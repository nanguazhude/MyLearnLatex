\newsection{阴谋}    %第六十三节:阴谋

\begin{this_body}

一只仙蛊,悬停在半空中,散发出清丽的华光。

柠檬色的光晕,全数笼罩在白凝冰的全身,深入她的空窍。

她的空窍四壁上,已经攀附上一根根黄绿色的光藤,密密麻麻地附着在上面。

片刻之后,砚石老人深吸一口气,将仙蛊收回自家空窍:“好了,你的空窍被我的永固蛊作用,可保你三个月无事。按照你的修行速度,三个月之后,你须来到我这里,再次加固空窍。”

自三叉山一役,已过去这么多时日。白凝冰的资质已经复原成十成,还原为十绝之一的北冥冰魄体。

十成的真元,带给空窍极大的压力。若非砚石老人用永固蛊,加固空窍,白凝冰不会存活至今。

虽然砚石老人对自己有救命之情,但白凝冰却没有诚挚的谢意,而是一直笔直地站着,冷若冰霜。

砚石老人撤开仙蛊之后,她睁开蓝色的眼眸,目光清冷,神情淡漠。

她早已经不再是初出茅庐的无知小儿了诱梦禁欢全文阅读。

在方源身边,她学到了很多东西。

砚石老人是通过仇九,主动找到她,明显是要利用她。而她加入影宗,充其量也不过是一场交易罢了。

砚石老人的态度很温和,微微笑道:“白凝冰,只要你诚心诚意归附我影宗,再次发下新的海誓。我便出手,将你转为男身。”

“哼,不必了。你帮我压制空窍,我暂时加入影宗,帮你对付方源,谋取他手中的定仙游,这就是公平的交易而已。我的男儿之身。还是我亲自拿回来,才够精彩啊。托庇于他人之下,岂是我白凝冰所为?”

说完这句,白凝冰转身就走。

一直到走出密室,白凝冰冷酷如冰的脸上,这才松动了一下,双眉皱起,目光冷冽。

这个七转智道蛊仙砚石老人,她很不喜欢。总觉得他有不可告人的目的。

对方虽然是蛊仙,但白凝冰从不惧怕死亡。一个不怕死的人,还怕什么蛊仙?

但对方和自己的目的是一致的,都是想要对付方源,白凝冰也就答应了这场交易。暂时加入到了影宗。

“方源……”

念头一起,白凝冰不禁又想起当日在三王福地的情形。

在万众瞩目之下,方源逆天而飞,以凡人躯,炼成仙蛊。

其后,三王福地崩溃,众人一片混乱。纷纷逃离三叉山。大同风刮起来,最终将整个三叉山都夷成了平地。

小兽王的名号,被万人念叨,很快就疯传南疆。方源失踪了。蛊仙的命令则下达到各方势力。

与其同行的白凝冰,自然成为了各大势力争相追捕的目标。不管是正道、魔道都想捉到她。

若非魏央顾念义气,故意放了白凝冰,在伤重濒死时分。又遇到仇九,白凝冰早就沦为阶下囚了。

仇九治好了她的伤后。便向她提出加入影宗,一起对付方源的建议。

白凝冰谋算方源失败,更加坚信方源有预言推算的蛊。她听到仇九的背后,有着同样可以布局推算的智道蛊仙,不禁意动。

她虽然不怕死亡,但如此丧失了生命,任由方源活着,实在不甘心,实在是人生的大败笔,太不精彩了。

所以她便答应了仇九,暂时成为影宗门人。并和砚石老人、仇九一起用仙蛊海誓蛊,定下约定:一旦杀了方源,她就脱离,恢复自由身。同时影宗上下,不得直接或者间接地再对付她。

待白凝冰的背影消失,砚石老人的脸色这才缓缓地阴沉下来。

白凝冰身上,有着一股魔性,令身为智道蛊仙的砚石老人,也感到难以掌控。

“此子无法无天,只是情势所逼,才与我联合,绝非久居人下之人。”砚石老人目光极为深沉。

“不过也不要紧,我为了永生大计,筹谋了这么多年,还怕这条小鱼翻腾出什么样的浪花来?哼!”

砚石老人冷哼一声,又将目光投向通天蛊的镜面。

就在这时,有一道神念向他传来——“我这里有第二空窍蛊的秘方,换你的神游蛊。”

神念的主人,自称琅琊老仙狐女仙途全文阅读。

“嗬嗬嗬嗬……”砚石老人大笑起来。

鱼上钩了!

谁说他这鱼饵,只钓方源?方源不过是一条小鱼,琅琊地灵才是真正的大鱼啊。

“琅琊福地,传说中收藏了无数秘方的地方。从洞天跌落成福地,当世堪称天下第一福地!更关键的是,这里面更收藏了不少仙蛊。天元宝皇莲首当其冲,我必要得之!”

砚石老人的眼中,闪烁着炙热无比的光芒,充满了贪婪的欲望。

他在很久很久以前,就开始谋划了。

甚至,琅琊地灵手中的第二空窍蛊,就是他暗中安排,刻意流落到地灵手中的。

以琅琊地灵爱好收藏秘方,喜好炼制蛊虫的秉性,必定会去炼制第二空窍蛊的!

“接下来,就是我苦苦等寻的良机了!”砚石老人一边冷笑着,一边传出神念,答应琅琊地灵,在宝黄天中进行了交接。

神游蛊从镜中落下,哪怕有许多蛊虫增益通天蛊的效果,也使得通天蛊裂痕满布,几个呼吸之后彻底损毁。

琅琊地灵不管这通天蛊,福地中还有数只通天蛊的存货呢。

地灵目光火热地看着手中的神游蛊,哈哈大笑:“第二空窍蛊我早就想炼了,哈哈,这样一来,实在太好了!”

他当即又取出一只通天蛊,沟通了宝黄天,开始大肆搜寻炼蛊材料。

“呵呵呵,入我瓮中也!”透过通天蛊,砚石老人看到这一幕,开怀大笑。

事关仙蛊,这场重量级的交易,自然落到许多蛊仙的眼中。

方源也一直在关注。

“是谁买下的神游蛊?”方源目光一闪。立即询问小狐仙。

“是琅琊老仙。”小狐仙脆生应答。

“琅琊地灵么……果然不出我的所料。”方源双眼微微眯起,又凝神看着通天蛊。

片刻后,他察觉到琅琊地灵开始大肆收买炼蛊材料,他的双眼闪烁出阵阵寒芒。

“哼,这么多蛊仙,同时抛售珍贵的炼蛊材料,而且专门卖给琅琊地灵,这明显是个陷阱啊。地灵虽然有智,但执念更深。极容易被蛊仙利用。等一等……”

方源忽然心头一震。

贩卖炼蛊材料的金道蛊仙铁甲子、魂道蛊仙王感仰、木道蛊仙檀香仙子、奴道蛊仙雪熊大仙……这些人的名字,很熟悉。不都是针对琅琊福地,参加第二波攻潮的蛊仙么?

一时间,方源眼中精芒烁烁,他意识到了一个原本埋藏在历史中。不为人所知的巨大阴谋。

“原来如此!前世五百年,琅琊福地先后,承受了七波攻潮,最终陨灭。这里面,原来有幕后黑手在操纵,一直在图谋不轨。”

先前方源只是看客,现在他身在局中。发现了这个真相。

“这个砚石老人,贩卖神游蛊,针对的不仅是自己,更大的目的在于琅琊福地。七波攻潮。他究竟组织安排了多少次呢?”

“如果他是南疆蛊仙,那么这些北原的蛊仙,为什么会听他的调动?是单纯的利用,还是直接的命令呢?”

毫无疑问完美重生路最新章节。智道蛊仙砚石老人,是有组织的。一个有组织的蛊仙。单单这个消息,就弥漫着一股无形的压力。

方源目前还不知道,这个组织就是他曾经听说过的影宗!

但这并不妨碍他的猜测这个组织的强大和神秘。

一个能横跨南疆和北原的组织,该有多么庞大?但这样的一个组织,方源从来就不知道,前世五百年也没有听说过。这样的组织该有多么的神秘?

“当然,这一切的推测,都建立在砚石老人是南疆蛊仙的基础之上。也许我只是恰逢其会?”

“再想想,第一波攻潮,会不会也是砚石老人安排的呢?余下的几波攻势,是否又是他的手笔?至少第七波攻势,是由天庭出手,绝对不是砚石老人的手笔。”

这点方源可以肯定。

天庭高高在上,源自中洲,根正苗红,绝非南疆的地仙可以插手的。

五大域中,最大的蛊仙组织,古往今来也只有天庭。

天庭之强,令人窒息。只有八转、九转的蛊仙才能进驻。

这点从结果上,也可以判定。

天庭出手之后,将琅琊福地中所有的秘方都收走了。

“前世那个时候的砚石老人,又在哪里呢?”

砚石老人,这是一个隐藏在历史深处的神秘人物!在他背后,还有一个神秘至极的组织,至少横跨北原、南疆两大域。

“智道蛊仙,神秘组织,天庭……永生之路果然是步步艰险啊。等我将这些阻碍,一一冲垮的时候,该是怎样的畅快呢?”

困难越多,强敌越多,方源却越是斗志昂扬。

他走的这条路,从一开始,就注定了寂寞孤独,注定了与世皆敌。天庭、神秘组织,就好像是盘踞在这条路上的两大巨兽。同时这条黑暗的路途上,布满了荆棘陷阱,充斥着阴谋暗算。

这是一条无比艰难的路。

似乎古往今来,从未有人走到尽头。

而方源则是独自一个人奋战,这是他一个人远征,一个人的圣战。

一切的动力的源头,只是一个最简单朴实,也最贪婪巨大,最被人嗤笑不屑,最让人难以理解的……

梦想。

关于永生的,似乎不切实际的梦幻泡影。

能不能成功,方源从未考虑过。

他只知道,就算失败身亡,自己也毫不悔恨。

今天,他通过一场交易,发现了一个原本隐藏着的阴影漩涡。然后他稍稍展望了一下自己的路,那是无比的黑暗,无比的艰险,几乎步步都是绝路。

“既然没有路,那就自己闯出一条来。”

方源微微带笑,目蕴神光,将纷杂的思绪清理干净,再次将目光投向通天蛊。

是时候买舍利蛊了。

\end{this_body}

