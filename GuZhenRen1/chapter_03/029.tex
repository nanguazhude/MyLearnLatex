\newsection{常山阴,做我的丈夫吧!}    %第二十九节:常山阴,做我的丈夫吧!

\begin{this_body}

广阔无垠的腐毒草原,到了夜晚,更加黑暗。文學馆

风在耳边呼啸,不知哪里的狼嗥,远远传来,好像幽魂的哭泣。

一团篝火,在草地上呼呼的燃烧着。

葛谣靠着篝火,渐渐驱除身上的寒意。

篝火上驾着锅,锅里烧着香气四溢的肉汤。

葛谣吞咽了一下口水,仿佛更饿了些。终于她吞咽了一下口水,问向方源:“常山阴前辈,这锅肉汤可以吃了吗?”

方源坐在少女的对面,两人之间隔着篝火。

“不急,这肉干才刚刚下锅,需要等水煮沸。之后再等片刻,等到里面的肉块完全酥软,吃起来才有意思。”方源一边取出推杯换盏蛊,一边淡淡地答道。

“哦,还要等这么久呀。”葛谣撇起嘴,俏丽的容颜在火光的照耀下,配上草原的长袍和精美的挂饰,更显出她的独特风情。

但这样靓丽的风景,却不能使方源的目光为之一顿。

他的目光集中在手中的推杯换盏蛊上。

推杯换盏蛊是五转蛊。但是到了北原,压制成四转。论容量,它甚至弱于一些四转蛊。论喂养,它代价高昂,占据五转蛊虫的巅峰。论真元损耗,它也是差强人意。但方源为何别的不选,独独选择它,浪费大量的精力,投入大量的资源,来炼它呢?

皆因它源自盗天魔尊之手,盗天魔尊偷天盗地,是历代所有尊者当中最富有的。

他花费毕生的精力,寻找传说中的空穴。

空穴最早记载在《人祖传》中,空穴是和光阴长河同等的秘禁之地。光阴长河中,有大量的宙道蛊虫。而空穴里。则生活着无数的宇道蛊虫。

它贯通五域,隐藏在无人知道的地方,从空穴中推开门扉,就能顷刻到达世间其他地方。通往它的门扉,又称之为空门。空门无处不在,可以在狭小的指缝间,也可以广布浩瀚的天空。只要有空间的地方,就有空穴的门扉悍妃:宠冠天下。

但是古往今来,极少有人能找到空穴。更遑论进入空穴。整个人族的历史中,似乎只有一人进出过。

盗天魔尊研炼出推杯换盏蛊的秘方,就是想要将这推杯换盏蛊,塞入空穴当中,然后带出空穴中的野生蛊虫。

但他失败了。却又成功了。

四百多年后,他的福地被人挖掘而出,引动各方蛊仙的争抢。推杯换盏蛊的秘方,也广为传播,强大的功效很快令它风靡五域,被无数蛊仙推崇。

方源先取出空窍中的金龙蛊。

原本四转的金龙蛊,已经被压成三转。它飞出来后。又投入到推杯换盏蛊里。

方源将大量的真元,都灌注到推杯换盏蛊中。

这只上金下银的杯盏,立时绽放出璀璨的金银光芒,缓缓地悬浮起来。

方源抽回手掌。然后对着杯盏,缓缓虚推。

推杯换盏蛊随之往前移动,随后它开始消失,先是边沿。然后消失大半,最后彻底消失在空气之中。

葛谣不禁站起身来。瞪大了双眼,惊奇地看着这一幕。

与此同时,在遥远的中洲,狐仙福地中。

小狐仙忽有所感,一个挪移,来到荡魂行宫的密室。

密室中,一只推杯换盏蛊,绽放着耀眼的光辉,缓缓悬浮而起。像是被无形的力量牵引,它慢慢前行,最终消失在空气里。

当推杯换盏蛊完全消失之后,方源又将手掌慢慢平摊,真元继续狂催。

忽然,就有一点金银之光,在他的手掌上空绽放。

随后,葛谣就看到空气中,先是露出一只杯盏的边沿,然后是一半的杯子,最终整个杯子从空中出现。

最后光芒消散,杯盏样的蛊虫,缓缓地落到方源的手掌中。

“成了。”方源低声喃喃一句,看到这个杯盏时,他就知道这事没有脱离自己的掌控。

“常山阴前辈,你在做什么呀?咦,这蛊好像有些不对劲。”葛谣走了几步,来到方源的身边,好奇无比。

“怎么不对劲了?”方源淡淡一笑,取出元石握在手心,快速回复着真元。

葛谣没有说话,只是盯着推杯换盏蛊猛看,忽然她双眼骤亮,叫出声来:“这只蛊不一样了,原先的是上金下银,如今的是上银下金。”

方源哈哈一笑。

没错!

推杯换盏蛊并非只有一只,而是两只。

这两只蛊一只上金下银杯,一只上银下金盏。两只组成一套,才是完整的推杯换盏蛊。方源临走前带了一只在身边,将另一只放在狐仙福地当中。

当他灌输真元,两只推杯换盏蛊同时进入空穴,在空穴中完成交换之后。福地中的那只,来到方源的身边。而方源装载着金龙蛊的杯盏,则回到了狐仙福地里。

昔日,盗天魔尊想要靠着推杯换盏蛊,盗取空穴中的蛊虫。他失败了,没有达成本来目的。但是推杯换盏蛊,却在另外一种程度上,获得了某种巨大的成功农家园林师最新章节。

依靠无所不在的空穴,一对推杯换盏蛊在空穴里完成交接,这就形成物资的运送。

更可贵的是,它只有五转,不是唯一的仙蛊。

在方源五百年前世,五域大混战期间,推杯换盏蛊成了各大势力都必备的蛊虫。就连蛊仙们,也都在争相使用它。

方源从这只推杯换盏蛊中取出了一封信笺。

信自然是小狐仙写的,说明了福地中的近况。

方源在腐毒草原,不过五六天的时间,狐仙福地中则已经过去了一个月左右。

信中言说,除了荡魂山外,福地一切都好。仙鹤门提出再次交易,小狐仙按照方源之前的叮嘱,婉言拒绝了。

交易次数若是增多,仙鹤门发现方源不在福地的可能性就越大。方源人在北原。但放心不下福地。靠着推杯换盏蛊,往来书信,他就可以幕后操纵,不教别有用心之人得逞。

方源看了信后,当即回了一封。

葛谣看得云里雾里,她不认识中洲的文字。

连同这封回信,方源又放进去三只四转蛊,分别是金缕衣蛊、横冲直撞蛊、骨翼蛊。

装的东西越多,推杯换盏时消耗的真元就越多。反而和两只杯盏的距离。没有丝毫的关系。

这是因为推杯换盏蛊,构思奇妙,借助了空穴这个奇妙的秘禁通道。

方源刚刚那次,只是一个尝试。确信推杯换盏蛊能够正常运作,他就开始正式地将身上的南疆蛊虫。放回到狐仙福地中,交由小狐仙喂养。

福地中,小狐仙趴在桌子上,睁着亮闪闪的大眼睛,盯着空气。

桌上的推杯换盏蛊里的金龙蛊,已经被她第一时间取走。

忽然,推杯换盏蛊漂浮起来。又钻入了空穴当中。随后,换成另一只,落到桌面上。

小狐仙连忙取出推杯换盏蛊里的东西,看到方源的回信后。她开心极了,不禁欢叫一声:“主人的回信!”

就这样再一次轮换,方源空窍中的真元,再次消耗得七七八八。

他不得不再次手握元石。补充自身的真元。

葛谣站在一旁,渐渐看出了些微端倪。好奇心旺盛的她。又不禁多问了几句。但方源却只是淡笑,没有正面回答她。

“哼,神神秘秘的,有什么了不起啊。”少女撅起嘴,气鼓鼓地坐回到原来的位置上。

她一屁股坐下,皱起好看的眉头,生气地盯着方源。

方源理都不理她,这令葛谣更加气愤。

她自幼便得父亲宠爱,是部族的族花,从未有人敢如此轻视她。偏偏方源一路上,将其视若无睹。

许多少年郎对她的热情追求,更助长了她的骄傲脾气。

葛谣又猛盯着方源看了一阵,方源恢复了真元,又在进行推杯换盏,仍旧没有理睬她。

这位北原少女的闷气,反而渐渐消散了。

“到底是常山阴,不是那些傻不愣登又肤浅卖弄的家伙能比的火影之水中无月最新章节。在他眼里,我又是什么样的呢?只是萍水相逢的后辈吧。”

想到这里,葛谣不禁有些心灰意冷,望着方源侧脸,渐渐不觉痴了。

方源用了人皮蛊,换了一个面孔,北原人特有的样貌,无疑更符合葛谣的审美观。

常山阴年轻时,就是常家部族少有的英俊少年。

他的五官端正,鼻翼挺拔,棕色的眸子目光深邃,薄薄的嘴唇抿着,无声地显示出坚强的个性。

他的双鬓已经微霜,流露出成熟男子的沧桑气息,对少女而言是一种强烈的吸引。

篝火随风晃动,火光照得方源面孔或明或暗,更突显出此刻他坚定稳重的气度。

葛谣的思绪渐渐飘散,她暗暗回想,方源究竟是个怎样的人呢?

第一次见面时的惊骇,微笑时的温和,指点自己时的智慧,战斗时的无双豪勇,还有扒下自己皮肤的冷酷冰寒。

这一幕幕,在少女的心中闪现,印象是如此的深刻,简直已经刻在了少女的内心深处!

“那他的过去呢?”葛谣不禁又想。

常山阴的过去,早已经成为一个关于英雄的传说,在北原大地上广为流传。

无数人敬重他,爱戴他,看好他。

他年少时,风头无两,是常家的未来希望。

他很早就成名,一流的驭狼术叫人刮目相看。

更关键的是,他正直公正,宽宏友善,从不欺压弱小,孝顺父母,帮助有困难的族人。同时也义气深重,不知多少次舍命保卫家园,为常家立下了汗马功劳。

他娶得娇妻,却迎来童年挚友的背叛。命运的无情捉拿,让这个男人失去了母亲,失去了结义的兄弟,失去了娇妻,甚至险些彻底失去了生命。

但他终究活过来。

靠着自己的努力,从死亡的深渊中艰难地挣扎出来,创造了一个常人难以想象的奇迹!

“这个男人肩上背负着无穷的痛楚,隐藏在暗处的是数不尽的伤痕啊。”葛谣想到这里,心中生出一股冲动,好想将方源抱在怀中,用自己的温柔来安慰这头受伤的孤狼,曾经的狼王。

篝火摇曳,烧得木材噼啪作响。

葛谣投注在方源脸上的目光,越加深情,渐渐不可自拔。

在温暖的火光中,有一种情愫在少女的心中酝酿、生长。

当方源完成这一轮的推杯换盏,又开始取出元石,汲取其中的真元时,葛谣暗自做了一个她人生中的最重大的决定。

她忽然站起身来,对方源大喊道:“常山阴!做我的丈夫吧!”

声音在深邃的夜里,在广阔的草原上传播,远去。

“你说什么?”方源皱起眉头,纵然他有五百年经验,也没有料到少女忽变的心思。

反应过来后,他展颜一笑:“不要闹了,小姑娘,我可是你的前辈。按照年龄,我大你二十多岁的人,我的儿子和你正好匹配。”

“不,常山阴,我就要你!”

------------

\end{this_body}

