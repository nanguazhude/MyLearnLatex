\newsection{交易仙蛊}    %第八十二节:交易仙蛊

\begin{this_body}

“你给我滚!老夫要收了你做徒弟,估计哪天就被你气死了。还是我的孩儿们省心,你给我滚,我不想见到你。”琅琊地灵连连摆手,对方源的拜师要求,毫不犹豫地拒绝了。

他原本能得到第二空窍蛊,但方源算计了他,使得他将炼成的第二空窍蛊拱手相让之外,还失去了神游蛊。

方源呵呵笑了声,对于地灵的拒绝也不放在心上。

地灵是蛊仙执念所化,简单又偏执,现在拒绝,就代表将来他仍旧会拒绝。

“真是可惜了,我原本还打算将神游蛊,当做拜师礼的。”

“哼,老夫最不待见你这种狡诈的人。还是毛民乖巧!实话告诉你,老夫已经收了十八个毛民徒弟。今后也只会收毛民做徒弟!”

“不说这话了,将第二空窍蛊给我罢。”方源伸出手掌。

琅琊地灵神情一滞,恋恋不舍地看了手中的仙蛊一眼。这原本是他想炼成的蛊,现在刚刚炼成,还未捂热,就要易主了。

但当年的约定,已经化为一种偏执,是组成地灵的一部分。他无法违背,也没想过违背。

“小子,你给我记住!”琅琊地灵低吼一声,将第二空窍蛊塞给方源。

这第二空窍蛊,宛若甲虫,两头尖尖,中间肥大。

甲虫有少年拳头大小,青玉似的,握在手中,温润清凉。

而在它圆滚滚的背部,还长着一只金色的眼珠子。金色的瞳孔。闪电般游弋不定,灵性十足。

“这就是当年,我在三叉山冒着奇险。千方百计想要炼出来的蛊。想不到会以这样的方式得到手中。”方源感慨地叹息一声,却没有急着用,而是将其收入囊中。

这第二空窍蛊虽然已经在琅琊地灵的主动配合之下,成了方源之物。但他还不是蛊仙,没有青提仙元,方源驱使不动。

“你已经拿到了仙蛊,如果你不想用掉最后一个机缘。那你现在就可以走了。”琅琊地灵下了逐客令。

方源却掏出神游蛊,面带微笑,在琅琊地灵的面前晃了晃:“你难道不想要这只仙蛊了吗?”。

琅琊地灵眉头一扬:“怎么。你想卖?”

他有通天蛊,沟通宝黄天,可以买到许许多多的炼蛊材料。哪怕是炼制第二空窍蛊所需的人窍,他也可以通过购买奴隶蛊师。然后杀掉获得。

第二空窍蛊是消耗蛊。方源用了一次之后,就不存在了。

琅琊地灵完全可以再炼一只。当然前提是,他必须有神游蛊。缺少神游蛊,他是万万炼制不成的。

方源却不直接回答,而是反问一句,道:“你说我再用一只第二空窍蛊,会不会生成第三空窍?”

“哼,做你的春秋大梦。”琅琊地灵立即嗤之以鼻。冷笑三声,“这是第二空窍蛊。不是第三空窍蛊。你想要凝成第三空窍?那你得先琢磨出第三空窍蛊的秘方来!”

方源点点头,神情认真:“我也是这么认为的。”

地灵不会因为想要收购神游蛊,而去欺骗他。同时,他也知晓第二空窍蛊的秘方,早就推算出这道理。

现在问一遍,只是稳妥起见,确认一下罢了。

这样一来,方源手中的神游蛊,以及第二空窍蛊的半成品,就失去原本的作用了。

方源已经拥有一只第二空窍蛊了,他是孤家寡人,没有亲信需要他提携,也就不需要第二只第二空窍蛊。

而且,他刚刚观摩了地灵炼蛊的过程,知道最后一步,极其凶险!光团几番膨胀、缩小,需要极强大的操纵能力,至少千人魂的底蕴,才能把持得住。

仙蛊岂是那么容易炼制的。亏得当年在三叉山时,方源重生之后,没有一门心思地去炼制第二空窍蛊。否则,依他当年的实力,绝对难逃失败的下场。单单炼仙蛊的反噬,就要使其濒死。

当然,他也可以卖到宝黄天中去。

但是这样一来,却会引起砚石老人的注意,暴露许多底牌。同时获得的东西,也难免被其他蛊仙动了手脚。

和琅琊地灵交易,就不一样了。

至少他不会以次充好,同时他财力雄厚,很想收购了神游蛊用之炼蛊。

而方源也避免了第二次暴露,安全得很。

“地灵,你再看看这是什么?”方源想了想,干脆将当年自己亲手炼制的第二空窍蛊的半成品,也取了出来。

这半成品,形态模糊,仿佛是模糊雕刻的粗坯,毫无生机。

它像是一块灰色石头,雕琢成的甲虫。大肚翩翩,头尾如尖锥,没有任何的足须和触脚。

虽然和真正的第二空窍蛊外形相似,但显然不好比。有着质的差距。

琅琊地灵见到这半成品,双眼不禁一亮:“想不到你居然进行到了这一步。不过,第二空窍蛊的炼制过程,最难的是在最后一步。前面的步骤,以炼蛊大师的水准,也能炼制出来。”

地灵语气中带着丝丝的喜悦。

方源有神游蛊,又有半成品。如果他收购过来,就只剩下最后一步,就能再炼出第二空窍蛊。

对于琅琊地灵来说,这样的诱惑,他难以抵挡。

“说吧,你想换什么?”琅琊地灵收回炙热的目光,看向方源。

方源看着右手中的神游蛊,立即脱口而出:“仙蛊无价,当然是以蛊换蛊。这可是蛊仙交易的老规矩了。”

琅琊地灵顿时脸色沉下来:“虽然是老规矩,但不适合我们这种情况。首先你的神游蛊,很不实用。只能当做万不得已之下的逃生手段。一旦挪移到火山熔岩底部,或者地心深处,简直就是自找死路。其次我用神游蛊。是用来炼蛊。第二空窍蛊对我一个地灵来讲,又有什么用呢?”

方源乐了,地灵讨价还价的时候,眼神直勾勾地盯着神游蛊看。虽然口气强硬,但神色已经出卖了他的内心。

“地灵,你太抠门了。我知道琅琊福地中,收藏着不少的仙蛊。神游蛊是六转仙蛊。我也不贪心,只换你一只六转的仙蛊就是了。”

地灵连忙摇头不止,又说了许多话。但方源死不松口。地灵渐渐急了,怒色流于表面,看着方源的目光像是要吃人。

方源见火候差不多了,开始收宫:“这样吧。我退一步。只要你一只消耗型的六转仙蛊。等我把这仙蛊用掉,你还可以再炼制出来,不是吗?”。

地灵神色缓了缓,方源的主动退让,让他有一种胜利的得意感觉。

他哼哼了几声,昂起头倨傲地看着方源:“也罢,就这么办吧。”

说着,他双手一展。凭空挪移过来五只仙蛊。

“琅琊福地的底蕴,真是雄厚。”方源心中大为感慨。一一瞧过去,忽然一楞。

“就要这只仙蛊了。”方源表情带着微微的怪异,手指着其中一只仙蛊。

这仙蛊不是别的,正是和稀泥仙蛊。害得荡魂山渐渐枯死的罪魁祸首,不想被琅琊地灵又炼制了出来。

双方迅速完成了交割,方源得到和稀泥仙蛊,收不进空窍去,只得暂时放入囊中。

“这个半成品我当初可费尽了心力啊,你得到它,可以节省一大笔开支。大家都是熟人了,便宜卖给你,你就转给我一千头毛民罢。我也不贪心,就要你刚刚炼蛊时,指挥的那些老毛民好了。”方源道。

“屁!”地灵勃然大怒,“你当我是三岁小孩儿?那些毛民至少都是炼蛊大师,放到宝黄天中,宝光至少有七丈高!”

方源嘿嘿一笑,他十分觊觎这些毛民。有了这些毛民在手,对他的帮助相当巨大。

“这样吧,我不要一千,只要八百。”

“屁的八百,这些毛民都是我的孩儿,我一个都不会卖的!”琅琊地灵怒极大吼。

“任何东西,都有一个价格嘛。我们还可以再商量!”

“不卖就是不卖!你再给我提这事,那你这个半成品拿回去吧,我不买了。”

琅琊地灵坚决的态度,令方源暗吃一惊。他把价格降到了底线,琅琊地灵明明有赚头,但就是不卖。看来是真的有感情了。

这种情形,也并不奇怪。

很多蛊仙在福地里豢养异人,就像是养宠物,看着他们一个个的长大,甚至还会花费大力气培养他们。若是他们死去了,蛊仙们也会伤心地掉落眼泪。

当然,这种情况绝不会发生在方源的身上就是了。

方源见买不来这些毛民,心中暗暗可惜,只好退而求其次:“既然如此,那我就换驭狼蛊的秘方罢。”

琅琊福地中,收藏着大量的秘方。从古至今,可谓浩如烟海。

驭狼蛊这种常见的蛊方,不可能没有。

方源一直在收购这些秘方,但自从他发现砚石老人的存在后,在宝黄天中,他却收敛了动作。

智道蛊师擅长推算,但不是凭空而来,也要收集大量的情报。在这些情报的基础上,加以推衍,得出结果。

方源若是在宝黄天中,大张旗鼓地要收购驭狼蛊的秘方,难保砚石老人不会推算到什么。

“拿去。这是一转到五转的驭狼蛊秘方。”琅琊地灵交给方源一大叠的秘方。

方源翻了翻,发现单单五转驭狼蛊秘方,就有八种。分别用不同的材料,不同的方法,得到相同的蛊虫。而一转到四转的驭狼蛊秘方,就更多了。

“这买卖值了!”方源心中暗喜。

“交易完成了,你可以走了。”琅琊地灵不耐烦地下逐客令。

方源却摆手,笑道:“不忙,不忙,我还有一笔交易,你肯定感兴趣。”

“哼,年轻人口气不要这么大。这个世界上能让老夫感兴趣的,可不多了。”琅琊地灵摸摸胡须,自傲地道。

“我这次卖的,是一个消息。这个消息就是,我什么时候用掉第二空窍蛊。”

琅琊地灵神情僵滞,他呆呆地看了方源一眼后,整个眉头都深深地皱起来,用强烈的鄙视和厌恶的目光,瞪向方源:“你这人怎么生得如此卑劣无耻!?你还有没有做人的底线?!”

“哈哈哈。”方源仰头大笑,“难道这个消息,你不感兴趣吗?”。

琅琊地灵顿时有一种被强暴之后,还要向罪犯道歉的屈辱感。

他能不感兴趣吗?

第二空窍蛊是仙蛊,而仙蛊唯一。如果方源一直留着不用,那他就永远也不肯再炼出第二空窍蛊来。

“你想怎么卖?”地灵强忍暴揍方源的冲动,最终忍气吞声地问道。

方源双眼眯起来,精芒闪烁不停:“我也不欺你,就用消息换消息。我想知道,王庭福地中那座八十八角真阳楼的一切信息。”

“八十八角真阳楼?你竟然知道这座仙蛊屋和老夫有关系!”地灵顿时大吃一惊。

\end{this_body}

