\newsection{散迫身亡不可待}    %第一百八十五节:散迫身亡不可待

\begin{this_body}

%1
啊——!

%2
凄厉的惨叫声,几乎要刺破耳膜。

%3
常极右痛得浑身颤抖,身躯肌肉贲发,背后六只怪物手臂四处乱舞,搅动出风声如雷。

%4
但方源只需要一只手,就将其牢牢按在半空当中。

%5
此刻他双目如电,一边操纵蛊虫,一边沉声断喝:“我儿,忍住!就看这个时候了!”

%6
常极右痛得几乎要失去理智,这是非人的痛楚。

%7
即便是化身成天尸,几乎没有痛觉,也让他感觉到整个身体的每一寸肌肤,每一丝肌肉都在被无情残酷地撕扯拉断。

%8
但方源的话,像是一记清泉,浇灌在常极右的心田。

%9
常极右拼命强忍,终于在快要眩晕过去的那一刻前,剧烈的痛楚消散如潮水般褪去。

%10
“成,成功了?!”常极右喘着粗气,头疼欲裂,勉强睁开双眼,竟然看到自己手臂上的肌肤已经回复成鲜活的血肉。

%11
甚至,就连他背后生长出来的六条怪臂,也有收拢缩小的痕迹。

%12
没有什么,比回复人身更吸引常极右的了。

%13
这一刻,他几乎要喜极而泣了。

%14
“不好!”忽然,方源开口,语调低沉。

%15
常极右刚刚恢复的手臂,几乎瞬间,又重新转化成天尸状态,变得如枯木一般。

%16
他身后的六臂,也重新长成,顽固至极。

%17
“不,不!怎么又变回去了?!父亲……”常极右惊惶大叫,下意识地就向方源求救。

%18
但当他看到,方源一脸伪装出来的疲惫之色时,常极右叫不出声来了。

%19
“父亲累了!是啊,这些天来,都是父亲帮助我,想了近百种方法,不眠不休。用坏了不知多少蛊虫。至少我见到的五转蛊。就有七八十只!我有什么理由责怪父亲呢?这都是我轻忽冒失,犯下的大错!父亲,孩儿对不起您的栽培啊……”

%20
常极右心疼又惭愧,又感受到一种为人子女,被狼王照料,宛若小草被大树庇护的幸福感受。

%21
“可恶,这天尸形态真是顽强。竟然连这个方法都不奏效!”方源暗中咬牙。在脑海中询问:“墨瑶,你还有什么法子,快快讲来!”

%22
方源自己构思出十几种方法,都在之前就被一一否决。

%23
现在试验的办法,都是墨瑶设想出来的。

%24
这也是方源的算计之一。

%25
一方面,是不断试验常极右。企图找寻到恢复人类状态的方法。另一方面,也是趁机激发墨瑶意志不断思考,消耗她的力量。

%26
墨瑶沉吟片刻,方才道:“还有一个法子,比较危险,试蛊之人承受的痛苦将是之前的三倍!极有可能,让他直接活活痛死!”

%27
“他是天尸。已经算是死了。怎么还会痛死?”方源疑惑。

%28
“他只能算是半个死人。身躯是死的,魂魄却还遗留在体内。除非他的魂魄,被吸进生死门中,这才能算是真正死亡。”墨瑶解释道。

%29
生死门乃是天地秘境之一,十分著名,在《人祖传》中早有记载。

%30
如今的时代,和太古那会儿不同了。

%31
生死门早就消失,几经易主。对亡灵的控制力大大降低。因此才有僵尸这样的怪物存在。这要搁在人祖时代,天地间是没有僵尸的。

%32
墨瑶的这个方法,风险极大,能让人痛得魂魄都碎掉,分崩离析。

%33
常极右魂魄一碎,那就会彻底死亡了。

%34
墨瑶又劝道:“小子,凡事有得必有失。杀招威力绝伦,缺陷便难以弥补。这些天来,你试验了这么多次,耗费家财。甚至多次向黑家等等赊账,也已经清楚其中的难度了。”

%35
“这是我最后的法子,我并不能保证一定见效。而且此法危险极大,九死一生!用不用这个方法,你看着办。你这个‘儿子’已经彻底化为天尸,战力卓绝,是个很好的战力。他对你又是言听计从。如此牺牲掉,实在太可惜了。我劝你不妨留他在身边调遣。”

%36
方源沉吟不语。

%37
“父亲,请您多加休息吧,孩儿不急……”另一边,常极右也道。

%38
方源凝神望去,常极右虽然化为天尸,相貌丑恶,但是一双眸子却仍旧清澈,饱含着对他的孺慕崇拜之情。

%39
方源面泛微笑:“孩子,为父已经想到了一个好方法。此法十分危险,你会有性命之忧。但是成功的可能极大。为父正在犹豫不决……”

%40
“唉……”脑海中,墨瑶一声幽幽叹息。

%41
方源虽然这么讲,但厌恶自身,矢志复原的常极右听了,只会有一个选择。

%42
果然,常极右闻言,犹豫了一下,双眼绽放奇光:“父亲大人!请你用这个方法吧,孩儿这样子简直是生不如死!”

%43
方源凝视常极右的双眸:“可是,你是我唯一的孩子啊……”

%44
“父亲!”常极右哭泣,拜倒在地上,他抱住方源的小腿,“孩儿也舍不定您啊。但是孩儿真的不敢以这样的面孔生活下去,哪怕只有一丝希望,孩儿也要努力挣扎!”

%45
方源沉默良久,这才发出一声长长的叹息声:“也罢。人总归是要依照自己的意愿而活着的。为父也不忍心让你这样痛苦下去了!你休息几天,让为父准备一下。几天后,咱们便做最后的尝试!”

%46
几天后。

%47
“啊啊啊……”

%48
常极右嘶吼的声音已经沙哑一片。

%49
“再坚持一下。”方源目光温暖,心中冷静如冰,勉励道。

%50
但下一刻,常极右的惨叫声戛然而止。

%51
这是他第三百零七次,痛得晕死过去了。

%52
“哼,真是不中用!”方源面沉如水,不满地冷哼一声,只得停止催动蛊虫。

%53
这个试验,必须要在常极右神智清醒的时候进行,否则无用。

%54
常极右昏迷过去,表明他的魂魄已经支离破碎,再一次达到了濒临崩溃的边缘。

%55
方源只得停下来。利用魂道蛊虫,为其稍稍滋补魂魄。

%56
“我说过的,这个方法希望不大,还是留常极右一条性命吧。”脑海中,墨瑶语气悲悯地劝道。

%57
方源冷哼一声,双眼眯起,眼缝中闪烁一抹锐利的寒光:“再试一次。再试最后一次!”

%58
常极右悠悠醒转,视野从模糊到清晰。他看到一旁站着的方源,在他眼中,“父亲”憔悴而且疲惫。

%59
这让他不禁心头揪起,惭愧地双眼流下泪来,哽咽出声道:“父亲……”

%60
“咱们再来一次吧。不要灰心。”方源微笑抚慰道。

%61
“父亲,如果这一次我还是痛晕过去,就不要再试了。父亲,这一切都是我的错,您实在该好好休息一下了。”常极右道。

%62
“好,试了这次之后,是该好好休息一下了。”方源叹息一声。语气中的深意却不是常极右能够感受得到的。

%63
先是三只五转蛊虫,一齐催动起来。

%64
然后渐渐加入其它蛊虫,这些蛊虫有的悬停在常极右的面孔上,有的钻进他的肌肤里,有的混入血脉,流喘到他的心脏当中。

%65
“啊……”剧烈的痛楚再次传来,常极右咬牙低吼,很快就张开嘴巴。发出大吼声,面容扭曲。

%66
方源手中动作有条不紊,依次加入蛊虫。

%67
蛊虫数量越多,常极右的痛楚就越是猛烈。他奋力挣扎,双眼翻白,痛到失去理智,状若癫狂。

%68
“停下吧。他的魂魄又要支撑不住了。”脑海中,墨瑶劝说道。

%69
但方源好似没有听到,仍旧再添加蛊虫上去。

%70
“可以了,这个数量已经超出了以往任何一次。常极右真的要支撑不住了。”墨瑶不忍目睹。

%71
方源冷哼一声。没有答话。

%72
“你这样太乱来了,这样搞下去,他会彻底死亡的!”脑海中,墨瑶意志提出抗议。

%73
“不冒点风险,怎么行?”方源冷笑,将蛊虫数量加到最大。

%74
他的双眼,绽放出兴奋的光:“再加三只蛊虫下去,就能彻底见到效果了。这是第三只。”

%75
“好,这是第二只!”

%76
“妙极了,只剩下最后一只,成败在此一举……呃!”

%77
方源神情一滞,周围蛊虫陡然间,如烟火般四下崩散。

%78
常极右不再挣扎,一动不动地悬浮在半空中。

%79
他死了。

%80
痛到令魂飞魄散。

%81
脸色却很安详。

%82
周围归于死寂。

%83
“是你害死了他。”脑海中,墨瑶语气冰冷。

%84
“是我害死了他。”方源眉头微微一扬,轻笑出声,“他也算死得其所了。至少让我知道,这杀招的缺陷是多么的顽固。”

%85
墨瑶没有再说话,似乎是不耻方源的冷血卑鄙。她将身影隐匿,消失在方源的脑海中。

%86
方源脸上的笑容,一点点消失。

%87
他越发认识到,脑海中的墨瑶意志,是个极其巨大的威胁。

%88
她究竟知不知道,针对六臂天尸王杀招缺陷的方法?

%89
是知道正确的法子,故意不想说。还是不想思考,以防削弱她自己呢?

%90
尤其是杀招缺陷如此严重,一旦关键时刻,墨瑶意志忽然冲击脑海,令方源思维一度紊乱。方源就算是想停下杀招,也不可能了。

%91
“墨瑶的这股意志,究竟是用的什么蛊?这些天来,我四处收集智道资料,也收购了不少凝练意志的智道蛊虫。但却从未发现有什么蛊虫,可以产生她这种缥缈隐秘,让人难以捉摸的意志!”

%92
“唉,我的时间不多了。现在八十八角真阳楼已经形成了近六十层。黑楼兰正全力攻取第三十九层。看来他的目标,极有可能是木鸡蛊。”

%93
“我得抓紧行动,不能再等下去!”

\end{this_body}

