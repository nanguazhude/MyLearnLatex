\newsection{关键问题}    %第一百五十五节:关键问题

\begin{this_body}

方源刚刚迈过来客止步碑,就有惊喜的发现。

一只借力蛊,出现在他的眼中。

这只上蛊力道的五转蛊,早已绝迹,比全力以赴蛊还要珍稀。它不能单独使用,而是得和其他蛊虫搭配起来。

和天力蛊搭配,就能使得蛊师借助天穹的力量。和地力蛊搭配,就能借取到大地的力量。和火力蛊搭配,便能使得蛊师从火焰中接取力量。和水力蛊配合,便可以从水流中汲取力量。

“我的杀招四臂风王,采用的是风霸王蛊、霸力蛊。这个蛊虫的组合,本身就是借力蛊的替代品,功效还不足原版的五成。我有了这只借力蛊,完全可以将杀招提升到全新的高度了!”方源心中暗喜。

当然,借力蛊还需要和其他蛊虫搭配使用。

到如今,天力蛊也已经绝迹。但地力蛊、火力蛊、水力蛊、风力蛊等等,却还广为流传着。

但它们使用这些蛊虫,却已经脱离了原先的作用范围。

地力蛊,常用于增长土地肥力,帮助蛊师栽培作物,或者搭配木道蛊虫使用。

火力蛊,则被一些炎道蛊师用作辅助蛊,能略微增强火道蛊虫的效用。水力、风力、电力等蛊,亦是如此。

方源来到晶壁面前,取出近十只泉蛋蛊,将借力蛊换取过来。

过了来客止步碑后,他要获取晶壁中的珍宝,还是只有换取一途。

方源继续朝前走。

这里晶壁中的珍宝。明显比之前的高出一两个档次。

流星雨蛊、星驰电掣蛊、风鬟雾鬓蛊、星火燎原蛊、水幕天华蛊……

在外界稀少无比的五转蛊,到了这里却很常见。反而四转蛊,变得比较少见了。但一旦出现四转蛊时。必定都是珍稀四转,价值和效果都可媲美普通的五转蛊虫。

方源目光四下扫射,身上笼罩着的血焰之光,不断消磨。

这层血焰之光,是他遮掩身份的保护。一旦消磨殆尽,他就会被八十八角真阳楼发觉,在瞬间铲除。

“在保护消失之前。必须找到那块楼主令!”

随着时间的推移,方源心中亦渐生焦急之感。

这些用来伪装身份,欺瞒八十八角真阳楼的蛊虫。并不好炼,且造价不菲。其中主材是黄金家族的上千斤血液,还需要经过九十八步的漫长提纯的过程。

如今王庭之争已经结束,方源要想大规模地弄到这些血液。就更不容易了。

更糟糕的是。眼前的水晶长廊居然出现了岔道!

方源不得不停下来,仔细辨认。

究竟哪一条,才是他想要走的路?

这个时候,前世中洲蛊仙攻破王庭福地的影像,帮了方源的大忙。

他选取了左边一道,钻了进去。

这段晶壁里的珍宝,价值又拔高一档。四转蛊已经消失,只有五转蛊。同时出现了五转珍稀蛊虫。

方源一边快步前行,一边迅速扫视。忽然他目光一定:“找到了!”

一枚楼主令,被封印在晶壁当中,距离地面只达方源的膝盖高度。

这枚楼主令,乃是中洲蛊仙的手笔。

树大招风,王庭福地屹立至今,早就已经吸引了各域蛊仙们的注意。中洲蛊仙更是在数百年前就开始布局。

楼主令一般掌握在盟主手中,且一旦离开王庭福地,就会自行摧毁。

近千年前,就有中洲蛊仙煞费苦心,暗中谋算,不仅买通了那一届的盟主,而且据说还消耗了一只仙蛊。

利用仙蛊的力量将楼主令篡改后,这块楼主令就一直隐藏在八十八角真阳楼里面,成为暗中的伏笔,等待后续的良机。

方源前世五百年,中洲蛊仙首先掀起五域大战,不久后创造了良机,最终攻破这里。

王庭福地乃是巨阳仙尊的布置,攻破这里,比攻破其他福地要艰难得多。

但巨阳仙尊已经逝去,中洲蛊仙上千年的筹谋,苦心孤诣的酝酿,终于得到了成果。

然而今生,这块楼主令落到了方源的手中。

换取这块楼主令的过程,十分顺利。但是要真正将它化为己用,却不简单。

到了此处,是最关键的一步。

方源脸色转为肃然,直接盘坐在地上,全神贯注。

一只只蛊虫随着他的心意,调动而出,不断撞在楼主令上。

叮叮咚咚……

撞击声仿佛是音乐般悦耳,楼主令渐渐地悬浮在半空中。每一次撞击,都会绽放出一层光晕。

直到它笼罩了三十八层光晕之后,它表面的灰白之色猛地消散,显露出楼主令三个大字。

光晕宛若气泡般,纷纷破裂。

楼主令失去了浮力,掉落下来,正好被方源一手接住。

他连忙咬破手指,将血液滴在楼主令上。

楼主令被血液一浸,整个金属令身,居然变成一块半透明的琉璃。方源也看过黑楼兰手中的楼主令,他立即察觉:自己手中的这块和平常的楼主令,大有区别。

“我刚刚的手段,只是参照前世影像,唤醒这块楼主令的真面目。看来这很可能是仙蛊的力量,也只有仙蛊才能篡改楼主令,使得八十八角真阳楼没有发觉。”方源看着手中的琉璃楼主令,陷入沉思。

十几个呼吸之后,笼罩他全身的血焰烟光彻底散去。

这一刻,世界都仿佛安静下来,方源能清晰地听到自己的心跳声。

“安然无恙。”心跳声渐渐消失,方源缓缓站起身来,吐出一口浊气。

“成功了。”他振奋地一握左拳。喃喃自语,却发现声音变得有些沙哑。同时浑身都被汗水打湿了,头脑更有一种微微的眩晕感。

真正开启这块楼主令。并让它承认自己有大不易。可以说,比炼制五转蛊虫都要困难得多。

稍差一步,就可能万劫不复。

方源顶着巨大的心理压力,终于功成。

“现在,只要我手持琉璃楼主令,便能自由出入秘藏阁了。再不用需要什么上等通关!”

掌握楼主令,相当于掌握了八十八角真阳楼的一小部分。

方源试着运用楼主令。几乎瞬间,在他的脑海中就浮现出黑楼兰等人的身影。

过了五十四关之后,他们已经闯到了第六十一关。目前。正在和一头金白虎虚像激战。

真正的金白虎,乃是荒兽级的存在。

金白虎虚像,拥有荒兽的气势,将黑楼兰等人压入下风。

黑楼兰一方虽然人多势众。但败象已显。

方源暗中关注了一会儿:“如果不出意外。黑楼兰等人支撑不了三刻功夫,就得退走。留给我的时间不多了!”

方源虽然进入了秘藏阁中,但只要楼主令的持有者走出八十八角真阳楼,那么他就得随之离开。

如今方源虽然有琉璃楼主令,可以自由停留。但是当前,却还不好这样暴露。

“光有琉璃楼主令,还远远不够。我还得找到那个缺口,将其彻底炸开。这样一来才能造成八十八角真阳楼的大漏洞,令我能够自由地取走晶壁中的珍宝。”

琉璃楼主令在手中轻轻一晃。下一刻方源消失在原地。

“这里应该就是中枢室了!”

再睁开双眼时,方源发现自己来到一处密室。

密室成圆形,闪烁星辉的墙壁围绕一圈,中央立着一张白玉圆桌。圆桌上堆砌着精致模型,类似沙盘,正是王庭福地的全貌。

不仅山川河流,中央的圣宫,甚至福地中的各个小塔楼都清晰可见。

方源调出蛊虫,一一飞到半空,化为一股股黑色烟气落入沙盘。

沙盘被黑烟浸染,很快就变成墨色。

墨色渐浓,形成大片粘稠的汁液,在沙盘上缓缓流淌。

方源目光注视,见得沙盘上一处黑液形成漏斗状,仿佛底下破开一个洞口似的,使得周围的黑液正缓缓地朝这个洞口注入。

“找到了,这就是那个漏洞!接下来,就是将此漏洞扩宽,将琉璃楼主令炼成一角楼主令。有了一角楼主令,我甚至能随意操纵八十八角真阳楼的其中一层!嗯?”

就在这时,方源动作一顿,双眼死死地盯住沙盘上的这处漏洞。

整个沙盘上,都被浓稠的黑油般的粘液覆盖着。因此刚刚第一眼时,方源还没有认出来。但此刻盯着看了一会儿,猛地发现这处漏洞不是别的地方,正是――土丘传承之地!

“这是怎么回事?难道说,和琉璃楼主令一眼,土丘传承同样也是中洲蛊仙早就安排下去的伏笔吗?!”

方源暗吃一惊。

但他很快镇定下来,察觉到这个猜测中的欠妥之处。

“不,不对。如果这只是攻破福地的预先伏笔,那么那句密语,以及灰白石板的线索,又该如何解释呢?这些线索,明显是传承线索。”

方源眼中精芒闪烁。

此刻,前世的影像,也不能给他带来帮助。

“会不会是中洲蛊仙,也发现了这处传承,然后没有勘破密语,最终这样利用了这处传承呢?不,也不对。设身处地去想,如果中洲蛊仙发现了这处很有可能藏有仙蛊的传承,他们肯定会心动的。那么这样一来,他们应该没有得到传承的线索,只是发现了这处漏洞?”

“当然,也有一种可能。他们也勘破不了传承的奥秘,攻破福地事大,最终选择破开这样的漏洞。但如此一来,地丘传承也被破坏了!”

方源左思右想,觉得这两种可能都有。现在没有确凿的证据,他也无法确认。

他不禁陷入迟疑。

一旦他利用前世影像中的手法,炸开这处漏洞,那么地丘传承肯定会彻底毁掉。

但如果不用,保留地丘传承,那么八十八角真阳楼的攻略,就止步于此了。

“八十八角真阳楼的价值,要比地丘传承高得多。实在不行,只有放弃地丘传承了。不过,这布置传承的人真是厉害,居然能钻破巨阳仙尊的布局……嗯?等等!”

方源陡然心神一震,想到了一个极关键的问题!(未完待续。。。)

\end{this_body}

