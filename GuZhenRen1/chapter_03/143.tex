\newsection{幽火蟒穴藏传承}    %第一百四十三节 幽火蟒穴藏传承

\begin{this_body}

当方源睁开双眼的时候,眼前的景象已经有了翻天覆地的变化。

天空是淡金色的,大地如春,有青绿的梯田,有静静流淌的河流,有低缓的丘峦,视野一片开阔。

这是宁静祥和的世界,和外界风雪交加,形成鲜明的对比。

这里便是北原最大的庇护所——王庭福地,每十年开启一次,专门奖励给北原的王者。

方源环顾周围,发现只有他一人。

虽然大家,都是进入了同一个门户,但是在跨入门户的瞬间,众人已经被打散,随意传送到福地中的任何角落里。

这是惯例,方源并不惊讶。按照先前的约定,接下来,他要向福地的中心进发。在那里,有着巨阳仙尊,曾经居住的寝宫——北原圣宫!

“我终于进来了。”方源调整了一下呼吸,王庭之争不过只是前戏,接下来才是重点。

他试着催动鹰扬蛊,晶紫真元随着意念,如臂使指,灌输到鹰扬蛊中。

刷。

一声轻响,伴随着剧痛,他的身后生长出两只宽大的,翼展超越一丈的漆黑鹰翼。

王庭福地中,并不禁凡蛊的使用。而至于仙蛊,任何的福地都是禁锢不住的。

有力的鹰翼,轻轻一振,便将方源整个人带上天空。

在天空中飞行,迎面轻风徐徐,风中夹裹着弥漫在整个福地中的独特馨香。

和外界的北原相比,这里宁静祥和,简直就是天堂。

方源也不着急。悠悠飞行,观赏着周围的风景。

王庭福地的地貌,和北原地貌极为类似,放眼望去。是一片片的平原。纵然有山丘,也都是缓坡,线条柔和优美,宛若没有勾勒线条的翠绿。舒畅婉转地流淌。

但和北原不同的是,每隔八里,大地上都会竖立着一座塔楼。

这些塔楼,让方源联想到了图腾柱子。每一个都高达八丈,笔直伫立,表面有黄金、白银裹皮,雕缀各色宝石、玛瑙,精美异常。

塔楼里,分成无数的间隔。宛若蜂巢。里面居住的。是一只只的蛊虫。

当福地里的虫群中。产生了蛊。这些蛊虫,往往就会脱离虫群,来到塔楼中居住生活。

塔楼是巨阳现在的布置。不管任何的蛊,都会从塔楼中寻找到自己的食物。

每一个塔楼中。都有数万的蛊。这些蛊,包含的种类繁多。常见的数量庞大,珍贵的较为稀少。

毫无疑问,每一座塔楼都是一笔巨大的财富,即便是方源,也要眼热。他甚至从某座塔楼中,见到了一支规模上千的星萤蛊群!

“可惜这些蛊虫,绝不能擅自取用。王庭之争开启之初,常有蛊师胆大包天,想要盗取,甚至攻打塔楼,夺取里面的野生蛊虫。结果都如同蜡烛,整个皮肉都融化掉,只剩下惨白的骨架落在地上,摔成一堆。”方源目光沉凝。

这是福地的伟力,是这片天地的力量。

只要是凡人,就无法抗衡。

就算是蛊仙,也要狼狈不堪。

受到的教训足够深刻,到如今,再没有蛊师想要打这些塔楼的主意。

“若追溯源头的话,开创王庭福地的蛊仙,乃是一位宇道蛊仙,姓名已不可考。因此,这片福地极为广阔,远超同等福地。巨阳仙尊还未成仙的时候,幸运地继承了这里,成为福地的新主人。等到巨阳成为仙尊之后,拥有无上伟力,便布置大手段,订下王庭之争的传统,也使得这片古老的福地能延续至今。”

方源一边飞翔,一边在心中回忆着。

仙尊的手段,已经超越了他的理解范围。也不知道巨阳仙尊是如何做到的,总之王庭福地在他的操弄下,再无天劫地灾的困扰折磨。

嘶嘶嘶……

大约飞行了半个时辰,越过无数的塔楼,在一座小山谷的上空,方源受到了一只巨蟒的挑衅。

这头猩红色的巨蟒,体型庞大,至少长达三十丈,蛇躯堪比塔楼般粗壮。

它的头上,长着一只尖锐的独角,一对通红的血眸,死死地盯住半空中的方源,不断地吐出蛇信。

它的蛇信,是诡异的紫色,上面缠绕着幽蓝的火焰。

“咦?居然是罕见的幽火龙蟒。”方源微微一怔。

就在这一刻,巨蟒猛地张开血盆大口,吐出一团马车大小的蓝紫火焰。

火焰飞速接近,空气中的温度顿时暴涨,还距离数百步远,方源的头发、眉毛都有了枯萎的迹象。可见蓝紫火焰温度的可怕!

方源轻轻一扬眉头,鹰翅一振,带着他猛地拔升,轻松地避开火焰的打击。

杀招——四臂风王!

他同时催动十多只蛊虫,空窍中晶紫真元开始剧烈消耗。而他的身侧,则冒出两只全新的青铜手臂。

随后,他如同陨落的星辰,悍然俯冲而下。

轰!

他狠狠地撞上幽火龙蟒,和它激战在一起。

一时间,烟尘飞腾,火焰四散喷涌,山谷震荡。

幽火龙蟒是异兽王,异兽是四转战力,它们当中的王者,皆可媲美五转蛊师。但方源早就是五转巅峰,施展了杀招之后,战力更强。

幽火龙蟒若是乖乖地蛰伏着,方源一心赶路,还未必能发现它。现在它主动挑衅,让方源见猎心喜,就在它的身上实践改良的杀招。

一炷香之后,尘埃落定。

方源浑身焦黑,站在面目全非,几乎完全坍塌的山谷中。

破碎的山石,将幽火龙蟒的大半个身躯掩盖。

方源咳嗽几声,吐出几口鲜血。

改良后的杀招,果然后遗症比之前小多了。当然这也是因为。幽火龙蟒没有人一样的智慧,不会从战斗中分析出方源的破绽来。

如果风被禁住,方源的后顾之忧就要大得多了。

这场战斗,并不容易。

王庭福地环境极佳。蛊虫众多,因此幽火龙蟒身上寄居着大量的炎道野生蛊虫。其中几只价值相当的高。

方源纵然杀招犀利,但针对炎道的防御并不突出。

如果省去躲避火焰的功夫,一炷香的战斗时间。还会缩短至少三分之一。

方源开始打扫战场。

这头异兽王,浑身都有价值。譬如蟒血,是用来喂养某些血道蛊虫的最好食料。蟒皮、蟒筋等等,放在凡人市集上,都会引发极大的轰动。

尤其是蛇躯中的幽火蛇胆,十分珍贵,在宝黄天中也有市场。

方源稍微处理了一下,为了节省时间,只将看得上眼的东西揣入蛊中。存储起来。

“幽火龙蟒。是以家庭的形式生活在地洞里的。如果有龙蟒的幼体。兴许还能挪到狐仙福地中放养繁衍,将来也算是一门进项。”方源想到这点,便四处搜寻。

很快。他就有了发现。

“嗯?这里原来有一道蛊师传承。”方源没找到小龙蟒,却意外地发现了一座火莲般。通红的巨石。

凭他的眼光,他很快就发现,这其实是蛊师的手法。

当他稍稍走近巨石的时候,这块酷似火莲的岩石,就层层分开,宛若火莲绽放。

火莲巨石彻底展开,露出当中的蛊虫,以及石碑。

石碑是和巨石一体的,上面刻着北原的文字。

方源一览无余,旋即就明白始末。

留下传承的炎道蛊师,名为火正君,是一位正道四转蛊师。他误闯入这片山谷,结果遭到幽火龙蟒的残害。伤重濒死前,他无奈地留下一身的蛊虫,并布下这道传承。

若今后,有人有缘来得此处,那么他留下的这套蛊虫,就是有缘人的。

火正君留下的蛊虫,本来有七只。但经过这些年,已经死去了四只,只剩下三只。

这三只蛊虫当中,只有一只能稍微入方源的眼界,乃是四转的炎瞳蛊。

蛊师催动炎瞳蛊时,目光所及之处,便会生出火焰,灼烧敌人。这样方便的攻击手段,常常叫人防不胜防。

但也有缺陷。

譬如持续催动,会导致蛊师本人的双眼被烧焦。须得使用上好的治疗蛊,以及搭配其他相应的蛊虫,来减少这样的后遗症。

这只炎瞳蛊,便是火正君的核心蛊虫。

抛开他留下来的蛊虫之外,在石碑上,还有他所记得的蛊方。

方源目光扫了三遍,将这些信息都存储到东窗蛊里。

他虽然不修炎道,但这些蛊方对于他今后炼蛊,甚至修行,都有旁敲侧击的启示作用。尤其是其中,关于炎瞳蛊的炼制蛊方,很有借鉴的价值。

按照这个蛊方所讲,四转炎瞳蛊是从三转的火眼蛊,搭配目击蛊,以及相干的一些炼蛊材料,合炼而成的。

三转的火眼蛊,方源知道,乃是一次性消耗蛊,作为侦察之用。它能将双眼改造成火眼,拥有洞穿迷雾之能。并非每次都能成功,一旦失败,就会眼瞎。

目击蛊,方源也清楚,黑家大军中的浩激流手中就有一只。浩激流曾用目击蛊,结合四转的换位蛊,一齐搭配使用。

方源将炎瞳蛊收起来,他没有打算走炎道。

炎瞳蛊和他本身的流派,并不相符。攻击手法,虽然方便,也很有局限性,十分依赖目光的接触。

这个世界上,千奇百怪的蛊虫太多。有无数的方法,可以隔绝视线了。

没有最强的蛊虫,只有最强的蛊师。

蛊虫只是大道载体,本质上只是工具。是蛊师将它们搭配起来使用,形成远超一般的效果。一些搭配效果更加卓绝的,难以破解的,便是杀招。

“这么说来,这应该就是我在王庭福地中,收获的第一份蛊师传承了。”方源想一想,感觉很有意思。

王庭福地中,埋藏有许许多多的蛊师传承。

因为这里有最得天独厚的环境,很多设立在外界的传承,还未等到有缘之人,就被天灾兽祸给破坏了。

再加上历届进入王庭福地的蛊师,都是经过战争洗礼的豪杰。就算不是英雄豪杰,那么至少也有两把刷子。

因此,使得王庭福地中的传承,十分繁多。只要是有缘人,就能有所收获。

方源将所得的三只蛊虫,都收入空窍,又将石碑摧毁成碎末。

最后,他接着搜寻。果然找到一处洞口,他顺着洞口进入地底深处,在深达三十丈有余的地洞中,他找到了六只幽火龙蟒的蛇蛋。

这让他有些犯难了。

如果是幼小的幽火龙蟒,那么方源可以摄走,带到狐仙福地中去,任凭它们自由捕猎。

但幽火龙蟒的蛇蛋,并不容易孵化。须得用幽火和蟒血日夜浸润,小龙蟒破壳而出之后,还得接受幽火龙蟒的言传身教,学会如何运用自身的力量去捕猎。

方源可没有这个闲情逸致,浪费宝贵的时间,去孵化这么几颗蛇蛋。

没办法,他只好将这些蛇蛋先收起来。随后,他钻出闷热的地洞,再不留恋此地,飞到高空,继续赶路。

ps:今天有重要的远方亲戚来,迟到了,十分抱歉!

\end{this_body}

