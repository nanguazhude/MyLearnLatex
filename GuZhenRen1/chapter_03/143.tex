\newsection{渺小者的前行}    %第一百四十一节:渺小者的前行

\begin{this_body}

车辚辚,马萧萧。

浩荡的队伍,多达百万,一路向着北原中央的王庭地区前行。

风雪在屋外肆虐,压得大胃马都抬不起头来。

地上的白雪,深可及膝,令人每一步的跋涉都比往常要艰难得多。

一只只的蓝田蛞蝓蛊,在蹒跚而行。它们的肚子里塞满了各类物资,将它们原本只有大象三倍大小的体型,撑到有小山丘般巨大。

一队队的马车,跟在蓝天蛞蝓蛊的身后。后者为这些凡人遮挡迎面而来的风雪,浑身都挂满了冰棱。

为了防止蛞蝓蛊冻死,平均有三位蛊师,负责一头蛞蝓蛊,专门给它们剔除身上的冰霜,同时利用炎道蛊虫取暖,维持体温。

大军向着王庭方向,一路前行。越来越大的风雪,让大军的速度越加缓慢。

大量的凡人奴隶,倒在路途中,再也起不来。

黑楼兰虽然可以下令蛊师救助他们,但却没有这么做。

尽管王庭福地,地域宽广,足以容纳五百万人。但在黑楼兰看来,福地中的这些资源都是他自己的,为什么要分给卑贱的奴隶?

每多出一个人,他要分出去的利益就多一分。

趁着风雪,故意牺牲大批无用的凡人,本来就是历代王庭之主的潜规则。

凡人的性命,不值得珍惜。他们就像是杂草,等到雪灾过后,就会迅速生长,然后蔓延,再然后像是蝗虫啃噬北原的单薄的资源,直到下一场十年雪灾的来临。

冷风又增大了,人们顶着风力闷头前行。

方源生出在大蜥屋蛊中,都能听到窗外呼啸的风声。

大蜥屋蛊内,温暖如春。催动它在这样的环境下前行,消耗的真元比之前要足足多出五六倍来。

不过对于方源来讲,单单一个五转巅峰的九成空窍,支撑这样的消耗,绰绰有余得很。

更何况,就在最近,他的第二空窍,也提升到了五转高阶的程度。

方源来到窗前,目光穿过半透明的密封晶窗,投射到左前方。

在那里,是马家的队伍。

马英杰继承了族长之位,马鸿运也出现了,甚至赵怜云就在他的身边。

这点,方源已经暗中打探过,并且叮嘱葛家的人,对马鸿运和赵怜云暗中关照。

记忆中,马鸿运在八十八角真阳楼中,收获过巨阳仙尊的一项传承。在方源接下来的计划中,他将是一个非常有用的棋子。至于赵怜云,现在还是个小孩子,没有任何威胁可言。同时又和马鸿运走得很近,方源打算先观察观察。

“五百年前世,马鸿运出现了。现在,尽管有我的影响,马鸿运同样还是出现了。那么在未来,他和赵怜云是否还能有前世那样的成就呢?”

经历了重生之后,方源对历史的改变这个命题,有一种源自内心最深处的兴趣。

历史的洪流,有惯性,也有变化。

以他亲身经历看来,地球上的蝴蝶理论,显得有些偏颇了。

五百年前世,马鸿运被赐姓,允许蛊师修行。是因为他在野外获得了舍利蛊,贡献给了马英杰。

如今,他则是因为黑楼兰的逼迫,导致马英杰做出了一个决定。这个决定,再次造就出了马鸿运。

过程不同,但结果相同。

这个眼前的事理,让方源沉思,让他不由自主地联想起一个词,那就是――命运!

命运这个词,远比宇和宙更加神秘飘渺。

传闻中,蛊师流派中似乎有过运道这个流派,但时至今日,谁也无法确定。

不过,和命运有所牵扯的大人物,不在少数。

《人祖传》中,就明确记载了宿命蛊。

天庭二代仙尊,智道的创始人,就掌握此蛊,算计了后世三位魔尊。

方源在三王福地时,被地灵告知――红莲魔尊其实是个大英雄,打坏了宿命的束缚,让天下苍生掌握自己的命运。

甚至,方源还在前世隐隐听到这样的传闻:巨阳仙尊就掌握着运道的蛊虫,因此修行路上鸿运齐天,屡屡避灾迎福。

“这个世界上,真的有一条命运的丝线,将所有的苍生都紧紧联系么?”方源不禁陷入遐想之中。

前世五百年,他虽然成为蛊仙,但却只是揭开了这方世界的奥秘的一角。

不管是前世,还是今生,他越是前行,变得越强大,越是感到自身的渺小和无知。

他越是感到渺小无知,前进的乐趣就更大,他就越是要前行!

“相比较这个世界,我等就是蝼蚁啊……”方源的血液里,骄傲和谦虚,偏执和通达一直并存着。

收起一时泛滥的思绪,方源将注意力集中当下。

“王庭福地拒绝蛊仙进入,我已经是五转巅峰修为,也许这是第一次,也是最后一次进驻王庭,亲身接触到八十八角真阳楼。呵呵,也许我能够在楼中,收获到仙尊关于运道的传承,也说不定呢。”

“但是,这次黑楼兰主动奔赴暖沼谷,强逼马家招降,此举相当古怪啊。”方源目光沉凝起来。

马家已经大败亏输,又是黄金血脉,黑楼兰这样做,是为了什么呢?

前世可以理解。

在五百年前世时,马家实力未衰,坚防固守,是个难啃的乌龟壳。黑楼兰无奈之下,才不得不招降之。

现在马家极度衰落,黑楼兰驱动大军,不辞辛苦地将马家逼降,如此处心积虑地打压马家,难道他和马家有私仇深恨?

方源微微地摇头。

没有任何的证据,能支持这个猜测。

“算了,或许这是黑楼兰的一时兴起,想要彰显他的功勋也说不定。这个只是细枝末节,我自身的实力才是永远的重点。”

想到这里,方源将心身投入空窍。

之前施展杀招四臂地王的伤势,已经恢复了。

关于杀招,他也进行了小幅度的改善。

将原本的土霸王蛊,改为风霸王蛊。其他搭配的蛊虫,也进行了微调。

如此一来,他就不需要脚踩大地,而是最好在风中作战。风越大,他的战力就越能发挥出来,施展杀招的后遗症就越小。

但方源仍旧不满意。

这只是一次遮掩和妥协,其实这个杀招的弊端,仍旧没有改变。

如果在无风的环境下作战,他催动杀招后的结果,不会比之前要好。

对于蛊师而言,禁风的手段太多了。

一旦他的这个弱点被公布,杀招将不再恐怖,对敌人的威胁将暴降谷底。

“其实就算此招改良的再好,我也不会满意。我的真正目的,是解决力道、奴道双修的弊端。四臂地王这个杀招,不过是初步成果而已。”

但这个成果,难逃变化道的藩篱。

方源要达到的目的,是彻底的,永久地改造肉身。而不是这种临时的形变。

然而,能达到这一步,已经耗尽了方源五百年的积累。

毕竟,方源前世是血道蛊仙,对于力道、奴道,算是旁敲侧击,只是广泛涉猎而已。

如果可能,方源也想速成血道蛊仙。但自从他重生以来,情况就不同了。他的本命蛊,不再是血道蛊虫。

成就蛊仙的关键之一,就是本命蛊。

本来,方源得了第二空窍,也有了新的机会。但那只关键的血道本命蛊,还埋藏在传承中,并未出世呢。

方源不可能枯等,局势逼人,他只有选择先强大自己,来应付接踵而至的考验,和潜伏在四面八方的敌人。

活下去,这才是第一要务!

方源也意识到,自己在力道和奴道上的底蕴不足。前世的广泛涉猎,让他能轻松地驾驭各道蛊虫,并且精通各道蛊虫间的精妙搭配。这其中,又以奴道造诣最为深厚。

但是要解决奴力双修这个偏僻的千古难题,要走在历史的前沿,做出创新和大胆的尝试,那么这些底蕴就不足了。

尽管方源现在,手握落魄谷的线索,兴许在获得落魄谷后,他就直接转修前景无比光明的魂道。

但方源从来不把希望,寄托在未来的某种可能上面。

即便他今后转修了魂道,那么关于奴力方面的努力,也将是他的一个宝贵的财富,对于他今后的修行有极大的帮助。

明白自己的不足,方源这些天都在广闻博览。

他利用那笔庞大的战功,换取了龙马精神、三心合魂等等杀招,同时还有大大小小数十套的力道小传承,以及四位奴道大师的心得,其中鼠疫、雷暴、豹突、马踏四大杀招,更是价值不菲。

方源眼界本就宽阔,这些天苦读冥思,关于奴道、力道上的认识和见解,比之前深邃了数倍。

和前世泛泛涉猎不同,今生他亲身体验,实践结合理论,触发了他无数的灵感。

但这些灵感,还远远不足以解决他的难题。

“其实要说到身体形态的变化,最早的记载就在《人祖传》中。人祖陷落于死境,为了救活自己的父亲,古月阴荒来到了成败山,并且斩杀了石人……”

方源忽然灵光一闪,信手翻开身边的《人祖传》。

这部蛊师世界的第一经典,蕴藏着无数的奥秘,即便是蛊仙这样的存在,也大多随手备份一本,时不时地翻看、感悟。(未完待续。请搜索,小说更好更新更快!)

------------

\end{this_body}

