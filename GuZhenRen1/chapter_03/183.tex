\newsection{演飞熊,墨瑶话虚道}    %第一百八十一节:演飞熊,墨瑶话虚道

\begin{this_body}

方源公认索要飞熊虚像蛊时,黑楼兰第一个反应就是断然拒绝。

这可是仙蛊啊!

就算是寻常的蛊仙,也苦求一只而不得。狼王就算再厉害,也不过是一届凡人,居然敢提出如此过分的要求?!

况且这飞熊虚像蛊,使用极为方便。

只要意念一动,随手一抛,便能使得蛊虫瞬间变化为飞熊虚像,进行战斗。

飞熊虚像的战力,虽然没有真正的荒兽飞熊那样恐怖,但绝对是半仙战力。拥有一只,便可傲视凡尘。

若是王庭之争决战时,黑楼兰手中有这么一只仙蛊。

哪里还能轮得到方源表现!

方源公然索要,甚为无礼。但黑楼兰愤怒之余,却又感觉到一种理所应当。

恐怕也只有常山阴这样的人,才有胆量当众开这个口吧。

再看手中的飞熊虚像蛊,黑楼兰渐渐冷静下来。

“不妙。飞熊虚像被我们打得濒死,这只仙蛊状态极差,不堪再用,必须得休养生息一段时日。”

“仙蛊受伤,就得用仙蛊疗伤。除此之外,就只有让它自我恢复。期间,需要喂食,不可间断。”

“如果我没记错的话,这只飞熊虚像蛊的食物,就是荒兽飞熊的血肉。这可难办了!我手中哪里有这种东西?唯有向部族求援才行。”

黑楼兰越是思量,心中的愤怒就越淡了。

他是十绝体之一的大力真武体。必须拥有力道仙蛊,才能晋升蛊仙,摆脱死亡的威胁。

飞熊虚像蛊并非他所需。他真正渴望的是一只力道仙蛊。

“我现在当务之急,是要寻找到力道仙蛊。仙蛊向来只会在某一层的最后一道关卡中,打通关卡的过程,我得靠众人之力。”

虽然黑楼兰并不惧怕常山阴,但如果他强行扣住飞熊虚像蛊,那么势必和狼王反目,生出隔阂。更加难以调动狼王之力。

反不如,将这仙蛊让给他,换来他的力量。真正的帮助到黑楼兰自己。

这样一来,还能照顾到他黑楼兰的名声,不至于旁人说他言而无信。

黑楼兰思考这么多,其实不过只是一瞬的时间。

“哼哼。”孙湿寒冷笑。“狼王大人。这可是仙蛊,你一句话就想拿走?之前我家族长是答应你五成的报酬,但并不代表我们一定要将仙蛊交给你。”

“看来我上次给你的教训,还不够啊。”方源闻言,眉头微微一挑,目光冰冷,杀机毫不掩饰,直冲孙湿寒而去。

孙湿寒面色骤变。情不自禁地向黑楼兰身边移动几步。

没有人怀疑,方源是否有直接动杀手的胆量。

狼王凶威。已经深入人心。

“哈哈哈。”黑楼兰仰头大笑,脚下一迈,走到方源的面前,一拍胸脯,“大丈夫行事,一言九鼎!之前许诺狼王五成,那就是五成。打通这关,我们也赖山阴老弟你的妙计。因此再多一成奖励,也不为过。只是仙蛊唯一,价值不可估量。你顶多占有六成,余下的四成,山阴老弟你打算如何弥补我们?”

“是啊,我们也出了大力气的!”

“我身负重伤,也激战不退,大家都看在眼里。水魔浩激流大人,甚至因此阵亡了!”

“我相信狼王大人,必定能给我们一个交代的。”

众人连声附和道,都是针对方源。

方源就算有实力,但是财帛动人心,眼看一只仙蛊就要落到他的手里,周围的人自然羡慕嫉妒恨,因此将苗头都集中在方源的身上。

众目睽睽之下,方源的眉头越皱越紧,最终缩成一个疙瘩。

黑楼兰当即暗笑:“狼王啊,狼王,就算你凶威赫赫,战力卓绝,到底也落到我的瓮中了。”

七天之后。

八十八角真阳楼,第七层,第九十关卡。

迷宫中,飞熊虚像仰天怒吼。

它浑身伤痕累累,口溢血迹,利齿被掰断了好几根,一只右眼也完全瞎了,硕大的眼珠子像是葡萄一般,挂在眼眶外面,十分狼狈。

而它的敌人,只有一位。

他身悬半空,背生六臂,和飞熊虚像小山般的体型相比起来,分外渺小。

但飞熊虚像如临大敌,甚至色厉内荏。双方激战已经持续了一刻多钟,它深刻地体会到眼前这个看似“渺小”存在的强大。

“飞熊虚像蛊,果然不愧是仙蛊!变化形成的飞熊虚像,有着不小的智慧,近乎狡诈。将来我对付强敌,完全可以任其进攻,减少我许多心神消耗。”

方源俯瞰脚下的飞熊虚像,心中评估着。

七天之前,黑楼兰集合众位强者,将第五层的最后一道关卡打通,获得飞熊虚像仙蛊。

方源毫不客气,当场索要。

其余蛊师嫉妒眼热,多加纠缠阻扰,方源便表示,愿意补贴另外一半的价值,态度强硬。最终黑楼兰将此蛊交给了方源。

有了此蛊,方源便在当天晚上,找到黑楼兰,向其借贷,背地里达成协议。

回去之后,他利用狐仙福地和宝黄天沟通,换取相应的蛊虫和食物,在第六天时将飞熊虚像蛊恢复到健康状态。

到了第七天,方源便暗中进入八十八角真阳楼的第七层,亲自试验飞熊虚像蛊,以及彻底完善的杀招六臂天尸王。

方源令飞熊虚像蛊毫无保留地攻击自己,而他自己则催动杀招六臂天尸王对战。

飞熊虚像蛊乃是仙蛊,有飞熊一半的战力。但却不是六臂天尸状态下的方源的对手。

对于这个结果,方源相当满意。

他在心中品味良久:“六臂天尸王这个杀招,的确是厉害!它以借力蛊为核心,可以借取天、地、水、火种种自然力量,加持自身,使得力量源源不断,滔滔不绝。”

“更妙的是,它令我暂时化身成僵尸之体。这样一来,身躯处于半死状态,恢复速度大涨,消除痛觉,仿佛力量无穷无尽,无所不能!”

如果是鲜活的肉体,不断发力,便会气虚力乏。但现在方源变成天尸,巧妙地规避了这种弊端。

这样的力量,这样的感觉,让方源迷醉。同时,也更让他警惕。

“我保持这样的状态,只能持续片刻时间。但激战时,往往让人全身心投入,而忘记时间的流逝。如此杀招,更令人感觉良好,容易沉迷。我万万不能忘记杀招的时间限制,否则后果就严重了!”

六臂天尸王这个杀招,虽然强悍,但有时间限制。超过限制,长时间催用,极有可能令蛊师彻底变成天尸。

程度之严重,更甚于十绝灾厄,即便是用阴阳转身蛊,也恢复不过来。

方源打算解决这个缺陷,但墨瑶提出的标准苛刻,一时间很难找到合适的试蛊之人。

“其实,这只飞熊虚像蛊的厉害,你还没有完全发挥出来。”这时,方源脑海中的墨瑶意志,忽然开了口。

“哦,怎么说?”

墨瑶对虚道似乎颇有了解:“虚道,讲究虚实相生。前期一转到五转期间,以虚为主,化虚来躲避攻击。因此虚道蛊师往往攻击不足,防守有余。但到了六转之后,虚道蛊仙就攻守兼备,虚实互转。防守时以身化虚,令一切攻势无功而返。攻击时由虚化实,令人防不胜防。”

“而飞熊虚像蛊,就是虚道仙蛊。斩杀荒兽飞熊,尽取其皮血骨魂所炼。单独的飞熊虚像蛊,战力只及真正飞熊的一半。但若给它配备其他蛊虫,譬如之前的斗空蛊、豪烈乱舞蛊、五行熊皮蛊等等,便能使得它的战力大幅度上涨。”

墨瑶的话,令方源频频点头。

他回忆之前闯关时的情景,正是因为飞熊虚像身上的蛊虫,令其战力暴涨,甚至曾经一度叫群雄束手无策。

但战后,斗空蛊被黑楼兰收入囊中。其余的蛊虫,也被他人瓜分。

方源单独秘密和飞熊虚像对战时,飞熊虚像身上已经没有一只蛊虫了。这才令方源战斗时,占据了上风。

他不禁有所领悟:“换个角度来讲,为飞熊虚像蛊搭配其他蛊虫,不就是许多蛊一起使用。这不就是杀招么?”

杀招,是蛊师匠心独运,动用多种蛊虫,一齐使用,形成一加一大于二的上佳效果。

墨瑶点头不已:“你领悟得不错,的确如此。不过,真正能配合飞熊虚像蛊的,只有虚道蛊虫。别的不说,单说一只弄虚蛊,就能令飞熊虚像蛊由实化虚,躲避一切攻击。”

“弄虚蛊?”方源心头不禁有一丝震动。

他完全能够想象,有了弄虚蛊配合的飞熊虚像,该是多么的麻烦。

这简直可以让飞熊虚像蛊的应用,发生某种程度上的质变。

墨瑶轻轻一笑,又道:“小子,你只要乖乖地将近水楼台还给灵缘斋,我便告诉你三处虚道传承的消息,其中一道甚至有可能是虚道蛊仙传承。”

“再说吧。”方源沉静下来,语气不咸不淡。

他对墨瑶意志的戒心,一直都没有消除过。

墨瑶虽然一直没有表现出敌意,但方源心中总有一股危机感觉。(未完待续。。。)

------------

\end{this_body}

