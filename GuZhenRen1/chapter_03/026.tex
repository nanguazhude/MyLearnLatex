\newsection{常山阴之死}    %第二十六节:常山阴之死

\begin{this_body}

%1
方源将心神探入空窍。

%2
白骨车轮进入空窍之后,漂浮在真金真元的海面之上,随着波涛,半沉半浮,一片死气沉沉的模样。

%3
蛊虫当然也会受伤、损毁、灭亡。

%4
“这只白骨车轮,已经濒临破碎,不能再用。除非我接下来能寻到骨竹蛊,再结合鬼火蛊,将它修复治疗。”方源在心中琢磨着。

%5
有许多治疗蛊,并不是针对人族身躯。

%6
有些蛊,比如狼烟蛊,专门用来治疗豺狼身上的伤势。又有些蛊,比如生铁蛊,是用来治疗锯齿金蜈的。

%7
骨竹蛊,再结合鬼火蛊,才能修复好白骨车轮的损伤,使之具备再战之能。

%8
“说起来,这白骨车轮蛊也是大名鼎鼎,出自八转魔道蛊仙沈桀骜之手。此人号称傲骨魔君。他天资卓绝,惊才艳艳,晋升六转时,苦于手中没有仙蛊,便想出一个杀招,名为白骨战车。白骨战车由白骨车轮等许多五转蛊组成,威力强悍,可媲美六转仙蛊!”

%9
“凭借这个奇思妙想,沈桀骜修到八转境界,又将杀招白骨战场发扬光大,结合了三只仙蛊,形成更厉害的大杀招白骨战场。他凭此纵横世间,不知屠戮了多少蛊仙,凶名广播,令正道一时束手。唉,我什么时候,也能修到这一境界?”

%10
方源前世五百年,是六转修为,离七转差了关键一步。后来炼了春秋蝉,遭到正道蛊仙围攻,不得不自爆身亡。

%11
每每想到血海老祖、傲骨魔君、幽魂魔尊这些人物,方源都不禁心驰神往。

%12
“男儿当世,自当如此,不受世俗拘束。纵横无忌,看谁不顺眼就杀谁。心恶时屠戮万物,心善时福泽苍生。天下皆随我心而动,主宰一切,将所有敢于反抗自己的敌人踩在脚下。这才是大自在,大畅快的人生啊!”

%13
方源在心中深深的感叹一声,又从怀中掏出皓珠蛊。

%14
皓珠蛊已经蒙尘,光芒黯淡。里面封印着定仙游,仙蛊逸散的气息也转弱。

%15
方源取出暗投蛊。

%16
此蛊和蒙尘蛊差不多形状。也是一个蚕茧模样,只是通体是幽深的黑色。

%17
方源调动真元,黑色的蚕茧便蠕动起来,数十根线头扭动而出,灵巧如蛇。攀上皓珠蛊。

%18
几个眨眼的功夫,皓珠蛊就被一层黑色的蚕茧裹住。

%19
此乃“明珠暗投”,亦是五域大战时,才研发出来的手段,专门用来遮掩蛊虫的气息。

%20
这样一来,定仙游的气息就更加的微弱了。

%21
“常山阴勇士,你是想封印了这只漂亮的玉蝶?”葛谣站在一旁。也渐渐看出了端倪。

%22
方源向她神秘地笑了笑,将黑不溜秋的圆珠收入怀中,又继续在战场上埋头寻找。

%23
这片战场,就是二十几年前。常山阴和哈突骨大战时留下来的。

%24
常山阴是四转巅峰蛊师,哈突骨则已经是五转初阶,同时又有一大帮的下属。

%25
他们两人原本是一同长大的玩伴,但是他们俩同时爱上的女人。最终选择了常山阴。从此仇怨结下,又因为后续种种。仇恨不断加深,最终只能用彼此的生命和鲜血来洗刷。

%26
哈突骨给常山阴的母亲下了毒后,常山阴为了找寻雪洗蛊,率领狼群深入腐毒草原。

%27
之后,常山阴在这里设下埋伏,哈突骨带着马贼一头扎进来时,大股的狼群从四面八方涌来。

%28
这一场惨烈的生死决战,打得昏天黑地。

%29
最终狼群尽没,而马匪也死的死,逃得逃。强弩之末的常山阴,和真元耗尽的哈突骨短兵相接,徒手搏杀。

%30
两人都杀红了眼,利用了一切所能利用的东西。

%31
他们扭打在一起,用牙咬,用手抠,拼尽全身所有的力气,然后双双瘫倒在地上,几乎连呼吸的力气都没有。

%32
这两个生死仇敌,曾经最亲密无比的伙伴,距离彼此只有两三步的距离,但是他们只能呼呼地喘着气,瞪着对方。

%33
他们都是强大的蛊师,一个是荣光耀目的英雄,一位是凶名赫赫的魔头,都失去了力气。在这一刻,脆弱得如同孩童,哪怕一只兔子跑过来,堵住他们鼻口,就能将他们窒息而亡。

%34
就这样僵持了片刻,哈突骨忽然哈哈大笑起来。他到底是五转蛊师,真元恢复速度比常山阴要快上一筹。

%35
他的真元首先恢复过来,足够他催动一次剧毒骨矛。

%36
眼看着骨矛射向自己,常山阴瞪圆了双眼,于绝境中迸发出奇迹般的一丝力气。

%37
靠着这点力气,他艰难地翻了半个身,但原本瞄准脑袋的剧毒骨矛,仍旧刺中了他的胸膛。

%38
剧烈的疼痛让常山阴发出怒吼。靠着用狼力蛊增幅过的力量,他掰断纤细的骨矛,握在手中,然后一步步地拖着身躯,挪移到哈突骨的身边。

%39
最终,常山阴将惨绿色的矛尖,插在了哈突骨的眼眶中,将这位毕生的死敌杀死。

%40
常山阴虽然胜了,但是骨矛上的剧毒,已经蔓延了他的全身。

%41
靠着刚刚恢复过来的一些真元,他催动了狼胎地葬蛊。

%42
此蛊乃是用一百零八头,不同种的怀孕母狼所炼,专门用来救命。只要有一丝气息,都能吊住。

%43
常山阴用了此蛊,钻入地中,陷入沉眠,苟延残喘。

%44
过了足足三十多年后,还是三转蛊师的马鸿运,被狼群追击到这里来。在走投无路的情况下,意外地发现了埋在地下的常山阴。

%45
马鸿运将常山阴救活之后,后者为了报答他的救命之恩,不仅帮助他击退了狼群,更为他效劳,成为四大将之一。在日后的草原争霸中,立下无数大功,将奴隶出身的马鸿运一举推向王庭之主的宝座。

%46
常山阴历经人生起伏,颇具传奇色彩,等到他重出江湖,他的故事就一直在北原广为流传,不是秘密。

%47
后来他又在马鸿运的帮助下,修行到七转蛊仙境界,成就“天狼将”的称号,更加位高权重。

%48
最终,他抵抗中洲侵略,战死沙场,他的后人就为其做传——这也是方源如此清楚事情过程的缘由。

%49
“嗯?找到了!”

%50
漫长的搜寻,终于有了结果。

%51
方源脚步一顿,发现了草地上的一条巨大狼尾。

%52
这条狼尾,沾满了泥泞,大半被毒草掩盖,几乎看不出来。若不是方源心中早有目标,又仔细搜索,根本不可能发现。

%53
“当初马鸿运就是在逃跑的路上,被这根狼尾绊倒。他拔出这根狼尾后,救活了常山阴,也救了他自己。”

%54
方源心潮澎湃,他抓起狼尾,用力将其拔出来。

%55
顿时土地翻腾,一只巨大的母狼身躯,双眼紧闭,浑身紫毛,肚皮雪白,被带了出来。

%56
它体型巨大,就算是它躺着,几乎也有一人之高。

%57
葛谣连忙跑过来,带着一脸的惊奇之色:“这是什么狼,怎么这么大?哎呀,好像是母狼,你看它肚子鼓鼓的,肯定怀孕了!”

%58
“这不是狼,而是一种蛊。”方源一边说着,一边从推杯换盏蛊中,取出锋利的匕首。

%59
他将匕首插在狼腹上,然后用力划开一道长长的口子。

%60
瞬时,鼓胀的狼肚破开,大量的羊水,混合着血水,从伤口处喷涌而出,将方源的下半身浇得湿透。

%61
葛谣倒是见机不妙,迅速往后蹦跳,避免了自己遭殃。

%62
然后她吃惊地张大嘴巴,叫道:“怎么狼胎里,竟然是一个人?!”

%63
和羊水一同流出来的,还有一个人。正是真正的常山阴!

%64
他双目紧闭,浑身伤势累累,尤其是胸膛处还插着半截骨矛。他浑身都是粘稠的羊水,表情痛苦,皮肤泛着惨绿色。

%65
方源迅速地蹲下来,然后伸出双手,似乎在察看常山阴全身的伤势,实则在常山阴的脖子上暗暗一捏。

%66
可怜常山阴这个英雄豪杰,成功斩杀了宿敌,又靠着蛊虫续命二十多年。本来再等十年左右,就会有命中的君主来解救他。但现在方源横插一脚,将这个日后的风云人物,大名鼎鼎的“天狼将”,未来七转的蛊仙杀死了。

%67
常山阴本就是奄奄一息,毫无意识,更谈不上防备,只剩下一丝微弱的气息。

%68
方源杀死他时,他的身体都没有颤动一下。更谈不上用意识,来引爆了蛊虫。

%69
方源将心神探入到他的空窍中,立即就发现里面的数只龟息蛊。

%70
龟息蛊也是存储蛊,和皓珠蛊差不多,都是用来封印蛊虫。

%71
常山阴在进入狼腹之前,为了防止体内的蛊虫饿死,就将它们一一封印在龟息蛊中。

%72
这些四转的蛊,像是椭圆的石头,比拳头要稍大一些。石头表面布满纹路,让人联想到乌龟壳。

%73
春秋蝉的气息一泄露,方源瞬时就将这些龟息蛊炼化。

%74
他将这些龟息蛊全部取出来,在葛谣好奇的目光注视下,都一一捏碎,露出里面的蛊虫。

%75
一共八只蛊虫,都隶属奴道,各个都是四转的珍稀蛊。其中个别的,甚至比普通的五转蛊还要珍贵。常山阴精心搭配,凭借此套蛊虫,在北原闯出赫赫名声。又借助此蛊,斩杀五转大敌。

%76
借助春秋蝉,方源将其全部收为己用。

%77
“如此一来,自己就有了一套精良的,源自北原本土的四转蛊了!”他的嘴角泛起微笑。

\end{this_body}

