\newsection{地丘传承}    %第九十一节:地丘传承

\begin{this_body}

墨线变化不定,持续了好一会儿功夫,这才渐渐定住,形成一副地形图。

图的中央,是一块隆起的土地。没有山峰陡峭,坡度很缓,上面张开一个豁口,仿佛通往地底的模样。

在这处地方,标注着两个字――地丘。

地丘四周,是沼泽和树林混杂的地形,在西南角上,还有一条河流。

在整个地形图的下方,还有四句――“土中蕴光,芒高万丈,百里天游,咏梅雪香。”

方源口中喃喃,咀嚼了半天,却参悟不透。

这四句,说是诗词,似是而非。说是蛊方,也有些模样。

但方源有一点可以确认,这四句谜题,很显然是故意留下来的线索。

更为奇妙的是,这地形图形成片刻后,又渐渐地从灰白石板上消失。

很快,方源手中的这块石板上面,就变得清白一片,空无一物。

但方源闭上双眼,却能轻松地回忆起这份地形图的每一个细节,丝毫不差。

这并非是他记忆力惊人,而是――

“画意蛊。这石板上,被种下过画意蛊。此蛊能形成图画景象,深入蛊师记忆深处,令蛊师永不忘怀。”

方源的双眼中,闪过一丝彻悟的光。

很显然,这是一个蛊师留下来的传承。

为了鉴定这些灰白石板,方源从葛家的族库当中,取得了许多一清二楚蛊,日光蛊,月光蛊等。

刚刚,他就动用了这些蛊虫。然后又很有技巧地灌注真元探测。

这些手法,都是鉴别灰白石板,特定的手段。

结果就是这些手段,成了开启这块灰白石板秘密的钥匙。

“伪造这个灰白石板的蛊师,不只是用了画意蛊。还用了其他蛊虫,才形成这种效果。这个蛊师留下传承,为了筛选出继承者,倒是花费了一番心思。”

方源笑了笑,没有想到居然会在这样的情况下,获得了一份蛊师传承的线索。

传承。是这个世界的文化特征之一。

不管是正道蛊师,还是魔道蛊师,都会选择留下传承,在这个天地中留下独属于自己的印记。

虽然幸运地获得了这个传承线索,但方源却并没有太多的惊喜。

五百年前世时,他也遭遇过许多类似的情况。现在早已经见怪不怪了。

绝大多数的蛊师,都会留下传承。

这样一来,就造成传承良莠不齐。有蛊仙传承,有四转五转蛊师传承,这些都是有看头的。但也有很多,是二转三转的传承,甚至一转蛊师也会留下传承。

再加上时光消磨。天灾人祸等等,很多蛊师探索传承,得到的结果,都是失望。

有些传承,早已经毁灭消亡。有些传承,被人捷足先登。有些传承,则是魔道传承,是被人精心设计过的陷阱,是心理阴暗的蛊师,在临死时的发泄。

“我现在忙得脱不开身。没有时间为了一个说不清楚的传承,放弃手中的计划,赶到远方去。再者,单凭这份地形图,我也不知道这个所谓的‘土丘’究竟在哪里呀。”

得了土丘传承图。只是一场小小的意外,很快方源就将其抛之脑后。

在接下来的日子里,他继续修行,同时开始炼蛊。

从之前两家的积累中,他得到了一份改良后的蛊方,方源觉得挺有意思。

从葛家族库中,他又取出一些蛊虫,花费了几天功夫,失败了两次后,将手中三转的鹰翼蛊,提升为四转的鹰扬蛊。

说起来,这只鹰翼蛊放在他手中,几乎就没用过。还是方源出了腐毒草原,来到红炎谷附近,葛家营地中时,碰到几家汇集开市,从市集上买来的。

狼王常山阴,可不是飞行的能手。方源自然不会轻易动用这项大师级的技艺。

这是一张底牌,将来一旦动用,必定会让世人大吃一惊。

常山阴失踪十几年,是一个相当棒的借口。谁都不知道他有什么际遇机缘,成为飞行大师,为什么不可能?

鹰扬蛊炼成的数天后,葛光亲自拜见方源,带来了最新的情报。

“马家已经彻底吞并了费家,成为天川英雄大会的主角……”

“猛丘英雄大会上,努尔家的代表是一位五转蛊师努尔图。”

“草府方面的赵家?嗯,那个赵怜云,马鸿运的妻子,日后成为智道蛊仙的奇女子,现在还不过是个稚嫩女童吧。”

“不管怎么说,马家这次大出风头,很显然是想大干一场,冲击王庭之主的宝座了。这和前世记忆,也是相符的。就是不知道马鸿运有没有出现?”

方源一边思索,一边回忆。

他模糊地记得,这次王庭之争,马家表现得极为强势,尤其是在前期,兵锋强盛,众志成城,屡破强敌。

但是树秀于林风必摧之,堆出于岸流必湍之。

马家出的风头太盛,先后被老资格的黄金家族盯上,几番恶战,虽然都得到了胜利,但都极其惨烈,元气大伤。

最后马家被黑楼兰逼住,八面重围。黑家人多势众,但马家防御森严,据险固守。

黑楼兰率众亲战,久攻不下,眼看大风雪就要来临,最终只能迫和。

马家臣服于黑家之后,获得了许多进入王庭的名额。马鸿运不知道走了什么狗屎运,居然也能进去。

正是借助了这次机遇,他获得了八十八角珍宝楼中的仙尊部分传承,从而日后崛起的资本。

这时,葛光开口,向方源请教道:“太上家老大人,如今各处英雄大会,已经进行得如火如荼。我们这边玉田英雄大会上,也是人雄层出不穷,高手相争,风起云涌。其中又以刘文武、黑楼兰两家实力最为雄厚,其他势力都被比了下去。”

“如今,月牙湖畔,只有我们一家势力留在这里。就算是大型部族,也都启程赶往玉田,参加英雄大会。经过这些天的休养生息,我们也已经消化了战果,稳定了局面。如果再不启程的话,时间上可就有些来不及了。”

方源点点头。

英雄大会是相互之间的试探,也是各方势力的合纵连横。

葛家虽然实力膨胀得厉害,但也只是个中型部族。如果参加不了英雄大会,脱离了游戏规则,将会受到排挤,影响很大。

葛家的族长虽然是葛光,但是自从方源担任了太上家老之后,他的决定就已经能主宰葛家的行动了。

这些天来,葛光等家族高层,也等得有些心焦了。

他们没有方源的前世记忆,不知道这次玉田英雄大会,会有一场精彩的龙争虎斗,因此结束的时间也是最晚的。

方源对此,早已经有所计划安排。

他摆摆手,对葛光道:“部族虽然稳定了局面,但只是表面现象。暗地里,还是人心浮动得很。真要作战,即便有中型部族的底蕴,却没有相匹配的实力。”

葛光垂首,一脸恭敬地聆听方源的训示。

方源继续道:“玉田英雄大会的争斗,才刚刚步入高潮而已,不急着去那里,我们要先去葱谷一趟。”

“葱谷?”葛光面露疑惑之色。

葱谷是一处类似月牙湖的地方,山谷广阔,长满青绿色的大葱,有独特的生态。

在那里,生存着大量的兽群,比月牙湖只多不少。当然,还有大量的野生蛊虫。

其中,就有一种闻名遐迩的二转蛊,葱爆蛊。

这种蛊,外形如葱,却非青白色,而是艳丽如火。一旦催动起来,它便散发出极其浓烈的气味。

野兽闻到这种气味,将会变得极其暴躁,展现出侵略性,容易攻击他人。

因此,葱谷是比月牙湖更加危险得多的地方。

怎么好端端的英雄大会不去参加,反而要去这等危险之地呢?

但紧接着,方源便说出了理由:“很久之前,我便在葱谷中放养了狼群。经过这些年的培养,应该也发展壮大了罢。”

“原来是这样!”葛光顿时眼前一亮。

奴道蛊师要培养出来,消耗的资源十分庞大,单单每天喂养野兽的食料,就是个庞大的数字。

这些天来,葛家为了照顾方源的狼群,消耗甚多,叫葛光有了十分沉痛、清晰的认知。

因此,很多奴道蛊师都会选择放养。

他们会选择一些环境恰当的地方,将兽群当做种子,放养在里面。

每隔一段时间,去查看验收。如果兽群壮大了,那就是收获。

当然了,收获的概率是比较低的,绝大部分情况下,是兽群受到削减,甚至会全军覆没。

但尽管如此,绝大多数的奴道蛊师,仍旧会选择这样做。

毕竟奴道的修行,对资源负担极大,能像方源这样,将狼群直接放养到福地中的凡人蛊师,能有多少呢?

方源这么一说,葛光顿时理解了。

“这么多年过去了,当初狼王的联系还在,但是狼群是多是少,我也不太知晓。但此次参加英雄大会,狼群自然是多多益善。有越多的狼群在手上,我们的底气也就越足啊。”

方源的话,让葛光连连点头,赞同地道:“太上家老言之有理,那我们何时启程?”

“就在今日。”方源道。(未完待续。。。)

------------

\end{this_body}

