\newsection{谋算艰难生不易}    %第一百四十九节:谋算艰难生不易

\begin{this_body}

%1
这是一座塔楼。

%2
形如春笋,挺拔瘦削,笔直伫立。

%3
塔身影影绰绰,七彩斑斓,仿佛是一团塔楼形状的彩墨液体,在不断晃动。

%4
“这才是八十八角真阳楼的初形,等到外界风暴万里、雪覆北原,王庭福地中一座座的小塔楼沉入地中,八十八角真阳楼才会一层层稳定下来。到那时,才能进入其中,进行探索。”方源站在居所的门口,仰头遥望着圆顶方向。

%5
此时,圣宫中一片欢腾惊叹。蛊师们雀跃欢叫,更有许多凡人跪在地上,五体投地,口中赞颂着巨阳仙尊的伟大。

%6
“但就算八十八角真阳楼成形,我也无法进入,因为我不是巨阳仙尊的后裔,身上没有他的血脉。要想进入其中,还得需要黑楼兰取得来客令。不急,真阳楼岂是那么容易闯的?黑楼兰迟早需要我这样的外援。”

%7
片刻后,方源收回目光,重新走回住处。

%8
将这处小殿的大门紧闭,门外的欢呼声立即减弱下去。

%9
来到专门的修行密室,方源端坐在蒲团上。

%10
“我现在的两大空窍,都已经晋升为五转巅峰。奴道方面有天青狼群,力道方面有杀招四臂风王。再配合我的战斗经验,已经算是凡人的巅峰了。但是要对付蛊仙……”

%11
方源眉头渐渐皱起。

%12
他深知仙凡之间的巨大差距,要以凡对仙,难如登天。

%13
现在对他而言。最好的情形是太白云生的脑海中,已经有了江山如故的仙蛊方。

%14
方源在北原外界不好动手,但是将来。在八十八角真阳楼中动手却是方便得很。利用蛊虫,窃取到太白云生的江如故、山如故二蛊,再将仙蛊方从他的脑海中盗取出来。

%15
有了这样的充足条件,方源完全可以跳出棋盘,直接扼杀掉未成仙的太白云生,利用琅琊地灵的第三次请求,要求他炼出江山如故蛊来。

%16
但是太白云生。作为一个凡人,脑海中拥有江山如故这个仙蛊方的可能性微乎其微。

%17
而且据前世隐约的传闻,江山如故是在他成就蛊仙时。天地感应,道纹相吸,两只蛊自发合并,形成的仙蛊。

%18
方源生性谨慎。从不把全部希望寄托在缥缈的未来之上。

%19
“如果情况糟糕。那么我就要对付蛊仙太白云生。到那时,我能依靠的就只有自己。天青狼群、四臂风王,却是显然不够的。”

%20
方源的实力,已经是凡人的顶峰。尽管兼修奴力二道,一直都没有解决最大的难题。但就算如此,他是五转强者当中,也是第一流的层次了。

%21
放在南疆,也就是商家天才族长商燕飞、武家族长武姬娘娘这种层次的人物。

%22
利用前世的经验。和重生的优势,再加上一系列的冒险。几番在生死线上的挣扎,方源的高速成长是可以惊骇天下了。

%23
但是这种实力,面对蛊仙,哪怕是一个刚刚晋升的蛊仙,也是不够看的。

%24
“要解决眼前的这个大难题,大约就只有两种方法。”

%25
“第一,是搜寻八十八角真阳楼,从中寻找到强大的蛊虫,或者独特的手段。”

%26
“第二,是暗中布置,尽量谋算太白云生,为真正动手的那一刻,做充分的准备。”

%27
“第三,是解决奴力兼修的难题,尽最大可能令自身的战力,再做增长。”

%28
方源静静地盘算着。

%29
他心里面很清楚,四臂风王已经将他的前世底蕴消耗殆尽了。毕竟他前世,擅长的是血道,而在奴力两道只是泛泛涉猎。

%30
所谓谋算蛊仙的布局和手段,也大多上不了台面,并不靠谱。方源前世也成功晋升过蛊仙,知道凡人蜕变蛊仙,会从头到脚,从内而外地洗礼一遍,甚至相貌上都会发生蜕变。

%31
真正能够寄予厚望的,还是八十八角真阳楼!

%32
日子一天天过去,天空中金光和银辉交相轮替。

%33
八十八角真阳楼还在不断酝酿,彩霞渐涨,最终将大半个圣宫都沐浴在霞光当中。

%34
王庭福地里鸟语花香,清风入沐。而在外界北原,却已经又连续下了一个多月的雪。

%35
风雪狂暴,寒风刺骨,大雪飘飞。

%36
天和地白茫茫一片,视野迷蒙。

%37
树木早就凋敝,被冻成一株株冰桩。丘峦被披上一层厚厚的雪衣,谷隘也被深雪堆满。

%38
兽群死伤无数,即便是生命力旺盛的杂草,也被霜冻成冰。

%39
人只要吐出一口吐沫,顷刻之间,便能化成冰疙瘩。

%40
但天无绝人之路,冥冥中,总会暗留一线生机。

%41
在北原各个地方,也有部族残留着。这些部族或大或小,靠着诸如暖沼谷这样的地利来抗拒风雪,在暴风雪中残喘生息。

%42
坐拥红炎谷的蛮家部族,便是其中之一。

%43
议事堂。

%44
家老分列两旁,蛮家族长蛮图则高居主位。

%45
三转外姓家老石武,跪倒在地:“族长大人,属下前来请罪。”

%46
“哦,石武你何罪之有?”蛮图微笑着问道。

%47
“启禀族长大人,在下负责看守的丁字号元泉,于昨日突然冰冻干涸,已经彻底荒废。”石武恭声回答道。

%48
“呵呵呵,无妨。”蛮图轻轻摆手,事实上他早已经知道这个消息。

%49
“石武家老起来吧,如今是十年大灾,元泉都会有突然冰冻干涸的可能。这不是你的罪责。”

%50
“族长大人海量宽容,叫在下惭愧!”石武作感激涕零状,心中则松了一口气。

%51
他姓石,对于蛮家而言,就是半个外人。虽然他也娶了蛮家的妻子。但总归还是被排挤着。

%52
他负责看守的元泉,出了问题,这种事情可大可小。不过幸好。此代的蛮家族长的确是开明之君,没有追究。

%53
“元泉之事,却也是家族大事。蛮多,如今我族内还有多少口元泉?”蛮图微皱眉头,开口问道。

%54
蛮多乃是蛮图的三儿子,体型又瘦又小,但精明狡诈。一直辅助蛮图,处理族中的内务。

%55
他立即答道:“父亲大人,如今红炎谷中。还有甲、乙、丙、戊、己、庚、辛七口元泉。其中庚、辛这两口元泉,本来就接近干涸,支撑不了一个月。”

%56
“嗯。”蛮图点点头,“也就是说。一个月之后。我们就只有甲、乙、丙、戊、己这五口元泉了。诸位家老怎么看?”

%57
“族长大人,此事难办得很。此乃天灾,人力难以回天,不妨缩减蛊师们的元石供给。”

%58
“红炎谷的这八口元泉,乃是我族这十年来,辛苦积累下来的。今年这个情况,比本族历史上都要强。这是多亏了族长大人您的英明领导,无需担忧。其他部族比我们情况要更糟。”

%59
“就算是元泉全部干涸了,又能怎样?等我们捱过这十年雪灾。届时否极泰来,新的元泉将接连冒出来的。”

%60
家老们你一言,我一语,纷纷发表自己的看法。

%61
蛮图点点头,安静地听完这些话后,才道:“元泉乃是蛊师修行的重中之重,不可轻忽。没有元石,蛊师的修为就难以推进。雪灾才刚刚开始,不久之后,就会有雪怪出没了。红炎谷的防御,真正依靠的还是蛊师。”

%62
说到这里,他顿了一顿,沉吟道:“这样吧,就将族库中的泉蛋蛊取出一只来种下去。就将这口新元泉,命为丁字号元泉。”

%63
“族长大人,泉蛋蛊可是五转蛊啊。”有家老心疼了,表示异议。

%64
蛮图扬起眉头:“五转蛊又如何?你们须知,有了元泉的供给,才能令我族蛊师不断进步。才能令他们保持战力,将来抵御雪怪时,更能保存自身。只要我族的蛊师们减员不多,我族的元气就能保住。等到雪灾过后,春暖花开时,便是我族大展宏图之际!”

%65
大展宏图……

%66
听了这话,众家老们纷纷眼前闪光。

%67
蛮图励精图治,蛮家部族在他的领导之下,不断壮大,这些年来已经吞并了不少中小部族。如今蛮家部族坐拥红炎谷,已然是一方霸主。

%68
但此刻众人听蛮图的话音,这位蛮家族长还想更进一步。如今的蛮家还满足不了他的野望!

%69
那提出异议的家老连忙请罪:“族长大人英明决断,在下只有俯首帖耳,甘效死命。”

%70
“愿为族长效死命!”其余家老随后齐道。

%71
“我有诸位,霸业可图。”蛮图大笑一阵,神色一定,下令道,“新的丁字号元泉就由石武家老负责。”

%72
众人无不向石武投来羡慕嫉妒的目光。

%73
看管一个元泉,可是一个肥差。

%74
石武这次真的感动了,眼眶泛红:“在下能得族长如此看中,必定竭尽心力,以报万一!”

%75
众人又商议了一会儿,这才散去。

%76
石武并未离开,而是跟随蛮多来到族库,领取了泉蛋蛊。

%77
“这便是泉蛋蛊吗?”。石武双手捧着,新奇地望着,语调都在颤抖。

%78
很快,他神色一定,向身旁的蛮多躬身行礼道:“还请三公子指教在下,如何才能催动得了这只蛊虫。”

%79
蛮多见他识时务,笑起来:“指教不敢当。这泉蛋蛊高达五转,所耗真元极多,就算是五转中阶的蛊师,也不能一次催动得成。石武家老你是三转修为,想要催动它,还得搭配持久蛊、又续蛊。但最快也要耗费大半个月的功夫,才能种下元泉。家老你无须担忧,这段时间我会在一旁协助你的。”

%80
“在下感激不尽。”石武连忙答谢。

\end{this_body}

