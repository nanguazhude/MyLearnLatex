\newsection{再获五转蛊}    %第四十三节:再获五转蛊

\begin{this_body}



%1
鬼王一路疾飞,顺着隐隐约约的仙蛊气息,来到一株雪柳下。

%2
“仙蛊的气息就停留在这里,再往前就没有了!”鬼王目光急切,将雪柳搜了个遍。

%3
“没有?没有仙蛊!这雪柳上的雪洗蛊都被采摘了,明显是有人来过。但为什么们没有仙蛊呢?若是仙蛊继续移动,气息自然也会转移。但这股气息就终止在这里,难道说仙蛊死了?”鬼王心中冒出这个猜测。

%4
但他不肯相信这个猜测,动用侦察蛊,将这附近的地域翻了个底朝天。

%5
“没有,还是没有!”鬼王恨恨咬牙,心中充满了遗憾和不甘。

%6
“等一等!”忽然间,他目光狰狞起来,想到了另外一种可能,“这腐毒草原中央,乃是紫毒福地,里面居住着七转蛊仙毒蝎娘子。难道是她取走了仙蛊?凡人的空窍,无法承载仙蛊,但是蛊仙可以啊。仙蛊进入了空窍,气息就不会逸散,这样一来,一切也能解释得通了。”

%7
“这么说来,毒蝎娘子安坐家中,就有仙蛊送到门口?可恨,可恶啊!”鬼王气得跺脚,他宁愿相信毒蝎娘子夺得了仙蛊,也不愿想仙蛊死亡的可能。

%8
但他根本没有想到,真正的仙蛊并没有毁灭,而是被方源用铁棺蛊暂时封印住了全部气息。然后顺着原路返回,埋在了半路中。

%9
鬼王顺着气息,一路往前。方源故意分段封印,蛊虫的气息一弱再弱,这就形成了思维惯性。

%10
鬼王总想着向前飞,根本就没有料到,在他来的路段中的某个位置,定仙游蛊就深深地埋着。

%11
当他想到毒蝎娘子的时候。嫉妒、遗恨之情,更让他钻进了死胡同。

%12
“毒蝎娘子乃是七转蛊仙,实力强大。我能召集花海三仙、红玉散人去攻伐琅琊福地,那是因为分别许诺了好处。却没有号召力,令他们一起攻杀紫毒福地。可恨至极!如果我早来十几日,说不定就能将仙蛊弄到手。”

%13
“算算时间,又要到紫毒福地开启门户,排泄毒气的时候。我可不是毒蝎娘子的对手,还是先行离开罢。”

%14
鬼王一跺脚。飞上高空,钻入阴云。

%15
阴云滚滚,鬼王不甘地注视了好一会儿,这才朝着他的老巢飞去。

%16
……

%17
今夜注定是多事的,葛家的牧场上。众人又将目光集中在方源的身上。

%18
如果方源杀了葛谣,取了她的蛊虫,那么蛛丝马迹蛊就会令其暴露。

%19
方源之前的谎言,也会被彻底的揭穿——你既然没有见过葛谣,又如何拥有她的蛊虫呢?

%20
到那时,任何的解释,都是说不通的超级微信!

%21
“竖子!”葛家老族长狠狠地瞪着葛光。愤怒之极,“常山阴恩人,如此正直宽厚,你怎么还要怀疑?!快给我跪下。向恩人磕头请罪!”

%22
“阿爸。”葛光着实惊了一下,没料到葛家老族长如此反应。

%23
他不是一直想要为妹妹报仇的吗?他想漏了一点,自己是在为他查漏补缺,没有做错啊。

%24
一旁的蛮家父子。倒是作壁上观起来。

%25
“葛老哥,贵公子言语凿凿。的确是有蛛丝马迹蛊吧?”方源面色十分平静,目光清明如水,“那就请你亮出来,当众催动一下罢。”

%26
“这个……”葛家老族长迟疑了。

%27
“葛老哥,你既然有证明我清白的手段,为什么要一直藏着掖着呢?哈哈哈,我高兴还来不及呢。”方源温和地笑着。

%28
葛家老族长察言观色了一阵,又看向一旁的蛮家父子。蛮家父子一直保持了沉默,好像成了旁观者,但是目光俱都是意味深长。

%29
“也罢,既然常山阴恩人执意如此,那老朽就得罪了。”葛家老族长终于咬了咬,取出蛛丝马迹蛊。

%30
此蛊形如黑蜘蛛,有拳头大小,身躯饱满,八只触脚黑毛绒绒,触脚尖端坚硬油亮,宛若小小的马蹄。

%31
葛家老族长向其灌输真元,蛛丝马迹蛊浑身缓缓地绽放出白色的微光。只要方圆五百里内,有蛊师动用标记过的蛊虫,它都会绽放红光,指明方向。如果蛊师将蛊虫一直藏在空窍当中,那它的侦测范围,只有方圆一千步。然而至始至终它都停在葛家老族长的手中,没有任何异变。

%32
看到此景,葛光扑通一声,直接跪倒在地上,向方源叩首:“常山阴叔叔,小侄错了!求证心切,冒犯了您。请您责罚吧!”

%33
“请起吧,我还要谢谢你,给我洗净了冤屈。你何错之有?”方源微微带笑,连忙上前一步,扶起葛光。

%34
时光回溯到当初,方源杀死葛谣的那晚。

%35
少女在临死前,向他哭泣:“常山阴!我不知道,我怎么挡了你成功的路。但就算是你杀了我,我也不恨你。也许你是想复仇吧?我一身的蛊虫,都留给你,希望能给你的成功带来一丝帮助。”

%36
“咳咳咳。”少女咳出满嘴的血迹,她惨然而笑,对方源哀求道,“我就要死了。在我临死之前,有一个小小的请求。希望你能抱抱我。我好想你温暖的怀抱……”

%37
但方源一动不动,目光冷冽地看着少女。

%38
他注视着少女,看着她脸上的表情一点点的坚硬,生命一丝丝的逝去。

%39
最终,花一般的少女成了一具冰冷的尸体。

%40
方源看着葛谣的面庞,陷入了良久的沉默。

%41
“居然将蛊虫全部留给了我?什么意思?”

%42
“她的确爱上了常山阴不假,但我杀死她,她能不恨我?她的爱,不过是少女情怀,几天酝酿。她的恨,却是丧失性命,冤杀之恨。孰轻孰重,一目了然啊。”

%43
“嘿!这少女终究还是太年轻,演技差得多,说话间眼中的恨意怎么能瞒得过我?我虽然缺少蛊虫,她一身的蛊虫也是精品,故意留给我……稳妥起见,我还是不能取。”

%44
接着,方源心念一动,毒须狼出动,将葛谣的尸体吞食干净。

%45
至始至终,他都没有碰一下葛谣空窍中的蛊虫。

%46
看到这个结果,葛家老族长也是吐出一口浊气,无限赞赏地看着方源:“常山阴兄弟,老朽今晚算是见识了。你不愧是草原上的大英雄,你的品行就像是今夜的月光,清纯如水,毫无杂质。再肮脏的地面,也不会使月光污俗。再浓厚的阴云,也不会遮蔽月光太久。我们葛家欠你良多,小儿无知莽撞,却还要怀疑你。我们葛家就只有这只五转的蛛丝马迹蛊,算是对今天的赔礼,请您一定要收下,否则老朽一生良心难安啊!”

%47
五转蛊难求,在众人惊讶的目光中,葛家老族长将蛛丝马迹蛊,送到方源的手上。

%48
方源推脱几次,但葛家老族长执意如此,他只好“勉为其难”地收下。

%49
这样一来,他就有了第二只来自北原本土的五转蛊。

%50
之后,众人继续酒宴,一直进行到下半夜,这才其乐融融地在月下分别。

%51
蛮图热情地邀请方源,到他家族做客。而方源则表明,自己不久后就要启程,去参加英雄大会。不过在临走之前,定会前去蛮家拜访的。

%52
看着方源和葛家父子骑着驼狼,渐渐远去的背影,蛮图脸上的笑容渐渐消退,变得难看起来。

%53
“看来这葛谣,恐怕真是死了。”蛮图语气沉郁。

%54
“父亲大人,不必忧愁。”一旁的蛮多,则冷笑一声,“这葛家想要借住红炎谷,有求于我们,一定跑不掉的。”

%55
被儿子这么一开导,蛮图脸色稍霁,他拍拍蛮多的肩膀:“你这点看得清,为父有些执迷了。这些年来,蛮家不断扩张,有你许多的功劳。可惜你只有丙等,资质不佳啊。今后父亲退位,让你大哥接管蛮家,你也要好好辅佐他。”

%56
“是,父亲你就放心吧。”蛮多应答得十分干脆、响亮,心中却在冷哼。

%57
自己也是父亲的儿子,凭什么就不能争夺族长之位,一定要让给大哥?资质不佳,难道就是不能成为族长的理由吗?

%58
不!

%59
“如果大哥登上族长之位,一定会整死我的。唉,真是可惜了。我原本向葛谣求婚,就是想将葛家成为我的妻族,成为我的势力。可恨天意弄人,葛谣居然死了!”

%60
……

%61
“逆子,给我跪下!”一到密室,只剩下父子二人,葛家老族长顿时沉下脸来,对葛光咆哮起来。

%62
“阿爸!”葛光吓了一大跳,虽然不明白自己父亲为何忽然勃然大怒,但他下意识地就先跪到了地上。

%63
“阿爸,我是你的儿子,你怎么打骂我都行,只要您能消气!但是孩儿有个小小的请求,阿爸消气之后,还请告诉儿子,您为什么这么生气。儿子以后一定改正,不使得阿爸您再生气了。”葛光道。

%64
葛家老族长嘿嘿冷笑,站在葛光的面前,手指着他的鼻梁:“我知道你心里很不服气,为父现在就告诉你原因,让你知道今夜是何等的危险!你以为蛮多求亲,是真的看上你妹妹的美色吗?”

%65
葛光楞了一下:“难道不是吗?葛谣可是我们一族的族花,多少少年一直在苦苦追求她。”

%66
“放屁!”葛家老族长咆哮一声,“美色不过是权柄上的浮雕,蛮多的背后是蛮图,他一直想要吞并我们葛家,所以蛮图才大力支持蛮多,迎娶你的妹妹。”

\end{this_body}

