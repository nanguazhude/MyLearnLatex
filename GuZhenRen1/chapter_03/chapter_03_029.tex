\newsection{常山阴,做我的丈夫吧!}    %第二十九节:常山阴,做我的丈夫吧!

\begin{this_body}



%1
广阔无垠的腐毒草原,到了夜晚,更加黑暗。

%2
风在耳边呼啸,不知哪里的狼嗥,远远传来,好像幽魂的哭泣。

%3
一团篝火,在草地上呼呼的燃烧着。

%4
葛谣靠着篝火,渐渐驱除身上的寒意。

%5
篝火上驾着锅,锅里烧着香气四溢的肉汤。

%6
葛谣吞咽了一下口水,仿佛更饿了些。终于她吞咽了一下口水,问向方源:“常山阴前辈,这锅肉汤可以吃了吗?”

%7
方源坐在少女的对面,两人之间隔着篝火。

%8
“不急,这肉干才刚刚下锅,需要等水煮沸。之后再等片刻,等到里面的肉块完全酥软,吃起来才有意思。”方源一边取出推杯换盏蛊,一边淡淡地答道。

%9
“哦,还要等这么久呀。”葛谣撇起嘴,俏丽的容颜在火光的照耀下,配上草原的长袍和精美的挂饰,更显出她的独特风情。

%10
但这样靓丽的风景,却不能使方源的目光为之一顿。

%11
他的目光集中在手中的推杯换盏蛊上。

%12
推杯换盏蛊是五转蛊。但是到了北原,压制成四转。论容量,它甚至弱于一些四转蛊。论喂养,它代价高昂,占据五转蛊虫的巅峰。论真元损耗,它也是差强人意。但方源为何别的不选,独独选择它,浪费大量的精力,投入大量的资源,来炼它呢?

%13
皆因它源自盗天魔尊之手,盗天魔尊偷天盗地,是历代所有尊者当中最富有的。

%14
他花费毕生的精力,寻找传说中的空穴。

%15
空穴最早记载在《人祖传》中,空穴是和光阴长河同等的秘禁之地。光阴长河中,有大量的宙道蛊虫。而空穴里。则生活着无数的宇道蛊虫。

%16
它贯通五域,隐藏在无人知道的地方,从空穴中推开门扉,就能顷刻到达世间其他地方。通往它的门扉,又称之为空门。空门无处不在,可以在狭小的指缝间,也可以广布浩瀚的天空。只要有空间的地方,就有空穴的门扉悍妃:宠冠天下。

%17
但是古往今来,极少有人能找到空穴。更遑论进入空穴。整个人族的历史中,似乎只有一人进出过。

%18
盗天魔尊研炼出推杯换盏蛊的秘方,就是想要将这推杯换盏蛊,塞入空穴当中,然后带出空穴中的野生蛊虫。

%19
但他失败了。却又成功了。

%20
四百多年后,他的福地被人挖掘而出,引动各方蛊仙的争抢。推杯换盏蛊的秘方,也广为传播,强大的功效很快令它风靡五域,被无数蛊仙推崇。

%21
方源先取出空窍中的金龙蛊。

%22
原本四转的金龙蛊,已经被压成三转。它飞出来后。又投入到推杯换盏蛊里。

%23
方源将大量的真元,都灌注到推杯换盏蛊中。

%24
这只上金下银的杯盏,立时绽放出璀璨的金银光芒,缓缓地悬浮起来。

%25
方源抽回手掌。然后对着杯盏,缓缓虚推。

%26
推杯换盏蛊随之往前移动,随后它开始消失,先是边沿。然后消失大半,最后彻底消失在空气之中。

%27
葛谣不禁站起身来。瞪大了双眼,惊奇地看着这一幕。

%28
与此同时,在遥远的中洲,狐仙福地中。

%29
小狐仙忽有所感,一个挪移,来到荡魂行宫的密室。

%30
密室中,一只推杯换盏蛊,绽放着耀眼的光辉,缓缓悬浮而起。像是被无形的力量牵引,它慢慢前行,最终消失在空气里。

%31
当推杯换盏蛊完全消失之后,方源又将手掌慢慢平摊,真元继续狂催。

%32
忽然,就有一点金银之光,在他的手掌上空绽放。

%33
随后,葛谣就看到空气中,先是露出一只杯盏的边沿,然后是一半的杯子,最终整个杯子从空中出现。

%34
最后光芒消散,杯盏样的蛊虫,缓缓地落到方源的手掌中。

%35
“成了。”方源低声喃喃一句,看到这个杯盏时,他就知道这事没有脱离自己的掌控。

%36
“常山阴前辈,你在做什么呀?咦,这蛊好像有些不对劲。”葛谣走了几步,来到方源的身边,好奇无比。

%37
“怎么不对劲了?”方源淡淡一笑,取出元石握在手心,快速回复着真元。

%38
葛谣没有说话,只是盯着推杯换盏蛊猛看,忽然她双眼骤亮,叫出声来:“这只蛊不一样了,原先的是上金下银,如今的是上银下金。”

%39
方源哈哈一笑。

%40
没错!

%41
推杯换盏蛊并非只有一只,而是两只。

%42
这两只蛊一只上金下银杯,一只上银下金盏。两只组成一套,才是完整的推杯换盏蛊。方源临走前带了一只在身边,将另一只放在狐仙福地当中。

%43
当他灌输真元,两只推杯换盏蛊同时进入空穴,在空穴中完成交换之后。福地中的那只,来到方源的身边。而方源装载着金龙蛊的杯盏,则回到了狐仙福地里。

%44
昔日,盗天魔尊想要靠着推杯换盏蛊,盗取空穴中的蛊虫。他失败了,没有达成本来目的。但是推杯换盏蛊,却在另外一种程度上,获得了某种巨大的成功。

%45
依靠无所不在的空穴,一对推杯换盏蛊在空穴里完成交接,这就形成物资的运送。

%46
更可贵的是,它只有五转,不是唯一的仙蛊。

%47
在方源五百年前世,五域大混战期间,推杯换盏蛊成了各大势力都必备的蛊虫。就连蛊仙们,也都在争相使用它。

%48
方源从这只推杯换盏蛊中取出了一封信笺。

%49
信自然是小狐仙写的,说明了福地中的近况。

%50
方源在腐毒草原,不过五六天的时间,狐仙福地中则已经过去了一个月左右。

%51
信中言说,除了荡魂山外,福地一切都好。仙鹤门提出再次交易,小狐仙按照方源之前的叮嘱,婉言拒绝了。

%52
交易次数若是增多,仙鹤门发现方源不在福地的可能性就越大。方源人在北原。但放心不下福地。靠着推杯换盏蛊,往来书信,他就可以幕后操纵,不教别有用心之人得逞。

%53
方源看了信后,当即回了一封。

%54
葛谣看得云里雾里,她不认识中洲的文字。

%55
连同这封回信,方源又放进去三只四转蛊,分别是金缕衣蛊、横冲直撞蛊、骨翼蛊。

%56
装的东西越多,推杯换盏时消耗的真元就越多。反而和两只杯盏的距离。没有丝毫的关系。

%57
这是因为推杯换盏蛊,构思奇妙,借助了空穴这个奇妙的秘禁通道。

%58
方源刚刚那次,只是一个尝试。确信推杯换盏蛊能够正常运作,他就开始正式地将身上的南疆蛊虫。放回到狐仙福地中,交由小狐仙喂养。

%59
福地中,小狐仙趴在桌子上,睁着亮闪闪的大眼睛,盯着空气。

%60
桌上的推杯换盏蛊里的金龙蛊,已经被她第一时间取走。

%61
忽然,推杯换盏蛊漂浮起来。又钻入了空穴当中。随后,换成另一只,落到桌面上。

%62
小狐仙连忙取出推杯换盏蛊里的东西,看到方源的回信后。她开心极了,不禁欢叫一声:“主人的回信!”

%63
就这样再一次轮换,方源空窍中的真元,再次消耗得七七八八。

%64
他不得不再次手握元石。补充自身的真元。

%65
葛谣站在一旁,渐渐看出了些微端倪。好奇心旺盛的她。又不禁多问了几句。但方源却只是淡笑,没有正面回答她。

%66
“哼,神神秘秘的,有什么了不起啊。”少女撅起嘴,气鼓鼓地坐回到原来的位置上。

%67
她一屁股坐下,皱起好看的眉头,生气地盯着方源。

%68
方源理都不理她,这令葛谣更加气愤。

%69
她自幼便得父亲宠爱,是部族的族花,从未有人敢如此轻视她。偏偏方源一路上,将其视若无睹。

%70
许多少年郎对她的热情追求,更助长了她的骄傲脾气。

%71
葛谣又猛盯着方源看了一阵,方源恢复了真元,又在进行推杯换盏,仍旧没有理睬她。

%72
这位北原少女的闷气,反而渐渐消散了。

%73
“到底是常山阴,不是那些傻不愣登又肤浅卖弄的家伙能比的。在他眼里,我又是什么样的呢?只是萍水相逢的后辈吧。”

%74
想到这里,葛谣不禁有些心灰意冷,望着方源侧脸,渐渐不觉痴了。

%75
方源用了人皮蛊,换了一个面孔,北原人特有的样貌,无疑更符合葛谣的审美观。

%76
常山阴年轻时,就是常家部族少有的英俊少年。

%77
他的五官端正,鼻翼挺拔,棕色的眸子目光深邃,薄薄的嘴唇抿着,无声地显示出坚强的个性。

%78
他的双鬓已经微霜,流露出成熟男子的沧桑气息,对少女而言是一种强烈的吸引。

%79
篝火随风晃动,火光照得方源面孔或明或暗,更突显出此刻他坚定稳重的气度。

%80
葛谣的思绪渐渐飘散,她暗暗回想,方源究竟是个怎样的人呢?

%81
第一次见面时的惊骇,微笑时的温和,指点自己时的智慧,战斗时的无双豪勇,还有扒下自己皮肤的冷酷冰寒。

%82
这一幕幕,在少女的心中闪现,印象是如此的深刻,简直已经刻在了少女的内心深处!

%83
“那他的过去呢?”葛谣不禁又想。

%84
常山阴的过去,早已经成为一个关于英雄的传说,在北原大地上广为流传。

%85
无数人敬重他,爱戴他,看好他。

%86
他年少时,风头无两,是常家的未来希望。

%87
他很早就成名,一流的驭狼术叫人刮目相看。

%88
更关键的是,他正直公正,宽宏友善,从不欺压弱小,孝顺父母,帮助有困难的族人。同时也义气深重,不知多少次舍命保卫家园,为常家立下了汗马功劳。

%89
他娶得娇妻,却迎来童年挚友的背叛。命运的无情捉拿,让这个男人失去了母亲,失去了结义的兄弟,失去了娇妻,甚至险些彻底失去了生命。

%90
但他终究活过来。

%91
靠着自己的努力,从死亡的深渊中艰难地挣扎出来,创造了一个常人难以想象的奇迹!

%92
“这个男人肩上背负着无穷的痛楚,隐藏在暗处的是数不尽的伤痕啊。”葛谣想到这里,心中生出一股冲动,好想将方源抱在怀中,用自己的温柔来安慰这头受伤的孤狼,曾经的狼王。

%93
篝火摇曳,烧得木材噼啪作响。

%94
葛谣投注在方源脸上的目光,越加深情,渐渐不可自拔。

%95
在温暖的火光中,有一种情愫在少女的心中酝酿、生长。

%96
当方源完成这一轮的推杯换盏,又开始取出元石,汲取其中的真元时,葛谣暗自做了一个她人生中的最重大的决定。

%97
她忽然站起身来,对方源大喊道:“常山阴!做我的丈夫吧!”

%98
声音在深邃的夜里,在广阔的草原上传播,远去。

%99
“你说什么?”方源皱起眉头,纵然他有五百年经验,也没有料到少女忽变的心思。

%100
反应过来后,他展颜一笑:“不要闹了,小姑娘,我可是你的前辈。按照年龄,我大你二十多岁的人,我的儿子和你正好匹配。”

%101
“不,常山阴,我就要你!”

\end{this_body}

