\newsection{十年风雪两茫茫}    %第三十五节:十年风雪两茫茫

\begin{this_body}

葛家老族长从下属通报中,听到常山阴的这个名字时,就有了疑惑。※r />

现在酒宴上,得到方源的确认之后,他表现得十分激动。

一众家老也是纷纷发出惊叹声。

“常山阴?”葛光坐在一旁,语气疑惑。

“你还小,不清楚也是应当的。”葛家老族长叹了一口气,又吩咐道,“儿子,快向常山阴敬酒。他不仅是你的救命恩人,更是我们北原上的英雄!”

“老族长。”方源苦笑一声,放下手中的酒杯,“我不是什么英雄,只是一位落魄的流浪者罢了。也许是长生天的祝福,让我从死亡的边缘,侥幸偷生回来。但一睡二十年,醒来已经物是人非。我是个不孝子,无颜回到部族了。”

说到这里,方源垂下眼泪。

家老纷纷叹息。

葛家老族长连忙开口劝慰道:“常山阴恩人,你这是哪里的话。如果你还不是我们北原的英雄,在座的谁还能是?昔日,哈突骨的那帮马匪是多么猖狂,不知多少的部族遭到他们的打劫,弱小的部族甚至被他们杀光,连牛羊都不放过性命。”

“你杀了他们,是为北原铲除了一大害。你的老母亲是受到奸人所害,并非是你的不孝。正相反,你的德行和正值广为传颂,我们都知道得很。你能回来,是我们北原正道的福音啊。”

“族长大人说的是啊!”

“想不到阁下竟然就是常山阴,我们今天能见到英雄,真是荣幸。”

“不错,常山阴英雄回归,是我们正道的大幸事!”

家老们接连开口称赞。

葛光双眼发光,他这才知道。原来方源竟然有这么大的来头,这么多的故事。这更加剧了他对方源的敬佩之情。

“过去的事情,就让它过去吧。诸位,相逢就是有缘,咱们喝酒吧。”方源却不想深谈这些,他对常山阴的过去比较了解,但能避则避最为妥当。

他表现出一副苦闷沉郁的样子。

众人察言观色,也都不提过去,只说酒宴的乐事。

明确了常山阴的身份后。方源得到了热情的招待。

宴会从上午,一直举办到深夜。众多家老都喝得躺下来,若非方源故意装醉,恐怕也脱不了身。

到了第二天,葛家族长又是宴请方源。

“常山阴恩人。这是区区的薄礼,算是感谢你对小儿的救命之恩。请你一定要手下!”宴席开始之前,老族长便赠给方源一百万元石。

方源有些意外,没有料到有这么多的谢礼。

葛家不过中小型家族,财政拮据,从营地布置、众人服饰便可见端倪。

他现在一穷二白,这些北原的元石可算是解了他燃眉之急。当即收下来:“葛家族长,我救葛光,并非图这钱财。然而实不相瞒,我两袖清风。正缺元石。我就不客气了。葛家的厚意,我定有后报美女娇妻爱上我。”

听到方源的最后一句话,葛家老族长,葛光。以及一干家老都笑起来。

能结交到常山阴这样的强者,和这样的英雄拉上关系。也是他们这种中小型的部族,一直想要做的事情。

酒宴继续下去,氛围比昨天还要热烈。

昨天那是第一次见面,今天双方都熟稔,方源挨个敬酒,将每个家老的名字都记得清清楚楚。

这让所有的家老,都感到受宠若惊,对眼前的常山阴更加亲近。

宴席间,不免有人好奇方源的经历。

方源就将早已准备好的话,当众说出来。大概是哄骗葛谣的翻版,只是要比欺骗少女时的话,更加精准。

重点是说到自己的修为,因为受伤,从四转巅峰落到初阶。

他的话没有破绽,众人一边惊叹,一边感慨,对方源越加敬佩。

而方源则唏嘘不已,对过去的成就丝毫不放在心上,目光沧桑,语气萧索。

葛家人看着传说中的英雄,也有着这么痛楚的一面,这么活生生的一面,他们为此伤怀,他们为此同情,心中对方源更加亲近。

到了第三天,葛家仍旧继续宴请,表现出极高的热情。

这次的酒宴中,出现了一位陌生的家老。他负责葛家的情报,方源来到营地的时候,他则奉命率队出去,搜寻葛谣。

“唉,一切都怪我。我有一个刁蛮的小丫头,平素时我娇惯得太厉害了。这次居然逃婚!”葛家老族长叹息道。

“对了,常山阴大哥,你从腐毒草原中归来。不知道一路上,有没有见到过我的妹妹。”葛光询问道。

方源毫无犹豫,神情自然无比,从容答道:“很抱歉,我一路行来,与狼为伴。你们是我见到的第一批人,所以我倍感亲切。”

葛光也只是随便问问,本来就没有什么期待。

再者,腐毒草原那么大,方源碰不到也极其正常。碰见了,反而是件稀罕的事情。

“我这不懂事的妹妹,如今音信全无,真不知道她逃到哪里去了。唉……现在麻烦了,蛮家族长的二子蛮多,就是向她求亲的人。现在妹妹逃婚,蛮多得不到人,恐怕会迁怒我们葛家啊。”

葛光皱起眉头,长吁短叹。

其他家老的脸色也沉下来,这些天来蛮家方面的施压,越来越重。

众人并不知道,他们这三天来热情招待的常山阴,已经把葛谣杀害。

“树挪死,人挪活。葛家老哥,你何必留恋此处呢。距离大风雪,只剩下一年多的时间。贵族大可以向北面迁徙,参加英雄大会,依附黄金家族。这样一来,才能在大风雪来临时,躲入王庭福地当中啊。”方源开口劝道。

蛊师世界生存不易,环境残酷。

在南疆,人们立下山寨,容易吸引兽潮冲击。而北原无山可凭,就要遭受风雪洗礼。

北原每隔十年,都会有一场席卷整个北原的巨大风雪。

届时,连续数月,天昏地暗,到处飘雪。寒风凛冽,如刀子一般。白雪皑皑,漫天遮地。整个茫茫草原都会冻结成冰霜的世界三国听风录最新章节。

每一次的大风雪,都会造成大量的生灵灭亡。狼、狐狸、鹰,花草,以及人类,都不能幸免。

尤其是大风雪中,会生成大量的野兽蛊虫,具备强大的攻杀力量。

往往一场大风雪之后,北原的大型部族会被削成中型。中型削减成小型部族,人口损失,伤亡巨大。

葛家老族长长叹一声:“山阴老弟,我已经老了,没有年轻时候的雄心壮志。依附黄金家族,就会将我们葛家拖入到战争的漩涡当中。成功了好说,一旦失败的话,后果实在太严重了。我们葛家举步维艰,承受不起啊。”

“王庭之争,不是我们这等小部族能够参与的。其实这里环境不错,土地肥沃,水草丰美。尤其是这附近,有一处红炎谷,有地火喷涌。大风雪的时候,我族搬到谷内,就能捱过这道关了。”

但是红炎谷被这里的霸主势力蛮家,给牢牢的控制住。

葛家要进入红炎谷,就得和蛮家商讨。

蛮多向葛谣提亲,就是一个非常好的良机。为了整个部族的利益,牺牲一个女儿的婚姻幸福,对于葛家老族长,以及葛光而言,是非常好的交易。

为了集体,而牺牲个体利益,这是体制中最常见的现象。

但现在葛谣逃婚,音讯全无。蛮家那边一直在施压,想要人,但葛家又交不出来。

方源听到葛家族长的这话,就立即明白对方的心意――就是想要蛮家了。

他对此非常理解,心中则暗暗可惜。

英雄大会,就是他下一站的目的地。若是对方举族迁徙,他还能一起顺路前行,路上的风险会小很多。

提到大风雪还有蛮家,众人的兴致都低落下来。

“山阴老弟,你真的不想回归常家了吗?”葛家老族长问道。

“当然不能回去,回去之后,自己这个假冒的不更容易穿帮么!”方源心中这样想,口中则道,“唉,我如今境况,无颜面对父老乡亲。”

葛家老族长点点头,表示理解的同时,还有些同情。

常山阴一睡二十年,醒来之后物是人非。母亲死了,妻子改嫁给了他的兄弟。常家成了他的伤心处,一时间无法面对也是人之常情。

之前的酒宴上,方源也提到,要去往英雄大会,同时尽快地恢复修为。

“老弟,你若是真的想要参加英雄大会,单靠现在你手中的这些狼群可不行。不如盘桓几日,稍作休整罢。”老族长建议到。

方源点点头,没有客气:“我正有此意,只是这样一来,就叨扰诸位了。”

“哪里,你能留在这里做客,是我们的荣幸!”老族长哈哈一笑,表示欢迎。

葛光也笑道:“再过几日,就是附近几个部族一起开市的好日子。常叔叔可以去看看。”

就这样,方源住了下来。

几天后,葛家拔营,一路往西南方向,和其他几大部族交汇在一起。

一个巨大的市集,迅速地在部族之间形成了。

方源拒绝了葛光的邀请,一个人走进这嘈杂热闹的集市。

\end{this_body}

