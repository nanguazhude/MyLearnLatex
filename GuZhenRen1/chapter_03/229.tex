\newsection{痛并快乐着}    %第二百二十七节:痛并快乐着

\begin{this_body}

黑楼兰心中叫苦不迭。

他战力出众,拥有极其强大的力道远战杀招,只是这样一来,便遮掩了他移动不足的弱点。

准确的说,黑楼兰主修力道,兼修暗道,移动能力并不俗。

但在如今的局面之下,只有飞行大师才能艰难生存。黑楼兰的弱点,在无相手的追捕下暴露无遗。

无相手肆虐,黑楼兰身上的蛊虫接连被抢。

尽管他击爆了不少无相手,追回了许多。但与此同时,新出现的无相手会抢走他身上更多的蛊虫。

此刻,黑楼兰飞猛扑,宛若一头蛮熊飞撞,满脸都是狰狞和愤怒。

即便他心知:在这样的情况下,方源优势巨大,但他也没法放手。

皆因方源好巧不巧捉住的蛊虫,就是黑楼兰的本命蛊!

本命蛊乃是和蛊师休戚相关,性命交修。一旦毁灭,蛊师就会立即遭受重创。方源拿捏住此蛊,等若捏住黑楼兰的一个巨大把柄。

说起来黑楼兰的这只本命蛊,也是命途多舛,多次被无相手抢走,又被黑楼兰艰难夺回。

细数的话,这已经是本命蛊第六次被抢了。

只是这一次,出现了意外,竟然落到了方源手中。

黑楼兰心中愤怒又焦急。

本命蛊落到大敌手中,这情况太严重了,必须追回来!

方源眼中冷芒一闪即逝。

他并不知道,自己手中的这只五转蛊就是黑楼兰的本命蛊。但看黑楼兰的表现。想必也是极为重要的关键蛊。

“哼!想要夺回蛊虫,别做白日梦了。”方源振动背后六翼,身形拔升。猛地飞离远处。

黑楼兰之前两次三番追杀方源,方源和黑楼兰之间早已经是深仇大恨。

但此刻,却非斩杀黑楼兰的良机。

无相手越来越多,方源自身躲避,就已经越加困难。想要在这样的环境下,再击杀强敌……即便成功,付出的代价必然惨重至极。

尤其是黑楼兰乃是大力真武体。若是把他逼上绝路,他冲动自爆,方源绝对吃不了兜着走。

“就让他在后面追。倒要看看他能坚持到什么时候!”方源心中冷哼一声,将手中躁动不安的蛊虫强行镇压下去。

很快,他甩脱黑楼兰。

方源一边飞行,躲避沿途的无相手。一边分出心神投入自家仙窍。

仙窍一直在不断生长。

从出了太白云生的仙窍之后。方源的仙窍就不断地吞纳天地二气。如今已经有四百多万亩地域,仙窍天空橙黄一片,大地并不平坦,一座座白石山峰,直起直落,仿佛刀削斧劈,突兀地相间矗立。

仙窍地貌和蛊师流派、积累、心性、感悟都有关联。

方源第一次成为力道蛊仙,当下只能粗略浏览。知道概况。要想深入了解,只能留待日后研究。

“时光流。也达到了一比十二。外界一天,仙窍中就是十二天。生长度也缓慢下来了……”方源心中琢磨。

此刻,真传秘境中的天地二气,已经被他吸纳得差不多了。

他现在是在等待仙窍彻底定型,这是主要目的。

一旦拥有蛊仙战力,一切都会改观。

“我的蛊虫就暂时寄存在你那里,我是不会放弃的!”黑楼兰无奈地看着方源越飞越远,只能如此咆哮。

追击途中,他受到无相手的阻挠,蛊虫损失数量众多。他自己躲闪无相手就已经相当困难,现在又要追击方源,当然会遭此恶果。

黑楼兰无奈之下,只得放弃。

这种情况下,若他还要强行追击,势必会引起方源的怀疑。万一被猜到这就是他的本命蛊,那就更糟糕了。

“只好凭借和蛊虫的联系,在躲闪无相手的过程中慢慢接近他,然后暴起突袭……可恨的贼人!”

黑楼兰将一口钢牙咬得嘎嘎作响,心中不停谋算。

无相手能抢夺蛊虫,但却没有炼化之能。一旦无相拳被击破,蛊虫就会得到解放,蛊师也会立即感应到蛊虫,并且可用心念调动召回。

方源强行镇压了黑楼兰的本命蛊,但却没有时间和精力去炼化。

因此黑楼兰和蛊虫的联系还在,留给了他翻盘追回的希望。

时间流逝,真传秘境中,无相手已经彻底占据主导优势。

到处都是淡蓝巨手,宛若一群群蜂群呼啸盘旋。

不管是蛊师,还是真传光团,都只有四处躲闪的份儿。

“春秋蝉……”方源口中干涩,压力重重。

随着仙窍生长,春秋蝉汲取光阴长河之水,带给第一空窍的压力,也愈来愈大。第一空窍的窍壁上,已经现出许多裂痕。

为了尽量地减缓第一空窍的压力,方源已经提前将第一空窍中的真元消耗一空。

他飞行度极快,动用数十只移动蛊,对真元的消耗极其巨大。

“好在此行之前,我就筹谋良久,准备充分,收购了大量的乞丐蛾。但即便如此,我还得注意真元,一旦真元消耗光了,后果就不堪设想!”

这些乞丐蛾,乃是用来存储真元的蛊虫。

它形如飞蛾,但双翅上分布圆形洞口。这些洞口,让乞丐蛾看上去残破,犹如乞丐衣衫褴褛,卖相不佳。

但事实上,乞丐蛾的双翅圆洞,却是越多越好。

圆洞越多,乞丐蛾就能储备越多越高阶的真元。

在凡人世界,这种乞丐蛾有价无市。但方源坐拥狐仙福地,在宝黄天中,乞丐蛾却是成群出售。

方源直接购买了一群。虽然五转、四转的乞丐蛾并不多,但三转、二转却是规模庞大。

此刻,方源身上就有近千只乞丐蛾。几乎是将整群都带在身上。

凡人买不起,也难以负担养蛊费用。但方源有蛊仙资产,有狐仙福地,喂养它们却是容易。

不止是乞丐蛾,还有鹰扬蛊等等,也是多有备用。

方源使用了某只鹰扬蛊一段时间后,就主动将其收回。催用另一只鹰扬蛊。

他背生六翼,看似同时催动三只鹰扬蛊,但实际上却有十几只鹰扬蛊。轮番替换来用的。

如此做的原因,就在于天地二气。

蛊虫在这样的环境中催用,会受到天地二气的反噬。

催用时间一久,便会损毁。

方源又要仙窍生长。一直都是天地二气汇聚的中心。受到的反噬之力尤其巨大。

因此,运用鹰扬蛊这些蛊虫时,就要轮番替换来用,尽最大可能延长使用的时间。

也多亏了他有狐仙福地,有仙资辅助,否则哪里能够如此奢侈。

天地反噬,也是蛊仙升仙的困难之一。

太白云生就在之前升仙渡劫时,毁坏大量蛊虫。若不是方源资助。他早在八十八角真阳楼外躲闪颠乱雷球时,就已经没有蛊虫使用了。

蛊师升仙。若能侥幸成功,原本的凡蛊通常都会损失惨重。

升仙风险极大,代价高昂。

很多情形是:新升的蛊仙两袖清风,浑身上下干干净净。过往的蛊虫积累,几乎都被消耗一空。

六臂天尸王!

躲闪飞行过程中,方源陡然化作天尸形态,六翼飞翔,八臂举拳。

砰。

下一刻,他直接和一只无相拳相撞!

方源毫无痛觉,直接将无相拳撞碎,又缴获一蛊。

方源看了一眼,顿时大喜。

这是他出了太白仙窍之后,收获的第三只野生仙蛊了!

方源顾不得查看,直接镇压,暂时收入囊中。

内有春秋蝉忧虑,外有无相手祸患,全神贯注的飞行途中,还要时刻注意真元、蛊虫的使用状态。

但不可否认,方源的收获也是丰厚。

方源深切地体会到,什么叫做痛并快乐着。

飞行大师的造诣,在这样的局面下,帮了他大忙。

“马鸿运究竟在哪里?”方源收好第三只仙蛊,得陇望蜀,四下张望。

运道无上真传,是他一直都非常感兴趣的目标。

当年的巨阳仙尊,很大程度就是凭借这个,称霸九荒十地,无敌天下。

之前方源对这道运道真传毫无办法,但现在借助无相手,反而可以抢夺运道蛊虫。这可是千载难逢的良机!

但他转了半天,一有功夫就不忘搜索,却一直也没看到运道真传的影子。

很显然,运道真传也在运动转移。

真传光团虽然保护包裹着马鸿运、赵怜云,但只是对鸿运齐天的认可,本身仍旧是野生意志。

野生蛊虫,躲闪天劫地灾乃是本能。

真传秘境那么大,场面有极度混乱,到处是无相手、真传流星,蛊师奔走,或追赶无相拳时出的灿烂攻势,极大地遮蔽方源的视野。

方源又要躲闪无相手,不能随意搜索。双方都在移动,碰不到面其实也属正常情况。

咔嚓、咔嚓……

像是蛋壳碎裂的声音,绵绵不绝。

真传秘境的裂缝,一直在不停扩大,先前只是一个面,现在渐渐蔓延开来,遍布四面八方。

越来越多的淡蓝巨手,顺着裂缝挤了进来。

时间流逝,方源蛊虫损失的情况,变得愈加严重。

一来,因为天地二气反噬,尤其是移动蛊虫,哪怕轮替交用,也达到了极限,损失了不少。

二来,无相手数量过了质变的点,有时候许多无相手一起围追堵截,将逃窜路线都给封死。方源因此也被捉住过两三次。

但每一次,他都是特意选择手指头较少的无相手。一旦脱身,任由蛊虫被捉走,也要优先确保自身安危。(未完待续!

------------

\end{this_body}

