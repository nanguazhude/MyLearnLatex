\newsection{荡魂山胆石}    %第五节:荡魂山胆石

\begin{this_body}

方源细细数了一下,金碗当中共有青提仙元七十八颗。文学馆再算上碗底的那层青绿仙液的话,就是七十八颗半。

仙元之充沛,是三王福地的上百倍!

方源当即从中取出一颗青提仙元,捏在手中,然后念头一动,呼唤定仙游蛊。

定仙游蛊像是一只碧玉,精心雕琢后的蝴蝶,精致华美,悠悠地飞过来。

方源脸色一白,整个身躯陡然颤抖,冷汗涔涔而出,双眼一黑。若不是他手扶着金碗的边,几乎就要一头栽倒下去了。

“主人,你小心呀。你魂魄伤势严重,不要轻易呼唤仙蛊啦,否则你就要再度昏迷了。”小狐仙叫出声来,语气焦急,目光关切。

“无妨,我自有把握。”方源咬着牙,摆摆手,向定仙游蛊缓缓地展开手掌。

定仙游蛊闻得青提仙元的气息,立即轻轻振翼,在半空中滑翔了一下,然后落到方源的手掌上。

接着,它便趴在青提仙元上,慢慢地吸允。

定仙游蛊的名字中,含着“仙”字,并非没有缘由。它的食物,便是仙元。

半晌后,一整颗青提仙元就被它吞吸完毕。

蛊师炼蛊,用蛊,还得养蛊。方源自炼成之后,就没有喂养定仙游蛊。

他昏睡了七天七夜,已经把定仙游蛊饿得慌了,原本莹润的蝶翼,早就黯淡了下来。

这尚是方源首次喂养。

定仙游蛊吃饱喝足,满意地一振双翼,飞上了半空中。莹莹的绿芒,再度显现。点点碎碎,伴随着它每一次振翼而挥洒,美轮美奂。仙蛊特有的气息。不断地散发着。

“仙蛊的喂养代价高昂,定仙游蛊每一次都得服用一颗仙元!不过这一次服用,六年内无需再喂养了。”

蛊虫转数越高,喂养的代价就越大,但同时喂养的间隔时间也大幅度地拉长。

一二转的蛊,每天都要喂养多次。但到了四转,几个月喂养一次即可。五转蛊可以一两年喂养一次,而每一次喂食,也都是代价巨大。

喂养了定仙游蛊。方源就不再管它,任由它在荡魂行宫中随意飞舞,只要不出荡魂山,不被那道电影杀了便成。

目前,方源的空窍还只是四转高阶的程度。装载不了完整状态的仙蛊。

装下春秋蝉的原因,主要是它状态太虚弱了。除此之外,也是因为本命蛊的关系。

事情分轻重缓急,了解了目前处境之后,方源确认暂时安全。此刻,又解决了定仙游的事情,接下来他就着手自己魂魄上的伤。

“走。小狐仙,带我出去寻找胆石。”

“哎!”地灵立即清脆地应答一声,“主人,你其实早该这么做了。快请跟我来。”

小狐仙带着方源。来到荡魂山上。

两人行走在崎岖不平的山石上,方源身躯摇摇晃晃,令地灵不免关切担忧:“主人,我带你直接挪移过去吧。”

“挪移什么?仙元要处处节省!”方源微瞪了一下眼睛。

小狐仙吐了吐舌头老婆,诱你入局最新章节。被方源吓了一小跳,心中不禁想:这位主人真是有气势啊。就算是受伤了,人家也怕怕。

“主人,这里有一块胆石。”走了一会儿,地灵驻足,指着脚下的一块石头,喊道。

这块石头,仿佛人胆,长在山石上,颜色混杂,不仔细分辨,只会当做一块普通石头。

方源连忙走过来,蹲在地上,用手敲碎这胆石。

胆石一碎,立即飞出一蛊,化成一道灰色的幽芒,钻入到方源的身体里去。

方源顿时觉得脑海一阵清明,一股全新的力量,补充到自己的魂魄中来,双耳中的嗡鸣声也减弱了许多。

荡魂山能震魂荡魄,是万物生灵的死地。此时受到地灵的压制,方源这才能自由行走。

但荡魂山又并非是纯粹的死地。

生灵的魂魄,在山上荡碎,挥洒而下,和荡魂山结合。久而久之,就能形成胆石。

而一些胆石中,藏有胆识蛊,可以壮人魂魄。

这就好像是毒蛇栖息的地方,常有解毒的药草。万物竞争,大道平衡,有生就有死。

荡魂山看似死亡绝境之地,但却蕴藏着一线生机。

但这胆识蛊只能存在一瞬,一瞬之内,不是自己消亡,就是被荡魂山再度荡碎。地灵虽然能压制荡魂山,令胆石却不能移动,只能就地采取。因此方源只能亲力亲为,在荡魂山上行走。

补充了这颗胆识蛊,方源就好像是沙漠中快要渴死的旅人,喝到了一口清爽的凉水。

他再接再励,在地灵这位称职的向导带领下,又陆续寻找到了十几块胆石,一一敲碎。

方源因此,又获得了八只胆识蛊的滋养。他的魂魄不仅伤势痊愈,甚至还赶超先前微微一丝。

“哈哈哈,有了这荡魂山,我就有一山的胆石。只要我不断吞服胆识蛊,终有一天,我能将自己的魂魄底蕴,增强到前无古人后无来者的程度!”

身处山腰,方源抖擞精神,欢喜的哈哈大笑。

狐仙是个幸运儿,一次机缘巧合,近乎捡漏似的,夺得了荡魂山,并且将其挪移到自己的福地中来。

但她又是个极为倒霉的家伙。五次地灾时,就陨落了。

“如果不是碰到了魅蓝电影,她迟早将成为当代奴道最强者!可惜啊,狐仙空有荡魂山,却没有来得及成长。可喜啊,这荡魂山落到了我的手中!”

狐仙福地最宝贵的地方,就是这座荡魂山!

当然,魂魄也不可能一味地得到胆识蛊的增强。一旦魂魄不够凝练,使用大量胆识蛊导致过度的膨胀,必是一场万劫不复的灾难。

但方源有五百年经验,重生以来人生大起大落,常在生死边缘挣扎。魂魄早就被挫折磨得凝练无比了!

生死面前不改色,惊厄来临心如冰。

对于方源来讲,至少再服用上百只胆识蛊,完全不成问题。

魂魄底蕴深厚的好处,影响多多。不管是炼蛊,还是奴役兽群都大有益处。抛开这些不谈,魂魄底蕴一旦深厚起来,方源至少不用太担心,魔无天的紫瞳这类的杀人手段。

荡魂山是秘禁之地重生一打造俊男集团。最早记载于《人祖传》中。

话说太日阳莽向天空冲刺,最终陨落身亡。

作为父亲的人祖得知后,异常悲痛,找到智慧蛊发难。

就是智慧蛊教会太日阳莽喝酒,最终才导致一系列的事情。

智慧蛊连忙道:“人祖啊人祖。你不要找我麻烦。你的儿子虽然死了,但也不是不可以复活啊。只要你进入生死门,带着他走向生路,走到阳光之下,那他就能复活了。”

人祖楞了一下,继而大喜,忽然又大怒。

他伸手捏住智慧蛊。质问道:“智慧蛊啊智慧蛊,你当我还像当初那样懵懂无知吗?生死门是凶险之地,进去了就再也出不来。你害了我儿子不说,还想再害死我吗?”

智慧蛊连忙道:“其他生灵不懂得进出生死门的窍门。所以它们出不来。但是这些窍门我知道啊,我统统告诉你。”

“你是活人,要进入生死门,就得从死路进去。这路非同寻常。乃是宿命蛊离开公平蛊时,留下来的路。称之为命途。命途中有大量的忧患蛊,你要从死路进去,就得拥有勇气蛊。这样你就不怕忧患的折磨了。”

“当你进入生死门中,看到公平蛊时,你已经死了。但同时,你也会看到你的大儿子太日阳莽的魂魄。你将其带走,顺着另一条路――生路回来。生路是宿命蛊拜访公平蛊时,走出的痕迹,也是命途。”

“但在这命途当中,有三个关卡。一座是荡魂山,一个是落魄谷,一个是逆流河。你要翻过荡魂山,就要敲碎山上的胆石,获得胆识蛊的帮助。要越过落魄谷,就要寻到信念蛊的帮助。要闯过逆流河,就要一刻不停地前进,千万不要有一步的停留。”

人祖听信了智慧蛊的话,便将其放走。

他很快就寻找到信念蛊。

自从他双眼瞎了之后,信念蛊的光就是他唯一能看见的明亮。

“人祖啊,我感受到了你要救回大儿子的坚定决心。我愿意帮助你,但是请你千万不要放弃这个决心。当你放弃的时候,我就会离你远去了。”信念蛊关照道。

人祖又找到勇气蛊。

勇气蛊和希望蛊,是一对好伙伴。

人祖拥有希望蛊,常常见到勇气蛊,和它的关系也不错。

得到勇气蛊的帮助之后,人祖便来到生死门前,踏步进入死路。

死路一片黑暗,大量的忧患蛊如黄色的萤火虫,海啸般地向人祖包围过来。这时勇气蛊发出光辉,帮人祖赶跑了忧虑蛊。

死亡需要勇气。

人祖成功地走下去,他的身体越来越白,越来越飘忽,渐渐变成了一个鬼魂。

他又能“看见”了。

当他来到死路的尽头,在一片深邃安定的黑暗中,他见到了公平蛊。

他为公平蛊巨大的身躯感到极为惊讶:“你就是公平蛊吗?为什么你的身躯这么巨大?山峰和你相比,就像是一粒微尘。大海和你相比,仿佛是一颗米粒。”

公平蛊的声音极为恢弘:“生死是世间最大的公平,当我身处生死门中,我就会变得无比的庞大。人祖啊,你来到这里,是想带走你的大儿子吧。尽管去吧,他就在那里。”

\end{this_body}

