\newsection{霜玉孔雀}    %第一百九十七节:霜玉孔雀

\begin{this_body}

眼前是华美的白玉大殿。

四根巨柱,支撑着穹顶。地面上,光可照人影。

不管是墙壁、巨柱,还是顶部,都是采用的洁白若雪的白玉石构建。

在大殿的中央,有一座二十三层阶梯的小高台。

高台上,伫立着一只青铜神鸟雕像。神鸟昂首,欲扬翅高飞,但无数条漆黑锁链,每一根都有古树粗细,攀附在神鸟的身躯上,紧紧地缠着它修长的脖颈,死死地绕在它纤细的双足,甚至残忍地穿透它的毛羽,扣在它的身躯内部。

神鸟双眼眯成刀锋一般细,皱眉张口,似乎愤怒尖叫。神情动人心魄,细微之处都栩栩若生,浑身上下满溢出一股绝不屈服,奋力抗争的精神,让人只看一眼都会印象深刻。

方源出现在阶梯上,看着占据整个视野的高大青铜神鸟雕像,黑眸中似乎燃烧着两团火焰。

“这、这、这!”异变带来的震惊,让墨瑶舌头都打结了一般。

她狠狠地咽下一口吐沫,伸出手指直指,惊呼起来:“这竟然是霜玉孔雀,王庭福地的地灵!!”

“啊,这有什么可奇怪的?”方源嘴角翘起一丝弧度,淡笑着道,“王庭福地本来就有地灵,要不然当年也不会承认巨阳仙尊,认他做主人。地灵乃执念结合福地的天地伟力所化,要铲除地灵,就要对付整个福地。地灵灭亡,就代表福地毁灭。反过来福地毁灭。也预示地灵必亡。现在王庭福地好端端地在这里,地灵必有也存活着。”

这话惹来墨瑶痛声斥骂:“混蛋小子。这个浅显的道理,老娘怎么可能不懂!”

墨瑶生前为了爱郎薄青,拼尽心血研究八十八角真阳楼,企图获得鸿运齐天蛊,来帮助薄青成功晋升九转。

最终,她虽然成功地进入了真传秘境,但却力有未逮,没有取得鸿运齐天蛊。只能退而求其次,利用里面的传承内容,再结合自身炼道宗师的底蕴,炼成招灾蛊。

正是因为如此,她对王庭地灵的价值,知道得一清二楚!

当年,巨阳仙尊还不是九转的时候。在王庭福地的继承权的竞争中,获得成功,成为王庭福地之主。

待他成就九转,无敌天下之后,拥有更好的长生洞天,王庭福地便成为巨阳仙尊的地上行宫之一。

巨阳仙尊布置八十八角真阳楼。将其设置在王庭福地当中。

这就形成一个关系――要图谋八十八角真阳楼,就得先进入王庭福地。王庭福地,相当于保护八十八角真阳楼的一层乌龟壳。

经过巨阳仙尊的布置之后,这层乌龟壳一直在发挥着巨大的作用,隔绝蛊仙进入。

但在这当中。有个巨大的漏洞,几乎任何蛊仙都能明了!

那就是王庭福地地灵的存在!

这个漏洞。在巨阳仙尊生前,是不存在的。因为巨阳是地灵之主,他想要让地灵做什么,地灵都会无条件地去服从。

但当巨阳仙尊死后,王庭福地就成了无主之物。只要达到地灵认主的标准,不管是谁都能成为王庭福地的新主人。

而八十八角真阳楼就设立在王庭福地当中,只要哪个掌握了王庭福地,就等于将八十八角真阳楼掌控在手中。

只要是蛊仙,稍微一琢磨,都知道这层关系。

墨瑶生前,研究八十八角真阳楼时,也在这方面耗费了海量时间和心血研究。

但是最终,她都毫无所得。

在这个方面她付出的一切艰辛努力,都最终化为了泡影。

巨阳仙尊是何等人物,哪里不晓得这个漏洞。他处理得极好,将地灵封印深藏,最终墨瑶探索时,连地灵的一根毛都没见到。

“没想到地灵就藏在真传秘境里面!只有在真传疾飞到一定程度时,才会出现漏洞,才有可能打通进入这里的门扉!”墨瑶心中极为震动,此刻,她已经琢磨出味道来了。

她越是琢磨,越是对方源刮目相看。

“这个小子,我太小看他了!他究竟是什么来头,居然知道进入这里的方法?”墨瑶心中又惊又奇。

方源的表现,让她十分惊异,远远超出了她的想象。

她却不知,这是方源照搬了前世中洲蛊仙的影像。

而事实上,中洲蛊仙之所以研究如此深入透彻,其实都是建立在墨瑶研究的基础上的。

墨瑶乃是堂堂炼道宗师,一心为爱郎,研究八十八角真阳楼,孤身闯入。她死后,留下宝贵的研究资料,被灵缘斋掌握。

之后过了万年沧桑岁月,灵缘斋一代又一代的人杰,不断深化研究。同时八十八角真阳楼也在岁月的销蚀下,漏洞越多,越加容易利用。

但灵缘斋心知,要啃下八十八角真阳楼这样的肥肉,不是自己一个超级势力能做到的。于是,灵缘斋秘密和其他几个中洲古派合作,密谋布置数千年。

灵缘斋牵头,始终掌握着最主要的研究资料。之后厚积薄发,中洲蛊仙大举进攻时,队伍都是由灵缘斋的当代仙子黑月率领。

缓缓拾阶而上,方源来到青铜神鸟的脚下。

这座高大的雕像,便是王庭福地地灵霜玉孔雀!

只是它被巨阳仙尊牢牢封印起来,动弹不得。从十余万年前,越过悠长岁月,一直伫立到现在。

离得近了,墨瑶便有所发现:“嘿,小子,这次你的计划实现起来,恐怕要难了。这霜玉孔雀身上的青泥,乃是七转仙蛊地牢蛊所化。它身上缠绕着的漆黑锁链,则是七转仙蛊地网蛊形成……呃!”

说到这里。墨瑶忽然想到什么,话音戛然而止。

一时间。在方源的脑海中她双眼瞪大,像是见鬼了一般神色。

皆因她忽然想起来:就在不久前,方源召唤了两只仙蛊,其中一只就是和稀泥!

地牢蛊、地网蛊,虽是七转仙蛊,要高于和稀泥一转。

但它们都是消耗蛊,用一次就消散。

两大七转仙蛊的力量合力禁锢着地灵,但历经十多万年的光阴洗涤。这股力量已经虚弱了不少。

而和稀泥恰恰却正克制这股力量!

“哈哈哈,看来你已经想到了,也不是太笨的么。”方源朗笑一声,从口袋中取出和稀泥仙蛊。

此蛊虽然高达六转,但因为是消耗蛊,运用较为方便,无须仙元灌注。只需轻轻一捏即可。

方源便轻轻一捏,将其捏碎。

蛊虫碎开来,流露出液体般的褐绿光辉。

光辉循着方源的意向,悠然飞起,融入到神鸟雕像的身上。

整个过程,墨瑶呆呆地望着。说不出话来。

光辉完全融入雕像之后,一丝微微的震动产生。这股震动太轻微了,以至于微弱到仿佛错觉。

但很快,这股震动越来越大。

整个神鸟雕像,都开始震荡。覆盖神鸟全身的青泥表现。陡然出现了一道道裂纹。而漆黑锁链随着震荡相互碰撞,发出接连不断的沉闷交响。

“成。成功了?!难道说,经历了十多万年的销蚀,两大仙蛊的力量已经所剩无几了吗?”墨瑶自言自语,神情复杂无比,有难以置信,有惊喜,有怀疑,有否定……

但接下来,震动却越来越小。

神鸟雕像很快沉寂下来,漆黑锁链不再碰撞,整个白玉大殿又重归安宁。

墨瑶大为失望,不禁长叹:“终究还是失败了!到底还是相差一转,这里的布置到底出自仙尊之手啊。”

“呵呵呵。”但在这时,方源却轻笑出声,“墨瑶,你何不看看神鸟的头颅?”

墨瑶探去心神,立即惊道:“地灵头脑上的青泥,正在缓缓融化!对啊,这才是和稀泥仙蛊的真正效果,我心系于此,患得患失,失了分寸,竟然没想到此层。呵呵,可笑,可笑。”

墨瑶忽然摇头而叹,脸上神情尽数收敛起来,再现传奇蛊仙,炼道宗师的大家风范。

其实,也由不得她不失态。

她在生前,实在为了王庭福地,为了八十八角真阳楼付出太多沉重的代价了。

现在,地牢蛊、地网蛊两大仙蛊的力量,虽然被光阴冲刷了大半,但还残留一部分。

和稀泥仙蛊虽然克制,但到底只是六转,要消融残留的力量,仍旧很艰难。

但千万别忘了,还有地灵的存在!

刚刚的震荡,起源就在地灵身上。

正是由它发力,导致大部分的仙蛊力量都用来镇压它,使得和稀泥仙蛊的力量趁虚而入。

青泥渐渐溶解,化为稀泥落在白玉地砖上,青黑色的泥点,在雪白的地砖上分外显眼。

一点一滴,很快就落了一地。

霜玉孔雀的头颅解放出来,但是到了脖颈处时,融化的速度大大减慢了。

很显然,地牢固、地网蛊的剩余力量已经反应过来,对抗和稀泥,令其效率大降。

“呵,就和前世影像中,展现的一模一样。”方源心中一笑,他仰头看向地灵霜玉孔雀,“王庭地灵啊,看看你的脚下,正是我将你解救出来。被封印了十多万年,重新呼吸的感觉如何?只要你认我为主,我将令你重获自由!”

霜玉孔雀愤怒的脸色,仍旧定格在脸上,听了方源的话后,它高傲地冷笑一声:“孤乃和福地一体,十多万年福地中经历的种种,都在我心中一一如实映射。你如果想成为我的新主人,就得达成一个条件。”

十多万年前,巨阳仙尊满足了地灵的这个条件,成为王庭福地之主。

今天,方源也面临这个同样的要求。

到这一步,前世影像已经再无指导价值。方源哈哈一笑,双眼放光,问道――

“什么条件?”

------------

\end{this_body}

