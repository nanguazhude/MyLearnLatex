\newsection{无上真传!}    %第一百九十五节:无上真传!

\begin{this_body}

“无双真传!”看着远处的白炽光团,方源目光骤亮。

他连忙取出特意蛊,又取出一只大魂虫。

他凝出一股意志,灌注在大魂虫身上,看着它飞向真传光团。

“无双真传,要凌驾于普通真传,我倒要看看,究竟有什么不同。”方源目含兴奋期待之情。

但出人意料的事情发生了。

大魂虫和真传光团的距离,越来越近,方源却感到自己和大魂虫之间的联系,越来越微弱。

当大魂虫,还未接近到三百步时,它和方源彻底断开了联系。

“这,这是怎么回事?”方源吃了一惊,看着无双真传循着先前的轨迹,往前悠悠飞走。而他放出去的大魂虫,却静止不动。说是灭亡,也不妥当,但不论方源如何调动它,它都静静地悬浮着,没有一丝响动。

“咦?这个无双真传并不简单,你要小心。”脑海中,墨瑶意志也轻咦一声,示警道。

“究竟怎么回事?”方源好奇,赶往大魂虫处,小心翼翼地接近,将其拾取在手,详细检查。

大魂虫并未有任何损伤,但是灌输在它身上的意志,却消散一空了。

大魂虫是二转蛊,被方源炼化,驱除了它远比的野生意志。

方源的意志,占据它的身躯,因此才能调动它,如臂使指。

但现在它身上的方源意志,包括用特意蛊凝练出来的特意,都莫名其妙地分崩瓦解,没有一丝留下。

大魂虫就成了空白之物,任何人灌注一丝意志,都能顷刻将其炼化。

这也是方源死活都调动不了它的原因了。

前几次,方源利用特意来试探真传,巧妙地规避了真传考验。用这个方法,屡试不爽。没想到还未探明无双真传。就栽了跟头。

这个无双真传,究竟藏着什么,将方源的意志清楚一空?

“哦,我想起来了!原来是它呀。”正当方源犹疑之时,墨瑶忽然开口。

旋即,她娇笑几声:“小子,该说你幸运好呢。还是不幸好呢?无双真传数量不多也不少,偏偏被你遇到最特殊的一个。”

“请指教。”方源面色平静下来。

“这份真传我看过,也给我留下了深刻的印象。当初我为了接近它,也耗费了不少心思呢。呵呵呵,这里面是巨阳仙尊亲手炼制的一只七转仙蛊,名为人气蛊。威能玄妙。甚至可以说是诡异,就算是当初的我,也耗费了不少代价,才探查明白的。”墨瑶语气感慨。

墨瑶生前,是七转蛊仙,灵缘斋的仙子,赫赫有名。

但是接近这份真传。花费了不小代价。方源如今只是凡人,连接近都接近不了,也理所应当了。

“人气蛊……”方源口中咀嚼着这个关键名字。

他忽有所悟,问道:“莫非和成仙三气有关?”

“小子,你猜的不错。”墨瑶一叹,解释道,“蛊仙成仙,讲究天地人三气。成仙之际。要彻底粉碎空窍,接纳天气和地气。天气、地气接纳越多,蛊仙的成就越高。但天气、地气也不能无止境的吸纳,必须和人气相互持平。”

“而人气,乃是蛊师自身积累的总和。蛊师本身的战力,肉体的强横,魂魄的深厚。对蛊虫的熟知,对天地的认知,对自我性情的觉醒,以及运气、才情、资质天赋、际遇经历、灵感经验等等。都会在升仙之际,转化为人气。”

“蛊师积累越多,人气就越多。人气越多,能接纳天地二气就越多,蛊仙的成就越高。而这只人气蛊的作用,就是吸纳旁人的人气,转化纯净,并在升仙的时刻,加持到自己的身上,从而大大地提升成为蛊仙的可能,以及加深蛊仙的根基和潜能。”

墨瑶的话,让方源大开眼界。

他不禁赞叹道:“居然有这样的仙蛊,这样的奇思妙想!了不起!”

人族历史上,蛊仙数目一直都极为稀少。成仙之难,万名五转蛊师往往怕只有一位,能够成功。

成仙关键中,首当其中的一点,就是蛊师积累。

很多蛊仙积累并不深厚,升仙过程中,凝聚的人气很少,但天地之气来的很多。如此一来,达不到平衡,天地二气压制、吞并人气,最终就使得蛊师被天地同化,身死道消。

但如果有了人气仙蛊,就能消弭这道难关。

收集他人的人气,提纯转化,集中在关键人选身上,就算积累不足,也不要紧了。

可以说,有了人气仙蛊,能极大地提高升仙成功可能,提高蛊仙的数量。

而提高蛊仙数目,就是明显增长一域的实力。

毫无疑问地讲,这是一只可以改变五域对峙大局的蛊虫!只要消息透露出去,任何一个超级势力都会为此狂热,趋之若鹜。

然而,人气仙蛊太强大了,转数高达七转,比目前的春秋蝉还高。

方源凡人之躯,承受不了人气仙蛊的余威。

他的意志,才刚刚接近仙蛊,就被仙蛊吞噬,提纯为一丝微末的人气,贮藏起来。

如果不是墨瑶提供情报,方源还蒙在鼓里,不知道真相。

“人气蛊虽好,但却并不适合我啊。”方源遗憾地望了白炽光团最后一眼,叹息一声,摇摇头,轻轻转过一个方向,毫不留恋地离去。

人气蛊的转数太高了,方源根本没有资格掌控。

人是万物之灵,蛊是天地真精。蛊对于蛊师而言,是工具。但人气仙蛊太强大了,仿佛冰刃,相比较起来,方源就是个婴儿,还未接近,就被冰刃冒出来的寒气冻伤了。

“最关键的是,我用不到人气仙蛊。我要晋升蛊仙,就得从凡人突破,掌控不了人气蛊。若我成了蛊仙,还要人气蛊干什么呢?帮助他人吗?”

人气仙蛊。并不适合独来独往的方源。只有那些超级势力,才会需要它。

除非方源自己组建势力。

在真传秘境中,悠悠飞行。

方源渐渐有所领悟,沉默了一会儿,他开口道:“墨瑶,我明白你所说的危机了。”

“呵呵呵。”墨瑶轻笑一声,她并不吃惊。“蛊师进入真传秘境,每接触一次真传,或者停留时间越长,这些真传的飞行速度就越快。因此催促蛊师,尽快选择真传。到最后,真传如流星般迅速。蛊师根本不可能捕捉得到。更因此,使得蛊师有生命之忧。”

顿了一顿,墨瑶继续道:“而生命危机,就是来源于真传。刚刚的无双真传,你已经看到了。人气仙蛊,还未接近,你的意志就被它转化为人气。想想看。当它如流星一般飞速像你撞来,你躲闪不及时,会怎样?”

方源轻哼一声。

这个情景,他完全可以想象。

凡人之躯,如果被白炽光团撞上,不管肉体魂魄意志,恐怕都会完全消融掉,转化为人气吧。

“就像我之前说的。你太弱了,只有凡人境界。这里的普通真传,每一个考验对你而言,都是一道难关。无双传承,就更难了。实话告诉你,在无双真传之上,还有最高层次的传承。号称――无上真传。”墨瑶又爆出猛料。

“无上真传?!”

“整个真传秘境当中,只有三道无上真传。”墨瑶此刻语气感慨万千,似乎沉湎于记忆最深处,“对于凡人来讲。要获得这三道无上真传,根本不可能。无双真传的难度,已经比升仙还难了。偏偏王庭福地,禁止蛊仙出入。这也就是为什么,八十八角真阳楼矗立这么久,其中八十八道真传,仍旧还有大半留在这里了。”

方源细想一下,对墨瑶的话十分赞同。

要进入真传秘境,就十分困难,机会渺茫至极。首先得是十年一届的王庭之争的胜利者,其次还得打通至少十层真阳楼,最后进入之后,还得成功地通过真传考验。

也就是说,十年一度,北原那么多的豪杰中,只有一位脱颖而出,掌控楼主令。

单单掌控楼主令还不行,还得要求胜利者的身边势力雄厚,不能在王庭之争中损耗太多,身边要有足够多的强者,又要足够全面,涉及各个流派,这样才能有希望打通十层真阳楼。

侥幸进入真传秘境之后,真传的考验又会刷下一批人出去。

真传考验极为不易,饶是方源这样的重生老怪,也险些栽跟头,更何况那些人呢?

更变态的是,就算通过了真传考验,这个真传就十分巧合地适合自己吗?

未必。

能够做到这一步的,通常都是龙中之龙,凤中之凤,雄上之雄,天资、性情、机遇三面齐全。这样的人物,心志高绝,很自然就会想多看看,多选择选择。

如此一来,等到在真传秘境中待不下去了,就悔之晚矣了。

说不定,还会死在这里面。

“真传飞行的速度越来越快了,你还有最后一次机会。不管是什么真传,快点拿了走人……嗯?!不好,快走!”墨瑶正劝,忽然语气一变,惊叫一声。

方源转头一看,便见一道真传,有成人大小,绽放七彩玄光,朝着他呼啸而来。在黑幽的秘境中,拖出一道绚烂的长长焰尾。

海碗大小的光团,是普通真传。

脸盆大小的光团,是无双真传。

那么成人大小的光团,是什么?

方源的脑海中,电光一闪,冒出四个字――无上传承!

整个真传秘境中,只有三道无上真传,现在就有一道,朝着方源而来!

------------

\end{this_body}

