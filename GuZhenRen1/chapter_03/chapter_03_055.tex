\newsection{古往炼道第一仙}    %第五十五节:古往炼道第一仙

\begin{this_body}



%1
时间一点一滴地流逝,阴云上数道人影默默立着。

%2
除去鬼王、红玉散人之外,还有三位女仙。

%3
这三仙各个美貌,窈窕多姿,正是花海三仙。

%4
青衣女子眉目清冷,为青索仙子。黄裳女子娇小白皙,号黄沙仙子。粉裙女子娇媚动人,为粉梦仙子。

%5
三人结伴,站在一起,和鬼王、红玉散人分隔一段距离,泾渭分明。

%6
青索仙子盯着下方的月牙湖,黄沙仙子和粉梦仙子小声交谈着,目光时不时扫过鬼王和红玉散人。目光中带着警惕。

%7
不说鬼王丑陋的外表,就让好美的三仙嫌恶。单就红玉散人,当年为了炼蛊,亲手杀了自己的父母双亲,背叛自己的兄弟,这就叫正道三女仙十分不齿。

%8
但琅琊福地非同小可,曾经居住在里面的蛊仙,乃是鼎鼎大名的“古往炼道第一仙”——长毛老祖。

%9
此人极精于炼道,有八转修为,具有古往今来屈指可数的炼道才华。

%10
他寿命悠长,横跨盗天魔尊、巨阳仙尊两代尊者。

%11
这是什么概念呢?也就是说两位九转蛊仙,都没有他活得久。

%12
他的炼道才华,就算是盗天、巨阳两大尊者都无比叹服,甘拜于下风。都曾经拜托过长毛老祖,请他为自己炼蛊。

%13
有后人统计过,长毛老祖一生,至少炼成了三十八只仙蛊。这还只是从确凿的历史事件中总结出来的,还不算那些传说和故事。

%14
但这样的人物,最终也抵挡不过光阴长河的冲洗,最终老死。

%15
据传闻,他死后化为地灵,仍旧在琅琊福地中炼蛊不辍。

%16
琅琊福地是长毛老祖的居所,因此收录了海量的蛊虫秘方,这其中当然有仙蛊的秘方。

%17
花海三仙虽然不待见鬼王、红玉散人,但仙蛊秘方的诱惑力,足以令她们三人抛开正道的身份,与这两位魔道蛊仙暗中合作。

%18
时间徐徐流逝。

%19
咔嚓。

%20
忽然一声轻响,月牙湖面上的虚空破碎开来,露出崭亮的闪电之光。

%21
“地灾开始了!”鬼王振奋高呼。

%22
一时间,月牙湖的中央上空,电闪雷鸣,轰隆不断。

%23
这无疑是地灾来临,造成了福地出现漏洞的景象。

%24
红玉散人目光炯炯,盯着这处漏洞,一眨不眨。

%25
花海三仙纷纷对视一眼,均看到眼中的兴奋之意,三人的呼吸也微微急促起来。

%26
仙蛊难寻,这五位蛊仙手中都没有一只仙蛊。当然五转蛊虫还是有的,并且数量众多,更各个都是精品。

%27
但再多的凡蛊,也抵不上一只仙蛊。

%28
蛊仙们对于仙蛊的渴求,比色中饿狼见到绝世美女而燃起的欲望还要庞大。

%29
轰隆隆……

%30
湖面上空,雷霆不断炸响,闪电不断劈斩。甚至还形成了电水雷浆,像是暴雨,倾盆而下重生魔法妻全文阅读。

%31
第二个漏洞、第三个漏洞……地灾造成一个又一个的漏洞,连续出现。

%32
“这是地灾——‘万雷电雨’,好恐怖的威力。”红玉散人看到这里,眼中流露出震骇之色。

%33
“如果落在我们的花海福地,恐怕就算我们姐妹三人联手,也要抵挡不住吧?琅琊福地不愧是长毛老祖所有,我们这次真的能够从中夺得仙蛊秘方吗?”花海三仙面面相觑,脸色发白。

%34
之前鬼王给了她们许多好处,花海三仙就兴冲冲地来了,现在三人均感到此行不易,要硬闯琅琊福地大不简单。

%35
但凡蛊仙,都是人中俊杰,智慧超俗。

%36
不管是花海三仙,还是红玉散人都谨慎警惕起来。

%37
鬼王将众人的神色都看在眼里,嘶哑一笑:“这狗屎般的天地一直都想搞平衡!福地有福,天地就降下灾劫,千方百计要消磨掉这福分。福地经营得越好,灾劫的威力就越强。你们看这场‘万雷电雨’,威力之强,简直可以媲美那些拥有秘禁之地的福地了。诸位不妨想想,这琅琊福地中会收藏多少秘方?必定有许多的仙蛊秘方,要不然贼老天怎么会降下这般强盛的灾劫来?”

%38
这话说得众仙心中砰然一动。

%39
想到仙蛊,这些人的目光中流露出一抹炙热。

%40
“鬼王说的不错。我也进去过一些福地,其中一些快要毁灭的无主福地,它们的地灾不过都是毛毛雨。但福地关隘重大,哪个蛊仙不想经营好了?”红玉散人苦涩一笑,“福地越好,灾劫就越强。蛊仙修行大不易啊……”

%41
“磔磔磔磔……修仙就是逆天,贼老天想要削弱我们,打压我们,我们偏偏就要逆天而行。”鬼王附和道。

%42
“二位的话有些偏颇。修仙其实是顺应天命。我们运用蛊虫,就是学习天地大道法则。我们经营福地,也是替代天地,养育万福,福泽苍生。”青索仙子反驳一句,声音清脆悦耳。

%43
红玉散人住嘴不说,顾念局面,不想纠缠这个话题。

%44
这就是魔道、正道的理念之别,从太古时就分辨,一直到今天都没有分辨出上下。

%45
鬼王嘿了一声,他手指着下方漏洞,道:“诸位,现在地灾愈演愈烈。为了防止地灵主动割舍掉这些漏洞,不妨现在就动手吧。”

%46
“也好。”红玉散人立即表示了支持。

%47
“还是鬼王先请吧。”三仙达成共识。

%48
鬼王嘿嘿一笑,取出一颗青提仙元,又催动一只蛊虫,叼起青提仙元,飞射到漏洞中去。

%49
这蛊虫到了福地当中,不知道是被镇压还是被地灾毁灭,一瞬间就和鬼王失去了联系。

%50
但鬼王的青提仙元也送入了琅琊福地,立即发生爆炸,和琅琊福地中的仙元产生对耗。

%51
对于蛊仙而言每一颗青提仙元都十分珍贵。平时的时候,蛊仙们都注重积累,不到万不得已,不会轻易使用。

%52
见鬼王先做了表率,其他四位这才依次各投下一颗。

%53
鬼王再投第二颗,按着顺序,其他人又投了第二轮。

%54
蛊仙死后,虽然形成地灵,但再不能产出仙元。地灵手中的仙元是越用越少,而鬼王一行人却是有四位,占据数量优势。

%55
但如此投了上百轮下来,这琅琊福地中,却仍旧有仙元使用天眼。

%56
除去鬼王之外,众仙脸色都有犹豫。

%57
“那长毛老祖乃是八转蛊仙,虽然死了,但留下的是白荔仙元。一百颗青提仙元,也比不了一颗白荔仙元啊。”轮到梦粉仙子,她捏着一枚青提仙元,却没有立即出手。

%58
鬼王目光闪烁着阴芒,冷笑一声:“仙子何惧之有?长毛老祖是盗天魔尊时的人物,百般延命,苟延残喘到巨阳仙尊时代,终于老死。他虽然留下了白荔仙元,但又经历了幽魂魔尊、乐土仙尊两大时代。曾经的琅琊洞天,已经跌落成福地。到现在白荔仙元能有多少?恐怕只剩下些仙水残浆了吧。”

%59
红玉散人也呵呵笑道:“鬼王说的有理啊。刚刚的地灾你们也见过了。琅琊福地中收藏着这么多的秘方,很多都涉及到仙蛊,福分太大了,每次灾劫都如此强烈。就算再多的白荔仙元,也要快耗光了吧。”

%60
“这世道向来是,撑死胆大的饿死胆小的!诸位都投了这么多仙元下去,难道要放弃吗?也许离成功不远了。”鬼王蛊惑道。

%61
三女仙对视一眼,由青索仙子道:“二位的话说得没错,但我们三姐妹的仙元,也都是一颗一颗的,辛辛苦苦节省下来的,不是大风刮来的。这样吧,我们再投个五十轮,看看情形。”

%62
如此又投了五十轮,琅琊福地终见不支的景象。

%63
鬼王大喜,磔磔狂笑。

%64
三女仙原本总觉得鬼王笑声嘶哑刺耳,现在听来却只觉得欢喜。她们仿佛看到一个个的仙蛊秘方在向他们招手。

%65
又投了三十轮,四仙的仙元进入福地后,各自膨胀,相互影响,发生连锁的爆炸,但福地终于是没了动静。

%66
这说明福地中仙元耗尽了!

%67
“诸位,我先行一步了!”鬼王忽然大笑一声,撑起青黑蝠翼,顺着漏洞,第一个冲进了琅琊福地当中。

%68
“不好!”红玉散人大叫一声,唯恐落后,化作一道红芒飞射进去。

%69
“这些魔道蛊师,果然奸诈狡猾!”

%70
花海三仙气得鼻子都歪了,连忙紧随其后。

%71
三仙进入福地当中,一身的五转蛊虫皆能催动,随意运用。

%72
“琅琊福地的仙元,果然消耗一空了!”黄沙仙子语气振奋。

%73
三仙冲过万雷电雨,来到福地深处。

%74
但见福地中充斥着洁白的云海,烟雾缭绕升腾。

%75
元海中,坐落了一十二个楼台,各个雕栏画栋,华丽堂皇,别有千秋胜景。楼台中,有的仙鹤飞舞,有的羽人盘旋,有的彩霞漫空,有的檀香逸散。

%76
“这是云土,踩之如实地,地力肥沃,远胜凡间土壤。”青索仙子跺跺玉足,语气兴奋。

%77
“果然是仙家老祖气象啊!”梦粉仙子一脸的惊叹。

%78
“传说中的十二云阁,每一个楼阁中,都收录了海量的秘方!想不到我今天能亲眼看到。”黄沙仙子感觉自己太幸福了。

%79
“哈哈哈,这些都是我的!”远处,鬼王刺耳的叫声传来,他正飞往其中一座云海楼台。

%80
至于红玉散人,已经极为接近另一座云阁。

%81
花海三仙的眉头,齐齐一皱,相互对视一眼后,兵分三路,分别飞向三座云阁。

\end{this_body}

