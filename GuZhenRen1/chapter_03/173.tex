\newsection{奖励的杀招}    %第一百七十一节 奖励的杀招

\begin{this_body}

一秒记住【】/manghuangji为您提供精彩小说阅读。

“杀!”

“该死的,又是一头万兽王!”

偌大的沙漠,炎热的空气扭曲视线,金黄色的沙海上,一只只沙虫,不断地从沙地下冒出来。

而蛊师们,则仅仅占着一块沙丘艰难据守。

这里是八十八角真阳楼第七层,第八十九道关卡。

每一道关卡,都是一个全新的小天地,有各种各样的考验。

一头沙虫万兽王加入战场,立即让蛊师的防线出现波动。

“守住,再守住三刻钟,我们就能打通此关了!”单刀将潘平高声呐喊,鼓舞着士气。

他是魔道出身,如今已经正式投靠了黑家,成为了黑家的外姓家老。

他个人战力极其出色,但此关考验的却是防御战。

要求闯关者,在特定的沙丘上坚守六个时辰。面对海潮般的沙虫攻势,除非是蛊仙,凡人的个体力量显得渺小,因此闯关的人数越多越好。

但潘平的话,没有起到多少效果。

战到如今,蛊师们伤亡惨重,疲惫不堪,已经快要达到极限。

“他奶奶的,又出现一头万兽王,这还让人怎么活?”大胡子抹了一把脸上的血和汗,怪叫起来。

他的本名不为人知,因为胡须浓密,被人直接喊做“大胡子”。

在之前的闯关中,他通了第六层的第十八道关卡,因此发迹。从默默无闻,到如今小有声名。

他原本只是二转蛊师,丢在圣宫中都找不到,小喽啰的角色。但因为通关的奖赏,令他拔升到了三转境界。

蛊师到了三转,就不一样了。

二转常见,三转就稀有了。普遍都是家老、长老。

大胡子到了三转,境遇立即得到改善。高质的真元,拉动他的战力。战力的提升。又带来丰厚的战利品。不仅如此,还有许多中型部族对他伸出橄榄枝——小型部族容不下也养不起外姓家老。而大型部族却还看不上他。

“这关可是第八十九道,要是通了关,奖励可不得了!我们这种人是不可能获得,但是这个任务的报酬,也足够我换取两只三转蛊了。”

大胡子想着,心头火热。他一边酣战,一边还得空瞅瞅沙丘顶峰的那几位大人物。

正是他们。发布任务,召集了近千名蛊师,来闯这道难关。

沙虫万兽王冲入防线,立即爆发激战。金光、箭雨、火焰各种攻击朝万兽王的身上落去。万兽王浑身甲壳忽然变作黄金色泽,防御大增,将这种攻势尽数抵挡。

它咆哮连连,闯入人群当中,掀起一阵腥风血雨。

常飚站在沙丘上。紧皱眉头。

“情况有些严峻。”他开口道,“这头万兽王的身上,居然有一只五转的黄金甲蛊。如果任由它肆虐下去,防线早晚会被彻底绞烂的。”

战到如今,这已经是第九头万兽王了。

根据身上寄生的野蛊不同。万兽王之间的战力也有高低差别。这只万兽王就比较棘手,因为有优秀的防御蛊,导致群攻的效果不大。

出现这种情况,一般都会由蛊师强者亲自出手,才有效果。

正所谓兵对兵,将对将。

常飚的言外之意,也正在于此。

马英杰就站在他的身边,此刻越众而出,开口道:“那就由我来动手吧。”

没有人提出异议。

万兽王出现,众人便轮番出手,这是当初就协商好的。按照顺序,也正好轮到马英杰。

不愧是马英杰。他亲自出手,统率天马群,轻松地拦下沙虫万兽王,立即稳住了局面。

“小马尊的确非同凡响啊。”沙丘上,众人交口称赞。

“常兄,你觉得马英杰怎么样?”潘平暗中传音。

马英杰曾经是马家少族长,师从马尊,但马家失败,被黑楼兰强逼臣服。如今马家盛极而衰,仅仅只是中型势力。

常飚知道潘平说的什么意思,他是想将马英杰也拉拢过来,成为杀狼同盟中的一员。

所谓杀狼同盟,就是对付方源的秘密组织。

潘平在星鹫峰被方源夺走传承,因此暗恨不已,企图报复。

常飚则和常山阴有夺妻之恨,不共戴天。

方源势大无比,能和黑楼兰分庭抗礼。尽管潘平、常飚二人多有斩获,得到了八十八角真阳楼的许多好处,如今都已经是四转巅峰。但要对付方源,他们都知道成功性很低,因此需要更多的强力帮手。

见常飚沉吟不语,潘平又接着道:“马家惨败,主要的原因就是常山阴!马英杰的师傅马尊,就是死在他的手上。他和常山阴可是有着深仇大恨。”

但常飚微微摇头,传音道:“不可。马英杰此人坚忍不拔,乃是一代人雄。以我看来,他现在一门心思都扑在部族上面,想要重振马家盛况。既然要振兴部族,他必定不会和狼王作对,反而会和他交好。狼王势大,我们杀狼同盟处在暗处,这是最大的优势。不能胡乱招揽外人,一个不妥,就会暴露我们自己的。”

潘平不愿放弃,执着劝说道:“常兄,做大事者岂能瞻前顾后呢?!常山阴这个小人,有万狼护卫。将来打杀起来,我们的力量恐怕会被狼群大大损耗。别忘了此人还有高绝无比的力道修为!马英杰统帅马群,人称小马尊,有大师之姿。我们正需要这样的强者,来对付狼群啊。”

常飚用眼光瞟了潘平一眼,心生不满。

潘平原是魔道蛊师,号称单刀将,行事无忌,大胆狠辣。这在常飚看来,却是鲁莽无智,难谋大事。

其实潘平也有不满。在他来看,常飚行事畏手畏脚,瞻前顾后,想得太多。不是英雄豪杰。

这是两人性格的差异。

常飚性格阴忍,当年谋算常山阴,也是借刀杀人。又爱惜名誉。过了这么多年,也不敢认常极右这个亲生儿子。

而潘平呢。看他在王庭之争时,纵横军阵,呼啸冲锋,就可窥破他的本性了。

但尽管有差异,不管是潘平还是常飚,都相互容忍。他们心底都清楚,狼王不是他们一个人能对付得了的。

“潘平是魔道出身。哪里晓得我们正道人物的心思!想要吸收马英杰?哼,异想天开!前一刻告诉他杀狼同盟的事情,说不定下一刻,就被马英杰出卖。示好狼王了。唉,我该如何劝说他呢?”常飚心中叹息一声。

他却也多智,立即计上心来。

他传音道:“潘兄,常山阴是杀了马尊不假,但你可别忘了。马英杰的父亲马尚峰是死在谁的手中啊。”

“呃……”潘平一愣,神情呆滞。

马尚峰就是潘平在乱军丛中砍死的。潘平也因此立下大功,得到许多奖赏。

当初的功劳,如今却成了阻碍自己复仇的绊脚石,这让潘平始料未到。

但潘平是个固执的人。他又提到:“我杀了他老父亲不假,但那是两军交锋,各为其主啊。既然常兄认为,马英杰是以部族为重,那么我相信你的这个判断。既然如此,我们就在这个方面设计。只要让马英杰认为,狼王的存在就是他振兴部族的最大阻碍,那不就好了吗?”

这话一说,轮到常飚愣神了。

他又瞟了潘平一眼,心想:“果然是智者千虑或有一失,愚者千虑或有一得。想不到这个潘平,也有脑袋灵光的时候。”

当下,他心思电转,谋算起来:“不错,此事的确有成功的可能。马英杰的弱点,就是马家部族!但是该如何谋算计划,才能让马英杰心甘情愿地加入杀狼同盟呢?”

常飚想到这里,忽然泛起一个念头,联想到之前的一个小情报。

他的目光,不由地投向左侧防线。

在那里,有一位二转的年轻蛊师。

“他就是马鸿运,原先是马英杰身边的奴仆长,现在则是马英杰最信任的心腹之一。不久前,狼王猎杀地魁兽群,马鸿运逃生出来,所获不小。奉献给马家之后,马英杰当即赏了他三枚青铜舍利蛊,奖励他的忠心耿耿,还将他奉献上去的蛊虫丝毫不取,都返还给他。”

“这小子运气真不错!赶在常家、葛家蛊师开进战场之前,顺利离开。青铜舍利蛊价值不菲,马家衰败,百废待兴,人心散乱。马英杰自己用不上青铜舍利蛊,便千金买马骨,竖立榜样。”

“马鸿运得了他最需要的青铜舍利蛊,一个晚上就将修为冲刺到二转,速度之快,连我都比不上呢。”

常飚想到这里,有点感慨命运的玄迷。

和马鸿运相比,常飚的出身可高贵多了,有大量的资源供给。但就算如此,常飚达到二转的时间,是马鸿运的数十倍。

皆因青铜舍利蛊自然生长,极其稀少,且又产地各异。也多亏了有八十八角真阳楼,马英杰闯关得的奖赏。

“马鸿运这小子,是捡了常山阴的便宜发家的。之前,常山阴还命常葛两家蛊师出动,很是逮捕了不少占便宜的蛊师。如果我将这个消息举报出去会怎样?”

常飚思量着。

“不……单单一个马鸿运,小小的二转,在常山阴的眼中怕是连蝼蚁都不是吧。他的份量还太轻,就算举报了,马英杰也不会为了区区的马鸿运,来对付常山阴。这个事情,还得继续筹谋和等待啊……”

三刻钟之后,沙丘上欢呼声震耳欲聋。

“胜利了,胜利了!”

“不容易啊,终于打通了此关。”

“也不知道这关的奖励,是什么?”

许多道好奇的目光,都集中在常飚、潘平、马英杰等人的身上。

此关一通,常飚等人的空窍中,出现了许多蛊虫。

其中一只东窗蛊中,存着信息。

“六臂天尸王?”常飚、潘平、马英杰等人查探一番后,面面相觑。

他们神情不一。

没想到会奖励一个力道上的杀招。按照内容的描述,这个杀招的威力可谓绝伦了!(未完待续)

------------

\end{this_body}

