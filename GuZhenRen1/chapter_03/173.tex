\newsection{联姻}    %第一百七十三节:联姻

\begin{this_body}

一秒记住【】/manghuangji为您提供精彩小说阅读。

一天之后。

晚宴。

“这次贵族子弟,救下小女,常某感激万分呐。这第一杯酒要敬马族长!”常飚举起手中的酒杯,笑道。

马英杰连忙举杯,谦虚道:“这只是机缘巧合罢了,不想竟是救下了常飚大人的爱女,这也是鸿运这个小子的荣幸啊。”

“哈哈哈。”常飚大笑,一饮而尽。

马英杰也同样干了杯中之酒。

马鸿运便坐在他的身边,此刻成了整个宴会的主角,数十道目光都集中在他的身上。

感受到这些探究、好奇、疑惑的目光,马鸿运显得有些局促不安。

常飚轻轻放下酒杯,以目光示意身旁座位上的潘平。

之前就已经商量好的,潘平会意,看向马鸿运道:“贤侄,说说你是如何英雄救美的吧。”

“我,我……”马鸿运讷讷,说不出话来。

说实在话,他也不知道自己是怎么救的。当初自己只顾着逃跑,背后是漫天飞舞的铁喙鸟,情况那么紧急,就算是救下常丽,他也没有多过脑子。

潘平瞪大了双眼,死活等不到马鸿运讲话。

好在常飚早有算计,摸清了马鸿运的性格,此刻目光一转,看向席中某人。

这人立即站了出来,来到中央,抱拳请示道:“常飚长老,诸位大人,在下便是当事人之一,整个情形都有幸亲眼目睹。马鸿运大人宽厚谦虚,不居功自傲。但小人却不忍英雄埋没,因此斗胆趁着酒兴,为诸位在座的大人讲述。”

一听这话。便知此人是个能言善道的主儿。

常飚点头:“那你就说说吧。”

这人便开口,说得言之凿凿,情意生动之外,还起伏跌宕,绘声绘色,将马鸿运描绘成了一个孤胆英豪,行事果决。有勇有谋,不惧艰险。

众人听了,时不时地叫好。交口称赞起来。

看向马鸿运的目光,也纷纷发生了变化,变得尊敬、温和、欣赏。

马鸿运瞪大双眼,耳朵里听着。像是听天书一般。他心中难以置信:“这人说的是我吗?我什么时候这么厉害了?莫不是他看错了?”

不敢相信的。还有一人,那就是马英杰。

马英杰是马家族长,一代英杰,熟知马鸿运的性情为人,怎么会被这花言巧语哄住?

他表面上淡淡微笑,听到妙处,也不断点头,向马鸿运投去认可的目光。心中却在琢磨:“要说马鸿运机缘巧合,救下了常丽。这没有什么奇怪的。但为何常飚长老。要派了托儿,为马鸿运这般分说呢?常飚有什么图谋?今天这场晚宴,虽然有数十位嘉宾,但真正的主角只有两位,那就是常飚和潘平二人。”

马英杰心中暗暗警惕。

马家在王庭之争中失利,由盛转衰,马英杰的师傅、父亲都没于战阵。艰难和挫折,让马英杰迅速成熟起来,成长为一位英杰。

他暗暗猜测常飚的企图,表面上却不动声色。

如今的马家衰败了,而常家因为方源的缘故,如日中天!潘平脱离魔道,转投黑家,如今的身份是黑家外姓长老。

不管是哪一位,都不是现在的马家,现在的马英杰能得罪得起的。

“好,好,好。”那人讲述完马鸿运的英雄事迹之后,常飚一连说了三个好字。

“果然是英雄出少年。”常飚看向马鸿运,不吝赞赏。

顿了一顿,他接着道:“自古都说,英雄爱美人,美人配英雄。不瞒诸位,自从小女被救之后,回来便常常默默寡言,神不舍舍。我问明缘由,方知小女是心有所属,对那位在危难间救下她的英雄少年牵肠挂肚。我举办这场宴会,一是为了表达感激之情,二则正是因为如此。”

这话一说,堂中大哗。

无数道目光,夹杂着羡慕、嫉妒、震撼、不敢相信的意味,望向马鸿运。

“这小子走了什么狗屎运?居然能得到常家小姐的垂青?”

“常丽清秀可人,不想却相中了这么一个傻小子?唉,早知如此,我也去铁喙鸟群的栖息密林里了。”

“这常丽虽然不是常飚的亲生女儿,但却是从小被收养,一直受到常飚的喜爱,是常飚长老的掌上明珠。马鸿运这家伙如果娶了常丽,老丈人就是常飚了呀!”

一时间,众人心绪沸腾。

马英杰很快从惊愕中反应过来,他迅速思量,犹有存疑:“难道这就是常飚,之所以大张旗鼓宴请我们的原因?虽有苏仙夜奔之事,但这事情也太好了点吧?”

更好的事情,还在后头呢。

当着众人的面,常飚掏出两只白银舍利蛊:“长江后浪推前浪啊,贤侄是咱们北原的少年英雄,不能不赏。这两只白银舍利蛊就是救命之恩的些微答谢,请贤侄收下。”

堂中喧哗声不由更大。

“啊?”马鸿运仓促之间,看向马英杰。

马英杰微微点头,笑着指点道:“长者赐不敢辞,鸿运,你还不赶紧跪拜谢恩?”

马鸿运连忙离座,走上前去,跪拜:“谢常飚大人赏赐。”

常飚哈哈大笑,也离开座位,亲手将两只白银舍利蛊交到马鸿运的手上。

众目睽睽之下,他亲切地拍拍马鸿运的手,问道:“不知贤侄对小女感观如何呀?”

“啊?”马鸿运抬头,涨红了脸,一时间不知道该怎么答话。酝酿半晌,终于从嘴里挤出几个字,“常丽小姐漂亮,很漂亮。”

“哈哈哈。”常飚仰头大笑,“这就好,这就好。贤侄,请回去坐吧。”

重新入座,继续开宴。

晚宴从傍晚一直持续到深夜,这才宾主尽欢,各自离去。

随着人流散去,常家嫁女,好运小子马鸿运的事迹,也随之广为流传开来。

到了第二天,常飚又宴请马英杰、马鸿运。只是这次,规模更小,不再是之前的大宴,只请了数人参加。

马英杰望着手中的请柬,目光沉凝。

回来之后,他一夜未睡,琢磨着这个事情。

小小的请柬,在他的手中,却让他感觉分外沉重。

他将请柬放在桌上,唤来仆从:“去,将马鸿运召来觐见。”

仆从连忙领命,来到马鸿运的居处时,赵怜云正在对马鸿运面授机宜:“你这笨蛋,踩着什么狗屎,居然走了这样的运道?不过这事情太好了,反而让人心中发虚。我想,马英杰族长必然会召见你询问此事,到时候你一五一十地说清楚就行,绝不能有半点隐瞒之处!”

“哦。”马鸿运立即答应下来。

“还有。”赵怜云大眼珠子一转,“常飚不是赏给你两个白银舍利蛊吗?族长召见你时,你就将这两只蛊献上去。”

“什么?”马鸿运双眼一瞪,叫了起来,“这可是我冒了生命危险,好不容易得到的!又是常飚大人赏赐给我的蛊虫,我用了它们,能立即晋升到二转高阶。这是多好的事情啊。”

“你这个笨蛋!”赵怜云气得一踢马鸿运的小腿骨。

马鸿运立即抱起小腿,痛呼起来:“你干嘛又踢我啊?”

赵怜云翻了个白眼,没好气地训斥道:“你懂什么?就算你成为了二转高阶,凭你的身手,有什么用?我们俩的立身之本是什么?不是二转的修为,是和马英杰族长的情分啊。你是怎么晋升二转的?是马英杰族长赏你的三颗青铜舍利蛊啊。你把白银舍利股献上去,就是表明忠心,你以为族长会贪污你的这两颗白银舍利蛊?哼,他自己又用不上,肯定会收下来,然后再还给你的。”

“咦,他既然收下了蛊虫,为什么又要还给我?”马鸿运疑惑地问道。

“白痴!”赵怜云又翻了个白眼,“马家现在衰败,族人稀少,百废待兴。整个部族,只有一个三转家老马由良,还是个残疾。马英杰新近上位,没什么人可用,他又一心振兴部族,正是提拔人才,培养心腹的时候。你虽然曾经姓费,但现在姓马,更曾经是马英杰的奴仆长。马英杰对你知根知底,用你比他人放心。你去献上蛊虫,表明忠心,他肯定欢喜,收下你的蛊虫,这是认可你的忠心。”

“但马英杰绝非是个小气吝啬的庸主。收下蛊虫只是作姿态,他肯定会还给你。为什么?就是要竖立一个榜样,鼓励族人们学习你的忠诚啊。我料定,他不仅会还给你白银舍利蛊,更会再加赏赐。你虽然能力不足,但是忠心是有的。这就叫千金买马骨啊。”

马鸿运听得懵懵懂懂:“什么叫千金买马骨?”

“唉,说了你也不懂。你就照我说的办吧。肯定有你的好处。”

“哦。”马鸿运挠挠头发,答应下来。

两人刚刚议定,马英杰的仆从便跑来传讯。

马鸿运依言,献上了两只白银舍利蛊。但是和赵怜云估计的不同,马英杰收下之后,却没有还给马鸿运。

这让马鸿运回去后,对赵怜云大加埋怨。

“难道是我估计出错?”赵怜云也有些疑惑了。

(未完待续)

------------

\end{this_body}

