\newsection{巨阳仙尊}    %第一百四十七节 巨阳仙尊

\begin{this_body}

%1
方源俯瞰地面。

%2
只见一处山丘,高高隆起,耸立在地面之上。

%3
山丘没有强硬的峰线,上面张开一个豁口,露出一个大洞,仿佛通往地底的模样。

%4
而在这山丘的四周,是一片沼泽地。

%5
沼泽地中,有着稀疏的小树林。

%6
在西南角上,还有一条河流。河水并不清澈,却源远流长,上下流都延伸出方源的视野之外。

%7
“土中蕴光,芒高万丈,百里天游,咏梅雪香。”看到这个地貌,方源的脑海中,不由自主地联想起这句密语。

%8
“难道这里便是地丘传承的所在地?”方源灵光一闪,顿时有些恍然。

%9
当初,他在一块灰白石板的赝品中,获得了地丘传承的信息。灰白石板上的画意蛊,直接将这副地形图送到他的脑海深处。

%10
因此方源印象深刻,就算是想忘,也忘不了。

%11
他振动双翼,在半空中盘旋了一番,再次确认这地貌和脑海中的地貌别无二致。

%12
“原来如此。我之前也有疑惑,以地貌地形为传承线索,这通常不靠谱。若是这传承铺设在北原外界,极容易被外力摧毁改变地貌。但是在这王庭福地中,却是不同了。”方源暗忖。

%13
这王庭福地,每隔十年,开启一次。王庭之争的胜利者们,蜂拥进来,也许会因为战斗等等原因,改变了这里的地貌。

%14
但是当王庭福地关闭之后,地貌种种就会缓慢复原。

%15
等到十年后,王庭福地再度开启时,便会恢复如初。

%16
“地丘传承……既然能设立在王庭福地当中,而且还能别出心裁,在灰白石板的赝品上动手脚。传承的主人花费这样的功夫和心思,里面的传承应该不会差。”

%17
怀着这样的心思。方源缓缓飞到土丘上,在土丘的洞口处观察了片刻后,他便召唤了几头天青狼。进入黑洞中探查。

%18
一盏茶的功夫之后,天青狼毫发无损地回到了方源身边。

%19
这个深洞。从外部看来黑黝黝一片,但是深入其中,却是长满了微光苔藓,并不黑暗。

%20
洞中什么东西都没有,空气湿润,只有土石和苔藓。

%21
方源也亲自下去探查,同样没有发现。

%22
他微微皱眉。重新走出来。对于这个结果,他早有心理准备:“这个传承并不简单,如果真的这么容易就得到手,恐怕早就被其他人得去了。当然。也不排除这份传承,被别人捷足先登的可能”

%23
但方源暗自分析,这种可能性很小。

%24
“要来到这里,至少得有两个条件。第一,是恰巧获得传承的线索。而要鉴别灰白石板,恐怕得要擅长鉴定的蛊师,才能办到。第二,需要蛊师进得来王庭福地。这就意味着,需要在王庭之争中生存下来。并且眼光独到,能成为优胜者。”

%25
“这个传承,很不简单。看来要想取得这个传承,必须要勘破密语了。”方源最后在心头总结道。

%26
土中蕴光,芒高万丈,百里天游,咏梅雪香。

%27
这个密语,究竟想表达出什么含义呢?

%28
方源苦苦思索,却得不到结果。脑海中思绪繁杂,没有任何的线索。

%29
“罢了,就先这样吧。待在王庭福地中,还有一段时间呢。”

%30
方源振翅而飞,率领着狼群,继续赶往福地中央的圣宫。那里才是他的计划重点,他花费了这么长时间潜伏,就是为了进入王庭福地。

%31
获得江山如故仙蛊是首要目标,除此之外,就是八十八角真阳楼中巨阳仙尊的传承!

%32
不止是他,大部分有能力有抱负的蛊师,都会积极地赶往圣宫。

%33
圣宫,是王庭福地的中枢,也是精粹所在之地。

%34
圣宫乃是巨阳仙尊的四大地上寝宫之一,也是最主要的寝宫。其余寝宫,分布于东海、西漠、南疆。

%35
而在中洲处,巨阳仙尊有一座规模更加宏大,更加辉煌的天上寝宫,坐落在长生天中。

%36
历史上,共出现了十位九转蛊师,被称为“仙尊”、“魔尊”。

%37
这十个人纵贯历史长河,从远古时代,到上古时代,到中古时代,到近古时代。每一个人都是各自时代的强者,纵横披靡,天下无敌。同时,又各有特色,差异极大。

%38
嗜杀的幽魂魔尊,神秘的红莲魔尊,智慧的星宿仙尊,不争的乐土仙尊……

%39
同样的巨阳仙尊,也是极富传奇色彩的人物。

%40
他本是魔道蛊师,出生在北原。一生福缘不断,好运连连。逢凶化吉不说,还能化灾为福。

%41
他成为魔道蛊仙之后,流连花丛,四处风流,无人能制。就算是当年中洲十大古派之一的灵缘斋的第一仙子,也被他纳为妻妾。

%42
也因此,他被灵缘斋招纳为外姓太上长老,转为正道。

%43
巨阳仙尊风流成性,成为仙尊之后,又上仙庭,成为四代仙王。他先后建立五大寝宫,拥有妃嫔数千万人。

%44
他精力旺盛至极,一千多岁时,他还从各地招收少女充实他的后宫。

%45
因此,在所有尊者当中,他拥有最多的子嗣。

%46
他的子女太多,大部分的子女他连名字都叫不出来。

%47
这些子嗣曾经遍布五大域,如今最主要的还是集中在北原当中。体内流着巨阳仙尊血脉的蛊师部族,被统称为黄金家族。

%48
“兄弟如手足,女人如衣服。”“家天下!”“美貌是女子天生的嫁妆。”“恨不能尽娶天下女子!”这些都是他的名言。

%49
尽管沧海桑田,时光流逝,但他在历史上留下的烙印仍旧熠熠生辉。

%50
尤其在北原,黄金部族把持着整个大局。巨阳仙尊仍旧影响着每一代人。

%51
圣宫,中枢大殿。

%52
夜。

%53
银辉灿烂,照耀着黑楼兰的脸上。

%54
他仰着头,望着中枢大殿上的牌匾。壮硕如熊罴的身躯,在银光中。默然伫立。

%55
作为盟军之主,身负黄金血脉,他一进入王庭福地。就置身于圣宫当中。

%56
中枢大殿的这块牌匾,极其巨大。长有二十丈。宽有八丈,上书三个大字——家天下!金光琉璃,夺目耀眼。

%57
中枢大殿,规格宏大雄阔,仿佛巨人居住的一般。在这巨大的牌匾下,黑楼兰肥胖的身躯,也显得渺小。

%58
“家天下么……”他仰头望着。神情却是复杂至极,有痛恨,有仰慕,有愤怒。有冷漠。

%59
“大人。”狈君子孙湿寒,缓步而来,轻声唤道。

%60
“何事?”黑楼兰回头,他脸上的神情已经尽数收敛起来,重新变化成往日里的狂放、粗鲁、暴躁。

%61
狈君子不疑有他。从怀中取出一封信来,禀告道:“这是单刀将潘平,不久前传达过来的信笺。他在信中,状告狼王常山阴,说他贪墨传承。公然勒索,行径十分恶劣,希望大人您能主持公道。”

%62
“哦?”黑楼兰伸出肥胖的右手。

%63
狈君子连忙将信,用双手奉上。

%64
“大人,不是小的多嘴。这个常山阴,真的是越来越过分了。居然连自己的袍泽,都欺压侮辱。唉,潘平大人是好心啊。本来有个传承,他念旧情,想着和常山阴分享。结果却被这样对待。狼人常山阴的确立下功劳,但这也不能代表他恣意妄为啊。如果人人都像他这样子,这还不乱套了吗?”

%65
狈君子趁着黑楼兰看信的功夫,在一旁小心翼翼地觐言道。

%66
黑楼兰冷哼一声,又将手伸过去:“拿来。”

%67
“啊,大人您这是……”狈君子懵懂。

%68
“这信只是潘平的片面之言,应该还有朱宰的来信吧。”黑楼兰目光犀利。

%69
狈君子立即谄笑起来:“大人您真是英明神武,小的敬佩得五体投地啊。”

%70
黑楼兰接过第二封信,目光扫视信上内容,面无表情,叫孙湿寒难以揣测。

%71
这信无非是朱宰表功,但结合第一封信,已经让黑楼兰大体明白了整个事情的经过。

%72
他用手一捏,两封信当即被一层暗光,腐蚀成渣滓。

%73
“潘平等人来到圣宫,你就去辎重营,给他们一些补偿就是。”黑楼兰接着嘱咐道。

%74
“是,大人。”狈君子躬身领命,但等了片刻,却听不到黑楼兰的下文,不由奇怪地抬起头来,问道,“大人,那狼王常山阴就不惩罚了吗?”

%75
“惩罚?可笑!”黑楼兰哈哈一笑,“我为什么要惩罚他?如果换做是我,我也会吞没这笔传承。只不过他的吃相有些难看就是了。”

%76
孙湿寒不甘心,急劝道:“大人,属下有些不同的想法。这狼王常山阴,自恃功高,桀骜不驯,可不能这样骄纵他啊。他虽然立下大功,但是没有大人您的部族的大力资助,他能有那么多的狼群吗?而且他可是北原英雄,威望甚高。大人若不稍作惩戒,恐怕今后他会变本加厉。甚至最后功高盖主,世人皆知常山阴之名,却不知大人您啊。”

%77
“哈哈哈。”

%78
“大人您笑什么?”

%79
“湿寒,你多虑了。这件事情一发生,他常山阴还有什么美名?恃强凌弱,贪墨传承,只会让他的威望大跌。而且他又不是黄金血脉,没有来客令,根本进不了真阳楼。”

%80
顿了一顿,黑楼兰继续道:“从这件事情上,就可以看出他常山阴也是凡人呐。有欲望,有私心,这就很好嘛。再说,我的手上还捏着常家、葛家。他现在已经五转巅峰了,如他这样的天才,肯定想更进一步。但如何晋升蛊仙,只有等到他加入黑家,我才会一步步透露给他。”

%81
“我知道你对常山阴没有好感,但接下来,我闯真阳楼时,还要用到他。这种芝麻大点的小事,以后就不要拿来烦我。听明白了吗?”

%82
“是,大人。”狈君子将头低下,声音微颤。

%83
“嗯,下去吧。”

%84
“属下告退。”孙湿寒怀着无比的失望之情,离开了中枢大殿。

\end{this_body}

