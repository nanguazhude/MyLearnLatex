\newsection{惊变}    %第二百三十一节:惊变

\begin{this_body}

救太白云生,还是夺回人如故?

这一瞬间,方源不免陷入犹豫。

人如故这只蛊,在凡蛊之时,方源就早闻大名。如今晋升成仙蛊,自然威能更高。

根据太白云生之前所言,凡蛊时,人如故能将凡人蛊师的状态,回到过去某刻。成了仙蛊后,人如故不仅能影响他人,更能影响自己。

这个自己,自然指的是太白云生。

太白云生乃是蛊仙,不难推测,仙蛊人如故能对蛊仙有效!

一个能令蛊仙恢复状态的蛊虫,人如故的价值之重大不言而喻。甚至可以说,价值远超普通仙蛊!

因此,太白云生才如此心急,忘记自己身处险境,也要提醒方源夺回人如故。

但太白云生此话,乃是情急之话,一时脱口而出,没有真正冷静思考。

一旦方源真正舍弃他,去试图夺回人如故仙蛊的话,那么太白云生就危险了。

纵然蛊师升仙,整个生命本质都受到了拔升。如此的高空,太白云生摔落下去,顶多重伤不会摔死。

但别忘了,地面上还有残留的其他蛊师。

“夺回人如故,必定浪费时间。我就算夺回人如故,太白云生却与我生出罅隙,该如何是好?”

太白云生呼喊,乃是头脑发热之语。一旦他深陷险境,命悬一线,必然会惊醒:比起仙蛊,自己的性命才更加重要。若是此刻。他却看到自己的“师弟”正在追逐仙蛊,而放弃自己,他会怎么想?

人心是多变的。也是易变的。

哪怕这句话是太白云生亲口喊出来的,彼时彼刻,换位思考一下,都会生出怀疑和猜忌吧。

“我若是掌控定仙游,也就罢了。关键是现在丢了定仙游,大同风幕不断收缩,隔绝内外。要想离开这里。只有开启星门,沟通狐仙福地!”

方源大有为难之处。

若开启星门,必定动用星光蛊。

无相手四处飞舞。风头正盛,力压其他一切存在。

星光蛊一出,必定会遭到无相手群的哄抢。方源纵为蛊仙,在无相手之下。也无保护周全的把握。

“但如今大同风幕已成。再无天地二气支援,无相手虽多,却不会再有新的出现。再加上八十八角真阳楼的碎块不断分解、还原成蛊,四下乱飞……”

无相手乃是无源之水,势盛却不能持久。

方源乃是五百年经验的魔道老魔,绝非短视之人。

“无相手势弱之后,必定是巨阳意志成为战局主宰。到那时,它收拾掉马赵二人。夺回运道真传,必然会优先解决我!”

到那时。巨阳意志夹裹巨阳仙元,拥有众多仙蛊,乃当之无愧的强敌。

方源作为弱者,要对付强敌,最佳的选择就是合纵连横,连弱抗强。

纵观战局,太白云生就是唯一人选。

哪怕是治疗蛊仙,也是蛊仙!

就算蛊虫被抢光,方源还可以借给他。

再者,黑楼兰这些血脉后裔,是巨阳意志的天然盟友。就算方源如何巧舌如簧,舌灿金莲,如此局面,这么短的时间,也说服不动他们转换阵营。

思索了这么多,不过只是短短一瞬的功夫。

方源身似闪电,毫不犹豫,向往下坠落的太白云生疾飞而去。

“师弟,你……唉!”见方源如此选择,冷静下来的太白云生目光闪动,流露出感动之色。想说什么,却终究只是长叹一声。

沿途有不少无相手干扰,方源左右转折,灵活若鸟,迅速拉近他和太白云生的距离。

但就在这时,一个淡淡的人影,忽然浮现在太白云生的左右。

“老东西,你终于落到我的手中了!”人影迅速化实,黑楼兰一把提住太白云生的衣领,将其擒拿。

太白云生纵为蛊仙,但没有一只蛊虫在身,黑楼兰简直是手到擒来!

黑楼兰立即催动蛊虫,将太白云生四肢禁锢,又令其昏迷过去。

看着手中的俘虏,黑楼兰心中大喜。

能在这么混乱的局面下,擒住太白云生,有很大的运气成分。但结果已经铸成,黑楼兰满脑子都是怎么要挟方源,利用人质,索回本命蛊的计划!

但他刚刚仰头,还未开口,就吃惊地看到方源身躯猛地拔升,向高空飞去。

“黑楼兰捉了太白云生,肯定是要挟我,索要那只蛊虫!”方源在第一时间,就猜到了黑楼兰的打算。

他心中一沉,却仍旧冰雪般冷静。

黑楼兰既然如此目的,就不怕他危害太白云生的性命。也就是说,太白云生暂时无忧。

既然如此,方源首先要做的,就是夺回仙蛊人如故!

他左右闪躲,规避无相手的捕捉。背后八翼拼命扑扇,风花蛊、烁蝺蛊种种交替使用,终于在一刻钟后,费劲全力,险险将无相拳击破,艰难地夺回了人如故仙蛊。

因为得到了太白云生的信任,人如故仙蛊在方源手中,却不挣扎。

方源没有调用意志镇压,直接将其放入力道仙窍。

“你师兄的性命就在我的手中,快快将老子的蛊虫还来!”直到此刻,黑楼兰才逮到机会大声威胁。

他满身大汗,追逐方源很不容易。

尤其是他本身移动不足,还提着太白云生。

他亦能理解方源的举动,心中信心十足。

只是刚刚喊完话后,就有一只无相手朝他射来,黑楼兰不敢硬抗,只得后退躲闪。如此一来,他好不容易营造出来的气势。立转虚弱。

方源身形宛若魅影,在无相手间穿梭,他向黑楼兰靠近。面无表情,心中则大叹麻烦。

方源自忖,自己一人无法抗衡巨阳意志,须得依靠太白云生帮衬。

但现在太白云生沦为人质,自然令方源束手束脚。

他虽有人如故仙蛊,却没有得到太白云生的允许可以“借用”。就算可以借用,黑楼兰撕票。太白云生身死,方源可以利用人如故来复活太白云生。

但这里面,也有一个极其重要的问题。

催动仙蛊人如故。是要损耗仙元的。

复活凡人,不要紧。复活一个仙人,损耗的仙元必然极重。

方源手中的青提仙元,原本总共也只有二十二颗。用了一颗。化为无限气态真元。只剩下二十一颗。

这是他抗衡巨阳意志的最大底气。

而复活蛊仙太白云生。至少得需要十颗青提仙元!

不到万不得已,方源不敢硬来。

黑楼兰不愧枭雄人杰,成为王庭之争的胜利者,绝非偶然。他有非凡的才情和天赋,一旦出现机会,更能及时把握。

“难道说,我真的要把他的蛊虫还回去?”

这一念泛起,又被方源强行压下。

黑楼兰如此着急此蛊。必然与众不同。就这样还回去,方源并不甘心。

更关键的是。黑楼兰不值得信任!

时间渐渐流逝,巨阳意志和无相手群的激战,如火如荼。

巨阳意志传承自巨阳仙尊,老道狠辣,原本的实力终于可以得到充分的发挥,展现出无双风采。

它借助运道无上真传光团,一面打击无相手,一面回收运道真传孤本。

坚持了这么久,运道真传光团却仍旧健在。

这道真传中,包含的蛊虫数量、种类之多,实在惊人至极。这一会儿工夫,被无相手夺取的运道仙蛊,就多达六只。

难怪是孤本。

仙蛊唯一,就算巨阳仙尊,穷极己力,也只能将这真传打造到如此地步。

六只仙蛊中的三只,被巨阳意志打破无相拳,顺利回收。其余三只,巨阳意志只能无奈地看着它们,被无相拳破空带走。

“可恶!可恨!”巨阳意志越战越怒,它早已经对赵怜云恨之入骨,如今对马鸿运也再无惜才之心。

“我倒要看看运道真传,能护住你们几时!”巨阳意志咬牙切齿。

若非它要保存运道真传,投鼠忌器,否则早就攻破光团,杀死马赵二人。

马赵二人相互紧拥,对此局面,无力施为,只能坐等受死。

巨大的压力,心理煎熬,让他们的感情迅速升温。

但这毫无作用。

承认他们为主人的霜玉孔雀,早已经死亡。

运道真传光团的体积,已经不足原先的三分之一。眼看着死亡的时刻,越来越近,马赵二人也失去了对生的希望。

方源则和黑楼兰不断僵持,相互讨价还价。双方都对彼此极不信任,这让这场谈判进展极为缓慢。

昂!

就在这时,忽然从风幕之外,传来一声刀鸣。

刀鸣清烈激澈,一时间响彻众人耳畔。

“什么?”

“这是!”

众人齐齐抬头,被这惊变吸引目光。即便是巨阳意志长龙,攻势也为之一滞。

只见头顶上空,原本浑厚的大同风幕,已然破开一道长长的刀口。

众人从风幕里面,透过刀口,完全看得到北原外界的天空!

高空中,单于部族的三位蛊仙,气喘吁吁。

这就是他们催动的仙道杀招插翅刀!

在他们的身边,则站着其他的十多位蛊仙。

原来他们三人密谋不久后,就被其余正道蛊仙找上门来。由于当中的黑家蛊仙诚意十足,一力推动,因此双方协商之后,迅速达成一致,因此提前发动了攻势。

刀口一开,贯通内外。

对于战场的任何存在,都造成巨大影响!

比方源等人反应更快的,是蛊虫!

八十八角真阳楼成为无主之物,被天劫地灾摧毁之后,断壁残垣分解出大量野蛊。

在这关键时刻,凭借本能,这些蛊虫立即腾飞而起,向着刀口冲去。

沿途中,无相手大抓特抓。

“这是冲破风幕,逃脱绝境的绝佳机会!”蛊师们双眼骤亮。

“啊!好多的蛊虫,这都是八十八角真阳楼所化!”北原蛊仙们的双眼也纷纷亮起灼热无比的光。

\end{this_body}

