\newsection{雪洗蛊}    %第二十八节:雪洗蛊

\begin{this_body}

葛谣渐渐放下了防备:“你说的话,好像有点道理。如果你真的是常山阴,那你就是我们北原的大英雄!为了区区一个外来人,我当然不会为难你。可是你怎么证明你的身份呢?”

方源淡淡一笑,神情微微变化,恰到好处地表现出符合前辈高手的一丝傲气:“我常山阴行不改名坐不改姓,我就是我,何须证明自己?小姑娘,你想现在走,我绝不拦着你”

方源的这招欲擒故纵,成功地让葛谣的怀疑动摇,心生巨大的犹豫。

方源趁热打铁:“小姑娘,防人之心不可无,你这点做得对。但是你想想看,之前一路上,我有多少次杀你的机会?别的不提,就提在空中飞行。我只要双手一松,你就落入地刺鼠群当中,绝对十死无生。我这样做了吗?相见就是有缘,你既然有逃婚的勇气,为什么就没有勇气来面对我呢?”

葛谣陷入了沉默。

方源察言观色,心知火候差不多了,便补上最后一句:“跟我走吧,你还小,独自一人在腐毒草原的深处,并不安全。之前你跟我讲过,要寻到雪柳不是吗?你想要雪柳上的雪洗蛊,带回部族,立下大功来抵消这场婚事?说实话,这不太现实。一两只雪洗蛊,改变不了局面。”

“正巧我也需要雪洗蛊,我就先带着你寻到雪柳,捉到雪洗蛊,然后再跟你回家,我来出面说服你的父亲。你看怎么样?”

“真的?”葛谣的双眼顿时亮起来。“我记得阿爸最推崇你了,说你是个大英雄。你说的话。阿爸一定会听取的。但你还需要雪洗蛊做什么呢?你的阿妈早已经过世……”

“唉!”方源语气萧索,神情落寞,先是低下头,随后苦涩一笑,“我自然知道:早在二十多年前,我的老母亲已然毒发身亡。我没有寻到雪洗蛊及时回去,是我的不孝啊。你知道吗,这二十多年来。雪洗蛊已经成了我的执念。我必须要捕捉一只,然后跪倒母亲的坟墓上,向她忏悔。”

说着,方源就流下了眼泪。

少女看到方源的眼泪,彻底相信了方源的谎言,她不由地一阵心疼,劝慰道:“常山阴前辈。这不是你的错。一切都怪那该死的哈突骨!”

“不要再说了,我们走吧。”方源摆摆手,朝前走去。

“对不起,我说错话了。我也不该怀疑你。”葛谣感到一阵的内疚,紧走几步,跟上方源的步伐。在他身后道歉道。

方源宽慰了少女两句,成功地令葛谣的内疚感更加深重。

两人继续朝着腐毒草原的深处走,紫色毒雾越来越浓重,就连毒须狼这种耐毒的野兽,都很少出现了。

行进了两三百里之后。方源和葛谣两人,都不得不时刻催动蛊虫解毒。又过了五六百里。紫色毒气浓郁如墙,几乎达到伸手不见五指的地步。

葛谣的雾雀蛊,在这种情况下,已然失去了作用。不过方源早有准备,仍旧能够侦察方圆数千步的距离。

“常山阴前辈,我们还是回去吧。或者换一个方向,说不定就能碰见雪柳了。再深入进去,恐怕……”葛谣面孔泛出紫色,快要撑不下去了。

但方源摇摇头,拒绝了这个提议,而是坚持往前走。

少女不太清楚,但方源却明白,这雪柳就是要生长在剧毒的环境当中。

而这片腐毒草原,也大有来历。

在腐毒草原的最深处,有一个福地,里面居住着七转蛊仙紫嫣然,号称毒蝎娘子。

她的紫毒福地中,栽种着无数的毒草,有大片的烂沼毒泽,生活着大量的毒兽,培育出大量的毒蛊。

这些毒物累加起来,毒性猛烈,甚至连福地都要承受不住。

因此每隔几年,福地的门户都会打开,将里面浓郁的毒气排泄出来。

这些毒气,大部分就化为腐毒草原中的紫色毒雾。

一些福地中,最底层的生物,也会趁着这个机会,偷逃出来。因此腐毒草原上,就有了大批的毒须狼。

天长日久,紫毒福地周围的环境,就发生了天翻地覆的变化。阴云常年笼罩,不见天日。紫雾缭绕,荼毒生灵。大量的毒草,生长出来。在微微腐烂的土地上,毒须狼群横行,使得这片草原成为生命的禁区。被凡人敬畏地称呼为腐毒草原。

腐毒草原的中央,是紫毒福地。而方源现在所处的地方,是腐毒草原的中部。

在草原内部,毒雾常年郁结,生长着大量的雪柳。方源心中清楚:按照这个程度,只要再坚持一下,继续前进,就必能见到雪柳。

果然如他所料,两人又前进了数百步后,方源发现了一株雪柳。

雪柳高达两丈,枝繁叶茂,树干漆黑如墨,但是垂下的千百柳条,却洁白如雪。在氤氲的紫雾中,显得高贵圣洁。

方源将葛谣领到雪柳面前,少女顿时开心地发出一声高呼。

两人开始仔细地甄别柳枝上的雪白叶片。

很快,葛谣就发现了其中一枚叶片,已然成了蛊。

这就是雪洗蛊,四转蛊虫,价值连城。雪洗蛊的解毒之能出类拔萃,得到北原蛊师的广泛认可。

两人搜寻了半天,寻到了三只雪洗蛊。

方源只要了一只,将其余两只都让给了葛谣,这让少女暗暗感激。

收了雪洗蛊,方源又取出怀中的黑色圆珠。

里面封印着仙蛊定仙游,此时正悄然溢出些微的仙蛊气息。这种气息,经久不散,很容易被蛊仙察觉。

“是时候了。”方源心念一动,唤出空窍中的一只蛊。

当即灌注真元进去。这蛊就化为一座赤红色的铁柜,将黑色圆珠死死封住。

顿时。仙蛊的气息被隔绝,再没有一丝逸散出来。

这是方源在三叉山上,缴获的铁家特有的蛊虫――铁柜蛊。

方源背着铁柜,和葛谣一起沿着原路返回。

走了两三百里,方源停下脚步,取出地藏花王蛊,将铁柜放入花心,种到地底深处。

此蛊高达五转。方源种得很是辛苦。他的真元效用不足,期间只好一边汲取元石,一边缓慢灌注真元。

足足耗费了两个多时辰,才大功告成。

方源曾经在青茅山时,开辟花酒行者遗藏。后者就是埋下地藏花蛊,藏下蛊虫。

地藏花蛊只是二转蛊,一路往上晋升。就能得到五转的地藏花王。

地藏花王,完全盛开时,比地藏花还要庞大十倍。暗金色的巨大花瓣,柔软如丝绸,花心处是暗金色的花液。

但是当地藏花王花蕾完全闭合时,它的整个身躯。比一个婴孩的拳头还要小。

完全缩在地里深处,不显露丝毫气息。

方源将地藏花王种下,又细心地销毁了地上相关的一切痕迹。到达此刻,他才算真正彻底的,将仙蛊定仙游隐藏起来。

方源的空窍。不能存下定仙游,他只能出此下策。将仙蛊就地埋藏起来,以待日后取用。

在这茫茫的草原上,谁会想到一只珍贵至极的仙蛊,就埋藏在这里呢?

不过也有破绽,就是仙蛊留下一路的气息。这些气息经久不衰,有可能引来蛊仙。

所以方源一路上,断断续续地封印仙蛊,也是为了防备蛊仙的搜查。

“不过这可能性,并不是很大。除非来的蛊仙,拥有可以侦察的仙蛊,并且这只仙蛊能洞穿地面,查探到地下数百里的深处。”

若定仙游,真的被某位蛊仙发现,夺走了。方源也只能认倒霉。

但就算如此,他也绝不会将定仙游带在身边。相比较仙蛊,他还是觉得自己的性命要更加宝贵。

埋下地藏花王蛊后,两人又继续往回走,回到那片战场。

到达此处,方源就改变了方向,不再沿着原路返回。而是寻了另一个方向跋涉,期间就靠葛谣的归心蛊指点方向。

回程并不顺利,期间方源和葛谣屡次遭受狼群的冲击。

其中有三四次,遇到千狼群,方源只好再次抱着葛谣飞到天空避祸。

当两人有惊无险地闯出来,到了腐毒草原的外部时,遇到的艰难险阻便小了许多。

……

眼前,一支上百头的毒须狼群,向方源和葛谣二人扑来。

方源朗笑一声,不惊反喜,振翅飞上天空,居高临下,伸手一指。

“驭狼蛊,去!”

一只二转的驭狼蛊,便化作一团蓝烟,往下落去,罩住百兽狼王。

百兽狼王激烈挣扎,魂魄上传来强烈的反抗之意。但面对方源百人魂的强势,它很快就一败涂地。

“嗷唔,嗷唔。”

当方源落在草地上的时候,这只百兽狼王摇摆着尾巴,像条小狗似的,围绕着方源脚边打转。

它带来的狼群,也在它的命令下,一动不动,成了方源的走狗。

这已经是方源收服的第二支狼群。

就这样一路下去,方源手中的实力越来越强,身边的狼群也越加庞大。

之前的逃亡式的跋涉,不知不觉间,已经变成闲庭信步的郊游。

当方源身边的狼群,扩张到两千多头,拥有四头百兽狼王后,腐毒草原也不再可怕。

有了安身立命的本钱之后,方源开始处理身上的,来自南疆的蛊虫。

北原人排外,以他原来的面貌,走到哪里都会遭到排斥和警惕,很不方便行事。而这些南疆的蛊虫,也是泄露他身份的巨大破绽。

生性谨慎的方源,是绝对不会允许这些破绽存在的。

所以身上的蛊虫,只要不是北原的,都必须处理干净。(未完待续。如果您喜欢这部作品,欢迎您来起点投推荐票、月票,您的支持,就是我最大的动力。手机用户请到阅读。)

\end{this_body}

