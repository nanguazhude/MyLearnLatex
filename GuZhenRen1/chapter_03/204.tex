\newsection{不自由毋宁死}    %第二百零四节:不自由毋宁死

\begin{this_body}

天地中心,三气融汇。

得到巨阳意志的帮助,浓厚气流几乎将太白云生包裹成一个气茧!

他身处其中,浑身伤势已然全无,整个肉体、魂魄、精神都在不断地升华着。

他回顾过往一生,一幕幕在他脑海中快速闪现。

大道的奥妙,在他心中流淌,灵光不断爆闪,以往桎梏他的修行难题,一个接着一个得到了完美的解答。

这一刻,天地仿佛是一位无私的教师,对太白云生倾囊而授。

但天地的奥秘,实在太过于浩瀚博大,太白云生知道得越多,越感觉到自己的渺小。

他只能从自己的宙道出发,不断精深,延展开去。

他对宙道的理解,达到了毕生的巅峰!

师法自然!

人乃万物之灵,蛊乃天地真精。蛊师用蛊、炼蛊、养蛊,其实是在不断交流,不断探索,不断向天地学习的过程。

升仙之际,蛊师再不用通过蛊虫,间接地了解天地,而是和天地形成直接的交流。

这种交流的宝贵机会,终其蛊仙一生,恐怕也只有这一次机会。

五转巅峰,是凡人蛊师的终点。而六转蛊仙,则是超凡脱俗,成就蛊仙的新起点。

在这个关键的起点,各个蛊仙们达到的成就,为将来积蓄的潜力也各有差别。

“真是美妙的体验啊,可惜我的人气不足了……”太白云生意犹未尽,一脸惋惜之色。

他的人气已经消耗殆尽,这还是得到了人气仙蛊的帮助,否则时间更短。

气茧消散。半空中重新显露出太白云生的身影。

三气融合压缩成混元三色气团,凝聚在太白云生原先空窍之处。

到此程度,升仙第二步纳气,已然完成。

接下来,便是升仙的最后一步――放蛊!

太白云生郑重无比。取出江如故、山如故、人如故三蛊。

这三只蛊,是他最核心的蛊虫,极为熟悉。其中人如故,更是本命之蛊。

“终于到了最关键的时刻……”太白云生首先将本命蛊,直接投进体内的混元三色气团当中。

轰!

耳畔骤然响起雷鸣般的幻听,太白云生浑身剧震。

三色气团。原本相互交融,不断流转。人如故闯入之后,像是点燃了炸药一般,令混元气团瞬间发生剧烈爆炸。

不破不立,死中再生!

以凡登仙,就在此刻!

这一声炸响。真正妙不可言,炸出一场生命的奇迹,炸出一片全新的天地!

凡窍已碎,仙窍生成。

好仙窍,里面天空湛蓝如水晶,大地荒野如黄石!

仙窍宛若刚刚出生的婴孩,急剧需求营养。对外爆发出一股绝强的吸摄之力。

呼呼呼……

天气、地气对准太白云生疯狂灌输,直达仙窍。

仙窍中,地域急速扩张。一百万亩、两百万亩、三百万亩……

而时光流速,亦从原先的一比一,不断攀升,到一比十,一比二十……

气流汹涌而来,仙窍渐渐不稳。

太白云生便再投入江如故、山如故,稳住局面。

仙窍面积不断扩张,冥冥当中亦接引光阴长河的支流。使得时光流速稳定攀升。

期间,太白云生不断放入蛊虫,始终维持仙窍的稳定。

五百万亩、六百万亩、七百余万亩!

一比三十,一比三十一,一比三十二。一比三十三!

到达此步,仙窍成长到了极点,宛若吃饱喝足,戛然而止。

但天空中,仍旧在倾泻清辉天气。地面上,金黄地气也不断涌上来。

到此刻,仙窍吸取天地二气的速度缓慢下来。起先是大口牛饮,现在如徐徐小酌。

天地二气汇入仙窍,如青金二色云雾,充斥整个仙窍。

浓郁的气流中,酝酿出颗颗仙元。

正是六转蛊仙的青提仙元,足足生成三十六颗。

苍穹中的劫云,地面上的灾尘,开始渐渐消退。

仙窍中,却仍旧有雾状的天地二气,并未散去。

它们凝聚在人如故、江如故、山如故等等蛊虫周围,使得这些蛊虫附近气雾浓郁,结成气茧。

二气融汇,天地交感。

在气茧中,个别的蛊虫开始酝酿着某种玄之又玄的变化。

“正是此刻。”太白云生心中划过一道蛊方。

这蛊方,只是理论蛊方,从老乞丐的蛊仙传承中来,是留下传承的蛊仙的理论推演。

太白云生并不能保证,这个理论蛊方的成功。

但他已经毫无退路了。

他寿命无多,即将耗尽。成仙之后,他感官达到一个全新的层次,尤其是对时间的感应提升最高,精确到每个呼吸。

这种新奇的提升,让他没有多久的高兴,就陷入到无声的压迫,甚至惶恐之中。

因为他感觉到,留给自己的时间真的不多了,宛如阳光暴晒下的小小泥水洼,而他的寿命就仿佛水洼中残留下的那薄薄一层的水迹。

原先他图谋寿蛊,因为方源暗中阻挠而失败。血道大殿中的经历,更成了他的心结。

现在他的唯一希望,就在于人如故蛊。

按照理论蛊方所示,将人如故蛊提炼到六转,便能给自己施展。

“给我炼!”太白云生轻喝一声,脑海中浮出无数念头。

这些念头,组成理论蛊方,直灌而下,进入仙窍之后,径直扑入人如故蛊的浓厚气茧当中。

这天地之气,乃是万物母气。相互交感,演化天地千材万物。

这两股气,便是万物的源头,可以替代任何一件炼蛊材料。

太白云生的念头。像是一把钥匙,又或是给了天地二气一个精准的方向。

念头扑入气茧当中,顿时引起激烈变化。

虽然不及刚刚,放蛊第一步时,三气爆炸般剧烈。但亦声势不小。

气茧爆发出吞吸之力,不断吸食仙窍中羁留的天地二气。

呼呼呼……

气流不断卷席而来,形成呼啸的狂风。

以人如故蛊为中心,化为一个风眼,大量的天地之气投入其中,化为炼蛊的资粮。

“就是这样。就是这样……”太白云生漂浮在空中,轻声喃喃,语气中有欣慰,亦有沧桑。

多番的努力和冒险,终于让他在此刻看到了成功的希望。

尽管过程中,充满了不可思议。

若没有真阳楼出手。太白云生必然陨落在第二步中。

想到这里,他回眸望了真阳楼一眼。

自己并非巨阳血脉,但偏偏真阳楼帮助了他,这是有目共睹的事实。

“巨阳意志莫非苏醒了?”太白云生暗中猜测道。他有蛊仙传承,见识并不比黑楼兰、耶律桑等人少。

“巨阳仙尊严禁蛊仙出入福地,我在这里升仙,已经犯了他的忌讳。他怎么还会出手帮我?”太白云生心中疑惑。

任凭他阅历丰富。也想不到真阳楼中发生的事情。

真阳楼中,霜玉孔雀尖啸连连,周身青泥碎片乱蹦,锁链不断晃荡,相互之间碰撞出铛铛声响。

封印它的力量,在和稀泥的侵蚀下,不断消融。

但霜玉孔雀却没有一丝高兴的心情,而是充满了惶急。

巨阳意志则在朗声大笑:“小麻雀,你这样的挣扎是没有用的。”

通过太白云生之手,巨阳意志利用仅有的手段。巧妙地打击到王庭地灵,从根本上削弱了它的力量。

尽管束缚霜玉孔雀的力量,不断减弱,但巨阳意志已经赢得了时间。

围困他的特意蛊阵,他已经参透大半。

仙尊布置。岂是那样简单?

只要巨阳意志,彻底腾出手脚来,自然有大把的手段,譬如金道、水道、炎道等等,将地灵重新封印。

“我沉睡得太久了点,不过不要紧。先将不听话的地灵收拾掉,再来个彻底扫荡,将真阳楼中的微小漏洞都消除干净,到那时八十八角真阳楼,又将是铁桶一般,足以再矗立十万年!”巨阳意志语气深沉缓重。

“我就算是死,也不会让你得逞!!!”听到“矗立十万年”这个话,霜玉孔雀彻底炸毛了。

地灵是执念所化,对于认主一事,本就是无法通融。霜玉孔雀,更和旁的地灵不同,骄傲无比,不肯任何的屈服。

但霜玉孔雀没有再用力挣扎,而是忽然诡异地萎靡下去。它躺倒下来,但目光仍旧宛若刀锋,充满了仇恨和绝然。

巨阳意志一愣,旋即意识到什么,怒道:“小麻雀,你竟敢如此!”

真阳楼外。

“劫云和灾土,都在渐渐消散,难道说太白云生大人成功升仙了吗?”不明真相的人,看到这里,心怀激动。

“想不到太白云生,居然达到了第三步!看他的样子,仙窍已成。就是不知道,他成就的是何等福地?”耶律桑口中喃喃。

蛊师升仙,若过第二步,便能生仙窍。

六转、七转蛊仙的仙窍便是福地,八转、九转蛊仙的仙窍便是洞天!

“福地分大中小三等,融合的天地人三气越多,福地就越高等。小福地方圆至多三百万亩,引动光阴小脉支流,生成仙元十余颗,资源贫瘠。中等福地方圆四到六百万亩,引动光阴中脉支流,生出仙元二十余颗,物产丰富。上等福地则有七到九百万亩地域,引动大脉光阴长河的支流,仙元数量超过三十,天地二气残留得多,相互交感,将凡蛊炼成仙蛊!”

黑楼兰目不转睛地望着,脑海中则划过相关信息。

他眉头微皱,太白云生得到真阳楼的帮助,渡过了第二步,到达第三步骤。目前看来,他升仙的希望很大。

“若是太白云生成仙,那么我又该用怎么样的态度,来对待他呢?”黑楼兰思考这个难题。

“若不出意外,太白云生必得上等福地无疑。然而炼成仙蛊,却有风险,会在仙窍中酿成灾劫。”方源目光不断闪动着。

以凡升仙,自然别具风险。

蛊师如此,蛊虫亦如此。

人吸纳天地二气,三气融汇,酿出天劫地灾。蛊虫吸纳天地二气,亦会产生灾劫。

在自己的福地中产生灾劫,外人难以插手。

真阳楼能影响太白云生,却影响不了他体内的仙窍福地中来。

换句话说,太白云生能依靠的,就只有自己的力量了。

“嗯?”方源忽然仰头,看向天空。

王庭福地的天空,白天金光灿烂,夜晚银辉温柔。

但此刻,金色的苍穹中却显出一道道的漆黑的痕迹,从中痕迹中,绽射出点点星芒。

这是外界北原的夜空。

随后,整个王庭福地开始颤抖起来。

众人惊呼不止。

原本快要消散的劫云、灾尘,又重新浓郁起来。大量的天地之气,像是扑火的飞蛾,义无反顾地灌输到太白云生的身体中,直达他的仙窍。

“怎么回事?!”黑楼兰瞠目结舌。

“到底发生了什么!”耶律桑抱着脑袋,失声大叫。

“王庭福地过度汲取天地二气,伤及根本,因此显露外界,要贯通北原了!”方源心头震动,目光从太白云生的身上,转移到八十八角真阳楼。

他猜到了真相。

霜玉孔雀居然如此骄傲,宁愿自我消亡,也不愿再受到巨阳意志的镇压。

它说到做到!

ps:本来早已经码好的,但是不满意,那样写节奏太慢了,于是改了好几遍。

\end{this_body}

