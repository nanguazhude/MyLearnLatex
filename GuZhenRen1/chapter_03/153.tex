\newsection{巨阳大局晶中宝}    %第一百五十三节:巨阳大局晶中宝

\begin{this_body}

%1
方源置身在水晶长廊之中。

%2
眼前光华流转,四壁均是晶莹剔透的晶壁,可照人影。

%3
方源目光逡巡一圈,后方、左右皆是墙壁,只有前方道路可走。

%4
“终于来到了秘藏阁。”方源的嘴角翘起一个微小的弧度,迈开步伐,朝前走去。

%5
水晶长廊修长通透,不出五步,方源便看到左右的晶壁中,出现了一件件蛊虫,或者蛊方等等,仿佛是琥珀中的虫子,静静地位于晶壁之内。

%6
这些都是秘藏阁的奇珍异宝,要想获得,得拿相应价值的东西换取。

%7
“嗯?这里有一只木道的天元宝王莲。”方源缓缓停下脚步,发现眼前的晶壁中,封印着一只熟悉的蛊虫。

%8
此蛊乃是一朵蓝白相间的盛放莲花,脸盆大笑。来历非凡,由元莲仙尊所创。

%9
共一套,分别是三转的天元宝莲,四转的天元宝君莲,五转的天元宝王莲,以及六转的天元宝皇莲。

%10
其中,天元宝皇莲乃是十大仙蛊第六,价值和春秋蝉不相上下。如今掌握在琅琊地灵手中。

%11
方源曾经用过一只天元宝莲,很大程度上弥补了他资质的缺陷,带给他相当大的帮助。只是此蛊往后晋升,方源不仅缺乏相应的蛊方,而且合炼代价高昂,须得利用一道天然元泉。这泉还得元力饱满,不能是那种使用了多年,底蕴不足的元泉。成功之后,这道元泉就彻底废掉。

%12
四转元莲,须废七口元泉。到五转,再废九口。到六转,再废十一口。

%13
这个数据,是基于南疆等地的元泉。若用北原的元泉,至少得再翻上六成上去。

%14
方源当时要晋升天元宝莲,实在过于困难。况且资质又有极大改善,他便舍弃了天元宝莲蛊。

%15
但现在,这朵五转的天元宝王莲,却是让他心动。

%16
“天元宝莲这一系,号称移动元泉。一经炼化,便可以产出天然真元。三转的天元宝莲蛊,我早就不合用了。但五转的天元宝王莲,却正合适我现在的状态。”

%17
但是要拿到这只天元宝王莲,按照规矩,方源就得用等价的宝物换取出来。

%18
“那就姑且换一换吧。”方源现在可以说是财大气粗,坐拥福地就等若蛊仙的资本。

%19
要换取天元宝王莲,对旁人来讲,需要舍了老本的大买卖。对于方源来讲,却是一个只关乎如何选择的小问题。

%20
“该拿什么东西来换呢?”

%21
方源将心神探入两个空窍。

%22
空窍中的力道、奴道蛊虫,当然是不能换得。第一本命蛊春秋蝉,更是拿都拿不出来。

%23
但除此之外,还有不少杂七杂八的蛊虫。

%24
这些蛊虫的单只价值,当然低于天元宝王莲,但是一个不行,可以用两个、三个一起去换。

%25
“嗯?这里怎么会有十八颗泉蛋蛊?”

%26
方源在第一空窍中,惊讶地发现,居然多出了一批珍稀的五转蛊。

%27
但他很快就想到了原因:“我差点忘了,这是我上等通关的奖励,被真阳楼直接送进了空窍之中。”

%28
按理说,这蛊师的空窍,乃是修行根本所在,隐秘中的隐秘。但真阳楼不愧是长毛老祖亲造,又有巨阳仙尊布置,拥有巨大威能,能够直接将蛊虫,送入蛊师空窍当中。

%29
泉蛋蛊虽然不及天元宝王莲,但同样也是五转蛊。

%30
泉蛋蛊外形如白色鹅蛋,乃是斩杀强大的蛋人皇所得。将泉蛋蛊种在地底深处,就能形成一道元水泉眼。

%31
元泉乃是蛊师修行的根本,因此大型部族、家族都会收集泉蛋蛊增强底蕴,以备不时之需。即便是蛊仙们也常常求购,种在自家的福地中,形成元泉,滋润万物。

%32
方源信手取出其中一只泉蛋蛊来。

%33
但此蛊品相不好,蛋壳表面布满微微裂痕。蛊中有明显的真元痕迹,显然是被人使用过。

%34
方源眼中精芒一闪即逝,冷笑一声:“看来此蛊应该是被催动时,遭到了飞手雪的抢掠。八十八角真阳楼中的通关奖励,都是如此来源。”

%35
巨阳仙尊为了实现他“家天下”的美梦,同时也为子孙后代谋利,煞费苦心布置千古大局。

%36
先是王庭之争,削弱铲除其他部族。后是利用十年雪灾,将北原各地的天材地宝充入八十八角真阳楼里。

%37
要知道,蛊师修行对物资依赖极大。

%38
巨阳仙尊设立八十八角真阳楼,就是釜底抽薪。将物资掠夺进来,给予自家血脉后裔,更绝了其他部族的崛起希望。

%39
方源五百年前世时,中洲蛊仙攻破王庭福地,正是利用了真阳楼的这一特性。

%40
他们派遣棋子,混进王庭福地。同时又故意洒下蛊虫,任凭飞手雪劫掠进去。最终从真阳楼的内部造成破坏,轰出关键性的缝隙。

%41
中洲蛊仙们毁灭王庭福地之后,又将巨阳仙尊的险恶用心,大肆披露,公之于众,引发了北原的大震荡,民意沸腾喧天。

%42
但震荡归震荡,北原仍旧是仅次于中洲的第二战力。由各大超级势力,黄金部族联合镇压,使得北原稳固不动。

%43
经由黄金部族统治北原这么多年,早已经根深蒂固,绝非民意沸腾就能撼动得了。

%44
民意再大,没有武力的支撑,就屁都不是。

%45
方源将手中的泉蛋蛊,挪近晶壁,向里面封印着的天元宝王莲靠拢。

%46
附近的晶壁,开始发出微亮的赤光,继而又产生橙光,橙光之后又有黄芒。

%47
赤橙黄三光交相辉映后,却是再无动静。

%48
方源便又取出两只泉蛋蛊,靠近晶壁。

%49
于是,又有一道绿色的光晕,闪烁而起。只是绿芒不盛,被其他三光压过一头。

%50
方源冷笑一声,取出第四只泉蛋蛊,一齐靠向晶壁。

%51
这次,绿光彻底盛开,和其他三光相差仿佛。

%52
晶壁由实化虚,微微颤动。里面的天元宝王莲从中悠悠飞出,而方源手中的四只泉蛋蛊却是不受控制,飞进了晶壁当中,取代天元宝王莲的位置。

%53
天元宝王莲悠悠落到方源的手中,方源真元一吐,顷刻炼化。

%54
八十八角真阳楼威能非凡,在楼中能令蛊师瞬间炼化凡蛊。

%55
“赤橙黄绿青蓝紫黑……在这里秘藏阁中,有这八种标准。彩光越多,说明宝物越是珍贵。五转的天元宝王莲只达到绿光,就能换取。这在八种标准当中,不过是中层而已。看来秘藏阁果然非同凡响,里面藏宝极多!”

%56
方源心中稍稍估算,对秘藏阁的价值,顿时又有了更深刻的认知。

%57
他伸出手来,抚摸眼前的晶壁。

%58
从手掌上,传来冰凉的触感。晶壁里面,就是方源刚刚付出的代价——四只泉蛋蛊。

%59
方源试着调动它们,不见一丝动静。

%60
他心中赞叹不已。

%61
毫不谦虚地说,在这当今天下,方源对八十八角真阳楼的了解,足可以名列三甲。

%62
皆因他手中掌握着琅琊地灵给他的详实情报。

%63
方源对八十八角真阳楼了解越深,便越能感触到此楼的鬼斧神工,绝妙才情。

%64
不说别的,单说这眼前的晶壁就大有来头。

%65
这晶壁乃是当初长毛老祖生剐了数万名蛊师,以某种秘法,取出他们的空窍。以他们的空窍窍壁为主材,再结合玄冰蛊、冰墙蛊、华玉蛊、缓更蛊、生机蛊种种合炼而成。

%66
这晶壁是封存蛊虫的绝佳地点,蛊虫在当中陷入自然沉眠,可保存数百年上千年不损分毫。

%67
按照情报中的记载,刚刚炼制出来的晶壁,时刻闪烁着温润如水的波光,人走在其中如梦似幻。

%68
到了如今,晶壁虽没有之前的风采,但仍旧晶莹剔透。

%69
时光荏苒,水滴石穿。

%70
光阴的力量,是最宏伟的天地伟力之一。

%71
纵然是九转之尊,也败于时光。真阳楼虽是仙蛊屋,但每隔十年就频频开动,亦有许多损耗。

%72
要不然方源前世,八十八角真阳楼也不会被中洲蛊仙攻破。

%73
如今晶壁光韵不再,甚至墙角下,都堆积了一层薄薄的晶粉。

%74
方源继续朝前走。

%75
晶壁中,封印着各种蛊虫、蛊方、炼骨奇材,更多的则是历代蛊师留下来的修炼心得。

%76
这些心得经验,弥足珍贵。能够到达秘藏阁的,都是上等通关的人杰,他们留下的东西根本就差不了。

%77
又行了一段距离后,方源停下脚步。

%78
前方水晶走廊延绵远去,看不到尽头。但一座石碑,挡住了方源的去路。

%79
石碑四四方方,有方源膝盖高度,上书四个北原文字——来客止步!

%80
八十八角真阳楼是巨阳仙尊设立,为他的子孙后代考虑的。但秘藏阁中,未必进来的都是身怀巨阳血脉的蛊师。

%81
巨阳仙尊考虑到这点,将更加贵重的物资,都布置在石碑之后。像常山阴这样的外人,就只能望宝惜叹了。

%82
但方源却不是常山阴。

%83
“来客止步?哼。”方源不屑地冷笑一声。

%84
他先试着越碑而过,结果被一道无形的墙壁挡住去路。

%85
“这么多年过去,八十八角真阳楼虽有损耗,但来客止步碑的功效仍在,不可强行突破。”方源试探一番后,盘坐下来,面对石碑。

%86
他此行筹谋良久,自然早有充分准备。

%87
当即,他掏出一只蛊虫,射进石碑当中。

%88
炼化八十八角真阳楼的第一步,开启了!

%89
ps:这个月会很忙,每天正常一更,可能断更。当然也有可能多更,只是这可能性有点小。最近新书都还不错,大家可以看看。

\end{this_body}

