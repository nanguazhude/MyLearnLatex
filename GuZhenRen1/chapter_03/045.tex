\newsection{小人之心君子之腹}    %第四十五节:小人之心君子之腹

\begin{this_body}

此后事情的发展,果然如方源所料的那样。

仅仅一天之后,葛家老族长拜访方源,说自己想通了,不想借住红炎谷,而是想举族迁徙,去参加英雄大会,争取进驻王庭。

方源知道葛家老族长的意思,无非是想再次利用常山阴,来脱离蛮家的掌控。

方源欣然同意,若他独自一人在草原赶路,也比较麻烦。有了葛家一族同行,其中的风险就会下降很多。同时,也是对自己的一个掩护。

“葛老哥看得明白,只是启程之事,宜早不宜迟。一旦有什么风吹草动,想必蛮家那边便会立即有所察觉。”方源叮嘱一声。

葛家老族长心中凛然,单听这话,他就明白,常山阴绝对是一个明白人。

方源又继续道:“本来是答应了蛮家,想去那边拜访。但既然葛家要迁移,保险起见,我就不去拜访了。就说我近日来修行,感到修为有回复迹象,索性闭关。我这就修书一封,要需要老哥找人,代为转交。”

如果葛家不迁徙,那方源拜访蛮家没有什么不妥。

但如今葛家要走,蛮家自然不愿意放过嘴边的这块肥肉。蛮家顾忌的是葛家和常山阴的联合,说不定蛮家就会软禁常山阴,掉过头来再来对付葛家。

先前,葛光找到风狼群的围杀,很大可能就是蛮家的手脚。蛮家到底是正道,顾及影响,杀害常山阴倒不会,但葛家这块肉太肥了,以各种理由软禁常山阴,蛮家也是能做得出来的。

葛家老族长听了方源这话。深深地望了他一眼,站起身来行了一礼:“在老弟面前,我这点才智算什么?我先前是糊涂了,整个局势还是老弟看得明白啊。”

“呵呵呵,身在局中,常常自迷,这是常有的事情,老哥无需挂怀。只要脱离这片地域,葛家就是海阔天空!”方源宽慰了葛家老族长一句。之后当场写信,再交给葛家老族长。

“葛老哥,我还要继续修行,就不送你了。”

“今日我下令准备迁徙,信一定会送到。告辞。”

葛家老族长拿着信,退出房门。

回到王帐之后,他便立即召集家老,下了命令,全族准备迁徙。

经过葛谣联姻一事,葛家的家老们,都对蛮家感官极差。纷纷赞叹此举英明。

葛家父子回到书房,老族长当场就拆开方源的信。

“阿爸,你这行为不太好吧?”葛光感到不好意思。

“嘿,今天为父就给你再上一课。这是常山阴写给蛮图的信。但他却没有动用信蛊,你知道为什么吗?”葛家老族长嘿然一笑。

“是因为他没有信蛊吗?不,如果他想要用信蛊,大可以向我们葛家借用啊。”葛光思索了一下。忽然眼前一亮,“难道他是故意这样做的?”

“呵呵呵。不错!他之所以用这普通的信,就是想让我们看的。葛家要迁徙了,接下来他会和我们同路,这封信就是他用来表明坦诚合作的意向。你过来,我们父子一起看看。”说着,葛家老族长便拆开了信。

信中的内容很简单,说明自身原因,要闭关恢复修为。对不能亲自拜访蛮家,表示遗憾,今后有机会一定会弥补这个遗憾。

在信的后段,方源又向蛮家求购骨竹蛊,表示愿意用高于市价两成的价格买卖。同时,还开了一大堆的炼蛊材料,以及三更蛊等等蛊虫,希望可以交易。

“原来常山阴叔叔,需要这些东西。阿爸,我觉得我们葛家应当尽量地满足他。毕竟他帮了我们葛家这么多。”葛光道。

葛家老族长却是盯着手中的信,眼冒精芒,心生一股寒意。

葛家和蛮家的这场争斗,多在暗处较量,没有撕破脸皮明争。这是正道的游戏规则。

牺牲者有许多人,葛家的一位家老因为蛮多的山门挑战而死,葛谣也命丧腐毒草原。除此之外,还有不少葬身狼口的蛊师。

在这场争斗中,不论蛮家,还是葛家,都不是胜利者。蛮家没有达到目的,葛家有许多牺牲。

惟独一人,却是实实在在的得益者。

这个人就是“常山阴”。

想想看吧,“常山阴”从腐毒草原而来,身上一穷二白,蛊虫都不全。现在呢?

在这场暗斗中,他赚得瓢盆满钵,单单元石方面就有一百多万的收益。更别提那只五转的蛛丝马迹蛊。

葛家老族长忽然明白:葛家利用了常山阴,但常山阴何尝不是利用了葛家?常山阴看似无辜地被夹在两族之间,陷入争斗的漩涡,惹上了不该惹的麻烦。但事实上,两边的人都不想得罪他,他反而左右逢源!

“我们不需要为常山阴准备这些东西。这信上的东西,蛮家会给他送过来的,甚至还极有可能无偿地奉献。”葛家老族长吐出一口浊气,好似要把心中的寒意驱除掉。

“啊?”葛光惊讶无比,“这不可能吧?常山阴叔叔明显在帮助我们,蛮家不会这么笨吧?”

“身处高位者,眼界是不一样的。这些东西,能值多少?不过十几万块元石罢了。对于蛮家根本不值一提,九牛一毛都算不上。付出这些代价,交好一位高手,何乐而不为?你再想想我们又给了常山阴多少?”

葛光立即想到那一百万的元石,还有那只五转蛊蛛丝马迹。

葛家老族长又深深叹息一声,这里面还有一层意思,解释给葛光听还嫌太早。

常山阴为什么要和蛮家做交易?

这其实不是做交易,而是做交情!以此手段,常山阴向蛮家表明,自己虽然违约,没有拜访蛮家,又身处葛家。但他却不是蛮家的敌人。他不想和蛮家成为死敌,而想成为朋友,因此可以进行交易。

蛮图不是傻子,自然能读懂常山阴在信中,释放出来的善意。蛮家如果拒绝这场交易,那就是拒绝方源的善意。如果按价买卖,一板一眼,那就是表明冷淡不满的态度。如果直接赠送,就是说明蛮家接受这股善意。愿意和常山阴成为朋友。

这场交易不是重点,重点是交易背后的东西。

这种隐晦而又含蓄的交流,正是正道高层经常玩弄的把戏。

葛家老族长忽然又灵光一闪,冒出一个念头:“这个常山阴,之所以如此帮助葛家。也许未必是因为正直的本性。而是因为他只有和葛家站在一起,才能获得最大的利益。”

蛮家本身势大,多出一个常山阴,只能算是锦上添花。但葛家弱势,多了常山阴,却是雪中送炭,倾覆实力天平的重要砝码。

这个念头。让葛家老族长浑身微微一抖,心中寒意骤盛,冰凉几乎彻骨。

老族长下意识地就又否定了这个猜测:“如果常山阴这等的英雄,都如此谋算。那这世间还有何正义和光明呢?我这是以小人之心,度君子之腹啊。”

三天之后。

蛮家父子等人,站在山丘上,看着葛家一族缓缓向南迁徙。

“父亲大人……孩儿有一事不明。想请教父亲。”蛮多开口道。

“说。”

“您将信中谈及之物,全数奉送给常山阴。这点孩儿能够理解。但为什么还要赠送给葛家,三万担粮草呢?葛家这块肥肉飞走了,我们还要往上面倒贴。这……”蛮多神情分外不甘。

蛮图目光深沉,看着葛家大部队离开的背影,只说了一句:“蛮豪,你来解释吧。”

站在一旁的家老蛮豪,则笑着解释道:“三公子无需忧虑,其实族长大人早有安排了。葛家想要就这么离开,那是他们想得太简单。三万担粮草中,已经被暗中用了许多引狼蛊。同时,已经有族人在前方勾引,大约有三支万狼群,在等候着他们呢。”

“原来如此!”蛮多顿时想明白了,“父亲英明啊。一旦葛家抵挡不住狼群,我族就会出动,将其救下,借此良机吞并。就算日后有人质疑,这三万担粮草,也足以表明父亲大人的坦荡和真诚,堵住那些怀疑者的嘴巴。只是……”

说到最后,蛮多语气迟疑。

蛮豪叹息一声,接口道:“只是这样一来,葛家也损失惨重,我族吞下葛家的获利,也要减少许多。甚至因为照顾伤员,还要投入一部分财富。”

但蛮多摇摇头,蛮豪说的并非是他的顾虑:“只是葛家有那个常山阴在。他号称狼王,这些狼群能阻止他吗?”

蛮图的眉头,微微皱起来。

蛮多的话说中了他的心思,他也是有这样的担忧。

可是葛家走的太干脆了,蛮家乃是正道,一举一动要考虑影响,短时间内只有引出这三支万狼群。

如果葛家撑过这狼群的攻杀,那蛮家只能看着他们扬长而去。但若在攻杀中,葛家损失惨重,那么蛮家就有出兵“援救”的理由。

这个计划,最大的变数,就是常山阴。

“三公子勿忧,这常山阴虽然号称狼王,但那是二十多年前的事情了。他现在的修为已经掉落到四转初阶。那次晚宴上,我们也暗中探查过,他的魂魄也不再是千人魂,如今只有百人魂的程度。”蛮豪带着不屑的语气道。

“呵呵,他就算是狼王,也不过是苟延残喘的老狼王。再者,他的手中有什么底牌?只有一千多只风狼,一千多只毒须狼,一千多只水狼。哈哈哈,面对成千上万的狼群,这些兵力能成什么事?依我看,不久后他的名声就要毁了。我们就等着吞并葛家好了。”

蛮多没有直接反驳,只是说道:“但愿如此吧。”

(未完待续)

------------

\end{this_body}

