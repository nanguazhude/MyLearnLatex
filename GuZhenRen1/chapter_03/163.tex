\newsection{墨瑶意志}    %第一百六十三节:墨瑶意志

\begin{this_body}

方源仔细打量着手中的招灾仙蛊。

此蛊高达七转,经过朱红巨碗的孕养,此刻已经彻底成形。

它有小拇指般大小,通体灰白,十分精致袖珍,如同蚕蛹。

随着方源的把玩,招灾蛊不断吸收他的气息,方源可以明显地感受到,自己和招灾蛊的那层内心最深处的联系,更加深厚绵长了。

方源心湖生出阵阵微澜。

没有感慨,那是假的。

前世他费劲千辛万苦,屠戮苍生,才炼成一只六转的春秋蝉。如今靠着重生的优势,年纪还未过百,就获得了招灾蛊,比春秋蝉还要更高一转。

虽然招灾蛊效用过于奇特,是主动牺牲自己,引动天灾地劫加身。但方源坚信,蛊师用蛊存乎一心。

像锯齿金蜈,都能被后来者,一位四转凡人蛊师“电锯狂魔”耍出不同的花样出来。那他方源,又如何不能开辟出新的妙用呢?

“其他的暂且不提,单单这枚招灾蛊的存在,就证明了传说中的运道流派,是的确有的。”

巨阳仙尊一生都幸运无比,传闻中,便是因为他独创的运道蛊虫。

然而,这个传闻一直没有得到确切的证实。

但如今,方源手握的招灾蛊,就隶属运道。乃是炼道宗师墨瑶,通过八十八角真阳楼中的排难蛊感应,炼制而成的仙蛊。

排难蛊、招灾蛊,这就是运道流派的蛊虫。

碗壁的墨文。更是一大强力佐证。

墨瑶甚至在文中直接指出,巨阳仙尊的运道蛊虫,能够抢夺他人的好运施加在自己的身上。同时还能将自己的坏运转移出去,祸水东引。

作为一个完整的流派,运道亦是结构井然,包含攻击、防御、转移、治疗等各个方面。只是针对打击的,是每个人都有的无形无色的运气。

运道被巨阳仙尊开创之后,一直秘而不宣,闷声发大财。

“巨阳仙尊即运气大盗。巨阳一死,运盗才消。”墨瑶在文中有感而发,这位奇女子极有个性。对仙尊之流也敢如此评头论足。

她还在文中推测,八十八角真阳楼的秘藏阁里,极有可能有着巨阳仙尊的运道传承!

“我如果能继承这份传承,是否就能复制出巨阳仙尊的成功之路呢?”方源为之怦然心动。

他沉思了片刻。将手中的招灾蛊。再次放入到朱红巨碗当中。

招灾蛊虽然已经彻底成型,但方源还炼化不了。

心中的牵绊,虽然浓重到了极点,但是距离真正成为招灾蛊的主人,还有一段质的差距。

方源现在还是凡人,没有仙元可以炼化它。

这种情况,和他当初炼出神游蛊不同。

在三王福地中,他在地灵的辅助下炼出定仙游蛊。他是主导者。因此定仙游一出世,就是他的蛊虫。不过还使用不了。要不是地灵的帮助,方源还不能传送到狐仙福地里去。

而炼制招灾蛊,整个过程是借助八十八角真阳楼的伟力回流,方源充其量不过是个协助者,身边又没有地灵的帮助,平心而论能够达到如此程度,已经非常不错了。

“我只有成为蛊仙,才能真正炼化它。在此之前,只能将招灾蛊留在这里了。”

没有成仙,就没有仙窍,无法装载招灾蛊。公然拿出去的话,仙蛊气息弥漫,必会引来无数的觊觎者。

方源并不担心黑楼兰、太白云生之流。他现在狼群众多,本身力道修为也足够。

他主要还是担心八十八角真阳楼中的巨阳仙尊的意志。

之前仙蛊雏形,气息微弱,如今仙蛊成形了,拿出去的话,会不会触动巨阳仙尊意志,令其苏醒呢?

方源不想冒着这个险,更何况他的主要计划,还没有完成,需要进一步的潜伏和等待。

“谁会想到,这处无名的山谷中,藏有七转仙蛊呢?不过在此之前,我还是先将这座近水楼台炼化一部分再走不迟。”方源心中思量。

近水楼台,乃是著名的仙蛊屋。方源身为凡人,居然打着炼化它的企图,这看似十分异想天开,但其实却大有可行之处。

强如八十八角真阳楼,都还有漏洞缝隙,供方源钻营。近水楼台如今乃是无主之物,自然也有可乘之机。

这说起来,还得谈到蛊屋的本质。

什么是蛊屋?

不说方源前世了,单说重生之后,他就遇到了不少的蛊屋。

其中,有最常见的蛊屋三星洞。回收时,化为种子。种下去后,形成参天大树,中间内空,分有三层。

有蜥屋蛊,外形如蜥蜴,色彩各异。眼眶成明窗,嘴口有门户,可以自行走动。

还有菇林蛊屋,用大量的菇房蛊栽种下去,众多的蘑菇房屋形成一片庭院。

以上是凡蛊,还有仙蛊屋八十八角真阳楼、近水楼台。

蛊屋发展到今天,已经繁复多杂,数不胜数。但若推根溯源的话,公认的开创者乃是绿龟七人众。

这七人乃是上古时代的魔道蛊师,乃是一母所生的七胞胎,从出生到身亡,一直都是情投意合,同进同退。

他们擅长防守,各个都是五转巅峰的蛊师,更有一招通力合作的防御杀招,名为“龟房”。他们依靠此招,抵御住蛊仙的三次攻击。创下名传青史的“三招之约”的佳话。

而这个龟房,便是蛊师历史上的第一座蛊屋!

因此,从本质上而言,蛊屋是数种、十数种蛊虫相互集合的固化定型的杀招。

菇林蛊屋,便是其中的典型。它是由大量的菇房蛊集齐而成的庭院。

八十八角真阳楼是巅峰的代表。具有防御、搜集、存储等功效。它由无数的小塔楼组成,排难蛊是主要基石之一。牺牲收集来的野蛊后,就形成伟力。从而塑造主楼。

还有白骨战车。

当年,傲骨魔君沈桀骜,惊才艳艳,晋升六转时,苦于手中没有仙蛊,便想出一个杀招,名为白骨战车。

白骨战车由白骨车轮等许多五转蛊组成。威力强悍,可媲美六转仙蛊。

白骨战车,其实就是蛊屋!

至于三星洞、蜥屋蛊、大蜥屋蛊。都是演变出来的分支,是真正意义上蛊屋的简化。

正常的蛊屋,是由数只蛊虫组合而成。而三星洞等,则简化成了一只蛊。因此威力大减。更加平民化。

而近水楼台,乃是七转仙蛊屋,这就意味着,组成它的蛊虫中,至少有一只七转仙蛊!

“以我现在的实力境界,还不能炼化仙蛊。但是炼化其他部分的凡蛊,还是可以的。”

这便是方源炼化近水楼台的底气所在。

炼化的过程,并没有太多关隘。

近水楼台本是灵缘斋的标志之一。由当年的仙子墨瑶执掌。

墨瑶为了爱郎,背叛了门派。炼成招灾蛊,牺牲了自己。这座仙蛊屋就成了无主之物。

如此一来,近水楼台对于方源而言,就是敞开怀抱、任其施为的小美人。

“好家伙,组合近水楼台的蛊虫,竟然多达三千多只。且每只蛊虫之间都是环环相扣,隐隐呼应。核心则是七转的水乳交融蛊,能令蛊师与水液彻底融合隐藏,水不灭则身不死。除此之外还有两只辅助仙蛊,分别是六转的移动仙蛊浪迹天涯,以及智道仙蛊乐山乐水蛊。”

炼化的过程,就是对近水楼台深入了解的过程。

仙蛊的炼化,远超方源的能力,不用妄想了。他着手的是一转、二转的凡蛊。

花费了三个时辰,方源炼得头晕眼花,将其中五百只一二转的凡蛊,彻底化为己用。

“这样一来,我便有了近水楼台一成的掌控权。就算被人发现,凭借这一成的掌控权,也能阻挡敌人片刻。片刻功夫,足够我警觉,赶过来了。”

又炼了一个时辰,方源将掌控权提升到一成三分。

越到后面,炼化的难度就递增。

“可惜我身怀春秋蝉,等于是定时炸弹。不能用宙道的一蹴而就蛊等等辅助炼化,否则绝不止这点成果。”

方源头昏脑涨,心知已经达到极限,两大空窍的五转巅峰真元也耗费得七七八八。

但正当他想抽回心神时,忽然一道身影出现在他的脑海里面。

“时光匆匆不记年,终于今天见有缘。”

随着这一声幽幽的叹息,一位身姿曼妙,脸罩黑纱,双眼幽光如夜的女子,出现在方源的脑海中。

“这是墨瑶仙子的意志,什么时候潜入到我的脑海里面?!”方源暗吃一惊。

时间过去了这么久,墨瑶并非尊者,居然还能残留着意志,可见她当初时的强大修为。

方源其实早有防备,但墨瑶仙子显然有特殊的手段,居然能令意志悄无声息地潜进方源的脑海中。

这手段非同小可。

墨瑶意志进入脑海,若是心怀不轨而发难,方源绝对要吃不了兜着走。

方源不是智道强者,只有智道蛊师,才对这类的意志有着克制威能。

还有更关键的一点。

人思考时,首先是泛出一个个的念头。这些念头在脑海中相互碰撞,发生变化,才会产生新的念头。这就是思考的结果。

现在,墨瑶意志进入方源的脑海当中。方源每思考一个念头,她就能获知得清清楚楚!

一旦被墨瑶意志,得知春秋蝉的事情,那会发生什么?

“不用害怕,有缘人,我无意加害于你。只是想询问你一个问题。”墨瑶意志幽幽地问道。

方源不用猜,也知道她想问什么,当即将薄青渡劫失败的事实告知。

得知这个噩耗,墨瑶意志一阵摇晃,顿时濒临崩溃!

方源心中为之暗喜,但令他失望的是,墨瑶意志终究还是勉强维持住了身形。

那双如夜般的眸子里,充盈着泪光。墨瑶仙子的神情十分复杂,既有悲伤,也有解脱。

就像她的诗中所言“岁月忽已晚,情仇已绵长”,她和薄青之间的爱恨情仇,必定是一个缠mián纠结的漫长故事。

“有缘人,既然你能炼成招灾蛊,又寻觅到此,开始炼化近水楼台,必然已经知晓了大概。既然他已经死了,那就让一切过去吧。唉……这或许是苍生之幸吧。”

墨瑶意志顿了一顿,接着道:“我已经辜负了灵缘斋,现在唯一的遗愿,就是将这座近水楼台归还回去。作为弥补,我将传授你我毕生的炼道心得。除此之外,还有八十八角真阳楼的秘辛情报!”(未完待续。。。)

\end{this_body}

