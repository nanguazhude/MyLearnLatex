\newsection{巨阳反击}    %第二百零三节:巨阳反击

\begin{this_body}

真阳楼中。<-》

地灵奋起全力,向封印自己的力量,发起了冲击。

嘣嘣嘣!

连续三声轻微爆响,锁在它身上的黑链,直接断去三根。

它身上的青泥,也已经彻底从它的脖颈褪下,甚至两翼上端都无青泥,显露出鲜活璀璨的毛羽。

“小麻雀,你胆敢如此!”巨阳意志勃然大怒。

“巨阳,你死了就死了,居然还囚禁我十余万年。今日我一定要冲破你这个破笼子!”霜玉孔雀发出刺耳的尖啸,同样的狂怒。

被囚禁这么多年,无法动弹,不得自由的憋屈郁愤,都化为它震撼封印的动力。

巨阳意志低吼,太阳般的意志陡然爆发,无数黄金洪流,向四面八方激射。

上万只特意蛊,早就将其层层包围。宛若堤坝一般,抵挡住黄金意志洪流。

巨阳特意,完全被特意蛊克制。整个堤坝岿然不动,但这个情形只持续了几个呼吸,吞食了巨阳特意的特意蛊,达到极限,被一个个撑爆。

几个呼吸之后,特意蛊构成的“堤坝”颤抖、分解、溃败!

巨阳特意虽然被克制,但量太大了,源自巨阳仙尊。哪怕经过十多万年的光阴洗礼,损耗了许多,但也不是上万只特意蛊能够压制的。

克制也是相对的。

水能灭火,但火焰足够猛烈,也能将浇上来的水直接烧成蒸汽。

小太阳般的巨阳意志,雄厚宏大。直接冲刷过去,虽然消耗不小,但却将上万只特意蛊直接撑爆。

没有了特意蛊的围困。巨阳意志像是被捆缚的囚犯,挣脱了束缚,立即全力操纵真阳楼。

但下一刻,足足五万只特意蛊,蜂拥而进,重新集结,形成新的包围圈!

正因为研究深邃。所以中洲蛊仙们从未小看过巨阳意志,他们为此准备充足。特意蛊又并非仙蛊,可以堆积数量。

巨阳意志刚获得自由。就被重新封锁,当即气得大吼一声。

它再度展开冲击,江河般浩荡倾泻,再度将特意蛊包围网冲得七零八落。

但很快。第三波特意蛊围了上来。这次数量更多,足足有十万余只!

中洲蛊仙为此耗费了数千年光阴,几大超级势力联手合作,大力收购,不断炼蛊,耗费天文数字般的炼蛊材料。

付出惊人代价,此刻皆被方源借用引动,发挥出了不俗作用。

看到自己再度陷落重围。巨阳意志却没有大吼大叫。

他是仙尊意志,懂得思考。

当他意识到自己落入算计。对方准备充足之后,他恢复了冷静。

他开始思考,很快就发现包围的破绽之处!

虽然特意蛊的重重包围,将巨阳意志的上下左右前后都布满,没有一个死角。

但这不意味着,这个包围没有破绽。

这些特意蛊,是被蛊仙炼化的。被炼化的蛊虫,无法直接汲取元气,这里也没有元气供它们吸取。

那么,催动这些特意蛊的真元,从哪里来?

这么多的特意蛊,甚至还有可能更多,对真元的需求极为庞大。巨阳意志稍加思索,便明白过来――这些特意蛊汲取的,并非真元,而是仙元!

一颗最低级的六转青提仙元,就能化成几乎无限的真元!

这些仙元,就是这些特意蛊的力量源泉。

而这些仙元被隐藏得很好,显然,还有其他辅助蛊虫和特意蛊相互配合,形成一个庞大隐秘的蛊阵。

“只要我切断它们和仙元的联系,这些特意蛊也就不足为惧了。”巨阳意志很快就想到这一层。

而要切断特意蛊和仙缘的联系,前提就是参透这个蛊阵。

就像是中洲蛊仙,为了克制八十八角真阳楼,将巨阳仙尊留下来的黄杏仙元一齐封印起来。巨阳意志,打的也是同样的主意。

当即,巨阳意志飞速思索,很快就找到十余道线索。

顺着这些线索,他同时深入推演,一幅幅模糊的蛊阵构造图,便逐渐清晰起来。

地灵当然不会坐以待毙。

察觉到巨阳意志被牵制,它鼓动全力,冲击身上的封印。

和稀泥对封印克制得很厉害,它身上的青泥布满裂纹,边缘处更是迅速消融。锁在羽翼上的漆黑锁链,再度连续崩断四根。

地灵霜玉孔雀尖啸一声,浑身剧烈一震,震破两翼上大部分的青泥,旋即它便迫不及待地张开两翼。

它浑身寒气四溢,洁白如雪,昂首激越,高洁华美。但宽大的翅膀还未完全张开,就被漆黑锁链紧紧拉住。

它动作猛地一滞,剧痛令其更加愤怒。

剩下的锁链,虽然只剩下三根,但却仍旧是束缚它自由的坚固枷锁。

“扯碎它!”霜玉孔雀猛地扭转优雅的脖颈,刀锋般的尖锐眼眉,闪烁着慑人的寒光。

“糟糕!”正在剧烈思考的巨阳意志,察觉到这个状况,不得不停下思索。

地灵一旦超脱他的掌控,那么八十八角真阳楼随时会被地灵扔出王庭福地。

没有王庭福地的保护,真阳楼将引来无数蛊仙的哄抢争夺。

到那时,众蚁噬象,再多的黄杏仙元,也禁不住无数蛊仙前赴后继的消耗。

但巨阳意志想要阻止地灵,此刻却力有未逮。

刚刚他一边思索,一边试探性的几次冲击,虽然稍有成效,恢复了一些手段,但仍旧有限得很,难以直接镇压。

此时真阳楼外……

吼!

峥虎张口咆哮,猛地一跃。

一时间。大地震动,小山般的峥虎,居然直接窜到半空中。硕大的狰狞虎爪。猛地拍向虫子般微小的太白云生。

太白云生轻喝一声,闪身爆退!

虎爪没有拍中他,但拍击时引起的庞大风力,仍旧让太白云生不大好受。

与此同时,天空中一片片柳风,飘零而下。

这些柳风,大如小舟。落下时姿态优美,宛若落叶。但动作似缓实快,杀伤力惊人。

太白云生浑身浴血。极为关注柳风动向。之前他不小心被贴上三片柳风,手中的防御蛊已然损失殆尽。

此刻,见柳风落下,他连忙左闪右躲。

地上狰虎扑杀。天上柳风接连飘零。太白云生只坚持了片刻,就感到难以为继,狼狈不堪,面临丧命之危。

“唉,太白云生凶多吉少了!”

“他的积累太雄厚,但对升仙的准备,并不充足。”

“太白云生大人,加油啊!”

就连方源。身在局中,此刻看到这样情景。也不禁紧张起来:“难道这一生,太白云生要升仙失败?”

他心中忽的一动,不顾反噬之力,强行催起察运蛊。仍见太白云生身上,气运如火云燃烧,如滚水鼎沸,旺的不可思议。

但下一刻,这片火烧云般的气运,竟陡然间消失了一小半!

“怎么回事?”方源还是第一次看到,气运的这般变化。

正这时间,真阳楼中陡然暴射出一道光柱。

光柱笔直,粗如古树,直照苍穹。

柳风被这光柱一射,顿时分崩瓦解,还原成清辉天气,醇和温良。

光柱连续扫射,天空中的柳风分解大半。光柱再射峥虎,峥虎哀嚎连连,不断闪躲,但光柱如影随形,根本闪躲不开。

地灾峥虎被越照越小,直至化为乌有!

“这是怎么回事?”众人震惊。

“八十八角真阳楼出手了!先祖在上,他保下了太白云生大人!!”有许多人一齐欢呼。

但也有一部分人嗤之以鼻:“放屁,太白云生又并非巨阳血脉,先祖保护他干什么?”

“这个!难道便是传说中的排难蛊不成?”耶律桑震撼不已。

黑楼兰更是怦然心动:“运道排难蛊,可以排遣天劫地灾,使其威力骤降,甚至直接泯灭于无形当中!如果我能得到这只仙蛊……”

排难仙蛊,是构成八十八角真阳楼的基石之一。也是巨阳意志目前可以调动的,有限的手段之一。

巨阳意志的动作,还不止如此。

几次冲击之下,令他能够强行抽取一丝黄杏仙元。催动了排难蛊之后,他又调动真传秘境中的人气仙蛊。

方源便看到,太白云生的火云运气,又陡然消去一小半。

人气仙蛊被催动起来,同样发出一道白光。

这次却是直接照在太白云生的身上。

人气仙蛊,可以为蛊师增长人气,提升蛊仙未来潜力。太白云生得此相助,身上的人气瞬间暴涨十倍!

人气一涨,自然就需要更多的天气、地气。

而天气、地气却是抽取的王庭福地。

太白云生发出大笑,真阳楼的举动令其始料未及,他打紧吸纳,大量融合三气于一身,不仅转危为安,而且潜力大涨,前景陡然一片大好!

地灵霜玉孔雀愤怒尖啸。

天地、地气的大量抽取,令其力量衰弱,一阵剧烈的虚弱感袭遍它的全身。

“小麻雀,你还嫩得很呢。”巨阳意志冷哼一声,它动用了手头上能够抽取出来的全部仙元,用在最关键的地方,取得了最大的成效,极大地缓解了危险局面。

到底是来源于仙尊的意志,智慧卓绝。

现在局面急转直变,又对巨阳意志有利起来。

地灵的虚弱,令巨阳意志有了充分的时间,来参透围困他的特意蛊阵。一旦冲垮包围,巨阳意志便能重获掌控权,再回头解开仙元封印,就容易太多了。

只要再解开仙元封印,八十八角真阳楼将彻底展现出八转仙蛊屋的无上威力!

“糟糕,还是低估了仙尊意志。留给我的时间,比预期少了一半还多。”依靠手中的琉璃楼主令,方源掌握着真阳楼中的情报。

他心中一沉:“现在只能寄希望于太白云生了。希望他能尽快成仙,炼出江山如故蛊……”(未完待续。。)

------------

\end{this_body}

