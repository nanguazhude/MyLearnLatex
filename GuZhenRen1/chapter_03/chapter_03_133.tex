\newsection{挑将}    %第一百三十三节:挑将

\begin{this_body}



%1
哗啦啦。

%2
一道水蓝激流,爆射而出。吊睛大虎一吼嗓子,爆发出雷霆般的声响。

%3
这是四转的虎吼蛊!

%4
虎吼音波,激荡空气,形成肉眼可见的涟漪。音波撞在激流上,将这股激流击溃成漫天的大雨。

%5
成虎自变身成吊睛大虎,攻势绝伦,立即就有了纵横捭阖之气象。

%6
虎吼蛊,爆发音波。虎爪蛊,犀利无端。虎皮蛊,防御卓绝!

%7
虎牙蛊,尖锐如枪,洞穿逞威。虎尾蛊,硬如钢鞭,甩摆如意!

%8
变化道的蛊师,为了形成杀招,会收集相应的蛊虫。一旦蛊虫收集全了,能够变化形态,战力就有质变性的巨大提升。

%9
当然,变化的形态越是强大,相应的蛊虫组合,就价值越高,越难收集。

%10
就算是蛊师能够变化,但也需要大量的训练。人生来是两腿直立行走,一头双臂。若是变化另一个形态,不习惯是最自然的反应。

%11
和飞行一样,只有大量的训练,加上天赋,才能将变化后的形态运用成熟。

%12
蛊师有养蛊、用蛊、炼蛊三大方面,这就是用蛊方面的深沉内涵。

%13
双方大军,都将目光集中在阵前。

%14
浩激流和成虎的激斗,已经进行到关键的时刻。

%15
成虎动用杀招,变化成吊睛大虎,牢牢地占据上风。水魔浩激流只能以闪避为主,一改之前的狂放攻势,显得狼狈不堪。

%16
然而随着时间的推移,战况仍旧僵持着。成虎虽然具备巨大优势,但却始终无法将这优势。转化为胜势。

%17
绝大多数的马家将士,还在大声叫好的时候,马家王帐中的诸位,却都蹙起眉头。

%18
“不妙,这水魔狡猾!成虎有危险了。”马英杰开口道。

%19
在座的不少强者。都赞同地微微点头。

%20
杀招虽然强悍,但实际上,是多种蛊虫同时催用。这样一来,消耗真元就加剧数倍。因此对蛊师来讲,是一把双刃剑。

%21
成虎催动了杀招,但水魔浩激流战斗经验十分丰富。一改硬打猛攻的风格。成虎短时间内拾掇不下水魔,等到他真元消耗见底,他就不得不变化人形。到那时,就是水魔浩激流大反攻的时候。

%22
马尚峰面容平静,但心中却是一沉。

%23
若换做以往,成虎的失败和他没有任何关系。但现在。成虎的胜败,已经不是他个人的事情,而是关乎到整个大军的士气。

%24
马家已经连续两场失利,马尚峰深知,投靠马家的各大部族,已经心生动摇。

%25
马家大军,是以马家为主。其他各个部族为辅的联盟。一旦人心晃动,那么情势就危险了。

%26
马尚峰自然不愿看到成虎的失败。

%27
但眼见战况仍旧僵持,成虎失败的可能越来越大,马尚峰也只得在心中暗叹一声,叫道:“费生成。”

%28
费生成立即出列,右掌抚心,施礼道:“属下在。”

%29
“第二阵,就由你来罢。”马尚峰道。为了消弭待会成虎败北的影响,他将希望寄托在费生成的身上。

%30
费生成亦是一员猛将。

%31
他之前在费家受到排挤,很不得志。马家便离间他。引他为内应,趁着费家内乱政变的削弱时刻,发动突袭,将费家吞并。

%32
费生成投靠了马家之后,也算是遇到了明主。屡立战功,受到着重的栽培。

%33
当即,他下到阵前,大声叫骂。

%34
“来者是费生成,王庭之争以来,已经斩杀过八位四转强者。上一场大战,他用麻木蛊,以一人之力,独战三位同级强者,表现惊艳。”黑家王帐中,孙湿寒开口道。

%35
一旁的耶律桑,则冷着脸。

%36
狈君子孙湿寒说的“大战”,正是马家和他耶律盟军之战。结果耶律桑败北,一路被马家追杀,几乎被杀成孤家寡人。原本归附耶律桑的祁连等,各大部族,最后都归附了马家。

%37
麻木蛊,是四转珍稀蛊,价值可媲美五转。一旦蛊师中招,全身麻痹,几乎动弹不得。虽然持续的时间很短,但在激烈的战斗中,却是十分要命的手段。

%38
黑楼兰嗯了一声,目光扫视左右,问道:“谁人可以出战?”

%39
话音刚落,就有一人朗笑一声,越众而出,道:“费生成不过如此,在下愿意出战,为您的霸业扫清一切障碍!”

%40
黑楼兰定睛一看,不是别人,正是单刀将潘平。

%41
潘平在之前,被刘家三兄弟的杀招打爆,战后被太白云生用人如故蛊救活。不只他,高扬、朱宰也因此得救。

%42
“好,便由你去吧。”黑楼兰点头应允。

%43
若换做大战之初,他是不看好潘平的。但经历十多场大战之后,潘平今非昔比,已然迅速成长为媲美裴燕飞如此级数的强者。

%44
“费家小儿,你不过是个背叛家族,求取荣耀的无耻之徒。你活着,就是一个耻辱,快来受死吧!”潘平上了场后,大骂一声,战意沸腾。

%45
费生成大怒,他最讨厌别人这么说他:“不过是个魔道的野种,之前让你嚣张,是因为没有碰到我!”

%46
当即,两人火并到一起。

%47
一时间,场面火爆,不分上下。可谓棋逢对手将遇良才。

%48
其实,这两人境遇十分相似。王庭争霸之初,他们两人都是郁郁不得志的人。潘平是魔道蛊师,颠沛流离。费生成受到家族打压,壮志难伸。

%49
但是因为这届王庭之争,两人声名鹊起,从战争中发家,实力都有了巨大的进步。

%50
之前的潘平,唯一的好蛊,就是单刀蛊。但现在他通过战功换取,身上的蛊虫俱都豪华精致,战力突飞猛进。已非先前,他手中只有单刀蛊独撑大局。

%51
而费生成亦大致相同。

%52
之前。在家族中他受到排挤,虽然一身蛊虫基本齐全,但缺乏强效手段。他也是从战场中获利,积累战功,兑换到媲美五转蛊的麻木蛊。和本身蛊虫搭配起来,战力顿时暴涨。

%53
两人的身影,纠缠不休,但却各自忌惮。

%54
潘平忌惮费生成的麻木蛊,而费生成则一直在凝神防备潘平的单刀蛊。

%55
说起单刀蛊,也算是潘平的运气。因为此蛊寄托在弯刀之上。而非蛊师的空窍或者身躯。

%56
潘平被炸成碎片之后,单刀蛊却侥幸存活下来。

%57
后来太白云生救活潘平,后者复生,原先一身的蛊虫都毁灭了,只剩下单刀蛊。幸好潘平身怀大量的战功,之前一直没有动用。

%58
至于高扬、朱宰就没有那么幸运了。

%59
他们被干掉后。蛊虫都损失殆尽。最令人遗憾的是高扬的五转波云诡谲蛊,也因此灭亡。

%60
五转人如故蛊,只能针对人体,不能救活蛊虫。

%61
不过两人却心态平和——能复活,便是大幸!

%62
后来,两人通过赊战功,基本将蛊虫补齐。数场大战之后。不仅将欠下的战功还了干净,还有结余。

%63
两对四转强者的激斗,牢牢吸引着众人的眼球。

%64
马家见费生成和潘平打得难分难解,又前后派出六位猛将。

%65
黑楼兰一一接下,遣去裴燕飞、高扬、朱宰等人。

%66
当第六对强者刚刚交锋时,成虎终于败下阵来。水魔浩激流无力追击,只能看着他安然撤退。

%67
黑家士气一振,但很快,第三组对决中,马家一方获胜。又将局面扳回来。

%68
双方又陆续派上强者,两军阵前,形成三十多个战圈。

%69
也就是说,有近七十位四转蛊师现身对战!

%70
这是相当浩大的场面。偌大的北原,人口数十亿。凡人占据大部分,四转蛊师只有数百规模,五转蛊师则不足半百。

%71
正是因为王庭之争,将这些人集结在一起,相互碰撞,相互竞争。在生死激斗中,产生更强大的蛊师,弱小者则被无情地淘汰。

%72
王庭之争,进行到最后的大决战。不管是黑家,还是马家,都是巨无霸似的庞然大物。

%73
不计算蛊仙的话,任何一方大军,其规模都大大超越了超级部族的势力。

%74
两军上下,无不看的心驰神摇,胸怀激荡。

%75
唯有方源比较冷静,前世他看过更大的场面,那是五域混战的跌宕大乱世。

%76
“盟主大人,属下请战!”一个年轻的四转蛊师,越众而出,忍耐不住心中的战意。

%77
此人不是旁人,正是葛光。

%78
葛光是葛家族长,原先只是三转蛊师,但是经过战争的洗礼,他存活下来,实力大增,已经在不久前成功晋升为四转。

%79
黑楼兰微愣,旋即将目光转向方源。

%80
方源乃是葛家、常家两家的太上家老,这两族都受他的节制。

%81
方源察觉到黑楼兰询问的目光,淡淡地下令:“葛光退下,你是一族之长,怎可轻易犯险?”

%82
葛光喏喏而退。

%83
方源又道:“常飚何在?”

%84
“大人,属下在此。”常飚一脸病容,之前大战的伤势还未好。

%85
但方源不管这些,只道:“你去上阵。”

%86
常飚张口欲言,心中充斥着一股愤怒。自从加入了黑家大军之后,每逢大战,他都得被方源屡次下令出战。

%87
就算他是出名的强者,也经不起这样高强度的连续作战啊。

%88
“可恶!常山阴这个家伙,是把我当畜生使唤么?!可恨我如今势弱,不能公然抗命。我暂且忍耐,来日方长。十几年前我能陷害你,十几年后我要将你真正陷于死地!”

%89
常飚心中愤怒地咆哮着,但表面上,只能选择听从方源的命令,拖着病躯,上了战场。

\end{this_body}

