\newsection{苏醒}    %第二百零一节:苏醒

\begin{this_body}

天地晦暗,大风呼啸。

一股无形的天地伟力,托着太白云生悠悠飞升而上。

他狂笑着,白须在风中狂舞。毫无平日温和仁慈之色,脸上尽是疯狂和扭曲。

圣宫上下一片惊惶,蛊师们抱头鼠窜,汇集成数股巨大的人流,纷纷向外逃命。

足足逃去万里之遥,人们这才惴惴不安地停住脚步。

蛊师强者们占据山头,或者直接悬浮于空,动用侦察蛊虫,神情各异地注视着圣宫上空的情景。

其余蛊师们,也都仰头注视着空中的太白云生,怀着崇拜、敬畏、爱戴、担忧、嫉妒等等神色。

关于太白云生晋升蛊仙的消息,已经不胫而走。

升仙!

这个在往日里遥不可及的词语,竟然才此刻,在众人的眼前上演。

“不可思议,太白云生还未调动蛊虫,竟然就引动天象变化!”看着天空中风云激荡,许多蛊师惊奇地叫出声来。

很快,惊叫声浪要掀起一波"gao chao"。

皆因不止是天气激变,就连人们脚下的大地,都开始微微地颤抖起来。

起先只是薄薄的微尘,很快就从地面上升腾起浩荡烟灰。

“这是地气的动荡!”

“根据族中记载,蛊师升仙,首先要碎窍!碎窍之后,便能勾动天气、地气!”

耶律桑、黑楼兰等人,来源于超级部族。比常人知道得要多得多。此刻,他们看在眼中。惊在心头。

升仙第一步,就是碎窍。

用真元拼命冲击空窍,将自己的空窍彻底冲垮,冲得粉碎。如此一来,原本晶壁包裹的封闭空窍,就成了蛊师身体中的漏洞。

“单单这第一步,就需要绝大的勇气。因为空窍彻底碎裂之后,就再也没法挽回。”古家族长古国龙叹息着。

“升仙艰难无比。百不存一!太白云生实在太有勇气了,居然敢踏出这一步。实在令我等佩服。”聂家族长聂亚卿一脸感叹。

黑楼兰神色复杂。

他知道内情,太白云生之所以升仙,恐怕就有他的一番力量促动。

“正是因为我,导致太白云生受到刺激,在心态崩溃的情况下冒险升仙。正好,我虽然熟知典籍。但也极需要亲眼目睹的观摩。太白云生的升仙,对我有极大的参考和帮助。只是……他失败也就罢了。如果他真的成就蛊仙,我又该用什么态度来对待他呢?”

黑楼兰想到这里,顿时皱起眉头,感到头疼。

方源坐于天青狼王背上,身边天青狼群环绕。

“升仙……”他目光深幽。有兴奋,亦有缅怀之色。

再用察运蛊看,便将太白云生身上的气运,如火如荼,原本就红火似乎火烧云一般。现在更加气息鼎盛。仿佛在猛烈燃烧!

“气运熊熊,前世他就成功升仙。今生恐怕问题也不大。”方源暗暗点头。

不同于黑楼兰的犹豫踌躇,面对成仙的太白云生,他亦早有定计。

再看下去,察运蛊不堪重负,竟然受到创伤。

自己的双眼,也有灼痛之感。

方源立即停止催动,并不惊异。

太白云生已经碎窍,引动方圆万里的天地之气。在这个范围动用蛊虫,必然会引发连锁发应,使得天地之气的剧烈激荡,形成反噬之力。

若一意孤行,强行催用,就算是仙蛊也要受创,甚至毁灭。凡蛊就更不用说了。

方源没有考虑过,在这个时候动用蛊虫,暗算钳制太白云生,也是这个原因。

这个时候冒然出手,搞不好还会牵连到他自己身上。

须臾功夫,天空中黑云滚滚,形成一道巨大漩涡。

于此对应的,地面上黄褐色的烟尘翻腾不休,亦形成一个漏洞。

“这太白云生积累雄厚,人气磅礴,竟然吸引了这么多的天气、地气!这个景象,竟然比我族的太上家老还要壮观。”耶律桑见此,心惊不已。

太白云生有仙道传承,一生行走北原,经历极为丰富,如今又几乎走完一生,因此积蓄浑厚无比。

两个巨大的漩涡,覆盖百里范围。

太白云生仰头上望,看着猛烈雄浑的漩涡,仿佛是巨兽张开狰狞大口。

和其相比,太白云生身躯渺小,宛若池塘边的飞虫。

但太白云生仍旧狂笑不已。

“来吧,来吧!”他大叫着,身躯颤抖,有恐惧,有兴奋,也有释然。哪怕失败,对他而言,也是一个解脱。

似乎是听到他的呼唤,酝酿了片刻之后的漩涡,开始缓缓转动。

天上地下,两大漩涡,宛若磨盘一样轰隆转动。

天空中,黑云漩涡电光激闪,雷鸣阵阵。地面上,烟尘漩涡紫烟升腾,爆吼声声。

似乎是碾磨一般,从黑云磨盘中挥洒下一股清辉之气。从烟尘磨盘里,往上冒出一股黄金之气。

正是天气、地气!

天气飘昂清扬,地气浑厚深敛。

而太白云生的身上,则冒出一股白色的人气。

这股人气,十分浓郁,宛若蚕茧一般,把太白云生紧紧包裹成一个庞大圆球。

天气垂下,地气上涌,在半空中相汇,和人气纠缠。

天地人三气,相互汇拢在一起,开始了融合。

“第一步碎窍,第二步纳气……”黑楼兰口中喃喃。

“碎窍一往无回,根本没有回头路,只能成功不能失败。而纳气则考验蛊师心性,蛊师的调控之能。此步极为关键,宛若悬崖上走钢丝。稍微平衡不好。人气稍重,便会自爆!天气稍多。便会消融成空!地气稍浓,便会化石窒息!难,难,难!”耶律桑摇头感慨,即便是旁观者,他都看得心中暗悸。

冲刺蛊仙,一旦开始便一往无前,再没有退路。危险性极高。因此很多五转巅峰蛊师,即便知道成仙之法的,非不大万不得已的情况,一般都不会选择冒险。

外行人看热闹,内行人看门道。

大多数人,纷纷张大双眼,兴奋地瞧热闹。

只有为数不多的。知晓内情的蛊师强者们,才看得大汗淋漓,心惊肉跳。

方源是最深有体会的一个。

他前世冲击蛊仙成功,成为血道蛊仙,对升仙的过程印象极为深刻。

“这第二步艰辛无比,其实不仅考验蛊师的操控平衡之能。更多的是考验蛊师的心性啊。”他暗叹道。

天气、地气袭身,三气汇拢,是凡人和天地交融的过程。

常人从出身起,从未有这么一刻,和天地如此亲密接触过。

天地是孕养万物的基石。天地之气相互感应,大道奥妙就会充盈蛊师心头。

蛊师体悟大道。良机千载难逢,极容易沉迷其中,不能自拔。这就容易造成倏忽,导致平衡三气失败。

更关键的一点是――

天地之气吸纳越多,将来蛊仙成就便越高。蛊师在此关键时期,往往贪欲过盛,过多吸纳天地二气,导致三气失去平衡,失败身陨。

太白云生没有让方源失望,他坚持下来,稳定局面,调控三气恰到好处,徐徐进展。

“这个太白云生的背后,恐怕有高人指点!”

“太白云生居然做到了,第二步也没有难住他。这个人,真是不简单……”

“还要再看。现在做到,并不代表他能坚持到最后。”

黑楼兰、耶律桑各个惊疑。

轰隆!

就在这时,异变陡生。

烟霞震荡,八十八角真阳楼剧烈摇晃。作为基石的圣宫,烟尘四起,一时倒塌了不知多少的华庭美院。

“怎么回事?”

“八十八角真阳楼!”

众人惊呼,目光转移。

“哼,也该是来了。”方源冷笑一声,却是心知肚明。

真阳楼中,地灵勃然大怒!

太白云生升仙,要吸纳天地二气。但这方天地,却非外界北原,而是王庭福地这个小天地。

太白云生消耗这个小天地的天地二气,等若削弱王庭福地,从根本上抽取地灵的力量之源。

地灵能不愤怒吗?

它被囚禁了无数年,如今终于获得脱身自由的希望。

它虽是执念所化,不懂得撒谎,但亦有智商,明白婉转潜伏的计策。

地灵忌惮巨阳意志,原本打算先按捺不发,等和稀泥侵蚀达到一定程度,再一举暴动。

但现在太白云生冲刺蛊仙境界,不仅抽取了地灵的力量,而且肯定会将巨阳意志惊醒!如此一来,地灵的计划就泡汤了。

它很可能因此,再遭巨阳意志的强力镇压。

逼不得已之下,它只好当机立断,趁着巨阳意志还在沉眠的当口,决心暴动反抗,打它一个措手不及!

因此,众人便看到真阳楼震动,殃及池鱼,圣宫多处建筑为此崩塌。

真阳楼中,地灵霜玉孔雀引吭高歌。

它这一击,效果拔群,身上覆盖的青泥直接褪到它的脖颈之下,身上的黑链亦被震落五六根!

“嗯?”

一声梦醒时的轻喃,在真阳楼的最深处响起。

“何人如此大胆,将我从梦中惊醒?”无边的幽暗中,一颗太阳般巨大的意志,缓慢苏醒开来。

它原本只是微光,但很快越来越亮,直至绽放九幽十地,横扫真阳楼中的一切角落。

方源炼化来止步碑的意识,根本抵挡不住,立即就被扫荡一空。他怀中的琉璃楼主令,在瞬间滚烫无比!

\end{this_body}

