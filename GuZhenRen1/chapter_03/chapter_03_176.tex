\newsection{升六角,难找试蛊之人}    %第一百七十六节:升六角,难找试蛊之人

\begin{this_body}



%1
方源生性谨慎,一直都认为世界上其他所有人都靠不住,真正能靠得住的只有自己。因此,他在得到六臂天尸王杀招之时,便着手钻研。

%2
这杀招,本来就是在他的基础上修缮而得。再加上这些天,方源在墨瑶的指点之下,炼道造诣不断深厚,眼界更今非昔比。

%3
因此,他对六臂天尸王的理解,已然了若指掌,清晰透彻。

%4
此刻,他看到这个完善的杀招,不需要试验,便知道此招已经极度完善了。

%5
墨瑶在原来的基础上,删减了不少辅助蛊虫,核心和六大支柱蛊虫都没有变动,又添加了几只辅助用的蛊虫进去。

%6
察觉到其中的巧妙之处,方源连声赞叹。

%7
脑海中,墨瑶意志介绍道:“这还要归功于潘常二人身上的心意蛊,有了他们详实的体会和心得,我才能修缮到如此程度。这个杀招已经完善到极致了,只要不超过一炷香的时间,不管催动次数多么频繁,也不会在体内积累尸气,产生尸斑。”

%8
方源点点头,心知墨瑶所言不虚。

%9
这个杀招的确是完善到了极致,按照这个思路,已经做到了最好。除非今后添加仙蛊,或者更换核心蛊虫。

%10
但添加仙蛊进去,必然将原本的体系打破,要再度重组,面无全非,正所谓牵一发而动全身。

%11
若是更换核心蛊虫,就等若更换了整个思路。思路一改。整个杀招就面目全非,重建之后。与其说是“六臂天尸王”,倒不如说是另一个新的杀招了。

%12
综合来看,六臂天尸王这个杀招,不管是核心蛊,还是支柱蛊虫,亦或者辅助蛊虫,都做到了极致,搭配完美。没有任何可以修改的地方了。

%13
尽管仍旧有后遗症——超过一炷香的时间,尸气就会对蛊师身躯造成损害。超过时间太长,甚至会令蛊师彻底变成僵尸。

%14
但这是蛊师使用不当产生的,完全可以规避。因为使用次数再多,也不会积累尸斑。蛊师可以频繁催动,来取得长时间催动杀招的效果。

%15
只是,方源对这样的巨大成果。仍旧有些不满意。

%16
“你有什么完善杀招的其他想法,本宗师可以洗耳恭听。”墨瑶发出幽幽冷笑。

%17
“做到这一步,六臂天尸王已经是极致了,没有可以改进之地。但仍旧有一点,让我并不满意,那就是这个杀招的后遗症。”方源道。

%18
“经过完善。现在的六臂天尸王,威力又翻上一番,时间长达一炷香。足够你应付凡间的一切挑战。你还有什么不满意的?你要知道,任何杀招都是蛊虫的组合运用,既是蛊虫的运用。就一定会有弊端。杀招的后遗症是不可避免的,只是有的严重。有的微弱。这样的力道杀招,威力之强,在我记忆中也属前五之列。小子,你不要人心不足蛇吞象了。”墨瑶教训道。

%19
方源冷哼一声。

%20
他的假象敌人,正是蛊仙,超脱凡俗的存在!

%21
有了六臂天尸王,方源哪怕舍弃奴道不用,都能纵横凡尘了。但是对付蛊仙,还远远不够。

%22
以凡战仙,这个目标太过于狂妄,过于耸人听闻了。方源不屑,也不方便告知墨瑶。

%23
当即,他接着道:“凡事预则立,不预则废。万一有一天,情况特殊,我过多使用了杀招,变成了僵尸怎么办?我也清楚,六臂天尸王这个杀招已经无法再改进了。接下来,我要解决掉这个后遗症。”

%24
墨瑶这才明白了方源的意思,她陷入沉默。

%25
方源年少得志,墨瑶没有料到他居然还有这种谨慎的一面。说实在话,这让墨瑶对方源有些刮目相看起来——

%26
“这个小子,有天赋,有资源,有机缘,更性格坚忍,能屈能伸。一方面勇猛精进,另一方面却又能兼顾稳妥慎重,顾虑周全。这个小子,迟早有一天,会名动天下的。”

%27
沉默很短暂,很快,墨瑶沉吟道:“要解决这个后遗症,有些麻烦。六臂天尸王,不是寻常的僵尸。尸气极度浓郁,蛊师一旦转变,甚至都不能利用阴阳转身蛊!”

%28
“你堂堂的炼道宗师,也没有办法?”方源不信。

%29
墨瑶不受方源的激将,语调仍旧很平缓:“我需要一位五转蛊师,彻底转变成六臂天尸王,然后在他(她)的身上,进行试验。我需要这个试验者的全面配合,时刻掌握他(她)的状态:身体方面,心里情绪,体会感悟……这些都是重要的参考信息。”

%30
墨瑶不愿深入思考,过分损耗自己。她故技重施,打算用活人试验,代替她深刻思索,艰难推演的过程。

%31
方源皱起眉头:“五转蛊师?还要他(她)全面配合?”

%32
“不错,最好是他(她)心甘情愿。因为动用奴隶蛊等等,控制他(她)时,会干扰情绪的正确抒发,以及摧毁该有的变身成僵尸的体会和感受。”

%33
这就难了!

%34
方源若要试图解决六臂天尸王后遗症的这个难题,就要寻找到一位五转蛊师,还需要他(她)心甘情愿,对方源言听计从,冒着身死道消的危险,全面配合整个试验。

%35
方源到哪里找这样的一个人去?!

%36
三天之后。

%37
中枢室。

%38
从墨液的漩涡当中,楼主令冉冉上升,缓缓飞到方源的手中。

%39
再一次催动炼道杀招墨化,五角楼主令已升成六角。

%40
方源喜忧参半。

%41
喜的是,自己现在拥有了六角楼主令,只待黑楼兰手中的楼主令升到四角,他便能夺过来,合炼成十角楼主令。

%42
有了十角楼主令,就能继承一道巨阳仙尊的真传!

%43
仙尊真传,对于任何人的诱惑都是相当大的。

%44
忧的是,为了这第六次墨化,方源真正掏空了老底。为了筹集资金,他再度贩卖之前的仙蛊残方,导致这些残方再无价值可言。

%45
同时,他还大量抛售石人、狐群、狼群、蛊虫,甚至还有珍贵的毛民,以及少数的气泡鱼。

%46
诸如之前的四转诗情蛊、金龙蛊、金风送爽蛊,五转的数只泉蛋蛊、金刚怒目蛊、点金蛊、松骨蛊、乌七蛊、蛛丝马迹蛊也都卖了出去。

%47
“现在我的手中,只剩下空窍中的奴力两道蛊虫。狐仙福地中,留着血海老祖真传之一的四转血颅蛊。三转骨肉团圆蛊,以及已经用去阴蛊,只剩下阳蛊的阴阳转身蛊。还有一些星门蛊、洞地蛊、通天蛊、神念蛊、葬魂蟾等辅助蛊虫。”

%48
这些蛊,有的干系重大,不可轻易流出。有的则必不可少,需要用到它。

%49
“狐仙福地几乎已经被我榨干了,需要休养生息一段时间。不过我手中有六角楼主令,可以控制八十八角真阳楼中的任意六层。这六层中的每一道关卡奖励,我都能随意抽走。这笔财富极其丰厚,若是尽取,我的财富将在原来的基础上,再暴涨六十多倍!”

%50
方源心里自有盘算。

%51
不过现在,这些财富是不能动的。

%52
黑楼兰纠集大批好手,如今已经将第五层推进到了最后一道关卡。

%53
不久的将来,他就能拥有一角楼主令。

%54
此令在手,他便能查看八十八角真阳楼中,任何一层中关卡的奖励。

%55
方源若是抽走其中的奖励,岂不是就露馅了么?

%56
“等到最后时期,王庭福地即将关闭,遣送众人出去时,再抽取奖励。到那时,才能神不知鬼不觉。”

%57
……

%58
轰!

%59
熊掌重重一拍,立即地动山摇,石块向炮弹般四处飞溅。

%60
烟尘散去,五六个马车般巨大的熊掌缓缓抬起,地面留下一个硕大的深坑。

%61
不幸被熊掌拍中的四转蛊师,成了一滩鲜红肉泥,混合着骨渣和脑浆。

%62
“速度太快了!根本让人反应不及。”

%63
“最后一道关卡,果然难比登天呐……”

%64
“这还只是一头飞熊的虚像,只有真正本体的一半威能。若换做真的荒兽飞熊,我们连逃都逃不了!”

%65
战场上的一众蛊师,各个皆有负伤,狼狈不堪,心有余悸。

%66
镇守五层第一百道关卡的,是一头荒兽飞熊的虚像。

%67
它体型庞大,宛若小山。一身白毛,洁若霜雪。身上野蛊齐全,各个精良。它攻势凶猛,偏偏动作轻灵,迅如电光,和它略显臃肿的体型很不匹配。这才交战不到一炷香的时间,闯关的蛊师们便已经损失惨重。

%68
“族长大人,我们还是撤吧。这一次我们是以试探为主,如今飞熊身上的蛊虫已经差不多探查清楚了。咱们的目的其实已经达到了。”孙湿寒站在黑楼兰的身边,劝说道。

%69
黑楼兰皱起眉头,扫视战场一圈。

%70
他是知兵善战之人,立即明白己方士气低迷。

%71
“我在八十八角真阳楼中闯关以来,遇到的最强对手就是这头飞熊虚像了。要消灭它,只靠我们是远远不够的,还得召集更多的帮手来。”

%72
黑楼兰心中思量稍定,冷哼一声,吐出一个字:“撤。”

%73
孙湿寒心中顿时松了一口气,不止是他,其余蛊师早有退意,只是碍于黑暴君的凶名,不敢提出来罢了。

%74
(ps:看了一下电影明日边缘,设定有点类似春秋蝉,推荐大家看看,可以帮助理解本书。)缠绵

\end{this_body}

