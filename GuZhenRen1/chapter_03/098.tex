\newsection{东方余亮}    %第九十八节:东方余亮

\begin{this_body}

“赵家连夜拔营而去?”王帐内,黑楼兰看了看手中的情报一眼,便随手将其抛在案几上。.

在他看来,赵家虽然是个大型家族,但却没有一支厉害的精兵,就连像样的蛊师强者都没有一位。虽然赵家族长乃是五转初阶,但是三年前,就被东坡空以四转巅峰的修为,挑战成功。因此威望并不高,执掌赵家这么多年,也没有太大的建树。

若是赵家投靠了东方部族,他兴许还会多关注几眼,毕竟五转蛊师哪怕再名不副实,也是不可小觑的。

但现在赵家抽身而退,连夜逃跑的狼狈,让黑楼兰心中尽是蔑视之意。

北原中人,钦佩勇武之士,最看不起的就是这种未战先怯,逃之夭夭的懦夫行径。

“恭喜盟主,贺喜盟主,我们还未真正动手,就吓跑了对方一个大型部族。”

“东方余亮看来要气炸了,他力邀半天的赵家,居然直接跑了,啊哈哈。”

“依我看,赵家虽然是个大型家族,但也不过如此,竟然如此胆小,哼……”

王帐中的诸位蛊师,纷纷开口,对赵家的态度也都并不在意。

一旁端坐着的方源,扫视了案几上的情报文书一眼。

赵怜云。

这个名字他一直记在心上,曰后的奇女子,马鸿运的妻子之一,成就智道蛊仙的人物。现在――还是个小女孩儿。

“看来,著名的虎狼羊之劝,已经上演了么……”

方源在心中冷笑一声。

前世五百年,赵怜云成为智道蛊仙,就有人为其做传。

这种文化传统,最早要追溯于《人祖传》。这个蛊道的第一经典,很多蛊师花费一生的精力和时间,都在琢磨。许多接触的蛊师蛊仙,人们为了纪念他们、赞颂他们,就会为其做传。

《赵怜云传》中,就记载着一段内容。

赵怜云在很小的时候,就表现出非同常人的聪颖和智慧。在“黑暴君黑楼兰”竞争王庭之主的大战中,赵家夹在东方部族以及黑家之间。

正当赵家犹豫之际,赵怜云以虎狼羊做比,劝说父亲,终于使得赵家族长下定决心,赶赴万里之遥,投奔马家。最终使得赵家不仅得以保全,而且还得到了马家的极高的重视和热情的接纳。

五百年前世的记忆,繁芜杂乱,但方源对这些东西,却记忆犹新。

皆因后来五域乱战,马鸿运、圣灵儿、赵怜云不仅成了北原蛊仙,而且还是抵挡天庭侵略的中流砥柱,标志姓的人物。

五域中,但凡这样的人物,他们的传记,都会被广为传播和阅诵。

“哼,像马鸿运、赵怜云,这种人物我迟早要扼杀在摇篮里。不过现在却还不忙……”方源按捺住心中的杀机,表面一片平静。

不管是马鸿运、赵怜云这些五域大战的弄潮儿,如今距离成就蛊仙,还有很长一段距离。方源有大把的时间去对付他们。

但马鸿运,方源还要留着,用来针对八十八角真阳楼。至于这个赵怜云,虽有杀她之心,但碍于此时的身份和情境,却是不好出手。

毕竟,方源现在扮演的是常山阴。堂堂常山阴,怎么会对一个年仅几岁的小女孩如此重视,甚至要动杀手呢?

“而且现在的当务之急,还是要对付东家部族!”念及于此,方源收起心神,又重新投注在王帐内。

在嘲笑贬斥了一番赵家之后,众人就将注意力集中在此次大战的对手身上。

东方家族,和黑家一样,同为超级家族,底蕴深厚,是雄踞在北原草府的庞大势力。

东方余亮作为此代东方一家的族长,可谓年轻有为。凭借智道上的修为,将整个家族的事务处理得井井有条不说,而且还有蒸蒸曰上的趋势。

虽然黑家的军力,更加占优势。但对方是擅于谋算的智道蛊师,实力也绝不容小觑!

“要论此战的最大威胁,那肯定是非东方余亮莫属!”

“不错,此子年纪轻轻,却博闻广识,琴棋书画、天文地理无一不通。他十一岁时丧失双亲,不仅要维持生计,还要照顾六岁大的妹妹东方晴雨。他的双亲给他留下了一笔巨大的遗产,但这小子却是人情练达,知道保护不住,竟然直接将这些家产都送给了一个当权家老,自己只留下很少的部分。”

“他在学堂时候,就表现得极为出色。出了学堂,就成为该家老的心腹。后来屡次立功,获得家老的赏识和引荐,竟然得到族中蛊仙老祖的指点,最终成就了如今的地位和实力。”

众人对东方余亮知之甚详,你一言我一语,道出他的跟脚。

方源细心地听着。

这些具体的东西,他前世都没有经历过,现在身临其境,顿时感觉这个东方余亮并不简单,值得重视。

“历史茫茫厚重,大浪淘沙,不知淘去了多少的英雄人物啊。”

就在众人议论纷纷的同时,作为众人议论的焦点人物――东方余亮,也在书房中谋虑着这场至关重要的大战。

咚咚咚。

三声轻微的敲门声。

“进来吧,妹妹。”东方余亮不用抬头,便知道来访的人是谁。

门被推开,进来一位身着淡黄衣裙,眉清目朗,婉约温柔的极美少女。

她肤如凝脂,眼如秋水,轻柔的声音充满了关怀:“哥哥,咱们从中洲移栽过来的玉杏花开了。哥哥,陪妹妹去院中赏花吧。”

东方余亮笑了笑,心知自己枯坐在书房中已经一天一夜,使得妹妹牵挂,用这借口要让自己放松宽怀一些。

“走吧,晴雨。”

兄妹俩走出书房,联袂而行,来到院中。

此时,天空下着霏霏细雨,天空阴云沉沉。

远望,天际和雨幕连成一片,形成墨绿的暗色。再近一点,透过院墙便可看到,东方家的无数旌旗,密密麻麻的如白馒头似的营帐。

人群穿梭在营帐之间,喧哗吵闹,正在为即将到来的大战做着准备。

小院中,却只有东方兄妹二人。

隔着雨帘,墙外嘈杂的声音反而更显得小院的幽静安详。

尤其是待东方余亮看到小院中的那株玉杏花,花瓣娇嫩小巧,得到雨水的滋润,温润光滑,嫩黄的色彩使得雨中的二人感到一股温馨之意。

“哥哥,听说赵家的人走了?”静默良久,东方晴雨小心翼翼地问了一句。

“放心吧,妹妹,这点哥哥早有预料。”东方余亮展颜一笑,轻轻地捏了一下妹妹的手。

东方晴雨微微仰头,便看见她的哥哥在这如纱的雨气中,一身白衣,面冠如玉,双眼深邃,透出一股运筹帷幄的气度,雍容淡定。

东方余亮又接着道:“我之所以力邀赵家,不过是想收集到一切能够收集的力量。赵家离开,无伤大雅。以我手中如今的实力,仍旧有战胜黑家大军的能力。”

东方晴雨心中的担忧消散了大半:“一切都逃不过哥哥的推算。不过这一次的对手,非同小可。不仅有黑楼兰,而且小妹还听说,曾经北原的英雄,狼王常山阴也投靠了他。哥哥,你可要小心啊。”

“呵呵呵,小妹,你还不放心你哥吗?不过……”东方余亮温声宽慰着妹妹,眼眸深处闪过一丝精光,“当初我们冒险,结识了黑楼兰,此人就对你心怀不轨过,被哥哥好好教训了一顿。但现在看来,这人还是不死心呢。这次哥哥要给他一个终身难忘的教训才好。至于常山阴,哥哥已经着手对付他了。这点哥哥早有预料,小妹,你就安心静养好了。你身子自幼就柔弱,不要过分担忧。你若卧病在床,才会令哥哥我分心呐。”

东方晴雨轻轻地点点头,她完全放下心来。

从小到大,都是哥哥在照顾她,关心她,为她着想。

她就像一朵幼嫩的小花,被哥哥这株大树遮蔽着。

这么多年来,她和哥哥相互携手,走过风风雨雨,这一次也一定能够平安渡过的。

“因为从小到大,哥哥一直都是这样淡定从容的样子呢。只是……若是自己没有重病,若是自己有蛊师修行的资质,那该多好啊。”东方晴雨在心中深深地叹了一口气。

兄妹俩就这样静静地,站在一起,看着眼前的玉杏花。

“妹妹,雨露湿重,站久了对身体可不好,你还是先回房休息去吧。”片刻之后,东方余亮道。

“嗯,哥哥你也不要过多艹劳了。”东方晴雨乖巧地答应道。

看着妹妹离开的背影,消失在转角处,东方余亮的脸色终于不再遮掩,眉头微皱,流露出忧色。

此战绝非他刚刚说的那般轻松。

“一个黑楼兰本来就不好对付了,现在又多了一个常山阴。五十万狼群啊,真不愧是奴道大师级的存在,单靠此人就改变战局,令原本只有微弱优势的黑家,一下子遥遥领先了本家。”

“接下来的大战,我方首先要解决的,就是这五十万狼群。否则胜利的希望,就极为渺茫了。”

“我不能输!蛊仙老祖好不容易答应下来,若是我完成了这项秘密任务,就由老祖出手,亲自为妹妹解决病症源头。为了妹妹,我一定要成为王庭之主,进入八十八角真阳楼!”

“在此之前,任何人敢拦在我的路上,都要有必死的觉悟!所以,狼王常山阴,你就先给我死在这战前的雨夜里吧。”

东方余亮仰起头,凝望着天空深沉的阴云,俊美的面庞尽显冷酷之色。

ps:有一番话不吐不快,但字数较多,都放在“作品相关”中的“答双穿,答威胁论”里面了。请大家移步,查看一下。

\end{this_body}

