\newsection{葛家营地}    %第三十四节:葛家营地

\begin{this_body}

%1
方源便随着葛光一行人,向西方继续前行。

%2
他们有驼狼,速度并不慢。

%3
一路上,队伍气氛融融。

%4
一方面,地方源有意接近,有所图谋。另一方面,葛光也是大力结交,心存敬意。

%5
北原的人蛮勇凶悍,另一方面便是豪爽直率。

%6
你若没有实力,北原人会看不起你,难以结交。但是当你的拳头很强很硬,北原人都会敬佩你。当你又对他们的脾气时,他们的热情会让你充分理解到什么叫做“相见恨晚”。

%7
仅仅两天功夫,方源就和葛光就打得火热。

%8
方源有意借助葛家这个跳板,来真正的融入北原。毕竟常山阴消失了二十多年,忽然回来,总得有个让世人接受的过程。

%9
同时,他身上元石匮乏,防御蛊缺乏,也需要交易获得。

%10
方源在常山阴的尸体上,没有搜到任何用来防御的蛊虫。想来应该是在哈突骨一战中,被打爆了。

%11
而葛光也对方源充满了感激,敬畏和好奇。

%12
感激自然是因为方源救了他的性命。

%13
敬畏是因为一路上,方源驾驭狼群,展现出一流的造诣。且他随意的指点,往往能点破葛光修行的关隘,着实是一派前辈的风范,高手的气度。

%14
好奇则是由于方源言语间涉及往事,时不时的感慨今非昔比的腔调,还有沧桑的目光,很显然是个有故事的强者。葛光自然生出探究的欲望,但是却不敢多问。

%15
五天之后,一行人回到葛家部族驻扎的营地。

%16
营地广大,外围突兀地竖立着一排厚实的土墙,土墙高达两丈。上面绿意葱葱,深绿色的藤蔓交织蔓延,大片的叶子下覆盖着一串串紫色的葡萄样的果子。

%17
这当然不是水果,而是木道神迷蛊。遇到兽群攻杀的时候,这些紫葡萄就会炸裂成浆水,浆水落到兽群的身上,就会令其神志不清,身躯摇晃不止,连站都站不稳。更遑论进攻战斗了。

%18
土墙之后,是高高耸立的瞭望塔。塔上一般都布置三名蛊师,一名防御蛊师,两名侦察蛊师轮流守望。

%19
营地大门洞开,许多蛊师都走出来夹道欢迎。

%20
“少族长回来了。是少族长回来了。”

%21
“少族长刚刚出发才几天,这就回来了?”

%22
“听说他们遭遇到了风狼群,险些丧命,幸亏有奴道高手相助!”

%23
“就是那个中年男子吗?这些狼都跟着他走,好厉害!就是不晓得是北原哪个部族的高手。”

%24
方源等人还未接近营地时,就碰到了在营地附近巡游警戒的蛊师。因此早有人提前回去通报。

%25
因此消息就走漏,很多人对方源指指点点。好奇无比。

%26
一些小孩子兴奋地大喊大叫,跟在队伍的后面蹦跳。

%27
方源坐在驼狼的背上,看着身旁的葛光向人群招手。他每一次招手,都能引发人群的欢呼声。可见这个年轻人在葛家十分具有威望。

%28
从一路上的交谈。方源已经彻底地了解葛光。他是葛谣的亲哥哥,是典型的北原人,豪爽重义气,视荣耀重于生命。勇武有加,同时北原中男尊女卑的传统观念也深入骨髓。因此对于妹妹逃婚一事。十分恼怒和反感。

%29
但是这种恼怒和反感,并不能说明兄妹俩的亲情浅薄。

%30
相反,如果他知道方源就是杀死他妹妹的凶手,哪怕身无半点真元,也会拿牙齿和手脚向方源报仇拼命。

%31
方源前世五百年,在北原讨生活,对北原人有着深刻的了解。

%32
一行人沿着大道,朝着营地中央走去。

%33
周围是一个个的帐篷,颇为类似地球上的蒙古包。这些都是凡人居住的房屋。

%34
很多人听到动静,纷纷掀起帘帐,走出来看到方源身边的狼群,都面色一变风流名将。又看到少族长,皆忙用右手捂住心口,向葛光行礼,并大声地问好。

%35
在南疆,凡人遇到蛊师,都有下跪。但在北原,勇武的男人的双膝只有跪天地、祖宗,还有长辈,寻常时候并不轻易下跪,哪怕是遇到族长、家老也是如此。

%36
这些人身上穿的,都是普通的皮袍。有些家境好的,女子带着首饰,男子在衣边镶着金线、紫线。有的家境坏些,穿的就破破烂烂,甚至打着补丁。

%37
不过这总比奴隶要好得多。

%38
一路上,方源看到跪在地上的人,都是奴隶。

%39
这些奴隶大多衣不蔽体,面黄肌瘦。在北原,这些奴隶的地位十分低下,生活也相当凄惨。

%40
在北原人的心中,豢养奴隶就相当于养育牛羊。奴隶买卖在北原最为盛行。

%41
在北原,居住帐篷的都是凡人。帐篷区分布在营地的外围,内围则是蛊师居住的地方。

%42
若是兽群冲击营地,首先遭殃的都是凡人。

%43
方源一行人走过帐篷区后,就来到蛊师居住的地域。

%44
草原上的蛊师们,居住的地方就不是帐篷了,而是蛊屋。

%45
蛊屋就是用蛊做成的房屋。简单点的蛊屋,直接就是一只蛊。复杂点的蛊屋,就是多只蛊相互结合起来,一起搭建的。

%46
在南疆,跋涉在山林间的大型商队,也都有蛊屋。

%47
当年青茅山上,贾家商队就携带着一座蛊屋,用的是一只木道蛊三星洞。

%48
它高达十八米,名副其实的参天巨木。树根粗壮,根根虬枝如龙蛇纠缠,一小部分裸露在地表,其余则深深地根植于地表之下。

%49
树干中开有三层,树干表面,也开了窗户。防御力绝非帐篷之流可比。

%50
用的时候,被后勤蛊师种下去,灌溉真元而顷刻长成。回收时,形成种子。

%51
但在北原,一般的蛊屋,都是三星洞这样的参天巨木。这样高耸的大树。在雷雨天气时,极为容易遭到雷劈。

%52
所以,方源第一眼看到的蛊屋,就是北原最常见的屋蜥蛊。

%53
这是一种二转蛊,外形如蜥蜴,色彩各异,最常见的是墨绿色,天蓝色,乳白色。它们体型庞大。堪比地球上的巴士。蜥蜴的两个眼眶,做成窗户。身子两侧,也开着窗户。

%54
蜥蜴趴在地上,嘴巴张开,就露出门扉。

%55
推开门进去后。是一条长长的过道。过道两侧,都是房间。过道的尽头,是茅房,暂时存储着粪便。

%56
当部族启程的时候,蜥蜴就会爬起来,粗壮的四肢,交替前进。

%57
当茅房中的粪便过多时。这些蜥蜴就会屙屎,抬起尾巴露出肛门,将粪便都排出去。

%58
居住在蛊屋中的家庭,至少有一名蛊师。

%59
这里的生活环境。明显要比帐篷区高出一个档次。

%60
在蛊屋的门口,时常站着大胃马,马缰就缠绕在蜥蜴的巨大牙齿上。少数的人家,还有驼狼。

%61
方源等人穿过这些蜥屋蛊后。就看到蛊屋菇林。

%62
这种蛊屋,就是用大量的菇房蛊栽种下去冤家眷属。形成的。每一座房屋,都是一只巨大的蘑菇,顶着肉质的灰色圆锥房顶,能顺利地滑落雨水,并不吸引雷击,在大风中也十分稳固。

%63
蘑菇粗壮的圆筒根茎,就是白色的墙面,上面也会开着窗户。

%64
几只菇房蛊组合在一起,俨然就是一个别样风情的小院落。数十只菇房蛊相互栽种,圈出一片草地,就形成小园林。

%65
生活在菇林中的,多是家老,或者家境富裕的蛊师。

%66
听到方源等人前行的动静,这些蘑菇房的窗户接连打开,露出一些北原妇孺的脸。一些活泼的孩童,则干脆跑出来,伸手摸摸风狼或者毒须狼的毛,比凡人家的孩童胆子大了很多。

%67
“常山阴恩人,前面就是我们葛家的王帐了。”葛光开口道。

%68
众人来到营地的中央,这里树立着上百棵的菇房蛊。

%69
一位老者,面容和葛光十分相似,带领着一众蛊师,主动迎接过来。

%70
方源猜他便是葛家的族长,为了表示礼貌,便下了驼狼。

%71
老族长快步走到方源的面前,右手抚心,对方源深深鞠了一躬:“尊敬的强者,你救下我的儿子,就是救了我们葛家的未来。快请进来吧,这里已经准备了上好的马奶酒,牛羊也在烧烤当中。你的这些狼群,我们也会有专人将它们喂饱。”

%72
“好。”方源点点头,随着葛家族长进入这片最大的菇林。

%73
在菇林中,最大的一间菇房蛊里,众人依次入座。

%74
喷香的马奶酒,装在皮袋子里,由貌美年轻的少女手持着,站在众人的身后。

%75
大量的美食,端上桌子。

%76
不一会儿,又有人将烤全羊,烤全牛,抬到房间中央。

%77
葛家老族长亲自下座,来到场地中央,用匕首切下牛羊的眼睛,又切下牛羊背上的,胸脯上的肉。然后盛放在金盘当中,双手捧着,亲自送到方源的矮脚桌案上。

%78
“恩人,请。”葛家老族长端起酒杯,站在方源的面前,向他敬酒。

%79
北原人最敬重好汉,也热情好客。在北原,主人向客人敬酒,客人若全部喝下,那就是对主人的尊重。相反,若是不喝就表示看不起,不屑的态度。

%80
当方源将满满一碗的马奶酒都喝下肚子时,屋子里的众人都大声地喝彩,由此气氛开始升温。

%81
葛家老族长敬过之后,便是葛光接着敬酒,方源同样一口气喝下去。紧接着,又是其他家老敬酒。方源无不满饮,显示出的豪气让众人更加开怀。

%82
酒过三巡,屋内的氛围已经被炒得热烈。

%83
“常山阴恩人,你的名字让我耳熟。你是常家部族的人吗?在常家,我也有熟人。我的二女儿就嫁在常家。兴许我们之间,还有亲戚关系呢。”葛家老族长放下酒杯,微微泛红的脸上,双眼炯炯透亮。

%84
“葛家族长,我知道你想问什么。我是常家元风一脉的族人,山字辈,家中独子。我的父亲是常胜纯,我的母亲是常翠。”方源叹了一口气,语气萧索地答道。

%85
葛家老族长的双眼,倏地大睁,震惊地看着方源:“你,你果真就是常山阴勇士!?”

%86
(ps:定时更新的时间设置错了,本来今天晚上20点的,设置成明天早上8点了,万分抱歉。呼,两更虽然比较辛苦,但既然承诺过,这个月就一定会咬牙坚持的。)

\end{this_body}

