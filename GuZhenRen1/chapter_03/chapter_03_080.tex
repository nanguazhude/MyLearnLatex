\newsection{算计地灵}    %第八十节:算计地灵

\begin{this_body}

%1
八日之后的夜晚。

%2
繁星点点,夜风习习。

%3
方源骑在白眼狼的背上,已经整装待发。

%4
葛光则站在他的身侧,汇报着近日来的情况:“这八天来,有大量的中小型部族,启程上路,参加英雄大会去了。如今留在月牙湖畔的,只有几个大型部族了。贝草川、裴燕飞等人,是第一拨离开的。”

%5
方源调整着坐姿:“嗯,之前我们主动出战,挑了三家,令各方部族多出许多忌惮。再加上英雄大会临近,他们启程也是正常。至于那些大型部族,家大业大,需要谨慎抉择。往往得等到英雄大会后半程时,才过去参加。”

%6
“太上家老明察秋毫,确实如此。”葛光立即拍了一个马屁。

%7
方源呵呵一笑:“月牙湖畔水草丰美,我族就暂且留在这里,不要动身,趁着这个时机消化战果才是。”

%8
“是,大人。”葛光松了一口气,他们之前合议,也是这个想法。

%9
葛家如今像是吃撑的胖子,路都走不动。这八天来,他们都在夜以继日地扩张营地,收编俘虏。统计物资,实力是一天天的暴涨。

%10
“那群水狼,还在那个方位吗?”方源问道。

%11
“是的,属下派遣侦察蛊师,多次打探。这群水狼有五千规模,到了晚上,便会进入那处水巢休憩。大人,真的不需要蛊师护卫吗?”

%12
方源冷哼一声,冷傲地答道:“有狼群在,就有我常山阴在。何须他人护卫?”

%13
葛光听出方源语气中似有不悦,连忙躬身让开道路,“晚辈就祝太上家老大人,尽收水狼,一帆风顺。”

%14
“嗯,你主持族中事物,也要小心。没有我坐镇,那些投降的蛊师尤其要多注意。”

%15
“是,大人。”

%16
方源留下风狼王,领着虚弱的龟背万狼王,已经一万八千头野狼,离开营地,开始狩猎。

%17
他首先顺着葛家探查出来的地图,一番把设计,来到水巢附近。

%18
水狼生活在水中,以鱼为食。偶尔,饿极了也会跑到岸边,吞食一些兔子、地鼠之流。

%19
狼群的到来,立即引起了这些水狼的警觉。

%20
为了保卫身后的家园,水狼成群结队地从水巢中涌出来,虎视眈眈地盯着方源。

%21
方源面无表情,端坐在白眼狼背上,手轻轻一挥,顿时无数野狼齐声呼啸,杀向水狼。

%22
水狼奋起反抗,双方搅成一团。

%23
狼嚎蛊!狼烟蛊!

%24
方源在后方几次出手,将局面牢牢地掌控在自己的手中。

%25
本来他的狼群规模就多得多,很快就杀败了水狼。

%26
水狼在地上丢下一千多只尸体,又被方源用驭狼蛊收编了两千多头,剩下一千多头逃窜到了月牙湖的深处。

%27
方源也不追击,而是捣毁了这处巢穴,又缴获了数百头的年幼水狼。

%28
然后,赶往下一处地点。

%29
又捣毁了数个狼巢,到了深夜,方源共收编了六千多头水狼,两千头龟背狼,一千多头夜狼。

%30
月牙湖畔,水狼最多,也生活着一些龟背狼、夜狼、风狼。

%31
但风狼速度快,难以捕捉。见机不妙,就会撤退。方源的目标中,有一群风狼,但战斗了片刻后,让它们跑了。

%32
狼群都很狡猾,方源为了捕捉它们,往往自身也要付出代价。如果代价太大,得不偿失,方源就会主动放弃。

%33
一些大型的水狼群,有着万狼王的存在,方源更不敢轻易开启战端。

%34
不过,他此次出行,狩猎狼群不过是个幌子罢了。现在表面的功夫做足了,他就寻了一处隐蔽地点,将狼群排布开来,然后动用推杯换盏蛊,联系小狐仙。

%35
小狐仙得到消息后,立即召出一群星萤蛊,借助星光和青提仙元,催动起星门蛊。

%36
星门蛊一套两只,借助黑天之力,能跨域沟通。

%37
方源等了一小会儿,便见夜空中的星光,纷纷投下来,集中在手掌中的星门蛊上。

%38
星门蛊,如蓝宝石般徐徐飞起,升到半空中后,星光暴涨,继而化为一道拱门。

%39
这次,方源却不急着先钻入星门,而是将受了重伤,虚弱不堪的龟背万狼王,以及大量的伤残狼群,送进星门当中。

%40
大量的野狼,宛若河水一般,波涛滚滚,流入星门之中,消失不见。

%41
这样一来,方源身边留下的尽是精锐和青壮狼群。无疑就大大地减轻了喂养负担。

%42
而那些伤残的野狼,将在福地中繁衍生息,生育出健康活泼的小狼崽子。经过狐仙福地的时光加速,飞快成长,最终成为方源新的兵源。

%43
将这些野狼调入狐仙福地后,方源也紧跟着回到狐仙福地。

%44
“主人,你关照人家每天都盯着通天蛊,人家很乖的,都照做了哦。那个琅琊老仙,果然又冒头了,还在宝黄天内大量收购炼蛊材料呢。”小狐仙见到方源十分开心,一把抱住他的大腿,用粉嫩的脸颊蹭着,还说出了一个重要的情报。

%45
“哦?是这样,他买了哪些东西?”方源闻言,精神猛振,连忙问道。

%46
小狐仙便从衣兜中,取出一张小纸片儿,递给方源。

%47
方源拿到眼前,细细浏览,这些蛊虫和材料,他印象深刻,皆是炼制第二空窍蛊所用。

%48
这说明了什么?

%49
琅琊老仙,就是琅琊地灵。他要重新开始炼制第二空窍蛊,绝对是渡过了第二次攻潮,想要炼制出第二空窍蛊的心不死。

%50
同时,那只神游蛊肯定也在他的手中。

%51
要不然,他怎么会在第二波进攻之后,就如此急不可待地大肆采购这些材料呢?

%52
“琅琊地灵受到砚石老人的算计,如今应该是守住了琅琊福地。这地灵虽然智力颇高,但一五一十,没有阴谋暗算之能。我还等什么呢?”

%53
方源想到这里,心脏都怦怦直跳起来!

%54
他立即退出了狐仙福地,回到北原月牙湖畔。

%55
然后,他率领狼群,马不停蹄地来到那处石林。借助当初盗天魔尊的布置,他再次来到琅琊福地之中。

%56
琅琊福地有了许多新的变化,十二云阁都受到了攻击,许多位置都是断壁残垣,有火焰灼烧,闪电轰击,冰霜冻结的种种迹象。

%57
这都是大战留下的痕迹。

%58
尤其是楼阁之外,洁白的云泥之上,浸透了大块的血斑。一只鹿一般的荒兽,体型如小山,倒在云泥上,彻底死亡。

%59
即便是失去了生命,它的毛皮也仍旧光滑,闪耀着五颜六色的彩色光辉,给人神圣璀璨之感。

%60
“你怎么来了?”琅琊地灵不耐烦地接见了方源。

%61
“这是怎么回事?”方源并没回答他,而是反问一句,显示出惊疑之色。

%62
“哼,一群胆大包天的小辈,居然敢打我家的主意,都被我杀了!”琅琊地灵面色阴沉,杀气四溢。

%63
方源好奇地看着地灵:“琅琊福地不是很隐秘嘛,他们怎么进得来?难道是你主动开启门扉……”

%64
“滚!我会那么笨吗?”地灵怒吼一声,“是这群该死的家伙算计我,在卖给我的东西上做了手脚。我原本买下神游蛊,是想炼出仙蛊第二空窍。结果炼蛊的过程中,忽然形成通道,闯进来几只小老鼠。”

%65
地灵说是小老鼠,但方源看看十二云阁的样子,便可猜测出当时的狼狈模样。

%66
但琅琊福地到底是当年,长毛老祖经营的家园。长毛老祖乃是号称“古往炼道第一仙”,和两代尊者平等交往,留下的底蕴极其深厚。

%67
第二波攻潮,也难以撼动这样的底蕴。至少方源知道,琅琊福地中有十二头荒兽,现在只是死了一头而已。

%68
不过,砚石老人还活着。第二波的攻势,不是终结,而是开始。接下来,还有好戏看。

%69
方源笑了笑:“我料得果然没错,你就是那个琅琊老仙,买下了宝黄天的神游蛊。看来,你是保住了神游蛊了。”

%70
琅琊地灵自得一笑:“那是当然的!要不然,那头九色灵鹿也不会死。”

%71
忽然,他神色猛地收敛起来,想到了什么,警惕地看向方源:“你小子来这里,是干什么?”

%72
方源向地灵行了一礼,施施然地道:“还能有什么?当然是请你出手,炼制第二空窍蛊了。”

%73
“什么?!”地灵大叫一声,怒视方源,双眼几乎要喷出火来。

%74
长毛老祖当年为盗天魔尊,炼制仙蛊失败,因此许下诺言,为盗天魔尊无偿地炼制九只蛊虫,不论仙凡。

%75
盗天魔尊用了其中六次机会,得了六只仙蛊。剩下三次机会,作为遗藏传承的一部分,留给了今后的有缘之人。

%76
在方源前世,这机缘被马鸿运得了。今生,方源提前到达这里,截了这个仙缘。

%77
之前,他利用其中一次机会,叫琅琊地灵炼制了星门蛊。现在再到这里,是用第二次机会,要炼出第二空窍蛊。

%78
琅琊地灵乃是长毛老祖的执念所化,根本无法拒绝方源的这个要求。

%79
但是,为了保护神游蛊,他也付出了沉重的代价。眼看着第二空窍蛊有希望炼成,结果被方源截胡了。

%80
琅琊地灵吹胡子瞪眼,语气森然地喝问方源道:“你小子不会就是算计我,围攻福地的幕后主谋吧?”

%81
方源摸了摸鼻子,一脸无辜状:“你难道认为,我区区一个凡人,能指挥得动那些蛊仙吗?我是看着你买下神游蛊,又在先前知晓你手中也有第二空窍蛊的蛊方,现在又看你再买第二份材料。因此才来到这里的。”

%82
琅琊地灵恨得咬牙切齿,手指着方源:“你们这些人类,各个奸诈狡猾。老夫击杀了那些蛊仙,没想到今天阴沟里翻船,居然被你这个小子坑了!”

%83
方源长笑一声:“你不是被我坑的,而是当年欠下盗天魔尊的承诺。怎么样,该为我炼制第二空窍蛊了吧?”

%84
琅琊地灵恨恨不已,把方源大卸八块的心都有了,但没有办法,只得为方源炼蛊。

\end{this_body}

