\newsection{江山如故}    %第二百零六节:江山如故

\begin{this_body}

%1
“察运蛊!”

%2
方源心中一动,调动真元灌注过去。

%3
下一刻他,的视野立即大变。

%4
察运蛊,乃是五转运道侦察蛊虫,能令他看到肉眼看不到的运气。

%5
此时再看太白云生的运气,原本旺盛宛若火烧云般的红火之运,此刻又消散了一小半。

%6
“之前巨阳仙尊意志帮助他,就令其损耗了运气。现在又出手助他,使得他的运气只剩下原先的三分之一了。”方源心道。

%7
他再看自家运气,仍旧是黑棺之运,浓郁如墨,比先前还要庞大几分。

%8
方源心中凛然,暗自冷笑,又看福地诸人。

%9
旁观太白云生升仙的众人,运气几乎都有了变化。

%10
绝大多数的蛊师身上,不知不觉间都已笼罩了一层小黑棺气运。

%11
唯有黑楼兰,仍旧是青运绵绵。

%12
扫视一圈,方源已感到双眼生疼,便明智地停止催动察运蛊。

%13
蛊师升仙的过程中,天地二气沸腾不休,在这样的环境下动用蛊虫,必然要遭受反噬之害。

%14
若一意孤行,则必然使得人蛊两伤,甚至灭亡。

%15
视野恢复正常,方源再看太白云生。

%16
便见他游走在雷霆之间,穿梭于云层缝隙,尽展大师级的飞行造诣。

%17
他的身上,已无防御蛊,纯粹靠着移动躲避颠乱雷球,仿若悬崖上走钢丝,惊险至极。

%18
为了躲避地灾羁绊狼烟,而冲入雷云当中,这固然是两害取其轻的明智之举,但也同样是无奈举动。

%19
“若非我有了青提仙元。化作无限真元,否则这样疯狂的速度早已经将空窍耗干了。”太白云生呼呼地喘着粗气,鬓发缭乱,体弱力乏,局面越加岌岌可危。

%20
嗡!

%21
忽然。一道颠乱雷球转变方向,向他砸杀过来。

%22
太白云生想要躲避,却已经迟了,一时间只能瞠目怒视,眼睁睁地看着。

%23
危急时刻,一道白色光柱及时扫射而来。直中这颗颠乱雷球。

%24
雷球在光芒的照射下,迅速消融,化为一蓬精粹的清辉天气。

%25
排难蛊!

%26
八十八角真阳楼的救援,又一次帮助了太白云生。

%27
“好险,我纵然成就蛊仙,但身上毫无防卫。被颠乱雷球劈中,必然受伤,后果不堪设想!”

%28
太白云生咽下一口吐沫,心有余悸。

%29
成为蛊仙之后,他的生命本质得到了升华。若换做凡人,一颗颠乱雷球就能取其性命。蛊仙不怕,纵然没有任何的防御蛊虫。但挨上十几计雷球,还是不成问题的。

%30
但关键是,这颠乱雷球还有特效。

%31
砸中之后,令太白云生念头颠乱,一瞬间思维极度混乱,无法思考。

%32
这个关键时刻,一旦太白云生悬停空中,下一刻必然有无数的颠乱雷球蜂拥而至。

%33
若是接连被这雷球砸中,太白云生连思考的能力都没有,就会彻底被无数雷球轰击、淹没。

%34
“你们快看。八十八角真阳楼又一次救下了太白云生啊。”

%35
“是啊,这种情况已经发生了至少十几次了。”

%36
“有了真阳楼的帮助,看来太白云生渡劫成功是板上钉钉!”

%37
场外,众人看到这一幕,又是惊奇又是羡慕。

%38
“如此大力度的支援。真的能行吗?”黑楼兰眼中精芒烁烁,心中却有不同观点。

%39
在这种环境下,天地二气沸腾,动用任何蛊虫,都会惹来强大的反噬之力。

%40
排难仙蛊,当然也不例外。

%41
蛊师历史上,关于渡劫时动用仙蛊帮助,最终反噬之力将仙蛊摧毁的例子,也有很多。

%42
只是大多数人,都是外行人看热闹,并不知道这等秘辛。

%43
“排难仙蛊,乃是八十八角真阳楼的基石之一,用来给王庭福地排难。若没有了它,王庭福地就会遭受恐怖的天劫地灾。巨阳意志绝不会为了太白云生,而牺牲掉这只珍贵的仙蛊。也就是说,巨阳意志的支援,也是有限的……接下来才是最关键的,一切都看太白云生,能否迅速渡过仙窍灾劫了。”方源面无表情,端坐在天青狼王背上,心也慢慢地揪起来。

%44
毕竟江山如故蛊,乃是他北原此行的首要目标。

%45
太白云生的速度,渐渐缓慢下来。

%46
他是内有忧外存患,外患是地灵陷害,内忧则是凡蛊升仙,在自家仙窍中酝酿灾劫。

%47
尤其是这内忧,牵扯了他越来越多的心神,导致他对外界的应对越加缓慢。

%48
陡然,他浑身一震,彻底停下身形。

%49
下一刻,无数的颠乱雷球向他轰砸过去。

%50
排难光柱绽放炫目光辉,也旋即爆发,笼罩太白云生。

%51
“最关键的时刻到了!”方源、黑楼兰、耶律桑等人,看得分明,俱都目光一闪。

%52
此刻,在太白云生的仙窍中,璀璨的光辉猛地爆发出来,充天彻地。

%53
仙蛊的气息,磅礴喷涌。

%54
耀眼的光辉中,一只仙蛊徐徐上升。

%55
只见它形如瓢虫,拳头大小,浑身碧玉也似。圆滚滚的背壳有天生纹路。一半是江河湖海,一半是山丘峰峦。

%56
“原来这仙蛊,是江如故、山如故两相合并而成。看来效果,便是江如故、山如故的结合和提升。”太白云生猜测。

%57
这是天地交感所炼,并非他亲自炼蛊,因此对江山如故的功效,还并不清晰。

%58
只是关注了几个呼吸的时间,太白云生便将整个注意力,转移在福地最中央。

%59
仙蛊江山如故,并非他的目标计划,只是顺带的惊喜罢了。

%60
此刻福地的中央,人如故蛊结成的厚茧,也裂开丝丝缝隙。五彩烂漫的流光,从缝隙中漫溢而出。

%61
破茧成蝶!

%62
厚茧消散。一只五光十色的蝴蝶,窈窕而飞,所行之处,洒下五色虹光,沿着蝴蝶翩翩飞舞的路径。型成一道弯弯绕绕的彩虹幻影。

%63
“成了,仙蛊人如故!啊哈哈哈……”看到这一幕,太白云生喜极而泣,仰头狂笑。

%64
多少年的积累,多少的担忧期待,多少次命悬一线。都在这一刻,得到了完美的结果!

%65
但内忧仍在。

%66
仙窍中,青色劫云滚滚,酿成一只青鳞巨蛇,蛇头长有烂银独角,蛇身长达数百丈。蛇尾化气,和遮天的劫云连为一体。

%67
而大地则开始震荡,轰隆隆,响如雷霆的声音震耳欲聋。

%68
刚刚成形的大地,出现裂缝,无数岩浆从裂缝中狂涌而出,浓烟滚滚。万里焦枯。

%69
“天劫——银角青鳞蟒,地灾——熔地!”太白云生心中一凛,独自面对天劫地灾,让他感到无以伦比的压力。

%70
他将希望的目光,投放在仙蛊江山如故之上。

%71
“来吧。”他意念一动,一颗青提仙元飞出,融入江山如故蛊中。

%72
瓢虫般的江山如故蛊,立即翻开背翅,化作一道极光,飞射而去。

%73
所到之处。方圆万里,大地回春,江山如旧,回到地灾摧毁前的那一刻。

%74
“好蛊,好蛊!”太白云生心中顿时大定。有了江山如故仙蛊,他就无惧熔地之灾。

%75
呜傲——!

%76
银角青鳞蟒张开大口,发出怪吼。

%77
它猛地窜出,庞大的身躯带起飓风阵阵,大风呼啸中,对准江山如故仙蛊扑去。

%78
太白云生骤然紧张起来,仙蛊各有威能,非是防御蛊虫,都比较脆弱。江山如故蛊同样如此,若是被银角青鳞蟒击毁,局面对太白云生将大为不利。

%79
好在江山如故蛊,是天地交感,在他仙窍中出生,受他操纵,如臂使指。

%80
太白云生积累丰厚,连忙调动江山如故仙蛊,不断逃窜。身后银角青鳞蟒数十次扑杀,打得山川倾塌,大地洞裂,掀起飞沙走石,卷动凛冽狂风。

%81
“糟糕!再这样下去的话……”太白云生心神灌注,无比集中,几个呼吸之后,额头便已涔涔现汗。

%82
他双目紧闭,身躯仍旧悬停在半空中,任由颠乱雷球疯狂轰砸。

%83
幸亏有排难仙蛊的光柱,保住他肉身安全。

%84
太白云生虽然早已经顾不上外界了,但却未忘记外患。

%85
糟糕的局面,促使他做出一个壮士断腕的决断。

%86
“八十八角真阳楼的支援,到底是有限的,说不准什么时候会停止支援。只好如此做了……现在人如故仙蛊,就在我的手上,只要有时间,就有希望!削,给我削!”

%87
太白云生在心中呐喊。

%88
弃车保帅,他毅然舍弃一部分的仙窍福地,开始削除天空。

%89
福地乃是蛊仙底蕴的根基所在,削减福地,就是自残,减少实力和潜力。

%90
但太白云生不得不这样做。

%91
就像当初方源在狐仙福地中渡地灾,为了排除魅蓝电影,舍掉四分之一的福地一样,太白云生也在壮士断腕!

%92
仙窍福地的天空,一层层地被削除出去。

%93
连带着浓厚劫云一起,被直接排除到体外去。

%94
他的体外正是王庭福地,霜玉孔雀欢鸣一声,立即吞并这些福地碎片,增加自身实力。

%95
巨阳意志则气得怒吼一声。

%96
他现在保着太白云生的性命,但后者却做出资敌的行为!

%97
巨阳意志又不得不保他,若是太白云生身死道消,整个仙窍都会被王庭福地吞并。

%98
而最关键的是,排难仙蛊已经满布裂痕,再对太白云生支援下去,恐怕就要被反噬之力摧毁了。

%99
排难仙蛊乃是八十八角真阳楼基石之一,绝不容有失!

%100
但若任由太白云生去死,对巨阳意志而言,也同样不利。

%101
一时间,巨阳意志陷入两难之境。

\end{this_body}

