\newsection{偷鞋(上)}    %第一百一十五节:偷鞋(上)

\begin{this_body}

%1
营帐中,一片安静。

%2
朱红色的案几上,文本堆成高高的一堆。

%3
时不时的,一阵风透过门帘,夹裹着野草的芳香气息,扑面而来,将最上层的文本也翻动几页。

%4
窗外风和日丽,时不时的,远远传来战马或者驼狼的嘶鸣。

%5
这反而更增添营帐中的静谧。

%6
身为马家少族长的马英杰,此刻正盘坐在蒲团上,伏案垂首,帮助父亲,处理着盟军要务。

%7
自从马家施展突袭,将费家灭族,英雄大会上最大的障碍,也就搬掉了。

%8
在随后的天川英雄大会上,马家力压成家,成为盟主。毒誓为盟之后,马家军力立即爆棚到六十万,军容鼎盛。

%9
再之后,马家东征西讨,从天川出发,一路向西南行军。沿途不断寻找兽群,或者顽固的地方势力出击,不断炼兵,整合战力。因为连连战捷,又吸收败亡的部族,因此自身再次壮大,营造出高昂的士气。

%10
如今,马家来到镜湖附近,终于碰到同级别的对手——宋家联军。

%11
目前,马家正在铸建第一道防线。

%12
“报!”这时,营帐外传来侦察蛊师的通报声。

%13
马英杰眼中精光一闪,心知此刻有通报,必定是有紧急或者重大的军情。于是,他放下手中的工作,唤道:“进来吧。”

%14
风尘仆仆的侦察蛊师听到召唤,刚准备踏入营帐,却被守卫拦下:“你懂不懂规矩?换鞋子,不要把帐内的地毯踩脏了。”

%15
侦察蛊师连忙道歉,换了鞋子之后,他进入营帐,见到马英杰,他立即单膝跪下:“属下见过少族长,此次带来玉田方面的重大军情。”

%16
他言简意赅地禀告之后,马英杰又详细询问,半柱香之后,这才令其退下。

%17
“黑家战胜了东方部族,赢得了关键的第一战胜利了。”马英杰心中有些沉重。

%18
王庭之争,已经进行了无数次,到了他这一层次,对这游戏的规则早已经揣摩透彻了。

%19
他心知首战得胜的重要意义,黑家击败东方之后,将会获得大量的赔偿。这些战争赔款,有着不少东方部族的最新蛊方,还有海量的战争物资。只要消化了这些资源,吸收了俘虏,黑家大军的战力将上涨五倍左右!

%20
“历来王庭之争中,首战尤其关键。一旦首战获胜,就获得了基础的资本。首战不胜,几乎都是被淘汰,很少有翻盘的例子。黑家已经完成了首胜,而我们马家还在和宋家僵持着……”

%21
费才小心翼翼地来到营帐门口,尽量不发挥出一点声音。

%22
把守入口的两位蛊师看了他一眼,便转移了目光。

%23
自从费家被灭之后,懵懂的费才被马英杰点中,幸运地成为了他的贴身奴仆,避免了其他族人的凄惨处境。

%24
他每天的工作,便是摆放营帐门口的鞋子。

%25
马英杰生性好洁,每次进入他营帐的来客,都要换鞋子,以免踩脏了他那名贵华丽的地毯。

%26
每个客人穿过之后的鞋子,费才都要负责清洗,然后再摆放上去。

%27
但这一次,却是和以往不同。

%28
费才将被侦察蛊师穿过的鞋子,拿捏在手中,犹豫了一下,终于将另一双鞋子塞进怀里。

%29
没有人注视到了他这个小动作,费才无惊无险地离开,绕过十多座营帐后,来到水池旁。

%30
他蹲在水池边,开始清洗那双被穿过的鞋子,态度十分专注,以至于身后来了一人,他都没有注意到。

%31
“喂,大呆瓜,洗个鞋子干嘛这么认真!”一只小手,猛地拍打在费才的肩膀上。

%32
费才被吓了一跳,回过头,看到一位小女孩,粉雕玉砌,一双眼睛精亮精亮的。正是赵怜云。

%33
赵怜云自从用了“虎狼羊”三说,劝说赵家族长,赵家便长途迁徙,几番波折之后,有惊无险地来到马家营地,并得到马家族长的亲自相迎。

%34
赵家成功地并入马家大军当中,受到马家高层的重视和热情款待。

%35
“是你啊,小云姑娘。”费才看见赵怜云,顿时流露出憨笑。父亲被杀,令他陷入无尽的悲伤当中。机缘巧合之下,他成为马英杰的贴身奴仆,也受到了老奴仆们的排挤,没有一个朋友。

%36
赵怜云捉弄了他几次之后,却被他认做唯一的朋友。因此他见到这个小女孩时,心中十分欢喜。

%37
“小云姑娘,我有东西要送给你。”费才压低声音,将头凑近赵怜云的耳边。

%38
赵怜云一把将他的脑袋推开,不悦地叫嚷起来:“喂,你个死大头,别靠这么近,不知道男女有别嘛。”

%39
费才被她一推,差点落到水里,他却不在意,偷偷摸摸地敞开上衣,将怀中被他带出来的鞋子暴露出来,邀功似的道:“你看这是什么?”

%40
赵怜云鄙夷地看了一眼:“原来是一双臭鞋子啊,一看就是被人穿过的。死大头,你蠢得要死啊,居然送我这样的礼物。我根本穿不上,也不会穿这双臭鞋!”

%41
费才却道:“小云姑娘,你前些天不是说缺元石花么。这双鞋子制作精美,咱们可以偷偷地卖到黑市中去,换些元石花差花差。”

%42
赵怜云扬起眉头,对费才刮目相看起来:“行啊,大呆头,你居然想到滥用职权,倒卖公物?行啊,平日里没看出来啊,你还有这一手。不过这双鞋子能卖多少钱?我每天的零花钱,都是这鞋钱的十倍还多。你有心了,嗯,还是你卖了吧,你的这身衣裳破破烂烂的,也该换了。”

%43
费才摸摸鼻子,摇头道:“不用,我这衣服还能穿呢。其实,也不是我想到的。那些老奴仆都这么做,反正鞋子数量多,经常被蛊师大人们穿出去。少族长大人又好干净,每隔一段时间就叫人把这些鞋子换新的一批。”

%44
赵怜云点点头。

%45
蛇有蛇路,鼠有鼠道。奴仆们身份卑微,但也有卑微者生存之道。

%46
尤其像费才这样的人,虽然是奴隶,失去了人身自由,但却贴近马英杰,平时极可能第一时间获得马家高层动向的情报。

%47
赵家如今参加了马家大军,赵怜云故意接近费才,其实有情报这方面的打算。

%48
就在这时,一阵喧哗声传来。

%49
“费才在哪里?快快滚出来!”

%50
“费才你闯了大祸了,居然敢私拿少族长的鞋子。”

%51
“少族长想要出营帐走走,结果发现没有自己的鞋。费才,你简直是胆大包天,罪无可恕!”

%52
一群老奴,叫嚣着,从入口涌了进来,在人群中搜索费才的踪迹。

%53
费才脸色骤然惨白一片:“糟糕了,我被发现了。小云姑娘,你快走,这事情和你无关,我可不能拖累你。我这就向少族长请罪去。”

%54
“请个屁罪!”赵怜云低吼一声,脸色极其难看,“你这个笨蛋,被人算计了还不知道!快跟我跑啊。”

%55
“啊?”费才懵懂不解,但被赵怜云拉着,一阵小跑,钻入巷弄小道。

%56
“可恶,这边的路口也被人堵了。”赵怜云仗着熟悉地形,带着费才不断辗转,结果发现四个入口,三个小后门都有人堵着。

%57
“小云姑娘,你快走吧,再不走就来不及了!”费才被赵怜云带的头晕晕的,早已经失去了方向感。他语气焦急,不愿意牵连到赵怜云这个唯一的朋友。

%58
赵怜云恨恨地一踱小脚,在心中连连咆哮:“老娘找个眼线,容易嘛!?这年头像费才这样傻的呆瓜,上哪里找去?那些老奴各个油滑如鼠,让他们告诉点消息,就伸手要钱,说的话还真假参半,虚虚实实。哼!他们这是嫉恨费才这个新人,想要除掉他。果然有人地方,就有江湖,就有算计。不成!老娘我可咽不下这口气,敢动老娘的人,找死!”

%59
赵怜云脸色阴晴不定,绞尽脑汁,耳畔传来老奴们的声音。

%60
“这么没有啊,水池那边也搜过了,没有!”

%61
“会不会已经走了?”

%62
“怎么可能,我们的人一直盯着呢,确确实实看到那小子进来的。”

%63
“那边还没搜呢,走。”

%64
听到老奴们的脚步声,赵怜云情急生智,忽然想到了一个办法。

%65
“大呆瓜,真是天不绝你,幸好我刚从市场上回来,买了一匹丝绸。”赵怜云说着,便从怀中掏出一段细腻柔滑的上等丝绸。

%66
这丝绸,她原本是想用来做衣服用的。

%67
“大呆瓜,只要你好好听我的安排,按照我的话说,说不定这次对你反而是件大好事!”赵怜云将丝绸塞到费才的手中。

%68
“啊?”费才一脸懵懂。

%69
赵怜云嘴皮子迅速翻动,告诉费才她的安排。

%70
十几个呼吸之后,费才主动走了出去,被老奴们发现。

%71
他们大喜过望,包围过来,但费才攥紧拳头,发了疯似的,将好几个老奴都打翻在地。

%72
“反了,反了,这狗才居然敢打我们这些老前辈!”

%73
“费才,你闯了大祸,少族长找你,我们来捉你,你居然敢反抗!”

%74
费才大吼一声:“少族长找我,我自己会走,你们这些小人,不要用你们的脏手来碰我!”

\end{this_body}

