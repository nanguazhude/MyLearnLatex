\newsection{战力激增}    %第八十八节:战力激增

\begin{this_body}

片刻之后。.

“哼……”方源寒芒,暗暗记下这些蛊仙的名号。

钓叟、巨石大仙、电煌天母……

“砚石老人,原来是你。”他终于明白针对他的是何人,皆因这些蛊仙中的绝大多数,都是五百年前世记忆中,攻打琅琊福地的人。

“这些蛊仙,大部分来自北原,但也有南疆、东海、西漠、中洲之人。再加上砚石老人,这个隐藏在暗处的蛊仙势力,真是庞大!”

因为这场不同寻常的交锋,倒是让方源得知到不少情报,也让他暗暗心惊。

到底是怎样的一股势力,是何人经营出来的?

在他前世,他居然一点都不知情,一直被蒙在鼓里。

这样的强大势力,潜藏在幕后,已经不是简单的一条毒蛇,而是猛虎栖于丛林。

它在策划什么,目标是什么?

未来的发展,五域的大战,它在其中,究竟起到了什么样的作用?砚石老人在这势力中,又处于什么样的位置?

再想想刚刚的交锋,宛若暗流汹涌,至始至终他们都没有全力出手,而是扮演得敲到好处。

旁人根本察觉不出来,只会以为是通常的竞价行为而已。

甚至就连自己,若不是前世五百年记忆,恐怕也只会觉得是自己运气不好,在购买物资的时候遭遇激烈的竞争。

他们暗中出手,悄然干扰,发现无法阻挡之后,又悄然收手。

整个狙击的行为,宛若投石水面,涟漪荡漾几圈后,水面又恢复了平静。

方源心中清楚,自己这次之所以能摆脱对方的阻击,得自琅琊地灵的蛊方只是一个基础。更关键的是,对方不愿意暴露,不想惹来怀疑,因此没有全力出手。

方源虽然前世是蛊仙,但现在却只是一个凡人。

虽然坐拥一个福地,但绝对无法匹敌两位蛊仙联手,更遑论仙鹤门,以及这样的神秘而且强大的势力。

“我之所以引起砚石老人的关注,恐怕还是因为定仙游蛊。”方源暗暗沉吟。

当初他以凡人之躯,在众目睽睽之下炼制出仙蛊,实在太显眼,太惹人瞩目了。

仙蛊,就算是蛊仙也很少拥有。更何况是定仙游这种六转极品蛊呢?自然会引起蛊仙们的觊觎了。

都怪这个风头出得太大!

但方源也没有办法。当时的情势所逼,他只能这么做。这么做,也是最好的结果,没有之一。

“这么长时间过去了,现在这个消息想必也已经传播开来了。仙鹤门肯定大力调查,不过这反而是一件好事。”

“他们调查得越清楚,就知道我的跟脚,我不过三十几岁,是一个中小家族破灭之后的流浪儿,一个散修,说得更实际些,那就是一个魔道蛊师!”

“这样的人,资质低下,资源缺乏。居然能够获知狐仙福地的景象,竟然能够炼出仙蛊来?这怎么可能?就好像是蚂蚁突然涨大,吞食了大象。肥猪忽然长出了翅膀,飞得比雄鹰还高。”

用作地球上的话讲,那就是不科学!

“这样一来,他们自然而然就会推测,是我背后有高人!这样的高人,至少是一位强大神秘的蛊仙。而我不过只是幕后人推出来的棋子罢了。”

“如此,他们想要对付我,就要考虑到这个幕后之人,或者是幕后组织的因素。如果是幕后组织,那么是一群人,还是一个超级家族,或是一个超级门派?谁都说不清。谁都想更深入的调查,去想方设法地弄清楚。”

“在不弄清楚之前,只要我一直龟缩在狐仙福地当中,不做碰触他们底线的事情,他们也只是以试探为主,不会撕破面皮动真格的。”

方源脑海中的念头,此起彼伏,电光火石一般,将当今的局面分析透彻。

“当然,这种情况只是暂时的。纸终究包不住火的,一旦他们发觉真相,或者失去耐心,那么我的大麻烦就来了。”

一旦仙鹤门进攻狐仙福地,留给方源最好的结果,就是自爆福地,两败俱伤,荡魂山谁都得不到。

失去了福地庇护,方源就成了人人喊打的过街老鼠。

之所以经营常山阴这个身份,也是他未雨绸缪,为将来留下一条后路。

狡兔三窟,方源身为魔道枭雄,自然深得精粹。

收拾泛滥的思绪,方源叹息一声。

他必须尽快修行,最好能在仙鹤门、神秘势力动手之前,再度成就蛊仙。但就算这样,也是巨大灾厄。渡不过去,就是粉身碎骨,身死道消的可悲下场。

“压力重重啊……”

虽然方源在三王福地中获得了最大利益,但经此一役,他就失去了主动。

春秋蝉、仙鹤门、荡魂山、神秘势力、福地地灾……

虽然他得到了狐仙福地,实力也因此暴涨,但他反而更加岌岌可危。

种种激烈的情势,险恶的局面,一步步地压迫他,像是鞭子或者镰刀,在背后驱赶他前行。

若是他稍慢一步,就是不堪设想的结果!

若换做他人,恐怕早就智穷力竭,被崩溃的局势碾压成渣。也只有方源,搜索枯肠,殚精竭虑,从危局中钻出一线生机来。

但就算他这样不断地努力着,局面仍旧没有得到好转。

就像现在,他又遇到了一个新的难题。

仙元石不足了!

他原先大量抛售仙蛊秘方,获得二十八颗仙元石。但用了几番,尤其是这次大肆采购之后,仙石花销太大,已然见底。

现在方源的手中,只有可怜的四块仙元石。但他投入的地方,还有太多太多。

万般无奈,他只好暂时停止了对狼群方面的投资。

但方源的计划是奴道、力道双修,如今有了第二空窍,令这个计划更加可行。

决定一位奴道蛊师实力的,通常有三个方面。

第一个方面,就是奴隶兽群的规模。

第二个方面,是蛊师空窍中,本身的奴道蛊虫。

第三方面,则是奴道蛊师的魂魄。魂魄底蕴深厚,驾驭的兽群就会越多,就能收编更强的兽王。指挥兽群战斗的时间也越长。

“经过我的大力收购,我的狼群规模已经在北原凡间,登上第二流的层次。第一流的,是当世的三大奴道大师――马尊、江暴牙、杨破缨。”

“但我的奴道蛊虫,都源自常山阴,只有四转。还得升上五转,才能应付北原大战。”

“幸好荡魂山正在消亡,但是仍旧可以运用。有了胆识蛊,大大减少了我的开销,但除此之外,我还要大量的狼魂蛊,继续转化狼人魂。”

面对急速的修行需求,方源手中的仙元石显得捉襟见肘。大力收购狼群,只是提升了奴道三大方面中的第一面而已。

接下来,他又在宝黄天,买下几种蛊方,许多炼蛊材料,以及今后修行不可或缺的五转狼魂蛊,统共十八只。

这样一来,他便又花去一颗仙元石,手中只剩下三颗。

方源将目光瞄向舍利蛊。

他的第二空窍,刚刚形成,只是一转初阶,需要重头修行。

但方源哪里来的时间?直接用舍利蛊堆出修为来,才可能在短时间内,对他提供帮助。

虽然有心,但碍于仙元石实在不多,方源只好退而求其次,只买下青铜、赤铁、白银、黄金舍利蛊。

之后,又选择了改造身体的蛊虫。

身体如皮囊,装载着魂魄。魂魄强盛了,身体强度不行,就会成为瓶颈,限制魂魄的增长。

同理,若是身体强度不够,那么使用力道蛊虫就要小心。一旦用力过度,还未打中敌人,就会筋肉撕裂,骨头自断。

当方源结束了这场交易,他手中只剩下两块仙元石。

这两块仙元石,他是留着备用,以应付突发情况。

之后两天,方源没有离开狐仙福地,就在荡魂山上进行修行。

“这是最后一只胆识蛊了。”攀登在一处峭壁之上,方源伸手捏开这只蛊。

啪。

随着一声轻响,一股黄褐色的泥水流淌了出来。

这是一只坏蛊,本身被和稀泥仙蛊的力量侵蚀,不再有增长魂魄的奇效。

“荡魂山生机渐弱,好的胆识蛊已经越来越少了……不过,经过这些天的采集,因为有大量的基数,我的魂魄还是达到了千人魂。”

方源闭上双眼,感受着体内的魂魄。

这千人魂,明显比之前更加凝实,有一种沉甸甸的错觉。它像是被硬塞在方源的身体内,有一种要满溢冲出来的感觉。

千人魂,是奴道高手的标志之一。当初的常山阴,就有千人魂。

方源缓缓睁开双眼,一个念头,唤来小狐仙,令其立刻挪移进了山体内的荡魂行宫。

盘坐在蒲团上,他取出一枚白银舍利蛊。

经过之前的测验,这白银舍利蛊并无问题,方源便灌注真元进去。

须臾之后,他睁开双眼,查看了第二空窍一番,满意地点点头。

“第二空窍,也到达三转巅峰了。”

这样一来,方源的第一空窍,是五转巅峰,晶紫真元。第二空窍是三转巅峰,雪银真元。

雪银真元,虽然比不上晶紫真元,但好歹能给方源提供一些帮助了。

毕竟在北原王庭之争中,三转蛊师是绝对的中坚力量。

“待我用了黄金舍利蛊后,成就四转巅峰,那就更妙了。当然最令我开心的,是第二空窍赋予了我另一个本命蛊的名额。这样一来,我一人就能拥有两只本命蛊!”

(ps:迟到的一章,以及迟到的新年祝福!本来就预计初三初四的时候更新,好歹没有失信食言。呼……吐出一口浊气,先谢谢清逸尔雅童鞋的提醒了。今晚同学聚会,提前回来,牌没打,酒也没喝,就是为了赶这章。我知道,我亏欠了很多人,很多粉丝。在这里,我要向大家道歉,很不好意思!我这个人面皮薄,又懒散,缺点一大堆,能得到诸位的认同,真的很开心。诸位锲而不舍的追更,也令我高兴又惭愧。)

(伴随着新的一年,我的生活也得到了许多改变。之前,也向一部分朋友们透露过全职的想法。接下来,我会以对待工作的心态,以及努力,来完成这本书!当然,二月是个调整月,更新不多。三月开始真正发力!本书原计划是年底结束,但是过了年,发现才写了一半。这本书非同一般,我终于能摆脱一些东西,好好去完成一个幼稚狂妄却又真实的梦想。)

(2014,我来了,继往开来。网文,我来了,两只脚踏进来。诸君,我来了,不想让诸位失望,不想亏欠你们,那就让我们携手――圆这个常人难以理解的,狂傲又偏执的梦!)

(,22:50,蛊真人。)(未完待续。)

∷更新快∷∷纯文字∷

\end{this_body}

