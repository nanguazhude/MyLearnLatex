\newsection{还有后手}    %第二百二十节:还有后手

\begin{this_body}

飞熊虚像的战力,虽然没有真正的荒兽飞熊那样恐怖,但绝对是半仙战力。最关键的是,它看上去体型庞大肥硕,但偏偏动作迅猛敏捷。

巨阳的意志长河,左绕右绕,企图绕过飞熊虚像,直接攻击方源等人,奠定胜局。

但仍旧被飞熊虚像挡住。

飞熊虚像,就算是没有方源操纵,本身也有智慧,近乎狡诈,完全可以放任其自行战斗。

呼。

肥厚巨大的熊掌凶猛砸下,带出猛烈的风声。

砰!

金沙般的巨阳意志,被熊掌砸碎,化为点点金星,旋即消散于空中。

但只是一条支流而已。

黄金长河般的巨阳意志,宛若巨蟒一般,将飞熊虚像紧紧缠绕。

同时,又分出无数支流,袭向飞熊虚像各处。

双方纠缠在一起,一时僵持住,谁也奈何不了谁。

巨阳意志分分合合,变化随意,飞熊虚像没法克制它。但巨阳意志也被它纠缠住,抽不开身来对付最关键的人物——方源。

巨阳意志继承了仙尊的部分战斗智慧,它审时度势,清楚地明白:要斩杀方源,至少得分出长河一半的意志出来。

但留下的一半,根本不会是飞熊虚像的对手。

若是强行出击,就等若将主动权拱手相让。

方源如果带着真传光团飞速撤退,拖延时间。等到飞熊虚像将纠缠它的巨阳意志剿灭。那么局面就危险了。

随意分兵的亏,巨阳意志在之前抽调出真传秘境时,就品尝过。

那时。方源通过手头中的琉璃楼主令,悍然和一小部分的巨阳意志对拼。结果两败俱伤,最终导致:巨阳意志冒着风险,付出巨大代价,却只能夺回树人关卡。

方源面色凝重。

他一面维持真传光团,延缓巨阳意志的扩展进度,一面观察面前的战局。

“若是巨阳意志分兵来攻。那就好了。可惜,同样的错误,它并没有犯第二次。”方源观察良久。见巨阳意志和飞熊虚像打得如火如荼、不可开交,却始终没有分兵迹象,只得在心中一叹。

对手没有犯错,方源这边的处境就越来越危险。

现在。绝大多数的巨阳意志都涌出楼外。抵挡天劫地灾,保护八十八角真阳楼。只留下两部分。

一部分,和飞熊虚像纠缠。另一部分,则一直在和方源、太白云生的意念拼杀,不停地在炼化八十八角真阳楼。

代表着八十八角真阳楼归属的无上真传,原先只有一丝的黄金色彩。但现在,这丝金芒已经扩张膨胀到巴掌大小。

方源、太白云生已经拼命压榨自身潜力,两人浑身都在颤抖。脸色苍白,汗滴涔涔。却难以抵挡巨阳意志炼化扩张的大势。

巨阳意志,真的是太过浑厚了。

太白云生在对抗中,就感觉自己像是面对着崇山峻岭,无望无助。

即便是方源这般意志如铁的魔头,此刻心中也升腾起一股淡淡的颓丧之感。

若是任何局面如此发展下去,迟早有一刻,巨阳意志扩张成功,将八十八角真阳楼重新掌控。到那时,方源等人绝对是死无葬身之地。

必须做点什么!

然而,方源已经陷在这道无上真传上面,抽不开身。

他手头上最大的战力,便是飞熊虚像蛊,现在也用了。

太白云生是被他诓骗过来的临时盟友,但和方源一样,已经并肩作战,在和巨阳意志对决,根本腾不出手来。

可以说,除了春秋蝉之外,方源手头的牌都用尽了。

“难道,真的要再次催动春秋蝉不成?”方源的脑海中,一个念头闪过。

但旋即,就被他否决。

“春秋蝉并不是一定成功的仙蛊,还有失败的可能。我身上的黑死霉运,如此深重。一旦我自爆,仅剩下的意志退缩于春秋蝉中,回溯光阴长河,必然是凶多吉少!就连九死一生的概率,都谈不上。”

如何才能破局?

“唉……方源,事不可为,咱们撤吧!动用定仙游,我们就可以回到你的福地里去。”太白云生在一旁建议道。

真的要撤退回去吗?

且不说撒手不管,巨阳意志便能迅速掌控八十八角真阳楼。到那时,受它阻挠,定仙游蛊还能不能催动成功。

就算成功撤退了,甘心吗?

方源绝不甘心!

他筹谋太久,处心积虑,几乎已经胜利了,滔天的富贵利益就在眼前,但就差那么一点!

就差那么一点啊!!

“真的不甘心啊……等等!”方源神情一动,他想到了马鸿运。

“我手中无牌,但却可以利用这个棋子。”脑海中灵光一闪,方源再无迟疑,当即扇动背后双翼,拖着无上真传起飞。

“我们要去哪里?”太白云生忙问。

“跟着我就是。”方源只是这样回答。

定仙游蛊在他的怀中,又被他炼化。就算太白云生一心想要撤退,没有方源的配合,也是无可奈何。

“休想逃跑!”长河巨阳意志怒吼一声,倏地散开分解,但又旋即凝聚一体,绕过飞熊虚像,向方源二人杀来。

飞熊虚像反映过来,紧随其后。

方源冷笑一声,方向一折。三方换位之后,飞熊虚像又处在方源身后。

方源且战且走,双方战战停停。时而长河巨阳意志赶到前边,时而方源又绕开,借助飞熊虚像阻击。

这一路有惊无险。

虽有各个大小真传,四处乱飞乱撞。但却碍于无上真传的威势,不敢接近。

靠着飞熊虚像的掩护,长河巨阳意志也奈何不了方源的脚步。只能稍作拖延。

但方源的处境,其实也一样。

他和太白云生,同样奈何不了另一部分的巨阳意志炼化八十八角真阳楼。在意志比拼的战场上,方源和太白云生被死死地压在下风,金色华光一直在稳步扩张,二人落败只是迟早之事。

“原来你在这里!”当方源终于在视野中发现运道真传时,他仰头大笑。

“你想干什么?”身后的长河巨阳意志。顿时感到极度不妙,惊惶地嘶吼起来。

太白云生也疑惑地转头看向方源。

方源哈哈一笑:“你说,两个无上真传相撞。会是什么后果?”

太白云生大惊失色,方源的疯狂让他为之心悸。

但他仔细一想,却又发现:这真是一个绝妙的好主意啊!

他和方源两人,是骑虎难下。手中的真传光团无比棘手。撒手是死。不撒手迟早也要死。

但如果让两道无上真传相互对撞,必然会造成不可想象的后果。

这后果是如此严重,就连巨阳意志也为此失态!

“你,你不要过来!”马鸿运大叫。

赵怜云则被他紧紧地抱在怀中。

在这个舞台上,就属他们两人实力最低。若非运气太好,早就在激斗中辗杀成末了。

机缘巧合之下,运道无上真传成了他们俩的保护神,但是却又被推入真传秘境。

进入真传秘境。运道真传的威胁就降为谷底,巨阳意志要对付天外之魔。简直就是瓮中捉鳖。

若非巨阳意志一直顾忌保存运道真传,又被方源步步紧逼,无法抽手,忙于应付,否则早就对付马赵二人了。

马鸿运能安然至此,可以说,还多亏了方源和太白云生。

但现在,方源拖着无上真传,主动赶到了他这里。

“他,他想干什么?!”赵怜云浑身颤抖,透过半透明的光罩,她恐慌地看到方源脸上那抹冷笑。

“不,你不能这么做!快停下,收手!!”身后的长河巨阳意志大吼大叫。

它慌了!

一旦两道无上真传相互对撞,必定是两败俱伤。

届时,不仅是运道真传这个孤本的损毁,还有八十八角真阳楼也会遭受到难以估量的创伤。

长河巨阳意志剧烈沸腾起来,组成意志的颗颗念头,都接连自爆。

金沙般的意志长河,爆发出刺眼夺目的光辉,战斗力暴涨一半。

飞熊虚像怒吼一声,被压入下风。

方源眉头紧皱,情况危急,他不得不分出心神,亲自指挥飞熊虚像去战斗。

五百年的战斗经验,不是盖的!

在他的指挥下,飞熊虚像更加灵动,不顾伤势,终究将长河巨阳意志阻挡下来。

“方源,你这是自寻死路!你想要对付马鸿运,别忘了他身上可有鸿运齐天蛊的威能!只要你对他抱有敌意,你的运气就会急速下滑。纵使你有半仙战力,一直硬抗,但又能坚持到什么时候呢?”长河巨阳意志见突破不成,又叫喊起来。

“方源,巨阳意志说的很有道理,你可要小心!”墨瑶意志也道。

方源冷哼一声,心中没有一丝犹豫。

富贵险中求,对方源而言,也只有乱中取胜,才有一线生机。

眼看着两大无上真传,已经相距不到千步距离。但就在这时,空间泛起涟漪,陡然间数十道身影,被挪移进真传秘境。

他们直接出现在方源的飞行路上,几乎都是熟面孔。

耶律桑、古国龙、边丝轩……

为首之人,五大三粗,双拳如钵,一脸狞笑地望着方源:“原来你叫做方源!巨阳先祖已经将你们俩的罪行昭之于众了。你们的阴谋破裂了,一切到此为止,都留下命来罢!”

不是黑楼兰,还是哪个?

太白云生面色骤变,他现在抽不出手来,全力抵挡着巨阳意志炼化手中的无上真传。

黑楼兰这些人,出现得太要命了。

卡在最后的关口,成了致命的一击!

“哈哈哈!”身后的长河巨阳意志大笑,“你们来得好!”

它对黑楼兰等人,不吝赞赏。

这一定是意志主体抽空通知,调来的这群援兵。

若在平常时刻,黑楼兰等人的战力对于巨阳意志来说,根本不值一提。但此时此刻,他们却成了救命稻草,压倒双方天平的最关键的砝码!

“方源、太白云生,你们这次死定了!”长河巨阳意志大吼一声,放缓进攻。

主动权再次被巨阳一方掌控在手中。

“败了!”太白云生长叹一声,又转头看向方源,勉强振奋精神道,“师弟,你钻进我的仙窍。我们撤!我有仙元,你有定仙游,巨阳意志又被天劫地灾拖住。再不撤,就来不及了!”

方源却摇摇头:“已经来不及了!一旦我们放手,八十八角真阳楼就会落入巨阳意志掌控。此楼有摄取仙蛊之能,覆盖整个北原。就算催动定仙游,也极可能失败。”

“我当然知道这一点。”太白云生语气急促,“但只有一搏,才有一线生机啊。再拖延下去,错失良机,我们连这丝机会都没有了!”

方源目光闪烁不定,看了一眼手中的无上真传。

真传光团中,金色的华光已经占据了光团三分之一的体积,扩张的速度越来越快。

方源深呼吸一口气:“没想到还是走到了这一步。唉,若不是局势所逼,我是不会动用这一招的!”

“师弟,你居然还留有手段?”太白云生闻言,又惊又喜又疑。

ps:通知。

本书qq书友群将于国庆节后并群。届时,《蛊真人》官方1、2、3群,以及真人幕僚团,一起合并到“蛊真人铁杆营”,方便聊天,不然窗口太多。保留《蛊真人》vip群,入此群须得全订阅。

造成的不便,还请诸位书友理解包涵。

谨请诸君移驾,请相互通知转告。

不胜感激!

\end{this_body}

