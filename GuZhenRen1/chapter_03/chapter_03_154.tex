\newsection{仙尊意志}    %第一百五十四节:仙尊意志

\begin{this_body}

%1
炼化八十八角真阳楼,乍一听异想天开,但其实并非妄想。

%2
事实上,方源也不是“第一个”炼化八十八角真阳楼的人。

%3
在他五百年前世,中洲蛊仙攻破王庭福地,就是先派遣的蛊师混入到八十八角真阳楼的秘藏阁中。

%4
事成之后,他们将整个过程,都用蛊虫存放起来,然后布告天下。

%5
王庭福地乃是北原蛊师的精神象征,意义非凡。记录着攻破王庭福地的影像,在五域流传,不仅是彰显中洲实力,而且是试图击垮北原蛊师精神支柱的毒计!

%6
更关键的是,这段详实的影像,是证明巨阳仙尊搜刮北原蛊虫,为自家后裔血脉谋利的铁证!

%7
影像一出,北原震荡,民愤喧天。

%8
虽然被各大黄金部族镇压住了局面,没有达到中洲蛊仙们预期的内乱程度,但的确营造出了一个暗流汹涌、人心离散的北原。

%9
中洲蛊仙宋且行,看过这个影像之后,就一针见血地评价说:“此等影像一旦流传天下,北原自由的精神,就将从巨阳仙尊的牢笼里解放!”

%10
方源对这段影像,自然印象颇深。

%11
他重生之后,立即醒悟到这段影像的巨大价值。

%12
北原之行,其实救治荡魂山,只是其中一个目的。

%13
方源性情谨慎,凡事先思败后思胜。

%14
“这个世界上哪里有心想事成的好事?荡魂山万一救治不下,那么我还可以从八十八角真阳楼中,获得其他方面的弥补。”

%15
中洲蛊仙攻破王庭福地的影像,对方源而言,具有极大的参考价值。

%16
但是单单这段影像,还不够。

%17
方源只能从影像中,看到表面的东西。但幸好,方源从琅琊福地当中,又获得了第一手的资料情报。

%18
如此一来,理论联系实践,炼化八十八角真阳楼的把握,就大大增加了。

%19
“以我如今的修为,要想完全炼化八十八角真阳楼,是不可能。但我可以炼化当中的一部分。”

%20
方源对现实看得透彻。

%21
他不过是一届凡人蛊师,要完全炼化仙蛊屋,至少得八转蛊仙的层次。

%22
方源的计划,只是炼化八十八角真阳楼的一部分。

%23
八十八角真阳楼历经沧桑岁月,早有损耗,暗布漏洞。它太宏伟了,太庞大了,就像是一个巨大的木笼子。

%24
方源和其比较,如同一只白蚁。

%25
一只白蚁的力量,是无法腐蚀整个木笼子的。但是却可以腐蚀掉一些边边角角。这两者的难度,有天地之差。

%26
眼前的来客止步碑,在被射入特定的蛊虫之后,发出一阵昏黄的光。

%27
方源趁机伸出双掌,调动空窍中的真元,灌注进去。

%28
同时,他的意志也随着真元,入侵到来客止步碑的内部。

%29
蛊师炼蛊,就是将自己的意志占据蛊虫身躯,而这个过程中,真元就是优良的载体。

%30
方源眉头紧锁,心神投入。

%31
来客止步碑,只是八十八角真阳楼的一部分。

%32
八十八角真阳楼乃是八转仙蛊,太过浩大。

%33
方源的意志一进入其中,就仿佛置身在一片黑暗当中。

%34
在这片广袤无垠的黑暗里,有一颗太阳般的存在。它发出极微弱的光,光晕仿佛呼吸一般,有规律地波动着。

%35
“这就是巨阳仙尊的意志么?”方源心中顿时十二万分的警惕。

%36
八十八角真阳楼乃是巨阳仙尊之物,被他炼化,自然里面就存在着巨阳仙尊的意志。

%37
巨阳仙尊的本体虽然早已经逝去了,但是他的这股意志,却寄托在八十八角真阳楼的躯壳中,长存了下来。

%38
“这样的意志,真是广博浩大,让我感觉仿佛面对一颗真正的太阳!仙尊的力量难以想象,这还只是巨阳仙尊一小股意志,又历经悠长岁月而残留下来。”

%39
“幸亏这股意志,陷入了沉睡当中,我动作轻微,应该不会惊动了它。千万不能把它惊醒,否则前世中洲影像中的那两位魂飞魄散的蛊仙,就是我的下场。”

%40
广阔无垠的黑暗空间,代表着八十八角真阳楼。

%41
巨阳仙尊的残存意志,大如旭日,镇压中央,陷入沉眠,散发微光。

%42
而方源的意志,和其相比,只是芝麻大小。亦散发着微微光明,潜伏在最最边缘的角落里。

%43
方源不断灌注真元,小心谨慎。

%44
随着真元侵入来客止步碑,他灌注在八十八角真阳楼中的意志,也越来越多。

%45
黑暗的角落里,象征方源的光明,不断膨胀,渐渐地将一片黑暗驱逐,自己占据。

%46
时间一点一滴的过去。

%47
方源小心翼翼,额头慢慢地布满汗珠。

%48
“想不到这碑炼化起来,如此不易。我的两个五转巅峰的空窍,双九成的真元都稍显不足。若非有刚刚得到的天元宝王莲,恐怕还得费一番周折。”

%49
整整两个时辰过去,方源这才吐出一口浊气,将贴在碑面上的双掌收回。

%50
他浑身疲惫不堪,主要是心理压力极大,比悬崖上走钢丝还要惊险。

%51
“终于成功了。”

%52
方源望着眼前的来客止步碑,一股亲近之意,从碑上传来,直达他的内心深处。

%53
但成功的喜悦很快消散,方源的眉头皱得更深。

%54
“前世的影像,果然有许多删减。我付出这么多的真元,但在影像中的蛊师,不过是五转中阶,中途没有休息,只炼化了短短半个时辰。”

%55
也许这蛊师空窍中,含有什么辅助蛊虫。但方源更相信影像删减的可能。

%56
中洲蛊仙传播的这番影像,主要目的是为了打击北原黄金部族的势力,解放北原他族的精神自由。

%57
深入到八十八角真阳楼,肯定有不想暴露的收获,或者见不得人的手段等等。

%58
同时,为了使得影像更生动精炼,引人入胜,删减无趣冗长的部分,也是认知场景。

%59
但这对方源而言,却是个大大的坏消息。

%60
炼化八十八角真阳楼本就是一项危机四伏的冒险,万一被前世影像误导,棋差一筹就极可能满盘皆输!

%61
“八十八角真阳楼果然非同凡响,我辛苦这么半天,恐怕连半分的程度都没有炼化掉。”

%62
方源心怀感慨,拍拍来客止步碑,站了起来。

%63
如果将八十八角真阳楼分成十成,那么巨阳仙尊的残留意志,占据当中的三成。

%64
一成有十分,方源炼化了来客止步碑,连半分都没有。

%65
“不过就算如此……”方源的嘴角,勾勒出一丝微笑。

%66
他悠然转身,往回走几步,随意来到一处封印了宝材的晶壁前。

%67
他目光微凝,伸出手,径直朝晶壁探去。

%68
若在之前,晶壁必如冰墙,挡住手的去路。但是现在,来客止步碑微光一闪,晶壁便光影变化,由实化虚。

%69
方源的手,像是探入水中一样,顺利地探入进去,将里面的宝材成功地取了出来。

%70
一旦炼化了来客止步碑,那么这段晶壁中的任何珍宝,方源都能随意获取,而不用付出任何代价!

%71
“哦?这应该是奔雷石吧……”

%72
望着手中的宝材,方源仔细辨认,这才确认。

%73
奔雷石是相当稀少的炼蛊宝材。如今已经几乎绝迹,宝黄天中也极为偶尔,才会贩卖。

%74
这种石头,乃是九天的雷霆相互轰炸时,从而形成的雷霆真精。

%75
但自太古时代,九天就陨落了七天,只剩下白天和黑天。这两天的雷霆相互对炸的几率,十分稀少。因此太古时代之后,奔雷石的产出就已经十分稀少了。

%76
雷道盛行时,大量损耗了奔雷石,用作炼蛊。

%77
因此如今奔雷石,存量极少。

%78
“天地变化,沧海桑田,雷道也已经变革,不再需要奔雷石。只有那些想要研究古代雷道蛊虫的蛊师蛊仙,才会对奔雷石感兴趣。”

%79
秘藏阁价值非凡,方源随手取出的一份珍藏,就是一颗奔雷石。

%80
但随后,方源却又将这块奔雷石,重新放回到晶壁当中。

%81
小不忍则乱大谋。

%82
因为要换取才能得到,晶壁中的珍藏数量,都是固定的。

%83
而这里的每一份珍藏,各大超级势力,甚至某些大型的黄金部族,都有记载。

%84
如果接下来,又有后来人,有了上等通关的战绩,来到这里。发现这里少了珍藏,那该会多么惊骇狐疑!

%85
方源毫不留念,试验了一番后,他再度朝水晶走廊的深处走去。

%86
再次来到来客止步碑时,他脚步微顿,速度放缓。

%87
几个时辰前,阻止他的那堵无形气墙,已经消散无踪。但这并代表,他就能安然进入当中。

%88
八十八角真阳楼乃是长毛老祖所炼,自然还有旁的甄别手段。

%89
但方源自然早有准备。

%90
他心念频动,一时间,五六只各样的奇特蛊虫从空窍中飞出,化为一团团彩色烟气,罩住他的全身。

%91
方源又检查了一遍,确认遮掩周全了,这才越过来客止步碑。

%92
各色的烟气顿时沸腾起来,形成一道血光,弥漫左右。

%93
方源环顾左右,发现晶壁中的珍宝,果然比前一段要好上一个档次。

%94
“嗯?这个是……”

%95
蓦地,方源目光微微一凝,见到晶壁中封存了一只五转的力道蛊虫。

%96
他心中大喜。

\end{this_body}

