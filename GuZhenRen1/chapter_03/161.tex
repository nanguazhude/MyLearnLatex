\newsection{近水楼台解玄秘}    %第一百六十一节:近水楼台解玄秘

\begin{this_body}

进行到这一步,方源已是叹为观止。

这位布置地丘传承的神秘蛊仙,绝对是炼道宗师的造诣。整个炼蛊过程,都是借助八十八角真阳楼的伟力,巧妙而又大胆。

方源虽然是主持之人,但是充其量还不如说是一位协助者。

炼蛊艰难,炼就仙蛊,更是难上加难,难比登天!

然而打个比喻的话,这颗巨石已经被神秘蛊仙推上了山巅。方源不过是最后出一把力气,轻轻一推,将这巨石推下山峰,砸落下去,就是功成。

嚯嚯嚯嚯……

地洞已经成了一个黑水漩涡。漩涡不断旋转,发出古怪之音。

随后,先是塔尖徐徐冒出,然后是塔顶、塔身,最后是塔基。一道全新的小塔楼,里面似有野蛊无数,从漩涡中生成而出。

“妙哉!这座小塔楼一出,差不多弥补了最后一个破绽。即便是有人因为动静赶赴到这里时,也看不出蹊跷了。”方源拍掌赞叹。

只是……

塔楼虽成,但是仙蛊怎么没有出现?

方源目光微微沉凝,发现小塔楼中,有一道强光若隐若现。心中更有一层微弱牵挂,与之相连。

他顿时明白,这便是仙蛊雏形。

他正要借助这层羁绊,勾动仙蛊出来。

忽然,这道强光宛若虚幻的光影,穿脱而出,向东南方疾驰而去。

“怎么回事?”方源心中一惊。到了此步,炼蛊四步“土中蕴光。芒高万丈。百里天游,梅咏雪香”已经完成,方源对这等异变一无所知。

“八十八角真阳楼重新凝聚。黑楼兰全面开放,着实是个大手笔。按道理,我应该前往楼中闯关,才是人之常情……”

仙蛊雏形已经远飞而去,方源望着圣宫犹豫了一下。

他虽然身处外地,但早已安排了棋子。不管是常家,还是葛家方面。都有消息实时地传达过来。

方源已经离开圣宫越久,就越会引起怀疑。若是因为卡关止步,需要狼王的力量。有人寻觅而来的话……

传递消息,也需要一个时间。

方源还不知道,他的敌人常飚间接地帮了他一个小忙。

此刻,唐妙鸣已是险之又险。成功地突破了关卡。引发楼中一片欢呼。

终究还是仙蛊为重,方源只是稍稍犹豫了一瞬间,便下定决心,展开双翼,追逐强光而去。

这强光形如蚕茧,紧贴地面而行,速度奇快,叫人咋舌。又因为草丛树木的阻挡。并不引人瞩目。

方源在高空追逐,尽量掩盖行迹。舍弃一切狼群,销声匿迹。

随着时间的推移,这点强光越发黯淡,但速度却又激涨几分。

他虽是五转巅峰,但速度并非凡俗顶峰,追得很是辛苦,但总归没有被彻底抛下。

归根结底,都是因为这只仙蛊雏形,不是单靠方源一己之力,独立完成。

和当初炼制神游蛊不同,这次方源只是协助者,靠的是八十八角真阳楼的伟力回流。又是凡夫俗子,能和仙蛊雏形有所联系,已经是十分努力的成果。

强光飞进一出山谷,忽然钻入一道瀑布当中,不见了踪影。

但靠着心中这点联系,方源觉察到雏形仙蛊的静止。

方源穿透瀑布,但却撞在湿滑的山石之上,一时间土石翻飞,水流乱溅。

“奇哉!”

方源大叫奇怪,心中的联系仍旧告诉他,强光就在这水中,但他分开水流,甚至彻底摧毁了这道瀑布,却左右搜寻不到仙蛊雏形的踪影。

“难道我要功亏一篑?不,这地方大有蹊跷!”

他升腾而上,俯瞰这片地带。

这小型瀑布已经变为一汪水潭,平凡无比。山谷无名,亦非钟灵毓秀之地。

方源将目光盯住这汪水潭。

心中的牵挂告诉他,雏形仙蛊明明就在其中,但任凭他穿透水潭,调水控流,也搜刮不到。

这个时候,就是考验蛊师手中的侦察蛊了。

方源自然不肯甘心,试了多种法子侦察。他虽然不擅侦察,但遥控狐仙福地,沟通宝黄天,却是不缺五转凡蛊。

直到尝试到第五十七种法子,花费了不菲的仙元石之后,他这才有所发现。

只看见水潭中,有一个影影绰绰的楼阁,宛若镜中花,水中月。

他进入水潭,却不能进入楼阁当中。

试了几次徒劳无功之后,方源啊呀一声,心中灵光一闪,终于认出此楼的来历。

“难不成这是失传多年的仙蛊屋——近水楼台?”

仙蛊唯一,仙蛊屋自然也唯一。

这近水楼台要逊色于八十八角真阳楼,只有七转级数。但同样名声远播,乃是当年蛊仙水妮的招牌标志。

水妮是八转蛊仙,开创水道的传奇。亦是中洲十大派之灵缘斋的创派祖师。

相比较八十八角真阳楼的恢弘浩大,近水楼台,则是可以依附水中,似虚似幻,奇妙无方。它可以随波逐流,也可以寄托于水雾当中,行于天空。还可以藏于冰霜角落,静处一隅。

没有进楼令牌,近水楼台甚至可以阻挡八转蛊仙。

不过此刻,此楼无主,门扉敞开。方源先前是被蒙在鼓里,现在发现了真相,自然有相应手法进入其中。

“要进入近水楼台,只能以身化水。有一个杀招,名为水灵变。统合几蛊,便可以令我化身水中精灵,战力暴涨,拥有主场优势。”

方源搜刮记忆,立即找出一种方法。

不过最终,狐仙地灵替他在宝黄天中。收购到了一只五转激突蛊。

此蛊能够让蛊师短时间内变身激流,向前冲刺一段距离。不过早已经被蛊师界淘汰,皆因蛊师化身激流时。一旦被炎道蛊虫攻击,轻则重伤,重则死亡。

不过方源并不用此蛊战斗,用作进入近水楼台,却是比整合蛊虫,形成水灵变杀招,要方便得多。

哗的一声。方源化身激流,冲进近水楼台。

楼台不大不小,有三层。雕梁画栋。罐瓶桌椅,尽显古时风采。

方源顺利进入之后,便收了激突蛊,一连上到第三层。推开一处房门。再次发现仙蛊雏形。

只见这仙蛊雏形,宛若蚕茧,只有小拇指大小。此刻正沉于一只朱红大碗当中。

这大碗,比水缸还大,碗口参差不齐,宛若鲨鱼锯齿。

里面一汪幽蓝水液,寒气四溢。

“原来如此。”

方源恍然大悟。

虽然进行了四步,炼出仙蛊雏形。但却还不是大功告成。

仍需温养,才能彻底孕育。

神秘蛊仙在这里布置了近水楼台。就是提供给雏形仙蛊一个上佳的温养之地。

朱红大碗表面,刻有几排文字。

方源辨认了一下,以他的渊博学识,竟然只能认出一小半的意思。

“这是墨人文字,传说中,乃是乾坤晶壁中书山文泉水滴溅落,形成的第一种文字!”方源心中颇为惊诧。

这种文字,早已经失传。就算是墨人,自从被迫从书山迁徙而出后,就很少知晓得了。

“宝黄天的蛊仙手中,应该有墨人文字的研究资料。毕竟很多蛊仙,都对书山感兴趣,一直在搜寻灰白石板,企图重现乾坤晶壁。”

方源心中一动,立即联系地灵小狐仙。

狐仙福地的价值,再次体现出来。让方源区区凡人,却能引动蛊仙这一层次的宝贵资源。

花费巨大代价,换取了一些墨人文字的资料之后,方源当场一一对比,初次剖析碗壁上的文字。

解译的结果,让方源又惊又喜:“原来这处传承,竟然是墨瑶所立!”

这墨瑶亦是惊才艳绝之辈,乃是灵缘斋第三十六代仙子。她身份特殊,乃是墨人。

但最终她打破种族桎梏,成为七转蛊仙。

她为正道做出的最杰出的贡献,乃是以情感化了万年前的魔道巨擘剑魔薄青。

薄青乃是一介散修,起于微末。但天资卓绝,独自一人开创剑道蛊虫,纵横五域,无人可敌。

他是浩瀚历史当中,最为瞩目的八转巅峰强者。战力卓绝,惊天憾地,一身剑道蛊虫,更是别出机枢,另辟蹊径,威力绝伦。被人称之为“剑劈五洲亚仙尊,为情所系幸苍生”。

其意是:此人战力恐怖绝伦,仅仅次于历代仙尊魔尊。幸亏他为情所系,由魔转正,真乃天下苍生的大幸事啊!

在当时,薄青更是被世人看好,冲击九转境界的不二人选。

可惜最终,他冲击九转失败,化为灰灰。身为他的贤内助,墨瑶也以命相拼,最终一同殉命。

“历史记载,墨瑶乃是货真价实的炼道宗师。难怪了……当年巨阳仙尊广招妃嫔,中洲十派都供奉美女蛊师。其中灵缘斋,更派遣了数位女蛊仙主动投入各大行宫,献媚仙尊。巨阳仙尊宠幸过的妃嫔当中,就有几位巨头,都来源于灵缘斋。”

既是枕边人,那对圣宫,对八十八角真阳楼的了解,就远超旁人了。

墨瑶乃是灵缘斋的第三十六代仙子,自然受到灵缘斋的全力栽培,调查八十八角真阳楼的隐秘消息,对她而言,却是轻而易举。

此处的近水楼台,也可以解释了。

墨文的中段内容,则详细介绍了这只仙蛊的信息——

招灾蛊!

ps:完蛋……写完这章,发现还没写到《人祖传》……难道,是我太啰嗦了吗……下一章,明天的下一章就是了。(未完待续

∷更新快∷∷纯文字∷

------------

\end{this_body}

