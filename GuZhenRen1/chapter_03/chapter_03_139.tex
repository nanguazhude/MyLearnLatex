\newsection{陷害蛊仙?}    %第一百三十九节 陷害蛊仙?

\begin{this_body}

%1
当马英杰从昏睡中,终于睁开双眼,第一眼看见的便是费才欣喜若狂的笑脸。

%2
费才大叫起来:“少族长,你终于醒啦!”

%3
憨厚的声音,让马英杰心中一暖。后者挣扎着坐起来,剧痛让他裂开了嘴,吐出一口血沫,艰声问道:“这是哪里?”

%4
费才挠了一下头发,惭愧地道:“我也不知道这是哪里,不过我们应该逃出了战场。”

%5
“战场?”马英杰陡然一惊,立即问道,“战场上怎么样了?”

%6
“我们失败了,少族长大人。好多人都在逃亡,更多的人投降了。”费才答道。

%7
马英杰脸色变得雪白无比,雄躯一颤,好悬再一头昏倒下去,幸亏费才在身后撑住他的脊背。

%8
赵怜云站在一旁,看着昔日光芒万丈的马家少族长,如今落到如此狼狈不堪的下场,心中也不是滋味。

%9
“唉,这马英杰其实也算是年轻有为,可惜就是碰上了狼王常山阴。不是你不努力,而是对手太变态啊……”

%10
马英杰闭上双眼,两道泪水默默流淌下来。

%11
好半天,他这才睁开通红的双眼,扭头看向费才和赵怜云二人,声音嘶哑地问道:“是你们救了我?”

%12
费才和赵怜云同时点头。

%13
“少族长,我们现在该怎么办啊?”费才问道。

%14
马英杰脸色阴郁,沉声道:“我们回去!这一场是我们马家败了,但在暖沼谷,我们还有一部分的族人。”

%15
当初,在英雄大会之前,马家阴谋策划了费家的内乱,将费家吞并,占据了暖沼谷。

%16
马家高层,为了以防万一。将一部分的老弱病残,都布置在暖沼谷内。

%17
如果马家获胜,就将他们接过来。如果马家失败,他们就是延续部族的种子!

%18
“要回暖沼谷?可是我们没有水,也没有干粮。要赶那么远的路……”赵怜云眉头紧蹙。

%19
“呵,小丫头。只要有我在,就有充足的水和粮食。你们用不着担心。”马英杰道。

%20
三人结伴而行。一路上遇到不少马家逃亡出来的族人,被马英杰一一收拢。

%21
“少族长大人,没想到我马由良,还有能见到您的时候!”马由良见到马英杰后,泣不成声。

%22
他是马家的三转家老,如今躺在担架上,缺少了一只胳膊,右小腿更是断裂,受伤很重。

%23
马英杰看到他。一对虎目也不禁泛出激动的泪花:“马由良家老,能见到你真是太好了!”

%24
他虽然一路上收拢了不少族人,但绝大多数都是凡人,马由良虽然重伤残疾,但到底是个蛊师。

%25
经此一战,马家大败亏输。

%26
战前。是大型部族,底蕴深厚到能冲击超级家族。但战后,马家已经彻底沦为小型部族,实力衰落到低谷,盛极而衰。

%27
对于现在的马家而言,每一个蛊师,都是部族最宝贵的力量和希望!

%28
“少族长大人。老族长他已经战死沙场了。”马由良放声痛哭起来,带给马英杰一个噩耗。

%29
马英杰身躯剧烈晃动了一下,尽管他早有心理准备,但此刻听到这个消息,心中还是充满了无尽的悲痛和哀伤。

%30
他咬紧牙关,整个人仿佛化身成为一个铁像。

%31
沉默了片刻后,他用力擦干眼眶中的泪水:“那么,从今天起,我就是马家的族长!马由良家老,你要振作起来。我们马家虽然失败了,但是并没有灭亡。当年,巨阳先祖订下规矩,不能尽数屠戮黄金家族的血脉。马家已经为战争付出了惨重的代价,现在就算是黑家也不能对我们赶尽杀绝。我们回暖沼谷去,我们要从失败中爬起来。我相信,马家的辉煌,不会经此而散的!”

%32
马由良怔怔地看着眼前的少族长,模糊的视野中他似乎看到了马尚峰、马尊的身影。他收起抽泣的声音,希望又从他心中升腾而起。

%33
下一刻,他用最深沉的声音回答道:“族长大人,我也相信!”

%34
成王败寇。

%35
在马英杰惨淡逃亡的同时,黑家盟军的无数营帐中,却是庆功的酒宴,欢呼的人群,温暖的篝火,以及丰富的美食。

%36
“我们胜利了,胜利了!”

%37
“马家的野心太大了,居然妄图成为超级部族。正是这个野心,毁掉了他们。”

%38
“祝贺黑楼兰大人,成为王庭之主!”

%39
“尊贵的狼王大人,请容许卑微的在下,敬您一杯酒。”

%40
王帐里,同样是觥筹交错。除去美酒佳肴之外,还有美丽热情的北原女子,在尽情地舞动曼妙的身姿。

%41
在座的,都是黑家盟军的首脑,大大小小的强者。修为至少是四转级数,可以说都是当代北原最为闪亮的人杰。

%42
黑楼兰坐在主位上,在他的左手边的首位,就坐着方源。

%43
本来方源的位置,因为太白云生的到来,以及陆续的几位五转族长,已经被挤到后面去。

%44
但经过和和马家的第三场大战后,方源暴露出五转巅峰的修为,同时以一己之力,对战三位奴道大师,将成龙、邬夜等强者斩杀。

%45
可以说,黑家能够大胜马家,奠定胜局,七成之功都在方源的身上。

%46
黑家上下震撼于狼王的恐怖战力,就在当天晚上,就调整了王帐中的排位。

%47
没有人对此发出异议。

%48
面对敬酒的强者们,方源来者不拒,但每次只喝一小口。这完全不是豪爽的北原勇士的风范,但是此时此刻,放在方源的身上,却演绎出狼王的高傲独孤之气。

%49
王帐中,一片欢乐的氛围。

%50
敬酒的蛊师,正是单刀将潘平。他在大战中,利用单刀蛊,幸运地取得了马家族长马尚峰的首级。因此在如今的战功榜上,只逊色方源一人,排在第二位。

%51
见方源抿了一口酒,潘平满怀感激地退下。

%52
北原人敬佩勇士。方源如此恐怖的表现,就算是历届的王庭之争中,也十分少有。

%53
飞行大师、奴道大师,双大师的光环笼罩他的身上。如此奴力双修,将让他的任何一个强敌,都会感到极度的头疼。

%54
看着潘平带着满足和激动之情。恭敬而退。周围看向自己的目光中,也都是敬仰、崇拜。或者忌惮之色。方源不动声色地放下酒杯,心中暗暗感慨:“不知不觉间,我已经到达这一步了啊。”

%55
凭借王庭之争这股东风,方源个人的战力急速膨胀。到今天这步,已经算是凡俗的巅峰。

%56
对于凡人而言,已经到顶了。

%57
再往上,就是仙的境界!

%58
之前的大战中,他屠戮成名强者,纵横披靡。无人可挡,这样的表现,就算是五转巅峰的蛊师也很少能够做得到。黑楼兰的风采,也被方源一人尽夺。

%59
奴力双修,尽管有巨大的缺陷。四臂地王杀招,即便是草创。很不完善,但已经足以作为基石,支撑着方源傲立凡尘。

%60
在三王山上,方源是借助福地之力,斩杀四转、五转的蛊师强者。

%61
而如今,他是凭借自己的实力办到。就算是铁家上一代族长,铁慕白重生。方源也有斩杀他的信心。

%62
短短时间,方源的战力实现了飞一般的暴涨。这番旁人终身都难以达到的巨大成就,是建立在五百年前世经验,狐仙福地,以及苦心筹谋的基础上的。

%63
“但是还不够啊,远远不够!世俗的巅峰,算得了什么?不成蛊仙,始终都是棋子。不提永生这个目标,就说近的——荡魂山还没有救下呢。”

%64
方源目光沉凝,胸中翻腾的是野望雄心的熊熊烈火。

%65
他扫了一眼身旁的太白云生。

%66
要救荡魂山,需要太白云生的仙蛊——江山如故。

%67
这是他北原此行,最主要的目标。甚至,八十八角真阳楼还在其次。

%68
在他的计划中:如果荡魂山救不活,那么八十八角真阳楼中的传承,将尽可能地弥补他的损失。

%69
但是要取走他人的蛊虫,是件很麻烦的事情。

%70
蛊虫的存亡,是在主人的一念之间。

%71
比方说方源,只需要他的一个念头,纵然是仙蛊春秋蝉,也会轻易自爆而毁。

%72
正是因为如此,往往蛊师战死,从他们的尸体上缴获的蛊虫,微乎其微。

%73
而太白云生的情况,要更加麻烦!

%74
江山如故这只仙蛊,还没有问世。它是由太白云生成就蛊仙之后,以手中的江如故、山如故两蛊为主要材料,炼成的独创仙蛊。

%75
于是,摆在方源面前的,就有两种方案。

%76
第一种,是活捉太白云生,利用魂道蛊虫搜魂,将他脑海中江山如故的蛊方,搜刮出来。再自己出手炼制。

%77
这个方案,风险太大。

%78
首先,方源不一定能活捉太白云生。活捉和斩杀,完全是两个概念。尤其是太白云生威望很高,又是宙道蛊师,一旦失败,影响很大。

%79
其次,就算活捉了,能不能得到江如故、山如故两蛊呢?万一太白云生一个念头,让这两蛊自爆,方源就竹篮打水一场空了。

%80
最后,现在的太白云生的脑海中,是否有江山如故这件蛊方,还是个未知数。

%81
一介凡人,能设想出独创的仙蛊秘方,这种可能性太小。尤其是太白云生,并非炼道大师。

%82
据前世隐约的传闻,江山如故是在太白云生成就蛊仙时,天地感应,道纹相吸,灵感爆发,使得两蛊自发合并,炼成六转仙蛊的。

%83
如果是这样,那么江山如故这件蛊方,就根本不存在。炼制江山如故的过程,完全不可复制了。

%84
第一种方案,并不可取。那么,第二种方案呢?

%85
其实不比第一方案好到哪里去!

%86
太白云生是进入王庭福地后,成就蛊仙的。第二种方案,就是在他成为蛊仙后,拥有了仙蛊之后再来对付他!

%87
这就意味着,方源要以凡人的身份,来陷害一位蛊仙!

%88
ps:第二更稍晚一些。另:庆贺色iunica同学,成为本书盟主!在此感谢色iunica 君的大力支持!

\end{this_body}

