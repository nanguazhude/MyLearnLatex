\newsection{换蛊}    %第五十七节:换蛊

\begin{this_body}



%1
“少年郎,您请喝茶。(文學館是上好的云烟茶哦,快尝一尝罢。”琅琊地灵做到云床旁,亲手为方源沏了一杯茶。

%2
方源坐在地灵曾经坐的位置上,无语地看着眼前的这个活宝地灵。

%3
刚刚方源又和地灵交谈了几句后,弄明白了——原来这地灵极其喜欢收藏秘方,看到精妙的秘方千方百计就想得来。

%4
若是旁人,地灵早就敲诈勒索了。但对于方源这个魔尊后人,他恐吓不住,只能软语相求。

%5
“英俊的少年郎啊,茶好不好喝呀?开不开心啊?看在这盏茶的份上,那个人皮蛊的秘方,你就换了吧。”琅琊地灵一脸讨好地笑着,对方源挤眉弄眼。

%6
方源沉默地喝着茶。

%7
原本心中仙风道骨,神秘强大的琅琊地灵的形象,正在迅速的崩塌。

%8
“少年郎,你就行行好吧,可怜可怜我这个老人家!我一个人生活在这里,多么苦闷,多么寂寞,多么饥渴啊。我每天只能看这些秘方解闷。难道你能忍心,残忍地拒绝我这个可怜的老头子吗?”地灵一副白胡子白头发白眉毛的老爷爷形象,眼巴巴地望着方源。

%9
“喂,你有点自觉好不好。你是地灵,不是人呐。”方源眼角抽搐着道。

%10
“少年郎,你说什么就是什么。只要你能把秘方给我,你怎样我都行!”老爷爷向方源抛媚眼死亡列车。

%11
方源强忍着将这货一脚踢下去的冲动,大吼道:“不换,就是不换。”

%12
琅琊地灵浑身僵滞了一下,然后放声痛哭,在地上打滚:“不,我就要。我要秘方啊。我要秘方啊。少年,你太残忍了,太凶残了,太没有同情心了。居然不换,换了你能死啊!你就换吧。”

%13
“真是,真是百闻不如一见啊……”方源满头的黑线,心中原本假象的,关于琅琊地灵的神秘强悍的形象,已经碎了一地。然后又被人狠狠踩成了渣滓。

%14
不过。这倒也显示出了琅琊地灵的赤子之心。

%15
地灵老爷爷在地上乱滚了几圈,哭号了许久,仍旧不见方源松口。

%16
他只好站起来,脸色眼泪鼻涕糊了满脸,连白头发白胡子白眉毛上都沾着。

%17
“少年郎。你良心大大的坏啊。实在太残忍了,真不愧是魔尊的继承人……”老爷爷一脸幽怨地看着方源,好像是一个被方源始乱终弃的怨妇。

%18
方源终于忍受不住这样的眼神,打了个寒颤,叹了一口气道:“罢了罢了,人皮蛊的秘方可以换给你。不过要等到时间成熟了之后。”

%19
“少年郎,你太好。你真是个大好人啊。那什么时候时机成熟啊?”地灵一蹦三尺高,开心极了。

%20
“呵呵呵,五百年后。”

%21
地灵老爷爷翘立的眉头,顿时耸搭下来:“要这么久啊……”

%22
“哼。这是我最大的让步。怎么,你不想换?”

%23
“换,怎么可以不换呢。算算五百年,我也可以等。少年郎。我会一直等你的哦,这是我们一生的约定。”老爷爷深情地道。

%24
方源捂住脸。深深地叹息一声:“我这里还有秘方,想换一只通天蛊。”

%25
一听到方源有求于自己,琅琊地灵立即换了脸色,身躯挺直了,脑袋微微后仰,一脸傲色:“哦,想换通天蛊啊……”

%26
他声音慢条斯理,又继续道:“实话告诉你,少年郎,我这里可有海量的秘方。你拿出的秘方,如果我这里有的话,我是不会换的。”

%27
方源自信一笑:“老家伙,你已经过时了。拿纸笔来,我给你先写一道。”

%28
刚写了一半,方源就停笔不写。

%29
“写啊,快写啊,还有什么?”地灵站在一旁,急得抓耳挠腮,双目放光。他已经确定,这是一道全新的秘方,他从未见过。

%30
“这可是五转秘方,你的通天蛊呢?”

%31
“在这,在这。”地灵随手一招,将一只通天蛊挪移到书桌上。

%32
方源写完这道秘方,将通天蛊炼化,收入空窍。

%33
一旁,地灵喜滋滋地看着这张秘方。

%34
秘方上的蛊虫,是五百年后的大时代,蛊师们研发出来的新蛊。在那个五域混战,烽火燃烧整个世界的年代,各种新蛊层出不穷。

%35
每当这种乱世,往往都会出现九转的蛊仙。

%36
方源有着前世记忆,虽然也遗忘了许多,但脑海中仍旧有大量的秘方。

%37
这些秘方,夸张一点说,简直就是领先一个时代!地灵当然没有见过。

%38
“你这里有神念蛊么?”方源收了通天蛊后,又问现代张天师全文阅读。

%39
“有啊。”

%40
“换么?”

%41
“难道你还有蛊虫秘方?”地灵老爷爷又惊又喜。

%42
方源含笑点头,摊开纸笔,埋头书写。

%43
但这次写到一半,地灵就笑了一声:“少年郎,你这个蛊虫的秘方我这里也有啊。”

%44
“哦?”方源停下手中的笔,神情有些惊愕。

%45
他不认为,这是地灵撒谎。地灵单纯,说有便是有,没有就是没有。

%46
“你不信的话,可以看看这个。”琅琊地灵伸手虚抓,一道牛皮秘方就被挪移过来,然后放在了桌上。

%47
方源拿到手上一瞧,竟然真是如此。

%48
“看来我记忆中的蛊虫,虽然是在五百年后涌现出来的,但未必是全新的蛊。”

%49
念及于此,方源又笑了笑,对地灵道:“不要紧,这道不行,我还有下一道。”

%50
但这次方源只写了三分之一,地灵就拍手笑道:“你这道蛊虫秘方,我虽然没有,但有个相似的。你拿去看吧。”

%51
说着,他便给了方源一个秘方,相似程度高达九成。

%52
方源顿时明白过来:“我的这道秘方,很有可能就是蛊师在老秘方上改良的。”

%53
“少年郎,你的这种秘方,价值不大。我可不会换的。”地灵道。

%54
方源收敛起心思。

%55
长毛老祖在世时,就喜欢收集秘方,研究秘方。他号称古往炼道第一仙,又活了相当长的一段时间,收集到无数秘方,可以说是集秘方之大成。

%56
而方源自己的那个新时代,还未真正走到最猛烈的时刻,至少还未出现大梦仙尊。方源记忆中的这些蛊虫秘方,和琅琊福地中无数时代积累的底蕴一比较起来。就显得浅薄了。

%57
“你再看看这道。”方源又埋头书写。

%58
但这些秘方,不是琅琊福地本来就有的,就是在古方上稍稍改良过的,入不得地灵的法眼。

%59
方源无奈。

%60
他记忆中,也有一些蛊虫。他可以确信是新蛊。但这些蛊虫关系重大,每一道都代表着庞大的利益。每一种蛊虫,都能改变战局。一旦从琅琊福地流出去,利益流失是小事,关键是推动历史进程,对于方源来讲弊远远大于利。

%61
“地灵,你再看看这道秘方吧。”方源想了想。书写出星门蛊的秘方。

%62
地灵瞧着一眼,就感兴趣。方源越写越多,他的兴趣就越是盎然。

%63
“这蛊虫秘方我没有见过,有些稀奇。有些稀奇。”老爷爷咂着嘴,喜不自禁。

%64
这是五转蛊的秘方,方源因此成功地换来神念蛊。

%65
“少年郎,这星门蛊有意思。居然有跨域传输之效。五大域相互都有隔膜,但它居然巧妙地借助了黑天的力量。这种蛊虫。历来只有洞地蛊,通天蛊最为经典常见。你这星门一出,完全能和这两只蛊相互媲美,平分秋色!这只蛊是你构思出来的吗?”地灵问道。

%66
“当然最强弃少!”方源毫不犹豫地就承认道。

%67
对于冒名顶替了这份名誉,他毫无愧疚之心。

%68
然后,他又恬不知耻地吹嘘起来:“之前的那几道秘方,有一些也是我研发出来的,或者根据秘方改良的。”

%69
此举无疑加深了他和琅琊地灵之间的关系。

%70
“小友,你有炼道天赋,叫我刮目相看啊!”地灵老爷爷不再称呼方源为少年郎,而是改称小友了。

%71
“不过你这星门蛊,也有弊端。需要在夜里牵引星光,才能催动。除此之外,这炼蛊的成功率也低。不过若是再添加几道辅料,却是可以将这成功率提升三成。”

%72
接着,地灵一连说出了几个材料的名字。

%73
方源皱着眉头听着,这些材料他听都没听过。

%74
看来,要不是太古的材料,要不就是极为偏门,极少用到的。

%75
地灵又接着道:“看来这个星门蛊,最佳的搭配还是星萤。有了星萤,就有星光,星门蛊不论何时何地都能运用。”

%76
“什么?”方源闻言,心中砰然一动,忙追问道,“星萤,什么是星萤?”

%77
“星萤你不知道?对了,这种蛊群在太古就稀有,上古年代已经绝迹了。太古九天还在的时候,大部分的星萤都生活在橙天里。”地灵爆出一个秘辛。

%78
方源顿时失望:“既然已经绝迹了,那还谈什么。”

%79
“所谓绝迹,不过只是说的凡间俗世罢了。我最近刚刚在宝黄天的交易中,就看到过一团星萤,好像是万象星君的货。”地灵回忆道。

%80
“真的?”方源双眼骤亮。

%81
他之所以换得通天蛊,又换神念蛊,就是想冒充蛊仙,在宝黄天里进行交易,换取物资,脱离仙鹤门的商贸封锁。

%82
地灵的这番话,让他对星萤的兴趣激增。

%83
方源眼珠子一转,狡猾地笑起来:“地灵,我这里还有很多秘方。但我只换星萤!”

%84
地灵摇头:“这个换不了。”

%85
“怎么?”

%86
“我没有星萤蛊啊。”地灵理所当然的回答。

%87
方源没好气地道:“你没有星萤蛊,不会用通天蛊,在宝黄天买下来吗?”

%88
地灵奇怪地看着方源:“我为什么要买星萤蛊?”

%89
“你不买星萤蛊,怎么换我的秘方?”

%90
地灵摇头,固执地道:“你的秘方,只能换福地中现有的蛊虫。”

%91
方源无语了,地灵死脑筋,这点都转不过弯来。到底不是人,只是地灵,这点变通都没有啊。

%92
最终,方源劝说的嘴唇都快要磨破了,还是不行。

%93
地灵就是认死理。

%94
方源只好作罢,忽然念头又起:“等等,地灵你说只换福地中现有的蛊虫?”

%95
“是的。”

%96
方源舔了舔干燥的嘴唇:“那我用仙蛊的秘方,来换你的天元宝皇莲,成不成?”

\end{this_body}

