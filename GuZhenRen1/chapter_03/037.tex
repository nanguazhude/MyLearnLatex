\newsection{钧力蛊}    %第三十七节:钧力蛊

\begin{this_body}

蛮家坐拥红炎谷,近些年打了几场胜仗,扩张得很厉害。

北原和南疆不同。

南疆是山林地带,易守难攻,长途远征,损耗极大。南疆的家族是徐徐经营,慢慢积累,步伐稳健,血统纯粹。

而北原是大草原,征战频繁。北原的家族,兴盛得快,衰败得也快。很可能几场大胜,能让一个小型家族,扩张成中型家族。让中型家族,膨胀成大型家族。也有可能一场大败,就让大型家族分裂成若干小家族。

蛮家最近吞并了石家,石家的家老石武,成了蛮家的外姓家老。

他是骨道蛊师,就是他买断了市集所有骨竹蛊,据说是要研发新的骨道秘方。

方源暂且将这人记下,又询问店主魂道和力道方面的蛊虫。

“我店中有一转斤力蛊,可增长蛊师一斤之力。每只蛊卖两百二十块元石。”

“有二转十斤之力蛊,增长蛊师十斤力气,单价是六百九十块元石。”

“还有三转一钧之力蛊,一钧就是三十斤,每只蛊价格四千五百五十块元石。”

“虽然没有四转的十钧之力蛊,但贵客若是需要,本店完全可以调货。每只十钧之力蛊,售卖三万六千块元石。”

店主嘴皮子开合间,熟练地报出一大串的价格。

末了,他又添加一句:“当然,这只是斤力蛊,钧力蛊。贵客若想走上古力道,想要兽力蛊,本店也有狼力蛊,可增长一狼之力,耐力持久。又有马力蛊。可增益一马之力,奔跑中最能发挥。”

方源之前,用的就是兽力蛊,走的上古力道路线。斤力蛊,钧力蛊则是最近数百年,北原盛行起来的主流。

如今力道式微,辉煌不再,但也有发展。

做出这个贡献的,就是北原有名的七转蛊仙楚度。号称霸仙。就是他研发出斤力蛊和钧力蛊。甚至一手炼成六转仙蛊“力贯千钧”。

一钧为三十斤。

千钧就是三万斤。

地球上的传说中,斗战胜佛的金箍棒,也不过一万三千五百斤重。二郎山的三尖两刃刀,则是两万五千二百斤。

楚度在三百年前成就蛊仙,他所研炼的秘方广为流传。很快就成为北原的力道主流。

上古力道,通常是兽力蛊,诸如青牛劳力蛊、彪力蛊、龙力蛊等等。因为上古时代的材料,到了如今,有许多稀缺,因此炼蛊成本颇高。

而霸仙楚度的这些秘方,胜在材料普通易于收集。成本低廉,炼蛊的成功可能也比兽力蛊要高出一线。

北原大概是五域中,力道最盛行的地方。北原常年征战,大小战事频繁得很。而力道成本低廉。于低转蛊师身上,提升的战力也比较明显。

很多北原蛊师,都兼修力道。

他们常常在酣战,真元用尽。这个时候就需要动用**作战。力气就很需要了。

地球上有句话,战争是科技提升的催化剂。这句话。放在这个世界,也颇为适用。北原也是新蛊虫层出不穷的地方。

从那个投靠蛮家的石武家老身上,就可管中窥豹,可见北原蛊师研发新蛊的风气。

霸仙后来战死沙场,死于从天庭下界的凤九歌之手。他死后,被后人尊称之为“力道的余晖”,死讯传出,曾令整个北原各个地方的蛊师失声痛哭。

“唉……这世间的英杰之众如夜空的繁星,才俊之多如过江的鲤鲫。五域太大了,一个地球的面积也比不上其中一域。尤其是五百年后的大世,龙蛇起陆,群雄竞逐,老怪们接连出山,新秀层出不穷。无数的英雄、枭雄,不同性情的正道、魔道,相互碰撞,进行生死的激战。实在是无以伦比的精彩和壮阔。”

“我之前悄悄用过推陈蛊,不仅将古铜皮,精铁骨,金钢筋洗去,连同身上剩下的力道兽影,也一并清除。一来这些力道兽影,源自南疆蛊虫,在北原也受到压制,不实用了。二来一旦打出兽影,也是暴露身份的隐患。”

“我现在最大的优势,就是拥有荡魂山,又有前世五百年经验,兼有福地资源,适合走奴道。但奴道也有缺陷,所以要兼修力道,防止斩首战术。我既然顶着常山阴的皮,也要融入北原,不妨就选择钧力蛊吧。”

方源心念迅速闪烁了几下,便下了决心,当场和店主预定了四转的十钧之力蛊。

接下来,他又看魂道蛊虫。

魂道方面的蛊虫,可比力道多太多了。

有鬼火蛊、鬼叫蛊、鬼脸蛊、鬼斧蛊等等,用作攻杀。有鬼笼蛊、鬼手蛊、鬼打墙,用作禁锢迷惑。

防御方面,有鬼卦衣蛊、魂盾蛊等。治疗方面,有鬼气蛊、鬼泣蛊等。侦察方面,有鬼眼蛊。移动方面,有魂飞蛊、神出鬼没蛊。

还有鬼兵蛊、鬼无常、气游鬼、九子鬼母蛊等涉及奴道。

这才是大道气象。

力道已经式微,而魂道经久不衰,从此处便可轻易得见。

力道的蛊虫,只有斤力蛊、钧力蛊、兽力蛊这些,大多用作攻击。防御、侦察、辅助等方面,少得可怜。难以构成完整的一套纯粹的力道蛊。

而魂道蛊虫众多,涉及各个方面。构成一套,更能相互增益,达到更佳的效果。

想那霸仙楚度,纵然是力道蛊仙,但却用其他蛊虫进行防御、侦察、存储等等。

方源抱着目的而来,琳琅满目的魂道蛊虫被他撇开,只重点查看凝练魂魄之用的蛊。

四转的蛊虫没有。但三转的一大堆。

有神魂蛊、龙魂蛊、冰魂蛊、梦魂蛊、月魂蛊、将魂蛊、怨魂蛊、诗魂蛊、马魂蛊、英魂蛊、气魄蛊、体魄蛊、云魄蛊、风魄蛊、虎魄蛊种种。

这些蛊虫,都能凝魂。且各有不同妙处。

比如凝练魂魄,形成冰魂,蛊师今后运用冰类蛊虫,效果就会有不小的增益。云魄、风魄同样有类似效果。

若用龙魂蛊凝成了龙魂,将来运使龙力蛊,或者龙鳞蛊、龙行蛊等等,皆有裨益。

方源挑选了一下,见没有更适合自己的,便选择了计划中的狼魂蛊。

自古奴魂不分家。

奴道最初就是从魂道方面衍生的分支,后来蛊师们结合了太古智道,令奴道真正的独立出来。

驭兽蛊、奴隶蛊,就是对魂魄上的控制,针对心智的主宰。

若是方源凝练成狼魂,对他奴役狼群也多有帮助。

三转的狼魂蛊,一只售价七千七百枚元石。方源大采购,一下子扫空店家的存货,将八只狼魂蛊都买下。

狼魂蛊的效用可以叠加,一只三转的狼魂蛊还不足以凝练方源的百人魂。

野生的蛊虫有着自己的意志,难以炼化。但方源将买来的蛊虫,都是已经被别人炼化过的。当场完成交接后,这些蛊虫都成了方源所有。

他将这些蛊,都收入空窍之中。离开这个店面,又去了其他商铺。

重地是二转、三转的驭狼蛊,以及三转狼魂蛊。

一百万的元石并不多,但方源要买的蛊虫,都是二三转的大众货色。

他倒是也想买四转的蛊。

不过这市集的规模还是小了,四转蛊稀少,又都不是他想要的。

连续转了三天,方源就成了市集中的名人。哪家商铺都知道,有这么一位四转高手,在大肆采购蛊虫。

到了第五天,方源大手大脚,花去了五六十万元石。

期间,他也会去逛逛市集中的牧场。

牧场上,贩卖许多的牲畜,卖得最多的是大胃马。这种马哪怕是凡人,都十分需要。而买下驼狼这种战斗用的坐骑的,都是蛊师。

也有许多蛊师,活捉了许多野兽来卖。诸如野牛、野马、鹰隼等等。而草原中,最常见的狼,自然也在其中。

方源左右问价,货比三家,逐渐看中了一群厚背狼。

方源手中有毒须狼群,风狼群。毒须狼群难以补充,又在白天战力削弱,注定要被淘汰。风狼群长于速度,而厚背狼则皮糙肉厚。若能补充进来,和风狼群搭配,一正一奇,才算是有个狼群的雏形。

但是到了第九天,方源意外地发现,竟然在一批水狼中,藏着一头幼年的异兽。

他当即不动声色,将这群水狼买下,讨了一个不大不小的便宜。

市集总共持续了十三天。

方源买下了一千多只水狼之后,又买了许多炼蛊用的各种材料,以及喂养蛊虫用的诸多食材,便在第十天离开市集。

这番采购,他手中元石只剩下三千多枚。

花钱速度如流水,叫葛光看着咂舌,暗暗佩服“常山阴”,果然英雄气概超越常人。

方源跟着葛家部族,又回到原先的地方驻扎下来。

他潜心修行,一边用狼魂蛊凝练自身魂魄,一边又进行炼蛊。

失败了两次之后,他成功地将三转金背狼皮蛊,炼成四转的天青狼皮蛊。

天青狼皮蛊用作防御,虽然是大众货色,但总算弥补了方源在防御上的不足。

这天,他凝魂完毕,暂停修行,门外传来一阵哭声。

“何事发生?”他推开房门,问守卫的蛊师。

蛊师黯然地道:“我们的大小姐葛谣死了。搜寻队伍在腐毒草原上,发现了她的衣服碎片。她被毒须狼群杀了!”

\end{this_body}

