\newsection{再探真阳楼}    %第一百七十节 再探真阳楼

\begin{this_body}

%1
方源再度进入中枢室。

%2
顾名思义,中枢室乃是八十八角真阳楼的中枢要地。

%3
密室成圆形,闪烁星辉的墙壁围绕一圈,中央立着一张白玉圆桌。圆桌上堆砌着精致模型,类似沙盘,正是王庭福地的全貌。

%4
不仅山川河流,中央的圣宫,甚至福地中的各个小塔楼都清晰可见。

%5
“又来到这里了。”脑海中,墨瑶意志发出一声感慨。

%6
方源没有搭理她,只是将视线投向白玉圆桌。

%7
自进入八十八角真阳楼以来,他先是利用来客令,达到上等通关,进入秘藏阁。又炼化来客止步碑,寻找到琉璃楼主令。再利用琉璃楼主令,进入到中枢室。

%8
但因为地丘传承的缘故,他选择暂时中止对八十八角真阳楼的攻略,炼成了招灾仙蛊。

%9
总的来说,地丘传承有利有弊。

%10
虽然方源得到了仙蛊,但其效果特殊,损己利人。

%11
又被墨瑶意志潜入脑海,尾大不掉,内患重重。但不得不说,墨瑶意志这些天来对方源的提点,令他收获匪浅。不提炼道心得,单单六臂天尸王、墨化这些杀招,改良的地魁尸蛊蛊方,就大有价值!

%12
除此之外,还有一个巨大好处。

%13
那就是对方源图谋八十八角真阳楼,大有用处。

%14
毕竟,墨瑶乃是炼道宗师,对八十八角真阳楼研究甚深,当年潜入王庭福地,是制造漏洞,立下地丘传承的传奇人物!

%15
方源有前世记忆,又有中洲蛊仙攻破八十八角真阳楼的影像,再有来自琅琊地灵的第一手珍贵资料,现在更增添墨瑶这个大帮手。

%16
此次前来,他信心十足。

%17
但旋即,他目光微微一顿,发出一声惊咦。

%18
他清楚地记得。原先圆桌沙盘上,是覆盖着一层粘稠黑液。黑液形成漏洞漩涡,向沙盘中的一处破洞缓缓注入。

%19
这处破洞,不是别处,正是那地丘传承所在。

%20
但现在,沙盘上墨汁不见踪影,照应地丘的那处地点,也恢复完全,不见丝毫破漏。

%21
如此情景,就像是对方源信心的一记重拳打击。

%22
他心中一动。正有所猜测时。脑海中墨瑶意志已知他意。轻笑起来:“有得必有失。小弟,你已经拿取了地丘传承,以八十八角真阳楼的威能,必定已经将之前的那处漏洞还原恢复。地丘上破洞不在,应当重新矗立了小塔楼。”

%23
“没有了漏洞,我该怎样将琉璃楼主令,炼成一角楼主令呢?”方源求教道。

%24
“没有漏洞,就打造出新的漏洞便可。”墨瑶傲然一笑,“为什么我要传授你炼道杀招墨化?你应该已经猜测到了,没有错,当初你见到的,覆盖沙盘的墨液。正是墨化杀招所致。”

%25
“没有漏洞,就打出漏洞?”墨瑶的话,尽显炼道宗师的气度,让方源眉头微挑。

%26
话虽说的容易,但怎么打出漏洞?

%27
至少炼道大师级的方源。没有这个能力。

%28
墨瑶接着道:“漏洞也不是随意就能打的。胡乱选取,只会惊醒沉眠中的巨阳意志。意志一醒,我们都要死无葬身之地。好在八十八角真阳楼历经沧桑岁月,曾经的完美已经不再,时光凿琢出各种瑕疵。当初我选取土丘,就是因为瑕疵最大。”

%29
“原来如此。”方源点头,心中暗喜。

%30
墨瑶意志,宛若无源之水,若要寻找瑕疵破绽,必然要进行大量的思索,这就带给她巨大的消耗。

%31
但墨瑶却没有遂了方源的心意,而是道:“小弟,你按照我的指点,用心神探入沙盘,我来告诉你八十八角真阳楼的运转奥秘,帮助你寻觅到破绽瑕疵。”

%32
“好的。”方源目光一闪。

%33
墨瑶此举,也并未出乎他的意料。

%34
她爱惜自身,不自己思索也没有关系。方源正好趁此良机,偷师学习,增强对八十八角真阳楼的理解。

%35
八十八角真阳楼奥妙非凡,方源将心神探入沙盘,宛若小舟置于汪洋,只感觉浩瀚无边,深不可测。一角一边,都值得自己深思借鉴。

%36
当下,他深感渺小,叹为观止。

%37
按照墨瑶的指点,方源前后找到五十四处瑕疵,其中有十三处瑕疵较大,形成破绽,可以和当初的地丘那处媲美。

%38
这个结果,令墨瑶大为感慨:“时光匆匆,已经过去这么多年了。就算是八十八角真阳楼,也难逃光阴长河的消磨啊。想当初,我找到的瑕疵不过三十八处,破绽仅有六处。”

%39
顿了一顿,她便又继续指点方源:“小弟,接下来,你就可以利用杀招墨化,炼化这几处破绽。”

%40
方源依言而行,心念频动,从空窍中飞出八百多只蛊虫来。

%41
这些蛊虫,从一转到五转皆有,涉及金、木、水、土、律、魂、血诸道,其中又以暗道为主。

%42
这些虽然都是凡蛊,但是其中十几只蛊虫,有的在中古时期就已经绝迹,有的甚至要追溯到远古时期。因此极其稀缺,现今在五域彻底绝迹,只有个别的蛊仙有所收藏。

%43
方源筹集这些蛊虫,花费不菲,至少投入了一块半的仙元石。

%44
仙元石价值极高,可直接补充蛊仙的仙元,帮助蛊仙修行,同时更是蛊仙交易时的硬通货币。

%45
就算是方源前世全盛时期,也不过积攒了六十多块仙元石。

%46
今生,他火中取栗,获得狐仙福地以来,最高纪录也就十二块仙元石。

%47
现今,他的手中只剩下两块。

%48
没办法,太多的地方需要动用仙元石了。

%49
购买狼群,收集智道情报,收购蛊虫等等种种,都需要仙元石。

%50
近千只蛊虫在空中飞舞,宛若蜂群,又似花雨。

%51
方源凝神屏息,指挥调度,终于时机成熟,轻喝一声:“墨化!”

%52
蛊虫化为一团乌云,云中雨滴坠下。啪啪啪打在沙盘上,落为颗颗墨滴。

%53
墨滴积少成多,渐渐覆盖沙盘表面。

%54
方源心神高度集中某处破绽,驾驭墨液向破绽处冲去。

%55
五转巅峰的空窍中,真元急速减少,墨液消耗大半,这才艰难地冲破一层隐约阻碍,形成漏洞漩涡,缓缓注入沙盘的破洞。

%56
见此情景,墨瑶便道:“将你手中的琉璃楼主令。投放进去吧。”

%57
方源依言而行。投入琉璃楼主令。

%58
琉璃楼主令沉入墨液漩涡当中。不见踪影。墨液消耗的速度陡然加快,大约半个时辰之后,乌云首先消散。两个时辰之后,墨液消失殆尽。一枚全新的楼主令,从破绽处悠悠飞出。

%59
方源一把握住。

%60
楼主令已经变了模样,在之前的基础上,在其边缘形成一处尖锐的凸起,仿佛独角。

%61
“一角楼主令,果然形象。”方源自语道。

%62
“在八十八角真阳楼里,巨阳仙尊后人打通一层之后,楼主令就会晋升成一角楼主令,对该层掌握控度。同时。能进入秘藏阁,取走其中的一件珍宝,无须换取。若是能打通十层,使楼主令晋升成十角楼主令,便能从秘藏阁中取走一道巨阳真传。”墨瑶介绍道。

%63
“巨阳真传?”方源怦然心动。

%64
墨瑶接着道:“不错。巨阳仙尊为子孙谋算。布置了王庭争斗的规矩,设立了八十八角真阳楼,并在楼中留下了八十八道真传。仙尊真传,自然非同小可。但历代王庭胜者,能打通十层的少之又少。我生前时,八十八道真传,还有五十三道遗留下来。不知道现在还有多少了。”

%65
“呵呵呵,说到这里,我也要羡慕你的运气了。我们发现的破绽,总共有十三处。每处破绽,墨化之后,便可晋升一角。如果你利用全了,就能得到十三角楼主令,完全可以取走其中一道巨阳真传!唉,不像我当初,只有六处破绽。”

%66
方源听了这话,火热的心就像遭了一盆冷水,他苦笑起来:“利用全部破绽?不瞒你说,我手中的仙元石只剩下两块,只能再支撑一次墨化杀招。”

%67
“是这样?”墨瑶目光一闪。

%68
方源在试探她的同时,她也在试探方源。

%69
仙元石,是衡量蛊仙的重要标准之一。方源虽然还是凡人,但坐拥狐仙福地,在墨瑶的心目中已经将其当做半个蛊仙看待。

%70
“原来这小子只有两块仙元石了?他这话是否是真话呢……应该是真的。巨阳真传的诱惑力,不是常人能够抵挡得。他是炼道大师,自然知道炼化的破绽越多,掌握更高的楼主令,便越方便攻略八十八角真阳楼。最关键的是,我潜伏在他的脑海当中,他就算想背着我偷偷炼化,也是不可能的事情。”

%71
“接下来,他应该会向背后的势力求援。哼,这臭小子隐藏得再深,到那时,也要露出底细,让我窥得一二。”

%72
一边这样想着,一边墨瑶和颜悦色地向方源出主意:“也不要紧。楼主令之间可以相互吞并。只要你吞并了此届盟主黑楼兰手中的那枚楼主令,说不定也能凑齐十角。”

%73
不愧是墨瑶,知道的秘辛真的不少。

%74
方源目光一凝,语调一提,故意道:“你是想让我对黑楼兰动手?你是想把我往死路上推么?黑楼兰虽然容易对付,但他的背后,可是黑家,是超级势力,有数位蛊仙罩着他!”

%75
“嘿,你这小子胆大包天,既然能伪装身份,潜伏到王庭,还怕区区一个黑楼兰?再说,我也不是要你一定杀了他,只是取得他手中的楼主令罢了。”

%76
方源目光一转:“黑楼兰不能杀,杀了他就是捅了马蜂窝,黑家我现在可惹不起。但那枚楼主令在他手上,他肯定珍惜若命。你说,我该怎么办?”

%77
“‘现在惹不起’,这是说,将来就不一定了么?这小子的野心真是不小……”墨瑶敏锐地察觉到话里的关键字。

%78
对于方源提出来的难题,她却不愿过多思考,在方源的脑海中,她摊开双手:“怎样取得他手中的楼主令,这就是你的事情了。”

%79
更新快纯文字

\end{this_body}

