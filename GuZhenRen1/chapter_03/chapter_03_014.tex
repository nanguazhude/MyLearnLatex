\newsection{方正的痛苦}    %第十四节:方正的痛苦

\begin{this_body}

%1
高耸入云的天梯山,高达百万丈。

%2
它位居中洲正中央,是传承之地,圣贤之山。古时,更是仙凡之梯,可以上达天庭。

%3
仙鹤门的一群精英弟子,如今站在天梯山的山脚,已经等待了有半个时辰。

%4
“我们还要再等多久呢?”

%5
“这个方源未免也太端架子了吧?”

%6
“嘘,小声点。他可是古月方正的亲哥哥,如今更是狐仙福地的主人!”

%7
“说起来,方正的这个哥哥真是太厉害了,居然连凤金煌、萧七星、应生机这些人都战胜了。”

%8
“这有什么?如果我背后有一位门派的太上长老支持我,替我催动定仙游蛊,我也能夺得福地啊。”

%9
“还是我派的长老们有谋算啊。表面上派出方正吸引火力,实际上真正的杀手锏是他的哥哥方源!”

%10
……

%11
仙鹤门为了将这件事情做真实,对门派中的弟子们都撒了谎。仙鹤门的弟子们,这才知道,原来自家门派当中还有一位古月方源的存在。

%12
这三个月来,古月方源成为了仙鹤门的弟子讨论最多的人物。他低调神秘,勾引众人的好奇心。不鸣则已一鸣惊人,替仙鹤门夺得狐仙福地,更给仙鹤门长脸,叫其他弟子脸上都有光。

%13
身后的议论声,不断地传入方正的耳中。

%14
方正站在人群的最前方,目光沉郁,仰望着天梯山。

%15
这些天来。他如行尸走肉,都不知道自己是怎么过来的。

%16
方正离开青茅山时。他就立志报仇,要为死去的族人们讨还公道。

%17
他背负着血海深仇,复仇的强大意念,支撑着他,令他刻苦修行。他比其他所有的弟子都要努力,几乎没有一丝一毫的懈怠。

%18
他曾经无数次幻想,当他找上方源的情形——他将方源打倒,让他跪在地上。就在青茅山上,为他所做的一切忏悔。泉下有知的族人们,也会瞑目了。

%19
所以当他在荡魂山上攀登时,无数次想放弃,但又无数次坚持下来。

%20
每当他想到方源,他的心中总是会产生一股强大的力量,支撑着他。让他继续攀登。

%21
他矢志要夺得狐仙传承,不仅是因为他不想辜负师傅、门派的期望,更是狐仙福地让他复仇的可能,增大了无数倍。

%22
然而,他万万没有想到的是,命运的打击来的如此突然。如此沉重。

%23
古月方源,他的亲生哥哥,无数次噩梦的主角,竟然突然出现在山顶!然后在众目睽睽之下,夺取传承。就连蛊仙都奈何不得!

%24
失败的方正,回到门派中。

%25
震惊!

%26
痛苦!

%27
迷茫!

%28
恐惧!

%29
他知道门派的谎言。他是知情者,知道事情的真相。但正是因为如此,他心中的阴影反而加倍扩大。

%30
这个阴影,就是至小时候起,方源就施加在他身上的。

%31
为什么哥哥那样的聪明?而我却蠢笨!

%32
为什么我那么的努力修行,却仍旧败在方源的手中?

%33
为什么在南疆时如此,在中洲时又是这样?!

%34
“难道我古月方正这一生,都要活在他的阴影之下,永远都战胜不了他吗?!”每当方正这样想的时候,内心深处就会升腾出千万分的不甘心,让他更有动力去刻苦修行。

%35
但是这一次却不一样。

%36
不一样了。

%37
一想到来之前,门派交给他的任务,方正就忍不住浑身微微的颤抖。

%38
福地在方源的掌握当中,门派为了得到狐仙福地,招揽方源。只要他愿意交出狐仙福地,他就是仙鹤门的长老。

%39
中洲门派中,由下到上,分外门弟子、内门弟子、精英弟子、真传弟子,一层层晋升上去。

%40
而在弟子之上,有长老,修为一般是四转,掌管门派各方职权。长老之上,则是掌门,修为至少是五转中阶,总领事务。

%41
掌门之上,就是太上长老。

%42
这些太上长老,都是蛊仙,平时神龙见首不见尾,都在潜修。等到门派存亡,或者重大事件发生时,他们就会踊跃而出,让世人知道仙鹤门为十大派之一的深厚底蕴!

%43
“我自加入仙鹤门,这些年来刻苦修行,从外门弟子,晋升内门。又从内门,擢升到精英。门派考核时,艰难地夺得精英弟子之首。而他方源只要开口一句话,轻轻巧巧就能成为门派长老。任何的弟子见到他,都要躬身行礼!”

%44
方正每每想到这里,心中都充满了无比的痛楚。

%45
如果方源真的成为了长老,那他方正今后见到这个大仇人,反而要躬身行礼!这样的人生还有什么趣味?还有什么意义?

%46
“师傅,难道我所做的一切努力,都是毫无意义的吗?”此时此刻,方正站在天梯山脚下,等待着方源的召见,不可避免地陷入到深深的自我怀疑当中。

%47
天鹤上人旋即安慰劝解道:“方正,你要端正心态。仙鹤门为了狐仙传承,牺牲了很多,其中甚至包括一只仙蛊!为了门派,我们应该从大局出发,暂时地放下个人的恩怨。方正,你要明白,是仙鹤门培养了你,如今门派需要你做出一些牺牲。你要有大局观啊,不能忘恩负义!”

%48
这样说着,天鹤上人却在心中叹息。

%49
他对方正十分了解,因此心中更加担忧。

%50
一直以来,复仇的意念,像是支柱支撑着方正前行,已经成为了他修行的执念。

%51
结果现在,门派的命令却要瓦解掉方正的这股执念,这比任何的伤势,都要致命。很有可能,方正就被这样打击,从此颓废不堪,一蹶不振。

%52
“但这有什么办法呢?要知道那可是一块福地啊,同时更有荡魂山这样的秘禁之地!山上的胆石,供给门派的弟子,整个门派的实力都将因此暴涨。除此之外,方源的手中还有血颅蛊,甚至还有仙蛊定仙游!这些东西的价值,实在是太大太大了,怎么可能是一个区区的精英弟子能够媲美的?”

%53
天鹤上人心中哀叹,嘴上则对方正道:“我的好徒儿,你要按捺住自己的复仇之心。小不忍则乱大谋,你就把它当做是一场心性的磨砺吧!见到你哥时,千万不要出手。在福地中,你万万不是你哥的对手。”

%54
说到这里,天鹤上人不禁又想起,出发时鹤风扬对他的嘱托——

%55
“我知道方源、方正这两兄弟的恩怨。如果有必要,牺牲掉方正也无不可。你可代替方正商谈!”

%56
鹤风扬身上的压力也很大,所有的太上长老们都在盯着这件事情。

%57
“师傅你是说,让我把这个当做磨砺?我,我尽量吧。”方正一双拳头松了又紧,紧了又松,显露出内心的挣扎、痛苦、郁愤。

%58
有人立志报仇,苦练功成,结果却发现仇人死了。这是痛苦。

%59
有人立志报仇,苦练功成,找上仇人,却打不过,仇人还生活得很美好。这更是痛苦。

%60
有人立志报仇,苦练功成,不仅打不过仇人,还得换一种和善的态度商谈,希望仇人成为自己的上司。这是痛苦中的痛苦!

%61
“嘿嘿,方正,你也不要过于纠结。方源的日子也不好过,福地有地灾。地灾之威,你是想象不到的。你哥哥就算有仙蛊,也终究是凡人。待会儿,他就会深刻地认识到地灾的恐怖了。到那时,地灾漏洞百出,必定损失惨重。你此行,大有成功的可能。”天鹤上人又安慰他。

%62
方正听了这番话,心情这才稍微轻松了一点。

%63
“地灾开始了。”鹤风扬轻声呢喃,他隐居幕后,一是为了保护这群精英弟子,二是防备其他蛊师的图谋不轨,三是关键时刻,若方源顶不住地灾,他就要出手相助。

%64
此时,他盯着狐仙福地隐藏在天梯山上的地点,察觉到地灾特有的毁灭气息。

%65
很快,他的嘴角就翘起,因为天梯山上,出现了异象。

%66
一块又一块,草原的虚影,出现在天梯山上。仿佛是烟云,也仿佛是光雾,虚幻并不真实。

%67
山上哪里能有草原?

%68
这就是福地的漏洞,还是较大的漏洞,能令外界之人窥视福地内部的景象。

%69
这样的漏洞,只能塞进去一些蛊虫进去。距离蛊师进出,还颇有些距离。

%70
那边天鹤上人已经叫道:“漏洞出来了,快,将电文纸鹤蛊飞进去。”

%71
方正咬了咬牙,在身后众人的注视下,灌注真元,催动蛊虫。

%72
电文纸鹤蛊如道闪电,飞射进漏洞中。

%73
但旋即,草原的虚影化为一团元气,消散于天地当中。电文纸鹤蛊飞了两圈,只好再次回转到方正的手上。

%74
“这是方源,自动割舍了福地,完全放弃掉了!看来他也是担心,漏洞形成通道,让外界蛊师进来啊。”鹤风扬微微吃了一惊,旋即又冷笑起来,“让你割,我看你能割弃到什么程度。每割弃一片福地,就是割自己的心头肉。”

%75
然而片刻之后,鹤风扬的面色彻底变了。

%76
“居然还在割弃福地?恐怕已经有数千亩了!倒是有些魄力,难怪能冒险抢了传承。”

%77
又过了片刻,鹤风扬脸色相当难看。

%78
“看来这次地灾,相当的眼中。不过他究竟还要割多少?已经数万亩了。这个败家子!”(未完待续。如果您喜欢这部作品,欢迎您来起点投推荐票、月票,您的支持,就是我最大的动力。手机用户请到阅读。)

\end{this_body}

