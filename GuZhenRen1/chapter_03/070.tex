\newsection{飞跃的修为}    %第七十节:飞跃的修为

\begin{this_body}

哗哗哗……

空窍中,九成真元海翻动起滔天的浪花,真金色彩绚烂耀眼,照耀在四周的窍壁上。

这窍壁也非寻常,晶莹剔透,莹润饱满,蕴藏着一股生机意味。这是四转巅峰境界的晶膜。

真金海面上,不断有浪潮喷涌,掀起滔天的巨浪。然后一波又一波地,冲刷在晶膜之上。

起先,晶膜如海边礁石,岿然不动。但是渐渐的,晶膜不堪重负,被真元狼潮冲刷出丝丝裂痕。

随着时间的推移,裂痕越来越多,越来越大。

真元海面徐徐下降,晶膜上也变得裂痕满布,脆弱不堪。但距离冲破却又差了那么一丝。

方源盘坐在床榻上,双目紧闭,心神几乎全数集中在空窍当中。

他有九成资质,隶属甲等,又积累深厚,此次冲刺五转境界,只靠自身底蕴便可行事。

此时真元海只剩下两成,海面波涛渐渐衰减,似乎有平静下来的趋势。

但陡然间,酝酿之后的真金海面,猛地爆发出一股冲天的浪潮。

这浪潮磅礴凶猛,超越先前的任何一次浪潮,简直像是海啸一般,轰然升腾,然后狠狠地砸在窍壁之上。

砰!

一声轻响,窍壁终于不堪重负,被这股巨大的浪潮砸得破碎。

大量的水晶窍壁碎片,宛若破碎的冰山,落到海水当中,化为精粹,慢慢消融不见。

片刻之后,晶壁上的碎片统统落尽,只余下一片光膜,包裹着四周上下。

光膜熠熠生辉,气息强盛,运超四转巅峰晶膜,正是货真价实五转初阶的气象!

随之而生的。是一丝淡紫色的真元,在海底产生。

蛊师一转为青铜真元,二转是赤铁真元,三转白银真元,四转黄金真元,到了五转便是紫晶真元了。

虽是一丝淡紫,但却是质变的差距。

从今以后。方源的真元就是淡紫真元!

“重生以来,四处辗转、颠沛流离,终于成了五转蛊师了。”荡魂行宫中,方源缓缓地睁开双眼,欣慰地一叹。

五转蛊师,在这凡尘中。绝对是一方霸主,真正的顶峰境界。能够修行到这一层次的人,何止万不存一啊。

若换做其他人,此时恐怕早已经欣喜若狂快。但方源早在前世,就有过一次经历,此时更多的不是狂喜,而是欣慰和展望。

“相比较前世的蹉跎。今生如此年轻,就进入五转,实乃佳绩了。”

方源如今不过三十小几,这样的年龄修行到五转,绝对是五大域中最顶尖的天才表现。

“不过五转境界,却不是我的终点,而是一个起点。接下来就是再次晋升成蛊仙!”

换做那些天才蛊师,谈及冲击蛊仙。也要心存犹疑。但方源有着前世经验,对今后晋升蛊仙,信心很足。

“晋升到蛊仙之后,春秋蝉的死亡危机就能彻底的解决了。”

方源再看春秋蝉。

自上次重生之后,春秋蝉就陷入沉眠,不断汲取光阴之河的流水吞食,自行恢复着。

此时。它的状态已经恢复了大半,一股淡淡的威压,充斥方源的整个空窍。

一直以来,春秋蝉就是一柄悬在方源脖颈上的铡刀。随着时间的推移。这个巨大的铡刀就会越接近方源的颈椎。

春秋蝉带给方源巨大的死亡威胁,逼迫着他不断冒险,尽一切可能冲击修为。

只有达到六转蛊仙的境界,方源的空窍,才能承载得住春秋蝉。

到那时,方源就能获得更多的自由和从容,再不像现在这般紧迫压抑了。

空窍中淡紫真元,在不断产生,沉淀在残存的黄金真元之下。

方源抽回心神,缓缓站起,走出了这处密室,来到另一处房间。

这房间中,摆着几个大坛。

坛子中,装满了绿色的汁液,里面各有一只紫晶舍利蛊沉浮着。

这些紫晶舍利蛊,共有六颗。绝大多数都是方源,或是小狐仙这些天,从宝黄天中收购而得。

紫晶舍利蛊能够拔升五转蛊师,足足一层的修为境界!方源为了这些紫晶舍利蛊,也花费不菲。

方源只需三颗紫晶舍利蛊,便能将修为提升到巅峰境界。但此时,他放眼望去,却见这坛水各异。

一坛中绿水浑浊,一坛中绿黄交杂,还有两坛中的紫晶舍利蛊上,都生长出紫黑色的细小绒毛。

剩下的两坛,却是绿水清澈,毫无变化。

方源冷哼一声,这六颗紫晶舍利蛊中,大部分都被人或多或少地动过手脚。

方源通过领先五百年的经验,测出这些舍利蛊身上的猫腻。

这六坛绿水中的紫晶舍利蛊中,只有两颗能用。方源当即从绿水清澈的那两坛中,取出紫晶舍利蛊用了。

用了两颗舍利蛊,方源的修为立即暴涨到五转高阶的境地。

方源再取出那两只浑身长满细小绒毛的紫晶舍利蛊,也接连用在自身之上。

两颗舍利蛊合力,终于将其修为擢升到五转巅峰!

短短片刻功夫,方源就从五转初阶,成为五转巅峰的蛊师,完成了一个巨大的跨越。且舍利蛊玄奇,令他毫无任何根基不稳的迹象。

蛊师修行,极为依赖资源。只要资源充足,修为便能一日千里。

理论上,只要准备充足的舍利蛊,蛊师的修为就能呈现飞跃式的进步。但实际上,却极少有蛊师这么做。

一来,舍利蛊只能推升蛊师的小境界,每转的大境界都需要蛊师自己突破。用多了舍利蛊,修为得来的太容易,反而令寻常的蛊师认知偏颇,突破大境界更难。

二来,蛊师修行并非境界高,就能代表一切。要在这残酷的世界中生存,还需要算计、智慧、机缘、经验、训练。暴涨的力量,若是没有足够的心智控制,恰如挥舞大刀的婴孩,伤人伤己。

三来,舍利蛊价值颇高,鲜有蛊师有这样的条件,能够进行如此奢侈的修行。就算是那些蛊仙的后代子孙,有这样的条件,他们的长辈反而更不会这样拔苗助长。

最后,还有更重要的一点,那就是资质。资质不足,再多的舍利蛊也没有用处。方源能够顺利地晋升五转,经验老道是一方面,甲等资质带给他的帮助很大。

这样连续吞食舍利蛊,令修为暴涨的修行方式,恐怕也只有方源能够这么做了。

换做其他的五转蛊师,哪怕再天才,例如商家家族也得按部就班地一步步来。毕竟蛊师修行,还得进行大量的蛊虫训练。修为再高,不能转换成战斗力,也不过是个人形肉靶而已。

前世经历是方源的巨大财富,让他能够顺利地掌控住这股爆发的力量。

算算时间,进行了这番修为的冲刺,不知不觉间已经过了一夜。方源走出荡魂行宫,来到荡魂山上。

果然如他所料,山石上又长出了许许多多的胆识蛊。

他几步走过去,蹲下身来,信手捏碎一只,却只得一滩黄泥。

自从荡魂山中了和稀泥仙蛊之后,渐渐衰死,结出来的胆识蛊中有一部分,都成了黄色泥浆。

方源面色不变,早料到如此,又接连捏碎七八只胆识蛊。

但在这个过程中,他只得了两只完好的胆识蛊。通过蛊虫的力量,他的狼人魂再次得到些微增强。

“和稀泥仙蛊的力量,已经侵蚀得更深了。记得当初,十颗胆识蛊中至少有四五颗完好的。如今却只剩下三四颗的样子。”方源叹息一声。

这些黄泥,虽然都是和稀泥,可以拿到宝黄天中售卖。但比之原本的胆识蛊,价值云壤之别,可以说是巨大损失。

不过好在,这漫山遍野的胆识蛊都是方源一人之物。

虽然胆识蛊中十只六七,都是废蛊,但是依靠庞大的基数,仍旧带给方源魂魄跨越式的大进步。

“如今我已是五转巅峰,下一步就是冲击蛊仙境界。不过此事艰难,还须得大量准备,徐徐图之。”方源暗暗谋算。

有了五转巅峰的空窍,春秋蝉的压力便大大缓解了。

当务之急,还是救治荡魂山。

不能看着荡魂山这就这样灭亡了。保住荡魂山,对于方源接下来的修行,帮助极大,哪怕方源晋升蛊仙之后,都有作用。

将山上的胆识蛊采摘完毕,方源又进入荡魂行宫里,和小狐仙合作,催动了通天蛊。

宝黄天中,一波波的神念相互纠缠。

刚打开通天蛊,就有一大股的神念,向着小狐仙喷涌而出。都是关于和稀泥的开价。

方源将大量的和稀泥抛售到宝黄天中,引起许多蛊仙的兴趣。

运用这些和稀泥,依靠仙蛊秘方,就能炼成和稀泥仙蛊。虽然此蛊只能使用一次,但却是货真价实的仙蛊。在仙蛊普遍稀缺的大环境下,蛊仙们对方源的和稀泥趋之若鹜。

对于这些和稀泥,方源的开价就是和稀泥仙蛊炼制的秘方。

起初,这些蛊仙只拿出破烂无比的残方,宝光一丈都不到。但这些天过去,这些蛊仙拿出来的秘方,已有三四丈的宝光了。

方源依旧不做理会,他还有大量的和稀泥可以抛售。现在出现的这些秘方,也不入他的法眼。(未完待续。。。)

------------

\end{this_body}

