\newsection{不永生都是屎}    %第五十九节:不永生都是屎

\begin{this_body}

如果没有外力作用,那么单靠他自己的力量,只能不断缓慢积累,等到时机成熟,再做突破。

但他因为有了春秋蝉,偏偏最缺的就是时间。同时若浪费时间,赶不上一些机缘,重生的优势也就没有了。

“这三次机会,的确相当珍贵。但一味地留到后面再用,看似充分利用了这三次机会,但却延误了我发展的脚步啊。具体情况要具体分析,盗天魔尊我不能一味地去效仿,马鸿运的选择也太浪费这三次机会了。”

深思熟虑了好一会儿,方源这才下了决定。

“地灵,我考虑好了。”方源缓缓地道,“这一次就拜托你给我炼制星门蛊吧。至于余下的两次机会,留着以后再说。”

“真的要我炼制星门蛊吗?你可知道,你现在放弃的是一个能炼成仙蛊的宝贵机会。或许,你可以赌一赌。就用刚刚的那道血神子的秘方。”地灵反而又劝说方源。

他对秘方非常喜爱,血神子的秘方虽然是残篇,但他仍旧想要收藏。

方源摇摇头,当一件事情决定下来,他就不会再有犹豫:“不,就炼星门蛊。”

“也罢。这个世界上,没有最强的蛊虫,只有最合适自己的蛊虫。仙蛊你用不了,但看来这星门蛊对你作用非凡穿越随我心。我这就来给你炼成这蛊。”

地灵老爷爷说着,念头一动,将五个蛊仙凭空挪移过来。

这五个蛊仙,一个丑陋不堪,一个面目红润,一个青衣,一个黄裳。一个粉裙。正是鬼王一行人。

他们进攻琅琊福地,相互抛出各自的青提仙元,消耗琅琊福地的白荔仙元。

但结果是地灵故意示弱,结结实实地摆了他们一道。

琅琊福地中有八转的天元宝皇莲,白荔仙元从未缺乏过。鬼王等人手中没有一只仙蛊,地灵将他们诱骗到福地中后。

等到这五人冲到云阁面前,地灵便禁锢了他们一身的五转蛊。

鬼王等人意识到不妙,连忙将所有的仙元都使用出来,结果仍旧比不过琅琊地灵。

地灵连一头荒兽都没有调用。就将这五仙生擒活捉。

五仙能修行到这一步,都很识时务,立即选择了保命,臣服在地灵的手下。

发觉自己被挪移过来后,他们很快就反应过来。纷纷向地灵行礼,并且齐声道:“属下拜见琅琊地灵大人!”

“嗯……”地灵抚摸着雪白的胡须,恢复到方源起初见他时那副世外高人的神情状态。

“这位是盗天魔尊的继承人常山阴,你们也来拜见一下。”地灵介绍道。

“盗天魔尊的继承人?!”五仙面面相觑,都看到彼此眼中的震惊。

原本心中对方源凡人身份的轻视,瞬间消失得干干净净。盗天魔尊是什么样的人?那可是九转蛊仙!

自太古,到远古、上古、中古、近古。再到现在,人族的历史上涌现的蛊仙也不过十人而已。

一位九转蛊仙的继承人……

这个身份,让五仙震惊之后,就是羡慕嫉妒恨。

“我怎么就没有这么好的运气。成为继承人呐?”

“这小子运气太好了,太逆天了。居然成为盗天魔尊的继承人!”

“盗天魔尊曾经设下许多传承,据说最大的传承就在空穴当中。不晓得他继承了几处?”

五仙思潮澎湃,纷纷对方源行礼。

蛊仙对凡人行礼。就像是大象对蚂蚁跪拜。但五仙并不觉得羞辱。

而方源安之若素,就算是九转蛊仙对他跪拜。他也不会觉得有多荣幸。换过角色,哪怕他向一位凡人乞丐膜拜,也不觉得羞辱。

在他看来——

但凡会死亡的,都是平等的。无非是早死、晚死的些微区别罢了。

所谓的身份和阶级,高贵和低贱,都不过是一群等死的蠢货,所玩弄的虚假的把戏。这种把戏通过对比,让其中一部分的蠢货觉得自己活得挺好,挺不错。

事实上,那些自认为高贵,有身份的蠢货,都不过是自欺欺人罢了。那些自认为自己低贱卑下的蠢货,更是悲哀。王侯将相宁有种乎!万物生来就平等,何须低头向他人?

“唯有永生,也只有永生,才是应该追求的东西啊!不能永生的话,一个九转蛊仙和茅坑里的一坨屎,有什么分别?!我也是一个超级大蠢货,但是我是个不想当屎的蠢货啊……”

这些心理的活动,心中的志向,实在不足以向外人言语,更不屑言语。

面对五仙的拜礼,方源淡漠地受了,拿眼看向琅琊地灵:“地灵,难道不是你来亲自炼蛊吗?”

“当然是我亲自炼蛊,不过用这五人打打下手,也算是废物利用了。”地灵嘿嘿笑着。

被人称之为废物,又被方源如此无视,叫五仙脸色都变得很难看。

他们心中充满了愤怒,此时却身陷囹圄,不好发作。只能无奈地闷着头,听地灵的调派。

但由长毛老祖所化地灵亲自出手,又有五仙在旁辅助的第一次炼蛊,结果却失败了。

“哼,这星门蛊难炼。非是我等错漏,而是本身就有个铁定的成功率。”地灵解释了一句,又对方源道,“你放心,星门蛊只是五转,必定给你炼成功的。”

“呵呵,那我就拭目以待了。”方源半躺到云床上,并不着急。

琅琊福地的时间流速,是外界的三十六倍。这里过一个月,外界北原才一天都不到。

地灵继承了长毛老祖的傲气,失败了一次,脾气有些糟糕。

他瞟了一眼,看到方源的云烟茶已经喝光,便随手指派一仙出来:“你!笨手笨脚的,胸大无脑。不要你炼蛊了。还不去给小友沏茶!”

被指派出来的女仙,正是黄沙仙子。

她心中生怒,却不敢发作,身家性命拿捏在琅琊地灵手中,只好咬着嘴唇,强忍怨怒,给方源沏茶去了。

但方源却道:“我不喜欢喝茶,你这里有酒吗?我只喜欢喝极品的美酒,你这里可是堂堂的琅琊福地。不会没有吧?”

“哼!怎么会没有?天马酒、清贫酒都是极品美酒,你要喝哪一样?”

“都取出来给我尝尝吧。”方源不动声色。

于是,黄沙仙子就成了给方源斟酒的侍女。

这天马酒,酒液乳白,散发着浓郁的奶香气息。喝到嘴里。更是香醇绵柔。而清贫酒,酒色平淡,宛若清汤寡水,一点酒气也没有。入了口后,也是淡然无味。但是此酒后劲极大,往往喝一口,就能七八天烂醉如泥。

趁着地灵炼蛊。方源浅尝辄止,然后堂而皇之地将这两坛酒,都收入狼吞蛊中。

第二次炼蛊,也失败了。

地灵脸色更加难看。吼道:“哼,我就不信了,给我继续炼!”

方源藏酒的举动,地灵当然也察觉到了。但此刻碍于脸面,就选择了无视。

“公子。请您大发慈悲,救救奴家吧。”看着地灵等人第三次炼蛊,站在一旁的黄沙仙子,泫然欲泣,小声地对方源哀求道。

她肌肤又白又嫩,仿佛刚刚剥下的荔枝。眉发乌青,双眼如湖,暗藏秋波。胸脯硕大雄伟,蛮腰小巧纤细,体态绝不止窈窕,可以说是惊心动魄。

此时又美目藏泪,别说男人,甚至女人看了都心动心软。

“如果能救下奴家,奴家愿意一辈子侍奉公子您,任由公子差遣。”黄沙仙子又软语哀求道。

这可是女蛊仙的哀求啊取悦。

换做别的凡人男子,恐怕早就激动万分,第三肢都怒竖起来。男人都有一种征服欲,尤其是征服高贵的女子。

但方源连瞧都不瞧她一眼,在前世这五仙都是炮灰,死于琅琊福地,没有一个幸存。他们为了贪婪和冲动,付出了应有的代价。

黄沙仙子虽美,但在方源心中,和茅坑中的一堆粪,又有什么区别呢?

“只要不得永生,自己也是茅坑里的一堆屎啊……呵呵呵。”方源心中冷笑。

黄沙仙子又欲张口哀求,她对自己的容貌美色十分自信,此时也抓住冥冥中的灵机,觉得此人或许就是自己逃生的唯一机会。

但她却不知道,方源是个比琅琊地灵更变态的变态。

“你太聒噪了,滚。”方源一伸脚,将半倾着娇躯的黄沙仙子当场踢倒。

黄沙仙子被踢倒在地上,神情一阵发怔,好半天才反应过来——自己居然被拒绝了?他究竟还是不是男人!

强烈的羞恼,冲击着她骄傲的心房,让她的脸色变得扭曲,看向方源的目光变得极为怨毒。

“呵呵。”方源冷笑一声,从云床上爬起来,走到黄沙仙子的面前,又照着她的脸面抬起一脚。

一声闷响,黄沙仙子再被踢翻,头部撞在地板上,又发出一声沉重的闷响。

她一身的五转蛊虫,都被收走。仙元也消耗光了,受到琅琊福地的压制,哪里是方源的对手?

“你!”

黄沙仙子肺都气炸了,门牙被踢断,满嘴的鲜血。她发出低沉的怒吼,神情一片狰狞,和刚刚泫然欲泣的可怜模样,完全判若两人。

“哼,区区美色,焉敢惑我?你再看我一眼,信不信我将你满嘴的牙都踢掉?”方源黑眸深幽,一片冷酷之色。

黄沙仙子娇躯颤抖不止,双拳狠狠捏紧,似乎用尽了平身最大的气力。

但她终究低下头颅,没有再看方源。

那边,星门蛊的炼制又失败了。

地灵气得跺脚,脸色又差了几分。这边黄沙仙子的事情,他也察觉到了。

走了过来,地灵便对方源笑道:“小友还请不要恼怒。这个小东西我刚刚俘虏的,还没有调教好呢。你不妨先玩玩她,让她跳跳舞,脱脱衣,想怎么玩就怎么玩。呵呵呵,材料用光了,我先去买点。”

听到这话,黄沙仙子如坠冰窟。强烈的羞辱感,像是海啸一样冲击她的心房。

“让我脱衣跳舞?!”

这个建议,让她心寒无比,恐惧无比。从小到大,她都没受到过如此的对待。让一代堂堂蛊仙被如此玩弄,黄沙仙子觉得自己还不如死了算了!

而一旁的鬼王、红玉散人,似乎被激起了身体的欲望,均舔了舔干燥的嘴唇,眼光灼热地盯着黄沙仙子。

平时和黄沙仙子情同姐妹的粉梦仙子、青索仙子,则默默无言,好像没有听见的样子。

眼看着人间的惨剧,就要发生,但方源冷笑一声:“脱衣跳舞?这倒不必,有什么意思?色欲?征服欲?哼,全都是无聊的东西。我现在感兴趣的,只有星门蛊。”

\end{this_body}

