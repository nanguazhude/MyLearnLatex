\newsection{参透密语}    %第一百五十六节:参透密语

\begin{this_body}

这个问题其实早就显明,但正常人却是极容易忽略过去。

为什么这位神秘蛊仙要大费周章,冒着巨大风险,去钻营、去利用巨阳仙尊的布局,来设置自己的传承?

“设身处地的想一想的话,如果我是这位蛊仙。我想要设下传承,那为什么非要去钻八十八角真阳楼的漏洞?我自己单独设置一个,花点心思,完全也能布置得很好么!”

“八十八角真阳楼乃是八转仙蛊屋,经由巨阳仙尊亲手布置。利用它的漏洞,风险实在太大。若是我一心留下传承,绝不会舍近求远,故意求险。除非……”

方源眼帘微垂下来,漆黑的眸子里闪过一丝电般的冷芒。

“除非——传承必须这样布置!”

念及于此,方源脑海中的迷雾,仿佛被一双大手猛地拨开。

人行事都会有动机。

只有巨大的动机,才会促使一位蛊仙去冒巨大的风险!

也许这个蛊仙,看不惯巨阳仙尊,和他有仇。但巨阳仙尊早已经逝去无数岁月,这个可能性极小。

那么,抛开感情这个因素的话,那么就只有利益了。

“神秘蛊仙为了这个利益,甘心冒着触犯仙尊布局的生命危险。他(她)到底在利用八十八角真阳楼的什么?”

念及于此,方源的脑海中不由地浮现出琅琊地灵给他的资料。

这份资料,他已经研读过成百上千遍。即便是在王庭之争的大战前夕。他也攻读不休。

就算是看了这么多遍,每一次他再看。或者回忆之时,胸中都不禁涌出一股赞佩之情。

八十八角真阳楼可谓巧夺天工,构思奇特得惊世骇俗。

这是一个狂想,能够实现出来,像是一场奇迹!

八十八角真阳楼本质上是一座仙蛊屋。

当年,巨阳仙尊想要为子孙谋划,主动寻找到长毛老祖,要求他炼制出一件长存不衰的传承重宝。

巨阳仙尊的要求太高。长毛老祖为了达到他的标准绞尽脑汁,苦苦思索百十日而不得。

忽然有一天,他灵感爆发,另辟蹊径,想出了一个独特的办法。

那么庞大的仙蛊屋,实在太过于庞大。长毛老祖只能退而求其次,将其分散成无数子体。即小塔楼。

这些小塔楼成千上万,以十年为期,吸引野生蛊虫进驻。

每当十年之期来临,它们就会陆续沉没,牺牲塔内的野蛊,同时响应北原外界的雪灾。获得奇妙的伟力。

这些奇妙的伟力,一股股凝结起来,量变达到质变,便凝成八十八角真阳楼的一层。

一层层叠加起来,最终形成完整的八十八角真阳楼!

换个角度来讲:每一次形成八十八角真阳楼时。都是一次重新炼蛊的过程。

长毛老祖不愧是古今公认的炼道第一仙,炼蛊造诣早已经超凡入圣。寻常人炼蛊。炼成成品已经极不容易。蛊仙炼仙蛊,更是如此。

但长毛老祖已经脱离了寻常炼蛊的范畴,可谓匠心独运,出神入化。

八十八角真阳楼炼成了吗?

其实没有彻底炼成。

若用正常的标准要求,真正完整的八十八角真阳楼,就是一座成形的塔楼,长存于福地当中,矗立于圣宫之巅。

但这个标准太难了。

当初巨阳仙尊提出之后,长毛老祖听了大皱眉头,当场开口说:若是真正炼成,那么这座仙蛊屋必得九转级数!

九转仙蛊什么概念?

传说中的力量蛊、智慧蛊、宿命蛊等,皆是九转仙蛊。这些仙蛊乃是珍中奇珍,早已绝迹,只能在《人祖传》中瞻仰它们的光辉。

就好比是九转蛊仙,被世人称之为“尊”。漫漫历史中,亦不过只出现了十位而已。

从未有过九转级数的仙蛊屋,但巨阳仙尊强求不罢休。

长毛老祖碍于仙尊之威,只得答应下来。

最终他想到了一个没有办法的办法,得到了“仅能存在一段时间的九转仙蛊屋”——所以,现在八十八角真阳楼乃是八转级数。

巨阳仙尊原本并不满意,但察明之后,却是改变态度,赞叹有加,并且直言:八十八角真阳楼本身就是凡人成就蛊仙之秘!

为什么会这么说呢?

皆因凡人成就蛊仙的难关,得需三气渡过。

一气为天。天在上,天威难测,冥冥浩荡。

一气为地。地在下,厚德载物,沉凝深重。

一气为人。人在中央,万物之灵,奋发狰昂。

凡人就如蝼蚁,唯有和天地相沟通,才能超凡脱俗,使得生命发成本质上的进化。

而八十八角真阳楼的形成,正是用了这三气。

王庭之争,是散布人气。十年风雪,是险恶天气。小塔楼沉没地底,是借助福地地气。

三气合一,使凡成仙。

三气合一,铸就八十八角真阳楼,搜刮北原资源,铸造黄金部族超级势力,使得巨阳仙尊的影响力千万载不磨不灭。

“等一等,难不成?!”

方源雄躯微微一震,眼中一道神芒陡然绽射而出。

“小塔楼本身就是八十八角真阳楼的一部分,布置地丘传承的神秘蛊仙,将这座小塔楼破掉,这就形成了漏洞。”

“漏洞一成,按照炼制八十八角真阳楼的原理,就会形成伟力的回流。从而凝练出新的小塔楼。”

八十八角真阳楼乃是八转仙蛊屋,很难破坏。但小塔楼却是易毁,这就好比蛊仙和凡人。因此巨阳仙尊将这些小塔楼,都布置在王庭福地当中。利用王庭福地的力量。将它们保护得严严实实,隔绝了绝大多数的伤害。

长毛老祖既然是炼道第一仙。自然也考虑到了小塔楼毁坏的情况。因此每当十年之期,凝造八十八角真阳楼主体时,凝聚起来的伟力,首先得回流,补足小塔楼的空缺。然后才是凝造主体。

但是!

“地丘传承布置已有许多年头,但此处的小塔楼却是仍旧没有凝练补充出来。而是被神秘蛊仙用了某种手法,封印住了这个缺洞,欺骗了八十八角真阳楼。我明白了。他(她)之所以这么做,无非就是想利用凝成八十八角真阳楼的这股伟力!”

“土中蕴光,芒高万丈,百里天游,咏梅雪香……如此看来,这句密语应该就是炼蛊之语!”

其实,方源之前也有这方面的猜测。

但他没有确凿的依据。

如今虽然也只是一个猜想的方向。还经过此次实践和探索,方源比原先增添了许多的信心。

专心思索的时候,时间总是过得很快。

黑楼兰等人,败于金白虎虚像,被传送出楼。方源为了不引发怀疑,只得动用琉璃楼主令。暂时也出了塔楼。

一群人陡然出现,站在塔楼一层的入口处。

“出现了,大人们回来了!”

“属下参见族长大人,拜见诸位大人。”

把守这里的黑家嫡系蛊师,立即上前行礼。

黑楼兰等人各个都是灰头土脸。或是身带血渍,狼狈不堪。

虽然失败了。但收获不少,更将关卡推进了十多关。

只是这次情况特殊,黑楼兰等人都将目光锁住人群中的方源,各种探寻、好奇、疑惑、审查的意味,流露而出。

“哈哈哈,此次闯关,诸位劳苦功高,尤其是狼王功劳堪称第一。回去之后,立即备下酒宴,我们要好好庆功三天!”黑楼兰哈哈大笑着,轻拍方源的肩膀,以示亲密。

上等通关,让方源在他心中的价值,又拔升一个档次。

“此次闯关略有心得,需要闭关巩固。希望黑楼兰大人勿怪。”方源淡淡一笑,却是直接了当地回绝道。

黑楼兰笑容一滞,旋即恢复过来,言称无妨,表现出大量容人的上位者气度。

至于他心中如何恼怒,旁人又是如何羡慕嫉妒恨,方源也不去管,他现在就想回去,静心钻研土丘传承的秘密。

六天之后。

方源推开窗户,俯视着脚下的圣宫,眼中闪过一丝喜悦的光。

土丘传承的密语,已经被他完全地破译出来了。

这种性质的猜谜,只要有了一个正确的方向,剩下来的都只是时间问题。

和方源之前猜想的一样,密语就是一道炼就仙蛊的秘方。

布置传承的神秘蛊仙,利用八十八角真阳楼的漏洞,借助凝造仙蛊的伟力,凝练出一只仙蛊。

只需要按照密语中所言,一步步地去炼蛊,哪怕是凡人蛊师,也能炼制成功。

皆因,炼成仙蛊并非是依靠凡人蛊师的力量,而是借助了八十八角真阳楼的伟力。

但这只仙蛊究竟是什么,方源也不得而知,只是根据密语猜测。

“至于这位蛊仙,极可能来源中洲。炼蛊的方法,充满了中洲的风格。再结合前世影像,这份传承应该有完整的一条链子,而我得到的应该只是其中一段。”方源心中估摸着。

在方源五百年前世,中洲蛊仙谋划并利用了这个漏洞,攻破了八十八角真阳楼。应该是发现了这份传承的前置线索。

这并不奇怪。

很多蛊师布置传承,都是这样一环接一环,一步接一步,并且留下一份份的线索。有缘者若是有能力突破过来,就能继承完整的传承。若是能力不够,就只能获得部分好处。

中洲蛊仙得到的,应该是这份传承的前段线索,并不知道后段密语。

否则能有仙蛊可取,他们一定不会这样大材小用。

而方源意外得到的,只是后面一段。没有前面的线索,极为费解。若非方源掌握八十八角真阳楼的情报,又有前世记忆,否则猴年马月也不可能参悟得出来。

\end{this_body}

