\newsection{妥协}    %第一百六十五节:妥协

\begin{this_body}

幽暗的密室中,萦绕着沁人心脾的檀香。

方源盘坐在蒲团上,手里捏着一只东窗蛊。

此蛊隶属信道,四转,形如瓢虫,但背上甲壳却是方形,宛若窗棂,专门用来储存重要的消息。

这只蛊是方源花费代价,从宝黄天中特意求购而来的。

里面的内容,都是关乎意志。

自从墨瑶意志潜伏到自己的脑海中后,方源就一直在收购这些相关的资料,花费了许多仙元石。

这只东窗蛊,早已经不是第一只了。

良久,方源缓缓地睁开双眼,目光冷如清泉。

这么多的珍贵资料,让他对意志有了更深一层的了解。

念、意、情三者一脉相承,其中前两者归属于智道,和其他流派亦多有牵连。

智道,开创之初,便是蛊师一力追求智慧的流派。

人思考时,就会在脑海中泛起一个个的念头。这些念头相互碰撞,或合并,或抵消,形成一个个新的念头,便是思考的结果。

这些念头,又分门别类,相互区别,且各有特色。

其中,最著名的便是执念。

蛊仙临死时的执念,会结合天地伟力,在福地中形成地灵。

蛊师们根据这些念头,创造出无数的相关蛊虫,诸如东方余亮曾经使用过的星念蛊,方源如今正用的空念蛊,以及蛊仙们常用的神念蛊。

一些念头凝聚在一起,形成“意”。

常言道:只可意会,不可言传。

意是无法用语言表达的,意之韵已经越文字的表述极限,只能用心,用人的灵性来体会。

最初,智道蛊师们创炼出来的,是天意蛊。用此来体悟天地大道的运行奥妙,以此来进一步地感悟天地至理,从而推进本身的修业。

后来渐渐展,逐渐有了杀意蛊、随意蛊、如意蛊、得意蛊、恶意蛊、画意蛊等等。

其中最著名的,就是人祖传中所记载的,传说中的意外蛊。

随着智道的展,一代代的蛊师们持续不断的探索,他们现:几股“意”纠缠在一起,便化成了“情”。

qing蛊、柔情蛊、诗情蛊等,都隶属此道。魅道就是从中衍生而出。

当中最出名的,同样来自《人祖传》――爱情蛊。

“念、意、情……”方源叹息一声。

了解的越深,他就越是明白一点――单凭此刻他在智道上的底蕴,还无法清除得掉墨瑶的意志!

双方的差距,太大了。

打个比方的话,在意志的造诣上,方源就如同一座土丘,而墨瑶意志则宛如山峰。而这山峰究竟有多高,多么雄伟,还笼罩在一层浓厚的烟雾中看不分明。

墨瑶的造诣,已经出方源的理解层次。在这些天的接触和试探中,给方源留下了深刻的印象,并带给他一种高山仰止,深不可测之感。

面对这样的**烦,自己处理不掉,又该如何是好呢?

盘坐在蒲团上,又静静思索了一会儿,方源目光冷然,下定了某种决心。

他将意识投入自家脑海,仅仅只是一个念头,墨瑶意志便感受得到,在漆黑的意识空间中,浮现出窈窕诱惑的身姿来。

“我接受你的提议,我会将近水楼台送还到灵缘斋。”方源传过第二个念头,道。

墨瑶的目光中闪过一抹奇异色彩。

她没有料到,方源居然这么快就选择了妥协。

按照这些天来的交锋,她清楚地明白,方源是那种意志惊人,极有主见,性情强硬勇猛之人。

性格决定命运,这种人不管成为枭雄还是英雄,都注定是人中之杰,人上之人。

她抛出六臂天尸王这个杀招,却藏着掖着,没有给全。这与其说起来是一个诱饵,倒不如说是表明她的态度。

她知道,以方源的聪慧,必然明白自己想要表达的意思。

她原本估算着,至少有七八天之后,方源才会选择商谈。但事实上,只是一天之后,方源便主动找上了她。

“唉,识时务者为俊杰,能屈能伸才是伟丈夫……可惜世上绝大多数的人,都囿于一己之能,骄傲得不肯低头。纵观历史,无数的人杰明明都知道退一步海阔天空的道理,但谈起来容易,轮到自己时,真正做到的又有几人呢?”墨瑶意志出一声幽幽的感慨。

“近水楼台虽好,但对于我而言,帮助并不大。你也知道一些我的情况了,如今我因为狐仙福地,被仙鹤门宣布是门派中人。灵缘斋和仙鹤门同属于中洲十派之一,要去那里送还近水楼台,可不是一件容易的事情呢。”方源接着道。

“呵呵呵。”墨瑶意志掩嘴娇笑一声,“小少年,我知道你什么意思。你放心,你是我的继承者,招灾蛊是你的,我也毫无害你之心。送还仙蛊屋事关重大,其中风险我当然清楚得很。要完成这个任务,你至少得需要蛊仙的力量。而我全力将助你成就蛊仙,你要钻营八十八角真阳楼,我也会全程辅助。至于我的炼道心得,你能领悟多少就是你自己的事情了。”

墨瑶说完这话,信手一挥,一道意念便浮现在方源的心头。

意念的内容,正是杀招六臂天尸王的最后关键部分。

方源立即不满地反问道:“地魁尸?你没有搞错么,这就是你所谓的第六只飞僵蛊?”

地魁尸蛊,他早就知道。

乃是斩杀地魁兽,抽经扒皮获得主材,集合数十种蛊虫,再用地底深处九百里下的阴土,以及数百年份的山吸草、晦潮花种种炼成。

虽然威力不俗,和修罗尸、天魔尸、血鬼尸、梦魇尸、病瘟尸同为五转,但却没有飞行之能。

没有飞行之力,如何算得上是“飞僵”蛊?

面对方源的质疑,墨瑶傲然一笑:“普通的地魁尸蛊,当然不行。但姐姐我是什么人?呵呵,我早已改良蛊方,炼制出新的地魁尸蛊。蛊师运用,化为地魁僵尸,虽然仍无翅膀,但却能利用地磁元能,飞遁上天,转折如意。”

说着,又一道意念传来,正是改良的地魁尸蛊的蛊方。

方源一览,顿时眼前一亮。

这蛊方,被墨瑶大胆又巧妙地加入了几种素材,尤其是其中一味元磁精元,乃是主材。正是主导地魁尸蛊改良的最重要因素。

方源稍稍思索,凭借前世五百年经验,便知此蛊方大为可行,不禁啧啧赞叹几句。

墨瑶意志在他的脑海中,得意地大笑几声:“小少年,你还挺识货的。不错,有炼道的些许天赋。但你要小心,这个杀招只是在你提供的基础上进行的草创,我劝你找几个人先实践一翻,再来自身验证。”

方源点点头。

六臂天尸王这个杀招,威力绝伦,是先前四臂地王、四臂风王的近十倍!

威力巨大,一旦失败,那么反噬之力自然也十分强悍猛烈。

方源先前可以试验四臂地王,是因为风险不大,可以承担。这招六臂天尸王就另当别论了。

……

“第三层,第三层形成了!”圣宫上下,欢呼的声浪煊赫动天。

彩色的霞光,仍旧浓郁。

随着时间的推移,八十八角真阳楼的凝聚度越来越快。尤其是新生的第三层,更加激了众人探索的欲望。

八十八角真阳楼的每一层,都有百道关卡。前面的关卡,较为简单。越往后,关卡便越难,奖励越丰厚。

对于大多数蛊师而言,无力攻克后面的关卡,前面的关卡奖励则称为众人争抢之物了。

一群蛊师,有十多人,正在急行出宫。

他们正穿过圣宫的东大门时,八十八角真阳楼第三层形成的动静,让这伙蛊师停下了脚步。

蒋冻回望了圣宫顶端的八十八角真阳楼一眼,冷哼一声,对队伍中的马鸿运道:“你们这些黄金部族的蛊师们有福了,八十八角真阳楼中的任何一关奖励,都能让你们飞黄腾达!”

马鸿运傻笑几声:“队长说的是,呵呵呵。可惜我血脉浓度不高,无法进入八十八角真阳楼。要不然进去开开眼界也好啊。”

即便是黄金族人,巨阳仙尊的后裔,也不一定能够进出八十八角真阳楼。

如果祖辈先人混合的血脉过多,导致血脉稀薄,达不到标准,也不能进入楼中。

蒋冻听了这话,心中的酸楚和嫉妒顿时减轻了许多。队伍中的其他蛊师,看向马鸿运的目光,也变得柔和。甚至有一人,还拍拍马鸿运的肩膀,安慰道:“你这小子也算倒霉了,不过没关系,这次跟随我们杀地魁兽,也能大赚一笔!”

“是,是,是。”马鸿运点一边头哈腰,一边赔笑着。

他机缘巧合,救下了马英杰。黑家强招了马家之后,他一番际遇之下,就改变了姓名,成了马鸿运,更能修行。

如今,他只是一转蛊师,资质不算低,也不算高。

他当然能进出八十八角真阳楼,但是凭借他的实力,就算进去了也没用。居住在圣宫的这些天,他每天混迹在野队中,外出狩猎,应付日常的开销,积累元石辅助修行。

至于对蒋冻所言,则是赵怜云在临行前特意叮嘱他的话。

(未完待续)

\end{this_body}

