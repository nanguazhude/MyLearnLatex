\newsection{三老三真传,意志可取巧}    %第一百九十三节:三老三真传,意志可取巧

\begin{this_body}

方源努力朝着一团碧光游去。

这团碧光,却比先前的赤光,更加难以接近。

它在缓慢地飞,不断转弯,方源追了片刻,这才追到,一把抓住。

它同样有海碗大小,方源小心翼翼地探入心神。

一股难以言喻的震动,从碧绿光团中,延伸到方源的身体上。

“小心,这是空绝老仙的东西!”脑海中,墨瑶意志忽然认出了此物,开口示警道。

“放心,它飞不走!”方源语气肯定,双手攥紧,牢牢握住光团。

“笨蛋,我不是说的这个,快松手。”墨瑶说到这里时,已经晚了。

玄妙的震荡,从碧光中传出,竟然一直影响到方源的空窍上。

方源空窍陡然震动!

九成的真元海面,掀起波澜大浪。

哗哗哗……

巨大的浪潮,扑打在四周的窍壁上。五转巅峰的剔透晶膜,在浪潮的冲刷下,产生龟裂的纹路。

方源大吃一惊!

这个真传的考验,极为奇妙,难以防备,居然直接影响蛊师窍壁。

“小子,快用蛊虫斩断你的双臂。壮士断腕,才有一线生机!”墨瑶意志在方源的脑海中高呼。

“斩断双臂?”方源脸色一沉,额头渗出冷汗。

他尝试过松开双手,但这团碧光发出一股极强的吸摄之力,将方源的双手牢牢吸住。

方源摆脱不得,空窍震荡得越来越强。

“有了!”情急之下,方源灵光一闪,调动第二空窍真元,传入第一空窍。

震荡通过真元,传播到第二空窍中去。

第二空窍也开始震荡不定,但如此一来,得到分担之后,第一空窍的危情得到了极大的缓解。

墨瑶轻咦一声。

碧光渐渐静止下来。震荡消失,似乎承认方源通过了考验。

“这是怎么回事,你居然有两个空窍?”墨瑶叫道。

两个空窍一起承担考验,虽然晶壁都有裂痕,但终究没有破裂。

“小子,你隐藏得真的挺深,居然有两个空窍!第二空窍……想不到一直流传的传闻。居然在你的身上,得到了证实。”墨瑶语气感慨万千。

方源不愿谈及这事,而是问道:“你刚刚提到空绝老仙,难道就是那位上古的炼道大宗师?”

“不错,正是他。炼道大宗师……纵观人族历史,古往今来。也不过只出了三位。后人称他们为‘三老’。分别是远古时代大宗师天难老怪,上古时代大宗师空绝老仙,中古时代大宗师长毛老祖。”

墨瑶继续如数家珍道:“这其中,天难老怪,性情怪癖,妄图炼化天空而身败陨落。空绝老仙,对空窍研究最深。著有《仙窍方略》,还助十绝体成仙。在他之前,从未有十绝体成功升仙的例子。而长毛老祖,则寿命最长,炼成仙蛊最多,八十八角真阳楼就是他所炼。甚至传闻中,他还能炼寿蛊,引得巨阳先祖攻打琅琊福地。但最终却不了了之。”

墨瑶说出了许多秘闻。

三老,三位炼道大宗师,每一个都是传奇。

他们是人族历史上,伫立在炼道上的三座巅峰,就算是九转尊者也掩盖不了他们的光辉和风采。

“空绝老仙的这份传承,考验有缘人的空窍。你居然有两个空窍,也算取巧过关了。看看吧。这份传承很不错,对你将来很有帮助。”墨瑶接着道。

方源重新探入心神。

这份传承,名为升仙秘要,记载空绝老仙对于各流派蛊师升仙过程中。可能遭遇的难关,以及如何应对,升仙的关键要点等等的总结。

方源心中思量:“虽然这不是蛊方,也不是仙蛊,但却具有极高的理论指导价值。蛊师得了,不仅对自己有巨大帮助,对其他各道更增了解,是绝对的重宝。”

“这份升仙秘要,时间久远,未免有些跟不上时代潮流。毕竟经过这么多年的发展,多种流派百花齐放,远非上古时代的单调。我要提醒你的是,这里面最精髓的并非升仙秘要本身,而是巨阳仙尊阅览时留下的批注。这份批注高屋建瓴,提纲契领,说是一字千金也毫不为过。”墨瑶又爆出一个猛料。

空绝老仙是媲美长毛老祖的炼道大宗师。

他的探索,再加上仙尊亲笔批注,价值极其巨大。

蛊仙阅览之后,必然能极大地增强升仙几率。并且通过仙尊的目光,远眺之后的发展。对将来的蛊仙之路,有极大的指导作用,能避免走许多弯路。

这份真传的诱惑力,是惊人的。尤其是对于方源这样的天才,渴望再进一步,不满足于凡俗红尘的,更是如此。

但方源还是没有选择它。

他有前世经验,重走血道就是了。

他来到真传空间,别有目的。

于是方源放手,任由碧色光团远离自己,用比赤色光团稍快的速度,飞向远方。

“你真的选择放弃?以后回想起来,可不要后悔啊。”脑海中,墨瑶叹息。

“后悔?呵呵,我的人生中早已没有后悔这两个字了。”方源轻笑几声,继续寻找。

他在黑暗虚空中划游了片刻,发现了第三颗光团。

这个光团,闪耀着灰色光辉,速度比前两个光团还要快,好像是悠然飞行的一颗流星。缓慢飞行中,拖出一条好看的霜色焰尾。

方源努力接近它时,墨瑶忽然开口:“其实你大可以不必每次犯险,你既然已经陆续收集了许多智道蛊虫,何不让它们发挥作用呢?”

“哦?你这是什么意思?”方源速度不停。

“小子,好好感谢我吧,我交你这一招,能让你避开真传的考验,就可探寻大部分真传的内容。”

墨瑶所说的方法,其实也不复杂。

方源追寻到真传光团之后,首先第一步就是要用心神探入,这就会触发真传的各种考验。

墨瑶便建议方源。利用智道蛊虫,产生意志。将这股意志送入真传当中,等候片刻后,再收回意志。

这样一来,就避免了蛊师和真传的亲密接触,属于投机取巧的手段。

“智道,有念、意、情。所谓的心神。即是心念和神念。心神探入,就是念头的探索和交流。念头如水滴,最易产生,也最脆弱。因此数量得多,颗颗水滴汇集成流,才能进行持续探索。这个过程中。又须得蛊师脑海中不断生产,方能绵绵不绝。而意志高于念头,宛如冰块。防御更强,产生不易。但却可单独行事,需要的时候再收回即可。”墨瑶解释道。

方源沉吟不语。

“臭小子,到了这个关头,你还在犹豫什么?呵呵呵。我知道你一直在收集智道的情报,同时也在收购智道蛊虫。你的那些特意蛊、刻意蛊、玩意蛊、留意蛊、新意蛊,现在不用,什么时候用呢?”墨瑶娇笑起来。

顿了一顿,她继续劝说道:“现在的你,想必也清楚:我的意志寄生在你的脑海里,已经和你深深纠缠。而你用这些蛊产生的意志,却必定纯净无暇。是我渗透不了的。你不是一直在防备我么?此法还有一个巨大的好处,那就是你可以用这个方法,规避真传考验,去查看无双传承的秘密。无双传承,可是比普通传承的价值,还要巨大呀。”

“你究竟……”方源目光迟疑。

一直以来,他和墨瑶之间的关系。都没有挑明。没想到墨瑶,竟然在此刻忽然说破。

她的意图究竟是什么?难道真的如她所讲的那样,是一心想让方源将七转仙蛊屋近水楼台,还给灵缘斋吗?

墨瑶坦诚布公的表现。让方源心绪一阵起伏。

也许,是自己太谨慎,一直过于防备墨瑶了?但意志不比念头,一颗念头十分简单,不擅长哄骗。执念形成的地灵,更是句句真话。

而意志不同,它是许多念头的结合,比念头更加复杂。哄骗他人,也是可以做到的。

方源微微摇头,将脑海中的杂乱思绪尽数排遣,现在不是想这些的时候。

不得不说,墨瑶的话完全站在方源的立场上,成功说服了他。

方源掏出收购而来的智道蛊虫。

他有地灵小狐仙,又能随时沟通宝黄天,站在蛊仙的肩膀上,要收购凡蛊还是比较容易的。

现在,他手中的这八只智道蛊,都可以凝出意志,但又各有区别。

特意蛊,能够产生特意,一定情况下能触发特殊动作。

刻意蛊,能产生刻意,这种意志寄生下去,宛若刻在铁石上面,消磨的难度是其他意志的数倍。

玩意蛊,产生的是玩意。玩意是最能凝成“情”的意志之一。

还有留意蛊、新意蛊、战意蛊等等。

值得一提的是,这些智道蛊虫,能产生分门别类的意志,同时也能克制相应的意志。

比方说,特意蛊能产生特意,也能吞噬特意。刻意蛊能凝成刻意,刻意难以消磨,但只要对症下药,找到刻意蛊,就能很轻松地驱除掉它。

方源为什么要收购这些蛊虫?

就是因为此点。

他要对付墨瑶的这段神秘意志。可惜他收购到现在,仍旧没有发现手中的任何一只蛊虫,能够对付墨瑶意志的。

现在这情况,选择什么意蛊倒并非关键。

方源随手选择了特意蛊,当即灌注真元,产出一股意志。

相像酷似方源自己,青年容貌,目光幽黑,但和本体不同,面颊轮廓硬朗,颇有决断的神色。

“意志只能寄存在魂魄当中,你要送出这份意志去探索,还得至少需要一只小魂虫。”墨瑶提醒道。

小魂蛊,乃是魂道的一转蛊,极为普通,却是魂道重要基石。许多更高转的魂道蛊虫,都是在它的基础上,一步步合炼而成的。

墨瑶的提醒,显得有些多余。

她话音未落,方源就唤出一只蛊虫。

它无实体,巴掌大小,宛若蝴蝶,悠悠晃晃,似光如影。常人若用手去抓,只能透体而过。

这是二转的大魂虫。

方源轻轻一拍,将自己的特意拍到大魂虫的体内。随后调动大魂虫,飞向最近的一团绿白相间的真传光团。

大魂虫的速度,可比他自己划游快得多了。

不一会儿,就钻入真传光团当中。等了片刻后,身躯残破的大魂虫,摇摇晃晃,飞回来。

方源取回意志,细细一看。

这第三个真传,也是普通真传,但对于一些蛊师,甚至蛊仙而言,价值比前两个真传要高的多!

------------

\end{this_body}

