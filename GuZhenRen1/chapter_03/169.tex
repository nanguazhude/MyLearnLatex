\newsection{墨瑶的弱点}    %第一百六十九节 墨瑶的弱点

\begin{this_body}

%1
深幽的大殿中,只站着方源一人。

%2
死一般的沉寂。

%3
昏黄的光晕链接一体,形成蛊屋形状。光屋中,躺着一地的尸体。

%4
近两百位蛊师,俱都惨死。

%5
这些尸体,有的浑身长满鳞甲,浑身僵硬如铁,有的生长怪臂,面孔狰狞。有的干脆自爆,残肢碎片各处洒落。

%6
方源脸色沉凝,在脑海中质问墨瑶:“这就是你构思出来的杀招?”

%7
墨瑶却理直气壮:“这些蛊师肯定是要死的,六臂天尸王乃是力道杀招,对蛊师身躯要求极高。现在看来,哪怕是你的力道底蕴,也难以承载这种烈度。”

%8
方源冷哼一声,表示不满道:“这就是你给我的解释?你以为我看不出来?这杀招缺陷甚多,一开始就死了五六十人,中途死去蛊师已近百,到了后期只剩下六七位蛊师还能支撑。缺陷至少有十七处!你最好别忘了,我们是有协议的。”

%9
“呵呵呵。”墨瑶一阵娇笑,被威胁了也不以为意,回道,“小弟弟,你不要性急嘛。六臂天尸王只是初步草创,有些瑕疵,也是应当。只要接下来不断修缮,就会慢慢地达到理想的程度了。”

%10
她一边回应方源,一边暗道:“这小子不简单,竟能察觉到十七处的缺陷……单凭他的炼道大师造诣,只能看出十一处。看来他的力道造诣,也已经达到了大师级的边缘了。”

%11
墨瑶却不知,方源其实耍了一个滑头。

%12
“我只看出了七处缺陷,故意添了一个十上去。就是想诈一诈她,看来这杀招缺陷远不止十七处了。”方源心中计较。

%13
墨瑶在揣度他的深浅,方源也在揣摩墨瑶的底细。

%14
六臂天尸王,缺陷重重,用来试验的蛊师全都死了。但方源不忧反喜。

%15
“按道理来讲。墨瑶有炼道宗师的境界。即便是草创杀招,也不可能有这么多的缺陷!墨瑶是炼道宗师不假,但现在这里的。只是她留下来的一段意志……”

%16
方源妥协是假,想要铲除掉墨瑶意志的打算。其实一直没有改变。

%17
现在的试探,就让他发现了墨瑶意志的一个弱点。

%18
“我虽然对智道并不熟悉,但之前花了大代价,从宝黄天中收购许多昂贵的情报,也是有价值的。”

%19
“智道是寻求智慧之道,分‘念’、‘意’、‘情’。念是基础,但凡人思考时。就会产生一个个的念头。许多念头凝聚起来,就形成意。不同的意纠缠不休,会形成情。念头如杂草,烧不尽吹又生。意变化万千。或如钢铁坚硬,或似苍空渺渺,或如烈火熊熊……而情则如水,浅薄时若小溪,剪不断理还乱。深沉时如汪洋。滔天覆地威力绝伦。”

%20
“这三者,情最难根除,意次之,念头最易摧毁,但也最易生成。我应该庆幸。现今留在我脑海中的,只是墨瑶的意志,而非她的情感。”

%21
“人思考时,产生念头,念头相互碰撞,或泯灭或融汇,最终产生新的念头,这就得到了思考的结果。那么一段意志呢?呵呵呵。”

%22
方源想到妙处,不禁肚中冷笑连连。

%23
并非将乱七八糟的念头胡乱集合在一起,就会形成意志。但意志,的确是无数念头的结合。

%24
意志一旦思索问题,就会动用本身的念头相互碰撞,产生新念头,得到思考的结果。

%25
“墨瑶乃是炼道宗师,她的意志,也继承了炼道宗师的境界。但却构思出缺陷众多的六臂天尸王,不是因为她能力有限,而是顾及自身安危,不敢深入思索啊。”至此,方源已觑得墨瑶意志的弱点!

%26
活人思考不休,尚且伤身,更何况一段遗留下来的意志呢?

%27
意志越思考,便越是损耗自身,越会孱弱下去。

%28
若是墨瑶还活着,她的意志所损耗的念头,就会在她的脑海中,得到补充。但现在的情况是——墨瑶已经死了!

%29
为什么她的意志,一直在近水楼台中沉睡,直到方源进来,才惊醒过来?

%30
就是因为,意志不能时刻清醒,清醒时思维敏捷,就代表思考。思考得多了,意志就会越来越孱弱,最终消亡。

%31
再看八十八角真阳楼,这座雄奇伟大的八转仙蛊屋中,寄居着巨阳仙尊的意志。

%32
但即便强如巨阳仙尊,他留下来的意志,如今也在沉眠着!

%33
不管是巨阳意志,还是墨瑶意志,都是无根之水,得不到任何的补充,只会越来越弱。

%34
“我的智道造诣,虽然浅薄无比,但是我有一个巨大的优势。那就是我还活着!现阶段是,虽然不能和墨瑶意志匹敌,但是只要给我充足的时间僵持,此消彼长,结果必然是我的胜利。”

%35
思考到这一步,方源便停止了这方面的思绪。

%36
毕竟,他的脑海中藏着一段墨瑶的意志。思考太多,念头就多,虽然现在用着空念蛊,但保不齐会被墨瑶发现破绽。

%37
“唉……”

%38
大殿中,响起方源一声幽幽叹息。

%39
墨瑶意志是个非常大的累赘,令方源思索问题时,都要有所顾忌。

%40
原本潜入王庭福地,谋夺江山如故仙蛊,就已经费尽心思,令其殚精竭虑了。如今又加上这么一个大敌,潜伏在自己的脑海当中,令方源投鼠忌器,每次思考问题的时候,都要做顾而言他,夹杂旁的念头。

%41
这番思索,换做以前,不过是沉心凝神就可达到的事情。但方源现在做来,却是感到心力交瘁,有一种精疲力竭的味道。

%42
墨瑶意志听了这声叹息,却是误解了方源,当即安慰道:“呵呵呵,小弟弟,你年纪轻轻,自当奋发图强,萧索叹息作甚?你放心。我已经改良了杀招,需要换下其中六只蛊虫,还要再添三只进去。”

%43
“哦?”方源眉头一挑。不动声色,“愿闻其详。”

%44
墨瑶便一一详说。

%45
方源有狐仙福地在手。沟通宝黄天,换取凡蛊不成问题。

%46
同时,方源又命葛光、常极右二人,暗中擒拿蛊师,押到大殿中来。

%47
三天之后,又是一场人体试验。

%48
试验的结果,比前一次好上不少。但犹有缺陷。

%49
墨瑶结合结果,又一次提出改良建议。

%50
方源依言而行,如此三番五次,不厌其烦。他心知肚明:墨瑶意志有炼道宗师的境界。但却不想深入思索,削弱自身。因此不断试验,用现实效果,来减少自身的思索力度。

%51
但就算如此,墨瑶意志也仍旧需要思考。如何改良杀招?想得到这个问题的答案。思索必不可免!

%52
这样思索的次数多了,墨瑶意志也就越来越弱了。

%53
方源不想打草惊蛇,他已经做好打持久战的准备。

%54
就这样过了大半个月,第十五次试验。

%55
大殿中,横尸遍地。

%56
方源将组合六臂天尸王杀招的十数只蛊虫全都收起。扫视一眼地面,满意的微微点头。

%57
试验者修为低微,难以承载六臂天尸王这个强大杀招,是必死无疑的。

%58
但怎么死亡,死亡后的形态如何,却是关键所在。

%59
现在,地上的这些尸体,俱都化成了僵尸,浑身昏黄鳞甲,背后长有六只臂膀,亦都是长有尖锐鳞片。手臂相同粗细,肌肉贲发,到了手部则是漆黑一片,指甲俱都尖锐狰狞。

%60
“六臂天尸王,是以借力蛊为核心,六大飞僵蛊为基石,其余十八只蛊虫为辅。严格来讲,是变化道的杀招,凝聚尸气,使人化身飞僵。如此一来,蛊师身躯强度就攀升十多倍的程度,可以承担更强的力道。”方源徐徐开口,总结这些天来的心得体会。

%61
“但也正因如此,尸气郁结浓重,使得此杀招的后遗症十分严重。就算是辅蛊中,有了三只专门增添生机之蛊,也只能做到维持杀招半刻钟的程度。大体就是如此吧?”方源询问脑海中的墨瑶意志。

%62
墨瑶笑了起来:“小弟说得头头是道,是这个理儿。用借力蛊,能借到天地气象之力,但就算是力道五转强者,也不能随意施展。因此我便想到用僵尸之体,僵尸之体强于鲜活**,且恢复卓绝,和借力蛊相得益彰。”

%63
“但蛊师是活人,尸气却是死气。你化身六臂天尸王的时间越长,身体中的尸气就越多,当尸气吞灭生气,你就会彻底化身成僵,再也转变不过来了。这就是这个杀招的后遗症。”

%64
“唉,生气、死气,泾渭分明,形同死敌。生死融合,是万古难题,已经超出我的能力范围。我在辅蛊中专门布置了三只增添生机的蛊虫,这已经是极限了。数量多了,干扰尸气,轻则令杀招威力大降,重则令整个杀招崩溃。数量少了,就使得战斗维持时间锐减,实用性大降。”

%65
方源听着,表面上不停点头,表示赞同,心中却嘿了一声。

%66
墨瑶此言,大有转移关键话题之嫌。

%67
生死融合,的确是万古难题。至始至终,从未听说过有人死了,又同时活着,生死不能并存。

%68
但墨瑶是炼道宗师,难道就不能构思出其他的杀招吗?

%69
为什么一定要用六臂天尸王?不过是她节省思索力度,因此在方源提供的杀招“四臂风王”,稍加改良而得。

%70
方源可以肯定,凭借宗师境界,墨瑶完全可以另起炉灶,构思全新的杀招。

%71
“墨瑶此人,乃是异族,是墨人。非我族人,其心必异。因此这些天来,前后用了上千名蛊师试验,她身为正道人物,灵缘斋仙子,却是淡漠视之。她的爱人是鼎鼎大名的剑仙薄青,想来这爱意中,恐怕还是推崇强者的心意多一些吧?”

%72
薄青号称“剑劈五洲亚仙尊”,和其相比,方源不过一届凡人。在墨瑶的心中,是否和死去的试验品差不多?

%73
“所以,这世间之事,还得靠自己。也只有自己,才是最可靠的。”方源暗中冷笑,也不点破,而是向墨瑶请教道,“如此一来,六臂天尸王这个杀招,可算完善了?”

%74
墨瑶摇头:“还不够,试验的对象只是低转蛊师,我们还需要高转蛊师。五转蛊师最好,若是力道五转,那就最完美不过了。”

%75
“五转力道?”方源眉头微皱,就他所知,身边除了自己,根本没有其他人选。

%76
方源摇了摇头,他决定将这个事情先放置一边:“这事先不急,墨化需要的蛊虫,我已经集齐了。是时候再探八十八角真阳楼了!”

\end{this_body}

