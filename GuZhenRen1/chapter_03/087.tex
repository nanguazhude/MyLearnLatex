\newsection{再收异兽狼}    %第八十七节:再收异兽狼

\begin{this_body}

“如此一来,狼群数量都达到了二十万!但是……还是有些不足啊。”方源的眼中,闪烁着思索的光,“普通的野狼,可以先不忙着积累,到时候自会有人主动送上门来。现在还是优先扩充高端的战力。”

宝黄天中虽然时不时的有兽皇出售,但每次都会在短时间内,被其他蛊仙买去。

方源知道,自己要想连续买下几头狼皇,并不现实。于是,他将目光集中在异兽上。

每一头成年的异兽,战力都可媲美万兽王。

但凡一流的奴道蛊师手中,都有一支异兽组成王牌队伍。

诸如,南疆犬王手中的狮獒部队,江暴牙麾下的钻山鼠大军,杨破缨的雷鹰群,马尊的天马群。

狼类的异兽有很多。此刻,宝黄天中正往外卖的,就多达四种。

第一种,是血森狼。

这种狼,体型庞大,简直和山丘一般。成年人站在它的脚下,就宛若狐狸站在大象的脚边。

血森狼上下,皮毛血红,狼毛枯槁,如同蒿草。狼背上长着一片“白色树林”。

这些“树”,其实是血森狼的骨骼,从背部衍生出来,竖直生长着。白色的骨树上,还长满了血红色的枫叶。一棵棵的树木炼成一片,就是俗称的“血森”。

血森狼产下幼崽之后,就会将幼崽放进背上的血森,广阔的空间供幼崽玩耍。血森中的血果,则是幼崽的食物。

血森狼就像是一座座移动堡垒,速度虽然不快,但是具有极强的碾压能力。这是很多万兽王都不具备的。

第二种,是鱼翅狼。

这是水路两栖的异兽,体型如象,身上长满了光滑的鳄鱼皮甲。同时身体的两侧,生长着尖锐的深蓝鱼鳍。背上也有一排类似鲨鱼背部的鱼翅,这些鱼翅连成一线,从狼头延伸到狼尾。

鱼翅狼是防御力最强的异兽狼,同时它还具有水下作战的能力。

第三种,是狂狼。

狂狼浑身银灰,长着三只眼睛,体型不大,和寻常的千兽王相同。

但因为体型而小看它的人,都遭到了惨重的代价。

狂狼一旦战斗起来,非常疯狂,动作极为迅猛,不杀死对手决不罢休。尤其是当它睁开第三只眼的时候,它的战力将暴涨五倍!

一旦它三眼齐睁,不管战斗结果如何,它都会战死沙场。

这是一种连自身都不顾,陷入战斗狂热中的可怕狼种。

第四种,则是白眼狼。

白眼狼的狼瞳,都是一片纯白色。它们的视力极强,就算在黑夜里,视力也不会受到任何的影响。

宝黄天中,待售的一头血森狼,三只鱼翅狼,两头狂狼,以及五只白眼狼,都被方源买下。

而与此同时,远在南疆的影宗福地中,砚石老人眯着双眼,盯着半空中的通天蛊。

方源大肆地收购狼群,接连买下狼皇、异兽狼,引起了这位智道蛊仙的关注。

“这个方源,他忽然买下这么多的狼群,是要做什么?”

这位身处幕后,来历神秘的七转智道蛊仙,皱起了眉头。

他尝试用蛊虫,进行推算。

但得到的结果,却是方源要豢养狼群经营谋利。

这个结果,并不令他满意。

“狐仙福地的环境,豢养狐群最为有利。狼虽然类似狐狸,但是终究还是有所差别的。”

福地有福,是恩泽之地。但每一片福地,又有差别,恩泽不尽相同。

比方说,狐仙福地最利于豢养狐狸。狐群在这片地方生存,成长会更好,繁衍得更多。而在琅琊福地中,利于炼蛊,更适合毛民生存。

影宗福地,则利于魂道蛊师的修行。

心底深处有一种感觉,告诉砚石老人,方源此举大有深意。

但智道也不是万能的,也有其弊端。否则,蛊师世界中早就智道独大,并非现在百花齐放,各道争鸣的气象了。

智道推演,需要证据。越多的证据,越可靠的证据,都会指引智道蛊师,推算出越加正确的结果。

但砚石老人,虽然身为智道蛊仙,但他万万不会料到,方源乃是重生之人。

砚石老人推算出,方源动用定仙游蛊,去了狐仙福地。但他不会料到,方源竟然跑去了北原。

若是方源在宝黄天中,出售神游蛊,兴许他还能联想到这一点。

但方源谨慎,将神游蛊直接转卖给了琅琊地灵。砚石老人缺少最关键的一点证据,导致他推算出错误的结果。

他接连又算了几次,都是这个结果。

“难道非得我祭出本命蛊不成?”砚石老人这样想着,浑身毛孔张开,从每个毛孔中泄露出一丝云烟。

白色的云烟,袅袅上升,在他的头顶凝成一片,形成翻滚不休的烟云。

烟云迷蒙,普一出现,浓郁的仙蛊气息就泄露出来。

但这只高达七转的仙蛊,气息飘渺不定,宛若夏夜的星空,神秘无迹。又仿佛千步外的莲香,若隐若无。

云烟每一次翻腾,都蕴藏着千万种变化,说不清道不明。外人逞强参悟,得到的都只是似是而非的结果。

这便是砚石老人的本命蛊,名为——天机!

天机仙蛊!

它能将天地的机密都泄露出来,哪怕蛊仙没有任何证据,也能直指真相。

砚石老人在发现杀人鬼医被种下奴隶蛊后,依靠天机蛊,推算出方源,会在将来的某天,回到南疆。

因此,砚石老人才布下了陷阱,等着方源跳进来。

此刻,砚石老人犹豫着,是否该使用天机蛊呢?

天机蛊虽然能力强大,却也有着弊端。

砚石老人并不是每次催动它,都能获得成功。十次催用天机蛊,至少有八次会失败。一旦失败,砚石老人就会受到仙蛊的反噬。

这种反噬,若是寻常伤势也还罢了,但偏偏极为厉害,任何人都得忌惮。

砚石老人的身体、魂魄毫无损伤,天机蛊的反噬只针对他的寿元。

一旦反噬,砚石老人就会失去,十年至七十年不等的寿命!

蛊师修行,提升的境界,对寿元无直接帮助。蛊师要延寿,最佳的选择只有一个,那便是寿蛊。

使用寿蛊,能直接增长蛊师寿命,毫无其他副作用。

除此之外,便动用其他旁门左道,来达到延寿的效果。但这些法门,无不有着弊端缺陷。

“我现在的身体,寿元还剩下八十年。就算最严重的反噬,削去我七十年寿命,我还有十年的时间,足够我完成我的逆天大计!而且这等小事,一般反噬的结果也不严重,正常是十三四年的寿命。但是……”

“我为此而动用天机蛊,值不值得呢?此番逆天,到了关键之处,敌手势必察觉。将来,我还需要天机蛊测算推演。”

“然而,若是能得到方源手中的定仙游,对于逆天大计也极有帮助。远的不说,就说进攻琅琊福地。若是有了定仙游蛊,我便进可攻,退可守。琅琊地灵哪里能奈何住我?也不至于这一次大败亏输。”

砚石老人左思右想,终究还是放弃了这个念头。

天机蛊虽然强大,但也是个坑。使用它的成功率太低,更关键是失败的结果很严重,砚石老人不敢随意浪费自己的寿元。

当初,他为了推算出方源利用定仙游,去了何地。先是浪费了七十年的寿元,结果得到的结果,让他傻眼了好一段时间。

居然在中洲!

竟然是狐仙福地!

他怎么会传送到那个地方的?他得了什么机缘,获得了狐仙福地的内部景象?

好嘛,方源龟缩在福地中不肯出来,砚石老人想要谋划定仙游的计划,还未开始就险些折戟沉沙。

好在他继续利用天机蛊,又耗费了八十年寿命,算出来最有可能得手的机会。

在将来的某天,方源将回到南疆,参加义天山大战!

为了这只定仙游,砚石老人整整耗去了一百五十年的寿元。

“罢了,既然已经算出了结果,那么就守株待兔好了。他豢养狼群的事情,无伤大雅。身为智道蛊仙,我亲手布下的局,还怕他区区一介凡人逃脱得了?哼哼。”

砚石老人冷笑几声,缓缓睁开双眼。

他的双眼,和白眼狼差不多,没有瞳孔,只有眼白。

他盯着通天蛊,嘴角的冷笑又浓郁了几分:“你一介凡人,居然耗去我一百五十年寿命,将来死在我的布局之下,也算是荣幸了。当然现在,你也别想好过!”

砚石老人一番运作之后,很快,在狐仙福地的方源,就觉察到了不妥。

“不好,有人在大量收购驭狼蛊的炼制材料!”

方源正打算收购这些材料呢,砚石老人却抢先了一步。

他赶忙出手,很快就遭到阻击。有不少蛊仙在和他故意抬价,令他花费更多代价,收购的材料却还不及原先的程度。

方源眯起双眼,寒芒如针:“这是针对我的行动。哼,琅琊地灵是不可能的,也没有这样的号召力。那么除了仙鹤门,就只有那个神秘的砚石老人了!”

“呵呵。”

忽然,方源又笑出了声。

若换做先前,被这样狙击,他只能徒呼奈何。但现在却不一样。

他的手中,有大量的驭狼蛊秘方。这些秘方,不仅从一转至五转应有尽有,而且每转驭狼蛊的秘方,都有不同种类。

这些蛊仙拦截的材料,只是大众化的驭狼蛊方而已。还有许多蛊方,别出机枢,另辟蹊径,尤其是琅琊地灵独自研发的秘方,这上面的材料这些蛊仙又怎么能了解呢?

\end{this_body}

