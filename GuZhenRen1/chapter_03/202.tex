\newsection{伏击巨阳意志}    %第二百零二节:伏击巨阳意志

\begin{this_body}

。

一道炽热如阳的意志,通过琉璃楼主令,向着方源脑海暴袭而去。

意志宛若黄金洪流,横冲直撞,竟然直接闯入方源的脑海!

意志乃是由特定的大量念头组成的,若任由巨阳意志冲刷脑海,极可能令方源失忆痴呆。

“糟糕!快丢掉那块该死的令牌!!”脑海中,墨瑶意志显现出身形来,尖叫出声。

她是蛊仙意志,临死前高达七转,意志底蕴比方源更加深厚。

她和方源一荣俱荣一损俱损,等若拴在一条绳上的两只蚂蚱。

脑海乃是关键之地,方源更加不愿在自己的脑海中开战。

“无须惊惶。”他冷笑一声,早有准备,并不惊惶,旋即催动身上的特意蛊。

特意蛊得到真元灌输,直接闪现到他的脑海当中,拦在巨阳意志面前。

黄金般的意识洪流戛然而止,被特意蛊彻底的拦下!

随后,特意蛊爆发出一股强大吸力,将巨阳意志不断吸食。一小吞大,宛若蛇吞鲸般,偏偏巨阳意志反抗不得。

“这是……”墨瑶意志看得一呆,下一刻恍然大悟,“原来巨阳意志,乃是特意!”

智道发展悠久,从远古时代就起源,一直传出到今天。

意志隶属智道,自然内容丰富。不仅意志的种数分门别类,效用更是五花八门。

就方源初步涉及,就收购了不少智道蛊虫。诸如特意蛊、刻意蛊、玩意蛊、留意蛊、新意蛊。专门用来凝造出特意、玩意、留意、新意。

这四种意志,又各有优劣长短,需要蛊师根据自身情况。自己选用适合自己的意志。

其中特意,蛊师能预设动作,在特定情况下自行触发。

这个效果,打个浅显易懂的比方,就如同老鼠夹。当老鼠触发陷阱(特定条件满足),老鼠夹便会发动,困住老鼠(预设动作)。

巨阳仙尊为子孙谋利。考虑到将来的危难局面,便选用了最为合适的特意蛊,凝出巨阳意志。放在八十八角真阳楼中。

只要危机达到一定程度,特定条件满足,特意便会触发巨阳仙尊的预设动作。至于这个预设的动作,究竟是什么。外人还不得而知。就算是重生的方源。也不清楚这点。

但他从前世影像中知道,巨阳意志,乃是特意。

知道这点,便好办了。

巨阳意志,尽管是仙尊意志,但也得遵守天地大道法则。意志被蛊虫凝造出来,同样受到相对应的蛊虫的克制。

因此,巨阳特意受到特意蛊的巨大克制。在方源的脑海中。还没有开始逞威施能,就被方源动用特意蛊。吞噬一空。

吞噬了这缕巨阳意志之后,特意蛊立即涨起肚皮,好像吃撑的胖子一般。

“哈哈,好意志。”方源赞叹一声,心中则暗惊,“好险!虽然受到克制,但毕竟是仙尊意志。纵然只是一缕,却也不容易对付!”

琉璃楼主令,乃是八十八角真阳楼的一部分。巨阳意志通过这个楼主令,袭击方源。但楼主令宛若通道,这个通道实在太小了,只能通过这么一缕巨阳意志。

若是通道再大点,局面又不一样了。

特意蛊吃下的这缕意志,和真阳楼中的巨阳意志相比,根本微不足道。

被方源小小地暗算一把,巨阳意志怒不可遏:“作弊者,你居然胆敢亵渎我巨阳的荣光!我要把你碾成碎片!”

轰隆!

真阳楼震动起来,绽放各色霞光,直冲九霄,威势浩荡绝伦,惊天动地。

巨阳仙尊虽然已经身死,单靠巨阳意志,还无法催动八十八角真阳楼。

但在八十八角真阳楼中,巨阳仙尊留下了大量的九转黄杏仙元!

野蛊可以直接吸收空气中的元气,被炼化的蛊虫丧失了这个能力,吸收的是蛊师真元。

八十八角真阳楼乃是仙蛊屋,听命巨阳意志,吸收巨阳仙尊留下来的仙元,自发催动,简直是轻而易举,理所当然!

“快跑!”眼看着八十八角真阳楼就要发威,墨瑶意志在方源脑海中焦急示警。

方源是炼化真阳楼的卑鄙贼子,巨阳意志一苏醒,就将其列为死敌。

方源并非巨阳血脉,这个仇恨太大了,方源直接被死死盯住,身陷绝境,空前的杀机即将降临!

“不着急。”危难当头,方源却显得云淡风轻。

他手中紧紧握住琉璃楼主令。

这块令牌,并不简单!

它乃是消耗了仙蛊,从楼主令的基础上加工而得。耗费中洲蛊仙,几大超级势力,数千年的准备和心血,乃是一把关键钥匙。

“死了还要作怪,哼,这都是什么年代了!”方源冷哼一声,毫无畏惧,心念一动,通过琉璃楼主令,发动埋伏。

真阳楼猛地剧震一下,漫天的烟光彩霞顷刻间崩散大半。正当凝聚的楼层幻影,宛若大风下的烛光,消散崩解。

一切戛然而止。

真阳楼中,巨阳意志发出惊天的怒吼,受到无数特意蛊的围攻剿杀。

这些特意蛊,都是中洲蛊仙准备上千年,不断累积,专门为了对付巨阳意志而布置的大杀手。

巨阳意志被彻底针对,一时间,群狼环伺。

他第一时间,就想催动蛊虫,对这些特意蛊进行剿杀。

八十八角真阳楼是仙蛊屋,是由许多仙蛊、凡蛊组合而成的固态大杀招。自然有清剿内敌的手段。

但中洲蛊仙研究至深,早就算计到这点。

配合特意蛊发动的,还有其他惊人手段,竟将巨阳仙尊留下来的黄杏仙元一齐封印。

真阳楼得不到仙元驱动,立即哑火。纵然巨阳意志暴跳如雷,也没有立竿见影的法子,一时间陷入困境。

真阳楼陷入死寂,方源放下心来。

前世中洲蛊仙就是靠此手段,斩灭巨阳意志,破坏八十八角真阳楼。

“可惜,我还是凡人,不是蛊仙,没法再从这里获取利益。”方源心中微微遗憾。

这是巨阳意志,最为虚弱的时候,但是他已经插不上手了。

其实,就算他成为蛊仙,单凭一位蛊仙战力,也讨不了好。中洲蛊仙进攻此地,一共有十一人,各个都是强者精锐,但胜利凯旋的只有三位。

不过,他们的牺牲是值得的,战果丰硕。

真阳楼乃是北原精神象征,骤然毁灭,北原上下遭受重创,士气一度降至谷底,在战场上节节败退。若非已是蛊仙的马鸿运,作为支柱,屡次在最关键的时候,发挥关键作用,否则北原早就被中洲吞并了。

对于现在的方源来说,至少一段时间内,他是安全的。

“真阳楼刚刚怎么回事?”

“难道说,太白云生升仙影响太大,连八十八角真阳楼都被他波及了吗?”

“王庭福地禁止蛊仙出入,太白云生在此升仙,是极其罕见的事件。这是否会破坏巨阳先祖当年的布置呢?”

真阳楼的异动,让不明真相的众人惊疑不定,纷纷猜测。

真阳楼陷入死寂,暂时别无动静。很快,众人的目光又重新被太白云生吸引。

太白云生悬浮在天地之间,不断吸纳天气、地气,结合自己的人气。

三气平衡,灌注于曾经的空窍之处。

一种玄奇妙极的变化,正在他的身上发生。

不管是他的身躯肉体,还是魂魄精神,都在受到天地之气的淬炼。他的整个生命本质,得到升华。

升仙三步,第一碎窍,第二纳气。

太白云生得益于完整的蛊仙传承,对此中关窍清楚明了,至今为此他发挥稳定,没有大碍。

但,升仙的过程,怎么可能只有这些障碍?

相比较磅礴的天气、浩荡的地气,人气只来源于太白云生一人。

是以,天地二气浩荡无穷,人气量少。

三气平衡,多余的天气、地气,就来不及吸纳,被排斥一旁。

随着时间推移,这些天气、地气,各自积压在苍穹和地面,不断积压凝聚,由量变发生质变。

“太白云生积累太浑厚了,吸引了这么多的天气、地气。现在这些天气、地气,要凝成灾劫了!”耶律桑双目烁烁闪光。

蛊师积累越浑厚,升仙的难度就越大。但若是成功,日后的成就也就越高。

这其中的风险和利益,是成正比的。

黑楼兰更无言语,一直紧紧盯着,尽全力观摩参照。

天劫、地灾,是如何形成的?

眼前的一幕,将形成的过程,全部展现在众人眼帘之中。

金黄色的地气,越加深厚,浊气凝形,渐渐化为一头独角大虎。而青色天气,则凝成无数巨大的碧玉柳叶。每一根柳叶,都如舟船大小。

独角大虎,大如小山,坐卧于地,昂首待跃。

碧玉柳叶,摇摇晃晃,铺陈于天,蠢蠢欲动。

“这是地灾峥虎、天劫柳风!”方源目光一闪。

太白云生止住狂笑,脸色凝重。

众人离得这么远,都能感觉到地灾、天劫的无边威势,更何况处于风口浪尖,天劫、地灾的目标太白云生呢?

一时间,观望的众人默然无语,氛围沉重。

而王庭地灵却在大笑:“脱困之机,就在此刻!”

(未完待续)

\end{this_body}

