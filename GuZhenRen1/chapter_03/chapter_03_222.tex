\newsection{十大凶灾乱歧牙}    %第二百二十二节:十大凶灾乱歧牙

\begin{this_body}

%1
清辉之天气,黄金之地气,越来越多,缠绕方源周身。

%2
黑楼兰等人徐徐退去。

%3
长河巨阳意志,怒吼连连,一时间却也不敢莽撞上前。

%4
它和方源激斗大战,体积不足原先一半。

%5
它心中焦急万分:“可恶!除非有主体意志支援援兵进来,否则单靠我是收拾不了方源的。”

%6
战场陷入短暂的安宁。

%7
方源见局势终如自己所料,如重释负地吐出一口浊气。

%8
八十八角真阳楼之争,几番波折,到如今局势已渐明朗。

%9
方源如悬崖边上走钢丝,狂风阵阵,好几次落入险境,丧失主动权。若非他精于谋算,手段众多,绝撑不到此刻。

%10
尤其是这一次,方源以命搏命,同归于尽的勇烈,这才堪堪将主动权夺回。

%11
“北原之行,终于到了最关键的时刻!接下来升仙至关重要,几乎决定胜利的归属!”方源强振精神,正如他之前所言,真正的战斗才刚刚开始。

%12
“师弟,你可要千万当心!”太白云生语气沉重,目光焦虑,心中更是担忧不已。

%13
他刚刚升仙,对升仙过程中的艰险十分清楚。他明白:若非巨阳意志助他,他根本不可能成功。哪怕他积累了一生,底蕴厚重!

%14
“唉,师弟年少气盛,万万不该做此选择!若是我提前得知,说什么也不能让他这样做!但现在说什么都晚了。唉,只盼他得到过恩师的指点,侥幸成功啊……”太白云生心中哀叹。

%15
方源和太白云生对视一眼,点了点头。

%16
随后,他缓缓闭上双眼,将心神投入体内。

%17
寻常蛊师、蛊仙,终其一生,都只有一个空窍。但方源曾在三王山险死还生,捉住机缘。因而。用过第二空窍仙蛊,身具两大空窍。

%18
经过王庭之战,他身上的空窍皆是五转巅峰,晶紫真元!

%19
蛊师升仙有三大步骤,第一步就是碎窍。

%20
方源碎的是第二空窍。

%21
蛊师一旦碎窍,就只有前进,再无退路。那是因为。他们只有一个空窍。

%22
但方源有两个。

%23
也就是说,哪怕此次渡劫失败,方源失去一窍,却仍旧还能保留一个空窍。

%24
保留一个空窍,就能保存性命。

%25
这就是第二空窍的好处之一。

%26
太白云生不知道这个秘密,若是知道。心中的担忧必会消散一大半。

%27
咔嚓!

%28
咔嚓!

%29
咔嚓!

%30
一道道雷电,接连不断,劈中巨阳意志。

%31
“方源小贼,你不得好死啊!!”巨阳意志怒吼声声,硬抗劫电,金沙般的意念一次次爆散。

%32
这雷电非同小可,呈现三叉戟形状。威力超凡。乃是十大凶宅之一。

%33
原本雪殇劫电并不频繁,至少间隔数十个呼吸,才会形成一次。但此刻,方源升仙碎窍,勾动天地二气,形成新的天灾地劫。

%34
天灾地劫又融汇在十年暴风雪灾中,顿时助长了雪灾的威力。

%35
雪殇劫电就好像是磕了药似的,兴奋起来。连连劈下,之间间隔不到一个呼吸!

%36
每一次抵御,巨阳意志都要付出沉重代价。

%37
但它偏偏不能一走了之。

%38
八十八角真阳楼需要它的保护。

%39
现在的八十八角真阳楼,几乎完全停止了运转,徒具其形。若是巨阳意志的保护一撤销,一眨眼的功夫,就会让真阳楼毁灭在恐怖的天劫地灾之中。

%40
八十八角真阳楼乃是巨阳仙尊留下的传承重宝。巨阳意志就是此楼的管理者。身具的使命,让巨阳意志只能硬着头皮死抗天灾地劫。

%41
如此一来,它却是成了方源渡劫时的保姆了。

%42
为暗算自己的仇敌防护,帮助他渡劫升仙。形势所逼之下,不得不这么做的巨阳意志,心中的憋屈与郁火可想而知。

%43
“方源小贼,等我撑过天灾地劫,我要将你抽经扒皮,我要喝你的血,吃你的肉啊!”巨阳意志咆哮,即便是雷鸣轰响不断,也盖不住它的怒吼。

%44
但很快,它的怒吼停下来,转为惊愕:“这,这是乱歧牙!”

%45
只见电闪雷鸣的天空中,出现十八道漩涡。

%46
漩涡飞速旋转,漩涡边缘向周围迅速扩张。

%47
每一道漆黑漩涡中,都有一根巨大的雪白兽牙,慢慢凝聚形态。

%48
一共十八道兽牙,洁白如雪,根根巨大无朋,粗如古树,长达十丈余。

%49
兽牙蓄势待发,强烈的危机感从巨阳意志的心头升起。

%50
同样是十大凶灾之一的乱歧牙,排位还在雪殇劫电之上,威力更加恐怖。

%51
“绝不能让十八根乱歧牙同时轰击!到那时,恐怕连我都抵挡不住……必须提前出击!!”巨阳意志战斗经验丰富,立即选择果断出击。

%52
它毫无犹豫,眨眼间分化出十条长龙般的意志,呼啸盘旋,悍勇无端地冲向天劫上空。

%53
然而冲锋的路上,充斥风刃冰雹,雪殇劫电更是咔嚓不断。十条长龙意志,在途中便折损半数,只剩下五条,成功冲撞到漩涡。

%54
然而冲撞的成果,也并不显著。

%55
只有三条成功,毁去漩涡。剩下的两条长龙意志,在冲突折损严重,被它们撞中的漩涡一阵摇晃之后,又复原过来。

%56
劫电咔嚓咔嚓,轰击的速度居然比先前又快了三分!仿佛是对巨阳意志的嘲笑。

%57
“可恶!可恨!”巨阳意志无奈,只得又分化十道意志长龙,再度展开惨烈的冲锋。

%58
真传秘境中,却仍旧一片安宁。

%59
方源闭着双目,面容平静,呼吸悠然。

%60
他的心神,大半都投放在碎裂的第二空窍。

%61
“我的第一空窍,本命蛊乃是春秋蝉。此蛊事关重大,若是升仙,就是宙道蛊仙。可惜我对宙道知之甚少,并且身具仙蛊渡劫,我也没有经验。不能轻易冒险。”

%62
“唯有第二空窍,本命蛊是全力以赴蛊,成就力道蛊仙。重生以来,我对力道多有研究,虽然没有血道擅长,但这份底蕴却足以升仙了。”

%63
此刻的第二空窍,已经破碎得一塌糊涂。形成漏洞,吸引着天气、地气的汇入。

%64
太白云生一生不过百岁,积累的底蕴就能升仙。方源有五百年经验,所知更强于拥有蛊仙传承的太白云生,升仙自然绰绰有余。

%65
“只是现在,并非升仙的最佳时刻啊。”方源暗暗苦笑。

%66
方源的力道修为。还没有达到极致。身上只有百钧之力,骨头是无常蛊,皮肉却是龟玉狼皮,一套力道蛊虫也不都是五转。

%67
这些都关乎升仙后的潜力,仙窍的大小,能尽善尽美自然最好。方源原本的打算,也是积累达到最浑厚的阶段。然后做足充分准备,再冒险升仙。

%68
可惜计划赶不上变化。

%69
局势所逼,方源也只能这样做了。

%70
天地二气交汇入体,与此同时,方源的身上也弥漫出浓厚的白色人气。

%71
人气浓郁,团成一个巨大气球,将方源整个身躯都笼罩遮掩。

%72
黑楼兰见此,瞳孔不禁微微一缩。

%73
耶律桑难以置信地道:“怎么可能。他竟然有这样浓郁的人气,这比太白云生还要强盛!”

%74
“想不到师弟的积累,如此深厚。如此可见,师弟此次冲仙也是早有准备,并非冲动冒险了。”太白云生也抱以诧异的目光,心中微安。

%75
清辉气,金黄沉重的地气。再加上洁白如雪的人气,三者交汇在一起,形成一个混色气团。

%76
如此,便已至升仙第二步——纳气。

%77
“可恶。小贼纳气了。援军怎么还没有来?”长河巨阳意志心急如焚。

%78
它不敢轻举妄动。

%79
一旦被沾染了天气,就会消融成空。被地气侵蚀,则会化为石块身亡。

%80
作为三气交点的方源,更是险而又险。

%81
他必须时刻做到三气的平衡,天气、地气过多,就是身死当场。哪怕是人气,一旦失衡过多,也会引发自爆。

%82
然而富贵险中求,越大的风险就是越大的机遇。

%83
三气平衡,相互交融,带给方源源源不绝的灵感之光。

%84
天地是孕养万物的基石,天地之气相互感应,大道奥妙就会充盈方源的心头。

%85
三气激荡,方源身处其中,已经撤销六臂天尸王的杀招,还原本来面目。

%86
他身上的伤势,迅速恢复。痊愈速度,比六臂天尸王状态下还要快上数倍!

%87
三气洗涤他的身躯,不仅是肉体,还有魂魄、精神、意念都在不断地升华。

%88
像是被冲刷洗净,重归婴儿的纯净无暇。方源的双眸变得越加清澈,黑白分明。一块块肌肉被重新凝塑,身上的伤疤自然脱落,露出全新的肌肤。头发也在疯长,从原先的短发,长成长发。长及腰间之后,又自然崩断。生长、崩断,如此循环往复……

%89
不管是体格、骨骼,还是眉宇、五官,都在发生着微妙的变化。

%90
与此同时,前世五百年以及重生以来的一幕幕经历,都在他的脑海中快速闪现。

%91
在极短的时间内,方源回顾了自己的一生。

%92
一股股难言的,复杂的情怀,以往都沉淀于记忆的深处,此刻又重新翻腾上来,充斥方源的心田。

%93
记忆中的,那些以往桎梏他的修行难题,也有如神助般,得到一个个的完美解答。

%94
这一刻,天地无私,敞开胸怀,将大道奥秘传授给方源。

%95
方源一面要应对自己的情绪,一面要珍惜良机,多加体悟大道奥妙,还要时刻不忘平衡三气。三气失衡,就是身死。

%96
多少蛊师升仙,败在了这一步!

%97
ps:通知,将于十月七号正式并群。

\end{this_body}

