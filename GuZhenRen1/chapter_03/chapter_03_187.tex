\newsection{影子深黑}    %第一百八十七节:影子深黑

\begin{this_body}

%1
方源低下头,注视着晕死过去的太白云生。

%2
他便是太白云生昏死过去的凶手。

%3
此刻,太白云生浑身浴血,躺在地上,深陷的眼窝,双目紧闭。

%4
他伤痕累累,雪白的发须凝结着血浆,早已不复北原第一治疗蛊师的风范。

%5
仅仅几步之遥,主殿大门外,海量的血兽咆哮着,嘈杂的声浪不断激打过来。

%6
方源抬起头,盯着它们,轻声开口:“闭嘴。”

%7
霎时间,门外死寂。

%8
血兽们纷纷闭嘴,宛若乖巧的猫狗,伏跪在地上,一动不动。

%9
方源利用六角楼主令,掌控了这一层,他就是这里的主人,这些血兽自然也受他的操纵了,听凭他的心念,指挥如意。

%10
方源闭上双眼,心神投入此层,静静感受。

%11
此刻,在这道关卡中,还留有不少蛊师。

%12
这些蛊师,一部分是太白云生拉动进来的蛊师,但在之前的战斗中被淘汰下来。还有一部分,则是各方势力的眼线,都是侦察蛊师。他们来源于黑家、马家、耶律家各个势力,关注着太白云生此行的成败。

%13
至于这座主殿当中,除去地上那些干枯破烂的尸体,只剩下方源和太白云生两个人。

%14
方源关上主殿大门,半蹲在地上,伸出右手掌,一把抓住太白云生的头颅。

%15
蛊虫早就准备好了,他接连催动。

%16
很快,太白云生的脑门上,亮起微微的白光,成为幽暗的大殿中唯一的光源。

%17
白光越来越盛,太白云生的脸上,渐渐浮现出痛苦的神色,眉头越皱越深。

%18
酝酿片刻之后,方源陡然睁开双眼!

%19
他的双眼。没有瞳孔,只有一片眼白。

%20
眼白绽放三尺微光,与此同时,大量的画面在方源的脑海中显现。

%21
太白云生从老年回溯到年轻时的记忆,都被方源提取出来。

%22
……

%23
一位老人行走于北原,天苍苍野茫茫,风吹草低。狼群嚎叫。

%24
……

%25
“老先生您的救命之恩,我们兄弟俩没齿难忘!”高扬、朱宰一齐跪倒在他的脚下。

%26
……

%27
一位紫发老乞丐,裂开嘴,露出仅有的几颗牙齿,怪笑道:“你要想成为什么样的蛊师呢?嘿嘿嘿,我这里恰好有三份完整的传承!”

%28
……

%29
“嗯。这个小子长得不错,就选他了。”墨人城中,一位墨人指着少年时期的太白云生,哈哈大笑道。

%30
再往前,更加年轻的时候……

%31
“为什么,你为什么要背叛我?!”新婚大喜之夜,太白云生踉跄而倒。帐外传来震天的喊杀声。

%32
而他的妻子一脸冷漠和仇恨,慢慢逼近他,目泛凶光,咬牙切齿:“太白云生,你要恨就恨你的父母,是他们吞并了我的部族,杀害了我的父母,我要为他们报仇!”

%33
童年时期……

%34
“我的儿子。你可是我太白部族的下一任族长!不准哭,不要再同情心泛滥了!要在北原生存下去,我们的心得硬起来!将来,你要领导我们太白一族啊。”父亲十分严厉地训斥着。

%35
……

%36
“啊啊啊……”方源痛得大吼。

%37
脑海中,不断闪现的画面,叙述了太白云生的传奇一生。如此宏大的信息,对方源的头脑是巨大的冲击和伤害。

%38
好在画面并非无穷无尽。终有结束的时候。

%39
阅尽太白云生的一生,方源立即停下蛊虫,一屁股坐在地上。

%40
他呼呼地喘着粗气,浑身都是大汗。良久。他的瞳孔才恢复焦距。

%41
搜魂,绝非意事。尤其是方源顾及太白云生的安全,不想伤害他,因此只能自己承担大部分的压力。

%42
停止了搜魂,太白云生仍旧昏睡着,但原先紧皱的眉头,却舒展开来。呼吸平稳,神情安详。

%43
反而方源的眉头,则微皱起来。

%44
“没有找到啊!”他遗憾长叹。

%45
“没有找到什么?”脑海中,墨瑶意志禁不住疑惑而发问。

%46
方源的这一切举动,都令她好奇。

%47
方源没有答她,只是眉头皱得更深。关于江山如故蛊的重大计划,他怎么可能告诉墨瑶呢?

%48
江山如故蛊是太白云生成仙之后,才拥有的仙蛊。

%49
有传闻说:此蛊乃是太白云生成仙之时,天地交感,灵光爆发,自发凝练而成。

%50
但还有一个可能,就是太白云生的脑海中,本就有江山如故蛊的仙方。

%51
如果真有蛊方,那么方源大可盗取蛊方,以及江如故、山如故两蛊,带到琅琊福地中去,叫琅琊地灵出手,替他炼制仙蛊。

%52
这样一来,他就不用虎口夺食,危险性大降。

%53
但方源这一次搜魂之后,结果糟糕。

%54
方源没有搜出江山如故的仙蛊方,这就说明传闻是真的。江山如故蛊,的确是太白云生在成仙之际,天地交感而得。

%55
也就意味着:方源要取得此仙蛊,就得从成仙的太白云生手中,抢夺此蛊。

%56
方源还不是蛊仙,以凡战仙,方源印象中还未有任何成功的例子,无疑比登天还难!

%57
但还能有什么办法呢?

%58
当初的三个选择,这个已经是最容易的一道。时间、精力都投入到这个计划中,方源虽然也没有把握,但也只有积极准备,冒险一试了!

%59
……

%60
八十八角真阳楼外,抖现太白云生的身影。

%61
“出来了,出来了!”

%62
“结果如何?有人看到太白云生杀进了主殿。”

%63
“不好,太白云生大人一动不动,似乎昏死过去了!”

%64
周围的蛊师,立即围拢过去。

%65
打量一眼,众人脸色微变。太白云生身上伤势沉重无比,让他们暗暗心惊。

%66
“还有呼吸!”一人伸出手指,探了探太白云生的鼻息,高喊起来,“快。谁是治疗蛊师,快来稳住老大人的伤势!”

%67
“我来,我来!”

%68
“我也是治疗蛊师!!”

%69
许多治疗蛊师纷纷主动出手,太白云生的威望和仁厚之名,早已经深入人心。

%70
毫无疑问地讲,他比黑楼兰、常山阴更得人心。

%71
“就连太白云生大人,都受了这么严重的伤势。唉。这次大举闯关,恐怕是失败了。”有人叹息。

%72
“闯关的时间已经结束了,大部分的蛊师都没有回来,这次伤亡太过惨重!”

%73
“你们谁看到高扬、朱宰两位大人出来了?”有人惊觉,忽问。

%74
众人四处张望,旋即面面相觑。

%75
没有人看到高扬、朱宰的身影。而八十八角真阳楼中,那道关卡仍旧存在。只是短时间里不让蛊师再进入。

%76
这意味着什么,众人心中都是雪亮。

%77
八十八角真阳楼凝聚至今,已经有五位五转强者牺牲了。

%78
如此沉重的伤亡,让广场陷入了一片沉寂当中。

%79
当太白云生睁开双眼时,发现自己已经躺在床榻之上,浑身虚弱乏力。竟连起身都困难。

%80
看到他睁开双眼,一旁伺候的丫鬟,立即惊喜地叫出声来:“老先生,您醒了,您终于醒了!快来人呐,快来人呐,老先生醒了!”

%81
很快,太白云生就听到一连串急冲冲的脚步声。

%82
一群治疗蛊师。来到他的身边,一齐为他检查身体。

%83
“家老大人,您放心,您的伤势已无大碍。只是您年岁大了,这次受伤颇重,伤及根本。今后须得注意保养,尤其是最近几个月。身子骨虚不受补,需要静静安养才是。”治疗蛊师的领头,温声劝慰道。

%84
太白云生为了拉起队伍闯关,不惜答应了黑楼兰。已经成为了黑家的外姓家老。

%85
太白云生眼神散漫,从苏醒过来就一阵发怔,听了这话,这才回复了一丝神采,他问道:“这是哪里?”

%86
“回家老大人的话,这里是黑楼兰大人的住所。自从老大人您闯关失利,险死还生,我家族长就十分关切,亲自将您接到这里来治疗修养。下人们已经禀告去了,相信很快,族长大人就会来看您的。”依旧是那位首领答道。

%87
“闯关失利,险死还生?”太白云生微微皱起眉头,脑海中的记忆开始复苏,他回忆起了最后那一幕——

%88
他费尽最后一丝真元,催鼓起防御蛊,在血兽的围杀中成功地挤进了主殿。

%89
但随后不久,他就昏死过去,失去了知觉!

%90
然后醒来时,就发现自己躺在了这里。

%91
“这么说,我真的是闯关失败了?!”太白云生语调陡然一扬,目光倏地变得尖锐无比。

%92
“家老大人……”围在床边的一群治疗蛊师,相互对视,想要劝说安慰,却都说不出口。

%93
于是,他们只好都低下了头。

%94
房间中,一片安静。

%95
太白云生目光发直,沉默了好一会儿,忽然仰头大笑:“哈哈,原来老夫失败了。功亏一篑,功亏一篑啊!”

%96
他起不了身,只能用手掌使劲地拍打床边,发狂大笑。

%97
“老大人,老大人!”治疗蛊师们慌了,连忙相劝。

%98
“可怜我朱宰、高扬,为保护老夫牺牲了生命!”太白云生双眼泪水横流,他的笑声充满了悲痛。

%99
“家老大人节哀,人力有时穷,家老大人您已经尽力了!”

%100
“家老大人,您能活着出来已经是万幸了。”

%101
“人死不能复生,老先生还请节哀顺便啊……”

%102
众人你一言我一语,相劝不止。

%103
但这些话,听在太白云生的耳里,却充满了讽刺的意味。像是一根根针一样,扎进他的心里。

%104
在最后关头,太白云生留下真元,没有选择救下朱宰、高扬,而是为了自己,催动了防御蛊,闯进了主殿当中。

%105
是他,为了自己私欲,视同伴的牺牲而不顾。

%106
这还是太白云生吗?

%107
这还是那位北原公认崇敬,救死扶伤,救治世人,消除疾苦的太白云生吗?

%108
为什么自己会这么做?

%109
偏偏在那个紧要的关头,自己根本就不假思索。选择了这样做!

%110
故意牺牲高扬朱宰,为自己换取机会,为的就是通关奖励的那只十五年寿蛊!为的就是自己的苟且偷生!

%111
这个选择,让太白云生对自己感到陌生,感到羞愧,感到自卑,感到悔恨!

%112
当初的不假思索、毫不犹豫。现在则化成道德的皮鞭,拷问他的灵魂,鞭笞他的良心!

%113
太白云生痛苦地闭上双眼,双手紧紧握拳。

%114
“族长大人到——!”

%115
“属下拜见族长大人。”

%116
一屋人都跪倒在地,黑楼兰面带微笑,来到太白云生的床边。

%117
见到太白云生痛苦的模样。黑楼兰眉头轻轻一皱,旋即又舒展开来:“太白家老,很高兴你能苏醒。情况我已经听说了,你和高扬、朱宰等人已然尽显我北原男儿的英勇,虽败犹荣!只要总结教训,将来必能打通此关,反败为胜。洗刷耻辱!”

%118
太白云生却没有睁眼,一言不发,神色痛苦。

%119
他已经想明白失败的缘由了。

%120
他进入主殿之后,成功脱离了血兽的围杀,倒在主殿中。但成功之后他狂喜大笑,心境大起大落,再加上身负重伤,因此昏迷过去。

%121
闯荡此关。是受时间限制的。

%122
时间用尽,昏迷中的他和其他外围的蛊师一样,都被强行传送出来了。

%123
明明距离成功,只有一步之遥,结果却因为昏迷而失败。

%124
但如此讽刺的结果,并非太白云生心中的痛苦来源!

%125
他的痛苦,在于他为一己私欲背弃同伴。

%126
这还是他太白云生吗?

%127
记忆中的一幕幕。又翻腾上来。

%128
从小到大,他一直坚信爱的力量。

%129
他从孩童时起,就颇有仁名。

%130
太白一族吞并其他部族,当他看到童年时的玩伴要面临成为女奴的悲惨命运时。是他开口,要娶她为妻。因此也宽恕了一批俘虏。

%131
但新婚之夜,他的妻子背叛了他。俘虏们联系外敌,突袭他的部族,他的父母因此而亡。

%132
之后的奴隶生涯,艰辛困苦,他一直受到内心的强烈煎熬。

%133
终于有一天,他好心为了一个素不相识的老乞丐舀了一碗水。老乞丐传给他三个仙道传承作为选择。

%134
第一份传承,能让人浴火踏焰,睥睨凡尘。

%135
第二份传承,能令人掌风浮空,逍遥天下。

%136
第三份传承,则是穿越生死,扶助苍生。

%137
太白云生选择了第三份。

%138
那一刻,他仿佛在黑暗中找到了光明,内心不再煎熬,他无悔,他浴火重生!

%139
过去这么多年,老乞丐的笑声,犹在耳边。

%140
穿越生死,扶助苍生也成了他的人生信条。

%141
其后的生涯,他真的这么做了。

%142
他收获了无数人的感激,他的仁名广为传播,他的光辉照亮整个北原。

%143
他是一个活生生的传奇。

%144
但现在!

%145
他失败了!

%146
他的失败,不在于没有获得寿蛊。而在于背弃同伴,背弃了自己的人生信条!!

%147
偏偏这一切,都还是他自己不假思索地去做的。

%148
他用了几乎一生,来竖立和实践自己的人生准则。然后在那一刻,他自己将这个准则摧跨。

%149
他见识到了自己的另一面,自己的自私。

%150
他曾经以为,自己就是众人眼中的那个人——夕阳下行走于草原,扶助苍生,悬壶救世。

%151
但现在他的心中,这个形象,已经渐近渐远,步履蹒跚。

%152
在落日的余晖中,他的身影拖得老长。

%153
影子深黑……

\end{this_body}

