\newsection{影鸦}    %第二十四节:影鸦

\begin{this_body}

穿过葵海的过程有惊无险。

葛谣回望后面,长长地吐出一口浊气,提着的心终于踏实地放下。

“没想到竟然这么简单,就穿过了鬼脸葵海。”

平复了心境,少女又不免看向身边的这些草人傀儡。

这些傀儡已经只剩下数十头,围绕在周围,组成一圈薄弱的防御。

葛谣从小到大都未见过这种蛊,今天算是开了眼界。

“这些蛊,战力虽然不怎样,但是却是最优良的炮灰。常山阴果然是有备而来的。”想到这里,少女又转过明眸,打量身边之人。

和方源越是相处,葛谣心中的好奇就越重,探究心就越强。

“他究竟是怎样的一个人?深入腐毒草原,又有什么目的呢?常山阴,常山阴……这个名字,我真的听说过。哎呀!”

少女忽然脸上涨得通红。

刚刚穿越葵海的时候,方源牵着她的手。那些鬼叫声,漂浮的鬼脸让她害怕,不由自主地靠近方源,渐渐地已经将他的胳膊抱在怀里,竟然都没有察觉。

葛谣连忙撒开方源的胳膊,挣脱开他的手。

拉开安全的距离,方源慢慢地停下脚步,转头看着身后的葵花。

“这些鬼叫蛊,鬼脸蛊倒都是不错的蛊,可惜我没有专门的蛊虫来捕捉它们。”

捕捉野生的蛊虫,也需要特定的手段。这两种蛊虫,都隶属魂道,捕捉它们的蛊更是特殊得很。

“是时候了。”方源目光一凝,将心中的遗憾抛去,将目光转向肩头的仙蛊定仙游。

仙蛊气息强盛,无法存入凡人空窍。春秋蝉是因为虚弱。才能勉强装着。

尽管方源一直令定仙游趴在自己的肩头,不要随意动弹,但仙蛊的气息仍旧在逸散着。如果被蛊仙察觉,就会引来杀身之祸!

好在方源为此,早有所准备。

他取出皓珠蛊。

“去吧。”他灌注真元,皓珠蛊顿时化作一团柔和的白光,飘飞到定仙游的身上,将它罩住。

皓珠蛊为四转存储蛊,专门用来封印蛊虫。使得蛊虫陷入沉眠,易于保存。

方源几乎消耗了全部真元,这才成功地将定仙游蛊封印。

定仙游蛊形如翠玉蝴蝶,封印在拳头大小的透明圆珠当中。但它是仙蛊,气息仍旧透过这颗明珠。逸散出来。

只是,比之前要减弱许多倍。

方源也不意外,这才是第一步罢了。

接下来的路途,顺利了许多。

许是因为那片葵海的阻拦,毒须狼群没有再出现。

两人继续深入,草原上的毒雾越加浓密,已经可以清晰地看到空气中飘逸的紫雾。

当两人开始咳嗽的时候。他们停下脚步,分别取出蛊虫,清理掉身上积累的毒素。

越往草原深处跋涉,毒雾就越是浓密。两人停下的次数也渐渐频繁。

心性活泼的葛谣,为避免说话过程中呼吸过多的毒雾,也开始保持沉默。

渐渐的,紫色毒雾已经开始渐渐遮挡视野。

“我们还要深入多远啊?”葛谣终于忍不住发问。

腐毒草原的深处。是生命的禁区。越是深入,里面的猛兽就越强大。很多前往探索的高手。都死在这里,有来无回。其中不乏三转巅峰,四转的强者。

“就快了。”方源淡淡地回答一句,脚步放慢,直至停下。

“到了吗?难道就是这里?”葛谣高兴地问道。

方源不发一言,谨慎地蹲下来,扒开地上长着的扭曲怪异的毒草丛,露出被草丛覆盖着的洞口。

这洞口,大如海碗,边口光滑。洞内幽深,一片黑暗。

葛谣看到这里,顿时目光一凝,呼吸转促:“这,这是地刺鼠钻出来的洞!地刺鼠成千上万,汇集成群,隐藏在草地下。只要地面稍微有一丝震动,它们就会从地下发出攻击。它们的脑袋就像是钢梭,一射出来,就能钻破人的脚面。甚至连铁蹄马的马蹄都会被轻易刺穿。”

“我们绝不能往前走了。一旦陷入包围,我们绝无生存的可能,只会被无数的地刺鼠淹没。而且一路上,我们没有碰到任何的毒须狼,这就说明这里的地刺鼠已经是霸主。说不定地底的地刺鼠王,就是一头万兽王!”

少女自幼就生活在草原上,对地刺鼠的强大,认识极为清晰。

“不,我的目的地,还在前方。”方源站起身来。

“常山阴!过度的勇武是自找死路。你是走不过去的,甚至刚刚走了几步远,就会遭到地鼠群的围攻。”葛谣赶紧劝告道。

但方源淡淡一笑:“谁说我要走过去?”

话音刚落,漆黑的骨翼在他背后生长而出。

“这是……”葛谣瞪大双眼,还不待她反应过来,方源已经一把抱起她。

在少女的惊呼声中,方源振翅而起,飞离地面。

葛谣只感觉一颗心提到了嗓子眼,风儿在她耳畔呼啸,她感觉自己仿佛腾云驾雾般,双脚虚不着地,这让她下意识地牢牢抱住方源的脖子。

过了一会儿,少女适应过来,在方源的怀中笑逐颜开:“常山阴,想不到你也能飞。阿爸有一只腾云蛊,小的时候也常常带我飞到空中玩耍呢。唉,可惜腾云蛊是四转蛊,不适合我用。就算我能用,阿爸也不允许,怕我从空中摔下来。”

葛谣感怀了一下,又不禁好奇地问道:“常山阴,你这是什么飞行蛊?为什么我从来没有见过?”

方源不答话。

少女并不甘心:“这是三转的鹰翼蛊吗?这种高度和速度,和鹰翼蛊差不多。但又好像不是。”

方源叹了一口气:“你的问题,实在太多了。与其发问,还不如把这些精力,用在前方的影鸦身上。”

“影鸦?”少女反应过来,便发现左前方有三只影鸦,正悄无声息的快速逼近。

影鸦大如鹰雕,浑身漆黑,飞行起来毫无动静,在这昏暗的腐毒草原上更是隐蔽。

少女的脸色顿时发白,声音打颤:“常山阴,你的飞行术究竟怎么样?实在不行,我们就飞到草地上防守好了。”

“放心。”方源的声音依旧平淡,“我双手抱着你,无法随意出手。接下来就看你的螺旋水箭射得准不准了。”

“什么?呀!”

少女还未听明白,方源就猛地振动双翼,悍然冲向那三头影鸦。

方源用实际行动,告诉了葛谣答案。

“这太疯狂了!他居然没有想到逃走,反而想要杀了这三头影鸦!”葛谣心头的震动,匆忙之间,发出两道水箭。

但这两道水箭,其中一道完全射空,另一道则擦着影鸦的翅膀而过。

“太慢了,再来!”方源一个漂亮的回旋,急速振翅,再次冲向影鸦。

“什么?喂,等等,不要硬拼啊。这里可不比地面,对方可是速度奇快的影鸦!”葛谣大叫。

两人三鸟在空中相互冲锋,快速逼近。

葛谣瞪大双眼看着,视野中一只影鸦迅速扩大,急速接近中,影鸦亮出钢刀般锋利的爪子。

眼看着利爪,就要抓到自己,葛谣吓得都浑身冰凉,手脚僵硬,动弹不得。

“要撞上了,我要死了!”就在她冒出这个念头的时候,方源忽然双翅一收,飞行的高度猛地下降,在影鸦的钢爪之下,擦身而过。

然后他的双翅猛地伸展出来,用力一振之后,在空中做出一个闪电般的转折,拔身飞上,窜到了影鸦的身后。

“快射!”方源断喝一声。

葛谣反应过来,下意识地伸手一指,飞出一道螺旋水箭。

方源的飞行技术太高超了,直接闪到了影鸦的背后。影鸦等若将后背,完全暴露给葛谣。

螺旋水箭顺利地射中影鸦,穿身而过,带出一溜血线。

影鸦顿时毙命,如断了线的风筝,摔落在地上,发出砰的闷响声。

草原沉寂了一下,瞬间地表微微震动,地洞中无数的地刺鼠,探出了脑袋。

那只影鸦的尸体,闪电般地被最近的几头地刺鼠瓜分,分成数份拖入地洞当中。

偌大的影鸦,瞬间消失,只剩下草地上的一滩血迹,以及几片内脏的小碎块。

这个场景,落在少女的眼中,顿时让她无比的紧张起来。真要摔到地上,就会立即遭到地刺鼠群的攻杀,绝对是十死无生的下场!

“你再想什么呢?还不快射!”方源的冷喝,打断少女的思绪,她匆忙出手,连续射出近十箭,这才将两只影鸦消灭。

两只影鸦落到地上,立即被地刺鼠群撕扯瓜分得干干净净,葛谣看得浑身冒出冷汗。

“你这准头太过差劲,还不赶紧用元石回复真元!”

在方源的喝斥声中,葛谣连忙取出一块元石,但忙中出错,这块元石脱了手,落下去,掉到了地上。

“你这个笨蛋!”

“对,对不起!”

少女的声音,带着一丝哭腔。

“专心,冷静!把你的水平正常发挥出来。接下来的影鸦肯定不少,我还要依靠你呢。”方源声音一缓。

“是,是。”葛谣连连点头,在方源的抚慰声中,心境慢慢平复。

\end{this_body}

