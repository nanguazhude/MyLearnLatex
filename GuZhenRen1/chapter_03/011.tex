\newsection{渡地灾(上)}    %第十一节:渡地灾(上)

\begin{this_body}

数月之后。

狐仙福地,狐群集结成千重军阵,紧紧地包裹着荡魂山。

方源背负双手,矗立在山巅之上,仰望着天空,满脸的凝重。

时间悄然逝去,今天便是第六次地灾降临之日!

饶是方源五百年前世,也是一位蛊仙,面对地灾,心中也颇不平静。

每一次地灾,都会越来越强,对于福地、蛊仙是生死攸关的严峻考验。方源执掌福地时,只剩下一年零三个月的时间。

这时间实在太短了,他只能尽量准备。凿开运河,调和水火是一;栽培狐群,大力繁殖是二;留着定仙游蛊,随时准备撤退是三。

至于漫空的云海,以及西部的那头魅蓝电影,他真的是有心无力。

微风渐渐停滞,远处的天空中,云海翻滚,一团光芒正在酝酿。

“来了。”方源瞳孔一缩,轻声喃喃。

云海之中,那团光芒陡然爆开,形成一扇恢弘的白光圆门,正对着广袤的福地。

光门闪耀着刺眼的光辉,一头巨大的怪兽,浑身黄褐色,好像是块巨大的岩石。从光门中徐徐而降。

“看这架势,荒兽之灾?!”方源眯起双眼,一瞬不瞬地盯着。

巨石不断地降落,悄无声息。

方源忍不住舔了舔干燥的嘴唇,心却是往下沉。

地灾有无数种,其中有荒兽成灾。

福地中,会出现一头或者多头荒兽。一齐冲击福地中枢,在福地中翻江倒海。尽情破坏。

如果不及时地剿灭它们,再广阔的福地也要被它们破坏毁灭。

“该死的,竟然是荒兽。但愿这头荒兽身上没有仙蛊寄生!”方源在心中忍不住咒骂一声。

荒兽若是身怀仙蛊,战力可超越普通的蛊仙!

这只神秘的荒兽,以一种完全不符合体型的轻巧之势,落到地上。

远看,它像是一方巨石,略显扁平。

但方源透过影像近观。才发现这巨石乃是一块黄褐色的甲壳,散发着金属的光泽。甲壳上,披着厚重的烂泥。

就在方源猜测,它究竟是个什么玩意儿的时候,一对庞大的螯足,仿佛是两只钢钳,从方块甲壳中伸出来。

紧接着。十八只细长的螯足,分别从两侧探出,落在地上,然后轻而易举地将它沉重的身躯撑高,远离地面。

“泥沼蟹!”看到这一幕,方源脱口而出。认出这只荒兽的真面目。

这是一只巨大的螃蟹,有山一样的雄阔身躯。当它撑起身体,高度能达到荡魂山的四分之一。

它的第一对螯足,比钢钳还要恐怖,轻轻一夹。就能断山石,剪蛟龙!

它其余的十八只螯足。虽然较第一对螯足瘦长纤细。但实际上,都比百年古木还要粗壮。

它的身上,寄生着大量的蛊虫,多为水、土两道蛊虫。有时候,甚至是整整一支蛊群。

“幸好狐仙福地中,仙元充沛!”方源咬了咬牙,心中不无庆幸。

早在泥沼蟹刚刚出现时,地灵就动手,施加天地伟力,禁锢了它一身的蛊虫。

不管是一转还是五转的蛊,都不能展现威能。

问题的关键是,这只荒兽身上有没有仙蛊。如果这只泥沼蟹拥有仙蛊,那又要看究竟是什么仙蛊。

仙蛊唯一,超凡脱俗,福地根本禁锢不住。

仙蛊,是影响整个大局的关键因素!

泥沼蟹完全伸展开肢足后,开始缓缓地开动身躯,向着荡魂山直行。

方源心念迅速地调动。大批的狐群,漫山遍野,仿佛潮水般,朝着荒兽涌去。

很快,它们就包围住了泥沼蟹。

爪牙在泥沼蟹的螯足上啃咬,身子骨坚实的金狐直接撞过去。

但泥沼蟹庞大无比,简直是个巨无霸,不断前进,普通的狐群无法阻止,反而被踩成肉酱。

方源面色冷酷,仍旧指挥着狐狸上前送死。

他繁衍这么多,就是要用来牺牲的。伤害都是积少成多,此刻哪怕阻止魂兽一丝前进的步伐也是好的。

但泥沼蟹以碾压之势,缓缓行进,岿然不动。好像是一座山峰在行走,对待脚下的狐群不管不问。

色彩缤纷的攻潮,打在泥沼蟹的身上,仿佛无数灿烂的烟花在绽放。

这是狐群中的百兽王、千兽王、万兽王在发力。它们身上寄生着许多蛊虫。

在群蛊的力量下,泥沼蟹身上的厚泥被尽数打落。

这只巨大的黄河搜有史以来,终于停顿了一下。

它忽然张开口器,喷吐出大量的泥沼。同时在腹部,像是开了无数的小闸门,黄色泥沼喷涌直下,仿佛黄泥瀑布一般。

泥沼落在草地上,瞬间形成大片的沼泽地。

黄泥当中,一只只造型奇特的螃蟹站起来。有的体型巨大,势如猛虎。有的螯足尖锐,如同钢针。有的肢节如八爪,速度奇快。

几乎眨眼睛,一支上百万的蟹军就成形了。

“果然是泥沼蟹!它可以随时随地自我受孕,生出无数的小螃蟹,形成大军。”方源脸色更加沉凝。

狐群和蟹军交锋,激烈地纠缠在一起。

狐狸大量缩减,损失惨重。蟹军的伤亡比狐群还要多数倍,但是荒兽不断地生产,海量的螃蟹源源不断。

方源连忙将潜伏在山外的狐群,都调动过来。

“幸亏我将发情蛊消耗光,大量的繁殖狐群,否则兵力绝对不够!”

这么短短的片刻,方源就感到脑袋眩晕。

他指挥的狐群数目,实在是太庞大了。即便他的魂魄。是常人的六倍,也吃不消。

有着螃蟹大军开道。泥沼蟹继续前进,恢复了先前的速度。

它身体两侧的螯足,交替落下,好像谱曲般,有优雅的韵律。

但它的脚下,却是惨烈的战场,血流成河,尸体堆叠。寸土寸血。

泥沼蟹敌我不分,每一只螯足踩下,都会爆发出一团惨烈的血浆。当螯足抬起时,地面上的深坑中,是狐皮肉酱,还有螃蟹的碎壳残肢。

这只荒兽的体格是如此的庞大雄阔。平心而论,它前进的速度并不快。

但也正因为如此。它带给人庞大的心理压力,看着它碾压过来,方源仿佛感受到死亡铡刀,正悬挂在他的脖子上。

“可恶啊!”方源咬牙切齿。

眼前的这头荒兽,是泥沼的君王。它浑身都是甲壳,常年潜伏在泥沼深处。连眼睛都退化个干净,根本没有一丝弱点。

方源调派狐群,竭力阻挡,但都没有用。

他只能眼睁睁地看着,泥沼蟹不断地接近!

“你能把他挪移走么?”方源忽的扭头。问向地灵小狐仙。

蛊仙不同,福地不同。地灵的威能也就有所差异。有的地灵可以挪移,譬如三王福地中的那头霸龟。有的地灵甚至不会。有的地灵可以呼风唤雨,有些地灵却可以随意地操纵时光的流速。

“我试试看。”小狐仙呼吸紧促,也感到巨大的心理压力,她调动仙元,拼尽全力,可爱的小脸憋的通红。

“啊呀呀呀!”她发出奶声奶气的娇喝声。

刷的一声轻响,巨大的泥沼蟹消失在原地,挪移到九千步之外。

“成功了!”小狐仙脸色潮红,喘息呻吟着。

方源也舒了一口气。

“主,主人,刚刚我足足用了一颗青提仙元。”小狐仙肉痛地汇报道。

“无妨。”方源冷着脸,继续调动万千狐狸,继续向泥沼蟹冲杀。

半盏茶的功夫,泥沼蟹再次冲杀近前。地灵不得不故技重施,将它挪移走。

又是一颗青提仙元损失了。

小狐仙肉痛无比,方源心也在滴血。

狐仙福地中,所有的青提仙元,只有七十八颗。方源用掉一颗,喂养了定仙游蛊。如今又用掉两颗,挪移了泥沼蟹。

他将来还得用这仙元,来炼蛊,来经营整片福地呢。

要用到仙元的地方很多,但狐仙已死,这些青提仙元就是无源之水,用多少就少多少,不能补充。

一些螃蟹,穿过疏漏的方向,冲杀上山。

方源看了冷哼一声,旋即命令地灵,开放出荡魂山的部分威能。

顿时,蟹军行走的地域,成了一片死域。无数的螃蟹顷刻死亡,身躯无损地躺在地上,魂魄却荡得粉碎,情形十分诡异。

“可惜,荡魂山的力量是对魂魄持久的杀伤,荒兽魂魄强大,足以支撑一时半刻。不能让它到达荡魂山,破坏这片珍贵唯一的秘禁之地!”

方源没有亲自参战。

就算是他动用那套力蛊,杀伤力也不足以破除泥沼蟹的甲壳。

更关键的是,泥沼蟹还没有使出仙蛊。方源也不知道,它到底有没有。

未知,是一种强烈的威慑,令方源不敢轻举妄动。

如此,泥沼蟹又攻杀进来。就在它第三次被挪移出去的当口,小狐仙的脸色骤然变化。

不等方源反应,她猛地伸手,抓住方源的胳膊,然后一起消失在原地。

下一刻,一道凶狠的霹雳闪电,打在他们原本站立的地方。

轰隆!

一声炸响,山石飞溅。

电光一顿,猛地折返上去。

化为一道人形闪电,发出凄厉的嗥叫。

行凶的,正是魅蓝电影!

(ps:需要解释一下,上一节写的不是讽刺,而是悲壮。为了明天,写的是石人,是狐群,也是方源。包括今天这两节,也是如此。与天斗,与地斗,与人斗。在残酷中求生存的悲壮。被他人玩弄,被天地玩弄之后,发出的不甘呐喊——我们不断地追求强大,但面对命运和现实时,又总是如此的弱小!!!)(未完待续。如果您喜欢这部作品,欢迎您来起点投推荐票、月票,您的支持,就是我最大的动力。手机用户请到阅读。)

\end{this_body}

