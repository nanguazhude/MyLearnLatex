\newsection{雪爪飞舞楼显化}    %第一百五十节:雪爪飞舞楼显化

\begin{this_body}

元泉之地,早已选择妥当。

石武带着泉蛋蛊,和蛮多来到目的地,准备妥当之后,便在此调度真元。先是灌输到持久蛊、又续蛊二蛊当中,然后再间接地注入泉蛋蛊。

泉蛋蛊悬浮在半空中,不断吸收真元,徐徐沉浮。

如此过了十七八天,石武每天只睡一个半时辰,吃饭、如厕无不抓紧时间。虽然辛苦,但成效可见。

泉蛋蛊经过真元的不断灌输,已经闪烁着一层华丽宝光。

这天,蛮多又来视察,见此景象,不禁欣慰地道:“功夫快要到了。你看这泉蛋蛊上,已经隐显裂纹,等到它彻底破碎,便是成功之时。石武家老着实辛苦了。”

“不辛苦,不辛苦。”石武一边炼蛊,一边谦虚道。

他神态疲惫,身体明显瘦削下去。以他如此修行,强行催动五转蛊,十分辛劳。但真正做了之后,便是功劳一件。

蛮多又道:“这次家老议事时,父亲大人特意关照了。石武家老一番辛苦,大家都看在眼里。并问家老你有什么需求,我们当尽量满足。”

石武感激涕零:“在下惭愧,劳烦族长大人如此关照,岂能得寸进尺,再有什么要求。只是最近在下有一份疑惑。”

“哦?请讲。”

“蛮多少爷,在下感觉最近的红炎谷中,却是越发寒冷了。我看这附近的花草鸟兽,都被冻死不少呢。”

蛮多脸色一沉。红炎谷的确出了问题,最近几次部族议事中,都被反复提到。侦察后的结果是,支撑红炎谷的地下熔岩,不知为何削减了许多。

这和历史同期完全不符。

刚刚的议事中,蛮图族长更是大发雷霆,要众人想出对策。同时尽量隐瞒,防止人心浮动。

“石武家老正在种元泉的关键时刻,我还是不要告诉他这个坏消息,以防他思绪不定,坏了手上的这件大事。”

念及于此,蛮多便扯谎几句,叫石武放宽心。

石武不疑有他,正要说话,忽然神色骤变,惊呼道:“糟糕,竟是飞手雪!”

蛮多转头一看,惊骇欲绝:“红炎谷中,怎么会出现飞手雪?!”

但见天空中,狂风骤起,一片片雪花,形如手掌,循着五转蛊的气息,漫天降下。

“不好,快来人,保护泉蛋蛊!”蛮多大叫一声,立即唤来附近的蛊师。

但雪势惊人,风割如刀,天地肆虐威能,众人抵抗一阵子后,就渐渐不支。

“雪怪!”

“有雪怪出现了!!”

祸不单行,风雪凝聚,形成一只高达两丈的雪怪。

蛊师们的防线很快被突破。天空中,大量的飞手雪相互凝结,成一只巨手。在众人愤恨不甘,却又无可奈何的目光注视下,巨大雪手一把抓住半空中的泉蛋蛊,狠狠一握。

砰。

一声轻响,雪手崩溃。

雪花散落到地上,里面的五转泉蛋蛊已然不见。

……

银辉宁静,如纱朦胧,挥洒在王庭福地当中。

嗷呜、嗷呜……

天青狼群在撒爪飞奔,或在天空中漫游,或是俯冲大地。

尽管在圣宫当中,有专门的蛊师喂养它们,但是狼群终究还是向往自由,向往宽阔天地的猛兽,不是囚笼里的金丝雀儿。

身为它们的主人,方源放任这些狼群。而他自己则张扬着鹰翼,缓缓盘旋于高空,俯瞰着脚下。

脚下正是地丘传承所在。

“土中蕴光,芒高万丈,百里天游,咏梅雪香……这句话到底是什么意思呢?”方源心中琢磨着。

这些天来,他每隔几天,都会亲自过来勘察传承现场。

直觉告诉他,这个地丘传承很不简单。

至于每次出游的理由,就是溜溜这些狼群,训练一番操纵狼群的基本功。

但就算如此,他也不能逗留这里太久。

方源现在位高权重,是仅次于黑楼兰的大人物,一举一动都牵引着众人的目光。不像以前那样子方便了。

这一次同样没有发现,为了防止他人的怀疑,方源只能暂时离开这里。

虽说以他如今的地位,完全可以用个人的名义,将这片地方圈起来。

但方源没有这么做。

如果这片传承价值极高,就算他实力超群,也会有人和他争抢。

他毕竟没有巨阳血脉,要进入八十八角真阳楼,都需要黑楼兰的来客令。

不过,他一直遣人监视着这里。

用的理由,也落到天青狼群身上。

他每次率领天青狼群外出狩猎,都会有一个线路。每次出行前,他都会调遣蛊师为自己侦察,哪个线路猎物最多,就去哪个路线。

一共有五六个路线候选,不管哪一个,都会经过地丘附近。

做戏做全套,方源接下来便带着天青狼群,按照规定的路线,继续狩猎。

王庭福地当中,兽群极多,物资十分丰盛。尤其是小塔楼的附近,虫群一片又一片,野蛊不在少数。

八十八角真阳楼,只有少部分的蛊师才有机会进入其中。大部分的蛊师,都会在广阔的福地中,不断游荡,或收服野蛊,或找寻传承。

方源沿途,就见到不少的蛊师。

至于地丘传承附近,当然也会有很多蛊师陆续经过。

方源倒不担心地丘传承,会被有缘人获取了去。他对此反而有些期待,如果有人意外地撞开传承,他势必会听到风声,到那时再去动手也不迟。

反正他以大欺小,抢夺传承的事情,早在星鹫峰上就已经做过了。

咳隆隆……

一阵连续的沉没响声,从地面传上来。

方源骑着天青万狼王,往下看去,便见一座小塔楼,闪烁着耀眼的光辉,正缓缓地沉入地面。

方源并不惊异。

在八十八角真阳楼凝聚的过程中,王庭福地中每隔八里就有的小塔楼,都会陆续沉入地底当中。

这些小塔楼,像是一座座蜂巢,里面塞满了闭门十年来,积累起来的野蛊。

很多人猜测:这些小塔楼中的野蛊,便是凝聚八十八角真阳楼的力量之源。从某种意义上讲,这无数的小塔楼,或许便是八十八角真阳楼中的一部分。

这个猜测一直都不得证实。

能证实这点的蛊仙们,都进入不了王庭福地。而能进入王庭福地当中的凡人,和巨阳仙尊差距比天地还大,又没有能力勘察。

但,方源是个例外。

“八十八角真阳楼,构思奇特,可谓巧夺天工。这些小塔楼,的确是真阳楼的一部分。”方源远比很多蛊仙都清楚这点,皆因他手中就掌握着八十八角真阳楼的全面信息。

这个信息的来源,就是琅琊地灵。而地灵的前身,便是负责炼制真阳楼的八转蛊仙长毛老祖。

“嗯?等等!”方源身躯微微一震,脑海中一道灵光,仿佛闪电般划破迷雾。

这一刻,他忽然发现地丘上,一个与众不同的地方——地丘附近,并没有这样的小塔楼!

“没错,的确是这样子的。”方源眼中精芒一阵爆闪,好几次他带着天青狼群外出。就算不到地丘上空,也会在地丘的附近遥望。

现在他回想起来,立即发现,地丘上的古怪之处。

“按照道理,每隔八里,都会设有一处小塔楼。其实每处塔楼,都照应北原相应的地域范围。但地丘附近,却是空无一物啊!”

想到这里,方源心情不免振奋起来。

这是个突破性的发现!

按照这个线索,他便很有可能解开地丘传承的奥秘。

但方源却没有直接回头。

忽然兴冲冲地跑回去,是会被怀疑的。

他按耐住微微激动的心情,按照之前的路线继续走下去。天青狼群从圣宫出发,绕了一个大圈子后,又回到圣宫。

圣宫宛若山峰,圆顶上八十八角真阳楼还在孕育当中。它散发出来的彩霞,已经彻底覆盖了整个圣宫。原本就富丽堂皇的圣宫,便渲染成一片锦绣壮观的宏大盛景。

方源每隔六七天,都会外出一趟,带着天青狼群狩猎。

然而计划不如变化,三天之后,漫天的彩霞陡然尽数收拢,在瞬间凝聚成形。

八十八角真阳楼的第一层,显化出来了!

这个叫人欢喜的消息,引发了圣宫上下的一片震动。一时间,人们茶钱饭后,都在讨论这个话题。

当然大多数蛊师,都只能观望。

黑楼兰迫不及待地进去。几个时辰之后,他带着一身伤势,狼狈不堪地走出来。

他的伤势不轻,但却难以掩盖脸上的惊叹之色。

第二次出发时,他带了许多黑家蛊师在身边。

黑家族人身居巨阳仙尊的血统,能自由出入真阳楼,不受拘束。

这个时候,还是族人较为可靠。能够自己独吞的东西,为什么要分享给外人呢?

黑楼兰再次出来时,一脸疲惫不堪的模样。而带进去的蛊师,也只有六成的人成功回来。

随后,各种小道消息风传开来,都是关于真阳楼、。

有说真阳楼鬼斧神工,为此叹为观止的;有说闯楼艰难,举步维艰的;也有说斩获颇丰,激动人心的……

一时间,人心浮动。

\end{this_body}

