\newsection{石人族大发展}    %第十六节:石人族大发展

\begin{this_body}

%1
鹤风扬仔细看着,闪着电光的信笺上,内容密密麻麻。

%2
方源要求的东西,无非三样,一是元石,二是蛊虫,三是材料。

%3
“要求五百万块元石?可以!别说是五百万,就是一千万,五千万也给得。”鹤风扬冷笑着。

%4
到了蛊仙这一层次,注重的是仙元。元石要多少有多少,根不在意。

%5
“嗯?还要泉蛋蛊五只?”鹤风扬微微一愣。

%6
泉蛋蛊,则是五转蛊,外形如白色鹅蛋,乃是斩杀特定的蛋人所得。

%7
将泉蛋蛊种在地底深处,就能形成一道元水泉眼。这就意味着,能种出元泉。

%8
每一颗泉蛋蛊,就代表着一道小型元泉。

%9
每道小型元泉,从成形到消失,至少能出产一亿块元石!

%10
但泉蛋蛊的价值,要比一亿元石还要高得多。蛊仙们常常求购,种在自家的福地中,形成元泉,滋润万物。

%11
有充足的元泉,便能使福地中元气更加浓郁,更利于蛊虫的培养和壮大。

%12
相反,若没有元泉,福地中的元气都是通过仙元稀释,而产生的。

%13
从某种意义来讲,元泉可以节省仙元。

%14
“想不到这个方源,区区凡人,也知道泉蛋蛊的重要性。不过这种蛊,就算是蛊仙也供不应求,怎么可能一下子给你五只?三只差不多,你经营的狐仙福地,迟早是我们仙鹤门的。”

%15
鹤风扬转移目光,又掠向下面的内容。

%16
下面,方源名列了许多蛊虫。

%17
“他要怎么多的低转蛊虫干什么?”鹤风扬心生疑惑。

%18
若是方源要求那些五转的精品,稀有而又强大的蛊虫,也就罢了。但偏偏他要求交易的,是一些二转、三转的蛊虫。四转的虽然也有。但也数目稀少。

%19
“哼,我明白了。这小子的戒心真是强啊!他害怕直接索要五转蛊,蛊虫身上会被我们动手脚,所以就打算自己炼蛊。炼蛊的过程,要求精密细微,若用了动了手脚的蛊虫,就会破坏整个炼蛊行为。反过来讲,炼成功的蛊虫,都是安全干净的。”

%20
鹤风扬顿时感到有些棘手。

%21
他原来的确是有这样的打算。现在看到这里,已然明白这个阴谋还没有实施,就已经破产了。

%22
“这个小子不好对付啊……不过,他怎么知道如此多的炼蛊秘方?看来他的背景绝对不简单!”鹤风扬心中暗凛。

%23
不过转念一想,他又笑了。

%24
“这个方源。不知好歹!蛊是那么好炼的么?每一次炼蛊,都是巨大的投资。炼的蛊转数越高,失败几率就越高,就越是亏。你小子才多少岁?炼蛊可不是单靠天赋就能成的,更重要的还是经验的积累。”

%25
鹤风扬并不知道方源的底细,更不知道定仙游蛊直接就是方源炼制的。如果他知道,他绝对笑不出来。

%26
鹤风扬继续看下去。

%27
“哦?这其中他还要了舍利蛊。一只黄金舍利蛊。三只紫晶舍利蛊?”鹤风扬目光为之一顿,知道方源的盘算。

%28
方源有四转高阶的修为,这点早已经落到十派蛊仙的眼中,不是秘密。

%29
方源用了一只黄金舍利蛊。就是四转巅峰。晋升五转初阶后,连用三只紫晶舍利蛊,便能一跃成为五转巅峰的蛊师。

%30
仙鹤门是中洲十大门派之一,家大业大得很。这些舍利蛊自然有。每年都会有弟子、长老贡将意外获得的舍利蛊献给门派,换取贡献额度。而门派也会将青铜到紫晶舍利蛊。作为师门任务的奖励,发放下去。

%31
这些舍利蛊,鹤风扬完全拿得出来,但是,让方源这么快就晋升五转,实在不符合仙鹤门的利益!

%32
“呵呵呵,这次先把一只紫晶舍利蛊交易过去,让方源这小子眼馋着。这些舍利蛊是他最想得到的东西,得卡住他,让他将胆识蛊交易出来!”

%33
鹤风扬继续看下去,除去蛊虫之外,接下来就是一些材料。

%34
这些材料,也大多普通得很。一些珍稀的炼蛊材料,虽然也有,但数量不多。

%35
“看来这小子,打定主意要炼蛊了。哼哼,炼吧,炼吧。不过他要星鹭胆汁、蟾蜍石、羽化酒等等,这些偏门至极的材料干什么?”

%36
这些材料,能够运用的秘方很少,有些价值堪比荒兽身上的部件。饶是鹤风扬,搞到这些东西也要费一番功夫。

%37
“不管这小子是故弄玄虚,还是分散我的注意力,或者真的要炼出什么稀罕的蛊虫。这些材料,我都不能一下子全给他。让他着急着急,更能试探出他的真正用意。嗯,这次就把羽化酒给他一坛吧。”

%38
这羽化酒,乃是极品美酒,为仙鹤门太上三长老所酿。还是鹤风扬在一百多年前,正式投靠三长老。在酒宴上,太上三长老心情上佳,就奖赏他三坛羽化酒。

%39
太上三长老喜欢喝酒,算是生活的情趣之一。但鹤风扬不喜欢,羽化酒对他而言,是最没有价值的。

%40
信中内容的最后,鹤风扬看到荒兽泥沼蟹的身体各个部件。

%41
他舔了舔嘴唇。

%42
这可是荒兽,能媲美蛊仙的存在!

%43
“这么全的荒兽尸体,看来这次地灾是荒兽之灾了。方源这小子运气真好,利用荡魂山绞杀了泥沼蟹的魂魄,一下子获得这么完整的泥沼蟹的尸身。”

%44
鹤风扬只一瞬间,就将地灾的情形,猜得不离十。

%45
“这泥沼蟹可以全部吃下,不过方源这信中没有要求石人?是他不知道石人,见识有限?还是当年迁徙进去的那群,仍旧存活着?情报终究太少啊。”

%46
狐仙福地的具体情况如何,方源是怎么得到定仙游蛊的,他的背后又有什么人物?这些问题,鹤风扬都不知道。

%47
中洲、南疆、北原、西漠、东海五大域,都是相互独立。各有屏障,身又广袤无比。

%48
仙鹤门尚且掌握中洲不全,更遑论将触角伸入南疆各域了。

%49
不过,自从方源抢夺了狐仙传承之后,仙鹤门中就已经派遣长老,前往南疆调查去了。

%50
……

%51
数天后。

%52
“到了,就将元泉种在这里罢。”草原上,方源停下脚步,对紧随身后的小狐仙道。

%53
小狐仙点点小脑袋。将手掌一扬,飞出泉蛋蛊。

%54
泉蛋蛊落在地上,就钻进土里深处。只是须臾功夫,方源就感到地面震动,哗哗流水的声音。越来越大。

%55
然后,砰的一声,一道泉水从地上喷出来,高达两三丈。

%56
泉水呈现乳白之色,水汽逸散,几个呼吸之后,方源就感到空气中的元气。变得十分浓郁了。

%57
“元气是万物的母气,元气越充沛,土地就越是肥沃,草木更加茂盛。兽群更加繁荣,石人更会得益不少。”方源满意的点点头。

%58
和仙鹤门的交易,获得成功。方源将泥沼蟹的尸身,全部卖掉。同时也得到了一大部分他想要的东西。

%59
其中,就有三只泉蛋蛊。

%60
这泉蛋蛊。高达五转,使用一次便消耗掉,化为一道小型的元泉。

%61
催动它对蛊师的真元要求很高,至少五转高阶的蛊师,消耗掉所有的真元,才能将其成功催动。

%62
方源只是四转高阶的修为,压根使用不动泉蛋蛊。不过好在有地灵小狐仙,作为方源的帮手。

%63
这三只泉蛋蛊,方源都埋设在石人一族的家园附近。

%64
经过第六次地灾的洗礼,狐仙福地中狐群已经稀少得可怜,不值得大力培养了。方源便将注意力,全部集中在石人一族的身上。

%65
如今,石人只剩下数百人,比狐狸更要稀少得多。但是有着荡魂山,再加上这三道元泉,石人一族将会迅速地壮大起来。

%66
“呵呵呵,主人,太棒了!有了这三道元泉,青提仙元的损耗就变少了。”小狐仙看着喷上高空的泉水,双眼笑得眯起来,很是开心。

%67
“这三道元泉,至少出产三亿块元石,可以支撑五六十年。不过对于整个福地来讲,还远远不足。就算是福地南部,也是少了。”方源道。

%68
元泉分小型、中型、大型。

%69
小型元泉,支撑五六十年,中型元泉大约是一百年有余,大型元泉则是数百年不等。

%70
喷涌的泉水,将周围的泥土冲开,泉口渐渐扩大。泉水冲击力不够了,高度便在慢慢降低。

%71
过个四五天,泉口变会正式形成,泉水就会潺潺流淌,浸透周围的泥土。在此后数月,浓郁的元气,将会凝结成第一批元石。

%72
“走吧,我们先回去。那边石人一族,应该高兴坏了吧。”方源命令地灵道。

%73
下一刻,两人挪移到荡魂山上。

%74
“天,好多的胆石啊!”

%75
“这里一片狼藉,果然经历了一场大战。也许那个男仙人已经死了。”

%76
“我们石人部族将引来辉煌的明天!!”

%77
“我们的子孙后代将无穷无尽,岩勇族长啊,我们可以建成世间最庞大的石人部族了!”

%78
石人们欢呼着,呐喊着,尽情地宣泄此刻的兴奋和幸福。

%79
荡魂山上,长满了胆石。五步之内,必有一颗。

%80
石人们热火朝天地敲碎胆石,壮大魂魄。

%81
很多小石人已经产生,他们也在敲打胆石,新生的魂魄很快就壮大起来,令他们可以驾驭更多的石头成为身躯。

%82
石人原先只剩下数百人,但仅仅一天,他们的人口就扩大了十倍!

%83
人口基数越多,增长得速度就越快。

%84
到了第二天早上,他们的人口破万。第三天傍晚时,他们有了三十万的族人。

%85
荡魂山上的胆石,也被采集一空。

%86
到了第四天,荡魂山开始发威,石人们只好恋恋不舍地离开这里,一路浩浩荡荡,返回自己的家园。

%87
在那里,还有意外的礼物(新生的三道元泉)在等待着他们。

\end{this_body}

