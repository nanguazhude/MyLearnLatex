\newsection{复原荡魂山}    %第二百一十一节:复原荡魂山

\begin{this_body}

%1
“主人,你终于回来了!”小狐仙早早地守候在星门那端,看到方源的身影后,立即跳到他的面前,一把抱住他的大腿,并且用苹果般可爱的脸蛋不断蹭蹭。

%2
“呵呵,我不在的这些天,辛苦你了。”方源脸上流露出温柔的笑容,伸出手,轻轻地抚摸小狐仙的小脑袋。

%3
小狐仙脑袋上的一对狐耳顿时微颤,脸上流露出幸福的神色,一只雪白的狐尾在小屁股后面轻轻摇摆起来。

%4
“这是……地灵?这么说来,这里就是一片福地了!”太白云生紧跟着,从星门中跨出来,看到小狐仙后,不禁吃了一惊。

%5
小狐仙称呼方源为“主人”的话,他听在耳中。

%6
方源只是区区凡人,却已经坐拥了一片福地!这种际遇,历史上也有人有过。最著名的一位,便是巨阳仙尊。他还是凡人时,就幸运地继承了王庭福地。此后修行过程中,王庭福地带给巨阳仙尊巨大的帮助。

%7
“哈哈,这座中洲福地名为狐仙福地,是我接了师傅之命,强行从中洲几个门派手中夺来的。”方源回道,神态傲然。

%8
顿时,太白云生的目光又发生了改变。

%9
仙凡有别,现在的他却他真正地用一种平等的目光看待方源,不禁问道:“想来,夺得这片福地的过程十分艰险吧?”

%10
“那是当然,现在回想起来,我还有些后怕呢。”方源哈哈大笑。“不过话说回来,成王败寇。一切就这么简单。我得到了福地,成为最终胜利者,风险越大,收益就越大!”

%11
太白云生点点头,心中感慨:这收益可大得去了!

%12
拥有地灵的福地,就是陨落蛊仙的仙窍。拥有这样的福地,就等若继承了蛊仙的仙窍,继续修行!

%13
同时。太白云生对方源好行险的性格印象,又加深一层。

%14
“主人,他是谁啊?气息好强。”小狐仙看到太白云生,察觉到他身上洋溢着的蛊仙气息。不禁小手一紧,攥着方源的裤子,显得有些紧张和戒备。

%15
“放心,他是自己人。这次来是帮助我们救活荡魂山的。”方源安抚小狐仙,“快将我们带到荡魂山去吧。”

%16
小狐仙听到这话,眼前一亮,主人说的话,她毫无保留地相信。

%17
于是立即放下戒备,带领方源和太白云生。一齐消失在原地。

%18
下一刻,方源、太白云生便被挪移到福地中央,见到了荡魂山。

%19
荡魂山已经被和稀泥,腐蚀得不堪入目。原本巍然耸立的荡魂山,此刻已经只剩下一个小土堆。

%20
之前在荡魂山中。开凿而成的荡魂行宫,自然也早就销毁了。

%21
“主人。你再晚一点,就见不到荡魂山了。”小狐仙语气悲伤。

%22
方源轻轻抚摸着她的小脑袋,转身看向太白云生。

%23
太白云生是个聪明人,见此情景,再结合之前方源的话,便料到方源请他来的用意。

%24
他对方源缓缓点头,旋即从空窍中,取出江山如故蛊。

%25
顿时,仙蛊的澎湃气息,充斥周围空间。

%26
“这是?”小狐仙瞪大水灵灵的眼睛,她还是首次见到江山如故蛊。

%27
此蛊它形如瓢虫,拳头大小,浑身碧玉也似。

%28
圆滚滚的背壳上有天生的纹路。一半纹路绵延缠绕,描绘江河湖海,一半纹路陡峭重叠,描绘山丘峰峦。

%29
太白云生轻喝一声,一颗青提仙元便化为一道流光,眨眼间注入到江山如故蛊上。

%30
仙蛊陡然爆发出冲天的碧玉光辉,光芒万丈,令人不可逼视。

%31
光辉笼罩着残缺的荡魂山。

%32
荡魂山表面的和稀泥,在碧光的照耀下,立即消停下来。和稀泥仙蛊的力量,被迅速中和,直至消失。

%33
碧光渐渐晦暗下来,太白云生又投入第二颗青提仙元。

%34
碧光重振旗鼓,覆盖住惨不忍睹的荡魂山,不断冲刷。

%35
小土丘般的荡魂山,在光辉的灌溉下,以肉眼可见的速度不断长高,不断壮大。

%36
“荡魂山又变回来了!”小狐仙拍动小手,高兴得雀跃欢笑。

%37
但好景不长,荡魂山只恢复两成,碧光再度消弱。

%38
太白云生面色微变:“这是什么山?竟然如此消耗青提仙元!”

%39
他成仙之际,生成三十六颗青提仙元。但之后狂催移动蛊,在颠乱雷球中持续闪避,消耗了一颗。又在自家仙窍中,抵御仙蛊形成后酝酿而成的天劫地灾,不断催动江山如故蛊,恢复仙窍旧貌,足足消耗三颗。

%40
太白云生十分清楚仙元的珍贵程度,现在他才刚刚出手恢复荡魂山,就先后消耗了两颗仙元。

%41
按照这架势,至少要消耗五六颗的青提仙元才行!

%42
方源微笑道:“荡魂山。”

%43
“荡魂山?”太白云生终于听清楚,这一次面色大变,双眼瞪大,震惊地道,“难道是?”

%44
“《人祖传》中,难道还有第二座荡魂山不成?”方源笑意浓郁。

%45
“想不到,我居然看到了传说中的荡魂山!这可是幽魂魔尊都赞不绝口的魂修圣地啊。”太白云生感慨万千,赞叹不绝。

%46
即便是他,也是首次看到这种传说之物。

%47
瞄了一眼方源,太白云生的目光又起了微妙的变化。

%48
他没有说话,在沉默中调度出第三颗仙元,继续催动江山如故蛊。

%49
这一次,荡魂山恢复到四成旧貌,碧光再度不济。

%50
太白云生又用了第四颗、第五颗青提仙元,直至第六颗,这才终于将荡魂山恢复十成。

%51
“真不愧是荡魂山,足足耗费了我六颗青提仙元。才修复完成。”太白云生仰望眼前的高山,口中喃喃。

%52
荡魂山美轮美奂。绝非凡石俗山可比。

%53
它乃是一座水晶山峦,通体粉红,散发着梦幻般的光姿,令人一见难忘。

%54
“主,主人,荡魂山的病终于好啦。”小狐仙双眼泛红,“呜呜呜……”

%55
她喜极而泣,低下头来。用粉白细嫩的小手背不断擦拭着眼泪。

%56
方源望着荡魂山,也吐出一口浊气。

%57
至此,他潜入北原的目标,终于达成!

%58
多少日夜的努力,纵横沙场的危险,小心翼翼的伪装,殚精竭虑的筹谋。都没有白费。

%59
现在,方源的心中燃烧着滔天的热焰。

%60
这股熊熊之火的名字,就叫做——野心。

%61
恢复完全的荡魂山,已经满足不了他的胃口。王庭福地中,还有良机!

%62
尤其是现在,巨阳意志被排斥在楼外。八十八角真阳楼等若无主之地。

%63
虚情假意蛊纵然仙蛊,难以捕捉。但方源鼓动三寸不烂之舌,已经取得了太白云生的信任,成功拉拢到了影响局面的关键战力。

%64
届时凭借蛊仙意志,强行炼化虚情假意蛊。并非不可能的事情。

%65
“走,回北原。”方源来去匆匆。和太白云生离开了狐仙福地。

%66
而在临走之前,他将噬魂蟾交给小狐仙。

%67
噬魂蟾乃是存储蛊,里面装满了魂魄,有人有兽,大多是方源从战场中收集而来的。

%68
借助荡魂山,将产生大量的胆识蛊。

%69
回到八十八角真阳楼里,支撑星门的星萤蛊已经损失大半,方源连忙将剩下的收回空窍。

%70
星萤蛊积攒起来,极不容易。如今的量,只能再支撑一次星门开启。

%71
“又回来了。”太白云生感慨不已。

%72
这一番,从北原直接回到中洲,又从中洲直接回来北原。简直是纵横恣意,咫尺天涯!

%73
太白云生只听过洞地蛊,通天蛊,可以令福地、洞天相互沟通,还从未见过这样的手段。

%74
“恐怕,也只有六师弟这样的人物,才配得上如此的厉害手段吧。或许,这是师傅传授他的。”狐仙福地一行,让太白云生更加信任方源。

%75
方源此时还只是凡人,却拥有福地,这事情本身就足以证明许多问题。

%76
荡魂山的壮丽光景,更是深深刻在太白云生的心中。

%77
眼见为实,耳听为虚!

%78
由不得太白云生不相信。

%79
方源先是表明真面貌,随后用寿蛊、秘辛等等取信太白云生。又解开太白云生心中疑惑——真阳楼为何无缘无故地帮助他渡劫。

%80
方源说的话,给太白云生展现的证据,几乎都是真的,只有在关键处说了谎话。

%81
九真一假,假也变成了真。尤其是这个假话,是太白云生心底最深处的秘密。被方源搜魂而得,太白云生从未向外人透露过。

%82
这其中最妙的地方,在于方源通彻了太白云生的心理。

%83
太白云生因为背叛高扬、朱宰而愧疚万分,一度否定自己存在的价值,受到刺激之后,更是冲动渡劫,带有轻生之念。

%84
但是当他成功渡过天劫,成为蛊仙之后,他站在全新的高度,看到了不一样的风景。

%85
他的心思开始活络起来,尤其是历经艰辛才渡过劫难,拥有仙窍福地,他不那么想死了。

%86
他想重新做人。

%87
但是他还有心结。

%88
他需要希望,需要认可!

%89
这种心理需求,太白云生自己都没有察觉到。但方源察觉到,并且给了他认可。

%90
尤其是方源的认可,在某种程度上,就代表着恩师的认可。

%91
在太白云生心中,十分敬重恩师。恩师的认可,让他觉得自己还有活着的价值。一个神秘的门派,全新的风景,勾动了他的好奇心。

%92
他是凡人的巅峰,历经红尘,但在仙途上,他还只是刚刚起步的孩童。

%93
十五年的寿蛊,就在他的怀中。他不再焦急,又心怀愧疚,他选择相信方源,不仅是因为方源将一切都解释通了,又给了他大量的不容反驳的证据。更更关键的是——

%94
他从内心最深处。愿意相信方源!

%95
但凡上当受骗的人,难道真的是因为他们蠢笨吗?

%96
不是。只是他们内心愿意去相信罢了。

%97
“你的仙蛊都还给你。”太白云生将定仙游、飞熊虚像蛊抛给方源。

%98
方源若无其事地接过,他这种对待仙蛊随意的态度,对太白云生的信任,再度让太白云生心生感动的涟漪。

%99
但当方源将琉璃楼主令握在手中,稍稍感知后,却是大吃一惊,连忙带着太白云生来到另一处关卡。

%100
古木参天,方源、太白云生二人置身在漫漫森林里。

%101
咆哮声、怒吼声接连响起。一棵棵巨树拔起树根,站立起来,化为高大树人。

%102
转瞬之间,方源、太白云生就被树人重重包围。

%103
太白云生面色凝重,单个树人他丝毫不放在眼里,但这里的树人数量极为惊人,要彻底清剿需要耗费许多代价。

%104
但下一刻。方源轻轻晃动手中的琉璃楼主令,立即将这道关卡化为己用。

%105
树人冲势顿止,杀意骤灭。

%106
太白云生正愣神之际,方源抓住他的胳膊,带着他直接挪移到黑楼兰、马鸿运的面前。

%107
此时这边的局势,已经产生了巨大的变化。

%108
霜玉孔雀半躺在地上。马鸿运、赵怜云二人龟缩在它的羽翼之下。周围是层层树人,对他们展开围杀。

%109
黑楼兰则是一人于不远处作战,虚情假意蛊就停留在他的肩膀上。

%110
他一边抵挡着树人的围杀,一边对马赵常三人不断发动攻势。

%111
但这些攻势,都在半途中被霜玉孔雀撑起的光罩抵挡。

%112
“黑楼兰大人。你现在也被树人攻击,和我们处境一样!我们和你无冤无仇。你为什么非要杀我们?”马鸿运大叫。

%113
黑楼兰冷哼一声:“你刚刚没听到先祖说吗?你的这个女人是天外之魔,来历神秘,祸患无穷,必须铲除!说不定她和假冒的常山阴,就是同一伙人!老祖宗刚刚要对她动手,就忽然消失,遭受了不测。你身为巨阳血脉,北原中人,居然是非不分,袒护一个天外之魔!”

%114
马鸿运十分气愤,立即反驳道:“她不是什么天外之魔,她是无辜的!黑楼兰大人,你不要白费力气了,我们俩个已经得到地灵认主,有地灵的帮助,你是杀不了我们的!”

%115
原来,巨阳意志被抽出楼外,这道关卡没有了主持,自动施行,树人们就将地灵,以及马鸿运等人,当做闯关者对待。

%116
黑楼兰欲替巨阳意志,斩杀赵怜云。马鸿运舍身相助,令赵怜云心神震动,真正爱上马鸿运。

%117
马鸿运本来就对赵怜云付出真心,两人真心相爱,让地灵主动认主。

%118
千钧一发之际,地灵为他们二人撑起防御护罩,抵挡住树人以及黑楼兰的致命攻击。

%119
至于常丽,则惨死在树人的攻击中。

%120
方源和太白云生倏地出现在树人的树冠上,俯瞰战局。

%121
“马鸿运……居然得到了地灵认主,哼,这小子的狗屎运,还真是强悍啊。”方源口中喃喃,语气复杂。

%122
“你认识他?”太白云生有些惊讶,没想到堂堂方源,居然也认识这个三转初阶的小人物。

%123
“是你!”黑楼兰很快察觉到远处的方源和太白云生,瞳孔猛缩,连忙跳开一边,如临大敌。

%124
他眼角抽搐,心已沉入谷底。

%125
巨阳意志忽然失踪,假冒常山阴的神秘男子再度出现,居然身边还站着蛊仙太白云生,双方关系还似乎十分紧密。这样的局面,对黑楼兰极为不利。

%126
马鸿运则毫无察觉。地灵撑起的护罩,不断收缩,已经被树人层层包围。他和赵怜云的视线,早就被茂盛繁密的枝叶遮挡。

%127
“现在怎么办?”太白云生目标瞟向面沉如水的黑楼兰,平静地问道。

%128
方源冷笑一声:“要颠覆真阳楼,就必须借助王庭福地地灵霜玉孔雀的力量!霜玉孔雀虽然认主,但其实是趁着巨阳意志不在,身上的封印力量无人指挥,而在苟延残喘。它的大部分力量,已经被再度封印。防御光罩岌岌可危,根本不能阻我。我去杀了马鸿运、赵怜云。你去对付黑楼兰,要夺得虚情假意蛊,相信以你蛊仙的战力,应该不成问题。动作要快,我们时间不多!”

%129
“仙凡有别,相互差距宛若鸿沟。你就放心吧。”太白云生淡淡一笑,飘摇而起,向着黑楼兰逼近。

%130
方源电射而下,直接向着马鸿运杀去。

%131
周围树人在他的意志下,层层叠叠,跟随而来。

%132
“可惜了……马鸿运,本来还想借助你这个线索,来把握将来大势。但谁叫你挡在我的路上?任凭你运气再好,今日也必死无疑!”心中杀机已起,方源的脸上浮现出一抹狞笑。

\end{this_body}

