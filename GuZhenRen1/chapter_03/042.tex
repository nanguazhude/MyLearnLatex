\newsection{蛛丝马迹蛊}    %第四十二节:蛛丝马迹蛊

\begin{this_body}

谈到葛家一事,原热烈的晚宴氛围,就忽然冷却下来。

葛家老族长连忙站起身,做出解释。

葛光坐在一旁,看着自己的父亲,心中满是苦涩和悲伤。

自己的亲妹妹死了,父亲心中已经哀伤至极,他多么希望这一切不是真的。但是如今,他却要向外人百般证实,自己的女儿千真万确已经死了。

这是一场多么痛苦的折磨啊。

但蛮图做出一副倾听状,却是不接受的态度。

常山阴是正道英雄,昔日又是常家举族培养的天才蛊师。自从斩杀了哈突骨等一帮马匪之后,为世人广为称颂,身的威望极大,在北原可谓家喻户晓。

所以,为了照顾常山阴的面子,蛮图做认真倾听的样子。

但就算是常山阴威望再高五六倍,也远不及他吞并葛家的利益。现在他好不容易找到一个正当的理由,蛮图当然不会这么轻易放过。

所以,在他“耐心”地听完葛家老族长的解释之后,蛮图便看向自己的三儿子。

蛮多心中冷笑,站了起来。

他向方源深深一礼,然后当众取出一只蛊虫:“常前辈请看。”

这蛊宛若蜻蜓,只是细长的身躯仿佛一截檀香,蜻蜓的尾部还在灼烧,冒着缕缕的熏烟。熏烟时而有色,色彩斑斓,时而无色无味。

“难道这蛊,是上古流传下来的追烟蛊不成?”方源目光一闪。

蛮多楞了一下,这才用佩服的语气道:“前辈法眼无差,见闻广博,晚辈佩服万分。的确如此,这蛊我也是侥幸获得。只要染上了这追烟,半年内都会久久不散,用来跟踪极为方便。”

顿了一顿,他又继续道:“实不相瞒,晚辈第一次见到葛谣时,便对她悄悄用了此烟。如今只要催动这追烟蛊,就能令追烟显形,肉眼可见。”

蛮多略带着得意之情,接口道:“葛家族长。若是用此蛊,在你们葛家营地发现了藏匿的葛谣,又当如何?”

他要吞并葛家,自然要谋算安排。追烟蛊是三子蛮多提议用的,没想到现在就有了作用。

面对蛮多饱含威胁之词。葛家父子反而心中松了一口气。

这追烟蛊好啊,如此一来,自己的话就能得到证实,可谓沉冤得雪。

反倒是方源心头咯噔一下,这追烟能相互沾染。自己曾经和葛谣并肩而行,甚至还抱过她。追烟一现,自己岂不是要暴露了?

如何是好?!

方源没有料到。会有这样的变故,心不断地往下沉。

他身上的这套蛊虫,源自常山阴,是奴道蛊虫。长于战阵厮杀,弱于个体拼斗。

如今这酒宴上,自己不过是四转初阶,还有蛮家、葛家两大族长。修为皆高于自己。更有一些家老,护卫。都有三转修为。

就算是自己将狼群带过来,两三千头的普通野狼,也颠覆不了局面!

一时间,方源脑海中思绪急闪,各种念头闪现,宛如电光火石。

当即,他朗笑一声:“这就再好不过。有了追烟蛊作证,那事情就真相大白了。贤侄,快快使来。”

他想要拖延,或者拒绝使用,都是不可能的。一来,自己这个局外人拒绝使用追烟蛊,十分奇怪,近乎于不打自招。二来,不管是蛮家还是葛家,都想用追烟蛊证实一些事情,大势已成,难以抗衡。三来,追烟蛊在他人手中,就算方源拒绝使用,难道蛮家就不用了?

因此,方源索性主动要求,先落个光明坦荡的印象。

听得方源此话,蛮图自然大喜:“既然常山阴老弟发话了,儿子,你就用吧。”

蛮多却是高兴不起来,他察言观色,看到了葛家父子反而隐隐期待的模样。

“难道葛谣真的死了?”蛮多按捺住心中的不安,心念一动,催起这只追烟蛊。

顿时,空气中原无色无味的追烟,显现出黑色的轨迹来。

“嗯?”

“咦,怎么回事?”

“这是……”

不出方源所料,自己身上附着着一团浓重如墨的黑烟,在灯火通明的晚宴上,分外地显眼。

一时间,众人的目光都集中在他的身上!

原热烈的氛围,戛然而止,无人作声,空气压抑而又凝重。

方源皱起眉头,脸上是恰到好处的意外、惊愕,让人一看就觉得他也是无辜的人。

似乎嫌弃自己还不够显眼,方源在众目睽睽之下,站起身来,离开座位,又走了几步。身上的这团黑烟,紧跟着他,同时在空中,拖出一道细长的黑烟尾巴。

蛮多目光闪烁,蛮图兴奋地道:“这么说来,常山阴老弟你是见过葛谣的,甚至还和她有过亲密接触!”

葛家老族长的脸色十分复杂,神情变幻不定,看着方源。

葛光到底是年轻,沉不住气,他腾的一声,从座位上站起来,盯着方源,一脸写满了怀疑,质问道:“常山阴叔叔,这到底是怎么回事?难道你在腐毒草原上,见过我的妹妹葛谣?”

方源脸上一片沉静,没有匆忙解释,而是踱步回来,重新坐到座位上,沉声道:“说实话,你们奇怪,我也奇怪呢,怎么我的身上染了这么重的追烟?”

蛮多阴测测地道:“常前辈,乃是北原的大英雄,您的疑惑,小子不敢妄加揣测。”

“常山阴前辈!”葛光按捺不住,目光灼灼,紧紧地盯在方源身上。

方源闻言,神情坦然地和葛光对视,又看向葛家老族长:“葛老哥,我的为人你是知道的。我用狼王的名誉,向你保证,我的确没有在腐毒草原上,见过你的女儿!”

葛家老族长用右手抚心,神色严肃庄重,郑重其事地行了一礼:“常山阴兄弟,老朽与你一见如故,相信你的话!”

一旁站着的葛光,欲言又止。

方源当然明白,单凭这句话,根不足以消弭众人心中的疑惑。紧接着,他眉头微皱,露出思索之色,沉吟道:“我一睡二十年,苏醒之后,就离开腐毒草原。在路上,收服了许多毒须狼。夜间寒冷,我也就抱着毒须狼取暖休憩。想来该是哪头毒须狼,吞食了葛老哥的女儿。是以,我的身上也沾染了这些追烟。”

“是这样?”葛光咬着牙,目光闪烁不定。

方源的解释,合情合理,他是亲眼看见方源从腐毒草原上走出来的,当时从风狼群手中救下他的时候,正是驾驭的毒须狼群。

“此事要证明,十分容易。我的狼群就寄养在葛家的牧场当中。诸位不妨移步,到牧场中查看一番,不就真相大白了吧?”方源又提议道。

这个建议让在场所有人都怦然心动。

“那就依常老弟的建议?”蛮图将目光扫向葛家父子,他还是有些不信葛谣死了。

葛家老族长直接站起身来:“还请诸位移步,一起做个见证人。”

众人一齐登上驼狼,来到葛家牧场。

方源的狼群,就寄养在里面,一个个养的膘肥体壮。

在夜色下,毒须狼群更是显得精神焕发。

靠着追烟蛊显化黑烟,众人成功地发现,狼群当中的几头毒须狼,身上染着比方源还要更加浓郁的追烟。

“长生天在上,仙祖保佑,原来杀害我女儿的凶手,就在这里!”葛家老族长看到这里失声痛哭,然后转身对方源深深一礼,刚要开口,就被方源所阻。

“葛老哥,你要说什么,我清楚。这几头毒须狼,就交给你处置了。”方源开口道。

葛家老族长感动地流下了泪水:“常山阴恩人,谢谢你!你不仅救下了我的儿子,更把杀害我女儿的凶手带到了我的面前。你将是我葛家世世代代最亲密的朋友!”

葛光则保持沉默。

方源救下自己的性命不假,但这个证据却不能完全撇清他的嫌疑。若是常山阴杀死妹妹,下令毒须狼将妹妹的尸体吞食干净,这一切也说得痛啊。

一旁的蛮图,却是暗暗焦急。

他至始至终,都不太相信葛谣死了。

如果葛家察觉了追烟蛊,配合葛谣养出这么一幕假戏,也很容易啊。

于是他嘴角扯动:“葛家老族长,你真的是错怪常山阴兄弟了。常老弟是北原的大英雄,多么正直可敬的人物,怎么可能做出谋害你女儿的事情呢。依我看呐,我们不妨顺着这追烟,深入腐毒草原,继续收集证据,彻底证明常老弟的清白!”

他以方源为借口,实则还是向要查清楚,葛谣到底是不是死了。

葛谣若是深入腐毒草原,那么草原上必然有一路的追烟。

但这时葛光忽然开口:“不必深入腐毒草原了,我有一法可彻底证实常山阴叔叔的清白。父亲大人,早年时你为妹妹准备了一套昂贵珍稀的水道蛊虫。你在这套蛊虫上,用过五转蛛丝马迹蛊,种下了印记。妹妹也知道这件事情,若她是被人杀害的,一定会按照您的嘱托,不去自毁这些蛊虫,而是将它们留给凶手。这样一来,也方便我们找出凶手,为她报仇。现在你只要祭出那只蛛丝马迹蛊,就可以了。”

葛光说到这里,目光深沉地看向方源。

\end{this_body}

