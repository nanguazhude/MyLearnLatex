\newsection{众人蜂拥各有意}    %第一百六十节:众人蜂拥各有意

\begin{this_body}

%1
八十八角真阳楼的惊变,让圣宫上下大为惊震。无数人为此日夜牵挂,提心吊胆,惶惶无措。

%2
好在八十八角真阳楼一层坍塌不久后,情况就稳定了下来。

%3
七彩霞光也不再缩减,而是重新开始持续的增长。

%4
许是被刺激的缘故,这一次霞光增长扩张的度,比之前还要快上三分。

%5
几天之后,浓郁如水的霞雾,重新凝聚出八十八角真阳楼的第一层来。

%6
一彻底成形,黑楼兰等人便立即火急火燎地进入当中。进出八十八角真阳楼的过程,相当顺利。

%7
这让黑楼兰大为安心。

%8
暴怒的黑楼兰,渐渐平静下来。

%9
八十八角真阳楼对他极为重要,他要为母亲复仇,就得晋升蛊仙。而作为十绝体之一的大力真武体,他要升仙就得从八十八角真阳楼中获得力道仙蛊才有希望。

%10
《人祖传》中,早就有过描述。

%11
人活着,可以没有力量,也可以没有智慧,但不能没有希望。

%12
黑楼兰重生希望,心情渐佳,开始狠狠地攻略百道关卡。

%13
新生的八十八角真阳楼第一层,关卡重置,黑楼兰之前的苦功化为硝烟飞散,一切都得重新攻克。

%14
对于黑沛等家老来讲,这是一个巨大的好事。

%15
“这就是否极泰来啊,重新闯关,能让我们获得更多的奖励!”

%16
“或许这只是仙尊老祖宗给我们开的一个小小的玩笑……”

%17
“八十八角真阳楼的每一层都有百道关卡,关卡越是靠后,难度便暴涨。历代王庭胜主通关的层数屈指可数,我们可不指望能攻克最后一关。只要将之前的关卡尽力通了,就能令我族实力极大飞跃!”

%18
家老们喜气洋洋,但对黑楼兰来讲,却是个坏消息。

%19
他若要从八十八角真阳楼中获得力道蛊仙,只有两种方法。

%20
一种是上等通关,进入秘藏阁。利用血脉身份,去换取珍藏中的仙蛊。

%21
第二种则是打通每一层的最后关卡,也有可能获得仙蛊。

%22
对于黑楼兰来讲,第一种方法,需要他本身拿出等量的珍宝出来换取,因此并不现实。真正有成功可能的,只有第二种方法。

%23
打通最后几关。非常艰难。现在关卡又重置,重新攻略这些关卡,势必要浪费掉他宝贵的时间。

%24
时间不等人,若是时间一到,他们就要被传出王庭福地。若在此之前,没有拿到力道仙蛊。黑楼兰不仅大仇不得报,而且必会迎来死亡。

%25
为此,黑楼兰力排众议,一意孤行,扩大规模招揽外族蛊师,彻底开放八十八角真阳楼。

%26
八十八角真阳楼随意进出,黑家不收任何费用。

%27
这样一来。排除黑家族长,圣宫上下欢腾一片。

%28
“黑楼兰气概无双,做了历代王者都没有做出来的事情。我耶律桑佩服佩服!”耶律桑第一个进入八十八角真阳楼,红光满面。

%29
他是耶律部族的当代族长,本来是此次王庭之主的热门人物,获得族中蛊仙的大力支持,不惜借出炎道仙蛊,寄托在他的身上。

%30
但耶律桑最终败北。为了保全身上的炎道仙蛊,不得不投靠到黑楼兰麾下。

%31
虽然最后胜利,成功地进入了王庭福地,但对他来讲,同为级势力却屈居黑家,本身就是一个耻辱。回到部族,势必将得到冷落、抛弃甚至惩罚。

%32
“如果我在八十八角真阳楼收获良多。那么我就能将功补过,风风光光地回到部族当中去!”耶律桑心怀激荡。

%33
“常山阴,你不要得意。只要你一日不成仙,我就都有机会。八十八角真阳楼就是我的崛起基石!”常飚目光阴冷,也是第一批进入楼中的蛊师强者。

%34
他并非独自一人成行,身边还有一位同伴。

%35
正是单刀将潘平。

%36
之前,潘平在星鹫峰中,被方源公然抢夺了机缘,心中愤慨难平。

%37
常飚知道这个情况之后,就故意接近,两人一拍即合,成为搭档。

%38
“去吧,去吧,都成为我的开路先锋,用你们的生命,给我拓宽道路。”黑楼兰肚中冷笑,在一旁催动楼主令,平静地注视着一波波盛大的人流,接连进入楼中。

%39
人流渐渐稀疏之后,一位相貌古朴,身材魁梧,一身白袍的太白云生也出现在圣宫顶层。

%40
“太白老先生。”黑楼兰招呼道。

%41
“族长大人气概惊人,老夫承情了。”太白云生交口赞叹道。

%42
他风度翩翩,气态悠然。既然都可以随意进出,不限名额,他也不急着第一时间冲进去了。能够将寿蛊当做奖励的关卡,必定靠后,他现在还不急。

%43
两人交谈了几句后,太白云生也进入楼中。

%44
“人多力量大啊。”很快,黑楼兰便心生感慨。

%45
从楼主令中传出讯息,短短功夫,依靠庞大的蛊师数量,前三十关很快就被连续攻克。

%46
但是到了四十关以后,单靠数量就不行了,需要特定的蛊师强者出力才有希望。

%47
潘平、常飚、耶律桑、太白云生相继出力,将关卡推进到五十三层。众人又不得不止步,碰到难题,需要奴道大师才能解决。

%48
“此关看来得须狼王出手,才能攻克。”太白云生手抚雪须,沉吟道。

%49
此届王庭之争,方源的表现留给世人极其深刻的印象。北原当代第一奴道蛊师的荣耀,已经送到他的手上。

%50
因此,此刻遇到困难,人们第一时间就想起了方源。

%51
“奇怪,怎么不见常山阴的身影?”耶律桑左右环顾,却搜寻不到。

%52
“狼王近日外出遛狼去了。”很快,就有人回答道。

%53
“常山阴的确非常人也,八十八角真阳楼随意进出,他居然都不动心!”人群中传来赞叹之音。

%54
潘平冷哼一声,语气阴阴:“诸位不要忘了,咱们的狼王大人可是早就进过楼里来了。据可靠消息,还得了上等通关。出楼之后。立即选择闭关,就算黑楼兰族长屡次邀约,他都不去呢。”

%55
众人熟知他和方源的矛盾,没有人愿意当面得罪这位名声鹊起的单刀将,针对方源的赞叹之声戛然而止。

%56
众人陷入到短暂的沉默当中。

%57
诸多蛊师强者们,亦都是神情微动,有些脸色不虞。

%58
潘平用心险恶。指出方源曾经获得巨大好处,又留给众人充分的想象空间,成功勾动了众人心中的嫉妒之情。

%59
若是之前,众人没有体验,还好些。

%60
但现在,众人一路闯关。很多人都有亲身体会,更能觉察出八十八角真阳楼的天大好处。这就更加引得人心深处的嫉恨!

%61
这时,一个年轻蛊师的声音,打破了沉默:“八十八角真阳楼既然给大家开放,谁都能获得好处。父亲大人能获得好处,那就是本事!”

%62
众人目光转移过去,便现此人不是旁人。正是常山阴的亲身儿子——常极右。

%63
顿时,潘平神情一凝,阴狠的目光瞪向常极右。

%64
常极右虽然实力稍逊潘平几分,但此刻他的心中满怀对父亲的崇敬之情,毫不示弱地怒瞪回去。

%65
潘平心中杀机沸腾,但却不敢出手。一时间,竟然被常极右这个后辈逼得下不了台。

%66
心情最复杂的人,还数常飚。

%67
他其实才是常极右的亲身父亲。此刻看着自己的亲身儿子,去维护自己此生的最大仇敌。

%68
他的心中涌起无限的酸楚和仇恨!

%69
“咳咳。”太白云生这时站出来打圆场,“诸位还是着眼于此关为好。”

%70
“为今之计,只有请狼王大人出手相助了。”

%71
“我们当中,太白老先生您最德高望重,只要您亲手书信一封,必能请得狼王大驾。”

%72
蛊师们你一言。我一语,常飚和潘平对视一眼,均心中暗急。

%73
若是让狼王过来,势必能打通此关。

%74
但这样一来。既让他收获关卡奖励不说,又让他施展手段,更增威信。这是他们两人均不想看到的情况。

%75
潘平欲言又止。

%76
他有心阻止,但好不容易趁着太白云生打岔的机会下了台。如果再被常极右杠上,他这脸面恐怕得丢!

%77
常飚一直对潘平暗中察言观色,见他几次张口,却没有说话,心中暗骂一声懦夫,便将目光一扫,示意人群中的一位暗线。

%78
这暗线立即会意,在人群中喊道:“要我说,狼王大人日理万机,恐怕不容易请到。还不如请唐妙鸣大人。她的奴道造诣已经直逼大师境界了。”

%79
太白云生不禁心中一动。

%80
他和方源接触不多,但着实领教了他的“高傲”。与其冒着邀请被拒绝的尴尬,还不如先让唐妙鸣试试。

%81
哗哗哗……

%82
一群又一群的极乐雪蝠群,从四面八方飞来,像是百川汇流成海,向着地丘上的洞中飞去。

%83
方圆百里,异香扑鼻。

%84
正是这种浓郁的异香,将极乐雪蝠群勾动而来。

%85
进行到这一步,炼蛊已经接近尾声。

%86
“此法乃是用兽命献祭,其实和八十八角真阳楼的运转,有着异曲同工之妙。只不过小塔楼是牺牲野蛊,统合力量罢了。”

%87
由于亲身实践,方源对此次炼蛊更加认知深刻。

%88
异香渐渐消散,没有了勾引,所剩不多的极乐雪蝠群慌忙逃窜。

%89
“快要成了!”方源鼻息转粗,双目紧紧盯住地洞,难掩兴奋之色。

\end{this_body}

