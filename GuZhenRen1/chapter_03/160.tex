\newsection{三份传承显本性}    %第一百五十八节:三份传承显本性

\begin{this_body}

王庭福地的夜空,银辉烂漫,如纱如雾,挥洒在广袤的地面之上。

一群天青狼群,在天空中自由奔跑,追逐着天空中的鸟儿,或者地上的猎物,却并非为了猎食,而是纯粹是撒欢游戏。

方源站在地丘之上,缓缓地睁开双眼。

这些天来,他不仅准备好炼蛊所需,而且还广纳狼群。

不仅是天青狼群,还有地上的普通野狼,包括龟背狼、水狼、夜狼等,已经多达二十多万。

“狼群已经散布在附近方圆百里,能起到不小的警示作用。而且天空中的天青狼群,已经成为了我狼王的标志。大多数蛊师看到了,都会明白这是我常山阴在狩猎,也会主动退避三舍的。”

如果不主动退去,那就说明心存歹意。

这种蛊师一旦被狼群发现,必然招致猛烈围剿。

当然,也有无意闯入的无辜者。但方源也管不了那么多,被狼群杀了,只能怪他(她)运气不佳了。

方源又将目光投向远方,即便他距离圣宫遥远,但仍旧可以看到天际处的灿烂彩霞。

圣宫中,已经是再次霞光漫空,美轮美奂――这是八十八角真阳楼,要再次凝结出第二层的征兆。

对于方源而言,这又有另一层含义。

小塔楼陆续沉没,牺牲塔楼中的无数野蛊,形成凝聚仙蛊屋的伟力。

这些彩霞,标志着这股伟力的积蓄,已经渐渐达到量变引发质变的节骨点上。

而方源要开启地丘传承,势必要借助这股伟力,令其回流。

“黑楼兰为了打通关卡。故意开发了八十八角真阳楼,惹得群情激荡,蛊师们趋之若鹜。消息传出,散步在圣宫之外的蛊师们,也在陆续赶来。众人的注意力。都已经集中八十八角真阳楼上。此刻正是我开启传承的佳时啊!”

方源目光矍铄,微微一笑,没有犹豫,直接动手。

“去。”他轻轻一拍肚皮,瞬间便从空窍中,飞出三十六只蛊虫。

这些蛊虫。形状奇特。体型袖珍,只有指甲盖的一半大,像是五角星,散发着乳白色的微光。

这都是一转的小光蛊,十分出名的光道辅助蛊虫。

随着方源的心念调动,这些小光谷纷纷飞入土丘的地洞当中。一时间。将地洞中的黑暗尽数驱散。

方源又调出十三只光栅蛊。

此蛊只是三转蛊虫,同样来源光道。一经发出,能变化栅栏,禁控目标。

光栅蛊同样飞进地洞当中,和之前的小光蛊混杂在一起,不见有任何异变。

方源微微一笑,手掌一甩。飞出三只五转光道蛊来。

这三只蛊,分别是用于增长速度的迅电流光蛊,治疗之用的chun光无限蛊,以及攻击蛊虫火光烛天蛊。

迅电流光蛊,光芒湛蓝,闪烁电芒,首当其中,飞进地洞。

一直没有动静的土丘,这时候终于发生了异变。

像是机关被开启一样,地洞周围的土壤开始缓缓延伸。相互闭合起来。

蓝色的电光,将之前的小光蛊纷纷击溃,形成一蓬散漫的淡蓝光晕,有蓬勃外涌的强烈趋势。

但与此同时,光栅蛊接连一起。形成光的栅栏,将淡蓝华光险险盖住。

华光喷薄欲出,这时chun光无限蛊飞了进来,绽放出无限碧绿霞光,温柔如水,将蓝华压住,两相僵持。

最终,火光烛天蛊飞进洞中,化作一股浓郁的赤芒,穿透绿霞,又刺穿蓝华,沉入洞中深处。

隆隆之音中,地洞洞口彻底合拢起来。在大地深处,三色光辉相互融汇,发生着方源也说不清道不明的奇妙变化。

方源目睹此景,心中稍稍松了一口气,知道自己没有参悟出错。这便是密语中“土中蕴光”之意。

虽然他心中有了八成有余的把握,但也担心出错。毕竟他掌握的,只是地丘传承的后半段线索。这个传承的前半段,则掌握在中洲蛊仙的手中。

不过这后半段线索,专门描述了如何开启这道传承。

方源凭借炼道大师的底蕴,结合八十八角真阳楼的情报,强行参透这段密语,就有了半道出手,强取传承的可能。

大约过了半个时辰,地底的隆隆之音,越来越小。

但是地面则滚烫起来,即便方源穿着北原的鹿皮靴子,也不能隔绝这股热意。

地面上洞口缓缓打开,里面光辉点滴不存,一片黝黑深暗。

方源见此,心中不惊反喜。

“土中蕴光”之后,便是“芒高万丈”。这要单纯从字面上解释,却是谬以千里了。

这密语也是一道考验,考验蛊师是否有炼道造诣。

密语关乎炼道,“芒高万丈”就不是单纯描绘景象之语,而是描述接下来如何炼蛊的步骤!

方源不慌不忙,依次投下五转防御的芒刺在背蛊两只,用于侦察的高瞻远瞩蛊三只,用于攻击的万箭穿心蛊一只,以及专门辅助的火冒三丈蛊九只。

但见洞中,灰黄的烟尘翻腾滚滚,却不漫溢而出。啾啾的鸟鸣声,又似箭枝疾飞,穿透空气的尖啸声,从烟雾中隐约传出。

这番异象持续了半柱香的功夫,又被黑暗掩盖。

大地再次合拢,地洞消失不见。

这次土丘再不发热,反而弥漫出丝丝寒气,冻得方源双脚丝丝发僵。

方源吐出一口浊气,将目光投向圣宫所在的方位。

“如果我所料不差的话,接下来圣宫应该大乱了……”

此时,圣宫。

一处偏殿,隐蔽的暗门在树荫的遮挡下,悄然开启。

一位雪发苍苍的老者,以及一位中年蛊师。相继而出。

“老先生,您请慢走。”黑沛大家老将太白云生送出暗门,拍着胸脯保证道,“您放心,有我在。担保您老的一枚来客令!”

太白云生呵呵一笑:“既是黑沛大家老的保证,那就是铁打的事情了。老夫自然放心的很,不必远送,告辞了。”

“告辞。”黑沛大家老右手抚胸,行了一礼,看着太白云生在茂盛的树木的遮盖下。走过转角,消失在自己的视野当中。

“仙尊传承的吸引力真是大啊,想不到就连太白云生,都来贿赂我。”黑沛心中感慨着。

自从黑楼兰开放了八十八角真阳楼之后,主持此事的黑沛大家老,就成了炙手可热的人物。每天。都会有各式各样的人物,前来明拜暗访,套交情者有之,攀亲戚者有之,贿赂者有之,色诱者更有之。

但太白云生的暗访,仍旧令黑沛暗吃一惊。

太白云生德高望重。是当今北原几乎第一的治疗蛊师,活人无数,品行正大光明,影响力极大。

黑沛万万没有料到,暗中贿赂的事情,会发生在太白云生的身上。

“说到底,太白老先生也是人呐。来客令就这么几枚,若换做是我,恐怕早就坐不住了。”黑沛嘿然一笑,下意识地抬头。遥望圣宫顶端的方向。

在那里,灿烂多彩的霞光,已经积蓄得宛若浓雾。

浓雾中,第二层八十八角真阳楼的虚影,已经隐约可见。

“用不了多长时间。第二层就会凝现出来了。”不远处,太白云生遥遥注视着。

多彩的霞光,映照在他雪白的须发上,他皱纹满面的脸上。

太白云生神情恍惚,记忆深处的一幕,浮现在心头。

那一个傍晚,天边的晚霞若火,锦绣绝伦。

还是十四岁的太白云生,遇见了那位改变自己一生的老乞丐。

“少年郎,你给了我一碗水,算是救了我老叫花子的命。你想要什么,说出来,老叫花子必全力满足你!”老乞丐有一头紫红色的乱发,时而疯癫,时而昏迷。但是当他清醒的时候,目光沧桑似海,整个人散发出一股令人难忘的深邃气度。

“我要成为蛊师!”年少时候的太白云生脱口而出。

“那你要想成为什么样的蛊师呢?嘿嘿嘿,我这里恰好有三份完整的传承!第一份传承,能让你浴火踏焰,睥睨凡尘。第二份传承,能令人掌风浮空,逍遥天下。第三份传承,则是穿越生死,扶助苍生。”老乞丐笑起来时,露出一口败落的黄牙。

少年的太白云生,皱眉思考了一会儿,最终选择了第三份传承……

目光中的迷惘渐渐消散,从记忆中抽回心神,太白云生苦笑一声,轻声喃喃:“说到底,我终究是个贪生怕死的人呐。”

年轻的时候,还不觉得,甚至因为见惯了生死,而对此淡漠。

但当太白云生渐渐老迈,往昔健康活力的身躯,变得腐朽不堪时,他便越来越留恋年轻时的美好。

很多时候,人的思想是随着境况而变的。

地球上,生死无法逾越,不得不看淡。但在这里,只要有一线生机,一丝希望,都会挣扎!

只有身临其境了,太白云生才越来越从死亡这个词中,察觉出一股大恐怖!

正因如此,他暗中考察,耐心等待了好几届,最终才看清时局,参加了此次王庭之争,进入了王庭福地。

“如果我能从八十八角真阳楼中,获取到寿蛊,就能增添寿命了。寿蛊虽然难寻,更难买卖,但相信八十八角真阳楼中总归会有。如果最终我无法收获寿蛊,那么就只好尝试晋升蛊仙了。”太白云生心中暗暗思量。

老乞丐给他的传承,非同凡响,乃是直达六转蛊仙的整套传承!

传承的内容,对凡人如何晋升蛊仙,有着详细的叙述。

因此太白云生十分清楚,晋升蛊仙的巨大风险。

晋升蛊仙的过程,需要天气、地气、人气的统一和交融。三方面稍差一筹,都会身亡魂散。

不到万不得已,太白云生并不想晋升蛊仙。因为就算成功晋升成了蛊仙,也不能增长寿命。

但太白云生的这套传承中,却又留给了他一份希望。

只是这份希望,十分苛刻,须得他晋升蛊仙成功时,才有达成的可能。

这些年来,太白云生一直在苦苦搜寻着寿蛊。但寿蛊天地生成,踪迹飘渺,又难捕捉,太白云生至今没有斩获。

“八十八角真阳楼中,应该有寿蛊的。我一定可以在这里寻找到寿蛊!”太白云生遥望着八十八角真阳楼的虚影,暗暗为自己打气。

但下一刻,他的双眼陡然睁大,看到了一幕令他感到不可思议的景象!(未完待续。。。)

------------

\end{this_body}

