\newsection{刺杀}    %第九十九节:刺杀

\begin{this_body}

秤砣般的十钧之力蛊,在方源的头顶静静地悬浮着。

方源闭目盘坐,浑身都笼罩在十钧之力蛊的光辉之中。

良久。

方源睁开双眼,光辉缓缓消散,十钧之力蛊只剩下原先一半的大小。

“还能再用两次。”方源默默估算。

十钧之力是消耗蛊,用完这只之后,他便拥有七十钧力道。这样的底蕴,放在四转初阶的力道蛊师身上,也说得过去了。

但全力以赴蛊被压制成二转,若一天不解决此事,力道上的修行便没有质变的可能性。可以说,短时间之内,力道修为上的战斗力还拿不出手。

再依次查看了第一空窍、第二空窍。

第一空窍中的本命蛊,是春秋蝉,仍旧隐去身形,沉睡休养。

九成的晶紫真元海,波光粼粼,映照着五转巅峰的晶壁一片紫色迷离。

海面上空,狼烟蛊宛如一团狼形的乌云。

海面上,漂浮着已经修复完整的战骨车轮,以及洁白柳叶般的雪洗蛊。

五转的蛛丝马迹蛊,像是乌贼一样潜游,时而和狼头鱼肚的狼吞蛊,一起玩耍。

而在真元海底,沉积着大量的驭狼蛊、不少的十钧之力蛊,以及一些狼魂蛊。

同时,还藏有对方源目前最重要的星门蛊、推杯换盏蛊、东窗蛊、葬魂蛊、马到成功蛊。

至于,狼嚎蛊、狼顾蛊、鹰扬蛊、狼奔蛊、敛息蛊都寄托在身体各处。

随着方源深入北原的日子,一天天增多,他的身体已经渐渐适应了北原的环境。第一空窍在北原的修为,已经可以达到五转中阶的地步。

只是方源一直在用敛息蛊,使得自身的气息压制在四转巅峰。

而在第二空窍之中,则是另外一番景象。

四面的晶膜,将空窍照耀得一片透亮。

九成的真金真元海上,波光荡漾。

经过这些天的修行。方源的第二空窍也从原先的三转巅峰,达到四转巅峰。

空窍的正中央,是三转的全力以赴蛊。

除此之外,便是可以将兽力虚影,转化为凝实气流的三转力气蛊。身体越是受伤,力量便越强的四转苦力蛊。

四转的横冲直撞蛊,三转的兜率花、元老蛊。四转的费力蛊,以及具有治疗作用的三转自力更生蛊。

至于原先其他的金刚怒目蛊、点金蛊、乌七蛊、血颅蛊、骨肉团圆蛊、阴阳转身蛊等,暂时都用不上,因此都留在了狐仙福地。

因为从琅琊福地中出来,方源第一时间出现在北原,因此第二空窍。被北原首先承认。四转巅峰的修为,没有受到丝毫的异域压制。

第二空窍的修为,能有如此飞速的进展,多亏了方源之前购买的那些舍利蛊。

但碍于仙元石有限,他无法再买紫晶舍利蛊。因此接下来,第二空窍的修为只能依靠方源自己的力量,按部就班地修行了。

“四转到五转。是质变的过程,差距巨大。今晚索性直接将第二空窍的修为,突破到五转初阶!”方源见时间还很充足,索性继续盘坐在蒲团上,决定冲刺五转境界。

第二空窍的修为,已经达到四转巅峰,底蕴已经积累足够。资质不能高于第一空窍,但由于琅琊地灵亲手炼制。因此资质上也是九成。

寻常蛊师,只消有了这两个条件,就有充足的冲刺五转境界的资本。

通常他们失败过几次后,经验充足了,都会成功晋升。

但在经验这块,方源向来都是强项,这个关隘对他并不存在。

更关键的是。第一空窍和第二空窍的真元,可以相互共用!

天下没有两片完全相同的树叶,同样的蛊师之间,真元各异。若不动用骨肉团圆蛊这类的蛊虫。蛊师相互灌输真元,会导致异种真元纠克,最终空窍爆炸。

然而不管是第一空窍,还是第二空窍,都是方源一人所有。两个空窍中的真元,可以百分百互借调用,本质完全相同。

“起。”

方源心中默念一声,第一空窍中的晶紫真元,便逆冲出来,冲入胸膛中央的第二空窍里面。

属于五转巅峰的真元,冲击四转晶壁,果然有着强烈的效果。

到了黎明时分,方源成功地突破到五转初阶。

这一次冲刺五转,可以说是他有史以来,最为轻松的一次。

“只是因为用了第一空窍中的真元,导致现在第二空窍也受到异域压制了。”方源感受了一下,虽然现在第二空窍中,有着淡紫真元。但是催动的效果,仍旧相当于原先的真金真元。

“但半个多月之后,第二空窍上的异域压制,便会消散。三个月后,第一空窍也将彻底融入北原的环境,不再受到异域压制!到那时,王庭之争也已经步入尾声了……”

方源呼出一口浊气,站起身来,活动活动筋骨。

一夜不眠不休的修行,让他感到微微的疲惫乏累。

他推开密室的门,门前的两位三转蛊师立即反应过来,向他行礼问好。

其中一位,还告诉了方源一个好消息:“狼王大人,我们的蛊师在野外幸运地捕捉到了一头鱼翅狼。正关押在笼子里,族长关照下来,若是大人结束了修行,就可到辎重营去,将其收服。”

这个消息带给方源一个意外的惊喜。

鱼翅狼乃是异兽,相当于四转蛊师战力。虽然方源在宝黄天中,收购了一批异兽狼,但因为解释不清,因此没有放出来。

若是有一头鱼翅狼,在身边护卫着。在战场上,方源无疑会更加的安全。

片刻之后,方源走进辎重营。

“土波参见狼王大人。”一位三转的蛊师,连忙出来迎接。

他长得又矮又胖,肥肥的脸上满是油光,谄媚地道:“狼王大人,小的已经等候多时了,这就领您过去。”

在土波的带领下。不一会儿,方源等人便隔着木笼,见到了那头鱼翅狼。

鱼翅狼体型大如象,此刻趴在笼子里,浑身包裹着鳄鱼似的的坚韧皮甲。

在它的背部,有一排类似鲨鱼的蓝黑色鱼翅,从狼头一直延伸到狼尾。

晨曦的光。照射在它的身上,这头鱼翅狼闭着眼睛,在昏睡蛊的作用下,已经失去了知觉。

“恭喜大人,鱼翅狼乃是防御力最强的异兽狼。有此狼护卫,大人如虎添翼了。”

“更难得的是。鱼翅狼不仅可以在陆地上作战,而且还能潜游水下,战力更强!”

两位护卫的三转蛊师,看见这样神骏的鱼翅狼,纷纷开口,恭贺方源。

方源微笑着,看着眼前的鱼翅狼。眼睛微微眯起来,漫不经心地问道:“捕捉到这头异兽狼,牺牲了不少人吧?”

土波知道是在询问自己,立即答道:“那是!牺牲了足足四位三转蛊师,二转蛊师丧命的至少有两百多人。若非汪家、房家两位族长及时支援,这头鱼翅狼就跑了。”

方源点点头,眼睛眯成了一条缝:“这鱼翅狼的身上,伤痕累累。但在我看来。却似乎有着旧伤?”

“嗯,是的。若非有旧伤,侦察蛊师也难以逃出生天,赶回来报信了。可见狼王大人,得到了长生天的眷顾。在大战来临之际,将一头受了伤的鱼翅狼送到大人您的跟前。”土波拍马屁道。

“幸运么……”方源喃喃一声,心中的不妥之感越来越强烈。

他也说不清。这不妥之感由何而来,只是莫名地感到一丝危机。

他询问了几句,也没有觉察出什么不对的地方。

鱼翅狼乃是异兽,相当于四转蛊师强者的战力。因为身上有旧伤。因此才被拖延,被生擒。

这一切都很合理。

唯一不合理的地方,便是方源心中的不安之感。

但方源却极其重视这种感觉。

这种感觉,他前世刚刚穿越过来时并没有。是他经历了数百年的磨难,无数次险死还生的经历之后,从丰富的人生经验中积累而出的一种直觉。

常言道,人老成精。一个人哪怕再笨,吃的亏多了,受到的磨难多了,见过的东西多了,自然而然就会形成生存的智慧。

事实上,不仅是人,普通的野兽也对危险的来临,有一种直觉和敏感。

在周围蛊师期待的目光中,方源掏出一只四转驭狼蛊。

“给你,你去收服了这头异兽狼。”令他人感到意外的是,方源并没有亲自动手,而是将驭狼蛊交给了土波。

“让小的用?”土波诧异,“可是在下的修为只有三转……”

“少废话,快用。”方源不耐地厉喝一声,强行将驭狼蛊塞给了土波。

土波无奈,不晓得狼王这样的大人物有着什么古怪的脾气,但碍于方源的威势,只得灌输真元。

他催动了好一会儿,累得浑身大汗淋漓之时,这才将四转的驭狼蛊缓慢地催动起来。

驭狼蛊化为一道奇光,颤颤巍巍地落到鱼翅狼的身上。

“唉……”一声充满遗憾的女性叹息,在众人的耳边突兀地响起。

刹那间,方源心中警兆陡升,他想都没想,身形爆退!

一股战栗感,瞬间席卷在场每个人的灵魂深处。

几乎在同一刻,土波陡然张大嘴巴,发出一声凄厉的惨嚎后,当场毙命!

两位三转蛊师护卫骇然莫名,他们根本就不知道土波是如何丧命的。一时间,他们下意识地跟着方源,向后飞退。

但旋即,其中一位忽然身躯抖震,身体还在半空中,就没了气息。

“是魂爆……”方源脑海中灵光一闪,脱口而出。

“狼王大人果然见识非凡。”一声女子的轻语,在他耳边响起,随之而来的是涌动的暗影。

暗影如剑,重重叠叠,仿佛黑孔雀乍然开屏,阴狠犀利,将方源的身躯裹住。

四转――多重剑影蛊!

叮叮当当。

顿时,密集的声音响成一片。

多重剑影斩在方源的身体上。宛若金铁相撞,爆发出一阵耀眼的火花。

方源的皮肤变成一片墨绿之色,若仔细观察,还有龟壳纹路连绵一片。

五转――龟玉狼皮蛊!

“女贼子!”剩下的另一位三转蛊师,看到方源正被攻击,立即大吼一声,转变方向。赶来帮忙。

偷袭方源的女蛊师,冷哼一声,却是不管不顾,催动得多重剑影越发狂暴。

同时,她张口一吐,吐出一条丝线长虫。

长虫宛若黑线。对周围的剑影视若无睹,直朝方源的耳朵钻去。

方源面无表情,目光冷冽宛若冰山,他陡然伸出右手,猛地抓住赶来支援的三转蛊师护卫。

“狼王大人!”三转蛊师十分惊愕,他赶来是为了保护方源,但万万没有想到。方源居然一把抓住了他。

趁着他惊愕失神的一刹那,方源将他一把抓到自己的身体右侧,挡在方源和鱼翅狼的中间。

几乎在同时,三转蛊师啊的一声,浑身抽搐,双眼翻白,口吐白沫!

丝线长虫乘此良机,一把钻入方源的耳朵之中。

方源闷哼一声。松开三转蛊师,双拳对准黑屏般的剑影直捣过去。

女蛊师察觉到这一击的庞大气力,轻笑一声,并不硬拼,乍然收起多重剑影蛊,身躯化为一道黑影,倏地退出二十步开外。

黑影落到帐篷的阴影处。又化为一个女子。

这女子娇小玲珑,一身黑衣,脸上蒙着黑色轻纱,只露出一对狭长的丹凤眼。

她浑身散发出幽暗阴寂的气息。给她的美丽增添一份鬼魅精灵般的魔力,叫人一眼看去,就深入人心,难以忘怀。

“晚辈无影剑边丝轩,见过狼王大人。”女子对方源微微一礼,身在敌营,众敌环伺,她却好整以暇,从容淡定。

方源冷哼一声,厉声责问:“你刚刚给我种下的是什么蛊?”

边丝轩轻笑一声:“乃是晚辈一次探险时,在某处遗迹中意外发现的诡异蛊虫。一经发动,便深入人耳,钻入脑髓当中。只要该人稍稍急速思索,此虫便会飞速涨大,直到将头脑撑爆。因此晚辈取名为爆脑蛊。”

方源脸色一沉。

边丝轩再行一礼,语气中充满了真情实意的敬佩之情:“前辈竟能察觉东方公子精心设计的必杀陷阱,甚至避过了魂爆的绝大部分威能,实在令晚辈佩服万分。能取得前辈的性命,也是晚辈的莫大荣幸,告辞了。”

话音刚落,她化为一道黑影,在各处建筑的阴影中飞速投射。

“是影剑客!”

“该死的,拦下她。”

闻声赶来的大批蛊师,纷纷怒吼出声,密集的攻击打在四处的阴影中,但边丝轩的那道黑影,早已经消失无踪。

她走了,还是仍旧留在这里?一时间,众人都不敢立即确定。

“属下来迟一步,请狼王大人恕罪!”

“狼王大人,您没有什么事情吧?”

担忧至极的众人,很快又将方源团团围住。

方源虽然肉体上没有受什么伤害,但皮毛被剑影削去甚多,显得比较狼狈。

“我能有什么事?一群无能的废物,被对方摸进辎重营都不知道!都给我滚!”方源气急败坏地吼道,心中却是暗喜。

不想一场刺杀,竟将盗天魔尊的一处传承线索,送到了自己手中!

爆脑蛊?

真以为我来不及遮挡?

哼,没见识的小辈……

在方源五百年的记忆中,这影剑客边丝轩也是重要人物。

她是马鸿运的妻子之一,将来成就六转蛊仙。正是因为她手中的这只“爆脑蛊”,马鸿运成功获得了盗天魔尊的一处传承。

只是这传承是什么,具体经过如何,马鸿运一直避而不谈,因此方源也不清楚。

只知道如何正确地开启这只“爆脑蛊”。

“马鸿运都避而不谈,可见这处传承的收获之大。应该是害怕说出真相,引起他人觊觎吧。”方源表面上满脸惊怒之色,内心却在冷静地分析着。(未完待续。。。)

\end{this_body}

