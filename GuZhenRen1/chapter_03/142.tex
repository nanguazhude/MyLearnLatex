\newsection{马鸿运}    %第一百四十节:马鸿运

\begin{this_body}

一位凡人来算计蛊仙,成功的可能性有多大呢?

纵然方源是五转巅峰,但也难以逾越仙凡的差距。

方源前世就是蛊仙,恐怕没有一个凡人,能比得上他对这等差距的深刻理解。

好在,方源的计划里,并非是一人敢干蛊仙。他要借助八十八角真阳楼的力量,前世的宝贵经验给了他光明的指引,还有琅琊地灵提供的珍贵信息,让他的计划更增几分把握。

这样一来,谋夺江山如故仙蛊的可能,就有了两成!

两成的可能性,在救治荡魂山的三个方案中,已经是最高的了。

能抗衡和稀泥仙蛊的,只有仙蛊。

凭借超越五百年的远见卓识,方源所知的,能在当下救下荡魂山的仙蛊,只有三只。

第一只,是土道六转化石蛊。这蛊如今掌握在西漠六转蛊仙孙醋的手里。

第二只,同样是土道六转仙蛊,名为东山再起。收藏在东海的海市福地当中。

第三只,就是宙道六转仙蛊江山如故了。此蛊还未诞生,并非自然形成。其主太白云生,目前还只是北原的一位五转蛊师。

要夺化石蛊,方源就要应付成仙已经十多年的蛊仙孙醋。

若是图谋东山再起蛊,那情况更糟,方源将置身于一群蛊仙的注视之下。以一届凡人身份,来换取仙蛊?这无异于揣着黄金的小孩,去逛黑市。

所以,关于江山如故蛊的第三个方案,才是风险最小,可能最高的。

哪怕太白云生成就蛊仙,那也只是一位蛊仙新嫩,对于境界和质变的力量,都不会很熟悉。

这样的对手,可比老资格的孙醋,以及海市福地的那群蛊仙,容易对付多了。

……

北原历,十二月。

风雪渐大,次数越加频繁。就算没有风雪,洁白的寒霜也遍布了整个北原。纵然太阳升空,往日里炽热的阳光,也变得软弱无力。

十年暴风雪灾来临的日子,已经越来越近了。

天川,暖沼谷。

“族长,这就是丙字号元泉。”马由良忧心忡忡地指着干涸见底的元泉,为马英杰介绍道。

马英杰的眉头深深皱起。

丙字号元泉,乃是暖沼谷中,仅剩下的三口元泉之一。

如今干涸,那么支撑马家的就只剩下甲字号、乙字号两口元泉了。这对于马家部族而言,是个噩耗。

北原的元泉,和南疆等地的元泉并不一样。

北泉大多泉水稀少,泉口狭窄,喷涌猛烈,底蕴稀薄,持续时间最短。

东泉量足,南泉潺潺,北泉剧烈,西泉精粹。

在南疆,一个中小型的部族,可以连续十几年,使用一口元泉。南疆的元泉,只要不过度开发,就可以持续取用,细水流长。

而在北原不同。

北原的元泉,形成快,溃散得也快。再加上北原战事浓烈,一个中小型部族,至少需要三四口元泉,才能支撑。

马英杰回过部族后,成为马家的新任族长。马家冲击超级势力失败,如今落败为小型部族。偌大的暖沼谷,都显得空旷。

马家的粮草、水源都是有的,而且准备充分。

但是元泉是货币,更是蛊师修行必备的重要物资。一旦雪灾来临,暖沼谷这类的地方,就成了最后的避难所。

不仅是兽群,更有其他的蛊师,前来栖息。

作为地主的马家,不仅要抵御风雪灾害,而且还要和这些人交涉。

元泉产出的元石,就是支撑蛊师战力的脊梁。现在马家的三个脊梁,断裂了一根。虚弱的马家,只剩下三成的底气。因为丙字号元泉的干涸,底气立马就消散了一成。

然而面对这样的难题,马英杰也没有对策。

如果他有一只“江如故”蛊,立马将丙字号元泉恢复原状,这个问题立即就解决了。但马英杰没有。

“族长大人,元泉真的这么重要吗?”回去的路上,费才问。

他回归家族之后,作为马英杰的救命恩人,立即解除了奴隶身份,如今已经是自由人。

同时,仍旧是马英杰的贴身随从。

马英杰忧心忡忡地点头道:“元泉干涸,对蛊师影响极大。而蛊师,则是一个部族的支柱。我们马家不仅需要蛊师们作为主要力量,来抵御雪灾时的各种灾害。而且雪灾之后,更要依靠蛊师战力,去抢夺新生的资源,去发展部族……”

费才哦了一声,边走边问:“那么我们能不能找到新的元泉呢?我是说,您看,咱们的暖沼谷这么大,说不定不止这三道元泉呢。”

费才的话中,充满了乐观的精神。

马英杰苦笑一声:“北原的元泉,的确会在短期内形成。暖沼谷中,也许有第四口元泉。但这个可能性太低了,几乎不可能。你要知道,每当十年雪灾来临,北原各地的元泉都会相继干涸、枯死。等到雪灾消退之后,新的元泉会大量地涌现出来。到那时,北原各地都有丰美的水草,间隔百里,兴许就有一口元泉。那将是每一个部族,每一个兽群发展壮大的最佳时机。”

“是这样……”费才这才明白过来,他活了这么大,却是懵懵懂懂,对这些状况不太熟悉。

“啊!”忽然他惊叫一声,从路边摔倒下去。

他们俩走的是山崖边上的路,好在不是峭壁,而是缓坡。费才一失足,就顺着缓坡一路滚下去,发出不断的惨叫声。

“你这家伙……”马英杰被费才这一连串的搞笑的惨叫声逗乐了,紧皱的眉头稍稍地舒展了一些。

“你这个路痴,现在连路都不会走了吗?快给我赶紧地爬上来……嗯?!”马英杰忽然顿住,双眼瞪圆,难以置信地看到一道崭新的元泉,在缓坡上喷涌而出。

这道元泉口,原本盖着一块山石。

但这块山石,被跌滚而下的费才给撞到了一边去。山石覆盖下的元泉,这才显露了出来。

很显然,这是一道最近形成的元泉。否则,战前马家探查,不会探查不到。

这道元泉水量巨大,短短功夫,就有上百颗元石,顺着泉水,喷洒而出,落到周围的地上。

“这,这竟然是一口新元泉,直接超过了甲字号元泉!”马英杰大喜过望,甚至眼眶都微微泛红,“这就是所谓的否极泰来吗?长生天在上,一定是先祖的保佑!”

“族,族长,我来了!”这时,费才吱呀咧嘴地赶上来,看到新泉后,同样瞪大了双眼,“奇怪,这里怎么忽然有一口泉水?”

马英杰哈哈大笑:“费才,你是上天送给我的幸运星。从今天开始,你就改名字吧。不要再叫费才了。费才,废材,我马英杰身边怎么可以有废材?从今以后,你就叫鸿运。费鸿运!预示着我们马家鸿运高照,否极泰来!”

然而,马英杰并没有高兴太久,七天后,黑家大军来到这里,包围了暖沼谷。

就在黑家大军驻扎下来的当晚,暖沼谷的三口元泉,同时化为黑水,彻底污染。

一封劝降信,随后送到马英杰的手中。

马英杰没有料到,黑楼兰明明已经取得了最后的胜利,居然还不放过自己!

被蛊虫污染过的元泉,已经再不能出产元石,都被废了。虽然还有元石的库存,但马家已经失去了继续居住在暖沼谷的可能。

“想不到黑楼兰如此睚眦必报!他在信中,要求我们马家归降。这也不算违背巨阳仙尊当年设下的规矩!可恶,可恨!”马英杰捏紧双拳,心中充满了愤怒、仇恨、无奈、无力。

“黑楼兰,号称黑暴君,一向暴虐粗鲁。看来他是因为之前的大战,对我们马家产生了忌惮之心。但碍于巨阳先祖的规矩,想将马家置于他的眼皮子底下,继续打压马家。”马由良瘫坐在椅子上,用低沉的语气分析道。

顿了一顿,马由良又道:“其实这样也不错。马家归降他黑楼兰,我们也可以进入王庭福地里了。”

马英杰摇摇头:“这正是黑楼兰的用心险恶之处。马家固然能够进驻王庭,但其他人呢?我问你,现在的部族中,姓马的亲族有多少?”

马由良脸色一白:“只有一百三十余人。”

“正是如此。”马英杰神情沉重地点点头,“我们马家想要发展壮大,就要招纳外人,就要大肆结亲、大量生子。但黑楼兰只要一个命令,不允许我们马家接纳外人,甚至只允许我族内部通婚。到那时,我们马家想要壮大,该是猴年马月的事情?”

马由良脸色更白了一分。

他意识到问题的严重性。

政治是肮脏的,不允许迎娶外族人这点很容易做到。黑楼兰只需要一个保持黄金部族血脉纯净的理由,就可以名正言顺地遏制马家的壮大。

“那我们该怎么办呢?”马由良失去了主意。

马英杰沉默了一会,终是下定了决心,他咬牙道:“我们将所有的外人,都赐姓马,接纳为本家!”

“族长大人,这样做的话,我们马家的黄金血脉,恐怕真的就要……”马由良迟疑了。

“我们必须防一手。黄金血脉,是我们马家的骄傲,绝不会被污染。如果局面好转,再将这些人贬斥出去,剥夺马姓就是了。”马英杰道。

马由良这才放下心来,缓缓点头,认同了族长的这个策略。(未完待续。请搜索,小说更好更新更快!)

------------

\end{this_body}

