\newsection{人肉试验}    %第一百六十八节:人肉试验

\begin{this_body}

%1
花园中,银辉烂漫,草木招展。

%2
坐在凉亭里,方源把玩着手中的地魁尸蛊。

%3
此蛊形如蚯蚓,通体嫩黄,有婴孩的前臂长短,此刻盘绕在方源的手指之间,宛若蛇般蜿蜒游走。

%4
正是用了墨瑶提供的改良蛊方,在地魁万兽王的尸躯血肉的基础上,炼制而成。

%5
不过,方源却没有催动试用。

%6
他生性谨慎,这地魁尸蛊乃是改良蛊方所炼,出了问题怎么办?

%7
虽说和脑海中的墨瑶意志达成了妥协,对方又是正道灵缘斋仙子,且又是炼道宗师,却不可不防。

%8
“不过宗师就是宗师,我的炼道造诣,已经是大师境界。但是和墨瑶相比起来,却仿佛幼稚如孩童。”

%9
回想起炼制地魁尸蛊的过程,方源心中感慨。

%10
整个炼蛊过程,墨瑶只是提点了三句,却是提纲挈领,画龙点睛。方源受益匪浅,醍醐灌顶,。

%11
他却不知,墨瑶意志此刻也在他的脑海中暗暗感慨。

%12
“双大师……没想到这小子不仅是奴道大师,而且还是炼道大师。这么年轻,就是双大师,这样的才情天赋,就算在我的记忆中,也是不多了。难怪他野心勃勃,想跻身尊者,媲美盗天、乐土、巨阳。”

%13
“年轻人少年得志,狂妄自大些也是自然。”墨瑶对方源媲美尊者的“壮志”仍旧嗤之以鼻,但却可以理解了。

%14
“不过,成为大师,单靠天赋才情还不行,还得有充分的资源供给。甚或是名师指点。看来这小子背景深厚啊。”

%15
之前,墨瑶已经知道,方源坐拥狐仙福地的秘密。现在再结合“双大师”的情报,她更觉得方源大有来头。

%16
“狼王大人,葛家、常家族长前来觐见了。”就在这时。有奴仆过来通禀。

%17
方源收回思绪,这两个人本就是他召唤过来的,淡淡而道:“宣。”

%18
“是。”门外,奴仆恭敬而退。

%19
不多时,两人过来,跪地而拜。连磕了三个响头,神态极敬且畏。

%20
经过王庭之争的洗礼,又身处高位,资源不缺,在狼王这棵大树之下,葛光、常极右已经都是四转蛊师了。

%21
方源瞥了两人一眼。没有令他们起身,而是直接问道:“我前些天,吩咐你们的事情,完成的如何了?”

%22
方源得胜,回归圣宫之后,就下令两位族长,令两族蛊师齐齐出动。替他打扫战场。

%23
“回狼王大人的话,战场已经打扫干净,共有……”葛光刚想回报收获,就被方源打断。

%24
这些战利品,他并没有真正放在心上,只是问道:“拘拿了多少蛊师?”

%25
葛光这次没有回答,而是跪在地上,以目示意身旁的常极右。

%26
众所周知,常极右乃是常山阴的亲生儿子,方源执掌常家之后。就命常极右担任族长之位。

%27
方源询问,葛光让常极右回答,也是看了这层关系,刻意示好常极右。

%28
常极右神情恭敬,目光中却又透露出狂热的崇拜。他朗声道:“人心叵测,欲壑难填,尽管有父亲大人公告,但仍旧有不少蛊师潜入战场,偷取兽尸或者野蛊。这几日,孩儿和葛光族长合力,共拘拿了一百八十多名蛊师,如今都关押在地牢当中。但仍旧有许多狡诈奸猾的,趁我们不备,偷偷进出,不劳而获。孩儿通过审问,已经掌握了不少情报。只要父亲一声令下,孩儿必定尽心竭力,将这些漏网之鱼也都捉拿归案!”

%29
常极右虽跪在地上,但上身挺得笔直,他鼻挺眉黑,狼背蜂腰,语气昂然,大有英武之气。

%30
方源微微一笑,他对抓捕漏网之鱼毫无兴趣,道:“战场宽阔广大,没有门户关隘可以把守,能够拘拿这么多蛊师已经不错了。你们俩多做得很好。至于其他人,能让他们偷去,也算是他们的本事,就不必追究了。你们退下,将这些俘虏都押到我这里来。”

%31
“是,属下(孩儿)领命!”

%32
不多时,两人将近两百位俘虏,都押解过来。

%33
依方源之命,专门腾出一座大殿,关押这些蛊师。

%34
“你们都出去,紧闭大门,护卫左右,方圆百步以内不得有外人出没。若有强者,提前告我。”方源将其他人员尽数遣散,只留下他一人以及俘虏。

%35
大门关闭,又没点灯,大殿陷入一片黑暗。

%36
这无疑加重了俘虏们心中的不安和焦虑。

%37
“狼王大人,你抓我们,意欲何为?你可知道,我乃是黑家族人。若论关系……嘿,黑楼兰大人还是我的表哥呢!”俘虏中,一位年轻蛊师叫出声来。

%38
方源冷笑,催动蛊虫,弹指一挥。

%39
砰。一声轻响,将其头颅打爆,宛若破裂的西瓜也似,间杂花白的脑浆。

%40
众人大哗,受了惊吓,顿时一片慌乱。

%41
旋即一人带头,无数人响应,纷纷跪在地上。

%42
“狼王大人,小的该死!”

%43
“千不该万不该,去偷大人的战利品,小人是被猪油蒙了心啊!”

%44
“求大人饶命,求大人饶命啊……”

%45
方源本来就凶威赫赫,如今一言不发,就动辄杀人,实在凶残。

%46
这些蛊师,又都非强者,多以一二转之流,出身不好,进不得八十八角真阳楼,面对方源这样的五转巅峰强者,想要对抗只能是自找死路,只有讨饶一途。

%47
“聒噪。”方源轻喝一声,声音响动大殿。

%48
随手一挥,又隔空杀了位求饶声喊得最凄厉的某人。

%49
“谁敢再吵闹,都杀了。”方源淡淡开口,音量不大,偏偏萦绕众人耳畔。

%50
整个大殿,噤若寒蝉。死寂无声,掉下一个针头都能听得见。

%51
方源这才满意,脑海中询问墨瑶意志:“接下来,该如何试验?”

%52
墨瑶轻笑一声:“这个容易,小弟且听我吩咐安置蛊虫便是。”

%53
她直接称呼方源为“小弟”。语气透出一股亲昵的意味。方源心中冷哼一声,却没有反驳,而是顺着墨瑶的嘱咐行事。

%54
墨瑶每关照一句,方源就应声飞出一只蛊虫。

%55
这些蛊虫,都是组成六臂天尸王杀招的蛊虫,地魁尸蛊、修罗尸蛊、天魔尸蛊等等。或被投放到大梁上,或是安置在角落里。

%56
一只只的蛊虫划空而去,在众俘虏的眼中,留下一道道彩色光影。

%57
这些人惶恐不安,却不敢询问什么。他们站在原地,不敢动弹。宛若小鸡崽子。

%58
方源杀的两人,无头尸体还躺在他们的脚下,从断颈处汩汩流血,血腥气味渐渐弥漫大殿。

%59
将这些蛊虫安排妥当,方源又按照墨瑶之言,接连调动真元灌注。

%60
这灌注的顺序似大有讲究,忽而选择东南角落里的蛊虫。忽而直奔西北角落,又忽的转移到左右两侧。时而是主蛊,时而是辅助蛊虫。

%61
即便是方源,也是一头雾水,暗暗警惕。

%62
待蛊虫全数催起,各团光辉渐渐勾连,很快形成一片瘟黄之光,覆盖大殿内部,形成一座光屋,将殿中一干俘虏俱都囊括进去。

%63
“这是?!”方源心头震动。瞳孔扩张。

%64
脑海中,墨瑶意志轻笑一声,解释道:“这就是蛊屋了。”

%65
方源不禁失声:“这六臂天尸王,其本质难道是一座蛊屋?”

%66
墨瑶呵呵一笑:“呆小子,你难道不知道蛊屋的本质就是杀招么……六臂天尸王是杀招。蛊屋也是杀招,本质相同,自然可以互相转换了。”

%67
方源眼中精芒烁烁,他察觉到墨瑶的言外之意,立即追问道:“照你这么说的话,岂不是所有的杀招,都能转化为蛊屋了?”

%68
“这是当然。”墨瑶斩钉截铁地回答道,“人是万物之灵,蛊是天地真精,大道载体。一只蛊虫,功效单一。杀招是什么?就是将不同的蛊虫组合起来,功效相互叠加,或令单方面效果绝伦,或令功效繁复,兼顾多面。”

%69
墨瑶点到即止,能不能有所领悟,就是方源自己的事情了。

%70
方源愣神,灵光闪现不停!

%71
墨瑶的这番话,像是给他捅破了那层窗户纸,将他的眼光抬升到一个全新的高度。

%72
“是了!蛊屋的本质是杀招,只是形式固定下来。譬如八十八角真阳楼,再好比近水楼台。前者覆盖王庭,影射北原,收刮资源,贮藏传承。后者飘渺隐匿,又有防护贮存之用。这些功用,单个的蛊虫提供不了,是蛊师们将这些蛊虫组合起来,才达到的效果。”

%73
“换个角度想,蛊屋不过是杀招的一种表现形式罢了。六臂天尸王既然能只作用我一人,为什么就不能形成蛊屋,同时作用许多人呢?”

%74
念及于此,方源再看眼前。

%75
蛊屋之中,那些蛊师俘虏的身体已经开始发生变化。

%76
“我,我怎么了?!”众人惊恐大叫,他们发现自己的皮肤上,迅速长出金色鳞甲。

%77
“啊,好痛,好痛!”“痒,太痒了,我受不了了,干脆杀了我吧!”很快,他们接二连三倒在地上,有的狂抓皮肤,将衣服都撕扯成缕,有的弯弓如虾,捂住心口,口鼻溢血。

%78
“这,这是什么东西?我的背后怎么长出了怪臂!”不久后,人们惊动的尖叫声达到了顶峰,声震屋瓦。

%79
手臂接二连三地长出来,每条手臂都不一样,有的墨绿,有的暗紫,有的枯黄,且又粗细不一,畸形难看。

%80
但方源从不计较外表这种细枝末节,他双眼微微眯起,感受到蛊师们散发出来的危险气息,他心中喜悦之余,又有凛然——

%81
“墨瑶对杀招的理解,可谓精深超凡。但为什么旁人就没有这样的领悟?还是因为大多数蛊师,乃至蛊仙,都没有将杀招随意转变形态,化为蛊屋的本事!炼道宗师……这样的境界,真是厉害!”

\end{this_body}

