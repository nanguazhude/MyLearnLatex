\newsection{洪福齐天马鸿运}    %第三十六节:洪福齐天马鸿运

\begin{this_body}

“来一来,尝一尝了啊,喷香喷香的米饼。”

“卖茶了,卖茶了,顶级的茶砖!”

“阿苏家的马奶酒,十年老字号,谁喝谁知道。”

……

杂乱的小地摊随处铺着,空气中香味扑鼻,吆喝声、讨价还价的声音此起彼伏。

方源行走其中,周围人摩肩擦踵,喜庆的氛围如此浓烈。

不仅有卖吃的,也有卖袍子的。

有普通的羊皮袍,狗皮袍,稍微高一档次的牛皮袍。还有漂亮的狐狸雪袍,战士们常常穿戴的镶着甲片的狼皮袍。

孩童们在小吃摊周围流连忘返,男人们着蹲在买铁器的摊子前讨价还价。女人们则挑选着瑙、宝石、珍珠、金银制成的首饰。

继续往里面走,方源便看到一个草草搭建起来的广场。

广场上,有巨大的木头笼子,笼子里装满了人。

笼子外,有专门的蛊师把守,一位腰腹便便的胖子在使劲的吆喝:“卖奴隶了啊,五个男奴才卖半块元石!”

方源瞥了一眼,便知道这是北原盛行的奴隶买卖。

这些人,是战败后被吞并的其他部族。被活捉后,失去了自由,成为了货物。

笼子里面有衣不蔽体,瘦成麻杆的小孩子,也有锁着铁链,低头跪坐着的男子。当然还有女人。

胖子吆喝了半天,额头现汗,见众人只是观望,他眼珠子一转,打开木头笼子,从里面拽出一个满脸泥垢的女子。

“快看哪,这可是上好的女人。”

哧的一声。他将女人本就破烂的皮袍撕开,露出乳.房。

“你们看看,多么丰满的胸!”

然后,他把女人调转个方向,当众拍拍她的屁股。

“再看看这,多么肥大的屁股,多能生养啊。买回家能生娃子,还能干力气活!”

整个过程中,女人神情麻木。宛若木偶,任人摆布。

胖子的卖力吆喝,总算勾得台下的人群一阵骚动。

当即便有人叫道:“这个女人怎么卖?”

“三两的元石,只卖三两的元石。”胖子立即竖起三根粗短的手指头。

三两元石,连半块都不到。

但台下刚刚问价的那人。却高喊起来:“什么,三两!你抢劫啊,有这钱我还不如再凑二两,买头大胃马合算!”

胖子满脸是油的肥脸,立即抖了一抖,往脚下吐了一口吐沫:“屁!买个女人能在床上骑,买了大胃马。你虽然也能骑它,但它能给你生娃吗?穷鬼,不买就滚!”

胖子乃是一转蛊师,被骂的那人只是凡人。立即缩着头灰溜溜地走了。

方源饶有兴趣地望了几眼,便将目光收回。

胖子卖的这些奴隶,只是凡人,当然卖不出什么好价格来。但如果卖的是异人的话。就好卖多了。若是蛊师,就是高等奴隶。价钱最高。

看着这些奴隶,方源就不禁想起马鸿运来。

此子就是奴隶出生,有着逆天的运气。

先是以奴隶的身份,参加战斗。结果本家部族溃败,他在逃命中阴差阳错地救下了少族长。因而建立了大功,赐予马姓,不再是奴隶。

他成了马家部族中的一位普通凡人。为了讨生活,他出门狩猎,因为技艺不佳,毫无所获,归来的路途中被一块石头绊倒。他发狠击碎这石块,结果从石头底下发现了一只白银舍利蛊。他将蛊虫奉献给本家的少族长。

少族长正需要这只白银舍利蛊,得之大喜。不仅重重地赏赐了马鸿运,还给他修行的机会。

马鸿运开启了空窍,资质乙等,也没有给力的蛊虫。经常被周围的蛊师欺负,一次把他踢到河里去。

结果不会游泳的马鸿运,被灌了一肚子的水,顺着湍急的河水,漂到下游去。

下游有个圣家族长的三女儿圣灵儿,正在洗澡,结果被马鸿运看个彻底。按照圣家的传统,貌美如花的圣家天才女蛊师,不得不认命,成了他的妻子。

马鸿运从此得到了圣灵儿的巨大帮助,蛊虫和元石从未短缺过。

圣灵儿甚至,还为他偷了家族的宝蛊,将他的资质增长到甲等。

此事东窗事发,圣家族长自然不愿女儿嫁给这么一个穷小子,便暗中派遣高手,要杀了马鸿运。

结果这个高手,在来的途中,和人发生争执,被另一个高手杀了。

马鸿运后来和圣灵儿结为夫妻,受到重用,被圣家族人嫉恨,暗中谋害。

他不得不逃遁到腐毒草原上去,就要被毒须狼群干掉的时候,发现了常山阴。他把常山阴救活之后,常山阴就成了他的肱骨之臣。

就这样,靠着屡次令人瞠目结舌的好运气,马鸿运一步步往上爬,后来甚至成了王庭之主。

在黑楼兰死后,他连续十多次,做了一百多年的王庭霸主。

后来,他连续得到盗天魔尊、巨阳仙尊的部分传承,成为蛊仙。还被一位蛊师,主动赠予了福地。

当烽烟和战火燃烧全天下的时候,马鸿运成为北原的中流砥柱之一,抗击中洲大军,风头无两。

“现在这个时间,马鸿运恐怕已有十三岁了吧。可惜,我不知道他的真正身世。马鸿运不过是他救下马家少族长之后,马家赐给他的姓名。马家是巨阳现在留下的黄金血脉之一,为了争夺王庭,这些年积极扩张,吞并了不少部族。也不知道马鸿运现在,有没有成为马家的奴隶。”

方源收回散漫的思绪,发现自己来到了一处赌石坊的门口。

坊门上贴着对联。

左侧是:小施勇气,得春夏秋冬禄。

右侧是:大展身手,获东南西北财。

横批是:时来运转。

赌石坊的生意非常好,让方源也心动了一下。

“凭借自己多年的经验,或许能发一笔小财。”这个念头。让方源笑了笑,他旋即又想到了马鸿运。

这个运气惊人的小子,受到别人欺骗,拿最低档的石头来给他赌。

结果在纯粹瞎蒙的情况下,他竟然开出了一只五转蛊!

方源还有正事,此时却不忙进赌石坊。

他走过赌石坊的门口,进入这片市集的中心地带。

这里就安静许多了,人流量骤然稀少,视野中几乎都是蛊师。就算是有凡人。也是蛊师的随从,跟在蛊师的身边,低着头手中提着东西。或者一些还未开窍的公子、小姐之流。

虽然没有守卫,但凡人和蛊师之间,自然而然地形成两个泾渭分明的区域。

力量高低铸就的无形门槛。分隔出两种不同的人生。

方源信步而走,所到之处,蛊师们纷纷投来敬畏的目光,走在方源面前的,都停下来,主动避让。

身后一些小声的议论声传来:“怎么出现了一位四转高手?”

“这人相貌颇为陌生,好像不是几大部族中人。”

“小心些吧。每次开市集,都会有不少的魔道蛊师进来销赃。”

四转和三转,是不一样的层次。

三转是中流砥柱,各大部族的家老都是这一层次。四转则是蛊师中的高手。大部分的族长都是四转,因此可以横行俗世。

四转初阶的气息,让他举手投足间,都牵引着众人的目光。

或敬畏。或好奇,或忌惮。

这里专门经营蛊师的生意。

有炼蛊坊。专门代替蛊师炼蛊,同时也出售、收购秘方。

方源缺少驭狼蛊的秘方,几进几出后,顺利地买到了一转到三转的秘方。

四转的秘方,向来都被各大部族紧紧把守,市面上很少看到。

随后,方源又去了酒楼。但没有寻到极品的美酒。

他来到最大的一家商铺。

“尊敬的强者,你的光临是小店的荣幸,快请进。”店主是一位三转蛊师老者,亲自出迎。

“我需要大量的驭狼蛊。”方源直接开门见山。

“好的,请上楼详谈吧。”老者将方源引入三楼的雅间。

令方源惊喜的是,这店铺中居然有一枚四转的驭狼蛊,售价是七万元石,方源当即拿下。

一番讨价还价之后,方源又收购了五只三转驭狼蛊,三十八只二转驭狼蛊。

店主开心得直搓手,心知这次遇到到了大主顾:“尊贵的客人,您还需要什么,请尽管开口。”

“你这里有什么防御蛊呢?”方源便问。

老者报了一大通的蛊虫名字,方源听了微微摇头,有一只四转蛊,却不合他的奴道。剩余的,则都是些普通蛊虫。

接着,方源又问了飞行蛊。

飞行蛊较为少见,因为飞行术难有成就。一般情况是大族中的优秀子弟,从小时候就开始培养。但这样一来,他们有家族的资源,从不缺飞行顾。

而其他的蛊师,也很少去尝试飞行。

店主一脸自豪地道:“客人,你算是来对了。整个市集里,就我家店中有一只四转的腾云蛊!”

但方源摇摇头。

腾云蛊升降虽强,但转向较差,速度中等,适合那些飞行术并不佳的蛊师。

对他来讲,反而不如使用三转的鹰翼蛊,能在空中随意转折,更加称心如意。

买下一只鹰翼蛊,方源便问及骨竹蛊和鬼火蛊。

他需要这两样蛊虫,来修复五转蛊战骨车轮。

结果二转的鬼火蛊,店中有许多。反而一转的骨竹蛊,却没有一只。

店主苦笑着道:“客人,真是不好意思。整个市集上所有的骨竹蛊,都被蛮家新近招揽的外姓家老买下了。不仅本店没有,其他店里也没有。”

“哦?此人是谁?”方源目光一闪。(未完待续)

\end{this_body}

