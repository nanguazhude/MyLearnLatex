\newsection{炼得地魁尸蛊}    %第一百六十七节:炼得地魁尸蛊

\begin{this_body}

一秒记住【】/manghuangji为您提供精彩小说阅读。

狼群过处,如巨浪卷席,血洗福地山河。

方源神情淡漠,端坐天青狼王背上,对脚下的无情杀戮视若无睹。

脑海中,墨瑶意志微皱眉头。

她是灵缘斋培养的一代仙子,正道人物,对方源轻易大开杀戒,自然没有欢喜之感。

“这少年,能伪装身份,在北原蛊仙的眼皮子底下,潜入王庭福地,是悍勇阴鸠之人。又能破析我的传承线索,炼成仙蛊,是有机缘,又手段非凡。更能屈能伸,知道事不可为,就立即妥协,城府深厚!到底是何跟脚呢?”

墨瑶的这段意志,不禁暗暗猜测起来。

她才刚刚从近水楼台中出来,寄托在方源的脑海中,虽然解析了方源的许多念头,一时间知道不少秘密。

但方源反应敏捷,立即采取措施,使得她接下来的进展并不大。

因此,墨瑶意志知晓方源的秘密,其实不多。

这当中,还包括了方源主动暴露给她的秘密。

方源拥有春秋蝉,她还并不知晓。方源是穿越者的最大秘密,她更无从得知。

这是蛊的世界。

墨瑶早已陨落,留下的只是一段意志罢了。无法催动蛊虫,墨瑶意志也就欺的方源智道造诣不深。

“我初来乍到,不知对方跟脚。这个少年又绝对是意志坚定,极有主见之人。还是先按兵不动,引导他将近水楼台归还了门派。只要到了灵缘斋。再劝他向善也不迟……”

墨瑶意志继承了本体的些许智慧,见劝说无功,当即闭嘴不言。

地上无有任何抵抗力量。方源目光幽幽,慢慢收回俯瞰的目光,端坐狼背,开始闭目养神起来。

他并非滥杀无辜之人。

今日纵狼卷席,一路屠戮,自有缘由。

其一,勾引地下的地魁兽群。地魁兽乃野兽。会被浓郁的血腥气息吸引,激发凶残气性,主动冒出地面。

其二。大量屠戮,就能带来大量魂魄。王庭之争起初时,方源就用葬魂蟾一直收集着。他有荡魂山,只要一日荡魂山不彻底死亡。这些魂魄他就多多益善。

其三。却是针对脑海中的墨瑶意志,也能起到试探之用。

至于肆意斩杀蛊师的恶劣反响,方源怡然不惧。

不说他前面发过公告,广而告之,已经有了铺垫。就算民意沸腾,众人怨愤仇恨又能如何呢?

他是凡俗巅峰,是大名鼎鼎的狼王常山阴!早已不是几年前那个如履薄冰的小人物了。

在这个蛊仙进不来的王庭福地当中,他咆哮一声。万众蛊师都要颤一颤。跺跺脚,圣宫上下都要抖三抖。只消念头一动。狼群呼啸,立时便是血染千万里。

地球上有句名言枪杆子里出政权。

拳头硬就是强权,强权就是真理!

不过地球上,在于集众。除去暴力和拳头,还得用大义来遮掩,用民心来调和。

而在这里,个人的力量可以凌驾于集体。民心、大义就孱弱了。别的不说,就拿方源现在来讲我管你狗屁的民怨,谁来灭谁!看谁不顺眼,尽数屠戮了就是。

睥睨众生,践踏民心,自由恣意,快哉如风!

不过方源矢志大道,对随意屠戮蝼蚁,倒真没有什么兴趣。今日大举杀戮,不过是他迈向崇高目标的一小步罢了。

想到此行目标,方源又缓缓睁开双眼。

吼!

一声兽吼,响遏行云。

远处的大地隆起一座山丘,随后轰的一声,土石迸飞,从中蹦出一头高达五丈的地魁万兽王。

方源双眸寒光闪烁,冷笑一声:“便是你了。”

当即指挥狼群攻了上去,一时间,他脑海中念头纷飞,如泡泡细雨,绵绵不休。

从高空俯瞰,万狼奔腾,如潮水一般,绵绵不休,场面极为壮观。

临近那地魁万兽王时,狼群忽然分出几股,一股股如蚂蚁攀象。同时,又有天青狼群呼啸而下,盘旋地魁万兽王左右,如鸟群绕树。

地魁万兽王纵身一跳,跳入狼群当中,左右横扫,狼血飚飞,狼尸遍地。大脚四踏,留下个个狰狞血坑。

方源面含微笑,端坐云霄。

狼群在他的指挥之下,进退有据,缓急分明,一波波冲杀上去,形成众蚁噬象的战局。

地魁万兽王咆哮连连,左冲右突,对狼群造成大量杀伤,的确悍勇。

而脑海中,墨瑶轻咦一声,对方源的奴道造诣微微吃惊。

“啧,想不到这小子才情丰富,年纪轻轻,但驾驭万狼宛若呼吸,指挥若定,如臂使指,以弱战强,消耗战术深得奴道三昧。这是大师级的造诣啊……”

不过也只是微微吃惊罢了。

墨瑶见多识广,不是平常的蛊仙,乃是灵缘斋的某代仙子,眼界很高,记忆中像方源这般年轻的大师,不在少数。

战斗持续片刻,地魁万兽王嘶吼连连,被群狼消耗得体乏气弱,凶威不再。

大量的地魁兽,从地底一个个冒出头来,参入战团。

方源淡淡微笑,早有预料,指挥若定。

这场大战,从一开始就已经注定了结局。一支万兽规模的地魁兽群,怎么可能是他的对手?

偌大的战场,辐射方圆千里,但被方源巧妙地割成数十块,用孱弱的狼群围困,再用矫健的异兽狼形成锋矢,往来冲突。

先是一块战团被狼群摧垮,吞并,其后第二块、第三块,局部优势累积起来,胜利的天平越来越向方源倾斜。最终狼群吞并的速度越来越快,直至将地魁万兽王绞杀当场。

“这小家伙奴道修为的确不俗。指挥起来虽然细腻兼并豪迈,锋锐夹杂柔和,但距离宗师级还有很大差距。”墨瑶心道。

蛊师养用炼三方面。不管哪一面,都是博大精深。

明明用的是一样的蛊虫,但偏偏某些蛊师用得就十分出色,甚至上升到艺术的层次。人们就把这样的人,称之为大师!

大师可遇不可求,单靠资源的堆砌,是培养不出来的。不仅需要天赋。还需要才情。

但大师之上,还有宗师。

大师和宗师相比,就如小草对比大树。抛除天赋、才情、资源之外。还得有机缘和悟性。

但凡达到宗师者,触类旁通任何流派,晓阴阳乾坤,知宇宙奥妙。超凡脱俗。乃是仙中之仙,贤上之贤。

墨瑶便是炼道的宗师,如今虽然陨落,但眼界还在。

她生前见过的大师无数,觉得方源造诣不俗,还是主要看在他年纪轻轻的份上。

不多时,地魁万兽王轰然倒地,伤重身亡。

鲜红的血液。汩汩流淌,很快在它的身边。形成一滩河塘般的血坑。

方源飞下身来,亲自抽经扒皮,处理一番后,就着血坑,群狼环伺,立即炼蛊。

墨瑶改良后的地魁尸蛊,需要最新鲜的血肉,又以地魁万兽王为佳,千兽王次之,百兽王再次。

因此,方源才大举出动,亲自出手,斩杀地魁万兽王。

炼蛊持续了三天三夜,大功告成。

方源达到目的后,便遣散了大部分狼群,放养野外。只带了异兽精锐,马不停蹄,回归圣宫。

战场陷入死寂,从小山般的血森狼尸中,忽然钻出一个血人来。

血人踉跄而行,东倒西歪,走了几步后终于不支,摊倒在地上。

他大口大口地喘着粗气,目光中尽是难以置信,口中喃喃:“我居然还活着?”

他伸出手来,狠狠地抹了一把脸,终于露出真容。

不是旁人,正是马鸿运。

原来他被蒋冻敲昏之后,摔在地上,不省人事。

那头地魁老兽王没有顾他,对蒋冻紧追不舍。

但随后,狼群席卷,屠戮万物,马鸿运便惨遭狼口。

若是寻常野狼,诸如龟背狼、水狼、风狼,马鸿运早已经被四分五裂,撕扯成无数肉块碎片,存于狼腹当中了。

侥幸的是,吃他的却是体型如小山般的血森狼。

这狼张开巨口,舌头一舔,直接将百步之内的草皮都舔掉。马鸿运混同地魁老兽王的尸体,成了血森狼的腹中之物。

若是顺其发展,马鸿运迟早会被血森狼消化,最后化为一堆狼粪。

但之后和地魁兽群开战,这头血森狼被围攻,战死沙场,开膛破肚,反漏了空气进去。

马鸿运昏昏沉沉地醒来,忙不迭地钻了出来。此时大战已经结束,战场上兽尸遍布。偶尔有伤重未死的野兽哀嚎喘息,更显得周遭一片死寂沉沉。

马鸿运喘息够了,体力渐渐恢复,浓郁的血腥气味充盈鼻口。

他心知不妙:“我得速速离开这里,过不了多久,许多野兽都会被血气吸引过来的。”

马鸿运生长于北原,这点求生的常识早已深入骨髓。

他连忙站起,看准圣宫方向,立即启程。

但刚走几步,他就顿住了。

他的视线,被一只蛊虫吸引。

这是地魁兽尸上的一只野蛊。

地魁兽死了,野蛊不是毁灭就是飞走,但这只野蛊恰巧被断骨扣住,飞不走。

“这应该是二转的……那什么蛊。”马鸿运虽然记不太清,但并不妨碍他知晓这只蛊虫的价值。

“好蛊虫,好蛊虫,我得了它,就算自己用不了,也可以卖不少元石呢。”

马鸿运怦然心动,连忙走过去,伸手一捉,轻易到手。

(未完待续)

\end{this_body}

