\newsection{方源苏醒}    %第三节:方源苏醒

\begin{this_body}

方源缓缓地睁开双眼。

眼前是一片迷糊的粉红色,待视野渐渐转为清晰之后,还原成薄纱的帐幔。

微风吹来,有风铃丁玲作响,粉色帐幔悠缓飘动,如梦似幻。

方源从床上坐起。

这床圆形,宽大无比,躺上四五十人不成问题。

床上红底金边的丝绸被褥铺着,温暖贴身。

方源环顾一周,发觉自己身处在一处宽敞的卧室里。

床边,是烧着的鼎炉,焚烧着香料。是以空气中,飘动着惹人情怀的暗香。

这卧室以金砖砌墙,银砖铺地,床边、拐角、桌椅、梳妆台镶嵌着大量的珍珠、玛瑙、水钻以及各色的宝石。

金碧辉煌,奢华典雅,逸散着前主人狐仙的金粉之气。

这是狐仙的荡魂行宫。

“倒是个温柔乡。”方源淡淡地评价一句,从床上下来。

他的身体不自觉地晃了一晃,脑袋中残留着眩晕。

方源也不奇怪,反而心知肚明——这是他在三叉山时,把自己压榨得太狠了。

白凝冰背叛、群雄逼压,还要算计地灵,先炼第二空窍蛊,后来又在光瀑天河中,炼成了定仙游蛊。整个过程中,更要承受着冒险豪赌的心理压力。对方源而言,不管是身体,还是精神,都到达了极限。

当他利用定仙游蛊,来到荡魂山巅之时,凤金煌、方正是强弩之末,他方源又何尝不是呢?

相对于他们,方源承受的心理压力更大。春秋蝉已经不能连续动用第二次,而他在十派蛊仙的面前抢走狐仙传承,等若是虎口拔牙。火中取栗,冒险至极!

方源第一个登上山顶,地灵就驱逐了其他竞争者。方源正式成为福地主宰之后,便立即命令地灵,关闭了整个福地。

交代了地灵几个关健要点之后,环境安定,方源便一放松,顿时就呼呼昏睡过去。

“也不知我这次睡了多久……”方源晃了晃脑袋,他现在仍旧感觉十分疲惫。从灵魂深处传来一种虚弱感。

同时,他双耳不断地嗡鸣,脑筋转动时明显有一种凝滞的感觉。思考问题,比平常要困难得多。

“不好,我这是伤了魂。”方源心中一沉。察觉到自己的状态并不妙。

其中的原因,主要还是炼制了仙蛊。

仙蛊岂是那么好练的?很多蛊仙炼制不慎,受到反噬,轻则伤,重则死。

方源能以凡人之躯,炼成仙蛊。主要原因还是秘法好,源自《人祖传》。其次材料好。重点采用了神游蛊。换个角度讲,等于是将神游蛊转换成了定仙游蛊。这就大大减轻了难度。

不像方源五百年前世,用了大量凡蛊,合炼成仙蛊春秋蝉。以凡成仙。这难度更要大上百倍。

“饶是如此,我的魂魄底蕴太浅,还是伤了魂啊。不过好在这里是荡魂山……”想到这里,方源神色一肃。轻声唤道,“地灵何在?”

嗖的一声。地灵狐仙乍现在他的面前。

“主人,你终于醒啦。”狐仙低着头,红着脸,看着自己的脚尖,声音糯糯的。

她是五六十左右的女童形象,粉嫩可爱,娇小玲珑,一身彩衣,身后一只毛茸茸的雪白狐尾,在左右晃动着,显露出她忐忑的心境。

“主人,你昏睡之后,我就自作主张,把你左臂的伤口治好了。本来想给你穿上衣裙,但是行宫中的衣裙,都不合主人的身材。”狐仙地灵汇报道。

她说的衣裙,乃是狐仙的衣裳,都是身材曼妙的女子穿着,能适合方源就怪了。

方源皱了皱眉头:“衣裳不过是细枝末节,我且问你,我昏睡了多长时间?这段时间里,可有强敌攻门?”

狐仙将一对亮闪闪的大眼睛,连连眨动:“主人,您这次昏睡了七天七夜。期间,并未有人攻门。”

“哦?”方源双眼顿时一闪。

他怎么也想不到,仙鹤门的鹤风扬会包庇他,替他挡下其他九大派的刁难。

但他却多少明白些,为什么蛊仙没有冒然进攻狐仙福地。

狐仙福地,不是三叉山上的三王福地。

这片福地还很年轻,拥有地灵,还有充足的仙元储备,更有荡魂山保护福地中枢。

这三者将狐仙福地,打造成坚固的堡垒,足以让大多数的蛊仙望而却步。

这片福地是多么难以攻打,方源深有体会得很!

五百年前世,他伙同近十位魔道蛊仙,一齐进攻这里。最终惨胜,只剩下他和宋钟。

宋钟乃是宋紫星的儿子,当时的魔道新星,现在还未出生。

“那时候,我已经是魔道的老前辈。宋钟这小子,继承了他老爹的遗产,一跃而上,和我交手数十回合,不分胜负,因此一战成名。”

回想前世,自己被宋钟这个后辈小生,踩着上位,方源就不禁冷笑连连。

“今生一切都变化了。待我寻机杀了宋紫星,哼哼,宋钟我看你怎么出生?”

宋紫星拥有一道血海真传,乃是上古荒兽戾血龙蝠。这龙蝠并非蛊虫,却是可以擒拿抢夺。

“拥有了戾血龙蝠,就意味着拥有绵绵不绝的血蝠大军。指挥血蝠,嘿嘿,可是我五百年前世,最拿手的手段之一。当然,这是以后的计划。现在还是借助这片福地,尽快的修行,争取早日重归蛊仙的境界!”

念及于此,方源不禁想起一个重要的问题:“地灵,距离接下来的地灾,还有多久?”

“主人您不问,我也想汇报您呢。目前,福地已经经受了五次地灾,如今距离第六次地灾,只剩下一年零三个月了。”狐仙的语气隐含着焦急和凝重。

“什么?只剩下一年零三个月!”方源顿时就坐不住,从床边上站起身来。脸色阴沉如水。

万物平衡,天道至公。有强就有弱,有福就有灾。福地有灾劫,每十年一次地灾,每百年一次天劫。

且不提天劫,单说地灾。地灾一发,威能浩瀚,往往天崩地裂。一旦福地抵御不住,就是毁灭的下场。

方源前世也拥有福地。对事情的严重程度,再清楚不过!

对于福地来讲,每一次地灾都是相当严峻的考验。地灾一次比一次强,狐仙就死于第五次地灾,而方源将面临着更为强大的第六次地灾。

“距离地灾。居然只剩下一年零三个月的时间了。怎么在《凤金煌传》中,没有提到?是了,凤金煌乃是灵缘斋弟子,父母双亲又是蛊仙,靠着他们的帮助,抵挡地灾并不困难。但是对我而言,情况却极为不妙了!”

第六次地灾来得太快。方源纵然有丰富的经验,但没有时间进行充分的准备。

除此之外,他还得防备着外在的强敌。

“我在众目睽睽之下,夺了狐仙福地。十派虽然没有举动,但一定在外面虎视眈眈。我有些明白了,想来这十派蛊仙按捺不发,也是瞅准了地灾将近。想要利用这次地灾罢?”

地灾来临,地灵自然要全力以赴。难以兼顾自己这个主人。方源目前还是四转高阶,很容易被针对。方源一死,福地就是无主,地灵将再选新主。

若是地灾造成较大的漏洞,外界的蛊师便可以自由出入这里。如果十派届时发难,必将雪上加霜,令情势更加险恶。

方源目光闪烁了好一阵子,因为伤了魂,想得脑门都隐隐发疼。

他停下纷乱的思绪,吐出一口浊气,最终决定:还是先看看具体的情况。尽最大努力做抵御地灾的准备。若是失败了,他只好舍弃狐仙福地,令其自毁,不留给正道任何资源,然后动用定仙游蛊撤退。

福地虽好,但怎能及得上自己的安危?

这样想定,方源便嘱咐地灵带他出去看看,他现在急需全面地了解这片福地。

“是。”地灵低着头乖巧地回答,又略显迟疑地加了一句,“主人,您不喜欢穿衣服的吗?其实穿上一身好看的衣服,会显得人很精神的,自己也会莫名的开心哒。”

方源:“……”

荡魂行宫中的衣裙,方源是穿不了的。不过好在他的兜率花中,储备着备用的衣衫。

换了一套黑色长袍后,方源跟随着地灵,踩着阶梯一路向上,来到山巅。

荡魂山上的风,刮得很大。

但地灵轻轻一挥手,顿时就变成怡人的微风。

“主人,这片福地的宇有六百万亩。宙有五倍流速。六百万亩地,皆是草原地形,青草以蓝度草、马蹄草、六神草为主,伴有七宝花,奶茶花等等。”

地灵一边介绍,一边就划出影像,悬浮在半空中,令方源观看。

影像中,正是一片典型的草地,色彩缤纷,宛若花毯,仿佛近在咫尺。

蓝汪汪的剧毒蓝度草,马蹄形状的马蹄草,六片细长的叶子,如玉般细腻光泽的六神草。还有七彩缤纷的七宝小花,杯子形状,盛着奶茶般花汁的奶茶花。

以这七种为主,但除此之外,还有许多其他的杂草、野花。

方源看了,连连点头。

不要小看这些花草,这都是修行资源。

这七种主要的花草,都能作为炼蛊的材料。一些花草上,寄生着蛊虫。六百万亩的草原,相当于地球上,四个香港的面积总和。这里面会有多少蛊虫?

这些野生蛊虫,极容易捕捉。只要方源一声令下,地灵就能将其完整无缺地拘拿过来,贡献到方源的手中!

(ps:关于第二卷184节的三王福地面积,已经做了修改,改成了九百万亩。也就是地球上文莱一国的面积。第二卷108节的定仙游方面,也做了修改。定仙游并非一次性的消耗蛊,而是可以重复利用。修改的部分,可来起点网观看。请读者朋友们不用担心以上bug。)

\end{this_body}

