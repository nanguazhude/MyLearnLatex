\newsection{方源你这个小王八蛋!}    %第十八节:方源你这个小王八蛋!

\begin{this_body}

%1
“不卖胆石,卖石人?”鹤风扬望着手中的信笺,阴沉着脸。

%2
狐仙福地中,荡魂山胆识蛊,才是仙鹤门最需要的东西。一旦有了胆识蛊激增魂魄底蕴,整个仙鹤门弟子的实力,都要再强上三分。

%3
而且胆石不能离开荡魂山体,只能就地开采。这样一来,借着这个由头,仙鹤门弟子就能出入狐仙福地。一来二去,方源的警惕渐渐下降后,仙鹤门再暗中动手脚就方便多了。

%4
但是方源却死活不出卖胆石,鹤风扬为之郁闷:“什么时候,我堂堂的蛊仙,居然受着一个凡人小贼的刁难!?”

%5
他咬牙切齿的想着,俊秀的少年面孔,此时显得有些狰狞扭曲。

%6
他空有一身战力,却无法施展。方源龟缩在狐仙福地当中,就像是缩头乌龟,又有定仙游蛊,随时可以脱身。鹤风扬,甚至整个仙鹤门都为此投鼠忌器,暂时都不敢动他。

%7
“看来狐仙当年迁徙进去的石人还存活着。不过,方源这小贼子一下子交易出这么多的石人,他究竟用了多少胆识蛊培养啊!”

%8
联想到这里,鹤风扬就感到心在滴血。

%9
距离上一次交易,已经过去数月了。狐仙福地中的时间,还要再延长五倍。也就有一年左右了。

%10
方源培养出这么多年轻力壮的石人,如果将消耗的这些胆识蛊交给仙鹤门,用来培养弟子,该多好啊。

%11
但是这方源贼子,宁愿培养石人,也不愿给仙鹤门人使用。其心可诛,可诛啊!

%12
令鹤风扬气愤的,还不仅仅是这个。更关键的。他还是在生自己的气——方源卖出这么多的石人,就算他是蛊仙,也不由为之心动。

%13
福地当中,若有充足的石人,蛊仙可以开发地下,开采大量的地底资源。

%14
各种金属、宝石、矿石,以及蛊虫,地底生物等等,源源不断。

%15
除此之外。石人若再多一点,就能建立地下城池,这就变相地为福地拓展了空间。

%16
福地中没有实力,蛊仙开采出来的资源,大多只是来自地表。这只是一个平面。但若是多了石人之后,连地底都能利用到,利益绝对要翻倍增长。

%17
而市面上,石人奴隶供不应求。

%18
石人一生大多时间,都在睡眠。一位普通的石人千岁而亡,一生当中,只能繁衍出四位子孙。

%19
若是用魂道蛊虫给石人增长魂魄。也不是不可以,也有大量的蛊仙尝试过,但从未推广成功。

%20
原因无他,只有一个。那就是成本。

%21
魂道蛊虫的价值,要比单个的石人昂贵多了。

%22
除服个别蛊仙对石人有特殊的需求,否则都是得不偿失的。

%23
这个世界上,也唯有掌握了荡魂山的方源。能这么大肆培养石人了。

%24
而且此次交易,方源在信中给出的价格。也颇让鹤风扬心动。就算仙鹤门自己不用,转卖出去,也会很有赚头。

%25
但交易就算有着便宜,鹤风扬仍旧心气不畅。

%26
他知道这是方源抛出来的饵。

%27
不怕你不心动,不怕你不吃!

%28
而正如方源所料,鹤风扬心动了,仙鹤门也会心动,其他的蛊仙也必然心动不已。石人奴隶买卖,将至少畅销一百年!

%29
“方源这可恶的小贼,真是狡诈。不过有了这批石人,倒是让太上大长老、二长老、三长老们看到了成果。也能让雷坦这家伙闭上臭嘴。而我也能松一口气了。”鹤风扬狠狠地喘了几口气,将心境平复下来。

%30
他双眼眯起,嘴角渐渐溢出丝丝冷笑:“不过方源你也别得意,你做初一我做十五,你不卖胆识蛊,那我也不卖舍利蛊。你不是想要黄金舍利蛊、紫晶舍利蛊么?没门!”

%31
……

%32
方源躺在躺椅上,看着仙鹤门的回信。

%33
因为魅蓝电影还把守在天梯山上,小狐仙不敢乱开福地门户,所以仙鹤门这次用了飞剑传书蛊。

%34
方源扫视了一眼,将内容尽收眼底。

%35
鹤风扬在信中,除了答应交易之外,还这次明确地提出:要交易胆识蛊。为了胆识蛊,他可以售卖舍利蛊。甚至可以亲自动手,帮助自己度过第七次地灾。

%36
方源冷笑连连。

%37
让鹤风扬进入狐仙福地,这危害比地灾还要大得多,绝对是不可能的。

%38
至于卡住舍利蛊,就能卡住方源的命脉?可笑的打算。

%39
“胆识蛊我是绝对不会交易的,但是石人可以出售,同时也不怕仙鹤门不心动。只是今后不能总和仙鹤门一个门派做交易,还得往外扩展。”方源思索着。

%40
有着前世五百经验,方源知道自己卖出的这价格,比蛊仙市场上要稍低一筹。

%41
仙鹤门做了这桩买卖,必定很有盈利。

%42
不过这也是方源故意安排的。

%43
目前而言,他需要仙鹤门弟子这个身份,也需要稳定和维护住这层虚伪薄弱的关系。

%44
“这份利益到手,也算是缓和了鹤风扬身上的压力。他想徐徐图谋我的福地,而我也正是需要这个‘徐徐’二字啊。等到我日后成了蛊仙,还怕仙鹤门的脸色?”方源淡淡一笑。

%45
他再看信笺,信中末尾,鹤风扬约定了时间,要求方源打开门户,将洞地蛊传递进来。

%46
这次买卖的规模很大,方源要卖出六万石人。但这些石人,不可能从福地门户走出去。

%47
目前,那道魅蓝电影,还在天梯山上徘徊。如果敞开门户,让它再冲进来,方源就有天大的麻烦了。

%48
这种情况下,就要用到洞地蛊了。

%49
此蛊高达五转,分有两只,一只母蛊,一只子蛊。

%50
作用就是架设在两地当中,形成宇道通路。蛊师从子蛊这端进去,便能从母蛊那边出来。同样的,从母蛊那边,也能顷刻到达子蛊这里。

%51
洞地蛊常常被用来连接两片不同的福地。平时的时候,洞地蛊用来输送各种资源。战时,援兵能够通过洞地蛊,展开快速的支援。

%52
“要进行交易,运走这些石人,使用洞地蛊是难免的。可是我却担心,你的洞地蛊会做手脚呢。还是我自己炼制的保险些罢。”

%53
方源念及于此,春秋蝉的气息泄露出一丝,随手就将手中的飞剑传书蛊炼化,迅速地再回一信。

%54
半天之后,鹤风扬接过回信,展开一看,只见上面名列了大量的蛊虫,还有各种材料。

%55
“哦?不想用我给的洞地蛊,而是改良了洞地蛊的秘方,想自己炼制出新的蛊虫?”鹤风扬面现怒色。

%56
“放屁!洞地蛊已经被广为运用了五百多年,秘方早就为蛊仙所知,偏偏到你那里就能改良了?方源这小贼,戒心太强,还要借此讹我一笔,叫我不敢在拒绝他的交易。我不用看都知道,这些炼蛊所需中,必有舍利蛊。或者泉蛋蛊,或者力道上的珍稀蛊虫。嗯?没有?”

%57
鹤风扬看了几眼,发现并没有舍利蛊、泉蛋蛊,同时也没有力道蛊虫的影子。

%58
倒是有大量的偏门材料,还有低转的蛊虫。最高的五转蛊虫,是土道蛊虫为山九仞蛊,专门用来增加炼蛊的成功可能。除此之外,又要求不少的四转蛊,有常见的移步换形蛊,有极其实用的旁推侧引蛊。

%59
鹤风扬的想法产生了动摇:“看这架势,好像是真的要炼蛊。只要运气正常,这些材料,已经可以炼出三只洞地蛊了。难道说,他真的改良了洞地蛊的秘方?不,方源只是区区凡人怎么可能。但万一是他背后的蛊仙呢?”

%60
如果是改良了秘方,那么新蛊必定比洞地蛊还要强大一筹。

%61
鹤风扬不禁怦然心动。

%62
他就算搞不到秘方,方源炼成之后,必定会将子蛊交给他。到那时,他堂堂蛊仙,也能从子蛊上逆推出许多东西,甚至很大可能还原出秘方来。

%63
两天之后,方源接到来信。

%64
方源展开看了看,果然如他所料,信中鹤风扬又找各种借口,故意削减了许多材料和蛊虫,目的不言而喻,就是为了试探方源真正的秘方。

%65
方源不禁摇头失笑,这鹤风扬就是心思太细腻了,这是他的优点,也是他的缺点。

%66
虽然打交道的次数少的可怜,但方源已经将鹤风扬看透了大半。

%67
他当即义正言辞地回信,要求鹤风扬不得削减任何东西,否则就炼不出蛊虫来。

%68
但鹤风扬虚以委蛇,继续扯皮,借口说得他都快要相信了。

%69
信来信往,几次之后,方源“无奈”选择了妥协,又列了另一份名目。

%70
“小子,你和我斗还嫩着呢。”鹤风扬得此信,这才交出名目上的一部分东西。然后,又接着找借口拖延。

%71
一来二去,足足磨了八九天的时间。

%72
当方源表现出极度的无奈和愤怒时,鹤风扬这才收手,觉得火候到了,拿捏着再三削减后的第六份名目,回去埋头钻研去了。

%73
但他炼道造诣不佳,苦心研究,又加试炼了多次,都见不到成效。

%74
中洲时间半月之后,他收到方源传递出来的洞地蛊子蛊。

%75
他如获珍宝,又继续耐心地埋头研究。

%76
这次,他研究了足足三天,终于成功地反推出秘方。

%77
看到这个秘方,鹤风扬狂怒暴吼:“方源,你这个小王八蛋!这明明就是寻常的洞地蛊秘方!”

%78
一时间,他痛声大骂方源卑鄙无耻狡诈下贱,更推而广之,涉及到方源十八代祖宗。

\end{this_body}

