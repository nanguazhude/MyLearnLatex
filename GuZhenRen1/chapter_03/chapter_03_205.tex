\newsection{胜负关键}    %第二百零五节:胜负关键

\begin{this_body}



%1
蛊师升仙,空窍便晋升成仙窍。

%2
仙窍比之空窍,是质的升华,等若开辟一个全新的小天地。

%3
六转、七转蛊仙的仙窍,常被人谓之“福地”。八转、九转蛊仙的仙窍,则被称之为“洞天”。

%4
蛊仙死后,若留下执念,执念便结合仙窍中的天地之力,形成地灵。

%5
每一个仙窍都是独一无二的,本就是蛊仙身体的一部分。从某种意义上来讲,蛊仙尽管陨落,但留下的地灵、仙窍,亦是他(她)生命的另类延续。

%6
正因为执念形成的地灵,所以地灵都认死理,不知变通。

%7
历史上,也有许多例子。譬如某某蛊仙强势,图谋某块福地,结果达不到认主要求,地灵不允,最终地灵直接自毁,令蛊仙竹篮打水一场空。

%8
巨阳意志,要强行封印王庭地灵霜玉孔雀,早就令后者继续了十多万年的怒火。现在还要继续封印它,这当然就引爆了霜玉孔雀的绝然之心。

%9
“这到底是怎么回事?”

%10
“太白云生大人升仙,结果令福地都破开漏洞了吗?”

%11
“王庭福地正在显露于世间,难以置信啊,这可是数万年来都没有的事情!!”

%12
“大事不妙了,我有一种极不好的预感。你们说,该不会福地要毁灭了吧?”

%13
福地中,众人惊呼不已,恐慌、疑惑的情绪迅速弥漫。

%14
苍穹中,裂开无数的伤口。原本金银交替的福地天壁,此时伤口累累,从这些伤口中,人们可以看到北原夜空中的星光。

%15
方源的目光,却仍旧一片沉静。

%16
“王庭福地,是不可能这样轻易自毁的。”他在心中低语,对于这一点他十分笃定。

%17
历史上,地灵自杀、福地自毁的例子多了,但绝非眼前局势。

%18
巨阳仙尊的布置,岂是那样容易推翻的?这可是九转蛊仙的大手笔,直接覆盖了北原一域!

%19
八十八角真阳楼就是最大的障碍。

%20
“换做以前,地灵怎么可能翻起这样的浪花?只是现在,巨阳意志刚刚从沉睡中醒来,被打了一个措手不及,暂时抽不出手来收拾局面。而封印地灵的力量已经在我的和稀泥仙蛊的影响下,消散了大半,这才让地灵有了余力挣扎。”

%21
分析到这里,方源不禁暗赞巨阳意志的果决和手段。

%22
他巧妙地借助了太白云生这个棋子,用仅能动用的手段,争取到了最宝贵的时间,从而让胜利的天平大大地倾斜过去。

%23
“然而地灵虽然无法真正做到自毁,但抽取自己的天地之气,这最基本的能力还是能做到的。”

%24
念及于此,方源目光远投,看向太白云生。

%25
在太白云生的头顶上空,原本已然消散的劫云,又重新汇聚起来。地面上的灾尘,同样再度弥漫。

%26
大量的天气、地气,喷涌而出,眼看就要形成全新的天劫、地灾!

%27
“这,这究竟是怎么回事?怎么还会有天劫、地灾?!不是已经度过了吗?”看到这一幕,耶律桑不禁瞪大了双眼。

%28
“这和记载中的完全不符。太白云生大有古怪,怎么会有连续两次的天劫地灾呢?”黑楼兰也万分不解。

%29
他们都不知道,其他人就更加不清楚了。

%30
看到气势浩荡的天劫、地灾又再度形成,许多人失声惊呼,对太白云生的状况表示极度的担忧。

%31
太白云生的面色,一片铁青。

%32
“怎么还有天劫形成?!我的传承中,可不是这样讲的!”他心中惊怒交加。

%33
因为不仅是外界,与此同时,在他的体内仙窍中,也正酝酿着另一场天劫地灾!

%34
凡蛊升仙,如同蛊师成仙,都是逆天质变的危险过程。蛊虫炼成仙蛊,自然也要勾动天地二气,如此一来,也便形成相应的天劫、地灾。

%35
内忧外患一齐夹攻,太白云生进退失据,唯恐顾此失彼,极是为难,情况十分堪忧!

%36
方源动用和稀泥仙蛊,唤醒地灵。

%37
太白云生的升仙,就成了地灵和巨阳意志相互较量的关键棋子。

%38
这一场升仙,早已经脱离了当事人的掌控。

%39
除了方源之外,其他人都被蒙在鼓里,糊里糊涂,不知道真相。

%40
唯有方源知道地灵的打算!

%41
地灵争时间,争不过巨阳意志,也把主意打到太白云生的身上。

%42
天气、地气,都是地灵主动牺牲自己的根本,提取出来的。目的就是要形成天劫、地灾,摧毁太白云生。

%43
消灭他对地灵有什么好处呢?

%44
这就涉及到仙窍的吞并了。

%45
凡窍不好相互吞并,但升仙之后,仙窍就是一个个的小天地。天地之间,则可以相互吞并,对蛊仙更大有好处。

%46
方源之前在天梯山,主动舍弃狐仙福地的四分之一领土,很快就被其他蛊仙夺去了。吞并其他仙窍,壮大自家仙窍,这里面好处多多。

%47
地灵若消灭了太白云生,后者一死,仙窍却会留下。仙窍留下来后,自然就要被王庭福地吞并。

%48
这吞并下去之后,福地底蕴大涨,地灵实力也随之暴涨。这就对它突破巨阳封锁,有着巨大的帮助。

%49
天空中,黑云滚滚,电闪雷鸣。

%50
无数紫色闪电,丝丝缕缕,汇成雷电之球。

%51
地面上,漫天的灾尘则沉淀下去,露出裸露的土地。

%52
土地上渐渐撕开一道裂口,长达数百丈,露出里面的鲜红的火焰,宛若洪荒巨兽张开血盆大口一般。

%53
火焰升腾而出,凝成道道狼烟,喷涌而上。

%54
“颠乱雷球,羁绊狼烟!”方源微微皱起眉头。

%55
这颠乱雷球不仅威力巨大,而且有摧毁蛊师心念意志,令其混乱的效果。狼烟封锁天地,蛊师只要陷入其中,便丧失五感,如坠迷宫深渊,不得挣脱。

%56
颠乱雷球速度不快,但配合羁绊狼烟,却是恰到好处,能将威力放到最大!

%57
方源不禁生出微微担忧之情。

%58
太白云生的失败,并不是方源他想看到的。

%59
捏着琉璃楼主令的手,微微发紧,方源目光深沉:“还是太弱了,太弱了!我现在的实力,只能因势利导……”

%60
呜呜呜!

%61
颠乱雷球,一颗颗破空激射,发出怪异的尖啸之音。好像是女子夹杂着哭音的尖叫声,十分刺耳。

%62
与此同时,狼烟滚滚,向着太白云生攀来。

%63
太白云生脸色发白。

%64
颠乱雷球方向时刻改变,忽左忽右,本来就不好躲避。此刻他的身边,一道道狼烟竖起,仿若置身在古木森林当中。

%65
“一旦撞进狼烟当中,必定凶多吉少!糟糕,仙窍中的天劫地灾,也在酝酿而出,而我的防御蛊已经损失殆尽!”太白云生咬牙,目光中绽放出一股狠意。

%66
拼了!

%67
他浑身猛地绽放出璀璨华光,数只移动蛊虫一齐催起,带动他的身躯飞射而上。

%68
狂烈的风,将这位老人的发须甩在脑后,恣意飘扬。

%69
人群中猛地爆发出一阵惊呼声。

%70
太白云生竟然直接冲向劫云,在他的头顶上,是密密麻麻,数不胜数的颠乱雷球,远远望去,宛若飞蛾扑火。

%71
“难道太白云生想寻短见了?”耶律桑紧皱眉头。

%72
“好!”黑楼兰却赞叹一声。

%73
方源看得眼前一亮。

%74
狼烟看似无害,没有任何攻击能力,只有困敌之效,但威胁反而比颠乱雷球更大。

%75
太白云生认清形势,做出了最佳的决断。

%76
只见他冲入雷云,在无数雷球的互相激射下,宛若游鱼一般穿梭自如。

%77
轰轰轰……

%78
雷球不断爆炸,但却影响不到太白云生的身形。

%79
他左右穿梭,在夹缝中寻觅生机,移动速度忽快忽慢,每每恰到好处!

%80
“好厉害!这是何等的飞行水准啊!”

%81
“想不到太白云生大人,居然是一位飞行大师!!”

%82
“他隐藏得太深了,我到现在才知道……”

%83
众人惊喜地发现,太白云生居然是一位飞行大师!

%84
“想不到太白云生,还擅长云道蛊虫。”耶律桑交口称赞,并不是很吃惊。

%85
太白云生是治疗蛊师,这些年行走北原,不知道遇到了多少危险。能存活到今天,必然有一把刷子。

%86
事实上,大多数的治疗蛊师,不是擅长防御,就是擅长移动。

%87
蛊师在被培养的时候,就特意这样训练。

%88
皆因战场上,治疗蛊师往往是第一打击对象,很多时候得不到队友的照顾。这就要求治疗蛊师本身,必须有一定的保护自己的能力。

%89
“太白云生作为北原第一治疗大师,不擅长防御,年纪又这么大,有如此的飞行造诣,并不突兀。”黑楼兰想到这里,不禁微微转头,看向远方天空。

%90
在那里,方源端坐在天青狼王背上,似乎也在关注着太白云生。

%91
和太白云生相比起来,同为飞行大师的常山阴就年轻多了。

%92
“小麻雀,你倒是蹦跶得挺欢畅的。”真阳楼中,巨阳意志怒极反笑。

%93
八十八角真阳楼剧烈震动起来,霞光满天摇曳,一道巨大光柱,直射而出。

%94
仙蛊——排难!

%95
排难之光,照在太白云生周围,消弭雷球,洞穿狼烟。

%96
“先祖真是眷顾太白云生,又再次出手了!”惊叫声迭起,羡慕嫉妒皆有之。

%97
“这情形太古怪了,太白云生并非巨阳血脉,居然能得到先祖的这般青睐!”有人则开始怀疑。

%98
“果然出手了。”方源暗叹,太白云生已经成为影响地灵和巨阳意志胜负的关键因素,地灵要害他,巨阳意志则要保他!

\end{this_body}

