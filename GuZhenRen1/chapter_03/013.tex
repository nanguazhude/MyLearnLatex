\newsection{有失有得}    %第十三节:有失有得

\begin{this_body}

%1
方源从荡魂行宫走出来。

%2
粉红色的水晶山,坑洞满布,碎石遍地,狼藉不堪。赤红的鲜血,黄褐的泥浆,狐群和螃蟹的尸体堆砌在一起。

%3
荒兽泥沼蟹巨大的身躯,压在荡魂山的山腰上。仿佛是凸起的一座小山头,异常的显眼。

%4
微风吹拂着方源的黑色长发,他那幽深的眸子,缓缓扫视战场。

%5
刺鼻的血腥气味,刺激着方源的鼻腔。

%6
紧跟在他身后的,是地灵小狐仙,眼眶通红,粉嫩的小脸蛋上犹有泪痕,仍旧一噎一噎的抽泣着。

%7
“主人,我们损失太惨重了。丧失了一百多万亩地,仙元只剩下七十一颗。四百七十万的狐群,只剩下三十一万。还有蛊虫,一下子损失了七十多万。”

%8
狐仙辛苦经营了这么多年,经此一役,整个福地的发展倒退四十年。

%9
但方源却不这么想。

%10
他面色疲惫,目光中却难掩喜悦之情。

%11
这场地灾,终究是叫他抵挡了过去。这是一个关口,如今过了这个关口,方源就有了喘息的时间。

%12
面对第七次地灾,他将有更充足的准备时间。

%13
“不要哭了,福地保住了,就是保住了希望。荡魂山也没有塌毁,更令我们有起步之资。虽然损失了一百万亩的广袤土地,但是短时间内,我们也用不到这些,并不妨碍我们的发展。”

%14
方源拍拍地灵的小脑袋,继续安慰道:“你看。过不了多久,荡魂山上就会长满了胆石,我们又控制了石人一族。最关键的是。那道魅蓝电影被驱逐走了。接下来,我们就可以毫无顾忌地经营。狐仙福地,将会再次繁盛起来!”

%15
这一场惨烈至极的激战,荡魂山附近死了无数的狐狸和螃蟹,同时泥沼蟹的魂魄,也被震荡成精粹,落入荡魂山上。

%16
可以预见。今后不久,大量的胆石必将在荡魂山上层出不穷。

%17
“主人说的话……也有道理。”小狐仙渐渐停止了抽泣,仔细想想。还真是这么一个理儿。

%18
地灾就像是一次洗礼。狐仙福地撑了过去,并非只有惨重的损失,事实上还有大量的收获。

%19
“真是可惜,泥沼蟹身上寄生着大量的蛊虫。如今也都被荡魂山杀死了。”小狐仙嘟起嘴。恨恨地看着泥沼蟹巨无霸似的甲壳。

%20
“庆幸吧。这头荒兽身上没有仙蛊,否则我们就未必能站在这里了。”方源长叹一口气。

%21
这是他此次幸运的地方。

%22
一只关键的仙蛊威能强盛,足以轻易的颠覆整个战局。

%23
即便这头泥沼蟹身上寄生了仙蛊,也被方源杀死了,那么方源如何能捕捉到仙蛊,还是个巨大的问题。

%24
说不得这只野生的仙蛊,会顶替魅蓝电影,成为狐仙福地的巨大内患呢。

%25
能够撑过此次地灾。方源已经心满意足了。

%26
毕竟他只是个四转高阶的凡人蛊师,却杀了连蛊仙都头疼的荒兽。

%27
“地灵。把战场收拾一下。泥沼蟹尸可要保存好,我下去休息了。”方源巡视了一番后,放松下来,便感到了强烈的疲惫感。

%28
他指挥数百万的狐群,魂魄、精神早已透支,急需要睡眠。

%29
“好的。”小狐仙清脆地答应一声,她看向泥沼蟹庞大的甲壳,双眼放出晶晶亮的光。

%30
每一头荒兽,都是一个移动的宝库。

%31
它身上的血液、毛皮、骨骼、内脏等等,都是炼蛊的好材料。

%32
“你这头该死的大螃蟹,人家要把你全部扒光苗疆蛊事!”小狐仙皱起琼鼻,嘴角咧开,露出小虎牙,一边走向泥沼蟹,一边咬牙切齿地自言自语。

%33
……

%34
这一觉,方源睡得十分舒爽。

%35
三天之后,他苏醒过来,躺在床上,一时间都不想动弹。

%36
他这次是真正的放松下来了。

%37
第六次地灾刚刚过去,方源争取了不少的时间,终于可以松一口气了。

%38
自第一次重生以来,他都是在辗转流离,挣扎抵抗。尤其是在三叉山时,更是殚精竭虑,拼尽了浑身解数。

%39
现在,终于保住了福地。对于方源来讲,就有一个稳定的后方。

%40
如果狐仙福地毁灭,那他就麻烦了。

%41
他暴露了很多东西,别的不说,就说定仙游蛊吧。

%42
方源还不是蛊仙,这只仙蛊无法收入空窍,气息就泄露出来,很容易就被蛊仙察觉,从而出手抢夺。

%43
现在,定仙游蛊放到了狐仙福地当中,又有仙元,总算是能养住。

%44
除此之外,还有第二空窍蛊的炼制。

%45
没有小狐仙调动仙元,方源就不能炼制这只仙蛊。至少得等到他成为蛊仙。到那时,神游蛊恐怕就为他人所有了。一切都晚了。

%46
保住了狐仙福地,对于方源来说,帮助实在太大了。

%47
方源躺了一会儿,便起床简单吃喝。填饱了肚子,他又再次睡去。

%48
这次睡了五个时辰,他悠悠醒转,感觉疲乏一扫而空,神清气爽,心境明彻,状态非常之好。

%49
“地灵何在?”他跺跺脚,呼唤一声。

%50
小狐仙嗖的一声,破空挪移,出现在他的面前。

%51
“主人,我把那只大螃蟹都拆开了,咱们一定能卖个好价钱!”小狐仙满脸通红,亲手拆解了罪魁祸首,让她既解恨又兴奋。

%52
“对了,主人,这里有三封信。是地灾那天,有人从漏洞里塞进来的。”小狐仙说着,取出三只蛊,交到方源手中。

%53
当漏洞大得成为通道之后,才能供人进出。在此之前,一些蛊虫可以顺着漏洞塞进来。

%54
这三只蛊。都是信道蛊虫。

%55
一只纸鹤模样,乃是三转电文纸鹤蛊。一只青鸟,栩栩如生。高达五转,是为传信青鸟蛊。一只小剑模样,三转的飞剑传书蛊。

%56
方源目光闪了闪,先取了传信青鸟蛊。

%57
青鸟一阵变化,化为一张信笺,竟是七转蛊仙凤九歌的来信!

%58
不过方源倒并未吃惊,他看到这只传信青鸟蛊的第一眼。就有这个预感了。

%59
凤九歌的来信中,语气缓和,先是表达了对方源的赞赏及恭喜。然后才说明他来信的用意——替他的女儿凤金煌约战!

%60
原来凤金煌回到灵缘斋后,一直郁闷至极,难以抒发。这段时间,她不断地勤修苦练。就是要找回场子。约战是她主动提出来的。要和方源正大光明的来一场较量,赌上灵缘斋和仙鹤门的荣耀!

%61
“哼,你想战,我就要战吗?”方源嗤笑一声,十分不屑。

%62
自己时间如此紧张,用来修行都不够,方源当然不会闲的没事干,接受凤金煌的这场挑战。

%63
凤金煌是天之骄女。有蛊仙双亲,有门派支持。就算她有福地。来了地灾,也会有一大群人主动帮她扛下来。

%64
但方源只是孤家寡人,什么事情都得靠自己打拼。根本就没有心情,陪同这种大小姐玩耍。

%65
“她的念头不通达,她想找回场子,我就一定奉陪么?可笑!”方源冷笑几声。和别人不同,他并不忌惮凤九歌。在不久的将来,凤九歌就会被天庭征召,成功飞升,想要下界再入中洲就不是那么容易的了。

%66
“不过,这封信中,这对父女似乎将我当做仙鹤门的弟子,这是怎么回事?”方源的眼中,闪过一丝疑惑。

%67
对方是蛊仙,自然不会犯这种低级错误。这样说来,其中必有隐情。

%68
方源又取来飞剑传书蛊。

%69
点开一看,又是一篇挑战书!

%70
但这信中,通篇都是咒骂方源的脏话,问候到方源八辈子祖宗,更用仙鹤门的门派荣誉挤兑方源,激将方源应战。最后更是威胁,如果方源不答应,他就将这封信的内容公之于众,使得全天下都知道方源是个胆小鬼、孬种!

%71
方源仔细地看完,淡淡一笑:“原来是剑一生这厮。”

%72
剑一生,乃是金道蛊师,长相和个性都十分猥琐,最擅长的就是偷袭,最讨厌的就是吃亏。

%73
他是天梯山上的魔道蛊仙之一,十足的真小人。

%74
方源前世五百年,被他偷袭过数次。终于惹得方源怒火勃发,掀起滔天血海,将剑一生堵在福地中,不敢出来应战。这一堵就是二十年,剑一生终于明白自己惹了不该惹的人。最后,他实在受不了,直接向方源跪地投降,毫无蛊仙的风范。

%75
至于这厮为什么要挑战方源,也算他倒霉透顶。

%76
方源割弃福地之后,将魅蓝电影赶到天梯山上,好巧不巧,被剑一生撞个正着。

%77
现在这时候,剑一生虽然已经是蛊仙,但却没有仙蛊在手,被魅蓝电影狠狠地痛扁了一顿。他狼狈逃窜到自家福地中,这才摆脱了魅蓝电影。

%78
回到家,他算了算损失,顿时怒恨交加,气得跺脚。查明了此事的元凶之后,他就发了飞剑传书蛊,要约战方源。

%79
“哼,他明明知道我还是凡人之躯,却以一届蛊仙的身份约战我。偏偏还说得如此理直气壮,标榜公平……这厮果然还是和前世一样无耻。不过,他怎么也把我当做仙鹤门的人?”

%80
方源怀着加倍的疑惑,点开电文纸鹤蛊。

%81
他展开稍稍一看,顿时瞳孔一缩,脸上流露出明显的诧异之色:“怎么!方正他还活着?”

%82
再仔细浏览下去,他心中的疑惑终于消释一空。

%83
“原来如此。这个仙鹤门的鹤风扬,也是精明人物,居然想到这个方法,来排除竞争对手。”

%84
“不过,他大大低估了我。我怎么可能加入仙鹤门?倒是信中提出的交易,正是我需要的。”

%85
想到这里,方源定下心来。

%86
他呼唤地灵道:“去,打开福地门扉,将那位和我容颜相似的年轻蛊师,挪移过来。”

\end{this_body}

