\newsection{捡漏}    %第一百九十六节:捡漏

\begin{this_body}

“无上真传!”方源瞳孔扩张,目光灼灼。

距离真传光团,还有千步之遥,方源就感觉到真传的澎湃气息。

这是普通真传、无双真传都没有的感觉。

在这股气息的影响下,方源脑海中念头攒动,无数念头凭空产生,记忆画面不断闪现!

如果说平时里,他的脑海就像是一汪深潭。那么此刻,他的脑海中念头勃发,宛若一条恢弘瀑布,陡然砸入深潭。

念头狂涌,掀起浪花重重!

这种感觉,无比奇妙,让方源一时间都找不到准确的词语去描绘它。

平时里,深埋于心底的记忆,一个个鲜活起来,吹去表面的灰尘,一个个栩栩如生,清晰无比地展现在他的脑海中。

大量的灵光,不断闪现,许许多多的奇思妙想在方源的脑海中凭空而生。一些修行时遇到的关隘疑难,在灵光的闪现下,不断地被瞬间攻克。

方源心中不由自主地涌起,一股强烈无比的自信心。

这是无比奇妙的感觉,仿佛能掌控一切,仿佛能解决一切的难题……

这是一种独特的“力量”,让人不禁沉迷、深陷。

“小子,快清醒过来,我也知道这种感觉是多么醉人。如果你不想死,就千万不要沉迷!快看看你自己罢!”脑海中,墨瑶大叫。

这声音提醒方源,他看了一眼自己,顿时悚然失色!

短短片刻功夫,他的身躯变得虚弱而疲惫,原本光滑的额头出现了皱纹,原先强健有力的臂腕。蒙上一层苍老枯朽的感觉,正在变得越加的瘦弱不堪。

如果不是墨瑶提醒,方源恐怕还得沉迷在奇妙的感觉当中,而忽略身体上的变化。

“难道这只蛊是?!”方源脑海中。想到一个令他都感到无比震惊的答案。同时他催动移动蛊虫,带动身躯急退!

真传秘境中。并不禁蛊虫使用。

无上真传,也并非针对方源追踪而来,而是直线飞冲。

方源迅速和其拉开距离,心有余悸地望着这道无上真传。划破暗空,远远飞走。

“这还只是无上传承的气息,就让我差点沉迷,死在当场。它的考验,又该如何艰难呢?”方源惊叹。

他连人气蛊这道无双真传,都接近不了。

对于无上真传,更是难以企及。别说接近了。稍微靠近一点,真传泄露出来的气息,就差点要了他的命。

正如墨瑶所说的那样,凡人境界的方源――太弱了!

真传虽好。他却无力承担。就好像是蜜蜂采蜜,蜂蜜若大如拳头,反能将蜜蜂溺死。

“这个无上传承中的仙蛊,该不会是……传说中的那只蛊吧?”方源开口,他几乎有八成的把握。但这个答案,太过于惊人,导致他都有些不敢相信。

“呵呵呵。”墨瑶娇笑连连,“小子,你猜的不错。想当初,我也不敢相信,但事实就摆在眼前,由不得你不相信!没错,这个无上真传中包裹的仙蛊,正是传说中的九转智慧蛊!”

智慧蛊!

在《人祖传》中,就早有记载。

这是九转仙蛊,能给予蛊师无穷的智慧!

但是要使用它,代价极为昂贵。

《人祖传》中明确记载,人祖、古月阴荒先后将中年奉献给它,这才得到它的帮助。

也就是说,要使用智慧蛊,就得消耗寿命!

方源还未靠近它,寿命就不断损耗,长出皱纹,**凡躯在很短的时间内加速老化。

对于方源来讲,智慧蛊就是货真价实的索命死神!

“真不知道,巨阳仙尊当年是如何得到它,又动用了何种惊天手段,将其封印在这里的。难怪巨阳仙尊千方百计,设想出各种延寿的方法,想来是用智慧蛊用多了。”方源感慨良多。

墨瑶则叹息道:“巨阳仙尊之所以动用智慧蛊,恐怕只有一个目的,那就是延寿。可惜智慧蛊用的越多,寿命越少,最终他也没有得到他想要的答案。”

根据史料记载,巨阳仙尊活了八千岁余,最终还是陨落了。

八千岁,在九转尊者当中,已经是中上等的成绩了。寿命最长的,是一代元始仙尊,有两万五千岁。寿命最短的,是红莲魔尊,只有三千岁。

其他的尊者,普遍有七千岁左右的寿命。

“真传秘境中,有三道无上真传。你说,其他两道真传是什么样子的?真是想见识一下啊。”方源双眼涌动起莫名的光辉。

墨瑶听出话音不大对头,连忙劝道:“小子,你不要瞎想,还是赶紧取走一道普通真传走人吧!你只有十角楼主令,也只能取走普通真传。就算你通过无双真传的考验,你也拿不走它们。更别提无上真传,你这个弱不禁风的小身板,连它们的气息都承受不住!”

方源哈哈大笑:“你这么一说,我更想去见识见识了!这次良机,千载难逢,错过这次,我恐怕就再无机会了。”

“臭小子,你疯了?你知不知道,就在刚刚的一会儿功夫,你至少减少了两年的寿命。你再这样耗下去,等到真传飞如流星,你不禁什么都得不到,甚至很可能要丧生于此!”墨瑶急道。

方源笑声不绝,却是不为所动。

他继续游走,看到普通真传、无双真传,都远远绕走,根本没有一丝取走它们的意思。

墨瑶意志看到这里,着急了:“你个疯子!天呐,你脑子里究竟在想些什么?放着大好的真传不取,就单单为了开眼界?你个蠢货,你死了,我怎么办?近水楼台怎么办?”

“我就算死了,你这股意志当然也存活不了。但你放心,近水楼台中还有你的意志。你大可以等到后面的有缘人,再将这个珍贵的任务托付给他(她)。”方源缓缓道。

墨瑶再劝,方源一意孤行,我行我素。

他不断探索。又过了数天。真传速度越来越快,方源险况迭发。

好几次。他都有生命之危,最终险险避开。

墨瑶屡劝不止,急都快抓狂了:“小子,我服了你了。你真是一根筋!好了,你不要逗留了,我告诉你另外两道无上传承究竟是什么。”

“这第一道是运道传承,巨阳仙尊开创此道,就是靠着运道造诣独领风sāo,无敌天下!第二道就是八十八角真阳楼的控制权。此楼乃是巨阳仙尊和长毛老祖合力炼成,能搜刮一域的蛊虫。有了它,就有源源不断的蛊修资源!”

她是真的急了。

平时里,这种内幕消息就算方源发问,她都未必有心情回答他。现在却是直接吐露。生怕方源再犯傻。

但怕什么,来什么。

方源呵呵直笑,仿佛没有听到似的,继续飞游。

几天之后,真传速度已经快如飞鸟,方源闪避一两道,十分容易。但真传秘境似乎正在缩小,大大小小的真传在狭小的空间飞射,一个个拖着长长的光尾,五颜六色,几乎已经交织成一片稀疏的光网,方源只能在夹缝中求生存。

到了这种程度,方源全神贯注,一刻都不敢放松。唯恐疏忽大意,酿成身陨的悲剧。

“快走了,到了此时此刻,你连普通真传都拿不走了。只要你停留在原地,危险就随着你的呼吸次数激增。唉,你居然在这么关键的时刻犯病!”墨瑶劝说累了,有气无力。

方源目光炯炯,忽然问道:“你说,有没有可能,两道真传相互碰撞,让我捡了漏子呢?”

墨瑶听了这话,愣住了。

但旋即,她尖叫起来:“你这个蠢货,你这个笨蛋!这么明显的差错,堂堂的巨阳仙尊会犯吗?原来你是这么想的,我高估你!我太高估你了!你这是聪明反被聪明误,怎么可能有这么明显的漏洞让你钻?你想得太美了,你太天真了!”

“哦,原来是我一厢情愿了。”方源呵呵一下,神情淡然,继续探索。

其实,他早已知道这一点,故意说这话,只是想挑逗一下墨瑶。

“你怎么还不走?你是一心想死吗?!”墨瑶被方源一阵挑逗,彻底抓狂了。

“我死不死,是我的事情,管你屁事。你已经死了,一个死人,在这里乱叫什么?”方源淡淡嘲讽,神情平淡,仿佛深陷险境的不是他。

“可恶!混蛋!白痴!”墨瑶似被点燃火气,痛声咒骂,滔滔不绝。

轰。

就在这时,万步的远处突然发生一声爆响。

方源循声望去,只见两道真传撞到一起,随后各自弹开,朝着不同的方向继续飞射。

方源呆了。

他没有想到,自己胡言乱语,居然真的发生了真传相撞的这一幕!

尤其是其中一道真传,好像还是一道无上真传!!

墨瑶也呆了。

“这怎么可能?”她声音陡然提高,尖锐刺耳,再也没有炼道宗师的风范,“我当初探索这里时,怎么没有碰到这样的好事?!”

但她旋即双眼一瞪,恍然大悟:“原来如此!那是运道无上真传,当年我拼死打出一道裂缝,想得到里面的鸿运齐天蛊,结果却让它飞走了。只好利用所得的部分传承,退而求其次,炼出招灾蛊!”

这道无上真传,被墨瑶打出一个裂缝,已经不全,因此没有遵循规范,和其他真传相撞了。

砰。

又是一下碰撞。

方源双目一瞪,这次撞击,好像从运道无上真传中,撞出一只蛊?

方源楞了一下,道:“你说,那是什么?”

墨瑶也楞了,旋即叫道:“那是一只运道蛊虫,你还愣着干嘛,还不快去取了?”

方源却有犹疑:“取了这只蛊虫,该不会让我迎来运道真传的考验吧?”

运道真传是无上真传,它的考验方源根本承受不住。

“屁的考验!撞击让真传光团的裂缝变得更大,它正在分解。你赶快拿了走人,再不走你就死在这里吧!”墨瑶吼道。

方源轻笑一声,左闪右躲,艰难靠近,一把将这只运道蛊虫捉到手中。

这是一只五转凡蛊,却非仙蛊。

“小子,你还不快走?!”墨瑶吼叫连连。

“哈哈,急什么。”方源朗笑,却未撤离,而是留在真传秘境,左右打量。

“算算时间,这样的火候,也该差不多了。”他口中喃喃。

“小子,你说什么?”

墨瑶话音刚落,便将方源洒下一蓬蛊虫,骤然一齐催动,形成一个漩涡。

漩涡绽放出澎湃吸摄引力,将真传的光辉各自吸取一道,投入漩涡当中。

几个呼吸之后,漩涡一滞,轰然崩解,显现出一道门户。

“正该如此!”方源毫不犹豫,投身进去。

下一刻。

他见到了地灵!

------------

\end{this_body}

