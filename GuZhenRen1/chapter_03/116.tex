\newsection{扬名}    %第一百一十三节:扬名

\begin{this_body}

面对黑楼兰的招降,东方余亮沉默以对。

他站在城墙上,看着眼前黑压压的敌军,微风吹拂着他的长发,他的衣摆也跟着风轻轻晃动。

他轻轻地叹了一口气。

纵然他是智计百出,常常谋算在先,但是双方的差距太大,终究到达了这最后一步。

智道并非无敌之道。

纵观历史,蛊师的流派层出不穷,百花齐放,就算是有诸如气道、力道等盛极一时,但终究没有破灭其他,独自制霸蛊师界的流派。

每一个流派,都有自己的优点,也有自身的缺陷。

尤其是蛊师流派,都是建立在物资的基础上。当时代变迁,环境变化,蛊师修行需求的物资减少时,流派的生命力也就随之老朽了。

只要稍微熟知历史的蛊师,便可以知道,漫漫的光阴长河,不知埋葬了多少的流派。

说起来,智道从远古时代就兴起,一直流传到今天。即便智道蛊师一直数目稀少,但这样生命力长久的流派已经可以说是,独树一帜了。

这个世界上,从来就没有无敌的流派,只有无敌的蛊师。

但能做到无敌的境界,整个历史中,也不过只有区区九位罢了。

东方余亮只是五转的智道蛊师,虽然占据世俗的巅峰,但是距离无敌还差得太远太远。

即便他身怀独创的杀招七星灯,但真元也随之消耗巨大,终究不能持久。面对黑家这样的庞然大物,他早已有独木难支,势单力孤之感。

“如果我是奴道蛊师的话,兴许还有力挽狂澜的机会。但就算是奴道,也要担心斩首战术。即便是狼王常山阴,也不敢独自一人,率领狼群脱离大部队。所以,只有晋升为蛊仙,才能站在凌驾于凡尘之上啊。”东方余亮在心中叹息。

这时,水魔浩激流上前搦战。

“风魔,你给我滚出来受死!”他大叫着,指名道姓。

风魔大怒,低吼一声:“浩激流,你休要张狂!”

说着,他从城墙上猛地一纵,人还在半空中,就催谷蛊虫,两道四叶风刃旋即形成,飚射出去。

“你怎么尽都是这些老招数!”水魔哈的一笑,不闪不避,悍然迎上。

轰轰轰!

风水双魔已经交手十多次了,对彼此都熟悉至极,此刻一交手,就呈现白热化。

风刃和水弹互射,半空中对撞,爆炸。

风魔攻势犀利,擅长穿插游走,水魔则招式狂放,磅礴浩荡。

这两人其名已久,一时间身影纠缠,声势煊赫,却不分胜负。

两军茫茫蛊师,都将目光集中在这两人身上。

四转蛊师的强悍,虽然很多人都有过深刻体会,但此时再看,仍旧有一种胆战心惊之感。

战斗片刻,水魔渐渐占据上风。

风魔的状态并不好。蛊师的战力,也有起伏的时候。

就像现在,大局已定,被黑家重重包围,东方部族士气低落,风魔自然也受到了影响。

看到原本势均力敌的老对手,被自己压在下风,水魔高兴的大呼小叫,攻势更加浩大。

将是兵之胆,看到这一幕,黑家大军的士气更加高涨,城墙上的东方盟军则陷入更深寂的沉默当中。

王帐中,黑楼兰哈哈大笑,又遣出一位四转强者,出阵搦战。

东方余亮便派遣一人应付。

但这两位四转蛊师的战斗,远不如风水两魔交手的激烈。

不仅雷声大雨点小,而且还在战斗中相互攀谈,甚至谈到某代祖上时两族之间,还有联姻关系。

东方余亮的脸色愈加难看,黑楼兰的笑意则更浓郁几分。

东方盟军的士气低落,人心也已经涣散。各方结盟的势力,几乎都已经开始找寻退路。

黑家大军士气高涨,四转的蛊师强者纷纷请战。

黑楼兰笑着,一一应允。

很快,两军阵前,就开辟出了十二个战圈。

“某家潘平,何人与我一战?”潘平气势汹汹,在得到黑楼兰的应允后,成为第十三个出战之人。

东方余亮却是一阵沉默。

战斗至今,他麾下的四转蛊师,也多有陨落。很多强者,瞻前顾后,已经消极怠工,阴奉阳违。

虽然建盟之时,各大首脑,知名的强者都用了毒誓蛊。但毒誓的内容,却不苛刻,仍旧有大量的漏洞可以钻。

作为盟主的家族,纵然希望将其余各大部族,死死的绑在自家的战车之上。但其他的势力,也不是傻子。所以这毒誓的内容,是传承了许多年,无数代人协调而得的。

此时潘平邀战,东方余亮竟发现,手中已无将可用。

他沉吟了片刻,终于下了一道命令。

“什么?东方盟主竟然下令,让我父亲出战?!”城墙后的营帐中,唐方看到眼前的传信使,脸色非常难看,双眼直欲喷火。

在之前的战斗中,唐家族长为了掩护族人撤退,被黑家两位四转蛊师围攻,受了重伤。其后,一直卧病在床,没有休养得好。

“这是盟主之命,难道你们唐家想抗命不遵吗?我知道贵族族长重伤,卧病在床。但卧病在床的族长多的去了,但得了盟主的命令,他们还不是都上去参战了?”使者语气强硬,看着唐方的目光中,带着毫不掩饰的不屑之意。

“你!”唐方大怒,低吼道,“他们那都是在装病,我父亲可是货真价实的受伤!”

“好了,小三,不要再说了。这一场战,我作为唐家族长,亲自参战责无旁贷。”这时,脸色苍白的唐家族长走了出来。

“哼,领命就好。”东方家的使者冷哼一声,拂袖而走。

“可是阿爸,你的身体……”唐方心中十分担忧。

“没有事的。”唐家族长轻轻地拍了拍儿子的肩膀,“这些天我一直在休养,身上的伤势好得七七八八了。今日一战,极可能是两方的最后一战。如果我不出战,场面上是过不去的。而且对整个家族而言,也是弊大于利。”

唐方咬紧牙关:“那阿爸你可要小心,阿姐还在对方手上,如果有机会的话……”

“嗯,我尽力而为。”唐家族长皱了皱眉头,走出了营帐。

他来到城墙,见过东方余亮,便下去战场,和潘平展开激战。

唐方站在墙头,目光紧紧地盯着父亲。

“少族长,没事的。族长大人虽然身上有余毒未清,但今天这一战,却不同以往,大家都留着手呢。”唐家的一位家老宽慰道。

唐方看着父亲和潘平打得难分难解,不温不火,心中的担忧也就消减了许多。

但就在这时,潘平陡然爆发,将腰间一直挂着的弯刀拔出。

众人的眼中,便见一道闪亮的白光,一晃即逝。

再定睛一看时,唐家族长竟然身首异处!

“啊,父亲!”唐方怔了半晌,随后发出悲切的呼喊。

这一变故发生的太快了,双方惊愕了几个呼吸,这才发出嘈杂的议论声。

“唐家族长唐幽,被我潘平讨取了!”潘平双眼冒着嗜血的光,一手提起唐家族长的人头,兴奋地呼喊起来。

唐方眼前一黑,当场昏倒下去。

“刚刚那是什么?”

“我只看到了一抹光,实在太快了!根本就没有看清楚。”

“不晓得是什么蛊,或者是潘平的杀招?”

潘平骤然斩杀了同级的强者,一时间风头无两。就连一直在闭目养神的方源,都微微睁开眼帘,向他投去注视的目光。

北原地域广袤,是百战之地。无数的战争,磨砺出层出不穷的强者。这些强者,或有一些手段底牌,一直雪藏着,不为人知。

潘平虽然是四转蛊师,但其实声名不显,在所有的四转蛊师中,并不出名。但经此一战,他踩着唐家族长的尸体,彻底扬名。

潘平满脸春风,得胜而归。

黑楼兰哈哈大笑,当场命人端去自己的酒杯,就杯中的美酒赏赐给潘平。

“谢盟主大人赏赐!”潘平在王帐中昂然站立,一口饮下杯中的美酒,目光顾盼,神采飞扬。

他得到这只蛊虫,纯属意外。有一次,几大部族开放集市,他看着这弯刀精美,就购买下来,当做把玩的器物。

但没想到,把玩的时候,发现这弯刀中的秘密。

弯刀的刀刃上,有一抹寒光。这寒光,竟然是一只神秘的蛊虫。

潘平费劲千辛万苦,才将这只蛊虫炼化。虽然不知道此蛊的名号,但屡次为他斩杀强敌,攻击极为犀利。

此届王庭之争,他一直怀着高度的期待。

原本,黑家和东方大军第一战时,他就主动请命,要第一个出征搦战。其时,便是打着利用此蛊,在众目睽睽之下斩杀强敌,一战扬名的意图。

但人家唐妙鸣,却指名道姓,要挑战狼王常山阴,让潘平好不郁闷。

让他更郁闷的是,方源根本不按照常理出牌,直接出手,略过了挑将的环节,引得两军大战。

潘平苦等良久的机会,就这样没了。之后虽然也有激战,但却不是潘平想要的环境。

“不过,今日一战,终于叫我抓住了机会。唐家族长唐幽,是早就成名的人物。经此一战,我的地位立即暴涨,几乎可以成为黑家第一战将。毕竟水魔浩激流,虽然名声更大,但一直没有拿下风魔。至于狼王常山阴,他是奴道蛊师,我不和他比……”

潘平环顾四周,感觉众人看他的目光都变了,心中的快感又浓郁了几分。

“这就是人上人的感觉嘛,嘿嘿,总有一天,我潘平的名声将响彻整个北原!”他在心中呐喊着。(未完待续。请搜索,小说更好更新更快!)

------------

\end{this_body}

