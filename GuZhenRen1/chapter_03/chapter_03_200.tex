\newsection{可怜人}    %第二百节:可怜人

\begin{this_body}

%1
“请太白老先生为我们做主啊!”

%2
“在下恳请老先生出面,为我们求情。”

%3
“我族上下,老弱妇孺,一族前途尽在大人您的手中了……”

%4
房屋中,一群蛊师一齐拜倒在地,或泣不成声,或满脸悲切,都在恳求太白云生相助。

%5
这些蛊师,身份超出寻常,都是各族族长。

%6
娄家、路家、席家……这些部族都是中型或中小型,在攻略八十八角真阳楼的过程中,他们伤亡惨重,再这样下去,恐怕就有灭族之忧。

%7
之前几次,他们联合上书,要求撤军,但都遭到黑楼兰的严厉拒绝。

%8
生存第一。

%9
黑楼兰的穷兵黩武,已经严重危害到这些中小型部族的生存和发展。

%10
但碍于黑暴君的凶名,这些族长不敢直接去找黑楼兰。他们合计了一下,便联袂一同来寻求太白云生的帮助。

%11
太白云生乃是当代北原的第一治疗大师,公认的慈悲心肠,在北原凡人心中威望第一。如今他又成为黑家的外姓家老,深得黑楼兰的器重。

%12
如果能得到太白云生的帮衬,说不定就能让自家部族得到宝贵的喘息,休养的机会。

%13
“你们都先起来说话。”太白云生劝道。

%14
“老大人,您不答应,我们就不起来!”

%15
“请您到我族看看去吧,您就会了解我们的痛楚和悲伤。我们的伤亡实在太惨重了,女人们失去了他们的丈夫。孩子们失去了他们的父母,老人们丧失了他们的儿女……”

%16
“老先生,请你劝劝黑楼兰大人吧。再这样下去,我们都要灭族了。”

%17
“哼,没有死在王庭之争,却灭亡在王庭福地,这话要传出去,黑家的名誉往哪里搁?这又该多么的讽刺啊!”

%18
族长们你一言,我一语。态度坚决,都不起身。

%19
“唉……”太白云生长叹一声,他对黑楼兰的做法其实也早有不满。

%20
明明可以徐徐图之。为什么这样操之过急呢?

%21
造成如此惨重的伤亡,实在让他不忍心。

%22
在太白云生想来:恐怕是之前第二关时,丢失了楼主令,刺激到了黑楼兰。还有一个原因。就是黑家太上家老们。催促得紧。黑楼兰迫于压力,不得不尽全力闯关。

%23
“来人,给诸位族长们上茶。”太白云生召唤下人。

%24
又转过头安抚一干族长:“老夫这边过去,劝说黑楼兰族长,我必当尽力而为。”

%25
“老先生,您真是仁厚的大好人呐。”

%26
“有老先生帮忙,真是我族的幸事!”

%27
“老大人,这里是我们的联合上书一封。我们就静候大人您的佳音了。”

%28
众族长们感恩戴德。

%29
太白云生接过上书,揣入怀中。辞别众人,便来到大殿,去见黑楼兰。

%30
黑楼兰却不在大殿中处理事务,而是在后花园中散心。

%31
太白云生便又赶过去,见到黑楼兰时,后者正在湖心亭中下棋。

%32
黑楼兰听了他的来意,起先严词拒绝,但太白云生好言相劝,黑楼兰渐渐松动:“这样吧,我可以暂时让这些部族休养,但他们必须得借出全部蛊虫。我大军进攻,不能因为他们的缺席,导致战力下降!借一只蛊虫,免一个名额。借二转蛊,免二转蛊师。借三转蛊,免三转蛊师,以此类推。”

%33
“族长英明仁义,在下先替各族族人,谢过族长的宽宏大量了。”太白云生见好就收,回去之后,便将这消息告知众位族长。

%34
众位族长面面相觑,最终接受了这个条件。

%35
蛊虫,乃是一位蛊师的根基所在。借出的蛊虫,容易损伤,不过总好过丢失了性命。

%36
这些部族的减员情况,着实太严重了一些。

%37
次日,黑楼兰便下令黑旗军前往各个部族住处,统一收缴蛊虫,过时不候。又派遣太白云生压阵,处理相关事务。

%38
连续三天时间,终于将此事处理完毕。

%39
太白云生回来汇报,得到黑楼兰的宴请款待。

%40
酒席间,黑楼兰频频敬酒:“老先生,请畅饮此杯!得幸有老先生加盟我族,才能令我大军减少无数伤亡。老先生你活人无数,功德无量啊。”

%41
“不敢当,只是谋事在人成事在天,长生天在上,尽力而为罢了。”太白云生面临惭愧之色,举杯回敬。黑楼兰的赞誉,让他想起高扬、朱宰。

%42
事实上,从血道大殿回来之后,他听到的每一句赞誉,都像是一种讽刺,鞭笞他的心灵。

%43
不过今天他处理了这个事务,缓解了各个中小部族和黑家的矛盾,从某种意义上来讲,救下了许多蛊师的生命。

%44
这让他感觉好受了一些。

%45
“长生天在上,尽力而为……”黑楼兰一饮而尽,放下酒杯,口中喃喃,感慨道,“老先生这话,说得不错,说得好啊。长生天看着我们,未来的路神秘不测,作为凡人的我们也只有尽力而为。太白家老你尽力而为,我黑楼兰也当尽力而为才是啊!”

%46
太白云生听着,顿觉黑楼兰话中有话,他便问道:“族长大人,何出此言呢?”

%47
黑楼兰哈哈一笑,正要开口,黑书从屋外进来汇报。

%48
他身带血迹,一脸恭谨,凑到黑楼兰的身侧:“启禀族长大人,黑旗军幸不辱命,已经处置妥当。”

%49
太白云生见其杀气萦绕周身,心中顿生不妙之感,他连忙问道:“什么处置妥当了?黑书!你们究竟处置了什么?”

%50
黑书并不答话,只是眉头一挑,斜眼瞥了一眼太白云生。目光中流露出对太白云生悲天悯人的行事风格的不屑。

%51
“哈哈哈,太白家老!正像你之前说的——尽力而为!你在尽力,我也在尽力!我要尽力冲击关卡。尽快完成太上家老的任务。你说说看,这些人却临阵脱逃,该当何罪呢?他们是依靠我,才能来到这里发展的。有好处就占,见坏处就躲,世间有这样便宜的事情吗?哼!如果人人都学他们这样,那我的大军还能存在吗?还会有人跟随我冲击难关吗?”。黑楼兰说到这里。言辞严厉,眼中更是凶光毕露。

%52
此话一说,处置什么。不言而喻!

%53
“黑楼兰,你!”太白云生腾的一下,站起身来。

%54
他对黑楼兰瞠目而视,脸上流露出震惊、愤怒。甚至仇恨之色。

%55
“大胆!”黑书不悦。就要出手,但被黑楼兰伸手所阻。

%56
“太白家老,你可别忘了你现在的身份。”黑楼兰收敛凶光,好整以暇地倒下一杯酒。

%57
太白云生双手紧紧攥拳,浑身都气得微微颤抖。

%58
他死死地盯住黑楼兰,想要咒骂,却骂不出来。

%59
黑楼兰嘿然一笑,亲自为太白云生斟酒:“老先生。坐下来吧,这样的良辰美景。好酒好菜,不可浪费了。”

%60
“去你的好酒好菜!”太白云生怒极挥袖,砰的一声,将酒杯直接扫落在地。

%61
黑楼兰不以为杵,反而手指着桌上饭菜,大笑道:“太白家老,这可是你的庆功宴啊。没有你出面,他们怎么可能这么信任我们,借出蛊虫呢?没有了蛊虫在手中,我黑旗大军才能三下五除二,轻而易举地镇压了他们。这一切都要感谢你啊。”

%62
太白云生听了这话,如雷震心口,他蹬蹬蹬后退三步,脸色瞬时苍白至极。

%63
“黑楼兰,你真是个卑鄙小人!”

%64
“卑鄙?这样天真的话,你也说得出口吗?太白家老,你也曾经是一族少族长,你痴长这么多岁数,怎么还看不清世情呢?所谓政治,从来就都是肮脏的。”

%65
太白云生想要张口反驳,但却找不出反驳的话来。

%66
他冷哼一声,拂袖便走,头也不回,一直疾步走出大殿。

%67
“族长大人,是否要派人跟踪监视太白云生?”黑书谏言。

%68
黑楼兰举杯饮酒,神情淡然,没有一丝紧张之色。他一瞥太白云生的背影,冷笑一声:“太白云生,不过区区一位治疗蛊师,怕他什么?名望么?呵呵,仅此一事,谁还敢信任他?好事不出门恶事行千里啊……”

%69
黑书闻言,心头微微一震:“但凡高位者,果然心机深不可测!”

%70
他现在才看出来黑楼兰此举,乃一箭三雕。

%71
第一,血腥镇压中小部族,杀鸡儆猴。第二,敲打太白云生,让他明白自己的身份。第三,打削太白云生的名望,稳固自己的统治地位。

%72
太白云生疾步来到事发地点。

%73
大屠杀已经结束,黑楼兰为了更好地威慑其他人,并未命令属下及时地清理现场。

%74
太白云生环顾四周,脸色发白,心头颤抖。

%75
圣宫染血,横尸遍地。

%76
不仅有强壮的男子,还有老弱妇孺。他们有的睁大双眼,死不瞑目;有的被削去手脚,腰斩两半,惨不忍睹;还有的衣衫凌乱,神情扭曲痛苦,显然临死前遭受了非人的侮辱……

%77
这一切,这一切……

%78
“都是我造成的啊,如果不是我相信黑楼兰,如果不是我居中调停……我太傻了,我太天真了!我居然没有看出黑楼兰的险恶用心!!这些人,都是信任我而死的呀!”

%79
太白云生浑身都颤抖起来,血腥气味扑鼻而来,满地的死尸构成一幅悲催的画面,对他脆弱的心境造成巨大的冲击。

%80
他佝偻着背,微风中花白的胡须微微晃动,深深的皱纹,呆滞的目光,紧紧握拳的双手,无声地宣泄出他内心深处的悲伤、愤怒、懊悔、自责!

%81
扑通。

%82
他无力地跪倒在地上,双手撑地,地上的血泊将他的双手,膝盖,裤腿都沾上鲜红之色。

%83
“对不起,对不起……”

%84
太白云生呜呜痛哭,一时间竟老泪纵横。

%85
“这是一位好人呐……尽管内心中也有阴暗。但生死之间有大恐怖。面对死亡,常人谁能不失态呢?”远处,一处隐蔽的角落里。方源一直悄悄地注视着。

%86
他目睹了一切的发生。

%87
原来黑棺气运,代表着严重的杀僧祸。这些部族的灭亡,就是最佳的明证。

%88
而他身上的死气,比这些族长们要浓重数十倍,甚至上百倍!

%89
“呵呵呵,真是有趣。运气么……”方源嘴角勾勒起一抹傲然冷笑,他倒要看看所谓的坏运气。能否阻挡他的脚步。

%90
“这一生,将没有人,也没有任何理由。任何困难,能影响我前进的决心!”他目光深沉且冰寒,投向太白云生。

%91
“谁说好人有好报的?”想到这里,方源冷冽的目光中。又增添一份讽刺的意味。“就让我推你一把罢。”

%92
他暗中催动蛊虫,形成飘渺幻音,一缕缕幽幽传入太白云生的耳畔。

%93
太白云生正处于心境崩溃,心神极度动荡之际。

%94
此刻,他忽然听到声音,和高扬、朱宰一般的语调。

%95
“我相信您!您一定会救我的,不是吗?”。

%96
“老大人,您可是我们的救命恩人呐。我们感激您,愿意冒着生命危险来帮助您!”

%97
“啊!”太白云生顿时惊呼一声。陡然睁开双眼。

%98
泪眼朦胧中,他看到地上的血泊。

%99
血泊晃动,赫然呈现出噩梦中,高扬朱宰葬身血兽群中的景象!

%100
太白云生浑身一颤,静如雕像,下一刻他忽然仰头大吼,声嘶力竭。

%101
“啊——!”

%102
凄厉沙哑的声音,回荡在这片血腥的屠戮场中。

%103
“呵呵呵,嘎嘎嘎,哈哈哈!”

%104
吼叫声结束之后,却是太白云生的大笑声。

%105
笑声癫狂恣意,是郁愤中的悲号,又是对世俗的,对自己的冰冷嘲讽。

%106
“疯了?疯了!”

%107
“太白家老他疯了!!”

%108
周围的黑家蛊师们惊叫连连。

%109
而方源则翘起微笑的嘴角,他是穿越者,又是重生者,丰富的经历让他对人心的把握,已经妙到毫巅。

%110
太白云生是一个好人。

%111
但他在血道大殿中的下意识行为,却是背叛。

%112
这个行为,让他对自己本身的价值存在产生怀疑和否定。

%113
一个人如果否定自己,怀疑自己,那么他无疑会陷入痛苦的深渊。

%114
很显然,太白云生是痛苦的,他心怀愧疚,几乎每天夜里都受到相似的噩梦的折磨。

%115
如果让他接受自己的卑劣行为,他又做不到,他这一生都在遵循着正义和仁爱之路,他想回归过去,但明明白白的事实横亘在他的内心深处,是他无法逾越的障碍。

%116
他在痛苦的深渊泥沼中沉沦,无力的挣扎,只会让他越陷越深。

%117
族长们的请求,带给他自我拯救,向良心赎罪的机会。

%118
如果真的能够办成,无疑是对他的宽解,是一计恰逢其会的良药。

%119
但事实却是,黑楼兰下令株连九族,一个活口都不留。太白云生反过来,倒成了助纣为虐的帮凶。

%120
良药变质成钻心的毒药,剧烈的毒素让他不能自己,让他心境濒临崩溃。

%121
就在这个关口上,方源轻轻一推,终于促成太白云生的爆发,歇斯底里的爆发。

%122
恐怕只有方源能够理解他了。

%123
从某种程度上来讲,搜过魂,知晓太白云生一生经历的方源,可谓是他的知己。

%124
但残酷的是,这位知己不是亲朋,而是算计他的敌人。

%125
隐藏在阴暗中的方源,静静地聆听着太白云生的大笑声,怀着欣赏的神情,淡淡微笑。

%126
这位可敬又可悲的老人,他嘶声力竭的怒吼声,大笑声,满脸的泪流中,是对自我的审问,是对残酷世情的批判,是对命运捉弄的无力反抗。

%127
“否定了自己的你,还能干什么呢?”方源轻声喃喃,眼眸黑幽深邃。

%128
“去血道大殿,重新闯关吗?不,到了那里,你满脑子都会是高扬朱宰的影子吧。去寻求帮助吗?谁还能给你帮助?黑楼兰?想来刚刚和他闹翻了吧。退一万步讲,就算得到了寿蛊,你还能安心使用吗?”。

%129
“呵呵呵。如今的你,还有什么路可以走呢?还有什么可以选择呢?否定自身,自暴自弃的你,悲痛欲绝,受到良心拷问的你,还能怎么选择呢?”

%130
方源的心中,一个答案越加清晰起来。

%131
“所以,就去选择吧,不要让我失望!”阴暗中,方源目光灼灼,盯着癫狂的太白云生。

%132
太白云生大笑,笑声苦涩,仿佛哀嚎。

%133
他撕扯着衣衫,涕泪交零,再无一丝五转巅峰强者的风度。

%134
他是位苟且贪生之人,也是位被残酷冰冷的世俗逼到角落里的好人。

%135
事实上,他更是一位——

%136
可怜人。

%137
不知什么时候起,忽然起了微风。

%138
风声渐大,天空中云翳生成。

%139
以太白云生为中心,氛围无声而又迅速地转变着。

%140
某种玄妙的变化开始了,天地开始微微震动起来。

%141
“终于没有令我失望啊……”方源兴奋到战栗,他抽身而退,悄无声息。

%142
他远远退去,一路后撤。

%143
因为……

%144
蛊师升仙,可不是闹着玩的啊。

%145
很快,风云激突,天地骤变!

%146
阴测测的浓厚黑云,仿佛随时都要坍塌下来,压毁圣宫。

%147
圣宫上下,一片惊呼大叫。

%148
“怎么回事?”

%149
“到底发生了什么?!”

%150
大多数人不明原因,陷入慌乱当中。

%151
“这个……该不会是?!”少部分的有识之士,目瞪口呆地望着这番动荡气象。

%152
“查,给我查!究竟是谁?!”大殿中,黑楼兰大发雷霆。

%153
“大人,是太白云生!”黑书满身大汗,跌跌撞撞疾跑过来禀告。

%154
黑楼兰神情一僵,一时间呆若雕像。

%155
几个呼吸后,他反应过来,神请扭曲地咆哮道:“撤!让所有人都撤离圣宫。当然,想死的可以留下来!”

%156
ps:háo已近!有许多预测帝猜到了某些剧情(俺恨你们!),当然,也有绝对猜不到的。

%157
另通知:企鹅读者群(一群、二群、三群、vip群)将于近期大量清人,想要留群的人,请于一周内冒历时久,群里潜水的太多,讨论情节的少,群数量太多,实在不便于管理。

%158
已接近第三大章结尾。接下来还有第四大章,第五大章。

%159
不会太监,也争取不烂尾。

%160
方源一直在追逐他梦想,我也一直没有放弃。的地方,别问我个人隐私问题,欢迎不放弃支持的读者朋友们。也请被踢的朋友童鞋们,多多理解,原谅我。

\end{this_body}

