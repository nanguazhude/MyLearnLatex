\newsection{星念蛊}    %第一百节:星念蛊

\begin{this_body}

%1
“终究是没有成功么……”东方余亮背负双手,看着窗外的细雨,轻叹一声。

%2
这只鱼翅狼,是他特意从野外活捉过来,付出了不菲的代价。

%3
而后,又在鱼翅狼的身上动了手脚。唯恐对方看出破绽,精心选择了一只五转魂爆蛊,种在鱼翅狼的身上。

%4
最后,他又调派了北原赫赫有名的魔道杀手影剑客,将启动魂爆的对应蛊虫交到她的手中,千叮咛万嘱咐。

%5
他精心设计了这个刺杀的计划,为此不惜损耗精力,反反复复测算推演了三四遍。直到将所有的破绽都消弭干净。

%6
不管是鱼翅狼身上的状态,还是它出现的地点、时间,甚至是发现鱼翅狼的蛊师的应对态度,他都考虑周详。

%7
但就算这样,仍旧是没有成功。

%8
关键时刻,狼王常山阴察觉到不妥之处,让他人出手。影剑客边丝轩见机不妙,不得不启动魂爆蛊。

%9
魂爆的力量,无形无色,肉眼无法察觉。但常山阴却在第一时间脱口而出,道出真相。

%10
虽然受到边丝轩的攻击和牵制,他只撤退到魂爆范围的最边缘,但是危机关头,他冷静非凡,竟然将赶来支援的自己人当做盾牌,挡住大部分的魂爆威能。

%11
“盛名之下果无虚士,不愧是名动北原的狼王。”东方余亮听了边丝轩的汇报之后,心中对方源的重视程度,又拔升一个档次。

%12
异兽狼就在眼前,他却能耐得住诱惑。谨慎无比的性格,危机下正确的判断,瞬间认出魂爆的眼界,将自己人当做肉盾的冷酷,以及被刺杀之后,却没有盲目追击的冷静……

%13
“狼王……”东方余亮口中喃喃,心中越加沉重。

%14
“东方盟主不必担忧。常山阴虽然在最关键的时刻,拿自己人当了肉盾,但他仍旧被我所阻,还是受到了魂爆的波及。尤其是,他还中了我的爆脑蛊。此蛊即便要不了他的性命,也会极大地遏制他的战力。可以说,狼王已经废了。”

%15
书房的角落里。边丝轩一身黑衣,站在阴影当中,声音清冷。

%16
“爆脑蛊?”东方余亮微微一愣,他还是首次听说这个蛊名。

%17
边丝轩轻笑一声,当即为他解释了此蛊的由来。

%18
“竟是这样……”东方余亮听了之后,眸子微微一亮。似松了一口气,便对边丝轩感谢道,“此次有劳影剑客出手,失了一张好底牌。”

%19
边丝轩没有说话。

%20
其实,她也暗暗感到心疼。

%21
自从她试验出爆脑蛊的用途之后,此蛊的确成了她压箱底的手段之一。很多次刺杀,都是靠着它。建立了奇功。

%22
但刺杀常山阴时,情况紧迫,她根本无法当场斩杀掉常山阴,来不及回收。在敌方包围过来前,她又必须撤退,先保全自身。

%23
“这是事前答应你的报酬。”东方余亮从空窍中唤出一只蛊虫。

%24
此蛊浑身漆黑,独角方壳,有拳头大小。给人敦实沉重之感。

%25
这是四转的叠影蛊。

%26
边丝轩的目光落向叠影蛊,不由地流露出些许热切的情绪。她虽然有多重剑影蛊,但攻势分散,遇到防御深厚的对手,战斗起来就分外艰难。

%27
若是能有叠影蛊,将多重剑影叠加到一块,就能形成攻势凌厉的至强一击。对她战力的提升。可谓幅度巨大。

%28
但很快,边丝轩又收回目光,没有接受叠影蛊。

%29
阴影中,传来她清晰冷淡。略带骄傲之意的声音:“这叠影蛊先寄放在盟主手上,待狼王死后,我再来取便是。”

%30
说完,她融入到阴影当中,消失不见。

%31
东方余亮微微一愣,只好将叠影蛊重新收回空窍。

%32
“这影剑客果真是讲信用,难怪做为魔道蛊修,却能在各大部族之间混得如鱼得水。很多正道蛊师不惜花费重金,专门请她出手。看来,她对爆脑蛊的信心十足啊……如果爆脑蛊真的能解决掉狼王,那我便少了一个心腹大患,这是最好的情形。”

%33
“但是,如果不能呢?如果爆脑蛊被常山阴成功解决呢?关键时刻,他宁愿选择抵抗魂爆,任由爆脑蛊钻入耳窍,这就说明他有一定的信心和手段,来解决掉这个麻烦……”

%34
东方余亮目光沉郁下来:“但狼王此次的确受到了魂爆的影响……现在变数又增多了,看来我得重新再推算一番。”

%35
想到这里,他移步走到书橱旁,扭转香炉顶盖,打开密道。

%36
沿着密道,他来到地下深处。

%37
这里,早就被他种下了一只地囊菌王蛊。

%38
此蛊是蛊屋的一种,里面空间狭小,但足够一人独自修行之用。最关键的是,地囊菌王蛊防御深厚,可以保证东方余亮的人身安全。

%39
进入蛊中之后,东方余亮就将入口关闭。整个地囊菌王蛊,团成一个圆球,又深入地底数丈,这才停住不动。

%40
地囊菌王蛊的内壁,柔软厚实,仿佛地毯。东方余亮直接盘坐下来,双眼缓缓闭合。

%41
他心神投入到自己空窍当中,调动五转真元,灌注到星念蛊中。

%42
他开始思考——

%43
“若是狼王常山阴解决掉了爆脑蛊,我该如何应付他?”

%44
这个问题在他脑海中刚刚产生,就在星念蛊的作用下,凝结成一个念头。

%45
普通的念头,无形无质,只能存于脑海当中。

%46
但这个念头,散发着湛蓝星光,不仅可以用肉眼瞧见,而且还能脱离脑海,探出头颅,直接飞到东方余亮的头顶上去。

%47
东方余亮很快想到:“要对付奴道蛊师,大体上有三种方法。”

%48
“第一种是王道之法,以奴道大师对付奴道大师。”第二个散发着星辉的念头,飞出东方余亮的脑海,和第一个星念飞到了一起。

%49
“第二种是霸道之法,用斩首战术,以猛将冲阵,于万军丛中硬取其性命。”第三个星念飞出,围绕着第一个星念盘旋。

%50
“第三种是诡道之法。刺杀常山阴,或者收买贿赂,或者以情动之。”第四个星念同样飞出去,和之前的星念一起纠缠,时而碰撞。但不管如何碰撞,四个星念始终都是四个,没有变化。

%51
紧接着。东方余亮又回忆:本方的军力,对方的军力,双方的粮草,常山阴的性格、动机,本方各个蛊师强者的信息,对方各大强者的资料。近期天气的预测和变化趋势,战场上地理环境,有多少山丘,有几个湖泊,周围又有多少兽群,其余各大势力介入的可能……

%52
蓬。

%53
瞬间,数以千百计的星念同时产生。然后涌出脑海,飞到东方余亮的头顶上空去。

%54
一时间,星光灿烂!

%55
东方余亮的脸色却是骤然一白,空窍中的真元海面也跟着下降一大截。

%56
他熟练地操纵着这些星念。

%57
一颗颗星念,有大有小,大的不超过大脚趾,小的也不小于小拇指。在狭小的空间中,这些星念相互碰撞。

%58
有的星念碰撞在一起。碰出三四个,甚至五六个全新的星念。

%59
有的星念则相互融合。有的星念,反而自己分化成数颗。

%60
千百计的星念,充斥整个空间,密集无比,环绕在东方余亮的身边。

%61
真元海面徐徐下降,东方余亮的心神完全投注当中。操纵着这些星念不断融合,不断碰撞,不断分化。

%62
随着他不断的努力,间或动用其他的智道蛊虫辅助。星念的数量渐渐减少。

%63
足足持续了两个时辰之后,原本成百上千的星念,只剩下八颗。

%64
但这八颗星念,每一颗都有拳头大小,星光烁烁,包含着复杂的念头。

%65
当这些星念,一一飞入东方余亮的脑海当中后,东方余亮的眼中便闪烁出缕缕智慧之光。

%66
他成功地推算出来,许多应付狼王的方法。这些方法皆是条理分明,层次清晰。

%67
若是换做寻常人,恐怕思索一两个月,也未必能将这无数繁芜杂乱的因素理清,更遑论从乱麻一团的局面中,寻找到解决的方法。

%68
但靠着智道的手段,东方余亮仅仅花费了两个时辰,就得到了答案。

%69
不过,这些答案却还并不唯一。

%70
取出元石恢复了真元,又休整了片刻之后,东方余亮又再一次推衍测算同一个问题。

%71
这一次,他只花费了一个半时辰,得到了七颗的星念。

%72
星念钻入他的脑海,他取读之后,又得到一些答案。这些答案和之前的那一批,十分相似,只在细节处,略微有些不同。

%73
东方余亮松了一口气,真正停歇下来。

%74
过了好一会儿,依靠着元石补充,他的真元才恢复巅峰状态。

%75
只是魂魄深处,还有一股虚弱疲惫之感,始终萦绕着,驱散不尽。

%76
智道的推演,不仅是催动蛊虫,消耗真元,而且还消耗魂魄的力量。推算的次数越多,运转的念头规模越庞大,推算的时间越长,魂魄就会消耗越大。

%77
虚弱是常态,若是推演难度大,魂魄会受到损伤,甚至直接消亡!

%78
当然,作为五转的智道蛊师,得到蛊仙指点的东方余亮,自然有一套完整的蛊虫组合。

%79
四转,炼精化神蛊!

%80
东方余亮的肉身,立即清瘦下去,肉体的精力在炼精化神蛊的转化之下,成为魂魄的资粮。

%81
魂魄得到大补,虚弱感很快便消散得干干净净。

%82
但是一阵强力的饥饿感,却紧接着传来。

%83
东方余亮摸摸肚皮,心中苦笑:“炼精化神蛊,虽然是我东方家的秘传蛊,对魂魄治疗效果良好,可惜治标不治本。魂道、智道关系紧密,若是我能有传说中的胆识蛊,想怎么算就怎么算。哪怕算得魂魄受伤,都能迅速复原。可惜胆识蛊,只在荡魂山上才有……”

\end{this_body}

