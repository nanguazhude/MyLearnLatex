\newsection{三尊说}    %第五十六节:三尊说

\begin{this_body}

%1
数天后。

%2
车辚辚,马萧萧。

%3
葛家众人来到月牙湖畔,依靠着湖边稀疏的马蹄树林,安营扎寨下来。

%4
到了这里,葛光等葛家高层,终于松了一口气。

%5
这里水草丰美,有大量的兽群,完全可以好好进行一场狩猎大会,补足粮草后再启程。

%6
“终于到达了这里。”方源心中感慨。

%7
当天晚上,他就借口要率领狼群狩猎,离开了葛家一族的视野。

%8
今夜无月,但是繁星点点。

%9
狼群在夜风中奔腾,遇到中小型的兽群,就一一吞食掉。

%10
狼群兴奋地发出嚎叫,这些天来它们都是半饥饿状态,今晚总算能吃个饱了。

%11
纵然有三万的狼群,方源仍旧小心翼翼地操纵狼群走向。狼群虽然规模庞大,但月牙湖畔也隐藏着危险,栖息着更加庞大的水狼群,三角犀群,以及上千头的异兽群,数以十万计,甚至百万计的虫群。

%12
当然这些虫群、兽群,都有着各自的地盘,不像草原上的兽群、虫群四处流动。

%13
月牙湖畔水草丰美,它们也不需要四处流动,寻找食物。

%14
只要方源、葛家不主动侵犯它们的地盘,它们也不会过来找麻烦。

%15
按照前世的记忆,方源沿着湖边一路往东,来到一处石林。

%16
这石林也普通得很,根根灰白、紫黑、青黄等等色彩的石柱,相互间隔,静静地矗立着。

%17
月牙湖边的石林有很多片,只是这一片比较特殊。若从有人高空俯瞰,只看灰白色的石柱,就会发现石柱依稀组成一个“盗”字。

%18
要说这片石林。来头可太大了。乃是盗天魔尊亲手布置的。

%19
当年盗天魔尊为了请长毛老祖,为其炼蛊,便和长毛老祖打赌,五场中赢了三场,长毛老祖惜败,不得不认赌服输,答应为盗天魔尊炼蛊。

%20
盗天魔尊一生都想要进入传说中的空门,便请长毛老祖炼制仙蛊——遁空蛊——就是能使人遁入空门的仙蛊。

%21
他拿出用毕生精力,钻研出来的秘方。长毛老祖看了相当兴奋,觉得秘方很好,十分正确,但是还可以改进一些地方。

%22
和盗天魔尊商讨之后,魔尊也十分高兴。大叹长毛老祖的炼道造诣要远高于自己。

%23
两仙合作,耗时二十一年,终于将“遁空蛊”炼成。

%24
但是此蛊虽然炼成,盗天魔尊也能催动,但死活进不去空门当中。

%25
魔尊心灰意冷,骄傲的长毛老祖,也大受打击。

%26
后来长毛老祖。和巨阳仙尊研究,也没有进展。遁空蛊成了长毛老祖一生中,唯一的失败作品。

%27
据说,长毛老祖临死前。也对此念念不忘。他的好友一言仙,不忍心看着老友就这样去了,便耗费五十载阳寿,为其测算。

%28
算得无数年后。历经三个大时代,会分别出现两男一女三位尊者。第一位是幽魂魔尊。第二位是乐土仙尊,第三位是大梦仙尊。而关于遁空蛊的难题,将会在大梦仙尊手中解决公主请上榻全文阅读。

%29
一言仙乃是八转智道蛊仙,精通测算,常常一言即中,因此号称一言仙。而推演出来的这道预言,就是后人皆知,鼎鼎大名的“三尊说”。

%30
果然世道变迁,不断发展,继巨阳仙尊之后,真的出现了幽魂魔尊。魔尊死后,出现的九转男性蛊仙,隶属正道,又真的号称乐土仙尊。

%31
如今乐土仙尊也已经老死,“三尊说”已经对了一大半,只剩下大梦仙尊还未出世。

%32
再说,长毛老祖得知这个推演的结果,既开心又难过。

%33
开心的是,遁空蛊的难题终于能够解决。难过的是,他却看不到那一天了。

%34
长毛老祖死后,化为地灵,只有一个执念,就是见到大梦仙尊,只要她将遁空蛊的难题解决,那么整个琅琊福地就归她所有。

%35
“当年,长毛老祖为盗天魔尊炼制遁空蛊,结果得了一个失败品,还查不出问题。大为羞愧,答应为盗天魔尊再炼九只蛊虫。炼制蛊虫的材料,都由他出。盗天魔尊后来又请长毛老祖,炼了六只仙蛊。盗天魔尊神秘失踪之前,他在五域布下自己的传承,并且和长毛老祖约定,将剩下的三次炼蛊机会,留给他的继承人。长毛老祖答应之后,双方就约定了暗号。”

%36
“然后距今十几年后,盗天魔尊的一处传承忽然开启,公之于众。传承的内容,是一个谜题,谜底直指琅琊福地。北原由此掀起了一个猜谜的热潮,无数人尝试破解和挖掘。流言飞起,却没有人成功。”

%37
“之后,马鸿运在战场上兵败,被迫流窜到月牙湖。在湖边,他遭到蓄谋已久的埋伏。慌不择路地逃到这片石林,结果意外地发现石林中的通道。通过这道秘密的通道,他进入了琅琊福地,见到地灵。对了暗号之后,琅琊地灵依约,为其炼了三只蛊虫。”

%38
“马鸿运得了这三只五转蛊,又修养完好,回到石林,大发神威,反败为胜,又席卷重来,最终第二次登上王庭之主的宝位。”

%39
方源一边回忆,一边缓步行走。

%40
然后双眼一亮,驻足在一根紫色的石柱下。

%41
这石柱也普通,但根部有一块石头,顶端平滑如凳。

%42
当年,马鸿运兵败之后,被敌军追杀,已经伤重濒死。他逃到这里,已经走投无路,伤重濒死。

%43
他坐到这个石凳上,偎依着石柱,仰头望天。

%44
忽然觉得,这石柱宛若他的爱妻容貌。

%45
他的神志近乎昏迷,用沾满血的手抚摸着石柱,深情地唤道:“怜云啊,怜云……我真后悔没有听你的劝诫。我想要见到你,想要当着你的面,对你说:我爱你,我错了……”

%46
这时。敌人已经赶到,眼看着马刀已经高举起来。

%47
但下一刻!

%48
马鸿运陡然消失,他由此进入了琅琊福地。

%49
方源坐在这个石凳上,试着将后背依靠在石柱上,然后抬眼仰望。

%50
却没发现,这石柱有什么类似女子容颜的地方。

%51
方源失笑一声:“看来这马鸿运,是睹物思人。心中思念深重,看着什么东西都能想成赵怜云。”

%52
这赵怜云也是个奇女子,日后成就智道蛊仙。是马鸿运的贤内助,帮助他谋划算计。不过现在,还只是个小小女童而已。

%53
方源取出匕首,割开一道伤口,将鲜血涂在这根紫色石柱上重生之盛世清雅。

%54
这紫色石柱。曾经被盗天魔尊暗施了神秘蛊虫。鲜血只是开启通道的第一标准,还有第二个标准,就是说出“想要”两个字。

%55
当方源刚刚说出这两个字后,他的身躯瞬间消失,视野骤然剧变。

%56
待他反应过来,发现自己已经身处在一处房间之中。

%57
房间中,丹炉飘香。龙柱金幔,鹤灯朱窗。

%58
一位仙风道骨的老者,盘坐在云床上,正在闭目冥思。

%59
他身形瘦长。白发如雪,胡须垂胸,面目似婴儿般红润,身穿一身宽大的衣袍。两袖飘飘浮动。

%60
“魔尊后人常山阴前来拜见。”方源以右手抚胸,微微一礼。

%61
“你叫常山阴?”老者微微睁开双眼。眼中闪烁着精芒,四下扫视着方源,目光宛若实质,“你的这皮囊有些意思。嗯……是用了人皮,还有愧梅子、秋声草。嗯,还有丹火蛊,药力蛊。还有一些……”

%62
地灵深深皱着眉头,用手抚摸胡须,神情迟疑。

%63
他竟然一眼就看穿了方源的伪装,甚至仅仅在观察之下,就反推出人皮蛊秘方的大体内容。

%64
“小子,你的这个蛊虫秘方换不换?我用同等的蛊虫,作为交换。或者你也可以挑选一个同等的蛊虫秘方。哦,对了!暗号,对暗号!”说了一半,地灵才恍然大悟,一拍额头,想起来还有暗号。

%65
方源耸耸肩:“魔尊当年约定的暗号么?哈,那就是没有暗号。至于人皮蛊的秘方,我现在还不能换。”

%66
“你居然不换?为什么!”地灵大怒,空气中降下沉重的压力,方源动弹不得,整个骨骼都被压得嘎吱作响。

%67
但他冷笑一声,不以为意:“我不换当然有我不换的理由。你却不需要知道!”

%68
地灵森然一笑:“呵呵,你就算不换,那我把你囚禁在此,把你的皮扒了,几下研究,也能反推出这个人皮蛊的秘方。”

%69
“不,你不会动手的。”方源语气笃定,“我是魔尊的后人,在完成当年的三蛊之约前,你不能拿我怎样。”

%70
按照地球思维的理解,地灵就相当于一段智能程序。

%71
只是长毛老祖所化的这个地灵,智能颇高,擅长恐吓别人。五百年前世,马鸿运就是被他吓唬得一愣一愣的,结果丧失了炼制仙蛊的宝贵机会,只炼成了三只五转蛊。事后,马鸿运懊悔不已。成就蛊仙之后,他每每想到这事,更常常扼腕而叹。

%72
“你!”地灵吹胡子瞪眼,眼中仿佛喷着怒火,做出一副怒而杀人的姿态。

%73
但方源不为所动。

%74
地灵老者瞪了半天,忽然像泄了气的皮球,整个人委顿下来。

%75
然后他哭丧着脸,下了云床,走到方源的面前,拉着他的衣袖口,软语哀求道:“少年郎,你就行行好,换了这道秘方吧!”

%76
方源无语!

%77
这,这什么情况?变脸也太快了吧!

%78
(ps:今天要去吃酒,时间上来不及了,只有两更。明天的更新,也可能要晚点。计划赶不上变化啊。总的来说,这个月是疯狂的月份,我居然坚持下来了!的确也有些不足的方面,以后慢慢提升吧。总之很感谢诸君的体谅理解,以及一直都相当给力的支持!)

\end{this_body}

