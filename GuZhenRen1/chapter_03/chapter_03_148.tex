\newsection{王庭圣宫}    %第一百四十八节 王庭圣宫

\begin{this_body}



%1
风在耳边呼啸。

%2
金色的天空,辉煌绝伦。

%3
天青狼群在空中漫步,方源骑在天青狼王的背上,大风刮得他的头发往后飘拂。

%4
他目光沉凝,面露思索之色。这几天来,他一直在参悟地丘传承的密语,可惜毫无进展。

%5
一座辉煌的宫殿,渐渐地从视野的尽头,冒出圆形平顶。

%6
察觉到这一点,方源的目光当即转移过去,将脑海中繁芜杂乱的思绪放置一旁。

%7
他的脸上,闪过一丝感慨之色。

%8
圣宫,终于到了!

%9
随着方源的接近,一座巍峨的宫殿,慢慢展露出它的全貌。

%10
它分有八层,高达八百余丈。第一层,也就是底层,占地面积最广。第二层居于第一层之上,第三层以此类推。

%11
每一层,都有城墙圈成一个个同心圆。

%12
雪白的城墙厚达三丈,墙体连接一体,毫无缝隙。城墙上,每隔一段距离,都有七彩塔楼。每一座塔楼,颜色不一,赤橙黄绿青蓝紫,相互间隔。

%13
层层堆叠之后,整个圣宫,看起来如同一座高耸的山峰。

%14
随着方源急速接近,整个圣宫宛若从地下拔升而起,横空出世,气贯苍穹!

%15
壮栽,圣宫!

%16
即便方源见多识广,此刻看见,也心生赞叹。

%17
“天空中飞来的是什么?”

%18
“狼王到了!”

%19
“和情报上说的一样,那应该就是天青狼群了。”

%20
方源的出现,也引起了圣宫中,蛊师们的注意。

%21
此刻,圣宫中已经驻扎了上万人。这些人运气好,进入福地之后,位置就在圣宫的附近。因此比方源早到。

%22
黑楼兰早有交代,负责接引的蛊师们,很快就反应过来。

%23
当方源徐徐降落下来后。早就有人站在第一层的巨大城门口。

%24
方源胯下的万狼王刚一落地,接引蛊师便迎了上去,拜见道:“狼王大人,您回来了。小的三生有幸,专门负责引您入城。你的居所在第八层,房间也早就为您打扫好了。”

%25
“嗯,不急。先带我瞻仰一番吧。”方源淡淡点头,下了天青狼。处于对巨阳仙尊的尊重。圣宫中一律不得骑乘坐骑,只能徒步行走。

%26
“遵命大人,这是小的荣幸。”

%27
跟着接引蛊师,走进圣宫。

%28
圣宫中,有大量的亭台轩榭,宫殿庭院。从外看,飞檐外挑,屋角翘起,铜瓦鎏金,美轮美奂。

%29
各个建筑分部合筑、层层套接。有的地方。广阔敞亮,格致宏大。而有的地方,则是廊道交错,殿堂杂陈,空间曲折莫测。

%30
而在亭台堂殿的内部。不管是立柱还是大梁,都布满了鲜艳的彩画和雕饰,充满奢华富贵的气象。

%31
“大人,这里是怡乐宫。伟大的巨阳先祖,如今居于圣宫,每天都会在这个宫殿中举办规模盛大的音乐庆典。据史料记载,每一场庆典中,都会有大量的妃嫔争相献舞,以期夺得巨阳先祖的青睐。”

%32
“大人,这里是春汤殿,拥有北原最大的温泉。巨阳先祖每隔七天,都会邀请成百上千的妃嫔,一起来这里泡水、嬉戏。”

%33
“这里便是飘香院,巨阳先祖曾经将神话中的‘酒池’、‘肉林’,移到这里。每天早晨,肉林中都会产生各种口味,无比鲜美的肉果。而每到晚上,酒池中则会生出各样香醇的美酒。”

%34
接引蛊师每到一处,都介绍一番,口才不错。

%35
方源安步当车,四处浏览,觉得蛮有意思。

%36
到了圣宫第四层,接引蛊师将方源引进正殿。

%37
“大人,这里是圣宫八大正宫之一的画宫。巨阳先祖多才多艺,尤其擅长美人画。这宫中的壁画,都是他一人所书。请这边进。”

%38
接引蛊师打开宫殿的侧门,请方源进去。

%39
圣宫的八大正宫,都有正门,不过都只能巨阳仙尊一个人进出。巨阳仙尊虽然已经逝去,但是这个规矩还是流传下来。后人遵守这个规矩,也是表达对巨阳仙尊的敬畏和爱戴之意。

%40
进去了宫殿,顿时宏大的壁画,就充斥方源的视野当中。

%41
画宫中空无一物,只有四周的巨大墙壁。这些墙壁上,绘有各式各样的美人,或妩媚妖娆,或清纯如水,或露齿欢笑,或低头凝思。神态均动人多姿,足足有八万有余!

%42
“能够记录在画宫之中的女子,都是巨阳先祖一段时期的宠儿。在当时,能够被仙尊亲笔题画,是天下女子的无上荣耀。巨阳先祖的妃嫔有无数,能记录在这里的,都是佼佼者,如今算是芳容永存了。”

%43
接引蛊师说到这里,一脸的感怀之色。

%44
方源没有说话,只是默看,同时暗道:“芳容永存到不至于。至少五百年前世,王庭福地就被中洲蛊仙攻破摧毁,圣宫也成了绝响。唉,真正的永生,即便强如仙尊也无法做到啊……”

%45
旁人来到这里,无不被圣宫的奢华富贵,堂皇锦绣之气摄住,就算不痴迷神往,心中也会生出敬畏之情。

%46
但方源却从这辉煌中,品味出一丝衰败和枯朽。

%47
没有永生,强如仙尊又如何呢?

%48
千古风流如巨阳,如今却已经烟消云散。留下的痕迹,像是印证,但在方源感觉,这种印证充满了失败的意味,带着淡淡的嘲讽和悲伤。

%49
游兴已尽。

%50
“走吧,直接带我去第八层的住处去。”方源叹息一声,吩咐道。

%51
接引蛊师连忙收起脸上的痴迷之色,迟疑道:“可是大人,圣宫盛景无数,我们才刚刚开始而已!除了这些之外,还有美妇宫,幼女宫,妩媚殿,纯真殿。还有异香宫,当年专门住着女性异人,甚至连毛民都有呢。还有玉像宫,专门用软玉制造美人像。供先祖享用。”

%52
接引蛊师心中焦急,他说的这些地方,单凭他自己的身份,还不能进去。

%53
他很想借着这个机会,大饱眼福。

%54
但方源没有满足他这个小小的愿望。

%55
巨阳仙尊晚年,就很少下界,到圣宫中居住了。而是居于长生天中。深居简出。

%56
而北原每年,都会为他选出大量的女子。充实圣宫。

%57
巨阳仙尊最后一次,来到圣宫也没有临幸这些女子。他建立了八十八角真阳楼,订下王庭之争的规矩之后,就彻底鸿飞冥冥,自此消失在世人的眼中。

%58
圣宫因此凋零,那些美貌如花的可怜女子,仿佛是金丝雀被关在笼子里。

%59
王庭福地虽大,但没有自由,再大的地方也是牢笼。

%60
最终,她们一一老实在这里。空耗了青春。她们无法逃出去,也没有人胆敢去搭救她们。

%61
在巨阳仙尊的伟大荣光之下,埋葬了不知多少女子的痛苦、哀怨、悲切。

%62
而在方源眼中,圣宫的价值并不高。

%63
它只是巨阳先祖的遗迹而已,没有蛊师胆敢在这里留下传承。而当年遗留下来的贵重物品。也早就被历代的蛊师们收刮空了。即便是后来中洲蛊仙,一齐出手查探,也一无所获。

%64
圣宫中唯一有价值的地方,也是整个王庭福地,不,更准确的说,是整个北原最有价值的地方。

%65
那便是八层的顶端——八十八角真阳楼!

%66
由巨阳仙尊提议,长毛老祖亲手炼制出来的蛊屋。

%67
八转仙蛊!

%68
不过现在,还不到时机。

%69
第八层的顶端,只是一片空白。只有当十年暴风雪彻底爆发,八十八角真阳楼才会渐渐出现。

%70
接下来的日子,方源便深居浅出,一边继续修行,一边等待真阳楼的开启。

%71
天青狼群,交给其他人去打理,无需方源挂怀。

%72
这期间,黑楼兰遣人唤他,透露出要招揽他的意向。

%73
加入黑家,成为外姓家老?

%74
面对这个提议,方源说要考虑考虑,流露出意动之色。但事实上,他是绝不会这样做的。

%75
人皮蛊能将他伪装成狼王,但到底是凡蛊,面对仙蛊的探查,极可能就会露馅。

%76
招揽外姓家老,是北原地区超级势力的惯常把戏。如此一来,即可将魔道蛊仙转为正道实力,对各大黄金部族把持北原大局极有帮助。

%77
而对于星鹫峰上的事情,黑楼兰只字未提。倒是有一股流言蜚语渐渐兴起,说狼王如何霸道,对方源在星鹫峰上的表现,如何恶劣,如何恃强凌弱,描绘得夸大三分,偏偏又显得真实可信。

%78
方源心中冷笑,这明显是有人在幕后推手,败坏他的名誉。

%79
“潘平极有可能,孙湿寒等人也有动机,甚至还有黑楼兰。不过就算我名誉扫地,又能如何?”

%80
巨阳仙尊当年,是名声恶劣,四处寻花惹草的魔道蛊师。现在呢?是人人敬仰的仙尊!

%81
他广纳后宫,踩踏了多少少女,破坏了多少幸福?

%82
但即便到今天,也没有人来公然指责他。

%83
世间一切,力量才是根本。

%84
在地球上,人言可畏,三人成虎。那是因为,人人都是凡夫俗体,世界规则不同,不能够让个体拥有超越群体的力量。

%85
但是在这里,却非如此。

%86
这也正是方源喜爱这个世界的原因之一。

%87
大半个月后,金色的天空,忽然如水般微微晃动起来。

%88
整片大地都开始微微震荡。

%89
凭空生风,一团光芒刺目如日,陡然盛放在圣宫之巅。

%90
光芒持续了三盏茶的功夫,渐渐消弭。而原本空无一物的地方,显现出一座塔楼。

%91
八十八角真阳楼!

\end{this_body}

