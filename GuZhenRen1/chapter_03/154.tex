\newsection{三等通关可为上?}    %第一百五十二节 三等通关可为上?

\begin{this_body}

方源只觉得浑身一紧,在撞进塔楼的瞬间,一股压力压迫他的身心。

但旋即,压力一空。

他彻底进入真阳楼内,眼前是一片湖。

天空湛蓝,湖水波光粼粼,周围则是朦胧的青山墨影。

手中的来客令,化为一滩冰凉的铁水,顺着他的手指缝洒落下去。

来客令只能用一次。

方源甩甩手,将铁水完全甩干净。

他环顾四周,发现自己置身在湖中心的小岛上。身边站着一人,便是黑楼兰。

“这里就是第五十四关卡。”黑楼兰没有看向方源,而是直接望着前方,“你看,那就是山阴老弟你要突破的难题了。”

方源顺着他的目光望去,只见不远处,就有另一座小岛。

小岛上,栖息着一群水蛇狮。

这些水蛇狮,长着一身亮蓝色的皮毛,光滑无比。四肢并非利爪,而是蛙蹼。它们的尾巴,是一条条剧毒的蛇,或盘绕在水狮的背部,或者挺立蛇躯,吐露猩红的蛇信。

不需要黑楼兰多做介绍,于此同时,就有一股冥冥中的信息,直接注入到方源的脑海里面――

“利用岛上的刺猬鱼,冲破水蛇狮群的防护,占领对面的岛屿。”

方源收回目光,再看自己这座小岛的边缘。

在岛边碧蓝的海水当中,他果然看见一只只刺猬鱼的身影。

这时,陆续有黑家族长们,一个个进来,站到了黑楼兰的身边。

“不管是水蛇狮,还是刺猬鱼,都是中古时代的野物。如今只有在东海深处,才能得见呢。”一位家老感怀道。

“要我说,这道关卡极不公平。水蛇狮本来就强,一只水蛇狮就堪比五六只刺猬鱼。但我们这边的刺猬鱼数量,只不过只有狮群的一倍左右。”一位黑家族人盯着眼前的小岛。为方源说明难度。

“山阴老弟无须担忧,今次过来,以试手为主。”黑楼兰拍拍方源的肩膀道。

毕竟,方源驾驭的是狼群。而此次却是鱼群,还是当今稀缺,在北原更是绝迹的刺猬鱼。

黑旗胜已经失败了八九次,最好的战绩,不过只拼杀了三成的水蛇狮群。这让黑家蛊师们都深刻地意识到了此关的难度。

常山阴虽然是奴道大师,但这里可是巨阳仙尊设下的真阳楼啊。

方源凝望片刻,皱起眉头。

他暗暗估算。单凭自己的实力。操纵这些刺猬鱼。绞杀了水蛇狮群却是不难。甚至可以说,是板上钉钉的事情。

但通关看似简单,其实却有极大讲究。

当初,巨阳仙尊设立八十八角真阳楼。为的是给后代留下传承,奖励才华出众的后辈。因此,真阳楼中每道关卡的奖励,都有上中下三等。

下等通关,获得最少的奖励,直接进入下一关卡。

中等通关,获得下等一倍有余的奖励,进入下一关卡时,还能获得关于此关的提示信息。

而上等通关。不仅奖励在中等的基础上,再翻一倍,而且能将闯关者传进真阳楼深处,名为秘藏阁的地方。

这秘藏阁中,拥有无数奇珍异宝。甚至还有巨阳仙尊的数道真仙传承。

但秘藏阁中的这些东西,不能随手取走,须得闯关者交换。

交换蛊虫可以,交换蛊方也行,甚至哪怕是自身的修行体悟经验,也能当做交换的物品。

当闯关者完成交换之后,这才会进入到下一关卡。

黑楼兰只需要下等通关,能令他进入第五十五关,他就满足了。但方源需求不同。

他需要上等通关,进入秘藏阁。只有在秘藏阁中,他才能孤身一人,动用自己早就准备好的手段,获取最大利益!

“下等通关,对我而言易如反掌。但中等通关、上等通关的标准是什么,我却不知道了。如今只有勉力一试!”

念及于此,方源深吸一口气,对黑楼兰点点头,示意可以开始。

黑楼兰便掏出一张令牌,对准天空一晃。

这令牌和来客令不同,乃是楼主令。黑楼兰第一个进入王庭福地时,便凭空得来此令,代表着他的尊贵身份。

楼主令一晃之下,空中波纹荡漾,闪现出数十只驭鱼蛊,一二三转皆有。

当方源接过这些蛊虫,轻松炼化之时,其他的人都被一股无形的温柔力量禁锢住,不能从旁支援。

唯有黑楼兰还可以说话,他提点道:“山阴老弟,你要注意时间。此关你只有一盏茶的时间可用。”

方源点点头,信手一洒。手中的驭鱼蛊,就化为数十道奇光飞出。

众人见方源这样随意,纷纷在心中惊呼。

若换做黑旗胜,势必要一只只点将,慎重无比。像这种驭鱼蛊种下之时,蛊师必定受到鱼群的反抗,稍不留意,往往就会驭兽失败。甚至还可能导致,魂魄上的反噬。

“狼王是否过于托大了?”

“我还从未见过这样驭兽的!”

“要糟……”

众人的心都提起来。

但旋即,他们双眼微瞪,只见鱼群被种下蛊虫之后,由静即动,纷纷朝岛外游去。仿佛本来就为方源所属,没有一个驭兽失败的例子。

水蛇狮群也被这事态惊动,纷纷一改懒散,从坐卧变为直立,狮头昂首低吼,蛇头长牙嘶鸣。

许多水蛇狮群纷纷跃入水中,建立水下防线。

而方源则傲然挺立,背负双手,身躯晃都不晃,显现出来的奴道造诣以及魂魄底蕴,叫众人无不暗赞。

“天呐,他居然做到了!”

“单单这一幕,就叫我大开眼界了。”

“狼王果然不愧是奴道大师,就是和旁人不同。”

“大师就是大师,兴许这次能一举功成,也说不定啊。”

众人的眼中,纷纷闪烁出期盼的神色。就连黑楼兰也一脸希冀地望着。

但鱼群并未如众人所想,直扑对面的水蛇狮群,而是朝四面八方分散游走。不断逡巡。

“这……狼王是想干什么?”

“狼王稳妥,看来他是在熟悉鱼群的习性!”

但时间渐渐过去,鱼群仍旧畅游。众人望眼欲穿,却始终不见鱼群和狮群的交战。

这下子,就连黑楼兰都隐隐急躁起来,催促道:“山阴老弟,抓紧时间啊。”

“不急。”方源面色平和,悠然回道。

水蛇狮群见鱼群久久没有来犯,一些水蛇狮子又从水中钻出来,爬到岛上去。水下防线因此渐渐松垮下来。

过了片刻。黑楼兰再催:“山阴老弟。如今可是已经过了半盏茶的功夫了!”

“不忙。”方源摆摆手。眼皮微垂,似乎有些睡意朦胧。

更多的水蛇狮群,重新上岸。老迈的狮王甚至卧在地上,闭眼假寐。

众人大失所望。纷纷在心中咒骂。

“这狼王雷声大雨点小,枉费我之前那么看好他!”

“大师又能如何,这里可是真阳楼啊……”

“看来这次,常山阴主要是想熟悉鱼群,积累经验,等到下次全力冲击!”

眼看着时限将至,众人残留的一点希望也被耗尽。

“可惜一块来客令,就被这样损耗掉了。”

“还是想想今天回家该吃什么好?”

“这次闯关失败,不晓得族长大人该如何对付常山阴呢?”

就在众人心神散乱之时。只听方源哈哈一笑,鱼群猛然发动冲锋,从四面八方冲向水蛇狮群。

“果然来了!”黑楼兰眼中精芒爆闪。

他心中早有猜测:“常山阴既然想将真正的大动作留到下一次,那么这次机会,他花费如此功夫来熟悉和训练鱼群。肯定还要试探狮群。洞悉它们的虚实和战力!”

鱼群的忽然冲锋,着实将水蛇狮群打了个措手不及。

刺猬鱼群像是一批饥饿的鲨鱼,以迅雷不及掩耳之势,将水中少量的水蛇狮群尽数消灭。

“原来如此!常山阴之前一边训练鱼群,一边麻痹狮群。”

“这次突击的效果,已经足足消灭三成的狮群,大师就是大师啊!”

“叹为观止的驭兽造诣,黑旗胜和常山阴相比,简直就是婴儿和壮汉。”

众人无不瞪大双眼,为方源顷刻间创造的战绩惊异。

吼!

见到自家儿郎被屠戮,老狮王怒气勃发,大吼一声,亲率狮群,跃入水中,对鱼群展开报复性的攻杀。

叫众人奇怪的是,方源一扫之前的强硬作风,在他的操纵下,鱼群一退再退。

狮群穷追不舍,行进到某处时,忽然队伍发生了混乱。

“怎么回事?”众人疑惑。

“原来这里有一处暗流漩涡!”黑楼兰一声低呼。

旋即,家老们眼睛发出了亮光:“我知道了!看来狼王大人广布鱼群,不仅是为了训练,还有调查地形的缘故!”

“对啊!兽群交战,如两军交锋。不仅要考虑敌我双方,还有顾虑地形地利的。”很多蛊师看到这里,差点要一拍大腿,张口叫喊起来。

水蛇狮群被暗涡困住,鱼群立即调转枪头,回扑过来。

狮群体型庞大,暗流对其影响甚深。但鱼群体型娇小,暗涡对鱼群的影响微乎其微。

一场精妙无双的攻伐战,在众人的眼前上演。

强大的水蛇狮群,宛若纸糊般脆弱不堪。而鱼群在方源的一手操纵之下,宛若一支百战百胜的精兵,配合默契,进退有据,分割包围,迅速蚕食。

鱼群时而骤起聚集,硬冲猛攻。时而分散四逸,使得狮群的反击徒劳无功。

“这简直像是一场戏耍!”

“好生厉害,好生厉害!!仅仅用了十几个呼吸的功夫,狼王就大获全胜了!”

“此关通了,通了!”

黑家众人无不惊喜交加,看向方源的目光,更是透出敬畏、叹服、忌惮种种。

“好,好一个狼王!”黑楼兰抚掌大笑。

方源亦是朗笑一声,皆因完全绞杀了狮群的那一刻,就有信息传入他的脑海――上等评价!

下一刻,方源身形蓦地消失在原地。

“怎么回事?”黑家众人看到如此惊变,眼珠子差点都瞪掉下来。

“竟是上等通关!!”唯有黑楼兰在心底惊呼。

------------

\end{this_body}

