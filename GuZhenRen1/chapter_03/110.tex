\newsection{东方余亮的后手}    %第一百一十节:东方余亮的后手

\begin{this_body}

%1
短短片刻功夫,东方大军便损失惨重,撤退的途中丢下大量的尸体。

%2
“逃啊!”

%3
“快跑,再不跑就来不及了!!”

%4
令东方余亮最担心的情况出现了,东方盟军出现了大溃败。将士们毫无斗志,一心逃跑。黑家大军看出便宜,一哄而上,展开冷酷的残杀。

%5
东方余亮咬紧牙关,连忙调动四转蛊师强者一齐出动,压住阵脚。

%6
强者的回头痛击,大大的遏制了黑家大军刚刚展开的追杀之势。但好景不长,随着黑家四转蛊师强者的出击,东方家的蛊师强者被一一拖住。

%7
东方余亮再次出手,但也被黑楼兰牵制。

%8
看着己方将士,追赶上敌军,展开冷酷的大追杀,方源则收手,将狼群归拢在身边。

%9
他的这个举动,立即赢得了身边的黑旗军大统领的好感。

%10
在他看来,狼王有收获战功的良机,但偏偏将到手的战绩让给了其他人。大统领开口,对方源很是恭维了几句。

%11
方源心知对方表达善意的意图,黑绣衣原本担当着方源的护卫,却临阵脱逃,若非水像蛊,方源早已身亡。

%12
但方源根本从未指望过别人的护卫,他向来只靠自己,再者黑绣衣乃是黑楼兰的爱将,凭此发难,也无法对黑绣衣怎样。就算能怎样,方源也不愿意看到这样的无谓内耗。

%13
当即,他随口敷衍几句。暗示大统领,此事他不会放在心上。

%14
大统领这才松了一口气。心中暗觉:狼王虽然高傲,但着实大度,的确非常人。

%15
终于,东方大军疯狂逃窜进了第一道防线。

%16
他们为此,付出了惨重的代价。

%17
整个大军中,有两成死在之前的大战中,有五成死在蛊师和狼群的追杀中。剩下三成的残兵,逃进了防线。

%18
四转强者中。也有两人,因为拦截黑家大军,而因此丧命。

%19
黑家大军并未停歇,而是冲杀上去,却被城墙上的羽箭精兵一通狂射,丢下数百人的尸体,狼狈逃回。

%20
这是东方家精心培养的队伍。投入的心血和资源,丝毫不弱于黑旗精兵。

%21
黑家一方的蛊师们,又接着冲了三次,皆被羽箭精兵射退。黑楼兰调动麾下的四转蛊师冲了上去,但被东方余亮设计,反而害了三人性命。

%22
“盟主大人。对方凭险而守,占有地利。我方鏖战许久,真元不足,难以再战之力,不如退去。再做打算。”众强者回归王帐,狈君子孙湿寒便建议道。

%23
一旁。方源皱起眉头,他有前世记忆,知道东方余亮最擅长处理残局,擅长以弱胜强。拖得越久,他收集到越多的情报,对黑家就越不利。

%24
此次虽然有了方源的插手,导致黑家比前世占据更大优势,但东方余亮却战力无损,是个巨大威胁。对付东方余亮,最佳的途径便是一击必杀,不给他阴谋算计的机会。

%25
黑楼兰听了狈君子的建议,沉吟不语。他之前就和东方余亮打过交道,熟知东方余亮的手段,此时却是不愿放过眼前的良机,于是他将目光投向方源。

%26
方源傲然一笑,对黑楼兰道:“黑家族长放心,只需一刻功夫,担保破掉这层防线。”

%27
众人为之纷纷侧目,狈君子冷笑一声,觉得方源夸大其词。

%28
黑楼兰则大喜:“就看狼王的手段了。”

%29
群狼再次蜂拥而出,汇集在一起,对东方部族的第一道防线,展开绵绵不绝的冲锋。

%30
东方部族严防死守,须臾功夫,城墙上便倒下密密麻麻的狼尸。

%31
黑家将士都看得动容,方源的打法不计牺牲,简直是拿狼群上去送死。

%32
尤其是,羽箭精兵最擅长远程打击,站在城墙上大出风头。四转强者们则担当起救火员的角色,四处支援,但凡有一处地方出现险情,他们便轮番出手,化险为夷。

%33
东方部族的防线,简直固若金汤。

%34
方源冷笑一声,不断催动狼嚎蛊,冲势轮番变化,让人目不暇接,每一次变化,都带给防线巨大的威胁。

%35
高强的烈度,让狼群牺牲巨大,短短功夫,就死了二十八万的野狼!

%36
方源面无表情,站在双头犀牛的背上,遥遥指挥着。王帐众人,看向他的目光,却悄然起了变化。

%37
狼王的狠辣无情,让人不禁心生忌惮。

%38
如此磅礴凶猛的狼潮,更叫他们生出个人渺小之感。

%39
“狼王常山阴,你好狠的心肠,是要赶尽杀绝么!”东方余亮面沉如水,狼群的大量牺牲,换来的是东方大军真元的剧烈消耗。

%40
他后悔极了,早知道如此,他宁愿舍弃黑楼兰,也要取了常山阴的命!

%41
终于,东方部族的蛊师们支撑不住了,他们来不及恢复真元,狼群的攻势让他们疲于应对,没有时间喘息。

%42
“撤退!”无奈之下,东方余亮只好下达了这个命令。

%43
正如方源所言,一刻之后,东方部族的第一道防线被冲破。东方余亮留下一批伤残蛊师断后,率领剩下的残军,以最快的速度向第二道防线撤退。

%44
“狼群疲惫不堪,不适合再追杀。”方源撤下狼群,让开道路,让黑家大军有机会再出动。

%45
这一举动,为他赢来了几乎所有人的好感。

%46
“我此战受伤不轻,你们追赶上去,尽力追杀,但要小心,东方小儿肯定有所布置安排。”黑楼兰稳坐王帐未动,派遣浩激流、潘平、汪家族长等人,前去追杀。

%47
众将激动地跨过残破的防线,追杀过去。

%48
但追击大军。刚刚要过防线,陡然间一记猛烈的爆炸响起。

%49
轰的一声。如雷霆炸响,顷刻间将数十位蛊师炸上了天,落到地面上的,只是一堆烂肉以及断臂残肢。

%50
轰轰轰……

%51
紧接着,大量的爆炸,接连发生,贯穿整个防线。

%52
追击的大军,顷刻间伤亡惨重。陷入混乱的当中。

%53
“是焦雷土豆蛊!防线的地下,被东方余亮埋下了大量的焦雷土豆蛊!”侦察蛊师返程,来到王帐前汇报。

%54
“我已经看到了!”黑楼兰脸色极为阴沉,摆手命侦察蛊师退下。

%55
他并不笨,立即明白是遭到了东方余亮的算计。

%56
埋设如此巨量的焦雷土豆蛊,需要大量的时间。这个时间,其实是黑楼兰给的。

%57
开战之前。东方余亮主动递出挑战书,故意扬言求战,结果被狈君子觐言,黑楼兰拖了几天,直到后军汇集才开战。这就给了东方余亮机会。

%58
爆炸连绵不绝,带给黑家大军不小的伤亡。最关键的是。追击的脚步因此遏制,黑家大军只能坐看东方残部安然撤退。

%59
焦雷土豆蛊,虽然只是二转蛊,威力有限,但架不住数目巨大。

%60
除此之外。还有不少三转的闷雷土豆蛊,以及少量的四转炸雷土豆蛊。

%61
蛊师们就算撑起防御蛊。保住了性命,真元也随之大量损耗。掌握飞行蛊虫的蛊师,毕竟是少数。这些人就算追上去,势单力薄,反而是给东方残部斩杀自己的机会。

%62
怀着兴奋激昂,立功迫切的蛊师们,最终灰头土脸,一身伤残,无奈地回归大部队。

%63
“今日已然大胜,东方小儿不过苟延残喘。接下来再杀他个痛快也不迟!”黑楼兰安抚了几句,便开始主持战后的工作。

%64
打扫战场,治疗伤员,整理战功,都是耗费精力和时间的繁琐事情。

%65
方源自然不愿将宝贵的时间,浪费在这里。他随意找了个借口,便离开了王帐,回到自己的大蜥屋蛊中去继续苦修。

%66
此战,他是大功臣,虽然只出手几次,但次次关键。凭他的战绩和实力地位,狈君子也得闭嘴,更没有人敢说什么闲话。

%67
“没有炸伤狼王么……”东方余亮时刻关注着战况,听到侦察蛊师的汇报后,他的心中颇有遗憾。

%68
此战他将方源列为头号大敌,威胁程度还在黑楼兰之上。

%69
常山阴太狡诈狠辣了,根本不顾及葛家的生死。虽然只是四转巅峰,但比黑楼兰要难对付得多。

%70
东方余亮原本估算着,常山阴催动狼群展开追杀的几率最大。

%71
但他精心布置的陷阱,没有坑杀一头野狼,反而杀伤了黑家大军中大量的蛊师。

%72
蛊师的性命,可比野狼珍贵多了。但东方余亮却高兴不起来。

%73
对于擅长算计的智道蛊师来讲,一个强大的敌人,并不难对付。但狼王纵然强大,却不自恃强大,冷静到冷酷的对手,就相当棘手了。

%74
正是因为方源的几次出手,导致东方大军从略微失利,变成大溃败。两方由此拉开差距,黑家占据明显优势,而东方大军则陷入下风,局势糜烂。

%75
看到这场大战告一段落,逆雨福地中,蛊仙东方长凡收回目光,伸手一揽,将石桌中央的烟气收入袖中。

%76
他无需推算,便知此战之后,东方余亮已经陷入绝对下风。除非黑楼兰犯下重大失误,否则进军王庭的希望已经基本渺茫。

%77
不过整场作战,东方余亮表现得可圈可点,以弱势军力抗衡,一度打成僵局。可以说,是充分利用了手中的力量。

%78
“尤其是他有意地保护了本族力量,羽箭精兵一个不失,这就是对家族的忠心。接下来,就是考验他对残局的处理了。”东方长凡他缓缓闭上双眼,对这点最为满意。

%79
别族伤亡惨重,那是他们的事情,只要东方部族损失不大就行了。

%80
王庭之争,本身便是巨阳仙尊当年设下的局,目的之一就是保护血脉后裔,削弱他族!

\end{this_body}

