\newsection{智道}    %第六十二节:智道

\begin{this_body}

虽然买下了十万只星萤虫,但交易远没有结束。 />

方源还需要解决这些虫子的喂养问题。

“星萤虫以星屑草为食,我这里就有大量的星屑草籽,可以卖给你。”万象星君最先传达过来神念碎空战神全文阅读。

“我需要草籽,同样也要你们的培育心得。先看宝光强弱罢。”方源通过小狐仙,向万象星君、摇光仙子以及帝渊三人,传达了神念。

很快,三份文书进入宝黄天,各种绽放着宝光。

这次不是万象星君,反而是摇光仙子的文书宝光最盛,足有六尺。

“我最爱栽草,星萤虫反而是为了星屑草的繁衍,才故意迁移过来的一小群。选我这份文书,绝对不会让你吃亏的。”摇光仙子传来神念。

方源沉思着。

原本他已经遗忘,毕竟前世他没用过,也用不着星门蛊。

现在经过琅琊地灵、通天蛊、万象星君这些人的接连提示,他的脑海中关于这块的记忆,是越来越清晰了。

记忆中,星门蛊一出,有关星萤蛊的一切都成了机密,绝不会有人交易出来。

“琅琊地灵得了星门蛊秘方,说不定就会炼成星门蛊,放到宝黄天来贩卖。一旦星门蛊放到宝黄天中,再想要获得这些机密,就困难无比了。”

想到这里,方源便做了决定,让小狐仙传出神念:“你们这三份,我都要了,就用仙元石结算。”

仙元石的购买力,相当强大。

一块仙元石分开三份,分别买下三份文书,还剩下四分之一块。

方源打开这三份文书。只见上面分别写着许多事项――

“星屑草只能种在天上,生长在云朵上。因此常常选用上佳的云土栽种……”

“星屑草喜阴厌阳,过分的阳光照耀,会晒死星屑草。但不能完全缺乏光线,需要的量大约是……”

“星屑草若是有星萤虫生存,帮助散播种籽,会蔓延更快,长势更好……”

三份文书,帝渊的最简略可靠。万象星君的别有创新,摇光仙子的最详实繁芜。方源取长补短,相互参照结合,很快就将一切都牢记于心,顷刻间成了培植星屑草的专家。

他心中暗笑:“有了这些。我就能成为日后最大的星萤蛊的卖家。将来星门蛊盛行起来,必能赚上一笔!”

“该买酒了。”方源对极品美酒念念不忘。

他已经收集到了三种极品美酒,只剩下最后一种。

宝黄天中,当然也有极品美酒贩卖。主要是蛊仙用来炼蛊,其次才是享受生活时的上佳饮品。

但就在方源要购买极品美酒的时候,整个宝黄天陡然沸腾了,无数道神念疯狂地激荡着。

“仙蛊。有人在卖仙蛊!”小狐仙惊叫一声。

镜面上画面一闪,出现一只仙蛊的图景。

“神游蛊!”方源瞳孔微微一缩。

这只在卖的仙蛊,赫然就是他所需要的神游蛊。有人先他一步,抢得了神游蛊的所有权。

“呵。终究还是晚了一步么。”方源楞了一下,随后轻笑出声。

他要炼制第二空窍蛊,就必须要神游蛊。但在三王福地中,局势所逼。将神游蛊炼成了定仙游。

按照《人祖传》中记载,要饮用四种极品美酒抗日之铁血军魂。才能获得神游蛊。方源一直在收集美酒,但到了如今,却是有人先他一步。

“没有神游蛊的话,那么第二空窍蛊就差最后一步,不能完成了。好在还有数十年的时间。”

“话说回来,还是我在三叉山上炼蛊过于高调,导致南疆人尽皆知。呵呵呵,所以被人抢先一步,也不奇怪。”

方源耸耸肩,他对这个事情,其实早有预料。

他用神游蛊炼成了定仙游的过程,很多蛊师都目睹了。此事肯定引起轩然大波,南疆的蛊仙不是呆子,怎么可能不去行动?

只是方源之前,还是寄希望于蛊仙之间的内斗,相互之间的掣肘,因此一直在努力收集。

人的一生中总有些事情,虽然希望不大,但是努力奋斗过了,才不会后悔。不去奋斗努力,那真的就连一点希望都有。

“世间之事,时常都是不尽人意,我早已经习惯了。不过对方既然将这神游蛊放进来,也是有买卖的意向。那就先看看罢。”

方源淡然自若,并没有患得患失,心境也保持着平和。

即便宝黄天是蛊仙最大的交易市场,也极少有买卖仙蛊的交易发生。历史上仙蛊交易的次数,屈指可数,并且大多数都是蛊仙们先谈妥了,才通过宝黄天作为中介,用宝光验证真假,确保交易的安全。

这些交易,绝大多数都是以仙蛊换仙蛊。但贩卖神游蛊的蛊仙砚石老人,提出的交易要求却是不同寻常。

“我要用这只神游蛊,换取第二空窍蛊的秘方。”

这个要求包含的信息量,就太大了,引起广泛的关注。宝黄天中,各个蛊仙的神念此起彼伏,相互激荡。

“神游蛊!想不到今天,竟然能见到这只传说中的蛊虫。”

“虽然是仙蛊不假,但神游蛊的效用也实在过于随意。当年人祖太子都因此身陷险境,我们这些小小的蛊仙就更危险了。”

“但神游蛊到底是仙蛊啊,备在身上,若是走投无路,陷入绝境之时,也可以搏一搏的。”

“与其谈论神游蛊,我倒更想知道第二空窍蛊的秘方!”

“不错,第二空窍蛊我早就听闻过,想不到竟然真的有这种蛊方?”

“第二空窍啊!啧啧,凡人还不能深刻理解这其中的价值。但是对于蛊仙而言,谁不心动?”

……

方源已经从座位上站起身来,凝神望着半空中的通天蛊。

他手中就有第二空窍蛊的秘方,换得这蛊,只在一念之间。

但是!

“这位卖神游蛊的砚石老人。恐怕是针对我而来的。他直接要求第二空窍蛊的秘方……贩卖神游蛊的时机,也不早不晚,偏偏挑中我在买卖的时候……好厉害!这蛊仙的流派恐怕是太古智道,擅长推演测算。我此刻开启通天蛊,在宝黄天买卖,已经被他算出来了!”

智道是蛊师中,十分神秘的流派。从太古就有,流传至今,人数一直都十分稀少。

智道的开派祖师。乃是太古时的星宿仙尊,二代天庭之主。她活了一万九千年,在九转蛊仙中长寿名列第二。

死亡前推衍天机,星宿仙尊算尽死后三百万年一切事家里有个狐狸精全文阅读。她推算出死后,天庭将长期无主。乾坤动荡,会出现三位魔尊。

她便着手布下三局,专门留着对付三位魔尊。并嘱咐后人照此进行,可保天庭三百年太平。

她死后,果然天地动荡,时代变迁,连续三个大时代。相继涌现出三位魔尊。

三位魔尊当代无敌,都攻过天庭,皆被星宿仙尊的布局所阻,功亏一篑。天庭因此屹立不倒。

“智道蛊仙……砚石老人……”方源双眼微微眯起,口中喃喃。被一位智道蛊仙注意着,绝不是件好事。

智道蛊仙,通晓天机。最擅长布局和谋算。往往坑人不漏痕迹,阴人不动声色。是最难对付的一类强敌!

“我在南疆出的风头。实在太盛了。引起了智道蛊仙的注意。呵呵呵,这一世我勇猛精进,冒着奇险在悬崖上一路冲刺,冲得太快了,以区区凡人之身,都引来了蛊仙的关注!”

这种情形,就好像是一只蚂蚁,引起了大象的关注。

一时间,方源仿佛看到,一双充满智慧和阴谋的眼睛正透过通天蛊,盯着自己。

空气中,弥漫着无形的压抑感。

但方源此刻心境早有不同,他仰头哈哈一笑,尽数驱散心中的压抑。

“好,有着智道蛊仙的算计,无疑更有意思了。哼,区区神游蛊,区区第二空窍蛊,怎么诱惑得住我?”

前世五百年的经验,培养出一股对潜在危机的直觉。

方源冥冥中感到,这只神游蛊极可能便是砚石老人抛下的鱼饵。

“仙蛊虽好,但我矢志永生,所谓的仙蛊也不过是修行证道的工具罢了。”

……

与此同时,南疆,生死福地。

一位老者,身着黑袍,散发着七转蛊仙的幽幽气息,静静地盘坐在蒲团上。

他满脸皱纹纵横,双眼一片漆黑,没有丝毫的眼白。

他盯着半空中的通天蛊,感受着宝黄天里纵横荡漾的无数神念,面无表情,一动不动。

正是砚石老人!

老人的身前,跪着杀人鬼医仇九。

他看了通天蛊半晌,脸上神情失望:“太师尊,看来那个方源没有上钩啊。”

砚石老人微微一笑,丝毫不见恼怒:“这条小鱼的确有点意思。能舍能弃,不过区区一介凡人,但这气魄却超乎大多数的蛊仙。不过他的胆子实在太大,居然将你变成他的奴隶,惹到我们影宗的头上。这就是自己找死了。”

仇九连忙叩首:“多亏太师尊归来,徒孙才免遭被人奴役的悲惨命运!”

“嗯……”砚石老人微微颔首,“小徒孙,你身上的奴隶蛊,我已经动了手脚。我算定方源必定会参与义天山正魔大战,届时你可卧底在他身边,伺机而动。”

“是,太师尊!”

“嗯,你下去罢,将你师妹白凝冰唤来。”

“是,徒孙告退。”

------------

\end{this_body}

