\newsection{力量才是根本}    %第四十四节:力量才是根本

\begin{this_body}

%1
“啊,蛮家想要吞并我们葛家?”葛光惊呼一声。

%2
葛家老族长长叹一声:“唉,你妹妹死了,他们根不愿意听为父的解释,为什么?就是想把这个借口,当做出兵的正当理由!但是为父找来了常山阴助阵。常山阴是北原的英雄,威望极高,背后又有常家,他蛮图也不敢强词夺理,今夜才遗憾收手。”

%3
葛光讶然:“原来这背后还有如此曲折?但阿爸,儿子有一点想不通,既然您早就看出蛮家的居心,为什么还要答应这场婚事,将妹妹许配给那个猴子蛮多呢?”

%4
葛家老族长屈起手指,狠狠地敲动葛光的脑门:“你这里不长脑子么?为什么答应婚事?你以为为父想委屈自己的女儿?!还不是因为大风雪将近,若是能和蛮家结亲,我们就能借住红炎谷,尽最大可能保全自己的部族!牺牲你妹妹一个人的幸福,却能保全整个葛家。这场婚姻,就是一场交易。唉,可惜你妹妹逃婚,死在了腐毒草原!”

%5
葛光皱起眉头,猛眨眼睛:“阿爸,我有些明白了。”

%6
“不,你还不明白。”葛家老族长十分了解自己的这个儿子,恨铁不成钢地继续解释道,“今夜蛮家父子,表面上是邀请常山阴,我们随同。实际上,真正的目的是要对付我们父子。而为父就推出常山阴,作为挡箭牌。”

%7
“那蛮多取出追烟蛊,用心险恶至极!你们都怀疑常山阴的时候,为父却对常山阴信誓旦旦地说,相信他。你真当为父不怀疑吗?为父还没有这么老糊涂呢!”

%8
葛家老族长一跺脚,语气苍凉:“但是为父不敢怀疑他啊。我们父子能和蛮家分庭抗礼,是借助了常山阴的力量。如果怀疑他。那蛮家父子就挑拨成功,离间了我们和常山阴的关系。常山阴若不站在我们这边,说不定今晚的夜宴,我们父子俩就回不来了。”

%9
葛光满脸都是惊讶:“啊,难道他们蛮家胆大到这种程度,想要杀了我们父子?”

%10
“哼,你以为你这次出去搜寻,为什么会遇到那么多的风狼围杀?北原上,用兽群借刀杀人的手段还少么?不过今晚这种情况。蛮家杀死我们倒不至于,但绝对会软禁。到那时,他们蛮家再用葛谣的事情为正正当借口,吞并我们葛家,而葛家失去我们。又群龙无首,结局堪忧啊。”

%11
听到父亲的解释,葛光总算明白了今夜的凶险,脸上流露出后怕的情绪。

%12
“为了部族,为了大局,就算常山阴真是凶手,我们也不能去怀疑他!你真以为我忘记了蛛丝马迹蛊吗?怎么可能!但你偏偏要提这茬。万一真的证实了常山阴就是凶手,怎么办?”葛家老族长语重心长。

%13
葛光陷入沉默,良久才道:“所以阿爸,才将珍贵的五转蛛丝马迹蛊。赠送给常山阴。就是要弥补我们和他关系的裂缝,让他站在我们这边是吗?”

%14
葛家老族长点点头:“你总算有点领悟了。儿子,你虽然资质出色,修为也高。但是要成为葛家的新任族长。还差得远呐。”

%15
“阿爸,有你在真好。儿子今后一定向您多多学习。葛家少了谁都行。就是不能少了您呐。”葛光心悦诚服地道。

%16
葛家老族长微微摇头:“岁月不饶人,阿爸我已经老了,今后葛家还是要靠你的。唉……经过这件事情,我也算认清了蛮图这个人。他就是贪婪的豺狼,再多的财富,也填不满他的心。”

%17
“如今你妹妹也去了,我们想要借红炎谷也失去了名义。总不能拆散了家族,全部投靠蛮家吧?不,葛家绝不能这样泯灭,否则我就是葛家的千古罪人!经过这一夜,为父已经想清楚了,不能停留在这里,早晚要被蛮家谋害的。几天之后,我们就启程,赶往英雄大会。”

%18
“阿爸,我们就这样走了,蛮家会轻易地放过我们?”葛光担忧地问道。

%19
“他们当然不想放过我们,但是我们随同常山阴一起走。蛮图忌惮常山阴,是不敢出手的。”葛家老族长嘿嘿一笑。

%20
“那我们这样利用常山阴前辈,是不是……”葛光有些不好意思。

%21
“你这傻瓜!为什么不利用?利用有什么不好?好的猎手,都善于利用周围的一切。当我们力量不足时,就要用智慧来弥补。这一切都是为了家族的生存啊!”

%22
葛家老族长斥责一番后,停歇下来,脸上浮现出复杂的神色:“不过这个常山阴,的确是名不虚传的英雄人物……也许他已经看出来了,但他始终站在我们这边。如此正直仗义,扶住弱小,这才是真正的正道楷模,是人世间的明光。儿子,庆幸吧,让我们葛家遇到这样的人!”

%23
……

%24
时光匆匆,又过去数天。

%25
房间内,方源看着手中的骨竹蛊,被鬼火燃烧殆尽,呼出一口浊气:“这是最后一只骨竹蛊了。”

%26
这些天来,他勤修不辍,将礼盒中上百只骨竹蛊,都全部用完。

%27
经过他大力的抢修,战骨车轮上最为严重的八道伤口,都被修复成功。虽然如今的战骨车轮上,仍旧是伤痕满布,但总算是脱离了危险期。

%28
将战骨车轮送回空窍,方源开始检视自己的空窍。

%29
他是四转巅峰的修为。

%30
因此空窍四壁,是透明的晶膜。真元海高达九成,都是真金真元。

%31
但是方源初到北原不久,在没有完全适应的情况下,真金真元只相当于初阶的淡金真元。

%32
所以,方源目前的修为暂时停滞了。用淡金真元,当然突破不了壁障,成为五转蛊师。

%33
“要加快适应的速度,也不是没有办法。最常用的,就是用三更蛊,加快自身的时间。但是这样一来。蛊师剩下的寿命就会硬生生地缩短三倍。”

%34
此举急功近利,方源当然不取。

%35
原倒不是因为他珍惜寿命,而是春秋蝉。

%36
方源若对自己用了三更蛊,他身上的光阴长河的流速,就加快三倍。那么,寄居在他身上的春秋蝉,恢复速度也会提升三倍。

%37
除了命蛊春秋蝉,以及藏在腐毒草原上的定仙游蛊之外,现在方源的身上还有不少蛊虫。

%38
首当其冲。两只五转的蛊,都来自北原土。

%39
一只是战骨车轮,目前正在修复,短时间之内无法提供任何的帮助。

%40
另一只则是蛛丝马迹蛊,可以用来侦察、追踪。

%41
“葛家底蕴还是有的。居然有一只五转的蛛丝马迹蛊。此蛊可用来追踪蛊虫,是捕捉野生蛊虫,防备它们逃窜的有力手段。可惜,葛家老族长虽然有这蛊虫,却不敢出去寻找自己的亲身女儿。”

%42
对于葛家、蛮家的争斗,方源洞若观火。

%43
蛮家的三子蛮多,野心很大。想要染指族长之位。但碍于自身修为不足,就看上葛家。想要通过迎娶葛谣,将葛家成为自己的妻族,帮助他争夺族长宝位。

%44
蛮图未必看出蛮多的真正用心。但是他对吞并葛家一直很感兴趣。

%45
而葛家则是想利用联姻,牺牲葛谣一人,借助蛮家的红炎谷,捱过十年一次的大风雪。渡过眼前的难关。

%46
葛家当然不想拆散整个部族,但蛮家却想着完全吞并这块大肥肉。但又怕被肉中的骨头卡住咽喉。

%47
葛谣逃走后。蛮家极力抓住这个正当的借口不放。甚至很可能暗中,对葛光出手。

%48
只要杀掉葛家族长父子,那么葛家就会群龙无首,陷入内乱,更方便蛮家吞并。

%49
葛光懵懂无知,葛家的老族长却是年老成精,渐渐看清了局面,发现自己是无法满足蛮图的贪婪,但又泥足深陷,只好隐忍不发,整日坐镇在部族中,不给蛮家暗中下手的机会。

%50
而常山阴的到来,带给了葛家脱困希望。

%51
葛家老族长也许在得知方源“常山阴”的身份时,就想到了利用。因此,他热情洋溢地招待方源,甚至一见面就送出一百万的厚礼。

%52
之后,他也利用得很好,借助常山阴的力量,和蛮家角逐。

%53
那场月夜下的晚宴,看似其乐融融,实际上三方的比拼,背地里暗流汹涌,藏着刀斧之险。

%54
结果是——

%55
蛮家一方攻势受挫,不仅没有针对住各家,而且还失去了葛谣这个正当的借口。但这方没有失败,他们仍旧是最强大的。

%56
葛家一方,推出常山阴,利用他的力量,成功地捍卫了自己的部族,暂时渡过了这次危机。他们成功了,但他们仍旧处于弱势地位。

%57
而方源,则是揣着明白装糊涂,借助这两方的争斗,达成了原先登台亮相的目的,同时从中得利,壮大自身。

%58
人是万物之灵,人与人之间的争斗,并非全是惨烈的简单搏杀。哪怕在民风彪悍的北原,也有智计和谋算的较量。

%59
至于葛谣……

%60
这个单纯的少女,不过是两大家族政治斗争的牺牲品。

%61
甚至根据方源的暗中猜测,葛谣的逃婚也颇为蹊跷。能够在外松内紧的葛家营地中逃出去,说不得就是葛家老族长暗中安排,试探蛮家之举。可惜现实总会有意外发生,谋算虽好也赶不上变化,也许是蛮家族长的大儿子、二儿子的势力出手,唯恐蛮多坐大,希望葛谣去死。总之葛谣逃到了腐毒草原里,碰到了方源,才发生了之后一系列的事情。

%62
摇摇头,方源将心中的这些猜测排出脑海:“任何计谋的基础都是力量。蛮家为什么不直接吞并葛家,是因为他们不是超级家族,只是大型部族,力量有限。葛家为什么转危为安,除了利用了我之外,身他们也曾经是大型家族,有着底子。如果葛家老族长,身有五转修为,恐怕早就取出蛛丝马迹蛊了。”

%63
“不管哪个世界,力量才是根呐。如此一来,精明如葛家老族长,恐怕要准备迁徙部族了。十年风雪的危机,还是要借助王庭避祸!”

\end{this_body}

