\newsection{大决战}    %第一百三十一节 大决战

\begin{this_body}

方源将常飚三人的神情,尽收眼底。

对于狼王常山阴的过往,方源大多是通过《常山阴传》得知,却不知道当初陷害狼王的真凶。

不过他即便知道,也没有心思去为死去的狼王报仇雪恨。

他是方源,所谓的常山阴,不过是一张面具罢了。

“从今日起,我便是常家部族唯一的太上家老。”方源开口,打破沉寂。

常飚浑身一颤,睁开双眼,连忙叩首道:“常飚拜见太上家老大人。”

“嗯。”方源点点头,“当年的事情,还需彻查。不过现在并非良机,至少要等到王庭之争结束之后。从今日起,我便是常家唯一的太上家老。常极右,你担任常家族长。常飚任第一家老。倪雪彤,你我缘分已尽,继续做常飚的妻子吧。”

因为巨阳仙尊定下的传统,在北原,女子地位低下,常常沦为货物而被交易。甚至有时候,家中来了尊贵的客人,主人会将自己的妻子派到贵客身边,为贵客侍寝。

“啊?”常极右惊愕失声,呆立当场。

倪雪彤没有说话。

常飚忍住心中震动,再叩首:“属下遵命!”

“都退下吧。”方源摆摆手,下了逐客令。他还要抓紧时间,继续修行。

三人恍恍惚惚地走出大蜥屋蛊,直到寒冷的夜风,将他们吹得浑身一颤,这才惊醒过来。

“我居然就这样过关了?”常飚心中涌起无限的喜悦和庆幸之情。

“不过,当年的事情,我做得滴水不漏!就算有些蛛丝马迹,经过这些年洗刷,也早就没有了。当年我故意接近常山阴,和他成为无话不谈的好友。如今常山阴遭逢巨变。又多年不见,感情生分了也很正常。”常飚在心中急速思索起来。

现在的情形,比他预料中的。还要好数倍不止。

“我自己虽然从族长贬为家老,但大部分的权力还在。常山阴叫我担任第一家老。可见他还是信任我的!而他将常极右立为常家的新任族长,可见他骨子里还是念旧情的!只要他还念旧情,一切都好办了……”常飚越想,越是振奋。

他沉浸在自己的世界里,却没有注意到自己的妻子倪雪彤的复杂神情。

曾经常山阴十分迷恋她的美貌,但是刚刚,常山阴却连多看她一眼都没有。

在来时的路上。倪雪彤万分担心,一旦常山阴将她重新抢夺到自己身边。这样一来,她就和爱郎常飚分离了,这该是多么痛苦的事情啊!

但是如今。情况比倪雪彤料想中的要好许多倍。

常山阴不仅暂时没有追究当年的事情,而且还叫她继续做常飚的妻子!

这是倪雪彤之前,梦寐以求的结果。她应该高兴才是,但是不知道怎么的,她的心中残留着余悸的同时。还有一股她自己也不想承认的失落感。

而常极右,则陷入到巨大的欢喜、疑惑和迷茫。

“我终于见到父亲了,他就在几步远的地方!他比我想象中,还要威严许多。”

“父亲没有认我这个儿子,而是直呼我的名字。他难道不知道。我就是他的亲生骨肉吗?”

“但父亲,为什么又让我担当常家族长呢?我这么年轻,只是三转修为,能行吗?”

“我懂了!这应该就是父亲给我的考验。他是在考验我这个从未谋面的儿子,如果我能将常家治理好,出色的完成他的考验。那么他是否会感到欣慰,会认我这个儿子?”

念及于此,常极右心中不禁激动起来,他下定决心,一定要尽最大努力,在今后的王庭之争中好好表现!

方源不会料到,他简简单单的安排,会带给常飚三人如此巨大的心理波动。

不过就算知道,他亦不会在意。

五百年前世,常山阴帮助马鸿运,登上王庭之主的宝位之后,也是重掌常家的大权。

和地球不同,当伟力能归咎于个体时,力量越强,权利便越大。

时至今日,方源已经再不是青茅山上的低阶小蛊师,受着体制的压制、剥削。如今他已经可以操纵、恣意篡改一个部族的权利构架。可以说,他已经站在世俗的巅峰。

他心知肚明,这一切都得归功于他手中拥有的强大力量!

“如今,我的第一空窍已经完全适应了北原,可以动用五转巅峰的真元。第二空窍,也达到了五转中阶的地步。两个空窍的资质都是甲等九成,使用如今的两套蛊虫,真元充沛得很。”

“但是奴、力二道的蛊虫,并不算极致的强大。力道上,自从有了五转功倍蛊后,爆发力已经变得足够高,但是我的身体却难以承担。”

之前,在和刘文武三兄弟的合体杀招“三头六臂”对战时,方源完全可以凭借力道战力,和其一较高下。

但是方源清楚,一旦他爆发出五百钧力量,不说对手如何,单说自己本身的肉体就难以承受。

“我的骨骼,是无常骨。浑身的皮肤,是龟玉狼皮。要承受得住五百钧力量,这是远远不够的。但是如果我要将肌肉、大筋改造,适合力道,就不会适合奴道。适合奴道,就不能适合力道。归根结底,还是奴力二道,相互之间虽有互补,但兼容的程度太低了。不像魂道和奴道,或者魂道和智道之流。”

这个问题,其实一直困扰着方源。

如果解决不了,那么方源的奴力两道,只能达到精深,谈不上巅峰般的强大。

虽说方源现在掌握着关于落魄谷的传承消息,但是未来是说不准的,什么事情都有可能发生。方源生性谨慎,在没有得到落魄谷之前,他还下定不了决心,去转修魂道,因此仍需要完善奴力二道。

方源闭目沉思了一会之后。缓缓地睁开双眼,取出空窍中的东窗蛊。

此蛊乃是存储蛊,专门存储信息。得自琅琊地灵。

东窗蛊中,有关于杀招“三头六臂”的详尽信息。这个杀招极为强大。能令刘文武、欧阳碧桑、墨狮狂三人,形成巨大怪物,战力暴涨到恐怖的地步。

黑家战胜了刘家之后,针对这个杀招提出要求,因此刘家支付的战争赔款中,就有这项。随后,就被方源用战功换取过来。

这些天来。方源没有事的时候,就在琢磨这个杀招。

蛊师同时催动多只蛊虫,蛊虫的效果相互搭配,形成更加强大的效果。这就是蛊师俗称的杀招。

杀招“三头六臂”。需要同时催动十八只蛊虫。蛊虫从三转到五转,消耗真元极多。同时,还得需要三位蛊师,单独的个体反而不成。

这个杀招,方源是用不了的。但这并不意味着。对方源没有价值。

杀招,或者蛊方,都是用蛊方面的精粹。

为什么这些蛊虫,相互搭配,就能有这样的效果?为何那些蛊虫。却反而不能呢?如果将这其中的某只蛊虫,替换成另一只,效果又会如何呢?如果敌人再次使用这个杀招,该用什么方法,来破除呢?

人是万物之灵,蛊是天地真精。

蛊的身上,蕴藏着天地的些微法则,大道的残片。

了解蛊,就是理解大道,理解这方世界的自然法则。就仿佛地球上,利用实验,获得科学定律一样。

这只蛊方,带给方源的启发很大。

“如果我生长出三个头颅,六只手臂,会怎样呢?”

他脑海中灵光一闪,像是打开了新的窗户。

他的肉体,就仿佛基石。奴力两道是矗立在基石上的楼阁。现在这块基石不大,两座楼阁只能建成低楼。如果将这块基石扩大,是不是就能同时承载两座高楼?

方源对自己的相貌,历来都不在意。

什么英俊美丑,不过都是外人的眼光。旁人的看法和他有什么关系?

只要战力强大,被认作怪物又有何妨?

北原历,七月。

天气日渐严寒,霜气凝结成冻,阴雨绵绵不休。

各路大军经过多次激战,数量已经锐减到不足五十路。

黑家虽然战胜刘家,但伤了元气,停驻营地,宛若受伤的猛兽,抓紧一切时间喘息和休养。

七月中旬。

独角地区,耶律大军击败七路大军的围攻。反击之日,耶律桑击杀五转蛊师多达三人。

然而这场战役中,最大的功臣,却是耶律大军中,祁连一族的隐家老,祁连族长的义子无名。

无名乃是五转中阶,暗道蛊师。在大军对峙中,他屡次进入敌营,暗杀敌酋,成功暗杀了两位五转强者,十三位四转蛊师,使得七路大军人心惶惶,士气低落。

北原历,八月。

杨家招揽了奴道大师江暴牙之后,实力大涨,一路凯歌,几番大捷,成为王庭之争后期,涌现出的新热门。

新晋的奴道大师,“豹王”努尔图,则率领大军,威逼陶家。陶家盟军在挑将过程中接连受挫,盟主陶幽审时度势,心知自己登上王庭主位,已经无望。便选择依附努尔图,努尔大军吞并了陶家之后,军力大涨。

八月中旬,黑楼兰下达军令,全军再起征程。

到了九月,王庭之争的格局,已经明朗。只剩下五路大军,最有希望。

拥有狼王常山阴、太白云生的黑家,新晋豹王领导的努尔家,拥有鼠王、鹰王的杨家、拥有马王的马家,身具仙蛊的耶律桑率领的耶律大军。

九月上旬,努尔大军和杨家接战。豹群承受不住鹰群和鼠群的双重打击,在坚守了大半个月后,努尔图战败。

十月初,趁着杨家消化战争赔款,耶律桑趁机发难。

杨家大军中有人主张开战,有人主张坚守,有人提议撤退,因为意见不合,导致大军进退失据,被耶律桑得逞。

耶律桑获得胜利后,却没有得意太久,就被马家盯上。

马家一路急行三天三夜,打了耶律桑一个措手不及。

相同的一幕,上演在他的身上。耶律大军还未消化战果,就被马家击溃。

耶律桑领着残兵一路败逃,主动投靠黑家大军。

十一月初,黑家大军迅速北上,一路建设八道防线,于该月中旬,和马家展开决战。

只有胜者才能进驻王庭福地,而败者赔款,在惋惜和失望中,迎接北原十年的大风雪。

一时间,这场决战吸引了无数道幕前、幕后的目光。

前几场激战,黑家占据些微上风,马家失去两道防线,退到第三防线坚守。

耶律桑心存报复,不断挑将,杀得马家大军上下闭门不出,士气低落。

马家无奈,不得不向背后的大雪山福地求援。

ps:状态不行,心烦气躁得很……

\end{this_body}

