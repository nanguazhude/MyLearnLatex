\newsection{影子深黑}    %第一百八十七节:影子深黑

\begin{this_body}

方源低下头,注视着晕死过去的太白云生。

他便是太白云生昏死过去的凶手。

此刻,太白云生浑身浴血,躺在地上,深陷的眼窝,双目紧闭。

他伤痕累累,雪白的发须凝结着血浆,早已不复北原第一治疗蛊师的风范。

仅仅几步之遥,主殿大门外,海量的血兽咆哮着,嘈杂的声浪不断激打过来。

方源抬起头,盯着它们,轻声开口:“闭嘴。”

霎时间,门外死寂。

血兽们纷纷闭嘴,宛若乖巧的猫狗,伏跪在地上,一动不动。

方源利用六角楼主令,掌控了这一层,他就是这里的主人,这些血兽自然也受他的操纵了,听凭他的心念,指挥如意。

方源闭上双眼,心神投入此层,静静感受。

此刻,在这道关卡中,还留有不少蛊师。

这些蛊师,一部分是太白云生拉动进来的蛊师,但在之前的战斗中被淘汰下来。还有一部分,则是各方势力的眼线,都是侦察蛊师。他们来源于黑家、马家、耶律家各个势力,关注着太白云生此行的成败。

至于这座主殿当中,除去地上那些干枯破烂的尸体,只剩下方源和太白云生两个人。

方源关上主殿大门,半蹲在地上,伸出右手掌,一把抓住太白云生的头颅。

蛊虫早就准备好了,他接连催动。

很快,太白云生的脑门上,亮起微微的白光,成为幽暗的大殿中唯一的光源。

白光越来越盛,太白云生的脸上,渐渐浮现出痛苦的神色,眉头越皱越深。

酝酿片刻之后,方源陡然睁开双眼!

他的双眼。没有瞳孔,只有一片眼白。

眼白绽放三尺微光,与此同时,大量的画面在方源的脑海中显现。

太白云生从老年回溯到年轻时的记忆,都被方源提取出来。

……

一位老人行走于北原,天苍苍野茫茫,风吹草低。狼群嚎叫。

……

“老先生您的救命之恩,我们兄弟俩没齿难忘!”高扬、朱宰一齐跪倒在他的脚下。

……

一位紫发老乞丐,裂开嘴,露出仅有的几颗牙齿,怪笑道:“你要想成为什么样的蛊师呢?嘿嘿嘿,我这里恰好有三份完整的传承!”

……

“嗯。这个小子长得不错,就选他了。”墨人城中,一位墨人指着少年时期的太白云生,哈哈大笑道。

再往前,更加年轻的时候……

“为什么,你为什么要背叛我?!”新婚大喜之夜,太白云生踉跄而倒。帐外传来震天的喊杀声。

而他的妻子一脸冷漠和仇恨,慢慢逼近他,目泛凶光,咬牙切齿:“太白云生,你要恨就恨你的父母,是他们吞并了我的部族,杀害了我的父母,我要为他们报仇!”

童年时期……

“我的儿子。你可是我太白部族的下一任族长!不准哭,不要再同情心泛滥了!要在北原生存下去,我们的心得硬起来!将来,你要领导我们太白一族啊。”父亲十分严厉地训斥着。

……

“啊啊啊……”方源痛得大吼。

脑海中,不断闪现的画面,叙述了太白云生的传奇一生。如此宏大的信息,对方源的头脑是巨大的冲击和伤害。

好在画面并非无穷无尽。终有结束的时候。

阅尽太白云生的一生,方源立即停下蛊虫,一屁股坐在地上。

他呼呼地喘着粗气,浑身都是大汗。良久。他的瞳孔才恢复焦距。

搜魂,绝非意事。尤其是方源顾及太白云生的安全,不想伤害他,因此只能自己承担大部分的压力。

停止了搜魂,太白云生仍旧昏睡着,但原先紧皱的眉头,却舒展开来。呼吸平稳,神情安详。

反而方源的眉头,则微皱起来。

“没有找到啊!”他遗憾长叹。

“没有找到什么?”脑海中,墨瑶意志禁不住疑惑而发问。

方源的这一切举动,都令她好奇。

方源没有答她,只是眉头皱得更深。关于江山如故蛊的重大计划,他怎么可能告诉墨瑶呢?

江山如故蛊是太白云生成仙之后,才拥有的仙蛊。

有传闻说:此蛊乃是太白云生成仙之时,天地交感,灵光爆发,自发凝练而成。

但还有一个可能,就是太白云生的脑海中,本就有江山如故蛊的仙方。

如果真有蛊方,那么方源大可盗取蛊方,以及江如故、山如故两蛊,带到琅琊福地中去,叫琅琊地灵出手,替他炼制仙蛊。

这样一来,他就不用虎口夺食,危险性大降。

但方源这一次搜魂之后,结果糟糕。

方源没有搜出江山如故的仙蛊方,这就说明传闻是真的。江山如故蛊,的确是太白云生在成仙之际,天地交感而得。

也就意味着:方源要取得此仙蛊,就得从成仙的太白云生手中,抢夺此蛊。

方源还不是蛊仙,以凡战仙,方源印象中还未有任何成功的例子,无疑比登天还难!

但还能有什么办法呢?

当初的三个选择,这个已经是最容易的一道。时间、精力都投入到这个计划中,方源虽然也没有把握,但也只有积极准备,冒险一试了!

……

八十八角真阳楼外,抖现太白云生的身影。

“出来了,出来了!”

“结果如何?有人看到太白云生杀进了主殿。”

“不好,太白云生大人一动不动,似乎昏死过去了!”

周围的蛊师,立即围拢过去。

打量一眼,众人脸色微变。太白云生身上伤势沉重无比,让他们暗暗心惊。

“还有呼吸!”一人伸出手指,探了探太白云生的鼻息,高喊起来,“快。谁是治疗蛊师,快来稳住老大人的伤势!”

“我来,我来!”

“我也是治疗蛊师!!”

许多治疗蛊师纷纷主动出手,太白云生的威望和仁厚之名,早已经深入人心。

毫无疑问地讲,他比黑楼兰、常山阴更得人心。

“就连太白云生大人,都受了这么严重的伤势。唉。这次大举闯关,恐怕是失败了。”有人叹息。

“闯关的时间已经结束了,大部分的蛊师都没有回来,这次伤亡太过惨重!”

“你们谁看到高扬、朱宰两位大人出来了?”有人惊觉,忽问。

众人四处张望,旋即面面相觑。

没有人看到高扬、朱宰的身影。而八十八角真阳楼中,那道关卡仍旧存在。只是短时间里不让蛊师再进入。

这意味着什么,众人心中都是雪亮。

八十八角真阳楼凝聚至今,已经有五位五转强者牺牲了。

如此沉重的伤亡,让广场陷入了一片沉寂当中。

当太白云生睁开双眼时,发现自己已经躺在床榻之上,浑身虚弱乏力。竟连起身都困难。

看到他睁开双眼,一旁伺候的丫鬟,立即惊喜地叫出声来:“老先生,您醒了,您终于醒了!快来人呐,快来人呐,老先生醒了!”

很快,太白云生就听到一连串急冲冲的脚步声。

一群治疗蛊师。来到他的身边,一齐为他检查身体。

“家老大人,您放心,您的伤势已无大碍。只是您年岁大了,这次受伤颇重,伤及根本。今后须得注意保养,尤其是最近几个月。身子骨虚不受补,需要静静安养才是。”治疗蛊师的领头,温声劝慰道。

太白云生为了拉起队伍闯关,不惜答应了黑楼兰。已经成为了黑家的外姓家老。

太白云生眼神散漫,从苏醒过来就一阵发怔,听了这话,这才回复了一丝神采,他问道:“这是哪里?”

“回家老大人的话,这里是黑楼兰大人的住所。自从老大人您闯关失利,险死还生,我家族长就十分关切,亲自将您接到这里来治疗修养。下人们已经禀告去了,相信很快,族长大人就会来看您的。”依旧是那位首领答道。

“闯关失利,险死还生?”太白云生微微皱起眉头,脑海中的记忆开始复苏,他回忆起了最后那一幕――

他费尽最后一丝真元,催鼓起防御蛊,在血兽的围杀中成功地挤进了主殿。

但随后不久,他就昏死过去,失去了知觉!

然后醒来时,就发现自己躺在了这里。

“这么说,我真的是闯关失败了?!”太白云生语调陡然一扬,目光倏地变得尖锐无比。

“家老大人……”围在床边的一群治疗蛊师,相互对视,想要劝说安慰,却都说不出口。

于是,他们只好都低下了头。

房间中,一片安静。

太白云生目光发直,沉默了好一会儿,忽然仰头大笑:“哈哈,原来老夫失败了。功亏一篑,功亏一篑啊!”

他起不了身,只能用手掌使劲地拍打床边,发狂大笑。

“老大人,老大人!”治疗蛊师们慌了,连忙相劝。

“可怜我朱宰、高扬,为保护老夫牺牲了生命!”太白云生双眼泪水横流,他的笑声充满了悲痛。

“家老大人节哀,人力有时穷,家老大人您已经尽力了!”

“家老大人,您能活着出来已经是万幸了。”

“人死不能复生,老先生还请节哀顺便啊……”

众人你一言我一语,相劝不止。

但这些话,听在太白云生的耳里,却充满了讽刺的意味。像是一根根针一样,扎进他的心里。

在最后关头,太白云生留下真元,没有选择救下朱宰、高扬,而是为了自己,催动了防御蛊,闯进了主殿当中。

是他,为了自己私欲,视同伴的牺牲而不顾。

这还是太白云生吗?

这还是那位北原公认崇敬,救死扶伤,救治世人,消除疾苦的太白云生吗?

为什么自己会这么做?

偏偏在那个紧要的关头,自己根本就不假思索。选择了这样做!

故意牺牲高扬朱宰,为自己换取机会,为的就是通关奖励的那只十五年寿蛊!为的就是自己的苟且偷生!

这个选择,让太白云生对自己感到陌生,感到羞愧,感到自卑,感到悔恨!

当初的不假思索、毫不犹豫。现在则化成道德的皮鞭,拷问他的灵魂,鞭笞他的良心!

太白云生痛苦地闭上双眼,双手紧紧握拳。

“族长大人到――!”

“属下拜见族长大人。”

一屋人都跪倒在地,黑楼兰面带微笑,来到太白云生的床边。

见到太白云生痛苦的模样。黑楼兰眉头轻轻一皱,旋即又舒展开来:“太白家老,很高兴你能苏醒。情况我已经听说了,你和高扬、朱宰等人已然尽显我北原男儿的英勇,虽败犹荣!只要总结教训,将来必能打通此关,反败为胜。洗刷耻辱!”

太白云生却没有睁眼,一言不发,神色痛苦。

他已经想明白失败的缘由了。

他进入主殿之后,成功脱离了血兽的围杀,倒在主殿中。但成功之后他狂喜大笑,心境大起大落,再加上身负重伤,因此昏迷过去。

闯荡此关。是受时间限制的。

时间用尽,昏迷中的他和其他外围的蛊师一样,都被强行传送出来了。

明明距离成功,只有一步之遥,结果却因为昏迷而失败。

但如此讽刺的结果,并非太白云生心中的痛苦来源!

他的痛苦,在于他为一己私欲背弃同伴。

这还是他太白云生吗?

记忆中的一幕幕。又翻腾上来。

从小到大,他一直坚信爱的力量。

他从孩童时起,就颇有仁名。

太白一族吞并其他部族,当他看到童年时的玩伴要面临成为女奴的悲惨命运时。是他开口,要娶她为妻。因此也宽恕了一批俘虏。

但新婚之夜,他的妻子背叛了他。俘虏们联系外敌,突袭他的部族,他的父母因此而亡。

之后的奴隶生涯,艰辛困苦,他一直受到内心的强烈煎熬。

终于有一天,他好心为了一个素不相识的老乞丐舀了一碗水。老乞丐传给他三个仙道传承作为选择。

第一份传承,能让人浴火踏焰,睥睨凡尘。

第二份传承,能令人掌风浮空,逍遥天下。

第三份传承,则是穿越生死,扶助苍生。

太白云生选择了第三份。

那一刻,他仿佛在黑暗中找到了光明,内心不再煎熬,他无悔,他浴火重生!

过去这么多年,老乞丐的笑声,犹在耳边。

穿越生死,扶助苍生也成了他的人生信条。

其后的生涯,他真的这么做了。

他收获了无数人的感激,他的仁名广为传播,他的光辉照亮整个北原。

他是一个活生生的传奇。

但现在!

他失败了!

他的失败,不在于没有获得寿蛊。而在于背弃同伴,背弃了自己的人生信条!!

偏偏这一切,都还是他自己不假思索地去做的。

他用了几乎一生,来竖立和实践自己的人生准则。然后在那一刻,他自己将这个准则摧跨。

他见识到了自己的另一面,自己的自私。

他曾经以为,自己就是众人眼中的那个人――夕阳下行走于草原,扶助苍生,悬壶救世。

但现在他的心中,这个形象,已经渐近渐远,步履蹒跚。

在落日的余晖中,他的身影拖得老长。

影子深黑……(未完待续。。。)

------------

\end{this_body}

