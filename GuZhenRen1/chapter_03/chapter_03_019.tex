\newsection{我们要屠仙}    %第十九节:我们要屠仙

\begin{this_body}

%1
狐仙福地,西部广袤的草原。

%2
辽阔的绿地,连绵一片,伸展出去。一只新生的小狐狸,正在和翩跹飞舞的蝴蝶嬉闹玩耍。

%3
在狐仙福地中,狐狸算是最强大的掠食者,因此生活无忧。

%4
虽说第六次地灾时,损失极为惨重,但终究是保留了火种。广袤的草原上,狐群缓慢而又坚定地延续着。

%5
忽然,空间一阵波动,两个身影陡然出现,将小狐狸惊得飞逃。

%6
这两人,一位是黑袍男子,高大英武,一头黑发,双眸如墨幽深。另一位则是可爱粉嫩的女童,一身彩衣,眼眸如星,背后雪白的狐尾蓬松柔软,狐尾尖上细细的长毛随着微风微微颤动。

%7
不是旁人,正是方源和地灵小狐仙二位。

%8
“位置不错,就在这里吧。”方源扫视一圈,从空窍中取出洞地蛊。

%9
此蛊形如核桃,外表如木,坚硬且凹凸不平,大如西瓜。

%10
洞地蛊高达五转,至少得是五转巅峰的蛊师,全力调动,才能催动成功。

%11
方源自然不可以,便将蛊交给了地灵。

%12
小狐仙催动之后,洞地蛊顿时爆发出强烈的赤芒,一头扎进草地中去。

%13
霎时间,红光冲天,方面百里的大地都在颤抖。

%14
须臾功夫,红光乍然消失,大地裂开一条缝隙,长达二十七丈。

%15
缝隙两边土壤隆起,若是从高空往下看,就好像是一个人的嘴唇。

%16
紧接着,裂缝往两侧缓缓张开,露出两排紧凑的方块巨石,仿佛是人的牙齿。

%17
“齿关”也张开后。便露出里面幽黑的大洞。

%18
“主人,我饿……”地裂的嘴唇一张一合,竟然发出巨大的响声,带动附近地面都在微微的颤抖。

%19
方源笑了一声,取出元老蛊。

%20
真元调动,慈眉善目的云老人脸上的喜色渐渐消失,因为大量的元石都被取出来,投入到地裂大嘴当中。

%21
足足投去二十万的元石,方源这才停下。将元老蛊重新收入空窍。

%22
地裂大嘴缓缓闭上,两排巨石牙齿不断碰撞碾磨,将元石咬碎,磨成粉末。

%23
然后“咕咚”一声巨响,好像是人大口吞咽食物的声音。地表都因此震了一震。

%24
吞下这些元石后,地裂大嘴安稳了下来,不再响动。

%25
到这一步,洞地蛊算是成了。

%26
落在这里,就不能再移动。等到鹤风扬那边的子蛊也催动成功,两边就能相互沟通。

%27
喂养洞地蛊费用极大。一年内,要喂养它二十万块元石。

%28
而每次使用时。也都要消耗大量的元石。

%29
凡人蛊师哪个独自能用得起?除了大型的门派、家族之外,也就蛊仙家大业大,能独自豢养、催用这些洞地蛊了。

%30
“地灵,今后这个地方须得严加防守。你奴隶一些狐群。就在这一带生存。”方源看着眼前的地裂大嘴,布置道。

%31
“是,主人。”

%32
……

%33
“为了保护家园,我们必须战斗!”

%34
“那个该死的仙人。又要卷土重来,我们必须迎难而上。为了美好的未来。族人们,举起你们的双拳!!”

%35
“虽然我们和其他两族有些恩怨,但这都是咱们石人一族的内部的小矛盾。这一次我们三个部族,各自出动两万勇士,组成联军,往西进军,直捣仙人的老巢。”

%36
“这是一场伟大的战争,一切为了人民的利益。”

%37
“我们的父辈祖辈们抛头颅,散英魂,打败了仙人,才有我们现在安宁的生活嫌妻当家。我们要追随先辈的足迹,前仆后继,英勇作战!”

%38
地灵调动狐群,陆续出现在石人部族的周围。石人高层趁势鼓动,很快就组成了一支联军。

%39
联军浩浩荡荡,行进到福地西部。

%40
一路上小规模的战斗,进行了五六次,皆是石人大军胜利,狐群节节败退。

%41
“看那里!那就是恶魔的老巢!”岩勇挺身而出,来到洞地蛊的面前。

%42
“大地,你是我们的母亲,养育着我们石人一族。为何你要包庇那个可恶的仙人?”岩勇痛声疾呼。

%43
这时地缝大嘴张开,小狐仙运用蛊虫变声。

%44
石人大军便听到一个温柔的女子声音:“石人啊,我的孩子们。不是我要包庇那个仙儿,而是那个仙人进入了我的肚子里,盘踞在我的心脏中,威胁我庇护他。我这就张开嘴,请你们将他消灭。我将给你们赐福!”

%45
石人们震惊,旋即热烈地欢呼起来。

%46
“大地母亲她开口说话了!”

%47
“我们是被大地母亲赐福的勇士!”

%48
“仙人是多么的卑鄙啊,居然威胁我们仁慈而又温柔的大地母亲。我们一定会将他碎尸万段!!”

%49
石人联军士气大振。

%50
地裂大嘴旋即张开,岩勇一马当先,高呼道:“石人们,跟我冲!”

%51
说着,他就跳入洞口中去。

%52
“冲啊,不能让我族的英雄孤军奋战。”

%53
“冲进去,我们无所畏惧,我们无所不能,我们要屠仙!”

%54
“大地母亲都在我们的这一边,这一战我们必胜!”

%55
石人们一个个如饺子般,跳进地裂大嘴中。

%56
在幽深黑暗的隧洞中,他们往下坠落,片刻后,他们落到实地上。

%57
“这是哪里?怎么黑乎乎的一片。”

%58
“这里比地底还黑暗,我们什么都看不见啊。”

%59
“看不见怎么战斗?”

%60
石人们正疑惑着,忽然听到巨大的咳嗽声。一团光明,在他们的头顶上陡然开裂。急速的气流瞬间形成,夹裹着他们,将他们喷射出去。

%61
“两百三十只,两百四十只……”仙鹤门的弟子站在洞地蛊的子蛊旁边,细细数着喷出来的石人。

%62
石人被地缝大嘴喷出来后,摔在地上。第一时间就被仙鹤门的弟子控制住,无法反抗,无法动弹。

%63
岩勇和一些石人家老,站在一旁,垂眉低头地看着这一幕,噤若寒蝉。

%64
被方源戏耍后勃然大怒的鹤风扬,终究是顾全大局,没有在一气之下,将洞地蛊子蛊给捏碎。而是种在了飞鹤山上。

%65
至于石人……

%66
方源不过是利用了他们一族的地母信仰,然后再利用狐群,以及石人高层的配合,便顺利地将六万年轻石人,都诱骗到飞鹤山卖了。

%67
人是万物之灵。

%68
这世界中末世之灯焚造吉。除了正统人祖血脉之外,还有异人。

%69
异人虽然较其他生命聪慧,但远不及人族智慧。毛民懵懂,蛋人纯真,石人憨蛮……

%70
就算是七八岁的聪明孩童,都能哄骗了他们。不管是南疆,还是中洲。常常有这样的事情发生——什么地方的某某孩子,碰到毛民或者石人,一路哄骗到市场上。异人被卖了之后,还懵懂无知。给孩子数钱。

%71
这笔买卖结算后,方源只落了一百六十多万的元石。

%72
六万石人虽然各个年轻力壮,但却没有一只泥沼蟹值钱。方源还要支付先前炼蛊的材料,使用洞地蛊也损耗了六万五千块元石。同时他还收购了一些普通材料。

%73
鹤风扬为了报复方源,将价格提高了一成。方源要求的东西也被他削减了很多。

%74
方源也不在意,他真正想要的东西,已经在之前到手了。

%75
交接完成之后,岩勇等石人顺着洞地蛊回归狐仙福地。如何解释,方源也给他们安排好了,相信石人部族不会反弹。就算有反弹,那就杀了,再养一批,反正荡魂上的胆石还多着呢。

%76
方源沉下心来,继续修行。

%77
但当他到达四转巅峰,九眼酒虫就失去了作用。

%78
酒虫能提炼真元质量,但也只能提升一个小境界。方源到达四转巅峰,本身就有了巅峰期的真金真元,已经到顶了。再往上升,那便是五转初阶的淡紫真元。

%79
方源便将九眼酒虫封存起来。

%80
这只从青茅山开始,就陪伴他一路走来的蛊虫。从一转的酒虫,一路合炼成为如今的四转,现在终于功成身退了。

%81
单靠真金真元洗练空窍,方源的修行进度一下子缓慢下来。

%82
当然因为身在狐仙福地,对比外界的普通蛊师,方源的修行速度至少还是他们的五倍。

%83
“就这样修行下去,等到第七次地灾,我将至少有五转中阶的修为,接近高阶。”

%84
方源如今是甲等资质,突破五转不成问题。但蛊师修行,越到后期,提升的时间就越长。

%85
五转中阶,还是已经算上他手中的那只紫晶舍利蛊的结果。

%86
这种程度,比他同期的五百年前世,不知道好过多少倍。如今他三十有余,却已经是四转巅峰。五百年前世,他这个岁数,还在二转境界中挣扎呢。

%87
“但是这个速度,还是慢了。福地有第七次地灾,我更有春秋蝉这个大内患!”

%88
方源算了一下,最多三年,他就必须离开福地,进行一系列的冒险。至少要寻求到水到渠成蛊、或者马到成功蛊,来作用春秋蝉,求得更多生机。

%89
“唉!如果有可能,我多么希望和凤九歌一样,能在狐仙福地中一直修行,达成蛊仙境地后,再纵横天地。”

%90
春秋蝉能令蛊师重生不假,但限制极多。就算重生成功,也要防备着它撑破空窍。若无此点重大弊端,方源的修行会从容许多倍。

%91
“接下来的三年,就是一边炼蛊,一边用胆石再次壮大石人,不断贩卖,换取各种资源了。”

%92
但方源的计划虽好,却不及命运的捉弄。

%93
仅仅一个多月后,一场重大的变故,让他不得不将提前出走福地。

\end{this_body}

