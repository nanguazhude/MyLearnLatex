\newsection{兄弟相见}    %第十五节:兄弟相见

\begin{this_body}



%1
天梯山上,漏洞密密麻麻,极其频繁地接连出现。

%2
不仅是仙鹤门的精英弟子们目不转睛地盯着,同样还有许多蛊仙,隐藏在幕后,密切地保持关注。

%3
趁着这个功夫,方正又连续试了三次,终于将电文纸鹤蛊成功地射进福地中去。

%4
一只青鸟,展翅飞来,旋即也顺着漏洞,钻进了狐仙福地。

%5
“这是传信青鸟蛊!他凤九歌究竟想要干什么?”鹤风扬看到这一幕,面色一沉。

%6
然后在下一刻,他的双眼瞳孔猛地缩成针尖大小,嘴巴猛地张大,脸上充斥着极端的震惊之色。

%7
“我的天!这么大的一块福地,他居然都割了?!”

%8
鹤风扬瞠目结舌,呆如石像。

%9
方源割弃了整整一百万亩的福地,天梯山半山腰上都是福地的烟影,大片的草原覆盖了众人的视野。

%10
一个蛊仙,最快地反应过来,剑光一闪,现出真身。

%11
“哈哈哈,好大一片福地啊。它是我的,谁也别想跟我抢!”剑一生兴奋地大吼着,就要将这片版图,扯进自家的福地里,壮大自己的地盘。

%12
但就在这时,一道电光飞射而出。

%13
“我日!”剑一生猝不及防,爆了一句粗口,被魅蓝电影直接像颗炮弹似的打飞出去。

%14
剑一生也不是易于的,当即和魅蓝电影战成一团。

%15
声势猛烈,地动山摇,看得仙鹤门的一宗精英弟子全都傻了眼。

%16
更令他们惊愕的是,紧接着十多个身影,出现在场中,像是一群饿狼。闪电般地瓜分了这一百万亩的狐仙福地。

%17
“你们这群该死的贱货!”

%18
“老子引走了怪物,劳苦功高,你们居然不留点给老子!”

%19
“我操你们八辈子祖宗!”

%20
“我诅咒你们拉屎卡住屁眼,生儿子头上长鸡巴!”

%21
剑一生气得哇哇大叫,他平生还未吃过这么大的亏,被魅蓝电影追赶得好不狼狈。

%22
“还有方源小贼,真是恶毒,胆大包天,居然算计我!有种地就和我一战!”他射出飞剑传书蛊。

%23
飞剑传书蛊速度奇快。且有破空之能,就算是没有漏洞,也能射进福地当中去。

%24
仙鹤门众人一脸呆滞。

%25
这,这就是蛊仙的风范吗?

%26
“这个剑一生,真给我们蛊仙丢人啊……”鹤风扬都不自禁以手掩面。

%27
就在这时。亮起白金色的光芒。

%28
光芒中有一道朱红色的门楼,高达十丈,有九彩门匾。

%29
粉红色的祥云汇拢而来,七彩的虹光照在方正的身上。只是眨眼的瞬间,方正就消失在了原地。

%30
将魅蓝电影,或者荒兽泥沼蟹直接挪移到福地之外,已经超出小狐仙的能力范围空间之悠然田居。但是要挪移一个方正。还是可以的。

%31
“进去了!”看到这一幕,鹤风扬心头一振。

%32
一道闪电霹雳,从天而至,正是魅蓝电影。但白金光辉带着朱红门楼。猛地收拢。

%33
差了少许,魅蓝电影想要冲进狐仙福地的企图,没有得逞。

%34
方正只感觉眼前一花,再定睛一瞧。周围的景象已经有了大变。

%35
他身处草原之上,脚边都是绿草茵茵。头顶上云海重重,投下浓重的阴影。不远处还有几处湖泊,波光粼粼。

%36
“到狐仙福地里了。”方正迅速反应过来,他身上的蛊虫都被禁锢了,一如他刚开始进入福地时的情形。

%37
一团烟影在他的面前,升腾而起,扩张成落地镜面大小。镜中显现出方源的身影,他坐着,背斜靠着椅背。翘着二郎腿,左手搭在翘起的膝盖上,而右手肘则撑在宽大的扶手上,手掌托着他的脸颊。

%38
一头黑发恣意地垂下,双眼半眯着,闲适慵懒的动作神态,却给人一种危险的邪魅阴暗感。

%39
“我可爱的弟弟,想不到居然能在中洲见到你。”方源开口道。

%40
他的声音,对方正来讲,是多么的陌生,又是多么的熟悉。

%41
方正身躯一颤,旋即双眼中爆发出浓烈至极的仇恨,低吼道:“古月方源,你个丧心病狂的恶魔,屠杀亲族的刽子手!我要亲手杀了你!”

%42
说着,他冲向方源。

%43
但这个“方源”只是一团光烟呈现出来的影像而已,方正扑扇了烟影。很快,散去的烟影又汇拢起来,形成完好无损的方源影像。

%44
方正手指着方源,叫道:“方源,你连亲自见我的勇气都没有吗?你这个懦夫!无耻的叛徒,毫无人性的畜生!大不了是死而已,你却为了苟且偷生,将亲族都杀了。这种大逆不道的事情,你怎么能做得出来?!你还是人吗?!”

%45
“呵呵呵。”方源朗笑几声,惬意地靠着椅背上,“我亲爱的弟弟,想不到你还是这般的愚蠢。不管我动不动手,他们的下场都是死。既然如此,为什么我就不能活着?没有我的反击,你以为你会被人带回中洲?相反,是我救了你啊。我可是你的救命恩人呢。”

%46
“放你的狗屁!歪理邪说,无耻至极!”方正听到方源自诩为他的救命恩人,鼻子都气歪了。

%47
方源唇边的笑意渐渐止住,他叹息一声:“方正,我的弟弟,你真是令我失望啊。这些年来你一点长进都没有。你的修为再高,也不过只是个别人的棋子罢了。好了,谈正事吧。仙鹤门的来信,我已经看了。什么许诺我长老之位的鬼话,今后就不要再说了。反倒是交易,我们可以做一做。”

%48
方正胸膛起伏不断,鼻息粗重,目光含恨,瞪着方源的影像。

%49
这对兄弟,面貌如此相似,几乎一模一样,身上更有最亲切的血脉联系。可惜,他们却是生死的仇敌。

%50
方正狠狠地喘了几口粗气,终于按捺下心中对方源的澎湃杀意。想到门派对自己的命令:“狐仙福地中,狐群、蛊虫我派并不感兴趣。但是荡魂山上的胆识蛊,还是有一定价值的。我派会陆续调遣弟子前来,你将他们接引到荡魂山上……”

%51
“停下。”方正的话还未说完,就被方源打断,“我还不相信你们仙鹤门的诚意。”

%52
“这是我要的东西,你们先给我备上,尽快地交给我。元石我没有,不过我却有荒兽泥沼蟹的尸体可以代替元石交易。详情都在信中。回去好好看看肥田仁医傻包子。”

%53
话音刚落,一道细小的电光飞射而来,落到方正的手中。

%54
却是那只电文纸鹤蛊。

%55
这只电文纸鹤蛊,已经被方源当仁不让地炼化,收为己用。里面的内容就是方源要的蛊虫、各种材料,以及泥沼蟹身上血、肉、骨骼、甲壳等等。

%56
方正抬起头,刚想要张口说话,忽然眼前景象大变——他又被传送出去了。

%57
“检查一下,没有什么不妥之处吧?”方正走后,方源却没有放松下来,而是叮嘱地灵。

%58
福地是无法禁锢仙蛊的。方源没有亲自面见方正,就是担心他的身上藏着仙蛊。

%59
仙鹤门家大业大,仙蛊绝不会少。

%60
方正虽然空窍不足以装载仙蛊,容易让仙蛊的气息流露出来。但蛊师世界千奇百怪。将仙蛊气息隐藏的手段,也有许多。这点方源不得不防。

%61
地灵检查了几遍,没有问题,方源这才放下心来。

%62
“渡过了地灾。算是否极泰来么?”方源眯起双眼,考虑着自己的处境。

%63
眼前的局面。比他原本料想中的,还要有利得多。

%64
仙鹤门为了吞下狐仙福地,竟然为方源这个敌人打掩护。如此气魄,真不愧是中洲十大门派之一!

%65
一切以利益为先,什么敌人、友人,都是建立在这个基础上。

%66
用俗话来讲,就是所谓的“大局观”。体制束缚下,大局观的要求下,方正就算再仇恨自己,又如何呢?还不是乖乖地和自己商谈交易?

%67
“一旦发觉拿捏不住自己,强攻会失去一切,仙鹤门就来和谈,来交易。就算被人发现,也不会说什么正魔勾结。因为仙鹤门已经承认,我就是他们门派中的弟子了!倒是算计得精细。”

%68
“不过,这也正是我需要的。哪怕这个弟子的身份如此虚假,却也足以震慑其他势力。看看剑一生和凤九歌的来信,就知道这个身份的宝贵之处。”方源心中思索,他并不介意这个身份。

%69
实质上,他还是魔道,还是个人,行事还是潇洒恣意,没有人能束缚住他。

%70
但同时,他又能交易,换取所需的其他资源。

%71
“本来,我还是想着前往琅琊福地,抢夺通天蛊。现在有仙鹤门交易,却不用多此一举了。倒是我抢夺了狐仙福地,仙鹤门绝不会善罢甘休,此时和谈商讨,是对我投鼠忌器,没有办法。我绝不能大意麻痹,给他们可乘之机。”

%72
方源暗暗告诫自己,至于弟弟方正,倒是次要的。

%73
杀了他,最多是用血颅蛊,提升自己一些资质罢了。坏处却是交恶了仙鹤门,也将自己置于险境。

%74
屠杀自己的亲弟弟,这是纯粹的魔道行径,一旦被外人所知,就会解读成方源背叛仙鹤门。到那时,十大门派还有无数魔道蛊仙,都会将贪婪的目光,集中在狐仙福地上。

%75
天下没有不透风的墙,一旦事情泄露,就算是仙鹤门想要做戏,也做不成了。

%76
方源目前的局面,提升资质已经是次要的了。

%77
就算是提升再多的资质,也需要资源来修行啊。

%78
所以关键还是如何稳住局面,充分地利用好福地中的资源,转换成自身的实力!

\end{this_body}

