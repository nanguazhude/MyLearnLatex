\newsection{灰白石板}    %第九十节:灰白石板

\begin{this_body}

%1
这处的水狼巢穴中,居住着四千多头水狼,是一支千兽群。

%2
水狼虽然很少踏足陆地,但是家园被外敌入侵,那又是另外一回事情了。

%3
察觉到方源的狼群汹涌而来,这支野生水狼群从巢穴中,一齐悍然出击,和方源的狼群绞杀在一起。

%4
受到阻碍,方源麾下的狼群,冲势一滞。

%5
但他冷哼一声,催动狼嚎蛊,同时又调派三路援兵,支援上去。

%6
野生水狼群只坚持了片刻,就承受不住这种压力,立即被冲散,大势已去。

%7
远处。

%8
“族长,我们的东西还留在那里呢。”一位柴家家老不甘心地眺望着。

%9
柴家族长柴章深深地叹了一口气:“罢了,丢了就丢了,总好过丢了性命。”

%10
“我们是不是留下来再看看?常山阴这样的人物,也许看不上这三头黑皮肥甲虫呢。”另一位柴家家老,抱着侥幸心理。

%11
但柴章却看得明白,冷哼一声:“你如果不担心冒犯了常山阴,惹来他的屠杀,你就留下来吧。”

%12
这位柴家家老顿时脸色一僵。

%13
“哼,这个想法,你以为仲家会想不到吗?就算常山阴看不上眼,我们也得不到这些物资的!唉,有常山阴这样的强者在这里,这月牙湖我们是不能呆了,还是继续出发,赶紧走吧。”柴章摆了摆手,语气中有无奈,有愤恨,更多的是无力。

%14
柴家只是小型部族,实力薄弱。尤其是在十年风雪来临之际,王庭争霸,龙蛇起陆,整个北原都是纷争不断,一片乱世。

%15
像柴家这样的部族,就像是乱世漩涡中的小舢板,风雨飘摇,随波逐流。只有依附更强大的势力,才能增加自己生存下来的概率。

%16
柴家拔营而走,走得十分干脆。

%17
片刻之后,仲家的侦察蛊师,带着一脸余悸,向仲费尤回报道:“大人,狼王大获全胜,翻手之间,就剿灭了那处狼巢。四千多头水狼,他收编了近三千头,而他仅仅只损失了三百头。”

%18
仲费尤以及仲家高层,听了都浑身一震。

%19
这样的战损比,着实恐怖!难怪他狼王的狼群,能够这么快就得到了补充。

%20
“族长大人,你是没有亲眼所见。常山阴的指挥,已经超凡脱俗,简直升华成了艺术!”侦察蛊师擦了擦头上的冷汗,补充道。

%21
仲费尤冷哼一声,不愿堕落了自家气势,强撑道:“常山阴手中,有一头水狼万兽王。而这支野生狼群的首领,不过是一头千兽王。一旦交战,水狼群便会受到万狼王的压制,战力被削弱。收编也更为容易。那三头黑皮肥甲虫,去向如何?”

%22
侦察蛊师便答:“都被常山阴夹裹了去。”

%23
仲费尤的脸色顿时难看起来。

%24
他这次是偷鸡不成蚀把米,不仅没有抢夺了物资,反而恶了柴家。

%25
说起来,柴家和仲家还是亲家姻亲,之前关系紧密。否则,也不会一起迁徙,驻扎营地时,都选择比邻而居,守望互助了。

%26
但是,现世是残酷的。

%27
如今王庭之争,对于仲家、柴家来讲,不仅涉及到利益,而且还关乎两族的生死存亡。

%28
以往的情意,不过是维护利益的手段罢了。到了抛弃的时候,就会毫不犹豫地抛弃掉。

%29
王帐内,一阵压抑的沉默。

%30
良久,仲费尤才吐出一口浊气:“常山阴这样的人物,哪怕集齐我们仲家上下全力,也难以匹敌。但北原却绝对不是他一家独大,比他强大的奴道大师多达三人!这场我们先记下来,等我们投靠了刘文武公子后,迟早有一天,会把今天的场子讨回来!”

%31
仲家家老们纷纷点头应是。

%32
不久后,仲家上下也拔营启程。

%33
一晃九天后,方源率领着壮大数倍的狼群,回到葛家营地。

%34
葛光率领着葛家高层,主动出迎十里地。

%35
“太上家老大人,您的修为回复了?!”当葛光察觉到方源四转巅峰的气息时,顿时又惊又喜。

%36
方源点点头,淡淡地回应道:“回复了,也该是时候回复了。”

%37
当年常山阴的修为,就是四转巅峰。后来和哈突骨马匪大战,伤重濒死,藏眠地底。

%38
不过现在,方源的第一空窍,已经达到五转巅峰。虽然仍旧受到北原的压制,但仍有五转初阶的气息。

%39
现在的四转巅峰气息,只是用了敛息蛊,故意伪装而成的。

%40
至于他的第二空窍,因为首先出入北原,被北原承认,倒并没有受到异域压制,仍旧是三转巅峰。

%41
就这样先收敛着,然后一步步的释放气息,不仅可以保留底牌手段,而且还能给他人一个循序渐进的接受过程。

%42
方源跟随着葛家高层,一路回到营地当中。

%43
葛家营地正在扩建,一路上看到的都是热火朝天的施工景象。大量的凡人奴隶,甚至蛊师奴隶,都被葛家族人恣意地调遣着。

%44
成王败寇,这就是战争的残酷,也是战争的美妙。

%45
葛家高层均是一脸洋洋喜气,葛家吞并了贝家、郑家,实力膨胀得厉害,这些天来一直在努力消化着,整个家族的实力也跟着上涨许多。

%46
“现在最大的麻烦,就是奴隶蛊稀缺。若是有大量的奴隶蛊的话,我们就能将这些奴隶蛊师放上战场。这将极大地增强我们葛家的战斗力!”葛光感慨地道。

%47
奴隶蛊,是能够操控人的蛊虫。

%48
但人是万物之灵,比野兽要难操纵得多。对魂魄的负担更大,尤其是要奴隶那些魂魄强盛的蛊师。

%49
因此,基本上,一位蛊师很少操纵,超过五个的奴隶。奴隶蛊师的数量,就更少。往往一位蛊师,驾驭一位奴隶蛊师,再多的话,魂魄负担就大了。

%50
而那些魂魄强盛的奴隶蛊师,要操纵他们,非得要比他们的魂魄更加强大。

%51
方源当然有能力,搞到大量的奴隶蛊。

%52
但这样一来,他就暴露了很多东西。葛家在他的计划中,不过只是一枚棋子罢了,犯不着为他们如此着想考虑。

%53
“接下来,我要继续闭关,进行修行。这些狼群,你们就代为照料。”方源开口道。

%54
“是。”葛光连忙恭谨地答道,心中却是暗暗叫苦。

%55
现在葛家扩张膨胀,正是用人之际,劳力紧缺。狼群庞大了,喂养的负担就更加沉重,这得消耗葛家多大的劳力啊!

%56
但方源下一句话,又说得这位葛家年轻的族长心花怒放——

%57
“我这次带来了许多物质,都是收编野狼时,顺带收刮的。你们都用着,但要记住,那三头黑皮肥甲虫身上的东西,务必给我好好保管。”

%58
“是,太上家老大人!”

%59
接下来的日子里,方源就在葛家营地中,深居浅出,刻苦修行。

%60
他的第二空窍,需要继续提升修为。魂魄上虽然是有千人魂的成就,但仍旧需要狼魂蛊,不断地强化,直到强化成千人级的狼人魂。

%61
同时,他的力道也要继续提升,钧力蛊不断地使用着。

%62
需要放松的时候,他就取出那三只黑皮肥甲虫身上的东西,进行查看观赏。

%63
柴家辛苦收集的这批物资,十分奇怪,都是一些灰白石板。

%64
但这些石板表面,描绘着漆黑的墨线,有直有曲,有粗有细。墨线纠缠在一起,有的类似文字,有的仿佛山水画面的景象。

%65
若这些石板是真的,那来头可就大了。往源头推溯,那可得追究到太古年间,人祖的九女逍遥智心。

%66
《人祖传》中记载,逍遥智心为了救活智慧蛊,便来到乾坤晶壁前。

%67
乾坤晶壁直上直下,屹立在虚空当中,宛若一面巨大的镜子。

%68
镜子中,有一座书山。

%69
书山上,有一道墨瀑垂下,砸落在山石中,形成文泉。

%70
墨瀑不断垂落,重重地砸在文泉中,激起万千的水花。这些黑色的水花,散漫在空中,一颗颗水滴,化为一个个的文字。

%71
这就是蛊师世界中,百族文字的来源。

%72
后来乾坤晶壁被破,就分裂成无数块的灰白石板。

%73
传闻,当集齐了所有的石板拼凑在一起,就能重组乾坤晶壁,使得蛊师能再进书山。

%74
翻开人族的历史就会发现,历代的蛊师,蛊仙,乃至仙尊魔尊,都有收集过这些石板的记录。

%75
也正是因为如此,很快就会出现了大量的石板的仿制品。

%76
这些赝品石板,和正品石板很难分辨,除非是经验丰富的鉴宝蛊师。、

%77
历史上,最具权威,也最有成就的鉴宝蛊仙,就是宝黄天的主人,拥有宝光蛊的那位——多宝真人。

%78
但就算是他,也只能辨认七八成。

%79
赝品石板实在太多,有太多的蛊师仿制,甚至当中还有盗天魔尊。

%80
盗天魔尊特意伪造了许多石板赝品,哄骗了许多蛊仙上当受骗。他制造的赝品石板,极其逼真,甚至可以把真品比下去。

%81
方源从未没有想过,将灰白石板收集完全,然后拼凑出书山来。

%82
哪怕是九转蛊尊,都没有成功过的事情,方源绝没有自不量力的冲动。

%83
他只是在休息的时候,试图鉴别这些石板。

%84
在前世,他经商养成了犀利的目光,也曾经贩卖过,伪造过这种灰白石板。

%85
如今鉴别这些石板,从中辨别出真假,剔除明显的赝品,也算是一种放松和消遣。

%86
但没有想到的是,就在他摩挲一件石板的时候,忽然发生了意外。

%87
这件已经被他判定成赝品的石板,在方源真元的灌注下,表面的墨线忽然流动变幻了起来。

\end{this_body}

