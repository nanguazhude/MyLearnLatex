\newsection{察运蛊}    %第一百九十九节:察运蛊

\begin{this_body}

%1
大半个月后。

%2
八十八角真阳楼。

%3
风镰鸟密密麻麻,数不胜数,铺天盖地一般,向联军扑杀下来。

%4
这些风镰鸟,长着尖锐如剑的鸟喙,翅膀弯曲锋利,仿若镰刀。它们飞行极为迅速,在空中划出道道残影,冲锋起来悍不畏死。

%5
“杀杀杀!”黑楼兰大吼,他身上黑色烟气翻腾不休,分出无数的黑色触手,将周围的风镰鸟一一擒拿、吞噬。

%6
激战已经持续了半个时辰,大军和风镰鸟剿杀在一起,双方都是伤亡惨重。

%7
地上是铺了满地的鸟尸,同样还有死去的大量蛊师。

%8
“啊啊啊啊……”黑楼兰咆哮,杀红了双眼,他猛地伸出右手。

%9
右手呈爪,一道道漆黑暗流凭空而生,在空中轻轻一个盘旋,随后接连汇集到他的右爪爪心处。

%10
眨眼间,成百上千道暗流,如百川归海般,在黑楼兰的右爪中汇聚成一个黑色的光球。

%11
光球深黑莫测,不断飞速自转,竟然发出风雷一般的呼啸之音。

%12
“去!”黑楼兰陡然怒目圆睁,大喝一声,鼓荡全身力量,猛地抬起右爪对准上空。

%13
他抓着圆球黑光,宛若举着千钧重物,整个动作迟缓吃力。

%14
随着这个动作,光球缓缓飞上天空。

%15
在半空中,它迅速膨胀,短短几个呼吸的功夫,竟然直接膨胀到小山一般巨型!

%16
一时间,黑色光球占据了联军头顶上的小半个天空。

%17
无数风镰鸟被其包裹,随后瞬间消磨融蚀,连骨头渣子都没有剩下。

%18
整个过程悄无声息,诡异霸绝,让人看了心中发寒。

%19
正是黑楼兰的招牌杀招——暗漩!

%20
暗漩持续了六个呼吸的时间,杀死数万头风镰鸟。

%21
风镰鸟群数量太多,又都聚集在盟军的上空中,因此杀伤数量惊人。

%22
暗漩消失之后。天空中出现一个清澈的大洞。但很快,周围的风镰鸟群像是流水一般,补充进来。整个蛊师联军的上空,又重新被风镰鸟群占据。

%23
这已然是八十八角真阳楼。第六十八层的第二十五关。

%24
风镰鸟的数量实在太多了,其中有包含大量的鸟王,阻击力量过于庞大。

%25
其实按照众人之前的经验,不难猜到攻略此关一定别有技巧,但黑楼兰却直接硬冲猛上,调动他所有能够调动的力量,直接平推。

%26
这就引来了风镰鸟的疯狂反扑。

%27
“哈哈哈,杀光这些死鸟!”那边耶律桑发出大笑。黑楼兰的爆发,勾动了他的战斗豪情。

%28
耶律家和黑家齐名,同样是超级势力。拥有蛊仙坐镇。耶律桑和黑楼兰地位相等,都是此代超级势力的当家族长。

%29
现在他见黑楼兰使出了杀招,杀伤了海量的风镰鸟,他自然也不甘示弱。

%30
“看我新研究出的杀招——炎魔!”

%31
耶律桑大叫一声,衣袖翻飞。真元爆发,疯狂催谷手中蛊虫。

%32
腾的一声,他浑身冒火,火焰冲天,逼开周围蛊师。

%33
方圆数百步内,只剩下他一人。

%34
他浑身一抖,身上冲天般燃烧的熊熊火焰。便分化出一部分,形成一个完全由火焰组成的三丈金刚。

%35
火焰金刚面容丑恶,獠牙怪角,肌肉贲发。正是火魔!

%36
火魔冲天而起,在天空中横冲直撞,但凡遭遇到风镰鸟。都是一点即燃,烧得鸟群哇哇直叫。

%37
沿途过去,火焰冲天,连成一线。大量的风镰鸟转瞬之间,就被烤熟。砰砰砰的落到地上。

%38
火魔肆虐八方,立即吸引了众多鸟群的注意力,为地面上的蛊师大军减轻了许多压力。

%39
“什么火魔,也不怎么样啊!”远处,黑楼兰大笑一声,得意张扬。

%40
耶律桑同样嘎嘎大笑:“黑家族长,你再看看!”

%41
说着,他浑身连连抖动,将火焰抖落,形成一个个火魔。

%42
黑楼兰面色一变,不仅是他,就连大军中的方源也是目光一紧。火魔威力如此惊人,居然不止一头。

%43
八十八角真阳楼带来机缘,不仅是方源、黑楼兰等人在进步,其他的蛊师也都大有斩获。

%44
当六头火魔加入战场后,耶律桑身上的火焰,这才完全被抖落光。

%45
“六头火魔,还好……”方源收回目光。

%46
黑楼兰的面色也稍稍缓和下来。

%47
耶律桑有才情,这个杀招威力不俗,但也仅仅如此。充其量,只能和暗漩媲美。

%48
火魔持续肆虐了半个时辰,威力越来越小,最终湮灭在风镰鸟群当中。

%49
但之后,方源、太白云生、奚雪、裴燕飞、陶幽、古国龙等人相继出手,杀招频出。五转蛊师的强大战力,改变了战局,令胜利的天平严重倾斜向联军。

%50
最终,他们成功通关,闯进第二十六关。

%51
“启禀族长大人,有十三位族长联合上书,请求暂停进军。他们担心伤亡过重,急需休养。”黑楼兰还未仔细地打量此关,他的亲信黑书就上来汇报。

%52
黑楼兰顿时皱起眉头,喝斥出声:“这点小小的伤亡,就喊累抱怨了!一群饭桶,无能的懦夫!这道上书驳回,告诉他们,必须前进。谁敢擅自后撤逃离联军,株连九族,杀无赦!”

%53
说着,他将族长的联合上书夺到手中,三下五除五,撕成碎末。

%54
“继续前进!往前冲!”黑楼兰挥起手臂,咆哮连连,指挥大军。

%55
他心中焦急。

%56
这些天来,八十八角真阳楼每隔一段时间,都会发出诡异惊人的震荡。

%57
再结合之前楼主令的神秘丢失,黑楼兰觉得事情很不简单,敏锐的直觉告诉他,好像有什么不得了的大事情发生了。

%58
为了避免夜长梦多,他必须尽快取得力道仙蛊。晋升成力道蛊仙之后,他才有实力,掌控隐隐大变的局面。

%59
这一次攻略,一直进攻到第二十八关。黑楼兰见众人实在支撑不住,伤亡巨大。必须休养,他只好意犹未尽地无奈收兵。

%60
……

%61
房间中,太白云生面泛苦笑,看着手中的几只东窗蛊。

%62
这些东窗蛊中。记载着蛊师先贤们的修行心得。其中大部分,竟然都是宙道心得!

%63
这些都是来自八十八角真阳楼的通关奖励。

%64
“难道说,这是命运的安排,要我晋升蛊仙吗?”

%65
太白云生之前,寻求寿蛊失败,耗尽身家,只能成为黑家外姓家老。

%66
他是宙道蛊师,五转巅峰,对时间感应敏锐,好些年前就察觉到自己年岁无多。

%67
晋升蛊仙。虽然能令蛊师的生命得到质的升华,但并不能增长寿命。不过太白云生的情况特殊一些。

%68
他的宙道传承,来历非凡,乃是蛊仙传承。

%69
此套传承的核心有三个,分别是江如故蛊。山如故蛊,人如故蛊。

%70
传承内容早已有推论:成仙之时,利用天地交感的无上良机,便可令人如故蛊成为仙蛊。

%71
凡蛊人如故,只能针对他人运用。

%72
仙蛊人如故,却能针对自身有效。

%73
“如果我有了人如故仙蛊,对自己不断催用。就能延寿了。可惜,人如故不是春秋蝉,却是救活不了高扬、朱宰。唉……”

%74
一想到这里,太白云生便揪心起来,强烈的愧疚和悔恨之情,袭上他的心头。

%75
……

%76
方源立在高塔顶端。遥望着黑家方向。

%77
察运蛊一直在催动,带给他与众不同的新奇视野。

%78
这只五转凡蛊,正是他在真传秘境中的唯一的意外的收获。

%79
察运,顾名思义,就是能视察运气。这是一只侦察蛊虫。

%80
在他的视野中。黑家生活的区域,各有无数如烟的运气。其中又有两道,极为浑厚,如鹤立鸡群,就算是宫殿房屋也遮蔽不了。

%81
一道来源于黑楼兰,气运宏大如巨柱,青云之色,绵延悠长。一道来自太白云生,气运仿若傍晚时分的火烧云一样,红光照人,积蓄在他的房屋上空,给人火焰般绚烂燃烧的感觉。

%82
“这两人运气正旺,但又有分别。黑楼兰身上的气运,给人一种持久之感。太白云生的气运,则像是一堆薪柴,在最后关头剧烈燃烧。这些天来,我暗箱操作,提取许多宙道心得,作为奖励,特别发放给太白云生。每一次他得到奖励,头顶上的火云运气,就旺盛鲜红一分。想必成仙的可能,就增大一分。”

%83
方源琢磨良久,心中不禁感叹这运道的神秘玄奇。

%84
但他只得到一只察运蛊,不知道其他传承内容,等于开了一扇全新的窗户,面对这个全新的流派,他只能慢慢摸索。

%85
再动用察运蛊,看他自己。

%86
方源的身上,缠绕着漆黑如墨的运气。这些运气,形成一个庞大的棺椁形状,将其浑身都覆盖在里面,散发着浓郁的死气、凶气。

%87
饶是方源已经查看过许多次,每次看到这里,他都是心中一沉。

%88
“我的运气看来不容乐观。不过,这些天来我劝说王庭地灵,丝毫没有进展。这个地灵太过高傲,一心想要冲破阻挠,恢复自由。真阳楼已经颤动了三十八次,终有一刻,巨阳意志会被惊醒的。”

%89
“嗯?”方源轻咦一声,他遥遥望见一群蛊师,各个身罩着小黑棺气运,向着黑家方向走去。

%90
片刻之后,他们都进入了太白云生的住处。

%91
方源稍微思索了一下,便明白这些人的去意。

%92
“正好可以借此机会,查明这黑棺气运的意义。”方源目光幽幽,对太白云生的方向,投去深深一瞥。

\end{this_body}

