\newsection{先挑狼王}    %第一百零三节:先挑狼王

\begin{this_body}

魂道的开辟者,傲居蛊师九转巅峰的传奇人物――幽魂魔尊,曾经评价过:

“天下之大,壮魂首选荡魂山,炼魂首选落魄谷。一山一谷若得之,则必可魂道大成,纵横世间不在话下!”

因此,荡魂山、落魄谷并称为魂修二圣地。

这两大圣地,究竟能带给蛊师多么巨大的帮助,方源自从掌握了荡魂山后,就深有体会。

他的千人魂,就是通过催用了荡魂山上的胆识蛊,才修炼而成。

胆识蛊是用于壮魂的极品第一蛊,直接增长底蕴,没有任何的副作用,效率极高。

普通的蛊师,若是将魂魄积累到千人级,通常需要二十年左右的时间。一些天才,有家族的支援,长辈的照顾,可以将二十年的时间缩短一半。

而方源呢?

他利用荡魂山上的胆识蛊,将魂魄壮大到千人级,只用了半年都不到的时间。

别忘了,这还是在“荡魂山受到和稀泥仙蛊的侵蚀,渐渐死亡”的基础上。

方源因为荡魂山,魂魄轻而易举积累到千人级,速度简直就是坐上火箭直冲九霄。但魂魄的修行,除了壮魂之外,还要炼魂,将魂魄提纯。

这个方面,方源的进展就缓慢得多了。

和壮魂的速度相比,他炼魂的速度简直像是乌龟在爬。

方源炼魂,一直采用的是狼魂蛊,将魂魄提纯改造,最终形成狼人魂。

但他采用过的狼魂蛊,从未有达到五转级数的,最高是四转。四转狼魂蛊提纯千人魂,好像是一瓶墨水,倒入湖中。想要把整个湖泊染遍,四转狼魂蛊的效率实在太低了。

方源之前,也努力追寻过五转的狼魂蛊,可惜没有成功。

没有五转狼魂蛊,其实还有一个方法。

那就是利用两更蛊、三更蛊,增加自身时光的流速,或者直接进入福地,从而达到加快自身修行速率的目的。

但这种方法,旁人可以使用,但方源却不能。

方源的第一本命蛊春秋蝉,正随着时间缓慢恢复。在方源没有成就蛊仙之前,它就是一个索命的刀刃,一直悬浮在方源的脖颈上。

“我现在的千人魂,完全受益于荡魂山。落魄谷与荡魂山并驾齐驱,我如果能得到它的话……”

一时间,方源心中甚至涌动起一股,干脆改走魂道的冲动。

“若是救活荡魂山,再掌控落魄谷,有魂道两大圣地这样的雄厚资本,改修魂道真的是明智之举。甚至比重修前世的血道,还更加前途光明远大!”

但旋即,方源又冷静下来。

“按照念头中的指示,落魄谷距离遥远,当务之急还是救活荡魂山,现在还不是前往落魄谷的时机。大战在即,而我现在这身力道、奴道的积累,也不能随便丢弃。”

方源靠着奴、力两道的修为,成为北原的风云人物。

但以他如今的实力,距离在凡间纵横无敌,还有相当遥远的路程。

一个影剑客,就让他灰头土脸。

虽然有第二空窍,力、奴兼修,五转巅峰修为,但方源置身在王庭之争这样的大环境下,仍旧显得渺小。

这场波及整个北原的战争漩涡,只要稍有不慎,饶是五转蛊师,也有陨落之危。

“如今我奴道有成,能影响整个战局。但明显是攻强守弱。力道方面,还不能做到自保。一旦被墨狮狂、边丝轩这样的人物近身,就麻烦了。和东方家的战斗,还是要谨慎。”

想到即将要展开的大战,方源并没有其他人的战意沸腾。

狈君子可以说是帮了他一个小忙,让他居于幕后,有更多的时间进行修行,增长战力。

接下来的日子里,方源一边温养第二空窍,一边炼制呕心婴泣蛊,同时和小狐仙通信,主持处理福地中的大小事务。

狐仙福地中,荡魂山的情况继续恶化,整个山峦日渐缩小。小狐仙每天都会从荡魂山上,清理出大量的和稀泥。尽自己最大的努力,来拖延荡魂山的生机。

星云覆盖福地东部,星萤虫群的规模,又在原来的基础上,上涨了三倍。根据小狐仙的初步估算,多出了五六十只的星萤蛊。

能在这么短时间内,一下子多出这么多的星萤蛊来,多亏了气泡鱼的作用。

这些气泡鱼,已经渐渐产生作用。

之前方源频繁出入狐仙福地,长时间维持星门蛊,导致星萤蛊数目降落谷底。现在星萤蛊数目增长上来,很是缓解了压力。

而在福地的西部,花粉兔大量的繁殖。

方源将之前的狼群,都抽调到北原中来。花粉兔的压力骤减,因此兔群规模上涨得很快。

小狐仙汇报了这个情况之后,方源立即将东部湖泊中大部分的水狼,都调到西部,弥补食物链的空缺。但就算如此,兔群的规模仍旧在涨。

为了防止兔灾,就在前几天,小狐仙往宝黄天中低价出售了一大群花粉兔。

方源最关心的毛民,暂时定居在福地南部。

这里原本是石人的故乡,现在骤然多了一批毛民,双方争夺生存空间,发生过几场小型冲突。

小狐仙在方源的叮嘱下,暗中帮助毛民,战胜了一个石人部落,并将石人俘虏转手卖给了仙鹤门。

仙鹤门方面,再三提出要针对胆识蛊的交易,都被小狐仙拒绝。方正作为谈判的代表,几次要求面见方源,也都被拒之门外。

至于宝黄天中,和稀泥又卖出一次,收获了第二份和稀泥仙蛊的蛊方。

之前方源出售的有关仙蛊的各大残方,隔了这么多天,小狐仙又转卖了一次,获得了十一块仙元石的进项。

相同的蛊方,在宝黄天中出售得越多,越多蛊仙得到,宝光就越低。因此,根本不能作为一项长久的收益。

这就像一个金矿,已经开采了一大半,今后的收益将日渐薄弱,并不值得过多的期待。

又对峙了三天时间,东方余亮亲手一封战书,传达到黑楼兰的手中。

这让黑楼兰惊愕了一下,问左右道:“难道东方家的后军,已经赶到了吗?”

狈君子孙湿寒便答道:“对方的后军还远在五千里之外,正在修建第五道防线。”

黑楼兰狰狞一笑:“东方家原本军力就低于我们,居然还敢分兵!”

孙湿寒也笑道:“东方余亮这是在玩火。我们不妨稍等片刻,等到后军汇集,军力将大大领先对方。到时候,再一举压上,将对方杀得人仰马翻。”

黑楼兰眼中凶光闪烁了几下,他和东方余亮有着私仇,年轻时行走天下,增长阅历时,就垂涎东方晴雨的美色,但被东方余亮好好教训了一番,吃了很大的苦头。

他虽然极想报仇雪恨,但也并非被情绪轻易支配的人。

“东方小儿的意图,傻子都看得明白。他想要战,我偏不给他机会。我方后军还有多久,才能赶来?”黑楼兰又问。

“大约三天的时间。”汪家族长在一旁答道。

“好。我便修书一封,约战东方余亮,四天后大战!”黑楼兰哈哈大笑一声。

东方余亮接到信后,交给文武诸将浏览。

东方盟军的高层,都被气得不轻。

黑楼兰在信中大放厥词,恣意张扬,宣称自己大发慈悲,多给东方余亮三天的时间,希望东方余亮不要辜负他的美意,好好享受一下最后的人生。

众将纷纷请战,但东方余亮却从容淡笑:“诸位稍安勿躁,此信早在我意料之中。这些天来我推演多次,已得出一计,诸位听我详细道来……”

四天的时间,一晃即逝。

决战的这一天,风和日丽,碧空万里无云。

深可及膝的青草丛生,双方展开军阵,绵延百里。旌旗如林,兵马如蚁。

双头犀牛宛若小山,背上王帐中,坐着黑楼兰、方源、浩激流、汪家族长、房家族长、叶家族长等等强者。

方源的位置,自然是左首第一。

而狈君子孙湿寒,则站在黑楼兰的身后,一脸的忠诚,俨然已经成为黑楼兰的心腹。

风声在耳边呼啸,刮得战旗猎猎作响。方源安坐着,举目遥望,只见对方军容齐整,王帐立于一朵白云之上,悬浮于半空当中。

王帐内,依稀可见东方余亮端坐中央,各个文臣武将分坐左右。以气势看,丝毫不弱于黑楼兰一方。

这时,方源的耳边蓦地出来黑楼兰的大笑:“哈哈哈,今日一战,便是我黑家纵横北原,踏上王庭主位的第一步。诸位,谁与我上前,挑动第一战?”

话音刚落,一批蛊师纷纷起身离座,大声叫嚷着,或者猛拍胸脯,要求出战。

黑楼兰的目光逡巡一番,落向其中一人身上:“潘平,就由你出战罢。”

潘平人高马大,头发赤黄相间,腰挎一柄金边银柄弯刀,闻言大喜,正要应喝下来,就听见阵前有人高喊:“小女子唐妙鸣,久闻贵方狼王赫赫之名,特来领教一二。”

“东方余亮胆气不小,居然提前挑阵!”

“来者是小狐帅唐妙鸣,四转中阶,居然直言挑战狼王大人,必有阴谋算计啊。”

一时间,众人的目光都投向方源,要看狼王有何反应。

\end{this_body}

