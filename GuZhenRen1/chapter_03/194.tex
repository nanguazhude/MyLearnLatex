\newsection{长生难难难}    %第一百九十四节:长生难难难

\begin{this_body}

%1
第三个真传,价值很高,皆因里面讲述的是延寿之法。

%2
俗语曰:老死不如赖活着。

%3
除去特殊原因,没有谁不想活着,不想活得更久。

%4
蛊师这个职业,虽然能够修行,到了六转地步也能够改善生命本质,但却不能直接增长寿命。

%5
蛊道发展以来,不知多少的能人志士,千方百计地研究出各种各样的方法,去延长自己的寿命。

%6
譬如青茅山的古月一代,选择化身成僵。又比如南疆的魔道蛊师百岁童子,利用还童蛊加持自身,平时只是孩童形象,减少生命流逝速度。

%7
这些法子,只是凡人的俗法,蛊仙拥有更玄妙的法门。

%8
第三个真传,就是针对蛊仙的延寿法门。设想出来的,不是旁人,正是巨阳仙尊。

%9
仙尊设想的延寿大法!可以想象,这份真传的价值!

%10
根据真传内容记载,某年某月某日,巨阳仙尊灵机一动,触发了这个奇思妙想。

%11
他将这个仙法命名为——夺舍。

%12
其核心,便是一只六转夺舍仙蛊!除此之外,还有各种五转凡蛊,多达三千只。

%13
“没有想到,夺舍仙蛊的出处是在这里啊。”方源看得砰然心动,在前世,五域混战时,夺舍仙蛊一度大放光彩,好几位高转蛊仙因此延寿,从而改变整个五域局势。

%14
夺舍仙蛊虽是六转,但却是能够改变世界格局的关键蛊虫!

%15
一瞬间,方源心中涌起一股冲动——这里就是夺舍仙蛊的源头,只要他拿取走,掐住源头。将来五域大战,他贩卖夺舍仙蛊,不知要获取多少惊天的利益!

%16
尤其是他有前世记忆,知道谁最需要这种夺舍仙蛊,只要善加利用,好处几乎无法想象!

%17
但很快,方源冷静下来。

%18
大魂虫上的特意,传来更多的信息。

%19
这道真传,只是一个空壳子,里面的夺舍仙蛊,还有配套三千凡蛊,都被旁人捷足先登,取走了。

%20
八十八角真阳楼中,藏有巨阳仙尊布置的八十八道真传。

%21
但经过这么多年的时间,当中一部分,被人陆续取走。墨瑶曾经说过,在她生前,她探索八十八角真阳楼时,八十八道真传,还有五十三道遗留下来。

%22
墨瑶是万年前的风云人物,虽说八十八角真阳楼闯关甚难,十年一度王庭胜者,极少有人达到十角程度。但日积月聚下来,总归会有些个别能人,进入到真传秘境里来。

%23
现在留下来的真传数量,绝对要小于五十三这个数字。

%24
“咦?这是……”和大魂虫上的特意,相互交流,方源忽然瞳孔微微一扩。

%25
真传虽然被人取走了,但方源拧出来的这股特意,还有新的惊人发现。

%26
“以魂夺体,不过苟延残喘之举,终难免一死。要想延寿,还是寿蛊第一。蛊仙之路,灾劫重重,肉体难抗,魂魄难存,长生即逆天……难难难!”

%27
真传里面,留下这么一段话。

%28
这段话,直接批评了巨阳仙尊的夺舍想法,语气竟大的不得了!又总结延寿万千法门,从来只有寿蛊第一。最后三个难字,感慨良多。

%29
最后,说这话的人也留下了他的名字——幽魂魔尊!

%30
看到这个名字,顿时就让人觉得,前面这段话说得理所当然。

%31
幽魂魔尊和巨阳仙尊一样地位,同样都是九转蛊师。但巨阳仙尊是中古时代的仙尊,幽魂魔尊在继巨阳之后,出现的魔尊。

%32
巨阳仙尊陨落,二十万余年,天地间出了杀性最重的魔道尊者幽魂。八十八角真阳楼名头这么大,幽魂魔尊无敌天下,探古寻幽,来到八十八角真阳楼中,也不奇怪。

%33
“难道夺舍仙蛊,是被幽魂魔尊取走了?不,看起来可能性很小。幽魂魔尊开创魂道,是玩弄魂魄的大行家。这个方面,巨阳仙尊还比不上他。从幽魂魔尊留下的这段话里,他对夺舍之法根本看不上眼。只是他想长生于天地,却孤独一人,困难重重,见到前人也同样在琢磨长生法门,因此有感而发,留下了这段话。”方源暗暗分析。

%34
哪怕是无敌于乾坤的九转尊者,也难逃时间的冲刷。

%35
时间,是英雄和美人最大的敌人。

%36
寿蛊难寻,位置不定,产量有限,且只有天地才能产生。

%37
九转尊者搜刮天下,寿蛊渐渐用光,又寻不到新的寿蛊,只好想方设法来延寿。

%38
天地第一人,无敌的威势后面,却是对寿命有限的窘迫和悲凉。

%39
所以,巨阳仙尊想到夺舍法门,幽魂魔尊感慨长生之难。

%40
“即便是仙尊魔尊,如此无敌的人物,最终也难逃一死。唉……风水轮流转,各领风骚一代代,盛极而衰,否极泰来,天地能有何长存者呼?”脑海中,墨瑶心情沉重,仰天而叹。

%41
方源却非这么想。

%42
他目光幽幽:“天地大道,讲究生态平衡,相互制约。但蛊师修行,却是搜刮天地,集齐资源,供养自我一人,因此是货真价实的逆天之路。也许这就是天灾地劫的缘由吧。不过也真因为如此,超越长生的永生才是我值得追寻的目标啊。”

%43
想到这里,他心中像是燃烧着一团火,更加昂扬奋发!

%44
仙尊魔尊只能长生,而他追求的目标,是更高一层的永生!这可是尊者们,都达不到的境界。

%45
套用方源前世一个人的话来讲,那就是与天斗,与地斗,与人斗,其乐无穷也!

%46
潮水只有击打在礁石上,才能碰触出美丽的浪花。人只有在不断的斗争中,才能感觉到生的光彩啊。

%47
前世的方源,只不过六转,接近七转的成就。

%48
对于历代九转的存在,他从未仰望,只因胸怀壮志。

%49
在这里,他看到仙尊魔尊的另一面:疲惫、衰弱、无奈。这更让他激发出自己的斗志!

%50
眼前的真传,已经是空壳,不值得再去浪费时间。

%51
方源继续搜索。

%52
接下来,他连续搜寻到两个普通传承,都是关乎智道。

%53
这两个真传,分别有两只六转仙蛊。

%54
一只是“虚情假意蛊”,另一只是“儿女情长蛊”。

%55
巨阳仙尊的智道造诣,十分深厚,这是众所周知的事情。尤其在情蛊方面,他继往开来,发明了许多新的蛊方。

%56
巨阳仙尊广布后宫,收集美人,让天下女子倾心,可见这些情蛊的威力。

%57
这两道真传,都不是方源想要的,都选择了放手。

%58
“奇怪,不是说,这真传秘境中不仅有普通真传,还有无双真传的么?怎么我寻了这么长时间,还见不到一个无双真传?”方源疑惑,问墨瑶道。

%59
这段时间里,墨瑶并未出声,似乎是被夺舍仙法真传打击到了,斗志消弭,在方源的脑海中一直隐匿不出。

%60
听了方源的问话后,她这才显出隐约身形,淡然开口:“哼,这才多久,你就不耐烦了?想当年,我在这里探索了足足三个月。真传秘境十分宽广,各个真传又在移动,无上真传数量稀少,你碰不到也很正常。耐心点吧,小子。”

%61
方源便又问:“说起来,你当初也来过这里,发现了什么好传承,给我透露透露吧。”

%62
这才是他想要问的话。

%63
“哼,我接触的真传当然不少,但在这里,真传四处飞舞,位置并不固定,告诉你也没用。只有靠运气不断摸索。嘿,你要注意时间,你接触的真传越多,停留在这里的时间越长,这些真传的飞行速度就越快,你的处境也就越危险。当年我乃蛊仙境界,也只能支撑三个月,差点死在这里。”

%64
墨瑶说到这里,语气中带着余悸。

%65
“就算是你,也差点死在这里?”方源立即抓住话中的重点。

%66
“小子,你是凡人境界,太弱。估计只能查看八道真传,最多待一个月吧。现在你已经看了五道真传,可要注意了!”墨瑶含含糊糊地提醒道。

%67
“到底是什么,让你堂堂蛊仙的性命受到威胁?”方源追问。

%68
但墨瑶只是娇笑,丢下一句“你若有缘,体验一下就知道了”,便重新隐匿起来。

%69
方源面色微微沉重。

%70
他一直有个疑虑。

%71
前世的影像中,中洲蛊仙一共有十一人,进入真传秘境。但掐掉其中过程之后,再见时只剩下九人,且各个负伤。

%72
难道说,那少去的两人,是陨落在真传秘境当中了?

%73
有时候,有墨瑶这个意志在,也是有好处的。

%74
方源得了墨瑶的提醒,加倍小心,开始探索。

%75
三天后,他接近第六道真传,利用特意蛊和大魂虫搭配,成功获取里面的真传信息。

%76
这道真传,出自巨阳仙尊之手,和第三道真传目的相同,同样是一道延寿仙法。

%77
却是巨阳仙尊通过研究人祖,创出一个崭新的流派,取名为“阴阳道”。

%78
要用此法,须得男女两位蛊仙相互配合,通过男女交合,达到一方延寿的目的。

%79
但此法有个弊端。

%80
一方延寿,必定另一方减寿。

%81
一言蔽之,就是将一方的寿命,挪移到另一方身上。

%82
这个真传,也被前人取走了。

%83
延寿仙法的受欢迎程度,可见一斑。

%84
“这已经是第六道真传了。”方源口中喃喃,想到墨瑶的提醒。

%85
按照方源的承受力,他最多能查看八道传承,超过这个数目,就有重大危机降临。

%86
如果想要获得真传,方源就要谨慎了。

%87
毕竟他只剩下两个选择机会,同时也得防备真传是否被人取走的情况。

%88
十三天后,方源遇到第七道真传。

%89
这道真传光团,有脸盆大小,白光耀眼,气势要高于普通真传。

%90
“小子,你可以开眼界了,这是一道无双真传!”脑海中,墨瑶出声道。

\end{this_body}

