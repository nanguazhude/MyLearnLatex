\newsection{幽魂魔尊}    %第十七节:幽魂魔尊

\begin{this_body}

%1
汩汩汩汩……

%2
五人高的巨大石鼎中,青蓝色的水浆,在不断地冒泡。

%3
水面看似沸腾,实则寒气四溢,哪怕石人伸手进去,都要顷刻间化成冰棍。

%4
方源站在大鼎面前,一边分出心神,小心地调控,另一边则取出元老蛊。

%5
他一心多用,催动元老蛊,从中飞出一块块的元石。

%6
扑通、扑通、扑通……

%7
元石落入鼎中,溅起一朵朵水花。

%8
元老蛊是三转存储蛊,最多能存百万元石。它形如水晶球,半透明,球中云翳凝结形成云老人。里面存储的元石多,云老人便是笑颜。元石少,就是哭容。

%9
大量的元石投入进去,云老人的笑颜渐渐消失,渐渐转为了苦瓜色。

%10
方源这一次炼蛊,前前后后损耗了大约五十多万块元石。

%11
若是以前,他自然损耗不起。但如今,他确实财大气粗。卖了荒兽泥沼蟹,又收购了许多,身上剩下的元石还有六百多万。

%12
元石投入进去,大鼎中形成一个大漩涡,搅得水浆急速旋转。大鼎都在微微颤抖着。

%13
炼蛊到了关键时刻,方源投注全部心神,无法再分心。

%14
他的额头,很快就渗出汗滴,只能轻声呼唤:“地灵。”

%15
“哎!”小狐仙清脆地应答一声,连忙将一块块的银锭子,抛入大鼎之中。

%16
银锭落入水中,立即使得漩涡转速缓慢下来。

%17
一块块的银锭,终于使得鼎中水面平静,结成银色的坚实冰块。

%18
最后,鼎中水液彻底被冻结,大量银色的气雾蔓延而出。将大鼎都冻住,甚至蔓延出五步之外,将地面也染成一片银光。

%19
方源狠狠地喘息几声:“炼蛊炼了三天,终于告一段落了。起!”

%20
银色冰块破裂,飞出一只只蛊虫。

%21
这些蛊,皆是三转蛊,形如小杯子,又似喇叭花朵。手掌摊开,足能放下三只这样的蛊。

%22
它们通体银光灿烂。用以装载液体,皆是存储蛊虫。

%23
小狐仙细细数着,笑逐颜开,拍着手雀跃道:“一百三十七,一百四十六。一百五十九!主人,你好厉害,一下子炼成了一百五十九只三转蛊。我们可以卖不少钱了。这些蛊,都是什么呀?”

%24
“呵呵,这些都是银盏蛊。接下来还要用它们再炼蛊,不会卖的。”方源笑了笑。

%25
这银盏蛊乃是方源前世三百八十年后,才由一位蛊仙研炼出来的蛊虫。不断合炼下去。达到五转后,才是方源所需的蛊。

%26
现在当然不可能卖出去。

%27
“这些天来,我用了大量的胆识蛊,将魂魄增强到常人的五十二倍。因此才不会太疲惫。才能一下子炼出这么多的蛊。”方源对这次的炼蛊结果,还是很满意的。

%28
魂魄底蕴增强的好处,现在开始一一展现出来。

%29
如果鹤风扬看到这一幕,一定再不敢小觑方源。方源展现出来的炼蛊造诣。已经大大地高于他了。

%30
但是脑袋还是有些眩晕。

%31
每一次炼蛊,都对魂魄造成负担。损耗大量心神,更何况方源这种大规模的炼蛊。

%32
换做先前,方源要恢复神魂,只能休息静养睡觉。但现在他却有一种更好的方法。

%33
“地灵,这大鼎已经废了,你把它处理掉。我去外面走走。”

%34
“好的,主人。”小狐仙立即吭哧吭哧地干起活。

%35
已经过了三四天,荡魂山上再次凝结出了大量的胆石。

%36
方源随意地踩碎几块,飞出的胆识蛊立即将他的魂魄恢复,甚至还有一丝的壮大。

%37
方源瞬时感到,脑袋再无眩晕,思考什么问题,都迅速如电。

%38
他畅快地朗笑一声:“这胆识蛊果然是神话中的蛊,效用妙不可言。索性我今天,就将魂魄增强到极限吧天下第一掌门。”

%39
六十八倍!方源感到神清气爽,动作轻快无比。

%40
七十七倍!方源思维如电,每一个念头都如电光火石般闪烁。

%41
八十五倍!方源魂魄之强,已经隐隐超过身躯的容纳极限。

%42
九十二倍!方源可以清晰地感受到自己的魂魄。在感觉中,魂魄是灰白色的,外形和方源的面貌别无二致,只是强壮无比,肌肉贲发,体格如熊似虎。而方源的肉体身躯虽然健壮,却只是狼背蜂腰。魂魄装在这躯体当中,有一种挤压的感觉。

%43
已经到达极限了吗?

%44
九十三倍!方源再容纳一只胆识蛊,魂魄再次得到加强。这次方源感受到前所未有的畅快和舒爽感觉。这种快感,竟比吸食毒品,品尝美食,男女交媾都要美妙百倍!让方源这样铁打的硬汉子,都忍不住发出一声畅快的呻吟声。

%45
无法形容的舒爽和迷醉,让人流连忘返,品味无穷。

%46
方源眼中寒芒一闪,却是心生警惕。

%47
他又踩碎胆石,魂魄再度变强。这次舒爽之感,比之前还要强烈数倍!

%48
九十七倍,九十八倍,九十九倍!

%49
从灵魂深处传来的舒爽之感,强烈到让方源的浑身都颤抖,筋骨酸麻透爽,难以用言语表达。

%50
一百倍!

%51
狂烈的迷醉感受,宛若飓风席卷,方源差点因此昏迷过去。

%52
“极限了,不能再用胆识蛊!”方源咬破舌尖,借着痛楚来令自己保持清醒,不沉迷其中。

%53
常人只能将魂魄提升到一百倍,俗称百人魂!

%54
这也是生死极限。若令魂魄壮大一丝,整个魂魄就会砰的一声,发生爆炸。就好像是吃撑了把肚子涨破一样。

%55
但魂魄爆炸,比肚子涨破要严重无数倍。魂魄彻底消散,顷刻毁灭,肉体在保存一段时间后,慢慢腐烂成白骨。

%56
方源若沉迷于舒爽的快感当中,忍不住再用一只胆识蛊。他连催动春秋蝉的机会都没有,直接灭亡,从此彻底消失在这世界里。

%57
“可惜我没有落魄谷。落魄谷中,有迷惘雾,能令魂魄松散。又有落魄风,能切割魂魄。魂魄经受这样的拷打折磨,就会越加凝练精诚。”方源心中遗憾地叹息一声。

%58
魂魄光是壮大,只是数量上的优势。还得精炼凝结,才是质量上的优势。

%59
这世界上。有许多专注于魂魄修行的蛊师,他们统称为魂道蛊师。魂道和力道齐名,也是上古时代的辉煌。只是力道到如今没落了,魂道却经久不衰,仍旧是当今的一大流派。

%60
开创魂道的那位蛊师。在整个世界的历史上,都赫赫有名。

%61
他便是幽魂魔尊!

%62
九转蛊仙,傲视宇宙,漠视苍生,真正的天下无敌,横霸整整一个时代的传奇。

%63
同时,他还是杀性最大的九转蛊仙。

%64
在所有的仙尊、魔尊当中。他杀的人最多。在他那个黑暗的时代,他将五大域都当做自己的屠宰场,万物齐喑,任他宰割。无力反抗。

%65
幽魂魔尊就曾说过:天下之大,壮魂首选荡魂山,炼魂首选落魄谷。一山一谷若得之,则必可魂道大成纯阳圣魂。纵横世间不在话下。

%66
因此,荡魂山、落魄谷乃是魂道蛊修心目中。并驾齐驱的两大圣地。

%67
方源得到荡魂山,已经是万幸,是捡了重生的大便宜。想要再得到落魄谷,却是希望渺茫得很,他根本就不知道落魄谷在哪里。

%68
“不过,虽然没有落魄谷,但是我却可以用其他魂道蛊虫代替。神魂蛊、龙魂蛊、冰魂蛊、梦魂蛊、月魂蛊、将魂蛊、怨魂蛊、诗魂蛊等等,都可以凝练我的魂魄,让我能继续壮魂,突破百人魂,达到千人魂,甚至万人魂的地步。”

%69
这些蛊,他不可能借仙鹤门得手。直接作用魂魄的蛊虫,若是被仙鹤门动了什么手脚,那就太危险了。

%70
但是这些蛊的合炼秘方,方源又知晓得很少。更关键的是,他还没有考虑好,究竟选择哪种蛊虫最合适自己。

%71
“就现在而言,百人魂已经足够应付场面,还是将精力放在如何买卖石人罢。”方源思维发散了一下,便又收拢回来。

%72
荒兽泥沼蟹已经全卖了出去,接下来交易,方源却不打算卖胆石。

%73
胆石卖出去,壮大仙鹤门的实力,这不是他想看到的。

%74
接下来的日子,方源不断地炼蛊,洗练自己的空窍。

%75
一个月的时间,匆匆流过。因为有着九眼酒虫的帮助,方源顺利地晋升到四转巅峰。

%76
同时,他还炼出了一百五十五只金杯蛊。

%77
金杯蛊和银盏蛊,差不了多少,都是三转蛊,也都是用来存储水液。

%78
接下来,他又用金杯蛊、银盏蛊一起合炼。耗费了七天六夜,运气不错,最终得到三只四转的金杯银盏蛊。

%79
他暂时放下手中的工作,将目光投向石人一族:“这么多天过去了,石人部族应该分裂了吧。”

%80
石人一生当中,大部分的时间,都是睡眠中度过的。

%81
一般而言,一位石人近三百岁时,魂魄积累到一定程度,能繁衍出一个后代。此后平均每两百年,繁衍出小石人。

%82
不出意外,一个石人活到千岁,寿终正寝时,能有四个子孙后辈。

%83
但因为胆石的缘故,石人部族人口疯狂暴涨,从数百达到三十万余。

%84
激增的人口,导致石人内部矛盾迅速地增加、激化、爆发。

%85
石人的社会体制,本来就相当的松散,最多只能组织十万人口。果然一番政变之后,石人部族分裂成三个,平均每个部族有十万左右的人口,分别以一道元泉为中心,重新安家落户。

%86
方源将新炼制的奴隶蛊取出来,交给地灵小狐仙。

%87
奴隶蛊,从一转到五转皆有。方源炼制的,都是三转奴隶蛊,足够驾驭住石人部族了。

%88
小狐仙将奴隶蛊挪移到万里之外,直接作用在关键石人的身上。

%89
石人的魂魄,哪里是方源百人魂的敌手,轻而易举地就被奴隶。

%90
转瞬之间,方源掌控了包括岩勇在内的三位石人族长,以及十多位石人家老。

%91
一下子,石人三十万人口,都在他的操控之中。

%92
翻手为云覆手为雨,此举已然有了蛊仙的一丝风采!

\end{this_body}

