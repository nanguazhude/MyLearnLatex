\newsection{两仙蛊入手!}    %第八十一节:两仙蛊入手!

\begin{this_body}

%1
“我要亲自看你炼蛊!”方源心思一动,又试探着提出新的要求。

%2
蛊师炼蛊的整个过程,都是隐秘的。一旦被人观看到,无疑就会泄露蛊虫秘方,或者蛊师独有的手法。

%3
“当然。这也是当初的约定内容之一。”琅琊地灵却是一口答应,出乎了方源的意料。“”

%4
方源扬了扬眉头,有些明白了。

%5
当初,长毛老祖练就“遁空蛊”失败,反而将盗天魔尊辛辛苦苦一辈子积累的材料,糟蹋尽了。这其中很多蛊材,都十分珍稀,饶是魔尊也是多凭运气才采集到手的。

%6
盗天魔尊可是九转魔尊,为了弥补他的损失,以及挽回自家声誉,长毛老祖答应为其炼仙蛊。

%7
允许盗天魔尊参观整个炼蛊的过程,既是故意泄露秘方,弥补魔尊,又是证明澄清自己的一种方式——整个炼蛊的过程,你都看到了。若是炼蛊失败,我可没有故意动手脚啊!

%8
但到方源这种情境下,他却没有了这种顾虑。

%9
地灵是由福地天地之力,结合蛊仙临死的执念所化,没有人的歹毒心思,直来直去,坦坦荡荡。

%10
不过,琅琊地灵可是古往炼道第一仙长毛老祖所化,虽然记忆丧失了大半,但仍旧保留了大部分的,当年长毛老祖身上的炼道天赋和造诣。

%11
观摩地灵炼蛊,对于方源而言,大有裨益。

%12
“第二空窍蛊,我也是炼制过的。这一次倒要看看琅琊地灵的手段。吸取经验。弥补我自身的不足。兴许也可窥探到昔年长毛老祖灼耀千古的一丝风采。”方源对这次炼蛊,饱含期待。

%13
琅琊地灵轻轻挥手,便带着他挪移到一处大厅。

%14
这厅堂极为广场。方圆至少有十里。里面的设施应有尽有,有青铜大鼎,有赤铁丹炉,有彩陶水缸,甚至还有窑洞。

%15
蛊师炼蛊,有时候也需要器皿辅助。

%16
陈设在这处大厅的器皿,简直像是一个全面的展览。叫方源也有一种大开眼界之感。

%17
他虽然有前世五百年经验,屡次重生,但实际上是颠沛流离了三百年。只有两百年左右的时间,纵横天下,成就伟大。

%18
这期间,他虽然广泛涉猎其他蛊道。但既要抵御地灾。又要经营势力,三要炼制仙蛊,精力分散,炼道上积累的底蕴,能傲视凡俗,但在蛊仙当中只能算个二三流的层次。和长毛老祖这样的存在,那就更不能比了。

%19
“这就是我的地字丙号炼蛊大厅。”琅琊地灵淡淡地介绍了一下,方源闻言。脸色微微一变。

%20
地灵是不会骗他的,这样的大厅。还只是琅琊福地中炼蛊大厅之一!

%21
地字丙号……这福地中到底有多少这样的地方?

%22
这时,地灵又一挥手。

%23
一瞬间,大厅当中,凭空出现数千毛民。

%24
这些毛民,被挪移过来,只是楞了一下,然后纷纷拜倒在地上,齐声高诵道:“毛民拜见琅琊老仙!”

%25
声音整齐一致,在大厅中回荡。

%26
琅琊地灵的脸上,流露出一丝慈爱的笑意:“孩儿们,都起来罢。”

%27
他再一挥手,将准备好的炼蛊材料,都分别派送下去:“今天要炼一只仙蛊,就按照我给你们的蛊方炼制。”

%28
毛民们显然经历这种事情,不是一次两次了。

%29
听到炼仙蛊的时候,毛民们骚动了一下,纷纷流露出兴奋的神情。

%30
但旋即,他们就镇定下来,开始钻研手中的蛊方,然后开始炼蛊。

%31
数千毛民,同时炼蛊,这样的情景,方源不是头一次见到。

%32
异人中,毛民最擅长炼蛊,天生就有炼蛊的才华。根据《人祖传》记,在太古年间,毛民们就开始大肆炼蛊了,甚至还将人祖的大儿子太日阳莽捆绑起来,企图杀他炼就永生蛊。

%33
毛民炼蛊,大多凭借天赋和灵感,随意挥洒,没有人类蛊师的斧凿痕迹。

%34
到了蛊仙这个层次,远超凡俗,通常为了炼蛊,也会豢养一些毛民。方源在前世,也豢养过毛民。

%35
但他那是血海福地,环境极端,毛民死了几批之后,他豢养的兴致也就低落下去了。

%36
方源在成就蛊仙之后,就曾经召集手中的毛民,为自己的魔教势力炼制大批蛊虫。

%37
正因为他有亲身经历,此时看来,就发现了眼前这批毛民的不俗之处。

%38
这些毛民,被培养得太好了。

%39
看他们的皮毛,各个油亮清爽,双眼有神,面色红润,可见生活环境极好,没有受到琅琊地灵的任何折磨。

%40
更关键的是,这些毛民灵性十足,动作矫健,一个个都是炼蛊的熟手、好手。

%41
在炼蛊的过程中,甚至有一两位老毛民提出意见,修改秘方!

%42
看他们随意挥洒的样子,方源都不由地砰然心动。

%43
这样的一群毛民,放到宝黄天中售卖,绝对是奴隶中精品的精品。至少得放出七丈以上的宝光,会被那些蛊仙风抢。

%44
这些毛民,造诣极为惊人,大多都是炼蛊大师。甚至有几位,就是提出修改意见的老毛民,已然是炼蛊宗师!

%45
方源如今,也只能勉强算是个大师罢了。

%46
有了如此重量级的阵容打下手,炼蛊进展自然比方源当初在三叉山时,快了十多倍。

%47
这些毛民炼成的半成品,先是汇集到几位老毛民的手中,然后再交给琅琊地灵。

%48
琅琊地灵将收上来的几件半成品,查看了一番后,直接捏碎其中几件,令毛民们重新炼制。

%49
如此几番下来,琅琊地灵终于满意。亲自出手,一蹴而就,获得第二空窍蛊的半成品。

%50
也就是方源如今手中。拥有的那个半成品。

%51
但方源却明白得很,琅琊地灵手中的这件半成品,品质上足以甩掉自己两条大街。

%52
终于,琅琊地灵当众取出了神游蛊。

%53
这神游蛊表面有明显的破损,伤痕密布,看得方源心头一揪。

%54
“小子,你现在反悔还来得及。”琅琊地灵停下动作。对方源劝道。

%55
方源皱着眉头,思索了一下,忽然轻笑一声:“不。还是请你出手。”

%56
“狡诈的家伙,蒙不住你!”琅琊地灵咒骂一声,捏着鼻子,又挪移出数只蛊虫。以及珍稀的实材。

%57
方源睁大双眼。只认出其中两件,都是增加炼蛊成功可能的珍品。放到宝黄天中,宝光至少六七丈的贵重宝物!

%58
琅琊地灵的一举一动,都牵引着无数道目光。

%59
毛民们看得如痴如醉,激动得浑身颤抖,一对对瞳眸中都是极度的崇拜之色。

%60
方源也是看得心驰神摇,琅琊地灵炼蛊时,简直是行云流水一般。带给人一种自然而然,没有烟火的气息。

%61
琅琊地灵展现出来的精妙操纵。令方源受益匪浅。

%62
虽然极想就这样看下去,但方源咬了咬舌头,迫使自己清醒过来,开始干正事。

%63
他当场盘坐下来,取出两只早已经准备好的蛊。

%64
一只称之神清蛊,一只唤为醒云蛊。

%65
这两只蛊,皆是四转。被方源灌注真元后,催动起来。

%66
神清蛊化为一道清风,直接钻入他的脑海当中。而棉花团似的醒云蛊,则化为一蓬微型白云,悬浮在方源的头顶上空。

%67
方源一边盯着琅琊地灵操纵着的光团,一边取出四种极品美酒。

%68
有了通天蛊沟通宝黄天,他采集极品美酒变得极为容易。为了此刻,他准备了至少有十四种极品美酒。

%69
察觉到方源这一举动,琅琊地灵极为不悦地冷哼一声,但终究没有对方源动手脚。

%70
方源暗松一口气,不顾耀眼的光辉刺得自己双目泪流,也要一只盯着光团猛瞧。

%71
终于,他看见神游蛊,渐渐化为一滩流水,和其他实材交织在一起。

%72
方源连忙打开酒坛,仰头灌下一大口。

%73
顿时,一股强烈的酒意,冲击他的整个身心。

%74
方源连忙驱使神清蛊,令自己振奋精神,神智重复清爽。同时头上醒云,滚滚翻腾,令他的心神时刻保持清醒的状态。

%75
接连喝下四种美酒,方源保持着清醒的状态,却发现身上毫无异变。

%76
光团还在变化当中,已经从原来的规模,膨胀了数倍,比大象还要大。

%77
“看来神游蛊没有消失,仙蛊唯一,我还有继续喝。”方源又取出酒坛,拍开封泥,一一灌入口中。

%78
光团渐渐缩小,琅琊地灵手里托着光团,神情越加肃穆。

%79
片刻之后,方源喝到第八坛极品美酒的时候,他已经醉眼朦胧。

%80
就在这时,琅琊地灵手中的光团猛地发生膨胀,瞬间又缩小成弹珠,膨胀缩小,膨胀缩小,如此三番五次,终于一定,形成第二空窍蛊!

%81
在第二空窍蛊形成的刹那之后,方源浑身一震,充斥身心的全部酒意,忽然长河入海般纷纷聚拢,凝结一点。

%82
冥冥中,玄机降下,道纹凝结,使得这点微微一爆,化成一蛊——神游蛊!

%83
第二空窍蛊。

%84
神游蛊。

%85
两只仙蛊同时入手!

%86
方源酒意彻底清空,不禁兴奋地站起身来,朗声一笑。

%87
“你倒是算计得深!”琅琊地灵满脸疲惫地看着方源,原本凝实的身躯变得微微虚幻。

%88
方源乃是得了盗天魔尊机缘之人,琅琊地灵虽然觊觎神游蛊,却是万万不能向方源出手的。

%89
毛民们睁大愤怒、鄙视的双眼,瞪向方源。

%90
这个可恶的人类,居然算计我们敬爱的至高无上的琅琊老仙大人!

%91
方源对这些目光毫不在意,他尽数收敛笑容,向琅琊地灵躬身一礼:“些许算计,难登大雅之堂。今日在下大开眼界,受益良多,又愧又佩,愿拜前辈为师。”

\end{this_body}

