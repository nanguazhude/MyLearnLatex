\newsection{投降}    %第六十九节:投降

\begin{this_body}

%1
战火纷飞,群狼如潮水般狂烈地奔腾着。

%2
围攻严家营地,已经持续了两个时辰。严家擅长防守,但因为突袭成功,再加上严家群龙无首,此刻严家营地已经破败不堪,到处都是断壁残垣,人的尸体,狼的尸体。

%3
“兄弟们,顶住,给我顶住啊!”在营地的最中央,还有严家残余的力量在负隅顽抗。

%4
但狼群汹涌而来,绵绵不绝,整个防御战线已经不支,摇摇欲溃。

%5
“杀,杀光这些侵略者!”

%6
“卑鄙无耻的葛家,我诅咒你们灭族灭种!!”

%7
除了这些咒骂声之外,还有老弱妇孺们嘤嘤的哭泣声。

%8
看着自己背后的亲人、朋友,已经疲惫不堪的严家蛊师们,又奋力从身躯中压榨出一丝丝的力量。

%9
在他们的心中,有一个意念,在支撑着他们——守住,一定要守住!我的背后,是妻子儿女,是父母双亲。如果守不住,他们就要命丧狼口!

%10
忽然,一头千狼王突破了战线,冲入了阵地内部。

%11
“糟糕!”

%12
“小心啊!”

%13
“快躲开!”

%14
前线的蛊师们怒目圆瞪,纷纷怒吼,但已经救助不及,只能眼睁睁地着千狼王张开血盆大口,就要屠戮里面的老弱妇孺。

%15
“畜生,你找死!”就在这时,一位躺在地上,伤重得只剩下一手一腿的蛊师,不知哪里来的力气,奋力一跃,主动投身狼口。

%16
千狼王一口咬在他的腰际,几乎要将他拦腰截断。

%17
蛊师口吐白沫,惨烈而又得意的一笑。

%18
他猛地抱住狼头,大吼一声:“孽畜,给我一起死吧。”

%19
说完,他轰然自爆,血水溅射,和千狼王同归于尽了。

%20
这一幕,被包围于此的葛家众人看在眼中,当即便有人叹息道:“严家儿郎,竟勇烈如斯!”

%21
方源淡淡点头。

%22
严家虽然进取不足,但擅长防守,十分团结。攻打这片营地,损失了他大量的野狼,着实超出他原本的估计。

%23
但方源并不心疼,当即冷哼一声:“再勇烈又如何?失败了往往就万劫不复,这就是王庭之争的残酷。可以了,去劝降吧。”

%24
这话令葛家蛊师们心中一凛:如果战败,眼前的严家就是将来的葛家。

%25
但目光移到方源的身上,众人又心中一宽:有狼王在此,葛家就算是傍上了大树。今后还得依靠狼王提携啊。

%26
原本激烈的战场,渐渐地安静下来。

%27
群狼停止了攻势,缓缓后退,并让开一条道路。一位葛家家老越众而出,顺着这条道路,来到了严家众人面前。

%28
“诸位,投降吧。”葛家蛊师大喊道,“识时务者为北原俊杰!”

%29
“放屁!老子绝不向卑鄙无耻的偷袭者投降!”

%30
“来吧,割了大爷的头颅去。”

%31
“严家儿郎宁死不降!”

%32
一些蛊师大吼着,但也有一些蛊师则显得目光涣散,神色犹疑。

%33
葛家蛊师冷笑一声:“你们不降,杀了你们也不费什么事情。但是你们考虑过身后的妻子儿女吗?这些人会因为你们的负隅顽抗,惨死在这里。是你们害了他们啊。”

%34
这话一说,严家残破的阵地上,便安静下来。

%35
萧瑟的风,吹拂在众人的脸庞。刚刚还在怒吼的那些严家蛊师们,一个个神情僵滞。他们回望身后,其中大部分人的脸色松软下来。

%36
唯一的严家家老,感受到众人斗志的崩溃,恨极葛家的攻心战术。但这股愤懑,转到口边,却又只化为了一声长叹。

%37
在众目睽睽之下,他走了出来,艰难地道:“我们严家……愿降!”

%38
“大人!”

%39
“家老大人……”

%40
严家蛊师们纷纷喊着,有难以置信者,有低声悲泣者,也有释然解脱者。

%41
而与此同时,葛家蛊师们欢呼起来。

%42
“胜利了,胜利了!”

%43
“大局已定了,我们吞并了严家!”

%44
同处一地,两边的情境却宛若云泥差别。

%45
“收拢部队,清理战场。”葛光脸上亦是难以喜色。论实力,葛家在严家之下,但此次一笑吞大,却是一举功成!

%46
“只要消化了这些战果,我们葛家的实力就能膨胀三倍,甚至能超过红炎谷的时期。这一切多亏了常山阴大人呐!”葛光想到这里,不由地将目光转移到方源的身上。

%47
说实话,接到方源的信笺后,葛光也犹豫过。

%48
但这犹豫只是一瞬,便转为了坚定。事实说明他当初的选择,是正确的。

%49
“如果当初拒绝了狼王,恐怕我们葛家会和严家一样的下场吧。”葛光心中对方源的敬畏,不禁又深了一层。

%50
坐在驼狼背上,方源打量着整个严家营地。

%51
入眼处,皆是断壁残垣,火烟缭绕,尸体随处可见,鲜血流淌在地面上。

%52
一个个躲藏着的严家族人,被搜索队搜捕出来。严家蛊师们皆被戴上镣铐,剥夺蛊虫,禁锢真元,成为俘虏,被严加看守。

%53
方源脸上一片平静,这种情形他见得多了,五百年前世五域大乱战时,情形比这还要惨烈恐怖。

%54
“严家既灭,接下来就是英雄大会了。当然在此之前,我还要返回狐仙福地。琅琊福地也要去一次,如果幸运的话……”

%55
三日之后的晚上,又是一个繁星点点的夜晚。

%56
方源率领着狼群,来到野外,先用推杯换盏蛊和小狐仙进行沟通,随后利用星门蛊,再次返回到狐仙福地。

%57
这一次,随同他一起进入狐仙福地的,还有数千只野狼。

%58
这些野狼,都各有伤残,或是老弱之兵,战力低下。

%59
若是寻常奴道蛊师,唯一的选择就是将这些狼,充当下一次战斗的炮灰牺牲掉,防止它们继续消耗更多的粮草。但方源坐拥福地,却有一个更好的选择。

%60
那就是放养。

%61
“这些野狼在福地中繁衍生息,几个月之后,就能产下狼崽子了。”方源将这些狼群都放养到福地西部。

%62
这世界野兽繁殖能力甚是强大,再加上狐仙福地六倍于北原的光阴流速,方源今后的兵源完全可以自我补给。

%63
这样一来,狐仙福地的西部,就成了狼群的豢养地。而北部,几乎被方源割舍。东部笼罩阴云,还有数十个湖泊,水汽旺盛。而南部,则是石人聚集之地。

%64
中央处,则是中了和稀泥蛊后,渐渐死去的荡魂山。

%65
“主人,主人,我依照你的嘱咐,已经将星光虫群安置好了。你快来看呀。”再一次见到方源,小狐仙显得极为开心,拽着方源来到福地东部。

%66
“主人,您往上瞧!”小狐仙娇呼着。

%67
方源便往天空上看,入目的是一大片浅蓝色的云彩。

%68
缕缕星光,挥洒下来,仿佛梦幻薄纱,随着微风轻盈晃动。景色优美奇异,宛若书画一般。

%69
方源再仔细看,便发现这云彩并非本身的蓝色,而是上面栽种了大量的星屑草。星屑草呈深蓝色,而草丛中飞舞着一只只的星萤虫,点点荧光汇集成海。而星萤虫群中,更有星萤蛊,绽放着货真价实的星芒光辉。

%70
“不错。”方源评价了一句。

%71
小狐仙顿时开心得眯起双眼,将小脑袋蹭蹭方源的手背,娇憨地道:“主人,要摸摸。”

%72
方源淡淡一笑,伸出手来,摸摸小狐仙的脑袋。

%73
小狐仙的雪白长尾,顿时高兴地打起了卷儿,头上的两只毛茸茸的耳朵,也温柔地垂落下来,脸颊上泛起幸福的红晕。

%74
方源从通天蛊中,买下了许多星屑草。但此草十分特殊,不能栽种在凡土当中,只能种在云中。

%75
买卖时,瑶光仙子曾建议方源,购买云土培育星屑草。但方源并没有采纳她的建议,皆因狐仙福地当中,就有一块规模庞大的阴云。

%76
这阴云乃是当初,方源消除地灾影响,水火相冲形成的大股阴云。

%77
阴云经久不散,笼罩在狐仙福地的东部,本是一桩不大不小的麻烦。久而久之,遮挡光线,会影响整个东部的生态。

%78
但现在,方源用它来安置星屑草,却反而转祸成福。不仅废物利用,而且还替方源节省了购买大量云土的开支。

%79
如今,阴云上已经种上了大片的星屑草。星萤虫群在其中生活,给福地东部洒下漫天的星辉,美轮美奂。

%80
“只要好生培养星屑草,星萤虫群的繁衍生息就得到了保障。将来就会有越来越多的星萤蛊可供使用了。若能培养到上佳的程度,星萤蛊数量过剩,还可以在将来放到宝黄天中售卖。到五域大战时期,星萤蛊可是最紧俏的蛊虫之一啊。”

%81
方源稍微展望了一下美好的未来,便又领着小狐仙,回到荡魂山。

%82
站在山顶,他取出葬魂蟾蛊。

%83
他在围杀严家高层,攻占严家营地的整个过程中,都在使用葬魂蟾收集战场上的魂魄。

%84
他将葬魂蟾中的魂魄,尽数放出。

%85
这些可怜的魂魄,刚刚出来,就被荡魂山的奇妙力量,震荡成最精粹的养料,滋养整个荡魂山。

%86
“到了明天,荡魂山上就会再次长满胆识蛊了。”方源满意地点点头。

%87
他之所以吞灭严家,收集大量的魂魄,就是其中的一个原因。

%88
“至于今晚,就是我冲击五转的时刻了。”回到狐仙福地,方源的修为再不受异域压制,是实实在在的四转巅峰。

%89
他出于巅峰境地,已经久矣。再加上如今的甲等资质,冲刺五转境界的时机,早已经成熟了!

%90
ps:元旦放假,我要出去旅游。转换一下心情,顺便搜寻一些稳定更新蛊、好心情蛊什么的。因此停更三天。一月四号继续更新。在此祝诸君元旦快乐,亲们,健康的身体,才是幸福之本啊!!!

\end{this_body}

