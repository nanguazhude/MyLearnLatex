\newsection{牺牲(六千七大章 )}    %第一百八十六节:牺牲(六千七大章 )

\begin{this_body}

来到王庭福地这么久,多次出入八十八角真阳楼,方源从未忘记此次北原之行的初衷。

那便是救治荡魂山。

而救治荡魂山,就得需要江山如故蛊。

要得此蛊,困难重重,关键就在一个人的身上。这个人便是太白云生。

从太白云生加入黑家大军以来,方源就一直对其保持关注。

数月前,方源便打探出一个情报:

在黑楼兰获得一角楼主令,可以查探每道关卡奖励后,太白云生就当即找上门去,询问寿蛊的存在。

但当时,八十八角真阳楼中并未存有寿蛊。

直到第五十五层凝聚时,黑楼兰这才告知太白云生,发现了一只十五年寿蛊。

第五十五层彻底凝聚成形后,黑楼兰更明确了寿蛊的地点,乃是在该层第八十五道关卡上充作奖励。

从那之后,太白云生就将全部精力,投入到第五十五层的攻略上。他多次鼓动蛊师,拉起庞大队伍,硬闯难关。凭借他的威望,广施钱财,以及向黑楼兰等强者屡屡求助等等,倒是进展颇佳。

功夫不负有心人,太白云生耗尽家财,终于是闯到了第八十五关。

方源当然不想看到太白云生得偿所愿。

太白云生老迈,寿命无多,想用寿蛊增添寿命,那是人之常情。

但若他真的拿到了这个寿蛊的话,他必然不会冒险升仙。不冒险升仙,就不会在升仙途中天地感应,形成仙蛊江山如故。

没有江山如故蛊,方源拿什么来救治快要彻底消亡的荡魂山呢?

尤其是这几日,更有情报传出:太白云生闯关,多次尝试,已见希望。不惜向黑楼兰赊账,甚至答应黑楼兰加入黑家,成为外姓家老的条件。

从而,太白云生利用黑家的这笔资金,招揽大量好手,阵容浩荡,闯入第五十五层,要做最后冲击。

尽管记忆中,前世太白云生是冒险升仙的。但是方源谨慎,不敢冒险,决定亲自出手相阻。

时机不等人。

方源原本准备先解决掉墨瑶意志,攘外必先安内。但此事一直没有进展,现在他只得动用楼主令,偷偷潜到第五十五层当中。

第五十五层的情况,已经被太白云生摸透。

这里流淌着一道波澜壮阔的鲜血长河,河水深处矗立着一座宏伟雄奇的血道皇宫。

血道皇宫中,分有三大主殿,六十九座辅殿。

每一座宫殿,都有大量的血道蛊虫,以及血兽护卫。

血道蛊虫成群结队,威势磅礴,但只要花费时间,陆续消耗,总有屠尽的时候。

麻烦的是血兽。

这些血兽,斩杀之后,过不了片刻时间,就会汲取血气复活。可谓杀不甚杀,斩之不尽。

太白云生的几次闯关尝试,都折在这些血兽身上。

但最后一次尝试,他在无意中,惊喜地发现:只要一路冲锋,闯进宫殿最深处,将挂在主梁上的令牌摘下,便能令此殿中的血兽自行消散。

这个发现,让太白云生产生了一股强烈的自信。他一穷二白,不惜卖身黑家,赊取资金,拉起一只蛊师队伍,冲击这道关卡。

方源秘密进入此层时,太白云生拉起的队伍,已经闯过了三十三座辅殿,正在巍峨的主殿中展开激战。

方源取出六角楼主令,心念一动,瞬间这一层彻底落入他的掌控当中。

六角楼主令,能令方源掌控真阳楼中的六层。

方源曾经动用了一次机会,掌控了第七层。现在动用第二次机会,掌控这第五十五层。

黑楼兰手中有一角楼主令,掌控的则是第五层。

方源心念一动,身形骤然消失在原地,下一刻便出现在主殿的主梁上。

他惬意悠然地坐在主梁上,身边不远处,就是那枚关键的令牌。

屏气凝神,太白云生等人战斗的情景,就在方源的内心显现出来。

越是往主殿深入,里面的血兽护卫就越多,战斗进行得十分惨烈。太白云生的这只队伍,已经只剩下不到五十人。

他们一路突进,伤亡惨重,沿途留下大量的尸体。

两大五转蛊师强者朱宰、高扬,轮番替换,充当队伍冲锋的箭头。

而太白云生,作为此届北原第一治疗蛊师,则被众人保护在队伍中央。他不断动用人如故蛊,一道道白芒激射,照在死尸上,令蛊师复活,或者照在活人身上,令他们的伤势复原,真元瞬间恢复。

但即便太白云生乃是五转巅峰,也狼狈不堪。

队伍中不断有成员倒下,他虽然竭尽全力,但实在救不了全部。

无穷无尽的血兽,从主殿各处的回廊、房间里汹涌而出,密密麻麻,堆积在队伍前进的道路上。

越是接近关键令牌,血兽就越是强大。

尤其是血兽能汲取血气,快速回复,不断重生。

太白云生等人突进到此地,身后斩杀的血兽又再度重生,追杀上来,前堵后追,众人情景越加危急。

“难怪前世,太白云生要冒险升仙。这关难度颇高,除非有血道蛊师出手。”方源掌控此层,对于太白云生的行径,如掌上观纹。

这道关卡,对于血道蛊师而言,不啻于仙境一般。有大量的血道野蛊可以收服,又有纯正的血兽可以吞食,用来增长血道修为。

可惜,血道不容于世,因为血海老祖的缘故,血道蛊师一旦出现,就会立刻受到正道的大力追杀。

因为血道战力可以速成,甚至就连魔道蛊师都普遍忌惮。

往往血道蛊师一旦被发现,不仅要受到正道追杀,而且还会受到魔道蛊师的暗算偷袭。

黑家大军当中,没有一位血道蛊师。就算有,也隐藏至深,不会被太白云生所用。

方源探知片刻,便清楚太白云生成功的希望,已经十分渺茫。

心中的担忧,顿时消散大半。

方源躺在大梁之上,从身上掏出《人祖传》,一边打法时间,一边留意太白云生的动向。

闯荡八十八角真阳楼,也会死人的。

方源要图谋江山如故蛊,当然不能让太白云生现在就死了。

“杀过去,一定要杀出重围!否则我们都得死!”太白云生满脸血污,嘶声力竭地鼓舞士气。

一波波的血兽,咆哮着,怒吼着,如同血色巨浪,向众人扑杀而来。

……

主殿中一片安宁。

方源悠然翻开手中书页,还调整了一下姿势,好让自己躺得更舒坦一些。

……

吼!

一头血兽,虎头马身,杀进蛊师当中,疯狂撕咬。

受到攻击的两位蛊师,一死一伤。

“人如故!”太白云生大喝,手掌一扬,一道白光骤然射出,照在那位伤者身上。

手上的蛊师,顿时恢复到前一刻的状态。血兽造成的恐怖伤势,不翼而飞。

伤者呼呼地喘着粗气,惊魂未定,暂且退入内围。他留下的防御空缺,立即被其他人补上。

但那位阵亡的蛊师,尸体被周围血兽一拥而上,粗暴地拖了出去。

太白云生没有出手。

不成仙人,哪怕是五转蛊师,真元都是有限的,需要谨慎运用。

而且这位死去的蛊师,身上蛊虫严重损失,尤其是防御蛊被洞穿摧毁。就算救活了,也会再次被击杀。

剩下的蛊师痛声咒骂,只能眼睁睁地看着同伴的尸体,被十数只血兽疯狂啃噬。

血兽对血气的感知极为敏锐,一丝血腥气味就会让它们发狂。它们通过吞噬强大生物的血液,来壮大自身。

很快,阵亡的蛊师浑身的鲜血都被汲取光,整个尸体变成干尸,破烂不堪,惨不忍睹。

吸取了他身上血液的血兽们,纷纷长出人的形象。有的瞳孔,化为人类瞳孔。有的长出几只人耳,还有的则生出灵活的人类手臂。

但得益于它们吞吸尸体浪费时间,太白云生等人压力骤减,趁机突进,往前了数十步。

……

方源将手中的《人祖传》,翻到第三章,第十四节。

北冥冰魄将他的二姐古月阴荒救醒,二人再次为如何救活父亲,求教思想蛊。

思想蛊道:“人啊,成败山已经消失了,不知道什么时候什么地方才能重新形成。能救活你们的父亲的其他方法,我也不清楚。但不要灰心,你们可以去找智慧蛊问问看。我就是它的母亲,智慧是思想的结晶。”

青出于蓝而胜于蓝,思想蛊不知道的事情,智慧蛊未必不知道。

北冥冰魄和古月阴荒便在思想蛊的指点下,找到了智慧蛊。

智慧蛊和人祖有过结,当年人祖动用规矩二蛊,曾经捕捉到智慧蛊。但最终被智慧蛊逃走了。

起先,智慧蛊并不愿意帮助北冥冰魄和古月阴荒。

但是看在他们俩,是母亲思想蛊介绍来的,便勉为其难地道:“人啊,我可以为你们指点迷津。但我需要报酬,你们中的一位需要将中年交给我。”

“我把中年交给你吧。”古月阴荒立即道,毫不犹豫。

她被弟弟北冥冰魄唤醒,又赋予人生的意义,就是救活父亲人祖。因此这时候答话,当仁不让。

北冥冰魄争不过姐姐,只好让她奉献了她的中年。

这就意味着,古月阴荒青年一过,就会直接跳过中年,迈入老年。

但为了救活父亲,她也顾不得这么多了。

智慧蛊得到了古月阴荒的中年,便指点她道:“在西边的黄金沙漠中央,有一片静止的蓝宝石海洋,风波不兴,平滑如镜。那是万物之源,全天下的生命都来源于那里。而在蓝海深处,有许许多多的生命蛊,照映天下万物。你们去潜入海中,如果能找到人类形状的生命蛊,就将它带上岸来。这块人形的生命蛊,便能赋予你们父亲新的生命。但要注意时间,不能超过一刻钟,否则你们就会被蓝海同化。”

末了,智慧蛊又添加一句:“要找到人形的生命蛊,十分不容易。只有真正理解生命真谛的人,才能做得到。你们如果做不到,不要怪我的方法不好使。”

古月阴荒还想在询问什么,结果智慧蛊说话就飞走了,没有给姐弟俩任何机会。

……

嘈杂的嘶吼声、鬼叫声,冲入耳膜。

血兽宛若海潮般,滚滚而来。太白云生等人则宛若礁石,一次次抗住血兽的攻杀,同时一步步地艰难向前走。

“快了,我们距离主殿大门,只剩下三百步了!”有人大叫,鼓舞士气。

“小心!”忽然,身边的蛊师大声示警。

那位鼓舞士气的蛊师一愣神,被一只熊身龙头的血兽狠狠撞中。

砰。

蛊师在瞬间被撞得胸骨尽碎,一口鲜血混合内脏碎块,喷吐而出。

他身躯飞射出去,身后蛊师竭力拦下,一时间阵型大乱,竟有分崩离析之危!

“撑住,给我撑住!!”太白云生满眼血丝,瞠目大吼,焦急无比。

这样的战局,一旦蛊师们分散开来,立即就会被周围无数的血兽撕扯成碎片。只有蛊师紧密抱团,才有一线生机!

但蛊师们节节败退,大量血兽冲破防线,凶悍绝伦地扑杀过来。

一时间,惨叫声迭起。

许多蛊师,还在尽量恢复真元。猝不及防之下,就被血兽撕成碎片。

一只螳螂般的血兽,挥舞手臂镰刀,冲在最前面。

刷!

一位蛊师被它直接削去脑袋,脖颈**血泉,大好的头颅被冲向高空,还未落下,就被飞在半空中的血兽一口叼住,咕咚一声,吞咽入肚。

但腥红的热血,落了下来,正好溅了太白云生一脸。

太白云生连忙伸手一抹,勉强睁开双眼,浓郁的血腥气味刺鼻难闻,更激起一阵血兽的咆哮。

“完了!”太白云生心沉谷底,正在这时,一道身影宛若猛虎下山,飞扑而来。

轰轰轰。

几下交锋,来者便斩杀了螳螂血兽,在最后关头,将阵型堪堪稳住。

是朱宰!

到底是五转蛊师,魔道的有名强者!!

众人都被他拯救,仿佛从悬崖边上拽了回来。

但朱宰亦付出代价,身上新增三道伤口,皆是深可见骨。

太白云生见此,连忙催动人如故,提供治疗。

朱宰回复到前一刻状态,新增的伤势消失不见,同时刚刚剧烈消耗的真元,也重新涨了回来。

只是他刚刚爆发战力,用的几只消耗蛊,却真的损耗了。

人如故,只能针对人体,对其他蛊虫则没有任何效果。

……

“这个人如故,的确效果奇佳,可惜不能对自身施为。”方源叹息一声,又将目光投到手中书籍。

古月阴荒、北冥冰魄二人艰苦跋涉,穿越黄金沙漠,在沙漠中央,见到了蓝海。

蓝海美不胜收。

正如智慧蛊所言,即便再大的风,也掀不起蓝海的丝毫波澜。

在周围柔软的黄金沙粒的包裹下,它就像是一块深蓝色的宝石,静静地镶嵌在这块黄金丝绸上。

姐弟俩潜入海底,果然发现海底深处,铺满了密密麻麻的生命蛊。

这些生命蛊,就像是一颗颗的蓝宝石。但大小、形状各异。

有的像马驹,有的似虎豹,有的类鹰鸽,有的如蛇蛟。

姐弟俩细细寻觅,看得眼花缭乱。花鸟鱼虫,飞鸟走兽,雪人毛民等等各种形状的生命蛊都见过了,却唯独找不到人类形状的生命蛊。

无奈之下,姐弟俩只好钻出海面,回到岸上。

刚刚离开蓝海,弟弟北冥冰魄手中把玩的那枚鹿形生命蛊,就忽然绽放出柔和的光辉,跳到沙地上,化为一头小鹿。

这是生命的诞生!

姐弟俩惊奇地看着这一幕,都瞪大了双眼。

直到小鹿蹦跳着窜出老远,姐姐古月阴荒忽然顿悟:“难怪智慧蛊最后说了那番话,又不让我们再问,就直接飞走了。我明白生命的真谛了。”

“生命的真谛,那究竟是什么?”北冥冰魄忙问。

古月阴荒便指着眼前的这片蓝海,反问道:“你说,如果我们真的找到一只生命蛊,是人形的蓝宝石。我们将它带出来,它会变成什么?”

北冥冰魄想了一下,答道:“就像是那头小鹿一样,变成真正的鲜活的生命吧。”

说到这里,他忽然愣住了。

古月阴荒含笑看他:“看来你也明白了。我们就是这样的生命。我们就是生命蛊所化!我们自己就是人形的蓝宝石!”

北冥冰魄彻底明白过来,人从哪里来?

智慧蛊在之前就已经说得明白:这片蓝海是万物之源,天下一切生命都来源于此。

人,当然也来源于此。

他们的父亲人祖,曾经就是这海底的一枚蓝宝石。机缘巧合之下,出了海面,形成鲜活的生命,闯荡世间,艰难生存,走到如今这一步。

但人是万物之灵,偌大的蓝海中会有多少人形蓝宝石呢?

一定数量极少,甚至极可能只有曾经的人祖一位。

想要在这样广阔的海洋中,寻找一枚小小的蓝宝石,这是多么浩大的工程!

这简直比古月阴荒在成败山,寻找唯一的成功蛊,还要艰难千万倍。

“我知道有一个方法,可以尽快地得到人形生命蛊。”古月阴荒忽然道。

“什么方法?”北冥冰魄忽然有了不好的预感。

古月阴荒微微一笑:“那就是让我沉入海底,和这片蓝海同化,重新还原成生命蛊。”

古月阴荒虽然变成了怪物,但本质上还是人。

生命的本质,没有改变。

既然是人,一旦同化之后,就会形成人形蓝宝石般的生命蛊。

这个推测,并没有错。

难怪智慧蛊说过:要注意时间,不能超过一刻钟,否则就会被蓝海同化。

智慧蛊说的话,没有一句是废话。

智慧蛊又说:你们如果做不到,不要怪我的方法不好使。

从这句话分析――极有可能蓝海当中,已经没有人形蓝宝石了。如果姐弟俩不愿牺牲自己,就不会找得到人形生命蛊。这样一来,就不要怪我智慧蛊,是你们做不到而已。

“不,姐姐你不能就这样牺牲自己。”北冥冰魄连忙阻止道。

他虽然想救活父亲,但也不愿意牺牲自己的亲姐姐。

“我能的,我人生的意义,就是救活父亲啊。”古月阴荒一脸平静地答道。

北冥冰魄忽然说不出话来。

是他赋予了古月阴荒这个人生的意义,古月阴荒就是为此而活的。换句话讲,如果救活了父亲,那么她的人生就没有意义,那么继续活着还有什么意思呢?

只要牺牲自己,就能救活父亲,这真是古月阴荒的人生最大意义啊!

“人,本就是天地间的宝石。只是宝石璀璨与否,需要我们自己的雕琢。我们的每一次努力,每一次选择,都是一次雕琢。”

“而人也只有献出生命,才能得到生命。”

古月阴荒悠然说完,便沉入蓝海。

北冥冰魄极力阻止,但阻止不了,古月阴荒成为怪物,力大无比,他不是对手。

超过一刻钟后,古月阴荒被蓝海同化,化为一块人形的蓝宝石。

但这枚蓝宝石,并非健全的人形,而是残缺了一小半。

这是因为之前,古月阴荒将自己“中年”交易给了智慧蛊。

人们往往在了解什么是生命之前,已将自己的生命消磨了一半。

北冥冰魄含着泪,带着这枚蓝宝石,离开了黄金沙漠。

他的心中没有成功的喜悦,而是陷入极大的愧疚当中。

从某种程度上讲,是他害死了他的姐姐。

……

“还差五十步!”太白云生竭力嘶吼着。

经过艰难的推进,每隔片刻就有同伴倒下,惨烈的激战,让他身边的蛊师,只剩下五位!

但成功近在眼前!

“再加把劲……”

“我要把这些血兽撕成碎片!!”

朱宰、高扬一左一右,护在在太白云生的身边,亦开口高呼,提升士气。

三十步!

两人倒下,只剩下太白云生、朱、高三人。

周围血兽,张牙舞爪,狰狞恐怖。它们前仆后继,成百上千,浩浩不绝。

十步!

朱宰、高扬奋尽全力,蛊虫因为过度催动,损失惨重,真元已然见底。

“二位之恩,老夫发誓,日后必当重重酬谢!”太白云生双眼放光,语气诚挚恳切至极。

“老先生,你说哪里的话!我们当时就是你救的,没有老先生,就没有今日的我们啊!”

“老先生的救命之恩,恩重如山!我们今天就算是死,也要偿还您的恩情。”

高扬、朱宰二人动情答道。

正道亦有假君子,魔道非无真豪杰!

五步!

“救我!”朱宰大吼一声,合身扑上,不顾防御,和前方拦路的血兽同归于尽。

太白云生伸出手掌,但却催不出白芒。

他失声叫道:“不好,我真元耗尽了!”

这个糟糕的消息,让高扬脸色瞬间苍白。

一直以来,都是太白云生充当核心。有人如故蛊的不断支援,众人这才能推进如此深入。

“没有关系,主殿中没有血兽敢进。只要我进入主殿,迅速恢复,朱宰还有救!”太白云生又大吼。

高扬精神一振。

太白云生说的没错,血兽虽然撕扯尸体,但主要还是汲取血液。只要尸体大半还在,就有救活朱宰的希望。

吼!

下一刻,一头巨蟒般的强大血兽,忽然从身后窜出来,张开血盆大口,将高扬一口吞下。

两步!

主殿大门尽在咫尺,太白云生浑身伤痕累累,踉踉跄跄地从血兽缝隙间挤窜而出。

关键时刻,他鼓动最后的宝贵真元,撑起一片金甲,替他抗下诸多攻击。

到了!

“只有人献出生命,才能得到生命。高扬、朱宰,我会记得你们的牺牲的!”太白云生奋进全身余力,猛地推开主殿大门。

他一下子栽倒在主殿中,身后血兽怒吼咆哮,张牙舞爪,却不敢探身进入主殿。

正如太白云生之前探查的情况。

太白云生瘫倒在地上,没有一丝力气。浑身的伤口,不断带来剧痛。

他狠狠地喘息几口,忽然哈哈大笑起来,笑声中夹杂着哭音。

他泪流满面。

“这关卡,有三座主殿,六十九座辅殿!在规定的时间内,全部打通,就能获得上等通关。打通两座主殿,四十六座辅殿就是中等评价。我打通一座主殿,二十三座辅殿,只能得到下等通关。但寿蛊到手了啊!”

“十五年的寿蛊,能够给我增寿十五年。十五年啊……”

太白云生正无限感慨着,忽然眼前一黑,晕死过去。rs

\end{this_body}

