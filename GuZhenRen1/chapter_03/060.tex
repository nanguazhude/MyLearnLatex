\newsection{今夜星空真美}    %第六十节:今夜星空真美

\begin{this_body}

%1
鬼王、红玉散人顿时无比的失望。

%2
青索仙子、粉梦仙子纷纷吐出一口浊气,神情复杂。

%3
黄沙仙子则心石落地,彻底放松下来,差点软倒下去。心中对方源对地灵的恨意,更深了数倍。

%4
“只要叫我逃生出去,只要让我卷土重来,我一定会将你们碎尸万段,方解我心头之恨呐!”

%5
“呵呵呵,小友不愧是魔尊的继承人呐。”地灵干笑了几声,取出通天蛊。

%6
这通天蛊催动起来,便化为一面椭圆镜子,飞上地灵的头顶,镶嵌在空中。

%7
镜子中一片漆黑,忽然画面一变,绽放出一片明黄之色。

%8
地灵催动神念蛊,神念一波波地发射到镜子里去,又一波波地接收着。

%9
同时,镜面上不断变化,出现一件件材料的实物图景。

%10
地灵似乎谈妥了,凭空取出两块仙元石,抛入通天蛊中。

%11
镜子表面仿佛一阵涟漪荡漾而起,仙元石没入其中,而一份份材料则被镜子吐了出来。

%12
“这次我足足买下二十份的材料,不信炼不成功!”地灵发狠地道。

%13
这一次炼蛊到最后阶段,星光烂漫,汇聚成一对椭圆的石头。

%14
“成了!”地灵哈哈一笑,拿着这两颗蓝宝石样儿的蛊虫,亲手交到方源的手中,“这便是星门蛊,算是完成了一项约定。”

%15
“我怎么知道能不能用?”方源手捏着这对星门蛊,反问一句。

%16
地灵立即吹胡子瞪眼,仿佛受到了莫大的侮辱:“你这是怀疑我炼蛊的能力?若是不成,你大可来换!”

%17
“也罢,那就先告辞了。”方源干脆至极,说明了去意。

%18
地灵盯着他猛看了两眼:“说走就走?那还剩下的两次机会。你都现在不用?”

%19
“当然先留着。出口在哪里?”

%20
地灵一挥袖,将五仙挪移走,脸色大大缓和下来:“你这里还有什么蛊虫秘方吗?都可以写出来,让我看看嘛。”

%21
“没有了。有的话,自然会和你交换的。”方源推脱道。

%22
“总感觉小友你肚子里大有存货!”地灵怀疑地看着方源,嘴里嘟囔着,“也罢,那我就等着你再来。”

%23
说完这话,地灵再一挥长袖。

%24
方源眼前一花。再定睛一看,便发现自己正站在石林中。身旁就是自己之前坐的石凳,背后就是那根紫色石柱。

%25
夜空中,繁星点点,温度降得很低。让人呵气成雾。

%26
刚刚的一切,好像是个梦幻,给方源一种如梦幻一样,不是很真实的错觉。

%27
再看看夜色,虽然在琅琊福地中坐了不少时间,但是在外界却只过了三十六分之一的时间。

%28
空窍中的星门蛊、神念蛊、通天蛊,以及狼吞蛊中藏着的两小坛极品美酒。则是明证,让方源恍然记起这场奇遇。

%29
“我原本打算用了这三次机会,哪里晓得地灵这处也可用秘方换蛊。因此还剩下两次机会。如果这次星门蛊能用得起来,耗费一次炼蛊机会绝对值得!”

%30
想到这里。方源再不停留,离开这处实力,回归到狼群之中。

%31
这是分秒必争的时刻,为防止夜长梦多。方源当即在狼王的拱卫下,盘坐在湖边草地上。开始写信。

%32
信很快就写好,然后他便催动推杯换盏蛊。片刻之后,通过空穴,他推杯换盏成功,从杯盏中取出一只四转蛊,外加一封信。

%33
此蛊形如一只枯死的小鱼,名为涸泽蛊,乃是水道蛊虫,却别有奇效。专门用来助长其他蛊虫的效力。

%34
这蛊一到北原,立即降为三转。

%35
方源取出信来,目光一扫,欣慰地点点头:“看来我不在福地的这些天,小狐仙勤勉有加,没有忘记我临行前的嘱托,炼成了整整二十只涸泽蛊。很好!”

%36
当即,方源又回了信去,在信中夸赞了小狐仙几句。

%37
“主人夸奖我了,好开心啊。”小狐仙接到信笺后,高兴得脸上浮现起红晕。

%38
“主人,人家好想你呀……”小狐仙旋即又趴在桌上,埋头书写,“主人不在,人家心里惶惶不安。主人你还好吗?主人,我这就再传一只涸泽蛊过去。”

%39
小狐仙用粉嫩的小手,将信纸好好的叠好,然后放入推杯换盏蛊中。

%40
同时,又放进一只涸泽蛊。

%41
过了好一会儿,推杯换盏蛊悠悠飞起,进入空穴,和另一只杯盏完成交接后,送到方源的手中。

%42
方源立即取出涸泽蛊,将信瞧了瞧。然后他汲取元石,补充真元,再次催动推杯换盏蛊。

%43
如此三番五次,他一共取出了八只涸泽蛊。

%44
“八只涸泽蛊一起用,效果相互叠加,应该有四转的程度了。”

%45
当即,这些涸泽蛊被方源一一捏碎,化为八道光圈,缠绕在推杯换盏蛊上。与此同时,在狐仙福地中,小狐仙也捏碎八只涸泽蛊,叠加到推杯换盏蛊上。

%46
推杯换盏蛊本是五转蛊,但是落到北原,就被压到四转。因此它只能传输四转的蛊虫。

%47
现在用了八只涸泽蛊后,推杯换盏蛊的效用提升到五转,这才可以传输五转蛊。

%48
方源等了片刻,首先取出通天蛊,郑重其事对放入推杯换盏蛊中,然后灌注真元。

%49
成功催动,两个杯盏在空穴中完成转换,装有通天蛊的杯盏,落到狐仙福地当中。

%50
小狐仙欢呼一声,立即将通天蛊取了出来。

%51
推杯换盏蛊上,则出现了丝丝裂痕

%52
涸泽蛊虽然能倍增蛊虫效用,但却是涸泽而渔焚林而猎。威能虽然暴涨,但蛊虫事后也会因为过分的爆发而毁灭。

%53
接着,他又把神念蛊送入狐仙福地,推杯换盏蛊上的裂痕加剧。

%54
方源沉默不语取出元石来,拿在手中,尽快地补充真元。

%55
之后。他将一对蓝宝石般的星门蛊,分出一只来,送入狐仙福地。

%56
幸好星门蛊不是仙蛊,而是五转蛊,使得方源可以动用推杯换盏蛊,将其勉强送入狐仙福地。

%57
将星门蛊送入福地之后,推杯换盏蛊已经伤的不成样子,裂痕满布,濒临破碎。

%58
这样的程度。还能使用最后一次。

%59
方源将这个杯盏,收入空窍,不再使用。

%60
他开始等。

%61
推杯换盏蛊是他所炼,拥有他的意志。但早就被他借给地灵小狐仙,因此小狐仙也能使用。

%62
先前几次。都是他在用推杯换盏蛊。

%63
这是因为,狐仙福地的时间流速是北原的数倍。小狐仙那边若是冒然催动,方源这边却很可能还没有准备好。

%64
“呼……现在就看小狐仙的了。如果她失败了,那最后一次,她将传来另一套的推杯换盏蛊。希望她能成功罢。”

%65
方源走之前,只炼成了一套推杯换盏蛊。但他临走前,关照过小狐仙。

%66
期间。小狐仙和仙鹤门做了几单交易,换取了不少炼蛊的材料,已经炼成了第二套的推杯换盏蛊。

%67
时间变得有些难熬。

%68
不管是通天蛊、神念蛊都是五转蛊,对真元消耗极大。即便是五转巅峰的蛊师。也用不了几个呼吸。因此往往只有蛊仙,或者地灵消耗仙元来用。

%69
至于为何不在琅琊福地中,动用推杯换盏蛊,那是因为方源还不想暴露蛊虫。一旦暴露。恐怕琅琊地灵就非得索要相应的秘方了。

%70
时间一分一秒的过去,放在狐仙福地中。却是几倍的流速。

%71
夜风寒凉,吹得方源也有些坐立不安。他从地上站起来,开始踱步。

%72
夜空中繁星点点,星光充足。方源手中握着的星门蛊,却是丝毫不见动静。

%73
“难道要失败?宝黄天中的那团星萤蛊,被其他人买走了?”随着时间的流逝,方源的心也不断下沉。

%74
他不再踱步,背负双手,伫立在草地上。

%75
举目远望,月牙湖波光闪闪,一片静谧。而身边狼群或站或伏,神态不一。

%76
这让他不禁想起当初,在青茅山上捕捉酒虫的那一幕。

%77
他失笑一声,不再患得患失,目光又重新清朗起来。

%78
一直以来,所有的担忧,所有的压力,所有的焦躁,都化为清风散去。

%79
他仰望夜空,吐出胸中所有的浊气,忽然觉得现在的生活是这样的美好,将所有的一切都投入到毕生最高的追求当中,心无旁骛,一切无悔。

%80
他的心一片通彻,如明镜不染尘埃,如这月牙湖通明沉静。

%81
自从重生青茅山以来,他一直殚精竭虑,现在忽有所悟。

%82
这种领悟说不清道不明,萦绕在他心中,最终只化为一句喃喃之语:“今夜的星空,真是美丽啊。”

%83
这是由衷的感慨。

%84
此话说了,方源浑身轻松,感觉自己好像卸下千斤重担,直欲飘然飞仙。

%85
他浑身的气质,似乎也发生了一些转变。原本的阴鸠气息,消散无踪,转成清新开朗之气。原本幽黑深邃的双眸,此时却绽放着清明的光,宛若新生儿,又仿若星辰。

%86
手中的星门蛊,这时微微震动起来,幅度越来越大。

%87
方源松开手。

%88
蓝宝石般的星门蛊,便悠悠飞上半空中。牵引出大量的星光,凝聚过来。情形灿烂如梦,优雅如幻。

%89
须臾后,星光形成一道巨大的圆形攻门。

%90
星门蛊终是成功了!

%91
方源静静地看着,嘴角微微带笑,有着开心,但眼眸中却是一片淡定。

%92
他从容地迈进星门,眼前的星光汇集成漩涡,带动他的身躯一路疾飞。

%93
十几个呼吸之后,他走出星门,重新来到狐仙福地。

%94
“主人,你终于回来了!”小狐仙高兴至极,一个跳跃,投入到他的怀中。

%95
方源呵呵笑着,摸了摸地灵的小脑地。

%96
“是啊,我又回来了。”他淡淡地说了句。

%97
(ps:感冒了,可能是醉酒时,夜里睡觉受凉了。悲催……而且最近开始忙了,对于更新我想说,我只能尽量。原本是想两更的,但头很昏沉,今夜就一更。诸君也注意保护身子,健健康康才是福!)

\end{this_body}

