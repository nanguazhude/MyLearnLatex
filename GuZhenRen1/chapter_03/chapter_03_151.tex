\newsection{竟是蛊仙传承!}    %第一百五十一节:竟是蛊仙传承!

\begin{this_body}

%1
砰!

%2
黑楼兰抬起一脚,将族人黑旗胜踢翻在地。

%3
“无能的东西!”黑胖子大声咒骂,脸色横肉抽动,目光凶残暴虐。

%4
大堂内噤若寒蝉,黑家的蛊师们都不敢吭声。

%5
黑楼兰人称“黑暴君”,性情暴虐残酷是出了名的。尤其是当他被第五十四关,连续阻碍六次,脾气就像是个火药桶被点燃了。

%6
“族长大人,是属下无能,属下该死!”黑旗胜匍匐在地上,五体投地,大声告饶不绝。

%7
“蠢货!我们黑家怎么会有你这样无能的蠢货?!”黑楼兰咬牙切齿,对着黑旗胜连踹数脚。见他被自己踢得吐血,黑楼兰心中的怒火这才稍稍平息了一点。

%8
周围的家老们,一个个默立当场,不敢说话。

%9
前一次,有过家老为黑旗胜求情,结果被黑楼兰当场打成重伤,现在还躺在病榻上呢。

%10
黑楼兰的脾气,原本在王庭之争的时候,还稍稍收敛着。但是到了福地之后,他暴躁的脾气,残酷的秉性,彻底展漏无疑。

%11
“你们也都是一群蠢货,无能的废物!呆呆地杵在这里干什么?说,给我说出闯过五十四关卡的好方法。想不出来,就罚你们的供奉。部族不养闲人!我给你们元石,给你们荣华富贵,提拔你们,栽培你们是干什么的?现在就是你们表现的时候了!!”黑楼兰大吼大叫,声音激荡着窗棂都在微微震动。

%12
众家老们暗暗叫苦,一个个像是霜打的茄子,低着头,偷偷地用眼神互视,却没有人敢第一个站出来发言。

%13
黑楼兰瞪着虎目,四下扫视,最终定格在黑沛家老的身上。

%14
作为资格最老的大家老黑沛,此时只能硬着头皮出列,恭敬地行了一礼:“大人,以属下看来,这第五十四关考的是奴道,难度极大,非大师级的造诣不能通过。黑旗胜家老虽然是我们部族培养的奴道蛊师,但他却不是大师。我们要想通过此关,恐怕需要狼王大人的力量了。”

%15
“哼,你是叫我请外援?是想让外人都看笑话,都认为我们黑家无能吗?”黑楼兰眼中凶芒大炽,恶狠狠地吼道。

%16
黑沛心头颤抖,又深深一礼,连忙道:“族长大人英明神武,领袖群伦,乃是此届王庭之主。有大人您坐镇,谁能认为黑家无能谁就是天下第一等的蠢货。所谓外援,那也谈不上。这狼王常山阴,本来就加入盟军,是大人的属下,调动他来助阵,本来就是他应该的。我想常山阴一定很感激大人,毕竟作为一个外人,能进入真阳楼,是他莫大的荣耀了。”

%17
黑楼兰被这么一说,脸上的怒色终于稍稍缓解了几分。

%18
众家老察言观色,纷纷在心中暗赞黑沛的口才了得,能成为大家老,果然有两把刷子。

%19
黑楼兰缓缓踱步,他很不甘心。

%20
目前,他手中仅有两块来客令,乃是之前闯第十二光,四十六关时的真阳楼奖励。

%21
他倒不是舍不得来客令,而是一旦邀请狼王进来,这五十四关的通关奖励就会落到常山阴的手中。

%22
若是自家族人,黑楼兰还可以凭借族长权威,将这些好东西都纳为自己手中。但是传统的规矩中,这奖励就应该归外援所有。

%23
八十八角真阳楼的奖励,一个个都非比寻常,就算是黑楼兰也要心动不已。

%24
不管是蛊方、蛊虫,还是其他什么,几乎每一件奖励都能令一个蛊师从平凡中崛起。

%25
黑楼兰踱了五六步,发生一声轻轻的叹息。

%26
他心中也知道,强逼黑旗胜也没有用。黑家培养了三个奴道蛊师,在王庭之争中死了一个,黑旗胜在他们当中,已经是最强者。

%27
但他不是奴道大师,任何的大师都需要充分的才情,不是单靠培养,就能直接培养出来的。

%28
黑楼兰停下脚步:“黑书何在?”

%29
“属下在。”黑书立在大堂门外,乃是黑楼兰的贴身蛊师随从,听到黑楼兰的召唤,立即进来拜见。

%30
“你去请狼王过来。”黑楼兰吩咐道。

%31
听到这句话,殿堂中的家老们纷纷暗松一口气。躺在地上的黑旗胜,更是浑身蓦地轻松下来——终于要解脱了!

%32
“是,族长大人。”黑书领命而退。

%33
砰。

%34
黑楼兰对准黑旗胜又是狠狠一脚:“你个废物,还躺在这里干什么?待会常山阴过来,你是想让他看到我们黑家无能的丑态吗?”

%35
“大,大人,我错了!”黑旗胜忙不迭的道歉。

%36
“给我滚下去养伤!!”黑楼兰吼道。

%37
“是,是,是大人!”黑旗胜挣扎着爬起来,跌跌撞撞地慌忙离开。

%38
不久,黑书带着一脸惭射,独自回来禀告:“族长大人,狼王大人并不在圣宫,而是外出放养狼群去了。”

%39
“什么?”黑楼兰扬起声调,眉头挑起,刚刚平静下来的脸上,又浮现出一层怒去。

%40
家老们心中一凛,大家老黑沛指责黑书道:“你这个小辈,好不懂得做事。狼王不在,你就空手而归吗?难道你不可以传信过去,说明来意,狼王还不屁颠屁颠地跑回来!”

%41
“大人,非是属下办事不利啊!”黑书委屈地叫苦道,“属下已经传信过去,但狼王也回了信,说自己率领狼群狩猎,已成习惯,不想中断,叫我们等等。若是等不及了,可唤唐妙鸣等人入楼。”

%42
一时间,众人惊异。

%43
大家老黑沛瞪大双眼,犹不相信:“居然有人如此淡然?他真的是这么说的吗?!”

%44
“证据确凿!族长大人,这就是狼王常山阴回的信蛊!”黑书说着,便递给黑楼兰一只星信蛊。

%45
此乃星道蛊虫,高达四转,传信最速。只是飞行中光芒闪耀,声势极大,容易被敌人截断捕捉。

%46
但在王庭福地,却是无忧。

%47
黑楼兰心神探入星信蛊中,冷笑连连:“这狼王倒是好心性,竟然能如此沉得住气。”

%48
“大人,狼王孤傲,世人皆知。以属下看来,狼王早已经暗笑不已,但却拿捏架子。”大家老黑沛的揣度,让其他家老都纷纷点头。

%49
“哼,他拿架子那也是应该的。他是奴道大师,还是飞行大师。你们如果是大师,我要求到外人头上?”黑楼兰一声怒吼,立即让大家老闭嘴,其他家老们的脑袋也垂得更低了。

%50
方源此刻倒还真没有,把注意力放在真阳楼上面。

%51
他俯瞰着脚下的地丘,心中满是震动:“这里果真没有小塔楼,按照周围的小塔楼比较,其实这里应该坐落着一座小塔楼,就位于地丘上面。但如今这里,只有一个漆黑的洞口……厉害,布置这道传承的蛊师,真是厉害!”

%52
王庭福地每隔八里,都会有一座小塔楼。其实,这都是八十八角真阳楼中的一部分。

%53
八十八角真阳楼乃是当年,长毛老祖亲自炼制的仙蛊屋,高达八转。

%54
后来,又经过巨阳仙尊的一手布置,相应整个北原。小塔楼居于王庭福地,几乎随处可见。每座小塔楼中,都有成千上万的野蛊,但无人敢触及分毫,谁动谁死。

%55
但现在看来,布置土丘传承的蛊师,不仅动了小塔楼,而且还凭此设置了传承。如此手段,如此才情,如此能力,立即让方源确认,这个神秘蛊师极不简单。

%56
“不,与其说是蛊师,倒不如说是蛊仙!哪怕历经沧桑岁月,巨阳仙尊的布置有些松动,但也不是凡人可以撼动得了的。只有蛊仙级别的人物,才能撬动一角,酿造成这样的布局。”方源眼中精芒闪烁不定。

%57
他意外得到的地丘传承,竟然是一道蛊仙传承!

%58
土中蕴光,芒高万丈,百里天游,咏梅雪香。这句密语,到底是说的什么?

%59
蛊仙的传承中,会有什么样的宝藏?

%60
“会不会有一只仙蛊?”方源大胆猜测。

%61
如果是仙蛊,那就可以媲美一层真阳楼。因为即便是八十八角真阳楼,它的每一层的最后关卡,也未必会有仙蛊奖励。

%62
“就算没有仙蛊,地丘传承布置得如此煞费心机,恐怕也有一份仙蛊秘方。”

%63
没有仙蛊,有仙蛊秘方,也是极大斩获。完整的仙蛊秘方,在宝黄天中,都不会贩卖。

%64
蛊仙们顶多只会卖残方,就算是有完整的仙蛊秘方,也要将其拆散,充进错误,才会拿出去贩卖。

%65
完整的仙蛊秘方,只有相互交换。但这种交换,在历史上也是屈指可数的。

%66
方源按捺下心中的遐想,又开始冷静思考。

%67
有了关键的线索,他现在思索起来,顿时突飞猛进。

%68
他结合种种线索,推测出土丘传承的布置时间,应该相当久远。至少是在乐土仙尊时代,就已然坐落在这里了。

%69
但随着思考的深入,又有更多的谜团产生。

%70
姑且将这个布局者称之为地丘蛊仙,那么他(她)究竟是谁?为何要在这里布置自己的传承?身为蛊仙,他(她)是如何进来的?更关键的问题是,他(她)是如何知晓这里的布置,如何明白八十八角真阳楼的原理?

%71
如果他(她)和太白云生类似,是在王庭福地中晋升成仙的话,那么一切就有了新的解释……

%72
方源想到头疼,终于不再想了。

%73
一只星信蛊,划破天际,向他飞来。

%74
方源接到手中一看,却是黑楼兰的催促信。

%75
“罢了,还是先去真阳楼,这里暂且放下。地丘蛊仙可以说是利用了巨阳仙尊的布置漏洞,我进入真阳楼后,也许能顺藤摸瓜,间接地参悟出这里的秘密!”

%76
念及于此,方源便回了信去。

%77
既然黑楼兰主动亲自来信催促,那方源也不便再拿捏姿态,索性直接回去。

%78
“这次还要借助狼王你的力量了。”黑楼兰见到方源后,大笑起来。

%79
他其实心中焦急得很。

%80
在王庭福地的时间说长不长,说短不短。只要北原外界的十年雪灾消退,王庭福地就会关闭,他们就得从福地中出来。

%81
在此之前,黑楼兰身上不仅背负着本家蛊仙的任务,而且更重要的是,还需为自己找到力道仙蛊。

%82
狼王回归,黑楼兰迫不及待,再组队伍。

%83
众人来到楼前,黑楼兰递给方源一道古朴令牌:“这是来客令,狼王你没有先祖血脉,对于真阳楼来讲,就是外人。要入楼一次,就得需要一枚来客令。”

%84
方源接过来客令,毫不作伪地轻笑一声,跃跃欲试地道:“正要见识一番真阳楼的气象!”

%85
随后,黑楼兰也不推开一层大门,而是径直一头撞“进”了门里去。

%86
方源紧随其后,手持着放光的来客令,效仿黑楼兰的做法,也入了真阳楼里。

\end{this_body}

