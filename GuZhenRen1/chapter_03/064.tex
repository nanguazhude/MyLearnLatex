\newsection{活棋}    %第六十四节:活棋

\begin{this_body}

舍利蛊是能够直接增加空窍底蕴,拔升蛊师境界的蛊虫。

从一转到五转,分别有青铜、赤铁、白银、黄金、紫晶舍利蛊。舍利蛊都是凡蛊,没有六转及其以上的仙蛊。

这其中青铜、赤铁、白银舍利蛊,都在凡间大量流通。但到了黄金舍利蛊、紫晶舍利蛊,因为能够直接影响四转、五转的蛊师战力,间接地影响势力格局,所以受到各大组织的严苛管禁,极少能在市面上流通。

但到了宝黄天,黄金舍利蛊、紫晶舍利蛊都是有的。

蛊仙用再多的紫晶舍利蛊,也不会增长修为。但紫晶舍利蛊的数量,仍旧较为稀少。

因为黄金、紫晶舍利蛊,除了增长修为之外,本身也大量地用作炼制仙蛊。在仙蛊秘方当中有重要的催化效用。

除去炼蛊,蛊仙若有后辈的话,也会收藏许多舍利蛊,留给后辈或者家族使用。

因此在宝黄天中,一到三转的舍利蛊成打卖,四转黄金舍利蛊就少了些,紫晶舍利蛊数量更少。

黄金、紫晶舍利蛊广泛用于仙蛊秘方,方源炼制春秋蝉用过,炼制第二空窍蛊也用过。他说中血神子仙蛊残方中,也需要大量的舍利蛊。

但凡涉及到仙蛊,价格一直居高不下。不过这情况,总比凡间的管禁要好太多了。

方源当即买下三只紫晶舍利蛊,又花费重金,收购了一只马到成功蛊。

此蛊比紫晶舍利蛊还要贵得多,可以增长炼蛊的成功率。

之后,方源又买下大量的低转蛊以及炼蛊材料。准备炼制五转的敛息蛊等等。

宝黄天虽然能检测宝光,但单看宝光的高度,并不能察觉出蛊虫身上到底有没有动过手脚。因此并不十分安全,所以方源还是尽量自己炼制,比较妥当。

将手中的残方。尽数卖去,又收购大量物资,再扣除宝黄天的手续费用,这番买卖之后,方源手中的仙元石达到二十八块。

当然,这些仙蛊的残方还可以重复买卖。但相应的宝光会持续下降。必须间隔一段时间,宝光才会回复。

宝黄天中的交易,首先看宝光。宝光强,价格就高。宝光下降,就代表卖价降低。

毕竟残方这种东西,越多人知道。价值就越低。如果持续贩卖,方源会越赚越少,对于之前购买的蛊仙,也会觉得亏。间隔一段时间再卖,对买卖双方都有好处。

卖方的卖价依旧,而买方多出了许多宝贵的时间,可以用来钻研残方。毕竟仙蛊唯一。

“这样一来,我手中最珍贵的秘方,只剩下春秋蝉秘法、第二空窍蛊秘方以及血神子的残方。其余残方,都会再卖。当然,至少要过数月。”

这个时间,当然是指的五域时间。

至于其他的秘方,诸如神游炼成定仙游,人尽皆知。放到宝黄天中,一寸宝光都不会有。

“石人,还是要尽量供应仙鹤门。这也是为了不让仙鹤门看出端倪。缓和矛盾,尽量拖延。”

如今荡魂山渐死,胆识蛊产量越来越低,每一次交易都让石人数目大减。仙鹤门的耐心也在渐渐消磨,迟早必有一战。方源也只能尽量拖延。

“好在他们只会以为。我困守狐仙福地,将我想成瓮中之鳖。怎么也不会料到,我会远去北原,甚至还能回来。就算是那个砚石老人,也不会料到。”

智道蛊仙,虽然擅长推演,但并非凭空臆测。而是根据一切的蛛丝马迹,推根溯源,进行推敲。

方源的核心蛊是春秋蝉,最大的优势就是重生。前世五百年的记忆,星门蛊、推杯换盏蛊等等,都是领先五域一个时代的蛊虫。

“残方暂时不能买卖,石人暂时只供应仙鹤门,但是我还有东西可卖!”

方源早有计划,这货物不是别的,正是泥土。

宝黄天中的泥土,都是珍稀之物。比如云土,比如腐土,比如盐土等等。很多福地出产土壤,贩卖到宝黄天中,与其他蛊师互通有无。

方源贩卖的泥土,当然也不普通,乃是和稀泥。

当初,地灾来临,荡魂山中了暗算,被和稀泥仙蛊的力量污染,整座山都在渐渐变成烂泥。

但这世间之事,都是福祸相依的。

荡魂山山石转变成的烂泥,都有和稀泥仙蛊的力量。这放到一些蛊仙的眼中,就是炼制和稀泥仙蛊的必备材料!

果然,方源将这些泥土放入宝黄天中后,便立即引发了大量的关注。

“今天到底是什么日子,居然有这么多的好货?”

“和稀泥,这是真正的和稀泥,有着强烈的仙蛊气息。”

“可惜我没有和稀泥仙蛊的秘方,得了这些泥也没有大用……”

蛊仙们神念激荡,很多人竞相发出神念,相互竞价。

“我需要和稀泥的仙蛊秘方,除此之外不换其他。”小狐仙在方源的命令下,传去神念。

这个苛刻的要求,顿时令蛊仙们嗤笑起来。

“用一堆泥,来换仙蛊秘方?你这想法也太过于贪婪了。”

“和稀泥仙蛊是一次性的消耗蛊,很明显秘方要比蛊虫更加重要。居然大言不惭地要换秘方?”

“这太不实际了,我劝你还是换一个吧,否则这堆烂泥只能存在宝黄天中无人问津。”

但小狐仙又道:“我当然没想要求完整仙方,谁的残方宝光越高,就取用谁的。”

蛊仙们这才不说话,开始观望。

须臾片刻后,一个蛊仙拿出一个残方,宝光一丈二尺。

这当然不入方源的法眼,他盯着镜面笑了笑,嘱咐小狐仙:“只要其他蛊仙每拿出一件残方。我们就加一斤的和稀泥。”

“好的,主人!”小狐仙立即脆声应答道。

狐仙福地中,和稀泥相当的多。

从始至终全部的和稀泥,都被小狐仙清理出荡魂山,都并未排出福地。而是转移到了福地西部。

这些年,福地中和稀泥不断转化,越来越多,简直快要形成一片泥湖了。

曾经的灾祸,成了如今的资本。

宝黄天中,蛊仙们的神念渐渐又荡漾起来。

每出现一份残方。小狐仙就抛出一斤的和稀泥。这种架势,很明确地告诉别人——咱这里有的是货!

一个个残方陆续抛出,且宝光也相继增强。从原先的一丈多,已经上涨到两丈有余,并且还有上涨的趋势。

方源看了一会儿,嘴角笑意愈加浓郁。

他的灵魂来自地球。营销的手段要远远超过这个世界。地球上商业发动,而宝黄天这里,仙元石只能勉强承担货币的角色,大多数的交易还处在原始的以物换物的阶段。

方源这种手段也算是一场造势,吸引更多蛊仙卖出和稀泥仙蛊的残方。

当然,这充其量也只是个小花招而已。

接下来,方源又嘱咐小狐仙。将其他事务也一一处理,安排下去。

“时间差不多了,该是回去的时候了。”

方源一直掐着时间。他在狐仙福地花费的时间,缩减到五分之一,就是北原流逝的光阴。

“主人,再见。你可要常回来看看呀。”小狐仙带着方源挪移到福地西部,在星萤蛊的照耀下,再次催动了星门蛊。

方源怀揣着许多炼蛊材料,以及低转蛊虫,踏入星门当中。

不多时。他从另一星门中踏出,回到北原月牙湖畔。

和他算计得差不多,此刻已经临近清晨,天边拂晓,一丝鱼肚白。在地平线上渲染。

清风拂面,纯净的湖水带起了阵阵微波。

空气清新至极,脚边的花草上沾着露珠,湖面上已经有影影绰绰的鸟群,在扑棱着翅膀飞舞。

方源深呼吸一口气,心中愉悦。

通过抢夺了马鸿运的机缘,方源此行得到了琅琊地灵的帮助。在琅琊福地中,他用了其中一次机缘,获得了星门蛊等等。

然后回到狐仙福地,再用通天蛊,将资源短缺的困难解决。

“这样一来,原本已经走尽的棋子,都有了新的发展之路。整个棋局都盘活了。”

天边渐亮,在狼群的簇拥之下,方源回到葛家营地。

“我要闭关炼蛊,闲杂人等不得打扰。”他交代一句后,就将自己锁在蜥屋当中,开始炼新蛊。

炼蛊的材料不管来自哪里,在哪里炼成的蛊虫,便是哪里的蛊。也就是说,方源手中的材料,虽然大多来自中洲,但只要在北原炼成,新的蛊虫都是北原本地的蛊,不会受到异域压制。

他首先要炼的,是推杯换盏蛊。

推杯换盏蛊本身是五转,原先的那套在狐仙福地炼成,现在已经濒临破碎,只能再用一次。通过宝黄天,方源现在又有了材料,自然要炼成一对北原的推杯换盏蛊。

与此同时,在月牙湖畔。

严家蛊师一行九人,骑着驼狼急速奔驰着。

“停,这里有狼群脚印!”严家族长忽然勒住狼头,看着眼前大片的狼群足迹,脸色惊疑。

“竟然有这么多的狼,这是万狼群啊……”其余蛊师也纷纷发出惊叹声。

“葛家大营,就在不远处。你们说这万狼群,会不会是朝着他们去的?”

“不妙!你们看这些狼的脚印,有毒须狼、风狼,还有龟背狼、夜狼等等。”

“野生的狼群都是单一的,这么多种狼群夹杂在一起,只能说明这狼群是受蛊师操纵!”

严家族长皱眉沉吟道:“葛家曾经是大型家族,但如今迁徙流浪,早已经不复当年盛况。不大可能养得起这样庞大的兽群。很有可能这兽群,就是魔道蛊师的手笔。我们这次是要向葛家求援,先去看看具体情况。如果葛家不妙,我们就悄悄撤走。如果能做个顺水人情,那就和葛家前后夹击狼群。”

“是,族长大人!”众人齐喝一声。

“走,去葛家。”

驼狼再次奔行,载着一行人,向着葛家大营跑去。

(ps:呜呼哀哉,最近中了感冒蛊,每天都到医院挂水。但大家仍旧在默默的一直支持,感谢!这次俺征集龙套,回馈诸君!这次征集的龙套,都是北原人,很多都会在英雄大会上露脸。经过一系列的情节后,有些龙套都会残留下来,等到五域大战时,和主角、以及南疆的龙套们相爱相杀……目前的《蛊真人》书评区,已经置顶了帖子了。)(未完待续。。。)

\end{this_body}

