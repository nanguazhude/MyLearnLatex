\newsection{未来大计}    %第六节:未来大计

\begin{this_body}

%1
人祖看到了自己的大儿子,顿时大喜过望,奔跑了过去。

%2
太日阳莽也成了白色的鬼魂,正躺在一座湖边,用碗舀起河水喝。

%3
河水如酒,香气四溢。

%4
太日阳莽喝得十分惬意自得。

%5
“我的儿,别再喝了,快和我一起回去吧。”人祖走上前去,喊道。

%6
“我尊敬的父亲,您怎么来了?太好了,和我一起喝酒吧。”太日阳莽睁开惺忪朦胧的双眼。

%7
人祖一把夺过他手中的酒碗,恨铁不成钢地道:“别再喝了,你就知道喝酒!谁不知道,死地中只有迷魂湖,里面的湖水就是迷魂汤啊。鬼魂喝下之后,稀里糊涂,再也不想离开了。”

%8
太日阳莽却道:“父亲,你错了。这不是迷魂汤,是安魂酒啊。喝了之后,能沉淀魂魄,剔除魂魄中的杂质,不再躁动,得到心灵的大宁静。这是天下极品的美酒啊。”

%9
“我不管是什么酒,总之你得跟我回去。”人祖拉住太日阳莽的手,却发现他山一样重,根本拉不动。

%10
太日阳莽摇摇头:“有的人死了,重于山。有的人死了,轻于羽。父亲,我前世有名声蛊,又沐浴荣耀之光坠落身亡。到了这里,身体变得比山还要重,自己都走不动一步,只能躺在这里啊。”

%11
“啊!”人祖顿时心中一沉,焦急地喝道,“我早告诉你,树大招风,出名不是好事。你还不快把名声蛊丢掉。”

%12
太日阳莽又摇头叹息:“名和利,生不带来,死不带去。我死后,名声蛊就离我远去了。倒是定仙游蛊,还陪伴着我。”

%13
定仙游蛊可以将太日阳莽的魂魄。带出生死门,甚至带到外界任何一个地方去。

%14
但只有走上生的命途,才能让太日阳莽真正的复活。

%15
“这可怎么办啊……”人祖发现自己被智慧蛊耍了。他虽然来到了生死门的最深处——无尽黑暗的沉迷死境,也发现了太日阳莽。但他却发现,自己带不走自己的大儿子。

%16
这时,公平蛊开口:“人祖啊,你还不明白吗?通往生死门的只有两条路,都是宿命蛊踩踏出来的。生死由命啊!万物有生就有死,如此天地宇宙才能循环不休。你的大儿子太日阳莽死了。这都是宿命的安排,你就认命了吧。况且他在这里生活得也很好啊,沉迷死境是世界上最安宁的地方。他喝着天底下最好的美酒,再无外界的纷纷扰扰,你体会不到这种幸福吗?”

%17
人祖立在原地。看着自己的亲生骨肉好一会儿,这才叹了一大口气。

%18
他知道带不走大儿子的魂魄,至少这一次是这样的。

%19
他只好告别的公平蛊,还有太日阳莽,离开了沉迷死境。

%20
他踏上另一条路,那是象征生的命途,从黑暗走向光明。

%21
但人祖很快就察觉。行走在这条路上,却比来时的死路,困难无数倍。

%22
生路上的忧患蛊,比死路上要多得多。人祖走死路的时候。越走越顺利,阻挡他的忧患蛊越来越少。但当他走生路,忧患蛊不仅多很多,而且他每前进一步。就有更多的忧患蛊飞过来,拼命地阻止他。

%23
很快。勇气蛊就支撑不住了:“人祖啊,忧患太多了,还会越来越多。死了安宁,生才有无尽的忧患。你快去前面的荡魂山敲胆石。你想要生存,单靠勇气不够,还得有胆识啊。”

%24
人祖连忙来到荡魂山,忍受住魂魄震荡的痛楚,取得胆识蛊。

%25
有了胆识蛊的帮助,他的魂魄壮大起来,虽然还受到震荡,但已经无忧。

%26
他翻阅了荡魂山,来到落魄谷。

%27
落魄谷像是一个迷宫,曲折蜿蜒。时而蔓延出茫茫一片的迷惘雾,能令魂魄松散。时而刮起凛冽如刃的落魄风,专门切割魂魄。

%28
人祖陷入到人生的低谷中,不辨方向。因为胆识蛊而壮大的魂魄,在迷惘雾中渐渐松散。松散的魂魄,受到落魄风的切割,一片片向地上掉落。

%29
人祖差点就要彻底迷失,好在这时,信念蛊飞出来,照亮了他的路。

%30
人祖走出落魄谷,只剩下最精炼的一团魂魄了。

%31
他着实松了一大口气,感到胜利在望。

%32
他来到逆流河,这是生路上最后的一道关卡。

%33
他逆流而上,更加艰难。

%34
无穷无尽的忧患,推挤着他,令他举步维艰。

%35
但他硬生生地坚持住,迈向光明。

%36
“快要到了。”眼看着就要大功告成,人祖望着眼前,只剩下最后一步。

%37
他长吐出一口浊气,松懈下来,忘记了智慧蛊的叮嘱,停下了脚步。

%38
这一停,人祖顿时就被河水冲刷下去。

%39
生活不易,就如逆水行舟,不进则退。

%40
人祖被一路冲刷到落魄谷底,他累的动弹不得,被困在了落魄谷中。

%41
……

%42
方源敲碎最后一块胆石,满足地呻吟一声。

%43
这一刻,他感觉舒爽得不得了。原先沉重的伤势,已经彻底痊愈不说,他的魂魄至少比原先增强了五倍!

%44
一种强大的感觉,从他内心的最深处漫溢而出。

%45
这种感觉,不是肉体上的强壮,而是精神的恢弘。不管是思索问题,还是一心多用,方源都有了一种游刃有余的感觉。

%46
“可惜,荡魂山上的胆石,只有一百多颗。而且里面蕴育出胆识蛊的,也不是全部。”方源心中微微遗憾。

%47
荡魂山被狐仙得到时,就几番易手,一片光洁。

%48
狐仙将其挪移到福地中,先后经营了数十年。几乎每年,她都会奴役大量的狐狸,来到荡魂山送死。用魂魄灌溉山峦,催生胆石。

%49
但这些胆石,也被狐仙用得差不多。狐仙死后,这片荡魂山上的胆石。还是近八年来慢慢生长出来的。因此只有一百多颗,到如今,也都被方源用光了。

%50
按照地灵小狐仙的叙述,想要催生一颗胆石,至少得需要牺牲近万头狐狸。若其中死的兽王越强越多,结出来的胆石也就越多。

%51
这个方法,方源不取。

%52
福地强盛时,狐仙这么干,理所当然。但如今福地衰弱不堪。狐群也是大幅度的衰减。屠杀狐群,不是考虑长远的明智打算。

%53
魂伤痊愈,方源没有松懈,开始静心思考将来打算。

%54
得到狐仙福地,对他的计划无疑有巨大的帮助。但同时也有巨大的影响。

%55
“现在,我有了荡魂山在手,魂魄底蕴会不断地增强。前世又有操纵血蝠群的经验心得,这些巨大的优势就该利用起来。接下来,走奴道是必然的选择。”

%56
“但奴道也有弊端,容易被突击斩首。所以力道修行,也不能放松!”

%57
这样一来。方源就要横跨力、奴两道,组建并喂养两套蛊虫。

%58
换做先前,方源颠沛流离,还养不起。但如今有了狐仙福地。却是毫无问题。

%59
“最完美的结果,就是将第二空窍蛊炼成。拥有第二空窍之后,一套奴道,一套力道泾渭分明。各在一个空窍中,互不干扰。同时真元量上。也足够操纵两套蛊虫。”

%60
届时,方源一挥手,兽群、虫群大军如漫天巨浪席卷,转眼间,就是血流成河,生灵涂炭。

%61
若是哪个不长眼的家伙,自以为是,突入进来,实施斩首之术。那么方源的力道修行,就会让他们明白,现实是何等残酷,花儿为什么这样红!

%62
想到这里,方源从空窍中取出第二空窍蛊。

%63
此蛊形如花生壳,一片金黄灿烂,表面上的纹路,如猩红的血丝。这是第二空窍蛊的胎盘形态,以伪蛊形态结合寿蛊合炼而成,能够长存四十年。

%64
“要想将第二空窍蛊真正炼成,仙元绝对足够,两只三更蛊也很容易到手,但还得需要神游蛊。”方源思索着。

%65
仙蛊唯一,天地中同一时间,只能存在一只仙蛊。神游蛊转化为定仙游蛊,已经不复存在了。这就允许新的神游蛊出现天地。

%66
而且神游蛊有一个非同寻常的妙处,那就是容易得到。不像其他的仙蛊,很难捕捉。

%67
《人祖传》中有着明确记载,只要喝下天底下四种极品美酒,就能在身体中凝成神游蛊。

%68
除开野兽酿造、自然孕育之外,因为人族酿酒工艺日盛,极品美酒的数量比古代要多得多。

%69
寻找四种极品美酒,虽然麻烦了些,但只要花费时间,就必能成功。

%70
对于方源而言,坐拥福地,又有小狐仙地灵辅助,炼制第二空窍蛊,大有希望。

%71
但问题的关键,也在于此。

%72
正因为神游蛊易得,很容易被旁人所趁。一旦其他人得到神游蛊,那么不论方源再喝多少极品美酒,也无可能得到神游蛊了。

%73
“南疆的飞家,就拥有极品美酒壮思飞。东海的七转蛊仙醉仙翁,打造酒海,藏酒无数,每年都开酒会。北原王庭中,有长生酒。这些大势力,或者蛊仙本身就坐拥一两道极品美酒,得到神游蛊的几率,比我大多了。”

%74
这样一算,方源不仅要抵御地灾,经营福地,而且还要尽快地收集四种极品美酒,炼制第二空窍蛊。

%75
“但这些还不是根本,春秋蝉才是最关键的所在。现在五倍光阴流速,春秋蝉恢复速度也变得极快。我重生两次,已经运气极佳,不能再赌第三次的运气。在这次春秋蝉回复过来之前,我至少得寻找到一举成功蛊,或者马到成功蛊,亦或者水到渠成蛊等等。”

%76
以前,是方源实力不够。现在,他有了福地,资本雄厚,已经可以追逐这些蛊虫了。

\end{this_body}

