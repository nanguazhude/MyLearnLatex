\newsection{狼王的狠辣}    %第一百零九节:狼王的狠辣

\begin{this_body}



%1
大战在继续。

%2
因为东方余亮的全力出手,连续屠戮了三支精兵之后,东方部族一方,开始占据明显的上风。

%3
“杀啊!”

%4
“让我们杀光这些黑家的狗腿子。”

%5
“把男人全部杀死,把女人统统纳入我们的营帐!”

%6
东方盟军士气高涨,攻势变得狂猛彪悍,将黑家联军压得喘不过气来。

%7
然而做为始作俑者的东方余亮,心中却有着担忧。

%8
“此刻虽然风光无限,但却是我方提前动用了底牌。杀招七星灯持续不了多久,如果再不怕黑楼兰、常山阴逼出来的话,恐怕……”

%9
念及于此,东方余亮双眼中的寒芒暴涨。

%10
他清明透彻的双眸,转向黑家的中军,在那里停驻着黑旗精兵。这支精兵实力雄厚,远超同济,战斗到如今,这支宝贵的力量一直都没有出动,镇压着大军阵脚。

%11
紧接着,东方余亮目光一转,又看向战场的某个角落。

%12
在那里,聚集着葛家一族的蛊师们,葛光等人正在浴血奋战。

%13
东方余亮目无表情,将星念之云划分两半,一半朝着黑旗精兵侵袭而去,另一半则当空飞下,罩向葛家部族。

%14
看到星云来袭,黑旗军的三大统领面色都变了。

%15
“注意防守!全军一齐催动战念蛊!”

%16
大统领一声令下,所有黑旗军的头目们,俱都奋力催发战念蛊。

%17
战念蛊,和星念蛊、空念蛊一样,亦是智道蛊虫之一。黑旗精兵的大小首领,都配备了三转至四转的战念蛊。

%18
这些战念蛊,原本是作用在黑旗精兵的身上。在作战的时候,战念蛊冲入他们的脑海,使得他们战意滔天,勇敢无畏。

%19
星念之云,奔袭而来,黑旗军的上空,亦升腾起一片赤红的念头。

%20
这些战念,虽然稀疏,但护在黑旗精兵的身旁,帮助他们堪堪抵御住了星念之云的冲击。

%21
“黑家不愧是超级家族,培养的精兵素质,就是和其他部族的精兵不同,远超同济。”看到这一幕,方源亦是心中暗赞。

%22
这些黑旗精兵都是黑家,在平日里不断积累,千挑万选的精锐蛊师,然后加以大量的训练,以及庞大的投资,才打造出来的王牌力量。

%23
他们各个都意志坚强,本身就对念头的冲击有着抵御能力。现在又被战念包裹,星念之云肆虐战场,到现在首次被遏制。

%24
当然这其中,还有关键的一点,便是东方余亮并未全力出手,而是将星云分成了两半,只用了一半来冲击黑旗军。

%25
黑旗军惊艳的表现,令人侧目,和其他溃败的精兵形成了鲜明的对比。

%26
而另一边,葛家方面却惨叫连连,在浩荡磅礴的星念打击之下,溃不成军,被大量屠戮。

%27
方源冷眼旁观,狼顾蛊被他用得妙到毫巅,令他可以清晰地看到葛家的惨状。

%28
葛家不过是他的一个棋子,用来伪装他的身份,他身为棋手,怎么可能因为一个棋子,而置身险境呢?

%29
“还不出手么……”东方余亮耐心地等待了片刻,一直在暗中催动侦察蛊,只要方源出力救援,他便能通过魂魄的波动,准确地找到方源的位置。

%30
但他左等右等,都没有等到方源的出手。

%31
狼王表现出来的冷酷绝情,令东方余亮,也不禁生出一股寒意。

%32
倒是黑楼兰看见自家的黑旗军有支撑不住的迹象,连忙闪现了过来。

%33
“东方余亮,受死吧!”他大吼一声,中气十足,显然已经将反噬的内伤养好了。

%34
东方余亮冷哼一声,从头脑中汩汩地冒出大股崭新的星念,冲向黑楼兰去。

%35
两人在半空中展开激烈的碰撞,相互缠斗不休,一时间不分胜负。

%36
有了黑楼兰的牵制,困扰黑旗军以及葛家的星念,没有了后援支撑,在继续肆虐了一会儿之后,统统消散。

%37
战场混乱了片刻,又回到僵持的状态当中。

%38
十几个四转战圈,已经一小半分出了胜负。四转强者或死或伤,其中风魔、水魔还在相互纠缠。影剑客边丝轩,以及飞电东破空两人,则在战场上穿梭。

%39
这两人都有强悍的移动蛊,即便遭到四转蛊师的拦截,也会轻易地脱身而出。

%40
他们再不断地寻找方源的踪影,可惜方源一直暗藏着,没有出手,导致他们一直搜寻无功。

%41
而同时,在逆雨福地,两位蛊仙一男一女相对坐着,一边品茶,一边看着石桌中央的烟影。

%42
烟影翻滚不休,将黑家和东方家的大战,展现得清晰淋漓,每个角落都细微可察。

%43
女蛊仙谭碧雅收回目光,对男蛊仙东方长凡笑着道:“看来这场大战,还是要看东方余亮和黑楼兰之战的胜负。哪一方得胜了,哪一方便能占据上风。东方余亮这个年轻人不错,明明军势要弱于黑家,却能打成这样的胶着战局,看来长凡兄的调教,颇有成效啊。”

%44
东方长凡高冠古面,一双眼眸中时刻闪烁着成百上千种琉璃之光,作为东方家族唯一的智道蛊仙,他淡淡摇头,语气冷漠。

%45
“实际上,我对东方余亮的指导,也只有过两三句话而已。但这个年轻人是不错,很有想法,回去后大肆宣扬,借助我的势成功上位。他有些天资,又懂得努力。我已允诺他,只要他能入主王庭,我便出手为他妹妹治病,将他做为后继者之一来培养。”

%46
“入主王庭?”谭碧雅微微一怔,轻笑道,“请恕小妹直言,此届王庭之争,恐怕东方部族的希望不大。今年的几大热门中,耶律家的耶律桑,被人普遍看好。此次耶律家的太上家老耶律莱,暗中将仙蛊寄托在耶律桑的身上。这在圈子里,已经是众所周知的秘密。”

%47
“耶律家虽然是黄金血脉,北原有数的超级家族之一,但已经连续八届,没有入主王庭了。正因为如此,耶律莱前些日子,还被黑家的黑城当众取笑。这次动用仙蛊,恐怕也是想找回场子。”东方长凡说完,轻笑一声,笑声中似有不屑。

%48
谭碧雅抿了一口茶,说道:“嗯,谈到黑城,黑楼兰便是他的第二十七房所育下的亲子。这是他的儿子,于情于理,他都会在背后大力支持的。因此黑楼兰亦是几大热门之一。历来王庭之争,不过是几大黄金家族的一场竞争游戏。谁能成为王庭之主,背后势力的支持极为重要。照我说,黑楼兰的赢面,可比你家的东方余亮要大得多。”

%49
东方长凡却缓缓摇头。

%50
谭碧雅见此,眼中闪过一丝感兴趣的光:“怎么?难道长凡兄私底下,也给了东方余亮仙蛊护身?或者有什么安排布置,可保东方余亮入主王庭么?”

%51
智道蛊师谋算之能,蛊仙们不是深有体会,就是早有耳闻。智道蛊师的数量十分稀少,东方长凡是北原有名的智道蛊师,如果他出手暗中布置,只要不公然破坏王庭之争的游戏规则,那么东方余亮便大有可为之处。

%52
但东方长凡却否认了谭碧雅的猜测:“非也,非也。此届王庭争夺,马家势大,可以说已经一只脚踏上了王庭主位。我东方长凡又岂会做无用之功呢?”

%53
他东方长凡,已经垂垂老矣,寿命无多。

%54
他也算到自己死期将至,因此为了家族,也为了自己的传承不断绝,当务之急便是挑选和培养继承之人。而王庭之主,还要放在其次。

%55
不是所有蛊师,有了一套智道蛊虫,就能成为智道蛊师的。东方余亮的天赋,让东方长凡十分满意,甚至隐隐忌惮。而令他更满意的是,东方余亮有一个体弱多病,无法修行的亲妹妹。

%56
这是东方余亮的软肋,只有拿捏住这一点,就不用担心他的忠诚。

%57
王庭之争,只是他为东方余亮布置的一个局。

%58
东方余亮失败之后,为了他的妹妹,肯定会上门求救,这就等若将把柄主动交到他的手中。

%59
如果东方余亮侥幸成功,那是意外惊喜。虽然答应过东方余亮救治他的妹妹,但结果肯定不会治好。

%60
谭碧雅十分诧异:“怎么?长凡兄,你居然看好马家?马家虽然是大型家族,摆在明面上的军势的确不俗,但马家却没有一位成为蛊仙的太上家老啊。”

%61
东方长凡早就等她这句问话,施施然答道:“碧雅小妹,你有所不知,大雪山福地已经秘密地和马家联络,暗地里支持他们了。”

%62
“大雪山福地,那帮魔道蛊仙?”谭碧雅面色一沉,这个消息对她的冲击有些大。

%63
她紧紧盯着东方长凡:“长凡兄长,你是怎么知道这个消息的?”

%64
东方长凡傲然一笑:“这都是我亲自推算而得,你尚是第一个知情的人。”

%65
谭碧雅立即信了七八分,东方长凡乃是智道蛊仙,亲自推算的结果,几乎等于事实。他的情况,谭碧雅也心知肚明,没有欺骗自己的动机。

%66
再者,大雪山福地中的那帮魔道蛊仙,向来对八十八角真阳楼觊觎有加,此次暗中扶持马家,向巨阳仙尊的传承下手,这样的事情,在之前也早就发生过多起。

%67
想到这里,她再也坐不住了。

%68
她是刘家的外姓太上家老,暗中扶持刘文武。刘文武一旦获得王庭之位,那么对于她在刘家中的地位,极有帮助。

%69
马家的存在,严重破坏了她的布局。她当然容忍不下,这便站起身来:“长凡兄,此事事关重大,魔道蛊仙皆是豺狼之徒,但其他同道还被蒙在鼓里。小妹这便去通知他们,请恕小妹告辞。”

%70
“去吧,去吧。”东方长凡缓缓点头,同时开放福地门扉。

%71
谭碧雅离开福地之后,东方长凡古井无波的脸上,这才流露出一丝笑意。

%72
这番谈话,不过是他对谭碧雅的局。

%73
谭碧雅也是一位精明的蛊仙,但奈何身在局中,又有所求,自然就被轻易算计了。

%74
东方长凡又将目光移到烟影当中,此刻战场已经出现变化。

%75
东方余亮久战之下,渐渐不支,只好选择撤退。主帅一退,大军士气骤降,在东方余亮的命令下,也开始撤退。

%76
撤退慌而不乱,显然是有大量训练过的。

%77
东方余亮早就预料到这点,因此在之前就花费了心血,在撤退这个方面。

%78
东方大军徐徐而退,夹带着时不时的反击,黑家许多蛊师反而大意之下,丧命在反击中。

%79
“风魔,你这个无胆鼠辈,这就想要跑吗?”水魔浩激流怒吼着,浑身伤痕累累,鲜血淋漓。

%80
风魔冷哼一声,并不答话,而是在沉默中后撤,坚持执行着东方余亮的军令。

%81
之前大军建立的防线,就在身后数百里外,只要撤进防线,稍微休憩片刻,东方盟军的战力将迅速复原。

%82
到那时,就该轮到黑家大军头疼了,而初战不利不过只是小节而已。

%83
然而这个时候,狼群忽然齐声嗷叫,汇集起来,再度形成浪潮,对东方大军展开亡命的冲锋。

%84
狼群和蛊师不同,蛊师惜命,狼群却悍不畏死。

%85
“可恶!”东方余亮看得睚眦欲裂,在狼群的冲锋之下,东方盟军死伤无数,一股慌乱的情绪很快蔓延全军,继而形成溃败之势。

%86
方源使出八分力气,大师级的奴道造诣,看得人目眩神迷。一波波的冲势,接连不断,东方大军像是泥土,在狼潮的冲刷下,掉落一块又一块。

%87
强烈的魂魄波动,令方源的方位暴露无遗。

%88
但方源早就公然现身,他站在重新安定的双头犀牛的背上,身边是众多汇集而来的蛊师强者。

%89
“狼王常山阴……”东方余亮咬牙切齿,双眼直欲喷火。

%90
这一战,他算是彻底领教了方源的狠辣和歹毒。

%91
说起来,此战方源不过两次出手。

%92
第一次出手,直接引动大军开战,令东方余亮许多布置安排没了用武之地。

%93
而这第二次出手,是逮住了东方大军最为脆弱的时刻,趁人之危,落井下石。要知道蛊师们拼杀了这么久,空窍中的真元早就所剩无几了,虽然还有一战之力,但往往是和野狼同归于尽。

%94
方源的狼群,同样损失惨重。但这已经赚大了,他的野狼可以轻易补充,北原中的野狼多的是!但对方牺牲的,却是宝贵的蛊师性命。

\end{this_body}

