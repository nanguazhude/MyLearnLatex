\newsection{上位者的良知}    %第九节:上位者的良知

\begin{this_body}

岩勇滚出荡魂行宫,穿过数条曲折幽深的暗道,这才来到荡魂山上。

看到这座粉红的水晶山,岩勇吐出一口浊气,远离了方源,他心中的压力和恐惧,消散了许多。

在山上跋涉了好一会儿,他这才被族人们发现。

“啊,伟大的,尊贵的族长,我们的英雄,您在这儿呢!”一些石人立即欢呼起来。

“请让我亲吻您的脚趾,表达我对您的崇拜吧。”一些石人跪倒在地上。

“大英雄,大英雄!你的勇气比天高,你的胆量比地厚。”一些小石人们,成群结队,夹道欢呼着。

岩勇笑着,无人知道他的苦涩。

耳边的欢呼声,如此热闹,身边许许多多的族人,簇拥而来,但他却感觉无以伦比的孤单。

他看着身边的这些族人。这些笑颜洋溢的石人们,三个月后,恐怕都得死了。因为开凿运河而累死。但他又能做什么呢?

在其他石人看来,荡魂山的胜利是多么的伟大,多么的值得歌颂。只有他明白,这不过只是那个仙人幕后操纵的一场游戏。

这个残酷而冰凉的真相,让他无比深刻的明白,那些石人的牺牲,这些艰难辉煌的胜利,是多么的苍白浅薄,是多么的可笑无力。

他带领着族人们,取得越来越多的胜利,他对方源的恐惧也就越来越深。

“那个仙人,他就是个魔鬼!他的心肠比我们石人的心还要冰冷,他的力量比山峦还要广大。我是如此的孱弱,我能怎么办?反抗就是死啊。我承认我的胆怯,我真的怕死啊。我没有睡够,我才一百八十岁而已。”

岩勇一旦想起方源的样子。他的心就被恐惧充满。

他那还未泯灭的良心,折磨着他。

他知道:他即将亲手,把几乎所有的族人们推向死亡。他受到良知的拷问,族人们的每一句赞美,都像个鞭子,将他的心鞭笞得伤痕累累。

“尊贵的,敬爱的族长啊,您终于回来了!大家都在等着您呢。”石人们让开一条道路,让岩勇畅通无阻地走上高地。

“我的族人们。这三天来,我们的族群壮大了许多倍!我们的远征,得到了辉煌的成果。但是这样的胜利,还是远远不够的。你们愿意和我一起,继续走下去。走向美好的明天吗?”岩勇居高临下,大声地问道。

石人们用尽最大的声音欢呼,表达对岩勇一百二十分的支持。

岩勇点点头,这样的情形他早有预料。

石人中,并非没有刺头,或者智慧的老石人。但在多次的激战中,他们都已经“壮烈”的牺牲了。

现在留下来的石人。大部分是刚刚出生的小石人,心思比较单纯。而那些老石人,几乎都是岩勇的铁杆支持者,甚至是岩勇的狂热崇拜者。

岩勇耐心地等待欢呼声结束。这才继续开口道:“这三天来,我一直在无人的地方思考——我们该怎么对付逃走的仙人。仙人拥有了仙元,才能鼓动那些妖狐大军,才有恐怖的战斗力。他一定是退到北部的水泽。或者东部的火坑里喘息去了。我们不能任由他恢复过来。”

“族长说的是啊!”

“族长太英明了,我们不能任由那个可恶的仙人积攒仙元。”

“当那个该死的恶魔恢复了实力。肯定又会找我们石人的麻烦!”

“但是我们该怎么办呢?水泽、火坑都是十分危险的地方,就算是我们石人也待不了多久。而且这两个地方那么的广阔,鬼知道仙人跑到哪里去。”

众石人七嘴八舌,议论纷纷。

岩勇打断众人的议论,他高喊道:“所以,我想到了一个唯一的办法。我们要用土将水泽和火坑,都填平了。这样一来,那个仙人就无法恢复仙元了!”

“天呐,这可是个疯狂的想法!”当即,就有石人惊呼出声。

“我伟大的族长啊,水泽是那么的宽广,令人望而气馁。而火坑有致命的炎热,我们怎么能够用土填平它们呢?这是不可能的事情啊。”有位老石人反驳道。

岩勇深深地看了一眼这个老石人,暗记心中。

这个石人居然敢反驳自己,证明他崇拜自己的程度还不够。将来就给他委派最沉重的工作,将其累死。

这时,又有一个老石人开口:“我们不能蛮干。我想到了一个办法,也许我们可以开凿出一条大运河,将大水引入火坑。水火的力量相互抵消,这样就省事多了。”

岩勇杀机更甚。

这个老石人心中很有智慧,比上一个反驳他的老石人更具威胁。

他当即在心中决定,将来委派这个老石人防护守备的任务。让妖狐大军将其杀死,尽快地除掉这个祸患!

岩勇不咸不淡地称赞了这个老石人一句,高喊道:“我的主意就是这个。我们必须尽快地开凿出大运河,将大水倾泻,将火灾平定,让仙人没有地方积攒出仙元!其实,白石老族长死之前,也教导过我。说北部的大水,东部的火灾都是那个邪恶的男仙人搞出来的,那是他力量的源泉。就好像我们石人吞食泥土一样。”

“原来白石老族长早有预料啊。”

“白石老族长,不愧是我们石人的贤者啊。”

“白石老族长已经有九百九十八岁,他知道的自然很多。”

众石人纷纷点头,对白石老族长表示称赞,同时对他的死表示惋惜和遗憾。

石人常年睡觉,相互之间,交流很少。这种距离感、神秘感,更使得死去的白石老族长的智慧深不可测。

按照地球上的话讲,如果白石老族长泉下有知,听了这话,说不定能被气得从棺材中跳出来。

但可惜的是,他被方源彻底杀死,魂魄都没有放过,被方源放到荡魂山上,震荡崩溃。其精粹落到山上,形成一颗胆石。

这颗胆石后来不知道是被哪个石人,给敲碎了。亦或者是方源本人,也说不定。

议论了片刻之后,石人们纷纷统一了意见,愿意在岩勇族长的带领下,开凿运河,贯通水火。

方源隐居幕后,将一切尽收眼底,见大局已定,便命令小狐仙。

地灵适时地开放荡魂山的一丝威能,众石人顿时感到魂魄震动,头晕目眩。很多小石人都当场昏迷了过去。

“不好,我们赶紧出去异界妖妃。荡魂山要开始发威了!”岩勇一句话,顺利地带领着族人们,走下荡魂山。

他们没有回归原来的家园,而是直接开赴东北,一路浩浩荡荡。

荡魂行宫中,方源透过蒸腾变幻的烟影,目送着这些石人远去,脸上神情无悲无喜。

“主人,你听说过石人的故事吗?”小狐仙的尾巴不安地晃了晃,语气委婉。

方源轻笑一声:“你是想劝我,对待石人要用怀柔的手段吗?”

“主人你好聪明哦。”小狐仙闪着的大眼睛。

“呵。看来你还不太了解。很多时候,仇恨和恐惧的力量,会比感恩更庞大呢。”

当初方源得知,有一支石人部族时,颇为惊喜。

石人居于地底,以泥土为食,善于挖掘。一支庞大的石人族群,甚至可以在地底深处,建造出一座地下城池。

石人可以为福地的主人,开采地底的资源,是一种很好的奴隶。很多蛊仙都要购买许多石人,迁徙到自己的福地当中来。

而对于狐仙福地而言,有着荡魂山,只要魂魄足够。哪怕只有一个石人,都能通过胆石,繁衍成庞大的种群。

方源完全可以凭此,大肆养殖石人,和其他的蛊仙进行奴隶买卖。

当初的狐仙,为什么千方百计地移来石人,也是这样的打算。

的确,石人吃软不吃硬。大多数的石人,都是铁骨铮铮的硬汉子,勇士,不惧死亡。很多蛊仙,都选择怀柔,潜移默化地榨取石人身上的利益。

狐仙就是用的这种手段。

但此法,方源不取。

太温柔了。

榨取的利益就要榨得干干净净!

这个世界竞争是如此的残酷,不仅是人和人之间的竞争,残酷的地灾天劫如洪水般,冲刷了古往今来多少的英雄豪杰。

蛊仙又如何?

如果不争取每一滴每一点的资源,尽快地武装自己,壮大自己,那么狐仙的下场就是最好的例子。

魔道中人,就应该争分夺秒,锱铢必较,搜刮一切,壮大自身!

“既然身为上位者,就该明白:规矩律法、情谊道德,都是压榨利益的工具罢了。宽和和良知,严酷和仇恨也都是如此。”方源心中冷笑。

和石人折腾了这么久,距离第六次地灾,还有七个月了。

北部的水泽,东边的火坑,都是地灾留下的创伤,福地的薄弱部分。地灾袭来,它们就是福地的弱点所在。

一个木桶能装载多少水,是由最短的那条木板决定的。一条铁链能拉起多重的物品,主要看铁环中最脆弱的那一节。

“就算被石人仇恨、憎恶,被无数人咒骂又怎样?”

“如果这世间单纯的仇恨、憎恶、咒骂有用,那还需要力量做什么?”

只要尽快地开凿出运河,弥补福地创伤,累死多少的石人也无所谓。有着荡魂山,将来多多地捕捉一些魂魄,什么石人要多少有多少!

\end{this_body}

