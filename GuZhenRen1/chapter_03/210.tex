\newsection{我们是同门师兄弟!}    %第二百一十节:我们是同门师兄弟!

\begin{this_body}

“蝼蚁,我要杀了你!”

又一次惨遭算计埋伏,巨阳意志自然大怒。

它立即分出一股磅礴的意志洪流,向远处的方源狠狠地冲刷过去。

不过这股意志洪流,经过内层的颠乱雷球、羁绊狼烟的冲刷,缩减三成。经过外层的暴风雪的消磨,又缩减两成。

到方源面前时,已经缩减了一半有余。

“这种程度,只能维持一刻钟不到的时间啊。”方源对迅速接近的攻击视而不见,心中冰雪般冷静,默默估算第三手段能够维持的时间。

他紧握琉璃楼主令,心神一动,下一刻他闪现到八十八角真阳楼中。

这是琉璃楼主令本就有的威能。

之前方源探索八十八角真阳楼,正是靠此暗中传送,才神不知鬼不觉,将所有人蒙在鼓里。

下一刻,他出现在一处冰山当中,让意志洪流扑了一个空。

天地一片玄白,冰寒刺骨的风,吹拂在他的脸上。

吼吼吼!

一只只三眼雪兔,如人般高大,肌肉贲发,从雪地深处钻出来,短短功夫,就将方源包围。

这里自然是八十八角真阳楼中的某一关卡,太白云生就在这里。

雪兔越来越多,旋即就成千上万,对方源虎视眈眈。

这些雪兔,论肉搏能力,不属于风狼、龟背狼。在冰天雪地的环境下,甚至战力还要超越普通狼群。

它们身上寄生着大量的野蛊。更增凶威。

要冲破它们的阻碍,无疑要耗费庞大的精力。以及大量的时间。

不过方源掌握琉璃楼主令,却不需要硬打硬冲。

他心念一动,琉璃楼主令上便有微光一闪。

一众雪兔脸现迷茫之色,杀气顿消。

方源一挥手,它们便一哄而散,钻入厚厚的积雪当中,转眼间就消失无踪。

利用琉璃楼主令,方源可以控制多层真阳楼。他之前特意留手。没有全部掌控,还留有名额。现在用出来,立即将这道关卡化为己用。

没有巨阳意志的阻碍,他顺利你地成为这个关卡的真正掌控者,挥散这些雪兔,自然显得轻而易举了。

咔嚓嚓……

冰川开裂,露出一个洞口。

洞口延伸向下。一直深入到冰山内部。

方源钻入这个洞口,迅速奔行,很快,他便见到太白云生。

太白云生须发如雪,被封印在一块玄冰当中,似乎陷入沉眠当中。一动不动。

他之前渡劫,被颠乱雷球劈中,陷入混乱当中,不能思考。因此被巨阳意志轻易摄入楼中。

太白云生并非巨阳血脉,又是蛊仙。不容易掌控。巨阳意志为了防止他捣乱,又在抓紧时间对付地灵。便趁机将其封印于此。

方源念头一动,玄冰自解,太白云生缓缓苏醒。

“常山阴……”太白云生渐渐恢复神智,他扫视一圈,回想起来,接着盯住方源。

尽管他一直对方源的残暴做派十分反感,但此刻却流露出感激之情:“是你救的我?”

方源傲然一笑:“不是我救下你,还能有谁?太白云生,你知不知道,你已经大祸临头,死劫将至了。”

“大祸临头,死劫将至?”太白云生神色一动,倒是成名人物,没有被方源的“危言耸听”吓到。

而只是缓缓站起,对方源淡然一笑:“愿闻其详。”

方源怨愤地望向太白云生,解释道:“我受师父之命,潜进真阳楼中,解放地灵封印,意图倾覆真阳楼,执掌王庭福地!师父又交给我琉璃楼主令,可用三大手段,对付巨阳意志。可是中途出现了意外,你忽然升仙,反而被巨阳意志借你之力,削弱地灵,导致如今危局。你以为受到巨阳意志的青睐?哼,他只是利用你维护真阳楼罢了!现在的你已经失去了价值,被封印起来,不是我救你,你必将落入巨阳意志手里,求生不得求死不能!”

“什么?”太白云生皱起眉头,流露出惊疑不定的神色。方源的话,不仅信息量大,而且别具冲击力。

太白云生盯住方源,眼中精光闪烁不定,眉头则越皱越深:“我之前还在疑惑,为什么会忽然得到巨阳意志的帮助。而后被摄进真阳楼中后,被无故封印。按照你所讲,倒是能将这一切解释得通。不过,你到底是什么人?为什么救我?你虽然解开我的封印,但并不代表我会无条件地相信你的话!”

方源仰头大笑一声:“我是什么人,你就睁大眼睛看清楚好了!”

说着,稍稍后退一步,当着太白云生的面,取出刀来,照着自己眉心缓缓割下。

一道血痕,随着刀锋一直延伸到肚腹处。

“你这是?!”太白云生着实吃了一惊。

方源淡笑一声,又反手拿刀,照准脑后向下,两手交替握刀,顺势而下,切出一道伤口。

两道长长的伤痕,很快渗出猩红的血迹。

剧烈的疼痛传来,方源却面不改色。

紧接着,他又在胳膊、大腿等关键地方,切割出大大小小的伤口。

整个过程,他手腕稳如山石,面色冷漠,仿佛切割的是他人,不是他自己,感受不到一丝痛楚。

“你这是做什么?”太白云生暗凛,稍稍后退一步,心中疑云重生。

但下一刻,他瞳孔一缩,脸上的惊奇之色再也掩盖不住。

只见方源咬紧牙关,先扒光衣服,又将表皮统统扒下。

他动作干净利落,三下五除二。浑身上下几乎一寸皮肤都没有残留,只剩下鲜红的肌腱。宛若血怪,雪白的牙床裸露,分外渗人。

随后,方源催动治疗蛊虫。

沐浴在一蓬翠绿的光辉当中,他浑身迅速生长出全新细嫩的皮肤。

肌肤渐渐覆盖全身,当绿光散去之后,展现在太白云生眼前的,已经是真面目的方源。

“啊!原来你不是常山阴。而是假扮的他。你,到底是谁?”太白云生虽然仁慈,但并不笨,看到这番景象,立即明白了真相。

他心中忌惮更甚。

眼前的方源,虽然只是个相貌普通的青年,但气度极为不凡。尤其是一双眼睛幽如古潭,深不可测。

他浑身上下洋溢着浓郁的五转巅峰气息,目光凛然如刃,太白云生识人无数,一看便知方源是一个身怀傲骨,心志坚定。不惧任何挑战的天才人物。

如此人物,太白云生纵观自己一生,见过的也不过屈指可数。

“我是谁?”方源淡淡一笑,嘴角微微勾勒出一丝傲气,神情生动。

他目光炯炯。注视太白云生,语气郑重。蕴含着令人下意识去相信的诚信:“我真名叫做方源,此次从中洲来北原,身负重大师命,与你一师同门。”

“方源?一师同门?”惊异之情,一波波接连不断地袭上太白云生的心头,“你究竟什么意思?”

“哼,一师同门还不懂?就是我和你的恩师,都是同一人,我们的关系就是师兄弟!”方源皱起眉头,语气显得有些不耐。

太白云生和方源之间,并不存在语言障碍,他当然听懂了,只是一时间没法接受这当中的重大意义。

当他听到方源解释之后,他的脑海中旋即浮现出一个身影来。

那是一位老乞丐。

有着一头紫红色的乱发,时而疯癫,时而昏迷。但偶尔清醒的时候,老乞丐的目光沧桑似海,气度摄人。

在年少时候,太白云生好心地给了老乞丐一碗水喝。

当老乞丐清醒过来后,便给他三个蛊仙真传的选择。最终少年的太白云生,选择了第三个真传。

这是太白云生记忆中,最深刻的印象,终其一生,都难以忘怀。

多少个日夜,他总忆起老乞丐的身影。

是他拯救了迷茫的太白云生,将太白云生从低谷中拉起来。可以说,是老乞丐铸就了太白云生。没有老乞丐的帮助,绝没有现在的太白云生。

“师父……”太白云生口中喃喃,一直以来,他都将老乞丐当做自己的无上恩师!

他不禁浑身都微微颤抖起来。

他尝试过打探老乞丐的行迹,而且一直没有停止过尝试。可是这么多年过去了,他几乎走遍北原,也没有任何进展。

老乞丐神龙见首不见尾,忽然出现,又神秘消失。

现在,太白云生陡然听到关于恩师的消息,一时间心中充满了激动、喜悦,当然更多的是难以置信。

“现在,你听好了。”方源手指着太白云生,语气很不客气,“恩师的名号为紫山真君,座下有六大弟子,我方源排行第五,主修力道,兼修奴道,两道皆是蛊仙真传。”

“紫山真君、紫山真君……”太白云生如获至宝,将方源凭空杜撰的名字在口中不断咀嚼,脑海中则不由地浮现出老乞丐的那头紫色乱发。

“师父他老人家,平时不修边幅,喜欢云游四海,寻幽探秘。我来北原前,听他说起过你,他曾经给你三个选择,你却选择了最没有用的宙道传承。哼,要是我,一定会选第一份,那份火道传承可是焚海天仙的遗传,威力惊人至极。”方源语气愤愤,向往又惋惜,神情真挚生动,好像真有这么一回事似的。

太白云生闻言,心脏顿时一抖。

他的这个经历,从未有对外人提起过。方源却言之凿凿,竟然准确地说出当初的情形!

他当然不清楚,方源曾经对他搜过魂,确认他是否拥有江山如故蛊的仙方。

“那么,你,你就是我的师兄弟了?”太白云生这时看向方源,目光已经有了巨大的变化。

“哼!”方源不满地撇撇嘴。“你这种充其量只能勉强算是记名弟子,恩师收的其实不少。根本不能和我这种亲传弟子相提并论。不过按照师门的规矩,只要记名弟子成功升仙,就晋升为亲传弟子。按照排位,现在你已经是我的……嗯,那个……五师兄了。”

“哦?”太白云生微微扬起眉头。

“哼,你得意什么!只是暂且因为蛊仙身份,排在我前面罢了。”方源冷冷地看向太白云生,“等我也成就蛊仙。奴力两道兼修,将你打下去,我就是你的五师兄了!当然,看你这么老态龙钟的样子,很显然寿命不多了。也许我根本不用升仙,等你老死就行了。”

方源的语气、态度都极不客气,甚至带着明显的愤恨之情。

这正是他的高明之处。

若是让他言之灼灼。举手发誓,恐怕太白云生还不会太相信。但正是这种包含拒绝、抵触情怀的证明,让太白云生悄然放松警惕,渐渐选择了相信。

太白云生虽然年岁颇长,但和五百年前世经历的方源一比,就小巫见大巫了。

方源没有正面解释。但三言两语,已经让太白云生明白了原委,看到了师门的一角。

太白云生沉吟:“那么,六师弟……”

方源旋即伸手:“别这么叫,我和你不熟!我的任务本来进行的好好的。都是被你搞砸了。你知道师傅花费了多少时间,才在王庭福地有所布置吗?现在搞成这样。你叫我回去怎么向师父交代?你还是叫我方源罢!”

太白云生被方源无力打断,生不起气,反而心中涌起一股愧疚之意,他呵呵一笑,对方源拱手一礼,诚恳地请教道:“那么……方源,事已至此,我该如何帮助你,才能尽量挽回师门的损失呢?”

成功了。

“呵呵呵,精彩,真是太精彩了,小子!你真是阴险啊,啧啧啧,仅凭三言两语,就诓骗了一个蛊仙战力!”墨瑶一直旁观,此刻终于忍不住在方源的脑海中显现出身影来,她对方源的表现赞叹不已。

方源心中冷笑一声,表面上则没好气地道:“嘿,说了这么半天,太白云生你总算说了句人话!恩师法眼无差,没有栽培一个白眼狼。知恩图报,才是我辈楷模嘛。嗯……现在局面很糟糕,最关键的是几乎没有多少时间了。我已经用光了师父留下来的三大手段,你必须积极配合我的行动。接下来,只能依靠我们俩了!”

“尽请吩咐。”太白云生又一礼。

“首先你把这个用了。”方源咧了咧嘴,将十五年寿蛊抛出。

“这是!”太白云生看到这个寿蛊,神情一变,流露出惊异之色。

“用了吧,你或许有了仙蛊人如故,但那东西耗费仙元可不少。我可不想关键时刻,你忽然老死了。”方源语气歹毒,神情冷漠。

但偏偏太白云生却感觉到了一股温暖。方源五百年前世,历经磨难颠簸,掌握人心已经妙到毫巅。

“这只寿蛊……”

“我手中的琉璃楼主令,能控制一部分关卡。真阳楼中的寿蛊,只有这一只,你赶紧用了,别磨蹭!”

但太白云生最终没有用,而是将它收入怀中。

他想到了高扬、朱宰。

至今,他还心怀愧疚。

手中的寿蛊,似乎残留着那股熟悉的血腥气味。

“你怎么回事?”方源发怒,恍做不知情。

“有些原因……”太白云生垂下眼帘,又抬起目光,坚定地注视着方源,“总之你放心,我将提供你最大的帮助!”

方源紧紧地盯住太白云生,咬牙切齿:“混蛋,你知道我为了拿这个寿蛊,不惜动用琉璃楼主令吗?”

太白云生沉默,眼神温柔如玉,却满是坚定之色。

方源的话,让他心中更添温暖,最后的那丝怀疑也烟消云散了。

对视了好一会儿,方源像是感受到了太白云生的决意,收回目光:“哼,要不是你成为蛊仙,是亲传弟子,我才懒得管你死活呢。好了,按照你那份真传内容,再看你渡劫的表现,你应该拥有仙蛊江山如故了吧?”

“嗯。”太白云生点头,“是的。”

方源双眼一亮,裂开嘴,露出雪白的牙齿,并不掩饰自己的喜悦之情:“很好,跟我来!”

言罢,一大群的星萤蛊,从他的空窍中飞出。

“哦,对了,这两个你先帮我拿着,放你仙窍里。我带不过去。”方源又随手一抛。

“仙蛊!”太白云生瞳孔一缩。

片刻之后,太白云生和方源二人,通过星门,回到了狐仙福地。(未完待续。如果您喜欢这部作品,欢迎您来起点()投推荐票、月票,您的支持,就是我最大的动力。手机用户请到m.阅读。)

\end{this_body}

