\newsection{无相手}    %第二百二十四节:无相手

\begin{this_body}



%1
“方源,你跑什么?”

%2
“胆小如鼠的贼子,你怕了吗?敢不敢和爷爷我大战三百回合!”

%3
“你根本不配伪装成常山阴,你连他百分之一的勇气都没有。”

%4
真传秘境中,方源一路飞驰,身后众人追赶,挑衅不休。

%5
面对这些辱骂和挑衅,方源充耳不闻,神情平淡。

%6
他双目紧紧注视四周,真传秘境虽然空间宽广,但并非坦途,还有许多的真传流星四下乱飞,时而相撞。

%7
之前巨阳意志收回了不少真传,但真传秘境中,仍旧还剩下许多。

%8
方源若是不幸被这些真传所阻,就会被身后众人追上。一旦他陷入包围,局面将更加危险。

%9
这时,一道暗红色的真传流星,忽然飞向方源。

%10
风花蛊。

%11
方源却是微微一喜,心头默念。

%12
一朵风凝成的巨大花朵,陡然出现在他的身边。风花一转,带着他的身躯灵敏转向。

%13
咻——!

%14
一声尖啸几乎要刺破耳膜,方源和这团无双真传擦肩而过。

%15
无双真传直射而去,身后追赶而来的众人慌忙躲避,一阵手忙脚乱。

%16
趁着这个功夫,方源速度激增,迅速和身后追兵拉开距离。

%17
黑楼兰等人自然不愿看到这种情况发生,他们各个都是精英,立即反应过来,打出无数远程攻击。

%18
方源冷笑一声,第一空窍中真元狂催。

%19
烁蝺蛊。

%20
这是罕有的宇道移动蛊。在它的帮助下,方源身形顿时消失在原地。

%21
下一刻出现时,他已经在前方五百步之外了。

%22
黑楼兰等人的大部分攻势。都因此落空。

%23
方源停下烁蝺蛊,又改换鹰扬蛊。

%24
烁蝺蛊虽然是五转蛊,但速度不佳,每次消耗许多晶紫真元,却只前行五百步。它主要用来突击移动,躲避攻势。

%25
真正要赶路,长途飞行。还是鹰扬蛊更好些。

%26
鹰扬蛊虽然只是四转蛊,但方源背生六翼,竟是同时动用三只鹰扬蛊!

%27
鹰扬蛊带他一路疾飞。消耗真元要少,速度还十分可观。

%28
看着方源遥遥飞走,追兵们咬牙切齿。

%29
“可恶!这个家伙还真滑溜,只顾着跑路。一点勇士的风范都没有。”

%30
“他不是升仙碎窍了吗?怎么还有这么多的真元可以动用?”

%31
“哼。蛊师保留真元的手段多了去了。碎窍又怎样?只要搞到鱼泡蛊、乞丐蛾、墓芋蛊等等都能存储真元!”

%32
“我的移动蛊快要受不了了。这天地二气的反噬,还真是厉害。”

%33
黑楼兰冷笑连连:“我们受到反噬,他也不好受。一定要继续追下去,不能停。对方是飞行大师,我们追不上很正常。但只要紧追不舍,让他没有时间升仙,我们就达到目的了!”

%34
方源回望一眼,见身后追兵丝毫没有放弃。心中的焦虑便又加深一分。

%35
不得不说,黑楼兰此举刚好打在他的七寸之上。

%36
一阵阵的痛楚。从第二空窍处传来。

%37
第二空窍已经碎掉,形成漏洞,此刻被一颗三色混元气团占据,只待放入本命蛊,达成最后一步,厚积薄发,炸出仙窍。

%38
若是炸不出仙窍,方源就会被天地二气吞噬性命,身死道消。

%39
若是炸出仙窍,且不管成就几等仙窍,单这关键时分,方源会心神放空,陷入毫无防范的脆弱时刻。

%40
这时刻有长有短,因人而异。

%41
一旦在这个时刻,被敌人所趁,方源的下场必然堪忧,极可能死无葬身之地!

%42
现在黑楼兰等人,将方源死死地卡在这一步上。

%43
方源不敢轻易放蛊,只得任由体内压缩到极致的三色浑圆气团阵阵颤抖。

%44
“不能再这样下去!若是拖延到一定时候,三色气团逸散开去,我用来升仙的积累都会竹篮打水一场空,全部化为乌有。”方源脑海中思绪不断,思索中如何打破这个危局。

%45
轰隆隆……

%46
就在这时,真传秘境陡然一阵动荡。

%47
轰隆隆……

%48
旋即,又是一阵动荡,让人不安恐慌。

%49
“这是怎么回事?”

%50
“发生了什么,真传秘境要瓦解了吗?”

%51
黑楼兰、耶律桑等人面现忧色,他们心知肚明:这是楼外的天劫地灾,终于波及到八十八角真阳楼本体了。

%52
方源和太白云生则是大喜,这证明巨阳意志已经拼尽全力,天劫地灾极为恐怖,即便是它也到达了极限。

%53
“该死,该死……终于撑过来了!”巨阳意志喘着粗气,后怕又庆幸,连咒骂方源的力气都没有了。

%54
原先厚达三丈,将八十八角真阳楼包裹得严严实实,宛若黄金甲胄的巨阳意志,如今只剩下薄薄一层,黯淡无光。

%55
三十六道乱歧牙,绝非两两叠加那么简单,威能暴涨了近十倍!

%56
整个过程中,还有雪殇劫电像是不要本钱似的,疯狂乱劈。原本一闪即逝的电光,几乎成了阳光般照耀毫无断绝。可见雪殇劫电可怕的频繁程度。

%57
巨阳意志险死还生,意念只剩下百分之一不到。

%58
它奋力出击,费尽千辛万苦,提前消灭了十一道漩涡。剩下的乱歧牙一齐爆发,巨阳意志又豁出老命来拼,拦截大半,仍旧剩下七道没有挡住。

%59
七道伤口,出现在八十八角真阳楼上。尤其是其中四道,直接洞穿真阳楼,打出四个通道。

%60
方源等人身处真传秘境,感受到的震荡,就是来源于此。

%61
“我的八十八角真阳楼……”巨阳意志查看一番后,心中滴血。差点忍不住哀嚎出声。

%62
损失太大了,损失的这些蛊虫,不仅数目庞大。而且还有一两只仙蛊在内。

%63
要重新补充这些蛊虫,重现八十八角真阳楼全貌,还不知要花费多少时日。

%64
“方源,你这个挨千刀的混蛋。你自己渡劫,却要让我来承担恶果!我要让你后悔出生在这个世界上啊!”

%65
巨阳意志恶狠狠地在心中诅咒。

%66
天劫地灾形成的巨大蚕茧,已见稀薄。

%67
这时从高空中,洒下数道光柱。透过黑漆漆的云层,照射在残破的八十八角真阳楼上。

%68
天劫地灾正在缓缓消散。

%69
巨阳意志将心头的巨石缓缓放下,这一刻差点喜极而泣:“这一次的天劫地灾。终于要结束了!”

%70
“我终于撑过来了!虽然八十八角真阳楼受创严重,但根本蛊虫没有丢失,毁坏的凡蛊数量再多,也可以补充。损失的仙蛊比较麻烦。但也可以重炼。就算被人捷足先登。也可以动用其他蛊虫替代。大局已定,接下来,就要斩除天外之魔,然后好好收拾这个方源贼子!”

%71
看着天劫地灾在眼前缓缓消散,报仇在即的巨阳意志,心情也跟着好转了许多。

%72
就在这时,一只淡蓝色的大手,渐渐显露出来。出现在巨阳意志的面前。

%73
“这,这是?!”

%74
残留的巨阳意志浑身巨颤。比看到七八十道乱歧牙,还要惊惶。

%75
“这是盗天魔尊的杀招——无相手啊。这该死的方源,运气怎么这么差,居然会惹来这样的灾祸!”

%76
九转蛊仙无敌天下,能在整个世界留下独特的印记。

%77
他们就算死了,某些杀招,也会在天劫地灾中再次呈现。

%78
“我知道了!这个方源胆大包天,一心想要杀掉气运之子马鸿运,结果受到鸿运蛊的反噬,运气差得一塌糊涂。因此,才惹来如此恐怖的天劫地灾!”

%79
巨阳意志想到了缘由,心中万分苦涩。

%80
说到底,源头还是巨阳仙尊留下的鸿运齐天蛊。阴差阳错之下,巨阳意志居然被自己的布置坑了一把。

%81
“当年盗天魔尊动用无相手,抢劫盗取无数蛊虫,全天下都无可奈何。不过还好,这只无相手只有五根指头,最多只能抢走五转蛊、蛊、蛊……我的天!”

%82
巨阳意志喃喃自语,忽然声调一扬,失态地惊叫起来。

%83
他看到天劫地灾缓缓消散,大量的无相手相继浮现而出。

%84
两只、三只……二十只、三十只……两百只、三百只……

%85
密密麻麻的无相手,浮现而出,成千上万!

%86
其中从一指无相手,到五指无相手,数目众多,占据绝大多数。

%87
还有六指无相手十六只,七指无相手九只。

%88
甚至,竟有八指无相手三只!

%89
“这不可能!就算方源受到鸿运齐天蛊的反噬,运气多么差,也不可能差到如此地步啊!”巨阳意志难以置信地大叫起来。

%90
“完了,一切都完了!”巨阳意志哀嚎,如果它能流泪,此刻必定是泪流满面。

%91
一时间,大量的黄金意志都涣散开来,仿佛沙硕从八十八角真阳楼的表面洒下。

%92
这是意念在自我崩溃!

%93
再无一丝斗志,受不了这样的打击。

%94
原来天劫地灾并未结束,而是统统转化为了这最后一劫。

%95
无相手!

%96
剩下的巨阳意志,则缩成一团,任由八十八角真阳楼袒露于外。

%97
真传秘境中,方源等人还蒙在鼓里,对这一切毫不知情。

%98
追杀一直在持续。

%99
受到天地之气的反噬,双方损失的蛊虫已经超过百数。

%100
双方都在死撑。

%101
方源身上的移动蛊,所剩无几,但他已经找到了破局的方法!

%102
带着身后追兵,绕过一个大圈之后,他忽然一转方向,向太白云生飞去。

%103
“快,师兄,开放你的仙窍,让我进去。只有在那里,我才能躲避追杀,顺利升仙!”方源暗暗传音道。

%104
“这个……”太白云生不可避免地陷入犹豫。

%105
仙窍乃蛊仙根本,将方源放入其中,一旦他失败,产生的恶果就要太白云生承担了。

%106
“这样关键的时刻,师兄你还在犹豫什么?只有我晋升成仙,我们才有一拼之力啊。你放心,我有十足的把握成功升仙!”方源喝道。

%107
ps:今晚九点准时并群,造成的不便,敬请原谅。

\end{this_body}

