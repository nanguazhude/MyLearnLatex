\newsection{一场戏}    %第八节:一场戏

\begin{this_body}

一秒记住【】/manghuangji为您提供精彩小说阅读。

三天后,伤重濒死的岩勇回到部族,唤醒了所有沉眠的石人。

“换了一个男仙人,他就是一个恶魔,要将我们所有人奴役!”

“不仅如此,他还要搜刮我们石人族的美男,供他玩弄。”

“我们石人天生地养,逍遥自在,怎么可以卑躬屈膝在淫威之下呢?”

“我们当场就反抗,仙人太强大了!但是我们石人不怕牺牲,毫无畏惧,终于将他打伤,打得他逃跑了。”

“其他族人都牺牲了,只剩下我回来。我要死了,但是那个仙人还活着。他逃走的时候,说他带领着他的狐群大军,将我们石人一族全部剿灭!”

岩勇有气无力,悲壮地哭诉着,带给族人们一个惊天的噩耗。

石人们既震惊又恐惧,既悲伤又愤怒。宣战者有之,报复者有之,提倡迁徙者有之,和谈者有之。

他们群龙无首,不管是继承人还是老族长,都死在方源的手中。石人一共八个部族,一片混乱。

有石人想问岩勇具体的情形,但岩勇伤势很重,从回来报警之后,就一直昏迷着。

就在他们还未商量出结果的时候,果然如同岩勇所说的那样,一波波的狐群开始进攻石人部族。

石人们奋起抵抗,但狐群比他们庞大得多,形势渐渐危机,八大部族不得不联合在一起,退居地底进行防御。

但狐群仍旧不放过他们,屡次进攻地底,哪怕每一次进攻都付出惨重的代价,但狐群仍旧源源不断。

石人们痛声咒骂方源,对他的仇恨和怒火,倾尽天河也浇不灭。局面一天比一天不妙。绝望的情绪在石人中间蔓延。

但就在这时,岩勇苏醒了过来。

石人可以通过睡眠养伤,他的伤势大好,立即率领众人,打了几场漂亮的反击战。

“我们石人,是勇敢的一族,根本就不畏惧死亡!”

“哪怕是仙人,也不能折辱我们!”

岩勇四处演讲,士气被他鼓舞起来。

“别看仙人这么强大。其实他外强中干,他只能派狐群来送死,他已经受伤了。”

同时,他又鼓吹仙人受伤,带给石人们希望。

绝望中的石人。紧紧地抓住这丝希望,仿佛溺水者抓着救命的稻草。

岩勇话锋一转,又谈到老族长们。

“他是被老族长们合力打伤的,老族长们的牺牲,是我们最大的哀伤。”

“尤其是白石老族长,他就死在我的怀中,临死前将整个部族托付给我。我看着他魂飞魄散。心中有愧啊。为什么死的不是我,而是他呢!”他说到这里,捶胸顿足,显得十分悲伤沉痛。

立即就有石人们相劝:“岩勇大人啊。你不要悲伤。你能幸存下来,带给我们警示,又带领我们走向胜利,已经十分了不起了。”

“不错。我们铁石族人,都佩服你呢。”

“既然老族长将部族托付给了你。那么就请你率领我们白石人吧。”

石人喜欢的是呼呼大睡,对权位并不热衷。尤其是现在,生死关头,惶惶不可终日,石人们迫切期望着一位坚强、勇敢的族人,来领导他们。

于是,岩勇便先继承了原本部族的族长之位,又掌握了白石部。

一个多月之后,他陆续掌握了其他部族,成为石人八大部族共同的领袖。

又过去半个月,他成功地带领石人,将狐群赶跑,成功保卫了家园。

“但这还不够。只要仙人不死,我们就没有未来。狐群还会重整旗鼓,重新侵袭我们的家园。”

“我们只有进攻,进攻那座仙山,将仙人彻底杀死,才能换来美好和平的生活。”

岩勇紧接着提出,要进攻荡魂山。

一些石人们却显得有些犹豫。

“我们刚刚饱受战乱之苦,正想要睡觉呢。”

“我们石人数量大减,恐怕没有力气去进攻那个魔鬼的老巢。”

“我们有大量的子孙,在惨烈的战争中诞生,需要好好的抚养。让他们长大成年。”

岩勇只好搬出白石老族长。

“我的族人们,我还会带着你们走向死亡不成?”

“进攻仙山,并不是我的主意。而是白石老族长死之前,告诉我的秘密。”

“他老人家说,这座仙山就是传说中的荡魂山。荡魂山上有胆石,我们石人得到这些胆石,就能让我们实力增强,让我们的部族壮大!”

白石老族长是所有石人当中,年纪最高,经验最丰富的石人。被广大石人普遍认为,是部族的贤者。

他的“遗言”,再加上岩勇如日中天的威望,终于鼓动了所有石人,组成远征大军,对荡魂山展开进攻。

方源为此,在荡魂山周围,特意布置了一些狐群,组成薄弱的防线。

岩勇不断地给周围的石人鼓舞大气:“看吧,仙人的妖狐大军已经所剩无几了。我们已经在走向胜利。”

石人一路胜利,高唱凯歌,士气旺盛地冲上荡魂山。

在荡魂山上,方源带着一批狐狸现身,和石人们展开了“大决战”。

方源展现出恐怖的实力,将许多石人杀死,石人尽皆胆寒。

但这时,岩勇站了出来,指出方源受伤,外强中干的“实质”,并和方源展开了“决斗”。

狐群被消灭,方源果真“不敌”,再次被岩勇打跑。

“等我从水火中攒够了仙元,我还会回来的!到那时,就是你们的末日!”撤退前,方源大吼,面目狰狞。

石人们身体粗苯,行动缓慢,又不熟悉荡魂山上的地形,只能任由方源“逃走”。

打跑了仙人,石人们对岩勇敬佩又崇拜,一齐大声地欢呼胜利。

“我的族人们,现在还不是欢呼的时候。”岩勇站出来,“荡魂山不可久留,这是一块魔地。白石老族长告诉我,只有每年的这几天,我们石人才能安全地进出这里。我们赶紧行动,采集地上的胆石吧。三天后,我们必须得离开这里!”

他们的身躯,都是坚硬的石块儿。为了维系这样强大坚固的身躯,魂魄负担很重。一旦石人活动过盛,就会损害魂魄。

因此石人一生中,会拿出八成的时间睡觉,用来孕养他们的魂魄。

当他们的魂魄积累深厚,就会满溢出一部分。这部分的魂魄落到石头上,就会形成新的生命。石人一族也因此得以繁衍。

当石人得到胆石之后,胆识蛊就会壮大他们的魂魄。他们的魂魄一壮大,满溢出来部分,就形成了小石人。

三天后,山中内部,荡魂行宫。

岩勇跪在地上,垂下脑袋,恭敬畏惧地汇报道:“启禀仙人,经过这三天的休养,我们石人部族已经新添了六千小石人。算上我们这些老族人的话,整个族群的人口,比战前整整扩大了三倍!”

方源大马金刀地坐在高高的云床上,俯视着下面的岩勇。

“很好,这样一来,你们石人就有足够的人力,可以用来开凿运河了。接下来,你还记得怎么做么,不用我再教你了吧?”

岩勇连忙答道:“至高无上的仙人,您之前的话我一句都不敢忘记,一直铭记在心中。”

“很好。”方源淡淡点头,“我给你三个月的时间,命你开凿出一条横贯东北的大运河。”

“啊,三个月?”岩勇听了一呆,“伟大的仙人啊,我们石人需要睡眠来养魂。如果活动过盛,就会累死的。大运河这样的长,只给我们三个月的时间,我们石人一族根本得不到休息。这样的话,运河开凿出来,我们石人一族恐怕也要死光了。”

“呵呵呵,当然不会死光的。我已经算过了,会剩下两百余人。”方源笑着道。

岩勇不禁浑身一颤,如今部族有上万的人口,开凿运河之后就只剩下两百多人。这是多么可怕的牺牲啊。

“三个月之后,我要见到一条大运河!如果没有,在杀死你之前,我会将所有的真相都告诉你的族人们。滚下去吧。”方源的声音,如寒冰一样冷酷。

岩勇听到这恶魔般的威胁,害怕得浑身如筛糠直抖。

他对方源有着深深的恐惧,不敢反驳什么,身躯团成一团,竟然真的滚走了。

“主人,当初引进这群石人,可是花费了很大代价呢。”一旁,地灵小狐仙委婉地劝说道,她不忍心看着大量的石人,就如此丧生了。

“你放心,石人一族我还有大用。想要增添人口还不容易么?”方源半躺在椅子上,双眼眯起来,取出空窍中的一只葬魂蟾把玩。

葬魂蟾,四转蛊,用于储藏。乃是狐仙遗留下的蛊虫之一。

它只有巴掌大小,一片灰白之色。背部长满黑色的小疙瘩,一双大大的眼珠子闪着诡异的惨绿色。

它只能储藏魂魄,将魂魄吸入腹中,腹部就越膨胀。

方源用此,将石人、狐群激战中死去的亡魂,都收集起来。

如今这葬魂蟾,肚皮瘪扁。里面曾有的魂魄,都被方源放到荡魂山,培育出了新的胆石。

方源用了一部分胆石,将自身的魂魄增强到常人的六倍。

而剩下的胆石,他留给了石人部族。

石人们不知道,他们的魂魄壮大,有着许多死去的石人的功劳。

(未完待续)

------------

\end{this_body}

