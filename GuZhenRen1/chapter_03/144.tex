\newsection{借力雪蝠御天青}    %第一百四十四节:借力雪蝠御天青

\begin{this_body}

%1
就这样,方源又连续飞了三天,却仍旧不见地平线上冒出圣宫的影子。

%2
方源暗忖:“看来我进入王庭福地之后,被送到很远的地方。否则以我的速度,还不至于三天,都未到达圣宫。”

%3
这一路上,方源也碰到过不少蛊师,以及凡人。

%4
他们都是黑家联盟的成员,进入福地之后,有能力的都向中央圣宫汇聚。没有能力的,则就近选择地方,居住下来。

%5
王庭福地并不安全,有着大量的野生虫群,还有凶禽猛兽。

%6
黑家大军的到来,打破了这里的默契,扰乱了兽群或者虫群原本划分好的各个领地。

%7
因此,碰撞和杀戮必不可少。

%8
但总体环境,比北原外界的暴风雪灾,要好上太多倍了。

%9
方源一路上,就看了好几起屠杀后的场景。有的是兽群横尸,有的则是人的断肢碎骨。

%10
王庭福地,比狐仙福地底蕴深厚了不知多少倍。狐仙福地没有天象的变化,但王庭福地却有。

%11
而且,王庭福地还有白天和夜晚之分!

%12
这点或许在凡人眼中,并不算稀奇。因为在北原外界,同样是白天黑夜,他们过惯了。

%13
唯有方源这样识货的人,才会知道这其中的难能可贵。

%14
福地中,拥有天象变化,已经份属难得。如果有白天、黑夜之分,就证明这方天地,已经具备了深厚的底蕴,至少在宙道法则上比较健全。

%15
一般而言,拥有天象变化的福地,算是第一等福地。方源的狐仙福地,还没有产生天象变化。

%16
而拥有白天、黑夜之分,往往是洞天的特征。

%17
福地是小世界,在福地之上,还有更加健全完善的小世界,便是洞天。

%18
王庭福地已经有了洞天的特征,已经算是一个准洞天了。琅琊福地是从洞天级数,跌落下来的。到如今,也没有日晚之分呢。

%19
夜幕降临了。

%20
方源望着天空,由辉煌的金色,渐渐转变为淡雅的银。

%21
王庭福地的白天,是锦绣辉煌的金之天。而到了夜晚,也非黑漆一片,而是银光璀璨的夜幕。

%22
飞行在高空中,方源亲眼目睹着天空的变化。

%23
银光挥洒而下,没有白天金芒的璀璨和炽热,变得温柔,隐藏着丝丝犀锐。

%24
方源飞行的速度渐渐慢下来,他俯瞰下方,目光逡巡了一番后,寻找到一处缓坡。

%25
凭借多年的经验,他知道这个缓坡是个良好的宿营地。

%26
但他并不急忙降落,而是围绕着缓坡,在半空中盘旋了几圈,身姿像是鸟儿一样自由舒展。

%27
最终当他确认,这个缓坡的安全可靠之后,他这才缓缓地降落下来,收拢了鹰翼。

%28
宽大有力的鹰翼,漆黑如铁。在方源停止催动鹰扬蛊后,便化为无形,渐渐消散在空中。只余下一两片黑羽,洒落在缓坡的草地上。

%29
方源心念一动,调动蜥屋蛊。

%30
旋即,一道奇光从他的空窍中飞射而出,落到跟前。

%31
光芒膨胀、暴涨,最终化为一只庞大的大蜥屋蛊。

%32
蜥蜴张开大嘴,吐出舌头,露出嘴里的门扉。

%33
它的舌头仿佛披上红地毯的阶梯,方源安步踏上,门户自动开启。当他进入蛊屋中后,门户又自动关闭,同时蜥蜴的大嘴合拢起来,不露出一丝缝隙。

%34
方源虽然精力旺盛,但终究还是肉体凡胎,疲惫积累到一定程度,就要进行适当的休整,如此才能保持饱满的精神和战力状态。

%35
喳喳喳喳……

%36
方源刚进去蛊屋不久,便听到屋外传来一阵嘈杂的声响。

%37
他目光一闪,自语道:“果然又是极乐雪蝠群么……”

%38
方源已经摸清楚了规律,每当白夜转化时,天边总会飞出来大群的极乐雪蝠。

%39
极乐雪蝠,浑身洁白如雪,浑身毛茸茸的,没有寻常蝙蝠的丑恶,外形十分可爱。

%40
这种兽群,规模极为庞大,每一支都有数十万的规模。其中夹杂着大量的兽王。万兽王很常见,甚至还有雪蝠兽皇。

%41
就算是方源如今双空窍,拥有四臂风王这等杀招,也无法与之抗衡,只能退避三舍。

%42
雪蝠只在空中狩猎,吞食风中的微粒或者飞虫,大蜥屋蛊并不在它们的食谱范围内。但方源谨慎起见,还是将大蜥屋蛊调到缓坡的背面。

%43
大蜥蜴俯首帖耳,缩起尾巴,团成一团。从高空看去,仿佛就是一块大石头。

%44
方源躺在床上,睡去不久,屋外极乐雪蝠的叫声不知为何,变得急促嘈杂,其中似乎还夹杂着狼嚎之音。

%45
“怎么回事?”方源被这声音惊动,睁开双眼,从床上起来,来到窗边。

%46
只见银色的夜幕中,两团兽群相互绞杀在一起。

%47
雪白的那团,规模庞大,便是极乐雪蝠群。而青墨色的那团,则是一支天青狼群,数目虽然没有蝠群多,但是极为勇猛精悍,配合默契至极。

%48
雪蝠群虽然众多,但在狼群的攻击下,死伤惨重。

%49
方源微微挑起眉头,有些惊异。

%50
天青狼,身上具有荒兽天狼的残留血脉,因此天生的幼崽就可以悬浮半空。而成年的天青狼,将这项天赋充分发挥,能够在空中随意奔腾。

%51
天青狼极为精锐,和其他普通的狼群不同。在天青狼群中,每一头天青狼,都至少是百兽王!

%52
天青狼群往往规模不大,但是战力极强。只是再强的兽群,也经受不住时间的洗礼,红尘的动荡。

%53
在如今的北原天空,已经很少有人能看到天青狼的身影。天青狼越来越稀少,已经相当罕见了。

%54
不过,作为北原最大的福地,王庭福地中生存着如此规模旁大的天青狼群,也不足为奇。

%55
狼群越战越勇,蝠群渐渐不支,留下数万头蝠尸,狼狈而退。

%56
天青狼群却是减员甚少,大部分的天青狼,落到地面上,对准热乎乎的蝠尸下口进食。还有少部分的天青狼群,悬浮在半空中,四处观望,保持警戒。

%57
方源心中一动:“进入王庭福地之后,我的狼群都四下分散。王庭福地幅员辽阔,我一时也无法将这些狼群尽数召集起来。这支天青狼,来的真是时候。不仅可以飞行,跟得上我的速度。还能护我周全,今后碰到蝠群,直接冲突出去即可。”

%58
正巧这时,好几只天青狼,发现了方源的大蜥屋蛊,团团包围了过来。

%59
方源直接出了蛊屋,收起大蜥屋蛊,便向高空的万狼王冲去。

%60
狼群立即沸腾,从四面八方向方源袭杀过来。

%61
但刚刚进食饱腹的天青狼,速度便慢,战斗的欲望也下降了一个档次。对于方源来讲,这是收服万狼王的最佳时机。

%62
方源冷笑一声,左转右突,身形在空中曲折盘旋。飞行大师的造诣,使得这些天青狼只能在他屁股后面吃灰。

%63
被他打上主意的这头万狼王,和其他两头万狼王不同,刚刚作战时,冲在最前面,本身受了伤,野蛊也损失不少。之前方源暗中关注,更知道它的身上的野蛊底细。

%64
冲到它的面前,方源直接开启四臂风王杀招,对万狼王一通暴揍。

%65
这只倒霉的万狼王,被方源打懵了。

%66
方源趁机催动五转驭狼蛊,将其收服。

%67
它一归入方源的麾下,便立即长嗥,立即便有三分之一的天青狼群也跟着叛变。

%68
方源长笑一声,在狼群的围攻下强收万狼王可谓艰险,但此刻既已成功,那就另当别论。

%69
有了狼群,局面顿时逆转。

%70
在方源的操纵下,狼群左冲右突,配合自己,很快又圈住第二头天青万狼王。

%71
野生狼群愤然攻击,方源便将手下的狼群,组成一圈防线,护住自身。自己则亲自动手,和万狼王绞杀在一起。

%72
一盏茶的功夫后,方源再次成功地收服第二头天青万狼王。

%73
大局顿定!

%74
剩下的最后一只万狼王,见此情景,拔腿飞奔,带着自己的麾下狼狈逃窜。

%75
方源先止血,照看浑身的伤势。稍稍打扫了战场,便收起大蜥屋蛊,转移了宿营地点。

%76
这片缓坡散发出来的浓郁血腥气息,很快就吸引了一波接着一波的兽群。如果方源还停留在那里,一定不堪其扰。

%77
在十里之外,他休息了两个时辰,这才振翅而飞,继续赶路。

%78
但这次和前几天不同了,方源的身边,环绕着两头天青万狼王,三十八头千兽王,两百五十六头百兽王,声势可谓浩大。

%79
就这样赶路,不知不觉间过了六天。

%80
在这个过程中,方源陆续发现了三处传承。不过都是小传承,在他眼中,收获可以忽略不计。

%81
值得一提的是,狼群壮大了。

%82
两头天青万狼王的基础上,又增添了一头。如此一来,方源麾下的天青万狼王数量,就高达三头。

%83
王庭福地,乃是货真价实的宝地。里面兽群数目庞大,在北原外界稀罕的天青狼,在这里却是常见之物。

%84
除了天青狼之外,方源在沿途又收拢了一批夜狼、风狼、龟背狼、朱炎狼。

%85
这些狼本来就是他的麾下,只是进入王庭福地之后,被分散在各处。方源收拢的只是当中的一小部分。

%86
ps:深夜还有第二更。

\end{this_body}

