\newsection{为了明天!}    %第十节:为了明天!

\begin{this_body}

%1
轰隆隆……

%2
爆炸声不绝于耳,烟尘也随之漫天而起。

%3
这里是石人开凿大运河的工地。

%4
石人们本身就是以泥土为食,很多石人身上寄生着蛊虫,可以为用。

%5
“族长大人,您不能再这样干了!昨天,又有三个族人累死了,他们死的凄惨啊,连一个后代都没有。”

%6
一位老石人跪倒在岩勇的面前,大声哭泣着。

%7
石人一旦累死,就是魂消魄散,彻底消亡,不会有剩余的魂魄形成小石人。

%8
岩勇捶胸顿足,低吼道:“我怎么可能不知道?我怎么可能不知道?我族又有英雄死了!为了我族光明的未来,美好的明天,他们奉献了他们的生命。”

%9
“正是因为如此,我们才更不能懈怠。我们从开工到现在,已经遭受了许多次狐群的打击。这些狐群的规模越来越大,显然那个活该千刀万剐的男仙人,正在不断地恢复仙元!我们要继续努力,挖开运河,让他失去力量的根基!”

%10
老石人楞了一楞:“但是,族长啊……”

%11
“你是个好石人啊,你是为我族考虑,我明白的。这些英雄不会白死的,看那边,我已经给他们立了英雄墓碑。将来子子孙孙,都会祭奠他们,感恩戴德的。”岩勇手指着远处,在那里竖立着的墓碑,已经密密麻麻。

%12
老石人望着墓碑群,叹了一口气。

%13
石人大批累死的情形,刚刚出现时,岩勇这位新族长就命人建立了这些墓碑。

%14
低落的士气,立即被鼓舞起来。每天都有石人劳累致死,但仍旧干得热火朝天。

%15
“人都死了妖女无心。要这些墓碑有什么用呢?”老石人是少有的清醒的一员,他为此感到深深的担忧。

%16
“族长啊。”他又苦苦劝道,“咱们石人也讲究个传承繁衍。这些累死的石人,魂魄彻底消散,连个后人子孙都没有啊。”

%17
岩勇面色不变,没有说话。

%18
他身边的小石人中,却有人不忿地叫道:“你个老家伙,是不是怕死啊!”

%19
老石人立即梗起脖子:“小子,怎么说话呢?我是老了。但我一直都是石人,石人怎么可能怕死呢?”

%20
“既然不怕死,磨磨唧唧的说什么?”

%21
“对啊。我们这是在为部族做贡献!”

%22
“为了集体,牺牲点个人利益算什么?”

%23
岩勇身边,有着一群小石人。此时纷纷开口叫嚷起来。

%24
“老前辈,你要是感到累了,就先休息休息吧。没有关系的,我的时间很紧,还需要赶到其他地方监督进度呢。”岩勇拍拍老石人的肩膀,越过他,继续赶路。

%25
一群小石人紧跟在岩勇的身后。叽叽喳喳,尽情的表达了他们对老石人的鄙夷。

%26
被这群小辈数落,这让老石人憋闷异常,气得七窍生烟。

%27
他想大声的反驳。但是看到工地周围,到处竖立着大石板。大石板上刻着种种的标语。

%28
“死了都要干!”

%29
“只要人心齐,三天之内,就能建成一条大运河!”

%30
“石人有多大胆。就有多大产。”

%31
“共创石人一族美好的明天!!”

%32
“岩勇族长万岁!”

%33
“为了石人光明的未来,奉献生命。奉献青春!”

%34
狂热的氛围,笼罩在石人们的心头。就算是累死的石人,在临死之前,脸色都是泛着笑容。

%35
老石人想要开口,但张嘴几次,终究没有说出什么来。

%36
他呆呆地跪在原地,好长一会儿,忽然伸出拳头,狠狠地击打在地面上。

%37
砰的一声闷响。

%38
老石人腾的站起,不发一言,佝偻着背影,向工地走去。

%39
岩勇赶到了下一段工地。

%40
河道已经初见规模,大量的成年石人,在河道中挖掘。而旁边,劳力稍弱一筹的小石人们,则组成巡逻小队。有的在监督工地,有的在刻制标语,有的在树立英雄墓碑。

%41
这些小石人,统称为岩卫兵,由岩勇一手组建。

%42
“向族长汇报!”五六位小石人,立即赶到了岩勇的面前,大声汇报这些天的劳动成果。

%43
“报告伟大的族长,我们这段工地,又开凿了五十里!”

%44
“报告伟大的族长,我们这里遗憾地牺牲了一百二十位族人,他们都是我族的英雄!”

%45
“报告伟大的族长,我们居然在工地中发现了三位偷懒睡觉的族人。这是我们的石人一族耻辱!一定要批斗!”

%46
“很好,很好。你们都是好样的!记住,一定要给死去的英雄们立碑。同时,将那些耻辱,都绑起来示众,当众批斗他们,让他们知耻而后勇。”岩勇关照道。

%47
“遵命!”

%48
“你们都是我族的未来,看到你们就像看到我族灿烂的明天。你们也要继续努力啊。”岩勇夸奖道。

%49
小石人们顿时激动得,都浑身颤抖起来。

%50
“一切为了石人一族!”

%51
“敬爱的岩勇族长,您就是我们光辉的旗帜!”

%52
“我们团结在您的身边,一齐走向灿烂光明的美好未来!!”

%53
他们纷纷大吼,目光狂热无比。

%54
岩勇却下意识地避开他们的目光,这些小石人的狂热,让他都感到可怕。

%55
他向远处眺望。

%56
远处,各个河段都在紧锣密鼓地挖掘着,岩勇可以看到许多石人的,面朝黄土背朝天的宽厚脊梁。

%57
宽达数十丈的河道,一段又一段,绵延出视野之外。

%58
这是多么宏伟的工程啊!

%59
每当岩勇看到这样的情景,心潮就澎湃起来——只要团结在一起,石人一族的力量是多么的强大,简直可以改天换地!

%60
但当岩勇又想到方源,那个有史以来最恐怖的恶魔时,他澎湃的心潮。就仿佛寒风吹过,乍然间变成雪白的冰川。

%61
不管是在外游曳的小股狐群,还是石板上的标语,组织起来的岩卫兵,都是那个恶魔的阴谋。

%62
如此多管齐下,硬是将石人一族的反弹削弱到最低的程度。

%63
岩勇亲手实施了这一切,看着运河一天天的成形,他对方源的恐惧就越来越深重。

%64
那个男仙人,不仅拥有恐怖的武力。更让人绝望的是,他的狡诈和阴险也是如此的深不可测。

%65
岩勇感到自己正在沉沦,正在走向深渊。

%66
他卑微的如同蚂蚁,而在他身后,是方源山一样高大的身影。在俯视着他。

%67
他像是一个行尸走肉,又像是个木偶,操纵他手脚的丝线,就掌握在方源的手中。

%68
每天,他看着一个个的族人死亡,他的心都痛如刀绞。

%69
看到族人们热火朝天,拼死拼活的开凿运河。他更感受一种凄凉的悲哀。

%70
“如果可能的话,我宁愿不知道这样的真相,也许活在骗局中,才会显得幸福快乐吧?”

%71
岩勇收回目光。一招手,带着小石人们,赶往下一河段。

%72
……

%73
“这是最后一只发情蛊了,去罢。”

%74
真元早已灌注。方源一弹手指,将青豆模样的蛊。射到空中。

%75
发情蛊一爆,化为粉红色的光粉,洋洋洒洒,落到下面的狐群当中。

%76
整个狐群将这些光粉,呼吸进去,立即骚动起来。

%77
很快,无数只公狐趴在母狐的背上,不断地耸动,向母狐体内注入生命的精华无敌唤灵。

%78
狐狸的孕期,各不相同。比如金狐,孕育两个月左右,就能生下小狐狸。每一胎,大约三到四只。而三尾狐这等荒兽,上百年的孕期都不够。

%79
一般而言,越强大的猛兽,孕期都会越长。

%80
不过在如今的狐仙福地中,狐群都较为普通,孕期较短。

%81
从石人开凿运河的事情,上了轨道之后,方源每天都用发情蛊,催生小狐狸,壮大狐群。

%82
狐仙没有仙蛊,在第五次天灾时,她被魅蓝电影直接杀死,因此身上一套主打的奴道蛊虫,都没有留下。

%83
但是在她的荡魂行宫中,却还留下不少的蛊。

%84
譬如葬魂蟾这样的辅助蛊,发情蛊这样的消耗蛊,以及一些奴道上的备用蛊虫,大多都是驭狐蛊。

%85
为了尽快地增强自己的实力,方源已经将这些蛊用了七七八八。

%86
狐群在他的努力下,数量暴涨了数倍。

%87
狐狸虽是杂食动物,但福地也支撑不住这般数目庞大的狐群。不到两年,很多狐狸都会因为找不到食物而饿死。

%88
不过方源已经管不了那么多了。他现在全部的想法,就是要撑过第六次地灾。

%89
几个月的时间,一晃而过。

%90
运河挖通,横贯东北。

%91
白茫茫一片的水泽,顺着宽阔的河道,奔腾滚滚,一路泛起万千浪花。

%92
哧啦……

%93
大水流入燃烧的火坑中,水火的力量相互碰撞,河流蒸发,形成大量的水雾,升腾往上。

%94
待河水平静下来,黑色的火焰,也被浇灭大半,只剩下边缘,还残留着三处地方。

%95
如此一来,水火调和,北部的大水消退,露出地面。

%96
大量的流水,顺着运河流淌,接连灌注到东部的数十个巨大陨坑当中,形成了一片片的湖泊。

%97
尽管北部还残留着大量的淤泥。东部也早就成了焦土,寸草不生,但水火调和,仿佛是流血的伤口结上了一层伤疤。

%98
只要让时间流逝,东方和北方,都会渐渐恢复生机的。

%99
在方源的秘密指示下,岩勇带着一百多位石人,伤痕累累地返回南方的家园。

%100
“主人,我们拿这些云朵怎么办?”小狐仙望着天空,厚厚的一层云海,心中犯难。

%101
尽管下了好几场的大雨,但仍旧有水汽悬浮于空,凝聚成云。

%102
这些云挡住天光,将东部大片的土地笼罩在黑暗当中,甚是烦人。

%103
要知道福地中可没有强烈的阳光,也没有大风吹拂。这些云层,会严重地影响到东部大部分地区的生态。

%104
“不用管它。”方源看了一眼云海,就收回了目光。

%105
云海的出现,他早有所料,只是细枝末节罢了。如今的重点,还是第六次地灾!

%106
只要撑过去,云海慢慢处理。撑不过去,一切都休提。

\end{this_body}

