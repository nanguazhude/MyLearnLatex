\newsection{偷鞋(下)}    %第一百一十六节:偷鞋(下)

\begin{this_body}

%1
这些老奴皆是老迈之身,费才又是一副破罐子破摔的疯狂样子,一时间只好包围着费才,不敢上前。

%2
费才怒目圆瞪,一脚踢开前面的老奴:“腌臜的东西,小爷我要去拜见少族长,别给我挡道。”

%3
老奴们羞怒交加,眼中闪动着阴毒狡诈的光,却不敢上前。

%4
他们已经看出费才的胸口处,鼓鼓囊囊。许多人都在心中不屑的冷笑:“这个傻小子,还真敢偷!偷了鞋也就算了,偏偏偷了少族长的鞋,哈哈哈,运气真不好啊。本想把他搬倒,让他去扫茅坑去。结果这下子,恐怕小命都没了。活该!少族长的贴身奴仆,是那么好当的么?”

%5
费才被老奴们半包围着,向少族长的营帐走去。

%6
守护营帐的两位蛊师,看见费才,目光像是打量死人。

%7
马英杰站在营帐门口,面沉如水。他处理公务疲乏了,想要出去走走,结果发现自己的鞋子居然没了。

%8
他唤来贴身的老奴,老奴便告诉他,这极有可能被新来的年轻奴仆偷去了,很可能拿去卖掉换钱。老奴又告诉马英杰,这其实已经不是费才第一次偷盗。

%9
马英杰自然大怒。他没有想到,自己一时仁慈,结果却为他招来一个小偷。

%10
费才心中惴惴不安,恐慌早已经充斥心头,但他牢记着赵怜云叮嘱他的话,抬头挺胸来到马英杰的面前,表现得雄纠纠气昂昂。

%11
马英杰看着他,心中暗暗称奇。

%12
那些押解费才而来的老奴们,跟在费才的背后,反而像是簇拥他而来的样子。更关键的是,费才毫无慌张,难道不是他偷的?

%13
不知不觉间,马英杰心中一部分的怒气,被好奇和疑惑所取代。

%14
“小人拜见少族长大人。”费才跪倒在地,声音洪亮。

%15
马英杰俯视脚边的费才,不悦地低喝道:“我的鞋子是不是你偷去了?”

%16
“小人从未有偷取大人您的鞋子。给小人一百个胆子,也不敢去做。”费才矢口否认。

%17
“他骗人,他的怀中鼓鼓囊囊的,揣着什么东西,一看便知!”身后,立即有老奴叫道。

%18
费才冷哼一声,敞开衣襟,露出一团雪白的高等丝绸。

%19
他将丝绸小心翼翼地取出来,缓缓打开,露出里面的鞋子。

%20
马英杰见这双鞋,正是自己穿的那双,不由地冷笑起来:“好,好得很,证据在此,一个小偷能做到你这般理直气壮,倒也难得了。”

%21
“请少族长明鉴。”费才却不反驳,只是用双手托住,带着恭敬的神色将鞋子摆放在地面上,然后额头贴地,一副任凭发落制裁的模样。

%22
“少族长大人,证据确凿,快请你狠狠地惩罚这个可恶的狗东西吧!”

%23
“是啊,他居然敢偷少族长的鞋子。将来,他还会偷更多的东西。”

%24
“他的手脚太不干净了,依老奴看,干脆把他的手砍掉!”

%25
老奴们纷纷觐言,内容恶毒狠辣,费才听得心头乱颤,但谨记赵怜云的话,没有开口做出任何的反驳。

%26
这样的情景,让马英杰生出了些许兴趣。

%27
处死一个奴隶,算不了什么事情。但马英杰向来以“英明仁爱”来标榜自己,约束自己,希望自己将来能接手家族,成为一代明主。

%28
尤其是现在,马家身为大军的首领部族,一举一动都被人看在眼里。因为偷鞋子这样的小事情,冒然处死一个奴隶,会不会被人传为残暴?

%29
马英杰心中也有这等顾虑。

%30
好的名声,营造容易,但维护艰难。

%31
于是他便问道:“我一向处事公正,费才,我给你自辩的机会。”

%32
费才顿时大松一口气,他按照赵怜云的吩咐,果真等到了马英杰的这句话。这给他带来了巨大的信心,他当即暗中决定,一切都按照赵怜云吩咐的那样回答。

%33
于是他答道:“我父亲因家族内斗而亡,少族长攻灭费家,便是为小人报了杀父之仇。少族长如此英明仁爱,小人又岂会做出恩将仇报的事情呢?”

%34
马英杰听到“英明仁爱”这四个字,心情顿时有了明显的好转,他温声问道:“哦?那你难道还有什么隐情不成?”

%35
但费才摇头:“没有什么隐情,只是小人想着报答少族长您。但小人能做什么呢?小人只是一个凡人,没法子为少族长您冲锋陷阵。小人蠢笨不堪,没法为少族长您出谋划策。小人只是您的贴身奴仆,只会洗鞋子,把鞋子摆放好。小人设身处地的想,这鞋子摆放在外面这么久,少族长您穿上,难道脚不冷吗?于是小人便用全部的积蓄,买下这片真绸,将少族长的鞋子包裹起来,捂在胸口,这样一来,少族长您穿上这鞋子,就不会感觉到冷了。”

%36
“哦?竟是这样!”马英杰听了这话,大为惊异。

%37
他有洁癖,若是费才直接将鞋子放到怀中,他反而厌恶。

%38
但用了丝绸包裹,却是不同。

%39
而且这片上等的丝绸,货真价实。没有谁会用这样的布料,去包裹鞋子的吧?

%40
“这个费才,是个好奴才,倒真是有心了。”马英杰思绪电转,看向费才的目光悄然发生了转变。

%41
如果费才所言是真,那么他的忠心昭昭,实在是令人感动!

%42
这时,费才猛地磕头:“少族长,我有罪!”

%43
“哦?你有何罪?”马英杰看着费才,嘴边已经流露出了明显的笑意。

%44
费才答道:“小人只顾着为少族长大人暖鞋,却忘了若是大人您要出去时,会极不方便。小人有罪,请大人您责罚吧!”

%45
马英杰长叹一口气:“我的鞋子可不止一双,今天我只是看见常穿的这双不见了,这才唤你过来。也幸亏如此,险些叫我错怪了我的一位忠仆。”

%46
“少族长大人,您不能听信他的一面之词啊!”

%47
“少族长大人,这小子妖言惑众,诡计多端,花言巧语得很呐!!”

%48
身后的老奴们,看到平时里呆呆傻傻的费才,竟然巧舌如簧地在他们眼皮子底下,硬生生地咸鱼翻身了,一个个都急得大叫起来。

%49
这时,费才又道:“请少族长明鉴!偷鞋子的事情,的确有过,但小人从未做过。反而是小人身后的这些老奴,做过许多次。小人担任了这份职务之后,这些老奴就多次暗示小人,因此对小人产生了嫉恨。小人不怕清查,也不怕惩罚,恳请少族长大人遣人明察,还小人一个清白!”

%50
费才当然不怕查,这是他第一次偷鞋子!

%51
他按照老奴们在他耳边“不经意”说的那样,偷了一双最精美的鞋子,好卖个高价钱。

%52
费才懵懂,跟了少族长身边这么长时间,也没有留心少族长脚下的鞋子模样,就这样轻易地落到老奴们的算计之中。

%53
幸运的是,他在关键的时刻,碰到了关键的人。赵怜云成了他的救星,在她的指点下,费才成功翻盘,转危为安。

%54
老奴们听到要调查,一个个脸色都变了,苍白如纸。

%55
蛊师的手段,自然丰富多彩。要彻查这等芝麻小事,自然是轻而易举的。

%56
这些老奴已经后悔死了,没有想到最后反而把他们自己都搭了进去!

%57
马英杰看着老奴们的神情变化,心中已经对费才确信了七八分。但他立志成为“明主”,自然不会单凭心中想法,就冒然下达命令。

%58
当即,他就唤来侦察蛊师,命令他调查这件事情。

%59
侦察蛊师得到马英杰的亲口命令,自然卯足了劲头调查。只花了一盏茶的时间,事情就水落石出了。

%60
事实面前,老奴们统统跪在地上,哭泣着,哀嚎着,害怕得体如筛糠,请求少族长的饶恕。

%61
马英杰冷哼一声:“你们这些奴才,媚上欺下,居然敢哄骗我!本来,你们依罪应当一一处死,但念在你们服侍了我多年,其中几位更是我在孩童时候,就伴随左右的人。我就饶了你们的狗命,统统给我发配到辎重营,给我大军服务去。清粪便,扫茅坑!”

%62
“谢少族长不杀之恩,谢少族长不杀之恩!”老奴们磕头如捣蒜,千恩万谢。

%63
“至于你……”马英杰将目光落到费才的身上,戏谑地微笑道,“你居然敢偷拿本少主的鞋子,胆大包天!今后,就将你发配为奴仆长,好生服侍本少主,好戴罪立功!”

%64
费才听得一愣一愣,好半天明白过来,马英杰说是发配,其实却是擢升。

%65
他连忙叩首答谢。

%66
马英杰哈哈一笑,挥手道:“好了,还不给我滚下去,好好想想服侍本少族长的好法子!”

%67
“是,大人。”费才退下之后,走在回去的路上,神情一阵恍惚。

%68
好半天,他这才清醒过来,自己这次居然因祸得福,成为奴仆长了!

%69
“这一切都得感谢小云姑娘……啊,对了,小云姑娘叫我若是没事,就要赶紧向她汇报的。”费才一拍脑袋,连忙转变方向,向约定的秘密地点走去。

%70
“什么,你居然成为奴仆长了?”赵怜云听到这个消息,不由地瞪大了双眼,惊喜地看着费才。

%71
她自忖这个布置,虽然巧妙,但也有风险,完全是看着马英杰的心情来。

%72
若是马英杰的心情糟糕,只消下达一个处死的命令,他费才就铁定完蛋了。但显然,这大呆瓜运气不错,不仅没事,而且还担当了奴仆长。

\end{this_body}

