\newsection{方源的尴尬}    %第五十八节:方源的尴尬

\begin{this_body}

%1
天元宝皇莲,乃是六转仙蛊。※天生地养的自然蛊,而是由元莲仙尊炼成。

%2
三转的天元宝莲,四转的天元宝君莲,五转的天元宝王莲,能够出产真元,为蛊师的真元恢复提供极大的帮助。

%3
而六转以上的天元宝皇莲,则能够出产仙元,用途极大。

%4
元莲仙尊凭此蛊,成为历代九转蛊仙中,仙元最充沛的人物。

%5
元莲仙尊死后,他的天元宝皇莲受到重重保卫,是天下有名的重宝。但保存了十几万年后,被一位胆大包天的七转蛊仙偷走。

%6
这位七转蛊仙,就是元莲仙尊之后,成就盗天之名的盗天魔尊。

%7
盗天魔尊死后,这只天元宝皇莲也损毁了。

%8
长毛老祖和盗天魔尊有过交流,魔尊失踪数百年后,长毛老祖就尝试炼蛊。最终重新炼成了一只六转的天元宝皇莲。

%9
后来他又按照自己的喜好,将其进行提升,达到八转的地步。

%10
仙蛊唯一,名称统一。六转之后往上晋升,名字并不会更变。

%11
举个例子——天元宝皇莲、春秋蝉,若从六转升到七转,仍旧是叫天元宝皇莲、春秋蝉。

%12
长毛老祖死后,执念结合天地之力,化为地灵。这只八转的天元宝皇莲,也一直留在了琅琊福地当中。

%13
“琅琊福地后来被一位称号鬼王的魂道蛊仙意外发现,引发了第一波攻潮。结果都被地灵俘虏。但琅琊福地的消息,因为鬼王而走漏,吸引了更多的蛊仙。之后组织了第二波,第三波等等攻潮。直至第七波,天庭中专门派遣数位七转蛊仙,分别携带着仙蛊下凡。”

%14
“一场惨烈的大战之后。琅琊福地终于崩解,而蛊仙也伤亡惨重。凤九歌就是死在此处。而琅琊福地中的海量秘方,都被充入天庭。中洲因此实力暴涨,以一洲之力,反击四大域。”

%15
有着前世的记忆,方源对接下来的发展,心知肚明。

%16
现在方源忽然提出,要用仙蛊秘方换取天元宝皇莲,地灵毫不犹豫地否决道:“这当然不行!天元宝皇莲可以产出仙元。因此我才有资本,抗衡屡次的灾劫。同时这些仙元,我还可以要用来炼蛊。天元宝皇莲绝不会换的。”

%17
“是这样啊。”方源点点头,虽然心中早有预料,但仍旧有一些失望的情绪。

%18
但地灵的下一句话。又点燃了他的希望。

%19
“不过,我这里还有其他的仙蛊。要看你有什么秘方了。七转的秘方,只能换六转的仙蛊。八转的秘方,可以换七转、六转的仙蛊。”

%20
方源眼中顿时绽放一道神光。

%21
他知道琅琊福地中,还有一只七转的驭兽蛊。

%22
此仙蛊可控天下任何的野兽、异兽、万兽皇,甚至荒兽、上古荒兽!

%23
长毛老祖当年,就是用了此蛊。收服了许多荒兽,甚至几头上古荒兽,埋在十二云阁的地基下面。

%24
有这些荒兽护卫,整个琅琊福地固若金汤。一直顶了六波蛊仙的狂攻,直到第七波时才支撑不住。

%25
“这驭兽蛊关系重大,地灵肯定也不会换。不过长毛老祖当年炼了许多仙蛊,绝不只有驭兽蛊、天元宝皇莲这两只。”

%26
方源怦然心动。想了想,开始埋头书写秘方。

%27
老爷爷地灵就站在他的身旁。背负双手,盯着看。看了一会儿,他哈哈一笑:“这是第二空窍蛊吧。”

%28
“咦?难道你也有这个秘方?”方源停下手中的笔。

%29
“当然,我这琅琊福地收录无数秘方,仙蛊秘方足有数千张。”地灵颇有傲意。

%30
“如此财富,难怪引得天庭都来进攻啊。”方源目光闪烁,在心中暗自感叹一声。

%31
“这样的话……”他沉思了一下,又换了一张纸,继续书写。

%32
他这次写的,是血神子的六转秘方。

%33
地灵看了秘方开头,顿时眼冒奇光。但渐渐的,光芒黯淡下来。

%34
当方源写了三行之后,他道:“你这血道秘方虽然奇妙,却是个残方。虽然补足了一些缺漏,但本身法理冲突,炼成功的可能性很小。你凭这张秘方,可换不来仙蛊。”

%35
方源叹了一口气。

%36
当年他拿到的这张血神子秘方,也是残篇。经过自己刻苦钻研,又请其他蛊仙共同修补,才有了这张秘方。

%37
他也知道这秘方并不十分正确。当年他之所以没有选择练就血神子,主要原因也有这个。

%38
后来他机缘巧合,获得了春秋蝉的正确秘方。于是便舍弃血神子,改炼了春秋蝉。

%39
但春秋蝉涉及到他最大的重生隐秘,就算是琅琊福地没有这道秘方,方源也不会拿出来换蛊的。

%40
他在三王福地时主动暴露,那是因为深陷困境,想要一搏,夺取惊天的利益。

%41
现在的情形,他又不是走投无路,根本不需要冒险。

%42
只是如此一来,这道血神子的秘方,方源也写不下去了。

%43
他所知的仙蛊秘方,也不过十几张。但绝大多数,都是残篇。血神子秘方已经是残篇中最好的情况。

%44
他只有两个完整且又正确的仙蛊秘方,一个是第二空窍蛊,另一个是春秋蝉。但这前者的秘方,琅琊福地中已经有了。后者方源不敢随意暴露。

%45
方源一阵沉默,思索了片刻后,开口问道:“地灵,我若是用五转的秘方,换一只星门蛊可以吗?”

%46
地灵摇头:“不可以。只能换我手中的现有的蛊虫。”

%47
方源不甘心,反问道:“地灵,你难道不想炼制星门蛊吗?要知道这可是一只全新的蛊啊。”

%48
“当然想啦。虽然不适合我用,但是我却可以放进宝黄天中贩卖。嗯……不过我到底什么时候炼,什么时候能炼制成功,那可说不准了。”地灵忽然反应过来。

%49
他在这个时候,倒变聪明了。

%50
“小友,你是盗天魔尊的继承人。在我这里,你有三次机会,让我给你炼蛊。但是炼制仙蛊,不管成功还是失败,都算一次。若是炼制凡蛊,我一定会将炼成的蛊交到你的手中。你若是想要星门蛊,我一定给你炼制出来。”地灵提议道。

%51
方源这才恍然,原来还有这个细节。

%52
难怪马鸿运他要选择三只五转蛊。当初他沦落到这里来时,只是一介凡人,可能还不晓得仙蛊的价值。再者,凡人之躯又用不了仙蛊。所以,他才选择了三只五转蛊,使得自身战力暴涨。回到外界后,力挽狂澜,重新归拢军队反败为胜。

%53
等到马鸿运成为蛊仙,他这才明白,昔日在琅琊福地的机缘是多么可贵。可惜到了那个时候,他已经悔之晚矣。

%54
马鸿运的尴尬,也是方源的尴尬。

%55
方源现在也是凡人,纵然有仙蛊,也运用不了。地灵小狐仙可以运用仙蛊,但不管是洞天蛊、通天蛊,还是星门蛊,都无法承受仙蛊。

%56
也就是说,方源即便拥有了仙蛊,也不能带会狐仙福地里去。

%57
而且这三次练就仙蛊的机会,也有个前提,那就是方源必须提供仙蛊的秘方。

%58
如果秘方有误,是练不成的。仙蛊唯一,如果别人已有相同的仙蛊,方源也是炼不成的。

%59
按照刚刚地灵所言,如果炼仙蛊失败,这次机会就丧失了。

%60
盗天魔尊当年能六次炼成仙蛊,一是因为长毛老祖有傲骨,不屑于故意失败。二是盗天魔尊才华横溢,能力强大,充分准备和利用了每一次的机会。令长毛老祖失去了大量珍稀的炼蛊材料,六次大出血。

%61
现在长毛老祖死了,化为地灵,虽然也不会在给方源合炼仙蛊时捣鬼,但他到底是地灵,并非长毛老祖本人,炼蛊的本事是要打折扣的。

%62
那么,究竟要不要耗费这么一次宝贵的机会,炼一只星门蛊呢?

%63
方源陷入沉思。

%64
若按照一般的常理,这三次机缘最充分的利用,就应该效仿当年的盗天魔尊,炼成三只仙蛊。

%65
但对方源来讲,困难重重。

%66
首先,他没有仙蛊秘方。其次,他没有盗天魔尊的能力,不能确保仙蛊炼成。炼蛊若是失败,就浪费一次良机。还不如改炼五转蛊。最后的关键,他只是凡人,就算是仙蛊也运用不了。仙蛊虽然价值很大,但对自己没有帮助,带出去反而引来祸端的话,那还不如一只五转蛊。

%67
甚至五转蛊也不适合方源,以方源如今在北原的实际情况,用四转蛊刚好。

%68
“当然,这三次机缘,我可以留着,暂时不用。反正这个机缘已经抢到了手中,就算马鸿运亲至,或者那处盗天魔尊的传承现世,也没有什么影响。但我现在,真的需要星门蛊啊!”

%69
方源陷入犹豫当中,感到很为难。

%70
他现在的力道修行,只有三十钧力气,急需改造自身皮肉的蛊虫。奴道修行方面,喂养狼群负担极重,没有家族支持,单靠他个人只能维持在三万左右。

%71
魂魄上,狼人魂已经炼成,就差去往荡魂山,再运用胆识蛊壮大了。境界上,明明有条件突破了,但因为异域压制,迟迟不能突破到五转。

%72
他现在的情况,就像是一个人下棋,把所有能动的棋子都走到不能在走的地步。他已经陷入一种僵局,一个深深的瓶颈。

%73
(ps:可恶,喝酒喝多了,今天头疼得很。不过两更是必须的,不能辜负大家对我的支持。新的一月,希望多一些保底的月票啊!)

\end{this_body}

