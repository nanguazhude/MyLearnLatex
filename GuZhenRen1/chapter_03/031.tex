\newsection{死在路上也丝毫不悔}    %第三十一节:死在路上也丝毫不悔

\begin{this_body}

即便是白天,腐毒草原上也是一片阴沉。文学馆厚重的阴云,阻挡了阳光的恩泽。

在低缓的土丘背后,一头驼狼悄悄地潜伏着。

驼狼体型庞大,堪比战马。它浑身长着又黑又长的狼毛,背山长着两块干瘦的驼峰。一对狼眼在昏暗中,闪烁着幽幽的光。

它趴在土丘上,一动不动,宛若石像。它甚至连呼吸都放得极为缓慢,乍一眼望去,仿佛是一块黑铁。

忽然,驼狼的一对修长的狼耳朵,颤动了一下。

在它的目光注视下,一只灰色的兔子,钻出土丘脚下的一个洞穴,开始觅食。

尽管窝边就有着丰美的野草,但灰兔不管不顾,直冲出去,寻找远处的草。

兔子不吃窝边草,吃了窝边草,它居住的洞穴就要暴露了。

驼狼看到灰兔出去,它的眼帘越加低垂,遮掩住巨大部分的狼眸,只剩下一丝缝隙。

灰兔一边吃草,一边高高地竖起双耳。一旦有什么风吹草动,它都会敏捷无比地抬起头来,四处张望,十分警觉。

驼狼耐心十足,灰兔吃得很欢,而它却一动不动,好像死了一般。

灰兔继续吃草,沉浸在美食当中。

当它吃饱了,它开始往回走。

就在这时,驼狼猛地出动农女的秀色田园全文阅读。它从土丘上窜出来,向灰兔杀去。

灰兔回去的路,被驼狼堵截了,它惊骇之下,只好转身奔逃。

它速度奇快,奔跑起来,仿佛化身成一道灰白色的闪电。在草丛中游走。竟然超越了驼狼的速度,很快就拉出一段距离。

但奔跑了一会儿后,它慢了下来。

灰兔的爆发力强大,但是耐力上却不如驼狼持久。

两者在腐毒草原上追逐,生死时速。上演着草原上最常见的一幕――猎食者和猎物之间的死亡游戏。

驼狼渐渐追近,眼看着灰兔已经落到自己的跟前,驼狼一跃而出,扑杀过去。

但就在这一瞬间,兔子竟然猛地提速。整个身体向旁边一窜。立即躲避了致命扑杀,并且拉开了和驼狼的距离。

这只灰兔也十分狡猾,刚刚的疲惫只是一种伪装,它还保留有余力。

驼狼没有扑中,闷头继续追击。

很快。双方又渐渐拉近距离。

驼狼再扑,仍旧没有扑中。

连续三四次后,兔子真的力竭了,终于被驼狼扑杀。

驼狼喘着粗气,趴在地上好一会儿,这才缓缓地站起来。在这残酷的竞争中,猎食者也并非风光无限。而是有着许多的辛酸艰苦。

辛辛苦苦捕捉到的灰兔,驼狼并没有享用这份美食,而是将它叼在嘴里,回往巢穴。

在巢穴里。还有母驼狼,以及数只新生的狼崽子需要喂养。

当时当这头驼狼,赶回到巢穴时,却只看到血迹和冰冷的尸体。

嗷呜!!!

它丢下灰兔尸体。愤怒地仰头长吼。它颈部的狼毛倒竖起来,仇恨的怒火让它双目通红。

一大批的毒须狼。从四面八方,向它包围过来。

在远处的山丘上,方源环抱双臂,居高临下,俯视着这片战场。

“呵呵呵,果然来了一头公狼。”他淡淡笑着,觉得最近的运气总算有些好转了。

驼狼是北原上较为优秀的坐骑。虽然方源手中有常山阴的四转狼奔蛊,但是消耗真元不少。远不如骑乘驼狼,来得方便快捷。

当方源意外地发现这个狼窝的时候,就将窝里虚弱的母狼和幼崽全部杀死,并获得了一只二转的驭狼蛊。

他没有急着离去,而是埋伏下毒须狼,等待着公狼的回归。

驼狼和毒须狼群的战斗,一开始,就进入了白热化。

驼狼体格巨大,愤怒的情绪令它战斗起来更加凶猛。狼爪抡起来,普通的毒须狼不是它的一合之将。

但在方源的指挥下,毒须狼表现得极为狡猾,并不硬拼,而是相互间巧妙配合,你退我进,你进我退,消磨着驼狼的战斗力。

一直磨了大半个时辰,驼狼气喘吁吁,再不复刚刚的勇武。

在它的身边,躺着六十多头毒须狼的尸体,都是它创造的辉煌战绩。当然若方源一心想要杀它,凭借方源的奴道造诣,只需要付出三十头毒须狼的生命贪欢,攻身为上最新章节。但方源是想活捉它,因此战斗起来,就不免有些束手束脚。

“火候差不多了。”方源看着驼狼在风中,不断颤抖的四肢,慢慢走下山丘,小心翼翼地接近。

如今,他身上大部分的蛊虫,都通过推杯换盏蛊,转移到狐仙福地中了。

距离驼狼还有两百步远时,方源手指一弹,催动二转驭狼蛊。

驭狼蛊轻轻一爆,化为一股轻烟,罩落在驼狼的身上。

驼狼连忙后退闪避,但轻烟亦步亦趋。驼狼发出嘶吼声,向方源展开冲锋。但遭遇到毒须狼群的强力阻击。

几个呼吸之后,轻烟完全融入它的体内。

驼狼无力地趴在地上,浑身都是流血的伤口,通红的一对狼眼,不再仇恨地望着方源,而是流露出一股臣服的意味。

“百人魂的确实用啊,若是没用胆识蛊的话,要奴隶这头驼狼,必须得耗费一番功夫呢。”方源心中感慨了一下,便又催动空窍中的狼烟蛊。

狼烟蛊飞射而出,化为滚滚浓烟,包裹住驼狼,以及大部分受伤的毒须狼。

片刻之后,浓烟散尽,驼狼身上的伤口彻底痊愈,甚至新生了茂密的狼毛。受伤的毒须狼们,也恢复了活力。

不过,虽然浑身无伤,但战斗力仍旧不在巅峰。

影响兽群战力的,不仅是伤势,还有温饱的程度。

狼群要发挥出最大的战斗力。不能太饿,饿了就虚弱。也不能太饱,饱腹下反而影响战斗力。

之前驼狼狩猎,为什么要耐心地等到灰兔吃饱返回?也是这个道理。

只有让狼群处在半饥饿的状态,才能让它们战斗厮杀起来,更加凶狠残酷。

战斗了这么长时间,不管是驼狼,还是毒须狼体力消耗都很大,都饿了。

方源心念一动。毒须狼们就开始啃噬地上的狼尸。而驼狼则将那只灰兔吞下肚子,然后又在方源的强制命令下,将死去的母狼和幼崽也都吞食。

方源站在原地,也取出干粮,就这凉水吃着。

距离杀死葛谣。已经过去了三天。

葛谣是必死的,当她看到定仙游的那一眼,就注定了她的死亡。

更何况,她先是目睹方源赤身裸体地到达北原,又看到他埋下仙蛊,以及运用推杯换盏蛊的情景。

她知道的东西太多了,在方源的心中。早就是必杀的目标。

只是方源初来乍到,战力薄弱,要在腐毒草原行走,葛谣的确能给他带来帮助。

但葛谣不能活着。她的天真既然能被方源利用,那就能被其他人利用。这样的累赘,魂魄不过常人标准,只需中了他人的读心蛊或者回忆蛊。就能让方源所有的安排成为一个笑话,一路掩藏的秘密公之于众。

方源杀人是早有预谋的。

随着葛谣和他一起闯过鬼脸葵海、钻地鼠群、影鸦拦截。之后寻到常山阴,换了皮肤,找到雪洗蛊,埋下地藏花王蛊,她的利用价值就在一步步的缩小。同时,她的威胁则在一步步的提高。

她对于方源的爱,更令伪装成常山阴的方源,感到如鲠在喉,如芒在背豪门通缉令,女人别跑。

一个热恋中的女子,自然会千方百计地了解所爱的人,包括现在、未来,过去。

当她发现真相时,会怎样?

更何况,她的背后还有一个家族,她还是家族的大小姐。

被这样的一个人热恋,不管方源再如何低调,也会成为众所瞩目之人。

别忘了蛮家的二公子蛮多,极其迷恋葛谣的美色。

方源若随葛谣回去,必定成为葛家、蛮家的众矢之的。何必为了一个累赘,去吸引这么多的仇恨呢?

方源并不惧怕仇恨,但他来到北原,绝非是来郊游的。他的时间很紧,简直要争分夺秒。荡魂山再一步步的死亡,春秋蝉在一步步的恢复,而他的修为还只是四转巅峰而已。

他必须通向成功,不能有失败。一旦失败,他就将万劫不复,尸骨无存。

他走的这条路,注定了孤独,只有两种结果,不是成功,就是毁灭!

所以,当两人接近腐毒草原外围时,方源便趁着人迹罕至,杀人方便的时候,痛下杀手!

杀死了葛谣之后,方源就令毒须狼将其分食个干净。魂魄当然也不会放过,用葬魂蟾吞了,如今已经送到福地中,被荡魂山彻底荡碎。

篝火旁的痕迹,方源也仔细地消除,没有留下任何的隐患。

总之,葛谣彻彻底底地在这个世界上消失了。唯一的痕迹,恐怕就是毒须狼群排泄出来的粪便吧。

呵呵呵。

所谓的美色,到头来,也不过是黄土一捧而已。

尘归尘,土归土。

美丽的少女,在天地中,不过和鲜花一样。或是被路边的脚步践踏,或是时间到了枯萎老去,成为丑陋的肥料滋润土地。

“没有永生,再美好的东西也是镜花水月啊。存在的价值,不过是刹那的芳华罢了。”方源越是经历多,越是认识到天地的残酷。没有永生,哪怕再有价值的东西,也会没有价值。

“所谓的流芳百世、遗臭万年,也不过是那些孙子的肤浅念想。所谓的精神不朽,不过后辈人拿来佐证自己的工具。人难道真的只能彼此印证彼此吗?地球上也就罢了。到了这个世界,旦有飘渺至极的可能,我也要追求啊!”

“哪怕死在追求的路上,死得比葛谣难看千万倍,我也丝毫不悔啊……”

方源早已心生死志。

但唯有将毕生的精力,都奉献给追求,才会令他死时,不感到一丁点后悔。

呵。

谁能明白方源这个穿越者加重生者的心?

他走的路,注定是无边的黑暗,注定是无比的孤独。

他朝圣的方向,只是心中的光明――永生――一丝微小到不存在的可能。

这个世界上,没有人明白他。

而他。

也不需要别人明白。

\end{this_body}

