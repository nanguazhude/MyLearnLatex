\newsection{今夜无眠}    %第一百九十八节:今夜无眠

\begin{this_body}

%1
“不,不要丢下我……”

%2
“救我,救我,恩公!”

%3
声音缠绕在太白云生的耳畔,他猛地睁开双眼,从床上一下子坐起来。

%4
呼呼……

%5
他喘着粗气,满身粘粘的汗液,很不舒服。

%6
又一次噩梦!

%7
灰暗的灯光下,这位五转巅峰的强者,显现出和年龄相符的龙钟老态。

%8
寂静的房屋里,太白云生的喘息声越来越小,紧皱的眉头也逐渐松弛下来。

%9
盘坐在床上,他陷入死一般的沉默,似乎在发呆,目光中透出一股极深的疲惫之感。

%10
自从闯关失败,他为了一时私心,置朱宰、高扬于不顾,令二者牺牲之后,太白云生就陷入到深深的自责和愧疚当中。

%11
几乎每天晚上,他都会做内容相似的噩梦。

%12
噩梦中,血液四溅,腥臭冲天,各种各样的丑陋血兽张牙舞爪,而他深陷重重包围当中,重新面对朱宰、高扬的哀求。

%13
他们叫他恩公,请他出手援救。

%14
但每一次,哪怕太白云生的空窍中有满满的真元,都会动弹不得,坐视族朱宰、高扬二人被血兽们层层包围,然后吞食血肉,最终啃噬得只剩下惨白骨架。

%15
整个过程中,朱宰、高扬求援不断,有时哀求,有时拱卫,有时怒骂,有时嘲讽。

%16
最后,当他们变成雪白的骨架,躺在猩红的血泊当中,只剩下骷髅脑袋的他们。仍旧在说——

%17
“我相信太白大人,他是这么的仁厚慈爱,他绝对不会放弃我们的!”

%18
“嗯。我也相信他!就算是死了,也相信……”

%19
太白云生痛苦、悲楚、无奈、悔恨!

%20
尤其到最后关头,在梦中的他都会无力地跪在地上,任由血花沾满他的雪须白发,痛哭流涕。

%21
他感觉自己再不是太白云生。

%22
这样的经历,不得不让他重新审视自己。

%23
但每一次审视,都是一次严重的否定。

%24
有时候照镜子时。他甚至都感觉自己像看一个陌生人!

%25
今夜无眠之人,远远不止太白云生一个。

%26
夜已经深了。

%27
王庭福地的夜里,银辉灿烂。挥洒这方天地。

%28
黑楼兰倚窗眺望,圣宫顶端八十八角真阳楼仍旧在凝聚新的楼层,绚烂的烟霞摇曳生姿,美轮美奂。

%29
“可恶!楼主令竟然弄丢了。这究竟是怎么一回事?”回想起白日闯关时的情景。黑楼兰咬牙切齿,捏紧双拳,一对眼眸中凶光四射,直欲择人而噬。

%30
历代,都没有新任楼主丢失楼主令的情况,偏偏在黑楼兰的身上发生了。

%31
“不管了!没有了楼主令,用不了杀招灰融,又能怎样?任何的困难。都不能阻止我取得力道仙蛊!娘亲,你泉下有知。就好好地看着我为你复仇吧!”

%32
方源漫步在圣宫庭院里。

%33
他面色并不好看,一脸沉凝之色。

%34
花庭中,玉泉石桥,繁花如春,如此优美的景色,他却无心欣赏。

%35
刚刚从白玉大殿中退出来,方源正为收服王庭地灵的事情苦恼。

%36
脑海中,墨瑶意志娇笑连连:“呵呵呵,想不到原来王庭福地认主的条件,居然是这样的。小子,依你这样的阴沉性格,想要寻找到一位真心爱的女子,而她也真心爱你的,可不容易啊。”

%37
不久前,方源面对王庭地灵霜玉孔雀,被当场告知成为王庭之主的条件——真爱!

%38
要让霜玉孔雀心甘情愿地认主,俯首帖耳,需要的不是一个人,而是一对真心相爱的男女蛊师情侣。

%39
巨阳仙尊要达到这个条件,十分容易。

%40
但对方源来讲,却困难重重。

%41
别的不说,单说让方源真心去爱某个女人,就几乎不可能了。

%42
“这要换做前世,我能够达到。但今生么……”方源踱步,走到石桥上停下来,他手抚桥栏,凝望着桥下的碧玉小湖,冷笑连连。

%43
湖面如镜,在夜辉的映照下,灿烂若银。

%44
夜闯八十八角真阳楼的行动,惊心动魄,又戛然而止。

%45
此刻,他身处静谧祥和的圣宫美景之下,和之前危机四伏形成巨大落差,令其不禁产生一种如梦似幻之感。

%46
“人生如梦,朝露夕花,宛若泡影……”方源口中喃喃,一时失神。

%47
微风乍起,吹起小湖一片涟漪。

%48
涟漪中,银粼泛起,在方源恍惚的目光中,浮现出一个女子的身影。

%49
那是他深埋于记忆里的女子。

%50
在方源颠破流离的前世,他在最失意的时候遇到她,世间的际遇是那样美妙,命运的捉弄又是那样的残酷无情。

%51
“如果说,真心相爱……我怎么会想起了她?”方源抚摸栏杆的手指,不禁微微用力。

%52
他陡然蹙起眉头,目光也随之变得冷冽如冰,女子的身影悄然消散,眼前仍旧是那一片银光粼粼的湖面。

%53
方源陷入沉思。

%54
现在的局面,不容乐观!

%55
霜玉孔雀,乃是福地原主的执念所化,傲气十足,但认可真情真爱。

%56
方源要达到它的要求,几乎是不可能的事情。

%57
如果收服不了地灵,那么方源图谋八十八角真阳楼的大计,就只能付诸东流,戛然而止了。

%58
“难怪前世,中洲的蛊仙们选择毁灭王庭福地,破坏八十八角真阳楼。蛊仙秘密深沉,个性十足甚至极端,比凡人更难以真心相爱。”

%59
“用了和稀泥之后,尽管融化缓慢,但地灵身上的封印力量正不断减弱。地灵被封印了十多万年,心中愤怒仇恨,无比渴望自由,一定会挣脱封印。巨阳意志虽然现在仍在沉睡,但只要封印被溶解到一定程度,或是地灵的反抗挣扎,最终都会将其惊醒!”

%60
这可是巨阳仙尊的意志!

%61
仙尊一怒,血流漂橹,万物齐哀。但地灵占据主场优势,只要福地不灭,便力量生生不息。

%62
两者互掐,比神仙打架还更要可怕。到那时,遭殃的就是方源这些凡人。

%63
尤其是方源作为解放地灵的罪魁祸首,若被巨阳意志盯住,必定凶多吉少,下场凄凉!

%64
方源本来的计划,是这个样子的。

%65
首先第一步,来到北原,混入王庭福地。

%66
这一步,他达成了。巧妙地利用前世记忆,提前斩杀了只剩下一口气的狼王常山阴,随后自己冒名顶替,成为黑家高层。最终隐藏在盟军当中,进入王庭福地。

%67
第二步,进入八十八角真阳楼,取得足够多的利益。再利用前世中洲蛊仙影像,收服地灵,在不惊醒仙尊意志的前提下,他就成为王庭福地新的主人。

%68
第三步,在成为王庭之主后,并不急着攻略八十八角真阳楼,而是隐藏幕后,顺其发展。

%69
八十八角真阳楼中,有仙尊意志,不好对付。王庭福地,更是北原所有蛊师心中向往的圣地,不容玷污,不容染指。

%70
一旦被发现情况,方源就会遭受整个北原的全力追杀!

%71
这样的恐怖力量,就算是方源逃到其他地方,也无济于事。更不会有任何一个超级势力,顶着北原全部势力,来包庇他。

%72
除非是等到五域大战,自顾不暇,五败剧伤时,这个情况暴露出去,方源才有活命的可能。

%73
成为地灵之主后,方源就会命令地灵,仍旧待在白玉大殿中,积极配合八十八角真阳楼的运转。

%74
这样一来,一切就顺当了。

%75
第一,他可以利用王庭福地的力量,来对付蛊仙太白云生。这样一来,把握激增不说。不管战果如何,方源已立于不败之地。就算抢夺仙蛊江山如故失败,有了地灵的保护,方源不虞性命之忧。

%76
第二,方源可以利用定仙游,自由往返狐仙福地、王庭福地了。狐仙福地,被超级势力仙鹤门虎视眈眈,每隔一段时间又有地灾,立于风口浪尖,并非理想中的修行之地。

%77
方源成为王庭福地之主,便可以狐仙福地作为前哨阵地,王庭福地作为主基地。一面修行,一面和仙鹤门周旋,争取最大利益。若事有不济,便舍弃狐仙福地,退进王庭福地。

%78
王庭福地受着仙尊布置保护,智道蛊仙推演困难。再加上方源日后修补漏洞,暗中主持,比狐仙福地要安全多了!

%79
第三,方源坐拥宝地,每一次真阳楼开启,都能混入其中,收服仙蛊等等珍贵的海量修行资源,再也不用以身犯险。等到良机到来,重走前世机缘,收服血道蛊虫,晋升血道蛊仙。稳稳当当,安安全全。

%80
但计划是美好的,现实是残缺的。

%81
方源尽了最大的努力,卡在了最关键的第二步。

%82
无法成为地灵之主,他就借不了力,区区凡人,难以成事。

%83
现在情势正在变糟,巨阳意志必定会被惊醒。方源要想逃脱,倒是可以利用洞底蛊,勾连狐仙福地逃生。但几只仙蛊,他却带不走!

%84
甚至,就算是他搞到江山如故仙蛊,如何及时安全地返回狐仙福地呢?

%85
没有了地灵的帮助,身为凡人的方源,根本做不到这点。

%86
“为今之计,只有先想方设法,说服地灵。如果我站在它的阵营,和它一齐对付巨阳意志,说不得能利用到它的一些力量。唉……我实在不想,用那最后的手段呐。”

%87
方源沉思片刻,暂时也想不出什么好的办法。

\end{this_body}

