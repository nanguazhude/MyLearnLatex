\newsection{贪生怕死因由何?}    %第一百五十七节:贪生怕死因由何?

\begin{this_body}

“说起来,这中洲的蛊仙亦是厉害,居然能看破八十八角真阳楼的漏洞,并且利用到这种层次。很显然,他(她)同样是炼道大师,不,能达到这一步,至少是炼道宗师!”

“居然将一只仙蛊,留做传承之物。这是真正的仙藏!他(她)利用赝品灰白石板当做线索,更是匠心独运,想法奇妙。不知道是想选择什么样的继承者……”

最关键的密语解决了,但更多的问题冒了出来。

方源摇摇头,将脑海中杂乱的思绪排清。

“不管怎么说,这份传承关乎仙蛊,我都要尽力一试。接下来就是准备相关的蛊虫,这至少得需要大半个月的时间……”

土中蕴光,芒高万丈,百里天游,咏梅雪香——这四句话虽然简单,但方源若非有炼道大师的底子,还破译不出。

单就方源目前理解的成果而言,要借助八十八角真阳楼的伟力回流,炼成神秘仙蛊,需要的蛊虫多达两百余只,其中有四转五转蛊虫就有二十八只。

要知道,这还只是一次性成功的数量。

方源要准备炼蛊,至少得有三倍的准备,以防止炼蛊过程中的失误,而导致炼蛊失败。但他失败的时候,就需要备用蛊虫了。

十六天后。

大殿中,黑楼兰尽显“黑暴君”的风范,狰狞咆哮,恣意地发泄着心中的愤怒。

被黑楼兰训得抬不起头来,甚至惨遭拳打脚踢的众家老们,一个个噤若寒蝉。

黑楼兰自从进入了王庭福地之后,脾气就越来越暴躁。八十八角真阳楼开启以来,他更是变本加厉。脾气仿佛火药桶般,动辄对属下训斥暴打。到如今,已经有三个黑家家老被打成重伤,仍旧躺在病榻之上。

“族长大人,非是我等懈怠。而是这七十八关实在太难。守关的金白虎虚像,实力强悍,已有荒兽的三成。我等凡躯,拼尽全力,也只能骚扰,没有强大的攻击手段。而且。一旦金白虎发动攻势,我方蛊师万难抵挡啊。”

作为家老之首的黑沛,待黑楼兰发泄了一通之后,小心翼翼地觐言道。

黑楼兰斜了他一眼,骂道:“你说的都是屁话!金白虎虚像攻势虽强,但只要我们万众一心。不计牺牲,肯定能在时限前,了结掉它!你们一个个都畏难不前,我黑家的勇武之名,都被你们丢光了!”

家老们被训得缩头垂眉,都不敢说话。

黑楼兰的话,其实不无道理。

对付金白虎虚像。已经不是第一次了。

若是真的不计牺牲,主动有人充当炮灰,即便丧生在金白虎虚像的爪下,也能为他人争取时间。

黑家众人的攻势,虽然绵绵无力,但只要时间充足,积少成多,必然能够群蚁噬象,将金白虎虚像打垮。

但事实是,金白虎虚像一旦发动攻击。众人都贪生怕死,畏缩不前,导致黑楼兰对八十八角真阳楼的攻略,一直卡在此关,无法再进一步。

大殿中回荡着黑楼兰的咆哮之声。

没有人敢在这个时候。去忤逆这个发怒起来就六亲不认的黑暴君。

黑楼兰发泄了一通后,阴沉着脸,坐到主位上。

他心中郁愤,尤其是看着这些默不作声的家老们,更是气不打一出来。

除了愤怒之外,他也有无奈。

王庭之争中,这些黑家的家老们各个奋勇争先,勇猛无畏。但是到了此处,却是畏手畏脚,胆气去哪里了?

其实,黑楼兰心中也理解。

王庭之争,赏罚公明,人人争先奋战,都是为了名利,为了强大,为了生存。

到了八十八角真阳楼,过关的奖赏,都归于族长之手。众人闯关的积极性就很低了。

最关键的一个原因是,王庭之争已经得胜,再无生存的危机,哪怕是圣宫之外的传承,也有许多。只要安然度过这段时间,出了王庭福地之后,必将有更加光明的未来。

将身家性命舍弃,充当炮灰,去成全别人,傻子才干这种事情呢!

黑家的家老们,各个都是人精。

保住性命是首要,除此之外,就算被黑楼兰骂得狗血喷头,又能怎样?就算被黑楼兰暴打,躺在病榻上,和死亡一比,又算得了什么呢?

黑楼兰对众家老的心思心知肚明。

“纵然我是五转强者,也操纵不了人心啊。人心一散,再强大的部族也不好带。也罢……”

黑楼兰心中叹息一声,开口道:“既是如此,那我就只好开放八十八角真阳楼,集众人之力攻关。”

请外援,需要来客令。

但现在的王庭福地中,除了黑家之外,还有其他大量的黄金部族族人。譬如耶律家、马家等等。

可以想见,一旦黑楼兰开放八十八角真阳楼给他们,那么这些人必定趋之若鹜。这样一来,就可以引做炮灰了。

家老们听了黑楼兰这话,相互之间用目光隐晦交流。此法能令他们退居二线,但他们却都有些不大情愿。

黑沛大家老出列道:“族长大人,此计虽妙,但却不可不防啊。这些人虽然身上流淌着先祖的血脉,但却并非我黑家族长。一旦通关,获得了好处,恐怕就吐不出来了。”

“是啊,大人。”黑旗胜家老也附和道,“想我黑家殚精极虑,励精图治,千辛万苦才在此届王庭之争夺得了魁首。这八十八角真阳楼都是我们的,何必要去让外人分羹呢?”

“此举虽有先例,但纵览历史,历来都是那些弱小部族侥幸得胜,本身没有能力攻关,只能发动其他黄金部族。而我黑家兵强马壮。人才济济,怎么可能借他人之手呢?”

“哼!”黑楼兰微挑眉头,“既是兵强马壮,为何连一头金白虎虚像都挑不过?你们这群狗屁东西,一个个惜身惜命。想要外人为自己拼命,却又怕外人得了好处。这个世间,哪有这么好的事情?”

和这些家老不同,黑楼兰心中早已焦急不耐。

他是大力真武体,必须得到一只力道仙蛊,方能晋升蛊仙。

也唯有晋升成仙。才能解决性命之忧。

但八十八角真阳楼中有没有力道仙蛊?力道仙蛊到底在哪一层?这都是未知数。

是以,他一心想打破惯例,促成黄金部族合力攻关的局面。他每打通一层,手中的楼主令就能获得晋升,才能更方便地继续攻略。

虽然卡在了此关,屡战屡败。但对黑楼兰来讲,却是一个良机。

他趁机发难,再次当堂咆哮起来。

愤怒的吼叫,回荡在大殿当中,震得人双耳都有嗡鸣之感。

碍于黑楼兰的威势和凶名,家老们只得妥协。

黑沛大家老忧心忡忡地道:“开放八十八角真阳楼,等若倾泻洪水。一旦势大,势必损失惨重。老臣建议,要加以遏制。狼王常山阴的教训,可是近在眼前啊。”

这番话,立即在众家老中引发了强烈的共鸣。

有人语气酸涩:“是啊,狼王太无赖了,拿了好处,一人独吞。到现在都在闭关,恐怕在暗地里偷着乐呢!”

有人不屑地嗤笑:“八十八角真阳楼都是我黑家的,让他来攻关。是看得起他。结果他却这样报答我们,哼,什么狗屁北原英雄,以我看实则是个忘恩负义之徒啊!”

有人目光阴冷:“依我看,这次开放八十八角真阳楼。就不要招呼常山阴了。给他一个教训!”

黑楼兰冷哼一声,对于方源的表现,他当然也有大量的不满。若是换做旁人,他早就动手了。

只是常山阴非同寻常,王庭决战的战斗身姿,至今还深深地印刻在他的心中。

说不忌惮,那是骗人的鬼话。

但明显排斥方源,也是不妥。不仅失了气量,万一惹恼了狼王,就算常山阴不出手,他还有天青狼群呢。

“黑沛大家老,既然你提出了这个建议,那么就说说看吧。”黑楼兰道。

黑沛微微一笑,侃侃而谈终道:“从明日起,不妨就开放了八十八角真阳楼。但进楼之前,不管是谁,都得缴纳费用。每天只有八百位进楼名额,且缴纳的进楼费用,依照排位顺序递增。同时,须得动用毒誓蛊,所得奖赏中的五成,都要归于我黑家所有。”

顿了一顿,他又道:“至于外人,想要进楼,就得高价购买我们的来客令!”

这番话,顿时让在场的家老们眼中发亮,纷纷赞好。

黑楼兰目光扫视一圈,上半身往后倚靠,缓缓闭上了双眼:“也好,此事就给你去办了,黑沛。”

黑沛大喜:“族长大人英明神武,谢族长大人赏识。”

开放八十八角真阳楼的消息一出,立即就在圣宫中引起了轩然大波。

无数人涌到报名的地点,一边大骂着黑家族人的黑心肠,居然将入楼费用抬得这么昂贵,另一边则纷纷慷慨解囊,甚至为了争夺一个入楼的名额,相互之间大打出手。

方源冷眼旁观,心中暗喜。

现在对他而言,继承地丘传承,才是当务之急。

就算是黑楼兰主动邀他出手,他还不大乐意呢。现在旁人都被八十八角真阳楼吸引了注意力,正是着手传承的大好时机!

(ps:这些天更新不及时,让诸位读者朋友们受了累,万分抱歉。此书本月一般晚20点更新,这点没有就是无更。就算深夜赶稿有成,也只会在次日晚20点更。这里说一下这个情况,对不住等更的诸君!)(未完待续。。。)

\end{this_body}

