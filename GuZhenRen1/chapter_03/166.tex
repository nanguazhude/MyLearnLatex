\newsection{死不足惜}    %第一百六十六节:死不足惜

\begin{this_body}

轰隆隆……

大地震动,如潮水般的狼群出现在天际,绵绵不绝,宛若浩荡江水,奔腾而去。

万狼奇奔,狼群混杂。有风狼,奔走如飞。有水狼,洁白如雪。有夜狼,漆黑矫健。又有龟背狼,敦实稳重,朱炎狼赤焰蒸腾。

百狼王、千狼王、万狼王领袖群伦,更兼有异兽狼,行于行伍,分外显眼。

诸如血森狼,背负骨林,高大如山;又有鱼翅狼,批甲如象,水6皆能;还有狂狼,银灰三眼,战痴斗狂。

王庭福地,淡金苍穹中,还有天青狼群,奔腾于空,嗷鸣声声,神骏非凡。

方源稳稳地坐在一头天青万狼王的背上,狂风也吹拂不动他坚毅如铁的面庞。

一对双眸,黑暗深幽,此刻俯瞰脚下,心思难测。

虽然进入王庭福地之初,狼群被打散,流落福地各处。但方源早就布任务,从进入福地就坚持不懈,除去召回旧有狼群之外,又新揽无数野狼。如今脚下奔腾的狼群,似江如海,规模多达五十万之众!

到了他这一步,已经是货真价实的凡尘巅峰。

王庭之争起初,他和江暴牙、马尊等人被称之为五大兽王。王庭之争结束之后,不谈天下公认,但已经是北原公推的当代第一奴道大师。

除去奴道造诣之外,更兼有力道惊人手段,飞行大师实力,叫人惊诧敬畏。

狼王常山阴谁人不敬,谁人不畏?

纵观北原上下,可堪其敌手者,能有几人乎?

不过此刻,在方源的脑海中,却是有道声音在“打击”他。

“呵呵呵,小少年,你居然想奴力合流,心志真是挺大。但姐姐我还是劝你趁早打消了这层心思罢。”

墨瑶意志接着笑道:“奴力二道,泾渭分明。奴力合流,乃是千古难题。你兼修已经大大不易,不如趁早改换,知难而退,干嘛尽做吃力不讨好的事情呢。唉,姐姐是个过来人。你要知道,道路万千,流派无数。蛊师之强,不在于走多少条道路,而在于谁走得更远。”

墨瑶好言劝解,但方源冷哼一声,冷傲否决:“事在人为,有什么可怕的?奴力合流千古难题,那是因为之前没有出现我方源这样的人物。我迟早要威凌天下,名垂千古,媲美星宿、红莲、乐土之流。”

星宿、红莲、乐土,皆九转蛊尊也。人族历史,漫漫长河,无穷岁月摩挲,到如今也总共不过十位。

方源还是个凡人,却妄想成为这样的人物,宛若蚂蚁想成大象。

这样的野心和口气,令墨瑶意志一时听得,也不免瞠目结舌。

……

“吼!”

“杀,将这些地魁小兽斩尽杀绝!”

“治疗蛊师何在?有人受伤,快来接应!”

兽群嘶吼,蛊师喊叫,一场小范围的激战已经到最关键的时刻。

不大的战场上,土坑遍布,残肢血溅。五六位蛊师,围绕着地魁兽王,展开鏖战。而周围,倒着大量的地魁兽的尸体。

这些地魁兽,人身蛇尾,面若蝙蝠,朝天鼻子,双耳招风,浑身漆黑,生有肉甲。胸膛前后左右,长有五六根肉鞭,短的有一丈,长的竟达两丈余。

肉鞭如手,灵活似蛇,抽甩间势大力猛,可攻可守。

地魁兽居于地底,但并非地下深处,而是往往在地下浅层筑巢。

它们价值很高,不管是眼珠子、毛皮、蛇尾都是良好稀少的炼蛊佳材。更稀罕的,乃是它们身上的长短肉鞭,越长越好,两丈以上的肉鞭,往往千金难求,有价无市。

地魁兽如今在北原,踪迹已经渐渐稀少。但在王庭福地,却规模众多。譬如这片地域,方圆万里,都是地魁兽群。

财帛动人心,又因为此地靠着圣宫不远,因此很多蛊师常常结伴同行,到此狩猎。

这支激战中的小队,正是其中一支。

队中的蛊师们,大部分人都狩猎地魁兽群三四次,经验较为丰富。

不过这次,他们却是遇到了麻烦。虽然引来的地魁兽群,只有一百余,规模在他们的控制之下。但没有料到,这只兽王,却不是寻常的百兽王,而是一头老迈的千兽王。这头兽王虽然不能战,老弱多病,但身上寄生的野蛊,却是货真价实的三转级数。

凡人蛊师中,一转乃是学徒,刚刚起步,战力孱弱,最为常见。二转,则是骨干、基石,十分普遍。

三转称之为长老、家老,乃是势力中坚,数量急剧减少。四转乃是势力脑,万人领袖。五转称为凡俗巅峰,更加稀少。

地魁千兽王本身不足为惧,但身怀数只三转野蛊,这样的战力,就出这只队伍能力之外。

若非交战之初,队伍领蒋冻性情刚毅,经验丰富,果断下令,率领一些好手,插入兽群,将地魁兽王死死拖住,再由其他人屠戮地魁小兽。否则,蛊师队伍早就溃败了。

“诸位,再加把劲!这头老兽已经快不行了,坚持就能得胜!”蒋冻腾挪间,大声叫喊,振奋士气。

众人应答参差不齐,勉强振奋精神。

从开战至今,已持续了大半个时辰。蛊师们真元消耗将尽,但幸好北原风气,这些蛊师大多都兼修了力道。

蛊师们奋力作战,多以肉搏为主,非到万不得已,才会去催用真元,救己救人。

吼!

就在这时,老兽王忽然蛇尾狂甩,打得空气爆响。蛇尾狠狠地击中一位蛊师,当场将其打飞。落到地上时,这位蛊师已经胸膛骨骼尽碎,死得不能再死了。

人有智慧,野兽也有它的狡诈。

老兽王体残虚弱不假,但烂船还有三磅钉。此时陡然暴起,立杀一人。

众蛊师都是一呆,士气陡降。

“糟糕,本来就是僵持,现在我方减员一人,我空窍中的真元不足两成,如何是好?”蒋冻眼珠子一转,立即想到了逃跑。

这支猎队不过是他临时组建,临阵而逃,虽然有害名誉。但和生命相比,名誉算得了什么?

北原人好杀伐,性勇悍不假,但却并不是傻子。

“以前艰难困顿,只得拼杀,才有生机。现在我已经入了王庭福地,遍地资源,正是积累雄起之时,怎么能将大好性命,牺牲在这里呢?”

“我家里有老有小,这些天虽然狩猎成果不错,但所得元石只能支撑我的修行。家里那小子,再过半载,就要踏上修行之路了……所以,对不住了,诸位!”

蒋冻目光目光猛地闪烁一阵,忽然后撤,将剩余真元,一概灌输到移动蛊虫。

衣袂划破空气,嗖的一声轻响,他绝尘而去。

剩下的蛊师们又愣了一下,领都临阵脱逃了,还打什么?

顿时众人飞奔逃散,士气低落谷底。

地魁兽王怒吼一声,紧追不舍。

“我草!”蒋冻闻声回头,忍不住爆了一句粗口,差点魂飞天外。

这头地魁老兽王什么人不追,单追杀他而来。想来是蒋冻手段狠辣,攻势猛烈,遂记恨于心。

“糟糕,这样下去,我命休矣!”

两者一逃一追,时间流逝,真元急剧消耗,蒋冻渐渐陷入绝望当中。

“蒋冻大人快跑!”

就在这是,一道声音远远传来。

蒋冻听了,转头望去,只见左前方立着一位年轻人,他识得,叫做马鸿运,参入队伍不久,只有一转修为,属于后勤蛊师,战力微薄。作战开始,他就被遣派出去,留作侦察棋子。

“好个傻小子!”蒋冻大喜过望,立即转变方向,向马鸿运奔来。

马鸿运眼睛瞪大,他见地魁老兽王渐渐追进,忍不住想要提醒蒋冻。但没有料到,蒋冻居然引着地魁老兽王,直接向他这边逃窜。

马鸿运撒腿就跑,但蒋冻度飞快,几个呼吸之后,就追赶到他的身边。

蒋冻哈哈大笑:“小子,你今天救了本大人一命,也算死得其所了。”

话音刚落,催动蛊虫,将马鸿运击昏,随后伸出大手,拎着衣领,将其向后用力一抛。

但地魁老兽王,放着送上嘴边的肉食不要,竟然仍旧紧追不舍。

蒋冻大笑声戛然而止,心情好似从天堂跌落地狱。

正当地魁老兽王追近之际,忽然隆隆之音传入耳中,广袤大地都开始微微颤动起来。

旋即从天边涌起一道灰线,渐渐的线条浓重而又清晰起来。

狼!

怎么这么多的狼?!

狼潮从天边奔涌而来,气势浩荡,宛若滔滔洪水席卷天下。

千兽王顿足,呆了片刻,浑身颤抖,转身就跑。

“朱炎狼、龟背狼……如此狼群,对了,之前有过公告,狼王出猎要狩猎此地!啊哈哈,天不绝我,我有救了!”

蒋冻呆了一呆,旋即兴奋大叫。他瘫坐在地上,浑身激动得颤抖,绝境逢生,双眼涌动出喜悦的泪花。

但下一刻,狼群奔腾而来,不减其,宛若浪潮,眨眼睛就将蒋冻撕碎吞没。

金空之中,青狼之上,墨瑶意志在方源的脑海中出一声悲悯的叹息:“方源,你放任狼群,所到之处,不管敌友,只要活物都屠戮一空。杀性太重,就算你不怕有干天和,难道就不担心其他蛊师的想法吗?”

“哼,我此番出行,早已提前出公告。这些人利欲熏心,蝼蚁一般,挡我兵锋,死不足惜。”方源淡然回应。

\end{this_body}

