\newsection{力道仙蛊出现}    %第一百八十八节:力道仙蛊出现

\begin{this_body}

。

黑楼兰皱着眉头,走出太白云生的房间。

“强者能胜不骄败不馁,想不到大名鼎鼎的太白云生,也不过如此。”他对太白云生的精神状态,很不满意。

“太白云生可是治疗第一人,如果他状态不佳,我闯关的难度就要激增不少了。”想到这里,黑楼兰仰头,向上方望去。

圣宫顶端,霞光万道,恢弘喷吐。

八十八角真阳楼已然凝聚了六十七层,目前正在初步凝聚第六十八层。

黑楼兰痴迷地望着,双眸中闪过势在必得的冷光。

就在这六十八层中,存在着一只力道仙蛊!

黑楼兰拥有一角楼主令,这只力道仙蛊刚刚被八十八角真阳楼摄取进来,他就知晓了。

这是一只六转力道仙蛊,名为飞熊之力蛊。

飞熊乃是荒兽,战力媲美蛊仙。

飞熊之力蛊,可以令蛊仙攻击之时,有一定几率爆发出飞熊虚像,打出飞熊神力。

这正是黑楼兰需要的仙蛊,有了它,他就能晋升成力道蛊仙了!

“这只仙蛊,我必须得到手。有了它,我的复仇,我的大计才有希望!”黑楼兰不禁捏紧双拳,但他很快又放松下来。

他的心腹黑书,一阵小跑,跪拜在他的脚下。

黑楼兰的脸色迅速恢复如初。

就算是面对心腹亲信,他也不流露一丝真实情感。

每天这个时候,黑书都会来汇报圣宫中各大强者的动态:“常山阴大人。仍旧在闯荡第七层中第九十道关卡,仍旧毫无进展。耶律桑大人,则在第四十九层中。攻破了第八十一关,身边蛊师损失三成……”

黑楼兰彻底开放八十八角真阳楼,是历届都罕见的慷慨之举,引动众人闯荡真阳楼的疯狂热潮。

每一枚来客令,就算都炒成天价,仍旧有无数人趋之若鹜。

人为财死鸟为食亡,利欲熏心使得许多人因此阵亡。

算上太白云生这次。已经陆续有常飚、潘平、浩激流、高扬、朱宰五位五转强者陨落。这对黑楼兰来说,是巨大的损失。

可供他调用的实力越小,他取得飞熊之力蛊的难度就越大。

当然他彻底开放八十八角真阳楼。也有一定的好处。参与其中的蛊师,只要生还下来,大多都有着飞速的提升。

也有不少的蛊师,晋升成功。涌现出许多四转、五转的新面孔。

八十八角真阳楼。本来就是巨阳仙尊,为了子孙后代谋福利而布置的。

“这第六十八层,就是我族攻略的重点,甚至要远胜之前的三十九层。这是我族某位太上家老大人,特意交代下来的任务。因此现在圣宫中的每一份力量,都需要珍惜。黑书,你将这个消息公布出去,从即日起。八十八角真阳楼关闭,所有人都得听我调度。第六十八层攻破。真阳楼便重新开放。”黑楼兰吩咐道,语气中充满了不容质疑的决断。

听到这竟是部族某位太上家老的任务,黑书顿时浑身一颤,神情变得极为肃穆。

“快去办吧。”黑楼兰挥挥手。

黑书忙领命而退。

每一届王庭之争,都是蛊仙们在下的棋。就算再强大的蛊师,也不过是棋盘上的棋子。

蛊仙为了谋夺某类仙蛊,资助某个部族参加王庭之争,这是最平常的情况。

黑楼兰确信,自己关闭八十八角真阳楼虽然会引起众人不满,但是用黑家蛊仙的名头压下去,没有人敢提出什么意义。

“太白云生的伤势很重,但只需要休息半个月左右,就能令他参战。毕竟他是治疗蛊师,处于战场后方。他需要寿蛊,要调动他,从这个方面出发即可。”

“常山阴和第七层较上劲了。呵呵,他的日子不好过啊。因为倏忽,导致自己的亲儿子常极右都折损在这道关卡了。为了挽回威信,他必须要这关卡打通。不过他之前向我索要飞熊虚像仙蛊,又赊账,欠了我一大笔资源。我调动他,还是可以的。”

“还有耶律桑,他身上可是有着炎道辅助仙蛊,可为强助。我当用重立诱之……当然最关键的,还是要调动出雪松子、黑城、黑柏三位蛊仙的力量。呵呵呵,该向他们送信了。”

……

黑柏看了黑楼兰的来信之后,分外激动:“等了这么久,终于等到木鸡蛊的出现了,想不到是在六十八层里头。”

他丝毫没有怀疑这信上的内容。

毕竟,黑楼兰可是黑城的儿子,是黑家的当代族长。如果他的忠诚都要怀疑,那全天下还有什么人值得信任呢?

一旁,黑城朗笑道:“贤弟,你是守得月明见花开,功夫不负有心人呐。信中可有提到力道仙蛊的事情呢?”

黑柏脸上一红:“惭愧,我只顾着木鸡蛊,却忘了贤侄的情况。贤侄是大力真武体,必须有力道仙蛊,方可晋升蛊仙。信中贤侄也说了此事,可惜还未等到力道仙蛊出现。”

“还没有力道仙蛊?”黑城皱起眉头。

之前,黑柏担忧木鸡蛊,被黑城安慰。这次,终于轮到黑柏反过来安慰黑城:“兄长勿虑,八十八角真阳楼总共要凝聚八十八层,如今还有二十层没有凝聚呢。”

黑城微微点头,叹息道:“暗渡仙蛊已经快遮掩不住了,这几乎可以算是我儿最后一次机会。唉……谋事在人成事在天,听天由命吧。”

黑柏心中热切,将话题又转到木鸡蛊的身上。他提醒道:“兄长,时间不等人,我们快快准备吧。给黑楼兰提供一些助力!”

“嗯。此事刻不容缓,信中说的是哪些地方?”黑城缓缓从座位上,站起身来。

黑柏也跟着站起:“是在魔血山丘附近。”

“走。”

两位蛊仙十分干脆。立即出发。

不多时,赶到魔血山丘上空。

魔血山丘,乃是北原著名地域,曾经正道中四大部族联手,剿灭了当时,为祸北原数百年的魔道匪帮。

这只魔道匪帮,十分猖狂。皆因背景深厚,有着魔道蛊仙的暗中支持。

激战之后,匪帮被全数剿灭。正道亦损失不轻。魔道蛊师的鲜血,染红了整片山丘,因此后人称之为魔血山丘。

黑楼兰在信中,要求帮助。

黑家两位蛊仙。早就准备多时。收走就走,此行就是为了帮助黑楼兰尽快地打通关卡。

但八十八角真阳楼位于王庭福地,为何黑家两位蛊仙却来到这里呢?

这还要从八十八角真阳楼的构造说起。

前文也阐述过:八十八角真阳楼乃是八转仙蛊屋。

由成千上万的小塔楼组成,排难蛊是主要基石之一。

小塔楼每隔数里,覆盖王庭福地,照应北原各地。平日时,吸纳王庭福地中的野生蛊虫。每当十年之期来临时,这些小塔楼就会陆续沉没。牺牲塔内的野蛊,同时响应北原外界的雪灾。获得奇妙的伟力。

这些伟力对整个北原进行一场巨大的搜刮,将北原珍贵的蛊道资源,都摄取到王庭福地的圣宫顶端,从而凝聚出八十八角真阳楼的一层层。

这一层层叠加起来,最终形成完整的八十八角真阳楼!

当年,巨阳仙尊提出设想,被长毛老祖驳回。长毛老祖乃是炼道大宗师,另辟蹊径,设想出绝妙构思,炼成如今的八十八角真阳楼。

每一次形成八十八角真阳楼时,都是一次重新炼蛊的过程。

利用的不仅是小塔楼中的野蛊,而且还有摄取出来的蛊虫。

譬如第六十八层,八十八角真阳楼先摄取到飞熊之力蛊,将其炼化,利用它本身的力量,形成第一百道关卡。

再摄取其他强大蛊虫,形成第九十九道关卡。随后是第九十八道,第九十七道……

就好像是搭建高楼,必须要先浇筑沉重的地基。

若是蛊虫不全,那么真阳楼就用蛊方、元石、传承秘密来替代。

常人都知道,八十八角真阳楼中关卡越难,奖励越是丰厚。其实反过来讲,正是因为奖励越厚,蛊虫越强,关卡才越是艰难。

可以说,摄取进来的每一只蛊虫,都是八十八角真阳楼的构成部分。

无数届的王庭之争,无数代贤者仙人的摸索,终于得到八十八角真阳楼的这个运转机理。

本来,这当中是毫无破绽的。

然而随着时间流逝,沧海桑田,漏洞就产生了,供北原蛊师钻营利用。

这个漏洞就出在“炼化”上面。

八十八角真阳楼搜刮北原,摄取蛊虫进来,“炼化”它们,再依靠它们的力量,形成关卡。

炼化蛊虫,靠的是什么?

意志!

当初,方源炼化酒虫,是怎么做的?

是用真元作为载体,让蛊师意志不断消磨蛊虫身上的意志,直至蛊虫身上全是蛊师意志,如此蛊虫才彻底成为蛊师的工具。

方源炼蛊时,曾经利用仙蛊春秋蝉不断讨巧。

蛊虫之间,高出二转,就会形成威压。方源利用春秋蝉的气息,逼压低转蛊虫的意志,令其龟缩一旁,从而再用自身真元灌注,其意志长驱直入,一口吞掉龟缩成一点的蛊虫意志。

八十八角真阳楼不是蛊师,它靠什么来炼化其他蛊虫呢?

巨阳仙尊为此,特意留下来他的意志!

巨阳意志!

方源曾经在炼化来客止步碑的时候,就见识过巨阳意志。真正个宏伟如日,雄奇浩荡。

相比较而言,方源炼化来客止步碑后,留下的那股意志宛若蝼蚁一般微小。

八十八角真阳楼炼化蛊虫,就是用的巨阳意志。

仙尊无敌,意志同样无敌,可以轻易消灭任何其他意志。

因此八十八角真阳楼炼化蛊虫,从未遇到任何困难。

但是,有一个问题。

意志是由念头组成的,若是没有灵魂提供载体,不断思考就会不断消耗。

为了抵抗这种惊人的消耗,意志常常选择深眠。

方源进入近水楼台时,里面存放的墨瑶意志,就从深眠中被惊醒。

巨阳意志虽然浩大如日,但巨阳仙尊逝去这么多年,也渐渐难以抵挡岁月的力量,也选择了沉眠!

当巨阳意志沉眠之后,八十八角真阳楼炼化蛊虫的速度就慢了。

这一慢,真阳楼的漏洞就出现了。(未完待续)

\end{this_body}

