\newsection{修复战骨车轮}    %第四十节:修复战骨车轮

\begin{this_body}

方源看了一眼拜帖,帖子上的话十分客气,用的当然是北原文字,行书看似蛮野却不恣意,落款是蛮家族长的名字――蛮图。

这是蛮图亲自书写的拜帖,以示诚挚之意。

内容表达了蛮家族长对常山阴的倾慕之情,又说之前的事情都是小孩子间的误会,常山阴没有杀死蛮家的外姓家老石武,这样的风度更是令蛮家上下钦佩。

所以,将在今晚摆上酒宴,诚挚地邀请方源赏光。同时这礼盒中,是石武家老冒犯了英雄,因此送来赔罪之物。

方源打开一看,笑了笑:“这蛮家倒是有心了。”

礼盒中,是上百只的骨竹蛊。

显然,蛮多回去之后,蛮家详细调查了方源,连他在市集中采购的情况都探听到了。

“山阴老弟,老夫有个不情之请啊。”一旁的老族长开口道。

方源摆手:“老哥的意思我知道,放心吧,在酒席上我会劝说一二,争取为葛蛮两家化干戈为玉帛。”

“啊,那就太感谢山阴老弟了!”老族长十分感动,颤巍巍地站起来,向方源深深一礼。

距离晚宴还有一段时间,方源送走了,就关上房门。

方源盘坐在床榻上,心念一动,从空窍中飞出一道光。

微弱的光辉散去,显露出战骨车轮。

这只五转蛊,体型庞大,几乎顶到房顶。它一出现,原本宽敞的房间,就立即显得狭小无比。

车轮上布满了裂痕,有几道严重的伤痕,几乎要毁掉全部的车轮辐条。还有一道最深的伤口。十分严重,几乎要把整个战骨车轮分裂开来,只差中间的一点白骨连着。看得人触目惊心。

这只五转蛊虫的运气并不好。本来就被常山阴打得残破不堪,它的原主人哈突骨死后,它就沦为野生蛊,食用战场上的尸骸残骨,艰难存活,伤痕则一直没有修补。

后来它又被方源和葛谣合力攻打。春秋蝉是六转蛊,也不能压服五转蛊虫。方源只好把它打得奄奄一息。这才收服了它。

正因为如此,整个战骨车轮已然濒临破碎。落在房间里时,从车轮上还掉落下几片骨头的碎片。

可以说是,惨得不能再惨了。

方源从礼盒中取出一只骨竹蛊,靠着春秋蝉的一缕气息。将其随手炼化。

这蛊虫,宛若一截骨头制成的一截竹竿,惨白惨白,仿佛一根白色的蜡烛。

方源将骨竹蛊拿捏在手中,真元随着心念而动,出了空窍,一路往上。流到舌底处。

鬼火蛊就寄托在他的舌底,已然化为一个蓝色的火焰团。

方源微微鼓起腮帮,轻轻一吐。

呼的一声,他吐出一团幽蓝色的鬼火。

鬼火准确地落在骨竹蛊上。附着在上端,静静地燃烧这。

方源拿捏着骨竹蛊的底部,好像是举着一个蓝火蜡烛。

鬼火不断地燃烧着,散发出阵阵冷意。阴寒刺骨。骨竹蛊的上端在鬼火的烧灼之下,慢慢融化。形成一缕白骨烟气,漂浮而出。

方源将骨竹蛊小心地靠近战骨车轮,白骨烟气像是受到吸引一般,自发地飘向战骨车轮的裂痕处。战骨车轮微微颤抖起来,裂痕开始一点点的修复。

不一会儿,幽蓝色的鬼火越烧越小,方源便又吐出第二团,将火苗增旺。

半盏茶的功夫过后,这只骨竹蛊燃烧殆尽。方源便又从礼盒中取出第二只骨竹蛊,继续用鬼火灼烧,形成白骨烟气。

如此循环,锲而不舍,用了三十多根的骨竹蛊后,方源已将战骨车轮上的那道最深的裂痕,彻底修复。

装满了礼盒的骨竹蛊,一下子用去了四分之一。想要完整修复的修复战骨车轮,单靠这些骨竹蛊还不够得很。

而战骨车轮也远远没有到可以作战的地步。

最深的裂痕修复好了,但是其他的伤痕仍旧遍布车轮表面。

它就像是一个病人,经过方源的抢救,从死亡的悬崖边缘往回拉了一步远。但这个病人仍旧是生命垂危,还需要坚持不懈的修复。

“一口吃不成胖子,战骨车轮的创伤实在太重了。但若非如此,我也不可能轻易地收服它。不过就算现在完全修复,凭我现在的真元也不够催动它的。此事不必急于一时。”

方源从床榻上下来,活动了一下酸麻的手脚,打开窗户,已是黑夜,天上月明星稀。

“时间差不多了。”想到今晚的宴会,方源出了房门。

走出庭院,蛮多和葛家族长父子,早已守候多时。

“蛮多见过常山阴大人!”蛮多看到方源,立即行礼,恭敬有加。

“你们等了多长时间?”方源点点头,随口问道。

蛮多立即答道:“不过区区三个时辰,何足挂齿。站在大人的门前,也是晚辈的荣幸啊。常山阴前辈,家父已经在不远处,备下了丰盛的晚宴,敬请您的大驾光临。葛家父子作为此行陪同,也会同去。”

方源看了一眼这个蛮多,心中微微可惜。

这个小子的确有才智。可惜天意弄人,资质不足,倒像是前世的自己呢。

“好,那就同去。”方源骑上自己的驼狼,和众人一起,出了葛家营地。

由着蛮多指路,连同随从一共十多人,皆骑着驼狼,向着远方奔驰。

凉爽的夜风在耳畔呼啸,茫茫草原仿佛在向身后奔跑。

月光如水,倾泻而下,近看视野良好,远看月华氤氲如烟。地上绿草广袤,翠绿欲滴。山丘舒缓,流向天际。

在这样的月夜,纵狼奔驰,大地无垠,月色正美。自由一股畅快!

众人奔驰了不一会儿,就看到一处山丘上站着一群驼狼。狼背上坐着一群蛊师。一位蛊师手中擎着大旗,旗面随风飘扬,上书一个大大的“蛮”字。

看到方源等人后,这群蛊师立即骑着驼狼,奔驰过来。

蛮多见此,立即笑道:“常山阴大人,前方正是家父,他来迎接您了。”

这是北原上的规矩――如果宴请尊贵的客人。主人都会十里相迎。

两群驼狼,在中间相会。

蛮家族长主动下了狼背,大笑着走过来:“哈哈哈,今夜月亮明亮得仿佛和太阳一般,这是照耀我们北原英雄归来。常大人。我是久仰你的大名了。”

蛮家族长身高九尺,身材魁梧得很,浑身肌肉贲发。他穿着宽大的皮袍,没有袖子,露出两个肩膀,古铜色的臂肉粗壮得,堪比普通男子的大腿。

但是他话音刚落。老天就好像跟他作对一样,给他开了个玩笑。

只见夜空中,一片阴云掩来,遮住月光。令这片草原陷入一片黑暗。

蛮家族长豪气的大笑声低落下来,十分尴尬。

还是蛮多精明,眼珠子一转,立即在一旁朗笑一声:“常山阴前辈。是我们北原鼎鼎大名的英雄。父亲,你则是我们蛮家人心中最崇敬的英雄。今夜便是英雄会!你们看。英雄之气,果然令天地激荡,风起云涌!”

这话巧妙地化解了尴尬,当蛮家族长走到方源的面前,脸色已经恢复了自然。

方源等人也下了狼背。

蛮图先向方源深深一礼,方源以右手抚胸还礼。

然后,蛮图故意地瞪了蛮多一眼,以责备的语气道:“胡说八道!为父怎么可以和常大人相提并论。常大人昔日名扬北原,尊称狼王,一手驭狼术独步天下。更斩杀了五转蛊师哈突骨,消灭一帮马匪,为北原除去大害,值得世人永远称颂。”

“呵呵呵,蛮图族长何必过谦虚?”方源也笑着道,“你是蛮家族长,统御百千蛊师。实实在在的开拓之君,带领蛮家连连得胜,又掌握着红炎谷,乃是一方霸主。我的修为已经落到四转初阶,族长的修为则是巅峰,远远超过在下。我虽有驭狼术,但说到底不过是控制禽兽畜生,蛮图族长却是控人,之间境界天差地别啊。蛮图族长才是名副其实的英雄豪杰!”

蛮图闻言一愣。

狼王早年以孤傲闻名,难以打交道,没想到居然如此健谈,态度如此谦和。

不过转念一想,他就想通了。

这常山阴年少成名,少年心性,自然有些张狂。如今已经是中年,又遭逢大难,心性经受了磨练,沉淀下来也实属正常。

蛮图没有想到方源如此好说话,不过能得到狼王常山阴的如此赞许,也隐隐让他十分开心。

他在心中,更对方源高看一筹。

方源修为虽然落到四转初阶,但他越级斩杀过五转哈突骨,因此蛮图丝毫不敢小觑方源。

当即他笑道:“常大人早在二十多年前,就已经是四转巅峰。现在因为伤势跌落,迟早都会修行回来,甚至更上一层楼。届时,我的这点修为,算得了什么?”

就在这时,阴云散开,月光再次照射下来。

“呵呵呵,二位都是当今北原的英雄豪杰。”葛家族长适时地开口,“更难得的是,虎狼相遇,却没有争斗,而是惺惺相惜。我等在此有幸见证这一盛事,可谓拨开云雾见明月啊。”

这番话,引得众人都笑起来。

“哈哈哈,葛老哥,你这话应景啊。快请,酒宴已经备好,就在不远处!”蛮图看向方源,做了个侧身邀请的动作。

他没有在自家营地中设宴,而是赶往这里,把酒宴设在靠近葛家营地的野外,更显出热情和诚意。

“好,请!”方源笑着答应,心中却是一紧。

这乌云来去之快,别有蹊跷。恐怕是蛊师出行,而自己埋下定仙游蛊,还不到一个月。仙蛊的气息还未散去,但愿仙蛊不被发现。(未完待续)

------------

\end{this_body}

