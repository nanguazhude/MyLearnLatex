\newsection{仙鹤门的图谋}    %第二节:仙鹤门的图谋

\begin{this_body}

%1
面对雷坦的逼问,鹤风扬微微一笑,却是沉默不答。

%2
倒是太上三长老笑了笑:“风扬长老的意思,老夫明白。若是不选择庇护这个什么方源,狐仙福地仍旧是十派争夺。但若承认方源是我仙鹤门人,却可以直接排除其他九位对手,惟独我仙鹤门一派来夺取这块福地。”

%3
雷坦听了这话,面色微微一变,目光闪烁,不再咄咄逼问了。

%4
这时,鹤风扬从座位上站起来。

%5
他先是向帮助他说话的太上三长老,拱了拱手:“三长老英明!当时事发突然,谁也没有料到会有一凡人,通过仙蛊定仙游,传送到荡魂山巅。并在众目睽睽之下,直接夺走狐仙传承。”

%6
然后他再道:“这狐仙传承,处于天梯山上。诸位大人想必都清楚,天梯山乃是上达天庭的梯子,纵然已经损毁,多少年弃之不用,但仍旧代表着天庭的威仪。从某种意义上讲,攻打狐仙传承,就是攻打天梯山。攻打天梯山,就是攻打天庭。”

%7
“所以,天梯山上的福地,纵然数目不少,但从未有人胆敢攻伐过。这次的狐仙福地,也是等候着时间开启了,我们十人才一起出手,帮助狐仙地灵扩开入口。并不算攻伐。”

%8
“即便是真的动手,进攻狐仙福地。福地有地灵在,没有至少三位六转蛊仙联手,恐怕都攻不进去。而福地中枢,更有荡魂山守护。没有五六位蛊仙精诚合作,谁敢夸言能登上山巅去?雷坦,你能吗?”

%9
雷坦冷哼一声,有心辩驳,却终究闭上了嘴。

%10
在福地中。地灵能自由随意地调动一切资源,战力媲美蛊仙。最关键的是,地灵能压制一切五转到一转的蛊虫。

%11
蛊仙要进攻一处福地,最强的利器就是仙蛊。因为福地无法禁制仙蛊。

%12
但仙蛊难寻,很多蛊仙都没有一只仙蛊。就算是有,也未必是用于攻伐的蛊。

%13
这是典型的攻弱守强的局面。

%14
若是硬要是进攻福地,往往都是数位蛊仙联手,造成数量上的优势,然后进行仙元的对耗。福地的仙元消耗光了。接下来才是进攻。

%15
然而事实上,除非特殊情况,很少有蛊仙进攻福地。

%16
因为太划不来了。

%17
不仅是仙元珍稀,难以积蓄,还有福地自毁的原因。

%18
一旦守方想不开。自己毁灭了福地,大同风刮起来,进攻方根本就缴获不到什么战利品。

%19
蛊仙进攻福地,往往到头来,什么都捞不到,反而自己损失惨重。除非是深仇大恨,否则谁也不肯干这等亏本的买卖。

%20
鹤风扬见雷坦不说话。又继续道:“那方源登上山巅,取了传承,立即就命地灵关闭了福地。方源和方正长得几乎一模一样,其他九派蛊仙反应过来后。下意识地都以为是我仙鹤门动的手脚,便纷纷对我质问。我当时便想,若是说出真相,恐怕狐仙传承悬而未决。仍要十派较量,辛苦争夺。不如索性承认下来。哪怕为此付出点代价,却可以排除其他九大派的竞争。这样一来,今后我们仙鹤门就可以暗中攻略狐仙福地,而不用担心其他势力了。”

%21
鹤风扬说明了原委,雷坦冷哼一声,继续逼问道:“此计大有问题!狐仙福地就在天梯山上,如今关闭,不攻打出漏洞,我们怎么进去?”

%22
鹤风扬似乎早等着有人如此刁难,哈哈一笑:“我岂会没有算计?福地有灾劫,狐仙死于第五次地灾。我已经算过,如今狐仙福地距离第六次地灾,只剩下三个月的时间。他方源区区一个凡人,如何懂得抵御地灾?纵然有地灵帮忙,没有门派的支援,到那时,福地必然破损不堪,出现漏洞。”

%23
雷坦嗤笑一声:“就算是出了了漏洞,你还能真敢攻打不成?你刚才也说了,那狐仙传承,就在天梯山上!”

%24
鹤风扬不假思索,立即答道:“强攻是下下策,不到万不得已,不应采用。那方源只是区区一个凡人,只要漏洞现出,再施阴谋暗算,还怕拿不下他?呵呵,地灾一过,他必定心生忧患,对外援如饥似渴。我已经想过,先用方正,以情动之,好生劝说,使其合作,进行交易。交易次数越多,他自然放松警惕,再晓之以理动之以情,说不得就能说服他加入我们仙鹤门!”

%25
“若是他冥顽不灵,我们再秘密动手。奴隶蛊就是一个好方法。地灵不好对付,但针对他一介凡人,手段太多了。如果能够因此而获得定仙游蛊,那无疑就是完美的结果。”

%26
一听到定仙游仙蛊,在场的长老都不禁怦然心动。许多人开始小声议论起来,鹤风扬描绘出的情景,实在美好得很。更关键的是,他的计划很有成功的可能。

%27
雷坦察觉到氛围转变,只得愤愤坐下:“说的好听,但愿如此罢。”

%28
太上三长老沉吟道:“除去梦翼蛊这等,需要魂魄催动的特殊蛊虫外。仙蛊当中,大多数都需要仙元催动,这定仙游蛊也不例外。这方源拥有定仙游,又有地灵狐仙能催动福地仙元,随时都能脱身。对付他,要慎之又慎。还有一点,他不过只是个凡人,却能拥有仙蛊,恐怕背景也不简单。”

%29
鹤风扬点点头:“这点,晚辈也有所预料。先前,谎称方源是我派弟子,也是对其他九派的试探。现在看来,他的背后并非是中洲九派。晚辈猜测,方源来自南疆,很有可能背后就是南疆的某个超级家族。但不管是武家、商家、铁家、翼家等等,都远在南疆,鞭长莫及。真要进入中洲,战力还要受到压制。我们仙鹤门乃是中洲十大派之一,对付他们,大有胜算。”

%30
此话一出,众长老纷纷点头,小声议论起来。

%31
“的确。强龙难压地头蛇。”

%32
“中洲可是我们仙鹤门的地盘!”

%33
“若是真来动手,哼哼。”

%34
“即便不算战力压制,我们仙鹤门也比四大域的任何超级部族,都要强上一筹。”

%35
三长老紧皱的眉头也略微松缓下来:“现在,还有一点疑虑。定仙游蛊,须得使用者记忆深刻,清楚具体的地貌。他方源不过一介凡人,又似乎远在南疆,怎么知道狐仙福地的景象?又如何将时间掐得如此精准?难道说。狐仙曾经在南疆有所布置?或者,却是天梯山上的魔道蛊仙使坏?”

%36
鹤风扬躬身行礼:“这点晚辈着实不知。此事实在蹊跷,原本只是一道血海真传的线索。当年门人因此叛变,逃亡南疆。数年前晚辈便派遣天鹤上人,前往南疆。清理门户,收回真传。但天鹤上人失败,反被方源夺了真传,拥有血颅蛊。此子心狠毒辣,竟然当场屠杀亲族,以血颅蛊之能提升资质。天鹤上人不甘失败,就带回方正。以图再取。”

%37
鹤风扬自然也万万没有料到,当年一个名不见经传的小人物,居然有一天以这种令他目瞪口呆的方式,悍然登场。并且破坏他的大计,给他惹下如此巨大的麻烦。

%38
这种感觉相当古怪。

%39
就好像是人走在路上,一只小蚂蚁,忽然跳到人的鼻子上。张牙舞爪。

%40
这只蚂蚁是哪里来的?胆大包天!

%41
人伸出两根手指就能碾死它。但偏偏局面特殊,人还拿捏不得。只能暂时任由这只小蚂蚁耀武扬威。

%42
“怎么又是这血海传承……”听着鹤风扬叙述原委,当场就有许多长老皱起眉头,感到心中烦闷。

%43
血海传承,源自魔道巨擘血海老祖。

%44
他杀人如麻,遗臭万年。以七转蛊师之能,竟然布置了数十万个传承密地,地点遍及中洲、南疆、北原、西漠、东海五大域。

%45
他在死前怪笑:“血道不孤,遗毒万世!”

%46
而今,果真如他所言,不知道多少凡人,因此受惠。血海传承已经被公认为,是全天下最普及,数量最多的传承。没有之一!无数正道为之头疼。

%47
“那个血屠,原本不过是个屠夫,不就是得到了血海传承,成了中洲有名的魔修吗?”

%48
“昔年,万龙坞的宋紫星,得到了血海真传之一,叛逃了门派,连着我们十大派都为此蒙羞。现在人家已经是七转的蛊仙,号称“血龙”。万龙坞为了剿杀这个叛徒,洗刷耻辱,先后出动了八位蛊仙,五位六转,三位七转。结果被他直接杀了四个,打残三个,打退一个!”

%49
“据说血海有九道真传。分别有血颅蛊、血手印蛊、血气蛊、血汗蛊、经血蛊、血影蛊、血战蛊,以及上古荒兽戾血龙蝠,六转仙蛊血神子。可谓是血道大成者啊……”

%50
“这么说来,血海真传已经有四道现世。一道血颅蛊,在这方源的手中。一道血手印蛊,在南疆当代商家族长的手中。一道戾血龙蝠,在宋紫星的手中。”

%51
“据说,那个商家族长已经获得了第二道血海真传了……”

%52
“那是传闻,未经证实,不足为据。”

%53
众长老交头接耳,小声议论。

%54
“好了,不谈血海真传。找了这么多年,就像是大海捞针,各种传闻倒是日益增多。听得老夫耳朵都生茧了。”太上大长老一挥手,顿时止住了议论声。

%55
他转移目光,看向鹤风扬:“风扬长老,这件事情,既然由你而起,那你就负责到底。夺了福地,立下功劳,门派不会吝啬赏赐。”

%56
“晚辈遵命!”鹤风扬含笑领命。

%57
见鹤风扬落到这么个好差事,一旁的雷坦有心阻止,但既然是太上大长老发话,他也只能无奈暗恨。

%58
但接着,太上大长老又道:“既然由你负责,那风扬长老,你就交出我素蛊吧。凤九歌来信索赔,要求一只仙蛊。你是负责此事的人,就给我平息此事。”

%59
雷坦顿时大喜。

%60
鹤风扬苦笑领命。

\end{this_body}

