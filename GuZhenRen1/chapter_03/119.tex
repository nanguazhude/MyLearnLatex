\newsection{威望隆重功第一}    %第一百一十九节:威望隆重功第一

\begin{this_body}

%1
太白云生身材高大,相貌奇古,鬓发苍苍如雪,满脸皱纹深皱。

%2
他已经有八十多岁,一双眼睛,却并没有老朽昏花,而是天性中悲天悯人的温柔,以及看破世俗的平淡。

%3
他从七岁时,就立志行走北原,救助苍生。

%4
他一生跌宕起伏,饱受命运颠沛。家族破灭,成为奴隶蛊师,被妻子暗算背叛,成为异人俘虏,奇遇中获得宙道蛊仙传承,将死时得到兄弟舍命救治……

%5
如今,他已经成了一个活着的传奇。

%6
虽然是孤身寡人,但却是公认的正道大蛊师。其仁慈之名,深入北原人心,威望之重,远超常山阴、黑楼兰、刘文武等等之流。

%7
就在黑楼兰面对战局,一筹莫展,已经心生退意的时候,他只身来到营外,手持着一份书信。

%8
黑楼兰解开书信阅览,顿知缘由。

%9
原来,当年黑家太上家老黑柏,看中太白云生,曾经多次指点或者搭救他。如今黑家大军陷入困难局面,一直暗中注视的黑柏,便书信一封,传达给太白云生,令其前来支援。

%10
黑楼兰知道太白云生之能,大喜过望。当晚设宴,笼罩招待。

%11
到了第二日,天刚亮,黑楼兰便迫不及待,排列阵势,请太白云生出手。

%12
在众人期待的目光下,太白云生施施然走到阵前,仰望着眼前的高山。

%13
古家擅长土道,已是北原出了名的。垒石成山,这在地球上是匪夷所思,无法达成的战术。但在这个世界里,眼前的这座十几天浇筑的新山,告诉人们,没有什么不可能。

%14
盟主古国龙,高居山巅,俯视山脚。

%15
看到一位白衣雪发的老人出阵,他周围的蛊师们都爆发出哄笑,或者不屑的讽刺,但古国龙的心中却升腾起不妙之感。

%16
他心知肚明,自己这招垒土成山,是建立在自家土道蛊师数量众多的基础上。旁家势力纵然难以模仿,但要破解,却并非没有途径。

%17
古国龙被黑家大军连败几场,军力受损,原先争雄的野心早已经淡了。他左思右想,决定投靠刘家。

%18
刘家刘文武仁厚英明,宽于待人,严于利己,比黑楼兰的名声好多了。在多日之前,他便暗中书信,向刘文武表达投靠之意。

%19
“刘文武公子已经回信,答应了我族的投靠,如今正在率军赶来支援。我只要固守待援,再支撑七天,便能拨开云雾见青天,脱离困境了。”

%20
古国龙心中暗暗为自己打气,就在此时,太白云生缓缓地伸出双手。

%21
他的手掌宽大,老茧丛生,皱纹遍布,使人联想到古树的树皮。

%22
他缓缓调动真元,双手皆绽放出微弱的银光。银光起先微弱,但很快就渐渐强盛,几下眨眼的功夫,银光强盛,已经令人不能直接注视。

%23
“山如故。”太白云生悠悠吟诵,声音响遏行云。

%24
山巅上,古国龙听了这声音,顿时脸上涌现出骇然之色:“不好,他竟是太白云生!”

%25
说时迟,那时快!

%26
只见银光一爆,化为一道笔直光柱,直接轰击在山巅。

%27
无数蛊师见机不妙,立即催起防御蛊虫,或者打出攻击,进行拦截。

%28
但银光无视任何拦截,普照山巅。

%29
人兽皆安然无恙,但古家脚下的山石,不管有多么庞大坚厚,在银光照耀之下,宛若烈日下的残雪,以肉眼可见的速度,化为一片片的虚无,好像原本就不存在似的。

%30
古家大军的脚下,失去支撑,纷纷坠落。一时间,立即人仰马翻,从五六丈高的半空中,跌落到山石上,死伤无数。

%31
再蠢的古家蛊师,此时也意识到了危机。

%32
他们纷纷惊呼起来。

%33
“这样的力量,这是太白云生大人的山如故!”

%34
“天呐,太白老先生为什么要帮助暴君黑楼兰?”

%35
“太白云生大人,当年为我族恢复元泉,是我族的救命恩人。现在却要让我们和他交战吗?”

%36
古家大军脚下的新山,是他们内心最深处的底气所在,现在轰然崩解了不说。太白云生的个人威望,更是动摇他们斗志的巨大因素。

%37
“哈哈哈,果然不愧是太白云生,一出手就是不同凡响啊。”黑楼兰坐于王帐当中,见此情景,发出张狂的笑声。

%38
他也没有料到,居然家族方面还有这一个暗手。

%39
不过,各大超级部族的太上家老们,都会时不时地从魔道、正道的凡人蛊师中,挑选出自己看好的种子,加以栽培。

%40
一旦这些种子,日后成就了蛊仙,常常就被超级部族吸纳,成为他们的外姓太上家老。

%41
这是超级部族,维护自身地位的发展策略之一。

%42
显然,太白云生就是被六转蛊仙黑柏看好,认为日后能晋升蛊仙境界的希望种子。

%43
看着敌军狼狈模样,黑家大军士气大振,很多人都发出轰然的大笑声,还有许多人高声叫嚣,要屠尽敌军上下老小。

%44
王帐中,蛊师强者们亦是欢欣鼓舞,唯有方源一脸沉静。太白云生的出现,早就在他的意料当中。

%45
前世五百年记忆中,太白云生就是从这当口,参加了黑家大军,并且一路辅助,带给黑楼兰巨大的帮助。

%46
黑楼兰最终能够战胜诸雄,很大程度上得归功于太白云生。

%47
但太白云生,生性仁慈,在一路辅佐当中,深刻认识到黑楼兰凶残暴虐的性情。是以,当他进入王庭福地之中,就在那里晋升为蛊仙,并未答应黑柏的要求,成为黑家的外姓太上家老。

%48
“杀!杀死他们,这群狗东西,居然敢垒土成山,负隅顽抗!”黑楼兰兴奋地吼叫着。

%49
阵前的太白云生,听了黑楼兰的话,却是皱起眉头,他悠悠地叹了一口气,却没有继续出手,而是传音,对黑楼兰劝道:“盟主,上天有好生之德,何必大开杀戒?历来王庭之争,无不血流漂橹,伤亡惨重。盟主既要入住王庭,不若收降古家大军,老夫愿作为说客!”

%50
古家大军垒土成山,用来对抗黑家,但是面对太白云生,这浇筑的山峦却成了他们致命的陷阱。

%51
现在的情形是,黑家大军牢牢地包围古家,密不透风。

%52
太白陨石只要信手而为,就能将古家折损大半。古家必然不会坐以待毙,但当新山殆尽,他们的军力也必然所剩无几,最后发动的冲锋,根本没有任何的威胁。

%53
但太白云生却没有这么做。

%54
黑楼兰眼中凶芒闪烁,他虽然心中早就杀机沸腾,但是太白云生的面子却要顾虑。

%55
太白云生可不是普通蛊师。

%56
他本身是极稀少的宙道蛊师,修为高达五转巅峰,在北原的威望如日中天,影响力遍及草原。

%57
黑楼兰沉吟了一番,回道:“那就听老先生这一会吧,不过老先生独自上山,实在太危险了。我遣六位四转强者,为老先生保驾护航!”

%58
太白云生点点头,随后在重重护卫之下,来到山上。

%59
他威望极厚,仁慈之名深入人心,所到之处,敌军自发地分开两旁,露出中间的过道。

%60
“不想在此时此刻,又见恩公了。”古国龙苦笑连连,上前见礼。

%61
当年,古家元泉干枯,被几大部族排挤,迁徙十分危险,就请了太白云生过来救治。太白云生没有收取任何的费用,无偿出手,是古家上下的恩人。

%62
在太白云生的劝说之下,古国龙尽管心仪刘文武,但奈何形势比人强,他不得不低头。

%63
太白云生上山不过一刻,便下了山。

%64
他上山时,只有七人。下山时,却带领着十多万人。

%65
是役,太白云生说服成功,古家大军全部投靠了黑楼兰,黑家大军因此军力暴涨。

%66
太白云生以一人之力,改变战局。又以深重威望,解救十多万人的性命,同时也为黑家立下大功。

%67
太白云生来到黑家大军的第一天,便荣登战功榜首位。与其相对的,是方源。

%68
狼王常山阴的名字,位于战功榜最后,鲜红而又巨大的负数,和太白云生的战功,形成鲜明的对比。

%69
当晚,黑楼兰下令,举办庆功喜宴,也是为太白云生接风洗尘。

%70
月明星稀,篝火冲天。

%71
觥筹交错,乐声响彻云霄,美貌的少女穿着北原的衣袍,带着金银玉石的琳琅雕缀,围绕着篝火翩翩起舞。

%72
黑楼兰频频向太白云生敬酒,他赞道:“有老先生在,任何的防线都将形同虚设!”

%73
太白云生手中,有两只北原世人都众所周知的五转蛊。

%74
一只名为“山如故”,一只名为“江如故”,皆是宙道蛊虫。

%75
前者能令大地厚土山峦丘谷,恢复到原来面目。后者能令江河湖泊溪流泉瀑,还原本来风貌。

%76
古国龙浇筑新山,原本此地乃是一片平坦的草原。因此在山如故蛊的作用下,还原本来地貌。

%77
古家原先的元泉,则是被江如故蛊,恢复成原先的状态,可以重新产出元石。

%78
而大军相争,浇筑防线,常常以土道蛊虫为主,建设绵延千里的高大城墙。这些城墙在山如故蛊的作用下,都会还原成平坦的草地。因此黑楼兰说“任何的防线都将形同虚设”,深具道理!

\end{this_body}

