\newsection{仙蛊和稀泥}    %第二十节:仙蛊和稀泥

\begin{this_body}

咔吧。◎文學館r />

方源一手捏碎胆石,顿时一团泥浆流出来,沾了他一手。

方源目光凝重至极,又随手取过一块山上的碎石子,紧紧地握着手中。

他的手上,沾着的黄色泥浆,以一种缓慢的速度,消融着石子,将坚固的石块转化成更多的泥浆。

片刻之后,方源就明显感到,手中的碎石块在缩小。

大约一盏茶的功夫,碎石块彻底化为泥浆,消失不见。

黄色的泥浆,顺着方源的手指缝隙,流淌到荡魂山上,对荡魂山持续不断地造成伤害。

方源沉默了一下,这才缓缓开口,问道:“情况有多严重?”

小狐仙擦了一把眼泪,一边抽泣着,一边答道:“现在大部分的荡魂山,已经都被黄泥浆侵蚀了。山腰以下的胆石,十颗中有六颗都是黄泥。主人,怎么办呀?荡魂山要死了……呜呜呜,都怪我,没有提前察觉到。”

方源摸了摸小狐仙的脑袋,安慰道:“这不是你的错,无须自责,罪魁祸首还是那只泥沼蟹啊,真不愧是荒兽,真不愧是地灾!”

地灵的能力,各不相同,也各有差异,跟蛊仙以及福地都有关系。

他叹息一声,接着道:“我原先还觉得庆幸,荒兽身上没有仙蛊。原来这头泥沼蟹早就用了消耗型的仙蛊,仙蛊的力量就蕴藏在泥浆当中。血肉碰到了无事,但是山石却要遭殃,会被渐渐地同化成泥浆。”

之前的地灾中,泥沼蟹喷吐出大量的泥浆,从泥浆中杀出螃蟹大军。

方源剿杀了海量的螃蟹,但事实上。真正的杀手锏反而是这黄泥土浆。

方源猜测,这应该是六转和稀泥蛊的效用。

和稀泥蛊是自然形成的蛊,一二转的非常多,三四转的也很常见,常常被蛊师用作建筑城池。五转的比较稀少,很多五转蛊师没有得力的五转蛊时,往往会选择和稀泥这种蛊,暂时使用着。到了六转,全天下就只有一只了。并且和稀泥仙蛊只能用一次。

和稀泥蛊只能作用泥土,想必泥沼蟹生活的那个沼泽,就曾经被和稀泥仙蛊作用了。泥沼蟹每天吞吐沼泽中的泥浆,因此就将和稀泥蛊的力量,带到了狐仙福地中来。

泥沼蟹死后。战场虽然被打扫过,但大量的黄泥水浆已经渗透到山里去,大地中去。

和稀泥仙蛊的力量是如此的隐蔽,黄泥水浆自然也没有仙蛊的半点气息,若非方源这次安排石人进山敲石,还未必能发现端倪。

不过就算他提前发觉真相,地灾来临时。他也没有能力阻止。

方源面沉如水。

荡魂山正在被仙蛊的力量侵蚀,渐渐同化为黄泥浆家养小仙女。这是一个天大的噩耗!

整个狐仙福地当中,最后价值的就属这座山。他还打算凭此山,培养石人贩卖。今后他自己进一步壮魂。还得依靠此山,绝不能坐视不管,任由局面恶化下去。

当即,方源就命令小狐仙。尽量将黄泥浆清理出去。

这样一来,就大大延缓了危情。

但荡魂山内部也被侵蚀着。此法治标不治本。这是和稀泥仙蛊的力量,要清除掉它,方源必须还得借助其他仙蛊的力量!

“我冒如此巨大的风险,侥幸得到壮魂的圣地。就算是我日后成就蛊仙,这荡魂山都有大用。绝不能让它就此毁去。老天不想让这方圣地落入人的手中,因此降下如此地灾,那我就要逆天而行。呵呵,与人斗,与天斗,人生之趣,不外如斯。”

以方源的见识,还不至于毫无办法,手足无措。

他想了十多种解决的方案,排除掉其中不现实的,去了一大半。再排除掉难度较大的,只剩下三种。

第一种,是土道六转化石蛊。此蛊现在西漠,由六转蛊仙孙醋执掌。孙醋是正道蛊师,用此蛊化沙成石,方便凡人在沙漠筑城,深受凡人的感恩戴德。他心慈手软,重视亲情,最宠爱重孙女。方源若能擒拿了他的重孙女当做人质,必能令其就范。

第二种,同样是土道六转仙蛊,名为东山再起。此蛊已经现世,藏在东海的海市福地当中,方源可以进入福地,用另一只仙蛊换取此蛊。

第三种,则是宙道六转仙蛊江山如故。此蛊还未诞生,并非自然形成。其主太白云生,目前还只是北原的一位五转蛊师罢了。

“我在南疆三叉山,当众炼成了仙蛊定仙游,不用想也知道,现在南疆已经传的沸沸扬扬。一个凡人却有仙蛊在手,恐怕南疆的那些蛊仙也都被惊动了,正满世界寻找我。”

方源在成就蛊仙之前,根本不用想再踏入南疆。

“天下五大域虽然独立,但是南疆的超级家族翼家,却是和东海神秘势力有着联系。我炼成仙蛊的消息,要传到中洲等地,至少需要两三年的时间。但是东海却未必如此。”

方源首先排除东海。

“至于西漠,是全天下商贸最为发达的地方。一座座城池,凭靠着沙漠中的绿洲生存着。我若能将石人卖到那里去,必定能大赚特赚。可惜,商贸发达,意味着信息情报也发达。我一个南疆蛊师,绝对是个肥羊。到达那里,恐怕刚刚进城就会被注意了。”

方源前世五百年流荡五大域,最后才选择在中洲落户,成就蛊仙。他对西漠的情况也比较了解。

“相比较西漠而言,北原是个大草原,各大部族在草原上放牧,迁徙,战斗,繁衍,流动性相当的大,一些中小型的部族更是管理混乱,更能令我浑说摸鱼。”

西漠和北原不同。

西漠中,人族依凭着绿洲生活,很多人常年聚集在一块儿。只要绿洲不失,谁也不会闲的没事,在危机四伏的浩瀚沙漠中长途跋涉,去远征另一块绿洲。

而北原,部族必须时不时的迁徙,寻找肥沃鲜美的草地。那里气候变幻无常,时不时强大的气象变化,也会在一夜之间,摧毁家园,令部族不得不再次。如此一来,一个个部族四处流动,矛盾碰撞,战斗必不可少。所以北原的蛊师是最多的,也是五大域中最擅长战斗的。

方源若选择西漠,绑架人质,要挟蛊仙,必定会给安稳的西漠造成巨大而又持久的轰动星空战神全文阅读。

但他若选择北原,就算他杀死五转蛊师太白云生,也不过绞荡风云一时。几个月后,人们就会遗忘掉他。

方源仔细考虑了一番后,觉得混乱的北原,更加适合他行动。

西漠的孙醋已经是蛊仙,而北原的太白云生如今还只是五转巅峰。

选定了方向,方源又苦思冥想,从记忆中搜刮出各种有用的信息,编织他的北原大计。

计划赶不上变化。重生以来,他的计划都是不断变动的。

青茅山上是第一次,他成了甲等资质,突破重大,因此改变。三叉山是第二次,方源一步冲天,原本的计划变得更加面目全非。

虽然义天山大战,还有利可图,但南疆方源已经混不下去了。

至于中洲,更加不可能了。

他一个凡人,名传正道十大派,被仙鹤门上下,还有天梯山众多魔道蛊仙都盯上。

只要他一天不成就蛊仙,就得龟缩在福地中,不能出来。

他原本也想打算,在狐仙福地中好好生存。

福地中有着充足的资源,方源是想效仿凤九歌,闷头苦练,尽快地达到蛊仙境界,将春秋蝉的这个最大内患永久地解决掉。

方源虽然是个冒险分子,喜欢以小博大,又常常在生死线上挣扎,但并不意味着他排斥安宁稳定的生活。

放着安全稳定的修行方式不用,偏偏喜欢上蹿下跳,四处得瑟折腾,冒风险胡乱尝试,那是脑袋坏掉了。

孤独、寂寞、枯燥,从来不是方源修行的阻碍。

如果心性都浅薄到忍受不了这些,怎么能够一路成功地走下去?

然而世事变幻,事与愿违才是人生常情。

方源想要闷头修行,福地发展势头也迅猛且良好。仙鹤门虽然是个外患,但方源始终掌握着主动权,对方虽然势大,一段时间内却还拿捏不住他。

未来前景堪称美好,顺风顺水,一切都在向好的方面转变。但就在这时,荡魂山出了毛病!

对于狐仙福地,荡魂山的重要性不言而喻。此山一出毛病,方源的贸易就从根本上坍塌了,他的修行计划也化为泡影。

为此,方源不得不再次更改他的计划,出走福地,远征北原。

“幸亏我未雨绸缪,多做了几手准备,否则此时就被动了。”

此后几个月,方源将全部的精力都投入到炼蛊当中。

他用四转的金杯银盏蛊,结合四转旁推侧引蛊,四转移步换形蛊,合炼出五转的推杯换盏蛊。

又先后炼得蒙尘蛊,皓珠蛊,暗投蛊,地藏花王蛊。

“地灵,我不在的日子,你就照着我的布置去做。”临走前,方源叮嘱道。

小狐仙眼眶泛红,依依不舍:“主人,人家在这里等你,你可要早去早回呀。”

说完,催动青提仙元,灌注到定仙游蛊中。

碧光一爆,方源骤然消失。

\end{this_body}

