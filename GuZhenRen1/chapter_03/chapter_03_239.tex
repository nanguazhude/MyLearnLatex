\newsection{不犯同样的错误(上)}    %第二百三十九节:不犯同样的错误(上)

\begin{this_body}

%1
“正是!”

%2
方源说着,在心底出一声冷笑,忽然反问墨瑶意志:“你不是一直想让我动用它吗?”

%3
“嗯?方源小子,你这话是什么意思?我哪里知道你有春秋蝉这件事!”墨瑶意志一怔,旋即否认。

%4
“呵呵呵……”方源一阵长笑,“墨瑶啊,我这个人生性谨慎,向来未虑胜,先虑败!所以从你潜藏在我脑海中的第一天开始,我就设想过最坏的情况。”

%5
什么是最坏的情况呢?

%6
“那就是我拥有春秋蝉,重生的秘密被你知道。我想,当初我第一次进入近水楼台,你的意志就已经获悉了这个秘密了吧?”

%7
墨瑶意志在方源的脑海中显现出来,她神情诚挚,摇头道:“方源小子,你是不是误会了什么?”

%8
方源不管她,继续道:“你得知这个秘密,却装作无知,主动现身。随后,你装作辜负了灵缘斋的愧疚样子,故意交给我一个将近水楼台还给灵缘斋的任务。事实上,这个只是你的一个幌子。”

%9
“你为了博取我的信任,可谓用心良苦。你帮助我构思六臂天尸王,教导我墨化等炼道杀招,辅助我偷偷潜进真阳楼,以及在战斗中你屡次示警,充任参谋的角色。这些行为,都是在降低我的警惕之心。同时,你一直在篡改我的念头,潜移默化地影响我的思考,真正用心险恶!”

%10
墨瑶意志声调抬高,脸现不悦之色:“小家伙。你究竟怎么了?居然这样想我,你吃错药了?”

%11
“呵呵呵。墨瑶,你很了不起。我在你身上栽了一个不小的跟头。你表演得惟妙惟肖,我几乎都当真了。巨阳意志苏醒时,你一面劝我远离这里,取近水楼台,回中洲,一面却影响我的思维,令我做出继续冒险的选择。”

%12
“在面对马鸿运时。你又一面劝说我帮助马鸿运,另一面却篡改我的念头,刺激我的杀机。引我的愤怒、狂热,让我冲动地去斩杀马鸿运,导致我黑棺气运大涨。平心而论,正常状态下的我算计利益。怎么会如此无脑呢?”

%13
“我初次见到王庭地灵霜玉孔雀。得知认主的条件。出了真阳楼后,居然回想起前世的一些事情,某个人物……我本人的性格还不至于软弱到这种地步。呵呵,这些都是我压在心底最深处的前世记忆。你也是你的手笔!你是想暗中出手,趁着我心境微微波动,窥探我前世的记忆!”

%14
墨瑶意志瞪大双眼,语气不善,充满了被冤枉的委屈:“方源小子。这一切明明都是你的选择,你现在落到这个下场。居然不肯承认自己的失败,反而归结到我的身上来了!”

%15
方源笑道:“哈哈哈,承认失败,我当然要承认失败!我的底蕴还是浅薄了,尤其是和你墨瑶、和巨阳比较起来,经验和眼界已经不是我的优势。但你知道,你最大的破绽在哪里吗?”

%16
墨瑶意志冷哼一声,双臂环抱在胸:“小子,你再这样说,我可要生气了!我一路如此帮你,你就是这样对待你的恩人的吗?”

%17
方源充耳不闻,自顾自地说道:“你最大的破绽,就是居然用近水楼台主动撞击七转无相手。”

%18
“你给我这个任务,要将近水楼台还给灵缘斋。”

%19
“为什么在紧要关头,你却放弃近水楼台,也要保护我呢?”

%20
“难道说,这些天的相处,已经让你改变了原则,倾向于我了吗?呵呵,我自问魅力还没有大到如此程度。”

%21
“这点就能充分证明,你所为的交还任务,根本就是你的幌子!”

%22
方源声调平淡,墨瑶意志听着却是声声如雷。

%23
她陷入了沉默。

%24
这一次,她没有反驳!

%25
方源淡笑一声,继续道:“除此之外,你还有其他破绽。其一,你将近水楼台依附我身,却隐瞒此点,显然是居心不良,想在关键时刻,用这个七转仙蛊屋制约我。”

%26
“其二,你若真心想和我合作,何必一直隐瞒你是何种意志?这就说明,你的意志身份一旦暴露,必然会引我的怀疑。”

%27
“其三,你故意估错时限,让我转变成六臂天尸之躯!我成了僵尸,竟然感到心灰意冷,这也是你做的手脚。”

%28
“其四,你多次捣鬼,暗中影响我的思绪,令我多次泛起催用春秋蝉的念头。我意志向来坚定,毫不犹豫,谋定而后动,一旦决定就会施行。怎么可能三番五次,泛起同一个念头呢?”

%29
“其五,你既然仙元稀少,我又成就蛊仙,为什么不及早地主动借出近水楼台呢?显然别有算计。”

%30
“呵呵呵……”墨瑶意志仰头大笑,反问道,“方源小子,你这是前言不搭后语啊。你别忘了,你的命是我救下的。没有我催动近水楼台,你早就被巨阳意志杀死了!你现在又说,我故意陷害你,估错时限,让你变成六臂天尸王。这样又救你,又害你,我有病吗?”

%31
“你当然没有病,你始终都清醒得很!既害我,又救我,这都是你的计划。呵呵,只要设身处地地取想,换位思考,你的动机不难猜测。”

%32
顿了一顿,方源继续道,“若我是一股意志,残留在世间万年之久,本体却已经死亡,所爱之人也渡劫失败。这个时候,我若见到了春秋蝉,我会怎么样呢?”

%33
“答案不言而喻,我当然是想利用春秋蝉,逆流光阴长河!春秋蝉的效果,是将蛊师的全部都献祭,仅仅保留蛊师的意识在蛊虫身上。到关键时刻,我混入春秋蝉当中,毁灭掉蛊师的意志,从而取而代之,回到过去,再度重生,改变自己的历史!”

%34
方源掷地有声地分析,彰显自信的非凡气度。

%35
墨瑶意志瞳孔收缩,再度陷入沉默。

%36
方源道:“然而,你要实现这个目的,却是困难重重,主要有两大难关。”

%37
“第一,你只是意志。不管是什么意志,大部分的蛊虫你都无法独立催动。春秋蝉不仅也在此列,而且需要更多,它需要蛊师自爆,将真元、血肉、身上的其他蛊虫都牺牲,作为力量推动。这股力量越强,就能回到更古老的过去。”

%38
“第二,春秋蝉是我的蛊虫,又是本命蛊,非同凡俗。你不是无相手,不好强行夺走。后来,我又找到了许多智道蛊虫,可以产生多种意志,因此你更加难以抢夺。”

%39
“这样一来,你就只好与我虚以委蛇,见机行事。一方面,你要壮大我、帮助我,提升我的底蕴,让我自爆时获得更强的力量推动春秋蝉。另一方面,你要因势利导,营造出一个绝境,让我顾不得巨大风险,不得不使用春秋蝉!”

%40
说到这里,方源叹息一声:“我之前警惕你,但随着你渐渐影响我的思维,终究还是一步步落入了你的算计和安排当中。你暗中影响我,使得我在真阳楼中不断犯险。又鼓动我对付马鸿运,对抗巨阳意志,对付黑楼兰。很多事情明明收益不佳,并不理智,我却一而再,再而三地冲动。你让我到处竖立强敌,就是为了方便营造绝境。”

%41
“但另一方面,你又害怕一个处理不好,令我死了。所以,在危机关头,我命垂一线,你不得不出手,暴露近水楼台,也要帮助我免受巨阳意志的致命一击。”

%42
“所以,当我要遭受无相手捉拿时,你驱使近水楼台主动撞去,就是害怕它随机抓走我的春秋蝉,担心春秋蝉的暴露。”

%43
意志之间的交流,内容虽多,却是迅捷无比。

%44
但方源这次说完,墨瑶意志却久久没有回应。

%45
脑海中一片沉静。

%46
而外界的巨阳意志数数,已经从十,倒数到了六。

%47
终于,墨瑶打破沉寂,鼓掌而笑:“呵呵呵呵,好,好分析!真是抽丝剥茧,条理分明。想不到,我筹谋了这么久,终究还是没骗得过你。你真叫我刮目相看,小子,就算你是重生之人,才情也是卓越。若是有生之年,我们也许会成为良友,也说不定呢。可惜,可惜。”

%48
方源通彻一切,和墨瑶意志摊牌。

%49
墨瑶意志再也无法伪装下去,这一刻,她选择坦白!

%50
她身姿曼妙,脸罩黑纱,虽是意志,但宛若静静开放的一朵夜莲,风采卓卓,气质绝佳。

%51
不愧是当初灵缘斋的第三十六代仙子,七转蛊仙,炼道宗师。就算被当面揭破自己的算计阴谋,也没有不好意思。见不到继续隐瞒下去的希望,她态度坦然,十分干脆。

%52
她叹息道:“当初我的本体冒险进入八十八角真阳楼,在真传秘境中收获一道真传,造就了我这股意志。留下土丘传承,一方面是为了有备无患,保留继续炼制招灾仙蛊的可能。另一方面,也是一个陷阱,主要针对灵缘斋。当年我为了薄青,早就背叛了门派。只是没想到,万年之后,等来的人是你。”

%53
“其实,早在你炼制招灾仙蛊的时候,我就现了你身上的春秋蝉。我乃炼道宗师,仙蛊炼制过程被我动过手脚。炼成的招灾仙蛊里,早有我的意志埋伏。这个布置,原本是要对付灵缘斋的追兵,因为没有人现可以炼制仙蛊的机会,而不去紧紧抓住的。”

%54
ps:恭贺“吉屎”同学成为本书第一盟主,谢谢支持。今天两更,稍后还有一更。

\end{this_body}

