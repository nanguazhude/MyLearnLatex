\newsection{第二空窍!}    %第八十五节:第二空窍!

\begin{this_body}

%1
就在北原风云交汇之时,琅琊福地中的交易,也步入了尾声。

%2
“你要八十八角真阳楼的消息?嗯……也不是不能告诉你。”琅琊地灵沉吟了一句后,便用犀利的目光盯住方源。

%3
琅琊地灵的执念,只是关乎遁空蛊,以及大梦仙尊。

%4
再说,如今巨阳仙尊早已仙逝,琅琊地灵也非长毛老祖,而是一种别样的生命体。

%5
“只是当初炼制这只仙屋蛊时,只有巨阳仙尊一人旁观。你又是如何知晓这个秘辛的呢?”琅琊地灵问道。

%6
方源笑了笑。

%7
在他五百年的前世,马鸿运进入八十八角真阳楼,获得巨阳仙尊的传承。

%8
后来,马鸿运又因为兵败,被人追杀,误入了琅琊福地,得到盗天魔尊的机缘。琅琊地灵看出了他的跟脚,便在交谈中,说出这个秘辛。

%9
马鸿运后来成就蛊仙,成为了北原中流砥柱,抗衡中洲的入侵的大英雄。

%10
在一次详谈中,他说出了这段秘密,之后便流传开来。

%11
当然,现在面对琅琊地灵的追问,方源却不能直接坦白这个事实。

%12
“我继承了盗天魔尊的传承,对于巨阳仙尊的东西,也有许多线索。之所以知道这秘密,其实也是模棱两口的推算。”方源答道,嘴上带着一丝傲然的笑意。

%13
琅琊地灵看着方源这种神色,忽然灵光一闪,脱口问道:“难道你是?”

%14
“不错,晚辈有些智道蛊师的手段。”方源随口便撒谎,偏偏神情恳切真挚至极。

%15
琅琊地灵叹息一声,看向方源的目光,便带着些复杂之意。

%16
智道来源悠远,推根溯源,乃是远古年代,三百万年之前的一位九转大能所创。

%17
此人是尊者当中,罕见的女性。

%18
她世号星宿仙尊,执掌天庭,为二代仙王。

%19
据传言,智慧蛊就曾经被她掌握在手,因此开辟了智道流派。

%20
她在死前,推演天机,算计后世三百万年,布下三局。

%21
在她死后,人族中陆续出了三位魔尊,皆杀上天庭去,却最终被这三局所阻。

%22
能在死后,算计三位同转的蛊尊,智道之能可见一斑。

%23
“小子,你竟然能够推算到这等地步,看来你的确得到了智道的精髓。”在得知方源同时还是智道蛊师之后,地灵的语气居然缓和下来。

%24
智道流传至今,真正继承者少之又少,但绝没有人会因此小觑智道蛊师。智道蛊师善于谋划推算,很多和智道蛊师作对的人,死都不知道怎么死的。

%25
而且,自古炼道、智道不分家。蛊师参悟天地,推衍炼蛊秘方,都要靠智道的许多手段。有很多的炼蛊大师、炼蛊宗师,设想秘方时遇到瓶颈,都会请智道蛊师出手相助。

%26
琅琊地灵很反感方源的狡诈,但当他知道方源有不俗的智道造诣之后,他的态度就有了缓和。

%27
琅琊地灵喜欢炼蛊,自然也喜欢钻研、创新蛊方。在这个过程中,他当然也遇到了许多关隘,单靠他本身的智慧难以解决。

%28
“也许将来,我还要请他帮助推衍秘方?”琅琊地灵的脑海中闪过这个念头。

%29
有道是:无欲则刚。

%30
琅琊地灵既然对方源有所求,态度自然好转了。

%31
他凭空召出一只四转东窗蛊,捏在手中。

%32
此蛊为四转信道,形如瓢虫,但背上甲壳却是方形,宛若窗棂,专门储存消息之用。

%33
琅琊地灵将缕缕意念灌注其中,片刻之后,松开手。

%34
东窗蛊翻开窗户般的甲壳,振翅飞舞,在空中绕了一个旋儿,乖巧地飞到方源的身边。

%35
方源探出意念,稍微查看了一下,脸上喜悦便现。

%36
“这些就是你想要的消息。现在该你了。”琅琊地灵提醒道。

%37
方源嘿嘿一笑:“第二空窍蛊既然成形,这么好的东西,我当然也有些急不可待。用之前,就会传信给你的。”

%38
“慢!”琅琊地灵这次却是长了心眼,“我怎么知道你会拖到什么时候用?你是智道蛊师,耍心眼是一流的,你还想凭此坑我第二回?这东窗蛊我们还没有交接呢,嘿,现在老夫只要意念一动,你手中的东窗蛊就会自爆。”

%39
方源皱起眉头:“难道你信不过我?”

%40
琅琊地灵一脸的怀疑之色:“哼!单凭你是盗天魔尊传人这点,我就信不过你。你小子又如此狡诈,还有智道手段,你出了福地,我还能追出去不成?万一故意拖着,借机勒索我怎么办?”

%41
“那你说怎么办吧?”方源不耐烦地道。

%42
“当然是现在,你就在老夫的面前,用了这第二空窍蛊。老夫这才放心啊。”琅琊地灵抚摸着花白的胡须,为自己能想到这个好主意而有些得意。

%43
方源咧开嘴:“嘿,我现在不过区区五转,没有仙元怎么能用这仙蛊呢?等到了福地外,靠我爷爷出手相助方可呢。”

%44
“你爷爷能助你,老夫难道不行吗?你不要废话,将第二空窍蛊借给老夫,由老夫亲自出手!”

%45
“这个……”方源只是推托。

%46
地灵见此,更是觉得方源要耍诈,态度变得更加强硬。

%47
方源无法,只好勉为其难,让琅琊地灵出手。

%48
琅琊地灵接过第二空窍蛊,灌注了仙元进去,便朝着方源一丢。

%49
这第二空窍蛊化为一颗豆大的绿光,跳进方源的身体当中。

%50
方源浑身剧震。

%51
绿豆般的光,先是直撞方源的腹部肚脐处。

%52
但在那里,被方源本身的空窍排挤,只好又跳上去,在方源的胸膛中央落定。

%53
随后,轰隆一声炸响,宛若晴天霹雳。

%54
豆绿之光,猛地爆炸开来,形成似实还虚的崭新空窍。

%55
第二空窍!

%56
“这就是第二空窍?”方源抚摸着自己的胸膛,一时失神,喃喃自语。

%57
“废话,不是第二空窍,还是什么?”琅琊地灵翻了一个大大的白眼,“第二空窍,对你修行大有裨益,成就蛊仙时,更有天大的好处。不过,你若还想有第三空窍蛊,那却不成了。这第三空窍蛊的秘方,还需要推演测算呢。”

%58
方源恢复神色,大有深意地看了琅琊地灵一眼。对方故意提了这茬,无疑是想今后和自己配合,推演蛊方。

%59
琅琊地灵被方源了然的目光,看得老脸一红。

%60
方源笑了笑,对琅琊地灵一礼:“此事以后再说吧,告辞。”

%61
“快滚,快滚!”琅琊地灵连连挥手。

%62
看着方源在自己的面前彻底消失,琅琊地灵这才松了一口气,脸上浮现一丝意图被识破的羞恼,咒骂道:“这个狡诈的小子。”

%63
但想到方源最后这话,似有合作之意,琅琊地灵的眼眸也随之闪烁了几下光。

%64
他捏着胡须,不由陷入遐想,他手头上早就积压了数十张未完成的蛊方。

%65
“那小子既然能推出这样的秘辛来,智道造诣已经相当不俗了。得他帮助,这数十张的蛊方,至少得有一大半能有所突破,其中十三张蛊方应该可以得到完善。但剩下来的七道蛊方,单凭两者合力,还远远不够。”

%66
“不过将来,要和这小子打交道,可得防范些,免得又被他算计。幸亏今天老夫谨慎,让他当场用了第二空窍蛊,否则还会受到他的勒索!咦,有些不对劲啊……驱动第二空窍蛊,我可是耗了仙元呢!”

%67
出了琅琊福地之后,方源便又顺着星门蛊,回到狐仙福地。

%68
狐仙福地,在他的经营之下,已经起了许多可喜的变化。

%69
在光线明媚的福地西部,大量的狼群,以及少量的狐群,栖息共存。

%70
狼群的到来,虽然打破了狐群的安逸生活,但并未带给狐群惨重的伤亡。

%71
这一切,还得归功于大量的成功移植在土地上的铁壳花。

%72
这些铁壳花海中,生活着数不尽的花粉兔。

%73
花粉兔容易繁衍,成为了狼群以及狐群的主要食物来源。

%74
而在东部,星光弥漫如幻,大大小小的湖泊星罗密布。

%75
大片的阴云上,种植着星屑草。星屑草长势喜人,原先只是移栽了阴云的中央地带,现在却已经蔓延到云朵的边缘了。

%76
“瑶光仙子的栽种经验,确实无误。星屑草是星萤虫群的食物,同时草群也会在星萤的帮助下,加速生长。”

%77
方源看得不住点头。正是因为之前,他从万象星君、瑶光仙子等人手中交易到了星萤虫群的培养心得,才使得自家少走了许多弯路,收获良多。

%78
“只是星萤虫群虽然也在壮大,但其中的星萤蛊数量却在减少。这些天,我三番五次动用星门蛊,星萤蛊的消耗速度要大大高过它们的形成速度。”

%79
方源又将目光,投向地面的湖泊。

%80
这些湖泊,都是他牵引水火,倒灌而成。

%81
现在这些湖泊中,生活着许多水狼。这些水狼在湖边筑巢,吃湖泊里面的青玉鲫鱼,有时候也会上岸,吞食在湖边汲水的地皮猪。

%82
青玉鲫鱼和地皮猪,皆是方源上次收购之物。如今大批豢养,在小狐仙的操持之下,都适应了新环境,成功地融合狐仙福地的食物链中。

%83
尤其是地皮猪这种野兽,繁殖力还强于花粉兔,几乎什么东西都能吃,甚至是泥土。整个猪群的规模正在逐渐扩大。若非水狼,以及放养的毒须狼群的狩猎,地皮猪的数量还要多出三四成来。

\end{this_body}

