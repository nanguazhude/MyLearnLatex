\newsection{没有什么不能出卖的}    %第六十一节:没有什么不能出卖的

\begin{this_body}

时隔数月,方源终于再次进入狐仙福地。(文學館r />

地灵小狐仙很是高兴。因为狐仙福地的时间流速,是外界的五倍。方源在北原是耗费了数月,对她而言则已经过了一年有余。

方源首先查看了身后的星门。

这片星门,比北原上的那面,还要微小。将将只有一人高,勉强让方源进出。

北原月牙湖畔的那面,却很高大,足有数丈高宽。

“星门蛊不分子母,消耗真元太多,五转巅峰蛊师都支撑不了三个呼吸,唯有用仙元才能长久催动腹黑权少,你先上全文阅读。小狐仙催动一只,另一只在我手中,也跟着催动起来,倒不需要我耗费真元。”

“然而星门蛊的催动,不仅需要仙元,还需要大量的星光,才能凝聚成门。北原是五域之一,繁星点点,星光充足。但在狐仙福地这里,星光太弱了,星光的来源只是一群星萤蛊。”

方源一边暗暗思索,一边将目光投向半空中飞舞着的一群星萤蛊。

星萤蛊是三转蛊,体型微小,和正常的萤火虫差不多。但是它们绽放的,却是货真价实的湛蓝星光。

这团星萤蛊,原本有五百多只。因为催动星门蛊,已经死去了三十二只了。而维持星门蛊的话,平均三个呼吸的时间,都会死去一只。

五百多只星萤蛊,看似众多,其实维持不了星门多久。

所以方源紧接着命令小狐仙,停止催动星门蛊。

星门消散,重新化为一颗椭圆形态的蓝宝石。小狐仙伸出粉嫩的小手一招,就将这只珍贵的星门蛊,招到自己的手中。

“主人,给你。”小狐仙双手捧着星门蛊。仰起头,睁着水汪汪的大眼睛,将星门蛊递给方源。

方源摸了摸她的小狐狸耳朵,温柔地笑道:“就放在你这里了,好好保管吧。今后我要用的话,都会通过推杯换盏蛊传信给你。”

“好的,主人。我一定好好保管的!”小狐仙郑重其事地将星门蛊放入衣服上的小口袋中,还用小手拍了拍口袋。

至于北原的那块星门蛊,则掉落在草地上。被万狼群严密地护卫着。

“主人,这里还有两块仙元石。是人家听了你的吩咐,用仙蛊秘方买下星萤蛊后,多出来的。”小狐仙又献宝道。

普通的元石,大小如同鸭蛋。是椭圆状的灰白石头。而仙元石同样是鸭蛋大小,却不是椭圆,而是圆珠状。通体晶莹透润,像是水晶般透明,却又有玉一般的润泽。

如果将元石比作凡人,那么仙元石就是蛊仙。

仙元石十分珍贵,可以用来补充仙元。同样的也是蛊仙交易时,采用的珍贵货币。偌大的蛊之世界中,只有天庭处出产仙元石。

“又见仙元石。”方源笑着感叹一声。

这两块仙元石,价值极高。两亿块元石也换不来。

五百年前世时,方源手中的仙元石最多也不过六十多块而已。

方源收起仙元石:“好了,我们先回荡魂山。”

小狐仙立即开心地嗯了一声,然后抓住方源的手。下一刻挪移到荡魂行宫里。

“快催动通天蛊吧。”方源命令道。

另一只星门蛊放在北原,他并不放心。

夜长梦多。方源要抓紧时间,尽快处理好狐仙福地这边的事务。

小狐仙连忙催动通天蛊。

洞地蛊连接福地,一旦种下,就不能更改。同样的,通天蛊也只能连接一处洞天。

方源的这只通天蛊,既然联通了宝黄天,今后就再也不能更换。

由小狐仙消耗微微的仙元,通体蛊化为一面无边的圆镜,镶嵌在半空中近身特工最新章节。

圆镜中,展现出宝黄天的景象。

这处洞天,空空荡荡,荡漾着一片柠檬般的黄光。没有寻常洞天里的山川、植被、兽群。

福地之上,就是洞天。

最著名的洞天,记载于《人祖传》中,就是赫赫有名的太古九天。

太日阳莽炼定仙游蛊,飞上九天,在青天里,采摘一节碧空的玉竹。在蓝天里,收集星光碎屑中的八角钻石。

太古九天,分别为白天、赤天、橙天、黄天、绿天、青天、蓝天、紫天、黑天。

但后来,人祖之子大闹天地,赤橙黄绿青蓝紫七天接连坠毁,到如今只剩下白天和黑天相互轮转交替。

宝黄天的来源,就牵涉到太古九天中的黄天。

中古时期,有一位八转蛊仙,号称多宝真人,意外得获得一块黄天碎片。他将其融入自家的洞天之后,就形成了宝黄天。

宝黄天是一处十分特殊的洞天,它空荡一片,没有山川树木,也没有鸟兽虫鱼,只存珍宝。

宝黄天中有着八转仙蛊——宝光蛊。

一转到三转的宝光蛊,十分普遍,四转、五转的宝光蛊,往往只有大型势力才能拥有。五转以上的宝光仙蛊,当然只有唯一的一枚,就在宝黄天中。

宝光蛊专门用来测验物品的价值。物品的价值越高,显现而出的宝光就越盛大。

商家城中,演武时都会测验双方身上的宝光,来评价蛊师手中蛊虫的价值,从而推测蛊师的战力。

宝光蛊的主要用途,不只是鉴定,还可以用来探宝。

宝光蛊的探测范围很小,往往要搭配其他蛊虫,扩大它的探测范围。

多宝真人当年,就是拥有仙蛊宝光,才寻得无数珍宝。久而久之,才有了“多宝”这个名号。

小狐仙不断催动通天蛊,镜面变化不定,闪烁出各种物品的图景。

蛊虫、兽群、异人、植被、矿脉、土壤、水、美酒等各种各样的物资,皆有贩卖。

忽然,画面一定,显现出一团星萤蛊。

“主人,还是那个万象星君卖的。”小狐仙用神念蛊沟通了几句后。对方源道。

“星萤蛊不能相互繁衍,我需要大量的普通星萤虫。你问他售价。”方源点点头,道。

小狐仙沟通了几句,转过来向方源汇报:“主人,万象星君说不卖。”

方源却不失望,反而笑了几声:“呵呵呵,这个世界上没有什么不能出卖的。不卖的往往只是利益不高,不令人心动罢了。”

说着,他便将记忆中的几道仙蛊残方。分别写在几张牛皮上,然后统统抛入通天蛊中。

这些牛皮,十分普通,但因为记载着仙蛊的残方,却显得价值连城。

通过通天蛊进入宝黄天后。一张张牛皮上都显出一丈到三丈不等的宝光。这些宝光五颜六色,姹紫嫣红,分外好看。

“告诉他,价钱好商量。”方源朗笑一声,对小狐仙道。

他不怕万象星君不动心天才邪女全文阅读。

仙蛊对蛊仙的诱惑力,是巨大无比的。这些虽然是残方,但就算如此。也是紧俏的商品。

积年累修的老字辈的蛊仙,哪一个手中不攥着数件,十数件的蛊仙残方?

但这些残方,他们往往都不会贩卖出去。

相同仙蛊的残方。相互结合起来,总会推演出正确的方子。按照这些方子炼成仙蛊之后,别人就无法拥有了。

因为这个关系,仙蛊残方很少交易。就算有。也不过是以残方换残方。

先前,方源就用了记忆中。最残次的秘方,宝光只有三尺,便和万象星君交换了一团星萤蛊。同时万象星君还补贴了方源两块仙元石。

现在这么多的秘方,宝光至少是一丈有余,由不得万象星君不砰然心动。

这些残方,投了进去后,立即在宝黄天中引起无数的专注。

大量的神念,蜂拥而至,小狐仙利用神念蛊接收得小脸蛋儿都有些发白了。

“主人,好多蛊仙传来神念,问我们这些残方怎么卖!”

方源笑了声:“你传了神念过去,就告诉他们:这些残方都卖,但要一个个来。先说这星萤虫,至少要十万只。”

小狐仙传了神念过去,立即有位号称“摇光仙子”的女蛊仙叫起来,说她福地中培育了不少的星萤虫,并且愿意交换。

方源呵呵一笑。

果然万象星君,也坐不住了,改变了态度。

紧接着,又有第三位蛊仙,自称“帝渊”,言说手中有星萤虫。

方源又等了片刻,再不见蛊仙发言,不禁心中感慨:这星萤虫不愧是太古之虫,流落到今天,已经如此残微。

有了竞争者,一切就好办了。

方源稳坐钓鱼台,让三家竞价。这是赤裸裸的阳谋,但修行到蛊仙这层次的哪有蠢货?三位并未恶性竞价,反而协商妥当,各出三万三千多只星萤虫。

这三群星萤虫,分别进入通天蛊中,各个绽放着宝光。

其中摇光仙子的虫群,宝光最弱,只有一丈八尺。帝渊的虫群,则显现出两丈的宝光。万象星君的宝光最强,有两丈三,这是因为他的虫群中还掺杂着不少星萤蛊。

“看来这万象星君手中,培养了大量的星萤虫群。我想起来了,星门蛊出现之后,万象星君就凭着贩卖星萤蛊,赚了大量的仙元石。”

方源忽然回忆起一个信息。

五百年的记忆,毕竟太多了,很多细节方源记得并不清楚,现在看到这里,原本模糊的记忆这才清晰起来。

星门蛊出现之后,导致星萤蛊的价格也水涨船高。后世宝黄天中,甚至都不贩卖星萤虫群了,只卖星萤蛊。

蛊是不能繁衍的,只有普通虫群才有这个能力。

当即,方源用了三张秘方,获得了十万只星萤虫。只要慢慢培养,今后的星萤蛊方源就不用花大价钱购买,自己就能自产自用了。

(ps:这个星期一天一更,生病了状态不佳,我需要调整的时间。)

\end{this_body}

