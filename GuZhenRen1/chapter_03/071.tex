\newsection{气泡鱼}    %第七十一节:气泡鱼

\begin{this_body}

通天蛊镶嵌在半空中,镜面中轮番闪现着狼的身影。

忽然画面定格,显现出一头龟背万狼王。

龟背狼体格宽厚,浑身墨绿,狼眼幽蓝。狼如其名,每只龟背狼的背上都长着一个乌龟似的甲壳。

甲壳十分沉重,龟背狼大约有三分之一的重量,都集中在里面。作用除了防御之外,还类似于驼峰,有着存储营养的功能。

在普通的野狼当中,这种狼的防御最强。

而通天蛊中,显现出来的这头万兽狼王,体型无疑更加庞大,足有普通龟背狼的十倍大小。圆形的龟背高耸着,青铜色坚固的龟壳锃光瓦亮,给人战力充沛,身强力壮之感。

“如今我的手中,狼群规模近三万。其中主要是三股狼群,分别是龟背狼、夜狼以及风狼。每一股狼群数目都近万。除此之外,还有少量的毒须狼,得自腐毒草原。百余头的水狼,是因为白眼狼而买下的。”

原本方源麾下的狼群,有三万多头。但几场战斗下来,损失不小,如今满打满算只有两万七千头。再刨除这次迁徙到狐仙福地中的老弱病残,手头上可战的狼群,就只剩下两万五千还不到。

这样一来,就需要补充了。

寻常的奴道蛊师,想要补充兽群,就得组织人手,进行抓捕,耗时耗力不说,太大的兽群还吞不下。

而方源拥有了通天蛊后,等若站在蛊仙的肩膀上,比普通蛊师多了一个选择――在宝黄天中收购兽群。

绝大多数的福地当中,都培养着野兽。这些野兽繁衍过多。对福地的生态造成压力,掌管福地的蛊仙们,就将其抛售到宝黄天中。这样一来,不仅可以减轻福地的压力。而且还能有所收获。

同样的,许多蛊仙想要培养福地,补充生态,就要购买一些野兽放进福地中豢养。因此宝黄天中,兽群的买卖十分平常普通。

宝黄天中,售卖的狼群种类繁多,琳琅满目。

北原的夜狼、风狼、水狼、毒须狼、龟背狼,南疆的电狼、苍狼、血狼、双头狼,西漠的丝绸狼、珍珠狼、沙狼、土豪狼。东海的墨水狼、贪狼、红狼、金背狼,中土的白眼狼、星狼、色狼、血狼……

这其中,不仅有普通的野兽。更有异兽,诸如白眼狼、土豪狼、贪狼。

当然,可媲美蛊仙战力的荒兽级的狼,却是没有的。不过据方源所知,琅琊福地中埋藏着一只荒兽桃太狼。

“我手中最大的三股狼群,其中夜狼中已有一头万兽王,风狼、龟背狼群若是也有一只万兽王统领,我就无需操纵那么多的百狼王、千狼王,极大的减轻驭兽时的负担。从而减少魂魄上的负担,收编更多的狼群。”

方源并不打算去更换狼群。那是多此一举。

他决定壮大手中的狼群。

小狐仙替他传去神念。询问这头龟背万狼王的价格。

售卖的蛊仙。自称南海龟仙,他想要的是一万株吹箫草。

吹箫草是北原特产。每根草的形状仿佛竖立的长箫。当风吹起来时,风声灌入草上的箫孔,就会发出一阵悦耳动听的箫音石激千重最新章节。

吹箫草本身就是炼蛊时的常见材料,同时大片的吹箫草丛中,还会渐渐衍生出一些音道、木道的蛊虫。

方源手中并无吹箫草,虽然知道吹箫草原的位置,但是现在时节不对,吹箫草都还是埋藏在土里的种子,并未生长。

方源知道了这个情况,却并不死心,而是关照小狐仙道:“你问问看,对方是否接受仙元石。”

小狐仙传达了神念,交流了几句后,回报道:“主人,他说有两万头龟背狼可以卖,但索要两块仙元石。”

方源嗤笑一声,这开价太高了。仙元石十分珍稀,龟背狼也不过普通野兽,不过买卖交易倒也需要个讨价还价的过程。

一番讨价还价之后,方源用了一块仙元石,将南海龟仙手中三万头龟背狼买下,同时还有一只龟背万狼王。

达成了交易之后,方源又分别买下了一万八千头风狼,兼风狼万兽王一头。又补充了夜狼两万,毒须狼五千,水狼六千头。共耗去两块半的仙元石。

这还不算完。

狼群进入狐仙福地,对原有的生态是一个严峻的考验,会产生诸多的影响。至少,方源得给这些狼群补充食物,否则这些狼在福地里会因为食物不足,而纷纷饿死。

狐仙福地中,也有不少的野兽。

比如野兔、野鸡等等,但这些动物,乃是狐群的食物来源。

狐仙在福地中,豢养了大量的狐狸,有红狐,金狐,云狐、风狐、秋水狐、流光狐等等。

这些狐群在第六次地灾中,损失惨重,但还是保留了种子。经过许多年的休养生息,也渐渐扩大规模,恢复了一些元气。

如果食物不足,势必会产生狼群和狐群的相互捕食,产生极为严重的内耗,得不偿失。

为了这些狐群和狼群,方源又在宝黄天中选择食种。

“黄金锦鲤、青玉鲫鱼、肥泥鱼,选哪一种好呢……咦?居然有气泡鱼的鱼籽!”

方源正挑选的当口,意外地发现了有蛊仙正在售卖气泡鱼的鱼籽。

他立即决定购买下来。

但气泡鱼的鱼籽,还是引起了一些蛊仙的注意,形成了竞价的场面。

方源当机立断,将价格提升到一块仙元石的高度。

“这些鱼籽不过两万多颗,居然有人出价一块仙元石!”

“这个人疯了,就算孵养出来,气泡鱼也过三千只的数量。虽然珍贵,但也值不了一块仙元石啊。”

蛊仙们纷纷传出嘲讽、鄙视的神念。

气泡鱼孵养困难。只有三成左右的成活率。在他们看来,方源花费的代价有些高了。

“如果我记忆不差,就是在今年年底,气泡海上两位蛊仙相斗。剧毒侵蚀整个气泡海,将其化为了绝境。到那时,气泡鱼的价格就要暴涨十倍有余了。”将这些鱼籽得到手中,方源呵呵冷笑。

能成就蛊仙的,毫无疑问都是人中龙凤,天之骄子。抛除一些不学无术的仙二代之外,绝大多数的蛊仙都是人精中的人精,方源若不出这么高价,要将这鱼籽得到手是极难的。

“再到后世众星之主。五域大战爆发,这气泡鱼的鱼籽价格,还要上涨至如今的百倍!”

气泡鱼是南海特产。比较奇特,半透明的鱼肚滚圆,如一颗圆溜溜的气泡。气泡前端点缀着两个芝麻黑点,便是眼睛。鱼鳍、鱼尾都十分袖珍,提供微弱的动力。

气泡鱼和其他鱼类不同,它游动的方式多是上下沉浮。把气泡鱼逼得急了,为了躲避天敌,它们会浮出水面,浮上高空去。

一些成年的气泡鱼,甚至终年就浮在高空当中。吞食空中的微虫。不进入海水里生活。

气泡鱼最大的作用。是能够增产蛊虫。

气泡鱼将虫子吞食到鱼肚中去,这些虫子就会在里面得到保护和营养。渐渐演化成蛊,然后钻破鱼肚,飞到野外去。

正是因为可是使蛊的形成率提升,气泡鱼在五域大战中,成了各大蛊仙竞相追逐的物品。

“这些气泡鱼慢慢培养,虽然只有三成的成功率,但只要细心照料。数十年后,我就能自给自足。百年之后,我便能形成至少五十万的规模。和星萤蛊一样,绝对是五域大战时的紧俏物资。”

气泡鱼的鱼籽,是一个意外的收获,算得上一个小小的惊喜。

“有了气泡鱼的话,肥泥鱼就不能选了。虽然肥泥鱼最容易豢养,但这种鱼会吃掉气泡鱼的鱼籽,饿的时候甚至连泥土都能果腹。”

“而相比较而言,黄金锦鲤和青玉鲫鱼,一个吞食黄金,一个则以玉石为食。都对气泡鱼的鱼籽,没有任何兴趣。”

方源并不缺少黄金和玉石,因为狐仙福地中生活着大群的石人。

这些石人生活在洞底,以泥土为食,长久之后,会在身上凝结出各种金属、玉石等等矿物。

“但是黄金对气泡鱼生长繁衍有害,会减少敷衍率。而温润的玉石却是没有这个弊端。所以还是选择青玉鲫鱼罢。”

方源思索了一阵,最终决定了鱼种。

在五百年前世,他也经营过一片福地。正因为有着丰富的经验,曾经经历过许多的挫折和失误,让他如今做出的抉择,都显得十分明智。

有了青玉鲫鱼之后,水狼群的食物就有了着落。方源又收购了大批的花粉兔,以及铁壳花,用来作为夜狼、风狼、龟背狼的食物。

又因为毒须狼,方源又选择了一些地皮猪。这些猪繁衍极为容易,能吃肉也能吃草,放养在福地东部,也能生存。

最后,方源又收购了不少蛊虫,以及大量的炼蛊材料。一共花去他八块仙元石,数十万块元石。

“一下子,八块仙元石花出去了。”方源并不心疼普通的元石,但是却在意仙元石。

之前,他大肆贩卖记忆中的残方,获得了二十八块仙元石,如今只剩下二十块了。

“手中的残方虽然还可以再卖,但越卖越贱。我缺少获利的手段,关于气泡鱼、星萤蛊的投资,也要百年之后才能见到效益。接下来的这些仙元石,得精打细算着用了。”

方源叹了一口气,将通天蛊交给小狐仙,让她来处理狼群、鱼群的接收,铁壳花、气泡鱼籽的布置等等工作。

而方源则一头扎进荡魂行宫,利用刚刚收买的材料,开始炼蛊。

ps:电脑坏掉了,送去修理,今天才拿到手里。

\end{this_body}

