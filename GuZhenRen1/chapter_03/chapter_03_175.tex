\newsection{杀招完善}    %第一百七十五节:杀招完善

\begin{this_body}



%1
嗷呜!

%2
狼嚎声忽然响起,一大群龟背狼群在前方出现。

%3
“狼,怎么出现了狼群?”潘平动势一滞,表情诧异。

%4
常飚面色一沉,因为方源的关系,他现在最讨厌的动物就是狼了。

%5
但他万万料不到方源掌控这道关卡的情况,他沉声道:“我们这次是来试探为主的,现在情况有了新的变化,暂且先杀了这群狼再看看。”

%6
“嗯!”潘平点头应是。

%7
两人合力动手,杀入狼群。

%8
起先,两人占据上风,大杀特杀,普通的龟背狼群岂会是他们俩的对手?

%9
但很快,狼群绵绵不绝,出现了其他品种,诸如朱炎狼、水狼、风狼等等。又出现异兽狼群,如狂狼、白眼狼等等。

%10
两人渐渐吃不消了。

%11
“怎么会有如此多的狼群?”

%12
“难道这关,还要考验闯关蛊师的厮杀能力不成?”

%13
大量的千狼王、万狼王加入战场,潘平和常飚的脸色渐渐难看起来。

%14
“这关好难!”潘平感慨道。

%15
“八十八角真阳楼,关卡越往后越难,尤其是第九十关至最终的第一百关,最是艰难。”常飚应和一声。

%16
又战了片刻,两人支撑不住了。

%17
“情况已经探查清楚了,第九十道关卡不仅有迷宫,还有狼群!”常飚沉声总结道。

%18
“这两边都是墙壁,地形狭窄。对我们用人海战术局限太大。怎么办?”潘平皱起眉头。

%19
“先撤退再好好商议吧。”常飚叹息一声。

%20
他和潘平都没有巨阳血脉,进入八十八角真阳楼,用的是来客令。

%21
来客令珍贵。因此他们俩每一次进出,都花费甚多。

%22
“好!”潘平早有撤退之意,他恨恨地望了一眼面前的狼群,“这些该死的狼崽子,迟早有一天我要将你们的狼王,踩在脚底下,好好凌辱!哈哈哈……”

%23
他赌咒发誓。口中的“狼王”自然意有所指,指的方源。

%24
“呃!”下一刻,潘平的大笑声戛然而止。震恐的神情凝聚在他的脸上。

%25
“怎么回事?竟然出不去了?”身旁,常飚也发现了这个严重的问题。

%26
本来两人用来客令进入楼中,只要意念一动,便能出去。两人进出多次。早已经驾轻就熟。

%27
但现在方源掌控了此层。五角楼主令可比来客令大得多,他们俩因此宛若笼中之鸟,身陷绝境了。

%28
“该死,这下怎么办?我的真元只剩下三成了!”潘平大叫,声音中充满了惊慌。

%29
常飚脸色严峻,轻喝一声:“冷静!”

%30
他的情况比潘平好些,空窍中的真元还剩下一半呢。但真元哪怕处在完美状态,眼前的狼群绵绵不绝。早晚也会消耗光的。

%31
“这个情况很少见,八十八角真阳楼怎么会出不去?这道关卡相当古怪。很可能考验的是蛊师的胆量!千万不能胆怯!”常飚思索了一番,又叫道。

%32
潘平听了他的话,惊惶之情稍稍缓解了一些。他记得,在过往的历史中,的确有些古怪的关卡,考验的不是别的,正是蛊师的心境。这些关卡,往往蛊师越胆怯,面临的怪物威能就越强。

%33
潘常二人强制镇定,企图冲出狼群的包围。

%34
但方源掌控此关,所谓迷宫,在他心里一目了然,清楚每个角落。

%35
他调动狼群,轻松至极,不管潘常二人如何冲杀,总会不断有狼群杀过来围追堵截。

%36
“不,我绝不能死在这里!该死的狼崽子,看你大爷的杀招!”潘平真元消耗殆尽,逼不得已,开启了“六臂天尸王”的杀招。

%37
他化身成八臂僵尸,战力暴涨,所到之处掀起狂澜,众狼损失惨重,莫能抵御。

%38
常飚珍惜真元,跟在潘平的身后,省了许多力气。

%39
好景不长,很快潘平的真元就彻底消耗殆尽了。

%40
常飚连忙将其救下:“危难关头,你我只有同心协力,才能有逃生的希望。你休息,用元石恢复真元,我来保护你!”

%41
常飚也使出六臂天尸王的杀招,将潘平牢牢护住。

%42
就这样两人相互帮助,反而稳定了局面。

%43
如此,过去了七八天的时间,常飚、潘平身上的真元消耗殆尽,又支撑不住了。

%44
“难道我就要死在这里了?”潘平仰天怒吼。

%45
“可恶,一定有出路,一定有出路的!”常飚失去了往日的风度,大吼大叫着。

%46
就在两人绝望的时候,忽然看见前方拐角处,竟然堆了一大堆元石。

%47
“有元石!”

%48
“小山似的元石,这么多,我没看错吧?”

%49
两人绝处逢生,大喜过望,连忙奋起余勇,杀奔过去,用元石汲取真元,又稳住局面。

%50
“我懂了,我懂了,这关原来是考验蛊师的耐力!”常飚欣喜若狂地大叫起来。

%51
“原来如此。”潘平闻言,也恍然大悟。

%52
两人大喜之余,却没有意识到,自己身上出现的异状。

%53
随着催动“六臂天尸王”杀招的次数越来越多,他们的身上出现了无法恢复的尸斑。

%54
“这座元石小山,足够我们足足再支撑两三个月的。”潘平一头扑在元石小山上,感动得喜极而泣。

%55
“快加紧时间,恢复真元吧。我们不能坐吃山空,应该还有其他的元石小山。我们渴了喝狼血,饿了此狼肉,就这样坚持下去,说不定能打通此关。”常飚双眼精芒闪烁。

%56
“常兄,你说得太对了!”潘平猛地坐起来,常飚描述的情景让他脸上红光满面。“这关如此艰难,简直就九死一生。不知道打通此关,会有什么丰厚的奖励!”

%57
常飚一边抵御狼群的攻势。一边长叹一口气,道:“我现在终于知道,为什么此关只能进不能出了。一旦能轻易撤退,这关卡还怎么考验闯关的蛊师?”

%58
可怜两人还不知道,这元石小山,是他们的大仇人方源故意丢在这里的。目的就是想继续试验,让他们进行使用更多次的杀招。

%59
情况都在方源的掌握之中。就算两人不愿用杀招。使用本来手段对敌,方源也能操纵狼群冲锋,营造出艰险的局面。让两人不得不使出杀手锏。

%60
方源这方面的担忧是多余的。

%61
两人越用杀招,越是数量顺当,潜意识中渐渐产生了依赖情绪。在接下来的战斗中,很少用到自身的原本手段。

%62
待到他们身上尸斑浓郁。已经严重危难到他们身体时。两人这才惊觉。

%63
此刻,早已经晚了。

%64
“我不甘心,我不甘心,竟然死在这里!狼王,我做鬼都不会放过你的!!”

%65
潘平先死了。

%66
临时前,发出对方源的诅咒。

%67
几天后,常飚也累死在激战当中。

%68
他死不瞑目,濒死的时候。他口中喃喃:“出口,出口到底在哪里?”

%69
他有太多的心事。太多的放不下。

%70
杀狼同盟还只在草创阶段,绑架马英杰入伙的筹谋才刚刚开始,最让他放心不下的是他的亲生儿子常极右。

%71
其实,常飚也是个可怜人。

%72
为了名声,他至始至终也不敢认自己这个亲儿子。只能听到常极右一口一声地叫他“义父”、“义父”。

%73
就算这样,他还觉得不保险。为了遮掩,他还认了几个孤儿,充当义子、义女。这其中的一位,便是常丽。

%74
他苦心孤诣,潜伏一生,筹谋了这么多,临死时什么都看不到。

%75
他不甘,他悔恨,他懊恼,但又有什么办法呢?

%76
“真想听一听,他叫我一声父亲啊……”这股临死前的强烈心声,最终化作一腔悲愤和遗憾,随着他生命烛光的熄灭,而不甘逝去。

%77
“第两千三百一十一次催动杀招,虽然是累死的,但身上积蓄的尸气,却是致命的主要缘由啊。”凭空一闪,方源出现在常飚的尸体旁。

%78
他微微带笑,试验达到了预期的目的,发现了问题

%79
哪怕杀招每次使用,都不超过该有的时间限制,但使用频繁,会令身上产生尸斑,积累多了,便会引发蛊师死亡。

%80
不管是潘平,还是常飚的死,都是这个问题。

%81
方源将两人身上的蛊虫,都收入自家囊中。

%82
此关都在他的掌控之下,潘常二人就算想自爆蛊虫,也不可能。

%83
常飚的手中,是一套风道蛊虫,十分精良。潘平的蛊虫亦是不俗得很,但以方源目前的眼界和资本来讲,却都只是聊胜于无。

%84
唯有潘平的那枚单刀蛊,有些收藏的价值,方源为此多看了两眼。

%85
几天后。

%86
书房中,方源手握着一只东窗蛊,闭目冥思。

%87
“敌意蛊,可凝造敌意。敌意者,攻势强猛,侵略如火,但稍欠寰转,没有持续之力。”

%88
“锐意蛊,可凝造锐意。锐意者,犀利无当,仿若刀枪,然刚则易则,难以恢复补充。”

%89
“恣意蛊,可凝造恣意。恣意则,百无禁忌,最能泛滥,却难以稳控,动辄伤人伤己。”

%90
这只东窗蛊中,记载着一份较为完整的智道传承。方源从宝黄天中收购到手,付出了手中所剩不多的全部仙元石。

%91
但物有所值,方源阅览之后,受益匪浅。

%92
方源沉思:“这份传承中,记载了十四种智道蛊虫,其中涉及到意志方面的,有六种。分别是:敌意蛊、锐意蛊、恣意蛊、转意蛊、意冷蛊、意乱蛊。前三者的作用,都是凝造出意志,且各有利弊。”

%93
智道是最神秘的蛊师流派,博大精深,源自星宿仙尊,从太古时代就有了。尽管修行的人数极为稀少,但仍旧流传到了今天,经久不衰,万岁长青。

%94
“那么墨瑶意志,是属于敌意、锐意,还是恣意呢?”

%95
方源之所以研究这些,就是为了防备和对付脑海中的这个巨大隐患。

%96
“敌意如火狂猛,锐意如枪犀利,恣意张扬难控,但墨瑶意志却神秘如海,隐秘飘渺,我之前和她交锋,只觉得有力难施,仿佛举起猛拳打在棉花上面。”

%97
方源仔细分析,觉得墨瑶意志不是当中的任何一种。

%98
他收集到的这份传承,只是智道一角,肯定还有其他的大量的智道蛊虫。

%99
“看来,还是要继续收集这些智道的消息啊……”方源暗叹。

%100
墨瑶意志之所以不好对付,就是方源不熟悉智道,不知道她的底细跟脚,难以施展手段。

%101
“俗话说,知己知彼百战不殆。我暂且忍耐,一面另其不断思考,消损自身,一面利用她,帮助我探索真阳楼。”

%102
正思索之时,方源脑海中墨瑶意志浮现出隐约的身形。

%103
“改好了,这一次六臂天尸王,算是真正的完善了,再没有任何的不当之处。”墨瑶道,她语气虚弱,神情疲惫。

%104
“看来这些天的思考,让她损耗了不少。”方源暗喜,看了修改后的杀招后,这股暗喜渐渐变成了大喜!

\end{this_body}

