\newsection{蛮家挑战}    %第三十八节:蛮家挑战

\begin{this_body}

%1
时隔一个多月,葛谣的死终于被葛家人发现。

%2
消息传来,葛家许多人为之哭泣。葛谣平素虽然刁难,但是心性善良,是葛家的族花,人们都喜欢她。她在葛家有众多的追求者。

%3
“我的女儿啊,是为父害了你……”葛家老族长为之神伤,竟然一病不起。

%4
葛光便代理族长之位,其余家老协助。葛家上下一片哀伤之余,心情更是沉重。

%5
葛谣死了,带来的影响很大。蛮家族长的二子蛮多对葛谣一见钟情,一直在向葛家要人,现在葛谣死了,葛家从哪里交出人来?

%6
蛮家是大型部族,这些年来扩张得厉害,连续吞并了许多小型家族。几场胜仗,打得蛮家上下士气高涨,对落魄的葛家更加气势凛然,几次交涉不断地逼迫葛光。

%7
葛家是迁徙过来的家族,本来就没有蛮族庞大,又丧失家园,又想寄人篱下,在大风雪中依托红炎谷。因此葛光非常被动,可谓焦头烂额。

%8
……

%9
房间中,方源盘坐在床榻之上,双目微微睁开。

%10
在他的右手掌上,放着一只狼魂蛊。

%11
此蛊只有大拇指大小,仿佛一只狼形的灰色小布偶,默默地散发出幽蓝色的光辉。

%12
“这是第九只狼魂蛊了。”方源当即灌注真元,狼魂蛊顿时膨胀起来,几个呼吸间,化为一头灰白色的狼魂。

%13
狼魂张口,发出无声的呼啸,向方源的身体撞去。

%14
这一撞,悄无声息,但是在方源感觉,却是身心俱震。目眩神迷。

%15
狼魂直接冲撞上他的魂魄,原本人形的百人魂,顿时一阵翻腾,失去了人形,和狼魂纠缠,形成一阵滚滚的魂雾。

%16
魂雾并不飘散而出,在方源的身躯中翻腾不休。时而露出狼头、狼尾,时而又化为方源的形态。

%17
半盏茶后,魂雾一定。重新化为人魂。

%18
只是这人魂,又有变化。

%19
方源原本的百人魂,完全是他原先的相貌,耳鼻眼俱是一般无二。但用了九次狼魂蛊凝练之后,百人魂虽然大体上还是人形。但头顶上却长着一对狼耳朵,长发比现实中垂到腰际,眼睛也变成充满野性的狼瞳。整个身型更加削瘦,鼻子也高挺上去。

%20
原本他的百人魂,体形庞大,几乎要满溢出皮囊去。现在却是凝练很多,魂魄的颜色也从原先的苍白。变成深邃一点的灰白。

%21
方源估算了一下,等到他将手中的狼魂蛊全部用完,百人魂的凝练差不多就凝练到极限了。

%22
到那时,他整个魂魄都会变成半人半狼的形态。俗称狼人魂。

%23
狼人魂,比原先的百人魂要强大数倍。

%24
拥有了狼人魂之后,方源就可以再度壮魂,将百人魂提升到千人魂神。甚至万人魂。

%25
当然万人魂也绝不是终点,上面还有亿人魂等等。

%26
“从理论上而言。魂魄可以无限地变强。当年开创魂道的幽魂魔尊,他的魂魄就绝对超越了亿人魂!他的魔尊之魂有千臂千手,三个头颅,正面的头颅有龙角、狮鬃、蛇瞳、象牙,左边的头颅是桃额、草发、三眼如花,右边的头颅有云鬓、电眼、火耳、金口。强大到不可思议的地步,威能浩瀚无边。至今,这个形象,还被很多人的迷信和崇拜。南疆中就有许多凡人,用泥土捏成酷似的塑像,加以膜拜和祭奠。”

%27
幽魂魔尊的魂魄,显然是古往今来第一人。方源如今的狼人魂,和起相比,巨人脚下的一只蚂蚁,还需要不断地成长。

%28
稍微休息了一下,方源又取出十钧之力蛊。

%29
此蛊好像是一个铁秤砣,拿在手中相当的沉重。

%30
方源一共买了五只十钧之力蛊。如今已经用到第三只,本身的力气涨到二十钧,也就是六百斤力量。

%31
四转的十钧之力蛊,比同转的兽力蛊,在效果上要弱一些,但胜在可以叠加。

%32
方源之前,用过一只昆仑牛力蛊,得到了昆仑牛的兽力虚影。但是用第二只昆仑牛力蛊时,就没有效果了,不会再增加一头相同的兽力虚影。

%33
但钧力蛊却没有这个限制,可以不断地叠加,直到达到本身的极限。

%34
当然,方源虽然有六百斤力气,但平时发挥出来的,自然不是全部。

%35
力道的普遍弊端,钧力蛊仍旧存在。否则,霸仙楚度就不会被称为“力道的余晖”,而是“力道的崛起”了。

%36
当然,相同的兽力蛊之间也可以叠加气力。只是需要额外地搭配兽胎蛊。

%37
关于兽胎蛊的秘方,流传下来的不少。就算有,炼蛊的材料在如今也比较稀缺。炼蛊的代价太高,成功率比不上钧力蛊,因此逐渐就被淘汰。

%38
方源调动真元,灌注到钧力蛊中。钧力蛊便飞到他的头顶上空,绽放出一片玄光。光芒笼罩着方源全身,徐徐稳定地改造着他的身躯。

%39
正在这时,门外传来敲门声。

%40
随后,一道声音传来:“常山阴叔叔,小侄葛光求见。”

%41
方源引他进来,却见葛光灰头土脸,肩膀上还插着一根白骨羽箭,狼狈至极。

%42
见到方源,葛光扑通一声跪倒在地,通红着双眼,恳求道:“叔叔,还请您再救一救侄儿啊。”

%43
方源目光一闪,心中立即有所猜测,口中则问:“到底发生了什么事情?难道是蛮家大举进攻,要冲杀葛家的营地不成?”

%44
葛光便答:“叔叔猜对了一半,正是来自蛮家的大麻烦。那蛮家族长的二子蛮多,打听到家父病卧床榻,时常昏迷不醒,就带着蛮家一干悍将上门挑战,要我交出妹妹葛谣。但我妹妹已经丧生,我哪里能交得出人来?任凭我多么解释,蛮多那小子就是丝毫不信。按照草原上的规矩,上门挑战我们葛家不得不接。如今,他已经杀了我方的三家老,还打伤三人,连我也败下场。”

%45
方源心道果然如此,近日来蛮家逼迫日盛,可谓盛气凌人。葛家多有忍让,反而助长了蛮家的嚣张气焰。

%46
方源虽然几乎整天埋头苦修,但并非闭门造车,对于外界的形势也十分清楚。

%47
“话说回来,常山阴回到北原,也需要一个更大的舞台来亮相。不妨就借此机会,正式宣告昔日英雄的归来开元风流全文阅读。”

%48
想到这里,方源便扶起葛光,道:“我这些天住在这里,对葛家多有叨扰,自然不能袖手旁观,快带我去吧。”

%49
“叔叔,侄儿叩谢!”葛光大喜过望。

%50
两人连忙出去,还未到营门,就听到外面喝骂的声音。

%51
“葛家尽皆无胆鼠辈,快快出来受死!”这是一个少年郎的声音。

%52
“蛮多,你不要欺人太甚!”一位葛家的家老怒吼着。

%53
“呵呵呵,欺负你又怎样。豺狼捕捉猎物,鹰雕欺负鸟雀,这是天经地义的事情!快快交出葛谣,否则我就一直挑战下去,直到把你们葛家的人全部杀光为止。”

%54
“卑鄙!若是老族长,你们安敢如此?”葛家家老反驳道。

%55
蛮多大怒:“哼,你们才是卑鄙无耻,明明亲口答应的婚事,现在居然交不出人。言而无信!我知道,你们是把葛谣藏起来了,一直想要拖延。先前说是逃婚,这次居然说死了。你当我蛮多是傻子啊?小小的兔子居然敢耍弄虎狼,那就要付出生命的代价。石武,接着给我打,给我挑战。葛家,快快派出你们的人来上场。哈哈哈!”

%56
葛家家老尽皆脸色苍白,一时间面面相觑,无人敢上。

%57
石武五大三粗,顶着光头,一脸横肉,狞笑着走上场。

%58
他是三转巅峰蛊师,实力强劲。葛家牺牲的那位家老,就是被他当场打杀的。

%59
他在场地中央踱步,看着葛家偌大的营地:“怎么还没有人上场呢?你们该不会是怕了吧!”

%60
葛家人一阵羞恼,无数双喷火般的眼睛,瞪着石武。

%61
“无胆的孬种们,就让大爷我给你们增添一点勇气。这里有十万块元石,谁敢上来赢了我,本大爷就将这些元石统统送给他。”

%62
葛家无人应声。

%63
石武哈哈狂笑:“葛家,葛家,不过是一窝兔子和绵羊!”

%64
“你笑够了么。”由葛光在前面开道,方源一脸平淡地走出人群。

%65
石武笑声顿止,瞳孔一缩,惊愕地看着方源。

%66
“四转蛊师!此人是谁?”不止是石武,就是蛮多等一行人的心中,也同时冒出这个大大的问题。

%67
“葛家居然还有一位四转蛊师隐藏着?”蛮多顿觉不妙。

%68
他此次前来找麻烦,就是趁着葛家老族长病重卧床的良机。但是没想到,葛家居然还有第二位四转战力!

%69
“我来之前,明明已经查探清楚了。这是从哪里冒出来的高手?”

%70
怀着这样的疑惑,蛮多从马背上跳到地面,他换了一个态度,右手抚心,对方源庄重一礼:“这位朋友,你不是葛家的人,何必来趟这趟浑水呢?”

%71
方源打量了一眼蛮多,觉得这少年有些意思。

%72
他是从葛谣的叙述中,第一次得知蛮多的存在。

%73
蛮多虽是蛮家族长的三子,但从小就体弱多病,资质不佳,只有丙等。他如今已经二十多岁,却只有二转修为,又黑又瘦,果然像个猴子。

%74
但他却非葛谣说的那样不堪,他有一双精明狡诈的小眼睛,里面燃烧着的是野心的火焰。

\end{this_body}

