\newsection{人生为何早有悟(六千二大章 )}    %第一百六十二节:人生为何早有悟(六千二大章 )

\begin{this_body}

方源看着碗壁上的墨文,微微咧嘴。

这招灾蛊,品级高达七转,比方源如今的春秋蝉还要高一个档次。效用极端奇妙,居然是涉及地灾天劫。

墨文中段,有详细阐述。招灾蛊能够替蛊仙招灾引难,将地灾天劫脱离原来的目标,勾连到自己的身上。

方源神情不免古怪起来。

这样的仙蛊,谁敢用?

天灾地劫是多么麻烦的毁灭力量,多么恐怖的天地伟力。好好的日子不过,用招灾蛊吸引来这些天劫地灾,这不纯粹自己找死么?

蛊仙墨瑶,堂堂的灵缘斋三十六代仙子,为什么要炼制这样的仙蛊?

墨文的后段,给予了理由。

原来她是真的想要找死!

当初,她和薄青相恋相知,成为正道伉俪,五域瞩目的传奇。

剑仙薄青天才卓绝,打遍天下无有抗手,人称“剑劈五洲亚仙尊,为情所系幸苍生”。

纵观天下,审视人生,摆在他面前的,只有冲击九转这个无上的目标了。

然而冲击九转,艰险无比,纵然是堂堂剑仙,也感到如履薄冰,把握至多只有一成半。

薄青志向高远,矢志冲击九转巅峰。墨瑶苦劝不成,只能含泪协助准备。

薄青布下传承,安排后事,第一次冲击失败,身受重伤七十载,躺在病榻上不能动弹丝毫,皆是墨瑶照顾起居。

伤好之后,薄青意欲再度冲击九转。

墨瑶心知此事艰难,尤其是冲击九转的最后关卡,有无边的地灾天劫降下。薄青纵然战力惊绝天下,却缺欠持久之力。

她为了一心帮助情郎,暗中背叛门派,将主意打到八十八角真阳楼的身上。

八十八角真阳楼乃是大名鼎鼎的仙尊布置,长毛老祖手笔,天下第一的仙蛊屋,墨瑶很早之前,就开始研究它,从它身上钻研出心得,帮助自己加深炼道宗师的造诣。

她虽然只是异人,没有巨阳血脉,但灵缘斋在巨阳时代,就供奉了许多杰出女子,成为巨阳仙尊的妃嫔。其中更有数位女蛊仙,得到过巨阳仙尊的欢心。

因此,灵缘斋中保存着许多巨阳仙尊的隐秘,其中对八十八角真阳楼,更是知之甚详。

墨瑶从这些资料中,得知八十八角真阳楼的秘密。

王庭福地位居北原正中央,内里天地广阔,又有黑白两天之分,作用八十八角真阳楼,福分极深。因此每隔一段时间,就会引来极其猛烈的地灾天劫。

巨阳仙尊考虑到这点,在组建八十八角真阳楼时,放进一只“排难蛊”,当做八十八角真阳楼的重要基石。

此蛊高达七转,乃是巨阳仙尊运道精髓之一,能够将王庭福地中的天劫地灾,都排遣到外界去。如此一来,便形成扩散整个北原的十年暴风雪灾。

反过来,巨阳仙尊又借助十年暴风雪灾,订下规矩,形成王庭之争的传统。

墨瑶在这一点上,发现了一个不是漏洞的漏洞。

原来王庭福地与八十八角真阳楼共生,每隔十年,都会引来强大浩瀚的天灾地劫。这时就需要王庭福地敞开一丝缝隙,配合排难蛊,将灾劫排遣出去。

王庭福地被巨阳仙尊布置,只能出入凡人,不允许蛊仙进出。但排难时,王庭福地打开一丝隐秘缝隙,洪水般的灾劫排放出去,仙尊隔绝蛊仙进出的布置就起不了作用。

墨瑶便是借助这丝缝隙,冒着九死一生的危险,逆着灾劫进入到王庭福地。

她在当中考察近十年,凭借宗师造诣,费尽千辛万苦,找到关键的节点――正是地丘处的小塔楼。

随后,她冒着惊醒巨阳意志的危险,将小塔楼摧毁,利用伟力回流,感应排难蛊,形成酿造“招灾蛊”的地洞。

成功地炼成仙蛊雏形后,她又在这处无名山谷中,随手布下仙蛊屋近水楼台。

在其中,墨瑶将雏形彻底温养成形,便带着招灾蛊,利用福地排难的良机,偷偷潜回外界。

她没有将这些布置毁掉,是为了以防万一。若是薄青再次冲击失败,招灾蛊也毁于天劫地灾,那么她还会再进来福地,再度炼成招灾蛊。

然而这一次出去,她就再也没有回来。

剑仙薄青第二次冲击九转,彻底失败,在浩荡天劫中化为灰灰。而墨瑶也随之一同身陨。

“原来如此。冲击九转,千难万险,墨瑶在此之前,偷偷设下了这道传承。她没有点明传承的真正内容,是因为这个行为,本身是对门派的背叛。但她终究还是将前置的线索,留给了门派。前世五百年后,中洲蛊仙图谋王庭福地,摧毁八十八角真阳楼,就是借助了她的这个前置线索。”方源现在回想,顿时有一种迷雾消散,一切都说通了的感觉。

墨文的最后,有一首诗――

仙路阻且长,相逢于天涯。

岁月忽已晚,情仇已绵长。

君思仙尽头,我思君安危。

生死两茫茫,为君梦还乡。

显然,这是墨瑶所著。

蛊仙之路,漫长艰难,却和君有幸相逢。不知不觉间,情恨纠缠,无法自拔。

君的目光,盯在仙路的尽头。我的目光则落在君的身上。

冲击九转,九死一生,我不愿生死相隔,只愿为君倾尽一生,助君圆梦!

为了保护心中的爱郎,为了助推薄青走上蛊仙的巅峰,墨瑶宁愿牺牲自己,利用招灾蛊,将天灾地劫引到自己的身上。

“真是奇女子……”方源叹息。

尽管他绝不会因为爱情,做出这样的牺牲,但却不妨碍方源理解这样的人。

甚至,这种了解程度,反而比旁人更深。

人生活在这个世界上,总有**,总有目标,总有意义。

墨瑶的目标,是为心中的爱郎。而方源的目标,在于追求永生。

**不同,目标不同,意义不同,铸造了形形色色的天下苍生,特立独行的英雄枭杰。

方源将目光重新落在朱红巨碗中。

蚕茧已经破开,招灾蛊已经彻底成形。它形如蚕茧,浑身灰白,只有小拇指大小,在水里载沉载浮。

招灾蛊乃是货真价实的自我牺牲之蛊,墨瑶用之舍己为人。

对于方源来讲,这蛊看似无用,其实仍旧有巨大价值。

首先,它是仙蛊。就算自己不用,放入宝黄天中贩卖,也能换得一笔巨额仙元石。

其次,它乃运道蛊虫,能吸引天灾地劫。灾劫虽强,但只要自身过硬,能够撑住,那么用来陷害他人却是别有妙用。

最后,它的成形乃是借助八十八角真阳楼中的基石之一――排难蛊。它和排难蛊可谓一体两面,一个招灾,一个排难。单凭这层关系,对方源接下来的八十八角真阳楼之行,将有巨大帮助。

然而心中虽然有一层羁绊,但想要彻底收服此蛊,还有一层关隘。

碗壁墨文中,墨瑶就有详细关照。

要收服招灾蛊,需要一定的条件。虽然不计仙凡身份,但却需要有缘人有一颗自我牺牲的心。

如果没有这颗心的话,强行收服,轻则导致仙蛊反噬,重则仙蛊自毁,殃及人身性命。

近水楼台中,方源站在巨碗之前,面无表情。

自我牺牲的心,他有么?

……

“咳咳咳。”唐妙鸣用手帕捂住嘴唇,秀眉紧皱,表情痛楚。

“大姐!”一旁的唐家三少唐方,呼唤一声,脸上充斥痛惜之情。

唐妙鸣半躺在床上,摆摆手,示意唐方不要担心。

唐方看着大姐手帕上的鲜血,深深地叹了一口气:“大姐,你何必这样拼命?闯了这关又能如何?父亲他们都已经去了,大姐你就是我唯一的亲人,你要是有个三长两短,我该怎么办啊?”

唐妙鸣伸出手来,轻抚唐方的头发:“三弟,你可是我们唐家的族长了,可别这么没有志气。这一次王庭之争,我们唐家损失惨重,险些被他族吞并。如今是千载难逢的机会,正要借助八十八角真阳楼,来令我们部族重新强盛起来。”

唐方不以为然地撇嘴道:“但是大姐,你强行闯关,造成重伤,得不偿失啊。三弟我看着分外心疼,这些天来连家族的事务都无心打理了。”

“什么?”唐妙鸣听了这话,脸色一肃,凌厉的目光紧紧盯住唐方。

她毫不犹豫地训斥道:“唐方族长,你身兼重任,岂能如此儿女情长?振兴部族,乃是你的职责,是一族之长的意义所在。今后,我不想再听到相似的抱怨,你听明白了么?”

“大、大姐,我错了。”唐方连忙从床边站起来,一脸惭愧,低头认错。从小到大,就属大姐对他最好。

唐妙鸣目光渐渐温柔下来,幽幽一叹:“三弟,我知道你性情跳脱,喜欢过浪迹天涯,无拘无束的生活。但是你身为家里最后的男人,就应该勇于承担。今后你人生的意义,就在于振兴部族,你可明白?”

“大姐教训的是,小弟明白了。你别生气了,本来身上就有伤势呢。”

唐妙鸣仍旧严肃:“回去之后,将《人祖传》的第三章第一节,连夜抄写十遍给我。”

唐方的心中,顿时充斥着一股温情。

从小到大,大姐的惩罚,就是让他抄书。

“大姐,你休息吧,我这就去抄。”

《人祖传》第三章第一节有载――

人祖的二女儿古月阴荒,为了将自己的父亲从生死门中解救出来,踏上成败山,寻找成功蛊。

但最后关头,她失败了,丧失了自我,成为了一个丑陋而又强大的怪物。

没有女儿的搭救,大儿子太日阳莽更是颓废沉迷,人祖困于落魄谷中,不能生还。

落魄谷如同一个大迷宫,曲折蜿蜒。时而蔓延出茫茫一片的迷惘雾,能令魂魄松散。时而刮起凛冽如刃的落魄风,专门切割魂魄。

人祖乃是魂体,在迷惘雾中寻不到出去的方向,落魄风则切割他的魂魄,令他越来越虚弱,处境也越发危险。

被落魄风切割下来的魂魄碎片,渐渐凝合起来,成为一个少年。

就这样,人祖的第三子诞生了。

他就是北冥冰魄。

“我的儿子啊,谢谢你的陪伴。我的时间不多了,在最后的日子里,因为有你的陪伴,父亲我一点都不寂寞。”人祖感叹道。

北冥冰魄外冷心热,虽然话不多,但对人祖十分孝顺。

看着人祖一天天虚弱下去,他的心情也越发沉重。

他决定把人祖救出去。

人祖感受他的决意,既欣慰又痛惜:“不要忙活了,我的儿子,你孝心我心领了。我现在明白了,生死强求不得。人总归是要死的,这就是人的宿命。”

北冥冰魄哭泣道:“父亲,我知道你说的话是对的。我也知道,我的努力会白费。但是看着你这样虚弱下去,我不做出努力,我的心里将会更加难受。就让我为您做些什么吧。”

人祖叹息一声,只能任由他去了。

北冥冰魄游荡在落魄谷中,他在这里诞生,落魄风不能削他的魂魄,**雾更不能遮挡他的视线。

他苦苦搜寻,但始终找不到出去的路。

就在他越来越绝望的时候,他碰到了一只蛊虫。

“哎呀呀,想不到居然被你发现了。”这只蛊虫形如瓢虫,肥胖若球,但动作敏捷无比,四处闪现在北冥冰魄的身边。

北冥冰魄眼睛亮起来,好奇地问道:“你是什么蛊?”

“我的名字,叫做意外。”这只蛊虫答道。

北冥冰魄目光黯淡下去:“原来你是意外蛊啊,可惜你不是成功蛊。”

意外蛊嗤笑一声:“年轻人,你莫要小看我。我可是令成功蛊又爱又恨的存在。意外的力量是很强大的。你知道你在这里遇见我,代表着什么吗?”

“什么?”

意外蛊摇晃着肥胖的身体,得意地道:“这里是什么地方?这里是落魄谷,是死境。你在这里,说明你已经死了。但你遇到了我,就是在‘死’中遇见了意外。那就是――‘生’了。抓紧我吧,我带你回到人间,令你重新复活。”

“真的吗?”北冥冰魄大喜,“能不能带上我的父亲一起呢?”

意外蛊摇头:“是你遇见了我,不是你的父亲,所以只能带你走。”

北冥冰魄失望极了,他拒绝道:“既然不能带上我的父亲,那我也不走了。我要陪伴我的父亲,直到最后的时刻。”

意外蛊大笑三声,用霸道的语气道:“人生的意外,可由不得你拒绝。年轻人,你必须得跟我走!”

话音刚落,意外蛊便强行带着北冥冰魄,瞬间离开了生死门,回到了人间。

北冥冰魄拥有了鲜活的血肉之躯,独自一人面临偌大的世界,感到分外的迷茫。

意外蛊消失了,他忽然记起人祖曾经说过的话,想起来他还有一个二姐,叫做古月阴荒。

这时候,思想蛊主动找到了他:“年轻人,你不要怀疑我,思想一向是人的朋友,我来是帮助你的。”

思想蛊告诉北冥冰魄,有关成败山,以及古月阴荒的事情。

北冥冰魄决定先见见自己的二姐。

当他看到到古月阴荒时,他难过地流下了泪水。

北冥冰魄企图和古月阴荒交流。但变成怪物的古月阴荒,一直在嘴里念叨着问题。

“这是哪里?”

北冥冰魄思考了一下,答道:“这是人间,生命可以在这里活动。我们的头顶上是天,我们的脚底下是地。”

“我是谁?”古月阴荒又问。

“你是人,人祖的二女儿,名字叫做古月阴荒。你是我的二姐啊。”北冥冰魄答道。

“二姐啊,你赶紧清醒过来吧。我们的父亲死了,被困在落魄谷里,我们得赶紧去救活他呢。”

“人祖?古月阴荒?救活?”怪物摇晃着脑袋,困惑无比,“我为什么要救活他?人难道不应该死吗?死亡有什么不好?人为什么活着?我为什么活着?”

这一次,北冥冰魄答不上来了。

人为什么活着?

北冥冰魄思考这个问题时,困惑蛊就悄悄地来到了他的身边,让他失去了对周围的感应。

随之一同的,还有爱情蛊、伪装蛊。

思想蛊看到它们,顿时头疼无比。这几只蛊,出了名的调皮捣蛋,经常结伴而行,就算是思想蛊也不想去招惹它们。

“爱情,你害的人还不够吗?为什么还不放过人呢?”思想蛊叹息道。

“别跟我讲道理,我就是蛮不讲理的。”爱情蛊语气刁蛮,“快滚吧,思想,我不待见你。”

思想蛊无奈,只能退走。

“又来了一个人吗?哈哈!”爱情蛊看到北冥冰魄,十分开心,因为又有了玩弄的对象。

它和伪装蛊是铁杆哥们,当即借助它的力量,伪装成思想蛊。

“年轻人啊,你的二姐把自己都忘了。你要想拯救她,就得寻找到意义蛊。”爱情蛊道。

北冥冰魄回过神来,不疑有他,对爱情蛊问道:“我倒是见过意外蛊的,请教你,这只意义蛊在哪里呢?我该如何找到它?”

爱情蛊用郑重其事的语气,哄骗他道:“人啊,你要知道,你们生活在这个世界上,都是有意义的。你只要寻找到意义蛊,就能让你的二姐清醒过来。你顺着我指的方向,一直走,一直走,你就能找到意义蛊了。”

北冥冰魄表示万分的感谢,立即上路了。

爱情、困惑、伪装三蛊望着他远去的背影,都哈哈大笑起来。

这个世界上,哪有什么意义蛊?

根本就没有这只蛊虫,北冥冰魄怎么找,也是找不到的。

“傻子,谁叫你们触怒我呢?我要让你们知道,爱情的惩罚,可是极为恐怖的!接下来,我们就一直跟着他,轮流玩弄他吧。”

爱情蛊的提议,得到了其他两蛊的认同。

就这样,北冥冰魄轮流受到三蛊的戏弄,苦不堪言。但他为了寻找到子虚乌有的意义蛊,仍旧坚持不懈。

这份精神,感动了思想蛊。

趁着爱情蛊不在的时候,思想蛊来到北冥冰魄的身边,要再度帮助他。

“思想,你来干什么?我们玩得正开心呢。”困惑蛊、伪装蛊十分排斥思想。

思想蛊笑起来:“我惧怕爱情,但我可不怕你们两个。年轻人啊,借助我的力量,清醒过来吧。”

北冥冰魄便借助思想蛊的能力,认清了真相,不再困惑,识破了伪装。

困惑蛊、伪装蛊只能败走。

北冥冰魄向思想蛊表示感谢,道:“谢谢你,思想蛊,因为有你,我想到了拯救二姐的方法。”

“哦?那是什么方法?”

“这个世界上,的确不存在意义蛊。但我可以创造出一个意义蛊来。”北冥冰魄自信地道。

人活着是没有意义的,但是却可以赋予一个意义。

北冥冰魄回到古月阴荒的身边,亲手创造出了一只意义蛊,按进古月阴荒的脑海当中。

“我活着,就是为了搜寻到成功蛊,救活父亲!我明白了,我懂了!”古月阴荒的双眼骤亮起来。

……

“活着的意义么……”唐方将手中的笔放下。

夜已经深了,在这王庭福地中,温柔的银辉覆盖大地。

连续的抄书,令他心生感慨。

“人活在这个世界上,总会感到迷茫。但只要找到自己人生的意义,就会找到方向,勇往直前。同时,也就明白自己想要什么,不想要什么,就会不惧牺牲。大姐要我抄书的用意,恐怕就在于此吧。”

他轻轻地推开窗扉,看着眼前辉煌锦绣的圣宫,想着形形色色的人物,有强者,有弱小。

他的心绪渐渐激昂起来:“每个人的生命,都有各种各样的意义。而我的意义,就是带领着部族,走向昌盛!”

……

与此同时,近水楼台。

“自我牺牲之心?”方源的嘴角泛起一丝傲然的冷笑。

他伸出右手,没有丝毫犹豫,伸入巨碗,将里面的招灾蛊直接拈了出来。

汲取着他的气息,招灾蛊身上光辉一闪,旋即便成为方源之物。整个过程顺利无比,没有丝毫震荡和反噬。

穿越者的身份,前世五百年的经历,早已让他看破了生死,什么亲情友情爱情,更都不是他的兴趣。

只有永生,这样崇高到遥不可及的目标,才能让他的生命之旅,显现出浓烈的趣味来。

这就是他赋予此生的意义!

但是追逐永生,并不是说他怕死,怕失败。

对于死亡、失败,他坦然接受。

甚至,永生究竟存不存在,都没有任何证据能够证明。

但是就算是不存在又如何呢?

方源享受这样的过程。他在追逐永生的过程中,寻找到了意义,感受到了此生的趣味盎然。

身体上的低级**,爱恨情仇的满足,他早就腻味了。

只有永生,才是值得追求的目标。

“因此,牺牲的觉悟,我早就有了啊。”方源目光幽冷,把玩着手中的七转仙蛊。

ps:感谢诸君的耐心等待,感谢“傲骨丶临风”同学对我的理解。什么都不说了,这一大章双手奉上!(未完待续。)

∷更新快∷∷纯文字∷

------------

\end{this_body}

