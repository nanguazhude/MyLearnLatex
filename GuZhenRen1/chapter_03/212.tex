\newsection{鸿运之因}    %第二百一十二节:鸿运之因

\begin{this_body}

%1
轰轰轰!

%2
六头高达三丈的巨大树人,高举巨槌般的拳头,一击接着一击,狠狠地砸在光罩上。

%3
霜玉孔雀悲鸣声声。

%4
宽大的羽翼,已经被黄金死死凝固封印,地灵只剩下头部还裸露在外。

%5
“两位主人,我要支持不住了。”地灵叹息一声,修长的睫毛阵阵颤抖。光罩越来越虚弱,在树人的围攻下岌岌可危,已然濒临破碎。

%6
更恐怖的是,方源已经快速逼近!

%7
王庭福地已经千疮孔,霜玉地灵又被封印,剩下的力量对方源而言毫不足虑,分秒可破。

%8
致命的杀劫已然来到,马鸿运赵怜云却毫不知情。

%9
就算察觉不到方源,浓郁的死亡气息也厚重得叫人喘不过气来。  首发 蛊真人212

%10
赵怜云沉默不语,依靠在马鸿运的怀中,闭着双眼。

%11
本质上,她终究是个女人。

%12
马鸿运不断低声安慰她,尽管他自己也毫无希望。

%13
“结束了。”方源身影电射,脚踩枝叶,进入树人包围的最内层。

%14
但正当他要痛施杀手,俯冲而下之际,脑海中墨瑶忽然惊叫一声:“嗯?这股气息!等一等……快住手!”

%15
“怎么?”方源动势一滞,停在树梢高处,不免惊疑。

%16
自墨瑶意志寄托在他脑海中以来,他还从未见过墨瑶有如此剧烈的情绪波动!

%17
此刻,方源的脑海中,墨瑶显现出身形。

%18
她竟然泪流满面,娇躯都在不断颤抖,口中不停自语:“这股气息,这股气息……没有错。绝对没有错!这是鸿运齐天蛊的气息啊!”

%19
“鸿运齐天蛊?!”方源心头一动,旋即回想起来。

%20
墨瑶前为了爱郎薄青,拼尽心血研究八十八角真阳楼,冒险深入真传秘境。

%21
她的原先目的。就是要谋夺无上真传之一的运道真传。想要盗取当中的鸿运齐天蛊。

%22
然而最终,她只成功了一部分。

%23
她虽然费尽千辛万苦。拼尽全力,将运道真传打碎一道裂缝。

%24
但鸿运齐天蛊,却是顺着这道裂缝飞走了,消失无踪。  首发 蛊真人212

%25
不过墨瑶也不是毫无所获。她得到了一小部分运道的传承内容。再结合自身炼道宗师的底蕴,她最终炼成了招灾蛊。

%26
但招灾蛊到底不如鸿运齐天蛊,墨瑶拿着招灾蛊帮助薄青渡劫,结果双双陨落。

%27
而方源则是找到了墨瑶留下来的传承,重新炼出招灾蛊,接受了将墨瑶的遗命任务,要在未来将七转仙蛊屋近水楼台归还给灵缘斋。

%28
现在墨瑶忽然感受到了鸿运齐天蛊的气息。这是她前拼尽全力,穷尽智慧,却仍旧谋夺失败的关键仙蛊。

%29
她如此失态、激动,也就不难理解了。

%30
“鸿运齐天蛊的气息。就在这个小子的身上!想不到啊,我墨瑶拼尽全力,九死一,结果放走了鸿运齐天蛊,结果便宜了一个凡人小子。呵呵呵,这就是命吗?”墨瑶仰天长叹,哭笑着,语气萧索至极。

%31
方源眯起双眼,眼中寒芒烁烁。

%32
他在心中发问:“这么说鸿运齐天蛊就在马鸿运的身上?你能确定?!”

%33
“当然,鸿运齐天蛊的气息,我是绝对不会忘了的!”墨瑶的回答,斩钉截铁。

%34
她接着又道:“鸿运齐天蛊,高达八转,乃是一次性的消耗蛊。它无形无质,本就是一团恢弘气运,用寻常手段万难捕捉。当年我本体就是没有捉住它,令其鸿飞冥冥,消失无踪。就算是真阳楼,也没有困住它。想来它一定是逃到北原,不断流浪,最终落在那个小子的身上了!”

%35
“原来只是一次性的消耗蛊……”方源心中顿时大为失落。

%36
鸿运齐天蛊,乃是巨阳仙尊运道真传的最高成就,价值极其巨大。

%37
方源无法在真传秘境中捞好处,如果能从马鸿运手中夺到鸿运齐天蛊,那就最好不过了。

%38
但可惜的是,鸿运齐天蛊是一次性的蛊,用了就消失了。

%39
“方源,你要小心!鸿运齐天蛊玄妙非凡,当年巨阳仙尊奇遇连连,次次逢凶化吉,福运不绝,很大程度上就是依靠鸿运齐天蛊。马鸿运用了鸿运齐天蛊,就是货真价实的气运之子!身上运气,堪称五域第一!!当年巨阳仙尊如何运气,他就有相同程度的运气。运气虽然无形无质,但也是实力的一部分,你可千万不能小看它!”墨瑶急切地关照道。

%40
“我知道分寸。”方源按捺杀机,目光冰冷,停在树枝上,没有冒然出手。

%41
他居高临下,俯瞰下方的马鸿运等人,任由脚下的树人发动攻击。

%42
其实,双方距离并不远,但就连霜玉孔雀都没有发现他。

%43
由此可见,地灵虚弱到了何等严重的程度。

%44
双方实力强弱,一目了然,看上去胜利果实唾手可得,斩杀马鸿运易如反掌。

%45
但方源不这么看,他曾是蛊仙,眼界自然超脱凡人。

%46
他清楚:所谓蛊师流派——智道、力道、炼道、金道、宙道等等,看似是一道道复杂精细的蛊虫系列的集合,但本质上而言,则是对天地自然的理解,是对乾坤奥秘的解析。

%47
运道也同样如此。

%48
因为巨阳仙尊的例子,运道甚至比其他流派,更能深层次地解释天地奥秘的某个方面。

%49
方源从未有小看过马鸿运,一直对他重视有加。

%50
因为前世记忆,他很明白,马鸿运今后的成就能有多高。

%51
在前世,他最终成为了北原的顶梁柱,抵抗中洲蛊仙进攻的领袖人物!方源被人围杀的时候,他仍旧在茁壮成长着。

%52
方源清楚地知道马鸿运的成长轨迹,现在他终于明白,马鸿运这身逆天气运的来源。

%53
那就是巨阳仙尊的运道最高成就——鸿运齐天蛊!

%54
而始作俑者,则是居于方源脑海中的墨瑶!

%55
运气虽然没有实质的形体,也摸不着,但方源却清楚这当中的玄妙威能。

%56
“难怪巨阳意志,为地灵认主,专门挑选了黑楼兰和马鸿运。黑楼兰是血脉后裔中最强之人,而马鸿运身上则有齐天鸿运。”

%57
方源感到相当棘手。

%58
马鸿运虽然只是小小的三转蛊师,但身上却有八转仙蛊的威能护身!

%59
八转仙蛊,仅次于传说中的九转智慧蛊、力量蛊等等。威能非凡,超乎想象,胆敢小觑必下场凄凉。

%60
方源只是个凡人,当初在真传秘境当中,连七转的人气仙蛊都不能接近。何况八转的鸿运齐天蛊?

%61
马鸿运看似人畜无害,但实际上一路走来,何曾吃过一次真正的亏?每一次都是因祸得福,奇遇不断,想伤害他对付他的人,反而都遭受了不测。

%62
虽然每一次,几乎都不是他正面出手。但总是会出现其他外力,或者各种意外,还有无数巧合,这些外力、巧合、意外叠加在一起,就形成惊人的威力!

%63
方源忽然想到巨阳意志。

%64
之前,巨阳意志要铲除赵怜云,结果马鸿运舍命力保。这就等若巨阳意志站在马鸿运的对立面上。

%65
就在巨阳意志要动手之际,结果被方源动手,抽出了真阳楼,现在正被天劫轰炸着。

%66
“就连巨阳意志,都惨遭毒手了么……”方源转换一个角度思考,顿时悚然一惊。

%67
“我拥有琉璃楼主令,可以侦察八十八角真阳楼里的情形。之前为什么会动手?其实完全是无意的!或许……听到赵怜云乃是天外之魔,有想要保下她,去探究一番的好奇?”方源回想当时的情况。

%68
巨阳意志被排出真阳楼,可以说完全是他一手造成的。但时机偏巧,令赵怜云、马鸿运双双逃脱一劫。

%69
“我现在向马鸿运动手,会不会也和巨阳意志一样,被某个外力介入,或者某个意外忽然发?”

%70
方源捏了捏手中的琉璃楼主令,顾虑重。

%71
“等一等!”方源又转念一想,“我听到的,其实都是墨瑶的一面之词。墨瑶意志似乎可信,但万一她是因为某种特殊的原因,欺骗我呢?”

%72
方源忽然想到了这点。

%73
于是他在心中,向墨瑶问道:“奇怪,之前我动用察运蛊,观察了圣宫上下几乎所有人。若是马鸿运身上有齐天鸿运,我怎么没有观察到?我只看到最有气运的两个人,一个是黑楼兰,一个是太白云。”

%74
墨瑶似乎早料到方源有此问题,立即答道:“这有什么奇怪的?鸿运齐天蛊乃是八转仙蛊,你的察运蛊不过是五转的凡蛊,是一只侦察蛊,令蛊师能看到肉眼不可见的气运。五转的侦察蛊,怎么能侦察到八转仙蛊的气运呢?”

%75
“既然看不到,那我怎么确信,马鸿运的身上就有鸿运齐天蛊?”方源并不罢休,继续追问。

%76
“方源啊,你真是太多疑了。”墨瑶叹息一声,旋即道,“有一个方法可以确认。那就是用侦察蛊,观察你自身的气运。每当你对气运之子马鸿运不利时,因为气运的相互影响,你的气运将会变得更糟,更坏。你如果不信,完全可以这样查看。”

%77
时间有限,方源立即催动察运蛊。

%78
他的视野,顿时发了变化。

%79
马鸿运、赵怜云的气运,只是一小股,和常人无异。

%80
而一股庞大的黑色气运,高达近丈,竟有贯穿天地之势!

\end{this_body}

