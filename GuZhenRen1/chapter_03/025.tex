\newsection{白骨战车}    %第二十五节:白骨战车

\begin{this_body}

足足一个半的时辰之后。

方源抱着葛谣,安稳地落在地上。

葛谣双腿发软,面色苍白,踩到实地上喘息了片刻之后,才渐渐回复过来。

就在刚刚的飞行中,她至少杀死了上百头影鸦,好几次都感觉自己在生死一线上挣扎。

飞行的过程,充满了危险,不仅遇到了好几波影鸦群的追杀,而且还引发了地刺鼠的攻击。

凶险的时候,不仅天空中密密麻麻的影鸦,地面上的还会往上暴射无数的地刺。

方源就在这样的围攻之下,振翅飞翔,不断转折、回旋、俯冲、拔升。在不可能中寻找可能,从紧密如雨的攻击中,冲出一线生机。

“我居然能够在这样的围攻下,活了下来?”这是葛谣双脚踏地时,心中冒出的第一个念头。

当劫后重生的庆幸,难以置信的欢喜,回想的后怕等等情绪消退之后,葛谣神情复杂地看向方源,后者正在治疗身上的伤势。

就算少女再没有见识,经此一役,也会认识到方源在飞行上的厉害。

更何况葛谣,并不是普通的草原少女,她的父亲是一族之长。在过往的生活中,葛谣耳濡目染,眼界开阔,知道的东西比同龄人要多得多。

“眼前的这个男人,他的飞行术竟然如此娴熟,如此强大。他藏在雄壮身躯里的,是飞鹰的灵魂么?这样的飞行术,完全可以媲美飞电东破空、水仙宋清吟、青蝠邬夜!这是北原第一流的飞行术!常山阴、常山阴。你究竟是怎样的一个人……”

方源迅速地将伤势打理好。

在宛如暴雨的攻势中,就算他飞行术再高超。也要不可避免的中招。

关键还是四转的骨翼蛊,在北原只相当于三转蛊的效用。同时他又抱着一个人,重量大增,又影响了灵活性。

“不过,有葛谣的帮助,还是利大于弊的。如果不是她用水箭、水龙击杀鸦群,不是她用水甲帮着防护,单靠我的真元必然是不够用的。”方源心思着。

“自己是南疆蛊师。到了北原,修为上就会受到压制,不过好在时间越长,自己渐渐融入北原,这种压制就会越少了。”

人是万物之灵,能适应环境,融入环境。

当然。当方源彻底融入进北原,修为彻底恢复后,再回到南疆时,他就会再次受到压制,需要再次的适应和融和。

“人可以渐渐融入环境,但是蛊虫却不能。南疆的蛊虫。会一直受到削弱。我身上的这些蛊,催动时消耗的真元量和以前一样,但是效用却大大削减。南疆的四转蛊,反而不如北原的三转蛊。”

但是杀了葛谣,她一身的蛊虫只有极小的可能。落到方源手中。

唯有这样子利用葛谣,才能发挥出这位三转中阶蛊师的最大价值。

“如果不是葛谣。我绝不可能行进得这么快。她身上的蛊虫虽好,我却不可能得到。这些南疆的蛊虫,我必须换掉,除了战力方面的考虑之外,也是我隐藏身份的巨大破绽。还是要到那处战场才行啊。”

方源暗暗感慨了一番,又取出皓珠蛊。

皓珠蛊中封印着定仙游,仿佛是一个洁白的琥珀。

当着葛谣的面,方源又毫不避讳地取出蒙尘蛊。

蒙尘蛊形如蚕茧,通体暗灰,摸在手中仿佛磨沙一般。方源灌注进真元,蒙尘蛊便微微一炸,爆成一蓬琐细的灰烟。

朦朦的灰色烟尘,自有灵性,全数落到皓珠蛊上。

原先散发着明亮白光的皓珠蛊,被这烟尘覆盖,顿时光芒变得黯淡下来。仙蛊定仙游的气息,再弱一筹。

这就是宝珠蒙尘。

方源五百年前世,到了中洲征伐,掀起五域混战之时,一些蛊师潜入他域作战,为了隐藏身份,遮掩气息,才研发的手段。

“你这是在做什么?”一旁的葛谣,好奇地问道。

方源没有答她,只是将晦暗的宝珠收入怀中,便继续启程。

两人不断前行,空气中毒雾越来越重,浓郁的紫色已经开始影响到视线。

两人不得不更加频繁地停下来,利用蛊虫解除身上积压的毒。

咔嚓。

脚下传来一声脆响,好像是踩到了什么枯树枝上。

葛谣疑惑地将目光投下去,然后便尖叫一声,兔子般迅速地往后蹦去。

“这,这里怎么有人的头骨?”她的声音打着颤儿。

“因为这里,原本就是一处战场。”方源走在她的前面,并不回头,反而是加快了步伐。

“战场?哎,你等等我呀,别走这么快!”葛谣赶紧跟上方源的脚步。

她越走越心惊。

原先是浓郁的紫色毒雾遮掩了她的目光,现在她走近了看,便发现微微腐烂的草地上,有很多白骨。有人的,也有狼的。

地面上也布满深坑或者沟堑,很显然这里曾经发生过惨烈的大战。

“死了好多人啊,到底是谁曾经在这里进行生死的搏杀呢?不过打出来的深坑沟洞,已经重新长满了毒草。再结合其他的痕迹,这片战场至少有二十多年的历史了。”

葛谣落在方源的身后,看着他不断地搜寻,像是找着什么东西,顿时就有了明悟。

“原来常山阴,深入腐毒草原的目的,就是这片战场啊。他究竟要寻找什么东西?等一等,二十年前的腐毒草原,的确有一场大战。阿爸曾经和人谈过,我当时就在旁边……”

一段记忆,被葛谣从脑海深处翻腾了出来。

那时候,葛谣还小。大约只有四五岁。她的父亲宴请一位贵客,将她带在身边。

在大帐中。大人们谈起草原上的英雄豪杰。

“说起来,常家这次出了个人物啊!”

“你是说狼王吗?”

“不错。此人是常家着重培养的奴道蛊师,狼群在他的手上,行动如风,攻势如雨,临时变阵随心所欲,实在是身手了得。更关键的是,他为人正派。对他的老母亲极为孝顺。这次他老母亲身中奇毒,必须是雪柳上的雪洗蛊可解。他就不顾阻拦,独自深入了腐毒草原。”

“唉,也正是因为如此,狼王才丧命的呀。”

“是啊,这件事情从头到尾,都是一个阴谋。是狼王的死对头哈突骨精心谋划的。哈突骨想杀了狼王,但狼王也不是好惹的,将哈突骨的一帮子马匪,都杀了个干净,为草原除了一大害啊。”

“只是可惜了狼王常山阴这个英雄啊,他也因此丧命在腐毒草原了。”

……

“常山阴?你是狼王常山阴!”想到这里。葛谣吃惊得张大了嘴巴,双眼死死地盯着方源猛看。

“哦?你也知道常山阴?”方源淡淡地回了句,仍旧在不断寻找着什么。

但葛谣旋即又摇头:“不,不对。算算岁数,常山阴若活着。应该早就有四五十岁了。怎么可能是你这样的年纪。而且,你的相貌、口音都不对。你不是常山阴!”

“呵呵呵。我不是常山阴,那我又是谁?”

“是啊,你到底是谁?为什么假冒成一个死去的蛊师?”葛谣心中全是疑问。

忽然她双眼一亮:“等等!虽然常山阴死后不久,他的老母亲也毒发身亡了。但他已经成婚,留下了血脉。他的儿子已经成长起来,也是个优秀的人杰。难道说……你就是常山阴的儿子?”

方源笑了笑,正要回答她,忽然耳畔传来一阵奇怪的声音。

这声音,就好像是老旧的车轮滚动,碾压在地上。

随着这个声音,紫色的雾气中,滚住一个巨大的白骨车轮。

它有两人高,宽有半丈,通体都是白骨所制。车轮表面,俱是尖锐骨刺。车辐中央,是一个巨大的骷髅头。骷髅头恐怖的眼洞中,燃烧着血红色的火焰。

“小心,这是哈突骨的五转蛊――战骨车轮!”

方源刚刚示警,车轮就猛地加速,气势汹汹,碾压过来。

金龙蛊!

方源双掌一推,金龙咆哮,狠狠地砖在车轮上。

车轮颤了颤,然后轻而易举地将金龙碾碎,直撞方源。

方源连忙展开骨翼,飞上天空。

但车轮竟然也拔地而起,悬空飞撞。

金缕衣蛊!

方源见实在躲闪不过,只好选择硬抗。

砰的一声,他被狠狠地撞飞出去,落到地上,摔得满身污泥烂草。

嗖嗖嗖!

三根螺旋水箭,接连射中车轮。

白骨车轮落到地上,溅起一大片泥土,它舍弃了方源,调转了方向,朝葛谣碾压过去。

葛谣连忙催动水迹蛊,撑起水甲,不断躲闪,不断还击。

方源也赶了过来,进行支援。

这是一场艰苦的战斗。

对方是五转蛊,原主人哈突骨赖以成名的核心蛊。哈突骨死后,它就成了野生蛊,以战场上的白骨为食。

白骨战车的攻势磅礴凶悍,往往将对手碾压成肉酱血泥。

方源受到压制,战力暴降,又没有带来五转蛊,无法和白骨车轮正面抗衡。

原本他的计划是躲避白骨车轮,搜寻常山阴的遗体。

不过现在有了葛谣在旁辅助,他便改了计划,选择了战斗。

足足战了两个多时辰,靠着方源定下来的牵制战术,两人狠狠欺负了白骨车轮智力上的不足,轮流喘息,渐渐消磨,终于将白骨车轮打得倒在地上。

战局一定,方源立即将双手放在白骨车轮之上,意志混同真元,一起扫荡过去。

白骨车轮是五转蛊,因此就算是六转的春秋蝉、定仙游的气息,都不能帮助方源瞬间炼化了它。

但车轮此刻满身裂痕,险些散架,已经惨不忍睹,濒临碎灭。方源又有百人魂,精神十足,意志如铁,真元不断灌注,又花费了一炷香的时间,终将其收服。

五转蛊到手!(未完待续。如果您喜欢这部作品,欢迎您来起点投推荐票、月票,您的支持,就是我最大的动力。手机用户请到阅读。)

------------

\end{this_body}

