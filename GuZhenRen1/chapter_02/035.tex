\newsection{通缉令}    %第三十五节:通缉令

\begin{this_body}

“你们要知道,紫幽山上紫枫叶多的是。那些人之所以买,也是图个方便快捷,省下采集的功夫罢了。唉,跟你们讲了也白讲。算了,算了……。”

老村长不断地摇头叹气。

方源饶头道:“俺们想多卖点元石,一是回本,二是孝敬一下二老。哪里料到费了这么长功夫,都卖不出去。”

他声音焦急,隐约带着一丝哭腔。

听了他的话,老村长的心骤然软了,心中的怨怼顷刻间消散了大半。

方源又道:“村长大人不急,俺们决定明天就跟着商队一起走。把价钱降下来,总是能卖的出去的。”

“跟着商队?谁允许你们能跟商队的?”老村长把双眼一瞪。

方源理所当然地道:“俺看到商队里,有很多的凡人呐。他们能跟,俺们为什么不行?”

老村长以手扶额:“那是人都是蛊师大人的家奴!你以为商队是随随便便就能跟的啊?万一混进歹徒怎么办?”

“啊?!”方源张大嘴巴,呆愣在原地,“那怎么办?商队明早就走了。”

“唉……”老人深深的叹了一口气,“算了,帮人帮到底吧。明早我求求人,能不能加入商队,就看你们俩的造化了。”

天刚刚破晓,淡青色的天空中残留着几颗残星。远望,紫幽山上一片暗紫色,幽静而又神秘。

经过一夜的休整,商队已经开始整装待发。

“把货物都检查一遍!”

“绳子都绑紧了没有,要是途中掉落一件,就打你们一百大板。”

“快快快,把我们的黑皮肥甲虫都喂饱了。”

蛊师们大声命令着,将凡人家奴们指挥得上下乱窜。有的脾气暴躁,手中拿着皮鞭,看哪个动作慢了,就上去抽一下。有的爱惜蛊虫,亲自喂食。

“陈大人。”老村长弯着腰,向商队的一位副首领请示行礼。

“哦,老张啊,我这正忙着呢。有什么事情快说吧。”这个陈姓蛊师道。

“是这样子的,我有两个后生,搞了些小买卖,“…”老村长还未说完,陈蛊师就突然高喊一声,“陈鑫,你个小子在瞎晃悠什么呢?赶紧去给翼蛇喂食去。你指望那些家奴能喂好?你那头翼蛇这些天,已经吞了三个家奴了!”

“是,家老大人。”陈鑫被捉住,耸搭着脑袋回道。

陈蛊师却不放过他,又斥责道:“告诉过你多少次,在山寨里叫我家老大人,在商队里就得称呼我为副首领。”

“是是是,副首领大人。”陈鑫答了声,飞奔而走。

“这个混小子……。”陈蛊师气哼哼的嘟囔了声,又转头看向老村长,“你刚刚说什么来着?哦!是想担保两个晚辈加入商队?”

“大人英明,正是如此。”老村长连忙答道。

“这样啊…。”陈蛊师故意沉吟起来。

老村长就是他点化成蛊师的。为的就是在这商队的必经之地,安插一个自己人。

商队行商,那些山寨是贸易重点,但是沿途的凡人村庄也不能疏漏,也很重要。

商队人多事杂,很多生活物资消耗掉,就得沿途补充。还有家奴,有时候商队遇险,一些家奴死亡,缺乏人手,商队就得从沿途的村庄里抽选凡人。

说起来,陈蛊师手中的家奴,已经略显不足了口在商队中,凡人命贱,也只是一种能说话能活动的消耗物资罢了。

“今后行商,我途经紫幽山落脚时还需要老张,不答应他岂不是寒了他的心?正巧我也缺人手,不过却不能这般轻易地一口答应。得掂量一下,才能卖好。”

陈蛊师正思索着时,一位商队的传讯蛊师奔跑了过来。

他手中抖着一叠纸,一边奔跑,一边大喊:“都注意了啊,新的通缉令,有新的通缉令!”

他一边喊着,一边随手贴在一只黑皮肥甲虫的身上。

“新通缉令?哪家的?有多少悬赏,拿来我看。”陈蛊师来了兴趣。

“是,副首领。”传讯蛊师连忙递来一张。

陈蛊师看了:“哦,是百家发布的通缉令。只要消息正确,就是一千块元石?这么高!”

陈蛊师眼中一亮,有了兴趣。

通缉令上一般有两个价格,一个是消息价,一个是缉杀价。

一千块元石的消息价,往往是要通缉那些闯出名头的魔道中人。但这张通缉令上的肖像,却是两个年轻人,皆长得五官端正,一个甚至还很漂亮。

一男一女,这是两个新人,

“一个是一转蛊师,一个是三转。消息价高达一千块元石,缉杀价也有五千八百块。啧,看来百家是恨极了这两个魔崽子了。嘿嘿…,“”陈蛊师幸灾乐祸地笑了笑,反正又不是陈家。

他浑然不知,这两个魔子已经就在他的附近。

老村长也顺势看了眼通缉令,心中不禁打个寒颤。

“蛊师的世界真是危险,看起来这么漂亮的少年郎,竟然是罪犯魔头!但愿他们不会跑到我们村子里来。”

“好吧,看在老张你这么多年辛辛苦苦的份上,我就答应你的这个要求了。”陈蛊师道。

“啊,谢大人!大人我这就把他们俩叫来。”老村长大喜过望。

陈蛊师摆手:“不必了,我这忙的很。你让他们俩到陈鑫那儿报到去。”

他对面见两个凡人,毫无兴趣。同时也没将眼前的人,和手中的通缉令联系到一块去。毕竟这是百家的通缉令,而百家寨远在数千里之外。陈蛊师下意识地就觉得,自己这边安全得很。

这是很惯常的思维。

就算是在现代地球上,发生在省城的命案,哪怕再恶劣凶残,其他省城的人也不会感到太大的危机感。哪怕现代交通如此发达。

除此之外,还有一个侥幸心理。

大千世界,芸芸众生,这两个贼子怎么会偏偏逃到我这里来呢?那我的运气也太背了吧,这是不可能的!

人们总觉得,不幸的事情不会发生在自己的身上。

再者,通缉令有很多,很多都是魔道巨头,或穷凶极恶,或乖僻阴毒,吸引众人眼球。像方白这两个新人,一个三转一个一转,能成什么气候?

陈鑫看到方白二人时,一点都没有联想到通缉令上。

方白二人皆形象大变,已经毁容的方源就不提了,白凝冰经过这些天的演练,也像模像样。

陈鑫一看就失去了兴趣,尤其是方源的容貌更让他有些嫌恶。

他只是一转修为,而方源在几天前就晋升二转了。

陈鑫匆匆辨别了一下,没有感觉到任何的蛊师气息,就召来一个老管家,让他来安排方白二人的工作。

“你们两个叫什么名字?”老管家问道。

到了这个时候,这才有人问他们名字。

“俺叫黑土,俺婆娘叫白云。”方源随口便道。

“女的?”老管家回首,皱了皱眉头。

他盯着白凝冰看了眼,看着她黑不溜秋的木愣愣的样子,居然叫做白云?这个黑土也是够丑的!

“女人麻烦。你们可得小心点了。出了事情,不要怪老夫没有提醒过你们!”老管家道。

“俺知道。俺这边有辆板车,车上是紫枫叶。俺婆娘就留在车上,照看货物,俺们也不想和其他人多接触。”方源道。

“哼,知道就好。”

老管家给他们俩安排了一个卸货的体力活。不过对于方源和白凝冰来讲,简直是不值一提。倒是白凝冰为了伪装出气喘吁吁的样子,有些劳心劳累。

不远处,几个家奴趁机偷懒,围坐在一块休息。

几个人的目光,都盯着方白二人。

“强哥,来了两个新人。有人看到,他们还带了私货!那可是一车的紫枫叶啊。”一个瘦子家奴有些兴奋地道。

剥削刚刚加入的新人,已经是商队老人的惯例。

强哥蹲在地上,眯了眯眼睛:“看到了,瘦猴,你先去试试他们。”

他体格健壮如牛,胸肌发达,但并不莽撞。

在这个蛊师为尊的世界里,凡人的勇武是不值得一提的。他能够成为这个小圈子里的中心人物,自然有些聪明。

瘦猴哎了一声,在众人的注视下,走到方源的身边。

“老兄,哪来的呀?别人都叫我猴哥,以后我们一起做事,要多关照。”瘦猴脸上堆起笑容。

方源瞟了他一眼,只说了一个字

“滚。”

瘦猴顿时眼眶一撑,流露出愤怒的神色。

方源也不看他,自顾自的搬货。他前世混过商队,很清楚这其中的龌龊。

用黑话来讲,瘦猴这是在“踩盘子”。先用话套出方源的底细,若是没有背景,那就合伙欺负,剥削出一些好处来。

事实上,不止是凡人如此,就是蛊师之间也差不多,只是大多吃相优雅好看一点罢了。

独行冒险,是和野兽搏杀。结队行动,是和同类拼斗。

有利益的地方,就有争斗。空间就这么大,每个人都想生活得更好,有更大的腾挪空间,那怎么办?

只有去侵占其他人的空间。

瘦猴未料到方源如此不给面子,一时间站在原地,狠狠地瞪着他。

方源对这种小角色,根本不放在眼里。凡人如草芥,就算是打杀了一两个,算得了什么?

只要不妨碍出货,上面的蛊师也不大在意的。

真要在意了,方源自然也有预备手段,来摆平这个事情。

一句话,要来招惹方源,这些家奴简直是在找死。

“怎么,还不滚,想要我走你吗?”方源又冷冷地瞥了瘦猴一眼。

瘦猴哼了一声,却不敢发作,铩羽而归。

这样强硬的态度,反而让强哥心生忌惮,莫不是这两人真有什么背景瓜葛?不然怎会这般厉色?保险起见,还是先探查清楚了情况再说罢。(未完待

\end{this_body}

