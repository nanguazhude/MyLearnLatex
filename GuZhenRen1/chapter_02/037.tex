\newsection{好人会有好报?}    %第三十七节:好人会有好报?

\begin{this_body}

“只是她为什么会在这里?还有,她为什么叫做张心慈呢?记忆当中,她明明是蛊师,怎么其他人却说她没有修行资质,只是个凡人?”

方源心生疑惑。

“难道她并非商心慈,只是长得相像?这未免也太像了!不,等等……”

方源脑海中思绪翻腾,忽然从记忆的深处挖掘到更陈旧的信息。

“这个商心慈,身世坎坷。她是商家族长年轻时,在外游历的私生子。她从小就没有父亲,饱受欺凌。母亲死后,她日子更难过。被族人逼迫,只得跟随商队外出行商。结果来到商家城时,她的血脉被商家族长感受到。商家族长惊喜交加,当众认下她,又感觉亏欠良多,因此以后即便她犯了许多错误,都替她包涵遮掩。”

想到这里,方源眼中精光一盛。

“原来如此,我明白了!”

结合眼前的现世,以及记忆中残旧的线索,方源推算出了真相。

多年前,商家族长还是商家少主的时候,在张家留下了血脉,正是商心慈。

商家和张家,自古以来,就关系不和。

商心慈生下来后,她的母亲出于种种考虑,没有将商家族长抖露出来。因此商心慈,便作为耻辱的私生子出生,一直跟随母姓。

这也是为什么现在,她仍旧叫做张心慈了。

她母亲死后,她被族人逼迫,只好外出行商。不想来到商家城后,被察觉出血脉。她的父亲也成了商家族长,位高权重,当众相认,让她的命运发生了天翻地覆的转变。

“也就是说,我如今的这个商队,就是她第一次行商的队伍!”

认识到这点之后,方源心中砰然而动。

要知道,这商心慈绝对是个潜力股啊,她是未来的商家族长。这点要说出去,恐怕全世界都不会相信吧。

当然,未来是可以改变的。

就算按照前世的轨迹,等到她成为族长时,天下变幻,沧海桑田,强盛的商家已经破败不堪了。

对于方源来讲,等到她成为商家族长。呵,这个投资期实在太长,也没有回报。

她的价值,不是在族长的时候体现的,而是当她被商家族长相认,成为商家少主之一。

而作为少主,她必将负责一部分生意。这也是商家为了培养家族继承人的传统。

恰恰是这点,正是方源需要的。

他需要销赃。

一个稳定的,良好的,安全的销赃渠道。

在他的重生大计中,这点是必不可少的。

他今后开启的传承密藏,必定更多。很多东西都用不到,只要交易,才能发挥出它们巨大的价值。

就比方说,如今存在白凝冰空窍中的那些骨枪蛊、骨枪螺旋蛊。

这些蛊,方源和白凝冰用一只就足够了,顶多再留一只备用吧。其他的如果不卖出去,只能烂在手中,甚至还要耗费大量的奶水来养活它们。

曾经,方源打算先将那个贾金生,当做暂时的销赃渠道,然后最好搭上贾富,培养出这条线。

结果,造化弄人,为了饱受花酒行者遗藏的秘密,他不得不把贾金生给杀了。

“这是上天送到我面前来的礼物啊!”方源心中长叹,看着远处的商心慈,眼中熠熠生辉。

商家背景雄厚,至少在那场席卷南疆的变革风暴之前,它是南疆的霸主势力之一。将贼赃销售给他们,苦主也不敢来找麻烦。

但正因为他们势力太强,也让方源担忧他们黑吃黑。

别看是正道,商家人的心都是黑的,这实在太正常了。

但是商心慈必定不会。

她是经过百年历史,变革动荡检验的。她的善良、温柔、仁慈、诚信,在方源前世传为佳话,美名流传整个南疆。

而且最重要的一点,她没有根基。

和其他的少主不同,她到了商家城后,完全是孤家寡人。日后,她犯下很多错误,并非是她不聪明,而是其他少主为了族长之位,暗地里给使她绊子,打压竞争者。

方源需要她的渠道,而她今后若要活得更好,同样也需要方源这个外援。

最关键的是,她好操控和影响!她的年轻,还有性情,都是方源眼中的“把柄”。

“呵呵呵。”一旁的白凝冰忽然笑起来,“你不会是看上她了吧?”

“什么?”方源的思绪微微一顿。

“你不要装了,咱们都是男人,看你的眼神我就懂了。话说回来,这小妞看得还挺顺眼。不过你想要得到她,可就难喽。除非我来帮你,找机会悄悄的掳走她。作为交换,你那只阳蛊就得先给我。”白凝冰充满诱惑地道。

但方源下一句话,就把她气得七窍生烟:“你也算男人?”

“你!”

事情有点麻烦啊。

方源首先要接近商心慈,然后要得到她的信任。然而时间是有限的,他必须趁着她到达商家城之前,搞定这一切。

白凝冰直接虏人的主意太馊了,风险太大,只会让一切变得更糟糕。

方源清楚的知道一点,商心慈之所以被屡屡欺骗,不是因为她笨,而是因为她太善良。

前世,有一个魔道中人,人称“夜君子”,擅长偷盗,狡诈如狐。

用相同的理由,骗了商心慈许多次。

有一次,他实在忍不住了,就问道:“我总是以这个说法告诉你,你就不担心我骗你吗?”

商心慈便道:“你说你资金周转不开,若是不借些元石救急,家里人都会揭不开锅,会饿死。我也知道你骗我的可能性极大,但你每次说,我都忍不住想,万一你这次说的是真的呢?我如果不先付款,那就会有几条生命消逝。虽然这种可能性不大,但我却不想赌这个东西。”

夜君子听了这话后,涕泪并流,被商心慈的人格魅力深深打动,当即拜倒在地上。

从此之后,他改邪归正,追随商心慈,忠心不二,立下许多汗马功劳。

太阳一点点被西方的群山吞噬,夜幕渐渐降下。

人群长龙也变得稀少。

最终,家奴们都领到食物,慢慢的散了。

“好了,今天就这么多。我明天还会来的……”她还未说完,一个身影就猛地窜到了她的面前。

这是怎样的一张脸啊!

没有眉毛,都是烧伤,头发也只剩下一些,还缺一只耳朵。不是方源还是谁?

商心慈着实吃了一惊,身后的丫鬟甚至惊叫出声。

“干什么?!”那个身躯魁梧的老蛊师,立即大喝。

“张小姐,你行行好,买了我的货吧!”方源不管这老蛊师,对商心慈大喊道。

白凝冰默默地站在远处,饶有兴趣地欣赏起方源的表演。

“我这里还有最后一块蒸饼,你拿去吃吧。”商心慈脸上浮起温柔的笑,她心中没有一点对方源的嫌恶,只是同情。看他这样的伤势,曾经的经历一定很惨痛吧。

唉,也是个可怜人。

方源接过蒸饼,却一把扔在地上:“我不吃这东西,我要卖货!我卖了我家的老房子,买了这车的紫枫叶。结果却卖不出去,眼看着叶子快要干枯了,我活着还有什么意思?!呜呜呜……卖不出去我就不活了我,我就一头撞死算了!”

他说着说着,就大哭起来。又是跺脚,又是嚎叫,神情极为激动,带出一丝疯狂。

白凝冰看得都呆了。

“这样的演技,我真是望尘莫及啊!”

若非她知道方源的底细,恐怕也被方源骗过去了吧。

她再看看周围的人的神情和目光,果然有许多的惊诧、鄙视、同情、冷漠,却唯独没有怀疑。

“这人谁呀,一下子就冲过去了,吓我一跳!”

“真是贪心啊,居然想要张小姐买他的货。”

“这种人活该,真以为买卖是那么容易做的吗?”

“唉,他一定是被压价了。想当年我也遭遇过……”

人群中议论纷纷。

“呔,你这个疯汉,竟然惊吓我家小姐,还不快滚开!”那护卫的老蛊师,舌绽春雷,跨前一步,把商心慈护在身后。

“张小姐不买我的货,我就不活了!张小姐,你是好人,求你买下这车货,救救我吧。”方源大哭不止。

商心慈心软了:“唉,你不要哭了。我买下来就是了,生命是最可贵的,你今后要好好生活,切勿轻生。张柱叔,给他三块元石吧。”

“小姐……”护卫的老蛊师皱起眉头。

“我日,这也行?!”

“要不然,我要去这样闹闹?”

一群摊贩激动不已。

“谢小姐,谢小姐。张小姐你就是我黑土的救命恩人呐!”方源大喜若狂,脸上带着泪痕,连连鞠躬。

老蛊师的眼皮子抖了抖,看向方源身后的板车:“这车紫枫叶,最多也不过值两块元石。三块太多了!”

“张柱叔……”商心慈声音柔柔的。

老蛊师深叹一口气:“小姐,不是我舍不得三块元石。但这对他区区一个家奴,还是太多了。会惹来觊觎和暗算的,小姐为了他的安全,两块元石刚刚好啊。再说,你现在这样给,恐怕明天所有的小贩都过来了。”

“张柱叔说的有理,那就给两块元石。”商心慈思索了一下,从善如流。

方源用颤抖的双手接过两块元石,他深深的看了商心慈一眼:“张小姐你是个大大的好人,你会得到好报的!”

\end{this_body}

