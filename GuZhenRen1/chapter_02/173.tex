\newsection{地灵——蛊仙之殇}    %第一百七十三节:地灵——蛊仙之殇

\begin{this_body}

%1
中洲,狐仙福地。

%2
一座高大的水晶山川,矗立在福地的中央。

%3
它名为荡魂山,通体粉红,散发着梦幻的色彩。

%4
此时此刻,来自十派的精英弟子们,如一只只蚂蚁,正努力攀登着山峰。

%5
在山腰附近,方正满身大汗,脸色一片苍白,双指紧紧扣着崖壁,大口大口地喘息着。

%6
万物生灵只要身处在荡魂山的附近,其魂魄就要受到震荡之苦。方正越往上攀登,就越是头晕眼花,魂魄像是被大风吹拂,有种晃晃欲倒的感觉。

%7
“哎呀呀,你要坚持不住了?那就放弃吧。你看你落后别人那么多,怎么可能获胜呢?还不如直接放弃,反正你已经没希望了。”狐仙地灵嘟着粉嫩的小嘴,忽然出现在方正的身边。

%8
狐仙地灵形如小女童,肌肤若雪,又透着粉嫩之色。一双乌黑发亮的大眼睛,散发着纯真的光芒。最引人瞩目的是她屁股后面,长了一根雪白无暇的狐尾,毛茸茸的,十分可爱,让人很想握住把玩。

%9
此时此刻,狐仙地灵就坐在虚空中,望着方正,打趣道。

%10
方正也不惊讶。

%11
他从山脚下攀登到现在的高度,狐仙地灵出现了好几次,每次都来看戏,十分调皮。

%12
方正已经找到了对付她的秘诀,那就是不搭理她。

%13
果然,狐仙地灵见方正不吭声,顿觉无趣,小嘴嘟得更厉害了:“你这个笨蛋小子,真是好无聊。千万不要成为我的主人啊,否则我的生活就太无趣了。嘻嘻,还是其他人好玩。”

%14
说完。她骤然消失在原处,找其他的精英弟子玩去了。

%15
狐仙地灵走后,方正的空窍中,寄魂蚤一阵轻微的颤动,传来天鹤上人的声音。

%16
“方正,努力坚持啊。现在还不是我出场的时候。你至少要达到山腰处,我们才有获胜的可能。坚持,再坚持。魂魄的力量,是可以挖掘的。你还有很多的潜能没有开发出来。”

%17
听到师傅的鼓舞。方正有些涣散的眼神,重新坚定起来。

%18
他在心中回答道:“师傅,你放心,我会坚持下去的。我只是想休息一下,喘一口气。”

%19
顿了一顿。方正又问道:“师傅,我一直有一个疑惑,地灵到底是什么东西?刚刚那狐仙地灵就在我的身边,你说我一把捉住她,是不是就不用再攀山了?”

%20
天鹤上人被吓了一跳,连忙喝斥道:“你这个小子,真是无知无畏。连地灵的主意都敢打!你知道地灵是怎么形成的吗?那可是蛊仙死后残留的意志和魂魄的碎片,再结合福地之力,而形成的灵体!”

%21
“什么,地灵的生前竟是蛊仙?”方正吓了一大跳。

%22
“不错。你刚刚看到的狐仙地灵。就是曾经的狐仙死后所化。只是再无生前的记忆,只剩下最终的执念。你别看她这么人畜无害的样子,在这狐仙福地当中,她能自由地操纵天地之力。直接抗衡蛊仙!她能随意地禁锢,一转到五转的任何蛊虫。只有六转的仙蛊。才能在福地里活动自由。方正,你实在是太胆大妄为了。你给我老老实实地攀山,千万不要有不切实际的想法。”天鹤上人教训道。

%23
“是,师傅。我再也不敢了。”方正连连点头,在心中认错。

%24
天鹤上人教训了方正,语气便一缓,又安慰鼓舞他道:“所以你明白了吧?只要你能成为福地之主,就能得到地灵的效忠。只要在福地当中,地灵就是堪比蛊仙的存在啊!”

%25
方正听得惊呆了。

%26
这是何等强大的臂助啊!等若是得到一位蛊仙的辅佐!!

%27
天鹤上人又接着道:“不过,地灵终究不能走出福地。狐仙传承的真正精髓,还在于这片广袤的福地啊。方正,你的层次还太低,不知道的东西太多太多了。等你继承了这片福地,你就会慢慢地明白,福地对于蛊师的帮助是何等的巨大!你真是太幸运了,碰到了一个有灵的福地,又得到门派的帮助。如果是无灵的福地,那就大打折扣了。”

%28
方正不禁好奇:“师傅,无灵的福地又会怎样?”

%29
天鹤上人答道:“无灵的福地,注定灭亡。就像是一头沉眠的神龙,所有人都能吸它的血,吃它的肉。直到它死亡,它也不会苏醒。方正,等你成为福地之主,你要将福地中的资源,上缴给门派。仙鹤门栽培了你,你也要回馈门派。门派壮大了,对你的保护就更大了。这个道理你明白吗?”

%30
“嗯,我明白的。是仙鹤门收留了我,没有仙鹤门的帮助,我也没有争夺传承的希望。更没有向哥哥复仇的可能。如果有可能,我不仅要回报门派,还要帮助师傅您复生!”方正连连点头,目光清澈,他对仙鹤门一直都充满了尊崇和感激之情。

%31
天鹤上人听了一楞,然后干笑几声:“笨蛋徒弟,人死哪能轻易复生?你有这个心意就好了。”

%32
……

%33
正道的荣光,笼罩着三叉山上下。

%34
在铁慕白的影响下,三王传承成了正道人士的探索乐园。

%35
但凡魔道蛊师,尽数被驱逐,不能踏进三叉山一步。

%36
“铁慕白太霸道了,直接圈了场,不让我们任何人进去。”

%37
“他堂堂的前辈高人,居然这么贪婪。吃了肉骨头,也不留点汤给我们喝!”

%38
“最关键的是,三王传承开启的时间越来越短,三道光柱也越来越弱。看来真像小兽王一个月前说的,这个蛊仙福地正在衰败,过不了多久就会彻底毁灭了。”

%39
……

%40
魔道蛊师们心中都越来越焦躁愤怒,百年难得一遇的大好机缘就在眼前,但他们却被摒除在外,看得见却吃不着。

%41
“撑死胆大的饿死胆小的,兄弟们,我们一起冲上去。那个铁慕白再厉害,还能屠杀了我们所有人不成?!”有人咆哮着,站出来鼓动众人。

%42
“我们可以趁着铁慕白进入传承之后,再闯上山。这样一来,我们的阻力就小很多了。”有人提议道。

%43
“这个方法有缺陷。我们从传承中出来时,说不定铁慕白也出来了。再说,谁也不知道会出现在三叉山的哪个角落,会被正道围杀的。”有人当即反驳了一句。

%44
“那又怎样?富贵险中求,不入虎穴怎得虎子?要想不冒险就能捡便宜。世上哪有那样的好事?!”

%45
正当魔道众人嘈杂怒骂之时,耳畔忽然传来百鬼呼啸之音,刚刚还晴空万里的苍穹,变得乌云滚滚。

%46
漆黑如墨的乌云中,传来一个刺耳声音:“嘎嘎嘎。铁慕白,你既然出关了,怎么不知会老朋友一声,嗯?”

%47
乌云沸腾,形成一个庞大的人脸,鹰钩鼻,深眼眶。注视着三叉山。

%48
“这样的笑声,这样的威势,是我们魔道中的巫鬼大人!”

%49
“我想起来了。巫鬼大人同样是五转巅峰,老一辈的强者。是铁慕白的宿敌啊!”

%50
“正道有什么了不起的,我们魔道中也有强者!上啊,巫鬼前辈!!”

%51
一时间,魔道中人尽数欢腾起来。各个双眼放光,有的大叫。有的呐喊。

%52
乌云推进,很快就笼罩住三叉山,声势庞大,像是千军万马一般。一时间,连阳光都被遮盖,三叉山上阴暗笼罩下来。

%53
正道蛊师们无不人心惶惶。

%54
“竟然有这样的威势!”

%55
“这是巫鬼老魔,想不到他竟然还活着?”

%56
“巫鬼老魔,他的年龄已经有数百年了!当年他在冲击六转境界的关键时刻,被初出茅庐的铁慕白大人无意破坏,因此他一直怀恨在心,曾经屡次屠杀铁家族人,打击报复。”

%57
……

%58
“巫鬼,这些年你躲到那个山脚旮旯里去了?今天,又想来尝尝失败的滋味吗?”山顶处,绽放出金色的光辉。

%59
金光中,铁慕白傲然挺立,背负双手,看着天空中的滚滚黑云,语气平淡。

%60
“哼,十多年前侥幸让你胜了一招半式,你还真抖起来了。小贼,今天我要让你死无葬身之地!”乌云如开水般沸腾,忽然凝聚成一只大手,向铁慕白抓去。

%61
这只手,庞大无比,比小型的山峰还要巨大。声威赫赫,简直慑人,竟似有捉星拿月的气度!

%62
乌云巨手似慢实快,一把抓下去。

%63
腐蚀的烟气迅速升腾起来,巨手将一片山峰都包裹,覆盖之处树木山石,都化水消融。

%64
“还是老花样罢了。”铁慕白一声冷哼,脚下一顿,化为一道犀利的金芒,直接破开乌云巨手,向天空冲去。

%65
金芒璀璨,如流星,似闪电,一下子撞入到漫天的乌云当中去。

%66
几乎是下一刻,雷霆般的炸响,接连爆发。

%67
乌云爆涌翻滚,金色的电光时时闪烁。

%68
两大五转巅峰的蛊师,在乌云内部交手,虽然看不清具体的情形,但众人仍可从战斗的余波中,体会到两者的强大!

%69
“巫鬼,我刚刚出道时,你就已经是成名高手。那时候,我遇到你,必定绕道走。但五十年后,我能从你的手中逃得性命。八十年后,你奈何不住我。一百多年后,你败在我的手中。如今又过了十几年,到了今天,你要把命交给我了。”

%70
乌云中传来铁慕白洪亮的声音。

%71
“呼呼呼……铁、慕、白!你太猖狂了,你运气好,出生在铁家,依靠家族,站着说话不腰疼。老夫要是有你这样的资源,早就能成就蛊仙了。”巫鬼嘶哑着声音,气息不稳。

%72
很显然,刚刚的激烈交手,是铁慕白大占了上风。

%73
“不过你以为,我没有后手吗?哈哈哈!”巫鬼喘息了一阵子,忽然又狂笑起来。

%74
随着他的狂笑,第三个五转巅峰的气息,猛地升腾起来。

%75
“铁慕白,这些年来,你有没有想念我啊?”一个阴测测的声音,传遍方圆千里。

%76
铁慕白的声音中难掩震惊:“骷魔,想不到你也来了!”

\end{this_body}

