\newsection{聪明人的默契}    %第四十三节:聪明人的默契

\begin{this_body}

%1
“可疑?”商心慈浓密的眼睫毛下,目光流转。

%2
张柱点点头,肃容道:“事实上,我从进入匪猴山的那天起,就开始怀疑他们了。小姐你要奖赏一百五十块元石。如此巨款,他们却一点都不心动。这实在发人深思。”

%3
他顿了一顿,继续道:“这些天来,我一直暗中观察他们。发现了更多的疑点。首先,他们很少和身边的家奴交流,恨不得变成透明人似的。其次,他们拒绝了很多家族的招揽,哪怕条件再优越。”

%4
“小姐还记得那晚他来投靠我们的样子吗?黑土他有这样大的力量,怎么还会被人伤成那样子?最后,据我观察,他的那个同伴,虽然是男人装扮,但其实却是个女子!”

%5
营帐内,一片沉静。

%6
半晌后,商心慈才微笑道:“力气大,并不代表就一定能打赢别人不是吗?双拳难敌四手,黑土受伤也很正常呀。其实你说的这些疑点,我都知道。”

%7
张柱却丝毫没有流露出惊异的神色,他了解商心慈,知道她的聪慧。

%8
“小姐……”

%9
商心慈眨眨眼,柔美的容颜显出一丝俏皮:“张柱叔,你憋了好多天了吧?发现我一直没有处理这个事情,所以今天就来提醒我。”

%10
张柱笑起来:“小姐,我什么都瞒不过你啊。但为什么,你一直放任他们俩在身边呢?”

%11
“因为我觉得他们并没有恶意。”商心慈的眼中闪过智慧之光,“他们在匪猴山被我们怀疑。如果当时他们不站出来,我们就看不到疑点。但他们为什么还要站出来,帮助我呢?”

%12
“这个……”

%13
“如果他们心怀叵测的话,一定会隐藏在一旁看戏吧,或者接受那一百五十块元石。但是他们没有。黑土他说报恩时,神情那么真挚,我觉得他说的是真的。他是真的想偿还这个人情。”商心慈道。

%14
张柱一阵哑然:“但他们一定不简单,肯定有秘密。”

%15
商心慈笑魇如花:“每个人都有秘密,我也有秘密,有秘密的就一定是坏人吗?这个世界是光明的,能够知恩图报的,再坏也坏的有限不是吗?”

%16
“话虽这么说,但我始终觉得他们有什么目的。也许他们在策划什么阴谋……等等,我知道了,他们一定是匪徒的内应。混入商队,配合魔道中人抢劫!”

%17
“这也说不通的。”商心慈摇摇头,“既然是内应,更应该隐藏起来,何必在匪猴山出风头呢。那么多人招揽他们,他们大可以加入其它的队伍,不更易于隐藏吗?为什么偏偏留在我们这里?我觉得他们一定是有什么苦衷。我们帮了他们,他们就回报我们。现在他们想要隐藏身份,我觉得我们应该再帮他们一把……”

%18
张柱摇头叹息:“小姐啊,你怎么总是这么为别人考虑呢?要知道防人之心不可无啊……”

%19
“张柱叔。”商心慈沉吟道,“如果真的被大劫的话,请一定不要因为守护货物,而去战斗。这些货物丢了也就丢了,没有什么大不了的。我娘的遗愿,是要我带着信物,却到商家城找某个人。但她也说了,如果那个人不愿意接纳我们,我们就靠这些财物生活下去。”

%20
“我娘走的急,要我找的那个人究竟是谁,也没有说清楚。但我想,钱财都是身外之物。娘亲已经离我而去了,张柱叔你和小蝶是仅剩下的亲人了。我不想看到你们再出事。”

%21
“小姐,你千万不要这么说……”张柱眼眶泛红,哽咽起来。

%22
“来一来,看一看了啊,正宗的沈佳丝绸!”

%23
“各种美酒,欢迎大家品尝。”

%24
“一气金光蛊,只卖五十块元石!”

%25
……

%26
叫卖声此起彼伏,人流涌动,在这个临时搭建的集市中,一片喧哗热闹。

%27
每当有商队前来,山寨就像是迎来了节日。

%28
不单单是商队售卖,也有金家寨的人过来兜售物品。

%29
最常见的,就是大量的黄金雕塑或者器具。

%30
锅碗瓢盆,以及人物、动物的雕像,刻工深厚,惟妙惟肖。再搭配上红绿黄蓝的宝石,或者珍珠,更显得华美精致。

%31
黄金山上因为得天独厚的地利,黄金几乎俯拾即是。

%32
生活在这里的人,哪怕是衣不蔽体的家奴,也带着些许的金戒指,金项链等等。

%33
许多少女,戴着的发簪、耳环、手镯,都是金光闪闪,锦绣华丽。她们三五成群,莺声燕语,别有风情。

%34
至于金家的蛊师,身上的穿戴和青茅山大体一致。上身短衫,下身长裤,有腰带、绑腿,脚上是青色的竹芒鞋。

%35
只是有的人的绑腿,是用掺了金丝的麻绳绑的。而有的腰带,衣服袖口或者裤脚,都镶着金边。算是黄金山这边的特色。

%36
南疆家族中蛊师的穿戴,皆大同小异。魔道蛊师的话,则是各种奇装异服。

%37
方源和白凝冰穿梭在人流当中,已经寻找了三四个金家寨的人,购买了一些牛奶、羊奶。

%38
为了喂养空窍中的骨枪蛊,方源已经尽了全力。但即便如此,已经饿死了三分之二的骨枪蛊了。

%39
“你这样大肆购买,不怕被查出来暴露身份吗?”白凝冰表示疑惑。

%40
“只要是伪装,就必定会有暴露的一天。我倒是无所谓,但是你身上的破绽太大。”方源看了白凝冰一眼,道。

%41
白凝冰顿时发出一声冷哼。

%42
她知道自己的破绽在哪里,那就是自己的性别。

%43
当初在村子里,就连老婆婆都能看出来。女人和男人的生理特征是有区别的,这个要伪装起来,必须要特定的蛊虫,可惜白凝冰没有。

%44
因此哪怕她穿着宽松的衣服,带着草帽,涂着黑灰,绑着裹胸,时间一长,也不能避免身份的暴露。

%45
方源继续道:“所以,与其隐瞒下去,还不如主动暴露一些,让他们觉得看破了许多东西,认为局势在他们的掌控中,会比较安心。”

%46
暴露是必须的,并不是坏事。要接近他们,就得暴露身份,只有这样才能获得信任。

%47
方源不可能主动去摊牌,这样做和他们之前表现的很不相符,太不自然。

%48
只有那边主动发现,然后试探,方源反倒能借助这个台阶,顺势暴露一些东西。

%49
白凝冰恍然:“这么说,你是特意等着他们发现,然后再做应对?”

%50
“你终于聪明一回了。”

%51
“哼!”

%52
然而三天的时间过去,方源期待的反应和试探,却迟迟不来。

%53
白凝冰终于找到挖苦方源的机会:“原来你也有算错的时候。”

%54
方源冷哼一声,心中琢磨:“从张柱的神情举止可以发现,他早就怀疑了。先前不试探,大概是因为在路上随时会遇到危险,因此忍耐不发。但现在商队依靠金家寨,十分安定,试探应该早就来了才对。除非……”

%55
方源的心中,浮现出商心慈的容颜。

%56
“真是聪明人呐,也具有魄力。应该就是她,阻止了张柱吧。有点麻烦,太聪明也不好啊。”

%57
方源叹了一口气。

%58
商心慈的温柔善良深入人心,让他有些低估了少女的聪慧。

%59
商心慈想要和方源达成聪明人的默契,双方心照不宣,揣着明白装糊涂。但方源的目的不同,这层默契反而成了他的阻碍。

%60
“既然如此,那我只好主动点了。”想明白这一层,方源叹了口气,主动找到商心慈。

%61
“你想要与我合伙?”帐篷内,看着阐明来意的方源,商心慈以及张柱的脸上都流露出诧异的神情。

%62
自己没有找他们俩,他们反而主动找上门来了。

%63
这有点出乎少女的意料。

%64
而张柱则心中一动:“终于要露出狐狸尾巴了吗?合伙……哼!”

%65
“张小姐,说来惭愧,我们需要元石,而在下对于行商自认有些心得。在下想要借贷一批货,赚到的元石,和你五五分账如何?”方源微微欠身,不卑不亢。

%66
“没有元石,一穷二白,就想借鸡生蛋?你也太自信了吧!”张柱眼中闪着冷光,“你凭什么觉得,自己一定能赚到钱?又凭什么觉得,我们张家一定会借贷给你呢?”

%67
“做生意,当然有赚有赔。我当然不可能稳赚不赔。至于为什么,可能是觉得张小姐你是个好人,应该会借给我吧。你要问我原因,我只能回答你,就是这么觉得的。如果我的感觉错了,就当这件事情没有发生过吧。”方源笑了笑。

%68
他缺少一只耳朵,身上全是烧伤的痕迹,笑起来有些可怖。

%69
但商心慈却看着他的眼睛,从中发现一种自信、决断、运筹帷幄的光彩。这种光彩,反而从丑陋中衬托出一种别样的魅力。

%70
“真是有趣,看来他也察觉到了我们的怀疑。所以,想和我达成聪明人的默契吗?”商心慈目光闪烁着。

%71
片刻后,她笑起来。

%72
这种别样的“坦诚”的交流方式,让她感觉到莫名的安全,还有新鲜感。

%73
“如果没有黑土你的话,我的货物剩不到四分之一。早在匪猴山,被那些猴子抢走了。既然你有这个想法,那这些货物就都交给你掌管好了。”她道。

%74
若是丫鬟小蝶在此,恐怕立即要大呼小叫了。

%75
方源表面上则楞了一下,欠身表示感谢。

%76
“小姐?你这是……”等到方源离开帐篷,张柱终于忍耐不住。

%77
商心慈调皮地眨眨眼,像个孩子:“这不是很有趣吗?你听听他刚刚说的话,还未做生意呢,就要和我五五分账。这语气,就好像必定能赚到钱似的……”

%78
“哼,他一个莽汉,有多少才华?”张柱不屑地嗤笑一声,“要说做生意的头脑,谁人能比得上小姐你呢?这么多年来,小姐你如何操持家业,发展壮大的。老夫历历在目呢。要不是张家那些小人眼红嫉妒……”

%79
“好了,过去的事情说得干什么呢。既然张柱叔你也认可我的才华,那么就该相信我。哪怕他黑土将这些货物都败光,我也能重新白手起家,不是吗?”商心慈道。

%80
“当然!”张柱斩钉截铁地道。

\end{this_body}

