\newsection{苍天无眼啊}    %第九十四节:苍天无眼啊

\begin{this_body}

“终于到达了商家城。”望着满山的建筑,百风叹了一口气。

他是百家的家老,当初他年轻时,游历南疆,就曾经来过商家城。如今再来,青春不再,物是人非,却是带着追缉魔道蛊师的任务而来。

“你们说,那两个魔道贼子真的走了这条路吗?”百莲语气担忧。

“应该就是这条路了。我们以无足鸟的坠落地点,分散了几路。其他方向,都没有收获。惟独此条道路,残留可疑痕迹。”一旁的铁刀苦道。

他心中也不太确定。

铁家虽然在追踪方面是南疆第一,但他是攻击蛊师,而且南疆多山林,环境复杂多变,容易藏身,追踪困难。

“也许他们两个早就死在路上,葬身兽口之下了。”一位同伴乐观地道。

这可能性很大。他们沿途发现了许多残骸,以及兽潮冲击商队留下的大量痕迹。

“我倒是希望他们还活着!”百战猎咬牙切齿。他的爷爷就是被方白二人杀死的,他要亲手把方白二人杀死,才能排解心中的滔天仇恨。

“好了,我们先进城再说吧。如果没有发现,就花些元石给族中送信,看看族长大人怎么安排。”百风率先迈出脚步。

一行人风尘仆仆地来到城门口。

巧合的是,这城门就是方白二人当初进入的关卡。

“要进入城池,每人十块元石。”城门的守卫拦下他们。

百风拿出了一道黄梨令牌,晃了晃。

守卫看了,确认无误后便道:“黄梨令牌可免三人的入城费。”

百风一行六人,缴纳了三十块元石。

“这位大哥,你可见过有这两个人进城么?”百战猎手指着城墙上的一则通缉令,问道。

这正是追缉方白二人的通缉令。

但是通缉令的表面,已经被一张新的通缉令,覆盖了大半。

这是常态。

每过一段时间,总会有新的通缉令产生。

城卫勃然变色,对百战猎低喝道:“你这是什么话?有我把守的城门,怎么会有魔道蛊师进入。你当我是瞎子吗?你是在污蔑我,污蔑商家的正直青年!”

百战猎神情一滞。

百风家老连忙道歉,在商家城,他就算是百家的家老,也得收纳起架子。

城卫见百风到底是三转蛊师,也不敢过多追究,只是口中骂骂咧咧。

直到铁刀苦神色不快,亮出身份:“行了,你闭嘴吧。你们商家是什么样,我铁家还不清楚吗?”

城门守卫这才止住话。

百家一行人吃了这记下马威,颇有些灰头土脸,走进了外城。

“我们先吃些饭菜,这些天奔波,都疲累不堪了。好好休息一下,并不耽误追缉。我正好知道第五内城里,有一家不错的饭馆。就在演武区,想当初我也参加过演武,打到第四内城呢。”百风家老提议道。

这建议甚好,赢得了众人的欢迎。

一行人进入第五内城,来到演武区。百战猎、百莲几位小辈,立即就被演武区的热烈氛围所感染。

走的路上,不断地听到路人兴奋交谈的声音。

“坛镜这一次终于报仇了,把施南生打趴下了。施南生扬言要报复回来,这两人仇怨越积越深了。”

“袁空掌握了病云蛾,战力又上升了一段,几乎可以在第五内城的演武场称霸了。”

“称霸,呵呵,等古月方正升上第四内城再说吧。”

……

“古月方正!!!”

百家一行人无不瞳孔一扩,如遭雷击。他们脚步倏地停住,数道眼神闪电般地射向那个说话的路人。

把那路人着实吓了一跳。

半柱香之后。

演武场中,方源施施然走上场。他此次的对手,是一位中年大汉,五大三粗,膀大腰圆,胡须如钢鬃,长得凶神恶煞。

当――!

一声脆响,代表着战斗正式开始。

“我认输!”中年壮汉第一时间,开口大叫。叫得干脆,叫得利落,叫得让周围的观战者大为不满。

“怎么又是这样?”

“丢不丢脸啊,还没打,就认输了。”

“丢脸总比丢命好啊,现在第五内城这里,谁敢和方正打?全力以赴蛊,实在太猛了!”

众人兴叹唏嘘。

战斗刚刚开始,然后便结束了。

主持的蛊师走上来,要去双方的藤讯蛊,修改了内容后,又返还给对方。

中年大汉转身离去,主动认输是一种耻辱,但形势比人强,谁叫方源心狠手辣至极,从不留手呢。他不敢冒险争面子。

方源却没有立即走,而是当众取出四万块元石,交给李然。

这一幕,又再次引发了众人的议论。

“又是四万块元石,真的又给了!”

“这个古月方正虽然狠辣,但说到做到,二十万元石,如今已经给的差不多了。只剩下三万块元石的缺口。”

“李然用这些元石购买蛊虫,在演武区也开始展露头角了,真是叫人羡慕嫉妒啊。”

“有恩必还,有仇必报。从这点上来讲,我很佩服古月方正。”

“这一场胜了,他就是二十九场净胜,还差最后一场,就要升上第四内城。”

“赶紧走吧,在这里谁敢对战他?连李好都死在他的手里……”

……

“古月方正,千辛万苦,终于让我们发现了你!”人群中,百战猎咬牙切齿,脸色恨得都扭曲起来。

其他的同伴,脸色也十分难看。

他们辛辛苦苦追缉这么久,一路奔波劳累,吃了多少苦头。结果到头来,却发现他们要捉拿的人,在商家城混得风生水起,简直是春风得意。

在此之前,他们还不断猜测,方源说不定已经惨死在野兽的口中,被消化成粪便排泄出去了。

此情此景,和他们的预计,形成了强烈的对比。

方源不仅没有困顿,反而活得很好,甚至换了蛊虫,战力和修为统统暴涨。

这叫他们情何以堪!

“苍天无眼啊,怎生叫这恶贼如此得志?”百风仰天长叹。

“他运气也太好了,救下了商家族长的私生女,居然得了紫荆令牌!在商家城里,我们根本动不了他。他受到商家的保护!”百莲恨得牙根发痒,从路人那里她了解到了很多。

杀死两位少族长的仇人,明明就在眼前,但偏偏他们拿方源没有办法。

“不仅是他,还有那个白凝冰,也有紫荆令牌。如今也参加演武,战绩相当不错。”身边的同伴叹息。

“这是什么世道啊,坏人得志,好人却惨死。唉!”

“其实要动他俩,也不是没有办法。他们可以参加演武,我们也可以。”铁刀苦沙哑着嗓子,目光尖锐如刀。

他从百家族长处得知,方源手中有着焦雷豆母蛊。

如今,他已经确信方白二人就是杀害他的少主人的凶手!

缉拿方白二人已经不可能了,那么就杀掉他们俩个!如此才能为少主报仇,铁刀苦才有脸面回转铁家。

“不错!这是个好方法。”百风眼前一亮,精神振奋起来。

借助演武场的规矩,方源就算有紫荆令牌,也保护不了他。就算杀不了他,夺走他的全力以赴蛊,也能遏制他的成长。

说实在话,他的成长速度太可怕了,叫人心惊。绝不能让他这样成长起来!

“等等,百战猎去哪了?”百莲忽然道。

……

方源走出这处演武场,沿途的路人都不由地给他让路。

无数目光,集中在他的身上,有畏有敬,也有冷漠、仇视等等。

“我虽然杀了许多蛊师,但名声却并不臭。每一次当众给李然元石,都是一场造势。二十万元石,还差三万了。”

方源一边走着,一边心中琢磨。

和李好一战,他缴获了三只蛊,却没有背山蛤蟆,也没有移形蛊。

背山蛤蟆被他硬生生打死了,移形蛊是李好的本命蛊,李好死后它也随之消亡。

不过那一战,观战人数众多,方源还是得到了六千多块的元石。

这一个多月来,不断连胜,除了自己留下一小笔元石,其余的凑齐四万,就交给了李然。

因为这番举动,再加上李然的回应,武家方面暂时没有对方源动手。

这就赢得了宝贵的发展时间。

“如今我已有二十九场净胜,再胜一场,就要升到第四内城去。第四内城竞争压力更大,强手如云,不过赢了之后,收获就更多。如今钢筋蛊已经用毕,我可以再添其他兽力虚影。只是……”

只是增添兽力虚影,是个水磨工夫,至少得两三个月,才能完全见效。

方源升到第四内城之后,压力巨大,急需增添新的力量,应对局面。

“最近的一场大型的拍卖会上,有只一蹴而就蛊。我若能得到它,就能一夜之间,获得新的兽力虚影。可惜我元石太少,连参加拍卖的资格都没有……”

钱到用时方恨少!

正当方源思索,该是向谁借钱的时候,一个少年忽然出现,拦住他的路。

“古月方正,你可还认得我!?”百战猎怒吼道。

“战猎!”

“战猎,你不要冲动。”

百家一行人随即赶到,形成一个针对方源的半包围。

双方对峙!(未完待续。请搜索,小说更好更新更快!)

------------

\end{this_body}

