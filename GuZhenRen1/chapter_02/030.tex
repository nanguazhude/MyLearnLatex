\newsection{村庄休整(第二更)}    %第三十节:村庄休整(第二更)

\begin{this_body}

看着方源走到幽豹尸体处,白凝冰不由地被吸引了目光。(138看书.138看书.)

只见前者蹲下来,开始掏幽豹的双耳。

不一会儿,方源就从雄豹的左耳,以及雌豹的右耳中,掏出来两片紫色的小巧树叶。

这是敛息蛊。

三转级别的草蛊,蛊师用之能掩藏自身气息,遮盖修为,起到一定程度的伪装作用。

每一头幽豹的耳朵中,几乎都会有一片敛息蛊。但是幽豹向来都是成双入对的行动,又都至少都是千兽王级的实力,擅长偷袭,速度又快又灵活,捕捉起来非常危险麻烦。

并且,幽豹又是紫幽山附近才有的特有猛兽。因此幽豹耳朵里藏有敛息蛊的信息,还未广为人知。

方源前世一百五十年,出了个正道人物,号称“猎王”的孙干。以他为先,大肆捕猎幽豹,获取敛息蛊,贩卖到市场中,收获暴利,因此发家。

从他之后,无数蛊师赶往紫幽山淘金。仅仅数年之后,幽豹就灭绝了。

不过现在,紫幽山还是一处僻静的地方。

这里,白日安全,夜间极为危险。没有山寨家族,但是却有家族的雏形――村庄。

虽然没有地听肉耳草来侦察,但幸运的是,得到了两只敛息蛊。

方白二人依靠此蛊,规避了许多危险。

紫幽山他们是不会上去的,他们现在的力量,已经可以在普通的山林中跋涉。但是名山大川,却没有实力深入。要探索这些地方,就连百家也要付出相当大的代价。更何况现在的方白二人呢。

他们绕着紫幽山脚前行,两日后,发现了山道。

人为开辟出来的山道,比普通山林还要安全得多。当然,运气不好,也会遭遇危险。

顺着山道前行,在一天傍晚,方白二人发现了袅袅的炊烟。

两人对视一眼,脚步加快。在一处山坳处,发现了一个村庄。

村庄周围砌着低矮的石墙,晚归的农人三三两两,扛着锄子等农具,走入村中。村子周边几处,还站着守卫。

不过这些人,都是些凡人,根本不足为虑。

“走吧。”方源率先走下村子。

“就这样过去?”白凝冰略显诧异。

二人的出现,很快就引来了村民们好奇、怀疑、猜忌的目光。

这个世界的村庄,大多都很排外。家族山寨更是如此,防御森严,唯恐潜入进什么间谍盗贼等等。

“请问两位远道而来的客人,是否是尊贵的蛊师大人吗?”还未到村口,就有两位容貌相似的守卫迎了上来。

白凝冰没有说话,按照原先的约定,一切由方源应对。

方源摇摇头:“两位小哥,俺们都是凡人咧。”

听了这话,两人明显松了一口气,脸色变得轻松起来。

更年轻的一位少年,不屑地打量了方源全身,有些厌恶地道:“我就说嘛,这人这么丑陋,怎么可能会是那些神通广大的蛊师大人呢?”

方源浑身都是烧伤,又缺了一只耳朵,相貌丑陋,让人嫌恶。

而白凝冰也换了普通的装束,一头长长的银发不仅削成短发,还染黑了。她浑身冰肌,因此肌白如雪,此时也故意抹黑。只是眼睛的颜色遮盖不住,于是带了个草帽,遮盖住上半个脸面。

两个人站在一起,活脱脱的两个凡人村民。

“小弟,不要乱说话。”年长的守卫训斥了一番,然后警惕地看向方白二人,“你们从哪里来的,干嘛要到我们这儿?”

“俺们是从山那头的村子来的。本来是拖了板车,带了药草还有腌肉,想出来卖的。唉,路上碰到了老虎。妈呀,吓死俺了。一路飞跑啊,捡回一条命。唉……俺们暂时不敢回去,就来到你们村,想睡一晚。就一晚,明天就走啦。”方源随口就道。

守卫警惕的目光稍减。

方源又道:“大兄弟,你别怪你弟弟。俺这伤是火烧的,那天家里起火,俺为了抢米出来,就烧成这个样子了。”

“唉,这年头,都命苦。”年长的守卫叹了一口气,“你们进村吧。如果没有人收留你们,你们就靠着墙角凑合一晚吧。”

说着,便让开了路。

看着方白二人进去村子,守卫又嘱咐他弟弟:“你现在就去和村子说下,就说有两个村外人来了,他老人家经验丰富,请他再掌掌眼。”

“哥哥,你也忒小心了。你也不想想,就他们这俩货,怎么可能是蛊师?再说了,我们这些凡人,蛊师干嘛骗我们啊?寻开心啊?”

“叫你去,你就去!”

“又叫我跑腿……”年轻人抱怨了一声,终究还是去了。

村庄内一片祥和。

饭菜的淡淡香气弥漫在空气中。一天劳动之后,一家人团圆聚餐的欢笑声,也传到方白二人的耳中。

身处在这样的环境中,令白凝冰不由地感到一阵轻松。

之所以伪装,一来是因为不想暴露行迹,那样将方便百家的追捕。二来,也是方源生性谨慎,在陌生的环境中喜欢藏巧露拙,方便应对突发和异常状况。

要寻到一户人家收留,很是容易,给上一颗碎元石,就足以让他们欢天喜地的腾出主房了。

但这样做,却不符合现在他们俩的身份。

方源有更好的方法。

他在村子里走了片刻,停在一处破落的民房跟前。

这户人家,只有一个老婆婆。原本有个孙子,可惜在外游玩时,遭了狼口。

在屋前,老婆婆正在井前打水,显得很吃力。

“大娘,俺来帮你吧。”方源脸上堆起憨笑,殷勤地跑了上去。

老婆婆看到方源,被他的容貌吓了一跳。

但方源表现热情,一脸憨笑,手脚麻利地帮她打了好几桶水后,老人家的戒心消除了。

“小伙子,你是外村的人吧?”老婆婆笑起来,裂开的嘴里,牙没剩几颗。

“是啊,俺想在大娘你这里住上一晚。大娘,俺替你干活,你看成不?”方源憨憨地道。

“成啊。”老婆婆欣喜地道。虽然平时有村民相互帮衬,但她还是需要这样的劳动力量。

白凝冰在身后看得一阵无语。

这方源这贼厮也太能装了!

打完水后,就是劈柴。方源还主动烧饭,麻利干练的动作让老婆婆连连夸赞。

“大娘,俺再帮你打几桶水吧,水缸盛满了再说。”吃完晚饭,方源又主动提起了水桶出去。

老婆婆连说不用,但方源执意如此。

水缸盛满之后,老婆婆老眼泛着泪花:“小伙子,你人实在啊。唉,可惜命和老太婆我一样的苦……”

显然,晚饭时方源编造的悲苦故事,给淳朴的老婆婆留下了深刻的印象。

对于凡人来讲,灯油也是精贵的,夜晚的房屋中一片黑暗。

只有窗户处,洒下来一片月光。

屋子里有两张床榻,皆十分简陋。不过白凝冰躺在上面,已经心满意足的很。这些天来跋涉的疲惫,在此刻缓缓消散。

方源则盘坐在床榻上,心神沉入空窍,检查骨肉团圆蛊。

这些天,他都没有动用这对蛊。

毕竟是篡改了秘方,合炼出来的。依照方源谨慎的性情,自然需要好好研究一番。

忽然,方源睁开双眼,一抹精光一闪即逝。

“大约没有问题,骨肉团圆蛊可以用了。”说着,他便唤出一对玉镯形状的蛊。

这两只玉镯,一只青色如草,一只红色似血,相互还扣在一起,不能分离。

先前,方源已经将它们炼化。但要发挥它们的妙用,还得舍弃其中一只,让白凝冰去炼化。

白凝冰盘坐起来,接过蛊虫,却没有忙着炼,而是看向方源:“接下来,你有什么打算?”

方源嘿了一声:“我还以为你不会问呢。”

尽管是在黑暗中,但白凝冰也能感受到此刻方源脸上挪揄的笑容。

她哼了一声。

方源也没想过瞒她:“接下来的目的地,是商量山。”

“商量山,商家?”白凝冰不禁微微挑眉。

商家是南疆首屈一指的势力,不弱于铁家,飞家,只是被武家压过一头。

商家以贸易威名南疆,甚至出了南疆,天下人但凡有见识的,都知道南疆的商家是贸易的中心,商家城是如何的繁华似锦,遍地元石。

白凝冰还是北冥冰魄体的时候,就向往过什么时候能去商家城看一看,但是此刻她却担忧起来:“我们在百家犯下案子,恐怕要被正道通缉了。去商家城,岂不是自投罗网?”

方源笑了笑:“如果整个南疆,只剩下两个地方能容得下我们。那商家城必是其中之一。商家虽然是正道领袖之一,但商家城确实不折不扣的自由之地,魔道中人最大的销赃处。不然你以为,商家为何是南疆首富?就连武家都在这方面,望尘莫及呢?”

白凝冰听了,不禁悠然神往:“传闻都说,在商家城什么东西都买的到,是否真是这样?”

方源摇了摇头:“说这话的,都是层次低的。这个世界上,有太多的东西有价无市了。就比方说――某只阳蛊?呵呵呵。”(未完待续。如果您喜欢这部作品,欢迎您来138看书文学注册会员推荐该作品,您的支持,就是我最大的动力。)

------------

\end{this_body}

