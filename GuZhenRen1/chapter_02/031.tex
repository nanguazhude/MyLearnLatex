\newsection{合作修行(第三更)}    %第三十一节:合作修行(第三更)

\begin{this_body}

黑暗的屋子内,白凝冰发出一声不悦的冷哼。[138看书.138看书.]

方源话锋一转:“不过现阶段而言,商家城是最适合我们的地方。那里商铺众多,完全可以买到适合自己的蛊虫。我们现在的蛊虫,纵然优异,但是却相互不搭调。只有搭配成套,才能发挥出更强大的战斗力,甚至越级挑战都有可能。”

方源此话,令白凝冰无比赞同。

但凡优秀的蛊师,都会有一套成熟的蛊虫搭配。

就比方白凝冰还是北冥冰魄体的时候,冰刃蛊、冰锥蛊、水罩蛊、蓝鸟冰棺蛊、冰肌蛊、霜妖蛊……

都是冰水一系的蛊虫,相互搭配使用,极其顺手,相得益彰。

但现在,他们二人手中的蛊虫虽多却杂乱得很,或是换上成套的蛊虫,战斗力将至少暴涨一倍。

正巧方源空窍中,囤积了大量的骨枪蛊,以及螺旋骨枪蛊。这些东西,正可以贩卖到商家城去。

就算没有这些蛊,方源也有天元宝莲,可以日产元石。

去商家城暂避风头,是第一目的。购买换取蛊虫,是第二目的。方源的第三目的,却不方便和白凝冰分说了。

这就要牵涉到解石。

在某个赌石场中的阴暗角落处,一块石头中暗藏着一只传奇色彩的蛊!

方源手上的蛊虫,有高达六转的春秋蝉,有牵扯到九转仙尊的天元宝莲,有篡改资质的血颅蛊,还有骨肉团圆蛊。

且不说骨肉团圆蛊了,血颅蛊几乎已经失去了价值,因为和方源血脉相关的族人几乎死的一个不剩了。天元宝莲,虽然效果卓越,但是没有发展的前景,方源不知道合炼秘方,方源修为越高,它的作用就越小。

春秋蝉就更扯淡了。

现在还在休养沉眠之中。你说你用它吧,有风险。你不用它吧,它恢复起来,又会撑爆空窍。

这货简直就是个定时炸弹。

春秋蝉、天元宝莲、血颅蛊、骨肉团圆蛊,算得上是方源手中最有价值的四蛊。并且这些蛊,都有个共同特点。

那就是,都是用于辅助修行的蛊虫。

它们带给方源的帮助大不大?

大!

巨大!

但是对于方源的战斗方面的帮助,却不是那么明显了。

商家城的那只蛊,却不同了。它是用来战斗的王牌,有了此蛊,再增添一些其他的蛊,搭配成套。那么方源才算是自重生起,有了一些同级无敌的架势,恢复前世一点魔道巨头的风采。

“要赶往商家城,还有很长的路途。先别提那么遥远的事情了,你还是先将这骨肉团圆蛊炼化一只罢。”方源收回话题。

白凝冰打探出了方源的计划,也就稍稍安心了一些。

和方源相处的时日越久,她就越发觉得方源太过阴险狡诈了。

尤其是今天他的表现,憨厚得连自己都差点要骗过去!在不久前,她亲眼目睹方源如何炼制骨肉团圆蛊的,前后联系方源的表现,她越发觉得心寒。

她暗暗提醒自己要小心,那个悲催的百家女族长就是前车之鉴呐!若不提防着方源一点,不定哪天自己被方源卖了,还在替他数钱呢?

微微摇头,白凝冰收拾好心中的情绪,她开始着手炼化骨肉团圆蛊。

这对蛊形状奇特,一青一红,白凝冰选了个红色的,将真元催动上去。

这骨肉团圆蛊,早在事先,就被方源炼化掉了。如今方源主动消褪意识,配合白凝冰。

后者轻而易举,没有花费太多的时间,就将其炼化。

她刚刚一炼化完毕,骨肉团圆蛊就发生了异变。

这一对相扣的玉镯,忽然完全消散在半空中。而几乎同时,方源和白凝冰的手腕上,各出现了一道环状的痕迹。

所不同的是,方源左手腕上的环痕,是青色的。而白凝冰右手腕上的环痕,却是红色的。

一股玄而又玄的感应,从方白二人的心中一齐升腾起来。

这种感应,让方源可以感觉到白凝冰,同时也让白凝冰感应到方源。仿佛是血肉相连,不可分割的那种感觉。

这种感觉可不太妙,让白凝冰像吃了苍蝇般恶心。

方源也紧紧的皱起眉头。

灰骨才子留下的卷轴中,可没有说明白这一点。

不过方源转念一想,旋即又释然:“这灰骨才子,也只是理论上的研究,从未使用过的经验。不知道也很正常啊……”

这点感觉只是细枝末节,接下来才是重点。

两人忍住心中的古怪,开始面对面盘坐在一张床榻上,开始双修。

他们分别伸出手掌,四掌掌心相对。

为了稳妥起见,先由修为薄弱的方源开始试验。

他从空窍中,调动出一股青铜真元,小心翼翼地灌入到白凝冰的空窍力。

青环和赤环陡然间,散发出明亮的光彩。

原本带着异种气息的真元,投入到白凝冰的空窍中后,立马转变得完全和白凝冰的气息别无二致。

“真的成了!”白凝冰轻轻的欢叫一声。

“但是我明明调动了一成的青铜真元,到了你的空窍中后,却只剩下六分,足足削减了四分。”方源的观察体会则更加细微。

白凝冰却不奇怪:“这有什么难以理解的。那个卷轴上不是说了吗?这骨肉团圆蛊,根据炼制两人之间的情谊深浅,分为五种品质。从低往高,分别是:骨肉相残、骨肉相连、情同骨肉、亲如骨肉、骨肉至亲。”

她继续分析道:“如果那对兄妹成了蛊师,单独合炼的话,依他们之间的感情,估计是最高品质的‘骨肉至亲’吧。若是我们单独炼制,凭我们之间的关系,嘿嘿,恐怕定然是最差品质的‘骨肉相残’了。如此两方一中和,因此才是这不高不低的‘情同骨肉’。”

这骨肉团圆蛊,是指的一个系列。类似豕蛊中,有黑豕蛊、白豕蛊、粉豕蛊等。

按照卷轴中所述,骨肉团圆蛊包含五种蛊虫。

最差品质的,是骨肉相残蛊,一成真元只能转换两分。

在之上,是骨肉相连蛊,一成真元能换得四分。

而情同骨肉蛊呢,则能有六分真元剩下。

到了更上一阶的亲如骨肉蛊,一成真元转换过去后,能剩下八分!

至于最高的骨肉至亲蛊,百分百的转换,没有任何消耗。

方源依着前世经验,临时篡改秘方,能得个三转的情同骨肉,也算运道不错了。

需知要研炼秘方,十分不易。得靠不少蛊虫,不断的推演,不断的实验。

方源能够篡改成功,一来是因为经验带来的灵感,二来是因为运气不错。

不过这撞运气的事情,他向来不喜欢。

这倒并非他天生是个扫把星,运气一直不好。方源的运气和普通人一样,时好时坏,但他不喜欢这种脱离掌控的因素。

他是个控制欲很强的人,喜欢操控局面,控制别人,当然也控制自己。

“就这么着吧。下面,该你来了。”方源说道。

总体而言,他对情同骨肉蛊这个结果,还是满意的。

白凝冰便调动一成雪银真元,灌注到方源的空窍当中。

这一下,可不得了!

刚刚,方源的深绿真元到了她的空窍中,形成一片绿水浮在海面上。不一会儿,就被雪银色的海潮吞没。真元海面只是微微涨了一丝。

如今,她的一成真元到了方源的空窍中,还未落入真元海,方源的整个空窍都颤抖起来。

方源连忙喊停。

这却不是骨肉团圆蛊出现问题,而是他的空窍只是一转高阶的程度,又装载了数百只的骨枪蛊,如今再装载雪银真元,就有不堪重负的危险了。

白凝冰犹豫了一下,如果此时灌注真元进去,说不定能令方源的空窍撑破。

不过就算撑破空窍,阳蛊也会在方源的一念之间被摧毁。

她想了想,最终还是将这股真元又调回去。

这成雪银真元原本是一成,到了方源的空窍时,转变了气息,只剩下六分。如今再回到她的空窍当中,又发生缩水,只剩下三分半有余。

倒是让白凝冰心底暗暗感叹,这骨肉团圆蛊的神奇。

方源将空窍中的骨枪蛊,都移到白凝冰的空窍当中去,这才继续接引雪银真元。

这股雪银真元,在白凝冰的体内时,还是她的气息。但是一旦到了方源的体内,就骤然缩水,转变成了方源之物。

这股雪银真元落入方源的空窍中,当即沉入底部,而他的青铜真元只能无奈地浮在上面。

两者和平共处,相安无事。仿佛这股白银真元,本来就是方源空窍中产生的。

若换做异种真元,这样密切的接触,立即就会发生爆炸,导致空窍受损。

因此寻常灌顶如走钢丝,十分危险,劳心劳累得很。

方源尝试着调动这股白银真元,冲刷周围的窍壁。

蛊师分九转大境界,每层大境界中又分初阶、中阶、高阶、巅峰四小境界。

初阶为光膜窍壁,稀薄而闪烁。

中阶为水膜窍壁,光辉流淌。

高阶为石膜窍壁,光芒凝练而稳固。

巅峰为晶膜窍壁,空灵剔透。

方源如今是一转高阶,周围窍壁正是石膜。

但雪银真元稍一冲刷,这层石膜竟然有承受不住,颤动不支的迹象!

(ps:嗯,今天封推,爆发一下更新!啦啦啦……)(未完待续。如果您喜欢这部作品,欢迎您来138看书文学注册会员推荐该作品,您的支持,就是我最大的动力。)

\end{this_body}

