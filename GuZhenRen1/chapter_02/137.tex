\newsection{英雄救美}    %第一百三十七节:英雄救美

\begin{this_body}

山石迸溅,群鳄哀嚎。

一场大战已经步入尾声,方源和白凝冰二人傲立火炭山,脚下的地面附近都是熔岩鳄的尸体。

而那只千兽王级的熔岩鳄王,浑身的骨架都被打折,背部的两个火山凸起,被硬生生地击垮。

它低声哀嚎着,趴在地面上,鲜血汩汩,从身上无数的伤口处流淌出来。它脚爪微微颤动,扒着土地想要钻回地底去,却已经力不从心。

它挣扎的幅度越来越小,最终死亡降临到它的身上。

千兽王一死,其余剩下的熔岩鳄顿时分崩离析,钻入地底,仓惶而逃。

稍微打扫了一下战场,方白二人再度启程。

焦黄、孟土则躲在远处的角落里,没有动身。这两位魔道著名的刺客杀手,此刻的脸色都很不好看,像是僵尸一般。

他们被吓到了!

“这两个还是人吗?竟然单凭一己之力,抗衡整个熔岩鳄群!”

“白凝冰的冰道,对熔岩鳄克制极大。这也就罢了,关键是那方正,简直是个人形怪兽。受伤越重,他就越强。最后轻而易举地,就能直接把熔岩鳄王撞飞。”

焦黄、孟土相互对视一眼,均看出彼此眼中的悸动。

方白二人的恐怖战力,大大出乎他们的意料。

他们并不在商家城生活,此时亲眼目睹了整个战斗过程,他们俩这才深深的明白方白二人的恐怖。

“这两个年轻人,真的只有二十几岁吗?他娘的,和他们一比较,我四十多年的生涯,简直像是活到狗身上去了。”孟土心有余悸地咒骂着。

“孟土老弟,你别这么说。你这样说,让我也无地自容了。”年龄更大的焦黄,深深的叹了一口气。“毫无疑问,这两个人都是天才!人比人,气死人啊。我们俩根本就不是他们的对手。早知如此,这笔买卖就不接了。”

“焦黄老哥,你这么一说,反而激起了我的好胜心,让我更加不甘心。这事情还不算完。我们虽然打不过他们,但我们还有机会!”孟土吐了一口吐沫,狠狠地说道。

“哦?是什么机会?”

“你想啊,焦黄老哥。他们两个是去三叉山,想要在三王传承中捞取好处的。三叉山那边,可是混乱无比。有四转、五转的强者。他们到了那里。一定会有冲突。我们瞅准时机,如果能捡到便宜,那就最好不过了!”

被孟土这么一提醒,焦黄的双眼也亮了起来。

他伸手拍拍孟土的肩膀:“老弟,你说的很有道理。走,我们也跟去三叉山!”

“刚刚那群熔岩鳄,出现的有些蹊跷。”走在路上。方源若有所思。

这群熔岩鳄出现的时机,方位都太巧了。一出现,就牢牢包围住了方白二人。白凝冰觉察不出什么,但拥有前世经验,老谋深算的方源,却从中嗅出了一丝阴谋的味道。

对这种味道,方源再熟悉不过了。

“有人故意设计了这个陷阱,想要对付我。那么究竟是哪一方呢?武家、百家还是商家?”方源暗中琢磨着。

“武家是因为我知道李然的身份。我现在一出商量山,他们出手很有可能。”

“而百家呢?我和他们结仇,掌握他们家族元泉干涸的大秘密,又敲诈了他们三百万元石。能不恨我吗?”

“还有商家。我在商家得罪的人也很多。商睚眦、商一帆,买下卫家那群人,又得罪了商蒲牢。商家少主相互竞争,我是商心慈的臂助之一。若是在外被铲除,商心慈就被削弱了。”

“算了,不去想了。兵来将挡,水来土掩吧。”方源摇摇头。将繁杂的思绪尽数排除脑外,心思清定下来。

换做以前,他实力弱小,凡事要殚心竭力,谋算一切。但现在实力增长上来,已经有一丝八风来袭,我自岿然不动的意味。

……

中洲。

云海上,狂风呼啸。

上万只飞鹤,在一齐展翅飞翔。

方正等仙鹤门的精英弟子,各踩踏或者盘坐在飞鹤的背部,赶往天梯山。

“方正老大,您的鹤群真是威武雄壮。此去天梯山,必定能大放光彩,横扫四方。”一位精英弟子道。

他说话时,动用了蛊虫。任凭周围风声呼啸,也阻挡不住他的声音,清晰地传入众人耳帘。

“诸位师兄弟抬举我了。这次区域天梯山,参加白狐蛊仙传承之争的,都是十大派的精英。我们若要取得传承,不仅需要实力,更需要运气。”方正道。

“方正老大,你太谦虚了。凭借你的上万鹤群,谁人能挡?”立即就有一位精英弟子道。

“方正老大,你就是我辈的楷模。难怪这次你刚晋升成精英弟子,就被掌门派遣出来。这次去天梯山,我等几人唯您马首是瞻!”一位精英女弟子恭敬地道。

来的一路上,方正和这些精英弟子都一一切磋过。

他实力强大,空窍中又有寄魂蚤。天鹤上人的魂魄就寄托在寄魂蚤中,随时随地为他临阵指导。又有上万的鹤群撑腰。

因此,方正将其他几位精英弟子,都一一击败。

他胜不骄,有气度,更谦虚,很容易就博得众人好感,被众人推为首领。都对方正钦佩不已。

“中洲十大门派,哪一个不是底蕴深厚?相信他们之中,也必有能人。我虽然有上万的鹤群,但也有指挥不力的弱点。关于指挥飞鹤,还需要和诸位多多讨教。”方正说着,向身边的人拱手。

“不敢,不敢。能和方正老大切磋,也是我们的荣幸。”

“这些天来,方正老大你勤修苦练,努力程度教我等汗颜。”

“方正老大,你进步神速,对控鹤极有天赋。先前只是缺少操练罢了,假以时日,必定能凌驾于孙元化之上。”

其他精英弟子们你一言,我一语地回应道。

他们的话语,完全是发自内心。这一路上,方正的进步他们有目共睹。

方正不禁笑了笑。他有天鹤上人随时教导,传授秘密心得,甚至有时候还代己操纵,能不进步神速吗?

又飞了一段时间,铁喙飞鹤们纷纷叫唤起来。

方正等人均心中了然。

“好了。时辰到了,我们下去,该给飞鹤喂食了。”方正一顿足,飞鹤群便在他的调动下,开始一个个地往下方的云层扎去。

顿时,众人周围都是一片白色的迷茫。

很快,云雾消失,众人随着飞鹤群飞下云端,扑向郁郁葱葱的地面。

飞鹤也需要进食。鹤群规模越庞大,对食物的需求就越高。不过幸好铁喙飞鹤,什么都吃。有时候甚至能吞吃碎石子果腹,十分容易养活。

方正拥有这支规模巨大的鹤群,但也比较麻烦。每隔一段时间,都要飞到地面上,给鹤群喂食。

“咦?下方有战斗!”在下降的过程中,忽然有一位精英弟子开口。

众人旋即发现了下方的异状。

四位魔道蛊师,发出阵阵阴笑,将三位女蛊师包围起来,正慢慢地逼近。

“我呸,是那四大淫贼。”很快,就有精英弟子带着厌恶的语气,道破那四位魔道蛊师的身份。

这四大淫贼,分别是东淫陈淫道,西贱郁八光,南骚施暴,北荡樊春耀。

他们一直混迹中洲,各个都有四转修为。配合起来,更能抗衡五转蛊师,十分强大。

“快看,被他们包围的竟然是天莲派的碧霞小仙子!”一位目光犀利的精英弟子,开口叫道。

“哼,魔道中人,人人得而诛之!”方正脸色冷峻无比,没有想太多,立即驱动鹤群猛扑而下。

“嘿嘿嘿,碧霞仙子,今日你在劫难逃了!”

“想不到今天运气这么好,能得到碧霞小仙子的芳泽。就算是受到再重的伤势,也值得了。”

四大淫贼挤眉弄眼,向天莲派的三位女蛊师渐渐逼近。

“可恶。”碧霞小仙子差点将一口银牙咬碎,她受了重伤,想要突围,却有心无力。

就在她渐渐感到绝望,要咬舌自尽的时候,忽然头顶上空传来群鹤齐鸣。

“什么人?”四大淫贼齐齐抬头,齐声喝问。

“仙鹤门精英弟子,方正!”方正踩踏在铁喙飞鹤王的背上,舌绽如雷。

他傲立鹤背,身躯雄健,浓眉虎目,紧紧地盯住四位淫贼,然后伸手一指。

身后的精英弟子们,还有上万只铁喙飞鹤,纷纷越过他,对四淫贼展开冲锋。

“我的天,这么多鹤!”

“是十大门派,仙鹤门的精英弟子……”

“糟糕,我们身上有伤,不是这群人的对手,先撤先撤!”

四大淫贼审时度势,掉头就跑,很快就跑得远远的,背影消失在众人的视野当中。

“这些魔道的渣滓,别的本事没有,就是跑的快。”精英弟子们哈哈大笑。

“你没有事情吧?”方正走下鹤背,来到碧霞小仙子的面前,温和地道。

“我,我没事……谢谢方正公子的救命之恩!”碧霞仙子看着方正,满脸红霞,目光中带着痴迷。

她原本以为自己这劫难逃,但没想到有英雄冲天而降。

方正英雄救美,给碧霞仙子留下了深刻的印象。(未完待续。如果您喜欢这部作品,欢迎您来起点投推荐票、月票,您的支持,就是我最大的动力。手机用户请到阅读。)

\end{this_body}

