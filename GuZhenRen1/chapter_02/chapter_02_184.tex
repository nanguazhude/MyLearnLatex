\newsection{宇宙奥义,红莲魔尊}    %第一百八十四节:宇宙奥义,红莲魔尊

\begin{this_body}

%1
“年轻人,仙元的消耗有点小小的出入,就此收手罢。”地灵也察觉到了这丝误差,提出了它的建议。

%2
方源面色一冷,断然拒绝道:“不行,八成三不是我的计划,我的计划是九成!”

%3
“世间不如意之事,十之八九。怎么可能处处顺心?年轻人,你已经被利益蒙蔽了双眼,头脑开始发热了。”地灵冷静地道。

%4
方源心中不禁咒骂一声,没有地灵的帮助,他是不可能调动仙元的。地灵现在已经开始打退堂鼓,方源必须要再次说服它。

%5
“霸龟。”方源吐出一口浊气,缓了缓语气,“仙元消耗超出一丝,也没有什么大不了的。我的计划中,还有留下来备用的仙元。”

%6
方源计划中,将仙元分作了十六份。

%7
其中八份用作炼蛊,四份用来支撑三王传承。三份多一点,用来斩杀蛊师。剩下一份不到的仙元,留作备用,用来应付突发情况。

%8
方源心思缜密,又为此事筹算良久,怎么可能没有后备手段?

%9
但地灵却没有被说服,它虽然是执念所化,却也有智慧,可以思考。

%10
“年轻人,你的算法其实不对。你有没有考虑过,炼蛊过程中的失败?你不可能不失败,一旦失败,就会浪费仙元。八份真元,用作炼蛊,本就捉襟见肘,剩下的那一份不到的仙元,正是用作浪费,允许你多次尝试。现在你将这份用掉,你炼蛊时失败次数稍微一多,第二空窍蛊就绝不会炼制成功了。”地灵反驳道。

%11
方源暗暗焦急,他现在还没有通过考验,不是福地之主,无法直接命令地灵。

%12
地灵若是不配合,他也没有办法。

%13
现在的福地中,还有一位五转蛊师。许多四转高手。这些都是不久后,他炼蛊时的生死大敌。

%14
方源现在斩杀这些人,就是提前铲除敌手,保护自己。

%15
但福地中剩下来的仙元,实在太少了。方源必须绞尽脑汁,进行精打细算。同时,还得面对地灵的严格审查。

%16
“年轻人。不要冒险了。如果你一意孤行,我是不会配合你的,更会取缔了你的考验资格,把机缘留给其他人。”地灵语气坚决。

%17
方源闻言,不禁深深叹气。

%18
地灵都是很固执的,这种情况下。要说服地灵几乎不可能。

%19
“别人可以不除,但剩下的那个五转蛊师,必须要死。他是南疆四大医师之一,号召力极强,必须除掉才能安全。”方源尽力地争取道。

%20
“炼蛊时的安全,你已经安排了那位少女。结合犬兽,完全可以抵挡一二。届时。我也会拼死护你。真正重要的,还是你炼蛊的成败。”地灵否决了方源的提议。

%21
方源面色一沉。

%22
剩下的五转蛊师仇九,号称杀人鬼医,虽然战力不显于世,但影响力极大。

%23
前世记忆中,他上了义天山,加入魔道一方,后来居上。直接坐上了第三交椅。振臂一呼,就有无数魔道蛊师云集。

%24
在他的治疗下,魔道一方伤亡锐减,士气大振,让正道无比头疼。

%25
直到商燕飞请来商家城的素手医师,配合圣手神医,才堪堪压住仇九的风头。

%26
义天山一战。让世人认识到仇九的厉害。杀人鬼医,也被公认为四大医师之首。

%27
义天山被正道攻克后,仇九被俘,武家族长怜惜他的才华。想要招揽他。但仇九拒不受降,痛声咒骂商燕飞以及素手医师,暴露出一段陈年恩怨,最终被恼羞成怒的商燕飞斩杀当场。

%28
但凡,五转蛊师,各个都是人杰,绝不能小觑。

%29
这些蛊师,能够从芸芸众生中脱颖而出,在这样残酷的环境下,大浪淘沙,优胜劣汰,登上世俗之巅峰,都是枭雄豪杰。

%30
一想到仇九这样的人物,将在炼蛊的最后关口攻杀过来,方源就寝食难安。

%31
“我的修为,虽然上涨到四转高阶,修行速度惊人。但在这样的大舞台上,还是薄弱了,无法和这些五转蛊师媲美啊。若是有前世六转的修为,不管来多少五转,都尽数杀了,如屠猪狗!”方源心中暗暗叹息。

%32
“你还想杀人?不可能,我不允许!仙元不能浪费在这里,要留下大部分用作炼蛊。”地灵听了方源的要求,却断然拒绝,“魔道蛊师就是这样,喜欢冒险。唉,年轻人你不要太激进了。这个世界上,为什么正道荣昌,永远压过魔道一头?那就是因为魔道激进,太过贪婪,把自己陷入绝境。而正道求稳,稳着稳打,岿然不动。”

%33
“霸龟,你错了。魔道生存不易,不激进,不在每次机会中,努力去攥取最大的利益,不冒险,怎么能修行,怎么能和正道争斗?魔道的贪婪、自私、激进、冒险,正是魔道的生存法则。只要利益巨大,在悬崖上走钢丝又何妨?一步踏错满盘皆输,将不可能化作可能,正是魔道的精彩之处!魔道的生命,就像是酒,浓烈芬芳啊。”方源大声地反驳道。

%34
地灵听到这样的话,叹息连连:“年轻人,你已经魔性深重,再不回头是岸,将来必走向毁灭。天作孽尤可活,自作孽却不可活。”

%35
方源大笑:“哈哈哈哈,霸龟,你太愚昧了。什么是作孽?作孽不可活,那不过是弱者们美好又无奈的期望,他们广为宣传,希望人人遵守,这样就能保护自己。而我就是要做一个作了孽,却仍旧逍遥法外,纵横天地,屠戮生灵,享用一切的魔头。我不仅要屠戮蛊师,更要炼成第二空窍蛊,成为最大的赢家!霸龟,你且感受一下,这是什么?”

%36
说着,方源催动春秋蝉,第一次外泄出春秋蝉的气息。

%37
地灵震惊了!

%38
“这,这是六转仙蛊的气息!好像,好像是春秋蝉……春秋蝉名列天下奇蛊第七!你一个区区凡人,怎么能拥有这等仙蛊?”

%39
为了说服地灵,方源毅然去赌。主动暴露春秋蝉的存在。

%40
春秋蝉是六转蛊,但第二空窍蛊同样是仙蛊,这样的利益足够促动方源去冒险。

%41
“凡人为什么就不能拥有仙蛊?实话告诉你,春秋蝉就是我炼制的。”方源又道。

%42
“我明白了,我懂了。原来你本是蛊仙,借助春秋蝉重生,来到过去改变历史!”地灵惊叹道。

%43
“哦?霸龟。你似乎知道春秋蝉一些情报,说来听听罢。”方源连忙请教。

%44
“从太古起,蛊师中就有两大流派,分别是宇道、宙道。宇和宙,是构成天地的基础。其中,宇代表上下左右。宙表示古往今来。就拿我的福地来讲,全盛时,宇上有九千亩地盘。宙上有六倍光阴流速。”地灵道。

%45
九千亩有多广呢?搁在地球上,就是朝鲜国土面积的一半。

%46
而六倍光阴流速,是根据光阴长河所讲。

%47
在全盛状态时,福地中的光阴是外界天地的六倍。在福地一天,外界就是六天。

%48
当然。到了现在,福地衰弱,不仅地盘被消磨了巨大多数,而且光阴流速也只有三倍了。

%49
这个世界上,还有许许多多不同的福地,面积和光阴流速都各有不同。

%50
地灵继续道:“春秋蝉就是宙道的蛊虫之一。当年有个名垂青史的主人,那就是红莲魔尊!他凭此抗衡仙庭,打破命运枷锁。带给后世天下人福音,从此人民将命运掌握在自己手中。”

%51
“红莲魔尊?!就是那位史上,最为神秘的魔尊?”方源惊异。

%52
蛊道中,九转为尊。

%53
魔尊就是九转魔道蛊师,仙尊就是正道九转蛊师。

%54
漫漫历史长河,魔尊、仙尊屈指可数,代表着无上的传奇。每一个大时代。都只出现一尊,从未有二尊同世的情形。因此尊者,代表的就是真正的,彻底的天下无敌!

%55
而这些九转尊者当中。红莲魔尊最为神秘,后世的记载最少。就连方源也只知道,有过这么一个大人物。

%56
现在,他还是第一次,听到有关红莲魔尊的具体信息。

%57
地灵徐徐地道:“这个世界,广袤无比,分为东南西北中五大域,既连绵成片,又相互独立。又有一条光阴长河,源自过去、流经现在、通向未来。这就是大世界的宇和宙。”

%58
“如果说历史,就是一个个的静止画面。那么光阴长河,就是一条细线。无数的画面,就串在这条线上。你凭借春秋蝉,便能破画而出,顺着细线,逆流而上,将意志和记忆,灌注到过去的一个画面当中。这个画面发生变化,之后的画面也会相应地转变。”

%59
“原来如此。”地灵的话,让方源受益匪浅。

%60
他总共动用过两次春秋蝉,每一次都有特别的体验,让方源对于春秋蝉,对于光阴长河的认知,也越来越清晰。

%61
地灵的这番话,更是提点方源,使得他产生了一股非同寻常的明悟。

%62
“想不到,你居然是未来的蛊仙,还炼成了春秋蝉。在这福地当中,你的确是炼蛊的最佳人选。也罢,既然你这么有自信,那就依你这次,你还要斩杀什么人?”地灵松了口。

%63
原先,它以为方源只是一介凡人,炼蛊时候必定会有仙元的大量浪费。

%64
而现在,它认识到方源蛊仙的身份,心中估计仙元浪费的程度,大为缩减,因此接纳了方源的建议。

%65
方源笑着承诺道:“地灵,你相信我是对的。我不会让你失望,炼蛊必定成功。将来,我还要靠着这第二空窍蛊,重新登上蛊仙之境呢。”

%66
地灵也笑了,对于方源产生了认同感:“不错,你是蛊仙,比凡人更加明白第二空窍蛊的价值。当你成为蛊仙时,第二空窍蛊将带给你无以伦比的收益!”

%67
(ps:修改了一个bug,关于神游蛊、定仙游蛊的转数问题,在第二章一百零八节。感谢热心读者的指正,谢谢。本书首发,更改的内容在起点可看。)(未完待续。如果您喜欢这部作品,欢迎您来起点投推荐票、月票,您的支持,就是我最大的动力。手机用户请到阅读。)

\end{this_body}

