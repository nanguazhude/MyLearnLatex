\newsection{背甲蛊和鳄力蛊}    %第五节:背甲蛊和鳄力蛊

\begin{this_body}

一般三转蛊师,要对付兽群,都是游击战。[138看书\&文学无弹窗小说网]不断利用元石,回复空窍真元。白凝冰这般硬抗硬打,坚持到现在,已经相当不易了。

蛊师若真元干涸,战斗力将暴降到低谷。

白凝冰开始频频回首,看身后峭壁,是否有攀登逃脱的可能性。

当她看到方源神情悠然,倚靠在峭壁上时,不由地勃然大怒,骂道:“方源,我在前面拼死战斗,你竟然在看戏?!”

方源冷哼一声:“曾经的北冥冰魄体,堂堂的白凝冰,现在居然连一支百兽群都对付不来了吗?”

白凝冰气急败坏:“有种的,你上啊!”

方源冷笑:“我若是有三转修为,早就将这群六足鳄铲除了,哪里还轮得到你?”

白凝冰不由地大喘气,气得七窍要生烟。

方源正色道:“白凝冰,我看你是大手大脚惯了。你以前是北冥冰魄体,真元恢复很快。现在落到甲等九成,还像先前那般用法,真元自然不足了。优秀的蛊师,都掐着真元一点一滴,从不滥用。接下来,你按我说的去战斗。你的战法太粗糙了,需要细腻起来。”

“哈?”白凝冰扯动嘴角,“我的战法还不细腻?你可知道,我多少次被当面族长、家老夸赞,我可是白家寨作战精微第一!”

“和一群凡夫俗子,有什么好比较的?你给我听好了……”

白凝冰冷笑三声,但方源照说不误。

声音不可避免地飘入白凝冰的耳中,起先她还不以为意,但渐渐的,她的神色开始发生了变化。

从轻视不屑,慢慢肃容,到最后神情变得凝重。

方源的话,可谓字字珠玑,句句精辟,偏僻入里,妙到毫巅!

这是他前世五百年的经验总结,时间积淀下来的精髓,怎能不让白凝冰这毛头小子,不,黄毛丫头感到震惊?

方源前世活了五百年,人老成精。这份经历,就算是古月一代,或者天鹤上人都比不了的。

这两个老怪,看似活了近千年,但绝大多数时间,都是在沉睡,苟延残喘。真正的活动时间算起来,顶多是两三百年罢了。

若换做平时,白凝冰听着方源的指导,也就听听罢了。她心高气傲,哪怕心中震骇万分,也不会去做。但如今,她面临鳄群压力,身体不由自主地照此实施,顿时收到立竿见影的效果。

六足鳄接连赴死,而她的情形却越来越好。

明明真元、体力都所剩无几,但珍惜使用后,减少无谓的出手次数,增加攻击效果,反倒令她的真元和体力,在战斗中,缓慢地,一丝一毫地重新积累起来。

一刻钟之后,六足鳄群伤亡过半,倒下两百多具尸体,终于停止攻势,开始缓缓后退。

随之,一个巨大的身影,在江面上缓缓升起。

它龇牙咧嘴,满口利齿如刃。黄色竖瞳,倒映着白凝冰的身影,散发出冰寒的杀机。

这是六足鳄群的王。

百兽王级的雄鳄王!

和其他的六足鳄有所区别,这只鳄王身躯更大,仿佛牦牛。它并非六足着地,而是只用两只后足迈步。

它似人般行走,肩背厚实如熊,一条布满甲片的鳄尾,在沙滩上拖出深深的痕迹。

空出来的另外四足,都呈利爪形态,可称四臂。它的臂膀粗壮,肌肉坚硬宛若石头。

白凝冰不禁苦笑。

若是她单独对付这头百兽王,必有胜算。但如今,经过刚刚的激战,她剩下的体力和真元,都严重不足。难以对付这头状态完美的百兽王了。

但就在这时,身后传来方源的声音:“接着罢。”

一道蓝白相间的光芒,射入到她的空窍当中,呈一朵莲花,种在了她的空窍海底。

顿时,她的真元海面开始快速的上升!

白凝冰又惊又喜:“这是什么蛊?”

“天元宝莲。”方源答。

“原来这就是天元宝莲!难怪古月一代也想得到它了。”白凝冰惊叹一声,旋即怒骂道,“你有这样的蛊,为什么不早借给我?”

方源呵呵一笑,自顾自说道:“幸亏这头鳄王还只是百兽级,记住,它的弱点是胸膛上的那块白皮。”

说完,他全身一晃,如水光波动,渐渐隐去了身形。

却是催动了隐鳞蛊。

“狡诈阴险!”白凝冰暗骂方源一声,旋即凝神向雄鳄王望去。

只见它的胸膛上,的确有一块白皮,但只有脸盆大小。还被它的四肢臂爪隐隐护住,要击中那里,谈何容易?

吼!

雄鳄王怒吼一声,一头向白凝冰撞过来。

白凝冰只能靠自己,狼狈侧翻,躲过这一击后,顺势甩手。

锯齿金蜈狠狠地砸在雄鳄王的后背上。

火星四溅,锯齿金蜈猛地弹起,差点把白凝冰都带倒。

雄鳄王背甲上印了一个白印子,除此之外毫发无损。

呼!

它鳄尾一摆,风声乍起。

白凝冰只看见眼前一条黑鞭抽来,又粗又大又长。她根本来不及闪躲,只能狂催天蓬蛊。

砰的一声闷响,她被抽飞出去,跨越十几米的距离,然后撞到坚硬的峭壁上。

白凝冰痛得直抽冷气,天蓬蛊是三转蛊,防御卓越,但并不能缓冲力道。

砰砰砰……

雄鳄王迈开两只粗壮的腿,在沙滩上踩出一个个的深坑,向白凝冰撞来。

白凝冰双眼一亮,看着雄鳄王冲撞过来,却不动弹。

雄鳄王凶威滔滔,张牙舞爪,若换做其他人,不是被吓得瘫软,就是急忙逃窜。但白凝冰到底是白凝冰,意志如铁。

“二十步,十五步,十步,五步!”眼看着雄鳄王就要杀到,在最后关头,白凝冰这才纵身一跃。

轰!

她险而又险地避开了雄鳄王,而后者则硬生生地撞到峭壁里,大量的碎石滚落下来,顷刻间就将它掩埋。

“畜生就是畜生!”白凝冰哈哈一笑,正要迈步趁胜追击,忽然想到什么,脚步一顿。

下一刻,雄鳄王巨尾狂甩,碎石如弹,四处飞溅。

白凝冰静静地看着,片刻之后,雄鳄王终于脱困而出。

它极为狼狈,一口利齿崩坏了一小半,鼻孔流出血迹。原本金黄的瞳孔,此刻变得通红。

它仰天怒吼一声,这次上身伏地,以更快的速度,再次向白凝冰冲撞过去。

白凝冰后退,微笑侧让。

轰!

一声巨响,峭壁坍塌,烟尘四起……

大半个时辰之后,伤痕累累的雄鳄王,徒劳地捂住胸口的那处白皮,却捂住不断流淌的血液。

随后,它扑通一声,倒在面目全非的沙滩上。

“这只血月蛊,其实也挺好用的嘛。正是有了它血流不止的特效,才令雄鳄王死得这么轻易。”白凝冰望着手中的红月印记,心中浮想。

鳄王一死,剩下的六足鳄虽然还有上百头,但都失去了主心骨,士气暴跌,纷纷入江溃逃。

“终于结束了。”白凝冰甩开手中的锯齿金蜈,累得一屁股坐在沙地上。

方源的身影显现出来,他蹲在雄鳄王的尸体旁,一阵摸索。

“找到了!”当他收回手掌时,两只手中各拿了一只蛊。

白凝冰看到这一幕,顿时气得呼吸一乱。自己拼死拼活的战斗,总算杀了雄鳄王,击退了鳄群。结果却是毫发无损的方源,出来收取战果。

方源端详一番。

这两只蛊,一只奄奄一息,仍旧微微挣扎。宛若龟壳,巴掌大小,只是凸起的表面,布满了鳄鱼般的鳞甲。

这是背甲蛊。

还有一只,毫发无损,却静静的一动不动,任由方源用食指和拇指轻捏着。

这是鳄力蛊。

它很小,堪比常人的一根手指。它仿佛是微型鳄鱼,有头有身有尾,但惟独缺少了鳄足。

不管是背甲蛊,还是鳄力蛊,都是二转蛊虫。

一般而言,百兽王的身上,寄生着二转野生蛊。千兽王的身上,是三转蛊。万兽王掌握着四转蛊。

“果然不出我的所料。”方源看着手中两蛊,并不奇怪。他经验丰富,观战良久,早已经窥破这只雄鳄王的虚实。

当下,他泄露出一丝春秋蝉的气息,真元一吐,就将这两只蛊虫炼化。

背甲蛊飞到方源的后背上,化为一大片鳞甲印记,宛若纹身,从方源的肩膀延至腰际。布满了后背。

顾名思义,它是能增强蛊师后背防御的蛊虫。

鳄力蛊则化作一道暗黄色的光,钻入到方源的空窍当中去。

它和黑白豕蛊的作用,大致相同。能永久性的增强蛊师力量,一鳄之力!市价很高,平时是有价无市的珍稀蛊虫。

“又是这样,顷刻间炼化蛊虫!”看到这样一幕,白凝冰不顾上愤怒,瞳孔猛缩成针尖大小。

在和方源之前的交手中,她就发现了方源的这个秘密。

回到家族里,她翻遍了资料典籍,查出了一些辅助蛊虫,能达到这种效果。

但此刻,她再次目睹,忽然觉得真相未必如此。

“这个家伙,底牌好多。天蓬蛊、血月蛊、锯齿金蜈也就算了,竟然还有天元宝莲!他的战斗技巧,完全凌驾于家族所授。刚刚他用的又是什么蛊呢?”

白凝冰想到这里,心中一股寒意顿生。(未完待续。如果您喜欢这部作品,欢迎您来138看书文学注册会员推荐该作品,您的支持,就是我最大的动力。)

------------

\end{this_body}

