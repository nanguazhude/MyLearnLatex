\newsection{无足鸟啊,振翅高飞吧!}    %第二十八节:无足鸟啊,振翅高飞吧!

\begin{this_body}

奔跑疾驰!

脚下是密道,似乎前所未有的漫长。但是再漫长的旅程,也会有终点。

一点光亮出现在方、白二人的视野中,并且渐渐扩大。

“是出口!”白凝冰脱口而出道。

“按照前世的传闻,这密道应该通往白骨山的一处悬崖峭壁。”方源没有开口,闷头奔跑,只在心中思量。

应该感谢失去的百生、百花这对兄妹。正是因为他们的情报,才吸引了方源来此处。也因为他们的牺牲,才让方源得偿所愿,获得骨肉团圆蛊。

忽然,身后传来紧凑的脚步声。

“终于追上了!”

“发现了那两个贼子!!”

“你们哪里走!”

百家蛊师气势汹汹,追杀过来。

百家族长一马当先,几位家老紧随其后。狭窄的密道将百家众人拉成长长的队伍。

直撞蛊。

陡然,一位家老忽然加速,身形如一颗炮弹,迅速拉近和方源的距离。

“白凝冰!”方源大叫一声。

白凝冰咬咬牙,手掌往后一甩。

血月蛊。

哧的一声轻响,一道血刃飞射而出,正中后面的家老,使其冲势一滞。

但紧接着,密道中彩光绽放,大量的远程攻击向方、白二人射来。

“白凝冰,接着!”方源又叫一声,将天蓬蛊再次借给她。

白凝冰同时催动天蓬蛊、铁刺荆棘蛊,再加上冰肌的防御,如此三管齐下,硬抗住大部分的攻击。

“百家,你们不想知道两个少主的下落吗?”方源大喊。

百家蛊师在方、白二人的手上,看到百生、百花。听闻此言,攻势顿时一滞。

“说,你们把我族少主怎么样了!”

“他们俩若有什么闪失,你们必将下场凄凉!!”

“把我的孩子还来!!”百家族长暴怒欲狂,伸手一指,飞出一道寒芒。

寒鱼蛊!

此蛊只有飞镖大小,形如小鱼,在空中划出弯曲的弧线,有追踪之效。

方源闷哼一声,躲避不开,被寒鱼蛊射中。

顿时寒气笼罩他的全身,他速度暴降。

隐鳞蛊。

方源心念一动,浑身波光荡漾,就要隐去身形。

“休想!”一位家老忽然伸开右手,对准方源,狠狠虚抓下去。

噗。

一声轻响,隐鳞蛊虽然藏在方源的空窍当中,却忽然碎裂毁灭。

方源心中一沉,连忙催动跳跳草。

他的两个脚底板上,顿时一阵酥麻刺痛痒,从皮肉中硬生生地生长出两蓬形如弹簧般的青草。

方源凭借青草的弹力,跳跃前行,速度激增。

“我先走一步,记得跳!”他留下这句古怪的话。

“什么?”白凝冰自然疑惑,但紧接着背后风声大盛。

她回头一看,心中顿时咯噔一下。

却是百家女族长接近过来,她双眼赤红一片,仿佛是一头被激怒的雌狮。气势之烈,便是白凝冰也忍不住心中一颤。

毕竟,对方可是四转蛊师。

轰!

百家族长一拳直捣,巨大的力量顿时破掉天蓬蛊的白光虚甲。

白凝冰大吐一口鲜血,唤出锯齿金蜈。

锯齿金蜈化为一道金光,缠绕住百家族长。

趁此良机,白凝冰飞奔疾驰,来到出口处。

竟然是悬崖峭壁!

在这一刹那的时间里,白凝冰忽然明白过来,方源留下的那话意思——竟然是要让她跳崖!

“你跑不掉了!”身后传来白家族长的爆喝。至于锯齿金蜈,已经被她的蛮力扯断成数段。

白凝冰的眼中闪过一丝挣扎。

这种高度,跳下去绝对是十死无生。但此刻局面危急万分,早已没有选择的余地。

她一咬牙,悍然跃下山崖。

“真跳了!”

“他死定了!”

一些百家蛊师纷纷惊呼,止步在悬崖峭壁处。

百家族长往下望去,只见白凝冰正在飞坠。她的额头上青筋直冒,咆哮起来:“活要见人,死要见尸,我一定要抓住他们俩个!”

身体在不断地下落,风声在耳边呼啸。

白凝冰从未料想过,居然自己有跳崖的这么一天。

“就这样要死了吗?虽然精彩,但是不爽啊……”死亡的气息扑面而来,白凝冰心绪剧烈翻腾。

“阳蛊还未到手,以女子的身份死亡,真是个悲剧。不过摔下去,一定会成为骨渣肉泥,倒也看不出男女来了。”

白凝冰也搞不清楚,自己居然在临死前,冒出这么古怪的念头。

就在此时,耳畔的风声陡然剧烈起来,一个她无比熟悉的声音传来:“白凝冰!”

白凝冰回头往身边一看,不是方源又是何人?

此刻方源亦在飞坠,但不同的是,他的脚下却踩着一只白骨大鸟。

无足鸟!

此鸟没有血肉,通体都是皑皑白骨。鹰首鹤身燕尾,两对翅膀,分布两侧,没有一只鸟足。

方源满身血污,显然刚刚在秘道当中,他也受了许多伤。

“抓住我的手。”他蹲在鸟背上,尽量将手伸过去。

啪。

一声轻响,两只手在空中紧紧握住。

方源再一用力,就将白凝冰拉倒鸟背上。

然而此刻,一块距凸起山石离他们已经不到百丈。无足鸟如一颗流星,向着这块山石砸落下来。

“小心,我们要坠落了!”白凝冰惊呼一声,一颗心霎时间提到了嗓子眼。

山石在她的视野中急剧扩大。

方源眼中厉芒爆闪,狰狞呐喊:“我的魔道怎么可能折损在这小小的白骨山,给我起!”

起!起!起!

呐喊声在山谷中回响。

哗哗哗!

在他的操纵下,无足鸟疯狂振翅,拼尽全力,缓和下坠的速度。

一阵咔嚓声响起,四只骨翅均出现裂痕。

方源空窍中真元急剧消耗,但天元宝莲散发温润光辉,喷涌出大量的天然真元。方源的真元海面消减的同时,又暴涨。

山石附近悠然闲逛的一群骨兽,感受到无足鸟的动静,纷纷抬头仰望,顿时吓得四散奔逃。

一只灰骨驼鸡,吓得将带有尖锐鸟喙的脑袋,狠狠地扎进白骨山石当中,屁股撅得高高。

驼鸡这种动物,天性如此。在害怕的时候,就喜欢这样自欺欺人。

就要撞上了!

白凝冰忘记了呼吸,方源怒瞪双目。

巨大的风压,将这块山石上的白色骨树压垮。最终,无足鸟擦过驼鸡的屁股,斜斜的逆冲而起!

驼鸡屁股后的修长鸡尾,被削的一根不剩。露出它圆溜溜的屁股。

“哈哈哈。”白凝冰在鸟背上大笑不止。

险死还生,压力尽去,她感觉一颗心,又从嗓子眼落回到胸腔当中了。

精彩,真是精彩。生死一线的精彩,向来是最为动人心魄的。这样的生活,不正是她一直想要的,一直追求的吗?

“无足鸟啊,冲向蓝天吧。”她近乎高歌着。

“族长!他们竟然还活着!”悬崖处,百家蛊师均脸色发青。

“追不上了,那是无足鸟,能一日飞行万里”铁刀苦无奈地叹了口气。

能飞行的蛊虫本来就少,追得上无足鸟的更是少之又少。五转之下,无足鸟乃是第一流的飞行坐骑。

“呜呼,苍天无眼啊!怎么能容忍如此恶徒苟活于世!”一些老人气的大喊,捶胸顿足。

百家族长双眼一片血红,拳头攥紧几乎要捏碎,牙齿咬得嘎嘣作响。

听着白凝冰的大笑声,她甚至产生了一股跳崖追敌的强烈冲动!

百家不是没有飞行蛊,一些蛊师已经各展神通,向方、白二人追去。但观其速度,傻子也清楚,要追上这两人,完全是痴心妄想。

一股强烈的苦涩之意,充斥百家蛊师的内心。

罪犯就在苦主们的眼前逍遥法外,他们却只能眼睁睁地看着。

“不!绝不能让他们逃脱!不能!!”一位蛊师老者发出怒吼声,全身忽然燃烧起熊熊的热焰。

“百战温。”百家族长脸色一变。

“家老大人!”众人震惊。

“爷爷!”百战猎泪流满面。

“族长,诸位!绝不能让这两个小贼逃脱,否则我们百家的尊严何在?两位少主的仇,更是不同戴天!我走后,请诸位多多关照我的孙子,这小子脾气和老朽一样的烈……”说到这里,他失去了声音。

他浑身的皮肉、骨骼都转化成了火焰,声带自然也消失了。

就连双眼,都变成了紫黑色的圆瞳。

他成了人形的火炬!

火焰熊熊燃烧,气温快速上升,众人连连后退,山巅处似乎在响起一曲悲歌。

火人蛊。

四转蛊虫,一经使用,全身焚烧,转为火焰。直至熄灭。这位百战温家老选择牺牲自己的性命,来获取这股强大的力量。

“好,本族长就站在这里,看家老建功!”百家族长神情动容。

但百战温家老已经听不见了。

他感到自己的力量前所未有的强大,生命在这一刻燃烧,达到顶点,璀璨无边。

紫黑色的圆瞳扫视周围众人一圈,他看了孙子最后一眼,然后毅然化作一道流焰冲天而起。

流焰以极快的速度,向无足鸟逼近。

“好!”众人看到这一幕,大声喝彩。

百家族长如钢铁般冰寒的脸上,也泛出了激动的神色。

“有强敌!”白凝冰脸色凝重。

她缺乏在高空作战的经验,一个处理不好,掉落下去,可就粉身碎骨了。

吼!

火人发出怒吼,紫黑色的瞳眸中,充满了杀意。

火光一闪,火人速度激增,如电般向方源扑来。

“抓紧了!”方源时刻关注着,操纵无足鸟猛地一振翅。

无足鸟一个提速,拉开距离,令火人扑了个空。

山崖上,响起一片遗憾、惋惜之声。

但很快,火人又再度扑上。

无足鸟一个侧飞,与其擦肩而过。白凝冰差点被甩下去,连忙紧紧抓住鸟背上突出的骨头。

吼!

火人又杀来。

方源冷笑一声,无足鸟忽然收缩骨翅,向地面俯冲。

火人紧随其后,冲刺速度快过无足鸟,渐渐拉近距离。

无足鸟忽然四翅一展,陡然拔高,逆冲而飞。

轰!

火人飞行,全靠火焰的推动**,没有翅膀的他,来不及变向,直接撞在山石上。

一时间,方圆近十亩的地域火焰蔓延,杀死其中的一切骨兽,摧垮大量的骨树。

火海中,一团火重新凝聚成人形,飞射而出。对方白二人,锲而不舍。

然而,蛊虫也要看什么样的人再用。

蛊师修行,养、用、炼三大方面,“用”占其一。不是什么人随随便便的拿出一只蛊,就能发挥出色的。这里面的门道、学问都很深厚。

火人蛊是同归于尽的手段,百战温家老自然是第一次运用。反观方源,虽然此生第一次驾驶无足鸟,但在他前世不知用过多少的飞行坐骑,使用心得丰富得超过百战温不知多少倍,简直是滚瓜烂熟,甚至深入到灵魂中,近乎成了一种本能。

悬崖边上,百家众人一直关注着空中的战局。

从希望和期待,渐渐转变成了愤怒。

但凡一个明眼之人,都能看出来,无足鸟在近乎戏耍一般,玩弄着百战温家老。

火人的怒吼声,原先听起来是那么的震撼人心,气势十足,但现在却变成了外强中干,凸显出悲凉和无奈。

“可恶啊……”有人狠狠地捏紧双拳。

“怎么会这样?”铁刀苦也为方源的技术吃惊,无可奈何地叹息。

‘方正,我要让你死!‘百战猎大叫着,仇恨的种子在他心中深埋。

百战温家老的牺牲,在方源的戏耍下,仿佛成了一个笑话,又像是讽刺的巴掌,扇在百家众人的脸上。

渐渐的,众人的愤怒之情,转化为了失望、绝望。

“追不上了。”

“难道我们就只能眼睁睁地看着吗?”

“古月方正……”很多人都念叨着这个名字,咬牙切齿。

无足鸟自由翱翔,又一次和火人擦肩而过。

“少年,你弄鸟的技术一流啊。哈哈哈!”白凝冰大笑,放下心来。

方源的脸色忽然一凝:“小心!”

轰!

剧烈的爆炸陡然发生,百战温知道追不上方白二人,毅然选择了自爆。

爆炸的火焰笼罩住整个无足鸟。

幸亏无足鸟乃是白骨之身,就连羽翼都是一片片的如刃的薄骨片。

火焰造成不了伤害,但真正危害的,是爆炸的震荡力量。

无足鸟身上的裂纹大增,被这一炸,失去平衡,一头向地面栽倒下去。

但坠落了一段距离后,在方源的操纵下,它重新掌握了平衡,振翅飞向远方。

“方源!”白凝冰却惊叫一声。

先前方源借给了她天蓬蛊,她有蛊虫护身,但方源却没有。

事发突然,方源也来不及催动蛊虫防御。

无足鸟冲出火焰,方源赫然浑身都燃烧着热焰。

狂风推动火焰的燃烧,右耳处的地听肉耳草更被烧毁。但他却神情如铁,仿佛并不是当事人一般。

无足鸟真正稳定之后,方源吐出兜率花,一蓬奶泉水迎头浇下。

火焰被熄灭了,但他浑身都是大面积的烧伤,脸部尽皆毁容,惨不忍睹。

白凝冰张口欲言,却一时说不出话来。

反倒是方源扯动嘴边,笑起来:“我喜欢无足鸟,你知道为什么吗?”

他的笑容,着实有些渗人。

“为什么?”

“因为它没有鸟足,只有翅膀。因此只能飞翔。当它落地之时,就意味着它的毁灭。”

孤注一掷,不飞则死!

蓝色的瞳孔一扩,开始熠熠闪光。迎面的风吹得银发飘扬,白凝冰的嘴角也微微上扬:“呵呵呵……那就让我们振翅高飞吧。”

无足鸟遥遥飞远,变成眼中的一个点,最终消失不见。

悬崖上,百家众人“望眼欲穿”,均没有说话。

一片死寂,笼罩住所有人。

天地是那么的宽广!

天空幽蓝,白骨山则皑皑如雪,阳光从斜面打来。无足鸟和方、白二人的背影,深深的印刻在他们的心中。

愤怒在酝酿,仇恨在发芽……

噗。

百家女族长陡然喷出一口鲜血,仰头倒下,昏死过去。

“族长!”

“族长大人!”

“快,治疗蛊师,快来救族长!”

悬崖上一片惊慌忙乱。

(ps:什么是魔道呢?一种道,必然有其思想和理念。对于什么是魔道,个人有个人的见解,但在我看来,无足鸟无疑诠释着一种魔道精神。魔道终究是残酷的,不仅是对于他人,还是对于自己。所以它自然不讨人喜欢了。本章4700字,你们懂的哈。)(未完待续。如果您喜欢这部作品,欢迎您来138看书文学注册会员推荐该作品,您的支持,就是我最大的动力。)

\end{this_body}

