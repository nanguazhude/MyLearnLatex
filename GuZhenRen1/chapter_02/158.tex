\newsection{毛民的传说}    %第一百五十八节:毛民的传说

\begin{this_body}

有关毛民的记载,最早记录于人祖传中。

话说,人祖挖下双眼,化为一儿一女。儿子为太日阳莽,女儿为古月阴荒。

而太日阳莽好喝美酒,一次喝出事端,被困在平凡深渊。最后因祸得福,得到菊花样子的名声蛊,逃出生天。

因为有了名声蛊,太日阳莽的名声渐渐大了。很快,世界上就传遍太日阳莽是个大酒鬼的消息。

一天,一群斑虎蜜蜂拖着蜂巢,主动找到太日阳莽。

“太日阳莽啊,听说你喜好美酒,一直都说天地四猴的美酒最好喝。但它们酿造的酒,哪里及得上我们的蜜酒呢?今天我们特意带来蜜酒,请您品尝品尝。”

这些蜜蜂一个个都有花豹子一般大小,身上的花纹好似虎纹,黄金打底,黑斑点缀。说话都很客气,但是隐含威胁强迫的意味。

太日阳莽心中叫苦,这叫身在家中做,祸从天上来。斑虎蜜蜂实力强大无比,单单一只,他也不是对手。更何况来了一群呢?

太日阳莽只好勉为其难,尝尝蜂巢中的蜜酒。

他刚刚喝下一口,双眼就发亮了。

蜜酒甜而不腻,醇香可口,十分好喝,是天地间的绝对佳品!

“好喝,好喝,太好喝啦。这蜜酒喝了,能让人感到自己是天底下最幸福的人!”太日阳莽一口口喝下肚,赞不绝口。

斑虎蜜蜂们都笑了,感到很高兴。

首领就问太日阳莽:“那你说,我们的蜜酒和天地四猴的酒相比,谁更好喝一些?”

太日阳莽已经喝得醉醺醺的,忘了斑虎蜜蜂的可怕,直接坦言道:“各有千秋,难有比较。”

斑虎蜜蜂们大怒,自己酿的酒居然和那群死猴子不相上下?这太日阳莽太可恶了,我们得好好教训他!

它们正要动手。忽然太日阳莽消失不见。

太日阳莽这一醉,醉了七天七夜。

朦朦胧胧中,他听到一个声音在黑暗中呼唤他:“太日阳莽啊,你快醒来。再不醒来,你就要被吃啦……”

太日阳莽惊醒了。

他发现自己被五花大绑起来,由一群野人抬着。

这群野人浑身长满了毛,双目幽蓝。已经点燃了篝火,篝火上还架着一个大锅。

野人们静静地坐着,说着悦耳动听的话语。

“我们要炼出永生蛊,正需要一个人作为药引。结果上天就送来了太日阳莽,真是可喜可贺啊。”

“人是万物之灵,人祖就是灵祖。太日阳莽是他的左眼所化。灵气十足。依我看,这次炼蛊能够成功!”

“快把他投入油锅,我们得到永生蛊,就能永生啦……”

太日阳莽听到这些话,大惊失色,连忙喊叫起来,大力挣扎。

但这些野人不为所动。

这个时候。太日阳莽的心底又响起之前的那个声音。

“唉,没有用的。这些野人,都是毛民,天地钟爱。与生俱来就有一项才华,能把蛊虫都炼化。”

太日阳莽一时间忘记了险境,好奇地在心底问道:“你是谁?”

那声音答道:“我是神游蛊,只要任何人喝下天地间四种极品美酒,就会在心田里孕育而成。我能让你挪移到任意的地方。”

太日阳莽大喜:“那就请你快出手啊。带我离开这里。”

神游蛊叹息道:“没有用的。只有你喝醉了酒,才能催得动我。你现在神智如此清醒,是不行的。”

太日阳莽恍然大悟:“难怪那次我被困在孤岛,差点被饿死。幸好得到名声蛊,才脱离了平凡深渊。原来是你害的我!”

神游蛊回答道:“唉,人啊,我也不是有意害你的。都是你喝醉了酒后,催动了我的力量。你不要怪我啦,上次你差点被斑虎蜜蜂捉拿,是亏了我你才脱险的。一害一救。咱们算是扯平了。”

太日阳莽也想起斑虎蜜蜂的事情,不再怪罪神游蛊。

他被毛民们投入到锅中。

大火在锅底炙热地燃烧着,水温渐渐上升。

“加玛瑙红椒!”一个毛民将璀璨珍贵的玛瑙红椒,投入到锅中。

锅中的水,立即变红了,甚至还染红了太日阳莽的身躯。

“加碧落狐烟婴!”一个毛民手中提着一个小狐狸,将起抛入锅中。

小狐狸浑身毛茸茸的,等着黑钻石般的双眼,十分可爱。但一碰到水,它就化为一股青色的烟气,融入水中。

锅中的水,渐渐滚烫,太日阳莽也变得绝望了,觉得这次在劫难逃。

毛民们陆续添加了许多辅料,以及蛊虫。

“加虚荣蛊!”一个毛民将一只蛊,抛入到锅中。

这只蛊很奇怪,长得好像是一只青色的大螃蟹。但和真的螃蟹不同,它的蟹壳里面是中空的。

一见到太日阳莽后,大螃蟹一般的虚荣蛊十分兴奋:“你,你就是太日阳莽吗?我听说过你,想不到能在这里见到你,真是三生有幸啊。我真是太高兴,太兴奋了。”

太日阳莽哭笑不得:“还三生有幸呢,我们马上都要死了。”

“死的问题我不关心。我只想向你请教,你是如何变得这么有名?我都快羡慕死了!我最崇拜向你这样的人啦。”虚荣蛊十分急切地问道。

“我现在可没有心情说这个,我要逃生。”太日阳莽在锅中挣扎,想要爬出去,但很快又被一旁看守的毛民强行摁下锅里。

“快告诉我,快告诉我吧!”虚荣蛊很不识趣,一心想要求教。

太日阳莽怒斥道:“你看看现在的处境,你还看不出来吗?”

虚荣蛊瞪着双眼,盯着太日阳莽猛看,忽然笑逐颜开:“我懂了,我懂了。要想红,就得忍住烫。谢谢赐教,谢谢赐教。太日阳莽啊,为了感谢你。我就为你做一件事情吧。”

说完,虚荣蛊砰的一声爆炸开来。

这爆炸也不剧烈,十分轻微,只发出一声砰的轻响。然后虚荣蛊,就化为一股无形的毒风,袭遍所有毛民的内心深处。

毛民们原本幽蓝透亮的双眼,都变得通红起来。

太日阳莽楞了好一会儿。这才惊醒。他顾不得感慨虚荣蛊的牺牲,连忙大叫起来:“你们毛民虽然能炼蛊,但我看也算不了什么。你们就算各个都永生,又能怎样呢?你们长得这么丑,浑身都是毛,简直丑死了。”

毛民们都愣住了。

换做先前。他们都不会搭理太日阳莽。

但现在,虚荣的毒弥漫他们的心田,蒙蔽他们的智慧。

听到太日阳莽的叫喊,当即就有毛民大声地反驳道:“胡,胡说!我们毛民最美了,浑身的毛发美不胜收!”

太日阳莽灵机一动:“你们的毛发再美,有我的头发美吗?”

他因为曾喝下了金刚猴酿的烈酒。头发都变成了火焰在燃烧。

毛民们听了他的话,都一阵发怔。

火焰之美,每时每刻都在缭绕变化。就算是他们,也不得不承认太日阳莽头发的动态美丽。

太日阳莽继续刺激他们:“你们就算得了永生,也不会有我美丽!你们看我的头发,和火焰一样的颜色,一样的动人。”

毛民们受不住激将,终于有成员躁动起来:“你有这样的头发。我也有。看我的!”

说着,他就用火把点燃全身。

他浑身的毛发都在燃烧,成为一个火人。

“哈哈哈,你只有头发美,我现在全身都美。”这个毛民大叫道。

很快,就有其他成员争相效仿。

他们一个个成为火人,火焰灼烧着他们。剧烈的痛楚让他们情不自禁地发出声声哀嚎。

但他们明知如此,仍旧不去扑灭火焰,而是夸夸其谈,展示自己的美丽。

神游蛊大喜过望。在太日阳莽的心中,对他交口称赞不绝:“人啊,你真是太聪明了,居然想到了这个方法。”

太日阳莽逃出油锅,成功求生。他冷笑着在心底回答道:“不是我聪明。是爱慕虚荣的人,都会变得愚蠢。他们常常为了虚无的美丽,而默默忍受痛楚,放弃真正应该追求的东西。”

……

月白宏金,骷髅石,龙歧牙,小秋草,以及五十块元石,一只花豕蛊,一只猪笼蛊。

这是交到方源手中,用来炼蛊的材料。

而在他的对面,这只毛民已经盘坐在地上,开始炼蛊了。

方源虽然不是炼道蛊师,但是前世经验丰富,在炼道上也算是涉猎颇广,知道许多秘方。

现在,他有三种选择。每个选择,都能炼出一只新的蛊虫。

传承中没有明说要炼什么蛊,这就意味着,他炼出来的蛊,至少要比毛民更棒。

方源看着毛民炼蛊的整个过程,心中已经猜测到他要炼的蛊虫。

他心中一声冷笑,脸上则涌现出崇拜的谄笑:“毛民大人,您真是太厉害。看看您炼蛊的手法,真是让小的叹为观止啊。您就是炼蛊大师,天底下没有您炼不成的蛊啊。”

“嘿?哈哈!你,识相。”毛民听了,哈哈大笑,很是高兴。

这一分神,炼蛊顿时失败。

毛民脸色骤变,惊惶大叫:“不――!”

但已经迟了。

天地伟力降下,化为一道闪电,将其劈成焦炭。

“呵呵。”方源淡淡地笑了声,收起手中的材料,在纸鹤蛊的引领下,向下一关卡缓步走去。(未完待续。如果您喜欢这部作品,欢迎您来起点投推荐票、月票,您的支持,就是我最大的动力。手机用户请到阅读。)

\end{this_body}

