\newsection{力道修行有玄机}    %第八十九节:力道修行有玄机

\begin{this_body}

八天后。

楠秋苑。

小广场上,星星石铺成的方砖,视野一片清晰,星光灿烂。

方源立在广场一端,背负双手,双脚微开,双目紧闭,呼吸深缓悠长。

在他的面前,竖立着十几根石柱子。

每一根石柱,直径皆有两丈之粗,黑幽刚硬。石柱之间,间隔摆放,几乎布满了整个小广场。

呼……

方源吐出一口浊气,陡然睁开双眼。

顿时,虚空中仿佛亮起一道闪电!

冲!

他迈开脚步,大步连跨,几步之后,赶到一根石柱面前。

脚步倏地停住,腰腹一扭,手臂顺势狂甩,拳头横扫过去。

砰。

一声闷响,石柱颤抖,被打出一个球形深坑,一时间石屑飞舞。

再来。

方源憋住气,或拳或掌,轮番进攻。同时飞腿、脚踢、肘撞、膝顶、肩靠……身体的每一部分,都被他化为犀利的进攻武器。

砰砰砰……

闷响声连绵不绝,石柱不断颤抖,石屑不断向四周乱飞。

方源一口气不停歇,攻击如暴雨疾风,绵绵无穷。

须臾功夫,这石柱就削减了一半体积,宛若暴风雨中的小树。

咔嚓。

方源猛地抽腿横击,一声脆响,本就岌岌可危的石柱,终于不堪蹂躏,断成两截,倒在地上。

方源缓缓收腿,呼吸微微急促,额头上渗出一层细密的汗珠。

他放眼往地上一扫。

这两根断裂的石柱,已经面目全非,表面坑坑洼洼,都是方源拳脚轰击的痕迹。

“这是我没有动用全力以赴蛊,寻常的攻击,的确难以发挥出兽力虚影。”

呔。

他轻喝一声,连续两个大跨步,猛地窜到最近的一根石柱前。

他右手化掌,向着石柱重重地拍击过去。

吼!

一只虚影棕熊,陡然在方源身后的半空中出现。它膘肥体状,张开血盆大口,露出森白的尖牙。同时举起右掌拍击,酷似方源的动作。

在这一刹那间,方源顿时感受到,自己的右掌中增添了一股澎湃的力量!

轰。

方源拍在石柱上,发出一声巨响。

两个人都合抱不过来的粗大石柱,顿时被拍断,轰隆一声,砸在地上。将地砖砸出一个浅坑。

一股反震之力,旋即向方源传来,震得方源右掌一阵酥麻。

方源买下棕熊本力蛊后,就一直在用,如今已经大功告成,身上增添了一熊之力。

“先前那根石柱,打了数十次,才堪堪将其摧毁。动用了全力以赴蛊后,只是一击,就将石柱拍断……”

方源感受着彼此间的差距。

全力以赴蛊不是攻击类的蛊,而是辅助蛊虫。但是它一出,顿时就有一种化腐朽为神奇的效果!

方源表现出来的前后战力,完全是天壤之别。

以辅助蛊充当核心,比较少见。大多数的蛊师,辅助蛊虫在整套蛊里面,只是充足主要支点。

举个身边的例子。

魏央就是以移动蛊光虹蛊为核心,攻击类刀光蛊为主要支点之一,辅助类光源蛊为支点之二。光源蛊能够极大的减少光类蛊虫的真元消耗。

全力以赴蛊能以辅助蛊的身份,却能充作核心。从这一个方面,也可以看出它的珍贵价值。

“不过……天蓬蛊的防御虽然可观,但缺乏刚性。如果换成金罡蛊,效果还要更胜一筹。”方源心道。

刚刚整个攻击过程,他都催动着天蓬蛊。否则自己的手、脚可就烂了。

力的作用,是相互的。

虽然方源的骨头都是铮铮铁骨,但是皮肉筋血,还都是凡人。刚刚一拍,若没有天蓬蛊保护,骨头不会有事,但会血肉模糊。

不过,金罡蛊暂时还没有货。

方源却已经没钱了。

一开始,他虽然有九十多万的元石,但是被白凝冰分去一半。

然后,为了全力以赴蛊,逛赌石坊花费了极多。

再加上,整个过程中衣食住行,还有喂养蛊虫,在商家第三内城,开销甚大。

而后,又是大换血,投入了大批元石。八天前,在演武场上他将仅剩下的八万多块元石,当众交给了李然。

到了现在,方源不仅两袖清风,而且还担负了债务。

因为按照毒誓的约定,他还需要交给李然十二万的元石。

方源知道,这是必须的投入。

名声还是其次,主要是稳住李然,展现出真诚合作的态势。尽管方源早在几天前,就用言而无信蛊消去了身上的毒誓蛊。

李然是方源的意外收获,杀李然,会留下证据,引发商家怀疑,绝对是个败笔,所以方源不取。

对方源来讲,李然这个棋子,虽然不好用,但很有用。将来说不定,方源还能通过李然,从武家套出五转的全力以赴蛊的合炼秘方。

现在,最主要的,是通过李然来稳住武家。

三天前,李然已经通过紧急联络方式,联系了接头人。他表示方源这个人,可以利用。这样一来,武家的报复可能性就很小了。

本来方源就有紫荆令牌在手,动方源便会惹来商家的关注。方源又是三转修为,得了全力以赴蛊,战力暴涨。真要报复,还得是三转或者四转蛊师出手。出动这种家老层次的力量,已经打破了一流家族暗中较量的默契。

留着李然,好处就在于此了。

“再来。”方源将散漫的思绪收回,口中低喝一声。

他沉下身子,把住手臂,心念一动,随之半成的淡银真元灌注到直撞蛊中。

直撞!

蛊虫的奇妙力量,立即让他的双腿,不受控制地向前飞奔。

砰砰砰!

在一条笔直的直线,他疾走十多步,连续撞倒三根石柱。到了第四根石柱,不断摇晃,却最终没有倒下。

这还是方源没有动用全力以赴蛊的情况下。

直撞蛊是方源买下的移动蛊,效果是强行突破。

若无阻碍,笔直向前冲出五十步,冲势顿止。若有阻碍,就会少于五十步。

这蛊比跳跳草要更适合方源。

跳跳草的弹跳力,是本身蛊虫的力量。而直撞蛊的力量,源自方源本身。

方源的力量越强,直撞蛊的强行突破效果就越好。和李然一战,方源同时催动了山猪虚影,简直是完美搭配,一下子就将李然重创!

横冲!

方源再催动一蛊,整个人顿时横向冲出去,像个霸道的螃蟹,撞得石柱倒地。

这就是横冲蛊,和直撞蛊类似,但是横向冲锋。

方源卖掉了跳跳草,购买了横冲蛊、直撞蛊,充当自己的移动蛊。

毫无疑问,这两只蛊能在移动中发力,极为适合力道蛊师。让移动本身,就成为了一种攻击的手段。

“等我到了四转,就可将这两只蛊合炼成横冲直撞蛊,接着使用。”

这算是长远投资,从侧面为自己省钱。

接下来,方源交替使用这两只蛊虫,化身为一头蛮牛,在小广场上纵横冲撞。

这次,他一心两用。

不仅用横冲蛊或者直撞蛊,还在同时动用全力以赴蛊。

石柱不断地被撞飞,然后狠狠地砸在地上,发出闷雷一样的响声。

“动用猪力虚影,能撞飞五根石柱,撞倒八根。动用熊力虚影,平均撞飞七根,撞倒五根石柱。动用鳄力虚影的话,能撞飞三根石柱,撞倒四根。”

方源细心体会这其中的微妙。

显然,山猪之力适合冲撞,冲锋的距离长,全程效果好。棕熊之力前半程力量强大,后半程就虚弱了。至于鳄鱼之力,不太适合冲撞这样的攻击方式。

“目前的全力以赴蛊,只能催动一只蛊的力量。不能同时催出猪力虚影和熊力虚影。也就是说,我的双猪之力,算是有些浪费了。换成其他兽力,能增强其他方面的进攻能力。”

“如今我有双猪、一熊、一鳄之力,又达到了身体承担的极限。是时候,动用钢筋蛊了。”

方源已经将钢筋蛊买到手上。

钢筋铁骨相互配合,可谓相得益彰。

不过钢筋蛊,和铁骨蛊不同。后者是消耗型蛊,前者则需要一段时间的连续使用,才会量变达到质变,在方源的身上形成钢筋效果。

招呼家仆过来收拾广场残局,方源便回到密室当中。

他一分一秒都不想浪费,开始催动钢筋蛊改造自身。

三天之后,第五内城演武场。

一位蓝衣大汉站在方源的面前,气喘吁吁,心有余悸。

“只是和这小子交战五六个回合,我受了重伤。全力以赴蛊,真不是盖的……招招都能打出兽力虚影,实在太恐怖了!”

这位蓝衣大汉紧紧地盯着方源,勉强振奋精神。

“不能输!只要击败他,按照演武场的规矩,我就能得到全力以赴蛊。有了此蛊,我也能修力道!”

蓝衣大汉想到这里,猛喝一声,再次向方源扑去。

水牢蛊!

他张开大口,吐出一团蓝色水球。水球迅速壮大,将方源整个人都罩入其中。

方源冷哼一声,在水中挥起拳头,随意一击。

猪力虚影!

水球晃动三下,强烈变形,却没有破。

“没有用的,为了对付我,我可是专门借的高利贷,买的这三转水牢蛊啊。”蓝衣大汉得意的大笑。

方源目光凌厉,毫不慌乱,又拍出一掌。

熊力虚影!

水球猛地颤抖,剧烈形变,差点就要破掉,但最终还是还原成球状。

“兽力刚猛,水流至柔,以柔克刚,你是打不破的。”蓝衣大汉轻吁一口气,放下心来。

但就在这时。

方源抬腿侧踢,鳄力虚影!

鳄鱼和山猪、棕熊不同,它是两栖生物,熟识水性。

啪!

水球乍然破碎,效果立竿见影。

“什么?!”蓝衣大汉大惊失色。

直撞蛊!

砰。

蓝衣大汉被方源撞飞出去,大喷鲜血,足足飞了二十步,这才落到地上。

他挣扎欲起,但爬了一半,就再次倒下去。

一动不动。

他死了。(未完待续。请搜索,小说更好更新更快!)

\end{this_body}

