\newsection{都是吃}    %第一百二十九节:都是吃

\begin{this_body}

“是这样啊。”商嘲风开口应答了一声。

方源出面,替商心慈直接拒绝了他的提议,他也不好再强求。

方源战胜了巨开碑,三转巅峰修为,就有四转的战斗力。同时他又有紫荆令牌,是商家的贵客。因此,商嘲风一直都在对方白二人,释放善意。

不过他心中却不以为然。

“哼,组建势力,需要时间,怎么可能一蹴而就呢?也罢,就让你们先吃吃苦头。到那时,我在来出手帮助,反而更能收到效果。”商嘲风心中思量着,表面上微微带笑。

双方又谈了片刻,商嘲风亲自将商心慈等人,送出府院大门。

对于外人而言,这是一个最明确不过的政治信号。

辞别了商嘲风,商、方、白三人走上大街。

今天是利市节,店铺都关门停业,街道两旁摆着各式各样的小地摊。

“来来来,又酸又甜的糖葫芦啊!”

“我跟你说,这块老玉,可是我家从祖辈上传下来的……”

“卖米啦,卖米啦,一袋五香精油大米,只要半块元石啦。”

小地摊一个接着一个,琳琅满目,卖各种东西的都有。排成一排,绵延出去。放眼望去,无数行人摩肩擦踵,有的驻足围观,有的讨价还价,有的东张西望。

利市节每年一次,不管是方源、白凝冰还是商心慈,都已经不陌生。

“算一算,我们来到商家城,已经有两三年了。”商心慈忽然开口,语气感慨。

“这两年发生的事情,真的太多了。”她叹息着,继续说着,“换做之前,我绝对想不到我是商家族长的女儿。”

然后,商心慈又看向方源。微微而笑,露出洁白的牙齿:“如果不是黑土哥哥你,我也不能到达这里来呢。”

对于方白二人,商心慈一直抱有深深的感激之情。

白凝冰在一旁默然无语,只是眼角微微抽搐了一下。

“是啊。我也没想到你的父亲居然是大名鼎鼎的商燕飞!不过。救你也是投缘。缘来缘去,聚散离合,此乃世间常情。”方源双眼望着前方,答道。

商心慈面色微变。她听出了话外之音:“黑土哥哥,你是要离开商家城了吗?”

“不错。不久之后,我和白凝冰,就要离开商心慈,前往三叉山。”方源答道。

商心慈咬了咬牙。想要说出挽留的话,但最终都没有说出口。

她和方源相处日久,虽然没有明确的交谈过,但却能感受到他心中的野望。

这个男人的野心太大,是不可能停留在商家城这一个地方的。

“不过,你放心。在离开之前,我会将你推上少主位,并且让你稳稳地坐上起去。”方源笑了笑,“走吧。我先带你去招揽属下。今天,咱们就将心慈你今后势力的班底,搭建起来。”

“什么,黑土哥哥,你真的有理想的人选?”商心慈略带诧异地反问一句。

组建势力。需要漫长的时间。

但凡忠心耿耿的下属,都是要经过长时间的培养。

方源现在的语气,让商心慈感觉,忠诚又有能力的下属。就像是大白菜,说能找到就能找到。

他究竟有什么样的把握。语气这么自信?

不仅是商心慈,就连白凝冰都不免为之好奇。

“跟我来,就是了。”方源走在前面,为两人带路。

七拐八拐之后,他们来到一个小巷。

在一家灯笼店铺,和丝绸铺子之间,摆着一个摊子。

方源走到这个摊子前,停下脚步。

摊子后,半躺着一个少年。

少年衣衫褴褛,背靠在墙角,眼睛半眯着,脸色很差,目光虚浮,一副沉浸酒色,不能自拔的颓废模样。

“难道黑土哥哥,要找的就是这个少年?”商心慈心中思量。

白凝冰也在用锐利的目光,打量这人。这人虽然是个蛊师,但只有一转中阶。看他的模样,年岁已经不小,却是这实力修为,简直是惨不忍睹。

“这位小哥,想要买什么……呃,方正大人!”少年察觉到有人,睁开双眼,话刚说了一半,忽然把声调一扬,显露出吃惊、震撼的神色。

方源如今算是商家城的名人,没有隐藏真面目,是以很多人都认得出来。

“白,白凝冰大人……”旋即,他又认出了白凝冰,语气激动得都结巴起来。

商心慈他虽然不认识,但是却能辨别出商心慈的蛊师气息。同时商心慈的绝色容颜,更是让他有一种目眩迷离之感。

“这是十块元石,你的东西我都买下了。你可以走了。”方源掏出一袋元石。

这少年的脸上,顿时涌现出大喜过望的神色。

但旋即,他有流露出些许的迟疑之色。

这摊子上的杂碎,是他收拾祖父遗物的时候,找到了一些零碎。经过他的辨别,都是破铜烂铁,并没有什么值钱的东西。

但为什么,方正大人要买下他的东西?难道这里面,真的有什么宝物?

如果有宝物,自己卖掉,岂不是亏了?

正这般思量着,方源忽然抬手一抛,将手中装元石的小袋子,抛给了他。

“你还在想什么呢?居然没有听清楚我的话。哼,我看中你的东西,是你的荣幸。你现在可以滚了。再不滚,你就连滚的机会都没有了。”方源寒声威胁道。

少年顿时被吓得一哆嗦。

他颤抖着双唇,结结巴巴地道:“方、方正大人,你怎么可以这,这样。做生意,讲究的是……是你情我愿。咱们不能强买强卖,你也是,也是有身份的人……而且这又是商家城里……”

啪。

方源甩手一个巴掌,将这少年打倒在地。

“滚。”方源居高临下,用冰寒的目光俯瞰地上的少年,语气平淡。

少年捂着脸,心惊胆战,害怕得浑身颤抖。他抬头看了一眼方源,目光触及道方源漆黑的眸子,立即转移开去。然后一声不吭地爬起来,跌跌撞撞地跑走了。

“黑土哥哥……”商心慈看着少年离去的背影,有些于心不忍。

白凝冰则面无表情,不以为意。

“心慈,我可是魔道蛊师,自有一套行事方法,讲究的就是纵情纵横。”方源随口解释了一句,理直气壮。

周围的摊贩都看向他。

他视线左右扫视一番,顿时众人纷纷避让目光,不敢和他对视。

若换做以前的方源,自然要虚以委蛇一番,动用哄骗、欺瞒的手段,和和气气地买下这小摊上的东西。

但现在,他实力大涨,今非昔比。能用最直接的手段,省事省力,为什么不用呢?

正道人物,要爱惜名声,展示风度,所以常向弱小表达和善。

方源却不是正道,而是魔道。

自古以来,大鱼吃小鱼小鱼吃虾米,就是这样弱肉强食的丛林法则。

只是魔道中人,都是撕扯血肉,一口口直接吞下。正道人物,则是一边吃,一边掉鳄鱼的眼泪,说自己被逼无奈。

总有许多愚昧的人,被虚伪所欺骗。或者愚蠢到自己欺骗自己,不愿意面对残酷的真相。

呵呵。

事实上,都是吃。

剥削者吃被剥削者,侵略者吃被侵略者,强大者吃弱小者,压迫者吃被压迫者,上位者吃下位者……

众生万物没有不吃的,不吃,就生存不下去。只是彼此的吃相上有差异罢了。

方源赶跑了那摊主,俯下身之,从摊位上挑选出了一枚令牌。

这令牌,似乎是黑铁所制,黑不溜秋,还半块残缺着。上面依稀刻着字,但被研磨得久了,又缺了一半,根本辨认不出来。

但方源却知道,这是一个饭字。

三百年前,有一个魔道蛊师,重伤落水,被一位河边浣纱的小姑娘所救。

小女孩心地善良,救下这位魔道蛊师之后,将其秘密安置在柴房里,还每天送饭给这位蛊师吃。

魔道蛊师康复之后,有感小姑娘的恩情,做了一块黑铁令牌,刻着一个“饭”字。

又把令牌掰成两半,一半交给小姑娘,一半留给自己。

魔道蛊师离开之时,仔细地叮嘱小姑娘:将来若有什么难处,可去丹火山鬼哭洞,寻找自己的帮助。就算是小姑娘死了,这份承诺,对半块令牌的新主人,仍旧有效。

小姑娘牢牢记住,但只过了不到五十年,丹火山就发生大战,火山喷发,鬼哭洞湮灭。而那魔道蛊师,也被铁家活捉,押在了镇魔塔中。

这半块残缺的令牌,也就失去了作用。在小姑娘的后人子辈中,辗转流传下来。

因为涉及到魔道,小姑娘老死之前,也没有向子女透露半分秘密。将这段往事烂在了肚子里。

小姑娘的后人,被家族驱逐。流浪到商家城定居后,家道渐渐没落。子孙无才又不孝,一脉单传下来,到如今只余下一位少年。

这少年被惯得很,公子脾性,更是吃喝嫖赌样样俱全。双亲死后,每隔一段时间,他就靠贩卖祖辈遗物过活。

但在一次利市节上,他的命运发生了转折。

份属魔道的蛊师三兄弟,来到这里闲逛,无意间发现摊子上的这半块令牌。(未完待续。如果您喜欢这部作品,欢迎您来起点投推荐票、月票,您的支持,就是我最大的动力。手机用户请到阅读。)

\end{this_body}

