\newsection{商家城喧腾}    %第一百二十二节:商家城喧腾

\begin{this_body}

“以三转巅峰,对决四转,方正居然胜利了。不仅以弱胜强,更战胜了巨开碑这样强大的存在!真是少年有为啊!”

“这是一场毫无疑问的大胜。巨开碑更是施展了杀招巨灵变,结果仍旧被方正给干趴下了。甚至主动认输。”

“八大兽影的攻击,真是恐怖……想不到方正居然有力气蛊。他运气也太好了,据说这只力气蛊,是他原先家族中的珍藏。”

茶馆中,酒楼里,街坊间,人们闲暇交谈,都以方源和巨开碑这一战为话题。

在双方都是巅峰状态时,能够以三转胜四转,这样的战例真的很少见。

整场战斗有近千人观战,全部过程都在众目睽睽之下见证,方源胜的堂堂正正,毫无取巧,整个战力的确在巨开碑之上。

至于,力气蛊的来历,方源故意散发出假消息,将众人的猜测往古月家族上面牵引。

“从今天开始,方正大人就是我的榜样!”

此战之后,很多人口中,都称呼方源为“大人”。他的战力,已经得到大多数人的认可。

“方正大人真的帅气无比,巨开碑的时代已经结束了。”一些女蛊师双眼冒光。

“一代新人换旧人,巨开碑倒下了,成全了方正的名声。”许多老蛊师感慨着,均有长江后浪推前浪之感。

“方正打赢了巨开碑,不知道能否战胜炎突。如果他再战胜炎突。他就是这些年来,雄霸演武场的唯一一人。只要他守擂十八场,他将成为商家的外姓家老!”

这一战,让演武场格局动荡。

旧有的格局被打破,新的势力产生了。

人们开始展望方源未来的情景,带着激动,藏着期待。

商家,已经十多年,没有外姓家老了。

如果方源只是以一丝微弱优势,艰难地战胜巨开碑。那么众人的期待远不会这么大。

当方源展现出凌驾于巨开碑的强大战力后,众人开始兴奋地分析――

“方正身上,有全力以赴蛊,苦力蛊、自力更生蛊,还有力气蛊,上古力道蛊师也不过如此!”

“全力以赴蛊让他攻势强悍威猛,霸道无俦!苦力蛊令他越战越强,尤其是和自力更生蛊搭配,巧妙得令人叫绝。力气蛊更是让他远战刚猛无比。将短板弥补成优势。八大兽影,一起轰杀的景象。简直是恐怖。这就是方正的大杀招!”

“巨开碑输的不冤。他虽然是四转,有修为上的优势,但是力道蛊虫对真元要求少,无形中削弱了这种优势。方正的蛊虫,实在是太好了,组合强大,搭配得更是紧密无间。巨开碑当场主动认输,也是心灰意冷了。”

“一个人的机缘际遇怎么可以强到这种地步?方正的运道,强到逆天了。他这才来商家城几年。就积累到了这种程度。蛊虫组合如此强悍,足可以让他今后扬名南疆,拥有自己的名号。”

“名号?他已经有了……现在很多人将他和白凝冰,合称为黑白双煞。不过更多的人,把方正称呼为小兽王。”

方源驾驭八大兽力虚影,战斗风格又是威猛霸道,横冲直撞。小兽王的称号,十分形象贴切。

至于为什么不直接叫“兽王”。

那是因为,兽王已是一位魔道成名蛊师。一个人占据百兽山,以百兽为伍。修为高达五转,蛮不讲理,穷凶极恶。正魔两道,都深为忌惮。

方源现在还只是三转巅峰,虽然打败了巨开碑,一战成名。但是和兽王这等五转蛊师比较起来,还是差了不少的。

从另个方面来讲,众人能以小兽王称呼他,也间接的说明了方源的未来,被众人看好。

“真期待他的守擂战啊。每一个雄霸演武场的强者,守擂战都是经典。想当年魏央大人,以区区三转中阶修为,比方正的修为还要弱,就能在演武场称雄。他的每一场战斗,都成为了经典!”有资深蛊师感怀道。

于是,许多人开始为方源算计前景。

“到了现在这个局面,方正真正的对手有几个?巨开碑、炎突,对了,还有白凝冰。”

“不过,巨开碑已经被方正索要去了巨灵意。这招真是狠呐。巨灵身、巨灵心、巨灵意三蛊里面,巨灵意最难得到。没有了巨灵意,巨开碑的杀招就不完整。对于方正来讲,威胁程度大降。”

“炎突将是方正的最大敌手,他和巨开碑并称为演武半边天,实力不容小觑。不过方正能战胜巨开碑,又有力气蛊弥补远战短板,和炎突战斗胜算将很高。”

也有人提到白凝冰:“白凝冰也绝不容小觑,她败给炎突之后,再上场时,修为已经突破到四转初阶!这么年轻的四转,这是何等的资质啊。不过她和方正的交情极为深厚,很可能就算在战斗中碰面,也会主动认输,成全方正。”

白凝冰和方源一直住在楠秋苑,这个消息已经广为人知。很多人都猜测,他们俩已经发生最亲密的关系。

然而,就在众人喧腾之际,方源忽然宣布,退出演武场。

消息传出,众人惊哗。

“为什么方正要放弃这么好的机会?”

“外姓家老,称霸演武场,就在眼前呀。”

无数人为之扼腕叹息。

“难道有什么内幕?”魔道蛊师们都不由地猜疑起来。

“是不是商家,已经不想招揽外姓家老,所以暗中逼迫方正放弃?”魔道蛊师向来缺乏信任感。

“亦或者商家,想要修改政策,取消掉外姓家老的举措?”

怀疑,让许多人心起伏不定。

商家的这个招揽魔道蛊师,成为外姓家老的政策,乃是南疆独一份。

在南疆的许多家族中,都设置有演武场。规模最大,最盛行的演武场,位于南疆第一家族――武家当中。

但这些演武场,都是对内居多,很少对外。商家的外姓家老政策,成了画龙点睛之笔。为商家吸引了无数的人才。

就算是在外行商,很多的魔道蛊师也不打劫商家的商队。正是因为外姓家老这个举措,让这些魔道蛊师想为自己留一条后路。

商燕飞察觉到不妙的苗头,连忙做出澄清。

虽然他暗中,的确布下巨开碑、炎突两大棋子,控制演武场。但这种东西,是不能曝光的。

商燕飞的威信,让这场无形的风波,渐渐地平息下来。

紧接着,几日后,方源也跟着当众宣布,自己将投靠商心慈,辅佐她登上少主之位。

顿时,群众的注意力被转移了。从演武场,转移到了少主之争上。

因为商睚眦的假账案被证实,原本属于他的少主之位,也被撤销,空余下来。

很多商燕飞的子女,都虎视眈眈地盯着这个位置,垂涎三尺。

“我早就听说了,方正这小伙子忠义。他有人生原则,就是滴水之恩涌泉相报,点滴之仇百倍报还。真是有担当。”有人为此,竖起了大拇指。

“方正为了商心慈登上少主位置,连快要到手的外姓家老的身份都不要了。这真是……”有人则无法理解这样的举动。

“说不定,他和商心慈之间,有什么暧昧的关系。换做我,难以做出这样的牺牲。”

“这很有可能啊。等等,如果是这样的话,那白凝冰怎么办?”

人们的八卦之魂,熊熊的燃烧起来。方源、白凝冰、商心慈组成的三角关系,成为了许多人茶钱饭后的谈资。

紧接着,随后几天,白凝冰也宣布退出演武场。

“白凝冰也投靠商心慈了!”众人惊诧。

“这世道怎么了?外姓家老都没人要了么?”很多人的价值观都被冲击了。

“白凝冰勇气十足,不愿意妥协和放手。为了爱情,主动插一脚。其实这三人之间的感情纠葛,从他们见面的时候,就开始了。”类似的谣言开始四起。

群众想象出无数情节,甚至商家城都开始出现画本和戏剧,描述方白商三人无比虐心的感情纠葛。

居然还很畅销!

这些细枝末节暂且不谈,方白二人的投靠,的确是让商心慈成了众人瞩目的焦点。

白凝冰如今已经是四转初阶,四转的蛊师,放到其他中小家族,那都是族长了!

方源虽然只是三转巅峰,但是战斗力已经凌驾于许多四转蛊师。等到他成为了四转,那还得了?

他们俩都十分年轻,成长空间十分巨大,如今被所有人看好。

骤然得到两大强者助臂,商心慈自然成为竞争少主之位的第二大热门。

在此之前,人们普遍认为,最大希望能继承少主之位的,是商一帆。

方白二人一投靠商心慈,立即改变了格局,形成了两强相争的局面。

……

商一帆个头不高,甚至有些矮小。鹰钩鼻梁,双目狭小,目光锐利。

“商心慈……”他坐在书房里的宽背木椅上,口中轻声喃喃,咀嚼着这个名字,目光闪烁不定。

在他面前,坐着一个人。

不是别人,正是商睚眦。

“一帆老弟,我这次来,是要帮助你的。商心慈有方白二人助臂,将是你最大的对手。尤其是方正,你要小心,这个人阴险狡诈,我就是栽在了他的手上!”

商睚眦咬牙切齿地说道。(未完待续。如果您喜欢这部作品,欢迎您来起点()投推荐票、月票,您的支持,就是我最大的动力。)

------------

\end{this_body}

