\newsection{灵光一闪蛊}    %第八十三节:灵光一闪蛊

\begin{this_body}

方源虽有紫荆令牌,但却不好明目张胆地去调查。

不过记忆中的李然,也参加演武。得了传奇蛊后,大杀四方,横扫演武场。

这就好办了。

方源有强行挑战的权利,于是借着这个名头,来到特别接待处。

报名参加演武的李然,有三位。

两男一女。

女子首先暂时排除,传闻中的李然是男性。

剩下的二人中,一位是老者,六十多岁,有薄名。三转修为,几年前他冲上第三内城的演武场,这些年状态不行,降回到第四内城。这显然也不是,记忆中的李然混得很不如意。

另一位是中年男子,五十多岁。二转修为,来商家城已经近十年了,却仍旧在第五内城演武区打转。他喜好赌石,生活很窘迫。在商家城曾经结婚生子,但是夫妻不和,早已分居。

方源暗暗调查,发现这人便是当初解石时,在赌石坊所见的那位。

连续跟踪几日,方源发现这个李然,生活极为困窘,有时候甚至吃了上顿,却没有下顿。

但这并非说明他没有本事。

他在第五内城的演武场,依靠战斗获取元石,维持生计。

方源查了一下他的战绩,居然胜多败少。

按道理来讲,他应该不愁吃穿。但这人却好吃好赌好嫖,花钱大手大脚,根本不知道节制。为了赌石,他很可能将身上的元石全部拿出来用。以至于接下来的大半个月,他都得四处找人借钱,吃最低劣的饭菜过活。

同样的,为了大搓一顿,他可以去最高档的酒楼,花去身上大部分的钱财。他甚至还吃过霸王餐,现在还欠着几家酒楼的饭钱。

“此人应该就是前世,得到传奇蛊的主人翁。但不懂得控制自己的欲望,就算是有传奇蛊,又能如何呢?难怪前世如流星般划过。璀璨一时,不久后就销声匿迹了。”方源暗暗摇头。

“他前世得到过传奇蛊,今生会不会再得到呢?我这样跟踪他,也不是长久之计。距离发现传奇蛊的日子,已经临近了。难道我真和这传奇蛊无缘?”方源紧皱眉头。

这些天跟踪李然,浪费了方源大量的时间和精力。若把这时间用来修行,也许提升更大。

关键是,方源心中没底,跟踪这个李然是否会有收获。

而且跟踪久了。还会被人发现。幸亏魏央担当重任,只有在方源演武战斗的时候。才会过来,否则方源也不方便追踪李然。

若被魏央发现他跟踪调查李然,又如何解释呢?

现在最大的疑点,还是传奇蛊为何不在星辰石中,这和传闻严重不符!

方源想不通,这实在是匪夷所思!各种因素都对上了,但最关键的一点上,却出了岔子。

不知道是不是思虑过重的原因,方源总觉得这个李然行为有些古怪。但到底古怪在哪里。他又说不上来。这只是一种感觉,似是而非,隐隐约约,连方源自己都不大确定。

就这样又过了二十多天,方源陷入到深深的困扰当中。那种古怪的感觉,越来越重,但方源却不知道源自哪里。

李然的行为。根本毫无不妥之处。

他生活在商家城近十年,若真是有不妥之处,旁人早就看出来了。

“我已经思维定势,想太多了。这个时候。需要旁人的观点来启发自己。”方源对自己的处境有很清醒的认识。

但他向来只信任自己,虽然白凝冰和他发了毒誓,但方源根本不信任白凝冰。他只是利用她来修行,骨肉团圆蛊可比酒虫要好用多了。

没有人帮忙,方源只能靠自己。

但他已经习惯了依靠自己,也喜欢凡事依靠自己。

于是他来到商铺:“店家,可有灵光一闪蛊?”

方源问了好几家店面,都是没有。终于有一家有存货,一问价格,却要两万九千块元石。

灵光一闪蛊,乃是三转的消耗蛊,用一次就消耗掉。谁会花这么多的元石,买这么一个玩意?

但事实上,真的有许多人用。

这些人就是专门推衍炼蛊秘方的秘方大师。

当他们遇到瓶颈,离成功只差一步之遥的时候,便会选择使用此蛊。此蛊一用,便能带给他们最关键的一道灵光,让他们捅破窗户纸,推衍出新的秘方。

一道新秘方的价值,就远远不止两万九千块元石了。

是以,灵光一闪蛊有价无市,若非方源拥有紫荆令牌,店家绝不可能卖给他。

和灵光一闪蛊功用类似的,还有仙人指。

只是后者是铁家的独有草蛊,市面上根本就不流通。

方源咬咬牙,买吓这只灵光一闪蛊。

此蛊形如蓝宝石蝌蚪,小巧玲珑,甩着尾巴,不断游动。

方源真元一催,此蛊顿时射入到他的脑海之中,化作一道白色灵光,如霹雳闪电一般,瞬间替方源划开重重迷雾。

方源眼中陡然暴射出精芒,终于察觉到古怪的感觉来源于哪里。

“是了,这李然生活颓废,毫无节制,看似一团乱麻,但却隐有规律。他每三天去七次酒楼,七天去两次赌坊,五天去一次青楼。而且,每隔十天左右,他都会去一家叫做‘富态祥和’的酒楼。”

原本杂乱复杂的情报,在灵光一闪蛊的作用下,透露出了其中的规律。

这就是古怪之处。

若真的毫无节制,不思前想后的人,生活定然糜烂不堪,怎么可能会有这样的规律?

三天之后,富态祥和酒楼。

“小二。会账。”李然从窗外收回视线,大叫一声。

“来咧,李然大人,这次总共五块半元石。”

“这是六块,不用找了。”李然抛出六块元石。

“谢大人,大人您慢走!”

蹬蹬蹬……

李然下了楼,走出酒楼大门,直至背影转入街角消失不见。

楼上,方源双目幽幽,唤来伙计。手指着李然刚刚坐的位置:“这边风景太差了,我要搬到那张桌子去。”

据他了解,每次李然来此处酒楼,都会坐靠近窗户的位置,这实在有些奇怪。

“没问题!”伙计咧嘴一笑,“窈窕淑君子好逑嘛。”

“此话何意?”方源眉头微微一扬。

“咦,客官难道不是想看秦艳楼的当家红牌,安渔姑娘的吗?嘿嘿,坐那个位置。正好可以看到安渔姑娘的房间,有时候运气好。真能看到她的身影呢。刚刚那个李然大人,每次都坐这个位置,过个眼馋的瘾。每次都是我给预留的位置,一次只需要半块元石就好了。”伙计贼兮兮地说道。

“哦,是这样?”方源不置可否,坐到李然的那个位置。

透过窗户,果真看到隔了两条街的秦艳楼。

秦艳楼高达八层,头牌姑娘安渔姑娘,乃是二转蛊师。房间在顶层。她本身姿容不俗,蛊师的身份更能激起男子的征服欲,据说单一夜作陪就要上万块元石。

依李然的身家,当然不可能支付这笔巨款,但他真的是来偷窥安渔姑娘的吗?

以他出入青楼的情况而言,他并非是那种痴情之人。

方源扫视窗外,秦艳楼周围的酒楼不少。为什么他单单选择这里?

如果他有增长目力的侦察蛊,也说得通。但就方源所知,他并没有此类的蛊虫。

这个距离看过去,就算是安渔姑娘主动站在窗边。也只能看到一个模糊的面庞吧?

“咦?这是……”方源忽然目光一凝,他看到了街对面的一家豆腐小店。

一对母子正忙着收拾铺子,准备打烊。

方源认得这对母子,他们就是李然的妻子和儿子!

“难道说,李然真实的目的,是想看看他们?”方源不由地想到。

他眯起双眼,在脑海中回忆出这片街道的地形图。

要偷窥安渔姑娘,有很多比这更好的地方。但要观察这对母子,却只有此处风景独佳!

“如果,李然真的是要观察母子,为什么要偷偷的旁观?按照我的调查,明明是几年前,他主动抛弃了这对母子。难道是他心中有愧?古怪啊……他若心中有愧,想要关心妻子和孩子,完全可以主动现身,何必躲躲藏藏呢。”

“不,也有可能正是心中的这股羞愧,让他无脸再见妻子儿女。但他若真的羞愧,为什么不痛改前非?蹊跷啊,他看似生活糜烂,不知节制,但事实上他颓废的很有规律。这种规律,正说明他很有自控之力。”

方源双目闪着幽光,脑海中思绪万千。

他夹了一口菜放入嘴中咀嚼,但却食不知味。

他感觉自己已经渐渐接近了整个事件的真相,就好像是一个在黑暗的房间中摸索的人,已经接近了房门。

所有的线索,所有的疑点,都在他的脑海中走马观花似的疯狂乱闪。

这样的急速思考,让他吃饭的动作都缓慢了下来。

他慢腾腾地放下筷子,然后缓缓地拿起酒杯,杯中的酒水如琥珀般剔透,散发着浓郁的酒香。

忽然,方源的黑色瞳孔猛地一扩!

手中的酒杯才举到一半,他整个手臂悬停在半空中,一动不动,宛若石雕。

仿佛有一道雷霆霹雳,在他脑海中咔嚓一声,炸响开来。

“原来如此……我明白了!”

他在心中兴奋的低喝一声,一道如闪电般耀眼犀利的光芒,在他双眼中一闪即逝。

牵涉到前世今生的迷雾统统消散,一切的疑点都豁然贯通,方源寻找到了答案!

同样的,也“寻找”到了丢失的传奇蛊!(未完待续。如果您喜欢这部作品,欢迎您来起点()投推荐票、月票,您的支持,就是我最大的动力。)

------------

\end{this_body}

