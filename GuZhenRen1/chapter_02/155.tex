\newsection{古铜皮蛊}    %第一百五十五节:古铜皮蛊

\begin{this_body}

深紫色的气罩犹如高塔,将方圆百丈的地域死死罩住。

气罩并不完全隔绝视线,在罩外不知有多少双眼睛,隐藏在暗处。

“白凝冰,你乖乖地束手就擒吧。你在这里,是插翅难飞!”铁家四老各占据东南西北四方,将白凝冰围在中央。

白凝冰盘坐在一块巨石上,双手拿捏着元石,补充着真元。

“白凝冰,你不要抱幻想了。你杀了我们铁家的人,还想逃脱?哼。”铁家四老各个目光沉凝,紧紧地盯着白凝冰,包围十分森严。

白凝冰眯着的双眼缓缓睁开,蓝眸中一片沉静,语气冰一般冷漠:“铁家四老,你们说再多又有什么用?想要擒杀我,尽管放马过来。我白凝冰就算死在你们手上,也一定会拉几个垫背的。”

顿了一顿,她又继续道:“一直维持着这个气罩,耗费你们不少的真元吧?呵呵,我知道四位擅长合击战术,而我只是四转初阶罢了。但请你们相信一点,你们若杀了我,就算不死,也定会重伤。万一我那同伴出现,你们又当如何呢?”

“你……”铁家四老为之气结。

白凝冰的话,虽然借了方源的势,但却正中他们的软肋。

“白凝冰,你休要嘴硬。来来来,就由我来会会你吧。”铁家四老之一,站了出来。

“呵呵呵。”白凝冰轻声而笑,施施然站起身来应战。她银发雪衣。夭矫不群,纵然身陷囹圄,却仍旧云淡风轻。有一股将生死置之度外的潇洒气度。

双方交战在一块,气罩中顿时飞沙走石,雪光四溢,金铁交击。

远处低矮山峰上,站着一群商家的蛊师。

“车轮战又开始了,这个白凝冰不愧是在演武场出名的人物。能够抵抗得住四位四转中阶蛊师的轮番攻打,不简单!”其中一人感叹道。

“白凝冰有出色的战斗天赋。的确很强。但另一个原因,也是铁家四老不敢动用全力。”有人分析。

“不错。铁家四老心有顾忌,害怕被白凝冰临死反扑。他们擅长合击战法。但独个的战力并没有同等修为的蛊师厉害。一旦缺少一位成员,整体实力就会暴降,难以争夺三王传承了。”

“我现在担忧的是,白凝冰手中拥有紫荆令牌。一旦她亮出来。我们要不要出手?”

“暂时不要出手。我已经传信回去。家族已经派遣了强援。易火家老已经在赶来的路上了!”

听到这个消息,商家的蛊师们都精神一振。

易火并非是寻常蛊师,而是商家五大家老之一,有四转巅峰的修为!他是商燕飞的得力干将,被派遣过来,一定会改变整个三叉山的局势。

……

“想不到事情发展成这样。白凝冰若是被铁家杀掉,商家那边的买卖可就要泡汤了。”一处草丛中,孟土目光炯炯。盯着气罩中的战斗,语气担忧。

他正当壮年。和搭档焦黄同为三转巅峰的蛊师。

这两人乃是魔道中著名的暗杀组合,当年连四转中阶的正道蛊师萧福禄,都命丧在他们的手中。

他们接到商家方面的承诺,若是能杀掉方白二人,就允许他们俩投靠商家。

他们已经在火炭山上动过手,但引来的纫鳄群并没有给方白二人,带来多大的麻烦。

两人并不死心,跟随方白二人来到三叉山,一直等待着机会。

“唉……这能有什么办法?我们擅长暗杀,却不擅长强攻。大庭广众之下动手,成功的可能性太低了。白凝冰若死,我们也没有办法。这场买卖,我们只能尽人事听天命!”老者焦黄叹息一声道。

“是啊,这暗中不知道有多少双眼睛盯着呢。说不定我们还为潜行接近,在半路上就被人发觉了。”孟土无力地附和一声。

他们是暗杀蛊师,讲究潜伏,不出手则已,一出手就要一击致命。在出手之前,他们要经过精心的算计,大量的准备,然后厚积薄发。

如果成功的可能性太低,他们绝不会出手,宁愿放弃买卖。

这也是他们混迹魔道这么长时间,却仍旧活着的原因。

每个出名的蛊师,都有自己独到的生存之道。

……

“嘿嘿嘿……这下铁家四个老头尴尬了。”李闲站在气罩前,看着里面的打斗,眼睛眯起来,充满了幸灾乐祸的笑意。

虽然这里,已经成了三叉山的目光聚焦点,但李闲却一点都不在意。

他十分自信,自己绝不会暴露在众人视线当中。

这份自信来自于他手中的五转蛊虫匿迹隐形蛊!

蛊虫到了五转,就变得稀有。很多五转蛊师,多年以来,手中甚至只有一两只五转蛊虫。

匿迹隐形蛊,只有五转的特定侦察蛊,才能窥破。但现在三王传承还才开启不久,连中段还未有人突破,还不至于引得那些五转蛊师出动。

李闲也有他的机缘奇遇,在四转的时候,就拥有了稀缺的五转蛊。

“这个铁柜蛊秘方,乃是铁家的炼道大师铁一般研炼出来。原本的用意,是想研炼出一种坚固保险的存储蛊。首次研炼出来之后,就给铁血冷试用。结果神捕拿它用来捉人,十分耐用。若在辅以化气蛊的话,效果奇好。从此之后,铁柜蛊就成为了铁家捉拿魔道蛊师的得力手段。”

“嘿嘿……但现在,铁家四老要维持这铁柜蛊和化气蛊,根本无法展开四人合击。若是四人一起出手,这紫色气罩就会消失。没有阻碍,把白凝冰放跑了,这面子可就丢大发了。哈哈哈,有趣有趣。铁家四老是骑虎难下了。”

李闲看到了一场好戏,嘴角忍不住上翘起来。

但当他想到一个人时,笑意却又缓缓消散。

“小兽王他竟然是没有过来救援!他究竟是看出了此番局面的微妙?还是冷血无情到这样的地步,直接放弃掉白凝冰了?不管是哪一种,都足以说明此人的可怕……我还是将那只蛊,送到他的手中好了。”

……

一只蛊,交到了方源的手上。

它形似臭虫,又扁又宽,头很小,身体呈现椭圆形状。通体都黄橙橙的,散发着铜yiyàng的金属光泽。

人们将其称之为“古铜皮蛊”。

铜皮蛊从一转到三转都有。但若晋升到四转,就是古铜皮蛊,防御力比三转的铜皮蛊要更加强大。

“李闲,你不愧是魔道中远近闻名的商人。这么快就能拿出古铜皮蛊,让我感觉和你做交易,的确是正确的选择。我这里没有茶,请喝酒吧。”

方源语气客气,面容和善,招待李闲,并为他倒酒。

“哪里,哪里。能和小兽王大人做买卖,也是我的荣幸。”李闲表现得十分谦虚,把自己的位置放得很低。

两人交谈了几句之后,气氛十分融洽。

若让不知情的外人看到,都会觉得这二人和善文雅,哪里会想到这两人皆是腹黑狠辣的魔道蛊师?

“李闲,你不必这么客气,直接叫我方正就可以了。我这里先交给你五万元石,算是接下来的订金。”方源取出元老蛊,大手一挥,就调出五万元石出去。

元老蛊形如水晶球,里面贮存的元石越多,球中云翳老人就越是和善。

李闲看着云翳老人笑开花的脸,心中对方源的评价不禁又更高一筹。

“有件事情,还要劳烦李闲你出手。”方源忽道。

李闲目光一闪,连忙道:“请讲。”

方源便将李闲带到山洞深处,指着一个石缸:“我做了一个石缸,却没有火类蛊。请李闲你帮忙,把这些铜块烧成汁水,装在石缸里。”

李闲松了一口气,笑道:“刚巧手中有批火系蛊。此事容易,举手之劳。”

山洞里温度飞速上升。

片刻之后,李闲就将这些铜块烧成金属汁液,几乎装满了石缸。

方源又取出火炭,堆在石缸下,为其保温。

然后在李闲惊骇的目光中,他轻轻一跃,便跳入缸中。

滚烫的铜水,哧啦一声,瞬间将他的衣服烧烂。方源全身都浸泡在铜水里,只剩下脑袋露在外面。

“小,小兽王大人,您这是为何?!”李闲此刻都能闻到烤肉的味道了。

方源一面催动空窍中的古铜皮蛊,一面咬着牙笑道:“李闲你不知道吗?使用这古铜皮蛊有个窍门,便是配合铜汁,能加快三成的速度呢。”

古铜皮蛊要催动一段时日,才能把蛊师的浑身肌肤养成古铜皮。若配合铜汁浇身,则会缩短时间。这点李闲也早有耳闻,但他没有想到这点。

为方源准备这些东西的时候,他还一直以为这是炼蛊的辅料。

皆因此种方法,最是痛苦残酷不过。

蛊师须得皮肤接触炙热滚烫的铜汁,而不能用其他任何的防御手段。除非是变态的自虐狂,才会选择这种方法吧。

但小兽王是自虐狂吗?

回去的路上,李闲一直在想这个问题,导致他都有些失魂落魄。

与此同时,还有一个人,也在念叨着方源。

“方源,你怎么还不出现?”白凝冰盘坐在地上,双眼半眯半睁,隐隐有种不妙的感觉。

“我和他动用过毒誓蛊,他绝不敢见死不救。但他现在还未出现,难道还留在传承里头,或者遇到了什么麻烦?”(未完待续……)

\end{this_body}

