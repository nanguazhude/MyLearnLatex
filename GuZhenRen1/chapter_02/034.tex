\newsection{深深的误会}    %第三十四节:深深的误会

\begin{this_body}

到了第二天,轮到白凝冰躺在床上,疼得动弹不得了。[138看书.138看书.]

方源则恢复了大半,到村东头锄田去。

老婆婆走进房屋,关心慰问。

白凝冰连说无事,可能是昨天干活有点累,休息一天就好了。

老婆婆笑得很有深意:“是干的有点累,这两天你们晚上动静有点大,我都听到了。”

“什么?”白凝冰一时间没有听懂。

“姑娘,你就别瞒着了,大娘我早就看出来了!”老婆婆笑着道。

白凝冰瞳孔一缩,难道身份败露了,怎么可能?心中顿时酝酿出一股杀机,但她又有些不忍。

她可以冷漠无情地旁观百家兄妹被烧死,但那是因为百家注定是他们的敌人。她虽然骄傲,但却不像方源那般毫无顾忌,对于有恩的人,她下不了手。

眼前的老婆婆是这样,曾经的白家族长也是如此。

老婆婆对白凝冰的心思毫无察觉,她抓起白凝冰的手轻轻拍拍:“姑娘,大娘这些天看出来了,男人哪有你这样的屁股和腰!难怪你戴着草帽,又不爱说话。大娘虽然老了,但仍旧是女人。咱们女人比那些臭男人有个优点,就是细心。”

“啊?”白凝冰一时间不知道该说什么。

老婆婆却很热心,一副体谅的语气:“大娘理解你,毕竟是女人,出门在外就得这样扮,把自己伪装起来。要不然出个三长两短,就不好了。”

白凝冰无语。

她最恨别人拿“女人”这样的词语,来**她的神经。但面对热情淳朴的老婆婆,她感到非常无奈。

老婆婆笑得眼睛都眯起来,忽然声音压低:“你们一定是小两口吧。这两天晚上你们的动静,大娘都听到了。不是大娘说你们啊,干那种事情,得悠着点。”

这话简直像是晴天霹雳一般!

白凝冰神情陡然凝滞,如五雷轰顶。

“大娘,不是你想的那样的。”好半天,她才艰难的挤出这句话的,神情僵硬至极。

“哎呀,还害臊什么呀。没啥的,什么话都可以跟大娘说。大娘我活了这么大岁数,什么没见过呀!”老婆婆眨眨眼,笑得咧开了嘴。

然后她的目光,似有意似无意地落到床单上。

白凝冰顺着她的目光看下去,顿时连死的心都有了。

不过话说回来,这床单的确是被她撕坏的……

接下来,老婆婆还跟她说了好些话,但白凝冰脑子里乱哄哄的,一句都没有听进心里去。

中午方源回来吃饭,老婆婆就在门口拦住他,好心的提醒道:“小伙子,你媳妇已经跟我说了。年轻人火气大,但也要爱惜自己的身子,更要爱惜媳妇。记住大娘的话了吗?”

“啥?”方源张大了嘴巴,一时间也没有反应过来。

老婆婆咂了一下嘴,有些不满又带着无奈:“你这小伙儿,什么都好,就是太憨厚。这么老实,是会吹亏的!”

若是古月一族,还有铁神捕,百花百生这些人,听到这话,说不定能气得活过来。

方源站在原地发呆,忽然目光闪了闪,这下终于反应过来了。

“哦……这事啊,嘿嘿……”他挠头憨笑几声,不迭地点头,“大娘,你教训的对,俺晓得了。”

在饭桌上,他见到白凝冰。

白凝冰冷冷地看他一眼,浑身都似乎冒着冰寒之气。

而方源的眼角,则一直微微抽搐着。

这个事情,给方源提了个醒。

方源演什么像什么,那是因为他经历丰富,眼见宽广。但白凝冰不是,因此她伪装起来,却是有破绽的。

幸好这个破绽很小,这个世界上凡人女子要出行,的确有装扮成男子,减少危险的习惯。

尽管对这个误会很腻味,但方源不得不承认,这个误会反而更能起到掩护身份的作用。

白凝冰的心情,则一下子变得糟糕透顶。

当天晚上,她再次质问方源,什么时候能将阳蛊给她。

方源只好明确回答,只要一达到三转修为,就交给她。

白凝冰冷哼一声,她了解方源,要她相信方源的诚信还不如让她去死!但是现在她却仍没有办法强取阳蛊。

“到了商家城,至少得买一只毒誓蛊。三只手蛊、豪夺蛊也行……”念及于此,白凝冰不由地对赶往商家城,更加急迫。

除了合作修行之外,方源的修炼内容又多增加一块。

那就是利用鳄力蛊,进行力量修行。

说起来,鳄力蛊能养到现在,有点出乎方源的意料。

最大的功臣,还得归功于百家。正是从百家得到了大量的鳄肉,方源才将鳄力蛊养到现在。

否则,它早就因为缺食而亡了。

鳄力蛊和黑白豕蛊相同,皆是永久性增加蛊师的力量。

但六转之前,蛊师到底还是**凡胎。就好像是一个碗,它不能装载一片湖。蛊师的躯体承载能力是有限的。

因此,方源之前还用不了鳄力蛊。好在,从白骨山中他获得了铁骨蛊以及玉骨蛊。

这两只蛊,都是消耗类的蛊。各有千秋,价值不分伯仲,都能够永久地提升蛊师的身体素质。

选择什么样的蛊,将决定蛊师今后的发展方向。

蛊师用蛊,有相当多的考究。有些蛊不能一起用,有些蛊一起用有一加一大于二的出色效果。

比方白凝冰曾经用过冰肌蛊,因此浑身肌肤都是“冰肌”。冰肌止汗止血,因此白凝冰今后就不能再用“血汗蛊”这类的蛊虫了。

而她用了玉骨蛊之后,原先的凡骨都成了玉骨。冰肌、玉骨,两者交相辉映,是一种上佳的使用搭配。

每个人的需求不一样,冰肌玉骨适合白凝冰,但却不适合方源。

考虑到商家城的那只传奇蛊虫,在方源的计划中,他最想要的组合效果是“钢筋铁骨”。

能从白骨山得到铁骨蛊,对他而言,是称心如意的。

在用了铁骨蛊之后,如今方源浑身骨骼坚硬若铁。身体底蕴的增强,已经能在两猪之力的基础上,再承担上一鳄之力了。

他的力气在不断地增加着。

七天一晃即逝。

按照原先的约定,方白二人从老村长处,得到了一板车的紫枫叶。

这种货物,价值很低廉,一板车也卖不到两块元石。方源也不放在心中,这只是他隐藏身份,跟随商队的敲门砖。

商队的到来,比老村长预料的,还要晚上三天。

一直到第九天,商队才迟迟出现。

原本平静的山村,陡然间热闹起来。

商队规模庞大。

一只只大如巴士的黑皮肥甲虫,载着货物和人,慢腾腾地爬着。

在它们的身旁,色彩斑斓的驼鸡,拖着板车。山地大蜘蛛的身上,绑着货箱。翼蛇蜿蜒游走,蟾蜍背着巨大的包裹。

这些坐骑蛊,构成商队的主体。除此之外,还有大量凡人赶着牛马骡车,或者背着竹娄之类的行囊。

“今年的商队,终于来了!”

“每次看到这些蛇,我心中都要害怕。”

“蛊师大人们真是了不起啊,能让这么凶恶的大蛇乖乖听话。”

“希望这次的腌肉能卖出去,我也不指望卖个高价,能有几颗碎元石我就心满意足了。”

“是啊,我们运气可没有外乡人好啊……”

“村长也太偏心了,那可是一板车的紫枫叶,就这样送出去了!”

村民们在村口铺了许多临时小摊,方白二人拖着一板车的紫枫叶,也夹杂其中。

这些人,有些是本村的。有些是外村赶来的,自身都带着货物。

有人的地方,就有利益之争。

方白二人哪怕在村东锄田七天,但一板车的紫枫叶,仍旧让身边的许多村民眼红嫉妒。

方源自然不会把这些话放在心里。

他暗暗观察往来的商队成员。

这个商队是个杂队,是很多家族势力,拼凑起来的。不像贾家商队,主体是以贾家为主。这个商队,除了公推的首领之外,还有一打子的副首领,仿佛是一只联合军。

这对方源来讲,是个好消息。

商队的结构越是复杂,越方便他混入进去。

“喂,你这车的紫枫叶怎么卖?”很快就有人过来问价。

“两块半元石。”方源道。

“两块半元石?你还不如去抢啊!”来人顿时瞪大双眼。

“爱买不买!”白凝冰一旁道。

“哼!”来人拂袖而走。

要真卖了这车货,方白二人拿什么借口加入商队?因此故意为之,不一会儿就气跑了三个问价求购的人。

一直到傍晚,这车紫枫叶都卖不出去。反倒是很多人的药草、腌肉、牛奶什么的,都售卖了大半。

毕竟商队人数庞大,这些东西也需要补充。

很多人都看方白二人的笑话,一些人甚至开始冷嘲热讽。也有好心人提醒他们适当降价。

但方白二人一概不理。

夜幕降临之后,方源装作一副垂头丧气的样子,拉着这车紫枫叶,来到老村长家。

老村长问明情况,叹了一口气:“你们呐,我明明嘱咐过你们,能买两块元石就不错了。一块半的元石也能出手。干嘛不听我的话?偏偏要卖两块半呢!”

(ps:晚八点还有更。)(未完待续。请搜索飄天文學,小说更好更新更快!)

\end{this_body}

