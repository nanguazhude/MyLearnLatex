\newsection{你这个无耻小人!}    %第一百一十一节:你这个无耻小人!

\begin{this_body}

转眼间,三四天过去。

“苦力蛊……”书房中,商睚眦看着手中的蛊虫,眉头拧成一个疙瘩,心中泛起一股冲动——想把这蛊捏死!

但他又不得不忍耐下来。

这苦力蛊,毕竟是他花费了八十一万的高价买下来的。捏死了,叫他心疼。

但每每看着这玩意,商睚眦的心中又十分郁闷。

这只苦力蛊,好像是无声的嘲讽,无时无刻的不停地提醒商睚眦他的愚蠢!

几天前的那场拍卖会上,他在众目睽睽之下,被方源算计。

如今,他已经成了众人的笑柄。就算是周围的亲族,也不待见他。

自己愚蠢也就罢了,但你是商家少主啊,此番行径简直是给商家抹黑!

商睚眦身为商家少主,一举一动,都在一定程度上代表着商家的形象。他在拍卖场的表现,不仅是他自身的耻辱,也羞辱了商家族人心中的自豪感、优越感。

至于商燕飞,倒没有什么表态。

不过,这更让商睚眦心中惴惴不安。

“不行,我一定要把场子找回来。我要让父亲刮目相看,我要让族人对我的印象改观!”商睚眦狠狠咬牙,下定决心。

“方正,你敢和我作对,敢戏耍我。我要让你付出惨烈的代价!”他的眼中阴冷的寒芒闪烁不定,开始琢磨着如何算计方源。

经此一事,他对方源的愤恨更加浓郁深厚。简直是深入骨髓的仇恨。

“少主,少主,大事不好啦!”

就在这时。一位心腹家奴奔跑过来,站在书房的门前大喊大叫。

“慌慌张张的,成什么体统?给我滚进来!”商睚眦不悦地喝斥道。

房门被推开,家仆扑通一声跪在地上,一脸的惊惶:“少主,事情不妙。不知从什么地方传起的,现在商家城几乎所有的大街小巷。都在流传。说少主你曾经和方正,争夺安渔姑娘失败。因此产生深切的仇恨,所以要教训方正。”

“安渔姑娘?那个秦艳楼的头牌?什么乱七八糟的。”商睚眦扯动嘴角。不屑地嗤笑。

但哪知家奴又接着道:“市井还在传闻,说少主你和方正有过隐秘交易,为了通过考评,做过……做过假账。”

“什么?!”商睚眦听得此言。顿时大吼一声。脸色剧变,腾的一下从座位上站起来。把书桌上的笔墨摆设,都带动得震倒下去。

家奴小心翼翼,语气急速:“少主,这事情传的有鼻子有眼。连具体的交易时间,具体的账目都传得清清楚楚。据说已经引起家族内务堂的注意,要派遣蛊师下来调查。”

“啊?!”

商睚眦惊骇欲绝,脑袋中像炸了雷霆。

在刹那间。他脸上血色褪尽,变得一片惨白。

他的心怦怦乱跳。浑身虚软,差点要瘫倒下去。伸手用力扶住书桌,这才勉强撑住身子。

这个打击,来的太快,太沉重,太突然了!

“完了,完了。假账一旦被翻出来,我一定会丢掉少主之位。这事触犯了家规,就算是母亲大人也不好为我说话。像父亲大人求亲,也没有用!我一旦没了这层身份,不知道多少人会对我落井下石!”

大难临头,商睚眦惊惶无比,陷入到极度的恐惧当中。

“怎么会有这样的消息流传出去?我明明做的很隐秘,除了方正之外,不可能有第三人知道的。不可能的,不可能的……”

商睚眦一个劲的在口中喃喃,眼神一片迷茫。

他到底还是太年轻了,没有经历过生死之间的磨砺。虽然掌管商铺两年,日理万机,但也只磨练出一张皮。遭逢大变,便变得六神无主,措手不及。

……

楠秋苑,湖中亭。

微风徐徐,吹动碧绿湖面阵阵涟漪。

湖泊不大,两岸是重峦叠嶂的灰石假山。假山周围,又种植翠竹和杉树。

湖面上,宽大的翠绿荷叶,如宝盘接连铺展,粉白的花苞点缀期间,还未绽放。金色、橘色的鲤鱼,在湖水中嬉戏,时而探水而出。

小亭金砖朱梁,珠帘画栋,显然精细雕琢。

亭中一张棋盘石桌,两位少年正在对弈。

一位少年郎身穿黑衣,双目幽幽如潭。一位少女白衣如雪,银发蓝眸,面色冷淡。

正是方源和白凝冰二人。

白凝冰下了一子,视线转到亭外,看着湖面:“这商睚眦真是蠢笨,反应真慢。我将消息发散出去,已经这么长时间了。到了现在他还不来?”

“放心,我已叮嘱了门卫,放他进来。他一定会来的。”方源笑了笑,展现出掌控全局的信心。

商睚眦惊慌失措之后,必定会对方源这边产生怀疑。一定会找过来,看看情况。就算不是怀疑,他也会过来联合方源串供,抵挡商家的这次调查。

白凝冰眯了眯眼,幽幽地叹了口气:“方源,我不得不佩服你,居然能在毒誓的内容中布下这道不是漏洞的漏洞。你早在两年前,就想对商睚眦不利了吧?只是一直忍耐着,直到如今时机成熟,才发动这个陷阱。”

方源呵呵一笑,提取一子,回答着:“商睚眦此人,气量狭小,定会报复我等。我怎会留着这样的祸害?只是先前不动,是因为商心慈初来乍到,还没有站稳脚跟。去了商睚眦,推商心慈上去,将来对你我都有大用。”

白凝冰没有再说话,蓝色的幽芒在眼中闪了闪。

她的心中,寒气涔涔。

这方源一计连着一计,紧密关联,叫人只要踏入其中,就如深陷泥沼,越来越不能自拔。更叫人心惊的是,他居然在两年前就已经算计到今天的情形。这份智谋,真是叫人胆寒。商睚眦和这样的人作对,简直是自找死路!

“方正,你在哪里?你给老子出来!”商睚眦的怒吼声,忽然传来。

湖中小亭,并无树木遮拦,商睚眦又得门童的告知,轻易就发现了方白二人。

“方正,你还有工夫在这里下闲棋?你知道外面都传成什么样子了吗?说,这一切是否是你做的!”商睚眦赶到方白二人面前,手指着方源,气愤无比的叱问道。

方源微微侧身,语气淡然地答道:“如果这一切是我做的,那我还会活着吗?想不到你越来越愚蠢了,商睚眦。你难道不记得我们一起使用过毒誓蛊?”

商睚眦冷哼一声,怒火稍稍减轻了一些。方源说的没错,如果是方源传播的这个消息,那么他早就死了。现在好端端的坐在自己面前,这证明消息并不是他传出去的。

但方源接下来的一句话,让他陡然间狂愤暴怒。

“不过这个消息虽然不是我传出去的,但却是我的同伴白凝冰传的。”

商睚眦楞了一下,原本缓和的脸上,爆炸似的发红,像是一个火星陡然落到了一盆火油当中。

一股庞大的怒气,从他心中窜起。

“是你,原来是你干的!你这个混蛋,我要把你挫骨扬灰!”他咆哮起来,双眼气得通红,目光如刀,狠狠地剐向白凝冰。气势疯狂,仿佛成了一头择人欲噬的虎豹豺狼。

“哦?你想要在这里动手?向我动手?”白凝冰缓缓地站起身来,绝美的面庞冷酷如冰,冰寒的语气中蕴藏着一丝不屑,“我是三转巅峰,演武场中无一败绩,我还有紫荆令牌,你真的想要与我生死搏杀?”

商睚眦脸上肌肉不断抽动,咬牙切齿,双眼好似在喷火一般,狠狠地瞪着白凝冰。

但他终究没有动手。

他只是三转高阶,又养尊处优,绝非白凝冰的对手。同时,白凝冰还有紫荆令牌,这令牌可是商燕飞亲自授予的。

“方正,你背信弃义,你不得好死!我们的交易,她怎么知道?不对,你违背了毒誓,怎么没有死?!难道你找到了什么接触毒誓蛊的方法?”商睚眦又转向方源,惊疑不定。

“非也,非也。”方源缓缓摇头,“毒誓的具体内容是——你我二人要保守秘密,‘不能泄露给不知情的第三者’。但在毒誓蛊之前,我早就将这事情告诉了白凝冰。所以白凝冰是‘知情的第三者’。外面流转的消息,也不是我传出去的。都是白凝冰的功劳。因此,我根本就没有违背当初的毒誓啊。”

商睚眦不由地张大嘴巴,眼中流露出惊愕的神色。

经方源这般提醒,誓约上的确是这样写的。

这是个漏洞,但当初商睚眦为什么没有发现呢?

一来是思维定势,没有想到方源早就泄露给白凝冰。“不能泄露给不知情的第三者”,单独看这句话,是没有漏洞的。

二来是他提出要用毒誓蛊,方源表现出一副没有准备的样子,事实上他早就有所预料,因此麻痹了商睚眦。

三来商睚眦为了抱住少主之位,走投无路,心中急切,再加上宣誓时毒誓蛊抽血的痛楚,让他难以静心思考。

现在发现,已经太迟了……

“方正,你这个无耻之徒!居然敢陷害我,这样坑我!你是阴险小人,卑鄙无耻至极!”商睚眦反应过来,气得浑身颤抖。

\end{this_body}

