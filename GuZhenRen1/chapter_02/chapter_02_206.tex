\newsection{今日暂且展翼去,明朝登仙笞凤凰!}    %第二百零六节:今日暂且展翼去,明朝登仙笞凤凰!

\begin{this_body}

%1
太古的荣耀之光,照耀着蚕茧。

%2
吸收着太古的光辉,蚕茧正在发生一种玄妙的转变。

%3
方源浑身金黄灿烂,在天瀑光河中逆流飞翔。面对五转蛊师的强大杀招,金汤蛊单纯的防御,渐渐不起作用。

%4
黄金的防护,被光芒渐渐冲走。

%5
尤其是骨翼上面,已经渐渐地被天瀑光河,冲刷出原本的黑色。

%6
恢弘磅礴的光明天河中,一个单薄的身影,顶着庞大的压力,艰难飞翔。

%7
如此奇景,吸引了众多蛊师注视。

%8
偌大的战场沉寂下来。

%9
地灵全力辅佐着方源,犬兽失去了指挥,大部分都在溃逃。而蛊师们也一个个停住脚步,纷纷仰头望去。

%10
他们的心中,都有一个相似的疑问——“这个情形,我明明没有看过,但为什么我如此熟悉?”

%11
“我想起来了,我想起来了!”忽然不知道谁喊起来,“这样的情形,在人祖传中记载过。难怪这么熟悉!”

%12
《人祖传》乃是天下第一经典,为世人广为传颂,没有人不熟悉它的。

%13
被这么一提醒,很多人猛然醒悟。

%14
“不错,人祖传的第二章第三节有着相关的记载。”

%15
“我也想起来了。太日阳莽为了炼成仙蛊定仙游,挥动双翼,沐浴在荣耀之光中,飞向太阳。”

%16
人们回忆出来,顿时引发一阵笑声。

%17
“这人是怎么回事?模仿太日阳莽,居然不顾性命?”

%18
“哈哈,他难道也是想炼成仙蛊定仙游吗?”

%19
“怎么可能!他有神游蛊吗?”

%20
方源当然有神游蛊!

%21
但有神游蛊还不够,人祖传中记载——

%22
太日阳莽担忧神游蛊,会趁着他醉酒,将他置于险境。神游蛊也感到惭愧。便指点他道:“你先去天上,在九重天中的青天里,有一片竹林。在竹林中,采摘一节碧空的玉竹。再到九重天的蓝天里,在夜晚的时候,收集星光碎屑中的八角钻石。然后你在清晨时分,飞向天空,借助朝阳的荣耀之光,将我变成定仙游蛊。我成了那个蛊后。就再也不会带着烂醉的你乱窜了。”

%23
所以,方源还要有青天里碧空的玉竹,以及蓝天里八角钻石的星屑。

%24
还有,太古的荣耀之光。

%25
他有吗?

%26
他原先是没有的。

%27
但重生之后,他杀了龙青天。取得了碧空蛊。

%28
此蛊,乃是五转蛊,源自太古时代,好似一节墨绿的竹干,巴掌大小,中间空通,摸在手中如玉一般细腻润泽。

%29
正是青天里碧空的玉竹!

%30
《人祖传》中描述了各种各样的蛊。仙蛊都是直接描述。诸如智慧蛊、力量蛊等。而凡蛊,则含蓄地点出来,描写得很隐晦。需要读者,深入的挖掘和细细的研究。

%31
但单有碧空的玉竹不行。他还得要八角钻石的星屑。

%32
他有吗?

%33
他当然没有,不过白凝冰给他悄悄种下了!

%34
没有错,那就是定星蛊。

%35
此蛊乃是太古星屑,宛若钻石。生有八角,晶莹剔透。种在方源的左前臂时。绽放星光,能将他的左前臂映照成一片半透明的幽蓝。

%36
青天里碧空的玉竹,以及蓝天里八角钻石的星屑,都有了,但方源若要炼仙蛊定仙游,还缺少一个条件——那就是地灵所说的——太古的荣耀之光。

%37
方源有么?

%38
至始至终,方源都没有。

%39
但方源没有,萧芒却有。

%40
萧芒掌握着一只太光蛊,此蛊乃盗墓而得,只是一只残蛊。每个月,只能催动三次,能催发出荣耀之光。三次一过,就要自毁。

%41
而天瀑光河这个杀招,从某种方面来讲,就是太古的荣耀之光!

%42
神游蛊、碧空蛊、定星蛊以及太古之光,一切的条件都具备了!

%43
当方源重生之后,忽然意识到这点时,他在心中就毅然舍弃第二空窍蛊,转为改炼定仙游蛊。

%44
但是要直接说服地灵,是根本不可能的。

%45
地灵的执念,就是要炼出第二空窍蛊来。

%46
那么,白凝冰、铁若男以及正魔两道的群雄,就成了方源的利用对象!

%47
他精心算计,运筹帷幄,顺势而为,终究营造出这个局面。当地灵意识到无论如何,也不能炼成第二空窍蛊的时候,它自然就会选择退而求其次,保护方源,保住这份希望。

%48
蚕茧吸收着阳光,微微颤抖,仙蛊的气息不可避免地逸散而出。

%49
群雄震惊!

%50
“这样的气息,怎么可能!?”铁若男、白凝冰等人差点把眼珠子瞪掉。

%51
“他在炼蛊,他的确在炼仙蛊?!他究竟是什么人?难道是太日阳莽重生?”易火、翼冲等人,嘴巴张的老大,下巴都要掉下来了。

%52
“定仙游!他真的在炼制定仙游蛊吗?想不到我风天语,居然有幸能看到炼制仙蛊的场面啊!”这位炼蛊大师双膝一软,跪倒在地上,泪流满面!

%53
福地中,所剩不多的毛民,也都纷纷跪拜下来。

%54
这一瞬间,它们疯狂地崇拜,这个正在炼制仙蛊的人!

%55
无法置信。

%56
太古的情形,就在眼前上演……

%57
难以想象的恢弘和壮美,让无数的蛊师都浑身颤抖。不晓得是因为激动,还是害怕,或是两者皆有?

%58
一瞬间,方源的身影,成为所有人目光的焦点!

%59
哪怕天瀑光河再刺眼,所有人都张大眼睛,一眨不眨地盯住。

%60
仙蛊的气息越来越浓,但就在这最关键的时刻,异变突生。

%61
“居然想借我之力,炼成仙蛊?哼!”萧芒不是笨蛋,明白过来后,立即停止杀招。

%62
天瀑光河断流!

%63
众人一同发出最大的惊呼声响。

%64
“不——!”风天语甚至哀嚎起来,痛彻心扉。悲痛绝望。

%65
在他的视野中,天瀑光河像是被剪断的绸带,向下无力地飘飞。只需要三个呼吸,光河中的蛊师就会飞出光河。而这短短的时间,绝对不够仙蛊孕育而出。

%66
但方源又岂会没有料到这点?

%67
三更蛊!三更蛊!

%68
他将两只三更蛊,都对着蚕茧催动。

%69
顿时,时光加速九倍,仙蛊气息暴涨!

%70
“他用了宙道蛊虫来加速?”风天语像是触电般,从地上爬起来。双眼放光,脸色涌现出强烈的红晕,竟然还有希望?

%71
但旋即,兴奋的红晕转为苍白,风天语如如丧考妣。又扑通一声,失魂落魄地坐在地上,哀嚎着:“没有用的。如此加速,只是饮鸩止渴,仙蛊孕育太快,稳不住气息,必然自爆而毁啊……”

%72
但方源怎会没料到这点?

%73
他取出一种蛊虫。

%74
此蛊其貌不扬。好似灰石圆片。

%75
上一世,方源从风天语手中得之。这一世,乃杀人鬼医仇九奉上。

%76
何蛊?

%77
百战不殆也!

%78
百战不殆蛊,五转消耗蛊虫。一旦用之,便能令蛊师一次炼蛊必成!

%79
下一刻,风天语惊呆了,抱住脑袋。狂喜地嘶吼起来:“这不可能!”

%80
皆因他感受到仙蛊的气息,竟然奇迹般地稳定下来!

%81
蚕茧破开。飞出一只绿光莹莹的翩翩蝴蝶——定仙游蛊!

%82
“真的是仙蛊啊!”

%83
“美轮美奂……”

%84
“他到底是何方神圣,竟然炼成了仙蛊?!”

%85
一时间,众人心中掀起惊涛骇浪,不管是知情的,还是不知情的,都怔怔无语,震惊到了极点。

%86
萧芒傻眼,魔无天呆滞。

%87
“眼前的是神话重演吗?”

%88
“我究竟是生活在哪个时代啊?!”

%89
方源飞出光河,身边定仙游蛊在环绕飞舞,每一次振动双翼,都挥洒出蓬勃的盈盈绿光。好像是玉屑,美不胜收。

%90
当然,方源也付出了不菲的代价。

%91
五转杀招不是开玩笑的,金汤蛊寿终正寝,金霞蛊等等也因此大损。背后的黑翼,也残破不堪。

%92
金汤褪下,露出方源的真容。

%93
一时间,整个战场哗然一片。

%94
“他是何人?”魔无天瞳孔猛缩。

%95
“是,是小兽王!”狐魅儿、李闲完全惊呆了。

%96
“就是他!”易火眼珠子都凸出来。

%97
“竟然是他?!”焦黄、孟土两人对视一眼,瑟瑟发抖,均看出彼此的惊恐、后怕、庆幸。我们居然胆大包天到要暗杀这样的强敌?一个能炼出仙蛊的男人!?

%98
“方源……”知情的白凝冰、铁若男等人,亲眼目睹着这场奇迹,仿佛是雕塑一样站着。

%99
数十万年前,太古时代,太日阳莽振动双翼,炼出定仙游蛊。

%100
而今,方源亦同样如此,以凡人之躯,众目睽睽之下,做出壮举。

%101
此事一旦传出,他势必将震动南疆,名传天下!

%102
“你真的炼成了仙蛊定仙游,了不起!果然不愧是未来的蛊仙。”方源的耳畔,传来地灵的赞叹声。

%103
方源朗笑一声:“定仙游蛊,能令蛊师纵横天下,去任何一个想要去的地方。但它乃是仙蛊,还需要你帮忙,用仙元来催动。”

%104
霸龟:“这是当然的。铜鼎中还残留着一些仙元,你想要去哪里,就先在脑海中想好了。你最好选个安全的地方,而且要注意,你脑海中的记忆画面,和现实地点要相差无几,不能差别太大。”

%105
“这我当然明白。”

%106
霸龟叹了一口气,语重心长地道:“用了定仙游,必能令你脱离重围。而没有了仙元,福地便会立即毁灭。死亡不过如此,对我来说,也是一个解脱。只盼你能日后炼成第二空窍蛊,不辜负这场机缘。离别之前,你还有什么话要对我说吗?”

%107
方源张口欲言,却说不出话来。

%108
他振动残破的黑翼,飞翔在空中,俯视四方。

%109
破开大洞的青铜大殿,血流漂杵的山丘战场,还有破碎不堪的蛊仙福地……

%110
别了。白凝冰。

%111
别了,铁若男。

%112
别了,南疆。

%113
地灵不知,方源此去冒着惊天的风险。但人生本来就是一场豪赌,此时不搏更待何时?

%114
男儿不展凌云志,空负天生八尺躯!

%115
这样想着,方源不禁壮怀激烈,心潮澎湃。于万众瞩目中,心潮奔腾。化为一诗。

%116
群雄只听他脱口长吟道——

%117
千古地仙随风逝,昔日三王归青冢。

%118
阳莽憾陨谁无败?卷土重来再称王。

%119
天河一挂淘龙鱼,逆天独行顾八荒。

%120
今日暂且展翼去,明朝登仙笞凤凰!

%121
吟罢,方源哈哈大笑。

%122
众皆怔怔无语。

%123
唯有地灵大叫:“好志向。我便送阁下一程!”

%124
仙元灌输到定仙游蛊当中,碧芒一闪即逝,带着方源消失在空中。

%125
只剩下他的衣裤,如断了线的风筝,从高空中飘落下来。

%126
“他消失了!”

%127
“用了定仙游,不知道到哪里去了。”

%128
“啊!怎么天地都在摇晃?”

%129
众人傻眼。

%130
这时狂风呼啸而起,山崩地裂。天塌地陷。一个个漏洞,沟通外界,旋即形成。

%131
“该死的,快逃。”

%132
“福地真正崩溃了。很快大同风就要刮起来了!”

%133
“再不逃就要丧命于此,我不想啊!”

%134
群雄惊惶失措,疯狂溃逃,整个三叉山在瞬间乱成一团。

%135
……

%136
中洲。天梯山。

%137
狐仙福地,荡魂山上。一场关乎狐仙福地归属的竞争,也到了最后的关头。

%138
“方正,加油,胜利近在咫尺了!”天鹤上人鼓舞道。

%139
方正不断攀爬,手脚已经磨破,鲜血淋漓。

%140
他接连超越了应生机、萧七星,浑身痛得已经麻木,脑袋更无法思考其他,眼中只剩下荡魂山巅。

%141
第一个登顶,已经成了他的执念!

%142
“我堂堂凤金煌,怎么可以输在这里?我自出生以来,从未输过。这一次也不例外!出来吧,梦翼!”

%143
凤金煌娇喝一声,从身后肩膀处生长出一对绚烂的羽翼。

%144
这对羽翼,极尽华美艳丽,各色光辉不断流转,璀璨夺目,轻轻一扇,便带着凤金煌徐徐上升。

%145
“什么?”

%146
“这是……”

%147
“传说中的仙蛊——梦翼!”

%148
九大蛊仙为之惊愕。

%149
大多数的仙蛊,是用仙元才能催动。但梦翼不同,催动它需要消耗的,是蛊师的灵和魂。

%150
凤金煌只是凡躯,现在强行催动梦翼,冒着极大风险,轻则失忆,重则痴呆。

%151
但凤金煌求胜心切,一心想要胜利,甘愿付出沉重的代价!

%152
在方正吃惊的注视下,她超过方正,重新夺回领先的地位。

%153
梦翼乍然收起,凤金煌攀在山崖边上,狠狠地喘着粗气,从灵魂中传来的眩晕感,强烈到让她几乎要昏厥过去。

%154
达到极限了。

%155
强行催动仙蛊,凤金煌能做到这一步,已实属不易。

%156
“我竟然输了!”方正瞪大双眼,失魂落魄。

%157
凤金煌的双手已经抵达了悬崖边缘。

%158
“我,我要……胜利!”

%159
这一刻的凤金煌,拼尽全力扬起头颅,迸发出最后一丝力量。

%160
她的双眼,亮如琥珀。容颜娇丽无俦。雪白的修长脖颈,在福地粉色的光辉中,流转着玉一般的光晕。

%161
她就像是一只雏凤,第一次向天地展开亮丽的羽翼。

%162
锦绣辉煌!

%163
一时间,就连蛊仙们都为其失神。

%164
方正仰望着她,狐仙地灵也怔怔地看着她,所有人都在等待凤金煌的胜利。

%165
凤金煌不负众望,她咬破嘴唇,艰难地将手臂搁在悬崖边上。

%166
然后她奋起余力,正要把沉重的身躯也拖上来。但就在这时!

%167
刷!

%168
玉光乍现,山顶处陡然出现一人。

%169
这人赤身裸体,左前臂破开一个洞,血流不止。年轻的身躯,魁梧有力,肌肉结实,又散发出历经百战,千锤百炼的深沉气息。

%170
“哥?!”方正惊呆了,一失手,从山壁上滑落下去。

%171
十大蛊仙同时傻眼,这,这是哪里跑来的裸男?!

%172
凤金煌仰起雪白的脖颈,从方源的脚底下,望着方源,惊愕至极,双眼瞪大,如同石像。

%173
方源浑身的肌肉,胯下的巨物,都毫无保留的映在凤金煌的一对秀眸之中。

%174
“果真传送到这里来了吗?切,定仙游蛊,就是有这个缺点,不能传送体外衣物。不过好在我将蛊虫,都存进了空窍之中,一并带来了。”

%175
方源迅速扫视周围,明白自身的处境。

%176
“咦?脚下的这人,不就是凤金煌么?”

%177
看来《凤金煌传》中,记载的时间还是靠谱的。自己先她一步登顶,这场竞争不禁仙蛊,按照规矩,那自己就是狐仙福地之主!

%178
“我成功了,一切的冒险都是值得的。第二空窍蛊炼制不成算什么?我现在拥有更好的狐仙福地,还有仙蛊定仙游!哈啊哈哈……”方源在心中狂笑。

%179
“可惜,她有仙蛊梦翼,自己目前还杀不了此人。”方源遗憾地俯视着悬崖边上的凤金煌,然后抬起自己的右脚。

%180
在十大蛊仙的注视下,他的右脚,结结实实地踩在凤金煌锦绣的容颜上。顿时一股滑腻如温玉的感觉,立即从脚底板传来。

%181
“给我下去吧。”方源一用力,将浑身无力,惊愕呆滞的凤金煌踩了下去。

%182
凤金煌本就乏力至极,哪里抵得上方源的力气,立时坠落下去。

%183
方源施施然转过身,面对着地灵狐仙。

%184
“小狐仙,还不叫主人?”

%185
“主,主人……”狐仙女童望着方源,都惊呆了。

%186
反应过来后,她刷的一下,用粉嫩的小手捂住一对亮晶晶的大眼睛。

%187
然后,她低下头,脸红到耳朵根,跺着小脚,摇晃着脑袋,羞涩地叫道:“主人,你羞羞,你这么大了,还不穿衣服!”

\end{this_body}

