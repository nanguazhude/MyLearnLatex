\newsection{骨肉团圆}    %第二十七节:骨肉团圆

\begin{this_body}

%1
“形如狮虎的头骨雕像,这不就是传闻中那个需要二人合力的机关?”方源心中立即闪过一个念头。

%2
“这雕像的尖牙上,还刻着字。”白凝冰又有新发现,她顺着字低声念道,“双子同心,三灵合一。有缘无缘,切莫强求……这是什么意思?”

%3
“这是开启此处机关的密语。双子,表示必须是两个人一起行动,才能开启机关。三灵,分别指的是人的心、掌、目。”方源回忆道。

%4
人体四肢中最灵巧的是手掌,五官中最灵动的是双眼,而人的心念思绪快若电光火石,最是灵活。

%5
所以,统称为三灵。

%6
“来,将手心贴在雕像的瞳孔上。”

%7
头骨雕像的眼眶中,塞着一对晶莹剔透的红宝石。宝石大如海碗,上面倒映出方源和白凝冰两人的身影。

%8
但是方白二人,将手心搭在宝石瞳孔上,久久却毫无异动。

%9
“呵呵,说的头头是道,结果却是错的。”白凝冰不放过挖苦方源的任何机会。

%10
方源亦是脸色一沉,前世的记忆中,这处机关被百花重点描述过。按道理来讲,就是这样做才对。但为什么却没有动静呢。

%11
“双子同心,三灵合一……”方源口中喃喃,“三灵合一他是做到了,但是同心,同心……”

%12
随着思考,他的眼中渐渐泛出一丝神采。

%13
难道说,要开启这道机关的两个人,必须是同心同德的么?

%14
若是这样,自己和白凝冰虽然在一起行动,但却是情势所迫,事实上貌合神离,各有算计。怎么可能做得到“同心”!

%15
想到这里,方源不由地再次将目光落到百生、百花的身上。

%16
于是,这对兄妹再次被方源踢醒。

%17
“恶贼,你究竟想要怎样?!”百生苏醒之后,发出愤吼。

%18
百花却也不哭泣了,一双大眼睛紧紧地盯住方白二人,流露出浓郁的仇恨之色。

%19
这次方源已经不耐烦解释。直接抓住两个人的手掌,分别贴在这一对红宝石的瞳孔上。

%20
不愧是命中注定的继承人,这两人的小手刚贴上去,红宝石就骤然发亮。

%21
咔嚓咔嚓。

%22
头骨缓缓张开嘴巴,露出里面一大堆的煤石和干柴。

%23
在黑色的煤石中央,是一个形制古朴的陶瓮。一份卷轴,依靠在陶瓮上。

%24
“这是何意?”方源将两人随手甩在地上,拾起卷轴一看,这才了然。

%25
原来这传承之主灰骨才子,资质不高,一生修行都为此感到苦恼。

%26
终其一生,他都在致力于研炼出一种蛊,能够帮助蛊师快速修行。

%27
辅助修行的蛊虫,已经有许多。最典型的,就是酒虫。但是这些蛊虫,大多珍稀无比,难以推广。

%28
灰骨才子心志很大,想研炼出一种能够广泛运用的优良蛊虫。

%29
但直到他寿命完结,失败过无数次,也没有试炼成功。

%30
在他生命的最后关头,也许是上苍的怜悯,使得他在布置白骨山传承的时候,忽然有了一个极其巧妙的灵感。

%31
在没有蛊虫辅助下,一位蛊师想要提升修为,最主要的方法是什么呢?

%32
那就是借助长辈的力量,进行灌顶。

%33
当初在青茅山,古月赤城就一直接受着他的爷爷古月赤练的真元灌输。

%34
但是此举,却有一大弊端。

%35
那就是真元各异,依靠长辈的优异真元来洗练窍壁,会留下异种气息,将来会极大地制约蛊师的发展。

%36
除非用净水蛊来洗去异种气息。

%37
然而,净水蛊也是数量稀少的蛊,寻常蛊师很难得到。就算是家老级人物,也要靠运气,耗费很大的代价。

%38
因此,灌顶方法也并不普及。

%39
于是这灰骨才子就有了一个奇妙的构想。

%40
如果有一种蛊,能够将别人的真元,转化为自己的真元,那么灌顶之后,岂不就没有了异种气息这个后遗症吗?

%41
经过一系列的尝试,他排除了许多可能,最终留下了一个成功性最大的方案。

%42
这个方案的名字,便是——“骨肉团圆蛊。”

%43
卷轴中言道:要炼制出这种蛊,必须得两位蛊师同时出手。而这两位蛊师,必须有血亲关系。父母和子女,或者双胞胎。依靠同根血脉中的联系,才能进行真元的转换。

%44
当然这个构想,灰骨才子已经来不及实践了。在他做了大部分的准备之后,他在最关键的地方无奈止步。

%45
他虽然有两个称号,却是单独一人。他缺少两个符合条件的蛊师。

%46
卷轴最后的内容中,显露出灰骨才子的无尽遗憾。

%47
他没有时间再做准备,只能留下这处高台。但如果后来人有缘至此,能够开启这道机关,看到这份卷轴,那就说明符合条件的蛊师出现了!

%48
“不妨尝试炼制,不管结果如何,请在我的坟前倾述一声。”卷轴中的这句话,包含了灰骨才子一生的执念。

%49
原来方源脚下的这座平台高塔,就是灰骨才子的坟墓。

%50
不用尝试,方源已经知道,骨肉团圆蛊的构想是成功的。因为在前世,百生、百花就是凭此双修,成为正道双星,以五转修行将百家的势力推到鼎盛阶段。

%51
但对于方源来讲,现在却有些麻烦。

%52
他原本以为,骨肉团圆蛊已经是成品,但事实上它还没有炼制,连半成品都算不上。

%53
要炼制出骨肉团圆蛊,他和白凝冰也不满足条件。

%54
除非古月方正在场。

%55
不过就算是这样,恐怕炼制出来的骨肉团圆蛊也不会很好。

%56
按照卷轴中所述,骨肉团圆蛊是一个系列的总称,而非特指一种蛊。炼制蛊虫的两位蛊师,越是情同意和,炼制出来的骨肉团圆蛊的品质便越好。

%57
依照方源和方正的关系,炼制出来的骨肉团圆蛊绝不会理想。

%58
在方源前世,这骨头团圆蛊,定然便是百生、百花所炼。但是如今提前了这么多年,他们虽然情同意和,但却还没有成长为蛊师,也满足不了这个标准。

%59
骨肉团圆蛊,是方源最重要的目标。如今材料、熔炉都已经被灰骨才子准备好了,只欠临门一脚的炼制。

%60
要放弃,方源自然不甘心。

%61
但是要炼制,先不说满足不了合炼条件,后面还有一群强敌正在追杀过来。

%62
时间紧急,方源咬了咬牙,决定冒险一试。

%63
他和白凝冰两人只占标准的一半,但百生、百花这对兄妹却占了标准的另一半。若是四人合作,或许有成功的希望。

%64
“来搭把手罢。”方源开始点燃头骨塑像中的干柴。

%65
火焰瞬间升腾起来,熊熊燃烧。

%66
“你打算强行炼制?”白凝冰吃了一惊,“这可不是什么明智的选择。”

%67
话虽如此说,但她终究还是出手,掌心贴在红宝石瞳孔上,往里面灌注真元。

%68
火焰骤然变色,从橘黄色变成幽蓝。

%69
陶瓮在火焰中接受烤制,里面沉眠的数只蛊虫被烧醒,开始疯狂的挣扎。陶瓮不断颤动,却并未破损。

%70
炼制的步骤并不繁琐,方源和白凝冰交替灌注真元。

%71
很快,就到了最关键的一步。

%72
此步需要两位蛊师身上的新鲜血肉,投入火中煅烧。卷轴中言明,此步骤血肉越多,效果越好。

%73
“正好有肉白骨在手,割下几块肉来也无妨。”白凝冰正欲动手,却被方源阻拦下来。

%74
“慢着,我还有一个更好的主意。”

%75
白凝冰顺着方源的视线,看向百生、百花这对胞胎兄妹。

%76
“你竟然要临时篡改秘方?”白凝冰顿时明白了方源的意思,眼中闪过一丝不忍。

%77
“你们,你们想要干什么?!”百生将妹妹护在身后,一时间,他感到了大祸临头、末日来临的焦躁和恐慌。

%78
人为刀俎我为鱼肉,在方白二人炼制蛊虫的时候,他们心知逃脱不掉,因此一直乖乖地呆在一旁,等待族人救援。

%79
但现在,百生心中极度懊悔!

%80
“感到荣幸吧,你们的牺牲,将见证一种全新蛊虫的诞生。想来,灰骨前辈泉下有知,也会很开心的吧?”方源狞笑着,向这对兄妹逼来。

%81
“妹妹,快跑!”百生大叫着,不退反进,冲到方源跟前,一把抱住方源的腿。

%82
“哥哥!”百花眼泪夺眶而出,正当她犹豫的功夫,方源已经将百生敲昏。

%83
眼看着方源向她逼来,巨大的恐慌蔓延在小姑娘的心头。

%84
她转身就跑,但怎么及得上方源的步伐?

%85
她很快就被方源抓住,一颗心沉落谷底,徒劳的挣扎中失声哭喊:“娘,你在哪里呀?”

%86
方源面容冷酷至极,又将她敲昏。

%87
三下五除二的,将这对兄妹的衣服都扒光,然后一手提着一个,将他们抛入到火中。

%88
他们一进入火中,顿时就被烧醒,剧痛传来,让他们发了疯似的奔逃。

%89
就见两个火人从火堆里跳出来,四肢狂舞,胡乱挣扎。

%90
方源冷哼一声,一脚一个,又将他们踢回到火焰里头。

%91
卷轴中,要求是新鲜血肉,方源却不想将他们踢死。因此踢进去后,他们又跑出来,然后又被踢回去。

%92
如此三番五次,百生、百花终于烧死在火焰中。

%93
他们的身躯,像是蜡烛一样渐渐融化。火焰由幽蓝色转化为血般的鲜红。

%94
然而,却迟迟转变不成大红大紫的颜色。

%95
按照卷轴中所述,只有火焰的颜色转变成红紫色,这关键一步才算成功。

%96
“怎么办?”白凝冰眉头紧皱,合炼蛊虫一旦失败,蛊师也会遭到反噬的。

%97
方源脑海中思绪念头剧烈翻腾:“百生、百花乃是命中继承的人,居然也不行?看来是因为他们还不是蛊师,因此有了差别。如此的话,只有再试!”

%98
想到这里,方源再不迟疑,催动兜率花,吐出一柄利刃。

%99
他伸直了前臂,利刃一削,鲜血横流,自身的一块血肉就被他抛入到一人多高的火焰当中。

%100
“你来。”做完这一切,他将利刃抛给白凝冰。

%101
“你确定有用?”白凝冰犹豫了一下,同样用利刃切自己的前臂。但因为冰肌的缘故,利刃像是砍在坚冰之上。

%102
不得已,白凝冰只好唤出锯齿金蜈,将自己的一块肉绞了下来。

%103
当她的这块肉,抛入到火焰之中后,顿时火焰顺利转变成了紫红色。

%104
“好,成败就在此一举了!一起灌注真元。”方源大喜过望。

%105
二人同时向头骨的宝石瞳眸中倾注真元,机关开始缓缓闭合。仿佛是一只骨兽,张开大口,将熊熊燃烧的火焰吞入口中。

%106
两排尖牙紧密地咬合在一起,牙关紧闭,火焰在其中燃烧,将头骨都灼烧成一片绯红色。

%107
砰的一声,似乎是陶瓮发生了爆炸。

%108
整个头骨都因此一颤。

%109
听到这声炸响,方白二人齐齐松手。

%110
方源一边紧盯着这边的动静,一边伸手向白凝冰。

%111
他没有开口,但白凝冰已经知道方源想要什么。

%112
她哼了一声,为此刻大局着想,只得将肉白骨交给方源。

%113
她没有顷刻炼化蛊虫的本事,但方源却有。

%114
肉白骨在春秋蝉的气息下,没有丝毫的抵抗之力,被瞬间炼化。

%115
不过方源虽然炼化,但是碍于修为,却用不来肉白骨。因此又转交给白凝冰。

%116
白凝冰接到手中,立即催用,一捧橘黄色的光芒笼罩在前臂的伤口上。几乎眨眼功夫,皮肉就长好,伤势痊愈了!

%117
但是白凝冰的三转巅峰的真元,也在瞬间消耗掉两成!

%118
肉白骨的弊端,就是需要瞬间消耗大量的真元。换做方源的青铜真元,哪怕是用到元海干涸见底,也催动不起来。

%119
紧接着,白凝冰又给方源治疗。

%120
方源脸色苍白,短短片刻,他没有冰肌可以止血,因此失血较多。

%121
前臂上的伤势虽然痊愈了,但是疼痛感却依旧强烈,整个心弦都随之痛得颤抖。甚至都有头晕目眩的感觉!

%122
不过两人都是意志如铁之辈,尽管如此,两人亦是面不改色,硬生生地撑住剧烈的痛楚。

%123
须臾之后,头骨缓缓张开,里面的火焰早已经消散。

%124
陶瓮以及百生、百花的尸骸,都没有丝毫残留。

%125
两只蛊虫出现在方源的视野当中。

%126
他们一青一红,如两个玉镯,相互套在一起。悬浮在半空中,静静地散发出温润的光泽。

%127
“这就是骨肉团圆蛊么?”没有时间细细查看了,方源捞到手中,瞬间炼化,就收入空窍。

%128
“走!”他拔腿飞奔,跑下高台,率先钻入大厅尽头的新密道。

%129
仅仅只是片刻,百家蛊师来到了这处大厅。

%130
“有人在这里炼过蛊!”空气中残留的气息,让许多家老都神色一动。

%131
“快看,这里有两位少主的衣衫。”很快,他们就发现高台上被方源扯烂撕碎的童装。

%132
看到此处,一股极其强烈的不妙感,冲击百家族长的心田,几乎让她眼前一黑。

%133
她甚至不敢去联想。

%134
“追!他们就在不远处,我的孩子也一定在他们的手中!”百家族长嘶吼着,双眼充斥血丝,一片通红。(未完待续。如果您喜欢这部作品,欢迎您来138看书文学注册会员推荐该作品,您的支持,就是我最大的动力。)

\end{this_body}

