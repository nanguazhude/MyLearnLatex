\newsection{青天碧空,狐仙金煌}    %第一百八十九节:青天碧空,狐仙金煌

\begin{this_body}

“想不到我的运气这般好,居然能无意间撞见这处薄弱地域。”龙青天目光闪烁,流露出明显的激动之情。

要知道三王传承,虽然只有百关,但每一关卡都有无数种选择,宛若分枝茂盛的参天巨木。

他能到达这里,比万中选一还稀有,实属幸运至极。

“这里天地压制,十分虚弱。竟然能令我动用一只蛊。哈哈,我就先用碧空蛊,毒烂这片福地,这样一来,就能勾通外界,形成通道。”

“通道一形成,我就能在此附近,随意催动蛊虫,不用再去遵守三王的破规矩了。然后,我在以此为基,不断辐射,大捞好处。呵呵呵,嗯?什么人!”

龙青天察觉到异状,猛地转身,就见远处陡现一人。

“小兽王方正?”龙青天眯起双眼,察觉到方源来意古怪。

方源没有答话,直接挥手,一记骨刺闪电般飞刺而出。

龙青天冷哼一声,急忙催动蛊虫防御。

他身边尽是青绿色的碧空蛊毒,已经侵蚀了好大领域。方源没有解毒手段,不敢碰触蛊毒,只能进行远战。

此时,方源的身上只剩下异兽虚影,但他却无四转的全力以赴蛊。因此不能随意催动而出。

不过好在他斩杀了那么多蛊师高手,缴获了许多蛊虫,可以用于远战。

几个回合之后,龙青天被方源顺利击败。

龙青天只能在同一时间,催动一种蛊虫。但方源却有地灵的协助,可以随意动用。两者之间,战力差距极大。

虽然胜利了,方源却脸色难看。

龙青天的尸体。一片惨淡青绿之色,明显是被碧空蛊毒侵蚀。

这位魔道成名的高手,在临死之前,阴狠地朝着方源一笑,悍然动用碧空蛊,将自己也毒杀。

方源没有杀掉龙青天,他明知绝境,就自杀了。

“不愧是一代魔头。”方源吐出一口浊气。

碧空蛊乃是太蛊奇毒,已然绝迹。在当今南疆。恐怕也只有四大医师才能救治。

方源若要取蛊吞窍,势必得碰触到蛊毒。若中了碧空蛊之毒,不消一个时辰,全身就要化成一片青光散去。

很少能有蛊师,中了碧空蛊毒。能够幸存下来的。

武家的武神通,已经算是极其幸运的个案。

就算方源拼着中毒,也未必能获得蛊虫。此处天地压制薄弱,龙青天完全可以一一自爆了蛊虫。

收益和风险如此不成比例,方源自然不会冒险。

“但如此一来,最后一份仙元预算,也就耗在了龙青天的身上。兽力胎盘蛊的资质。只能停留在八成九了。”

方源心中十分遗憾,到头来,千辛万苦,还是没有达到预计的九成。

“到底还是自己实力太弱。这种谋算之事。只能尽人事,听天命。”

龙青天的这个突发事件,弄得方源有些措手不及。

方源终究是人,不是神。无法料到这层变化。

他虽然有前世记忆,也尽量地回忆清楚。但这种细节,只要当事人不说,谁会知道?

况且,他重生以来,改变了不少事情。龙青天这件事,到底是不是前世发生的,还很难说。

“兽力胎盘蛊也就罢了,更糟糕的是,这片天地已经中了碧空蛊毒,迟早要毒发!到那时,这片天地就会溃烂,化成一片青光,形成巨大漏洞。”

方源望着这片天地,青色在不断地加深,不断地往外蔓延。

如此漏洞,将加大地加剧福地的衰亡。也就意味着,对地灵的虚弱。

“换做是年轻点的福地,只要仙元充足,哪怕解不了毒,也能剔除这块病变之地,弥补漏洞。但这片福地,实在是太老迈了,仙元匮乏。此处一旦形成漏洞,必将是压死骆驼的最后一根稻草。唉,留给我炼蛊的时间不多了,我必须抓紧时间了。”

至此,方源的无情屠戮,终于告一段落。他回到大殿,争分夺秒,继续炼蛊!

时间一天天的过去,春秋蝉带来的压力与日俱增。

方源平均每天,只睡半个时辰,疯狂地压榨自己的潜力。

在他拼了命似的努力下,炼蛊过程中虽有多次必不可少的失败,但总体上进展相当迅速。令地灵也频频夸赞。

在方源向着成功迈步的同时,他的亲生弟弟也意气风发。

中洲,天梯山。

狐仙福地的中央,荡魂山上,方正努力攀爬着,超越一个又一个的身影。

在这段时间里,他成了风云儿,吸引了许多人的注意。

“是他,那个仙鹤门的方正!他又超过了一人!”有人嫉妒羡慕。

“方正……”望着方正不断攀升的背影,碧霞仙子目光复杂。

“这小子果然有古怪!看这样的趋势,恐怕他将是第一个登顶的!”对方正抱有敌意的魏无伤,此时也不得不承认,方正获胜的可能性很大。

方正手脚并用,努力攀等。

他大口喘息着,在天鹤上人的帮助下,超越了一个又一个的十派精英。

最终,在他的头顶,只剩下三人。

萧七星、应生机、凤金煌!

“这小子终于赶上第一梯队了,不枉我动用一次我素蛊。”鹤风扬一直保持关注,见此情形,暗暗松了一口气。

“再不出意外,此次的优胜者将会在这四人中产生了。”一位蛊仙传念道。

“仙鹤门的上升速度很快嘛,呵呵,不过鹿死谁手,不到最后时刻,还说不准呢。”

“的确,仙鹤门目前只是第四,和前三名差距明显。接下来就看他能不能,在有限的时间内赶超了。”

蛊仙们相互交流着。

整个传承的争夺,到了此时此刻,终于步入到最后阶段!

……

大殿中,一团玄光五颜六色,足有水缸大小,浮在空中狂飙乱转。

方源主持着玄光,双眼通红一片,充满了血丝,忽的开口吟道:“取柳石黄三两。”

顿时,大殿浮雕脱离青铜方砖,化为实体,正是柳石黄。

此乃上古石材,今昔难见,被地灵取来,又分出三两,主动飞入到炫彩玄光当中。

方源心神灌注,不敢有一丝马虎。待见玄光忽然变得通黄一片,又开口道:“取雪球蛊三十只。”

三十只雪球蛊,汇入黄色光团当中去。水缸大小的光团,仍旧黄蒙蒙一片,但是体积不断缩小。

最终,化为一点,形成一颗土黄色的石子,毫不起眼。

方源小心翼翼地接过这枚石子,长舒了一口气。此番炼蛊,一直持续了两天一夜,终于到此刻暂告一段落。

他累极了,立即躺在地上,沉沉睡去,恨不得直接睡上七天七夜方好。

但过了不到半个时辰,他就被地灵准时唤醒。

这石子不能长存,再过一刻钟,便会蒸发消散。到那时,方源前功尽弃,竹篮打水一场空,一切都得从头开始。

“炼蛊艰难啊,必须是力道蛊师,还得精通炼道。即便是我这样,拥有前世记忆和底蕴的,也感到艰难困苦,好几次差点功亏一篑,艰险无比。难怪前世,没有人能炼成第二空窍蛊了。”

方源心中感慨,拍拍自己的脸颊,让自己努力清醒过来。

休息了半个时辰,他感觉好多了,但是脑袋仍旧昏昏沉沉,这是消耗过多心神,太劳心劳累了。

第二空窍蛊可是高达六转,炼制仙蛊自然非同寻常。

方源此时不用照镜子,都知道自己形象必定头发散乱,脸色苍白,眼袋深黑,憔悴不堪。

“地灵,还有多少仙元?”他问道。

“还剩下五份仙元。”地灵立即答道,声音中透着一股虚弱。

此时距离方源斩杀龙青天,已经过去了十八天。

福地衰弱的速度,让方源也暗暗心惊不已。

为了支持三王传承的开启,仙元消耗甚大。这番异状,已经让外界沸沸扬扬。

这段时间,又有更多的蛊师赶来,进入福地探索,其中不乏成名高手。

“只剩下五份仙元,但是炼蛊才进行了三分之二。地灵,今天是几月几号?”方源面色凝重,又问道。

“按照你说的历法,已经是十月十九日。”

“十月十九日,按照我的计划,再过五天,就是炼制第二空窍蛊的最后一步。十月二十四号……咦?历史上的这一天,不就是那凤金煌得胜,继承狐仙福地的日子么。”方源忽然想到了凤金煌。

此女天赋卓绝,又背景深厚,继承了狐仙福地之后,更是一飞冲天。在日后,凭此修行成蛊仙,威仪笼罩四野,气度光耀八方。

方源前世逃离南疆,来到中洲后,成就了蛊仙,建立血翼魔教,此女便是死敌。大小战数百次,最终方源联合数位魔道蛊仙,一齐攻杀狐仙福地,付出了血的代价,才将其艰难打杀。

“目前我的影响,恐怕还波及不到中洲去。凤金煌得了狐仙福地,收益齐天!就算是第二空窍蛊,也比之不及啊。”

方源叹息一声,前世大敌要一飞冲天,自己却鞭长莫及,无法阻挡。

方源还不知道自己的亲弟弟还活着,并且此时正跟凤金煌等人激烈角逐。

他重生带来的影响,已经波及到了中洲十大门派。(未完待续)

------------

\end{this_body}

