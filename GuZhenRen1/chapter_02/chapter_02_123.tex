\newsection{竞争压力}    %第一百二十三节:竞争压力

\begin{this_body}

%1
商睚眦继续道:“我是输了,但是我还没有彻底失败。父亲将我流放到捕奴大队里去,三年!三年之后,我还会回来的。不过在此之前,我有一些东西要送给你。”

%2
说着,商睚眦递给商一帆几份文书。

%3
商一帆接过手,扫视一眼。文书上记载着商睚眦的秘密资产,布置的暗线棋子,还有一些得力的亲信人手。

%4
“睚眦哥哥,你这是……”商一帆做出惊讶的表情。

%5
“一帆老弟,你是所有兄弟姐妹当中,最有希望得到少主之位的人。老哥就在此助你一臂之力,希望你能够马到功成。这些物力人力,你尽管用着。三年之后,我回来时,你再交还给我就行了。”商睚眦叹着气道。

%6
他被贬到捕奴大队中历练,自然不可能带着下属、侍卫。他不是去享受的,而是去受罚的。

%7
如今他没有了少主的身份,这些势力早晚要分崩离析,还不如趁着还能掌控,先交给商一帆打理,让他代管着。三年后,商睚眦回到家族,还不至于重新开始,白手起家。

%8
商一帆连忙站起身来,抱拳动容道:“睚眦兄如此帮助,一帆必不敢忘。将来若有成为少主之日,必定十倍报还。”

%9
“哎,你我兄弟,何谈相报。呵呵呵……”嘴上虽这么说着,商睚眦的嘴角却咧开来,流露出浓郁笑意。

%10
一番热切的交谈之后,商一帆亲自将商睚眦送出大门。

%11
望着他离去的背影。商一帆脸上的笑容,渐渐转变成冷笑。

%12
“商睚眦。你打的好算盘,想让我替你保管势力?呵呵,那我就将计就计,先将你的势力接收过来,再分化吸收,有借无还。”

%13
他看穿了商睚眦的打算。

%14
“三年?三年能发生多少事情,你还真以为自己还有希望。哼,真是天真!这么天真。难怪会被方白二人算计。真是给我商家丢脸!”商一帆嗤笑连连。

%15
但想到方源和白凝冰,他的脸色有阴沉下来。

%16
“商心慈……”他口中咀嚼着这个名字。

%17
商睚眦下台,他原以为这个少主之位手到擒来,但没想到半路杀出个程咬金,商心慈异军突起,成了他最大的阻碍。

%18
商一帆双眼眯成一条缝,精光烁烁。自言自语地分析:“商心慈最大的优势,就在于父亲的宠溺。这等偏爱,其他哪个子女有过?但家族规矩立在那里,众目睽睽竞争少主,父亲也不能公然给她开小灶。”

%19
“反而,因为父亲大人的偏爱。导致其他兄弟姐妹对商心慈都有意见,或者敌视。商心慈最大的优势,反而是她的劣势。”

%20
“商心慈第二优势,才是她实际上的最大优势,那就是方正和白凝冰!这两人真是……”

%21
想到方白二人。商一帆的脸上也涌现出古怪、羡慕、嫉妒、无法理解等复杂的神情。

%22
“这两个家伙,不晓得他们想的什么东西!居然连外姓家老都不要做。赶去帮助商心慈去!”

%23
这感觉,就好像是有人放弃西瓜不去吃,改去捡芝麻。

%24
“商心慈是怎么笼络的这两人?运气真好啊,轻而易举,就收下了这两员大将。那白凝冰已经是四转修为,方正甚至能战胜巨开碑!”

%25
这就是两个四转战力。

%26
就算是大哥商囚牛,也没有四转的麾下。

%27
在十大少主之上,商家目前的少族长商拓海,也只是有两个四转强者的属下。

%28
但这两个属下,也不属于商拓海的私军,而是家族专门调派给他的帮手。

%29
商拓海作为家族少主,亲自掌管着一只商队。在外打拼,的确需要强大的助手来应付各种情形。

%30
现在好了,商心慈还未当上商家的少主,就有两个四转强者相助。

%31
这个情形,不知道让多少商燕飞的子女眼红、忌惮。

%32
“不过,就算她有方白二人相助,又能怎样?这个少主之位,一定会是我的!”商一帆舔了舔嘴唇,想到了什么,精神一振。

%33
……

%34
与此同时,在楠秋苑。

%35
“我们商家有着繁复全面的家族规矩,尤其是关于继承人方面,规矩更是森严。”魏央站在方白二人,还有商心慈的面前,侃侃而谈。

%36
“心慈小姐你要登上少主之位,就得通过商家的考核。这个考核,相当传统,历来都是一个内容,那就是经商。”

%37
商家以商立家,商家发展,离不开经商贸易。商家选拔少主,就是以经商为内容来考核。

%38
“不要小看经商,认为这只是赚钱的买卖。经商可以考研一个继承者的方方面面,经商的过程中会碰到各种各样的问题。考验一个人计划谋算,灵活应变,实力修为等等。”

%39
“家族方面,会给任何一个参加竞争的子女,调拨一笔十万元石的款子。三个月后,选取赚取钱财最多的那位,成为新任的少主。”

%40
魏央对家族政策十分了解。

%41
“那么,大约我们需要赚到多少元石,才能够赢得这场竞赛?”方源问道。

%42
商家的少主之争,全城都瞩目,整个商家都在重视。方源有自知之明,打算遵守这里的游戏规则。

%43
“一般而言,只要心慈小姐最终手中有三十万元石,就能淘汰掉大部分的人。如果有六十万,就有很大的竞争力了。六十万以上,七十万到八十万之间,这种成绩在历代的少主选拔中,都是第一流的成绩。不过……”说到这里,魏央语气一缓。

%44
“这次的竞争对手当中,有一位对手,母亲就是族长大人的表妹,是商家中人,势力很大。他叫做商一帆,是此次少主之争中最大的热门。有他的母亲在他的背后帮助他,他至少会有六十万的成绩。心慈小姐要胜过他,就要做到更优秀的成绩。”

%45
商一帆至少会有六十万的成绩,商心慈要得到少主之位,就要做到第一流。

%46
但商心慈在商家,无权无势,母亲更是张家人,在政治上还有劣势。不可能像商一帆这样,背后有人撑腰。

%47
唯一的靠山商燕飞,却正因为是族长的缘故,更不能公然偏袒她。

%48
因此,现在落到商心慈双肩的压力很大。

%49
魏央说完,用暗藏担忧的目光,注视着商心慈。他知道,七十万到八十万之间的成绩,都是可遇而不可求的。历代很少有少主,能达到这样的成绩,因为这需要才华,更需要帮手,还需要运气。

%50
本来商燕飞也没有打算,让商心慈这么快就竞争少主之位。

%51
一切都是方白二人,在背后推动。

%52
商心慈听完魏央的介绍后,沉默了一下,忽然问道:“不知道,商家历史上最好的成绩,是有多少?”

%53
魏央微微一愣:“当然是超越了八十万,达到九十几万。能做到这种程度的,在商家历史上,少之又少。满打满算,也不超过二十个。当今的少族长商拓海,达到八十九万,引发轰动。不过小姐你的父亲,却在其时,达到过九十二万。单这项成就,就足以载入商家史册。”

%54
说着,魏央语气一顿:“不过,商家历史上最好的成绩,是一百一十一万。他叫做商鬼才,有妖孽般的天资,可惜是个十绝体……”

%55
“九十二万,一百一十一万……”商心慈听得双目炯炯发亮,微微握紧秀气的拳头。

%56
她在经商上,可谓才华横溢。对于此次竞争,她虽然是被方源鼓动,本身也有意愿,并不是被动的。

%57
看到商心慈这番神情,魏央也微微放下了心:“好了,我也该走了。不过临走前,我还有特别关照你们。此番竞争,受到商家上下的重视。千万不要铤而走险,做出一些违规的举止。相信你魏大哥吧,历来在竞争中作弊的,都没有侥幸成功过。”

%58
少主乃是商家的未来,事关重大,不容许有一丝马虎。

%59
商睚眦还是少主呢,因为一次假账,就被驱逐流放。从中便可看出商家对少主之事的极端重视。

%60
商一帆有母族势力在背后撑腰,顶多也只能在规矩之中帮衬他,不敢为其作弊。

%61
随后,魏央又不放心的交代几句,便离开了楠秋苑。

%62
他是家族重臣,商燕飞的心腹,按理来讲,应该避嫌。但他仍旧出入楠秋苑,为商心慈讲解这么多,足可见他的情义。

%63
“魏大哥为了我们,也扛着许多压力。接下来,我们不能再麻烦他。”商心慈道。

%64
方源不置可否。

%65
白凝冰则皱眉道:“要在三个月内经商,将十万元石翻六倍,谈何容易?”

%66
她虽是天资卓越,但是对于经商,一窍不通,现在感到十分为难。

%67
哪知商心慈美眸一转,露出微笑,自信地道:“其实,我已经有了一个好想法。只要按照这个想法实施,保管能翻出六十万来。”

%68
“哦,是什么想法?”白凝冰便问。

%69
商心慈深深看了一眼方源,和盘托出:“我也是因为去演武场,看黑土哥哥的比斗时,泛起的灵感。我们可以做情报生意。”

%70
“情报生意?”白凝冰皱眉。(未完待续。如果您喜欢这部作品,欢迎您来起点投推荐票、月票,您的支持,就是我最大的动力。)

\end{this_body}

