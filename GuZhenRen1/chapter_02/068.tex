\newsection{大大低估了方源的无耻}    %第六十八节:大大低估了方源的无耻

\begin{this_body}

“古月方正?”听到魏央的汇报后,商燕飞微微地皱起眉头。

古月这个姓很特别,商燕飞搜索记忆,很快就找到出处。

“古月一族,青茅山上的三家之一,好像有几百年的历史了。青茅山……”商燕飞双眼微微一亮,他想到了大半年前的一个情报。

青茅山遭受到神秘打击,在一夜之间,从绿水青山变成冰天雪地的绝域。

没有人知道,在青茅山上到底发生了什么。

至今覆盖整个青茅山的冰雪还未完全融化。

但人们从种种痕迹中,看到蛊师间发生大战的痕迹。

如今青茅山的覆灭,成了一个神秘事件,已经在南疆广为流传,对于造成这一切的原因更是众说纷纭。

但是对于某些人来讲,青茅山上的痕迹,就太明显了。

十绝体的秘密,在蛊师界的高层,已经算是半公开的信息。

商燕飞在得知这个情报的第一时间,就猜到,这一切很有可能是十绝体之一的北冥冰魄体造成的。

“如果是这样,那古月方正的来历就能解释了。墓碑山的调查队伍,也传回了书信,没有任何疑点。”

商燕飞在心中迅速勾勒出整个事件的大概轮廓。

“那么接下来,只剩下一个疑问。他们为什么要隐姓埋名,躲在商队当中呢?”

商燕飞已经猜出许多可能,但这事情还正在调查当中。

“对了,你和他们切磋了一番,他们的身手如何?”商燕飞问道。

魏央面容一肃,恭声地禀告道:“天纵之资!假以时日,此二人必能超越我。”

“哦,竟然能得到你这么高的评价?”商燕飞小小的惊异了一下。

魏央点点头,继续道:“他们两人虽然用蛊虫遮盖了气息,但是在战斗中能坚持那么久,至少应该都是三转修为。再凭他们的年纪就可推测,这两人应该都是甲等资质。”

商燕飞笑道:“可是魏央你要知道,甲等资质,只是一种天资,并不代表一切。魏央你只有乙等资质,却是我的外姓家老。这些年,死在你手中的甲等有多少?呵呵,你就是最好的明证。”

寻常山寨中,出个甲等资质,就是件了不起的事情。

但对于商家而言,甲等资质却很常见。

一来是因为家大业大,族人众多。二来,可以招揽魔道蛊师,能从演武场脱颖而出的魔道蛊师,资质必定不俗。三来,是商家财富如山,完全有能力购买到改变资质的蛊。

“承蒙族长大人夸奖,属下愧不敢当。”魏央谦虚了一声,继续道,“属下亦知道这个道理。因此更加确信,这两人绝非池中之物。”

“他们二人深富战斗才情,对战斗局势有一种天生的敏锐,简直就像是为战斗而生。他们的蛊虫残缺不全,但偏偏凭借着那两三只蛊虫,坚持许久功夫,实在叫人意外。”

“就心性而言,这两人身具韧性,深陷劣势,却毫不气馁。在我施加的压力下,他们能临阵突破,不断的调整,配合更加默契,阵脚越来越稳,进步极其明显。”

“具体来说,两人又有区别。古月方正生性直爽,勇气绝伦,横冲直撞,天生有一股气概。而那白凝冰,则思维敏捷,凡事谋定而后动,战斗中一直在试图寻找我的破绽,目光犀利。尤其是古月方正,在我坦言拥有光源蛊后,他主动认输,十分坦然,心胸之开阔绝非常人。”

“族长大人,这两人实乃是平阳岗的幼虎,浅沙滩的雏龙。一阳一阴,一刚猛霸道,一阴柔谋算,珠联璧合,双星闪耀,交相辉映。若能招揽之,为商家效力,大善!”

商燕飞不禁动容。

先前他还没有把方白二人看在眼里,但听魏央如此说,不由地兴趣大增。

“魏央你跟随我这么多年,你的目光我是信任的。不过你无需妄自菲薄,就算是将来他们成长起来,也未必及得上你。你受制于资质不足,否则以你的才华,会比现在更加卓越。你无须挂怀,接下来若有脱胎蛊,我会帮你留下来。”商燕飞道。

脱胎骨,能令蛊师资质上涨,珍稀无比,价值巨大。

魏央顿时感动得双眼泛红:“族长栽培之恩,属下没齿难忘!”

“嗯,我商燕飞是绝不会亏待任何一个忠心跟随我的人。你下去吧,这些天继续招待他们,探探他们对我商家的想法,看看能否招揽之。”

“是,属下领命!”

……

转眼三天之后,楠秋苑。

会客厅中,方源和商睚眦面对面的坐着。

方源悠然地品着茶水,而商睚眦的脸色却不好看。

“这几天来,我满怀诚意地来找阁下。但阁下的开价,竟然一次比一次高。原先不过是六十五元石,几乎每天涨数万,到今天,阁下居然开价八十万!阁下是否在戏耍我,当做消遣?”商睚眦胸中憋闷无比,咬牙切齿地道。

换做先前,他早就发火,把手中的杯盏狠狠地摔在方源的脸上。

但是现在不行。

为什么?

因为这家伙能上达天听!

不晓得他们究竟和父亲大人是什么关系!

这些天来,魏央一直招待着这两人,陪同他们逛街游城。

魏央是什么人?他号称第三干将,乃是父亲大人的心腹!

他的行为,更多时候,代表的不是他本身,而是商燕飞的倾向和意志!

但这不知道从哪里冒出来的两个家伙,究竟怎么会让父亲大人如此礼遇呢?

商睚眦百思不得其解。

事发之后,他就疯狂的展开调查。

但没用。

他只是区区一个少主,势力局限于商家城。远没有商燕飞那般巨大的能量。

查不到结果,商睚眦就只好猜测。

父亲大人究竟为什么要示好?

是因为他们手中的传承吗?不,一道传承可能振兴一个普通的家族,但是商家却不同。除非是六转蛊仙的传承,否则都是锦上添花罢了。

还是看重他们两人是人才?也不对,演武场那么多忠心耿耿的魔道蛊师,巴巴地想要依附商家,而且又都很能打。那些人父亲都看不上,还能看中他们?反正商睚眦是看不出来什么。

两种猜测被排除之后,一个念头不由自主地冒出来。

难道在他们两个当中,有人是父亲的私生子?

那个商拓海,如今的少族长,不就是父亲的私生子吗?

当商睚眦再一琢磨,又觉得不对。

亲生血脉的意义重大非凡,商拓海一被发现,就被牢牢地保护起来。哪像眼前这两人?

商睚眦苦苦思索,毫无进展。

正因为如此,他对方白二人更加忌惮。往往未知是最恐怖的。

方源察觉到,这些天来眼前的这个商睚眦少主,已经越来越焦躁不安,越来越不耐烦。

这正是他想看到的。

每次加价,都是他故意为之。

如果他一次性从六十五万加到八十万,那肯定谈崩掉。但一次次增加数万,反而能磨掉了商睚眦坚决的反对之心。

时机成熟了。

方源放下手中的茶杯,微笑道:“一个货物都是有其价值的。对于其他人来讲,这只是纯粹的蛊师传承罢了。但是对于阁下,却是保住少主之位的最后希望。”

“既然如此,那价格就应该高一些。随着评定考核的日期越来越接近,这道传承的价值就越来越大。因此我每隔一日就加价一次,难道不是件理所应当的事情么?”

“呵呵,如今开价低了,怎么能对得起这个最后的希望?怎么能对得起商家少主这样的重要权位?要知道只有商家少主,才能竞逐少族长之位呢。”

商睚眦听了这番话,眼角气得直抖。

方源这是在要挟,这是在坐地起价!

做人怎么能这么无耻?

商睚眦恨不得把方源大卸八块。但他想到少主之位,终于还是硬生生地忍耐下来:“你的打的好算盘。我要是花八十万买下来,这就是一笔亏本买卖,反而会降低我的评价。负责考评的那些家老,不是傻蛋!所以,这价格我根本不可能买的。”

方源早料到商睚眦会如此说,他的嘴角微微翘起,浮现出一抹笑意:“所以我有一个很好的办法。明面上我卖给你六十五万,但实际上你给我八十万。你保住了你的少主之位,我也卖出了一个理想的价钱,不是皆大欢喜么?”

商睚眦顿时变色,瞪圆了双眼看向方源:“你这是要我做假账?这绝对不成!要是被发现,不管什么时候,我都会立即被撤销少主之位,并且会受到相当严重的惩罚。”

方源轻佻眉头:“话不能这么说。谁说是假账?我卖给你秘方,这是一码事。而你乐善好施,觉得我为人正直,送给我一些元石当做礼物。这两件事情毫无关联嘛!”

一时间,商睚眦只能愣愣地看向方源,不知说什么好。

他原先觉得方源这人无耻,现在他发现,原来之前他大大的低谷了方源的无耻程度!(未完待续。请搜索[.138.看.书.],小说更好更新更快!)

\end{this_body}

