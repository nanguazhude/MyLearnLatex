\newsection{演得好一场戏}    %第十九节:演得好一场戏

\begin{this_body}

第三天,百家族长再招来方源。以狩猎的名义,送了他许多元石。

晚宴时,方源除了偷瞄百莲之外,又主动向百家族长敬酒,神情诚恳,带着感激之色。

百家族长嘴上不说,心中却很是满意。

这天晚上,百莲主动拜访方源,言及自身一友,受了毒伤,唯有清热蛊治疗,效果最佳。但清热蛊较为稀少,因此,向方源借蛊。

“终于忍不住了么?”方源心中冷笑着,二话不说将清热蛊借给了她。

此事之后,百莲似大为感激。

方源又主动攀谈,热情洋溢,两人越走越近,关系急剧升温。

到了第五天的晚宴。

百家族长忽然直接问道:“贤侄,你们古月一族是否想在白骨山安家立寨?”

方源连忙站起来,神情强自镇定,但掩盖不住眼里的慌乱:“白骨山与百家寨相邻,可以说是贵族的地盘。晚辈岂敢冒犯。”

百家族长心中笑了,越加肯定古月一族的目的地,就是白骨山。

她又虚情假意地道:“贤侄有所不知,此地局面复杂。我百家压力甚大,若是古月一族能在白骨山栖息,作为我族的盟友,那就再好不过了。”

方源连忙否认。

百陌行也在劝说,方源似有意动,但嘴上却不松口。

晚宴结束之后,百莲也来旁敲侧击,方源神情复杂。仍旧没有承认。

“哼,这个小子,嘴巴倒牢靠得很。”晚宴后营帐密谋,百陌行咬牙叹息。

“这才是一个家族的少主,应该有的表现。我并不意外。应该再添上一把火了。”百家族长目光悠悠。

于是到了次日清晨。

方源还未起床,就被营帐外的争吵声吵醒。

他走出营帐一看,只见百战猎和百莲正拉拉扯扯。

“百战猎,我跟你说过多少次了,不要来纠缠我了。感情的事情是不能强求的!”百莲甩脱百战猎的手,冷若冰霜。“我还有事,你先走吧。”

“今天的狩猎大比,就要开始了,你还有什么事情?又是去找古月家的那个小白脸了对吗?”百战猎愤怒地低吼道。

“你说什么话呢!方正少爷人很善良,清热蛊说借就借了。要不是他帮忙,百盛景能有这么快康复么?”

“莲儿你能不能不要这么单纯。清热蛊,呵呵,我看他是想和你亲热吧。你难道没有察觉到他看你的眼神吗?”百战猎急道。

百莲瞪眼:“百战猎,你给我适可而止吧!啊。方正少爷……”

争吵的两人,发现了营帐外站着的方源。

方源的脸色有些尴尬。目光又有些担忧。他对百莲道:“原来是百莲姑娘。有什么事情,先进帐说吧。”。

“你小子!”百战猎勃然大怒,想要找方源麻烦,但半途中就被百莲阻拦住。

“百战猎,你想干什么?你疯了吗?这可是我族的贵客!”

“什么贵客,不过是丧家之犬罢了。”百战猎呸了一声,手指向方源的鼻子,“小子,有种的我们比试一番。来一场真刀实枪的较量!谁若输了,就不得再纠缠莲儿。”

“哼,我是一转,你是三转,你也好意思说这种话。堂堂百家,难道没有公平吗?”方源脸色变得难看。

“这个世界上,从来就没有公平。只有强弱。你不敢比试,就是孬种!原来古月家盛产孬种啊,啊哈哈哈……”百战猎仰头大笑,声音吸引了周围许多人的关注。

“怎么回事?”百陌行这时赶了过来。

百莲一番说明。百陌行立即痛斥百战猎:“你真是胡闹,竟敢对贵客无礼!”

百战猎昂着头:“他连应战都不敢,不是勇士。既然不是勇士,何须礼待他?”

“你!”百陌行怒瞪。

百莲则道:“你这种邀斗,谁答应了谁就是傻子。方正公子是因为伤势,才导致修为下降。若在他强盛期,你未必是他的对手。”

一位妙龄少女如此向着方源说话,若真是方正在此,恐怕已经生出无比的感激之情。

但方源却是心中冷笑:演得好一场戏!

“我代替方正少爷,和你比就是了。”百莲接着道。

百战猎气得直喘粗气:“你为什么要替他出头,再说,你凭什么能代表他?就他一个小白脸,窝囊废物,我一个能胜他十个!我不和你比,小子,你有种的就站出来。一声不吭,还是个男人吗?”

“你要比就比,谁怕谁!”方源一副受不住激将,梗着脖子,以冲动的语气道。

“家老大人,你听到了吗?他答应了!”百战猎立即高兴地嚷嚷起来。

百陌行皱起眉头:“人生应该勇于接收挑战。方正贤侄,你的勇气我们有目共睹。但你是我族的贵客,若有个三长两短,我们难以向古月一族交代啊。而且双方修为不一,挑战起来有失公允。”

“家老大人说的是,晚辈考虑不周了……”方源故意迟疑道。

眼看着方源有向后退缩的趋势,白战猎和百莲迅速交换了一下眼神。

百战猎接着开口,挑动方源的火气,激将他。

百莲咬着牙,走到方源的面前,睁着一双水汪汪的大眼睛,以轻柔的声音道:“方正公子,在下有个不情之请。”

“哦,你说什么事情?”

“恳请公子答应了这次挑战,帮我摆脱百战猎的纠缠。我真的受不了他的骚扰了。”说着,百莲泫然欲泣。

一个少女,哀求般地向一位男生求助,让他赶跑另一个很可恶的追求者。

尤其是这位男生,也同时对少女有好感。

你说,这位男生能不答应吗?

于是,方源当即拍着胸脯,一口答应下来:“百莲姑娘切勿忧愁,你的事情就是我的事情。我定当鼎力相助。”

顿了一顿,方源语气又有些迟疑,“只是我的修为,暂时要比他弱小。万一输了……”

“公子放心,小女子早有定计。”百莲一笑,如水仙花开。

她转过身去,对百陌行道:“家老大人,方正公子虽然答应了比试,但若真的进行决斗,恐怕要伤了和气,而且也不公平。我有一个提议,不如就借这次狩猎大比的机会,双方分五人小队,比拼此次狩猎的战绩。”

“嗯,不错的提议。”百陌行摸了摸胡须,微微点头,“那你们就挑选人手罢。但为了公平,五人的修为必须相当。”

百战猎不悦地冷哼一声。

“是。”百莲则连忙高兴地行了一礼。

……

半个时辰之后,两方人马出发。

方源这边,有三转的白凝冰、百莲,以及另外两位女蛊师,年龄和百莲相仿,均有二转修为。

其中一位,名为百盛景,对方源十分感激。

先前正是她中的毒,被方源的清热蛊解救。

队伍其热融融,方源虽然修为最低,但却是中心焦点。

“公子勿忧,我们在之前就已经打听到详细的情报,知道哪里有更珍贵的猎物。跟我们走,就对了。”百盛景性格较为活泼,负责侦察。

众人跟着她走,果然接下来,打杀了不少独特的猎物。

方源出力不多,整个过程更像是郊游。

“方正公子,听他们说你们古月一族,要迁徙到白骨山上来。是这样子的吗?”满载而归的途中,百盛景似乎是随口问道。

“这些都是空穴来风罢了。”方源笑了笑道。

“公子你温文尔雅,是正人君子,可比那个什么百战猎好多了。唉,如果将来能定居在白骨山,我们以后就能常常见面啦。”百盛景继续道。

方源呵呵笑了一声,又偷瞄了一下身边并肩而行的百莲。

百莲一脸忧容:“要安家立寨,绝不是简单的事情。首要的条件,就是寻找到一口元泉。然而元泉周围,元气浓郁,必然生活着兽群或者强大的野生蛊虫。先贤创立山寨,定要经过激战,剿杀掉兽群或者野蛊。这个过程,必然伴随着流血和牺牲。”

说着,她看向方源,“其实白骨山上,生活着大量的骨兽。这些骨兽体制强硬,很难对付。白骨山上,也没有泥土,都是皑皑骨石。要想在这个山上生存立寨,不是不可能,只是付出的代价,会很高昂。”

“哦,是这样吗?”方源笑容显出一丝勉强,目光中流露出一丝忧心忡忡的意味。

然而他装作不经意地问道:“我对白骨山比较有兴趣。百莲,你是半个地主,跟我讲讲,这座山上到底有多少的危险。”

百莲笑了笑:“公子要听,那我就说说。”

她一边夸大其词,一边暗暗催动空窍中的积虑蛊。

此蛊效用如春风细雨,悄无声息,默默地影响身边十步以内的范围。能令人心思加重,忧虑更深。

方源的笑容变得渐渐地少了,眼中的忧愁溢于言表。

“方正公子不必担心,这次我们定能胜过百战猎。”百盛景故意开解道。

方源点点头,回答了几句,却显得心不在焉。

接下来,他问的问题,更多了。

都是关于白骨山,尤其是后山的某一段。

百莲都给予了耐心的解答。

这一幕的情形,在营帐中如实同步地显现着。

“小鱼上钩啦。”百家女族长得意地笑起来,“后山……”

她开始对比桌上的地图。

\end{this_body}

