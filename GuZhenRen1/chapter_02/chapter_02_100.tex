\newsection{闭嘴,你这头肥猪!}    %第一百节:闭嘴,你这头肥猪!

\begin{this_body}



%1
“这一场战斗结束,九成的真元消耗掉了七成,还只剩下两成……”收获了一场胜利后,白凝冰离开了演武场。

%2
她一边走着,一边进行总结。

%3
每场战斗之后,她都会进行总结。今天和铁刀苦一战,虽然没有杀掉他甚是可惜,但是白凝冰的收获也很大。

%4
“刚刚那场战斗,虽然激烈,但却短暂。我却消耗了这么多的真元,想想看我还是甲等资质的真元恢复力。战斗激烈的时候,我身上的这套蛊虫的真元消耗就显得多了。”

%5
很多蛊师组合蛊虫,都难在此处。

%6
大威力的蛊虫,谁不喜欢?但往往蛊虫效能越大,消耗的真元就越多。

%7
威力大,消耗真元少的当然也有。那都是珍稀蛊虫,价格就昂贵了。

%8
白凝冰手头上虽然还有余钱,但是已经不那么宽裕了。她是聪明人,很快就意识到自身的处境,主动开始节约。

%9
以前,她花钱大手大脚,对元石价值没有清晰的概念。那都是家族在全力培养她。

%10
现在,她自己当家掌钱,一段时间下来,她已经完成了意识上的重要转变。

%11
方源在提升,白凝冰也在飞速的进步。

%12
“如果我有方源的天元宝莲,那就好了。至少在三转,不要动心真元消耗的问题。”白凝冰想到这里,心中不免涌起一阵羡慕嫉妒的情绪。

%13
方源那家伙,走的是力道。力道蛊虫对真元的消耗最少,以他九成甲等资质,真元很充沛。天元宝莲放在他那里,根本展现不出真正的价值来,简直是明珠暗投!

%14
“若是能从方源手中。收购天元宝莲呢?”

%15
白凝冰摇摇头,很快就将这个不切实际的念头,从脑袋里驱散出去。

%16
方源是什么人,她太清楚不过了。

%17
从未见他吃亏过,若真有一天,从他手中取走天元宝莲,那自己必定要付出更严重的代价。

%18
“没有天元宝莲辅助,那我就只好削减手上的蛊虫了。”

%19
仔细思索了片刻后,白凝冰决定要舍弃雪球蛊。

%20
雪球蛊。是三转蛊里面中高端的,原先她打算用作远程攻击手段。

%21
但这蛊实在有些消耗真元。

%22
一发雪球,还好一些。但一场激烈的战斗中,怎么可能只发一记雪球呢?

%23
五六发雪球之后,真元的消耗量就大了。

%24
“舍弃雪球蛊。保留冰锥蛊。冰锥蛊虽然只是二转,但也能起到一定的牵制作用。若是配合冰爆蛊,也能造成威胁。”

%25
冰爆蛊是三转蛊,能引爆冰块,形成瞬间的强大杀伤力。

%26
冰块越大越久越寒冷,爆炸的威力就越大。

%27
刚刚和铁刀苦的战斗中,就是冰爆蛊建功。

%28
当然冰块引爆之后。白凝冰若被波及,自身也会受到伤害。但若是化身冰晶,伤害就下降一个档次。

%29
一想到刚刚的精彩战斗,白凝冰舔了舔嘴唇。感到有些兴奋。

%30
一般的冰道蛊师,追求防御,或者是困敌的能力。只有火道蛊师、雷道蛊师等追求爆炸力量。

%31
但白凝冰偏偏反其道而行之。冰爆蛊是她的一个灵感,没想到效果之好。超乎原先的预估。

%32
“冰爆蛊在战斗中的效果很好,很精彩。以后要继续发扬。”白凝冰暗暗打算着。

%33
她原先崇尚刚正狂猛的进攻,凌厉的冰风切碎一切。但是和方源朝夕相处之后,她耳濡目染,慢慢的对一些的阴险的招数感兴趣了。

%34
她对世间的一切精彩,都很感兴趣。

%35
“冰晶蛊、冰刃蛊、冰锥蛊、冰爆蛊……这样的组合,能否击败方源?”

%36
白凝冰心中都是将方源当做第一假想敌。

%37
“方源走力道,有全力以赴蛊,近战方面他比刚刚那个家伙还要威猛霸道,以我现在的手段,要胜他并不容易。除非我能重现我的杀招。”

%38
白凝冰的杀招,乃是她自创的冰刃风暴。原先是用旋踵蛊,搭配旋风蛊、冰刀蛊,形成冰风龙卷。

%39
但现在白凝冰眼光开阔了,这种搭配已经过时,冰风龙卷欺负二转的倒也不差,放到三转这个层次,就有些勉强了。

%40
这些天,白凝冰一直琢磨着如何改良自己的杀招,但一直没有头绪。

%41
她忽然想到一件事情,步伐加快:“对了,今天是方源升上第四内城的第一战。现在去看的话,应该还来得及!”

%42
……

%43
“杀,杀了他!”

%44
“朱八,快动起来,还不把这小子撞死!”

%45
“依我看,朱八不用自己动手,方正那小子就要承受不住了。”

%46
“嘿嘿,这小子麻烦了。一上来就碰到朱八,正克制他的力道打法。”

%47
演武场边,议论纷纷,人头攒动。

%48
“看来方源的这场战斗,吸引了不少的观战者。也是,演武区并不禁观战者的修为。只要蛊师有钱,就能到这里观战。”白凝冰来到演武场边,对周围扫了一眼,旋即将目光投到场上。

%49
这是个中型演武场,地形是普通的石砖。

%50
方源正和一位体型庞大的蛊师,激战在一块。按照旁观者的议论,方源的对手,应该就是那个“朱八”。

%51
朱八体型庞大,他身高足有三米,浑身长满了肥膘,显得脑袋、手脚都仿佛变小了一般。就好像是地球上的相扑手,被放大了好几倍。

%52
这个世界上,不管是蛊师,还是凡人,身高一般都不会超过两米。朱八的这种身高,显然是动用了蛊虫的力量,才产生的变化。

%53
事实上,这很正常。

%54
很多飞行流派的蛊师,会将自身的骨骼改成中空,降低重量。一些地遁蛊师,则用缩骨蛊,将自己变成一个侏儒。这样一来。在钻地洞的时候,消耗就少很多了。

%55
白凝冰对朱八并不了解,但是听着身边人断断续续的讨论声,也明白了朱八的打法。

%56
朱八是擅长防御的蛊师。

%57
他浑身的肥膘,给他提供了相当棒的保护。

%58
方源围绕着朱八,拳脚如飞,打在他的身上。朱八的身躯受到重击,肥膘就宛若水流般晃荡,将方源的力量分担到全身各处。然后又利用蛊虫。将这股力量聚集起来,通过反震的方式,“回赠”给方源。

%59
朱八身上的蛊,据说是得自一个魔道传承。究竟有多少蛊,是什么名字。他一直都没有曝光。

%60
但他的这种打法,叫人无奈,尤其克制力道蛊师。

%61
“方正,你歇歇力气吧,没有用的。你每一拳的力量,我都能反震给你八成。一顿拳脚打下来,我没什么事情。你反而受伤。你现在还没有感觉到吗?”朱八瓮声瓮气地说着。

%62
他直接坐在演武场上,任凭方源围着他进攻,有一种岿然不动的气度。

%63
山猪虚影、棕熊虚影、鳄鱼虚影,不断地在方源的头顶上闪现。这一幕。好像就是他暴打背山蛤蟆时的景象重现,但是朱八却无动于衷。

%64
你奋尽全力,对方却像个没事人一眼,这个情形如何不叫人气馁?

%65
但方源锲而不舍。一声不吭,围着朱八暴打。

%66
从战斗刚开始。就这样子。如今已经持续了一炷香的时间了。

%67
“呼!”方源吐出一口浊气,忽然收起攻势,后退一步。

%68
他的嘴角溢出一丝鲜血,来自朱八的反震之力,震荡他的肺腑五脏,使得他受了内伤。

%69
他有铁骨,有钢筋,又有天蓬蛊,还藏着金罡蛊,但是这震荡之力,却透过这些防护,专攻他的内脏。

%70
他的内脏并没有强化,哪怕一直用自力更生蛊恢复着,伤势也越来越重。

%71
方源停下攻势,反震之力也就消失,几个呼吸之后,他的内脏伤势就痊愈了。

%72
继续进攻!

%73
他猛跨一步,再次扑上,双拳狠砸如流星,拳拳打出呼啸的风声。朱八满身肥膘再次晃动起来,一拳拳地承受着,又一波波地反震回去。

%74
“方正,你有治疗蛊,我也有治疗蛊。你受到八成的攻击,我只受两成。你是耗不过我的,不如认输吧。”朱八继续劝说道。

%75
方源只是强攻不语。

%76
轮到嘴角溢血的时候,他就停下手中攻势,用自力更生蛊恢复治疗。

%77
他虽然消去了一猪之力,但是却又添上了牛力、马力和半个龟力,自力更生蛊的效果已经赶超了肉白骨。

%78
这样的情况,不断地持续下去,观战者都有些不愿意了,有些甚至打起了哈欠。

%79
“太无聊了……”

%80
“真不知道方正打得什么劲。”

%81
“今天的朱八也有些反常啊,怎么一直都没有主动进攻?”

%82
“这不是他一贯的风格吗?先防守,耗费对手的真元和战斗激情,然后在一举出动,将对手揍趴下。”

%83
“问题就出在这里呀,方正这小子太激情了,打了半天没有起色,一点都看不出气馁的样子。他是力修,消耗的真元也少。这样僵持下去,何时能到头啊。”

%84
“这个情形,你们还看不出吗?朱八是在调戏新人呢。”

%85
这话引得周围人一阵嬉笑。

%86
方源在这些常年厮混在演武区的蛊师来讲,的确是嫩得不能再嫩的新人了。

%87
“方正,你这样打下去,真的没什么意思。我对你的全力以赴蛊没有丝毫兴趣,作为前辈,不想欺负你这个新人。你还是主动退下罢,我答应你不选你的蛊虫就是。”朱八呵呵地笑着。

%88
他承受着方源的攻击,每一拳都有山猪冲撞、鳄鱼撕咬、棕熊拍击的力量,但他依然能笑出身来。

%89
“这是怎样的实力!”

%90
“不愧是我们第四内城的八号人物啊。”

%91
“这气场不是盖的。”

%92
观战者们尽皆动容。

%93
“闭嘴,你这头肥猪!”方源陡然爆喝一声。

%94
全场一静。

\end{this_body}

