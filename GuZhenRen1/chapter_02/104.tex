\newsection{真想扼杀掉}    %第一百零四节:真想扼杀掉

\begin{this_body}

方源不禁又想:“依照铁家的反应,铁若男应该还不知道我手中有血颅蛊。也许我现在的甲等资质,会是我最好的伪装。”

这点,方源猜得八九不离十。

事实上,铁若男根本就没有怀疑过方源的身份。

原因,也出在资质方面。

铁若男一直认为,方源是古月阴荒体。十绝体的弊端,她在铁家也了解了许多。如今这么长时间过去了,若真是方源,早就自绝于天地了。

而且,从铁刀苦的情报上,方源勇猛无畏的战斗风格,也算是一个佐证。

在铁若男的印象中,方正就是这样直来直往的性格。而方源那样阴险的人,怎么可能打出如此狂猛的攻势?

铁若男的第四次拜访,终于见到了方源。

会客的厅堂中,只有方源和铁若男两人。其余人皆被安置下去。

“想不到你我再见面时,却是这番景象。”铁若男不胜唏嘘。

方源和方正是胞兄弟,相貌酷似。

她唏嘘,方源比她更唏嘘,深深叹息道:“过去的事情,就让它过去吧。我不想再提了。”

铁若男眼中锐芒一闪:“不,有些事情是不能忘的。我这次来找你,就是为了当年的事情。你一定知道,我的父亲铁血冷是怎么死的。请你如实相告!”

方源深深地看向铁若男,后者毫不畏惧,和他对视,目光中流露出坚定的决心。

她剑眉星目,这些年许是不在外闯荡了,微黑的皮肤已经转白,配上高挺的鼻梁和朱唇,显露出巾帼的飒爽英姿。

毫无疑问,她是一个美人。虽然容貌上不及白凝冰、商心慈,但身材健美,尤其是一双长腿。再配上她独特的气质。让她出类拔萃,更能勾起男人心中的征服欲望。

但方源不关心她的美貌,从她的目光中,方源察觉到她对自己的身份,似乎毫无怀疑。

这个情况很好。

那么接下来。就是如何解释当年的事情。

方源清楚。如果他不拿出一个像样的解释,铁若男是不会善罢甘休的。

对于这点,方源也有了应对之策。

于是他又叹了一口气:“每每想到青茅山,我的心就一阵阵发痛。你虽然死了父亲。但我却丧失了所有的族人,被迫流浪。商家城虽好,但到底是异乡,再没有家的味道。”

说着,说着。他眼眶泛红了。

看到方源这般,铁若男坚定如铁的目光,不由地柔和下来。

同是天涯沦落人,两人皆是受害者。相比较自己而言,失去了所有族人的方正,无疑更加可怜啊。

“你知道吗?你杀了我族的一位少主,若非我一力阻拦,你就会受到我族的制裁。”铁若男换了一个话题道。

方源脸色一变,急忙分辨道:“铁刀苦的那件事情我知道。但我不是故意的!我挖下了坑,是要对付草裙猴,谁叫你们铁家一路追踪我,还踩中了陷阱?这是自找死路,怎么能怪我!”

“杀人偿命。天经地义,不是吗?”铁若男面色严肃起来。

方源心中冷笑:“若是如此,那我该有多少条命才能偿还?”

脸上也在冷笑:“铁若男,发生了这么多事情。你我都不再天真。你们铁家在这件事情上占不住理,之所以没有动手。主要还是因为我手中的紫荆令牌吧?”

铁若男倒也坦诚:“紫荆令牌,的确是主要的原因。但是紫荆令牌,只能保你在商家城的安全,出了商家城,我们铁家势必不会和你甘休。如果你能告诉我当初的真相,我可以担保,只要我在一天,铁家就不会因为这件事情与你为难。”

方源心中稍稍惊讶了一下。

这个铁若男,看来这些年发展得很好。就算是铁家少主,有这样权柄的不多啊。

“你若不相信我,我们可以使用毒誓蛊。”铁若男接着道。

又用毒誓蛊?

说实在话,方源合炼言而无信蛊,炼得都有些腻味了。

“真相其实也没有什么,你如今是铁家的少主,难道还猜测不出来吗?”方源垂下头,用眼角的余光暗暗打量铁若男。

他此言以试探为主,但铁若男并未觉察出来。

少女微微一笑:“其实就算你不说,我心中也有数。”

方源语调顿时一变:“你已经知道了?”

铁若男幽幽地叹了口气:“十绝体在蛊师界上层,根本不算秘密。能造成那种景象,直接冰封整个青茅山,就算是五转蛊师也做不到这一步。只是我没有想到,你的哥哥不是古月阴荒体,而是北冥冰魄体。”

“什么?”方源心中诧异地叫了一声,不过面部表情却未有多少变化,只是恰到好处地将双眼眯起来。

“她怎么会认为我是古月阴荒体?”方源感到一阵荒谬。

“等等……难怪当时,族长古月博会莫名其妙的维护我。难道说,我的修为进步,被他们认为是十绝体了?”方源这么一想,忽然明白了一些事情。

“若是她这么认为,北冥冰魄体反而成了我的最大伪装。这么说来,她应当没有怀疑我的真实身份。只要接下来,我不露出破绽的话……”

想到这里,方源的脸上浮现出复杂而又凄苦的神色。

他没有说话,只是长叹一声。

说多了,就容易露出马脚来,言多必失!

铁若男见他如此神色,更加确定心中的答案。她把声音放缓放柔:“方正,我知道你心中很痛苦很复杂。毁掉你的家园,害得你流浪在外,杀害了你全部的族人的凶手,却是你亲生哥哥……”

方源挥手打断她的话,眼眶泛红:“你不要说了,你既然已经清楚,何必再来问我呢。”

“可是我需要明确的答案。以上这些,都是我的猜测!”铁若男目光逼迫而来。

方源沉重地点点头,默默地流下泪滴。

铁若男见此,再不好逼迫,幽幽地道:“你知道吗,我曾经又赶回到青茅山,看着漫山遍野的冰雪,心中一片茫然。我知道父亲死于方源之手,若方源健在,杀父之仇不共戴天,我必定要杀他泄恨。但是他也已经死了……”

“不甘心又能怎样呢?子欲养而亲不待,仇欲杀而已身死。人生大憾呐!”铁若男长叹,殊不知自己的大仇人方源正坐在她的面前。

方源冷哼一声,语气夹杂着一丝不悦:“方源毕竟是我的哥哥,他人已经死了,你还想怎样?”

铁若男双目炯炯发亮:“我还想知道一些事情。当初我父亲接到一份神秘信笺,信上的内容我现在已经得知。是说你古月山寨中藏有血海传承,所以我父亲才不顾身上之伤,第一时间赶往青茅山。你和白凝冰,是否知情?”

方源摇摇头:“要是有血海传承,我早就取用了,否则怎会在旅途中如此狼狈。”

铁若男饱含深意地看着方源:“血海传承,遗祸无穷,乃是当年魔道蛊师血海老祖的遗毒。追根究底,我父亲的死因源头,也是这道传承。方正,如果你真的继承了这个传承,希望你能将它交给我,算是让我稍稍弥补一些遗憾。”

方源继续摇头:“没有就是没有。”

铁若男沉默了一下:“根据情报,我知道你手中有一只蛊,血气盎然,曾经用作远程进攻。但是你转修力道之后,却几乎不再使用它。这是为什么呢?”

方源楞了一楞,旋即恍然。

“你怀疑我取了血海传承,却故意隐藏?哼,你指的应该是这只蛊吧?”

方源心念一动,从空窍中取出血月蛊来,主动抛到铁若男的手中。

“这是我族的血月蛊,你难道没见到我哥哥使用过吗?当时冰川爆发,我族族长和白家族长合力,拼死保下了我和白凝冰。族长将手中残留的蛊,都交给我。我和白凝冰在流浪途中,许多蛊都饿死了,只留下这只血月蛊,因为它容易喂养。”

方源一席话,不仅解释了铁若男的疑惑,同时也解答了自己和白凝冰二人,为什么能逃得一命。

铁若男检查了一遍血月蛊,神色松缓下来:“原来是这样,你家族长用心良苦,为了留下火种,不惜牺牲自己,真是壮举!”

方源哼了一声:“所以我更应该好好的活下去,重建古月山寨。谁要阻挡我,我就消灭谁!”

这算是解释了他在演武场的心狠手辣。

“虽然我和你相处的时间不多,但是可以明显的感觉到,你变了很多。”铁若男看向方源。她只是感慨,并没有怀疑。

遭逢巨变后,人也会发生变化,这是很正常的。

方源坦荡荡地和她对视:“人都是会变的。你不是也变了么?”

铁若男却摇头:“我只是一直在走我的路而已。”

此话后,两人陷入了沉默。

良久,铁若男这才开口:“铁刀苦我会带回去。我承诺,铁家今后也不会继续追究这件事情了。依附商家,是个能重建家族的好路子,许多人因此成功,也祝愿你能成功。”

说完这句后,少女站起身,很干脆的走了。

方源望着她离去的背影,双眼眯起来。

他有一种隐隐约约的预感,这个铁若男很不简单,将来恐怕会给他造成巨大的麻烦。

“真是想提前扼杀掉啊……”方源心中充满了遗憾。(未完待续。如果您喜欢这部作品,欢迎您来起点()投推荐票、月票,您的支持,就是我最大的动力。)

\end{this_body}

