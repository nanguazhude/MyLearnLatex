\newsection{死的才干净}    %第五十四节:死的才干净

\begin{this_body}

%1
天空阴沉沉的,乌云压得很低,一副山雨欲来的架势。

%2
商队在山道上沉闷的前行着。

%3
商队里原有的大型的黑皮肥甲虫,已经死的一个不剩。翼蛇剩下两只,各有残疾。倒是背囊蛤蟆剩的较多。皆因它们体型小巧,行动迅速,容易躲避野兽的冲袭。至于驼鸡,虽然和背囊蛤蟆差不多,但是它们一旦受到惊吓,都喜欢把头埋在泥土里去。这种自欺欺人的逃避方法,使得它们死伤最重。

%4
商心慈夹杂在人群中,看着身边的一排十几只的背囊蛤蟆,眼光复杂莫名。

%5
这些背囊蛤蟆上背着的货物,都是张家的。

%6
方源和白凝冰虽然没有去参加商议,但是商队的首脑们仍旧是调拨了一大批商货,交到她的手中。

%7
“这就是实力的原因么……”商心慈在心中叹着气。

%8
以前,张柱在的时候,这些首脑的态度平淡。但如今,他们态度客气,甚至带着一些讨好。

%9
还有那些蛊师,以及家奴。

%10
现在他们看向自己的目光中,都带着敬畏和忌惮。

%11
“这一切的转变,都来源于他们两个。”商心慈将目光投向不远处的方源,以及白凝冰。

%12
她的目光有些复杂。

%13
一方面,她从方白二人的身上,收获到安全感。另一方面,心性善良的她又对方白二人动辄杀人的狠辣,感到忌惮心惊。

%14
“呵呵呵,看来我们俩把那个小妞吓坏了。”白凝冰和方源并肩前行,察觉到商心慈的目光后,低声浅笑。

%15
自从杀了欧家父子,已经过去了七八天。

%16
影响扩散到整个商队。当然也包括商心慈和小蝶。

%17
小蝶如今在方白二人面前,大气都不敢喘一声。商心慈的目光也躲闪,不敢和方源对视。

%18
这种反应,都在方源的意料当中。

%19
这主仆俩,从小在张家寨子里长大,受着正道观念的熏陶。当方白二人展现出魔性之时,不可避免的就会意识到彼此间的区别。要接受两个魔道人物,还有一段心路要走。

%20
方源对此并不担心。

%21
外在的压力,会让她们不得不接受和妥协。人总是要活着的。接下来。几次兽群袭击之后,心中的疏远就会渐渐消失了。

%22
“现在的问题,是他。”方源将目光投向陈鑫。

%23
这个青年蛊师,好像就是那个和张柱一起逃跑的家伙。之前以为,已经被白羽飞象撞成肉泥了。没想到竟然活着。

%24
蛊师就算是修为不高,但有一两只奇特的蛊虫,就不能小瞧了。

%25
方源失了地听肉耳草,就没有侦察手段,被他逃了一命。也不知道他究竟了解到什么程度。

%26
但不管什么程度,方源都有应对手段。

%27
他心思谨慎,凡事不虑胜先虑败。当初计划的时候,就考虑到被人发现的情况。

%28
所以,他再杀了张柱之后,就特意展现了强大的实力。如果真有人发现。看到方白二人如此强悍狠辣,大多都会心生惧怕,隐而不发。

%29
欧家父子是撞到枪口上的,若不是欧飞找来。方源也会寻机发难,或者在兽群袭击的时候。展现出强大手段。

%30
当然,如果被揭发,方源也有许多手段反击。商队首脑贾龙,就是一个很好的棋子。

%31
他是贾富的人,而方源在古月山寨时,得到过贾富的令牌。只要出示令牌,在运作几番,就能获得贾龙之信任。

%32
但凡一个计划,哪怕设想再完美,实施起来,都会发生差错。

%33
所谓谋事在人,成事在天。

%34
方源就算是经验丰富,老谋深算,也有失败的可能。不过,也正因为如此,人生才会如此精彩纷呈。

%35
方源第一个设想的,就是最坏的结果。

%36
如果他运气极为糟糕,在杀害张柱的时候,被人看到,留下铁证。最后导致商心慈认清真相,对“恩将仇报”的他产生深深的仇恨。

%37
那么该怎么办呢?

%38
很简单,直接把商心慈杀掉就是了。

%39
她现在还只是一个凡人,杀她何其简单。商家的族长,也不知道她是自己的亲生女儿,手脚干净一点,不会有来自商家的后遗症。

%40
现在就结果看来,张柱死了,方源的运气不是最好的,留下了尾巴。但也不是最坏的,至少商心慈还被蒙在鼓里。

%41
方源可以确信这点。因为商心慈还年轻,在方源的眼中,她心中的情绪无所遁形。

%42
“前方发现鳄象群!”

%43
“有一支鳄象群向我们冲来了!!”

%44
“戒备,戒备!”

%45
就在这时,前方的侦察蛊师飞奔而回,带来一个坏消息。

%46
商队微微骚乱了一下,但旋即就回复了平静。

%47
“只是鳄象罢了,大家不要慌。”

%48
“我们人手不足,绝不能就地固守。”

%49
“不错,大家都分散,进入雨林中去!”

%50
蛊师们下达了最明智的命令,商队众人早就心弦绷紧,连忙四散而去。

%51
若在匪猴山之前,他们遇到这事情,恐怕还有人牵挂货物,犹豫不决。但如今,他们毅然舍弃,对这些商货看都不看一眼,专注于逃命。

%52
方源和白凝冰,一路护着商心慈主仆二人,一头扎进雨林当中。

%53
他对鳄象群的到来,毫不奇怪,因为正是他的设计。

%54
轰隆隆……

%55
鳄象群奔腾而来,很快周围就传来惨叫声,树木被撞倒的声音。

%56
方源带着商心慈等人,在雨林中谨慎规避,但鳄象众多,还是碰到了一头。

%57
鳄象体型较小,和牦牛出不多,这使得它更加灵活。

%58
它全身长满了类似鳄鱼身上的鳞甲,防御厚实,远超白羽飞象。象尾也形似鳄鱼的尾巴,拖在地上。

%59
“啊!”看到这只鳄象如小山般冲撞过来。小蝶发出一声惊呼。

%60
商心慈也是脸色苍白。

%61
“不用担心。”方源淡淡地嘱咐一声,面对鳄象悍然反冲锋过去。

%62
一人一象在半路上,狠狠地撞在一起,发出一声巨响。

%63
方源倒退两步,身上的白光虚甲晃动了三下。而那只鳄象,则头骨碎裂,血液喷涌,被撞倒在地上,推出十多步的距离。然后撞在一根大树的树干上。这才猛然停住。

%64
“好厉害!”看到这一幕,小蝶吃惊得瞪圆了双眼。

%65
这只是一只普通的鳄象,还不是百兽王。方源有雪银真元,又有双猪一鳄之力,料理起来自然分外轻松。

%66
不过。商心慈和小蝶二人,却没有看过这般火爆的场景。

%67
那张柱没有巨力在身,又是治疗蛊师,所以战斗的时候,都是躲闪和辅助为主。

%68
方源如此硬打硬撞,恣意张扬的战斗风格,自然给主仆二人留下深刻印象。

%69
大约一个时辰之后。鳄象群渐渐撤离,商队众人这才陆续从雨林中走出来。

%70
统计了一下,只牺牲了个把蛊师,还有十几个家奴。损失并不大。

%71
将货物整理好,商队再次前行。

%72
几日后,他们脱离了象牙山地界,向墓碑山进发。

%73
在之后的半个月内。商队先后遭遇到了黑岩熊,铁冠鹿群等等的袭击。

%74
因为方白二人几乎形影不离的保护。商心慈和小蝶毫发无损。

%75
这般朝夕相处,主仆二人的态度也发生了转变。

%76
商心慈对方源亲近了许多,交谈时有说有笑,眼神再无一丝躲闪。而小蝶则彻底转变为方源和白凝冰的崇拜者。

%77
崇拜强大,是所有生物的共性。因为只有强大,才能有更高的几率生存下去。

%78
况且方源和白凝冰虽魔道,却表现得极有原则。在主仆二人看来,他们别无所求,从不对她们动手动脚,只是还恩。这个行为,充满了英雄气息。就算是正道,有多少人能做到这点?

%79
哪怕方源丑陋,在主仆二人心中,也比许多虚伪造作的正道人士可爱许多倍。

%80
几日后,商队进入墓碑山地界。

%81
周围开始出现僵尸。

%82
墓碑山,曾经不叫墓碑山。一百多年前,在山上还有一个大家族。

%83
一个魔道蛊师改变了这一切。

%84
他曾是这个家族的一个家奴,新婚当天,美丽的妻子被家族的一位蛊师霸占后凌辱致死。

%85
他将仇恨深埋于心,机缘巧合之下,获得了魔头“僵王”的传承。

%86
卧薪尝胆近百载后,五转修为的他挥动僵尸大军,突袭了这个家族,将所有人屠戮一空。然后连他们的尸体都不放过,转化为僵尸。

%87
做完这一切后,他在山寨的废墟上,竖立了一块巨大的墓碑。

%88
墓碑上,刻着他妻子的名字。

%89
此事震动南疆。

%90
至此之后,此山就被人称之为墓碑山。山中游荡着僵尸,它们杀死野兽或者过往的路人,吸取血液为食。而它们的尸毒,则会感染尸体,形成新的僵尸。

%91
因此,墓碑山僵尸不绝。

%92
为了保障商路安全,每年都会有家族组织打僵队,清理这些僵尸。

%93
但不管怎么清理,僵尸总是杀之不尽,绵绵不绝。

%94
毕竟打僵队规模有限,南疆多山,途中艰险,劳师远征都耗费甚巨。投入多,而收获少,又不可能真的把墓碑山翻个底朝天。但凡有漏网之鱼,一段时间后,就会形成僵尸规模。几次大规模的联合后,世人的热情就都消耗光了。

%95
这一夜,商队驻扎在墓碑山的山脚下。

%96
星空璀璨,方源仰望着黑沉沉的墓碑山影,眼中闪过思量的光。

%97
“时机已经成熟,这些人也就失去了利用价值。是时候解决掉这些麻烦了。”

%98
陈鑫是要死的,但独独解决掉他,恐怕更加麻烦。他到底知道些什么,又向谁透露过,除了他还有没有其他看到的人?方源一概不清楚。

%99
但方源也不想去弄清楚。

%100
因为在他的计划中,陈鑫要死,其他人也要死。

%101
都死了,才算是干净。

\end{this_body}

