\newsection{拍卖(下)}    %第一百零八节:拍卖(下)

\begin{this_body}

“诸位请看,这是一根千年风柳。”女蛊师指着台上的事物,徐徐介绍着。

“风柳这种植株,乃是罕见珍稀的炼蛊辅料。生活环境特殊,需要风的力量,才能持续不断的成长。因为飓风山常年大风不止,因此产有大量风柳。”

“寻常的风柳,只有数十年,上百年。但作为此次的拍品,这根风柳,有上千年的年龄,可用做五转蛊的合炼辅料!底价五万元石,现在开始拍卖。”

女蛊师的话音刚落,就有人开始叫价。

“五万五千!”

“五万八千。”

“六万五千。”

“七万。”

价格不断地上涨,许多人都参与了竞价,大多数都是专业的炼蛊师。

千年风柳相当少见,就算是沮家,在飓风山上屹立了数百年,如今已只有十八条存货。

风柳没有树干,只有扎根在山石中的根,以及仅仅一条的枝。

当大风刮起来时,这根唯一的枝条,就会随风飞舞,仿佛海底的修长水草。

台上的这根千年风柳,已经被人为的盘起来,但总长度绝对超过百米。

方源试着掺和一脚,报了个价格。果然那边商睚眦立即开口,将价格提高五千。

但最终,风柳被一位炼蛊大师收入囊中,花了近十万的价格。

“千年风柳,已经是第八件拍品。在此期间,我尝试出手,每次都遭到商睚眦的狙击。我若不出手,他就不出手。看来此人参加拍卖,是专门过来和我作对到底的。”方源目光闪了一闪,对商睚眦的狭小气量认识得更加深刻。

时间在不知不觉中流逝。

第九件拍品。第十件拍品……

“下面展出的,是第十二件拍品。这是一只追风蛊,四转移动蛊,可令蛊师有追风的速度。底价十八万。”女蛊师清脆的声音,在整个拍卖场中回响。

“十九万。”

“二十万!”

“二十二万!”

“二十五万!!”

……

追风蛊是热门拍品之一,一出现,就掀起了拍卖场的第一个高氵朝。

很多人都参与竞价,以往悄然安静的包厢,更是接连喊出报价的声音。

风、光、电、云四类蛊中。移动类的蛊都十分出色。追风蛊是四转蛊虫,带给蛊师的速度增幅很大。

方源试着喊了个价格,那商睚眦立即跟上,把方源的报价压下去。

方源不再报价,两人的价格旋即淹没在随后的报价声中。

不时有新的竞争者。加入这场角逐,但同时更多的人望着节节攀升的价格,无奈地选择放弃。

当追风蛊的价格,上涨到三十五万的时候,只剩下两位蛊师在竞争。

一位是翼家家老翼不悔,另一位则是飞家的家老飞鸾凤。

翼家和飞家的关系并不融洽,两位家老的竞争也掺杂着火气。

最终。追风蛊以四十四的高价,被翼不悔家老买下。

方源尽管有两百多万的元石,却没有参加角逐。

追风蛊蕴藏风的道纹,他若运用。身上的各大兽影都会有干扰作用,效果会大打折扣。因此追风蛊并不适合方源。

“若是四转的仙风蛊,我倒是有意向拍买,可惜只是追风蛊而已。”

方源掌握着一道秘方。可以用仙风蛊和七颗明星蛊等一些辅料,一起合炼成“定仙游蛊”。

入梦游、逍遥游、定仙游、酒神游。合称为四大移动蛊。

这四种蛊,都是五转蛊。

其中,入梦游可以让蛊师进入他人的梦境。逍遥游,最擅长闪避攻击。定仙游,是一次性的消耗蛊,但却能让蛊师传送到天底下任何想要去的地方。酒神游,则最为特殊,最早出现在《人祖传》,这里暂且不表。

方源有前世记忆,因此掌握着逍遥游蛊、定仙游蛊两道秘方。其中定仙游蛊,最为方源所需,可惜合炼成功率极低,而且合炼的材料都筹集不全,只能暂且作罢。

“好了,接下来是本次拍卖会的第十三件拍品。一只……苦力蛊!”

女蛊师在台上洋洋洒洒的介绍一番,其实不用她说明,许多蛊师的眼睛已经亮起了光辉。

“这只苦力蛊我势在必得,谁也阻止不了我。二十万元石!”不待女蛊师报价,方源已经开口高喊。

方源的话引发了巨开碑的不快:“年轻人,就是急躁。我出二十五万。”

“三十万。”另一旁,商赑屃紧接着开口。

“五十万!”方源报出一个惊人的价格。

整个拍卖场顿时一片嘈杂。

“听着声音,应该就是古月方正!”

“他真是财大气粗啊,为了一个苦力蛊,叫出五十万的高价。”

“他在演武场赢了那么多次,财力很雄厚。”

人们纷纷感叹,一些力道蛊师一脸的苦涩,他们原本还对苦力蛊抱有期待,也想尝试一下,但没想到方源这样一搞,他们还未报价,就提前出局了。

就连台上的女蛊师也是一脸的意外。

苦力蛊乃是四转蛊,但力道蛊虫向来对真元要求不高,三转境界也能勉强运用。一般市价都在三十八万左右,比追风蛊的价格还要差一些。

没有想到,方源一开口就是五十万,直接高出十二万元石上去。

“现在的年轻人,真是有干劲啊。”巨开碑感叹一声,不再报价了。五十万的元石,足够他买些其他蛊,对他更有帮助。

“八哥、九哥……”商赑屃犹豫地看向身旁两位少主。

九哥商狻猊没有说话,只是看向商蒲牢。

“十弟,这就要看你的意思了。你想要争,我们做哥哥的自然力挺你到底。”商蒲牢笑着鼓励道。

商赑屃咬了咬牙:“五十万……实在得不偿失,罢了,就让他方正得去吧。”

他也放弃了角逐。

方源展现出势在必得的决心。一下子将两位竞争者都打下去。

“五十万第一次。”

“五十万第二次。”

“五十万第三……”

“慢着,我出五十一万。”就在女蛊师即将一锤定音的时候,从五号包厢中传出商睚眦慢条斯理的声音。

“哼,商睚眦,就凭你也想阻止我?六十万元石。”方源立即开口,语气中流露出一股不屑之情。

商睚眦冷笑:“方正你想玩,我就陪你玩下去。六十一万。”

拍卖场一片轰然。

“方正和商家少主杠上了!”

“怎么会这样?”

“方正有紫荆令牌在手,乃是商家贵宾,不惧商睚眦的少主身份。但商睚眦掌管商家城的大小商铺。财力更雄厚。这将是一场龙争虎斗。”

……

“七十万!”方源大叫一声,“商睚眦你栽在我手中,乃是我的手下败将。看来上次是你接受的教训不够啊。”

商睚眦立即反驳:“放你的臭屁,看这次究竟是谁教训谁!七十一万!!”

此话传出,众人又是一阵沸腾。

商睚眦和方源的矛盾。在商家城高层不是秘密。但是对于寻常蛊师而言,却是个大八卦。

一时间,人们开始纷纷猜测,商睚眦究竟为何和方源结仇。

“很显然,商睚眦曾经在比斗中败给方源,因此要报复他。”

“何必计较这些原因。睚眦少主气量狭小,就算是走路时与路人磕磕碰碰。他都会报复。再加上方正是个无法无天的主儿,两人不闹出矛盾才稀奇呢。”

“你们都猜错了,我已经打听到了,商睚眦和方正二人一次同上秦艳楼。都对头牌安渔姑娘有意。但最终方正得手,因此商睚眦记恨在心。”

“真的假的呀?”有人旋即表示怀疑。

刚说话的那人,表现是凛然无畏的样子,一指某处座位:“你们看。那里就坐着安渔姑娘。你们不用怀疑我,只需问问她就成。”

于是。众人向安渔姑娘投去询问的目光。

安渔姑娘也楞了,没想到忽然牵扯到自己。但她忽然看到,人群中老鸨正向她眨眼示意。

她顿时明白,这是一场灵机一动的炒作。须知像她这种身份的青楼女子,只有这样炒作,才能令自己身价更高。

她没有正面回应,垂下头,脸上布满了红晕。只做出这番神态,顿时就让怀疑的人相信了大半。

“果然是这样。”

“看安渔姑娘的神态,答应呼之欲出了!”

“你们不知道内情,但我知道。历来美人配英雄。安渔姑娘为什么喜欢方正呢?”

“为什么呀?”

“嘿嘿……因为商睚眦单薄干瘦,下面不行啊。但方正却龙精虎猛,每次冲击,都能动用全力以赴蛊。山猪的冲撞力,棕熊的拍击力,鳄鱼甩尾似的绞劲,还有骏马的奔驰力,青牛的持久力,石龟的耐力……啧啧,这样的男子,哪个姑娘不喜欢呢?”

“哦——!”顿时,许多男子都发出心领神会的顿悟声。

许多女子也不由地夹紧双腿,满脸红晕,陷入到遐想当中。

安渔姑娘的头,垂得更低了,心中则暗暗欢喜:“老娘要火了,老娘要火了!”

拍卖场大厅里众人集体八卦的时候,方源和商睚眦的竞价,已经飙升到八十一万。

“方正,今天有我在,你必输无疑。放弃吧,你是斗不过我的。”商睚眦得意的大笑,他又一次增加一万元石,就是为了恶心方源。

“你以为我会怕你!不就是八十一万而已,有什么了不起的。”方源发出一声冷哼。

“少主,可以了,该收手了。”一位家奴小心翼翼地觐言道。

商睚眦犹豫了一下:“我心里有数。方正对苦力蛊势在必得!等方正再加价,我再加一次,然后再收手。这个蠢货,花了八十多万,买只苦力蛊,足足是两倍多的价格。叫我出了口恶气,真是令人开心,哈哈哈!”

这时,方源的声音传来。

“不过这次,我就放你一马,苦力蛊让给你了。”

家奴:“……”

商睚眦:“……”

\end{this_body}

