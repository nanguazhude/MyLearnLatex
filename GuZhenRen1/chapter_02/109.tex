\newsection{拍卖(终)}    %第一百零九节:拍卖(终)

\begin{this_body}

“哈哈,现在的商睚眦恐怕脸色铁青吧。”九号包厢中,魏央忍不住笑起来,“不过,方正老弟,你没有买到苦力蛊,真的不要紧吗?”

一旁,商心慈也向方源投来关切的目光。

方源笑了笑:“苦力蛊我的确势在必得,不过,我却不想做这个冤大头,花八十一万买只蛊。我宁愿自己合炼。八十一万……足够我尝试许多次了。”

“但合炼失败的可能性很大,而且对黑土哥哥你也会有损伤。”商心慈语气担忧。

方源轻轻地摇摇头,此事他另有定计,却不能和外人明说。

“嘿!这个方正,把商睚眦给耍了。商睚眦花了足足八十一万,就买了只苦力蛊。”群众的眼睛是雪亮的。

“商睚眦这个蠢货,简直是给我们少主丢脸!”商家少主们皆现有怒容。

“话说方正演得还真像,连我都被他骗了。”有人感慨。

“旁观者清,当局者迷啊。我早就意识到有些不妥之处了。”有人马后炮。

“商睚眦气量狭小,喜好报复,因此一直跟价。但方正也没有得偿所愿,两人都是输家。所以拍卖场中不能怄气啊。”巨开碑心中感叹着。

“真正的赢家,只有拍卖场。”

“沮家开心了,一只苦力蛊,卖了八十一万!”

众人议论纷纷,交头接耳。

但事实上,他们忽略了还有一个大赢家。

那就是骤然出名的安渔姑娘。

啪。

商睚眦奋力将手中的青瓷杯盏,扔到地上,摔个粉碎。

五号包厢内,家奴们瞬时跪地垂首。大气都不敢喘一下。

商睚眦坐在座位上,鼻息粗壮,额头上青筋直冒,满脸的怒容。

被方正坑了!

八十一万啊,买了只苦力蛊,自己根本用不上。

商睚眦感觉到自己的心在滴血!

事实上,他是精明的。他从那次挫折中,汲取了教训,吃一堑长一智。这一年多来励精图治。将商铺经营得很好。

但性格决定命运。

他就是小肚鸡肠的人。仇恨令他智昏,中了方源的圈套。

“方正、方正,若不是毒誓蛊,我一定会杀死你,把你千刀万剐啊!!!”商睚眦在心中不停地咆哮怒吼。

拍卖会继续下去。

第十四件拍品,第十五件,十六件…十八件…二十八件……

一波波的竞价高潮掀起来,气氛热烈,让人们很快遗忘了方源和商睚眦之争。

“下面是第三十二件拍品――四转风气蛊。”女蛊师的声音仍旧清脆悦耳。

风气蛊形如一只蝴蝶,它有青蓝色的双翅。每一次扇动,都会有碎钻般的星芒在周围的空气中产生。自然是十分吸引人们的眼球。

风气蛊,是很特殊的蛊虫。它吸收生命的活力,从风中诞生,是天然蛊。到目前为止,还未有秘方大师,研究出炼成它们的秘方。

秘方大师一般分为三大流派。过去流派,研究消失的力道、气道等等蛊虫秘方,试图还原。现在流派。研究天然蛊,企图钻研出合炼它们的秘方。还有未来流派,专门创造新蛊虫的炼蛊秘方。

风气蛊不仅出身特殊。用途也特殊。

它针对一支种族群体而施展,用一种无形的力量,营造出一种集体中流行的爱好或者习惯。

上古时代,蛊师们用它来对付兽群。比如要对付一支钢针猪群,蛊师用了风气蛊后,这支钢针猪群,就忽然形成了一种,喜欢用全身皮毛蹭石头的习惯。

钢针猪的皮毛。如根根铁针,攻防一体。蹭了石头之后,皮毛渐渐损毁,蛊师们对付起来,不费吹灰之力。

但后来,蛊师们渐渐发现,风气蛊用来统治部落、家族,是绝好的利器。

有的家族,粮食缺乏,却爱好酿酒。用了风气蛊后,改变了酿酒的习惯后,粮食增多,家族壮大起来。

风气蛊不仅可以用来对内,也可以用来对外。

在历史上,有个很著名的例子。

两个家族相争,弱小的一方动用了风气蛊,使得强大的家族忽然兴起了女子裹小脚的风气。

这使得这个家族中的女子,劳动力大减。女性蛊师也战斗力大降,最终被弱小的家族翻盘逆转而灭。

说到底这是蛊的世界,有千奇百怪的蛊虫。

女蛊师洋洋洒洒地介绍一通后,道:“风气蛊,底价二十六万元石。”

“三十万。”翼家的家老翼不悔首先报了价。

“三十五万。”飞家的飞鸾凤毫不示弱。

“三十七万。”一位秘方大师喊道。

“三十八万。”魏央开口。他执掌风雨楼,也希望用这风气蛊使下属办事,更加尽心尽力。

“五十万!”方源再次高喊。

全场一静。

方源沉寂良久,这一次的声音又让众人回想起,不久前他和商睚眦的争锋。

“五十万元石,买一只风气蛊?方正,你还想坑我?当我傻子吗!”商睚眦咬牙切齿,眼中燃烧着愤恨之火。

他刚刚花了八十一万,再花五十万,可就要破产了。

“五十万一次……两次……三次,成交!”女蛊师喊道。

五十万的价格,稍微超过了众人预期,没有人加价。

风气蛊对势力有效果,对个人用途极少,也使得大多数人兴致缺缺。

“方正老弟,这风气蛊四十六万就可以拿下的。”魏央叹了口气。

商心慈却有不同见解:“不,考虑到翼家、飞家的两位家老,很可能较劲,使得价格一路突破上去。叫价五十万,一下子打消他们的想法。也是正确的抉择。”

“四十六万……五十万……多了四万元石而已。魏大哥,这只蛊我就收下了。”方源挥挥手,不在乎地道。

“怎么,你难道也想重建古月山寨?”魏央有些吃惊,没有料到方源对风气蛊真的感兴趣。

刚刚,他还以为,这是方源替他喊价的呢。

“当然有需求,不过此事未成,却还须保密。”方源笑了笑。没有作过多的解释。

“哼。神神叨叨的。”白凝冰见不惯方源这作风,知道他必有图谋,不由地暗暗警惕。

风气蛊之后,是一套餐风蛊。

三十八只餐风蛊,合成一套,一起拍卖。

餐风蛊只是二转蛊虫,但很实用。能令蛊师以风为食,填饱肚子。

沮家寨处于飓风山上,擅长风类蛊虫。餐风蛊就是他们的特征之一。

时间不断流逝着。

第三十四件拍品,三十五……三十八……四十四……

方源再无出手。倒是魏央出手一次,将一只光类蛊成功买下,算是如愿以偿。

白凝冰也连拍三次,最终入手一只三转的龙卷蛊。

正当方源感到无趣之时,最后的第四十九件拍品登场了。

“这是本场拍卖会的压轴宝物。它并非是蛊虫,也不是珍贵的炼蛊辅料,它是一则秘方。”女蛊师徐徐地介绍道。

接着,她又补充一句:“这则秘方,由于十分珍贵。因此还未鉴定。”

这句话无疑勾起了大多数人的好奇心。

一般珍贵的炼蛊秘方,都要谨慎鉴定。因为一旦交给秘方大师鉴定,这秘方就有泄露出去的危险。

秘方。秘方,当然是知道的人越少越好。没有秘密众所之知的秘方,其价格会连一张白纸都不如。

女蛊师经验老道,特意没有说话,留给众人反应的时间。

看到众人涣散的目光,又集中到自己的身上,女蛊师浅浅一笑,抛出一个炸弹:“这则秘方。事关天元宝君莲。”

“天元宝君莲?我没有听错吧!”

“三转天元宝莲,四转天元宝君莲,五转天元宝王莲……这个系列的蛊太有名了,简直是如雷贯耳啊。”

“这是元莲仙尊的核心蛊。据说,谁能炼制出六转的天元宝皇莲,就有机会继承元莲仙尊留下来的传承遗藏!”

“想不到沮家寨中,居然有这等收藏……”

拍卖场中人群沸腾了。

女蛊师接着开口:“想必诸位都对天元宝莲有所了解,这里我就不多做介绍了。这是一份关于如何合炼出天元宝君莲的秘方。底价五十万元石!”

秘方的价格,原比蛊虫要高得多。

四转天元宝君莲的秘方,比天元宝君莲本身还要昂贵。

授人鱼不如授人以渔。从理论上讲,有秘方,就能有无数的天元宝君莲。

“等一等,我有个疑问。这份秘方,可需要天元宝莲为合炼材料?”方源忽然开口,高声喝问。

女蛊师面色微微一变。她有心不想回答,但却知道方源乃是紫荆令牌之主。

她一直想刻意地回避这个问题,但没有料到方源如此才思敏捷,发现了关键之处。

无奈之下,她只好实话实说:“虽然原则上,秘方中的内容,不会公布。但商家以诚信为本本,这秘方中的确需要天元宝莲为主料。”

众人不禁哗然。

“要以天元宝莲为主料,我们从哪里找得到这玩意?”

“难怪沮家得了这道秘方,也没有合炼出天元宝君莲呢。”

“这秘方有什么用?前不着村后不着店的,太尴尬了。”

“好险,得亏了有方正问了这么一问。”

“最关键是,这秘方是真是假,还没有鉴定呢。”

“这秘方能被沮家收藏,料想不差。我出五十二万元石。”一位秘方大师报价道。

尽管如此,事关天元宝君莲的秘方,仍旧对一些蛊师有很大的吸引力。

“五十五万。”

“五十八万。”

价格交替上升,越来越慢,最终卡到六十六万。

“六十七万。”魏央的最后报价,使得他终究买下了这个秘方。(未完待续。如果您喜欢这部作品,欢迎您来起点投推荐票、月票,您的支持,就是我最大的动力。)

\end{this_body}

