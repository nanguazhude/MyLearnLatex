\newsection{路}    %第十二节:路

\begin{this_body}

%1
百丈高空,狂风呼啸。

%2
“死!”天鹤上人狂吼一声,奋起残余力量,化作一道白光,冲向古月一代。

%3
“想要杀我,你太天真了。”血鬼尸张开血盆大口,獠牙外龇,不退反进,硬是撞上去。

%4
轰!

%5
巨大的轰鸣爆响声中,巨力涌来,将两人远远炸飞。

%6
两败俱伤!

%7
“青山不改,绿水长流。师弟,今日之仇,来日必报。”古月一代嘎嘎大笑,振动背后黑的蝠翅,正要飞遁。

%8
忽然一个巨大的阴影,笼罩下来。

%9
他抬头一望,正是那只伤势极重的铁喙飞鹤王。

%10
“居然没死?可恶……”古月一代面色骤变。

%11
被铁喙飞鹤王拦截住,天鹤上人也赶过来前后夹击。最终,飞鹤王坠落大地,血鬼尸被打爆。

%12
天鹤上人抓住古月一代的头颅,悬停在半空中,目光散乱了好一会儿,这才回过神来。

%13
他仰天大笑三声,然后笑声转为哭声。

%14
发泄了片刻之后,他飞身赶往青茅山。却被白凝冰自爆的冰川镇压,他不得不启用存息葬玉蛊。破冰而出时,他疯狂搜寻,但最终只从冰层下,挖出了奄奄一息的方正。

%15
……

%16
墙壁上的画面,忠实地显现出当时的情景。

%17
接下来的事情,就乏善可陈了,都是天鹤上人赶回来一路上的风景。

%18
仙鹤门乃是中洲最古老的门派之一。历来,门派中人要出去执行紧急任务,都会随身携带着一只蛊虫,记录执行的过程。

%19
不管任务成功与否,蛊师回山复命,门派中都会根据记录的过程,来进行相关的考核评定。

%20
天鹤上人虽然是门派长老,但是也不能免除这个规矩。

%21
只是他资格太老,仙鹤门中的蛊师很少有资格来对他进行考核评价。就算是当代的仙鹤掌门人也没有。

%22
不过此刻,盘坐在天鹤上人面前的这位老人。却非比寻常。

%23
鹤风扬,太上长老,六转蛊师,号称鹤羽飞仙!

%24
他此刻身着白袍,系着黑腰带,大袖翩翩。盘坐在蒲团上。

%25
他面如少年。温润如玉。眉毛碧绿修长,眉间一直垂到腰间。幽深的双眼盯着墙壁上的画面,然后悠悠地收回目光。

%26
天鹤上人恭谨地站在他的面前,大气都不敢喘一口。

%27
其实说起来,天鹤上人沉睡近千年,鹤仙人还属他的晚辈。然而所谓资格,从来不是论年龄,而是论实力。

%28
天鹤上人虽然是五转,和鹤风扬只有一级之差。但这差距。却是天壤之距,云泥之别。

%29
五转是凡人,六转已成仙!

%30
“天鹤上人。”鹤风扬徐徐开口,声音如溪水般清澈柔缓。

%31
“在。”天鹤上人连忙应道。

%32
“你此次斩杀了门派叛徒,却并未寻回血颅蛊。反倒被两个小辈戏耍镇压……”房间中,回荡着鹤风扬的声音。

%33
天鹤上人头垂得更低了。面露惭愧之色。

%34
“不过,那其中一个却是北冥冰魄体。你被镇压,也并不奇怪。欣慰的是,事发之后你做了补救。我问你,你带回来的那个少年,真的是画中小子的孪生兄弟吗?”鹤风扬问道。

%35
“在下已经查明,正是如此。那逃走的是哥哥方源。被抓来的是弟弟方正。更妙的是,这兄弟俩一直感情不和。在我将他哥哥屠杀族人的影像,给方正看过之后,他现在是恨不得立即杀了他哥哥呢。”天鹤上人阴笑着说道。

%36
鹤风扬微微点头:“看来你已经有所计划了。但是你的时间足够吗?”

%37
天鹤上人面色一变。语气陡然变得低沉下来:“太上长老提点的是。在下已经达到寿命的极限,是一个将死之人了,只剩下几天可活。就算是存息葬玉蛊,也改变不了这个状况。”

%38
他这种情况,只有寻到寿蛊,才能从根本上解决问题。

%39
但寿蛊难寻,仙鹤门中倒是有几只寿蛊,却轮不到他。一直牢牢掌管在掌门和几位太上长老的手中。

%40
“所以在下在不久前,就舍弃了一生积蓄和蛊虫,换取了一只寄魂蚤。我将接引方正入山门,并收他为徒。之后随身指点他,炼成至亲血虫。再督促他杀死他的亲哥哥,为仙鹤门夺回血颅蛊!”天鹤上人道。

%41
鹤风扬微微扬眉:“听起来,你很自信自己的这个计划。不过,那个方正真的会一直受你摆布吗?”

%42
“他虽然资质不俗,但到底还是个孩子。在下死后,将魂魄存放在寄魂蚤中,将一直伴随他成长,指点他修行。路已经给他铺设好了,他只有走下去!”

%43
说到这里,天鹤上人拜倒在地上,叩首恳求道:“请长老再给在下,一次机会!”

%44
鹤风扬沉默片刻,这才道:“也罢,就许你这最后一次。”

%45
“谢长老,谢长老!”天鹤上人大喜。

%46
“去。我等待着你的好消息。”

%47
“不出二十年,必有成功之果!”天鹤上人激动得语气微微颤抖,唯唯而退。

%48
……

%49
南疆,万程山,铁家城。

%50
高大厚重的黑石城墙,延绵数千里。铁家城从半山腰起,无数的石屋、铁楼,依次排列,一直绵延到山顶。

%51
阳光照耀下,城内人来人往,车水马龙,一片鼎盛昌隆之景。

%52
家主阁就坐落在山顶附近,防御严密,周围人流明显稀少。巡逻队交替巡查,蛊师各个精悍逼人,一丝不苟。

%53
家主阁的楼顶,有两个人站着。

%54
一位中年男子,脸色冷酷如铁,眼中则隐藏着几分温柔。

%55
站在他身边的,是一位少女,目光悲伤却又坚定。

%56
正是铁若男。

%57
“你才回来几天?但这已经是你第十九次向我请辞。你父亲的死,令我万分悲痛。你早年丧母,现在丧父,但你要记住,你还有我这个舅舅。你是我的侄女,我是不会让你去冒险的。”铁家族长叹道。

%58
铁若男目光灼灼。直视铁家族长:“舅舅你知道吗?父亲虽然去了,但我悲伤中却又替他感到高兴。父亲一生立志铲除邪恶,惩治罪犯。他做到了,就算是身上负伤,也没有退缩。他坚持如一,走完了自己的人生道路。而现在。该我走下去了。”

%59
铁家族长脸上愣住。

%60
在这一刻。他在铁若男的身上,看到了铁血冷的影子。

%61
真像啊,这双眼睛,这样的目光。

%62
恍惚间,铁家族长仿佛回到了年轻时候,铁血冷就站在他的身边,盯着山巅的镇魔塔,坚定地道:“我要打尽天下的罪犯,让世界充满正义和爱!将魔道中人都关押到镇魔塔里去。哪怕把镇魔塔都塞满!”

%63
昔日的誓言,还犹在耳边。但是挚友已经不在……

%64
眼前的眸子重合在一起,铁家族长微微摇头,晃散回忆。他用一种既欣赏又爱怜,既担心又鼓励的目光,看向面前倔强的少女:“你选的这条路。可不好走啊。”

%65
铁若男没有答话,而是转头遥望山巅。

%66
在万程山的山巅最高处,矗立着一座雄伟的铁塔。

%67
它气势磅礴,似乎顶住了苍穹,踩踏着高山。白云如雾,在它周围缭绕,使得外人看来迷蒙模糊。又给它增添一分神秘色彩。

%68
这塔不仅是南疆盛景,更天下闻名。就算是中洲,也多有人耳闻之。

%69
镇魔塔!

%70
塔身高有百丈,分列近百层。塔楼形制古朴。巍峨沧桑,正气堂皇。自从建成以来,铁家蛊师关押了多少魔道蛊师进去。数百,上千,成万?

%71
就算是铁家族长也未必清楚。

%72
它是正道的象征,是铁家蛊师心中最深处的骄傲。多少的魔头魔子,将野心埋葬在这里,留下悲痛、悔恨、不甘、遗憾。

%73
魔道蛊师谈它脸色惨淡,正道中人说之眉飞色舞!

%74
铁若男开口,语气坚定如铁,似对铁家族长,也似乎自言自语:“万程山巅有一座镇魔塔,我的心中也有一座镇魔塔。这条道路,父亲没有走完,那就让我接替他继续走下去!”

%75
……

%76
“坚持不住了……”陈翠花头昏眼花,时不时犯恶心,想要吐。但吐又吐不出来,浑身虚弱,一阵阵的乏力感不断袭来。

%77
她原计划是坚守三天,但一天刚过,她就知道自己先前太过乐观了。

%78
蛇毒带给她的危害,越来越严重。她知道自己已经被逼入悬崖,必须尽快寻找到一只治疗蛊虫。

%79
“真是该死。若不是那两个小贼,说不定我早就捉到了野生蛊虫,解除了蛇毒了。”她心中焦躁不安,自从中了蛇毒,她一直都在尝试着寻找治疗蛊虫。但茫茫山林,充满了危险,她又没有什么捉虫手段,到现在都没有什么进展。

%80
白凝冰举起手中匕首,正要往自己的右耳割下。

%81
“且慢。”方源忽然伸手,一把将其手臂抓住。

%82
锋利的刀锋,离白凝冰的右耳只差分毫之距。

%83
要使用地听肉耳草,就需得割掉右耳,替换上去。左右不过是个耳朵,也不是什么大事情。相比较即将的大收获,白凝冰更不觉得有什么可惜。

%84
魔道中人,心狠手辣。对别人狠,对自己更狠!

%85
而往往这样能舍能弃的人,才能成就一番大事。

%86
“不需要割了,她出来了。”方源说着话,便开始动身。

%87
他利用地听肉耳草,远远跟在陈翠花身后。

%88
不一会儿,他就听到了打斗声。

%89
两人靠着茂盛的灌木丛,偷偷接近。只见这魔道女蛊师正和一只双头山猪激战。

%90
观战片刻,两人眼中均冒出兴奋的光。

%91
陈翠花明显状态很糟糕,战斗力比见面时下降了一半不止。而这只双头山猪,则是一头独行的百兽王,拥有一只防御蛊。

%92
“这将是一场消耗战,我们正可以捡个便宜。”

%93
“她埋下去的果真是焦雷土豆蛊!”

%94
“有点奇怪。这蛊师激战这么久,真元竟然还没有耗尽?”

%95
“看来她有些辅助蛊,类似天元宝莲,鱼泡蛊等等……”

%96
又看了片刻,方源觉察到时机来临了,唤出隐鳞蛊,交给了白凝冰。

%97
白凝冰会意地点点头。隐去身形,悄悄接近。

%98
轰!

%99
一声爆响,双头山猪再次踩爆了一颗焦雷土豆蛊。

%100
这一次,它彻底倒下了,再也怕不起来。它倒在地上,不断挣扎。整个肚皮都被炸开一个豁口。肠子缓缓流出。鲜血一股股的淌外来。

%101
“这头该死的山猪,皮还真是厚啊。炸了半天,才炸死掉!”陈翠花喘着粗气,靠在树干上,眼前一阵阵发黑。

%102
这场激战令她浑身无力,极度的疲累感袭来。

%103
心中的恐惧强撑着她,不让她就此晕过去。

%104
“不行了。我得赶紧回到山洞去,如果晕倒在野外,实在太危险了!”

%105
她正要动手。忽然耳边传来一阵剧烈的风声。

%106
“奇怪,怎么会有风?”这是她人生最后的疑问。

%107
白凝冰耐心地一步步潜行近身,出手如雷霆电闪。他利用山猪死亡,魔道女蛊师心神放松的破绽,一击必杀!

%108
陈翠花的脑袋,被锯齿金蜈摧枯拉朽。拍个稀烂。无头的身躯扑通一声,倒在地上。

%109
“得手了。”方源朗声一笑,疾步走过来。

%110
白凝冰已经蹲下,心神探入陈翠花的空窍当中,旋即有些失望地道:“空窍中只有三只蛊。”

%111
但凡蛊师炼化的蛊虫,都有些呆板,不似野生蛊虫那般机灵。

%112
魔道女蛊师已经死亡。空窍壁渐渐晦暗,这三只蛊却还停留在里面。

%113
不过这却难不倒方源。

%114
“强取蛊!”他心念一动,轻而易举地将这三只蛊取出来。

%115
分别是:一只饭袋草蛊,一只铁刺荆棘蛊。以及一颗焦雷豆母蛊。

%116
这些当然不是全部。

%117
方源一阵摸索,在魔道女蛊师的脚上发现了一只毛脚蛊,一只跳跳草蛊。最后又在她稀烂的脑浆中,发现了一只书虫。

%118
但可惜的是,这只书虫已经在刚刚,被白凝冰一击拍死了。

%119
“这只书虫应该是传承之物。里面的内容,帮助了她学习了很多蛊师的东西。必然也记载了一些秘方等等。”方源遗憾地道。

%120
“书虫也就算了。这个魔道蛊师身上,居然也没有治疗蛊?”白凝冰大失所望。

%121
方源沉默,没有开口。

%122
没有治疗蛊,这个结果,他先前也预料过。魔道女蛊师受累于蛇毒,若是有治疗蛊,哪怕不是消毒的,也不至于这般狼狈。

%123
他感到疑惑的是,刚刚在战斗中,魔道女蛊师表现得很奇怪。尤其在真元方面,她的恢复速度超过甲等,可以和十绝体媲美。

%124
但她并非是十绝体,只是甲等资质,收刮出来的这些蛊,也没有帮助她恢复真元的功用。

%125
“原来胸是这么裹的呀。”这时,白凝冰拆开了魔道女蛊师的裹胸布。

%126
方源目光一凝,看到尸体胸口处的有一个银边三角纹路。

%127
“竟是三更蛊!”方源诧异。

%128
“三更蛊,什么东西?”白凝冰扬眉问道。

%129
“此蛊高达五转,能令时间加速三倍。用一次就没有了,是消耗类的蛊虫。它作用在蛊师身上,就会在胸口形成这样的银色三角形的纹身。”

%130
随后,方源详细解释了一番。

%131
三更蛊,作用在蛊师身上,能令这个蛊师的个人时间,加速三倍。

%132
光阴之河,滚滚流淌,流速始终如一。对于正常人来讲,一天就是一天。

%133
但对于中了三更蛊的人来讲,一天就相当于三天的浓缩。

%134
被种下三更蛊的陈翠花,修为必然进步神速。别人努力一天温养空窍,她却相当于用了三天来温养。修行效果,自然明显。

%135
还有一个巨大的好处,就是真元的恢复速度。在她身上,光阴之河流速加快,是常人的三倍。因此真元的恢复速度,自然也是三倍。

%136
这也是为什么,在刚刚的战斗中,她真元恢复速度快的叫方源和白凝冰吃惊。

%137
然而,万物平衡,有利就有弊。

%138
三更蛊有如此大利,就有大弊!

%139
首先最大的坏处,就是损耗寿命。光阴流速增快三倍,表现出来,就是生命大幅度缩短,削减到原有的三分之一。

%140
陈翠花得到传承只有一年不到的时间,但对她来讲,却过了两三年。

%141
其次,任何需要时间酝酿的伤势,都会加重。

%142
绿蟒的毒素,需要时间才能慢慢渗透,伤害渐渐加深。别人中了这毒,一天就是一天的效果。但陈翠花中了此毒,一天就是三天累积的效果。

%143
这也是她被蛇毒搞的如此狼狈不堪的主要原因了。

%144
“能被种下三更蛊,看来这个女子接收的传承,至少是个五转传承。可惜明珠暗投,落在此等人手中。”白凝冰冷哼一声,看着脚下的无头尸体,有些不屑。

%145
从长远来讲,三更蛊危害很大,将蛊师的生命硬生生缩短到三分之一。但事实上,它对于魔道蛊师却实用的很。

%146
皆因魔道蛊师,向来单打独斗,没有家族、门派的资源支撑。又要防备正道征剿,首先就得以生存为主。

%147
只有活下来,才是最重要的。其他什么都是虚的。

%148
而修为越高,生存的几率自然就越大。

%149
三更蛊将蛊师的生命浓缩,将一生的灿烂凝聚起来绽放,如同飘零的樱花,易冷的烟火,短暂却又精彩。

%150
但如果没有三更蛊,樱花的幼苗来不及绽放,就已经被铲除了。

%151
“这也不奇怪。用三更蛊的蛊师,各个都是勇猛精进之辈。但这女子,却性子不符。生性胆怯,喜好拖延,没有一往无前的勇气。落到今天这个下场,也是应该。”方源撇了地上的尸体一眼,便收回目光。

%152
白凝冰的脸色,却有些难看。

%153
没有治疗蛊,是什么下场。眼前的这个魔道女蛊师,就是最好的例子。

%154
这是否也是他未来的写照呢?

%155
方源却笑笑,拍拍她的肩膀,安慰道:“这世间的路,都是人走出来的。但各人有各路,别人的路纵然再宽广,也未必适合自己。你我皆走在自己的路上,有什么好担忧的呢?”

%156
白凝冰闻言,神情一愣,渐渐飞扬。继而点点头道:“你说的是。”

%157
方源眯着眼睛,遥望眼前的茫茫山林。

%158
杀了这魔道蛊师,得了数只蛊虫,令他实力暴涨。

%159
但他也知道自己缺治疗蛊,同时并不把希望记挂在自身的运气上面。

%160
接下来,他也许会获得一只治疗蛊,也许会如这魔道女蛊师一般,直至死亡,都一无所获,望眼欲穿。也许下一刻,就被兽群践踏陨落,一生野望转头空。

%161
但这又有什么关系呢?

%162
路在脚下,继续前行即可。

\end{this_body}

