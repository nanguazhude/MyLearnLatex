\newsection{大电文狗}    %第一百五十节:大电文狗

\begin{this_body}

第十三轮……

第十四轮……

第十七轮……

第十八轮……

方源一关关地闯下去,手中的犬兽数量不断增加。到了第十九轮结束,他积累到了八十多头犬兽。

其中有四十多头菊花秋田犬,二十多头电文犬,十九头刺猬犬。

在这期间,他又斩杀一人。是位水道三转巅峰的蛊师,得了六只蛊。

不过,却始终没有碰到白凝冰。

“第二十轮了。”方源一直在心中算计着。

三王传承,每过十轮,难度上都会暴涨数倍。

第二十轮开始,就会出现百兽王,规模上百、数百甚至近千的犬兽大战。

迷雾中再次出现三团光影,分别处在左、右、前方。

前边这团光影,一片橘黄色,好似栲栳大小。

左边的光影,则是一片幽蓝的电芒,不断地在闪烁着,有磨盘大小。

右边的光影,却是水中月,雾中花,似有似无的青白样子。三团光影中,就属这团光影形体最小。

方源清楚得很,光影的大小暗喻犬兽的数量,光影越大,说明犬兽的数量就越多。

前方的橘黄光影,代表着有菊花秋田犬,有两百多只。左边的幽蓝光影,说明有一百五十只左右的电文犬。右边的青白光影,则表示有阴犬群,数量最少,只有一百出头。

方源首先就排除了菊花秋田犬。

菊花秋田犬群,一旦有犬王,更加团结。数量越大,战力越强。两百多只菊花秋田犬,对于方源来讲。完全是个灾难。

其次他又排除了阴犬群。

阴犬很特殊,它没有实实在在的躯体,像是一团犬形阴气,漂飞在半空中。寻常的攻击根本杀不了它们,它们还会穿透山石、遁地、藏水等等。

方源手中的犬兽。没有这样的手段。电文犬也只是代表速度快,没有外放雷霆电力的才能。

略微思索了一番,方源选择了左面方向。

迷雾消散,狗群在他身边环绕,默默随行。一只二转的驭犬蛊从天而降,落在他的手中。

当他彻底走出迷雾时。电文狗群正散漫在附近。

山丘上,一只体格庞大的犬兽,正趴在草地上眯眼小憩。

它比寻常的电文狗要大上两倍有余,浑身上下是深蓝色的皮毛,毛尖锐利,闪烁着微微的电芒。

这是百兽王——大电文狗!

“天赐良机!”方源洞悉整个战局。双眸中精芒爆闪了一下,瞬间做出抉择。

他单手一挥,悍然压上全部的兵力。

八十多头犬兽,将他包裹在中间,向大电文狗冲去。

大电文狗的反应也很灵敏,双耳颤动了一下,立即睁开双眼。闪电般站起来。

它仰头大啸,召集山丘附近的电文狗群。

得到王的召唤,周围的电文狗群顿时纷纷响应,从四面八方向狗王身边汇集而来。

电文狗的速度虽然很快,但方源占据了先机。

他将手中所有的力量,都集中在了一起,毫不犹豫地直捣黄龙!

一些靠的近的电文狗,比方源更早一步,集合到狗王的身边。

“冲!”方源眼中闪过绝然的光,这样的时刻最忌讳犹豫。既然下定了决心。他就一往无前。

上来阻挡的电文狗,很快就被方源冲散。

方源的阵势密集,而这群电文狗却是阵势松散,集齐过来的数量稀少。

“汪!”

作为百兽王的大电文狗,看到手下被屠戮一空。立即被激起凶性,四爪奔腾,向方源杀来。

方源心念一动,二十多头电文犬分出两队,从左右包抄过去。十九头刺猬犬,排成一个阵型,悍不畏死地冲向大电文狗。而剩下的大部队——四十多头菊花秋田犬,则包裹着方源,稍稍散漫开来,跟在刺猬犬的队伍后面推进。

大电文狗一头扎进方源布置下来的包围圈中,被困在中央。

方源不顾心力的剧烈消耗,尽全力调动麾下犬兽,施展出精妙的配合。

大电文狗速度比电文狗更加惊人,但在方源有意的包围中,它的最大优势受到了遏制。

它仰头狂啸,企图召集自己的麾下。

它的属下,正疯狂地朝这边用来,前来救驾。

方源一面要围困住大电文狗,一面又要抵挡外界的狗群攻潮,压力巨大,不多时脑袋就隐隐作痛,一层汗渍密布额头。

局势有些危险。

若换做正面硬碰,就算是得胜,也是惨胜,手中的兵力会所剩无几。因此方源打的是擒贼先擒王的主意。

但若控制不住这头百兽王,那么方源就要被包饺子。不单失败,无望进入下一轮,甚至还会命丧当场。

在这个蛊仙福地中,除去传承给予的蛊虫,其余的蛊都不能调动。因此,蛊师极容易伤亡。

方源身上的力气蛊、全力以赴蛊也都不能调动。

当然,他还有最大的王牌——春秋蝉。依现在春秋蝉的状态,倒是可以勉强再催动了。但催动春秋蝉本身,就有巨大风险。极有可能白白自爆而亡。所以方源不到山穷水尽,万不得已的情况,是绝对不会胡乱动用的。

额头上的汗渍汇集成汗珠,从上而下地滚落下来。

方源顾不得擦拭,手中紧紧地扣着二转驭犬蛊,一直都没有出手。

他在慎重地寻找时机。

用一转驭犬蛊,收服普通犬兽,那是随心所欲。但用二转驭犬蛊来收服百兽王,就有失败的可能性。

因为百兽王的身上,寄生着天然蛊虫。这些蛊,很有可能会坏了方源的好事。

方源的机会只有一次,若是二转驭犬蛊被摧毁。他就完蛋了。因此,不得不谨慎!

“汪!”

大电文狗被逼急了,忽然张开大口,吐出一团蓝色的电浆。

深蓝色的电浆,好像是黏液一般。扑的一声,兜头洒下。砸在草地上,却不消散,仍旧在闪烁不定,将草地山石都打得噼啪作响,很快就焦黑一片。

“这是二转的电浆蛊!”方源立即认出了制造出这股电浆的罪魁祸首。

电浆蛊的攻击力。并不强大。

虽然攻击很突然,罩住了方源手下不少的菊花秋田犬,但方源立即在第一时间,就将处在这片电浆中的犬兽,都抽调出来。

这些犬兽浑身皮毛,都被电得焦黑。同时速度也不灵活。被电流麻痹。但仍旧保持了一定的战斗力。

“汪、汪、汪。”

大电文狗连连喷吐,电浆覆盖一片又一片的草地。

方源的眉头,深深的皱起来,脸色更加凝重。

电浆蛊的攻击力并不强,但覆盖在地面,会持续一段时间。在这段时间内,这片地域就成了方源麾下犬兽的活动禁区。

方源的部队。本身就被包裹在中央,活动的范围并不大。

电浆一覆盖,极大的加深了这个弊端。这给方源调度队伍,轮换防线,分摊伤害的行为,造成了极大的阻碍。

无奈之下,方源只好转换队伍。

他将原本布防在外线的电文狗,调到内部,来对付百兽王。将大部队菊花秋田犬,调出去。抵挡外部的一百多只的电文狗大军。

电文狗在电浆覆盖的范围内,活动自如,甚至身体受到电流的刺激,速度变得更快一分。

但整个场面,却对方源更加不利。

电文狗面对自身种族中的百兽王大电文狗。都不免生出畏惧之心。这让方源操纵,更加耗费心力,同时指挥效果还打折扣。

而外界,菊花秋田犬的速度,并不如电文狗,只能结合成密集阵势,进行被动合作防守。

若换做电文狗,方源还能通过速度,进行游击牵制。

“坚持,必须坚持住!百兽王的第一只蛊,已经被探测出来了。接下来还要再接再励……”方源纵然身处险境,但心中仍旧冰雪般冷静。

电浆蛊每次催动,都要间隔五息的时间。电浆蛊本身,也需要休息。

在这个五息的时间内,方源可以催发出二转驭犬蛊,而不用担心受到电浆蛊的攻击。

但是,方源并不知道,这头百兽王的身上,是不是还有其他的蛊。

为了保险起见,他必须要将试探持续下去。

时间一点一滴地流逝,场面对方源而言,变得越来越艰难。

他手中的电文狗,损失了十四头,只剩下九只。刺猬犬因为不和大电文狗同属一族,倒是损失的较少,不过也只剩下十一头。

菊花秋田犬的损失最为惨重,从接近五十的数量,只剩下二十头不到。

但方源却迟迟没有试探出,眼前这只百兽王身上的第二只蛊。

方源决定出手!

“不能再等了。这头大电文狗的身上,很可能只有一只蛊寄生着。”

他必须冒险。

再等下去,他岌岌可危的外部防线,就要被彻底攻破了。

二转驭犬蛊!

趁着大电文狗发出一记电浆的时候,方源迅速出手,打出关键一击。

在方源紧紧的注视下,驭犬蛊顺利地飞到大电文狗的身上,种在它的魂魄当中。

一股精神上的无形冲击,通过这只驭犬蛊,向方源袭来。

方源脑袋本来就隐隐作痛,受了这击,顿时一阵头晕目眩。

这是百兽王的魂魄,不甘情愿的反抗。

收服百兽王,并不容易。

方源身躯晃动,咬牙坚持下来。

一股心灵的联系,沟通着方源和大电文狗。后者刚刚还在暴动,恨不得将方源碎尸万段,但此刻却汪的一声,讨好地向方源摇动大尾巴。

围攻的电文狗群,听到狗王的叫声,顿时停住冲击。

刚刚还胶着惨烈的战场,一下子安静下来。

方源站在原地,缓缓转动头颅,扫视战场。

这场冒险是值得的!

他虽然有不少的损失,但是看看场上——至少有一百二十头的电文狗。

不要忘了,还有一头寄生着电浆蛊的百兽王。

方源实力大增!(未完待续。如果您喜欢这部作品,欢迎您来起点投推荐票、月票,您的支持,就是我最大的动力。手机用户请到阅读。)

\end{this_body}

