\newsection{魔无天}    %第一百九十一节:魔无天

\begin{this_body}

三叉山峰巅,光柱只剩下一道。

但一片灰黑平原,如虚影飘烟,展现在众人的面前。好像是隔了一层纱,如梦似幻一样。

山峰上怎么会出现平原?

这是萧芒一拳洞穿福地,将其打出漏洞所致。

一时间,三叉山上寂然无声,蛊师们有的呆滞,有的震惊,有的面面相觑,都没有动弹。

萧芒心中冷哼一声,悄悄地向人群中某处打了个眼色。

人群中顿时跑出一名蛊师:“萧芒大人神威无敌,将福地硬生生打穿了。这样一来,我们就再也不受三王传承的限制,都可以进去了!”

喊完这句,他越众而出,几下蹦跳,就穿过漏洞,进入了福地当中。

他又当着众人的面,催动蛊虫。

一只,两只,三只……在这漏洞附近,天地压制薄弱,让他能催动三只蛊虫。

这一幕,顿时引发人群的躁动,无数人惊醒过来,鼻息转粗。

能够动用蛊虫,就代表着有自保的能力。也就意味着,闯荡福地的风险大降。

但表演还没有结束,这名蛊师忽然又往回跑,几下蹦跳,又跳出福地,成功地回到三叉山上来。

“哈哈哈!”他大笑三声,向萧芒一拱拳,“谢大人通融!”

萧芒呵呵一笑:“不用谢,不用谢。我只是觉得,天地宝物,人人都应有竞争的权利。只是有限的几个人独吞,那就太过了。但是,接下来能抢得多少,就靠诸位自己了。”

“萧芒大人恩义无双啊!”

“萧芒大人是多么好的人呐,他的哥哥就是萧山萧大侠。”

“壮栽,真是龙兄虎弟也!”

“纵观三叉山上几大五转蛊师,也就是萧芒大人,能为我等这些弱小着想,真是正道楷模。多么仁慈啊……”

众人欢呼声不断,马屁奉承的话,如潮水般传入萧芒的耳朵里。

人潮汹涌,大量的蛊师都冲入福地。

“一群白痴。”萧芒满面春风,面带温暖的笑容,内心中却是不屑的冷嗤。

“相比较铁慕白、巫鬼这些人,我来到三叉山的时间。还是太晚了。犬王、信王的传承,都被人捷足先登,只剩下爆王传承!真是该死!我早就想父亲还有各大家老请命,但这些人就只关心萧山的病情!哼!他死了不是更好,这样我就能成为少族长了……”

“唉,我软磨硬泡。这才得到允许,来到三叉山,可是已经晚了!先前那些人的积累,我怎么比得上?唯有轰破福地,形成漏洞,造成混乱局面,我才能从中获利啊!”

“对于这片福地来讲。进去的人越多,负担就越重。呵呵呵,天地压制越来越弱,就算你们获得了传承又怎样?我完全可以动用蛊虫,进行抢夺!爆王、犬王、信王的传承,都是我的。就算我得不到,你们也别想得到!”

……

“好了,大体就是这样了。”方源看着周围。满意地点点头。

此处青铜大殿,居于山丘之上。山丘并非险峰,而是向周围蔓延成缓坡。

这样的地形,并不容易防守,但好在犬兽众多,靠着数量,稍稍能够弥补一些。

在过去的一个多时辰。方源一直在安排白凝冰布阵,并交代她出现何种情况,又如何应对。

轰……

就在这时,整个天地微微一颤。青铜大殿抖下簌簌灰尘。

“不好了,那萧芒动用太光蛊,将福地中的一块击穿,形成通道。大量的蛊师,汹涌进来,相互厮杀,不断争抢,场面一片混乱!”地灵传音道。

方源呵呵一笑,却不惊惶。

前世记忆中,也有此一幕。萧芒的到来,就喻示着此事发生。

“乱的好,他萧芒要乱中取胜,我也要这混乱局面,帮助我拖延时间。”方源一双黑眸幽幽闪光。

“咦?这群人中,居然还混进一个五转蛊师……这年轻人好生厉害,他之前居然屏蔽了我的感应。直到他动手,我才发现了不妥之处!”地灵忽然又道。

方源眉头一皱,这情形超出意料:“什么人?”

他面前影像顿现,只见一位青年男子,一头黑发及腰,双眼重瞳,瞳色深紫。一对黑眉粗重浓厚,眉末高高上挑,分出几岔,张扬如狂,恰似如火焰燃烧。

他魔气凛然,有不可一世,毁天灭地般的恣意霸道,宛若孽龙降世。

“魔无天!”方源目光一凝,认出他来。

此子乃是魔道天才,继承了上古传承,乃魂道蛊师。不管是威名还是实力,方源还不能和其相比。

记忆中,义天山之战,魔无天更是力斩数位正道五转大蛊师,凶名赫赫,魔焰滔天。最后魔道一方溃败,魔无天打破包围圈,扬长而走,无人可阻之。

“前世,魔无天可没有来到三叉山!看来我重生带来的影响,已经波及到这种层次的人物了吗?”

正当方源沉思的时候,画面中魔无天似乎察觉到被人注视,微微侧身转头,居然面向方源而视。

“原来是这个方向……”他开口,轻声喃喃,嘴角勾勒出一丝阴沉诡秘的笑。

“不妙,他似乎察觉到了什么,正在向大殿这里冲刺!”地灵及时发出警讯。

方源双眼眯成一条缝,魔无天居然目标直指自己,带着强烈的敌意。他抱有什么企图,又到底发现了什么?

“混乱只会持续一段时间,铁慕白等人迟迟不出现,就会引起有心人的怀疑。时间不多了,必须即刻炼蛊!地灵,升起迷雾。风天语,你随我进大殿,辅佐我炼蛊!”

时间紧迫,方源呼喝一声,带领着风天语,双双进入青铜大殿。

至于,后者带来的毛民,则留在外面,拱卫大殿,形成最后一道防御。

望着方源和风天语离去的背影,白凝冰眼中冷芒一闪。

迷雾开始升腾,迅速弥漫,很快就遮蔽了这处大殿,并将山丘上的犬兽尽数掩盖。

……

青铜大殿,宽阔雄伟,回荡着方源和风天语两人的脚步声,更显得此处的幽静、空阔。

此时大殿上的青铜砖面上,已经一片空荡――绝大多数的材料、蛊虫,都已经在炼蛊时消耗光了,只剩下几个浮雕。

方源走到铜鼎面前,和风天语一起,盘坐下来。

“这是最后一步,真正的关键时刻!”他深呼吸一口气,眼中清光如水。

风天语则鼻息粗壮,表现得相当兴奋。对于一位炼道蛊师来讲,能炼制仙蛊,是平生最向往之事。

“开始罢。”方源取出第二空窍伪蛊,直接抛入到铜鼎当中。

铜鼎无火自燃,底部只剩下薄薄一层的仙元,在此刻急剧消耗,悍然燃烧!

燃烧成的青气如烟,袅袅娜娜地升腾起来,包裹住第二空窍的伪蛊。

伪蛊悬浮在铜鼎上空,被这清气一化,就形成漫空的黄光。

方源和风天语一齐灌注心神,竭力调和青烟和黄光。

也不知过了多久,青烟化为颗颗青草,悬空生长。而黄光化为花朵,飘零而落,点缀其中。

“时候到了!”方源取出匕首,割破动脉,喷出自身精血。

此步必不可少,只有过了此步,炼制出来的第二空窍蛊,才是方源所有。否则就是无主之物,一旦炼成,就会凭空飞去。

大股的精血,浇灌过去。青烟黄光顿时嗤嗤作响,化为一片赤色云烟,犹如血海狂涛。

血水翻腾,却只团成圆球,悬浮于空,并不扩散。

云烟不断演化:血海渐渐平静下来,凝固成田,一大片的赤稻,猩红如血,长在田地上。

方源看到此处,吐出一口浊气,连忙动用蛊虫,治疗了伤口。

饶是如此,他也失血过多,脸色苍白一片。

“野草芳华,血气如海。三百岁为春,五百岁成秋。神机无限,扩游四野,添三更,再三更,三更得九。九为极,大功告成!”

他早就将秘方背得滚瓜烂熟,但此刻仍旧回想了一遍。

“三百岁为春,五百岁成秋……接下来,就是动用寿蛊了!霸龟!”方源猛地大喝。

地灵早就严阵以待,听得方源召唤,连忙调出两只寿蛊。

这寿蛊一大一小,仿佛是参须,犹如老树根,摸在手中,一片粗糙沧桑。

小的那枚,是三百年寿蛊,如青蛇盘成一圈,可增蛊师三百年阳寿,无有任何遗毒。大的那只,则如虬龙飞天,张牙舞爪,可涨五百年寿命,同样没有副作用。

两蛊的价值不言而喻,风天语看到此处,双眼冒光,浑身都在颤抖。

方源先将三百岁寿蛊,抛入到云烟当中。

云烟吞掉寿蛊,顿时如滚水般翻腾。

这一刻,云烟仿佛成了一头青鳞长蛇,滑不留手,企图从方源的掌控中逃窜!

方源骇得一跳,措手不及,差点就被这青蛇逃窜。

他反应过来时,好似这青蛇已经大部分从他手中溜走,只剩下尾巴一截还在。

方源紧紧咬牙,双眼瞪圆,一片赤红!

他全部心神毫无保留地灌注进去,拼尽全力,死死地掌握云烟,不令其脱离控制。

一旦云烟脱离,他就功亏一篑,先前一切的努力都会化为泡影!(未完待续。如果您喜欢这部作品,欢迎您来起点投推荐票、月票,您的支持,就是我最大的动力。手机用户请到阅读。)

------------

\end{this_body}

