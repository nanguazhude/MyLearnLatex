\newsection{换骨剧痛(第五更)}    %第三十三节:换骨剧痛(第五更)

\begin{this_body}

“大娘啊,什么好事啊?”方源立即憨憨地问道。(138看书,小说更快更好.138看书.)

正问着,从外面走来一个老汉。

白凝冰不着痕迹地眯了眯眼,这老汉是个蛊师!

不过,只有一转初阶的修为,老态龙钟,毫不足虑。

“小伙子,这位就是我们村的村长。”老婆婆介绍道。

方源和白凝冰连忙站起来。

方源挠挠头,显得有些局促:“村长老大人,俺是……”

他还未介绍完,村长就笑着摆手道:“我知道你们,两个外乡人。”

蛊师之间可以感受气息。但方白二人有敛息蛊在耳朵里面,蛊师气息收敛起来,至少村长是看不破的。

这两个外乡人,老村长找在几天前,就得到过村民的汇报。但他并不放在心上。

这些天来,除去方白二人外,前前后后也有不少人在这村里借宿过。

“来就来了,还带什么东西呢。”老婆婆抱怨了一声。

老村长没有空手而来,手中提着一串鱼。

“这是今天早上在水塘钓来的。你牙齿不好,喝喝鱼汤,能补身子。”老村长笑起来。

老婆婆白了他一眼,带着接过鱼来:“我到厨房做鱼汤。”

声音透着喜悦。

两人的神态、语气,白凝冰还看不出来。但方源目光闪了闪,心中了然――这老汉和老婆婆怕是黄昏恋了。

“大娘,还是俺来吧。”方源立即道。

“你们坐,你们坐,坐下谈。有好事咧!”老婆婆忙不迭的摆手,示意方白二人坐下,“你们的事大娘我都和村子说了,我们村长可是蛊师呢,他能帮你们。”

“蛊师!”方源顿时将眼睛瞪得大大的,显然惊呆了,楞在原地。

白凝冰看到他这副神情,差点忍不住翻白眼。她也努力做出吃惊的神情,但显然和方源差远了,这点她也心知肚明。

老村长看到方源这样的表情,哈哈大笑,一下子觉得这个丑汉憨得有些可爱。至于身旁的白凝冰,则显得有些呆木。比较起来,老村长反而更喜欢方源。

“两个后生仔,都坐吧,不要拘谨。”他招招手,首先坐下来。

方源扭捏了一下,这才哼哧哼哧地坐下来,神情紧张。白凝冰也跟着他后面学,但总显得有些不自然。

老村长却也没怀疑什么:“我听说了,你们是想来贩药草和腌肉的,结果撞见了野兽,运气不好啊。你们大娘都和我说了,你们在这里帮了大娘这么多忙,我这里呢,有些紫枫叶。过些日子,有支商队来。你们拿去这些紫枫叶卖了,兴许能回点本钱。”

白凝冰一听,原来这就是老婆婆所说的好事,顿时失去了兴趣。

“这,这,这……”方源高兴得结巴起来,然后双眼流泪,哽咽了,“村庄老大人,您可真是好人,大大的好人!”

老村长抬手拍拍这个魔头的肩膀,同情地勉力道:“这些话,用不着多说。谁能没有个落魄的时候。不过这东西也不能白送给你们,不然村民们会有看法的。这样,你们白天到村东头来锄田,干个七天,商队也就来了。”

他倒是挺感激方白二人的。

他是这个村土生土长的凡人,从小和老婆婆青梅竹马。但世事无常,老婆婆嫁给了别人。在一次商队买卖时,有蛊师看他机灵,就点拨了他,使其成了蛊师。

因为此事,他就成了村长。虽然对老婆婆仍有情愫,但双方都是有儿有女的老人,影响不好,就没有过于来往。他虽然身为村子,也不能太明目张胆的帮衬老婆婆。

他其实暗中考察过,也觉得方白二人心性不错,勤劳肯干,都是老实孩子,就是运气背了点。

因此老婆婆跟他说了,他就立即答应帮助方源。

……

“你在吃饭的时候,答应老村长。难道真的要去锄地七天,要那车什么紫枫叶?”夜晚屋中,白凝冰不解地问道。

“紫枫叶当然不是我的真正目的。你刚刚没有听到么,有商队要经过这里。”方源答道。

“那又如何?你不是也提到过,紫幽山常年都有商队路过吗?”

“商队有时候一年一次,有时候半年一次。我只是没有想到,居然在近日就有商队。我刚刚旁敲侧击过了,这只商队是东西向的。即便不到商家城,也至少和我们顺路。”

白凝冰恍然:“你是想加入商队,借他们的力量,到达商家城?”

她略微思量,越发觉得这是一个好方法。

首先,地听肉耳草损毁了,他们缺乏侦察蛊虫,自己行动恐怕会很麻烦。

其次,他们犯了案子,百家必定追捕,绝不甘休。融入商队,是一个很好的掩盖行迹的好法子。

最后,他们有敛息蛊,完全可以伪装成凡人。商队对外来的蛊师很有戒备,但是对凡人却没有那么强的警惕性。

就算是被发现,也没有什么大不了的,到那时一人三转一人二转,脱身的能力还是有的。

商队的首领,一般都是三转修为。类似贾富这等四转的头领,还是比较少见的。

“不过就算我们伪装成凡人,要想随随便便加入商队,恐怕也不成吧?”白凝冰想了想,心中仍有顾虑。

方源嘿嘿笑了声:“当然不可能随随便便加入商队的,就算是凡人,也需要担保人。不过我想老村长会替我们解决这个问题的。”

白凝冰这才释怀,把担忧放下。

“也是,我是白担心了。凭方源这个狡诈的家伙,怎么可能漏算这点?”她心道。

“好了。七天之后,我必能突破到二转初阶。不过除此之外,我想是时候用了铁骨蛊、玉骨蛊了。”

白凝冰撇了撇嘴,有些不以为意:“我先前就说过要用了,你却一直不同意,白白养了这么多天,浪费了许多奶泉水。”

方源叹了一口气:“少年,你是无知无畏,不知道这其中的痛楚。今晚我们就不双修了,好好养足精神,明晚我先用了那铁骨蛊。”

翌日,方白二人依约去了村东头的田间劳作。

方源特意从早上,一直劳作到太阳落山。他巨力暗藏于身,这点运动量根本不值一提,但已经叫其他农人刮目相看,甚至有些仰望。

到了晚上,他盘坐在床榻上,沉入心神,将真元灌注到空窍中的铁骨蛊上。

此蛊如一根骨头,两端圆头,中段细长。通体乌黑,如同钢铁所制。

这蛊还是三转蛊,催用时要求瞬间消耗大量真元。

方源区区一转,本来还不能用。但白凝冰灌入了许多白银真元,这才让他有了勉强使用的资格。

借来的雪银真元几乎消耗一空,尽皆投入到铁骨蛊中。此蛊先是大放乌光,紧接着就融化成一股铁汁,从空窍中飞出来,融入方源的骨骼。

痛!

剧痛!

难以想象的剧痛!

方源仿佛被滚烫的烙铁直接印在心窝之上,铁汁流到哪处骨骼,哪处骨骼就仿佛是放在火炭上烧烤。

这种痛,痛彻心扉,让方源的脸色都扭曲了。

紧接着,方源额头开始冒出一涔涔的冷汗,不到片刻,浑身都被汗渍打湿。

半晌功夫后,方源终究坚持不住,发出哼哼的声音。

白凝冰这才变色。

黑暗中,她看不清方源之前的神情和状况,但听这可以压抑的声音,就判断出此痛非同小可!

她知道方源的意志如何的强悍刚硬,甚至浑身都被火焰笼罩,都一声不吭。

此时用了铁骨蛊后,却发出这样的声音,可见痛楚何等剧烈。

蛊师养用炼三大方面,在用上,有些蛊的使用体验极其特殊。铁骨蛊便是如此,一经使用,就是极限剧痛。偏偏这种痛楚只能忍受住,一旦在中途晕厥过去,那就前功尽弃了。

历史上,有许多知名的蛊师,因为用了铁骨蛊之类的蛊虫,而硬生生的疼死掉。

方源咬牙坚持,实在忍耐不住剧痛如潮水般的冲刷,就从鼻腔中哼哼几声。

到最后,他痛得几乎全身麻木,牙齿也咬虚了。

当全身的骨骼都被铁汁水,染成一片黑色之后。他终于放松下来,几乎与此同时,一阵强烈的眩晕感就袭击过来。让他双眼一黑,几乎要就此昏死过去。

但他强撑着,缓了几口气,慢慢躺下来。

“结束了?”黑暗中,传来白凝冰试探的声音。

“当然。”方源嘶了一口冷气,声音沙哑但平静地道,“睡吧,明天还要干活呢。”

“嗯。”白凝冰见方源神智还清醒得很,心中其实大为失望。如果他昏迷过去,说不定就能弄到阳蛊了。

“不过也不对,阳蛊存在方源的空窍当中。我没有特殊的手段,怎么能取出来?”想到这里,她一些阴暗的心思也就散了。

可到了第二天,方源却没有起来干活,而是赖在了床上。

剧痛的余韵,还折磨着他。只要稍稍一动弹,强烈的痛楚就仿佛是一把钢锯,在他的神经上狠狠锯搓!

痛楚暂时主宰了他,令他虚弱无力。不说农活,连最普通的下地走路都不成了。

白凝冰这才意识到,昨晚方源是强撑门面。

这一天的农活,是她一个人干的。

当晚,方源恢复到能活动的状态了,白凝冰不信邪,也用了玉骨蛊。

她这才知道,方源承受的何等剧痛!

骄傲如她,也不禁痛得呻吟起来,甚至抓破了床单。

但最终,她浑身颤抖地坚持到了最后。意识到自己的成功后,在强烈的眩晕的袭击下,她躺倒在床榻上,彻底昏睡过去。

(ps:累了,不过好久没爆发了,爆发一下也挺爽的。呵呵,明天上午有事情,只能14点更了。)(未完待续。如果您喜欢这部作品,欢迎您来(本站)订阅,打赏,您的支持,就是我最大的动力。)

------------

\end{this_body}

