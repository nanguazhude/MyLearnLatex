\newsection{卖不出去}    %第六十三节:卖不出去

\begin{this_body}

三百一十?

看到老者比出的数目,方源微微挑起眉头。

酒虫市价五百八十块元石,书虫有时候略高一点,达到六百。黑白豕蛊也是六百块元石。但以上这些蛊,皆是一转的珍稀蛊,因为数量稀少,所以价格昂贵。13800100.

普通的一转蛊,大约在二百五十块元石左右。

而生机叶这种消耗类的一转蛊,则是五十块元石一片。

将骨枪蛊的卖价,定为三百一十,已经算是不错。通幽商铺的这个老者,没有胡乱压价。

不过即便如此,方源也要尽量地把价格提上去。

讨价还价对他来讲,在前世就已经练得炉火纯青。

两三句话后,老者无法,只得将价格再提高十枚元石。

一枚骨枪蛊,三百二十块元石。

“好,就按照这个价格成交罢。”方源一扬手,顿时从空窍中飞出一蓬的光点。

五十六只骨枪蛊,悬浮在老者的面前,将老人家吓了一小跳。

“这么多……”他顿时有些后悔。每只提高十枚元石,那就得多付近六百块元石出去。

事实上,方源在白骨山时,足足取了两百只骨枪蛊。

但一路行商过来,食料不足,已经死了大半,只剩下这么多了。

“五十六只蛊,那就是一万七千九百二十枚元石。在下这就命人取元石过来。”老者将骨枪蛊都收入怀中。

“不忙,且再看这只蛊。”方源微微一笑,唤出一只螺旋骨枪蛊。

“这是二转蛊,似乎是骨枪蛊的……”老者脸色浮现出惊异的神色。

“不错,骨枪蛊合炼成功,就可得到螺旋骨枪蛊。它具有一股钻劲。威力较为可观。”方源适时地介绍道。

老者试验了一番,的确如方源所说,报了个七百八十块的价格。

几轮交锋后,方源将价格提到八百块一枚。

这些螺旋骨枪蛊,他保养得很好,空窍中有二十只。

这样一来,就卖出了一万六千块元石的价。

“再看这只蛊,七千块元石。”方源报了个价,取出骨刺蛊。接着介绍一番。

老者捏着骨刺蛊,却不敢胡乱试验,,苦笑道:“此蛊是三转蛊,但伤敌一千自损八百。骨刺穿透皮肉,必然疼痛万分。要运用它,还得再搭配治疗蛊。七千这个数太高了,六千五百块元石,这个数字恰恰好……”

“我们就别讨价还价了,我让多点,就六千七百块罢。”方源道。

刚刚两轮交锋。让他老者察觉到方源之棘手,他擦了擦头上的汗渍,索性一咬牙:“成交。”

“那就是四万零六百二十块元石。”方源眼珠子一转,报出了准确的数字。

老者忽然弯腰。对方源鞠躬行了一礼:“贵客,您要卖的这些蛊虫,在下执掌商铺这么多年,从未见过。且它们之间又相互关联。层层递进,似乎是一脉相承。请问贵客。这些蛊是否属于同一传承。”

方源点头:“明眼人都能看得出,不错,这是我机缘巧合,继承了一个传承,得了这些蛊虫。”

老者面色骤喜:“既是如此,那贵客手中必定有相应的炼蛊秘方。贵客的好运道实在羡煞旁人,不如将这些秘方也一并卖给我铺如何?”

方源皱了皱眉头。

物以稀为贵,骨枪蛊、螺旋骨枪蛊、骨刺蛊,不仅他有,百家也有。卖也就卖了。

但完整的合炼秘方,他是得自于肉囊秘阁,天下独此一份。这种东西,却不能轻易出手。

“只要是六转以下的蛊虫和秘方,都是有价之物。不过,你能出多少元石呢?”方源想了想,问道。

只要价格合适,这秘方也不是不可以卖。

他需要钱。

卖了骨枪蛊等等,得到的四万块元石,对他的计划来讲,远远不够用。

老者竖起两个手指:“二十万元石!”

方源卖了那么多蛊虫,不过四万。无形的秘方,却能卖到二十万。

授人鱼不如授人以渔。

捕鱼的法门,远比鱼本身要有价值得多。

因为能捕鱼,就代表着鱼源源不绝。

对于通幽商铺来讲,有了秘方,就代表骨枪蛊等等,能不断地制造出来。

那就变成一个长期的贸易点,独此一家别无分店,因此白骨秘方的卖价自然要高得多。

但方源却冷笑一声:“二十万,这个价也亏你报的出?”

老者老脸一红,这个价格的确是低了。他旋即又道:“三十万!”

方源摇头不语,作势欲走。

老者狠狠咬牙:“五十万!”

“这价钱还像点样子。六十八万卖给你。”方源悠然地嘬了口茶。

老者满脸苦涩之意:“五十万已经是我的最高权力,贵客,您卖了这么多的骨枪蛊,其实我们也可以雇佣秘方大师,来倒推您的秘方。能卖到五十万,真的已经不错了。”

方源摇头,态度坚决:“我手中的这份秘方,是独一份,天底下独一份!最少六十五万,否则不卖。这商家城里的商铺,也不是只有你一家,不是吗?”

“贵客这就有所不知了,商家城的商铺虽不止一家,但却都受我家少主管辖。贵客您在我这里卖不出去,在其他地方也必定卖不掉的。如果你不卖秘方,恐怕这些蛊虫也卖不掉了呢。”老者拱手一礼,话语软中带硬,蕴藏威胁之意。

“哦?那我倒要试试看呢。”方源收回所有蛊虫,起身便走。

“贵客,我好言相劝,您还留下来,卖了吧。”老者做最后的挽留。

方源不理睬他,直接迈出房门。白凝冰也只好紧随其后。

“贵客,您一定还会到我这里来的。”老者冷笑着。目送方白二人离开通幽商铺。

方白二人刚走,老者便来到密室,催动真元,射出一蛊。

这蛊化为一道光,投入到密室墙上的铜镜中。

铜镜表面泛起一阵涟漪,下一刻显现出一个年轻人的面庞。

“属下见过少主。”看到这个年轻蛊师,老者连忙跪倒在地。

“你有何事?”这年轻人便是商家少主之一,名为商锱铢。他年仅十八,正是青春年华。但他长期耽于酒色,面庞消瘦,肤色苍白,双眼无神。

老者便把方源的事情说了。

商锱铢的眼中,顿时闪过一抹亢奋阴鸠的光。

他大叫道:“好极了。真是天无绝人之路。我正愁着如何保住少族长的位置,结果上天就送来了这份大礼。必须把这个传承拿下,有了这个大业绩,今年的考评我就能撑过去了!”

“属下必定竭尽全力,只是属下不过负责通幽商铺,其他的铺子……”

“我会安排的,哼。这两个人想要卖出蛊虫,只有向我低头!”商锱铢不屑地冷哼一声。

八宝商铺……

“对不起二位,上面关照下来,二位要卖蛊。请往通幽商铺。”

元芳楼……

“原来是两位贵客,贵客若是能卖秘方,一切都好商量的。”

不倒阁……

“二位是贵客,小女子做生意。又岂会将客人赶走。但实在是无奈呀。”

连续走了三个商铺,方源都没有将蛊虫卖出去。

“哈哈。想不到你也有吃瘪的时候。看来那个老东西说的没错,这的确是人家的地盘。”白凝冰毫不吝啬对方源的打击。

方源屡次遭拒,面色却仍旧平淡:“商家族长商燕飞子女众多,但商家少族长之位只有一个,少主之位只有十个。每年商家族长都会进行考评,选出少族长,剔除十位少主中成绩最差的一个,空出来的位置由其他子女补上。”

白凝冰脑海中顿时灵光一现:“原来如此,这么说这个背后出手的商家少主,不是竞争少族长之位,就是保少主之位。否则绝不会如此大动干戈。”

方源抚掌笑道:“这是商家第三内城,不能动武。就算是百家的人来追捕我们,也不能在此处动手。对于商家少主来讲,限制更加巨大。考评在即,一举一动都有其他人虎视眈眈。我们不急,先找个地方住下来再说。”

到了商家城,就不愁食料。

大不了多养几天这些骨枪蛊,看谁能耗过谁!

与此同时,第一内城。

“都调查清楚了吗?”商燕飞站在窗前,望着庭院景色。

“这位姑娘的确是族长您的亲生骨肉。并且,她的魂魄也属正常,并非是被人夺舍。已经联系了张家那边的族人,确认了她的身份。但有一点比较奇怪,她是参加商队来到商量山。但是我们却找不到商队中和她同行的其他人。”外姓家老魏央垂首汇报道。

商心慈闻言不语,心中暗自长叹:“苍天呐,谢谢你给我一个弥补愧疚的机会。我已经辜负了一个女子,对于这个女儿我绝不会再辜负!”

……

商心慈坐在桌前,看着袅袅的茶气在面前升腾,眼中失神。

从小蝶的口中,她得知了商燕飞的身份。

她没有想到,自己的父亲居然是堂堂的商家族长。一个权势滔天,一举一动都能影响整个南疆的男人!

那他为什么不去见娘亲呢?

商心慈天资聪颖,很快就悟到商燕飞之所以抛弃母亲和她的原因。

但母亲直到临死前,还念着他。

商心慈心中有悲伤,也有愤恨,更多的是茫然。

她不知道该怎么去面对这个突然出现的父亲。

但逃避是不可能的,门外传来敲门声:“慈儿,我可以进来吗?”

小蝶顿时紧张起来。

是商燕飞。

\end{this_body}

