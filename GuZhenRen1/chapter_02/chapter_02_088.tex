\newsection{名声蛊}    %第八十八节:名声蛊

\begin{this_body}

%1
方源冲出五十步后,冲势才止住。他回转身子,面对李然,正要继续进攻。

%2
但李然已经抬起手,高声大叫道:“等一等,不打了,我认输了!”

%3
此言一出,全场一静,继而大哗。

%4
“搞什么,这就不打了?”

%5
“我们专门花元石来这里,就是要看全力以赴蛊的。”

%6
“你这个怂货,还是不是男人,站起来再打啊!”

%7
人群义愤填膺。

%8
很多人气得直喘粗气,觉得元石白花了,开始骂骂咧咧。但也有一部分人十分理解李然。

%9
“这根本打不了,李然认输很明智。”

%10
“刚刚一击,已经足以说明差距。再打下去,李然肯定有姓命危险。”

%11
“这个李然,也是演武场的老油条了。经验丰富,他这么做,我毫不意外。”

%12
当。

%13
一声清脆的钟响,宣布这场演武结束。

%14
演武场边,众人开始离场,方源也作势离去。

%15
“方正,你先别走。”李然忽然开口,叫住他。

%16
方源皱起眉头,转身看他:“你想干什么?”

%17
旁人脚步也不由一缓。

%18
“方正,我对你有恩,你现在却将我打伤,你是恩将仇报,你得赔偿我!”李然叫道。

%19
这话说的有些无耻,明明是你自己不自量力地想要挑战,结果受了伤,怎么还倒打一耙,说人家“恩将仇报”呢。

%20
很多人听了这话,都不由地嗤之以鼻,对李然更加不屑。

%21
找方源强行挑战,那是气不过,人之常情。现在还纠缠方源,就有些胡搅蛮缠了。

%22
方源摇摇头,转身就走:“你脑袋也被我撞坏了吧?”

%23
许多人发出哄笑。

%24
但李然挣扎着站起来,朝着方源喊道:“方正,我知道你这个人!你恩怨分明,号称滴水之恩以涌泉相报,星火之仇燎原往复。商心慈给你一些小恩小惠,你却冒着生命危险救下她,保护她来到商家城。商家族长要给你奖赏,你都曾经一概推拒,说恩情两清。是商家族长硬塞给你紫荆令牌的!”

%25
“方正,我对你也有恩!你说,如果不是我挑选了那块星辰石,你能得到全力以赴蛊吗?不能!嘿嘿,旁人也就无所谓了,但是我知道你,我清楚你。你这个人,虽然横行霸道了一点,但有恩必还,否则你都睡不好觉。是不是?你想想看,你欠我的情,你今后能睡得着觉吗?”

%26
“哼,那是你没看清方源的真面目!”人群中,白凝冰听到李然的这番话,心中不由地冷笑。

%27
方源得了全力以赴蛊,也激发了她的好奇心。

%28
毕竟,方源一直都是她的假想敌。

%29
但方源却止住了脚步。

%30
众目睽睽之下,他转过身,面对李然,脸色有些凝重。

%31
“听你这么一说,好像我的确欠你的情。但是一来,本身就是你冒犯我在先。二来,全力以赴蛊我是不可能给别人的。你说我该怎么办?”

%32
方源的这番话,让正在离开的众人都不由地停下脚步,纷纷驻足观望。

%33
白凝冰心中也轻咦一声,感到意外。

%34
“全力以赴蛊虽然是三转蛊,但它来源上古时代,如今已经是独一无二,真正的价值难以评估。你就给我十万元石,就不算欠我的人情了!”李然琢磨了一下,道。

%35
“这个李然是傻子吗?”

%36
“居然狮子大开口,蠢成这个样子,唉……”

%37
“这种要求都能提出来,真是恬不知耻啊!”

%38
众人纷纷摇头,心中对李然很不耻。

%39
方源沉思一下,果然摇头。

%40
“十万元石,不能抵消这份情。给你二十万元石,我心才安。”说罢,他一扬手,唤出元老蛊,将里面的元石尽数拿出来。

%41
演武场地上,顿时多出了一堆元石。

%42
“这是八万多块元石,我目前手中只有这么多。等我将来有了钱,再补给你!”

%43
“什么?!”方源的话,让众人吃惊不已。

%44
“他居然真的给了?还,还主动提价到二十万!”许多人瞠目结舌。

%45
“我没有看错吧!这个李然虽然没有拿到全力以赴蛊,但是有这么多的元石补偿,也不算差了。”大多数人都眨着眼睛,看着一堆元石,差点流下了口水。

%46
“这个方正真是……”很多人看着方源离去的背影,一时间都涌起怪异的神色,不知道说什么好。

%47
虽然没有如愿以偿地看到全力以赴蛊的威力,但是李然和方源的有趣对话,却让众人不无收获。

%48
这一场战斗,很快通过众人之口,一传十十传百,在商家城迅速地流转开来。

%49
方源拥有紫荆令牌的事情,广为人知,打消了不少宵小的不良居心。

%50
很多人都开始羡慕李然,也有很多人对方源承诺的二十万元石,抱有怀疑。

%51
但不管如何,方源恩怨分明的名声,算是打响了。

%52
回到楠秋苑。

%53
“你真打算给那个李然,二十万的元石?”白凝冰疑惑地问道。

%54
这不是方源的风格啊。

%55
“当然。”方源言简意赅地回答。他当然不会告诉白凝冰,这其实是他和李然之间的秘密约定。李然帮助方源演一场戏,同时告知方源合炼秘方,而方源则补偿给李然二十万元石。

%56
白凝冰沉默了一下,有些不信,冷笑起来:“花二十万元石,就买一个名声,值得吗?”

%57
方源呵呵一笑:“你难道没有听说过名声蛊的故事?”

%58
白凝冰目光迟疑:“你想说明什么?”

%59
“名声就是一道桥梁,可以令人跨越深渊。名声就是一块通行的令牌,比紫荆令牌还要宝贵,可以令人通行无阻。二十万连紫荆令牌都买不到。我花二十万,就买来名声,这可是天底下最讨便宜的买卖,哈哈哈。”方源笑道。

%60
白凝冰冷哼一声,想到他有预知蛊,便姑且相信了他。

%61
关于名声蛊的故事,来源于人祖的传说……太曰阳莽一次喝得酩酊大醉,醒来的时候,头脑生疼,忘记了醉酒时发生的一切。他发现自己不知道为什么,被困在一处孤峰之上。孤峰周围,都是几千丈宽的深渊。

%62
深渊里充满了漩涡似的风,一团团的风,都是惨绿色的,这是“平常风”。风中刮着尘土,都是暗黄色的“凡俗尘”。

%63
太曰阳莽心沉谷底。因为他认出来,这是平凡深渊,从未有生物能飞跃过去。他被困在这处孤峰上,出不去了,迟早要饿死。

%64
所幸孤峰之上,还有一片密林。太曰阳莽饿了,便来到这片密林寻找野果充饥。但是这片密林很奇怪,黑色的泥土像是沼泽,带着的气息。每棵树都没有树叶,枯瘦的枝干像是怪爪。偏偏当风声吹来的时候,却还有树叶沙沙作响的声音。

%65
太曰阳莽找不到食物,陷入绝望,知道自己命不久矣。

%66
几天过去,他饿得四肢无力,只能依靠着树干,瘫躺在地上。

%67
他渐渐地昏迷过去。

%68
在迷迷糊糊中,他听到许多人的声音在交谈。

%69
“喂,你看你看,这个人终于昏倒了。”

%70
“嗯,果然不出我的所料,他要完蛋了。”

%71
“其实平凡深渊可以出去,只需要得到名声蛊就好了。”

%72
“名声蛊就在细语密林的中央,被一块石头压着。可惜他不知道,哈哈哈……”

%73
“嘘,咱们小声点说话,万一被他听到就不好啦。”

%74
“没事没事。他已经昏迷过去了,再过不久,就被黑泥埋没,转化成养料,滋养我们这些树了。”

%75
听到这里,太曰阳莽悚然惊醒。

%76
原来这片密林,是细语密林。他曾经听到的树叶沙沙的响声,是密林中的细语声。

%77
按照他听来的消息,太曰阳莽走到密林中间,搬开石头,取得了名声蛊。

%78
名声蛊像是一朵菊花,花瓣金黄灿烂,散发着一种似香似臭的气味。

%79
名声蛊对太曰阳莽道:“年轻人,谢谢你把石头搬开,解救了我。为了报答救命之恩,我决定帮助你渡过平凡深渊。”

%80
名声蛊告诉了太曰阳莽,该如何使用自己。

%81
太曰阳莽大喜,来到平凡深渊,把名声蛊塞到嘴里,然后用尽力气大声的呼喊……

%82
奇怪的是,不管他叫得多用力,都没有任何的声音,但是却震得平凡深渊不断地陡荡,动静之大,简直是山崩地裂一般。空气中还充满了美妙的香气。

%83
太曰阳莽也不疑惑,因为他从名声蛊处得知:名声本身寂然无音,却能广泛传播,引发剧烈震动。

%84
随着他的叫喊,半空中出现了一道金光的桥梁。但金光桥梁长度有限,距离对岸还有很长的一段距离。

%85
太曰阳莽太饿了,太累了,试了几次,效果次次削减,自救无望。

%86
名声蛊叹了一口气:“唉,你很久都没有吃东西了,腹中虽然有气在,但是量少,从腹中调上来,要经过肚子,胸口,喉咙,最后才出口,路途太长。我们必须减少路程,这样,你把我按在你的两个屁股中间。”

%87
太曰阳莽依此做了。

%88
名声蛊就落到了他的下体附近,化为了一个菊花小洞。

%89
“好了,你现在可以调气再喊了。”名声蛊道。

%90
太曰阳莽便调出一股气,通过这个洞,排出体外。

%91
噗——!

%92
恍惚间,太曰阳莽双耳似乎听到一声闷响。空气中顿时臭不可闻,但是那道金光大桥,却变得雄伟壮阔,横跨千丈,搭在对岸。

%93
臭名声永远比好名声,来得容易和稳固。

%94
太曰阳莽连忙登过光桥,越过平凡深渊,走到对面,成功自救。

\end{this_body}

