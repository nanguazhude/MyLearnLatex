\newsection{商家内城}    %第六十二节:商家内城

\begin{this_body}

方源缴纳了两百块元石,和白凝冰一起进入内城。

内城虽然是建造在山中,但街道宽阔,足可供十辆马车并排而行。

一进入这里,人流骤然减少,只剩下外城的一半不到。

但是蛊师却已经随处可见,一转满地走,二转蛊师掺杂其中,偶尔有一两位三转。

普通的凡人非常少,毕竟要进入这里,每人得需要一百块元石。很多蛊师纵然有贴身家奴在身边服侍,也不愿花这份冤枉钱。

内城的照明,普遍取用了一种火炭石。

火炭石燃烧长久,并且没有烟雾产生。方白二人平均走百步,就会看到墙壁上挖开洞,一堆火炭石在里面燃烧着。

尽管火炭石散发的热量不高,但是这么多的火炭持续不断地燃烧着,仍旧让内城中空气的温度拔高,并且干燥。

不像外城,各种建筑物都有,杂乱得很。这里的建筑物,已经统一构造,外形相似,都用了耐热的红色岩石。

各种支路,从街道两旁延伸出去。

同时,街道上每隔五百步,都会出现一根巨大的圆柱。

圆柱表面塑造了螺旋形式的石梯,绕着圆柱延伸向上,石梯外侧有护栏。

通过石梯,人们可以去往上一层,或者下一层的街道。

内城并非是寻常意义的城池,而是立体的。从上至下,无数街道,房屋林立,相互贯通,四通八达。

方源和白凝冰一路向山内前行,这里还不是他们的目的地。

到了某处地方,又出现关隘。

把手的蛊师,修为更高,防御更加森严。

“二位请住,有令牌么?”守卫拦住方白二人。

商家针对不同身份的人物,发放各种权限的令牌。

“我们还是首次来此。”方源道。

他当然没有令牌了。

“既然如此,请每人缴纳两百块元石。”守卫道。

方源交了元石,守卫放行。

二人由此来到第四内城。

商量山经过商家数千年的经营,整个山体内部都被商家改造,挖开通路,雕塑建筑,分区布局。

因此内城极大,从内而外,分外五区。

第一内城也叫中央内城,是商家政权中心,亦是军事重地。

第二内城,别名家城,是只提供给商家本族子弟居住。

第三内城,环境优雅,空气清新,是高档区。

第四内城,是中档区。第五内城,是低档区。

再往外,就是外城,人流量极大,货物装卸之处,管理较内城而言,比较混乱。

这种建筑构造,好比是地球上的白蚁山。

白蚁山是高达四米到十米,蚁群在里面生活,细小的通道相互勾连,繁复精细至极。

二人一进入第四内城,顿感空气湿润,气温渐降。

第四内城比第五内城,要高一个档次,不仅体现在入口税多了一倍,还体现在方方面面。

首先,采光在不用廉价的火炭石,而是大量地种植了一种一转草蛊。

名称为月光爬山虎。

这种藤叶植株,附着在街道两旁的洞壁上,蔓延开去,随处可见。

它们的根茎是深蓝色的,叶片又宽又大,散发着微微的淡蓝月光。一段通道,就有成千上万的叶片,柔和的蓝光连绵一片。

因为大量的藤叶,这里空气潮湿,水汽蔓延。在贴近地面的地方,沉降凝聚成雾气。

月光在雾气中折射,形成光雾缭绕。让人走在街道上,有一种漫步仙境之错觉。

这里的建筑,已经多了雕纹装饰,有人造的草坪,摆放着花坛,偶尔间还有假山,亭台等等。

路上行人,更加稀少。

二转蛊师已经成为主流,毕竟对于一转蛊师来讲,单单入口的两百块元石,就是一笔很大的支出了。

最明显的感觉是,走在第五内城,街道上还很嘈杂。到了这里,就安静许多了。

二人一路向深处进发,来到城门口。

“没有令牌,二位要进入第三内城,就得缴纳六百块元石。”守卫蛊师中的头领,已经是三转修为。

方源交了元石,终于来到第三区。

这里又和第四内城不同。

所有的建筑石料,都采用了星星石。

这种石头,是蛊师广为采用的炼蛊辅料,在黑暗中能散发出璀璨的星光。

整个第三内城,都采用星星石。不仅是建筑物,甚至连街道上都铺着星星石料所制的石板。

放眼望去,星光连绵一片,视野清晰,再无光雾阻挠。

空气也清爽无比,放眼处,亭台楼阁,红墙绿瓦。更移栽了竹林、名木,打造了假山,甚至勾引了泉水,流水潺潺。

街道上行人稀少,幽静怡人,恍若星宫。

“真是财大气粗啊……”白凝冰稍稍估算了一下,单就眼前范围的设施造价,就是一个令其眩晕的数额。

商家乃南疆财富第一,有人说拔根腿毛来,都比其他人的腰还粗。此话虽然夸张些,也并非空穴来风。

商家富如山,整个商家城就是一座立体的大山。商家行商,遍布南疆各处。

商家的财富究竟有多少,没有一个外人能说得清。

但方源至少知道,单单这个第三内城的造价,就能抵得上十几个古月山寨的财富总和。

到了此处,连二转蛊师都少见了。

偶尔间,两人见到一位蛊师,几乎都是三转级数。

这里便是方源的目的地了。

再往深处进发的话,就是第二内城。

但要进入第二内城,就不是元石的问题了,还需要商家颁发的令牌。并且这令牌的规格要高到一定程度。

“通幽商铺。”方源看了眼牌匾,迈步走了进去。

这是间买卖蛊虫的店铺。

“两位贵客,请里面雅座。”一位负责接待的少女,立即走了过来,轻声细语。

她气息流露,竟然是位一转蛊师。

方白二人虽然都穿着凡人的衣衫,一个丑陋,一个落魄,但这位蛊师少女却仍旧态度恭敬,显现出优秀的素质。

方源和白凝冰被引入雅室。

这是单独的房间,檀木桌椅,雕梁画栋,洁白的墙上挂着字画,字体龙飞凤舞,笔力刚虬。

透过窗棂,可看见庭院,院中青木红花,鸟鸣啾啾。

蛊师少女奉上两杯香茶,就退了出去。

她前脚刚刚离开,后脚就有一位老者走了进来。

“不知二位贵客,来到鄙店,想买还是想卖?”老者是个二转蛊师,脸上堆着笑,拱手问道。

“既是想买,也是想卖。”方源一边端起杯盏,一边答道。

老者哈哈一笑,两道光从他体内飞出,分别悬浮到方源和白凝冰的面前。

是两只书虫。

书虫只是一转蛊虫,却相当珍贵,堪比酒虫。

在市面上刚一露面,就会被人抢购。因此常常有价无市。

它形如蚕蛹,虽分头、胸、腹三个体段,但通体如纺棰一般,有些浑圆可爱。

它浑身洁白,表面像是涂了一层釉质,带着油亮的光。

摸在手中,也是光滑润和,仿佛是上佳的瓷器。

书虫是存储类的蛊,和兜率花类似。

区别只是,兜率花存储实物,而书虫却存储无形的知识和讯息。就算是自毁,也只是炸成一蓬无害的白光。

“二位请细细浏览。”蛊师老者道。

两只书虫都被他炼化,算是借给方源和白凝冰的。

方白二人各调动一股雪银真元,灌注到书虫中去。

书虫顿时化作一道白光,汇入两人的额头眉心之中。

顿时,方源和白凝冰的脑海中,都多了一段丰富的信息。

这些内容,就好像是反复背诵之后,深深地印在脑海当中的一样。

白凝冰暗暗咂舌,通幽商铺出售的蛊虫之多,简直成千上万!种类繁多,令人有眼花缭乱之感。

其中,不乏有书虫、酒虫之类的珍稀蛊虫。从低到高,涵盖一转到五转。

当然,六转是绝对没有的。

每种蛊虫,都有详细的简介,说明了作用。更标有价格,有些阶位较高,珍贵稀少的蛊虫,还特别详注了各种令牌标准。

这就意味着,蛊师只有拥有商家发布的令牌,才有资格购买这些蛊虫。

方源需要大量购买蛊虫,粗略浏览了一遍后,他收回心神,将书虫唤出,还给了老者。

他现在身上虽然有上万块元石,但距离收购这些蛊,还有巨大的缺口。

关键不仅是钱的问题,一些计划收购的蛊,还有令牌标准。

“这位客人,看中了什么?若是购买的量多,我们通幽商铺还有相应的折扣。”老者微笑着道。

看到方源和白凝冰的雪银真元后,他语气更加客气。

方源摆摆手:“不忙,我这里有蛊虫要卖。”

言罢,唤出一枚骨枪蛊。

老者也不意外,来通幽商铺买蛊虫的多,卖蛊虫的也不少。

他捏住骨枪蛊,只看了地一眼,脸上就不由地流露出一丝惊诧之色。

骨枪蛊他从未见过。

他当然没有见过。

这是灰骨才子的独门蛊虫,从不在市场上流通。

“请贵客指点。”老者面容肃穆,拱手道。

方源点点头,喝了一口茶:“好说,这名为骨枪蛊,乃是一族蛊虫。我借给你用,你试试手便知。”

老者当场试演了一番,沉吟道:“这枚蛊,虽然只是一转,养的也不太好。但攻击不俗,又奇特,可值这个数。”

老者伸出手掌,比了个数字。

\end{this_body}

