\newsection{白羽飞象}    %第四十八节:白羽飞象

\begin{this_body}

白凝冰强忍住这股冲动,双眼眯起来。

她到底是冰雪般聪颖,想透了一些东西:“商心慈身上到底有什么东西,值得你这么费心尽力?”

先前她以为,方源是觊觎商心慈的美貌。现在她推翻了这个想法。她了解方源,只有重大的利益,才会令他如此大费周章。

但是相处这么多天,白凝冰已经了解到商心慈的身世,她在张家受到排挤,本身也只是一个凡人,根本就没有修行的资质。

商心慈的确貌美如花,但这样的容貌,放在她的身上,并非是优点,而是缺点。

这样的容貌,会引来贪婪和罪恶的魔爪。而她偏偏却没有自保之力,若非身边有一位忠心耿耿的三转蛊师,她商心慈早就被人捉去,沦为玩物了。

这样的一个人,到底有什么价值?她的商业才华么,和方源相比,根本不算什么。

白凝冰百思不得其解。

方源不语,没有回答白凝冰。

“那边两个,搬快一点,别磨磨蹭蹭的!”不远处,一位蛊师手指着方白二人,大喝道。

方白二人脚步加快,白凝冰压低声音:“你这样搞,不怕暴露吗?万一被人发现的话,嘿嘿,这些人必定会和你不死不休!”

“那他们发现了吗?”方源反问一句。

“呃……”

” 章节更新最快” 两人放下木箱,往回走。

方源为了消除自己嫌疑。第一次引诱兽群时,就牺牲了自己的货物。几次兽群袭击过后,张家的货物损失最多。很多人都对商心慈报以同情。甚至商心慈也找过方源,劝慰过他。

但现在白凝冰仔细想想,忽然发现,方源的货物看似牺牲很多,但其实真正有价值的货,一直都保存到现在。这些货占了众价值的一大半,他真正损失的并没有多少!

他手段隐蔽至极。若非自己巧合之下发现,估计还会被蒙在鼓里。

想到这里,白凝冰心中就有不忿“这个家伙。居然连我都瞒着!”

两人再搬起一个箱子。

方源似乎知道白凝冰的心理想法,轻笑一声:“欺骗敌人,最好从欺骗自己人开始。况且,我也不是有意要瞒着你的。你有你的作用。”

“哦。什么作用?”白凝冰不由地问道。

“警示作用。你是我身边最亲近的人,你如果发现了蹊跷,那么其他人也差不多了。”

“可是,我也是今天巧合之下……”

方源摇摇头:“巧合也代表着某种趋势。不过,也差不多是时候了。”

白凝冰眼中一亮:“你想干什么?”

冷翡枭猫到底没有冲到第三防线,第二道防线坚挺地耸立着,将来犯的枭猫群击溃。

””战后,生还者清点伤亡。打扫战场。

“这是第几次兽群袭击了?”

“我想回家!”

“真是该死,这一趟的运气也太差了。”

“我们该不该继续往前走?也许留下来。等待其他商队的帮助,是个好主意。”

……

众人的情绪极其低迷,很多人怨声载道,普遍觉得前途未卜,不想再继续前行。对死亡的恐慌、焦躁、畏惧等等情绪,弥漫在整个营地当中。

“贾首领,为什么每一次都安排我陈家部守在第一防线上?你究竟有何居心!”

“陈副首领,我一直都是公平公正。你们陈家实力最强,如今家同舟共济,互帮互助。能者多劳,自然要担当起更多的责任。”

争吵声忽然传来,惹来许多人的关注。

贾龙和陈家副首领陈双金,相互瞪眼,气氛紧张。

“我陈家实力最强?呵呵,贾首领你睁着眼睛说瞎话,你贾家剩下多少好手,大家伙心理都清楚!”陈双金冷笑。

“放屁!我家贾平这样的好手,都牺牲了!你们陈家呢?”贾龙痛骂。

“二位,现在可不是争吵的时候。”林家副首领走过来,劝说道。

最终,贾、陈二人不欢而散。

”蛊真人 第四十八节:白羽飞象”“连贾龙大人和陈双金大人,都吵起来了。贾家和陈家,不是走的很近吗?”

“唉,现在这个关头,人人自危。都想着怎样保存自家实力,再好的交情也没用啊。”

“据最新消息,贾家那两位少主闹得不可开交,陈家好像已经彻底倒向贾贵那边。”

“原来如此。贾龙大人是贾富的下属,难怪对陈家没有好脸色了。”

几位蛊师在一旁低声交谈着,方源心中微动。

数天之后,已经人心涣散的商队,前行到象牙山。

象牙山高耸入云,栖息着大量的象群。这里气候独特,山脚到山腰,潮湿闷热,布满雨林。山腰到山巅,皑皑白雪,干爽冰冷,雪松林立。

众人无不小心翼翼,叫他们高兴的是,进入象牙山的几天,都没有遭到兽群的冲击。

“这是好运气终于到了吗?”

“否极泰来,也该如此了。”

“可惜,货物几乎损失光了,这次要赔进去许多钱啊。”

“哼,知足吧,能捡回一条命,就不错了!”

“过了象牙山,再过墓碑山,双将山,就是赵家寨。到那里我非得好好睡个三天三夜不可。”

……

众人七嘴八舌,展望未来,士气稍稍有所”蛊真人”回升。

“咦,下雪了?”一人仰头,发觉点点洁白正从半空中往下飘落。

“胡扯,这是象牙山山脚,怎么可能下雪?”有人不信,但是抬头一看,脸色一僵。

“真下雪了……”

“该死,这不是雪。是羽毛!”有人忽然大叫道。

商队的许多蛊师听了这话,顿时一个激灵。

白色的羽毛,难道是白羽飞象?

就在这时。狂风骤起,白羽翻飞,如大雪倾覆。

昂……

数百头大象齐鸣,在半空中飞踏而下,向地面上的商队俯冲而来。

“该死的,真是白羽飞象!”

“怎么会招惹到它们,它们应该生活在山腰之上才是啊。”

“结阵。速速结阵迎敌!”

已经来不及了,象群俯冲而下,所到之处。无不人仰马翻。

这些白羽飞象,浑身都长满了白色的羽毛。两根长达一丈的弯曲象牙,粗壮而又尖锐。巨大的冲击力,让它们所向披靡。

行进中的商队。被打了个措手不及。象群仅仅一次冲击。就带走了上百人的性命。许多家奴被踩成肉泥,车厢被象牙洞穿,黑皮肥甲虫顷刻间死了三头。翼蛇和鸵鸟到处乱窜,造成踩踏事故。

””一时间,场面混乱至极。

“蛊师,所有的蛊师,都向我这边集合!”贾龙在人群中高喊。

但他刚刚聚集了十多人,象群又再次俯冲下来。将蛊师击散。

象群飞上天空,又开始酝酿第三次冲击。

“唉……”贾龙长叹一声。知道反击的希望已经渺茫。他只好高喊:“大家都快逃,逃到周围雨林里去!!”

不消他说,很多人已经冲入了雨林。

但是白羽飞象冲撞力量极强,冲入雨林当中,树木在瞬间被撞断,无数人惨遭大象的蹂躏。

这些飞象外表圣洁高雅,但生性最是嗜杀。

昂!

一只飞象,对准商心慈等人,如一颗流星直冲下来。

“小姐,快跑!我来引开它!”危机关头,张柱挺身而出,发出一道红光,击中飞象。

飞象大怒,转折方向,对准了张柱。

张柱是治疗蛊师,攻击和防御并不强大,在雨林中狼狈逃窜。

飞象冲击而来,带起呼啸之声。

张柱拔腿狂奔,及时向前一扑。飞象重重地撞在他的身后,折断数棵大树,白色的羽毛洒落一地。

“好险!”张柱擦擦头上的冷汗,刚爬起来,顿时眼前一黑。

砰。

一棵粗壮的树干,被白羽飞象用鼻子甩过来,结结实实地撞在张柱的身上。

生死存亡之际,张柱催动防御蛊,浑身罩住一层金光。

噗。

金光溃散,他大吐一口鲜血,被远远地撞飞出去。

他眼冒金星,头昏脑涨。倒在地上,不能动弹。

隐约间,他听到象足踏地的声音,且越来越大。

他脊椎骨生出一股寒气,丰富的战斗经验告诉他自己的生命危在旦夕!

他来不及多想,连忙就地一滚。

差之毫厘,白羽飞象就从他的身边撞了个空。

轰!

一声巨响,白羽飞象狠狠地撞在山壁上,两根象牙深深地插进山石当中。

飞象高声长啸,努力摇晃脑袋,同时四足往后倒退。

张柱模糊的视野终于变得清晰,他晃晃悠悠地站起来,看到这幅清晰,不由地冷汗淋漓。若刚刚他慢上一拍,必定就被撞得粉身碎骨了。

他查看了一下空窍,还剩下五成真元。防御蛊状态萎靡,已经濒临毁灭的边缘。

“必须赶快回到小姐身边去!”他心中焦急万分,自己身为蛊师,都如此危险。小姐和小蝶都是凡人,简直是危在旦夕,命悬一线。

那头白羽飞雪还在拔牙,张柱转身就跑,向记忆中分别的地点追去。

来到分别的地点,商心慈已经不见踪影。

张柱正迟疑该往哪个方向追,一位蛊师跑来,身后追着三头白羽飞象。

“救救我!”他大喊道。

“该死。”张柱咒骂一声,他认出此人乃是陈家的青年蛊师,名叫陈鑫。

张柱心系小姐安危,哪里顾得上救陈鑫,连忙拔腿飞奔。

陈鑫看到张柱,像是溺水者看到了最后一根救命的稻草,也跟着追上去。(未完待续。)

138看书网138看书网www.13800100.com

\end{this_body}

