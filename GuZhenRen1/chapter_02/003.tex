\newsection{鳄杀机少女悲鸣}    %第三节:鳄杀机少女悲鸣

\begin{this_body}

“锯齿金蜈……”白凝冰伸出芊芊玉手,摩挲着这只三转蛊的暗金甲壳,口中喃喃,神情有些复杂。(ωωω.⒈з⑧00⒈00.cοм无弹窗小说**【138看书//文字首发】网**)

她和方源大战,吃过这锯齿金蜈不少的亏。没有想到,居然有这么一刻,被方源主动借给了自己。

方源使用锯齿金蜈的方法,给她留下了深刻的印象。

白凝冰立即照葫芦画瓢,将锯齿金蜈当做大剑来回挥舞。

时不时的,她心念操纵,锯齿金蜈伸缩身躯,宛若鞭子挥舞。银边锯齿转动着,在空中划出一道道诡异扭曲的光边。

“天蓬蛊!”她将天蓬蛊收入空窍,眉头一扬,暗暗灌注白银真元。顿时浑身亮起白光虚甲。

“命运真是玄奇,想不到有一天,我居然能用你的蛊虫。”她看向方源,嗟叹道。

方源沉默,而是盘坐在温暖的煤石旁,闭上双眼。

他将心神投入到空窍当中,甲等资质饱满的真元海,立即呈现。

九成!

原先只是四成多一点,如今凭空增长了一倍多。

“虽然修为从三转下降到一转,几年的苦功都耗去了。但这一切都是值得的!”方源心中很满意。

历来蛊师修行,有三大重点。

一资质,二资源,三蛊虫。

这三方面,缺一不可,是重中之重!

原先,方源只是丙等资质,靠耍弄手段,用资源和蛊虫来千方百计地来,尽量地弥补资质的短板。在青茅山的几年,过得相当艰难辛苦。虽然修行进度不错,但也是他竭心尽力,殚精竭虑,筹措冒险的结果。

如果他当初的资质是甲等,那他的人生完全是另一种风光,早就是三转了。

“造化弄人……我现在是甲等资质了,但山寨这样安定的成长环境却没有了。在资源、蛊虫两方面,却比不上从前。”

方源现在在外闯荡流浪,修为太弱,随时都有生命危险,自然比不上在青茅山安逸稳定。当然更没有稳定的贸易,用来互通有无。

“幸好有天元宝莲,资源上最大的难题解决了,至少三转之前都不愁。”方源心神扫过,在九成青铜真元海地,一朵蓝白莲花扎根在海底空窍上,花瓣饱满,充满圣洁仙灵之气。

这天元宝莲,乃是三转,发展潜力巨大。为了炼化它,直接废掉了古月一族的根基元泉。

它相当于一口微型的移动元泉,当初在方源三转修为时,它就能不断恢复方源的真元,令其达到乙等的恢复速度。

三转修为,是白银真元。如今方源是一转初阶,青铜真元。天元宝莲在方源空窍中,令他的真元恢复速度达到极高的程度。

“我如果用一转蛊,真元恢复惊人,近乎无穷无尽。用一两只二转蛊,真元海面虽然会下降,但不断消耗不断回复,海面也会相对稳定。用一只三转蛊,真元就暴降,消耗远远大过回复,支撑不了片刻,真元海就会彻底干涸。”方源心中暗算。

毕竟是青铜真元,而且还是初阶的翠绿,质量太差。

除去这天元宝莲之外,方源还有一些蛊。

首先是本命蛊春秋蝉。

这六转蛊,天下奇榜第七,一旦成为本命,就再也不能移到空窍外。如今稳居空窍中央。

经过又一次重生,它的气势不再,黄绿的强光消褪个干净,一片萎靡,极度虚弱。

它隐去身形,随着时间的流逝,默默地汲取光阴的河水,开始再一轮的恢复期。

方源心中有明悟:“短时间之内,绝无可能再用春秋蝉了。这样的危险状态,一旦用了,就是沉溺于光阴之河,白白自爆送死。”

没有了春秋蝉的压力,其他蛊虫都释放了本性。

二转的四味酒虫,胖乎乎的身躯上四种光辉轮换着闪烁。在高高的真元海面上,不断戏水,很是欢快。

鲤鱼化石般的隐鳞蛊,静静地躺在海底,任凭真元海水冲刷它的鱼鳞。

一只头部长有一对铁钳的黑甲虫,在海面上空振翅飞翔――此乃强取蛊。

和它一同盘旋嬉闹的,是阴阳转身蛊中剩下来的白甲阳蛊。

四转的血颅蛊也沉在海底深处,偶尔表面闪烁一下鲜红的血光。

至于其他的蛊虫,还有血月蛊,如今化作红月牙印记,藏在方源的掌心中。地听肉耳草,成为方源的一只耳朵,平时不显。兜率花寄托在方源的舌苔上。

而天蓬蛊、锯齿金蜈,则在刚刚借给了白凝冰。

算一算数量,方源现在手头上共有十二只蛊虫。

这个数目太多了!

一般而言,低阶蛊师手中有两三只蛊,是常态。到了四五转,才会上升到四五只蛊的样子。堂堂神捕铁血冷的蛊,也不过七只左右。

别看古月一代,还有天鹤上人,那都是特殊情况,两者皆是有数百年积累的老怪。

方源拥有的蛊虫数量,是寻常蛊师的三四倍。数量太多,就会给蛊师带来沉重的经济负担,还有后勤压力。

这些蛊虽然是方源精挑细选,易于养活。但如今,兜率花携带的物资有限,仍旧给方源带来负担。

首当其冲的是四味酒虫,它需要美酒为食。兜率花中有不少酒,但统共算起来,也只能支撑它半年。

“在这半年里,必须寻到新的酒。或者将四味酒虫逆炼,还原成酒虫。”

然后是强取蛊。

强取蛊的食料难以寻找,兜率花中也收藏不多,只能支撑五个月。

其次是地听肉耳草。

地听肉耳草以参须为食,这个兜率花中却有不少,可以支撑近一年。

而血颅蛊、血月蛊,皆是需要鲜血,需要小心算计。

阴阳转身蛊若是阴阳齐全,就是完整的太极光球,阴阳两气转化衍生,不需要蛊师提供食物。但如今缺少了阴蛊,单单留下阳蛊。方源就得每隔一段时间,将其放出,汲取周遭的阳气。

养活阳蛊,相当重要。有了阳蛊,才可钳制白凝冰。有了白凝冰这个便宜保镖,方有生存的保障。

这就意味着,方源今后不能随便就往山洞、地洞里面钻。万一被困在特殊环境中,没有阳气,阳蛊饿死了。到那时,绝望暴怒的白凝冰将变成方源的索命大敌。

现在方源的情景,有点尴尬。

他手上的蛊虫,普遍转数较高。三转、四转,甚至还有六转。偏偏他现在修为只有一转了。

高阶的蛊虫,对他而言,使用起来相当累赘,极不趁手。

而且,更关键的是――他在治疗、移动方面,严重缺乏,是他的两大短板。

“接下来,必须着手解决这些事情,收集野生蛊虫。但愿运气能好些,能碰到适合自己的蛊。刚刚遭遇梭箭鱼群能够逃生,是运气好。但不可能每次运气都这么好的。”

方源思虑稍定,这才带着微微凝重的神情,缓缓睁开双眼。

他刚睁开眼,就看到白凝冰手中托着一个白色蛋壳,走过来。

“你看看。我刚刚催动锯齿金蜈,让它钻入地下。没想到这沙滩下,藏了一枚蛋。被锯齿金蜈捣破了。”白凝冰开口道。

这蛋有半个脸盆大小,白色的蛋壳彻底破碎,蛋黄只残留了一些。

方源只扫了蛋壳一眼,面色陡然变得紧张起来:“不妙,这是六足鳄的蛋。难道这处沙滩,是六足鳄群的产卵之地?”

他迅速地站起身来:“快,催动锯齿金蜈,看看这沙滩下还有多少这样的蛋。”

白凝冰目光抖寒,指向方源后方:“来不及了,你看!”

方源一回头,就看见滚滚黄沙水的江面上,漂游来数百根“枯木”。

“枯木”一个个游上岸,皆是体型庞大的鳄鱼。

这些鳄鱼,背甲浑厚,牙齿尖锐,长有三对足。此刻无数双通红的双眼,紧紧地盯住白凝冰。

六足鳄群!

白凝冰下意识地将手一松,白色的蛋壳掉落在沙地上,蛋壳破碎,蛋黄流淌。

吼!

六足鳄发出愤怒的嘶吼,迈开六足,纷纷向二位少年杀来。

……

与此同时,青茅山,古月山寨旧址。

白雪皑皑,冰川遍地。

一群人默默站立雪地中,如根根铁钉,钉在这苍白的世界里。

“父亲……”铁若男跪在雪地上,口中嘶吼,双眼泪流。

五六天前,铁血冷和古月一代激战,临死之前,仍旧记挂亲生闺女的安危,将山丘巨傀蛊和铁手擒拿蛊飞出。

山丘巨傀蛊形如青铜面罩,罩住铁若男的脸面,保护住她。铁手擒拿蛊,则化为巨手,带她远离青茅山这块是非之地。

事后不久,这两只已经被血狂蛊污染的蛊虫,皆化为血水。

铁若男发了疯似的,赶回青茅山。但在途中,遭遇危险,被兽群围困。

危难之际,铁家的援兵赶到。原来铁血冷行事稳重,保险起见,早在之前,他就向家族发送了求援信笺。

铁若男得了援兵之助,赶回到这里。但见满山冰雪,冻结一切生机,而父亲杳无音讯,彻底失踪。

她和铁家援兵一起搜寻了几天几夜,最终无奈地接受这个残酷的事实。

神捕陨落,父亲已死!

“父亲啊――!”铁若男声音嘶哑,声音中充满了极度的悲伤,宛若大雁哀鸣。(未完待续。如果您喜欢这部作品,欢迎您来(本站)订阅,打赏,您的支持,就是我最大的动力。)

------------

\end{this_body}

