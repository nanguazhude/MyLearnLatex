\newsection{一,二,五}    %第八十五节:一,二,五

\begin{this_body}

“我认输!”李然半跪在地上,大声地喊道。

对手停下冲锋的脚步,及时收手,也不想将李然逼入绝境。这对双方都没有好处。

李然吐了一口鲜血,摇摇晃晃地站起身来。

为了尽快地合理地脱离战斗,他又故意挨了对手几下攻击。

主持的蛊师走上演武场,宣布这场战斗的结果。

李然早就心急如焚,在取回藤讯蛊后,他装做示意的样子,走出了演武场。

草草的料理了一下自己的伤势,他就急忙往自己的住处赶去。

“该死的,怎么会这样子?到底发生了什么事情,为什么和花苞蛊突然失去了联系?”

李然的心中笼罩着一层厚重至极的阴云。

花苞蛊是他炼化的蛊,被方源炼化的那一刻,他就立即感应到。

“通常出现这种情况,只有两种可能。一种是花苞蛊被摧毁了,第二种情况则是被人炼化!难道是我被发现了?不,情况也许不是那么糟糕,可能只是我的屋子遭贼。他娘的,我在商家城生活了整整八年,大小毛贼都清楚,居然还有人看得上我那个破地方!”

李然快步而走,星辰石重要至极,就算是他的心性,也不禁暗暗焦急。

他后悔了。

他应该把花苞蛊,藏到更隐秘的地方,而不是单纯地放在床板下的暗格当中。

但事实上,这也不是他的错。

他孑(jié)然一身,为了隐藏自己,没有朋友,几年来从未和妻、儿照面。常去的青楼、赌石坊、酒楼,人来人往,也不是能藏东西的地方。

倒是可以藏到当铺或者钱庄,但要大张旗鼓地收藏一颗杂等顽石,也太过奇怪了。将来商家调查,这就是一个巨大的疑点。

在商家城,龙蛇混杂,做盗贼的蛊师并不少。但要破解花苞蛊,需要三转蛊。但通常三转蛊师,怎么会看得上自己的这个破烂住处呢?

李然潜伏了整整半年,连他自己都快忘记了过去。藏花苞蛊的时候,他很有自信,但现在他的自信全部转变成了自责。

“但愿事情可以挽回!”

他忐忑不安地赶回到住处,房门虚掩着。

他一下子推开房门,顿见住处一片凌乱。

“果真遭贼了!”他心中顿时冒出这个念头。

遭贼并不可怕,只要自己的身份没有暴露,一切还有挽回的余地。

“没错!”李然安慰自己,“我每次和武家,都是单线联系。手中根本没有留下任何的证据。那只传奇蛊,也被包裹在星辰石中,不解开石头,谁会明白它的价值?只要我找到那个贼,凭我八年来经营的关系……”

这么一想,他渐渐地稳住情绪,将慌乱排除心中。

“要不要报案?借助城卫军的力量,帮我缉拿盗贼?不,还是先礼后兵,能安安稳稳地拿回星辰石最好。城卫军也不可靠,不会为了我这个小人物尽心尽力的。也许,我该雇佣一位铁家的蛊师?”

“嗯?这是……”这时,他眼神一凝,在掀开的床板上赫然发现了一只蛊。

心音蛊!

此蛊二转,青黑色,婴儿的小拇指尖差不多大小。形如螺蛳,一端大,一端小,表面有螺纹。

“心音蛊都是两只配套,可以令两位蛊师在一百步内,利用心声交谈。难道说……这是贼人特意留下来给我的?!”

李然眼中闪过一道犹豫的光,接着咬牙将心音蛊塞入耳中,仿佛是个耳塞一般。

“你是谁?”李然灌注真元,催动心音蛊,凝神聚念,在心中试着发问。

“我是谁这个问题并不重要。重要的是,李然这个名字应该是假名吧?呵呵呵。”旋即,方源的声音在李然的心中响起。

刹那间,李然瞳孔猛缩成针尖大小,如遭电击,整个人呆立在原地。

“不好,他发现了我的身份!”李然惊骇欲绝。

整个事情发展到最坏的结果!

但他到底是潜伏八年之久的卧底,陡然遭到如此剧变,仍旧勉强镇定了精神,凝聚心力,在心中对方源道:“假名?什么假名,你什么意思?”

他一边说着,一边小心翼翼地踮起脚尖,在简陋的房间中行走。

然后背靠在墙壁上,侧身看向窗户外的街道。

“心音蛊的有效范围,只有一百步,这个神秘男子一定就在我的附近。”他急速思索着对策。

心音蛊的使用,要凝练心神,才能展开对话。普通的思绪想法,是不会传出去的。

但方源早已经算到他此刻的心理状态,轻笑出声:“你用不着伪装,李然。我不是你的敌人,只是你的合作者。为了表示我的诚意,我们可以见个面。”

“见面?”李然着实楞了一下。

紧接着,方源的声音又在他心中响起:“现在你要走出房门,出了大门左转。”

“我凭什么听你的?”李然在心中叫道。

方源淡淡一笑:“不要试探我知道多少,我知道的永远比你想象的多一些。这样,我数到五,你好好考虑一下。”

“一。”方源数道。

李然急速思考。

就算是花苞蛊落到对方手中,但对方也没有证据证明自己就是武家的卧底。

他潜伏了八年,怎么可能手中留有明证?

“二。”方源的声音不紧不慢。

如果此时,听从方源的威胁,落到有心人的眼中,那么势必就从另一个侧面证明了自己的卧底身份。

但如果不听他摆布,星辰石怎么办?

“五。”方源的声音继续在他心中响起。

李然顿时方寸大乱,在心中咒骂:“该死,你会数数么?!”

“呵呵呵,看来你已经考虑好了。”方源笑道。

李然鼻息沉重,双手攥成拳头,狠狠一咬牙,转身离开小屋。

他走出大门,来到人来人往的街道上,然后毅然向左转。

走了五十多步后,他的心中再次传来方源的声音:“向右转,走到第三个岔口,向左。”

“他能看到我的行踪,势必就在我的身边。到底是哪一个?”李然目光锐利,仿佛鹰隼一般,扫视周围人群。

“我劝你不要东张西望,这可不像你平时的作风呢。啧啧,潜伏了八年,别到此时功亏一篑,你说是吗?”方源的声音很快传来。

“可恶……”李然将牙齿咬得嘎吱作响,方源的威胁让他只好垂下头,一直盯住前方。

在方源的指挥下,他七拐八绕,终于方源叫他停下。

“转过身。”方源紧接着道,“在你的视野中,只能看到一家酒楼。去这家酒楼的三层,我就在那里等你。”

李然转身一看,顿时心中一悸。

这家酒楼,正是他常来的富态祥和酒楼。

“难道说……”他心中立即涌动出一股强烈的不妙之感。对方掌握的情报如此之多,叫他觉得自己仿佛是被剥了衣服,站在冰天雪地当中。

他走进酒楼。

熟识的店中伙计看到他,热情地打招呼:“李然大人,您来啦,楼上请!”

李然神情凝重,勉强挤出一丝笑容,带着十二分的警惕,登上楼梯。

刚走了一半,还未到二层。

“等等。”方源忽道。

李然停住脚步,心中发问:“怎么?”

“下楼,出门。”方源指挥道。

李然哼了一声,只好转身下楼。

店中伙计又看到他,感到疑惑,连忙上前:“怎么,大人您不想用餐啦?”

李然摆摆手,挥退他,走出酒楼大门。

“在你对面,有那个卖烧饼的小摊贩,去买几个烧饼。”方源继续道。

李然眼角抽搐了一下,但最终仍旧听了方源的吩咐,将烧饼买到。

“好,现在你返回酒楼,到三层上来。”方源又道。

李然额头冒起青筋,拿着烧饼,重新返回酒楼。

店中伙计再次见到他,纷纷投来异样的目光:“李然大人,您要买烧饼,你说一声就是了,小的给您跑腿。”

“滚开。”李然咒骂一声,吓得伙计肩膀一缩,连忙噤声。

他走到三楼,在楼梯口站住。

“继续走啊,两三步后左转,你就能看到我了。”方源传音道。

李然依言而动,转过一个立柱,他终于看到了方源,就坐在他常坐的那个位置上。

然后,他就看到方源对他伸手,做了个邀请入座的动作。同时心中响起声音:“请吧。”

李然默然不语,闷着头走到方源的面前坐下,然后一瞬不瞬地盯着方源。

说实在话,方源如此年轻的面孔,让他心中着实吃了一惊。

刚刚对话,方源将他吃的死死的。他下意识地勾勒出对方源的印象――一个老奸巨猾的中年男子,或者老年,带着兜帽,将面孔隐藏在黑暗当中。

但是事实却大相径庭。

用心音蛊的声音,都是一个调子,不能说明年龄,甚至不能判断性别。

如果不是亲眼所见,李然绝不会想到,这个神秘人居然如此年轻!

第五内城的火光,透过窗户,映照在方源的脸上。他面容虽然普通,但是双眼却漆黑如墨,宛若深潭,深不可测。

李然将方源的面貌,深深地印刻在脑海中。

虽然这只是他第一次见到方源,但他坚信,自己这一生不会忘记方源的样貌。

皆因自己就是栽在这个少年的手里,还栽得如此不明不白!

\end{this_body}

