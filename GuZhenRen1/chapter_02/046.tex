\newsection{二转高阶:}    %第四十六节:二转高阶:

\begin{this_body}

商队一路西行,穿山越岭。

半个月后,方源等来到阴森静寂,枯树怪影的魂木山,他抛售掉一小半的货,买下许多特产木材。

二十多天后,他们来到巨雨山。

巨雨山上,坑坑洼洼,这都是雨滴砸出来的大坑。

这里一旦下雨,颗颗雨滴都大如酒坛,冲击力巨大。因此,巨雨山上的童家寨,是建于山体空洞之中。若是露天建造,几场雨下来,就会被雨滴砸毁掉了。

在童家寨子里,方源在黄金山上收购的黄金油灯,大受欢迎。

驻扎期间,下了三场雨。方源也因此采集了不少蓝油雨滴。这些雨滴,是一种用途广泛的辅料。

这些巨大的雨滴,既是童家寨的灾害,逼迫他们只能在山体内部生活。但又是天材地宝,是他们最主要的贸易商品。

离开巨雨山后,商队来到方砖山。

这山上的石头,很奇特。每一块都是砖头形状,只是有大有小,有粗有细。

大方家族,已经在方砖山上生活了上千年,是个大型家族。

他们的住宅,比童家寨要好得多,都是砖瓦结构。山寨周围建筑了高大的城墙,城墙内有塔楼,城墙外有碉堡。

方源记得大方家,有一个怪癖的家老,喜欢木雕。

联系上他后,他在魂木山收购的,造型怪异夸张的魂木,被这位家老全部买下。

就这样一路行商,在方源的操作下,张家的商货时增时减。但每一次的变化,都能赚取丰厚利润。

次数一多,自然就引起了其他人的关注。

方源将一切都推到商心慈的身上,商心慈的经商才华渐渐得到传扬。

时光流逝,又经过了四家山寨后,商队渐渐接近啸月山。

这天晚上,商队背靠着一处峭壁,驻扎下来,结成临时营寨。

“我们已经进入啸月山的地段,接下来的一段路程,都是荒山野岭,虫兽满布,连村庄都没有。从今晚开始,大家须提神防备。”贾家首领嘱咐道。

帐篷内的各家副首领,都表示纷纷点头。

商心慈眼中闪过一抹光亮。

这条商道走到现在,是最危险的一段。经过了这段,就能到达血泪山,再经过几家山寨,便是商量山商家城。

“好了,接下来安排这一周的布防位置。”贾家首领接着道。

半个时辰之后,众人商讨结束,纷纷从帐篷中走出来。

张柱就迎了上去:“小姐,晚饭已经准备好了。今天还是要邀请黑土和白云来就餐吗?”

“当然了。”商心慈点点头道,“我还要在宴席中向他讨教行商的心得呢。”

这些天来,方源赚多亏少,将商货颠来倒去,赚了五六倍的利润。令商心慈、张柱、小蝶这三个知情人,都刮目相待。

方源依照约定,将所获分出一半,交给商心慈。但在商心慈的心中,区区元石怎比得上方源对经商的经验和见解。

她没有修行资质,只是个凡人,经商是她最骄傲的能力。

然而,就在这个引以为傲的领域,方源展现出的实力让商心慈也不得不承认,她不如方源良多!

许多看似荒谬的决定,在之后的交易中,都显现出惊人的成效。

商心慈并非是一个自暴自弃的人,在明白自己和方源的差距之后,每天都会宴请方源。

方源指点几句,便能让她受益匪浅。

她在经商方面,天赋卓越,如海绵吸水,迅速成长。

和方源的交流越多,她便越佩服他。

“唉,小姐,你要小心。这两个人明显有故事,不是普通人。”张柱叹了一口气,眼中流露出担忧的神色,他害怕商心慈越陷越深。

“放心吧,张柱叔,我有分寸的。”商心慈的确是蕙质兰心,她从不和方源谈及其他方面,只限于经商。说话点到即止,更不会旁敲侧击。

她觉得方源和白凝冰二人,虽然神秘,但并不危险。

方源经商凭借的都是正规手段,也从不赖账,每赚一笔都和商心慈五五分成。这样的行为,在无形中,带给商心慈更多的安全感。

然而今晚的这次宴请,方源却没有接受。

“我今天感觉有点累,就不去了。”他摆摆手,对前来特意邀请的小蝶道。

并非商心慈的每次宴请,方源都会点头。三次宴请中,往往只有一次方源会去。

小蝶撇撇嘴,埋怨地看了方源一眼,嘴里嘟囔了几句,只得离开。

先前方源拒绝,小蝶还吵闹几次,为小姐不忿。但是随着方源越赚越多,小蝶的态度也在转变,从不忿不平,变成了迁就和无奈。

不论哪个世界,哪个层次,有实力的人都会引起尊重。

方源闭上帐篷,白凝冰已经盘坐在其中的一张床榻上。

昏暗中,她的蓝眸透着微微的亮光。

先前方源拒绝小蝶,她还有些意外。但是几次之后,白凝冰意识到此举的妙处。

俗话说,无事献殷勤非奸即盗。方源拒绝邀请,不迁就商心慈,反而会让少女觉得方源对她别无所求。

“开始吧。”方源也盘坐到床榻上,背对着白凝冰。

白凝冰双手呈掌,搭在方源的后背上,一成的雪银真元随着心意,调动而出,灌入到方源的体内。

骨肉团圆蛊各闪着青红之光,经过它的转换,六分雪银真元进入到方源的空窍中。

哗哗哗。

方源沉入心神,全神贯注,催动这些真元,洗练空窍四壁。

在一转时候,他的空窍还不能承受这样大的冲刷力度。但是如今他到达二转中阶,窍壁底蕴深厚,已经能够直面雪银真元的不断冲洗。

只是持续的时间不能太长,每隔一段时间,都需要休息一会儿。

窍壁上水光波动,不断流转。洁白的光辉不断增强,一些地方已经有固化凝结的迹象。显然,方源距离二转高阶,已经是临门一脚。

方源修行经验丰富,心中有数的很。打算就在今晚,趁热打铁,冲上二转高阶!

时间悄然流逝,不知不觉间,已经到了深夜。

嗷呜——!

忽然,传来一道苍狼王的叫声。

叫声打破营地的静寂,然后紧接着,无数声狼啸此起彼伏,响应它们的王。

“狼袭,狼袭!”

“该死的,都起来,有狼群袭击营地!!”

“好多的苍狼,简直数不胜数!”

……

许多人惊叫怒吼,营地在短时间内苏醒,然后沸腾起来。

“嘿,听这声势,好像狼群规模蛮大的。”白凝冰侧耳倾听了一会儿,笑道。

行商的旅途中,也碰到过好几起兽群的冲击。经历过之后,她如今也不吃惊了。

“这里是啸月山,有人说这里生活着南疆所有的狼。在月圆时分,狼群都会仰望圆月,狼啸能响彻整座大山,此起彼伏,连绵不断。只是第一晚,就遇到苍狼群,这运气有些不太好。”方源微微睁开双眼,一心二用,空窍中的修行并没有停下。

“该死的,苍狼太多了。”

“治疗蛊师,治疗蛊师在哪里?!我父亲受伤了,他流了好多血……”

“东南方的防线已经被突破了,快去支援!”

局势比白凝冰预想的,要严重的多。从发现狼群,半盏茶的功夫,营地的一处防线就被冲破一个缺口,狼群突入到了营地内部。

“啊——!”

“拿起武器,和这帮畜生们拼了!”

家奴的惨叫声,战斗声,呐喊声,不断传来。

“我们要出去吗?”白凝冰问道。

“出去干什么?你能干什么?别忘了你现在的身份。”半晌,方源才心不在焉地回了一句。

“可是狼群已经冲过来,你刚刚收购的货物,可要遭受损失了。”白凝冰笑道,言语中充满了幸灾乐祸。

“那就损失吧。”方源重新闭上双眼。

又过了片刻,帐篷外传来丫鬟小蝶的声音:“黑土,黑土!你们两个在吗?”

“什么事?”白凝冰道。

“天呐,你们真的在?这么大动静,都没吵醒你们两个!有许多苍狼冲进了营地,虽然场面已经得到了控制,但万一有一两只漏网之鱼呢。小姐让我叫你们过去。有张柱大人的保护,那边会很安全的!”小蝶喊道。

“不必了。既然已经控制了局面,就不去惊扰你家小姐了。一两只苍狼,凭我的力气,还应付得来。”方源道。

小蝶有催促几句,但仍旧遭到拒绝。最后她在帐篷外恨恨地跺脚:“真是不识好人心,你们死了可别怪我没提醒你们俩。哼!”

抛下这句话后,她匆匆离去。

苍狼群持续攻击了一炷香的功夫,这才撤退。

这一次的兽群冲击,造成了商队组建以来,最惨重的损失。

三位蛊师战死,十几位受伤,家奴死伤无数,货物各有损失。尤其是张家的货,损失最多。

这些货都是方源借贷过来的,初步估算了一下,仅仅这一夜,方源就损失了近千块元石。

天亮后,清点损失时,他面带苦笑,心中却一点都不在意。

因为就在昨夜,他晋升到了二转高阶。仅仅是过了数月而已,这样的修行速度叫当事人之一的白凝冰,也暗暗惊异。

(ps:有些迟到了,sorry,已经是尽了最大努力了!)

\end{this_body}

