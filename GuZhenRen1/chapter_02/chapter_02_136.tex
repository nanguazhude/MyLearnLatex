\newsection{白狐蛊仙传承}    %第一百三十六节:白狐蛊仙传承

\begin{this_body}

%1
中洲门派林立,和南疆家族制度不同。

%2
家族中,以亲情维系,族人为栽培对象。但在门派里,师徒关系取代血脉亲情。门派广收门徒,只要有资质,有品性,就可收录门内。

%3
正因为如此,方正才能被吸收到仙鹤门中,成为其中一员。

%4
仙鹤门中,从下到上,分为外门弟子、内门弟子、精英弟子、真传弟子、门派长老、门派掌门人,以及太上长老。

%5
三年小考,可选取得内门弟子的身份。八年中考,获胜者能成精英弟子。十五年大考,则可晋升真传弟子。

%6
真传弟子再往上,就是门派长老。

%7
仙鹤门的门派长老,都至少有四转修为。掌门必定是五转,几位太上长老皆是六转蛊仙,甚至还有七转!

%8
中洲是东南西北当中,最为强大的地域。而仙鹤门,则是中洲十大门派之一。比商家还要强大。

%9
因为门派选徒,几乎都是不问出生,择优选取。因此仙鹤门中,几乎没有丙等资质的成员。

%10
乙等资质最为常见,甲等也不在少数。

%11
方正的甲等资质,的确是天才。但在仙鹤门这样的巅峰势力中,像他这样的天才,也不少。

%12
“方正,你天资聪颖,又勤学苦练,如今已有四转中阶的修为。你的这份修为,已经可以成为门派长老。但你入门时间较短,需要完成大量的门派任务才能考核你的忠心。希望你以后继续努力,在大考中取得优胜,成为真传弟子。”仙鹤门的掌门,高座在掌门人的宝座上,俯瞰着阶梯底下,跪着的方正。

%13
“是。掌门的教诲,我一定牢记在心。”方正应答道。

%14
“现在,就有一项师门任务,要交由你和几位精英弟子去办理。任务内容回去好好琢磨,去吧。”掌门人说着,便飞出一只书虫。

%15
方正接过之后,恭声告退。

%16
回到住处,他躺在床上,倒头便睡。

%17
操控蛊虫。需要消耗大量心力,有时需要精神高度紧张集中,有时候又需要一心多用,分出注意力。

%18
和孙元化一战,方正翻尽了底牌。竭尽了心力。

%19
他太累了,强撑着面见了掌门,完成了晋升精英弟子的仪式。脑仁子仍旧不断地疼痛,头像是变成了重锤,脖子双肩都扛不住,有浑浑噩噩,头重脚轻之感。

%20
方正一连睡了两天两夜。然后被一阵敲门声惊醒。

%21
推开房门,是一群精英弟子。

%22
这群弟子有男有女,大多数是三转,也有个别的是四转修为。

%23
和家族不同。家族的忠心往往不需要考验。门派却需要如此。

%24
越往上,精英弟子、真传弟子、长老的名额都卡得十分紧,非常稀少。很多人修行到四转,却在考核中被淘汰。因此这就造成。门派中很多弟子和长老,都是四转修为的现象。

%25
但不管修为怎样。长老们的战斗力一定大于弟子。皆因他们都是一步步严苛考核出来,万中无一的俊秀人物。

%26
“方正,我们都有同一个任务。这一路上,我们彼此间要多多关照。”

%27
“方正,你和孙元化的战斗我看了,很精彩!”

%28
“希望我们在一路上,多多切磋啊……”

%29
这些人都表现得相当和善。他们都知道方正手中,拥有一群上万只的铁喙飞鹤群。虽然方正不能操纵随心,但这种强大的力量,让他们心中敬畏。

%30
“诸位同门,有礼了。说来惭愧,这些天我呼呼大睡,还从未阅览过任务内容。”方正向众人拱手,有些赧然。

%31
“原来如此。那我直接告诉你好了,我们这次的师门任务非同小可!方正,前几个月,天梯山那边出了一件惊天动地的大事,你可知道?”一位精英弟子开口道。

%32
“天梯山?”方正连连点头,“这事情太大了,我当然清楚。天梯山上,忽然出现了白狐仙子的传承。白狐仙子可是正道著名人物,六转的蛊仙,占据着狐仙福地。这个传承一出现,就引动许多蛊仙出世。只要继承这个传承,就能拥有狐仙福地。现在天梯山,已经被许多蛊仙重重包围了。”

%33
“方正,我就直接告诉你好了。我们这次的任务,就是前去天梯山,拼尽全力获取这道传承。我们仙鹤门的蛊仙大人,已经和其他几位门派的诸位大人商议好了。他们为了免伤和气,并不出手,选择门派中的几位精英弟子,来一场君子之争。”

%34
方正听到此处,顿时双眼瞪圆,眼眸中绽放出渴望的光彩:“你是说真的吗?”

%35
这可是涉及到蛊仙的传承呐!

%36
“当然了。我们的运道到了,如果能成为继承者,必定能一飞冲天。将来冲刺蛊仙,也有莫大的希望。但是除去我们仙鹤门,还有天莲派、灵蝶谷、古魂门、天妒楼等其他中洲十大势力。所以我们此行,要精诚合作,众志成城!”

%37
“这是当然的!”方正立即点头应和道。

%38
……

%39
南疆的火炭山,是一座半死不活的火山。

%40
它高达千丈,山巅是个圆圈大洞,里面岩浆流淌,有时候灰尘滚滚,就像是个巨大烟囱。

%41
山上最大的资源,就是火炭石,几乎到处可见。

%42
这种火炭石,燃烧长久,热量不高,无烟雾,被商家城大量采用,广泛用于第五内城。

%43
方源和白凝冰踏入火炭山,已经有数天时日。

%44
他们从商量山出发,要达到三叉山,此处火炭山乃是必经之路。

%45
踩在火炭石上,脚下传来阵阵热意,空气干燥,无一丝水汽。放眼望去,满眼都是暗红的炭石色彩。

%46
这山上也有树木。

%47
但这些树木,各个矮小精瘦,枝叶如针般细小。并不遮蔽视野。

%48
因此火炭山上,较之其他山峦,视野更加开阔。

%49
方源和白凝冰跋涉在山间,若从高空鸟瞰,就如同两只微小的蚂蚁在爬。

%50
事实上,他们的确受到着关注。

%51
“嘿嘿,他们终于到了。”

%52
“火炭山是必经之力,他们要前往三叉山,肯定会来的。”

%53
在一处不显眼的角落里。隐藏着两位魔道蛊师。

%54
他们是两个光头,老者叫做焦黄,壮年成为孟土,乃是魔道蛊师当中赫赫有名的暗杀两人组。

%55
“只要杀了这两人,商家城那边不仅有上万的元石奖赏。还会给我们俩黄梨令牌。这样的好买卖,怎么能放过?”孟土兴奋地舔舔嘴唇。

%56
“但是他们俩都是四转蛊师,我们只有三转巅峰。这买卖好处是大,但是风险更高!接了这份生意,我不知道是对还是错……”焦黄到底年老持重一些,脸上有着忧虑。

%57
“焦黄老哥,你不要被他们的修为吓到。四转又怎样。不过都是四转初阶。手中的蛊虫,大多还都是三转蛊呢。再说,我们也不是没有杀过四转蛊师。想当年的那个萧福禄,不就是死在我们的手上吗?”孟土鼓气道。

%58
一提到萧福禄。焦黄顿时皱起眉头,紧张无比:“我不是跟你多次强调过吗?别再提萧福禄了。他是萧家太上长老的孙子,我们杀了他,得罪了蛊仙。这件事情千万要烂在肚子里。不要再提了。”

%59
“哼。蛊仙又怎样?这么多年来,我们不还是在逍遥自在吗?”孟土不以为意地撇撇嘴。“蛊仙也不是万能的,更何况这两位新晋的四转蛊师?老哥,你可不要忘了,我们这些天辛辛苦苦给他们搭建好的那个陷阱。嘿嘿嘿!”

%60
提到那个陷阱,焦黄的脸色顿时松缓下来。

%61
他沉吟道:“只要方正二人中了这陷阱,激战之后,真元大损,我们就有可趁之机了!”

%62
“正是如此啊。”孟土说着,忽然双眼放光,激动地低呼起来,“快看,他们中了陷阱了!”

%63
方源和白凝冰顿住脚步。

%64
附近的山地,忽然剧烈的颤抖起来。碎小的山石,顺着两旁崖壁,滚滚而下。

%65
地面上,泥土翻开,一只只的熔岩鳄,从中钻了出来。

%66
“是熔岩鳄群。有一只千兽王!”白凝冰双眼一凝,看到了这只兽群的首领。

%67
这只熔岩鳄王,体型巨大,有三头大象叠加起来的身躯。

%68
它浑身长满暗红色的鳞甲,四肢粗壮的腿支撑着它雄壮的身躯。一条鳄尾,散发着金属光泽,长达十米。在它的背部,高高隆起两个凸起,仿佛是两座微型火山。随着它的呼吸,两股黑烟从火山口直冒而出,时强时弱。

%69
熔岩鳄王钻出地面,就紧紧地盯住方白二人。其中更多的注意力,集中在方源的身上。

%70
方源和白凝冰都是四转初阶的蛊师,但是他却有四转中阶的真元。

%71
熔岩鳄王感应敏锐至极,立即觉察出两人当中,方源的气息更加可怕一些。

%72
它张口嘶吼,上千只的熔岩鳄向方白二人,缓缓包围过来。

%73
面对如此险境,方源洒然一笑:“不过是一群地底鳄鱼罢了。白凝冰,你我合力,杀了这头熔岩鳄王!”

%74
“好。”白凝冰淡淡的答着,双眼中却战意磅礴,豪雄顿生。

%75
当初,他们从青茅山漂流出来的路上,也曾碰到过熔岩鳄王,远远避开。

%76
现在,他们今非昔比,都有四转修为。哪怕这熔岩鳄王,乃是千兽王,也阻挡不住两人的脚步!

%77
(ps:想出一个更佳的方案,所以修改了大纲,耗费了许多时间。负责任的讲,三叉山这章,将会很精彩。明天,继续。)

\end{this_body}

