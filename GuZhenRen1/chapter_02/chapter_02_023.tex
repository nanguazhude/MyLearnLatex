\newsection{白骨传承(上)}    %第二十三节:白骨传承(上)

\begin{this_body}

%1
密道并不长,方白二人走了不久,就来到一处大厅。

%2
纯白的大厅,都是骨头所制。在大厅中央,有一个大缸。

%3
大缸中,盛满了白色的液体,仿佛是牛奶,散发着香气。

%4
方源走近大缸,相关的记忆不禁浮现而出。

%5
按照前世百花所述,这大缸下方,应该联通着一个泉眼。

%6
此泉,乃是奶泉。

%7
泉水如奶,味道清甜可口。不仅是上佳的饮品,而且营养丰富,小孩喝了能助长发育,老人喝了能身强体壮。

%8
奶泉乃是白骨山特产。

%9
方源前世,百家迁徙到这里来之后,开发了五个奶泉眼。使得奶泉成为了百家的特产,不仅供应自身,而且还作为商品交易,每年都会吸引许多商队前来贸易。

%10
“这缸中有蛊……”方源说着,用目光示意白凝冰。

%11
虽然按照记忆,此处毫无危险。但方源生性谨慎,毕竟这不是自身经历,情报都是道听途说而得来的。

%12
有风险的事情,还是让别人来做吧。

%13
白凝冰冷哼一声,催起防御,探手如缸。

%14
“里面好多的蛊!”很快,她眉头一挑,露出吃惊的神情。

%15
将手收回来时,白凝冰的手中已经抓满了一把蛊。

%16
这些蛊,大小如常人食指,通体皆白。一头圆润,一头尖锐,如微型长枪。

%17
这便是骨枪蛊。

%18
“此蛊虽是一转,但是架不住量多。这个大缸中,几乎盛满了这种蛊虫。”白凝冰有些兴奋地道。

%19
骨枪蛊,就类似于古月山寨的月刃蛊。属于这道白骨传承的基础之物。

%20
前世百家发现了这处传承之后,族中大量的蛊师,都装备了此蛊。骨枪蛊也成了百家蛊师的一个特征。

%21
“你再翻翻看,这缸中应该还有二转的蛊。”方源站在一旁,神情平淡。

%22
白凝冰捞了几次,终于发现了二转蛊。

%23
此蛊和骨枪蛊类似,但是骨枪表面。有着螺旋刻纹——螺旋骨枪蛊。

%24
此蛊是骨枪蛊的进阶,激射出去,攻击更高,比骨枪蛊的直射多出一股钻劲。

%25
大缸中,骨枪蛊占据绝大多数。只有少部分的螺旋骨枪。

%26
“如此一来。我总算有一种常规稳定的进攻手段了。”方源捏着一只螺旋骨枪蛊,心道。

%27
焦雷土豆蛊很不稳定,需要种植在地上,别人不踩的话。种了也是白种。如果是记性糟糕的蛊师,忘记自己种下焦雷土豆蛊的地点,还会殃及自身。

%28
而且焦雷土豆蛊的使用,有条件限制。必须是在泥土中使用,土地越是肥沃越好。至于在白骨山这种特殊地方。焦雷土豆蛊就不能使用了。否则,方源不介意再来铺设一两个陷阱的。

%29
“可惜没有三转蛊。”白凝冰颇为遗憾,她选了一只螺旋骨枪蛊,揣在衣兜里,打算之后炼化。

%30
不过这种二转的蛊,对她来讲,只图一个新奇。真正用来战斗,并不能发挥出三转蛊师的实力。

%31
见大缸中毫无危险,方源开始行动。

%32
他一把把地捞出蛊虫。依靠着春秋蝉的气息,真元狂吐,顷刻炼化。

%33
“你,你这是……”白凝冰看得瞠目结舌。

%34
短短功夫,方源已经炼化了数十只蛊虫。并且他还在继续!

%35
因为春秋蝉,他能够瞬间炼化。又因为资质和天元宝莲,导致他真元的恢复速度大过炼化的消耗速度,使得他动作不停。

%36
这像是一场疯狂的表演!

%37
方源的空窍中多了两百只的骨枪蛊。二十多只的螺旋骨枪蛊。

%38
炼蛊向来是蛊师的一道难关,白凝冰虽然见过方源几次瞬间炼化蛊虫。但都没有这一次这么有视觉冲击感。

%39
在方源手中,炼化蛊虫仿佛是吃饭喝水一样,不,像翻手眨眼一样容易的事情。

%40
简直太轻而易举了!

%41
“他到底有何种底牌啊?”白凝冰心中十分惊疑,这一刻方源在她心中的形象,变得更加高深莫测起来。

%42
不过表面上,她却撇嘴,以平淡的口气道:“你一下子拥有这么多蛊,你养得活吗?”

%43
方源笑了笑:“当然养不活了。”

%44
骨枪蛊、螺旋骨枪蛊,都是以奶水为食。因此被养在这处奶缸之中。

%45
别看这奶泉水满满一缸,其实消耗巨大。皆因缸底联通了一处泉眼活水,才至如此。

%46
方源若要养活这么多的蛊,除非自己拥有一口奶泉。

%47
“虽然养不活,但我喜欢尽量多带一点,要不然可就便宜了百家那帮人。”方源笑了笑,指着大纲又道,“好了,你可以把剩下来的蛊都毁掉吧。”

%48
大缸中的蛊实在有些多,方源虽然炼化了很多,但考虑到一转空窍的承受力,因此还剩下不少。

%49
片刻之后,白凝冰神情复杂地看着一地的残碎蛊尸。她心中极为清楚,这些蛊虫的巨大价值。

%50
毁掉它们,如同摧毁一座元石堆成的小山。饶是白凝冰,也心疼不已。

%51
不过,若是留下来,就会被百家得去。与其资敌,倒不如毁掉它们。

%52
两人离开这处大厅,顺着另一个密道,走入第二个白骨大厅。

%53
大厅的中央,竖着三根白骨柱子。

%54
柱子的顶端,雕刻成人手,只是削去皮肉,剩下白骨。

%55
白骨人手中,各握着一只蛊虫。

%56
三根柱子,就有三只蛊虫沉眠着。

%57
方白二人走近,柱子表面刻着字,是关于这三只蛊虫的说明。

%58
“肋骨盾蛊、飞骨盾蛊、臂骨翼蛊……”白凝冰目光扫视,口中呢喃。

%59
很快,她目光一凝,看到一行刻字——“三选其一,心满意足。白骨山传承,留待更后人。”

%60
这意思很明显,就是只能选择一只蛊虫。把这机缘,让给以后的有缘人。

%61
肋骨盾,能令蛊师生出两排肋骨,护于胸前。防御力卓越。乃是三转蛊虫,优点是:除去起初生长阶段,需要大量真元。在此之后,不需要真元就能维持。类似天元宝莲,炼化之后。无需灌注真元。就能使用。

%62
飞骨盾蛊,用了之后,能形成三面飞旋的骨盾,面积小巧。悬浮在蛊师周围。

%63
臂骨翼蛊,则是前臂处生长出一对骨翼,扇动骨翼,能略微的增加移动速度。最主要的,是对出手速度的增幅。

%64
“肋骨盾能配合他先前的背甲蛊。形成一前一后的防御,但背甲蛊注定要被我淘汰的。只有一个肋骨盾,防御面太小了。我有了跳跳草,也不专职博斗,臂骨翼蛊就不大用得上。”

%65
方源思考了一下,选择了飞骨盾。

%66
他捏碎手骨,将飞骨盾蛊也炼化,收入空窍。

%67
至于其他两只蛊虫,却是一动不动。

%68
这是正道传承。此处考验有缘人的把持能力。若真是控制不住心中的贪婪欲念,取了三只蛊,后面的密道就会悄然改变。虽然不会有什么险恶的陷阱,但是收获会大大减少。

%69
正道传承,不同于魔道传承。

%70
通常而言。都是宅心仁厚的设计。蛊师有缘撞见就有收获。只是收获多少,不同罢了。

%71
白凝冰见方源没有动手,也不敢妄动,唯恐有什么机关。

%72
二人顺着密道。进入到第三个大厅。

%73
此厅再无其他密道,在洞壁处盘坐着一具人骨。

%74
人骨面前。摆放着一本大书。

%75
这本书,用骨头所制。长有一臂,宽有半臂,厚达八寸。

%76
方源示意白凝冰捡起来,见没有危险,这才接过到自己手中。

%77
这本书,在前世被百家的兄妹命名为“灰骨巨书”,里面记载着许多的炼蛊秘方,同时还有这道传承之主——灰骨才子的平生过往,以及设立传承的缘由等等。

%78
方源翻开一看,果真如此。

%79
这是货真价实的正道传承。

%80
书中最后言道:这具尸骨,正是灰骨才子本人。后来的有缘人若是有心,不妨拜祭一番,磕上三个响头。祭拜之后,劈开尸骨脑壳,可见一蛊。此蛊乃是灰骨才子的本命蛊,后来人得到它,要造福世间,匡扶正义云云。

%81
方源见此一笑,将灰骨巨书随手递给白凝冰,他跪倒在地,恭谨地磕了三个响头。

%82
这是货真价实的响头。

%83
额头碰到坚硬的地面,发出砰砰砰的三声闷响。

%84
白凝冰诧异,想不到方源还有这一面!

%85
方源叩首完毕,站起身来,大厅丝毫不见动静。

%86
他不以为意,微微带笑。

%87
这处大厅,再无其他密道,但却不是终点。此处考验后来人的心性,若是心性纯良,知道感恩,必会叩首。

%88
若是磕上货真价实的三个“响”头,那就会显出新的密道。

%89
但这只是一层。

%90
如果不禁磕了响头,并且一动不动这处尸骨,敬重前辈尸躯,那就会出现第二条密道!

%91
“前世百生、百花都叩首磕头,但百花怕疼,没有响动之声。是这百生,触动了第一层机关。但百生想要开颅取蛊,却被百花阻拦,要留给前辈一个清静。因此是百花触动了第二层机关。”

%92
方源浮想联翩,看了眼百花、百生。

%93
他们被白凝冰提在手中,仍旧昏迷着。

%94
白凝冰也看向这两人,叹气道:“看来我们接下来,还得靠这两个护身了。不过我很好奇,这灰骨才子的本命蛊到底是什么。我们开颅取蛊吧。”

%95
方源摇摇头:“这正是此处设计的精妙之处。好奇心能让人将这颅中蛊,想得十分美好,比看到实物更加心动,不要着急。”

%96
他话音刚落,一处骨壁往后一缩,滑动开来,露出一个全新的密道。

%97
“原来如此。”白凝冰似有所悟,正要动手,又被方源拦住。

%98
“这洞口虽然正确,却不是最有价值的一道,再等等。”

%99
等待的时间,是最漫长的。

%100
尤其是方白二人,还在被百家围追当中。

%101
足足等了半盏茶的功夫,白凝冰心中渐生焦躁,又一道暗门滑开,密道悄然出现。

%102
“哈哈,就是这个了。”方源大笑一声,走上前去,抬起一脚,将灰骨才子的尸体踏碎。

%103
出现了这处密道,这尸骨就没有用了。

%104
方源从颅中取出一蛊,是一只三转的骨刺蛊。

%105
(ps:更新有些迟了,不好意思,晚饭耽误的功夫有点多,抱歉!)

\end{this_body}

