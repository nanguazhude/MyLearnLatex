\newsection{仙鹤门,方正}    %第一百三十五节:仙鹤门,方正

\begin{this_body}

蛊师世界,广袤非凡。

东面,是沧海滔滔,海岛星罗棋布,点缀其中。名为东海。

西面,是戈壁沙漠,绿洲如珍珠,洒在漫天黄沙内里。号称西漠。

北面,是苍莽草地,人称北原。南面,是十万大山,俗称南疆。

而在东海、西漠、北原、南疆的包拢下,有一处中央地带,名为中洲。

中洲幅员辽阔,绵延亿里,元气最为充裕,门派林立。多少英雄豪杰,正道魔道隐藏其中,可谓人杰地灵。整体实力,较之东南西北,最为强盛。

在中洲南部,群山之上三万丈,云海之巅,苍穹之中,悬空着一座大山。

飞鹤山!

飞鹤山雄壮而又飞逸,悬空在玄白云海之中。

阳光透过云雾照射,山上绿树丛丛,满山黛绿欲滴。

山浪峰涛,层层叠叠。

每当狂风呼啸,山上云海如滚水翻腾。松柏竹林漫卷如涛,万鹤齐鸣。

铁喙飞鹤,丹火鹤,凤尾鹤,云烟飘渺鹤,星辰极光鹤……数万种飞鹤,或是盘旋,或是栖息在松树上,或是盘踞在山石上。可谓气象万千,巍然壮观。

飞鹤山上的万鹤,闻名中洲。而山上的蛊师,则名传天下。

这就是仙鹤门。

中洲十大门派之一,占据中洲的巅峰势力。

此时,在仙鹤门的门派比武场上,一场战斗已经进行到最为关键的时刻。

两位青年,相似的服饰,激战在一块。双方身影或进或退,时而纠缠一起,时而触电般乍分。

“太。太强了!”

“难以想象,交战的双方都只有二十几岁。”

场外,观战的众人对紧紧地盯着激战的双方。震惊、钦佩的神情,纷纷流露在脸上。

“孙元化师兄,是老牌强者。上次三年小考中的第一,他有这样的实力我并不吃惊。但是方正师弟,居然有着这般的实力,实在叫人惊讶!”

“不错。这次门派八年中考,方正师弟是一匹最大的黑马。谁也不会料到。他竟然能达到决赛。”

“这些年来,方正师弟默默无闻,简直像是一块最普通的山石。谁也不关注他,根本就不起眼。没有想到这次中考,他一飞冲天。名传仙鹤门上下。”

许多人都发出感慨和唏嘘,有艳羡,也有嫉妒。

方正一脸凝重,虎目放光,和孙元华打得难分难解。

这些年来,他已经长高,宽厚的背膀。狼背蜂腰,养成了一股沉凝精悍之气。

忽然,方正甩出一道碧绿狂风,逼退孙元华。趁势后退。

“孙师兄,你认输吧。”他开口道,平静的语气中充满了自信。

“小师弟,你有什么底牌。尽管翻出来吧。”孙元华笑了笑,同样自信的回道。

“那好。”方正忽然一吹口哨。哨音传播开去。很快,远方就传来群鹤的鸣叫。

众人循声望去,无数双眼睛倏的瞪大,无数张嘴猛地张开。

“这是铁喙飞鹤群!”

“天呐,这么多的飞鹤,我这是出现了幻觉了吗?”

“怎么可能?这竟是万兽王!方正修行到四转中阶,已经是天资纵横!但他居然能控得一支数量上万的鹤群,这是什么手段?”

众人一片哗然,心中均是十分震惊,感到难以置信。

就连场外的几位门派长老,都被惊得从座位上站立起来。

孙元化的脸上,顿时变得极为凝重。

铁喙飞鹤群,气势汹汹,阵势磅礴,让他感受到极大的压力。

不过,他并没有想放弃的打算。

他的双眼中涌现出一抹坚定的光芒:“方师弟,你的确厉害。不过这么多的飞鹤,你控制得过来吗?我还没有输,因为我也有一支鹤群!出来吧!”

飞鹤山有千万飞鹤,这样的地理优势,被仙鹤门人充分利用。

方正有鹤群,孙元华不愧是饱受关注的门派天才,也雪藏了一支鹤群。

听到孙元华心中的召唤,很快,就从天边飞出一支鹤群。

鹤群的规模和方正的铁喙飞鹤群形成鲜明的对比,只有数百只。

但孙元华控制的这些飞鹤,和铁喙飞鹤截然不同。它们的身上,大部分都覆盖着白色羽毛,但在双翅尖端,还有尾翼、鸟爪都呈现深重的蓝色,在阳光的映照下,绽放着金属光泽。

同时,它们在飞舞的过程中,还能看到在它们的身上,缠绕着一丝丝的蓝色电光。

“小心,这些飞鹤乃是幻电鹤。性情凶悍,单打独斗,普通的铁喙飞鹤绝不是对手。”方正的心中,忽然出现天鹤上人的声音。

“我知道了,师父!”方正立即回应一句,双目绽放精芒,一边紧紧地盯住这支幻电鹤群,一边调动自己手中的铁喙飞鹤群。

“要相撞了!”

“铁喙飞鹤群太庞大了,简直就像是一个大怪兽。孙师兄的鹤群,恐怕连给它塞牙缝都不够呢。”

“不,孙师兄还有胜机。鹤群交战不是单纯的比拼数量,还要比较双方的操纵能力。”

“孙师兄一直勤修苦练,操纵鹤群的能力在门派中,绝对是一流。现在就看方正的了。”

“老实讲,我不太看好方正。方正虽然是天才,但到底也是人。这些年修行到四转,又做师门任务,积攒了这套优秀的蛊虫组合,已经耗费了他巨大的精力。我不相信,他还有时间,还有能力将鹤群操纵好。”

众人议论纷纷,一个个心潮澎湃。

这样的大场面,在门派的三年小考,八年中考里,都不常见。只有十五年的大考,才会偶尔有。

方正虽然拥有数量庞大的鹤群。但并非得到所有人的看好。

现在的情景,就像是两军对垒。

方正有上万只的飞鹤,军队规模庞大,但数量多了,就很容易操纵不过来。他毕竟才是四转中阶,灵魂只能达到一定的强度,心力也有限。

而孙元化虽然飞鹤数量少,但却是一批精兵良将,指挥起来如臂使指。再加上他平时花费许多时间操练。在门派中他优秀的控鹤术都广为人知。

眼看着两支鹤群,就要在空中相撞。

就在这时!

孙元化忽然纵身一跃。

他催动空窍蛊虫,整个身体化作一道闪电。咔嚓一声,就逼到方正的近前。

方正没有想到孙元化会突然袭击。

这样一来,他孙元化就要陷入到铁喙飞鹤群的围杀中。把自己陷入极为危险的境地。

孙元化展开狂风暴雨般的攻势,简直像是发了疯一般,强度前所未有的威猛剧烈。

方正失了先手,被孙元化压在下风,只能全力抵挡。

两只鹤群在空中交锋,地面上,方正和孙元化在激烈的搏斗。

幻电鹤群数量虽少。但抱成一团,一路冲锋,铁爪撕裂无数铁喙飞鹤。

反观铁喙飞鹤群,数量众多。却一片混乱,像是一大蓬的无头苍蝇。

“铁喙飞鹤群,竟然阻挡不住这一小拨的幻电鹤!”

“孙师兄威武!他选择的战术,实在太恰当了。”

“没错。方正心力有限。指挥这么一大群飞鹤,简直就像是婴儿耍重锤。笨拙不堪。他现在又被孙师兄全力猛攻,自身难保,只能拼命防守,精神极度集中。哪有空余的心神,去顾及头顶上的飞鹤群。”

“但铁喙飞鹤的数量,实在是太庞大了。有许多百兽王,还有一些千兽王,甚至万兽王。幻电鹤群里只有三只百兽王,一只伤残的千鹤王。幻电鹤群为了避开这些兽王,左冲右突,即便杀伤了大量的普通铁喙飞鹤,但自身损员也十分惨重。”

“孙元化和方正打得难分难解,一攻一守,彼此熟悉至极,根本分不出胜负。现在就看这飞鹤的比拼结果了。”有长老看出了胜负的关键。

如果铁喙飞鹤群,侵吞了幻电鹤群,那么必然是方正胜利无疑。

但若是幻电鹤群冲出铁喙飞鹤的包围,赶下来帮助孙元化,那方正就是输家。

“坚持,坚持住。努力分出心神,照看一下铁喙飞鹤,将幻电鹤全部扼杀!你若能做到,这就是巨大的突破,对你将来将有极大裨益。”方正心中,天鹤上人的声音不断地临阵指导着。

方正竭尽全力,按照天鹤上人所说的去做。

但往往就在他踏上成功的边缘,都会被孙元化的攻势打断节奏。

孙元化到底年龄比方正大,操纵鹤群下过苦功。在他的分心操纵下,鹤群虽然损失惨重,但已经快要突破铁喙飞鹤的包围。

“方正师弟,想要击败我,你还差得远呢。”他冷笑一声,道。

这样的话语,被方正听到耳中,忽然就勾勒起了他记忆深处的一个画面。

当初,在南疆青茅山,古月山寨中,他和自己的亲生哥哥方源展开擂台战。

方源也是这么对他说的。

“不,我不能输!”

“我还要为家族,为我的舅父舅母,为我的族长,为青书大人报仇!”

“这些年来,我努力为了什么?我要站到哥哥的面前,将他击败。怎么可以现在就输给孙元化?”

方正的双眸中,似乎猛地燃烧起熊熊的火焰。

他强忍着脑仁的疼痛,分心他顾!

铁喙飞鹤群忽然一动,像是巨兽张开大口,将幻电鹤群顷刻吞没。

看到这一幕,孙元化面如土灰。

赢了!

全场掀起赞叹之声。

方正成为八年中考的第一名,击败孙元化,成为仙鹤门的精英弟子!

\end{this_body}

