\newsection{家宴}    %第七十节:家宴

\begin{this_body}

%1
这处并不宽敞的庭院,就是商燕飞设下的家宴场地。

%2
露天的酒宴,因为是在山体之内,因此不用担心什么刮风下雨。

%3
庭院并不奢华,也不雅致,甚至有些破败。

%4
庭院里摆放着十几个桌案,围成一圈,已经开始显得有些拥挤。

%5
桌案上放着些水果等小吃,还有竖着的标牌,表明这是谁的位置。

%6
已经有三人,到达这里,坐了下来。

%7
“大哥,不知道这次父亲大人召集我们,有什么事情。”商嘲风将一枚红色的水果扔进嘴里,一边含糊地问道。

%8
老大商囚牛正襟危坐着,正闭目养神,听了这话,他睁开一丝眼缝,声音低沉:“父亲大人闭关出来,想念我们,开设一场家宴,有什么奇怪的?”

%9
“大哥的话虽然有理,但父亲大人哪次开设家宴,没有重要的事情?你不觉得今天这些桌案,有点多么?”一旁,商蒲牢接口道。

%10
商嘲风嘿嘿笑了声,商囚牛却再次闭上双眼。

%11
商蒲(pu)牢目光闪烁了一下,他执掌风月区,管理大小青楼,消息最为灵通不过。其实已经隐隐听到风声,他正要继续说话,忽然耳朵一动:“有人来了。”

%12
三人的目光都转向小院的门扉。

%13
吱呀一声,院门被人推开,走进来三个人。

%14
魏央当先,而方白二人随后。

%15
这庭院方白二人先前已来过,正是当初召见的那个私宅。

%16
“此处私宅乃是当年族长大人。还是少主之一时,遭到其他几位少主的联合打压。族长大人以退为进。主动放弃少主之位,成为普通族人。那段落魄之时,就居住在这里。后来族长大人功成,为了警醒自己以及后辈,就将此处保留下来。历来的家宴,都是在此地召开。”

%17
魏央一边开门,一边介绍着。

%18
紧接着,他发现了院中的三人:“嗯。原来已有三位少主到了。”

%19
商囚牛,商嘲风以及商蒲牢,都纷纷站起身来,向魏央抱拳:“魏央家老,有礼了。”

%20
魏央是商燕飞的五大干将之一,商家重臣。但凡少主想要竞争少族长之位,都绕不过魏央的评定。

%21
“三位少主都好。这两位乃是族长大人今天邀请的贵客。”魏央拱了拱手,神情平淡。他是家老,地位比少主还要高一筹。又是重臣,不会去巴结这些少主。

%22
“二位,请这边坐。”魏央将方源和白凝冰引上各自的座位。

%23
商囚牛等三人面面相觑,均看出彼此眼中的惊疑。诧异和好奇。

%24
这是家宴,何时邀请过陌生的外人?

%25
这两人究竟是什么身份?居然做的位置,比我们还靠近父亲大人的主位。

%26
魏央也坐了下来,他满含微笑,接着道:“我向二位介绍一下。这是商囚牛,族长大人的长子。如今执掌商家的寄养场。这位是族长四子商嘲风,掌管商家城内的所有斗蛊场。这位是商蒲牢,风月区的青楼都由他负责。”

%27
商囚牛体格雄健,声音低沉,一看就是性情沉稳之人。他年龄最大,已近三十。

%28
商嘲风一头乱发,鼻梁很高,散发着狂野之气。

%29
商蒲牢则最为清秀,身子单薄,面色白皙,长有一对桃花眼,意态风流,显然是常年流连于花丛之中。

%30
“囚牛见过两位贵客。”商囚牛率先抱拳一礼。

%31
魏央没有主动介绍方白二人,三位少主都是精明的人,自然不会傻到追问。

%32
“三位少主有理,我乃黑土,这位是我的同伴白云。”方源介绍道。

%33
这两个名字,明显是假名。

%34
这更让三位少主有些摸不准方白二人的来历,只能打哈哈,把场面糊弄过去。

%35
快要临近晚宴,陆续有少主赶来。

%36
有管理赌石场的商貔貅(pixiu),酒楼绸庄的负责人商狻猊(suānni),管理拍卖场的商负屃(xi),执掌代练司的商赑屃(bixi)。还有方源已经熟悉的商睚眦。

%37
魏央介绍给方白二人,这些少主看到方白二人,均或多或少地流露出异色。

%38
他们一一坐下,人多话也跟着多了,小小的庭院渐渐热闹起来。

%39
快要临近开席之时,门扉忽然被推开,一位少主行色匆匆地闯了进来。

%40
这人身材高瘦,浓眉虎目,乃是商狴犴(bi’àn),掌管着商家城的城卫军。

%41
城卫军处理纠纷,协调矛盾,维持治安,最是繁忙不过。

%42
和方白二人客气了几句,商狴犴还未坐下,忽然主位上火焰一闪,现出商燕飞。

%43
商燕飞此次穿了一身白袍,袖口边角都镶有金边。一头鲜红的血发肆意散开,垂至腰间,配合英俊至极的面貌,形成他独特的气场和魅力。

%44
“儿等见过父亲大人。”众少主纷纷起身,然后半跪在地上,齐声道。

%45
“族长大人。”魏央站起身。

%46
同一时间,方白二人也起身行礼。

%47
“都坐。”商燕飞半躺在宽背座椅上,随意地挥了挥手。

%48
顿时,洒出一片绚烂的七彩华光,如雨滴,似云雾。

%49
华光落到众人的桌案上,化为一份份精美佳肴,小院内顿时菜香四溢。

%50
方源一眼便认出来,这是锦绣食盒蛊。

%51
商燕飞特有的五转蛊,专门用来储藏佳肴。佳肴放置在里面是什么样的状态,取出来便是什么样的状态。

%52
方源前世在商队里打拼,古月山寨灭亡后,他更是无依无靠。时值义天山突然崛起,魔道蛊师结成联盟。声势浩大,碰触到正道底线。

%53
各大家族联合起来。围攻义天山。

%54
联军首脑之一的商燕飞,犒赏三军,就用的这锦绣食盒蛊。

%55
只是一挥衣袖,数万人都有了美食犒劳,极大地振奋了联军士气。

%56
从那之后,锦绣食盒蛊就成为了商燕飞的一个标志,传为世人口中的趣谈。

%57
当时,方源靠着穿越者的优势。成为了底层的小头目,加入了一支行商队伍,负责押送物资供给正道联军。

%58
他亲眼看到商燕飞动用锦绣食盒蛊的盛况。

%59
漫天的七彩霞光,蒸腾灼照,绚烂夺目,映照天地,气象宏大。

%60
“想不到今生。我‘提前数年’就看到这锦绣食盒蛊,同时自己还成了商家的座上宾客。”方源暗生感慨。

%61
前世今生形成鲜明对比,这就是重生的巨大优势。

%62
而这重生的优势,来自春秋蝉,正是他整个前世努力奋斗的积累和成果。

%63
商燕飞布下菜肴,紧接着又对众子道:“今天有两位贵客在此。你们都要一一上前敬酒。囚牛,你是老大,你先来做个榜样。”

%64
父亲大人亲口吩咐下来,商家少主虽然心中疑惑,却无人敢怠慢。

%65
商囚牛立即起身。举起酒杯,声音低沉:“囚牛敬二位贵客。”

%66
刚说完。他便一仰脖子,就杯中酒一饮而尽。

%67
方源陪了一杯酒,白凝冰仍旧只喝水。

%68
这些少主当中,囚牛最大,已近三十,面貌上更显得相当老成,乍一看还以为是四十岁的。

%69
反观他的父亲商燕飞,宛若是二十来岁的小年轻。父子俩若站在一块对比,也是件蛮有趣的景象。

%70
“宴会开始之前,就已经和二位贵客聊过啦。初次见面,相谈甚欢。二人贵客若有空,可到我场下去玩玩,斗蛊绝对有趣呢。”商嘲风亦站起身来。

%71
方源嘴角微笑,虽然是初次见面,他却对商嘲风比较了解。

%72
此人好斗狠争胜,性子又有点阴鸠。前世差一点成为商家少主,曾经一度是商心慈的最大掣肘。

%73
“四哥的斗蛊太血腥了,还是美人歌舞怡情。”商蒲牢反驳一句,跟着向方白敬酒,一对桃花眼闪着光,“小子愿请二位贵客畅谈风月。”

%74
“有空的话,一定一定。”方源说着场面话,含笑喝酒。

%75
这场家宴,在他眼里,也算是一场名人宴。

%76
这些商家少主,大多数都在将来的南疆,有一番自己的演绎。

%77
商燕飞子女众多,这些人能够在激烈的竞争中脱颖而出,自然有过人的能力。堪称人中龙凤,石中美玉。

%78
且又各有不同性格,以及行事之风格,此时他们齐聚一堂,宛若明珠,散发着或明或暗的光辉,交相辉映。

%79
长子商囚牛,二子商睚眦,四子商嘲风,八子商蒲牢且不多说。

%80
九子商狻猊,狮口阔鼻,似乎用了什么蛊虫,每次呼吸时,从鼻腔都都喷出两股淡黄烟气,缭绕在身边。

%81
十子商赑屃,身材矮胖敦厚,似乎并不起眼。但方源却知,他身上蕴藏巨力。但比力气,比自己还要强上数倍有余。

%82
十二子商狴犴,后来和铁家联姻,成为了继铁血冷之后,南疆公认的第二神捕。

%83
十三子商负屃,颇有智计,正道联军围攻义天山时,他屡屡献计,令魔道蛊师吃尽苦头。

%84
还有二十一子商貔貅,年纪最小,大器晚成。许多年后,商家破落,他投身魔道,成为南疆中恶名远播的魔头。

%85
“父亲,对不起,我来晚啦。”敬酒的过程中,院门被推开,走进来一位少女。

%86
她是商螭吻,排行十六,掌管演武场。双眸灵动,面若桃花,性情活泼,是个俏美人。

%87
看着众子都敬了酒,最终商燕飞也举起手中的酒杯,面对方白二人:“感谢二位,一路上保护了心慈,才使得我新得了一位女儿。”

%88
此言一出,满堂皆惊。

\end{this_body}

