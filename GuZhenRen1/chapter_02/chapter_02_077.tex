\newsection{星辰石}    %第七十七节:星辰石

\begin{this_body}

%1
这是一块黑色的石头。

%2
足有脸盆大小,黑色的石头表面闪烁着点点蓝色的荧光。

%3
乍一眼望去,仿佛是黑幕中稀疏的星辰在闪耀。

%4
这是星辰石。

%5
但是和周围的星辰石不同,这颗星辰石四四方方,有点像砖头。

%6
然而就是这块埋没在角落里的石头,被人切出那只传奇蛊虫。而那个幸运儿,本来只是个无名之辈,就凭此蛊在力道上闯出威名,有了一番旁人羡慕的成就。

%7
方源站在这里,目光逡巡四周,有点明白为什么这颗顽石无人问津了。

%8
这是一家极豪华的大型赌石坊,生意很好。

%9
虽然是在第三内城,但身边却形成了人流。

%10
顽石按照分类,分有杂等、下等、中等、高等、特等五大区域。

%11
杂等区域,每块顽石平均只卖十几块元石。下等区域的顽石,平均售价是上百块。中等上千,高等上万,特等顽石往往数十万一枚。

%12
而这枚星辰石,就处于杂等区域。

%13
杂等顽石,是最不被看好的顽石,品相最差。绝大多数都是实心的石头。就算切出蛊来,也大多都是死蛊的残骸。

%14
尤其是自从魔道蛊师卫神经出世之后,杂等顽石的市场就受到了严重的冲击。

%15
卫神经最擅长造假,人称假大师。

%16
以大量出售造假的杂等顽石发家,各大家族几乎都有他的通缉令。在南疆属于人人喊打的过街老鼠。

%17
杂等顽石越来越不受欢迎。但也并非无人问津。

%18
有时候,一些蛊师心情好,出来逛街,也会小赌怡情。顺手买来几颗杂等顽石,赌的玩玩。或者是专业的赌徒,拿来试手气。

%19
毕竟杂等顽石,价格便宜,对于一些蛊师而言,赌输了也不心疼。

%20
但问题就在于,这里是第三内城。

%21
在这里的蛊师。绝大多数都是三转蛊师。

%22
三转已是家老级别,到达这一层次,大多都有些家产了。就算是试手,也会挑一些几率更大一点的低等顽石。

%23
赌杂等顽石,对他们来讲,有些掉档次。

%24
当然,像这样的大型赌石坊,人来人往,也会有些蛊师。将手伸向这些杂等顽石。

%25
然而……

%26
“呵呵呵,他们又怎么会想到。这块垫柜台的星辰石中,藏着传奇蛊虫呢。”方源心中暗笑不已。

%27
没有错,这块星辰石被当做踮脚石,压在柜台的一角下。

%28
人来人往,谁会注意杂等区域中,数十排的长条柜台中的一个普通柜台呢。

%29
这些柜台上,都摆满了杂等元石,一颗挨着一颗,一颗叠着一颗。廉价的。好像是地球上菜场里摆着的一摞摞的蔬菜瓜果。

%30
来到这里的蛊师,都会将目光集中在这些元石上。

%31
所以,这块星辰石从来无人问津,以至于渐渐沾满尘土,更不起眼官心计全文阅读。

%32
也不知道这块顽石是何时,被何人充作垫脚石,在这期间。更不知道顽石身边经过了多少人。

%33
直到有一天。

%34
一个混得并不如意的蛊师,走到这里的时候,被这个柜台的角绊了一下。

%35
按道理来讲,这只能怪这个蛊师走路不长眼。柜台是不会动的。走过许许多多的蛊师,也不见有谁被撞一下。

%36
就像是电线杆,它伫立在街道上,不招谁也不惹谁,但总会有人不长眼,走路的时候撞到电线杆上。你说这事情,能怪电线杆吗?

%37
然而不管是哪个世界,解决争端都不是通过讲道理,而是凭借实力。

%38
农夫养的一只公鸡报晓,吵得农夫睡不成懒觉。于是农夫一气之下,就将公鸡宰掉了。这似乎没什么不妥。

%39
不管哪个世界,大人物往往只要退让一小步,就能解决矛盾。但最终争端解决,通常是大人物分毫不退,而小人物付出巨大的牺牲。

%40
蛊师被绊了一下,尽管没有摔倒,但他很生气。他把怒气发泄在这块踮脚石上,好!既然你敢绊我,我就把你切掉!

%41
于是,蛊师买下这块石头,当众切解。

%42
然后,传奇之蛊现世。

%43
很戏剧性的一件事情。正是因为其戏剧性,导致在方源的前世里人人相传。

%44
而现在,这件戏剧性的事情,还没有发生。

%45
这块星辰石,还好端端地充作垫脚石,默默无语,安安静静。

%46
而方源的脚,距离它只有仅仅两尺。

%47
“这位客人,您是第一次来赌石吗?”一位伙计走了过来,他看方源双眼迷茫,并不晓得他是在用眼角余光观察脚下的这块石头。

%48
方源摸摸鼻子,答道:“差不多吧。”如果不算前世的经历的话,重生以来的确还只是第二次。

%49
菜鸟啊。

%50
好忽悠!

%51
伙计的笑容越发灿烂:“那您里面请。说实话,这里的石头价格最便宜,但便宜没好货啊。品相都不好,奇形怪状的。能赌出什么来?让我来为客人介绍一下吧。”

%52
“哦?那你说。”方源微微扬起眉头,跟着伙计走向赌石坊里面。

%53
“客人您知道顽石是怎么产生的吗?”伙计走在前面,回首问道。

%54
不待方源回答,他便自己答道:“俗话说,人是万物之灵,蛊是天地真精。人灵气十足,最擅创造,智慧最高,所以创造出了许多,大自然中并不存在的全新蛊虫。蛊是天地精髓,没有什么智慧,但小小的身躯中却蕴藏着大道的法则碎片。”

%55
“天地留一线生机,一些蛊虫受到重创,或者没有食物,就会有可能陷入沉眠。天长日久,它们的身边就会凝结成一层石衣。石衣增厚,就成了石皮。石皮再增厚,变成石肉。石肉经过天长日久的增厚凝实,最终转变成顽石。这些顽石紧紧地包裹着蛊虫,受到蛊虫身躯里法则碎片的影响,渐渐变得与众不同,和周围的土石差别很大。人们发现这些顽石,小心地收集起来,这就是您现在看到的顽石了。”

%56
方源点点头,一边走着,一边答道:“嗯,顽石的来历我也隐约听说过。”

%57
越往赌石坊里面走,蛊师就越多超级特工系统。

%58
许多蛊师都在挑石头,或者小声的交谈议论着。

%59
赌石坊和空旷,窃窃私语更显得静寂。

%60
伙计侃侃而谈:“客人既然想要赌,那就不能草率。咱们店虽然是赌石坊,但不坑人。总得让客人您赌的放心,赌的明白。咱们店里,主要经营老汉石,推车石,星辰石,寒冰石。每种石头,根据品相不同,分为五等,价格也不同。有的高达十多万,有的只有几块元石。”

%61
“客人您刚刚在的地方,就是最便宜的杂等区域,为什么不建议您在那买呢。因为杂等区域里的顽石,品相最不好。就举星辰石的例子吧。”

%62
“星辰石中,一般都是星光一系的蛊。这些蛊蕴藏着星辰的运转的法则,因此顽石渐渐形成了星光点点的斑纹。星类的蛊中,最常见的是一转星镖蛊,因此顽石会呈现出类似飞镖的形状。或者是星箭骨,顽石一般都是长条形,一端尖锐,一端扩散。还有流星蛊,顽石像圆球却带着尾巴。这些外形的石头,有蛊虫的可能性会很高。”

%63
“当然,除去外形,还有其他许多因素。比如星斑,星斑越多,就代表着里面有蛊的可能越大,蛊虫的转数越高。客人,您现在看到的,就是中等区的石头。你会发现,品相无疑更好。当然,这里的价格也稍微贵一点。”

%64
“嗯,的确如此。”方源随意附和道。

%65
伙计以一种自豪的语气道:“本店配有最优秀的解石师傅,至少都有二十年的丰富经验。其中有一位段大师,擅长酸液解石,解石已经有五十多年了。”

%66
“小人强烈建议客人您,选好了顽石之后,就在本店解石。解石师傅更专业,有一套专门的蛊虫用来解石。”

%67
“解石方法也要选择恰当。比如要解星辰石,最好用元磁法。解老汉石,就用酸液法。”

%68
“顽石中如果有蛊,都是虚弱至极的。选择不恰当的解石方法,很容易对蛊造成致命一击。到时候后悔就迟了。”

%69
“客人,如果您在高等、特等区选中石头,本店将免费为您解石。如果在中档、低档区,就要另付解石费了。至于杂等区,呵呵,那就不建议客人花钱请解石师傅了。通常请师傅的钱,比买石头的钱还要高出数倍。”

%70
曾经在青茅山,贾家商队来的时候,方源曾经自己解过紫金石。

%71
但他这种法子,堪称暴力解石。

%72
还得亏紫金石比较松软,癞土蛤蟆生命力顽强,当然还有方源经验丰富的缘故。

%73
暴力解石的方法不能广泛使用,很可能把活蛊杀死。

%74
方源要解石,就得借助这店中解石师傅的手段。

%75
逛了一圈,方源最终在下等区,随意选了几颗石头。老汉石,推车石,星辰石都有。

%76
伙计不由失望,巧舌如簧,极力鼓动,想劝说方源买上几颗更高档的,但方源怎可能上他的当。

%77
费了半天劲,就这么一个结果。伙计的声音变得有气无力:“这位客人,你有令牌吗?如果黄梨令牌以上,都会有价格方面优惠。请您出示一下。”

%78
然后下一刻,他神情骤变,他的嘴巴张的老大,眼珠子差点都要瞪掉下来。

%79
“紫荆令牌?我的老天!”

%80
(ps:今天状态不佳。)(未完待续。如果您喜欢这部作品,欢迎您来起点投推荐票、月票,您的支持,就是我最大的动力。)

\end{this_body}

