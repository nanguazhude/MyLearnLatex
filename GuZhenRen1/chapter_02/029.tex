\newsection{幽豹殉情}    %第二十九节:幽豹殉情

\begin{this_body}

%1
残阳如血,西天火烧云连绵一片。

%2
晚霞之下,飞行了大半天的无足鸟,开始缓缓下落。

%3
经过一系列超越极限的拉升俯冲,承受住火人的爆炸,它浑身都布满了裂纹,已经再无力飞翔了。

%4
砰。

%5
一声轰响,在方源尽量的操纵下,无足鸟终究坠落到了一片森林当中。

%6
一时间烟尘四起,群兽奔逃。

%7
“这是何处?”白凝冰跳下鸟背,环视周围。

%8
周身的树木又矮有粗,但是枝叶特别繁盛,不想白骨山上的骨树那样稀疏。这里的山林棵棵如华盖般,叶片皆是紫色。

%9
淡紫、暗紫、紫红、嫣紫……

%10
晚风吹来,放目远眺,形成一片紫色的浪潮。

%11
“紫色的山林……我们一路往北,按照行程来算,这里应该是紫幽山附近。”方源估摸着道。

%12
他眉宇间笼罩着一层忧色:“紫幽山白天安全,夜晚却极为危险。天快黑了,我们赶紧离开这里,找一处尽量安全的落脚点。”

%13
“也好。”白凝冰点头道。

%14
一个多时辰后,他们幸运地找到了一处山洞。

%15
山洞的原主,是一头布兜熊。

%16
这种熊的腹部,长着一个天然的育儿袋,类似袋鼠一般。

%17
干柴被烧得噼啪作响,篝火静静地燃烧着,架起的铁锅中肉汤已经沸腾,散发着浓烈的香气。

%18
肥嫩嫩的熊掌,也已经烧烤好了。除去这些,兜率花中还有从百家多带走的美食。

%19
两人大快朵颐,一直紧绷的心弦慢慢地放松下来。

%20
白凝冰忽然轻笑一声,幽蓝的双眸看向方源:“你看看,这就是报应啊。你烧死了那对兄妹,没过多久,自己也被烧成这个样子。”

%21
火光映照在方源的脸上,他脸上恐怖的伤势令其显得有些可怕和丑陋。若是胆小的女生,见到他的面目,恐怕当场就要吓得惊叫起来。

%22
方源却笑了笑,不以为意,甚至暗暗为此高兴。

%23
“幸好有肉白骨在手,你想要恢复原貌也不困难。只是把你全身烧伤的皮肉削掉,再用肉白骨治疗,就能长出全新的皮肉。不过,你现在的一转修为,可用不来肉白骨。你可以求我,兴许我大发慈悲,心生不忍,能为你治疗呢。”白凝冰不放过任何一个挖苦方源的机会。

%24
方源做了一个扬眉的动作,尽管他现在的眉毛已经被烧光。

%25
“为什么要治疗,这样的情况不是很好吗?”他笑起来,“我们把百家仅有的两个少主统统杀死,还戏耍了百家族长和家老,你觉得他们会放过我们吗?现在这样的伤势,正好省去了我改头换面的功夫。”

%26
地听肉耳草已经毁灭,方源如今残缺右耳。耳朵里有软骨,这伤势并非肉白骨能够治愈。但他即便有治疗手段,也情愿缺着一只耳朵来改变形象。

%27
昔年,有魔头白鳝子,被人捉去,关押在大牢中。装疯卖傻,把自己拉的屎涂在自己身上。甚至自削第五肢,成为太监。他的仇敌终于认为他彻底疯了,因此放松警惕,被他逃走。日后,他归来报复,将仇敌一家老小杀得干干净净。

%28
也有正道巨头武姬娘娘,年幼即位时被亲生姐姐夺权,只能隐忍。但姐姐嫉妒她的美色,处处与其为难,她便自削鼻梁,作践自己,终于赢得生存空间和成长的时间。十多年后,她推翻姐姐的统治,重新执掌大权。将姐姐五官都削去,令其求生不得求死不能。

%29
历来成大事者,都擅长隐忍,不耽于皮肉美色。

%30
这点不管是正道,还是魔道,不管是男性,还是女子,尽皆如此。

%31
武姬娘娘掌权之后,尽管有治疗手段,但从未将鼻梁复原,以此来自我警惕。因此武家乃南疆第一家族,盖压铁家、商家、飞家,霸主地位无人可以撼动!

%32
耽于皮肉美色者,几尽肤浅,难以成事。

%33
不管是这个世界,还是地球的历史,都充分说明此点。

%34
周幽王为了爱妃褒姒一笑,烽火戏诸侯,最后什么下场?众叛亲离,被蛮族斩杀。

%35
吕布之于貂蝉,吴王之于西施,项羽打仗居然把虞姬带着身边,呵呵,这些人又是什么下场?

%36
反观曹操矮小,孙膑残疾,司马迁宫刑……

%37
爱美之心,人皆有之。但成就和皮肉美色毫无相干,能有毅然割舍的心性,才是成就大业的基石。

%38
“事实上反倒是你,蓝眸银发,实在太耀眼了,需要改变一下。”方源打量了白凝冰全身,道。

%39
白凝冰冷哼一声,没有说话。

%40
方源接着道:“无足鸟受损,我们只是飞了数千里。虽然距离白家寨较为遥远,但是我们做了这个案子,百家一定会来缉拿我们。处境还是危险的。若是他们发布通缉悬赏,那我们今后的日子就更加困难了。”

%41
白凝冰皱眉思索了片刻,切了一声:“也罢,这身行头我也腻了,换换造型的话,想来也是一个精彩的体验。”

%42
接下来,二人开始总结此次的损失和收获。

%43
损失是有的。

%44
地听肉耳草、锯齿金蜈、背甲蛊,铁刺荆棘,隐鳞蛊,无足鸟,都在追击战中损毁了。

%45
不过对于方源来讲,能逃出生天,才是最重要的。

%46
活着,才有可能,才有希望。

%47
这是一切的基础。

%48
为了活下去,就算是舍弃掉春秋蝉又能如何?

%49
一句话,能舍能弃方是大丈夫!

%50
而收获呢?

%51
方源空窍中,大量的骨枪蛊、螺旋骨枪蛊。

%52
三转级数的飞骨盾,玉骨蛊、铁骨蛊,用于治疗的肉白骨,以及记载着各种合炼秘方的几本骨书。

%53
除此之外,还有在百家营地收获的清热蛊。

%54
当然最重要的,还是最后关头冒着风险,合炼成功的骨肉团圆蛊。

%55
相比较收获,那些损失并不严重!

%56
毕竟这是一道完整的传承,那个花酒行者虽然为五转强者,比四转的灰骨才子还要高一筹。但方源在花酒传承中的收获,却输给了这个白骨传承。

%57
原因无它,白骨传承是灰骨才子用心设计,筹谋良久。花酒传承却是仓促而行,灵机一动的产物。

%58
事实上,方源还只是走了白骨传承中的一条主线。还有许多岔道支线,另外肉囊秘阁当中,绝大部分的齿关都没有敲开。这些东西,都便宜了百家寨。

%59
他们掌握了此处,只要耗费时间,花费精力,必定能将整个传承都吞掉。

%60
“不过也无所谓了。计划中的蛊虫,都被我取来了。只要这骨肉团圆蛊能够发挥效果,就远远超过其他的东西。只是地听肉耳草损毁了,这就有些麻烦。”

%61
在方源的理念当中,只有实用才有价值。

%62
损失了锯齿金蜈,可以用螺旋骨枪勉强替代。没有了铁刺荆棘、背甲蛊,却还有天蓬蛊以及飞骨盾。唯有地听肉耳草损失了,在侦察方面,就出现了短板。

%63
以前是少治疗、少移动,现在这两方面差不多补齐了,侦察方面却出现了错漏。

%64
人生之事,不如意者十之八九啊。

%65
紫幽山的夜晚,比白日热闹得多。这一夜,方源和白凝冰轮番守夜,都没有睡好。

%66
时不时的,从洞外传来野兽的嘶吼声,扑杀之声。

%67
尤其是到了黎明时分,一场激战就在洞口附近展开,熟睡中的方源也被惊醒。

%68
这是两只千兽王大战!

%69
长有双翅的黑羽蟒,主动挑衅一只幽豹。

%70
双方你来我往,展开厮杀大战,动静闹得很大,声势惊人。

%71
幽豹是紫幽山的特有猛兽,它们身躯矫健,一身花斑紫皮,速度极快,常在山林中穿梭,留下一道道幽雅的魅影。捕猎时,悄无声息,猎物常常还未反应过来,就沦为它们的肚中餐。

%72
方白二人看得提心吊胆,他们可以说是被堵在洞中,难以脱身。

%73
随着时间的流逝,幽豹渐渐不敌,落入下风。

%74
这是一头怀孕的雌豹。

%75
幽豹向来是雄雌搭配,如今雌豹有孕,自然是雄豹出去猎食了。不想被黑羽蟒钻了空子。

%76
最终,雌豹死在黑羽蟒的缠绕之下。

%77
但黑羽蟒也未走脱,被归来的雄豹撞见,又一场生死激斗之后,雄豹杀死了凶手,却只得到雌豹冰冷的尸躯。

%78
黎明到来了。

%79
晨曦的光,照耀在幽豹一身优雅华美的皮毛上。

%80
但是雌豹却已经死亡。

%81
雄豹徘徊在雌豹的身边,发出呜咽的悲鸣声。它们俩的距离是如此之近,又是如此之遥远,相隔了生与死。

%82
“怎么还不走?”白凝冰暗暗叫苦。

%83
“放心吧,幽豹雄雌同心,一方死,另一方必然不会独活。”方源叹了口气,“我回去先睡个回笼觉。”

%84
他回到山洞深处睡觉,白凝冰则在洞口留守。

%85
雄豹徘徊了一番,匍匐下来,伸出舌头,舔舐雌豹的伤口。

%86
雌豹的伤口,一片漆黑,这是黑羽蟒的毒素所致。

%87
雄豹生活在这里,只要闻一闻,就能分辨出这种毒素。但如今,它却是不管不顾。

%88
最终,它明亮的双眼越来越暗淡,眼皮子越来越沉重。

%89
当等到中午时分,它也死了。静静地和雌豹躺在一起,漂亮的皮毛令它们俩看起来,仿佛是精美的工艺品。

%90
亲眼目睹整个过程,白凝冰也不由地叹了一口气。

%91
不久后,方源醒来,精神饱满地走出山洞,便看到白凝冰依靠在洞壁,正看着这两具幽豹的尸体发呆。

%92
“收获怎样?”方源问道。

%93
白凝冰耸耸肩,有些意兴阑珊:“该飞的蛊虫都飞走啦,我可没有捕蛊的手段。再说了,昨晚的战斗你不也看过了吗?那些蛊虫死的死,伤的伤,残留的都不是我们需要的。呵呵,若非如此,你这种人又怎可能主动去睡觉?”

%94
方源笑了笑:“虽然是两只千兽王,但身上蛊虫的确不怎么样。不过没有收获却也未必,嘿嘿。”

%95
说着,方源就向幽豹的尸体走去。(未完待续。如果您喜欢这部作品,欢迎您来138看书文学注册会员推荐该作品,您的支持,就是我最大的动力。)

\end{this_body}

