\newsection{法则碎片,大道之痕}    %第九十三节:法则碎片,大道之痕

\begin{this_body}

%1
密室中,方源盘坐在蒲团上,双目紧闭。

%2
淡银真元不断地催动,灌注到自力更生蛊上。

%3
自力更生蛊,位于方源的空窍当中,正在海水中潜游。

%4
它形如蟑螂,体型扁平,通体黑褐色。长丝触角,分有前后双翅。

%5
这是三转的治疗蛊,方源前段时期大换血时所购。

%6
自力更生蛊几乎可以说是,力修蛊师的绝配之蛊。

%7
蛊师的力量越强,自力更生蛊的治疗效果便越优秀。反之,力气越小,疗效就越差。

%8
自力更生,自力更生,便是从力气上获取出更新、生长的效能。

%9
方源已有双猪一熊一鳄的力量,此刻催动自力更生蛊,治疗效果已然和肉白骨持平。这还不是它的极限,当方源今后继续增添新的力量时,它的疗效会更超越肉白骨,再不断地往上拔高提升。

%10
但自力更生蛊也有弊端。

%11
那就是不能给他人治疗,只能给蛊师本身使用。

%12
这就大大限制了它的应用。

%13
再加上它价格昂贵,属于珍稀蛊,高达四万五千块元石!比剑影蛊还要贵得多。所以力道蛊师虽多,但是很少人有买得起的。

%14
和上古不同,如今力道蛊师大多都位于底层,干苦力买卖。低转境界的多,高转的少,出彩的更少。傲立巅峰的极为稀有,纵观南疆,也就武姬娘娘一人罢了。她还是凭借的上古力道传承。

%15
这是个力道没落的时代。

%16
现在,方源正在利用自力更生蛊,进行自我的疗伤。

%17
在和李好的一战中,他受了伤。

%18
导致受伤的罪魁祸首,不是李好,也不是背山蛤蟆。而是方源他自己。

%19
在激烈的战斗中,他攻势狂放威猛,持续良久。导致肌肉拉伤,肌腱断裂,就连天蓬蛊也十分萎靡,需要修养。

%20
幸好他事先用了铁骨蛊,铁骨铮铮,没有什么事情。否则连骨头都要断裂。

%21
常人发力过猛,就会拉伤肌肉。炮弹打得越多。炮口就越容易炸膛。

%22
任何的力量,都需要基石承载。这点,在力道蛊修身上,尤为如此。

%23
方源虽然有天蓬蛊,铁骨。但是力量太强,招数绵绵不断,肌肉承担不住,内脏、血液等等的负担,都很大。

%24
一战下来,他浑身都是暗伤。

%25
就连出汗,都渗出血渍。

%26
“用自力更生蛊疗伤。这已经是第五个时辰了。主要还是钢筋蛊效果,导致治疗艰难。”方源心中有分寸。

%27
在此之前,他一直用钢筋蛊改造自身,锻炼置身。导致肌肉中的大筋。都染上淡淡的黑色,透着金属光泽,变得更加坚韧。

%28
如今这些大筋不是拉伤,就是直接断裂。要续接生长,要恢复治疗。比常人的筋肉困难多了。

%29
为何如此?

%30
常人也许说不清楚,只知道有这样的现象。但方源心中清楚,他前世到底是蛊仙,知道这其中涉及到天地规则。

%31
人是万物之灵,蛊是天地真精。

%32
蛊虫身躯或大或小,蕴藏着天地大道的法则碎片。

%33
钢筋蛊的效果,是将蛊师身体中的筋腱,锻炼得如钢铁般坚韧。这就使得方源的大筋上,永久附着相应的法则之力,大道之痕。

%34
自力更生蛊要治疗,不仅是恢复本身的肉筋,更要克服、覆盖这层法则之力。

%35
好在这段法则,不和治疗法则对立冲突。否则就不是治疗方源,而是伤害方源了。

%36
同样的,棕熊本力蛊,黑白豕蛊、鳄力蛊也是如此。

%37
它们本身蕴藏着关于“气力”的一段大道法则碎片,附着在方源的体内,就成为兽力虚影潜伏起来。全力爆发之后,才会肉眼可见。

%38
什么是兽力虚影,究其本质,就是力量的道纹——天地大道的痕迹!

%39
再举个例子——毒誓蛊。

%40
它是关于约束的法则,布置在蛊师体内,同时约束对方。

%41
这道纹平时并不可见,但是在言而无信蛊的作用下,就能显露出来,然后被其消灭。

%42
言而无信蛊中蕴藏的法则,就是和毒誓蛊对立的。只是前者更强一些,所以能起到克制的作用。

%43
再拓展开来,为什么光虹蛊、移形蛊,需要蛊师本身洁净呢?

%44
也是如此原因。

%45
力道的法则,会干扰到光之法则,空间法则的运转。力之道纹如果太强大,就会知道导致后两者使用失败。

%46
“蛊师用蛊,其实利用的是各种大道法则的碎片。蛊是碎片的载体,是一种天然的工具。蛊师炼蛊,同样如此,也是融合法则,凝练法则。养蛊、用蛊、炼蛊……蛊修不是小道,是真正的大道!在修行中,蛊师师法自然,通晓天地。所以才有长生之法,才有成就不朽的希望。”

%47
方源心中早有明悟。

%48
……

%49
“因为困难多壮志,不教红尘惑坚心。今身暂且栖草头,它日狂歌踏山河。”

%50
书房中,商燕飞口中喃喃,咀嚼着方源长啸之诗,面容上饶有趣味。

%51
“因为困难重重,反而壮志凌云。红尘滚滚,也不能束缚自心。真是好大的气概!尤其是最后一句。这是把我商家城,当做草头吗?”

%52
“族长大人,依我看,方正应该说的是那对手李好,弱不禁打罢了。”一旁的魏央抱拳道。

%53
商燕飞摇摇头:“无妨,我还没有心胸狭小到这种程度。只是有些可惜啊,我错过了一场好戏。不过虽然没有亲眼看到,但也能想象得出方正大发神威,打出气势的情景。”

%54
商燕飞身居高位,眼界宽广,心胸开阔。对方源的志向,大度包容,并且加以欣赏。

%55
被商燕飞一提,魏央脑海中顿时浮现出当时生动的画面。语气唏嘘:“的确如此。当时那个场面,方正气势刚猛至极,盖压全场,震慑众人,无人出声!”

%56
商燕飞抚掌,从座位上站起身来,慢慢踱步到窗口:“千军易得,一将难求。可惜这个将,心高气傲。连我商家城都不放在眼里。也是,他虽然艰难困苦,但都闯了过来,并且屡屡收获,修为更是勇猛精进。这就把他的性格培养成了。”

%57
“族长大人。分析的极是。年轻人,总是有这种闯劲,无法无天。”魏央垂首,附和道。

%58
商燕飞看向窗外,渐渐地眯起双眼。

%59
方源把商家城比作草头,这样的志向大得吓人,绝对是雄心壮志。但商燕飞本人。并不在意。

%60
他是站在巅峰,俯瞰南疆的男人,自有胸怀容纳。而商家城,也不是被人比作草头。就真的是草头了。

%61
他在意的是前一句。

%62
因为困难多壮志,不教红尘惑坚心……

%63
这个世界上,志大才疏的人多的是。而有志向又有天赋才华的,很少。简直是万中无一。不仅有志向才华,更有坚强意志的。则简直是罕见!

%64
有志向,有才华不可怕,但再增添这股绝不屈服的钢铁意志,就令人担忧了。

%65
这样的人,在历史上,常被人称之为英雄、枭雄,或者奸雄。

%66
这种人,往往能改变历史,创造历史!

%67
这样的人,怎么会屈居人下呢?怎么能为商家所用呢?如果不能用,魔道有此子,必定绞动天下风云,掀起腥风血雨。远非正道之福啊……

%68
想到这里,商燕飞忽然开口:“我听说,方正夺了那个李然的机缘,为了心安,主动偿还二十万元石。现在还了多少了?”

%69
魏央便答道:“已经还了十三万,还差七万的缺口。不过依属下来看,也快了。”

%70
听了这个回答,商燕飞心中的担忧缓解了下来。

%71
他点点头:“也是。但凡志向宏伟者,从不拘小节,区区二十万元石算得了什么?呵呵呵。”

%72
又想到方正救商心慈的事情,商燕飞不由地失笑一声。

%73
此子性情有些可爱,行事很有原则,有恩必还,有仇必报。却非那毫无礼仪道德,无所顾忌的危险性格,有这点,就有招揽控制的希望。

%74
只是他现在正是风头正劲之时,招揽希望不大,须得缓缓图之。

%75
等到他被现实重挫,以情感动,替商心慈招揽到麾下,有这样的臂助,自己也就稍稍放心了。

%76
“也罢。方正,就让我看看你能走到哪一步吧……”商燕飞心中沉吟着。

%77
他目光犀利,对整个局面洞若观火。

%78
方源轰杀了李好,在整个演武区都引发了轰动。

%79
全力以赴蛊的强大,让人万分瞩目。

%80
很多人,开始意识到方源是个硬茬。但另外一方面,也更激起了其他强者的注意和觊觎。

%81
方源接下来会渡过一段安稳的时间,但紧接着,他的日子就不好过了。

%82
依他的心性,必定会撞得头破血流。到那时,他就会认识到自己的弱小。

%83
就先借着演武场之力,压一压他,磨一磨他。

%84
正如商燕飞所料的这样,方源迎来了不战而胜的轻松日子。

%85
李好是什么人?

%86
第五内城的演武场中,他是当之无愧的第一人,本来都要升到第四内城中去了。

%87
连他都惨死在方源的铁拳之下,还有什么人能制住方源?

%88
但是强行挑战,不能取消,原先想捡便宜的魔道蛊师们,只好捏着鼻子强行登场。

%89
刚开始,还会有人顾及着面子,试着和方源交手几次。

%90
但方源重伤数人,打死一人之后,他们再不敢托大了,往往战斗一开始,他们就主动认输。

%91
十九胜,二十场胜,二十一场胜……

%92
方源的上场,不再是战斗,而成了走过场,领元石。

%93
可以说是春风得意。

%94
在这样的情况下,一支百家的追缉队伍,来到了商家城脚下。

\end{this_body}

