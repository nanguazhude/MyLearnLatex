\newsection{硬气蛊}    %第一百一十六节:硬气蛊

\begin{this_body}

“那场战斗你看了吗?真是精彩啊!”

“白凝冰遭逢首败,炎突也胜的并不轻松。”

“花了二十块元石,原先还有一些心疼。但是看过之后,顿时觉得这钱花的太值了。”

白凝冰和炎突之战,如同巨石抛入湖泊,在商家城中引起广泛的反响。

大街小巷,都在谈论着这场战斗。

“姜还是老的辣,终究还是炎突大人获胜了。”

“但白凝冰虽败犹荣,能够打到这种程度,她的未来不可限量啊。”

“真是可惜,没有看到这场战斗。”

“白凝冰虽然失去了本命蛊,受了伤,但冰晶蛊容易补充,对她的实力造成不了太大的影响。”

“我现在十分期待方正和巨开碑的一战。”

“没错,这必将是一场龙争虎斗!”

方源和巨开碑之间的战斗,被安排在七天之后。因为白凝冰的突出表现,受到了无数人的关注。

而白凝冰却失踪了。

那场战斗之后,她并没有回楠秋苑。

“不会出什么事情吧。”商心慈对此表示担忧。

“放心吧,我了解她。她傲气嶙峋,却遭逢惨败,让她独处吧。”方源反过来安慰商心慈。

白凝冰虽是女儿身,却有一颗男儿心。

但凡男人都如雄狮苍狼,受了伤,会独自找一个无人的角落,默默的舔舐伤口。

而女人却不同,受到一些委屈,都会有倾述的欲望。她们渴望得到保护和安慰。

商心慈点点头,一双美眸温柔如水,关切地看向方源:“可不可以不要去战斗?那个巨开碑。可是和炎突齐名的人物啊。白云姐姐已经败了,失去了冰晶蛊。巨开碑可是力道蛊师,若是黑土哥哥你失去全力以赴蛊的话……”

白凝冰失去了冰晶蛊,可以得到补充。方源若失去全力以赴蛊,却无从弥补了。

方源淡淡而笑:“正是因为如此,我才更不能输。好了,接下来这几天,我都要闭关炼蛊。你先回去罢。”

巨开碑的实力,方源早就打探过了。经过白凝冰和炎突一战后。他又得到了更多的推测。

巨开碑是四转初阶的蛊师,和炎突不相上下。方源自我估量,若是按照现今的手段,要战胜巨开碑,只有三成的希望。

三成的胜机。看似不高。但事实上,作为考虑到双方修为相差一个大境界,三成已经相当不错了。

“若是炼蛊成功,有那蛊相助,我将有六成的胜机!但愿能炼蛊成功罢。”

方源钻到密室,开始炼蛊。

……

第一内城。

书房中,亮着柔和的光。

商燕飞静静地注视着眼前。一团彩色的烟气悬浮在半空中。彩烟翻滚,演绎着白凝冰和炎突一战的景象。

商燕飞将这场战斗从头看到尾,这才收了彩色烟气。

白凝冰输掉了。

失去了本命蛊,还受了伤。

根据风雨楼收集的情报显示。白凝冰如今正在素手医师那处养伤。

这样一来,她设计自己二子,也受到了惩罚和教训。

商燕飞闭上双眼,往后倚靠在椅背上。

如今白凝冰战败。没有通过考验,但她的天资、才情已经被公认。假以时日,必定能超越炎突,有一番大的成就。

这就是天才。

商燕飞也是爱才之人,惜才之人。

看到白凝冰的这番表现,更生出替商心慈招揽的渴望。

“白凝冰之后,就轮到方正了。不知道他能带给我什么样的惊喜。不过,听说似乎他的手中有一朵天元宝莲……”

天元宝莲这种东西,商家城的活宝门内也存着两棵。同时还有一棵天元宝君莲。

但是要取出这些宝莲,却要耗费更大代价。

活宝门把守着,就算商燕飞贵为家族之主,也必须得遵守祖宗定下来的规矩。

“方正是力道蛊师,若有天元宝莲,却无法发挥它的价值。不如给心慈。”

商燕飞沉思了一会儿,坐正身体,发出一道纸鹤密信。

纸鹤顺着暗道,飞到一处密室里头。

巨开碑、炎突二人旋即生出感应,汇集到密室当中。

“主上又有密信到了!”炎突展开密信,瞧了一眼,旋即就递给巨开碑,“这是给你的。”

巨开碑浏览一番,自言自语:“主上猜测方源的手中,可能有一株天元宝莲。如是此战我得胜,便叫我试着索要这株草蛊。若是没有这蛊,也不可选全力以赴蛊。”

说着这话,巨开碑的眉头微微皱起。

他原本是想索求方源的全力以赴蛊。这只传奇蛊,对于力道蛊师的他而言,具有极大的吸引力。

但偏偏,商燕飞传来这样的命令。

“主上是起了爱才之心,想保护那个方正成长起来。”炎突分析道。

抬眼看到巨开碑眉头紧锁,他又劝解安慰道:“巨兄,你要注意心态啊。我们二人已经是商家的隐家老,族长的命令就得遵守。再也不是之前,在外闯荡无拘无束,想干什么就干什么的日子了。”

巨开碑沉重地点点头,颇为感慨地道:“说起来,那段时间真是值得怀念。独行于天地,自由自在,不受人管束。”

“难道巨兄,还想重新成为魔道蛊师不成?”炎突语气严肃。

巨开碑嘿了一声:“炎老哥,你还不知道我吗,只是随口说说罢了。魔道蛊师虽然无拘无束,但是压力很大,风险也大,说不定哪天就横尸野外,连个收尸的人都没有啊。”

炎突这才脸色转缓。

巨开碑和炎突,都曾经是独行的魔道蛊师。

在南疆闯荡出一些名声,屹立不倒数十年。但他们越来越累。魔道蛊师的生活虽然极为自由,但是压力很大,为养蛊的食料,为元石,为自身安危等等操心。

两人渐渐地厌倦了这样的生活,但是又无脱身改变的勇气。

直到某一天,两人在野外遭遇。

魔道蛊师之间,极其缺乏信任。两人察觉到对方后,第一时间展开进攻。务必使得自己掌握主动。

哪知这一战,双方势均力敌,从白天打到黑夜。期间用智谋,安陷阱,竭尽全力。无所不用其极。

战到天亮时,双方都没有了力气,真元也消耗殆尽,浑身伤口满布。看着近在咫尺的对方,却无能力补上最后一刀,杀死对方。

当晨光同时照在两人的脸上时,两人都做出了同一个决定。

“我厌倦了这样的生活了。唉。如果这场战斗生还,我就去商家,改投正道。”炎突轻声喃喃。

“真是累啊,这场战斗之后。我就去商家争取那外姓家老的位置!”巨开碑则狠狠咒骂。

不知道为何,两人异口同声地说出来。

说完之后,彼此间面面相觑。

一阵静默之后,两人同时哈哈大笑。

这就是缘分。像是上天的安排。毫无信任感可言的两位资深魔道蛊师,在这一天。同时收获了一生的至交好友。

像是要彻底抛掉过去生活的阴影,他们开始改变,选择信任对方。

这种信任,是对彼此双方毫无保留的信任,两人从一个极端,走到另一个极端。

“不过话说回来了,炎老哥,你这次损失惨重,可是灰头土脸啊。”巨开碑挤眉弄眼,挪揄道。

旁人绝对想不到他会有这副表情。平时的巨开碑,严肃如铁,一丝不苟。但事实上,这只是他闯荡魔道时,练就的一副面具。

只有在炎突这个至交好友的面前,他才展现出真性情。

炎突冷哼一声,眉头渐渐舒展,叹着气道:“这个白凝冰不简单,总有一天会超越我们俩的。我也只是赢了一招半式而已,如果她是四转,结果就不好说了。”

巨开碑点点头:“你的那场战斗,我也乔装看了,的确是后生可畏。”

炎突拍拍巨开碑的肩膀:“那个方正,和白凝冰齐名,一同来到商家城,又住在一起,关系不一般。他们号称是演武双星,又有人称之为黑白双煞。白凝冰如此实力,方正必然不弱。巨老弟啊,当你最强时候就是你的最弱时刻,你要小心。”

巨开碑得意一笑:“老哥,你且看这是什么?”

说着,他从空窍中掏出一只蛊来。

此蛊如拳头大小,甲虫形态,黑色似铁。

“咦,硬气蛊!”炎突的脸上惊讶之后,涌现出喜色,“太好了,巨老弟。你动用杀招时,防御下降,有了此蛊,就弥补了短板。这蛊珍贵,并不常见,你要好好喂养啊。”

巨开碑点点头,收起硬气蛊后,发出一声叹息:“可惜,这只是一只硬气蛊,而非力气蛊。若是有力气蛊的话,我的战斗力就能得到升华,引起质变,暴涨两倍战力绰绰有余……”

炎突哈哈一笑:“你这是贪心不足蛇吞象!上古气道早已经消失,力气蛊已经绝迹,你能得到一只硬气蛊,已经是运道旺盛了,别不知足。”

巨开碑也笑起来:“说的也是。不过得陇望蜀,也是人之常情嘛,哈哈哈……”

上古气道,比力道还要更早出现。

气道蛊虫,能转换有无形,效果通常都十分奇妙。在上古,气道鼎盛的时候,十位蛊师中有八位都是气道。

然而,花开花落,盛极而衰,乃是自然规律法则,气道也不能免俗。

盛极一时之后,气道渐渐衰弱,然后被力道所取代。

白凝冰的朝气蛊,巨开碑的硬气蛊,方源的风气蛊,都和气道有关。

\end{this_body}

