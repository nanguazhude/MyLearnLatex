\newsection{修行得上瘾了!(第四更)}    %第三十二节:修行得上瘾了!(第四更)

\begin{this_body}

方源吓得连忙停止。

他虽然用过酒虫,但是酒虫也只是洗练真元,将真元品质拔生一个小境界。哪里用过差距如此巨大的雪银真元?

逼不得已,方源只好减少每次调用量,同时减缓冲刷洗练的力道。

“想不到这真元品质太高,也有弊端呢。”方源有些苦恼地道。

黑暗中,白凝冰听到这话,忍不住翻了一个白眼。

这是何等的机缘!

若是让其他人知道了,只怕要气得一巴掌把方源拍死。

这真是幸福的烦恼。

两个时辰之后,这六分的雪银真元才堪堪消耗完毕。

仅仅只是这一次的修行,赫然让粗糙的石膜,有了一种抛光过的明亮感。就好像是粗糙的玉石,经过粗加工后,初步显露出圆润的品质。

“快,快,再来一成真元。”方源催促道。

就在这样的修炼中过了一夜。

等到院子里的公鸡打鸣,窗外的天空开始微微透亮,方源睁开了双眼。

他的眼中神采熠熠,透着一股难以遮掩的兴奋感。

收回手掌,他狠狠地握成拳头:“一转巅峰了!”

就是这一夜的修行,让他从一转高阶,晋升成了一转巅峰。

何其之速!

方源修行以来,算上前世,也从未有此夜这般的舒爽、畅快!

几乎让他欲罢不能。

打个比方的话——

以前丙等资质的时候,修行若蜗牛爬。用了酒虫之后,能走路了。等甲等资质,又有天元宝莲,仿佛是在奔跑。

他娘的,用了这骨肉团圆蛊胡,简直是插了两个翅膀在飞啊!

白凝冰的神色,亦是发自内心的震动。

她冰雪聪明,很快意识到此蛊还有另一个更大的作用。

“此蛊了得,恐怕还能冲刺大境界!”

酒虫只能提升小境界,冲刺大境界时,得靠蛊师自己的真元。因此资质凸显重要。

但若有了这骨肉团圆蛊,冲刺大境界时,能借来他人的空窍。

这是何等的优势!

“有此蛊白骨山绝对物超所值,哪怕牺牲了天元宝莲,也是值当的。”

这时,晨光扫进这间农房,白凝冰喟然长叹一声:“天下英才何其多也!”

灰骨才子不过是四转蛊师,却能研发出骨肉团圆蛊来,真是奇才!

但凡有称号的蛊师,都有过人之处。相比较而言,方白二人如今还都是无名之辈呢。

……

太阳正烈,田地中。

“大娘,你就好好歇着吧。俺来帮你!”方源一把夺过老婆婆手中的锄头。

老婆婆当然争不过年轻力壮的方源,手中东西被夺,她却更加高兴,笑得咧开了嘴:“哎呀,你这小伙儿上哪里找去哟。”

距离第一次双修,时间已是过去了两天。

原本计划是住宿一晚就走,但自从用了骨肉团圆蛊后,方源享受到修为突飞猛进的快感,就决定稍稍改变一下计划,至少将修为突破到二转再走。

反正那块石头被发现,还有两三年的光阴,也不急于一时。

白凝冰也没有反对,二转和一转之间的差距很大。成为二转蛊师,对接下来的行程也会有巨大的帮助。

于是他们俩就赖在村子里,不走了。

老婆婆也没有赶他们。

事实上,她巴不得方白二人永远留下来才好呢。做事这般勤快,一个嘴虽然笨点,一个长得丑点,但都是老实孩子啊。

方白二人白天干点活,他们一个身怀双猪之力,一个拥有一鳄之力,在凡人中堪称神力。做些农活,简直是小菜一碟。况且一个老太婆家,能有什么粗重的农活?

每天方白二人只睡眠个把时辰,但仍旧精神奕奕。

这样的生活状态,和在野外跋涉的风餐露宿、朝不保夕相比起来,简直是天堂一般。

方源抓紧一切时间,扑在修行上,简直是上瘾了般。

依靠骨肉团圆蛊,他的修行速度,用一个成语来形容,便是一日千里!

第一夜,他就晋升到一转巅峰。这些天来,他有意稳扎稳打,空窍四壁已经晶莹剔透。按照这种速度算计,恐怕过不了几天,他就能突破到二转上去。

随着修行次数增多,骨肉团圆蛊在方源心中的地位极速上升,已经超越了血颅蛊、天元宝莲,仅次于春秋蝉了。

酒虫不必说了,只能提升一个小境界,和骨肉团圆蛊没法比。

血颅蛊呢,投资太多,回报时间太长。天元宝莲虽好,却只能给单独的蛊师提供帮助。

对于骨肉团圆蛊,按照白凝冰的赞语,这是一种可以改变天下格局的蛊。

这话一点都不夸张,深得方源的认同。

骨肉团圆蛊位阶低,即便是一二转的蛊师,也能够使用。它虽然合炼条件不少,但是标准真心不高,凭借一个家族的底蕴,想要炼出这蛊还不容易?

这蛊对于独行侠没有吸引力,对于家族、门派却是神器!

以此途径,让家族长辈来提携后辈,能将培养蛊师的时间、资金缩小多少倍!

有了骨肉团圆蛊,已经不是培养蛊师,而是批量制造蛊师。

这就是天与地的差距。

因为此蛊,方源在昨天甚至动摇过念头,有改变重生大计的冲动。想要将单打独斗,转为建立势力。

说起来,体制组织这玩意只是一个工具,拿来用罢了。

方源前世就建立过血翼魔教,今生若要组建势力,更可谓轻车熟路。

不过建立势力,最主要的目的只有一个,那就是——霸占地盘,圈占资源,依靠层层控制,来集齐他人之力供自己修行。

蛊师修行,如逆水行舟,更是个积累的过程,自然需要资源。

因此五转蛊师及其以下,若有组织的帮助,必定更易于修行。等到了六转及以上,引发质变,超凡成仙,天地伟力集中自身,那组织就没有用。

知道以上这点,就很好理解方源的计划了。

方源前世为了筹集资源,这才建立了血翼魔教。这世重生了,知道各种密藏传承,独吞这些资源,完全能更快更自由地冲刺到六转。何必多此一举,再耗费大量的时间精力,来组建什么势力呢?

不过现在得了骨肉团圆蛊,若是组建势力的话,却比之前设想要方便快捷得多。投入更少,见效却更快,批量制造蛊师,能迅速地形成一股力量。

然而方源又仔细斟酌了一番,直到今天早晨,关于组建势力的冲动已经完全消散了。

如果真的靠骨肉团圆蛊,来组建势力,那他就是绝对自找死路了。

这样的神器,一旦暴露出来,必定会引起所有大势力的觊觎。

别说他白手起家,就算是一族之长,有些根基,也不敢大肆运用骨肉团圆蛊。

太招人嫉妒了!

哪怕就是武家,南疆的第一家族,恐怕也不敢自己单独使用。

大规模运用骨肉团圆蛊的条件,方源估算了一下,至少是武家、飞家、铁家、商家这些庞大势力,形成统一联盟。才能抵挡住天下人的贪欲。

“我若要组建势力,恐怕刚有起色,就要遭到围歼。到那时,哪怕自己侥幸逃脱,辛苦组建的基业也必定付之东流了。到最后,反而连累自己成为丧家之犬,人人追捕。”

方源完全冷静下来,知道此路完全是死路一条,又坚定了此生单打独斗的大战略。

“不过说起来,百花百生也够智慧。在前世,他们虽然利用骨肉团圆蛊,成长为正道双星。却一直没有将这种力量,运用于百家势力的组建。在加上他们有意的误导,因此虽然传出了骨肉团圆蛊的名声,却令世人大大低估了这蛊的作用。”

方源现在回想起来,忽然就察觉到这对正道双星的智慧。白骨山此行,能够将这对天才兄妹扼杀在萌芽状态,也算是一个巨大收获。

但方源转念又一想:“这骨肉团圆蛊,只能合炼的两个人使用。百生、百花一定是用的卷轴上的合炼方法。没有我的篡改之法,合炼骨肉团圆蛊的难度要大大增加。也许他们根本没有想到这个前景也说不定……”

琢磨得多了,方源也感觉脑仁疼。

他一边在烈日下锄地,一边摇摇头,将发散开来的思绪收拢回来。

毕竟这些东西已经不重要了,想了也无用。

“可惜,我还没有得到才思蛊、彗心蛊、玲珑蛊、思绪如电蛊等,方便思考的蛊虫。再不济,有只书虫也算凑合。看来只有到商家城,再行采购了。”

老婆婆的田地本来就少,方源很快就锄了一半,白凝冰看得有趣,就主动把活揽过去。

她原先在青茅山,也看过农夫锄田。但那时她还浑浑噩噩,如今身临其境,却想体会其中的精彩。

到了傍晚十分,他俩在老婆婆家的门前吃饭。

“你们一定饿了,锅里还有番薯,一定要管饱!”老婆婆笑眯眯地盛了两大碗的米饭。

这米饭自然不是饭袋草里的精米,而是粗粮。

但方白两个都不是挑剔的人。

“你们慢慢吃,大娘有好事要告诉你们俩个。”饭桌上,老婆婆笑得有些神秘。(未完待续。如果您喜欢这部作品,欢迎您来138看书文学注册会员推荐该作品,您的支持,就是我最大的动力。)

\end{this_body}

