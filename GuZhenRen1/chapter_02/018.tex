\newsection{授人以柄(第一更)}    %第十八节:授人以柄(第一更)

\begin{this_body}

吼!

山林中,一只肥硕的黑熊,人立起来,足有两米高。\{请在百度搜索<strong>138看书</strong>,首发全文字无弹窗阅读\}

它向方源和白凝冰咆哮着,但两人却无动于衷。

黑熊愤怒了,四肢着地,向两位少年飞奔而来。

别看熊的样子笨重,但事实上它们的奔跑速度相当的快。通常是人类的两倍。

眼看黑熊距离自己只剩下五十步左右,但方源的嘴角却露出得逞的微笑。

砰。

一声炸响,泥土飞溅。

黑熊发出惨嚎,如遭到当头棒喝,冲势顿时被遏止。

遭到莫名其妙的攻击,让它十分愤怒。迅速调整方向,再次向方源冲来。

但刚冲出十步,地面上又发生爆炸。

吼!

黑熊胸腹被炸得血糊一片,它双眼通红,愤怒到了极点,再次扑来。

“野兽就是野兽,缺少智慧啊。”方源悠然一叹,转身后退。

黑熊在后面锲而不舍地追击,但每走几步,就会引爆一次爆炸。

又冲了数十步,黑熊浑身是伤,皮毛没有一丝完好。四肢都有残疾,一瘸一拐,再不复刚刚的凶猛姿态。

它的愤怒消褪了,求生的本能占据了上风。

尽管方源站在它面前,不超过二十步的距离。但它选择了退却。

但方源早就算到它的撤退方向,在路上挖下了一个深坑,同时埋下了至少五只焦雷土豆蛊。

轰!

一声巨响,战斗结束。

同时,在营帐中,一缕烟气不断沉降悬浮。

烟气中,画面闪烁,如实同步地显现着方源战斗的整个过程。

“陌行家老,你怎么看?”看到战斗结束,百家族长出声道。

整个营帐中,只有她和百陌行两人。

“如果没有看错,这个古月家的少主,应该用的是焦雷土豆蛊吧?这种蛊是一次性消耗品,埋在地上,吸取地力而成长,然后受到震荡而引爆。在二转蛊虫中,威力不俗。不过此蛊在白骨山,就要受到极大的削弱了。白骨山上没有泥土,就连山石都是白骨。焦雷土豆蛊根本就种不下去。”百陌行琢磨着道。

百家族长却微微摇头:“你分析的不错,但我重点不是看的这些。你有没有发现,从埋设焦雷土豆蛊,到战斗结束,都是方正一个人**完成的。他身边明明有三转的护卫,自己用二转的焦雷土豆蛊,又十分吃力。每埋下一颗,就要利用元石恢复真元。但他却一直坚持自己动手。这说明什么呢?”

百陌行双眼微微一亮:“老朽明白了。这个方正为人正直,并不是那种偷奸耍滑之辈。他答应参加狩猎,哪怕麻烦劳累,也不想动用外力作弊。”

“偷奸耍滑之辈,往往心志不坚。正直之徒,常常坚定不移。我们要从这两个人的身上,套出元泉的位置,最好的方法就是旁敲侧击,以智取胜。呵呵,我现在对昨晚计划,更有信心了。”百家族长笑道。

……

“在下幸不辱命。”半柱香后,方源将一张破烂不堪的熊皮,交到百家族长的面前。

“呵呵呵,短短工夫,贤侄就杀了一只成年黑熊,不愧是古月一族的少主。”百家族长脸上露出一丝恰如其分的惊讶,旋即转为笑颜。

“贤侄不妨回去休整一下,青铜舍利蛊稍后就送到。”

“谢族长,晚辈告退。”

方源和白凝冰退出中央大帐,回到昨晚的住处。

片刻之后,便有蛊师送来青铜舍利蛊。

方源取了,当即就在营帐中用掉,修为从中阶晋升到高阶。

蛊师的小境界,容易突破,是水磨工夫。但大境界却需要资质支撑。

舍利蛊、石窍蛊等等,很多的蛊虫均能让蛊师舍掉水磨工夫的时间,在短时间内拔高修为。

然而一转高阶,仍旧只是一转,这些微的提升,实在难以对如今的局面有什么改变。

到了晚上,百家族长又在营帐中,设下晚宴,邀请方白二人。

百家寨子有传统,狩猎大比期间,天天都有篝火晚宴。超大型的篝火晚宴,在营地中露天举行。而中央营帐的小型宴会,则只会邀请大比的前几名。

方白二人因为身份不同,仍旧是座上宾客。

“来,我为贤侄介绍一下我族的希望之星。你们年轻人可以好好亲近亲近。”席间,百陌行挑起话头。

在中央大帐中坐着的,有四位青年。

两男两女,无一例外,均是三转修为。

其中一位男子,正是百陌行的侄子,名为百陌亭,身材瘦削,是第一天狩猎后的第三名。

两位女子,一位名叫百草率,长得相当草率,却是第四。一名叫做百莲,容貌清丽,肌肤白皙,睫毛又浓又密,有一股清新自然的气质,是百家公认的族花。

两女相邻而坐,形成鲜明的对比。

“百战猎,见过两位贵客。”最后一位青年男蛊师,主动开口,抢过百陌行的话头。

他身材魁梧,含有傲气,战意腾腾。目光扫视了方白二人,先在方源的身上顿了一顿,露出一丝不屑的笑。然而目光紧紧地盯住白凝冰。

白凝冰如冰雪仙子,银发蓝眸,论姿容美貌还盖过百莲一头。更关键是,她是三转巅峰修为,引发了百战猎的兴趣。

他哼了一声:“看来你们古月一族中,是阴盛阳衰吗?”

白凝冰面色如冰,丝毫未变。

方源的脸色则冷下来,有些难看。

百家族长打哈哈道:“这是我们族中,年轻一辈的第一高手。贤侄勿怪,说话向来如此。”

“不敢。”方源扯动嘴角,面向百家女族长,“战猎兄的确是人中龙虎,方正佩服。”

他的语气复杂,表现得恰到好处。有些寄人篱下的忍耐,有自身修为薄弱的无奈,还有年轻人不甘心的傲气。

就连白凝冰都不禁为之侧目。

百战猎冷哼一声,方源心中不住地冷笑。

他知道百家的困窘境况,但百家却不知道他的真正底细。局面上虽然对他不利,但方源却牢牢占据着情报上的优势。

“如何利用好这个优势,就是脱身的关键。青铜舍利蛊是个好兆头,这表明百家忌惮那支不存在的古月大部队。不想硬来,想用计谋诓骗。这个百战猎是否就是百家的下一步棋呢?细细品味,他刚刚的话,措辞似乎过于强硬了。”

“如果他真是百家的下一步棋,倒有些麻烦呢。不如我亲自将‘把柄’交给对方,把‘弱点’呈现出去……”

若是让百家这般随意地搭建陷阱,方源无疑会越来越被动。

如此一来,倒不如授人以柄,故意暴露出一些虚假的弱点,争取一些变相的“主动”。

想到这里,方源目光一扫营帐,顿时计上心来。

他看向对面的百莲。

盯得久了,百莲似乎察觉到方源的目光,方源便转过眸子,看向别处。

酒宴进行着,方源时不时地偷瞄百莲,但又避免和百莲的目光接触。

带到酒宴后半程,方源不断偷看,次数越来越频繁。

这幕情景落在百家族长,和一些家老的眼中。

家老们眼中流露出笑意。

少年慕艾,人之常情。百莲是百家的族花,吸引古月家少主的目光,是很自然的事情。

酒宴结束之后,百陌行就兴冲冲地求见族长:“族长,酒宴上有一幕,你可看到了吗?”

百家族长含笑点头:“且让我再谋划一番。”

一夜无话。

到了狩猎大会的第二日,百家族长又招来方源,让方源狩猎一条地角犀。

方源故技重施,利用焦雷土豆蛊,炸翻了地角犀,并带回了犀牛角。

百家族长夸赞一番后,便奖励了一只清热蛊。

清热蛊如同甲虫化石,半透明的玉石质地。握在手心中,有一股清凉之气。

此蛊专用来解毒,是二转的治疗蛊。

方源得了此蛊,总算是弥补了最大的短板。

到了当天的篝火晚宴。

“这是我的儿子,这是我的女儿。百生、百花,你们站起来,向这位哥哥敬一杯酒吧。”百家族长说道。

一对胞胎兄妹,便站起来,小大人般模样举起酒杯,一齐道:“百生(百花),敬古月少主酒。”

他们微微鞠躬,神态肃穆,显示出良好的教养,一点都没有孩童的顽皮之态。

方源微微一愣,不禁细细打量这对兄妹。

按照前世的记忆,眼前的这两个孩童,将成为正道的双子星,在一段时间内风头无两。最终双双达到五转,将百家寨的势力推到历史上所没有的高度。

同时,他们也是白骨山传承的继承者。其中,百生更是百家今后的族长。

一个家族中,族长之位通常是由族长的亲生儿女继承。但像古月山寨这般,族长膝下没有儿女,则会从纯正血脉的族人后辈中择优录选。

人总是有个成长的过程的。百生百花这对胞兄妹,可能在未来变成一方豪雄。不过现在这两人,年龄还太小,连学员都算不上。

方源将目光收回,注意力又转移到百莲的身上。

宴会继续着。

期间,方源不断地偷瞄百莲。百战猎找茬,语气越来越不客气。百陌行的侄子百陌亭,则在偷看白凝冰。

(ps:晚上九点半有第二更。)(未完待续。如果您喜欢这部作品,欢迎您来138看书文学注册会员推荐该作品,您的支持,就是我最大的动力。)

------------

\end{this_body}

