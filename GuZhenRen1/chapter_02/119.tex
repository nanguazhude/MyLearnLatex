\newsection{方源vs巨开碑(中)}    %第一百一十九节:方源vs巨开碑(中)

\begin{this_body}

“两个兽力虚影……我刚刚没有看错?”场外,炎突也不禁瞪大了双眼。

“能同时打出两个兽力虚影,他用了什么蛊?”巨开碑面色变得凝重,双目紧紧地盯住方源,忽然瞳孔一缩,“来了!”

直撞蛊!

方源面色冷酷,双眼如黑潭深不可测,他向巨开碑发动冲锋。

与此同时,在他的头顶半空处,升腾起骏马和山猪的虚影。

山猪之力让他的冲撞力量变得更强,而骏马之力则让方源一下子就提速,冲锋的速度更快。

双力加持,方源仿佛化身凶蛮的山猪,奔腾的烈马,身形在巨开碑的瞳孔中不断放大。

龙行虎步蛊!

面对方源的强势攻击,巨开碑身形一跃,龙吟虎啸声再起。

这次他主动避让,方源和他擦肩而过。

直撞蛊的缺陷就在于此,冲锋时是一条直线,很容易就被对手判断出来,从而从容躲避,导致攻击无功。

“巨开碑主动躲闪了!”

“他向来猛打硬冲,很少躲闪的。开战以来,这是他第一次躲闪。”

“方正能同时打出两头兽力虚影,巨开碑的惯力蛊却还要重新蓄力,躲避开来,是明智的举动。”

“方正还是太嫩了,虽然有好蛊虫在手上,但是战斗经验远不如巨开碑。这次攻击处理的就不理想……呃。”

场外的这人,评价的话还未说完。场上异变突生。

方源和巨开碑擦肩而过之后,忽然停止催动全力以赴蛊。双兽影消散,冲撞之力顿时暴降到低谷。随后他撞到一根黑石柱上。

黑石柱倒塌下来,方源冲势顿止。

横冲蛊!

方源侧身横冲,双拳直捣。棕熊和白象两大虚影,陡然升空。

距离太近,巨开碑措手不及,被狠狠击中,身体高高的抛飞出去!

刚刚评价方源“太嫩”的那人。哑然无语。

许多人的眼中,都放出光亮。

方源这一次攻击,看似普通。实际上包含了许多东西,打得分外精彩。

全力以赴蛊可以催动,爆发出强大攻势,当然也可以撤销。

催动和撤销之间。不断转换。让方源收发自如,灵活至极。

同时他又利用了地形,及时停住了冲势。

巨开碑被方源一击而中,也不是他不聪明,而是他用惯了惯力蛊,丰富的经验反而让他有了思维盲区。

方源再度冲上去!

巨开碑奋起抵抗,但方源两头兽力虚影同时爆发,狂猛得仿佛怒狮。凶悍的叫旁观者都看得心惊胆战。

巨开碑成了沙包,竟然被打得毫无还手之力。憋住一口气,疲于防守。

“天呐,方正以区区三转巅峰,居然压制住了巨开碑!”

“难以想象的一幕,竟然在我的眼前发生了。”

“全力以赴蛊到底是传奇蛊,威能妙用,就算是四转惯力蛊,也不能媲美!”

“方正已经将全力以赴蛊用得出神入化了,他虽然没有丰富的战斗经验,但是才情天赋足以弥补这一切。”

场外一片嘈杂之声。

轰!

烟尘翻腾而起,巨开碑高大的身躯,仿佛是飞起来的麻袋,连续撞到数根粗大的黑石柱子。

噗。

他忍不住再喷一口鲜血,挣扎着要爬起身,但方源已经扑杀而来,不给他任何喘息之机。

脚下一踏,骏马、青牛两道兽力虚影,同时出现。

巨开碑就地一滚,狼狈地躲开这一击。

方源的右脚狠狠地踩在地上,发出一声闷响,土石飞裂,整个演武场都发出微微的震动。

在他的脚下,一个明显的凹坑顿时呈现出来。

直撞蛊!

方源再度展开冲锋,棕熊和白象的虚影同时闪现。

“可恶啊!”这次巨开碑已来不及躲闪,只好咬牙巨臂格挡。

吼吼!

从他的身上,陡然浮现出两头龙象虚影,升腾到半空中,让方源的棕熊、白象双影顿时相形见绌。

方源结结实实地撞到巨开碑的身上,巨开碑稍稍退后,反而他自己被撞飞出去。

两兽力虚影,能力压一龙象力。但是两龙象力压过来,方源就要吃亏了。

“终于打出了双龙象力……”巨开碑呼出一口浊气,方源的战力凶猛得让他吃惊,只有双龙象力爆发,才能挽回局面。

惯力蛊,虽然有四转,但的确不如全力以赴蛊。

巨开碑催动惯力蛊,力量不断积蓄增强,然后打出兽力虚影。但什么时候能打出来,巨开碑也不知道。就算打出来,爆发了一下,巨开碑还要重新蓄力,这就引起战斗力上的暂时低落。

巨开碑的力量,就如同一个水缸。惯力蛊像是一个桶,每次提水出来。

反观,全力以赴蛊收发如心,完全凭方源的心意,就可以爆发出来。甚至,方源不想爆发的时候,就停止催动。一发一收,称心如意。让方源进退有据,威猛的同时又兼并灵活。

“如果我有全力以赴蛊的话……唉!可惜,我就算胜利了,商燕飞也不允许我索取全力以赴蛊。”巨开碑心中很是遗憾。

成为魔道蛊师时,他自由自在,想干什么就干什么。投靠了正道,虽然安稳,资源充足,但做起事情来,却感到束手束脚。

“如果我就真的选了全力以赴蛊,会怎样?”

这样的一个念头忽然冒出来,又旋即被巨开碑打消。

这样做,无疑就会得罪商燕飞。商燕飞不仅是五转强者。本身更执掌商家。自己若不听命令,今后就算逃走,也再无宁日。

“等等!我想这么多干什么。现在最关键的,还是将方正击败!”巨开碑目光一凝,催动龙行虎步蛊,向方源攻去。

方源刚刚受伤,吐出一口鲜血,铁一般强硬的手臂骨,都发生了骨裂的现象。

剧痛传来。方源反而咧嘴发笑。

全力以赴蛊!

棕熊、白象、青牛三个虚影,一齐爆发出来。

饶是巨开碑,也不由地双眼一瞪。流露出惊愕的神色。

刚刚发生在方源身上的情形,在他身上同样上演了。

巨开碑原本想趁胜追击,主动发动进攻,但反而被直直地轰飞出去。胸膛上坚硬厚实的象牙白甲。破碎成一个大洞。他大吐一口鲜血。黑石柱被撞倒,塌毁在地上,掀起烟尘。巨开碑咬紧牙关,赶忙爬起身来。

“三头兽力虚影!”场外众人大哗。

刚刚的一幕,看得十分清楚。方源打出了三头兽力虚影!

“这是怎么做到的?”许多人面面相觑。

“刚刚打出两个兽力虚影,现在直接打出了三个……”很多人都无语了。

更多人已经有所猜测:“难道说……”

巨开碑死死盯住方源,从口中挤出一个词:“苦力蛊!”

没错,正是苦力蛊。

苦力蛊是四转蛊。是力道蛊师的绝配。蛊师受伤越重,感受到的痛苦越深。发挥出来的力气就越大。

当然,发挥出来的力气也有上限,要看蛊师个人的具体底蕴。

方源身上,有八大兽力虚影,蕴藏的力量仿佛是一个大水缸。全力以赴蛊,好像是一个可以随时开关的竹管,在不断地往外提取出水。

如今,方源受了伤后,又动用了苦力蛊,仿佛是水缸周围,破开了许多小洞。全力以赴蛊给这些洞口添上竹管,让方源随心所欲地往外调用。

单个用全力以赴蛊,方源只有一个竹管取水,因此只能发挥出一头兽力虚影。但如今,他受到伤势的影响,又用了苦力蛊,仿佛在水缸上又增添了两个新的出水口。

因此,方源现在动用全力以赴蛊,就能同时打出三头兽力虚影。

“想不到苦力蛊,在他的身上!拍卖会的时候,这蛊虫不是被商睚眦买下了么?这只苦力蛊,是不是商睚眦手中的那只?”场外的炎突,也猜出了答案。

他的双眼眯起来,脸色前所未有的凝重。

“如果方正手中的苦力蛊,不是商睚眦的。那么在拍卖场上,他无疑戏耍了商睚眦一番。如果这苦力蛊正是商睚眦买下的,那就更可怕了!商睚眦前不久,因为和方正一起做假账,被剥削了少主之位。也许这只苦力蛊,就是方正的战利品!这小子,不简单。巨老弟,你要把持住啊……”

炎突不禁对巨开碑的处境,担忧起来。

巨开碑感觉到嘴里一阵阵的发苦。

先是全力以赴蛊,现在有是苦力蛊……这两只蛊,都是他巨开碑梦寐以求的,苦苦追寻,却没有结果。

但方源年纪轻轻,就手握两蛊,这样的运道、机遇,叫他这个前辈也不免羡慕眼红。

有了苦力蛊,方源受伤越重,身体破损就犹如力气水缸漏洞,战斗力也就越来越强。

换句话讲,巨开碑每一次的伤害,反而助长了方源战力。

这种感觉,可不好受。心理承受能力差一点的,都不想和方源打了。方源越打就越强,受的伤越重,力气就越大。更关键,他有八大兽影,如果一齐爆发出来,就算是巨开碑的三龙象齐出,也不济事。

更叫巨开碑气馁无比的是,他知道方源手中还有一只自力更生蛊。

自力更生蛊,对于方源来讲,治疗的效果极其可观。

方源完全可以一边受伤,一边治疗,将自身力气一直维持在某种可怕的程度上。

全力以赴蛊、苦力蛊、自力更生蛊,三者形成巧妙紧密的搭配,组成牢固坚实的基础。

做到这一步,方源计划中的蛊虫组合,已经小成,构建了框架。效果是出众的,就算是四转的巨开碑,也要在这套蛊虫下吃瘪。

方源在商家城两年多,已经做到了许多蛊师大半生都没有的积累。巨开碑这样的老资格蛊师,也得眼红嫉妒。(未完待续。)

\end{this_body}

