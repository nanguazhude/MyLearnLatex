\newsection{再炼仙蛊!}    %第二百零五节:再炼仙蛊!

\begin{this_body}

%1
铁若男跨前一步,目光如刀:“方源,你是个聪明人,你也相当清楚自己的处境!”

%2
“我当然清楚我自己的处境。”方源一脸平静,“而且我更清楚你们的处境。”

%3
他的嘴角泛起一丝冰冷的笑意:“仙蛊的确没有大功告成,但也因此我保留下了大批的仙元。足够我绞杀了你们!”

%4
但这时霸龟显现出来,泪流满面,声音哽咽:“没有用的,没有用的。杀了他们,仙元必然损耗过多,再不能支持接下来的炼蛊了。还有时间上也来不及了,福地漏洞太多,不久之后就会彻底崩溃,烟消云散。”

%5
对于它而言,最大的执念就是成功地炼成第二空窍蛊,自我的毁灭却并不在意。

%6
现在方源只炼到这种程度,还差最关键的一步,也是最困难的一步,但环境已经不允许了。

%7
“年轻人,你计算严重失误,我们的仙蛊——炼制失败了!”地灵大声地哀嚎起来。

%8
“炼制仙蛊?没想到,居然这么热闹啊。”炎军施施然踏入大殿。

%9
上一世他被风天语阻挡,这一世,他畅通无阻,进入了青铜大殿。

%10
“呵呵呵,方源,你还不是福地的主人,地灵不会完全听你的命令。炎军少族长既然到了,其他人还会远么?今天,你输定了。”铁若男终究还是忌惮着地灵,只能继续心理战。

%11
“不,我还有希望,这次炼蛊不算失败。”方源忽的淡淡而笑。

%12
他转移视线,看向白凝冰。

%13
白凝冰顿时心中一紧,若是方源以阳蛊相逼她,让她死战,她该何去何从?

%14
但方源目光滑过白凝冰,转向了地灵霸龟。

%15
这只威严苍老的霸龟,此时已经哭得一塌糊涂。绝望伤心无比。

%16
“霸龟,你还想不想炼成第二空窍蛊?”方源心中传音。

%17
“难,难道你有办法?”霸龟心中陡然燃起希望的火苗,“没错,你是重生的蛊仙,前世经历过,应该早料到如今的局面了吧!”

%18
“不。情况超乎我的预料。现在的我,只是一个凡夫俗子,无法力挽狂澜了。”方源接着坦言,“霸龟,我对不住你,这次炼蛊的确是失败了。”

%19
地灵的嚎哭声更大了。

%20
但方源话锋紧接着一转:“不过。这种失败只是暂时的,我们还可以保留希望。”

%21
地灵哭声减弱下去,传音问道:“这话怎么说?”

%22
“你死后,我就是唯一熟知秘方的人。加上第二空窍蛊的半成品,也在我的手上,所以地灵你必须保护我。”方源道。

%23
“我当然得要保护你!你是福地中唯一满足标准的人选,又是重生的蛊仙。就算这次失败,将来也有极大可能,将第二空窍蛊炼制成功。”地灵理所当然地答道。

%24
“很好,地灵,你能有此觉悟非常好。但是你能护住我一时,却不能护住我一世。过不了多久,你就要泯灭了。以剩下的仙元,也铲除不尽福地中所有的敌人。就算你将我挪移到外界去。三叉山上也有成群的蛊师在。”方源的嘴角渐渐地翘起一个阴险的弧度。

%25
这次重生以来,一切进行的都很顺利。越来越接近成功,让他也不禁暗暗有了一丝的兴奋。

%26
“那你要我如何保护你呢?”地灵反问道。

%27
方源呵呵一笑,意味深长地道:“霸龟啊,你还记得《人祖传》第二章的第三节么?”

%28
地灵愣住,猜到了方源的想法,但语气迟疑:“你是说……不。不行,还差两个条件,不能满足。首先,你需要太古的荣耀之光。”

%29
“呵呵呵。”方源听到这里。忍不住得意地大笑,他一抬手,指着青铜壁顶,喊道,“你看,光来了!”

%30
从他从地上站起来时,他就暗中嘱咐了地灵,和它共享了视野,因此对殿外情形一目了然。

%31
之后,和铁家众人交谈,不过是拖延时间,等待良机。

%32
这已经是方源第三次运用春秋蝉,经验积累下来,令他对重生有了更深的了解。

%33
重生之后的影响,的确存在着蝴蝶效应,能把一些东西变得面目全非。但也有历史惯性,许多大事件的发生,是日积月累的矛盾,欠缺一个导火索罢了。没有了原来的导火索,也会有新的其他的导火索。不是单靠自己一个人的影响,就能随意改变的。

%34
大殿之外。

%35
萧芒被嘤鸣犬皇纠缠得不耐烦,直接飞上了高空。

%36
四转,聚光蛊。

%37
五转,太光蛊。

%38
五转,江河日下蛊。

%39
杀招——天瀑光河!

%40
光河滚滚,洪波翻腾,向福地里倒灌而下。

%41
强光耀眼,卷起惊涛骇浪,映照得山丘一片白炽,无数人都眯起双眼。

%42
“不好!”铁若男等人正在大殿当中,铁白棋见此,连忙竭力阻挡,但光河大势已成,气象磅礴,只能削弱部分。

%43
恢弘的光河,如同瀑布,重重地轰在青铜大殿上。

%44
大殿壁顶立即破开大洞,光瀑冲刷下来,正对着方源。

%45
“太古的荣耀之光!”地灵惊呼一声。

%46
方源哈哈大笑,狂催真元,调动四转骨翼蛊,五转金汤蛊。

%47
他背生一对宽长的黑翼,金汤如水,包裹他的全身,蔓延到黑翼上,将他全身染成一片灿烂夺目的黄金。

%48
“他,他想干什么?”

%49
“方源!”

%50
在铁若男、白凝冰等人惊诧至极的目光中,方源飞振双翅,迎着天瀑光河扶摇直上!

%51
轰……

%52
耳畔连绵不绝的巨大轰鸣,强大的冲击力,让方源逆飞困难至极。

%53
但他之前杀了许多五转蛊师,得了不少的移动蛊,譬如金霞蛊等等。此时他都催动起来,逆着光河,飞天而上。

%54
他飞出青铜大殿,立即引起战场上无数的惊叹和疑问。

%55
“那是什么东西?”

%56
“金光闪闪的,好像是一头大鸟!”

%57
“不。那是一个人,仿佛是黄金浇筑的战士!”

%58
方源的视野中,是白茫茫的一片。为了催动众多的移动蛊,还有防御蛊,他的真元消耗极快。

%59
“时不我待,地灵助我,神游蛊!”

%60
仙蛊神游。落在他的手掌心上。

%61
“碧空蛊!”

%62
一道绿光,从空窍中射出,落到另一只手上,化为一截碧空的玉竹,中间贯通着。

%63
“他到底想要干什么?竟然直接迎着杀招,飞上去了!”铁若男瞪着双眼。一脸的惊疑。

%64
“小兽王这是想不开,要主动寻死吗?”炎军张大了嘴巴,仰头呆呆的望着。

%65
“不妙,我了解他!方源做的每一件事情,都极有深意!不管怎么样,我们不能让他得逞,必须要破坏!”白凝冰语气急促。不知为什么,她心里的不妙感越来越强。

%66
铁若男点点头,颇为赞同白凝冰的话。

%67
“四老!”她断喝一声。

%68
“明白!”四老立即分散东南西北四方,半跪在地上,伸出左手扶住右手腕,右手成爪,两两相对。

%69
杀招——无极搜锁!

%70
几乎同时,方源左前臂中。亮起幽蓝的光芒,隐隐可以看到里面的定星蛊。

%71
此蛊乃太古星屑,宛若钻石,生有八角,晶莹剔透!

%72
同时,在方源的周围,空间破碎。四道锁链如蟒蛇般探出。

%73
“现在才反应过来?哼,晚了!”方源冷笑。

%74
神游蛊忽的飞起来,一头扎进他的左前臂中,一口将定星蛊吞掉。

%75
四老噗的一声。齐吐鲜血,他们和定星蛊失去了联系。

%76
方源身边的四条锁链,失去了目标,胡乱飞射,何况就被天瀑光河冲刷摧毁。

%77
神游蛊吃了定星蛊后,便钻进中空的碧空蛊中,迅速地结成了一个蚕茧。

%78
“嗯?居然有个怪人,顶着我的杀招飞上来?!”萧芒惊诧莫名。他的杀招别人躲还来不及,现在居然有个傻瓜,硬顶着上来!

%79
“他是何方神圣?究竟想要干什么?”易火、翼冲等人仰头,看得目瞪口呆。

%80
“等一等,这样的情形,好像在哪里见过!”李闲紧紧地皱起眉头,精明如他,脑海中有一道灵光闪现。

%81
“怎么可能,他好像在……在炼蛊?”风天语仰头望了片刻后,看清楚一些后,立即陷入到震惊当中。

%82
“居然在杀招中炼蛊,真是胆大包天啊!”魔无天双目喷发出锐利的紫芒,长达两寸。

%83
没有错,方源正在炼蛊。

%84
这就是方源的计划。

%85
当他奴隶了杀人鬼医之后,就注定第二空窍蛊的炼制会失败。

%86
因为到达最后一步,融合神游蛊的时候,单靠方源一人是无法完成的。在上一世,方源是靠着炼蛊大师风天语,还有地灵的协助,才勉强成功。

%87
这一世,他没有风天语的帮助,这一步肯定失败!

%88
不过,方源重生之后,根本就没有真心实意地去炼制第二空窍蛊。

%89
他欺骗地灵,故意拖延时间,就是营造出这样特定的环境,说服地灵,转炼此蛊。

%90
和第二空窍蛊一样,此蛊也是仙蛊,秘方就记载在《人祖传》第二章的第三节中!

%91
《人祖传》是蛊道第一经典!

%92
初读起来是故事,其实寓意深刻,更记载着上古秘闻,里头有各种各样的蛊。有些蛊,直接描述,诸如智慧蛊、力量蛊等。而有些蛊,则含蓄地点出来,描写得很隐晦。需要读者,深入的挖掘和细细的研究。(未完待续。如果您喜欢这部作品,欢迎您来起点投推荐票、月票,您的支持,就是我最大的动力。手机用户请到阅读。)

\end{this_body}

