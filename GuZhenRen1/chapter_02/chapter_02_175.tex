\newsection{福地有灾劫}    %第一百七十五节:福地有灾劫

\begin{this_body}



%1
半个月后。

%2
三叉山,某处山头。

%3
一群灰白色的山猿,足有近千头,喳喳怪叫,把铁若男包围得水泄不通。

%4
铁若男深吸一口气,忽然扬手,洒出一大蓬的金针蛊。

%5
金针蛊并非天然孕育,而是铁家蛊师合炼得到的蛊虫。每一根金针蛊,都是二转蛊,形如食指长的金色细针。

%6
金针蛊射入猿群当中,这些猿猴有的直接动弹不得,有的当场毒发毙命,有的发狂颠乱,竟然开始攻击周围的同伴。

%7
铁若男连连洒出金针,山猿大军一片混乱,损失惨重。在惨嚎声中,它们狼狈奔逃。很快,原本喧闹的战场,就平静下来。

%8
一大片的山猿,倒在地上,有的已经死亡,有的奄奄一息。

%9
铁若男缓步走到它们身边,再次洒下金针蛊。

%10
但这次,这些金针蛊却有治疗的效用,射入山猿体内,化为一团团金光,穿梭在伤患处,令许多山猿重新恢复了行动能力。

%11
金针蛊本身,并不出奇。但是配合毒液蛊,就能形成毒针。搭配僵滞蛊,就能让敌人动弹不得。配合乱神蛊,则能让敌人敌我不分,陷入混乱当中。若在配合生机蛊,还能有救治的作用。

%12
这四种战术搭配,铁若男前后只花费了七八天,就已经演练成熟。并以此在实战中,独自击败了一支近千头的山猿群。

%13
“若男这个孩子,资质卓绝,悟性十足,更重要的是性子坚韧刚毅,的确是铁家的栋梁之才。”铁慕白在不远处看着。脸上没有任何的神情,心中却藏有许多赞赏。

%14
对于这位铁家的老族长,一生不知见过多少天才的崛起,又目睹了无数天才的陨落。

%15
他深知:险恶艰苦的生活环境,总能催生出许多天资卓绝之辈。但天赋只是一个方面罢了。难能可贵的是天才的秉性。

%16
一个天才,若能吃得了苦,耐得住寂寞,将来才有可能会有大的成就。

%17
性格有缺陷的天才,只能是流星,耀眼一时。

%18
铁慕白为什么要教导铁若男?一来铁若男、铁血冷的血脉。和他有着渊源;二来铁若男历经磨砺,性情已经被打磨得如同青石,没有一丝的浮躁,只剩下稳定坚忍。

%19
铁若男就像是一块璞玉,稍一打磨,就绽放出璀璨夺目的光辉。

%20
“老族长。”铁若男攀上山头。来到铁慕白的身旁,抱拳行礼。

%21
少女对眼前的这位老人,充满了敬佩和尊重。

%22
就在半个月前,这位铁慕白老人,以一敌二,大战魔道两位五转强者。

%23
他先是利用五转的点金蛊,轻松地和两大魔头周旋。之后。又动用五转的金汤蛊,使两大强敌知难而退,战意削弱,最后各自罢手。

%24
铁慕白的强大,仿佛是一盆冷水,浇在魔道蛊师的心中。让他们刚刚暴涨起来的气焰,一下子就衰弱下去。

%25
此战的结果,是正魔两道共同竞争三王传承。但明眼人都可以看得出来,铁慕白游刃有余,根本没有使出全力。

%26
“不错。能在这么短的时间内,就掌握了这些战术变化,很好。”铁慕白淡淡地赞赏了一句,然后信手一挥。

%27
刷。

%28
一大蓬的金针蛊,挥洒出来。

%29
但和铁若男的金针蛊不同。铁慕白挥洒而出的金针蛊,极为细小,简直像是雨滴拉成细丝。

%30
洒在空中时,就像是一团金雾。

%31
金雾随风而动,刮过一片山石。但见整块巨石,发出稀稀疏疏的响声,好像是上千只蚕,在吃桑叶般。

%32
铁若男瞳孔一缩,立刻就识得此变化的厉害之处。

%33
金雾过处,山石已经被渗透,被洞穿,布满了细微的无数孔洞。山石旁边的树木,也被洞穿,顷刻之间生机毁灭。

%34
若人中了此蛊,浑身上下都会被上万的细小金针洞穿,五脏六腑都会被破坏掉,实在是恐怖的杀招!

%35
铁慕白又信手一挥,飞射出三根金针蛊。

%36
这三根金针,却又生出变化,又粗又长。正常的金针蛊,只有一个手指长,但这三根金针,堪比一个手掌的长。

%37
三根金针蛊,飞到一只山猿头顶,闪电般刺下。

%38
一只垂直电射,直接扎进山猿的头顶。其余两根,则分别从山猿的左右太阳穴中刺入,大部分直接没入山猿的脑颅中,只剩下一小部分露在外面。

%39
这山猿被铁若男治好,刚要逃跑,就中了金针。

%40
山猿发出一声惨嚎之后,忽然几下跳跃,跪倒在铁慕白的脚下。

%41
一对猿眼瞪得老大,脸上更是充斥着无比的惊惶、恐惧、愤怒之色。

%42
但奇诡的是,它的身体不受它的控制,毕恭毕敬跪着,一动都不动。安安静静的,发出不了一丝的叫喊声。

%43
铁若男从未料到这番奇景,一下子都惊得呆住。

%44
铁慕白笑了一声,俯视着脚下的山猿,淡淡地开口道:“金针蛊配合起雾蛊,就能形成金雾。这金雾看似飘渺虚弱,但实际上攻击极其强悍,尤其擅长破除蛊师的防御。我二十八岁时,行走南疆,凭借此招纵横陆川江一带。”

%45
铁慕白顿了顿,又接着道:“金针蛊再搭配操偶蛊,就能控制生灵躯体。我四十二岁后,修行到四转巅峰,结束闭关,开始试剑天下。来到铁幕山时,遭到魔道一伙山贼,共五十多位蛊师围攻。凭借此招,令其中三十八位反水,最终将其一网打尽,为民除害。”

%46
铁若男听得心驰神往。

%47
她早年时,就跟随父亲,走南闯北。但亦听说,这位老族长的英雄事迹。

%48
老族长是甲等资质,刚刚修行时就崭露头角,成为当时铁家的第一新星。而他也不负众望,在五十岁之前,就成为四转巅峰。

%49
他结束闭关,试剑南疆,过山跨水,一路纵横,打出自己的名声。

%50
回归到铁家之后,他成为铁家族长,带领铁家,苦心经营。一时,铁家风头强劲,令武家、商家都黯然失色。

%51
他的一生,充满了荣耀和光辉。无数的战绩,不论是单打独头,纵横南疆,还是领袖群雄,铲奸除恶,都是胜多败少。

%52
尤其是他作风强硬,敢打敢拼,十分强势,在位时令多少敌人都闻风丧胆。就算是正道中人,听到铁慕白的名头,也会心头压抑。

%53
现在,铁若男听着老族长平淡的回忆,就不禁心潮澎湃。

%54
她的脑海中,不由地浮现出一幕幕情景。

%55
昔日,英雄不老,俊美清隽,一身青衫,纵横山水。斩强敌于一手,无人可挡,万众瞩目。

%56
然而岁月无情,终究将那少年,催成老者。

%57
但铁慕白仍旧是铁慕白。

%58
哪怕他再年老不堪,也掩盖不住他一生的光辉事迹。

%59
这些战绩,是笼罩在他全身的夺目光环,即便是层层历史的尘埃,也掩盖不住这样的光辉。

%60
“老族长大人,我一定不会辜负您的教导,金针蛊在我手中,不会堕了您的威名!”铁若男掷地有声地道。

%61
但老人欣慰地点点头,拍拍铁若男的肩膀。

%62
“你这孩子,铁骨铮铮,流着我们铁家的血,有我们铁家儿女的担当。这很好。我把我的一生所学,传授给你,也是希望你将来能够扛起铁家的一面旗帜。那个小兽王方正,是留给你的一场考验,你有信心么?”

%63
“我有信心,也有计划。老族长放心,方正这个家伙已经彻底堕落魔道,我必将授首之!”铁若男目光坚定无比。

%64
“很好,胜不骄败不馁,你能从打击中走出来,从磨难中汲取力量,这是许多年轻人都做不到的事情。你只要保持住,将来一定会是铁家的荣耀!接下来,我就将这两种变化的原理、心得、经验,以及衍生出来的其他变化,都传授给你。”

%65
就这样,两人一个用心的教,一个尽力的学。

%66
大半个时辰之后,铁慕白传授完毕:“好了,你还有什么不明白的地方,可以问问我。”

%67
铁若男悟性上佳,早已经将传授的东西全部铭记在心。

%68
但她想了想,还是开口问道:“这些天来,我发现三王传承的开启时间,越来越短,每次进入的蛊师人数也越来越少,三道光柱早已经不像原先的,那般粗壮明亮。现在很多人都在盛传,说福地已经接近毁灭的边缘。这是真的吗?”

%69
铁慕白点点头:“的确是真的。”

%70
“有些东西,你现在还接触不到。万物平衡,有阴就有阳,有水就有火,有福就有灾厄。”他将目光转向三叉山的顶峰,叹了一口气道:“每一片福地,每隔十年,就会有一场地灾。每隔百年,都会有一场天劫。这片福地源自上古的一位神秘蛊仙,后来被三王继承,改造成传承之地。”

%71
“这片福地,已经老了,天年将至。若是有地灵还能支撑些许时日,可惜连地灵都没有。”

%72
“没有地灵的福地,就是一艘正在沉没的海船巨舟。任何人都能进入其中,搜刮里面的财富。搜刮得越多,这艘巨舟的漏洞就越大,沉没得就越快。这块蛊仙福地,已经行将就木,支撑不到十年之期的地宅,就会因为仙元耗尽而毁灭的。”(未完待续。如果您喜欢这部作品,欢迎您来起点投推荐票、月票,您的支持,就是我最大的动力。手机用户请到阅读。)

\end{this_body}

