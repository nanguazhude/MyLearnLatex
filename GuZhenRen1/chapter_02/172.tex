\newsection{且让他飞得再高些}    %第一百七十二节:且让他飞得再高些

\begin{this_body}

%1
《人祖传》第二章第三节有载——

%2
太古的阳光,普照万物生灵。

%3
夏蝉嘈杂,喧闹出生命的热量。而浓绿的树木,形成一大片的阴凉,并随着风儿摇摆。

%4
一大缸的美酒,摆放在太日阳莽的面前,他却皱起眉头,没有大口畅饮的欲望。

%5
“神游蛊啊,神游蛊,你可害苦了我。现在我有美酒,也不敢畅饮。就怕喝醉了之后,被你带到另外的险地。”太日阳莽苦恼地长叹道。

%6
他前两次,一次被神游蛊带到了平凡深渊,一次被带到毛民的油锅里。幸亏运气好,两次都险死还生了。

%7
神游蛊道:“人族太子啊,我也不是故意陷害你的。其实,每一次都是你醉酒之后,动用了我的力量。我也是无辜的呀,况且我也曾经救过你一命,不是吗?”

%8
神游蛊的确从斑虎蜜蜂的手中,救下过太日阳莽一次。

%9
太日阳莽神情颓丧:“唉……过去的事情就不提了,现在我因为有你,都不敢喝美酒。我的生活,变得了无生趣了。”

%10
神游蛊听他这么一说,也感到惭愧:“既然这样,那我教你一个方法。你先去天上,在九重天中的青天里,有一片竹林。在竹林中,采摘一节碧空的玉竹。再到九重天的蓝天里,在夜晚的时候,收集星光碎屑中的八角钻石。然后你在清晨时分,飞向天空,借助朝阳的荣耀之光。将我变成定仙游蛊。我成了那个蛊后,就再也不会带着烂醉的你乱窜了。”

%11
太日阳莽听了。顿时大喜过望。

%12
但他仔细一想,又觉得希望渺茫:“蛊啊,我生来脚踏实地,没有烟云那般轻巧,也没有鸟儿的翅膀,怎么能到青天之上,采摘玉竹。又怎么能收集星光碎屑中的八角钻石?更不可能飞向朝阳旭日了。”

%13
神游蛊道:“也是啊,人是不会飞的。不过没有关系。我们可以求助智慧蛊啊。它的智慧深不可测,一定会有办法。”

%14
太日阳莽和智慧蛊早有交情,太日阳莽之所以喝酒,就是智慧蛊教他的。

%15
但智慧蛊当初教他喝酒,只是想让他不要烦自己。察觉到太日阳莽要找自己,它连忙躲了。

%16
太日阳莽没有找到智慧蛊,十分沮丧。

%17
但神游蛊又道:“智慧蛊找不到。我们可以去见思想蛊。它是智慧蛊的母亲。”

%18
太日阳莽就找到思想蛊,寻求飞翔的办法。

%19
思想蛊便道:“你找我算是找对了,因为思想天生就有自由的翅膀。不过每个人的思想,都是不一样的,能有什么样的翅膀,就看你自己的了。”

%20
说完。思想蛊散发出温润的光辉,点化了太日阳莽。

%21
在光辉中,太日阳莽的背后,生长出了一对洁白纤细的羽翼。

%22
这对羽翼,十分漂亮。洁白如雪,没有一丝一毫的污渍。就像是白鸽的翅膀。

%23
思想蛊瞧了一眼,便道:“嗯,你这对翅膀叫做自我,每个人都有自我思想。这双翅膀非常灵便,也非常自由。但是你要小心,不要被阳光过度照射,否则自我不是膨胀变大,就是缩减变小。”

%24
“年轻的人啊,你要切记我的叮嘱。飞得过高,就会摔得越重啊。”思想蛊最后,语重心长地说了一句。

%25
太日阳莽得到名为自我的思想羽翼,十分高兴,当即就飞向了天空。

%26
他飞啊飞,越飞越高。

%27
人生来就不会飞翔,像鸟儿一样自由地飞翔,带给太日阳莽十分新奇的感觉。

%28
他在天空中自由自在地玩耍,十分开心。并且同时,他也牢记着思想蛊的叮嘱,从不在阳光下过度照射。

%29
每当晴天的时候,他就飞到云层里躲起来。

%30
就这样,太日阳莽一直往上飞,终于飞到极天之上,青天的尽头。

%31
在那里,一株株的玉竹,凭空生长着,蔓延着墨绿色的繁盛枝叶。

%32
这些玉竹,根系沉于虚空当中,竹尖也贯穿到虚空里面,从外面看,只有中央的一节节的竹干。

%33
太日阳莽信手折取一节。

%34
这节墨绿的竹干,就像是玉做的一样,巴掌大小,中间空通,润泽沁凉。

%35
太日阳莽得到了这节玉竹,很是高兴。他又继续往上飞。

%36
太古的苍穹,分有九重,依次是白天、赤天、橙天、黄天、绿天、青天、蓝天、紫天、黑天。

%37
太日阳莽在青天里,采摘了碧空的玉竹。几天之后,他又飞上了更高一层的蓝天。

%38
在夜晚时分,蓝天中星光璀璨,星辰玩耍奔行间,洒下无尽的星屑。这些星屑汇集成海,星辰银色的烂漫光河,在整片蓝天中流淌。

%39
太日阳莽振动思想的双翼,一头扎进星河中遨游。

%40
他在无边无际的星屑中,辛苦的寻找。那些七角的、十六角的星屑,他都不要。他只要八个角的星屑,这种星屑仿佛是一颗颗的钻石,晶莹剔透,完美无瑕。

%41
他找了好久,终于找到了一颗。

%42
在找到的第二天,太阳刚刚从东方升起的时候,他就飞向冉冉上升的朝阳。

%43
朝阳如一颗红彤彤的灯笼,散发着温暖的光。

%44
这阳光也非比寻常,乃是荣耀之光,能照耀万物生灵,贯穿光阴长河。

%45
太日阳莽左手捧着碧空的玉竹,右手抓着八角钻石的星屑,一边飞向旭日,一边唤出神游蛊。

%46
在荣耀之光的照射下,神游蛊吞下星屑,然后钻入到玉竹中间去。

%47
“太日阳莽啊,我需要时间结茧化蝶,最终变成定仙游蛊。在这段时间内,你要一直向着太阳飞,不要断了荣耀的光辉。但是你更要小心,思想蛊曾经说过,注意你背后那对自我思想的翅膀。我一旦化蝶成功,你就速速飞到云中去。切记,切记。”神游蛊关照道。

%48
太日阳莽哈哈一笑:“蛊啊,你就放心吧。我连平凡深渊都闯了过去,在毛民那里也能求生,有了名声蛊,又勘破了虚荣。荣耀的阳光,也不能拿我怎样。”

%49
“这我就放心了。”神游蛊的声音渐渐微弱,璀璨的阳光渐渐地凝成一根根的丝线,组成光的茧,将神游蛊和玉竹包裹起来。

%50
太日阳莽振动洁白的双翼,飞向太阳。

%51
在这个过程中,光茧也越来越重,越来越厚。

%52
片刻之后,光茧陡然震破,从中飞出一只绿光莹莹的翩翩蝴蝶:“我终于成功了,从今天起,我就不是神游蛊,而是定仙游蛊,啊哈哈哈。”

%53
定仙游蛊绕着太日阳莽,高兴地飞舞起来,忽然它惊呼一声:“啊,不好!太日阳莽,你快看你的翅膀!”

%54
在阳光的照耀下,太日阳莽背后的双翼,竟然已经变成原先的三倍大。

%55
“不要大惊小怪的,我早就注意到了,有什么关系呢?翅膀越大就越有力,我就飞得越高,飞得越快。”太日阳莽哈哈大笑道。

%56
“赶紧躲到云层里吧,不要再飞了。”定仙游蛊担忧地道。

%57
“没有关系,没有关系的。”太日阳莽毫不在意。

%58
身后自我的翅膀,越长越大,最终比他的整个人还要庞大。太日阳莽振翅飞翔的速度,也越来越快。

%59
“定仙游蛊啊,你说九重天之上,会有什么?”他向着更高空发起冲击。

%60
“别飞了,别飞了。你要是掉下去,我可帮不到你呀。”定仙游蛊十分担忧。

%61
“有什么关系,我怎么可能掉下去呢?你看我的翅膀,是多么的强大,多么的有力!”太日阳莽刚刚反驳了这一句,背后的翅膀就膨胀到了极限,发生了爆炸。

%62
失去了翅膀,太日阳莽立即往下掉落。

%63
最终,他砸在地上,摔得粉身碎骨。

%64
人祖的大儿子,就这样一命呜呼了。

%65
……

%66
南疆,火炭山。

%67
红褐色的山石上,摆放着一坛酒。靠着火炭山的地热,酒水保持着一定的温度。

%68
阳光照耀下来,方源端起酒杯,一饮而尽,舒服地感叹道:“这种棉曲酒,最适合温着喝。”

%69
一旁,白凝冰也坐着,却没有动眼前的酒水,而是遥望着三叉山方向。

%70
在那里,三色光柱冲天而起,直贯苍穹。

%71
“你居然还喝得下酒?如今已经过了数月,三王传承已经连续开启了两次。铁慕白控制了三叉山,赶走了所有的魔道蛊师。这几个月来,我们潜伏在火炭山上,就这样望着吗?”白凝冰不满地道。

%72
她并不害怕死亡,只想追寻最精彩的生命。

%73
也许是被铁家四老围困受到刺激,或者是因为方源的实力已经反超了她。这些天来,她一直苦修不辍,毫不顾忌自己的资质正越来越高。

%74
“依我看,我们也能进入三王传承当中。只要看准时机,趁着铁慕白等人进入传承,我们就能动身。凭我们的实力,三叉山上也没有人能阻止我们。”白凝冰的话,很是激进。

%75
但方源好整以暇,神情悠然地摆手:“不着急,不着急。铁慕白号称铁家的荣耀,乃是上一任的族长,五转巅峰的修为,他是多么高高在上的人物啊。和他相比,我们就是他脚下的老鼠而已。就让他飞高点,再飞高点吧。”

%76
说完,他举起酒杯,对着三叉山的方向,微微而笑,喃喃轻语道:“来,铁慕白前辈大人,晚辈我敬您一杯酒。”

%77
白凝冰瞟了方源一眼,但见那双黑色的双眸,幽深如潭,深不可测。(未完待续。如果您喜欢这部作品,欢迎您来起点投推荐票、月票,您的支持,就是我最大的动力。手机用户请到阅读。)

\end{this_body}

