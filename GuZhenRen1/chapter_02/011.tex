\newsection{谋划}    %第十一节:谋划

\begin{this_body}

土石飞溅,烟尘升腾。

白凝冰被炸翻在地,一个鲤鱼打挺,又站了起来。

她有天蓬蛊的防护,毫发无损,但空窍中的天蓬蛊却受到震荡,身上的白光虚甲也黯淡了三分。

“什么东西作鬼?”白凝冰咒骂一声,凝神望去。

但见爆炸的原处,已经化为一个小型的深坑,直径有两三米。

山洞内,那魔道女蛊师得意的哈哈大笑:“好!这次炸不死你,小娘皮你有种的再冲来啊?”

“哼。”白凝冰冷哼一声,虽然心中忿怒,但她却不是冲动之人。

刚刚的爆炸,虽然被天蓬蛊防护下来。但是若再承担几次,恐怕天蓬蛊就要毁了。

“刚刚那是什么手段?从地底忽然发生爆炸性的攻击,如果我脱离地面,是不是就能避开那种攻击呢?”白凝冰心中思索,她并不笨,虽然在其他方面懵懂单纯,但战斗上却敏锐聪颖至极。

“我并没有飞行蛊虫,跳跃前行也总会碰触地面。不对,我未必要强攻啊。这家伙刚刚的话,明显是想激将我冲杀过去,呵呵。”

想到这里,白凝冰阴笑一声:“难道你以为你躲到洞口中,就安全了吗?我只要守着洞口,不愁你不出来。”

“哈哈哈,那你就守。我早就准备好了大量的吃食,外门风大雨大,看谁能耗过谁!”魔道女蛊师立即反驳道。

白凝冰冷笑不语,时间越是拖延,对她越有利。因为这魔道女蛊师身中绿蟒毒害,时间越久,她就越虚弱。

但就在这时,方源却对魔道女蛊师抱拳道:“山野偶遇,我们纯粹是路过。唉,为难你,其实就是为难我们自己。但愿我们再也不会相见。告辞了!”

说着,他转身就走。

“走什么?她不过只是个三转蛊师。只要摸清楚那个爆炸的手段,我们赢定了!”白凝冰扬起眉头。

方源冷哼一声:“你也是三转,我才一转。我们还是赶路要紧,别惹这么麻烦了。不怕一万,就怕万一啊。”

白凝冰楞了一下,旋即领悟到方源正在表演。虽然不知道方源打了什么算盘。但是凭借对他的了解。白凝冰还是选择了配合方源,故意生气地道:“你这个家伙,总是这么胆小。唉,算了,就饶你一命。”

她深深地望了魔道女蛊师一眼,毫不掩饰自己的杀机。随后,她跟随着方源走进山林,走出了魔道女蛊师的视野。

拉开足够的距离后,白凝冰首先打破沉默:“她的那个爆炸手段。并不足虑。她刚刚和我交手,一直没有使用。直到她退入山洞,我踩在那处地点,才发生了爆炸。我猜测她应该是事先埋设的蛊虫,但并不能移动。我们完全可以勾引一群野兽,利用野兽来冲击山洞。试探她的手段。”

一番话令白凝冰的战斗才情,显露无疑。

但方源却笑了笑,反问一句:“然后呢?”

白凝冰一愣。

方源眯起双眼,眼中闪烁着精芒:“按照你所说的,引来一群野兽试探出了她的手段,那又如何?把她逼入绝境,她知道自己没有幸存的机会。必然会拼死战斗,临死也要拉个垫背的。我们就算保住了性命,也会有所损失的。”

“而且,就算是最后把她击败。她必定会抱着‘让我们什么都得不到’的想法。将她手中的蛊虫全部毁掉。蛊师要毁灭自己的蛊虫,只需要一个念头。我们没有针对性的手段,杀了她,得不到她的蛊虫,又有何益呢?”

白凝冰眉头微微皱起。

先前他们防备这个魔道蛊师,是害怕她的偷袭,是为了保护自己。但是当他们俩发现了这个魔道蛊师的虚实之后,他们的意图自然而然地就发生了转变――

将这虚弱的魔道蛊师杀了,取得她身上的蛊虫,壮大自己!

野生的蛊虫,种类繁多,但阶位合适,又容易喂养的太少了。蛊师身上的蛊虫,必然是经过其主人的精挑细选,肯定会考虑到各方各面。若是能得到,比捕捉野生蛊虫要靠谱多了。

但是很少有人,能够在杀敌之后,尽夺蛊虫的。

除了战斗时的损耗之外,蛊师念头一动,就能令蛊虫自毁。很多战败者,只要有充分的反应时间,都不会将自身的蛊虫留给杀死自己的仇敌。

要杀死这个魔道蛊师不难,但要尽量得夺得她身上的蛊虫,却不容易。

“你身上不是有一只强取蛊么?”白凝冰忽然道。

“一只强取蛊,能起到的作用太小了。对付野兽还行,对付蛊师,想要成功,需要苛刻的条件。”方源摇头。

白凝冰忽然又想到什么,担忧地道:“我们这般离开,要是弄巧成拙,让她轻松逃脱了。如何是好?”

方源哈哈一笑,以笃定的语气道:“短时间内,她是绝不会逃的。”

正道蛊师,不管是家族传承,还是师门流派,都会经过一定程度的培养,素质较高。

而魔道蛊师,往往良莠不齐。

有的是正道的叛徒,逃犯,这些人受过培训,有蛊师的底蕴。而有的则是农人、猎户,偶然间开了空窍,得到了一些传承,算是半路出家。

“这个魔道女蛊师,口音粗鄙,战斗技巧不足,也没有足够的生存经验。她每一次宿营栖息,都会留下明显的痕迹。就算是受了伤,也不会掩盖血迹这等重要的线索。我看她体格粗鲁健壮,大手大脚,极可是原先是个农妇。只是得了个小传承罢了。”

方源继续分析道:“刚刚那种爆炸,应该是她事先埋下去的一种二转草蛊,名为焦雷土豆。不管是谁踩了,都会发生爆炸。一个农村妇人,能有多少见识?受了蟒毒,没法处理,伤势越来越严重,心中惊恐,没有安全感,就下意识地在洞口铺设了许多焦雷土豆蛊。”

“我们若是强逼她。会令她做出极端过激的事情。但主动离开,却能让她舒一口气,暂时缓住情绪。她必然会怀疑,我们是否真的走了。外出有风险,说不定就碰上我们。而那些焦雷土豆蛊,却带给了她巨大的安全感。所以短时间之内。她是不会走的。”

白凝冰面无表情。一直默默地听着。

尽管她十分不愿意,但也不得不承认方源的分析,极其有道理,简直是入木三分,自己不及!

“你分析的不错,但她蛇毒在身,拖下去不是个办法,终究还是会走出山洞的。”白凝冰反驳了一句。

方源点头,指了指自己的右边耳朵:“所以。我们要监视她。”

他的地听肉耳草,虽然只是二转,但侦察范围,却堪比好多三转蛊虫。

白凝冰微微摇头:“哼,你此举也有弊端。催动地听肉耳草,要不断地消耗真元。纵然你有天元宝莲。能补充真元消耗。但是人的精力也是有限的,需要休息,需要睡觉。你总不可能一直窃听?”

面对这般疑问,方源直接翻了个白眼:“你怎么变笨了?对方只有一个人,我们可有两个啊。”

蛊虫可以相借,他们俩完全可以交替使用地听肉耳草,轮流休息。

白凝冰神情一滞。旋即双眼中闪过一抹羞恼之色。

“该死!这么简单的问题,自己居然会想不透?”她暗暗咬牙,有些埋怨自己犯下如此低级的错误。

方源暗笑。

归根结底,还是白凝冰不愿看到方源压过自己一头。所以下意识地就想处处反驳方源,反而顾此失彼,乱了阵脚。

方源很乐意看到这样的反驳,皆因白凝冰每一次反驳失败,都会使他进一步的折服白凝冰。

这种折服,是微乎其微,潜移默化的。就连白凝冰自己也感觉不到。

等到哪天她反应过来时,她已经在不知不觉间为方源所用了。

对于方源来讲,魔道女蛊师只是目标之一,白凝冰则是目标之二。

……

陈翠花心惊胆战。

她原本是个农妇,一次翻泥耕种时,意外地落到了一处地洞。

在地洞中,她发现了一具尸体,稀里糊涂地获得了传承,成为了蛊师。

蛊师!

陈翠花怎么也料想不到,自己竟然有一天,成了高高在上的蛊师大人!!

但短暂的狂喜之后,她就迎来了厄运。

一只大如耕牛的山豹,浑身缠绕着青色的旋风,袭击了她的村庄。

全村人都死光了,而她被山豹一路追赶,靠着蛊虫,侥幸逃脱。

在野外独自流浪了大半年,她手中的蛊虫越来越少。终于在最近,遭遇到了一只绿色巨蟒。虽然杀了巨蟒,她却中了蟒毒。

尤其在今天,她还遭遇到了两位蛊师。

这已经是她第三次碰到蛊师了。前两次带给她的沉痛教训,让她学会了保护自己。

但她到底还是半路出道,缺乏蛊师的基础。

一想到刚刚的战斗,陈翠花心中就慌。

她根本就不是那个女孩的对手!

幸好,自己事先埋下了许多焦雷土豆蛊。幸好,那个男孩比较懦弱胆小,选择了离开。

陈翠花看到他俩的身影完全没入山林后,着实松了一大口气。

但她不敢肯定,他们俩真的离开了。

她的侦察蛊虫,能令她看到三百五十步内的所有事物。清晰的程度,如同摆在她面前一样。

但却没有透视之能。

“还是等等。再等三天,我再出去。”陈翠花暗道,如今的她已经学会了谨慎和耐心。

\end{this_body}

