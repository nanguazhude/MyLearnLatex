\newsection{黄金山}    %第四十二节:黄金山

\begin{this_body}

%1
“居然是他,嘿嘿,老天开眼啊。”强哥一伙人看到方源的身影,顿时眼光发亮,兴奋起来。

%2
“我已经等不及看到他手臂被折断的样子了。”

%3
“张家小姐心善,他冲撞了张家,也没有把他处死。原来是要在这里派他上场。”

%4
方源缓步上前,走到石桌旁,径直坐下。

%5
猴王盯着他,伸出手臂。

%6
双方的手搭在一起,在无数双目光的注视下,开始了较量。

%7
猴王使出力气,但方源的手臂如钢铁浇筑的一般,一动不动。

%8
猴王的瞳孔微微一缩,流露出一丝震惊之色,这是有史以来它碰到的力量最强的人类!

%9
方源心中暗笑:就算是双猪之力,也能胜过这只猴王。更何况他如今又增添了半鳄之力,再加上猴王之前比试过许多轮,已经力气不济。可以说,这是方源必胜无疑的局面。

%10
“说起来,这猴王的力气,也并非十分巨大。之前的那些蛊师,有的拥有一熊之力,有的拥有一马之力,结果却输掉,倒也并非力气真的不如这猴王。而是掰手腕,并不能真正发挥出全部的力气。”

%11
事实上,熊力、马力、兔力、鱼力、龟力、鳄力种种,皆有不同。

%12
这种不同,不仅在于力气的大小,更在于力量的擅长领域不一样。

%13
熊力在于拍击,马力在于奔腾,兔力擅长跳跃,龟力长于负重,鳄力在于咬合。每种力量,都有它擅长的领域。

%14
也就是说,在某个领域,某种力量会得到最大程度的发挥。

%15
而掰手腕,更多的是看臂腕上的力量。

%16
这点,匪猴非常擅长。从它们的体格上就能看出端倪,它们的上肢比下肢往往要粗壮一倍有余。它们自打出生起,就开始掰手腕,有着大量的训练基础。

%17
换另一种比试方式,许多失败的蛊师说不定就能胜过匪猴王了。

%18
从这方面引申,每种力量都有其独到之处,不能单凭力量大小而划分它们的优劣。

%19
“就拿人来讲,一个人挥拳的力道,必定小于脚踹的力道。正常的情况下,一个人拥有的力气,不可能全部都爆发出来。我有双猪之力,半鳄之力,还有自身之力,但是掰手腕的力道,绝不可能是它们之和。当然,要想将全部的力气,集中在一个动作上,也不是不可能。那就得借助那只传奇蛊虫了……”

%20
掰手腕,并不能充分展现出方源的真正力量。但是他到底是底蕴深厚,因此稳胜不败。

%21
不过,场面上却也不能做得如此明显。

%22
于是,方源故意流露出吃力的神色,手臂颤抖,作出和猴王僵持的模样。

%23
然后,才慢慢地,一步步压倒猴王。

%24
当结果出来时,几乎所有人都楞了。

%25
“居然真的赢了!”

%26
“此人天生神力!”

%27
人群中微微骚动,惊叹声迭传。

%28
“打听一下此人是谁?可以的话,立即招揽过来!”许多家族的头领,都怦然心动。

%29
和那些投入大量资金,培养出来的蛊师相比,方源性价比太高了。

%30
根本就不需要投资,就能使用,为商队创造利润。

%31
“张家运气好,捡了一个宝贝。”一时间,许多蛊师看向张家的目光中,多了几分羡慕。

%32
“难怪我们打不过他!”看到这一幕,强哥等人无不咋舌。

%33
“这是个怪物啊。”

%34
“现在想想看,我还真是幸运,居然没有被他打死。”

%35
他们现在回想一番,顿时后怕不已。

%36
原本还想找机会报复方源,现在看到这里,一点报复的**都没有了。反而有些担心,今后方源来主动找他们的麻烦。

%37
陈家的老总管的脸色难看。

%38
“没有想到这傻子居然有这么一股蛮力。糟糕了……但愿副首领不要责怪我。”他小心翼翼地瞥了一眼陈家的副首领。

%39
陈家副首领紧紧的皱起眉头,他想到的更多。

%40
他开始怀疑张家的用意。是否之前要人,就是一个圈套呢。

%41
张家看出这个家奴的价值,所以故意将其囚禁起来,然后上门要人?

%42
他越想越觉得合理,不由地冷哼一声。任是谁感到自己被耍了,被欺骗了,心情都不会好的。

%43
但是错误已经铸成,他只能捏着鼻子吃下这个闷亏了。

%44
“我没有看错吧?”丫鬟小蝶捂住嘴,看到这样的结果,吃惊的说不出话来。

%45
商心慈脸上消去担忧之色,嘴角泛起微笑。

%46
“我们走。”张柱挥手,指挥队伍前行,双眼中亦散发着复杂的光。

%47
方源得胜,张家商队开始渡过关卡。

%48
方源连胜两场,张家队伍走了一大半。到了第三场,方源为了伪装,故意输掉,张家队伍到底还是被取了许多财货去。

%49
不过就算这样,方源的表现已经足以让所有人刮目相看。

%50
回归商队后,他受到热烈的欢迎。

%51
“张小姐,幸不辱命。”他抱拳对商心慈道。

%52
商心慈美眸闪光,重新打量了方源一番,声音轻柔如水:“我娘说人不可貌相,黑土你给我上了生动的一课。我真的很感谢你,这是一百五十块元石,是答谢你的谢礼。”

%53
“一百五十块元石?”丫鬟小蝶吃了一惊,“小姐,你干嘛给他这么多!”

%54
方源后退一步,义正言辞地拒绝道:“小姐,我是来报答你的恩情,不是因为这些元石。请把元石收回去吧,我是不会要这些报酬的。”

%55
小蝶连忙附和:“小姐,你看他都说不要了,你还是收起来吧。”

%56
商心慈却坚持道:“这不是报酬,而是谢礼,是我答谢你的礼物。”

%57
方源正色,语气严肃:“别说是一百块元石,就是一千块元石,我也不会要。张家小姐,我虽然只是个凡人,但请你不要侮辱我!”

%58
“这样啊……”商心慈无奈,只好将元石收回去。

%59
“哼,算你还识趣。”小蝶撇撇嘴。

%60
一旁的张柱,沉默不语,目光则更加复杂。

%61
“救命之恩,难以报答。请让我再为小姐出把力。”方源抱拳又道。

%62
匪猴山上猴群众多,沿途商道,每隔一段距离,都会盘踞一股猴群,设下关卡。

%63
此后方源屡屡出手,刻意表现,有输有赢。

%64
商队走走停停,在匪猴山上耗费了二十多日,这才最终离开了这座高山。

%65
商队的货物削减了近一半,众人心情不免低落。

%66
张家则成了唯一开心的队伍。

%67
因为方源的出力,导致他们的损失远远小于预计。

%68
方源因此出名,许多家族都派遣了家奴,专门找上门来。

%69
他们都想要招揽方源,开出丰厚的条件,但方源皆一一拒绝,仍旧留在张家。

%70
“算你小子还有点良心,不枉费小姐如此待你。”小蝶因此对方源转变了态度

%71
这小丫头,口直心快,没有城府,她的态度如何,一直都不在方源的考虑之内。方源在意的是商心慈,以及她的护卫蛊师张柱。

%72
商心慈温柔善良,聪慧灵敏。蛊师张柱更是老成稳重,经验丰富。

%73
尤其是后者,方源已经感觉到,这个张柱已经或多或少,开始怀疑他了。

%74
私下里,白凝冰也提醒方源:“你当初拒绝那一百五十块元石,是一个败笔。按照你表面上的身份,如此巨款,怎么可能不心动?谨慎起见,我们应该暂停修行,防止张柱暗地里的侦察。”

%75
但方源却拒绝了这个建议,仍旧每晚修行不辍。

%76
白凝冰也没有不配合。她对身份暴露,其实抱着无所谓的态度,事实上她更乐于见到方源的失败。

%77
雪银真元对方源帮助巨大,修炼速度如展翅高飞。

%78
在正式离开匪猴山地界的那天晚上,方源从二转初阶晋升到中阶。

%79
当风尘仆仆的商队来到黄金山脚下时,他也将鳄力蛊用毕,永久性地增添了一鳄之力。

%80
黄金山上,有大量的露天金矿。山土中蕴藏着丰富的黄金,甚至山溪中舀出一蓬泥沙,筛一筛后,都能收获数十颗黄金颗粒。

%81
在白天,阳光挥洒在黄金山上,往往会反射出一层朦胧的金黄光晕。光晕笼罩住整个山峦,有锦绣辉煌之美。

%82
如果这座黄金山,放到地球上,那必定引起腥风血雨的抢夺和争战。但是在这个世界上,元石才是货币,黄金只能沦为一种金属矿物,最大的用途是担当炼蛊的辅料。

%83
黄金山上,坐落着两个家族。

%84
山阳处,是黄家寨。山阴处,是金家寨。

%85
一山不容二虎,从青茅山就可联想而知,黄家寨和金家寨的关系并不和睦。

%86
商队的到来,自然引起两家的欢迎。但是两家的联合通知也到了——商队只能选择一个山寨。选择黄家寨,就不能再进金家寨。反之亦然。

%87
商队人多事杂,流动性大。之前两家都有过借助商队,来偷袭彼此的斑斑劣迹,因此便下达了相应的严令。

%88
究竟去哪个山寨,商队的首领之间发生了分歧。

%89
他们各有需求,商量了之后,这只混并起来的商队就分成了两半,两股人马分别进入两家山寨。

%90
当然,他们不能进入山寨,大部队只能在寨子周围临时驻扎。

%91
一安定下来,张柱就私下找到商心慈:“属下暗中观察多日,黑土、白云二人十分可疑,我建议将他们俩逐出我们的队伍!”

\end{this_body}

