\newsection{商心慈}    %第三十六节:商心慈

\begin{this_body}

%1
商队行进了一天,到了傍晚时分,选在一处山谷附近停下。

%2
这天大家伙的运气都不错,只是遇到了三小股兽群。

%3
打杀了两股,驱逐了一股后。除开一些损耗,从屠杀的兽群里获得的,反而有些赚头。

%4
傍晚的天空,晚霞一片片。

%5
红艳、橘黄、葡萄灰、茄子紫……霞云绚烂,色彩交杂,又变幻不停。时而形如怒吼雄狮,时而如天马奔腾,时而似花海盛放。

%6
霞光照耀着翠绿如宝石般的山谷,在山谷中商队布置妥当,在某个角落圈出一小块地方,人声嘈杂。

%7
“来一来,看一看啊,今天刚宰下来的新鲜兽肉!”

%8
“酪浆,香甜的酪浆……”

%9
“衣服只剩十件,清仓甩卖了啊。”

%10
方源和白凝冰亦在人群当中。

%11
他们拖着板车,占据了一块地方,左手旁边是卖野菜的摊子,右手边上则是卖牛奶的。

%12
白凝冰颇感兴趣地来回打量周围:“想不到商队里,还会有这种小集市。”

%13
“有消耗就会有交易,有消费就会刺激市场。”方源答了句。

%14
白凝冰目光闪了闪,这话真是精辟。

%15
她看向方源:“这些紫枫叶,你都要卖掉吗?”

%16
方源微微点头:“已经加入了商队,这些紫枫叶随手甩卖掉。留在自己手上,反而会引来一些鼠辈觊觎。”

%17
再者,紫枫叶也不易保存。

%18
才过了一天多,方源车上的紫枫叶,已经开始出现干枯的现象。随着时间流逝,价值会越来越贱。

%19
当然区区两块元石,方源自然不在乎。

%20
不过随地丢掉,却不符合他们现在的身份,会惹来怀疑的。

%21
“商队里的小集市,分两种。我们参加的这种,只是凡人间的交易,基本上每天都会有。还有一种,是蛊师之间的交易,每七天一次。”方源道。

%22
白凝冰掩盖在草帽阴影下的蓝眸,微微一亮:“如果能参加蛊师的小集市,对我们会有帮助。此行到商家城还有一大段距离,为防止突发情况,我们至少需要一只侦察蛊。”

%23
“这点我早有计划,不过现在还早了点。”方源想到兜率花中的某个东西,自信地笑道。

%24
两人正小声谈着话。

%25
一个男性家奴步履蹒跚地走了过来。

%26
他衣衫褴褛,满脸血污,形如乞丐。走到方源旁边的摊子前,看着盛牛奶的陶罐,吞咽了一口口水:“这位兄弟,能给我一口牛奶喝吗?”

%27
“去去去。别妨碍我做生意!”卖牛奶的摊主不耐烦地摆手。

%28
这家奴只好挪步,走到方白二人的板车前:“二位兄弟……”

%29
还未说完,方源就走上前去,抬起一脚将其踹翻,恶声喝道:“滚。”

%30
家奴被踹倒在地上,黑色的泥泞沾满了破烂的衣衫。更牵扯到伤口,痛得他龇牙咧嘴。

%31
他艰难地爬起来,用仇恨的目光盯着方源:“好,我记住了,大家都是凡人,谁没有落难的一天。哼……”

%32
方源面色一冷,又踹上一脚。

%33
啪。

%34
这家奴再次倒地。

%35
“你再说一个字试试?”方源居高临下。

%36
家奴狠狠地瞪了方源一眼,爬起来,却再不敢啰嗦。

%37
但紧接着,他又被方源踹倒。

%38
“我不喜欢你的眼神。”方源抱臂在怀,声音冷漠。

%39
家奴垂头低眉,再不敢看方源,默默地爬起来,也不敢在这块继续乞讨,走远了。

%40
看着他离去的背影,白凝冰疑惑地问道:“奇怪,商队里怎么还有乞丐?”

%41
“这很正常。一定是这家奴犯了错,或者他的主人今天心情不好。总之是被蛊师打了,还取消了供应给他的饭菜。”方源耸耸肩头,目光却冷冷地瞟向不远处的一个角落。

%42
角落里,三四个健壮的家奴正在捕捉新面孔,合计着怎么欺负新人。

%43
看到方源这边的情形后,他们纷纷将视线扫向其他目标。

%44
凡人命贱,地位极其低下,生存艰难,如走钢丝。在商队里,蛊师动辄打杀,命贱如草。反正沿途能从村庄中补充进来。

%45
每一次商队遇险,都会有大量的凡人丧命。

%46
除此之外,凡人之间也有近乎惨烈的黑暗竞争。方源刚到这里,就有两拨人马想找他的麻烦。

%47
他当然不怕这些麻烦,但能轻易解决,他都会争取提前解决。

%48
当然也有一些凡人,活得光鲜漂亮。

%49
这些人,大多有背景,和某些个蛊师沾亲带故,于是狐假虎威。

%50
那个讨饭的家奴走后,就来了两拨人。

%51
一拨的头目是个老者,眼神精明,问了方源价格之后,立即将价格压低到四分之一。方源估摸着这老东西的身份,应该是个总管,负责家奴调派。

%52
另一拨的领头,则是个女子。媚眼如丝,居然穿着丝绸的衣衫。方源立即了然,这一定是某个或某些男蛊师发泄的工具。

%53
两人皆是前呼后拥,虽然都是凡人,但彼此阶级极为分明。

%54
他们将价格压得很低,打的就是低买高卖的主意。这些人手中都有些余钱,不像大多数家奴朝不保夕,混得惨的连饭都吃不上。

%55
方源虽然根本看不上这车紫枫叶,但为了演足全套,不露出丝毫马脚,还是拒绝了压价的两人。

%56
那老者和颜悦色,却暗含威胁。女子则骂骂咧咧的走开了。

%57
“只要有第三人来问价,我就卖了这车烂叶子。”正当方源如此盘算之时,小集市中忽然骚动起来。

%58
一些人在兴奋的欢呼,高喊。

%59
“那个好心的张家小姐又来了!”

%60
“张小姐心慈仁善,简直就仙女下凡!”

%61
“真是个好人啊,今晚我不会饿肚子了……”

%62
“怎么回事?”白凝冰眺望,只见集市入口,出现一抹绿色倩影。

%63
方源也纳闷,这什么情况?

%64
“张小姐!”“张仙子!!”一群家奴蜂拥而去,一时间,集市入口摩肩擦踵,人头攒动。

%65
这些人大多是被蛊师责罚,没有饭吃的。方源之前踹倒的那个男子,也在其中,伸长了脖子和手臂。

%66
“大家不要急,都有的,都有的,慢慢来。”绿衣少女说道。

%67
她声音温柔,音量不高,刚说出口,就淹没在人群的呼喊声中。

%68
“都他娘的闭嘴!排好队,一个个的来。谁敢抢,敢在大呼小叫,老夫立即劈死他!”忽然一个声音如炸雷轰响,回荡在小集市中。

%69
一个苍老却身材魁梧的蛊师,昂首站了出来。虎目扫视一圈,沸腾的小集市瞬间安静下来。

%70
这就是蛊师之威!

%71
没有人不相信他刚刚喊的话。以蛊师的身份,只要心情不好,随手杀两三个凡人,又算得了什么?

%72
众人推推搡搡,很快就乖乖地排好长队。

%73
队伍前头,绿衣女子提着篮子,一个个的分发蒸饼。

%74
整个小集市,鸦雀无声。

%75
无数双目光,看向绿衣女子,包含着尊敬、崇拜,以及爱戴。

%76
白凝冰生了好奇心,问向旁边的摊主:“这个女子是谁?”

%77
“连张心慈小姐你们都不晓得?两位是新来的吧?”

%78
“张心慈?”方源回首,紧紧皱眉,“把你知道的都说出来!”

%79
摊主想到刚刚方源踹倒家奴的狠辣劲,也不敢隐瞒:“这张家小姐,也是咱们商队里的一位副首领呢。她没有修行资质,和我们一样都是凡人。但她在家族很有背景,她身边的蛊师就是她的护卫。我活了这么多年,说实在话,从未见过这么好心善良的人。张家小姐真是太善良了,几乎每天晚上都会带些食物,来送给饿肚子的家奴。哪怕是刮风下雨,都是这样……唉,老天不公啊,这么好的人却不能让她修行。”

%80
白凝冰点点头,笑了笑,对方源道:“果然林子大了,什么鸟都有。”

%81
方源却未有回应。

%82
白凝冰奇怪地瞟了一眼方源,却见到方源神情很不对劲。

%83
后者目光灼灼地盯住绿衣少女,眉头几乎都拧成一个疙瘩。

%84
这绿衣女子,一头黑发披于双肩,细致乌黑,尽显柔美。眉如淡柳龙烟,眼似明月清波。肌若雪白,嘴唇粉嫩。

%85
脸上未着粉黛,线条极为柔和。分发蒸饼,时而浅浅一笑,纯洁无暇。

%86
她身着绿衣裙,散发着清新素雅之气。她清雅如兰,秀美如莲,温柔若水,论姿容竟和白凝冰不相上下,却别有千秋之美。

%87
女子容貌精致,只能算是好看,好比普通饮料。唯有气质在身,才能称之佳人,如同陈年美酒。

%88
毫无疑问,这绿衣女子便是个绝色佳人。

%89
但管你有多美,气质有多动人,在方源心中,都是个屁!

%90
方源看的不是她的容貌,任凭多美的容颜和气质,掀开皮肉,都是白骨。

%91
他心中大奇,想到某个人物,心道:“这女子不就是商心慈么?”

%92
商心慈,乃是商家的少主之一。

%93
在家族中,族长的子女皆称之为“少主”。唯有获得家老认同的继承人,才可称之为“少族长”。

%94
单凭商家少主这个身份,商心慈就是天之骄女。

%95
世人都说商家人心黑贪婪,但这商心慈却独独是个例外。她生性柔弱,不善争斗,心慈手软,是商家中最不会做生意的人。

%96
她做生意,不仅经常亏本,而且时常被人欺骗。她很容易就相信别人,关键是被骗一次也不长记性,被人屡骗不止。

%97
作为商家少主,她一度被商家上下引为耻辱,最不被人看好,但总算念在是商家族长的血脉,因此留情,没有驱逐出家族。

%98
她对凡人也一视同仁,抱有强烈的同情心,多有体恤和救助。甚至好几次,在拍卖会上赊账买下全部的奴隶,让商家族长数次痛骂。

%99
然而命运实在是精彩,到了最后,反倒是她成为了商家之主!

%100
(ps:修改了一下,耽误了时间,抱歉。今后的更新时间修改一下,第一更是下午14点,第二更是晚上20点。爆发更新会提前说明的!)

\end{this_body}

