\newsection{群雄并现}    %第一百七十六节:群雄并现

\begin{this_body}



%1
“想不到连福地都会毁灭……”铁若男听了这般秘闻,轻声叹息。

%2
“天底下岂有不灭的存在?天道循环,万物竞法,在偌大的乾坤之中,我们凡人宛若海底里的一捧沙石,渺小至极。唯有晋升成蛊仙,才算是超凡脱俗,有了根基,成沙石变成海岛,能抗衡海浪波澜。”铁慕白的语气,饱含感慨。

%3
“那福地面临毁灭,身处在福地当中的人,岂不是很危险吗?”铁若男又问道。

%4
“不错。”铁慕白点点头,“福地消弭的那一刻,会刮起大同风,和天地融为一体,福地中的一切,都会还原成天地间的最基本的元气。此风威能浩瀚,能扫荡一切尘埃,就算是蛊仙也忌惮,连仙蛊都能消灭。”

%5
“竟然有这等厉害的风?”铁若男惊叹不已,双眼中流露出浓重的担忧。

%6
铁慕白摆摆手:“无妨,我已经踏入福地数次,对其了解越来越深,一直都在观察感悟。福地的确在变得衰弱,但距离毁灭还有一段时间。在这段时间里,福地中的天地伟力会越来越弱,对于我们蛊师的限制也会渐渐减少。慢慢地,我们可以动用一两只蛊虫,然后三四只,五六只……到最后阶段,福地漏洞百出,形成隧洞,我们甚至随意进出。”

%7
铁慕白遥望远方,目光深邃,充满了智慧,仿佛已经望见了未来。

%8
“到了最后阶段,才是三王传承的大决战。来自福地的压制,几乎没有了,任何人都可以发挥出自己的全部实力,尽情地去掠夺、战斗。到那个时候。三叉山上就远不止三位五转蛊师,那才是真正的风云会聚,龙蛇起陆。”

%9
铁若男灵光乍现,恍然大悟。

%10
她终于明白,铁慕白为什么没有和那两位魔道蛊师死战了。

%11
皆因。未来的敌人将远不止这两位。过早拼尽全力,反而会让其他人捡了便宜。

%12
三王传承越到最后,蛊师得到的奖励就越丰厚。

%13
真正了解福地的高人们,已经再为最后的大决战准备了。

%14
……

%15
中洲。

%16
天梯山。

%17
狐仙福地。

%18
“终于……爬到了半山腰了。”方正喘着粗气,满头的汗滴滚落下来。

%19
他手脚又酸又麻,几乎已经脱力。独自以一人之力,攀升到如此高度,已经压榨了他几乎全部的灵魂潜力。

%20
山风吹来,一阵阵强烈的眩晕感,让方正几乎有栽倒下去的感觉。

%21
他的整个视野,都在天旋地转。灵魂被压榨到极限,他甚至连思考的能力都丧失了。

%22
恍惚间,他听到空窍中,寄魂蚤传来的声音:“好,很好,我的好徒儿,你坚持到了这一步。真的很不容易。你成功了,接下来就看为师的吧!”

%23
说完这话,方正就感到,一股无形的力量,传达到他的魂魄最深处。

%24
那种栽倒的感觉,顿时消失不见了。就好像是一个小孩子刚刚学会走路,忽然被大人扶住。

%25
视野中的一切都清明起来,一直困扰着方正的眩晕感,也在以极快的速度,消失不见。

%26
方正狠狠地呼吸了几口空气。他感觉棒极了!

%27
就好像是跋涉在沙漠中的旅人,已经快要渴死,忽然喝到了甘甜的泉水。又仿佛是常年工作不眠不休的人,呼呼大睡了七天七夜。

%28
一切都好起来,舒爽起来。

%29
“好徒儿。抓紧时间,快往上爬吧!”天鹤上人催促道。

%30
“是,师父!”方正一双虎目,绽放出夺目的光彩。他仰头向上望去,凤金煌等人,依旧遥遥领先,但他们的速度越来越慢。

%31
方正心中,前所未有的,生出一股自信来。

%32
“我能行,在师父的支持下,我一定能打败这些天之骄子,成为狐仙传承的唯一继承人!方正,要加油啊!”

%33
心中为自己打气一番后,方正开始攀爬。

%34
他动作又快又稳,上升的速度很快,简直是一个爆发。这个表现,一下子就吸引了有心人的注意和重视。

%35
狐仙福地开启时,一直沟通着外界的天地。

%36
因此,潜伏在福地之外的十位蛊仙,立即发现了方正的异状。

%37
“咦?这个孩子,本来已经到达极限,即将被淘汰,怎么忽然生猛起来了?”

%38
“有古怪,这样的速度比凤金煌等人还要快上一筹!”

%39
“这个小辈,是仙鹤门的人。原来如此……”

%40
十道蛊仙的意念相互交流着,很快就有人发现了方正的底牌。

%41
“鹤风扬,你倒舍得下大本钱。寄魂蚤并不稀罕,不过你为了保护寄魂蚤的运转,动用了我素蛊吧?”

%42
鹤风扬作为仙鹤门太上长老之一,拥有一枚六转的“我素蛊”,已经广为人知。

%43
我素蛊,能令其他的蛊虫,在福地中正常的使用。

%44
它是消耗类的蛊,只能动用三次。三次之后,就会化为乌有。

%45
“鹤风扬,你真是好手段,想不到埋下这步暗棋。”

%46
“不敢当。此代的新人中,万龙坞出了应生机,灵缘斋出了凤金煌,灵蝶谷出了萧七星,我们仙鹤门实在难以争锋,只好出此下策。”鹤风扬话语谦虚,很是低调。

%47
其他蛊仙只能呵呵的笑着。

%48
他们当然也有六转蛊虫,但却没有我素蛊,无法形成如此有效且巨大的帮助。

%49
蛊虫六转,就是仙蛊,成就唯一。同一时间,广袤的天地当中,只会有这么一只。

%50
仙蛊比五转蛊还要罕见,很多六转的蛊仙,甚至没有一只仙蛊。

%51
方源前世五百年,成就蛊仙。为了炼制第一只六转春秋蝉,花费了无数时间和精力,刚一炼成,就被正道围攻。主要目的就是抢夺这只仙蛊。

%52
现场的这十大蛊仙,因为背靠各自门派,有些积蓄。但手头上,也只是一两只仙蛊罢了。

%53
鹤风扬有我素蛊,这就表示其他蛊仙一定没有!

%54
微妙的在于。鹤风扬的举动,也没有超越底线,更没有打破十派协商出来的游戏规则。所以其他蛊仙只能眼睁睁地看着,无法强行干预。

%55
“可惜,我手中的仙蛊擅长攻伐,若用出来。就是攻打狐仙传承,却是不美。”

%56
“我虽有保护魂魄的防御仙蛊,但弟子们却无仙元催动。否则,狐仙传承必属于门派。”

%57
“情况还没有定下来。仙鹤门的这个弟子,有了寄魂蚤的帮助,可以说有了很大优势。但是他毕竟落后了许多。究竟结果如何,还要再看看……”

%58
“鹤风扬八十年前,动用过一次我素蛊,这一次是第二次动用。也就是说,他手中的我素蛊,只能再催动最后一次了。他花费这么大的代价,只是换取一个竞争传承的优势而已。”

%59
其余蛊仙各自思量一番后。都选择按捺不动,继续观望。

%60
……

%61
三叉山上,三道光柱,冲入天际。

%62
三王传承再度的开启,强烈吸引着着整个南疆的蛊师。

%63
正如铁慕白所估测的那样,三叉山上,龙蛇开始汇聚,风云接连汹涌。

%64
“你听说了吗?就在昨天,李飞乐来到了三叉山。”

%65
“啊,你是说稳固如山李飞乐?”

%66
很快。三叉山上又迎来一个成名的高手。

%67
李飞乐修为达四转高阶,土道蛊师,号称“稳固如山”,最擅长防守。他原本是李家寨的新星,受到家族的大力栽培。但此人对于力量的追求。有着强烈的执着。

%68
执着成迷,就是执迷。

%69
为了追求最大的力量,他杀人炼蛊,为正道不容,转为了魔道。

%70
李飞乐只是一个引子,接下来的一个月里,陆续有成名的蛊师登上三叉山。

%71
有奴道蛊师章三三,在三转巅峰时,就斩杀了四转蛊师,如今是四转中阶,人称驭兽大师。

%72
有正道高手陶子,手中有闻名南疆的五转治疗蛊——灵桃蛊。

%73
还有云落天,云家的少族长,风道蛊师,二十三岁时就修行到三转巅峰,如今三十五岁,距离四转巅峰,只有一步之遥了。

%74
越来越多的成名蛊师,如闻到血腥味的鲨鱼,踏上三叉山,竞争三王传承,要来分一杯羹。

%75
这些蛊师,至少都得有三转高阶的修为,有个别的,达到四转境界。

%76
一个多月后,三叉山上迎来第四位五转蛊师。

%77
王逍。

%78
他是巫山之主,魔道蛊师,毒修五转,货真价实的一方霸主。

%79
两个月后,武家的武阑珊赶来。

%80
武家作为正道第一家族,一直都雄霸南疆。派遣过来的武阑珊,乃是当今武家家主的表妹。甫一登山,她就和王逍大战一场。

%81
虽然此战也以平局收场,但她表现出来的战斗力,已经凌驾于骷魔、巫鬼二人。

%82
此后又过了八天,魔道中名扬南疆的仇九,出现在三叉山的山脚,同样引起了轰动。

%83
仇九同样有五转修为,但是战力薄弱。他擅长治疗,是赫赫有名的杀人鬼医。和素手医师、九指游医、圣手神医并称为南疆四大医师。

%84
他性情怪癖,有个规矩,但凡要他治疗的人,都需要为他杀死一个人。

%85
他救一命,就要杀一命。因此人称:杀人鬼医。

%86
久在河边走,早晚会湿鞋。蛊师们混迹在这残酷的世界中,伤病在所难免,就会求到医师手上。

%87
四大医师都是正魔两道通吃,杀人鬼医一登上三叉山,就受到铁慕白的热切邀请,酒席间两人畅谈甚欢。

%88
时光匆匆流逝,三王传承开启了又关闭,关闭了又开启。

%89
这一天,三王传承再次开启。

%90
三叉山的山脚下,走来两个身影。

%91
很快,就有人认出了两人的身份,惊诧地叫出声来:“是黑白双煞,他们又来了!”(未完待续。如果您喜欢这部作品,欢迎您来起点投推荐票、月票,您的支持,就是我最大的动力。手机用户请到阅读。)

\end{this_body}

