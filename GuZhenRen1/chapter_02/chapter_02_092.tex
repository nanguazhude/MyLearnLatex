\newsection{方源vs李好(下)}    %第九十二节:方源vs李好(下)

\begin{this_body}

%1
吼吼吼!

%2
兽力虚影疯狂闪现,声势惊天动地,背山蛤蟆痛得呱呱大叫,猛地发力撞向方源。

%3
方源冷哼一声,及时闪过。

%4
背山蛤蟆背负山峰,吨位太重,速度很慢,方源闪避的很轻松。

%5
横冲蛊、直撞蛊接连使用,背山蛤蟆屡次出击,连方源的皮都没蹭到。

%6
众人看得目瞪口呆。

%7
背山蛤蟆一扫众人心中岿然不动的形象,被方源揍得直叫唤。

%8
“这个小子,居然如此凶猛!”李好也是看得心惊,狠狠咬牙,催动背山蛤蟆高高跳起。

%9
“臭小子,我要把你砸成肉泥!”李好咧开嘴,发出森然的冷笑,心中杀机沸腾。

%10
但是,背山蛤蟆投下的阴影,却没有主动罩向方源。

%11
反而,落在李好的身上。

%12
背山蛤蟆巨大的身躯,开始往下落,重重地砸向李好。

%13
看到这一幕,大多数人都楞了一下,没有反应过来。唯有魏央在内的少数人,眼中精光一闪,看出了李好的战术。

%14
单纯用背山蛤蟆,砸向方源,攻击十分容易落空。

%15
方源有横冲蛊、直撞蛊,完全可以躲闪。

%16
但是背山蛤蟆砸向李好,李好可以用移形蛊,来和方源的位置进行调换。只要时机掌握得好,完全可以让方源来不及反应,被背山蛤蟆砸死。

%17
背山蛤蟆这样的重量,就算砸不死,也绝对能重创他。

%18
然而背山蛤蟆腾空的那一刻,方源第一时间,就冲向李好!

%19
横冲蛊、直撞蛊连用,他很快就赶到了李好的身边。

%20
“该死的家伙,他居然看破了!”这一刻,李好无比气恼。

%21
方源现在和他还有一段距离,但他现在用了移形蛊,方源就有充分的时间,躲开背山蛤蟆的攻击。

%22
但如果他现在不用,一旦方源贴近他,他就算是利用移形蛊换了位置,也来不及逃跑,会和方源一同被背山蛤蟆砸死。

%23
无奈之下,李好只能催动移形蛊,和方源的位置互换。

%24
轰!

%25
背山蛤蟆砸落到地面,方源果然用一记横冲,躲开来。

%26
他一直都注重横冲蛊、直撞蛊的交替使用时间。一般都会间隔三个呼吸。算上冲锋五十步左右的时间,这样一来,他始终会有移动蛊可用。

%27
细微之处,往往决定成败。

%28
方源前世丰富的战斗经验,让他毫无破绽。

%29
他再次冲向背山蛤蟆,拳打脚踢,攻势狂猛,硬打硬落。兽力虚影轮番闪现,凶悍绝伦。

%30
背山蛤蟆再次陷入到狂风暴雨的攻击当中,被揍得呱呱直叫,碎石纷飞。

%31
原本口中叫嚣的观战者,都陷入了沉寂。

%32
许多人张大嘴巴,吃惊地看着这一幕。

%33
全力以赴蛊被方源用的如此威猛霸道,就算是背山蛤蟆也成了挨打的沙包,陷入弱势地位!

%34
这边打得如火如荼,场面火爆剧烈。而李好这边,则风平浪静。

%35
李好万万没有料到,方源会如此锲而不舍地攻击背山蛤蟆。

%36
以往的对手,和李好交战,无一不舍弃背山蛤蟆,企图攻击李好。李好一败,背山蛤蟆也就毫不足虑。

%37
这才是聪明人的选择啊!

%38
但是偏偏,方源选了一个最愚蠢的攻击对象。

%39
他把火力全部集中在背山蛤蟆身上,对正主李好不闻不问。

%40
李好被晾在一边,仿佛成了无关局面的看客,处境尴尬!

%41
兽力虚影不断闪烁,交汇在空中。方源绕着背山蛤蟆,凶猛殴打。

%42
蛤蟆原本庞大威武的体型,在此刻却显得如此笨拙。

%43
“不好,背山蛤蟆都被打吐血了!”远在一旁的李好,看到这一幕,顿时手脚发凉。

%44
移形蛊!

%45
他眼中绽放奇光,摄住方源。

%46
刹那间,方源视野大变,定睛一看,已经被李好挪到远处去了。

%47
反观李好,则取代了方源的位置,站在背山蛤蟆身边。

%48
他伸出一双手掌,贴着背山蛤蟆,展开治疗。

%49
蛤蟆的伤势,让他心中暗惊。

%50
他在演武场中战斗这么多场,还从未见过这么严重的伤。

%51
“难道这场战斗,我会失败?败给这样的一个年轻晚辈?不,不可能!”失败的强烈预感,第一次出现在李好的内心深处。

%52
方源冷笑一声,展开冲锋。

%53
他怎么可能容许李好在自己的眼皮子底下,公然治疗背山蛤蟆?

%54
眼看着方源冲杀过来,李好狠狠咬牙,只得放弃治疗,开始反向奔跑。

%55
待方源冲到背山蛤蟆跟前时,他再用移形蛊。

%56
如此一来,他再次站到背山蛤蟆的身边,而方源则身处远地。

%57
不过方源毫不在意,仍旧埋头冲锋。

%58
李好的治疗屡屡被打断,方源次次冲击也毫无所获。

%59
但尽管如此,他仍旧不断冲击,锲而不舍。

%60
几次下来后,反倒是李好主动停止治疗,脸色阴晴不定。

%61
这一幕,叫观战的许多人摸不着头脑。

%62
但到底还是有聪明的人。

%63
“方正每次冲锋,看似无用,其实对李好的真元进行了剧烈的消耗。”

%64
“不错。移形蛊虽然作用玄奇,但也有弊端,消耗真元量大,就是其中之一。”

%65
“距离越远,对象的实力越强,李然催动移形蛊消耗的真元就越多。”

%66
“方源身负双猪,一鳄一熊之力,李然每次动用移形蛊,都会消耗不少的真元。再加上他还要对背山蛤蟆治疗,就算是三转的真元,也禁不住这样使用。”

%67
李好也是意识到这点,这才不得已停下治疗。

%68
他的真元已经不多了。

%69
若换做以前的对手,他到底是三转的白银真元。但如今方源的修为也是三转,这些回合下来,方源已经牢牢地建立了真元优势。

%70
眼看着方源再次冲来,李好眼中闪过一丝犹豫的光,不得不挺身而上。

%71
方源捏拳拍掌,立即舍掉背山蛤蟆,攻向李好。

%72
啪啪啪。

%73
他拳拳打爆空气,声威赫赫,刚猛无比,如同潮水拍击礁岸。

%74
几个回合下来,李好就支撑不住!

%75
他的蛊虫虽然补齐,但真正的核心还在于背山蛤蟆和移形蛊。

%76
为了减少移形蛊的真元消耗,他也消去了潜伏在身体里的兽力虚影。

%77
方源的攻势,实在是猛烈至极,如狂风暴雨,压得他呼吸都困难。

%78
李好靠着之前力修的底子,勉强抗衡几下,不得不再次动用移形蛊,将背山蛤蟆换来。

%79
方源也不追李好,对准背山蛤蟆展开攻击。

%80
山猪、棕熊、鳄鱼的虚影,轮番闪现。

%81
一时间,山石翻飞,背山蛤蟆大口吐血,疯狂反击。

%82
但方源交替使用横冲、直撞两蛊,背山蛤蟆的反击显得如此笨拙不堪。

%83
“怎么会成了这个样子……”

%84
“连李好大人,都不得不参战,为背山蛤蟆分担压力。”

%85
“方正的攻击,狂猛得可怕。把李好和背山蛤蟆都压制住了。”

%86
战局进行到这里,出乎大多数人的意料。自从李好抛弃力道,转修辅助以来,他们还从未见过李好落到如此逆境。

%87
李好的战术,相当不错。利用移形蛊,配合背山蛤蟆,一旦打出配合,效果极佳。

%88
就算是方源,也没有破解得了这个战术。

%89
但是……

%90
方源他不需要破解啊!

%91
他根本就没有想去破解,直觉抡起铁拳,以不变应万变。不管你哪个出现在我面前,一通狠揍就是了!

%92
此举看似蠢笨,却大智若愚。

%93
狂猛、霸道的气势,顿时就展现了出来。

%94
“这是个好方法。”有人眼前一亮,“将来我如果对战李好,也学方正,不管其他,对背山蛤蟆展开猛烈攻击。”

%95
此言一出,顿时就遭到身边人的否定和讥讽。

%96
“屁!你也想学他,脑袋烧糊涂了吧?方正能这么干,是因为他是力修,真元消耗少。你一个火修,敢把攻击浪费在背山蛤蟆的身上,正是李好想看到的。”

%97
蛊师没有真元,战力必定暴降,几乎就等于凡人了。

%98
战斗中,一方真元较多,往往就有优势。真元量相差越大,优势就越大。

%99
一些想效仿方源的蛊师,听到这话,顿时噎住,反驳不得。

%100
有人一拍脑袋,大悟道:“我忽然发现,力修也是有优势的。”

%101
“没错。”立即就有人附和出声,“力修的攻击,借助身体,因此力蛊有个普遍的优点,就是消耗真元较少。”

%102
“每个蛊修流派都有各自的优缺点,力道能在上古时代煊赫一时,不是没有道理的。”

%103
众人再次将目光,投向场中。

%104
李好和他的背山蛤蟆,在方源的猛烈攻势下,节节败退。

%105
全力以赴蛊对真元的消耗,是很少的。

%106
方源真正的攻击力量,来源于几大兽力虚影。

%107
但这些兽力虚影本身,根本就不,需,要,消,耗,真,元!

%108
这才是最变态的地方!!!

%109
换做其他蛊师,放到方源的位置上,打个数十回合早就萎了,但方源却越打越猛,持久无比。

%110
他的气势不断攀升,拳拳都带出风声,霸道猛烈,仿佛是猛虎咆哮,巨熊嘶吼!

%111
在他身上,众人仿佛看到了上古力修的傲世风采!

%112
心脏急速跳动,胸中热血沸腾。方源越打越爽。

%113
重生以来,他一直如履薄冰,积压在心中的郁气,随着拳脚的狂热爆发,统统宣泄出去。

%114
他心中的阴郁一扫而空。

%115
毫无疑问,得到全力以赴蛊,是方源人生的一个转折点。

%116
在此之前,他东奔西跑,朝不保夕,食不果腹。遇到一个稍大的事情,就需要殚精竭虑地思索办法。

%117
但是当他得到全力以赴蛊后,他终于有了傲人的资本,可以用拳头去解决许多事情。

%118
就像现在,他根本不需要破解李好的精妙战术,直接举拳横扫。

%119
魔是狡诈,魔更是霸道!

%120
扫天荡地,席卷山河,血溅乾坤,一力降十会!

%121
你凶狠我比你更凶狠,你蛮横我比你更蛮横!

%122
魔!魔!魔!

%123
杀!杀!杀!

%124
方源打得酣畅淋漓,心中一股情绪在猛烈的激荡着。终于压抑不住,化作长啸之声。

%125
“因为困难多壮志,不教红尘惑坚心。今身暂且栖草头,它日狂歌踏山河!”

%126
从今日起,便走上雄起之路罢!

%127
一扫尘埃,笑对沧桑。

%128
踏青山,蹈蓝海,缚苍龙,击长空!

%129
沐浴风雨,砥砺魔魂,举旗高歌猛进,逆天逆命逆乾坤!

%130
轰!

%131
一拳狠狠捣下,背山蛤蟆再支撑不住,山峰崩塌,鲜血喷涌,被方源活活打死。

%132
横冲直撞!

%133
李好高高飞起,飞出十几步远,又重重落下。

%134
扑通一声,落在泥浆中,一动不动。

%135
黑色的泥水混合着殷红的鲜血,很快就污染了他的花袍。

%136
他为他的轻视,付出了生命的代价。

%137
战斗戛然而止,方源傲立原地,阳刚猛烈的气势盖压当场。

%138
演武场中,似乎还回荡着他的长啸之音。

%139
除此之外,一片静默,无人出声!

\end{this_body}

