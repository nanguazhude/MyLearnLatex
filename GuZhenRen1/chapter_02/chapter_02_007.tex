\newsection{成为主导者}    %第七节:成为主导者

\begin{this_body}



%1
日上三竿,蓝天白云。域名就是的全拼,请记住本站域名!

%2
阳光照耀大地,黄龙江滚滚长流,江边茂密的丛林,汇集成一片绿色的海洋。

%3
在捕兽树的树冠上,藤蔓攀附,八九个叶片牢笼,宛若聚合的河蚌,高高竖起。

%4
忽然,其中一个牢笼,猛然颤抖了一下。

%5
噗。

%6
一记鲜红的月刃从里面飞射出来,射破叶片。

%7
一个少女,身穿浅色衣袍,浑身笼罩着一层白色虚甲,从叶片牢笼中钻出来。

%8
她身手敏捷,脚尖在树杈上连点几下,不断飞跃,最终安然地落在地上。

%9
正是白凝冰。

%10
整个过程中,捕兽树一动不动,一片死寂。

%11
白凝冰看了一眼这株捕兽树,心中不由浮现出昨天,方源关照她的话。

%12
“对于任何脱困的猎物,捕兽树都不会再进攻。因为能够脱困的猛兽,已经不是捕兽树能够对付的了。捕兽树虽然没有智慧,但是这种进化出来的本能,让它更适应生存。”

%13
“阿嚏。”

%14
白凝冰忍不住打了一个喷嚏,她一边揉了揉鼻子,一边目光四扫,打量周围。

%15
这一片捕兽树林,大多都在树冠上高举着叶片牢笼,宛若一只只碧绿的贝壳。

%16
“看来浅滩上的血腥味,吸引了许多野兽。捕兽树林在昨夜斩获颇丰啊。”白凝冰一边在心中琢磨,一边活动四肢,晃动脑袋。

%17
她昨夜睡得很不舒服,鳄尸坚硬,夜间也比较冷。虽然极为疲累,但她好几次都被冻醒。

%18
因此带着黑眼圈,精神不佳。但经过一夜修行,体力倒是恢复到了大半。

%19
此刻,她特意站在有阳光的地方,借助阳光,驱散身体内的寒气。

%20
“白凝冰,将我放出来。”一个声音传来,正是方源。他不用借助地听肉耳草,都听得到这番动静了。

%21
白凝冰将目光投向另外一棵捕兽树。在这棵树的树冠上,吞下方源的叶片牢笼,还在老地方。

%22
她暗暗笑了笑,并未答话。反而闭目养神,沐浴在阳光中,故意拖延。

%23
一直过了一刻多钟,她这才射出血色月刃,将藤蔓切断。

%24
贝壳似的叶片,在树干间磕磕碰碰,最终砸落在地面上。

%25
白凝冰慢慢地悠然走近,再催出一记血月蛊,切破叶片,方源这才从里面钻出来。

%26
“怎么这么晚?我可是早就醒了,还修炼了一会儿。”方源面色红润,精神抖擞。

%27
昨晚取出来的衣服,披风则早就被他收了起来。

%28
白凝冰冷哼一声,方源这样的状态,有点出乎她的意料。

%29
她原以为,方源也和她一样,睡得不好,又冷又饿。因此故意拖延,多折磨他一下。

%30
没想到方源状态这么好。

%31
“时间已经不早了,今天还要赶路,先吃饭吧。”方源吐出兜率花,一一取出煤石,铁架,铁锅,水囊,干饼等等。

%32
他动作麻利,很快就煮好了一锅肉汤。

%33
然后就地搜寻了一番,在捕兽树的树干青苔上,采摘了一大把的树菇。

%34
这些树菇,干瘪细长,泛着暗紫色,或者黑色。

%35
白凝冰看着方源将这些树菇都放进锅里去,不禁质疑:“野外的植物,可不能乱吃,很可能蕴含毒素。”

%36
“嗯,你说的对。”方源点点头,“你也可以不吃啊。”

%37
白凝冰冷笑:“你中毒了,我手中可没有治疗蛊虫。”

%38
方源一脸淡然,取出汤勺,在白凝冰的注视下,喝了一大口的肉汤。

%39
白凝冰冷哼一声。

%40
一直到方源喝下五六口的肉汤后,她这才确认,这汤没有危险。

%41
取过汤勺一喝,顿时双眼微微一亮。

%42
这肉汤可比昨天的好喝多了,透着一股鲜味!

%43
她旋即将目光移动到,肉汤表面浮着的树菇上。很显然,前后差别都因为这把树菇。

%44
她不禁打量方源一眼,方源坐在石头上,正低头喝着汤,啃着干饼,很有精神气。

%45
明明是睡的环境都是一样,白凝冰再对比自己这颓废的样子,心中不可避免地生出一丝,连自己都不想承认的佩服情绪。

%46
当然,如果她知道方源昨夜偷偷取出披风、衣服保暖的话,那注定又是另外一种情绪了。

%47
方源感受到白凝冰的目光,正落在他的身上。

%48
他却没有抬头,只是嘴角浅笑,佯装没有察觉,继续吃喝。

%49
自从青茅山一役,白凝冰主动救下他,就让方源感受到她体内纯粹的魔性。

%50
魔是疯狂,是不可理喻,是偏执于自己的路。白凝冰的魔性,让方源看到了她身上可以利用的价值。

%51
但白凝冰这个人,也是复杂的。

%52
一方面,她纯粹稚嫩,得了新生,北冥冰魄体的麻烦也暂时解决。因此她不愿放弃,更热爱生命,不甘死亡。

%53
但另一方面,她魔性凛然,追求精彩,因此有无以伦比的洒脱性情。她并不畏惧死亡,如果死亡足够精彩,她绝对会坦然赴死。

%54
这样的一个人,就好像是幼龙,对世界充满好奇,天性桀骜,野性难驯。她有着自己的路,自己的野心和志向。

%55
白凝冰还没有成长为魔头,现在只能算魔子。但这是一头真魔,不可改变她的道路,也改变不了她的方向,更不可能让她臣服。

%56
真魔只忠于自己,在黑暗中孤独朝圣,只走自己的路。

%57
真魔可以去敬佩他人,但永不可能臣服于他人。

%58
真魔皆是自己的君王,至高无上!

%59
方源了解白凝冰,因为方源了解他自己。他知道白凝冰绝无可能臣服于他,但不臣服,并不代表不能折服。改变不了她的道路,并不代表不能去利用。

%60
若方源本身有三转修为,自然不需要白凝冰。但现在他只有一转初阶,白凝冰的利用价值就高了。

%61
然而想要折服她,利用她,却需要花费一些心思。

%62
白凝冰聪颖高傲,不能强行压迫。只能借助一些小事情,或者外力来旁敲侧击,慢慢磨她的性子。

%63
偷偷地取出衣物保暖,并非是方源小气。刚刚白凝冰故意拖延,方源没有去追究,也并非是因为他大度。

%64
“要折服白凝冰,必定耗费相当长的时间。不过也不必着急,慢慢来,我也正需要时间来恢复修为。”

%65
两人吃饱之后,已经接近正午。

%66
地面上,依稀全是野兽踩踏出来的脚印。他们继续出发,由白凝冰在前面开道,往丛林的东南方走去。

%67
越往深处走,丛林中的树木越是高耸。先前的捕兽树,只有三四米,但渐渐的,出现五六米的树。时不时,出现七八米的树王,如鹤立鸡群。

%68
当然也有枯死的树干,倒在地上,上面布满了厚重的青苔。或者在某些关节,抽出细枝绿芽。或者是断木,被雷劈成两半,显露出天地之威。

%69
这些庞然大物,无比密集的生长在这里,遮天蔽日。

%70
越深入于此,便越感到阴凉。

%71
强盛的阳光,被茂盛的枝叶组成的城墙阻挡,只能透过一丝缝隙,钻进来一些斑驳的光影。

%72
风一吹,树叶沙沙作响,光影摇晃,宛若碎裂的金子。

%73
丛林中,亦并不平静。

%74
时不时地,会见到鹿、狐狸、兔子等等动物窜动的身影。

%75
最多的是鸟儿,各种各样的鸟,有的三五只,有的成群结队,或者飞翔于天际,或者驻足在枝叶间相互攀比彼此的歌喉。

%76
偶尔间,会从远处传来猛虎的咆哮声。

%77
两人时走时停,依靠着地听肉耳草,方源规避了许多危机。但有些地方,规避不了,就得靠白凝冰的战力,来强行闯关。

%78
三转的修为,已经足以应付野外寻常的问题。

%79
夜幕再次降临,方源寻了一个安全的宿营地,这是一处乱石岗。

%80
白凝冰疲惫不堪,哪怕是坚硬的石头上,她也是一躺就睡。

%81
到了第二天起来,她浑身都不舒服,脑袋也运转不便,很显然是落枕了。并且她的喷嚏,越打越勤快,显然是寒意入体。

%82
方源看在眼里,仍旧赶路。

%83
他们俩行走的很慢,因为缺乏移动蛊虫代步。先前方源拥有千里地狼蛛,白凝冰拥有白相仙蛇蛊,都是代步的蛊,俱都高达五转。可惜一死一逃。

%84
不过方源并不急躁,他修为薄弱,正需要时间不断修行。

%85
每天白日里赶路,就算是歇息的时间,他也抓紧一分一秒,修行不辍。

%86
到了晚上,他通常都要修行到深夜。

%87
几天后,白凝冰生病了,感染了风寒。头昏脑胀,战斗力急剧下降,浑身发热。

%88
方源只好停下来,取出兜率花中的草药,给她口服之外,还有外敷的药膏,涂在身上,火辣辣的,能拔除湿气。

%89
一直休息了六天,白凝冰这才好转。

%90
这一次生病,给了她当头一棒。期间有几次,她昏昏沉沉,四肢无力,连药膏都没法涂抹。还是方源出手,帮助的她。

%91
“若无方源,我此次恐怕凶多吉少……”白凝冰康复之后,心中叹息。哪怕她极不愿意承认这点,但终究是事实。

%92
以前,她语气强硬,嘴巴很凶。但经此一事之后,她的话变少了,常常沉默。有时候赶路,能半天都不说一句话。

%93
她的话变少了,自然方源的话语权就增强了。如此一来,他慢慢地成为两人中的主导者。(未完待续。如果您喜欢这部作品,欢迎您来文学注册会员推荐该作品,您的支持,就是我最大的动力。)

\end{this_body}

