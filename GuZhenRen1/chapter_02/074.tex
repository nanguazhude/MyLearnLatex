\newsection{晋升三转}    %第七十四节:晋升三转

\begin{this_body}

哗哗哗……

空窍中,潮涨潮落。一波波的真元呈浪,向窍壁冲刷而去。

浪花如雪,又闪耀着银光,绚烂非凡,正是雪银真元。

房间中,白凝冰一双手掌抵住方源后背,通过骨肉团圆蛊不断地为其灌输真元。

真元一旦被骨肉团圆蛊转化,就成了方源之物,只能由他调用。

方源调动着绵绵不绝而来的雪银真元,不断冲刷周围窍壁。

他如今已是二转巅峰,本身真元一片暗红,窍壁呈晶膜,光莹剔透。

当初,他在青茅山,以一转巅峰丙等资质冲刺二转,不得不借助元石之功,耗费了三四天的功夫,这才勉强完成。过程艰难无比。

而到他以二转巅峰冲刺三转,资质有限,还不得不借助人兽葬生蛊这种外力。

如今却大不一样。

不仅自身资质,增长到甲等九成,再不用借助人兽葬生蛊不提,竟然身边还有白凝冰为其强力臂助。

人生际遇,果真奇妙万分。就算是方源当时,也绝对料想不到。

厚实坚固的晶膜表面,在雪银真元的冲刷下,很快就出现裂纹。

裂纹以肉眼可见的速度,迅速扩张、蔓延。片刻之后,整个晶膜都裂痕满布。

咔嚓嚓……

晶膜彻底破碎坍塌,无数的晶体碎片,坠落道真元海中,激起多多浪花。

随后,这些碎片就化为无数晶莹的白色光点,在真元海中渐渐消散。

一片全新的白色光膜,全面取代了晶膜的位置。

与此同时,一丝淡银真元,出现在海底深处。

淡银真元。正是三转初阶的特征。

此刻,方源正式突破二转巅峰,晋升到三转。

不提六转蛊仙,三转蛊师已经算是中坚力量,不管是正道还是魔道,都有真正立足的能力。

“距离出走青茅山,不过是将近一年光阴,我就重新修到三转境界。这样的修行速度,已经是青茅山上的三倍有余。并且资质甲等,前途一片光明。”方源握了握双拳,心中欢喜。

算算用时,不过三个时辰的样子。

真是快啊。

若是凭他自身之力,也能突破到三转。但至少得要一天一夜的功夫。

雪银真元,果然效果卓绝。

此时再细细查看空窍。

原先一片惨白的骨枪蛊、螺旋骨枪蛊,已经被贩卖一空。

暗红的二转巅峰真元,还保存大半。

白凝冰的灌注没有停息。

一股雪银真元,从天而降,如瀑布一般,冲入真元海。一阵翻腾后,渐渐沉入海底深处。

境界越高的真元,质量便越好。赤铁真元以及淡银真元,只能被雪银真元挤上去。

在真元海底。一朵蓝白相间的花骨朵儿,随着海流摇曳生姿。

正是三转的天元宝莲,号称移动元泉,一天能为方源供应大约五十枚的元石。

圣洁的宝莲旁边。就是邪气凛然的血颅蛊。

血色的骷髅头上,两个深洞一般的眼眶中。似乎偶尔间有紫焰闪烁。

在血颅蛊的不远处,有一颗水晶球,静止不动。

水晶球中,云涛幻灭,形成一位鹤发童颜,仙风道骨的老人。

老人拄着拐杖,胡子一大把,面色平淡。

正是元老蛊。

原先云老人表情是喜笑颜开,但是被方源分去一半的元石后,神情就变化为平淡了。

还有一只玉坠般的甲虫化石,半透明的碧绿色,散发着一股清凉之气。

乃是二转的清热蛊,专门用来解读。

而在波澜起伏的海面上,四味酒虫在海面翻滚,嬉戏浪花。它肥嘟嘟的身躯,不断闪烁着红、蓝、绿、黄四色,分别代表辣、苦、酸、甜四味。

真元海上空,则有天蓬蛊和阳蛊绕着雪银瀑布,在嬉戏飞旋。

天蓬蛊如大瓢虫,半圆形乳白色的甲壳上,点缀着点点黑斑。而阳蛊则散发着温暖的白光。

最重要的本命蛊,还在沉眠当中,不断地汲取光阴长河中的水,恢复着生机。

除了空窍中的这些蛊,还有骨肉团圆蛊,形成手镯印记,戴在方源的手腕上。

手掌心中,寄托着血月蛊。

舌苔上,有兜率花的印记。

在左耳中,则藏着敛息蛊。

两只脚底板上,有跳跳草蛊。

肉白骨是白凝冰所得,已经还给了她。原先还有一只铁钳黑甲虫模样的强取蛊,方源曾经用来夺取过白凝冰空窍中的蛊虫,已经在行商途中因为缺少食料而饿死了。

这些就是方源现有的所有蛊虫了。

“天元宝莲要保留,但是我却没有合炼下去的秘方,现在能用用,等到修为上了四转,它的作用就越来越小了。”

“血颅蛊对我几乎无用了,要培养子嗣血脉,实在太过于麻烦,耗时又耗力。当初古月一代,也是迫不得已。这个可以换掉,毕竟是血海老祖的真传之一。兴许能从宝界中换到一只好蛊。不过宝界乃是商家根本,我就算是有紫荆令牌在手……此事还得从长计议。”

“春秋蝉、元老蛊、骨肉团圆蛊不用多说,血月蛊尽管好养活,但攻击不足,无法配套。兜率花最好也能换掉,在拍卖场有更好的选择。敛息蛊要暴露,但跳跳草必定要剔除掉。它本来就是救急用的。”

“除去这些之外,我还有购买大量的力蛊,在侦察、移动方面也要有所补充。同时现在已经稳定,得到商家的信任,赌石坊那边,也可以尝试接触了。”

方源仔细思量着。

他身上的蛊,本来就是东拼西凑来的,并不搭配成套。而且又有缺失,严重影响他的战斗力。

现在到了商家城。难得有稳定发展的机会。他必须趁此良机,抓紧时间,务必在两三年内,使得蛊虫配套,战力成型,修为尽可能的提高。

过了这个时间,紧接着就是三王传承,然后是义天山正魔大战,波及整个南疆。都是风云际会。群英称雄之时。若没有相应的实力,只能沦为牺牲品。若是有参与的资格,凭借方源重生的优势,必定能够获取巨大的好处。

……

“属下魏央族长大人!”书房中,魏央跪倒在地。

商燕飞停下手中的笔。抬起头来:“魏央,坐,这里只有你我二人,不必太拘束。”

“谢族长赐座。”魏央站起身,坐到一旁。

商燕飞笑起来,眼中闪过回忆之色:“你啊,还是这么一本正经的样子。想起我们第一次见面时。你还没有称雄演武场,而我也还只是一个商家少主。一晃这么多年过去了,我能登上族长之外,还多亏了你在我身边帮衬着。”

“属下愧不敢当!”魏央连忙站起身来。抱拳道,“属下才智愚钝,只知效死力罢了。族长大人英明神武,魏央不过是锦上添花罢了。”

“呵呵呵。我虽是英明神武,但总归是势单力孤。唯有集合你们之力,才能成势,才能坐大。双拳难敌四手,一个好汉也得三个帮。你说是吗?”商燕飞饱含深意地看向魏央。

魏央立时发觉商燕飞意有所指,但却不解其意,只得抱拳:“属下惭愧。”

商燕飞陡然转过话题:“我原先还以为,那白凝冰是女扮男装,毕竟许多家族都是重男轻女。但是今天听素手医师说,白凝冰曾向她打听阴阳转身蛊的事情。看来此中还有内情。不过她究竟先前是男是女,都是细枝末节,已经不重要了。重要的是,她和那方正,是否能为我商家效力。”

魏央恍然大悟:“属下明白了。”

“嗯,明白就好,下去吧。”商燕飞挥挥手。

“属下告退。”

看着书房的门打开又轻轻关上,商燕飞倚靠在座椅中,缓缓闭上双眼。

方白二人能在百家手中抢夺传承,又能护送商心慈一路,可见两人勇谋兼备。

资质又甚好,据情报上的信息都是三转。

他们才二十岁不到啊,真是好天资!

关键是,他们还能知恩图报,这就是品行端正,让人放心。

还有一点,他们并非是泥腿子出生,本来就是两大山寨的少族长,本身就有着深厚的正道烙印。

商燕飞执掌商家这么多年,看过年轻俊彦无数,但极少能有方白二人这般,令其动心的。

但商燕飞要招揽方白二人,却并非为了自己,而是为了商心慈。

他睁开双眼,忽然化作一道火光,消失在书房。

再出现时,他已经置身在一处巨大的走廊中。

走廊两侧,立着高大的石柱。地面铺着白银色的正方大石砖。商燕飞站在石柱旁,如筷子旁的一只蚂蚁。

他缓步前行,偌大的走廊中空无一人,只回荡着他的脚步声。

片刻后,他来到走廊的尽头。

一面朱红色的巨门,展现在他的面前。

巨门有走廊的立柱之高,宽有近十丈。门上没有把柄,却雕刻着一张巨大的人脸。

人脸往外凸出,闭目沉睡,用的是阳刻的手法,惟妙惟肖。

商燕飞来到门前,仰望着朱红巨门,没好气地道:“活宝门,别装睡了,有意思吗?你这把戏已经玩烂了。”

巨门上的硕大人脸,陡然睁开双眼,瞪向商燕飞,埋怨道:“哎呀呀,小飞飞,你长大了,越来越不可爱了!”

它说这话时,巨口大张,呼出一阵狂风,商燕飞赤发一阵飞舞。声音如雷霆轰鸣,整个走廊都回荡着嗡鸣之声。

商燕飞眼角抽搐了一下:“废话少说,我这次是来换宝的。”(未完待续)

\end{this_body}

