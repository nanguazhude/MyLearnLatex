\newsection{熊力虚影}    %第八十二节:熊力虚影

\begin{this_body}

%1
当方源站在演武场的时候,已经将脑海中纷杂混乱的思绪排除。

%2
他凝神静气,打量眼前的对手。

%3
汤雄。

%4
他身高八尺,膀大腰圆,手臂能有方源的大腿粗,坦胸赤脚。

%5
他的胸口长满了黑色的胸毛,散发出彪悍之气。

%6
“小兔崽子,就是你杀了我弟弟?今天我要让你给他陪葬!”汤雄怒目圆瞪,紧紧地盯着方源,眼中燃烧着仇恨的火焰。

%7
周围热气蒸腾,宛若置身在七八月中午,酷热的烈日之下。

%8
地面滚烫,黑色与红色交织,这是熔岩地形。

%9
此处是中型演武场,场外站着数十人,稀稀疏疏。绝大多数都是来看汤雄如何复仇的。

%10
至于方源,虽然连胜两场,但终究火候不到,还没有打出名声来。

%11
“汤家兄弟从小就相依为命,现在汤青死了,只剩下汤雄了。”

%12
“呵呵,我坐看汤雄如何虐杀了这个小子。”

%13
“呃,这个小子叫什么名字?”

%14
“好似叫做古月方正,一个无名之辈。”

%15
“这小子太不懂事,居然敢坏规矩,打了两场已经连杀两人。”

%16
“这应该是从外面刚过来的魔道蛊师吧……”

%17
“唉,年纪轻轻不会做人,手下留情些就不会有今天了。”

%18
观战者议论着,普遍都不看好方源。

%19
魏央做了伪装,双目炯炯盯着演武场。汤雄也走的力道路线。二转巅峰,但能爆发出三熊之力,更打到第四内城里去。对于现在的方源来讲,是个不弱的对手。

%20
当——!

%21
一声清脆的钟鸣,演武正式开始。

%22
汤雄大吼一声,迈开双腿,如同一头蛮牛,向方源直直地冲撞过来。

%23
这个演武场的地面,都是暗红色的熔岩石块。方源就算是脚蹬皮靴,踩在上面。也觉得发烫。

%24
但汤雄赤着双脚,却毫不在意。

%25
嘭嘭嘭。

%26
汤雄一双大脚,踩踏在熔岩地形上,每一步都发出闷响,同时将熔岩石块踩得四溅,一个个的足迹深深的印在地面上。

%27
方源眯起双眼,目光凌厉犹如刀锋!

%28
尽管汤雄气势汹汹,但他怡然不惧,嘴角泛起一丝冷笑。向着汤雄悍然反冲上去。

%29
“他疯了吗?”

%30
“居然和汤雄玩硬的?”

%31
“拖延还能活命,他这是自找死路。”

%32
旁观的众人看到这一幕。纷纷摇头。

%33
方源还是少年,体型不及汤雄一半的雄壮。两人急速接近,仿佛一头小羊和一头大牛对撞。

%34
砰!

%35
两人狠狠地撞在一起,巨大的力量将两人弹飞。

%36
汤雄连退六步,脸色涌现出吃惊的神色。这个小子的力气怎么这么大?

%37
方源则连退三步,笼罩全身的白光虚甲一阵摇晃。

%38
撞击的结果,让观战者大为惊异。

%39
有的人微张嘴巴,有的人连眨眼睛,没有料到方源年纪轻轻。却有这般气力底蕴。

%40
“我的力气居然不如他?难怪弟弟死在他的手上!”汤雄眼神变了,第一次正视方源。

%41
方源甩了甩发麻的胳膊,面容依旧冷酷,对撞的结果并没有出乎他的意料。

%42
他有双猪一鳄之力,最近这段时间一直在用棕熊本力蛊,因此还得增加些力气。而汤雄,本身只有双熊之力。

%43
熊豪蛊!

%44
汤雄忽然大吼一声。浑身肌肉发生明显的膨胀,整个人大了一圈,临时再增一熊之力。

%45
熊掌蛊!

%46
他的手掌、脚掌都笼罩了一团鹅黄色光晕。光辉散去之后,他的手脚变大三倍有余。均化为厚实庞大的熊掌。

%47
呼!

%48
他飞扑上来,猛地用力,挥起右掌向方源狠狠地拍去。

%49
熊掌还未拍中方源,劲风就扑面,刮得他衣角向后飘飞。

%50
方源怡然不惧,左手捏拳,直捣上去。

%51
拳掌相击,发出一声闷响,平分秋色。

%52
但紧接着,汤雄手臂横扫,另一掌也拍过来。

%53
方源以攻对攻,啪啪啪,拳掌交击,风声呼啸。

%54
看到方源与汤雄对攻,丝毫不落下风,周围的观战者们都流露出异色。

%55
“居然能和汤雄如此对打,这个少年有两把刷子!”

%56
“他叫什么名字来着?”

%57
“这是个硬茬,汤雄碰到麻烦了。熊豪蛊是有时限的,一旦效果消失,他就会落入下风了。”

%58
但就在这时!

%59
忽然间,一声熊吼爆发开来。

%60
汤雄的背后,猛地浮现出一个黑熊张开血盆大口,朝天咆哮的虚影。

%61
兽力虚影!

%62
汤雄福至临心,使出兽力虚影。

%63
原本这记极为平凡的拍击,却有了一熊之力!

%64
方源来不及躲闪,匆忙之间只能竖起双臂硬架。

%65
轰。

%66
一声巨响,他整个人都被砸飞出去。白光虚甲闪了闪后,忽的涣散。

%67
双臂直接被震麻了,一时间都使不出力气。

%68
他在空中竭力稳住平衡,扭腰翻身,双脚落地。

%69
再催天蓬蛊,白光虚甲再次浮现,只是更加萎靡透明,防御力远不如之前。

%70
“那是熊力虚影,汤雄爆发了!”

%71
“熊力擅长拍击,汤雄攻击了这么多次,出现一次熊力虚影也很正常。”

%72
“原本还是僵持局面,但只是一次熊力虚影,就让那小子阵脚大乱,汤雄已经占据上风了。”

%73
“按照统计,每场战斗,他平均能使出五次熊力虚影。那个小子如果拿不出应对措施,肯定要倒霉了。”

%74
演武场外。众人议论纷纷,嘈杂一片。

%75
兽力虚影一出现,就打开局面,调动了观战者的热情。

%76
“我身上有双猪一鳄之力,猪力擅长冲撞,鳄力长于撕咬,熊力还未养成。用拳用掌,不可能发出兽力虚影。而且天蓬蛊也不擅长防御近战肉搏,再遭受一两次兽力虚影的攻击,恐怕就毁了。”

%77
方源心念一闪。便决定转变战术。

%78
跳跳草。

%79
他心中默念,弹簧似的青草旋即在他脚底板上生长出来,然后钻破皮靴底部。

%80
汤雄冲过来,方源脚下一蹬,如青蛙一般,远远跳开。

%81
同时,他左手掌照准汤雄的方向一切。

%82
一记鲜红的月刃,顿时凌空飞射而出,落在汤雄身上。打得他身上的防御光晕一阵乱晃。

%83
血月蛊虽然攻击不足,当终究还是三转蛊。

%84
汤雄楞了一下。再次向方源扑去。

%85
方源故技重施,只远射血光月刃,改变战术,不和汤雄近战。

%86
汤雄只好动用移动蛊,和方源展开追逐战。

%87
方源且打且退,汤雄也不是没有远战手段,但和近战能力一比,就弱得太多了,根本就威胁不了方源。

%88
汤雄气得连声咆哮。咒骂激将方源,周围的观战者也配合得发出一阵嘘声。他们渴望看到激情四射的近战肉搏。

%89
但方源怎么可能轻易被激?

%90
他蛊虫并不成套,近战还未养成,远战也能只算是勉强凑合。

%91
时间流逝,双方真元不断下降。

%92
方源的优势越来越明显,他是三转初阶的淡银真元,而汤雄只是乙等资质。赤铁真元。

%93
尽管力道蛊师真元消耗少,但也禁不住持续使用。

%94
当方源的血月蛊在汤雄的身上造成五六个伤口时,后者不得不主动认输。

%95
他的治疗蛊并不出色,难以止住伤口处不停往外流的鲜血。

%96
“你给我等着。总有一天我要把你拍成肉酱!”汤雄捂住伤口,退下演武场。

%97
他来时凶神恶煞,去时脸色苍白,脚步虚浮。

%98
“想不到第四内场的汤雄,都失败了。”

%99
“那蛊是什么?能形成血红月刃,造成的伤口还会流血不止,我怎么没见过?”

%100
“这小子绝非池中之物,用不了多长时间,就能升到第四内城的演武场去。”

%101
成王败寇,场外众人将更多的目光集中在方源的身上。

%102
于是,方源收获了第三场胜利,同时按照规矩,从汤雄的身上取走那只熊豪蛊。

%103
这是汤雄手中,价值最高的蛊了。此蛊一去,汤雄就只剩下双熊之力,战斗力直接下降三成,更不足为虑。

%104
但方源心中却高兴不起来。

%105
传奇蛊并不在垫脚的星辰石中,又会在哪里呢?

%106
难道说不是这块垫脚石,而是另一块?

%107
但方源回到赌石坊中,那个柜台的一脚已经被修复了。

%108
或者根本就不是这个赌石坊?

%109
方源暗暗摇头,星辰石,垫柜台,赌石区,这些要素结合在一起,就只有这家赌石坊。

%110
“如果没有传奇蛊,我走力道也就没有什么优势。上古时代力道盛行,到了现在,已经衰落的不成样子。失去了传奇蛊,若再走力道,至少得有一个上古力道的传承。可惜,我记忆中仅有的三个上古力道传承,一个在东海,一个在中洲。”

%111
“第三个虽然在南疆。但如今已经被武家占据,早在几年前就在开采。说起来,武家的武姬娘娘也走的力道。正是得益于这道上古传承,才稳固南疆战力第一的宝座。”

%112
当然,这里的第一宝座,只是指的世俗。六转蛊仙那种层次,已经超脱凡俗,不计算于内。

%113
方源又花费了七八天的时间,暗中调查,没有任何进展。

%114
“唉,只剩下最后一个线索,实在不行,就只能放弃了。”方源心道,他不是那种钻牛角尖的人。

%115
最后的这个线索,就是李然——前世得到传奇蛊的主人公。(未完待续。如果您喜欢这部作品,欢迎您来起点()投推荐票、月票,您的支持,就是我最大的动力。)

\end{this_body}

