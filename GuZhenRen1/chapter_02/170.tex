\newsection{魔道的觉悟}    %第一百七十节:魔道的觉悟

\begin{this_body}

%1
“原来三王传承,还有这样的一线生机。我若是得到令牌,肯定更能放开手脚,不至于提前退出去。”包同大为感叹道。

%2
李强则主动端起酒杯,向方源敬酒:“小兽王大人的一席话,真是字字千金。这杯酒庆贺阁下斩杀了百岁童子这个力道败类!”

%3
人走茶凉,刚刚李强还在和百岁童子亲切交谈,如今换做方源,他立即改了词,把百岁童子定性为败类。

%4
“哈哈哈,好说好说。”方源却不端起酒杯,而是看着百岁童子的这些干儿女,不耐烦地挥手道,“今天我斩除罪魁,心情好,就放你们这些人一条生路。不想留下来的,都给我滚。快滚,快滚,留着碍我的眼!”

%5
百岁童子一死,这些干儿子、干女儿早就心中焦惶,听到这话,不禁面面相觑。

%6
“怎么?留下来,想给我杀吗?”方源淡淡冷笑。

%7
立时,人群骚动起来,许多人狼狈而走,酒席瞬间空了一半。

%8
但百岁童子的这些干儿女中,还有少部分留了下来。

%9
“方正大人,您是我的救命恩人呐!”一个干儿子猛地跪下,涕泪交加地喊道,“我是被百岁童子那个家伙逼得认贼做父,小兽王大人您威加四海,气盖八方,救我于水火当中,您是我的救命大恩人呐!”

%10
“小兽王大人,您的强悍已经彻底征服了我的心,请让我留下来。伴随您左右,伺候您吧。”一个漂亮的干女儿娇滴滴地哀求道。

%11
“小兽王大人。您拯救小的于灾难当中,您的大恩大德我永世不忘,恩同再造,请让我叫您一声干爹!”一个七老八十的老头子,跪倒在地上,动情地呼喊着。

%12
哗啦啦。

%13
瞬间,方源的面前跪倒了一片。

%14
百岁童子一死,这群势力的首脑就没有了。立即分崩离析。大多数人逃离出去,而另外一部分人则改弦易辙,想要依附方源。

%15
“哈哈哈……”方源大笑起来,“说的真是动听啊,不错,不错。”

%16
一群干儿女的脸上,也涌现出喜悦之色。

%17
但紧接着方源笑声一敛。面色阴沉下来,低喝道:“一群阿谀奉承之辈!杀人就是杀人,罪恶就是罪恶,什么大恩大德。此等虚伪的赞赏,我从来不屑。我喜欢杀人,我喜欢罪恶。听听,多么直接,多么纯粹。你们也都给我滚,想要报仇的,快去积攒实力。我等着你们挑战我!”

%18
干儿女们既惊愕,又恐惧。纷纷愣住。

%19
“嗯?”方源从鼻腔中淡淡地哼了一声,心念一动,兽影扑杀下去,当场击毙一人。

%20
众人如梦方醒,齐声尖叫,纷纷向洞外狼狈逃窜,屁滚尿流。

%21
留下来的蛊师们,脸色都不好看。

%22
方源喜怒无常,动不动就杀人,让身边的人心中很有压力。百岁童子虽然可恶,但和他相比较起来,可爱了不知多少倍了。

%23
唯有白凝冰,端坐在方源的左手边位置上,蓝色的眼眸半睁半闭,面色平静如冰。

%24
李强的酒杯一直端着,没有落下,此时他也忘了尴尬,勉强笑道:“小兽王大人,斩草要除根呐。这些人放走了,万一日后发迹了呢?保险起见,还是都杀了为妙。小兽王大人记不全这些人也不要紧,我记得。由在下代劳,杀掉这些人,算是刚刚情报的谢礼了。”

%25
“无妨,无妨。”方源将背依靠在椅背上,淡淡一笑。

%26
放走了这些人,自有他的打算,不过却不能明说。

%27
想了想,方源道:“我走魔道,就从未怕得罪人。只要自身不断地强大,复仇算什么?十个人复仇,我就是杀十个人,百个人复仇,我就杀百人。若全世界复仇,我就杀了全世界!如果我被人复仇成功,那就证明我不够强,不够努力,懈怠了修行,死了也活该!”

%28
方源说着这话,眼中厉芒频闪,左右扫视,如恶虎猛兽一般,无人敢和他对视。

%29
“小兽王对别人狠,对自己更狠!”

%30
“这个方正魔性太强了!不惧报复,不怕死亡,置生死于度外……”

%31
“方正疯魔了,心理根本就不正常。和这样的人做敌人,绝对是一场噩梦!”

%32
众人听了方源这话,心中冰凉一片。

%33
方源成功地震慑了众人,便适可而止,展颜一笑:“我们喝酒吧。”

%34
众人端起酒杯,战战兢兢,仿佛伴随一头吃人的猛虎,念及自身安危,原先的美酒也变得索然无味。

%35
但接着,方源又谈及三王传承,暴露出许多秘密。

%36
众人心神完全被吸引过去,一个个的隐秘听入耳中,很多人都兴奋地鼻息粗重起来。

%37
唯有李闲忧虑惊疑:“这个小兽王打得什么主意?居然主动暴露出这些珍贵的情报,他究竟想要干什么?”

%38
一个时辰之后,酒宴结束了。

%39
方源杀了百岁童子,鸠占鹊巢,还主持酒宴。而其他人意犹未尽,纷纷觉得不虚此行。

%40
走出洞口时,他们甚至还都有些恋恋不舍,想要从方源的口中听到更多的消息。

%41
至于酒宴的原主人,那个百岁童子,被撕成两半的尸首还在地上,血已经渗透到了土地中,惨白的骨头月光之下,散发着冰冷的光。

%42
众人谈笑着,走过他的尸体,没有人往这个失败者,投去一瞥。

%43
这就是魔道失败者的下场。

%44
成者王,败者寇。

%45
所有的魔道蛊师,或多或少都有着这样的觉悟。

%46
……

%47
天空中下着淅淅沥沥的小雨。

%48
阴沉的天空,寒风陡峭吹拂。

%49
细雨洒在少女的头发上,肩膀,后背,乃至全身。

%50
“若男少主,人死不能复生,还请节哀啊。”铁家四老中的首领,此时站在少女的身后,关切地劝慰道。

%51
但少女没有说话,以往明亮如星的双眼,此时失神而又空洞,再无往日的坚定并且犀利的目光。

%52
铁若男愣愣地看着眼前的墓碑。

%53
这些墓碑是一块块切开来的山石,上面刻着沉眠者的名字。

%54
铁沐、铁刀苦、铁线花、铁傲开、铁霸修……

%55
这一个个的名字,都能牵扯出铁若男内心深处,最鲜明深刻的记忆。

%56
但曾经和她并肩战斗,一起前行的伙伴们,已经成为了土地中最冰冷的尸体。如同铁若男的心一样,再无一丝温度。

%57
“是我害了你们,我没有尽到一个首领该尽的责任!”

%58
“你们死了,我却独独活着。我是一个懦夫啊……”

%59
“这一切多么像一个噩梦,父亲啊,我给您丢脸了。”

%60
铁若男陷入深深的自责中,除此之外,还有懊悔以及迷茫。

%61
这个天之骄子,在经历了丧父之苦后,努力攀升,如同一颗冉冉上升的正道新星,受到无数人的瞩目和祝福。

%62
但是数月前的一场战斗,方源亲自将这颗新星打落谷底,成为阴沉的角落中,砸在地上,浑身裂纹满布的灰暗陨石。

%63
“唉……”铁家四老之首的铁铉之,看着阴雨中少女单薄瘦削的背影,发出一声深深的叹息。

%64
但就在这时,一个苍老的声音,轻轻的,在他的身后响起:“已经几个月了,若男这个孩子还是这样子吗?”

%65
铁铉之悚然而惊!

%66
什么人,居然如此接近自己,自己却一直没有发觉!

%67
刹那间,他浑身汗毛乍起,闪电般转身,下意识地就要动手。

%68
但一只干瘦如柴的手,轻轻地搭在他的肩头,同时还伴随着一个声音:“铉之啊,稍安勿躁。”

%69
铁铉之顿时浑身僵直,空窍中沸腾的真元海面,被一股强大的无形巨力笼罩下来。

%70
好像是千丈的青山巨峰,陡然镇压下来。

%71
堂堂的铁家四老之首,四转高阶的铁铉之,在这一刻,动弹不得,浑身都被禁锢住,像是琥珀中的一只小虫!

%72
但当他看到来人的面貌时,铁铉之充斥心头的惊骇欲绝,旋即转为狂喜之情。

%73
“啊,是老族长您!”铁铉之脱口而出道。

%74
此时,站在他面前的干瘦老人,就是铁家上一代的族长,铁慕白!

%75
“族长之位,我早就退位让贤了。如今我也不是家老,铉之啊,你直接称呼我慕白吧。”老人温和地摆摆手,笑着道。

%76
“晚辈何德何能,如何敢直接称呼老族长您的名讳!”铁铉之深深地弯下腰,恭谨地对老人一礼。

%77
对于眼前的老人,铁铉之的心中充满了崇敬、孺慕之情。

%78
“名字不过是一个代号罢了,铁慕白这个名字,本身就是用来称呼的。没有什么不妥。”老人言语淡然,眼眸沧桑,已经看破了名利。

%79
铁铉之还想说话,但老人却微微摆手,慢慢走上前去,来到铁若男的身边。

%80
他站到墓碑前,背对着铁若男。然后伸出手来,抚摸着石碑表面,轻轻一叹:“铁家人,死在哪里,就葬在哪里。这是铁家从创建以来,就立下的规矩。你知道为什么吗?”

%81
铁若男仍旧半跪在地上,似乎没有听见,无动于衷。

%82
老人继续道:“因为对铁家人来讲,战死沙场,是最大不过的荣耀!铁霸修、铁沐、铁刀苦、铁线花、铁傲开这些人是这样,你的父亲铁血冷也是这样。将来我死了,也会这样。你死了也一样。”(未完待续。如果您喜欢这部作品,欢迎您来起点投推荐票、月票,您的支持,就是我最大的动力。手机用户请到阅读。)

\end{this_body}

