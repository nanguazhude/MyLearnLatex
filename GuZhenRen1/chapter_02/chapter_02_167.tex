\newsection{地灵}    %第一百六十七节:地灵

\begin{this_body}



%1
砰。

%2
一声闷响,微尘溅起。

%3
方正重重地摔在地上,面部着地,摔了个狗吃屎。

%4
他从高空落下,但没有持续多长时间,便落到实地,没有摔死。

%5
“呸呸呸!”方正趴在地上,吐出口中的草叶和泥土,浑身一阵阵的酸痛和气闷。

%6
“这是哪里?”他缓了几口气,狼狈地站起身来,定睛打量周围。

%7
周围是一片青草茵茵,方正仿佛置身草原。青草随着微风轻轻摇摆,无数的花朵,色彩缤纷,组成花毯,延绵到天际。

%8
在远处,突兀地矗立着一座高大陡峭的水晶山峰。

%9
水晶山峰,呈现半透明状,一片粉红,如梦似幻。

%10
方正站在这里,可以清晰地看到,山峰上面已经有人影在攀登。和整个水晶山相比,这些人渺小得如同一只只蚂蚁。

%11
“第一个攀上巅峰的人,就能获得狐仙传承!”陡然间,方正想起那个神秘小女孩的话语。

%12
那个神秘的小女孩是谁?

%13
登上对面的山峰,真的就能继承这道狐仙传承?

%14
方正正疑惑不定的时候,心底深处传来天鹤上人焦急的声音:“笨蛋徒弟,你还在犹豫什么?赶紧去攀登这座荡魄山啊!”

%15
“师父!”方正又惊又喜,“那个小女孩是谁?还有,为什么我的蛊虫都动用不了了?”

%16
“你现在赶紧跑过去,抓紧一切时间。我给你解释缘由,你好好听着!”天鹤上人匆匆地解释起来——

%17
“这次的狐仙传承非同小可,乃是一道完整的蛊仙传承。继承人只有一个,哪个能得到它,就能一飞冲天。日后成仙得道,大有可能。而其他人,不会得到任何的好处,什么收获都不会有。你刚刚所见的那个小女童,就是狐仙传承。你只有第一个登顶,成为她的主人,才能得到蛊仙传承。”

%18
方正一边跑,一边望着山峰。

%19
他自小生活在南疆,南疆多山。自然知道“望山跑死马”的道理。

%20
“师父,我距离这座水晶山还很遥远。他们那些人却已经在攀登山峰了,这差距太大。我的移动蛊也催动不起来,怎么能赢?”

%21
天鹤上人哼了一声:“无知!这是狐仙的福地,这方天地与众不同!狐仙虽然已经陨落。但是意志还在。你的移动蛊,已经被狐仙的意志禁制住,当然运用不了。其他人也是一样,什么蛊虫都运用不了。你想要到达山峰,只能凭借自己的体力,完成这项考验。”

%22
方正听了这话,不禁气馁道:“动用不了移动蛊。又有这样大的差距,我怎么能追得上去?奇怪,我们接连进入门楼,怎么之间的差距这么大?”

%23
天鹤上人答道:“这是蛊仙的福地。光阴流速不一样。外界一天,这里已经过去了五天。你们在外界,一个个进入门楼,时间差距很短。但一进入这里。时间差距就猛地拉大了。要不然干嘛大比武,决出胜负名次呢?每一个名次。都代表着一定的优势。最后的那名,基本上就没有获胜的希望了。”

%24
“原来如此。”方正这才真正的明白,大比武中获得的名次是多么的重要。

%25
但可惜,他自己有力未逮,实力上的确不如别人。

%26
十大派是中洲魁首,藏龙卧虎,底蕴深厚至极。就算是南疆的超级家族,也比之不上。在这代的精英弟子当中,方正和碧霞仙子、古霆、魏无伤等人,只能算是第二层次。

%27
第一层次上,分别有凤金煌、应生机,萧七星等人。他们的背后都有蛊仙撑腰,跟脚背景极其深厚。从一出手起,就受到悉心栽培。海量的资源供给,从不缺乏。甚至,偶尔还会有蛊仙级的存在,亲自教授,面授机宜。

%28
相比较而言,方正这样的人物,也不过是乡下小子。虽然有所奇遇,但怎么能抵得上蛊仙的支持?

%29
“我的名次不高,那也没有办法。大比武我已经拼尽了全力,就算是铁喙飞鹤群也几乎损失殆尽。”方正心中遗憾,又很无奈。

%30
但他想着想着,忽然又觉得不对劲。

%31
“师父,既然在这里蛊虫无法催动,你怎么仍旧能和我对话?师父,还是您老人家有一手,您真是太厉害了!”

%32
“哈哈哈……”天鹤上人忍不住大笑起来,“笨蛋徒弟,你终于意识到了这点了?不过,你高看师父我了。我生前只是五转,死后魂魄寄托在这寄魂蚤中,更不算什么。之所以能和你说话,那是我仙鹤门的太上长老鹤风扬大人的手段!方正啊,你要记住,在这世界上,只有蛊仙才能对付蛊仙啊。”

%33
“太上长老鹤风扬?”方正心中充满了惊愕,仙鹤门中隐藏着许多蛊仙。其中就有一位鹤风扬,不想他居然和自己有所牵扯。

%34
天鹤上人语重心长地道:“方正,狐仙的传承非同小可,拥有着地灵。也就是说,这片福地拥有自己的意志!就算是蛊仙强攻,都讨不了好。地灵单纯又固执,继承着原主人的执念。硬拼的话,就算地灵不支,也会选择自我毁灭,导致福地崩溃。强攻的蛊仙,也会白白受伤,损失惨重,竹篮打水一场空。因此,他们派遣你们这些精英弟子过来。”

%35
“这次大比武,灵缘斋中出了一个凤金煌,的确厉害!这少女堪称骄子中的骄子,傲视群伦,人中龙凤。你底蕴不足,败给她也很正常。但我们仙鹤门,却未必会输。你就是我们仙鹤门的翻盘希望!”

%36
“我?”方正不由地瞪大了双眼,没有想到自己居然被寄托了这么大的希望。

%37
他忽然明悟过来。

%38
这场狐仙传承的角逐,看似是十派精英弟子争锋,但其实是十大派的较量。自己这些人都是棋子,而十派的蛊仙则是棋手,他们居于幕后,排布棋子,相互之间角逐竞争。

%39
而他自己懵懵懂懂,被仙鹤门选为了秘密武器。

%40
“师父,门派对我的希望太大了点吧?我现在蛊虫都动用不了,怎么可能做唯一的胜利者?”

%41
“呵呵呵……”天鹤上人得意地笑出声来,“方正,你可知道这座水晶山是什么?这就是人祖传中赫赫有名的荡魂山。”

%42
“荡魂山!”方正惊呼一声。

%43
人祖的传说,在这个世界广为传播,方正自然知道荡魂山的大名。

%44
天鹤上人继续道:“这天底下的生灵,只要进入荡魂山的范围,它们的魂魄就要受到震荡。越接近峰巅,这股震荡之力就会越强。很多生灵魂魄不强,往往到达山腰处,就会被荡得魂飞魄散。”

%45
“当然,这是正道传承,不像魔道传承那般危险。荡魂山的威能,已经被地灵压制住,使得你们这些精英弟子,有登上巅峰的可能。若是有人支撑不住,就会被地灵传送出去。可以说,你毫无生命危险。”

%46
天鹤上人的语气越加严肃:“方正,接下来我要说的是重中之重,你仔细听清楚了!”

%47
“你小小年纪,就惨遭灭族,而凶手偏偏是你的亲哥哥。你立志复仇,意志坚毅。遭受这样的大变,你的魂魄凝练,远超同龄人。但和凤金煌、萧七星这些人相比,还有差距。”

%48
“但你不要灰心丧气,鹤风扬大人的谋算,就精妙在此处。他暗施手段,将寄魂蚤复苏。接下来,我就会用我的魂魄来支持你,将你推上荡魂山之巅!”

%49
天鹤上人是五转蛊师,一生磨砺难以计数,同时寿命悠长,经验丰富,魂魄上的底蕴自然要远远强过凤金煌、萧七星等人。

%50
这些天之骄子,虽然意气风发,前途不可限量,但现阶段比起天鹤上人,魂魄上的底蕴仍旧不足。

%51
方正又喜又忧:“有师父相助,我就有获胜的可能。但这样作弊,是否有失公允?若是被其他蛊仙发现了,又该如何是好呢?”

%52
天鹤上人哼了一声,训斥道:“傻小子,这可不是作弊,而是能力,是才华。有能力才华,自然就得到蛊仙传承。这就是公平!你放心,你的身后可是仙鹤门,有鹤风扬大人护着你,其他蛊仙不会轻易动你的。况且蛊仙们,不能进入福地。一进入福地,他们就会受到地灵的攻击,因此就算是他们发现端倪,也只能干瞪眼!哈哈哈!”

%53
“地灵有这么厉害?连蛊仙都不是对手?”方正感到诧异。

%54
“一方拼死攻击,一方投鼠忌器……而且地灵和福地一体,在福地中,地灵就是仙神,天地都要受其操纵。掌握地灵,就能掌握福地。”天鹤上人答道。

%55
方正忽然灵光一闪:“哦!我懂了,刚刚看到的那个小丫头,就是地灵!”

%56
“哈哈哈!”天鹤上人大笑,“徒弟,你还不算太笨,总算领悟到了。不错,那个小女孩儿,就是狐仙地灵。一路向前冲吧,勇往直前。这荡魂山非常难爬,至少要爬个一年半载。以防万一,为师接下来不会多说话,只在暗中帮助你。你有我的帮助,至少有八成的机会夺得狐仙传承!只要得到狐仙传承,你就是一飞冲天,从乞丐变成富翁,一口气吃成胖子。这是一步登天啊!”

%57
“我明白了,师父!我会尽最大的努力,不会让您老人家失望的!”方正一双虎目闪烁着坚决勇敢的光。

%58
“去吧,整个仙鹤门都在看着你。”天鹤上人一句话,就鼓动了方正,将他体内的热血点燃。

%59
方正干劲十足!

\end{this_body}

