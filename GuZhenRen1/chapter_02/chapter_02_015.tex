\newsection{轰!}    %第十五节:轰!

\begin{this_body}



%1
“白骨山,终于要到了。”一处山坡上,方源望着远方的白色山峰,兴叹道。

%2
在他的身边,白凝冰默默地立着。

%3
两人均是衣衫褴褛,满脸疲色。

%4
就在刚刚,他们艰难地甩脱了气钢猪一家五口的追杀。

%5
气钢猪是很特殊的兽群,它们数量不多,通常以家庭为单位,成员数量不会超过十。但只要是成年气钢猪,都至少是百兽王!

%6
追杀方源和白凝冰的这家五口,猪爷爷已经是千兽王,猪爸爸、猪妈妈是百兽王,就连猪儿子、猪女儿,身上都寄生着一转蛊虫。

%7
距离夺取猴儿酒,又过去了五天。经过艰难的跋涉,白骨山已经遥遥在望。

%8
南疆广阔多山,普通的丘峦都不配称呼为山,只有至少高达千丈,才可有资格称之为山。

%9
白凝冰站在山坡上,凝神远眺。

%10
她是第一次,目睹白骨山。

%11
青茅山周围,还有丘峦陪衬。但这白骨山,却像是孤独的将军,周围地形十分平缓。它拔地而起,高耸入云,一片惨白之色。

%12
这种白,不是雪的白,而是骨的白。

%13
白骨山,顾名思义,山上的每一块石头,都是骨质。人们又称之为骨石。

%14
白骨山上并非死寂的绝域,山上生长着许多特殊的植被,生活着大量的骨兽,同时也有不少骨系的野生蛊虫。

%15
白凝冰看着看着,就皱起眉头。

%16
任何的高山长河,都是元气精华凝聚之地。白骨山上并没有人迹,完全是一座野山。山上聚集着大量的野兽、蛊虫,致命的植物。危险程度绝对更高一筹。方源执意要深入此山,有什么目的呢?

%17
或者说,到底有什么东西,在如此强烈的吸引着他?

%18
方源此刻则浮想联翩。

%19
白骨山,如今还是个荒山,没有人烟。但这个情况,在十年之后,就会发生彻底的改变。

%20
一个大型山寨,将举族迁徙过来,并在这里发展壮大。

%21
这个家族姓百,百家寨。

%22
在未来,将以白骨山为中心,辐射方圆数千里,成为此地的霸主。

%23
最令方源印象深刻的,并非是百家寨势力得到腾飞。因为在这个世界上,个人的力量完全可以凌驾于集体。

%24
而是百家寨的一对同胞兄妹。

%25
百生和百花。

%26
这对兄妹,在十八岁的那年,在白骨山后山比试,误打误撞发现了一处山洞。

%27
在这山洞中,他们开启了传承。一位正道四转蛊师的完整传承。

%28
这个蛊师名字不详,只留下称号——“肉骨上师”。

%29
百生和百花,因此受益,接受了传承之后,成长为正道双星。在近百年后,双双晋升五转,接掌百家寨。

%30
两位五转蛊师的力量,将家族势力推上一个巅峰。

%31
“完整传承中的蛊虫,必然涵盖攻击、治疗、防御、移动、存储、侦察六大方面。我得了此传承,就可站稳脚跟,进退有据。”

%32
先前方源和白凝冰从青茅山逃遁出来,修为参差不足,蛊虫也不全面。就好像是在惊涛骇浪中行舟,在悬崖边上行走。运气稍一不好,就有陨落之危。

%33
辗转颠沛之后,机遇来了,杀了一个重伤的魔道女蛊师。夺了她的蛊虫,有了饭袋草蛊,跳跳草蛊,算是能勉强求生了。

%34
但终究还是有缺陷的。

%35
不仅是因为一直缺少治疗的蛊,还因为修为太低微了。

%36
就算是方源晋升到了一转中阶,那又如何呢?青铜真元到底还是青铜真元。

%37
他现在靠的,就是甲等资质,还有天元宝莲针对真元的快速回复能力,使得他勉强应对局面。

%38
但真正算起来,他的战斗力微乎其微。若非是白凝冰,他在江边搁浅时,就惨死鳄腹了。

%39
能走到这里,还是靠的白凝冰。

%40
不过靠人,终究不如靠己。

%41
“如果得到白骨山传承,很多问题,就都能得到解决了。”方源心中暗道。

%42
首先是玉骨蛊。得了此蛊,就能令他浑身骨骼,摆脱凡胎俗骨的脆弱,使骨骼更坚硬,更坚韧。他现在的身子,只能承担双猪之力。用了玉骨蛊,就能在这基础上,再增添一鳄之力。

%43
然后是治疗蛊,方源记得这传承中,有一只很著名的三转治疗蛊“肉白骨”。在前世被百花得去,令其成了著名的治疗蛊师。

%44
最后是方源最看重的“骨肉团圆蛊”。

%45
此蛊,乃是骨肉上师独创,天底下唯此一份。此蛊妙用非凡,在前世令南疆各大势力为之而侧目。

%46
若按照作用划分,可将世间蛊虫大抵划分七大类。

%47
一攻击,二防御,三治疗,四侦察,五存储,六移动,七修行。

%48
不管是酒虫、四味酒虫,还是人兽葬生蛊、舍利蛊,或者是天元宝莲,皆在修行类中。

%49
这骨肉团圆蛊,亦是修行类中的奇蛊。

%50
和阴阳转身蛊一样,它是一对蛊,分别作用在两位蛊师身上。能令这两位蛊师进行双修,修为齐头并进,事半功倍。

%51
“我若得了骨肉团圆蛊,借助白凝冰之力,修为必能突飞猛进。在三转以后,以骇人的速度暴涨!尤其是在前期,比酒虫的效果,还要好得多。骨肉团圆蛊,我志在必得!”

%52
想到这里,方源不着痕迹地用眼角余光,瞄了白凝冰一眼。

%53
白凝冰懵懂不知,还在望着白骨山。

%54
方源心中冷冷一笑,正要出发,忽然几个身影急速飞来。

%55
“嗯?正道的蛊师!”不管是方源还是白凝冰,均心中一震。

%56
一共四名蛊师,迅速靠近。在距离百步之时,他们落在地上,向方源和白凝冰二人靠拢。

%57
领头的一位老蛊师,散发着很明显的三转气息。其他三位,却是二转。

%58
他们服饰一致,行动默契,气息精悍。

%59
“这荒郊野岭,怎么碰到了正道蛊师?”

%60
“蛊师和野兽完全是两个概念。我虽然是三转巅峰,锯齿金蜈钝边,又有方源拖累,恐怕不是这些人的对手。这下麻烦了……”

%61
四位蛊师一步步逼来,方源和白凝冰二人均心中叫苦。

%62
……

%63
黄昏。

%64
天边残阳如血,晚鸦嘎嘎叫着,飞回巢中。

%65
铁傲天一脸冷色,走在队伍的中央。

%66
从青茅山出发时,他身边有八人,各个都是族中好手。如今却只剩下三位。

%67
一想到这一路上的牺牲,铁傲天的心就开始滴血。

%68
伤亡太惨重了!

%69
这远远出乎他的意料。

%70
不是他们修为不足,而是运气实在太差!

%71
追查到白凝冰和方源的痕迹之后,他们开始顺着黄龙江漂流而下。

%72
但黄龙江水流湍急,难以留下痕迹。就算是动用了蛊虫,也有侦察追踪的强手,他们仍旧追过了头。

%73
不得已,他们逆流而上,耗费了一段时间,终于找到了方源和白凝冰的竹筏搁浅的位置。

%74
然后紧接着,麻烦来了。

%75
他们遭遇到了大批的六足鳄群的袭击。

%76
说来真是倒霉,原来这浅滩是六足鳄群的一处产卵地。遭到彻底的破坏之后,原本霸占这里的六足鳄群也覆灭了。

%77
兽群之间,也是有势力划分的。原来这块地盘的主人,已经死亡。这块无主之地,自然而然就吸引了周围的几支六足鳄群的注意。

%78
正当它们要将势力蔓延到这里时,铁家的追捕队伍登上了岸。

%79
“什么东西,竟然敢闯入我们的地盘?”

%80
“这块地盘是我六足鳄群的!”

%81
“想抢我的地盘,找死啊……”

%82
野兽的领地观念,绝不容侵犯。于是一场大战爆发,在两支千兽群,三支百兽群的围攻下,铁家队伍痛失两人,被迫逃离。

%83
方源处理痕迹的手段,相当老道。这使得他们的追查,一直得不到有效的进展。

%84
在蛊虫的帮助下,他们终于摸清楚了方源的前进方向。

%85
然后,彩色的噩梦降临了。

%86
一头轩辕神鸡从天而降,盯上了他们,将他们认作为食物。

%87
如今,整个逃离过程,深深地埋藏铁傲天的心中。轩辕神鸡的身影,成了他每晚从梦中惊醒的罪魁祸首。

%88
轩辕神鸡带走他们三位同伴的性命。其中就有他们最专业的侦察蛊师,还有三转修为的防御蛊师。

%89
损失相当惨重。

%90
如今,负责侦察的蛊师,都是轮流客串的。

%91
虽然遭受了如此的重创,但铁傲天并没有想放弃。

%92
他是铁家堂堂的四公子,甲等资质,从小到大都被寄予厚望。他修行刻苦,继承了铁家人刚强坚毅的性格特点。

%93
支援铁血冷父女,是他第一次正式的出山任务。

%94
然而他只救了铁若男,神捕铁血冷牺牲了。这和他理想的结果,有相当大的差距。

%95
但如果他追捕到血海传承的魔道余孽,替铁血冷报仇,那将是何等的成功啊。

%96
这样的功劳,将转化为他日后竞争族长之位的资本。令他获得更多族人的支持。

%97
他并不担心魔道余孽的实力。在追踪过程中,他们已从蛛丝马迹里发现,方源和白凝冰二人的战力有限,也许是因为有伤在身,只相当于一位三转蛊师的战力。

%98
“虽然损失了许多同伴,但我本身就是三转,还有铁刀苦也是三转。在加上其他两位二转蛊师的策应,实力上牢牢压过那两人。一旦将魔道余孽追捕到手,那么之前的人员损失也会成为我‘坚韧不拔’、‘决不放弃’等等优良品质的证明!”铁傲天眼中闪着光。

%99
“四公子,前方又发现了些微痕迹。看来我们追捕的方向没有错!”这时,负责侦察的蛊师回来禀报道。

%100
“哦?快带我去看看。”

%101
一盏茶的时间,两个深坑被挖开,暴露出一大堆草裙猴的尸骨。

%102
“这些草裙猴死后不到一周,公子,看来我们就要追上那两个家伙了!”铁刀苦惊喜地道。

%103
铁傲天深呼吸一口气,神情陡然振奋!

%104
“终于啊,大功要告成了。”他捏住双拳,兴奋地踱步。

%105
他仰望西方天空,晚霞映照在他年轻的脸上,他双目闪闪发光。

%106
一切的忍耐和努力都没有白费,此刻终于就要得到结果!

%107
“虽然太阳就要落山,但是我却从中看到了未来和希望啊……”他心中长叹,忽然起了兴致,要登上这片山坡,享受这美妙的时刻。

%108
身边的几位蛊师,注视着他,均流露出敬佩仰慕之色。

%109
“四公子,到底是四公子啊!”

%110
“这一路上,我们都产生了放弃的想法。惟独四公子一直坚持,如今终于要得到正果了。”

%111
“从四公子的身上,我看到了家族的希望,和光明的前途。”

%112
“此生我将追随四公子,矢志不渝!”

%113
几人看着铁傲天缓缓攀上山坡,一时间都看痴了。他们仿佛看到日后,铁傲天一步步登上族长之位的情景。

%114
然后在下一刻,剧烈的爆炸猝然发生。

%115
轰!!!!!

\end{this_body}

