\newsection{近乎讹诈}    %第四十五节:近乎讹诈

\begin{this_body}

黎明时分,东方欲晓,晨曦微露。

夜间的寒气凝结成露珠,残留在草叶上。金家寨临时的集市,正在被拆除。帐篷被收起来,地摊的地毯被一一卷起。货物包扎好,重新装上。

在金家寨停留了多日后,商队准备再次启程。

对于商人而言,不管货物卖掉了多少,又收购了多少,总归是有赚头口因此,许多人的脸色虽然疲惫,却充满了喜气和笑容。

小蝶的脸色却奇差无比。

“小齤姐,我刚刚查看了一下,那黑土几乎把所有的货物都给换了。在其中,我居然还看到了满满三车的金簪草!”

“金簪草?”商心慈闻言,柳眉微微一皱。

小蝶愤惶不已,拽着商心慈的胳膊道:“就算是我这个门外汉,也知道金簪草这种东西根本卖不出去。偏偏他还换了那么多。小齤姐,这个黑土完全是在胡闹!”

“小蝶你先消消气,稍安勿躁。”商心慈拍拍小蝶的手,“这金簪草应该是他昨晚换的,我不知情。不过他换来的其他货物,却也有一定的道理。

你想他一个凡人,能做到这一步,已经不错了。”

“小齤姐,你怎么在替他说话呢?我这也是为小齤姐你好啊。这些货本来就是我们的,何必白白的被外人败光呢?关键是他又赔不起!张柱大人,您也不劝劝小齤姐……”小蝶嘟着嘴。

一旁的张柱叹了口气:“小齤姐,小蝶说的有道理。先前借给他,是想试探他。现在已经看透了,何必再让他折腾下去?我们都相信小齤姐你的能力但是能减少的损失,为什么不尽量减少呢?将来到了商家城也能少奋斗一些。”

“起……”商心慈沉吟不语,到底还是年轻,眼中闪过犹豫之色。

先前她还觉得,这方源换的商货有些道理。但是这金簪草……,

换来这么多金簪草,绝对是最大的败笔。

金簪草虽然易于保存,但需求很少,根本就卖不出去。一下子手中囤积这么多,迟早要烂在手里。到最后,不得不降价抛售,可以说是注定亏本。

“请问您是张心慈小齤姐吗?”就在这时,一位满头大汗的中年男蛊师面带着急之色,一阵小跑到商心慈的身前。

看其腰带铁牌,是为二转蛊师。

商心慈面露微笑:“正是小女子请问阁下是?”

中年男子一抱拳:“我是族长大人的亲卫,奉族长之令有一个不情之请。”

“哦?请说。”

“阁下在昨晚应当收购了一大批的舍簪草。这事情原委是这样子的,我们族长大人酷爱金簪草,因此亲手养了一些观赏。无奈小少主偷偷挖取,背着族长交易。现如今,族长已经将小少主关了禁闭。同时想赎回卖给小齤姐的金簪草,恳请小齤姐你能放手。”

亲卫蛊师说得客气但软中带硬,显露出坚决的态度。

“小齤姐……。”张柱脸色凝重,忍不住出声提醒道。

亲卫蛊师代表着金家族长,这事情可大可小,处理不好,会很严重。

商心慈看了一眼张柱,点点头表示明白:“贵族长的心情,我十分理解。其实我也是一位爱花之人,金簪草必定全部交给贵族,不留一棵。”

“小齤姐通情达理,实在令人高兴。”亲卫蛊师脸色一缓,露出微笑。

商心慈继续道:“这笔交易,是由我的一位下属负责的。我这就将他唤来。”

方源早就暗中关注着这边的动静。

“黑土,你有麻烦了,小齤姐叫你过去呢。”小蝶来唤他。

方源赶到亲卫面前,抱拳道:“收购金簪草的买卖是我负责的,听说贵族要重新买回去?”

亲卫蛊师见来的人,居然不是蛊师,而只是个凡人,诧异之后,脸上流露出微微的不屑和倨傲。

他鼻腔一哼:“没错。凡人,开心吧。族长大人宽容仁慈,愿意用三千块元石买回你那三车的金簪草。”

“这么多?”一旁,小蝶吃了一惊,眼中流露出明显的喜色。

张柱紧皱的眉头,也渐渐的松缓开来。三千块元石,应该是按照金簪草的市场最高价算得,可见金家族长的诚意。

但方源却摇摇头:“金簪草名贵非凡,只出三千块元石就要买回去,这是否太显得没有诚意了?”

亲卫顿时皱起眉头:“什么?按金簪草的市价,最高也就这么多。凡人,你按什么价收购的?”

方源摸摸鼻子:“不谈收购价,我们都是做生意的人,自然要低买高卖。三千块元石太少子,不卖!”

“你!”亲卫咬咬牙,伸出手掌五指分开,“好吧,那就再涨两千块元石,五千块元石!”

小蝶吃惊得瞪眼,她看着亲卫伸出的五根手指,脸色很快泛出兴齤奋的红色。

“五千块元石?这可是你说的,不能反悔!”她高兴地几乎要蹦起来,没想到事情如此发展,方源已经赚翻了。

但方源却仍旧摇头。

亲卫脸色一寒,威胁道:“凡人,你不觉得你贪心了吗?这些金簪草本来就是我族之物。你们私下交易,已经是不被允许的。没有交易的凭据,我甚至可以说,你们偷偷窃取了金簪草!!”

亲卫的怒气,让张柱不禁心头一跳,他看向方源:“能卖就卖吧。”

方源哈哈一笑:“你卖我买,本就是两厢情愿。更何况,这还是你们的少主卖给我的。如果硬说是偷窃,那我也没有办法。金家强盛,大可以欺凌弱小,直接抢回去。喏,货物就在那里,你大可动手。只是据我所知,收购金簪草的并非我一家。许多人都有份,金族是否都要抢过去?”

方源早料到有此情形,因此昨晚只买了大部分。剩余的金簪草,那蛊师都卖给了其他人。

“你!”亲卫大怒,但受方源挤兑,只能咬牙切齿。

他手指着方源:“就你家收购的最多,臭小子,你是想给我难堪吗?”

“当然没有这个想法,我只是想做成这笔买卖罢了。”方源拱拱手道。

“哼,罢了!那就再加两千块元石,七千块!凡人,把你收购的金簪草都提过来。”亲卫低喝道。

“黑土,不妨就卖了罢。我们做生意的,讲究和气生财。”商心慈也扛不住这份压力了。

“既然小齤姐你都这么说了……”方源兵点头,紧接着话锋一转,“那我就退让一步。八千块元石,你们可以买回去我手中全部的金簪草。”

此话一出,商心慈几人都神情一滞。

亲卫反应过来,怒不可遏:“什么?你个混蛋!!”

方源脸上泛笑:“在商言商,还请蛊师大人不要生气。其实我原本是想买到一万块元石的,大人若是坐不了主,不如我和贵族的族长面谈?”

“不必了!”亲卫甩手,极其厌恶地看了方源一眼,“你一个凡人,有什么资格面见族长大人?快快滚去提货,你如此趁人之危,本人记下了。哼!”

这样说,无疑是答应了方源的提价。

很快,双方就完成了交易。

方源用不到五百块元石的货物,只是经过一晚上,简单的转手,便换来了这八千块元石的收益。

“小齤姐,满满两箱的元石!”小蝶眉开眼笑。交易的整个过程中,她都是心惊胆战,现在看到元石,顿觉得一切都值了。

连带着看向方源的目光,也发生了变化。

“你是不是早知道?不对,应该是瞎猫碰到死耗子吧!”她上下打量着方源,评价道。

“这样做,恶了金家,区区八千块元石,却有些不值。”张柱一直紧紧皱起眉头,他有些不满地看向方源,叮嘱道,“今后不要在这么弄险了。”

方源微微一笑,转身看向商心慈:“按照原先的约定,这这八千块元石里有小齤姐的一半。我的那份元石还请小齤姐帮我保管。”

“都办妥了?”金家族长站在山坡上,远远望着商队缓缓离去。

一位家老站在他的身边,禀告道:“是,族长。卖出去的金簪草,都买回来了。只是张家趁机讹诈,实在可恨。”

舍家族长挑起眉头:“哦?说来听听。”

家老便详细说了。

金家族长笑了笑:“区区八千块元石罢了,不必在意。那个张家小齤姐,却有些智慧,推出一个凡人家奴来顶杠试探,让她赚去了这笔钱财。”

“族长大人,属下担心的是,这张家是不是已经知道我族的秘密,所以才趁机讹诈呢?”

“呵呵呵,不要疑神疑鬼了。他们要是知道金簪草对我族的重要性,何止讹诈区区八千块?要么就全买下来,要么就不会买。不过,为了以防万一,还是派遣蛊师盯住他们,直到他们离开黄金山地界。看看是否有人偷偷潜去黄家寨。若是有,务必当场格杀!”

说到这里,金家族长杀意毕露。

家老心中一凛:“是,族长大人!”

最后再望一眼身后的黄金山,方源转过头来,嘴角流露出笑意。

不远处,马车中,商心慈掀开窗帘,将视线投注到方源的后背上。美眸中光辉流转,不知道在想些什么。(未完待续

------------

\end{this_body}

