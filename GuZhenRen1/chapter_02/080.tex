\newsection{这就动手?!}    %第八十节:这就动手?!

\begin{this_body}

魏央感到苦恼,这些天他一直都很头疼。

商燕飞想要招揽方白二人,但在魏央的旁敲侧击之下,方白二人对此却没有流露出丝毫的兴趣。

这也很正常。

年轻人嘛,血气方刚,心怀理想,不想寄人篱下,总觉得这世间的事情,只要努力了就能达到。

天真,天真呐。

魏央可以理解,想当年自己不就这样子的么。

所以,他没有直接邀请方白二人。

一旦说白了,被方白二位拒绝,那就是没有寰转的余地。魏央处事老辣,他已经制定了计划,令方白二人一步步进入瓮中。

他并没有想出什么阴谋诡计,招揽这种事情,讲究你情我愿。强行逼迫,反而会坏事。

尤其是魏央这些天和方白二人相处下来,知道他们的性格绝不会妥协。

所以魏央打算潜移默化,用堂堂正正的阳谋取胜。

对此,他极有信心。

“蛊师修行靠的是什么?资源!正道蛊师还好说,魔道蛊师……呵呵。”

魏央曾经也是魔道蛊师,从演武场走出来的。

他太清楚魔道蛊师的艰辛了。

蛊师的修行离不开元石、蛊虫以及种种饲料。修行越高,蛊师对这方面的需求也就越高。

刚开始的时候,蛊师还能依靠自身能力满足这种需求。

但是越往后,都会发现需求越来越大。竞争越来越强,有时候反而连个人温饱都不能解决!

魏央在一转的时候,就踏入演武场。从一转到二转,再到三转,他越来越强大,却又感到自己越来越弱小。

就好像是饱学之士,知道的越多,就越会明白自己的无知。

魏央越强大,就越明白自己的弱小。

个体是如此的弱小,唯有相互依靠。依托于家族,才能更好的生存。

当他明悟到这点之后,他接受了商燕飞抛来的橄榄枝。

他就是这样一路走过来的。

“当方白二人有朝一日能够领悟到自身的弱小时,商家的招揽就能水到渠成了。不过在此之前,我不能坐看方正走弯路,这会浪费他的青春和精力,浪费他的大好资质!”

在魏央看来,方源选择力道,完全是一种错误。

是年轻气盛。是懵懂无知。

力道?

商家城演武场中,走出来的魔道蛊师。哪一个是以力道称雄的?没有!

力道是一个注定底层,注定登不上台面的蛊师流派。

想要靠着力道出人头地?没有希望的!

然而,事情的发展并没有随着他的意愿而转移。

这些天来,魏央主动找到方源,劝说了好几次。但每一次都被方源回绝,并且态度始终坚定不移。

魏央充分认识到了方源性格中的执着,他铩羽而归多次,回去后冷静思考,决定换一种劝说方式。。

今天。他再次找上门来。

“魏大哥,你建议我进入演武场,参加演武?”方源神情微微疑惑。

“是的。魏大哥就是靠着演武场,一步步修炼出来,好的经验当然要分享了。”魏央神情相当诚挚。

紧接着,他竖起三根手指:“你参加演武,有三大好处。”

“第一。能赢取钱财。你每参加一次演武,只要胜利,都有元石可拿。观战的人越多,元石就越多。”

“第二。能得到蛊虫。演武场的规矩中有一条,胜利者能从失败者的手上,选取一只蛊收为己用。”

“第三,参加演武能锻炼你的战斗技巧,同时实战检验你手中蛊虫的组合威力。以便让你做出相应的调整。”

方源听到魏央最后一句话,顿时心中了然。

魏央这些天,力劝他放弃力道,改换其他流派。失败之后,他决定旁敲侧击,借助演武场实战,让方源充分意识到力道的缺陷,转而让方源放弃力道的修行。

“力道修行,乃是我深思熟虑的选择,对未来大有帮助,怎么可能会改变?不过,演武场也是我接下来的打算……”

想到这里,方源点点头,答应了下来。

魏央心中大喜:“先前你和我比武,只是借用了演武场地。商家城的演武,已经形成完善的制度,你要真正参加的话,还得先报名。跟我来!”

于是两人来到第五内城的演武区。

这里远比第三内城要热闹许多,人流穿梭着,熙熙攘攘的声音传入方源耳中,一股浓烈的氛围扑面而来。

“听说了没?李好又胜了,这已经是他第十三个胜场,看来不久之后,他就能升到第四内城。”

“他有三转修为,进入第四内城是板上钉钉的事情。”

……

“刚刚王大汗和马德全的战斗看了没?很精彩啊!”

“王汗输给了马德全,丧失了最主要的雨滴蛊,几乎是废了。”

……

“赵大熊和张牛的战斗要开始了,快去看看!”

“这两个人都是走的力道,场面一点都不好看。还不如去看乔大和乔二的兄弟之争呢。”

……

魏央特意做了伪装,他一边在前头给方源带路,一边道:“你要在第五内城,得到净胜三十场,就能升到第四内场。在第四内城的演武区,净胜八十场,就可升入第三内城。什么是净胜场数?举个例子吧,如果你胜利十场,失败两场,那就是净胜八场。”

“演武场分有三档,低档的在第五内城,中档的在第四内城,高档的在第三内城。老弟你是演武场新手,不管是胜场还是败场都是零。按照规矩,你得先从第五内城的低档演武区开始打起。”

“你先不要着急,以我对你的战力评估,你很快就能从低档区,升到中档区。至于高档区,可得花费一番功夫。来的快,应该要有一年半载的时间。”魏央拍拍方源的肩膀道。

他承认方源有战斗才情,但是力道向来只能在中低档区逞能,到了高档演武场,强手如云。力道就会显露出颓势,被其他流派死死压制。

在魏央的带领下,二人来到一处大堂。

大堂中嘈杂一片,都是申请挑战,查看场次等等的蛊师。

有参加演武的,也有专门过来观赏战斗的,还有专门设立赌局盘口的。

魏央不去和他们挤位置,而是轻车熟路地推开一扇小门。

小门后是一个狭长的过道,两位蛊师把守着。

其中一位立即走上前来。对魏央方源二人道:“这里是特别接待处,闲杂人等一概不许进入。”

魏央出示了一下家老令牌。两人连忙行礼放行。

过了通道,也是一处厅堂,布置有四张办事桌。

其中有三桌,已经在接待他人。

这里比外面的大堂,要幽静多了。

方源用古月方正的名义,完成了报名。缴纳了五百块元石之后,得到一只藤讯蛊。

这蛊如一段藤蔓,长有一片宽大的碧绿树叶。

一转蛊,里面记载着方源的一些信息。

这蛊不是方源买来的。只是商家城借的。方源负责喂养,但不能修改里面记载的内容,唯有商家方面能够修改。

当然一转的藤讯蛊,破解并不困难。只是商家方面也有信息备份,再加上群众监督,极少出现弄虚作假的情形。

“演武场的规矩不多,比较自由。如果你想参加战斗。就到这里来申请一下,商家方面会给你安排对手。可能比你强大,也可能比你弱小。你也可以指定具体对手,但需要对方同意。一人一个月有强行挑战的权利。不允许对手拒绝。”

“此外,演武场对战斗场数也有限制。一人一天最多只能战一场。每隔十天,至少参战一场,否则消除胜利场数一次。如果你的净败场达到五,你的资格就被取消了。想要参战,就得重新报名了。怎么样,今天有没有兴趣来战一场,试试手?”魏央笑道。

方源点点头。

这里是特别接待处,接待员办事效率很高,很快就给方源安排了一个对手:“时间是在未时三刻,地点在五号演武场,地形是草原。”

距离未时三刻,只有半刻钟的时间。

魏央带领着方源来到五号演武场时,对手已经在里面等待着了。

他是个青年,身材高瘦,容貌普通,一身青衫。

看到方源走入场地,他眼中闪过一丝喜悦之色。

看方源相貌,明明是个少年,能有多高的修为?反观自己,最近晋升到二转中阶,此战正是出关以来第一战,当讨个好彩头。

方源走入场地,五号演武场是个中型的斗场,比上一次和魏央交战的石板演武场要大出一倍不止。

演武场中绿草茵茵。方源脚蹬皮靴,踩在上面,感受着脚底下松软的泥土。

周围观战者只有两三个人,其中包括伪装过面貌的魏央。

观战也需要支付元石,方源现在不过是无名小辈。他的对手汤青倒是小有名气,只是闭关久了,关注他的人本来就不多,已经消失殆尽。

当——!

一声清脆的钟鸣,战斗正式开始。

“在下汤青。”男青年风度翩翩,向方源拱了拱手。

跳跳草。

方源心念一动,脚下用力一踩,身形猛地窜出。

“我操,这就动手?!”汤青大吃一惊,没料到这少年如此无耻,竟公然偷袭。

这家伙一点规矩都不讲!

(ps:今天就这一更,准备利用晚上的时间,炼个静心蛊。最近压力大,比较烦躁。下一章传奇蛊面世!)(未完待续。如果您喜欢这部作品,欢迎您来起点()投推荐票、月票,您的支持,就是我最大的动力。)

\end{this_body}

