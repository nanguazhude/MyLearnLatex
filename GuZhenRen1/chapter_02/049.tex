\newsection{力所能及就可心安理得}    %第四十九节:力所能及就可心安理得

\begin{this_body}

张柱放声痛骂,但终究还是被卷入其中。一路险死还生的惊险追逐,两人甩掉一头飞象。最终跑到一处绝壁,被两头飞象堵住出路。

地面震动,飞象冲撞过来。

“小子,分开跑!”张柱大喊一声,向左边飞扑。

“妈呀!”陈鑫大叫一声,没有来得及动作,飞象就一头撞过来,然后绝壁被砸出大坑,象牙插进了山石当中。

两只飞象,皆暂时受制。

“天佑我,我命不该绝!”张柱狠狠地喘了几口粗气,瘫软在地上……

山壁在摇晃,两只飞象皆在怒吼。不断用力甩动头颅,碎石屑飞溅,象牙插进去的洞口在不断扩大。

张柱看得一阵心惊,知道这两头飞象很快就会脱困,连忙强撑起身子,努力爬起。

他刚刚站起来,忽然风声传来,浑身一震!

哧。

一根白色的骨枪,带着螺旋的纹路,从他背后刺穿到他的胸前,然后枪尖死死的钉在地上。

鲜血顺着骨枪,流淌下来,滴在地面上。

张柱动作一滞,微微张开嘴巴,从嘴角处立即溢出一股鲜红的血液。

他缓缓低头,看到这致命的螺旋骨枪。

起先,他还以为是白羽飞雪的象牙,但是旋即,他意识到这明显是蛊师的攻击。

“是谁?”他想要扭头,看清在背后暗算自己的凶手。

但下一秒。

哧,又一根骨枪激射!

这一枪,直接从他的脑后射入,然后从他的嘴巴里窜出来,枪尖定死在地面上。

张柱被牢牢的固定住,双眼徒劳地睁大,瞳孔却缩成一点。

他死了。

死不瞑目。

一个隐蔽的角落里,方源远远的望着。

这些天来,他已经将张柱的底细摸透。这个张柱是他的障碍,必须要铲除掉才行。

两根白色的骨枪渐渐崩解,化作点点白光,消散在空中当中。

张柱没有了支撑,一头栽倒在地上。

一头白羽飞象拔出牙齿,冲到张柱的尸体面前,一脚踏去。就将其踩成肉泥,骨骼粉碎。

白色的羽毛飞散,飞象拔地而起,又飞上了半空中。

看到此处,方源这才收回视线,张柱已经彻底死亡了。飞象踏碎了尸体,令方源省下清理犯罪现场的工作。

他悄悄退去。

在他离去之后不久,另一头白羽飞象也飞离而去。

它的象牙将山壁洞穿,留下两个碗口大小的深洞。山壁凹进去一大块,碎石洒落周围。

忽然,碎石堆一阵颤抖,然后从中冒出一个头颅来。

“我的天,吓死我了!还好我有埋地蛊,逃过了这一劫……”陈鑫从地里钻上来,喘着粗气,冷汗涔涔,后怕不已。

这埋地蛊能令蛊师,潜入地下,暂时躲藏起来。缺点是,一旦使用,蛊师就只能埋在一个地方,不能动弹。并且催发时,需要蛊师持续催动大量的真元。

陈鑫一直被追着跑,直到最后关头,才有了时间使用出来。

“情况越来越混乱了,居然有蛊师暗杀张柱。”看了一眼张柱的尸体,已经被踏成了肉泥,面目全非,陈鑫咽下一口口水,仓惶逃离。

象群足足肆虐了一个多时辰,这才离去。

商心慈和小蝶相互搀扶着,走出雨林。

她们浑身泥泞,狼狈不堪,小蝶脸上更有青紫色,显然是奔逃的时候,撞到了什么地方。

“小姐……”她被吓破了胆,死亡离她如此之近,走路的时候身躯都在颤抖着。

商心慈拍拍她的手,权作安慰。但事实上,她也是脸色煞白。

一路上,尸体遍地,血流于野。破烂的车轮,惨死的驼鸡,还有黑皮肥甲虫、翼蛇的尸体,横在商道上。

随着幸存的人们渐渐聚集到一起,呻吟声、哀嚎声、哭泣声,响成一片。

作为商队首领,贾龙脸色铁青。这一次伤亡太惨重了,整个商队严重减员,十不存一,被彻底打残。

收拢了队伍之后,只剩下一百多人。其中大部分是蛊师,少部分是凡人。

实力最大的贾家、陈家都是伤筋动骨,更遑论其他队伍。林家只剩下三位蛊师,一些不幸的家族队伍,甚至已经死光。

雨林中也有危险,很多人不是死在白羽飞象的踩踏冲撞下,而是受到了雨林中猛兽、毒虫的亲切问候。

“白云,能见到你实在太好了。刚刚在雨林中,谢谢你替我们引走了一头白羽飞象。”在人群中,商心慈发现了白凝冰,主动道谢。

方源并不放心白凝冰,担心她和张柱勾连,因此自己亲自去袭杀张柱。而白凝冰则负责暗中跟随商心慈,保护她的性命。

“这没有什么,我向来有恩必报。张家小姐,救你的不是我,而是你曾经的善行。”白凝冰道。

她平时沉默寡言,几乎不说话。若是说话,也是刻意的改变、压低声音。

此时她不再隐藏,用正常的音调说话,语气冷淡,声音清澈,明显是女音,令商心慈和丫鬟的脸上都微微流露出一抹异色。

“对了,白云,你有没有看到张柱叔?”商心慈问道,神情焦急,“我找遍了,都还未发现他。”

白凝冰心中叹了一口气,方源已经归来,她便知道张柱必定死亡。

“小姐勿忧,张柱是蛊师,身手不凡。说不定正在回来的路上。”她劝慰道。

“但愿如此。”商心慈眉头紧锁,心中的不安越加浓重。

那边,贾龙首领站到高处,喊道:“大家听着,这里的血腥味将很快引来其他的兽群。我们必须尽快撤离。大家动作快点,能带的货物,尽量带上。搬不动的,必须舍弃。一炷香的时间内,我们必须离开这里。”

危险还未有过去,众人只得强振精神,将悲痛按捺在心中,忙碌起来。

“救救我,谁来救救我!我的血还在流……”

“带上我,我只是残了一条腿,我还能走的。”

“求你了,我给你元石作为报酬。两块、三块?四块都行!”

伤势严重,不能动弹的家奴,都发出哀求声。

很少有人得到帮助,这些伤残不能带来劳动力,反而是累赘。许多人都被无情的舍弃。

看着众人远去,很多人都发狂,大声咒骂起来。

有的人在地上爬,想要追赶上商队。

“救救我,张家小姐,你最心慈仁善了!”

“张小姐,求您行行好……”

商心慈脚步迟疑,双唇颤抖,脸上白得毫无血色,目光慌乱游离。

山风吹拂她的绿色裙衫,她发鬓缭乱,柔弱得好像是风雨中的小草。

“张小姐,快走吧。现在不是你发善心的时候。”方源走到她的身边,扶住她的胳膊,强行带着她往前走。

平日里如麻雀般叽叽喳喳的小蝶,此时也闭嘴不言,闷头赶路,腿脚发颤。

“相信我,一切都会好起来的。”方源语气柔和。

商心慈捂住心口,不断地深呼吸。仿佛空气稀薄,令她喘不过气来似的。

刚开始她只是用鼻腔呼吸,渐渐的,她张开嘴巴,大口吞吸着空气。

她脚步越来越虚浮,四肢也越加无力,不是方源扶着她,她恐怕都要瘫倒在地上。

血腥气扑鼻而来,她浑身都被汗水湿透,山风一吹,她结结实实地打了个寒颤。

就在这个寒颤之后,商心慈的呼吸渐渐地平缓下来。

继续走了十多步,她不再大口呼吸。三十步之后,她合上嘴,鼻息也不再浓重。五十多步之后,她的脚步重新显得有力起来,不需要方源再搀扶。

山道陡峭向上延伸,她走到一个坡上,一阵山风吹来,彻底吹乱了她的发鬓。

她伸出手来,一边走一边整理。

当她将发髻重新整理整齐之后,她的目光中的迷茫、惊惧、担忧都消褪了,只剩下坚定之色。

“谢谢。”她对方源道。

方源点点头,松开搀扶她的手。

站在坡上,她缓缓地停下脚步,回望了身后一眼。

“你知道吗?这是我从出生以来,走过的最艰难的一段路。”她幽幽地叹息着,脸色仍白,声音轻柔无比。

方源嘴角泛起一丝笑意,这就是商心慈吗?果然不愧是名动南疆之人。

就连白凝冰都不禁侧目,对商心慈有些刮目相看。

对于一个凡人少女,能在遭逢惨变之后,如此迅速地调整过来,真是不容易。

在这段路上,不断有哀求哭泣的声音传过来,这声音对方白二人没有什么,对商心慈来讲,却是最大的折磨和拷问!

尤其是张柱失踪,最大的依靠消失的情况下,商心慈能勇敢地直面这一切,确是叫人赞叹。

这一段路,是普通的山道,也是艰难焦灼的心路,商心慈咬住牙关,没有倒下,坚强地走过来。

仿佛在一瞬间,她成熟起来。

方源忽然轻笑起来,目光深邃地看向商心慈:“张家小姐,你心地善良,为什么不去救助那些被抛弃的人呢?”

这话换来小蝶的怒目瞪视。

商心慈苦涩一笑:“如果我能救得了他们,必定会出手的。可惜我就算尽全力,也救不了这些人。”

“呵呵呵。”方源朗声一笑,“这就是我最欣赏你的地方了。不理智的善良,都是犯罪。你虽是凡人,却令我敬佩。张家小姐,人生之路,风雨很多,有时候路途泥泞得很,只要做到力所能及这四个字,就可以心安理得了。”

商心慈看向方源,美目中波光一转。

她早就隐隐猜测到,方源和白凝冰并非凡人。刚刚方源的话音,更叫她确认了这一点。

在她的认知中,她在无意中给予了方白二人一些帮助,这些只能是小恩小惠,但却赢得了方白二人的认同和欣赏。

之后,方白二人数次回报自己。先是匪猴山出手相助,然后是帮助自己赚钱,刚刚又是救助自己性命。

自己一个弱女子,被家族变相驱逐,货物也损失了大半,有什么值得他们图谋的呢?

没有!

在这样的情况下,他们却仍旧站在自己身旁。仅仅只是这个动作,就知道他们俩虽然神秘,但品性纯良正直,心中藏着真善美。

碰到他们俩,是自己的幸运。

念及于此,商心慈的心中不由地泛起感动的涟漪,她深深的凝视方源,诚挚地道。

“谢谢。”

虽然只有两个字,却流露出她内心深深的感激之情。

白凝冰忍不住翻了个白眼。

若是让商心慈知道,几乎所有的灾难都是方源一手导演的。不知道她又会是什么态度?

(ps:本章3500字左右,所以有些迟了。)(未完待续。请搜索[.138.看.书.],小说更好更新更快!)

------------

\end{this_body}

