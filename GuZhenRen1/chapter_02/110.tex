\newsection{心慈之志}    %第一百一十节:心慈之志

\begin{this_body}

%1
随着天元宝君莲的秘方,被魏央购走,这场拍卖会也顺利结束。

%2
沮家到底是屹立数百年的家族,深有底蕴。这些沮家的收藏,让参加拍卖的不少人,都有或多或少的收获,也让许多人开了眼界。

%3
“大家难得欢聚一堂,先都别走,让我来做东。”方源挽留下众人。

%4
酒楼中,玉盘珍馐,佳肴美酒。

%5
“方正老弟,这是你要的秘方。”酒过三巡,魏央将一只铭心蛊取出,交给方源。

%6
铭心蛊,形如瓢虫,手指头大小。通体赤红一片,背部浑圆的甲壳上,印有四颗白色的爱心状斑点。

%7
铭心蛊也是存储类的蛊。

%8
和书虫差不多,存储的是信息。

%9
铭心蛊,从一转到五转都有。一转的铭心蛊,背部甲壳上只有一颗白色心状斑纹。二转的有两颗,如此类推。

%10
魏央掏出来的这只铭心蛊,有四颗斑点,这就表明这只铭心蛊高达四转。

%11
不过,要储藏天元宝君莲的秘方,动用四转的铭心蛊也很正常。

%12
方源为了这个秘方,花了六十七万元石。为了避嫌,又让魏央来报价。

%13
看着这只铭心蛊,方源却没接手,而是道:“既然魏大哥已经炼化了它,不如现在就用了,省得我再炼化一次。”

%14
“也好。”魏央点点头,真元灌注过去。

%15
铭心蛊砰的一声轻响,化为一道粉色流光。在魏央的意念操纵下。流光扑到方源的心口,转瞬间消失不见。

%16
立时,方源的心中就涌出一个秘方。

%17
关于如何合炼出天元宝君莲的方子。

%18
合炼前的主料蛊虫。各项辅料,所有的步骤,以及过程中的注意事项,都一应俱全。

%19
这些内容,方源想忘都忘不掉,就像深深的刻在了心头。这就是铭心蛊的效果,记忆深刻。如刻骨铭心。

%20
一股淡淡的喜悦之情在方源的心中泛起:“我手中有天元宝莲,但它只是三转。现在用挺不错,但是当我到达四转。它的辅助效果就立马衰弱下去了。如今有了这份秘方,将来若能合炼出天元宝君莲,无疑能给我带来巨大帮助。”

%21
方源不知道有关天元宝莲的秘方。如今能得到这个方子,乃是意外之喜。

%22
当然。这份秘方没有鉴定过。不过以方源丰富的经验。初步判断,这秘方虚假的可能性很小。要不然,商家也不拿出来拍卖。

%23
“不过将来还是得用些蛊,来推衍一下,防止其中的陷阱或者错漏。”

%24
“我如今买下这道秘方,恐怕其他人都会有些想法。一定会有许多人猜测,我的手上,是否有一株天元宝莲……”

%25
“不过。天元宝莲虽然珍稀,但并不唯一。不像血颅蛊那般烫手。否则。我也不会明目张胆地买下这秘方了。如今我的蛊虫组合,已经渐成,修为不断突破,顾忌越来越少,买下这秘方也没有什么大不了的。”

%26
若换做以前,方源不会这么明目张胆的,来买下这个秘方。

%27
但是现在,他离三转高阶,只差半步之遥,也就是几天内的事情。

%28
再动用白银舍利蛊后,那就是三转巅峰!

%29
他的实力已经今非昔比,又有紫荆令牌护身,再加上众人只是猜测怀疑,因而此事的影响他完全能够承担。

%30
“算算时间,方正来到商家城已经两年有余,实力进步之快,超出意料。天元宝莲……”魏央喝着酒,他虽然有所猜测,但终究没有开口。

%31
“难道黑土哥哥手中有天元宝莲?”商心慈也在暗暗思考,却没有问。

%32
她修行的第一天,商燕飞就告诉她蛊师圈子中的许多忌讳。其中就有一条,不得随意询问其他蛊师拥有的蛊虫。

%33
蛊虫对于蛊师来讲,是立身之本,是隐私,是秘密,是底牌。

%34
蛊虫一旦暴露出去,蛊师就极容易遭到针对。

%35
所以蛊师之间,不得询问彼此的蛊虫,这是一个大忌讳。

%36
“方源有天元宝莲,因此买了秘方。这是可以理解的。但是他放弃了苦力蛊,却买了风气蛊,这是为何?”

%37
坐在一旁的白凝冰没有说话,在心底寻思,方源的举动让她有些看不透。

%38
“唉,若是我能成为少主就好了。商家少主每年都有机会,向家族申请三只蛊虫,家族会无偿地为其收购。”商心慈叹息一声。

%39
这是商家培养少主的一大政策。只要要求不过分,商家都会集家族之力,来全力收购少主们想要的蛊。

%40
商心慈若成为少主,想要一只苦力蛊,借助整个商家的力量,简直是小菜一碟,易如反掌。

%41
方源很早之前,就在寻求苦力蛊。如今好不容易,在拍卖会上等到了,却被商睚眦所阻,失之交臂。商心慈很想帮到方源。

%42
方源拍拍商心慈的肩膀,面带微笑:“无妨无妨,说不得商睚眦会将那只苦力蛊,主动交到我的手上。”

%43
“二哥花了八十一万,买了那只苦力蛊,已经成为笑柄。要让他主动送出,恐怕不成……”一旁的商螭吻摇摇头,觉得方源在异想天开。

%44
“难道黑土哥哥,你已经想到了什么妙计?”商心慈双眼闪亮。

%45
魏央等人,也看过来,一脸好奇之色。

%46
方源手指着白凝冰,似乎胸有成竹:“这一切还要归功于凝冰。”

%47
“我?”白凝冰顿时一愣。

%48
“快说说,是什么奇思妙想?”众人不由地更加好奇。

%49
“两三天后,此事便见分晓。容我先卖个关子。”方源打个哈哈。

%50
他又看向商心慈,面容一肃:“心慈。你真的想成为少主吗?要成为少主,就是落入政治漩涡,从此身不由己。商家的情况。你现在一定比我更清楚。商家的少主之间,竞争激烈,为了一个位置,争夺得头破血流。你要成为少主,就有被倾轧的危险啊。”

%51
此事是关键中的关键,方源需要问个清楚。

%52
若商心慈没有此心志,就是个扶不起来的阿斗。

%53
方源炯炯的目光注视下。商心慈浅浅一笑:“黑土哥哥,就算是现在的我,难道能得自由吗?同样也是身不由己啊。”

%54
“加入商队。来到商家城的这一路上,可谓是艰难险阻,险死还生。我早就静心反思,世间如海。我们就像是一艘艘的小舟。小舟随波逐流。看似逍遥,却有很多的无奈和痛楚。只有修为越高,势力越大,小舟变成大船,才能抗衡风雨,给自己在乎的人提供避风的港湾。”

%55
商心慈的话,平平淡淡,一点也不慷慨激昂。但众人却听出此中的一股豪情。

%56
“好。有志气。”魏央笑了一声。

%57
商螭吻亦投去惊异的目光。

%58
她和商心慈相处时间并不短,却还未瞧出后者温柔的模样下。暗藏有这样的雄心壮志。

%59
“我家小姐乃是行商的奇才,当一个商家少主,绰绰有余。”小蝶站在商心慈的背后,一脸骄傲的插嘴道。

%60
“小蝶……”商心慈面色一窘,略带嗔意,看了小蝶一眼。

%61
小蝶吐了吐舌头。

%62
“哈哈,说的好。既然如此,那我们两人就助你一臂之力,成人之美,助心慈你成为商家少主。”方源哈哈大笑,放下心来。

%63
商心慈能有此志,也不奇怪。

%64
穷人的孩子早当家,商心慈童年并不幸福,饱受家族欺凌。商队一行,更教她清楚地认识到世界的残酷,自身的渺小柔弱。再温柔的人,受到这样的刺激,也会奋发图强。

%65
但商心慈心性善良,和方白二人不同。她想要变强,除了为了自己,更多的是想给周围的人幸福。

%66
“呵呵呵,今年的少主考核已经过去。要等到来年,心慈妹妹才有机会了。不过,竞争少主之位,十分激烈。父亲大人的子女众多,每年只有一个少主的位置,但却有数百的竞争者。”商螭吻微笑着,主动为商心慈出谋划策。

%67
但是她的心中,却有些不以为然。

%68
商心慈修为低微,如今只有一转高阶。资质也不行,连乙等都不到,没有发展的潜力。

%69
她的母族是张家,张家向来和商家积怨深厚,这更是她的政治大劣势。

%70
她孤身一人,势单力薄,谁会去支持她?

%71
唯一的优势,在于商燕飞的宠爱。商燕飞为了她,耗费巨大代价,几乎是逆天改命一般,将毫无修行希望的商心慈,打造成一位蛊师。这是其他的子女,都没有的待遇。

%72
但这优势,从某种方面来讲,也是劣势。

%73
商心慈被孤立了,就算是商螭吻的心中也暗藏着对她的羡慕嫉妒。

%74
种种原由,商心慈要成为商家少主,真的是极其艰难,希望渺茫。

%75
商心慈的这些劣势,方源自然也心知肚明。在他前世,足足六年之后,商心慈才成了少主。不过此一时彼一时,前世商心慈势单力孤,今生却多了方源的臂助……

%76
“心慈放心,有我等助你,不需明年,今年就可让你登上少主之位!”方源哈哈大笑,一副运筹帷幄的架势。

%77
“那我等就拭目以待啦。”商螭吻表面笑着附和,暗下撇嘴,觉得方源越说越不靠谱,胡吹大气。

%78
魏央放下手中酒杯:“方正老弟,事关商家少主之争,非同小可。我身为家老重臣,却不能掺和其中的。”

%79
“不需魏大哥相助。此计早在两年多前,就已经埋设下来。一切多亏了有凝冰。”方源笑得道。

%80
“哦?”

%81
一时间,桌上众人都将目光集中到白凝冰的身上。

%82
作为视线的焦点,白凝冰依旧是一脸冷漠,但心中却疑惑重生,不由暗中腹诽。

%83
“这关我什么事?”

%84
“凝冰,凝冰,叫得肉麻死了。真以为我们关系多好似的!”

\end{this_body}

