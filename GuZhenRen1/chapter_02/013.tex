\newsection{猴脑美食}    %第十三节:猴脑美食

\begin{this_body}

哗哗哗……

空窍中,海潮起伏不断,持续地冲刷着周围窍壁。(

..)

海潮一片翠绿,浪花翻腾不休。这是一转初阶的真元。

一转蛊师拥有青铜真元,初阶为翠绿色,中阶为苍绿色,高阶为深绿,巅峰为墨绿。

周围的窍壁,明光灿烂,在翠绿浪潮的冲刷下,底蕴一丝丝的增加着。

蓦地,白色的光辉骤然一盛,旋即晦暗下来。白色的光芒相互凝练,形成如水般的光辉波纹。

这一刻,窍壁光膜深化成了水膜。

“成了。终于从一转初阶,晋升到了中阶。”一股淡淡的喜悦,随即在方源的心中升腾起来。

初阶蛊师的空窍四壁为光膜,到了中阶则为水膜。

水膜一成,意味着方源修为回到了一转中阶。

“距离斩杀那魔道女蛊师,已经过去了二十多天。这期间,我x夜修炼不辍,没有浪费一丝一毫的时间。修为晋升中阶,也不奇怪。”方源缓缓地睁开双眼。

“可惜,若是有一只酒虫在身上,速度会更快。”

酒虫能令一转蛊师的真元,提升一个小境界。用酒虫来辅助修行,可以达到事半功倍的效果。

方源重生一来,除去春秋蝉不算,第一个得到的就是酒虫。

然而后来,他在商队中又收购了一只,将两只合炼成了如今手中的四味酒虫。

可惜的是,四味酒虫只对二转蛊师有效,对于现在的方源,没有任何帮助。

事实上,方源在青茅山上一共收获了三只酒虫。第三只,是杀了古月药乐缴获的战利品。

这酒虫一直保管在方源手中,直到铁家父女突来青茅山。他为了防止暴露,忌惮铁血冷的侦破手段,便主动将酒虫和其他一些战利品都摧毁了。

事情已经发生,究竟此举是对是错,方源既然做了,就没有后悔过。

虽然现在,他若是保留酒虫在手,必然对自己有帮助。但至少,整个过程中,真的没有让铁家父女查出古月药乐这事情来。

“如果将四味酒虫逆炼,就能得到一转的酒虫。可惜逆炼的条件,要求很多,我现在还达不到。”

蛊师养蛊、用蛊、炼蛊。其中炼蛊,大致分为合炼,逆炼两大方面。

合炼,是将低转的蛊虫晋升为高转。逆炼却恰恰相反。

要逆炼四味酒虫,方法也有很多。可以用浪子蛊,这样一来就能得到两只一转酒虫。或者用幡然蛊,这个法子只能得到一只酒虫。用化蝶蛊的话,则会得到幽气蝶。用莫测蛊的话,得到的蛊虫就不定了,甚至有可能会是一种全新的蛊虫。

对于方源来讲,最好的方案,当然是得到浪子蛊。

但浪子蛊比酒虫还有珍贵得多,入手难度太大。最实际的方案,还是幡然蛊。

用了幡然蛊,虽然会损失一条酒虫。但方源还是会毫不犹豫的选择去做。

保留四味酒虫,等到二转修为再用,那是在安定的情况之下。现在方源朝不保夕,颠沛流离,一切以生存为首要前提。

而修为提升越快,自然生存的几率越高。

吱吱吱……

山洞外,传来猴子的叫声,以及一阵脚步声。

方源耳朵动了动,这脚步声他最熟悉不过。

果然没有多久,白凝冰就出现在方源的面前。

他浑身上下,都披了一层荆棘衣衫。这件特殊的衣衫,完全是碧绿的藤蔓和黑色尖锐的木刺构成,带给白凝冰防御效果的同时,也能起到攻击作用。

对那些徒手肉搏者,这些堪比钢铁般坚硬的木刺,将会给他们留下深刻的印象。

这层荆棘衣衫,正是铁刺荆棘蛊所化。

白凝冰杀了那魔道女蛊师后,就看中了这只三转草蛊。将其索要去,炼成了他的本命蛊。

“真是麻烦。每次进出山洞,都要如此小心翼翼。那个什么焦雷土豆蛊,你就不能埋少点吗?”他一边嘴中抱怨着,一边盯着脚下,时而跳跃,时而跨步,谨慎地避开一些地点。

“有备无患,防患未然。”方源淡淡地答了一句,然后目光转向白凝冰的腰间。

在她的腰间,挂着一串猴子。

这些猴子,体型袖珍,最大的超不过手臂长短。它们有圆溜溜的小脑袋,浑身黑色的猴毛细致而又浓密,光滑泛着幽光。最令人瞩目的,是它们腰间,都生长着树叶。

这些绿色的树叶,围成一圈,连成一体,遮住猴子的胯骨和屁股。活脱脱的是一件草裙。

这些猴子,自然就是草裙猴了。

草裙猴味道鲜美,这些天来,白凝冰和方源都是以此为食。

这几只被捕捉过来的草裙猴,也意识到了接下来的结局。极力挣扎,却无济于事。

整个山洞中,都回荡着它们凄厉,惊惶的叫声。

算算时间,也是到了午餐的时候了。

“烤了吃吧,我有点饿了。”白凝冰摸摸自己的肚皮,将用麻绳串起来的草裙猴扔在地上。

“我先处理一下,你来烧烤。你烤出来的肉,比我好吃多了。”她蹲下身来,正要动手杀猴。

方源忽然阻止道:“今天教你一种新吃法。”

当即,他手指一勾,从白凝冰的空窍中就飞出一道血光。

血光落在方源的手中,化成一片血月印记。

正是血月蛊。

方源的手掌中,开始泛出微亮的红芒。

血月蛊虽然是三转蛊虫,但方源此举却不是用来攻击,因此勉强能操纵起来。

白凝冰不知道方源打算,蓝色双眸一眨不眨地盯着。

就看到方源忽然伸手,将其中一只草裙猴提起来。他牢牢抓住草裙猴的脖子,使得它脑袋动弹不得。

然后,将泛着红芒的手掌边,贴着猴子脑壳,缓缓绕了一圈。

昔日,他用月谷蛊解石。如今则用血月蛊切猴头。

手掌边沿绕了一圈之后,方源空窍中的真元,就彻底见底了。

手掌中红芒消失,方源嘴角含笑,轻轻捏住猴子脑袋,微微一提。顿时就将半边脑壳,揭了开来。

完整的白色猴脑,裸露在空气中。

草裙猴疯狂挣扎,痛得撕心裂肺。尖叫声锐利刺耳,在山洞中回荡不尽。

但方源有两猪之力,手如铁钳,丝毫不动。草裙猴的手脚,都无爪而且细嫩。抓挠在方源手臂上,根本伤不了他一根汗毛。

“尝一尝这新鲜的猴脑罢。”说着,方源取出汤勺,递给白凝冰一个。

“就这样生吃吗?”白凝冰看着白花花的,夹杂着些许血丝的猴脑,有些迟疑。

方源嘴角含笑,当先示范。

他用汤勺切入猴脑,挖下一块,放入嘴中,喉结滚动,就吞咽了下去。

在刚刚的一瞬间,草裙猴的叫声,陡然拔高了八度。浑身如触电般剧烈颤抖,四肢狂舞,癫痫了一般。

白凝冰试着,也挖下一勺猴脑,放入嘴中。

顿时,一股香草气息弥漫在舌尖,丝毫的血腥气味都没有。宛若豆腐般的细嫩口感,轻轻一抿,整个猴脑就融化在口腔中,丝丝甜意,透着清凉,满口生津。

她喉结滚动,咽下这口猴脑。幽蓝的双眼顿时绽放出亮光,白皙剔透,清丽如仙的脸庞也露出一丝陶醉的神情。

“竟然这般好吃!”少女发出惊叹。

“也只有这种草裙猴,才有这般的美味。它们的猴脑,是天然的美食。”方源说着,又挖下一勺,放入嘴中。

他晋升到了中阶,也算是一桩小小的喜事。此时吃下猴脑,更觉得尽兴。

白凝冰朗声一笑,动作比之前快了一倍,吞下第二口猴脑。

而这只被开颅的草裙猴,在经历了短暂的疯狂之后,已经奄奄一息。明丽如宝石的眼珠子,也彻底晦暗下来。

其他的草裙猴,看到同伴如此的惨状,都发了疯似的挣扎,想要逃窜出去。

然而在麻绳的束缚下,它们的挣扎终究是无用的,最终的效果,只能在地上胡乱翻滚。

吃这猴脑,有点类似地球上的甜筒,或者冰淇淋。二人合力,三下五除二,就将猴脑消灭掉。

“再来,再来。”白凝冰意犹未尽,舔了舔嘴唇,挥着汤勺叫道。

方源此刻的真元也回复到九成,他故技重施,又切开一个猴脑。

一时间,美食入口,二人吃的不亦乐乎。而山洞中,则回荡着草裙猴痛苦而又尖锐的叫喊声,哀嚎声。

吃了猴脑,接着烤猴肉。

草裙猴的猴肉,肉质鲜美,尤其是胸脯的那一块。整个猴肉的味道,有点类似在甘蔗片上的烤鱼。

饭袋草。

方源心念一动,从空窍中射出一点绿芒。

绿芒落在地上,顿时生长出绿芽。

在几个呼吸的时间内,绿芽生长成绿叶,绿叶渐渐长大,延展成藤蔓。藤蔓蜿蜒,覆盖了十米范围后,抽出几个枝头,长出硕大的叶片出来。

这些叶片,饱满似桶。起先,只是一个手指头大小,然后渐渐膨胀,涨大成拳头般,最后成碗般大小,才算成熟。

白凝冰轻车熟路,首先摘过来一片。

解开叶片顶部,里面是一捧米饭。颗颗饭粒饱满圆润,洁白无瑕。

白凝冰仰头开口,倾倒了一嘴米饭。咀嚼吞咽后,又吃猴肉搭配口味。

“饱腹就是一种幸福啊。”她发出一声感叹。

“那些草裙猴的情况怎样?”方源则问。

白凝冰目光一凝:“不深入那片树林,没有关系。但一旦深入,那三只猴王就出现了。你要夺的酒,就在树林最中央,恐怕搞不到手了。”(未完待续。如果您喜欢这部作品,欢迎您来文学注册会员推荐该作品,您的支持,就是我最大的动力。)

\end{this_body}

