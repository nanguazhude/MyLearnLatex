\newsection{紫荆令牌}    %第七十二节:紫荆令牌

\begin{this_body}

%1
商燕飞见方源拒绝了元石,便说出另一个方案:“你二人如今受到百家的通缉,这显然是一个误会。就由我出面解释一番,将这通缉令消掉,你们看如何?”

%2
方白二人受到通缉,商心慈也是清楚的。

%3
“我怎么就没有想到,给黑土哥哥消掉通缉令,这无疑是他最需要的。”商心慈心底十分认可这个方案。

%4
商家乃是南疆霸主,而百家不过是一个普通的家族山寨,并且元泉渐渐干涸,外强中干。商燕飞的“解释”,百家自然会听取的。

%5
但方源却摇摇头。

%6
这个方案看似动人,他却心中别有谋算。

%7
若是让商家出面,无疑就给自己打上了商家的标签。对他将来接近魔道中人,靠拢义天山的大事是巨大阻碍。短期有益,长远有害。

%8
这通缉令看似很麻烦,其实对方源的危害不大。

%9
哪个魔头身上没有十几张,几十张的通缉令?方源前世的通缉令足有上百张!

%10
但通缉令再多,又能如何?

%11
讽刺的是,当他前世创建了血翼魔教,称霸一方之后,很多家族都主动撤销了对他的通缉。

%12
这世界本质如此,终究一切还得是实力说了算!

%13
方源的计划中,本就打算在商家城待上个两三年的时间,这期间百家不可能在这里抓捕他们。

%14
两三年之后,方源将蛊虫凑足一套。实力大增,更加不怕百家的通缉。

%15
其实百家本身的处境都岌岌可危。自顾不暇,哪里能顾得上方白二人!

%16
所以商燕飞的这个答谢方案,在商心慈看来,可能最需要。但在方源心中,却最是没有价值。

%17
于是他直接摇头拒绝:“我们俩和百家的冲突,源自一道传承。说心底话,我们抢了传承,杀了百家两个少主。根本从未后悔过。我身上的伤,就是出自百家之手。迟早有一天,我要找百家算账。我这个人,有恩报恩,有仇报仇。滴水之恩涌泉相报,星火之仇燎原往复!”

%18
说到这里,方源毫不掩饰狰狞的杀意。

%19
一时。惹得院中少主人人侧目,心生诧异之情。

%20
“这是十足的魔道气性啊……”有人反感。

%21
“一个人要对一个家族复仇,简直不知天高地厚。”有人不屑。

%22
“呵呵呵,居然在父亲大人面前,如此坦言。这人是傻子,还是胆气十足?”有人感到有趣。

%23
商家乃是正道霸主。但方源在商燕飞面前,居然坦言要对付百家。

%24
他气焰嚣张,令商心慈暗暗心惊,却并不意外。这才是黑土哥哥的本色,不是吗?

%25
商燕飞也丝毫不以为杵。在他看来,方源很坦荡直率。魏央的评价。果真不差。这种人能一眼看到底,比较一旁沉默不语的白凝冰,他无疑更喜欢方源。

%26
“无须族人大人出面,为我二人消除通缉令。我正需要这通缉令不断勉励自己,鞭笞自己,不断地变强。感谢燕飞大人的好意。”方源抱拳道。

%27
“既是如此,二位想要什么谢礼,尽管说来。不要再提什么恩情两清的话,我商家要答谢,就必定会答谢,这是我商燕飞的行事规矩。哪怕你们过后将这些谢礼扔掉,那就不关我的事情了。”商燕飞皱起眉头,故作不悦之色。

%28
顿时,庭院中的轻松氛围,就转为丝丝的凝重。

%29
一众少主举杯喝酒的动作,都不由地变得小心翼翼。

%30
这就是五转蛊师,商家族长商燕飞的霸道。

%31
我要感谢你,你就算不愿意,也得受着!不要也得要!

%32
方源目光扫视一圈,哈哈大笑:“如果我真的不要呢?”

%33
商燕飞淡然一语,目光坚决:“那也由不得你。”

%34
众少主看到两人杠上,一边暗骂方源是个夯货,脑袋里缺根筋,商家族长的感谢都往外推,外面多少人想求都求不到呢。另一边倒也不禁佩服他的勇气。

%35
商心慈手心都攥出汗来,为方源担忧。

%36
魏央哈哈一笑,打圆场道:“我想到了,方正兄弟受了伤,容貌尽毁。族长大人何不遣来医师,为他回复原貌呢?”

%37
“嗯,这个想法不错。魏央你便去把素手医师唤来。”商燕飞点点头道。

%38
方源沉默不语,魏央化光而去。

%39
不多时,魏央回来禀告:“素手医师已被请入屋内,请方正兄弟移步。”

%40
方源对自身容貌从来就不大在意,不过此时倒不可再逞强。

%41
他看向一旁坐着的白凝冰:“你也一起来吧,顺便检查身体一下也好。”

%42
白凝冰正欲嗤之以鼻,自己有什么伤病自己不清楚?不过转念一想,顿知方源似乎别有用意,便答应下来。

%43
二人暂时离开小院,来到屋内。

%44
这屋子他们也很熟悉。当初,他们就在这里,等候商燕飞足足三个时辰,也没有等到。

%45
素手医师是位身材曼妙的女子,面罩白纱,身着白衣长裙,正坐在位置上品茶。

%46
她和商燕飞关系复杂,恩怨情仇都有,在商家城中位置很特殊,乃是五转的治疗蛊师。

%47
“难得我今天心情好。”她把手中杯盏轻轻放下,看向方源,“是你想要疗伤?先去沐浴去。”

%48
又伸出芊芊玉指,指向白凝冰:“尤其是你,小姑娘,你面目上涂的什么东西,又脏又丑,都给我洗干净了再来。”

%49
白凝冰一路上遮掩容貌惯了,到了商家城也依旧如此。她脸上涂着一大片丹砂,充作胎记。又抹上特殊的黑油,头发遮住眼眉,弄得其貌不扬。

%50
听了这话。白凝冰不禁诧异。

%51
方源也流露出诧异的神色。

%52
魏央连忙解释:“这是素手医师大人的规矩。但凡看病之人,都要在此之前沐浴一番。将身体清洗干净,使用香精,换上白袍,否则大人是不医治的。不过二位放心,我已经准备妥当,热水已经布下,就在里屋,二位请吧。”

%53
二人又进了里屋。果然里面已经布置了两个大木桶。

%54
每个木桶旁,都各有两位凡人婢女站着,显然是要服侍二人洗浴。

%55
白凝冰顿时皱起眉头,不悦地道:“你们都出去,我自己会洗。”

%56
“这个……”魏央语气迟疑,这四人乃是素手医师的手下。素手医师生平好洁净,若是赶走婢女。她却唯恐你自己敷衍了事,洗不干净,一般是不医治的。

%57
“我这两个就不用走了,魏大哥你先出去吧。她不懂享受就算了。”方源笑道。

%58
魏央把这顾虑说了,但白凝冰执意如此。魏央也不坚持,退出里屋。关上房门,反正治疗的主角是方源。

%59
方源三下五除二,褪去衣衫,进入木桶。

%60
木桶中水温刚刚好,两位婢女一位洒下香精。一位为方源搓身,动作娴熟。显是老手。

%61
白凝冰站在木桶前,却没动弹,一阵迟疑。

%62
方源惬意地躺在木桶里,双臂搁在木桶边缘,轻笑一声道:“白凝冰,你我二人身份已经暴露,到了商家城也无需遮遮掩掩,难道你不敢以真面目见人不成?”

%63
白凝冰立即冷哼一声。

%64
方源继续道:“我叫你一起来,我可是为了你好。这素手医师我早就听闻过她的大名,在南疆,她和九指游医、圣手医师、杀人医,并称为四大医师。你待会可以问问她,关于阴阳转身蛊的事情。”

%65
阴阳转身蛊!

%66
白凝冰双眼顿时眯成一条缝,眼中精芒闪烁不定。

%67
这可是白凝冰心中最大的痛,是方源钳制白凝冰的主要手段。方源却将其堂而皇之地说出来,是要做什么?他有什么险恶的阴谋?他到底是什么目的?

%68
白凝冰的脑海中,顿时各种疑问纷至沓来。

%69
商心慈带给她的心神上的冲击,还残留着许多,以至于白凝冰都有些风声鹤唳。

%70
方源这个家伙,简直是深不可测!

%71
就连商心慈都不知道自己的身份,他却明白,否则不会一路故意接近。

%72
他怎么做到这点的?

%73
白凝冰猜不到春秋蝉,那是六转蛊,对她来讲太遥远了。但她想到了另一个答案,那就是预知能力的蛊。

%74
“方源必定有预知类的蛊,能看到未来的景象。原先以为他熟悉白骨山,是因为借鉴了先人的经验,现在看来也应该是此蛊的作用。只是不知道他掌握的预知蛊,到底是哪一种,到底是哪一转?”

%75
毫无疑问,白凝冰此时的压力是巨大的。

%76
尽管她明白,但凡预知类的蛊,都有各自的严重弊端。有时候甚至预知的结果,都是混乱的,错误的。

%77
但是当她现在想要对付方源时,她都会不由自主地去想——我这个办法,会不会被方源预知?我用这个方法对付他,会不会被他将计就计?

%78
能够预知未来的敌人,真的是太可怕了。

%79
里屋中,热气渐渐弥漫。

%80
白凝冰站在原地,却感到手脚发凉。

%81
模糊中,她依稀看到方源躺在木桶里,身旁婢女为其洗浴。

%82
她感觉到,方源正看着她,用那双幽黑如墨的双眼,波澜不惊宛若深潭般的双眼,静静的盯着她。

%83
她仿佛听到方源在她心中的低语:你要怎么做呢?白凝冰!没有错,这就是我的底牌,预知类的蛊。你还想对付我吗?那就来吧!我已经看到了未来,你没有胜算的……

%84
但事实上,方源早已经在闭目养神。

%85
白凝冰洗不洗,都是他一石二鸟的试探。既试探白凝冰,又试探商燕飞。

%86
两位婢女的手法很老道娴熟,水稍微凉了,就会立即添加。

%87
这里屋并不大,摆上两个木桶,已显得拥挤。可见当上商燕飞放弃少主之位,沦为普通人的窘困。

%88
但这亦数正常。

%89
英雄常遇末路。这话并非是说。英雄的命运都是多舛多灾的。而是末路、困境才能造就出英雄。

%90
商燕飞算得上是一位英雄,但他更是奸雄、枭雄。

%91
足足洗了半个时辰。婢女这才停下动作。

%92
方源穿上准备好的白衫,走出房门时,白凝冰仍旧还站在原地,蓝眸中思绪不断闪动。

%93
“你们都出去,我一个人洗。”方源出去后,白凝冰将两位婢女尽数赶出里屋。

%94
方源微微一笑,白凝冰思考的越多,她就越有压力。越是思考。她心中的斗志就越遭到削弱。

%95
有时候,优点并不一定一直都是优点。

%96
白凝冰若是直肠子的莽汉,也就罢了。偏偏她冰雪聪明,她越聪明,想的就越多,想的越多就会觉得方源越深不可测,难以战胜。

%97
白凝冰可能觉得沐浴没有什么问题。但方源却从这件微不足道的小事中,看到她对自己低头。

%98
聪明人总是多疑的,从某种程度上来讲,是白凝冰帮助方源压倒她自己。

%99
回到正屋,方源见到素手医师。

%100
素手医师并不和他废话,伸出手掌。搭在方源的肩头。

%101
一股洁白的水光,顿时泛起涌动,笼罩住方源浑身上下。

%102
清新凉爽之意,旋即袭遍方源全身。

%103
原本烧伤的皮肤,以肉眼可见的速度。开始康复生长。同时方源的脑袋右侧,开始生出肉芽。

%104
随着时间的流逝。肉芽茁壮成长,渐渐形成耳朵形状,软骨也渐渐形成。

%105
方源咬牙忍住,酥麻酸痒之感,形成潮水,不断地挑战着他的忍耐极限。

%106
很快,他长出新嫩泛红的皮肤,身上烧毁的汗毛、眉毛也开始长出来。

%107
半盏茶的功夫后,他伤势痊愈,不仅完全恢复了容貌,而且右耳也长了出来,和左耳对称一致。

%108
素手医师收回手,淡淡地评价了一句:“现在看起来就顺眼多了。你走吧,顺便把你那同伴也叫走。哼,她把我的婢女赶走,我怎么知道她自己洗没洗干净?咦……”

%109
就在此时,房门打开,白凝冰从里屋走了出来。

%110
她一身白袍,洗尽伪装,露出真容,瞳眸幽蓝如碧空,冰肌玉骨,脸颊微微泛红,带着刚刚沐浴而出的湿润之气。就算是身为女子的素手医师,也不由地被白凝冰的绝色打动。

%111
素手医师顿时对白凝冰的印象,大为改观,柔声道:“妹妹好姿容,叫姐姐也差点看呆了眼。”

%112
这态度简直发生了一百八十度的大转弯,方源忍不住翻了个白眼。

%113
不过也知道,这素手医师性情如此,一生都在追求美好的事物。用地球上的话讲,就是颜控!

%114
白凝冰摇摇头:“我不需要治疗,只想问问有关阴阳转身蛊的事情。”

%115
“妹妹想问什么,姐姐一定知无不尽。”素手医师温和地道,然后冷眼看向方源,“至于你,还待在这里干什么,走走走!”

%116
对待白凝冰和方源,完全是两个态度。

%117
方源摸摸鼻子,被赶出正屋。

%118
刚出了门,就看到魏央。

%119
“方正老弟?”魏央迟疑地看了他一眼。

%120
方源点点头,眼中流露出一丝感激之色:“劳烦魏大哥一直守在这里。”

%121
“哈哈,想不到方正老弟你是一表人才啊!”魏央竖起大拇指,夸赞道。

%122
其实,方源面貌普通,充其量只能算个中上。但他双目幽幽,有一股坚定不疑的气质。

%123
最关键的,还是他受伤时相貌太丑陋,两相对比一下,就觉得方源“英俊”了。

%124
但魏央很快又浮起苦笑:“老弟啊,你既然叫我一声大哥,那大哥就得说说你。商家的谢礼为什么不要?我知道你有你的原则,但商家也有商家的规矩。都说入乡随俗,况且族长大人也不是要对你不利!这是好事啊。”

%125
“但如果你还要坚持,恐怕好事就变成坏事了。向来识时务者为俊杰,你也不想心慈小姐夹在中间两头为难吧?”

%126
方源皱起眉头:“我也正是顾虑到这一点,所以接受了治疗。”

%127
魏央笑容更加苦涩:“就这份治疗。当不得谢礼。商家若不拿出像样的谢礼,就会引起外人的嘲讽。破坏商家的声誉和形象。日后,万一有商家少主遇难,谁会来救呢?所以说,这份谢礼你必须得接受。”

%128
魏央一边说着,一边察言观色,见方源眉头皱得更紧,再次劝道:“你啊你,我都不知道说什么好了。别人向往无比的事情。你却往外推。方正老弟,胳膊扭不过大腿。你若是真不想要,你可以先接受了,让这场面混过去,再找机会还给商心慈。同时,也这不很好吗?”

%129
方源思索片刻,沉吟道:“嗯。这个法子好。一来没有违反我做人的原则,二来也不会让魏大哥你难做。但魏大哥,你说我要什么好呢?”

%130
魏央脱口而出道:“当然是令牌!”

%131
方源心中暗笑,这正是他想要的。原本是想借商燕飞之口说出来,但商燕飞似乎有意招揽自己,故意不说。

%132
现在。就借魏央之口好了。

%133
“令牌?”方源流露出一丝疑惑之色。

%134
“你刚来商家城,虽然知道需要令牌,但还不清楚令牌的重要性。相信魏大哥,有一枚高等令牌真的很重要。有时候你再有钱,没有令牌也无用。”魏央苦口婆心地劝道。

%135
方源点点头:“虽然不太明白。但既然魏大哥都这么说了,那就令牌好了。”

%136
魏央顿时生出一股被信任的感动。

%137
他拍拍方源的肩膀。叹道:“老弟啊,你我一见如故,魏大哥怎么会让你吃亏呢?”

%138
回到院中,方源当即就对商燕飞拱手:“小子莽撞了,刚刚听魏央大哥的指点,原来高等的令牌这么重要。小子想要向族长大人讨要两枚令牌。”

%139
商燕飞眼中光芒一闪,他故意不提令牌,自有打算。没想到反被魏央乱了计划,这个魏央啊,是好心办了坏事,没有再想深一层。

%140
这两人救了我的亲生女儿,这令牌绝不能给低了。

%141
也罢!

%142
在商家城,他们那点元石算什么?不出一两年,就会用光。就算有令牌,还得渐渐地来依附我商家。

%143
念及于此,商燕飞朗声一笑,大手一挥:“准了。就给你们二位一人一枚紫荆令,当做谢礼吧。”

%144
此言一出,不少少主同时倒抽一口冷气。

%145
魏央也不由动容。

%146
商家城有黑白赤橙黄绿青蓝紫九等令牌,黑石令牌最低,紫荆令牌最高,代表着商家贵宾,权限几乎等于半个家老!

%147
就算是方源也未料到,商燕飞竟然给了紫荆令牌。他原先估计着,大约是绿青蓝三色。倒是有些低估了商燕飞的气量。

%148
商燕飞当场掏出一枚令牌,紫荆木所制,巴掌大小,正面是商家二字,反面是商量山全景缩影图。

%149
但这却非真正的紫荆令牌。

%150
商燕飞又唤出一只蛊来:“此乃商家特制的令牌蛊,要取你一点鲜血。”

%151
蛊虫如蚊子一般形状,飞到方源的手臂上,叮了一口,又飞回到令牌表面。

%152
啪。

%153
一声脆响,令牌蛊忽然自爆,化为一团血光,汇入到令牌当中。

%154
令牌似乎毫无变化,但是当它转交到方源的手中时,令牌表面开始流转紫色光辉,如水如影。

%155
至此,这才是真正的紫荆令牌。

%156
旁人拿了这紫荆令牌,也没有作用。唯有方源亲手执掌,才能使用。

%157
这也是商家的保密手段。

%158
旁人万难模仿。若是以蛊虫当过信物,别的蛊师兴许还能反推秘方,研炼出来。

%159
事情妙就妙在,紫荆令牌并非蛊虫。只是其中留有令牌蛊的力量。

%160
随着时间的流逝,令牌蛊的力量会越来越小,紫荆令牌也会失效。

%161
这对商家来说,又是个优点。

%162
令牌的消耗,会使令牌的数量维持在一个稳定的范围。

%163
方源的紫荆令牌若是失效了,就得重新回到商家城,请他们重新制作一个。这也是商家对令牌的一种控制。

%164
方源得了紫荆令牌,一众少主看向方源的目光,顿时发生了转变。

%165
先前因为是魔道蛊师,还多多少少有些轻视、不屑,现在完全是平等的态度。

%166
紫荆令牌只有族长,或者十位家老合议,才能发放。目前流出在外的紫荆令牌,也不过两百多枚罢了。

%167
“你那同伴呢?”商燕飞又掏出一枚令牌。

%168
“还在素手医师那里,属下这就去催。”魏央正欲起身,正巧白凝冰回来,出现在众人的面前。

%169
她面色冷酷,神情凝重。皆因刚刚从素手医师口中得到消息,要想转回男身,非得需要对应的阳蛊不可。

%170
当然,事无绝对,若是能请动一位六转蛊仙出手,也大有成功可能。

%171
看到她的容颜,众少主都不由地呆了一呆。

%172
(ps:这章难写至极,求鼓励……)(未完待续。如果您喜欢这部作品,欢迎您来起点投推荐票、月票,您的支持,就是我最大的动力。)

\end{this_body}

