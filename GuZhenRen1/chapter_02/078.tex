\newsection{谨慎和忍耐}    %第七十八节:谨慎和忍耐

\begin{this_body}

紫荆令牌在方源的手中,散发着流光溢彩,明晦间如梦如幻。

伙计紧紧地盯着这面令牌,神情呆滞。

半晌后,他忽然惊醒,用恭敬无比声音道:“尊敬的贵客,恕小人有眼无珠!请您稍等片刻,小人这就去唤掌柜的来。”

掌柜的是个中年白胖子,听到紫荆令牌出现,立即屁颠屁颠地跑过来。

见到方源后,他立即深深一礼:“贵客,您的到来,实在是令小店蓬荜生辉!”

方源语气淡淡,指着他特意挑选出来的石头:“我赌了一些小石头,劳烦你们解石。”

掌柜的连忙一看,脸色微微一愣,怎么都是下等顽石?

他不由地迅速扫视伙计,目光仿佛在责备:拥有紫荆令牌的,都是大人物,大顾客,你居然叫他选这些蛊?

伙计局促不安地站在一旁,有口难言。

掌柜向方源垂首:“贵客您要解石,毫无问题。紫荆令牌的拥有着,可在各大赌石坊免费解石。至于这些石头打过折扣,只有……六百五十块元石。”

“我知道,赌石坊的规矩,先付款后解石。”方源点点头,从元老蛊中取出相应数目的元石。

掌柜的麻溜地接过元石,立即转身叫那伙计:“你速去,将段大师,黄师傅,张师傅,赵师傅,马师傅都请出来。”

又转身对方源道:“贵客,解石台在坊内,请您稍移尊步。”

解石的五位老师傅,都在赌石坊后面的小庭院里休息着。到了他们这种层次,寻常的顽石已经不会让他们出手了。而一些年轻的蛊师,大多是他们的学徒。则在台前负责正常解石。

伙计一路飞跑,来到小院,说明来意。

“哦,请我们四个老家伙一起出去?”五位老师傅都亮起了双眼。

“难道有人买掉了那几块特等的顽石?”老师傅们顿时感到手痒了。

伙计摇摇头:“只是些下等顽石。”

老师傅们脸色顿时一变,流露出不快的神色。

段大师一声冷哼。

叫他们去解下等顽石,简直是一种侮辱,对他们身份的轻贱。

但伙计紧接着又道:“来人是大贵客,拥有紫荆令牌。掌柜的特意叫我来请五位的!”

“什么,紫荆令牌?”

“你确信没有看错?”

“商家数千年来。外发的紫荆令牌也不过数百枚罢了。居然出现了紫荆令牌的拥有者?”

“快快快,你们还愣着干嘛?”

五位老师傅忙不迭地跑出小庭院,走上工作台。

工作台上,年轻的解石师们都吓了一小跳,连忙行礼问好。

老师傅们纷纷挥手。将他们这些后生晚辈赶下台去。

这番动静,很快就引起旁人注意。许多道目光,都投射过来。

人们纷纷私语,不由地升起好奇探寻之心——

“这是怎么回事?”

“五位当家的老师傅,一齐出马,这样的情形难得一见。”

“难道说,有人买下了一批特等顽石?”

方源在掌柜的陪同下。已经来到当众解石的工作台前。

但他并不靠前,保持低调,只是远远地望着。

但五位老师傅年老成精,目光扫视。看到商铺掌柜正像个小跟班似的,跟在一个年轻人的身旁,哪里还不知道方源的身份?

但方源只是远远观望,并不走上前去。

五位老师傅心中不禁猜测:“看来这位小主儿。并不想张扬。”

这也很正常。

很多人在解石之前,都是这样。赌输了多没面子啊!

“看来我这次得好好表现表现。争取给这个贵客留下深刻印象,能攀上关系那就再好不过了。”五位老师傅均暗暗盘算,磨搓双手,跃跃欲试。

方源选的几块石头,被伙计优先呈上去。

围观的众人看到这些石头,皆跌破眼镜。

“什么呀?这都是些下等石头嘛!”

“我没看错吧,这些破石头……”

“将下等顽石给老师傅们解,这简直是对他们的侮辱啊。”

但更叫他们诧异得瞪眼的一幕发生了。

五位老师傅双手捧起这些下等顽石,小心翼翼地放到自己面前的石台上,神情十二分的认真严谨,然后均唤出拿手蛊虫,开始当众解石。

五位老师傅大多都是二转巅峰修为,唯有段大师是三转。一时间,各种蛊虫齐齐亮相,看得众人目瞪口呆。

“这是怎么回事!”

“难道这些石头,大有来头,不是表面上这么简单?”

“五位老师傅如此慎重严肃的神情,我还从未见过。”

……

台上,蛊虫飞舞,五位老师傅卖弄着自己的独到手段。

段大师,修为最高,是赌石坊的当家解石大师。他擅长酸液法解石,先取出一个海碗,然后唤出一条蛇蛊,喷出绿色酸液。

然后,他小心翼翼地将顽石放置在酸液当中。

不断有气泡,咕咕地冒出来。

半晌后,他双手罩住一层流光,取出缩小了一半的顽石。再将它放置到另一种酸液当中去。

而张师傅用的是元磁解石法。他双手呈掌,掌心相对。顽石在他的两只手掌中央,凌空悬浮。不断有细小的石屑,被元磁的力量吸摄而出。

还有其他师傅,有的唤出蛤蟆蛊,利用蛤蟆的舌头不断舔舐顽石。有的用火烤烧,有的控制一团小旋风,将顽石放入风中切割。

一众年轻的解石师,在老师傅们的身后,看得目瞪口呆。

区区下等顽石,何须如此大张旗鼓,简直是杀鸡用牛刀,大材小用!

难道老师傅们此举,蕴藏着深意?不行,我得好好观察。

观察的结果,让众人瞠目结舌。

这些下等顽石,解开来之后,没有一个有料的。不是实心,就是空心,甚至连一具死蛊都没有。

“什么玩意啊,还以为有蹊跷呢。”

“原来都是普通的下等石头啊。”

“这些老师傅跑来干嘛,真是浪费我的时间……”

围观的众人大失所望,纷纷拂袖而走。

听到这些话,掌柜的脸色有些苍白。以往他见赌客选中的石头中,没有开出蛊虫来,心中都无限欢喜,但是此刻,他却恨不得自己塞进几只蛊虫进去冒充。

紫荆令牌的贵客,可不能轻易得罪啊!

他小心翼翼地看向方源,刚想要说些什么安慰的话来。

但方源已开口微笑道:“不妨事,看来我今天手气不佳,就不再赌了。下次有时间再来吧。”

掌柜和伙计,一直将方源恭送到门口。

他们还想送出街道,但被方源阻止。

一直到方源的身影,消失在街口,掌柜这才直起腰来。

他转身就给身边的伙计,一个暴栗。叫你怠慢了贵客!

伙计捂住生疼的脑袋,不敢说一句话。

方源今日来此,只是试探罢了。

如今发现了目标,此行就已经得到圆满。

至于那星辰石却不能直接出手,还得费一番波折。

方源要得到那蛊,就得解星辰石。他没有手段去自己解石,要凑齐一套专门解石的蛊虫,所耗甚大,因此就得交给赌石坊中的老师傅。

老师傅当众解石,便会引发轰动。若要动用紫荆令牌,要求他们秘密解石,也不妥当,更会让人疑心。

若方源一上来就开出传奇蛊虫,未免运气太好了。

方源毫不怀疑商家的庞大能量。商燕飞刚刚调查过他,想必也知道了方源在贾家商队中赌出癞土蛤蟆的事情。

若方源两次赌石,一次赌出癞土蛤蟆,一次赌出传奇蛊。商燕飞这样精明之人,能不起疑心吗?

赌石牵涉利益重大,若商燕飞怀疑方源有手段可测石中是否有蛊,那他必定会动心,甚至动手。

方源根本就没有这样的手段,但他身上一些蛊虫却不能曝光。

尤其是春秋蝉。

尤其在商量山,商燕飞还不是首要威胁。真正的首要威胁来源于六转蛊师。

霸主级的势力,几乎背后都有六转蛊师。

蛊师到达六转,均深居浅出,一闭关就是十几年。皆因力量达到质变的程度,他们均有不得已的苦衷,需要为自己负责。

但若是春秋蝉暴露,这些六转蛊师必定会蜂拥而至。

小不忍,则乱大谋。

方源生性谨慎,怎可能在这些小地方犯下大错误?

杜绝一切暴露春秋蝉的诱因,哪怕担负着星辰石被人发现的风险。

依照前世记忆,传奇蛊被发现的可能性很小。但就算被发现,方源也有相应的弥补和解决的办法,总好过自己惹来怀疑,进而春秋蝉曝光的威胁。

“接下来的这些天,我便多去赌石区逛逛,买一些石头赌一赌,给人留下一些印象。”

方源一路思量盘算着,走回楠秋苑。

白凝冰已经从风雨楼回来了。

“正等你呢。”说着,她将购买的冰晶蛊递给方源。

方源动用春秋蝉的气息一催,就将其炼化,然后抛回给白凝冰。

白凝冰帮助他修行,他帮助白凝冰炼蛊,这是毒誓的一部分内容。

白凝冰拿着冰晶蛊,走进密室炼化去了。

方源也走进另一间密室,他要开始炼一种蛊。

蛊名——言而无信。(未完待续)

\end{this_body}

