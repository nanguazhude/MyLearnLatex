\newsection{各有算计}    %第六十一节:各有算计

\begin{this_body}

%1
蒸羊羔,烧花鸭,清蒸八宝猪,江米酿鸭子,抓炒鲤鱼,什锦套肠,麻酥油卷儿,蜜丝山药,拔丝鲜桃,八宝丁儿,清蒸玉兰片,猩唇,驼峰,鹿茸,熊掌,三鲜木樨汤,蜜蜡肘子,三鲜鱼翅……

%2
片刻之后,方源和白凝冰坐在三层的雅座上,面前则摆着满满一大桌的菜肴,各个色香味俱全。

%3
方源取了竹筷,随意吃了几口,他先吃了清蒸玉兰片,清新爽口,令人精神一振。又用了一片拔丝鲜桃,甜而不腻。再取用熊掌,肥嫩可口,蒸羊羔也是脆嫩有加,沾上一些调味料,更显得风味俱佳。

%4
白凝冰喝了一口三鲜木樨汤,顿时满口生津,齿唇留香,也被调动起食欲。

%5
“和你相处这么多天,难得见你大方一回。”她一边吃着,也不忘对方源一番冷讽。

%6
方源一笑,没有搭话,他知道白凝冰心中的疑惑和焦躁。

%7
他一路接近商心慈,费尽心力,不辞辛苦地保护她。到了目的地,却故意分别。这事情令白凝冰看不透。

%8
而方源如今已经是二转巅峰,距离三转只有一步之遥。当初他和白凝冰设下三转之约,如今已经到了最后关口。

%9
但方源当然是不会守约的!在他眼中,诚信这种东西,不过是无可奈何之下的妥协,或者是一层美妙的伪装,一张逼真的面具。

%10
关于这点方源清楚,白凝冰也清楚。

%11
所以她心中焦躁。

%12
因为她已经预感到,方源这混蛋必定会毁约。但她偏偏拿方源没有办法,阳蛊在方源手中,她投鼠忌器得很。

%13
如今没有百家的追捕,没有兽潮,方白二人同桌吃饭,看似和谐温馨,但实际上关系已经紧张到了极点,再往前发展一步,就是分崩离析。

%14
而这一步,就是方源正式晋升三转之时。

%15
一旦他晋升三转,他和白凝冰之间的最后一点寰转的余地都没有了,双方必须硬碰硬,面对面。

%16
怎么处理白凝冰呢?

%17
方源一边吃饭,也在一边思索。

%18
他的修为不可能停滞不前,总有迎来矛盾爆发的那一天。

%19
现在的情形,十分微妙。

%20
方源碎石因为掌控着阳蛊,而占据上风,然而事实上白凝冰也掌握着方源的把柄。

%21
她和方源一路同行,亲身参与了白骨山传承,又亲眼目睹了方源如何吸引兽群袭击商队。

%22
她知道的太多了。

%23
和丁浩那个潜在威胁相比,白凝冰对方源的危险性无疑要大得多。

%24
“如果将白凝冰干掉,商心慈那边就不用担心了。但是我的修为不足,要杀白凝冰也许在行商路上是最好的机会,但那个时候我还须借助她的力量,确保生存的几率。况且在商心慈的身边,也不好下手。白凝冰这家伙,对我其实一直都暗中戒备着。她本身又有冰肌玉骨的防护,做不到一击必杀。而且她战斗意识十分出色,经过最近这段时间的磨砺,更加老辣……”

%25
方源脑海中思绪翻腾,同样的白凝冰心中也在苦思冥想。

%26
“从青茅山出走,到现在,终于有了一丝喘息之机。我一定要取得阳蛊,变回男儿身!直接暴力抢夺,成功率不大,除非我将方源瞬间斩杀。但方源这混蛋,修为虽然只有二转,但综合战力要强悍得多。他就是个怪胎,战斗经验无比老辣。而且他心思极为深沉阴狠,无恶不作,没有什么道德可以束缚他,什么事情都能做得出来。”

%27
“更关键的是,他的底牌我还未探得清楚。不过我也不是没有优势,他现在修行,得靠我出力。而我也知道他很多秘密,也许我可以利用这些东西逼得他妥协!动用毒誓蛊,和他签订不容反悔的契约,除此之外,还可以利用强取蛊、豪夺蛊、妙手空空蛊等等,将阳蛊偷出来……”

%28
白凝冰绝不蠢笨,她一路上一直忍耐观察,一直在思索谋算。

%29
二人同桌吃饭,相隔不过两三步远,看似好友,但心中都在算计对方。

%30
外在的压力消退下去,他们二人间的矛盾顿时凸显出来。在商家城这相对安全的环境中,他们也有功夫可以思考这些东西了。

%31
但越是思考得深入,这两人就越觉得棘手!

%32
方源行事无所顾忌,白凝冰又何尝不是呢?在她的生活理念中,只要活得精彩就可以了。等等,道德原则?那是什么东西?

%33
他们俩其实非常相像,同样蔑视世俗,同样意志强悍,同样对力量极为渴望,同样只相信自己。

%34
从其他人的角度看,他们都是该死的恶魔,都是危害社会的渣滓,他们的死亡就是对世界的造福。

%35
正是因为有如此的相似点,他们俩人才觉得棘手。

%36
真正难以对付的敌人,往往就是自己。

%37
更关键的是,他们互有把柄捏在对方手中。如果不一击制胜,让对方有了喘息之机,那么另一方也必定接近完蛋!

%38
两人越想越头疼。

%39
“白凝冰这家伙不好对付啊。”方源暗暗咬牙,手上可用的资源太少了。

%40
“方源这个混蛋简直没有弱点呀……”白凝冰眯起双眼,目光冰寒。

%41
两人想不出什么好点子,不约而同地抬起目光,看向对方。

%42
目光在半空中对撞,然后一触即分。

%43
面前的菜肴,虽然鲜美,但两人心思沉重,都吃不出味道来。

%44
虽然是打了个半折,但仍旧花费了方源十五块元石。

%45
不愧是商家城,物价不菲。

%46
两人填饱肚子,出了酒楼。

%47
方源刚走到大街上,就听着路人议论。

%48
“你们知道吗?刚刚在南门,商家族长竟然出现了!”

%49
“怎么可能?”

%50
“真事情,他来的快,去的快,整条街都轰动了……”

%51
“吹牛的!商家族长是什么样的人物,怎么可能无故出现在街道上?”

%52
谣言纷飞,有人说是商燕飞,有人则否定。

%53
方源选择东门进城,而商心慈则进的南门。这事情传到东门这里,已经被扭曲得面目全非。

%54
在白凝冰听来,这不过是小道消息中的一个,并没有留意。周围人很快又谈论起其他话题。

%55
但在方源这样的有心人耳中,却是最明确不过的信息。

%56
他不由地暗笑一声,看来商心慈这事情,并没有超出记忆的轨迹。

%57
接下来,就等着果实成熟,砸在自己的头上了。

%58
“快看,这是飞天蓝鲸,翼家的商队来啦!”忽然,街道上有人手指天空,惊呼一声。

%59
一时间,街道上的行人都不约而同地停住脚步,仰天望去。

%60
巨大的阴影,笼罩下来。

%61
天空中,缓缓飞来一头蓝色巨鲸。

%62
说是“飞”,也不恰当,倒不如说是“游”的好。

%63
飞天蓝鲸,这是一种能在苍穹中自由游荡的巨兽。

%64
它居于东海九天之上,性情温纯平和,常有蛊师用御鲸蛊控制住,用于行商。

%65
飞天蓝鲸体型巨大,犹如小山一般,行商队伍藏身在它的体内,飞于高空之中。比在山林中行商的队伍,不仅危险性大减,而且速度更快。

%66
但飞天蓝鲸每天都得吞食上千斤的食物,不是大型家族根本养不起。

%67
整个南疆,拥有飞天蓝鲸的,只有翼家。

%68
翼家亦是南疆的霸主之一,地位和商家相差不多,和东海势力更有千丝万缕的联系。

%69
“真是叹为观止。”白凝冰轻叹一声。

%70
她回想到曾经的白家寨,这种飞天蓝鲸一旦照着白家寨砸落下去,整个白家寨都要被摧垮吧。

%71
巨大的身影在商量山上一动,飞天蓝鲸缓缓地降落到一侧山峰处。

%72
方源可以远远地望到,它张开大口,密密麻麻的黑点从它的口中缓缓移出。

%73
这些黑点,就是翼家的商队。只是距离太远了,看不清楚。

%74
“翼家的商队来了,看来市场又要动荡了。”

%75
“听说翼家这次带来了一只五转蛊,要在商量山拍卖。”

%76
“翼家和东海有联系,这次肯定又带来许多东海的特产,值得购买。”

%77
路上行人的话题,已经完全转移到翼家商队身上。

%78
方源和白凝冰一路向上走。

%79
古月山寨只是占据青茅山一隅,而商家城却几乎覆盖了整个商量山。

%80
在南疆,就算是第一家族的武家,也没有如此面积广阔的山城。

%81
整个商家城,各种建筑都有,有竹楼,有土坯房,有砖瓦房,有窝棚,有树屋,还有蘑菇房,洞窟,塔楼以及碉堡等等。

%82
这些建筑混搭在一起,像是给商量山披上了一层色彩驳杂的衣裳。

%83
商家城作为南疆的贸易中心,占地面积当之无愧为第一。

%84
但如果有人认为,这就是商家城的全部,那就大错特错了。

%85
这才只是商家城的外城。

%86
方源和白凝冰来到一处巨大的山洞前。

%87
“二位要进内城?请每人缴纳一百块元石。”守在这里的蛊师,伸手索要道。

%88
“单单进去,就需要一百块元石?”白凝冰表示惊异。

%89
“内城空间较小,这也是为了防止无关人物涌入进去。同时也方便维持治安。”蛊师客气地解释道。

%90
商量山经过商家数千年的经营,不仅将商家城遍布山体表面,而且更深入山体内部,建成内城。

\end{this_body}

