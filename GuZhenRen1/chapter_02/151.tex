\newsection{狗胆蛊}    %第一百五十一节:狗胆蛊

\begin{this_body}

%1
眼前迷雾重重。

%2
方源在雾中缓步行走,在他的身边,环绕着一大群的犬兽。

%3
十一头的刺猬犬,一百三十二头的电文狗,十七头菊花秋田犬。除此之外,还有一只百兽王——大电文狗。

%4
受到天地伟力的拘束,这群犬兽默默地跟着方源,安安静静,乖乖巧巧,不发出一点声音。

%5
迷雾的世界,一片沉静。

%6
安静到方源的每一次呼吸,都能清晰可闻。

%7
虽然过了第二十轮,手中的犬兽数量达到前所未有的最高,方源的心中却并不乐观。

%8
“这三王传承,每过十轮,难度就要翻上数倍。我虽然拥有上百头的犬兽,但要渡过接下来的十轮,在第一次就冲刺到第三十关,并不容易。”

%9
第三十关,是方源首次冲刺犬王传承,对自己设下的目标。

%10
前世的三王传承,历经多年,才被众多蛊师瓜分完毕。方源此次在三叉山,必定也要历时良久。想要一次性地夺取传承,并不现实。

%11
三王传承,不管哪一道的难度都相当大。方源就算是拥有百兽王,拥有上百头的犬兽,充其量也不过是暂时站稳脚跟。

%12
一旦接下来,他在任何一轮关卡中大意,就有可能一败涂地,好不容易积攒的犬兽丧失一空,甚至还会命丧于此。

%13
魔道传承,向来都是这般残酷无情。

%14
“这道犬王传承,越往后面,越对奴道蛊师有利。我前世也走过奴道,但今生终究还只是力道蛊师。”

%15
经历了刚刚的那场惊险大战,方源的脑袋还残留着眩晕感。

%16
这是剧烈地调动精神。消耗大量心力指挥犬群作战,所留下的后遗症。

%17
但若换做是魂魄强大的奴道蛊师,这种症状就会小很多。

%18
奴道,是奴役兽群、虫群的蛊师流派。要同时指挥大量的虫兽,对心力、精神的消耗很大。因此奴道蛊师常常动用特定的蛊虫。来锻炼和加深自身的魂魄底蕴。

%19
在犬王传承中,天地伟力一丁点都不约束和限制,蛊师们的魂魄底蕴。

%20
所以,奴道蛊师在犬王传承中,会更加如鱼得水。

%21
“不过奴道蛊师也十分艰苦,一来食料负担极重。虫兽规模越大,吃的自然就越多。二来收服强大的虫兽,并不容易。很有可能遭到反噬,令自己变成白痴。三来奴道蛊师过于依赖虫兽,自身薄弱,很容易被独自针对。”方源心中思量着。

%22
蛊师各大流派。都有各自的优劣和难易。

%23
世事变迁,沧海桑田。在当今的蛊师界,气道已经几乎湮灭,力道彻底没落,只剩下些许残光。奴道则中规中矩,难以兴盛。偶尔出现几位奴道强者,也往往如流星一般。只能璀璨炫目一时。

%24
“奴道蛊师,往往对于资源要求十分庞大,且又劳心劳力。就算是中小型的山寨,也养不起奴道蛊师。最多是大型、超级家族,为了大型战事,特意栽培数位奴道蛊师。在魔道中,奴道蛊师经营更加困难,尤其少见。所以我不取此道。”

%25
方源定了定神,收起散漫的心绪。

%26
这一会儿工夫,他感到脑袋渐渐清明。眩晕感消失了很多。

%27
他转动双目,开始打量周围。

%28
在他的左右两边,以及前方,各亮着一团光影。

%29
之前的光影,都是模糊一片。但二十轮之后。光影都变得无比清晰。

%30
左边的光影中,是一群菊花秋田犬,有大约两百多头。有的趴在地上栖息,有的在嬉戏打闹,有的围着母狗肚下喝奶。

%31
方源聚精会神地看着,目光不断搜索。

%32
菊花秋田犬的兽王,和其他犬兽不同,实力越强,体型却会变得越来越小。

%33
尤其是在情势纷乱的战场上,兽王夹杂在狗群中,毫不起眼。是一种效果不错的自我保护的方式。

%34
“百兽王会在哪里?”方源目光逡巡间,光影忽然一变,显出一只蛊。

%35
这只蛊,有拳头大小,形状好似一颗鹅卵石,深褐色。表面光滑,好像覆盖了一层油光。

%36
“狗胆蛊。”方源第一时间,就认出了此蛊,“在兽群之后闪现出来,这就表示如果我选择对战这群菊花秋田犬,若是能获胜,就会得到一只狗胆蛊……”

%37
不一会儿,这团光影消散了。

%38
方源没有做出选择,而是将目光投向正前方。

%39
正前方的光影,正显示着一支黑黝黝的狗群。

%40
这些狗相互叫喊,一个个捉对,向对方冲刺,然后狠狠地撞在一起。充分展现出了这种狗的凶狠猛烈。

%41
这是铁甲狗。

%42
这种狗的身上,长着皮甲。皮甲黑幽厚重,如铁一般,防御力十分可观。

%43
在成长的过程中,铁甲狗都会相互对撞,用来消减身躯增长时的强烈的瘙痒。

%44
铁甲狗的百兽王,体型和大电文狗差不多,在狗群中十分显眼,方源一眼就找到了它。

%45
随即,光影变幻,化为一只蛊。

%46
这只蛊,方源也很熟悉,刚刚还用了的。

%47
就是二转的驭犬蛊。

%48
“选择铁甲狗群,若是得胜,就能得到一只驭犬蛊么……”

%49
方源口中喃喃,旋即又将目光投向右边。

%50
右边的光影,却毫无犬兽的踪影。而是显示了一片山腰处的风景,灰白山石陡峭,树木苍翠葱茏,正随风摇曳。

%51
显示了一会儿,这团光影就消失了,没有像之前那样,出现蛊虫的影像。

%52
“这是一次退出犬王传承的机会。”方源心中了然。

%53
犬王传承布置在福地当中,进入其中的蛊师们,受到天地伟力的约束,并不能随意进出。

%54
但三王传承,也不是绝对的有进无出,而是留下了一线生机。

%55
每当一定阶段,传承都会可能提供退出的机会。

%56
就如同这次,方源若选择右边。当他走出迷雾时,就会被传送出去。

%57
出现的地点,就是光影中显现的那处山腰。而这处山腰,也并不远,在三叉山的某处。

%58
但方源没有选择右边。

%59
“我手中有资本,还可以再冲一下。现在就退出,有些可惜了。当然,接下来未必会出现这样的机会,能让我从容退出。”

%60
这其实也是一次无形的考验。

%61
许多蛊师,明明有实力,但为了稳妥,选择放弃,退出传承,白白地浪费了大好机缘。

%62
有的蛊师,没有资本,却鲁莽前行,错过退出的机会,死亡之时悔之晚矣。

%63
方源从不缺乏勇气,毅然放弃了退出的机会。对于他而言,只剩下两个选项。

%64
“左边,是菊花秋田犬。此犬最为团结,犬王的位置不明。但若得胜,会得到狗胆蛊。狗胆蛊乃是二转蛊,乃是奴道的辅助蛊,专门用在犬兽身上。狗胆蛊一经催动,就能让一定范围内的犬兽胆气倍增,扫空怯弱,战力全部发挥出来。”

%65
方源不由地想到,刚刚的那一战。

%66
他被迫动用电文狗,来围攻大电文狗。结果,电文狗均怯弱不前,战战兢兢,战力大打折扣。方源只好不断消耗心力,一个劲的催促它们战斗,导致精神剧烈损耗。

%67
兽王,都会对普通野兽,产生一种压制和威慑。

%68
但方源若在刚刚,动用狗胆蛊,将效果覆盖到自己的电文狗群上。

%69
那么这些电文狗,就会胆气十足,再不惧怕,可以十分勇敢地针对大电文狗展开攻击。

%70
所以说,狗胆蛊虽然不是用来进攻的蛊,但却是绝好的辅助蛊虫。它针对犬兽,能让犬兽勇气倍增,犬群的士气永远高昂!

%71
“狗胆蛊只是二转,能令普通犬兽,克服百兽王的威慑。再往上,就是三转的狗胆包山蛊,能抵制千兽王的威慑。到了四转,则是狗胆包海蛊,能抵消万兽王的气息压制。若能侥幸晋升到五转,便是大名鼎鼎的狗胆包天蛊!能让阉狗叫春,老狗成疯,蔫狗狂浪,就算是面对犬兽之皇也有干翻抡倒的勇气……”

%72
方源对此一清二楚,更知道这犬王传承中,关于狗胆系列的蛊,从一转到五转都拥有。

%73
前世就有许多人,获得了三转、四转的狗胆蛊。还有一位幸运儿,获得了唯一一只的五转狗胆包天蛊。

%74
“我要在犬王传承中大展宏图,拥有狗胆蛊是必须的。”

%75
方源不得不承认,自己对狗胆蛊,有着一股渴望。

%76
但他脚步却没有迈向左边,而是直指前方——

%77
他选择了铁甲犬群。

%78
他虽然很想得到狗胆蛊,但却没有被欲望冲昏头脑。

%79
菊花秋田犬群,极为团结,打群战时战力极强,是个硬骨头。同时,它们当中的犬王,隐藏得很深,方源并没有找到。这是一个巨大的隐患。

%80
但铁甲狗群就不同了。

%81
它们的狗王很明显,而且方源的手中,还有一项专门针对铁甲狗群的利器。

%82
方源走出迷雾,铁甲狗群相当警觉,立即有所反应。

%83
作为百兽王的大铁甲狗,仰天大啸,狗群奔腾,很快就汇集到它的身边。

%84
整个过程,方源都静静地看着。

%85
他没有像上次进行突击,上一次是因为有二转驭犬蛊,这次他的手中却没有。

%86
他需要将这支狗群歼灭打散,才能通过这关,得到一只驭犬蛊作为奖励。(未完待续。如果您喜欢这部作品,欢迎您来起点投推荐票、月票,您的支持,就是我最大的动力。手机用户请到阅读。)

\end{this_body}

