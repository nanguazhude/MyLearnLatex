\newsection{奔逃}    %第九节:奔逃

\begin{this_body}

约莫过了半个时辰,熔岩鳄王发出最后一声惨叫,再无声息。////

随后方源和白凝冰两人,就听到轩辕神鸡啄食的声响。

但轩辕神鸡食量很大,一只熔岩鳄王,显然并不能满足它的胃口。

这一夜,方源和白凝冰都没有睡。

熔岩鳄王的惨叫之后,他们又听到白猿的哀啼,吞毒蛙的惊鸣,蜂群飞舞的嗡嗡嘈杂声。最主要的,还是轩辕神鸡嘹亮的鸡鸣。

直到黎明时分,轩辕神鸡冲霄飞走。在空中,划出一道七彩斑斓的飞行光虹。

看着光虹的尾端,彻底在天空中消散,方源和白凝冰二人这才敢走出山洞。

两人来到原来的山谷。

山谷已经面目全非,坍塌了大半。熔岩鳄王肚皮翻上,仰躺在地上,死的不能再死了。

它的肚皮被轩辕神鸡啄开,里面的血肉、内脏都被吞食。只留下残破的骨架撑着的一身暗红鳄皮。

两人搜寻了一番。

他们很快就找到了一片片红色琉璃的碎片。这是炎胄蛊的残骸。

很显然,熔岩鳄王催动炎胄蛊防御,结果被轩辕神鸡强行摧垮,导致炎胄蛊毁灭。

至于其他的熔岩炸裂蛊,以及积灰蛊,都不见影踪。

这也不奇怪。

宿主一死,寄居在野兽身上的蛊虫,宛若失去了家园,便会主动离开,再次流荡。

积灰蛊是很理想的治疗蛊,最适合方源目前的情境。

但世事不如意,十居**。

没有得到积灰蛊,这也在方源的意料当中。不过,他们俩倒也并非没有收获。

熔岩鳄王的尸体中,还残留了一些血肉。

轩辕神鸡吃了大头,到底还是留了些残羹剩炙的冷汤水,留给了方源和白凝冰二人。

方源和白凝冰两人辛苦了一个上午,将鳄肉切割开来,储存到兜率花中。

“这些鳄鱼肉,足够再喂养鳄力蛊三个月的。我们再去其他地方看看。”

方源和白凝冰来到白猿群的领地。

原先茂密繁盛的树林,白猿群在其中嬉闹,玩耍。

如今到处都是倾折的断木,白猿的断肢残体夹杂其中。一些老弱残幼的白猿,守护在亲族的尸体旁,发出嘤嘤的哭泣声,整片树林都弥漫着悲伤、凄惨的氛围。

轩辕神鸡在昨夜,给这支数千规模的白猿群,带来了灭顶之灾。如今只剩下两三百的规模,残留着几只百兽王级的白猿,还各个带伤。

白凝冰看得双眼发亮:“此刻是白猿群最衰弱的时候,要不要动手?”

方源却阻止她。

倒不是他怜悯这些白猿,而是知道,在某种方面来讲,此刻的白猿群更加危险。

“哀兵必胜,不要惹这些白猿。一旦招惹它们,它们必定狂怒攻击,不杀死我们誓不罢休。那几只百兽王,虽然有伤,但还不是你一个人能对付得了的。”

白凝冰闻言,看了方源一眼,最终放弃了动手的打算。

二人又来到西南方向的腐烂沼泽。

沼泽已经变得面目全非,被轩辕神鸡翻了一个底朝天。

生活的环境遭到了巨大的破坏,沼泽的势力重新洗牌。神鸡虽然飞走了,但是沼泽却并不平静。各种毒物在厮杀,在乱窜。

方源和白凝冰二人站在沼泽边缘,仅仅一会儿工夫,就远远望见了三场争斗。

一场是两只色彩斑斓的毒蛇交战,最终一只将另一只吞下。但刚吞下不久后,来了一只螃蟹大小的蝎子,将毒蛇扎死。

第二场是一只毒粉飞蛾和幽蓝色的蛤蟆。战斗突然爆发,蛤蟆舌头一伸一缩,将飞蛾吞入腹中。然后片刻之后,飞蛾在蛤蟆的肚中窒息而亡,而蛤蟆也被毒粉毒死。

第三场则是一只脸盆大小的黑蜘蛛,从泥浆中翻滚出来。它的身躯表面沾满了蚂蚁。一会儿工夫后,蚂蚁获胜,就地将蜘蛛彻底啃噬掉。

看到这种混乱的场面,方源和白凝冰掉头就走。

最后他们来到狂针蜂群的蜂巢处。

原本房屋大小的蜂巢,已经彻底倒塌。周围静悄悄的,没有一只狂针蜂剩下。

两人走近。

旋即,一股芝麻般的香气,飘入白凝冰的鼻腔中。她吸了吸琼鼻:“什么味道?”

“就是蜂巢的味道。狂针蜂并不酿蜜,它们的蜂巢是一种上佳的炼蛊辅料。同时,也是一种食物,人可以直接食用。”方源说着,就伸手掰开内部的一块蜂巢。

咔嚓。

蜂巢碎片是暗黄色的。

在白凝冰好奇的目光下,他将蜂巢放入嘴中,一口咬碎,几下咀嚼,就吞入肚中。

蜂巢的味道,让他想起地球上饼干的味道,又香又脆。

但毫无疑问,这种纯天然的食物,比饼干要好吃多了。带着丝丝的甜味,却一点都不油腻,反而有一股清爽之气。

“嗯,味道不错!”白凝冰也掰开一块,吃下去,顿时觉得口中生津,甜美的味道让她一直微皱的眉头,都不自觉地舒展开来。

“我们的腌肉、干饼都快要吃完了。不如采集一些蜂巢,放在你那兜率花中。”白凝冰提议道。

方源回望了天空一眼,神情有些担忧:“我正有此意,不过动作得快一点。”

“你是担心熔岩鳄王,还有白猿尸体的血腥气,引来其他猛兽吧。放心,今天无风,要吸引其他野兽过来,需要一段时间。这段时间,足够我们挥霍了。”白凝冰笑了笑。

方源摇了摇头,刚想要说话,忽然面色一变。

嗡嗡嗡……

蜂群急速飞行的声音,传到了两人的耳中。

白凝冰循声回望天空,只见一蓬乌云,由数不胜数的狂针蜂组成。正向他俩杀来。

狂针蜂巢虽是被轩辕神鸡推倒,把蜂巢最中心,也是最美味的那一大块吃掉。但狂针蜂却没有消灭多少。

狂针蜂难以对轩辕神鸡造成伤害,后者吃饱喝足之后,也没有耗费力气,却剿除这些微不足道的小东西。

狂针蜂群家园被毁,智力低下,就对轩辕神鸡展开追杀。

但轩辕神鸡飞上高空之后,它们就有力未逮,追之不及,只好回转蜂巢,准备重铸家园。

然后,它们就看到两个少年,正站在它们的家中,吃它们的蜂巢。

这样的情况,还有什么好犹豫的?

先前针对轩辕神鸡的怒火,完全转嫁到方源和白凝冰的身上。

一时间,无数的狂针蜂,振动双翅,如万千雨点,急速笼罩而下!

白凝冰楞了楞。

“还不快跑!?”方源转身就跑,撒开了双腿。

被他这一提醒,白凝冰也旋即转身,向方源追去。

狂针蜂群在他们身后,紧追不舍。

方源在前边跑,白凝冰落在后边。他们没有移动蛊虫,白凝冰很快就被蜂群追上。

砰砰砰。

白凝冰撑起天蓬蛊,白光虚甲一阵晃荡摇曳。顷刻间,遭受到上千次的攻击。

狂针蜂的蜂针,如钢似铁,硬度很高。在加上高速飞行,不下于箭雨攻潮。

巨大的数量,引发质变。

白凝冰的真元海不断下降,蜂群的攻击,不可轻视!

更麻烦的是,狂针蜂群中的几只,已经成蛊。

三转的狂针蜂蛊有洞穿之能,就算是天蓬蛊也防护不住。白凝冰的后背很快就开了几个血洞,令她不禁痛哼几声,奔逃的速度受到刺激,竟然突破了往常自身的极限。

白凝冰从未有想过,居然自己有一天,能徒步跑得这么快。

山石和树木仿佛都在向她撞过来。她不得不打起全部的精神和注意力,一旦被什么东西绊倒,她身后的蜂群必然会在第一时间,将她团团包裹。

群攻之下,她必死无疑!

有白凝冰在身后分担压力,方源这边的情况要好的多。

他催动真元,灌注到后背上的背甲蛊中。

他的背后皮肤,顿时凹凸不平,微微地隆起来,变成又厚又硬的鳄鱼皮甲。

普通的狂针蜂射在鳄鱼皮甲上,往往无功而返。而数量稀少的蜂蛊,则都被白凝冰吸引了过去。

又奔逃了半刻钟,狂针蜂群仍旧紧追不舍。

方源和白凝冰都气喘吁吁,速度开始缓慢下降。

“有救了,前面是湖!”情势越加危急之时,方源忽然大叫一声。

白凝冰大喜过望。

只见树木渐渐稀少,一片蓝白的水光出现,夹杂在翠绿的颜色中,且越来越多。

两人冲出树林,一片湖水便跃入眼帘。

方源没有犹豫,扑通一声,跳入湖水中。

白凝冰紧随其后。

嗖嗖嗖!

狂针蜂认定了他俩,竟然也射入到水中。

白凝冰身上的白光虚甲一阵剧烈的震动,顷刻间她遭受到大量的攻击。

剧痛传来,她抿紧双唇,双手划动,极力向随水中深处潜去。

半晌之后,方源和白凝冰二人从另一处登岸。

他们身上蜂巢的味道,已经被彻底洗去。再望原处,还有大量的狂针蜂,在不甘地盘旋着,时不时地攻击着水面。

狂针蜂体型虽小,但十分精悍。就算是暂时地落入湖水中,只要不深入,也能飞上来。

“该死的……”白凝冰轻声咒骂,心中还残留着余悸。

她的脸色非常难看。

不管是轩辕神鸡,还是白猿群,狂针蜂群,都不是她能够应付的。

如果昨夜她被轩辕神鸡发现,必定成为神鸡的腹中餐了。

三转的修为,在这个残酷的大自然中,只是底层罢了。

“我有点受够了。我们什么时候,才能到白骨山?”

“嘘……安静!”方源满脸的凝重,他半蹲着,手指着岸边的一处篝火残留的痕迹。

白凝冰顿时眉头紧锁。

毫无疑问,这是其他人留下来的痕迹。(未完待续。

------------

\end{this_body}

