\newsection{青铜舍利蛊}    %第十七节:青铜舍利蛊

\begin{this_body}

曾经生机勃勃的青茅山,如今成了冰天雪地。(138看书.138看书.)如此惊变,早就引起了附近周遭势力的关注和调查。这些个月来,青茅山覆灭的消息更已经渐渐传播到了远方。

“不敢回想。一想起来,就是痛楚啊。”方源坐下来,满脸哀戚之色。

“来人,上酒。”百家族长见方源不想说,也不追问,吩咐家奴送上两坛酒。

白凝冰无动于衷,她向来不喝酒,只喝水。

方源直接拍开封泥,直接咕咚喝了一口,又落下泪来。

百家族长吃惊:“贤侄,又何以落泪?”

“喝了贵族的酒,醇香醉人,不禁就想起我族的青竹酒,曾经在青茅山上喝酒的日子。”方源一边抹着泪,一边道。

家老们的唏嘘声更大了,不少人安慰方源。

方源的遭遇,让他们也产生了共鸣。毕竟百家的元泉已经有枯竭的迹象,若寻不到新的元泉,古月一族的惨状,就是他们百家的未来。

百家族长又好生劝慰了一番,这才将方源劝住,停住了眼泪。

“家园被毁,哪个能不痛心?贤侄的心情,我能体会。但只要有人在,就有希望。贤侄啊,不要悲伤,相信过不了几天,你就能和族人汇合了。”百陌行试探道。

方源作懵懂未觉状,擦了把眼泪,随口应答道:“是啊,应该过不了几天吧。”

听到这个回答,百陌行和百家族长不禁快速的对视一眼。

晚宴之后,百家族长召来百陌行,秘密商谈。

“族长,情况不妙啊。古月山寨被破,这些残众为什么要特意赶来白骨山?很有可能就是抢占这块地盘。我们要不要先下手为强呢?”百陌行一脸担忧地道。

“呵呵呵。”百家的女族长却轻笑起来。

百陌行微楞:“族长何以发笑?”

女族长眯起双眼,安抚道:“陌行家老稍安勿躁,此事有利有弊。只要运作得好,却能省下我们不少功夫呢。”

被这一提醒,百陌行顿时若有所思起来。

没有错!

百家的元泉,用了许多年,已经快要干涸枯竭了。需要尽快找到新的元泉,这一次行动,以狩猎为名,其实就是要在白骨山上寻找到合适的泉眼。

百家队伍刚来不久,却还没有探查出什么东西。但古月残众长途跋涉,要来白骨山,必然是知晓了什么情报。

可以推测出来,古月家族必然掌握了关于元泉位置的情报。

百家族长看到百陌行的神情变化,继续道:“看来你也想到了。其实但凡名山大川,都是元气凝聚之地,总会存在元泉。但是要探查元泉的准确位置,却不简单。需要消耗大量的人力、物力。”

“我们百家周围,有方家,有廖家,有范家。都是实力强悍的家族,一直以来相互制衡。我们百家元泉枯竭的事情,若是传出去,恐怕就会惹来他们的趁人之危,以及落井下石。所以我们之前商议,要秘密寻找元泉的位置,这一次也是打着‘狩猎’的幌子。然而这样一来,做起事情来束手束脚,探索元泉时消耗的人力、物力将更大。”

百家族长说到这里,欲言又止,忧心忡忡。

百陌行接着道:“所以,族长你是想从古月家,直接得到有关元泉位置的消息?”

“正是如此。”百家族长点头,眼睛绽放着亮光,“古月家的族长、家老们肯定不好对付,但谁叫那两个少年落到我们手上呢?这可是天赐良机啊!”

百陌行皱起眉头:“不过,我看那两个少年也不笨。那个丫头一看就是心志坚定的死硬派,那个小子虽然修为薄弱了些,但是处变不惊,遇事沉静。第一次见面,被我们包围时,一丁点都没有慌乱。想要撬开他们的嘴,恐怕不是一件易事。”

百家族长嘿了一声:“若没有这等心志,怎么配当一族的少主?这两个少年,都很优秀。如果不优秀,那我可就要怀疑他们的身份了。”

百陌行又道:“所以族长你要三思啊。若是严刑拷打,恐怕他们未必服软。他们沿途还留下了痕迹,古月残众很快就到。这些人可都是标准的走投无路的亡命之徒了。”

百家族长微微摆手:“这点家老不必担忧,我已有一计。”

“哦?老朽愿闻其详。”

百家族长细细说了,百陌行浑浊的老眼越听越亮。

百家族长说完,百陌行忍不住交口称赞:“此计大有可为!我观那古月方正,的确是心系家族,席间两次落泪,乃是性情中人。他到底是个年轻人,族长你布下这局,就如把蜂蜜摆在小熊罴的面前,把胡萝卜放在幼兔的鼻端。不愁他不上钩啊。”

……

方源撩开营帐的门帘一角。

夜已经深了。但百家这处临时的营地,却灯火通明。

营帐错落有致,每隔一段距离,都会有铁架火炬。时不时的还有巡逻的蛊师队伍。

“方正公子,有什么事情吗?”方源刚刚撩开营帐,门口的两个守卫立即走过来。

方源故意打了个酒嗝:“席间喝了许多酒,这里何处可以方便一下。”

“公子,这边请。您是我族的贵客,族长已经特意安排了方便之地,不到三十步。”其中一位守卫立即说道。

“你给我指个方向,我方便时不喜欢有人在附近。”方源摆手道。

“不敢违背公子的命令,木屋就在那处。”守卫指了指,躬身退下。

方源去了木屋,撒了一泡尿后,装作醉眼朦胧的样子,故意走错方向。还不出二十步远,立即就有几个巡逻的蛊师走来:“贵客走错方向了,您的营帐在那边呢。”

“是这样吗?我怎么记得是在那边。”方源打了个酒嗝。

“贵客这边请吧。”百家蛊师脸上浮显出虚伪的笑容,隐含着一种强硬。

方源被这些人重新带回营帐。

营帐中,点着灯。

东西两面,各设有两个床榻。白凝冰正盘坐在床榻上,进行修行,消耗真元冲刷空窍。

听得方源进来的声响,她睁开双眼,以目光询问。

方源扫了她一眼,倒在床榻上:“凝冰,你也早点睡吧。这些天把你累坏了,过不了多久,就能和族人汇合了呢。”

他越说越迷糊,说到最后,已经闭上双眼,只余鼻息,仿佛是已经睡着了。

白凝冰的瞳孔微微一缩。

她知道方源必是表演,故意这番说话,是为了防止有窃听监视的蛊虫。刚刚他出去方便,这么快就回来,也说明了这里防卫森严,要想夜里偷偷溜走,绝无可能。

思虑及此,她不禁暗暗担忧。

自己虽然是三转巅峰修为,但是蛊虫不佳,导致战力不足。

在这营地中,百家族长是四转蛊师,还有五六位三转的家老,大量的二转蛊师。

“人为刀俎我为鱼肉”就是如今的情景。百家寨虽然是正道阵营,但人为财死鸟为食亡,一旦利益巨大,肯定会杀人灭口。

白凝冰知道,方源手中的蛊虫几乎各个都是精品。尤其是天元宝莲、血颅蛊,一旦暴露,必定会引来百家蛊师的觊觎。

他们之所以现在没有直接动手,是因为方源扯了虎皮,利用不存在的古月族人,暂时诓骗了他们。

几天之后,他们不见古月族人的到来,必定会有所怀疑。到那时,方白二人走投无路,局面就相当危险了。

“该怎样破局呢?”白凝冰微微皱起眉头,看向对面的方源。

方源已经侧身躺过去,背对着她,听其呼吸声,平缓均匀,倒像是真的睡了。

“你倒是沉得住气!”白凝冰冷哼一声,心中微恼,又无奈。

……

次日。

风和日丽,阳光明媚。

三声鼓响,百家族长召集族人。

“我们百家一年一度的狩猎大比,从今日起连续七天,都是你们展现实力的时候。按照惯例,夺得名次的,都有重赏!接下来,尽情地展现你们的勇武吧。”

百家族长一挥手,寨门打开,早已经跃跃欲试,蠢蠢欲动的蛊师们,顿时急不可耐地蜂拥而出。

不一会儿,就汇入到各处的山林当中去,不见了踪影。

刚刚还拥挤的整个营地,顿时显得宽敞和空寂。

“方正贤侄,昨晚睡得可好?”百家族长回过头来,笑眯眯地对方源道。

方源拱手行了一礼:“多谢族长款待。晚辈昨晚一躺下就睡着了,醒来时已经是天亮。”

“哈哈哈。”百家族长笑了笑,拍拍方源的肩膀,一副亲切的架势,“贤侄可想参加我族的狩猎大比么?也让我们看看古月男儿的英姿!”

方源面露难色,推诿道:“惭愧!晚辈不久前受过重伤,修为从三转掉落,幸得族中救治,捡回来一命,但如今只有一转中阶。”

其实不用方源说,他身上的一转气息,早就一目了然。

“这点贤侄不必担心。作为我族的贵客,自然有所优待。这样吧,只要贤侄能猎到一头成年黑熊,那么这只青铜舍利蛊,就是你的奖励了。”百家族长轻轻地拍拍手,身旁的一位蛊师献宝似的,掌心一摊,呈现出一只手指头大小,圆球状的蛊。

方源看着这只蛊,心中冷笑,眼里却流露出热切的神情:“那晚辈就恭敬不如从命了。”(未完待续。如果您喜欢这部作品,欢迎您来138看书文学注册会员推荐该作品,您的支持,就是我最大的动力。)

\end{this_body}

