\newsection{有才无德}    %第一百四十五节:有才无德

\begin{this_body}

“李哥,你就出手吧……”狐魅儿坐在李闲的怀中,嗲声嗲气地撒娇着。

李闲圆鼻大目,额头宽阔,脸上散发着一层油腻的光。他身材矮胖,此时粗短的双手,一只揽在狐魅儿娇柔的小腰上,另一只则搭在她的腿上。

李闲修为,有四转高阶!但他很少出手,他做买卖谋生,是魔道中著名的大奸商,人脉广阔。

此时,他怀抱着狐魅儿,一脸色眯眯的笑道:“魅儿,你身若无骨,娇肤如玉,这些天不见,你又变漂亮了。”

狐魅儿瞟了一个白眼,嗔道:“李哥,奴家和你说正事呐。那黑白双煞得罪了奴家,奴家都是你的人了,你可要为奴家做主啊!”

“嗯嗯嗯……”李闲一边口中敷衍着,一边十指大动,在狐魅儿的身躯上游走。

狐魅儿娇喘两声,面上腾现出两抹红晕。配合水汪汪的大眼睛,诱人之极。

但她心底却是冰雪一般的清醒。

她又劝说两句,见李闲始终嗯嗯哼哼,就是不答应,反而大占自己便宜。

狐魅儿便推开李闲,站到地上,脸上带着哀怨:“李哥,你到底给奴家一个准信!奴家这般央求你,你都不可怜可怜奴家吗?”

“哎哟,我的心肝儿,我的小宝贝儿。不是你李哥不想办,而是那黑白双煞实在有些棘手。他们可是两个中转蛊师,你李哥我势单力孤,只有一个人啊。”李闲摊开双手,一副无可奈何的样子。

狐魅儿哼了一声:“好吧。那奴家就降低标准,只要求对付那个小兽王。这个方正,最是可恨了!那个白凝冰。李哥可以暂且不去管。”

“这样啊……”李闲犹豫起来。

“李哥!”狐魅儿一看有戏,又主动投怀送抱,在李闲肥胖的耳边吐气如兰。

李闲顿时感觉,自己小腹处有一团欲火升腾起来。

他双眼目光变得迷离:“好,魅儿你既然这么要求了,那李哥就帮你出了这口恶气。不过……”

他话锋微微一转:“再过一两天,三王传承就要重新开启了。这个节骨眼儿,可不能出差错。等三王传承之后,李哥再来教训那个小兽王。你看好么?”

三王传承开启在即,所有人都盯着呢。

狐魅儿撇撇嘴,对于李闲的推脱,也不好多说什么。

两人又耳鬓厮磨一阵子后,狐魅儿便告辞。

李闲虽然极力挽留。但狐魅儿态度坚决,最终李闲只好将狐魅儿送出洞府。

他站在洞口,眼巴巴地看着狐魅儿的背影,渐渐远去,最终消失在山林当中。

“哼,这个小妖女!”李闲回转洞府,脸上色眯眯的神情消失不见了。流露出谨慎阴沉的神色。

“想要挑拨离间,把我当枪使,对付小兽王?这个小妖女,还真以为自己魅道大成了。可笑!”李闲嘴角笑意很冷,“要对付黑白双煞,我当然有办法。但我李闲什么时候,做过亏本生意?呵呵。”

“这个小兽王方正很不简单。他看似粗犷蛮横。实则心思细腻。赶在三王传承开启之前,连挑三位魔道蛊师。这一定是他蓄谋已久的计划!”

蛊师们来自天南地北,汇集到三叉山,是为了什么?

还不是为了三王传承!

方源的嚣张,没有为他惹来麻烦。这些天来,只有他找别人麻烦,为什么?

因为所有人都眼巴巴地盯着三王传承,都在暗中准备,做着努力。谁想在这个关键时刻,和小兽王这个疯子血拼,错失这等机遇良缘?

没有人!

“这个小兽王,就是利用了人们的这个心理,从而肆无忌惮,建立了名声。两天之后,三王传承开启,谁想费力不讨好地阻止他进入?”李闲摇摇头,已经预见了未来。

自从方源杀了薛三四之后,就算是他,也对小兽王心存忌惮。

之前,方源打爆横眉暴君、费立,都没有引起李闲的重视。但是方源当中毁约,趁势杀掉飞天虎后,李闲这才意识到方源的棘手!

何也?

正道选拔,通常将人才分为四等。

第一等,是有才有德。有才华又有品德,最为上等,可以独当一面。

第二等,有德无才。虽然没有才华,但是却有品德。有孝顺心,就会听父母话。有忠诚,就会听头领的话。有诚信,就不会毁约。用了能让人放心,至少不会出大的纰漏。再说,很多才能也是历练出来的。

第三等,有才无德。虽然有才华,但是没有德操。有才能,可以胜任职位,但是高层用着却不放心,生怕哪天被背叛。

第四等,则是无才无德。没有才华,也没有德操。没有利用的价值,用了也不放心。

放到魔道当中,什么是最棘手的魔道中人?

第四等,无才有德。你没有才能,却要讲究美德,生活在理想当中。往往还没人灭你呢,你就自己毁灭了。

第三等,是无才无德。虽然没有才华,但也没有德操。至少能做些卑贱下流的事情,混口饭吃。

第二等,是有才有德。有才华,能有口饭吃。但同时却还有德操,心中还有孝顺、恭良、忠贞等等束缚自己的绳索。在魔道这个尔虞我诈的环境中,这就注定这类人,不会混得风生水起。

而第一等,则是有才无德!

方源就是这样的人。

他有才华,简直是才华横溢。本身资质就出众,甲等资质不是谁都有的。

其次他有战斗天赋,能越阶战斗。横眉暴君、费立、薛三四的修为都比他高,却都惨死在他的手中,这就足以说明一切。

之后,他还有经营之能。在商家城中,他混得风生水起。短短几年工夫,就组建了这么一套强大的蛊虫搭配。

这样一个才能卓绝的人物,如果有德操,李闲还不觉得难以对付。

但方源却偏偏是个没有德行的人。

他面对狐魅儿,见死不救,根本不顾及美色。他蛮横无理,动不动就杀人,可以说漠视生命。

他打杀横眉暴君等三人,次次都把对手打成肉酱,可以看出他生性暴虐凶残。

他还欺骗薛三四,当众毁约,犹自得意。这就更可怕了。

说明他行事肆无忌惮,做事毫无下限,背叛信手拈来,根本没有一丝心灵上的愧疚。

李闲深深的明白:小兽王这样的人物,若把他放到正道,放到和平安稳的环境里,必定受到周围人的排挤、打压、驱逐、关押。但若要在魔道这种动荡不安,竞争残酷的环境中,他就是猛虎上山,蛟龙入海!只要运道不是太差,绝对能有巨大成就!

“这样的人物,若我提前发现,必定镇压打杀掉。但现在发现,太迟了,太迟了啊……”李闲心中发出深深的感慨叹息。

方源已经成长起来了,身边又有一位同伴相助。李闲虽然有四转高阶的修为,但他擅长的是做买卖,而不是战斗。

当然,能人之外有能人。

三叉山上,也有能斩杀方源的人物。那就是孔日天、龙青天等四大五转蛊师。

但李闲怎么可能指挥得动这样的巅峰人物?

而且,这四个人相互之间,都处于微妙的制衡状态。谁都不敢轻举妄动,都盯着三王传承,怎么可能把心思放在其他地方。

“哼,这个小妖女,居心不良。竟然想要撺掇我李闲。与这么一个棘手的魔道人物为敌,怎么可能?我不仅不会得罪他,反而要和他建立良好的合作关系。魔道中人,都是利益至上。”

“不过,小兽王杀了薛三四,也得罪了一个人物。飞天虎拜了一个干爹,乃是四转高阶的百岁童子。这些天,百岁童子一直在闭关炼蛊,这次传承开启,他一定会出现。到那时,很可能就要找黑白双煞的麻烦了。”

“呵呵呵。我暗中和小兽王交好,表面上则作壁上观。静看百岁童子和黑白双煞这两虎相争,然后再看结果,伺机而动!”

不管结果如何,李闲都不会有多少损失。

这就是李闲。

魔道中的大奸商,最擅长的就是占便宜。

哗哗哗……

亮金色的真元浪潮,在方源心念调动下,冲刷着空窍四壁。

哪怕是在三叉山上,他也一直勤修不辍。

他的实际修为只有四转初阶,原本只是淡金真元,但是在九眼酒虫的精良下,真元质量提高一个小境界,成为亮金真元。

亮金真元,给他的战斗带来了巨大的帮助。这也是他连续打爆三位四转中阶的重要原因之一。

“这些天来,我打出名声。怕我恨我忌惮我的人,一定很多很多了。但这又有什么关系呢?”

“呵呵,我走的是魔道,就该如此。况且这里也不是地球,个人的伟力凌驾于组织之上。”

环境不同,游戏的规则就不同。

若在地球上,即便走魔道,也要多少顾及品德。

因为地球上,个人实力差距不大。魔道中人也只有集众,才能成就雄图霸业!

“再过两天,就是三王传承开启之日。不会有人来找我麻烦的,趁着这个时间,先把横冲直撞蛊合炼出来罢。”

方源对自己的未来,有着精准的计划和安排。(未完待续)

\end{this_body}

