\newsection{他死了}    %第一百七十九节:他死了

\begin{this_body}

很快,方源就冷静下来。

第二空窍蛊只是一个希望,要想实现它,还得需要谋划和努力。

“炼制第二空窍蛊,不管成功与否,都会将大部分的仙元消耗掉。没有了仙元,就无法抗衡接下来的地灾天劫,也就是说,即便我成为福地的主人,这片福地也已经无法挽救,必须舍弃。”

“所以,第二空窍蛊就是我此行的最大收获。”

但整个事情,并非那么简单单纯。现在方源面对的局面十分复杂。

炼制第二空窍蛊,残留的仙元会急剧减少,福地大幅度衰弱,地灵的帮助也会越来越小。

同时,还要考虑到福地中的这些蛊师。如果他们发现了这处大殿,必将如同前世那般,一齐围攻,破坏炼蛊的行动。

方源双眼爆闪着精光,一番思索之后,主动地提出要求:“地灵,把第二空窍蛊的炼制秘方告诉我吧。”

地灵霸龟缓缓点头,低沉的声音,传入方源的耳膜:“那你可听好了。以下就是秘方的全部内容。”

“腐土血粉,地中藏花。玉骨成瓣,冰肌化茎,花心金舍利。星火烂漫,汇拢冰雪成原。其下有阳云升火如丹,其上有阴云落沙似金,中空增添兽影,直至电光霹雳,生兽力胎盘,便可集人窍……野草芳华,血气如海。三百岁为春,五百岁成秋。神机无限,扩游四野,添三更,再三更,三更得九。九为极,大功告成!”

秘方内容丰富。洋洋洒洒,有近万字。

方源越听,脸色越凝重。

炼这第二空窍蛊,需要上千个步骤。前期就涉及到上百种的材料,中期大量消耗四转、五转的蛊虫,越到后期,更是艰难,竟然要动用到另一枚六转仙蛊!

“不妙,炼制第二空窍蛊的难度。比我估算的还要高出许多。前世的情报,只谈及到秘方的前半部分,更夹杂着当事人对秘方的许多曲解。”

方源心头一沉,原本有五成的把握,听了秘方之后。只剩下三成左右。

“你不要过于担心,在这大殿中,早已经储备了大量的材料、蛊虫,专门用来炼制第二空窍蛊。”

地灵说着,整个大殿都泛起玄青之光。

灿烂的光辉中,大殿表面的浮雕纹刻,突出砖石。化为实体。

这个变化,让方源惊异:“这明显是一种储藏手段,我还是第一次见到。唉,上古的许多手段。都泯灭在光阴河流里了。”

一件件的材料,五花八门,让他看了眼花缭乱。一只只的蛊虫,从一转到五转。至少得有五千只,其中四转蛊就有六百多枚。五转蛊竟多达八十有余!

在外界十分稀罕的五转蛊虫,在这处大殿,竟然有近百只。而市价昂贵的四转蛊虫,在此处则似乎成了烂大街的货色。

“好多精品!还有许多我都没有见过的蛊……这些力道蛊虫,我如果得到,战力绝对要暴涨十倍!还有十多颗的黄金舍利蛊,八颗紫晶舍利蛊!我用了它们,顷刻之间就修成五转巅峰啊!”

方源看着这些四、五转的蛊虫,一时间都想放弃第二空窍蛊的炼制,直接吞并了这些蛊虫算了。

但这股冲动,很快就被他自己打消。

有地灵在一旁看守,这些蛊只能用于炼制第二空窍蛊,他是无法私自使用的。

再者,第二空窍蛊若是炼成,对未来的帮助极大,甚至对方源六转时,产生巨大的好处。

第二空窍蛊,是长线投资。尤其是到了六转之后,收益将膨胀到天!

“还有一只极为关键的蛊虫。”霸龟说着,双眼完全睁开,全力调动这座大殿。

大殿中青光暴涨,浓郁到逼人的地步。

铜鼎中的仙元,也开始一丝丝的剧烈消耗起来。

方源眯起双眼,但见这青光当中,禁锢着一只仙蛊。

它如一块圆形宝玉,宝玉通体橙黄,宝云中空,里面有一团紫色烟气。

这烟气时刻变化不停,忽而化作飞马,忽而变成仙鹤,有时又成筋斗云,有时又作白霹雳。

随着它的出现,一股浓烈的酒香,旋即弥漫在整个大厅。

方源呼吸几口,就感到一阵醉意,双眼朦胧,脑袋眩晕,连忙屏住呼吸。

“这就是六转的仙蛊――神游。”地灵介绍道。

神游蛊!

方源瞪大眼睛,目光一动都不动,凝视着这只蛊。

神游蛊十分的神秘,也十分的传奇,最早见于《人祖传》中。

《人祖传》是蛊道第一经典。初读起来是故事,其实寓意深刻,更记载着上古秘闻,里头有各种各样的蛊。有些蛊,直接描述,诸如智慧蛊、力量蛊等。而有些蛊,则含蓄地点出来,描写得很隐晦。需要读者,深入的挖掘和细细的研究。

在人祖传中,神游蛊最早出现于太日阳莽的身边。

太日阳莽喝下了,天底下四种极品美酒,肚中酒气郁结,就凝成神游蛊。

神游蛊可带人遨游天地,无处不到。但催动时,须得人酒醉神迷,同时到达的地点也无法控制。

太日阳莽吃了神游蛊的很多苦头,被神游蛊带到许多险境,好几次险死还生。

“神游蛊虽然是四大移动蛊之一,效能强大,但缺陷也太大,谁敢运用?就算是太日阳莽,最终也要将神游蛊,炼成定仙游蛊。难怪这个福地原主人,要将此蛊,转化为第二空窍蛊了。”

这定仙游蛊,也是四大移动仙蛊之一。能带人走到心中想去的地方,不管天涯海角。不过前提是,人的脑海中,必须要有这个地点的具体印象。若是这个地点发生了巨大改变,那么使用定仙游蛊,也会遭到失败。

方源细细一想。便理解了这位上古时代的力道蛊仙。

神游蛊,虽然贵为六转,但是每一次动用,都具有极大的风险。虽然高达六转,但运用价值很低。因此,太日阳莽将其转化为定仙游蛊,而这位上古力道蛊师,则想要利用它,炼成第二空窍蛊。

地灵霸龟将神游蛊。重新慎重地封印起来:“你打算什么时候开始炼制?”

“不急,先让我好好研究一下这个秘方。”方源当场盘坐下来,闭上双眼,开始静心冥思。

人是万物之灵,蛊是天地真精。

蛊师修为高深之后。就会明白一个道理,那就是蛊师用蛊,并不是单纯地将蛊虫当做一个工具。而是理解天地的一个途径。

蛊虫,是大道法则碎片的载体。炼蛊,并非随意胡乱蒙,而是基于对法则的理解。

一道秘方,不仅是炼蛊。更是研制者对于天地的体悟。

方源从这道秘方中,可以学习到这位福地原主人,上古力道蛊师的许多探索和感悟。和他自己两相对照,更加深了对大道的体悟。绝对是受益匪浅。

“蛊师天生只有一个空窍,要多出第二空窍,此乃货真价实的逆天而行。难怪要用到神游蛊了。”

良久,方源睁开双眼。对整个炼蛊过程,都有了深刻的理解。

“地灵。开始炼蛊吧!”他开口道。

“好。”地灵立即答应一声,将两样材料,一只蛊虫送到方源的面前。

第一份材料,乃是青泽腐土,采自腐毒沼泽的千丈地底,本身含有剧毒。哪怕是方源用手触摸一下,不出几个呼吸,整个手臂都要被毒烂。

第二份材料,是一捧血色的粉末,来头也是甚大。乃是八种太古荒兽之血,混合在一起,凝固起来后磨成粉末。

而那只蛊虫,却是常见,方源在青茅山时,就接触过。

乃是一株地藏花蛊。

地藏花,乃是储藏之蛊。花酒行者栽种在山洞内,储藏了数只蛊虫,最终都被方源得到。

方源在地灵的辅助下,将青泽腐土和八荒血粉混合在一起。

待混合成均匀的土壤之后,他试着将地藏花栽种下去。

地藏花蛊,一进入其中,便衰败而死。不管是青泽腐土中的剧毒,还是八荒血粉的凶猛血力,都是地藏花难以承受的。

不过,方源早料到此情形,也不气馁。

地灵又取出一株地藏花蛊,他信手种下。

那位上古时代的力道蛊仙,早已经料到每个步骤中出现的失败可能,因此备份充足。

接连几次失败之后,方源终于成功地种下地藏花蛊。

青泽腐土中的剧毒和八荒血粉的血力,形成一种微妙的平衡,从而让地藏花蛊发生异变。

这就是秘方的第一步――腐土血粉,地中藏花。

接下来,则是“玉骨成瓣,冰肌化茎,花心金舍利”,需要动用玉骨蛊,冰肌蛊,还有黄金舍利蛊。需要蛊师娴熟的炼蛊技艺。

第三步“星火烂漫,汇拢冰雪成原”,很容易误解成星火蛊和雪原蛊。若真如此,则会造成火力不足。实则两句要联合起来理解,真正的答案是要用星火燎原蛊,汇同雪原蛊,这样才能达到平衡。

方源按部就班,一一实现。

到了第四步“其下有阳云升火如丹,其上有阴云落沙似金”,更考验他一心多用之能。

方源先是运用阳云蛊,并用丹火蛊。然后又催动阴云蛊,又用金沙蛊。

但见阴阳二云,一上一下,混杂了前面步骤所得的烟气。橘黄色的圆球丹火,一颗颗的,从阳云中接连上升。黄金的沙硕,如细雨朦朦,从阴云中不断降落。

“阴上阳下,逆反平衡……到了此处,就是关键!去吧,白象兽影。”方源双目神光绽射,催动全力以赴蛊,头顶便现出白象虚影。

白象虚影一头撞入到阴阳二云的中央,受到丹火和金沙的磨搓。

轰的一声轻微爆响,白象虚影化作一团白光,在半空中成团旋转。

“再来,黑蟒兽影。”方源伸手一指,又牺牲掉一头兽力虚影。

黑蟒投入进去,很快就化为一条黑芒,绕着白光,两者不断纠缠。

“石龟兽影、骏马兽影。”从方源身上,再飞出两道兽影。

四大兽影,相互纠缠,形成彩光漩涡,却离成功还差一筹。

“怪哉!怎么还不融合?”方源感到奇怪,他遇到了第一个难关。

彩光漩涡在丹火和金沙的摩挲下,越来越小,各种气息始终不得融合。

眼看着就要失败,方源忽然灵光一闪:“等等,难道说……”

他开始操纵四种兽影。

白象兽影,踏实淳朴。黑蟒虚影,阴冷纠缠。石龟虚影,敦厚如山。骏马兽影,却是奔腾如飞。

这四种兽影,先前只是胡乱纠缠。如今在方源的操纵下,展现出各自的神韵和真意。

轰隆!

一声雷霆炸响,异变产生。

阴阳二云滚滚而动,阴云下沉,阳云上升,两者连成一体。

云气滚滚,混杂不休,其中有电闪雷鸣,不断炸响。

“原来如此。难怪需要力道蛊师,单纯拥有兽影,还不能成功。须得这蛊师体会到各自兽影的真意,才能开启融合。”方源松了一口气,暗生明悟。

咔嚓嚓……

电光爆闪,雷霆炸响。

声音宛若战鼓,越来越密集。终于到达极致,连成一片。

轰轰轰……

在连续不断地轰鸣声中,云气彻底融合,混成一色,忽又猛地爆炸。

狂风骤起,云烟尽数散去。

半空中,只留下一只蛊。

此蛊,高达五转,形如圆盘。表面粗糙,犹如杂草土胚。盘中央,有一头猛兽,马头象牙,龟身蛇尾。

“这就是兽力胎盘蛊了。”方源看到这蛊,心神一松,顿时一头栽倒,沉沉睡去。

他这番炼蛊,前后共耗去五天五夜。期间几乎不眠不休,一心多用,心神憔悴不堪,已经疲累到极致。

这一场觉,足足睡了一天一夜,方源苏醒过来。

他磨搓着兽力胎盘蛊,回想秘方:“其下有阳云升火如丹,其上有阴云落沙似金,中空增添兽影,直至电光霹雳,生兽力胎盘,便可集人窍……”

“接下来,便是集人窍!”想到这里,他站起身来,对地灵道:“霸龟,是时候了。”

霸龟也很干脆:“好,福地各处,你想去哪,我尽可传送。”

说着,方源的眼前,便浮现出一张大地图,以及种种画面,尽显福地风貌。

蛊师、毛民、犬兽等等,一目了然。

“就是此处。”方源搜索,目光一定,指了个地方。

下一刻,方源消失,出现在铁慕白的面前。

“是你。”铁慕白正在探索传承,看到方源突然出现,微微一惊。

但他旋即又镇定下来,刚想要说什么话,但方源却赶时间,信手一挥,幸运地打出岩鳄兽影。

啪。

铁慕白被岩鳄甩尾,脑袋像个西瓜,当场被抽爆――

他,死了。(未完待续)

\end{this_body}

