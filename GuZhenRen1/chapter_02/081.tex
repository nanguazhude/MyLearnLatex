\newsection{剑影蛊(两章 合一)}    %第八十一节:剑影蛊(两章 合一)

\begin{this_body}

%1
说时迟,那时快!

%2
方源身形如电,扑向汤青。

%3
呼啦!

%4
劲风骤起,汤青眼前一黑,还没来得及反应,方源一拳便捣中他的胸膛。

%5
刹那间,汤青只觉得一阵剧烈的痛疼骤然袭来,差点让他当场疼晕过去。

%6
耳畔风声呼呼,视野快速倒退,他整个人都被方源这蓄势一击,揍得飞出去。

%7
然后他落到地上,视野天旋地转,一会儿是草地,一会儿是演武场的棚顶。他擦着地面,一路翻滚,草皮都被他剧烈的摩擦掉,肥沃松软的黑土翻腾出来。

%8
他浑身都是草屑和黑泥,青草汁液、泥土气息以及鲜血的气味,混杂在一起,扑入他的鼻腔。

%9
他躺倒在地上,目光涣散,一路翻滚过来,令他浑身都散了架,身上各处都传来酸痛之感。

%10
但这些感觉,和他胸口处的剧痛相比,简直是毛毛雨!

%11
他低头往胸口一看,顿时倒抽一口冷气。

%12
只见自己的胸膛左侧,被硬生生打低两寸,皮开肉绽,惨白的肋骨都露出来,很显然是断掉了。

%13
大股大股的鲜血,不要钱似的往外喷涌。

%14
方源巨力在身,突袭得手,让汤青遭受重创。

%15
汤青瞪圆了双眼,惊骇之后,狂怒仇恨如火山岩浆,在他心中喷发。

%16
“这个小兔崽子,卑鄙无耻,偷袭我!让我重伤,我要杀了他。把他千刀万剐!”

%17
“死来!”就在这时,汤青的耳畔传来方源的低喝。

%18
方源连跨几步,赶到汤青面前,抬起一脚,照着汤青的双腿中间狠狠踩下。

%19
汤青只觉得眼前一花,下意识地眯起双眼。方源催动天蓬蛊,浑身罩着一层白光虚甲,光芒让他心尖抖颤。

%20
竟然是三转蛊师!

%21
方源先前用着敛息蛊,此时出手战斗,再不能掩盖一身三转气息。

%22
直把汤青骇得魂飞天外。这么年轻的三转蛊师?!

%23
他下意识地狂催防御蛊虫。顿时一股青风绕体。

%24
方源这一脚,蓄谋已久,本来能将汤青下体踩得稀巴烂。但被清风所阻,力道大大降低。

%25
“啊——!”汤青张开大嘴,神情扭曲至极,发出凄厉至极的惨叫。

%26
虽然有所防御,但毕竟是要害受到重击,他如遭电击,腰腹用力一弹。从地上坐了起来,伸出双臂下意识地护住自己的裆部。

%27
方源眼中冷芒一闪。左手捏拳,猛地直捣出去。

%28
这一拳,他用尽全力。

%29
风声呼啸!

%30
砰!

%31
笼罩着一层白光的拳头,重重地击中汤青的脸面。

%32
汤青的惨叫声戛然而止,他以更快的速度向后仰倒,后脑狠狠地砸在地面。

%33
一动不动,气息全无。

%34
致命一击!

%35
他整个鼻梁都被打进脸部里去,眼球爆出,凸出眼眶一大半。头骨在刹那间碎裂。

%36
鲜血缓缓流下,浸染身边的泥土和绿草。

%37
方源居高临下地俯视着,场面像是凝固了一般。

%38
足足几个呼吸之后,周围响起惊呼声。

%39
“死,死人了!”

%40
“有人被活生生的打死了!”

%41
观战的两三个蛊师,都瞠目结舌。就连魏央也不免涌起一丝异色。

%42
演武场中,虽然战斗频繁。但死亡并不多。

%43
一来演武双方都有手段,不敌也可以认输。二来有主持演武的蛊师能及时出手。三来战斗双方通常也比较克制,毕竟都在演武区混的,平日里抬头不见低头见的。虽然因利益而争。但不会真正闹出人命来。

%44
方源能杀了这汤青,有两个主要原因。

%45
第一是汤青刚刚闭关出来,对于战斗的敏锐大大降低。换做平时,他及时动用防御蛊,也不会落到这般下场。

%46
第二是方源相貌年轻,又用了敛息蛊遮掩气息,导致汤青心神放松,然后突然暴起偷袭,第一击就重创了汤青,随后两下兔起鹊落般快速,让主持演武场的蛊师都没有来得及反应,汤青就被他干掉了。

%47
主持演武场的蛊师,一路飞奔,赶了过来。

%48
但当他看到汤青凹下去的脸孔,还有从头骨裂纹处往外渗出的脑浆和血水,只得打消施救的想法。

%49
“年轻人,你下手太狠了!”他瞪了一眼方源,语气中有不满。

%50
刚刚的战斗虽然短暂,但他尽收眼底,方源掌控全局,原本可以留手,放过汤青一条性命,但他没有那么做。

%51
方源无所谓的耸耸肩:“一条人命罢了,有什么大惊小怪的。按照演武场的规矩,我把他打死了,他身上的东西都是我的战利品,对么?”

%52
主持演武场的蛊师不悦地哼了一声:“他身上的东西都是你的,不过藤讯蛊我们得回收。年轻人,我得提醒你,你对生命一点都不尊重,这样的想法很危险!”

%53
“实在是抱歉。”魏央带着一脸歉意,从身后走过来,“我的这位老弟,一直都在外面闯荡,第一次来这里参加演武。”

%54
那蛊师并没有认出魏央来,脸上流露出一丝厌恶之色:“哼,你们魔道蛊师就是这样,侵略成性,嗜血屠杀。算了,和你们说不通,把你的藤讯蛊拿来。”

%55
方源掏出藤讯蛊,主持蛊师修改了里面的记录,还给了方源。

%56
方源心神探入进去,原本零的胜场数,此时已经变为一。

%57
他又搜了汤青的尸体,得到四只蛊。三只二转,一只一转,都是普通的蛊,总价值大约两千元石。

%58
汤青的死很突然,以至于没让他来得及自爆蛊虫。

%59
只是他刚刚闭关出来,身上的元石只有二三十块。

%60
“方正老弟。以后最好尽量少下杀手。”出了演武场后,魏央劝说方源道。

%61
“虽然说在演武场中,生死勿论,但是……”魏央缓缓摇头,“没必要每一场都做生死拼杀。演武区说大也大,说小也小,总会碰到比你强的对手。大家抬头不见低头见,毕竟都生活在商家城里,得饶人处且饶人吧。”

%62
方源微微皱起眉头:“可是我饶了别人,别人未必饶我啊。我之前也碰到过魔道蛊师。二话不说就动手。你不动杀手,人家动杀手,我过往的经验告诉我,能下杀手就下杀手,否则夜长梦多,绝不要给敌人机会!也正因为如此,我才能活到现在。”

%63
魏央一阵哑然。

%64
他忽然想到,眼前的这个方源不正是以前的自己吗?下手狠辣,不信任任何人。追根究底是没有安全感。

%65
魔道蛊师是最没有安全感的。

%66
和正道蛊师不同,正道蛊师有家族依托。有族人帮衬,有稳定的资源供给,所以他们有安全感,能信任彼此。

%67
而魔道蛊师都是散修,没有家族势力支撑,就得自己搞元石,弄食料。很多时候吃了上顿没下顿,朝不保夕,能有安全感么?

%68
他们为了自己能生存下去。就只好铤而走险,杀人越货。魔道蛊师势单力薄,往往不敢去动正道人马,因此就把魔爪伸向彼此。

%69
这样一来,恶性循环。魔道蛊师之间就更不能信任,往往一见面就动手。

%70
他们更没有安全感,因此招数就更加狠辣。

%71
魏央是从魔道转了正道。丰富的经历让他对正魔两道的认知,比常人更加深刻。

%72
为什么正道荣昌,魔道只能被压在下风?

%73
就是这样的原因。

%74
方源出手狠辣,魏央十分能够理解。正是因为这份理解。导致他对方源心生怜惜之意。

%75
“他虽然修为达到了三转,身具战斗才华,但终究是个孩子啊。唉,看到他攻势这样凌厉,就知道他吃过多少苦头。现在想想,如果他手下留情,反而显得奇怪吧?”

%76
“他毫不留情,就和其他的魔道蛊师一样,刚刚参加演武都有这样的毛病。我以前不也这样?呵呵,算了,等时间一长,他就会慢慢转变了。这种过程是潜移默化的,不能强行催导。”

%77
想到这里,魏央便不再劝说,而是话锋一转,谈到刚刚的战斗。

%78
“方正老弟,你之前说过,用了不少蛊改造身躯,增加力量。你具体用了多少蛊?”

%79
“魏大哥既然问,小弟知无不言。一共用了三种蛊,黑白豕蛊,鳄力蛊,前几天刚买了一枚棕熊本力蛊,如今正在用着。”方源微笑道。

%80
“呵呵呵,黑白豕蛊给你增加了两猪之力,鳄力蛊带给你一鳄之力。但是老弟啊,你想过没有,若刚刚是一头山猪冲撞你的对手,会造成什么样的伤势?”魏央以一种启发的语气道。

%81
方源已知魏央的用意,索性配合他,脱口而出道:“若那人没有防御,肉体凡胎怎能抵得过野猪冲撞?必定被撞得腰折肚烂,惨不忍睹。”

%82
魏央脸上笑意更浓:“如果将那人的头颅放在一条鳄鱼的口中,鳄鱼一口咬下去,又会怎样?”

%83
“必定咬得稀巴烂,像脆皮的瓜果摔在地上。”方源答道。

%84
魏央紧接着又问:“老弟你有双猪之力,一鳄之力,还有你本身的气力。你第一拳打下去,却只能将对手的胸膛打个坑,断几根肋骨。你第三拳打在他的脸上,却只能把头骨打出裂缝,你说这又是为什么?”

%85
这次不待方源回答,魏央就继续说道:“拳击脚踩,只能发挥出人体的一部分力量。老弟你身上虽有双猪一鳄之力,但真正能发挥出来的有几成呢?力道修行的最大弊端,就在于此!人一身的气力,就像是个大水缸,里面装满了水,但真正作战时,用到的水只是所有水中一小部分。”

%86
“黑白豕蛊、鳄力蛊、棕熊本力蛊,这些蛊能永久增加力量,都是价格不菲的蛊。你投入这么多的钱,结果发挥出来的效果。可能十分之一都不到。如果将这些钱财投入到其他方面,无疑更见成效啊。”

%87
“原来魏大哥说这么多话,还是想劝我放弃力道。”方源淡淡一笑,一副这才听明白的样子。

%88
“这个道理我也懂,人的身体和野兽的身体构造不同,的确难以发挥出它们的力量。但是用蛊存乎一心,我也听说很多力道蛊师,能使出兽力虚影,发挥出蛊虫的全部力量。”

%89
“呵呵呵,的确是有这么一回事。当年我在演武场时。也遇到过不少的力道蛊师。很多人用蛊纯熟,在偶尔的攻击中,能出现兽力虚影,发挥出蛊的全部力量,十分有威胁。但你其实有所不知,这样的攻击次数,实在太少了。而且,攻击方式也很局限。只有特定的招式,才能发出兽力虚影。很容易就能躲避。”魏央苦口婆心地继续劝道。

%90
方源沉默了片刻,心想正是因为如此。才需要那只传奇蛊啊。

%91
口中则道:“魏大哥,你的好意我领了。但是力道是我的选择,我还是想亲自验证一下。”

%92
“唉,那你可要好好验证。”魏央叹了一口气,方源如此“执迷不悟”,他也不好继续强劝。

%93
好在方源已经参加演武,魏央觉得:接下来的事实会打动方源的心。

%94
方源要在商家城生活两三年之久,时间还有的是,魏央也不着急。

%95
一转眼。已是半个月后。

%96
赌石坊中,掌柜弯着腰,站在方源的身侧,陪着笑脸。

%97
“魏大哥,你也来挑几颗石头玩玩?说不定能赌到好东西呢。今天我请客!”方源笑着道。

%98
魏央就站在方源的身旁,他摇摇头:“今天是汤雄强行挑战你的日子,想不到你杀的那个汤青。竟然是此人的弟弟。方正老弟你还是谨慎为好,汤雄为了报仇,特意从第四内城降到第五内城。他小有威名,能爆发出三熊之力。来势汹汹啊。”

%99
他虽然这么说,神情却不显得担忧,反而隐藏着期待。

%100
这些天,方源又赢了一场演武,这已经是第三场了。

%101
魏央希望这个汤雄能够“打醒”方源,让他舍弃力道,转换流派。

%102
“兵来将挡水来土掩,有什么好怕的。魏大哥不赌,我就出手了。我已经看到了几块好石头了。”方源眼中放着亮光,伸手点选了几枚顽石。

%103
掌柜的连忙指挥伙计,小心翼翼地将选中的顽石取下来。

%104
方源忽然轻咦一声,指着柜台一脚:“这块垫脚的石头,好像也是星辰石?”

%105
掌柜的楞了一下,连忙笑道:“贵客您真是慧眼如炬啊,这块石头还是小的几年前垫上的。那时柜台脚被一赌徒踢坏了,小的就顺手拿了一块四四方方的星辰石,正巧垫上去了。”

%106
方源皱起眉头:“石头是用来赌的,怎么能做垫石?简直是宝珠蒙尘!今天最后一块就选它吧。”

%107
“是是是,贵客教训的是!”掌柜的点头哈腰,心中却不以为然。

%108
星辰石品相很重要,多以箭枝状、流星状为佳。这块石头一看就是坏石,以前摆在柜台上无人问津,所以被他塞到下面,平衡柜台。

%109
几位伙计一起合力,将柜台脚下的星辰石抽出来,和其他石头一块,送到解石台。

%110
解石台上,几位年轻的解石小师傅,正在为一位中年蛊师解石,动作缓慢细致。

%111
魏央对方源解石,其实很不赞赏。看到他如此选石,更是暗暗摇头。就算是他不赌石的人,也知道品相的重要性。方源最后一块石头,简直是瞎选。连他都有些看不过去,难怪方源这些天泡在赌石区,元石花下去,却不见有所收获。

%112
就算偶尔赌出一些蛊,都是一二转的,或者干脆是残骸尸壳。

%113
在魏央心中,方源如此赌石纯粹是浪费钱。但他没有劝说,方源元石越少,就越需要依靠商家,这是他喜闻乐见的事情。

%114
方源心中激荡,脸色平静,目光中饱含期待地看着解石台。

%115
这些天来,他在赌石区各家赌石坊流连忘返,又故意选取一些坏石,给人留下印象。

%116
虽然他已经暗中克制,小打小闹。但赌石就像是无底深渊,短短一个月不到,就吞掉他近十万的元石。

%117
不过想到即将得到的传奇蛊,别说是十万,就算是五十万元石也是值得的。

%118
这只赋有传奇色彩的蛊虫,虽然只是三转,但效用奇特,已经绝迹。

%119
很有可能,这就是全天下最后一只了。

%120
魏央皱起眉头:“要注意时间,待会就要赶去演武场。掌柜的,解石能快些么?”

%121
“当然,当然。”掌柜自然认得魏央这位外姓当权家老,连忙点头。

%122
他跑到解石台上,伸手连挥,对那些解石小师傅道:“去去去。”

%123
把这些解石小师傅,都赶到了一旁去。

%124
“我的石头啊……”那个中年蛊师惨叫一声。

%125
解石是个精细活儿,被掌柜的一干扰,小师傅们手忙脚乱。令中年蛊师选的好几颗石头都损坏了。

%126
“李然,你别叫。你出的元石。本店如数奉还给你。”掌柜的喝道。

%127
那中年蛊师胡子拉碴,瞪着双眼,语气很不忿:“万一这顽石里面有蛊呢?”

%128
掌柜不屑地嗤笑,对中年蛊师挥手:“得了吧,李然。你赌石赌了这么多年,都是赌的下等石头,出过什么好东西的?你别闹,再闹连赔偿都没有!”

%129
“我呸,店大欺客。狗眼看人低,总有一天我李然会出人头地的!”中年蛊师嘴里嘟囔着,语气愤愤,终究还是没有叫嚷。

%130
“李然……”这个名字唤起了方源的某段记忆。

%131
他的眼中,不由地流露出一丝古怪之色。

%132
貌似这个李然,就是赌出传奇蛊的那个人呐。当然,也可能是同名同姓……

%133
但不管如何。李然出人头地的机遇,已经被方源抢走了。

%134
五位老师傅接连登场,解石开始了。

%135
年轻的小师傅们无语地看着,这些前辈卖弄着手段。明明很多顽石可以用很简单。耗费真元很少的法子去解,但偏偏老师傅们非得动用全力,弄得真元大损,步骤繁杂,却只能提高一丝成功率。

%136
十几块顽石,包含有杂等、下等、中等,被老师傅们迅速解开。

%137
“出蛊虫了,是剑影蛊!”

%138
“三转的剑影蛊,难得啊……”

%139
“活蛊,绝对的活蛊。恭喜贵客了。”

%140
老师傅们连连向方源拱手,心中均松了一口气。

%141
方源这些天连连解石,没有像样的收获。搞的这些老师傅心中也惴惴不安。

%142
掌柜的屁颠屁颠地跑过来,一脸喜色:“贵客您大赚了!剑影蛊能卖到三万两千块元石,贵客您只花了八千块元石买的石头啊。”

%143
周围人无不羡慕地看向方源,那个李然更是撇撇嘴,语气酸涩的嘟囔:“哼,走了狗屎运了。”

%144
“行啊,老弟,剑影蛊和我的刀光蛊齐名。建议你不要卖出去,留着自己用吧。”魏央也恭喜道。

%145
“呵呵呵,这叫做苦尽甘来,我就说我的运气不会那么差的。”方源笑道,“魏大哥,今天比试结束,我请你吃酒,咱们庆祝一下。”

%146
魏央点点头,也不客气。他却不知方源表面谈笑风生,其实心中却掀起惊涛骇浪!

%147
剑影蛊只是一个意外,并非是他的目标。

%148
那块垫脚的星辰石,切开来后,根本没有蛊,完全是实心的废石!

%149
怎么会这样?

%150
那只传奇的蛊,哪里去了?!

%151
一时间,方源心中思绪起伏,心潮汹涌激荡。

%152
无数的疑惑,充斥他的脑海。

%153
怎么会这样?!

%154
传奇蛊虫难道不是藏在这块星辰石中的吗?

%155
如果不是这块顽石中,又会在哪里呢?也许,不是这块石头,或者不是这家赌石坊?

%156
或许这传闻根本有误?我又该去哪里,寻找那只传奇蛊呢?

%157
事情的发展,大大超出了方源的意料。原本以为唾手可得的传奇蛊虫,如今已经不翼而飞,而意外收获的剑影蛊,根本不能弥补方源这些天来的元石投入。

%158
“如果没有传奇蛊,我的这些努力都白费了。真是该死,怎么会如此?传闻或许有误,但能流传如此之广,必定不是空穴来风。尤其是这细节如此详尽,相关传闻无一差异,可信度很大。但偏偏……”

%159
方源暗暗咬牙。

%160
没有这只传奇蛊,他的力道修行,就宛若空中楼阁,水中花月。

%161
“难道,真的要我改变修行方向?若是这样一来,三王传承也会受到很大影响!”

%162
解石的结果,很大程度上打乱了方源的大计。但偏偏他不不知道问题,究竟出自于哪里。

%163
曾经引为依据的传闻,此刻也蒙上了一层厚厚的神秘迷雾。

%164
“时间差不多了,我们该去演武区了。”这时,魏央开口提醒道。

%165
(ps:嗯,本章是我合炼的两更连体蛊。至于那只传奇的蛊,哪里去了呢?计算失误了,捂脸……传奇蛊,传奇蛊,本书中肯定会出现的……我真心不是有意的,求大家轻吐槽。我已经蹲在墙角,开始默默地捡节操。)(未完待续。如果您喜欢这部作品,欢迎您来起点()投推荐票、月票,您的支持,就是我最大的动力。)

\end{this_body}

