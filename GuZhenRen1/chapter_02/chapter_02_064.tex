\newsection{接见}    %第六十四节:接见

\begin{this_body}

%1
商燕飞面含微笑,走入屋内。

%2
小蝶慌忙行礼,商心慈却坐在桌旁,一动不动,甚至连瞳眸都未转动。

%3
商燕飞也坐到桌旁,声音轻缓温和至极:“慈儿,现在感觉怎么样?”

%4
商心慈并非刁蛮性格,她站起身来,退后几步,轻轻万福一礼:“商家族长大人不用过多挂怀,小女子只是情绪过于激动才昏倒的,如今已经恢复,耳清目明,没有事了。”

%5
商燕飞连忙摆手:“呵呵呵,慈儿你没事就好,坐,坐下来说罢。”

%6
商心慈只称呼他为商家族长,如此刻意疏远的语气,让他心中一疼。

%7
商心慈重新坐下,小蝶反应过来,给商燕飞倒茶。

%8
“说说吧,这些年你是怎么过的?”商燕飞温柔地凝视着商心慈。

%9
“过的还行。”商心慈回答简略,明显不想深谈。

%10
倒是小蝶抱怨道:“小姐从小到大,都受族人排挤。夫人去了后,他们更加变本加厉,还想吞并我们家的财产,太可恶了。老爷,你可得为小姐做主啊!”

%11
“小蝶,好好倒你的茶。”商心慈白了小蝶一眼。

%12
小蝶顿时闭嘴,不再说话。

%13
商燕飞吃了一瘪,丝毫不恼,心中更添怜爱之意。

%14
他笑了笑:“对了,你们是怎么过来的?从张家到这里,可是很长的一段路呢。”

%15
“老爷,你差点就看不到小姐了。我们这一路。可是险死还生。整个商队数千人,最后只剩下我们四个。幸亏我们遇到了黑土和白云两位大人相助。要不然……”小蝶又憋不住,脱口而出。

%16
“小蝶!”商心慈狠狠剐了小蝶一眼。

%17
小蝶只好又闭上嘴巴。

%18
商燕飞一边将“黑土白云”这两个名字暗记在心,一边微笑道:“接下来你们就住在这里,这里很安全,闲暇时可以去庭院里走走,去外面街上逛一逛。你们刚来到这里,还不熟悉,容易迷路。我会指派你一个丫鬟过来。她熟悉这里的环境。我先走了,你们好好修行。”

%19
商燕飞看得出商燕飞还需要时间,来调整自己。

%20
此时要给她空间和时间来适应。

%21
“老爷人真好,虽然是商家族长,却这么和善。小姐,他毕竟是你的亲生父亲啊……”看着商燕飞离去的背影,小蝶劝道。

%22
“我知道。从见他第一面时就明白了娘亲的良苦用心。唉,娘临死前嘱咐我来商量山,却不明说。因为她也不敢肯定,他是否会认我这个女儿……虽然他现在认了我,但我心中却不是滋味。这一切来得太突然了……”

%23
“小姐,不管如何。不管你去哪里,小蝶都在你身边。”小蝶坐到桌前,捉住商心慈的手,给她鼓励。

%24
“嗯。”商心慈感动的点点头,将另一只手搭在小蝶的手上。

%25
“当然。如果小姐能留下来,那就更好了。要知道。这可是商家啊!天呐,荣华富贵唾手可得啊小姐。张家就算给商家提鞋,都不配啊,小姐!”小蝶做了个鬼脸,叫道。

%26
“你呀。”商心慈又好气又好笑,拿小蝶很无奈。

%27
小蝶咯咯的笑起来,笑声渐渐感染了商心慈,化解了一丝她心中的积郁。

%28
商燕飞走出房屋,脸上的微笑立即消失殆尽。

%29
他黑袍血发,面容英俊,双眼闪烁着阵阵寒芒,习惯抿紧的唇角泄露出他坚定果敢的性格。

%30
他是商燕飞,这一代的商家族长!

%31
他心狠手辣,为了族长之位,已将两个兄弟,一个姐妹逼得自杀。

%32
他杀伐果断,刚刚上位时,一个边远山寨自以为天高皇帝远,袭击了商家的商队。他力排众议,耗费巨资远征,将山寨一干老小尽数屠戮。将所有乞降的俘虏斩杀,头颅堆成小山,摆在当初反对远征的家老们的面前。

%33
他手腕强悍,上位以来,巩固权势,提拔亲信,打压异己。一下子确立了十多位外姓家老。仅仅只用了三年,整个商家高层就只剩下他一个人的声音。

%34
他眼光卓绝,有经营才华。在位这么多年,商家的商队规模,扩大了三倍有余。上百个家族表示依附商家,成为一股庞大的隐形势力。

%35
他任人唯贤,就算是对自己的子女也不例外。原先商家少主之位,有十五个。他上任之后,直接缩减了三分之一。

%36
更难得的是,他天赋异禀,甲等资质,执掌家族的同时,修为也在不断提升,并驾前驱,惹人羡慕嫉恨。

%37
这才是商燕飞,站在南疆凡俗顶端的男人。

%38
“属下见过族长大人。”一位少女蛊师,向他躬身行礼。

%39
“田蓝,从今天开始,你就要尽心竭力地服侍慈儿小姐。明白吗?”商燕飞冷漠地道。

%40
“属下明白。”少女点头。

%41
“到了慈儿小姐身边,多留意,多打听。慈儿小姐可能不太爱说话,但她身边的丫鬟却是心直口快,她就是你的突破点。我要求你打探到她们是究竟怎么来到商量山的。”

%42
“是,属下领命。”

%43
“嗯,你这就去吧。”

%44
“属下告退。”

%45
田蓝是商燕飞收养的孤儿,忠心耿耿,用起来放心。办事能力也强,三天后,她圆满地完成了任务。

%46
“黑土白云?魔道蛊师……”商燕飞摩挲着下巴。“如果这事情是真的,我倒要好好的感谢他们俩个。不过为了预防其他的可能,还是再调查一番的好。”

%47
念及于此,商燕飞唤来魏央:“你去找一找这两个人,一个名字叫做黑土,另一个叫做白云。一男一女,魔道蛊师。容貌特征是……”

%48
第三内城。

%49
“已经过去三天了。那两个人一直都住在楠秋苑吗?”

%50
“是的,少主,从那天之后,他们就深居浅出。最多出来买些牛羊的奶水。下属估摸着这些奶水应该是用来喂养那些蛊的。”

%51
“可恶……”商睚眦咬了咬牙,目光如刀,恨不得把方白二人捏碎。

%52
眼看着考评会就要到了,十位少主当中,他已经是最后一名。如果再没有一点成绩,他铁定就要被淘汰了。

%53
商燕飞子女众多,但商家少主之位只有十个。

%54
普通子女和少主之间,商家的待遇可谓天差地别。少主万众瞩目,能掌管商量城的一个产业,风光无限,油水丰厚。少族长更是不得了。

%55
但如果只是普通子女。那待遇几乎和一般的族人没有什么两样。

%56
商睚眦已经体会到少主的权利滋味,要让他过回普通子女的生活,还不如让他去死!

%57
所以,那道传承必须得到。

%58
不仅要得到,还得把卖价压低。

%59
只有压得越低,他的成绩就越好。

%60
但偏偏这方白二人。却是倔强得如石头一般,不肯向他低头。

%61
若是在第四内城或者第五内城,他还可以耍点偏门的小手段,强迫他们低头。但这两人却住在第三内城,给商睚眦一百个胆子。他也不敢在这里动粗。

%62
“这样下去可不成,他们能耗。我可耗不起。看来还得我亲自出马了,哼!”

%63
商睚眦越想越坐不住,终于带着一帮心腹,来到楠秋苑。

%64
楠秋苑乃是一处园林,供贵客居住。

%65
在第三内城,没有客栈,都是这样的小型园林。

%66
在这里住一天,需要三十块元石。十天就是三百块,一个月下来,就是近千枚元石。

%67
楠秋苑的价格,在第三内城,还算是便宜的。有些大型园林,一天就是上百块元石。而有些特级园林,你有元石也住不到,必须拥有商家令牌。

%68
在商家城,可谓寸土寸金。尤其是在第三内城,物价很高。

%69
商睚眦带着人造访的时候,方白二人正在荷塘边的亭中下棋。

%70
商睚眦哼哼两声,不阴不阳地道:“二位真是好兴致,这两天怎么没去其他商铺里问问呢?”

%71
“有什么好问的。想买的人,自然会找上门来,你说是么?”方源淡淡一笑。

%72
“你!”商睚眦顿时大怒,方源的态度让他很不爽。

%73
他咬了咬牙,艰难地将心中的情绪压下,昂首傲慢地道:“我是看你们俩也不容易,算了,就五十万元石吧。你们开心了?”

%74
“五十万元石?呵呵,不卖。最低价六十五万。”方源看了商睚眦一眼,便将目光重新转移到棋盘上。

%75
商睚眦双眼眯成一条缝,几步走到方源身边,低声威胁道:“你们这两个魔道蛊师,还在装!我早就查过你们的底细了,这道传承是从百家手里抢来的吧?据说你们还把百家的两个少族长都杀死?奉劝你们见好就收,赶紧脱手。告诉你们,这几天有一只铁家队伍,已经到达了商量山,专门四处打探你们的踪迹。你说,如果我把你们在这里的消息告诉他们,会怎样?”

%76
“哦?那你就告密去吧。”方源哈哈一笑。

%77
“你!”商睚眦手指着方源,再也忍耐不住,“我告诉你,五十万元石已经可以显示我的诚意。你除了卖给我,还能有其他选择吗?没有了!这是我的地盘,你最好看清形势。”

%78
就在此时,一道光从天而降,化为一名蛊师。

%79
这蛊师扫视一周,然后微微弯腰行礼,对方白二人道:“黑土、白云二位阁下,鄙人魏央,奉族长之令,敬请二位来第二内城一叙。”

%80
“什,什么?父亲大人点名要接见他们两个?!”商睚眦只感觉一道晴天霹雳劈中自己,眼珠子瞪大,流露出惊恐之色。

%81
这下糟糕了!

\end{this_body}

