\newsection{才貌双全,心慈风采}    %第一百二十八节:才貌双全,心慈风采

\begin{this_body}

“小弟在这里,敬大哥您一杯。”商一帆站起身来,弯腰鞠躬,高举起酒杯,脸上堆满了笑容。

此刻的大堂中,灯火辉煌,歌舞在表演,丝竹声流淌。众人觥筹交错,氛围正烈。

“好。”商囚牛坐在主位上,向商一帆举起酒杯,一仰脖子饮下这杯酒。

一旁的谋士为其代言:“一帆少爷,你能够参加我家囚牛少主的晚宴,毫无疑问这是明智之举。你放心,我家少主绝不会亏待你的。”

“囚牛大哥一直都是小弟的表率和榜样。其实,小弟一直想和囚牛大哥您多多亲近,多多学习的。”商一帆笑着道。

商囚牛乃是当今少主当中,第一大派系,正好商一帆的母族也和其瓜葛甚深。商心慈若也选择商囚牛这边,必定会遭到商一帆排挤。

不过,商囚牛事先也主动地向商心慈发去请帖。

商心慈若来,他肯定会欢迎。

为什么?

就是因为商心慈身边,有方白两大四转战力。

这是其他所有的少主,都不具备的。很多人都对此表示羡慕和嫉妒。

“少主,属下打探到了,那商心慈去了……”这时,有蛊师弯腰进场,对商囚牛一番耳语。

商囚牛听着,目光微微的暗淡下去。

商一帆一直在小心翼翼地察言观色,心中不免琢磨:“看来商心慈并没有选择这边,难道是商蒲牢那派?”

商囚牛面皮上已经历练出了火候。喜怒不形于色,神态上没有变化。让商一帆只能猜测,看不出端倪。

但不多时,商一帆也接到消息。原来商心慈,已经拿着请帖,参加商嘲风的节日晚宴了。

这就意味着,她选择了商嘲风一派。

甫一得到这个消息,商一帆不禁私下窃喜:“商嘲风、商负屃虽然是第三大派,但始终被压着。怎么可能是囚牛大哥的对手?我夺得此次少主之位,又多了几分胜算!”

但一旁的张老总管却摇摇头,脸色有些沉重:“少爷,如今囚牛少主和蒲牢少主两派斗得难分难解。少爷想要借助囚牛少主之力,来对付商心慈,恐怕可能不大。”

“这……”

“这样激烈的时刻,囚牛少主是不可能再另外树敌。商心慈这次的选择。有一些巧妙,避开了漩涡,暂时作壁上观了。等到两大派暂时地分出高下,少主之位的竞争也已经结束了。”

“商心慈有两大四转战力,不论投到哪一派,都会被吸纳招揽。引为盟友。商嘲风少主一定也会扶持她。”

张老总管冷静地分析局势,鞭辟入里。

“这么说来,我就算加入了商囚牛这一派,也还对付不了她商心慈?”商一帆脸色难看起来。

“只能说,不能借助商囚牛的力量。”张老总管摸摸自己花白的胡须。“不过,她商心慈的问题也不少。最主要的。还是缺少得力的人手。我听说她最近,一直在试图招揽周全。哼,这怎么可能?”

张总管嗤笑一声,继续道:“周全曾是一族之长,心高气傲,才能十倍于我。怎么可能去依附她这样乳臭未干的黄毛丫头?我已经暗中出手,制造了大量的流言蜚语。周全极好脸面,已经堵死了他的路。商心慈原先只是想暗中招揽周全,这次招揽不到,将大失威信,出师不利,甚至成为商家城的一个笑柄。呵呵呵……”

说到最后,张总管阴笑连连。

商嘲风的书房,布置简单,宽大的石头方桌,还有高背石椅,带着粗犷、坚厚的气息。

在商一帆和张老总管暗中交流的同时,商嘲风、商心慈以及方源、白凝冰四人,也在商量着事情。

谈话已经进行了一段时间了。

说实话,商心慈主动投靠自己,这让商嘲风有些意外。

他原以为,商心慈会投靠商囚牛,或者商蒲牢。自己这一派,只是第三派系,目前蛰伏着,积攒实力。平时的时候,十分低调,很多问题和矛盾,都选择退让。没有想到,能得到商心慈的青睐。

商心慈本身修为只有两转,资质更加不行,在商家城的势力基础也几乎为零。

但她却有方源、白凝冰投靠。

这可是两大四转战力,就算是当初商燕飞在少主的时候,也没有这样的干将。

能得两位如此虎将,没有一个少主不对此眼热嫉妒的。

商嘲风自然也在私底下,暗中多次感慨商心慈的好运道。

现在,只要吸纳了商心慈,就能间接地得到方白二人的助力。这对于商嘲风来讲,是不能拒绝的诱惑。

书房中氛围融洽。

在刚刚的谈话中,商嘲风已经诚恳地表达出自己,将全力支持商心慈上位的心意。

“心慈妹妹,你靠着前段时间收购蛊虫转卖,已经赚到了三十万元石。可以说是独占鳌头,但是单靠这三十万还是远远不够的。接下来,你有什么计划和打算呢?”

商嘲风说完,又补充一句:“你说出来,我只要力所能及,会给你最大限度的帮助。”

商心慈和方源对视一眼,方源向她微微点头。

商心慈便坦白道:“不瞒嘲风哥哥,我是想做情报生意。”

“情报生意?”商嘲风不禁扬起眉头。

“是的。”商心慈便详细说出自己的计划。

哪知商嘲风听闻后,脸色一变,摇头道:“你要做演武场的情报生意?不妥,不妥。”

他一连说了两个“不妥”,很不赞同商心慈的计划。

“哦,有什么不妥之处。还请嘲风哥哥指教。”商心慈语气诚恳地道。

“心慈,我劝你改掉这个计划罢。你初来乍到。不太清楚,也是情有可原。演武场这块,还是不要乱碰的好。”

商嘲风叹了一口气,继续道:“商家城的演武场,和南疆其他所有家族的演武场不同。在这里,魔道蛊师可以成为我商家的外姓家老。这是我们商家独有的政策,这些年来,吸收了许多人才。当今商家高层中五大家老。就有三位是外姓家老。”

“但是,这项政策有利有弊。利越大,弊端风险也越高。外姓家老,乃是商家的高层,一旦引狼入室,对我商家也是巨大的祸端。魔道蛊师,毕竟是魔道蛊师。就算是改邪归正,其忠心也要商榷。同时,魔道蛊师之外,还有那些居心叵测的名门正道。所以,商家的演武场,都是历代商家高层最重视的地方。”

“曾经。有一个少主,突发奇想,想在演武场动刀,开设赌场。结果赌场刚开了两天,就赚了五十万的元石。但到了第三天。赌场被查封,少主也被贬。流放出去。演武场不能乱动,这个例子就是最好的警示啊。”

商嘲风说出了一个秘辛。

这个秘辛,是方源、白凝冰,以及商心慈都不知道的。

方源虽然有重生的记忆,但涉及商家内政,这般细微隐秘的历史事件,发生的又突然,结束的又迅速,不知道也不奇怪。

演武场,是个禁区。当年的一个少主,碰了这个地方,丢了少主之位不说,还被家族流放。

现在,商心慈还不是少主,却在打演武场的主意。这在商嘲风看来,实在是无知者无畏。

“心慈啊,你的计划另辟蹊径,设想的很好。如果做成,必定是日进斗金。但这个可能性太小了,不要拿自己的前途开玩笑。我掌管斗蛊场,你十三哥掌管拍卖场。你完全可以着手这些方面,有我们助你一臂之力,必定能和那商一帆一较高下的。”商嘲风温和地劝说道。

商心慈沉默。

白凝冰皱起眉头。

方源嘴角微微带笑,张口欲言,但目光瞄了一下商心慈后,心中主意一变,将嘴里的话又咽到口中去。

一时间,书房中氛围变得凝重起来。

半晌后,商心慈经过一阵剧烈的思考,忽然展颜一笑:“嘲风哥哥,对于演武场这件事情,我和你的观点不同。此事大有可为!”

“哦?”商嘲风皱起眉头,双目盯住商心慈。

商心慈感受到商嘲风目光中的压力,反而笑魇如花,展现出自信的风采。

只听她继续道:“前个少主开设赌场失败,简直就是飞蛾扑火,自寻死路。被家族流放,还算我们商家仁慈。为什么?”

书房中,只剩下商心慈的温润的声音在侃侃而谈。

“就像嘲风哥哥刚刚说的那样,我们商家的演武场,是重中之重,历代高层都极度警戒的地方。在演武场中开设赌场,为了巨额的利益,极容易产生幕后黑手,暗中操纵演武的结果。这对于商家来讲,是对外姓家老政策的巨大破坏。赌场看似利润很高,对于商家的演武场来讲,却是一只巨大的蛀虫。牺牲商家的利益,来喂饱自己。必须要除去,才能使得演武场继续茁长成长。”

商嘲风不断颔首,商心慈的话鞭辟入里,本身更有一种气质,让人不由自主地去信服。

“但是我做情报生意,却和赌场的性质完全截然相反。我知道,商家的最大情报组织风雨楼,一直在调查演武场的这些魔道蛊师。必须调查清楚,才能确认他们是否居心叵测。但这些调查,一直都在暗中进行,从不明目张胆。为什么?”

“呵呵。这就是因为,大多数的魔道蛊师心中,都充满了不信任,怀疑,甚至是疑神疑鬼。他们常年生活在朝不保夕,命悬一线的糟糕情境里,不能有一丝的放松,心理压力太大。到了商家城,还要被调查,他们肯定会反感、厌恶。商家为了保证吸收到更多更好的人才,自然不能去正大光明的调查。但是不调查,又不可能,更不放心。”

“在这种情况下,我做情报生意。调查每个蛊师的大体实力,蛊虫,战绩,出售贩卖。同时请权威、高手,预估重要场次的战斗结果,以及搞出许多排名。这对于演武场的蛊师们来讲,也是极为重要的情报。很多蛊师奔着外姓家老的位置,更希望能多多的宣传自己,也希望能在排名中看到自己的名字。”

“这种情报,当然不会涉及隐私,只针对他们表现出来的实力,进行总结和比较。这样一来,对于演武场的魔道蛊师而言,具有相当大的帮助。同时对于商家,更能加强对演武场的掌控。我想,父亲大人一定暗中开心不已,怎么会取缔这样的事情呢?”

商心慈说完,微笑着看着商嘲风。

这一刻的她,双目炯炯发亮,智慧而又自信,风采无限。

而白凝冰的眉头,已经舒展开来。

“就是这样……”方源暗笑。

商嘲风微微张开嘴巴,看着商心慈,神情微怔。

几个呼吸后,他反应过来,忍不住鼓掌拍手,赞道:“说的好,分析的太精彩了。心慈妹妹,你真是眼光独到,蕙质兰心,令我不得不佩服。这份生意,我一定全力助你。你现在缺少人手,我这就给你调人!要多少,有多少!”

商心慈却微微摇头:“谢谢嘲风哥哥的好意,人手方面我们已经有了人选。”

商嘲风表面上要出人,帮助商心慈,实际上,也有自己安插人手,控制这个情报生意,同时对商心慈加强掌控的用意。

但商心慈瞬间就看出来,委婉地拒绝了商嘲风。

不是所有的帮助,都是好意的。

但商嘲风却不肯善罢甘休,之前避之不及的情报生意,现在在他心中,已经转变成一个丰富的大宝藏。

他继续劝道:“心慈妹妹,我知道你最近一直在招揽周全,但是此人恃才傲物。当初,商囚牛都亲自延请过他,都被他拒绝,甚至斥责说:‘小辈乳臭未干,安敢招吾百年之人?’。”

周全如今已经有一百多岁,不肯寄人篱下。尤其是给小辈当下属。

商嘲风对于商心慈招揽周全一事,也很不看好。

周全的能力是有的,得到公认,但是太高傲了。

“退一万步讲,就算你招揽到了周全。一个人能顶什么用呢?你还是没有中下层的属下。这些人要有一定的才干,最重要的是要对你忠心耿耿。这样的势力,要构架出来,需要时间,漫长的时间。你没有可以放心的属下,就算是做了这个情报生意,也很有可能被商一帆破坏,甚至盗取成功的果实。”商嘲风虽然有他的死心,这席话,说得极有道理。

商心慈听了,也不禁皱起好看的弯眉。

“这点……嘲风少主无须担忧。我已有打算了。”这时,方源忽然开口。(未完待续。如果您喜欢这部作品,欢迎您来起点投推荐票、月票,您的支持,就是我最大的动力。手机用户请到阅读。)

------------

\end{this_body}

