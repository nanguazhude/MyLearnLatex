\newsection{浅滩休整借蛊虫}    %第二节:浅滩休整借蛊虫

\begin{this_body}

五天前,方源在青茅山重生。

当时他在血罩内白凝冰携手,屠杀中倒也匆匆收缴了些战利品。

但这些蛊虫,都带着伤。这五天来,方源一直在江山漂流,它们缺乏食物,都陆续死掉了。

治疗方面的蛊虫,方源一直都缺乏,没有寻过称心如意的。

“可惜我的蛊虫,都在自爆中死去了。要不然也至于受伤如此……”白凝冰仰天长叹。

但方源却笑了笑道:“不要太悲观。并非凡事都需要蛊虫才能解决的。”

“嗯?”白凝冰疑惑地望来。

只见方源暗催真元,张嘴一吐,红光乍现,一个灯笼般的花蛊现出,缓缓自转,悬浮半空。

正是兜率花。

方源心念一动,兜率花通体红芒飞涨,红霞烂漫中,几件事物飞出。

有绷带,有大药罐,有小药瓶。

“小药瓶的粉末,可以消毒消炎,少量即可。大药罐中的药浆,则可以止血生肌。应该会用绷带的吧?”方源说着,递给这些东西分作两份,递给白凝冰一份。

“这些都是凡人的手段,不过学堂中倒也有过教授。”白凝冰接过去,嘴一撇,“你这人准备得倒挺周全的。”

说着信手揭开大药罐,顿时一股冲鼻子的恶臭扑面而来,令她不禁头往后仰,叫道:“怎么这么臭!”

方源笑了笑,没有搭话。

他褪下衣衫,先倒了小药瓶的粉末,伤口处顿时传来疼痛感,火辣辣的。又打开大药罐,里面药浆如烂泥,一片黑绿色,味道也不好闻。

但方源却在前世早就习惯了,不为所动。

他手上掏出一捧黑绿药浆,均匀地涂抹在伤口处,动作娴熟老练至极。

然后再用绷带绑住伤口,一圈圈缠绕,很快就处理完毕。

因为药浆的原因,伤口处很快就传来一阵阵的清凉之感,取代了原先的火辣和疼痛。

“你这药浆还挺管用!”一旁,白凝冰仍旧在处理着伤口,龇牙抽气道。

她白袍破散,此刻涂药,几乎袒胸露乳,*光乍泄,却犹不自觉。

一边用药,她还一边嗟叹:“唉,现在想想,若是有一只可以治疗的蛊虫,该有多好。”

方源看了一眼她,索性又唤出兜率花,取出两套服饰。

他准备充足,本来就打算跑路的,因此准备了多套衣服。再加上白凝冰和他年纪相仿,体型也差不多,自己的衣服倒也挺适合她。

“接着。”他将一套衣服递给白凝冰。

白凝冰接过衣服,嘿了声,微微惊异:“想不到你准备得还挺充分。”

“凡事有备无患。”方源应了一句,将身上剩下的衣服,湿透了的鞋袜也都脱去,换上新衣。

顿时,舒适干净的衣服换上去,令他感觉好多了。

白凝冰也换上衣服,将破烂不堪的白袍随意地扔在沙滩上。但她的脸色却很不好看,显然此刻脱离险境,包扎伤口以及换衣服时,都让她意识到了自己的身躯的转变。

“接下来,你打算怎么办?什么时候把阳蛊给我?”她走上前,皱着眉头问道。

方源将换下的黑袍和鞋袜都拾起来:“我不是说过么,此行要去白骨山。至于那只阳蛊,至少得等到我三转吧。”

白凝冰的眉头皱得更紧,声调一扬:“还要等到你修到三转?”

她从未想过,自己有朝一日,会陷入到如此尴尬的境地。堂堂男儿,竟然变成了女子。生死一线的刺激消退之后,这该死的古怪感觉,就涌上了她的心头。

若有可能,她一刻都不想再忍受了。

方源抬头看了她一眼,没有说话,而是走到江边,利用江水洗涤衣服。

这身黑袍虽然有破洞,但修补一下还能再用,不像白凝冰的那件白袍。自己不知道要在野外生存多久,这衣服不能浪费了。

白凝冰是个聪明人,方源的沉默让她意识到自己的真正处境。

现在,她空有三转修为,但是浑身一只蛊虫都没有。就算是有,恐怕也不敢拿方源怎样。阳蛊被方源炼化,只要他心念一动,就能自毁。

阴阳转身蛊都是一对,若这只毁掉,那到时候,白凝冰恐怕再也变不会男儿身了!

看着方源的背影,白凝冰一阵咬牙,心中郁闷至极。想她堂堂的白家天才,居然会沦落到如此境地,受他人的摆布。

这感觉,让心高气傲的白凝冰分外不爽。

“现在我们没有一只治疗蛊,要是再遇到危机,该怎么办?问题还远远不止这些,我身上一只蛊虫都没有,根本就没有战斗力。不行,我得捕捉一些野外蛊炼化了,否则我连自保之力都没有!”

白凝冰正说着,忽然肚子传来叫声。

“可恶!”她捂了捂肚子,一阵饥饿感袭来,“喂,洗衣服的,快把你的肉干取出一些来,我都快饿扁了。”

在竹筏上漂流的五天,他们都是靠着方源带着的肉干为食。

肉干虽然生硬,嚼起来仿佛干木材,但却能果腹,提供能量。

方源站起身来,双手用力将黑袍的水整掉,又甩了甩之后,这才回答白凝冰道:“你急什么?给我拿着它。”

白凝冰皱着眉头,勉为其难地伸手接过黑袍。

方源再唤出兜率花,取出了一袋肉干。

白凝冰毫不客气地夺到手中,取来就啃。他嚼得牙齿嘎嘣响,牙关都酸,却不亦乐乎。

方源看了看她,脸上带笑。这个白家天才什么时候被饿过?想想前世自己的经历,此刻很是理解她的感受。

白凝冰啃下一块肉干,舔了舔干燥的嘴唇:“肉干管饱,就是太硬了。唉,不过能有肉干吃,我已经很知足了。”

方源脸上的笑意更浓,在白凝冰惊讶的目光下,他取出一口铁锅。

“你居然连铁锅都带来了?太好了,我们可以用水把肉干煮烂了吃!水的话,就取江水。不过烧火还得木柴,恐怕得砍伐树木。”

说到这里,白凝冰扫视周围,感到为难。

他们搁浅着陆的这处浅滩,一面临水,三面则是高耸的峭壁。峭壁上是茂密的丛林,浅滩上却无一根木材。

白凝冰要去木柴,就得攀上峭壁,砍掉树木。

这换做他以前风光的时候,做这些不过是小菜一碟,信手拈来。但现在她手上空无一蛊,想要攀上这滑溜的峭壁,恐怕得大费周章。

白凝冰暗暗为难,就在这是,方源又取出了一堆煤石。

煤石可比木材要好用多了,白凝冰看了自然惊喜。

紧接着,方源又取出打火石,火油,还有铁架。很快就将一切都搭建好。

白凝冰看到这里,神色严肃起来,湛蓝的眸子盯住方源:“你准备得太充分了,是不是早就想离开青茅山了?”

方源准备充分得有些过分了,连这些东西都带着,聪明如白凝冰发现了端倪。

“你觉得呢?”方源笑了笑,并没有正面回答白凝冰的话,而是手指着铁锅,“你可以去舀一些江水了。”

白凝冰咬牙,方源这种态度,让她恨得牙痒痒。

她舀来水,方源已经将火升好。

先将铁锅里的水烧开,方源再将一小袋肉干都倾倒进去。不一会儿,一股肉香味儿,就飘荡而出。

白凝冰闻了,又下意识地舔了舔嘴唇。

方源取出筷子和汤勺,和白凝冰两人立即大快朵颐。

烂熟的肉干,只是咀嚼几下,就吞咽下去。滚烫的肉汤,更让两人通体舒泰。唯一美中不足的是,这黄龙江水含着泥沙,吃在嘴中,口感欠佳。

但在这颠沛流离的情况下,能有这待遇,还能有什么不满足的?

“还不太饱啊,再来半袋肉干吧。”白凝冰欲求不满,摸摸肚子道。

方源立即拒绝:“不能再多吃,必须省着点用。”

“干嘛这么抠!你看这丛林就在眼前,会有多少的野味?”白凝冰不满地叫道。

方源瞪了她一眼:“我当然知道丛林里有野味,但野味也代表着野兽。你现在能解决多少只野兽?遇到兽群怎么办?万一我们遇到伏击的野生蛊虫怎么办?就是猎杀了野兽,它的肉是不是有毒,能不能吃?你有辨别毒性的蛊虫吗?”

白凝冰噎住,哑口无言。

方源冷哼一声,白凝冰是白家天才,自然心高气傲,这番敲打已经足够,再教训下去反而过犹不及。

他盘坐下来,将锅取下,在铁架挂上刚洗过的黑袍,利用煤石残留的温度,熏干它。

方源接着道:“天色不早,我们就在这里渡过一夜。明天再探寻丛林。这块地方我特意挑选的,三面峭壁,必定极少野兽能到达此处沙滩,相对安全一些。不过也不能麻痹大意,我们轮流守夜。”

这就是两个人的好处了。

方源心念一动,又唤出锯齿金蜈,以及天蓬蛊。

“这两只蛊虫,暂时借给你用。你多多熟悉一下罢。”方源道。

他现在只有一转初阶的修为,催动三转蛊虫十分勉强。就算有甲等的恢复能力,还有天元宝莲,也发挥不出三转蛊虫的应有威能。因此不如转给三转修为的白凝冰。

白凝冰接过蛊虫,不禁深深的看了方源一眼。

蛊师之间,蛊虫可以相互借用。

蛊虫中,寄居着的是蛊师的意志。只要得到主人的承认,其他人也能沟通蛊虫,运用它们的力量。当然了,没有自己炼化得那么称心应手。

而且,只要原主人心念一动,想要反悔,其他人就会立即失去操纵权。

但就算如此,蛊师之间也很少借蛊虫给别人。

虽然有情势逼迫的原因,方源此举,却透着大气,让白凝冰侧目。(未完待续。

\end{this_body}

