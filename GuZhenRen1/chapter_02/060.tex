\newsection{父女相认}    %第六十节:父女相认

\begin{this_body}

%1
商燕飞念头一动,瞬间消失在黑暗当中。

%2
再出现的时候,他已经在商家外城当中。

%3
嘈杂的人声旋即传入耳帘,街道两旁排列着各种小摊。

%4
周围行人,只看到忽有血焰一闪,凭空就出现了一名黑袍血发的英俊男子。

%5
“哎妈呀,吓我一大跳!”

%6
“这人是谁?居然敢在商家城滥用蛊虫?”

%7
许多人投来惊疑不定的目光,只有几位认出商燕飞,但一时间又不敢确信。

%8
商燕飞没有这些目光放在心上,他循着血脉间的感应,犀利的眼神投到商心慈的身上。

%9
二女正驻足在一个小摊前。

%10
“小姐,这个簪子好看!”小蝶取出小摊上的一个玉簪,放到商心慈的秀发上比了比。

%11
商心慈勉强笑了笑,离别了方源之后,她心情低落。

%12
小蝶倒是被这城中的繁华景象迷住,又活泼起来。

%13
忽然,商心慈心有所感,转头望去。

%14
她在第一时间,就看到了商燕飞。

%15
在人群中,黑袍血发的商燕飞如鹤立鸡群,分外显眼。

%16
但吸引商心慈的,却并非他的形象,而是源自血脉亲情的一种神秘联系和呼唤。

%17
没有说一句话,当商心慈第一眼看到商燕飞时,她就知道了这个中年男子的身份。

%18
这是她的父亲!

%19
父亲……这个词,对商心慈来讲,是多么的神秘,多么的遥远,蕴藏了多少悲怜,积压了多少心酸。

%20
小时候,她无数次问她的母亲,有关父亲的话题。但母亲缄默其口。而如今,她终于见到了父亲。

%21
“原来娘临终前,要我来到商家城,是这个缘由!”她恍然大悟,顿时眼泪就止不住地往下掉。

%22
商燕飞看到商心慈的第一面,就联想到了她的母亲,明白了她的身份。

%23
像,真的是太像了!

%24
这孩子眉宇间的温柔,和她如出一辙啊。

%25
商燕飞心口蓦地一疼,恍惚间他仿佛又看见了她。

%26
那是在春雨蒙蒙的下午,荷塘边杨柳随风摇摆,在破旧的屋檐下,还是商家少主的商燕飞,偶遇了正在避雨的张家少女。

%27
才子佳人,一见钟情。倾诉衷肠,私定终身……

%28
然而世事无奈,江山美人往往如鱼和熊掌,哪能兼得。

%29
年轻的商燕飞,心中燃烧着旺盛的火焰,这火焰源自男人天生就对强大和权势的野心。

%30
在野望和柔情之间,在责任与逍遥之中,在强敌的逼压和佳人的承诺之间,他最终选择了前者,放弃了后者。

%31
所以,他击败了一干兄弟姐妹,登上了商家族长的宝座。所以,他成为了五转蛊师,拥美无数,如今儿女满堂。所以,他不能再找那个她,因为张家和商家乃是世代的仇敌。

%32
人在江湖,身不由己。

%33
族长的身份,成就了他,也束缚住了他。

%34
王的一举一动,都会牵引风云变化,都在受着世人瞩目。他身为商家族长,怎能因为儿女私情,而不顾家族影响?

%35
这些年,他努力说服自己,把愧疚和不安深藏,用大义和责任来麻痹。

%36
他以为自己已经忘了那一切,但是当他陡然间看到商心慈时,他心底最深处的那些记忆,那些温柔,如春雨般淅沥而来,瞬间笼罩住他的心扉。

%37
此刻,他心潮起伏,汹涌!

%38
血浓于水的亲情,化作一条长河。而惭愧又将这道长河,泛滥成江海,瞬间淹没席卷了他。

%39
他轻轻迈出一步,瞬间消失在原地,下一刻出现在商心慈的面前。

%40
小蝶惊呼一声,周围人皆投来诧异、惊骇的目光。

%41
但当事的两位,却毫无察觉。

%42
“你,你叫什么名字?”商燕飞努力开口,他的声音带着磁性,流露出浓郁的温情。

%43
商心慈却没有回答。

%44
她那一对美眸中,泪水扑簌扑簌地往下掉。

%45
她后退一步,紧紧地抿着嘴,定定地瞧着商燕飞,目光中流露出倔强……

%46
就是这个男人,伤了娘的心。

%47
就是这个男人,让我从小到大饱受歧视和欺凌。

%48
就是这个男人,让娘魂牵梦绕,就是死前也在挂怀。

%49
就是这个男人,他就是我的……父亲。

%50
这一刻,她情绪激荡至极,无数的情感混杂在一起,形成狂暴的漩涡,将她的心神都吞没。

%51
她昏了过去。

%52
“小姐!”被商燕飞气势所摄的小蝶,蓦地惊醒,尖叫一声。

%53
但商燕飞早先她一步,将商心慈揽在怀中。

%54
“是谁在公然违反城规,动用蛊虫,想进监牢吗?”一队城卫军察觉到异常,骂骂咧咧地赶了过来。

%55
“啊,族长大人!”一看到商燕飞,他们顿时脸色大变,齐齐跪倒在地上。

%56
整个街道轰动。

%57
“你,你竟是商家的……”小蝶结结巴巴,说不出话来。

%58
商燕飞伸出手,抓住小蝶的胳膊,血焰一闪,三人在原地瞬间消失。

%59
……

%60
“都排好队,一个个来。要进商家城,每人缴纳十枚元石。进入城中,不得滥用蛊虫,违者将压入监牢至少七天!”驻守在城门外的守卫高喊着。

%61
在城门口的墙上,张贴着许许多多的通缉令。有些通缉令,已经老旧泛黄,被盖在下面,只露出一角。有的通缉令,则是全新,随意地贴着。

%62
方源和白凝冰随着人流,渐渐接近城门,果然发现其中有一张通缉令,白家寨发布,通缉他们。

%63
“百家寨……”看到此处,方源不禁暗暗冷哼一声。

%64
“二位请住。”城门的守卫来到方白二人的面前。

%65
两人都穿着普通衣衫,看似平民,白凝冰将草帽压低一点。

%66
“这是二十块元石。”方源并不紧张,递过去一个钱袋子。

%67
守卫检查无误后,立即让开了路。

%68
尽管他旁边的城墙上,贴着密密麻麻的通缉令,方便照看对比。但他却始终没有看一眼。

%69
那些通缉令,都不过是表面文章罢了。

%70
商家以利为先,只要付得起元石,就能进入城中。

%71
每天都会有大量的魔道蛊师,从守卫们的眼皮子底下过去,大家心照不宣。

%72
魔道蛊师打劫而来的赃物,都会在商家城出售。同时他们要补充蛊虫,商家城也是最好的选择。

%73
可以说,商家城的繁荣,魔道蛊师是其中一个巨大的支柱。

%74
当然,魔道蛊师也不能太明目张胆,大张旗鼓地进城。毕竟商家城乃是正道门庭,也需要顾及一些影响。

%75
二人走过高大的城门,入目的是一道极宽敞的大道。

%76
街道上人头攒动,道路两旁栽种着高大笔直的绿树。阳光照耀,绿树成荫,各种面食摊,卖烧饼,卖豆腐脑儿,卖菜,卖首饰的小贩,都缩在树荫下,摆着五花八门的地摊。

%77
二人走过一段,街道两旁开始有高大的竹楼,黄色的土坯房,或者白墙灰瓦的砖房。

%78
商铺、酒楼、客栈、铁匠铺纷纷出现在眼前。

%79
“这位大兄弟,要住宿吗?我们这边便宜得很,每晚只要半块元石。”一位中年妇女走上前来,脸上堆满了笑容。

%80
方源瞪了她一眼,一言不发地走了。

%81
他容貌吓人,中年妇女心中一跳,不敢再纠缠他,便把目标转移到身后的白凝冰身上。

%82
“这位兄弟,出门在外多不容易。咱们客栈好啊,晚上还有小妹呢。你要到青楼一条街去,那可不便宜。咱们凡人在外行商就是拿命换钱,血汗钱扔到那些地方砸不出一朵水花。还是咱们客栈的小妹好,便宜!有熟透的桃子,也有嫩得能掐出水的雏儿。大兄弟要哪种的?”

%83
中年妇女压低声音,神情暧昧。她看方源和白凝冰的打扮,将他们误认为凡人。

%84
白凝冰被她说的一头黑线。

%85
“滚。”她冷哼一声,声音冰寒彻骨。

%86
中年妇女脸色骤变,身体一僵,停在原地。

%87
“原来是个雌的。”

%88
“哈哈哈,张大姐你这次看走眼了吧……”

%89
周围同样在拉人的同行们一阵哄闹,中年妇女只得讪笑。

%90
这一路行来,白凝冰对于伪装已有心得,进步很大,叫经验丰富的捐客都认错了。

%91
四季酒楼。

%92
半个时辰后,方源停步在一栋高达五层的楼阁前。

%93
这楼白墙黑瓦,朱门大柱,酒气飘动,菜香四溢。乃是商家城里的一座出了名的酒楼。

%94
“二位客官,快里面请。”店小二很会来事,看到方白二人,连忙走过来邀请道。

%95
两人走了半天,一路向上攀登,也是饿了,便迈入酒楼。

%96
“二位客官,请这边坐。”店小二在前面引路。

%97
方源微皱眉头:“这大堂里太吵,我们去楼上。”

%98
店小二顿时露出难色:“不瞒二位,楼上的确有雅座,但只对蛊师开放。”

%99
方源哼了一声,调出一股雪银真元。

%100
店小二连忙鞠躬:“小的有眼不识泰山,恭请二位上楼!”

%101
来到楼梯口,店小二止步,一位年轻貌美的少女,长相甜美,立即走了过来,娇声道:“二位大人,不知要上几楼。我们四季酒楼分有五层。一层大堂可供凡人饮食,二层供一转蛊师,三层供二转蛊师,菜价打八折。四层供三转蛊师,打五折。五层供四转蛊师,免费招待。”

%102
方源笑了笑:“便去四层罢。”

%103
少女招待听了这话,脸上神色顿时又恭谨了几分,行了一个万福礼道:“敬请二位贵客出显真元。”

\end{this_body}

