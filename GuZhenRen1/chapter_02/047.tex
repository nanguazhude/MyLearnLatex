\newsection{被发现}    %第四十七节:被发现

\begin{this_body}

流血的一夜过去,晨光照耀在破损不堪的营地。

在沉闷的氛围中,众人打扫战场,收拾货物,带着沉痛的心情重新启程。

然而,这一次的狼群袭击,只是一个开始。

几天之后,他们再次遇到苍狼群的袭击。

这次的规模,比上一次还要庞大。不过商队有了充分的戒备,损失反而较之上一次,有些减少。

击退了这群苍狼,商队还未喘过气来,便在三日后,遭受到了电狼群的袭击。三只狂电狼,九头豪电狼,杀死了足足十五位蛊师。最终,它们留下一地的狼尸,只剩下一头受伤的狂电狼,率领着残狼撤退。

商队中虽然很多蛊师有心复仇,却有力未逮,不敢深入啸月山,只能眼睁睁地看着狼群撤退。

这一次袭击,让商队的首领,以及副首领都意识到自身的危险处境。在当晚,他们就决定加快步伐,尽快脱离啸月山抵挡。

然而即便如此,在此后的半个多月内,他们仍旧遭到了狼群的频繁袭击。

苍狼、电狼、雪狼、双头狼,甚至还有血牙狈……

出了啸月山,商队上下着实松了一口气。

过了一段平和安稳的日子,当他们进入白虎山时,再次遭遇了兽群的袭击。

这一次,是龟背老猿。

” 章节更新最快” 这些白色猿猴,体型庞大,背后有甲壳,甲壳上龟纹清晰。猿群的袭击。虽然没有造成人员的大量伤亡,但是货物损失很多,让无数人心中滴血。

方源的货物亦遭厄难。十几辆车的货物,只剩下不到一半。

商队士气大落,这些人拼死拼活的行商,无非是为了赚取钱财。但是这些损失,让许多人此行做了无用功。

“这一趟,白跑了。”

“昨天我清算了一下,前前后后。只赚了不到两千块元石!”

“我这边更惨,货物已经损失了三成。”

“再惨能有张家惨吗?他们的货物已经损失一大步了!”

“唉,早知道如此。还不如缩在家族里面,何苦冒险,却只赚了这么点钱!”

……

就在这样的氛围中,五天之后。一支白虎群来袭。

商队再次遭受损失。

又七天之后。一群炎虎突袭,商队的营地被火焰覆盖,大量的货物被烧毁。

众人士气降落到低点,已经有许多人亏的血本无归。

十天之后,就在他们为即将离开白虎山地界而欢呼时,一只彪出现了。

五虎一彪,彪是长着翅膀的虎。

””如虎添翼,说的便是它。

一头彪。至少是千兽王。因为有飞翔的能力,更加难缠。

为了抵抗这头彪。商队的某家副首领都因此不幸丧生。

彪尾随着商队,不断骚扰,长达上百里。最终商队商议,决定弃车保帅,壮士断腕般舍弃了近百位家奴。

这些家奴大多都是伤残,他们哀嚎咒骂,又或者哭泣求饶,但都没有改变得了命运。

最终彪饱餐一顿,满意离去。

远离了白虎山,商队好好休整了一番,各家首领不吝赏赐,总算是振奋了士气,缓过精神来。

商队的规模,已经瘦减了不到原来的一半。

不过经过这番残酷的淘汰和磨砺,商队上下有了一些精锐的气质。

“我行商多年,这是最困难的一次。”

“这些野兽,不知道发了什么疯,攻击这么频繁!”

“此行结束之后,我就退银老了。”

“不管如何,这条商道需要重新评估风险……”

“主要还是因为,这些大山都没有人烟,没有山寨驻扎和清剿,这些野兽恣意生长,得不到遏制。”

有人感叹,有人心灰意冷,有人则仍旧保留希望。

但”蛊真人 第四十七节:被发现”是坏运气,似乎纠缠上了这支商队。在此后的旅途中,不仅有各种兽群冲击,而且还多了许多虫群和野生的蛊虫。

商队人数不断减少,人们不再关心盈亏问题,他们开始体味到生死存亡的压力。

很多货物都被主动舍弃,来追求前行速度。

落霞漫天,残阳如血。

商队穿行在山林之间,众人沉默不语,神情疲惫麻木,士气低落。

很多人都打着绷带,带着轻重不一的伤势。在坎坷的山道中,他们深一步浅一步的前行着。

昨天下了雨,山道泥泞湿滑,很不好走。

一辆装载满满货物的板车,它的右边轮子,不幸地陷在泥泞当中。拖车的驼鸡,扬起脖子,一阵嘶鸣,奋力踏步,却拽之不动。

这时,两只手搭在板车后面,用力一抬,将车轮抬出这个泥坑。

出手的正是方源。

他轻松的拍拍手,数千斤的货物,在他手中并不嫌重。

然而,板车虽然脱离了泥淖,车轮却莫名卡住,不再转动。

一旁的白凝冰弯下腰,检查板车轱辘。

在商队这么长时间,她为了伪装身份,也学到了不少东西,已经彻底融入进去。

“这是什么?”她探出手,抹了一把车轴”蛊真人”,眼中闪过疑惑的光。

车轴磨合处,似乎藏着什么东西,随着车轮前行,不断被碾磨成黑灰色的细小粉末。

这些细小粉末,落在地上,根本看不清楚。

白凝冰用手摸了一把,手上沾满了这种细粉,一磨搓,粉末就化为一片油腻。

“哦,这是我加在车轮上的油粉,可以润滑,能使板车前行更流畅的。”方源走过来,从怀中取出棉布手帕。抓住白凝冰的手,三两下就擦干净。

然后,他蹲下身子。伸手摸索了一番,车轮便又能动了。

“走吧。”他将手中的油粉擦去,笑着拍拍白凝冰的肩膀。

两人接着前行。

白凝冰越走越慢,心中的疑惑升腾,形成浓郁而化不开的雾霾。

她感到有什么不对的地方。

“方源什么时候弄的这油粉?我怎么一直都不知道……是起初的时候,还是在黄金山、啸月山?古怪啊……他对商队其实一直并不放在心上,商队损失那么多。也没见他皱一下眉头。怎么会如此细心,还关注到板车润滑的这个小问题?古怪,古怪!”

“等一下!”

忽然间。一道灵光如同闪电,在白凝冰的脑海中闪过。

刹那间,她身躯陡然一震,瞳孔猛缩成针尖大小。

””一个可能。在她脑海中似声音在深谷中不断地回荡。

她停驻在原地。心中充满了震惊!

好半天,一直走过她身边的驼鸡,忽然鸣叫了一声。这声音让她惊醒。

方源的背影,已经远去,渐渐要没入前方的人群之中。

“这个家伙……”白凝冰低下头,在草帽遮盖的阴影下,她那双蓝色的眼眸闪烁着阵阵寒芒。

太阳彻底落入西山,繁星渐渐出现在夜空之上。

在一处河滩旁。商队停止了前进,决定驻扎在这里。度过一晚。

然而刚刚将营地搭建了一半,就有一群冷翡枭猫,出现在营地附近。

“有兽群,是枭猫!”

“立即停下工作,结队防备!”

“这群该死的畜生,我的晚饭才刚刚吃了一口……”

人们咒骂着,四处奔跑,得益于前一段的苦难和磨砺,商队众人很快就组成了严密的三道防线。

冷翡枭猫体型如花豹,敏捷异常,脸部仿佛猫头鹰。一对巨大的瞳眸,几乎占据脸部的一半,在黑暗中闪着青色的幽光。

枭猫王凄厉地嘶叫一声,枭猫群如潮水般,向营地涌来。

“杀!”前线的蛊师大喝一声。

一时间,五颜六色的光辉爆闪,火焰燃烧,岩土飞砸,电光激闪……

在攻击中,无数的枭猫倒下,但后续的枭猫前赴后继。

“天呐,这是个大型的冷翡枭猫群。”有人大叫。

“啊,救我!!”防线某处终于支撑不住,一位蛊师被三只枭猫一起扑倒。惨叫声戛然而止,鲜血飞溅。

“快,堵住那个缺口。”两位蛊师被派来增援。

但已经无济于事,缺口越来越大,渐渐殃及整条防线。

“撤,撤退!”不得已,众人撤退到第二防线。激烈的攻防战中,局面陷入僵持。

“将板车还有马车车厢链接在一起,把货物都堆砌起来,形成高墙!”

在第二防线之后,第三条防线正在紧急搭建。

大量的家奴搬运着货物,忙得大汗淋漓,没有人能够偷懒。

方源搬起一个巨大的木箱,白凝冰忽然走过来,搭手另一边。

表面上她似乎在帮方源的忙,然而实际上她凑在方源的耳边,咬牙切齿地道:“你这个家伙,这群冷翡枭猫就是你引来的吧?”

方源似乎吃了一惊:“何出此言?”

“别装了。那些细粉大有古怪,我不相信你这样的人,会好心考虑这些细枝末节!”白凝冰压低声音。

“呵呵呵,终于被你发现了呢。”方源没有否认。

白凝冰忍不住磨牙,原来商队这一路上遭遇到频繁的袭击,都是方源的“功劳”!

两人抬着一个木箱,缓慢移动,周围人流穿梭,大喊大叫,注意力都集中在这场袭击战中。谁也不会注意听方白二人的窃窃私语。

“你为什么要这么做?”沉默了一会儿,白凝冰问道。

“呵呵。”方源笑了笑,“你猜?”

这个回答,顿时让白凝冰泛起要揍人的强烈冲动。(未完待续。)

138看书网138看书网www.13800100.com

------------

\end{this_body}

