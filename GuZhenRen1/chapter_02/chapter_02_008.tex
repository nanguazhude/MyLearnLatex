\newsection{轩辕神鸡}    %第八节:轩辕神鸡

\begin{this_body}

%1
时光流转。

%2
大半个月之后,白凝冰整整瘦了一圈,但整个人浑身上下,都流露出一股精悍的气息。

%3
她的精神更加矍铄,蓝眸张望间,时不时地闪过一抹精芒。

%4
毫无疑问,她适应了这种颠沛流离的生活,并且从中汲取到许多东西。

%5
方源清楚:白凝冰虽然沉默寡言,但并不代表她认输了。

%6
她心里面憋着一股劲,认真学习,努力适应。偶尔间,她会提出一些反驳意见,虽然浅薄,却脱离了当初的稚嫩。

%7
方源可以感觉到,白凝冰每一天都在进步。

%8
但这情况,并未出乎他的意料。要折服白凝冰,并不容易,任何真魔都有旺盛的反抗精神。

%9
烈日下,两人潜伏在灌木丛中,小心翼翼地观察这处山谷。

%10
山谷里,趴着一只巨大的鳄鱼,正在酣睡。

%11
这是一头熔岩鳄。

%12
三头大象叠加起来,才有它的体型。它浑身长满暗红色的鳞甲,四肢粗壮的腿支撑着它雄壮的身躯。一条鳄尾,散发着金属光泽,长达十米。

%13
尤其引人注意的是,在它的背部,高高隆起两个凸起,仿佛是两座微型火山。随着它的呼吸,两股黑烟从火山口直冒而出,时强时弱。

%14
“这头熔岩鳄,可是千兽王!要斩除它,风险太高。”白凝冰面色凝重。

%15
千兽王的身上,寄居着野生的三转蛊虫。再配合兽王本身强悍的身体素质,哪怕是三转巅峰的蛊师单对单,也难有胜算。

%16
“富贵险中求!能在山林中,遇到一只鳄并不容易。鳄力蛊需要鳄鱼肉喂养,但是如今已经消耗大半了。先试探一番罢。”方源道。

%17
熔岩鳄群生活在地底,只有其中的兽王才有能力钻出地面,呼吸新鲜湿润的空气,享受烈日阳光。

%18
白凝冰咬了咬牙,站了起来。

%19
自从得了鳄力蛊,都是她一直在使用。如今她力量大增,但离一鳄之力还有一段距离,没有圆满。

%20
熔岩鳄王正在沉睡,白凝冰一直接近到五十步,它才猛地睁开赤金双瞳。

%21
扑哧!

%22
它缓缓地撑起身躯,调转头颅,鼻腔中喷出两股高温热气。

%23
白凝冰面色凝重,即便催动了天蓬蛊,却仍旧感受到了一股灼热的气息扑面而来。

%24
她没有取出锯齿金蜈,而是甩手一记血色月刃。

%25
三转的月刃射中熔岩鳄的后辈,削掉了几片厚甲,也成功激怒了熔岩鳄王。

%26
它对准白凝冰,猛地张开巨口,吐出一颗暗红色的熔岩火球。

%27
熔岩火球足有磨盘大小,白凝冰不敢迎接,躲闪过去。

%28
轰!

%29
熔岩火球在半空中划过一道轨迹,砸在山石上。

%30
爆炸声中,山石飞溅,烈焰飞炎。

%31
一朵小型的蘑菇烟云升腾上空,渐渐消散。爆炸的地点,出现了一个大坑,坑内外还有暗红色的熔岩流淌,并渐渐冷却。

%32
“三转熔岩炸裂蛊。”方源看到这一幕,心中顿时有数。

%33
……

%34
半刻钟之后,方源在山壁上抛下绳索,将白凝冰拉上来。

%35
熔岩鳄王咆哮了几声,并没有追击。方源和白凝冰的身影消失之后,它重新趴在地上,开始闭目享受日光浴。

%36
这是因为白凝冰的攻击,都以试探为主。熔岩鳄王并没有觉得她是个威胁。只是当做擅闯领地的野兽,驱赶了事。

%37
“这只熔岩鳄王,一共有三只蛊虫寄生着。一只是熔岩炸裂蛊,一只是炎胄蛊,一只是积灰蛊。三者皆是三转蛊虫。攻击、防御、治疗,最基本的三方面也都涵盖了。”离开山谷后,方源在路上总结着。

%38
白凝冰紧紧皱起眉头,她刚刚亲身试探,已经知道要斩杀这头鳄王,难度极高,几乎已经不可能。

%39
“那只熔岩炸裂蛊也就算了,炎胄蛊的防御,血月蛊不能突破。唯有用锯齿金蜈近身战。但如果是这样破了炎胄,锯齿金蜈恐怕也要废掉。这些天来,屡次用它开路,它的银边锯齿已经破损不堪。就算是破除了炎胄蛊的防御,熔岩鳄王还有积灰蛊,可以治疗自身。它体力绝对比我俩加起来都充沛,打持久战我们必输无疑。更关键是,它能钻入地下,回到地底老巢去。我们根本没有能力阻止它回巢。”白凝冰道。

%40
方源点头:“你分析的很有道理,但我更想要斩杀它了。积灰蛊以灰烬为食,很好养活。是一只很适合我们的治疗蛊虫啊。”

%41
“哼,蛊虫虽好,也得有命去享用才是。你虽然掌握着阳蛊,但你却别想指挥我,让我冒着生命危险去战斗。”白凝冰冷哼道。

%42
“不能力敌,我们可以智取。若换做其他野兽,也就算了。但这熔岩鳄王,我们却可以吸引其他兽王,引起它们之间的争斗,然后渔翁得利。”方源道,他并不想因为困难而放弃。

%43
在不可能中创造可能,也是他最喜欢做的事情。

%44
若换做其他地表的猛兽,除非是刚刚迁移过来的,否则各有各的地盘领域。彼此之间,都知道对方的存在,不会有引起它们争斗的希望。

%45
但熔岩鳄王却不同。

%46
它平常时期,生活在地底,偶尔间钻上来透透气。仿佛深海的鱼,跃出海面。

%47
它的存在,并不被其他兽王熟知。它只能算是偷渡客。

%48
除开狡电狈这等兽王,大多数的兽王都智慧不高。只要将某一头兽王引来,它们彼此之间都会感觉到生命的威胁,从而展开激战。

%49
等到它们拼得两败俱伤,方源和白凝冰二人就可以趁火打劫了。

%50
方源的话,让白凝冰眼前一亮。

%51
她点头道:“我们没有移动蛊,这个方法仍旧冒险。不过比力战熔岩鳄王,还是多了许多成功的希望。可以一试。”

%52
和人类社会差不多,野兽之间,也各有地盘。

%53
强大的兽王,往往统领着兽群,盘踞在自然资源丰富的地点。宛若人族形成山寨,占据元泉。

%54
势力之间,必有接壤,只要往其他方向探索过去,自然就会有所发现。

%55
此后连续五天,方白二人以熔岩鳄王为中心,四处探索。

%56
西北方向是他们来时的路,不必再探。

%57
绕过山谷,在东南面,他们发现了白猿兽群。首领是一头老白猿,千兽王级。白猿速度很快,在引来的途中,就会被白猿追上,团团包围。因此他们立即放弃了这个打算。

%58
在西南方向上,是一处腐烂的沼泽地,臭味熏人。这是毒物的世界。

%59
毒蛇盘踞在枯树根下,拳头大小的毒蜂团团飞舞,一张纸巨大的蜘蛛网上,盘踞着脸盆大小的黑蜘蛛。

%60
从沼泽中心传出的雷鸣般的蛙叫声中,方源判断,这片沼泽之主是一只治疗蛊——吞毒蛙。它高达四转,体型袖珍,以毒素为食。蛊师若是中毒了,催动它吞吸了毒素,就能起到治疗效果。

%61
它移动速度不行,但要深入沼泽,引它出来,更加不易。

%62
方源和白凝冰二人手中,都没有治疗蛊。万一被其中毒物咬伤,将分外麻烦。再者,在偌大的沼泽中搜寻一只体型袖珍的吞毒蛙,十分困难。

%63
二人最后在东北方向,发现了一个房屋大小的蜂巢。

%64
里面生活着一支数量庞大到恐怖的虫群——狂针蜂。

%65
这蜂群更加不好惹。

%66
狂针蜂一旦成蛊,蜂针将有洞穿之能。也就是说,就算是白凝冰顶起天蓬蛊的白甲护身,也会被三转的狂针蜂洞穿。

%67
夜风呼啸。

%68
风灌入山洞,吹得洞中篝火摇曳。

%69
这是一处小土丘,临时发现了里面的一个山洞。

%70
这山洞位置并不理想,首先不背风,风从洞口中直吹进去,在夜间带来深重的寒意。其次洞顶露天,宛若天井,能看到夜空繁星。

%71
方源和白凝冰沉默着,围坐在篝火旁。

%72
方源面无表情,白凝冰则失望地叹了一口气:“这些天我们把周围都探索遍了,你的主意虽好,却没有合适的对象。看来,我们只能放弃熔岩鳄王了。”

%73
“谋事在人成事在天,我们实力不足,有些事情就得靠运气。罢了,原本还想养着鳄力蛊。但现在看来,斩杀熔岩鳄王很不现实。明天我们就启程,继续向白骨山进发。”方源无奈地点点头。

%74
但就在这时!

%75
忽然从外面传来,熔岩鳄王愤怒的咆哮声。

%76
“怎么回事?”

%77
“是那头熔岩鳄王!”

%78
两人对视一眼,立即走出山洞,远远望去。

%79
只见山谷那边,七彩虹光闪耀,火焰升腾,声势滔天。

%80
在绚烂耀眼的彩霞当中,一只巨大如小山般的锦鸡,显露出来。它鸡冠如黄金,高高耸立。身上羽毛五光十色,不断变化,闪耀非凡。

%81
“不好,是轩辕神鸡,万兽王级的飞禽!熔岩鳄王要遭劫了。”方源当即道。

%82
“轩辕神鸡?”白凝冰疑惑。

%83
“这是独来独往的万兽王,飞行在天空中,觅食时落到地面。神鸡数量稀少,分布天下。它们的身上寄居着各类虹蛊。一旦作战,必是漫天七彩霞光,五色霓虹绚烂。唉,熔岩鳄王是没有指望了。我们快退进去,这神鸡目光极为敏锐,堪比鹰隼。若是发现我们,袭杀过来,可就大大不妙了。”说着,方源就退入山洞内。

%84
白凝冰抿着嘴唇,紧随其后。(未完待续。

\end{this_body}

