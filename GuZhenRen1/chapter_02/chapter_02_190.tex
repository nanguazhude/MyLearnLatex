\newsection{只差最后一步}    %第一百九十节:只差最后一步

\begin{this_body}



%1
十月二十日。

%2
大殿内,淡红色的光芒幻灭,照耀四周。

%3
青铜砖面上的浮雕,已经消失一大半。

%4
方源满脸苍白,形容枯槁,瞪着通红的双眼,死死地盯着光团的每一丝的变化。

%5
一切,都寂然无声。

%6
十月二十一日。

%7
地灵传来不好的消息,一位五转强者,进入了蛊师福地。

%8
方源看了影像,立即认出此人:“原来是萧家的萧芒。他拥有五转太光蛊,是光道的强者。前世时,三叉山上就有他的身影,终究是来了。”

%9
地灵吸了一口气:“太光蛊?这么说来,这萧芒能催动出太古时代的荣耀之光!这对于我们来讲,是个巨大威胁啊!”

%10
地灵担忧不已。

%11
太古时代有九天,分别是白天、赤天、橙天、黄天、绿天、青天、蓝天、紫天、黑天。

%12
太古的阳光,非同寻常,乃是荣耀之光,能洞穿九天,挥洒温暖,恩威施加万物生灵。

%13
而到了如今,赤橙黄绿青蓝紫七天已经不在,只剩下白天和黑天。而阳光也再无太古的荣耀,衰弱不堪,只能洞穿白天。

%14
五转的太光蛊一经催动,就能爆发出太古烈日的荣耀光辉。此光一丝攻击力也没用,但却能透过一切隔膜,洒照天涯海角。

%15
也就是说,这片福地也阻隔不了太古之光。

%16
方源冷笑一声:“霸龟,你且宽心。他的太光蛊,乃是盗墓而得,只是一只残蛊。每个月,只能催动三次。三次一过,就要自毁。”

%17
地灵这才松了一口气:“那就好。这些天来,我变得越来越虚弱了。到了最后关头,还得靠你自己啊。”

%18
“呵呵。我向来喜欢靠自己。”方源答了一句,不再说话。开始继续炼蛊!

%19
十月二十二日。

%20
噗。

%21
“糟糕,又失败了!”

%22
方源大吐一口鲜血,双眼一黑,差点昏死过去。

%23
他咬紧牙关,手撑着地面,只感到天旋地转,双眼金星直冒。耳畔嗡鸣不断。

%24
尤其是胸中烦闷至极,几欲呕吐。

%25
半晌之后,这种糟糕的感受才稍稍缓解了一丝。

%26
方源吐出一口浊气,慢慢坐稳。

%27
“炼蛊失败,就遭反噬。这处步骤,我已经失败了三次。倒也不是我技术不佳。我已经做到最好,但此步就是要运气,赌那十分之一的成功概率。唉!没有时间了!”

%28
方源脸色苍白,强忍住反噬的痛楚,开始第四轮的冲刺。

%29
而此时,铜鼎中的仙元,已只剩下四份不到。

%30
十月二十三日。

%31
方源停下动作。看着手中的这只蛊虫,眼中精芒烁烁。

%32
此蛊乃是甲虫,大肚翩翩,头尾如尖锥,没有任何的足须和触脚。它形态模糊,仿佛是模糊雕刻的粗坯,毫无生机,仿佛一块灰色石头。

%33
地灵却欢喜地道:“年轻人。我果真没有看错你!你炼成了这伪蛊,只差最后一步,就能去伪成真,炼成真正的第二空窍蛊了!”

%34
“没错,就差这最后一步。”方源的语气很复杂,既有轻松,又有沉重。

%35
炼制这第二空窍蛊。犹如登山。前面步骤多达数千,失败了不知多少次,方源几乎不眠不休,但终于达成此步。过往的努力和付出。并没有白费,是以轻松。

%36
但这最后一步,却最为关键,是质变的一步,要动用到仙蛊神游蛊。

%37
方源虽然炼成过春秋蝉,但还从未用仙蛊炼仙蛊,因此这最后一步也是他最没有把握的一步,所以他的心情又沉重。

%38
“三百岁为春,五百岁成秋。神机无限,扩游四野,添三更,再三更,三更得九。九为极,大功告成……这最后一步,得用寿蛊,用神游蛊,还得用两只三更蛊。”方源心中琢磨着。

%39
前面的步骤,他都能理解,甚至能修改。但秘方到了此处,他只懂得这其中三分真意。

%40
“地灵,福地中又有什么变化?”方源忽然开口问道。

%41
“来了两拨人马,数十位三转蛊师,各自由一位四转蛊师带领着,声势浩大。”地灵将画面展现给方源看。

%42
“原来是车家和左家,啧啧,两家的族长领头,大部分的家老都过来了吧。”方源看了一眼,就认出了跟脚。

%43
整个三叉山,就位于左家的冷颤山和车家的飞来山之间。

%44
这两个家族,不断扩张,近些年来,一直在三叉山一线竞争角逐,都有侵占之心。

%45
但三王传承爆发开来,彻底打乱这两大家族的大计。

%46
整个南疆十万名山,还有无数无名的杂山乱峰,又遍布猛兽野蛊,环境险恶,极难行走。

%47
其他势力,只能派遣精英前来。但这两个家族,却是近水楼台,先前一直按捺不动,此时发觉传承有异,终于派遣大部队前来。

%48
对于方源来讲,这是个坏消息。

%49
在最后关头,众人定然齐攻福地中枢,也就是这座大殿。这些车家、左家的人马,都是方源的敌人。

%50
“除了他们,届时还有李闲、狐魅儿、易火、孔日天等强手。最后关头,我要全力炼蛊,抵御外敌只能靠地灵,还有白凝冰、风天语。这形势险恶,却还只是外部。”

%51
“最后一步,需要连续用两只三更蛊,就会导致我身上的时间流速加快九倍!对于春秋蝉来讲,却是大补药。届时,压力暴涨,危机空窍。这是内因。”

%52
“内外交迫,危机四伏。但我也只能咬牙坚持下去,已经努力到这样地步,就差一步就能登上峰顶。不赌一下,我是不会甘心的。若真能成功,我就拥有第二空窍。今后加以培养,到了六转,也不会落后凤金煌太多了。”

%53
在方源的重生大计中,青茅山只是个起点,商家城也只是一个平台,第二空窍蛊也是垫脚石。

%54
但正是因为这些一次次的积累下来。他才能往更高一层冲刺。

%55
接下来的很多机缘,一环套一环,没有一定的修为、实力,根本就没有参与的资格!

%56
“生灵万物,优胜劣汰,这机缘更是要寸步不让,时机也要争分夺秒。这样才不愧这重生之躯啊……”

%57
方源长叹一声。开始休整,为最后的一天做准备。

%58
十月二十四日。

%59
方源从沉睡中醒来,缓缓地睁开双眼。

%60
“好多天没有睡得这么舒服了,接下来就是大战!”他站起身来,缓步踱出大殿。

%61
大殿外,受到地灵的指引。已经站立着两人。

%62
“主上!”风天语一看到方源,立即跪倒在地,将一只蛊虫奉上。

%63
此蛊其貌不扬,好似灰石圆片。不是别的,正是百战不殆蛊。

%64
“属下幸不辱命,已经闯过百关,获得信王传承。收得这些毛民。”风天语又道。

%65
在他的身边,站着数百位毛民,各个浑身长着浓密长毛,默然站立着。

%66
“善。”方源点点头,淡淡地称赞一声,并不意外。

%67
这毛民有个秉性,喜欢追随比自己更会炼蛊的人。风天语闯了百关,有这些追随者并不奇怪。

%68
方源又走到白凝冰的面前。

%69
白凝冰凝望着眼前恢弘的青铜大殿。目光中透出一丝了然:“看来,这便是福地的中枢所在了。”

%70
说完,她目光移向方源:“哼,你最好记得你的承诺。”

%71
方源笑了笑:“你放心好了。”

%72
他望向白凝冰的身后,近十万只犬兽,漫漫无涯,或盘踞在地。或相互嬉戏,或奔逐打闹。

%73
方源微微皱了皱眉头,这是白凝冰掌控薄弱了。换做是章三三、巫鬼或者武神通的任何一位,都能令这狗群排布紧密。一动不动,如同军队。

%74
但白凝冰毕竟是赶鸭子上架,先前都没有任何的奴道训练,能做到这一步,已属不易。

%75
事实上,白凝冰现在脑袋昏昏沉沉,举手投足间都感应差池,有种魂魄沉重,身躯犹如提线木偶的感觉。

%76
一下子,掌控这么多的犬兽,实在是难为她了。

%77
“接下来你听我的安排,去一一布置防守。无论敌手如何挑逗,都不要主动进攻。切记,切记。”方源叮嘱道。

%78
“嗯,既然是你安排,那成败都不关我干系。”白凝冰冷声道。

%79
“呵呵,不论成败,都会给你阳蛊的。”方源微笑着担保。

%80
“哼,你最好说到做到。”

%81
……

%82
“两道光柱接连消失,这就意味着信王、犬王的传承,被人夺走了!”清晨的三叉山顶,蛊师们震惊无比,人声鼎沸。

%83
“这次传承开启,非常古怪,一直持续到今天,导致福地极速衰败。”有人早就怀疑了。

%84
但相比较这个,更多的人关心的是传承去向。

%85
“到底是哪两个幸运儿,继承了传承?”

%86
“我想信王传承,应该是铁慕白大人获得了。他自从进去后,就从未出来过。”

%87
“犬王传承,恐怕得是巫鬼。”

%88
“不,是我族的武神通大人。”

%89
“哼哼,依我看我魔道的驭兽大师章三三,也有胜算呐。”

%90
众人争吵了一阵子后,终于有人发现了诡异之处。

%91
“奇怪,这次传承几位五转蛊师,都未出来。是怎么回事?”

%92
“信王、犬王传承都被继承,但为什么其他人也未出来?”

%93
“他们是被滞留在福地当中了。这片福地,已经接近溃灭,过不了多久,就会门户大开,任由我们随意进出。”一个嘹亮的声音传播开来。

%94
“是萧芒大人!”顿时就有正道蛊师,认出了说话者的身份。

%95
“这萧芒来到三叉山,却没有进入传承,他想搞什么鬼?”魔道蛊师们心里嘀咕着,萧芒的到来,压制了魔道的气焰。

%96
成功地吸引了众人的目光,萧芒傲然一笑:“接下来,我就动用太光蛊,替尔等打开福地的门户!”

%97
话音刚落,他就怒目圆睁,狂催真元,高举拳头。

%98
太光蛊!

%99
天意蛊!

%100
空拳蛊!

%101
杀招——太古光拳!!

%102
三蛊齐催,整片天空一暗。

%103
众人震恐地看到,一个光芒组成的拳头,大如山峰,从天而降,又忽然消失,击在冥冥中的某处。

%104
太光蛊虽无一丝攻击力度,但结合其他两蛊,就形成强烈无比的攻击!

%105
轰隆。

%106
无形的膜胎被洞穿,福地摇颤,巨大的漏洞形成门户,沟通了外界。(未完待续。如果您喜欢这部作品,欢迎您来起点投推荐票、月票,您的支持,就是我最大的动力。手机用户请到阅读。)

\end{this_body}

