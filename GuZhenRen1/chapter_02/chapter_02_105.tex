\newsection{苦力蛊}    %第一百零五节:苦力蛊

\begin{this_body}

%1
铁若男果真走了。

%2
如她所言,带走了铁刀苦。

%3
铁刀苦是什么样的人,方源也很清楚。

%4
能和白凝冰长久激战,很显然是个好手,战力不容小觑。如今也被铁若男招揽至麾下。

%5
“这个铁若男必须要铲除,留下她绝对是个巨大的隐患。”方源心中有强烈的预感,皆因铁若男和他,以及白凝冰有共同之处——

%6
都是找到了自己人生之路的人。

%7
这样的人,有大毅力,大勇气,只要不夭折,一定有巨大成就。

%8
铁若男也是甲等资质,同时更有铁家在背后撑腰。

%9
她是铁家八少主之一,从这点上比较,方白二人尽管各有一块紫荆令牌,却也算不得什么了。

%10
“若就这样按部就班的发展下去,我和白凝冰都要被她渐渐甩下去。唯有依靠三王传承,才能缩短其中的差距。”方源心中有谱。

%11
按照常规的发展速度,方白二人都不会是铁若男的对手。后者乃是铁家少主之一,资源雄浑,方白二人比不上。

%12
唯有不走寻常路,冒险走捷径抄近路,才有缩短差距的可能。

%13
而三王传承、义天山正魔大战,就是这样的“捷径”和“近路”。

%14
春来春又走,秋去秋回来。

%15
光阴的长河流转不息,又是一年多过去。

%16
楠秋苑,密室。

%17
方源盘坐在蒲团上,额头不断地渗出汗滴,咬紧牙关,忍耐坚持。

%18
他正在炼蛊。

%19
已经到了关键时刻。

%20
一团赤、橙、绿三色的光团,悬浮在半空当中。有栲栳大小。将他的面庞映照上一片彩霞。

%21
一心四用!

%22
“石龟负力蛊,去。”

%23
随着方源心念一动。顿时,一道淡墨色的光,从他空窍中飞出来,一头扎进三色光团当中。

%24
石龟负力蛊的添加,立即让三色光团产生剧烈反应。

%25
原先光团中,只有赤、橙、绿三色,如今又添加上一道黑墨色。

%26
四色相互角逐绞杀,谁也压不过谁。陷入到一场大混战中。

%27
一时间。四色光轮彪转,就仿佛是水突然沸腾,又好像是有一只大手在其中猛烈搅拌!

%28
轰。

%29
突然间,一声轻微的爆响。

%30
光团炸裂开来,形成一片四色光雨。光雨来得快,去的快,眨眼间,密室就沉入一片黑暗当中。

%31
一切烟消云散。

%32
“又失败了……”幽暗中,方源轻轻地叹了口气。

%33
他的鼻腔中,缓缓地流出两道血迹。同时。魂魄也因为炼蛊失败,而受到反噬。让方源感到一阵阵的头晕目眩。

%34
算起来,这已经是第四次失败了。

%35
“这一年半来,我从第四内城,打到了第三内城。但苦力蛊却一直买不到。”

%36
方源知道其中的原因。

%37
一来,是因为苦力蛊的确稀少至极,而且价格昂贵,就算是在商家城也不多见。

%38
二来,则是因为掌管商铺这一块的商睚眦。暗中对方源下绊子。

%39
商睚眦是商家的少族长之一,正掌管着商家城商铺这一块。方源要买苦力蛊,商睚眦就暗中做手脚。千方百计的阻止方源。

%40
在两年前,方源来到商家城,曾经敲诈勒索过商睚眦。因此就和他有了过节,商睚眦靠着白骨传承,通过了家族考核之后,保住了自己的少族长之位。虽然碍于毒誓蛊,不能对方源有致命攻击,但是这样报复方源。还是可以的。

%41
商睚眦到底也是商燕飞的儿子,吃一亏长一智,经过挫折后,奋发图强,一扫颓废,谨于酒色,越加精明。

%42
他依靠家族制度,来对付方源,使阴刀子,就算方源有紫荆令牌,也无济于事。

%43
拥有紫荆令牌,算得上商家的贵宾。但到底还是外人,不如商睚眦的商家少族长身份。

%44
买不到苦力蛊,方源只好自己炼。

%45
但合炼苦力蛊的成功率,实在太低。

%46
算上今天这一次,方源已经前前后后合炼了四次。先后损失了棕熊本力蛊、骏马驰力蛊、青牛劳力蛊、石龟负力蛊。

%47
除此之外,还有珍贵的辅料,以及大量的元石。

%48
幸好方源在演武场上逢战不殆,才积攒了资本,承受住炼蛊失败的损耗。

%49
“唉,这次失败,只有等到伤势复原,再去尝试了。”方源叹气。

%50
炼蛊需要谨慎。

%51
炼蛊失败,会令蛊师的身躯、魂魄都受到反噬伤害。身躯上的伤害,很容易解决。但是因为一心多用,魂魄上受损,却是非常麻烦的事情。

%52
蛊虫的转数越高,越是珍稀,失败的后果就越严重。

%53
所以,炼蛊大师们也常常因为炼蛊失败,而重伤卧榻,甚至因反噬而死亡。

%54
直接治愈魂魄之伤的蛊,不是没有,但都非常珍稀,通常都被大家族秘而不宣地掌握着。

%55
方源这次魂魄受伤,会导致他其后一个多月,都会伴随有轻微的眩晕症状。

%56
眩晕感会让方源的战力受损,尤其在高手对战中,这些微的破绽就更显得致命。

%57
所以方源通常,都是每隔一个月尝试一下。留下充足的时间,进行魂魄上的休养。

%58
炼蛊虽然失败,但今天的修行还没有结束。

%59
方源静心等待,不一会儿,听到门外响动,他便打开密门,迎来白凝冰。

%60
这一年多来,白凝冰进步神速,也打到了第三内城,形成了一套蛊虫组合。和方源并驾齐驱,鲜有败绩,共称为演武场此代两大新星,受到许多人的瞩目。

%61
两人都没有说话,只是点了点头。

%62
白凝冰盘坐到另一个蒲团上后,便伸出手掌,紧贴方源的后背,向其灌注雪银真元。

%63
白凝冰故意压制自己的修为。如今仍旧还是三转巅峰。

%64
但方源修为进步神速,如今离那三转高阶只差区区半步之遥。

%65
两个时辰之后,白凝冰停止真元的灌入,缓缓抽回手掌。

%66
方源慢慢地睁开双眼。

%67
他的气息更加充盈,隐隐有一种满溢之感。这是修为即将有所突破的征兆。

%68
“离三转高阶越来越近了,用不了多少天,就能突破。”方源心中很平静的分析着。

%69
“等到突破到高阶,就用了那只白银舍利蛊,直接将修为催到三转巅峰。这样一来。就能赶得上白凝冰了。同时。也能延缓春秋蝉对空窍的压力。”

%70
春秋蝉是方源的本命蛊,高达六转。

%71
这一年多来,在沉眠中吞吸光阴长河中的水,休养生息,气息越来越强,对方源的空窍再次产生压力。

%72
但这次的情况,比青茅山之时,要好上许多倍。

%73
方源的修为,进展神速,有三转的空窍。并没有青茅山时那般紧迫了。

%74
这一切的功臣,还是骨肉团圆蛊,以及白凝冰。

%75
修行既已结束,白凝冰缓缓站起身来,率先走出密室。

%76
整个过程,她都没有说一句话,冷若冰霜。

%77
但方源其实也不是多话的人,这一年多来,两人早已经习惯这样的相处方式。

%78
方源暗暗寻思:“等我晋升到三转巅峰。白凝冰的真元对我的帮助,就不多了。不过,我和她朝夕相处。从她的气息中的微弱变化,隐隐可以察觉到她距离四转境界,已经不远了。”

%79
哪怕白凝冰极力拖延,但是十绝体的诅咒,并没有消失,一直在发挥着作用。

%80
按照白凝冰透露,她的资质又回升了两分,达到九成三分。

%81
再增七分。达到十成,她将再次还原为北冥冰魄体。

%82
“等到她有四转修为,就有黄金真元,对我冲刺四转境界,帮助极大。”

%83
“从某种方面来讲,白凝冰和我是同命相怜。我有春秋蝉,她有北冥冰魄体……”

%84
七天之后,方源从魏央处收到一个好消息。

%85
拍卖会!

%86
“飓风山上发生了百年一见的天灾,沮家寨被毁,沮家残众归附商家城,要拍卖大量家族收藏,维持生计。因此不久后,将有一场拍卖会。”

%87
说完,魏央特意告诉方源:“方老弟,你的运气到了。我查到,这次拍卖会上就有一只苦力蛊。”

%88
方源对苦力蛊的需求,魏央自然清楚。方源也曾经拜托后者四处打听。

%89
“关键这次的拍卖会,受到商家城上下重视,商睚眦少主却不方便捣鬼的。”魏央又道。

%90
商睚眦和方源的矛盾,在商家高层并不算秘密。

%91
“苦力蛊!”

%92
“沮家寨!”

%93
方源眼中一亮。

%94
沮家寨有数百年的历史底蕴,此时变卖家产,定然有许多好东西。苦力蛊就是其中之一,方源一直苦苦追寻而不得。

%95
“这一年多来,我积攒了近百万的元石。要拍买苦力蛊,有很大把握。”

%96
“像沮家寨破灭的例子,倒不罕见。这世界除了人祸,还有天灾。飓风山上大风终年不歇,虽是灵山,有元泉根基,却环境恶劣。沮家的库藏中,说不定还有许多令人心动的宝物。”

%97
“参加拍卖会!”方源立即这样决定下来。

%98
……

%99
“怎么,这沮家家产中有一只苦力蛊?”一处书房中,商睚眦缓缓皱起眉头。

%100
他一只手上,拿捏着文书,上面记载着大部分拍卖会上的内容。

%101
另一只手,则摆在书桌上,食指敲动着桌面。

%102
“想不到方源的运道这么好,我千方百计地阻止,终于还是让他要拿到苦力蛊了。这次拍卖会,连父亲都在关注,我不好再干涉。”

%103
“不过,方源啊……要想拿到苦力蛊,可不那么容易。我也可以参加拍卖会,有我来阻击你,定要叫你损失惨重!”

%104
商睚眦不由地发出一声冷哼。

%105
当年,方源勒索敲诈他,一直让他怀恨在心。

%106
他气量狭小,见不得方白二人崛起,一定要见到方白二人的凄惨,才能稍减心头恨意。(未完待续。如果您喜欢这部作品,欢迎您来起点投推荐票、月票,您的支持,就是我最大的动力。)

\end{this_body}

