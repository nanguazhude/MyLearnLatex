\newsection{商家内城}    %第六十二节:商家内城

\begin{this_body}



%1
方源缴纳了两百块元石,和白凝冰一起进入内城。

%2
内城虽然是建造在山中,但街道宽阔,足可供十辆马车并排而行。

%3
一进入这里,人流骤然减少,只剩下外城的一半不到。

%4
但是蛊师却已经随处可见,一转满地走,二转蛊师掺杂其中,偶尔有一两位三转。

%5
普通的凡人非常少,毕竟要进入这里,每人得需要一百块元石。很多蛊师纵然有贴身家奴在身边服侍,也不愿花这份冤枉钱。

%6
内城的照明,普遍取用了一种火炭石。

%7
火炭石燃烧长久,并且没有烟雾产生。方白二人平均走百步,就会看到墙壁上挖开洞,一堆火炭石在里面燃烧着。

%8
尽管火炭石散发的热量不高,但是这么多的火炭持续不断地燃烧着,仍旧让内城中空气的温度拔高,并且干燥。

%9
不像外城,各种建筑物都有,杂乱得很。这里的建筑物,已经统一构造,外形相似,都用了耐热的红色岩石。

%10
各种支路,从街道两旁延伸出去。

%11
同时,街道上每隔五百步,都会出现一根巨大的圆柱。

%12
圆柱表面塑造了螺旋形式的石梯,绕着圆柱延伸向上,石梯外侧有护栏。

%13
通过石梯,人们可以去往上一层,或者下一层的街道。

%14
内城并非是寻常意义的城池,而是立体的。从上至下,无数街道,房屋林立,相互贯通,四通八达。

%15
方源和白凝冰一路向山内前行,这里还不是他们的目的地。

%16
到了某处地方,又出现关隘。

%17
把手的蛊师,修为更高,防御更加森严。

%18
“二位请住,有令牌么?”守卫拦住方白二人。

%19
商家针对不同身份的人物,发放各种权限的令牌。

%20
“我们还是首次来此。”方源道。

%21
他当然没有令牌了。

%22
“既然如此,请每人缴纳两百块元石。”守卫道。

%23
方源交了元石,守卫放行。

%24
二人由此来到第四内城。

%25
商量山经过商家数千年的经营,整个山体内部都被商家改造,挖开通路,雕塑建筑,分区布局。

%26
因此内城极大,从内而外,分外五区。

%27
第一内城也叫中央内城,是商家政权中心,亦是军事重地。

%28
第二内城,别名家城,是只提供给商家本族子弟居住。

%29
第三内城,环境优雅,空气清新,是高档区。

%30
第四内城,是中档区。第五内城,是低档区。

%31
再往外,就是外城,人流量极大,货物装卸之处,管理较内城而言,比较混乱。

%32
这种建筑构造,好比是地球上的白蚁山。

%33
白蚁山是高达四米到十米,蚁群在里面生活,细小的通道相互勾连,繁复精细至极。

%34
二人一进入第四内城,顿感空气湿润,气温渐降。

%35
第四内城比第五内城,要高一个档次,不仅体现在入口税多了一倍,还体现在方方面面。

%36
首先,采光在不用廉价的火炭石,而是大量地种植了一种一转草蛊。

%37
名称为月光爬山虎。

%38
这种藤叶植株,附着在街道两旁的洞壁上,蔓延开去,随处可见。

%39
它们的根茎是深蓝色的,叶片又宽又大,散发着微微的淡蓝月光。一段通道,就有成千上万的叶片,柔和的蓝光连绵一片。

%40
因为大量的藤叶,这里空气潮湿,水汽蔓延。在贴近地面的地方,沉降凝聚成雾气。

%41
月光在雾气中折射,形成光雾缭绕。让人走在街道上,有一种漫步仙境之错觉。

%42
这里的建筑,已经多了雕纹装饰,有人造的草坪,摆放着花坛,偶尔间还有假山,亭台等等。

%43
路上行人,更加稀少。

%44
二转蛊师已经成为主流,毕竟对于一转蛊师来讲,单单入口的两百块元石,就是一笔很大的支出了。

%45
最明显的感觉是,走在第五内城,街道上还很嘈杂。到了这里,就安静许多了。

%46
二人一路向深处进发,来到城门口。

%47
“没有令牌,二位要进入第三内城,就得缴纳六百块元石。”守卫蛊师中的头领,已经是三转修为。

%48
方源交了元石,终于来到第三区。

%49
这里又和第四内城不同。

%50
所有的建筑石料,都采用了星星石。

%51
这种石头,是蛊师广为采用的炼蛊辅料,在黑暗中能散发出璀璨的星光。

%52
整个第三内城,都采用星星石。不仅是建筑物,甚至连街道上都铺着星星石料所制的石板。

%53
放眼望去,星光连绵一片,视野清晰,再无光雾阻挠。

%54
空气也清爽无比,放眼处,亭台楼阁,红墙绿瓦。更移栽了竹林、名木,打造了假山,甚至勾引了泉水,流水潺潺。

%55
街道上行人稀少,幽静怡人,恍若星宫。

%56
“真是财大气粗啊……”白凝冰稍稍估算了一下,单就眼前范围的设施造价,就是一个令其眩晕的数额。

%57
商家乃南疆财富第一,有人说拔根腿毛来,都比其他人的腰还粗。此话虽然夸张些,也并非空穴来风。

%58
商家富如山,整个商家城就是一座立体的大山。商家行商,遍布南疆各处。

%59
商家的财富究竟有多少,没有一个外人能说得清。

%60
但方源至少知道,单单这个第三内城的造价,就能抵得上十几个古月山寨的财富总和。

%61
到了此处,连二转蛊师都少见了。

%62
偶尔间,两人见到一位蛊师,几乎都是三转级数。

%63
这里便是方源的目的地了。

%64
再往深处进发的话,就是第二内城。

%65
但要进入第二内城,就不是元石的问题了,还需要商家颁发的令牌。并且这令牌的规格要高到一定程度。

%66
“通幽商铺。”方源看了眼牌匾,迈步走了进去。

%67
这是间买卖蛊虫的店铺。

%68
“两位贵客,请里面雅座。”一位负责接待的少女,立即走了过来,轻声细语。

%69
她气息流露,竟然是位一转蛊师。

%70
方白二人虽然都穿着凡人的衣衫,一个丑陋,一个落魄,但这位蛊师少女却仍旧态度恭敬,显现出优秀的素质。

%71
方源和白凝冰被引入雅室。

%72
这是单独的房间,檀木桌椅,雕梁画栋,洁白的墙上挂着字画,字体龙飞凤舞,笔力刚虬。

%73
透过窗棂,可看见庭院,院中青木红花,鸟鸣啾啾。

%74
蛊师少女奉上两杯香茶,就退了出去。

%75
她前脚刚刚离开,后脚就有一位老者走了进来。

%76
“不知二位贵客,来到鄙店,想买还是想卖?”老者是个二转蛊师,脸上堆着笑,拱手问道。

%77
“既是想买,也是想卖。”方源一边端起杯盏,一边答道。

%78
老者哈哈一笑,两道光从他体内飞出,分别悬浮到方源和白凝冰的面前。

%79
是两只书虫。

%80
书虫只是一转蛊虫,却相当珍贵,堪比酒虫。

%81
在市面上刚一露面,就会被人抢购。因此常常有价无市。

%82
它形如蚕蛹,虽分头、胸、腹三个体段,但通体如纺棰一般,有些浑圆可爱。

%83
它浑身洁白,表面像是涂了一层釉质,带着油亮的光。

%84
摸在手中,也是光滑润和,仿佛是上佳的瓷器。

%85
书虫是存储类的蛊,和兜率花类似。

%86
区别只是,兜率花存储实物,而书虫却存储无形的知识和讯息。就算是自毁,也只是炸成一蓬无害的白光。

%87
“二位请细细浏览。”蛊师老者道。

%88
两只书虫都被他炼化,算是借给方源和白凝冰的。

%89
方白二人各调动一股雪银真元,灌注到书虫中去。

%90
书虫顿时化作一道白光,汇入两人的额头眉心之中。

%91
顿时,方源和白凝冰的脑海中,都多了一段丰富的信息。

%92
这些内容,就好像是反复背诵之后,深深地印在脑海当中的一样。

%93
白凝冰暗暗咂舌,通幽商铺出售的蛊虫之多,简直成千上万!种类繁多,令人有眼花缭乱之感。

%94
其中,不乏有书虫、酒虫之类的珍稀蛊虫。从低到高,涵盖一转到五转。

%95
当然,六转是绝对没有的。

%96
每种蛊虫,都有详细的简介,说明了作用。更标有价格,有些阶位较高,珍贵稀少的蛊虫,还特别详注了各种令牌标准。

%97
这就意味着,蛊师只有拥有商家发布的令牌,才有资格购买这些蛊虫。

%98
方源需要大量购买蛊虫,粗略浏览了一遍后,他收回心神,将书虫唤出,还给了老者。

%99
他现在身上虽然有上万块元石,但距离收购这些蛊,还有巨大的缺口。

%100
关键不仅是钱的问题,一些计划收购的蛊,还有令牌标准。

%101
“这位客人,看中了什么?若是购买的量多,我们通幽商铺还有相应的折扣。”老者微笑着道。

%102
看到方源和白凝冰的雪银真元后,他语气更加客气。

%103
方源摆摆手:“不忙,我这里有蛊虫要卖。”

%104
言罢,唤出一枚骨枪蛊。

%105
老者也不意外,来通幽商铺买蛊虫的多,卖蛊虫的也不少。

%106
他捏住骨枪蛊,只看了地一眼,脸上就不由地流露出一丝惊诧之色。

%107
骨枪蛊他从未见过。

%108
他当然没有见过。

%109
这是灰骨才子的独门蛊虫,从不在市场上流通。

%110
“请贵客指点。”老者面容肃穆,拱手道。

%111
方源点点头,喝了一口茶:“好说,这名为骨枪蛊,乃是一族蛊虫。我借给你用,你试试手便知。”

%112
老者当场试演了一番,沉吟道:“这枚蛊,虽然只是一转,养的也不太好。但攻击不俗,又奇特,可值这个数。”

%113
老者伸出手掌,比了个数字。

\end{this_body}

