\newsection{黄龙江上竹筏倾}    %第一节:黄龙江上竹筏倾

\begin{this_body}

%1
黄龙江,南疆第三江,全长八千多公里,发源于黄果山,流经玄冥山、龟背山、青茅山、白骨山、雷磁山等,最后流入到海。

%2
如果鸟瞰整个南疆地图,黄龙江如几字形,贯穿了南疆一半有余的面积。

%3
几环咆哮卷沙腾,一路狂涛气势宏。 裂岸穿峡惊大地,带云吐雾啸苍穹。

%4
黄龙江水流湍急,黄水滔滔。河中鱼鳖蛇蚌,别有生机。此刻,河面上,一只竹筏在水浪中颠簸流离。

%5
这碧青竹筏相当的破烂,伤痕累累。竹筏中央竖着一根简陋的桅杆,挂着白色的破旧风帆。桅杆周围堆着物资,起到稳固重心的作用。竹子之间则用麻绳捆扎着。一些地方,箍了又箍,显然是在江面上,又临时紧急加工了许多次。

%6
江水滚滚向前,竹筏乘着水势,随波逐流。

%7
江水每一次拍击,都让竹筏发出不堪重负的声音,听着让人提心吊胆。

%8
这个似乎随时要散架的竹筏上,载着两个少年。

%9
一个少年郎,面容普通,身穿黑袍,黑眸黑发。另一位则是少女,一身白袍,蓝眸银发,盛颜仙姿。

%10
正是方源和白凝冰二人。

%11
自从青茅山一战,白凝冰自爆北冥冰魄体,将天鹤上人暂时困住后。他们费力破冰而出,斩了青矛竹,扎了这竹筏后,便立即跑路远遁。

%12
方源的千里地狼蛛已经死了,白凝冰的白相仙蛇,在之前就主动飞走,再无音讯。

%13
两人没有蛊虫代步,单凭自身脚力,速度太慢,必定会被天鹤上人追击到。因此方源就只好采取了这个办法。

%14
黄龙江在青茅山有着分脉支流,当初那只五转的吞江蟾,就是顺着黄龙江的主河道,意外地流落到青茅山脚边的。

%15
竹筏从支流,汇入到主河道,一路顺江而下,一日千里有余,速度自然是极快的。

%16
“已经过去了五天,看来那老家伙,是不会来了。”方源立足在竹筏上,回望身后一眼,喃喃地道。

%17
竹筏的速度,终究快不过铁喙飞鹤王。但铁喙飞鹤王毕竟是兽力,总得要休息,比不得竹筏借助水势,延绵不绝。时间越长,方源就越安全。

%18
况且,方源记得:当初天鹤上人斩杀了古月一代后,是独自一人回来。铁喙飞鹤王极可能已经死亡。

%19
耳边江水滔滔轰鸣,白凝冰看了方源一眼,她虽然听不清方源话的内容,但也知方源的意思。

%20
她哈哈一笑:“有什么好担心的!那老东西若是追来,我们反身死战就是。在这黄龙江上作战,肯定十分精彩。不过,如果死在这里,恐怕要给鱼虾果腹了。呵呵,这也挺有趣。”

%21
方源没有理睬她,而是凝视远方。

%22
算一算,五天的水路,已经很接近白骨山了。

%23
在他的记忆中,白骨山中藏有一个密藏传承,乃是一对正道四转蛊师所设,留待有缘人。

%24
“白骨山的传承,我前世并未亲自经历,只是耳闻。但据传这传承当中,有些关卡需要两人同心协力,才能通过。”

%25
方源想到这里,不着痕迹地看了白凝冰一眼。

%26
他虽然和白凝冰同行,但只是碍于情势,有强敌压迫。他自己又只是一转初阶,闯荡外界,非得有援手不可。再加上,白凝冰成了女子。而自己掌握了阳蛊,等若抓住了她最大的把柄,令她不得不选择妥协。

%27
如果真的进入白骨山,自己和白凝冰真的能同心协作吗?

%28
这是一个相当大的问题。

%29
崩。

%30
陡然间,一声闷声炸响。

%31
“不好,绳索又脱落了。”这声音太令白凝冰熟悉了,她立即开口惊呼道。

%32
江水力道猛烈,这五天来不知多少次,将捆绑青矛竹的麻绳冲烂掉。不过幸好方源出发前,准备得很充足。

%33
“快点取麻绳,这里暂时有我。”方源连忙蹲下身子,用双手按住分裂开来的地方,使糟糕的局面不在扩大。

%34
江水冲势迅猛,把住竹筏需要十足的力量,白凝冰远不能胜任,唯有拥有双猪之力的方源。

%35
好在之前这种情况发生了很多次,白凝冰处理起来也有了经验,连忙去取竹筏中央的简易桅杆上的麻绳。

%36
“来了,来了!”她赶忙过来,并且递去麻绳。

%37
方源麻利地取来,迅速缠绕,忙得满头大汗,圈了好几道麻绳,这才勉强将这边固定住。

%38
“竹筏已经破损不堪,按照这种态势,只能再坚持一天。一天之后,我们就得靠岸。”方源叹了一口气。

%39
黄龙江并不安全,滚滚的江水当中不知道潜伏了多少的危险。若是竹筏在江水中央崩溃,方源和白凝冰落下水,皆会有无法预测的生命危险。

%40
咄。

%41
忽然一声轻微的闷响。

%42
“什么声音?”方源顿时皱眉。

%43
白凝冰侧耳,表情疑惑:“有什么声音吗,我怎么没听到?”

%44
方源耳廓生出参须,几乎紧接着,咄咄的声音,连绵不断。竹筏随之不断的轻微震动。

%45
“江水里有什么东西,正攻击竹筏!”白凝冰惊呼一声。

%46
一道黑线,嗖的一下,从竹筏旁的江水中飞射而出,和白凝冰擦肩而过。

%47
这黑影速度极快,几乎视线都捕捉不到。白凝冰只觉得耳边一凉,脸颊上有股液体流下。下意识地伸手一摸,是血!

%48
“这是什么鬼东西!”她咒骂一声,仰天望去,只看到一条梭状的黑鱼,从半空中落入江水当中。

%49
“是梭箭鱼,该死的,赶紧靠岸!”方源大叫,连忙去扯风帆。

%50
这梭箭鱼两头尖,中腹大,如同梭子。只有在大江大海中,才有其身影。常常上百只,或者上千只出没。它们食肉,成群结队地出发,常常狩猎比它们体积大上数十倍,甚至数百倍的猎物。

%51
嗖嗖嗖!

%52
一支支黑箭,从江面下激射而出。

%53
竹筏急剧震动,大量的梭箭鱼射中竹筏。好在青茅竹,乃竹中佳品,十分坚硬,堪堪抗住。大量的梭箭鱼一头插在竹筏底部,使得竹筏危如累卵。

%54
风帆调整,借助江风,竹筏倾斜方向,向江边快速靠去。

%55
但江水中的梭箭鱼群,并未有放弃。大量的黑影潜游急窜,猛烈冲击。

%56
啪。

%57
一根青矛竹破开,一条梭箭鱼撞破竹筏,动势已尽,落在白凝冰的脚边。

%58
它浑身鳞片紧凑,鱼头呈锥子状,闪着幽光。白凝冰看着它干瞪眼,她自爆之后,浑身上下的蛊虫皆被冻死。阴阳转身蛊只是救活了她,却没有能力令她的蛊虫复生。

%59
啪啪啪。

%60
紧接着,大量的竹片破裂之声传来。

%61
竹筏支撑住第一波冲击,已经足够优秀。再也不能支撑第二波。

%62
江水弥漫,竹筏破损,开始沉没。

%63
“快,快,快!”方源咒骂着,保护风帆。风帆若失,竹筏就无动力,方源和白凝冰落入江水,必死无疑!

%64
梭箭鱼群酝酿出第三波攻击,大量的梭箭鱼如箭雨逆射,竹子洞破,麻绳破裂,竹筏开始大崩解。

%65
天蓬蛊!

%66
方源勉强催动三转蛊虫,顿时空窍中的真元海,以一种恐怖的速度在暴降。

%67
这还是他有九成甲等资质,且有天元宝莲的情况下。

%68
一转初阶的青铜真元,质量上难以满足天蓬蛊的要求。

%69
就算是凝成的白光虚甲,也虚弱不堪,不复三转修为的气象。

%70
砰砰砰。

%71
梭箭鱼撞在白光虚甲上,发出一阵阵的闷响,不能伤害方源。但白凝冰却已经负伤,她在竹筏上疯狂闪避,躲避着竹筏下射上来的梭箭鱼。同时站在方源身后,靠他抵挡了大部分的攻击。

%72
情势危急无比,风帆也被射破,许多烂洞布满帆面。竹筏的动力越来越小。竹筏还剩下三分之一不到,大量的江水漫上来,盖过脚面,几乎就要沉没了。

%73
“该死的,难道五转蛊师都杀不死我,我却要死在这群小小的梭箭鱼上吗?”白凝冰长叹。

%74
再有一波冲击,竹筏必定崩溃,他们落入水中,必死无疑。

%75
然而……

%76
梭箭鱼群的攻击,迟迟未至,令白凝冰屏气凝神,提心吊胆。

%77
“靠岸了,梭箭鱼群不会游到浅水区的。呼!我们暂时安全了。”方源吐出一口浊气,全身酸软无比。

%78
这些天来,他几乎不眠不休,掌控风帆,时刻调整竹筏漂流方向。几乎已经达到了体力的极限。

%79
白凝冰也狠狠地抽着冷气,她一身白袍都被血染红,身上伤害数十处,幸好她有战斗天赋,极力躲避,又有竹筏削减梭箭鱼的冲势,因此都是轻伤。

%80
方源看了一眼白凝冰,旋即自己身上也传来阵痛。

%81
他也负伤了,血流不止。

%82
天蓬蛊只是催动了几分钟,空窍中的真元海就彻底干涸。没有了防护,他以血肉之躯,自然抗不过梭箭鱼。

%83
原本还计划着,再漂流一天。

%84
但天有不测风云,人有旦夕祸福。计划永远赶不上变化,距离白骨山还有一段距离,但方源此刻,必须靠岸了。

%85
风帆已经失去作用,方源尽最大的努力,这才让竹筏避过礁石,搁浅在一处滩地上。

%86
两人淌水,踩在松软的沙滩上,上了岸。

%87
白凝冰捂着伤口,一屁股坐下来。她脸色苍白:“这样下去,必定失血过多,凶多吉少!你身上有什么蛊虫可以治疗,快拿出来。”

%88
方源苦笑,他哪里来什么治疗蛊?

%89
(ps:新的大章,新的开始。上一章魔性不死,写了许多人的魔性,正邪思想的碰撞,魔道的思想理论。这一章魔子出山,写闯荡,写挣扎,写奋发。行文更加明快,节奏突出。大多数读者朋友们应该会更喜欢。)(未完待续。

\end{this_body}

