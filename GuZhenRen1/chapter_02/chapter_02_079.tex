\newsection{何等天真!}    %第七十九节:何等天真!

\begin{this_body}

%1
(ps:有两个无伤大雅的bug。上一章节有一个,是我码字码快了,没有解释清楚,已经做了修改。感兴趣的朋友,可去起点网看看。还有一个漏洞,是统计方源空窍中的蛊,少了一只焦雷豆母蛊,还有一只饭袋草蛊。这个也添上去了。感谢热心的读者大大提出来,十分感谢!另外:下次如果更新异常,会提前通知的。不会再让大家苦等。前一次是我做的不到位,在这里向大家道个歉。其实,我是有节操的……)

%2
密室中,弥漫着清晰的星光。

%3
方源盘坐在蒲团上,开始着手炼蛊。

%4
和白凝冰分别之后,他又去了一次商铺,买齐了炼蛊所用的一切。当然为了防止秘方暴露,他也添加了一些不需要的东西,充作掩饰。

%5
他心念一动,先唤出一只蛊来。

%6
这只蛊,像个球形气囊,拳头大小,橘黄色。表面皱皱巴巴的,有些厚实,手感不是很舒服。

%7
这是一转的臭屁蛊。

%8
唯一的作用,就是散发出恶臭味,没有攻击力。

%9
别看此蛊似乎作用不强,但被广泛用作炼蛊材料之一。

%10
方源默默灌注淡银真元,使得臭屁蛊通体缓慢膨胀,悬浮在空中。

%11
与此同时,他伸手拍开身前的一个泥坛。

%12
顿时一股腐烂的臭气,弥漫开来。

%13
这个坛子中装着满满的腐臭沼泥,黑漆马糊的。这种泥,也是一种被广泛运用的炼蛊辅料。

%14
“饭袋草,去。”方源心念一动,从空窍中射出一颗种子。

%15
种子射入到泥坛之中,随着空窍中真元海面下降,在腐臭沼泥中,钻出墨绿色的藤蔓。

%16
藤蔓不断延展,探出泥坛,覆盖方圆两尺之地后,抽出枝头,长出硕大的叶片。

%17
这些绿叶,饱满似桶。起先很小,迅速膨胀成碗一般大小。

%18
方源一一摘开这些叶片顶部,露出里面的洁白饭粒。

%19
但方源却没有取用,而是将这些饭粒都倾倒在腐臭沼泥当中。随后他缓缓地降下臭屁蛊,一心两用,同时催动臭屁蛊以及饭袋草。

%20
臭屁蛊散发出橘黄色的光,和饭袋草散发出的绿芒,渐渐融合混杂在一起。

%21
混杂的光辉渐渐内敛,深入泥坛。

%22
泥坛不断震动,最终停止下来,从中飞出一只新蛊。

%23
通体黄褐色,烂泥一般,散发着微微臭气,仿佛一坨大便。

%24
这就是屙屎蛊。

%25
别看它其貌不扬,但能令人腹泻。很多医师都用它来治病排毒,很具有实用价值。

%26
方源面不改色。

%27
蛊虫千千万万,犹如形形色色的人。有圣洁高雅的天元宝莲,有充满哲思禅意的春秋蝉,有邪意鬼魅的血颅蛊,当然也会有些看起来恶心,闻起来恶心,摸起来也恶心的屙屎蛊、臭屁蛊等等。

%28
合炼出二转的屙屎蛊,还只是第一步。

%29
方源接着又取出大力蛊,同样采购于商铺。此蛊能在五十个呼吸之内,令蛊师暴涨一倍的气力。

%30
方源用大力蛊、屙屎蛊,合炼成了大力屙屎蛊。

%31
大力屙屎蛊是三转蛊虫,作用和屙屎蛊相同,外形也相似,但真实效力强大了三倍有余。

%32
大力屙屎蛊就不如屙屎蛊受欢迎了。

%33
并非蛊虫转数越高越好。

%34
三转蛊虫吃的比一转蛊虫多得多,大力屙屎蛊虽然效力强大,但是多使用几次屙屎蛊也能达到相同的效果。也有庸医误诊,滥用大力屙屎蛊,让人腹泻脱力而亡的事情。

%35
有了大力屙屎蛊,就是第二步。

%36
接下来,方源便取出狮吼蛊。

%37
此蛊形如狮头,一经催用,就能化作一个石磨大小的狮头虚影,爆发出狮吼般的强音。一般用来发信号,吸引或者惊跑野兽,作战的时候用的好,也能让敌方蛊师乱了阵脚。

%38
方源同时催动狮吼蛊,以及大力屙屎蛊。

%39
于是半空中出现这样的一幕,狮头虚影张开大嘴,开始吞吃大力屙屎蛊。

%40
但是吃了三分之一的量,狮吼蛊就吃不下去了,狮头虚影更有一种要崩解的趋势。

%41
于是方源心念一动,调出一只从商铺中购买而来的纸鹤蛊。

%42
这蛊分有淡蓝、深青、粉红等等几色,专门用来传信。

%43
淡蓝色的纸鹤蛊晃晃悠悠地飞向狮吼蛊,然后被狮头虚影一口吞下。

%44
一心三用!

%45
若是有外人看到这一幕,必定会吃惊得叫出声来。

%46
寻常蛊师,一心两用就已经很了不起了。一心三用者,寥寥无几,这是一种令蛊师成就炼蛊大师的强大天赋!

%47
而寻常的炼蛊师,都得借助其他的辅助蛊虫,才能做到一心多用。

%48
方源同时催动狮吼蛊、大力屙屎蛊以及纸鹤蛊,只是凭借他自身之力。

%49
但是对他而言,这却并非天赋,而是经验的积累。

%50
任何东西,都能熟能生巧。方源用了五百年的时间,慢慢积累,艰苦训练数百年。从一心一用,到一心二用,再到一心三用,甚至有时候可以达到一心四用!

%51
当然再往下,他就必须借助辅助蛊虫之力了。

%52
狮头虚影吞下纸鹤蛊后,顿时光线不再涣散,凝结如初。

%53
方源继续催动它,吞食大力屙屎蛊。

%54
又吞下三分之一后,狮头虚影故态重萌,有涣散的危机。

%55
方源毫不慌乱,冷静如常,取出第二只深青色的纸鹤蛊。

%56
如此施为,到了狮头虚影将大力屙屎蛊全部吞下,又变得不稳定起来。

%57
方源再用第三只粉红色的纸鹤蛊。

%58
狮头虚影这次吞下之后,原本是白色的光影,此时忽然缩小到原来体积的一半,并且开始绽放出淡蓝、深青、粉红三色。同时狮头虚影不断发出:“吃屎,吃屎……”的洪亮声音。

%59
声音在密室中回荡往复,震得方源双耳嗡鸣

%60
方源并不担心声音泄露,这是专门的密室,有坚固的防御力,隔绝内外。

%61
他心中一喜,这蛊到此时已算是炼成!

%62
这就是言而无信蛊!

%63
但这言而无信蛊,只能存在十个呼吸的功夫。这个时间一到,它就会自动灭亡。

%64
方源不敢怠慢,将言而无信蛊催使到自己的身边。

%65
在淡蓝、深青、粉红三色光辉的轮番照耀下,方源的全身上下,显露出一股深紫色的烟气。

%66
“吃屎,吃屎……”

%67
这些烟气在狮头虚影不断的咆哮声中,不断颤抖翻腾,两个呼吸之后,所有的烟气凝聚成两只毒誓蛊的虚影。

%68
方源心念一动,狮头虚影张开大口,一口就吞下一只。随着它的大力咀嚼,毒誓蛊发出尖锐的不甘的鸣叫,然后戛然而止。

%69
狮头虚影又再度张口,一口吞下另外一只毒誓蛊。一番咀嚼,毒誓蛊被消除干净,方源的身上一丝深紫烟气都没有留下。

%70
几个呼吸之后,狮头虚影砰的一声轻响,化作漫天的三色光雨。

%71
光雨稀稀疏疏地落下,照的密室一片绚烂。

%72
方源在彩色的光雨中,嘴角翘起,勾勒出一丝冷笑。

%73
毒誓蛊?

%74
呵呵呵呵呵呵……

%75
想用毒誓来束缚我,何等天真!

%76
不过,倒也不能怪他们。前世两百年后,这言而无信蛊才悄悄现世。

%77
它源自西漠一个部落酋长之手,又过了数十年,才渐渐传遍天下。又过了十多年,便有人研究出白纸黑字蛊,取代毒誓蛊的位置,就算是言而无信蛊也不能破解。

%78
但是没有了言而无信蛊,此后百年后,就出现了混淆黑白蛊,能破解白纸黑字蛊的约束力量。

%79
毒誓蛊、言而无信蛊、白纸黑字蛊、混淆黑白蛊,这只是约束与破解之间的斗争角逐中的一小段。在此之前有,在此之后还会延续下去。

%80
方源拥有五百年经历,等若领先整个世界五百年!要破解毒誓蛊,并不困难。

%81
毒誓蛊的约束一去,方源顿感轻松。更妙的在于,他只是消去了自己这边的约束,而商睚眦、白凝冰身上却仍然有毒誓蛊的力量。

%82
如此一来,对付他们两个人,方源就牢牢地占据主动了。

%83
“商睚眦此人气性狭小,我占了他的大便宜,必定会报复我。不过我亦早有谋算,就让他暂且先蹦跶一会儿。至于白凝冰,如今已在我手掌心中拿捏,短时间内可以利用,倒不急着翻脸。”

%84
“这些天,元石一下子分出一半去。又连续买了许多蛊,耗费了数万元石。再算上请客吃酒,支付楠秋苑的房租,还有赌石坊的费用,购买蛊虫食料等等,手中的元石只剩下四十二万。”

%85
人无远虑必有近忧,方源想到这里,就皱起眉头。

%86
他要凑齐一套蛊虫,这些钱看似很多,但实际上是不够用的。

%87
普通的一转蛊虫,市价在五百元石以内。二转蛊在五百至一千元石内,三转市价普遍不低于一千元石,不高于一万元石。

%88
到了四转蛊,就是一万到十万之间。五转蛊十万以上,百万以下。

%89
六转蛊从未有卖过的。

%90
以上的价格,只是针对普通蛊虫。一些珍稀蛊,譬如酒虫,虽然是一转蛊,但却是二转蛊的价格。

%91
而舍利蛊的价格更离谱。

%92
青铜舍利蛊的价格一般在两千元石左右,赤铁舍利蛊上升到八千,白银舍利蛊暴涨到五万,黄金舍利蛊接近三十万!

%93
方源手中有四十二万元石,还买不起两只黄金舍利蛊的。

%94
舍利蛊都是天然蛊,无法通过蛊师合炼而得。很多时候,有钱还买不到。

%95
“我要在商家城休整两到三年,四十二万恐怕连我一年的修行都支撑不了。蛊师修行,依赖大量资源。境界越高,所需钱财就呈几何形式暴涨。接下来我出入赌石坊,更得花钱。有必要寻找一个赚钱的行当来做。”

%96
商家城繁华似锦,竞争压力极大,但处处都有机会。

%97
只要你有能力,赚钱很容易。没有能力,就被淘汰。

%98
这里是另一个大自然,仍旧充满了残酷的竞争。

%99
方源结合前世经历,以及今生见解,对于赚钱,心中已经拿定主意。

%100
那就是演武场。

\end{this_body}

