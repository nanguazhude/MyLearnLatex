\newsection{春秋蝉!}    %第二百节:春秋蝉!

\begin{this_body}



%1
方源缓缓仰头,透过青铜大殿的窟窿,望着天空。

%2
福地的苍穹已经形成巨大漏洞,沟通着外界。以方源的角度,能够看到南疆的太阳。

%3
太阳低垂,已近黄昏。

%4
“申时三刻,这个时辰,凤金煌应该刚刚继承狐仙福地吧?”方源想着。

%5
前世,他伙同魔道蛊仙,进攻狐仙福地,攻占荡魂山。最后付出极大代价,斩杀凤金煌,险死还生。

%6
凤金煌死后,正道齐哀,为其作《凤金煌传》。传中写明生平事迹,凤金煌一生有三大奇遇。

%7
第一个奇遇,是三岁时睡觉,在梦中就获得仙蛊梦翼。

%8
第二个机缘,就是狐仙传承。在今日申时两刻,成功登顶。

%9
“凤金煌有仙蛊梦翼,狐仙传承必是她的囊中之物。要不被她落下太远,就得拥有第二空窍。可惜啊,到头来功败垂成!”

%10
他在心中叹息,涣散的目光却渐渐重新凝结起来。

%11
他还有翻盘的希望,他还没有输得彻底!

%12
因为他还有——春秋蝉。

%13
无极搜锁,能封禁五转蛊,但是却禁不住六转的仙蛊!

%14
方源抬眼,最后看了一眼白凝冰等人。没有什么好说的,这一次,若成功还有希望,失败一切休提。

%15
“他怎么如此平静?”白凝冰、铁若男等人,都在此刻感到莫名的不妙。

%16
砰。

%17
方源悍然自爆!

%18
刹那间,春秋蝉绽放出青、橙两色光辉,一股玄奇奥妙的气息,扩散开来。

%19
方源所有的蛊虫,全部的真元,一切的血肉和魂灵,都自爆开来。

%20
每一次催动春秋蝉,都是一场豪赌。

%21
方源身处绝境,毅然赌上自己所有的一切!

%22
自爆的一切。都尽数灌注到春秋蝉之中。

%23
春秋蝉化为精芒一点,承载着方源仅剩下来的一点意志,破碎虚空,来到这个世界中最著名的秘禁之地——光阴长河。

%24
蛊师世界有南疆、北原、西漠、东海、中州五域环绕,这是宇。又有一条光阴长河,贯穿过去、现在、未来,便是宙。

%25
宇和宙是构成世界的基础。

%26
哗哗哗……

%27
光阴长河中河水澎湃。潮起潮落,波涛汹涌。

%28
每一滴的光阴之水,都是苍白色的,但是亿兆兆的水滴,每一次的相互撞击、纠缠、旋转,都会迸发出最灿烂炫目的流光溢彩。

%29
在这苍凉而又缤纷的河水当中。春秋蝉如流浪的游子,回归到家乡,双翅振动,载着方源的意识,不断逆流而上。

%30
对于方源来讲,这已经是他第三次催动春秋蝉了。

%31
第一次,他进入光阴长河。没有丝毫的经验。第二次,他无奈自爆,时间太短。

%32
这一次,他有了心理准备,终于体会到逆流而上的感觉。

%33
这种感觉,是如此的玄奇美妙。无数的光影,好像是电影倒放,呈现在他的意识当中。

%34
仿佛是过了一个呼吸。又好像是漫游了无数年头。

%35
险恶的巨浪,一次次拍打过来,春秋蝉很快就失去了活力,自爆的能源消耗殆尽,它挣扎了一下,一个猛子扎进其中的一个水滴当中,消失不见。

%36
方源一眨眼。眼前的景象已经大变!

%37
“慢着,慢着,一切都好商量。我可以答应你,将正确的路径告诉你。但是你必须保证我的生命安全。我身上就有一只毒誓蛊……”

%38
耳边,熟悉的求饶声传来。方源循声低头望去,便看见脚下的王逍。

%39
他楞了一下,继而心脏砰砰跳动,脸色变化,流露出遮掩不住的狂喜之色。

%40
“哈哈哈,成功了,我又成功了,我赌赢了,又再次重生过来!”方源振臂大呼,仰天大笑。

%41
王逍:“……”

%42
方源陡然间这样的表现,让他心里又惊又疑。

%43
“什么重生?什么又成功了?这家伙该不会精神不正常吧?不过话说回来,魔道蛊师中疯魔的大有人在。该死,我居然碰到这样的一个疯子!”

%44
被方源踩在脚底,躺在地上的王逍,这样想着,求饶声不禁更大了。

%45
“哈哈哈……”方源笑声不绝,这种从绝境中逃出生天,一切重来的感觉真是太爽,太美妙了!

%46
他首先查看了自己的空窍。

%47
空窍中央,作为本命蛊的春秋蝉,再度萎靡虚弱。原本光滑油亮的外表,如今像是枯黄的秋叶。

%48
方源念头一动,春秋蝉便渐渐隐去身形,陷入沉眠,重新汲取时间的力量恢复去了。

%49
“这样一来,春秋蝉这个内患,就暂时解除了!”方源的笑声更大了,眼中精芒闪烁不停。

%50
他又打量周围。

%51
这明显还是在福地中,脚下踩着的是王逍,身边还有一个尸体,就是云家的少族长云落天。

%52
他是被白凝冰消灭的,尸体被地灵耗费仙元,挪移了过来。

%53
一想到白凝冰,方源笑声陡然一止,再笑不下去。

%54
就是这个家伙,筹谋良久,忽然反叛,令自己纵然炼成了第二空窍蛊,也功败垂成,陷入绝境。若不是有春秋蝉,若不是这次运道也很好,方源就彻底栽了。不管是死亡,还是被关押进镇魔塔,都再无翻身的机会。

%55
魔道就是这样,犹如悬崖上走钢丝,一着不慎满盘皆输。坠入山隘谷底,再也无力回天。

%56
“自己真是大意了,一心想着凤金煌,想着炼制仙蛊,反而忽略了身边潜藏的一头真魔!也是这白凝冰演得逼真,筹谋了数年,故意表现,麻痹自己,然后陡然爆发。呵,自己终究只是凡人,不是战无不胜的神啊。”想到这里,方源满嘴的苦涩。

%57
方源他在进步,在勇猛精进的成长,白凝冰同样如此。

%58
这就是现实的残酷,也是命运的美妙。

%59
在这天地中。人人都是主角,人人又都是配角。

%60
“每个人都有各自的优势,白凝冰是北冥冰魄体,铁若男有铁家大背景,凤金煌三岁就在梦中寻得仙蛊,而我则辗转挣扎,苦修数百年。炼成了春秋蝉。”

%61
方源想到这里,忽又情绪激昂起来,展颜朗笑,脱口吟道:“看万山红遍,层林尽染;漫江碧透,百舸争流。鹰击长空。鱼翔浅底,万类霜天竞自由。怅寥廓,问苍茫大地,谁主沉浮?”

%62
天若有情天亦老,大道无情至公,每一个存在都有问鼎的机会,就看如何把握。如何争锋!

%63
这个世界上,没有人生来就是一辈子的配角。也没有人,会是永恒的主角。

%64
万物相争,优胜劣汰。

%65
正是因为天下英杰的争锋,相互之间实力的碰撞,底牌优势的较量,才显得历史如此厚重精彩,世间如此丰富曼妙啊。

%66
念及于此。方源胸中一阔,苦闷和仇恨,惊悸及悲喜,都化为烟云消散。

%67
他的心一片清明,魔的执着又令他双眼熠熠闪光。

%68
他开始冷静思考。

%69
“原来我是重生到这个时候了。按照上一世的发展,我是拷问王逍前往巫山的正确之路,可惜他至死都没有说。我无奈之下。只好杀之,并用兽力胎盘蛊,吞噬了他和云落天的空窍。”

%70
方源沉默不语,神情严肃如冰。脑海中迅速地回忆着接下来要发生的事情。

%71
“之后,我费了好大劲,说出春秋蝉,才说服了地灵。又斩了杀人鬼医仇九、武神通、章三三。并在章三三的身上,意外收获到一只奴隶蛊。正要继续杀人的时候,出现意外情况,不得不救场,杀了龙青天,却碍于碧空蛊毒,没有任何一丝收获。”

%72
“再之后,就是炼蛊,防守大殿,白凝冰背叛了……”

%73
方源脑中思维如电,几乎闪电般回忆了上一世,然后他双眼眯成了一条缝,两只手不由自主地紧握成拳。

%74
“情况危急啊!”站在重生的高度,让他对局面洞若观火,上一世的疏忽变成了这一世的警惕。

%75
首先,炼制仙蛊的大体情况,白凝冰已经知晓。

%76
她秘密和铁若男合谋,铁家方面也早有准备。说不定那铁白棋,就已经隐藏在三叉山中,准备关键时刻登场。

%77
敌暗我明,不妙!

%78
其次,在炼蛊的最后几天,萧芒会赶到这里。还有意料之外的魔无天,竟然达到了五转修为,也会参加角逐。

%79
前世,方源斩杀数位魔头,导致魔道群龙无首。魔无天一来,以五转修为,轻而易举地就整合了魔道蛊师的力量。可以说,因为他的缘故,大大加快了群雄围攻大殿的速度。

%80
敌强我弱,更不妙!!

%81
最后,还有最严重的一点……

%82
方源抬起左臂。

%83
上一世自己被蒙在鼓里,这一世他已经知道:白凝冰偷偷地给自己种下定星蛊,就在自己的左前臂中。一旦铁家四老发动无极搜锁,那么他将无所遁形,逃到天涯海角,都要被虚空锁链拘拿。

%84
“现在想想,铁家之所以没有提前发动,一来是因我斩杀铁慕白,情势变化超出他们的预料。二来,也是想做得利的渔翁,觊觎胜利的果实。三者,更见他们的谨慎。无极搜锁虽然能够封禁五转蛊虫,但也有被克制的手段。万一擒拿了我,却没有搜到阳蛊怎么办?因此就将这个,留作最后的底牌了。”

%85
定星蛊悄无声息地种下,方源就如瓮中之鳖,怎么逃也逃不了。几乎已经等于身陷囹圄了。

%86
(ps:这本书是我六年梦想的实现,所以承载了许多东西,和其他书不太一样。其中有一点,就是主角方源会失败。魔从来就不是战无不胜的,魔也会失败。不描写出失败,就无法写出“魔”。任何的冒险都是有代价的,如果哪一天,方源的死会让书变得精彩,甚至升华,那么他会被我写死,那就是属于魔的悲壮。)

%87
(唉,这本书的世界,和现实差不多,人人都有成为主角的可能。也许,方源死后,我会描写方正,白凝冰或者铁若男?因为,他们的身上也蕴藏着各种各样的魔性。谁说主角必须始终一个人?大家要有心理准备啊。不想看的,不要勉强自己。书那么多,不差我这一本。爱看的,还请尽力支持我,毕竟写这本书真的赚不到什么。付出大量的时间和精力,来写这本书,我当然也希望能至少有个吃盒饭的钱啊。)

%88
(怅然……对于我来讲,青春已经渐渐远去,就疯狂这一把了。请大家理解我的离经叛道罢。嗯,从明天起,开始双更了!)(未完待续。如果您喜欢这部作品,欢迎您来起点投推荐票、月票,您的支持,就是我最大的动力。手机用户请到阅读。)

\end{this_body}

