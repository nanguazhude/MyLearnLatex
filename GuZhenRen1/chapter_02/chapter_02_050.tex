\newsection{滴水之恩,当涌泉相报!}    %第五十节:滴水之恩,当涌泉相报!

\begin{this_body}

%1
“伤亡统计出来了,目前我们还剩下一百三七人,其中蛊师七十八人,凡人五十九位。”一位副首领当众汇报道。

%2
此时,在这个破烂的帐篷内,商队幸存的头领都集结在一起,共同商讨出路。

%3
氛围凝重而且压抑。

%4
商队首领贾龙眉头紧锁,听到这里,眉头几乎拧成了一个疙瘩。

%5
原先上千人规模的庞大商队,想不到已经只剩下这么点人。七十八蛊师听起来蛮多的,但是三转蛊师,只有在座的十二位。二转蛊师有二十八位,剩下的三十八位都是一转蛊师。

%6
在这个数量上,还要剔除掉伤残者。实际上,真正还保留战力的,只剩下一半不到。

%7
凡人命贱,那些家奴可以舍弃,但是蛊师就不同了。

%8
每一个蛊师,都是家族宝贵的财富,不可能放弃。

%9
虽然贾龙知道,这些伤残蛊师对商队来讲,是一个巨大的包袱,为了维持他们的生命,治疗他们的伤势,商队原本就崩溃的后勤必将雪上加霜。

%10
但是贾龙却不敢放弃他们,一旦放弃,士气将彻底崩解,蛊师人人自危,整个商队将彻底垮掉。他们最终的下场,将是被兽群吞没。

%11
这时,作汇报的副首领话锋一转:“唯一值得庆幸的是,商队的货物还剩下不少。将那些无主的商货平均分配下来,将是一大笔的元石,可以弥补我们许多损失。”

%12
刚刚的灾难中,人死的多,货物损失较少。

%13
如果平均分配,幸存者反而能因此受益。

%14
听到这句话,帐篷中几乎所有人都眼前一亮。

%15
商人逐利,哪怕他们身处险境,也没有改变这个本性。

%16
众人面面相觑了一阵,副首领陈双金咳嗽一声:“平均分配,我觉得有些不妥。刚刚那战,我陈家牺牲最多,斩杀的白羽飞象也最多。这些无主的商货,我至少要三成!”

%17
“三成?”

%18
“怎么可能!”

%19
“你们陈家牺牲多?我尉迟家牺牲了一位二转巅峰的青年俊杰呢。”

%20
“不管怎么说,我们止家至少得要两成。”

%21
……

%22
众人七嘴八舌,渐渐争吵起来。利益当前,没有人不眼红的。

%23
唯有商心慈,夹杂在他们当中,默不作声。

%24
实力强大的队伍,希望要求更多。实力弱小的一方,则坚持平均分配。

%25
争吵声越来越大,传出帐篷外,引来无数打探的目光。

%26
商心慈忽然站起身来。

%27
帐篷中陡然一静。

%28
“诸位。”商心慈美眸扫视一圈,“当务之急,不是这些商货,还是该怎么活下去。也许兽群在下一刻就将到来!我们已经是一根绳上的蚂蚱,被命运栓在了一起。然而我们每个人的力量有限,接下来只有同舟共济,才能有生还的希望。”

%29
说到这里,她顿了一顿。

%30
“我提议,先将我们各自的商货都贡献出来,取出其中对我们有帮助的物品。在这里,我代表张家,先做一个表率。我自愿将手中所有的货物,都无偿的贡献出来。”

%31
“什么?”

%32
“无偿的贡献出来?!”

%33
一时间,许多人都惊呆了,陈双金、贾龙等人的脸上也变幻不定。

%34
“我累了,希望诸位能尽快地商讨出,行之有效的方法。告辞。”说完,商心慈点头致意,转身掀开了门帘。

%35
她走出帐篷不到五步,帐篷内陡然爆发出更高的声浪。

%36
如今增添了张家的货物,这份利益更重,叫人越加疯狂。

%37
商心慈脚步一缓,捏紧双拳,深深的叹了一口气。

%38
她也是商人,自然要逐利。刚刚所谓的“自愿奉献”,当然不是出自真心。

%39
只是如今情势所迫,她如同怀揣着巨款的婴孩,走在大人身边,不得不自保而已。

%40
回到自家帐篷内,小蝶眼圈通红,正缩在墙角,嘤嘤哭泣。

%41
她和商心慈几乎从小到大,就一直生活在一起,刚刚白羽飞象带给她的惊吓,到现在才爆发出来。

%42
“小蝶。”商心慈心中暗叹一口气,坐到她的身边安抚道。

%43
“小姐,我好怕。呜呜呜……张柱大人一直都没有回来,该不会,该不会……”小蝶一头扎进商心慈的怀中,痛哭流涕。

%44
商心慈拍拍她的后背,劝慰几句,但小蝶仍旧哭泣不止。

%45
“小蝶,张柱叔很可能永远不会回来了。”商心慈沉声道。

%46
此话一出,她明显地感觉到怀中小蝶的身躯一颤。

%47
“小姐……不会的,不会的!”小蝶抬起头来,双眼通红,不停地摇头。

%48
“接受这个事实吧,小蝶!”商心慈陡然轻喝一声,“现在只能靠我们自己了。不要哭泣,不能哭泣,哭泣解决不了任何的问题!”

%49
“靠自己?可是我们都是凡人呐,要不是张柱大人,我们这一路根本不能成行。”小蝶眼泪汪汪,情绪很低迷。

%50
“没有错。那些人如果不是看在张柱叔的份上,怎么可能让我成为副首领之一。小蝶你说的没有错,我们是凡人。但是凡人也有凡人的力量!”商心慈的眼中闪过坚定的光。

%51
她目光灼灼地盯着小蝶,双手把着小蝶的胳膊,用力摇晃一下:“小蝶,你相信我吗?”

%52
小蝶看着眼前的商心慈,感到一股莫名的力量。

%53
这股力量来源于商心慈,然后弥漫到她的身上,感染着她的心,使得她感觉自己仿佛正在被光明照耀着。

%54
“小姐……”小蝶目光闪动着,她从未看过这样的小姐,这一瞬间,她觉得小姐前所未有的明媚漂亮。

%55
“我相信你!”她轻声而又坚定地回答道。

%56
“很好,小蝶。你听好了,我们正处在危险当中。有时候人比野兽还要可怕,我们带上所有的积蓄,一起去找黑土和白云。”

%57
“好的,小姐,我都听你的!”

%58
……

%59
一盏茶的功夫后。

%60
帐篷内,方源和商心慈相对盘坐着。

%61
方源玩味地看着眼前的佳人:“你刚刚说,你已经将我们的货物都无偿地贡献出去了?给了那些贪婪的鬣狗?”

%62
“是这样子的。”商心慈坦然承认道。

%63
方源的嘴角翘起,目光中充满了对商心慈的欣赏。

%64
真是个聪明人啊,能舍能弃,有魄力!

%65
张柱一死,商心慈只是凡人,哪怕代表张家,地位也和其他副首领极不平等。

%66
这些蛊师要想吞并商心慈的财货,简直是轻而易举。

%67
将商心慈杀掉,把责任推到兽群身上,到时候谁也不能说什么。张家也不会为了区区一个商心慈,而去大肆调查。

%68
因此,对于商心慈来讲,她手中的这些商货就变得无比烫手,能给她带来灾祸。

%69
于是,她明智地将这祸端毅然舍弃,抛给了其他人,尽量地保证自己的安全。

%70
但是她又深深的明白,这种安全也不可靠。所以她来到了这里。

%71
“我是来向黑土你道歉的,很对不起。”商心慈向方源弯腰,“你借贷了这些商货,按道理来讲,这些货物都是你的。我这次越俎代庖,十分莽撞和唐突。为表达歉意,请收下这些。”

%72
商心慈这次带来了两个木箱子。

%73
不用打开,方源就知道,这木箱子中都装满了元石。

%74
有很大一部分,还是自己赚的,五五分成给的对方。

%75
方源不由地看向商心慈。

%76
两人的目光在半空中碰撞,渐生出心有灵犀之感。

%77
双方都是聪明人,很多话不用说的太明白。

%78
在商心慈的认知中,方源故意隐藏身份,很有可能就是蛊师。按照他在相处过程中的种种表现,商心慈觉得方源此人很可靠。如果谁还能帮助她,方源无疑是最佳的人选。

%79
因为,商心慈对方源很放心。

%80
然而,商心慈心中也有顾虑。首先,她不知道方源的真正实力如何,一转、二转、三转的差别很大。其次,方源隐藏身份,必有苦衷,未必可以暴露身份来帮助自己。

%81
在没有得到方源的承诺前,商心慈捐献了财货。安全有了最基本的保障后,她带着所有的积蓄,来找方源。说是来道歉赔罪,实际上是含蓄地想要邀请方源,希望得到他的帮助。

%82
她知道方源和她是一类人,这是聪明人的默契。

%83
两人对视良久,方源微微而笑,打破沉默:“这些元石,还请张家小姐带回去吧。”

%84
商心慈面色微微一变,一颗心顿时往下沉。

%85
在她的理解中,方源拒绝了这些元石,无疑是拒绝帮助她。

%86
但对此,她又能怎样呢?

%87
说起来,方源已经回报她很多。能做到这点的,世上已经很少人了。商心慈也不能要求什么。

%88
但是黑土如果不帮她,还有谁呢?白云吗?不会,他们俩男女关系亲密,自然同进同退。

%89
商心慈想不出还有什么人选,她绝美的容颜上,泛起苦笑:“我明白了。但这些元石,就当我送个黑土你的吧。匹夫无罪怀璧其罪,你若不收着,恐怕我也得贡献出去。”

%90
方源哈哈一笑:“张小姐,我想你理解有些偏差。在刚刚加入商队那会儿,我和白云负伤严重,几乎命悬一线。实力降到最低,偏偏还被一些强奴觊觎。哼,真是虎落平阳被犬欺。是你庇护了我们,给了我们俩充分休养的机会。这份恩情,说是救命大恩,也不为过。”

%91
“而我这人,有恩必偿,有仇必报。你虽是凡人,但我认同你,甚至欣赏你。我们在茫茫人海相逢,是一种缘分。滴水之恩,当涌泉相报。更何况是救命的恩德呢?元石你带回去,我必尽全力保护你!”

%92
“啊……”商心慈闻言,不禁低声惊呼。

%93
她睫毛微颤,双眼泛红,很快就氤氲一片。

%94
在她的视野中,因为泪水,方源变得模糊。但是在心头,方源的形象却无比清晰地刻印下来。

\end{this_body}

