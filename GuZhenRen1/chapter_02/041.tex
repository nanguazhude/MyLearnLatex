\newsection{掰手腕}    %第四十一节:掰手腕

\begin{this_body}

匪猴身强体壮,大如象。成年的匪猴高达一丈,浑身肌肉贲发,上臂比下肢粗壮两倍有余,猴尾如铁棍,能砸碎山石。

匪猴浑体皮毛如金,上面布满黑色的老虎斑纹。令人感到奇特的是,从它们的腰部会自然生长出皮毛,遮盖住裆部和尾部,宛若皮裙。

吼!

这只匪猴群的猴王,忽然张开巨口,发出洪亮的吼声。

它们的吼叫声,如狮虎一般雄浑。

吼吼吼……

猴王的吼声,引得群猴响应。

一时间,风云ji荡,声浪卷席,将白色的氤氲雾气都吼散许多。

众人视野陡然开阔,这才惊觉山道两旁的山峦中,站满了匪猴。足有上千只匪猴,将商队包围住。

它们身形庞大,树木和它们个头齐平。有些幼树,只及得上它们的腰部。

在商队的最前方,体型更加庞大的猴王,大马金刀地坐在一张石凳上。一个如水缸般的灰石酒坛,就摆在它的身边,酒香浓烈。

猴王吼叫了一声后,就闭上了嘴巴。其他的匪猴,仍旧吼叫不止。

这反而更彰显出猴王的气势。

它目光平静,眼眸中透着灵光,坐着不动。反观那些普通的匪猴,则盯着商队的货物,蠢蠢欲动,跃跃欲试的模样。

猴、狐、狈这类野兽,都有智慧。

这只匪猴王的智力只及得上三岁孩童,不及狡电狈,但也足够聪明,可以交流。

商队的首领贾龙眯着双眼,看着这猴王,忽道:“贾俑,你上。”

“是,首领。”贾俑站了出来。

他身材高大又肥胖,挺着一个大大的肚子,看起来就很壮实。

他是防御型蛊师,本命蛊是水甲蛊。二转修为,擅长水战。一次机缘巧合,在江河中游泳,捕杀了一只大如小舟的龟。从它的身上,缴获了一只龟力蛊。他用了之后,永久增添了一龟之力。

见贾俑走近,群猴的吼声蓦地大盛,声浪震撼山林。

贾俑满脸凝重之色,撸起袖子,战到猴王面前。

猴王身形巨大,只是坐着,还高出贾俑一头。

它看着贾俑,吼叫一声,立即就有几只匪猴,扛着一张石桌,哼哧哼哧地走过来。

石桌巨大如床,极其hou重,落在地上,发出闷响。

又有两只匪猴,搬来石凳。都摆放在猴王的面前。

猴王拍拍石桌,砰砰声如擂动巨鼓。

贾俑咽下一口吐沫,坐下来,伸出右臂,手肘搭在桌面上。

猴王亦伸出左手,两手紧握在一起。

身旁的一只母匪猴,忽然大叫一声。

贾俑和猴王听到这型号,同时猛地发力,开始了这场别开生面的较量。

匪猴崇尚力量,掰手腕是猴群中最主要的社交活动。小猴子一生下来,就能掰手腕。掰手腕,在匪猴群中不仅是游戏,更是化解纠纷的常用手段。

当年,正道蛊师冠天侯,本身只有五转修为,这样的实力自然不能杀上巅峰。他就是利用匪猴的这个习俗,不断掰手腕,打到匪猴山之巅,胜过猴皇。然后得到猴群的认同,定下协议,开通商路。

此后但凡商队进出匪猴山,都要遵照这个协约,和匪猴掰手腕。

赢了的人,将得到匪猴的承认,免受过路费。输了的人,则要任凭猴群拿取一部分的货物。

如此一来,商队可以经商,匪猴也尝到好处,乐此不疲。

多年来,商队遵守协约,商路渐渐兴盛,协议也渐渐稳固。

石桌旁,贾俑满脸通红,神情扭曲,已经拼尽了全力。

但是却仍旧架不住猴王的力气,只见手臂渐渐向一侧倾斜,最终砰的一声,猴王的手臂压倒贾俑。

得胜了!

猴王站起来,举起双拳高兴的擂胸。

猴群大叫大喊,声势惊天。

贾俑垂首退去。回来的路上,两旁的匪猴极尽嘲笑之能事。有的掀开皮裙,露出红屁股,对向贾俑,有的做鬼脸,有的摇手指。

“想不到我居然有被畜生嘲笑的一天……”贾俑无奈地叹了一口气,脸上全是苦笑。

贾龙面无表情,往后招招手。

属于贾家的队伍开始前行,猴子们蜂拥而上,从敞开的货车上肆意摘取货物。

贾家耍了个小心思,在精品煤石上,盖了一层色彩华丽的丝绸和绢布。猴子们都被这些彩布吸引了注意力,放过了灰不溜秋,但市价更高的精品煤石。

猴群们肆意嬉戏,很多猴子将布裹在自己的臂膀上,腰上,披在身后,场面喧闹至极。

“贾平何在?”贾龙沉声一喝。

贾平缓缓走出,他和贾俑形成鲜明对比,骨瘦如柴,似乎弱不禁风。

“我来替你报仇了。”他走过贾俑的身边,随手拍拍他的肩膀。

“有贾平哥哥出马,自然手到擒来。”贾俑抱拳拱手,苦笑一声。

贾家族人看到贾平出马,都舒了一口气,流露出安心的神情。

看到贾平到来,猴群发出怪叫声,充满了轻视和不屑。

猴王已坐了下来,满不在乎地提起酒缸,喝了口猴儿酒。

“到底是畜生,就会以貌取人。”贾龙冷笑一声。

这贾平看似瘦弱,其实却有双熊之力。只是用了盘筋蛊,使得他浑身肉筋纠结,如树根盘踞,因此就将肌肉紧凑压缩。

贾平坐下,伸出手来。

他的手,还不及猴王爪子的四分之一。但发力之后,只是僵持片刻,他就将猴王击败。

一时间,猴群吼叫声一滞。

猴王瞪大双眼,流露出难以置信的神色。

贾龙轻笑一声,挥挥手,示意队伍继续前行。

堵路的猴子自动让开道路,却不在动手。任由贾家队伍前进了一小半,猴群叫起来,又将路堵住。

猴王拍拍石桌,不信邪,向贾平挑战。

贾平含笑,又是一场胜利。

“诸位,我先行一步了。”贾龙拱手,打了声招呼后,贾家队伍全部通过这道关卡。

“好了,下面轮到我林家了。林动!”林家的副首领喝道。

其他人也不争,商队内部早就商量好了顺序。

时间在流逝,商队也在一段段的前行。

为了度过匪猴山,尽量的减少损失,各大家族特意培养了许多蛊师。

牛力、虎力、象力、蟒力、马力……蛊师们一一上场,各显神通,有输有赢。

大多数人都过了关卡,终于轮到张家的队伍。

张柱的脸色并不好看,他是治疗型的蛊师,并不擅长力量。

况且和猴王掰手腕,只能凭借自身气力,绝不能动用蛊虫。一经发现,就是作弊,会引来猴群的攻杀。

张家的这只商队,除了他这个三转之外,也再无其他蛊师。因此在商队中,实力处于末流。

商心慈在张家过得并不如意,作为私生子,招受许多排挤。尤其是当她母亲病逝后,情况越加糟糕。

依照母亲的遗嘱,商心慈变卖了家产,组建了这个商队。

张家许多人巴不得这个家族的耻辱,死在外面才好。因此也没有支援派遣蛊师进来。

“张柱叔不必太在意,左右不过是些货物罢了。只要人在就好。”商心慈心很细,察觉到张柱的脸色,轻声宽慰道。

“最后只剩下张家了。”

“啧啧,不用看,他们是输定了。那个张柱我比较了解。”

“张家这只队伍,据说是张家的丫头自己组织的。因此只有一个张柱撑门面。”

许多蛊师站在关卡后,抱着看戏的想法,看向这边。

他们或多或少,都失去了一些货物,心情自然并不大好。

比较产生幸福,倒霉的人往往看到更倒霉的人后,心情就舒坦了。

很多人都将目光投向张家队伍,想要寻求心理上的慰藉。

“货物是死的,人命才重要。张柱叔不必上去,我们直接让猴群取货物就是了。”商心慈道。

“唉,小齤姐有所不知。不比的话,根本过不去,这群猴子认死理,必须掰手腕。小齤姐,输人不输阵。咱不能叫其他人看不起,我这就去了!”张柱一拱手,硬着头皮也要上。

“等等!”就在这时,方源从人群中钻出来。

“张小齤姐,你对我有救命之恩。这一阵,就让我吧。”他拱手道。

“就凭你?”丫鬟小蝶翻了个白眼,“都这个时候了,你又不是蛊师,不要添乱了好不好!”

商心慈也微微笑道:“黑土,你的心意我领了。这不是闹着玩的,猴王力气巨大,你没看到前面几位蛊师,都骨折了吗?”

“小齤姐,就算是骨折,我也要报答小齤姐的。”方源坚持道。

“你这个人怎么这样子,不知道天高地hou。你骨折了,还不是我家小齤姐要出力治疗你。”小蝶厌恶地摆手,“你不要来捣乱了好不好?”

“张家小齤姐,你有所不知。我从小力气就超出常人,孩童时成年人都没有我力气大。这一次我必去无疑!”说罢,方源便转身,向猴王走去。

“黑土!”商心慈想要阻拦,却被张柱拦住。

“小齤姐,他又不是蠢货,能站出来一定有他的底气。有时候,我们也要相信别人。”张柱劝道。

实际上,他对方源根本就没有信心。只是觉得,趁机教训一下这个给他们惹麻烦的凡人也好。

“咦,你们快看,张家居然派出了一家奴!”

“哈哈,张家无人,派出一个家奴来丢丑吗?”

方源的身影,很快就引起了其他人的注意。(未完待续

------------

\end{this_body}

