\newsection{离开商家城}    %第一百三十四节:离开商家城

\begin{this_body}

%1
他原以为,方源是想教训他一顿。但没想到,方源是想杀他,甚至心甘情愿地付出紫荆令牌这般昂贵的代价!

%2
“为了区区的我,如此好勇斗狠,值得吗?”得到方源这般的“重视”,周全想哭!

%3
他是个正常人。

%4
他也怕死。

%5
要不然,周家灭亡之后,他成了孤家寡人,又受重伤,早就趁机死去了。

%6
但他活了下来。

%7
求生是每个生命的本能。

%8
他拒绝商心慈,是因为看不上她。他生性高傲,曾经是一族之长,怎么可能屈居于一个黄毛丫头?

%9
但他没想到,事情会闹得这么大!

%10
现在他居然要面临着死亡的威胁了!方源的疯狂,是他怎么也料不到的事情。

%11
“早知道如此,我从了那女娃,也就算了。居然会落到这步田地!”周全的心中涌起强烈的悔恨之情。

%12
他虽然高傲,但他并不愚蠢。

%13
生存和高傲相比起来,当然优先选择生存。要不然他也不会作为丧家之犬,苟且偷生到现在了。

%14
“对了,我明白了!我屡次拒绝商心慈,严重地打击了她的威信。方正又是商心慈最大的支持者,所以他要相反设法地除掉我!商一帆误我啊……”

%15
感受到方源身上越来越浓郁的杀机,周全思绪万千。

%16
以他的智慧,只需要稍稍试探,就明白先前街坊上的流言,是商一帆捣的鬼。

%17
起先,他还很沾沾自喜,藏着得意。这些流言,将是他拒绝商心慈的一个绝好挡箭牌。同时从流言中,他也能看出商一帆对自己的重视。实在不行,他还可以投靠商一帆去。

%18
商一帆和商心慈两者,对他周全没有任何区别。只要他愿意辅佐,他就能将其捧上少主之位。

%19
这是周全的自信。

%20
但现在他无比后悔。

%21
正是因为这股流言。才引得方源对他的杀机。他现在想要投靠商一帆,也已经迟了。

%22
方源这个疯子,完全不按常理出牌。周全这次彻彻底底的栽了!

%23
周全趴在地上,被揍得浑身骨架都散了,脸又被方源踩着。根本动弹不了。

%24
他张开口。想要求饶。

%25
但话到了嘴边,又说不出口了。

%26
“场上这么多人看着,当众求饶,脸面就彻底丢光了。但是不求饶投降。我的老命也玩完了啊……”

%27
性格决定命运。

%28
关键时刻,周全高傲的劣性仍旧在发挥着作用。

%29
“俗话说,识时务者为俊杰。周全,你既然不识时务,那就不是俊杰。我杀你损失一块紫荆令牌。足以让你骄傲了。你去死吧。”方源狞笑一声,脚下渐渐用力。

%30
周全只感觉巨力压迫而来,他终于抛弃一切的犹豫,想要开口求饶。

%31
但方源脚下是那么的用力,卡着他的腮帮子。他想要张口说话,却无能为力。

%32
周全急了!

%33
“等等,我不要死啊。我要求饶,我要投降,你倒是让我说话啊……”

%34
他在心中咆哮。同时奋起余力,挥舞手脚。

%35
他的手抓住方源的小腿,但方源身躯如钢铁浇筑的一般,纹丝不动。

%36
“我命休矣……”就在周全绝望的时候,忽然听到一道熟悉的声音。

%37
“黑土哥哥。脚下留情。”商心慈赶到了现场。

%38
“心慈,看来终究还是没有瞒住你。我知道你求贤若渴,你不要替这家伙求情了。这样的人死不足惜。”方源冷声回答一声,但是脚下却在悄然收力。

%39
“不。黑土哥哥,我要说。”商心慈却很坚持。

%40
她继续道:“哥哥你和周全老先生接触的时间不长。但我了解更深。周老先生,一直立志于重建周家。他的肩头担负着重任,心有壮志难酬。他曾经也很感伤地对我倾述,说放不下昔日的亲人。他的夫人临死前,曾嘱托他重建家园。这些年来,他背负着重担,艰难打拼。他是有苦衷的……”

%41
“是这样。”方源收回了大部分的脚力,面色微变。

%42
“我怎么不记得向你倾诉过?”周全心中奇怪,他夫人死时,他都不再现场。

%43
但他旋即明白过来,这是商心慈和方源演的一场戏。

%44
其实,方源和商心慈还是想招揽自己的!

%45
他们以商家城为舞台,当众演了一场好戏。刚刚的话,是铺设好的台阶。

%46
既是宣扬了商心慈的仁慈,以及求贤若渴的心,又给了自己一个台阶。

%47
“真是好算计,好算计……我堂堂周家族长,今日栽在了这几个小辈手上。真是长江后浪推前浪啊。”周全咬着牙,心中长叹。

%48
有愤怒,有仇恨,也有凄凉和无奈。

%49
“原来如此。想不到周老先生,也是有大志向的人。不过你还是愚不可及,辅佐心慈,也不和你重建周家的壮志冲突嘛。你为了理想,死都不怕,我也很敬佩。但你却不知,死很容易,但为了理想而忍辱偷生,背负重担继续前行,才是真正的勇气。”方源大声地道。

%50
周全听到这里,哪里不晓得这是方源给他的台阶下。

%51
这很可能就是最后一个台阶了。

%52
如果他不抓住,那么他的生命就结束了,再没有任何的机会。

%53
念及于此,这位老人家张开了口:“唉!江山代有天才出,达者为师,今日听了你们这番言语,让我惊醒!”

%54
方源松开脚。

%55
商心慈大喜,赶忙将周全搀扶起来。

%56
周全忍着全身的剧痛,颤颤巍巍地站起身来,又对着商心慈拜倒下去:“周全,拜见心慈小姐。”

%57
……

%58
“你说什么?周全居然认那黄毛丫头为主了?”书房里,商一帆听到这个消息后,错愕了半晌。

%59
“这不可能!周全的性子我知道,就算是当初商睚眦担当少主,统领商家城商铺时,也招揽不到他。她商心慈何德何能,居然能得到周全的效忠?!”商一帆反应过来后,高声惊呼。

%60
“这事情确实属实。”张老总管叹着气道,“商心慈还是个雏儿。自然没有这个能力。但是她身边,却有方正和白凝冰二人。老实说,我低估了方正。想不到他粗中有细,也是个有心计的人。他特意将事情闹得不可收场,强逼周全认主。”

%61
“周全若不认主。方正当场就要杀他。现在。整个街坊店铺,都在说着这个事情。到处都是流言,说周全为了重建家族,忍辱偷生。卧薪尝胆。被方正一语点醒后,终于选择归附求贤若渴的商心慈。现在商心慈的威望,已经达到了某种巅峰!”

%62
商一帆闻言,勃然大怒:“这么说,我们之前花费那么大力气。散布谣言,反而给他们造势了?骗子,都是骗子!这些流言一定都是他们故意发布出去的,好一个君臣相遇的故事。我呸!”

%63
“一帆少爷,你稍安勿躁,这场比试还远没有结束。商心慈虽然手下有了人才,但未必能令他们归心。接下来,依靠着夫人的势力帮忙,我们还是大有胜算的。”张老总管冷静地道。

%64
在他的劝说下。商一帆的心情渐渐地平复下来。

%65
他咬牙切齿,双眼闪烁着阵阵阴芒:“你说的不错。组建势力,可不是一朝一夕的事情。她得到了这些人,几乎都是方正威逼利诱,怎么可能真正归心?嘿嘿嘿。接下来,我就挑拨离间,再用重金收买,不信没有效果!”

%66
……

%67
在商心慈统筹。方正二人护卫,周全辅佐。卫德馨、雄家三兄弟等人同心协力之下,关于演武场的情报生意,终于搭建起来。

%68
正如同商心慈所料的一样,这生意刚一开张,就引发剧烈的轰动和反响。

%69
在开张的第一天,就赚回了投入进去的全部成本。

%70
第二天,仍旧引来轰动。

%71
第三天,热潮还未退去。

%72
足足七天之后,商心慈的三十万元石,已经增长至四十四万。

%73
商一帆的阴谋诡计,没有得到任何的进展。商心慈组建的势力,似乎是铁桶一般,严密周瑾。众人如此归心,让其他少主,都诧异无比。

%74
商一帆十分恐慌,因为他知道:如果任由商心慈发展下去,凭她这样的狂猛势头,必是最后赢家。

%75
他开始借助母族势力,影响商家高层。

%76
商心慈的情报生意,涉及到演武场,本来就是个敏感话题。商家高层数位家老合议,正要勒令商心慈停止买卖的时候,商燕飞站了出来,一扫众议,力挺女儿。

%77
商燕飞的表态,简直是对商一帆的最后一击。

%78
数月后,商一帆和其母族势力回天乏术,惨败在商心慈的手中。

%79
商心慈继商睚眦之位,成为十大少主中的新贵!

%80
但离别的悲伤,冲散了成功的喜悦。

%81
“黑土哥哥,你们真的这么急着走吗?”商心慈走出城门,十里相送。

%82
“你已经成功登上少主的位置。以你的才华,必定能坐稳的。心慈,天下没有不散的筵席,我们还会有再见面的时候,你无须太多感伤。”

%83
方源安慰着,又话锋一转:“临走之前,我还有一事提醒你。凡事要把目光放长远,商家十大少主之上,还有少族长商拓海。商拓海之上,还有商家的五大重臣家老,你的父亲商燕飞。商燕飞之上,还有商家的太上家老……”

%84
“哥哥,你放心。当年,商拓海成为少族长,是占据天时。其余商家少主,拥有地利。我无天时也无地利,只有投资人才,拥有人和,才能与他们抗衡。哥哥,你要有什么需要,就通知我。只要我力所能及,一定会为您办到!”商心慈的眼中,闪着智慧的光芒。

%85
这番话,让方正二人都不由地为其侧目。

%86
果然不愧是日后崭露头角,成为商家族长的女才子!

%87
“好,后会有期。”方源深深地看了商心慈一眼,转身就走。

%88
白凝冰跟随在他的身边。

%89
两人一黑一白的背影,渐渐地没入山道林荫中。

%90
商心慈和两位丫鬟,久久地站在原地,望着方白二人身影消失的方向,没有动弹。

%91
“黑土哥哥,三叉山危险,请珍重!”商心慈美眸中酝酿着一层水雾,心中则在暗暗地祝福。

%92
(ps:有人问,为什么要改?很简单,不改就要被屏蔽。你说改不改?

%93
有时候慷慨就义很容易,为了理想苟且偷生却很难。男人需要的是忍耐和坚持!

%94
这句话是给周全的,也是给我的。

%95
我很幸运的是,能得到大家这么多人的支持。其中有许多人,一直在支持我。哪怕我更新渣且不稳定,哪怕我节操时常碎掉一地。

%96
投诉其实也不只这一次了,写这本书不赚钱,太小众,各反面都有许多压力。

%97
唉,不管是我,还是大家。其实坚持到这里,真的不容易。

%98
告诉大家一个好消息,现在我有些时间了。

%99
明天,咱们继续。)(未完待续。如果您喜欢这部作品,欢迎您来起点投推荐票、月票,您的支持,就是我最大的动力。手机用户请到阅读。)

\end{this_body}

