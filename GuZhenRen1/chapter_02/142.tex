\newsection{何等的穷凶极恶!}    %第一百四十二节:何等的穷凶极恶!

\begin{this_body}

天梯山,乃是中洲第一山。又号称是传承之地,圣贤之山。

它高达近百万丈,山势雄伟,气度宏大至极。山峰直插苍穹,隐于霜雪云雾当中。

非常奇特的是,在这山上,嶙峋怪诡的山石并不多。

一块块的方形巨石,横卧着,从山脚一直延伸往上,形成一副阶梯。

但这阶梯规模是如此巨大,以至于中洲中有古老传说,称此山可上达仙界,是天和地之间沟通的桥梁。

围绕着天梯山,历史上有许多或动人,或神秘,或悲壮,或欢喜的故事。

天梯山是成为中洲蛊师心中的圣地,是最接近仙庭的地方。许多蛊师一生求仙未果,临死前将自己的墓地,安置这里。其中一大部分人,同时将自己的传承,也布置在这里。

天梯山上隐藏着无数的传承,但只有有缘人,才能获得。

每年,笼罩天梯山的云雾消散之后,天梯山都会涌来庞大的人流。许多传承会被继承走,又有许多新的传承布置下来。

但是,今年却与往常不同。

今年,在天梯山上,狐仙福地开启,引得蛊仙现世。他们商议后联手,将此山圈住,安排各自的门派后辈,来一场比拼较量。

此刻,在天梯山脚下,一场年轻人之间的激战,已经步入尾声。

万鹤齐飞,缭绕在方正的身边。而魏无伤气喘吁吁,衣衫褴褛,被陷入重重包围当中,目光死死的瞪着方正。

“不,我还没有输!我还有杀手锏,我还有底牌!”魏无伤处于下风,却不甘心认输。

换做平常情况,他可能已经甘拜下风了。但在这现场,不知多少位精英蛊师。正目不转睛地观看着。

魏无伤代表的不仅仅是他自己,还有他身后的天妒楼。同时,其他人也就算了,自己心仪的碧霞仙子,也在看着。

“不能输!”怀着这样的心念,魏无伤悍然催动了空窍中的一只蛊虫。

这只蛊虫,他一直雪藏着。就算是门派中重要的考核中,也没有使用出来。

这蛊一经催动,便产生了一股无色的微风,轻轻地吹拂起来。

温柔的轻风环绕在他的身边,将他的衣摆吹动,将他的发梢吹拂。

但方正却如临大敌。

皆因天鹤上人在他心中。已经大声地提醒他:“不好!这小子手中居然有伤风蛊。防御,尽全力防御,此蛊刮起伤风,看似温柔无害,其实厉害无比。天妒楼的小子,果然不弱。看来这就是他的杀手锏了!”

伤风吹来,方正完全放弃攻势。用全部的力量来进行防守。

伤风看似轻柔,但所到之处,吹得群鹤哀鸣,无数只铁喙飞鹤仿佛折断双翼,从高空凄离坠落。

伤风刮在方正的身上,将他浑身的防护之光,吹得不断摇曳。

方正额头滴下冷汗,咬牙防守。空窍中的真元不断消耗,灌输到自己的防御蛊虫里。

两人僵持了片刻,最终魏无伤在心中无奈叹气,停止催动伤风蛊。

不是他不想去战斗,而是他的空窍已经接近干涸。

蛊师身上都有个共同点,那就是当真元耗尽时,蛊师的战斗力就要暴降到谷底。

一转到五转的蛊师。都会受到真元的限制。只有超凡脱俗,达到仙的层次,成就蛊仙,才会有可能拥有无穷无尽的真元。

“输了。”魏无伤目光黯然。

他算计得很清楚:此刻自己真元耗尽。没有再战之力。但方正一直在指挥飞鹤战斗,真元消耗较少,必定还剩余不少真元。

“魏兄不愧是天妒楼的精英弟子,手段如此奇特犀利,教在下大开眼界,又涨了一份见识。在下的真元,也被兄台消耗殆尽啦。这场切磋,我们就算平手可好?”方正却笑道。

“什么?”魏无伤神情惊愕。

方正说的这是什么话?他自己的情况,自己最清楚。怎么可能会有消耗方正真元的蛊虫?

但魏无伤旋即明白过来,这是方正在撒谎。

“方正是在搭个台阶,让我好下场。”明白了方正的意图后,魏无伤的脸上涌现出复杂的神色。

名门大派的弟子,通常都不能随意出手。

皆因他们代表的不仅仅是自己,还有他身后的门派。

再加上碧霞仙子在场,这场战斗魏无伤实在输不起。

方正既然主动搭个了抬价,魏无伤犹豫了一下,旋即双手拱拳,向着方正道:“方兄少年英姿,魏某心中佩服。仙鹤门果然底蕴深厚,才能教授出兄台这样的人物。这次切磋让我受益匪浅,就依方兄所言,算做平局吧。”

表面上魏无伤这样说,但实际上,他却在对方正暗中传音:“方正,你这次手下留情,我魏无伤记住了,将来必有一报。但是碧霞仙子乃是我的意中人,我不会在这方面让步的。我努力积累,将来还要和你切磋!”

方正浅浅而笑,表面上点头应是,暗中却是头疼。

魏无伤又暗中道:“方正啊,你要小心。追求碧霞仙子的人物众多,远不止我一个。就算依你这样的强大战力,也有四大竞争者。他们分别是天河陈大江,紫电腾空古霆,九死小悲风汤如气,还有母老虎赵淑野。你和碧霞仙子走的这么近,他们必定会找你麻烦。你可最好不要败了。”

方正连续听到这四个鼎鼎大名,目光不由地一阵闪烁,直感觉到自己脑门子更疼了。

而天鹤上人,则在他的心中哈哈大笑。

果然,如魏无伤所说,三天后,古魂门等人来到天梯山。为首的古霆,在得知碧霞仙子和方正的事情后,立即找上门来挑战。

方正为避免这无妄之灾,选择避战不出。

古霆自然不肯善罢甘休,每天都会来到方正临时居住的山洞口,挑衅约战。

一连七日,天天如此。

古霆的话,越骂越难听。仙鹤门其他弟子不忿,找上门去,被他一一击败。

古魂门气势大振,到了第八天,直接率众堵在方正的洞门口,不断叫骂。

“方正你这个缩头乌龟,还不滚出来?”

“方正你躲一时,还能躲得了一辈子?乖乖地放弃和碧霞仙子往来,古霆大哥就会大发慈悲地放你一马。”

“仙鹤门也不过如此,竟然教出你这般懦弱的弟子。”

……

一牵扯到仙鹤门,在洞中的方正顿时发出一声叹息,只得无奈地走出山洞。

旁人骂他,他都可以忍受。但是一旦涉及师门,性质就变了。作为弟子就要维护师门,这是中洲的价值理念。若不维护,将来回到飞鹤山,也会被人弹劾,受到惩罚。

天鹤上人在他心中,发出大声的鼓噪:“嘎嘎嘎……方正我徒,你现在知道了吧?我之前一直劝你的话,没有错吧。一味的忍让,会让人误解成容易欺负。在这个世界上,你表现得越无害越懦弱,只会吸引更多想要欺负你的人。战斗吧,把这古霆击败!让古魂门的人,统统闭嘴!你的名声,将大涨一截!”

“唉……人在江湖,身不由己,我算是感受到了。只好战斗了!”方正经历此事,心中也产生了一股明悟。

古霆,来吧,我们好好打一场!

……

就在方正和古霆交战的同时,远在南疆的三叉山上,一场众人瞩目的战斗,已经完结。

战场一片狼藉,鲜血喷洒了一地,山石崩裂,树木摧垮,打出来的坑洞环遍四周。

方源傲立在场中,而他此次的对手费立,则跪在地上,向他磕头求饶。

“方正大人,请你高抬贵手,放我一马吧!”费立对方源不断磕头,哀声苦求道。

他是四转中阶的修为,也是力道蛊师。原本意气风发,但此刻浑身浴血,右臂被方源撕掉,两条腿都被打折,狼狈悲惨至极。

“你既然想求饶,那就先把你的费力蛊贡献出来。我再考虑考虑,答应不答应你。”方源俯视着脚下的费立,双眼冷芒四射。

费立犹豫了一下,只好交出费力蛊。

此蛊乃是他的本命蛊,核心蛊,能让敌人一举一动,都要更加浪费力气,加倍力量上的损耗。

费力蛊交到方源的手上,失去了本命蛊,又让费立遭受重创,大吐一口心血。

方源接过费力蛊,目光闪了闪:“我考虑过了,费力蛊还不足以换你的小命。”

费立瞪大双眼,不顾自身沉重的伤势,叫道:“方正大人,这可是我最珍贵的蛊虫了!”

轰!

方源心念一动,兽影悍然扑下,将费立打成一团血色肉酱。

“穷鬼。”方源望着脚下,面目全非的尸骸,不屑的嗤笑一声。

然后,他转移目光,扫视战场一圈。

观战的人不在少数,见方源目光扫来,都下意识地选择避开。

方源哈哈一笑:“怎么?飞天虎薛三四没有来吗?你们告诉他去,大家都是力道蛊师,三天后,我要登门,和他切磋!”

此言一出,许多人哗然。

小兽王太过于生猛,前几日杀了横眉暴君,这次又把费立打得跪地求饶,结果还不放过他,把他轰成肉酱。

接下来,他还要挑战飞天虎薛三四!

这是何等的穷凶极恶!(未完待续。如果您喜欢这部作品,欢迎您来起点投推荐票、月票,您的支持,就是我最大的动力。手机用户请到阅读。)

\end{this_body}

