\newsection{坑}    %第十四节:坑

\begin{this_body}

%1
这个世界上,可不仅仅只有人类才会酿酒。

%2
关于酒的最早记载,出于人祖的传说。

%3
人祖挖去左右双眼,分别化为大儿子太日阳莽,二女儿古月阴荒。

%4
太日阳莽和古月阴荒朝夕相处,不禁对自己的妹妹动了情思。但古月阴荒却屡屡拒绝了太日阳莽的追求。

%5
太日阳莽为此困恼不已。

%6
他知道自己需要帮助,因此求教智慧蛊。

%7
起先,智慧蛊并不搭理他,能躲多远就躲多远。但是太日阳莽锲而不舍,智慧蛊不堪其扰,只好对他说了一个法子——

%8
“在世界的东方,栖息着一群蜜桃猴。它们酿了酒,你先去喝了酒,再来找我罢。”

%9
太日阳莽便去了东方,喝了酒。

%10
蜜桃猴群酿的是果酒。太日阳莽喝了之后回来,从此以后,他的脸蛋就变得红扑扑的。他咂咂嘴回味道:“原来酒是甜的。”

%11
智慧蛊笑了笑,又对他说:“在世界的西方,有一群通灵猴。它们酿了酒,你再去喝。”

%12
通灵猴酿的酒,是苦酒。太日阳莽又去了西方,喝了酒,从此以后,他的舌苔就是黄褐色的。他一脸苦兮兮的样子回来:“原来酒也有苦的。”

%13
智慧蛊便对他说:“酒有苦甜,爱情也是如此,人生更是如此。在那北面,有一群金刚猴。它们也有酒,你去喝喝看。”

%14
金刚猴酿的是烈酒。

%15
太日阳莽喝了很对胃口,他喜欢烈酒,喝得酩酊大醉。

%16
他觉得这酒太对他胃口了,醉了之后更想喝。喝了碗里的,还想喝坛子里的。

%17
最终,他吐得一塌糊涂。酒劲上来,让他难受得快要死了。

%18
他感觉自己的身体内,仿佛有火焰在燃烧,有岩浆在流淌。

%19
“太热了!”他大叫一声,所有炎流都逆冲头顶,头发腾地燃烧起来。从此以后,他有了火一样不断燃烧的头发。

%20
当太日阳莽醒来的时候,他发现智慧蛊正看着他。

%21
“你喝了烈酒,有什么感悟呢?”智慧蛊问他。

%22
太日阳莽叹气道:“我算是明白了,再好的酒喝多了也会吐的。凡事都应该适可而止。”

%23
智慧蛊哈哈大笑,又说道:“在世界的南方,栖息了一群天水猴。它们酿了酒,也蛮不错的。你去喝喝看吧。”

%24
天水猴酿的是清酒,和烈酒是两个极端。

%25
太日阳莽淡淡地品酒,不禁忘却了烦恼,醉眼朦胧间,微醺而飘然。

%26
智慧蛊再问他感受。他轻轻地摆摆手道:“已得酒中趣,勿为醒者传。”

%27
智慧蛊轻轻一笑,悄然而去……

%28
因此,在这世界上,人类并非第一个酿酒的族群。反而猴子,走在了人的前面。

%29
一般的猴群,都会酿酒。

%30
根据猴群不同,各种口味的酒都有。但人们将猴子酿造的酒,统称为猴儿酒。

%31
方源栖息在这处山洞,一方面是感到了修为突破之兆,停下来特意冲关。另一方面,则是猴儿酒。

%32
蛊虫转数越高,食物的需求量就越大,同时喂食的周期也会延长。

%33
方源的准备足够充分,但是兜率花的容量到底是有限的。如今经过长途跋涉的消耗,已经有一部分空余容量,装一些猴儿酒最好不够。

%34
酒能消毒,同时也能驱寒。若是逆炼四味酒虫,也需要酒作为辅料。就算逆炼的条件满足不了,充当四味酒虫的食料,也是不错的。

%35
然而取猴儿酒并不容易。

%36
这支草裙猴群,虽然是百兽群,但规模近千。有三只百兽猴王。

%37
猴群向来团结,一旦对敌,自然是群起而攻。白凝冰虽然是三转巅峰,但终究势单力薄,强攻纯粹是自找死路。

%38
方源尽管提升到一转中阶,但这种提升对于局面的帮助,微乎其微。

%39
但方源执意要抢夺猴儿酒,对此白凝冰表示担忧。

%40
“所以需要智取啊,跟我来吧。”方源拍拍白凝冰的肩膀,站起身来。他小心翼翼地踩着地面,避开埋下的焦雷土豆,走出山洞。

%41
山洞外,树林苍翠,阳光照耀,一片鸟语花香。

%42
二人走了片刻,接近猴群外围。

%43
方源细心观察,终于选了一处坡地。

%44
借助较高的地势,他眺望了一番,最终满意地点点头,踩踩地面,说了一句:“开挖吧。”

%45
一炷香的时间,向阳的这处坡面,被两人合力,挖开了一个三丈深,直径五丈的大坑。

%46
焦雷豆母蛊。

%47
方源蹲在坑底,心念一动,唤出一个蛊来。

%48
这只蛊,形如土豆,表面凹凸不平,又满布一个个的细孔小洞。外形不堪,但却是货真价实的三转蛊。

%49
“我真元不行,还是借给你来用。”方源将焦雷豆母蛊递给白凝冰。

%50
白凝冰拿在手中,催动真元灌注进去,不久就看到豆母蛊的表面,那些细孔小洞中都往外生长出一根根的翠绿幼苗。

%51
幼苗迅速长大,开花,结果。

%52
几乎呼吸的功夫,手指头粗细的墨绿色果实成熟。随着枝牙枯萎,都掉落在白凝冰的手掌中。

%53
方源取来这些墨绿果实,细心挑选筛选,剔除了其中坏、死、空果,只留下三分之一不到的好果。

%54
这些果实,便是焦雷土豆蛊,二转蛊。埋在土中,便能借助土地肥力而生长。一旦有生物踩踏附近,造成震荡。焦雷土豆蛊便会发生自爆。

%55
方源取出其中一枚,顷刻炼化。捏在手指中间,用青铜真元灌注进去,焦雷土豆蛊顿时绽放出微微的碧光,缓缓悬浮而起。

%56
接着他心念一动,焦雷土豆蛊倏地一下,便钻入地底。

%57
方源故意深埋,直到豆蛊距离他脚下足有一臂距离的深度,这才停止。

%58
空窍中的真元海面,不断地飞速下降,周围的地力也汇聚到焦雷土豆蛊的身上。在方源的感知中,这枚小小的草蛊,在顷刻之间,就长成了一颗拳头大小的土豆。

%59
这才是成熟的焦雷土豆蛊,只要受到震荡,就会被引爆。

%60
白凝冰看到这里,不免有些奇怪:“我看那个魔道女蛊师,将焦雷土豆蛊埋得很浅,易于引爆。你这样埋,就算是我在上面跺脚,恐怕也引爆不了吧?”

%61
“这是当然。”方源回了一句,继续埋头铺设。

%62
白凝冰微微撇嘴,方源没有正面回答她,骄傲如她也不再追问。而是目光沉凝,开始自我思索。

%63
和方源相处的时间这么久,她知道方源绝不会做无用的举动。

%64
方源将坑底都埋了一遍后,这才直起身来,擦擦额头的汗渍后,开始号召白凝冰一起埋土。

%65
但土只埋了片刻,坑底地面增高了一丈五的高度,方源就喊停,然后继续埋下焦雷土豆蛊。

%66
白凝冰看到这里,顿时领悟到了方源的意图。

%67
“原来如此。焦雷土豆蛊只是二转蛊,一颗引爆,威力有限。你这样一次性埋下这么多,一旦引爆了,哪怕是千兽王,恐怕也不好受。但你如何将那三头猴王引到这里来呢?”

%68
焦雷土豆蛊虽然是攻击蛊,但是它不能移动的局限性,也大大削弱了它的实用价值。

%69
“这有何难?只要到时候,逮两三只草裙幼猴,在这坑上品味猴脑。它们的惨叫声,必会引来愤怒的猴群。起初时,只是普通的猴群,你所要做的就是挡住它们,杀退它们。然后就应该是那三只猴王了。”方源一边埋下草蛊,一边回答道。

%70
白凝冰不由地点头。

%71
野兽终究是野兽,智慧有限。方源这样的计策,粗鄙不堪,但绝对会好用。

%72
“只要将那三只猴王斩杀,取猴儿酒简直是探囊取物。当然,若是能捉几只猴王身上的野生蛊,那就更好不过了。”白凝冰琢磨着。

%73
接下来,整个下午他们都将精力耗费在这坑中。

%74
不断地催生焦雷土豆蛊,然后由方源亲自动手埋设下去,接着两人填土。一层土后,又是另一层土。直到将这坑填平。

%75
两人累得汗流浃背,好在方源有双猪之力,白凝冰也将一鳄之力修炼圆满。

%76
但第二天,方源却没有动手,而是继续挖坑。

%77
白凝冰表示不解。

%78
方源便道:“一个坑并不保险,总得有预备手段。多做一丝准备,都是一件好事情。”

%79
于是,接下来的三天,白凝冰总算意识到了方源近乎变态的“谨慎”。

%80
足足挖了五个坑,埋下了大量的焦雷土豆蛊。当然,第一个坑的规模最大。

%81
由于充足的准备,解决草裙猴的计划,进行得非常顺利。

%82
只是引爆了两处陷坑,猴群就溃散了。

%83
三只猴王两死一伤,伤者带领猴群逃遁迁徙。死者被炸成了碎渣,身上的蛊虫自然也没有幸存。

%84
方源收获了大量的猴儿酒,多得兜率花都装不下。

%85
逆炼四味酒虫是足够了。喂养酒虫方面,在两年内都不成问题。如果碰到商队,这些猴儿酒也能卖到好价钱。

%86
“走之前还有一件事情,我们把战场打扫一下,把炸开的两个坑都填了。”方源道。

%87
“要不要这么谨慎啊?”白凝冰现在一想到坑,就感到痛苦。

%88
方源扫了她一眼,只是一句话就让白凝冰乖乖的干活——

%89
“你忘了我们是怎么追查到那个魔道女蛊师的么?”

%90
就是依靠魔道女蛊师在途中,留下的各种痕迹。

%91
挖下坑,可别到头来,把自己埋进去!

\end{this_body}

