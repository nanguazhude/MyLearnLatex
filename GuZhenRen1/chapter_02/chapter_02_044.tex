\newsection{送到嘴边的肥肉}    %第四十四节:送到嘴边的肥肉

\begin{this_body}

%1
“这么多的货,可不便宜啊。至少得五万块的元石。”金家寨的蛊师,用怀疑的目光打量方源。

%2
从商心慈处得了许可,方源就立即找了好几位金家蛊师。眼前的这位,已经是第六位了。

%3
“元石我没有。”方源摇摇头,“不过我却可以用我的货物,来换取你的货物。”

%4
“易货?”蛊师微微扬起眉头,不是很吃惊。易货也是常有的事情,尤其在商队里经常发生。

%5
对于他来讲,只要价值相差不大,易货都不吃亏。

%6
“那你打算用什么货来换呢?”

%7
方源便领着他,当场看了货。

%8
金家的蛊师皱起眉头:“你这货市价还要比我的便宜呢。”

%9
“但是在这黄金山,你却能卖出好价钱,不是吗?”方源笑了笑。

%10
金家蛊师眉头皱得更紧:“价钱定高了,就卖不动了。”

%11
“那就慢慢卖,总有卖完的一天。物以稀为贵,到那时你坐在家里,安心收钱。”方源微笑道。

%12
蛊师哈哈一笑,他说这么多,不过是想要压价罢了,其实心中早就砰然而动。

%13
“你很不错。虽然是个凡人,却不卑不亢。我有三个铺子,你有没有兴趣来帮我?我给你一个掌柜的身份!工钱好商量。”蛊师拍拍方源的肩膀。

%14
方源婉言拒绝,倒让这蛊师有些遗憾。

%15
“黑土,你搞什么鬼啊!”这笔交易完成之后。小蝶面罩寒霜地跑过来。

%16
“你居然把小姐的货,都给换了?你想干嘛?你也太大胆了!”小蝶气得跺脚。“你知不知道,我们这些货,都是小姐精挑细选的。运到商家城去,能卖出高达一倍的价钱!快,你给我换回去!”

%17
方源面色一冷:“你们家小姐已经把所有的货品,都借贷给了我。也就是说,这些货物都是我的。嗯……我处理我自己的货,有什么问题?”

%18
方源目光扫过小蝶。寒芒一闪。

%19
小蝶顿时一个激灵,感到心中一悸。

%20
方源决定暴露更多一些,这还是第一次对小蝶不假辞色。

%21
小蝶原本伶牙俐齿,但是此刻,心中却有一股寒意油然而生:“你,你……我告诉小姐去,你等着!”

%22
她极力掩藏自己的内心情绪。但仍旧有些慌乱地跑开了。

%23
她的小报告,当然没有对方源造成什么困扰。

%24
不过,商心慈带来的家奴们,也对方源此举很有看法。很多人私下议论,都认为这个黑土疯了。

%25
方源知道原来的货物价值,这也让他察觉到了商心慈的商业才华。

%26
但是后者毕竟是第一次行商。天赋是有,但是经验却远远不足。行商,并非是把货物运到终点站,赚取差价那么简单。

%27
真正的行商高手,是走一路赚一路。是用敏锐的目光。及时地发现商机;是熟知各座名山的特产,各大家族的需求;是建立关系网。左右逢源。

%28
当然,这对于商心慈来讲,这些要求太高了。她毕竟今年才刚刚十六岁,虽然有天赋,但终究只是位稚嫩少女。

%29
方源前世有近百年的时间,都用来经商。混过商队,担任过头领。开过商铺,以及赌石店。还举办过拍卖会。

%30
论经验和眼界,商队里的那些首领、副首领,给方源提鞋子都不配。更别说刚刚行商的商心慈了。

%31
“距离商量山,还有一段距离。如果我全心全意操作,至少能将这批货的价值炒到七八倍!”

%32
这种倍率,已经相当恐怖。再高的话,方源已经做不到了。碍于种种现实条件,七八倍已经是这个世界上的顶尖水准。

%33
“当然,如果采取不正当的手段,别说七八倍,就是七八十倍也是轻而易举。”念及于此,方源不禁想到地球上的一首定场诗——

%34
守法朝朝忧闷,强梁夜夜欢歌,损人利己骑马骡,正直公平挨饿。修桥补路瞎眼,杀人放火儿多,我到西天问我佛,佛说:我也没辙!

%35
哈哈,所谓体制和律法,都是压榨大众的,限制弱者的。

%36
不管哪个世界,都是弱肉强食!

%37
因此,就算是法治的时代,也有无数权贵钻营体制的漏洞,逃避法律的制裁。更何况这个地方割据,靠实力说话的蛊师世界呢?

%38
方源前世,曾苦心经营,有家财近千万,产业无数。然后被强者随意吞并,倾家荡产,落魄街头。

%39
在此后四百余年,他每当回想起来,都万分庆幸拥有这段经历!

%40
感谢困苦,它教会人真理!

%41
正是这段经历,他一朝惊醒,地球上被一直洗脑而养成的种种框架束缚,皆轰然粉碎。

%42
人被蒙蔽,往往不是因为眼前景象,而是心中枷锁。

%43
对于方源而言,如果恪守商业道德,作为正规商人,顶多只能赚个七八倍。

%44
若是稍稍用些不正当的违规手段,成为不良商贩,那就是十几倍。

%45
若是抛开职守,尽情的坑蒙拐骗,当个大奸商,那就有数十倍的利润。

%46
若是直接烧杀抢掠,连本钱都不用。无本生意,永远是最赚的!

%47
但方源此刻行商,却别有目的。因此违规手段,一概不取用。这倒让他多了些缚手缚脚的感觉。

%48
不过,就在商队将要启程的前天晚上,一个金家的蛊师暗中找上门来。

%49
“有一笔私下里的买卖,你有兴趣吗?”这位蛊师,就是之前和方源易货过的其中一位。

%50
方源原本并不在意,但是半盏茶的功夫后,他改变了原来的想法。

%51
“你是说,有人要卖金簪草?”他心中不禁诧异万分。甚至怀疑自己听错了。

%52
金簪草对于金家而言,无法用其他材料替代。是极其重要的战略物资。正是因为有它作为炼化的辅料,金家才能大量合炼出金蚕蛊。现在居然有人要贩卖?

%53
在方源的记忆中,正是因为金家大量装备了三转金蚕蛊,才实力大增,从而灭掉黄家,制霸一方。

%54
“等一等,现在黄家寨仍旧存在。也就是说,金家还没有补全残缺的金蚕蛊秘方?不对呀。按照这个时间段,他们应该有了眉目才是。要不然,怎么会大量种植金簪草呢?”方源心中思绪电闪。

%55
他故意试探道:“我的货已经换得差不多了。金簪草是个偏门的材料,虽然稀少,但很少人需要……”

%56
一见方源有拒绝的迹象,那蛊师顿时急了:“这价格好说,咱们可以好好谈谈嘛。”

%57
方源目光闪了闪。看样子对方很急迫的样子,他便试着继续压价。

%58
几轮交锋后,金簪草的价格被压到最低,已经惨不忍睹。

%59
对面的蛊师面色发白,神情也变得难看,语气不耐烦:“你赢了。就按这个价格,可以成交了吧?”

%60
这价格压得很低,大大低于金簪草本身的培育成本。卖出去的话,可以说是血本无归。

%61
金家蛊师也明白这点,心中暗暗滴血。

%62
方源也知道到了极限。但他却摇摇头道:“这价格太低了,说实话。你这样的态度,让我不大相信啊。”

%63
金家蛊师顿时咒骂起来:“你个杀才,价格明明是你压的,你居然反过来嫌低了?”

%64
方源摊开双手:“你刚刚也说了,这是私下里的买卖,没有交割凭据的。万一你卖给我假货,我找谁说理去。你看,商队明天就要出发了。到时候我即便吃了亏,也不得不走。”

%65
“你怀疑的倒是……”金家蛊师的气消了些,“货你不用担心,百分百真货。实话告诉你吧,这是我家少主偷偷拿出来卖的。”

%66
方源眼中精芒一闪即逝,话说到这里,终于套出了些有用的信息。

%67
他故意惊讶了一声:“你们少主偷拿的?”

%68
“这些金簪草其实是族长喜爱的植物,因此特意种植了三亩。谁叫我们族长爱好特殊呢。所以你不用担心,金簪草又不是什么重要的东西,儿子拿了老子的东西卖掉,就算被发现,也顶多被打骂几句罢了。”那蛊师道。

%69
方源听到这里,心中顿时了然。

%70
原来是这样!

%71
金家一直立志于修复残缺的金蚕蛊秘方。到如今这个时间,已经有了眉目,知道了金簪草的重要性,因此种下了三亩。

%72
然而,为了防止黄家等等不必要的注意,这个消息被局限在家族高层当中。哪怕是少主都不知道。只以为金簪草是族长种下看得玩的。

%73
金簪草的培育周期很长,一般需要四年光阴,才能从幼苗长熟。

%74
记忆中,金家就在一年之后,发动了攻势。三转的金蚕蛊,攻势相当犀利,消灭了黄家,开始独霸黄金山。

%75
如果这三亩金簪草没有了,市场上也搜刮不到这么多的量。那么金家要消灭黄家,恐怕要推延数年光阴了。

%76
很显然,金簪草是个绝对的烫手货!

%77
要真买到手里,就是捅了金家这个马蜂窝了。

%78
换做常人,一定躲避还嫌来不及,但方源却看到其中的大便宜。

%79
这块大肥肉,虽然滚烫得很,但都送到嘴边了,能不吃吗?

%80
说实话,方源虽然有天元宝莲,但是对于元石的需求还是有的,并且这种需求还不小。

%81
到了商家城,他要购买蛊虫,需要大笔的元石。单靠天元宝莲每天的积累,不仅麻烦,还远远不够。

%82
“那个少主,显然是个败家子。估计平时很得族长的宠爱,说不定最近欠了钱,手头紧,就把主意打到了金簪草上。呵呵呵……”

%83
想到这里,方源不禁笑了。

%84
这块肥肉吃下去,不仅对购买蛊虫有帮助,而且能给商心慈留下深刻印象。这是一举两得的好事。

%85
金家蛊师看到方源的笑容,也笑了:“这么说,你答应了?”

%86
“当然。”方源看向他,这笔买卖做成了,这家伙肯定倒霉。

%87
“但他倒霉,关我什么事?”方源现在考虑的是,这块肥肉该怎么吞下去,才不会烫到他的嘴。

\end{this_body}

