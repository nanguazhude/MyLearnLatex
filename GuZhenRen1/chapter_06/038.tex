\newsection{天庭援手}    %第三十八节:天庭援手

\begin{this_body}

气海之上,白气遮天,汹涌澎湃。

气绝魔仙居高临下,俯视气海老祖,魔威浩荡。他的头顶兮地虚影显露无疑,令他稳稳利于不败之地。

“气海小辈,老夫修为可能向你借蛊吗?”气绝魔仙又开口道。

方源面露沉凝之色。

之前的一番交手,已经令他看出不少气绝魔仙的端倪和底细。

气绝魔仙拥有的种种杀招且不去论,他的底牌显然是天地秘境兮地。依靠这座天地秘境,他能够克制全天下的气道蛊仙。

换做其他流派,气绝魔仙不会有这么大的优势。但偏偏气海老祖专修气道,因此受制十分严重。

从此也可窥见气绝魔仙的性情。

他虽然行事霸道,但并非无脑,知道要发挥自己最擅长的优势。因此,他在重生之后,并不和三方恋战,明智撤退。随后途中完善弥补了兮地,增长战力。现在又来到气海老祖面前,强借仙蛊。

一旦他拥有了八转气道仙蛊,必然实力大涨,真正的在当今世间战力之巅站稳脚跟。

方源单靠气海老祖这层身份的手段,的确不是气绝魔仙的对手,但是他并不慌张。

方源行事向来喜欢谋定而后动。

这次气绝魔仙的威胁,方源早已得知情报,早就改良了气海大阵。只要他退守气海大阵,便可再抵抗一阵子。

若是气绝魔仙冲入大阵,方源还可将这座大阵直接引爆,从而制造出撤退的良机。

说起来,这个自爆的手段方源还是从房睇长那里获得的。当初房睇长为了豆神宫,故意设计陷害方源,就是要牺牲算不尽这个智道蛊仙,引爆他布置在豆神宫中的仙蛊屋,从而冲破瓶颈,一举炼化豆神宫。

方源识破算计,将房睇长俘虏,搜魂之后获得自爆仙蛊屋的手段。而后他经过改良,布置到了气海大阵之中。

一旦自爆,必定威力恐怖。就算威胁不了气绝魔仙的性命,也绝对能够阻止干扰他一段时间。

在这点上,方源自信十足。

只是……

方源若是借助大阵自爆而撤退的话,他这层气海老祖身份的价值就要降低许多了。

为了保住这层身份,方源当初在宿命大战中都刻意收敛了气道手段。气海无量杀招也只是运用了一次而已。

若是方源撤退,气海老祖就算是被气绝魔仙击败。今后气海老祖的声望会大为降低,在东海的号召力更会大打折扣。

气绝魔仙微微皱眉,他两番喊话,气海老祖都是不应,这让气绝魔仙有限的耐心达到了极致。

他冷哼一声,正要继续出手,忽然一道血色虹光贯穿天际,笔直地冲向他的面庞。

气绝魔仙迅速躲闪开来,然后他便看到了诛魔榜出现在了天边。

“天庭,又是你们!”气绝魔仙冷笑出声,“我不来找你们的麻烦,你们倒是主动来招惹我。宿命之战大败,你们还有何底气如此行事?”

“天庭恪守人族正道,行事凭借的不是底气,而是信念!”从诛魔榜中传出回应。

这声音却不是古月方正,而是一道女声。

方源在一旁听了,顿时心头微动:“秦鼎菱亲自来了?原来我的运势显现会有外援,竟是应在她的身上!”

气绝魔仙伸出手指,对准冲来的诛魔榜连连虚点。

大气汹涌变化,迅速凝聚成团。刹那间,数百颗浑白气团流星赶月一般,射到诛魔榜上,引发阵阵爆炸。

诛魔榜冲势顿止,被炸得剧烈震荡。

气绝魔仙正要说什么,忽然面色微变。他迅速抽身,闪电般飞射转移。

原来之前诛魔榜射出去的那道血色虹光,竟然又回转杀了过来。

气绝魔仙闪过血色虹光,不想从这层血光中忽然又透射出金芒。

金芒在气绝魔仙的头顶上迅速凝聚,化为一柄金色巨剪。

咔嚓咔嚓。

金色巨剪在气绝魔仙的头顶连连剪下,似乎是攻击落空了。

气绝魔仙却是面色骤变。

“这是运道手段!”

他连忙催动侦查杀招,顿时心疼不已:“糟糕,这金色巨剪是在削减我的气数!”

气绝魔仙十分注重维系自身气数,之前重生成功,而后又拆解乱流海域,弥补兮地,已经令他的气数消耗很大。这一次金色巨剪的攻击,更让他雪上加霜。

“气海、天庭,这笔账老夫记下了。将来若有机会,老夫会好好的和你们算一算的。”气绝魔仙声音震荡天地。

抛下这句话后,他就立即钻入冲天彻底的浑白气流当中,迅速撤离战场,消失不见。

“这就又退走了吗?”诛魔榜中,古月方正看得有些呆愣。

反倒是其余的天庭成员一个个面色凝重。

秦鼎菱叹息一声,指点方正道:“这才是气绝魔仙的可怕之处。他时刻保持冷静,对局势有着明确判断。单靠我方只能给他增添麻烦,但现在情景,气海老祖绝不会坐视不管。依凭气绝魔仙的战力,他对抗我方和气海的联手,也并非难事。但是这样一来,战况焦灼,他之前要强借气道仙蛊的计划,就没有成功的可能了。与其如此,他干脆撤退,保存实力。”

方正得到提点,顿时明悟过来:“我明白了。气绝魔仙到底是上百万年前的人物,他刚刚重生,对新生的流派接触得并不多。之前对付诛魔榜的血道攻势,还有秦大人您的运道金剪,都是以躲闪为主。等到他熟悉了这些流派,又将自身的杀招进行改良。我们要对付他,可就更加困难了。”

“正是如此。”秦鼎菱点点头,“都随我出来吧,让我们和气海老祖好好谈谈。”

天庭蛊仙鱼贯而出,秦鼎菱领头,面带微笑,古月方正紧随其后。

诛魔榜被主动收去,更彰显了天庭诚意。

“得亏了诸位仙友之助,才令老朽渡过此劫。”方源长叹一声,满怀感激地道。

秦鼎菱身披金甲披风,身姿高挑,肩宽腿长,她英气十足。但这股英气和凤金煌的不同,带着天生的高贵,仿佛生来就要凌驾于众生之上。

但此刻,秦鼎菱却是面浮微笑,对方源流露出主动亲近的意思。

她开口道:“若非老祖你有伤在身,怎会令气绝魔仙不断叫嚣呢?况且,我观老祖始终从容不迫,定是有着后手,还请老祖不要怪我等随意插手才好。”

这话说的真叫客气,让方源听了都有一种如沐春风之感。

天庭的态势和五百年前世很不相同。五百年前世,天庭是让别人没有台阶下。而这一世,天庭的头领都主动给别人找台阶下。

经过宿命一战,天庭蛊仙的心态和大战之前是完全不同的。宿命大战前,天庭掌握宿命蛊,优势极大。但现在宿命蛊毁了,天庭实力大减,一方面积极休养生息,另一方面则开始主动合纵连横。

气海老祖落到天庭眼中,就是最佳的合作对象。

气绝魔仙想要找高手切磋,自然会第一个找气海老祖。毕竟不管是方源,还是凤九歌都仙踪缥缈,难以寻找。但气海老祖就待在气海,很容易就能找上门去。

天庭既然想要和气海老祖联合,自然就想到利用气绝魔仙。

毕竟锦上添花永远比不上雪中送炭。

天庭援助气海老祖击退气绝魔仙,可不是随意插手,而是筹谋良久的。

不管是方源耐心等待气绝魔仙,天庭也同样如此。

“诸位仙友贤达,快请进来,让老朽略备薄酒,微表感激之情吧。”方源将天庭诸仙引入气海大阵。

他布置酒宴,款待天庭诸仙。

酒席间,又将弥留作客在这里的夏家诸仙,也邀请入席。

夏家诸仙对天庭成员们也抱有感激之情,态度很是热情。

之前,气绝魔仙来找麻烦,和气海老祖一战,他们都看在眼里,忧在心头。

东海夏家的地盘已经被吴帅彻底侵占,只能将复兴的希望寄托在气海老祖的身上。气海老祖若是战败,亦或者身陨,对夏家上下都是严酷打击。

所以,天庭帮助了气海老祖,也等若是帮助了他们。

秦鼎菱见到夏家诸仙,立即表示慰问,对两天联盟表示强烈的谴责,并且做出承诺一定会在将来出手,抗击这个天下最大的异族势力。

这番态度更加赢得夏家上下的好感。

夏家太上大长老一时间高兴得满脸红光。

因为,这和其他东海超级势力说说空话不同,天庭的确是做出了实际行动的。之前诛魔榜携带三仙,主动进攻两天洞天,这是全天下蛊仙都知道的事情。

秦鼎菱又接着询问夏家蛊仙的近况。

夏家蛊仙们皆答:自己在气海这里受到气海老祖很好的款待,一直感恩在心,只是心中忧愁何时才能重返家园。

秦鼎菱更加明白了气海老祖的意图。

“他这是在养望啊!野心真的不小。”

对于这一点,秦鼎菱却是乐于见到。只要气海老祖有图谋,有野心就好,双方之间就有彼此合作的基础。

方源自然也不是平白无故请夏家蛊仙入席参宴的。

透过秦鼎菱对夏家蛊仙的态度,方源更加确定天庭的来意。同时,他也让本土势力的夏家蛊仙为自己做一个公证人。毕竟身为东海蛊仙,却和天庭接触紧密,是很会降低气海老祖在东海的号召力量的。

酒过三巡,方源主动开口:“天庭诸友今番助我,此情老朽记下了。将来天庭若有什么麻烦,老朽也愿挪一挪老迈之身,为天庭略尽绵薄之力。”

这是明确的表示了。

天庭诸仙听到这话,不由齐齐精神一振。

秦鼎菱缓缓放下手中杯盏,面带微笑看向气海老祖,终于步入正题:“不瞒老祖,眼下天庭的确遭遇了一些问题。我等前来是想请老祖援手的。”

“哦?请尽管开言。”方源故作豪爽地道。

秦鼎菱却没有道出天庭产生气功果的事情,而是取出一只信道凡蛊,交给气海老祖。

方源探入心神,立即动容。

这信道凡蛊中的内容,就是那剩下的一半元始真传!

\end{this_body}

