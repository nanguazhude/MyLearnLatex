\newsection{天庭窘境}    %第四节:天庭窘境

\begin{this_body}

寒灰仙姑首先冲冰晶洞天的门户中缓缓飞出。

安魂洞天留守在外的两位七转,立即迎接上去。

“姑姑。”高瘦的蛊仙关切地询问道,“事情谈得如何?”

寒灰仙姑微微摇头,这个时候,夜天狼君也从冰晶洞天中飞出来。

两人又交谈了几句,便当场分别。夜天狼君领着狼群,寒灰仙姑领着魂兽大军,重新启程回转各自的洞天大本营。

一路上,三仙交流不断,皆不知晓魔尊幽魂其实一直潜伏在他们的身边。

魔尊幽魂原本想狩猎魂兽大军,但经此一事,便就悄然改变了主意。

光听安魂洞天这个名头,就知道它是以魂道立业。再看寒灰仙姑等三仙皆是魂道蛊仙,安魂洞天中一定有充沛的魂道修行资源。

别的不说,单单他们手中掌握的这支魂兽大军,对于魔尊幽魂而言,就是一笔丰厚的大餐。

“气潮席卷五域两天,黑白两天的洞天势力都会按捺不住。呵呵,我正可从中借力,提前打击其他尊者势力,帮助我更快崛起!”

魔尊幽魂不单要对付方源,更要对付其他尊者的势力,诸如天庭、长生天等等。

眼下宿命蛊已毁,魔尊幽魂就有重生,重修回尊者的可能,其他尊者也是如此。

眼下方源踪迹不明,魔尊幽魂当然要着眼全局。他虽然杀戮成性,但杀的方式也多种多样。借刀杀人也是一种高明的杀戮之法,正适合他现在采用。

寒灰仙姑三仙还不知道自己和安魂洞天,都已经笼罩在魔尊幽魂的庞大阴影之下。另一边,夜天狼君回到了夜狼洞天。

夜狼洞天中诸多蛊仙立即朝拜。

夜天狼君道:“冰晶仙王等人无非是想和我们结成联盟,共抗五域。我感到为难呐。”

蛊仙夜治点头道:“是很为难。眼下局势,宿命已毁,五域合并,天地二气彻底交融,气潮卷席之下,任何洞天都隐藏不住。我等势小力单,将来若是五域攻伐,必定抵挡不住。不过,咱们未必要抵抗五域。我们是人族,冰晶仙王等人则是异族。五域强,两天弱这是显而易见的事实。我们何不投靠五域呢?”

夜天狼君哈哈一笑:“夜治知我,我也有这样的心思。我们和异族联合,定然会被视作人族奸细。就算五域没有攻我的心思,一旦联合,也就有了。”

夜治皱眉:“但是要投靠五域,也不妥当。首先,五域之中的北原,是绝对投靠不了的。其次,其他四域,也未必肯冒然接受我等。最后一点,我们若是要投靠五域,恐怕首先就要遭受冰晶仙王等人的围攻。”

夜天狼君叹息:“是啊,所以我已和安魂洞天的寒灰仙姑商定,一方面两方共同进退,另一方面积极联合人族洞天势力。至于投靠五域中的哪一域,依我看,中洲天庭是最佳的对象。我们可以先暗中接触。”

夜天狼君乃是枭雄之辈,不仅是修为高达八转,政治手腕也是非常老辣,统领夜狼洞天无人不服。

他既有这样的决断,夜狼洞天中的蛊仙自然不敢有什么异议。

天庭。

中央大殿终于修补了起来,目前是秦鼎菱主持着大局。

宿命蛊修复失败,天庭经历惨烈厮杀,残破不堪,直至现在,仍旧大多都是废墟。

“这是耻辱!”秦鼎菱放下手中信蛊,暂时休息。她面色平静,目光却是闪烁,心底又浮现记忆,怒火不断地积蓄。

红莲的图谋大成之后,龙公死在冲锋的路上,方源旋即退走。但三域蛊仙却想要将天庭彻底灭亡,无人退却。

生死存亡关头,元始仙尊的手段开始发挥作用。各种怒气、晦气、丧气等等降临三域蛊仙的身上,令他们状态暴跌。

原来这记手段是用来对付方源的,结果被无极魔尊虚影出手拦截下来。红莲大计成功后,一缺抱憾亭的双尊对弈也迎来解决,无极畅笑,双尊虚影同时消散。没有了无极虚影的力量压制,元始的手段便又开始发挥作用。

还有一点值得庆幸的是,凤九歌虽然临阵反戈,但只是帮助方源炼蛊,此刻并没有出手为难天庭。

天庭得以喘息,反攻回去,艰难地扳回局面。

三域蛊仙无可奈何之下,以龙宫、劫运坛、玉清滴风小竹楼压住阵脚,不得不徐徐撤退。

出了天庭,还有人想劫掠中洲,结果这个时候发生了第一次气潮。

三域蛊仙的仙窍皆连不稳,只得彻底撤离。

天庭蛊仙也不再追击,在亲眼目睹了方源炼制宿命蛊后,他们的斗志和战意都不强盛。

大战结束,千疮百孔的天庭舔舐伤口,收拾残局。

龙公阵亡,紫薇仙子也叛变,投靠了魔尊幽魂,天庭蛊仙们商议了一番后,便推举秦鼎菱暂代首领之责。

战后的这段时间,都是秦鼎菱主持大局,处理繁杂如山的大小事务。

秦鼎菱表面平静,心中却是憋着一团火焰。堂堂天庭居然战败,彻彻底底的战败!她急切地想要洗刷耻辱,但理智又告诉她,不能冲动。因为一切都变了,天庭不再是以前的天庭,敌人也不再是以前的敌人,甚至天地都在剧变。

休憩片刻,秦鼎菱开始继续处理事务。

“嗯?”她手中捏着一只信道凡蛊,面露微微异色,“夜狼洞天?”

夜天狼君不愿和异族绑在一条战船上,他更想要投靠更加强大的五域一方。而五域当中,虽然天庭落败,但到底是数百万年来的人族第一势力。

夜狼洞天和长生天交恶,当今五域中,也唯有天庭能够有这样的底气,来对付长生天诸位蛮勇恶徒。

在这封信道蛊虫中,夜天狼君更是详细讲述了异人的种种不轨之心。

“哼,果然是非我族类其心必异!宿命蛊刚刚毁去,这些异人就迫不及待地要跳出来。”秦鼎菱冷哼一声,心中顿时涌动起浓郁的杀机。

冷静下来后,秦鼎菱开始思索这个难题。

对于夜天狼君和夜狼洞天,秦鼎菱了解很少。所以只是一份信道蛊虫的条件下,她无从夜天狼君的投靠是否具备诚意,亦或者根本就是诈降?

“假使这夜天狼君真心投靠,我方又该如何面对洞天异人势力,以及长生天呢?”

夜天狼君在信中明确地告知了先祖来历,并没有隐瞒他们和长生天的仇恨。

秦鼎菱感到为难。

眼下的天庭是前所未有的衰落。

气潮卷席天下,天地二气融汇,拥有虚窍的天庭蛊仙比黑白洞天的蛊仙,都更加行动自如。

但天庭却不能这样做。

首先,绝大多数的天庭成员寿命都极其有限。其次,还必须要考虑到支援星宿意志。

所以,天庭中的一部分成员,会继续进入仙墓沉眠,帮助星宿意志,防止她被天意彻底同化。

如此一来,天庭的战力就会再跌落一个层次。

一时间,秦鼎菱拿不定主意。

“若是紫薇仙子仍在,就好了。”秦鼎菱叹息。这是她心忧的又一点。

紫薇仙子、正元老人被魔尊幽魂奴役,不知所踪。前二者熟知天庭隐秘,后一者更是魔尊之魂,谁敢小觑?

秦鼎菱思索一阵,觉得此事事关重大,便召来周雄信等人商议。

周雄信建议道:“此事的确重大,但目前的情况还不明朗。夜天狼君这六位蛊仙,不过是我中洲上方的黑天中的八转,其余四域的黑天中呢?我们不知道到底有多少洞天,这些洞天当中有多少存在八转蛊仙。我愿亲自动身,前往黑白两天秘密探查!”

周雄信专修信道,深知情报的重要性。

他身怀虚窍,根本不在意天地二气的融汇。

秦鼎菱采纳了他的这个建议,决定暂且继续和夜狼洞天方面保持接触,另一边彻底探清详情。

君神光交出一只信道凡蛊,道:“这里的名单已经整理好,都是天赋、才情不缺的蛊仙种子,更重要的是对天庭,对十大古派忠心耿耿。只是其中有一位,我不能一人做主,尚需诸位一起商议才能定夺。”

君神光整理出的这份名单,名列着许多凡人蛊师,皆是资质不凡之辈。

天庭此战战败,仙墓也被方源彻底捣毁,几乎一切都在重建,都在百废待兴。再加上五域乱战的重大威胁,天庭早已决定要大力扶持,不惜代价,栽培出更多的蛊仙来。

秦鼎菱取过蛊虫,探入神念。

名单上有不少名字:仙鹤门举荐的孙元化,灵蝶谷举荐的萧七星,古魂门的古霆,天妒楼的魏无伤……等到了灵缘斋,名额却是空着,没有书写。

秦鼎菱叹息一声,面色复杂起来。

灵缘斋的具体名额,她当然知道,定然是凤金煌。

但是宿命大战后,凤九歌临阵倒戈,背叛天庭,不知所踪,身为其女的凤金煌处境就相当尴尬。

她原本被龙公收为弟子,乃是将来的大梦仙尊。但现在龙公阵亡,宿命一毁,大梦仙尊也并不是非她不可了。

周雄信顾虑的是:若是天庭继续栽培凤金煌,将来她也学着她老爹叛变,天庭该怎么办?

凤金煌虽然被龙公收为徒弟,但龙公教导出来的大徒弟布局数百万年,把天庭的宿命蛊给毁了。凤金煌这个二徒弟,将来会不会继承这个“优良传统”呢?

“这个名额先缓缓,再议吧。”

ps:我的天!你们都是这么猛的吗?这些天涌现了许多新盟主,感谢午夜摩尔、方源小朋友、书小逗、小王子cc、仙道杀招万我、月曲涯、佛祖讲经、玄门真人、★紫电狂神☆等人数万打赏。感谢古月魔尊的惊人打赏,真的猛啊,恭喜你成为本书的第一盟主,第一粉丝!

衷心感激支持本书,打赏、投票的书友们!蛊仙战队是你们支撑起来的,实在太给力了。

众筹群的氛围也非常热烈,加微信加的我手都有点痛了,痛并快乐着!目前是众筹盟主,助力战队活动。听取了大家的意见后,将来或许有可能会众筹简体书。非常感谢大家。今天一更,从下一月开始双更!

\end{this_body}

