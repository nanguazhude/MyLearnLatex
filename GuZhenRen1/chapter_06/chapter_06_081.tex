\newsection{大意}    %第八十一节:大意

\begin{this_body}

%1
当紫薇仙子等人紧追而来的时候,他们见到的是漫漫无际的黑蝠群。

%2
在地底深渊的这一层中,黑蝠群是凌驾于万物之上的唯一霸主。而今它们遭受到了侵犯,彻底陷入到了疯狂和暴怒之中。

%3
太古黑蝠驱动着上古黑蝠,上古黑蝠们奴役着荒兽黑蝠,荒兽黑蝠卷席亿万的普通黑蝠。当影宗、长生天、天庭三方追杀进来,狂怒的黑蝠群便将无处发泄的愤怒全都向他们倾泻。

%4
饶是三方蛊仙,当今世间最顶尖的强者,也不由地冲势一滞。

%5
“怎么这么多的黑蝠!”

%6
“杀掉黑蝠群中的太古黑蝠,才能将这些兽群打散。”

%7
“太多了,这些黑蝠群的规模太过惊人。不能和它们浪费时间,别忘了我们的首要目标是方源!”

%8
三方杀进黑蝠群中,宛若在汹涌澎湃的恶浪中逆袭。

%9
这些蛊仙都是当代顶尖的八转,黑蝠群虽然规模惊人,但真较力起来,还是难挡诸仙锋芒。

%10
只是诸仙杀出血路来,却不见万年斗飞车、幽魂、青仇等人的踪影。

%11
“怎么回事?”

%12
“他们在哪里?竟然一点线索都没有留下!”

%13
“这不可能,我们明明顺着他们打斗的痕迹,一路追踪过来的。”

%14
群仙一阵惊疑。

%15
劫运坛猛地爆涌金芒,消融周围万千黑蝠,瞬间清空出了一大片的空间。

%16
劫运坛中接着传出冰塞川不耐烦的声音:“紫薇仙子,告诉我,方源去了哪里?”

%17
正元老人冷哼一声:“我们岂会知晓?”

%18
秦鼎菱冷笑:“影宗的人,你们的把戏真以为我等不知?你们吊在后面,一方面是追击不上,另一方面难道不是在干扰我等吗?”

%19
秦鼎菱语气十分犀利,一番话挑拨离间,又述说事实,顿时让影宗极为被动。

%20
方源、幽魂、青仇、气海、气绝都忽然消失,这太奇怪了。让追在最后面的三方势力都着急跳脚。

%21
紫薇仙子眉头紧锁,沉声道:“我们也和主上失去了联络。我深深怀疑,已有其他势力插手其中。当务之急是要找到他们,若是我们内讧,只会让这插手的势力得逞。”

%22
冰塞川声音冷漠,毫不掩饰冰寒的杀机:“那我就要好好瞧瞧智道大能的本事了。”

%23
“哼,那你就瞧好吧。”紫薇仙子说着,旋即动手,催动杀招,迅速推算。

%24
秦鼎菱没有说话,杀意也是毫不掩盖。一旦让她发现影宗耍弄阴谋诡计,那么天庭必会出手针对影宗进行围杀。

%25
吱吱吱……

%26
周围黑蝠再次涌来,其余蛊仙犹豫了一下后,还是纷纷出手,为紫薇仙子抵御兽群。

%27
紫薇仙子浑身笼罩紫金光雾,光雾弥漫,不断扩散。

%28
紫薇仙子面色逐渐发白,仙元剧烈消耗,直至身躯也微微颤抖起来,她这才猛地睁开双眼,露出被紫光完全充斥的双眸。

%29
紫金光雾迅速消失,紫薇仙子的双眸也逐渐恢复成了正常情况。

%30
她的气息明显虚弱了几分,显然是刚刚的推算用尽了全力。

%31
在群仙注视之下,她吐出一口浊气:“我算出来了,有一个战场杀招卷席了他们,让我们一同扑空。”

%32
“战场杀招?”

%33
“会是什么人在此埋伏?”

%34
“哼,你有什么证据?”

%35
紫薇仙子冷笑一声,她身为智道大能,自然不会凭空说话,当即分开分化蝠群,指点群仙分辨出一缕快要逸散干净的土道气息。

%36
群仙捕捉到这缕气息,立即着手分析。

%37
“的确是土道仙招!”

%38
“好像真的是一记战场杀招。”

%39
“且看我的手段,将它找出来!”

%40
天庭中某位蛊仙出手,出手时他自信十足,但很快就铩羽而归,毫无成果。

%41
很快,这缕土道气息迅速消散,留下漫天的黑蝠群对三方群仙继续群袭。

%42
众仙脸色都十分难看。

%43
有的人仍旧怀疑影宗,有的人则在思考接下来该怎么挖掘完全收拢的仙道战场,有的人则不得不地对黑蝠群动手。

%44
长生天方面,冰塞川还是有些将信将疑。

%45
紫薇仙子的心情最为沉重,她一边应付黑蝠,一边喝问天庭:“这缕土道气息你们应当最为熟悉不过!天庭,你们怎么说?”

%46
秦鼎菱沉默了片刻,终究还是开口:“没错,这很类似乐土仙尊的手法。我们的想法都是一致的找到这处仙道战场,然后摧毁它!”

%47
“原来是乐土仙尊的势力,难怪……”冰塞川顿时恍然。

%48
历史上的十大尊者中,乐土仙尊是最为擅长仙道战场杀招的。在这方面的造诣,他是世间公认的尊者第一!如今,他布置下来的轮回战场仍旧还屹立在中洲,为十大古派共有。

%49
至于乐土仙尊的势力为什么出手?

%50
在场的诸仙没有人意外。

%51
乐土仙尊和幽魂魔尊的对立矛盾,世人皆知。幽魂魔尊屠戮天下,带来的恐怖和混乱,让五域民不聊生,正消魔涨。乐土仙尊从修行开始,就致力于拨乱反正,平复世间,镇魔消灾,造福世人。

%52
乐土仙尊、魔尊幽魂两人的理念,存在着强烈的对立。

%53
“要找到乐土的战场,还要破坏它,这个难度……”冰塞川不断摇头,饶是强者如他,也感到了迟疑和为难。

%54
“乐土仙尊的势力为何会出手?他怎么会专程在这里埋伏呢?”有蛊仙发问。

%55
这个问题让紫薇仙子的心情更加沉重,脸上完全被一层浓厚的阴云笼罩。

%56
“一定是方源!”尽管没有任何的证据,但紫薇仙子还是脱口而出,十分确信。

%57
她和方源交手太多次了,基本上每一次的失败者都是她。这种失败的滋味,紫薇仙子品尝了太多次,几乎已经习惯了。

%58
每一次当她以为胜券在握的时候,方源这边都会闹出幺蛾子来。

%59
现在的境况带给紫薇仙子的感觉也是如此,真的再熟悉不过。

%60
秦鼎菱也认可紫薇仙子的推测:“我们追杀方源,换个角度来看,却是被方源一路引领着。一逃一追,这个路线是由方源操控的。他若是和乐土势力配合,形成现在这个局面,的确是最为容易的。只是乐土势力为何和方源这等魔头合作?我们毕竟没有亲眼见证,这究竟是不是乐土留下来的势力?若是,又是哪一支乐土势力呢?”

%61
乐土仙尊乃是年代最近的尊者,他留下来的势力自然也是最多的。

%62
但和其他尊者不同,乐土仙尊似乎从不刻意扶持势力。

%63
天庭是元始仙尊、星宿仙尊、元莲仙尊相继入主,苦心经营出来的势力。长生天本是巨阳仙尊的仙窍洞天,他留下的遗产打造出来的基业。影宗是幽魂魔尊死后,魂魄在暗中经营无数岁月,以僵盟为壳,悄然积累出来的组织。

%64
乐土仙尊不同,他扶持、帮助了无数势力。比如东海的鲛人王庭,是他不忍心看到鲛人被欺凌。又比如南疆的菇人乐土,是他不想看菇人被无情屠戮。

%65
乐土仙尊有将仙窍改造成乐土的手段,他在生前帮助过了大量的蛊仙,许许多多的洞天福地。

%66
他甚至连太古传奇都帮,比如苍蓝龙鲸。

%67
乐土仙尊帮助的对象,不仅是人族,还囊括了异人,甚至包含万物生灵。

%68
所以,在这点上,他是所有尊者中最为博爱,最为仁慈的。虽然天庭的元莲仙尊也有救助万物的行动,但他绝大多数都是侧重于拯救自然,修复环境。

%69
其他的尊者都有攻伐手段,惟独乐土仙尊没有。

%70
乐土仙尊也在各地留下了真传。南疆的菇人乐土就是其中之一,所以乐土领袖,南疆蛊仙陆畏因号称是乐土传人,也无人反驳。当初宿命大阵之前,紫薇仙子还亲自进入菇人乐土,借走了仁蛊。

%71
众仙一时间难以猜测。

%72
乐土如此做派,导致很多势力、蛊仙都能和乐土挂钩。究竟当中是哪一股势力,还真的不好说。

%73
“乐土仙尊一辈子都在消除幽魂的影响,以仁爱示人,带来和平。他的势力会和方源这等魔头合作?但这并非没有可能。天庭不也暂时和我方合作吗?”紫薇仙子的心几乎都被阴霾充斥。

%74
她越是深思,越是觉得糟糕。

%75
天庭之所以和影宗合作,无非是想让方源、幽魂对耗,最终能够一并剿除这两大魔头最好!

%76
若是有这样的想法,乐土势力和方源合作,又有什么不可能的?

%77
以乐土仙尊本人的身份和性情,绝不会做埋伏的勾当。但仙尊已逝,他留下来的势力,或者说是传人有什么性情,谁也说不准啊!

%78
“这个情况一定超出了主上的预料。”

%79
“但还有更糟糕的情况。那就是这一切都是方源的阴谋暗算。他不惜放弃龙宫、万年斗飞车,也要将主上诱骗到这处战场中来。”

%80
“不管真实情况是如何的,我一定要拼尽全力,尽快救出主上!!”

%81
紫薇仙子预感相当不妙,心情焦躁不已。她可是智道大能,任何预感和心情变化,不管多么重视多不为过。

%82
魔尊幽魂对付方源,当然是有优势的。但若是对上另外的尊者势力,那就不好说了。更关键的是,这个尊者势力还是历史上最针对幽魂的乐土!

%83
与此同时,土道战场之中。

%84
魔尊幽魂看着周围黄沙滚滚的景象,面色冷酷如冰:“好,很好。你就是陆畏因,当代的乐土南疆传人?”

%85
一位蛊仙足踩黄云,对魔尊幽魂微微一礼:“魔尊大人慧眼,小仙已在此恭候多时了。”

%86
一旁,方源出声催促:“陆畏因,人我已经引来了。接下来你可别指望我出手,你得让我看看你的诚意……呃!”

%87
方源忽然收声,猛地爆退。

%88
但来不及了!

%89
刹那间,魔尊幽魂忽然现身在他的面前,也不知是用的什么手段。

%90
“方源,你大意了。真是难为你逃到了这一步,给我死吧!”魔尊幽魂陡然伸手,一把抓住方源的脖子。

%91
方源瞪大双眼,无法挣脱!

\end{this_body}

