\newsection{大小气功果}    %第三十节:大小气功果

\begin{this_body}



%1
漫天的气潮,滚滚浩荡,很快就卷席整个冰晶洞天。

%2
魔尊幽魂深深叹息,这太遗憾了。他原本还打算让双方死斗,最终自己收拾一切,吞食蛊仙魂魄可比吸收魂核更有帮助得多。

%3
可惜他的图谋,都别这一场气潮给破坏掉了。

%4
“我手中没有荡魂山,缺乏胆识蛊,只能吞食魂兽和魂核。但这些却是需要一定的消化时间,远不如胆识蛊可以直接助长魂魄底蕴。”

%5
“怎么会这么巧?前不久,气海老祖和吴帅交手,引发了气潮。这一次争战,也引来了气潮卷席。”

%6
魔尊幽魂想到这里,隐有所悟:“难道说,蛊仙之间交手,会引发气潮么?”

%7
魔尊幽魂只是猜测,不敢肯定。

%8
“可惜,我的气道境界只是普通。若是还在的话,定然能有决断。”

%9
他拥有搜魂、吞魂之术,但这些手段却不能直接助长流派境界。能够直接拔升流派境界的途径,唯有吸纳真意。魔尊幽魂全盛时期的种种境界,绝大多数是他借助搜魂而得的记忆,默默苦修、揣摩而得,非常不容易。

%10
“也罢了。”

%11
“这一次安逊表现极佳,立下功劳,极可能被吴帅召见。”

%12
“我寄托在他的身上,就能接近吴帅。或许会有机会,奴役吴帅,夺取龙宫!”

%13
魔尊幽魂一计不成,又生出全新的计划。

%14
他对拥有梦道仙蛊的龙宫十分觊觎!

%15
汹涌的气潮中,道道寒光凝聚一起,还原成冰晶仙王。

%16
冰晶仙王气息衰弱,脸色苍白,施展极冰晶光镇仙棺杀招的代价很大。

%17
但他几乎以一己之力,差点封印了诛魔榜,也是此战奠定胜局的最大功臣。

%18
两天蛊仙们纷纷来探望他,看他的目光已经明显变得不同。

%19
实力能证明一切,能区分身份。毫无疑问,拥有极冰晶光镇仙棺杀招的冰晶仙王,在战力方面要超出大多数的两天八转蛊仙。

%20
“虽然保下了冰晶洞天,但镇族的杀招暴露了。”冰晶仙王击退天庭,但心中却高兴不起来。

%21
当下,他也只能安慰自己这招镇仙棺很难被克制,一段时间内还能依为凭仗。

%22
“诸位仙友,我没有伤,只是需要休养。还请诸位一齐出手,护佑我的洞天。”冰晶仙王开口道。

%23
“好说,还说。”

%24
“我们两天联盟同为一体,这点小事算不了什么。”

%25
“大家速速动手,尽全力维护冰晶洞天!”

%26
两天蛊仙纷纷出手,抗衡和清缴洞天中的气潮。而冰晶仙王则负责修补洞天窍壁的漏洞。

%27
气潮虽然澎湃浩荡,但在诸多蛊仙的合力之下,很快就平定下来。

%28
因为救治及时,气潮导致冰晶洞天的损失,前前后后并不多。

%29
“副盟主,你快来看看,我们这里有新的发现!”大智仙母飞回来时,带给冰晶仙王一个意外的消息。

%30
片刻之后,两天诸仙来到了事发地点。

%31
只见一颗庞大如小山的果实,悬浮在半空中。这颗果实半透明,全然由天地二气组成,形如葫芦,又仿佛花生。

%32
“这应当是气功果罢?”

%33
“气潮卷席之后,人们常常会发现有类似形状的气功果产生。蛊仙得之,用于仙窍,能迅速地调和仙窍天地二气,极大推动自家仙窍的二气相融的进展。”

%34
“可是通常的气功果,不过大如车马。这颗若是气功果的话,怎么会如此庞大?”

%35
两天诸仙议论纷纷,拿捏不定。

%36
“主上,这颗真是气功果吗?”安逊暗问魔尊幽魂,“若真的是,那岂不是冰晶洞天因祸得福?只要吸纳了这么大的气功果,那冰晶仙王的仙窍岂不是可以彻底转变,再不受气潮所制?”

%37
“这的确是气功果,但是福是祸,还不好说。”魔尊幽魂呵呵冷笑。

%38
果然,冰晶仙王试着收取这颗气功果,却惊愕地发现更深层的秘密。

%39
“不妙!”冰晶仙王脸色白上加白,“这气功果还在培育壮大之中,它竟汲取我洞天的天地二气不断成长。我若冒然铲除,必定会造成洞天震荡,生灵涂炭,还会掀起更加暴躁凶猛的气潮出来。”

%40
众仙哗然,在得到冰晶仙王的允许下,他们纷纷上前侦查。

%41
“这气功果绝不能让它发展壮大。有哪位仙友能够帮助我,铲除掉这个隐患呢?我冰晶仙王绝不吝啬酬劳。”冰晶仙王神态严峻。

%42
但蛊仙们面面相觑,都是束手无策。

%43
要暴力铲除气功果,几乎所有人都能做到。但对冰晶洞天的危害极大,直接动摇根本。

%44
而且根据众仙推算,直接铲除,引爆了气功果,形成气潮之后,还会有第二颗气功果继续产生。

%45
这样一来,既要永久地铲除隐患,又要不伤害冰晶洞天,此中难度就着实高了。

%46
大智仙母提议:“我等不成,皆因我们都不主修气道。或许盟主大人有法子?”

%47
冰晶仙王满脸忧愁,深深一叹:“眼下,也只有找盟主大人求助了。”

%48
吴帅很快就得到了冰晶仙王的求援信。

%49
“我记忆有限,并不知道气功果还有这等变化。或许本体那边清楚得很。”

%50
吴帅便又将冰晶仙王汇报上来的详细战报,传达到了方源本体这边。

%51
“哦?这就引发了两天洞天的隐患了么,比五百年前世提前了很多啊。”方源本体果然是知晓的。

%52
五百年前世,两天蛊仙也是联合在了一起,多处攻伐,给五域带来相当大的麻烦。

%53
气潮之下,五域蛊仙受制严重。但两天蛊仙却是因为出身跟脚,受到气潮的影响更要小得多。

%54
再加上蛊仙大战达到某种界限之后,必定会引发气潮产生。

%55
两天蛊仙即便战力低下,但借助气潮,也能顺势击败五域蛊仙。

%56
然而好景不长,很快两天蛊仙们发现自己的洞天大本营中,出现了气功果。

%57
这些气功果完全是汲取洞天的天地二气成长,尾大不掉,若是坐视不管,威胁又会不断增长,最终能摧毁整个洞天世界。

%58
“冰晶仙王抗击天庭,使得气潮卷席了整个洞天,这正是提前结出气功果的缘由了。”

%59
方源将这份情报传给吴帅,忽然心中一动,脑海里灵感一闪。

%60
“等等!”

%61
“或许我能够借助气潮,来对付天意啊。”

%62
他现在的情况十分糟糕。

%63
主要根源是在于天道道痕加身,短时间内炼化不掉,天道道痕自然演化出天意。

%64
天道道痕对方源没有恶意,但天意却是想要铲除,千方百计地为难他,破坏他的至尊仙窍。

%65
更糟糕的是,一旦方源不及时铲除天意,天意积累到了一定程度,哪怕至尊仙窍门户不开,洞天里的天意也会和五域外界的天意相互呼应,令方源的位置暴露。

%66
方源的位置一暴露,若他在气海中逗留,这层气海老祖的身份也就要告吹了。

%67
然而,方源要铲除天意,着实困难。

%68
至尊仙窍太过庞大,方源无从下手。天道演变他也无法预测,并不能提前知晓天意会在何地产生。

%69
“不如就效仿气潮,在我至尊仙窍中不断卷席。像是刷子不断冲刷,所到之处,清缴一切的天意!”

%70
方源仔细深思,越发觉得此法大有可为之处。

%71
看似危险,好像火中取栗,其实按照他的气道底蕴、智道底蕴等等,有很大的实现可能。

%72
当然,这个法子最终能否成功,能否见效,不到最后关头,方源也不知晓。

%73
“若是智慧蛊仍在,那该多好。”

%74
智慧蛊提供的灵感,都是正确的。同时有智慧光晕相助,方源推算出催发气潮的杀招会极其便利快捷。

%75
与此同时,气绝洞天。

%76
“这里就是洞天的最中心!”罗家三位蛊仙各个带伤,终于艰难突破到了这里。

%77
“好,好一颗气功果……”罗木子失声,其余两仙也是满脸的惊奇。

%78
按照罗家三仙的推算,气绝洞天的最中央,应当是藏着洞天中最有价值的宝物。

%79
此时三仙一看,这里赫然只有一颗十分精致小巧的气功果。这颗果实凝如实质,表面上有荧光流转,熠熠生辉。

%80
它只有一颗葫芦大小,但九转仙材的气息却是确确实实。更有一层巨大的生机,宛若山岳海洋,镇压在罗家三仙的心头。

%81
“这颗气功果绝不普通,大有奥秘,我等须得好生研究。”

%82
“奇怪,气功果乃是气潮下的产物,怎么会被气绝魔仙隐藏在最中央呢?”

%83
“难道说,他早就料到了在遥远的未来,会有气潮这样的灾祸吗?”

%84
“动手!”罗家三仙正在商讨的时候,忽然一声轻喝,数位异人蛊仙显露身形,纷纷出手。

%85
罗家三仙猝不及防,被一下子卷入到战场杀招中去。

%86
“不好,竟有埋伏!”

%87
“还有人发现了气绝洞天?”

%88
“你们究竟是什么人?!”

%89
罗家三仙喝骂。

%90
为首的异人蛊仙只有手指头大小,背后生透明薄翼,乃是一位小人蛊仙。

%91
“哼,罗家三仙,我萧荷尖发现这里比你们还要早二十年呢。”小人蛊仙冷笑一声,“动手,杀光他们!”

\end{this_body}

