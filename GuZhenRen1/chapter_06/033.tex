\newsection{一气不绝兮}    %第三十三节:一气不绝兮

\begin{this_body}

“好!”

“盟主大人威武啊!”

“这就是我两天联盟的力量!”

龙宫中诸仙士气顿时大涨,纷纷喝彩。

孽龙在前头横冲直撞,凶蛮无比,而龙宫则跟在后方,轻松悠然。

“这个龙人蛮子!”罗家太上大家老罗足眼皮子直抖,咬牙切齿。

玉清滴风小竹楼中两家蛊仙都感到了巨大的压力。身后的孽龙紧追不舍,简直是一座山脉横行无忌,相比起来,玉清滴风小竹楼显得十分娇小。

更关键的是,玉清滴风小竹楼需要不断避让气柱,前行的路线相当的曲折。而孽龙帝藏生却是直来直往,双方的距离在迅速缩短。

“这可如何是好?”南联蛊仙脸色发白。

武庸却是淡淡一笑:“吴帅以力压人,正是扬长避短。我若是他,也定然会这样做的。诸位勿忧,宿命大战之后,玉清滴风小竹楼不仅修补完整,而且还被我加固几分。暂时抵挡住帝藏生不成问题。对方虽强,但我方却大可不必和他们纠缠,只要抢先夺取了气绝真传即可。”

“盟主说的是!”诸仙纷纷点头称是。

任何的争战都有目的,武庸就看得十分深刻:此次争斗的目的只有一个,那就是夺取气绝真传。哪一方能达到这个目的,哪一方就是胜利者。就算在拼斗的时候吃一些亏,又能算得了什么呢?

毕竟孽龙的确是太强大了。

它是太古荒兽中的战力第一!

整个五域两天,能够和孽龙正面抵抗的人,一只手就能数的过来。满打满算不足五人!

就在吴帅、武庸展开追逐的时候,第三座八转仙蛊屋也悄然奔赴到气绝洞天之外。

正是诛魔榜!

诛魔榜仍旧由方正操纵,但此次不仅来了凤仙太子、白沧水,还有新近疗伤完毕的车尾,以及信道大能周雄信。

方正上一次主动攻伐两天,虽然最终被冰晶仙王击退,但收获仍旧很大。凤仙太子、白沧水从前两个洞天中缴获的战利品十分丰富。

方正虽然战败,但却只是丢了天庭的脸面而已,里子并没有丢。

天庭原本并不清楚气绝洞天这件事情,毕竟太古白天如此空阔,就算气潮之下,气绝洞天显露形迹,一时间天庭也发现不得。

但别忘了,吴帅麾下的夜狼天君却是暗中向天庭投诚。萧荷尖加入两天联盟的事情,根本遮掩不住,夜狼天君因此获悉了气绝洞天的秘密,立即告知了天庭。

气绝魔仙在历史上非常有名,是和无极魔尊作对的恐怖大能!

尤其是他专修气道,在当今环境下,气潮汹涌,全新的气道资源大量涌现,层出不穷。五域各大势力都想要在气道上多加钻研。

天庭也不例外!

尽管天庭已经拥有了元始真传等等,但天庭若是抢夺了气绝真传,就意味着其他势力得不到。这样此消彼长之间,更能确定天庭的优势。

因此方正没有休整多长时间,便又听从秦鼎菱之命,率领四大八转蛊仙,操纵诛魔榜再次杀向白天,来到了这里。

“这里就应当是气绝洞天了。这是怎么回事?”方正刚到,就看到了一场惊变。

只见无数的孔隙不断浮现,浑白的气流汹涌澎湃,不断旋转。无数的孔洞接连产生,沟通洞天内外。

这些孔洞越来越多,不断浮现,密密麻麻,太过密集,让人看来不免有一种恐怖之感。

十几息之后,气绝洞天的巨大虚影逐渐显现。在虚影的表面,是无数浑白气流的包裹。气流中,则有无数孔洞不断地吞吐。

周围的大气被恐怖的无形力道,狠狠地吸摄到洞天中去。以至于形成剧烈的呼啸声,充斥天庭五仙的耳畔。

“气绝洞天似乎正在发生剧变!”

“吴帅、武庸等人恐怕已经在洞天中动手了。”

“该怎么办?”

凤仙太子等人都看向方正,这一次仍旧是方正为主。

方正感受到了压力,眼前的抉择让他感到十分艰难。一时间,他陷入犹豫当中——究竟该怎么做才好?

气绝洞天的中心。

轰轰轰!

气柱崩溃的轰鸣声不绝于耳。

孽龙所到之处,扫荡一切的障碍。但好景不长,这些气柱崩溃之后化为一股股强劲的气流,无数股气流相互纠缠,覆盖在孽龙身躯表面,不断对它施压,从四面八方来围剿它,困杀它。

孽龙嚎叫,但冲势不断减缓,从前方传达过来的压力越来越大。

玉清滴风小竹楼、龙宫也不好受。

这些气流宛若一道道海啸巨浪,两座八转仙蛊屋置身其中,仿佛成了惊涛骇浪中的小舢板。

一道道巨大的气流,冲撞得两座仙蛊屋剧烈颤抖,有时候甚至令仙蛊屋在原地打转。

一股深蓝的气流,笔直前射,玉清滴风小竹楼急忙躲闪。但深蓝气流太过庞大,速度快得惊人,玉清滴风小竹楼也是费尽全力,折腾了三个呼吸,这才成功脱离。

“这是电气!”南疆蛊仙惊叹,“我还从未见过如此庞大的电气,这样的质量足以是八转级别的仙材了。”

搜集电气,须得在雷雨天气。

电气的采集并不困难,但高转的电气仙材却是比较罕见的。

八转电气更是罕见至极。

若非情况紧急,南疆蛊仙们都想采集这些电气。

“都怪吴帅!这么乱来,导致气绝洞天彻底暴动了。”罗足咬牙切齿。

武庸则神色沉着:“气绝真传才是首要目标!”

刚刚的电气让玉清滴风小竹楼损失了一些蛊虫,武庸迅速弥补过来。

这些仙材本身因为道痕太过浓郁,形成了极端的环境。若是有风道的气流仙材,倒是能让玉清滴风小竹楼如鱼得水。

气绝洞天如此异变,把环境变得十分恶劣,但总体上却是让武庸处境好了许多。

因为玉清滴风小竹楼遭受影响,但吴帅方面所受的影响更大。

龙宫的速度要比玉清滴风小竹楼慢得多。

凌厉的刀气、漆黑的硬气、晦暗的死气……种种气流庞大无比,皆有八转程度。吴帅硬着头皮,操纵龙宫硬生生地顶过去。

孽龙已经被甩到后面去了。

它的身躯太过庞大,同时遭受到至少十几股的气流牵扯,已经难堪大用。

“这就是气绝洞天的底蕴么……”吴帅神情严峻,双眼精芒闪烁不定。

他落入了下风,但大局未定。

大殿中的两天诸仙也震撼得无语。若非有龙宫护身,谁敢闯荡这样的险恶之境?许多八转蛊仙同时强烈地感到自身的渺小和虚弱。

“气绝魔仙不愧是曾经和无极魔尊放对的存在啊!”

两座八转仙蛊屋都未放弃,艰难地向最中央前行,力图抢夺气绝真传。

而在洞天之外,方正已经不需要犹豫了。

“我们先撤,静观其变。”方正下令,诛魔榜开始迅速撤离。

凤仙太子等人都沉默不语,认可了这个决策。

皆因此刻乾坤大变,四面八方都汹涌气流——竟又有气潮产生了!

在这气潮中作战,会令蛊仙遭受反噬。天庭蛊仙也不例外。

“看来之前的推算是没有错的。”

“我们的仙窍中的天地二气,和外界存在差别。所以不管是开放仙窍,还是激烈战斗,只要达到一定程度都会引发新的一轮气潮。”

“气绝魔仙本身是五域蛊仙,跟脚在此,此刻孔洞无数,等若洞天大开。再加上洞天中,吴帅等人和南联蛊仙交手,因而演变出了庞大气潮。”

“我们作壁上观,争取渔翁得利罢。”

天庭蛊仙们相互交流,对气绝洞天牢牢注视着。

强大的气潮顺着孔洞,迅速流入到气绝洞天之中。

武庸、吴帅等人顿感前进艰难,原本狂暴强大的气流变本加厉。八转仙蛊屋虽强,但在这些气流面前也身不由己,时常被卷席偏离方向。

咚咚咚!

两座仙蛊屋逐渐靠拢中心,诸仙都听到了一声声沉闷的鼓声。

“这是怎么回事?”

“是什么气流,竟发出鼓音?”

“不,是心跳声!你们快看!”

两座仙蛊屋终于来到了气绝洞天的最中心点,所有的蛊仙都看到了这样的一幕——

在无数气流的集中交汇之地,宛若龙卷风中的风眼,周围狂暴而这里却是一片的平静。

在这静谧的环境中,有一颗气功果娇小无比,玲珑剔透,不断地吞吸着周围的狂躁气流。

无数的气流不管什么种类,被吸入气功果后,都在瞬间转变了性质,沦为气功果的一部分。

和之前纯粹如一的情况不同,此刻的气功果中,竟有一个婴孩的胎盘!

气功果仿佛孕妇的肚子,一个人类的婴孩蜷缩着身躯,在里面孕育,形体一分分地迅速涨大。

“怎么会这样?”

“这究竟是怎么回事?”

“这是怪物!它居然在无止境地吞吸八转仙材气流,不断壮大自己!”

“它要出世了,快阻止它!”

众仙勃然变色,吴帅刚想要催动梦里轻烟杀招,气功果似乎有感,猛地一颤,轻轻自爆开来。

婴孩顺利出世,小胳膊小腿,营养不良的样子。

这时,一片凹地的虚影在他头顶显露。这片凹地充满了玄妙意味,竟漫溢出九转的气息。但这份气息远比吴帅之前到手的人海雏形,要完善完美得多。

这是一片天地秘境!

婴孩忽然举起双臂,好似用双手虚托这片天地秘境的虚影。

“兮——”他长啸出声。

下一刻,周围的气流宛若海纳百川,倦鸟归巢,纷纷涌入到天地秘境的虚影之中。

气流卷席的速度是如此的惊人,竟在几息之内,统统投入到神秘的天地秘境里去。吴帅、武庸诸仙周围一丝气流都不存在了。

苍穹中,只剩下小孩和两座八转仙蛊屋。

“我们又出来了。”罗足发现自己来到了外界。

“不,是整个气绝洞天都化为了气流,被他统统吸了进去。”武庸难掩震撼之情。这一切超乎意料,他感觉自己是被利用了。

这是一个局!

“你到底是何人?”龙宫中传出吴帅的喝问。

婴孩已经长成童子模样,他鼻孔一吸,将头顶上的天地秘境虚影迅速吸入鼻中。

然后,他扫视周围,看了看龙宫和玉清滴风小竹楼,又立即发现了隐藏着的诛魔榜。

童子微微一笑,悠然唱道:

登山寻仙处,寸步间高险。

浮尘似光流,暗蛊藏心沟。

金玉如一梦,万年恨寂寥。

五域九天功,尽在一气中。

唱罢,他声调稚嫩,但语气沧桑地道:“老夫还能是谁?气绝魔仙是也。”

“气绝魔仙?!”众仙心头狂震。

“等等,若他是气绝魔仙,难道刚刚的那片天地秘境,便是《人祖传》中所记载的‘兮’?”诛魔榜中,周雄信想到了什么。

《人祖传》中有两片天地秘境十分神秘,具体的描述并不多。

一处是“乎”,一处则是“兮”。

------------

今天更新推迟及未来写作计划

人祖传写好了,但是回头一看,还是不满意啊。和当下的剧情无法贴合,如果直接抛出来,会有疏离感,简直就像是两本书了,无法用人祖传的情节,来推动常规的情节。

越是写到后期,就越难写人祖传。比同时写两本不同的书还要困难,因为要做到1+1>2。

所以,我打算重新抛掉人祖传,重新梳理,要晚一点。

还有,本来4月份打算双更的,结果个人能力不足,没有做到啊。实在惭愧!

这样,这个诺言我还是要完成的。我在四月份欠下的章节呢,就在今后,以及5月份还吧。

原本5月份打算完结此书,恐怕还是困难了一点,再看具体的情况吧。

啊啊啊……

我也想写得高产一点,但完全写不快啊。《人祖传》太难了,这种哲理寓言,我算是把自己坑惨了!

------------

\end{this_body}

