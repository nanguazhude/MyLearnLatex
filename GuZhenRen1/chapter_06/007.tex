\newsection{通缉方源}    %第七节:通缉方源

\begin{this_body}

气潮过后,天高云稀,一片清爽。

一朵黄云在高空疾飞,云尾在沿途洒下无数光沙。

黄云中坐着一位女仙。她身材肥硕,膀大腰圆,宛若水缸一般,正是西漠超级势力莫家的成员。西漠蛊仙界中有个她的称号——肥娘子。

肥娘子对于这个称号一点都不厌恶。她以胖为美,得知这个称号后,就十分喜欢,从此之后自己便公开自称。久而久之,连她原本的姓名都掩盖在这个称号之下了。

“咦?”黄云飞行之间,肥娘子忽然神色微变,察觉到了某处异常。

于是黄云微微一折,斜飞下去,稍稍改变了路线后,静止悬浮半空。

肥娘子从黄云上站起身来,俯视脚下的巨坑。

这个巨坑已经被风吹拂黄沙,自然掩盖了一部分,但是肥娘子却是西漠土生土长的人,对于沙坑这种地貌再熟悉不过。

“这不是自然形成的沙坑,而是当中有某个事物被抽走了,才留下的空地。”

“这股气息……是上古荒植的气息!”

肥娘子眼中陡现精芒,她立即明白:是有蛊仙偷偷出手,将这里的一株上古荒植给盗走了。

肥娘子冷哼一声,心中生出怒意。她虽然不知道这株上古荒植是什么,但这里是狼漠,是她莫家的领地。盗走这株上古荒植,就是盗取莫家的财物。

“这株上古荒植,既然能够隐藏自身,瞒过我族上下的侦查,定是不俗得很。恐怕它早就被某些人发现,只是一直没有取走。如今气潮会祸,这个盗取上古荒植的人必定是担忧气潮危及这株上古荒植,这才动手的。”

肥娘子暗中分析了一番,又细心查看现场。

来者手脚很是干净,没有落下明显的线索。她失望之下,只得用信道蛊虫通告莫家,希望家族能派遣侦查好手,将这可恶的盗贼追踪拿下!

肥娘子最后用信道蛊虫记录了一些现场的情况,便再次启程。她此次出来,自然是有要务在身,不能耽搁在这里。

一路飞行,她见到了一片绿洲。绿洲广阔,在中央有一座石头城池。城墙虽然矮小,但规格庞大,城中生活着海量凡人,正是沙狼城。

沙狼城乃是莫家的主要城池之一。

肥娘子并不遮掩行迹,直接从高空降落到城主府。

城主府中早有人准备接驾,见到肥娘子降临,纷纷跪倒一地,口中高呼:“我等拜见上仙!”

“都起来吧。”肥娘子撤销了黄云,脚踏地面,走到众人面前。

“上仙,您的接风宴会已经准备妥当了。里面有沙狼城最好的美酒佳肴,还有今天刚刚采摘下来的新鲜的瓜果。”城主谄媚地道。

肥娘子却是摆摆手,她没有兴趣,只道:“给我准备一间闭关的密室。另外我让你搜集具备天资和忠诚的蛊师,此事办得怎么样?”

“已经差不多了。城中绝大多数的蛊师都已经筛选完毕。”城主连忙答道。

“很好。三天之后,我会带走他们。现在,你下去去忙吧。”

“是,在下必定竭尽所能,将这个任务完成得尽善尽美。”城主连忙保证。

密室的门缓缓闭合,肥娘子又随手开启此中的蛊阵。

她面露疲惫之色,神念探入仙窍,见到仙窍内的景象,不免又是一声深沉的叹息。

她的仙窍原本是万里黄沙,一片平整。在沙漠中心地带,肥娘子精心布置了沙田六千多亩。

这些沙田主要生产两类仙材,分别是各色的光沙,以及冷暖流沙。这是肥娘子经营了许久的贸易。

但宿命一战,肥娘子承受了五根天道道痕,仙窍就开始发生演变。地貌变得起伏不定,沙丘越来越多。更有一道地沟,像是一刀直接横劈下来,触目惊心地横贯整个沙漠中心,千亩沙田损失惨重。

“我没有智道手段,根本无法推算,去克制这种种自然演变。除非是请智道蛊仙出手,但是我的家底浅薄,根本请不起啊。还不如将这笔费用节省下来,以防将来不测。”肥娘子心中哀叹。

莫家也有专修智道的蛊仙,但和肥娘子关系不近。事实上,就算关系亲近,智道推算也要耗费仙元,不能白白为肥娘子牺牲吧。所以,总得要一笔酬劳的。

“另外,我最近吞吸天地二气,也令仙窍不稳。”

此刻,五域界壁彻底消失,五域的天地二气没有了阻挡,正在不断地积极融合。西漠的天地二气也正在缓缓改变,这就导致西漠本土的蛊仙的仙窍根基,和外界的天地二气产生了差异。

蛊仙吸收了这些有差异的天地二气,必然会令仙窍不稳,造成动荡。

但蛊仙最好的应对方式,就是积极配合,不断地频繁吞吐外在的天地二气。若是一味闭关,将来等到外界的天地二气改变很大,再冒然打开仙窍门户沟通外界,造成的动荡将更加巨大,仙窍受损的程度会更加严重。

“要是我有气道造诣,对于天地二气造成的损失,也能弥补很多。”

智道是高大上的流派,气道流派则是式微已久,专修气道的蛊仙很少。

肥娘子的想法,只是种种幻想,不切实际。她最后只有自己安慰自己:她虽然没有智道、气道的帮助,但其他蛊仙也没有。所以,她不至于被淘汰,被立即甩下去。

肥娘子心中有着浓重的危机感。

这种现象,在当下的五域蛊仙心中普遍存在。

五域一统了,谁都明白将来的日子不会太平。所以不管是独自修行的蛊仙,还是各大势力都在为将来的大战积极准备着。

这一次肥娘子奉命前来沙狼城,就是为了莫家征集更多的蛊仙种子,加以栽培。

事实上,不只是莫家,更不只是西漠,整个五域乃至黑白两天中的大小洞天势力,都开始积极战备。

在此之前的过往时期,大体上五域两天都是比较和平的,超级势力栽培蛊仙,首要考虑的是资源供养和储备。

但现在各方都在积极战备,甚至穷兵黩武!

几乎所有势力都将资源放在了蛊仙后面。有一个普遍的观点:若是培养的蛊仙不够,将来敌人将自己消灭,来不及或者舍不得利用的修行资源都归了别人。那岂不是太亏了么。

肥娘子看着仙窍中的沙田,心在滴血。她无法舍弃这六千多亩沙田,这是她唯一的进项,是她修行的经济支柱。

肥娘子只能去力保。

未来难测。她根本不知道气潮会持续多久,但她也得硬着头皮去保。

“恐怕当今世上就只有方源,才会明白气潮究竟会持续多久吧。他毕竟拥有春秋蝉,从未来重生过来的。”肥娘子对这一点,着实羡慕不已。

肥娘子在仙窍中重建沙田,修补当中的蛊阵。

出现地沟的地方,显然是无法弥补的。所以,原本整整齐齐的六千亩沙田,就被劈成了两半。

“若是有仙阵就好了。”肥娘子有些后悔,莫家也有阵道蛊仙,当初铺设沙田的时候,莫家的阵道蛊仙还主动询问肥娘子,需不需要自己帮忙。

但肥娘子拒绝了他。

一方面是不想让自己的仙窍暴露,另一方面也是贪图便宜。

毕竟建设仙阵,需要她手中的一只仙蛊始终坐镇。肥娘子总共不过两只六转仙蛊。

其次,酬谢莫家阵道蛊仙也是不菲。

“若是当初布置了仙阵,至少我的损失会少很多啊。唉,谁能料到自家仙窍也不安全呢?”

肥娘子想到这里,不禁又暗暗羡慕族中的那位阵道蛊仙。现下这个时期,这位莫家阵道可是相当吃香。很多人都找他出手,为自己的仙窍中的资源点布置保护的仙阵。

肥娘子暗恨自己修行的土道,土道擅长改变地貌,但是在西漠土道蛊仙是最多了。肥娘子的土道传承也只是稀疏平常。

至于兼修?

肥娘子根本就不敢有这个狂妄想法。

她现在修行一门土道,还嫌精力、财力远远不够。

兼修两门就是自己作死!

况且,只有那些十分罕见的优秀传承,才有兼修的手段。比如雷鬼真君、龙公若留下传承的话。

肥娘子修补好了仙窍中沙田,还得采购一些沙硕。

沙田是以沙蕴沙,没有足够多的仙材沙石,会大大影响产量。

肥娘子取出蛊虫,沟通宝黄天。

宝黄天中各种消息纷呈,大量意志攀谈,无数神念不断交流。

值此天地剧变,蛊仙们不能随意走动,一方面他们得频繁吞吐天地二气,跟上世界的脚步,另一方面他们还得照看仙窍,和肥娘子一样不断修补维护。赵怜云放任不管的情况是比较罕见的。

所以,蛊仙们需要仙材,需要其他流派的蛊仙帮忙,因此宝黄天的交流就变得十分热烈喧哗了。

“我们这边又来了一场气潮,一片生灵涂炭。”

“这气潮究竟什么时候结束啊!真是该死,我在海里的渔场算是彻底毁了。”

“最讨厌的是行动不自由了。底蕴越强的仙窍,越要更长的时间才能调整天地二气,跟上五域的转变。”

蛊仙们一片唉声叹气,谈论着气潮的一切消息。

除此之外,就是关于方源的通缉令!

有神秘势力描述了方源的困境,并且出重资收购方源的最新情报。

方源拥有至尊仙窍,至尊仙体上道痕不互斥。更可怕的是,他能够吞食他人仙窍,跳跃灾劫,暴涨修为。但眼下,方源得到了太多的天道道痕,实力暴跌到谷底。谁发现他的具体位置,必有重大奖赏!

至尊仙胎蛊的秘密彻底暴露了,这不是天庭的手笔,而是紫薇仙子、正元老人受魔尊幽魂之名,发布出来的通缉令。

“太强悍了!这个至尊仙体真是恐怖,居然可以兼修所有的流派。”

“我终于明白为什么方源修为提升得这么快了!难怪一眨眼,他就成了八转大能。”

“谁能有这样的机缘啊?”

“上天为何偏爱一个天外之魔呢!我若是得到他这样的待遇,说不定也能得到他这样的成就。”

蛊仙们各种羡慕嫉妒恨。

“不能让他成长下去,他吞食他人仙窍,是我们蛊仙的祸害和天敌啊。”

“让那些个子高的头疼去吧。我只是个小人物,怎么会碰到方源呢。”

“这么多的仙材重赏,不知道会是哪个幸运儿能够拥有。”

“幸运儿?暴露了方源这个大魔头的位置,得罪了他,还说幸运?方源的战力我们都亲眼见过,就算是暴跌能有多低?人再虚弱,捏死一只蚂蚁也是轻而易举的。反正我就算无意间发现他,也不敢暴露他的情报。”

肥娘子听取各个蛊仙的议论,心绪起伏不定。

这个消息许多天前,就出现在了宝黄天中。一直喧嚣尘上,到现在还在讨论,热度极高,一点都没有衰减的迹象。

宿命一战,全天下人都看在眼里。方源的表现,的确让人惊悚震怖。亚仙尊的战斗力为天下所知,天下第一魔的名头已经安在了他的身上,无人不服,无人不晓!

“这样的大人物,我还是一辈子不遇到最好了。我就算再肥,也是小身板。卷入这种大能之争,连渣都不会剩下。”

肥娘子很有自知之明,她只想看个热闹。

然而她并不知道,自己曾经就和方源亲自交过手。甚至她刚刚发现的盗走上古荒植的盗贼,就是方源!

就在肥娘子闭关的时候,两个狼狈万分的身影,相互搀扶着,一步步挪到了沙狼城城门口。

城门的守卫并没有驱赶他俩,因为已经辨认出来这两人都是蛊师。

守卫只是走到两人身边,伸出手来:“入城费,每人一块元石。”

“还有入城费?”当中的一位落难蛊师瞪大双眼,正是彭达。

他搀扶着的中年蛊师,自然就是莫利了。

两人因为方源挪走千愿树造成的气浪而得救,费尽辛苦,终于回到了沙狼城。

“我这有。”莫利早有准备,取出两颗元石,放到守卫的手上。

彭达盯着这两块元石看。

一路上,他受到莫利的教导,已经可以内视自己的空窍,同时发现了自己身上的三只蛊虫。更明白了元石的价值,不只是货币,更能补充蛊师最宝贵而且稀少的真元!

“入个城,就要两块元石啊。”彭达感叹道,“莫利大叔,这钱我会还你的。”

莫利拍拍他的肩膀:“小子,走吧。大叔再落魄,也不愁这点小钱。这段时间,你就先住在大叔的家里。你人生地不熟,又身无分文,独自一人怎么活?”

莫利一路上指点彭达,发现他真的对蛊师修行一无所知,算是彻底相信他失忆的话了。

“大叔,真的太感谢你了。你救过我的命,又对我这般照顾……”

“说这些干什么!”莫利打断了彭达的话,领着他往城中走去。

莫利的家在沙狼城的中部地带,是一座大房子,门前的庭院栽种了不少花木。

“夫人,我回来了!”莫利敲开大门,正是他的妻子开的门。

“当家的,你终于回来了!”莫利的妻子是一位二转蛊师,见到自己的丈夫惊喜交加,“我听说了气潮降临,我还以为你……”

“我命大没有死成,不过商队全毁了,只剩下我和这位小兄弟活了下来。唉!”莫利大叹了一口气。

“活着还不够你庆幸的吗!”莫利的妻子翻了个白眼。

“大娘。”彭达适时地问候道。

莫利妻子笑呵呵地道:“小兄弟,我家老漠多亏了你一路照亮。”

彭达顿时脸上一红,赫然道:“忏愧啊,这次多亏了大叔照料我,没有大叔援手,我就要死在沙漠里头了。”

“出门在外,相互照应是应该的。快进来吧。”莫利妻子很是热情。

莫利和彭达二人历经艰险,终于安全。

在莫利的家里,匆匆洗了一把澡后,两人就各自倒在床上,呼呼大睡。

一直从中午睡到晚上,彭达被莫利唤醒,来到桌上大吃一顿。

酒足饭饱之后,莫利对妻子道:“咱家还有不少的余钱罢,这一次虽然商队毁了,但只要我人在,就能重现拉起一支商队来!”

但妻子却面露犹豫之色:“当家的,这个事情我还没来得及跟你说。我正在犹豫,万幸你活着回来了。”

“什么事?”莫利奇怪地问道。

“你走后不久,城主就发布了命令,筛选资质甲等,又对莫家忠诚的蛊师。据说这是上头下的令,要将这些蛊师栽培成蛊仙呢!”

“哦,还有这样的事情?”莫利迟疑了一下,“城主亲自发布的公告?我路过城门时却是匆忙了,并没有看到。”

莫利妻子继续道:“我们家的儿子听闻这个消息,就想加入。”

莫利失笑一声:“难得这小子想要求上进。可惜他只是乙等资质。”

莫利妻子笑道:“资质其实并不重要,重要的是对莫家的忠诚。只要展现出忠诚,咱们的儿子就有机会举荐上去了。”

莫利顿时恍然,问道:“多少钱?”

莫利妻子说出了一个数字,让彭达吃惊得张大了嘴巴。

莫利妻子愁眉不展:“就算动用了我们家全部的余钱,还是不够。当家的,你拿个主意吧。”

莫利沉默了片刻,毅然道:“我去借!难得我们儿子不再好吃懒做,要求上进,这是好事情。豁出去我这张老脸,定然能借来足够的元石。”

“当家的……”莫利妻子感动不已,又有些担忧,“只是咱们的商队已经全毁了,只怕……”

莫利笑了笑,拍拍自己的胸膛,自信地道:“别担心,我可是一位三转蛊师,并且身体硬朗,实力不损。谁都知道,这样的我只需要时间,不算糟糕的运气,就能翻身!”

------------

\end{this_body}

