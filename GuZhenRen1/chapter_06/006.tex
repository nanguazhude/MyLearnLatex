\newsection{方源救灾}    %第六节:方源救灾

\begin{this_body}

“醒醒,快醒醒,小子!”

彭达呻吟一声,勉强睁开双眼,见到一位络腮胡子的大叔,正用粗糙的手掌拍打他的脸庞。

彭达恍惚了一下,这才辨认出眼前之人的身份。

“莫利大叔。”他叫出声来,“我不是在驼背上睡觉的吗?”

彭达看着周围,不禁再一次蒙住了。

周围一片沙硕,商队毫无踪影,只剩下他和莫利二人,形象都狼狈不堪。

“痛啊。”彭达呻吟一声,他发现自己身上到处都是伤痕。

莫利大叔望着彭达,叹了一口气:“你小子,真不知道该说你什么好,命还真是大!很多蛊师拼命逃离,都死了。你在睡梦中,却是活了下来。不过,若非是自己及时发现了你的微弱气息,把你挖出来。恐怕你也被活活埋死了。”

“莫利大叔,你又一次救了我啊。”彭达抓住莫利的双手,感激不尽,“但这到底是怎么一回事?”

“这是一场天灾,我也从未见过。”莫利大叔发出一声沉重的叹息。

“那……咱们的商队呢?”

“唉,现在已经只剩下我们两个了。”

什么?!彭达不由地瞪大双眼,心中大叫了声我曹:“这个世界怎么这么险恶,动不动就是天灾啊!我只是睡了一觉,就差点又死了!我的天,我究竟来到了一个什么样的世界。简直是不把人当人看,生活环境也太恶劣了吧!”

“至少我们还活着。”莫利敲了敲彭达的脑袋,“看你样子,你小子还不满足?在那样的天灾中能活着就是最大的万幸了!”

“你看我。”莫利用手指指着自己的胸口,“我奋斗了大半辈子,才拥有了这么一支商队。现在都没了!看开一点吧,这鬼日子只有这样了。唉,若是能够成仙该多好!”

“成仙?这个世界上还有仙人吗?”

莫利看了彭达一眼:“你真的把什么都忘了?!连仙人的事情都忘了?唉,以后再向你解释吧,咱们先离开这里。”

彭达便跟着莫利启程,前往最近的绿洲。

彭达跟在莫利身后,小心翼翼地问道:“这种灾祸多吗?”

“当然!”莫利用沧桑的语气道,“咱们在沙漠中讨生活并不容易。沙尘暴就是经常造访的杀手。有时候还会有飞刀飓风,每一股风都能凝聚成巨大的风刃,所到之处,切割万物。除了自然灾祸,还有恶兽。比如这里就是狼漠,为数最多的便是沙狼。”

嗷呜——!

莫利正说着,一群沙狼忽然从沙漠地底钻了出来。

“狼,狼!”彭达吓得跳起来,“天哪,它们居然从沙底下钻出来了!好多头,怎么办啊大叔?!”

莫利面色凝重,出声咒骂:“该死!这些沙狼怎么就没有被天灾消灭呢。”

不过想想也不奇怪。

气潮乃是天地二气相互交融形成的现象,根源在于天道。天道从不杀绝,万事留一线生机。莫利、彭达能够活着,自然其他生命也有存活下来的可能。

沙狼群从四处冒出,但很奇怪地,虽然发现了彭达和莫利二人,却没有来攻击,而是迅速集结,口中呜嚎,紧盯沙漠地面。

簌簌簌簌……

一连串渗人的沙响,一只只金色蝎子从地下钻出来。这些金蝎一个个都有磨盘大小,看得彭达心中寒气直冒。

金蝎群和沙狼群展开了惨烈的厮杀。

沙狼但凡被金蝎的尾针蛰中,就会立即口吐白沫,倒在地上,最终死亡。而金蝎也难以抗衡沙狼的爪牙,常常被狼爪撕烂。

两群野兽厮杀,智慧薄弱,对近在身边的莫利、彭达视而不见。

彭达看得心惊胆战,脸色发白。这个世界也太危险了,不管那一方兽群获胜,最后肯定还会拿他开刀。

“我们必须突围!”莫利咬着牙关,面色坚毅。

“可是大、大叔,我们周围到处都是金蝎和沙狼啊。”彭达欲哭无泪。

“你想等着被吃吗?”莫利说着,就要动身,“小子,能战吗?”

“啊?我,我不能啊,我失忆了。”

“这些沙狼和金蝎可不管你是否失忆。跟紧我,尽全力自保。突围至少还有希望。”莫利说到这里,哈哈一笑,“如果突围失败,你就要去填野兽的肚皮了。不过你放心,你不是一个人上路,还有大叔我呢。”

彭达不禁大翻白眼,与其这样的死法,他倒是宁愿直接被活埋啊!

莫利开始突围,彭达大叫:“大叔,等等我!”

他只能拼命跟在莫利身后。

他们俩不动弹还好,一有举动,顿时惊动了蝎群和狼群。

几乎同时,就有数头金蝎和沙狼,一左一右杀向他们。

莫利低喝一声,催动凡蛊,但只击退了两头野兽,两人就被兽群包夹。

“突围失败了!”莫利叹息一声,放弃了反抗。

彭达抱头抓狂,陷入了绝望,双股战战:“我,我要死了吗?!”

至尊仙窍。

此刻,不管是小五域,还是小九天,都是一片混乱,乱象纷呈。

小南疆中山峦崩塌,地貌起伏不定;小中洲里河流改道,形成泛滥洪水;小西漠中沙暴漫天,吞没各处绿洲和城池……

小北原。

刮起了狂风暴雪,一朵朵洁白的火焰,在风雪中飘摇,附着到哪里,就灼烧到哪里。

这是异火仙材——寒冰焰。

在冰雪道痕浓郁的地方,会有一定程度上的概率,产生这种奇妙的火焰。这种火焰中大半都是冰雪道痕,但是焰心却是浓郁的炎道道痕。

“绝不能让这些寒冰焰四处扩散!”雪民蛊仙雪儿正在主持大局,全力赈灾。

她被方源任命,经营三圣山,执掌雪晶阵,一心操持雪民一族的繁衍壮大。

寒冰焰在三圣山中扩散,一旦被火焰附着,不管是雪民还是雪怪,都在痛嚎声中被灼烧殆尽。

雪民们一片混乱,相互奔逃,踩踏至死的情况正在不断地发生着。

混乱的汹涌人潮中,只有少数的雪民蛊师正努力地维持秩序。但很可惜,依凭他们的修为也难以对这些仙材寒冰焰有什么办法。

“不好,寒冰焰附着到了冰道晶精之上了!”雪儿脸色煞白,一时间焦急万分。

她刚刚为了救下更多的雪民族人,结果顾此失彼,让寒冰焰出现在了雪晶大阵中。

冰道晶精乃是雪晶大阵的核心之物,方源用大阵将冰道晶精散发出来的冰霜寒气,,均匀持续地传输、覆盖到周围每一寸地方去,将周围改造成适合雪民生存,雪怪产出的环境。

雪儿悔恨交加,看着寒冰焰不断灼烧冰道晶精,却是无能为力。

雪儿不禁噙泪,自责不已。雪晶大阵是方源托付给她的,是雪民一族壮大的根基,结果因为她的失误而面临毁灭。

她觉得自己愧对雪民族人,更愧对方源的期待。

但就在这时,一道身影忽然出现。

“方郎!”雪儿一愣,旋即脸上涌出惊喜之色。

方源伸手虚抓,连抓几记,就将大阵内的所有寒冰焰都摄取了过来。

方源忽然闷哼一声,身躯微微一颤,全身上下浮现出一道道的苍白丝线。

从这些苍白丝线上,雪儿顿时感到一股充天彻地的浩瀚威严!

“这是……天道道痕!天哪,这么多的天道道痕,足有上千吧?”雪儿吃惊地捂住嘴巴。

“接下来就交给你了。”方源对雪儿微微一笑,旋即带着寒冰焰,瞬间消失在雪儿的面前。

雪儿顿时心头一空,旋即又涌现出浓郁的担忧:“方郎身上竟有这么多的天道道痕,难怪他在宿命大战后,便四处潜行,极少出手。唉,我真是太没用了,根本就不能帮助他分担什么。就连这份雪晶阵也没有守护好!”

方源出现相助雪儿,缓解了小北原雪民一族根据地的危机后,就又立即回到闭关的密室,继续潜修。

炼制宿命蛊时,方源位置得天独厚,再加上他全力收取天道道痕,导致他身上的天道道痕足有三千多道,道道完整无缺!

可以说是,红莲筹谋的大计中最大的获利者。

只是这个好处太庞大了,方源一时间也难以消化。至尊仙体道痕不互斥,天道道痕加身,也就意味着至尊洞天中增添了三千多道天道道痕。

这些天道道痕对于整个至尊洞天,产生了剧烈的影响。

这些影响范围巨大,并且涉及方方面面。大到山川起伏,小到一条山间小溪的改道。

“天之道,在于损有余而补不足。万物平衡,相互制约。即便在雪晶阵这样的极端环境中,天道也可以改易,令冰雪道痕中产生炎道道痕。一旦让寒冰焰肆虐之后,小北原的环境就会面目全非。”

“当然,放任不管的话,我的至尊洞天会因此受益,建立成一副完善的生态平衡,前景广阔。”

“只是,这种方法太过耗费底蕴了,并且也是红莲魔尊想要看到的!”

方源冷冷一笑。

宿命蛊已毁,红莲魔尊已经大计得逞,方源对他的利用价值也就没有了。所以红莲算计到这一层,利用天道道痕来限制方源,约束他的成长速度。

偏偏方源虽然早就算到了这一点,但是也不得不主动跳下这个坑。因为方源发现至尊仙胎蛊也可能被魔尊幽魂动过手脚,方源需要凭借这些天道道痕,来遏制住魔尊幽魂的手段。

红莲一直在利用方源,从七转层次的未来身杀招就可看出来。但方源也在利用红莲,只有利用这位魔尊的大计,方源才能顺利地解决宿命蛊这个天大的难题,同时利用这份百万年的筹划,帮助他自己跳出魔尊幽魂的棋盘。

赵怜云任凭天道道痕演化仙窍,方源却是主动掺和一手。

天道道痕在改变至尊仙窍,他也积极参与,既抗衡又配合,全力保存自家底蕴。

就是这番合作和抵抗,让方源感悟良多。

天道的恢弘无情,人道的积极抗争,这两者在方源的心中交相辉映。

“时间差不多了。”良久,方源再次动身。

这一次,他直接收起至尊仙窍,真身本体来到五域外界。

他身在西漠,眼前有一株矮树。

这矮树高不过八尺,树枝细短歪曲,宛若怪爪张扬,显得丑陋不堪。但奇妙的是,它绽放出巨大的光影。这光影极其巨大,高达五六十丈,光影如树,树上繁花茂盛。树枝洁白似雪,花朵粉红若樱。花叶之中还夹杂着累累小果,并未成熟,颜色各异。

正是那株千愿树。

千愿树周围,则是方源布置下来的仙阵。

此刻,方源收起仙阵,连带着千愿树也连根拔起,皆被送入至尊仙窍之中。

此番动作,立即引得风云激荡,一股澎湃的气浪向四周汹涌扩散而去。

方源并不管这些,身化虹光,直射天穹而去。

气浪卷席周遭沙漠,发出隆隆轰鸣之声。

彭达、莫利突围失败,被沙狼群围住,正要下手,气浪蔓延而来。

沙狼群顿时混乱不堪,四处溃逃。

彭达骇然:“又是天灾到了!”

莫利瞪大双眼,看了一眼,顿时笑道:“哈哈,我们得救了!这不是天灾,充其量只是一场飓风罢了。我们快走!”

ps:感谢盟主修仙得道西兰花的打赏!

感谢盟主¥陈天宇¥的惊人打赏!

感谢盟主方源劳模的打赏!

------------

\end{this_body}

