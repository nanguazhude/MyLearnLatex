\newsection{李小白咏春}    %第十节:李小白咏春

\begin{this_body}

%1
“两位蛊师大人,请进,请进!”店小二点头哈腰,将华松、安崇迎接进店里来。

%2
“这是京城著名的茶楼,十分热闹,我曾经多次来过。”华松暗中传音,给安崇介绍道。

%3
两位蛊仙伪装成了蛊师,来到华文洞天的京都。

%4
安崇倒更愿意按在云头,俯瞰京城。但既然华松有这样的乐趣,他也只好客随主便了。

%5
“我要五楼楼上的雅座。”华松显得轻车熟路。

%6
两人来到五楼,进入房间,打开窗户就看见街道上人潮汹涌,第一会场就在街道的那头,青铜大门前几乎挤满了人。

%7
“这便是我华文洞天的当代才子们啊,也是我华文洞天未来的希望。尊使请看。”华松感叹一声,递给安崇一只五转的侦查蛊虫,可观察目标才气。

%8
安崇端详了一番,当面用了,顿时视野就发生了变化。他看到无数才子的头顶上,都有着才气。这些才气五彩斑斓,各有高低,形态不一,着实令他大开眼界。

%9
“既然是有如此蛊虫,可侦查才气,为何还要举办这样的选拔呢?”安崇问道。

%10
华松呵呵一笑:“尊使有所不知。才气便如同修为,就算拥有众多,临场发挥不出也是枉然。况且此次选拔,都是当场作诗,做不得假。更考验书生才子们的素养,许多才气浓郁的书生,未必有灵感,能够做出最高水平的诗词来。”

%11
安崇点点头:“我看这里布置的大阵不仅相互勾连,似乎还能够助长书生们的才思?”

%12
华松点头:“不错,尊者慧眼!在这座仙阵中,书生们往往都能超常发挥,更加体现出各自的才情资质。”

%13
说话间,房门被敲响。

%14
店小二得到华松的允许后,端来一盆盆的美酒佳肴。

%15
“八宝野鸭、金丝酥雀、熊猫蟹肉,都是我们茶楼的招牌菜,请两位客官享用。”店小二道。

%16
华松随手赏了他一枚元石,将店小二打发掉。

%17
华松对安崇介绍道:“尊使,这里的茶虽然是凡茶,不过却是华语老大人年轻时的创作,当时他是一位四转蛊师,夺得状元,因此此茶又被他命名为状元茶。”

%18
“哦?”安崇顿时起了兴趣,这可是八转蛊仙年轻时候的作品。

%19
“那我得要好好品尝一番。”安崇喝了一口,闭上双眼,细细品味,恍惚间,心头便升腾起一股微微的兴奋,好像是苦笑十数载,终于功成名就,人生得意近在今朝!

%20
“好茶,好茶。”安崇真诚赞叹,“这茶虽然是凡茶,但竟有一丝人道韵味!”

%21
宿命大战,中洲天庭连续施展的数记人道杀招,招招威力绝伦,轰动天下。人道之威可谓无人不知,无人不晓。

%22
就在二仙品茗之时,第一会场的青铜大门悠悠打开。

%23
拥挤在门前的人潮,顿时发出轰响。

%24
“打开了,打开了!”

%25
“别挤啊。”

%26
“快让我进去。”

%27
人潮汹涌,纷纷涌入门内。

%28
李小白也在人群之中,不过位于后端。

%29
他一边随着人流往前,一边揣摩着此次选拔的规则。

%30
“天下诗会共有十八个会场,每一个会场都有一个题目,所有人同时答题。不管人数多少,最终都依凭作品的质量,只有一半的人能够晋级。”

%31
“晋级到后面的会场后,仍旧是取半。如此一路晋升上去,打通整个十八会场,才算是达到了选拔的要求。”

%32
“若是中途失利,就要退回之前的会场。若是一连失败,直接退出第一会场,那就算是彻底失败了。”

%33
“不过,天下诗会共举办七天。每个人都有三次彻底失败的机会。”

%34
“如此一来,足以让华文洞天方面,挑选出优质的蛊仙种子了。偶尔失误失常,还有重来的机会。但如果有哪位才子,连续七天都发挥失常,正说明了本身素养就不足,根本不配洞天栽培成为蛊仙。”

%35
李小白收回思绪,已是来到了第一会场。

%36
会场一片空阔,挤满了书生。

%37
书生们有男有女,有老有少,人数之多,一眼望去,没有一万也有数千。

%38
这还只是第一天。

%39
李小白在会场中又等待了一炷香的时间,第一会场这才将所有的书生才子接收完毕。

%40
人山人海,一片嘈杂之声。

%41
好在第一会场乃是仙阵空间,可以随时如意地扩张,容纳更多的书生也在能力范围之内。

%42
咚咚咚!

%43
一阵鼓声,同一个声音传入书生们的耳畔:“天下诗会第一题——春,时限半盏茶。”

%44
说完这话,声音就消失不见了。

%45
“第一个题目是春?”

%46
许多书生皱起眉头,也有许多书生喜上眉梢。

%47
李小白暗暗沉思:“咏春的诗词着实太多了,这个题目乍一看很容易。毕竟书生们或多或少,都创作过有关题材的诗词。虽然天下诗会要求当场创作新诗,但只要将原先著作稍微修改一番,有时候也能勉强算是新诗。”

%48
“然而实际上,这个题目还是有难度的。”李小白一脸沉着之色。

%49
他知道,这一次创作诗词,主要还得和周围的人进行比较。只有比半数书生都要强,他才能够成功晋级。

%50
意识到这一点的书生不在少数。

%51
很多人都开始苦思冥想起来,有的人直接盘坐在地上,有的人则背负双手,四处踱步,有的人垂着头,口中轻轻呢喃。

%52
李小白想的却是:“我该抄哪一首呢?”

%53
咏春的诗词,在他记忆中藏有很多,不乏传世经典。

%54
但第一次就拿出传世经典,并不太好。这让李小白以后不好发挥了。他才气规模并不属于第一流的层次,若是场场拿出传世名篇来,肯定惹来怀疑。

%55
若是他运气好,李小白或许还会稍稍激进一些。但现在明显气运低迷,李小白便以稳重为先。

%56
在李小白思考的时候,已经有不少书生创作出了新诗。

%57
于是各种光晕闪耀,光晕五颜六色,或强或弱。每当光晕在书生的身上消散,书生们就都有收获。

%58
有的人得到了蛊虫,有的人直接拔升修为,有的人恢复了真元,有的人消散了疲惫。

%59
这是济文才杀招。

%60
华文洞天的铸造者,最初始的原主,在陨落之前将这记杀招布置下来。正是有了济文才杀招,才鼓励越来越多的书生奋发读书,最终形成华文洞天这般文风鼎盛的现状。

%61
李小白终于敲定了他的诗。

%62
他轻轻地咳嗽了一声,开始朗诵。

%63
“月夜。”

%64
“更深月色半人家,北斗阑干南斗斜。”

%65
“今夜偏知春气暖,虫声新透绿窗纱。”

%66
李小白朗诵刚毕,耳畔就听轰隆一声轻响,全身升腾出浓绿的光辉。

%67
腾腾腾。

%68
李小白周围的书生,无不感到一股无形的压力,绿光逼得他们向四周散去,留出好大一块空白的地方,只剩下李小白一人站在中央。

%69
“好,好强的光辉!”

%70
“名篇出现了啊!”

%71
“没想到这么快,就有名篇问世。不知道是什么人创作的?”

%72
许多书生都被打断了思绪,一个个用羡慕、探究的目光,盯着李小白。

%73
李小白表面云淡风轻,实则心里有些纠结:“哎呀,抄的有点过了,效果也有点太好,惹来不少注目。”

%74
他扫视四周,心中盼望着会有一些人站出来,不要让他如此独领风骚。

%75
绿光则不断地钻入他的空窍之中,令他的修为向上攀升。

%76
“哦!有名篇出世了。我来读读看。”茶楼中,华松顿有所感。

%77
读了一遍李小白的著作,华松满意地连连点头:“妙啊,妙啊!这个李小白年纪轻轻,作诗却是老道。”

%78
“通常咏春的诗词,多是柳绿桃红。但这首诗却偏偏反其道而行之,写夜幕来遮掩春光,可谓巧思。”

%79
“全诗后一句,必定是李小白从自身深切的生活中挖掘而得。透露出一股清新、欣悦,充满生机的意境。”

%80
“观诗如观人,这个李小白心境上佳!”

%81
华松啧啧赞叹一番,像是品味到了世间绝味之佳肴。他看着安崇,微笑道:“不知尊使对此诗有何见解呢?”

%82
安崇顿感头大,腹诽着:“如果你不说,我还不知道这诗有这么些好处呢。哎呀,眼下要我点评,我该怎么办?”

%83
ps:9点第二更。

\end{this_body}

