\newsection{天下诗会}    %第九节:天下诗会

\begin{this_body}

%1
太古黑天。

%2
“来了!”安逊兴奋地低呼一声。

%3
在他的悄然注视下,一支上千头的魂兽群向这埋伏圈徐徐赶来。

%4
一进入埋伏圈,安逊便立即发动早已埋伏好的己方魂兽,对野生魂兽进行围攻。

%5
而安逊一边指挥魂兽大军,一边催动仙蛊,发出种种仙道杀招。

%6
他的杀招覆盖下去,每每都令己方魂兽战力暴涨,速度、防御等等方面激增不小。

%7
这场战斗持续了半柱香,安逊成功斩杀了四百多头野兽魂兽,而其余的魂兽都被他俘虏。

%8
“主上!”安逊将一堆战利品魂核,主动供奉给魔尊幽魂。

%9
魔尊幽魂张口一吞,将这些魂核尽数吞入体内,开始徐徐消化。

%10
自从上一次,他暗中跟随安逊外出,便将他暗中收服之后,魔尊幽魂的局面就便算打开了。

%11
他暗中传授安逊种种魂道修行的奥义,并且根据安逊掌握的魂道仙蛊,亲自为他开创魂道杀招。

%12
安逊实力不断上涨,接连领取外出俘虏魂兽的任务。

%13
魔尊幽魂专门为此,开创了两个手段,一个能侦查到野生魂兽的位置,另一个则是一座简陋仙阵,布置下来后,可以吸引野生魂兽前来。

%14
有了这两个手段,安逊捕捉野生魂兽的效率大大提升,每一次的任务都完成得很好。

%15
寒灰仙姑将自家侄儿不仅没有辜负自己的期待,而且还更进一步,展现出了光彩,心中欣慰不已。

%16
安魂洞天中,只有寒灰仙姑一位八转,自然是以她为首。

%17
在她之下,就是安逊、安崇两人。上一次冰晶仙王主动开放洞天,邀请寒灰仙姑,寒灰仙姑就专门带上了他们两个,一方面增长他们的见识,另一方面是给他们露面的机会,让他们结识一番同等层次的蛊仙。

%18
寒灰仙姑在洞天中商议的时候,安逊、安崇在洞天门户之外,也和其他洞天势力的七转精英混了个脸熟。

%19
寒灰仙姑却不知道,安逊早已经成为魔尊幽魂的奴仆。甚至整个安魂洞天,包括她在内,都笼罩在了魔尊幽魂的阴影下,前途危险。

%20
任务完成得越来越多,越来越好,安逊在和安崇的竞争中,占据了明显的上风。

%21
寒灰仙姑看在眼里,其余的洞天蛊仙也看在眼里。

%22
这一天,寒灰仙姑将安崇召唤到面前来,吩咐道:“自上一次冰晶仙王主动号召我等商议之后,他便积极推动,想要迅速促成洞天本土势力的大联合。但是中洲黑天这一片,异族众多,我人族却只有两位八转。幸好我已联络上骷髅姥姥,他是东海黑天的八转蛊仙。你此去东海,便是代表安魂洞天,和骷髅姥姥接触,表明我等联手的诚意。”

%23
“是。”安崇领命,心中苦涩。

%24
他知道寒灰仙姑的用意。

%25
派遣他这个任务,是要彻底帮助安逊奠定胜局,让他成为安魂洞天着重栽培的对象。

%26
若是安逊表现得和自己差不多,安崇自然会不忿、恼怒。但偏偏安逊表现得极其良好,远超过他,就连安崇也明白自己取胜无望。

%27
他接下这个外交重任,又从安魂洞天的库藏中领了不少仙材,充当此次拜礼。

%28
离开安魂洞天,他就直接穿过天罡气墙,下到五域来。

%29
他遮掩行迹,从中洲去往东海。然后再从东海升空,穿越天罡气墙,去往骷髅姥姥之处。

%30
界壁存在的时候,蛊仙们要跨域十分不容易。很多强大的蛊仙都选择从黑白两天中取道。

%31
但现在界壁消失了,就连安崇这样的黑天蛊仙,都选择走五域路线。

%32
黑天中环境险恶,碰到太古荒兽,可就麻烦了。安崇虽然仙窍中藏着一头太古魂兽傍身,但他到底只是七转修为。尤其是眼下,气潮泛滥,五域蛊仙们都纷纷休养生息,外出活动十分稀少。所以安崇取道十分方便。

%33
一路上安崇发现,五域也遭受巨大的灾害,被气潮祸害得不轻。

%34
他倒是没有遇到什么意外,顺利地来到东海,随后升上太古黑天。按照寒灰仙姑留给他的联络方式,他顺利地来到碎骨洞天,见到了骷髅姥姥。

%35
“宿命已毁,气潮汹涌。值此天下大变之机,我等若不联合的话,恐怕未来难测。尤其是我等人族势力,定要同心协力,紧密团结。”骷髅姥姥没有刁难安崇,很快就答应下来。

%36
骷髅姥姥又道:“然而,我东海黑天这里,也是异族势力更胜一筹。寒灰仙姑想要联合我等,那冰晶仙王又岂会单单坐视?他定然也在四处联络黑天其他洞天的异族势力。”

%37
安崇知道骷髅姥姥不会凭空废话,连忙请教:“不知前辈有何见解?”

%38
骷髅姥姥道:“我们既然要联合太古黑天的人族势力,为何不连太古白天也一起算上呢?也不瞒你,不久前我已联络了不少人,反响积极。在他们当中有一个重要人物,还需要贵方出力争取。”

%39
骷髅姥姥顿了顿,继续道:“此人也是八转蛊仙,人称华语老仙,执掌华文洞天。他在东海白天这块,有着很大的声望。如果将此人争取过来,必定能够带出一串的白天蛊仙,能节省我们很大精力。”

%40
安崇闻言暗忖:去参见华语老仙本不是他的任务范畴,但自己若是能超额完成的话,却是能够有机会,和安逊再度竞争。

%41
想到这里,安崇便下定了决心,对骷髅姥姥道:“前辈既有如此美意,晚辈便前往华文洞天走一遭。”

%42
骷髅姥姥赞了一声:“好,不愧是安魂洞天的人。华语老仙生性不喜打打杀杀,所以一直对老身的联盟提议兴趣不大。你这次去若是能够成功,老身我这里重重有赏!”

%43
安崇便离开碎骨洞天,前去太古白天,拜见华语老仙。

%44
华语老仙虽然对联盟兴趣不大,但安崇此行却是代表着寒灰仙姑。对方同样是八转蛊仙,华语老仙因而并不怠慢,打开门户,将安崇迎接进来。

%45
见到华语老仙之后,安崇将一路上早已揣摩千遍的说辞拿出来说,但华语老仙却是不为所动。

%46
“我华语洞天不涉世征伐,自成一体,恪守中立。然而虽不参与联盟,但你我两方却是可以互换资源。”

%47
安魂洞天中盛产魂道资源,有许多对于华文洞天而言,十分适宜。

%48
安崇见说服不了华语老仙,能够有这样的交易也挺好。双方彼此之间还是头一次接触,以后交易多了,交情不就有了吗?

%49
有了更深厚的交情,再提出联盟之意,那就又有新的希望了。

%50
辞别华语老仙,安崇却被其他蛊仙留下。

%51
招待他的是蛊仙华松,华松对安崇道:“尊使此次长途跋涉而来,且务必逗留几日。恰巧凡间朝廷举办天下诗会,届时会有许多华美文章创出,实在是叫人期待。”

%52
安崇早已从骷髅姥姥那里得知,华文洞天这里以诗词歌赋为美。蛊仙一个个饱读诗书,喜好文章远远胜过美酒美人。

%53
安崇还想和华文洞天搞好关系,虽然自己对文章诗词什么的不感兴趣,但也不好拂了华松的一番美意,当即应承下来。

%54
天下诗会召开了。

%55
这是前所未有的盛事,在华文洞天之前的历史上从未有过。因此举办出来,立即吸引了诸多才子,从华文洞天的各地蜂拥而来。

%56
除了有志于在天下诗会大展风采的才子,当然还少不了看热闹的人。

%57
所以,作为诗会召开之地的京城,一个多月前就已经是人满为患。

%58
人潮汹涌的街道上,李小白和姜先生并肩而走。

%59
“天下诗会共设了十八个会场,小白啊,你要改变你此时的困境,眼下就是最好的时机。你的才华我了解,足够你通过前十个会场。但是后面八个,你却要看运气了。唉,为师这一次也帮不了你,说不定还会成为你的对手。”姜先生叹息道。

%60
这样的盛事,姜先生也非常想参与其中,和各方的才子进行一场交流和较量!

%61
李小白闻言,就有翻白眼的冲动,心中叹息:“我现在几乎已经肯定,老师你会成为我的对手啊。”

%62
李小白乃是方源分身,是特意投入华文洞天,攻略洞天的一枚棋子。

%63
刚开始的时候,李小白一路顺风顺水,甚至得到了苏琪涵苏家大小姐的青睐。李小白因此成为了天下十大才子之一,并一度有可能成为苏家赘婿。

%64
然而,就在李小白上京的路上,天地震动,山洪倾泻。李小白一行人伤的伤,死的死,被困在山谷里数个月后,李小白这才脱困,险死还生。

%65
他因此误了时期,没有见到当朝皇帝。没有皇帝的首肯,之前天下十大才子的荣誉也被他人替代。

%66
李小白想要联络苏琪涵,却得知苏琪涵得知他遇难的消息,立即动身前往施救,结果却是失踪了,至今未归。

%67
李小白来到京城,屡遭挫折,被人暗下毒手,无意间得罪权贵,很不如意。若非老师姜先生接济、相助,他恐怕要流落街头,成为一个乞儿。

%68
他心头则是一片雪亮。

%69
“本体虽然毁掉了宿命蛊,但是强大气运却是几乎耗尽在这一战中。连累我这个分身,也变得倒霉万分。”

%70
“这一次天下诗会,乃是华文洞天积极储备蛊仙种子的选拔活动。老师虽然不清楚这个实质,但既然参加了,必定会成为我的阻碍。”

%71
“我现在这么倒霉,一定会遇到老师。届时我击败老师,恐怕我这位老师要恨我一辈子。可是这个机会简直是千载难逢,我若不抓住,迎难而上,将来再要遇到可就难了!”

\end{this_body}

