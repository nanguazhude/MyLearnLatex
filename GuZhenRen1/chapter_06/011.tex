\newsection{李小白纠结}    %第十一节:李小白纠结

\begin{this_body}

%1
李小白的期盼没有落空。

%2
在他咏出名篇之后,很快就有三人相继脱颖而出。

%3
一位壮汉,带着金色面具,浑身笼罩在一层绚烂夺目的金光之中。

%4
一位老翁,头戴斗笠,手拄拐杖,一道青气缭绕周身。

%5
还有一位女子,身姿曼妙,一身粉裙,面带薄纱,咏了诗词之后,身边形成无数光蝶飞舞,美不胜收。

%6
这三人都创造出了名篇层次的诗词,和李小白的月夜一个档次。

%7
李小白暗自松了一口气:“这个世界拥有蛊虫,可以辅助创作。再者蛊师也可延寿,拥有更多生命的体验和积累。所以,相比较地球,名篇更容易出现。”

%8
《月夜》在地球上,绝对算是难得的佳作。但是放到华文洞天中来,并不能傲视群雄。

%9
包括李小白在内的四人,各据一方,相互对望。

%10
壮汉、老翁和女子都将更多的目光投注在李小白的身上。

%11
他们对彼此的身份,都有猜测和了解,但是李小白却是忽然冒出来的。

%12
“这个年轻人是谁?”

%13
“才气虽有,但不是特别雄厚啊,竟能创作出名篇来?”

%14
“呵呵,有意思的少年郎。”

%15
李小白的老师姜先生也看到了自家学生的表现。他欣慰地点点头:“很好,李小白,看来你这一次超常发挥了。继续努力吧。”

%16
结果没有异常,李小白成功晋级,被传送了出去。

%17
定神一看时,他又来到了另一处会场。

%18
“这是第五会场,请诸位书生耐心等待。”一道声音传入李小白的耳中。

%19
李小白顿时明白过来:“原来晋级也有区分。我弄出来的诗,足够我连跨三个会场,直接跳到了第五会场了啊。”

%20
“这个规矩倒是挺合理的。”

%21
“只是这样一来,能够到达这里的蛊师,无一不是创作出名篇级的诗词。在这些人中取半数晋级,压力顿时暴涨了数倍。”

%22
李小白皱眉凝望四周,他发现了两位熟面孔。

%23
正是之前的壮汉和女子。

%24
发现了李小白的目光,女子向他轻轻点头,薄纱后面的容颜似乎微微笑了笑。而那位壮汉却是傲气得紧,瞥了李小白一眼后,就开始闭目休整了。

%25
整个第五会场,暂时就只有他们三人。

%26
“也不知道那个老头子,是晋升到哪个会场去了。”

%27
“我还是装作休整的样子吧。”

%28
书生们创作诗词,无不是拼尽全力去想,惟独他只是要找一找合适的诗词罢了。这就太过轻松了,需要稍加掩饰一番。

%29
李小白在第五会场等了将近半盏茶的时间,终于开始了第二场比试。

%30
他的竞争对手一共有二十几人,这不禁让李小白心里压力剧增。

%31
第二场比试的题目出来了,两个字——旅行,时限仍旧是半盏茶。

%32
“旅行?这个题目和春一样,同样宽泛,好写,但创作出名篇佳作来,难啊!”壮汉心中叹息。

%33
女子则在思量:“这题目我并不擅长,如何是好?”

%34
李小白也在琢磨:“该抄哪一首呢?”

%35
他想了片刻,觉得不能找太过经典的,那效果肯定要爆炸,但又不能找差太多的。差了点,自己就要失败,从第五会场跌落回去。

%36
这让李小白颇感为难,此中程度真的不好拿捏。

%37
他对自己是了解的,但是对其他人却是不够熟知,不清楚究竟该拿出什么份量的诗词,能够让他胜出,并且又不大出风头。

%38
“第一会场,我的做法有点欠妥。这一次我就不争第一了。”李小白暗自告诫自己。

%39
他耐心等待。

%40
大约半盏茶的功夫,有书生作出了诗。随后接连几人,纷纷朗诵自己的作品,各有光晕斑斓。

%41
李小白松了口气,暗道:“看来名篇不是随便就能作的,这些人在第一场超常发挥,这一场算是正常水平吧。”

%42
李小白心里渐渐有了谱,但他仍旧没有急着动手,他还在等待。

%43
时限快要到的时候,他终于等到了名篇问世。

%44
是那位壮汉。

%45
他朗诵的诗词铁马金戈,充满了战意,使得他全身金芒滚滚,似乎有铁骑刀枪在嘶鸣碰撞。

%46
第二个名篇是那位女子。

%47
她创作的诗词婉约温柔,描述一位大家闺秀倚窗而望,听闻旅人们的叙述,自己幻想着自己的旅行,心思精妙。

%48
“好了,终于该我了。”李小白咳嗽一声,徐徐道,“簌簌衣巾落枣花,村南村北响缲车,牛衣古柳卖黄瓜。酒困路长惟欲睡,日高人渴漫思茶。敲门试问凡人家。”

%49
这一次他是抄的苏轼的浣溪沙。

%50
朗诵完毕之后,不管是壮汉还是女子都为之心头一动,暗自分析。

%51
“好词!这首词前面写景,后面抒情,情景交融,富有情趣。”

%52
“虽是写景,但却用声音串联,不同一般的景物堆砌,非常传神。而后面抒情,意趣盎然。这个少年郎虽是蛊师,但却谦和有礼,不贸然传入凡人的家院,果然是有君子之风,温文有礼啊。”

%53
“读他的词,就可见他的人品。这个少年书生真的不错!”

%54
周围的书生们看向李小白的目光,隐隐发生着一些变化。

%55
李小白吐出一口浊气,装作疲惫不堪的样子。

%56
“这一场比试,我应该能晋升。不仅如此,表现得也很完美。既没有太出风头,也没有太低微。”

%57
李小白暗自满意,稍稍等了一会儿,时限到了。

%58
他却有些傻眼。

%59
绝大多数的书生都没有创作出诗词来。

%60
“此番天下诗会,我定要拿出最佳的水准来。若是搪塞之作,别人不说,我自己也会自惭形秽的。”

%61
“我尽力了,但没有创作出我满意的诗作,就算失败,我也不会后悔!”

%62
“此番能够听到三位的名篇,已经是不虚此行了啊。”

%63
“哈哈,大不了掉到第一会场重新来就是了。告辞!”

%64
那些没有创作出诗词的书生们走的很洒脱,另一些凑出新作的书生们则有些面带愧色。

%65
李小白表面淡定,内心纠结不已:“你们怎么可以这么高风亮节啊?”

%66
他虽然晋级了,但这一次又出了风头。刚刚他感觉还挺满意,没想到这些书生这么不济事。

%67
“哼,都怪我太高估他们了。”李小白腹诽不已。

%68
没的说,他成功晋级,来到了第九会场。

%69
耐心等待了一会儿工夫,人终于到齐了。这一次人更少,十人还不到。

%70
许多人都是有着名望的才子,相互熟知,各自打着招呼,表面上云淡风轻,实际上却是暗自紧张。

%71
到达这一步,竞争的压力又上涨了数倍。

%72
“这次我一定要稳住,不能太出风头了!”李小白暗自捏拳,不断地告诫自己。

%73
但这一次的对手们却都发挥得很好,一半以上的蛊师都作出了名篇层次的诗词来。

%74
最后只剩下李小白一人。

%75
李小白表面紧张,内心从容,将早已准备好的诗词朗诵出来。

%76
一番光影变动,他的修为直接拔升到四转,还新添了一只蛊虫。

%77
这一次,他精心筹谋,选取的诗作质量刚刚卡在线上,使得他成功晋升,但却以胜者中垫底的身份。

%78
就在李小白有些得意的时候。

%79
噗!

%80
有人忽然大吐一口鲜血。

%81
哐当。

%82
有人直接昏倒在地。

%83
壮汉身躯一晃,吃力地盘坐在地上。而那位女子则咳嗽了好一阵子,面色苍白如纸。

%84
惟独李小白一人,浑然无事地站在原地。

%85
众人的目光再次集中在他的身上。

%86
“这个少年很强啊!到目前为止,他已经一连创作了三个名篇佳作了。”

%87
“他居然没有受伤,显然刚刚他创作时,没有拼尽全力啊。”

%88
“厉害!诗会之后,我一定要和他结交!”

%89
李小白:“……”

%90
他无语了。

%91
我的天,你们都这么拼的么?!

%92
不就是创作诗词吗,何必如此拼命呢?

%93
他很想表演一番自己吐血的模样,但是事发突然,现在他就算要装,也已经来不及了。

%94
李小白只好继续面色平淡,内心纠结地进入到下一轮。

\end{this_body}

