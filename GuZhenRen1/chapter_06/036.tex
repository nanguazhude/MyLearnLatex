\newsection{兮的碎片}    %第三十六节:兮的碎片

\begin{this_body}

太古白天。

彩霞飞舞,绚烂缤纷。

东海七转蛊仙青辉子在漫天的光霞中飞舞,屡屡施展仙道杀招,不断地收取这些霞光。

霞光被他收纳到手中,捏成一颗颗的圆球。

圆球随后又被送入他的仙窍之中,埋在云土里。一段时间之后,这些霞光圆球便能破壳而出,生根发芽,长出一株株的幼苗。再加以雷水、磁音浇灌,最终就能结出流光果。

流光果乃是六转仙材,本身是由浓郁的极光凝结而成。赤橙黄绿青蓝紫等等各种颜色都有。

青辉子没有直接踩在流光果,而是拥有独到的种植手段。正因如此,使得他成为宝黄天中三位贩卖流光果的主要蛊仙之一。

“有了气潮冲击天罡气墙,我也能透过气墙的缝隙,进入太古白天中搜寻仙材。这是我的机缘啊。”

青辉子暗自感慨。

霞光在两天中相当常见,而且产量极其丰富。

青辉子主动收取霞光,使得栽种流光果的成功下降得极多,能令他大赚特赚。

青辉子一边收取霞光,一边时刻关注着下方的天罡气墙。

他并没有多少的实力能够在太古白天中生存,甚至他连天罡气墙都突破不了。所以他必须时刻注意时间,一旦被气潮冲刷出漏洞的天罡气墙有弥补的迹象,他就要迅速回归东海了。

这可不是说笑的。

一旦青辉子麻痹大意,被封堵在了太古白天之中。依凭他的实力,恐怕是等不到气潮再次冲刷,就要在太古白天险恶的环境中饮恨身亡了。

可以说,青辉子是冒着生命危险来采集这些霞光的。

蛊仙修行不易,青辉子纵容已是七转蛊仙,也要为生计筹谋。

“兮——!”

就在这时,青辉子忽然听到一声长啸。

随后,风起云涌,青辉子惊骇地发现,他下方的天罡气墙居然被一股恐怖的无形力量牵扯。

几个呼吸之后,方圆数十里的天罡气墙就都投入到最中央的漩涡中去了。

“怎么回事?!”青辉子吓得肝胆俱裂,“难道是哪一个太古荒兽出没?但我还从未听过有这样威势磅礴的太古荒兽!”

天罡气墙彻底消散,露出一块凹地形状的天地秘境。

而后,这片天地秘境虚影很快又投入到了一个童子模样的蛊仙鼻中。

青辉子看到这位蛊仙童子,顿时身心狠狠一震,宛若冰雪从头顶浇下去,一直浇到他的内心深处去。

“啊,气绝魔仙!”青辉子陷入一片绝望之中,“我怎么这么倒霉?只是采集彩霞,居然碰到了气绝魔仙!!”

气绝洞天一战,虽然发生没有多久,但根本隐瞒不住,已经是天下皆知。

虽然没有众目睽睽杀招,令天下蛊仙亲眼目睹整个战斗的过程,但是气绝魔仙的相貌和手段,都传播开来,流传很广。

在蛊仙界中,有数位人物是公认的不能招惹,一旦遇到就要扭头逃跑的恐怖存在。

排在第一位的,便是方源。

这个魔头手段众多,实力超绝,并且行事无所顾忌,心狠手辣至极,连天庭都拿他没有办法。

吴帅也在名列其中,他是龙人首领,掌握龙宫,奴役帝藏生,对人族抱有深深的恶意。

而气绝魔仙便是新晋增添进来的人物。

至于天庭、气海老祖因为他们的阵营,倒不会随意欺侮弱小。因此不在这个名单之中。

青辉子看到了气绝魔仙之后,立即就辨认出了他,吓得呆愣在原地,宛若石像。

气绝魔仙皱着眉头,看着自己的手掌。

他的手掌仍旧是又小又嫩,身体仍旧是童子之躯。

回想起重生的那一幕,气绝魔仙心中暗恼。

若是再给他一段时间,他定然可以完美重生,拥有成年人的身躯。但被孽龙横冲直撞,他不得不提前复活,导致身躯彻底定型,潜力有限。

吞吸了这片天罡气墙,他总算是将之前激战中的损耗弥补了。

休整结束,他又要启程。

“小辈过来。”气绝魔仙看向青辉子,随意呼喝道。

青辉子一个激灵,满脸苦色,只得提心吊胆地乖乖飞到气绝魔仙的面前。

气绝魔仙微微一笑,伸出手指一弹,顿时探出一股幽魂气流。

气流迅速地将青辉子完全包裹,青辉子下意识地想要挣扎,但耳畔立即传来气绝魔仙的低喝:“不要乱动!”

青辉子咬牙,不敢再妄动分毫。

他放弃挣扎,任凭这些气流钻进他的肉身,在他的魂魄中恣意流窜。

流窜了好一会儿,气绝魔仙这才将青辉子体内的气流全都吸纳到自己的体内。

一瞬间,他接受到了海量的情报,对当今时代的了解深刻了数十倍。

“果然是个精彩缤纷的大时代。有意思!”气绝魔仙双眼放光,啧啧有声。

旋即,他又命令青辉子道:“将你手中的蛊虫都拿出来。”

青辉子面皮狠狠抽搐了一下,有那么一瞬间,他想出手反抗。但很快他的理智占据了上风,他乖乖地将所有的蛊虫都交了出来,递给气绝魔仙。

这番举动倒是让气绝魔仙重新打量了他一下。

气绝魔仙取走这些蛊虫,一一品鉴,对于青辉子的光道仙蛊,他只是看了一小会儿,重点反而放在那些极品凡蛊身上。

尤其是搜魂蛊这类的蛊虫,让气绝魔仙大感兴趣。

到最后,他竟然将光道仙蛊都还给了青辉子。

青辉子大感意外。

气绝魔仙笑道:“小辈,你很识趣,没有妄图反抗老夫。你以为老夫是什么人?岂会贪图你的些微仙蛊?快滚吧。”

青辉子先是难以置信,随后狂喜:“气绝前辈的大恩大德,晚辈没齿难忘。”

“哈哈哈,我搜你的魂魄,何谈什么恩德?”气绝魔仙大笑,“你尽管记恨老夫好了,这根本无所谓。这个时代虽好,但纵观天下,尊者不出,能够值得老夫重视的也没有几个。老夫虽然缺少仙蛊,但还不至于沦落到抢夺光道仙蛊的地步。老夫要夺就要夺气道的仙蛊!”

青辉子顿时心头一动:“前辈,您是想去气海?”

但下一刻,气绝魔仙却已不耐地一挥袖子。

青辉子顿时被一股气流死死包裹,动弹不得,任凭气流带着他直接飞射出去。

气绝魔仙哼了一声,看也不看青辉子,照准气海的方向俯冲而下。

“也不知道这气海老祖究竟如何,呵呵,希望他手中的气道仙蛊不会让我失望。”气绝魔仙信心十足,直扑气海。

他在东海的高空疾飞,周身气流夹裹,速度极快,跨越一片片的海域。

他丝毫也不遮掩行迹,一路带着轰鸣的声浪,震天动地,魔道巨擘的汹汹威势显露无疑。路途中他碰到数位东海蛊仙,把这些人吓得四处乱窜。

“咦?”半途中,气绝魔仙忽然神情惊愕,在高空止住了身形。

他感受到了自家天地秘境的震动。

“难道是?”气绝魔仙微楞,旋即想到了什么,脸上涌现出一抹惊喜之色。

他暂时将前往气海的计划抛之脑后,方向一折,向西南方飞去。

一段时间后,他来到了一片海域。

这里的天空白云茫茫,遮盖视野,不见星光。就连烈日高照的时候,阳光射不破浓厚的云层,只是稍微增添一点光亮罢了。

海水表面上一片平静,内里却有无数股的激流。这些激流五花八门,水质皆不相同。它们相互穿梭,在这片海域中流淌,朝夕改变,毫无规律可言。没有蛊仙的防护手段,凡人尸躯在刹那间,就被巨流冲刷成渣。

正是乱流海域。

气绝魔仙仔细侦查,片刻后,他喜上眉梢:“妙哉,妙哉。这片海域内心必然有一片兮的碎片,因此才造成如此的乱流景象。”

“没想到我重生之后,气数如此强盛。”

“先取了这块碎片,融汇一体,再去找那气海老祖的麻烦罢。”

念及于此,气绝魔仙一头扎入海水之中,再不见踪影。

\end{this_body}

