\newsection{陆畏因请喝茶}    %第一百一十七节:陆畏因请喝茶

\begin{this_body}

%1
咕嘟咕嘟。

%2
茶水烧开了,升腾缕缕黄褐色的气雾。

%3
在山顶的小木屋中,一个小茶几对面却是坐着影响天下的强者。

%4
一位是当今天下第一魔头——古月方源。

%5
另外一位是南疆乐土传人陆畏因。他或许综合战力还登不上亚仙尊一级,但在防御手段上却妥妥能和亚仙尊较量。

%6
窗户大敞着,山间清新的空气传入肺腑,混合着茶的香气。

%7
方源远眺,窗外是一片青山绿水,世外桃源。白鹤翻飞,灵猿啼叫,荒植荒兽寻常可见,性情温驯。

%8
山脚山腰都有树屋、村庄,炊烟袅袅,鸡鸣狗叫,一片安静祥和的氛围。

%9
而在半空中,浮着许多座的小山。这些山都是浮土捏造,能天然悬浮于空中,小山上各有特色,环境各异。有的青藤纠结,如发垂下。有的细雨纷飞,映照七彩红光。也有的瀑布宣泄,一股股水流一直垂落下去,直至汇入地面上的宽阔河水之中。

%10
而方源身处的这座山,则是矗立在地表,是菇人乐土的最高峰。

%11
方源刚刚坐下,便开口询问陆畏因:是否幽魂未亡?

%12
陆畏因只是微笑,没有回答,而是埋头做茶。

%13
茶做好了。

%14
这是太康清心茶,运用八转仙材制造。茶香清新,又似乎混合了一丝雨后泥土的味道。

%15
茶水和这住处,还有这片菇人乐土都相得益彰。它们并不缥缈,并不是仙人高高在上,而是充斥着一股农家乐趣,平凡的生机。

%16
“请。”陆畏因将杯盏双手端送到方源面前。

%17
杯中的茶水刚刚还滚烫,但几个呼吸之后,热气就彻底消散,成了凉茶。

%18
方源轻轻品了一口,但陆畏因却直接端起大碗似的茶杯,仰起脖子,喝了一大口。

%19
方源眼中精芒一闪,顿有所悟,也学着陆畏因的喝法。

%20
咕咚。

%21
喉结滚动之下,方源咽下一大口茶。

%22
这一喝,方源大感不同。茶水并不冰冷,而是用一种温润的凉,瞬间浸透方源的五脏六腑。心中燥热之气一扫而尽,茶水清润爽口,还带着一丝甘甜。

%23
方源咕咚咕咚,将碗中的茶水直接喝光。

%24
“好茶。”他放下茶碗,面露异色。

%25
只是喝了一碗茶,他的体内竟增长了上千土道道痕!

%26
蛊仙渡一场天劫,平均的道痕收获也只是七百五十而已。而地灾平均下来,道痕收益在两百五。

%27
这碗太康清心茶,居然增长这么多的道痕。等若蛊仙冒着生命危险,渡了一场天劫和地灾。

%28
陆畏因这才徐徐道:“这片菇人福地得到乐土仙尊大人的救助,改良成了乐土,从此无灾无劫。乐土仙尊大人还在这里留下了一份道德真传,这片菇人乐土因此改名为道德乐土。”

%29
“乐土仙尊在未成就尊者之前,便在这里修行学艺。等到他成就了尊者,改造成道德乐土后,他还驻留在这里生活了一段时间。太康清心茶便是在这个时间段,乐土仙尊大人开创出来。”

%30
“食道虽然流传了下来,但被幽魂敝帚自珍,几乎是单脉流传。乐土仙尊大人一生都在弥补幽魂之殇,为天下苍生拨乱反正。因此,他对食道研究很深。”

%31
“他在道德真传中留下了许多食道成果。在他看来,食道发展潜力极大,只是修行的人太少。若真正发展起来,必定能够造福世人。就比如这碗太康清心茶,本质上是将土道八转仙材消耗形成的食道杀招。所以,它才能够为蛊仙增添土道道痕。”

%32
“尊者才情,果然是深不可测。”方源赞叹不已。

%33
乐土仙尊为了克制幽魂的食道手段,开创出了吃亏是福、众生皆苦两大杀招。这两个手段在伏杀魔尊幽魂的战斗中,起到了关键作用。

%34
但乐土仙尊做到的不只是这一地步,他还将食道推上了更高层次!

%35
现如今的五域蛊仙界中,几乎所有的茶、酒,皆是食道蛊虫残方。这已经是得到蛊仙公认的事情。

%36
单单喝酒、喝茶,只是图一个口舌的畅快,彰显自家财富底蕴,亦或者是平日里锻炼炼道手法的成果。

%37
而乐土仙尊却是彻底补足了这份残缺,将茶、酒等等完善成食道杀招。如此一来,食道终于显露出它的独到光彩了!

%38
“乐土仙尊的食道手段,能补全茶、酒,将其推成杀招。品食之后,蛊修就能增添道痕。这绝对是改变蛊仙界格局的手段!”方源心中了然。

%39
这个手段影响极大!

%40
历来,蛊仙要增长道痕,最主要的手段还是渡劫。

%41
拥有这等食道手段,等若打破了这个现状。

%42
方源拥有吃力仙蛊,也能做到这一点。但仙蛊唯一,而乐土的这个食道手段的理念可以借鉴,从而开创出更多类似的杀招来。

%43
吃力仙蛊这种东西不能推广,但杀招是可以推而广之的,这完全是两个概念!

%44
方源眸光深幽起来。

%45
单从这份太康清心茶,就可看出陆畏因的深厚底蕴。

%46
这片道德乐土底蕴深厚,绝不缺乏土道八转仙材。陆畏因单凭此茶,就能积累出远超常规的土道道痕。

%47
难怪他能够和幽魂如此拼杀。

%48
方源再透过窗外,看这片菇人乐土。他心中原本对这处基业的评价,不由地拔高数筹。

%49
有着这个食道手段,对于蛊仙修行极有好处。道德乐土中的蛊仙也不少,这些蛊仙的战力很可能远超同转!

%50
而陆畏因至始至终,都只是一人出手,从未命令过这些福地中的蛊仙亮相。

%51
此人的城府和隐忍,再次叫方源刮目相看。

%52
察觉到方源目光中的深意,陆畏因摆手谦虚道:“方源仙友,切莫高估了我以及这片道德乐土。我们虽然掌握着这个食道手段,但施展的代价着实高昂。道德乐土

%53
中盛产土道资源,而除我专修土道流派之外,绝大多数的蛊仙皆是菇人,他们是修行毒道的。还有数位人族蛊仙,修行木道、水道。”

%54
“这种食道手段,其实效率并不高。比如这碗太康清心茶,是消耗大量的土道八转仙材,方才有数百土道道痕的收获。”

%55
“当然,方源仙友乃是至尊仙体,道痕不互斥,土道道痕收获应当超越常规。”

%56
陆畏因最后总结道:“就连乐土仙尊大人也在真传中遗言,他说这个食道手段和巨阳仙尊、元莲仙尊、红莲魔尊等等相比起来,根本不值一提。”

%57
尊者的手段都是改天易地,气魄雄阔至极。

%58
比如元始仙尊,领袖人族崛起,开创天庭。星宿仙尊开创智道,玩弄天意,真正奠定天庭屹立万载而不倒的基石。

%59
巨阳仙尊留下血脉,建立八十八角真阳楼,将整个北原占作自家庭院。

%60
红莲魔尊布局百万多年,摧毁了宿命蛊。

%61
元莲仙尊留下手段,形成神帝城。壁画世界中推衍人道,奠定人道基业。

%62
这些都是改变天地,影响的不只是一代人,而是整个世界发展的无上事绩!

%63
方源微微皱眉,听陆畏因这话的话音,让他有了一些猜测。

%64
他直接就问:“既然乐土仙尊遗言中提及元莲,是否已知元莲仙尊会打造神帝城?”

%65
陆畏因点点头:“乐土仙尊大人行走天下,拯救苍生的途中,获悉了元莲仙尊的手笔。元莲仙尊当年留下豆神宫,便是为了遏制幽魂,趁着他还未成尊之时谋算他。青仇的核心乃是仇恨蛊,经由豆神宫中孕育,青仇和豆神宫关系紧密至极。”

%66
“元莲仙尊的本来谋算,乃是在宿命大战时期,将青仇、豆神宫和帝君城合并,形成神帝城,保卫天庭。”

%67
“然而,在此之前却出现了意外。方源仙友和房家合作,将青仇赶跑,抢走了豆神宫。随后豆神宫纵然和帝君城合并,但却缺少了青仇。”

%68
“没有青仇,神帝城战力不足。居于其中的元莲意志,恐怕是想保住人道成果,索性没有去天庭参战。”

%69
方源神情复杂:“原来如此。”

%70
元莲的计划,原本很完美。

%71
宿命大战时,青仇能感知到幽魂残魂所在,它本身拥有九转仇恨蛊,乃是亚仙尊战力。不仅能够帮助龙公维护天庭,而且还能剿灭幽魂残魂。

%72
可惜的是,方源无意中将这个计划破坏了。

%73
青仇原本在豆神宫中,好端端的孕育着仇恨蛊,壮大自身。但被赶跑后,这个过程就被打断了。导致后来宿命大战发生时,青仇战力不足,还和豆神宫分散。

%74
直至宿命大战结束,元莲意志便和秦鼎菱联络,这才有了赤心行者、九灵仙姑联手,将青仇重新捉回天庭的秘密行动。

%75
所以事实上,不是元莲仙尊算计了方源,而是方源先破坏了元莲仙尊的大计。这才导致元莲意志没有办法,只能在大战时忽然发动,对计划进行弥补。

%76
那么,方源究竟又是如何甘愿接受盟约,和房家展开合作,驱逐了青仇的呢?

%77
其中的关键契机是大盗仙蛊。

%78
有了大盗仙蛊之后,方源以杀招鬼不觉9为核心,大盗仙蛊7等等为辅助,开创出了大盗鬼手杀招,将青仇身上的八转魂兽令强行盗取了出来。

%79
没有了魂兽令,青仇实力大减,最终令房家获胜。

%80
而大盗仙蛊来自于哪里?

%81
其实就是房家收获的那一份盗天真传。

\end{this_body}

