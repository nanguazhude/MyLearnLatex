\newsection{八转仇恨蛊}    %第五十三节:八转仇恨蛊

\begin{this_body}

%1
西漠。

%2
八转层次的激战还在继续。

%3
青仇怒吼,舌头吐射,速度之快宛若迅雷。

%4
砰砰砰……

%5
包围着青仇的血色人影,被舌头轻易贯穿,崩解爆散。

%6
每一个血色人影被击爆,赤心行者的脸色就白上一分。但他眉头紧锁,牙关咬合,斗志不减一分一毫,仍旧全力维持着战场杀招。

%7
“这样下去,赤心行者坚持不了多久的。”和青仇交战着的九灵仙姑一直在注意着赤心行者的状态。

%8
他们好不容易牵制住青仇,让赤心行者撑起战场杀招,暂时圈住了这头太古传奇荒兽。

%9
然而青仇拥有不弱于人的智慧,它在发现自己破解不了这处战场杀招之后,就将主要的攻势都集中在了赤心行者的身上,以图突破。

%10
一旦赤心行者坚持不住,战场消散,那么青仇必定会扬长而去。有了这次经历,它必定警惕至极,短时间内再想要捉拿它可就是千难万难了。

%11
“也罢!只有使用这一招了。”九灵仙姑一咬牙,毅然做出了某个重大决定了。

%12
仙道杀招九灵变!

%13
她陡然身躯一震,从兽体变回人形。她是变化道大能,按照常理而言,必定是变形之后战力更强。但她此刻还原人形却是气势勃发,反而更胜之前一筹。九股灵光从她的仙窍接连绽射而出,形成一个庞大的九色光罩,光罩分为九层,层层色彩鲜明。

%14
青仇一直将注意力放在赤心行者身上,九灵仙姑忽然爆发令它大吃一惊,连忙分调火力。

%15
但九灵变杀招进行到这一步,已经算是酝酿完毕,再无破绽。

%16
九灵仙姑清叱一声,九色灵光罩猛地一震,九层光混淆一体,变得极其缤纷绚烂。

%17
随后混彩光球猛地一变,变作一头犀牛,头大身小,浑身五光十色,就向青仇撞来。

%18
青仇怪啸一声,暂时舍弃赤心行者,不闪不避,悍然冲向这头犀牛。

%19
轰的一声巨响,地动山摇。

%20
两个庞然大物相撞之后,竟然是青仇小退一步,大头犀牛只是停留原处而已。

%21
青仇又惊又怒,没有料到这头犀牛竟然如此强大。在它看来,这头彩光犀牛身形半透明,还能看到位于中央的九灵仙姑,一点都没有实体变化的那种厚重感觉。但是一次交锋,却是让青仇明白:这头彩光犀牛比之前的实体变化只强不弱!

%22
“青仇在对撞中落入下风,这还是交战以来首次!九灵变不愧是变化道的传奇杀招。”赤心行者看到这一幕,不禁又喜又忧。施展九灵变杀招代价很高,蛊仙修成九灵变杀招之后,一生当中只能运用九次。

%23
如此珍贵的杀招,九灵仙姑选择将其中一次用在了青仇身上,显露出了她志在必得的决心!

%24
青仇怪啸一声,被激发了凶性,再次杀上九灵仙姑。

%25
但九灵仙姑身外的混彩光球却是猛地一变,从大头犀牛变化成了南明火雀。

%26
火雀双翅一振,一飞冲天。

%27
青仇扑了一个空,来不及抬头,直接催动防御手段。

%28
果然下一刻,青仇就听到一声火雀的尖啸,然后背部被狠狠一撞。呼啦一声,漫天的火焰将它卷入其中。

%29
青仇仓促之间催动的防御手段,只是支撑了几个呼吸,便瞬间被破。

%30
青仇悍然反击,一时间兽吼声,雀啸声交相争鸣,两方打得难解难分。

%31
赤心行者终于有了喘息之机,连忙抓紧时间进行调整。他一边调息,一边观战,不由心忧。

%32
“这青仇实在是皮糙肉厚,虽然处于下风,战斗节奏被九灵仙友牢牢掌控,但它却有充足的耐力。”

%33
“反观九灵仙友虽然占据上风,把握战局,却是不可持久。若无外力,拖延下去,必定是脱力阵亡的下场。”

%34
赤心行者想到这里,决定冒险。

%35
他耐心等待,暗中蓄势。足足等了个把时辰,终于让他见到了战机。

%36
机会只有那么一瞬间,赤心行者蓦地出手!

%37
他张口一吐,吐出一口鲜血。这鲜血芬香扑鼻,飞到半空中滴溜溜旋转,好似一颗微型血日,绽射无穷血光,直扑青仇。

%38
青仇虽然和九灵仙姑酣战,但从未放过对赤心行者的警惕。

%39
然而,当它运用杀招去抵挡这颗小小血团时,却毫无用处。血团好似虚幻,穿越阻击,直接射在青仇的脑门上,化为一道红日印记。

%40
青仇顿时变色,发觉自己每一个动作,都会牵扯到红日印记。

%41
这是一个极强的封印,开始迅速发威,牵制它的动作,干扰它的杀招,降低它的斗志。

%42
“该死,该杀!!!”青仇怒吼咆哮,掀起攻势狂澜。

%43
九灵仙姑拼力纠缠,节节败退。

%44
轰的一声巨响,青仇在爆发中终于打破了赤心行者的战场杀招。

%45
然而,赤心行者的嘴角却浮现出胜利的微笑。

%46
青仇浑身上下都被一道道血色丝线束缚,亿万根血色丝线的源头都是它额头正中央的血日印记。

%47
“成功了。咳咳咳!”赤心行者咳出大口大口的鲜血,脸色惨白如纸。

%48
九灵仙姑连忙压制住行动愈加缓慢的青仇,同时传音:“赤心仙友,你可需要我的帮助?”

%49
赤心行者缓缓摇头,艰难地调动杀招,强忍着剧痛开始疗伤。

%50
九灵仙姑费了一番手脚,终将青仇镇压。

%51
看着被血红丝线捆扎得如同粽子似的的青仇,九灵仙姑赞叹道:“你这是何手段,居然能俘虏青仇!”

%52
赤心行者勉强回了一口气,答道:“这是我临时自创的手段,还未起名字呢。”

%53
九灵仙姑看向赤心,不由微瞪妙眸:“仙友真叫人刮目相看!”

%54
赤心行者微微摇头,苦笑:“别高估了我。我这是侦查到了青仇的本命仙蛊乃是八转仇恨蛊,这才有此一招。换做其他传奇太古,这就是笑话了。之所以能施展此招,也是多亏了九灵仙友你不计代价,催动了九灵变杀招,这才让我有了偷袭的良机。”

%55
九灵仙姑也跟着摇头:“俘虏青仇,关乎捕杀方源的大计,这是为天庭,为天下苍生的大事,我怎会顾惜自身这点小小的代价。倒是这只仇恨蛊,莫非是人祖传中记载的那只?”

%56
“正是。”赤心行者点头。

%57
九灵仙姑感慨道:“这也就能说得通了,为何青仇能够分辨出它的仇敌。可惜它没有杀招彻底运用。我们俘虏了它,将仇恨蛊取出来用,以天庭的杀招定然能够寻出方源、幽魂等等魔头的准确位置!”

%58
赤心行者收摄气息:“走吧,此地不可久留。”

%59
他们在这里激战良久,虽然用战场杀招节省了大量时间,还掩藏了绝大多数的战斗踪迹,但必定有蛊仙赶来这里巡视侦查。

%60
别的势力不说,莫家集结的大批人马说不定已经快要赶来了。

%61
九灵仙姑、赤心行者激战良久,状态都不佳,当即押解着青仇赶往中洲。

%62
果然,他们离开只是一小会儿,就有莫家的数位蛊仙驾驭着仙蛊屋,气势汹汹地来到此地。

%63
“这里发生了激战!”

%64
“虽然痕迹很少,但已可看出是八转层次的大战。”

%65
“难怪我族蛊仙陨落得如此迅猛。”

%66
“快追,我们有仙蛊屋,他们逃不了多远的!”

%67
莫家一群人循着踪迹,追杀天庭二仙而去。

%68
又过了数天。

%69
风沙滚滚中,一位蛊师艰难跋涉而来。

%70
他来到了战场中央,看着深坑已经被风沙掩埋大半,他仰天咆哮。

%71
这人正是杀死彭达,夺走了盗天真传的莫利!

%72
他千辛万苦地赶来这里,想要一睹杀害他妻儿的凶手真面目,结果却是来迟了。

%73
“我一定要找到你,把你碎尸万段!”

%74
“不管你是谁,是什么怪物,我都要你死啊!!”

%75
莫利跪在沙漠中仰天长啸。

%76
而在他的身边,漂浮着彭达的魂魄。彭达冷笑:“呵呵呵,你区区一个凡人,也想要报复仙兽?”

%77
“仙兽又能怎样?”莫利双眼血红,死死瞪着彭达的魂魄,神情狰狞至极,“我有盗天真传,我有盗天真传!我一定能修成蛊仙,成功报仇!”

%78
彭达再笑:“还有一种可能,这头仙兽早已经被蛊仙们杀死了。你已经没有报仇的对象了。”

%79
莫利一下子愣住,沉默如石。

%80
好一会儿,他猛地伸手,一把握住彭达魂魄的咽喉,状若疯癫:“不!它不会死的,我决不允许它死!它一定要死在我的手里!!”

%81
莫利接着大喊:“我要修成蛊仙,我有盗天真传,我一定能报仇!我一定能报仇的!说,快给我解释下一只蛊虫该怎么用!”

%82
莫利掐着彭达的魂魄,同时催动蛊虫,莫利手掌和彭达接触的地方迅速浮现出缕缕白烟,并发出嗤嗤的声响。

%83
彭达魂魄顿时发出尖锐的嚎叫,痛不欲生:“我说,我说,这只叫做电路蛊。”

%84
“电路蛊?究竟是什么东西。”

%85
彭达魂魄迟疑了一下:“这个东西叫我怎么解释呢?你不是天外之魔,不能理解什么叫做电路。简单来讲,就是雷电行走的路。”

%86
莫利顿时嗤笑一声:“这有什么难懂的。老子又不是没有看过《人祖传》!《人祖传》中就有明文记载,雷电蛊为了寻求出路,选择和炎煌雷泽合作,最终他们走出了太古蓝天。这应当就是电路的真意了!”

%87
彭达魂魄愣住了。

%88
莫利赞叹不绝道:“盗天魔尊真不愧是尊者啊。居然有如此奇思妙想,他一定是从《人祖传》中获得的灵感,因此才开创了出了电路蛊啊!”

%89
彭达魂魄:“……”

\end{this_body}

