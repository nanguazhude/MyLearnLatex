\newsection{人道改运}    %第十二节:人道改运

\begin{this_body}

至尊仙窍,小中洲。

方源站在千愿树下,仰望苍穹,只见天空中升腾着炙热光斑,点点团团,时隐时现,绚烂夺目至极。

再看地面周围,沙尘滚滚,风声呼啸。

方源眼眸中闪烁着冷光。

天道道痕的数量正在增长!

原本他因为炼蛊而得了三千多道道痕,如今已经逼近三千五百道。

伴随着天道道痕的增长,方源仙窍中原本的各大道痕,正不断减少。

这一幕,不禁让方源回想起了他在疯魔窟的见闻。

在疯魔窟的倒数第二层中,方源见到了世界的演变。无数的世界破灭,凝聚成天道道痕。天道道痕相互接连,亦或者自行扩散,化为一个个的新生世界。

此刻,至尊仙窍虽然没有摧毁,化为天道道痕。但天道道痕却是在不断地改变着仙窍中的种种其他道痕。

当然,天道道痕也不是一味地增长。有时候,它也在衰减,一条天道道痕化为更多的其他道痕,扩散到天地之间。

通过这些道痕的改变,天道轻易地篡改着整个至尊仙窍的环境。

就像方源眼前正在发生的这一幕:天空中出现光斑,证明光道道痕正在迅猛增长。而地面风声呼啸,沙尘滚滚,意味着风道道痕、土道道痕不断变多。

“长久以往下去,千愿树周围就会化为一片荒漠罢。”趋势是如此的明显,方源不禁冷笑起来。

他之前认为千愿树是种植在沙漠之中,现在来看,这层认知虽然不算错,却也偏颇了。

千愿树乃是人道上古荒植,汲取人意,凝聚果实。它最适合生长的地方是闹市区、大城池,人意越多越好。

然而,天道却不愿意千愿树成长起来,所以天道道痕演变,逐渐营造出了烈日高挂,沙尘漫天,与世隔绝的环境。

千愿树为了生存,不得不改变自身的习性,努力适应这样的艰苦环境。

但现在它被方源挪移到了至尊仙窍中,仙阵中自有沙漠环境,让它顺应习性,不至于因为环境突变适应不了而灭亡。

但仙阵之外却是水草丰茂,土地肥沃,光照柔和。

仙阵并不自闭,千愿树的人道气息被天道感应。而这些天道道痕还未被方源掌控,并且缺乏智能,又不知晓仙阵内的黄金,于是就自发地演变仙阵外围的环境,企图遏制千愿树的发展。

如此便让方源看出端倪来。

“千愿树三百年生长完全,六百年一开花,九百年一结果。结出的果实,不多不少,正好一千枚。”

“这个认知是世间公认。但现在看来,却非如此。”

“千愿树受到天道压制,真正的潜力很大!”

“我今后用人意不断地浇灌,同时渐渐改善它的生存环境,它必定能大大增产!”

这当然是今后的计划,方源此刻来到千愿树这里,却是为了另外一件事情。

他心念一动,煮运锅便从头顶显露形态。

七转煮运锅,大如水盆,浑身金黄作色,笼罩一层洁白光晕,贵不可言。

锅边厚实,有大拇指头的厚度,锅外是八条雕龙,龙尾齐聚锅底,相互缠绕,又延伸下去,形成八爪的支架。锅口大张,里面的正是方源的气运,却是一片灰败之色。

虽然煮运锅只是七转,层次有点低,不能够将方源的全部气运都容纳到一口锅中来。但是管中窥豹可见一斑。

方源的磅礴运势几乎都消耗在了宿命大阵之中,战后他的运势十分低迷。这是天道、气潮之外的第三大隐忧。

命是定数,运是变数。

运势不佳,代表着他有着不好的变化趋势。

不过如今方源实力雄厚,有着亚仙尊级的战力,就算变化趋势再怎么糟糕,他也有着抵抗能力。

这方面本体的问题是不大的,叫他担忧的是分身的运势。

煮运锅外,是各大分身的运势。

宙道分身的气运,仍旧是一片光阴长河,但是河水却再不潺潺流淌,水流混乱,并且流淌的前方突兀断裂,令人揪心。

战部渡的气运,之前是一只雄鹰,展翅高飞,如今雄鹰落水,周围全是水流般的运势,正在为难他。

李小白的气运,仿佛鲜花绽放。花瓣红艳,燃烧着一层赤火。在赤火中,花瓣正在娇艳盛开。看似上佳,实则隐忧重大。稍不留意就是玩火自焚,身死道消。

吴帅的气运,乃是巨龙盘踞,龙爪中有三只各扣着一枚龙珠。一只龙珠中有着龙宫缩影,一只龙珠里藏着密密麻麻的蚂蚁,第三只龙珠里则是帝藏生的身影。吴帅的龙运最为强势,但周围却有刀兵之气,从四面八方向他逼来。

方源主运欠佳,分身的运势也自然不妙。

分身的情况,方源早已察觉,之前不是没有动手尝试,皆是失败。

天道道痕束缚着他,每当他运用杀招,便会横插一手,令方源催动杀招失败。即便偶尔成功,威能也大大降低。

“但这一次,应当不会了。”方源眼中精芒一闪即逝。

方源一心两用,一边调动千愿树周围的大阵,一边催动煮运锅。

千愿树不断摇曳枝干,大阵迸射出璀璨的虹光,尽数投入到煮运锅中去。

煮运锅开始改变各个分身的气运。

周围天地忽然狂风大作,响起晴天霹雳。一道道天道道痕,浮现而出。有的笼罩四野,有的则浮现在方源的身躯之上。

天之道,损有余而补不足。

人之道,损不足而补有余!

在人道仙阵的辅助下,方源硬是将各大分身的气运都做了一定程度上的改善。

宙道分身的气运之河,变得稳定下来,不在混乱。

战部渡的气运之鹰,在水中开始褪下毛羽,好像要转变成鱼,有了适应之像。

李小白的气运之花,花瓣在火焰的灼烧下,显现出琉璃之色,不再有玩火自焚的危险了。

而吴帅的气运之龙,仍旧盘踞着,气运本体不见变化。而周围的刀兵之气,却是消弭了很多,仿佛浓雾化为了薄雾。这意味着他仍旧会要动手,但这一次的刀兵之争的规模却远比之前要小很多。

“分身的问题算是解决了。真正的麻烦仍旧是天道道痕。”方源如今是越发的感受到了红莲的用意之深。

ps:有一个紧急通知。我也是醉了,这个事情我是今天傍晚码字的时候才知道的。感谢朋友们的及时通知。

蛊界群中有人行骗,请诸位书友注意。本人在此郑重声明:实体书众筹的活动还未展开,任何以此为理由,企图让书友们汇款的行为,都是诈骗!

实体书众筹的活动,会在今后展开,届时会有专门的通知(就像此次通知)。以简装版的实体书价格,就可购买精装版(含蛊真人亲笔签名)的诈骗理由,请诸位书友警惕。

哦,晚上9点还有第二更。

------------

\end{this_body}

