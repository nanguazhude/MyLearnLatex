\newsection{尊者无敌之秘}    %第一百二十节:尊者无敌之秘

\begin{this_body}

%1
“元境……还在疯魔窟的最底层?”一时间,方源的脑海中思绪泛滥。

%2
人祖传中的确记载了元境,但其描述是模糊的,模棱两可的,就看阅读的人如何理解。

%3
“就算元境是有这样的威能,那它是否就藏在疯魔窟的最底层呢?”方源不可能直接就听信陆畏因的一面之词。

%4
但的确是有这种可能性。

%5
方源也去过疯魔窟,当时就从种种痕迹上判断,乐土不仅去过,而且还开辟了道场。

%6
方源无法进入最底层,但乐土乃是仙尊,能力不好揣度。

%7
“如果这一切都是真的。那么乐土的目的,难道就是疯魔窟?”

%8
方源心中掀起波澜,面容上仍旧不动声色。

%9
陆畏因察言观色,越发心中感叹:“这古月方源不愧是当世第一魔头,难怪成为天意、尊者们的争夺对象。要让他相信,单凭刚刚这些完全不够!”

%10
陆畏因只好继续开口道:“疯魔窟的确是在最底层,元境十分特殊,它既是起始也是终极,蕴藏着整个天地的无上奥妙。无极魔尊正是以它、九转衍化仙蛊为核心,方才能推演天地世界,产生新的天道道痕啊。”

%11
“从某种方面而言,疯魔窟的重要性,比天庭还更重要!”

%12
陆畏因徐徐开言,吐露出更多的秘辛。

%13
一百万前上古时代,人族的第一位魔尊——无极魔尊攻上天庭,以摧毁天庭进行威胁,逼迫星宿意志进行棋盘赌局。

%14
棋局暗通疯魔窟,双尊对弈的过程中,碰撞思想的火花,不断推陈出新,开辟天道新成果。

%15
这些成果都顺着棋盘,输入到疯魔窟中,助推无极的推衍。

%16
星宿虚影故作不知,实际上早有准备,将计就计。

%17
其后历代尊者,诸如狂蛮魔尊、元莲仙尊、巨阳仙尊都参观过一缺抱憾亭,知晓双尊对决,更明白疯魔窟的重要意义。

%18
宿命大战中,双尊对弈的身影在一缺抱憾亭中消散,但这只是双尊赌斗告一段落而已。真正决定胜负的最终场地,还在疯魔窟!

%19
历代尊者知晓这一点,也纷纷掺和一手。比如元莲仙尊,在疯魔窟的第八层虚空中,创建了青莲道场。狂蛮魔尊打造了蛮荒大世界,乐土仙尊大人也建立了黄土大世界。

%20
如此一来,情势就更加复杂。

%21
方源心头一震:“这恰是我长久以来的疑惑之一。有一部分尊者为何要在疯魔窟的第八层中,大费周章地创建道场,维护一个大世界呢?难道说,他们也想对元境下手?”

%22
“尊者都已经成就尊者,元境对他们而言,并非必须之物。等等,或许并非如此!”方源脑海中仿佛有一道电光一闪即逝,划破重重迷雾,让他见到了真相一角。

%23
“哈哈。”陆畏因大笑两声,“方源仙友不愧是纵横天下,尊者不论多少,都算计不倒的人物。你已经猜到一部分缘由了。”

%24
“没错。”陆畏因继续道,“宿命蛊已毁,尊者重生就成了可能。然而越强的蛊仙重生,越是麻烦,付出的代价越大。稍有不慎,即便复活,也没有之前的实力。就比如气绝魔仙,因为受到方源和武庸等人的干扰,实力下滑,并非他的巅峰战力。”

%25
“尊者也是如此。他们重生之后,虽然再无尊者关卡,可以一路上升回到九转修为。然而,尊者之间亦有激烈的竞争。尊者留下重生的手段,但是毕竟是有限的。所以尊者们无不想确保自己顺利重生,最好重生的那一刻,便有昔日的巅峰战力。”

%26
方源连连点头。

%27
他完全能够理解尊者们的想法。

%28
宿命蛊在的时候,是不能重生复活的。但尊者们几乎都留下了手段,宿命蛊一毁,这些手段一旦发动,就能令尊者重生。

%29
然而,尊者之间立场不同,阵营不同,彼此也互相谋算。

%30
谁不想成为天地间最强的人?

%31
谁会希望除了自己之外,还有其他尊者?

%32
尊者们相互忌惮,相互拆场子,虽然死了,但他们都一直在暗中交锋。

%33
这种交锋,因为经历的时间太长,一般而言,越是后面出世的尊者越有利。比如乐土仙尊就是如此。

%34
但也不绝对。

%35
比如元莲仙尊还在幽魂魔尊之前,但他布置下来的神帝城和青仇,就坑了幽魂一把!

%36
又比如乐土仙尊进入红莲石岛中,取走悔蛊,就十分容易。

%37
但总体而言,还是这样的大趋势。

%38
越是后面的尊者,在这种交锋上越有优势。

%39
因为整个蛊修世界,都是在不断进步的。远古时代,元始仙尊那会儿的蛊修流派不过四五道而言,气道称雄。但现在呢?流派五花八门,主修流派至少有二十道!

%40
就拿神帝城举例,神帝城是元莲造物,在当时是木道、画道的巅峰水准,还能推演人道。但它在运道方面,就比较薄弱。因为巨阳仙尊是元莲之后的人物。

%41
甚至,严格来讲,神帝城还是不是木道流派的巅峰,已经说不准了。

%42
元莲仙尊的时代,已经过去了三十万年以上。这个漫长的时期,木道也不断发展,涌现了无数的人才,他们各有成果,推动木道流派,有了巨大的提升。

%43
唯有画道,应当还是巅峰。

%44
为什么呢?

%45
画道被元莲仙尊藏着,一直都没有流传出来啊。

%46
所以,画道这一块儿仍旧是世间顶级。

%47
这就是为什么,有一部分的尊者一直扣着自己的流派真传,没有广泛传播。

%48
比如巨阳仙尊的运道真传。

%49
陆畏因继续爆出惊人的消息:“方源仙友,你可知宿命大战前夕,龙公就秘密将生死门,挪移到了疯魔窟中去了。”

%50
方源目光闪烁:“是否天庭也知道生死门中,藏着幽魂本体,所以提前搬离,不然幽魂有可乘之机?难怪魔尊幽魂控制了紫薇仙子之后,挖掘天庭库藏,始终都没有找到生死门。”

%51
陆畏因又喝了一口茶:“龙公不仅是将生死门搬走,而且还将天庭仙墓的一部分,也搬到了疯魔窟中!”

%52
方源心头微动:“如此说来,我摧毁的仙墓只是天庭中的一部分,还有另一部分留在了疯魔窟中。这是天庭提前做好的退路吗?倒是一个正道超级势力的求稳风格。毕竟天庭仙蛊有限,苏醒的蛊仙难以保证生前的巅峰战力,并且星宿意志也需要一部分沉眠的蛊仙提供。”

%53
“我正要说到星宿意志。”陆畏因目光如炬,紧紧盯着方源,“方源仙友可知,星宿意志已经很长一段时间没有显形,指点天庭该如何行事了。”

%54
方源终于动容,皆因他听到了话音之外的暗意。

%55
方源也盯着陆畏因:“你的意思是,星宿已经依靠这股意志,重生复活了么?”

%56
陆畏因点头:“我虽然不能完全肯定,但当中的可能已是十之八九。星宿仙尊若是重生,必定潜藏在第八层中,图谋第九层的元境。一旦她进入元境之中,她将重回智道无上大宗师的境界,令天地元气为己用,成为真正完整的道主!”

%57
方源大皱眉头。

%58
他原本估料着,尊者没有重生。但从陆畏因的交谈中,他却得知恐怕已经有尊者重生了。

%59
并且这位重生的尊者,极其恐怖,乃是尊者当中唯一的智道流派,也是唯一的女子。

%60
这位女子之强,即便身死,也能谋算后世三位魔尊。即便此中内幕,和天庭表面宣传出去的有很大差距,但事实是天庭依靠星宿的谋算和布局,的确是在三大魔尊的冲击下,维护住了天庭。

%61
这份功绩无法抹杀,令方源每每想来,都深感可怕!

%62
而星宿仙尊为什么不再是智道无上大宗师,这一点方源心知肚明。

%63
星宿仙尊消逝的时候,是三百万年前的远古时代。

%64
时间太长了!

%65
智道流派发展到了今天,早已经有了翻天覆地的进步和提升。

%66
准无上大宗师,是说蛊仙对此流派的认知已经达到天地的极限,碰触到了那层天花板,几乎穷尽了此流派的所有奥秘。但还差一些边角,没有踏足。

%67
而无上大宗师境界,是不仅完全洞悉了流派的全部奥妙,而且还能推陈出新。站在这种层次上,每一次蛊仙的进步,都是前无古人的创举,是对整个流派的上限的提高,乃至对天地都是或大或小的晋升。

%68
星宿仙尊曾经的智道无上大宗师境界,放在今天,早已经不是了。

%69
即便她的意志和天意纠缠一块,时刻学习天庭中的智道成果,也是如此。

%70
且不说星宿意志和分身不能等同,就说天庭的智道成果,只是整个智道流派的一部分而已。

%71
只有当星宿仙尊进入元境,方才能令她的智道境界达到无上大宗师,成为道主。

%72
“什么是道主?”方源又从陆畏因口中,听闻了新的概念。

%73
他来到菇人乐土这一趟,真的是太超值了。成尊的条件他已经全部获悉,尊者的图谋他也真正知晓了大概,而道主这个词,很可能关乎尊者无敌的秘密!

%74
果然,下一刻陆畏因笑道:“历代尊者为何能无敌天下?尊者和亚尊之间的差距,就在道主之上。尊者的手段为何长存至今,很难消散?也在于道主这个关键。”

%75
不待方源催问,陆畏因便紧接着说道:“所谓道主,便是一道之主。当蛊仙成为尊者,拥有无上大宗师境界,领袖一道,自身的进步便是流派的进步,便是天地的进步,那么他就是道主了。”

%76
“身为道主,蛊尊能够感知天地自然中自身主修流派的全部道痕,并且加以炼化!炼化之后的道痕,蛊尊能够随意操纵。”

%77
方源不禁微微动容:“加以炼化?还能随意操纵?”

%78
陆畏因继续道:“每位尊者成尊之后,都有一个游历天下的过程。比如元莲仙尊,又比如乐土仙尊大人。事实上,在这个过程中,他们广泛炼化天地间的道痕,将整个天地都打造成了他们的战场杀招。”

%79
这就是蛊尊无敌之秘!

\end{this_body}

