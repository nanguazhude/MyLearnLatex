\newsection{八转狗屎运}    %第一百一十三节:八转狗屎运

\begin{this_body}

%1
“我池家此次遭遇困难,巴家可没有什么表示。甚至我要上位的关键时刻,巴家还在暗处干扰,想要扶持池须芝。可恨这池须芝自从娶了巴家女仙,野心一天比一天大。”池谤回忆起来,语气含恨。

%2
说着,他一拍桌子:“我意已决,答应武家的结盟,至于巴家……我会书信一封,给予解释,希望他们能够理解。”

%3
池谤答应结盟的消息,传到武庸这里。

%4
武庸哈哈大笑:“池谤小儿,不足为虑!”

%5
显然,武庸的计划开展得相当顺利。

%6
池曲由阵亡,池家内虚。而巴家改朝换代,原来的太上大长老巴十八,因为被方源俘虏,取走了仙窍而退位重修。巴德上位,造成巴家动荡。

%7
武庸敏锐地察觉到了这两家的破绽。叱咤传承这档子事,是上天将机会送到了武庸手中,武庸岂会不一把抓住?

%8
“大喜,大喜啊。”这时,武八重忽在门前求见。

%9
“进来吧。”武庸开口,“何喜之有?”

%10
他很少见到稳重的武八重有如此喜色。

%11
武八重便道:“武碑在探索悟空天坑时,意外发现了大量的乾清一气!”

%12
武庸微微一愣,旋即叫好:“这的确是个好消息。”

%13
武独秀临死之前,有三大憾事。

%14
首件是凶峡七鬼,如今武庸已经尽数铲除。其次则是没有炼成八面威风仙蛊。

%15
这八面威风仙蛊的仙材,武家已经筹备多年,只缺一份主材,便是乾清一气。

%16
武碑这次意外收获的乾清一气,足够武家开炼八面威风蛊了。

%17
“武碑的确是我家的福将啊。该赏!”武庸迅速思考一番,旋即开口。

%18
“八面威风蛊乃是八转仙蛊,若要开炼,是一件大事,必须得郑重,做多方筹谋。武家上下都要拼尽全力炼蛊。”武庸说到这里,叹息一声,“炼制八转仙蛊,难、难、难!就算是方源,有琅琊福地之助,也难以炼制八转啊。兹事体大,我们虽然筹备全了仙材,也要慎重考虑。”

%19
武八重心中的激动渐渐平复下来。

%20
“大人所言甚是,是属下莽撞了。”

%21
武庸始终保持冷静和理智,武八重对他越感佩服。

%22
最后,武庸对武八重道:“炼制八面威风蛊的事情,先放在一边。池谤已经答应和我家结盟,现在我需要做的是向巴家施压,让他们自顾不暇,难以干扰此事。”

%23
武庸未雨绸缪,思虑周详,已经开始提前消灭障碍。

%24
又是一番计划安排。

%25
武八重退出书房,在心中暗暗赞叹:“武庸大人如此雄韬武略,我武家何愁大事不成?”

%26
数天后,至尊仙窍。

%27
方源的宙道分身坐镇四元方悔血炼池。

%28
“起!”宙道分身忽然大喝,八转仙元疯狂灌输。

%29
轰隆一声。

%30
四四方方的水池中,鲜红如血的池水忽然逆冲向天,宛若倒挂的瀑布。

%31
整座四元方悔血炼池都发出嗡嗡的声音,被催动到了极致。

%32
血水瀑布冲到半空中,在最顶端形成一道漩涡。

%33
漩涡猛烈自转,透射出璀璨光辉。

%34
漩涡汲取了大量的血水之后,渐渐平缓下来。

%35
最终,漩涡中缓缓飞出一只仙蛊来。

%36
漩涡消散,逆冲的血水瀑布也砸落到四方池中,血水只剩下池底薄薄的一片。

%37
血光消退,宛若倦鸟归林,又回到池中小亭的亭顶。那里正是血本仙蛊的位置。

%38
方源宙道分身调息片刻,念头一动,悬浮半空中的仙蛊就降落,飞到了他的手掌中。

%39
这只仙蛊乃是蜣螂模样,俗称屎壳郎。

%40
它全体都是黄金作色,身躯分为头、胸腹、尾三段。头部如半月铲,两侧有船桨一般的触角。胸腹部有横行的隆脊,尾部椭圆。

%41
在胸腹部有两对足肢,尾部有一对。每一个触脚都十分粗壮,触脚的最末端还有黑色的坚硬毛刺。

%42
正是狗屎运仙蛊,经过方源的一番炼制之后,已然上升到了八转层次。

%43
狗屎运仙蛊乃是煮运锅的核心仙蛊。

%44
巨阳仙尊开创运道,留下三大真传:己运、众生运、天地运。煮运锅便是己运真传中精华凝聚的巅峰造物。

%45
宿命大战之前,方源几乎将所有的运道仙蛊,都投资到了煮运锅仙蛊屋上。

%46
这一次他将煮运锅彻底拆掉,升炼其中的运道仙蛊。

%47
狗屎运仙蛊是首要选择,方源只是第一次尝试升炼,就成功了。

%48
四元方悔血炼池即便炼制八转仙蛊,成功率也在五六成,这个概率极其恐怖。宣扬出去的话,必定会震动整个蛊仙界,乃至于可能引起长生天、天庭的疯狂进攻。

%49
有了八转狗屎运,煮运锅的转数就已经提升到了八转层次。

%50
换做寻常蛊仙,必然已经见好就收。但方源却远远不满足,接下来他还要升炼气运仙蛊、连运仙蛊、运筹仙蛊、察运仙蛊等。

%51
宙道分身继续催动四元方悔血炼池。

%52
随着八转仙元迅速消耗,原本接近干涸的池水,开始缓缓上涨起来。

%53
这池水乃是炼海雏形中的炼水,结合光阴支流的河水,通过水炼仙蛊为核心的炼道手段,形成的奇妙造物。

%54
宙道分身又打开仙蛊屋中的库藏暗门,里面的运道仙材一份份投放到了水池当中。

%55
其中就有宠心。

%56
方源已经将白兔姑娘、妙音仙子二人,接应回来。

%57
用的是定仙游杀招。

%58
现在的方源,因为体内道痕太过庞大,曾经的翠流珠杀招已经不合用了。

%59
但是没有关系,方源开创复合杀招硕果累累,已经改良出了天地游杀招!

%60
九千多道天道道痕辅助,天地游杀招载着方源直接传送到藏地附近。

%61
在浩荡的地脉中心,方源悠然地顶着地脉强压,将整个藏地收入仙窍之中。

%62
白凝冰却已提前溜走,方源也不管她。眼下在她身上浪费时间,显然毫无必要。

%63
刚刚升炼狗屎运仙蛊,九转运道仙材宠心缩小了一点。这一次池水冲刷,又将它表面上的一层冲刷,融入进了池水当中。

%64
如此利用仙材,已经是妙到毫巅的手段。

%65
血光再次从血本仙蛊上弥漫开来,覆盖整个水池。

%66
池水掀起潋滟,里面的仙材彻底消融,和血红的池水混为一团。

%67
酝酿片刻,一切就已经就绪。四元方悔血炼池处理仙材的效率,是非常惊人的。

%68
宙道分身手掌轻轻一抛,将察运仙蛊抛入池水。

%69
第二轮的升炼开始了!

%70
太古黑天,镇运天宫。

%71
黑楼兰的福地中,一场灾劫刚刚消散。

%72
黑楼兰面露喜色,这一场灾劫的收益相当可观。单凭她自己是不能撑过去的,但有巨阳仙尊指点,冰塞川亲自护持,黑楼兰轻轻松松就渡过灾劫。

%73
“我来到镇运天宫的时间并不长,但这些天来,我已经渡过数次灾劫,实力不断暴涨,简直是一日千里。”

%74
“我之前的预感无差,看来长生天是真的想着重栽培我!”

%75
黑楼兰观察自家仙窍。

%76
她初次拜见过巨阳仙僵之后,就被冰塞川施展了宙道手段,自家仙窍光阴流速提高到惊人程度。

%77
与此同时,长生天又支援黑楼兰海量的力道仙材,根本不存在拔苗助长的隐患。

%78
力道仙材如此充沛,自然引发出力道灾劫。

%79
而长生天掌握着天底下最强的运道手段,对于渡劫帮助极大,黑楼兰直接躺赢。

%80
背靠天底下第二大的超级势力,黑楼兰又被巨阳仙僵看重,她深深感受到依靠势力的优势。

%81
根本不需要操心养蛊、炼蛊的事情,也不需要留心收集蛊方、杀招等等,自有人直接送上来。

%82
甚至仙窍如何经营,如何布置资源点,都有专门的智道蛊仙为她谋划。

%83
黑楼兰只需要在用蛊方面多多琢磨,锻炼自己的杀招。

%84
她虽是仙二代,但命途多舛,经历坎坷,遭遇方源之后,更是颠沛流离,多少次面临生死险境。所以,得到栽培之后,她除了万分感激之外,更极其珍惜这样的机会!

%85
她深深的明白,这样的机会是多么的难得和可贵。

%86
她珍惜每一分每一秒的时间,排除必要的休息,每一天她都在勤修苦练。

%87
每隔一段时间,巨阳仙僵就会召见她,加以考量和指点。

%88
这一次召见,巨阳仙僵都对黑楼兰的自虐式修行有些看不过去,宽慰道:“黑楼兰,我的子孙后代。不要太过刻苦了,世间万事都讲究一张一弛,平衡有度。蛊仙修行也是如此。你长此以往,紧绷心弦,反而不美。”

%89
黑楼兰面对巨阳仙僵,却仍旧坚持自己的想法:“先祖大人,我这点刻苦算得了什么?在过去的一段时间里,我在方源身边修行,亲眼见证他的奋发努力。世人皆以为他得到尊者资助,方才有如今成就。但在我看来,即便没有任何人资助他,仅凭他这份品性,也足以能够脱颖而出,为祸一方。”

%90
巨阳仙僵微微一愣,微笑点头,对黑楼兰越加欣赏:“看来你是将方源当做了你的敌人,心中的目标,赶超的对象。很好,有这份心气劲,这才不愧是我巨阳的后代!”

%91
“但是,你要赶超方源,难度之高,宛若凡人登天呐。也不瞒你,在最近这段时间,方源的实力节节暴涨,程度极其惊人。”

%92
黑楼兰疑惑地看向巨阳仙僵。

%93
巨阳仙僵便解释道:“眼下察运仙蛊虽不在我这里,但我这处镇宇天宫却蕴含类似察运的杀招,并且因为盘踞太久,五域两天的运势都在我眼底。”

%94
“最近这段时间,我就屡屡瞧见,在那东海方向,夏家的大本营方位上,运势不断上涨,层层勃发,态势之猛烈,即便是我也生平罕见。”

%95
“魔尊幽魂、天庭以及我方劫运坛,没有斩杀得了方源,终于让他得到了修养之机。他得到数位尊者资助,又有上佳品性,屡次重生经历丰富。给与他时间越长,他就成长得越高。你要追上他,还得要拼死努力才有一丝希望。”

%96
黑楼兰面露凝重之色,立即语气坚定地问道:“还请先祖赐教,究竟是什么方法?”

%97
巨阳仙尊却微微摇头:“时机未到,你且继续苦修罢。”

\end{this_body}

