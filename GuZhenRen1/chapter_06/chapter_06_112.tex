\newsection{先对运道下手}    %第一百一十二节:先对运道下手

\begin{this_body}

%1
两天中的这些洞天,普遍拥有海量道痕。

%2
这些道痕的主要来源并非渡劫,而是历代蛊仙们的奉献。

%3
比如安魂洞天的道痕,就多达七百多万!这么多的魂道道痕,导致洞天中拥有大量的超级资源点。

%4
最初的安魂洞天,只有十万多的魂道道痕。它吞并了一大块九天碎片后,从此无灾无劫。

%5
受到天意扶持照顾,安魂洞天中产出蛊仙,一代又一代。

%6
这些蛊仙都是人族,虽然可以随意选择修行的流派,但绝大多数都选择魂道。

%7
因为洞天中的魂道资源,种类繁多,储备丰富,可谓得天独厚。即便蛊仙选择其他流派,数量上也非常稀少,他们在获取其他流派的修行资源,要远比魂道困难得多。

%8
事实上,太古两天也不是寻常的六转、七转蛊仙,能够进出探索的。

%9
通常而言,只有很少的七转强者,以及普遍的八转,才能在太古两天中探索,冒着风险,采集到一定量的野外资源。

%10
而当这些魂道蛊仙死后,他们的仙窍几乎都融汇合并到安魂洞天之中。

%11
这些蛊仙虽然没有至尊仙窍,合并仙窍时,会产生颇多内耗损失。但因为主流道痕始终是魂道道痕,导致合并的收益多数时期大于损失。

%12
沧海桑田,历代累积下来,便有了今天七百多万的惊人积累!

%13
四元方悔血炼池是方源修行以来,从五百年前世到现在,整个生涯中最为庞大的一笔投资。他将整个炼道都投入进去,还累及宙道、律道、智道、血道等等。

%14
拥有四元方悔血炼池,方源炼道方面的实力暴涨,成为当今天下第一人。再算上他拥有准无上的炼道境界,就算是数遍历史长河,炼道之中方源也绝对是三甲之列。

%15
但宙道、律道等其他方面,因为也投入了不少仙蛊,相应的底蕴随之下滑。

%16
这段时间休养生息,安土重山堡至今都没有修复,仙元储备已经脱离了警戒线,仙蛊付出了不少,但因为各大主流道痕暴涨,方源的综合战力不降反升!

%17
方源的日子好过了。

%18
想想以前,方源几乎是朝不保夕,被尊者算计强推前行,被天庭四处撵着追杀。而现在,魔尊幽魂阵亡,天庭、长生天也在休养生息,让方源的外部环境再无高压,使得他能够能更加从容的制定修行方略。

%19
站在长远来看,打造出四元方悔血炼池绝对是一本万利!

%20
“接下来就是升炼运道仙蛊,到达八转。”方源望着四元方悔血炼池踌躇满志,早有了定计。

%21
宿命蛊还在时,天地间乃是九命一运。现在宿命蛊毁掉,运道的影响无疑暴涨!

%22
尤其五域乱战乃是大势所趋,越是混乱的情况下,运道越能发挥奇效。

%23
“我现在的仙材储备多得史无前例,已有升炼八转运道仙蛊的实力。但打造完整的炼海还早得很,有心无力。嗯?”

%24
方源正琢磨着,忽然从宝黄天那方面,得到了黑菟的情报。

%25
“她们还活着,仍旧困于地脉。有些意思,她们躲避在藏地之中。好得很,藏地乃是十地之一,每一座藏地都蕴藏丰富仙材,尤其是有一份九转运道仙材宠心,正合我当下之用!”

%26
方源不禁心生喜悦之情。

%27
九转运道仙材,自然能够炼制九转运道仙蛊。

%28
但单单一份藏地宠心,还远远不够。

%29
方源手中的运道仙材,多是七转,少部分八转。显然是没有底蕴去冲刺九转运道仙蛊的。

%30
但用九转仙材,炼制八转仙蛊,却是能略微提升炼蛊的成功率。

%31
有了宠心,方源可谓是如虎添翼!

%32
南疆,池家大本营。

%33
砰。

%34
现任的池家太上大长老,狠狠一拍桌子,咬牙切齿:“岂有此理!”

%35
这一位七转蛊仙看样子颇为年轻,他皮肤苍白,一对酒色过度的黑眼圈,此刻满脸怒气。

%36
正是池谤。

%37
他池曲由之子,池曲由生前力捧,池家内定的下一任太上大长老。池曲由虽然在宿命大战中战死,但是因为做了充分的布置,最终令池谤成功上位。

%38
池谤上位之后,却是感到和之前大不相同。

%39
他之前是在父亲的庇护之下,有池曲由为他遮风挡雨。而现在,他却要直面压力。

%40
这次,南疆超级势力羊家就给他出了一道难题。

%41
事情的起因矛盾,是一道音道传承。这份音道传承,名为叱咤,历史悠久,声名远播,从很久远的时代持续传承下来的。

%42
上一代的叱咤传人,只有七转修为,但上上代却是八转蛊仙。

%43
也就是说,这份叱咤音道传可以让人修行到八转,蕴含巨大利益。

%44
更关键的是,这份叱咤传承就位于池家领土之中,但又十分靠近羊家。

%45
池曲由战死,池家损失了唯一的八转蛊仙,池家内部空虚。羊家果断出手,侵犯池家边疆,将传承之地围拢,目前在积极探索那一个音道福地。

%46
羊家如此行径,池家必须有所表示。被侵犯了领地,这关乎池家颜面。若是应对不当,便会令池家颜面扫地,难堪至极。

%47
但是,池家究竟怎么做,才能护住脸面呢?池家能不能击退羊家的入侵人马?

%48
能不能打?怎么打?如果不能打,又该怎么办?

%49
这是池谤面对的困惑。

%50
“这件事情,你们三位怎么看?”池谤询问眼前三位七转蛊仙。

%51
池大霹操着大嗓门,立即道:“当然是反攻过去,把这群羊家的狗崽子们赶尽杀绝!”

%52
“小声点,小声点。”池谤连忙摆摆手,又看向一旁的池规。

%53
池规皱着眉头道:“太上大长老,此事还需斟酌。”

%54
“怕什么!”池大霹瞪起双眼,“羊家没有八转蛊仙,此次只来了两位七转而已。若这样我们池家都不反击,那会被整个南疆正道耻笑的。”

%55
池规叹息一声:“别看只是区区两人,但这两人确实出了名的强者。一位羊阴光,擅长冻魄玄光,难以防备。另一位羊文魁则是智道、信道兼修,乃是羊家的智囊人物。”

%56
“依凭我们的侦查,羊家的其余蛊仙都镇守在羊家领地,但也难保羊家伪装,暗中抽调。即便羊家没有伪装,我们也得考虑到羊家的仙蛊屋。羊家既然入侵我方领地,极可能已经出动了仙蛊屋,只是藏于羊家两仙的仙窍中,按捺不发而已。”

%57
“嗯。”池谤点点头,最后看向池伤,“你如何看待此事?”

%58
池伤一身白袍,面容英俊,但目光却有些痴傻。

%59
“池伤?池伤?”池谤轻声唤道。

%60
但池伤却是听不到一样,双眼呆愣地看着前方。

%61
屋内三仙对望一眼,齐齐一声叹息。

%62
很显然,池伤又走神了。他乃是池家最有望成为阵道大宗师的天才,号称阵痴,经常性走神。为了思索一个阵道难题,甚至忘了吃饭。若非家族关照,他很可能因为忘记吃饭,沉浸在思考中而饿死。

%63
池伤根本不理睬池谤,池谤心中恼怒,但表面上还得和颜悦色地对他。

%64
参加宿命大战之前,池曲由就特意将池谤唤进书房,秘密嘱托,关照过他:池伤便是父亲留给池谤的最佳助力。

%65
事实上,书房中的三位池家七转蛊仙,便是池谤的赖以依靠的心腹了。

%66
但这些心腹却无法帮助池谤解决难题。

%67
池谤问了一圈,结果还是得他来拿主意。

%68
正在这时,忽有一只信道蛊虫顺着宝黄天传达过来。

%69
池谤神念探入,迅速阅览之后,双眼放光,流露出喜悦之色:“好极了,武家武庸盟主听闻了此事,愿意与我们池家合作。驱逐羊家之后,叱咤传承家一概不要,武家还会派遣仙蛊屋来助阵。但需要和我家达成更进一步的盟约,主要让我家为武家的资源点建设多处仙阵。”

%70
“还有这等好事?”池规大感意外。武庸开出的条件十分优渥,令他暗暗心动。

%71
“你们二位怎么看呢?”这一次,池谤直接舍了池伤,索性只询问池大霹和池规。

%72
池大霹双手拱拳:“一切全凭大长老做主。”

%73
池规则有些犹豫道:“我族这一次依靠武家击溃羊家,目前看是大占便宜,但武庸乃是当代枭雄,怎会做吃亏买卖呢?”

%74
池大霹道:“你忘记了吗?当初,武家势弱,羊家夺取武家毒瘟瀑布,虎视武家诈尸山。结果先被我家池曲由大人驾驭仙蛊屋逼迫,随后被武家夺回毒瘟瀑布。武家和羊家可是有仇呢。”

%75
池谤一拍额头:“对,池大霹说的对。说起来,武家似乎一直在拉拢我族。当初义天山梦境还在的时候,武家每次做仙缘生意,必定要把我池家捎上。武庸是枭雄,我父虽然故去,但我池家仍有拉拢价值。这一次,不妨就让武家替我们出血出力,哈哈,我池家坐享其成!”

%76
池规见池谤已经做出决定,只好压下心中的不安。

%77
他想了想,补充道:“此次,我们若和武家结盟,还得考虑巴家的情绪和看法。”

%78
“是啊,巴家……”池谤叹息一声,大感头疼。

%79
超级势力雄踞已久,关系盘根错节,真要处理起来,十分复杂,掣肘颇多。

%80
巴家一直想冲击南疆霸主的地位,想取代武家。巴家和武家之间,很不对付。

%81
而巴家和池家过往关系紧密,不只是普通的联盟,而是联姻!

\end{this_body}

