\newsection{幽魂的真正本体}    %第一百一十八节:幽魂的真正本体

\begin{this_body}

%1
“原来盗天魔尊出手,不只是天相杀招,还有这一处,只是我一直没有察觉罢了。”方源口中喃喃。

%2
他忽然心头又一动,脱口问道:“青仇是否也没死?”

%3
方源追杀战末期,神帝城急吼吼地拔地飞来参战,真的只是为了镇压气绝魔仙和长生天吗?

%4
方源想到的是青仇的那几块碎尸。

%5
神帝城的主要目的,究竟是什么?

%6
这很值得商榷!

%7
陆畏因脸色微肃,感叹道:“这一点,我也是回到这里休养了许久后,才琢磨出来的。神帝城参战的主要原因,究竟是什么?当时的情况,九转仇恨蛊的确是毫无气息,似乎彻底毁灭了。”

%8
“然而,青仇本质上是由杀招形成的生命。它究竟是死是活,我也无法肯定。元莲仙尊的这记手段,着实匪夷所思,令人叹为观止!”

%9
能够将杀招转变成太古传奇,这手段太玄妙,即便是方源、陆畏因也只能干瞪眼,无法揣度。

%10
陆畏因继续道:“青仇生死状况,我无从推算。但幽魂魔尊的状况,我却是一清二楚。”

%11
方源微微一笑:“愿闻其详。”

%12
陆畏因卖了这么久的关子,终于回到正题上来。

%13
他答道:“幽魂魔尊目前的状况,可以说是死,也可以说是活着。因为他真正的本体,一直都藏在生死门中。”

%14
方源点点头,感叹一声:“果然不出我的所料。”

%15
魔尊幽魂当时自爆,就已经让方源感到略微不妥。等到天道道痕尽数炼化后,方源反思,心中的疑虑就更多了。

%16
堂堂幽魂,就这样死了?

%17
要知道幽魂可是所有尊者当中,最为逆天的存在!

%18
方源五百年前世,僵盟健在,影宗潜伏,幽魂暗中掌控了天庭。龙公始终没有出现,显然是被幽魂暗害。他还设计杀害凤九歌,顺势侵占琅琊福地,他操纵大局,是五域乱战的幕后最大黑手。

%19
只要大梦仙尊没有诞生,他幽魂就是最大赢家。

%20
于是,又有了方源入侵狐仙福地,身边有宋钟协助,斩杀凤金煌的事件。

%21
凤金煌是谁?

%22
凤金煌童年便有梦翼仙蛊主动来投,是大梦仙尊种子!

%23
方源是谁?

%24
方源是天外之魔,宿命未毁的情况下,只有借助他的手才能有所改变。

%25
宋钟是谁?

%26
宋钟明面上是宋紫星的儿子,实际上是他的血胎所化。

%27
而宋紫星呢?

%28
他号称血龙,是影宗中人,是幽魂的分身!

%29
方源五百年前世,可谓内幕重重。

%30
那个时候,宿命蛊仍未毁灭,气绝魔仙无法重生,其他尊者也只能对幽魂干瞪眼。

%31
幽魂猖獗,最终逼得天意不得不选中方源,回到过去,来改变这一切!

%32
义天山大战,表面上看,方源成功阻止了幽魂。

%33
但真的是这样吗?

%34
幽魂得知方源重生回来,立即明白是天意所为。他若是按照原来计划发展,即便可以掌控天庭,操纵五域大战,也始终都会遭受天意阻截,一个方源不行,就会有第二个、第三个天外之魔重生回来,幽魂将始终无法真正立足宇宙之巅。

%35
所以,他改变想法,那一刻他才真正开始和红莲魔尊合作。

%36
最终,他将至尊仙胎蛊送给了方源,打造出了完整的天外之魔!

%37
宿命大战之后,宿命蛊被方源摧毁了。

%38
幽魂想要夺回至尊仙体。

%39
他的谋算很完美。魂道境界不缺,拥有至尊仙体,他可以吞并安魂洞天等等,顷刻之间就能重新崛起!

%40
如此一来,他就领先所有的尊者,占据先手。

%41
先手优势之大,超越想象,一旦幽魂得逞,就是唯他独尊,宇宙天地都会被他彻底踩在脚下。

%42
但是他没有得逞。

%43
红莲魔尊算计了他,天道道痕束缚住方源的同时,也保护了方源。

%44
幽魂没有得到至尊仙体,这发展的差别就大了去了。

%45
方源追杀战,天庭、长生天同时出手,又有乐土仙尊的传人陆畏因参战,终将魔尊幽魂击杀。

%46
回顾幽魂的谋算,方源设身处地想:若他是幽魂,又该怎么去做?

%47
于是,他很快就发现了幽魂大计中的一个巨大破绽——

%48
太弄险了!

%49
梦境大战,龙公击败紫山真君,攻破福地,将魔尊幽魂生擒活捉,取走了生死门。

%50
魔尊幽魂生死操控在天庭手中,若是紫薇仙子没有贪图情报和线索,将魔尊幽魂直接斩杀,岂不糟糕?

%51
若是方源绝不会这样安排。

%52
幽魂呢?

%53
他纵然是杀性第一,也绝不痴傻愚蠢吧。

%54
所以,最大的一个可能,就是魔尊幽魂并非幽魂的真正本体!

%55
如果方源处在幽魂的位置上,他就会这样做——用一个分身伪装成本体,被天庭擒拿俘虏。

%56
一方面,能够降低甚至打消敌人戒备心。

%57
另一方面,分身也是一个陷阱,紫薇仙子最终就中招了。

%58
第三方面,分身掌握的情报和真传,远不如本体,纵然被俘虏,也不会过多资敌。

%59
最后一个方面,就算分身被杀了,也没有关系,本体还保全完好。

%60
对于幽魂的怀疑,方源还有两大证据。

%61
第一个证据,就是气绝魔仙。

%62
气绝魔仙是方源追杀战中的关键人物。

%63
方源和幽魂利用各自的真传,全力争取气绝魔仙成为己方盟友。

%64
最终气绝魔仙被方源争取了过来,这才令魔尊幽魂最终自爆毁灭。

%65
方源战后反思,总觉得自己的这场争夺战,胜利的太轻易了。和自己预料的,有些许差距。

%66
如果魔尊幽魂只是分魂,那就说得通了。

%67
幽魂料不到气绝魔仙的重生,因为宿命蛊被毁,一切未来的轨迹就都崩溃瓦解,一片混乱了。

%68
幽魂算计的是分魂被天庭俘虏,所以他只会精挑细选一些记忆,放入分魂当中。这些记忆既不会引发天庭的怀疑,也不会有过多的珍惜内容,比如价值极高的传承。

%69
所以,方源在这方面战胜了幽魂。

%70
第二个证据,是方源最近才得到的。

%71
便是他通过察运,揣摩自身气运,发现自身纯银光柱上端,笼罩着三团云层。其中一层,漆黑如墨。

%72
谁还能对他造成困扰?

%73
这团黑云,应当就暗示着幽魂!

%74
幽魂未死,岂会对方源善罢甘休?影宗、僵盟累积了多少年的资源,其中更艰难地筹集了十绝体,这才打造出了至尊仙胎蛊。幽魂会放弃?怎么可能!

%75
另一方面,气绝魔仙的重生,已经告诉天下人,尊者重生也绝非不可能的事情。

%76
幽魂真正的本体,恐怕早已经开始筹谋重生了。

%77
推而广之,其他的尊者难道就没有这个念想吗?

%78
以前是有宿命蛊阻拦,没法复活。现在宿命蛊都毁了,谁不想复活?!

%79
谁也不知道,究竟会是哪个尊者先真正彻底的复活。谁也不清楚,究竟会是什么时候复活。说不定,就在下一刻,就有尊者复活了。

%80
所以,对于方源而言,情势是严峻的。

%81
一旦尊者复活,他的亚仙尊战力的价值就大打折扣。天下第一魔头的地位,恐怕会成为标靶。

%82
追杀战后,陆畏因和他告别,告诉他要争分夺秒,也就是这个意思。

%83
方源也是这样做的。

%84
追杀战后,他是确确实实在争分夺秒,没有一刻放松。

%85
一等到他实力回升,可以应对潜在风险的时候,方源就来到了道德乐土。

%86
他来这里,当然不是为了见陆畏因的,也不是为了获得什么乐土真意,或者继承乐土真传。

%87
他来的目的只有一个——

%88
“告诉我,陆畏因,如何才能成尊?”

%89
宿命蛊被毁,整个天地的未来都混乱起来,什么事情都可能发生。

%90
不仅是整个蛊仙界,蛊师界也预感到了未来的动荡大趋势。

%91
方源在抗争,凡人蛊师们也在挣扎。

%92
悲风山脉。

%93
有四位蛊师正联袂同行,深入山脉进行探索。

%94
他们皆有四转修为,配合起来,更能抗衡5转蛊师。在中洲蛊师的世界中,他们是赫赫有名的魔道强者。

%95
他们分别是东淫陈淫道,西贱郁八光,南骚施暴,北荡樊春耀,合称中洲四大淫贼。

%96
“又有悲风刮来了。”郁八光首先示警。四人当中他最擅长侦查。

%97
“快,取出蛊屋来。”陈淫道立即开口,通常行事,四人都以他为首。

%98
蛊屋是由樊春耀保管的,他负责治疗和后勤。

%99
四位蛊师连忙躲进蛊屋。

%100
悲风在蛊屋外猛烈吹拂,刮得蛊屋咔咔作响。时不时的,樊春耀就要填补蛊虫,将那些承受悲风而亡的蛊虫更替。

%101
“亏老本了,亏老本啦!”樊春耀咋咋呼呼,心疼不已,“这一路走来,我们光是抵御悲风,就亏损了太多。悲风山脉中真有好东西吗?”

%102
陈淫道摇头苦笑:“我当然不能保证。但是只要还有一丝希望,我们就得这么干!前一段时间,悲风山脉上有蛊仙大战,说不定遗漏了什么好东西,能让我们升仙。”

%103
一提到升仙两个字,其余的三位蛊师的呼吸都有些粗重起来。

%104
“只要升了仙,谁再敢对我们四位仙人动手?我们就不会被追杀了。”

%105
“一旦我们成了仙人,那些高高在上的女蛊师,还不是任由我们拿捏?哈哈哈。”

%106
“瞧你这出息!升了仙,凡夫俗子有什么好的。我们能对仙子下手了啊!”

%107
几位蛊师相互对视,然后一起嘿嘿嘿的笑出声来。

%108
他们却不知道,天庭早已经将战场打扫干净,那些太古年兽尸躯等等都被收刮走了。

%109
当四人来到山峦崩塌的废墟上,搜寻了七天七夜,什么都没有捞着。

%110
就在四人气急败坏,准备离开的时候,他们终于有了发现。

%111
“这里有个人!你们快来。”

%112
“有个人有什么好奇怪的,这里埋了不少尸体。是生活在山中的村民。”

%113
“不,这个人还活着!而且他周围还有异象。”

%114
“什么?”

%115
其余三位蛊师一拥而上,就看到了双目紧闭,仍旧有一丝气息的王小二。

%116
而在王小二的身边,几朵野生的花草,不断盛开凋零,青枯二色在呼吸间迅速转换。

%117
------------

\end{this_body}

\newsectionindepend{目标完成了!}

\begin{this_body} %begin a body

%118
这是本月最后一天,吐出一口浊气,如释重负。

%119
之前斗胆提出的小目标,我终于是完成了!

%120
本月一天一更,稳定更新,从未缺席。

%121
很难。

%122
有好几次,我都差点坚持不住。生活是方方面面的,总会牵扯精力和时间。要想像学生时代那样,全力学习一般的去全力写书,这是不可能的事情。我是一个成年人,成年人的世界充斥着一种东西,叫做——不容易。

%123
另一方面,《人祖传》接下来的篇章,也着实耗费了我许多心血和脑汁。

%124
下个月,就有新的《人祖传》篇章和大家见面了。

%125
《人祖传》越到后期难度越大,几乎是倍增。因为很多意象和概念,只要前文出现过,基本上就用不了了。

%126
能够一边构思《人祖传》,一边完成一天一更的目标,我现在回顾,都觉得有些侥幸。

%127
好在坚持了下来,真的完成了。

%128
压力真的太大了!

%129
这个月一大半,我都是十二点之后睡的。有许多天,尤其是最近这个星期,基本上都是构思到夜晚三四点。

%130
六百五十万字的正文,要结尾,要填太多的坑,收束太多的线索。但这个月,已经填了不少,几个线索也在收束。

%131
更多的精力都耗费在《人祖传》。有时候心里很慌,因为《人祖传》的资料无从查起,很多时候学习是没有明显成效的。很多时候,打破障碍的,只是我的一个点子,或者十天,十几天前学习到的东西,或者是前一刻无意中看来的一句话。

%132
正文是可以把控的,《人祖传》是没有办法把控。有此带来的不安,导致心理压力巨大,很多时候提前躺在床上,想要睡觉,闭上眼,死活睡不着,又爬起来构思。

%133
想得身心疲惫,把自己丢在床上,这才能睡得着。

%134
成果是显而易见的。

%135
本月更新的质量,是我较为满意的。不管是行为结构,还是支线的描绘,都顾及到了。尤其是断章的水准,提升了不少。

%136
《人祖传》也构思出了新篇章。

%137
预计在十月初旬,会有一场大戏,发生在疯魔窟。前期的情节已经酝酿了大半。

%138
最后呼吁一下:

%139
希望广大的读者朋友们,尽量支持《蛊真人》的正版阅读,尽量在起点中文网阅读。

%140
一章内容,我至少构思大半个小时,写要几个小时。但读者朋友们阅读,可能十分钟不到,付费也就几毛钱,甚至很多起点币是赠送的。

%141
《蛊真人》这部作品,也不会更新太长时间,衷心希望大家能够尽量支持正版!

\end{this_body}

