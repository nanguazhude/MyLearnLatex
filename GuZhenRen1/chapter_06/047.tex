\newsection{至死后生}    %第四十七节:至死后生

\begin{this_body}

萧七星等人漫步在街道上。

大战刚过,原本繁华热闹的街道此刻却是一片萧瑟,行人罕见。即便是有,也多是伤员。

“房睇长枉为西漠正道!他屠戮无辜,造成多少的伤亡。就这样让他死了,真的是便宜了他。”陈大江恨声道。

应生机叹息,附和道:“是啊,这里曾经是多么生机勃勃的地方。”

“呵呵呵。你们真是好笑。”魏无伤笑道,“这里只是壁画世界,是元莲仙尊的杀招。这里的人也都不是真实的存在。死伤再多,再积累几年,又会恢复旧观的。”

“是啊。比起这些,我倒是更好奇这一次门派会给与我们什么样的奖励!”古霆道。

萧七星双眼骤亮:“我们以凡人之躯,屠仙成功,并且为神帝城铲除了内患。这样大的功劳,恐怕不只是门派,甚至天庭都会亲自赐下奖赏吧。哈哈哈!”

萧七星大笑,心中得意。

这一次平定豆神兵灾,铲除房睇长,就属他表现最佳,出力最大。

要论功行赏,他当属首位!

就在众人结伴而过的巷口中,两位乞丐盘坐在角落里,冷漠地打量着周遭的一切。

这些十大古派的蛊仙种子的言谈举止,自然也落在了这两位乞丐的眼中。

“这一次还真是多亏了你啊,沈兄。”其中一位乞丐秘密传音道。

乞丐模样的沈伤笑了笑:“我该称呼你为方源仙友呢,还是房睇长仙友?”

身旁的乞丐回应:“我乃是分身而已,严格来讲,如今甚至连分身也算不上,只是一份意志残存罢了。所以,不用称呼我为方源了。至于房睇长,他已经死了,自爆而亡。我现在无名无姓,只是一个烂乞丐而已。”

沈伤眼中精芒一闪:“仙友太过谦虚了。即便没有我,仙友也早就筹谋以死脱身。”

乞丐点头,并不否认:“我之前谋算,若是能够利用豆神兵侵占夺取壁画世界最好不过。但这种可能并不高。为虑胜先虑败,我自然要准备万全之策。自爆令我的肉身、魂魄都彻底消失,但这股意志却是暗存下来。”

顿了一顿,方源的意志继续道:“我的确要多谢你出手相助。若是单靠我自己的手段,被元莲意志发现的可能很高。但是由你为我掩护,那就完全不一样了。沈伤仙友你的人道造诣,日渐深厚了。”

沈伤嘿嘿一笑:“的确如此。自从宿命大战,我破解了尊者的人道杀招之后,人道上便有不小提升。如今来到这片壁画世界,整天扮做乞丐,细心观察,专心体悟,收获颇丰。当乞丐有当乞丐的妙处,乞丐是最底层的人,用仰望旁观的目光看人间百态,体会仁慈刻薄,感受饥饱冷暖。”

说到这里,沈伤话锋一转,问道:“接下来,仙友你何打算?是离开这里,还是继续停留?”

方源意志没有犹豫:“我无法离开这里。即便如今的我只剩下一股意志,也只是暂时脱离了元莲意志的关注,并没有掌握离开这里的途径。沈伤仙友你呢?”

沈伤也摇头:“我也找不出离开这里的方法。那些中洲十大古派的蛊仙种子,我已经暗中调查过。他们进出这里,是因为得到元莲意志的首肯。不过我相信,只要我继续潜伏,不断钻研,人道造诣持续积累,总会达到质变的时候。到那时,我就能找寻到离开这里的方法了。”

方源意志深深地看了一眼沈伤,他完全看得出来,沈伤在这里是乐在其中。

不过想想,也毫不奇怪。

元莲仙尊的这个手笔真的很大,专门开辟出了一个世界,来推演人道。

在这里,人道至高,其余流派都是辅助。人道发展最大,而其余流派都十分式微。

同时,这里遵循凡间的规则,仙凡之间的差距很小,更有利于人道的发展。

还有内外隔绝,外来者进入了这里,就和五域隔绝,连宝黄天都沟通不了。这就杜绝了情报流传、信息污染。从而打造出了这个与世隔绝的人道天堂。

沈伤专修人道,来到这里,简直是饕餮遇到了绝世珍馐,如鱼得水,甚至是深深眷恋。

方源意志此时寄身的,也是沈伤的一只乞丐蛊。

摆在他面前的只有两条路。一条是等待本体救援,另外一条则是跟着沈伤身边暂且修行,借助沈伤的力量脱困。

有关豆神兵灾的前后情报,都汇集到了秦鼎菱的手中。

“房睇长自爆而亡?”秦鼎菱眉头微蹙。

这个结果让秦鼎菱有些失望,她原本还指望着活捉了这个方源分身。再通过他,来推算出方源本体的位置来。

“没想到方源分身居然悍然自裁,不过这倒也符合方源的狠辣心性。”秦鼎菱没有多少怀疑。

“若是能活捉了他,那就好了。在壁画世界中,仙凡的差距并不大。可惜这些蛊仙种子还没有成长足够,把握不住这次良机。”秦鼎菱心中叹息。

萧七星等人认为屠仙功劳重大,但在秦鼎菱的心中,却是一个差评。

不过就算如此,秦鼎菱也决定下来,对这些蛊仙种子重重嘉赏。他们都是重点栽培的对象。

解决了奖励一事,秦鼎菱又将注意力投到东海之争上。

东海之争,即是人族和异人之争。

双方以气海老祖、吴帅为主导,在太古两天以及东海海域进行了轮番交手,战况越发激烈。

气海老祖主动献身,救下华文洞天之后,便顺势统合了两天中的数个洞天势力。

而吴帅虽败,但却不甘示弱。一方面号召两天团结到他的身边,另一方面则对那些保持观望的洞天势力实施突袭、入侵。

吴帅的方针十分鲜明直白。

在攻略五域之前,他必须将两天肃清,令大后方稳定。

一方面,这是因为两天洞天之间,距离较短,可以相互出兵。若是吴帅放任不管,将来大军在外,洞天之间只需要佯装进攻,就能令吴帅大军进退失据。

另一方面,吴帅麾下的诸多洞天,都结出了或大或小的气功果。吴帅每攻打下一个洞天,就用异人蛊仙们想到的方法,来尝试铲除气功果。吴帅一方十分需要这样的试验。

吴帅激进的策略下,两天中越来越多的人族势力,纷纷寻求气海老祖的庇护。

气海老祖一边防守洞天,一边整合势力。但总体而言,仍旧处于下风。

吴帅的两天联盟壮大的速度远比气海老祖更快。

他积极维护异人的利益,让越来越多的异族洞天投入他的麾下。在两天之中,异人洞天数量可比人族洞天多很多。

最后一批观望的异人势力,纷纷主动投靠吴帅。一方面是大势之下,身不由己。另一方面,不投靠的话,洞天中的气功果内患就没有解决的希望。

吴帅声势大涨,但气功果一事,仍旧困扰着他。两天联盟尝试了不少方法,但铲除气功果都宣告失败。

气海老祖那边钻研气功果的奥妙,也是进展缓慢。

吴帅和气海老祖两方交手,虽然摩擦频繁,但伤亡都不大。就是因为双方的重心,都不是剿灭对方,而是要清除气功果的重大内患。

“天底下最知晓气功果奥妙的,恐怕便是气绝魔仙了。可惜我此时实力不济,无法对付他。”方源心中惋惜。

天意的大麻烦已经被他解决了,但是天道道痕如何炼化,方源仍旧毫无头绪。

被方源念叨的气绝魔仙,此刻却是来到了西漠。

“应该就在附近了。”气绝魔仙低空飞行,四处搜寻。

他向气海老祖借蛊不成,便运用气道侦查手段,查探其他气道仙蛊。

结果一路寻求,让他前后找到数只气道仙蛊。

这些气道仙蛊大多是野生仙蛊,庞大气潮之后的自然衍生之物。少部分则是气绝魔仙抢夺其他蛊仙说得。

气绝魔仙也没有做绝,只要夺得气道仙蛊,便放任蛊仙离开。

一路走走停停,气绝魔仙便逐渐离开东海,来到了西漠。

“我算到的一只气道仙蛊应当就在附近。”气绝魔仙正念叨着,一个蛊仙忽然出现在他的侦查范围内。

八转蛊仙赤心行者!

------------

今天无更

请假

------------

\end{this_body}

