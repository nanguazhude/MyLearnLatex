\newsection{方源的气运}    %第一百一十四节:方源的气运

\begin{this_body}

%1
至尊仙窍中过去了两个多月,全新的煮运锅终于呈现在方源的眼前。

%2
它浑身金光灿烂,是一口大锅。仔细看的话,锅内外时不时闪过一抹白金之光,有着强烈的富贵之气。

%3
锅边相当厚实,有儿童的拳头般厚度。

%4
在锅的外沿,有八条雕龙,每一条都是栩栩如生。

%5
八条雕龙的龙尾齐聚锅底,相互缠绕,又延伸下去,形成八爪的支架。

%6
八条龙尾,又分别对准东、南、西、北、东北、东南、西北、西南八个方位。

%7
每一条雕龙的龙爪都扣在锅外表面,龙身蜿蜒向上,龙头搭在锅边,朝着锅内。龙眼紧闭,仿佛在沉睡。

%8
煮运锅已经是八转仙蛊屋了!

%9
和之前的煮运锅相比,不仅体型庞大了一圈,八条雕龙也全部清晰精致,不像之前七转的时候,总有一条模糊粗糙。

%10
如今的方源拥有九千多道自在天痕,构思复合杀招难度暴降。所以,哪怕方源的运道境界不高,也完全有能力改造煮运锅,让它威能更加强劲。

%11
但方源仔细思考了一番后,还是暂时打消了这个想法。

%12
如果将煮运锅打造得类似四元方悔血炼池,那么放到身外去用,威能将大打折扣。偏偏煮运锅还真得放在身外运用。

%13
方源之前和秦鼎菱交手,秦鼎菱要害方源的运势,关键时刻就是七转的煮运锅显形,顶在方源头顶上空,抵挡住了秦鼎菱的八转运道杀招。

%14
到了方源这等层次,因为面对的敌人太过强大,手段太多,所以防御也得全面。方源不仅要遮护自家肉身、魂魄,还得维护自身的气运。

%15
强者对于气运都相当重视。

%16
这是自身实力的重要部分!

%17
即便是复活重生的气绝魔仙,也有这方面的心思。当初他在气海之上,和气海老祖、秦鼎菱交手,为了保护自身气数,明智果断地撤退。

%18
而所谓气数,便是气道流派,运道效用。

%19
方源升炼煮运锅,就是为了保护自身的气运。煮运锅的结构、出身,决定了它弱于攻伐,重在守御蛊仙气运。而运道的攻伐手段,多在众生运、天地运两道真传之中。

%20
参与煮运锅构造的运道仙蛊,都高达八转,在防御气运方面绝对够用了。

%21
当然,煮运锅中仍旧有不是八转的仙蛊。

%22
比如食道仙蛊煮,还是原来的七转。

%23
方源不是不想升炼它,只是缺乏相应的八转食道仙材。四元方悔血炼池的血本仙蛊情况也是一样的,它仍旧是七转,缺乏八转血道仙材升炼。

%24
八转的食道仙材、血道仙材,短时间内还真不好收集。

%25
这和对应的流派有很大关系。

%26
越是昌盛的流派,它们的仙材就越多,越容易收集,仙蛊就越多,修行的蛊仙就越多。

%27
蛊仙修行当中,创造出来的仙蛊方越多,培育出来的仙材越多,又能增添整个流派的繁荣。

%28
食道、血道从未有真正昌盛过,修行的蛊仙向来罕见。相比食道,血道的修行者要多一点,但绝大多数都只是蛊师而已。血道蛊仙很少,对血道繁荣帮助很小。

%29
一方面,是因为天庭的诛魔榜。另一方面,开创血道的血海老祖也不过七转修为罢了。

%30
说起来,运道也不是很昌盛的,只是因为巨阳仙尊将其发扬光大,太过声名远播。

%31
真正修行运道的蛊仙,数量也不多。

%32
自从巨阳仙尊时代过去之后,历史上出现的运道强者几乎是没有的。勉强算的话,恐怕就只有秦鼎菱了。

%33
这主要还是因为,巨阳仙尊将三道真传藏得很深,并未大肆流传。

%34
秦鼎菱能够成为天下知名的运道强者,也只是从巨阳仙尊汲取了一些基础,而后还是她自己改变原有的流派,体悟天心,方才有所成就。

%35
所以,对于这次能够升炼这么多的运道仙蛊,方源感到非常幸运。

%36
因为运道并不繁荣,运道仙材获取难度很大,他能从两天洞天中搜刮出这么多的运道仙材,是非常难得的一件事。

%37
当然,能够升炼成功,最大的功臣还是四元方悔血炼池。

%38
如果没有四元方悔血炼池,绝对不行!

%39
四元方悔血炼池升炼八转仙蛊,有高达五六成的成功率。这种成功率,已经可以批量生产八转仙蛊了。

%40
成功率高,直接就导致仙材的损耗暴降,为方源节省了大量运道仙材的损耗。

%41
换做寻常蛊仙,用方源手中的全部运道仙材,升炼运道八转仙蛊,成功一只,绝对已经是庆幸得不行,欢天喜地了。

%42
寻常蛊仙要炼制八转仙蛊,通常筹集和消耗的仙材至少是方源的十几倍以上!

%43
雪胡老祖就是一个好例子。

%44
为了筹集炼制八转鸿运齐天仙蛊,他把自身底蕴都耗尽了,极限压榨手底下的魔道蛊仙们,最后连累得整个大雪山福地都崩溃。

%45
就这还是因为,雪胡老祖当时俘虏了马鸿运,可以节省大量运道仙材!

%46
不提雪胡老祖了,就算当初的巨阳仙尊、盗天魔尊,不也要和长毛老祖合作嘛。

%47
拥有四元方悔血炼池,方源炼道实力绝对是古今前三。这个优势才刚刚发挥,就已经表现得十分恐怖,带给的收益外人难以想象!

%48
至少方源清楚,就算曾经和巨阳仙尊合作的长毛老祖,也没有他这样高的成功率。

%49
摆在方源眼前的新难题,反而是他有点担心升炼八转仙蛊太过成功,导致今后八转仙蛊太多,从而令喂养方面出现问题。

%50
八转仙蛊的喂养,难度都不小的。

%51
当初为了喂养悔蛊,方源可是直接打造了一个庞大的花雾太泽!

%52
不过方源早已在这方面未雨绸缪,他的智道造诣总是在默默地支撑着他。

%53
之前方源借鉴时差洞天,不惜花费大量物力、精力,构造出年华分池,形成遍及整个至尊仙窍的宙道分区。

%54
有这些分区在,又有九千多道自在天痕,会极大地缓解八转仙蛊们的喂养问题。

%55
方源念头一动,催动八转煮运锅,查看自身气运。

%56
煮运锅静静地悬浮在他头顶上空,以前还微微沉浮,现在却是岿然不动,缓缓自转。

%57
以前的煮运锅转数比方源修行低,导致方源无法查看自身的完整气运,每次都只能查看其中锅内的那一部分。

%58
但现在,煮运锅已经可以装得下方源全部的气运。

%59
方源也终于再次一次目睹自身气运的全貌。

%60
只见他此时的气运,宛若一道纯银光柱,虽然细小但修长高耸,一股笔直冲天之势,似乎要贯通苍穹。

%61
在这紧密的纯银光柱的底层,有一团黄气包裹,仿佛尘土飞扬,隐有拱卫之意。

%62
而在银柱的上方,还有三片云团:一团漆黑如墨,一团金霞烂漫,一团星光璀璨。

%63
如此景象,让方源不禁陷入沉思之中。

%64
良久,他收拾思绪,又观察气运。

%65
但这一次,他不是关注自身,而是着重煮运锅外的分身气运。

%66
分身气运中,最有气象,最为浩大的,自然非气海分身莫属。

%67
气海老祖的气运,形若大海汪洋。浑白气流涌动,形成漫漫海流波涛。

%68
“气海分身的气运景象,倒是和名号一致,都是气海之象。”方源心知缘由。

%69
气运景象和个体的状态、底蕴有很紧密的联系。

%70
气海老祖的气海气运,不只是来源于他的称号,也有他坐镇气海,操纵气海大阵的原因。更关键的是,在几天前,方源还将无量气海杀招,彻底挪移给了气海分身。

%71
因由是气海分身企图建立战斗体系,想要无量气海杀招作为核心。

%72
“气海分身拥有两百多万道气道道痕,同时又有诸多气道仙蛊,掌握完整的元始气道真传,更是正气盟盟主,位高权重。如此种种,便有了如此磅礴浩大的白色气海气运。”

%73
“只是这白色气海,虽然澎湃庞大,但内里气流随意乱窜,混乱不堪,也泄露气海分身的隐患。”

%74
方源仔细思虑,觉得这隐患来源于两个方面。

%75
第一,是气海分身企图构造战斗体系,思绪紊乱。若是成功,或许能将气海调理顺当。若是失败,恐怕仍会维持眼前的内乱之景。

%76
第二,是气海分身的身份问题。气海分身和本体之间的关系,虽然暂时没有暴露,但时刻都有暴露的危机。熟知内情的气绝魔仙,已经被神帝城镇压了。

%77
而一旦这个秘密泄露,对于气海分身必然是严重打击,所以气海气运就有一种分崩离析的感觉。

%78
“气海分身是我的最强分身,八转修为,战力也是亚仙尊级别。所以有如此浩大的浑白气海气运。但照此来看,气海分身还有一个缺憾,就是进步空间有限。”

%79
方源拿自身纯银光柱气运,和浑白气海气运对比,很明显就能发现纯银气柱有上扬飞天的冲势,但气海气运却是根基不稳,只能稳住局面。

%80
“若是气功果的计划见效,或许气海气运还能有云雾升腾之象。可惜,那些留在两天洞天中的气功果,自从被气海分身汲取大半后,成长非常缓慢。很难再令气海分身有什么气道道痕的收获了。”

\end{this_body}

