\newsection{梦道大师}    %第一百二十二节:梦道大师

\begin{this_body}

%1
此时摆放在方源手掌心中的梦道仙蛊,宛如一颗浑圆如球的红宝石,然后又被劈成了一半。

%2
这只圆滚滚的半球形蛊虫看起来就十分可爱,并且朱红的色彩也非常瑰丽。

%3
它的头部,只占整个身躯的一成。巨大的背部表面圆润光滑,有着黑色的纵横纹路,让人联想到了龟壳。

%4
这是梦道七转仙蛊——梦甲蛊。

%5
它是防御性的梦道仙蛊!

%6
方源手中的仙蛊有很多,但梦道仙蛊只有一只,就是八转如梦令,龙宫的核心仙蛊。

%7
如梦令形如蜻蜓,圆脑袋,长身躯,一对粉红宝石般的复眼,有四对透明的翼翅。轻盈的翅膀上细细查看,还会发现有着一层淡淡的七彩斑斓的光晕。

%8
两者结合来看,方源发现了一点,似乎梦道仙蛊都十分漂亮。

%9
“不知道梦翼仙蛊,入梦游又是何等模样?”方源不禁遐想。

%10
他带着前世五百年的记忆,知道眼下梦翼仙蛊是在凤金煌手中,入梦游仙蛊则在毒蝎娘子那里。

%11
方源也不是没有动过这方面的心思。

%12
但一直以来,他对凤金煌都难以下手,而毒蝎娘子那块,则把握很小。

%13
五百年前世,入梦游和定仙游、酒神游、逍遥游,并公认为天下四大移动仙蛊。方源虽然强大,但毒蝎娘子却可依仗入梦游,转瞬穿梭到任何一处梦境之中。

%14
“如今,梦道还未真正彰显,并没有到梦境四处外显的时间。梦道仙材罕见,梦道仙蛊更加稀少,但我已经掌握了其中三只了。”方源对这个成果也是比较满意的。

%15
前世五百年,梦境外显的规模,要比幽魂梦境还要庞大,堪称是五域泛滥。

%16
如此一来,世人皆可入梦探险,再不像现在这般,梦境只被高层垄断。

%17
海量的梦境提供海量的机会,让蛊修们挖掘出海量的梦道蛊材。有了这些蛊材作为基石,梦道流派这才真正进入到飞速发展的阶段。

%18
所以,方源眼下这段时间,只是梦道刚刚开幕的时期。盗天梦境、乐土梦境、幽魂梦境的遗留,算是一把火,但烧不开大锅中的冷水。

%19
要让梦道这锅水彻底鼎沸起来,幽魂梦境的规模还是太小了,需要的是整个天地发力。

%20
对于方源而言,做梦蛊、如梦令的重要性,都要小于梦甲蛊。

%21
因为梦甲蛊乃是防御仙蛊。

%22
有了这只仙蛊,他就有了资本,可以抵抗梦道灾劫,从而帮助纯梦求真分身成功升仙!

%23
方源回想之前的察运结果。

%24
纯梦分身的气运,是一个粉色的沙盘,规模很小,但周围不断有黄沙落进来,转变成粉沙,壮大沙盘。

%25
沙盘气运预示着纯梦分身将得到乐土这边的资助,此时看来,果然如此。

%26
方源现在再次催动煮运锅,来观察气运。

%27
他自身的银柱气运,并没有多少变化。只是光柱隐约大了一圈,光柱底部的黄沙消弭了大半。

%28
而纯梦分身的气运,则仍旧是粉色沙盘模样,只是再没有黄沙落进来。沙盘中的粉沙堆得高高,处在一种崩溃或者突破的关键时刻。

%29
“我吸收乐土南疆真传,得到资助,令纯梦分身有了突破的契机。但即便如此,还是有着巨大风险,需要我再做更多更充分的准备。”

%30
方源立即明白过来。

%31
煮运锅升上八转之后,方源从运道上获得的便利有很多。

%32
他现在越发感觉到运道的好处,难怪当年巨阳仙尊能够崛起!

%33
疯魔窟是必须要去的,但方源不可能立即就去。

%34
眼下是他实力暴涨的关键时期,方源打算将实力提高到极限,这才会前往疯魔窟。

%35
尽管陆畏因那边将情势渲染得相当紧迫,但方源可以察运自己,知道还有一段珍贵的时间。

%36
“磨刀不误砍柴工,先让我钻研梦道的手段,帮助纯梦分身升仙!”

%37
方源回顾自身,他在梦道上的造诣可谓当世第一人。

%38
重生带来的梦境探索的常识,仍旧是眼下最尖端的研究成果,但这份优势正在迅速减弱。

%39
然而方源重生之后,亦在梦道方面有着巨大收获。

%40
他原本记忆自带解梦杀招。从影宗那边获得了引魂入梦、梦中换魂、纯梦求真变。在龙宫收获杀招梦里轻烟、梦启、梦中之梦。现在又从南疆乐土真传中,获取梦道杀招三世梦渡有缘人。

%41
方源掌握的梦道杀招,远比梦道仙蛊要多。

%42
许多梦道杀招的核心仙蛊是其他流派的,不得已之下,用海量的梦道凡蛊替代梦道仙蛊的作用。比如引魂入梦、梦中换魂等等杀招。

%43
“我最需要的,是以梦甲仙蛊为核心,开创出梦道防御杀招。其次是梦道的转移腾挪的手段,万一危险时刻转移出去,便能争取到宝贵的喘息时机。再其次是治疗手段,攻伐手段最后增添一些。”

%44
虽然方源的梦道境界只是普普通通,但是他却对开创新的梦道杀招饱含信心。

%45
原因就是他有自在天痕加身,可以大量开创复合杀招!

%46
长生天、天庭肯定在筹谋疯魔窟之争,所以根本不会出动兵力,去和长毛老祖联手,讨伐方源的异族大联盟。

%47
更别提来找方源的麻烦。

%48
至于南疆以武庸为首的南联,方源不去找他们的茬,他们已经十分庆幸了。怎么可能来到方源这里找死。

%49
外部环境前所未有的安定,方源闭关静修,开创梦道杀招,每一天都有可喜的进展。

%50
十几天后,方源忽然停手,微微发怔。

%51
他晋升梦道大师了!

%52
这一切都是厚积薄发,日积月累而得,所以顺利而又自然。杀招开创的过程中,就不知不觉的成功了,彻底迈进新的层次。

%53
重生以来,方源一直迫于外界压力,从未在梦道这个流派上花过功夫,投入过精力。

%54
一直到了现在,他才开始真正修行梦道,在这个方面下功夫。

%55
因为积累的很深厚了,这便导致他一旦开始发力,就提升了梦道境界。

%56
大师境界——蛊修对于蛊虫的运用,已然上升到艺术的层次,脱离了肤浅的匠气。

%57
最明显的特征是产生了专有直觉。对于蛊虫的认知,对于该流派的仙道杀招的设想,已经化为了蛊修身上的一种本能。近乎于天赋,就好像是与生俱来的感知那般自然、轻松。

%58
比如方源对于很多种梦道蛊虫的搭配,是很模糊的,只能一一尝试,才能知道成败结果。但现在,他在尝试之前,就会时不时的有直觉告诉他,一些尝试可行,一些尝试根本不需要去做,肯定失败,尝试的话也只是浪费时间、精力和物力而已。

%59
“很好!有了梦道大师境界,对我而言是如虎添翼!我在梦道上的成就,绝对是站在当下蛊仙界中的最前列。但……再过一段时间,就不好说了。”

%60
方源前世五百年的记忆中,涌现出了许许多多的梦道天才。其中又以凤金煌、毒蝎娘子这等人物最是惊才艳艳,提升梦道境界对她们而言,难度远远小于常人。

%61
有的人天生就适合某个流派,天赋这种东西是最让人无语的。尤其是凤金煌、毒蝎娘子这种大梦仙尊种子!

%62
方源有自知之明,早有估计未来在梦道境界上,会被其他人超越。

%63
但方源绝不会放弃梦道上的发展,前世五百年的错误他不会再犯了。

%64
梦道绝对是大趋势,未来的潮流,新流派的优势是很巨大的。

%65
“但现在还不是真正修行梦道的良机!”方源对自身处境也有清晰的认知。

%66
他先得成尊,才能抵抗外在的危机,才有将来徐徐发展梦道的时间和机会。

%67
从陆畏因口中得知了成尊的四个条件之后,方源深思熟虑,估计自己最后可能成尊的流派便是炼道。

%68
先炼道成尊,然后还能图谋其他流派,继续成尊。

%69
这是至尊仙体的优势!

%70
其他尊者只能干瞪眼。

%71
幽魂魔尊纵然知道如何炼制至尊仙胎蛊,但追杀方源失败后,他在进度上已经落后方源一大截了。

%72
方源的成长空间远超其他尊者。

%73
与此同时,中洲,灵缘斋。

%74
在灵缘斋专设的炼蛊秘室中,凤金煌此次炼蛊已经到了最关键的时刻。

%75
“快要成功了。”凤金煌双眼发亮,“只要炼成这只梦道凡蛊,我就能在梦境中拥有防御之能。这绝对是一个巨大的突……噗!”

%76
忽然,凤金煌面前的火光爆散,炼蛊失败,凤金煌大吐一口鲜血,仰头而倒。

%77
“怎么会这样?!”凤金煌惊愕万分,这是不应该的事情,这份梦道凡蛊方她前前后后推敲了十几遍,怎么会有疏漏?

%78
就算有什么错误,也绝不是在这个关头发生的。

%79
“这不是我的蛊方错误,而是蛊材有问题!”凤金煌头脑炸裂般的疼痛,但仍旧想明白了这件事。

%80
“我在炼蛊的时候,也做了检查。蛊材一定是被人动了手脚,而且定是仙人的手笔,所以我也察觉不到。”

%81
“被人算计了。”

%82
“有人想要置我于死地!”

%83
凤金煌想到这里,一股彻骨的寒意就从内心深处不断涌出。

%84
她挣扎着想要坐起来,但在半途中,又一头摔倒在地上。

%85
她满脸金纸之色,气息越来越弱,身上的力气也倾泻而出。

%86
“我就要死在这里了。”凤金煌咬牙,眼眶泛泪。她娇美华贵的脸庞贴在冰冷的地砖上,却无力再抬起来。

%87
“师父……”临死之前,凤金煌想到了龙公,又想到了母亲,还有她的父亲。

%88
“真的不甘心啊!”凤金煌在心中呐喊。

%89
门外忽然传来争吵之声。

%90
“大人,大人,您不能进去啊!”

%91
“我赵怜云乃是当代灵缘斋的仙子,能够随意征用任何的炼蛊密室。如何不能选择最好的一号室?”

%92
“仙子大人,一号密室早已经有人了。您冒然强闯,万一干扰了里面炼蛊的人,造成什么意外的话……”

%93
“意外?哼!能有什么意外?”话音刚落,砰的一声,密室的第三道内门就被强硬蛮横地推开了。

%94
“啊,这是怎么回事?”看管炼蛊密室的蛊师大惊失色。

%95
“她炼蛊失败,遭受了反噬。快让开,你区区一位蛊师,如何能救得如此伤势?”赵怜云一把推开蛊师,眼中闪烁着幽芒。

%96
蛊师满头大汗,立即联想到了凤金煌和赵怜云的争斗历史,心中慌乱无比,手足无措,满头大汗。

%97
“还不快滚?干扰我救治,你能担待得起吗?”赵怜云又喝道。

%98
蛊师咬牙:“仙子大人……啊!”

%99
却是被赵怜云一挥手,就被一团玄光裹住,强行飞送了出去。

%100
赵怜云走近凤金煌,缓缓地半蹲下来。

%101
两位绝色佳人相互对视。

%102
炼蛊密室中一片死寂。

%103
命运弄人,曾经两人之前的地位,已经互转。

%104
凤金煌却没有临死前的半点慌张,面色平静至极:“原来,要我死的人是你们。”

%105
她聪颖至极,立即从赵怜云的身上,联想到了李君影和徐浩二仙。后者两人和凤九歌的矛盾十分深厚。

%106
赵怜云叹息一声,向凤金煌缓缓伸过手去。

%107
凤金煌毫无反抗之力,旋即沉沦黑暗之中。

\end{this_body}

