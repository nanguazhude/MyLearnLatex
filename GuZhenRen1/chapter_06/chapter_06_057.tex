\newsection{邀请气海}    %第五十七节:邀请气海

\begin{this_body}

%1
华文洞天。

%2
京城,苏家。

%3
在某个客房的庭院里头,树立着一座书屋。

%4
这座书屋乃是凡蛊屋,形如树屋,绿叶葱茏,枝繁叶茂。

%5
此刻,李小白就身处屋内,盘坐下来。

%6
在他面前站着一位美妙少女,正是苏琪涵。苏琪涵神情微凝:“李郎,你真的考虑清楚了吗?本命蛊可不是随意更换的。”

%7
李小白点头微笑:“琪涵,你不要过于忧虑。我是深思熟虑过的,我最感兴趣的就是作诗。所以对我而言,诗心蛊是最合适不过的本命蛊了。只是以前没有条件,不能随意更换本命蛊。但既然这座书屋威能不凡,有辅助替换本命蛊的功用,我又岂能放过这样的机缘呢?”

%8
苏琪涵叹息一声:“你既然已经决定,那我就不再劝你了。我去屋外为你护法。”

%9
“嗯。”李小白也不客气,直接闭上了双眼。

%10
自从他得到了书屋传承之后,便回到京城,和苏琪涵重逢。李小白在之前的诗词大比中虽然落选,但已经彻底证明了自己。苏琪涵便将他邀入苏家暂住,也无人反对,反而成为一时美谈。

%11
李小白并不吝啬,住进苏家后,便将书屋传承给予苏琪涵分享。两人几乎终日在书屋中渡过,不断修行钻研,相互之间交流密切。

%12
两人之前虽然是因为机缘巧合,成就了夫妻之实,但感情其实不多。苏琪涵之前的行动,更多的是出于自身的品性。

%13
但这些天相处过来,苏琪涵越发欣赏李小白的才华,喜爱更甚。

%14
李小白得了这个机缘,根本一点都不隐瞒苏琪涵,都拿出来给她分享。从这点上,让苏琪涵看到了李小白的品性纯良,愿意为自己付出,绝非薄情寡义的读书人。

%15
两人虽然还未正式定亲,但苏琪涵早已经将李小白当做自己一生中唯一的夫君。

%16
苏琪涵步出书屋,心中仍旧忧虑:“李郎得到的这座书屋,价值很高,包含了诗词歌赋等多种信道门类。然而这些门类却都局限于蛊师修行,没有蛊仙的晋升之道。我曾多次暗示李郎,只要他入赘我苏家,就能获得我苏家书仙的修行法门。可惜李郎对书法并不喜爱,一心想要扎在诗上。唉,他既然如此喜爱,那我也无法劝阻什么。如果将来他一事无成,终为蛊师,我便努力奋发成就蛊仙,成为朝廷官员,一生都养着他护着他。看着他作诗念诗,只要他开心,我就开心。”

%17
华文洞天中,有一个统一的朝廷。

%18
朝廷中的大官,都是蛊仙身份。苏琪涵乃是苏家爱女,掌上明珠,天赋、才情惊艳世间,自然能着接触到苏家修仙的法门。

%19
“大小姐,您歇息去吧,站着怪累的。有我们在,定保李大才子万无一失。”护卫和奴婢上前劝说苏琪涵。

%20
苏琪涵却是摇头:“你们都站在院外去,把守大门就好。守护李郎的事情,我亲自来。”

%21
书屋越发震荡,内里发出微微光华,透出窗外。

%22
半柱香的时间过后,书屋静止下来,树叶树枝不再摇摆颤抖。

%23
书屋中,李小白缓缓地睁开双眼。

%24
替换本命蛊的过程相当顺利,李小白展开内视,只见空窍当中一只蛊虫宛若心脏,朱玉质地,表面笼罩着一层淡金光晕。

%25
正是刚刚替换而来的五转蛊虫——诗心。

%26
诗心蛊隶属信道,在整个信道的修行法门中,属于偏门。信道修行中最多的是盟誓,比如海誓蛊、山盟蛊、诺言蛊、白纸黑字蛊。其次是传讯,比如鸿雁蛊、纸鹤蛊、蝶信蛊、飞剑传书蛊、飞鸽传书蛊等。再次则是看听读写,比如水文蛊,就是书写一类的。兽语蛊则是和读说相关。

%27
诗词歌赋等等,根本不是主流。

%28
李小白做出这样的选择,也难怪苏琪涵会担忧他的前途。

%29
但她却不知道李小白的真正打算。

%30
“这个洞天中环境特殊,有着济文才杀招。只要我开创出上佳诗词,就能得到奖励,迅速晋升。五域乱战将至,这种快速提升个人实力,又没有后遗症的路途,当然是最珍贵的。苏琪涵虽然能接触到蛊仙之法,但是却不知道外界大局。”

%31
“改换了诗心蛊,接下来就是练习诗意蛊、诗情蛊,用采诗蛊采集外界诗词,提升诗壁蛊。另外诗境蛊也要练习一番,这可是类似于战场杀招的蛊虫,十分优异。”

%32
李小白正思索着,书屋的门被缓缓推开,苏琪涵带着关切的目光步入屋内。

%33
“李郎。”苏琪涵轻声呼唤。

%34
李小白淡淡微笑,对她点点头:“一切顺利得很呢。”

%35
苏琪涵展颜,美眸似水流露出柔情:“这就好。”

%36
轰——!

%37
正说话间,忽然天空震荡,轰鸣声响彻寰宇。

%38
“怎么回事?”李小白、苏琪涵都被惊动,连忙走出屋外,就见万里晴空,纯净透蓝,一阵狂风迅速扑面而来,呼啸而过。

%39
狂风只有极其短暂的一阵,卷席过后,再无异变。

%40
京城中生活着的凡人、蛊师乃至蛊仙,都感受到这场突如其来,又迅速消失的异变。绝大多数人都莫名其妙,唯有少部分的蛊仙官员知道内情,喜形于色。

%41
“成功了!”在华文洞天的某处,华语老仙满脸激动,对气海老祖深深一拜,“恭喜老祖,成功破解了气功果。老祖拯救了华文洞天上下,恩同再造!今后旦有驱使,华某定是拼尽全力,绝无二话!”

%42
方源扮演的气海老祖哈哈一笑,扶起华语老仙:“不必多礼了。”

%43
一旁的诸仙都欢喜无限。

%44
“老祖神威!”

%45
“吴帅那边连连试验,损毁了多少洞天。老祖首次尝试,就大获成功,完全是吴帅不能媲美的。”

%46
“不知老祖何时才能为我等解决气功果的内患呢?”

%47
“是啊,我等洞天中的气功果都已经巨硕至极,距离自爆的界限越发接近了,情势都很危险啊。”

%48
蛊仙们大声称赞的同时,又催促方源再度行动。

%49
方源却摆手道:“此次试验虽是成功,但老夫却察觉到了当中的瑕疵。只需再闭关三日,便能将其改良妥善。届时再为诸位扫平祸患,岂不更美?”

%50
众仙大喜,齐声道:“全听老祖您的安排!”

%51
方源微笑,眼中精芒一闪即逝。

%52
天庭。

%53
很快,秦鼎菱那边便接到了这个消息,当即面露喜色:“好,好啊!当初我选择和气海老祖合作,果然是正确的。”

%54
气海老祖不负期待,不仅开创出了解决气功果的手段,而且还一次试演成功。根据情报所述,华文洞天不光是安全地铲除掉了气功果,而且果实爆散,形成一股无害的气潮,迅速卷席整个洞天,将洞天底蕴提升了好一截去。

%55
秦鼎菱心情大好。

%56
这段时间以来,天庭已经逐渐从宿命大战的战败阴影中走出来了。经过秦鼎菱等天庭众仙的筹谋和奋斗,好消息一个接着一个。

%57
先是异族势力得到遏制,其次青仇被俘虏,现在气功果的祸患也要被铲除了。

%58
“只是接下来,该先清除天庭中的气功果,还是谋夺雷电蛊呢?”

%59
秦鼎菱一个人拿不定主意,便召集天庭群仙商议。

%60
“气海老祖不愧是气道大能呐,有了这个手段,他那方势力便稳立不败之地了。”

%61
“吴帅联盟至今还在试验手段,他们的气道造诣薄弱得很。”

%62
“只要气海老祖拖延得住时间,说不定无须开战,就能等到吴帅联盟自行崩溃。”

%63
秦鼎菱听到这番乐观的估计,当即摇头:“吴帅,枭雄也。兔子逼急了还会咬人,更何况这位龙人。一旦他得到这个消息,恐怕会立即挥师和气海老祖开战。”

%64
“不错,气海老祖可以安全铲除气功果,吴帅不能。时间拖得越久,对吴帅而言越不利。吴帅兴师极有可能。”

%65
“若是这样一来,我们就得提前行动,让气海老祖出手,为天庭解决气功果的祸端呐。”

%66
“说的对。一旦开战,气海老祖分身乏术,并且定会遭受吴帅全力猛攻。此刻局势已如水火,一旦交手绝不会轻易罢战,恐怕真得要分得个生死了。”

%67
议论到这里,秦鼎菱又看向九灵仙姑:“仇恨蛊炼化得如何?”

%68
九灵仙姑摇头:“刚有突破。仇恨蛊的状态颇为奇异,已然和青仇融为一体。我们不久前才找到方法能够炼化仇恨蛊。若要炼出仇恨蛊来,需要的时间颇长。不过好消息是,若是成功,青仇也能为我方所用。”

%69
秦鼎菱点头:“既然仇恨蛊还未彻底夺来,那么雷电蛊也不着急去抢。我意全力隐瞒这个消息,邀请气海老祖秘密前来天庭,将天庭隐患悄然解决。做成这事之后,再和吴帅全面开战。”

%70
群仙听到这话,都没有异议。

%71
“只是我们该如何酬劳气海老祖呢?”有蛊仙问道。

%72
“此事不难。气海老祖对我天庭收藏的气道仙蛊,有着极大的兴趣。之前书信往来,他就屡屡商讨要借这些气道仙蛊使用,只是我一直未允,都推托了去。”秦鼎菱微笑。

%73
又有蛊仙补充:“此事还得防备另外一人,便是气绝魔仙!”

%74
秦鼎菱点头:“我也考虑到了。这个魔头也是气道大能,若是和吴帅合作,便会再次平衡双方处境,是一个巨大隐患。”

%75
“气绝魔仙难以追踪,而今不知去向。他刚刚重生,对现今的各大流派并不熟悉。他现在在宝黄天中一直大肆贩卖气道仙材,全力换取各个流派的传承,即便是蛊师传承也不放过。吴帅手头上有着和他交易的筹码!”

%76
秦鼎菱再次强调:“所以当今之计,便是立即邀请气海老祖前来天庭,同时全力封锁消息,争分夺秒!”

\end{this_body}

