\newsection{方源的分身气运}    %第一百一十五节:方源的分身气运

\begin{this_body}

%1
除了气海分身之外,其次就是吴帅气运了。

%2
他的气运规模削弱了很多。

%3
因为龙宫摧毁,短时间内难以搭建,四大龙将尽皆陨落,甚至帝藏生至今仍旧被镇压在天庭内部。

%4
现在的吴帅气运,仍旧是一条飞龙形态。这和吴帅分身的龙人身份密切相关。

%5
仔细深看,就会发现吴帅的飞龙气运,好似由一只只的蚂蚁构成的。

%6
数不清的蚂蚁,凝聚一体,形成蚁龙气运。

%7
“吴帅代我主持异族大联盟的俗务之外,就是钻研本身的奴道手段,企图将战力提升一层。所以方才有蚁龙之象啊。”

%8
方源再看宙道分身的气运。

%9
宙道分身的气运还是微小的光阴长河的模样。但相比较之前,光阴河面更加宽阔,水面上波涛翻腾。

%10
战部渡的鱼鹰气运,也没有多大改变。

%11
但李小白的气运,则是模样有了巨大差异。

%12
原本他的气运如花伸开,并且花蕊中积蓄着一层薄薄的花蜜琼浆。彰显着李小白的修行有了阶段性的成果。

%13
而现在他的气运模样大变,化为了一面白墙。

%14
白墙上已经有了三首诗词,长短不一。仔细一读,便知是李小白“创作”出来的三首绝妙好诗。

%15
诗的每一个字,都是漆黑墨色,在白墙上更显清晰。

%16
而在白墙的周围,还飞悬着无数的漆黑墨字,正要飞上白墙。

%17
“这是诗壁气运了。”方源心中有了明悟。

%18
李小白自从接受了书屋传承之后,就一发不可收拾,走上了扬名天下的路途。

%19
他不仅改换了本命蛊,而且还行走天下,四处采诗,并将自己开(piāo)创(  qiè)过来的诗词,形成三记杀招。

%20
这三记信道杀招,让他成为华文洞天中公认的蛊师第一人,更成了被无数人看好的蛊仙种子,甚至有评论说他是洞天数百年难得一出的才俊!

%21
李小白如此优秀,当然吸引了华语老仙的注意。

%22
种种迹象表明,华语老仙已决定全力栽培李小白!

%23
李小白的近况,气海分身、方源本体都十分清楚,但从未有所干涉,只是旁观。

%24
当初方源打造李小白这个分身,是想图谋华文洞天中的信道传承,帮助自己弥补信道方面的缺憾。

%25
然而,因为华文洞天乃是信道洞天,监管极严,导致方源不能大力扶持,李小白独自一人推动得相当艰难。

%26
所以结果是,方源已经过了宿命大战、追杀战,把魔尊幽魂伏杀后,天道道痕都解决了,李小白这一块才见明显的起色。

%27
而华文洞天也在正气盟中,他们信道真传的大部分内容,在正气盟内部的交流中已经可以获取了。

%28
世间之事向来如此。有时候计划赶不上变化,有时候变化跟不上计划。

%29
这样一来,李小白这个棋子就又有了新的安排。

%30
所以,方源接下来都会对他放任自然,顺其发展。

%31
最后,是纯梦分身的气运。

%32
纯梦分身仍旧是五转层次,要以纯梦求真体升仙,会引来梦道灾劫。即便是目前的方源,也没有自信应对。

%33
但纯梦分身的气运却也改变了。

%34
他的气运原先是一团粉色雾气,缥缈虚弱,不断变幻。但现在这团粉色雾气,却凝聚起来,化为了一个沙盘。

%35
沙盘仍旧是粉色,规模很小,但周围不断有黄沙落进来,转变成粉沙,壮大沙盘。

%36
“沙盘气运……这有什么预兆?”一时间,就连方源都感到了困惑。

%37
发动智道手段后,方源眼中精芒一闪即逝,有了一些猜测,还留待考证。

%38
方源细数一下,不知不觉间,自己创造的分身已经多达八人了。

%39
打造这些分身,是方源学习魔尊幽魂的成果之一。

%40
魔尊幽魂就是靠着分身,搭建出了影宗、僵盟。方源的这些分身也带给他相当大的帮助。

%41
“八位分身当中,唯一牺牲的便是房睇长了。嗯?等一下,这是……”方源忽然发现了一块气运。

%42
这块气运十分隐约,几乎是透明的,带给方源的感应也十分的微弱。

%43
方源差点漏掉,但仔细分辨之后,脸上浮现出了一抹惊喜之意:“这是房睇长分身?他竟然还活着!”

%44
宿命大战期间,方源被元莲算计,不仅失去了豆神宫,还搭上了房睇长分身。

%45
时间过去这么久,豆神宫已经和帝君城合并成了神帝城,更成为天下公认的第一仙蛊屋,方源也以为房睇长分身完蛋了。

%46
没想到这一次察运,居然发现了端倪。

%47
“以前我怎么没有发现?”

%48
“我明白了。”

%49
“煮运锅提升到了八转,我的察运手段比之前提升了数十倍威能。还有房睇长分身那边,显然是最近一段时间,有所发展。两相合并,便使得本体和分身之间的气运联络,又加强了不少,这才让我分辨出来。”

%50
神帝城虽然拥有壁画世界,但留下安居乐业杀招的元莲仙尊,早在巨阳仙尊之前。

%51
因为这层关系,神帝城对于运道方面是较为弱势的。

%52
发现房睇长分身还活着,并且隐约仍在壮大,这是一个意外之喜。但方源并没有改变自己的计划。

%53
短时间内,他还不想打上神帝城去。

%54
神帝城不是蛊仙,手段固定,难以改良,放在那里并不可怕。而方源是亚仙尊,终于迎来了可以安心休整,独自修行的美妙时光。

%55
四元方悔血炼池建成,大把的洞天留待吞噬,方源的实力会不断地突飞猛进。何必这个时候,去冒风险和神帝城死磕?

%56
察看了自身和分身的气运,方源了解到了更多东西。

%57
煮运锅能够让方源本体大幅度提升分身的气运,但方源没有这样做。

%58
一来,本体、分身的状况都挺好的,经过之前的困境和磨难,气运自发回升,否极泰来。所以没有必要。

%59
二来,消耗的不仅是仙元,还有方源本体的气运。毕竟是把锅内的气运吐到分身那边去的。

%60
“接下来该升炼木道仙蛊了。”方源再次投入到炼蛊大业里。

%61
至于为什么选择木道,原因很简单。

%62
方源收集到的木道仙材众多,可以支撑升炼。异族大联盟中的萧荷尖、青森大圣,都是小人八转蛊仙,他们的洞天都是木道洞天,上贡了充足的木道仙材。

%63
“先将仙蛊天元宝皇莲升炼到八转吧。”

%64
神帝城,壁画世界。

%65
新增的兽吼壁画当中,一座行营正在紧锣密鼓地搭建着。

%66
行营选址显然经过深思熟虑,不仅依靠山险,而且傍着一道河流。

%67
赵淑野站在山石上,居高临下指挥着上百位匠人全力劳作,建造行营。

%68
“这匠人蛊虽然战力微弱,但用处着实广大呢。”赵淑野身边站着碧霞仙子,看着热闹嘈杂的工地,有感而发。

%69
赵淑野立即昂头,骄傲地道:“还是咱们碧霞仙子慧眼识珠。匠人蛊的前景,远比贼人蛊、军卒蛊、捕快蛊要高得多。除了改造仙材,搭建蛊屋,他们还能炼蛊!”

%70
碧霞仙子点头:“是啊,若非你的这些匠人,我们岂会连建十座行营,一步步深入到兽吼壁画的深处呢。不过,你一个人如何把控这么多的匠人?这牵扯的心神太过庞大了。你也用了组织蛊?”

%71
组织蛊是人道蛊虫中的某个类型,能令人蛊之间形成组织。房睇长的门派令蛊,便是此类。

%72
赵淑野哈哈一笑,毫不藏私遮掩:“对,我用了工社令蛊,已经建造了一个工社。我自己带着社长蛊。选择了十几位高转的匠人蛊,赐予他们工头蛊。副社长蛊我也给出了一个,手头上还有三个。不着急,等到我的工社继续壮大下去,就有用武之地了。”

%73
两人正谈论着,忽然有紧急军号传来。

%74
碧霞仙子面色微变:“又有兽群来袭了。”

%75
赵淑野无所谓地道:“有萧七星他们在,不必有多担心。”

%76
果然,下一刻,萧七星的声音便响彻整个行营:“全军出击!”

%77
砰砰砰砰砰……

%78
一连串的闷响,大量的白烟被风吹散后,空阔的大营中塞满了军队。

%79
数量最多的士卒蛊,小头目级别的伍长蛊、什长蛊,更高一级的百夫长蛊,乃至千夫长蛊都用了出来。

%80
萧七星的大军在这段时间,规模上又有涨幅。

%81
神帝城真的是人道修行的天堂!

%82
大军冲向兽潮,两方宛若两波巨大的海浪,轰然一声对撞。

%83
僵持了一阵后,原本整齐的阵线开始出现参差的锯齿,大量的野兽和士卒在交锋最激烈的地方倒下。然后,又有大量的士卒和野兽填补上来。

%84
嗷吼!

%85
数只强大的野兽出现了。它们体格明显超出其他一大截,像是堡垒般在战场横冲直撞。

%86
萧七星局中指挥,面色流露出一抹凝重:“越是深入,兽群就越强大。还请仙友们出手。”

%87
“好说。”陈大江、古霆等人纷纷开始行动。

%88
这一场大战持续了半个时辰,兽潮退走了,萧七星等人也伤亡不小。

%89
半天之后,行营彻底建成,后续的补给队伍,夹杂着商人、小贩、平民,甚至是乞丐,入驻行营。

%90
这些行营就像是一颗颗最强硬的钉子,死死地钉在各自的位置上,成为人道十子深入探索的坚实依靠。

\end{this_body}

