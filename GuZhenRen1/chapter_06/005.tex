\newsection{凤赵之争}    %第五节:凤赵之争

\begin{this_body}

中洲,湖心山。

灵缘斋的公共福地大本营,就坐落于此。

福地中,有一处地方,名为静雨谷。

凤金煌徒步入谷的时候,正好下起了雨。

细雨。

雨安静地下着,悄无声息。

沐浴在雨中,连带着凤金煌都不知不觉间放缓了脚步。

空气中透露出清新的草香,缭绕起一层薄薄的绿意。

这样的雨,并不冷,像是玉,有着一种内敛的温和。

凤金煌知道:这是静雨。当这样的雨一下,任何的声音都会被吸收,天地一片安静,静得能让人听到自己的呼吸,自己的心跳。

静雨乃是六转仙材,这片山谷便是静雨的资源点,因此得名为静雨谷。

凤九歌背叛天庭,不知所踪后,他的妻子也是凤金煌的母亲——白晴仙子便主动申请,调到静雨谷中采集静雨,终日不出山谷一步。

每当静雨飘洒而下时,山谷中就会传出琴箫之声,似是白晴仙子的哀婉凄切的哭诉。

凤金煌正是听闻了这个传闻,心中不免担忧,这才赶来山谷,探望自己的娘亲。

现在静雨飘飘,果然从山谷中传出一阵琴箫之音。

琴声缥缈,宛若九霄白云,古意昂然,悠扬悦耳。而箫声却是高亢,高低起伏近乎突兀,宛若厉鹰纵横天穹,尖锐破风,撕扯苍岚。

起初,时而琴声响彻,时而箫声回荡,接连交替,像是双方诉说,你一言我一语。

随后,琴声和箫声开始相互接触,一点点合奏,逐渐地融合交汇。白云缥缈,飞鹰在云中纵横。两种突兀的乐声,不断合并,竟有一种难以言说的美感。

最终,琴声和箫声变得浑然一体,不分彼此,形成一种混音,既有琴声的悠扬,又带着箫声的清厉。独一无二,美不胜收!

凤金煌被这等美妙的乐声吸引,不由露出陶醉之色。她深入谷内,顺利地见到了自己的母亲。

白晴仙子坐在一处竹亭中,两只蛊虫宛若白玉圆珠,在她身边环绕飞旋。正是两只音道凡蛊——琴蛊、箫蛊。

当凤金煌走进亭中,乐声缓缓停息下来。

白晴仙子早已知晓凤金煌的到来,缓缓转身:“这首仙魔一同歌,乃是你爹与我相恋时创作而出,是我们的定情之歌。我还记得当初,你爹为我第一次演奏时,就说:仙魔本一家,元始仙尊起初是被诸多异人称之为原始魔尊呢。这首歌曲意义非凡,你爹一直珍藏不用,可是在宿命大战中,他却是用了。”

说到这里,白晴仙子话锋一转,凝神看向凤金煌:“这一次的名单已经批示下来了。”

凤金煌点头:“我知道。上一次门派举荐了我,结果天庭没有音讯。这一次门派将孙瑶替换上去,很快就得到了天庭的批准。”

白晴仙子平淡地道:“原本门派仍旧是想将你推荐上去,但李君影、徐浩从中阻碍,最终改成了孙瑶。这样的结果,你有什么感想呢?我的女儿。”

“呵呵。”凤金煌微微一笑,笑颜灿烂若花,一瞬间,带给竹亭一道明媚的阳光。

“娘亲,你又何必如此考较我呢?自从爹临阵反戈,叛离天庭,我就有了心理准备。眼下,梦道仙蛊没有被收缴了上去,情况已经非常不错了。”凤金煌道。

白晴仙子闻言,也微微露出一丝笑意:“正道中亦非没有强取豪夺,只是往往披上一层大义和道德。你能明悟到这一层,已经很不错了。不过你也不要过于担心,你爹离开天庭,却是开创了命运歌,卓绝战力举世共睹,天庭和灵缘斋也不敢轻易地拿捏你的。”

凤金煌流露出一抹不悦之色:“娘亲,你不必为爹说话了。我虽然能理解爹,但我不会原谅他。他忽然间背离天庭,根本不提前告知我们,这就是抛妻弃女!总有一天,我会找他算账。”

“呵呵呵。”白晴仙子笑着摇头,走到凤金煌的面前,拉着她的手,“你误解你爹了。那首仙魔一同歌,他当做定情之歌,一直珍藏,除了在我面前演唱,从未用于其他方面。他独独在宿命大战中用出来,便是想告诉我,仙魔一同,他对我的感情从未变化,就算是脱离正道,重归魔道,也无妨我们一家三口之间的感情。”

“你信吗?只要我稍稍流露出一丝脱离灵缘斋的动向,他定然会主动接我。他只是也了解我,知道我不想成为魔道中人。所以,他给了我们俩个自由选择的机会。他并不想让他的理想,成为我们的拖累啊。”

凤金煌冷哼一声:“爹宁愿帮助方源这个大魔头,也不愿帮助师父。师父的确固执,总是叨咕宿命什么的,但他终究对我们是一片好心,真正为我着想的。爹此举,未免也太过冷漠无情了。”

“娘,你既然没有什么事,那我就先走了。”

“你呀。”白晴仙子叹息一声,却是没有挽留女儿。

凤金煌出了静雨谷口,便见到两位女蛊师。

一位娟秀,一位可爱,都是凤金煌的数人,有些惴惴不安地站在谷外守候。

见到凤金煌后,她们立即面露喜色,迎了上去。

“师姐,我们听说你来到这里,所以就赶过来了!”面容娟秀的便是秦娟。

“师姐,对不起,我、我也不想侵占了你的名额……”孙瑶可爱圆润的脸蛋上,尽是不安之色。

秦娟、孙瑶平素就和凤金煌交好,两人一直都是凤金煌的跟班。

只是最近几年,凤金煌跟随龙公修行梦道,和这两位相处的时间就稀少很多了。

凤金煌朗笑一声,伸出手指点了点孙瑶的额头:“你呀,这一次表现得很不错,为什么要道歉呢。举荐的名额只是一桩小事,难道没有资助,我就不能成仙了吗?对于我而言,成仙绝不是什么障碍。只是成就什么蛊仙,才是我要考虑的事情。但对于你而言,这一次却是一番难得的机遇,好好抓住吧。”

说完孙瑶,凤金煌又看向秦娟:“你一向有着主张,能够坚持,这一次,道痕加身,你怎么会眩晕过去呢。若非如此,这一次举荐的名额便会是你的。”

秦娟不敢反驳:“师姐教训得是,是我修行懈怠了。”

“不是这样的,师姐。是秦娟师姐她当时身上有伤,所以忍受不住剧痛,这才昏倒的。”孙瑶连忙澄清道。

“原来是这样。”凤金煌点点头,惋惜道,“这的确是很大的损失。方源摧毁宿命蛊,将其炼成无数天道道痕,分散到天下万众身上。天道道痕承受得越多,将来就越有好处。不管是修行,还是经营仙窍,都有利无害。不过没有也无妨,充其量这只是一场际遇罢了。决定你一生成就的,还是要看你自己的才情和努力。”

“是,秦娟谨遵师姐教诲。”秦娟躬身一礼。

凤金煌再笑一声:“走吧,好好指点一番你们。”

秦娟、孙瑶顿时面露喜色,后者更是蹦跳起来,欢呼道:“真是太好了,我们可是好久没有聚会了。”

三人结伴而走,逐渐远离静雨谷。

谷内的白晴仙子缓缓收回侦查手段。

“女儿你长大了呢。”白晴仙子面露欣慰之色。

凤金煌对自身处境,对天庭和十大古派都有很透彻的理解,而对凤九歌的行为也抱有自己的想法,最后御下的功底也非常扎实。

“或许,也是你叛离了天庭,促进了女儿的成熟吧。她的路,就让她自己来走了。”

“我身为母亲,所能做到的,就是祝福。”

“当然,若是门派中有某些人不识好歹,仍旧得寸进尺,那我白晴也不会一再避让。”白晴仙子面色清冷。

与此同时,在另一处的含情峰。

灵缘斋的太上长老徐浩、李君影主动拜访赵怜云。

赵怜云已经成就蛊仙,不久前被灵缘斋调派,成为含情峰峰主。

她能够达到今天的成就,除了自身努力之外,在外多亏了徐浩、李君影的扶持。在蛊仙的修行、仙窍的经营方面,赵怜云也多受二仙指点,少走了许多弯路。

赵怜云将这些恩情记在心头,这一次徐浩、李君影联袂拜访,她立即出关,打断修行,亲自招待。

只是听闻二仙来意之后,赵怜云感到了为难。

“二位前辈,是想让我发动当代仙子的影响力,来对付凤金煌和白晴仙子?”

“不错,眼下正是最好时机!凤九歌临阵反戈,导致我天庭大败,宿命被毁。此时我们乘胜追击,必定能够彻底战胜对手。”徐浩面露狠厉之色。

他和凤九歌的仇恨,可谓由来已久。曾经更是恨不得凤九歌去死,按捺住了关键消息情报,隐瞒不报。凤九歌手腕也非常强硬,将他们夫妻二人排斥到灵缘斋的权利边缘。绝大多数时候,二仙对抗凤九歌都是处于下风,只能相互抱团取暖。

对于他们而言,凤九歌临阵叛逃是一件令人惊喜的好消息。

“这个……”赵怜云迟疑,“凤九歌已经背叛天庭,二位前辈和他有仇,却又何必连累到他的妻女呢?。”

“怜云啊,我们万不可有妇人之仁。想想吧,你当初和凤金煌争夺灵缘仙子之位时的情景。”李君影劝道,“不抓住这次机会的话,将来想要后悔可就晚了。你不必太多顾虑。再告诉你一个消息,门派举荐的名单原本是凤金煌,但此刻已经不再是她了。”

“二位前辈,得饶人处且饶人。”赵怜云劝说,她还是不太愿意。正所谓冤家宜解不宜结,况且当年她和凤金煌争夺仙子之位,都是正面举措,凤金煌有着深厚的背景,也从未动过什么卑鄙的手段。

赵怜云、凤金煌虽然是对手,但赵怜云暗中也钦佩凤金煌的为人,对她并无多少恶感。

徐浩却丝毫听不进劝告:“当年凤九歌那魔头,可曾饶得我们?”

“你太宅心仁厚了。怜云。”李君影皱起眉头。

赵怜云心中暗叹,转变路线,继续劝说:“二位前辈,此事还得深刻分析。门派虽然举荐了他人,但天庭当时并没有直接剥夺凤金煌的名额啊。由此看来,看来天庭高层对于她的态度,还是摇摆踌躇的。”

“正是天庭摇摆踌躇,我们才要助推一把,将此事定性!权利的斗争,永远不会终止,只有彻底分出胜败,才能罢休。”李君影饱含深意地道。

赵怜云再叹一声,只得答应下来。

她没有办法,欠下徐浩、李君影太多人情。虽然心底并不愿意,但是人在江湖,身不由己。这两人对于自己帮助如此巨大,自己若拒绝,他人会怎么看?

赵怜云乃是半个天外之魔,虽然成为当代仙子,但在灵缘斋中仍旧受着排挤。若是失去徐浩、李君影这层关系,她就会彻底变成孤家寡人。

二仙见赵怜云的答应,喜形于色,当即担保必有重酬。

赵怜云却对重酬不感兴趣,送走了二仙,她继续闭关潜修。

宿命一毁,对于赵怜云而言,也是重大利好的消息。马鸿运的重生,将会减少极大的阻碍。

只是马鸿运的魂魄仍旧在方源的手中。

所以,赵怜云全力修行,就是想增长自身实力,为将来搭救爱郎做最好的准备。

密室中,赵怜云沉下心来,探查自己的仙窍。

怜云福地,总计有八百六十万亩,光阴流速比较外界,是一比十三。地貌主要有草原、平原。

但是当天道道痕加身后,怜云福地正从各个方面,迅速地演变着。

先是大地迸裂,形成巨沟,随后降下暴雨,形成贯通东西的巨大河流。河水泛滥,形成洪灾,淹没周遭。

三天三夜后,洪水消退,留下大片大片的肥沃泥地。一道道小沟小河,连通巨河主干,形成精致繁密的河流网络。

赵怜云全程目睹了整个仙窍的变化,令她目眩神迷,心头震撼。

“有了天道道痕,仙窍就能自行演变,变得极其自然、平衡。虽然暂时是有损失,但是潜力和前景却是暴涨了无数倍。”

赵怜云拥有盗天真传神不知,灾劫不用加身,此时有利有弊,经营仙窍一直是她的软肋。但此刻天道道痕演变,让她这块短板拔升了许多。

“若是我有更多的天道道痕就好了!”

“唉,真是可惜,我总共也只不过承载了六根半的天道道痕。并且这些道痕都是一段段的,并没有一根真正完整的天道道痕。”

“方源能有多少天道道痕?他主持炼蛊,当时可是最接近这些天道道痕的人。”

赵怜云叹息一声。

她无从想象,也找不到答案,但她知道,自己和方源的差距又拉大了无数。

ps:众筹活动暂告一个段落!目前已经基本筹集了白银大盟的资金。多谢大家的支持,万分感激!至于名单,因为统计量太大,所以要过一段时间才能公示出来。

这一次众筹活动,举办的比较仓促。但大家太棒了,仅用了一两天就达到了。蛊仙战队的排名竟然排在了第二位,我们蕴含的力量真正发挥出来,连我们自己都害怕!

很多书友询问实体书的筹集,这一次活动筹集多下来的资金,都会用于接下来的实体书众筹。这个活动如果举办,会及时地通知大家,并且会更加正轨,不会这么仓促了。

感谢盟主大梦仙尊护道人的打赏!

感谢盟主冬三巳羊的打赏!

感谢盟主不得与飞的打赏!

感谢盟主风月£的打赏!

感谢盟主巴尔巴罗萨的打赏!

------------

\end{this_body}

