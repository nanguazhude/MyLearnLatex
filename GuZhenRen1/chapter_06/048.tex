\newsection{气绝收血气}    %第四十八节:气绝收血气

\begin{this_body}

气绝魔仙见这蛊仙头上缠着一道头箍,身着一席青袍,身姿挺拔,面容英俊,便心头一动:“我算到的那只气道仙蛊,莫不就是应在此人的身上?”

这念头刚动,气绝魔仙便直接出手!

轰隆一声巨响,一个巨大的白色气流巨手,就向赤心行者拿去。

几乎在气绝魔仙发现赤心行者的同时,赤心行者也发现了气绝魔仙。

气绝魔仙并不认识赤心行者,但赤心行者对于气绝魔仙却是早有天庭第一手的情报。

见白气大手向自己飞来,赤心行者心头一跳,连忙躲闪撤退。

赤心行者乃是天庭蛊仙,诛魔榜的前任榜主,战力高强。他曾经在宿命大战期间,来到西漠伪装成血道魔仙,吸引西漠正道诸多超级势力的视线,为中洲牵扯住了一批很大的敌对力量。

纵观西漠蛊仙界,赤心行者有信心和几乎所有的西漠蛊仙交手。但此时他的对手乃是气绝魔仙,当今世间最强的蛊仙之一。

赤心行者非常清楚,他绝非是气绝魔仙的对手。敌我差距太大,这场战斗的输家一定是他。

所以赤心行者这一撤,撤得十分干脆利落。

下一刻,气绝魔仙便见赤心行者脑后升腾起一团血色光晕,旋即后者的速度骤然变快,轻松躲过他的气流大手。

气绝魔仙轻咦一声,暗自嘀咕:“又是血道?当今天下,似乎血道十分盛行。”

气绝魔仙当然不愿就此轻易地放过赤心行者,起身追赶。

两仙一追一逃,很快就跨越万里之遥。

气绝魔仙惊讶地发现,自己依靠的主要移动手段居然追不上赤心行者,当然也没有拉开差距。

“我的腾挪手段已经被我改良了一番,没想到还追不上此人。此人究竟是谁?八转的血道蛊仙……”

气绝魔仙搜索记忆,渐渐地有了眉目。

他之前搜魂了青辉子,得到许多情报。赤心行者的真实身份并未暴露,但他为祸西漠,又是八转血道如此敏感的身份,有关他的情报早就传播到了宝黄天中,进而全天下的蛊仙都有所耳闻。

又追了一阵,气绝魔仙忽然出声低啸:“你玩哪里逃?”

他酝酿完毕,催起一记杀招。

赤心行者顿感周身气流卷席,严重干扰他的速度。

他心中一叹,只得返身和气绝魔仙交手。

气绝魔仙招式磅礴浩荡,浑白气流翻腾,简直是海啸山洪,威不可当。

真正一交手,赤心行者毫无意外地就落入了下风。但他手段也是不弱,血光弥漫,血花盛开,毫无邪气,反而温阳正派,坚守阵脚。

赤心行者奋力抗争,几个回合下来,全身都被汗水打湿,心中感叹不已:“这气绝魔仙不愧是名垂青史的老怪,实力真的太强了!”

赤心行者感到自己就像是一个小舢板,在气绝魔仙掀起的攻势狂澜之下,只能全力躲闪,拼命拖延时间。

又是数个回合下来,赤心行者已经危如累卵,险象丛生。

这时,忽然听到一声兽吼,一头太古荒兽从沙漠下猛地钻出,向气绝魔仙的后背突袭而去。

气绝魔仙被打了一个措手不及,但他临危不乱,伸手一拍,一记气流大手巨大如山,就要拍中太古荒兽。

但这头太古荒兽就在要被大手拍中的时候,忽然全身绽射华光,变作一只太古飞燕,宛若黑色利箭,射向气绝魔仙。

“原来是变化道蛊仙!哈哈,有点意思。”气绝魔仙微微错愕,一边不得不收手应付,一边口中不吝赞赏。

来援的自然是九灵仙姑。

她和赤心行者一道来到西漠,互为支援。

有了她的支援,赤心行者的处境顿时好转了许多,终于从战死的悬崖边缘挽救回来。

“不可久战!这老魔头实在厉害,我们不会是他的对手。眼下之所以能够抵挡,只是他还不熟悉我们的手段而已。”赤心行者急忙传音。

他和气绝魔仙交手,彻底感受到了当世巅峰强者的恐怖。气绝魔仙之所以没有收拾掉他,只是刚刚重生不久,还未完整的适应过来。真要交手,天庭两仙被气绝魔仙杀死是早晚之事。

“我有办法,跟我来!”九灵仙姑传音,且战且退。

气绝魔仙追杀天庭二仙,一路辗转,来到一处沙漠。

九灵仙姑忽然变作一头太古荒兽,一口含住赤心行者,扎进沙漠底下去。

气绝魔仙冷哼一声:“自作聪明!”

他双手一拍,轰隆一声,天地震动。浑白气流宛若海啸,狠狠地拍打在沙漠表面,将方圆百里的沙漠硬生生地拍下去十几丈的深度!

九灵仙姑闷哼一声,立遭重创。

赤心行者连忙出手,为她疗伤。

九灵仙姑不敢回头迎战,一门心思地向地底转去。

气绝魔仙正要俯冲钻沙,继续追杀,忽然听到一声兽吼。下一刻,一头巨大的太古魂兽从附近的沙漠底层钻了出来。

见到这头太古魂兽,气绝魔仙瞳孔也是微微一缩。

这头太古魂兽形状奇异,姿态狰狞,有着龟一样的壳,四只虎爪,龙尾,蛇颈,人头。披头散发,双眼全是凶光和怒火。

这是一头传奇太古荒兽,并且还拥有诸多仙蛊!

“啊,你找到青仇了。”看到气绝魔仙被太古魂兽缠住,赤心行者惊呼一声。

这头太古魂兽正是青仇。

豆神宫被房家夺取后,青仇脱离了豆神宫,隐匿一角,再无消息。

但宿命大战,豆神宫和帝君城融汇一体,元莲意志出来主持大局。宿命大战后,元莲意志自然将种种情报交到了天庭手中。

秦鼎菱不仅知道房睇长的真正身份,而且还获知了青仇的情报。豆神宫囚禁青仇这么多年,元莲意志对它太过了解了,掌握着搜寻它的秘法。

秦鼎菱得到这个秘法,便暗中命令九灵仙姑搜寻青仇,进而将其收服。

青仇乃是太古传奇,不仅战力出众,而且还对幽魂的一切有着天然的甄辨能力。天庭若是得到它,不仅能对付方源,更能对付魔尊幽魂,堪称是一举两得。

九灵仙姑不久前搜寻到了青仇位置,但苦于实力不足,强行出手毫无自信。此刻赤心行者遭遇气绝魔仙,便让她心生一计,顺势而为,引得气绝魔仙和青仇缠斗。

“真是太好了,照此下去,我们便能坐山观虎斗,或许能渔翁得利。”赤心行者笑了一声。

“青仇不会是气绝魔仙的对手,接下来只有看情况再行动了。”九灵仙姑一边交流,一边拼尽全力,转向地底深处,离战场越远越好。

然而出乎天庭二仙预料的是,气绝魔仙很快就察觉到了不妥,张口吐出无边气流,弥漫天地。

他投身其中,很快就消失不见。

青仇没有了对手,也选择了远离战场,逃出天际。

地底深处,天庭二仙面面相觑。赤心行者忽道:“糟糕!那道血海真传保不住了!”

九灵仙姑楞了一下,这才反应过来:“你真的找到了那份血道真传?”

赤心行者点头,叹息道:“天庭搜集到的线索没有错,关键是气潮卷席,让血海真传露出了端倪。我搜寻了大半个月,终于来到最终地点。但就在那里,遭遇到了气绝魔仙。”

九灵仙姑默然。

另一边,气绝魔仙折返回到了原处。

“果然是在这里!”他侦查一阵,发现气数果然仍旧落在这里,没有转移。

尝试了一会儿后,他成功的得到了这份血海真传。

血海老祖当年布置了数十万个传承密地,地点遍及中洲、南疆等地。其中的真传,却是只有九道。

眼下这一道,已经落入气绝魔仙的手中。

“血道真传……妙哉,有了这个真传,我再对付血道蛊仙将容易数倍。”

“还有这只血气仙蛊,也正合我用呢。哈哈哈。”

气绝魔仙大笑一声。

他之前催动杀招,几乎都是消耗仙材气流。现在每当他获得一只气道仙蛊,他的实力就随之上涨一截。

虽然和历史中的巅峰还有一段距离,但和方源交手的时候,已然提升极多。

------------

今天无更

这一周定为两天一更,明天更新,后天不更。

原因是人祖传自从被打乱节奏之后,就变得超级难写。这些天一直在琢磨这个事情。

强行写其实也可以写,但是就是水文了。以前犯过这种错误,现在不想继续犯下去。水文的确钱会多一点,但书写到大后期,我唯一想的就是保证质量。

优秀的小说,每一个情节,每一个章节都应当是贴合主要大纲的,都应当是紧凑的。水文的话,读者一章章的追更新看不出来,回过头来一看,就会发现有点良莠不齐。

就是这样!

\end{this_body}

