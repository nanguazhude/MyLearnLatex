\newsection{成尊的四个条件}    %第一百一十九节:成尊的四个条件

\begin{this_body}

%1
在这个蛊师的世界中,从现在往前回溯,三百多万年的悠悠岁月当中,有这么十位蛊修,傲立在乾坤之巅,俯瞰万灵,横蔑苍生。

%2
漫漫历史长河中,五域两天曾涌现过无数的天才、豪雄。若说这些人是天上的繁星,交相辉映的话,那么这十位蛊修,就宛若太阳一般,耀眼夺目,横亘长空。

%3
他们的光辉经久不衰,他们的传奇不断流传。

%4
他们就是十大尊者!

%5
从古至今历数下来,分别是——

%6
三百万年前,远古时代:元始仙尊、星宿仙尊。

%7
一百万年前,上古时代:无极魔尊、狂蛮魔尊、红莲魔尊。

%8
三十万年前,中古时代:元莲仙尊、盗天魔尊、巨阳仙尊。

%9
十万年前,近古时代:幽魂魔尊、乐土仙尊。

%10
这十个人都是巅峰,横扫天下,纵横披靡,盖压八荒,无人可敌!

%11
他们的手段威能恐怖,历久弥新,几乎是亘古永存。

%12
他们的智慧和谋略,更是纵横古今,贯穿天地,以万物苍生为棋子。即便是死了,也影响万代。

%13
对于这些人,方源有着天底下最深切的感受。

%14
方源深深的明白,自己和这些尊者的差距。即便他现在已经是天下第一魔头,亚仙尊战力,也没有自信和尊者对战。

%15
尊者横扫天地,除此之外,皆为蝼蚁。

%16
宿命大战给天下人都上了生动的一课,任何一个尊者遗留的手段,能够颠覆整个大局。

%17
唯有尊者才能抗衡尊者。

%18
说句难听的话,方源之所以在宿命大战中成功,只不过是在十大尊者、天意之间的交锋的间隙中,挣扎求生了而已。

%19
对于这些投资自己,将自己充作棋子的尊者们,方源每有一分的赞叹,就有一分挑战之意。每有一分忌惮,就有一分比肩之情。

%20
他们既然能够成尊,我方源为何不行呢?

%21
况且,要追求永生,所谓的九转尊者修为也只是我方源的一个踏板而已啊。

%22
成尊的计划,早已经在重生的那一刻,就已经定好了!

%23
多少年前,方源刚重生回来的那些夜晚,他仰望着夜空的明月,卑微如蚁。而如今,他已经攀升上了苍穹,尊者之位只是他下一步的阶梯。

%24
但如何才能成就尊者呢?

%25
方源不知道,不清楚,不了解。

%26
成尊有大秘!

%27
这个秘密从未流传出去过。

%28
十位尊者似有默契,共同保守了这个秘密。

%29
世人无法得知,只能通过一些共同性质,来强行分析。

%30
十大尊者中,总共有九男一女,说明尊者不限制性别,男女皆可成尊。

%31
尊者战力无敌,但并不全能。巨阳仙尊、盗天魔尊就曾经邀请长毛老祖,共同炼制仙蛊。

%32
王不见王。每一个大时代,都只出现一尊,从未有二尊同世的情形。尊者之间的交锋,顶多是宿命大战时那样,用留下的手段相互对决。

%33
每个尊者都有大气运。有的一出生就有异象,顺风顺水,比如红莲魔尊。有的则初期不显,等到某个时期,这才猛地爆发出来。

%34
似乎每一位尊者在成长历程当中,都会有一位护道人存在。每一个护道人都在尊者的成长历程中,起到了至关重要的作用,或敌或友。

%35
世间公认,每一位九转蛊尊,所主修的流派境界都是无上大宗师。

%36
尊者仍旧受到寿命的束缚,历史上有大量的尊者全力搜寻寿蛊延寿的记载。

%37
十大尊者皆是人族,从未有异人成尊的例子。

%38
然而,光凭这些却是无法真正推算出成尊之秘。历史上最大的一个推算成果,便是三尊说。一言仙预言,未来会有幽魂魔尊、乐土仙尊、大梦仙尊。

%39
这个推算,也没有涉及到成尊之秘的真正内容。并且随着方源摧毁了宿命蛊,这个推算已经不作数了。

%40
那么,成尊之秘究竟是什么呢?

%41
方源迫切地想知道这个秘密。

%42
气绝魔仙的成功例子近在眼前,尊者重生也成为极大可能。

%43
因此,方源应对这个未来,最王道的做法,便是自己也成尊!

%44
方源深深的明白一点:不管他如何吞并洞天,或者争取仙材、道痕,都无法成尊。所以,他来到陆畏因这里的最大目的,便只是成尊之秘了。其余的东西,比如乐土真传、乐土真意什么的,对于如今的方源而言,都只是其次之物。

%45
茶几对面,陆畏因喝下一口凉茶,笑了笑。

%46
他对于方源的这个问题,毫无意外。

%47
他紧接着抬头,将茶碗放下,竖起三根手指头,直言道:“要成就尊者,须得满足四个条件。”

%48
“第一,蛊仙的仙窍本源产出白荔仙元。”

%49
“第二,蛊仙主修流派的道痕,至少有三十万规模。”

%50
“第三,蛊仙的主修流派的境界,必须是无上大宗师。”

%51
“第四,蛊仙拥有前三项条件后,须得突破天道封锁,方能令仙窍本源发生质变,产出九转黄杏仙元!”

%52
出乎方源的意料,陆畏因居然直接坦诚,说出了这个成尊之秘。

%53
方源原本还以为,陆畏因会以此为交易的内容,要求方源付出一些什么。或者更准确地说,乐土仙尊之所以选择方源为继承人,一定是有目的的。

%54
乐土仙尊的图谋是什么,方源并不清楚。

%55
但他已经决定,只要能获取成尊之秘,他也愿意进行某些程度的退让,和乐土仙尊进行合作。

%56
然而,陆畏因并没有因为乐土仙尊的目的,就扣住这个秘密不放,而是直接坦言。

%57
这就好像双方交易,原本是一手交钱一手交货,但其中一方提前把货交给了对方。

%58
陆畏因此举没有令方源放松,反而心生警惕。

%59
陆畏因城府很深,方源曾经和他交过手,还失败了,没有夺回上极天鹰。方源追杀战中,陆畏因和方源等人合力,将魔尊幽魂伏杀,更让方源看到了他的隐忍性情。

%60
这样的一个人,岂会如此不智?

%61
方源仔细琢磨成尊的四个条件。

%62
“前两个条件,合起来便是八转蛊仙并渡过三次万劫。我在气道、宙道、炼道、变化道上已经达到了这个标准。”

%63
“第三个条件,主修流派必须是无上大宗师。这点我没有达到,但炼道、宙道、奴道已经是准无上,相差不大。”

%64
“第四个条件,突破天道封锁?”

%65
方源微微皱眉,继续问道:“天道封锁是指什么?”

%66
陆畏因仍旧为方源释疑解惑:“天道讲究平衡,损有余而补不足。万物平衡,相互制约。木秀于林风必催之,堆出于岸流必湍之。而人道恰恰相反。损不足以奉有余,吞噬弱小,不断学习,自身成长,取长补短,增益己能。”

%67
“天道当然不愿意蛊仙越来越多,越来越强,所以降下灾劫。灾劫便是天道封锁之一。而要成尊,天道降下的灾劫将远超万劫!”

%68
“除此之外,还有寿蛊。寿蛊乃天道之蛊,根本无法炼制。十大尊者即便无敌天下,也深受束缚。尊者时代,天地间寿蛊的产出会越来越稀少。”

%69
“第三道封锁,便是宿命蛊。不过这一点,已经被方源仙友你摧毁了。”

%70
方源思忖:“天道封锁基本上有三种手段,宿命蛊已毁,寿蛊暂且无忧,剩下的便是灾劫封锁,是我要面对的最大难关。”

%71
“可有方法应对天道的封锁呢?历代的尊者是如何做到的?”方源问。

%72
陆畏因便道:“并非没有取巧之途,运道便是其一。历代尊者气运都极其浓厚,他们的运气主要来源于两个方面。第一方面便是天意垂青,第二方面则是人道钟意。”

%73
“三百万多年前,异人势大,干扰平衡,天道便想削除,所以天意垂青,令元始成为仙尊。”

%74
“人道钟意的最佳例子,便是红莲魔尊。他是人道之子,气运浓厚至极,刚刚出生,便迎来天地灾劫。天庭三老不得不出手,为红莲遮挡灾劫。”

%75
方源扬起眉头:“让我想想,人道手段是否也是突破天道封锁的途径?”

%76
方源前一段时间,就是利用人道手段,干扰天道演化,争取出手时间。在这方面,他很有经验。

%77
陆畏因点头:“不错。人道手段也是突破天道封锁的好方法。正因如此,历代尊者都有一个共同点,那就是从《人祖传》中,阅读出心得,开创了顶级的人道杀招!”

%78
又继续交谈了一会儿,面对方源的提问,陆畏因简直是知无不言言无不尽。

%79
方源彻底明白了成尊的标准后,陷入沉思当中。

%80
他仔细思量,发现制约他成尊的只有两点:一个是境界,另一个是突破天道封锁。

%81
后者有章可循,方源在运道上有着煮运锅,人道方面也有手段,只需要不断提升发展即可。

%82
真正麻烦的是境界这一点。

%83
准无上大宗师境界,距离真正的无上大宗师看似很近,其实却很难达到。

%84
若是环境允许,时间充裕,方源自然可以勤修苦练,不断感悟,他相信自己有生之年,总会有某个流派达到无上大宗师的境地。

%85
但现在,尊者重生的阴影已经笼罩了他的前途。

%86
“有什么办法能够让我突破这一点呢?”

%87
梦境?

%88
显然不行。

%89
方源早已经发现,他手中的梦境来源于幽魂魔尊,但最多只能提升到大宗师境界。

%90
真意?

%91
“难道说,乐土仙尊留下的真意,能够令我达到土道无上大宗师?”

%92
好像是察觉到了方源心中所想,陆畏因摇了摇头:“乐土仙尊大人的遗泽,只能令你成为土道大宗师而已。不过,乐土真传中早已有了解决之法。”

%93
方源看着微笑中的陆畏因,眼眸中闪过一抹幽芒:“哦,愿闻其详。”

%94
陆畏因便道:“方源仙友熟读《人祖传》,定然知晓天地秘境之一的元境。其实,只要达到元境,任何人都能成为某一流派的无上大宗师。”

%95
“元境?”方源动容,“你知道它在何处。”

%96
陆畏因双眼骤放精芒:“就在疯魔窟最底层!”

\end{this_body}

