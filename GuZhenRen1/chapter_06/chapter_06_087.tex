\newsection{置之死地而后生}    %第八十七节:置之死地而后生

\begin{this_body}

%1
灾劫持续不断,人祖的骨架开始显露出金芒。

%2
灾劫越发狂暴,人祖的骨架被不断烧灼,变得金光灿烂。

%3
人祖开始挺直腰板,昂首挺胸,忍受万千折磨,浑身骨骼宛若黄金浇筑,直直地站在灾劫当中。正是万劫金骨。

%4
而在人祖的骷髅脑袋上,也逐渐生长出了一个冠冕,便是通天骨冠。

%5
人们所承受的灾难,将成为他们来日的桂冠。

%6
虽然灾劫仍旧持续着,但人祖这时终于有了开口的能力:“我知道,宿命必有安排,灾劫就是它对人的安排!但我能坚持下去,任何看似不可忍受的灾劫,其实对于我来讲,都能忍受得住。”

%7
强蛊惊疑不定。

%8
自己蛊却是恍然:“我明白了。人啊,你是从平凡深渊中走出来的人。所以你能忍受不可忍受的灾劫呢。”

%9
强蛊见灾劫也收拾不了人祖,只好强自镇定:“人啊,你可别赖皮。你有希望蛊,灾劫会一直持续下去。难道咱们之间的赌约,就任凭你这样拖延下去吗?我可不管!灾劫这事,就当你过了。现在该轮到我们吃你了。”

%10
“你想吃我的什么?”人祖叹息道。

%11
强蛊哈哈大笑,指着人祖的胸骨处:“接下来,我要吃你的心!”

%12
人祖微微一震,强蛊的选择太致命了,人若是没有心,该怎么活呢?

%13
“快把你的心都拿出来,让我们吃!”强蛊迫不及待喊道。

%14
人祖苦叹,犹豫了一下,他先将同情之心取了出来。

%15
强蛊直接将同情之心,投入到困境的脖颈中,直接落入肚里去了。肚皮涨大了一点。

%16
困境中,人常常先失去同情之心。

%17
人祖接着又将高尚之心取出来,困境吞了,肚皮涨了不少,有些难以消化的样子。

%18
人祖再将自己原来的本心取出来:“希望蛊啊,快离开吧。我可不想连累你。”

%19
希望蛊寄居在人祖的本心中,却是没有飞出来,它道:“我才不走呢,这就是我的家。人啊,你索性连我也丢进去吧,我并不怪你。”

%20
人祖无奈,在强蛊的催促下,又将本心取出给困境吃了。

%21
困境吃了之后,肚皮又涨大许多。

%22
困境也常常让人失去希望。

%23
没有了希望蛊,困扰人祖一身的灾劫便逐渐消失了。这让人祖压力大减,却又怅然若失。

%24
“快,人啊,把你最后一颗心取出来,给我们吃!”强蛊指着人祖胸膛中的孤独之心。

%25
人祖犹豫为难,孤独之心不仅是自己蛊的寄托之所,更是他最后一颗心。没有了这颗心,人祖的性命也就不保了。

%26
强蛊大笑威胁:“快!你若不拿出心来给我们吃,我们就直接动手,把你整个都吃了!”

%27
自己蛊已被欺骗,它满不在乎地道:“人啊,你给他们就是。我无所谓的,你也不会死!我们是最强大的。”

%28
“给我!”强蛊一把夺过人祖手中的孤独之心,直接顺着困境脖子上的伤口,将其投入到它的肚子里。

%29
这下,人祖彻底没有了心。

%30
他直接栽倒在地上,没有了生息,再也爬不起来。

%31
人祖死了。

%32
“人啊,你死了,我们也自由了。”规矩蛊飞走了。

%33
“没有办法,人遇到了他一生中最大的困境。”勇气蛊、刃蛊等等也接着飞走。

%34
人祖的尸体上,只有自己蛊不断盘旋。态度蛊也想走,但被自己蛊死死拽住,没有得逞。

%35
强蛊欢笑:“哈哈哈,人啊你也不过如此。咦?困境你怎么了?”

%36
强蛊寄托的困境,捂住涨大到极致的肚皮,疼的满地打滚。

%37
孤独极难排解消化,越强大的困境中越显得孤独。

%38
轰!

%39
陡然间,困境的肚皮猛地涨破了。

%40
人祖的孤独之心,还有他的皮、肉凝聚成了一个男孩。

%41
强蛊目瞪口呆:“你是谁?”

%42
男孩叫道:“我就是人祖的儿子——大力真武!”

%43
话音未落,他跳起来,一把抓住了强蛊。

%44
强蛊使劲挣脱,自己蛊趁机飞上来,咬了它一口。

%45
强蛊受伤,虚弱了。

%46
“哪里逃!”大力真武大叫一声,直接一把抓住强蛊,将它按进自己的胸膛。

%47
强蛊落入他的胸口,被关押在了心房之中,怎么也出不来。

%48
“别白费力气了。这是我的勃勃雄心,你是出不来的。”大力真武大笑。

%49
随后,他从困境的尸体中找出了人族的其他几颗心脏,放回到人祖的胸膛中。

%50
人祖又重新活了过来!

%51
……

%52
毫无疑问,人的心乃是动力之源,是人一身的致命弱点之一。

%53
幽魂开创的食道杀招吃心,相比较吃苦、吃亏两招,明显更加优秀。后两招即便击中敌人,也只能在胜败的天平两端增添砝码。而吃心杀招却是致命至极,一旦中了,若无提前防备,几乎便能定局!

%54
这是能致胜的手段!!

%55
方源中招。

%56
安土重山堡防御极其出众,这不假。但是蛊仙流派众多,偏偏食道流传很少,很少有针对食道的防御。所以在宿命大战中,西漠一方的众多仙蛊屋面对天庭两仙联合施展出的——坐吃山空食道杀招,尽数中招,无可奈何。

%57
方源也掌握了一些食道传承,同时他也深知幽魂拥有着深厚的食道造诣。

%58
安土重山堡在方源的组建之下,当然可以防备食道。

%59
然而,眼下的安土重山堡已是被幽魂打破!

%60
即便完整的安土重山堡,也难以尽数挡下吃心杀招,仍旧会让方源中招。

%61
一瞬间,浓郁的死亡阴影笼罩到了方源身上。

%62
他无法挣扎,无力挣扎!

%63
至尊仙窍中的天道似有所感,竟然主动停止演化万劫,全数扩散,束缚方源全身。

%64
方源无法调动手段防御,只能静静等死。

%65
陆畏因见机不妙,连忙调动土道战场,黄沙凝聚如蟒如龙,纠缠在幽魂巨人身上,迅速勒紧,企图禁锢幽魂。

%66
幽魂冷笑,数百只手臂勉强撑住,防御得很消极。他的四只眼眸仍旧死死地盯着方源,绝大多数的精气神都催谷着吃心杀招。

%67
“糟糕!”陆畏因的一颗心猛地沉下去。

%68
气海老祖、吴帅更是心中冰凉一片。他们拼命攻击,却换不回幽魂的一个回眸。

%69
幽魂宁愿重伤,也死死抓住这个战机。

%70
他一定要将方源致于死地!

%71
“我……就要结束在这里了么?”方源脑海还是一片干涸,很少的念头在调动。这是刚刚催动气海无量杀招的后遗症。

%72
幽魂真的太强大了。

%73
不愧是曾经的魔尊!无敌于天下的男人!

%74
那个时代,他杀得全天下一片昏暗,无人敢发出声响,亿万万生灵只能在他的阴影中瑟瑟发抖。

%75
直至乐土仙尊出世,拼搏一生,才为天地万命治愈身心伤口。

%76
幽魂在还未全力发挥的时候,个人战力和周围对手差距不大,却始终把持着整个战局。等到他完全爆发的时候,战力暴涨,立即和周围拉开差距。更让人绝望的是,他的每一次选择都是如此犀利阴狠,立竿见影,不给对手留下任何喘息的机会,更遑论生还的余地。

%77
“不,就算只有一丝希望,我也要尝试到底!”方源无法调度任何手段,不过就算能动用杀招,也防备不住吃心。

%78
除非是有逆流护身印!

%79
方源满脸痛苦之色骤然消失,嘴角上翘,流露出一丝神秘的微笑,极其耐人寻味。

%80
下一刻,他蓦地出声大喝:“你此时不出手,还待何时?气绝!”

%81
幽魂一对眼睛鼓瞪,一对眼睛却同时眯起:“这是……方源故意诈我?不对!”

%82
幽魂的一个头颅回转,正看到气绝魔仙对幽魂动手!

%83
轰!

%84
气绝魔仙催动杀招,头顶上的兮地竟若流星一般,凶猛飞射,重重地砸在幽魂的背上。

%85
幽魂被打得趔趄,差点一头栽倒在地上。

%86
噗噗噗噗!

%87
他浑身上下陡然破开许多大洞,像是漏气一般,无数晦暗的魂气顺着大洞,向外喷射。

%88
不仅如此,幽魂环绕周身的上百只粗壮黑臂,在瞬间掉落下来,宛若枯朽腐烂的树枝。

%89
------------

\end{this_body}

\newsectionindepend{《蛊真人》不是黑暗文}

\begin{this_body} %begin a body

%90
呼……这一段的人祖传终于写完了。

%91
显而易见,这段人祖传,是迄今为止最长的一段。是构思了大约一周的成果。

%92
很难!

%93
我当初修改了不少次,单就意象而言,就得慎重选择。比如困境、强弱、灾劫。困境中包含的苦头、亏、灾劫,这三种意象是经过精挑细选的。在刚开始,这种意象至少有十个。随后的几天,我忍痛割爱删减了大半。最终关口,我又将难处、易处这两个意象去掉了。否则的话,全文会更长。

%94
这段人祖传,我需要呼应正文。所有,有了人祖吃苦、吃亏、忍受灾难、被吃心身亡的几个历程。分别照应了幽魂的食道杀招,以及方源本身的处境。

%95
许多读者朋友应该可以察觉到,这一次的人祖传和正文之间的联络,更多的是一种对比。两条线之间隐隐呼应。比如《人祖传》中人祖遭遇了一生中最强的困境,方源同样如此。人祖垂死挣扎,幽魂如此,方源也是如此,其他人比如王小二还是如此(王小二是个重要配角,大家往下看就知道了)。又比如人祖曾用态度蛊+自己蛊,企图欺骗困境,也是和方源在万年斗飞车那边利用自身意志,企图欺骗幽魂一样。这些东西都是相互交映的。

%96
人祖是《人祖传》中的主角,他身上的蛊虫在他和困境互动的情况下,又有什么样的精彩反应?这一点绝对不能不考虑。在这方面,我至少思考了三天。

%97
还有十绝体的问题,在这个章节中,终于又有可喜的进展——大力真武体诞生了!

%98
等到这段大剧情写完,大家可以重新看一遍。一天一更,大家可能觉得节奏缓慢,但其实顺下来看,会有不一样的发现。

%99
《蛊真人》这本书,你第一遍看和第二遍看,感受是不同的。尤其是这段章节,真正的精彩除了表面上的方源挣扎,对抗追杀,其实还有更多深层次的东西。比如人祖传和各个人物之间的照应,这从整体来看,才能发现这种铺设上面的美。又比如各方势力、各个人物之间的谋算,尤其是气绝、幽魂、方源三者之间……

%100
大家看完今天晚上的第二更,为萧真人盟主的加更,就会更加明白一些了。

%101
最近看到了一位读者朋友的书评,感触尤深。大意是:《蛊真人》这部作品其实是成长性的,起初是偏向黑暗风格一些,但得到了中后期越发大气磅礴。单说这本书是黑暗文,其实是偏颇的。

%102
这个书评触发了我对自身的审视。

%103
《蛊真人》写了六年左右,这么长的时间,的确风格上有所变化。诚如这个书评所言,刚开始偏向黑暗风,其后渐有所变。

%104
一方面,是外部原因,本书从开书一来,就屡遭举报,直至最近也是如此,并没有随着时间流逝而安全,反而越发惊险。国家大局如此,只得转变笔锋。

%105
另一方面,是内部原因。身为作者的我,成长不少。尤其是前段时间断更,有特别的感悟。

%106
写《蛊真人》前期的时候,我凭借的是一腔热血,桀骜和不驯,一门心思写,不管什么成绩和钱财。

%107
起初关注者寥寥,但而后越来越多,反响很大,两极评价极多。

%108
所以,写到《蛊真人》中期,我的耳畔尽是嘈杂之音,仿佛置身热闹街市。种种俗事纷至沓来,令我心境摇曳,心湖浮躁。

%109
写到现在,到了《蛊真人》的后期,因为《人祖传》,正文各种线索太多,令我举步维艰,写的异常艰难。时常有心无力,越感自身之渺茫。

%110
前段时间的断更闭关,拯救了我!

%111
就像是周围的闹市,逐渐逐渐收缩。身边如江海般的人流,越来越少,越来越少,直至仅剩下街口的一张书桌,一盏昏黄的台灯,一个我。

%112
一直萦绕耳畔的诸多声音,也逐渐消散全无,只剩下我砰砰直跳的心声。

%113
闭关的时候,我从未看过任何QQ留言,微信短信,就是一门心思地去想,去写。终于将眼前的瓶颈突破。

%114
其实,人祖传中人祖面临的最大困境,方源面临的最大困境,何尝不是我迄今为止写作方面的最大困境?

%115
如今,我再看《蛊真人》。这本书跨越了六年左右的光阴,遭受许许多多的打击、折磨、诅咒、质疑。你看书评,那些骂声、质疑声、咒骂声从未停止过。

%116
起先我怀揣激烈,随之心绪起伏不平,最终沉淀下来。

%117
只剩下的,是对完成这本作品的渴望和坚定。

%118
不管什么排行,也不管什么收入,不管世外纷杂,只是这个念头,纯粹的创作的念头。

%119
让他们举报吧,让他们质疑吧,让他们不屑一顾,让他们怒不可遏。

%120
我只管写成这部,我也只能管这么多。

%121
最近这段时间,我越发认识到,我个人的渺小,我的能力是非常有限的。

%122
我会就这样写下去,直至迎来《蛊真人》的完结。

%123
九月开始,我暂定一个小目标,那就是每天稳定一更,维持一个月!期间有新盟主,尽量加更一章。

%124
但如果在此期间,我觉得无法写出理想标准中的文字,我还是会选择闭关。届时,这个小目标就只能暂时放弃了。

%125
至于,闭关多长时间我也不清楚,反正要将难关攻克,首先得让自己满意。

%126
你若觉得好,你就支持。不支持,咱绝不强求。

%127
就这样吧。

%128
哦,另外,今天晚上有第二更,是新盟主的加更。同样,也是一个画龙点睛的章节。

\end{this_body}

