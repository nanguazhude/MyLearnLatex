\newsection{吃心}    %第八十六节:吃心

\begin{this_body}

%1
漫无边际的苍云,波起峰涌,浪花飞溅,惊涛拍岸。似海非海,云山雾绕,幻化不休,意象万千。

%2
这是一记杀招。

%3
这记杀招是方源的杀手锏,是他的底牌,甚至可算是他如今最强大的攻伐手段!

%4
正是仙道杀招无量气海!!!

%5
这一招奇!

%6
经过方源的改良后,它不仅以八转仙蛊大气为核心,近十只气道仙蛊为辅助,数十万凡蛊奠基。而且还囊括气道仙材,可以通过消耗仙材气流,助长此招威能。

%7
这一招妙!

%8
它能汲取天下万气,生气、死气、剑气、刀气、暮气、朝气……储藏的仙材气流越多,它的威能就越大。若用它来攻,它就消耗仙元、仙材。若用它来守,它就吞吸外在气流。

%9
这一招威能浩瀚!!

%10
气相留下此招,气相洞天凭此招硬生生撑过数次万劫。万劫……这世间大多数的八转蛊仙,连一次万劫都不敢去渡,拼命拖延仙窍时间。气相洞天却是仅凭着天灵操纵,连番渡过多次万劫。

%11
方源吞并气相的洞天之后,便利用智慧蛊的光晕,赶在宿命大战之前,将气海无量杀招改良!而后,他在大战中,便用此招,硬生生地破解了元始气墙,令天下人,五域蛊仙无比震惊。

%12
元始气墙乃是元始手笔,不只是气道仙蛊,也有气道仙材。气流这个流派的特点之一,就是能够利用气道仙材替代蛊虫,施展杀招。

%13
方源利用无量气海杀招破解了元始气墙,也就汲取了气墙当中的海量仙材气流。

%14
方源破解了第一次,又破解第二次!

%15
在不久前的天庭中,他再次利用无量气海,吞吸了元始气墙中的仙材气流,将后者破解。

%16
连破两次气海,再加上方源之前探索气绝洞天,获取到的大量仙材气流,导致无量气海中的气流储备,暴涨到了一种十分骇人的程度。从而导致此招的威能,也上涨到令方源都感到恐怖的地步!

%17
方源念头调动,果断地打出无量气海杀招!

%18
气海翻腾,一道巨大的白色气柱凝聚而出,从高空压下,直朝肉球巨怪而去。

%19
气柱极其粗壮,这种规模达到了史无前例的程度,即便当年的白相也得叹为观止。

%20
不仅如此,气柱表面还笼罩了一层灿烂的鎏金光晕,给人极其坚硬,霸道至极的感受!

%21
白色气柱还未撞上肉球巨怪,就有无边无形的恐怖巨力,从四面八方涌来,将肉球巨怪牢牢禁锢在半空之中。

%22
轰!

%23
白色气柱宛若天柱一般,撞上肉球巨怪,然后狠狠冲刷。

%24
巨大的轰鸣声让战场边缘的雪儿,一时间都为之失聪。

%25
气柱刺眼至极的白色光辉,让躲在冰晶大阵中的雪儿都不得不紧闭双眼,防止双眼被强光刺瞎。

%26
肉球巨怪无法抵御气柱的冲击力量,被直接压在了小北原的草地上。

%27
草地上的青草连带着一层地皮,早就在气柱轰临之前,就被无比的狂风卷走。

%28
肉球巨怪被气柱重重地碾压在草原大地上!

%29
大地颤抖,上百道巨大的裂缝成形,从肉球巨怪的身下地面迅速蔓延,宛若蛛网。

%30
白气巨柱还在持续,崩山裂海!堂堂万劫所变的肉球巨怪被碾压到了地下去,竟孱弱如鸡仔,无力抗争。

%31
一丈、两丈、三丈……

%32
随着肉球巨怪越陷越深,它的身躯也越缩越小。它被白色气柱冲刷,被不断地消灭!

%33
巨怪表面无数的黑洞死死闭合起来,似在拼命蜷缩,一阵阵凄厉的哀鸣声从巨怪身体内部传出。

%34
这是天意在哀嚎!

%35
白气巨柱足足持续了二十个呼吸的时间,这才缓缓消失。

%36
高空的云海不仅缩减了八成,而且剩下的云海变得稀薄无比。方源立即将其调走。

%37
而肉球巨怪则更加凄惨,原本体型如山,此刻只剩下龙宫大小。

%38
肉球巨怪一动不动,好似死尸躺在洞底。而洞底和地面足有百丈差距!

%39
方源本体大喘粗气,脑仁子伴随着呼吸一阵阵的抽疼。为了这一招,他费尽了心思,调度损耗的念头太多,超出了脑海承受的极限。

%40
三千多天道所化的万劫,直至此刻,终于是消停了一会儿。

%41
它像是被方源这一击凶狠的反击给打懵了!

%42
良机难寻,方源连忙加紧力道,利用自在天痕杀招炼化天道道痕。

%43
短短片刻,方源最新炼化的天道道痕已经破百。和之前的成果一起算,方源已有一百六十多道天道道痕!

%44
当方源炼化的所有天道道痕数量正式突破两百时,天道终于反应过来。洞底的肉球如烟般消散,天空中开始下起了冰雹。

%45
然而,方源原本可以乘胜追击,多做应对,但此刻他却只能勉强应付,将注意力重新放在外界。

%46
安土重山堡损伤惨重,破败不堪,幽魂巨人在这段时间不断轰击,成果极大!

%47
幽魂誓杀方源,即便陆畏因等人不断纠缠围攻,幽魂巨人都不惜硬顶着攻势,也始终保持着对方源的凶猛火力。

%48
方源此刻拼力修复,但难挡局面轻颓。

%49
轰。

%50
安土重山堡挡不住,终于被打出一个破洞,内外联通起来!

%51
透过这个小小的破洞,幽魂巨人四只巨眼立即迸射出冰寒冷酷的杀机,因为他看到了屋内盘坐着的方源。

%52
他看到方源刚刚施展了无量气海杀招后的脸色。

%53
方源状态不佳!

%54
幽魂心头一动,无以伦比的丰富战斗经验让他立即决定,和方源一样,悍然动用压箱底的手段。

%55
仙道杀招吃心!

%56
“此招是我晚年方才从《人祖传》中领悟而出。方源,一切都结束了。眼下的安土重山堡,绝对抵挡不住这一招的。”幽魂杀意盈胸。

%57
果不其然。

%58
下一刻,方源闷哼一声,脸色扭曲,痛得他捂住心口。

%59
心中剧痛,仿佛是被猛兽一口咬中!

%60
《人祖传》第五章,第三十三节有载

%61
人祖盯着骷髅头颅道:“接下来,我要吃困境身上最重要的部分。”

%62
强蛊大笑:“这可是昏招!人啊,你的这个选择很愚蠢。”

%63
困境抖擞身躯,体格剧烈缩小。与此同时,无数灾劫从它的脖颈中飞袭人祖。

%64
灾劫如火,煅烧人祖的骨骼。灾劫如锤,将人祖敲打得骨屑翻飞。灾劫如风雨,卷席人祖飘零孤单。灾劫似电雷轰闪,不断劈打人祖。

%65
恐惧蛊大叫:“天哪,这太可怕了,这样下去人必死无疑啊。”

%66
勇气蛊则鼓气道:“人啊,别怕。”

%67
自己蛊和态度蛊联合在一起呐喊:“我很强大,我很强大,一切的灾劫都是毛毛细雨,微微小风!”

%68
弱蛊和背叛蛊、恐惧蛊飞到了一起,想要带着人祖逃跑:“人啊,快跑吧。你可千万不要信自己蛊的话,它已经被骗得太彻底了。”

%69
人祖被困在灾劫之中,无法逃脱。

%70
灾劫笼罩时,不是你想走就能走的。

%71
“太可怕了,我们走吧,别管人了。”弱蛊、背叛蛊和恐惧蛊三个就一起飞走了。

%72
“你们这些叛徒!”自己蛊愤怒不已,抓着这三只蛊分别狠狠咬了一口。

%73
三蛊受了伤,却没有被自己蛊阻挡,撤离逃离了人祖。

%74
自己蛊还想要去追,但这时被规矩蛊相劝:“不要追了。灾难中因为弱和恐惧而逃窜,背叛自己和勇气,只会走上死路。它们跑了不是很好吗?弱蛊和背叛蛊派不上用场,至于恐惧蛊……如果灾劫没有出现,那恐惧是徒劳的。但眼下灾劫已经发生,恐惧只会增加人的痛苦。”

%75
“你说的很有道理。”自己蛊被劝住了。

%76
希望蛊闪着光,支持着人祖:“人啊,只要你在灾难中心怀希望,一切就都还有希望。”

%77
强蛊却大笑不止:“人啊,你真是可怜啊,轻易地听信希望。你难道不知道吗?灾难中光凭希望,只会让灾劫持续不断,正是希望延续了人的痛苦。当你不抱任何希望,反而能在灾劫中感到轻松。”

%78
人祖始终闭口不言,艰难地抵挡灾劫的打击。

%79
不管他的骨头身躯如何弯折,也始终没有倒下。

%80
强蛊很快发现了原因在人祖的脚踝上,长了一对独立的翅膀。翅膀虽小,却顽强扇动,始终让人祖独立。

%81
强蛊继续打击人祖:“人啊,你倒是开始令我敬佩你了。可惜你快要支撑不下去了,你看你的骨头,都已经被灾劫折磨得裂缝连连。”

%82
人祖沉默。

%83
原本人祖的骨头光滑温润,一丝裂缝都没有,但现在骨头架子面目全非,裂痕漫布,似乎下一刻就要彻底崩碎成渣。

%84
其他的蛊虫都缄默,唯有希望蛊发出了光:“人啊,别放弃!灾劫并不可怕,可怕的是人没有了希望。”

%85
希望蛊的光,透过人祖的骨缝,照射到骨髓里。

%86
世间万物在生命的旅程中,皆会产生裂痕,但那正是希望的光照射进来的地方!

%87
人祖的骨缝吸收了希望的光,竟开始吞吸周围的灾劫。

%88
每一条裂缝就像是一张大嘴巴,裂开的幅度越大,好似嘴巴张的就越大,吞噬的灾劫就越多。

%89
“怎么会这样?”强蛊吃惊不已。

%90
人祖的骨头架子吞食的灾劫越来越多,骨架竟随之越变越坚硬。

%91
灾劫能毁灭人,也能磨砺人!

\end{this_body}

