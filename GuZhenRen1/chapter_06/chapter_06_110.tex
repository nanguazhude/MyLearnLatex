\newsection{地藏}    %第一百一十节:地藏

\begin{this_body}

%1
当白凝冰勉强睁开双眼,首先看到的是一片柔和的黄光。

%2
“你也醒了。”耳畔似乎有熟悉的声音传来,但听不真切。

%3
几个呼吸之后,白凝冰的视野变得清晰起来。

%4
她看到了身边的妙音仙子,以及黑菟姑娘。

%5
白凝冰浑身无力,勉强用手掌撑起上半身:“这是哪里?”

%6
她环视周围,发现自己正身处在一个洞窟之中。洞窟非常奇特,形状若球,白凝冰等三仙就在球底。

%7
球形洞窟的内壁,是大片大片土黄色的稀疏土壤。时不时,有一阵阵的红光闪现,在黄土中凝成一片好似血脉状的朱红流光。

%8
“等等,这些是……”白凝冰的神情开始惊疑起来。她发现球形洞窟的内壁中,每隔一段距离,就镶嵌了一份仙材。

%9
她首先发现的是,正前方的一块铜材。

%10
铜材刚硬霸道,插在土壤之中,闪烁着冷硬的金属光泽。

%11
“这莫不是七转仙材霸铜?”白凝冰辨认出来。

%12
然后,她旋即又发现就在霸铜不远处,有一片洁白如雪的铁块。

%13
铁块的气息比霸铜还要高出一筹,赫然是八转仙材白雪铁!

%14
“怎么会这样?霸铜产出之地,不会再有其他七转铜材。但我只粗略一览,这里的七转铜材不少于五份。”

%15
“而且白雪铁向来是雄鸡铁矿中凝聚而出的,怎么在这里独独生出一份白雪铁?”

%16
“看来白凝冰你也辨认不出此地的蹊跷啊。”黑菟姑娘失望地道。

%17
妙音仙子微笑:“我们也只是比你提前苏醒了片刻,对于这里也是非常疑惑。这处地方太奇妙了,好像是仙材的储藏室。不仅是金道仙材,还有其他流派。你看那边。”

%18
妙音仙子手指着白凝冰的左方。

%19
白凝冰顺着指点望去,只见左边的一小块土壤里,生长着十几个大圆萝卜。

%20
“我知道,这是白酒萝卜,萝卜内生出白酒。虽然只是六转,但是食道仙材,相当罕见啊。”白凝冰道。

%21
“再看这里。”妙音仙子又指点。

%22
白凝冰发现了一团风。

%23
这风凝聚一团,线球大小,通体墨绿作色,风声低沉得很。

%24
白凝冰曾经获得过白相洞天,跟着方源身边很久,也算是见多识广。她见到这团风后,眯了眯双眼,想起了它的记载。

%25
“这是幽沉风,八转的风道仙材,只有在地底深处才能见到,就算是天庭蛊仙也对其渴求。”

%26
金道仙材霸铜、白雪铁等,食道仙材白酒萝卜,风道仙材幽沉风,除了这些之外,白凝冰等人又发现了赤蛇铁、兰花生气、睡梦土等等。

%27
“这是什么?”白凝冰等人在头顶上方,发现了一只小狐狸。

%28
它毛茸茸的狐狸尾巴,和后肢两腿,都被困在土壤中,见到白凝冰等人发现了它,它害怕得缩成一团,发出呜呜的轻鸣。

%29
三女仙瞪圆了眼睛,呆愣了一会,齐声道:“碧落狐烟婴!”

%30
这是八转变化道仙材,《人祖传》中都有记载。

%31
“这到底是什么地方?”

%32
“碧落狐烟婴明明已经等于绝迹了,没想到这里还有。”

%33
三仙很快又发现了紫府土。

%34
这是八转仙材,土壤呈现紫色颗粒的状态,表面升腾着袅袅紫烟。紫烟只有半寸高,在紫烟的顶端,烟气又重新凝聚成更细小的颗粒,洒落到下方的紫府土上。

%35
“这种紫府土绝对是完全绝迹的。因为效用太过厉害,在历史上也只是惊鸿一现。”妙音仙子肯定地道。

%36
紫府土效用非凡。一缕紫烟直接放进空窍中,就能够提升修为。而紫府土放进仙窍中,只有份量足够,同样能提拔蛊仙的修为!

%37
三仙粗略搜寻了一番后,发现这里的仙材种类极多,虽然每一份的量稀少了些,但总体价值仍旧十分巨大!

%38
“这里最多的仙材,就是晶精。土道晶精、水道晶精、雷道晶精等等。莫非这里是传说中的天地秘境——乎地?”黑菟姑娘猜测道。

%39
妙音仙子摇头:“虽然《人祖传》中记载不详,但乎地既然能和兮地并称,显然也应当是和气绝魔仙手中的兮地相差不多。”

%40
白凝冰微笑:“真是奇妙又神秘!这里居然还有天晶,天晶不应该是九天的产物吗?我们此刻应当是在地底深处啊。”

%41
三仙不断钻研,很快就推翻了之前的结论。

%42
“原来最多的仙材,并非是晶精,而是埋藏在土壤中的大片地血经脉啊。”

%43
每隔一段时间,地血经脉就会闪烁流光,好像是大地中生长、蔓延的一片血管丛。正是因为它,白凝冰才发现之前的朱红流光。

%44
顺着地血经脉,三仙发现了源头,也看到了他们最重大的发现。

%45
一颗小巧玲珑的玄黄小心脏。

%46
以它为源头,分散出网状的大片地血经脉。

%47
三仙面面相觑,纷纷摇头,辨认不出这个玄黄小心的来历。

%48
“但很明显,这是运道仙材。”

%49
“它高达九转,是这里价值最大的仙材了!”

%50
“这里没有任何人为的痕迹,大自然的鬼斧神工,有时候能叫人瞠目结舌。”

%51
“天地浩瀚,真是神秘有趣,精彩绝伦。我辈蛊仙若活不出精彩,真的等若是白活一场,有什么意义?”白凝冰口中赞叹不绝。

%52
妙音仙子见到白凝冰这副模样,心中无奈叹息。

%53
白凝冰的性情是专注自我,追求精彩,真的不顾及生死,更别提什么道德伦理。

%54
妙音仙子原本对白凝冰非常反感,但之前一战,若非白凝冰拼死激斗,妙音仙子和白兔姑娘现在都已经成了天庭的阶下囚。

%55
一旁的黑菟则道:“太好了,有了这些仙材,一定对主上大有帮助。”

%56
她口中的主上,自然是指方源。

%57
妙音仙子看了她一眼,心中再次无奈叹气。

%58
她知道:白兔姑娘其实和白凝冰半斤八两,也是正常人。白兔姑娘修行的真传,诡异绝伦,乃是极品的双流派兼修法门。

%59
白兔姑娘光暗两道兼修,取极光为暗,暗极生光的理念。当白兔姑娘的修为是六转时,一旦转变,黑菟姑娘就是七转修为。当白兔姑娘拥有七转修为时,黑菟就能拥有八转修为!

%60
但是这个传承,对修行者的神智情绪,有着巨大的影响。

%61
白兔姑娘单纯善良,而黑菟狠辣无情,两者各位极端。不仅如此,一旦认可和崇拜之情,在白兔、黑菟的内心深处积累而成,就几乎刻印下来,难以改变。

%62
方源追杀战后,黑菟姑娘对方源更加崇拜了。因为魔尊幽魂都被方源伏杀!

%63
听到黑菟要将仙材奉献给方源,白凝冰顿时面沉如水,双眼透射出危险的光,向前迈出一步:“你居然有这样的愚蠢想法?”

%64
“你说我愚蠢?”黑菟扬起眉头,怒顶道,“真以为我怕你?”

%65
“两位!”妙音仙子连忙站在两人中间,“我们逃到这里,多么不容易。拼尽全力得到的性命和自由,不是用于内斗的。我们更该考虑的是,接下来怎么办!”

%66
白凝冰、黑菟被劝住,齐齐冷哼一声。

%67
三仙继续探索。

%68
他们挖通一小块内壁,透过小洞看到外面狂暴的地气,仍旧在汹涌翻腾。

%69
“我们还在地脉当中呢!”

%70
“怎么办?”

%71
“单凭我们的实力,就算联手闯出去,风险也极大。”

%72
这个发现,让三仙陷入忧愁困扰之中。

%73
好在这里并不禁止和外界的交流。

%74
三仙顺利地沟通宝黄天,从中迅速了解到,方源追杀战的后续,以及最近这段时间,东海风起云涌。

%75
黑菟大喜:“主上终于摆脱险境,成为了天下第一魔君!”

%76
妙音仙子则有些担忧:“没想到天庭居然还有神帝城这张底牌。强如气绝魔仙都栽了。唉,天庭的底蕴真的太过深邃了。”

%77
黑菟冷哼一声:“区区一座仙蛊屋,算得了什么。手段固定,改良困难。只要主上试探出了神帝城的所有手段,破解这座仙蛊屋根本是轻而易举的事情。”

%78
黑菟双眼放光,用坚定的语气道:“我要回去,主上已经放出话来,召回下属。我要立刻赶回去,继续追随主上!”

%79
妙音仙子点点头,认可黑菟的想法。

%80
她知道,五域界壁消失,宿命蛊彻底摧毁,将来必定是一个乱世。要在这乱世中生存,最好依附于强者。

%81
妙音仙子当然也知道方源的无情和冷酷,但她现在已经没有第二个去处了。

%82
和黑菟相比,妙音仙子对方源的忠心很有限。

%83
她想的更多的是她自己。

%84
所以,当初紫薇仙子俘虏了妙音仙子之后,妙音后来也是配合的。

%85
但现在情势不一样了,方源成为公认的天下第一魔仙。就连天庭都被他隐隐压入下风!就算妙音仙子投靠其他势力,谁敢收留她?不怕被方源知晓后算账么。

%86
白凝冰对着二女冷笑:“看样子,你们还想跟着方源?呵呵,你们俩到底有没有脑子?之所以沦落到这种境地,是谁造成的?不正是方源将你们抛弃了么。他能够抛弃你们一次,当然能够抛弃你们第二次、第三次。”

%87
“我们为何总要依靠他人?眼下就是我们获得自由的最好良机!我们和方源的盟约虽然还在,但已经被紫薇仙子出手压制住。世人都以为我们战死了,就连天庭都被蒙在鼓里。你们居然还想回去?”

%88
妙音仙子叹息:“白凝冰你别忘了我们的身份。我们是被通缉的魔道蛊仙,名列诛魔榜上,五域正道捉拿,人人喊打。不投靠方源,我们还能怎么办?单打独斗,总会有一天被人发现,引出一波波围剿追捕的蛊仙强者。”

%89
白凝冰哈哈大笑:“逃生的话,一定很精彩。你难道一点都不期待么?”

%90
妙音仙子大翻白眼。

%91
黑菟冷哼一声,直接道:“就算是方源主上再次抛弃我,我也是心甘情愿!”

%92
白凝冰、妙音仙子双双无语。

\end{this_body}

