\newsection{都要杀方源}    %第六十八节:都要杀方源

\begin{this_body}

%1
三方混战当中,紫薇仙子一伙以及气绝魔仙忽然现身参战,顿时将情势变得更加复杂。

%2
“你们去抵挡天庭。”魔尊幽魂公然下令。

%3
紫薇仙子、正元老人纷纷出手,杀招宛若光潮,震慑人心,卷席秦鼎菱等人。

%4
秦鼎菱等人除了气海老祖之外,都位于仙蛊屋中,倒也不惧幽魂一方的忽然增援。

%5
只是天庭众仙俱都气愤至极,他们不是愤怒紫薇仙子、正元老人的叛变,而是痛恨魔尊幽魂的奴役手段。

%6
对于紫薇仙子、正元老人的忠诚,天庭上下都十分信任。这两人之所以叛变,天庭诸仙都明白完全是魔尊幽魂的原因。

%7
魔尊幽魂呵呵一笑:“自己人对付自己人,这种感觉如何?能否救下他们,就看你们的手段了。这样的机会今天错过了,将来恐怕不会再有。”

%8
天庭诸仙闻言,顿时攻势又凝滞了几分。

%9
秦鼎菱细眉如剑,狠狠蹙紧,厉声下令:“幽魂是要让我们投鼠忌器,他要拖延时间。不必顾惜紫薇、正元的性命。我们杀了他们,就是为他俩解脱!”

%10
关键时刻,秦鼎菱展现出了首领的大局观,以及果断坚韧的品质。

%11
魔尊幽魂闻言,也不由微微扬眉。

%12
他的动作越发又快了几分,再次扑向青仇。

%13
青仇和其交战,很快就落入下风。

%14
“阻止他!”眼看着魔尊幽魂又要故技重施,从青仇身上汲取力量,秦鼎菱连忙求援,“还请老祖出手!”

%15
不消她提醒,气海老祖已经扑向魔尊幽魂。

%16
但下一刻,一道气流喷涌而出,宛如天河澎湃,拦截住了气海老祖。

%17
气海老祖止住动作,凝神看向出招的人:“气绝!”

%18
气绝魔仙微微一笑,拦路道:“气海,上一战被天庭搅局,意犹未尽。这一次就让我们公平对决。”

%19
天庭诸仙心头猛沉。

%20
关键时刻,气绝魔仙竟然是魔尊幽魂的援手!

%21
双方究竟达成了什么协议?

%22
紫薇仙子也感到惊喜,气绝魔仙绝对是意料之外的强援!

%23
她眼中紫芒一闪,便有了顿悟:“很长一段时间之前,气绝魔仙就开始在宝黄天中大肆收购传承,涉及各个流派。他是上古时期的人物,因为宿命蛊被摧毁而重生。借助这些传承,能让他迅速跟上当今的大时代!”

%24
“气绝之所以支援我等,很可能是主上和气绝达成了交易,贩卖了大量的传承给他。”

%25
紫薇仙子猜得很准。

%26
气绝魔仙虽然没有在宝黄天中直接公开身份,但他也并未有多少掩饰,动向和意图都昭然若揭。

%27
天庭能够发现气绝魔仙的举动,幽魂又怎能发现不了呢?

%28
因为历史缘由,气绝魔仙和天庭是无法合作的。但幽魂也是魔道,更掌握了海量的传承,正是气绝魔仙当下最理想的交易对象。

%29
“方源,今日你在劫难逃了。”魔尊幽魂冷笑,再次突进到青仇的背上。

%30
下一刻他悍然下手!

%31
青仇嘶吼咆哮,但力量再次被魔尊幽魂疯狂汲取。身为太古魂兽,即便是有准九转的仇恨蛊,青仇也被魔尊幽魂克制得死死。

%32
青仇气势急剧下滑,魔尊幽魂则精神振奋。

%33
他抽吸完毕,狠狠一踢,就将青仇直接踹落下去。

%34
随后,他闪电般扑向方源。

%35
仙道杀招——纯梦求真变!

%36
原本围绕在龙宫周围的梦境,陡然缺失一大块,化为纯梦求真体,迅速飞入魔尊幽魂的仙窍之中。

%37
魔尊幽魂竟然身怀充足的仙蛊,可以催发纯梦求真变杀招!

%38
他之前故意放任方源,明明有手段却不急着施展。这一下,他暂时解决了青仇,又用援手挡下天庭、气海老祖一方,再次对方源下手!

%39
群仙动容。

%40
整个战局似乎从始至终,都在魔尊幽魂的掌控之中。

%41
历史上,幽魂魔尊以嗜杀闻名,但并不是说幽魂解决问题的方法只有一个字——杀。他同样冷漠、狡诈、阴险,只是生前的时候不需要他动用种种手段,只凭本身实力就能碾杀一切不服。

%42
如今魔尊幽魂的实力,和他的敌人相比,并没有太大差距。所以,他也不吝动用种种手段,增添自身实力。

%43
一时间,大量的梦境被魔尊幽魂汲取,重新夺回。

%44
紫薇仙子等人力拼天庭,而气海老祖和气绝魔仙对战,处于下风。

%45
气绝魔仙头顶天地秘境“兮”,着实克制同一流派的气海老祖。

%46
情势危急,方源再不能凭借梦境来阻敌。

%47
龙宫大门忽然敞开,飞出数位纯梦求真体,突破梦境防线,扑向魔尊幽魂。

%48
砰砰砰。

%49
纯梦求真体自爆,化为梦境弥漫半空。

%50
魔尊幽魂连忙闪躲,并没有被梦境笼罩。

%51
但趁此机会,吴帅催动龙宫,已是从梦境的另一侧的漏洞处钻穿。万年斗飞车紧随其后。

%52
两座仙蛊屋跑路的意图昭然若揭。

%53
“哪里逃?!”魔尊幽魂冷喝一声,飞速绕过梦境的阻碍,再度逼近。

%54
方源是他的首要目标,必杀之人。

%55
眼前的良机太难得了,方源逃到哪里,魔尊幽魂便决定追到哪里。

%56
魔尊幽魂紧紧盯着龙宫和万年斗飞车,眼中闪烁着的尽都是志在必得的杀机!

%57
下一刻,龙宫、万年斗飞车忽然分散,向着两个不同的方向撤离。

%58
魔尊幽魂楞了一下。

%59
这可怎么办?

%60
万年斗飞车、龙宫藏身在梦境中一段时间,内外隔绝,如今的魔尊幽魂根本不知道方源藏身在哪个仙蛊屋中。

%61
魔尊幽魂只有一人,要同时追两座仙蛊屋,显然是分身乏术。

%62
但魔尊幽魂几乎瞬间做出了决断,他呼喝道:“气绝魔仙,你来追杀万年斗飞车!事成之后酬劳少不了你的。”

%63
气绝魔仙嘿了一声,用兮地吸摄了气海老祖的一记杀招,轻松抽身,追杀万年斗飞车去了。

%64
而魔尊幽魂仍旧负责龙宫。

%65
紫薇仙子亦得到命令,徐徐而退。

%66
“我们怎么办?”天庭蛊仙询问秦鼎菱。

%67
秦鼎菱看来一眼紫薇仙子等人,几乎不假思索:“先杀方源!”

%68
但刚刚困扰魔尊幽魂的难题,同样困扰天庭一伙——方源究竟在哪座仙蛊屋中?

%69
“随便猜一个,赌运气?”秦鼎菱眉头紧皱,虽然她不愿意这样做,但天庭缺乏智道蛊仙,似乎也只能这样做了。

%70
“我可以帮助你们。”紫薇仙子忽然停住撤退的脚步,“我有法门推算方源的位置。”

%71
天庭诸仙皆是楞了一下。

%72
“是何法门?耗费多久?”气海老祖询问。

%73
“加上我等相助,只需片刻。”冰塞川的声音传来,众仙转头望去,便见天边飞来劫运坛。

%74
“总算赶来了。”紫薇仙子吐出一口浊气。

%75
长生天的这批人自然是她唤来的。之前紫薇仙子等人是接到魔尊幽魂之命,隐瞒长生天,准备偷偷袭击天庭。但情况变化太快,紫薇仙子在赶来的路上,分析局势后,便有了借助长生天之力的打算。

%76
得到魔尊幽魂的允许之后,紫薇仙子立即联络了长生天方面。

%77
劫运坛飞近,冰塞川再度道:“我家主上说了,这世间下棋的人不需要太多。”

%78
群仙心头一震。

%79
冰塞川的主上当然就是巨阳仙尊。

%80
种种迹象表明,这位尊者图谋深远,还在世间留有深厚的意志和力量。

%81
就眼下而言,方源是最有望成就尊者的一位。巨阳仙僵当然不想这种事情发生。为将来计划,他当然要趁着方源还未彻底强大起来时将其铲除!

%82
秦鼎菱咬牙,陷入艰难的抉择当中。

%83
和魔尊幽魂、长生天联手?

%84
前者乃是彻彻底底的魔道,而后者更是在宿命大战中的强敌,宿命蛊被毁,他们都有份!

%85
和这些人联手?

%86
秦鼎菱只是思考了几息时间,从牙缝中挤出一个字:“好。”

%87
就像冰塞川的传话所言,这个世界下棋的人未免太多了些。方源是注定的敌人,更是人形大宝库,先除掉他,将来再好好对付幽魂、巨阳!

\end{this_body}
\newsectionindepend{致歉和接下来的更新说明(必看)!}
\begin{this_body}
%88
致歉和接下来的更新说明(必看)!

%89
我非常抱歉。最近的更新很不稳定,而且隔三差五就请假。

%90
事实上,我非常理解大家的痛苦。因为我不仅是作者,同样也是读者。当我读到某部网文,隔三差五就请假,我的阅读体验会很难受。我会失望,我会失落,我会恼怒,有时候我甚至会暗骂——这作者真是个傻逼!

%91
然而,事实很残酷。尽管我不想承认,但我不得不说,在最近的更新问题上——我就是那个傻逼作者。

%92
我真的很想稳定更新,甚至加更,爆更!

%93
所以,我曾经做出过加更的承诺。

%94
哪个作者不想稳定更新,频繁更新?

%95
对于网文而言,更新频繁,就意味着人气,意味着收入。哪个想自己的钱袋子过不去?

%96
我是凡人,我是个俗人,我当然也想多赚点钱。

%97
但残酷的事实又告诉我——我能力不够!

%98
我能力真的不够。

%99
我没法做到稳定更新,做到加更,我只能频繁请假,努力构思,然后再一点点的写出来。

%100
有读者会问:那你以前不也有稳定更新的时候吗?

%101
那我只能苦笑地回答这些读者:是,我以前更新是稳定,甚至曾经加更,曾经爆更过。但那是创作的前中期。

%102
创作,越写到后面越难。前期是白纸作画,信马由缰。中期是线索逸散,耐心跟进。后期是最难的,因为要把所有发散出去的线索,都收拢起来,把所有的坑都填了,让所有的人物都有自己当有的结局。

%103
做到这样的程度,后期才是真正的合格的后期。

%104
100万字的后期是难的,200万字的后期是很难写的,300万字那是难上加难。诸位,我们的这个已经600多万字了,难度可想而知。

%105
《蛊真人》这部和大多数的网文不同,通常的网文它的情节都是线性的,一条线顺下来,很长很长。地图不断地换,很多很多。几乎所有前中期的人物,除非是主要配角,越到后期这些人物几乎是消失匿迹。

%106
而《蛊真人》它更像是一个舞台剧,一个大舞台。不管是什么样的人物,到了后期都有机会出场。白凝冰、黑楼兰这等重要配角我就不提了,我举个例子——青辉子。这个人物偶遇气绝魔仙,一度陷入危机绝境,幸好气绝魔仙没有取他性命。事实上,青辉子是本书中期的一个小人物,我曾经一笔带过。

%107
又比如莫家的肥娘子,这个人物在方源仙僵时期出现的。到了本书后期,她又出场了。并且她的登场,影响到了莫利。而莫利和彭达的际遇,又会影响到其他人物。

%108
每个人物都有自己的想法,有自己的酸甜苦辣,他们的每一个动作不管有意无意,又会促使另外的人物的生命轨迹的变化。

%109
我认为这样的一个世界,才称得上有灵性!

%110
所以,蛊真人的情节设计,是很繁琐的,是非常艰难的。尤其是到了后期,五域两天的大舞台!这样多的人物!我每一次动笔,都很谨慎。稍不留意,我就要吃设定,人物崩溃或者逻辑错误。

%111
这还只是正文的难度。

%112
中还有一部分,构思难度比正文还要高出数倍!

%113
不用我说,大家都知道——那就是《人祖传》。

%114
《人祖传》本身它就很难啊,我不仅要创作一个神话,更要这个神话水准高出盘古开天,女娲造人,亚当夏娃的程度。因为后者这些都不涉及哲理,《人祖传》它是包含哲理的,深度更深,可以精读细读下去。因为我设计的对话、情节都要符合一定的道路。

%115
就随便举个例子吧。

%116
炎煌雷泽把仇恨蛊捏死,结果发现仇恨蛊是杀不死的。这个情节很小,但蕴含一个很朴素的道路——冤冤相报何时了,不能用仇恨来泯灭仇恨。要除掉仇恨,得用其他法门。

%117
再举个例子。

%118
人祖炼制财富蛊,他先后用了辛苦蛊、忧患蛊、悲伤蛊。然后愚蠢蛊被他诱骗,丧生了。智慧蛊也被诱骗,但最终逃了出来。最后,人祖失去了双手,才炼成了财富蛊。

%119
这个情节就是对财富的阐述。要赚取财富是很辛苦的,维护财富要具有忧患意识,失去财富会悲伤。愚蠢的人葬身在过大的财富中,而具有智慧的人也常常被财富迷心,但真正有大智慧的往往会最终看清这一点。最后,财富是要用人的双手创造的。

%120
这样的例子还有很多很多。

%121
《人祖传》的难度,我觉得比通常网文的难度还要更高。

%122
它也是越往后越难。因为我要在后期将人祖的十子都写进去,将怎么破坏九天的写进去,将前中期的一些伏笔(乾坤晶壁等)都要囊括进来。

%123
正文困难,《人祖传》更难。而我要做的,是要将两者巧妙地结合起来写。

%124
这是超难的事情!

%125
至少对我而言,是非常困难的。

%126
《人祖传》我要一章一章的写,然后巧妙地嵌入到正文中,给正文点睛。而这些章节的顺序还不能乱,绝不能前后颠倒。

%127
实话实说,我的压力是非常大的。

%128
尤其是最近,经常是睡不着觉,熬到三四点才能勉强入睡。构思情节,构思人祖传想到脑壳都要烧糊了的感觉。

%129
我记得在好几天前,预约了一位老中医看病。结果前一天晚上精神焦虑,睡不着觉,起来整理大纲,到了2、3点困了睡觉。结果第二天起来,早饭都没来得及吃,就乘车赶去医院。路上堵车,导致预约过了,重新挂号非得等到下午才能看到。干脆就再预约吧,预约只能又等到下周。现在中国的医疗资源真的越来越紧缺了。

%130
这种事情我通常不会说出来,也不是好面子,只是觉得一个大老爷们,不屑于卖惨博取同情。

%131
我现在说出来,只是表述现实,希望大家明白我如今是怎么样的写作状态。

%132
其实有很多时候,我也在想,要不就放弃《人祖传》?要不就干脆随便水水文,每天当然能做到稳定更新。文章的品质差点就差点,至少读者不会反感太大。

%133
每天有更新,每天就有稿酬,何乐而不为呢?

%134
我再一想,不能!

%135
有很多读者朋友,好几年跟过来,读到现在,到了后期,我就写这样干巴巴,毫无诚意的文章来搪塞他们?

%136
这不能!

%137
我创作本书的初心,就是要写一个不一样的魔头,不管其他什么东西,就是一个很纯粹的创作欲望。卖得好不好是次要的,钱包鼓不鼓是次要的,吃的饱不饱是次要的。所以我能到本书后期就马马虎虎吗?

%138
那不能!

%139
我若是这么干了,简直就是对不起我过去数年的努力和心血。对不起我当初的志向和初心。

%140
我更愿意看到的是,这本书的后期质量是好的,符合我的标准的,在水平线上的。

%141
大家现在跟读,每天一个章节,其实是看不出好坏来的。因为视线局限了。只有速读一大部分下来,才能觉察到这一部分的水准。

%142
我不想瞧到有这样类似的评论——《蛊真人》这本书前中期都挺好,后期崩了!逻辑错误一大堆,作者随便吃设定,《人祖传》更是写成狗屎!实在叫人失望。

%143
这是本书很关键的时期。

%144
很难。

%145
不仅是我在创作上感到从未有过的艰难,而且每次构思失败,都会让我感到自身的渺小,深深的挫败感。

%146
但我不会放弃,宁求质量,数量、稿酬都放一边。

%147
我希望大家都能多些理解,多些宽容。

%148
实在理解不来,我建议大家放一放。因为说实话,的确这样跟读,会影响阅读体验。我建议大家可以等本书完本再来读,如果能有个正版订阅支持的话,那就多谢捧场了。这本书真的花了我太多心血。

%149
最后,很不好意思,今天还是没有更新。

%150
这两天的构思成果,还是太差了,被我否决掉了。

%151
大纲上我决定还是要变动一下,方源被追杀的情节很繁琐,因为我会采用一个不同寻常的写法。里面涉及到的人物,我还要好好琢磨一下,才能放心地写出来。

%152
我和大家约定一下吧。

%153
以后的更新,只要是晚上八点到八点十分之间没有的,那就代表当天是没有了的。大家就不要再刷,再等更新了。

%154
本书到了后期,更新方面很不乐观,我估计是起不来了。稳定更新都很艰难。

%155
建议大家放一放,看看其他优秀的网文。如果实在要看本书,不妨重头来看。很多读者朋友在五刷、六刷。《蛊真人》这本书还是可以反复看的,这点我还是有自信的。

%156
最近就有不少读者重刷本书,还有几位好心人将看到的错别字、错误,都通过联络我。这些错误,我只要看到都会修改。我准备在最近一段时间,进行一场本书的大规模修订。

%157
我用的拼音输入法,错别字多,这方面我没有做好,会改正的!

%158
说了这么多,很不好意思。我当然更愿意将写这篇文稿的时间,用来构思《人祖传》。不过我看了这么多的书评之后,还是觉得有必要和大家沟通沟通。

%159
多谢大家一直以来的支持。

%160
蛊真人拜谢。

%161
如果大家不支持了,我也非常能够理解,请一路好走,好人一生平安。多谢你曾经的捧场。因为这的确是我没做到位,我能力不够。

%162
到了后期,我尽管心里很想,但我做不到稳定更新了。我只能这样写,才能写出符合我标准的来。

%163
我这样做,可能对不起某一些读者朋友们。但我觉得,作为一个作者,首先不能对不起自己的作品。

%164
《蛊真人》这部作品,就像是我的孩子,我一步步把他拉扯大。我不能在这部作品快要完成的时候,把它搞砸了!

%165
那就真的太可惜了。

\end{this_body}
\newsectionindepend{今天更新了狂蛮传番外}
\begin{this_body}
%166
今天更新了狂蛮传番外

%167
最近有不少朋友向我提了很多宝贵的建议。

%168
有人说:真人,咱们可以先写完正文,然后再补充《人祖传》啊。谢谢你的建议,但其实实施起来很困难。正文和《人祖传》是相辅相成的。如果正文写完,就好像是一座建好的大楼,再向里面塞小房子(人祖传)。最后看起来,会很别扭,破坏行文布局的空间和美感。

%169
这个想法挺好,其实我也早就想过。但若真的这么实施,最后很可能《人祖传》的质量会很糟糕。因为在一个建造好的建筑里搭房子,施展的余地真的太小了。

%170
虽然这个建议不会采纳,但也衷心地感谢你们。

%171
另外有个建议,我采纳了。

%172
有读者朋友说:真人,你不妨在你构思的期间,写一写番外啊,这些番外也很有趣。

%173
我想了想,这个建议真的好啊。这些番外虽然也要考验设定和逻辑,但构思和书写的难度,比正文+《人祖传》要轻松很多很多啊。

%174
所以,我今天补充了狂蛮传第二篇,大家可以去起点,蛊真人,作品相关里看到。这个番外是免费章节,我不会收钱的,大家可以免费看,也算是我这段时间对大家道歉的赔礼了。

%175
另外今天我修改了许多处的错别字。

%176
具体是在第一大章,有2、3、8、9、15、16、17、19、21、22、23、24、25、26、27、29、30、32、33、34、35、36、37、38、39、42、44、45、47、48、50。

%177
今天暂时修改到50小节。未来还会修订更多,希望广大读者朋友重读的时候,将看到的错误尽量汇报给我。谢谢了!

%178
今天更新了狂蛮传番外 (第1/1页)

%179
『加入书签,方便阅读』

\end{this_body}

