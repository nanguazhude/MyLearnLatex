\newsection{乐土深意}    %第九十三节:乐土深意

\begin{this_body}

%1
“什么?”吴帅第一次听到这个消息,不禁微楞。

%2
乐土仙尊的仁慈厚爱之名,谁人不知谁人不晓?他一生都致力于抹平幽魂之殇,没有一记攻伐招式。他的所言所行,都是导人向善,助人为乐。

%3
他帮助的人,何止千千万万?甚至不只是人,还有异人,还有太古传奇。他还多次出手,修复山川河流,为天地经营资源点。

%4
南疆菇人乐土、东海鲛人王庭、苍蓝龙鲸都是得到乐土仙尊的帮助。

%5
甚至中洲的轮回战场,也是当初乐土仙尊不忍看到中洲蛊仙杀戮,而创下的战场杀招,调解纠纷。

%6
而方源呢?

%7
方源杀人如麻,冷酷无情。为了一己私利,摧毁宿命蛊,至人族大业于不顾。

%8
他为了修行,四处掠夺资源。敲诈勒索、埋伏暗算,烧杀抢掠,无所不用其极。

%9
他所到之处,就是鲜血,就是灾难。

%10
而今幽魂阵亡,方源就是货真价实的第一魔头!

%11
乐土仙尊算到方源,并不奇怪。其余九大尊者,也算到了方源。但乐土仙尊居然要让方源成为他的继承人,要让乐土势力都辅助方源登上尊者之位?

%12
这未免也太匪夷所思了一些!

%13
“乐土仙尊是脑壳坏掉了吗?”吴帅反应过来后,第一个念头就是这样的。

%14
旋即,吴帅又想:“难道乐土仙尊,竟是想引导本体向善?这倒是符合他的一概作风。”

%15
没错。

%16
方源是魔头。

%17
刚刚的一仗,传扬出去,必定惊天动地。方源天下第一大魔头的位置是彻底坐实了!

%18
若是让这样的魔头返邪归正,那的确是对天地,对万物苍生的大好事啊。

%19
吴帅惊愕,方源倒是面无表情,没有惊奇的情绪外露。

%20
因为他在之前,和陆畏因沟通的时候,就听到过这番话了。

%21
当时方源回想到的,却是在红莲石岛上的一幕。

%22
乐土仙尊闯到石莲岛上,和红莲意志对话,的确是说过,要抢夺方源这个继承人。

%23
为此,乐土仙尊将悔蛊夺走,安排到了苍蓝龙鲸之中。

%24
正因如此,才有了后来,方源闯荡苍蓝龙鲸洞天乐土,取走悔蛊,释放沈伤的后续。

%25
当时,功德榜中显现终极任务,似有意邀请方源坐镇洞天,接取这个任务,成为苍蓝龙鲸的主人。但方源没有接受这个利诱,而是毅然选择外出,参与宿命大战,摧毁了宿命蛊。

%26
宿命大战之后,陆畏因主动联络上方源,提出辅佐之意。

%27
这两个事情其实是一脉相承的。

%28
方源接受起来,并不是那么困难的。

%29
“我重生以来,屡屡得到尊者真传。这是历代尊者对我的资助,以我为棋子,来摧毁宿命蛊。”

%30
“我原以为乐土仙尊也是同样的想法,没想到他另有深意。”

%31
方源脑海中念头不断闪烁、碰撞。

%32
他此番遭受如此巨大的困境,堪称九死一生。最惊险的时刻,便是幽魂巨人对他施展吃心杀招,方源是真的无力抵挡,只能冒险一搏。

%33
当时他强颜欢笑,伪装完美,成功哄骗了气绝魔仙出手,令后者反叛了幽魂。这才打断了吃心杀招,让他存活下来。

%34
那一刻,方源的大半个身体,其实已经进了棺材里了。

%35
而造成如此惊险的源头在哪里呢?

%36
外灾是魔尊幽魂全力追杀,想要夺回至尊仙胎蛊,以及诸多梦境。

%37
内患是红莲魔尊的暗算,他利用三千多道天道道痕,让方源几无还手之力。

%38
宿命蛊已经毁了,方源已经没有利用价值了。他之前得到多少帮助,此刻就要遭受多大的反噬!

%39
曾经资助方源的尊者们已经抛弃了方源,并且杀心坚决。

%40
在这样的一个大局下,却有一位尊者特立独行,在关键时刻伸出援助之手,拉了方源一把。

%41
乐土仙尊究竟是何深意?

%42
他究竟看中了方源身上的什么?

%43
他凭什么认定方源会成尊?

%44
甚至不惜一切代价,要辅佐方源成就尊者呢?

%45
周围一片都是土褐,地气滚滚。

%46
飞地在浩荡磅礴的地脉中前行着。

%47
“据我所知,三尊说中,下一代的尊者便是大梦仙尊。难道乐土仙尊是觉得我很有成为大梦仙尊的潜质吗?”方源眼神闪烁,打破沉默,试探道。

%48
陆畏因回道:“三尊说深入人心,广泛流传,但如今宿命蛊已毁,一切就做不得数了。很可能接下来的仍旧是大梦仙尊,但也可能不是,对么?”

%49
“方源仙友其实无须多言,将来到菇人乐土一行,接受乐土真意,便一切都清楚了。届时在下必定扫榻恭迎。”

%50
陆畏因微微淡笑,不愿多说。

%51
经此一战,他自信方源定会前来菇人乐土,但绝非现在。

%52
陆畏因深知方源秉性,方源狠辣,但骨子里是谨慎!

%53
就算他陆畏因在此战中贡献这么大,若有机会,方源也定会杀他。

%54
只是陆畏因一直戒备,没有给方源机会。幽魂自爆的时候,他救下吴帅、气海老祖,表达诚意的同时,又告诉方源他有脱身之法,这才让方源将他接入安土重山堡中。

%55
而到了飞地,他又掌握着掌控飞地的手段,始终都有利用价值,让方源不想动手杀人。

%56
若是他昏死,方源绝对会痛下杀手,再搜魂他陆畏因的魂魄,寻得一切内情。

%57
最终,方源选择在半途离开。

%58
他身受万劫纠缠,当下最紧要的,就是解除万劫!

%59
他现在战力低落,若非如此,早已对重伤的陆畏因下手了。

%60
临别之前,陆畏因态度诚恳地叮嘱方源:“方源仙友,之前的处境想必你感触极深。诸多尊者要除你,绝不只是之前的困境。我们的时间是非常紧张的,真实的处境远比宿命大战还要凶险。稍慢一步,就是身死道消的悲惨下场。所以,还望仙友你能够尽快履行诺言,前来菇人乐土一趟。”

%61
方源点点头,带着吴帅离开。

%62
两仙离开地脉,一路往上钻。半个时辰之后,方才钻出地表,出现在一片沙漠之中。

%63
方圆千里,荒无人烟。

%64
吴帅忽笑一声:“陆畏因似乎要有麻烦了。他此次相助我们,本身又统领异人势力,恐怕会遭受南疆正道的联合讨伐。”

%65
方源嗯了一声:“的确会有麻烦。南疆正道中以武家为首,武家首脑武庸乃是当代枭雄,绝不会放过这样的良机。但以陆畏因的手腕,我更信他已有解决之道。”

%66
吴帅再问:“临别前,陆畏因的那番话是否危言耸听?”

%67
方源却是满脸肃容,微微摇头。

%68
吴帅心头微沉。

%69
“走吧。”方源道,“速速觅地休整,我们的确要争分夺秒!”

%70
地脉。

%71
“真叫方源这个魔头跑了!”天庭的仙蛊屋中,秦鼎菱怒目圆瞪,咬紧牙关,双拳也暗自捏紧。

%72
这是她最不想看到的情况,但却发生了。

%73
劫运坛中,冰塞川也发出叹息之声。

%74
他也没有想到会是这样结果。之前的魔尊幽魂是那样的强势,结果却是他阵亡,而被追得四处跑路的方源,仍旧在跑路。最大的意外就是陆畏因,这个乐土传人简直是自绝于正道,是乐土仙尊的耻辱!

%75
当然,还有地脉和飞地。

%76
五域一统,界壁消失,五域的蛊仙界的注意力都集中在了宿命大战上。大战余波未歇,就又有气潮卷席天地,让蛊仙们疲于应对。基本上没有人有这个闲情雅致,去深入地底查探异变。

%77
地脉磅礴浩瀚,宛若滚滚黄河,声势浩大至极。在地脉扇面前行,非常不易。

%78
承载方源的飞地早已经不见踪影,众仙在地面中艰难行进,心中对追上方源的期待越发微小。

%79
众仙目光所及之处,地脉周边,土道资源多得纷杂如星。而在地脉之中,更有高品质的仙材层出不穷。

%80
“天地异变,大地也变得如此陌生了。”

%81
“这些土道资源非常雄厚,可惜要在地脉中收取,十分费劲。除非是有优异的土道造诣。”

%82
“单单我们一路走来所见的土道资源,若是充分利用,绝对能改变一方局势。坐拥这处地渊的古魂门有福了,吸纳了这些,必定势力大涨一截!”

%83
这些都是蛊仙中的最顶层的那一撮人,眼界广阔,高瞻远瞩。

%84
地下的资源如此丰富,影响十分深远。如何高效地挖掘这些资源,然后利用,将会影响整个天下局势。

%85
“快,取出备用蛊虫,又有三百多只蛊虫被毁了。谁来和我联手,将新蛊替换上去?”

%86
仙蛊屋在这地脉中前行,时刻地气剧烈侵蚀,仙蛊屋上刻印下来的土道道痕越来越多。

%87
除非是土道仙蛊屋,其他流派的仙蛊屋根本不能在地脉中待太久。

%88
“方源已经消失很久了。”

%89
“除非我们找寻到一块他们那样的飞行小岛,否则……”

%90
“那陆畏因明显有手段,能够操纵飞行小岛。他可是乐土传人,当年乐土仙尊主修的便是土道。”

%91
天庭一方的蛊仙们虽然没有明说,但是言下之意都有收兵的想法。

%92
秦鼎菱保持着沉默,她很不甘心,最终竟然眼睁睁地看着方源逃跑了!她也知道,方源消失了这么久,想要再追上去,希望十分渺茫,几乎没有。

%93
但她下意识地还是这样做了。

%94
经此一战,让她对方源的忌惮暴涨数倍。魔尊幽魂亲自追杀,方源居然连这都能捱过去,甚至还能设伏斩杀了幽魂!

%95
“古月方源……我还是低估你了。”

%96
“难怪当初,即便是专修智道的紫薇仙子,也一而再再而三地败在你的手中。”

%97
“等等,影宗那帮人呢?”

\end{this_body}

