\newsection{气绝压气海}    %第三十七节:气绝压气海

\begin{this_body}

%1
至尊仙窍。

%2
灾难还在持续,整个局势正在逐步滑落深渊。

%3
方源这一次损失很大。为了挽救资源点,他已经将仙窍内的异人和蛊仙分配到了各处,令他们密切关注,全力照料资源,并且大量散发能专门侦查天意的凡蛊。

%4
再加上方源的宙道分身,以及本体亲自出手,四处剿灭诞生演变出来的天意。总算是勉强保住了至尊仙窍各大资源点。

%5
然而,天道道痕只要不被方源炼化,它演变出来的天意就会绵绵不绝,并且越来越多。

%6
即便方源已经动用了全力,他也无法真正阻止局势变得更恶劣下去。

%7
他只能勉强保住资源点,但是至尊仙窍空间广阔至极,资源点所占据的只是很少一部分而已。在大部分的地区,都有天意产生,就像是密密麻麻的大军,逐渐包围各地的资源点。

%8
方源深知,自家局势已经到了一个关键的时刻!

%9
他已经来不及清缴仙窍中的天意。为了防止洞天中的天意和五域天意勾连,他不得不在气海大阵中单独开辟了一个秘密空间,专门隔绝天意洞察。

%10
而外界大局,气绝魔仙的复活也出乎方源的意料,气绝魔仙扬言要会一会气海老祖等人,这绝不是张口说说的。

%11
内忧外患!

%12
“接下来,就看这一座仙阵的效用了。”方源心神投注到一座律道大阵之上。

%13
这座大阵位于小蓝天之中,凭空而立,乃是由五界大限阵改良而成。

%14
方源拥有智道造诣,早已明白局势会发展到如今地步,因此未雨绸缪,早就开始图谋解决之法。

%15
许久之前,他得到灵感,决议仿造气潮,来对付自家仙窍中的天意。

%16
而这一座律道大阵,便是他的研究成果。

%17
八转仙元早已经堆积一批,储备在了仙阵之中。伴随着方源的一个念头,这座恢弘广阔,覆地十多里的大阵开始缓缓催动起来。

%18
呼呼呼……

%19
大阵发出呼啸的风声,风声开始震荡天地时,无数气流从大阵周围产生,宛若漫天蒸汽,白茫茫一片,伸手不见五指。

%20
大阵忽然绽射玄光。玄光击打到了白气之上,像是一记巨手狠狠一推。

%21
下一刻,大约三成的白茫蒸汽就滚滚开动,向前喷涌而去。

%22
玄光钻入白气之中,不断改造白气,使得这些松散的白气越发凝聚。

%23
终于,被推动而出的白气成功转变成了一股气潮,开始向东南方坚定不移地推行。一路上,巨大的气潮宛若一柄大刷子,将周围的天地二气彻底冲刷一遍。

%24
单论威能,方源制造的这股气潮还是不能和正统相比,但威能已然不俗,能轻易夺取凡人蛊师的身家性命。

%25
这也是为什么方源要将大阵布置在小蓝天中,就是防止大意之下屠戮了生灵,自讨苦吃。

%26
这股气潮足足前行了半个时辰,一路直行,几乎要推行到小蓝天的边缘窍壁。

%27
方源仔细检查成果,发现他之前故意布置的天意,都被剿除得干干净净,一点渣滓都不存在。另外,气潮开赴而出的这条路径上,天地二气也调和得极好,十分平静和谐。

%28
方源吐出一口浊气,结果是可喜的。

%29
有了此阵,他终于能解决掉天意这个大麻烦了。

%30
这对他而言,意义极其重大!

%31
若是解决不了天意,他几乎就得困守在气海大阵中,防止自己的身份和位置暴露在天意之下。

%32
方源心头如山般的压力,顿时骤减一半。

%33
现在他再看气潮,又有了更深一层的理解。

%34
“气潮的产生,根源在于天地二气的差异。就好像是悬崖上下有高低的落差,水流经过就会产生瀑布。气潮也是如此,正是五域界壁消失,地脉一统,存在差异的五域天地二气在交融中,会形成一段时期的气潮。”

%35
“等到将来,天地二气彻底融汇,五域之间不分彼此,也就不会有气潮产生了。”

%36
“而五域界壁又是什么呢?事实上,它的本质也就是气潮啊。”

%37
五域之间天地二气存在差异,因此五域之间交界接壤的地方,天地二气就有落差。但那个时候地脉不同,天地二气无法像现在这样融汇,最终便郁结沉凝成五域界壁。

%38
蛊仙在五域界壁中施展杀招,会遭受反噬。而在气潮中也是一样。究其原因,都是因为自身的天地二气和界壁中存在差异。

%39
而陶铸毕生研究的五域界壁,换句话来讲,就是气潮!

%40
只不过,他的研究途径是从律道着手。而主要的成果之一——五禁玄光气,单从这个杀招名称来看,就和气道脱不了关系。

%41
方源拥有陶铸真传,又有半份元始真传,再加上气相的底蕴,自身气道的境界,如此种种雄厚积累,这才让方源极快的有了可喜的成果。

%42
“接下来就是稍微调整这座大阵,然后在各地布置分阵,使得气潮连贯汹涌,不断剿除天意。”

%43
“天意的麻烦可算解决。但天道道痕的炼化,仍旧是个大问题。”

%44
不过方源接下来,也只得将这个问题暂时搁置一旁。皆因他要防备气绝魔仙的到来。

%45
气海老祖这个身份,非常有价值。方源在宿命大战时,也千方百计地隐藏,从未主动暴露过相关方面的线索。

%46
而要维系住气海老祖这个身份,方源当下的首要障碍就是气绝魔仙了。

%47
方源一面催动大阵,积极平定内患,一面耐心等候气绝魔仙上门挑战。

%48
依凭方源的推算,气绝魔仙在最近这段时间专程过来交手的可能极高。

%49
等候了一段时日,这一天,气海上空忽然发生异变。

%50
苍穹一片白色气浪,浩浩荡荡,滚滚不休,排山倒海一般向气海汹汹盖下。

%51
气浪之中,气绝魔仙的童子身躯若隐若现,魔威滔天。

%52
很快,他老气横秋的声音震荡天地:“气海小辈,老夫气绝魔仙专程为你而来。”

%53
“气绝仙友所来何事?”方源缓缓现身,催动气海大阵,立足海面,仰头而望。

%54
气绝魔仙呵呵一笑:“老夫新近重生,苦无仙蛊傍身。听闻你乃是当世仅存的八转巅峰气道强者,因此特意向你来借蛊。”

%55
“哈哈哈,好一个借蛊。”方源哈哈一笑,他充分感受到了气绝魔仙的霸道,“仙蛊就在我的身上,但你能否借到,那要看你自己的本事了。”

%56
气绝魔仙仰头一笑,幼嫩的小脸上激荡起浓郁的战意:“痛快!小辈你很和我胃口,那就来先接我这一招罢。”

%57
仙道杀招——一气大擒拿!

%58
顿时,一只浑白气流大手飞出,好似小山一般,向方源直接镇压过来。

%59
正是这一招让帝藏生惨痛怒吼,它不仅威能卓绝,而且十分玄妙,能聚能散,让敌对蛊仙很不容易应付。

%60
方源深呼吸一口气,神色平静,临危不乱。他伸出手掌,缓缓向上一推。

%61
轰隆!

%62
一道气墙覆盖方圆上百里,像是苍穹升腾而起,气度极其恢弘。

%63
气墙和大手撞上,大手被撞得不断后退,并不能抵挡气墙的上升。

%64
气绝魔仙微微一笑,伸手一指,气流大手陡然崩解,化为无数琐细气流试图钻入气墙当中去。

%65
但这个尝试也以失败告终。

%66
方源催动出来的气墙,不仅磅礴浩荡,更紧密异常,让气绝魔仙毫无空子可钻。

%67
气绝魔仙这才微微变色:“好一招气墙,此招何名?”

%68
方源坦陈道:“专为气绝仙友草创,还未起名呢。不知仙友可有好的想法?”

%69
气绝魔仙不禁双眼眯起,绽放精芒。

%70
气海老祖草创此招,远比开创已久更加可怕。因为这说明了气海老祖的气道境界十分高超,能够在这么短的时间里就能依凭战况,对自身的气道杀招进行成功的改良!

%71
气绝魔仙却不知道,事实上,方源真正改良的是五界大限阵,气墙杀招不够只是附带罢了。

%72
方源虽然失去了智慧蛊,但智道的手段仍旧是丰富的。

%73
气墙不断上升,将遮天的浑白气流不断挤退回去。

%74
等逼到气绝魔仙面前时,气墙中忽然生出变化,两只大手一黑一白,分别从左右两路向他合攻而来。

%75
气绝魔仙竟不闪不避,任凭他自己被黑白大手团团捏住。

%76
大手开始施压用力,但下一刻,就听见手心中传来一声呼啸:“兮——!”

%77
两只黑白大手顿时被一个凹地的秘境虚影,给吸纳了进去。

%78
这次轮到方源惊愕,他敏锐地察觉到气绝魔仙头顶上的这片天地秘境,似乎比之前还要完善很多。

%79
气绝魔仙心知寻常手段,无法对气海老祖生效,他直接动用了杀手锏。

%80
天地秘境——兮!

%81
方源不断尝试出手,种种气道杀招打到气绝魔仙面前,都会被兮地直接吞吸进去。它就像是一个无底洞,无论方源怎么填充都填不满。

%82
气绝魔仙头顶兮地,占据上风,稳立不败。

%83
他甚至开始悠然点评方源的种种气道杀招。

%84
“这黑白双手构思不错,应当是采用了阴阳二气,利用两者间的相互吸引和消磨,来巧妙地增长此招威能。”

%85
“你的这身气道防御也能入眼。每一丝每一缕的衣线,都是一股股的罡气凝聚而成,你是仿造的天罡气墙么?”

%86
“可惜啊可惜。若是你在我重生之时,还能对我有所威胁。遗憾的是,老夫最近刚刚拆解了乱流海域,令兮地几乎补全。我的这一招,正克天下气道,气海小辈你当然也不会例外!”

%87
“怎么样?老夫有无能力借你的仙蛊呢?”

%88
气绝魔仙之所以能够号称“气绝”,正是因为有这一招,克制了天下所有气道。

%89
任何的气道杀招都会被兮地吞吸,除非杀招威能超过兮地的承受极限。遗憾的是,气绝魔仙几乎补全了兮地,引发了战力上的质变。

%90
从某个方面来讲,他对付方源,比对付掌握龙宫的吴帅还要轻松得多。

\end{this_body}

