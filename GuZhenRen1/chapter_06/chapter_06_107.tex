\newsection{风风火火搞修行}    %第一百零七节:风风火火搞修行

\begin{this_body}

%1
至尊仙窍,小中洲。

%2
一座崭新的宙道仙阵,已经布置到了关键时刻。

%3
“去。”方源神念一动,一大团的昙花精蜜,扑洒上去。

%4
这可是八转宙道仙材。

%5
下一刻,玄光爆闪,化为无数光点,又紧接着猛地扩散。蓬的一声,像是银白色的烟花绽放。

%6
光点迅速消散,毫无迹象,仿佛方源费了大力气搭建的大阵,彻底失败了。

%7
然而,当方源催动侦查杀招之后,他就看到了一个水池虚影,悬浮在半空中。这个水池酷似年华池,但并不大,只有年华池的一成规模。

%8
方源观察片刻,满意地点点头,这一次布阵相当成功。

%9
他旋即念头一动,位于小九天之中八转年华池开闸放水。一股光阴支流,蜿蜒流下,汇入到刚刚搭建成功的分池之中。

%10
当分池的光阴河水积蓄到一定的程度,便开始自行运转,河水透底而出,灌注到正下方的地面。

%11
年华分池的最下方,是一片白色浓雾。浓雾掩盖下的浅滩中,有着大量的荒植——太泽花。

%12
这些花有的呈花骨朵竖立着,大若房屋。有的则绽放开来,花瓣接连一体,形如喇叭,从花心处,不断向外弥漫出丝丝缕缕的白色雾气。

%13
这里正是方源精心营造出来的花雾太泽浅滩。

%14
八转仙蛊悔的食物,便是新鲜的太泽花雾。

%15
光阴支流灌输下去,旋即带动整个太泽花雾浅滩的时间迅速加快。光阴支流的河水继续向周边扩散,但花雾太泽浅滩周围迅速显现一圈浅蓝色的半透明光墙。

%16
于是,时间加速的效果,就只局限在了光墙圈裹的范围内。

%17
“成功了,相当完美。”方源仔细检查了几遍,没有发现丝毫问题,大感满意。

%18
浅蓝色的半透明光墙,是方源从时差洞天中获取的良方妙法。

%19
时差洞天中的各大资源点,就都是这样,被光墙包裹,里面的时间流速比其他地方加快了许多倍。因此造成时差洞天中,宙道资源产量脱俗。

%20
方源在原有的基础上,增添了年华分池,每一个分池负责调控一个宙道分区。八转年华池相当于主坝,而这些年华分池相当于副坝。从光阴长河中引入的光阴支流,先是汇入到年华池中,然后最大的一股,铺散整个仙窍,还有许多分流汇入到副坝中,加速资源产出。

%21
这样一来,方源的资源产出又能往上再翻十多倍!

%22
当然,这个方法不只是用于加速资源产出,还能作用在蛊虫身上。

%23
蛊虫在时间缓慢的宙道分区中,食物的需求就下跌很多。

%24
这个方法在其他蛊仙手中见效甚微,原因很简单——道痕互斥。宙道分区本质上是宙道手段,若是影响的蛊虫是其他流派,并且转数较高,那么效果便会越小。

%25
方源拥有自在天痕,却是可以减少内耗。他尝试之后,发现此法效果喜人。

%26
一方面减少蛊虫的食物需求,另一方面增大资源产出,方源的蛊虫食物危机,还未发作就安然渡过了。

%27
气海老祖走了,但这几天,方源都没闲着。

%28
东海蛊仙战死,遗留在海底的仙窍福地,几乎都已被他吞下。

%29
而太古两天之中的洞天,方源打算暂时留着。

%30
一方面,至尊仙窍需要食物,每隔一段时间吞并一个洞天,就能令灾劫倒计时重新计算,让方源能灵活布置。

%31
另一方面,洞天仿佛盆栽,栽种气功果。尽管希望不大,但若是气功果继续壮大,不仅是对异族大联盟上下的钳制,更能滋补气海分身。

%32
至于方源之前吞下的三个洞天,那是必须的。

%33
方源的仙元太少了,需要大量补充。

%34
八转蛊仙补充仙元,单靠仙元石效率太低,王道方法还是自产。

%35
但是仙窍产出仙元,靠的是仙窍本源。而仙窍本源是否健壮,参考的是仙窍的发展度。

%36
为了提升发展度,方源必须得吞下三个洞天。

%37
宿命大战就消耗了方源大量的仙元储备。而后,方源根本无法停歇,在气潮中四处奔走,寻找对付天道道痕的方法。找到方法之后,方源进入天庭,立马就发生了追杀战。

%38
这一段时间,方源大战太过频繁,每一战几乎都是仙界顶尖的强敌,烈度太大。

%39
到了现在,方源的仙元储备已经滑落到警戒线之下,短时间内真的需要休养生息,不能再动手了!

%40
“多亏了我抢来了年华池,宙道手段不缺,让仙窍时间流速达到能力极限。否则,仙元早就消耗光了。”

%41
“可惜我没有八转的天元宝皇莲,否则不会为此事发愁。”

%42
天元宝皇莲仙蛊能够产出仙元,历史上拥有它的元莲仙尊,乃是尊者中仙元储备最雄厚的人!元莲仙尊从修行开始,似乎就未对真元、仙元的负担担忧过。反观他的敌人,和元莲消耗,常常陷入真元、仙元干涸稀少的窘境。

%43
方源现在缺乏仙元,其实很耽搁他的休养。

%44
他探索梦境,需要仙元。

%45
他修行魂道,需要仙元。

%46
他的仙蛊损失众多,需要抢炼。低转的仙蛊是大量的,需要升炼。受伤的仙蛊,通过炼道手段迅速修复,也需要仙元。

%47
因为仙元缺少,必须留最后一笔以防意外,方源的这些修行都只能暂时搁置了。

%48
唯有复合杀招的构思,因为难度实在是小,所以还能勉强进行下去。试验新杀招,无疑也需要仙元。但单独这一点,倒是在方源能够承受的范围内。

%49
积累仙元、推算复合杀招之外,方源还在做资源点的搬迁,以及仙材的统合工作。

%50
两天洞天中的资源点,能够搬迁的,方源都尽量搬迁过来。毕竟这些洞天留在太古两天之中,防御范围太广,有人要攻打的话,很难及时防御。

%51
海量的仙材被统一整理,收进至尊仙窍的几处库藏之中。方源仿造天庭,建造了几座巨大的明库,也造了几处价值高昂的暗库。

%52
异族蛊仙们受到天意关照,真的很富庶。

%53
其中,木道的仙材最多。因为萧荷尖、青森大圣都是小人,除了他们之外,还有两座木道洞天。

%54
其他流派的仙材当然也不少,但叫方源颇感惊喜的是,他搜刮上来的运道仙材也规模惊人!

%55
比如说运道上古荒兽打滚猫,零零散散的收上来后,方源最后发现竟有好几十头。

%56
这种小猫十分精致,成年人伸开手掌,就能让它在上面打滚。

%57
打滚猫长得十分讨喜,雪白的皮毛,粉嫩的爪子,黑闪闪的大眼睛,叫起来的声音非常娇嫩,喵喵喵……让人听了心头发痒。

%58
还有吉利宝玉,这是八转运道仙材,时刻散发出璀璨的华光,蕴含相当浓郁的运道道痕。

%59
塞牙水也是一种,表面上看去它和普通的泉水没有什么区别,但真要喝下去,就会塞在牙缝之间,分外难受不说,还能令蛊仙的运势暴降,陷落低谷。

%60
“异族大联盟中,并无运道专修。但他们这些人能在人族大势下,开辟世外桃源,隐居生活,繁衍生息,当然是运势浓厚的。更别提他们还受到天意的关照!”方源仔细琢磨,有这些运道仙材收获,也能解释得通。

%61
这些运道仙材对于方源而言,价值很高,来的也正是时候。

%62
因为,他需要升炼大量的运道仙蛊,最终令煮运锅提升一截。煮运锅的层次对于现在的方源而言,还是有些低了。

%63
方源预计,将来还会有一大批的仙材,甚至是仙蛊入账。

%64
因为那些东海蛊仙俘虏,还在至尊仙窍内关押着!

%65
方源的勒索信,已经送给了东海各大势力。

%66
东海正道首脑们接到勒索信后,又都转送给气海老祖,希望听听他的意见,请他做主。

%67
气海老祖刚刚将依附自己的两天洞天,都跑了个遍,吞吸了所有的气功果。

%68
如今,他的气道道痕直接突破了两百万大关!

%69
短短半个月不到,气海分身的实力就有惊天动地的暴涨。

%70
气海分身不动声色,这张牌暂且先藏着,能藏多久就藏多久。

%71
“如果留下来的那些气功果,能够再次生长,就好了。”气海老祖暗自期待,但他却也知道这个可能不太大。

%72
因为气功果的产生,是因为气潮的冲击席卷。各大洞天几乎都坐落在天脉上,方才有了重点郁结。

%73
方源追杀战的时候,五域气潮就已经没有动静了。五域之气的交融程度已近乎完全。

%74
气海老祖也在琢磨自己的修行。

%75
他缺的不是道痕,也不是仙元,而是杀招。

%76
气海老祖的杀招,能拿得出手的就那么几个。虽然得到了元始真传,但上面的气道杀招天庭太熟悉了,必须得加以改良。

%77
气海老祖的野心当然不只是这个程度。

%78
他还想创建出一个适合自己的战斗体系!

%79
杀招零零散散,可以欺负一些八转强者,或者个别的亚仙尊。但是面对魔尊幽魂、龙公这等人物,那就得需要一个优秀的战斗体系。

%80
影响蛊仙战力的主要因素,是修为。但杀招、仙蛊、仙元、天时地利人和等等,也都是影响因素。比如逆流护身印这等特别的杀招,可以令蛊仙越阶挑战。战斗体系当然也算在其中。

%81
对于绝大多数的蛊仙而言,战斗体系并不是必要的。因为单单前面的因素,就足以造成战力方面的差距,令蛊仙分出胜负来。

%82
但对于方源、气海来讲,它却是必不可少的。

%83
因为修为、杀招、仙蛊等等常规因素,不能决定胜负,大家都差不多。这个时候,谁有一个成熟、优秀的战斗体系,那就很重要了。

%84
方源发出来的勒索信,辗转一番后,最终都到了气海老祖的桌面上。

%85
气海老祖看都不看一眼,因为这些信中的内容,他一清二楚。

%86
“老夫伤势不轻,需要闭关休养。”气海老祖一句话,就让东海正道们失去了对抗方源的心气劲。气海老祖都不出头,谁还能正面抗衡方源?

%87
乖乖缴纳仙材,赎回家族蛊仙吧。

%88
气海老祖又关照一句:“老夫已经联络天庭,或许他们会派遣盟军支援我等。”

%89
东海正道蛊仙们情绪比较复杂,都不好说什么。

%90
毕竟气海老祖身受“重伤”,还为东海正道苦主们联络了天庭,并非不管不顾,不闻不问。

%91
作为一个亲近正道的散修,气海老祖做的已经相当到位了!

%92
至于夏家的那些蛊仙,都很从心的缩在气海大阵中休养。他们的太上大家老已经被方源斩了,剩下的几个蛊仙变得很乖很听话。

%93
他们之前并不乖,尤其是异人蛊仙们从中洲逃回来。这些夏家蛊仙们在当时都非常兴奋,认为重新夺回家园的良机到了。

%94
结果……

%95
结果就被气海分身顺水推舟,把他们的最后一座仙蛊屋拆散了,把他们的大长老给杀了。

\end{this_body}

