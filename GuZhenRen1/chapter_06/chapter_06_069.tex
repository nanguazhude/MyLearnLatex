\newsection{青家之仇}    %第六十九节:青家之仇

\begin{this_body}

%1
黄沙漠风吼,青州暖风流。

%2
砂岩成山,热气蒸腾。砂岩下一片沉静的大湖,绿树成荫,在黄昏的微风中树叶沙沙作响,构成一副西漠世外桃源图。

%3
西漠青家的福地大本营,就坐落在这处人称“青州”的绿洲之中。

%4
青家的蛊仙们几乎齐聚一堂。大殿冰冷的地砖上,有着几块蛊虫的碎片,牢牢牵扯住青家蛊仙们的目光。

%5
“青恒的命牌蛊碎了?!”

%6
“怎么会这样?”

%7
“青恒牺牲了,他此次离家是领命讨伐南疆蛊仙冥幽。难道真是死在冥幽的手中?”

%8
青恒乃是青家七转强者,既然被派遣出去讨伐冥幽,自然是被家族看好。最近不知是何缘故,青家获取豆神宫的秘密竟传出些许风声,令其他超级势力频频试探。青恒之死,令青家上下都怀疑真相,这是不是一次来自西漠势力的刺探呢?

%9
“报——!有确切战报,的确是冥幽斩杀了青恒,此人实力了得,狠辣冷酷。”

%10
青家蛊仙们皆连大怒。

%11
“外域蛊仙欺人太甚!”

%12
“杀了这个冥幽,将他挫骨扬灰,让其他西漠势力好好看看冒犯我青家的下场!”

%13
“在杀他之前,最好将冥幽生擒活捉,再邀请西漠其他蛊仙观礼。唯有当众将冥幽抽筋扒皮,羞辱折磨,一直到他死去,才能解我心头之恨。”

%14
“杀!”

%15
“杀了他!”

%16
“区区外域蛊仙,孤身一人来我西漠,竟敢斩杀我族蛊仙,实在是不知天高地厚。”

%17
“冥幽虽能战胜青恒,但修为只是七转。若是太上大家老亲自出手,未免太看得起他了,也会惹出我青家以大欺小的闲话。依我看,不妨让青荆、青华兰二人一同出马。”

%18
“不,为保万无一失,再加派青沉前去。三仙合力,必能将那冥幽擒拿捉捕!”

%19
“让他好好尝尝痛苦和后悔的滋味。”

%20
“要快!别让这小兔崽子跑了。”

%21
“杀冥幽,扬我青家之威!”

%22
“冥幽必死无疑,待日后我族彻底炼化豆神宫,我青家必定登临西漠蛊仙界的巅峰绝顶!!”

%23
……

%24
一句句的话在青仇的耳畔回响,一幕幕的情景在青仇的脑海中浮现。

%25
和以往支零破碎的记忆完全不同,这一次的记忆展现出了完整的全貌。与此同时,心中的仇恨宛若汹涌澎湃的潮水,不断卷席青仇的整个身心。

%26
恨!

%27
恨啊!

%28
坏着越加强烈的仇恨,青仇睁开了双眼。

%29
时光恍然,已过万千。

%30
青仇双眼赤红如血。

%31
它旋即发现自己躺在地上,然后它立即回忆起来,这是之前它和魔尊幽魂交手,被后者击落,直接从高空坠落到地。

%32
“吼!”

%33
青仇大叫一声,猛地从深坑中站立起来。

%34
仇恨蛊九转的气息越发浓郁,仇恨的力量再度蜂拥而出,迅速流遍青仇全身上下。

%35
“冥幽、冥幽!”

%36
“幽魂、幽魂!”

%37
青仇朝天怒吼,血目狰狞,披头散发,虽然魔尊幽魂已经紧追龙宫而去,但此刻仇恨蛊的力量却让青仇对魔尊幽魂的方位了若指掌。

%38
轰隆!

%39
青仇龟壳如山,虎爪龙尾,蛇颈人头,它飞身而起,卷席大气,掀起猛烈风压,直冲魔尊幽魂的方向追去。对于高空中布阵的紫薇仙子等人,它则不闻不顾。

%40
这群悬浮高空的蛊仙中,有一位蛊仙老者,灰袍大袖,白发飘飘,正是方源的气海分身——气海老祖。

%41
气海老祖见到青仇飞走,不由暗中吐出一口浊气,放下了担忧。

%42
之前在西漠的时候,哪怕方源顶着见面曾相识杀招改头换面,也被青仇直接追杀。

%43
青仇拥有仇恨蛊,又遭受因果神树杀招的影响,对于魔尊幽魂,以及和幽魂有关联的亲朋下属都有强烈感应。

%44
方源本体都瞒不过青仇,气海老祖乃是方源分身,按照这层关联当然也会被青仇感应。

%45
但现在,幽魂和方源对立,双方都巴不得对方身死道消,因此方源和幽魂成了死敌,再不是青仇的眼中钉了。

%46
“虽然青仇的担忧略去,然而紫薇仙子这边却是有着不小的麻烦。”气海老祖又将目光转向紫薇仙子。

%47
紫薇仙子身着紫金宫装,肤若白雪,青丝如瀑。她身姿窈窕神秘,正在空中不断飞舞,手中动作不断,十指纤纤宛若穿花蝴蝶,各种蛊虫接连不断地飞出,相互勾连。高空中,一座精妙的智道仙阵已经搭建好了雏形,初见端倪。

%48
正元老人在一旁协助。

%49
秦鼎菱、车尾等天庭蛊仙看着紫薇仙子,神情复杂至极,有仇恨愤怒,也有同情悲悯。

%50
一座三层圆坛几乎贴着智道大阵,悬浮高空。白玉栏杆,金霞辉映,正是八转运道仙蛊屋劫运坛。

%51
劫运坛由冰塞川主持,目光如鹰锐利至极,时刻紧盯着天庭蛊仙以及气海老祖,不敢有丝毫的松懈。

%52
劫运坛保护着紫薇仙子等人,以及智道大阵。

%53
代表着巨阳仙尊的长生天势力,早就和魔尊幽魂领袖的影宗进行合作。

%54
“方源有成尊之资,我们再不能放纵他了。让他这样的人魔头成为尊者,对我们在场的任何人都没有好处。”冰塞川还在暗中传音,劝说秦鼎菱、气海老祖等人,浑然不知气海老祖就是方源分身。

%55
秦鼎菱只是聆听传音,始终沉默不语。

%56
高空中大风呼啸,吹得她披风飘荡,一身金甲紧贴身躯,更显其高挑曼妙,高不可攀。

%57
她时而盯着智道大阵,时而又转移视线看向梦境。

%58
这片梦境乃是方源所留。方源原本寄希望于梦境阻敌,结果反被魔尊幽魂算计。方源不得不逃离战场,这片梦境仍旧留在原地,绚丽多姿,缓缓流转不休。

%59
“方源此刻身在何处?我只看到龙宫、万年斗飞车从梦境中飞走,脱离此处战场。所以,事实上方源的位置还有第三种可能,就是仍旧藏身在这梦境里头!”秦鼎菱暗道。

%60
正是因为这个原因,她暂时留下来看管这片梦境。

%61
方源太狡诈了,留在梦境中看似很蠢,但不排除方源故意反其道而为之的可能。

%62
眼看着智道大阵接近完善,气海老祖心头压力越发巨大。

%63
“不能再等下去了!”气海老祖脑海中念头电转,不断思量,“看紫薇仙子的态势,又有白凝冰、影无邪等人作为线索,这阵恐怕是有推算本体方位的威能。本体在信道方面建树不大,若是信道造诣深厚,切断本体和白凝冰、黑楼兰等人盟约关系,即便再多两三位紫薇仙子,也无能为力。”

%64
“紫薇仙子等人实力强劲,又有劫运坛守护,冰塞川等人对天庭诸仙防备甚深。我此时若是突然出手,根本不足以摧毁大阵,更会惹来天庭怀疑。”

%65
气海分身的实力是有的,但并不能够一锤定音,镇压全场,所以关键人物还是秦鼎菱!

%66
气海老祖一面听着冰塞川的传音规劝,一面暗中询问秦鼎菱:“看这架势,影宗和长生天都对智道大阵很有把握。但是我们能够相信他们吗?不管紫薇仙子能否推算出结果,大阵都是受她掌控的。她说出一个答案,我们怎么知晓是对是错?还是紫薇仙子故意隐瞒?”

%67
气海分身果断开始挑拨离间,绝不能任凭冰塞川将天庭蛊仙们也争取了去。若是那样,本体那边将要遭受影宗、长生天、天庭的联手追杀,压力太大了!

%68
若是方源五百年前世的天庭,那是杠杠的,绝对不屑于和其他势力联手。

%69
但是宿命蛊被方源摧毁,天庭的精神旗帜就毁了,又被几番攻上天庭,损失惨重,声望大跌。可以说,人族第一势力的称号已经是名存实亡,饱受普遍质疑。

%70
天庭蛊仙亦都是精英人物,痛定思痛,开始自我转变。秦鼎菱领袖的天庭,开始主动合纵连横,放下身段。之前秦鼎菱主动出手,帮助气海老祖赶跑气绝魔仙就是明证。

%71
所以,天庭和影宗、长生天联手的可能性并不低。

%72
气海分身的话一针见血,天庭和紫薇仙子缺乏最基本的信任。这样一来,紫薇仙子推算成功与否,又有什么区别呢。

%73
秦鼎菱沉默半晌后,方才传音回应气海老祖:“还请气海老祖出马,追踪龙宫和幽魂。”

%74
气海老祖微微一愣:“难道你已经知道方源的方位?”

%75
“我不知道。”秦鼎菱对气海老祖微微一笑。

%76
气海分身深深看了秦鼎菱一眼,旋即恍然大悟:秦鼎菱心中有数,她从未信任过投靠影宗的紫薇仙子,长生天也是一丘之貉,信赖他们皆是愚行。她之所以停留于此,一方面是监督梦境,另一方面是重整阵脚。

%77
天庭缺乏八转巅峰战力,从天庭中追杀出来,也将地利丢掉。看似气势汹汹,实则稍有大意,就会损失惨重。

%78
天庭已到了输不起的地步了!

%79
秦鼎菱当然要克制,要稳重。

%80
眼下,幽魂追击龙公,青仇追杀幽魂,而气绝魔仙追击万年斗飞车。

%81
秦鼎菱知道:身怀兮地的气绝魔仙太过克制气海老祖,即便万年斗飞车中藏有方源,气海老祖追了上去,也拿气绝魔仙没有办法。

%82
反倒是让气海老祖追上魔尊幽魂,可以影响战局。

%83
对于天庭而言,方源一定要铲除,影宗也是,长生天同样不是什么好东西。但天庭的实力不同往昔,十分衰弱,就算气海老祖站在天庭这一边,但他仍旧不是天庭中人啊。

%84
秦鼎菱深叹一声:“方源狡诈至极,这一分兵,看似简单,实则让他压力剧减,更让我等不得不驻足,暂时收手。”

%85
“魔尊幽魂强于气绝魔仙,若是让他斩杀了方源,夺得战利,必然会让昔日魔尊再度崛起!”

%86
“还请气海老祖为天下苍生谋算,最好让这两个魔头相互消耗,让我们正道成为最后的赢家。”

%87
秦鼎菱不愧是人杰!她领袖的天庭虽然实力弱小,但她始终都有放弃希望,而是进行谋略布局,尽最大可能扩大胜利的可能。

\end{this_body}

