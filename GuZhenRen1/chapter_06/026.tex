\newsection{人道奠基}    %第二十六节:人道奠基

\begin{this_body}

街道上人流攒动,两旁的店铺几乎都挂着鲜明的招牌,还有布幡。

“哇!这里就是壁画世界啊。真是了不起!一切都好像真的一样啊!”孙瑶只觉得身躯微微一震,旋即视野大变,真正进入到了这里,不禁大呼小叫起来。

周围的行人都向这位小姑娘投来诧异的目光。

但孙瑶却是毫无所觉,她睁大双眼,眸光发亮,不断扫视周围,满脸都是兴奋和好奇。

孙瑶是灵缘斋的蛊师,被门派重点栽培的对象,是蛊仙种子。

原本灵缘斋的名额是属于凤金煌的。可惜因为凤九歌公然叛变,导致秦鼎菱的犹豫。

灵缘斋便及时修改了举荐的对象,因此孙瑶顶替了凤金煌。

然而门派的栽培,却是和孙瑶料想的不太一样。灵缘斋只是告诉孙瑶这个壁画的来历,便将她送进了这里。

孙瑶一时间在街道和各个店铺中流连忘返。

她从小到大就在灵缘斋中长大,平素生活都是师哥师姐,都是修行,都是门规戒律,很少接触世俗。

现在的一切,对她而言,都是那么的有趣和新奇。

她用元石买下好多串糖葫芦,一路吃着,一路看着。

她在小巷中看到剪纸的艺人,那一片片红纸在剪刀下,迅速化为栩栩如生的蝴蝶和花朵。

她进入戏院欣赏戏曲,那些人画着各色的脸谱,穿着华美夸张,唱功悠长高亢。

她在街口停留,看到两个壮汉表演胸口碎大石的杂技。

忽然,她的肩膀被人轻轻拍了拍。

孙瑶转身一看,便见一位少年,国字脸,忠正的模样。

还不待孙瑶开口,这位气质忠正的少年便憨厚地笑道:“你是十大古派的人吗?”

孙瑶顿时瞪圆双眼,惊喜地叫道:“师兄,你也是吗?”

少年便点头,抱拳介绍道:“在下风云府陈大江。”

“原来是陈师兄!”孙瑶心头一震,连忙还礼。陈大江乃是风云府的当代首席弟子,孙瑶早已得知大名,如雷贯耳。

孙瑶有一种见到名人的感觉。陈大江崭露头角的时候,她并不起眼,是凤金煌的跟班而已。

和孙瑶一样,陈大江是风云府的推荐人选,也被送到画中的世界里来。

两人在陌生的地方相遇,尽管只是第一次见面,却有一种天然的亲近。

“陈师兄,不知道门派将我们送到这里来历练,究竟还有什么安排。”孙瑶询问道。

陈大江摇头:“对此我也一无所知。我原以为进入这里来,就能够得到下一步的指引。没想到门派似乎什么安排都没有。”

两人暗中商量无果,决定结伴而行。

“给我一点吃的吧,求求你们了,我们娘俩已经三天没有吃东西了。”

“给点钱,施舍一点吧。”

“我家老父亲去了,小女子卖身葬父,请求贵人成全!”

两人漫无目的,逛到一处街道,却是一片饿殍,许多人面黄肌瘦,依靠着墙角坐着。许多人纷纷乞讨,捧着双手或举着一个破碗。

街道上的行人则是行色匆匆,避之不及,鲜有停下来施舍救济的人。

“怎么会这样?只是相隔了两个街道而已啊。”孙瑶见到这一幕,不由大为吃惊。

陈大江沉声分析道:“这是神帝城的壁画世界。帝君城中历史上发生的每一幕重大的情景,都会融汇成一幅幅的壁画。看来这是历史上的某一次饥荒灾年,帝君城中的情形了。”

孙瑶不由地泛起了同情之心:“陈师兄,虽然他们只是画中人,不是真实的存在。但他们太可怜了,我仍想帮助一下他们。”

陈大江闻言,顿时对孙瑶刮目相看,他欣然笑道:“不怕师妹笑话,我也有此想法呢。不如我们两个一起行动?”

“好呀!”孙瑶雀跃,一口答应下来。

两人出钱又出力,不仅施舍元石,而且还购买稀粥、包子等等,送到这些乞丐的手中。

“好人呐!”

“多谢恩公,两位恩公的救命之恩,我们没齿难忘。”

“两位恩公,请让小人跟随二位以报答恩情啊!”

落难的人们一片感激和称颂。

就在这时,一个轻佻的声音插入进来:“你们在干什么?”

孙瑶、陈大江转头,便见一位蛊师一身白袍,腰际玉带,身姿挺拔如剑,眉宇间透着高傲之气,向他们走来。

“原来是灵蝶谷的萧七星萧兄!”陈大江抱拳招呼道。

孙瑶啊了一声,萧七星同样是鼎鼎大名的人物,慌忙道:“灵缘斋孙瑶拜见萧师兄。”

萧七星打量孙瑶:“你是这次灵缘斋的人选?看来凤金煌真的被雪藏了啊,这也难怪,她老爹公然背叛天庭,凤金煌有这样的下场也很正常。”

这话一出,孙瑶皱起眉头:“凤师姐才不会在意这些事情呢。”

“呵呵,她的确是心高气傲。可惜啊可惜,落毛的凤凰不如鸡。”萧七星露出嘲讽的笑意,“你们居然真的再救济这些难民?他们并非真人啊,你们这样做到底是为了什么?”

“不为什么,就是想帮助他们。”陈大江笑了笑。

“呵呵,同情心啊……”萧七星摇了摇头,不屑一顾的样子,“这么说来,你们都不知道此番际遇的重要性了,居然在这里浪费时间!”

“哦?”陈大江双眼一亮,“萧兄知道些什么?”

萧七星望着陈大江:“告诉你也无妨,这片壁画世界栩栩如生,乃是元莲仙尊的安居乐业杀招所致。我们都得到门派举荐,被天庭栽培,送入这里来深造。相信你们也察觉到了,这里也有蛊师修行,但却以人道为主。”

“门派的意思呢,就是让我们修行人道。争取将来晋升蛊仙,以人道为主修。这片世界虽然虚假,但人道的修行却是真实不虚的。我们从中得到的人道蛊虫,也都能拿到外面去用。”

“原来是这样!”孙瑶双眼瞪大,旋即又流露出些微疑虑,“但是门派为何并不告诉我呢?”

陈大江则干脆问道:“萧兄从何得知此中内情?”

萧七星微微一笑,挺起胸膛,傲然道:“灵蝶谷也未告知我这番内情,是我问了我家的老太爷,这才得知的。”

陈大江哦了一声,见孙瑶还是疑惑,便解释道:“萧兄身份高贵,家学渊源,他的太爷便是萧白虹蛊仙大人。”

“哦,原来是这样。”孙瑶看向萧七星的目光又变了一变。

“好了,我现在要去军营,加入城卫军。我已打探清楚了,只有城卫军中能够得到兵卒蛊、伍长蛊、什长蛊、百夫长蛊等等。这些人道蛊虫十分优秀,可为人道修行的坚厚基石。我们不妨同去,彼此间也好有个照应。”萧七星提议道。

陈大江却摇头:“萧兄先去吧,这里的难民我还未帮完呢。”

萧七星看了一眼陈大江,点点头:“你还是老样子啊,那么你呢?”

他看向孙瑶。

孙瑶犹豫了一下,也摇头道:“不了,萧师兄,我、我打算帮一帮陈师兄。”

“哈哈。”萧七星笑了一声,笑声意味深长,“那我就不打扰二位做好事了。”

他径直离开,心中冷笑:“两个蠢货,根本不知道自己错过了什么样的机缘!”

萧七星不禁回想起太爷叮嘱他的话:“七星啊,这一次你进去壁画世界,是难以想象的大机缘。若是有可能,你甚至能继承元莲真传,成为神帝城之主啊。你要拼尽全力,把握这次机会!记住,一定要抓紧时间。你们只是第一批人选,接下来天庭还会安排一些六转蛊仙进入。这些人也是天庭着重栽培的精英,只是气潮未平,他们都不得不休养生息。”

“你进入壁画世界后,要争分夺秒,努力建立前期的优势,以便将来应对那些蛊仙级别的对手。不要害怕,这些蛊仙在画中世界,也只能遵循画中世界的规矩来。仙凡之间,几乎平等。”

萧七星每每回想起这番话来,就不由地心头火热。

这是绝世机缘,他当然想要把握住!

孙瑶、陈大江不明内情,萧七星便想哄骗他俩,为自己出力,帮衬自己。可惜这两人居然要做好人好事,真是性情不合,不足与谋。萧七星看破这点,也就不愿在他们身上浪费宝贵的时间和精力了。

“这是馒头和粥,慢慢吃,千万别噎着。”孙瑶将一碗粥和馒头,放到沈伤的面前。

沈伤仍旧是浑身伤势,形如乞丐,坐在墙角的阴暗处,宛若石像。

孙瑶也不以为意,毕竟这样默不作声的乞丐也有很多。她做这件好事,帮助他人,并非是为了听一些感恩颂德的话的。

沈伤眼中蕴藏精芒,看着近在咫尺的孙瑶,又看了看街道远处快要消失的萧七星的背影。

“这些人皆是天庭的蛊仙种子啊。为了迎接大时代的浪潮,天庭已经做出了长远筹谋。”

“萧白虹鼠目寸光,害了自家的重孙子还不自知。孙瑶、陈大江先前被蒙在鼓里,而后虽然得知了部分内情,却又秉持本性,率性而为。这两人若是继续下去,必然会受益匪浅啊。”

沈伤暗自点评。这三人虽然传音对话,但却瞒不过沈伤这样的人物。

沈伤心底一声叹息。

“元莲仙尊的这记手笔,真的是磅礴浩大,令人崇敬啊。”

“历来人道真传,当属《人祖传》。然而此书看似肤浅,实则深奥至极,古往今来唯有一小撮人,领悟到了其中精髓,创造出了人道杀招。”

“然而,对于一个流派而言,这些杀招也不过是只鳞片爪而已,仿佛是云端的明月,不是常人能够企及的东西。人道有高端硕果,却无底层基石,因此一直都发展不起来。”

“但元莲仙尊却是创造了这片画中世界,汲取百万年来的人间情景,从中衍化出了人道的种种基石。他为人道发展做出的贡献,着实太大了!堪称是人道的奠基人呐。”

“这里的人道进展,对我而言,也有巨大的帮助。不知道我发疯的时候发生了什么,居然也能混入进来。”

“可惜的是,我始终联络不上方源分身房睇长。他那边究竟又是什么情况?”

沈伤之前就是和房睇长合作,成功地破解了尊者的人道手段,为宿命大战的胜负天平增添了一块重大砝码。

沈伤知道,房睇长曾经炼化过豆神宫,有着很大的权限。若是能够和房睇长再次串联合作,那么对于他们彼此都是一件大好事!

ps:热烈庆贺“努力起床的污妖王”书友,成为本书的白银大盟主,目前的粉丝榜第一!衷心感谢你的支持,也记得你的留言——努力更新高质量的篇章。

也感谢之前参与众筹的诸多书友,具体清单我终于在今天整理好了,待会发到作品相关中去。

本书能走到今天,着实离不开大家的支持,蛊真人感激不尽!

------------

\end{this_body}

