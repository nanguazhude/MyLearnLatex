\newsection{自己来争!}    %第一百二十三节:自己来争!

\begin{this_body}

%1
水汽呼啸,云光中三位蛊仙疾驰而来。

%2
“快,快,快!”为首的蛊仙赫然便是灵缘斋的太上大长老。

%3
包括太上大长老在内的灵缘斋三仙,各个面色焦急不安。

%4
原来,赵怜云强行送走了看守炼蛊密室的五转蛊师,这五转蛊师被送到外面,心中惶恐不安,连忙通讯,告知灵缘斋的高层。

%5
得到消息,太上大长老等人都坐不住了,立即起身,就往这里赶!

%6
凤金煌不能出事!

%7
之前,虽然因为宿命大战中凤九歌反叛,令白晴仙子、凤金煌这对母女失势。然而天庭始终是冷处理,并未真正惩罚这对母女。

%8
按照正道的规矩,也不好惩处。因为白晴仙子、凤金煌都没有背叛门派,凤金煌更是龙公之徒。

%9
方源追杀战时,秦鼎菱下达了紧急命令。灵缘斋的太上大长老亲自操纵仙蛊屋,载着凤金煌,亲眼看着她将方源遗落的梦境全部收取。

%10
凤金煌立下的这场功劳并不小!

%11
太上大长老更在离开之际,得到秦鼎菱的暗中传音,关照她:要好好的照看凤金煌。即便宿命蛊被摧毁,梦道仍旧是未来的大势。

%12
灵缘斋高层顿时明白了天庭的用心——凤金煌仍旧被天庭看好,暗中栽培。凤金煌收取的那些梦境,都被她带了回去,天庭一份都没有索回!

%13
三仙直接飞射到一号炼蛊密室之中。

%14
“怜云仙子,你要冷静……呃。”三仙诧异。

%15
他们并没有看到凤金煌身亡,眼前的一幕反而是凤金煌和赵怜云的对峙。

%16
凤金煌嘴边残留血迹,但三仙在侦查手段下,立即发现凤金煌健健康康,毫发无损。

%17
“这是怎么回事?”

%18
“难道真的是赵怜云救下了凤金煌?”

%19
“太奇怪了。赵怜云一直以来都是倒凤一派的人物,凤金煌此次炼蛊失败,她来的甚是蹊跷,明显是要发难。但为何会是如此情形?”

%20
三仙疑惑不已。

%21
从场面来看,反而是凤金煌的气势占据上风,而赵怜云的脸色则有些难看。

%22
“凤金煌,你很好!”赵怜云深深地看了凤金煌一眼,留下这句意味深长的话,转身便走。

%23
她和赶来的三仙擦肩而过,竟招呼都不打,大违平时的风范,可见她心情之败坏。

%24
凤金煌用手背一下就擦干了嘴角的血迹,笑着拜见三位太上长老。

%25
三仙纷纷明悟过来,心中惊叹。

%26
“看来应当是凤金煌的智计,她一定是查探到了什么,将计就计,从而引蛇出洞,让赵怜云等人败露形迹了。”

%27
“自从宿命大战之后,凤金煌的进步就非常明显。果然困境能磨砺人呐。”

%28
“只是这两方之争,如何是好?”

%29
三仙暗中苦恼无比。

%30
一方是倒凤派,徐浩、李君影乃是派系中坚,他们一手扶持赵怜云成为当代灵缘斋仙子。有了赵怜云的加入,此派已然是灵缘斋的第一大势力。

%31
而另一方是白晴仙子、凤金煌。这对母女也都不是好相与的,跟脚深厚。凤金煌看似落魄,实则仍旧得到天庭看重。别忘了,她的父亲还是凤九歌。凤九歌虽然背叛正道,但只要一天不死,那灵缘斋上下就不能忘记他的存在!

%32
两派相争,成为灵缘斋的最大内患,灵缘斋的太上大长老却束手无策。

%33
“凤金煌,你今后若要炼蛊,不妨来我主峰。”太上大长老看着凤金煌沉默了一会儿,这才神色和蔼地道。

%34
却是对之前赵怜云找麻烦的事情故作不知。

%35
毕竟没有什么证据。

%36
唯一的证据已经参与炼蛊,而被摧毁了。

%37
况且就算有证据,太上大长老恐怕也会按捺不发。因为公然发作,只会引爆内患,令门派迅速衰落。

%38
太上大长老是成熟的。她知道:眼下的灵缘斋陷入内斗,就像是一个火山口,一定会在将来喷发。她要做的,就是拿个锅盖,盖住火山口,尽全力遮掩,拖延时间。在拖延出来的时间中,她需要挖开正确的渠道,让火山中的岩浆流淌倾泻出来,释放内压,维持稳定。

%39
赵怜云一路面色铁青,飞回自己的含情峰。

%40
她没有入殿,而是站在峰巅,看着阴沉沉的天空。

%41
她知道会有人来。

%42
果然,没有一会儿,徐浩、李君影便双双飞至。

%43
这对夫妻的脸色也不好看,在来的路上,他们已经得知此番谋划已经失败了。但具体情况是怎样的,还需从赵怜云的口中得知。

%44
没有等徐浩、李君影开口,赵怜云已经抢先询问:“二位仙友,凤金煌如何发现蛊材中的破绽?此次事情败露,二位仙友有什么话可说的?”

%45
徐浩哑然。

%46
李君影黯然道:“是我小瞧了凤金煌。怜云,你也知道我号称幻灭,这种手段本就擅长。尤其是最近这段时间,我晋升为影道大宗师。即便是将蛊材放到太上大长老面前,她也未必能发现端倪。”

%47
赵怜云面色不善,气得一拂袖:“可是事实如此,二位又作何解释呢?”

%48
徐浩开口道:“或许,龙公曾经给予凤金煌一些护身保命的手段。又或许,这就是梦道的威能。毕竟梦道,我们都不熟悉。就像那大魔头方源,曾经也是多次凭借梦道以弱抵强啊。”

%49
徐浩接着道:“好了,这件事情是我们的过失。怜云仙子,按照我们的约定,你帮我们做成这件事情,我们之间就互不相欠了。但这件事情既然失败,那就算了,毕竟是我方的过失,就当做事成来处理。从今以后,我们两不相欠!”

%50
听了这番话,李君影面露急色,张口欲言,但却被徐浩伸手按捺下来。

%51
赵怜云这才面色稍缓,对两仙点了点头,一言不发转身即走。

%52
二仙见赵怜云身入峰顶宫殿,只得回返。

%53
半路上,李君影气愤不平:“这赵怜云若非我们相助,她岂会能拥有盗天真传神不知?若非我们相助,她岂会成为本派的当代仙子?如今她却要和我们划清界限,真的是一个白眼狼。更可气的是,夫君你居然就这样轻易地放过了她。你可要知道,她是我们对付凤九歌的最佳利器啊!枉你专修智道,唉。”

%54
徐浩呵呵一笑:“夫人,不必忧虑。此事失败,我们也大有收获。一来,从太上大长老等人的身上,试探出了天庭的心意。二来,将赵怜云牢牢绑在了我们的阵营之中。三来,还测探出了凤金煌的一丝底细。”

%55
李君影微微一愣,旋即有所明悟:“也对,此番谋算失败了,但赵怜云也和凤金煌彻底结仇。就算我们不找赵怜云,将来凤金煌也要‘回报’她。敌人的敌人便是朋友,所以她仍旧是我方阵营的人。”

%56
“正是如此。”徐浩点头道。

%57
李君影微笑:“夫君,你不愧是专修智道,我错怪你啦。”

%58
徐浩则苦笑,目泛忧苦之色:“凤金煌不愧是凤九歌之女,心性、才情还有梦道手段,都超出我的意料。我们原本计划,是用凤金煌充当诱因,来诱使凤九歌发怒,前来攻打灵缘斋,从而造成他和天庭的直接冲突。”

%59
徐浩、李君影的目标,始终是凤九歌。

%60
凤九歌虽强,十个徐浩、李君影加起来,都不是凤九歌一人的正面对手。

%61
但何必要正面作战呢?

%62
人是万物之灵,杀死一个比自己强大得多的敌人,有许许多多的好办法。

%63
身为智道的徐浩,便想出了一个以天庭为刀,铲除凤九歌的计谋。

%64
这就是智道蛊仙的风采,擅长的就是因势利导,借势成事!

%65
“此番谋划失败,是我小瞧了凤金煌啊。此次之后,她定然会缩到其母白晴仙子身边,我们接下来下手的机会,可就要少得多了。”徐浩忧心忡忡。

%66
凤九歌曾经打压徐浩,令他的生活暗无天日。这是仇恨。

%67
凤九歌在灵缘斋的时候,一家人都霸占修行资源,让徐浩、李君影修行缓慢。这是利益。

%68
凤金煌钻研梦道,得到天庭看重,走在时代前沿。这是未来的祸患。

%69
灵缘斋、天庭都是指望不了的,真正能代表徐浩利益的,只有他自己。

%70
他自己若不争,谁能替他着想?

%71
“修行一途,就要竞争,拼了性命争夺一线胜机啊。”徐浩口中喃喃,他的眸光再度坚决起来。

%72
数天之后。

%73
东海。

%74
无名海域中的小荒岛上,方源放出了自家的梦道分身。

%75
“开始吧。”方源淡淡地道。

%76
梦道分身点点头,深呼吸一口气,下一刻悍然碎窍!

%77
完美的纯梦求真体的空窍,和寻常的五转空去不同,本身就在不断的破碎和凝聚成形的过程中循环。而它的窍壁始终有一层膜,纤薄不堪,却牢牢限定着梦道空窍。

%78
此时,方源的梦道分身只用真元稍稍一冲,就直接冲碎。

%79
下一刻,无形的天地伟力源源不断地涌来,托举着纯梦分身缓缓升上半空。

%80
轰隆隆!

%81
天空中乌云翻腾,雷声滚滚。

%82
小岛周围波涛越来越大,很快就变成惊涛骇浪!

%83
环境变得恶劣起来,在这种环境下运用蛊虫,不论仙凡,都会遭受反噬。

%84
反噬严重的话,即便仙蛊都要损毁!

%85
天意似乎已经暴怒!

%86
方源这个可恨的贼子,居然想提前成就梦道蛊仙。

%87
这不是天意的原本计划。

%88
它违背了宿命的安排。

%89
不可饶恕!

%90
不可原谅!

%91
方源仰望天空,目光平静而又坚定。

%92
天道不允又如何?

%93
那就争!

%94
那就战!

%95
这世间最能依靠的,不是家族也非门派,不是亲人也非爱人,而是自己啊。

\end{this_body}

