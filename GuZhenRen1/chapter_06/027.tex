\newsection{迷雾中的方源}    %第二十七节:迷雾中的方源

\begin{this_body}

%1
至尊仙窍,小南疆。

%2
一处山谷之中,石狮诚俯身查看谷内的石菇。

%3
这些石菇若是长势良好,都会高达半丈,形如矮树,菌盖如伞。然而眼下的这些石菇,却个个矮小,宛若大树被砍伐后留下的木桩。并且浑身上下也不是健康的灰石之色,而是呈现一种扭曲的五色斑斓,一看就知道大为古怪。

%4
“这片石菇都毁了!”石狮诚检查完毕,确认了这一点后,脸色相当难看。

%5
他虽然修为只有六转,但被方源器重,安排在小南疆中总领石人一族。石狮诚对方源十分崇敬,一门心思想要发展石人一族,不负方源期待。

%6
这片石菇就是他为了石人一族着想,亲自布置建设出来的中型资源点。为了建造这片资源,石狮诚专门向方源宙道分身请求了许多仙材,作为前期的投入。当时,他还信誓旦旦地保证一定会有理想的收益!

%7
结果,现实像是一个巴掌,狠狠地打在了石狮诚的脸上。

%8
“怎么会这样?”

%9
“我明明是按照族中的记载,去精心栽培石菇,没有半点差错。”

%10
石狮诚十分疑惑。前几日这些石菇都长势喜人,一切正常。结果几日一过,这些石菇就变成了这样。

%11
“这让我如何向方源大人交代呢?!”石狮诚眉头紧皱,唉声叹气。

%12
就在这时,光芒一闪,半空中传送过来一股方源意志。

%13
意志凝如实质,栩栩如生。

%14
石狮诚啊了一声,立即行礼:“属下拜见主上。”

%15
“嗯。”方源意志摆了摆手,“你这里的情形我已经知晓,石菇被毁,怪不得你,是一场天灾。你且先回去,过段时间就有另外一批仙材送达,你再建设第二座石菇山谷便是了。”

%16
石狮诚大喜,连忙拜谢,又欲言又止。

%17
方源意志露出了然的微笑:“你是想问究竟是什么原因,致使这些石菇被毁吗?”

%18
石狮诚点头:“主上明察秋毫,下属还想问:这些石菇是否还能废物利用呢?若是直接抛舍,未免有些可惜。”

%19
建设石菇山谷的前期投入很大,当然在方源眼中不算什么,但在石狮诚看来却是一笔天价投资了。再加上他前前后后在这里投入的心血和时间,他不愿意看到自己一切的努力都化为泡影。

%20
但方源却摇头:“内中详情你不必多想,我若能解决,你便无忧。若是我不能解决,你知道也没有用。至于这些石菇一个都不能用,就算是当做蛊材炼蛊,也只会酿成更大的损失。你且去吧。”

%21
“是,主上!”石狮诚恭声退下,心中不免忧虑,“怎么这次麻烦这么大?听主上的口音,也未必能解决此事?”

%22
石狮诚离开,这片山谷中就只剩下方源意志。

%23
方源意志面色凝重。

%24
他先扫视整个山谷,山谷中的无数石菇,都成了扭曲之物,原本光滑的表面形成无数细碎的皱褶。色彩鲜艳斑斓,宛若毒物聚集,给人一种恐怖之感。

%25
这当然只是表象。

%26
放在方源眼中,便可见到,这些石菇中原本的土道道痕,都被打乱冲散,夹杂了许多其他流派的异种道痕,有风道、雷道、毒道、暗道、炎道等等,不一而足。

%27
正因如此,这些石菇只能当做废物舍弃。至少目前,方源没有什么手段能够利用它们。

%28
“天意!”方源意志忽然冷笑一声,发现了山谷中潜藏的天意。

%29
他砰的一声,化作漫天云雾,笼罩整个山谷,对天意围追堵截。

%30
一场意志间的交锋,很快落下帷幕。方源完胜,他成功地将此地潜藏着的天意彻底清缴,当然付出的代价也不少。

%31
这股意志重新凝聚之后,再不像之前那般栩栩如生,而是虚幻若影,给人随风飘逝的感觉。

%32
方源意志叹息一声。

%33
此番得胜,不过是治标不治本。新的麻烦出现了!

%34
石菇山谷并非只是个例,最近这段时间,在至尊仙窍各地都出现了潜藏的天意,许多资源点都被破坏。

%35
方源本体因为主持炼蛊,摧毁宿命蛊的同时,获得了三千多道天道道痕。

%36
这些道痕并不属于他,强加于他身上。天道道痕自行演化,导致至尊仙窍的各处环境发生变化,变得越发平衡和稳定。

%37
这番变化若是长远来看,自然是利大于弊,大大提升了至尊仙窍的发展潜力。但若从近期着眼,却会令方源损失惨重。

%38
方源不是超级势力,身家都在至尊仙窍之中。这片基本盘若是血亏,必将影响他的战力发挥。方源不愿落入红莲魔尊的算计之中,一直都在努力,统帅洞天中的人族和异人,不断地对抗天灾。同时他自己先后收取了千愿树、人海雏形,利用人道来克制天道变化。

%39
如此双管齐下,至尊仙窍的情况大为好转。但方源没有料到,天道道痕自然演变到一定程度,居然产生了天意。

%40
天道大公无私,对方源并无恶意,改变环境都是自然演化而已。但天意却非如此。

%41
天意由天道演变而生,更加灵活机智。天道无法剿除方源这个天外之魔,影响平衡的存在,天意便更进一步,开始特意摧毁至尊仙窍中的各处资源点。

%42
天道能够改易其他流派的道痕,天意便借助这样的威能,彻底令各大资源点的道痕紊乱,从井然有序变得糟糕透顶。

%43
石菇山谷只是当中之一罢了。

%44
“我掌握着天消意散杀招,但是至尊仙窍实在太大,天道不断演变,能够在任何地方产生天意。我要清缴这些天意,着实困难!即便全力出手,效率也很低。”

%45
方源清楚,治根之法便是直接炼化了这些天道道痕,让它们真正属于至尊仙窍,属于自己。

%46
但如何去做,他一窍不通。

%47
他还从未听说过,有人修行过天道!唯一的办法,只能用人道来压制天道。

%48
方源思谋良久,找到两条可能的出路。

%49
第一条便是帝君城。

%50
“帝君城乃是元莲仙尊精心布置,如今已然成了人道圣地。我若得之,必定能极大地克制天道。”

%51
第二条路则是疯魔窟。

%52
“疯魔窟是无极魔尊晚年布置,第八层中就包含天道演变。而且疯魔窟中魔音肆虐,令生灵发疯,各种仙材皆是废物,道痕混杂。这种情形完全和石菇等等资源点一模一样。我若能掌握疯魔窟的全部秘密,应当能改善我的现状。”

%53
但这两条路都非常的危险。

%54
帝君城位于中洲中心,方源要去,必定和天庭为敌。以他当下的实力,明面上是能战胜天庭的。但是!方源忌惮的是元莲仙尊,是红莲魔尊。

%55
元莲仙尊算计了他,令他损失了房睇长这个分身。之前察运,方源并未发现房睇长的运势,可见凶多吉少,陨落的概率极大。

%56
红莲魔尊也算计他,若非天道道痕,方源仙窍中怎可能有天意产生?当然,红莲也算计了魔尊幽魂。天道道痕对于方源而言,既是危机也是机遇,更是一层保护伞。至少魔尊幽魂可能布置的手段因此无法发动。

%57
方源要取帝君城,就要面对元莲仙尊、红莲魔尊的手段。

%58
而去疯魔窟?

%59
那里也危险重重。

%60
疯魔三怪之前答应方源,会参加宿命大战,结果大战前后他们三个一丝人影都不见。方源不是没有联络过他们,他们已经彻底失联了。

%61
疯魔窟究竟出现了什么情况,让疯魔三怪都神秘失踪?

%62
方源猜测,或者这是长生天出手。智慧蛊的损耗,让方源意识到这是巨阳仙尊的谋算。巨阳一击更让方源清楚,巨阳仙尊始终留着后手。

%63
眼下宿命蛊已毁,尊者复活再非绝不可能之事。方源若去疯魔窟,极可能要面对巨阳仙尊,甚至无极魔尊的手段。

%64
还有一件令方源忧虑的事情,那边是光阴长河的异变。

%65
自从宿命蛊被毁之后,光阴长河的河水就变得狂暴至极,极其凶恶惊险。即便是方源动用万年斗飞车,也只能勉强进入河中,支撑片刻就得回转五域,不能持久探索。

%66
方源倒是明白缘由。

%67
之前光阴长河能够平稳前行,濯濯流淌,是因为万物皆定,一切都被宿命蛊规定下来,不管什么存在都有特定的生命轨迹。正因如此,宙道蛊仙往往能通过观察下游,来“预知”事物。红莲魔尊之所以能够布局,最终摧毁宿命蛊,便是观察下游得到了许多未来秘辛。

%68
但是如今宿命蛊一毁,万事万物都没有了固定的轨迹,完全能够有任意的变化。因此光阴长河也就变得狂暴至极,一切都浑浊动乱不堪。

%69
光阴长河如此异变,春秋蝉暂时也不能使用了。

%70
春秋蝉之前能穿梭河水,是因为河水温和。如今河水变得狂猛暴乱,春秋蝉一旦发动,进入河内,必定会被河水直接拍碎。除非方源研发出了其他杀招,能够保护春秋蝉。

%71
“河水虽然麻烦,其实问题不大。我真正顾虑的还是红莲魔尊啊。他可是在光阴长河中种下过石莲岛的人。摸不准他在光阴长河之中还有什么布置。”

%72
宿命蛊终究是毁了,方源终于拨开了头顶上高悬的那柄铡刀。

%73
然而,他的眼前是一片迷雾,他行走在崇山峻岭之中。在这迷雾的包裹中,究竟哪一块是实地,哪一块又是悬崖呢?

%74
那通向永生的,他矢志毕生追求的道路,在这迷雾中是否存在呢?

%75
若是存在,它又在哪里?

%76
ps:有书友制造宠物药品,目前寻求销售渠道,我看了一下,感觉产品不错的。有经营相关方面的书友,若和这位书友有合作意向的,不妨和我联系,我来牵线搭桥。

\end{this_body}

