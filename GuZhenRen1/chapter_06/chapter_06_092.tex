\newsection{助方源成尊}    %第九十二节:助方源成尊

\begin{this_body}

%1
陆畏因当即回应方源:“方源仙友,请随我来!”

%2
方源却回道:“不急。”

%3
下一刻,他冷笑一声,背负双手,迎着三方蛊仙,诸多仙蛊屋,云淡风轻,从容而笑:“好得很,你们谁来第一个送死?”

%4
话音刚落,他全身气势暴涨,八转气息圆融畅通,竟无一丝疲态。

%5
诸仙纷纷瞪眼。

%6
“难道他不是强弩之末?”

%7
刚刚汹涌的冲势立即一滞。

%8
诸仙迟疑不决。

%9
趁着这个功夫,方源哈哈一笑,催动仙道杀招,带着陆畏因等人,化作一道剑虹,飞遁远去。

%10
和之前不同,方源已是炼化了不少天道道痕,从天道道痕的重重束缚中,挣脱出了珍贵空隙。

%11
他施展仙道杀招,比之前容易了数倍。虽然仍旧要用人道大阵等手段冲击天道道痕,才有出手的空间。

%12
“方源逃了,快追!”

%13
“他在戏耍我们呢。”

%14
众仙如梦初醒,纷纷狂追不舍。

%15
种种杀招打去,对方源等人轰炸不断。

%16
似乎情景又落到了之前一般,只是少了魔尊幽魂、气海老祖、青仇,多了一个陆畏因。

%17
然而……

%18
“务必要谨慎小心!方源这等魔头的临死反扑,绝不容小觑。”

%19
“让其他两方顶上,我们多留些气力。”

%20
“方源狡诈,依我看,未必前方没有第二座战场杀招,或者仙道大阵啊。”

%21
“是啊,方源在南疆一战,就是用宙道大阵坑陷了南疆正道联盟的大部队。陆畏因也是阴险小人,此次布置土道战场伏杀幽魂,手段和方源如出一辙。我真是看错了他!”

%22
“堂堂乐土南疆传人,居然和方源狼狈为奸!!”

%23
众仙正气浩然,义愤填膺,又暗中传音,纷纷合计。

%24
他们一边追杀,声势浩大,一边也都有些心虚!

%25
此时情况看似和之前相同,但其实已经发生了翻天覆地的变化。

%26
方源携伏杀幽魂之威,让三方都投鼠忌器。

%27
而方源向来行事阴险,卑鄙狡诈,狠辣无情,种种战绩深入人心,让三方蛊仙无不顾忌重重。

%28
一方面,即便方源真的是强弩之末,他的临死反扑都会让任何一方大出血,自家局面崩坏;另一方面,三方的对手不仅仅只是方源,还有其他两方啊。

%29
之前,是有幽魂、气绝承担了正面主力,所以三方乐得追在后面,伺机捡便宜。但现在幽魂没了,气绝魔仙看似气势汹汹,其实暗地里正在勒索方源,敲诈好处呢,怎可能真的强硬出手?

%30
说起来,三方蛊仙忌惮方源,气绝魔仙难道就不忌惮了么?

%31
“方源这小子,几乎在伏杀战中,就没有怎么出手过。他到底恢复了多少?现在看他施展杀招,明显更多更频繁。他究竟还隐藏了多少实力?还有没有第四座仙蛊屋?”气绝魔仙心中嘀咕。

%32
方源之前连毁龙宫、万年斗飞车,旋即就拿出安土重山堡。这三座仙蛊屋都是八转,奴道、宙道、土道的巅峰杰作。

%33
方源如此财大气粗,真的惊着了气绝魔仙。就算方源拿出第五座仙蛊屋来,气绝魔仙也毫不奇怪。

%34
“方源的底蕴,到底有多深厚?!历代尊者对他的投资,未免也太过了啊!”气绝魔仙直嘬牙花子。

%35
方源一边撤离,一边用真传交易,态度忽强忽硬,让气绝魔仙始终犹豫不决。

%36
“我究竟该不该立即动手?”

%37
“方源还剩下多少实力?除掉他,收益会不会更多?”

%38
“不过现在这样,一边追杀,一边接收真传,似乎也挺好……”

%39
所有人当中,就属气绝魔仙最为享受追杀的过程了。

%40
“到了!就在这个方位的下一层。”陆畏因忽然传音。

%41
方源操纵光虹,猛地一折,瞬间突破坚厚的土石,来到了下一层。

%42
轰隆隆……

%43
方源等人的耳畔立即充斥轰鸣之音。

%44
无边的地气宛若江河奔腾,海潮汹涌。方源三人置身其中,宛若河中蚂蚁,微不可查。

%45
“这是……地脉显形?!”方源楞了一下,旋即辨认出来。

%46
陆畏因点头含笑:“不错。五域合一,地脉也随之一统,地脉融汇迁移,便会显露真形。快落到那处小岛上去。”

%47
方源艰难催动杀招,剑虹直射而下。

%48
剑虹艰难维持,在半空中终于支撑不住,彻底消散。

%49
三仙同时跌落到一座褐黄色的小岛上,姿态狼狈。

%50
这座小岛面积很小,只有数亩。岛上一片荒凉,没有一丝生机。小岛缓平,但在边缘处有三块巨石,宛若柱子,扎根在浩荡的地脉之中。

%51
正是因为这三根巨石柱,使得小岛在地脉的冲刷下,也岿然不动。

%52
“方源你哪里逃?”

%53
“快追!”

%54
“这是什么地方?!”

%55
这时,三方追兵也下到这一层,追了过来。

%56
有人催发杀招,却发现这里土道道痕极其浓郁,压制住了绝大多数的异种流派。

%57
蛊仙催动杀招都十分费力,除了土道流派,唯有仙蛊屋才能纵横往来。

%58
“不妙,方源来到这里,恐怕是有意为之。”

%59
“不管他想要干什么,必须阻止他!”

%60
“快撞塌那座小岛。”

%61
三方蛊仙气势汹汹,越发逼近。

%62
方源、吴帅、陆畏因则是猛然动手,击溃了小岛周边的三根巨石大柱!

%63
没有了巨石大柱扎根,小岛开始顺着地脉的流动而漂移。

%64
三方蛊仙追来,但小岛加速极其惊人。双方差之毫厘谬以千里,小岛险险地拉开距离,然后几个呼吸的功夫,就将三方蛊仙彻底甩远。

%65
三方蛊仙目瞪口呆。

%66
这是什么鬼玩意?

%67
明明只是一座岛,居然速度如此惊人!

%68
“真的摆脱他们了。这座小岛究竟是何物?还有我们会去往何处?”吴帅抚摸着小岛坚硬的地面,惊叹地问道。

%69
陆畏因不断咳血,刚刚摧毁巨石柱,让他伤势雪上加霜。

%70
他没有回答吴帅,而是望向方源:“方源仙友,应当已经回忆起来了吧?”

%71
方源点点头,第一眼看到这个小岛,他脑海中就有一道灵光闪现。到了落到岛面,他已经回想起来。

%72
“这是飞地。”方源回答吴帅。

%73
吴帅虽然是他的分身,但方源并没有将一切记忆都复制给他。这是防备万一。

%74
若是将来,吴帅战败被俘,遭受搜魂等手段,那么这些记忆就都资助敌方去了。

%75
“在五百年前世,五域一统,两天合一,五域地脉也相互融合,形成不少特殊地形。世人汇同古今,将其统称为十地。它们分别是地渊、地沟、地道、地穴、地牢、产地、飞地、阵地、藏地、墓地。”

%76
“这座小岛便是十地之一的飞地,只在地脉中凝聚成形。可以在地脉中不断穿梭,迅疾若飞。速度之快,绝非一般。”

%77
吴帅恍然:“原来如此。”

%78
他继续感慨道:“天地剧变,地底深处亦有别样精彩。”

%79
他知道了这座小岛的本质,又问:“那我们能去向哪里?”

%80
方源微微摇头:“飞地我也只是听闻,并未亲自接触过。只晓得它能顺着地脉前行,但方向却是难以掌控。不知陆仙友,有什么手段?”

%81
这一场埋伏战,陆畏因的表现极其突出,让方源对他十分重视。

%82
这位当代南疆的乐土传人,实力非常强劲。

%83
先是布置出了幻沙转影战场,这座战场的优秀到困住幽魂、气绝,完美地呈现了战场应当展现的威能。

%84
然后,陆畏因操纵战场,让围攻幽魂的诸仙配合更加默契。幽魂曾在激战中,屡屡想要对青仇下手,都被陆畏因阻拦。

%85
其次,陆畏因在关键时刻,展现出了强劲的防御手段,竟挡下幽魂的狂攻,给气绝魔仙争取到了施展兮兮杀招的时间。这是埋伏战的转折点。

%86
再其次,陆畏因用众生皆苦、吃亏是福杀招,克制了幽魂,让幽魂遭受反噬重创。

%87
最后,他还强撑战场,压制幽魂自爆的恐怖威能,为方源的两大分身解围。

%88
战场被冲垮之后,还是他引领方源,来到这块飞地,逃出生天。

%89
陆畏因的战力,绝对是当世一流。更让方源欣赏的是他的性情。他擅长隐忍,刻意留着众生皆苦、吃亏是福两大杀招,在关键时刻才用出来,让幽魂吃了血亏。

%90
如此谋定后动的城府,方源相信,陆畏因定然掌握着操纵飞地的手段。毕竟这涉及土道,是乐土仙尊最擅长的领域。

%91
果然,陆畏因没有叫方源失望。

%92
他微笑点头:“在下无才,但乐土传承中却有相应的手段,能够操纵飞地方向和停止地点。”

%93
吴帅闻言,瞳孔微缩,不禁有了戒备。

%94
陆畏因恍若未觉,继续微笑道:“按照这条地脉,途经西漠,再贯东海,然后直达南疆,再远上北原。不知方源仙友想在何处停靠?”

%95
方源也微笑:“你之前不是请我去南疆菇人乐土中作客的么?”

%96
陆畏因点点头:“方源仙友,如今魔尊幽魂已是被我们联手伏杀,在下的诚意相信仙友已有估量。这不只是我的诚意,同样也是乐土仙尊的诚意!”

%97
“在下虽不知内幕,但乐土仙尊留下的传承中却有关照——古月方源将是五域乐土传人!乐土势力皆要辅助他,全力助他成就蛊尊!”

\end{this_body}

