\newsection{与天庭合作}    %第三十九节:与天庭合作

\begin{this_body}

%1
方源看到这半份元始真传,脑海中首先浮起的一个念头,却不是和真传相关的。

%2
“我是否应当趁着这个时机,将天庭的这帮蛊仙一锅端了?”

%3
这个想法很有诱惑力。

%4
秦鼎菱如今是天庭的暂代领袖,失去了她,天庭必遭打击。古月方正是天意布置,专门为了克制方源这个天外之魔的关键人物。方源斩杀了他,自然是挣脱了一份束缚。而同行的凤仙太子、白沧水等人,如今都是天庭的中流砥柱。缺失了他们,对于天庭,对于中洲正道都是强烈打击。

%5
但方源念头一转,还是放弃了这个想法。

%6
他如今实力不成,因为天道道痕加身,战力方面不能保证。

%7
别看现在天庭诸仙谈笑,似乎很放松的样子,事实上第一次来到气海老祖这边,怎么可能没有防备?

%8
只是为了表现诚意和友好,故意装作没有防备的样子而已。

%9
就算方源发动气海大阵,天庭诸仙也必定能第一时间取出仙蛊屋困守。

%10
若是在此时对付天庭诸仙,不管胜败,方源这身气海老祖的身份就要遭受世人的强烈怀疑。

%11
说心底话,方源倒是很愿意和天庭合作。

%12
纵然之前天庭方面千方百计地刁难方源,企图扼杀方源,甚至一度将方源逼入绝境。但只要利益足够,有益于追逐永生,方源和天庭合作是毫无心理障碍的。

%13
天庭和气海老祖合作,是因为有利可图。

%14
方源和天庭合作,同样会收获巨大。

%15
长远的利益且不去说,单论眼前方源就借助了天庭之力,击退了气绝魔仙,保证了气海老祖这层身份的价值。同时,天庭还主动将剩下的一半元始真传送了过来!

%16
方源手捏着这只信道凡蛊,心中颇有感慨。

%17
当初他为了谋求这份气道真传,花了不少心思。龙公精明老辣,明明盗天手段能够削除他的寿元,他却是伪装成功,用了半个元始真传成功稳住了气海老祖。

%18
剩下的一半真传,龙公是死活都扣在手中,方源无法得手。

%19
方源原本已不抱希望,但没想到宿命大战之后,这剩下的一半元始真传居然通过秦鼎菱到了自己手上。

%20
剩下的一半元始真传中,方源垂涎已久的三气归来杀招便赫然在列。

%21
“秦鼎菱不容小觑啊。”方源心中感叹。

%22
秦鼎菱还未说出什么要求,直接就将剩下的一半元始真传送给了气海老祖,这手笔真的很大。秦鼎菱和天庭的诚意十足地表达了出来。

%23
“若是秦鼎菱知道我就是方源,不知会有什么神情和想法?”

%24
龙公之前资敌了一次,现在秦鼎菱又接着这样做了。若是秦鼎菱明白眼前此人就是宿命大战的最大凶手,而自己却巴巴地送真传到对方手中,恐怕会气得当场吐血吧。

%25
“贵方的诚意,老朽是真真切切地感受到了。秦仙友,当下有什么麻烦,老朽必定倾尽全力相助天庭!”方源将信道凡蛊直接收入仙窍之中,对于天庭的麻烦,他其实也有猜测。

%26
秦鼎菱见气海老祖收了信道凡蛊,不由地暗自松了一口气。

%27
她此行最担忧的就是气海老祖不出手相助。毕竟气海老祖之前可是隐修,事不关己高高挂起。也是被重生后的方源带来的消息触动,气海老祖明白这样隐修下去,也会被天庭找麻烦,所以才一改行事作风。

%28
秦鼎菱并不知道,天庭究竟是怎么和气海老祖结仇的。但既然气海老祖听信了方源之言,显然方源给出了足够多的证据,让气海老祖相信。

%29
这是一笔糊涂账,眼下宿命蛊已毁,未来大变,天庭也追查不出什么来。秦鼎菱只能依据现状来进行推测和判断。

%30
秦鼎菱有所顾虑是很正常的。

%31
毕竟之前气海老祖和天庭是敌对关系,因为龙公斡旋,这才转为中立。当下为了将气海老祖拉拢过来,秦鼎菱必须拿出令气海老祖无法拒绝的筹码。

%32
气海老祖收下了元始真传,立即表明坚决的态度,要出手帮助天庭。

%33
秦鼎菱大感欣慰。她的谋算当然还不只是这些。元始真传只是一个诱饵,气海老祖要催动当中的杀招,得用合适的气道仙蛊。

%34
而这些气道仙蛊都在天庭的库藏之中。

%35
这就是接下来诱惑气海老祖的筹码了。

%36
“或许还能借助气绝魔仙的压力,促使气海老祖加入我天庭。”这个灵感在秦鼎菱的心头一转,暂时被她按捺下来。

%37
她面对微笑,对方源述说了天庭面临的难题。

%38
方源抚摸胡须,心中了然。果然不出他之前的猜测,天庭中也出现了气功果。

%39
龙公牺牲,仙墓摧毁,天庭再无气道高手坐镇,因此要解决气功果的麻烦也分外困难。

%40
秦鼎菱送元始真传这么干脆,也希望这份真传能够帮助气海老祖,让他更快地解决气功果的麻烦。

%41
方源思考了一番,点头道:“气功果一事,老朽应下了。但眼下并无手段,还得闭关一段时日,研习元始真传。”

%42
这时夏家太上大长老在一旁忽然开口:“老祖在上,如今您有伤在身,还得先休养好才是。您的安危可不只是关乎东海局势,更维系着人族和异族的天下大局啊。”

%43
他有点担心,害怕气海老祖和天庭合作,将夏家之事搁置一边。

%44
秦鼎菱笑了一声:“夏家仙友勿忧,接下来我便要说到两天联盟之事。吴帅乃是当今魔头,入侵过天庭的罪人,天庭绝不会任由他发展下去。我等此次前来,一是为了请老祖出手相助,二是为了对付这股异族势力。”

%45
说到这里,秦鼎菱顿了顿:“天庭愿意拿出仙材,资助老祖和夏家仙友,帮助夏家收复失地。将来开战,必要时天庭必定派遣援手,相助诸位,共抗异族。”

%46
“啊!”夏家太上大长老惊叹,当即作揖而拜,“天庭高义,不愧是人族第一势力。在下感激不尽,夏家上下感激不尽!”

%47
秦鼎菱面带微笑,和夏家太上大长老攀谈了两三句,双方关系迅速拉近。

%48
天庭纵然在宿命大战中失利,也仍旧不缺资源,天庭缺乏的是人手。

%49
此番前来东海,天庭一方面利诱气海老祖,另一方面则是瞄准夏家蛊仙,拉拢东海本土势力,为将来五域乱战打下基础。

%50
此番筹谋,可谓长远。

%51
可惜的是,天庭和夏家诸仙却不知道,吴帅和气海老祖本是一体。这是一个惊天的大骗局。

%52
方源没有犹豫,点头道:“两天联军,老朽绝不会容许它们在东海存在,将来必定要将他们全部驱逐,乃至彻底剿灭!”

%53
秦鼎菱伸手:“依我方之见,两天联盟中亦有我人族势力,只是目前异族牢牢把控两天联盟而已。这些人族势力和我们并无不同,皆是人族,应当同气连枝,团结一体才是。所以要对付两天联盟,不妨先分化他们。”

%54
方源立即听出话外之音:“难道两天联盟中已有人族势力,投靠了贵方不成?”

%55
秦鼎菱却是缓缓摇头:“并非如此。而是我等之前秘密擒拿了一位八转蛊仙。她号称骷髅姥姥,乃是当今两天中的碎骨洞天之主。”

%56
骷髅姥姥被安崇说动,想要加入两天联盟,结果半途中被天庭暗中设伏,擒拿了过来。

%57
这个情报当然是夜天狼君提供的。

%58
寒灰仙姑想要联合两天中的人族势力,在两天联盟中争夺话语权,早就将夜天狼君引为助力之一。她却没有料到,夜天狼君早已经投靠了天庭。

%59
“骷髅姥姥?”方源顿了顿,询问道,“此人当下可在此处?”

%60
秦鼎菱摇头:“我天庭向来以德服人,劝说了骷髅姥姥一番后,便将她放走了。”

%61
夏家太上大长老听到这话,顿时瞪了瞪双眼,为天庭的“天真”感到有些着急,欲言又止。

%62
方源却是饱含深意地看了一眼秦鼎菱,微笑道:“看来骷髅姥姥便是当下的关键人物,而诱使她投靠我们的契机,应当还是要落在气功果之事上了。”

%63
秦鼎菱哈哈一笑,高举杯盏,称赞道:“老祖法眼,万事通透。那碎骨洞天中也出现了气功果的问题,我们若是解决了这个难题,就能拉拢骷髅姥姥乃至其余人族势力。”

%64
“妙计啊!”

%65
“拉拢了这批人族势力,此消彼长,必定能让两天联盟大摔一跤,甚至直接解体也有可能。”

%66
“非我族类其心必异,两天中的人族蛊仙毕竟和我们同宗同族,只要我等展现出热情和诚意,拉拢他们应当并不困难。”

%67
夏家诸仙议论纷纷,许多人抚掌欢笑,酒宴上的氛围更加热烈。

%68
方源面带微笑,高坐主位,看着场中的蛊仙各种神态。

%69
若是他们知道,自己的计划还未实施,就已经被敌人所知,不知会有什么感想?

%70
方源目光扫视一圈,最终又停留在秦鼎菱的脸上。

%71
“此人虽只是专修运道,但领袖天庭已算合格。”

%72
“到了她这种程度,理应达到了流派互通的程度了。寻常智道八转也不过如此吧。”

%73
“当然,她和紫薇仙子相比,还是有不少差距的。”

\end{this_body}

