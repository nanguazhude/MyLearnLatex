\newsection{非到末路不甘休}    %第一百节:非到末路不甘休

\begin{this_body}

%1
方源手中捏着一只信道蛊虫,蛊虫中记载的正是有关之前追杀战的战报。

%2
大致内容是:天庭大胜,魔尊幽魂阵亡,方源逃窜,气绝魔仙、毛里球被镇压,正元老人被俘虏,影宗紫薇仙子、影无邪在逃,气海老祖、白凝冰、白兔姑娘、妙音仙子失踪,死亡概率极大。

%3
“看来当今天下,即便没有仙蛊屋排行榜,神帝城乃是仙蛊屋第一,也是实至名归了。”方源感叹一声。

%4
在他五百年前世,五域乱战时期,有信道大能联手排出了不少榜单。每一个排行榜都竞争激烈,仙蛊屋排行榜亦是如此。

%5
但是方源五百年前世,并未出现过神帝城。

%6
方源重生带来了种种巨变,如今仙蛊屋排行榜还未出世,就有了稳居榜首的神帝城。

%7
“照这样看来,即便将来有了仙蛊屋排行榜,神帝城第一的位置也会岿然不动。”方源估量着。

%8
神帝城能够镇压毛里球、气绝魔仙,方源并不感到奇怪。

%9
神帝城乃是由豆神宫、帝君城组合而出,虽然没有九转仙蛊,不是九转仙蛊屋,但内里却蕴藏壁画世界。

%10
这是元莲仙尊施展的安居乐业。它是人道杀招,画道效果,形成众生图。很明显,它是九转层次的杀招,尊者手段。

%11
在尊者手段下,龙公败过,帝藏生跪过。气绝魔仙即便兮地没有重创,也不过是亚仙尊级别,而毛里球距离亚仙尊还有些差距呢。

%12
“我的万年斗飞车、龙宫已毁,即便存在,也不过是八转顶尖层次。而神帝城却算得上半座九转仙蛊屋。如果当年八十八角真阳楼没有摧毁,还能和它比拼争夺第一位置。”

%13
当年的八十八角真阳楼,是由长毛老祖、巨阳仙尊合力炼制。它能分化成无数小塔楼,搜集野生蛊虫。合而为一后,形成主楼,还能为王庭福地排解灾难祸患。

%14
巨阳仙尊逝去,但留下八十八角真阳楼,使得巨阳即便死了,也能操纵整个北原政局。北原举办无数次王庭之争,黄金血脉在北原仙凡两界中确立无上地位,八十八角真阳楼居功至伟。

%15
而现在,神帝城乃是中洲人脉最大集结点,是中洲最大的人才存储、培养之地,人道圣地。论格局,和八十八角真阳楼不相上下。

%16
对于这样的仙蛊屋,方源也难免忌惮。

%17
至于天庭方面,为什么要将此战战报宣告天下,方源则完全能够理解。

%18
天庭方面太需要振奋士气了。

%19
自从宿命大战战败,方源当着全天下的面,将宿命蛊摧毁,天庭就像是被抽掉了脊梁骨,失去了精神旗帜,还被方源狠狠踩踏在脚下。

%20
这个结果影响极其巨大,整个中洲都笼罩在一层阴影之下。天庭蛊仙们愤恨图强之余,也有恐惧隐藏心底。而中洲十大古派更是士气衰败,无精打采。

%21
中洲乃是四战之地,这种情况着实危险。一旦被认为虚弱可欺,等到五域乱战,就会引起四域围攻。到那时中洲再强,宛若虎豹,也会被群狼所噬。

%22
一方面稳定内部,另一方面震慑外部,秦鼎菱必须要这样做!隐藏战果,或许能有些战术上的优势,但秦鼎菱乃是天庭领袖,要着眼天下。所以,她选择放弃战术上的优势,而甘愿在大势上挽回颓势。

%23
“天庭人才济济,即便失去了紫薇仙子这等大能,也有秦鼎菱稳定大局。天庭的底蕴实实在在,不是吹嘘的。”方源叹息一声,心中颇为认可秦鼎菱的决断。换做是方源自己,也会这样去做。

%24
到达他们这等层次,五域两天便是棋盘。彼此争锋,没有格局、眼界,即便能逞一时威风,最终也会落得惨败下场。

%25
回过头来,再看这场方源追杀战。

%26
毫无疑问,最大的输家便是影宗。亚仙尊战力的魔尊幽魂阵亡,影宗蛊仙死的死,失踪的失踪,逃到逃,被抓的被抓。

%27
然后便是方源。

%28
方源损失极其惨重。

%29
如今,整个至尊仙窍惨不忍睹。原本最有生机的小南疆,沦为一片片的荒山和废墟,只剩下几处资源点。其余小四域,以及小九天也都好不到哪里去。

%30
诸多异人、人族,尽管受到方源力保,也损失不小。

%31
末代战兽王此次抵挡万劫后,至今仍在昏死的状态,并未苏醒。

%32
仙蛊的损失,是方源有生以来最为巨大的一次。

%33
首先八转仙蛊屋就损失了两座,龙宫的核心仙蛊如梦令、万年斗飞车的似水流年都保下来,各有损伤,需要休养。而其余的大多数仙蛊,都在战斗中损毁,包括防备蛊、斗蛊等等。

%34
土道蛊虫损失也不少,虽然安土重山堡仍在,但幽魂自爆的威能相当恐怖。单单为了抵挡这一击,方源就损失了七只土道仙蛊,土道凡蛊不计其数。

%35
安土重山堡因此受到了严重削弱。

%36
这段时间来,方源先后经历宿命大战、抵消万劫、方源追杀战,八转仙元已近干涸。可以说,几乎耗尽了宿命大战前的辛苦积累。

%37
十二生肖战阵刚有起色,经过幽魂一战,又得重新积蓄。

%38
逆流河还只存一丝,还是难堪运用。

%39
气海无量杀招威能暴跌谷底,再不算是一张王牌。

%40
而从魔尊幽魂身上,方源根本没有获得什么的战利品。幽魂自爆,让一切都烟消云散。

%41
不得不说,幽魂难缠至极,即便是死,也没有一点好处留给方源。

%42
追杀战后,方源最大的收获便是保全自身性命,还有三千多道天道道痕,以及气海分身。但这些不是他本来就拥有的,就是他在宿命大战的收益,或者是谋算天庭的成果。和魔尊幽魂一点瓜葛都没有。

%43
幽魂就算是战死了,也让方源感到万分难受。

%44
影宗、方源都是输家,长生天同样如此。他们失去了毛里球这样的太古传奇强者,什么战果都没有。只是和影宗、方源比较起来,损失少了许多。

%45
在此之后,便是输多赢少的气绝魔仙。气绝魔仙从魔尊幽魂、方源两边敲诈了不少传承,不乏尊者真传的内容。但他兮地损伤惨重,又失去自由,被神帝城囚禁。

%46
所以此战最大的胜者,反而是天庭。

%47
幽魂毁灭,方源远遁,损失惨重。随后天庭击败长生天,劫运坛仓皇逃离中洲。豆神宫先后镇压了气绝魔仙、毛里球,最后秦鼎菱等人俘虏了正元老人,通缉紫薇仙子、影无邪二人。气海老祖、白凝冰、白兔、妙音疑战死。

%48
这样的战果,着实不小了。对于整个中洲上下,都是一剂强心剂。

%49
天庭战报传遍天下,不只是方源在分析输赢得失,全天下的蛊仙都在分析着天下大势。

%50
其中有一人,最为欢喜。

%51
“妙,妙啊!”大厅主位上,这位蛊仙手持战报,连连称赞,眼中浮现出喜悦之色。

%52
他中年模样,相貌普通,眉细眼长,体格强健,神态举止间流露出七分霸道,三分阴狠之气。

%53
正是武家第一太上家老,八转风道蛊仙,当世枭雄——武庸!

%54
坐在大厅左手第一位的,则是武家太上二家老武八重。他在武家中战力并不突出,但资历最老,为人稳重,敢于担当,曾经在武家被各大势力联手刁难之时,挺身而出,稳住局面,乃是武家肱骨重臣。

%55
武八重恭敬地请教武庸道:“不知何妙之有?还请大人明示。”

%56
武庸放下手中信道蛊虫,目光灼灼,指点江山道:“五域两天,强者为尊。所谓天下大势,不过是最巅峰的那些强者之争。强者僵持不下,才有其余弱者,以及超级势力的发挥余地。若产生尊者,自然便是一人无敌,独领风骚,群雄垂首。”

%57
“眼下尊者不生,天下大势便在四人身上。其一魔尊幽魂,其二古月方源,其三气绝魔仙,其四气海老祖。除此之外,便是天庭、长生天这等超级势力,拥有劫运坛、神帝城这类仙蛊屋,可以参战,影响大势。”

%58
武八重闻言,顿感困惑,旋即便问:“神帝城如今已经被公认为第一仙蛊屋,可镇压气绝魔仙。劫运坛只是稍逊一筹,却也是宿命大战中大放光彩的极强仙蛊屋。大人为何将天庭、长生天排在那四人之下。”

%59
武庸微微一笑,从容答道:“这两座仙蛊屋的确强大,但也只是强盛一时。仙蛊屋最大的缺陷,便是手段固定单一。一旦被人摸清,就可针对,而仙蛊屋却难以改良,远不如亚仙尊灵活多变,可以迅速调整。”

%60
“神帝城是很强大,能镇压亚仙尊气绝魔仙。但当它的手段被摸清楚,威慑力必然下跌一个档次。天庭缺乏亚仙尊,是他们最大的弱点。此战若非气海老祖站在天庭一方,秦鼎菱必然不会有如此战果。”

%61
“原来如此。”武八重恍然,“属下现在明白大人的喜色了。这场追杀战,幽魂阵亡,方源狼狈逃窜,气绝魔仙被镇压,气海老祖失踪,战死概率极高。决定天下大势的四位亚仙尊,对耗惨烈,正是我等发展的良机啊。”

%62
武庸点头:“这等强人当然是死光了才好。最遗憾的便是方源逃出生天,唉,这个魔头深不可测,决不可以常理揣度。说不定等到他重现天日,又会比之前更加强大!”

%63
谈及方源,武庸满脸凝重之色。

%64
武八重也是忧虑重重。

%65
方源可是和武家矛盾极深,而今幽魂阵亡,方源绝对是天下第一大魔头。这要让他喘息过来,对付武家,那可就是天塌地陷般的恐怖灾难了。

%66
武庸继续道:“亚仙尊这个层次,我们是插不上手的。所以,我们要趁机全力发展,拼命壮大,将来方源来找我们麻烦,我们至少能有一拼之力。若是这种情况将来没有发生,那五域乱战也是必然大势。”

%67
“中洲、北原有天庭、长生天把守,经营无数岁月,铁通一般,暂且不谈。”

%68
“余下三域南疆、西漠、东海,皆是一盘散沙。”

%69
“时代浪潮已然掀起,武家正当乘势而起。西漠沙海遍地,绿洲如星,易守难攻。而东海却是资源丰盛,从无尊者出世,民风最为温和。我们应当先整合南疆蛊仙界,再收东海,其次西漠。囊括三域资源,助我渡劫,成为亚仙尊,再培养八转蛊仙,构造顶尖仙蛊屋。最后,再向中洲、北原动手!”

%70
“大人!”武八重还是首次听到这番言论,不由瞪大双眼,为武庸描绘的大略震撼兴奋,同时又有忐忑不安。

%71
这可是侵吞天下之志啊!

%72
武家能被武庸带上巅峰吗?又或者被时代的浪潮冲毁成渣?

%73
“风道从未有过尊者。若有可能,我或许可以弥补这份空缺。将来的事情谁说得准呢?”这句话武庸并未说出口,只是心中念叨。

%74
他端坐主位,目光似乎穿透大厅,远眺苍穹。

%75
下一刻,武八重便听武庸悠然长吟道:

%76
“永生缥缈非我求,长生无为老愧羞。”

%77
“界壁消散乱世起,宿命一去竞自由。”

%78
“鹰击长空鲸霸海,不试怎知龙与蚯?”

%79
“凡夫俗子岂识我,非到末路不甘休!”

%80
武八重听闻动容,站起身来,又旋即跪拜在地,声音打颤,发誓道:“属下愿随大人左右,为我武家霸业,肝脑涂地,不死不休!!”

%81
------------

\end{this_body}

\newsectionindepend{《蛊真人》伴随着大家一起成长}

\begin{this_body} %begin a body

%82
今天的更新,不知不觉间又到了第一百节了。最近得玉心道者的提醒,我时隔数年,再次关注了一下《蛊真人》的数据。单看本书的起点方面的数据,总推荐票数,已近三百万。起点会员点击量突破了一千万。本书写到了六百四十五万字,根据细纲推断,突破六百六十六万应该不难。

%83
让我感触最深的是,写今天这章的时候,武庸的这首唱诗几乎是我一气呵成的。当然,对仗并不工整,但感觉到位了,味道也有了,所以我就不修改了。

%84
回想从前,我写第一首唱诗的时候,那叫一个难啊。前后想了三天,这才琢磨出来一首。当时就在想,什么时候我能够写这样的唱诗,能够一气呵成呢?

%85
现在,我很想回答六年前的自己:“别着急,慢慢打磨,一点一滴积累,六年后你就有这个功底啦!”

%86
就像《蛊真人》这本书的描述内容:人创造了时代,时代又影响了每一个人。在这里,远不止是主角,所有的人都在成长。每个人的实力层次不同,想法就不同,处境不同,计划就不同。而每个人的举动,造成的结果,又在相互影响,彼此交织。

%87
正是这样,才交汇出一副乱世图,会有一个波澜壮阔的乱战大时代!

%88
我想到许多读者的留言,有不少人都在说这样的一个阅读体验:《蛊真人》陪伴了他们许多年,从初中或者小学开始看,现在是大学毕业或者成为了高中生。

%89
这种感觉,让我感到很好,平静的心湖中充盈着一股淡淡的欣慰和喜悦。

%90
这种欣慰和喜悦,来源于成长。

%91
《蛊真人》这本书,伴随着的不仅仅是书中角色的成长,也有作者我的,还有无数阅读本书的读者朋友们的。

%92
这样就很好。

%93
真的很好。

\end{this_body}

