\newsection{七七玄魂鸟}    %第一百零一节:七七玄魂鸟

\begin{this_body}

%1
“武八重,快快起来。”武家大厅中,武庸哈哈大笑。

%2
武八重顿时感到一股无形之风,力度坚韧柔和地将他搀扶起来。

%3
“有你相助,我便无后顾之忧了。”武庸态度欣慰。

%4
“惭愧。”武八重垂首抱拳,“属下修为不足,有负大人厚望。但今后只要大人一声令下,属下必定肝脑涂地,报效死力!”

%5
武八重并非逢场作戏,而是心悦臣服。

%6
武庸实乃雄主,他刚刚上位时,就铲除了安插在武家内部的奸细。

%7
虽然出了方源冒充武遗海,利用武家,暗中偷取梦境的事情,但武庸力挽狂澜,很快就将武家失地全数收复。

%8
南疆梦境一役,武庸和天庭达成协议,归还各大超级家族仙蛊,声威无两。其后,他又率领南疆群仙,围杀方源,将方源从南疆赶跑到西漠,狼狈逃窜。

%9
五行山脉一战,他在方源和陆畏因的算计下,稍稍受挫。但事后,陆畏因也没有争得过他,终究还是让武庸担当了南联盟主。

%10
宿命大战,武庸率领南联诸多精英,主动寻战,悍然出手,不顾紫薇仙子的要挟。此战之中,武庸率领众仙狂攻久久不绝连环阵,施展无限风杀招,不惜永久损耗风道道痕,不管是战斗的英姿,还是做出的贡献,都深深印刻在南疆蛊仙心中。

%11
宿命大战之后,武庸回到家族休整,多番改良杀招,乃至仙蛊屋玉清滴风小竹楼。与此同时,他还主持武家政务,将武家上下内外,打理得井然有序,蒸蒸日上。

%12
不管是自我修行、个人才情、战斗才能还是领袖之力,武庸都是武八重生平仅见的英杰。即便是他的母亲武独秀,在武八重等众多南疆蛊仙的心中,也沦为武庸之下。

%13
“我垂垂老矣,一生也就这样了。能够追随武庸这样的人物,正是我的幸运啊。真是期待武家在他的领导之下,能走到什么样的地步。”这是武八重的心底话。

%14
嗖。

%15
一声轻响,一只信道仙蛊忽然从门外飞入大厅。

%16
信道仙蛊落到武庸手中,武庸神念探索,瞬间获悉当中内容。

%17
武八重见此,心中猜测:“能用信道仙蛊传递消息,必然是重大之事。不知是好消息,还是坏消息?”

%18
旋即,他就将武庸微笑:“此乃双喜临门,你看看。”

%19
武庸说着,就将手中的信道仙蛊递给武八重。

%20
武八重接过信道仙蛊,仔细阅览,这才知道,原来之前武庸竟做了一场大局,成功引诱到了凶峡七鬼陷入阵中。

%21
“凶峡七鬼,真的是凶峡七鬼!”武八重神情激动。

%22
凶峡七鬼,是一个南疆蛊仙界有名的魔道组织,满员是七位成员。每一代的凶峡七鬼,都给南疆正道带来极大骚扰和不小的损失。

%23
凶峡七鬼气焰嚣张,也多次侵犯武家利益。同时,武家作为南疆正道的领袖,自然要挺身而出。

%24
于是,武家和凶峡七鬼便算是杠上了。

%25
每一代的武家太上大长老,都致力于剿灭凶峡七鬼。而一代代的凶峡七鬼,也始终和武家周旋,但凡武家这个庞然大物有虚弱之时,他们就像是阴暗中躲藏的恶狗,猛地扑出来,撕咬下武家的一块血肉,然后迅速远遁。

%26
武家乃是超级势力,几乎常年稳居南疆第一的宝座。但凶峡七鬼也不简单,有许多证据表明,他们的传承极可能来源于幽魂魔尊。每一代的凶峡七鬼只要有一人存活,便意味着传承不断。给存活者一段时间,他(她)的身边便会又出现另外六位魔道蛊仙。

%27
一代代的生死斗争,令武家和凶峡七鬼成为不共戴天的死仇。武家成了每一代凶峡七鬼最大的强敌,而凶峡七鬼则成为武家的心病,甩脱不掉。每一个七鬼新人成员的投名状就是报复武家。

%28
武独秀临死遗言中有三大憾事,其中首条便是凶峡七鬼。武独秀一生中,幕后指挥,甚至亲自出手,斩杀了十多位凶峡七鬼的成员,但从未有机会将他们一网打尽。

%29
只要凶峡七鬼中存活一人,隐忍暗藏一段时间后,他们就会重新满员。

%30
但是现在,按照信道仙蛊的情报所言,凶峡七鬼居然已经尽数陷入到武庸设计的陷阱之中。

%31
武八重如此激动,也就难怪了。

%32
武庸收起信道仙蛊,迈开步伐,一边向门外走去,一边嘱咐道:“凶峡七鬼是我武家必须要解决的祸患。若是放任不管,将来必出纰漏,拖我族后退。为了确保万无一失,这一次我将亲自出手。”

%33
武八重紧随其后,主动请战。这一战若能全歼凶峡七鬼,对武家意义重大!

%34
但武庸却拒绝了他:“我此去要带走玉清滴风小竹楼,正狮子搏兔亦用全力。武家大本营还需要你来镇守。”

%35
“是。”武八重无奈,只得领命。

%36
在他满怀期待、恭敬的目光中,武庸身影如风,冲天而起,迅速消失不见。

%37
武家大本营位于武仪山上,而武仪山坐落在南疆西南一端。

%38
若俯瞰整个南疆,粗略来看:在黄龙江、赤龙江两条线交叉成\texttimes{},武家势力领土,便是在这个\texttimes{}的下口。而商家则在\texttimes{}的右边部分。

%39
在所有的南疆超级势力当中,武家位于最南端。这个地理位置让武庸十分满意,皆因武家不像中洲处于四战之地。

%40
武家领地周围,几乎是在黄龙江、赤龙江的隔绝之下,地理十分有利于崛起。

%41
“现在武家最大的弱项,便是蛊仙数量不足。将来即便吞并周遭领地,也难有足够的蛊仙进行镇守。”

%42
武庸一路思索,想到这里,他就想到了方源。

%43
正是方源搅动风云,祸祸南疆,使得南疆历经多次大战。不只是武家,整个南疆蛊仙界都因此减员众多!

%44
武庸一路向东北疾飞,速度极快。

%45
耗费了一小段时间,尸山已然在望。

%46
这座江边上的尸山高达数百丈,山上生活着大量的僵尸,紧靠着赤黄江心漩涡。

%47
更准确的来讲,这座尸山名为尸皇芋顶天。一位南疆八转蛊仙转变成僵,在这里受到超级势力的联手围攻。八转仙僵战死,天长日久,战场废墟中八转仙僵破碎的尸体,在江边磅礴的灵气滋养下,再配合江水冲刷上岸的水草,渐渐生根发芽,最终长成了尸山。

%48
尸山上盛产变化道中的僵尸蛊,还有气道的尸气蛊。

%49
南疆僵盟总部在没有暴露之前,曾经看中这片宝地,开出高价,想要向武家收购,作为大本营所在,结果被武家拒绝。

%50
后来姚家从武家手中夺走。武庸拿出玉清滴风小竹楼,力挽狂澜,尸山又复归武家。

%51
因为这座资源点不仅本身产出很大,而且位于战略要冲地位,武庸便派遣武家七转蛊仙武镇驻防。

%52
宿命大战之后,五域兴起气潮。气潮一阵阵卷席天地,不仅令五域蛊仙被逼静养休整,而且还叫许多真传秘地,乃至两天洞天显形露迹。

%53
气绝魔仙曾在西漠中夺取的血气仙蛊,就是血海真传中的一道,因为气潮暴露了自身踪迹。

%54
尸皇芋顶天中,也出现了真传。种种迹象表明,这道真传正是当年那位八转仙僵大能所留!

%55
这等层次的真传,或许不入气绝魔仙、方源的眼界,但到底是八转大能的传承,足够让世间任何一个超级势力重视,凶峡七鬼动心也合情合理。

%56
武庸便因此设计布局,布下陷阱,如今终于将凶峡七鬼都勾到了瓮中,只待他来捉鳖了。

%57
尸山之巅,激战已经持续了一段时间。

%58
“好一座东南西北五向风大阵!可惜,我们凶峡七鬼对你武家了解甚深,这座大阵困不住我们!”

%59
“大阵一破,就是你武镇授首之时!”

%60
“武镇,尸皇芋顶天中出现了尸皇真传,你却隐瞒不报。就算你等到了武家援军,你今后的日子也不好过。反不如背叛武家,投靠我们。”

%61
凶峡七鬼在风道大阵中四处破坏。

%62
他们对大阵颇为了解,因为历代凶峡七鬼和五向风大阵交战的次数都不少。

%63
风道大阵已经岌岌可危。

%64
而在外操纵大阵的武镇,也是浑身伤痕,状态不佳。

%65
忽然,他接到武庸的传音,脸上顿现喜色,连忙催动手段。

%66
五向风大阵顿时一变,一道清风徐徐吹入,化为武庸真身。

%67
“什么?这大阵居然还有变化!”

%68
“武庸竟来了!武家的支援居然如此迅猛?”

%69
“糟糕,这是一个陷阱啊。”

%70
凶峡七鬼身心震动不已,惊愕之后,他们纷纷被激起了凶性。

%71
“杀!”

%72
“就算是死,也要让武家损失惨重。”

%73
“杀了武庸,武家就是一个笑话!”

%74
凶峡七鬼纷纷向武庸扑去,但暗地里他们则在商量如何迅速突围。

%75
既然是武家的陷阱,自然越是纠缠,越对他们不利。况且武庸乃是当世高手,亚仙尊之下就是他这等层次的人物了。

%76
当今蛊仙界公认的:武庸、秦鼎菱、冰塞川、千变老祖等等,属于同一个档次,都是仅次于亚仙尊的强者。

%77
凶峡七鬼自知不是武庸的对手,但他们还是有希望的。

%78
武庸为了引诱这些蛊仙,动用了五向风大阵,虽然有所变化,但根基还是原来的五向风大阵。

%79
正因为这座大阵不是战场杀招,才使得凶峡七鬼放下戒备,轻举妄动,一同出击。

%80
眼见凶峡七鬼杀来,武庸冷冷一笑:“色厉内荏之徒,不足挂齿。”

%81
下一刻,大风骤起,寒气逼人。

%82
再一刻,风声呼啸,暴雪倾泻!

%83
凶峡七鬼当中,三鬼被冻僵,而后两个呼吸,其中二鬼被直接冻死。

%84
“这是什么杀招?”

%85
“竟如此冰寒!”

%86
这是武庸最近这段时间闭关,所构想出来的杀招——雪风无归。

%87
见凶峡七鬼阵型散乱不堪,武庸眼中阴芒闪烁:“宵小无胆之辈,非我之敌。”

%88
下一刻,凶峡七鬼身上尽冒绿叶,生长出枝条。

%89
仅剩下的五鬼惨叫连连,体内生机尽数被夺取。尸体上,长出茂密葱茏的柳条。

%90
正是武庸又创的一记杀招,名为——杨柳风!

%91
七鬼阵亡,但魂魄飞出尸躯,顷刻间融汇一体,化为一头太古魂兽。

%92
魂兽形如飞鸟,通体漆黑,羽毛如墨,长有七只鸟头。

%93
武庸这才神色振奋了一下:“原来七鬼真传,便是当初幽魂魔尊企图打造传奇太古魂兽的半成品——七七玄魂鸟。楼来!”

%94
武庸一招手,玉清滴风小竹楼宛若一道碧绿流星,砸中七七玄鸟魂兽。

%95
七七玄魂鸟拼死抗争,手中杀招不断。可惜它神智时而清晰,时而混乱,最终不敌,被玉清滴风小竹楼镇压。

%96
“陆畏因,我错信了你!”七七玄魂鸟在昏死的最后一刻,发出不甘的惨嚎。

\end{this_body}

