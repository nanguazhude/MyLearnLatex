\newsection{三气仙蛊}    %第五十九节:三气仙蛊

\begin{this_body}

天庭始终充斥着光明。

方源伪装成的气海老祖,此时已经在天庭之中。

他悬浮半空,仰望面前的气功果。

天庭中结出来的气功果,巨大的远远超乎常理,仿若山峦矗立,巍峨至极。

“这样庞大的气功果,放到寻常八转洞天之中,几乎都放不下,要撑破天壁。也就是天庭有着如此雄厚的底蕴,并且又曾大量融入苍玄子的宇道果实,方才有这样的空间可以容纳气功果。”

方源心中感慨。

他上一次来天庭,是以真面目参加宿命大战,令天庭惨败。这一次来却是伪装成了天庭友人,反而是要帮助天庭解决气功果的内患。

时局易变,常常让人难以预料。

“如何?”秦鼎菱此刻就站在方源的身边,发问道。

方源微微点头:“老夫还是第一次见到如此规模的气功果,由此可见贵方底蕴果然深不可测,天下第一实至名归。”

顿了一顿,方源又继续道:“经过刚刚一番侦查,我之前的手段虽然也能生效,但这颗气功果实在太过庞大。最好还是布置仙阵,气道仙阵规模越大,越是能节省时间。”

秦鼎菱微微意外:“没想到老祖你居然连仙阵都设想出来了。”

方源摇头:“并非如此。布置出来的气道仙阵,只是增益我的气道杀招。真正铲除气功果的手段,仍旧是由我催发的气道杀招。”

秦鼎菱闻言点头,这才合理。

她笑了笑,当即掏出一只气道八转仙蛊来:“天庭的气道仙蛊并不缺乏,为了解决气功果的内患,天庭也愿意拼尽气道资源,尽全力达成此事。老祖请看这只仙蛊。”

方源的目光顿时被这只仙蛊吸引。

“这莫非就是……人气仙蛊?”方源瞳孔扩张,面世微变。

人气仙蛊形如蜜蜂,只有拇指,不断扇动翅膀,悬浮在秦鼎菱的手掌上空,浑身散发着一股氤氲之气。

秦鼎菱将人气仙蛊递给了方源。

方源将手掌摊开,用掌心接住,目不转睛看着,口中称赞:“好蛊,好蛊。此蛊和人道大有关联,发展潜力极其巨大!”

秦鼎菱微笑,又紧接着取出第二只仙蛊,向方源递过去。

这只是地气仙蛊,它宛如蝴蝶,土褐色,同样散发着氤氲之气。

方源左手一只人气仙蛊,右手一只地气仙蛊,左看右望,兴趣盎然:“妙啊,妙啊。”

“这是天气仙蛊。”秦鼎菱说着,又递给方源第三只八转仙蛊。

天气仙蛊好似蜻蜓,身体是苍绿之色,有八片浅蓝色的薄翼,轻轻扇动的时候,同样浑身上下会散发出一股半透明的气流。

方源面露郑重之色:“这只天气仙蛊和天道联系密切。实不相瞒,我开创而出的铲除气功果的手段,也是体悟到了天道的一丝奥妙,方才大功告成。若是我有此蛊,只怕能更早闭关而出啊。”

“蛊修各大流派当中,以天道、人道最为特殊。老祖你已将气道这条路快要走尽,因此触类旁通,更能得到其他流派的真意。”秦鼎菱称赞一句,话锋一转,“这三气仙蛊都是三气归来杀招的核心,乃是天庭当之无愧的重宝。唯有天庭的成员方能拥有啊。”

秦鼎菱见方源如此表现,心中非常高兴,开始暗示气海老祖想要这些蛊虫,就得加入天庭。

方源闻言,脸上顿时流露出一丝犹豫之色,心中则是笑了笑。

他刚刚的神情语态,只是表演罢了。

若他真的主修气道,那三气仙蛊的诱惑必然是无以伦比的。方源之前已经尝试过多次,这三气仙蛊乃是三气归来杀招的核心,几乎不可替代。若要强行用其他气道仙蛊取代,就算勉强改良出了三气归来杀招,那也丧失了实战价值。

所以,方源尽管此时已经早已得到了完整的元始气道真传,但真正受益的并不多。

更何况,气海老祖只是方源的一个身份而已。而且,方源手中的八转仙蛊并不缺乏。态度蛊、变异蛊、慧剑蛊、似水流年蛊、魂兽令、悔蛊、春、夏、大气、界、加、偷生、水炼、升炼、如梦令,简直是一抓一大把。

在方源眼中,也只有九转仙蛊能令他激动起来。

九转仙蛊太稀罕了。

他原本有一只,正是智慧蛊。可惜这只仙蛊早就被巨阳仙尊动过了手脚,方源不得不牺牲在宿命大战之中。

秦鼎菱暂借了三气仙蛊,让气海老祖多加熟悉,又安排了数位蛊仙,和气海老祖一同改良气道仙阵。

借助群仙之力,数日之后,气道仙阵就推敲完毕。

方源和天庭诸仙便开始布阵。

这座气道大阵的主阵之人当然还是方源。

但是其他阵眼之中会安排天庭的蛊仙镇守,一方面是天庭相助方源,另一方面则是起到监督的作用。

气海老祖毕竟是外人,虽然签订了盟约,但仍不失天庭成员。天庭若无防范怎么可能?

事实上,不只是辅阵的蛊仙进行防范。

这座气道大阵的核心仙蛊——天气、地气、人气三气仙蛊,也都是秦鼎菱暂借给方源所用。一旦有所异变,天庭就能立即回收,让方源失去整个气道大阵的掌控。

而这座气道大阵,改良的过程中有数位天庭蛊仙全程参与。就算这些蛊仙气道造诣薄弱,但到底还是有着眼光,想隐瞒这些人很难很难。事实上,方源也没有在这座气道大阵中隐瞒什么东西。

大阵绝对是没有问题的。

“地气蛊,去!”方源首先抛出地气仙蛊,灌输仙元。

地气仙蛊迅速飞出,然后一头钻入气功果下层的地底中去。

轰!

忽然一声巨响,随后气功果下的地面开始剧烈颤抖,无数的地气宛若浓烟升腾而出。

方源开始收摄这些地气,在他身旁的天庭蛊仙们也纷纷出手,有的辅助收取地气,有的则开始布置小阵,在地气升腾的洞口边缘做文章。

就在方源布阵的同时,在天庭的另一端的地底深处,有一个巨大的地下广场。

青仇小山般的身躯,就静止在地下广场的深处,宛若石像一动不动。

轰……

忽然,一阵轰鸣声从远处传来。

地下广场微微震荡,地砖瞬间裂开数条细缝。

但镇守这里的赤心行者临危不乱,立即出手,稳定局势。地砖上的裂缝又以肉眼可见的速度迅速消散。

“什么动静?”青仇忽然开口,但古怪的是,它的话音却是九灵仙姑的声音。

九灵仙姑竟深入到了青仇体内,强行炼化仇恨蛊。

“看这番动静,应当是气海老祖开始布阵了吧。他采用了地气仙蛊,作为大阵的第一个根基,所以牵动了地脉。而我们这里正是主地脉之一,因此能影响到我们。”赤心行者开口回答,“之前你正值关键时刻,所以这个消息我就没有通知你。你尽管放心,这种影响程度干扰不到我们的。”

“嗯。”九灵仙姑立即放下心来,“那我们也抓紧一些,争取在气功果铲除的同时,车彻底炼化了仇恨蛊,给天庭喜上加喜。”

天庭蛊仙们都已从宿命大战的战败阴影中恢复过来,知耻后勇,他们远比之前要更加努力,斗志满满。

数日后,气道大阵搭建成功,方源主阵,催动手段,开始铲除气功果。

气功果在方源的杀招下,被不断地消磨。只是天庭的这只气功果着实太巨,要彻底铲除它需要一段时间。

如此又过数天,气功果被削了一成多一些。

秦鼎菱匆匆来到方源面前,带着一脸凝重之色:“老祖,东海生变,情况不妙。”

方源一惊:“发生了何事?”

“吴帅联盟忽然大举进攻,兵分三路,狂猛突袭投靠我们的洞天势力。猝不及防之下,我们的损失有些惨重。”秦鼎菱道。

“我秘密前来的消息,应当是封锁了的。怎会如此?吴帅知晓的竟如此迅速?”方源变色,语气中流露出一抹焦躁的意味。

“老祖是想回援东海吗?”秦鼎菱故意询问。

“正有这个想法。”方源紧皱眉头,“只是我这个手段,非得竟全功才行。若是半途而废,恐有不良异变呐。”

秦鼎菱对方源深深一拜:“还请老祖以大局为重!吴帅反应如此迅捷,出乎所有人的意料。但是眼下若是老祖回援,恐怕是遂了吴帅的心愿。只有当我们彻底铲除掉气功果,方能有一个安全的大后方。届时,老祖你引领我天庭诸仙杀回东海,就说是特意来我天庭请援,不想吴帅无耻偷袭,便可堵住悠悠之口。再然后结合你我双方力量,将吴帅打杀,彻底收拾了两天联盟。这才是取胜之道啊。”

方源不语,只是深深地看向秦鼎菱。

良久,他掌控嘴唇,语气沉重:“吴帅一方大举进攻,绝对是准备良久。他们筹谋划策,天庭是否知晓一些风声?”

“绝无此事。”秦鼎菱摇头,态度坚决。

方源冷哼一声,神色挣扎了好一会儿,这才无奈一叹:“也罢,我就留在这里主持大阵,直到彻底铲除了气功果。只是这段时间,还需你天庭派遣援手,支撑东海大局!”

秦鼎菱也跟着一叹:“这件事情我方定尽力而为。”

方源咬了咬牙,又道:“一旦彻底消灭了气功果,老夫就得立即回援。然而终日不眠不休操持大阵,战力必会下滑。还请天庭出让三气仙蛊,助我杀敌取胜。”

秦鼎菱面色微变,方源索要酬劳的语气虽然委婉,但胃口着实太大了,大大超出秦鼎菱的心理承受底线。

她摇头拒绝:“三气仙蛊乃是我天庭重宝,非是天庭成员方可……”

她还未说完,就被方源打断:“东海诸仙也都是老夫的麾下啊。”

秦鼎菱仍旧摇头:“老祖的牺牲我铭记于心,但我虽是天庭领袖,却无法出让三只八转仙蛊。我最多只能拿出其中一只来作为补偿。”

“一只岂能抵得上洞天中的万般生灵?”方源微怒。

轰!

正在两人讨价还价之时,忽然苍穹一声巨响。

一座八转仙蛊屋径直撞了进来,来势极其凶猛!

天庭众仙无不惊愕。

皆因侵犯天庭的仙蛊屋正是龙宫!

“怎么可能?!”秦鼎菱再顾不得和方源商量,飞出大阵,前去迎敌。

龙宫大门敞开,吴帅端坐龙椅,从龙宫深处看来:“秦鼎菱,我们又见面了。”

“吴帅,你何以至此?”秦鼎菱喝问,身后已经汇集了数位蛊仙。

吴帅冷笑一声,将脚边的一颗头颅踢下。

这颗头颅顺着龙椅下方的阶梯,一路滚到大殿中央。他死死的瞪着双眼,赫然便是夜天狼君。

“气海老贼能铲除气功果,你天庭就想隐瞒消息,偷偷解决内患?呵呵,想的真美!我这次来,就是将你天庭捣毁,顺便再杀了气海老贼。看这天下谁能阻我崛起!”

吴帅说到这里,猛地站直身躯,手掌一挥,口中低喝:“杀!”

下一刻,大殿中的群仙汹涌而出,带着凶狠的神色,悍然扑杀下去。

------------

\end{this_body}

