\newsection{悲风山脉王小二}    %第七十三节:悲风山脉王小二

\begin{this_body}

%1
“难。”方源深深一叹。

%2
几乎话音刚落,外界幽魂的攻势再度猛烈起来!

%3
幽魂冷喝:“方源,你真是令我太失望了。你还要躲避到什么时候?实话告诉你,你的位置已在刚刚,被紫薇仙子推算出了。”

%4
“寄希望于欺骗,这纯粹是弱者的行径。哼,我将至尊仙胎蛊安排给你,简直是明珠暗投!”

%5
“出来与我一战,至少你死后,世人不会说你懦弱无为,欺软怕硬,贪生怕死。”

%6
“可恶!”吴帅的念想被打断,不由咬牙切齿。

%7
方源冷哼一声,眉头紧皱,并不答话。对他而言,名声算得了什么?方源在意的是,紫薇仙子是否真的推算出来?还是幽魂在刻意哄骗自己,就连他刚刚的犹豫都是表演,都是对方源的试探呢?

%8
气海分身已经离开那边,具体的情况方源并不清楚。

%9
“就算此刻,那边的智道大阵还在运转着。也不能排除一种可能紫薇仙子凭此拖延天庭、长生天,暗中提前告知幽魂推算结果。唉,不能管这些了。”方源叹息一声,很快就将这个问题抛之脑后。

%10
不是他找到了真相,而是龙宫目前的状况,已是岌岌可危,非得他出手不行了。

%11
方源打开仙窍门户一丝缝隙,放出早已准备好的太古年兽!

%12
数头太古年兽嗷嗷乱叫,冲出龙宫,一些试图冲垮黑气涡流,另一部分则向着魔尊幽魂杀去。

%13
“这都是无用之功。”魔尊幽魂出手,全身黑烟飘飞,掀起惨淡阴风,笼罩方圆千里。

%14
他手指连点,每一击都只是一缕黑烟,偏偏速度奇快,威能强悍。黑烟射中太古年兽,立即膨胀,化为粗壮藤蔓似的浓烟锁链,将太古年兽禁锢。

%15
太古年兽挣扎,黑烟锁链却是得到黑烟涡流的增幅,越发粗壮,越锁越紧。太古年兽起先还能挣扎,很快就动弹不得,沦为标靶,随后被涡流卷席而出。

%16
“这都是小把戏,你真是令我太失望了,方源。”魔尊幽魂飞悬高空,俯视脚下的黑烟漩涡。

%17
堂堂八转仙蛊屋龙宫仿佛就是一个玩具,一旦被他摸清底细,就只能任凭玩弄,操弄于鼓掌之中。

%18
吴帅咬牙,催动梦里轻烟杀招。

%19
魔尊幽魂嗤笑一声,张口吐出数团魂球。魂球带着尾巴,形似蝌蚪,在空中甩出幽暗的魅影,迅速扎进之前的太古年兽身中。

%20
太古年兽身上,迅速浮出一层浅薄黑气。黑烟锁链在黑气中迅速融化,令黑气在眨眼间浓郁十多倍。

%21
太古年兽重获自由,却发出怒吼,杀向龙宫!

%22
其中一头太古年兽更是直接对准梦里轻烟杀招撞去。

%23
这些太古年兽竟都被魔尊幽魂当场策反。

%24
梦里轻烟卷走了太古年兽,吴帅深深地感到无力。现在幽魂掌控了战局,有了充沛的空余和主动,梦里轻烟杀招已不能对他构成威胁,沦为一个小麻烦。

%25
魔尊幽魂现在展现出了这几个魂道杀招,都是方源、吴帅没有见过,甚至从未听闻的手段。

%26
九转尊者的底蕴深不见底!

%27
高空中,智道大阵仍旧在运转不休。

%28
紫薇仙子坐镇中枢,仍旧一副全力推算的模样。

%29
长生天、天庭诸仙围绕着智道大阵。

%30
秦鼎菱沉默地望着智道大阵,就在刚刚天庭那边传来消息,留守的几位蛊仙成功地将帝藏生镇压。

%31
帝藏生身中元始气道杀招,状态本就跌落到了谷底。方源逃走的时候,直接将它舍弃了,顾不上它。此次被天庭蛊仙擒拿镇压,并不奇怪。

%32
由此一点,也可看出方源的仓皇。他连帝藏生这样的存在,都顾及不上,正说明了他此刻的软弱。

%33
也正是这个事实,让秦鼎菱对方源大起杀心。

%34
如此良机,可谓千载难逢!

%35
镇压了孽龙之后,留守的蛊仙们开始积极修复天庭洞天。破漏之处,被迅速弥补,天庭此番劫难已算过了大半。

%36
这个情况让秦鼎菱松了一口气,但还有其他的担忧。

%37
“气海老祖远去不久,按照龙宫的速度和方向,此时已经到了肠廊了吧?”秦鼎菱暗自估算。

%38
方源逃窜,驾驭龙宫出了太古白天,直接往中洲而来。这让天庭一方又急又怒,都在担心自家地盘遭受战斗余波,损失重大。

%39
肠廊是悲风山脉的北方出口,乃是上古年间,一位兽人强者身陨后,结合大地山峦所化。由中洲十大古派之一的黑天寺掌控。

%40
秦鼎菱早已将情况传讯到了黑天寺,让他们出动仙蛊屋,在肠廊一带自由行动。若有便宜就出手,若是敌势太强,那就直接舍弃肠廊,保存自身实力最为重要。

%41
黑天寺接到天庭之命,如临大敌,连忙调动几乎全派力量,展开行动。

%42
可是他们左等右等,都未等到龙宫和幽魂。

%43
秦鼎菱不清楚的是,龙宫早在半路上就被魔尊幽魂截断,眼下还在悲风山脉的上空。

%44
悲风山脉十分广袤,绵延十多万里,总体而言山势较为中洲其他山脉显得平缓。这里资源并不丰富,许多山峰都是光秃秃的一片。所以尽管位于黑天寺的势力边缘,也未被黑天寺重视。

%45
这里最多的是风道资源,当初风云府的一位七转蛊仙来到这里潜修多年,最终一举突破,成为八转蛊仙。此人正是悲风老人,可惜命途多舛,他在宿命大战被方源俘虏,而后就被龙宫奴役。在刚刚的激战中,悲风老人已然战陨。

%46
不仅是他,四大龙将在这一战中,尽皆牺牲。就连最强的帝藏生,如今都被天庭镇压囚禁。

%47
悲风山脉的某处山头。

%48
几个放羊的少年正发生着口角。

%49
被欺负的只有一个少年,他最为瘦弱,在指责声中连连后退。

%50
“王小二,这处山头已经被我们包了。没有你的地方。”

%51
“你快给我们滚,滚的越远越好。”

%52
几位少年身穿厚实的皮袄,身材粗壮,满脸憎恶之色。

%53
王小二的衣服不仅打着补丁,有许多地方是破烂的,仿佛乞丐。

%54
他弱弱的抗议道:“可是我的羊也要吃草啊。我如果不让羊吃饱,回家后会被舅舅、舅母打的。”

%55
几个少年哈哈大笑,其中一个最为粗壮,狠狠出手,将王小二推搡在地。

%56
“你去隔壁山头放羊吧,那里还有点草。快滚!再不滚,我就把你的腿打断!”粗壮少年恶狠狠的威胁道。

%57
王小二挣扎着起来,他不敢再抗议,只得驱赶着自己的那一小群羊,离开这座山头。

%58
他迈开幼小孱弱的腿脚,在穷山恶水中跋涉。山上根本没有路,王小二时常跌倒,在坚硬的怪石上摔得一块块青紫,反倒是他身后的羊群走得自在悠闲。

%59
王小二好不容易寻到了一小块草地,羊群跑了一阵,又渴又饿,不需要他驱赶,直接蜂拥而上,争抢吃食。

%60
王小二累得瘫坐在一块巨石上,望着羊儿哄抢,心中哀叹:“最近,山间的悲风越刮越猛,草地越来越少了。所以,他们驱赶我是不想我的羊分了那片草地。”

%61
草地很小,羊群吃了一阵,就啃噬光了。壮硕的羊吃的最多,剩下大部分的羊仍旧饿得哼叫。

%62
王小二摇头不已,这座山中草地稀少,而且四处分散。要将这些羊群喂饱,非得辗转山头。不仅耗费时间,更浪费体力。

%63
“羊啊,羊啊,你们至少还有的吃。今天我把你们吃饱了,肯定回去得晚。一定又会被舅舅、舅母打骂,只能吃馊饭馊菜了。”

%64
即便中洲乃是五域中最为发达的地方,但对于广大的凡人而言,生活仍旧困苦。吃不饱是常有的事情。

%65
就在这时,忽然有隆隆轰鸣之声传来。

%66
王小二抬头望天,还奇怪怎么晴空万里,却有雷声作响呢?

%67
然后,他就逐渐张大了嘴巴,看到高空中一颗流星坠落下来。

%68
流星在他的视野中越来越大,风压迫来,轰鸣声震耳欲聋,空气中的温度也迅速升高。

%69
王小二吓得呆立在原地,而身边的羊群感受到了危机,嗷嗷叫唤,四处乱奔。

%70
不只是羊群,山上的野物也四处狂奔。原本贫瘠的山头,忽然变得热闹嘈杂。

%71
流星越来越近,王小二看清楚了。

%72
这颗“流星”竟是一头山般巨大的野兽!

%73
野兽轰的一声,砸落下来。好巧不巧,正是王小二生活的那座山!

%74
山峰崩裂,乱石飞溅,地动山摇,烟尘四起。

%75
山上的小村庄在顷刻间,被撕扯被摧毁。人的惨叫声,混合着凄厉的兽鸣,顺着狂风气浪,隐隐传到王小二的耳中。

%76
轰隆、轰隆、轰隆……

%77
在王小二惊恐至极的目光中,巨大的裂痕向四周迅速扩散,很快蔓延周遭。周围的山峰也跟着龟裂,然后一座座接连倒塌下去。

%78
之前驱赶王小二的那些少年,因此尽数罹难!

%79
“我也要死了吗?”王小二一屁股坐在地上。

%80
他像是被抽空了所有的力气,只能眼睁睁地看着漫天的烟尘,好似史前巨怪一般扑来,要将他彻底吞没。

%81
烟尘中,风浪夹裹着无数的碎石。

%82
一块碎石砸中王小二的额头。

%83
他昏死之前的最后一个念头,却是在疑惑怎么从天上就掉下一头巨兽来呢?

\end{this_body}

