\newsection{中计}    %第八十节:中计

\begin{this_body}

%1
气绝魔仙赶来,立即对气海老祖动手。他头顶上的天地秘境兮地,实在太过克制气道了。

%2
气海老祖本是龙宫最依赖的助力,此刻被压制,让龙宫的处境雪上加霜。

%3
不过很快,万年斗飞车也紧随而来,全力作战。方源这边险象环生,却因此始终艰难地支撑下来。

%4
魔尊幽魂冷喝一声,身影宛若鬼魅,手掌再次拍中龙宫。

%5
龙宫表面迅速浮现无数魂兽的身影,一轮下来,如走马观花。龙宫崩碎,内外皆受到重创。

%6
龙宫催出梦里轻烟杀招,幽魂迅速撤离。

%7
下一刻,气绝魔仙忽然手指指向龙宫。

%8
气流猛地汇集,然后在瞬间爆炸。

%9
龙宫外部廊亭早已经被炸得全毁,只剩下一个光秃秃的主殿。但此刻遭受气绝魔仙的一记重击,大殿的屋顶中终于现出一个破洞,内外可见。

%10
魔尊幽魂见此,冷冷一笑,不吝夸赞:“气绝,做得不错!”

%11
到了这种程度,任凭谁都能知晓龙宫已然岌岌可危了。再加把劲,打碎龙宫便在顷刻。

%12
方源、气海老祖之间可以相互配合,气绝魔仙和魔尊幽魂之间同样可以!

%13
气绝魔仙参战之后,力压气海老祖,和幽魂相互配合。虽然并不默契,但两人战斗经验极其丰富,总能找到合击的地方。

%14
而正是这个配合,将龙宫推向了万劫不复的境地。

%15
当紫薇仙子、秦鼎菱、冰塞川等人赶来时,正好看到龙宫崩坏成无数碎片,四处飞洒。

%16
龙宫大殿中的方源、吴帅二人露出真身,但旋即又被万年斗飞车接应,双双入内。

%17
万年斗飞车可比龙宫快得多,向前一窜,继续飞逃。

%18
三方群仙为之一愣,旋即大喜。

%19
“快,方源真的不行了!”

%20
“龙宫竟被摧毁!!”

%21
“万年斗飞车虽快,但论防御,还逊色于龙宫。”

%22
“一定要除掉方源,洗刷我天庭的耻辱。”

%23
“方源摧毁了我王庭福地,和我长生天有深仇大恨。杀掉他,夺取他身上的一切真传和蛊虫。再将他的魂魄抽取出来,拷打折磨千万年!”

%24
“方源,终于到了影宗来收拾你的时候了。”

%25
“先杀了方源,再来相互竞争。绝不可给方源任何浑水摸鱼的机会!”

%26
群仙蜂拥而至,都对方源展开凶狠的追杀。

%27
方源遭遇到了重生以来最大的困境。他遭受天下排名前三的超级势力追杀,身后是当今蛊仙界最强的一小撮强者。秦鼎菱、紫薇仙子、冰塞川都是历史上留名的蛊仙,更有魔尊幽魂这样的恐怖存在。

%28
“方源,你是逃不了的。”魔尊幽魂心中冷笑,当他看到所有人的火力都集中在方源身上,便没有驱除其他人,而是任凭众仙追杀。

%29
“都留神留力,戒备幽魂和长生天。”秦鼎菱吩咐下去。

%30
这真是一副奇景。

%31
天庭、长生天、影宗这三方联合起来,共同追杀方源,但又彼此忌惮,刻意留手。

%32
方源没有营造出三方内讧的局面,皆因这三方都不是普通蛊仙。同时,方源和他们三方结的仇太深了,他身上的利益也太大了。

%33
即便三方刻意留手,火力也足够凶猛,杀招也足够奇妙诡异。

%34
万年斗飞车飞了没多久,便伤痕累累,破败不堪,飞行的途中洒下无数蛊虫死尸碎片。

%35
“这样下去,根本没有出路!”吴帅大吼,“本体,我们北上的路已经被彻底封死了。眼下只能被逼着逃向西南。这是毫无希望的,我们支撑不了多久,万年斗飞车远比龙宫更脆。”

%36
方源本体盘坐着,一面抵抗内部万劫,一面指挥道:“那就钻入地下,看看有无生机。”

%37
万年斗飞车在高空中连续急晃,晃过魔尊幽魂催发的一记黑色魂球,却仍旧被其余漫天的攻势覆盖。

%38
白银色的船体在恶浪般的杀招下艰难抵御,很快,一部分的船体黯淡下去,然后化为白色的灰,在高空飘洒。最后在瞬间,被狂风卷席,彻底消失得无影无踪。

%39
万年斗飞车宛若不断受伤的飞鱼,一个猛子扎了下去,几乎是向地面坠落而去。

%40
“跟上它!”

%41
“再有三轮攻势,这艘破船定会被摧毁。”

%42
“方源必须死!”

%43
“若是我等争抢不过,那就连至尊仙窍也一起毁灭,千万不能便宜外人。”

%44
三方势力几乎都有相似的沟通交代。

%45
轰!

%46
万年斗飞车宛若一颗银白色的流星,直接坠入地下深处。

%47
烟尘爆涌,大地震颤,但却震慑不住方源身后的追兵,他们顺着万年斗飞车撞出来的地洞,鱼贯而入。

%48
轰轰轰……

%49
种种杀招打来,不断地攻击在万年斗飞车之上。

%50
万年斗飞车迅速解体,危如累卵。

%51
“哈哈哈,方源昏聩了。在这地底下,万年斗飞车行动不便,让我们能轻易打中。”

%52
“上天无路,入地无门!方源你这个魔头,屠戮无辜,危害天下,今天的情景就是你种下的恶果啊。”

%53
“触犯了我们,这就是你的凄惨下场!”

%54
群仙怒吼。

%55
方源的落败逃窜,让他们士气几乎暴涨到了极点。

%56
再努力一把,方源的死亡已近在眼前。

%57
万年斗飞车中,吴帅满身血迹。在这等狂轰滥炸之下,万年斗飞车根本来不及修复。

%58
轰隆隆……

%59
群仙酝酿片刻,再度纷纷出手,掀起攻势狂澜。

%60
无数杀招相互间隔,尽力避免内耗,仿佛海啸时的恐怖巨浪,一波波一层层地向万年斗飞车卷席而去。

%61
“死吧!”

%62
砰。

%63
生死存亡之际,万年斗飞车却是撞断前方的土石,陡然闯入到一个空阔的地下空间。

%64
万年斗飞车有了空间迅速腾挪,立即避开绝大多数的杀招,险死还生。仿佛是差点要溺死的泳者,忽然吸收了一点空气。

%65
“怎么回事?”

%66
“我们是追到了地渊了!”

%67
“难道这就是方源的打算?”

%68
“追,绝不能让他跑了。”

%69
群仙惊愕了一下,旋即冲破土石,纷纷杀进地下空间。

%70
原来方源被魔尊幽魂追赶,先是从中洲北部的悲风山脉往东南突破。气绝魔仙从东边赶来,配合幽魂,将方源往西南逼迫。最后,紫薇仙子等人从南边参战,让方源不得不选择向西。

%71
这一路向西,终是到了古魂门的领地。

%72
古魂门乃是中洲十大古派之一,历史十分悠久。古魂门的驻地之下,有一座巨大的地下深渊。

%73
地下深渊的每一层都至少有数亿亩的面积。空间十分广阔,深幽神秘,各种溶洞甬道,有的仿佛迷宫,有的积成巨大的地下湖泊,有的空旷如平原。

%74
这里面生活着成千上万种的生物,在方源前世,从地渊中涌出恐怖的兽潮,祸及整个中洲。

%75
中洲十大古派花了数年时间,才剿除了中洲地表上的兽潮。而后不顾阵营之分,广邀正魔两道的蛊师、蛊仙,一起深入地渊探索。

%76
结果发现地渊几乎深不可测,一层又一层。直至方源重生,已经发现了一百零七层。还有多少层,无法估量。

%77
“这里环境复杂,生物繁多,越是往下太古荒兽就越多,的确是眼下最好的逃生场所了。”万年斗飞车中吴帅心中生起明悟。

%78
依靠着地渊,万年斗飞车的处境开始好转。

%79
但对于方源的谋算,以幽魂为首的追兵也是了然。

%80
许多蛊仙都不再留手,手段尽数使出。他们之前相互戒备,但此刻追杀到这里,谁也不甘心放弃。

%81
方源处境仍旧险恶,万年斗飞车的损毁速度缓慢了许多,但仍旧坚定地朝着毁灭的悬崖边缘滑落。

%82
方源和吴帅合力,竭尽心智,最大程度上利用地利,仍旧甩开不了追兵。

%83
充其量,只是将追杀的队伍拉开了层次。

%84
处于第一梯队,最接近万年斗飞车的,便是魔尊幽魂。

%85
随后是气绝魔仙、气海老祖、青仇。

%86
第三层次的则是天庭、长生天和影宗紫薇仙子等人。

%87
砰!

%88
濒临崩溃的万年斗飞车再次撞破土石,深入下一层。

%89
无数蝙蝠飞舞,惊惶逃窜。

%90
这一层中,生活着亿万的蝙蝠,宛若黑色的潮水,磅礴浩瀚。

%91
普通的蝙蝠被万年斗飞车撞成一滩滩的碎肉,而被惊扰的太古黑蝠却是逆着潮流杀来。

%92
在这一层中,黑蝠群是当之无愧的霸主。它们被万年斗飞车惊扰,势要惩治这个胆大包天的入侵者。

%93
“嗯?”魔尊幽魂随后赶到。

%94
他看到满天黑蝠,顿时惊喜交加。

%95
“方源,天不作美,看来这一层就是你的埋骨之地了。”魔尊幽魂忽然张口长啸。

%96
啸声传播开来,所到之处黑蝠瞬间丧命,即便是上古黑蝠也不能幸免。唯有太古黑蝠身姿踉跄,在空中艰难扑腾,勉强支撑。

%97
魔尊幽魂啸声一扬,从这些黑蝠的尸体中冒出无数的蝙蝠魂魄。

%98
这些魂魄又在啸声中迅速交融,化为漫天黑雾笼罩万年斗飞车。

%99
万年斗飞车在黑雾的侵蚀下迅速瓦解,再不能支撑下去。

%100
“收拾核心仙蛊,准备弃船!”方源本体果断下令。

%101
就和之前舍弃龙宫一样,他们在最后时刻尽全力回收核心仙蛊。龙宫的核心仙蛊如梦令,已经被顺利回收。

%102
但兵凶战危,仙蛊屋面临被摧毁的情况下,回收核心仙蛊也只是尽人力、听天命的事情。甚至有时候,因为太过急迫,过快地抽取出了核心仙蛊,反而令核心仙蛊受伤极重,在事后损毁。

%103
下一刻,万年斗飞车彻底崩溃,方源本体和吴帅彻底置身在黑雾之中。

%104
“他来了!我来挡住他。”吴帅挡在了方源身前。

%105
“我说过,这一路逃窜都只是徒劳的挣扎。”幽魂迫近,语气冷酷如冰。

%106
气海老祖等第二梯队的人刚刚赶来,正好看到漫天的黑雾缭绕,银白碎片点缀其中,不断消散,幽魂正杀向方源本体的这一幕。

%107
“不!”气海老祖鞭长莫及,只能在心中怒吼。

%108
死亡降临,方源却忽然微微一笑:“是你中计了,幽魂。”

\end{this_body}

