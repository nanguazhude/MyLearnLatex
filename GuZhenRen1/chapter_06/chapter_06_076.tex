\newsection{幽魂尊位}    %第七十六节:幽魂尊位

\begin{this_body}

%1
雷声轰鸣,震撼摧残着至尊仙窍。

%2
小南疆在万灭雷森的肆虐之下惨不忍睹,无数山峦倒塌,大地被劈得龟裂,坑洞无数。战兽王一直浴血奋战,他的身上伤势越来越重,处境越发艰难。

%3
他在用自己的性命死撑,但难挡大局。

%4
“和天地之威相比,我纵然是八转蛊仙,又算得了什么?”战兽王老者在至尊仙窍中勤修苦练,眼界大开,实力大进,自信十足。但此刻在万劫之下,他充分感受到了自己的微渺和弱小。

%5
万灭雷森覆盖的范围又扩张了,而方源本体却要应付外面的战局,这就令他非常吃亏。对于万劫,方源全力以赴还很危险,现在他的精力却被牵扯了大半,必须去和幽魂激战。

%6
这可是万劫,多少八转仓皇狼狈地倒在这道关卡上!

%7
方源的情况还要更加特殊,这不是他的第一场万劫。他不只是连续跳过了两场,而且这场万劫是他身怀的三千多道天道道痕的猛烈演变。

%8
如此形成的万劫,即便在漫漫人族历史之中,绝对是十分罕见了。

%9
情势所逼之下,方源不得不一心二用。他一面应付正面战局,一面洞察至尊仙窍。

%10
他看着天道道痕转变成雷道、云道,形成这场万劫,忽然有了明悟。

%11
“我有些明白了。”刹那间,他明白了灾劫的本质!

%12
以前,他就已经知道:灾劫受到天道控制,而天道讲究平衡,损有余而补不足。每一位蛊仙和仙窍,都是道痕大量凝聚,这就成了天道必须平衡的对象。

%13
所以每隔一段时间,天道就发威,形成灾劫为难蛊仙。但其中为何能演化成种种灾劫?方源并不清楚。

%14
现在方源心底的这一层疑惑,已然被眼前的景象彻底点透。

%15
原来不管是什么样的灾劫,其实都是天道道痕的变化。天道道痕可以变作任何流派,所以才有了飞霜跳雷劫、四炎云盖劫、龙吟劫、天鼓雷音劫、风花劫、雪月劫、玄白飞盐劫……

%16
这些灾劫,涵盖了各种流派,种类繁多芜杂,根本让蛊仙无从猜测,应对艰难。而天道甚至还会仿效尊者的手段,比如盗天魔尊的无相手。

%17
各种类型的灾劫只是表面,本质上它们都是天道道痕!

%18
蛊仙修为越高,积蓄的道痕就越多,和自然环境形成的差距就越大。越不平衡,引出的天道道痕数量就越多,从而形成的灾劫就越强。

%19
世间无人能够修行天道,所以就算蛊仙渡劫成功,也并不能存储天道道痕。这些天道道痕都转化成其他流派的道痕,在蛊仙仙窍中存留下来。

%20
回想一下,在疯魔窟中,天道道痕能衍化各种奇怪的小世界,而这些小世界破灭之后,便会还原凝聚成天道道痕。

%21
而方源从琅琊派中所得的仙劫锻窍杀招,其实原理也相关联。

%22
仙劫锻窍杀招是将福地洞天当做炼制的本体,通过杀招和仙窍之外的五域天地勾连在一起,从而影响灾劫,并利用灾劫锻炼仙窍本身。

%23
琅琊福地曾经寄托在北原月牙湖。月牙湖附近,充斥着浓郁的水道道痕以及炼道道痕。每次都运用仙劫锻窍杀招应付灾劫。通常都会形成和水道、炼道有关的灾厄。渡过之后,仙窍中就会增添水道、炼道的道痕了。

%24
这招其实就是影响天道道痕演变的方向。

%25
巨阳仙尊的真传中,也有运道手段来削弱灾劫。比如狗屎运、鸿运齐天。这些便是运道道痕参与天道道痕演变灾劫,从而发生对蛊仙有利的变化,让灾劫威能下降。

%26
兽灾洞天中的杀招万物大同变,是限定天道道痕的转变结果,将每一次灾劫都转变成变化道的兽灾。

%27
方源从影宗真传中获取的石洞天机杀招,能够准确地推算出灾劫内容。石洞天机杀招以天机仙蛊为核心,很显然也是从根源——天道道痕出发,进行推算。所以推算的结果才如此精确。

%28
仙劫锻窍、运道手段、万物大同变、石洞天机……这些手段都在方源手中,但很遗憾,他都暂时运用不了。

%29
天道道痕束缚着他,纠缠着他,让他很难自如地催使这些杀招。

%30
唯有用人道手段抗衡天道,方能给方源争取出一些空间和时间,来催动其他杀招。就像之前在天庭中,方源动用无量气海杀招破解元始气墙。

%31
方源在至尊仙窍中布置了人道大阵,以千愿树为核心。但天意并不蠢笨,早已熟知方源的这层底细。万灭雷森同样出现在了小中洲,开始对人道大阵狂轰滥炸,让方源更加被动。

%32
嗷吼!

%33
兽吼连连,一大群魂兽出动,冲向高空的雷云。

%34
它们冲势很猛,完全不顾及自身安危。

%35
这些都是魂兽,大量的荒级魂兽,上古魂兽不在少数,而太古魂兽则有四头。

%36
这些魂兽大多是方源从青鬼沙漠中捕捉的。其中的这几头太古魂兽,还是方源帮助房家夺取豆神宫一战中的战果。

%37
这些魂兽一直都被寄养在小黑天中。

%38
此时方源面临难关,只得动用这部分的底蕴。方源出手十分干脆果断,直接将所有的魂兽都调集过来,拼死也要冲散雷云,来给他反击争取机会。

%39
魂兽在万劫之下显得脆弱不堪,仿佛纸糊的一样。不一会儿功夫,就有近百头荒级魂兽在雷击下灰飞烟灭,上古魂兽也折损了十几头。

%40
方源心若冷铁,毫不动摇。

%41
魂兽群已经起到了作用,吸引了大量火力,给方源分担压力争取时机。

%42
这些魂兽即便包含了太古魂兽,然而没有仙蛊傍身,在万劫面前也只是一群高级炮灰。不过这也算是物尽其用了。

%43
方源根本不敢用魂兽群,去对付魔尊幽魂。

%44
事实上,自从那些太古年兽都被幽魂策反后,方源就彻底息了这方面的心思。

%45
一股股庞大的力量宛若汹涌的潮水,涌入到幽魂的体内。

%46
青仇剧烈挣扎,催发一记记杀招,疯狂地轰击自身。然而魔尊幽魂硬是扛着青仇疯狂的反扑,不断汲取它的力量。

%47
天地间之所以有魂兽产生,根源就出在魔尊幽魂开创的魂道。

%48
幽魂对魂兽太了解了,不仅能令魂兽忠心耿耿,还有手段从魂兽身上汲取力量增补自身。

%49
青仇挣扎的力道越来越弱。

%50
魔尊幽魂看着青仇大感满意,青仇体内的仇恨蛊气息已经开始接近圆满。这蠢货简直是送命又送宝。

%51
幽魂一面牢牢压制着青仇,一面将大半注意力放在龙宫上。至始至终,他都没有放松过丝毫,一直死死盯着方源本体所在。

%52
整个战局仍旧在他的掌控之下!

%53
呼!

%54
忽然,风声骤起,大气磅礴,两记巨手左右拍门,袭上幽魂。

%55
这两只巨手一黑一白,雄浑浩大,呼应玄妙,赫然是气道杀招——阴阳大杀手!

%56
原来关键时刻,方源的气道分身终于赶赴过来。

%57
魔尊幽魂冷笑一声,身上冒出滚滚魂烟。烟雾中飞出两头太古魂兽,分别撞上两只气道大手。

%58
气道大手狠狠一捏,将两头太古魂兽捏得惨叫,一时间却也收拾不了它们的性命。

%59
气海老祖一时间抓也不是,放也不是。

%60
继续抓着这两头太古魂兽,他的攻势就被牵制。若是放掉,这两头太古魂兽还会来找他的麻烦。

%61
气海老祖低喝一声,双手一甩,索性让黑白气道大手直接飞射远去,带着两头太古魂兽脱离战场。

%62
魔尊幽魂却已达到了目标,用两头太古魂兽挡下了气海老祖,同时还牵扯了气海老祖的精神。毕竟气海分身还需要维持黑白气道大手,用来捏住两头太古魂兽。

%63
气海和幽魂的第一轮交锋,算是一场简单的交换。

%64
魔尊幽魂付出了两头太古魂兽,气海老祖则被消耗了一截精力、仙元和时机。双方算是不相上下,甚至魔尊幽魂还吃亏一些。毕竟两头太古魂兽可是八转战力!

%65
但是从大局来看,魔尊幽魂却是占了便宜,继续把控着大局。

%66
皆因他暗中控制了安魂洞天,拥有许多太古魂兽,暂时舍弃两头算不了什么。反倒是气海老祖最该珍惜的时机没有抓住,他原本想要干扰幽魂,此刻反被幽魂干扰,浪费了战机。

%67
魔尊幽魂扬长避短之余,更在汲取青仇的力量,不断壮大自身,同时还极大地削弱青仇。整体而言,他是大赚特赚。

%68
这一轮交锋,只是战斗的微小缩影,却能展现出幽魂丰富至极的战斗经验,高瞻远瞩的战略眼界。他对战斗的把控程度,是如此的周密,毫无破绽,令方源这等人物也感到窒息。

%69
方源自然不会甘心,一面积极修补龙宫,蠢蠢欲动,一面令气海老祖出手,再掀攻潮。

%70
幽魂一心三用,一方面镇压龙宫,黑烟涡流不断对其剿杀,一方面迅速吞吸青仇的魂兽力量,一方面催动杀招,抵挡气海老祖攻势。

%71
龙宫、青仇和气海老祖都是当今天下的巅峰战力,结果以三打一,都未扳回局面!

%72
实事求是而言,魔尊幽魂此刻的实力并不太过出众,但他此刻对龙宫已经占据全面优势,并且正克青仇,而对气海老祖更是摸清了底细。

%73
“他的魂道手段一套又一套,招招之间交相辉映,紧密结合,是极其优异的战斗体系。反观我的气海分身,因为经营时间太短,手段稀少,太容易被看透了。”方源本体洞察战局。

%74
气海分身主要的常规攻伐手段,就是阴阳大杀手。主要防御手段则是天罡布衣。底子太薄,放在魔尊幽魂这等存在的眼中,交手几次后,就能立即做出针对性的对策。

%75
不管是战力、眼界、经验,魔尊幽魂都是世间巅峰。曾经的他可是纵横天下,无敌世间的魔尊!

%76
而在历代尊者当中,幽魂魔尊更是被天下公认的杀性第一。

%77
他是经过惨烈厮杀,一路浴血奋战,在重重包围中杀出一条血路的强者!

%78
在他的尊者宝座下,堆砌的是如山似海的白骨。

\end{this_body}

