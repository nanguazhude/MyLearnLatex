\newsection{房睇长之陨}    %第四十六节:房睇长之陨

\begin{this_body}

神帝城,壁画世界。

宽阔的广场上,萧七星为首的大股人马正和一群豆神兵卒对峙。

这群豆神兵卒的规模,比萧七星这一方还要庞大。

“怎么打?”孙元化刚问,对面的豆神兵卒已然展开了冲锋。

萧七星拦住准备动手的诸人,微笑道:“诸位暂且充当后援,且看我来对付它们。”

萧七星言罢,呼啸一声,手指连连弹动,大量蛊虫飞出空窍。

这些蛊虫落到地上,迅速化为一个个的人形。大多数都是普通士兵,少部分明显是精锐模样,还有极少部分则更加魁梧勇悍。

“这就是兵卒蛊、伍长蛊、什长蛊?”古霆微微眯起双眼。

“应该没错了。”魏无伤大略数了一数,“大概有五百多人,看来萧七星在军营中混得风生水起,竟积累了这么多的人道蛊虫了。”

“结阵!”萧七星低喝一声,催动空窍中的箭矢军阵蛊。

他掌控的这些士卒顿时集结起来,百夫长统领什长,什长统帅伍长,伍长身边着围拢着兵卒。

而整个大军则宛若一个巨大的箭矢,不闪不避,直接朝冲过来的豆神兵卒反冲过去。

“杀啊——!”

萧七星统领的大军,发出震天的喊杀声,反观豆神兵卒这一边却是寂静若死。

两方人马仿佛对冲的巨浪,对撞之后,开始相互渗透。

萧七星双目如电,绽射星芒,全神贯注地操纵自家人马。

混战之中,他掌控的人马始终团结成一小块一小块,保持着最基础的军阵。

反观豆神兵卒却是散乱不堪,毫无组织。

经过前期的僵持,场面逐渐地被萧七星掌控。一旦局部占据了优势,这些优势便像是滚雪球一般不断壮大。很快,局部的优势演变成了整体优势。

最终,萧七星一方大获全胜,豆神兵卒被斩杀个干净。

“怎么样?”萧七星得胜归来,一脸兴奋,环视其他蛊仙种子,炫耀之情溢于言表。

赵淑野翻了一个白眼,没有理会。

“萧兄果然厉害!”陈大江竖起了大拇指。

“哈哈哈。”萧七星大笑,对陈大江道,“当初你若是和我一道,今时今日也会有如此成就。”

“还是继续前行吧。只是清缴了一处壁画而已。”魏无伤道。

“不可大意。这些兵卒不过只是黄豆兵卒,乃是豆神兵中最基础的角色。”古霆开口。

众人继续前行,不断征讨豆神兵卒。

片刻后,他们来到了神帝城最繁华的街道。

豆神兵卒正在恣意地屠杀城中居民,一片混乱。

孙瑶见此,顿时气得双眼喷火。她第一次进入这里,就是来到这个街道,领略到了繁华盛景。没想到这样美好的景象,却被豆神兵卒彻底破坏了,到处都是尸体,血液横流。

“这就得由诸位出手了。”萧七星难得谦虚道,“这里地形狭窄,视野不清,我的人马横陈不开。”

其余蛊仙种子自然没有什么异议,纷纷冲上前线,齐齐出手。

萧七星则留在最后方,一边操纵身边的兵卒徐徐推进,一边细心观察。

他心中深知:自己此次来到这里竞争元莲真传,身边的这些蛊师便是他最大的竞争对手。知己知彼百战不殆,眼下就是他了解这些人的最佳时机。

这些中洲十大古派的精英蛊师,一出手果然不同凡响。

萧七星积累很多,这些蛊师也毫不逊色。

应生机不断取出医师蛊,变作人形,到处救死扶伤。

陈大江则入了公门,身边环绕着一群捕快,同时自己也手持着朴刀、铁链,参与前线作战。

而孙瑶更令萧七星刮目相看,她身边环绕着三位舞女,一个个长袖飘飞,将豆神兵卒层层削弱。同时孙瑶还掌握着人道杀招——助人为乐,此招增益盟友十分有效。

至于魏无伤着忽隐忽现,身影在混乱的战场上如鱼得水。

“这个家伙已经掌握了贼人蛊。”萧七星心中了然。

这些中洲十大古派的精英,一旦出手便大放光彩。萧七星也感到了重重压力,心中警惕,“还真不能小看了这些人!”

这一场混战持续了半盏茶的功夫,整个街道这才被清缴干净。

没有多做停留,萧七星等人再度向其他壁画进军。

一处处的豆神兵卒都被他们相继剿灭。

当然,反击行动的远不止他们这一批。壁画世界的土著们也自发地组织出了不少队伍,都在奋力抗战,努力击杀豆神兵卒。

叶凡和洪易就混在其他的队伍当中,暗中交流着。

“这些豆神兵卒真是奇妙,种类繁多,令人大开眼界。”

“十大古派的这些人也成长起来了,我们要小心遮掩,不能被他们发现。”

他们是最早的进入壁画世界中的人。

此刻,他们的积累要超出中洲这批人一大截,但叶凡和洪易却只能暗中行事。毕竟五域外界才是主体,一旦他们俩被天庭发现,那形势就糟糕了。

反击的队伍越杀越快,效率不断提高,死伤则迅速下降。

他们有了充足经验之后,完全洞悉了豆神兵卒的种类。黄豆兵卒只是炮灰,绿豆兵卒能射飞箭,红豆兵卒要小心它的自爆,黑豆兵卒身躯最是坚硬等等。

摸清楚了不同的豆神兵卒的特性,有针对性地展开攻势,这些豆神兵卒远比外表更加羸弱。

反抗的队伍们不断获胜,夺回的地盘越来越多。

最终,他们汇集到了一处,来到了神帝城的城墙上。

城墙外是汪洋一般的豆神兵卒,萧七星第一眼看到这样的情景,不由地倒吸一口冷气。

敌军势大,已经不是他这支队伍能够处理的,只有联合其他队伍,才能守住城池。

豆神兵卒汹涌而上,毫无阵型,乱糟糟的一片,杀奔过来。

萧七星等人占据城头,居高临下,不断狙杀。

豆神兵卒宛若蚂蚁,开始攀附城墙。

守城的一方拼死反抗,一步不退。

整个战场宛若巨大的绞肉机,分分秒秒都有大量的生命消亡。

这一战足足持续了三天三夜。

最终,萧七星一屁股跌坐在地砖上,浑身一点力气都没有。原本洁白若雪的军袍已经被鲜血染红。

周围传来连绵的欢呼声,他们胜利了,这股最大规模的豆神兵卒已经被他们彻底消灭。

大局已定!

接下来,就是清缴各个壁画里的残留小股的豆神兵卒。

“等一等,我们似乎能够走出神帝城去。”

“顺着豆神兵卒入侵过来的路线,我们可以反攻到它们的老巢!”

清缴的过程中,众人又有了一项大发现。

大军再次集结,终于一路杀到房睇长的面前。

在这壁画世界中,房睇长实力大损,但身边仍旧留有一股豆神兵卒。

“原来你就是幕后黑手,罪魁祸首!”

“你要为你的罪孽付出代价!”

“为什么,为什么你要屠戮无辜,他们和你究竟有什么仇怨?”

群情激愤,但是萧七星等人却是隐秘地对视一眼,不留痕迹地朝后方挪去。

“原来豆神兵灾的根源竟然是他。”

“他是房家的蛊仙房睇长,曾经炼化过豆神宫。”

“没想到他还活着,并且还在壁画世界中兴风作浪!”

“难道说,门派将我们送达到这里,就是为了铲除他吗?”

“奇怪,为什么天庭不出手,直接将房睇长斩杀了呢?或许这是故意留给我们的考验,也或许必须按照壁画世界中的规矩,才能对付他?”

萧七星等人迅速交流,面色凝重至极。他们都已经是各自门派的重点栽培对象,因此对于蛊仙界的情报知道的并不少,尤其是宿命大战的参与者们,他们知之甚详。

“杀啊!”

“给我们的亲友们报仇!”

“我要把他碎尸万段!!”

壁画世界中的蛊师们向房睇长发起了进攻。

“一群渣滓。”房睇长坐镇中枢,指挥若定,身边的豆神兵卒阵型变化,仿佛行云流水。尤其是各种兵卒相互配合,默契无限。

壁画世界的土著们一时间损失惨重。

萧七星等人也非常震惊。

“明明这股兵卒规模并不多啊,怎么会打成这个样子?”

“这是有了蛊仙指挥,完全是天地之差啊。”

“我们该怎么办?是战是撤?”

“战!我们人多势众,房睇长虽强,但敌我之间的差距并不是仙凡之别。我们大有希望!”

“屠仙!!”

不知是谁喊出来的口号,让萧七星等人兴奋异常。

他们一出手,顿时就让前方交战的蛊师们感到了强援。

事实上,不只是他们,叶凡、洪易两人也混迹在战场中,暗中出力。

房睇长冷哼一声,豆神兵卒在他的指挥下,阵型不断变幻,守得稳如泰山。

萧七星等人苦战良久,骇然发现敌我战损达到了恐怖的差距,房睇长身边的豆神兵卒损失得微乎其微。

“这可该如何是好?”

“仙人到底是仙人!”

“更别说房睇长还是智道蛊仙呢。”

萧七星等人状态跌落谷底,斗志消退,都有了后撤之心,不敢和房睇长争锋。

但就在这时,房睇长忽然身躯一抖,吐出一口鲜血。他冷哼一声,仰望天空,冷笑连连:“元莲意志,你终于还是忍不住出手了。”

房睇长被元莲意志压制,操纵豆神兵卒再无之前的随心所欲。

萧七星等人顿时感到压力骤减,一时间惊喜交加,纷纷挥军杀上。

房睇长身边的豆神兵卒不断缩减,最终被屠灭个干净。

房睇长不得不亲自出手,在大军中杀得七进七出,纵横捭阖,无人能敌。

好景不长,他的状态不断下滑,开始负伤。

又一段时间过去,房睇长浑身浴血,踉踉跄跄。

“没想到我房睇长居然陨落在此!”房睇长悲啸一声,猛地自爆。

轰!

围攻在他身边的蛊师们无不尸骨无存,被炸成粉碎。

短暂的震惊之后,幸存的蛊师们掀起震天的声浪,欢呼雀跃。

“我们胜利了!”

“我们终于将魔头杀死了。”

“呜呜呜……爹娘,孩儿为你们报仇了。”

萧七星等人振奋不已,满脸通红:“我们屠仙了!屠仙成功了!”

------------

请假,今天无更

还要继续调整大纲……

\end{this_body}

