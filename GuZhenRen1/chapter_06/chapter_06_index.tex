\newchapter{魔尊永生}    %第六卷:魔尊永生

\input{chapter_06/chapter_06_001.tex}
\input{chapter_06/chapter_06_002.tex}
\input{chapter_06/chapter_06_003.tex}
\input{chapter_06/chapter_06_004.tex}
\input{chapter_06/chapter_06_005.tex}
\input{chapter_06/chapter_06_006.tex}
\input{chapter_06/chapter_06_007.tex}
\input{chapter_06/chapter_06_008.tex}
\input{chapter_06/chapter_06_009.tex}
\newsection{李小白咏春}    %第十节:李小白咏春

\begin{this_body}



%1
“两位蛊师大人,请进,请进!”店小二点头哈腰,将华松、安崇迎接进店里来。

%2
“这是京城著名的茶楼,十分热闹,我曾经多次来过。”华松暗中传音,给安崇介绍道。

%3
两位蛊仙伪装成了蛊师,来到华文洞天的京都。

%4
安崇倒更愿意按在云头,俯瞰京城。但既然华松有这样的乐趣,他也只好客随主便了。

%5
“我要五楼楼上的雅座。”华松显得轻车熟路。

%6
两人来到五楼,进入房间,打开窗户就看见街道上人潮汹涌,第一会场就在街道的那头,青铜大门前几乎挤满了人。

%7
“这便是我华文洞天的当代才子们啊,也是我华文洞天未来的希望。尊使请看。”华松感叹一声,递给安崇一只五转的侦查蛊虫,可观察目标才气。

%8
安崇端详了一番,当面用了,顿时视野就发生了变化。他看到无数才子的头顶上,都有着才气。这些才气五彩斑斓,各有高低,形态不一,着实令他大开眼界。

%9
“既然是有如此蛊虫,可侦查才气,为何还要举办这样的选拔呢?”安崇问道。

%10
华松呵呵一笑:“尊使有所不知。才气便如同修为,就算拥有众多,临场发挥不出也是枉然。况且此次选拔,都是当场作诗,做不得假。更考验书生才子们的素养,许多才气浓郁的书生,未必有灵感,能够做出最高水平的诗词来。”

%11
安崇点点头:“我看这里布置的大阵不仅相互勾连,似乎还能够助长书生们的才思?”

%12
华松点头:“不错,尊者慧眼!在这座仙阵中,书生们往往都能超常发挥,更加体现出各自的才情资质。”

%13
说话间,房门被敲响。

%14
店小二得到华松的允许后,端来一盆盆的美酒佳肴。

%15
“八宝野鸭、金丝酥雀、熊猫蟹肉,都是我们茶楼的招牌菜,请两位客官享用。”店小二道。

%16
华松随手赏了他一枚元石,将店小二打发掉。

%17
华松对安崇介绍道:“尊使,这里的茶虽然是凡茶,不过却是华语老大人年轻时的创作,当时他是一位四转蛊师,夺得状元,因此此茶又被他命名为状元茶。”

%18
“哦?”安崇顿时起了兴趣,这可是八转蛊仙年轻时候的作品。

%19
“那我得要好好品尝一番。”安崇喝了一口,闭上双眼,细细品味,恍惚间,心头便升腾起一股微微的兴奋,好像是苦笑十数载,终于功成名就,人生得意近在今朝!

%20
“好茶,好茶。”安崇真诚赞叹,“这茶虽然是凡茶,但竟有一丝人道韵味!”

%21
宿命大战,中洲天庭连续施展的数记人道杀招,招招威力绝伦,轰动天下。人道之威可谓无人不知,无人不晓。

%22
就在二仙品茗之时,第一会场的青铜大门悠悠打开。

%23
拥挤在门前的人潮,顿时发出轰响。

%24
“打开了,打开了!”

%25
“别挤啊。”

%26
“快让我进去。”

%27
人潮汹涌,纷纷涌入门内。

%28
李小白也在人群之中,不过位于后端。

%29
他一边随着人流往前,一边揣摩着此次选拔的规则。

%30
“天下诗会共有十八个会场,每一个会场都有一个题目,所有人同时答题。不管人数多少,最终都依凭作品的质量,只有一半的人能够晋级。”

%31
“晋级到后面的会场后,仍旧是取半。如此一路晋升上去,打通整个十八会场,才算是达到了选拔的要求。”

%32
“若是中途失利,就要退回之前的会场。若是一连失败,直接退出第一会场,那就算是彻底失败了。”

%33
“不过,天下诗会共举办七天。每个人都有三次彻底失败的机会。”

%34
“如此一来,足以让华文洞天方面,挑选出优质的蛊仙种子了。偶尔失误失常,还有重来的机会。但如果有哪位才子,连续七天都发挥失常,正说明了本身素养就不足,根本不配洞天栽培成为蛊仙。”

%35
李小白收回思绪,已是来到了第一会场。

%36
会场一片空阔,挤满了书生。

%37
书生们有男有女,有老有少,人数之多,一眼望去,没有一万也有数千。

%38
这还只是第一天。

%39
李小白在会场中又等待了一炷香的时间,第一会场这才将所有的书生才子接收完毕。

%40
人山人海,一片嘈杂之声。

%41
好在第一会场乃是仙阵空间,可以随时如意地扩张,容纳更多的书生也在能力范围之内。

%42
咚咚咚!

%43
一阵鼓声,同一个声音传入书生们的耳畔:“天下诗会第一题——春,时限半盏茶。”

%44
说完这话,声音就消失不见了。

%45
“第一个题目是春?”

%46
许多书生皱起眉头,也有许多书生喜上眉梢。

%47
李小白暗暗沉思:“咏春的诗词着实太多了,这个题目乍一看很容易。毕竟书生们或多或少,都创作过有关题材的诗词。虽然天下诗会要求当场创作新诗,但只要将原先著作稍微修改一番,有时候也能勉强算是新诗。”

%48
“然而实际上,这个题目还是有难度的。”李小白一脸沉着之色。

%49
他知道,这一次创作诗词,主要还得和周围的人进行比较。只有比半数书生都要强,他才能够成功晋级。

%50
意识到这一点的书生不在少数。

%51
很多人都开始苦思冥想起来,有的人直接盘坐在地上,有的人则背负双手,四处踱步,有的人垂着头,口中轻轻呢喃。

%52
李小白想的却是:“我该抄哪一首呢?”

%53
咏春的诗词,在他记忆中藏有很多,不乏传世经典。

%54
但第一次就拿出传世经典,并不太好。这让李小白以后不好发挥了。他才气规模并不属于第一流的层次,若是场场拿出传世名篇来,肯定惹来怀疑。

%55
若是他运气好,李小白或许还会稍稍激进一些。但现在明显气运低迷,李小白便以稳重为先。

%56
在李小白思考的时候,已经有不少书生创作出了新诗。

%57
于是各种光晕闪耀,光晕五颜六色,或强或弱。每当光晕在书生的身上消散,书生们就都有收获。

%58
有的人得到了蛊虫,有的人直接拔升修为,有的人恢复了真元,有的人消散了疲惫。

%59
这是济文才杀招。

%60
华文洞天的铸造者,最初始的原主,在陨落之前将这记杀招布置下来。正是有了济文才杀招,才鼓励越来越多的书生奋发读书,最终形成华文洞天这般文风鼎盛的现状。

%61
李小白终于敲定了他的诗。

%62
他轻轻地咳嗽了一声,开始朗诵。

%63
“月夜。”

%64
“更深月色半人家,北斗阑干南斗斜。”

%65
“今夜偏知春气暖,虫声新透绿窗纱。”

%66
李小白朗诵刚毕,耳畔就听轰隆一声轻响,全身升腾出浓绿的光辉。

%67
腾腾腾。

%68
李小白周围的书生,无不感到一股无形的压力,绿光逼得他们向四周散去,留出好大一块空白的地方,只剩下李小白一人站在中央。

%69
“好,好强的光辉!”

%70
“名篇出现了啊!”

%71
“没想到这么快,就有名篇问世。不知道是什么人创作的?”

%72
许多书生都被打断了思绪,一个个用羡慕、探究的目光,盯着李小白。

%73
李小白表面云淡风轻,实则心里有些纠结:“哎呀,抄的有点过了,效果也有点太好,惹来不少注目。”

%74
他扫视四周,心中盼望着会有一些人站出来,不要让他如此独领风骚。

%75
绿光则不断地钻入他的空窍之中,令他的修为向上攀升。

%76
“哦!有名篇出世了。我来读读看。”茶楼中,华松顿有所感。

%77
读了一遍李小白的著作,华松满意地连连点头:“妙啊,妙啊!这个李小白年纪轻轻,作诗却是老道。”

%78
“通常咏春的诗词,多是柳绿桃红。但这首诗却偏偏反其道而行之,写夜幕来遮掩春光,可谓巧思。”

%79
“全诗后一句,必定是李小白从自身深切的生活中挖掘而得。透露出一股清新、欣悦,充满生机的意境。”

%80
“观诗如观人,这个李小白心境上佳!”

%81
华松啧啧赞叹一番,像是品味到了世间绝味之佳肴。他看着安崇,微笑道:“不知尊使对此诗有何见解呢?”

%82
安崇顿感头大,腹诽着:“如果你不说,我还不知道这诗有这么些好处呢。哎呀,眼下要我点评,我该怎么办?”

%83
ps:9点第二更。

\end{this_body}


\input{chapter_06/chapter_06_011.tex}
\input{chapter_06/chapter_06_012.tex}
\input{chapter_06/chapter_06_013.tex}
\input{chapter_06/chapter_06_014.tex}
\input{chapter_06/chapter_06_015.tex}
\input{chapter_06/chapter_06_016.tex}
\input{chapter_06/chapter_06_017.tex}
\input{chapter_06/chapter_06_018.tex}
\input{chapter_06/chapter_06_019.tex}
\input{chapter_06/chapter_06_020.tex}
\input{chapter_06/chapter_06_021.tex}
\input{chapter_06/chapter_06_022.tex}
\input{chapter_06/chapter_06_023.tex}
\input{chapter_06/chapter_06_024.tex}
\input{chapter_06/chapter_06_025.tex}
\input{chapter_06/chapter_06_026.tex}
\input{chapter_06/chapter_06_027.tex}
\input{chapter_06/chapter_06_028.tex}
\input{chapter_06/chapter_06_029.tex}
\input{chapter_06/chapter_06_030.tex}
\input{chapter_06/chapter_06_031.tex}
\input{chapter_06/chapter_06_032.tex}
\input{chapter_06/chapter_06_033.tex}
\input{chapter_06/chapter_06_034.tex}
\input{chapter_06/chapter_06_035.tex}
\input{chapter_06/chapter_06_036.tex}
\input{chapter_06/chapter_06_037.tex}
\input{chapter_06/chapter_06_038.tex}
\input{chapter_06/chapter_06_039.tex}
\input{chapter_06/chapter_06_040.tex}
\input{chapter_06/chapter_06_041.tex}
\input{chapter_06/chapter_06_042.tex}
\newsection{自愿跟随}    %第四十三节:自愿跟随

\begin{this_body}



%1
杀——!

%2
吴帅的呼啸激荡无边风云,响彻洞天乾坤。

%3
华文洞天的许多蛊仙纷纷变色,感受到了吴帅的凌厉之威,不免惊惶担忧。

%4
华文洞天和平太久了,从未有过这样的经历。即便是当代洞天之主华语老仙,也是意外至极。

%5
因为华文洞天乃是信道洞天,最擅长收集情报和种种线索。一旦有外来侵略,必定会先一步发觉。

%6
然而吴帅这一次攻击,不仅华语老仙等人都蒙在鼓里,毫不知情,而且吴帅的突袭入侵堪称是大手笔,竟能将蛊仙传送到洞天各处,针对各处要地和华文洞天的蛊仙,施展暴风骤雨般的突袭。

%7
可以说这一场突袭极其成功。

%8
华文洞天损失惨重至极,不说姜先生这一票精心选取上来的未来潜力,就算是华文洞天中的蛊仙也被灭杀了大多数。

%9
“坚持住,这是我们的家园,绝不能有失。我已联络到了援兵,绝不可让这帮卑鄙无耻的暴徒得逞!”华语老仙呼喊,稍稍稳定了军心。

%10
随后,他张开嘴巴,舌头攒动,不断吐出一个个的文字。

%11
这些文字宛若暴风大雨,带着森然寒气,向龙宫笼罩过去。

%12
仙道杀招——冷言冷语!

%13
龙宫遭受此招,表面立即结上一层层的冰霜,速度减慢许多。冰霜越结越多,眨眼间已经有了成人小腿高度。照此下去,整个龙宫都要被冰封起来。

%14
吴帅冷哼一声,催动杀招进行防御。

%15
龙宫表面的冰霜立即开始融化,层层递减,并且威势不断攀升,绽射出一道道橙光。

%16
橙光射向华语老仙,华语老仙口型一遍,又施展了一记腾挪手段。

%17
仙道杀招——快人快语。

%18
下一刻,他速度飙升,完美躲闪橙光攒射,轻松自如。

%19
仙道杀招——恶语相向!

%20
华语老仙再次从口中喷吐字潮,这些紫黑色的字句,宛如毒箭,射在龙宫上引发爆炸。

%21
轰隆隆……

%22
连续的爆炸逼得龙宫不断后退,同时更有大量的毒雾向龙宫内部渗透进来。

%23
吴帅坐镇主位,见此临危不乱,立即催动手段防御,化解了这场危机。

%24
华语老仙名不虚传,的确是有两把刷子的强者,竟和龙宫打得不相上下,招招有来有往。

%25
当然,这也是因为龙宫本身乃是奴道仙蛊屋,最擅长的方面是奴役。而吴帅为了保留底牌,四大龙将即便已经补齐,也不拿出来亮相。经常用的是孽龙,已经成为龙宫的一个招牌了。

%26
孽龙在肆虐!

%27
两天联盟的蛊仙们在烧杀抢掠!

%28
吴帅根本不着急,因为局势对他太有利了。

%29
华语老仙越战越急,他知道自己已经被龙宫纠缠住了,现在想要挽回华文洞天的危情,只有依靠外力。

%30
但是外援真的能及时赶到吗?

%31
华语老仙虽然万分期待,但是自己也知道希望并不是很大的。他虽然自己名望很大,华文洞天也和周边的洞天势力联络紧密,但是要让这些人支援自己,去对抗这么样的强敌?

%32
华语老仙换位去想,他自己周边的洞天若是遇到敌情,向自己求援,敌人正是两天联盟的话,他也会犹豫不决。

%33
两天联盟的蛊仙们不断肆虐。

%34
烽烟四起,一片生灵涂炭。

%35
“华语老仙,你还有什么底牌快施展出来吧,再不用的话,可就救不了你这洞天了。哈哈哈。”吴帅大笑,十分猖狂。

%36
华语老仙沉默,脸色铁青。

%37
两人交手的余波危及天地,打得洞天窍壁四处破漏,内外沟通。

%38
忽然,吴帅大笑戛然而止,随后惊怒地吼道:“气海老贼!”

%39
话音刚落,无穷气流从华文洞天之外,蜂拥而入,汇集成一只大手。

%40
大手如山,拍中龙宫,直接将龙宫远远拍飞。

%41
随后一个伟岸的身影,出现在了洞天苍穹的顶部,俯视龙宫。

%42
“是气海老祖!”华语老仙仰头,惊喜之情溢于言表。

%43
气海老祖出现得让他意外,但细细一想,又觉得理所当然。气海老祖乃是当下吴帅最大的敌人,他绝不会坐视吴帅恣意整个两天势力的。

%44
嗷吼!

%45
龙宫退下,孽龙顶上。

%46
气海老祖却背负双手,没有出手的意思:“吴帅,你罢手吧,我可任由你方蛊仙全部撤走。否则真要动起手来,你的两天联盟就不会存在了。”

%47
“你……”龙宫中传出吴帅气急败坏的声音,“好一个气海老祖!”

%48
孽龙冲势顿止,吴帅似在考虑。

%49
华语老仙松了一口气,用感激万分的目光投向气海老祖。

%50
气海老祖此举是想维护华文洞天的生机,不想让这座洞天彻底沦为废墟。这很不容易!眼下两天联盟的蛊仙们各自散落一方,气海老祖乘机就能各个攻破,给两天联盟施以重创。

%51
但是气海老祖没有这么做,而是顾及到了华文洞天。而在此之前,华语老仙根本就没有和气海老祖有所接触过。

%52
“对于我这样的外人,气海老祖就能做得出这样的维护之举。他真不愧是名传五域的强者,东海正道的领袖人物啊!”华语老仙心中感慨万分。

%53
然而下一刻,孽龙猛地出动,龙宫紧随其后。

%54
“气海老祖,当日一战,间隔太久。今天你再接我这一招!”吴帅嘶吼。

%55
“卑鄙!”华语老仙看到吴帅这近乎偷袭的举动,顿时在心中大声咒骂。

%56
但气海老祖却是微微一笑,似早有准备,从容淡定:“来得好。”

%57
轰!

%58
下一刻,孽龙和龙宫俱都冲撞在一道坚固凝实的气墙之上。

%59
气墙崩溃,但孽龙和龙宫的冲势也消耗殆尽。

%60
“气海老贼,你这么精神,那我就放心了。咱们来日再战!”吴帅见讨不到什么便宜,索性徐徐后撤。

%61
龙宫和孽龙率先脱离战场,去往洞天之外。

%62
两天联盟的蛊仙们纷纷抽身而退,不敢久留,深怕龙宫撤离,把自己遗留在这处险地。

%63
吴帅等人来得快,去得也快,留下烽烟四起,一片混乱的华文洞天。

%64
华语老仙没有急着去赈灾,稳定局面,而是立即飞到气海老祖面前,深深弯腰一礼:“今日得老祖您出手相助,方能击退恶敌。华文洞天上下都要感激您的救命恩德啊。”

%65
气海老祖朗笑一声,抓住华语老仙的双手,将他抬起:“你我皆为人族,守望相助,共抗异人,那是应当的。”

%66
华语老仙恨声道:“吴帅倒行逆施,企图颠覆人族大局,丧心病狂,卑鄙无耻。我华文洞天上下必与之为敌,不死不休!”

%67
气海老祖点头:“当今天下,乱战将至,谁能独善其身?仙友明白这点就好!”

%68
华语老仙再次拱手作揖:“我观天下诸仙,唯有老祖您威布四海,名传天下,更有德操令人敬服。今后华语洞天上下,愿跟随老祖您鞍前马后,听候调遣。”

%69
气海老祖大笑:“华语老弟,就让我们联手,共抗异人,包围家园,拼出一个美好的未来吧。”

\end{this_body}


\input{chapter_06/chapter_06_044.tex}
\input{chapter_06/chapter_06_045.tex}
\input{chapter_06/chapter_06_046.tex}
\input{chapter_06/chapter_06_047.tex}
\input{chapter_06/chapter_06_048.tex}
\input{chapter_06/chapter_06_049.tex}
\input{chapter_06/chapter_06_050.tex}
\input{chapter_06/chapter_06_051.tex}
\input{chapter_06/chapter_06_052.tex}
\input{chapter_06/chapter_06_053.tex}
\input{chapter_06/chapter_06_054.tex}
\input{chapter_06/chapter_06_055.tex}
\input{chapter_06/chapter_06_056.tex}
\input{chapter_06/chapter_06_057.tex}
\input{chapter_06/chapter_06_058.tex}
\input{chapter_06/chapter_06_059.tex}
\input{chapter_06/chapter_06_060.tex}
\input{chapter_06/chapter_06_061.tex}
\input{chapter_06/chapter_06_062.tex}
\input{chapter_06/chapter_06_063.tex}
\input{chapter_06/chapter_06_064.tex}
\input{chapter_06/chapter_06_065.tex}
\input{chapter_06/chapter_06_066.tex}
\input{chapter_06/chapter_06_067.tex}
\newsection{都要杀方源}    %第六十八节:都要杀方源

\begin{this_body}



%1
三方混战当中,紫薇仙子一伙以及气绝魔仙忽然现身参战,顿时将情势变得更加复杂。

%2
“你们去抵挡天庭。”魔尊幽魂公然下令。

%3
紫薇仙子、正元老人纷纷出手,杀招宛若光潮,震慑人心,卷席秦鼎菱等人。

%4
秦鼎菱等人除了气海老祖之外,都位于仙蛊屋中,倒也不惧幽魂一方的忽然增援。

%5
只是天庭众仙俱都气愤至极,他们不是愤怒紫薇仙子、正元老人的叛变,而是痛恨魔尊幽魂的奴役手段。

%6
对于紫薇仙子、正元老人的忠诚,天庭上下都十分信任。这两人之所以叛变,天庭诸仙都明白完全是魔尊幽魂的原因。

%7
魔尊幽魂呵呵一笑:“自己人对付自己人,这种感觉如何?能否救下他们,就看你们的手段了。这样的机会今天错过了,将来恐怕不会再有。”

%8
天庭诸仙闻言,顿时攻势又凝滞了几分。

%9
秦鼎菱细眉如剑,狠狠蹙紧,厉声下令:“幽魂是要让我们投鼠忌器,他要拖延时间。不必顾惜紫薇、正元的性命。我们杀了他们,就是为他俩解脱!”

%10
关键时刻,秦鼎菱展现出了首领的大局观,以及果断坚韧的品质。

%11
魔尊幽魂闻言,也不由微微扬眉。

%12
他的动作越发又快了几分,再次扑向青仇。

%13
青仇和其交战,很快就落入下风。

%14
“阻止他!”眼看着魔尊幽魂又要故技重施,从青仇身上汲取力量,秦鼎菱连忙求援,“还请老祖出手!”

%15
不消她提醒,气海老祖已经扑向魔尊幽魂。

%16
但下一刻,一道气流喷涌而出,宛如天河澎湃,拦截住了气海老祖。

%17
气海老祖止住动作,凝神看向出招的人:“气绝!”

%18
气绝魔仙微微一笑,拦路道:“气海,上一战被天庭搅局,意犹未尽。这一次就让我们公平对决。”

%19
天庭诸仙心头猛沉。

%20
关键时刻,气绝魔仙竟然是魔尊幽魂的援手!

%21
双方究竟达成了什么协议?

%22
紫薇仙子也感到惊喜,气绝魔仙绝对是意料之外的强援!

%23
她眼中紫芒一闪,便有了顿悟:“很长一段时间之前,气绝魔仙就开始在宝黄天中大肆收购传承,涉及各个流派。他是上古时期的人物,因为宿命蛊被摧毁而重生。借助这些传承,能让他迅速跟上当今的大时代!”

%24
“气绝之所以支援我等,很可能是主上和气绝达成了交易,贩卖了大量的传承给他。”

%25
紫薇仙子猜得很准。

%26
气绝魔仙虽然没有在宝黄天中直接公开身份,但他也并未有多少掩饰,动向和意图都昭然若揭。

%27
天庭能够发现气绝魔仙的举动,幽魂又怎能发现不了呢?

%28
因为历史缘由,气绝魔仙和天庭是无法合作的。但幽魂也是魔道,更掌握了海量的传承,正是气绝魔仙当下最理想的交易对象。

%29
“方源,今日你在劫难逃了。”魔尊幽魂冷笑,再次突进到青仇的背上。

%30
下一刻他悍然下手!

%31
青仇嘶吼咆哮,但力量再次被魔尊幽魂疯狂汲取。身为太古魂兽,即便是有准九转的仇恨蛊,青仇也被魔尊幽魂克制得死死。

%32
青仇气势急剧下滑,魔尊幽魂则精神振奋。

%33
他抽吸完毕,狠狠一踢,就将青仇直接踹落下去。

%34
随后,他闪电般扑向方源。

%35
仙道杀招——纯梦求真变!

%36
原本围绕在龙宫周围的梦境,陡然缺失一大块,化为纯梦求真体,迅速飞入魔尊幽魂的仙窍之中。

%37
魔尊幽魂竟然身怀充足的仙蛊,可以催发纯梦求真变杀招!

%38
他之前故意放任方源,明明有手段却不急着施展。这一下,他暂时解决了青仇,又用援手挡下天庭、气海老祖一方,再次对方源下手!

%39
群仙动容。

%40
整个战局似乎从始至终,都在魔尊幽魂的掌控之中。

%41
历史上,幽魂魔尊以嗜杀闻名,但并不是说幽魂解决问题的方法只有一个字——杀。他同样冷漠、狡诈、阴险,只是生前的时候不需要他动用种种手段,只凭本身实力就能碾杀一切不服。

%42
如今魔尊幽魂的实力,和他的敌人相比,并没有太大差距。所以,他也不吝动用种种手段,增添自身实力。

%43
一时间,大量的梦境被魔尊幽魂汲取,重新夺回。

%44
紫薇仙子等人力拼天庭,而气海老祖和气绝魔仙对战,处于下风。

%45
气绝魔仙头顶天地秘境“兮”,着实克制同一流派的气海老祖。

%46
情势危急,方源再不能凭借梦境来阻敌。

%47
龙宫大门忽然敞开,飞出数位纯梦求真体,突破梦境防线,扑向魔尊幽魂。

%48
砰砰砰。

%49
纯梦求真体自爆,化为梦境弥漫半空。

%50
魔尊幽魂连忙闪躲,并没有被梦境笼罩。

%51
但趁此机会,吴帅催动龙宫,已是从梦境的另一侧的漏洞处钻穿。万年斗飞车紧随其后。

%52
两座仙蛊屋跑路的意图昭然若揭。

%53
“哪里逃?!”魔尊幽魂冷喝一声,飞速绕过梦境的阻碍,再度逼近。

%54
方源是他的首要目标,必杀之人。

%55
眼前的良机太难得了,方源逃到哪里,魔尊幽魂便决定追到哪里。

%56
魔尊幽魂紧紧盯着龙宫和万年斗飞车,眼中闪烁着的尽都是志在必得的杀机!

%57
下一刻,龙宫、万年斗飞车忽然分散,向着两个不同的方向撤离。

%58
魔尊幽魂楞了一下。

%59
这可怎么办?

%60
万年斗飞车、龙宫藏身在梦境中一段时间,内外隔绝,如今的魔尊幽魂根本不知道方源藏身在哪个仙蛊屋中。

%61
魔尊幽魂只有一人,要同时追两座仙蛊屋,显然是分身乏术。

%62
但魔尊幽魂几乎瞬间做出了决断,他呼喝道:“气绝魔仙,你来追杀万年斗飞车!事成之后酬劳少不了你的。”

%63
气绝魔仙嘿了一声,用兮地吸摄了气海老祖的一记杀招,轻松抽身,追杀万年斗飞车去了。

%64
而魔尊幽魂仍旧负责龙宫。

%65
紫薇仙子亦得到命令,徐徐而退。

%66
“我们怎么办?”天庭蛊仙询问秦鼎菱。

%67
秦鼎菱看来一眼紫薇仙子等人,几乎不假思索:“先杀方源!”

%68
但刚刚困扰魔尊幽魂的难题,同样困扰天庭一伙——方源究竟在哪座仙蛊屋中?

%69
“随便猜一个,赌运气?”秦鼎菱眉头紧皱,虽然她不愿意这样做,但天庭缺乏智道蛊仙,似乎也只能这样做了。

%70
“我可以帮助你们。”紫薇仙子忽然停住撤退的脚步,“我有法门推算方源的位置。”

%71
天庭诸仙皆是楞了一下。

%72
“是何法门?耗费多久?”气海老祖询问。

%73
“加上我等相助,只需片刻。”冰塞川的声音传来,众仙转头望去,便见天边飞来劫运坛。

%74
“总算赶来了。”紫薇仙子吐出一口浊气。

%75
长生天的这批人自然是她唤来的。之前紫薇仙子等人是接到魔尊幽魂之命,隐瞒长生天,准备偷偷袭击天庭。但情况变化太快,紫薇仙子在赶来的路上,分析局势后,便有了借助长生天之力的打算。

%76
得到魔尊幽魂的允许之后,紫薇仙子立即联络了长生天方面。

%77
劫运坛飞近,冰塞川再度道:“我家主上说了,这世间下棋的人不需要太多。”

%78
群仙心头一震。

%79
冰塞川的主上当然就是巨阳仙尊。

%80
种种迹象表明,这位尊者图谋深远,还在世间留有深厚的意志和力量。

%81
就眼下而言,方源是最有望成就尊者的一位。巨阳仙僵当然不想这种事情发生。为将来计划,他当然要趁着方源还未彻底强大起来时将其铲除!

%82
秦鼎菱咬牙,陷入艰难的抉择当中。

%83
和魔尊幽魂、长生天联手?

%84
前者乃是彻彻底底的魔道,而后者更是在宿命大战中的强敌,宿命蛊被毁,他们都有份!

%85
和这些人联手?

%86
秦鼎菱只是思考了几息时间,从牙缝中挤出一个字:“好。”

%87
就像冰塞川的传话所言,这个世界下棋的人未免太多了些。方源是注定的敌人,更是人形大宝库,先除掉他,将来再好好对付幽魂、巨阳!

\end{this_body}
\newsectionindepend{致歉和接下来的更新说明(必看)!}
\begin{this_body}
%88
致歉和接下来的更新说明(必看)!

%89
我非常抱歉。最近的更新很不稳定,而且隔三差五就请假。

%90
事实上,我非常理解大家的痛苦。因为我不仅是作者,同样也是读者。当我读到某部网文,隔三差五就请假,我的阅读体验会很难受。我会失望,我会失落,我会恼怒,有时候我甚至会暗骂——这作者真是个傻逼!

%91
然而,事实很残酷。尽管我不想承认,但我不得不说,在最近的更新问题上——我就是那个傻逼作者。

%92
我真的很想稳定更新,甚至加更,爆更!

%93
所以,我曾经做出过加更的承诺。

%94
哪个作者不想稳定更新,频繁更新?

%95
对于网文而言,更新频繁,就意味着人气,意味着收入。哪个想自己的钱袋子过不去?

%96
我是凡人,我是个俗人,我当然也想多赚点钱。

%97
但残酷的事实又告诉我——我能力不够!

%98
我能力真的不够。

%99
我没法做到稳定更新,做到加更,我只能频繁请假,努力构思,然后再一点点的写出来。

%100
有读者会问:那你以前不也有稳定更新的时候吗?

%101
那我只能苦笑地回答这些读者:是,我以前更新是稳定,甚至曾经加更,曾经爆更过。但那是创作的前中期。

%102
创作,越写到后面越难。前期是白纸作画,信马由缰。中期是线索逸散,耐心跟进。后期是最难的,因为要把所有发散出去的线索,都收拢起来,把所有的坑都填了,让所有的人物都有自己当有的结局。

%103
做到这样的程度,后期才是真正的合格的后期。

%104
100万字的后期是难的,200万字的后期是很难写的,300万字那是难上加难。诸位,我们的这个已经600多万字了,难度可想而知。

%105
《蛊真人》这部和大多数的网文不同,通常的网文它的情节都是线性的,一条线顺下来,很长很长。地图不断地换,很多很多。几乎所有前中期的人物,除非是主要配角,越到后期这些人物几乎是消失匿迹。

%106
而《蛊真人》它更像是一个舞台剧,一个大舞台。不管是什么样的人物,到了后期都有机会出场。白凝冰、黑楼兰这等重要配角我就不提了,我举个例子——青辉子。这个人物偶遇气绝魔仙,一度陷入危机绝境,幸好气绝魔仙没有取他性命。事实上,青辉子是本书中期的一个小人物,我曾经一笔带过。

%107
又比如莫家的肥娘子,这个人物在方源仙僵时期出现的。到了本书后期,她又出场了。并且她的登场,影响到了莫利。而莫利和彭达的际遇,又会影响到其他人物。

%108
每个人物都有自己的想法,有自己的酸甜苦辣,他们的每一个动作不管有意无意,又会促使另外的人物的生命轨迹的变化。

%109
我认为这样的一个世界,才称得上有灵性!

%110
所以,蛊真人的情节设计,是很繁琐的,是非常艰难的。尤其是到了后期,五域两天的大舞台!这样多的人物!我每一次动笔,都很谨慎。稍不留意,我就要吃设定,人物崩溃或者逻辑错误。

%111
这还只是正文的难度。

%112
中还有一部分,构思难度比正文还要高出数倍!

%113
不用我说,大家都知道——那就是《人祖传》。

%114
《人祖传》本身它就很难啊,我不仅要创作一个神话,更要这个神话水准高出盘古开天,女娲造人,亚当夏娃的程度。因为后者这些都不涉及哲理,《人祖传》它是包含哲理的,深度更深,可以精读细读下去。因为我设计的对话、情节都要符合一定的道路。

%115
就随便举个例子吧。

%116
炎煌雷泽把仇恨蛊捏死,结果发现仇恨蛊是杀不死的。这个情节很小,但蕴含一个很朴素的道路——冤冤相报何时了,不能用仇恨来泯灭仇恨。要除掉仇恨,得用其他法门。

%117
再举个例子。

%118
人祖炼制财富蛊,他先后用了辛苦蛊、忧患蛊、悲伤蛊。然后愚蠢蛊被他诱骗,丧生了。智慧蛊也被诱骗,但最终逃了出来。最后,人祖失去了双手,才炼成了财富蛊。

%119
这个情节就是对财富的阐述。要赚取财富是很辛苦的,维护财富要具有忧患意识,失去财富会悲伤。愚蠢的人葬身在过大的财富中,而具有智慧的人也常常被财富迷心,但真正有大智慧的往往会最终看清这一点。最后,财富是要用人的双手创造的。

%120
这样的例子还有很多很多。

%121
《人祖传》的难度,我觉得比通常网文的难度还要更高。

%122
它也是越往后越难。因为我要在后期将人祖的十子都写进去,将怎么破坏九天的写进去,将前中期的一些伏笔(乾坤晶壁等)都要囊括进来。

%123
正文困难,《人祖传》更难。而我要做的,是要将两者巧妙地结合起来写。

%124
这是超难的事情!

%125
至少对我而言,是非常困难的。

%126
《人祖传》我要一章一章的写,然后巧妙地嵌入到正文中,给正文点睛。而这些章节的顺序还不能乱,绝不能前后颠倒。

%127
实话实说,我的压力是非常大的。

%128
尤其是最近,经常是睡不着觉,熬到三四点才能勉强入睡。构思情节,构思人祖传想到脑壳都要烧糊了的感觉。

%129
我记得在好几天前,预约了一位老中医看病。结果前一天晚上精神焦虑,睡不着觉,起来整理大纲,到了2、3点困了睡觉。结果第二天起来,早饭都没来得及吃,就乘车赶去医院。路上堵车,导致预约过了,重新挂号非得等到下午才能看到。干脆就再预约吧,预约只能又等到下周。现在中国的医疗资源真的越来越紧缺了。

%130
这种事情我通常不会说出来,也不是好面子,只是觉得一个大老爷们,不屑于卖惨博取同情。

%131
我现在说出来,只是表述现实,希望大家明白我如今是怎么样的写作状态。

%132
其实有很多时候,我也在想,要不就放弃《人祖传》?要不就干脆随便水水文,每天当然能做到稳定更新。文章的品质差点就差点,至少读者不会反感太大。

%133
每天有更新,每天就有稿酬,何乐而不为呢?

%134
我再一想,不能!

%135
有很多读者朋友,好几年跟过来,读到现在,到了后期,我就写这样干巴巴,毫无诚意的文章来搪塞他们?

%136
这不能!

%137
我创作本书的初心,就是要写一个不一样的魔头,不管其他什么东西,就是一个很纯粹的创作欲望。卖得好不好是次要的,钱包鼓不鼓是次要的,吃的饱不饱是次要的。所以我能到本书后期就马马虎虎吗?

%138
那不能!

%139
我若是这么干了,简直就是对不起我过去数年的努力和心血。对不起我当初的志向和初心。

%140
我更愿意看到的是,这本书的后期质量是好的,符合我的标准的,在水平线上的。

%141
大家现在跟读,每天一个章节,其实是看不出好坏来的。因为视线局限了。只有速读一大部分下来,才能觉察到这一部分的水准。

%142
我不想瞧到有这样类似的评论——《蛊真人》这本书前中期都挺好,后期崩了!逻辑错误一大堆,作者随便吃设定,《人祖传》更是写成狗屎!实在叫人失望。

%143
这是本书很关键的时期。

%144
很难。

%145
不仅是我在创作上感到从未有过的艰难,而且每次构思失败,都会让我感到自身的渺小,深深的挫败感。

%146
但我不会放弃,宁求质量,数量、稿酬都放一边。

%147
我希望大家都能多些理解,多些宽容。

%148
实在理解不来,我建议大家放一放。因为说实话,的确这样跟读,会影响阅读体验。我建议大家可以等本书完本再来读,如果能有个正版订阅支持的话,那就多谢捧场了。这本书真的花了我太多心血。

%149
最后,很不好意思,今天还是没有更新。

%150
这两天的构思成果,还是太差了,被我否决掉了。

%151
大纲上我决定还是要变动一下,方源被追杀的情节很繁琐,因为我会采用一个不同寻常的写法。里面涉及到的人物,我还要好好琢磨一下,才能放心地写出来。

%152
我和大家约定一下吧。

%153
以后的更新,只要是晚上八点到八点十分之间没有的,那就代表当天是没有了的。大家就不要再刷,再等更新了。

%154
本书到了后期,更新方面很不乐观,我估计是起不来了。稳定更新都很艰难。

%155
建议大家放一放,看看其他优秀的网文。如果实在要看本书,不妨重头来看。很多读者朋友在五刷、六刷。《蛊真人》这本书还是可以反复看的,这点我还是有自信的。

%156
最近就有不少读者重刷本书,还有几位好心人将看到的错别字、错误,都通过联络我。这些错误,我只要看到都会修改。我准备在最近一段时间,进行一场本书的大规模修订。

%157
我用的拼音输入法,错别字多,这方面我没有做好,会改正的!

%158
说了这么多,很不好意思。我当然更愿意将写这篇文稿的时间,用来构思《人祖传》。不过我看了这么多的书评之后,还是觉得有必要和大家沟通沟通。

%159
多谢大家一直以来的支持。

%160
蛊真人拜谢。

%161
如果大家不支持了,我也非常能够理解,请一路好走,好人一生平安。多谢你曾经的捧场。因为这的确是我没做到位,我能力不够。

%162
到了后期,我尽管心里很想,但我做不到稳定更新了。我只能这样写,才能写出符合我标准的来。

%163
我这样做,可能对不起某一些读者朋友们。但我觉得,作为一个作者,首先不能对不起自己的作品。

%164
《蛊真人》这部作品,就像是我的孩子,我一步步把他拉扯大。我不能在这部作品快要完成的时候,把它搞砸了!

%165
那就真的太可惜了。

\end{this_body}
\newsectionindepend{今天更新了狂蛮传番外}
\begin{this_body}
%166
今天更新了狂蛮传番外

%167
最近有不少朋友向我提了很多宝贵的建议。

%168
有人说:真人,咱们可以先写完正文,然后再补充《人祖传》啊。谢谢你的建议,但其实实施起来很困难。正文和《人祖传》是相辅相成的。如果正文写完,就好像是一座建好的大楼,再向里面塞小房子(人祖传)。最后看起来,会很别扭,破坏行文布局的空间和美感。

%169
这个想法挺好,其实我也早就想过。但若真的这么实施,最后很可能《人祖传》的质量会很糟糕。因为在一个建造好的建筑里搭房子,施展的余地真的太小了。

%170
虽然这个建议不会采纳,但也衷心地感谢你们。

%171
另外有个建议,我采纳了。

%172
有读者朋友说:真人,你不妨在你构思的期间,写一写番外啊,这些番外也很有趣。

%173
我想了想,这个建议真的好啊。这些番外虽然也要考验设定和逻辑,但构思和书写的难度,比正文+《人祖传》要轻松很多很多啊。

%174
所以,我今天补充了狂蛮传第二篇,大家可以去起点,蛊真人,作品相关里看到。这个番外是免费章节,我不会收钱的,大家可以免费看,也算是我这段时间对大家道歉的赔礼了。

%175
另外今天我修改了许多处的错别字。

%176
具体是在第一大章,有2、3、8、9、15、16、17、19、21、22、23、24、25、26、27、29、30、32、33、34、35、36、37、38、39、42、44、45、47、48、50。

%177
今天暂时修改到50小节。未来还会修订更多,希望广大读者朋友重读的时候,将看到的错误尽量汇报给我。谢谢了!

%178
今天更新了狂蛮传番外。

\end{this_body}


\input{chapter_06/chapter_06_069.tex}
\input{chapter_06/chapter_06_070.tex}
\input{chapter_06/chapter_06_071.tex}
\input{chapter_06/chapter_06_072.tex}
\input{chapter_06/chapter_06_073.tex}
\input{chapter_06/chapter_06_074.tex}
\input{chapter_06/chapter_06_075.tex}
\input{chapter_06/chapter_06_076.tex}
\input{chapter_06/chapter_06_077.tex}
\input{chapter_06/chapter_06_078.tex}
\input{chapter_06/chapter_06_079.tex}
\input{chapter_06/chapter_06_080.tex}
\input{chapter_06/chapter_06_081.tex}
\input{chapter_06/chapter_06_082.tex}
\input{chapter_06/chapter_06_083.tex}
\input{chapter_06/chapter_06_084.tex}
\input{chapter_06/chapter_06_085.tex}
\input{chapter_06/chapter_06_086.tex}
\newsection{置之死地而后生}    %第八十七节:置之死地而后生

\begin{this_body}

%1
灾劫持续不断,人祖的骨架开始显露出金芒。

%2
灾劫越发狂暴,人祖的骨架被不断烧灼,变得金光灿烂。

%3
人祖开始挺直腰板,昂首挺胸,忍受万千折磨,浑身骨骼宛若黄金浇筑,直直地站在灾劫当中。正是万劫金骨。

%4
而在人祖的骷髅脑袋上,也逐渐生长出了一个冠冕,便是通天骨冠。

%5
人们所承受的灾难,将成为他们来日的桂冠。

%6
虽然灾劫仍旧持续着,但人祖这时终于有了开口的能力:“我知道,宿命必有安排,灾劫就是它对人的安排!但我能坚持下去,任何看似不可忍受的灾劫,其实对于我来讲,都能忍受得住。”

%7
强蛊惊疑不定。

%8
自己蛊却是恍然:“我明白了。人啊,你是从平凡深渊中走出来的人。所以你能忍受不可忍受的灾劫呢。”

%9
强蛊见灾劫也收拾不了人祖,只好强自镇定:“人啊,你可别赖皮。你有希望蛊,灾劫会一直持续下去。难道咱们之间的赌约,就任凭你这样拖延下去吗?我可不管!灾劫这事,就当你过了。现在该轮到我们吃你了。”

%10
“你想吃我的什么?”人祖叹息道。

%11
强蛊哈哈大笑,指着人祖的胸骨处:“接下来,我要吃你的心!”

%12
人祖微微一震,强蛊的选择太致命了,人若是没有心,该怎么活呢?

%13
“快把你的心都拿出来,让我们吃!”强蛊迫不及待喊道。

%14
人祖苦叹,犹豫了一下,他先将同情之心取了出来。

%15
强蛊直接将同情之心,投入到困境的脖颈中,直接落入肚里去了。肚皮涨大了一点。

%16
困境中,人常常先失去同情之心。

%17
人祖接着又将高尚之心取出来,困境吞了,肚皮涨了不少,有些难以消化的样子。

%18
人祖再将自己原来的本心取出来:“希望蛊啊,快离开吧。我可不想连累你。”

%19
希望蛊寄居在人祖的本心中,却是没有飞出来,它道:“我才不走呢,这就是我的家。人啊,你索性连我也丢进去吧,我并不怪你。”

%20
人祖无奈,在强蛊的催促下,又将本心取出给困境吃了。

%21
困境吃了之后,肚皮又涨大许多。

%22
困境也常常让人失去希望。

%23
没有了希望蛊,困扰人祖一身的灾劫便逐渐消失了。这让人祖压力大减,却又怅然若失。

%24
“快,人啊,把你最后一颗心取出来,给我们吃!”强蛊指着人祖胸膛中的孤独之心。

%25
人祖犹豫为难,孤独之心不仅是自己蛊的寄托之所,更是他最后一颗心。没有了这颗心,人祖的性命也就不保了。

%26
强蛊大笑威胁:“快!你若不拿出心来给我们吃,我们就直接动手,把你整个都吃了!”

%27
自己蛊已被欺骗,它满不在乎地道:“人啊,你给他们就是。我无所谓的,你也不会死!我们是最强大的。”

%28
“给我!”强蛊一把夺过人祖手中的孤独之心,直接顺着困境脖子上的伤口,将其投入到它的肚子里。

%29
这下,人祖彻底没有了心。

%30
他直接栽倒在地上,没有了生息,再也爬不起来。

%31
人祖死了。

%32
“人啊,你死了,我们也自由了。”规矩蛊飞走了。

%33
“没有办法,人遇到了他一生中最大的困境。”勇气蛊、刃蛊等等也接着飞走。

%34
人祖的尸体上,只有自己蛊不断盘旋。态度蛊也想走,但被自己蛊死死拽住,没有得逞。

%35
强蛊欢笑:“哈哈哈,人啊你也不过如此。咦?困境你怎么了?”

%36
强蛊寄托的困境,捂住涨大到极致的肚皮,疼的满地打滚。

%37
孤独极难排解消化,越强大的困境中越显得孤独。

%38
轰!

%39
陡然间,困境的肚皮猛地涨破了。

%40
人祖的孤独之心,还有他的皮、肉凝聚成了一个男孩。

%41
强蛊目瞪口呆:“你是谁?”

%42
男孩叫道:“我就是人祖的儿子——大力真武!”

%43
话音未落,他跳起来,一把抓住了强蛊。

%44
强蛊使劲挣脱,自己蛊趁机飞上来,咬了它一口。

%45
强蛊受伤,虚弱了。

%46
“哪里逃!”大力真武大叫一声,直接一把抓住强蛊,将它按进自己的胸膛。

%47
强蛊落入他的胸口,被关押在了心房之中,怎么也出不来。

%48
“别白费力气了。这是我的勃勃雄心,你是出不来的。”大力真武大笑。

%49
随后,他从困境的尸体中找出了人族的其他几颗心脏,放回到人祖的胸膛中。

%50
人祖又重新活了过来!

%51
……

%52
毫无疑问,人的心乃是动力之源,是人一身的致命弱点之一。

%53
幽魂开创的食道杀招吃心,相比较吃苦、吃亏两招,明显更加优秀。后两招即便击中敌人,也只能在胜败的天平两端增添砝码。而吃心杀招却是致命至极,一旦中了,若无提前防备,几乎便能定局!

%54
这是能致胜的手段!!

%55
方源中招。

%56
安土重山堡防御极其出众,这不假。但是蛊仙流派众多,偏偏食道流传很少,很少有针对食道的防御。所以在宿命大战中,西漠一方的众多仙蛊屋面对天庭两仙联合施展出的——坐吃山空食道杀招,尽数中招,无可奈何。

%57
方源也掌握了一些食道传承,同时他也深知幽魂拥有着深厚的食道造诣。

%58
安土重山堡在方源的组建之下,当然可以防备食道。

%59
然而,眼下的安土重山堡已是被幽魂打破!

%60
即便完整的安土重山堡,也难以尽数挡下吃心杀招,仍旧会让方源中招。

%61
一瞬间,浓郁的死亡阴影笼罩到了方源身上。

%62
他无法挣扎,无力挣扎!

%63
至尊仙窍中的天道似有所感,竟然主动停止演化万劫,全数扩散,束缚方源全身。

%64
方源无法调动手段防御,只能静静等死。

%65
陆畏因见机不妙,连忙调动土道战场,黄沙凝聚如蟒如龙,纠缠在幽魂巨人身上,迅速勒紧,企图禁锢幽魂。

%66
幽魂冷笑,数百只手臂勉强撑住,防御得很消极。他的四只眼眸仍旧死死地盯着方源,绝大多数的精气神都催谷着吃心杀招。

%67
“糟糕!”陆畏因的一颗心猛地沉下去。

%68
气海老祖、吴帅更是心中冰凉一片。他们拼命攻击,却换不回幽魂的一个回眸。

%69
幽魂宁愿重伤,也死死抓住这个战机。

%70
他一定要将方源致于死地!

%71
“我……就要结束在这里了么?”方源脑海还是一片干涸,很少的念头在调动。这是刚刚催动气海无量杀招的后遗症。

%72
幽魂真的太强大了。

%73
不愧是曾经的魔尊!无敌于天下的男人!

%74
那个时代,他杀得全天下一片昏暗,无人敢发出声响,亿万万生灵只能在他的阴影中瑟瑟发抖。

%75
直至乐土仙尊出世,拼搏一生,才为天地万命治愈身心伤口。

%76
幽魂在还未全力发挥的时候,个人战力和周围对手差距不大,却始终把持着整个战局。等到他完全爆发的时候,战力暴涨,立即和周围拉开差距。更让人绝望的是,他的每一次选择都是如此犀利阴狠,立竿见影,不给对手留下任何喘息的机会,更遑论生还的余地。

%77
“不,就算只有一丝希望,我也要尝试到底!”方源无法调度任何手段,不过就算能动用杀招,也防备不住吃心。

%78
除非是有逆流护身印!

%79
方源满脸痛苦之色骤然消失,嘴角上翘,流露出一丝神秘的微笑,极其耐人寻味。

%80
下一刻,他蓦地出声大喝:“你此时不出手,还待何时?气绝!”

%81
幽魂一对眼睛鼓瞪,一对眼睛却同时眯起:“这是……方源故意诈我?不对!”

%82
幽魂的一个头颅回转,正看到气绝魔仙对幽魂动手!

%83
轰!

%84
气绝魔仙催动杀招,头顶上的兮地竟若流星一般,凶猛飞射,重重地砸在幽魂的背上。

%85
幽魂被打得趔趄,差点一头栽倒在地上。

%86
噗噗噗噗!

%87
他浑身上下陡然破开许多大洞,像是漏气一般,无数晦暗的魂气顺着大洞,向外喷射。

%88
不仅如此,幽魂环绕周身的上百只粗壮黑臂,在瞬间掉落下来,宛若枯朽腐烂的树枝。

%89
------------

\end{this_body}

\newsectionindepend{《蛊真人》不是黑暗文}

\begin{this_body} %begin a body

%90
呼……这一段的人祖传终于写完了。

%91
显而易见,这段人祖传,是迄今为止最长的一段。是构思了大约一周的成果。

%92
很难!

%93
我当初修改了不少次,单就意象而言,就得慎重选择。比如困境、强弱、灾劫。困境中包含的苦头、亏、灾劫,这三种意象是经过精挑细选的。在刚开始,这种意象至少有十个。随后的几天,我忍痛割爱删减了大半。最终关口,我又将难处、易处这两个意象去掉了。否则的话,全文会更长。

%94
这段人祖传,我需要呼应正文。所有,有了人祖吃苦、吃亏、忍受灾难、被吃心身亡的几个历程。分别照应了幽魂的食道杀招,以及方源本身的处境。

%95
许多读者朋友应该可以察觉到,这一次的人祖传和正文之间的联络,更多的是一种对比。两条线之间隐隐呼应。比如《人祖传》中人祖遭遇了一生中最强的困境,方源同样如此。人祖垂死挣扎,幽魂如此,方源也是如此,其他人比如王小二还是如此(王小二是个重要配角,大家往下看就知道了)。又比如人祖曾用态度蛊+自己蛊,企图欺骗困境,也是和方源在万年斗飞车那边利用自身意志,企图欺骗幽魂一样。这些东西都是相互交映的。

%96
人祖是《人祖传》中的主角,他身上的蛊虫在他和困境互动的情况下,又有什么样的精彩反应?这一点绝对不能不考虑。在这方面,我至少思考了三天。

%97
还有十绝体的问题,在这个章节中,终于又有可喜的进展——大力真武体诞生了!

%98
等到这段大剧情写完,大家可以重新看一遍。一天一更,大家可能觉得节奏缓慢,但其实顺下来看,会有不一样的发现。

%99
《蛊真人》这本书,你第一遍看和第二遍看,感受是不同的。尤其是这段章节,真正的精彩除了表面上的方源挣扎,对抗追杀,其实还有更多深层次的东西。比如人祖传和各个人物之间的照应,这从整体来看,才能发现这种铺设上面的美。又比如各方势力、各个人物之间的谋算,尤其是气绝、幽魂、方源三者之间……

%100
大家看完今天晚上的第二更,为萧真人盟主的加更,就会更加明白一些了。

%101
最近看到了一位读者朋友的书评,感触尤深。大意是:《蛊真人》这部作品其实是成长性的,起初是偏向黑暗风格一些,但得到了中后期越发大气磅礴。单说这本书是黑暗文,其实是偏颇的。

%102
这个书评触发了我对自身的审视。

%103
《蛊真人》写了六年左右,这么长的时间,的确风格上有所变化。诚如这个书评所言,刚开始偏向黑暗风,其后渐有所变。

%104
一方面,是外部原因,本书从开书一来,就屡遭举报,直至最近也是如此,并没有随着时间流逝而安全,反而越发惊险。国家大局如此,只得转变笔锋。

%105
另一方面,是内部原因。身为作者的我,成长不少。尤其是前段时间断更,有特别的感悟。

%106
写《蛊真人》前期的时候,我凭借的是一腔热血,桀骜和不驯,一门心思写,不管什么成绩和钱财。

%107
起初关注者寥寥,但而后越来越多,反响很大,两极评价极多。

%108
所以,写到《蛊真人》中期,我的耳畔尽是嘈杂之音,仿佛置身热闹街市。种种俗事纷至沓来,令我心境摇曳,心湖浮躁。

%109
写到现在,到了《蛊真人》的后期,因为《人祖传》,正文各种线索太多,令我举步维艰,写的异常艰难。时常有心无力,越感自身之渺茫。

%110
前段时间的断更闭关,拯救了我!

%111
就像是周围的闹市,逐渐逐渐收缩。身边如江海般的人流,越来越少,越来越少,直至仅剩下街口的一张书桌,一盏昏黄的台灯,一个我。

%112
一直萦绕耳畔的诸多声音,也逐渐消散全无,只剩下我砰砰直跳的心声。

%113
闭关的时候,我从未看过任何QQ留言,微信短信,就是一门心思地去想,去写。终于将眼前的瓶颈突破。

%114
其实,人祖传中人祖面临的最大困境,方源面临的最大困境,何尝不是我迄今为止写作方面的最大困境?

%115
如今,我再看《蛊真人》。这本书跨越了六年左右的光阴,遭受许许多多的打击、折磨、诅咒、质疑。你看书评,那些骂声、质疑声、咒骂声从未停止过。

%116
起先我怀揣激烈,随之心绪起伏不平,最终沉淀下来。

%117
只剩下的,是对完成这本作品的渴望和坚定。

%118
不管什么排行,也不管什么收入,不管世外纷杂,只是这个念头,纯粹的创作的念头。

%119
让他们举报吧,让他们质疑吧,让他们不屑一顾,让他们怒不可遏。

%120
我只管写成这部,我也只能管这么多。

%121
最近这段时间,我越发认识到,我个人的渺小,我的能力是非常有限的。

%122
我会就这样写下去,直至迎来《蛊真人》的完结。

%123
九月开始,我暂定一个小目标,那就是每天稳定一更,维持一个月!期间有新盟主,尽量加更一章。

%124
但如果在此期间,我觉得无法写出理想标准中的文字,我还是会选择闭关。届时,这个小目标就只能暂时放弃了。

%125
至于,闭关多长时间我也不清楚,反正要将难关攻克,首先得让自己满意。

%126
你若觉得好,你就支持。不支持,咱绝不强求。

%127
就这样吧。

%128
哦,另外,今天晚上有第二更,是新盟主的加更。同样,也是一个画龙点睛的章节。

\end{this_body}


\input{chapter_06/chapter_06_088.tex}
\input{chapter_06/chapter_06_089.tex}
\input{chapter_06/chapter_06_090.tex}
\input{chapter_06/chapter_06_091.tex}
\input{chapter_06/chapter_06_092.tex}
\input{chapter_06/chapter_06_093.tex}
\input{chapter_06/chapter_06_094.tex}
\input{chapter_06/chapter_06_095.tex}
\input{chapter_06/chapter_06_096.tex}
\input{chapter_06/chapter_06_097.tex}
\input{chapter_06/chapter_06_098.tex}
\input{chapter_06/chapter_06_099.tex}
\newsection{非到末路不甘休}    %第一百节:非到末路不甘休

\begin{this_body}

%1
方源手中捏着一只信道蛊虫,蛊虫中记载的正是有关之前追杀战的战报。

%2
大致内容是:天庭大胜,魔尊幽魂阵亡,方源逃窜,气绝魔仙、毛里球被镇压,正元老人被俘虏,影宗紫薇仙子、影无邪在逃,气海老祖、白凝冰、白兔姑娘、妙音仙子失踪,死亡概率极大。

%3
“看来当今天下,即便没有仙蛊屋排行榜,神帝城乃是仙蛊屋第一,也是实至名归了。”方源感叹一声。

%4
在他五百年前世,五域乱战时期,有信道大能联手排出了不少榜单。每一个排行榜都竞争激烈,仙蛊屋排行榜亦是如此。

%5
但是方源五百年前世,并未出现过神帝城。

%6
方源重生带来了种种巨变,如今仙蛊屋排行榜还未出世,就有了稳居榜首的神帝城。

%7
“照这样看来,即便将来有了仙蛊屋排行榜,神帝城第一的位置也会岿然不动。”方源估量着。

%8
神帝城能够镇压毛里球、气绝魔仙,方源并不感到奇怪。

%9
神帝城乃是由豆神宫、帝君城组合而出,虽然没有九转仙蛊,不是九转仙蛊屋,但内里却蕴藏壁画世界。

%10
这是元莲仙尊施展的安居乐业。它是人道杀招,画道效果,形成众生图。很明显,它是九转层次的杀招,尊者手段。

%11
在尊者手段下,龙公败过,帝藏生跪过。气绝魔仙即便兮地没有重创,也不过是亚仙尊级别,而毛里球距离亚仙尊还有些差距呢。

%12
“我的万年斗飞车、龙宫已毁,即便存在,也不过是八转顶尖层次。而神帝城却算得上半座九转仙蛊屋。如果当年八十八角真阳楼没有摧毁,还能和它比拼争夺第一位置。”

%13
当年的八十八角真阳楼,是由长毛老祖、巨阳仙尊合力炼制。它能分化成无数小塔楼,搜集野生蛊虫。合而为一后,形成主楼,还能为王庭福地排解灾难祸患。

%14
巨阳仙尊逝去,但留下八十八角真阳楼,使得巨阳即便死了,也能操纵整个北原政局。北原举办无数次王庭之争,黄金血脉在北原仙凡两界中确立无上地位,八十八角真阳楼居功至伟。

%15
而现在,神帝城乃是中洲人脉最大集结点,是中洲最大的人才存储、培养之地,人道圣地。论格局,和八十八角真阳楼不相上下。

%16
对于这样的仙蛊屋,方源也难免忌惮。

%17
至于天庭方面,为什么要将此战战报宣告天下,方源则完全能够理解。

%18
天庭方面太需要振奋士气了。

%19
自从宿命大战战败,方源当着全天下的面,将宿命蛊摧毁,天庭就像是被抽掉了脊梁骨,失去了精神旗帜,还被方源狠狠踩踏在脚下。

%20
这个结果影响极其巨大,整个中洲都笼罩在一层阴影之下。天庭蛊仙们愤恨图强之余,也有恐惧隐藏心底。而中洲十大古派更是士气衰败,无精打采。

%21
中洲乃是四战之地,这种情况着实危险。一旦被认为虚弱可欺,等到五域乱战,就会引起四域围攻。到那时中洲再强,宛若虎豹,也会被群狼所噬。

%22
一方面稳定内部,另一方面震慑外部,秦鼎菱必须要这样做!隐藏战果,或许能有些战术上的优势,但秦鼎菱乃是天庭领袖,要着眼天下。所以,她选择放弃战术上的优势,而甘愿在大势上挽回颓势。

%23
“天庭人才济济,即便失去了紫薇仙子这等大能,也有秦鼎菱稳定大局。天庭的底蕴实实在在,不是吹嘘的。”方源叹息一声,心中颇为认可秦鼎菱的决断。换做是方源自己,也会这样去做。

%24
到达他们这等层次,五域两天便是棋盘。彼此争锋,没有格局、眼界,即便能逞一时威风,最终也会落得惨败下场。

%25
回过头来,再看这场方源追杀战。

%26
毫无疑问,最大的输家便是影宗。亚仙尊战力的魔尊幽魂阵亡,影宗蛊仙死的死,失踪的失踪,逃到逃,被抓的被抓。

%27
然后便是方源。

%28
方源损失极其惨重。

%29
如今,整个至尊仙窍惨不忍睹。原本最有生机的小南疆,沦为一片片的荒山和废墟,只剩下几处资源点。其余小四域,以及小九天也都好不到哪里去。

%30
诸多异人、人族,尽管受到方源力保,也损失不小。

%31
末代战兽王此次抵挡万劫后,至今仍在昏死的状态,并未苏醒。

%32
仙蛊的损失,是方源有生以来最为巨大的一次。

%33
首先八转仙蛊屋就损失了两座,龙宫的核心仙蛊如梦令、万年斗飞车的似水流年都保下来,各有损伤,需要休养。而其余的大多数仙蛊,都在战斗中损毁,包括防备蛊、斗蛊等等。

%34
土道蛊虫损失也不少,虽然安土重山堡仍在,但幽魂自爆的威能相当恐怖。单单为了抵挡这一击,方源就损失了七只土道仙蛊,土道凡蛊不计其数。

%35
安土重山堡因此受到了严重削弱。

%36
这段时间来,方源先后经历宿命大战、抵消万劫、方源追杀战,八转仙元已近干涸。可以说,几乎耗尽了宿命大战前的辛苦积累。

%37
十二生肖战阵刚有起色,经过幽魂一战,又得重新积蓄。

%38
逆流河还只存一丝,还是难堪运用。

%39
气海无量杀招威能暴跌谷底,再不算是一张王牌。

%40
而从魔尊幽魂身上,方源根本没有获得什么的战利品。幽魂自爆,让一切都烟消云散。

%41
不得不说,幽魂难缠至极,即便是死,也没有一点好处留给方源。

%42
追杀战后,方源最大的收获便是保全自身性命,还有三千多道天道道痕,以及气海分身。但这些不是他本来就拥有的,就是他在宿命大战的收益,或者是谋算天庭的成果。和魔尊幽魂一点瓜葛都没有。

%43
幽魂就算是战死了,也让方源感到万分难受。

%44
影宗、方源都是输家,长生天同样如此。他们失去了毛里球这样的太古传奇强者,什么战果都没有。只是和影宗、方源比较起来,损失少了许多。

%45
在此之后,便是输多赢少的气绝魔仙。气绝魔仙从魔尊幽魂、方源两边敲诈了不少传承,不乏尊者真传的内容。但他兮地损伤惨重,又失去自由,被神帝城囚禁。

%46
所以此战最大的胜者,反而是天庭。

%47
幽魂毁灭,方源远遁,损失惨重。随后天庭击败长生天,劫运坛仓皇逃离中洲。豆神宫先后镇压了气绝魔仙、毛里球,最后秦鼎菱等人俘虏了正元老人,通缉紫薇仙子、影无邪二人。气海老祖、白凝冰、白兔、妙音疑战死。

%48
这样的战果,着实不小了。对于整个中洲上下,都是一剂强心剂。

%49
天庭战报传遍天下,不只是方源在分析输赢得失,全天下的蛊仙都在分析着天下大势。

%50
其中有一人,最为欢喜。

%51
“妙,妙啊!”大厅主位上,这位蛊仙手持战报,连连称赞,眼中浮现出喜悦之色。

%52
他中年模样,相貌普通,眉细眼长,体格强健,神态举止间流露出七分霸道,三分阴狠之气。

%53
正是武家第一太上家老,八转风道蛊仙,当世枭雄——武庸!

%54
坐在大厅左手第一位的,则是武家太上二家老武八重。他在武家中战力并不突出,但资历最老,为人稳重,敢于担当,曾经在武家被各大势力联手刁难之时,挺身而出,稳住局面,乃是武家肱骨重臣。

%55
武八重恭敬地请教武庸道:“不知何妙之有?还请大人明示。”

%56
武庸放下手中信道蛊虫,目光灼灼,指点江山道:“五域两天,强者为尊。所谓天下大势,不过是最巅峰的那些强者之争。强者僵持不下,才有其余弱者,以及超级势力的发挥余地。若产生尊者,自然便是一人无敌,独领风骚,群雄垂首。”

%57
“眼下尊者不生,天下大势便在四人身上。其一魔尊幽魂,其二古月方源,其三气绝魔仙,其四气海老祖。除此之外,便是天庭、长生天这等超级势力,拥有劫运坛、神帝城这类仙蛊屋,可以参战,影响大势。”

%58
武八重闻言,顿感困惑,旋即便问:“神帝城如今已经被公认为第一仙蛊屋,可镇压气绝魔仙。劫运坛只是稍逊一筹,却也是宿命大战中大放光彩的极强仙蛊屋。大人为何将天庭、长生天排在那四人之下。”

%59
武庸微微一笑,从容答道:“这两座仙蛊屋的确强大,但也只是强盛一时。仙蛊屋最大的缺陷,便是手段固定单一。一旦被人摸清,就可针对,而仙蛊屋却难以改良,远不如亚仙尊灵活多变,可以迅速调整。”

%60
“神帝城是很强大,能镇压亚仙尊气绝魔仙。但当它的手段被摸清楚,威慑力必然下跌一个档次。天庭缺乏亚仙尊,是他们最大的弱点。此战若非气海老祖站在天庭一方,秦鼎菱必然不会有如此战果。”

%61
“原来如此。”武八重恍然,“属下现在明白大人的喜色了。这场追杀战,幽魂阵亡,方源狼狈逃窜,气绝魔仙被镇压,气海老祖失踪,战死概率极高。决定天下大势的四位亚仙尊,对耗惨烈,正是我等发展的良机啊。”

%62
武庸点头:“这等强人当然是死光了才好。最遗憾的便是方源逃出生天,唉,这个魔头深不可测,决不可以常理揣度。说不定等到他重现天日,又会比之前更加强大!”

%63
谈及方源,武庸满脸凝重之色。

%64
武八重也是忧虑重重。

%65
方源可是和武家矛盾极深,而今幽魂阵亡,方源绝对是天下第一大魔头。这要让他喘息过来,对付武家,那可就是天塌地陷般的恐怖灾难了。

%66
武庸继续道:“亚仙尊这个层次,我们是插不上手的。所以,我们要趁机全力发展,拼命壮大,将来方源来找我们麻烦,我们至少能有一拼之力。若是这种情况将来没有发生,那五域乱战也是必然大势。”

%67
“中洲、北原有天庭、长生天把守,经营无数岁月,铁通一般,暂且不谈。”

%68
“余下三域南疆、西漠、东海,皆是一盘散沙。”

%69
“时代浪潮已然掀起,武家正当乘势而起。西漠沙海遍地,绿洲如星,易守难攻。而东海却是资源丰盛,从无尊者出世,民风最为温和。我们应当先整合南疆蛊仙界,再收东海,其次西漠。囊括三域资源,助我渡劫,成为亚仙尊,再培养八转蛊仙,构造顶尖仙蛊屋。最后,再向中洲、北原动手!”

%70
“大人!”武八重还是首次听到这番言论,不由瞪大双眼,为武庸描绘的大略震撼兴奋,同时又有忐忑不安。

%71
这可是侵吞天下之志啊!

%72
武家能被武庸带上巅峰吗?又或者被时代的浪潮冲毁成渣?

%73
“风道从未有过尊者。若有可能,我或许可以弥补这份空缺。将来的事情谁说得准呢?”这句话武庸并未说出口,只是心中念叨。

%74
他端坐主位,目光似乎穿透大厅,远眺苍穹。

%75
下一刻,武八重便听武庸悠然长吟道:

%76
“永生缥缈非我求,长生无为老愧羞。”

%77
“界壁消散乱世起,宿命一去竞自由。”

%78
“鹰击长空鲸霸海,不试怎知龙与蚯?”

%79
“凡夫俗子岂识我,非到末路不甘休!”

%80
武八重听闻动容,站起身来,又旋即跪拜在地,声音打颤,发誓道:“属下愿随大人左右,为我武家霸业,肝脑涂地,不死不休!!”

%81
------------

\end{this_body}

\newsectionindepend{《蛊真人》伴随着大家一起成长}

\begin{this_body} %begin a body

%82
今天的更新,不知不觉间又到了第一百节了。最近得玉心道者的提醒,我时隔数年,再次关注了一下《蛊真人》的数据。单看本书的起点方面的数据,总推荐票数,已近三百万。起点会员点击量突破了一千万。本书写到了六百四十五万字,根据细纲推断,突破六百六十六万应该不难。

%83
让我感触最深的是,写今天这章的时候,武庸的这首唱诗几乎是我一气呵成的。当然,对仗并不工整,但感觉到位了,味道也有了,所以我就不修改了。

%84
回想从前,我写第一首唱诗的时候,那叫一个难啊。前后想了三天,这才琢磨出来一首。当时就在想,什么时候我能够写这样的唱诗,能够一气呵成呢?

%85
现在,我很想回答六年前的自己:“别着急,慢慢打磨,一点一滴积累,六年后你就有这个功底啦!”

%86
就像《蛊真人》这本书的描述内容:人创造了时代,时代又影响了每一个人。在这里,远不止是主角,所有的人都在成长。每个人的实力层次不同,想法就不同,处境不同,计划就不同。而每个人的举动,造成的结果,又在相互影响,彼此交织。

%87
正是这样,才交汇出一副乱世图,会有一个波澜壮阔的乱战大时代!

%88
我想到许多读者的留言,有不少人都在说这样的一个阅读体验:《蛊真人》陪伴了他们许多年,从初中或者小学开始看,现在是大学毕业或者成为了高中生。

%89
这种感觉,让我感到很好,平静的心湖中充盈着一股淡淡的欣慰和喜悦。

%90
这种欣慰和喜悦,来源于成长。

%91
《蛊真人》这本书,伴随着的不仅仅是书中角色的成长,也有作者我的,还有无数阅读本书的读者朋友们的。

%92
这样就很好。

%93
真的很好。

\end{this_body}


\input{chapter_06/chapter_06_101.tex}
\input{chapter_06/chapter_06_102.tex}
\input{chapter_06/chapter_06_103.tex}
\input{chapter_06/chapter_06_104.tex}
\input{chapter_06/chapter_06_105.tex}
\input{chapter_06/chapter_06_106.tex}
\input{chapter_06/chapter_06_107.tex}
\input{chapter_06/chapter_06_108.tex}
\input{chapter_06/chapter_06_109.tex}
\input{chapter_06/chapter_06_110.tex}
\input{chapter_06/chapter_06_111.tex}
\input{chapter_06/chapter_06_112.tex}
\input{chapter_06/chapter_06_113.tex}
\input{chapter_06/chapter_06_114.tex}
\input{chapter_06/chapter_06_115.tex}
\input{chapter_06/chapter_06_116.tex}
\input{chapter_06/chapter_06_117.tex}
\newsection{幽魂的真正本体}    %第一百一十八节:幽魂的真正本体

\begin{this_body}

%1
“原来盗天魔尊出手,不只是天相杀招,还有这一处,只是我一直没有察觉罢了。”方源口中喃喃。

%2
他忽然心头又一动,脱口问道:“青仇是否也没死?”

%3
方源追杀战末期,神帝城急吼吼地拔地飞来参战,真的只是为了镇压气绝魔仙和长生天吗?

%4
方源想到的是青仇的那几块碎尸。

%5
神帝城的主要目的,究竟是什么?

%6
这很值得商榷!

%7
陆畏因脸色微肃,感叹道:“这一点,我也是回到这里休养了许久后,才琢磨出来的。神帝城参战的主要原因,究竟是什么?当时的情况,九转仇恨蛊的确是毫无气息,似乎彻底毁灭了。”

%8
“然而,青仇本质上是由杀招形成的生命。它究竟是死是活,我也无法肯定。元莲仙尊的这记手段,着实匪夷所思,令人叹为观止!”

%9
能够将杀招转变成太古传奇,这手段太玄妙,即便是方源、陆畏因也只能干瞪眼,无法揣度。

%10
陆畏因继续道:“青仇生死状况,我无从推算。但幽魂魔尊的状况,我却是一清二楚。”

%11
方源微微一笑:“愿闻其详。”

%12
陆畏因卖了这么久的关子,终于回到正题上来。

%13
他答道:“幽魂魔尊目前的状况,可以说是死,也可以说是活着。因为他真正的本体,一直都藏在生死门中。”

%14
方源点点头,感叹一声:“果然不出我的所料。”

%15
魔尊幽魂当时自爆,就已经让方源感到略微不妥。等到天道道痕尽数炼化后,方源反思,心中的疑虑就更多了。

%16
堂堂幽魂,就这样死了?

%17
要知道幽魂可是所有尊者当中,最为逆天的存在!

%18
方源五百年前世,僵盟健在,影宗潜伏,幽魂暗中掌控了天庭。龙公始终没有出现,显然是被幽魂暗害。他还设计杀害凤九歌,顺势侵占琅琊福地,他操纵大局,是五域乱战的幕后最大黑手。

%19
只要大梦仙尊没有诞生,他幽魂就是最大赢家。

%20
于是,又有了方源入侵狐仙福地,身边有宋钟协助,斩杀凤金煌的事件。

%21
凤金煌是谁?

%22
凤金煌童年便有梦翼仙蛊主动来投,是大梦仙尊种子!

%23
方源是谁?

%24
方源是天外之魔,宿命未毁的情况下,只有借助他的手才能有所改变。

%25
宋钟是谁?

%26
宋钟明面上是宋紫星的儿子,实际上是他的血胎所化。

%27
而宋紫星呢?

%28
他号称血龙,是影宗中人,是幽魂的分身!

%29
方源五百年前世,可谓内幕重重。

%30
那个时候,宿命蛊仍未毁灭,气绝魔仙无法重生,其他尊者也只能对幽魂干瞪眼。

%31
幽魂猖獗,最终逼得天意不得不选中方源,回到过去,来改变这一切!

%32
义天山大战,表面上看,方源成功阻止了幽魂。

%33
但真的是这样吗?

%34
幽魂得知方源重生回来,立即明白是天意所为。他若是按照原来计划发展,即便可以掌控天庭,操纵五域大战,也始终都会遭受天意阻截,一个方源不行,就会有第二个、第三个天外之魔重生回来,幽魂将始终无法真正立足宇宙之巅。

%35
所以,他改变想法,那一刻他才真正开始和红莲魔尊合作。

%36
最终,他将至尊仙胎蛊送给了方源,打造出了完整的天外之魔!

%37
宿命大战之后,宿命蛊被方源摧毁了。

%38
幽魂想要夺回至尊仙体。

%39
他的谋算很完美。魂道境界不缺,拥有至尊仙体,他可以吞并安魂洞天等等,顷刻之间就能重新崛起!

%40
如此一来,他就领先所有的尊者,占据先手。

%41
先手优势之大,超越想象,一旦幽魂得逞,就是唯他独尊,宇宙天地都会被他彻底踩在脚下。

%42
但是他没有得逞。

%43
红莲魔尊算计了他,天道道痕束缚住方源的同时,也保护了方源。

%44
幽魂没有得到至尊仙体,这发展的差别就大了去了。

%45
方源追杀战,天庭、长生天同时出手,又有乐土仙尊的传人陆畏因参战,终将魔尊幽魂击杀。

%46
回顾幽魂的谋算,方源设身处地想:若他是幽魂,又该怎么去做?

%47
于是,他很快就发现了幽魂大计中的一个巨大破绽——

%48
太弄险了!

%49
梦境大战,龙公击败紫山真君,攻破福地,将魔尊幽魂生擒活捉,取走了生死门。

%50
魔尊幽魂生死操控在天庭手中,若是紫薇仙子没有贪图情报和线索,将魔尊幽魂直接斩杀,岂不糟糕?

%51
若是方源绝不会这样安排。

%52
幽魂呢?

%53
他纵然是杀性第一,也绝不痴傻愚蠢吧。

%54
所以,最大的一个可能,就是魔尊幽魂并非幽魂的真正本体!

%55
如果方源处在幽魂的位置上,他就会这样做——用一个分身伪装成本体,被天庭擒拿俘虏。

%56
一方面,能够降低甚至打消敌人戒备心。

%57
另一方面,分身也是一个陷阱,紫薇仙子最终就中招了。

%58
第三方面,分身掌握的情报和真传,远不如本体,纵然被俘虏,也不会过多资敌。

%59
最后一个方面,就算分身被杀了,也没有关系,本体还保全完好。

%60
对于幽魂的怀疑,方源还有两大证据。

%61
第一个证据,就是气绝魔仙。

%62
气绝魔仙是方源追杀战中的关键人物。

%63
方源和幽魂利用各自的真传,全力争取气绝魔仙成为己方盟友。

%64
最终气绝魔仙被方源争取了过来,这才令魔尊幽魂最终自爆毁灭。

%65
方源战后反思,总觉得自己的这场争夺战,胜利的太轻易了。和自己预料的,有些许差距。

%66
如果魔尊幽魂只是分魂,那就说得通了。

%67
幽魂料不到气绝魔仙的重生,因为宿命蛊被毁,一切未来的轨迹就都崩溃瓦解,一片混乱了。

%68
幽魂算计的是分魂被天庭俘虏,所以他只会精挑细选一些记忆,放入分魂当中。这些记忆既不会引发天庭的怀疑,也不会有过多的珍惜内容,比如价值极高的传承。

%69
所以,方源在这方面战胜了幽魂。

%70
第二个证据,是方源最近才得到的。

%71
便是他通过察运,揣摩自身气运,发现自身纯银光柱上端,笼罩着三团云层。其中一层,漆黑如墨。

%72
谁还能对他造成困扰?

%73
这团黑云,应当就暗示着幽魂!

%74
幽魂未死,岂会对方源善罢甘休?影宗、僵盟累积了多少年的资源,其中更艰难地筹集了十绝体,这才打造出了至尊仙胎蛊。幽魂会放弃?怎么可能!

%75
另一方面,气绝魔仙的重生,已经告诉天下人,尊者重生也绝非不可能的事情。

%76
幽魂真正的本体,恐怕早已经开始筹谋重生了。

%77
推而广之,其他的尊者难道就没有这个念想吗?

%78
以前是有宿命蛊阻拦,没法复活。现在宿命蛊都毁了,谁不想复活?!

%79
谁也不知道,究竟会是哪个尊者先真正彻底的复活。谁也不清楚,究竟会是什么时候复活。说不定,就在下一刻,就有尊者复活了。

%80
所以,对于方源而言,情势是严峻的。

%81
一旦尊者复活,他的亚仙尊战力的价值就大打折扣。天下第一魔头的地位,恐怕会成为标靶。

%82
追杀战后,陆畏因和他告别,告诉他要争分夺秒,也就是这个意思。

%83
方源也是这样做的。

%84
追杀战后,他是确确实实在争分夺秒,没有一刻放松。

%85
一等到他实力回升,可以应对潜在风险的时候,方源就来到了道德乐土。

%86
他来这里,当然不是为了见陆畏因的,也不是为了获得什么乐土真意,或者继承乐土真传。

%87
他来的目的只有一个——

%88
“告诉我,陆畏因,如何才能成尊?”

%89
宿命蛊被毁,整个天地的未来都混乱起来,什么事情都可能发生。

%90
不仅是整个蛊仙界,蛊师界也预感到了未来的动荡大趋势。

%91
方源在抗争,凡人蛊师们也在挣扎。

%92
悲风山脉。

%93
有四位蛊师正联袂同行,深入山脉进行探索。

%94
他们皆有四转修为,配合起来,更能抗衡5转蛊师。在中洲蛊师的世界中,他们是赫赫有名的魔道强者。

%95
他们分别是东淫陈淫道,西贱郁八光,南骚施暴,北荡樊春耀,合称中洲四大淫贼。

%96
“又有悲风刮来了。”郁八光首先示警。四人当中他最擅长侦查。

%97
“快,取出蛊屋来。”陈淫道立即开口,通常行事,四人都以他为首。

%98
蛊屋是由樊春耀保管的,他负责治疗和后勤。

%99
四位蛊师连忙躲进蛊屋。

%100
悲风在蛊屋外猛烈吹拂,刮得蛊屋咔咔作响。时不时的,樊春耀就要填补蛊虫,将那些承受悲风而亡的蛊虫更替。

%101
“亏老本了,亏老本啦!”樊春耀咋咋呼呼,心疼不已,“这一路走来,我们光是抵御悲风,就亏损了太多。悲风山脉中真有好东西吗?”

%102
陈淫道摇头苦笑:“我当然不能保证。但是只要还有一丝希望,我们就得这么干!前一段时间,悲风山脉上有蛊仙大战,说不定遗漏了什么好东西,能让我们升仙。”

%103
一提到升仙两个字,其余的三位蛊师的呼吸都有些粗重起来。

%104
“只要升了仙,谁再敢对我们四位仙人动手?我们就不会被追杀了。”

%105
“一旦我们成了仙人,那些高高在上的女蛊师,还不是任由我们拿捏?哈哈哈。”

%106
“瞧你这出息!升了仙,凡夫俗子有什么好的。我们能对仙子下手了啊!”

%107
几位蛊师相互对视,然后一起嘿嘿嘿的笑出声来。

%108
他们却不知道,天庭早已经将战场打扫干净,那些太古年兽尸躯等等都被收刮走了。

%109
当四人来到山峦崩塌的废墟上,搜寻了七天七夜,什么都没有捞着。

%110
就在四人气急败坏,准备离开的时候,他们终于有了发现。

%111
“这里有个人!你们快来。”

%112
“有个人有什么好奇怪的,这里埋了不少尸体。是生活在山中的村民。”

%113
“不,这个人还活着!而且他周围还有异象。”

%114
“什么?”

%115
其余三位蛊师一拥而上,就看到了双目紧闭,仍旧有一丝气息的王小二。

%116
而在王小二的身边,几朵野生的花草,不断盛开凋零,青枯二色在呼吸间迅速转换。

%117
------------

\end{this_body}

\newsectionindepend{目标完成了!}

\begin{this_body} %begin a body

%118
这是本月最后一天,吐出一口浊气,如释重负。

%119
之前斗胆提出的小目标,我终于是完成了!

%120
本月一天一更,稳定更新,从未缺席。

%121
很难。

%122
有好几次,我都差点坚持不住。生活是方方面面的,总会牵扯精力和时间。要想像学生时代那样,全力学习一般的去全力写书,这是不可能的事情。我是一个成年人,成年人的世界充斥着一种东西,叫做——不容易。

%123
另一方面,《人祖传》接下来的篇章,也着实耗费了我许多心血和脑汁。

%124
下个月,就有新的《人祖传》篇章和大家见面了。

%125
《人祖传》越到后期难度越大,几乎是倍增。因为很多意象和概念,只要前文出现过,基本上就用不了了。

%126
能够一边构思《人祖传》,一边完成一天一更的目标,我现在回顾,都觉得有些侥幸。

%127
好在坚持了下来,真的完成了。

%128
压力真的太大了!

%129
这个月一大半,我都是十二点之后睡的。有许多天,尤其是最近这个星期,基本上都是构思到夜晚三四点。

%130
六百五十万字的正文,要结尾,要填太多的坑,收束太多的线索。但这个月,已经填了不少,几个线索也在收束。

%131
更多的精力都耗费在《人祖传》。有时候心里很慌,因为《人祖传》的资料无从查起,很多时候学习是没有明显成效的。很多时候,打破障碍的,只是我的一个点子,或者十天,十几天前学习到的东西,或者是前一刻无意中看来的一句话。

%132
正文是可以把控的,《人祖传》是没有办法把控。有此带来的不安,导致心理压力巨大,很多时候提前躺在床上,想要睡觉,闭上眼,死活睡不着,又爬起来构思。

%133
想得身心疲惫,把自己丢在床上,这才能睡得着。

%134
成果是显而易见的。

%135
本月更新的质量,是我较为满意的。不管是行为结构,还是支线的描绘,都顾及到了。尤其是断章的水准,提升了不少。

%136
《人祖传》也构思出了新篇章。

%137
预计在十月初旬,会有一场大戏,发生在疯魔窟。前期的情节已经酝酿了大半。

%138
最后呼吁一下:

%139
希望广大的读者朋友们,尽量支持《蛊真人》的正版阅读,尽量在起点中文网阅读。

%140
一章内容,我至少构思大半个小时,写要几个小时。但读者朋友们阅读,可能十分钟不到,付费也就几毛钱,甚至很多起点币是赠送的。

%141
《蛊真人》这部作品,也不会更新太长时间,衷心希望大家能够尽量支持正版!

\end{this_body}


\input{chapter_06/chapter_06_119.tex}
\input{chapter_06/chapter_06_120.tex}
\input{chapter_06/chapter_06_121.tex}
\input{chapter_06/chapter_06_122.tex}
