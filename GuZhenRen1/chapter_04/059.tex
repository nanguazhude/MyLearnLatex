\newsection{归还八大荒兽}    %第五十九节:归还八大荒兽

\begin{this_body}

%1
北原,琅琊福地。

%2
轰轰轰……

%3
巨大的冰刺神猿,每一次行进的脚步声,都宛若巨鼓轰鸣。

%4
走到云阁之前,它停下脚步,忽然仰头长吼,吼声激荡风云,远远传播开去。

%5
方源和琅琊地灵站在远处,正在交谈,听到声音后,皆投去目光。

%6
“这头小猴子!”琅琊地灵伸出左手,抚摸胡须,哈哈一笑。

%7
他受到的气道封禁,已经从原来的十几层,削减到只剩下六层。多亏了北原墨人王出力相助。

%8
因此琅琊地灵的左手,已经可以自由活动。如今只剩下右手仍旧动弹不得。

%9
方源也含笑着,看着云阁缓缓升起,露出云土地基下的洞窟。云雾飘渺,渐渐笼罩住冰刺神猿。冰刺神猿缓缓踏入洞窟,高大如山的庞大身躯渐渐消失在方源的视野中。

%10
最后云阁缓缓下降,将洞窟重新fēng阴。洞窟中的蛊虫催动起来,冰刺神猿陷入沉眠当中。

%11
不远处,其他的云阁也在上演同样的情形。

%12
只是埋入云阁地基的,不是冰刺神猿,而是凤羽熔岩鳄、金砂乌骓等等荒兽了。

%13
“冰刺神猿、凤羽熔岩鳄、金砂乌骓、青龙藤、泥沼蟹、桃太郎、铁冠鹰、地魁荒兽,你借给我的八头荒兽,我都完好无缺地还给你了。其实隔段时间,就把它们放出来溜溜,也有好处。总是让它们沉眠,虽然节省食料。却影响它们的战力。”方源道。

%14
琅琊地灵翻了个白眼:“臭小子,你话里有话,以为我听不出来?是想以后再借我的荒兽。帮你打架吧?哼,我的这些荒兽可是我好不容易积攒下来,是琅琊福地最主要的防御力量。你下次再借,绝不可能一下子借八头!”

%15
方源不以为意地笑笑,伸出手来:“好了,我还了你的荒兽,按照协约。你该将江山如故、人如故两大仙蛊还给我了。”

%16
琅琊地灵一听,顿时气势陡落,讪讪一笑:“这个……方小子。啊不,方老弟,江山如故、人如故两大仙蛊果然玄妙非凡,我还没研究透彻。不妨多宽限一段时日如何啊?”

%17
方源脸色一板:“想都别想。拿来!”

%18
“方老弟,不不,方兄方兄,你不可以这么不近人情啊。你要知道我受到气道封禁,实力发挥不出来,根本研究不出什么名堂。这场交易,我十分吃亏!”琅琊地灵叫道。

%19
“做生意你情我愿,又不是我逼你的。你吃亏也是你自己的事情。现在想反悔?晚了!快拿来,你可别忘了我们都是用山盟蛊发过誓的。而且还牵扯到你的朋友墨坦桑。谁要是反悔,可得赔上性命的。”方源冷笑不止。

%20
“地灵啊,你还是还了仙蛊吧。”墨坦桑赶了过来,眼巴巴地望着琅琊地灵。

%21
琅琊地灵看看墨坦桑,终究受不住后者的眼神,狠狠一跺脚,万分不舍地将江山如故、人如故两只仙蛊,交还给方源。

%22
“臭小子,每次都能让你讨了便宜去。给你给你,你这个奸诈的家伙!”琅琊地灵口中喋喋不休。

%23
方源不悦地冷哼一声:“我讨便宜?我讨了什么便宜?你的八头荒兽我都给你喂得饱饱的,什么惨斗都没有发生,只相当于活动热身了一下。我为你节省了一笔开销,还没找你算账呢。”

%24
“算账,算什么账?我不也养了你的青云直上仙蛊嘛,我为了收买食料,也付出了许多仙元石。你要算账的话,咱们就好好算算,到底是谁付出的多!”琅琊地灵跳脚,立即还以颜色。

%25
荒兽喂养,当然比喂养仙蛊要简单的多,更廉价得多。

%26
方源这点说不过琅琊地灵,不过他掌握着琅琊地灵的弱点,因此气势毫不低落,冷笑道:“琅琊地灵,你还想不想再研究江山如故、人如故两只仙蛊了啊?想不想啊?”

%27
琅琊地灵脸色顿时一变,讨好地笑道:“想,做梦都想啊。”

%28
“想就行了。”方源拍拍琅琊地灵的脑袋,“斤斤计较有意思嘛,谁叫你不放心我,硬要扣押我的青云直上仙蛊当做保险的呢?”

%29
“滚!”琅琊地灵气得一把扫开方源的手,“臭小子没大没小,老夫可是你的前前前前辈!”

%30
“嗯?”方源一瞪眼,亮了亮手中的两只仙蛊。

%31
琅琊地灵顿时变脸,笑呵呵地拍着方源的小腿道:“方源小子,没办法,老夫就是看你顺眼,我们可是忘年交啊!”

%32
旁边站着的墨坦桑,目睹这一切,深深的无语了。

%33
方源将两只仙蛊放入自家仙窍,又取出三张仙蛊方。

%34
在狐仙福地的五个月来,方源已经和琅琊地灵又完成了六笔交易。

%35
仙蛊方残缺越多,推算时消耗的青提仙元就越多,前后的时间也越长。方源现在交付的三张仙蛊方,原本的完善程度只有六成,足足耗费了大半个月,这才勉强完成。

%36
琅琊地灵接过仙蛊方,看了一眼,摇头叹道:“小子,老夫不得不说,你的确是推算蛊方的天纵之才。实在可惜了,你现在堕落成了仙僵。”

%37
方源顺势便问:“我也想找到摆脱仙僵的法子,你这就没有办法吗?”

%38
“方法当然一大堆,但几乎都是老法子,跟不上时代喽。有的风险太高,有的可能性很小,有的材料早在很久就绝迹了,有的法子连我都不敢确信。”琅琊地灵摇摇脑袋,“毕竟老夫是地灵,平常可不会考虑什么仙僵的问题。”

%39
“方兄或许可以混入北原僵盟分部。僵盟的高层,几乎都是仙僵,他们一直在研究如何摆脱仙僵之躯。回复生命的方法。且我在早些年前,曾经听到一些风声,说北原僵盟内有人。在这方面得到了突破。”墨坦桑这时建议道。

%40
方源点点头:“我亦有此打算,但却不好露面。我破坏了八十八角真阳楼,毁灭了王庭福地,不方便公开露面。万一被推算出来,那就糟糕了。”

%41
“如今东方长凡已死,能够推算得出方兄跟脚的蛊仙,整个北原恐怕都不会有了。”墨坦桑道。

%42
“也难保有些千岁老怪。或者潜藏在暗处的智道蛊仙。”方源摇摇头,叹息一声。

%43
“咦?小子,你不也是智道蛊仙吗?你能推算出这么多的仙蛊方。智道造诣已然极强,你可以自己用你的智道手段,防止他人推算自己啊。我记得很多智道蛊仙都这么干过。”琅琊地灵道。

%44
方源心中苦笑一声。

%45
他哪里是什么智道蛊仙?只不过借助了九转智慧蛊的一些威能,滥竽充数。厚颜冒充的。

%46
当下。他只好这样说:“我的智道手段,十分偏重于推算蛊方,其余方面就薄弱很多。况且,马鸿运、赵怜云二人是两个活生生的目击证人,亲身经历了王庭福地的变故。恐怕我的形态相貌,已经被秦百胜拷问出来了。”

%47
谈到了秦百胜,墨坦桑目光一闪,忍不住赞道:“此人的确是个人物。我原以为他只是战力出众。没想到智谋也是高妙。他俘虏马、赵二人,抢到运道真传的精髓核心。原本要被北原各大蛊仙联手剿灭的。结果北原时间的这一个多月来。他合纵连横,主动抛弃利益,说动黎山仙子出面,利用山盟蛊,勾连左右,硬生生团结了一大批散修,甚至连耶律家都被他说动。”

%48
“原本一场大战在即,结果现在被他上下折腾,居然化解了。还要在他的福地中,搞一场拍卖大会,特意拍卖运道真传的核心仙蛊,甚至连马、赵二人,都成了拍卖的货品。”

%49
说完这番话,墨坦桑饱含深意地看了方源一眼。

%50
他知道方源和黎山仙子,关系匪浅。

%51
当初,方源向琅琊地灵借荒兽,他就在场。琅琊地灵一口拒绝,态度刚硬,皆因荒兽乃是琅琊福地的最主要防御力量。若是方源怀有歹意,和之前进攻琅琊福地的势力是一伙的,琅琊地灵就危险了。

%52
但方源出乎意料地,拿出了山盟蛊。

%53
用此蛊发誓,彻底打消了琅琊地灵的顾虑。

%54
之后又用江山如故、人如故两只仙蛊,引诱琅琊地灵,一下子击中琅琊地灵的软肋。要知道之前,琅琊地灵为了研究江山如故,不惜向太白云生开出五块仙元石的高价。

%55
琅琊地灵被两只仙蛊打动,又被方源言语刺激,一下子借出了八头荒兽。

%56
“秦百胜能舍能弃,识时务,手段也厉害,可谓文武双全,的确叫人佩服赞叹。”方源淡淡附和了一句,旋即话题一转,问地灵道,“你这边可有白莲巨蚕蛊的线索?”

%57
方源手中的净魂仙蛊,还未喂养,经过这段时间,变得更加虚弱。

%58
而喂养净魂仙蛊,则需要白莲巨蚕蛊的肉。

%59
之前方源已经询问过琅琊地灵,结果地灵手中并没有白莲巨蚕蛊的蛊方,甚至连残方都没有。

%60
所以这次,方源只问线索,不提蛊方的事情。

%61
琅琊地灵摇头:“小子,这白莲巨蚕蛊乃是幽魂魔尊所创。老夫本体经历了盗天、巨阳两大时代,终于寿终正寝。幽魂魔尊是在巨阳仙尊之后崛起的,那个时候,我这个地灵已经在福地里足不出户了,哪里来的什么线索?”

%62
“不过,我这里倒是有一个变化道的奇妙杀招。这杀招来源于自当年的盗天魔尊。盗天魔尊年轻时候,就靠着这套蛊虫,遮掩真容,躲避追杀,效果十分出色。你有这个杀招,就能变化外貌,改变气息,伪装身份。就算是蛊仙算出你的身份来,嘿,说不定也能当面否认,推说算错了,蒙混过去!”

\end{this_body}

