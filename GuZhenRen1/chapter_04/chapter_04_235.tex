\newsection{致命的卧底}    %第二百三十六节:致命的卧底

\begin{this_body}

%1
眼看着仙蛊屋黑牢,宛若一颗流星,撞向银色巨人。

%2
这时,银色巨人的三只手臂,发生了变化。

%3
其中一个手,猛地张开五指,从掌心中冒出一团星云漩涡,深蓝漩涡飞速扩张。很快膨胀成正常星云磨盘的五六倍,自我旋转比之前更快了五六成。

%4
另一个手臂横折在自己的胸口,前臂骨陡然变化,凝聚出一块巨大的藏青厚盾。

%5
这个盾牌如此高大,竟然将银色巨人高达数十丈的身躯,遮住了大半。

%6
最后一个手臂,手指垂下,仿佛拈着一片淡黄色的半透明薄纱,比起前两者并不起眼。

%7
仙蛊屋黑牢撕裂空气,已近在咫尺。

%8
银色巨人便将手中的星云磨盘,往前一挡。

%9
黑牢轰的一声,被星云磨盘吸纳进去。

%10
星云磨盘开始迅速搅磨,发出咔咔的巨响,体积忽然膨胀又猛地收缩,如此变化数遍之后,彻底爆炸。

%11
星云破灭,碎屑的蓝光中,仙蛊屋黑牢宛若一只蛮牛冲出。

%12
银色巨人便用藏青厚盾,往前硬顶。

%13
黑牢重重地撞在厚盾上,僵持了几个呼吸之后,厚盾和黑牢相撞的部位附近,出现大面积的裂纹,旋即大盾轰然破碎。

%14
但经过这两重防御,黑牢的攻势已经不足先前的三成。

%15
最终黑牢仙蛊屋撞在戊土纱之上,软软绵绵,虚不着力,像是一支冲到射程尽头的飞箭,抵在银色巨人的胸膛,却只给后者带来微不足道的损伤。

%16
黑牢尽管是仙蛊屋,但是催动它的蛊仙,却只有黑城一人而已。

%17
从这点来看,上古战阵天婆梭罗,和仙蛊屋还是十分相似的。

%18
催动这两者的蛊仙越多,它们爆发出来的力量就越强。

%19
但两者之间,又有区别。

%20
仙蛊屋对蛊仙的流派、修为,没有硬性规定。而战阵则有具体的要求。

%21
比如天婆梭罗战阵,就要求里面的蛊仙,最好修行同一种流派。如此银色巨人的威力将会更强,操纵起来也会更灵活敏捷。

%22
像此时战斗,方源表现出星道力量,墨坦桑是气道,琅琊地灵是特殊存在,其余的毛民蛊仙倒都是炼道蛊仙。

%23
这种情况,天婆梭罗的威力还未达到极限。幸亏有这么多的毛民做了基础,否则流派太过于杂乱的话,银色巨人根本就无法成形。

%24
这还只是天婆梭罗的组阵要求。

%25
天婆梭罗乃是上古战阵中,排名第二的强大战阵,要求不是特别严格。

%26
根据历史记载,有很多其他的战阵,要求不是刁钻,就是古怪。通常而言,很难满足要求,组成战阵。

%27
战阵的这个弱点,也正是它渐渐消散于历史中的原因之一。

%28
同样是阵道流派的至高成就,仙蛊屋就比战阵方便多了,这是时代进步的象征。

%29
历史滚滚,从古到今,每一个时代都涌现出繁星多的英雄枭雄,鬼才怪杰。这些人的智慧一代代的积累叠加起来,就形成了当今百花齐放百家争鸣的蛊修时代。

%30
虽然远古、上古时代的很多天材地宝,猛兽植株都绝灭了,但是在今天的这个时代,总体来讲,是要远远优于古代,并且是在不断地向前进步的。

%31
材料蛊虫只是基础,蛊修的智慧才是关键。

%32
轰!轰!轰!

%33
秦百胜几乎将全部的注意力,都集中在苍穹之上。

%34
他目光灼热,高举双手,一次次向空中捏取。

%35
伴随着他的这个动作,天空中响起一声声雷霆般的炸响,一块块的空间宛若镜子般破碎,一只只的蛊虫被显露出来。然后在一股无形而又玄妙的力量下,这些蛊虫被拿取下来,一只只尽数落入秦百胜的手中。

%36
秦百胜气势非凡,摘取这些蛊虫,仿佛手到擒来。

%37
此刻的他气势浩瀚,宛若天神战仙,正在摘星拿月,带给人一种无法匹敌,高山仰止的感觉。

%38
和他一同收取蛊虫的,还有其他两处地方的神秘黑袍蛊仙、姜钰仙子二人。

%39
“他们正在收取我的仙蛊屋炼炉,不能让他们得逞。否则整个琅琊福地就会大面积坍塌!”战阵中央,琅琊地灵焦急怒吼。

%40
但没有用。

%41
虽然银色巨人挡住了仙蛊屋黑牢的冲撞,但却无法阻止它的干扰。

%42
好几次,银色巨人在地灵的操纵下,想要接近秦百胜,都被黑城拼命拦下。

%43
轰!

%44
这时,陡然一声巨响,天空中电闪雷鸣,首次在破碎的空间中,显现出一只仙蛊。

%45
“好!”秦百胜见此,忍不住叫了一声好,满脸喜色。

%46
他伸手去捉,按道理来讲,这种仙蛊是琅琊地灵之物,但此刻却毫无反抗之力,被秦百胜“轻而易举”地捕捉到手,收入囊中。

%47
很显然,秦百胜一定掌握着什么特殊的手段,可以夺取他人的仙蛊。

%48
这仙蛊一失,顿时引起了广泛的影响。

%49
琅琊福地一阵天摇地晃,壮阔的海面上掀起巨大的海啸,冲击海中的三片大陆。

%50
在汪洋的远处,世界的边角,忽然间发生了剧烈的坍塌。空间大面积缩小,使得原来这片空间中的海水、生物,都涌到附近的海域中去了。

%51
“秦百胜收取了一只仙蛊,居然引发了仙窍福地这么大的反应?”方源目睹此景,心中掀起波澜,顿时有了一些隐隐猜测。

%52
“住手,快给我住手!!”琅琊地灵急得连连叫喊。

%53
其余的毛民蛊仙也是十分担忧,取走了仙蛊,对琅琊福地影响极大,已经严重威胁到三大陆中亿万毛民的生命安全!

%54
但是黑牢挡住银色巨人,秦百胜又在周围布下防护手段,并不那么容易干扰。

%55
一只只的仙蛊被秦百胜取走,琅琊福地四处坍塌,造成海啸、飓风、洪涝等等天灾,三大陆上毛民损失了至少数百万。

%56
这是典型的神仙打架,凡人遭殃了。

%57
方源见此,更加确信刚刚他心中的猜想。

%58
琅琊福地的真正空间,已经远超常规的福地,和优秀洞天有的一拼。但明明琅琊福地只是福地而已。

%59
那琅琊地灵,究竟是如何做到的呢?

%60
方源早就知道,仙窍的环境很大程度上,是因为仙窍中蕴藏的道痕。

%61
于是他很自然就想到了:“琅琊地灵似乎有一种独特的手段,能够将仙蛊融入这片仙窍的天地之中。这样一来,仙蛊中蕴藏的道痕碎片,就成了仙窍所有。如此一来,就大大开拓了仙窍。”

%62
“但现在秦百胜强行取走八转仙蛊屋炼炉,里面的仙蛊失去的话,琅琊福地也就失去了大量的道痕,这才导致了环境的剧变,内部空间的坍塌。”

%63
方源猜的没有错。

%64
正是因为仙蛊屋炼炉,被镶嵌在这片天地里面,才使得琅琊福地借助仙蛊中的丰富道痕,开拓出广阔空间,产生无数资源,底蕴超级雄厚。

%65
上一任的琅琊地灵手笔颇大。

%66
炼炉融入仙窍天地,这样一来,就等若就整个仙窍都充当了炼炉内部,将仙窍中所有的生灵,都纳入到炼炉仙蛊屋的炼制范围内。

%67
正是有了炼炉的帮助,才使得上一任琅琊地灵,培育了大量的毛民蛊仙。

%68
但这样做的话,也有相当大的弊端。

%69
首先就是仙蛊屋炼炉,不能轻动,等若陷在仙窍福地中了。

%70
一旦动弹,仙窍福地的内部环境,必定发生翻天覆地的缩减,酿造成无数惨重的损失。

%71
这也就是,琅琊地灵宁愿动用上古战阵,也不早早祭出仙蛊屋炼炉参战的原因。

%72
“给我滚开啊!”银色巨人掀起巨掌,将黑牢仙蛊屋远远拍飞。

%73
随后巨人迈开大步,冲向秦百胜。

%74
但下一刻,黑牢仙蛊屋又飞了回来,狠狠地撞击在银色巨人的右肩膀上,将它撞歪过去。

%75
这种情形,已经发生了十数次。

%76
黑城上路贼船,就没办法轻易下船,此刻也是拼了!

%77
哪怕黑牢仙蛊屋,在激斗中损伤很多,变得残破不堪,黑城也咬牙坚持。

%78
“没有用的!我还能坚持很久!仙蛊屋是由少数仙蛊,无数凡蛊堆砌而成,只要仙蛊不损,这些凡蛊消耗多少,我就能补充多少!仙蛊屋若是如此轻易就破损,那还叫仙蛊屋吗?”仙蛊屋中传出黑城大吼。

%79
“你说的不错!”琅琊地灵开口,却罕见地赞同了黑城的话。

%80
黑城一愣,旋即脸色剧变。

%81
他发现自己的黑牢仙蛊屋中,许多蛊虫已经被琅琊地灵强行炼化。其中甚至包含了两只仙蛊。

%82
黑城说的没错仙蛊屋若是如此轻易就破损,那还叫仙蛊屋吗?

%83
因此,八转仙蛊屋炼炉虽然被收了不少蛊虫,甚至还损失了仙蛊,但只要核心还在,威能只是不断减弱而已,强行炼化他人蛊虫的威能还是一直作用着的。

%84
“哈哈哈,半柱香时间已经到了。你们都输了,一败涂地!我的炼炉中还留着两大毛民蛊仙呢,由他们坐镇,你怎么抢占我的仙蛊屋?”琅琊地灵咆哮,发出胜利的宣言。

%85
琅琊福地的底蕴真是雄厚,打到现在这种程度,居然还有毛民蛊仙没有露面。

%86
不过想想也合理,炼炉这等重地中不可能没有留守的力量。

%87
但这时,秦百胜却丝丝冷笑:“你确定是这样么?”

%88
话音还未落下,天空中降下两个毛民蛊仙,对秦百胜施礼:“拜见副使大人。”

%89
这两个特异留守炼炉的毛民蛊仙,竟然是影宗在琅琊福地里的卧底!

%90
有他们配合,里应外策,八转仙蛊屋炼炉的丢失已成定局!

\end{this_body}

