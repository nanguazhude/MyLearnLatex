\newsection{隐秘惊人弃净魂}    %第一百六十二节:隐秘惊人弃净魂

\begin{this_body}

%1
方源面容转肃,继续搜魂,不管东方长凡再如何劝说。

%2
这一次,历时最长,方源一鼓作气,将东方长凡掌握的秘密,彻底搜刮干净。

%3
他知道了东方长凡成长的一生,无数的场面、经历。

%4
“蛊仙韩东已经死了,被虚兽拖进尸山之后,本是俘虏,结果在天劫地灾中丧命。”

%5
“原来东方长凡在碧潭福地中,早就做了布置,是以六转仙蛊屋茅草屋为核心。但关键时刻,茅草屋却神秘地失踪了!这才让我等魔道长驱直入,入侵了碧潭福地。”

%6
“果然……在我捏爆东方余亮的肉身,抽取出东方长凡魂魄的那一刻,就将体内的全部仙蛊引爆销毁了!”

%7
“白莲巨蚕蛊!想不到太白云生改造仙魂,就用了大量的白莲巨蚕蛊……他的仙窍中,还藏着用剩下的一小批白莲巨蚕蛊。”

%8
“东方长凡,纵观一生,不甘于平凡,先是英雄,后是枭雄。”

%9
“咦?这是紫山真君!太白云生的宙道传承,就是他给的。没想到东方长凡的智道传承,也是来源这个神秘老者!!”

%10
接憧而至的秘密,让方源应接不暇。

%11
好半天,方源这才渐渐反应过来,全盘接受了这大股信息的冲击。

%12
紫山真君,只是方源给这个神秘的紫发乞丐老者的命名。这个神秘老者似乎身怀无数珍贵传承。来头很大,目的不明,可以说东方长凡、太白云生能有今天。他起了举足轻重的作用。

%13
姑且就先用紫山真君来称呼他罢。

%14
在方源的前世记忆中,根本就没有这个紫山真君的影子,就算是五域大战,他都没有出现,是隐藏了,还是有其他的身份,或者老死了?依照东方长凡、太白云生的情况推测。这位老者的年岁必然很大了。

%15
“紫山真君先不必管,我的净魂仙蛊早就需要食物。这份白莲巨蚕蛊数量不多,但正好可解我的燃眉之急。本来我暂时还不想去管那处福地,现在看来却有些势在必行了。”

%16
方源捏爆东方余亮的身躯,导致仙窍落地。形成无主福地。

%17
公共福地也就罢了,但这种单独的福地,基本上都会生出地灵来,可谓易守难攻。

%18
东方长凡败亡之际,为了防止方源讨便宜,就将掌握的仙蛊都重点销毁掉了,以防资敌。但他的仙窍内还有不少凡蛊,或许是价值低容易获得,或许是他自己不重视的。或许还有失去控制的原因罢。总之是留在了福地中,保留了下来,没有被毁。

%19
其他的凡蛊也就算了。但白莲巨蚕蛊,却是方源一直孜孜不倦追求的食料。

%20
方源一边踱步,一边思考。

%21
北原蛊仙界,原本波澜不惊,但自从王庭福地毁灭,就产生了剧烈的动荡。宛若掀起滔天波澜。北原蛊仙们惊怒交加,要寻找罪魁祸首。但八十八角真阳楼中的仙蛊逃出。使得北原蛊仙激烈争抢,令北原蛊仙界一片混乱。

%22
拍卖大会之后,以八转蛊仙为首,各大势力商议,北原蛊仙界不再那么混乱,有了新的规矩和格局。但因为影宗、中洲蛊仙一行人的缘故,仍旧是暗流汹涌。

%23
如今被方源这一闹,超级势力东方部族毁灭,北原蛊仙界还没有安稳多久,又再次剧烈激荡起来。

%24
当下,正魔两道无数蛊仙,都在围绕着碧潭福地进行争战夺抢,一片混乱。

%25
北原的局势,可谓一团糟!

%26
事实上,自从王庭福地毁灭之后,北原就没有消停过,已经彻底脱离了原先的轨迹。

%27
有着前世记忆的对比,这种情况让方源都感觉:北原似乎已经被他玩坏了。

%28
方源这次冒险,又有运气成分地夺下了智道传承,原本是想安安稳稳地发展一段时间,帮助狐仙福地渡劫。利用智道传承,参悟出见面似相识这类仙道杀招。

%29
外面风声太紧,还是不要露头的好。

%30
但白莲巨蚕蛊的诱惑力太惊人了,方源知道,虽然有些手段保住维系着,但净魂仙蛊真的已经饿到了极限,必须要有白莲巨蚕蛊喂养。

%31
“不过仙蛊唯一,虽然珍贵,净魂却不是必不可少之物了。我有我力仙蛊,充当核心,净魂就是毁灭,也不会令我伤筋动骨。当然,它若能再用,和我力一起,必能令万我杀招威力上涨一倍左右。而且它还能助长我的魂魄底蕴。”

%32
方源陷入犹豫。

%33
净魂虽好,但也不足以让他冒着性命危险去救。

%34
仙蛊只是外物,自己的性命才最宝贵。这一点,方源始终拎得清,从未被贪欲蒙蔽过理智。

%35
几天后。

%36
北原,蛊仙郄世民陨落地点附近。

%37
时不时的,便有蛊仙出现,或魔或正。在这里停留片刻之后,有的便迅速离开,追寻空气中残留的仙蛊气息而去。有的则试探郄世民遗留下的仙窍,所化成的福地,失败后垂头丧气而走。

%38
来自中洲的蛊仙们,远远隐藏,一边眺望着这里的情况,一边用神念在隐晦交流。

%39
“目标还没有出现吗?”

%40
“对方既然能杀得东方长凡,又在这里做了布置,遮掩了东方长凡留下的福地。就是想不被人发现。很可能就是想要攻略此处福地,他也知道夜长梦多,应该会回来的。”

%41
“已经不太可能了。我们已经暗暗守在这里这么多天,也许对方早就察觉了我们的行踪。趁此良机,还不如前往碧潭福地。”凌梅仙子道。

%42
“哼,阁下是觉得我黑天寺的招牌仙道杀招知白守黑,无法遮掩尔等,会被人轻易察觉吗?”天聋老人立即不悦地冷哼道。

%43
凌梅仙子沉默,她知道刚才的话,有些冒失了。黑天寺同样是十大古派之一,名声需要时刻维护,不容轻易冒犯。

%44
凌梅仙子身旁的傲雪仙子道:“听说碧潭福地中,竟然有一头万里芝马跑了出来。”

%45
话语含蓄,潜在的意思却很明显,分明是在赞同凌梅仙子之前的提议,也向往碧潭福地中的丰富财货。

%46
“哼,区区这些财货,就能引得诸位如此心动?别忘了我们这是来干什么的。这位仙僵十分可疑,力道战力相当不俗,很可能便是我们要找的仙僵沙黄!”残阳老君隐于一旁,插言道。

%47
“区区财货?说得云淡风轻,东方一族就算不如我等门派,但毕竟也是超级势力。底蕴绝对比你我私人,要深厚得多。哦,你残阳此次收获不菲,仙鹤门的任务奖励,更是丰厚。你一人吃饱了也就罢了,还要我等这些人,仍旧饿着肚子,这未免过分了吧?”陈振翅饱含怨气。

%48
他这次的门派任务,还没有完成,就出了这档子事情,被凤九歌召唤过来。

%49
和他相同情况的,还有许多人。

%50
别人畏惧残阳老君的追命火,他却独独不惧。他是飞行宗师,有着仙道杀招傍身,就算是中了追命火,也能甩脱了去。

%51
在中洲蛊仙界,更是盛传着这样一句话:“振翅九重天,往来一飞烟”。“振翅九重天”就是说的陈振翅,振翅高飞,可以飞进太古九天,天地都不能约束他。

%52
后者“往来一飞烟”是指天莲派的蛊仙步飞烟,意喻此位女仙往来纵横,灵巧翩跹,宛若一缕灵动飞舞的烟云雾氲。

%53
这一次,女蛊仙步飞烟也来到北原,代表十大古派之一的天莲派,参加对八十八角真阳楼倒塌一案的真相调查。

%54
此时,中洲当代两大飞行宗师就齐聚于此。

%55
陈振翅开口,步飞烟却一直保持沉默,没有说话。

%56
倒是老算子帮腔道:“碧潭福地被这样轻易攻破,主要原因就是六转仙蛊屋凡草屋,失踪得十分蹊跷。不是我眼红那些资源,而是刚刚我暗中推算,算出这碧潭福地里面留有真凶的线索。”

%57
“凡草屋……嘿,我更愿意叫它的原名茅草屋!此屋说起来,还是近古时代的三茅魔仙所创,几乎耗费了他一生财力。他晚年期间凭此纵横北原,成为一霸,风光一时。三茅魔仙又孕育一女,十分溺爱。那时,东方一族有蛊仙东方玉,和此女相爱,脱离正道,入赘魔道。三茅魔仙死后,他便继承了这座凡草屋,杀了他的妻子,弃暗投明,回归家族。为了摆脱昔日阴影,东方玉又向茅草屋中掺和其他蛊虫,取名为凡草屋。其实功效没有提升多少,纯粹只是为了面子上好看一点。”风云府的蛊仙洪赤明道。

%58
他在未成仙前,以凡人蛊师的修为,深入北原,有过一段生活经历。对这些消息,掌握得很多。

%59
此时故意提起,其实心意上也是想去碧潭福地搜刮。但他没有明说,表达得也是十分含蓄。

%60
“好了,我们到底是继续这样守候下去呢?还是去碧潭福地,找寻真凶线索?或者找那个黎山仙子,她既然和那个力道仙僵一起作战,关系紧密,一定知道很多秘密。”有蛊仙不耐地道。

%61
“不妥。黎山仙子付出重伤代价,抢夺到方寸山后,立即撤离这里。如今,已然龟缩在大雪山福地中。大雪山福地乃是北原魔道老巢,换做以往也就罢了,现在雪胡老祖一个劲地要炼制鸿运齐天仙蛊,一直坐镇在那里。”

\end{this_body}

