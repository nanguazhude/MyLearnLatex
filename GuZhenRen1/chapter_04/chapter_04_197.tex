\newsection{招揽方正}    %第一百九十七节:招揽方正

\begin{this_body}

%1
中洲,狐仙福地。

%2
荡魂行宫中,方源展现原形,八臂僵躯,盘坐在床榻上,双目闭合。

%3
而在他的仙窍中,正发生着一场意志的交战。

%4
“方源,你不得好死!”墨瑶意志愤怒大叫,但下一刻方源星意就恶狠狠地扑上去。

%5
两股意志立刻混淆一团,陷入凶险的意志交战。

%6
很快,方源的星意就以肉眼可见的速度,消散下去。而墨瑶的假意虽然也有损失,但比方源损失的要远远小的多。

%7
方源却不以为意,只是观战。不像第一次那样,一连用许多智道杀招,去辅助自己的星意作战。

%8
等到参战的方源星意已经快要完全溃败的时候,方源微微一笑,又发出一股星意,撞进意志的战团里去。

%9
不一会儿,方源的星意再度损耗严重,方源便再次补充第三股。

%10
墨瑶意志一方是无源之水,方源却是有本之木,就这样连续几股星意投放下去,终于将墨瑶假意磨损得奄奄一息。

%11
方源从战团中抽回星意,同时一连催动数种杀招,将墨瑶假意再度禁锢,同时又滋养她的假意,让她缓慢恢复状态,方便下一次方源的搜刮。

%12
墨瑶假意被方源玩弄得散漫一团,连咒骂的力气都没有,更遑论维持基本的形体了。

%13
这股墨瑶假意的最大价值,就是红莲魔尊的传承线索,已经被方源压榨出来了。剩下的就是一些炼蛊相关的记忆。

%14
参加中洲炼蛊大会这段时间,方源不忘从墨瑶假意中搜刮这些炼蛊记忆,同时还有意锻炼自己意志大战的手段。

%15
起初时。不再动用智道手段的情况下,方源星意节节败退,根本不是墨瑶假意的对手。

%16
墨瑶不愧是灵缘斋的某代仙子,身为墨人女子,却得到剑仙薄青的青睐。成为其唯一的伴侣。即便历数灵缘斋的所有仙子,墨瑶必定是名列前茅的。

%17
这个老妖婆,即便主修的炼道,对于智道方面的造诣,也远远超过方源。

%18
方源为了磨砺自己,故意和墨瑶假意进行纯粹的意志大战。刚开始。战损比几乎达到惊人的一百比一。方源这边损失一百,墨瑶假意才损失一。

%19
不过后来,随着战斗次数增多,方源经验迅速累积,更关键的是还有东方长凡智道传承的内容指导。使得方源在意志战斗这块,迅速提升。

%20
从原本几乎空白一片,到现在和墨瑶假意战斗得有声有色,进步极其巨大。

%21
原本的一百比一的战损比,如今也缩减到了四比一。

%22
也就是说,四股方源星意,便能战胜墨瑶假意。

%23
对于方源窃取自身修行经验,还把自己当做免费打手。任意错边揉捏,墨瑶假意当然极为愤怒,但受制于人。就算再怒也毫无办法。

%24
星意进入脑海,方源仔细查阅了星意中抢夺过来的墨瑶记忆。

%25
随后,他缓缓睁开双眼,嘴角流露出一丝笑意:“有点意思。”

%26
这次的记忆,却不是以往的炼蛊经验了。而是墨瑶生前和剑仙薄青一起探险,在中洲的真阳山脉中。斩杀了一群荒兽,搜索战利品时。发现山洞中的一株芙蓉石钟乳。

%27
芙蓉石钟乳,乃是罕见的七转仙材。石钟乳每年滴出一滴石心液。这石心液更加珍贵,乃是八转仙材!

%28
薄青当时要剑斩石钟乳,却被墨瑶阻止。

%29
斩断石钟乳,只会得到一份七转仙材。但若放任下去,多年之后,就会有不少的石心液。八转仙材的价值,可比七转仙材要高得多。

%30
薄青便采纳了墨瑶的建议,做下许多精妙蛊阵布置,将这处山洞完美掩藏。

%31
“这么多年过去了,若是这处山洞没有被发现的话,那么积累的八转仙材石心液就极其可观。我即便用不了,收藏起来也是好的。将来可以放到宝黄天中,以物易物,换取想要的炼蛊仙材!八转仙材,蛊仙们都会十分渴求的。”

%32
方源暗暗欢喜的同时,对墨瑶假意的认知,也加深一层。

%33
“这墨瑶假意,手段不少。这种珍贵的记忆,都被她故意隐藏起来。直至现在,大部分的记忆都被我搜刮获知,才被我发现。这种手段,就算是东方长凡的智道传承中都没有记载啊。我得敲过来!”

%34
方源暗暗叮嘱自己。

%35
不过当下,却不能继续对墨瑶施暴了。

%36
需要徐徐恢复她的假意,否则她承担不住。

%37
方源并未直接前往真阳山脉,去取石心液。而是起身,打开牢门,来到亲弟弟方正的面前。

%38
方正的状态十分古怪。

%39
他禁闭双眼,躺在地上,脸色时而狰狞时而松弛,有时候会大叫,甚至手舞足蹈。简直像中了梦魇一般。

%40
这当然是方源施加在他身上的手段。

%41
方源信手一挥,将这手段消除。

%42
方正大汗淋漓,呼呼地喘着粗气,好一会儿才缓缓坐起身:“你对我做了什么?”

%43
他望着方源喊道,目光充满了仇恨。

%44
方源呵呵一笑,淡淡地道:“我的好弟弟,我给你看的,都是真实发生的事情。是我记忆中的影响,结合天鹤上人的记忆影响。你现在应该知道,青茅山之所以发生如此惨祸,根本的源头并不在我,而在于你所谓的师傅天鹤上人啊。若是他不来这里,强取血道真传,说不定现在我们兄弟俩还在青茅山幸福快乐地生活吧。”

%45
方正呆傻,一时间竟忘了说话,或者说他不知道怎么答话!

%46
方源的确说的是事实,族人虽然是方源杀的,但为了生存迫不得已。若不是天鹤上人过来抢夺血海真传,青茅山的三座山寨也不会覆灭。

%47
当然这个事情的起因,天鹤上人也没有隐瞒方正。只是美化过程是必须的。

%48
方正也不是没有思考过这个问题,但每次思维触及到这个上面时,他都会下意识地选择回避。

%49
但现在方源将天鹤上人搜魂,将他的记忆影像直接放到方正的脑海里重放。

%50
这简直是将血淋淋的残酷现实,拎到方正的眼前。就算他想回避。也回避不了!

%51
“呵呵呵。”空气中回荡着方源的冷笑,“我的好弟弟,你难道就没有想过,你的师父天鹤上人对于青茅山覆灭,有着不可推卸的责任?你还让他做你的师父,你这是认贼作父啊。”

%52
方源的话。无疑是字字诛心,直接刺到方正的内心最深处。

%53
“你住嘴!”方正陡然大吼,从未有这么一刻,他如此的愤怒!

%54
“你住嘴,不要说了。你这个恶魔。明明是你杀了全族的人,是你杀害了舅父舅母,是你,是你,你这个刽子手!没有师父的搭救,我早就死了。不准你诋毁我的师父……”

%55
“哦?”方源嘴角翘起嘲讽的弧度,轻笑出声,“天鹤上人之所以救你。还不是要利用你,把我搜寻出来?呵呵呵,不久前仙鹤门还利用你。进攻我的狐仙福地。那一场你差点死了吧?我愚蠢的弟弟啊,他们这样利用你,你都心甘情愿吗?”

%56
“不要说了,闭嘴,闭嘴!”方正捂住耳朵,紧闭双眼。脸上愤怒又恐慌。

%57
方源脸色转冷,声音冰寒:“我懦弱的弟弟。你连真正的事实都不敢去面对。哪怕你现在有五转蛊师的力量,实则在内心深处仍旧不过是个没长大的孩子罢了。不过你我到底是亲身血缘。作为哥哥的我,再给你一次机会。来我的身边吧,我会锻炼你,让你真正的坚强起来,成熟起来。”

%58
“不……不!”方正睁开双眼,他的眼睛充满血丝,仇恨愤怒地看向方源,“你这个凶手!居然还想让我给你卖命?你死了这条心吧!就算,就算仙鹤门利用我,我也不仇恨他们。因为他们也培养了我。而你!是你亲手杀死了舅父舅母,杀死了全寨的族人。我恨你,一辈子都不会原谅你的!只要给我一次机会,我绝对会亲手杀了你,报仇雪恨!!”

%59
方源双眼渐渐眯起,静静地看着方正。

%60
他不怒反笑:“呵呵呵,看来你在仙鹤门的确学到了不少东西呢。你这么想要杀我,真是让我有些期待啊。”

%61
说完这话,方源缓缓转身,步出牢房。

%62
这些天来,方源将主要精力,耗费在炼蛊大会上。但也没有忘了利用智慧蛊、星念蛊进行推算。

%63
推算见面似相识杀招,暂时已经耗尽了方源的潜力。方源便将推算万我杀招,目前已经成功地将挽澜仙蛊,也融合进去。如此一来,万我杀招就有三大力道核心了。

%64
之后,方源又将气囊蛊的蛊方改良了一番,流水线炼蛊的步骤增多,更加细分,危险性也减少了三成。

%65
改良忆念蛊、星念蛊、恶念蛊蛊方时,却遇到了瓶颈,方源的智道修行才刚刚起步,智道境界惨不忍睹,大大拖累了方源推算这些蛊方的脚步。

%66
最近这些天,方源又将目光放到仙蛊方血神子上。

%67
他渐渐修复血神子,发现其中一些奥秘。就算血神子仙蛊炼制出来,想要催动它,还得需要血缘亲人。杀得一位亲人,在血神子仙蛊的力量下,就能得到一位血神子。

%68
若是这位亲人不反抗,甘心奉献,那产生的血神子便会亲近蛊仙,如臂使指。反之,若引来亲人的仇恨愤怒,血神子便有可能反噬自己的主人。

%69
这便是方源招揽方正的原因了。

%70
如果方源有其他法门,脱离了仙僵,就将方正利用在这方面。总之,既然方正曾经给方源惹来不少麻烦,方源肯定是不会放过他的。

\end{this_body}

