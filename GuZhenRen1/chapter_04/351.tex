\newsection{事生差异超预料}    %第三百五十二节:事生差异超预料

\begin{this_body}

他的师尊,七转散仙月下老人,盘坐在蒲团上,听到陆钻风的声音,却是一动不动,仍旧双眼闭合着。

良久,山洞外的萧家老祖再次开口,声音传入洞内:“月下老哥,我愿再出十枚天龙珠,请您出手相助这一次。”

月下老人这才缓缓睁开双眼,对陆钻风道:“徒儿,你就先去义天山,替萧山解了这场危局罢。”

陆钻风连忙跪下,向月下老人恭敬地磕了三个响头。

“师尊,徒儿去了。”

虽然口中这么说,但陆钻风却仍旧跪在地上,没有动弹起身的迹象。

月下老人一笑:“就你这贼性子,小心思。”

说着,向陆钻风吹了口凉气。

凉气沁入陆钻风的腹中,始终盘桓不散。

“这只仙蛊,能保性命。你去吧。”

陆钻风大喜,又磕了三个响头,起身后,嬉皮笑脸地道:“徒儿其实也为师尊你着想啊。徒儿丢了这条小命不要紧,可今后师尊身边没有人伺候,可怎么办呢?师尊对徒儿有大恩,徒儿还没有还,怎么可以轻死呢?”

“速去,速去。”月下老人叹气,对陆钻风连连挥手。

武神通携带兽群强攻义天山,义天寨刚刚草建,便遭毁灭。

魔道一方士气大沮,萧山决意奋死反击。

但就在当晚,五转魔道蛊师陆钻风,潜入正道营地,偷偷斩杀武神通,携带他的首级,夜上义天山。

萧山得之,大喜。对陆钻风道:“陆兄高义,这一次得赖陆兄,救了我们所有人的性命了。”

陆钻风也是乖觉之人。连忙拱手,诚恳地道:“萧大哥谬赞了。在下不过是捡了个便宜。若非萧大哥及诸位同道奋战,将武神通重伤,我又怎么会轻易就取了他的性命呢?若说这份功劳有十分,我不过只占一分,其余九分当归诸位。”

众人听闻此言,顿时觉得陆钻风更加顺眼了许多。

萧山心中一沉,没有捧杀得了陆钻风,连忙换了另一番言语。稳住陆钻风,将其拉拢进义天寨。

就这样,正道的第一波攻势,在萧山以及陆钻风等人的出击之下,土崩瓦解。

方源隐在一旁,打了一通酱油,至始至终冷眼旁观,并未惹来怀疑。

他想起上一世,自己搜集到的情报。

“看来今生,武神通重伤未死。萧山情势危急,他背后的萧家老祖同样去请了月下老人出手。”

“这个月下老人,乃是南疆著名的散仙。和武家有间隙。萧家老祖又出重资,使得他临时添加赌注,将陆钻风提前推进义天山。而武家女仙武当芷遭受算计,选中的棋子丢了性命,虽然也转化了不少战意,但受到的损失却是最大的。”

“义天山正魔两方交锋,其实背后的主导,却是南疆蛊仙们的较量。凡人如蚁啊……”

“就让他们相互算计好了,此战之后。我转化的战意,已经超越了萧山。成为第一人了!”

如此,方源闷声发大财。暗中转化的战意越来越多,却没有人注意到。

时间流逝,第二次正魔交锋爆发。

原来,第一次交锋,武当芷吃了一亏,她并不甘心。

月下老人临时加注,她又有何不能呢?

于是,她再选中一人,又联合商家蛊仙,不再重蹈之前独自作战的错误。

落在义天山的战场上,武家再次派遣蛊师讨伐义天山,同时此行之中,商家的两位新晋家老:炎突和巨开碑,也加入了其中。

双方在义天山脚下叫阵,互有胜负,僵持不下。

这时,蓝眉鹤、飞鼬王一齐上山,加入义天寨。

这两位蛊师可不简单,乃是南疆蛊师界中,闻名遐迩的飞行大师!

义天寨因此占据上风,将正道诸人逼退。

为了对抗蓝眉鹤、飞鼬王,正道请来女蛊师红飞鱼,她同样是飞行大师,却隶属正道势力。

但红飞鱼一人之力,难敌两位魔道飞行大师的联手。

红飞鱼生死一刻之际,商家的援军赶到,便是白光刀客魏央。

魏央经过商燕飞的提携,资质提升,修为达到四转初阶境地。这一战,他不仅救下了红飞鱼,而且成就了他的威名。

红飞鱼重伤退下之后,他以一敌二,大战蓝眉鹤、飞鼬王,最终成功拖延时间,撑到正道援兵来支援,蓝眉鹤、飞鼬王无奈撤退。

此战之后,魏央得到正魔公认,跻身进飞行大师的行列。南疆三大飞行,成为以魏央为首的四大飞行大师。

魏央的加入,使得正魔大战重新陷入僵持状态。

不过,随着二代僵王走上义天山,奴道再逞威能,正道中缺乏武神通这样的奴道高手,不得不撤退。

于是,第二次正魔交锋结束。

这背后,南疆蛊仙们阴谋算计,合纵连横,自然是凡人蛊师们想象不到的。

义天山上唯一的明白人方源,也是闷声发财,视若无睹。

他以黄沙的身份,在战场上,还和魏央照过面。

这位商燕飞的干将,曾经在商家城时,对方源和白凝冰多多照顾。此时此刻,魏央进步很大,方源却已然成仙,可谓物是人非。

此战之后,方源转化的战意,已经遥遥领先。规模之庞大,就算南疆蛊仙们掌握的战意都相加起来,比之方源也是稍差一筹。

中洲,天莲派。

轰隆隆……

爆炸声、轰鸣声,不绝于耳。

方圆十多万里,都化为一片荒地,飞沙漫天。

荒地中又分布着深坑,燃烧着火焰,堆积着一截截的冰川,散布着闪电霹雳的痕印。还有血水化成道道长河,在这处战场上恣意流淌。

因为八转大战,原本山峦叠嶂的地貌。彻底面目全非。

就这,还是天莲派驻地选址之处。乃是名山大川,拥有深刻的天然道痕。否则的话,狼藉不堪的战场景象,还会更加严重。

一轮交锋停歇下来,以剑仙薄青为首的影宗众仙,后撤一段距离,呼呼喘着粗气。

凝望眼前,三座仙蛊屋悬浮在高空之中。

一座揽雀阁。小巧精致。一座岳阳宫,锦绣辉煌。

但影宗众仙的目光,都集中在最中央的那座仙蛊屋上。

这座仙蛊屋,乃是八转级数,号为天池!

从外表看去,这就是一片池塘,面积一点都不大,甚至可以说是世间最袖珍的仙蛊屋了!

但就在这座天池当中,青荷曼妙,朵朵莲花或盛开。或闭合。

就在这每一朵莲花当中,就有一个仙窍福地,或者仙窍洞天!

八转仙蛊屋天池。乃是天莲派的镇派之基,由元莲仙尊所创,它的最大作用就是存放仙窍!

“不把天池攻打下来,天莲派此番损失,连伤筋动骨都算不上。”宋紫星沉声道。

“但是它太难攻打了。十大尊者当中,元莲仙尊最擅治疗和回复。这座天池,更是秉承了他的风格,若不是一击而溃,短时间内再重的创伤。它都能回复过来。”余木蠢脸色凝重。

七星子仙僵摇摇头:“不只是天池,其余两座仙蛊屋也是难堪得很。而且薄青他最大的弱点。已经被对方掌握了。”

仙僵薄青的最大弱点,就是体内的墨瑶残魂。

这个残魂就像是木桶一圈。最短的那个短板,太容易被针对了。

前世的时候,薄青就因此身亡,里面的魂魄被天庭蛊仙打散。今生,仙僵薄青更是丧失了慧剑仙蛊。此蛊乃是薄青生前,防备智道手段的王牌。

失去之后,更让仙僵薄青战斗起来,束手束脚,十分被动。

“走吧!拖延的目的已经达到了。”七星子仙僵首先撤退,身化长虹,飞速撤离。

薄青等人,紧随其后。

“逃脱宿命,又强攻十大古派,你们如此胆大包天,罪无可恕!还想走?”天莲派的三座仙蛊屋,一齐追杀出来。

从天池中,更是传出监天塔主的声音,充满了恼怒和仇恨。

七星子微微色变,对其余蛊仙道:“眼下情形,只有让一人舍命断后,为其他人争取逃脱的时间。”

“我去。”仙僵薄青立即开口。

“还是我去吧。青、蓝二位,还有宋紫星,你们的战力都比我高。对接下来的大计,会更有帮助。”余木蠢一脸淡然,主动留下。

三仙目光闪烁,没有丝毫犹豫,飞速撤离,独独留下余木蠢殿后。

“螳臂当车,不自量力!”

“呵呵,魔道中人就是擅长背信弃义。”

“你以为你能挡住我们,谁给你如此狂妄的自信?”

正道蛊仙们怒喝不止,三座仙蛊屋碾压过来。

余木蠢的脸上,展现出视死如归的微笑:“你以为我们就没有布置了吗?”

说着,他落到了地面,一座规模巨大的蛊阵,闪烁出冲天的华光。

关于此战的消息,中洲根本无法遮掩。情报被方源获悉之后,他不禁目光闪烁,此事已经和前世差别太多,仙僵薄青居然没有死!

如此一来,盗取他身上仙蛊的方源,必然是他的眼中钉了。

方源将心神投入被封印的仙窍,在那里,躺着一只神秘仙蛊。

这只仙蛊,就是方源从仙僵薄青身上盗取的收获之一。但事实上,这只仙蛊的原来主人,却非薄青,而是墨瑶。

正是因为这只仙蛊本身,充斥墨瑶的意志,方源才能利用墨瑶假意,这么快地将这只仙蛊炼化。

至于其他剑道仙蛊,真正的主人是薄青,里面有大量的薄青意志。所以方源用相同的蛊阵,要将这些仙蛊炼化,耗费的时间要多出很多。

“这些剑道仙蛊是别想了,肯定赶不上这场大战。至于这只神秘仙蛊,这些天我也尽量调查了,可惜就算是琅琊地灵那里,我也没有丝毫收获。”

“现在,正魔交锋已经过了第六次。仙蛊屋中我的战意规模,恐怕已经是南疆蛊仙总和的三十多倍。若是此刻和他们交锋,我不出一炷香的功夫,就能将南疆的这些战意,都剿灭得干干净净。”

哪怕南疆蛊仙之中,也有不少人有着智道造诣。

但那又如何?

不提方源的智道宗师境界,单单战意的规模,就已经占据了绝对的优势。

南疆蛊仙们已经彻底丧失了希望,没有胜出的可能了。

可以说,这些蛊仙们已经不足虑,但方源心中仍旧充满担忧:“转化了这么久,仙蛊屋内的纯净战意,居然还有余量。不知道所剩下的,究竟还有多少?我能否赶得及?”

方源抬头看天。

天空中,夜色深重,雾霾重重。

方源心头不禁有些压抑。

惊鸿乱斗台中的纯净战意,其规模大大超出方源的预料。

他原本的计划,是提前收取仙蛊屋,拍拍屁股,潇洒走人。但现在,时间已经拖到现在,何时是个头?(未完待续)

\end{this_body}

