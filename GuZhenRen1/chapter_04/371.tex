\newsection{截取最大的成果!}    %第三百七十二节:截取最大的成果!

\begin{this_body}

“万劫来了!”方源咬牙切齿,死死等着灰色的哀云,慢悠悠地覆盖下来。[看本书最新章节请到棉花糖小说网www.mianhuatang.cc]

仙僵薄青也不由地眯起双眼,浑身肌肉紧紧绷住。

但灰云降下,却似毫无伤害。

“怎么回事?难道有无害的万劫?!”方源紧张半天,却不见动静,双手四处乱摸自己的身体,惊疑不定地大叫道。

仙僵薄青没有回答他。

天庭蛊仙中有人,认出了这是万劫灰忆。但方源、墨瑶残魂却不知道这点。

不过,很快,方源、墨瑶残魂都有所明悟了。

因为魔尊幽魂身陷灰雾当中之后,便雾气翻涌,浮现出他记忆中灰色的部分。

童年的各种经历,少年的遭遇,青年时期的历险……在整个生命历程,幽魂生前的心灵拷问,和一次次的命运抉择。

而与此同时,太白云生夹裹着方源,向外一路疾飞。

黑楼兰最后回望一眼漫天的灰雾,紧随其后。

方源体内,影无邪陷入梦境之中,面对星宿仙尊。

他咬牙切齿,发狂大叫,朝着星宿仙尊扑去。

结果,却扑了一个空。

这是梦境,咫尺天涯。

他看似和星宿仙尊很近,但实际上不论他前行多少步,他和星宿仙尊的距离是始终不变的。

星宿仙尊手扶琴弦,却不吟诗,琴声婉转清悠,宛若山泉潺潺。

和影无邪的气急败坏,形成了鲜明的对比。

“快放我出去,快放我出去。天意!你这该死的天意!!”他手指着星宿仙尊,痛声咒骂。

但星宿仙尊无动于衷,没有更多的喜,也没有更多的怒。

影无邪连连眨眼,强自冷静,口中呢喃:“不要慌张,要冷静,要冷静。我是对引魂入梦知根知底的。这梦境还没有达到外显的程度,总会消散,不会一直困住我的。还有,方源身边的蛊仙。恐怕也会救我。我一醒来,就要立即回去,将方源的身份直接披露出来!现在就希望,我醒来之后,距离义天山不会太远!”

但影无邪注定不能如愿了。

因为接下来。黑楼兰、太白云生遭遇了南疆蛊仙,双方且战且退。

不仅让黑楼兰、太白云生没有闲暇,去想方设法唤醒影无邪,而且还距离义天山越加遥远。

“影无邪”、仙僵薄青二人,因为地陷,一直困在地面上,无法动弹。

一股股的灰雾,闪烁着迷离的光彩,不断变化着画面,围绕着仙僵薄青。

那是墨瑶早年的记忆。身为异人,遭受排挤的悲惨历史。

“哼。”仙僵薄青发出冷笑。

面对过往的不堪和阴影,墨瑶残魂表现出了坚定的一面。<strong>在线阅读天火大道Http://wWw.qiushu.cc/</strong>

“奇怪。你怎么会不受灰雾的影响?”薄青又望着身旁的影无邪。

方源心道:“你这个问题,我还想知道呢!”

按照常理而言,灰忆影响魔尊幽魂,又影响墨瑶残魂,没道理不波及方源魂魄。

但偏偏,方源身边没有任何动静。

很明显,这场万劫灰忆,放过了方源。让方源没有因此露馅。

“难道说,这是天意的故意纵容吗?之前天意不加干涉黑楼兰、太白云生,也是想让他们将我的肉身,带出去。远离这里吗?”

方源心中有许多猜测,这些猜测的背后,有着一股冰寒的冷意,让方源稍微想想,都感觉汗毛倒竖,凭空而生出许多惊悚。

但现在。面对仙僵薄青的质疑,方源只能装傻。

“我也不知道啊。”方源挠头,又手指向前,“不过你这个挺有意思啊。”

薄青冷哼一声:“看来你是刚刚出生,又只有九个时辰的寿命,所以根本来不及体验这些东西吧。”

仙僵薄青体内的墨瑶残魂,没有过多怀疑方源,而是主要的注意力,都放在魔尊幽魂身上。

对幽魂本体,墨瑶残魂很是担忧。

紧接着,灰雾中呈现出幽魂魔尊生前,继承兽人蛊仙的食道真传的画面。

幽魂魔尊的心性,也让方源惊叹不已。

天庭蛊仙更是炸开了锅,奈何陷入虚化状态,无法出手。

这时,苍穹之中,又降下第三场万劫太清空宇!

无数青鸟,对魔尊幽魂千刀万剐,威势恐怖绝伦,让方源、墨瑶残魂都纷纷变色。

接着,方源便看到灰雾中呈现的种种画面。

羽圣城、盗天真传、八十八角真阳楼之中、光阴长河中的红莲……

“原来,幽魂魔尊生前,就知道羽圣城的地点了。”

“盗天魔尊叫做本杰孙,他虽然和我一样是天外之魔,却不是来自地球啊!”

“寿蛊……看来天意才是仙尊、魔尊的最大制约。”

“红莲!鬼脸!我上一次重生失败,竟然是幽魂魔尊出手助我!我明白了,上一世大力真武仙僵被灭,十绝大阵无法完成。魔尊幽魂便借我之手,重新来过!”

“什么,人祖传就是人祖的真传?”

随后,方源看到魔尊幽魂自恃吞魂之术,却在金光魂魄的眼前吃瘪。又看到十大分魂,残存有七,重回了人间。

然后,结合墨瑶残魂的灰忆,知道了慧剑仙蛊的创始起因,薄青、墨瑶之间的恩怨纠葛。

“那个绽射金芒,呼呼大睡的魂魄,就是太日阳莽吗?传说中,人祖的右眼所化?”

“薄青也就是青!那个紫,竟然就是太白云生的师傅,我胡乱编造的紫山真君?!”

“原来当初,墨瑶和薄青的历史真相,是这样的。”

“天意!原来这些绝佳的仙蛊材,竟然充斥天意,所以影宗忌惮,不得不放弃。我得到手中,一直对我却似乎没有什么危害。不过将来呢?我将这些仙材随身携带,是否意味着我就受到了天意的关注?”

“原来如此。影宗的智道蛊仙大荔,便是被镇压的八转大力真武仙僵!一切都说得通了。”

这时,太清空宇万劫消散。阴油毒万劫降临,继续摧残魔尊幽魂。

接下来的一幕幕,让方源更加震动!

他终于明白,自己拥有春秋蝉的秘密。早已经被影宗获知。

正因为影宗的帮助,使得他推翻八十八角真阳楼之后,过了这么久,都未暴露出去。

反而,影宗的秦百胜等人。替他挡灾,和凤九歌等人血战一场。

影宗之强,居然直接算尽了此次的灾劫。

“重生之秘,我拥有春秋蝉,经此一战,广为人知了!”

“影宗、魔尊幽魂,似乎要利用我天外之魔的身份,结果弄巧成拙,让影宗北原分部几乎损失殆尽。不,从某个角度而言。岂不是说明天意的厉害?”

“影宗能算尽灾劫,用的是什么手段?这简直是渡劫的一大利器!”

“幽魂魔尊……似乎已经成为最强的尊者了。他生前或许和其他尊者相差伯仲之间,但他通过食道真传,创出吞魂之术。也就可以夺取其他的修行记忆和经验。所以不像梦境直接提升境界,然而经过这么多年的积蓄和发展,他广泛涉猎其他流派,阵道、智道上绝对有大宗师之境!算尽灾劫,也不奇怪。”

“如此一来,他无动于衷,就太反常了。难道说。他是故意如此做的吗?”

阴油毒、灰忆渐渐消散,但荆虬劫随之降临。

监天塔施展最终的反扑,竟然能操纵万劫,直接进攻十绝大阵。

魔尊幽魂的防守之战。让目睹这一切的方源瞠目结舌。他曾为自己积累的战斗技艺而骄傲,但和魔尊幽魂对比起来,却是不算什么了。

最终,监天塔崩解,天庭一方失败,就连监天塔主也随之牺牲。

两败俱伤!

魔尊幽魂的体格。只剩下几丈的高度。并且再无之前的凝实,气息虚弱。

墨瑶残魂驾驭着仙僵之躯,飞上高空,质问魔尊幽魂,结果被灭。

“你就是我最后的阻碍吗?呵呵呵,真是可惜。若是没有砚石的临终提醒,说不定还真会让你惹出些麻烦。不过现在……哼哼。你忘了你一身的剑道仙蛊,都是我的分魂意志,你更忘了,这些年是谁偷偷地喂养着它们。”

魔尊幽魂的这句话,让方源不由心头一紧。

趁着这段时间,他已经参悟出仙道杀招引魂入梦的不少内容。

这还有归功于影无邪的身上,几乎只有专门组成引魂入梦的蛊虫。

虽然影宗专门为他设计的这个杀招,根据修为不同,分成了几个层次。但影无邪分理归纳得很好,将用到的蛊虫,都各自堆成一堆。

这样做,无疑能让他更快,更简明地运用蛊虫,更快地催动杀招。但同时,也更加方便方源参悟其中的奥妙。

方源心中明白:“面对魔尊幽魂本体,我恐怕只有一次出手的机会。一次不成,就算我有态度蛊,让魔尊幽魂反应过来,我将再无机会!”

魔尊幽魂一口吞下墨瑶残魂,便对方源道:“无邪。你来护卫我,只需要十几个呼吸的时间就能大功告成!”

“是。”方源心中一片冷静,飞升上来,站到魔尊幽魂的身边,一脸警惕地盯着周围。

“没有机会了。”天庭蛊仙们纷纷叹息,陆续撤离。

“筹谋了数万年,殚精极虑,苦心经营,终于在今天炼成了这只九转至尊仙胎蛊。有了这只仙蛊,我就能超越历史诸尊,成就半个天魔之身,达到前无古人后无来者的无上境界!历代仙尊、魔尊,都不会是我的对手,哪怕是人祖复活,也要对我甘拜下风!哈哈哈……”

魔尊幽魂仰头大笑。

“至尊仙胎蛊?”方源心中一动,“这蛊虫是何用途,能让魔尊幽魂达到前无古人后无来者的无上境界?这种境界是否就是永生?”

“不过越到最后关头,我越不能大意。”魔尊幽魂双眼眯起,开始侦查。

方源的心,几乎漏跳一拍!

但最终,魔尊幽魂只是喷吐幽气,清除了侦查蛊虫。

“果然是这样。最后一句诗词,揭示了鬼不觉的真正用途,就是防备鬼魂的感知。当初盗天魔尊能盗取落魄谷、荡魂山,穿梭沉迷死境,而不被无尽魂魄纠缠,就是因为此招!”

“因此,魔尊幽魂没有怀疑我。看来影无邪中的魂魄,也是他的分魂。只有对自己,他才如此毫无保留的信任罢。”

接着,方源便目睹着魔尊幽魂飘向至尊仙胎蛊。

最后关头,方源反倒是平静下来,不再紧张。

诗词的提示,是让他直接摧毁了这只至尊仙胎蛊。因为仙蛊脆弱,双手一捏,就能摧毁。但魔尊幽魂,哪怕他再孱弱,要对付他始终都有未知和风险。

但方源却毅然决定,直接对付魔尊幽魂,夺取至尊仙胎蛊!

“引魂入梦。”他忽然出手。

他没有参悟出全部,这是不完整的仙道杀招,和八转层次的原版有差距。

但此时此刻,魔尊幽魂已经虚弱到了极点。

方源一招得手!

随后,他脱离影无邪的身躯,学习魔尊幽魂,直接让魂魄飘入至尊仙胎蛊中。

这一刻,他摆脱星宿仙尊的指示,截取了魔尊幽魂数万年筹谋计划,牺牲无数,耗费海量代价,而千辛万苦才凝出的心血成果!

ps:月票破900的加更!(\~{}\^{}\~{})<!--80txt.com-ouoou-->

------------

\end{this_body}

