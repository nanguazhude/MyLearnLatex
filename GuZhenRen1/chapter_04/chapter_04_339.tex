\newsection{星眸仙蛊}    %第三百四十节:星眸仙蛊

\begin{this_body}



%1
中洲,星象福地。

%2
方源精神集中,一丝不苟,正在炼制全力以赴蛊。

%3
大量的仙材,已经融化为一团浆液。

%4
黄褐色的浆液,悬空漂浮着。

%5
方源心念一动,第二本命蛊虫五转全力以赴蛊,就投射出去,射进浆液之中。

%6
“起。”

%7
旋即,方源暗喝一声,伸出手掌,五指张开,照准地面虚抓。

%8
垮啦啦。

%9
一阵响动,地表土石翻飞,将浆液彻底包裹起来,形成一个土球。

%10
这是一招炼道杀招,隶属于土道。在不久前,方源从琅琊地灵处换得。

%11
这算是他加入琅琊派的小福利。

%12
当然,不是无偿的。

%13
方源也为琅琊地灵,推算了一道仙蛊方,完成到三成左右。

%14
这个炼道杀招,虽然不是蛊仙级数,但非常好用。方源改良后的全力以赴仙蛊方,在炼蛊过程当中,就多番运用这个炼道杀招。

%15
血炼杀招血练蛇!

%16
方源雄躯一震,眼窍、鼻窍、耳窍等七窍尽皆往外流血。

%17
血流不止,化为七条长蛇,蜿蜒游荡,蛇信吐出,发出渗人的蛇鸣。

%18
七条血蛇,攀爬上土球。

%19
土球被缠绕成一个蛇球。

%20
方源将一只只凡蛊不断投出去,血蛇昂首张口撕吞。每吞下一只蛊虫,它的蛇躯就沉入土球一分。

%21
一个多时辰之后。血蛇完全融入土球。

%22
在土球的表面,满布着密密麻麻的赤红蛇鳞。

%23
整个土球,也仿佛是有了生命,赤红蛇鳞不断地闪烁着微光,明暗之间。就像是土球中有生命在呼吸。

%24
方源再用火烤。

%25
这是三种不同颜色的火焰,轮番炙烤蛇鳞土球。

%26
蓝色的火光中,土球表面微光闪烁的间隔便得悠长,赤红的蛇鳞上凝出一层白色的冷霜。

%27
方源又用青色火焰。

%28
烧烤之后,土球上的微光明显变得活泼起来,冷霜中长出了一层薄薄的青苔。

%29
最后方源用某种黄色火焰。

%30
在火焰之中,土球表面的青苔硬结起来。转为褐色的血茧。血茧上点缀着点点金光。仿佛是金沙掺和进来。

%31
黄色的火焰烧烤土球的时间,最为漫长。

%32
足足炙烤了一天一夜,才见整个褐色的土球,转变成纯金之色。

%33
达到这一步后,终于暂告一个段落。

%34
方源松了一口气,他得到了宝贵的喘息,随后休息了两天两夜。

%35
状态恢复完整,他再拿出金色土球。

%36
原本褐色的土球大如脸盆,但烤成金黄色泽之后,这个土球只比婴儿头大了少许。

%37
接下来,是一步关键。

%38
方源深呼吸一口气,催动某只特定的蛊虫,轰击金球。

%39
砰的一声,金色土球爆炸,无数的金粉顿时扩散,弥漫方圆三里。

%40
“失败了!”方源脸色一沉。原本期待的目光陡然晦暗下去。

%41
漫天的金色粉尘,空无一物。若是成功,则会形成一只仙蛊的半成品出来!

%42
但可惜,这次并没有。

%43
之前投入的仙材、仙元石还有无数凡蛊,都打了水漂。唯一的收获,就是方源增添了一些炼蛊经验,下次尝试的时候。或许能稍稍熟练一点。

%44
叹息一声,方源很快就恢复正常的心态,平心静气。

%45
炼制仙蛊不容易,成功率太低。

%46
有此失败,相当正常。接下里再继续积累仙材,继续尝试便是。

%47
方源前世炼成春秋蝉,今生又经历了星念仙蛊之后,在这方面的心理承受能力直线上升。

%48
休整完毕后,他利用定仙游,出了福地,来到南疆。

%49
夜色正浓。

%50
方源一路飞升,直上云霄。

%51
在浮云之上,繁星更见清晰璀璨

%52
星眸!

%53
这只仙蛊,方源也是最近才得到手。

%54
正是玉露福地中的收获。

%55
方源催动这只仙蛊,顿时他双眼的瞳孔,化为一团水涡,不断旋转。

%56
一股无形的吸引力,将夜空中的星芒,徐徐地向方源的眼内吸摄。

%57
六转仙蛊星眸,位列十大奇蛊榜中的第十位!

%58
这是一只侦查仙蛊。

%59
方源并不缺乏侦查手段,但左思右想之后,最终还是决定选择这只仙蛊。

%60
皆因,此蛊的侦查威能,已经超越大多数的侦查仙道杀招!

%61
借助星眸,方源可以将夜空中的星辰,暂时化作自己的眼睛,来进行观察。

%62
黑天中的星辰之光,照耀五域大地。但凡有星辰之光照耀的地方,就能被方源查探。

%63
方源完全可以身在中洲,借助夜空中的星辰,观察到南疆、北原等等地方。

%64
因此,此蛊名列第十。

%65
“可惜,若是没有弊端,这个排名还能上升个两三位。”

%66
星眸仙蛊的弊端,就是不是拿来就能用的。在使用之前,还需一个沟通星辰的过程。

%67
方源此时就在做这个事情。

%68
他眼中的漩涡,越转越缓,星辰之光被不断筛选、排除,最终一些星芒宛若无数道丝线纠缠在一起,形成一道小拇指粗细的星光丝线。

%69
左眼一道,右眼一道,一共两道星光丝线。

%70
这两道星光凝聚而成的丝线,一头连着方源的瞳孔,另外一头则是黑天中的一个星辰。

%71
一颗颗的青提仙元,接连消耗。

%72
星眸仙蛊催动不停,一股玄妙的力量,透过星光丝线,一直传上去,直至传到终点处。

%73
方源明明睁大双眼,但此时的视野中却是一片黑暗。

%74
但是随着星眸仙蛊的不断催动。他渐渐能从黑暗中勉强地看到一个微弱的蓝点。

%75
时间渐渐流逝,很快一夜就过去了。

%76
方源视野中,仍旧被黑暗占据。但正中央的地方,那个蓝点儿已经扩大了不少。、

%77
“假以时日,这个蓝点就会越来越大。逐渐还原出星辰的全貌。但愿我选的这颗星辰,体型稍小,否则要完全化为己用,需要的时间和精力可不小!”

%78
星辰体积稍小的话,方源完全化为己用的时间就大为提前。

%79
虽然小星辰的星光,比较暗淡。大星辰的星光,更具穿透力。

%80
但方源还是宁愿如此。他想要更快地实现星眸仙蛊的妙用。

%81
天空渐白。方源终于停手。

%82
他清点了一下,这才一夜时间,自己催动星眸耗费的仙元,就达八颗之多!

%83
“星眸第十,弊端就是耗费太多。要炼化一颗星辰,不亚于十场六转仙蛊的炼制。幸好这点,我却可以支撑得住。”

%84
“且不说我的星道境界。已经达到宗师,又有智慧光晕,星眸还可以掺和其他仙道杀招,应用范围并不狭隘。更关键的是,若是让星眸和定仙游搭配起来的话……”

%85
方源思维散发出去,想到当中的妙处,不禁嘴角泛笑。

%86
焚天魔女想必就是知道这点,才用星眸仙蛊来做文章的罢。

%87
不得不说,定仙游和星眸仙蛊两者,十分搭配。

%88
接下来的日子里。方源一边尝试炼制全力以赴仙蛊,一边琢磨玉露仙子留下来的战场杀招。这当中最重要的内容,就是如何在仙窍中搭建战场杀招,而不损伤仙窍。

%89
与此同时,方源还趁着闲暇功夫,运转星眸仙蛊,炼化黑天中的星辰。

%90
并且。他一直对外界的情报,保持着密切的关注。

%91
时间在方源勤修苦炼中滑过。

%92
北原、中洲等地的情况,和前世并无多少变化。

%93
但在东海之中,登天野的局势有了改变。方源前世时,大量的散修、魔仙闯入登天野,将宋家、蔡家、若来家辛苦维系的局面,搅得乱七八糟。

%94
这个过程中,七海蛇女还取笑挖苦宋亦诗,导致宋亦诗前去灭了海边小村,把李逍遥的祖先都杀了。

%95
但今生,这些散仙、魔修却遭到了宋家、蔡家、若来三家的联合痛击,还未撼动局势,就分崩离析了。究其原因,应当便是焚天魔女带着方源闯荡,让三家蛊仙提前警觉起来。他们三家内斗的时候,更加克制。一旦有了外来势力,企图染指,三家就会立即联手。

%96
方源最关心的,还是落魄谷。

%97
他知道,落魄谷中正上演着百日大战!

%98
以凤九歌为首的中洲蛊仙一方,对战秦百胜领袖的影宗蛊仙。战况究竟如何,方源有心探索,但又能力不足。

%99
他只能在落魄谷的方圆百里之外,布置一些侦查蛊虫。

%100
这个距离已经是很是危险了。

%101
再接近的话,被两方蛊仙发现的可能相当的大。

%102
不管是中洲,还是影宗,都不是吃素的。他们在谷中打生打死,肯定也会担忧第三方来捡便宜。也必然做了许多侦查、反侦查,布置战场杀招,布置蛊阵等等准备。

%103
“今生和前世不同,我故意留下雪松子一条性命,就是为了增强影宗方面的实力。百日大战的具体情形,肯定会发生一些变化。”

%104
前世影宗失败,但也有人成功突围。而中洲一方虽然胜利,但却折了凤九歌,只能算是惨胜。

%105
今生的百日大战,究竟结果如何,方源也十分好奇。

%106
不过在没有分出胜负之前,他还不想掺和进去。失去了我力,全力以赴蛊还未炼制出来,战力底气不足。

%107
不知不觉间,光阴流逝,前世百日大战结束的日子,已经过去三天。

%108
方源掐着手指推算,人已经蹲在北原,耐心守候。

%109
落魄谷方面终于有了反应!

\end{this_body}

