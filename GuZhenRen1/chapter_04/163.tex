\newsection{八转对抗风云荡}    %第一百六十三节:八转对抗风云荡

\begin{this_body}

一提到八转蛊仙雪胡老祖,中洲众仙都沉默下来。<strong>最新章节全文阅读www.Mianhuatang.cc</strong>

这些人,都是七转修为,战力出色,在各自门派中能有前五位置。但面对八转蛊仙,只有失败的战果。

凤九歌虽然和八转蛊仙有一战之力,但终究还是七转的修为。

而身为魔道的雪胡老祖,更是战力雄浑恐怖,在中洲收集的情报中,隐隐有北原八转第一人的迹象!

就在中洲众仙沉默的时候,三位魔道蛊仙隐着身形,悄悄进入中洲蛊仙们的埋伏圈里。

他们进入地下,来到方源杀死东方长凡的现场,这才显露行迹。

这三位蛊仙,为首的身材高大,浑身上下散发着彪悍之气,他有一双飞扬入鬓的眉毛,鹰钩鼻子,双眼厉芒闪烁。一对耳朵却不是人耳,而是形如羽翅,金光闪闪。

他便是北原大雪山福地,第四支峰峰主厉鹏王。

在他左边的那位,虽然比厉鹏王要矮一头,但身材敦实厚重,顾盼之间,气概流露,有着霸气,带着冰寒龙威,正是第五支峰峰主,坐拥冰湖宫殿的龙胆魔君。

最后一名蛊仙,走在厉鹏王的右边。双眼如点漆,一身白袍,身体欣长,也做书生打扮,但比自在书生少了一份风流洒脱,多了一份磅礴大气。乃是第六支峰峰主公子墨。

中洲蛊仙埋伏在侧,手段高超。这三位魔道蛊仙却未发现。

他们集中注意力,紧盯着眼前。

眼前空无一物,但在他们的感知下,却发现有一座福地隐藏着。

“这处便是那东方长凡丧命之后,而留下的福地?”

“动手吧。时间拖得越久,变数就越多。在这附近,就是郄世民死后遗留的福地。”

“既是雪胡老祖下令,我等三人自当竭尽全力,攻破这处福地。这就动手吧!”

三人雷厉风行,又充分准备了特别手段,在外界做了粗浅布置后。立即打破空间。杀入福地。

见到这三位进入之后,中洲蛊仙们这才继续用神念交流。

“好家伙!大雪山福地中第四、第五、第六峰主一齐出动了。”

“大雪山中前三支峰都是地位稳固,其余的却可以将竞争排位。(www.QiuShu.cc 求、书=‘网’小‘说’)这三位向来明争暗斗,如今一同出手,众志成城,也只有雪胡老祖才能指挥得出来。”

“没有等来那个仙僵,却等来了大雪山福地中人。我们是否出手擒拿他们?”

“不妥。这三仙战力非凡。还要超出自在书生这等人物一筹,等同于我们。我们虽然人多势众,但疏于配合。要战败他们三个容易,杀死他们却很难,活着擒拿更是难上加难。”

对于这点,残阳老君深有感慨地道:“是啊,北原是久战之地,五域中战力第一。一位位蛊仙战斗经验都十分丰富,有着许多独特的手段,保命功夫更远胜我中洲。一个个滑溜无比。之前太丘、碧潭福地一战,损命的蛊仙,屈指可数。这若放在其他四域,死伤数量绝对要高出两三倍的。”

“活捉他们也不是不可能,嘿嘿,只要凤九歌大人亲自出手即可。”有蛊仙这样说道。

北原此行,凤九歌的强悍战力。留给中洲蛊仙极为深刻的印象。

但凤九歌却道:“我们退。”

说着,不给众仙询问的机会,他自己率先就撤,隐秘非常,不漏丝毫马脚。

众仙面面相觑一番,只得悄悄跟上。

一行人在极远处汇合,众仙纷纷向凤九歌表示不解。

之前设下埋伏圈,是凤九歌的主意。现在等来了三位魔道蛊仙,凤九歌却又第一个撤离。

“我们这些人众志成城,还怕他们三个?”

“不战而退,有损我中洲十大派的威名啊!”

“来来回回,究竟意欲何往?”

蛊仙们离碧潭福地越来越远,语气中含有怨气,只是碍于凤九歌的威名,发作不得。

凤九歌扫视众人一圈,语出惊人地道:“黎山仙子抢得方寸山,回归大雪山福地休养。我已得到消息,北原正道八转蛊仙药皇已经启程,前往大雪山福地,向雪胡老祖索要方寸山。”

“什么?竟有这等事情!”

“两大八转要对抗了吗?”

“这北原真是……八转蛊仙向来镇压大势,就因为方寸山,便如此随意地出动了。还真是好战。想当初的荡魂山之争,我们中洲十大派也不过派遣弟子争夺。”

众仙议论纷纷,脸色动容,都在消化这个惊人的消息。

这时,凤九歌淡淡一笑,又道:“那个力道仙僵,十分谨慎警觉。大雪山福地的魔道三仙,不过是他借力打力,是他或者他背后势力的试探。我们要吞下这诱饵,反中了对方的算计。不如这就前往大雪山,看看有无机会,强攻进去,俘虏黎山仙子出来。”

众仙无言,一时都说不出话来。

大雪山福地,乃是北原魔道第一巢穴,不下于中洲十大古派的大本营。

凤九歌胆魄大至如斯,居然敢打大雪山的主意!

……

大雪山福地,第一支峰。

雪胡老祖深坐在宽背大椅上,手中拿捏着一座小山,不断转动,静静地观赏。

这座小山表面全是裂痕,惨不忍睹,山体上横尸遍野,都是小人的尸体,血流遍地,隐隐传来幸存者的哀嚎哭泣之声,当真是一片愁云惨淡。

不是旁的,正是方寸山。

黎山仙子追击方寸山,和小人蛊仙大战,结果引来其他蛊仙出手,有正有魔。

一番乱战之后,小人蛊仙濒死,带着方寸山乱窜,结果被战场边缘的黑楼兰逮个正着。

黑楼兰顺势杀死小人蛊仙,和黎山仙子一道拼死护住战果,一路拼杀,九死一生,最终回归大雪山福地。

这也导致了方寸山如今濒临崩解的惨状,山上的小人族更是死伤惨重。

“这就是方寸山了?”这时,一位女仙,从后面缓步走出。

她号称万寿娘子,乃是雪胡老祖之妻,七转修为,北原四大炼道蛊仙之一。

她一出现,就将目光投向方寸山,对于传说中的这座山,很有兴趣。

“夫人,你这次参悟可有心得?”雪胡老祖微微转过头,一边温言询问,一边将手中的方寸山递给她。

万寿娘子点点头:“我已经悟透了炼制鸿运齐天蛊的全部步骤,不过要熟悉掌握,还有一段距离。”

她说着,接过方寸山,放到眼前细细打量。

方寸山上,因为小人蛊仙已死,没有了主心骨支柱,幸存的小人们瑟瑟发抖,挤成一团团。

“怪物啊!”

“好可怕,好可怕……”

“千万不要吃我,我的肉太少,还不够您塞牙缝呢。”

万寿娘子美丽端庄,凤目灼灼,在小人看来,却是大怪兽,相当可怕。

万寿娘子端详片刻,将方寸山从眼前拿开,轻笑道:“黎山妹妹有福了。她早年本就是土道蛊仙,兼修土道。后来因为得到山盟仙蛊,已然将其换成核心仙蛊,成为半道出家的信道蛊仙。但其实她的福地还是木道福地,经营花草。有了小人一族辅助,正是相得益彰了。黎山妹妹是我大雪山的成员,如今方寸山归于我们之手,这正证明,我大雪山气运隆昌,越发兴旺。”

“夫人,你此言差矣。”雪胡老祖却是摇头,“恰恰相反,这正是气运低垂之迹象。我要炼制鸿运齐天蛊,将马鸿运当做炼蛊主材,因此鸿运袭我,令我遇到了阻碍。你看这方寸山,已经濒临崩溃,要修复它,势必要花费极大代价。这既是三当家之物,自然不会放弃,要令她狠狠大出血一次。这就间接影响了我搜集炼蛊材料的大计。匹夫无罪怀璧其罪,这次她抢夺了方寸山,引发矛盾,建立仇恨。更搅动风云,令正道出手。你看,这是药皇的来信。”

药皇之信?

万寿娘子大吃一惊,连忙翻开信笺浏览内容。

信中,药皇阐述:方寸山位于碧潭福地,本是东方部族之物。如今东方一族被灭,天下震动。身为相同血脉,同为巨阳仙尊的子孙,我药皇应正道各方请求出面,要和你雪胡老祖商讨,将方寸山归还正道一事。

信中药皇措辞分外客气温和,但万寿娘子知道,这是正道的风格正道向来就是这样,哪怕打生打死,恨得牙根痒痒,表面上也会客客气气,语气含蓄,保持风度风采。

万寿娘子从中感知到药皇的咄咄逼人,不由担忧地看向雪胡老祖:“夫君,你距离下一次万劫,只有十多年。可不能轻易动手!若是在战斗中有个闪失,伤了本源,损了底蕴,就算炼出了鸿运齐天蛊,也积重难返,得不偿失。到了万劫来临之际,就因为这些损伤而功亏一篑的话……”

雪胡老祖哈哈大笑:“夫人,你还是老样子。药皇这信,是我的一劫,但更是我的机会,我当牢牢把握住,迎难而上才是。”

雪胡老祖说到这里,忽然一顿,神色微变,带出一丝决意:“嗯?他来了,来的好啊!”(想知道《蛊真人》更多精彩动态吗?现在就开启微信,点击右上方“+”号,选择添加朋友中添加公众号,搜索“zhongenang”,关注公众号,再也不会错过每次更新!51read)(未完待续。)<!--80txt.com-ouoou-->

\end{this_body}

