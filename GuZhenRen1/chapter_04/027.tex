\newsection{人药蛊}    %第二十七节:人药蛊

\begin{this_body}

。

韩立躺在床上,悠悠醒转,睁开一丝眼缝.

在床榻旁边,他的父母跪在地上,正对一位蛊师老者磕头不止,连连感谢.

"墨大夫,谢谢你救活了我的儿子!"

"墨大夫,您的大恩大德我们两口子会记挂一辈子的!"

"爹,娘……"韩立艰难地张开口,轻声呼唤道.

"我儿,我的心肝唉!你终于醒了!"韩立之母听到声音,一把扑到床边,喜极而泣.

"小崽子,你终于醒了!快谢谢墨大夫,要不是他恰好走到这里,你就真的死了."韩立的父亲也是欢喜异常,走过来,赶忙提醒道.

"墨,墨大夫,谢谢你救了我的命."韩立见到墨大夫,眼中闪过一丝惊惧.

"嗯,我是驻守在韩家村的蛊师,救下你也是应该.你今天运气好,恰巧赶上我巡视村子的时候."墨大夫笑了笑.

随后,他在韩立父母的千恩万谢声中,施施然离开了这里.

"嗯?这倒是有趣."方源并未走远,潜藏在远处,将这一幕经过看在眼里.

他连运成功之后,韩立的气运就急剧降低,迅速转变成了淡薄的黑死运气.

韩立若是蛊师,实力强劲,可以无视这运气.但他只是一个凡人,还是凡人中弱小的孩童,立即被黑死运气反噬,吃饭的时候噎死了.

方源当即想要动手,赶去救援.

他手中有大量治疗的凡蛊.区区噎死,只要死的时间不长,完全可以重新救活.

只要人不死.运气就会源源不断.韩立若就这样死了,方源的糟糕运气也只是被缓解了许多,仍旧会再变坏,治标不治本.

韩立若生还,每时每刻都会和方源连运,不断地改善方源的坏运气.

但是他刚启程时,忽然发现一道身影.从村中的一间大屋中跑出来,并且迅速向韩立家赶去.

看他催动的移动蛊,很显然.他是一位二转蛊师.

方源眼睛一眯,悄悄接近村子,决定静观其变.

这二转蛊师来到韩立家门口,整顿了衣裳.故意装作云淡风轻的样子.随后"意外"地发现了韩立噎死的情形.

在韩立父母的恳求下,他当场出手救治,将韩立救活.

"听他们的谈话,这个墨大夫应该就是驻村的一位治疗蛊师.不过看样子,韩立似乎和这个墨大夫有隐秘联系."方源心中一动.

蛊师势力为了控制凡人,都会派遣一位到两位的蛊师,轮番驻守在村子里.

像青茅山的古月江牙,就是一名这样的驻守蛊师.

待韩立一家三口沉沉睡去.方源悄悄潜入屋内.他来到韩立身旁,使了一只凡蛊.令其睡眠更沉.

方源伸出八只大手,拿捏韩立身体的各个部位,不一会儿他就发现了端倪.

韩立的身上,寄生着好些蛊虫,浑身血肉也似乎被蛊虫施加了影响.

这些布置让方源隐约感到很熟悉.

僵尸的弊端,就是思维蒋,方源想不出关键,只得调动脑海中储藏的意志帮助思考.

他搜刮记忆,这一次很快发现真相:"原来这个所谓的墨大夫,是要炼人药蛊啊."

这人药蛊,是一种三转蛊.用人类孩童当做炼蛊的主要材料,历时数年,慢慢炼成.

人药蛊一旦炼成,就只能使用一次,能给人增添寿命,但后遗症颇大.

"那个墨大夫已经不年轻了,显然是想借助人药蛊增长寿命,于是把主意打到韩立的身上.难怪韩立割草的时候,可以直接徒手拿捏匕首草的草叶,这是因为他的身体已经被暗中改造.难怪韩立出事之后,墨大夫就飞一般的赶来,这是因为韩立身上的蛊虫可以时刻监视他的动态.也难怪韩立噎死后,轻而易举地被救活了,因为依照人药蛊炼制步骤,会令人的身体机能大大提高."方源恍然大悟.

不过他虽然发现了墨大夫的阴谋,但却暂时不想出手阻止.

"墨大夫若神秘死亡,韩家村一定会遭到调查,途生变故.他若不死,为了炼制人药蛊,肯定会多加照顾韩立.距离人药蛊的炼成,还有三四年的时间,而我只是需要韩立活着罢了."

方源再次催动察运蛊,查看韩立的运气.

自从连运成功之后,韩立的运气便一落千丈,再无先前特别气象,而是仿佛周围凡人一样,整个运气淡薄,宛若袅袅轻烟.

方源仔细观察,发现韩立之前气运中的黑色,已经消失,呈现出灰白颜色.

方源松了口气.

他总结经验得出:黑色代表死运,没有了黑色,韩立暂时也就没有死亡的危险.

正所谓否极泰来,大难不死必有后福,韩立经过这次死亡的危机后,气运明显的好转了.

"由此想来,我在北原,靠着自身实力和谋算,艰险地渡过了灾劫.从大同风幕逃生后,原本的黑棺气运也应当消散了大半.不过就算如此,韩立也险些被我克死.看来我还得再找一些目标,和他们一一连运."

方源目光闪烁了几下,终于还是在临走前,在韩立的身上秘密布置了一些蛊虫.

凭他的巧妙手段,可以确保二转的墨大夫发觉不了.

有朝一日,墨大夫若要向韩立动手,必然吃不了兜着走.

"小子,我给了你三张护身符.今后的日子,就靠你自己了.嘿嘿……"

方源悄无声息地来,悄无声息地走,人不知鬼不觉.

他先远离沙井绿洲,找到一个无人的角落.这才再次催动定仙游.

下一刻,他出现在中洲.

巨大的天河从苍空而落,斜穿整个中洲.最终流出中洲东部,流向东海.

在天河的中下游,由于天河的冲刷,形成大片的肥沃平原.

平原上物产丰富,人杰地灵,无数的中小型门派在这里扎根.

在这些门派当中,有一个很不起眼的小门派.名字却极大气,叫做众生书院.

方源隐藏在树林中,远远视察着这座众生书院.

书院坐落在山谷之中.庭院房屋并不多,有一个小广场.似乎正值门派内举行大比,广场上摆设了两个擂台,数百位蛊师站在擂台下围观着.

凡人蛊师的战斗.持续时间很短.泛善可陈.不过在当事人眼中,却是精彩无比.

尤其是对于人群中的少年洪易而言,这场大比更是意义重大.

"我身负牛力,虎力,又因奇遇得到蛊虫‘过得去’,战力已经达到二转一流层次.这一次大比,我要不鸣则已一鸣惊人,夺得书院首席位置.这样一来,我就能让父亲低头.将我娘的牌位放到宗族祠堂里去!"洪易捏紧双拳,心中暗下决心.

与此同时.远处的方源也是双眼骤然一亮,口中喃喃:"发现你了,洪易."

这洪易运气不输给韩立,也是五域大战时期,涌现出来的七转传奇.他力道,魂道兼修,天资卓绝,极其擅长自创杀招.

他奇遇极多,得到过上古石人王的躯壳,参悟了星宿仙尊流传下来的一道传承,还在机缘巧合之下,收服了一只上古荒兽白角麒麟,当做了坐骑.

不过虽然发现了目标,方源却没有立即动手.

"现在是晴天白昼,碧空万里,我要连运,动静不小,恐怕会被发现."

这里不比韩家村.

韩家村中,只有一个二转蛊师驻守.而这里,却是蛊师的一个门派聚集地.

不仅如此,山谷周围就有中小势力三四家.

整个天河中下游的平原,门派林立,密密麻麻.在这胁人蛊师当中,还潜藏着几位散修蛊仙.这些蛊仙都开宗立派,方源若直接动用连运仙蛊,说不定就被其他人发现.

"只有等到晚上,众人熟睡.我再布置大量凡蛊,尽力遮掩气息,方可一试……"

方源年老成精,永远不缺乏耐心.

等到夜幕降临,众生书院寂静下来,方源缓缓睁开双眼:"很好,布置在山谷周围的近万只蛊虫,也都准备就绪了.为了尽数掩盖仙蛊气息,这一次连运要持续大半夜的功夫……嗯?"

就在这时,他的仙窍中传来一阵空间的波动.

一只推杯换盏蛊,在他的仙窍中凭空闪现而出.

推杯换盏中,藏有一封信笺.

"难道是太白云生出事了?"方源探入心神,很快发现来信的并非太白云生,而是黎山仙子.

雪山立盟之后,方源就交给黎山仙子以及黑楼兰,每人一只推杯换盏蛊,分别和自己手中的形成对应的一套.

信中内容,让方源眉头顿皱.

这是黎山仙子为黑楼兰发出的求救信!

黑楼兰正在被他的父亲率人追杀,危在旦夕.

"怎会如此?可恶!"方源深深地叹了一口气,碍于誓言,他只得放下手中的事情,必须迅速赶往北原,帮助黑楼兰.

不过,短时间之内,洪易仍旧生活在这里,方源要再寻找也很方便.

"这一次,就暂且放过你."方源匆匆回收大部分的蛊虫,尽量不留下痕迹.随后,他钻进一处山洞中,悄然动用定仙游.

下一刻,他来到大雪山福地第三支峰的密室里.

密室中,黎山仙子正急得踱步,见到方源后,她立即走近:"你终于来了!可把我急坏了,情况紧急,小兰正受到那恶贼的追杀."

"你怎么不去帮她?"方源皱眉.

"我原本准备现身,但小兰却强烈要求我继续隐藏,不愿将我俩的关系公之于众.只有你一个人来吗?太白云生呢?"

"他另有要务."方源又问,"追杀她的蛊仙有几位?"

"只有两位.一位是黑城,一位是雪松子.我在黑楼兰的身上放了侦察蛊,她情况很危急.快,我把景象用念头传给你,你快去救她!"

方源探查念头,再次催动定仙游.下一刻,他便出现在黑楼兰的身旁.

(未完待续)

------------

\end{this_body}

