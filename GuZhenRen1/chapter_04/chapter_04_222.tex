\newsection{石龙瞳}    %第二百二十三节:石龙瞳

\begin{this_body}

%1
北原,大雪山福地。

%2
天空中云层似盖,白雪如尘。诸峰在雪中默默矗立,山林覆盖着积雪,宛若万千云朵。

%3
一道虹光,从第一主峰处贯穿长空,宛若一颗流星,飞在空中。

%4
大雪山诸峰峰主,几乎都投去目光。

%5
“老祖又发布命令了。”

%6
“不知道这次,又是谁接到采集仙材的命令……”

%7
“但愿不是我。”

%8
虹光在空中划过一道优美的弧线,直接投入第三雪峰之中。

%9
许多峰主暗暗松了口气。

%10
第三雪峰之中,黎山仙子一身锦缎皮裙,白银打底,群面上绣着月下草原暗香图。

%11
这是北原风格的服饰。

%12
黎山仙子头戴宝蓝珍珠丝带,此刻正闭合眼帘,大部分的心神都投入仙窍当中。

%13
在她的仙窍里,鸟语花香,温暖如春,和冰窟雪地的大雪山完全是两个极端。

%14
丛林茂密,翠绿江山如织似画,有大量的珍稀作物,也有数量稀少的荒植,甚至上古荒植。

%15
在她仙窍中央,一处齐腰深的草丛中,隐没着一座小山。

%16
这座小山,乍看之下,连一个小土丘都比它更起眼。

%17
但它却是方寸山,《人祖传》中记载,是世间少有的秘境之一。

%18
在这方寸山中,大量的小人。飞进飞出。

%19
他们在极为茂盛的草丛中飞行,相对于体型微小的小人而言,这根根绿草,简直如大树般高耸。

%20
小人们在草丛中宛若蜂群一般辛勤劳作。

%21
黎山仙子心神暗中关注,看了一会儿。心中满意。

%22
自从她夺了方寸山之后,这些小人起先并不和她合作,十分抵触。

%23
但黎山仙子并非无能之辈,恩威并施,在小人族中扶持傀儡上位,毫不顾忌地打杀顽固分子。

%24
小人没有羽民的固执。但也有忠贞之辈。

%25
黎山仙子杀了大约四成的小人,终于收复方寸山中的小人一族。

%26
之后,黎山仙子将方寸山正式迁入自家仙窍。有了小人族的辅助,她仙窍中的木道资源,开始疯狂的增产!

%27
毛民天生有炼道道痕。所以有炼蛊的天赋。羽民有云道道痕,背生双翼,可以在空中自由飞翔。而小人从一出生,身上就带着木道道痕。有他们生存活动的地方,草木就会极为茂盛。而他们更有世代传承,是整个五域天地中最为擅长培育植株的种族。

%28
“小人的确非同凡响!我这片玉阴草,可是七转仙材。一直以来都是蔫萎枯稀,小人族在这里生活了大半年。就将这片草群扩张成微型草场,每一根玉阴草都生机盎然。可惜,小人的数量还是少了点。之前哪一战。让小人族人口损失大半。若是有之前的人口规模,我现在的压力必定减少大半。嗯?同时有两只信蛊?”

%29
黎山仙子念头一动,感应到两只信道蛊虫,一只在外界,划过长空而来,另一只则是钻出福地中深藏的推杯换盏蛊。

%30
黎山仙子睁开双眼。首先信手一招,将天空中的虹光。招入自己手中。

%31
探去心神,往里一瞧。

%32
不出她的意外。又是雪胡老祖委派下达给她的采集仙材的任务。

%33
“遮天尘,石龙瞳,象牙焰……这一次是让我采集这三种仙材啊。”黎山仙子的眉头不由地深深皱起。

%34
这三种仙材,石龙瞳、象牙焰都是七转仙材,遮天尘更是八转级数,十分珍贵。

%35
象牙焰,乃是上古荒兽象死后,在象牙内部中静静燃烧的一团火焰。这团火焰只有豆丁大小。一旦烧透象牙,接触到外界的空气,就会熄灭。

%36
上古荒兽大象,可是媲美七转蛊仙的存在。这种大象死后,并非象牙中都会产生象牙焰。

%37
种种条件结合下来,可见象牙焰的珍稀程度。就连五域最大的市场宝黄天中都是难得一见的仙材。

%38
而相比较象牙焰,石龙瞳、遮天尘获取的难度更大。

%39
遮天尘是太古九天中的黄天,才有的独特气象。可是黄天早已经因为人祖十子而崩解,黎山仙子须得首先寻找到一块黄天碎块世界。

%40
而这个碎块世界,必须足够大,道痕足够多,能支撑她进入其中进行探索。

%41
进入黄天碎片世界之中后,黎山仙子还得准备特殊的手段,才好收集的遮天尘。采集的过程中,自然充满了危险。

%42
对于遮天尘,黎山仙子宁愿耗费重金,在宝黄天中收购得到。

%43
而石龙瞳,是指上古荒兽石龙的瞳眸眼珠。

%44
这种石龙,却非是普通的猛兽。

%45
而是石人部族中的墓地里,埋葬的石人尸体石块十分众多,经过天长日久的积累之后,在某一刻,引来晴天霹雳。

%46
霹雳炸毁墓地,从中生出一条石龙。

%47
和石人一样,石龙的全身都是由石头组成。

%48
在太古、远古时代,异人占据大半天下。只有超级势力的石人部落,或者历史极为悠久的大型石人部落,才有石龙的伴生。虽然出了人祖和十子,还有元始仙尊、星宿仙尊,但人人族的人口基数并不多。

%49
石龙是石人部落的守护兽,石人都认为,石龙是本族的祖宗们,先贤们的复生。在异人种族大战时,石龙一度大放光彩,被公认为世间第一战争巨兽。

%50
石龙没有任何的痛觉,防御十分坚厚,勇猛无畏。就算受了伤,若有甘愿奉献自己的石人。会主动投身石龙的创伤之处。石人牺牲自己,用自己的身躯化为石龙的一部分,石龙的伤势会瞬间痊愈。

%51
这套战法,十分变态,让其余种族都极为头疼。

%52
时代不同了。

%53
黎山仙子现在要寻一头石龙。是非常困难的。

%54
因为石人部落已经十分稀少,更多的石人是蛊仙们豢养的奴隶。

%55
就算黎山仙子发现了一头石龙,凭她的战力,要击败一头皮糙肉厚得,在远古时代号称第一战争巨兽的石龙,也是有力未逮。

%56
所以石龙瞳虽然是七转的仙材。但论获取的难度,反而比把八转仙材遮天尘更大。

%57
“雪胡老祖下派给我们诸峰峰主的采集任务,越来越重了。这三种仙材,只有象牙焰稍微容易一点。遮天尘、石龙瞳要成功寻获,实力只是一方面。更多的要靠运气。唉,如果我能够好运,在宝黄天中直接收购就好了。”

%58
黎山仙子深深叹息。

%59
她之前对方寸山小人族的前期投资,已经落实下来,现在开始收益。

%60
加之黑楼兰决心独立自主,蜿蜒拒绝了她的资助。使得黎山仙子最近这一小段时间,开始迅速累积财力。

%61
这种累积速度,是之前的两倍还多。

%62
遮天尘、石龙瞳的获取。太过耗费精力和时间。黎山仙子宁愿消耗财力,去代替时间精力的损耗。

%63
将来自雪胡老祖的信蛊收起,黎山仙子又探查第二只信蛊。

%64
能够从她仙窍中直接出现。只有很少的几个来源。

%65
黎山仙子查看之后,发现原来这只信蛊,来自于方源。

%66
在这只信蛊中,方源许以重利,邀请黎山仙子出手作战,为他斩杀一位北原僵盟中的仙僵。

%67
“方源……”黎山仙子口中沉吟一声。离开床榻,站起身来。

%68
她一边低头思考。一边缓缓踱步。

%69
当她踱步到窗口时,她抬起头看着窗外的雪花飘飞。群峰静默,冷笑一声。

%70
“这个方源,之前伪装身份,化名沙黄,加入僵盟。这次又要我出手,合作斩杀另外一头仙僵,必然有所图谋!”

%71
“许以重利,财大气粗。谁叫他抢夺了东方一族的许多资源,又有东方长凡的智道传承呢。这段时间,他发展的必定是顺风顺水,如火如荼啊。”

%72
“现在的北原局势,暗流汹涌。中洲蛊仙的调查团,还潜藏着。这种情况下,方源仍旧想要动手,可见僵盟那边的利益,对他而言十分重大。是想冒领其他仙僵的贡献,换取僵盟中对如何重获新生,摆脱仙僵身份的研究成果吗?”

%73
想到这里,黎山仙子双眼眯起。

%74
中洲,天庭。

%75
白玉宫殿群,一片光明。

%76
四位蛊仙,分别站在东南西北四个方位,将九转仙蛊屋监天塔围在中央。

%77
老态龙钟的监天塔主,立于东方位置,一顿手中的拐杖,浑浊的老眼中迸射出一缕精芒:“修复宿命,现在开始!”

%78
他手中的拐杖顿在空中,仿佛击在一片铁石上,发出铿锵之音。

%79
瞬间,光芒骤然绽放,无数的蛊虫飞舞,形成一座巨大的蛊阵。

%80
一道道的光痕,五彩斑斓,彼此交接,形成一幅囊括百里之地的巨大立体阵图。

%81
首次参与修复的女蛊仙白沧水,震惊道:“这座炼道蛊阵,再近一步,恐怕就是一座八转的仙蛊屋了!”

%82
仙蛊屋本身就是蛊修流派之一阵道的最高结晶。

%83
“呵呵呵,这座炼道蛊阵可非同凡响呢。沧水仙子且往下看。”西方位,白衣少年模样的炼九生笑了笑。

%84
然后,他凭空一抓,抓出一大把的石龙瞳,抛向前方。

%85
石龙瞳进入炼道蛊阵,只是几个呼吸功夫,就被完全分解,融入到这个奇妙的炼道蛊阵当中。

%86
“这么多的石龙瞳!还有这样的仙材处理速度……”白沧水见之失声。

%87
狐仙福地,荡魂行宫。

%88
“哼!”方源一把将手中的信道蛊虫捏碎。

%89
他被拒绝了。

%90
“还是想限制我的发展么……这群气量狭小的女人。”方源脸色阴沉,“嗯?”

%91
就在这时,一只信道蛊虫飞出现在他的仙窍中。

%92
“难道黎山仙子改主意了?”方源一愣,查看一眼,十分吃惊,“这是琅琊地灵的求援信!?”(天上掉馅饼的好活动,炫酷手机等你拿!关注起點中文网公众号

\end{this_body}

