\newsection{勒索敲诈长恨蛛}    %第一百四十九节:勒索敲诈长恨蛛

\begin{this_body}

方源弹服了荒兽鱼翅狼,便将其收入自家仙窍。

这头鱼翅狼神智已经崩溃,很快就陷入昏迷当中,也不能控制麾下的鱼翅狼群。

方源没有什么好方法,只能任由那些普通的鱼翅狼四下溃逃。

他也不觉得有多可惜,活捉了荒兽鱼翅狼,已经是一笔丰厚的成果了。

接着,他继续四下搜刮,不过遇到的深潭,都是普通资源,虽然量大,却乏善可陈。

东方部族的大本营中,肯定有不少好东西。但深潭千千万万,方源的侦察范围并不广阔,再加上周围还有魔道蛊仙竞争,因而搜刮多与少,还得看运气。

不久后,他运气又来了。

他看到一处深潭。

这深潭中积蓄的,却不是水,而是火。

火焰在熊熊燃烧,这是个独特的火焰深潭。

方源连续换了几种侦察手段后,便发现这深潭中,生活这一群长蛇猛兽。

方源辨别其身份,便发现是幽火龙蟒。

成年的幽火龙蟒,体型庞大,至少长达三十丈,蛇躯堪比塔楼般粗壮。

幽火龙蟒蛇鳞猩红,头上长着一只尖锐的独角,一对通红的血眸。蛇信是诡异的紫色,伸缩间,可以发现蛇信上面缠绕着幽蓝的火焰。

这是一处蛇窝。

虽然没有培养出荒兽级的幽火龙蟒,但龙蟒数量极多。蟒蛇简直堆积在一起。蛇身蠕动,看起来有些渗人。

方源却是看得砰然心动,幽火龙蟒浑身都有价值。譬如蟒血,是用来喂养某些血道蛊虫的最好食料。蟒皮、蟒筋等等,是炼制许多蛊虫的上佳材料。尤其是蛇躯中的幽火蛇胆。十分珍贵,在宝黄天中也有市场。

“幽火龙蟒,向来以小家庭的形式生存。东方部族不知采用了什么方法,让这么多的幽火龙蟒一起生活在一起。这手段虽小,却颇有成效,极易幽火龙蟒的繁衍!”

方源看出了些微门道。

说起来,他曾经在王庭福地时。就得到了几只幽火龙蟒的蛋。如今已经成功孵化。出生的幽火龙蟒生活在狐仙福地里,但因为数量太少,一直被方源忽视。

如果他能得到这么一群幽火龙蟒,大量豢养在狐仙福地中。今后,就能源源不断地得到的幽火蛇胆,放到宝黄天中贩卖,也算是一笔小小的收入。

方源虽然做着胆识蛊的大生意。但此时他迅速估算了一下,贩卖幽火蛇胆的利润,也叫他有些小心动。

心动不如行动,方源雷厉风行,立即飞出力道巨手去抓捕。

他还拿捏不住力道,许多幽火龙蟒都力道巨手捏死了。毕竟这招变化,取的势大力沉的特征,在精微控制方面颇有缺陷。

他连抓几十把,深潭中还有许多幽火龙蟒,便渐渐有些不耐烦。

他缺乏有效手段收取。只能这样蛮干。但如今却是时间有限。方源最后抓了几把,凑足了一万之数,便果断停手,迅速离开了这个火焰深潭。

之后,他一路上搜刮资源,不时地看到许多凡人蛊师。

这些蛊师有的在深潭之间的地段,惊惶逃窜。有的隐形匿迹。躲藏在地面下、山石堆里等等。

方源气势滔天地飞走之后,他们都从惊骇欲绝中捡回一条性命。

有些隐藏的蛊师,还自以为躲避了过去,其实早就被方源发现,只是方源没有刻意对他们动手而已。

之前,皮水寒等人打破福地,和第一波荒兽群激战时,就造成了巨大的轰动。早已经惊动了碧潭福地中的东方一族。

五域中,大多数的凡人,终其一生,都没有机会接触到蛊仙。

对于庞大的人口基数,蛊仙的数量极其稀少,只是一小撮中的一小撮。

同时,蛊仙也有生存压力,终日奔波忙碌,基本上没有时间逍遥自在。

对于大多数人而言,蛊仙只是传说,虚无缥缈。但碧潭福地这里的凡人,却都是东方一族的核心,知道许多秘辛。

因而他们更加明白仙凡之间的差距,宛若天地鸿沟。

他们没有想做任何的无谓抵挡,一边四处分散躲避逃生,另一边则紧急联络东方一族的蛊仙。

不得不说,这是相当明智的举措。

方源这些魔道蛊仙,更看重实际利益,根本懒得对这些凡人动手。因此许多东方一族的凡人,反而比荒兽更幸运,逃过了连绵杀劫,仍旧幸存着。

方源飞过一片片的深潭。

随着时间推移,他心中越加紧迫,于是开始有选择的放弃一些常见资源。

这些资源,他完全可以收购得到,并不珍稀。

同时,方源心中时刻保持着戒备,依照东方长凡老谋深算的性格,他不可能不在碧潭福地中有所布置。

渐飞渐远,方源忽然发现一片无水的深潭。

这些深潭,已经完全干涸。数量很多,大约有五六十座,彼此之间相距也不愿,遍布方圆数十里地。

方源心中一动,身形在空中一折,迅速接近过去。

“这片深潭是我先发现的,里面的长恨蛛理应给我!”

“你胃口太大了点吧?已经收取了这么多长恨蛛了。这片长恨蛛的繁衍地如此广阔,你一个人居然还想独吞?!”

方源飞近,便发现一处深潭边缘,有两位魔道蛊仙站在地面上,正为归属产生争执,气氛凝重紧张。

见到方源飞过来,这两个魔道蛊仙纷纷脸色大变,迅速停下争论,反而并肩站在了一起。

“这位前辈。不好意思,这里的长恨蛛,是我们两兄弟发现了。按照规矩,嘿嘿……”其中一位立即开口道,满脸的戒备紧张的神色。

“我听到风声。西北方向上还有一座深潭,里面养的是大量的气泡鱼群。好几位蛊仙正在争相收取呢。”另一位则企图祸水东引。

方源狞笑一声,状极凶恶:“你们俩的意思,是想让我就这么走了?”

话音刚落,两只力道巨手就飞了出来,悬浮在方源的左右两侧。

一副蠢蠢欲攻的架势。

两位魔道蛊仙见此,脸上顿时变得铁青。同时心中暗暗叫苦不迭。

方源的凶威。他们是在不久前亲眼目睹的,连皮水寒都被压入下风。

他们两个不过是六转垫底,面对如此强势的方源,一旦发生冲突,后果十分不妙。

但身后的这片资源,规模实在有点大。

人为财死鸟为食亡啊!

其中一位魔道蛊仙,终究还是硬着头皮道:“前辈。我们俩首先发现的这处资源,这是不争的事实啊。您和皮水寒大人定下规矩时,我们可都在场的。”

“哼,我魔道向来横行无忌,讲什么规矩?!凡事都讲规矩,你们干嘛不直接加入正道去?再说了,你们算什么?也就是皮水寒,才有资格和我讲些规矩。”方源气势凶狠,两只力道巨手缓缓推进,赤裸裸的耀武扬威。

对面的两位魔道蛊仙被方源直接开口教训。不敢吱声,却又留恋不退。

另一位魔道蛊仙,似乎下定了决心,忽然开口道:“前辈,这些长恨蜘蛛对我而言,却有大用,在下一直苦苦追寻。不如在下用一个等价的消息。来交换如何?”

“哦?什么消息?”方源缓缓停下力道巨手。

“这个消息,是有关于方寸山的。”刚刚出言的魔道蛊仙,直言不讳。

“方寸山,哈哈哈!”方源仰头大笑,复又威胁道,“好,我就跟你换了,但愿你这消息属实,不然的话……”

“属实,一定属实。晚辈就是从那个方向上来的,以晚辈的修为,无法突破,于是知难而退。但对于前辈您而言,根本就是留给前辈您的啊。”说着,这位魔道蛊仙就暗中传音,将具体消息内容告诉到了方源。

方源听完,双眼眯成一条缝,慢条斯理地道:“好,这个消息我暂且听着,是否属实,还待验证。你想把你刚刚收来的长恨蛛,都转给我。”

“前辈,你!”魔道蛊仙顿时瞪大双眼,脸上气得通红。

方源狞笑一声,理所应当地道:“我怎么知道你的这个消息是真是假,万一是假的怎么办?这些长恨蛛就是担保。放心,我只要这些,剩下的都归你们。”

“前辈,您一句话,就要了这里总量两成的长恨蛛啊!”魔道蛊仙咬牙切齿,心中充斥愤怒,却不敢动手。

另外的那位魔道蛊仙,站立于一旁,暗笑不动。

方源若是要了深潭中的长恨蛛,他必定会和正在被敲诈的魔道蛊仙联手,一起对抗方源。但现在方源偏偏要的是收入囊中的那一份,这就不同了。

收入仙窍中的长恨蛛群,基本上就是吃进肚子里的肉,很难吐出来,属于蛊仙个人所有。

袖手旁观的魔道蛊仙,见方源没有侵害到自家的利益,看到身旁蛊仙的遭遇,还有一种幸灾乐祸的快意。

被敲诈的魔道蛊仙,将身旁之人毫无动静,一颗心仿佛沉入冰河,一片寒冷。

吃进肚子里的肉,要吐出来,这是多么痛苦的事情。

“好了,你快快拿出来。我还要赶路,时间紧迫呢。”方源不耐烦地催促道。

“前辈,您要是反悔,那我岂不是……”被敲诈的魔道蛊仙,咬着嘴唇,从牙缝中挤出话来。

“什么?!你居然认为老夫是这样的无耻之徒?老夫年岁这么大,还要骗你们这两个小辈不成?!”方源十分不悦地回答,同时又飞出两只力道巨手。(未完待续。)

\end{this_body}

