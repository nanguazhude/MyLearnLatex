\newsection{手段似小有深意,越是明白越心惊}    %第七十节:手段似小有深意,越是明白越心惊

\begin{this_body}

方源言之凿凿,黑楼兰选择了相信。\$顶\$点\$小说。(23)

二人隐去身形,大摇大摆地走过星辉假眼蛊的范围,果然没有引发蛊仙的注视。

随后,两人又在途中,不断发现类似星辉假眼蛊的侦察蛊虫。

这些蛊虫,有的埋在土中,有的伪装成树叶,有的潜伏在溪水中,随波逐流。

方源一一动用手段,或隐身,或动用蛊虫针对,在或遁地顺水种种,晃过这些侦察蛊虫。

两人原本想走到第八星殿的下面,这样一来,直线距离最短,就最为接近第八星殿。碰到机会,也能迅速反应,用最短的时间冲上第八星殿,做那收拾残局,讨便宜的人。

但走着走着,前方传来一股血腥气味。

方源和黑楼兰对视一眼,后者眼中流露出疑惑的光,传念道:“奇怪,这段路程中的侦察蛊数量、种类,也越来越多了。”

“走上去瞧瞧看。”方源回道。

二人越发小心翼翼,躲避或骗过侦察蛊虫,发现越来越多的战斗痕迹。

片刻之后,他们终于到达血腥气味最浓烈的源头。

只见一头黑豹的尸体,躺倒在碎裂的乱石堆中。

黑豹体型硕大,堪比大象。皮毛油光锃亮,富有野性的魅力。残留的气息,彰显出黑豹的身份――这是一头货真价实的荒兽。

“这是荒兽绝影豹啊。”方源低声叹道。

“是谁斩杀了这头绝影豹,却还把荒兽的尸体遗留在这里?”黑楼兰目光流转。美眸中流露出一丝疑惑之色。

方源蹲下身子,半跪在地上,用手抚摸了一下脚旁的沟壑。

随后。他取出一只宙道蛊虫,小心催使起来。

这只蛊虫乃是五转,十分珍稀,水晶圆球样子。名为回溯蛊,乃是宙道侦察蛊,能回溯一定范围内,过去发生的事情。

回溯蛊催动。展现出一个模糊的,没有声响的战斗画面。

画面中,两个巨大身影在剧烈战斗。旁边还远远站着一个微小身影。

回溯蛊只是五转凡蛊,不管是石磊或者绝影豹、万象星君,都是六转级数,仙体兽体都有道纹道痕。因此侦察的效果十分有限。

黑楼兰凑过来。看了一阵。

无声的战斗画面十分模糊,并不分明。但从大致的体型,可以猜测出其中一个,就是绝影豹。

和绝影豹激战的,不似人形,好像是一头虎类荒兽。

而一直远远站着的,倒是人形,很有可能是位蛊仙。但他的面目一片模糊。根本看不清楚。

不仅如此,战斗画面也是时断时续。并不连贯。

方源、黑楼兰屏气凝神关注,一时间除了山林间的微风,毫无任何声响。

催动了十几个呼吸之后,回溯蛊噗的一声,冒出袅袅焦烟,战斗画面骤然消失。

强行回溯蛊仙级的战斗画面,让这头回溯蛊已然死亡。

方源微微心疼了一下,将回溯蛊的焦尸,又投入自家仙窍,争取不留下任何蛛丝马迹。

这种回溯蛊,哪怕他最近一直在宝黄天中收集,统共只收集了五六十只。这还是五域和平时期,要是乱战中后期,回溯蛊广泛运用,十分紧俏,都是自产自用,宝黄天中根本就没有的卖。

黑楼兰大感兴趣:“你这头蛊虫很妙,叫什么名字,在宝黄天买的么?”

蛊虫种类浩如烟海,黑楼兰又专修力道,不是宙道,只了解一些大致情况,对于其他流派,局限于当中的经典蛊虫。

像回溯蛊这类的蛊虫,属于偏门稀罕之物,她不了解也相当正常。

方源并不答话,而是又取出一蛊。

此蛊却是他自己按照记忆而炼制出来,按照历史进程,提前了三百多年面世,名为线迹蛊。

蛊虫外形独特,仿佛一条小型马鞭,给北原孩童玩的玩具。方源用巨大狰狞的僵尸怪爪拿着,更显得此蛊袖珍。

方源抓着马鞭柄部,对准眼前的空气,轻轻一抽。

啪的一声,原本空无一物的眼前,忽然浮现出丝丝缕缕的痕迹。

“这是什么东西?”黑楼兰美眸立即盯住了痕迹,一眨不眨。

啪啪啪啪。

方源连续抽动马鞭形状的线迹蛊,顿时空中浮现出来的丝线纹路,越来越多。

黑楼兰目睹一切,很快察觉,方源抽动的位置,正是之前战斗画面中激战的地方,除此之外,就是那位模糊蛊仙站立的位置。

脑海中一点灵光闪动,黑楼兰瞳孔微微一扩,脱口而出道:“难道说,这些痕迹就是道痕不成?”

方源这才自得一笑:“不错,正是道痕。不管是蛊仙、荒兽,身上都充斥道痕,他们停留的位置,就会有道痕残留一段时间。只是这些道痕十分微弱,就算是再敏锐的蛊仙也体察不到。除非动用我研发出来的线迹蛊,才能显现出来。我从中观察,就能看到道痕种类,从而推测出他们的一些情报。”

方源再次毫不客气地,将前世他人的创作,无耻地安在自己身上。

“线迹蛊……”黑楼兰口中喃喃,看向方源的目光起了微微变化。

方源伸出手指,指向万象星君曾经站立的地方,续道:“你看那里,道痕显化,形成人形,又是湛蓝颜色。很明显,曾经有一位星道蛊仙,抱臂观战。道痕主体一直停留在那里,周围虽有一些道痕,但都紧紧围绕主体,可见整场战斗中,这个星道蛊仙袖手旁观,并未动手。你再看道痕密集程度,好像是一个六转蛊仙。”

黑楼兰竖起双耳。顺着方源手指的方向看着,听得全神贯注。

方源又指向绝影豹死去的位置,那里残留着黑色的暗道道痕。以及红色的炎道道痕。

他又道:“暗道道痕明显是绝影豹的,红色道痕应该就是杀死绝影豹的存在。至于到底身份如何,是人是兽,还不确定。你看这些道痕密密麻麻,相互纠缠交错,可见当时战斗较为激烈。暗道道痕稀疏,而红色道痕却较为浓密。可以判断杀死绝影豹的存在,也是一个六转级数。”

黑楼兰听得嘴唇都微微张开。

她旋即将目光,投注到方源手中的线迹蛊上。目光微微灼热。

这个线迹蛊,虽然只是五转凡蛊,但太好用了。凭借此蛊,就能侦察出这么多有用的情报。

关键是炼制它的思想。十分独特。另辟蹊径,绕过常规,出人意料。

黑楼兰虽然刚刚晋升成仙,但早已懂得情报的重要性,越发看重方源“发明”的线迹蛊。

若是石磊、万象星君在此,恐怕也要出一身冷汗。

他们此时动用的,都是正统侦察、反侦察手段,线迹蛊恰巧钻了他们反侦察的空白之处。方源的手段。毕竟来源于五域乱战时期,是整个五域的智慧结晶。等若领先于现在蛊仙半个时代之多,当然优势巨大!

啪啪啪。

方源又抽动几下马鞭,更多的道痕显现,但同时又有许多道痕消失。

这些道痕十分虚弱,在线迹蛊的作用下,凝聚显现出来,寿命大大缩短。

道痕相互交错,越来越杂乱无章,方源终于停止抽动手中的马鞭,叹了一口气道:“这线迹蛊效果也就如此,道痕显现只能持续一段时间,需要观察者拥有经验。并且催动仙蛊,也会留下相应道痕。过于激烈的战斗,手段频出,道痕纠缠在一起,十分混乱,根本看不出来。不过好在,这场战斗中双方动用的手段较少,还能看得清一个大概。”

黑楼兰闻言,眼角微微一跳。

单凭区区一只五转凡蛊,就能侦察出这么多有用的情报,这还不满意?

她目光牢牢盯着线迹蛊,直到方源将其收入仙窍。她敏锐地发现,线迹蛊的鞭体原本完好,但用过之后却是十分残破。显然点化道痕,也要付出代价。

黑楼兰暗暗估计,这只线迹蛊起码还可以再用两次。顿时,她在心中对线迹蛊的评价,又拔升一个层次。

一共三次机会!

区区一只五转蛊虫,却能有三次机会,侦察六转级数的存在,获知这么多的情报。

就在黑楼兰差一点忍不住开口,要询问方源关于线迹蛊卖不卖时,方源又取出一只侦察蛊虫。

这只一只气道蛊虫,方源催动起来,令其汲取气息。

随后,方源将这只气道蛊虫放回去,又取出另外一只蛊虫。

他就这样在战场中四处乱逛,不时的动用蛊虫。这些蛊虫,黑楼兰只认识一小部分,还是似是而非,不敢确信的程度。剩余的大部分,黑楼兰干脆两眼一抹黑,根本认不出跟脚。

她下意识地抿紧双唇,首次觉得方源的深不可测。

在北原之行时,她只觉得方源老谋深算,算是生平大敌。和方源合作,她也是不得不这样做,自己十分清楚这是与虎谋皮。

她借力方源,成功渡过天灾,晋升成仙,原本以为可压过方源一头了。但很快,方源又向她和黎山仙子借力,将仙鹤门的蛊仙都唬住。整个过程不战而屈人之兵,令黑楼兰都没有出手的机会。既然没有战损,方源就没有双倍的补偿,等若不费一兵一卒就达到了目的。

反观黑楼兰自己,为了哄得方源出力,将狂蛮魔尊的宝贵真意,都让给了方源。方源也真是狠人,直接一口吞了,连口汤都没给黑楼兰留下。

后来,方源做胆识蛊生意,主动让利给黑楼兰、黎山仙子。虽然过程中,需要黑楼兰的力气仙蛊,但黑楼兰却始终感觉不舒服,仿佛受制于人。

现在黑楼兰看到方源动用的这些小手段,心头不禁凛然。

这些小手段,都是用的凡蛊,看似不起眼,但黑楼兰是个识货的人,哪里不明白这些小手段背后代表的深沉含义!

\end{this_body}

