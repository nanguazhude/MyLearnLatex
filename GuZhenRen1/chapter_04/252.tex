\newsection{仙窍移植,福地搬迁}    %第二百五十三节:仙窍移植,福地搬迁

\begin{this_body}

方源开始布阵。[看本书最新章节请到棉花糖小说网www.mianhuatang.cc]

按照蛊阵图所示,他在星象地灵的带领下,在星象福地中分别寻找到八个阵眼位置,并将对应的仙蛊,一一布置下来。

星象福地的宇道道痕比较缺乏,地灵没有瞬移之能。八个阵眼位置又稀奇古怪,或远或近,导致方源耗费在这一块的时间不少,完成的时候,已经过去了大半天。

完成了这一步骤,心思缜密,心情谨慎的方源,又将这八个阵眼,好好检查了一遍。

确认无误之后,方源这才回到原来地方,将那头挑选好的刺脊星龙鱼,收入自家的仙窍。

在他的命令下,星象地灵打开福地门户。

方源走出去,来到地渊。

福地门户散发出的光,只能照耀周围一圈。地渊里十分昏暗,一片寂静。

这是地渊的深处,就算是拥有者中洲十大古派之一的古魂门,也还未探索到这里。

不过这里虽然没有人烟,但隐藏着无数的荒兽,甚至上古荒兽,蕴含许多凶险。

事不宜迟,方源将刺脊星龙鱼放出来。

刺脊星龙鱼有正常鲸鱼大小,却形似鲤鱼,浑身上下包裹着一层汪蓝的鱼鳞。在它的背脊处,有骨刺长出体外,高高隆起** ,然后长长地延伸出去。

方源按照小阵图,将四只仙蛊分别安放在这头刺脊星龙鱼的身体各处。

一处在鱼嘴中,一处在背上的某根骨刺里,一处在左边的鱼眼里。最后一处则在鱼腹深处。

刺脊星龙鱼虽大,但和星象福地是不能比的。

因此方源这次布置。只是花费片刻的时间,就彻底成功了。

又细心检查了一遍后。方源彻底确认:整整十二只仙蛊都已经布置妥当。

他开始向周围洒下无数凡蛊,甚至飞出一两只仙蛊,布成一套强大的蛊阵。

接下来,终于到了搬迁福地过程中最关键的部分。

方源严格地按照方法,陆续催起手中仅剩下来的四只仙蛊。

星象福地的门户倏地紧紧闭合,消失在空气中。

方源全身上下,开始绽放出淡黄色的光辉。

光辉虽然明亮,但并非那种刺眼逼人的那种,反而很是温和。

光辉整整持续了十七个呼吸。然后缓缓收敛,最终全都凝聚在方源的两个瞳孔里。<strong>小说txt下载HtTp://Www.80txt.Com/</strong>

方源深呼吸一口气,心念一动。

下一刻,一道微弱的黄色光柱,从他的右眼中射出来,旋即射中庞巨如鲸的刺脊星龙鱼。

刺脊星龙鱼顿时剧烈一颤,身上布置妥当的小型仙蛊阵,立即启动,爆发出一股红色光晕。

刺脊星龙鱼在这层红光的笼罩下。仿佛穿戴上一层贴身的红光铠甲。

整个刺脊星龙鱼僵硬如石,一动不动,宛若雕塑。

方源狠狠喘息了两声,随后从他的左眼中再射出一道黄色光柱。正中星象福地的隐藏点。

在光柱牵引下,星象福地居然被摄取出来,形成一个蓝色光团。

蓝色光团只有蚕豆大小。和庞大似鲸的刺脊星龙鱼,形成了鲜明的对比。

摄取出星象福地的这一刻起。方源仙窍中的青提仙元,就开始以一种恐怖的速度剧烈消耗起来!

方源左眼艰难转动。带动黄色光柱。黄色光柱牵引着蓝色光团,开始了缓缓移动。

而那头宛若石像的刺脊星龙鱼,也在方源的右眼移动下,向福地慢慢接近过去。

轰隆隆,地动山摇。

吼吼吼,黑暗中无数猛兽惊惶怒吼。

福地的转移,自然引起了天地二气的强烈动荡。一场强烈的地震,已经在所难免,开始在深渊里爆发。

方源不管这些细枝末节,全神贯注地集中在刺脊星龙鱼和星象福地两者上面。

头顶上砸下来巨大的石块,也被他早就布置的手段防住。

就算有荒兽袭击,方源自信也能争取一段时间!

他营造出了一个短暂的,不受干扰的环境。

像是搬动两座大山,方源转动眼眶中的两个瞳仁。此时此刻,不管是他的肉身,还是他的魂魄,都在拼尽全力。

时间实际上并没有耗费多少,两者终于接触。

但在方源看来,却漫长得如同过了一年!

当星象福地和刺脊星龙鱼,接触到的那一刻,方源陡然间轻松下来,仿佛万斤重担在一刹那间完全卸下。

方源精心布置出来的大小两套仙蛊阵,在这一时刻,起到了关键性的作用。

星象福地仿佛是游子回家,乳燕归巢,安置在了刺脊星龙鱼的身上,化成了刺脊星龙鱼的仙窍!

这就是琅琊地灵用来搬迁福地的方法。

必须借助荒兽,或者上古荒兽,将福地当做仙窍,移植到(上古)荒兽的体内。

然后(上古)荒兽移动,便能带着仙窍四处转移。

但在这个转移的过程中,时间是很有限的。承载仙窍福地的(上古)荒兽,再不能被收入其他仙窍,也不能动用任何其他的仙蛊,否则都会影响它身上的仙蛊阵的运转。

具体情况就是:方源不能再把这头刺脊星龙鱼收入自己的仙窍,带着它迅速转移。

同时也不能动用其他的仙蛊,或者仙道杀招帮助它。

只能靠刺脊星龙鱼自身的肉体力量,悬浮游动,转移到其他地方。

毕竟仙窍福地虽然寄存在刺脊星龙鱼的体内,但只是靠着大小两个仙蛊阵的运转,强行粘合的。外部稍有影响,就会导致蛊阵破碎,不仅转移福地会立即失败,而且当中的仙蛊也会遭到破坏。

不过即便有如此巨大的弊端。也无法掩盖这个方法的巨大利用价值。

创造这个搬迁方法的人,并非是长毛老祖。但在历史评价上,却和长毛老祖齐名。

他的名字就叫做空绝老仙。

蛊仙历史长河中。仅有的三位炼道大宗师之一,远古时代的传奇。

他对空窍、仙窍,有着超乎时代的研究和认知。正是因为他的努力,才使得十绝体可以升仙,成为十绝蛊仙,成就上等福地。

这套仙道杀招,被空绝老仙命名为仙窍移植大法。

这是方源至今为止,所见到的涉及核心仙蛊最多的仙道杀招。

然而这套多达十六只核心仙蛊的仙道杀招,只是一个残招。空绝老仙最初的设想。是真的将蛊仙的仙窍福地,完美地移植到荒兽或者上古荒兽的身上。而不是现在这种样子只靠着大小蛊阵,仙元的剧烈消耗,将两者暂时地强行粘合在一起。

长毛老祖得到空绝老仙的这部分传承之后,在空绝老仙的成果基础上,也曾经做过许多尝试,但统统失败。

在墨人智道蛊仙一言仙的提点下,长毛老祖发现这个残招本身,已经具备惊人的价值!

蛊仙们完全可以用它。来搬迁那些陨落蛊仙遗留下来的仙窍福地。

时光荏苒,沧海桑田。

无数年过去,现在轮到方源在这上面受益了。

“走,去那边。”方源指挥刺脊星龙鱼开始移动。

因为仙窍移植大法的弊端。刺脊星龙鱼的速度不足原先的一半,鱼身上压力巨大,仿佛负山而行。

这就导致了。每一次福地的搬迁转移,其实距离都很有限。

琅琊地灵就算搬走了琅琊福地。但肯定距离之前的地点不远。

这也是方源不搬迁狐仙福地的原因之一。

“不过好在地渊已经是极好的隐藏地点,将来这层地渊曝光。进行大开发。我也可以将这份星象福地,当做最可靠的前线营地,进行最及时的补给和休养。”

“而且我知道仙窍移植大法的弊端,但别人不知道啊。黑楼兰发现星象福地不见,恐怕还会以为福地已经毁灭了。毕竟这种福地搬迁,已经超出了蛊仙认知的常识!”

就这样,方源带着刺脊星龙鱼,往地渊深处走去。

他行走的路线,自然是经过前期的大量探测,精心绘制出来的路线图。

他继续深入,规避了许多荒兽。在这些荒兽、上古荒兽划分好的领地之间的狭缝里,方源带着刺脊星龙鱼缓缓游走。

方源的目的地,是在星象福地原来地址,更下三层的地渊某处。

但下了两层之后,方源运气糟糕地碰到了一头荒兽独眼巨猴。

这头独眼猴子十分活泼,常常在地渊深处四下游荡。发现了刺脊星龙鱼之后,它兴奋地追了上来。

方源值得和这头独眼巨猴,展开激战,吃力地维持着刺脊星龙鱼的安危。

但最终,方源吃亏在没有战场杀招,无法有效地阻止独眼巨猴的行动,终究还是在半途中,让独眼巨猴重创了刺脊星龙鱼。

承载着仙窍福地的刺脊星龙鱼,是非常脆弱的,为了游动,它已经拼尽全力,因此一点还手的余力都没有。

方源终于赶跑了这头该死的独眼巨猴,但刺脊星龙鱼伤势过重,已经无法转移。

“幸好仙蛊都没有大碍。否则有仙蛊损失,我还不得赔死?!”方源检查之后,沉重的心情终于舒缓下来。

这是不幸中的万幸。

既然事情已经变成了这样,方源就只好退而求其次,将这个星象福地安置在附近了。

在仙窍福地脱离刺脊星龙鱼的那一刻,后者便彻底崩解,化为一滩烂肉血水,就连兽魂都烟消云散。

这是仙窍移植大法的必要牺牲,方源早已经知晓施展此法的代价。

就算刺脊星龙鱼不受伤,达到目的地后,它仍旧会力竭而亡。

最终,星象福地成功搬迁,隐藏在全新的位置地点。

这一切都是值得的!(未完待续……)<!--80txt.com-ouoou-->

\end{this_body}

