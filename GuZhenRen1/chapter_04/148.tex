\newsection{弹弹弹,弹服鱼翅狼}    %第一百四十八节:弹弹弹,弹服鱼翅狼

\begin{this_body}

黑楼兰神情愈加振奋,连忙回道:“小姨妈放心,我已经深入水潭,在地下打洞了!”

高空中,黎山仙子陡然吐出一口鲜血,这是盟约的反噬,但还在她承受范围之内。

她随手擦去血渍,继续向黑楼兰秘密传去消息:“好,我就在高空为你侦察,你悄悄潜过去,注意不要叫其他蛊仙发现了。尤其是方源那个家伙,他城府极深,工于心计,就连皮水寒都被他玩弄于鼓掌之间。我们和他定下雪山盟约,有效时限已经在逼近,你要多多防备他。”

“那是当然!”黑楼兰迅速回道。

方源轰然降临在碧潭的上空,身躯带动大风,吹得潭水表面掀起阵阵波澜。

方源有前世记忆,虽然没有侦察仙蛊,但凡道侦察手段比周围蛊仙,先进了一个时代。他眼眸扫视一番后,立即查明这深潭中,除去潭水,别无他物。

“原来是油水。”他微不可察地皱起眉头。

油水自然不是寻常的水,而是大海深处才有的资源。作为炼蛊的材料,在东海十分普遍。放到北原、西漠这些地方,就比较珍稀了。

小份额的油水买卖,在宝黄天中几乎找不到。那是因为利润太低,宝黄天抽取的费用多,完全得不偿失。;

若是宝黄天中有这等买卖,基本上就是超大份额,但也只是赚辛苦钱。又因为东海每年出产的油水,基本上都自我消化掉,因此这种超大份额的交易。次数也极为稀少。

东方部族在这里不仅是存储油水,而且用意巧妙。这口潭水极为深邃,勾连地底深处。引出缕缕地气。

油水本身就是深海中,大地沟中产生的资源。

“东方部族建立此潭,就是利用油水产生油水。每隔一段时间,抽取清水兑进去,利用地气和油水不断交融,渐渐恢复油水原本的浓度。”方源见多识广,脑海中思绪一转,便知晓了东方部族的运作手法。

油水可不是好捞的,需要针对的蛊虫。

方源微微皱眉。就是因为如此。

他还知道:东海中,就有一小撮特定的优秀蛊师,专门深入海底去捞油水,世代传承的蛊虫秘方,相互搭配起来,可以打捞、汲取这种油水出来。

这些人就是以此为生,蛊虫搭配的重点,就是打捞油水,战力不强。但社会地位较高,受到东海大小势力的普遍欢迎。

方源前世也去过东海,那时候接触过一位专门捞油水的老蛊师。

老蛊师年岁大了,膝下只有一位孙女。想招方源为孙女婿,入赘进去。可惜最终,被方源拒绝了。

很快。方源的眉头舒展开来。

他伸出食指,指向潭水表面。手指轻轻一勾,顿时潭水如龙般升腾而起。乖乖地灌入方源的仙窍之中去。

换做拍卖大会之前,方源一定无法可想,但拍卖大会之后,他有了力道仙蛊挽澜,专门针对水势,区区油水,还不是手到擒来?

“只是油水虽然珍稀,却不是仙材。动用仙蛊挽澜来收取它,完全是杀鸡用牛刀了。若非这潭油水份额较多,我之后计划,还要去参加中洲炼蛊大会,前期正适合用到油水。否则绝不会收取的。”

方源催动仙蛊,需要耗费青提仙元。青提仙元的价值,要比油水贵重多了。

若不是因为炼蛊大会,方源恐怕会一走了之。

此时,方源背负双手,凭空站立,油水自行不断地灌输到他的仙窍之中。

他微微侧身,转头看了看黑楼兰的方向。

黑楼兰那边毫无动静,似乎已经赶去另一处深潭了。

方源眼中阴芒一闪,他对黎山仙子、黑楼兰的把戏隐有猜测,但并不着急。

掌握重生优势,他对茅草屋、方寸山又岂会不知?事实上,他知道的必黎山仙子还要更加清楚。

前世五域大战时,各种仙蛊屋相互交锋,屡屡掀起大型战争的高潮。

仙道杀招容易被针对,被克制,但仙蛊屋向来攻防一体,即便有着不同的擅长之处,其他方面却基本上没有短板。

尤其是仙蛊屋中,载着越多的蛊仙,威力就越强。在战场上横冲直撞,几乎是无可阻挡的战争堡垒。

“茅草屋若能轻易被找到,那就不是茅草屋了。”方源将脚下的油水收取一空,身影拔升,向另一处深潭飞去。

陆续收了几处类似油水的资源后,方源终于碰到了一个深潭,里面栖息的兽群,叫他眼前一亮。

感受到方源强势入侵了自家地盘,狼嚎声四起,数万头鱼翅狼从潭水中奔驰而出,结成大阵,对半空中的方源虎视眈眈。

这些鱼翅狼,方源在伪装成狼王常山阴时,也用过一些。

十分好用。

它们体型有象般大小,水陆两栖,是为数不多的能够在水下战斗的狼群。

它们的防御力极强,一身光滑的鳄鱼皮甲,身体两侧,生长着尖锐的深蓝鱼鳍。背上也有一排类似鲨鱼背部的鱼翅,这些鱼翅连成一线,从狼头一直延伸到狼尾。

速度虽然不快,但常常被方源当做狼阵中的坚实盾牌。

当然,如今这些凡兽,已经勾不起方源多大的兴趣。

方源目光灼灼,落在狼群中的首领身上。

这是一头巨大的鱼翅狼,高达三丈有余,体型线条极为流畅,力量和美的完美结合。

此时,这头鱼翅狼低伏着身体,狼口咬合,边缘张开,露出白森森的尖锐牙齿。硕大如灯笼的狼眼,紧紧地盯着方源,仿佛下一刻。就会纵身猛跳,将方源扑食。

方源哈哈一笑。不由起了收服之心。

他虽然没有奴兽仙蛊,但通过其他方法。也可稍稍控制这等荒兽。

驭兽仙蛊只有一只,存放在琅琊福地当中。使用此蛊奴役荒兽,见效一瞬之间,荒兽进退如意,心随意动,这是奴道的无上杰作。

方源没有奴兽仙蛊,其他奴道蛊仙自然也没有。但蛊仙用蛊,妙用存乎于一心之中,自然有各种方法、杀招。部分取代奴兽仙蛊的作用。

比方说飞熊虚像蛊,就是虚道奴役荒兽的独特手段。其他水道、火道、光道、智道等等,亦有不同方法。否则太丘之战,东方长凡怎可能操纵得了那么多的虚兽?

各种流派相互独立,但又互通。流派之间相互借鉴、相互交流、相互竞争,才能有如今百花齐放的蛊师格局。

就算没有这些手段,一只荒兽从小养到大,养出了感情,只有亲近之心。可以护身,可以放出杀敌。仙鹤门中鹤风扬的九宫鹤小九,就是人兽交心的典型例子。

当然,这些手段绝比不上奴兽仙蛊的效果。

奴道。既然形成流派,奴道奴役兽群的手段,自然要优秀于其他流派的。

深潭边上。双方对峙了片刻,荒兽鱼翅狼仍旧按捺不动。方源身上的气息。让它如临大敌,不敢轻举妄动。

既然如此。方源便主动出击,两只力道巨手一左一右,分别压上。

荒兽鱼翅狼大吼一声,微微后退一步,身上顿时显出一层琉璃水甲。

下一刻,它猛地一跳,直接跳到高空,还要超过方源的高度。

它张开血盆大口,方源就感到一股腥风扑面而来。

伴随着腥风,还有无数的水弹,从荒兽鱼翅狼的口中急速喷吐。

方源不闪不避,又飞起两只力道巨手直接迎上。

水弹暴雨一般打在力道巨手上,却只是起了一丝延缓的作用。这些水弹连凡道杀招都称不上,自然更比不上仙道杀招了。

与此同时,刚刚飞出去的两只力道大手,从底下抄上来,分别抓住荒兽鱼翅狼的两只后腿,猛地拽下。

可怜的荒兽鱼翅狼才刚刚跳到高空,距离目标方源还有一段距离,就被拽了下去。

轰隆一声巨响,荒兽鱼翅狼和地面亲密接触,一些倒霉的普通鱼翅狼躲闪不及,被它们的首领压成一团团血糊糊的肉泥骨渣。

荒兽鱼翅狼大怒,倒在地上,挣扎欲起,还想扑腾,但下一刻,它就看到一只力道巨手向它逼近,越变越大。

力道巨手威势凛然,还未拍中鱼翅狼,后者就感到呼吸压抑,心房震颤。

鱼翅狼想躲,但后腿被可恶的两只力道巨手死死抓住,它身躯一扭,但仍旧躲闪不了,被力道巨手狠狠地拍中脸面。

荒兽鱼翅狼惨嚎一声,狼首被狠狠地拍进地面,直接陷进去,顷刻营造出一个大坑。

好几个呼吸之后,它这才勉强抬起头来,脱离地面。

但这一下重击让它视野模糊,涕泪并流。

它摇晃头颅,还想抗争,这时第二只力道巨手再次拍上。

荒兽鱼翅狼再度惨嚎,地面又被砸出一个大坑。荒兽鱼翅狼心中十分悲愤,敌人强大又十分无耻,这根本不给它反映时间啊。

方源又飞出力道巨手,总共五只,齐齐飞舞,将荒兽鱼翅狼摁在地上,狠狠地揍,打得烟尘四起,地面迸裂。

荒兽鱼翅狼身边的狼群,在四处乱窜,悲愤哀嚎,有的喷出水弹,数量挺多,但很多都在半途相互碰撞,就相互消损了。方源个体不大,目标小,落到他身上的水弹就更少了,完全是毛毛雨,可以忽略不计。

“这是什么声音?”不远处,半月蛮师听到动静,目光远眺过来。

待他看清方源正在虐打荒兽鱼翅狼,他的瞳孔立即一缩。

他虽然已经知晓方源有着七转战力,但当他亲眼看到方源抱臂在胸,云淡风轻地暴揍一头荒兽时,他的心脏仍旧不禁狠狠一震。

根本没有什么悬念,荒兽鱼翅狼毫无还手之力。

若是方源只是六转战力,兴许还会胶着一些。但跨越了一层大境界。七转层次的战力,让荒兽鱼翅狼只能被动挨打。

尤其是。这种福地中豢养的荒兽,基本上不会身怀野生仙蛊的。

有仙蛊的。基本上都被蛊仙第一时间收走了。仙蛊多珍贵了,怎么可能留给一头荒兽?

不过,鱼翅狼被方源狠揍,根本抬不起头来,但仍旧态度很顽强。

方源冷哼一声,决定加重力道。

他心念一动,四只力道巨手分别抓住鱼翅狼的前肢、后肢,然后直接将它提到了半空中。

荒兽鱼翅狼被揍得惨不忍睹,整个身体呈大字型。四条腿被力道巨手狠狠抓住,不断地发出腿骨碎裂的声音。

余下的一只力道巨手,呈现巴掌状,照着荒兽鱼翅狼的狼脸左右猛扇。

啪啪啪!

硕大的狼头在巴掌的循环重击下,迅速左右甩动。

底下的狼群哀嚎声顿时放大一倍,大部分的狼群已经崩溃,边缘处已有狼群在溃逃。

“这是名副其实的吊打啊!”半月蛮师看到这一幕,锃光瓦亮的额头,都渗出冷汗来。

方源凶威之盛。叫人不得不心惊胆战。

荒兽鱼翅狼被打得几乎成了一滩软泥,哀嚎得声音都变了,但它勉强睁开的眼缝中,流露出的目光仍旧十分顽强。透着深入骨髓的仇恨。

狼本来就是一种特别记仇的猛兽。

“还不服软?”方源也感到有些头疼,服从强者,是野兽的生存本能。按照常理而言。如此巨大的实力差距,早就应该让荒兽鱼翅狼认怂了。

方源仔细观察。又有了发现。

“原来如此,这鱼翅狼到了发情的时候了。正是雄性气息最为浓烈的时刻,难怪态度强硬。”

他心念转动,忽然有了一个古怪的主意。

在方源的操纵下,力道巨手从巴掌状,发生改变。

大手缓缓屈起中指,和大拇指合并。

然后,半月蛮师就遥遥看到,这只力道巨手慢慢下沉,沉到荒兽鱼翅狼的两腿中间。

忽然,大拇指微微放松,中指猛地弹出。

准确地命中这头雄狼的命根!

“嗷呜!”荒兽鱼翅狼如遭电击,猛地一震,眼珠子差点瞪掉下来,仰头长吼,声音中充满了猝不及防的无边痛楚。

弹弹弹!

“嗷、嗷、嗷呜!”鱼翅狼引吭高歌,音调一声高过一声。

远处,半月蛮师张开大口,目瞪口呆地看着这惊人的一幕。

“这、这、这……”他结结巴巴,说不出话来。

虽然他明白,这是方源在压服荒兽鱼翅狼,但此种方法,简直闻所未闻,见所未见!

刹那间,方源的形象,在半月蛮师的心中发生了巨大的改变。

方源一身黑袍,看不出本来面貌。但在外人心中,自然有推测的形象。

原本在半月蛮师心中,方源是狂猛凶狠的壮汉形象,虎背熊腰,双眼凸出,凶光四射,直欲择人而噬。

现在,这个形象产生过来变化。

壮汉形象已经没有了,虎背熊腰顶多算是狼背蜂腰,眼中目光阴狠,白牙森森,透露出一丝变态疯狂的气质。

荒兽鱼翅狼终于被击溃心理防线,彻底认怂了。

力道大手一松,它被砸在地上,一动不动。它那庞大健美的身躯,此时弓成一个大虾似的,两只前爪捂在两只后腿之间,发出呜呜地低泣声,孱弱得仿佛被风吹雨打过的的柔弱小花。

方源呵呵一笑,自语道:“这倒是个好方法。”

又转过头,看向远处的半月蛮师。

他当然早就发现了后者的窥视。

半月蛮师自然看不到方源的真面目,但却感到方源的目光犹如实质,仿佛是毒蛇巨蟒一般,缠绕在他的身上。阴凉滑腻的蛇鳞,紧紧贴在他火热的雄躯上。

半月蛮师从内心深处生出一股恶寒,雄躯一颤。尤其是自家胯下,还感到丝丝凉意。

“这人好生凶残,绝不是我能对付的,今后看到他,一定要退避三舍!唉,世道多艰,生活不易啊。”半月蛮师咽了一下口水,身形一窜,迅速远去,竟有些落荒而逃的意味。(未完待续……)

\end{this_body}

