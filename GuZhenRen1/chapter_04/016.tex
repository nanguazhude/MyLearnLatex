\newsection{三张十成仙蛊方!}    %第十六节:三张十成仙蛊方!

\begin{this_body}

“琅琊地灵,你这样做,未免有些不妥吧?”墨人王看了一眼太白云生,语气担忧…对方可是祸害了王庭福地,推翻了八十八角真阳楼的凶人!一旦惹恼了对方,墨人王很担心方白二人会对琅琊福地不利。

琅琊地灵极其率真,没有听出墨人王的话外之音,反而得意洋洋地道:“哼!那个臭小子狡诈奸猾,敢对我老人家这样的恶劣态度,我就是故意的!我要好好整治一下他,让他懂得尊敬老人的道理。哈哈哈,他能推算出什么蛊方来?他有什么底蕴?明明是力道仙僵,当我老人家看不出来吗?哼,小太白是对他盲目信任。他要是能推出仙蛊方来,我老人家直接把我捣药的药舂给砸碎了吃掉!”

太白云生一脸悲悯地看着嚣张大笑的琅琊地灵,没有出声。他其实很想说:自己不是对方源有信心,而是对智慧蛊有信心。

……

方源静静站在智慧光晕之中。

片刻之后,他睁开双眼,眼中闪烁出喜悦之光。

“成功了。”他感慨一声,刚刚的片刻功夫,便让他将第一件残方推算成功,完善到十成。再看这道蛊方时,方源也惊叹:“这最后一步,堪称神乎其神,竟然是自己完善的蛊方。”

他自己都有点难以置信。

推演蛊方,一是底蕴,二是灵感。炼道底蕴不足,推算蛊方会十分困难。但底蕴深厚的蛊师,推算蛊方时,也往往会遇到难题关卡。就这也被卡住,一步不得寸进。常常数年。甚至数十年,上百年的例子都有。

这个时候,要突破难关,就需要灵感。

但灵感难得至极,很可能连续出现,也很可能许多年都不见得冒出一个来。

就算有灵感,那只是一个方法。需要验证。

如果成功,那就是真正的灵感。如果不成功,那就只是一个想法。一个失败的尝试而已。

“智慧蛊的威能,就在于能令蛊师灵感无限。我本身就是炼道大师。底蕴算是够了,就差灵感。”方源再次充分地感受到九转蛊的强大。

这还只是他蹭用智慧光晕,若是有朝一日,真正地彻底催动智慧蛊,不知道是什么景象。脑海中,还有大量的乐意残余。

刚刚的蛊方完善度极高。就差临门一脚。因此。方源并未消耗什么乐意。

他沉下心来,再度开始推算蛊方。

第二件蛊方。是一个九成七的仙蛊方。

这一次,方源消耗的乐意也是不多,只消耗了三分之一,卡住仙蛊方的难关就算过了。

这道难关一过,就解决了最大的麻烦,接下来一马平川,思路清晰,一目了然。就算方源单靠自己的能力,都能将其推算完成。

但是到了第三件仙蛊方,方源遇到了些微麻烦。

他一连消耗了三颗青提仙元,这才将这道仙蛊方完善。

这道仙蛊方难题颇多,解决了一道,又来一道。因此虽然完善度也极高,高达九成六,但为了推算它而消耗的青提仙元,足足是之前的两三倍。

但即便如此,按照协议来算,方源也是大赚特赚!

因为急着用,方源沟通宝黄天,通过神念蛊,联系上琅琊地灵。

琅琊地灵哈哈大笑,还以为方源来认怂。所以当方源交给他一道十成的仙蛊方时,他惊愕在当场,一时间反应不过来。

几个呼吸之后,他的全部注意力已经被仙蛊方中的内容所吸引。

“妙啊,秒啊!竟然是这样方法,攻破难关。我怎么就没有想到,我怎么就没有想到!”他大叫起来。若非双手被气道封印绑着,此时此刻他估计已经在猛拍大腿了。

赞叹声忽然止住,琅琊地灵的脸上浮现出尴尬的神色。

他没有忘记,就在不久前,他还极不看好方源的能力。没想到没过多久,方源就将其中一份仙蛊方推算成功,交给了他。

琅琊地灵抹不开脸面,哼哼唧唧道:“臭小子,想不到你思维怪诞,居然也能瞎猫碰到死耗子,将这仙蛊方给蒙出来了。这次算你走运好了,按照之前的约定,完善一份九成以上的六转仙蛊残方,报酬十块仙元石!我是用宝黄天送给你,还是你亲自过来取?”

宝黄天交易,当然需要一定的手续费用。但方源亲自去取的话,就要动用定仙游。

每次催动定仙游,都得消耗一颗青提仙元,代价也不菲。

方源想了想,利用宝黄天虽然代价更低廉,但却难免留下痕迹,保不齐被某个有恶意的智道蛊师推算。

但亲自去取,代价也太高,方源有些心疼。

琅琊福地在北原,狐仙福地在中洲,两者不在同一地域,利用洞地蛊是不成的。唯有利用星门蛊。

“地灵,依你的脾气,你那里应该有星门蛊吧?”方源问道。

“嘿,臭小子你挺了解我的嘛。”琅琊地灵坦言承认。

他从方源手中得到了星门蛊的蛊方,炼制出的第一套星门蛊给了方源,自己后来又炼了第二套收藏。

方源点点头,传念道:“宝黄天并不安全,我们的交易会被很多人发觉,次数多了,就会引来大麻烦。定仙游虽然方便,但是代价太高。不如我们两大福地沟通起来,建立星门。”

“建星门?”琅琊地灵连连摇头,没有多想就否决了方源的提议,“臭小子,我被你坑得够多了。建立星门非同小可,你那边遭殃,我这边也会跟着遭殃。万一琅琊福地被人发现,我老人家就没有好日子过了!”

方源哈哈一笑,没有再劝,而是再传给琅琊地灵第二张仙蛊方。

“你竟然又完成了一道?!”琅琊地灵大吃一惊。

这样的速度,未免太快了吧?

“这样的解决之道,简直是奇思妙想!你居然想到用这个蛊,还有知心草这样的炼蛊材料,你怎么想到的呀?”琅琊地灵喋喋不休,品味到这张十成仙蛊方的妙处,惊叹不已。

“哈哈哈,老头子,长江后浪推前浪,一代新人换旧人!你已经老了,整天呆在琅琊福地里面,不知道外面的发展。就算你有个至交好友墨人王,但他是墨人,怎么能接触到我们人族修行的精华?再说了,我是谁?!我是智道的绝世天才!总有一天,我会超越星宿仙尊。”方源嚣张长笑,大放厥词。

琅琊地灵哑然无语。

他很想反驳,但事实就摆在眼前。曾经困扰他无数年的两道仙蛊残方,就这样轻易地被方源补齐了。

方源忽然叹息一声,道:“唉,你知道天才的寂寞吗?从一生下来,我就知道自己的不凡,周围没有人能和我平等交流。甚至有人认为我是一个怪物,他们恐惧我的智慧,嫉妒我的天份!我只好隐藏自己,显得和正常人一样,以至于其他人再也看不出我身上都快要满溢而出的惊天才华。智道就好像是天生为我设计的流派,但我阴差阳错却成了力道仙僵。你说,这是不是命途多舛,我的才华已经惹来了天妒!……

说到这里,琅琊地灵终于忍受不住,跳脚道:“喂,臭小子,你不要这么臭屁啊!推算两道仙蛊方,有什么了不起的!有种的你将最后那道仙蛊方也推算成功,我老人家才会佩服你。”

方源沉默。

琅琊地灵昂起头颅:“哼,小子,我老人家承认你的确有点天资。但那又怎样?这道仙蛊方,连我本体都被难住。你先好好琢磨个七八年,兴许就知道天地之大,群星璀璨。你就会懂得谦卑了,啊哈哈哈……”

“你说的是这张仙蛊方吗?”方源嘴角一咧,抛出第三张仙蛊方。

这简直是杀手锏!

琅琊地灵的大笑声戛然而止,旋即震惊地大叫起来:“这,这是什么?搞什么,你搞什么?!”

声音中还夹杂着一丝恐慌的情绪。

方源语气悠悠,长长一叹,悲郁地道:“我说过,我的天份常常惹来恐惧和嫉妒。唉,没办法,我就是这样的天才啊,生来如此……”

琅琊地灵这一惊,非同小可。

他毕竟只是一个执念所化,没有城府。

这最后一道仙蛊方,带给他的冲击力实在太大了。

“你居然真的推算成功了?难以置信!难以置信!这道仙蛊方居然是这样解的,这思路简直是硬生生地掉了个头啊。怪道连我本体都想不到啊……真正应了一句老话:峰回路转,柳暗花明!峰回路转,柳暗花明!!”

琅琊地灵喃喃不休,他先是彻底的震惊,随后陷入狂喜之中,最后他终于平静下来,心中弥漫着一股挫折感。

这张仙蛊方一直困扰着他,甚至就连他的本体,也都被难住,无法踏出最后的关键一步。

有时候,琅琊地灵甚至以为,这道仙蛊方已经彻底地走入了死胡同里,根本就没有解决之道。

他先前抛给方源,存心想刁难方源,从未想过得到真解。

但他现在得到了。

他没有做到的事情,就连本体都没有解开的难题,方源推算出来了!

这由不得他心中不生出挫败颓丧之情。)

\end{this_body}

