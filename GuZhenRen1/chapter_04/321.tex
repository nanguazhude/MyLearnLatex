\newsection{再重生鬼脸红莲!}    %第三百二十二节:再重生鬼脸红莲!

\begin{this_body}

传说中,这个世界上有这么一条河流。

它贯穿始和终,深藏因和果,流淌全世界的每一个角落,无处不在。

有了它的浇灌,世界才得以正常的运转,一切方能变化。

这条河,就叫做光阴长河。

这里是天地秘境,宙道的温床,无数的宙道蛊虫,在这里繁衍生息。

亿万万顷的河水,恣意流淌,从不间断。

河水澎湃浩荡,潮起潮落,波涛翻滚。

每一滴的光阴之水,都是苍白无色。但亿兆兆的光阴水滴,相互碰撞、交融,却能迸发出世间最灿烂炫目的流光溢彩。

这种光彩的多姿美妙,难以用语言来描述。方源一直都觉得,这是世间最动人的壮美景色之一。

当九转仙蛊屋监天塔,迸发出蓄势已久的一击,方源被殃及池鱼,难以逃脱,避无可避。

他仅剩下的唯一选择,就是催动春秋蝉。

然而,六转的春秋蝉有着巨大的弊端。每一次催动都有失败的可能!

在青茅山上,方源成功催动过一次。

在三叉山上,方源再次成功催动了一次。

如果算上五百年前世,方源催动春秋蝉,已经成功了三次。

如今,他终于碰到了失败。

催动春秋蝉失败了。

他的意志虽然进入了光阴长河,但就在启程的那一刻。春秋蝉自爆开来,化为无数的碎片。

方源的意志,仿佛就是失去小舟,而落水的婴孩。

波涛翻滚的光阴长河,能在瞬间将方源的这股意志。吞噬毁灭,彻底消融,连渣子都不剩。

“终究,还是失败了么……”

死亡来临的这一刻,方源仅剩下来的意志,反而出奇的平静。

没有焦躁,没有不甘。也没有懊悔。

当初选择这条路。就已经预料到了可能发生的结局,此时的这个情况早就在方源的设想当中。

没有办法了。

已经拼尽了全力。

“如果再给我一次机会,我仍旧会这样活着吧。呵呵呵,那么就这样吧,我的蛊仙冒险物语,就到这里终结吧。虽然没有留下什么传记和传承,不过……也无所谓了。”

方源的意志迅速消沉。无弹窗,最喜欢这种网站了,一定要好评]

他很平静。甚至感到一种幸福。

如果他还有脸面,恐怕此时嘴角翘起,下意识的带着微笑。

死在自己追求的路上,还有什么好遗憾的呢?

“呵呵呵……嗯?”

方源心中的笑声戛然而止,异变就在此刻发生。

苍凉壮阔的光阴长河之中,忽然冒出一个鬼脸。

鬼脸先是朝着方源的残存意志,挤眉弄眼,似乎是在发出无声的嘲笑。

然后,漆黑的鬼脸脸颊鼓起,表情十分痛苦。好像是在呕吐。

一颗巨大的花骨朵儿,从鬼脸的嘴巴中湿淋淋的冒出来。

鬼脸的嘴巴几乎要被撑爆,嘴边夸张的咧开,竟然直接咧到耳根子处。

吐出花骨朵后,鬼脸一下子轻松下来,又开始挤眉弄眼地望着方源,流露出滑稽古怪。却又阴森恐怖的笑容。

而那多花骨朵儿,破开河面,徐徐绽放。

时间似乎在这一刻停止。

眨眼间,花骨朵儿盛放开来,竟化为一朵娇艳的红莲!

从莲心中绽射出一股微薄的红光,照住春秋蝉失事的地点,于是红光之中的时间,开始回溯。

与此同时,承载红莲的鬼脸在河水中徐徐下沉。

就像是电影倒放一般,一切所发生的事情都在往回倒退。

又仿佛是泼出去的水,自动地退回到了脸盆中来。

方源的意志本来已经消散得差不多了,只剩下点点分毫,可以直接忽略不计。但在红光的照耀之下,他的意志迅速复原,并且无数的蛊虫碎片也跟着出现。

然后这些蛊虫碎片,一齐拼凑到一起,化为完整的春秋蝉!

春秋蝉载着方源完好的意志,重新开始启程。

红光消散,红莲转瞬间衰败溃散,而那个鬼脸儿也旋即被滚滚不绝的光阴河水,冲涤得干干净净。

仿佛一切都是幻觉。

但春秋蝉却已经被强行复原。

它载着方源的意志,方源仅剩下来的希望,冲向河水当中。

艰难的……逆流而上。

回到过去!

……

星象福地。

魔雾缭绕,毒血蒸腾,已经将三层蛊阵都腐蚀破坏,周围地表都化为了一层浅浅的毒泥烂沼。

方源满脸郑重之色。

“接下来,就是最难处理的仙材地极天罡了。”

他取出一份仙材,拿在手中。

这份炼蛊材料十分奇特,是由泥和气组成。泥气自发地拘束成一团。

上面是淡青色的罡气,下面是黑色的泥土。

罡气是九天之上的天气。太古九天之外,都有一层厚实的罡气墙。蛊仙要进入九天中探索,往往就得突破罡气墙。

而黑泥,则是十地之下的浓郁地气凝聚成的精华。

天地二气本身就难以共存,但是在此刻,这地极天罡中两者却达成和谐的统一。不仅和平共存,而且相互之间不断转化。不断有黑泥化为罡气,又不断有罡气化为黑泥。

方源手掌摇晃一下,这团地极天罡迅速浑浊,黑泥罡气混淆一块,形成一团灰雾缭绕。

但不摇晃,静置十几个呼吸之后,黑泥就会沉淀下来,罡气则在上。又出现黑白分明,相互微微循环的奇象。

“处理这种仙材。最是麻烦。寻常的炼道杀招,都不能完美处理。唯有用公认最强的,处理仙材的四大仙道杀招静眠电蟒,映雪,闷雷石鼓。风磨,方可一蹴而就。可惜这四种杀招我都没有。要处理地极天罡,只有卖力气,下苦功了。”

方源心中念头一闪,脚下一蹬,雄躯顿时拔空,轻轻一跃。整个人便跳进了龟壳之中。毒血之内。

噌!

方源亮出尖锐的指甲,分别在六只怪臂上切出伤口。又在自己的胸膛,后背等处,戳出伤口。

血炼杀招血丝游。

从这些伤口中,游出一丝丝的血迹。

血迹很快就融入深紫色的毒血当中,旋即龟壳大锅中的这些毒血,仿佛被牵引一样。开始从方源的伤口里钻进去。

剧痛传来,方源闷哼一声。

仙僵是没有痛觉的,方源能感受痛楚,自然是用了蛊虫手段。他需要通过感知痛楚,来明白仙材处理到了什么程度。

方源的血液和龟壳大锅里的毒血,不断交融,形成循环,在方源的身体内进出不断。

这个过程变得稳定之后,方源将早就取出来的地极天罡,一口吞下。

咕咚一声。地极天罡被他吞入腹中。

这是他前世的独创,血道炼蛊的诡谲法门。他将这个命名为肉身血炼法。

地极天罡进入他的身体内,不断地被血液冲刷,微微溶解在血液中。

这些血液,又透过方源浑身上下的伤口,从体内流出去,汇入龟壳大锅之中。沉入锅底。

同时,大锅内的其他毒血,则通过伤口,流入方源的体内,再冲刷地极天罡。

时间缓缓流逝,三天两夜过去。

方源浑身是伤,脸上痛得狰狞扭曲,八根怪臂都插入毒血之中,獠牙外龇,双眼赤红一片,喘息如牛。

“还,还剩下一个晚上,我就能将体内的地极天罡,彻底处理好了……只要撑过这一步,炼制变形仙蛊的过程中,最艰难的一步就算渡过去……呃!”

忽然间,方源痛苦的神情僵住。

紧皱的眉头下,炯炯发光的碧绿眼眸,忽然变得一片迷茫,丧失了几乎所有的神光。

炼制仙蛊的过程当中,必须时刻保持高度的注意力,怎么容得他这般分神?

噗!

他顿时仰头,猛地吐出一大口的毒血。

随即,他轰然倒下,摔倒在巨大龟壳中的毒血中,溅起一蓬猩仇无比的血浪。

血浪冲出龟壳的边缘,飞溅到草地上,很快一大片绿油油的青草,被腐蚀成一股股的暗红烟气。

“主人!”星象地灵大喊一声,满脸担忧之色,扑进龟壳当中。

忽然,血泊中冒出方源的脑袋。

他扑腾了两下,旋即重新站起身来。

他满脸的迷茫迅速消退,双目重新变得神采奕奕,口中喃喃:“这是……这是?”

听到方源这般自言自语,星象地灵心中不由更加担忧了:“主人不会因为炼仙蛊失败,直接变傻了吧?”

方源先是望着自己的双手,然后目光又扫视龟壳和血泊,最后目光的焦点,停留在星象地灵的身上。

“现在是什么时候?”方源问道。

星象地灵心中顿时咯噔一下,但仍旧答道:“从主人你开始炼制仙蛊,已经过去好多天了啊。”

“哈哈哈哈……”方源仰头大笑。

星象地灵心中一片冰凉:“完了,主人真傻了!炼制仙蛊失败,身受重伤,他连具体时间都不记得了,还笑得这么开心!”

方源心中的喜悦,不足为外人道也。

“又重生了!”

“尽管催动春秋蝉是失败了,但是因为一场异变,让我起死回生,由败转胜,意志再次回到了过去。”

“我还在炼制变形仙蛊……是一年多前么。时间居然这么短!”

“依照我的底蕴,那么多的仙蛊在身上,至少得有数百年啊。只回到一年多以前,是因为催动春秋蝉失败的原因吗?”

“还有,那个鬼脸、红莲,究竟是怎么回事……到底发生了什么?”

\end{this_body}

