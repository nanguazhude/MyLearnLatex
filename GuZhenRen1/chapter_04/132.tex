\newsection{太丘非是传承地}    %第一百三十二节:太丘非是传承地

\begin{this_body}

北原,太丘。[

一场激战,已经到了最关键的时刻。

一方是三位蛊师,另一方则是一群万兽规模的羊群。

两位蛊师顶在前方,在气势汹汹的羊群面前,节节败退。而后一人,一身白衣已染上层层血迹,面冠如玉,但神色委顿疲惫,气喘吁吁,眼帘低垂。

忽听一声吼叫,一头万兽盘山羊王,率领身边精锐羊群,向这三位蛊师展开冲锋。

冲锋速度越来愈快,两边的羊群像是激流汹涌的浪花,不断分散,留出道路。

几个呼吸的功夫,羊王等已经近在眼前。

“公子,快撤吧!”

“情况危急,我们殿后,公子,来日方长啊。”

前面的两位蛊师,见此危情,睚眦欲裂,口中连喊。

但身后那人,却好似状态极为不佳,没有听见也似。

其中一位蛊师,面现坚定之色,当机立断,猛地开口:“东破空,你是飞行大师,速带公子撤退。这里由我顶着!”

蛊师东破空身躯一震,看向身边的战友,面现一抹犹豫之色:“谭武枫……”

意欲留守的这位蛊师,正是谭武枫。

早年乃是魔道蛊师,和水魔浩激流,并称为风水双魔,在北原赫赫有名。

浩激流投靠了黑楼兰,结果方源捣毁王庭福地,他陨落其中。反观风魔谭武枫,投靠东方余亮。在王庭之争中失败,却反而保存了性命,由魔转正。一直依附着东方家族的上代族长东方余亮。

但如今。谭武枫要留下来殿后,以一人之力对抗万兽羊群。他这是用自己的生命为队友争夺宝贵的时间,让他们撤退!

这根本就没有生存的任何希望,名副其实的十死无生。

东破空心中震撼,又感动。

谭武枫虽然本来高强,但他身为正道,心底对谭武枫的投靠动机。忠贞,都心怀疑虑。除去疑虑之外,还有细微鄙视和不屑。

但此刻。东破空心中,这些疑虑、鄙视、不屑都统统烟消云散,化为一股内疚悲壮之情。

情真意切之下,他脱口而出:“好兄弟。我这便带公子去了!”

谭武枫瞪着牛眼。望着几乎近在咫尺的羊群,心中焦急万分,爆粗口喊道:“还不快滚?!”

东破空扭头便走,眼中两行热泪蓬勃而出。

他知道自己必须争分夺秒,要想逃出生天,单靠谭武枫的牺牲还远远不够,逃亡的路上还需要他飞行大师的拼力发挥。

就算这样,把握也不到三成。

然而。就在这时。

那位血染白衣的蛊师,忽然睁开双眼。猛地扬起头颅,口绽春雷:“七星灯!”

喊着,他整个人弹射而起,身边七团灯火,颜色各异,呼呼呼,绕着白衣蛊师周身急促旋转。

哗!

一大蓬的星念,在杀招七星灯的加持下,猛地喷发而出。

刹那间,这片天地都被染成晶莹的湛蓝,星光璀璨耀眼,映照在猝不及防的羊群脸上身上。

盘山羊群的万兽王惊惶高叫,但惯性太大,羊群一头扎进无数的星念当中。

“公子!”

“余亮大人!”

谭武枫、东破空见此突变,真的又惊又喜。

海量的星念,结成一股庞大的星云,不断剧烈翻腾。须臾之后,星念损毁大半,稀疏下去,露出死去的万兽羊王,以及数十头羊群精锐。

羊群失去羊王,顿时分崩离析。在好几头千兽羊王的瓜分下,原本庞大的羊群,分成数股分别向四面八方急急退去。

血染白衣的蛊师,从半空中缓缓落地,身形猛地踉跄一下,差点一头栽倒在地上。

他虽然打出了致命一击,翻盘成功,但此刻已是油尽灯枯的状态。

“公子!”谭武枫、东破空连忙上前,搀扶住这位蛊师的两只手臂。

这位蛊师,正是东方余亮。北原第一智道蛊仙东方长凡死后,继承其衣钵的传人!

他面色惨白得吓人,此刻强忍眩晕,勉强一笑:“好了,终算是打退了这波万兽羊群,可以有一段喘息之机了。”

“公子……”谭武枫、东破空相互对望一眼,均是感动又佩服。

东破空一边扶着东方余亮,令其缓缓坐下,一边忍不住劝道:“公子,太丘乃是北原十大凶地之一,我们进来三天,已经遇到十几波的兽群。原本十几人的队伍,已只剩下我们三人。我丧命于此,并不打紧。关键公子贵重,再这般下去,若有个不测,该如何是好?此时情形,咱们不若退去,下次再闯?”

东方余亮坐在地上,闻言不禁苦笑,连连摇头:“你们不清楚情势,此次是我最后的,也是唯一的机会。这一次我已经孤注一掷,仿佛开弓后射出来的飞箭,再无回头之路,也不能回头。我们抓紧时间,尽量休息,恢复战力罢。”

说完,他便闭上双眼,手捏两块元石,进入休憩的状态中。

东破空、谭武枫对视一眼,均看到彼此眼中的坚定之色,相互之间点点头后,也纷纷盘坐在东方余亮身侧,拿出元石打坐起来。

在太丘外围,蛊仙陆青冥收回视线,赞赏道:“想不到东方余亮区区一介凡人,居然能闯进太丘中如此深度。这一次,他示敌以弱,诱敌深入,盘山万兽羊王等若送上门去给他施行斩首战术,居然真的渡过了这一层难关。”

站在他身旁的蛊仙韩东,冷哼一声:“这太丘乃是十大凶地,盘踞着荒兽,上古荒兽,就是我等蛊仙也不敢擅闯。东方长凡将传承之地,设立在这里,就是要防备我们啊。”

蛊仙苏光附和道:“不错。东方余亮等人,乃是凡人蛊师,气息微弱,不受荒兽、上古荒兽的重视。又有东方长凡留给他们的路线图,因此一路上只是碰到普通兽群,闯入太丘深处。若是我们出马,一丝蛊仙气息泄露出去,恐怕就会引发荒兽、上古荒兽的警觉和对抗了。嗯?又有蛊仙来了外围。”

其余两位蛊仙闻言,顿时有感,一同将视线投向东南方向。

只见那处,有三位蛊仙刚刚赶来,均是一身黑袍,头戴面具,只露双眼。

“嘿,又是想来分一杯羹的家伙。”韩东双眼阴芒一闪,语气森森。

三位黑袍蛊仙,缓缓停下。

不是别人,正是黎山仙子、方源、黑楼兰三位。

黎山仙子乃是北原知名蛊仙,当初又和东方长凡定下誓约,不图谋其传承,因此不好露面。

而方源、黑楼兰事关八十八角真阳楼倒塌的泼天大案,更是要隐藏真正面容。

三人缓缓停下,悬浮半空,立在太丘外围,并不深入。先是望了望东方余亮三人,随后又纷纷扫视周围一圈。

黑楼兰轻笑了一声:“东方长凡的智道传承,的确够吸引人。太丘外围,隐藏着这么多的蛊仙。单单我发现的,就超过了二十位。”

黎山仙子接道:“我发现了二十六位,大多数都是魔道蛊仙,但还有东方家的蛊仙。”

方源匆匆一瞥,却只发现十九位蛊仙。

他虽然战力提升上去,但侦察手段却仍旧比不过老牌蛊仙黎山仙子。至于黑楼兰,竟然也在侦察上面超出他一筹,这一点让他心中暗暗警惕。

“难怪东方余亮急着继承智道传承。看来,东方家的蛊仙也在觊觎这份传承。毕竟得了这道传承,就有希望成为北原的智道第一人。就算是自家蛊仙,也抵御不了这样的诱惑啊。”黎山仙子感慨道。

“东方长凡是否百密一疏?和正道超级势力都订下盟约,互不侵犯,却漏掉了自家的蛊仙。”黑楼兰说着,眉头微微皱起。她总觉得,东方长凡不会如此大意。

“东方长凡是北原智道第一人,他将传承线索留在东方余亮的脑海中。这种智道手段,却是不能轻易破解。现在蛊仙们虎视眈眈,又按耐不动,看来是想等着东方长凡开启传承,再行抢夺了。”方源道。

他此行,是想得到完整的智道传承。

一份智道传承,只有越完整,价值才越大。

但现在看来,他的愿望基本上要落空了。觊觎传承的蛊仙数量这么多,待会抢起来,极可能形成瓜分的局面。

“不过好在我这一次战力提升许多,拥有一拼之力。待会抢夺起来,不能乱出风头,惹得众人围攻,还需策略。”

方源定下战术。

他战力虽强,但现在蛊仙众多,他也不能以一敌众。当然,也不会蠢到以一敌众。

然而就在这时,异变突生。

一道刺眼的光芒乍起,光芒迅速消散,原地的东方余亮、谭武枫、东破空三人已然不见踪影。

“这是传送的蛊阵!?”

“隐藏在地下,我们居然没有发现。”

“东方长凡布局果然没这么简单,这太丘并不是真正的传承地点,却是阻挡我们的难关。”

“快,趁着蛊阵的痕迹还残留着,气息没有消散,快去查看线索!”

蛊仙们恍然大悟,纷纷现身,向太丘深处冲去。

吼吼吼!

一头头荒兽,甚至上古荒兽,都被惊动,从太丘各处涌出。(未完待续。)

\end{this_body}

