\newsection{穿透界壁}    %第六十一节:穿透界壁

\begin{this_body}

比方说,一位水道蛊仙,福地就是一片海。23us得到浪迹天涯蛊后,蛊仙就会在福地中豢养幽冥水母、深海闪电鳗鱼等。这样一来,他喂养仙蛊浪迹天涯,就能就地取材,自给自足。

像太白云生的两只仙蛊,都可以经营自家福地,从而得到食料。

黑楼兰的楼兰福地中,土地里蕴藏着大量的重金矿藏,这些重金,就是力气仙蛊的食料。

但方源和他们俩的情况不一样。

他们俩的仙蛊,大多是升仙时,本命蛊、核心蛊提炼而成的,和自身流派相统一。

方源的仙蛊,大多是谋夺而来。本命蛊虽然是春秋蝉,但他没有宙道传承,走的是力道路线。

方源手中的仙蛊十分杂乱,涉及宙道、宇道、水道、智道等多个流派。而且最关键的是,方源的福地死了。

他的仙窍成了死窍,种什么死什么,再不能像寻常福地那样经营发展,也就不能为喂养仙蛊提供方便,更不能自产仙元。这样又导致仙元石的负担更重。

方源手中的仙元,不仅要承担交易,还要有一部分炼化成青提仙元,还要负责喂养仙蛊。和黑城激战一场后,方源的经济就差点崩溃,主要原因就是这个。

因此摆在方源面前的第二难题,就是仙窍福地。

仙窍福地已死,他的修为就停滞。

福地涉及到方方面面,带给方源极大的拖累。哪天方源真正摆脱仙僵之躯。他才能正式展开蛊仙的修行。

但摆脱仙僵身份,回复人身,方源就蹭不了智慧之光了。这又是一个难题。

至于最后一个主要难题,就是战力。

如今净魂仙蛊饿的虚弱,不堪使用。方源的底牌――仙道杀招万我,也就暂时不能动用了。

方源的战力也随之下降。

不管是面对今后的中洲十大古派,还是北原蛊仙的追捕,谋夺东方长凡的智道传承,帮助琅琊地灵活捉荒兽。谋取关于落魄谷的盗天传承,对付姜钰仙子夺取暗渡仙蛊等等,都需要高强的战力。

“影响蛊仙战力的方面。主要有四个。蛊仙本身的战斗造诣、仙蛊、杀招以及仙元。现在有胆识蛊的生意,仙元暂时不缺。蛊仙战斗造诣也属于日积月累的底蕴,短时间之内,无法拔升。仙蛊方面。血神子残方虽然足够。能够推算出完整的血神子仙蛊方,但我现在的财富距离炼制仙蛊的标准还很远。”

方源思考了一下,就算能够炼制血神子,他也不想这么做。

血道早就被打上魔道的标签,就算是十大古派研究血道,也都是偷偷摸摸的进行,不敢昭然若揭。况且天庭还有诛魔榜,方源还不想这么早就登上大名。

况且方源手中的仙蛊。几乎都要养不活了。喂养负担太重,再添加新的仙蛊。方源自己都缺乏信心。

于是,摆在方源面前,能够迅速拔升战力的,就只有杀招这个方面。

而杀招,又包含仙道杀招、凡道杀招。

“现在我可算拥有了充足的仙元石,手头上比之前宽裕很多,不仅可以购买更多更好的凡道杀招,而且还能不断试验、推演,重现前世的某些强大杀招。”

思考好了,方源便雷厉风行,立即着手。、

中洲。

以凤九歌为首的一群蛊仙,站在一幕巨大的光壁前。

光壁宽广无比,连接天和地。白光凝实,宛若墙壁。当中又有金芒闪烁,气象恢弘,堂皇磅礴。

此正是中洲胎膜,名为圣贤界壁。

蛊师世界五大地域:中洲、南疆、东海、西漠、北原,皆笼罩一层胎膜,形成天地极限,相互隔绝。

“老算子,你这次推算的结果如何?”残阳老君问向队伍中的一位青年蛊仙。

青年蛊仙缓缓睁开双眼,一对眼睛中尽是云雾升腾,变化万千。

良久,云雾散去,还原成正常的黑瞳眼白。

青年蛊仙带着微微欣喜的语气道:“可以了,这处界壁较为薄弱,是我们这些天来搜寻到的第三弱点。但比第一、第二弱点,稳定得多。我建议咱们就以此为突破口,打穿界壁,闯进北原!”

“哈哈哈,好,咱们寻了近千处地点,终于找到了一个理想的薄弱地点了。”天聋老人大笑一声。

凌梅仙子吐出一口浊气,感慨道:“寻找了这么多天,总算得到良好结果,不容易啊。”

“已经浪费了不少时间,下面就开始吧。”凤九歌言简意赅地道,“谁先来?”

傲雪仙子和凌梅仙子对视一眼,一起站了出来,齐声道:“这一次,就由我天妒楼打头阵罢。”

说着,双双出力,一齐催动仙道杀招无双偃月斩。

这杀招核心,是一只七转偃月仙蛊,另有两只六转仙蛊辅助。形成的仙道杀招,消耗仙元极多,且时刻牵扯精神,单凭一位蛊仙发挥不出威能,非得两位蛊仙一齐催动方可。

仙道杀招催动起来,顿时形成一轮青色的弯月。

弯月体型不大,只有水缸大小,但凝如实质,熠熠生辉,仿佛一件美轮美奂的艺术品。唯有偶尔泄露出来的恐怖气息,才让人知晓它绝非外表看起来那般无害。

周围的蛊仙们,纷纷远离傲雪、凌梅两位仙子。

两仙子酝酿足够,一齐娇喝一声,放出青色弯月。

弯月飞驰而去,速度极快,月光温柔,整个过程悄无声息。

青色弯月斩在圣贤界壁之上,顿时斩出一道七十六步远的路途。

“不愧是天妒楼的招牌杀招之一,直接斩出七十六步远的路。了不起啊。”蛊仙陈振翅赞叹道。

凤九歌一马当先,众蛊仙随后,一起进入圣贤界壁。

一踏进圣贤界壁之中。很多人的面色都微微一变,他们的身躯骤然变得沉重,念头也运转迟缓,身体中的仙窍也开始渐渐震荡起来。

“快,加紧速度。”凤九歌催促道,众人走了七十四步,再走不下去。

就是这一会功夫。圣贤界壁已经迅速恢复,原本七十六步深的路途,已经削减了两步。

残阳老君主动站了出来:“接下来。便让老夫来露一手。”

他取出一只仙蛊,仙蛊形如一段燃烧一大半的蜡烛,顶端燃烧着微弱烛光。

残阳老君凑近烛光,用嘴轻轻一吹。

烛光摇曳。顺着残阳老君吹出的气流方向。挥洒出无数莹莹光点。

光点打在白金色的圣贤界壁上,界壁宛若积雪消融,迅速化开,形成一个六十三步远的通道。

众人连忙前进,天聋老人站出来,也催动一只仙蛊,打通六十七步的道路。

就这样,一众蛊仙轮番出力。不是催动仙蛊,就是动用仙道杀招。打出通道,深入圣贤界壁内部。

他们越是深入,身上的压力就越重,甚至转化成一股向后拖拽的巨力,仿佛这个世界不愿他们离开中洲。

脑海中的念头,也越发迟缓,这让蛊仙催发仙道杀招,变得奇难无比。作为智道蛊仙的老算子不得不动用各种手段,甚至是仙道杀招,加持其他蛊仙的脑海。但就是这样,傲雪、凌梅两位仙子的无双偃月斩,也使用不出来了。

最关键的,还是他们体内的仙窍。

越是深入,仙窍的震动幅度就越大,福地中大地开裂,山石滚落,无数生灵遭殃惨死。

这些来自十大古派的蛊仙,各个战力出众,是响当当的强人。走了数百步后,大多脸色发白,一些修为薄弱者,更是身躯都开始微微晃动。

上千步后,大多数蛊仙都已经浑身大汗,有些面色苍白,傲雪、凌梅两位仙子已经相互搀扶着走了。

唯有凤九歌一人,仍旧昂首走在前端,神色如常。每一次出手,皆是仙道杀招,打出超出百步的通道。

大约走了三千多步,前方白金之光已见稀薄,并且隐隐透露出一抹青色。

“老算子你算的不错,这里的确是圣贤界壁的薄弱之处,咱们只走了三千多步而已。”当即有人赞叹道。

“过了圣贤界壁,就是北原的甘草界壁。咱们再加把劲,一举突破它!”残阳老君气喘吁吁地鼓舞士气道。

青年蛊仙老算子则有些担忧:“我能力有限,只能算到圣贤界壁。但愿这处甘草界壁,不会太厚。”

半刻钟后,众蛊仙脱离了圣贤界壁,正式进入甘草界壁。

这界壁,又和圣贤界壁大相径庭。蛊仙们深入其中,就感到一股无形的庞巨推力,正不断地排斥他们这群人。

仙窍福地的震动幅度更大,损失叫许多蛊仙都心疼无比。

“这一次北原之行,希望能收服一两只仙蛊才好,否则可就亏大了。”天聋老人道。

“唉,我年轻的时候,也穿过界壁,到过北原。整个过程,只要了一盏茶的功夫,轻松得很。”蛊仙洪赤明大为感慨道。

他穿透界壁时,还只是四转蛊师。修为越高,穿透界壁就越加困难。

半天之后,这群人不得不停下脚步。

他们置身在甘草界壁,到处都是青绿的浓雾,浓雾中一片片疯长的草叶,如蛇海,如发丝,不断缠绕,不断绞绕,阻挡着他们的前行。

领头的凤九歌汗流浃背,嘴唇苍白:“不行了,这处甘草界壁十分坚固,我们要换一个方向,哪怕多走点路。”

众人皆连连点头,大为认同凤九歌的决断。

老算子开始推算,这个期间,界壁不断恢复,众仙跟着后退。

“走这边。”终于,老算子算出一个方向。

众仙转折,历经一天一夜,终于苦尽甘来,打通甘草界壁,来到北原。

“先修整三天,再谈其他。”凤九歌有气无力地道。

整个队伍中,只有他一人还能勉强站着,其余人都是东倒西歪,更有甚者像是一滩烂泥,直接躺在地上,不想动弹一下。

\end{this_body}

