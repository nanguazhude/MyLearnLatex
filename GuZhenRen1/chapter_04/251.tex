\newsection{向琅琊地灵借蛊}    %第二百五十二节:向琅琊地灵借蛊

\begin{this_body}

当即,方源再不迟疑,利用定仙游前往琅琊福地。<strong>80电子书wWw.80txt.com</strong>

琅琊福地已经面貌大变。

汪洋大海上,三大陆地相互对望。高空中,漂浮着一片由云泥构造出的第四大陆。

曾经因为激战而损毁的福地,经由太白云生出手,已经恢复了旧貌。

太白云生虽然没有机会参加攻防战,但在战后修复的工程中,也收获了不少报酬。

当然,因为组成炼炉的某些仙蛊丢失,导致福地内部空间收缩。这些消失的空间,以及相关的损失,太白云生可没有能力去复原。

十二云阁,已经重新被琅琊地灵建立起来,平均分部在云泥大陆上。并且,围绕着十二云阁,还有许多的毛民蛊师,开始新建更多的建筑。

“我打算以十二云阁为中心,建立十二云城。这片高空的云泥大陆,将作为毛民精英们的全新生存之地。”琅琊地灵见到方源后,主动为他解释。

方源点头,表示认可。

琅琊福地因为丢失了不少仙蛊,导致内部空间缩减了不少。而里面毛民的人口却是众多,即将达到三块大陆的上限。这种情况下,自然要有新的地方去容纳更多的毛民繁衍生息。

但琅琊地灵接下来的一番话,让方源吃惊不小。

“除此之外,我还组建了琅琊派!门派制度参照当今的中洲,我就是派中太上大长老,毛民蛊仙们担当其他太上长老。我的上一任太过迂腐了,拿这些宝贵的蛊仙战力,专门去炼蛊。哼!要实现毛民制霸的宏图伟业,必定是要流血的。”

“接下来,我要对这些毛民蛊仙,进行集训,培养出他们的战斗能力。等到他们有了自保能力,我就将他们都分派到福地之外去,让他们完成各种各样的门派任务。”

“还有这海上的三个大陆,将会在我的影响下,爆发战争!从中决出三位国主,组建三个陆地国家,国家之间进行相互的竞争。每年,我琅琊派都会在这三个毛民国度里,招收最精英的毛民,并将这些毛民,带到云盖大陆上居住,形成琅琊派凡尘基石。并且,从中选择一位五转巅峰的毛民蛊师,作为琅琊派的掌门人!”

新的琅琊地灵和上一任完全不同,或许也是受到秦百胜强攻的刺激吧。琅琊地灵决心对整个琅琊福地进行铁血改制,手段强硬,一副枭雄大帝的样子。

一朝天子一朝臣,因为琅琊地灵的转变,底下生存着的这些毛民都将遭受战火的折磨和考验,往昔和平安宁的平静生活将一去不复返。<strong>最新章节全文阅读www.Mianhuatang.cc</strong>

“琅琊地灵用战争这样残酷的方法,的确能挑选出最出色的毛民种子。但这样的方法,未免太过于奢侈了些。”

方源暗暗可惜,战争中牺牲的民决不在少数。

这种损失让他颇为心疼,还不如交给他,用来炼蛊。

但这话,方源只是在心中嘀咕,万万不会说出来的。

在琅琊地灵看来,毛民是全天下最优秀的物种,人类以及其他的异人都是低下一等的。

方源豢养毛民,当做炼蛊奴隶。琅琊地灵要知道这个情况,绝对不会接受,甚至会和方源翻脸。

所以,方源虽然正缺少豢养毛民的商业经验,但之前谈论琅琊福地攻防战报酬的时候,明明知道琅琊地灵手中定有最珍贵的豢养经验,方源也从未提出过这点要求。

“不过照此下去,琅琊福地经过这番铁血改造,毛民蛊仙们励精图治,战力而且会迅速强盛起来。也许前世历史中,琅琊福地也发生了这种改变,才撑过了整整七波攻势。”

“不过琅琊福地这里面的水深,虽然跳出来两个毛民蛊仙内奸,但难保剩下的这些蛊仙中,不会有第三个,第四个内奸潜伏着。秦百胜他们居然能发展出这等内奸,手段简直是匪夷所思。”

方源心中琢磨着,嘴上则和琅琊地灵谈及此番来意。

琅琊地灵有些吃惊,重提旧事道:“哦?你现在就想将我的这套仙蛊借过去,搬迁你的福地?我可记得,我事先和你讲过,借我这套仙蛊可不便宜。就算念及你帮助我防御强敌的份上,也需要两千块仙元石的吧?”

方源便取出两千块仙元石,递给地灵:“你看,我都已经准备好了。”

琅琊地灵吃惊不已,一边接过仙元石,一边上下打量方源,目光发生了隐隐的变化。

他是知道方源底子的,心中十分惊异:“这小子才晋升多久,就有这么多的仙元石?”

“想当初他和上一任交易,推算仙蛊方赚取仙元石,可见处境是多么的困窘。看来这些仙元石,应该大部分都是外借的了。这可不好,欠下一屁股债,还有人情,以后可不好还。”

琅琊地灵摇摇头,对方源的这番行动,并不认同。

但他同时却也理解。

他知道方源犯的案子,实在太大了。一旦查明真相,他就完蛋了,肯定会被追杀到天涯海角。

所以,方源他必须要搬迁福地。

不过这一点,却也正是琅琊地灵准备逐渐吸纳方源,企图将方源转化为自己走狗的主要原因。

正因为方源这样的处境,四面八方都是敌人,才会在走投无路的情况之下,加入毛民势力,成为人族内奸。

但琅琊地灵却并不知道他完全猜错了!

这两千块仙元石,还真的都是方源的私有之物。

方源现在有钱,十分有钱!

他和灵缘斋做的第一批胆识蛊交易,就收获了一百八十块仙元石。

在中洲炼蛊大会之后,方源展现出强大的炼蛊造诣,鹤风扬亲自邀请他去仙鹤门看看去,这是仙鹤门方面对方源态度的明显转变。

方源现在掌控着狐仙福地,夹在灵缘斋、仙鹤门两大超级势力之间,可谓左右逢源。

外部环境史无前例地宽松下来后,又考虑到将来追杀的危险处境,方源便断然加大了胆识蛊的生产。

这项决定,使得他每个月单单胆识蛊的纯收益,就一路暴涨到三百块仙元石左右!

别忘了,方源收服了琅琊福地之后,还完整地继承了万象星君的四大经济支柱。还有他亲自出面,和西漠萧家达成的长恨蛛的买卖。

总共六大经营项目,让他每个月赚取的仙元石,直接突破了一千大关!

更还有龙鱼、幽火龙蟒,没有投放市场。若是再算上这两项,方源初步估算了一下,每个月的纯利润能够达到一千五百!

方源自从成为星象福地之主,时间早已经过去了两个月。

手中的仙元石积余,自然超出了两千。

回想起来,从刚刚升仙的斤斤计较,锱铢必较,甚至半块仙元石都要精打细算,到如今月入上千,手头宽裕,方源向前成功跨出了巨大的一步。

他这一步的步伐是如此巨大,琅琊地灵不知内情,哪怕已经高看方源一眼,仍旧猜测错误。

方源虽然是修为垫底,还是仙僵,但仙元石的收益,已经达到通常七转蛊仙的程度了。

付出两千块的仙元石,方源和琅琊地灵又用信道仙级杀招,订下相关契约。

最后方源还抵押给地灵十只仙蛊,这才成功借得一整套仙蛊,数量多达十六只,用来搬迁福地。

方源马不停蹄,利用定仙游,再回到星象福地。

星象福地中有三头荒兽刺脊星龙鱼,方源早已经考察过,此时也不犹豫,直接选了其中最强壮的一头。

他先是盘旋围绕在刺脊星龙鱼的周围,不断动用手段,观察刺脊星龙鱼的型体,骨骼,血肉等等方面。

确认刺脊星龙鱼状态良好之后,方源盘坐在半空,面对这头荒兽,闭上双眼。

刺脊星龙鱼早已经被驯服,乖乖地悬停在半空中,一动不动。一双大如马车的死鱼眼,闪耀着些许星芒,此时倒映着方源的身影。

方源心中平静,脑海则渐渐兴奋起来。

无数的星念,宛若漫天的繁星,动若萤火,在他略显狭小的脑海中尽情飞舞回旋。

这些星念旋即又分化为两拨。

一拨占据脑海右半部分,交汇成刺脊星龙鱼的立体图。

另一波星念,则占据方源孬好的左半部分,形成星象福地的全景。

完成这一步后,方源深吸一口气,毅然催动仙元,灌注到借来的某种智道仙蛊当中,并且按照地灵交付的具体方法,开始推算。

两股星念飞速前行,猛烈对撞,掀起漫天的星云风暴。

风暴中,无数的星念不断闪烁,有的在合并,有的则开始分解。

一天两天三天……在星象福地中,足足过了七天六夜,疲惫不堪的方源这才缓缓地睁开双眼。

脑海中消耗的星念,多达十六万。催动智道仙蛊消耗的青提仙元,更耗费高达六十六颗。

付出如此巨额代价,方源得到的是两套蛊阵图。

这两套蛊阵图,一大一小。

大的蛊阵图,对应星象福地,有八个阵眼。小的蛊阵图,对应刺脊星龙鱼,有四个阵眼。每一个阵眼中,需要一只特定的仙蛊压阵。大小两大蛊阵图,加起来就需要十二只仙蛊。<!--80txt.com-ouoou-->

------------

\end{this_body}

