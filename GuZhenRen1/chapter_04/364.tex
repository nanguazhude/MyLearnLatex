\newsection{灰忆}    %第三百六十五节:灰忆

\begin{this_body}

这浩劫地陷,根本就是蛮不讲理!

不仅将渡劫蛊仙,困在地上,而且还影响他们的念头。

没有念头调动蛊虫,蛊仙如何施展手段?

除非是有智道的手段,进行抗衡。

但还有个前提,这种智道手段必须之前就已经布置妥当。否则的话,一旦优先中了地陷,那么蛊仙就算是有智道的后手,也无法施展了。

先下手为强,后下手遭殃。

失去了先手,基本上就完蛋了。

由此可见,浩劫之强,更遑论之上的万劫。

“可恶!!”影无邪挣扎不休,却毫无结果,反而越陷越深,气得大叫。

“不要挣扎,也不要多想。这些都会让我们处境更糟,此时已不是我们能够解决的了。”仙僵薄青却很冷静。

准确的说,里面的墨瑶残魂,十分冷静。

她有经验。

生前,她帮助爱郎薄青渡劫,见识过的浩劫,比这个还要恐怖!

相比较而言,浩劫地陷反而显得温和无害。

“看来只能寄希望于本体的搭救了。”影无邪叹息一声,凝望魔尊幽魂。

只见魔尊幽魂同样泥足深陷,一双漆黑大脚,已经半数陷入地表之下。不仅如此,在他的左脚的前脚掌部分,还被紧紧的缝在地面上。

这是之前,魔尊幽魂身中仙蛊屋绣楼攻击的结果。

管中窥豹,由这点可见,魔尊幽魂虽然强势无比。但实际上已经用足全力,却并未掌控全局。

此刻。他的六只眼眸, 有两只瞪住监天塔。使其动弹不得,另外四只则紧紧注视着苍空。

在这苍空之上,穹顶之中,灰色的哀云越发浓密,开始徐徐降下。

第二场万劫降临。

但是它却和第一场的风雷囚笼不同,风雷囚笼之快,快到难以想象的地步,一眨眼间,风雷加身。已经身陷囹圄。

而这场万劫的速度却极为缓慢。

慢到似乎足以让渡劫之人,做出无数的反击。

但魔尊幽魂却没有动弹。

他静静地站立着,仿佛是一座沉默的山峰。

“难道本体已经中了招吗?”影无邪大叫,看得焦急无比,“这么好的机会,你倒是反击啊!!”

但直到灰色哀云将魔尊幽魂几乎完全笼罩,后者都没有丝毫动弹,只是一直盯着监天塔,将其牢牢控制。

天庭蛊仙叫苦不迭。

他们利用虚化的手段。让监天塔躲避魔尊幽魂的攻击。但没想到,魔尊幽魂的虚道境界,远超想象,绝对是大宗师级数。

监天塔的虚化战术。反而被魔尊幽魂利用,再无法返回实体。无法返实,如何能对战。如何能骚扰?

“万劫来了!”影无邪咬牙切齿,死死等着灰色的哀云。慢悠悠地覆盖下来。

仙僵薄青也不由地眯起双眼,浑身肌肉紧紧绷住。

浩劫地陷。就已然让他们毫无还手之力,现在万劫降临,是否就是薄青和影无邪的末日?

但灰云降下,却似毫无伤害。

“怎么回事?难道有无害的万劫?!”影无邪紧张半天,却不见动静,双手四处乱摸自己的身体,惊疑不定地大叫道。

仙僵薄青没有回答他。

倒是天庭蛊仙中有人认出了这个万劫,但影无邪是无法获知答案了。

“这难道是灰忆?”那位认出来的天庭蛊仙叫道。

“什么是灰忆?”

天庭蛊仙陷入回忆之中:“我在年轻的时候,曾经继承过一位八转蛊仙的传承。当年这位八转蛊仙,就是遭遇万劫灰忆而惨败。弥留之际,勉强留下了传承。这个万劫,对肉身毫无伤害,却直接勾出心底最深处的记忆。这些记忆,都是曾经带给渡劫蛊仙重大的心里创伤,或者人生的阴影。”

“你们千万别小瞧了这个万劫。那位八转先贤,就是在这个万劫之下,苦苦支撑了不到一炷香的功夫,就斗志丧尽,灰心丧气,再无一丝战意,了无生趣。”

“常言说的好,自己才是自己最大的敌人。类似幽魂魔尊,无敌天下,那么他最大的敌人,不就是他自己吗?人活在这个世间,就算是成为九转尊者,也总有弱小的时候。也总是从弱小一步步修行,不断变得强大的。幽魂魔尊屠戮天下,杀性之重,恐怖淋漓,后人都推测他是不是童年的时候,受到过什么刺激。”

“谁能没有隐藏在内心最深处的苦痛?谁能没有一些难以启齿的羞耻之事?谁能没做过违背本性的抉择呢?成长中,谁又能没犯过错?妙极,妙极!这场万劫真是妙极!恐怕魔尊幽魂,是要栽在这里了。”

万劫灰忆。

正是因为认出了这个万劫,魔尊幽魂才没有动弹。

他知道,什么样的攻击,都无法消解此劫。唯有投身其中,直面过去种种不堪、阴影、羞愤耻辱,才能渡劫。

萦绕在他身边的灰色云雾,忽然有了光彩,还散出声音。

呈现在众仙面前的,是幽魂魔尊的童年一幕。

“杀了她!杀了她!杀了她!”

一干蛊师,死死包围着一家三口,双目赤红,满脸狰狞地怒吼。

“爹!你不能杀娘啊!!”一位男孩护住身后重伤的母亲,嘶声力竭地呼喊着。

“哼,魔道中人,人人得尔诛之!大义灭亲,方是正道所为!!有什么不能杀的?不仅能杀,而且必须杀。只有杀了她,才能洗清我族的耻辱!!”为首的蛊师老者,义正言辞,张口怒喝。正是男孩的爷爷,家族的族长。

族长的话。得到了众多家老,还有精英蛊师的响应。

他们振臂高呼。

“杀!杀!杀!”

“杀!杀!杀!”

哧。

一声轻响。鲜红的血液喷溅。

男孩连忙转身,下一刻,他瞪大双眼,眼眸却缩成针尖大小。

只见他的父亲,已在瞬间越过自己。他满脸通红,紧要牙关,虎目含泪,心中激烈的情绪难以压抑。而他手握着的利刃,则已经深深地插在自己爱妻的心口。

男孩张口想要呼唤。

但却终究没有发出任何声音。

从那一天起。他开始变得沉默寡言。

很显然,这个小男孩正是童年时期的幽魂魔尊。

在万劫灰忆的影响下,他童年的阴影,不再只埋藏在当事人的心中,而是展现在了众人的面前。

灰雾一变,又呈现另一出画面。

大约是几年之后。

男孩已经长大稍许,拘谨地站立在爷爷的面前。

身为族长的爷爷,喝了一口茶,悠悠地问道:“我让你饱读咱们家族的历史典籍。这些天你有什么收获吗?来,告诉爷爷。”

“爷爷。”男孩先行了一礼,这才道,“孙儿这些天来收获很多。颇有心得体会。”

“哦?说给爷爷听听。”老人饶有兴趣地道。

“孙儿纵观历史,发现世间有一条最大的道理,那就是杀。”男孩平静地道。

“杀?”老人顿时皱起眉头。语气微沉,“解释给爷爷听听。”

“是。”男孩继续道。“我们肚子饿了,要食物喂饱肚子。就要杀猎物果腹。我们有敌人,就要杀掉他们,解除威胁。世间太平了,就要杀掉功臣,掌控权利……”男孩侃侃而谈,却没有注意到老人越皱越紧的眉头。

男孩又继续道:“纵观历史,就是你杀我,我杀你。什么是英雄?就是杀的敌人很多很多。什么是失败者?就是杀不过对方,被人杀了。”

“其实,杀虽然只是一个字,但里面也有很深学问。如何杀,是用蛊虫亲自动手,还是雇佣蛊师替自己出手?有时候,不能光明正大地去杀,明杀的话,会惹大麻烦,那就选择暗杀。暗杀又有很多种分别呢,比如说……”

“够了!”老人猛地怒吼,气极之下,将手中的杯盏砸在地上。

碎片溅射到男孩的脸颊上,顿时划出一道口子,血液慢慢流下。

老人腾的一下,站起身来,手指着男孩,十分生气地吼道:“我让你饱读史书,是要让你瞻仰我族先贤的功绩,知道我族辉煌的历史。是要让你明白礼义廉耻,让你清楚正道荣耀。你居然给我悟出个杀?这是哪门子的邪理?!你,你,你,给我禁足一个月,待在屋子里好好反省你的过错!!”

“是,爷爷。”男孩领命,语气低微,但眼底深处却闪着倔强的光。

灰雾中,画面再变。

几年后,男孩已经成为一转蛊师少年。

“呵呵呵,今天我总有炼成了匿息蛊,躲在书房,让爷爷大吃一惊!”少年悄悄地潜入书房。

“嗯?不对。我只是一转蛊师,爷爷却是四转。他要发现我,易如反掌。不如先躲到密门之后,藏在暗道里,然后再出来吓爷爷一跳。”少年又改变主意,开启密门,躲藏了进去。

不久后,他就听到动静。

他不敢开启密门,只能倾听声音。

两个人进了书房。

他的爷爷脚步沉重急促,显得怒气冲冲。

“这个逆子!他真的要谋反?要害他的亲爹?!”老族长勃然大怒,手掌拍在桌案上,发出砰的一声巨响。

“族长,证据确凿!族长你获得八十年寿蛊的情报,已经泄露出去了。少族长谋害之心,确定无疑。”

一个沙哑的声音。

少年瞬间听出来,这是家族的一个家老,族长的心腹。

“哼!这个逆子,想要从我手中夺权,可能吗?!”族长大喊。

“族长大人,少族长身边,可也有不少高层呢。”只听那沙哑的声音又道。

沉默了片刻。

少年爷爷的声音,这才低沉的响起:“逆子的势力,的确不容小觑。若是公然打杀,不仅有损正道名誉,也会酿成家族内斗,损耗族力。嗯……那就好好准备一下,先下手为强,找个机会,将他悄悄的暗杀了。人死灯灭,只要杀了这逆子,其他家老自然要离散的。”

“族长英明!”

密门后的暗道中,少年紧紧地捂住自己的嘴,浑身上下都在颤抖着。

\end{this_body}

