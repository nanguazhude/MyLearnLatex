\newsection{返实蝠翼}    %第六十四节:返实蝠翼

\begin{this_body}



%1
一只风狼,低着头,四肢几乎匍匐着,在茂盛浓密的草地中行进。

%2
被它盯上的猎物,是一头花粉兔。

%3
兔子两只耳朵直愣愣地竖起,倾听一切的行动。而兔首则埋在硕大的花朵中,汲取着花粉吃食。

%4
清风吹拂着这处山岗,绿草茂盛,繁花盛开,带来阵阵花香。

%5
风狼慢慢接近,而花粉兔却是毫无察觉。

%6
忽然,一道漆黑的阴影,迅速地从地面掠过。

%7
花粉兔一惊,抬起头来,哪怕狐仙福地中没有鹰群,也吓得它飞奔而走,恰巧逃过一劫,让风狼无功而返。

%8
飞在半空中的方源,将这一幕看在眼里。

%9
不由地,他心中泛起思绪:“现在我手头宽裕了,自身福地又成了死窍,不如先经营好狐仙福地。自北原之行后,狐仙福地中兽群规模下降谷底,十分稀薄。我现在已经直接贩卖胆识蛊,再进行石人贸易就是画蛇添足。”

%10
“狐仙是奴道蛊仙,狐仙福地是草原地形,最适合的就是豢养狐群。我将狐仙福地经营得生机勃勃,福地中产生青提仙元的效率就越高。小狐仙用这些青提仙元,也就能减少我自家仙元的消耗。等到我将来复活,若是合并了狐仙福地,得到的也会好处更大。”

%11
念头一转,方源又查看背后的蝠翅。

%12
自从将蝠翅成功移植到自己的后背,这已经是他第三次试飞。

%13
浓缩后的蝠翼。全力展开,也达不到方源手臂一半的长度。通体黄褐色,看起来较为微小。不如之前的轻虚蝠翼景象大气。

%14
但方源从不看重表面形象,他更看中实用价值。

%15
全新的轻虚蝠翼,带给方源一倍以上的速度提升,同时续航飞行能力,大为提升。毕竟之前的蝠翼,只是一抹虚影,而现在的蝠翼。却是货真价实的肉翅。

%16
蝠翼的肉膜、皮毛、骨骼、血肉都经过方源一遍遍的压缩、精粹。蝠翅中,分布着许多蛊虫。

%17
这些寄生在蝠翅血肉皮蛊中的蛊虫,正好可以组合成改良后的移动杀招轻虚蝠翼。

%18
“不。现在用轻虚蝠翼这个词来冠名这个杀招,已经不符合实际。不如就叫做返实蝠翼。”方源为这个改良后的移动杀招,取了新的名字。

%19
他心中对这个返实蝠翼的移动杀招,比狮毛甲还要满意得多。

%20
要搁在宝黄天中。这个移动杀招。能卖到六块仙元石之多!

%21
皆因,返实蝠翼大有发展潜力,只装一对蝠翅,并不是它的极限。它最多能移植三对蝠翅。这三对蝠翅的主人,至少得是荒兽。当然要是上古荒兽、太古荒兽那就更好。

%22
假设三对荒兽蝠翅皆成功移植到方源的背上,那么他的速度将暴涨到三倍多。

%23
轻虚蝠翼原本就是五域乱战时期,性价比最高,流传最广的移动杀招之一。在它的基础上。再增加三倍多的速度,就十分优秀了。

%24
虽然仍旧不必上浪迹天涯、平步青云这等的移动仙蛊。但方源若再遇到肥娘子,单凭后者的光沙遁,也逃不出方源的追杀了。

%25
耳畔风声呼呼传来,方源操控背后双翼一振,身形陡然拔升上去。

%26
从半空飞到高空后,方源索性闭上双眼,静静地感受着气流划过蝠翼的感觉。

%27
他的仙僵之躯,感受不到任何的痛,触觉方面几乎为零。这既是优点,也是弱点。

%28
方源移植蝠翅的时候,却宁愿花费心思,也要保留蝠翼部分的痛觉、触觉。唯有这样,他才能更加清晰地、真实地,感受每一次振翼飞行带来的反馈,感受到气流的强弱,感受到蝠翅的状态。

%29
丧失痛觉,让人麻木,使得方源更能狂暴地战斗。

%30
但飞行又是另一回事。

%31
飞行讲究细微操纵,尤其是方源的飞行造诣,达到准宗师的地步。只有恢复了痛觉、触觉,才能真正彻底的发挥出他的飞行造诣。

%32
若是僵化的蝠翅,感觉会十分粗糙,尤其在激烈的战斗中,依赖蝠翼的时候,细微的感触便能带来精妙的应对、变化,反之则会粗犷莽撞。

%33
“移动能力增强,便可进可退。集齐三对蝠翅之后,返实蝠翼杀招在凡道杀招中,足以称得上优异,只差我用浪迹天涯仙蛊一筹。”

%34
方源是力道蛊仙,成就力道仙体。浪迹天涯仙蛊则是水道法则,方源使用这只仙蛊,水道、力道并不完美融洽,会有相互牵制的弊端。

%35
若是火道蛊仙催用浪迹天涯仙蛊,速度比方源还要更慢一些。

%36
最适合的就是水道蛊仙,使用同一只仙蛊,消耗等量的仙元,速度却会更快。

%37
更关键的是,返实蝠翼是凡道杀招,比催用浪迹天涯仙蛊,性价比要高出许多来。

%38
这便是人的智慧。

%39
通过众多凡蛊的相互组合,形成稍逊于仙蛊的效用。

%40
用蛊、养蛊、炼蛊,皆是博大精深。

%41
“荒兽尸体难得一见,就算宝黄天中有蛊仙售卖新的兽尸,也未必是蝙蝠。要再收集两对蝠翅,我得主动出击。看来,是时候冒些风险,去往繁星洞天一趟了。”

%42
八转、九转的仙窍,称之为洞天。

%43
繁星洞天的主人,是一千七百年前的八转蛊仙七星子。

%44
七星子早已陨落,方源五百年前世,有一日繁星洞天从中洲高空坠落,分裂成数十块仙窍碎片,散落在中洲大地上。

%45
方源已经是蛊仙,却进不得这洞天碎片里去。

%46
洞天碎块世界,已经十分孱弱,方源修为太高,进入其中,就会将虚弱至极的洞天碎块,直接撑碎。

%47
好在方源有血翼魔教,底下有一干精英下属。

%48
和几位蛊仙一番交手之后,众人瓜分了一块较大的洞天碎块世界。

%49
方源等仙,便派遣各自的蛊师弟子、下属,前往洞天碎块世界中冒险,获取资源。

%50
方源在几位得力下属的脑海中,都留下一股意志,方便监视和指引。

%51
因此他对繁星洞天的景象,较为熟悉。

%52
“今生这个时候,繁星洞天还在九霄之上的高空中,并未陨坠堕落,也没有碎成碎片。我记忆中有繁星洞天的景象,也就可以动用定仙游,提前进入洞天,收刮资源。我记得洞天当中,有不少荒兽,甚至上古荒兽,其中就有一头星魔蝠荒兽。我此番进入其中,杀死星魔蝙蝠,夺得蝠翅。同时还能看看情况,有没有可能捕捉一头荒兽,带给琅琊地灵。”方源心中谋算。

%53
类似繁星洞天这种地方,在方源的记忆中还有许多处。

%54
但不是没到时机,就是十分凶险。

%55
繁星洞天是八转蛊仙的仙窍,还远远未到陨落坠毁的时候,比方源前世经历中还要强盛许多倍。

%56
“我前世,只是派人间接地探索了洞天碎块世界。那个时候,天灵已经不存在了。但今世我若进去探索,天灵想必健在,能牵动繁星洞天的全部防御力量。若非我已成仙僵,还没有资格进入其中探险。”

%57
方源对此时的繁星洞天,知道的不多,只能凭借前世经验去推测。

%58
前世的时候,他也不是这场机缘的最大受益者。凭此番机缘崛起的,是一位八转仙僵。一直以来,强势的十大古派也在这位仙僵手上吃瘪。

%59
后来,这位仙僵入主中洲僵盟分部,轻易击败之前的分部首领,自号星宿仙僵,坐稳第一交椅。

%60
他一度绞动风云,是五域乱战时期,割据一方的枭雄霸主。

%61
而真正的繁星天灵,方源也没有亲眼见到过,更不知道洞天认主的条件。

%62
几天后,黑楼兰带着力气仙蛊,通过星门来到狐仙福地。在方源石巢中,她和方源,以及数千毛民合力,炼制了第八批气囊蛊。

%63
这之后,黑楼兰并未立即告辞,而是停留在狐仙福地中,听方源讲述繁星洞天的一部分情报。

%64
单靠方源现在的实力,要进入繁星洞天,还有些勉强。

%65
于是,方源邀请了他的盟友。

%66
但黎山仙子此刻,正在和秦百胜合作,山盟蛊大放异彩,令秦百胜合纵连横,消弭敌意,并且开始大力筹措一场盛大空前的北原拍卖大会。

%67
秦百胜开给黎山仙子的酬劳,是一只仙蛊。

%68
重利当前,黎山仙子当然选择秦百胜,而放弃了这次和方源的合作。

%69
至于太白云生,他现在正和鲨魔等人一起,下深海,探索玉露福地。

%70
纵然太白云生有心帮助“自家师弟”方源,但碍于身中“死期”仙蛊,他也不能分身,只能爱莫能助了。

%71
唯有黑楼兰空暇,答应和方源一道,共同探索繁星洞天。

%72
按照大雪山盟约,此次是双方共同探险,黑楼兰的任何损失,方源都没有双倍赔偿的义务。同时战利品,按照方源六成、黑楼兰四成这个样子分配。

%73
在狐仙福地休整了几天后,黑楼兰便钻进方源的仙窍。

%74
方源按照脑海中的景象,试着催动定仙游。

%75
一连试了三四次,都没有成功。

%76
方源也不意外。

%77
他记忆中的景象,虽然也是繁星洞天的一部分,但那时已经坠毁下来,景象发生了变动。

%78
好在方源记忆中,景象众多。

%79
试了十几次后,消耗了十几颗青提仙元,他终于成功。

%80
碧光一闪,下一刻,他出现在繁星洞天里面。

\end{this_body}

