\newsection{义天山大战(上)}    %第三百一十七节:义天山大战(上)

\begin{this_body}

%1
禁仙绝境,是惊鸿乱斗台的力量。

%2
这片区域如今笼罩了义天山方圆数千里,任何一位蛊仙进去其中,都会受到巨大的压制,一旦过于深入,就会有性命之危。

%3
赌约成立之前,南疆的蛊仙们当然都进行了试探。

%4
但就算是八转蛊仙,都铩羽而归。

%5
加之迫于竞争者太多的外在压力,这才促使了这份旷世赌约的成立。

%6
但方源却敏锐地察觉到,这片禁仙绝境也有漏洞。

%7
这个漏洞,就是仙蛊。

%8
仙蛊能够在禁仙绝境中使用,只是威力下降得厉害。萧家的太上长老,就将一只仙蛊寄生在萧山的身上,又暗中留下仙元,一定程度上保护着萧山的生命安危。

%9
毕竟,萧山还是萧家太上长老的血脉后辈。

%10
“仙蛊能在禁仙绝境中运用,但是蛊仙却进入不得,难道说这片禁仙绝境针对的只是蛊仙本身?”

%11
方源心中早已经萦绕着这个猜想。

%12
那么,仙蛊和蛊仙之间,主要有什么区别呢?

%13
显然,道痕不是关键。仙蛊是道痕的载体,蛊仙身上也蕴含道痕。

%14
方源很自然地就想到了仙窍!

%15
或许仙窍就是这个最关键的区别。

%16
仙蛊是不会有仙窍的,但蛊仙必然身怀仙窍。如果禁仙绝境针对的是仙窍,说不定方源就可以针对这个漏洞,亲自参加义天山大战了!

%17
“我和别人不同,身上有两窍。第一凡窍中,封印着春秋蝉。第二仙窍,已经死亡,存放着绝大多数的蛊虫。”

%18
“若是我能够将仙窍封锁封印。重新沦为凡人五转蛊师,是不是就能亲身进入禁仙绝境?”

%19
方源越是深入思考,越觉得大有希望。

%20
他虽然没有封印仙窍之法。不过他有智慧蛊辅助,又有智道大宗师境界。完全可以自己推演啊!

%21
“并且我还掌握着转移仙窍福地的方法,并且成功地挪动了星象福地。这方面对我的推算,大有借鉴价值。”

%22
“对了,还有焚天魔女的炎道取窍之法,或许我也可以从中得到更多的灵感。不,我也不一定要死盯着她的取窍法门。她因为空绝老仙的传承得益,才创造出这份炎道取窍之法。我也许能从她那里,间接地得到空绝老仙的某些研究成果。”

%23
方源心思越加活泛起来。

%24
空绝老仙。乃是公认的炼道无上大宗师,和长毛老祖、天难老怪齐名。他对仙窍、空窍,研究最深,至今还未有人超越过他。

%25
若是能得到他的一些传承,相信对于方源推演封印仙窍的法门,必定有着举足轻重的巨大作用。

%26
方源之前投下的赌注,并不多。

%27
按照赌注的标准,他只能选择一位三转高阶蛊师,并且这位蛊师还必须得在三个月后,才能登上义天山的舞台。

%28
他手中的仙材众多。本钱雄厚,完全可以加大赌注。但他毕竟是初来乍到,若是冒然行动。惹来其他蛊仙觊觎、注意,甚至是八转蛊仙出手对付方源,那就危险了。

%29
好在赌约进行的时候,可以随意增加赌注。

%30
八转大力真武仙僵,是方源计划中的必得之物。除此之外,惊鸿乱斗台也让他怦然心动。

%31
毕竟这可是仙蛊屋啊!

%32
而且这座仙蛊屋中,蕴藏着食道的秘密。这么多年过去了,居然蛊虫都没有一只被饿死。

%33
若是从中获取了食道之秘,兴许还能对方源手中的仙蛊喂养难题。有着巨大的帮助作用。

%34
方源悄悄地离开义天山周围,又利用定仙游。秘密回到狐仙福地。

%35
接下来的时间里,他一面留神关注义天山大战。一面则积极推演封印仙窍的法门。

%36
推算法门,进展十分迅速,智慧蛊不愧是九转仙蛊,就算是方源蹭用智慧光晕,带来的帮助也极其巨大。

%37
一个多月之后。

%38
义天山正魔大战,已经声势浩大得卷席南疆凡间。基本上南疆的各大势力、强者,几乎都被这场大战吸引了注意力。

%39
既萧山创建义天寨之后,引来第一波正道攻潮。这是距离义天山最近的超级势力,武家所派遣而来的强者。

%40
武神通。

%41
武家家老,四转巅峰,奴道蛊师。

%42
奴道蛊师向来擅长以一敌众,最怕斩首战术。所以武神通的身边,还有三位四转蛊师,保护他的生命安全。

%43
这三位四转蛊师,有两个都是依附武家的山寨族长。还有一个则是武家家老,贴身护卫武神通。

%44
兽群如潮水一般,攻上义天山。

%45
魔道一方刚刚建立起来的义天寨,被兽群淹没。

%46
许多魔道蛊师惨死,但依靠萧山、孙胖虎、周星星三位五转的强攻,使得武神通重伤败退,两位四转正道丢了性命。

%47
武神通的攻击,虽然被打退,但对刚刚创建的义天寨而言,无异于当头棒喝。

%48
武神通不死,萧山等人寝食难安。

%49
正当他们三个打算孤注一掷,拼着重伤之躯,冒险下山,强杀武神通时,一位魔道蛊师上了山。

%50
并且,他还带来了武神通的首级。

%51
此人姓陆,名钻风,同样是五转蛊师,人称南疆第一神偷。他的背后有魔道蛊仙撑腰,曾经偷偷潜入过铁家的镇魔塔。

%52
陆钻风仗着他神出鬼没的潜行手段,斩杀了武神通,解决了义天寨的危机。正道第一波的攻势,彻底瓦解。

%53
武家并不甘心,很快又命第二位家老出战。并且此行,联合了商家的两位新晋家老:炎突和巨开碑。

%54
第二次正魔交锋,双方互有胜负,僵持不下。

%55
这时,蓝眉鹤、飞鼬王一齐上山,加入义天寨。

%56
这两位蛊师可不简单,乃是南疆蛊师界中。闻名遐迩的飞行大师!

%57
义天寨因此占据上风,将正道诸人逼退。

%58
为了对抗蓝眉鹤、飞鼬王,正道请来女蛊师红飞鱼。她同样是飞行大师,却隶属正道势力。

%59
但红飞鱼一人之力。难敌两位魔道飞行大师的联手。

%60
红飞鱼生死一刻之际,商家的援军赶到,便是白光刀客魏央。

%61
魏央经过商燕飞的提携,资质提升,修为达到四转初阶境地。这一战,他不仅救下了红飞鱼,而且成就了他的威名。

%62
红飞鱼重伤退下之后,他以一敌二。大战蓝眉鹤、飞鼬王,最终成功拖延时间,撑到正道援兵来支援,蓝眉鹤、飞鼬王无奈撤退。

%63
此战之后,魏央得到正魔公认,跻身进飞行大师的行列。南疆三大飞行,成为以魏央为首的四大飞行大师。

%64
魏央的加入,使得正魔大战重新陷入僵持状态。

%65
不过,随着二代僵王走上义天山,奴道再逞威能。正道中缺乏武神通,不得不撤退。

%66
第二次正魔交锋,告一段落。

%67
“看来因为我的缘故。义天山大战也发生了不小的变化。”方源一面关注,一面和前世对比。

%68
五百年前世时,这个阶段应当是义天山牢牢占据上风。

%69
这是因为,王逍加入义天山,打杀了许多正道高手,凶焰极盛,甚至一度威逼萧山,企图夺得群魔之首的位置。

%70
但在三叉山上,王逍被方源杀死。导致今生义天山没有王逍的加入,险些被正道压入下风。

%71
幸亏有二代僵王。提前出场,挽回了魔道的颓势。

%72
二代僵王来自南疆僵盟分部。很有可能是某位南疆仙僵投下来的棋子。他是五转巅峰蛊师,曾经在墓碑山,留下过传承,成就了一位奴道蛊师丁浩。

%73
但今生,丁浩也被方源杀了。

%74
“前世大战,丁浩为了挽救师尊二代僵王的性命,牺牲了自己。现在若是二代僵王遭遇劫难,还有谁来替他挡灾呢?”方源对此有一些好奇。

%75
义天山大战如火如荼地进行下去,但方源推算仙窍封印法门,却遇到了困难。

%76
他面临一个瓶颈,必须要借助其他底蕴,才能迅速跨越。

%77
否则单靠自身努力推算,至少得要十七八年的光阴。

%78
方源就向焚天魔女求教。

%79
焚天魔女手中,有着空绝老仙的部分传承,足以让方源跨越这道关卡。

%80
但焚天魔女却趁机要价,谈到借用方源的仙道杀招见面似相识。

%81
焚天魔女、黑楼兰、黎山仙子等人,当然也想要来分一杯羹。但南疆蛊仙联合一起,进行赌斗,让她们难以插手。

%82
她们是北原蛊仙,一旦出现,就会被南疆蛊仙排挤打杀。

%83
在天下五域当中,南疆是最为排外的地域。东海则最不排外,其次是中洲。

%84
所以,黑楼兰等人要掺和一手,希望借助方源的见面似相识。

%85
方源自然是不愿意的。

%86
借用仙道杀招,和借用仙蛊,是两个不同的概念。

%87
仙道杀招一旦被借用,当中的秘密就会被其他蛊仙洞悉,优劣之处了然于心,很容易就能设想出克制仙道杀招的方法。

%88
方源无奈之下,选择妥协。

%89
他借出见面似相识,得到焚天魔女手中空绝老仙的一部分传承。

%90
不过,他付出了巨大代价,也迅速得到了回报。

%91
跨越了瓶颈之后,他推算法门的速度,再次加快,一路顺风顺水。

%92
ps:之前码了三章,但感觉太拖节奏,所以都忍痛删掉了。今天这两章,其实是大纲中六章的量,压缩起来的。这样一来,前面的坑都填了好多,明天就是*了。嗯……希望大家喜欢。

\end{this_body}

