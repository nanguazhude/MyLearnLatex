\newsection{你四我六}    %第八十二节:你四我六

\begin{this_body}

%1
北原,琅琊福地。

%2
“地灵,我为你带来了一只活着的荒兽,你先看看货吧。”方源说着,打开仙窍,同时催动十几只蛊虫,将一头狮子拿出来,轻轻地放在地上。

%3
这头狮子,大如巨象。浑身褐黄色,浓密的鬃毛几乎覆盖了大半个身子。

%4
它沉睡着,发出鼾声。正是黄玉狮子。

%5
原本它伤势很重,几乎濒临死亡。但只要吊住一口气,凭借方源此时的能力和物力,将它治疗好了不成问题。

%6
琅琊地灵手抚胡须,绕着黄玉狮子看了一圈。

%7
他现在已经活动自由,身上的气道封禁也完全解开。

%8
“好狮子,好狮子,就算是睡着了,也是一脸的凶相。不错。”说着,琅琊地灵伸出手来,拨开黄玉狮子的嘴皮,露出黄玉狮子闭合的牙齿。

%9
“好牙口。”琅琊地灵赞叹一声。手顺着一路摸下去,一直摸到狮子的后腿中间。

%10
整个过程,琅琊地灵一直在默默催动蛊虫,查看了一番后,满意地道:“不错,精力旺盛,可以配种。不过可惜我的福地是炼道福地,对于豢养这些猛兽却是不擅长。”

%11
琅琊地灵养了十二头荒兽,已经养出心得,看其老练的样子,完全不是生手。

%12
但是琅琊福地,是炼道福地,在这里炼制蛊虫,会有更多的成功可能。但对养兽,却没有任何辅助能力。

%13
而狐仙福地,则是奴道福地,擅长豢养狐狸。在狐仙福地中的狐群,有更多的可能会养出荒兽狐。

%14
福地、洞天千差万别,各有专长。

%15
这是因为,福地洞天都是蛊仙仙窍。蛊师升仙时,汲取天地二气,根据各自对天地的理解,用各种流派的蛊虫炸出仙窍世界。

%16
琅琊地灵对这头黄玉狮子十分满意,很爽快地交给方源八十八块仙元石,同时还有杀招见面不相识的详细内容。

%17
方源当着琅琊地灵的面,立即浏览了一番。

%18
这个杀招,来历非凡,源自盗天魔尊。

%19
“不愧出自盗天魔尊之手,别出机枢,奇思妙想。”方源啧啧赞叹,“这里面蛊虫只有三百多只,但涉及变化道、力道、智道等六大流派。嗯……有些蛊虫比较稀少,看来要重现这个杀招,还得等候一段时间,将蛊虫筹集齐全了再说。琅琊,你手中有这些相关的蛊么?”

%20
琅琊地灵摇摇头,嘿嘿一笑:“我手中留着这些蛊干什么?我又用不上这个杀招。见面不相识,见面似相识,见面曾相识,这三个杀招是一个系列的。你拿到的,是凡道杀招,能变化成他人不认识的人物,若是认识的,就要露馅了。后两者呢,都是仙道杀招,扮演认识的人,也可能不露馅。盗天魔尊似乎也留下了这个传承,你要有机缘的话,或许有生之年,真的能够收获其他两个杀招,也说不定呢。”

%21
方源苦笑一声:“我哪有这等运气,偏巧就能得到另外的两道仙道杀招?说不定这个传承,早就被其他人得了去。”

%22
话虽然这么说着,但他心中却不由动了动。

%23
他想到了在北原之行时,收获的那个关于盗天魔尊,关于落魄谷的传承。

%24
他早已经炼成了进入传承的钥匙开门蛊、关门蛊。

%25
但他却迟迟没有动身。

%26
蛊师传承,不是那么好得的。尤其是魔道传承,常常比正道要困难得多,危险得多。盗天魔尊的传承,简直就是龙潭虎穴。情况不明,万我杀招又用不了,方源宁愿多等一会儿,不肯轻易冒险。

%27
方源对其期待并不大,因为之前早已有所揣测“我得到的传承线索,明显不是唯一的。盗天魔尊不愿传承埋葬,因此广收薄种,大网捞鱼。这道传承已经过去了这么久,说不得已经有人光顾过了。别的不说,尤其是幽魂魔尊。他似乎就是利用了荡魂山、落魄谷,修行的魂道。盗天魔尊的这份传承,既然关乎落魄谷,恐怕幽魂魔尊已经进去过。说不定,已经把落魄谷收走了。”

%28
收拾心神,方源取出三张十成仙蛊方,交给琅琊地灵:“这是这一次的交易。”

%29
琅琊地灵接过仙蛊方,查看了一番,确认无误,对方源的天资不禁再次赞叹道:“臭小子,我老人家很少佩服别人,你小子推算仙蛊方的这个才华,的确叫我老人家刮目相看。”

%30
“地灵,我正要和你说这件事情呢。”方源道,“关于仙蛊方的交易,我想要修改一下交易的具体内容。”

%31
“哦?”

%32
“你的仙蛊方完善程度高的,都已经被我推算成功。现在剩下一些完善程度低的,六成、七成,甚至五成以下的都有。我要推算这些仙蛊方,虽然也能够完成,但耗费的时间太长了。动辄一两个月,使我抽不出空来做其他的事情。我建议将这个交易的内容修改一下……”

%33
方源还未说完,琅琊地灵就连连摇头:“不改,不改,为什么要改?我花了这么多的仙元石,不就是想让你推算仙蛊方吗?你是智道蛊仙,又这么擅长推算仙蛊方,为什么不照旧继续下去呢?大不了,我提一点价钱,多支付你一点报酬好了。”

%34
“琅琊地灵,你不要先忙着拒绝,你听我说完,你一定会感兴趣的。”方源自信一笑,“我的建议是这样的,以后推算仙蛊方以你为主,你每遇到一个难题,进行不下去了,我就帮你推算一步,跨过这个难关,每一次只收你两块仙元石。”

%35
“什么?这样交易啊……”琅琊地灵陷入沉思状态。

%36
方源的建议,的确打动了他。

%37
要知道他是长毛老祖的地灵,炼道境界高达准大宗师级,推算仙蛊方也很有一手。

%38
这些仙蛊残方之所以没有推算成功,就是遇到了难关阻碍,想不通,就走不出去。

%39
若通过这个难关,许多仙蛊方或许就一马平川,顺利完成。如此一来,琅琊地灵只需要付给方源两块仙元石即可,比之前的报酬要便宜得多了。

%40
当然如果仙蛊残方的难关比较多,方源或许也能赚更多仙元石。

%41
但琅琊地灵有自信,凭借自己的炼道境界,一定是之前的情况更多一些。

%42
“如此一来,方源小子,你的收入可就低了。啊,我差点忘了,你现在卖胆识蛊,生意火爆,是不是看不上这点小钱了?”琅琊地灵恍然大悟地道。

%43
方源笑了笑:“地灵,这样交易对你也有好处啊。你得到了这么多的仙蛊方,又恢复了自由,可以炼制仙蛊了。而炼制仙蛊,十有八九都会失败,成功的可能太低太低,消耗的资源如海似渊。我这样帮你节省了仙元石,你还不谢谢我?”

%44
“哼,就知道你小子会这么说。”琅琊地灵翻了一个白眼,点头道,“这个我可以答应你,不过我也有一个条件。”

%45
“什么条件?”

%46
“这事情说来话长,你知道秦百胜搞的那场拍卖会吗?”琅琊地灵虽然不能走出福地,但对北原的情报局势,还是十分清楚的。

%47
“当然知道。”方源点点头。

%48
琅琊地灵继续道:“你知道前段时间,有神秘势力攻击我的琅琊福地。他们都失败了,而我也在这个过程中,俘虏了好些蛊仙。我打算将他们都卖掉,准确的说,是卖掉他们的福地。”

%49
“卖福地?”方源诧异地看着琅琊地灵,没想到他的手笔竟然这样大。

%50
“这生意好做么?”

%51
方源神情严肃地道:“比我的胆识蛊生意,还要火爆千百倍!蛊仙们买了这些福地,吞并下去,自己修为增长不说,还省去了好多次的天灾地劫。但你为什么不自己留着吞并呢?”

%52
琅琊地灵摇摇头:“要完全吞并一个六转福地,至少需要相应流派的大师级境界。彻底吞并一个七转福地,需要宗师级境界。而吞并一个八转洞天,则需要大宗师境界。我炼道境界虽高,但对其他方面涉猎不多。当然更主要的原因,是我不想吞并。我的琅琊福地已经达到了福地的巅峰,再进一步,就要升上洞天。洞天的灾劫可比福地大多了。当年琅琊洞天,就是被我精心削减,落为福地的,我又怎么可能再升上去呢?”

%53
方源恍然:“这倒也是。等等,你的条件,该不会就是让我代你出面,到拍卖会上卖这些福地吧?”

%54
“琅琊地灵竖起一个大拇指,有些讨好意味地道:“你这小子就是聪明!”

%55
方源瞪大双眼:“你想害死我吗?不干,绝对不干!我一卖这些福地,将会立即成为众矢之的!太扎眼了,谁有这么大的能耐,活生生俘虏了这些蛊仙?而且我这样跳出去,被那个神秘势力知道了,一定会把我当做和琅琊福地一伙的人。将来进攻你这里,说不定还要打到我那边去!北原墨人城的城主墨坦桑,不是你好友么,让他去就是了。”

%56
“唉,他是墨人成仙,出身不好,一直被人族排斥的。而且他还有家室,有墨人城需要照顾。他就更不合适了。”琅琊地灵连连摇头。

%57
“哦!你不想让他去送死,是想让我去送死?!”

%58
“别说的那么难听么,你不是有了见面不相识杀招吗?只要有这个杀招在,别的蛊仙不可能认出你的真面目的。”琅琊地灵低声下气地道。

%59
方源上下打量了地灵一眼:“我说你怎么这么大方,将这个珍稀的杀招给我,原来你早就计划好的。”

%60
琅琊地灵终于忍耐不住,气得瞪起了双眼,声调一扬:“臭小子,我还不知道你?!你胆大包天,什么时候害怕过?连王庭福地都给你毁了,连八十八角真阳楼都被推倒了,只要利益多,你就是亡命之徒!还怕这等小事?说吧,你要多少?”

%61
方源冷哼一声:“卖福地所得,咱们分了,怎么说,至少也得四六分啊。”

%62
“四六分?你也太贪心了吧,居然一张口就要四成?!”琅琊地灵大吼起来。

%63
方源眼中阴芒一闪,面沉如水:“你理解错了,地灵,我说的意思是我六你四!”

%64
琅琊地灵愣了愣,饶是他活得够久,也被方源的无耻和贪婪的嘴脸惊到了。

%65
他深呼吸几口气,瞪着方源的脸:“这种话,你也好意思说出口?”

%66
“怎么不好意思?”方源脸皮似城墙般厚实,“我可是冒生死大险的,没有让我动心的利益,我是不会做的。这个价钱,已经是最低了,还是我看在咱们俩之间是老熟人的情况下,开的友情价。”

%67
“友情价,我看你是坑熟的吧?”

%68
“这么说是没得谈了?你再考虑考虑,我就是这个价,反正距离拍卖会还有许多时间,我先走了。”方源打开星门。

%69
“你给我滚!”

\end{this_body}

