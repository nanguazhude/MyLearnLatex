\newsection{浩劫下各展其能}    %第三百五十九节:浩劫下各展其能

\begin{this_body}

%1
“黎山仙子,你可以马上联络到焚天魔女吗?”方源目光透过仙蛊屋,紧紧注视战场,询问道。

%2
目前的战况,援兵越多,自然越好。

%3
焚天魔女没有加入新盟约之中,这估计是黑楼兰、黎山仙子的打算,防备方源留的一手。但此时此刻,黑楼兰、黎山仙子都陷入局中,害怕焚天魔女不出手相助吗?

%4
黎山仙子明白又中了方源的谋算,白了方源一眼,当即尝试。

%5
不一会儿,她脸色微微苍白,摇头道:“不行!这处十绝大阵,真是厉害。居然连我的信道手段,都能禁住。”

%6
方源点点头,没有说话,心中暗道:“看来只有冲破大阵,才能联络这个强援了。”

%7
其实,焚天魔女若能赶来,方源还不太放心,将她安置在仙蛊屋内。

%8
若不安置在仙蛊屋内的话,外面凶险,又说不过去。

%9
焚天魔女暂且不做考虑。

%10
有了黑楼兰、太白云生、黎山仙子的增援,影宗方面拾掇方源不下。

%11
而对付监天塔,七转修为的影无邪,又稍显不足。

%12
僵持之中,天劫渐渐增强。

%13
天劫一重又一重,没有丝毫重复。

%14
忽然,天空中金光绽射,光明大放!

%15
啪啪啪啪……

%16
一连串的清脆激响。

%17
密密麻麻,成千上万颗的金刚珠子,陡然从天而降。

%18
“浩劫!”影宗蛊仙们尽皆动容。

%19
“哈哈哈。这是舍利金刚劫!”监天塔内,天庭蛊仙中有人抚须而笑,叫破这波灾劫的跟脚。

%20
此劫已然超脱了十大凶灾的范围。类似雪殇劫电的凶灾,也只是天灾地劫的范围。

%21
而这舍利金刚劫,却是天灾地劫更上一层的浩劫!

%22
蛊师升仙之后。就有灾劫傍身。根据自身仙窍的时间,每隔一段时日,就会产生灾劫。

%23
灾劫祸福相依,渡不过去,蛊仙陨落,渡得过去,底蕴大增。修为上涨。

%24
其中六转蛊仙。拥有青提仙元。十年一地灾,百年一天劫。历经三百年,三次天劫之后,成为七转。

%25
七转蛊仙,掌握红枣仙元。十年一地灾,五十年一天劫。百年一浩劫。历经三百年,成就八转。

%26
而八转蛊仙。仙窍中凝成白荔仙元。十年一天劫,五十年一浩劫。百年一万劫。三次万劫之后,成为无敌天下的九转蛊仙!

%27
地灾、天劫、浩劫、万劫,灾劫威能由小到大,依次递增。

%28
地灾、天劫就已经很艰难了,浩劫更是恐怖。无数蛊仙因此折戟沉沙,仙光永堕,乃是蛊仙强者时刻铭记,念念不忘,日夜防备的修行难关。

%29
至于万劫。修行界中有个俗语,俗称万劫不复,基本上没有蛊仙渡过。一旦渡过一次万劫,都是无上的荣耀,令蛊仙瞻仰敬佩。

%30
人族漫漫历史之中,渡过三次万劫的,只有十位。即十大尊者。

%31
剑仙薄青渡过两次万劫,被五域蛊仙看好,有望成为剑道仙尊。第三次时失败了一次,险死还生。再尝试时,身死道消。

%32
当今北原的雪胡老祖,千方百计想要渡过第一次万劫,因此不惜代价,苦炼鸿运齐天仙蛊。而百足天君、药皇,离一次万劫还有一段距离呢。

%33
别看他们年龄不小,其实都有延缓仙窍的宙道手段。有的蛊仙畏惧天灾地劫,甚至能耗费巨大代价,将仙窍中的光阴长河的支流完全断截。如此一来,仙窍的时间完全静止,灾劫暂时就不会逼近。

%34
但如此一来,仙窍中就无法自产仙元,同时不渡灾劫,蛊仙身上的道痕就无法大幅增长,修为也会停滞不前。

%35
而且延缓时间的宙道手段,都有弊端。静止仙窍中的时间,也无法长久,付出代价更大。

%36
但当今蛊仙界,为了充分准备渡劫,绝大多数的蛊仙都会采用宙道方法,延缓仙窍时间。

%37
就算是蛊仙中的强者,也不例外。

%38
这是因为,修为越高,面对的灾劫威能也随之提高,甚至更加凶险。

%39
天庭中的八转蛊仙,平常时候,都陷入沉睡之中。

%40
这种沉睡的手段,就是天底下最为优秀的延缓法门之一。

%41
对于八转蛊仙,每五十年就有一次生死存亡的巨大考验,这就是浩劫。

%42
如今,苍穹之上就降下第一场浩劫。

%43
舍利金刚劫!

%44
虽然不是针对自己,但天庭蛊仙们惊喜之余,都不由地流露出忌惮之色。

%45
金刚珠子,噼里啪啦地砸进阴云当中。

%46
这浓郁的阴云,顿时以肉眼可见的速度稀薄起来。

%47
很快,最上层的阴云,仿佛是被滚水泼雪,迅速消融。

%48
见此,影宗诸仙脸色都阴沉至极,眉头紧蹙。

%49
“妙哉。”天庭蛊仙们则纷纷微笑,他们也采用拖延战术,这一刻终于收到了良效。

%50
“不用我们出手,或许就能让对方一败涂地了。哈哈哈。”碧晨天大笑。

%51
监天塔主的目光深沉,用饱含敬畏的语气,感慨无限地道:“对方纵然强极一时,但又如何强得过这个天地呢?人应当敬畏天地,和天地相比,纵然蛊仙之能,又算得了什么?即便是九转的蛊尊,又哪一个能得长生?敬畏天地,顺应天意,才是修行之道啊。”

%52
面对浩劫之威,十绝仙僵无生大阵也渐渐支撑不住。

%53
“不能硬挡!”大阵由人主持着,自然比玉露福地中的那些战场杀招要灵活多变。

%54
阴云洞开,主动将金刚珠子泄露进阵内。

%55
啪啪啪……

%56
金刚珠子打在阵内三座仙蛊屋上。

%57
监天塔主冷哼一声。心知此刻己方已被影宗算计,让监天塔替影宗分担压力。

%58
监天塔主自然不愿意。

%59
但他却无可奈何。

%60
身在阵中,身不由己。

%61
他只能催动仙蛊屋,尽量躲闪。

%62
“监天塔中,有着宿命仙蛊。炼制宿命的主要蛊材之一。便是天意!而这些灾劫,也都是上天之意!监天塔,监察天下,是顺应天意,替天行道。因此它在灾劫之下,受到的创伤,天生就比其他仙蛊屋要少很多。”监天塔主开口道。

%63
他的言下之意很明确。就是仍旧执行拖延战术。等到十绝大阵露出破绽。再催动仙蛊屋,施展致胜一击!

%64
对此,其他天庭蛊仙们都没有异议。

%65
“不妙了。此时必须由我们来分担更多的压力,否则仙僵十绝大阵绝不会支撑到最后。”

%66
“就算是万劫又如何?此番逆天行事,我们早已经在东方长凡身上,预知了不少。万劫出现,也在意料当中!”

%67
“牺牲其他仙蛊屋。惟独留下羽圣城,就是用在此刻!!”

%68
影宗蛊仙们都流露出铁血之气,满脸坚毅之色。

%69
羽圣城上,亮起朦胧之光。

%70
这种光,仿佛是粉色,但事实上是一片纯白的光,顶多带一些微粉的晕。

%71
奇妙的是,这种光,让人一看,就感觉心中温暖。

%72
但本质上。这并非是一种光。

%73
而是愿力的逸散时的情形。

%74
每一座仙蛊屋,虽然方方面面都兼顾着,但基本上也都有各自的特长。譬如仙蛊屋黑牢,它的特长就是能擒拿、豢养、奴役荒兽、上古荒兽。若是能提升转数,自然也能收服太古荒兽。

%75
九转的监天塔,则是催发宿命攻势,让任何目标都无法防御。无法豁免。可惜受制于九转仙蛊宿命的状态,此时已经无法发动。

%76
而羽圣城,这座影宗千方百计要谋夺的仙蛊屋,自然也有特殊之处。

%77
它的特殊之处,就在于城内能吸收历代羽民的愿力,自动炼成愿力凡蛊。

%78
这座羽圣城,从很久远的时候,就被羽民们占据。羽民们一代又一代,生活在太古绿天碎片世界当中,存储的愿力凡蛊数量浩如烟海。

%79
虽然被天随人愿仙道杀招,消耗了不少,但仍旧还有大半,没有来得及用。或者说羽民们,没有其他运用的方式。

%80
他们没有,但影宗来历极大,源远流长,自然是有的。

%81
此刻,羽圣城中的愿力凡蛊,就被不断地催动起来,不计后果的剧烈损耗。

%82
愿力蛊是一次性的消耗蛊。

%83
得到愿力的加持,羽圣城的各个方面,都有了大量的提升。

%84
它主动四下飞走,拦截舍利金刚劫,替十绝大阵分担了巨大的压力。

%85
“好厉害的灾劫,只是短短功夫,惊鸿乱斗台要支撑不下去了。”黑楼兰惊呼出声。

%86
黎山仙子一脸苍白之色,不由地目光动摇。

%87
如此恐怖的浩劫,若是她独自面对,也只能挨得过两三颗而已。

%88
而这浩劫中,金刚珠子成千上万,并且还有后续,绵绵不绝!

%89
方源却大笑:“哈哈哈,我等的就是此刻。就让你们看看,惊鸿乱斗台真正的厉害之处!给我吸!”

%90
下一刻,他脑海中念头如烟花爆散,星光璀璨无限。

%91
全部的精神、注意力,都投注在惊鸿乱斗台中。

%92
这座仙蛊屋,可以将外来的打击,暂时用宙道手段封印住,然后自己驾驭,转而对付敌人。

%93
只是这个手段,要操作起来,须得一心一意,要牵扯蛊仙巨量的精神。

%94
方源之前还摸不透仙蛊屋的状况,也无法三心二意,一边操纵仙蛊屋移动,应付敌人,一边发动这个手段。

%95
但现在,有太白云生等其他人的帮衬,羽圣城又忙于应付浩劫,方源终于能抽出手来,启动惊鸿乱斗台的特殊威能!

%96
ps:月票破600,加更一章。

\end{this_body}

