\newsection{薄青仙僵出世}    %第三百四十五节:薄青仙僵出世

\begin{this_body}

%1
方源的问题还有很多。

%2
咨询了琅琊地灵,并未带给他想要的答案。

%3
对于鬼不觉,方源仍然没有确切的线索。目前为止,他只知道鬼不觉是针对魂魄的。

%4
“神不知已经如此绝妙非凡,和它齐名的鬼不觉,应当也不差。”

%5
“其实,相对于鬼不觉,我更在意那股歌声!”

%6
每每想到这里,方源就下意识地皱起眉头。

%7
星宿仙尊在梦境中所唱的诗词,十分古怪。

%8
按照道理来讲,诗词来源于梦境,应当和现实无关。但是方源在落魄谷中的经历,却表明这首诗词,大有现实的意义,近乎于一首预言诗。

%9
“要说预言,这向来是智道大能的拿手好戏。就比如一言仙,预言三尊说。难道这个梦境,也是星宿仙尊留下的预言梦境不成?”方源猜测。

%10
星宿仙尊可是智道第一人,她死后留下手段,接连算计三位魔尊,保住天庭不失。

%11
这等耀眼得刺目的战绩,让她的智道地位稳固如山,后人只能仰望。

%12
她的梦境特殊,有预言的作用,那也并不奇怪。

%13
况且,方源虽然有五百年前世,但那个时候,正是梦道迅猛发展,如火如荼之时,梦境层出不穷,远未被人探索清楚。

%14
这等特殊的梦境,方源不清楚,也并不奇怪。

%15
“歌声寥落,英雄落魄。难挡命途多舛……这梦中诗词的第一句,说的应该就是落魄谷盗天真传,还有凤九歌了。那么接下来,第二句话呢?”

%16
“折剑沉沙,千古兴亡。不尽天河滚荡。”

%17
方源口中低喃。

%18
“天河……落天河?”

%19
“折剑沉沙……是说剑道第一人的薄青?”

%20
“他丧生在灾劫之下,已经无数年,兴起时五域瞩目,风光无限。败亡时尸骨无存,光辉消散。可称得上千古兴亡。”

%21
经过落魄谷之后,方源已经记得全诗。

%22
他推算之后,越发觉得。这首诗的第二句话和剑仙薄青有关。

%23
再结合前世记忆。方源探索落天河的决心,越发的坚定了。

%24
他本来就想去。

%25
虽然危险,但是落天河源头的那些上古、太古级的荒兽、荒植的残尸碎体,可都是上佳的仙材啊。

%26
虽然比不上方源在地沟中,收获的那些八转、准九转的仙材。

%27
但这笔仙材的数量十分庞大,综合来计算,总体价值还要高出方源在地沟中的收获呢!

%28
尽管落天河底。阵亡了大量的蛊仙。但在探索初期,几乎每个蛊仙都大捞一笔,很是发了横财!

%29
“看来薄青仙僵出世的事件中,似乎大有价值可捞。只是此刻还非前往探索的良机。一来,剑光爆发时毫无规律,擦着就伤,挨着就死,太过危险。二来落天河底生活着无数上古、太古荒兽、荒植,暗流汹涌,漩涡遍布。危险重重。”

%30
方源虽有改良过的仙道杀招见面曾相识可用,但此招也有局限。

%31
方源目前,只能变作人形生命,不能变化猛兽等其他物种。

%32
原版的见面曾相识,倒是什么都能变化。

%33
为什么呢?

%34
因为,方源没有炼成变形仙蛊。见面曾相识中,就有变形仙蛊这个核心。一旦有了变形仙蛊。方源才能化作其他物种,脱离只能人形态伪装的桎梏。

%35
但就算有了变形仙蛊,添加进去,方源也不能进入落天河底。

%36
除了上古、太古荒兽荒植之外,落天河中还有无数险恶的天然陷阱,足以要了方源的小命。

%37
只有等到薄青的剑光爆发完毕,落天河中被剑光冲刷洗荡,那些猛兽恶植都被暂时清空,天然陷阱也几乎被全数破坏。这才是进入落天河底的最佳良机。

%38
短时间内,方源也只能等待,没有更好的办法。

%39
再说凤仙太子。

%40
八转蛊仙凤仙太子,乃是北原蛊仙界的巅峰之一。但真实的身份,却是灵缘斋安排多年的中洲间谍。他在蛊师阶段,就潜入北原。一路升仙,成为货真价实的北原蛊仙。

%41
一方面因为灵缘斋的背后支持,另一方面也是他自身的努力和才情,总之他达到了今天这样的地步。

%42
凤仙太子对灵缘斋忠心耿耿。

%43
他清楚,凤九歌对于现在的灵缘斋的战略意义。

%44
凤九歌失踪的这段时间,他十分忧虑。等到他接到回风子的传信,得知凤九歌出现时,他十分欢喜。

%45
只是,他并不知道,这个消息当中的“凤九歌”是方源假扮的。

%46
他等候几天,却不见凤九歌来他这里。

%47
“我是灵缘斋在北原方面的首脑,凤九歌既然已经脱身,为什么不来找我?难道他碰到了什么强敌或者麻烦?”

%48
凤仙太子心知凤九歌的战力,但他也知道凤九歌和秦百胜的激战。

%49
凤九歌被困这么久,才忽然出现。这个情报本身,就暗示着凤九歌狼狈的状态。

%50
“但凤九歌既然已经脱身,为什么不第一时间来找我呢?如今其他中洲蛊仙已经回去,他想要回归中洲,靠他一己之力,恐怕不成。必须得借助我的力量。难道他信不过我?”

%51
凤仙太子正疑惑的时候,接到了来自灵缘斋本部的来信。

%52
信中表明,赵怜云在数日前,成功地收取了神不知,并且获知鬼不觉,就是落魄谷中的那份盗天真传。

%53
凤仙太子既能成就八转,自然也是心思通透之辈。

%54
接到这信之后,他恍然明悟。

%55
“原来如此。”他长叹一声。

%56
有人的江湖。就有纷争。再团结的组织,也存在内斗。

%57
凤九歌的强势太久了,定然引起了灵缘斋内部势力的许多不满。这种不满,在凤九歌失踪的时候,爆发出来。

%58
单单看着赵怜云收取神不知的日期。凤仙太子就知道,这里面是有人故意拖延了时间。

%59
他一下子就明白了凤九歌的打算:“看来凤九歌是按捺不发,想要看清楚门派内部是有何人对他不利。亦或者,他早已经看清楚,只是找寻不到打压的借口,想趁此良机,抓住他们的把柄。将这些人打入深渊!”

%60
“所以。凤九歌他绝不会这么早,主动来我这里。因为他知道,他影响不了我,我会第一时间将这个消息,传回给门派。”

%61
“但我这里最为安全,也是他回归中洲必须依靠的力量。他不是无智之人,是有谋略的英雄豪杰!外有影宗。内有暗斗,很可能他身上还有伤。就算他不来我这里,也必定靠近我这里。一旦有意外发生,他就能及时向我求援。因此我若搜寻,应当先搜寻附近周围。”

%62
想到这里,凤仙太子眼中精芒阵阵闪烁,当即行动起来。

%63
八转蛊仙全力搜索,自然效果不同凡响。

%64
很快,凤仙太子就发现了凤九歌。

%65
“哈哈哈,九歌老弟。你让我一番好找啊。”他状极开怀,拍拍凤九歌的肩膀,大笑着。

%66
“还望太子贤兄勿怪才是。”凤九歌脸色还是有些苍白。

%67
“我懂的。”凤仙太子连连点头,“走!要养伤,还是我的洞天最为安全。”

%68
这位凤九歌,自然不是方源假扮的货色。

%69
两人进入洞天,一谈话。便知落魄谷之事,另有他手。

%70
凤仙太子微怒:“回风子办事不利,应当严惩。至于这收取落魄谷的蛊仙,恐怕来自影宗。”

%71
凤九歌脑海中一闪方源的影像,他沉吟不语。

%72
凤仙太子又道:“见到你时,我就传信回去宗门了。你能活着回来,这就是最大的好消息!可不能让弟妹,还有小侄女悲伤苦候了。等到你伤势养好,我给你举办庆功宴,大庆七天七夜,方才会给你安排回归中洲的事宜。到那时,我们哥俩必须得喝个痛快!”

%73
凤九歌苦笑。

%74
他知道凤仙太子的意思。

%75
对于凤仙太子而言,他身在中洲,是半个局外人,考虑的是整个门派的兴衰和利益。

%76
凤九歌要打压其他门中其他蛊仙,若程度过于激烈,势必会形成门派内耗。凤仙太子并不希望看到这点。

%77
但是一想到妻子和女儿,凤九歌的心又变得柔软下来。

%78
也罢!

%79
“还是太子贤兄,考虑得周到。”凤九歌拱手,说的话也很含蓄。

%80
听到这句回答,凤仙太子笑声更欢。

%81
消息传到灵缘斋,整个高层一片欢腾。

%82
凤九歌失踪,带给这些蛊仙很大的心理压力。凤九歌存在的时候,他们下意识有些忽略了他的作用。这一次,他失踪不明,让灵缘斋的蛊仙们认识到了凤九歌对于整个灵缘斋的重要意义。

%83
凤金煌破涕而笑,白晴仙子的面上也带起微笑。

%84
几家欢喜几家愁。

%85
“凤九歌没有死?接下来的日子,我们俩恐怕不好过了。”徐浩和李君影心情沉重。

%86
他们故意拖延,隐瞒赵怜云的消息不报,此事虽不明显,但明眼人一看就知。

%87
即便凤九歌抓不到什么把柄,但当他回来之后,肯定会对徐浩、李君影不满,有所打压是自然而然的事情。

%88
凤九歌的消息,提前传回灵缘斋,自然就没有了高层会议。也没有白晴仙子调查薄青等事。

%89
不过,薄青仙僵出世的事件,仍旧无可阻挡地发生了。

%90
剑纵中洲!

%91
无数道剑光,震骇仙凡。

%92
等到剑光平息之后,无数蛊仙,开始赶往落天河源头。

%93
不过在他们之前,方源已经率先来到事发地点。

%94
入眼处,血水已经将落天河源头染成一片红色。满眼都是仙材!

\end{this_body}

