\newsection{血毒棠花第七灾}    %第一百七十五节:血毒棠花第七灾

\begin{this_body}

ps:

“炼堂长老今年是耗费心血栽培的,这三人的水准,就算放在中洲也是中上等了。(WWW.qiushu.CC 好看的小说所谓名师出高徒啊。”有长老夸赞。

“呵呵呵,不敢当,不敢当。这三个孩子天赋是有的,更主要的是院长大人大力支持,否则绝不会锻炼出他们三人如此熟练的手法。”炼堂长老态度谦虚。

书院院长沉默不语。

炼堂长老话锋一转,谈及洪易:“说起来,院长的儿子洪易也有这般天赋,如今进入了决赛。关键他还不是炼堂弟子,平时疏于练习,能取得这个成绩,叫人刮目相看。”

众长老相互看看,一边暗暗鄙视炼堂长老溜须拍马如此露骨,一边纷纷附和。

“是啊,是啊。”

“有院长大人的血脉流淌,能弱到哪里去呢?”

“刚刚进入书院时,还不起眼。但现在洪易已然是一群弟子的首脑人物,将来必定也会是一方领袖人物的。”

院长冷哼一声:“犬子哗众取宠,难道我还看不出来吗?他能挤进决赛,是走了狗屎运。广泛涉猎,毫无定性。样样精通就是样样稀疏,待这场比试结束,就让他禁闭七天,好好反省一下。”

众长老默然。

院长大人有好几个儿子,洪易是最好的一个,却是庶出,和父亲关系并不融洽,有着叛逆的性格,因此常受院长的调教打压。

这一次洪易参加炼道比试,也是瞒着他的父亲偷偷报名的。

此刻场中,洪易满头大汗,死死地盯着手中的一团火焰。

在火焰中,蛊虫已经渐渐成型。

“终于我也进入了最后一步。可惜,我的时间耗费太多了!”洪易在百忙之中,观察他人。

待他看到曹宇、谢兰、鲁文三人的火焰已经缩成灯芯大小时,洪易便知道,自己此番要胜,已经无望了。

事实上,他炼道天赋是有的,甚至比曹宇三人更多。但他平时练习很少,一是主修方向不是这个,精力分散,时间有限。二是没有财力支持,虽然有些奇遇,但父亲不支持,只让他专修主道。

“可恶!父亲已经察觉到我的意图,我要让娘的牌位放进祖宗祠堂中,父亲恪守祖宗规矩,怎么可能会同意?他希望我和其他兄弟姐妹一样,乖乖的听话,不去冒犯他的威严。可是洪家对我娘亲,真的不公,不公啊!我若不为娘讨回这口气,真是枉为人子了!”

“也罢。如今之计,我只能兵行险招,冒险一搏了。这炼制红颜蛊的最后一步,其实在炼蛊大师手中是可以一蹴而就的。只是火焰难以掌控,所以他们故意放缓速度。我当然不可能一蹴而就,但速度超过他们,还是有机会获胜的。”

洪易心中下定决心,便立即施为。

不管是围观的弟子,还是台上的长老很快就发觉了洪易的举动。

众人纷纷摇头。

“真是天真呐。”

“洪易想兵行险招,但怎么可能翻盘?除非他是炼道大师级的人物!”

“他当然不是炼道大师,这是自取灭亡啊。快看,他的火焰已经失控了。”

“糟糕!”洪易心中大呼不妙,他手中的火焰忽强忽弱,燃烧时噼啪作响,似乎下一刻就要爆炸。

爆炸的威力并不可怕,毕竟书院出了考题,还要顾及弟子们的安全问题。

“失败了!!”洪易心中一沉,他手中的火焰已经完全脱离了掌控,甚至飘飞出去,离开了他的手掌心。

一时间,洪易满嘴苦涩。

“最终还是失败了啊……阿嚏!”

他最近都是连夜练习炼蛊,临时抱佛脚,却受了凉气,此时放松下来,满身的汗,不禁打了个喷嚏。

喷嚏一冲眼前的火焰,火焰忽的一下竟是灭了。

一只炼制完好的红颜蛊,啪的一声,掉在广场的地砖上。

“炼,炼成了?!”洪易傻眼。

众人石化。

“噗!”一位喝茶的长老,将口中的茶水都喷出来。

就连书院院长,洪易的父亲也下意识地站起了身,满脸都是古怪之色,心中惊异极了:“这,这口喷嚏,居然打出了类似炼道大师的手法,竟然一蹴而就,瞬间跨越最后的难关,将红颜蛊炼成了!洪易这小子……这是什么狗屎运啊……”

狐仙福地。

方源、太白云生、黑楼兰以及地灵小狐仙,联袂立于高空,默默等待着地灾的来临。

“这一次究竟会是什么地灾呢?”小狐仙仰望一旁的主人。

方源摸摸她的小脑袋瓜:“放心,这一次渡劫,和上一次不同。我们有三位蛊仙,又搬走了荡魂山,撤离了许多资源,大大减少了福地的底蕴。并且,我还增添了自身的运道,渡劫的把握已增至八成。”

天道损有余而补不足,讲究平衡。

福地的底蕴越深厚,地灾天劫往往就会越强。所以方源将荡魂山,以及其余的珍稀资源都搬入太白云生的仙窍当中去。

这样一来,就减少了福分,降低了地灾天劫的难度。

在王庭福地一行,他又体会到了运道的妙用。知道自己这个主人运道越强,天劫地灾的威力也往往越弱。

可以说,该做的准备,方源都已经做了。

不过就算如此,还有两成的概率,方源抵御地灾失败!

原因就在于,地灾种类繁多,通常都不会知道地灾究竟是什么。地灾千奇百怪,遇到偏门稀奇的,甚至没有见过的,抵挡的难度就大了。

静静的等待中,地气沸腾起来,地灾终于开始了。

一朵朵的花骨朵,钻破狐仙福地的地面,冒出尖角,几个呼吸之间,迅速生长,一只只鲜红欲滴的花骨朵儿,遍布狐仙福地。

“这是……”众仙迟疑之间,花朵全面盛开。

这些花儿十分巨大,宛若脸盆一样。花瓣柔弱如绸,层层叠叠,一朵花的花瓣至少有六层,有上百片。

“这是血毒棠。”方源沉声道。

他认出了此花,心中满是无奈。

没想到这场地灾,竟然是血灾的一种。这血毒棠从生长到盛开,再到凋零,只有十个呼吸的时间。当它凋谢之后,它的花瓣、根茎都会融化成一汪毒血。毒血污染福地,会造成大量的生灵灭亡,损失通常都会很大。

要克制这种血毒棠,唯有用专门的木道手段。除此之外,打烂盛放的血毒棠,就会立即使这种花化为一滩毒血。

但事实上,方源就算有木道手段,也来不及了。

这么多的血毒棠,遍及整个狐仙福地,怎么克制?

因此,尽管方源这边有三位蛊仙,面对孱弱不堪的血毒棠花海,竟然束手无策!

血毒棠很快凋零,一汪汪的毒血融汇成薄薄的水面。水面不高,只顶到成年人的脚腕,但却蔓延了整个狐仙福地。

福地东部的湖泊,已经全部被污染。西部的狐群、狼群不断中毒,无数狐、狼殒命,尸体倒在血泊中,血液流淌下来,又增添新的毒血。

南部的石人奴隶,在沉睡中被惊醒,许多石人的身上都直接长出血毒棠。花朵凋零之后,毒血流淌,石人发出阵阵哀嚎,但很少有死的。

太白云生心中一阵发凉,不禁仰望苍穹,叹息道:“老天果然不会让我等好过!”

方源苦笑。

黑楼兰则安慰道:“渡劫艰险,如今也算不错了。这漫地的毒血,虽然腐蚀大地,损失严重,但你们看,地气已经平静下来,这场地灾已经算是过了。”

小狐仙则眼泪汪汪:“主人,咱们得尽快清理了这些毒血。土壤已经被污染,今后数年几乎种不出什么东西了。花粉兔、狼群、狐群都损失惨重啊!”

方源旋即出手。

他催动挽澜仙蛊,不断地将毒血吸取到自己的仙窍中去。

他的仙窍已经是死地,死气沉沉,毫无生机,装这些毒血根本不怕污染。

但毒血遍及福地,覆盖范围实在太大,方源就算用了挽澜仙蛊,效率也是低下。

他心中不免有些后悔:“早知如此,我不该将拔山仙蛊融进杀招万我,应该优先融合仙蛊挽澜才是。这样一来,我的效率将大大提升,从另一个方面,减少了仙元的消耗!”

人算不如天算,就是指的现在这种情况了。

足足耗费了一天一夜的时间,方源这才将福地中的毒血全部抽取,放置在自己的仙窍中。

在这期间,他马不停蹄,不眠不休,没有丝毫的停歇。消耗的青提仙元,更不在少数。

毒血泛滥,这是不能拖的。拖的时间越长,腐蚀越深,后遗症就越大。

虽然毒血抽取出来,但是福地表面一层的泥土,已经充斥血毒,手捏上去,一片糜烂猩红。若不加处理,大几十天后,这层土壤就会彻底腐烂,化为毒血,更污染其他泥土。

方源立即决定,将这层泥土铲除。

虽然太白云生手中有江山如故,可以将整片土壤回复到之前的状态。但方源却不取此法。

就这样,方源、太白云生、黑楼兰三人联手工作,又耗费数天时间,终于铲除了这层土壤。方源统一将这些毒土,放入仙窍中去,饶是仙僵之躯,也感到十分疲累。

工程量实在有些大,从这个角度来看,方源当初割舍了福地北部,反而是件好事了。(小说《蛊真人》将在官方微信平台上有更多新鲜内容哦,同时还有100抽奖大礼送给大家!现在就开启微信,点击右上方“+”号“添加朋友”,搜索公众号“qdread”并关注,速度抓紧啦!)

\end{this_body}

