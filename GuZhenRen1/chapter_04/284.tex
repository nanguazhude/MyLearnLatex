\newsection{野生仙蛊,蛊阵痕迹}    %第二百八十五节:野生仙蛊,蛊阵痕迹

\begin{this_body}

在混战中,方源悄然脱身。

重新回到夜叉章鱼的聚居地,这里的夜叉章鱼群还剩下一半。

方源故技重施,又将剩下的夜叉章鱼,努力勾引出去。

“糟糕!又来了一头上古级的夜叉章鱼,还有十头荒兽章鱼。它们几乎倾巢出动了!”夜叉龙帅陷落在混战之中,看到夜叉章鱼的援军,简直苦不堪言。

这是什么仇,什么怨?

其余仙僵亦大叫冤枉,这都什么事?!

偏偏他们还必须先救出星象子才行。

“坚持住!”夜叉龙帅高呼,再也不吝啬仙元,疯狂消耗。

焚天魔女积威甚重,仙僵们都不敢违抗焚天魔女的命令,只能在混战中咬牙坚持。

让他们庆幸的是,这头口蚯居然没有躲到重土带下面去,反而十分疯狂,庞大的身躯扭摆甩抽,掀起一阵阵的狂风气浪。

仙僵们不敢硬抗,只能遥遥攻击,积少成多之下,很快打得口蚯皮开肉绽。

原本这个局势,是相当清晰的。

口蚯虽然是上古荒兽,但也不是夜叉龙帅等人的对手。

但夜叉章鱼群过来搅局,就形成了三方混战。

你打我,我打你,相互牵制掣肘,场面一片混乱。

夜叉章鱼数目最多,将仙僵和口蚯,都包裹进去,形成一层极其厚实的包围圈。

“星象子。你可千万要坚持住啊!”雷雨楼主大喊。

他虽然对方源十分不满,但一想到方源若陨落在此,他就浑身冒冷汗。

方源的安危。让众仙僵牵肠挂肚。

与此同时,当事人方源正施施然地,潜入夜叉章鱼群的驻地。

夜叉章鱼群,往往生活在地沟两旁的悬崖峭壁之上。

在眼前的这块峭壁上,有一块巨大的凸起山岩。这群规模庞大的夜叉章鱼,就在这块山岩上打出一个个的洞穴,成为家园。

方源潜藏行迹。钻入一个较大的洞口。

洞口很是宽大,毕竟要供夜叉章鱼出入。荒兽级的夜叉章鱼。体型就十分庞大了,更别论上古荒兽级的夜叉章鱼王。

方源一进入洞口,便有一股浓烈的油腻气味,扑鼻而来。

地洞延绵伸展向前。朝洞中瞧去,只有一片漆黑。

不过方源侦查的手段,极为丰富,这点黑暗根本难不住他。

他伸出手来,抚摸地洞四壁。

洞壁上面,沾满了黑油,十分粘稠滑腻。

原本干燥的地洞,为何会有黑油?

这个就要从夜叉章鱼的习性说起。

夜叉章鱼,是一种喜欢在地沟黑油河中嬉戏捕食的猛兽。因此它们身上。常年沾满了粘稠的黑油。

出入这些地洞的时候,黑油就从它们的身上,蹭到四面上下的洞壁上了。

方源之前闻到的油腻气味。也正是这些黑油散发出来的。

在黑油当中,还有许多奇特的杂草或者针刺腐木,它们扎根在洞壁之中,黑油赋予它们充足的营养。

咕嘟,咕嘟。

就在方源伸手抚摸,拉扯出一长串的黑油的时候。手边不远处,粘附在洞壁上的黑油表层。忽然冒起了气泡。

然后数十只,比小指头还细微的小瓢虫,接连从气泡中飞出来。

它们被方源的动作惊起,疾飞而出,旋即又落到远处洞顶的黑油之中。

这些黑油,像是胶体,粘性极大,死死地依附在地洞四壁。里面不仅有腐烂的草木,还有大量的虫群。

虫群的规模如此庞大,必然会产生一些野生蛊虫来。

方源只是稍微一扫,就发现自己身边,至少栖息着四只野生蛊虫。分别两只土道蛊虫,一只暗道蛊虫,一只水道蛊虫。

当然,这些都是凡蛊,方源一点收取的兴趣都没有。

“夜叉龙帅能够豢养大量的夜叉章鱼,他的仙窍是否就仿造的这种结构?”方源思维发散,心中遐想。

夜叉龙帅的修行秘密,也让方源好奇。

毕竟仙僵的仙窍是死的,充斥死气,豢养活物,完全格格不入。

夜叉龙帅究竟是如何做到的?

或许深入研究这里的环境,能够窥探到夜叉龙帅的秘密一角,不过现在可不是深究的时候,方源抛开这些想法,继续进发。

半盏茶的时间后。

嗷!

猛兽临死前,发出惨烈的嘶叫。

但这个叫声,只是在很小的范围内回荡,并且迅速消失。

杀死它的凶手,正是方源。

正是他催动蛊虫,才限制了惨叫声音的传播。

这里已经是夜叉章鱼巢穴的深处,不仅洞壁表层的黑油,厚达七八寸,而且还有一些猛兽,在狭缝中,在角落里生存着。

这些野兽,大多以腹黑犬为主。

腹黑犬专门吞食腐烂的肉,还有骨头。夜叉章鱼们捕猎归来,只食用新鲜的血肉,骨头和腐肉是不会食用的。

这些腹黑犬,就吃夜叉章鱼们剩下来的食物垃圾。夜叉章鱼们也乐得如此,正好有清洁工可以为他们清理巢穴。

这是大自然奇妙的生物共存。

方源第一次发觉腹黑犬和夜叉章鱼的互存关系之后,就变作了腹黑犬,深入探索。

可惜这些腹黑犬,各个划分了底盘,领地意识比夜叉章鱼还要强烈。

方源遭到了腹黑犬的猛烈攻击,反不如人形作战来得干脆利索。

如果将地洞巢穴,从上至下,分成两半。那么方源此时的位置,已经跨越了上半部分。到达下半部分了。

然而,关于遗藏的线索还未出现。

地洞每隔一段距离,就有岔道口。夜叉章鱼群的所有成员的住处。都通过地洞,相互联通。在地底下,这样的无数地洞,组合成一个四通八达的路线。

方源继续前行。

他开始发现留守巢穴的夜叉章鱼。

这些章鱼,虽然只是荒兽。方源爆发出真正战力,一定可以拿下它们。

但是方源都一一绕过,哪怕浪费时间。也不和夜叉章鱼作战。

他打杀腹黑犬,可以借助凡蛊。就能遮掩动静,神不知鬼不觉。但是对付荒兽,至少得有战场杀招来掩盖战斗的波动。

他将大多数的夜叉章鱼,都引了出去。

但这里还剩下一头上古级的夜叉章鱼王。数头荒兽夜叉章鱼。

若是把夜叉章鱼王惊动,可就糟糕了。

时间悄然流逝,仍旧是一点线索都没有,方源渐渐有些焦躁。

错过这一次机会,下次进来,可就要费一番周折了。

虽然有定仙游,但也不能堂而皇之地在阴流巨城中催动啊。毕竟,催动仙蛊的动静,可是不小的。

放在蛊仙的侦探范围内。可不只是一些光影效果,仙蛊的气息宛若黑夜中的火炬那般明显。

偏偏这段时间,方源还要留在阴流巨城炼蛊。焚天魔女是不会放任他。脱离自己的视线的。

定仙游是万万不能暴露的。

一旦暴露,牵一发而动全身,方源弄塌八十八角真阳楼的真相,恐怕旋即就要为人所知。

方源此时还不知道中洲的调查队伍已经撤走了。

“实在不行,就只有放弃了。时间有限,万一那边战场停歇下来。口蚯被分尸,夜叉龙帅等人见不到我的踪影。可就难以解释了。”

方源暗自估算。

时间是有限的,顶多只有一炷香的功夫。

现在,他已经耗去了八成时光,仍旧毫无所得。

方源不得不抓紧时间,提起速度。

最终,他来到地洞巢穴的最深处。

在这里,栖息着一头夜叉章鱼王。

呼噜声如雷轰鸣显然,它正在沉睡。

“仙蛊气息!”方源心头微微一震。

他在这头夜叉章鱼王的身上,感受到了一股野生仙蛊的气息。

原来,这片巢穴存在已久,而这头夜叉章鱼王的年龄也是最大的。天长日久的积累,它身上的寄生蛊虫中,衍生出了一头野生仙蛊。

心头的喜悦刚刚消散,方源便感到一阵庆幸。

幸亏他之前捣乱时,没有勾引出这头夜叉章鱼王。

若是这头夜叉章鱼王出现,那群仙僵必定激动得要疯狂。捕捉野生仙蛊的**,能让他们杀红双眼,奋不顾身。

但仙蛊只有一只,不管捕捉的成果如何,他们必定还会前来巢穴探索。

“或许,这就是前世北原僵盟,发现了这附近遗藏的缘由?”方源忽然灵光一闪。

犹豫了一下,方源决定赌一赌。

他变成一头腹黑犬,小心翼翼地接近夜叉章鱼王。

若是这头夜叉章鱼王身上的野生仙蛊,刚巧能侦破方源的身份,那就糟糕了。

不过这个可能并不大。

方源决定一赌。

“巢穴广阔,地洞无数,但这里是巢穴的最深处,最有可能的地方,也就是这里。”

就在这时,夜叉章鱼王相互缠绕着的触角,忽然解开,夜叉章鱼王的眼帘睁开一条缝隙,琥珀色的瞳眸中,正映着方源变化成的腹黑犬的影子。

方源却似毫无察觉,径直绕过夜叉章鱼王的身躯,朝它身后走去,自然无比。

在夜叉章鱼王的身后,是一堆白骨,白色巨骨上还有残留着的腐肉。

章鱼王又闭上双眼,继续沉眠。

方源精神抖振:“是这里了!这里道痕外泄,是蛊阵埋设的迹象……”

仙窍中无数蛊虫被催动起来,仙元疯狂消耗,破解这座蛊阵。

很快,方源就发现,这是一座超巨型的宇道蛊阵,因为年久失修,才露出了蛛丝马迹。

“很有可能,我苦苦寻求的宝藏,就在这个巨型蛊阵之中。但要破解这个蛊阵,非得催动蛊虫,动静很大。看来只有冒险了……”

方源下定决心,随意叼起一块骨头,远离夜叉章鱼王。

他迅速返程,到了某处停下,催动仙蛊,直接向附近的一头荒兽夜叉章鱼杀去。

激烈的战斗,像是捅了马蜂窝,留守家园的夜叉章鱼们蜂拥而来。

方源且战且退,快要到出口的时候,寻到机会,动用定仙游。

三息时间,他瞬移到巢穴最深处。

此时的夜叉章鱼王,仍在地洞出口处。方源争分夺秒,破解这座宇道蛊阵。

夜叉章鱼们在战场流连一番,见不到敌人,躁动渐渐平复,开始往还。

它们在洞中穿梭的速度很快,方源时间极其有限。

更不妙的是,真正着手破解的时候,方源越发感觉到这座庞大蛊阵的繁杂和精妙。

要完全破解,至少得要数月光阴!

ps:之前答应过读者朋友们要爆照,答应过大家的话,不能不算数。就在微信公众号上爆吧。玩微信的,搜索蛊真人公众号,添加即可。

\end{this_body}

