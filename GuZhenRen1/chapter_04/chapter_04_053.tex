\newsection{意外,神秘蛊仙的阻击}    %第五十三节:意外,神秘蛊仙的阻击

\begin{this_body}

%1
方源的提议,由不得鹤风扬不心动。

%2
当方源启动埋伏,鹤风扬便陷入重重包围。他原以为这次行动已经彻底失败,最佳的结果,就是利用拓宇仙蛊,破空重伤突围。

%3
就算逃跑的过程中,仙元消耗巨大,他也顾不得了。毕竟红枣仙元和生命比较起来,还是后者份量更重。

%4
但回到仙鹤门中,鹤风扬的日子也绝不好过。

%5
仙鹤门对他如此期许,他充分准备了一年多,结果却失败了。他的对头蛊仙雷坦,要知道这件事情,恐怕要兴奋地来到他门前嘲笑讽刺!

%6
太上二长老会对他失望无比,鹤风扬恐怕数年都抬不起头来,这个败绩会成为他终身的污点,只有一有机会,就会受到其他人明里暗里的嘲笑挖苦。

%7
如果和方源达成胆识蛊的贸易,对他鹤风扬来讲,远比突围的结果要好上太多倍了。

%8
鹤风扬心中思量:“狐仙福地有四大蛊仙,八大荒兽驻守,再加上地利优势,简直是固若金汤!即便十大古派中的任何一个,都有能力攻打。但也只是有能力,他们未必能抽得出足够多的蛊仙战力来。就算能抽调出来,进攻狐仙福地,打一场蛊仙混战,消耗的仙元等等资源将极其庞大。”

%9
“就算能抽调战力,又有勇气投资大笔军费,但对方也是蛊仙,便可轻易摧毁荡魂山。到那时就算是狐仙福地攻打下来,也极可能竹篮打水一场空啊。”

%10
在中洲外界。一个钧天剑派拥有三位蛊仙,就可和十大古派之一的仙鹤门叫板。

%11
这是因为要对付三位蛊仙,投入的物力、人力就大到仙鹤门需要慎重考虑了。若是失败。不仅是名誉损失,而且巨大的损失很可能引发门派动荡,间接地引发其他问题。

%12
蛊仙击败容易,但斩杀却难。混到这份田地,谁会没有逃生的底牌?

%13
若是打杀不成,彻底得罪了蛊仙。惹恼蛊仙暗算凡人,仙鹤门家大业大。恐怕就要四处救火。

%14
现在方源展露出如此势力,单凭仙鹤门一派,几乎可以说已经绝了强行攻打的希望。除非是几大古派联手。但联手的话。又涉及到方方面面,涉及到利益分配。

%15
若是鹤风扬答应和方源合作,建立关于胆识蛊的贸易,那么这个结果。绝不能算是彻底失败。

%16
因为仙鹤门要夺得狐仙福地的主要目的。就是胆识蛊。

%17
能够用最小的付出,获得最大的利益,这才是仙鹤门各大太上长老们最关心的问题。

%18
在此之前,仙鹤门早就要求方源开放胆识蛊的交易。

%19
“若是能和对方建立贸易,我便不用重伤突围,省下大笔的仙元,回到门派,也有一个交代。”鹤风扬心中考虑良久。

%20
方源察言观色。虽说鹤风扬城府很深,面无表情。但长时间思考就说明了他的意动。

%21
于是,方源继续道:“我的胆识蛊,放到宝黄天中,一百只贩卖一块仙元石。卖给你仙鹤门,一百二十只胆识蛊卖一块仙元石。不过,作为交换,我需要你仙鹤门承认我的狐仙福地,是你门派的附庸势力。且这个附庸,随时可以脱离。”

%22
只要有这个附庸身份,其余九大古派要对付方源,就绕不过仙鹤门。这是正道的游戏规则,相当于仙鹤门是老大,狐仙福地是小弟,打小弟也得先问问老大。

%23
鹤风扬皮笑肉不笑地道:“方源,你这是想拿我派当做你的挡箭牌啊?你方如此势力,完全可以独自建立宗派了,不是吗?”

%24
“任何人,都不能在天梯山上开宗立派。我可不想离开天梯山这块风水宝地呢。”方源笑道。

%25
天梯山曾经直通天庭,就算是蛊仙也不敢在此山动武。仙鹤门若不是宣布方源是门派叛徒,占据清理门户的大义,也不敢贸然攻打狐仙福地的。

%26
“再者说,狐仙福地成为仙鹤门的附庸,也能照顾贵派的名誉不是吗?”方源继续道。

%27
鹤风扬再次陷入沉默。

%28
良久,他吐出一口浊气:“事关重大,我不能做主,还须我亲自回去禀告门派。”

%29
方源点点头:“鹤兄亲自去说,自然最好。不过苍郁仙子还得留在此地,再做客一段时间了。”

%30
鹤风扬和苍郁对视一眼。

%31
苍郁仙子道:“风扬大人,尽管归去,我在这里等候便是。”

%32
“我会快去快回。”鹤风扬向她重重地点头,随后深深地看向方源等人。

%33
他要把这些人,都印刻在内心最深处。

%34
“地灵,开放门户,让鹤兄出去。”方源关照道。

%35
“是,主人。”小狐仙立即脆生应答。

%36
天梯山上,忽然绽放出白金光辉。

%37
光辉中,一道朱红门楼显现而出。门楼高达十丈,有九彩门匾。天空中,粉红祥云汇聚,霞光万道形成光梯。

%38
鹤风扬破门而入,出来的时候,却走的正门。

%39
这番动静,立即引起潜伏左右的九大派蛊仙的关注。

%40
他们交流起来。

%41
“他出来了!”

%42
“这么快?难道说什么像样的抵抗都没有吗?”

%43
“鹤风扬进去的时候,是动用的拓宇仙蛊,出来时已经能走正门。这显然是得手的标志啊。”

%44
“不对劲,你看他的神色,毫无一丝真正得胜的喜色,更非自持的云淡风轻。”

%45
尽管他们心中疑惑,却没有一人直接露面问询。

%46
此番是中洲十大派的暗中交锋,直接面对却是丢了脸面。

%47
鹤风扬无暇顾及他们的感受,一门心思地想快速回去飞鹤山。汇报此事。

%48
他一路疾飞,很快就远离天梯山。

%49
鹤风扬眉头紧锁,心中不甘又疑惑:“方源这个势力。到底是从哪里冒出来的?除了方源之外,其余三位蛊仙都用面具遮脸,用蛊虫阻挡侦察,他们不想暴露行迹,是因为什么?”

%50
“难道说,他们是其余九派里的蛊仙?不愿我派得到荡魂山,讨了这个便宜。因此暗中提前和方源商议好了?”

%51
刚刚想到这里,鹤风扬就将这个念头否决。

%52
纸包不住火,秘密都是暂时的。九大派不会如此干,这有违大派的行事风格。

%53
“那么,他们或许是中洲小派的蛊仙联合?”鹤风扬又想到。

%54
中洲门派林立,除去占据巅峰的十大古派。还有大型门派。中型门派,小型门派,微型门派。成千上万的门派中,当然也有蛊仙。

%55
若真是他们的话,这些蛊仙都在中洲有着根基,防止根基被古派打压,或被古派挑拨拆分,因此不愿暴露身份也说得通。

%56
“对了。方源的手中有定仙游仙蛊!这些人未必是中洲蛊仙,兴许来自外域。蛊仙实力越强。就越难通过域界壁。每隔一段时间,又要落下仙窍,在成仙的地域中,重新汲取地气来稳固福地。但对方有定仙游,却可以解决这些麻烦!”

%57
鹤风扬念头闪烁,很快就猜到了方源等人的身份。不过他证据太少,不能确定。

%58
“总之,狐仙福地的水很深,在探清底细之前,不能轻易动手。我这次栽了大跟头,归根结底,就是敌暗我明,吃了情报缺乏的亏!”

%59
鹤风扬总结经验教训,忽然停下,悬浮高空。

%60
“什么人?居然敢陷我于此!”鹤风扬提起十二分戒备,开口爆喝。

%61
“嗯,警觉性很高。”一个声音飘渺传来,或高或低,开头沙哑的男音,后面又转为清脆的女声。显然来人特意动用蛊虫遮掩了原本的声调。

%62
两道身影浮现出来,一前一后,将鹤风扬夹在中间。

%63
随之变化的,是整个天空。原先晴空万里无语,光明灿烂,现在却转为幽暗,雾气重重,冤魂四处飞舞。

%64
这是战场杀招!

%65
可以营造成一个战斗环境,增幅相应的某些蛊虫。

%66
不是一般的蛊仙,能够拥有的。

%67
“我乃仙鹤门蛊仙鹤风扬,二位遮掩容貌,鬼鬼祟祟,意欲何为?”鹤风扬再喝一声。

%68
堵住他的两位蛊仙,浑身罩着一层暗光,看不清容貌真相。

%69
“仙鹤门又如何?”前面的神秘蛊仙冷笑一声。

%70
后面的蛊仙则更直接:“照杀不误!”

%71
话音刚落,说话的蛊仙就猛地动手。

%72
五指铺张开来,掌心处爆射出数千道灰白丝线。

%73
丝线向鹤风扬急速逼近,鹤风扬心头一跳,这明显是专门用来困人的凡道杀招。一旦被这丝线缠住,想要脱身恐不容易。

%74
鹤风扬连忙唤出鹤群。

%75
他的仙窍中,豢养了海量飞鹤,随身携带,从不缺兵力。

%76
鹤群乌压压一片,铁喙飞鹤,丹火鹤,凤尾鹤,云烟飘渺鹤,星辰极光鹤,幻电鹤混杂在一起,却又井然有序,组成紧密阵型。

%77
一头荒兽九宫鹤飞出,长啸一声,将周围的灰雾震荡开去。

%78
鹤风扬踏上九宫鹤的背,长发飘扬,目光如电,积压的火气彻底爆发出来:“真以为我是泥捏的人物?今日就让你们尝尝我鹤羽飞仙的厉害!”

%79
丝线飞来,铁喙飞鹤群主动飞出,舍身挡住。

%80
另一位神秘蛊仙鼓起腮帮,吐出冲天火焰。这火焰惨绿惨绿,在这战场上,更增威势。烧得飞鹤只只坠落。

%81
鹤风扬不甘示弱,催起奴道杀招。

%82
顿时一群幻电鹤飞出,喷吐雷光,汇聚成柱,贯通战场。又有一群丹火鹤飞出,浑身燃起火焰,伤敌不伤身。

%83
两群飞鹤,宛若两计重拳,分别冲向两位神秘蛊仙。

%84
轰轰轰……

%85
爆炸声响起,双方激烈交锋!

\end{this_body}

