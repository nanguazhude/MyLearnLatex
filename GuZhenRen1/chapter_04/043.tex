\newsection{难以对付的黑家父女}    %第四十三节:难以对付的黑家父女

\begin{this_body}

“黑城,你,你居然反过来威胁我!把我惹毛了,我让你做的龌龊事情大白天下!!”雪松子怒吼道。

黑城冷笑:“那你赶紧去做啊。实话告诉你,苏仙儿就是你们大雪山的棋子,数十年前,故意安排了一场局让我钻,好刻意接近我,打入黑家高层。所谓苏仙夜奔,不过是一场阴谋。不久之后,苏家就被灭门,你以为是我黑家出的手?哼!是你们大雪山为了苏仙儿而收尾,自己扫清了痕迹!”

“什么?这,这我怎么不知道?”听到这个惊人的消息,雪松子表示难以相信。

“你知道什么?你不过是第七支峰的主人而已。大雪山的高层,永远只有三位当家。我知道你不相信,但我手中有确凿的证据。”黑城掏出一只东窗蛊,抛给雪松子。

雪松子接过东窗蛊,投入神念察看,片刻后额头上冷汗涔涔。

黑城仰头望天,长叹一口气,眼眸中流露出一抹萧索之意:“当年,我虽然是公认的蛊仙种子,但却是年少无知,落入了算计。苏仙夜奔,呵呵,苏仙夜奔,不过是正魔两道相互倾轧渗透的一场精心策划的计谋罢了。”

“我和苏仙儿成亲,又有了楼兰,真的以为日子就会如此幸福下去。然而有一日,一位蛊仙忽然来到我的面前,抛给我铁一般的证据。而你所看到的,不过只是这些证据中的一部分罢了。”

“怎么会这样?怎么会这样?”雪松子口中喃喃。

有了这些证据,一切都颠倒过来。

哪怕黑城用阴阳延寿法。算计了自己的爱妻,但这却是正魔不两立,黑城大义灭亲的堂皇举动。绝对不会引来非议。反而会人人称赞。

“你坑我!你竟然坑我!!”雪松子伸手指着黑城,气愤至极。

他原以为自己一直占据主动地位,抓住了黑城的把柄,没想到却反被黑城陷害了。

雪松子傻乎乎地跟在黑城后面,捕猎黑楼兰,又和黎山仙子等人激战,搞得现在他自己处境堪忧。黑楼兰、黎山仙子。以及中洲古派的势力,成了雪松子心头挥散不去的阴影。

黑城收拾情怀,复又将目光投注在雪松子的身上:“而现在。我又得到了新的证据。还记得黑楼兰渡劫的时候,喊了黎山仙子什么称呼吗?黎山仙子和苏仙儿,极可能有血缘关系。我的手中有苏仙儿的鲜血,也有黑楼兰的血。今后和大雪山对质时。又有了一项确凿的证据。”

雪松子深呼吸几口气,强自镇定下来:“好个黑城,果然不同凡响。我前前后后不过勒索了你几十块仙元石,就被你拖下水,跳进了坑。不错,我必须得承认,就算你对黑楼兰出手,千方百计地想擒拿她。也可以推脱为一个父亲管教儿女。希望她迷途知返的期望。只要你不对黑楼兰动用阴阳延寿法,我手中就不会有你真正的把柄!不过你的图谋也到此为止了。你势单力孤,对方可是和中洲势力联手的。呵呵,黑楼兰已经成为力道绝仙,你更加难以擒拿她。”

黑城陷入了沉默。

雪松子越说思维越是敏捷:“我的处境堪忧,你的日子也不会好过。你害了苏仙儿,黑楼兰复仇之心深入骨髓,中洲势力说不定也会找你的麻烦!”

“所以,你们俩个已经拴在了一起,只有与我方合作,你们才有胜利的希望。”一个女声忽然插进雪松子、黑城二人的谈话。

“什么人?!”雪松子大吃一惊。

一个曼妙的身影,渐渐浮现。这是一位女蛊仙,一身紫衫,妩媚动人,她娇丽的脸上带着诱惑的浅笑。

雪松子并非孤陋寡闻之辈,楞了一下,反应过来:“原来是姜钰仙子。”

他旋即就将目光投向黑城。

北原蛊仙都知道,姜钰仙子乃是黑城的第二十七房,同样是暗道蛊仙,拥有仙蛊暗渡,可以隐藏灵机,干扰推算。

黑楼兰之所以能撑这么久,而没有因为大力真武体而自爆,也是因为姜钰仙子出手,动用暗渡仙蛊,为黑楼兰遮蔽封印了十绝体的气息,防止天地感应。

哪知黑城见到姜钰仙子,却态度冷淡至极,哼了一声:“你果然来了。”

“许多年前,当我第一次带着苏仙夜奔的证据,来到你的面前时,就曾经说过一句话――当需要我出现的时候,我便会出现。”姜钰仙子的脸上浮现出神秘的笑容,“怎么样?我之前就说过,单靠你自己是抓不了黑楼兰的,反而会越闹越大。你只有和我方合作,才有希望。”

雪松子看了看黑城,又看姜钰,目光在两人之间逡巡一番。

黑城和姜钰两人对话的态度,让他暗暗吃惊。姜钰仙子的身份很不简单,似乎是某个神秘势力的代表。黑城之所以能够识破苏仙夜奔的骗局,还在于姜钰仙子的提醒。

黑城垂下眼帘,姜钰仙子能够潜进距离他如此近的距离,都没有被发现,这让黑城心中对姜钰的忌惮又加深一层。

姜钰仙子虽然表面上是北原散修,是黑城的爱妾,但实际上却一直笼罩着神秘的迷雾当中。

黑城陷入沉思。

他并不惧怕方源的万我大军,事实上心中还有些许不屑。尽管他没有和万我大军正面抗衡的手段,但蛊仙作战讲究全方面的比拼。单从万我大军追不上黑城这一点上看,万我大军对黑城就没有太大的威胁。

之前一战,若不是黑楼兰一方主动撤退,等到力道虚影消散,黑城毫无疑问地就会占据上风,甚至奠定胜局。

就算是雪松子。被万我大军撵得上下乱窜,心中也不担心将来和方源再战。

原因就在于:万我是仙道杀招,方源投入的仙元太多。却无法真正杀敌。雪松子只要动用凡道杀招,就可以支撑下来。连续消耗几次,作为仙僵的方源还能有多少仙元?一旦哑火,就是雪松子发威,收拾方源的时候了。

黑城心中忌惮的,是方源的背景,是仙鹤门。是中洲蛊仙。

他也没有料到,调查定仙游后,会牵扯到这么一个庞然大物。八十八角真阳楼的倒塌。已经被东方长凡算出,是中洲蛊仙在搞鬼。

中洲蛊仙既然连巨阳仙尊的布置,都能破坏。何况对付区区一个黑城呢?

和黑城一样,雪松子也有相同的忧愁和担心。

他们都被方源的“背景”吓着了。若是知道仙鹤门正要千方百计地对付方源。他们绝对不会这般紧张。

“如今看来,似乎只有和你们合作,才有成功的可能。不过在合作之前,作为基本的诚意,你是不是应该告诉我,你所代表的究竟是何方势力?”黑城思考之后,对姜钰仙子问道。

姜钰仙子沉吟片刻,她知道黑城是个难以对付的人。若是撒谎或者拒绝,恐怕对方就会立即拂袖而去。

于是她决定实话实说:“告诉你们也无妨。我所代表的势力,广布五域,名为――影宗。”

黑城、雪松子面面相觑。

中洲,狐仙福地。

“二十八颗青提仙元,半块仙元石。”方源检视自己手中的财富。

经过黑楼兰渡劫一战,方源前段时间,好不容易积累起的资本,立即被打回原形。

大战之前,方源的青提仙元多达九十一颗,但万我杀招催动了近五十次,形成近五十万的力道虚影大军。

激斗中,方源又屡次催动乐山乐意仙蛊、浪迹天涯仙蛊等,消耗了十几个青提仙元。

因此,剩下的青提仙元不足三十颗。

“不仅如此,我还欠黎山仙子十五块仙元石,琅琊地灵十五块仙元石。黎山仙子的欠债能拖多久,就拖多久。欠琅琊地灵的仙元石要尽快还上,时间拖得越久,利息就越高。”

方源和琅琊地灵的关系,还只是一般。

琅琊地灵借给方源仙元石的时候,虽然并未狮子大开口,但仍是按照惯例,收取一成的利息。

也就是说,方源至少要还给琅琊地灵十六块半的仙元石。若超过一个月,就要再此基础上,再添一块半。

两个月,就是三块仙元石。以此类推。

就算没有超过一个月,仍旧要多交付一块半仙元石,这是所谓的保底利息。

没有利益可言,旁人怎会无缘无故地外借?

本来,方源和黎山仙子等人的盟约中,有过约定。一方支援另一方,付出的代价可以得到双倍的补偿。

这个约定,在之前方源一战黑城时,就履行过。

但方源这次不仅仅只是付出代价,还得到了巨大的利益。

他的力道境界上升到宗师级,变化道境界简直是从无到有,暴涨到大师级,飞行造诣提升到准宗师级数。

还要再加上小家子气蛊,免除的十五块仙元石欠债,这些利益远远超过方源付出的代价。因此黎山仙子方面,不会做出补偿。

“现在回想起来,这个黑楼兰恐怕是故意设局,企图利用魔尊真意,让我卖命。”现在方源回想起来,觉得是落入了黑楼兰的算计当中。

若非魔尊真意,方源绝不会消耗如此多的青提仙元,去形成力道虚影大军。若是战局不佳,方源完全可以撤退。毕竟盟约中,可也没有死战不退的条款。

他和黑楼兰等人立下的雪山盟约,还是很宽松的。

关于这点,方源在战斗时,就隐约有所察觉。不过这是堂堂正正的阳谋,就算方源发现了黑楼兰的企图,他也要一头钻进去,毕竟甜头太大了。(未完待续……)

\end{this_body}

