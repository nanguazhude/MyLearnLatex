\newsection{地灵认主,第二片福地}    %第二百零二节:地灵认主,第二片福地

\begin{this_body}

得到断臂之后,方源小心翼翼地收入仙窍。9;\&\#32;\&\#25552;\&\#20379;\&\#84;\&\#120;\&\#116;\&\#20813;\&\#36153;\&\#19979;\&\#36733;\&\#65289;

随后,他细心打扫,尽量扫除自己留下的痕迹。

回到狐仙福地,他也没有休息,马不停蹄地开始施为,前后耗费了四天三夜的时间,才将这断臂炼化,最终转变成宋紫星的首级。

然后他利用定仙游仙蛊,去往地渊。

“星象地灵,我带来了宋紫星的头颅!”在幽暗的地下世界,他高喊道。

眼前先是出现了一点星光,旋即星光大盛,宛若泉水一般喷涌而出。

一个呼吸之后,绚烂的星光形成一道四方大门。半透明的星光大门,悬浮于空。门头有一个门匾,上书四个大字星象福地!

方源眯着眼睛,静静地看着。

一般的福地洞天,都有门户。因为有门户,才代表有交流。有交流,才能汲取天地之气,壮大自身。

所以不败福地是非常特殊的,因为它没有门户。这是非常奇怪的特例。

在方源的打量下,星光大门敞开一条小小缝隙。

“是你?你真的带来了宋紫星的头颅?”从门缝中传出声音,星象地灵却没有出现,有着防范之心。

方源呵呵一笑,直接将头颅扔了过去。

♀长♀风♀文♀学,☆.∧.n↓et

事实胜于一切雄辩!

大门缝隙倏地又张开一些,一股吸引力凭空产生,将头颅立即吸了进去。

“真的是宋紫星的头!”门内侧立即传出地灵惊喜交加的声音。

轰!

大门彻底敞开,星光铺就的大道在方源面前徐徐展开,星象地灵匍匐在门口。心悦诚服地对方源道:“星象地灵恭迎主人。”

“嗯。”方源点点头,脸色平静。迈开大步,踏上星光大道。跨进星象福地中去。

踏进去的那一刻,星光充斥方源的视野。

但一眨眼功夫,星光消散,显现出星象福地的独特地貌。

整个星象福地,就是一个巨大的盆地。

仿佛是一个巨碗,巨碗的边缘是一连串连绵的山脉,围成一圈,仿佛高耸的围墙一般。<strong>棉花糖小说网Mianhuatang.cc</strong>

而巨碗内,是向中央不断倾斜的广袤坡地。

夜幕笼罩这方世界。方源仰头望去,天空漆黑一片,但也有细微的小星点,虚弱地闪耀着。

就像狐仙福地只有白天,没有夜晚。星象福地只有夜晚,没有白天。一般而言,福地都没有天象的变化。只有成为洞天,或接近洞天这一层次,才有天象变化。

这不是方源第一次来到这里了。

事实上。这片福地还是他种下来的。

上一次来,这里的夜空繁星,还很光亮,星辉灿烂。映照福地地表,可见度很高。但现在确实星光晦暗,可见星象地灵的小日子过得并不好。

这是常情。

没有了蛊仙。福地已经不能自产仙元,地灵和福地一体。要维护自身的存在,就要经营福地。

但地灵哪有蛊仙的力量和智慧呢?

所以基本上。地灵都会寻求主人,帮助自己。生存是任何生灵的最基本的本能。只是各个地灵的认主条件,有难有易罢了。

方源目光在周围重重扫视一眼,旋即落到身旁的星象地灵身上。

“带我巡视福地。”方源命令道。

星象地灵是个男童模样,悬浮在半空中。他身穿一个粉蓝色的小肚兜,小胳膊小腿白嫩如藕,粉雕玉砌,十分可爱。

自从认主之后,星象地灵都仰望着方源,清澈的瞳眸中尽是仰慕敬爱之色。

听到方源的话后,星象地灵连忙点头:“主人,请跟我走。”

认主前后的态度,简直是天壤之别。

一团星光自动汇聚到方源的身上,带着他的身躯飞速飚射。

这速度惊人至极,就算是宋紫星的血虹闪,都比不上。

方源只觉得眼前星光缭绕,周围景色迅速倒退,依他的视觉都只能捕捉到模糊的影像。

星光倏地一停,眼前的景色已经大变。

“主人,您已经到了福地的东南角上。这里是箭竹林,福地中规模最大的豢养地。”星象地灵脆生生地介绍道。

方源展目望去,只见这片箭竹林规模很是庞大,蔓延出方源的视野尽头。

箭竹根根笔直,有点类似青矛竹,但并非青矛竹的碧绿,而是漆黑的竹身,竹子上没有任何的叶片,只有一条条的枝条。每一截的箭竹的纹路边缘,都长有三两根的竹枝。这些竹枝根根笔直,宛若箭矢一样。

在箭竹林的脚下,有大量的星洞石,无数的暗香菇,还有密密麻麻的白影草。

方源看得暗暗点头,这样的生态建设,已经做到不能再好。

“如果我没猜错,这里是产出星镖蛊的地方吧?”方源道。

“主人明鉴。”星象地灵立即回答。

星镖蛊只是一转凡蛊,由各种普通昆虫进化形成。这和星萤蛊的产生不同。

星萤蛊是由星萤虫群中的一部分,升华变异产生的。

而星镖蛊,这是任何种类的昆虫,钻进表面布满洞口的星洞石中,陷入沉睡。经历七七四十九天之后,有一部分的昆虫从沉睡中醒来,就形体大变,化为星镖蛊。

星镖蛊的食物,就是箭竹林的竹汁。

箭竹林中向来空气湿热,这种竹子本身都会慢慢地向外渗出竹汁。

“以前的时候,这里竹汁渗出,还会形成浓郁的蓝雾,是豢养星镖蛊最好的环境。可惜最近一直都没有得到很好的浇灌,就成这个样子了。”星象地灵用怀念的语气道。

“看来以后都需要大批采购月井水。”方源口中喃喃,他见多识广,知道箭竹林需要什么灌溉。

“主人明智。”星象地灵立即道。神情变得微微振奋起来。

万象星君虽然留下来不少的仙元,但仙元石却是很少。因而星象地灵财政颇为紧张。不能在宝黄天中肆意采买想要的东西。

相比较灌溉箭竹林,其他的地方更需要财力维持。

随后。在地灵的带领下,方源又去了福地的正北方,视察了陨石群坑,这是豢养星火蛊,乃至产出流星天陨蛊的地方。

再去了福地的正中央,是一片湖泊,波光粼粼,面积广袤,名为碎星湖。也是蛊虫豢养地。产出星河蛊。

最后去了几片草场,全都是星屑草,漫漫无涯也似。种植在黑土地上,草叶肥嫩,色泽如油,比狐仙福地中的空中草场,不论规模还是品相都要好得多!

“箭竹林、陨石群坑、碎星湖、星屑草场。”方源悬浮在空中,望着脚下这片湖水,脸上露出满意之色。

以上四者便是万象星君的经济四大支柱。

万象星君中规中矩。以贩卖星镖蛊、星火蛊、流星天陨蛊、星河蛊、星萤蛊为主。

在方源苦苦寻求四大支柱的时候,他早就已经有了。方源目前也只是有胆识蛊、长恨蛛的贸易,剩下的幽火龙蟒和龙鱼两项,还未挤入市场。

万象星君乃是六转一阶的修为。即是渡过了一次天劫。他也是散修,但修为方面比狐仙要高一截。

本来按照他的身份和底蕴,是不可能将福地发展得这么好的。

但他有奇遇。

他的奇遇就是繁星洞天。繁星洞天的主人七星子。为了延寿,转为仙僵。又探索星宿仙尊梦境,陷入其中。被困沉眠。

万象星君意外地发现,繁星洞天联通中洲外界的空间缝隙。每一年的固定时间,这个缝隙都会产生。

于是万象星君每年都进去,搜刮里面的资源壮大自身。虽然每次进去的时间都很有限,但架不住次数多。

万象星君因此发展得很不错,将自家仙窍福地经营得蒸蒸日上,远超其他散修蛊仙一大截。就连石磊见到他,也曾在心中暗赞,这万象星君的实力已经和十大古派中的寻常六转,差别不多。

但可惜的是,万象星君被宋紫星打伤,又渡劫在即,不得不铤而走险,与虎谋皮。他邀请了仙猴王石磊,一同探索繁星洞天。结果第八星殿中,陷入星宿仙尊的梦境,魂魄毁灭,只余肉身。

方源夺了肉身,将其福地种于地渊之中。

这才有了星象地灵认主的种种后续发展。

“有了这四大经济支柱,我的财力比之前直接翻上一番!日进斗金,真的是要日进斗金了!”方源心中喜悦。

万象星君经营福地这么多年,早已经打开市场,在宝黄天的市场中占据独属于自己的份额。方源继承他的福地,也就免掉了挤进市场的艰难发展期,直接继承了万象星君的渠道,省去了他好多麻烦!

类似星镖蛊、星火蛊这种常规资源,一般都是走量,有着激烈的竞争。想要抢占市场,分一杯羹可不容易。会面临排挤、价格战,甚至蛊仙的威胁袭杀等等。

一句话,断人财路如杀人父母。

蛊仙世界**裸,你想要做生意,就得先展现实力。没实力,生意做得再好,也是被抢杀被掠夺的料!

所以方源对于手头中的龙鱼、幽火龙蟒,一直很慎重。想要借助萧家的交易令,再挤入西漠的市场。如此才能少惹麻烦,多赚些钱财。

当然,类似胆识蛊的垄断买卖,就随便你卖了。别人想竞争,也竞争不起来,只能眼巴巴地望着。

巡视了一番福地,让方源十分满意,不过他还有更大的期待。

“地灵,福地中都留了那些仙蛊,带我去看看!”下一刻,他吩咐地灵道。(未完待续……)<!--80txt.com-ouoou-->

\end{this_body}

