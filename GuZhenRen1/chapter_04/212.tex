\newsection{炼制成功}    %第二百一十三节:炼制成功

\begin{this_body}

星象福地。

魔雾缭绕,毒血蒸腾,已经将三层蛊阵都腐蚀破坏,周围地表都化为了一层浅浅的毒泥烂沼。

方源满脸郑重之色。

“接下来,就是最难处理的仙材地极天罡了。”

他取出一份仙材,拿在手中。

这份炼蛊材料十分奇特,是由泥和气组成。泥气自发地拘束成一团。

上面是淡青色的罡气,下面是黑色的泥土。

罡气是九天之上的天气。太古九天之外,都有一层厚实的罡气墙。蛊仙要进入九天中探索,往往就得突破罡气墙。

而黑泥,则是十地之下的浓郁地气凝聚成的精华。

天地二气本身就难以共存,但是在此刻,这地极天罡中两者却达成和谐的统一。不仅和平共存,而且相互之间不断转化。不断有黑泥化为罡气,又不断有罡气化为黑泥。

方源手掌摇晃一下,这团地极天罡迅速浑浊,黑泥罡气混淆一块,形成一团灰雾缭绕。

但不摇晃,静置十几个呼吸之后,黑泥就会沉淀下来,罡气则在上。又出现黑白分明,相互微微循环的奇象。

“处理这种仙材,最是麻烦。寻常的炼道杀招,都不能完美处理。唯有用公认最强的,处理仙材的四大仙道杀招静眠电蟒,映雪,闷雷石鼓,风磨,方可一蹴而就。可惜这四种杀招我都没有。要处理地极天罡,只有卖力气,下苦功了。”

方源心中念头一闪,脚下一蹬。雄躯顿时拔空,轻轻一跃,整个人便跳进了龟壳之中,毒血之内。

噌!

方源亮出尖锐的指甲,分别在六只怪臂上切出伤口。又在自己的胸膛。后背等处,戳出伤口。

血炼杀招血丝游。

从这些伤口中,游出一丝丝的血迹。

血迹很快就融入深紫色的毒血当中,旋即龟壳大锅中的这些毒血,仿佛被牵引一样,开始从方源的伤口里钻进去。

剧痛传来。方源闷哼一声。

仙僵是没有痛觉的,方源能感受痛楚,自然是用了蛊虫手段。他需要通过感知痛楚,来明白仙材处理到了什么程度。

方源的血液和龟壳大锅里的毒血,不断交融。形成循环,在方源的身体内进出不断。

这个过程变得稳定之后,方源将早就取出来的地极天罡,一口吞下。

咕咚一声,地极天罡被他吞入腹中。

这是他前世的独创,血道炼蛊的诡谲法门。他将这个命名为肉身血炼法。

地极天罡进入他的身体内,不断地被血液冲刷,微微溶解在血液中。

这些血液。又透过方源浑身上下的伤口,从体内流出去,汇入龟壳大锅之中。沉入锅底。

同时,大锅内的其他毒血,则通过伤口,流入方源的体内,再冲刷地极天罡。

如此循环,地极天罡以极为缓慢的速度。不断消融缩减下去。

真阳山脉中的一处山头上,大风卷席。呼呼吹鼓。

毛民蛊仙余木蠢忽然伸手一指。

在他手指的方向,五六个龙卷风。便开始缓慢的靠拢。接近到一定的程度时,忽然合而为一,形成一个巨大的龙卷风,呼啸旋转。

余木蠢手掌一翻,取出一颗石头珠子也似的仙材。手指屈弹,将这颗石头珠子射进巨型龙卷风中。

这石头珠子虽然很小,很不起眼。但射入龙卷风中之后,却是发出嘈杂刺耳的声响。

龙卷风的呼啸声都减弱了许多,也转得不快了。仿佛是一个壮汉吃撑了的感觉。

不过随着时间推移,龙卷风越转越快,在迅速地“消化”珠子。

半炷香之后,龙卷风完全消化掉了珠子,整根风柱被染成黑白二色,忽白忽黑。

“地极天罡珠?”本多一看到这里,忽然灵光一闪,脱口而出。

余木蠢有些意外,不吝夸奖道:“嗯,你小子见识不浅。要处理地极天罡珠,依靠刚刚的小龙卷风是不行的。只有大龙卷才有能耐。”

本多一强忍心头的震动。

地极天罡,已经是极为难以处理的仙材。本多一心知肚明,就算自己拼尽全力,也得耗费数年光阴,才能慢慢地腐蚀掉一团地极天罡,用于炼蛊。

他是凡道蛊师,要处理仙材,通常都是以年计算的漫长时间。

而地极天罡珠呢?

则是大量的地极天罡,相互挤压,渐渐凝聚起来的精粹珠子。一颗珠子,至少得抵上百份的地极天罡。

地极天罡珠最是坚硬,极难处理。就算是蛊仙,也要动辄数月、一两年的时间,才能消磨一颗珠子。

但余木蠢大师只用这么短的时间,就处理好了一颗地极天罡珠?

本多一忽然想到了什么,声音颤抖地道:“难道,难道余大师您用的这个炼道杀招,就算传说中的风磨吗?应该是了,只有传说中的四大仙级炼道杀招,才能将仙材处理到这种程度啊!”

“不错,正是风磨。”余木蠢回应道。

本多一眼中一片火热。这可是处理仙材的最佳手段之一,就算他不是蛊仙,根本掌握不了仙道杀招。此时内心中也充满了羡慕,幻想着将来哪一天,我能用出风磨来完美地处理仙材,那该多妙啊!

地极天罡珠处理好了,余木蠢开始向眼前无数的龙卷风中,投放仙元石。

一颗颗的仙元石,被扔进龙卷风柱中。仙元石没有地极天罡珠那般坚硬,旋即就被风刃切碎磨成粉,爆闪出一阵阵耀眼的青芒。

这时天地交感,附近的群山开始微微震动,天空中乌云密布,响起声声闷雷。

“怎么回事?”本多一惊慌失措。仰头四望,发现他置身的这处山丘,周围的空间都浮现出一道道的光线。

这些光线,自然便是道痕。

有的是红色的炎道道痕,有的是蓝色的水道道痕。这两种道痕大多残缺。较多的是土道、木道道痕,比较完整。更多的却是一种闪烁着银光的道痕。

这是律道道痕!

这些道痕密密麻麻,但排布并不均匀。有的相互纠缠叠加在一起,显得较为密集。有的稀稀拉拉,比较稀疏。

本多一又发现,余木蠢选择炼蛊的地方。却是律道道痕集中最多之地。

“这片山丘看似普通,原来大有不凡之处。难道余大师会选择这里进行炼蛊!奇怪,普通的山川肯定没有这么多的道痕。”本多一发现的越多,心中就有更多的疑惑。

他心头越加压抑,因为群山震动的越加强烈。天上的乌云更加浓厚。

风雨欲来,大难将至!

就连那些龙卷风柱都偃旗息鼓似的,体型微缩。

余木蠢却昂首身上高空,哈哈大笑,笑声中充满了自信和豪迈。

他炼制仙蛊到了这一刻,终于迎来了最关键的一步。

“来吧。”他忽然一手指天,一手指地。

天空中,忽然电光激闪。电光如水,汇聚出一条巨大如龙的电蟒。

雷电是阳,蟒蛇是阴。雷电组成的巨蟒。蕴含阴阳,消去了雷电的狂暴,反而显出温和的样子。

这是静眠电蟒!

四大炼道仙级杀招之一!

而在地面上,无数的龙卷风柱相互合拢,很快形成唯一的超巨型龙卷风柱。

风柱高耸,几乎连天接地。静眠电蟒慢慢地游动起来。缓缓下降,缠绕住巨型风柱。

电光丝毫都不耀眼。巨蟒绕柱的过程,显现出一股难以言述的惊心动魄的美感。

这时。天地二气忽然升腾而起。

一股极其强烈的危机感,充斥本多一的心头。

“余大师,怎、怎么回事啊。怎么这情形,像是记载中蛊仙要渡劫的样子?”本多一大叫。

“啊,你猜对了。是个小麻烦。”余木蠢淡淡地道。

“小麻烦?!”本多一眼睛瞪圆,口干舌燥。

星象福地。

方源浑身是伤,脸上痛得狰狞扭曲,八根怪臂都插入毒血之中,獠牙外龇,双眼赤红一片,喘息如牛。

一共耗费了三天三夜的时间,他终于将地极天罡处理掉了,完全融入到毒血当中。

毒血原本满满一锅,如今却只剩下一半不到。

“最艰难的一步已经渡过,地灵,将那些俘虏都抛进来。”方源怒吼一声。

星象地灵连忙答应。

一时间,大量的生物,譬如牛马猪狗等等,都抛入毒血之中。

惨叫声,哀嚎声,怒吼声连成一片。

毒血极为浓稠,仿佛是沼泽一般,这些生物被抛进来后,极力挣扎,却只会沉得更快。

他们的血液、肉、骨,都被溶解分化。

很快,龟壳大锅内的毒血水位开始慢慢上涨。

“还不够,还不够。”方源大叫催促,赤红的双眼流露出兴奋和残忍。

地灵没有善恶,只有固执和忠诚。

星象地灵又开始抛入大量的异人。这些异人有毛民,有石人,有雪人,有墨人,有蛋人,有羽民,有鲛人……

毒血水位迅速上涨,方源脸上还是不满足,他心中掐算着时间,仍旧大喊:“再多,再多一些。”

于是他特意准备的人类俘虏,也被抛入了大锅内。

“啊,饶命啊!”

“好痛,痛死我了!”

“我做鬼都不会放过你的!!”

方源无动于衷,只是关注着炼蛊的进展。一个蛊师老者恰巧被抛到他的身边,挣扎欲起。

方源抬起脚,将老者踩进毒血深处,老者剧烈挣扎了几下,最终毒血表面上只留下老者一只手掌,手指如勾,抓向天空,仿佛是弱者对天地,对命运,对方源的强烈诅咒和控诉。

方源嘎嘎大笑,在有限的时间内,毒血水位终于恢复到了原先的状态,和龟壳边缘平齐。

他开始向毒血中抛入大量的仙元石。

一百块,两百块……毫不犹豫。

无数的冤魂在毒血上空缭绕,毒雾翻滚不休,毒血再次缓慢下降。

直到七天之后,毒血终于降至底部。整个龟壳里只剩下一汪毒血,连方源的脚踝都覆盖不到。

方源缓缓俯身,从这汪最汪毒血中拾出一蛊来。

变形仙蛊,成了!

中洲,真阳山脉。

本多一瘫坐在地上,口中喃喃:“终于渡过灾劫了……”

山丘已经崩塌了一大半,宛若战场般狼藉不堪。

不管是龙卷风柱,还是静眠电蟒都已经消失无踪。

大雨倾盆而下,将本多一浇成落汤鸡。

余木蠢将炼成的仙蛊,收入怀中。然后又抛下一只信道蛊虫,留给本多一。

“小子,这是我的炼道传承,留给你。自然炼蛊法你可以学习,但若不成蛊仙,万万不可动用此法炼蛊。因为这会招来天灾地劫的。”

余木蠢说完这话,凌虚踏空而去。

本多一浑身一颤,连忙跪下,双眼迸发出闪亮的光来,大喊:“师傅,你放心,我一定不辜负这份传承!”(未完待续)

\end{this_body}

