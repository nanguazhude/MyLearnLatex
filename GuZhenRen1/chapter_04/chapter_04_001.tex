\newsection{回归狐仙福地}    %第一节:回归狐仙福地

\begin{this_body}

%1
缓缓睁开双眼,方源醒来。

%2
入目的是莹润的粉光,晶莹的山壁,方源恍惚了一下,这才反应过来——自己已经通过星门,从北原回归了狐仙福地。

%3
只是北原之行,劳心伤神,艰险无比,堪称九死一生。方源纵然是六臂天尸之躯,心中也充满了疲惫。

%4
回归狐仙福地之后,他稍稍布置了一番,就倒头呼呼大睡。

%5
这一睡,也不知道睡了多长时间,这才悠悠醒转。

%6
太累了,就算是现在,方源也只想躺着,不想起身。

%7
“我的身体已经充斥死气,彻底转变成了僵尸,躯体不知道疲惫。但我的魂魄却是原来的魂魄,也有极限,也会损伤。”

%8
脑海中一个念头闪过,方源缓缓坐起。

%9
心灵安宁,有种舒爽的感觉。

%10
睡眠也有安魂养魂之效。

%11
方源吐出一口浊气,稍微舒展了一下肢体,只觉得浑身像是生锈了一般,带给他沉重且不灵活之感。

%12
他也不奇怪,心知自己这是使用杀招万我,造成的后遗症——魂魄受到了极大消耗。因此驾驭如此强健的僵尸身躯,就显得勉为其难。

%13
“地灵何在?”方源张口出声。他的声音变得十分沙哑难听,好像是沙硕和碎冰块一起剧烈摩擦的声音,带给人干涩又冰寒的感觉。

%14
“主人!”下一刻,方源的耳畔就响起地灵小狐仙的声音。

%15
声音仍旧娇脆动听,但此刻却又夹杂着担忧和难过。

%16
方源回过头。便看见小狐仙通红着双眼,出现在他的左手边。

%17
“我睡了多久?”方源点点头,问道。

%18
“主人你已经睡了两天三夜。”小狐仙便答。

%19
方源恍惚了一下。两天三夜这是狐仙福地的时间。狐仙福地五天,外界五域才过得一天。也就说从方源脱离大同风幕,回到现在,北原的时间也只是一天都不到。

%20
方源心中安定下来。

%21
他没有睡过,之前和黑楼兰商定过,确定了暂时的合作,不久后他还要再返回北原。

%22
小狐仙见方源沉吟不语。还以为他正在难过,便张开小口,安慰道:“主人。主人,你不要难过了。你变成这个样子,虽然好难看,但一定会有办法变回来的。人家相信。总有一天。主人你还会变成原来的样子。主人你要振作呀!”

%23
方源哑然失笑,伸出一只手臂,摸摸小狐仙的脑袋瓜儿。

%24
小狐仙躲闪了一下,终究任由方源恐怖的僵尸大手,落在自己的脑袋上。

%25
她垂下头,闷声不吭。

%26
方源轻轻揉了揉,小狐仙终于绷不住,哇的一声哭起来:“主人。你的手掌虽然冰冷冷的,但人家还是很喜欢的!”

%27
说完。一把抱住方源的大腿,嘤嘤地哭着。

%28
小狐仙还是这么可爱,如五六岁的女童,粉嫩纯真,娇小玲珑。她穿着一身彩裙,身后雪白的狐尾此刻耸搭在地上,显示出心境的失落。

%29
地灵是执念结合天地伟力形成,不同于意志,是不会欺骗人的。

%30
小狐仙说的话都是坦率的真话。

%31
方源一脸平静,没有说话,只是嘴角龇出的獠牙,微微收拢了一些,抚摸小狐仙脑袋的动作,也变得越加温缓。

%32
他分出心神,内视。

%33
第一空窍,死寂一片。里面无一丝真元,只有方源的第一本命蛊。

%34
原本通透莹润的晶紫窍壁,变为灰石质地。石壁上布满了微小的裂痕,造成这个伤痕的罪魁祸,便是空窍中央的春秋蝉。

%35
春秋蝉身交替闪烁着青、黄二色,六转仙蛊的澎湃气息充斥整个空窍。

%36
若是原本的五转巅峰空窍,恐怕已经被撑破了。但此刻空窍已死,反而更能承担仙蛊压力。虽然也有极限,但此刻却是距离较远。

%37
同时,方源还注意到,空窍中干涸见底,再不能自产真元。

%38
“这就是转为僵尸的代价,空窍已死,不能再产真元了。”方源心中思量。

%39
他不由地想到古月一代,古月一代转为血鬼尸,空窍也无法产真元,因此对天元宝莲这类的蛊虫,具有极强的需求欲望。

%40
“我此刻情形,却比古月一代要好多了。我之前已经升仙,有了自己的青提仙元十九颗,真元无限!”

%41
想到这里,方源便将心神从第一空窍,挪移到第二仙窍中来。

%42
刚刚新生不久的力道仙窍,此刻也彻底灰败,福地沦为死地。天空灰蒙蒙的,一座座白石山峰皆已倒塌,大地乌黑腐烂,散阵阵臭气。地面裂痕满布,宛若刚刚经历过大地震一般。这是强行装载了智慧蛊造成的。

%43
在这个仙窍里,装有方源几乎全部的蛊虫。

%44
最吸引眼球的,当然是仙蛊。

%45
在这里面的仙蛊数量,已经多达七只!

%46
这个数目要说出去,绝对要惊掉世人的下巴。仙蛊唯一,极其难得,普通蛊仙手中往往都没有一只仙蛊。

%47
若再算上借给黑楼兰的定仙游蛊,及春秋蝉,方源手中的仙蛊已经达到九只。

%48
这个数目,足以让大多数的七转蛊仙都汗颜。

%49
有关近水楼台的仙蛊有三只,分别是:浪迹天涯蛊、乐山乐水蛊、招灾仙蛊。另外四只仙蛊,则是方源击破无相拳,辛苦抢夺得的。

%50
除去仙蛊,还有凡蛊。

%51
其中价值最高的凡蛊,同样是方源击破无相拳抢来的。这些蛊虫,源自真传秘境中的各道真传,每一个都是精品,都有区别大众蛊的玄妙之处。

%52
但论重要性,还是第二本命蛊——四转的全力以赴蛊。以及苦力蛊、借力蛊、自力更生蛊、炼精化神蛊、地力蛊、水力蛊、风力蛊、电力蛊、火力蛊、潜魂兽衣蛊、敛息蛊等。

%53
正是这些蛊虫。构成了杀招万我的基石,得以让方源在大同风幕中奠定最终胜局。

%54
除去这些,就是其他的杂蛊。比如方源在北原之行中。动用最多损耗也最多的鹰扬蛊。能迅转向的风花蛊,缴获自单刀将潘平的单刀蛊,值得一提的战骨车轮蛊、星门蛊,蛊仙基本配备的洞地蛊、通天蛊、神念蛊等等。

%55
其中数量最多的,当属乞丐蛾。有近千只,有存储真元的效用。在北原之行的最后关头,帮助方源良多。

%56
视察了一遍空窍、仙窍。方源又将目光投向自己的身躯。

%57
不用照镜子,他都知道,自己已经成了彻彻底底的怪物。身高两丈。青面獠牙,血红双眼,浑身肌肉贲,如块块硬石。最为引人瞩目的。是他共有八臂。除去原有的两只人臂。其余六只,各有狰狞形态,分别源自五转飞僵蛊——修罗尸蛊、天魔尸蛊、血鬼尸蛊、梦魇尸蛊、病瘟尸蛊、地魁尸蛊。

%58
形态外貌是丑是美,方源根本就不在乎。地灵小狐仙在意,是因为她是女蛊仙白狐仙子的执念所化。

%59
方源在乎的是,僵尸之躯修为停滞,无法晋升微微一步。这对他的魔道追求,对他的梦想是最严重的阻碍。

%60
他野心勃勃。一心想永生。永生代表最强,千灾不损。万劫不磨,无人可以伤及性命,寿命无穷无尽。

%61
僵尸之躯,虽然无寿命,但魂魄还会受损,肉体遇到更强大的攻击也会陨灭,有许多的弊端和弱点。很多蛊师、蛊仙,在逼不得已的情况下,常常会转成僵尸。这个方法,只能说是无奈下的苟延残喘,是在地狱中仰望天堂。

%62
“如何才能解除僵尸之体,从死复生,再度拥有鲜活肉身呢?”方源在心中自问,显然,这是一个必然要去解决的难题。

%63
他虽然没有答案,却有腹稿。

%64
拷问仙窍中的墨瑶意志,是一个方法。借助智慧蛊,是第二个方法。

%65
小狐仙哭声渐歇,方源收回手臂,思绪平定下来,问道:“我昏睡的这些天,可有什么事情生?”

%66
小狐仙听到主人问询,立即站直身躯,抹抹眼泪,懂事地答道:“回禀主人,主要有两件事情。第一件事情是您的师兄太白云生大人早已苏醒,这两天一直想来看望您,有时候急得还哭鼻子哦。但主人之前有过命令,因此都被我劝阻了。第二件事情是仙鹤门,他们终止了和咱们的石人贸易。”

%67
方源闻言,微微皱起眉头。

%68
仙鹤门忽然终止石人贸易的原因,方源心知肚明。狐仙福地中有荡魂山,这是中洲十大派众所周知的事情。仙鹤门图谋荡魂山,又不好直接强攻狐仙福地,因此一年多前,派遣方正谈判,后来达成了石人贸易的协定。但其主要目的,还是荡魂山、胆识蛊。

%69
而后,仙鹤门多次催促和强调,要求方源开放胆识蛊的贸易。但都遭到了方源的否决。

%70
这一年多来,狐仙福地方面死不松口,仙鹤门的耐心终于被耗尽,因此忽然终止了石人贸易。至于他们是想表达愤怒,逼迫方源再谈判,还是别有他图,这却要再看事态展。

%71
“狐仙福地并非世外桃源,地处天梯山,外部情形一直堪忧。此次仙鹤门难,必有后文,我还得小心应付。”

%72
方源若是有仙窍福地,对狐仙福地的依赖性将大为减少。甚至可以直接吞并了狐仙福地,金蝉脱壳,扬长而去。可惜他现在成了僵尸,仙窍成了死窍,一切休提。

%73
方源想了想,暂时也琢磨不出些什么来。

%74
“可恶,变成僵尸之后,思维大大僵化,思考起来度太慢,效率太低。”方源眉头越皱越深,开始感受到僵尸之躯的弊端。

\end{this_body}

