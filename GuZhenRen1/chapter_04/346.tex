\newsection{八转慧剑蛊}    %第三百四十七节:八转慧剑蛊

\begin{this_body}

%1
这只仙蛊只有成人食指大小,通体银白作色,闪烁着耀眼的光泽。它形如蜈蚣,但蜈蚣的百足处,却是一根根的柔软须毛。

%2
“这个好像是……剑眉仙蛊?”

%3
方源想了想,脑海中闪烁一抹灵光。

%4
这是他从凤金煌给与的资料中,得知的情报之一。

%5
剑仙薄青当年有一只七转仙蛊剑眉,并非攻伐之蛊,而是用于剑道修行。

%6
蛊仙催动它,便能令蛊仙本身的一对眉毛改造。每一根细细的眉毛中,都刻印一条剑道道痕。

%7
道痕对仙蛊的威能,有着巨大的增幅作用。

%8
人的眉毛不管粗细长短,真要细数下来,眉毛的根数绝对不少。

%9
“这只仙蛊的价值很高,但对我而言,却是聊胜于无。毕竟我不走剑道这个流派。”方源同样地,将这只仙蛊也用墨瑶气息包裹严实,小心收好。

%10
片刻之后,第三只仙蛊也主动出现在方源的面前。

%11
这只仙蛊就比较特别。

%12
它形态不定,仿佛是一条蛇,又形似剑。它的全身,仿佛是一股浅蓝色,半透明的水液。

%13
被吸引出来后,不断地在方源周围环绕飞舞。

%14
水液形成一截剑刃的形状,栩栩如生。但由于不断游动着,剑刃仿佛丝绸般柔滑,却又给方源极端锋利之感。

%15
方源的仙僵之躯,本身的防御并不坏。但是这只仙蛊环绕周身的时候。每到之处,方源就感到最靠近它的肌肤就传来一阵割裂之感。

%16
结合情报,方源揣测,这只仙蛊应当是七转浪剑蛊。

%17
根据灵缘斋中的历史记录,当初剑仙薄青曾着手。准备将这只浪剑蛊提升八转级数。但是失败了。

%18
炼蛊失败,让薄青损失惨重,七转的浪剑蛊彻底毁灭。

%19
但后来,薄青又花费了大量的时间和精力,重新炼制出六转浪剑蛊。之后,又不辞辛劳地将六转浪剑蛊,再次提升到七转级数。

%20
第四只仙蛊。是飞剑蛊。外形看去。是一副银翅蜻蜓模样。

%21
这只七转仙蛊,是剑仙薄青作战时,最常用的仙蛊之一。

%22
他以此蛊为核心,辅助其他仙蛊,和大量凡蛊,组成了不少剑道杀招,在历史上几乎都留下一笔深刻的印记。

%23
比如其中的无形飞剑杀招、云霄飞剑杀招、穷追飞剑杀招、万里飞剑杀招……

%24
第五只仙蛊。是专门用来移动的仙蛊,名为剑遁蛊。

%25
它形如金针蜜蜂,一经催动,身形化作一柄利剑,刺破空间,向前穿遁。

%26
速度之快,几乎可以和同级的气遁仙蛊媲美。薄青一生当中,无数显赫的战绩之中,就有一例。

%27
他在七转时候,和一位同级蛊仙放对。三个回合之后。蛊仙不敌,催动七转气遁仙蛊飞逃。

%28
薄青驾驭剑遁仙蛊追赶。

%29
虽然没有追杀,但距离也没有被拉开。

%30
随后薄青又催动仙道杀招一剑纵横,此招的核心仙蛊之一,就是这只剑遁仙蛊。

%31
一招使出,敌人首级飞离,生死立判。

%32
收了五只仙蛊之后。方源还不想就此结束,继续对沉眠的薄青仙僵施为。

%33
墨瑶假意有限,此刻已经消耗了六成五。

%34
“接下来,不知又会是什么仙蛊?”方源心中充满了期待和紧张。

%35
他就像是在刀尖上跳舞。

%36
若是仙僵薄青醒来,对他将极为不利。

%37
不过方源已经知道,这里占据尸躯的是墨瑶残魂,却并非毫无应付手段。

%38
凭他三寸不烂之舌,口灿莲花的功夫,忽悠仙僵和残魂,并不是没有机会。

%39
怕就怕仙僵薄青一醒来,就惊怒交加,什么时间都不留给方源,直接将他斩了。

%40
方源当然不会主动唤醒薄青仙僵,他不会把自己的生命,寄托在他人的反应之上。

%41
这一次,勾动仙蛊,却始终不见动静。

%42
有史以来,最长的时间。

%43
一个时辰都过去了大半,仍旧没有任何端倪。

%44
“难道薄青仙僵身上,已经没有了仙蛊?这应当不太可能吧?若只有这么些仙蛊,怎么能催发出覆盖整个中洲,威力又无以伦比的剑光呢?”

%45
整整一个时辰过去了,方源手中墨瑶假意,已经所剩不多。

%46
他忽然心中一动。

%47
他在落天河河岸附近,布置了不少侦查用的凡蛊和蛊阵。忽然间,大片的凡蛊,还有不少蛊阵,被连续拔出。

%48
“有其他的蛊仙来了!”方源脸色微沉下来。

%49
之前,方源在河面上捡取仙材,耽搁了一长段时间。算一算,此刻有蛊仙到来并不奇怪。

%50
但若这位蛊仙,发现了河底的异状,接近方源,那方源就麻烦了。

%51
方源担心的不是这位蛊仙,而是仙僵薄青。毕竟外人可没有墨瑶假意,也不清楚状况,万一惊醒了仙僵薄青,让后者看到方源正在盗取他的仙蛊的话?

%52
呵呵。

%53
结果不言而喻。

%54
很可能,方源连说话的时间都没有,就给仙僵薄青斩了。

%55
“墨瑶假意已经差不多消耗光了,必须及时离开。毕竟离开的途中,我还要靠着气息继续掩饰。所以不能在薄青仙僵的面前,就将墨瑶假意彻底用光。”

%56
方源距离薄青太近了。

%57
直接催动定仙游,仙蛊气息洋溢,必然会惊醒薄青中的墨瑶残魂,简直是自寻死路。

%58
不过,就在方源想要抽身离开的时候。

%59
第六只仙蛊,终于姗姗来迟,出现在他的眼中。

%60
这只仙蛊外形很是简单。就像是一个气泡。飘飘荡荡,其貌不扬。

%61
但它逸散而出的气息,却叫方源不由地瞪大了双眼。

%62
这竟然是一只八转剑道仙蛊!

%63
难怪把它勾出来,耗费了这么长的时间。

%64
“但这是什么仙蛊?”一时间,方源思考不出这只八转仙蛊的跟脚。

%65
他也来不及想了。

%66
立即采用之前的方法。将其收入仙窍。

%67
走人!

%68
赶紧走!

%69
这一次来,收获已经极大,远远超出方源之前的意料。

%70
大赚特赚,瓢盆满钵,已经不足以形容这一次的收获了。

%71
相比较而言,地沟传承中的收获,都大大不及这次!

%72
接下来。来落天河探索的蛊仙。会越来越多。万一惊醒了仙僵薄青,嘿嘿……

%73
所以此地不可久留!

%74
离开落天河的过程,很是顺利。

%75
那位前来的蛊仙,有七转修为,已经消化了心中的震惊。此刻,他的注意力都被河水中惊天动地的猛兽激战所吸引。

%76
不时的,就有上佳的仙材。浮出水面。随后又被汹涌的波涛吞没。

%77
这位蛊仙想要抢夺这些血肉残肢,但神情犹豫不决,又有些不敢动手。

%78
毕竟这个时候,已经过了最佳时机。想要虎口夺食,冒的风险会很大。

%79
方源远远地离开落天河,心头微微放松。

%80
就在他催动定仙游的时候,他忽然全身一震。

%81
“我想起来了!这只八转仙蛊,恐怕就是传闻中的那只慧剑蛊!”方源的眼中闪烁着狂喜之色。

%82
“这只仙蛊,还是薄青独创。昔年,就算他战力出色。也吃了不少智道蛊仙的苦头,为了专门对付智道蛊仙,他就创造出了慧剑仙蛊。这只仙蛊不单单是剑道仙蛊,更兼具智道之妙啊。”

%83
想到这里,方源立即决定,将来他收服这些仙蛊时,必定首选此蛊!

%84
智道宗师的境界。让方源隐隐有了一股明悟。只要掌握了慧剑仙蛊,为其所用,那么方源的收获,将远远不止一只八转仙蛊这样简单。他的智道手段,必定突飞猛进,有巨大的质变!

%85
安全地回归狐仙福地之后,方源又意识到另一要点:“东方长凡的智道传承,智道宗师境界,六转星念蛊,八转慧剑蛊,还有九转智慧蛊……不知不觉间,我的智道前途似乎更为光明啊。就这现状,力道、宙道对我更有帮助。但论前景的话,智道的前景远比力道、宙道广阔辽远。”

%86
得到的这些仙蛊,都被他一一妥善安置。

%87
大量的墨瑶假意,被方源制造出来,用做消耗品,暂时封印住这些仙蛊。

%88
意志只要不沉眠,在思考,本身就会在大量消耗。当初,在王庭福地时,墨瑶假意就靠着乐意仙蛊,自己壮大自身,消耗多少,就弥补更多。

%89
方源如今虽然没有了乐意仙蛊,但自有其他手段,更是智道宗师,制造墨瑶假意这种事情,轻而易举。

%90
只是此中,需要凡蛊无数,分门别类,方源花了一些功夫才准备好。

%91
而不久前在落天河底的时候,事发突然,方源时间紧迫,来不及制造墨瑶意志。

%92
回想一下,这一次的收获,让方源都觉得有些匪夷所思!

%93
盗取他人仙蛊的良机,极其稀少。

%94
因为正常情况下,仙蛊也会因为一个念头,就自己毁灭。

%95
方源是捡了仙僵薄青沉眠的大便宜。而在蛊仙清醒的时候,仍旧能抢夺盗取对方仙蛊的,历史上似乎也只有盗天魔尊一人能够做到了。

%96
“星宿仙尊留下的预言梦境,好生厉害!接下来的几句诗词,又有什么真意蕴藏着?”

%97
方源想到这一次巨大收获的起因,既疑惑又期待。

%98
歌声寥落,英雄落魄,难挡命途多舛。

%99
折剑沉沙,千古兴亡,不尽天河滚荡。

%100
忧愁……

%101
幽夜漫漫魂梦长,问何处安乡?

%102
物换心移几春秋,唯天意苍茫。

%103
前两句话,方源已知真意,但后面三句话,究竟预言着什么?

%104
这里面实在太模糊笼统了,方源一时间也猜测不出。

%105
ps:今天第二更,月票突破100的加更。还是老规矩,计算月票时间是中午12点的样子。感谢大家的支持,高潮……终于是来了。这一章,接近末尾了。欢迎大家猜猜剧情,我觉得这是《蛊真人》这本书,区别于其他书的优点之一吧。

\end{this_body}

