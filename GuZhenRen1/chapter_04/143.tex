\newsection{天地雷下险成就}    %第一百四十三节:天地雷下险成就

\begin{this_body}

对于方源而言,智道传承的确珍稀,但即便获得了完整的传承,却暴露了真实身份,引得中洲、北原蛊仙追杀,那绝对是得不偿失的!

残阳老君的一把火,烧得群魔束手,皮水寒远逃。

方源意识到不妙,不禁心生退意,自然停下了手。

自在书生这位七转战力,也没有冒然动手。之前方源处理追命火的表现,让他更加高看。他心思自己是靠着千解才渡过劫难,更觉得方源深沉。

尤其是方源身边,还有黑楼兰、黎山仙子两位黑袍蛊仙,于是自在书生便按捺住动手的,只是目视方源这边。

意思十分明显,就是静待方源三人出手。

一时间,墟蝠尸山的上空陷入了沉寂,而废墟大殿里仍旧是咒骂声不绝于耳。

东方部族的八位蛊仙,身体动弹不得,却保留了说话的能力。

自从东方长凡夺舍开始,他们就痛声大骂,咆哮质问东方长凡。

东方长凡的星意对此是充耳不闻。

他见外界平静下来,群魔一时束手,淡笑出声:“终于等到了好时机,夺舍重生,便在此刻了。残阳老君,之前的几大步骤,我都详细叙述给你听了。关键时刻已经来到,还须劳烦你代为防守了。”

当断不断反受其乱,星意下定决心,便再无犹豫!

异变再起,撑起的广阔血幕,陡然在血幕表面形成一道巨大的漩涡。

漩涡中一股血能酝酿而出,凝聚成形,陡然爆发,化为一道血光巨柱直射而下,正中虚阵中心,笼罩在东方余亮的身上。

残阳老君双眼一瞪,大骂一声:“该死!”

见到这个变化,残阳老君这才明白:原来这血幕,并非单纯用于防守。而是一种纯化过程,抽调出来的八股仙窍本源,仿佛是八块金属熔化在一起。然后靠着上方魔道蛊仙们的打击淬炼,消除本源中八位蛊仙的意志及气息,最后再灌注在夺舍的目标身上。

东方长凡布局精妙深邃,把魔道蛊仙们都算计在内。就连方源也被蒙在鼓里,被东方长凡利用,成了打铁匠。

如今血能倒灌,充入东方余亮的体内,自然血幕就会薄弱下去。血幕削弱了,防御顿时薄弱。残阳老君门派重任在身,巨大利诱在前,他不站出来顶缸,能行吗?

果然见到血幕削弱,魔道蛊仙们俱都精神一振。

“看来之前的攻势,还是有效果的!”

“破开血幕有望,大家一齐动手啊。”

众仙纷纷动手,无数的攻击宛若暴雨倾盆而下。

血幕要一边向内输送血能,另一边要承受攻势,顿时显得吃力,摇摇欲坠。

残阳老君不得不出手,使出仙道杀招尽处余晖!

一道光罩,在血幕外围撑起,仿佛是落日的余晖凝聚形成,淡薄得近乎虚无。无数的攻击穿透这层光罩,落在血幕之上,威能却不足先前的一半。

这正是“尽处余晖”的作用,不是全部防御,而是将绝大多数的攻势削弱。

就连“流川”这等仙道杀招,都能削弱三成效果。如今大部分的魔道蛊仙出手,都是凡道杀招。被残阳老君这般削弱,便显得雷声大雨点小。

东方长凡的脸上,现出喜意。

如此一来,情势对他无比有利了。

虚化大阵不断地抽取出八股仙窍本源,交融一体,形成血幕。

血幕承受着攻击,里面蕴藏着的他人意志气息,都渐渐被消灭。随后再转化成纯净的血能,灌输到东方余亮的体内。

东方余亮本是五转蛊师,巅峰级的空窍。此刻他的魂魄已被东方长凡清除一空,东方长凡的魂魄掌握着年轻的躯壳,将全部的注意力集中在自己的空窍之中。

空窍在血能的灌输下,被轰然撑破。

东方余亮的气息,却不降反升。

一种惊人的转变正在发生超脱凡俗,冲刺蛊仙!

咔嚓!

外界的天空中,陡然闪过一道晴天霹雳。

旋即几个呼吸间,晴朗的高空乌云滚滚,覆盖千里周遭。

大地震动,轰鸣,像是千万牛群迁徙踩踏地面,无数的烟尘四起。

天地二气被勾动,天劫地灾形成!

“竟然有人在里面冲刺升仙?!”

“这到底是怎么回事?”

众蛊仙连忙停下攻势,莫名其妙。

咔咔咔!

一道道闪电,从高空劈下。闪电呈现青金之色,蕴藏的杀伤力,叫魔道蛊仙们纷纷变色,急忙后退。

“了不得!这是天罡雷劫啊。”

“底下这凡人到底有多倒霉,不受天地待见。刚刚升仙,居然遇到如此强大的天灾!”

“这又是什么地灾?”

众仙纷纷退到边缘,远远凝望。

只见墟蝠尸山周围,地表上现出无数的圆形坑洞。坑洞深不可测,一片漆黑。

通通通!

一颗颗漆黑雷球,从坑洞中出来,砸在墟蝠尸山上,轰然爆炸。

电光激闪,冰山只支撑了几个呼吸,就被无数雷霆炸碎,化为块块碎冰,冰屑四下飞溅。

这下就连自在书生都动容起来,暗自心惊:“天劫是天罡雷,霸道犀利。地灾是地鬼雷,灵诡轰动。这两大雷劫,号称天地双雷,一旦相互结合,更是威能暴涨。就算是如今的我碰到,也要头大!”

天地双雷相互结合,上方坠落,下方轰炸,电光刺目耀眼,形成雷霆炼狱。

无数道的电芒,密密麻麻,数目之多,竟然结成一大片闪电森林!

碎裂的冰块,彻底消弭,就连寒气也没有一丝残余下来,都被雷霆剿灭。

末日般的场景,让观看的魔道蛊仙们都一一屏住呼吸,心头压抑。

墟蝠尸山内,残阳老君满脸肃穆,全力支撑着仙道杀招尽处余晖。

这道奇妙的防御杀招,兢兢业业地削弱着天地双雷的威力。

但残阳老君竭尽全力,也只削弱了一成半。

他体内的红枣仙元消耗迅速至极,很快,他的额头上就隐隐见汗。

“怎么会有这般强大的雷劫!”东方长凡的星意喃喃自语,一脸担忧。这个情形几乎是他预计的,最为糟糕的一幕了。

不想自己这一次夺舍,就碰到了。

运气真是差到了极点。

血幕在电劈雷轰下,剧烈损耗,损失之惨重叫东方长凡星意心疼无比。

这些损失,将大大影响东方余亮的仙窍底蕴。

一刻钟,两刻钟,半个时辰,一个时辰……随着时间的流逝,天地雷劫不仅没有减弱的趋势,甚至还不断增强。

太古荒兽级的墟蝠尸体,被摧毁得面目全非。

周围的宇道道痕,原本密布万千,如今被雷道击毁,几乎全数清空。

残阳老君已现不支情形,仙元的惨重损失,让他心情沉重。

“是了!天地灾劫威能离谱,完全是因为你东方长凡的缘由。”残阳老君忽然道,“你东方长凡已经是死人一个,按照宿命的安排,就该魂归生死门中,怎么可以仍旧在这里活蹦乱跳?你现在不仅想重生,更想夺舍成仙,这是货真价实的逆天!怪不得天地都会震怒!!”

东方长凡的星意原本表情凝重,听到这番话后,反而双眉扬起,哈哈大笑起来:“自从红莲仙尊破坏宿命,这天底下万物的命运,都掌握在我们自己的手里。就算逆天又如何?我命由我不由天,我筹谋日久,做了多少的准备和努力,此行必能功成!”

他话刚说完,就听咔嚓一声爆响。

原本电光如雨,电闪雷鸣,就已经绵绵不绝,充斥耳膜。

此刻的这声雷响,却是力压群雷,响彻云霄,震得蛊仙双耳都嗡鸣起来。

东方长凡的星意,被一下子震散!

一片耀眼的白光,充天彻地!让方源、自在书生等一干魔道蛊仙惊惶急退,让残阳老君也不得不紧闭双眼。

白光当中有一道神雷,猛然落下。

这神雷刚猛无俦,所向披靡,顷刻间轰烂尽处余晖、炸破血幕,直朝东方余亮杀去。

神雷速度极快,几乎眨眼间,就从乌云盖顶的高空,摧枯拉朽,杀到了东方余亮的面前。

东方余亮不过凡躯,真要被这雷炸着,必定粉身碎骨,东方长凡的夺舍重生大计就此戛然而止,彻底失败!

生死存亡之际,一直皱眉,双目紧闭的“东方余亮”强行睁开双眼!

刺眼的白光中,他勉强捕捉到那一抹雷电之光。

一股强烈的惊怒在他的心中升腾起来:“摇光!堕星雷!劫中藏劫,贼老天你这么想我死?!那我偏偏要不死!”

东方长凡在心中咆哮。

下一刻,虚阵大放光辉。

轰!!!

惊天的爆炸中,无数道白炽光柱从墟蝠尸山中,不断四射。

光柱扫到一些倒霉的魔道蛊仙,立即心震神摇,魂魄晃动,有几位甚至直接从空中栽倒下去。

恐怖的爆炸,掀起滔天的气浪,将方圆数十里的植被,都掀起来。

天空都仿佛坍塌下来,地动山摇。

魔道蛊仙们慌忙撤退。

这一撤,连退三十里,白色光柱渐渐消散,众仙这才稍稍安定下来。

原本巨大如山,岿然不动的墟蝠尸山,已经荡然无存。

空地上方,出现一道年轻的身影,昂然漂浮着,凝视着方源等人。

东方余亮!

\end{this_body}

