\newsection{炼蛊大会的由来}    %第一百八十九节:炼蛊大会的由来

\begin{this_body}

%1
------------

%2
在众人的议论声中,方源漫步走出。

%3
他所到之处,拥挤的人群都不自觉地让开一条路。

%4
尽管方源是魔道蛊修,让正道厌恶反感,但方源表现出来的炼道的造诣叫人不得不敬佩。尤其是在这个中洲炼蛊大会这样的特殊环境,特殊时期,就连魔修都能堂而皇之地出入正道门派驻地,蛊师们的心中比平常时候,多了一种只以炼蛊论英雄的心理。

%5
的山门处,几位蛊师来自各个势力,大声吆喝着,贩卖着情报,生意很是火爆。

%6
“号外,号外,最新出炉的情报,本届中洲炼蛊大会最全成绩表,连第七场的成绩都有哦。”

%7
“你想知道本届最有希望夺冠的百大炼蛊强人吗?那就快来看看吧!一份影像蛊,最低价格只要五十元石。价格从优,售完为止!”

%8
“知名智道蛊师对本届中洲门派小比排位赛的推测,还有十大古派不得不说的恩怨情仇!一份情报,只要十块元石!物廉价美!”

%9
“没想到这么快,就有第七场的成绩了。”方源闻言,心中一动。

%10
不过等到他看见这位贩卖情报的蛊师着装,也就是释然。

%11
这位贩卖第七场成绩情报的蛊师,穿着“电语宗”的门派服饰。

%12
电语宗是蛊仙创造的超级势力,当然比之中洲十大古派完全是比不上的。电语宗的创派祖师糜蓝光,原先是风云府的雷道蛊仙。因为违背了门规,被废去修为。逐出门派。

%13
他为了维持生计,被迫转修信道。结合雷道修行的经验,创造出许多别具一格的信道蛊虫。结果反而比之前。发展的更红火。

%14
他成立的电语宗,经过一百多年的发展,已经跃升为中洲势力中,擅长信道的一流门派。甚至就连十大古派,有时候也向糜蓝光购买情报。

%15
方源走到这位蛊师的面前,丢给他十块元石:“给我来一份。”

%16
电语宗的蛊师手忙脚乱地接过元石,定睛一看,认出了方源:“啊,阁下。阁下莫非便是方源大人!?大人能向小人问询,是小人三生有幸,怎敢收大人的钱财?请大人收回,这份情报将免费送您。”

%17
说完,将十块元石又向方源双手奉上。

%18
方源淡淡一笑:“就当赏你的。”

%19
那蛊师谄媚地笑道:“那小的谢大人赏赐!大人这是您要情报,小的代表电语宗感谢您购买本宗的情报。如果您能在百忙当中,挤出一点时间,接受我宗的访谈,小人隶属的电语宗还将有仙材奉上!”

%20
方源摆摆手。接过蛊师递来的信道蛊虫,径直往前走。

%21
“大人请慢走。”蛊师见方源拒绝,不敢再劝,毕竟方源是魔道五转蛊修。站在原地,目送方源。直至方源消失在人群中,他才转身。继续大声哟呵,贩卖情报。

%22
方源一边走。一便分出心神,探入刚刚的信道蛊虫。

%23
这种信道蛊。只有一转,炼制成本极为低廉,自然不值一提。

%24
关键是里面的情报,才是价值十块元石的东西。

%25
方源查看一番后,发现中洲炼蛊大会中,连胜七场的蛊师,就有八千多人。

%26
这些人能闯过七场比试,基本上都有准大师级,当然这其中,不排除有一些运气特别好的,实力较弱的。

%27
不过这里的实力较弱,只是相对而言,绝非是洪易之流。类似洪易这种,早在前三场比试中就被淘汰干净了。

%28
起先报名参加大比的,有数十万人。这些人中绝大多数,都是中洲蛊师,也有来自南疆等地的其他四域中人。七场之后,只剩下八千多人,连一万都不到,可见淘汰率是多么的惊人。

%29
而在这其中,连续七场都在前三名之列的,有四百多人。

%30
这些人至少是炼蛊大师级的优秀人才。

%31
别看四百这个数目很多,放到中洲这样的范围内,就可发现这些人真的只是一小撮,站在炼道成就金字塔的顶端。

%32
任何流派,大师级的人物都是稀少的。而炼道相对其他流派,更加难以精深,大师的数量更少。

%33
在这四百人中,连续七场比试都获得第一名的蛊师,仍有八十多人。

%34
毫无疑问,这八十几人是焦点中的焦点,风云中心的人物。

%35
方源的名字便赫然在列。

%36
按照专家的评估,他的名字只排在中上的位置。

%37
这还是方源主动高调的结果,可见中洲炼蛊大会竞争何等激烈,各种人物层出不穷,要想在这些炼道精英当中脱颖而出,十分不易。

%38
当然,这个排名更多的是猜测,并不代表炼蛊真实的水准。

%39
前七场的比试内容,并不足以测量出方源的极限,同时方源来历神秘,旁人就算想要猜测,也无从猜起。这些原因,也导致方源的位置,在三十多名。

%40
方源仔细浏览这个排名,很快就发现许多熟悉的人名。

%41
火工龙头、方槐、天泯灭、凤金煌、余木蠢、吕品田、都平滩、曲文……这些人的名字,都在方源的前世记忆中有着高上的地位。

%42
毫无疑问,都是人中豪杰,排列在一起群星璀璨,让方源都有一种耀眼之感。

%43
“这些人中,不乏五域乱战中的明星。有的现在虽然排列在后并不起眼,但在将来却晋升成仙。有的在有生之年,更达到炼道宗师的境界。”

%44
方源心中生出一股压力。

%45
方源现在是炼道准宗师,离宗师境界只有一步之遥。但别看这一步,其实非常遥远,非得有质的突破方可。

%46
其实,方源要说炼道天赋,那是稀松平常的。

%47
天赋不行,就用刻苦努力来弥补。

%48
方源通过前世今生的积累,硬生生地将炼道境界一步步提升,超越大师级,达到准宗师的地步。

%49
五百多年的修行生涯,才修炼到准宗师,是不是太废柴了?

%50
看看郑山川,年纪轻轻,却已然是炼道大师!

%51
这当中,天赋的重要性绝不可否认。

%52
但还有一点。

%53
郑山川是主修炼道的蛊师,主要的精力都投放其中。而方源不管前世今生,都是兼修炼道,前世蹉跎岁月三百多年,后两百年主修血道,今生主修力道。

%54
一个主修,一个兼修,投入的精力就天差地别了。

%55
方源若不主修血道、力道,恐怕都活不到现在。他可没有郑山川这样的好运气,可以有一位师傅一路保驾护航。

%56
而且流派境界难以提升,越往上难度越大。在梦境改变五域大局之前,想要提升境界,只有靠个人的努力积累和天赋才情。

%57
可以说,方源能有炼道准宗师的境界,已经相当不易!

%58
“我这个实力,要挤进本届炼蛊大会的前十,都有些困难。而要从不败传承中获益,则至少得是决赛第六!”

%59
不败传承,是炼道传承。中洲炼蛊大会也是由小壮大的,起初它只是中洲十大古派为了争夺不败传承,而举办的瓜分利益的形式通过炼道的对决,判断输赢,从而为自己,为门派争取最大的利益。

%60
但到了后来,因为不败传承牵扯到的利益太大,对宿命蛊的修复都帮助,天庭主动接手过去。

%61
严格来讲,中洲十大古派皆是天庭内部组织派系的地上代理人。天庭就是十大古派的幕后老大,由老大接手,十大古派也只有乖乖地接受,并且还要积极配合。

%62
有了天庭在幕后推手,中洲炼蛊大会这才真正开始扬名,并且最终将影响力推广到全天下。使得如今,每一次炼蛊大会都会吸引其他四域的炼蛊人才主动参加。

%63
“按照往届的成绩推论,能够获得决赛前三的,都是炼道宗师级人物,甚至偶尔有准大宗师级!”

%64
每一届的中洲炼蛊大会素质不一,但前十中,至少大半都是炼道宗师,只有第七位上,挂着一些炼道准宗师的蛊师。

%65
有时候涌现的人才不多,前十中炼道宗师只有三四位,那么炼道准宗师就能挤进前六。但有时候涌现的人才过多,前十中的炼道宗师能有七八位,那么能荣幸地登入前十宝座的准宗师就少了。

%66
方源这一届大会,素质不高不低,炼道宗师大约是五人。

%67
“我最厉害的,还是血道炼蛊手段。但此时手中血道蛊虫不多,要公然动用那些血道炼法,就是给仙鹤门再次攻打狐仙福地的借口。要想获得前六,必须得动用手段了!”方源沉思着,内心下了决断。

%68
十六天后,中洲。

%69
安祖地沟,驱邪派。

%70
驱邪派乃是超级势力,高层拥有三位蛊仙坐镇。它的势力范围绵延数百万里,影响笼罩数十个大型势力,数百个中型势力,成千上万的小型势力,以及宛若蚂蚁般渺小的无数蛊师家庭。

%71
作为第八场的比试地点之一,驱邪派的门派总驻地绝对拥有这个资格。

%72
成千上万的蛊师,在这一天蜂拥而入,都是来观赏精彩的炼蛊对决。当然真正的比试场地,空间有限,只能容纳一千多人。这一千多人,自然都不是普通蛊师了。

%73
不过在容纳普通观众的几处巨大的广场上,驱邪派已经做了妥善布置,一旦第八场比试开始,就会有影像出现在半空中,同步显示。

%74
ps:多谢打赏,多谢推荐票,多谢月票,这段时间多谢大家的支持了!接下来无疑会更加精彩。(,!

%75
@!好消息,本站书友群开通了,群号如下2969158,验证请发用户名

\end{this_body}

