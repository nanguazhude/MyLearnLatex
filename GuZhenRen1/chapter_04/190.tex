\newsection{方正心底的呼唤}    %第一百九十节:方正心底的呼唤

\begin{this_body}

%1
方正和另外一名仙鹤门的长老,带领着十多位弟子,进入比试会场。

%2
场外,是嗡嗡嗡的声音。一千多位蛊师坐在四周的高台上,一边等待着第八场比试的开始,一边进行着低声的交流。

%3
但因人数太多,即便是最微小的交谈声,汇集在一起,也变得十分嘈杂。

%4
“是仙鹤门的长老和弟子。”一看到方正这群人,周围的蛊师们不由地目光停顿在他们的身上,议论的声音也不自禁地压低下来。

%5
相对于大小势力,中洲十大古派中人,便宛若皇子行于平民当中的感觉。

%6
“那个年轻人竟然穿着仙鹤门长老的衣服,我没看错吧?”

%7
“小声点,他可是五转蛊师,气息货真价实!”

%8
“大惊小怪什么,维持青春容颜的蛊虫又不是没有……”

%9
“快看那,是万龙坞的弟子和长老。”

%10
下一刻,众人的目光又旋即被另外一群人吸引过去。

%11
这群蛊师的数量,比仙鹤门还要多出一倍。有四位长老率领,弟子中也颇多精英弟子。

%12
“万龙坞……”仙鹤门的长老看到,不禁目光微凝。

%13
十大古派中的万龙坞,在最近可谓出尽了风头。原因都出在凶雷恶人的身上。

%14
凶雷恶人,虽然只是六转蛊仙,但战力却极为出色。

%15
闭关两年多,参悟出仙道杀招雷神子。而后破关而出,行走中洲,一路挑战无数蛊仙,胜利居多。平手很少,失败鲜有。

%16
凶雷恶人越是胜利,势气就越盛,因为切磋赌斗的关系,也获得不少资粮。导致手中的雷神子数量,居然不减反增,如今已有了三头之多!

%17
在没有雷神子傍身之前,凶雷恶人便已经有了战平七转蛊仙的骄人战绩。

%18
现在有了三头雷神子,更是战力暴涨,甚至超越了一般的七转蛊仙。在他挑战的战绩当中。不少的七转蛊仙都败在他的手中。

%19
以六转修为,战胜七转蛊仙,这是相当难得的事情。尤其难能可贵的是,凶雷恶人一而再,再而三地战胜七转蛊仙。一时间风头无两,所属门派万龙坞也是名声大涨。

%20
乃至于,整个中洲蛊仙界都开始隐隐认可凶雷恶人为当代中洲六转战力第一人的位置。而万龙坞某些好战的长老更是振奋,直接喊出凶雷恶人就是下一个石磊,下一个凤九歌的口号出来。

%21
凶雷恶人听闻之后,立即回信呵斥:“石磊也就罢了,凤九歌大人却不是我能比得上的。以后这种话少说!”

%22
言下之意,石磊还是可以叫板的。

%23
仙猴王石磊是战仙宗的七转蛊仙。战力十分突出,但始终被凤九歌牢牢压在下方,不得翻身。

%24
凤九歌这个变态。数千年都难得一出,中洲十派已经被他打得不服不行,又敬又畏。

%25
一连串的挑战胜利,助长了凶雷恶人的气焰,开始公开叫板石磊。

%26
战仙宗中人自然极为不忿,但偏偏性情暴躁直接的石磊。却是古怪地保持了缄默。

%27
如此一来,反而更增加万龙坞的气势。觉得石磊面对凶雷恶人,已无应对的把握。当起了缩头乌龟。

%28
当然,只要稍微熟知各种内情的人,就知晓凶雷恶人的叫板,不过是万龙坞对战仙宗的试探。战仙宗发现了繁星洞天,又发现了其中蕴藏着的星宿仙尊的梦境,已经暗中攻略经营。因为要四下抽取蛊仙战力,这番大动作已经被其他九派隐隐察觉了。

%29
但这些背后的秘密,往往只有蛊仙一级才会知晓。

%30
所以万龙坞的这群长老弟子,见到方正一行人时,显得相当的趾高气扬。

%31
“哦,是仙鹤门的人呐。”

%32
“呵呵呵,你们这次来干什么?坐看你们的长老是如何失败的吗?”

%33
“没有用的,我们万龙坞的龙头大人,一定是最后的获胜者,这点毫无疑问!”

%34
万龙坞的蛊师们,一开口,就展开冷嘲热讽。

%35
中洲十大古派之间的竞争,尤为激烈,双方并不对付。

%36
尤其是仙鹤门已经孱弱很久,强势起来的万龙坞中人,更加轻视仙鹤门。

%37
仙鹤门的弟子们不由大为憋气,但却反驳不得,保持沉默和万龙坞一群人擦肩而过,最终坐在安排好的位置上。

%38
“嘁,一群孬种。”有人不屑。

%39
更有人嘲笑:“哈哈哈,果然不愧是仙鹤门的风格啊。”

%40
仙鹤门一行人,脸色更加难看几分。

%41
旁观者见此,生出许多疑惑,有人便问:“仙鹤门、万龙坞同为十大古派,为何万龙坞这么嚣张?”

%42
当即就有人回答:“十大古派中,也有强弱。仙鹤门弱于万龙坞,这基本上是公认的。除此之外,就是万龙坞在这一场的参赛蛊师,实在太强大了,就是那个号称火工龙头的炼道蛊师!”

%43
“什么?火工龙头竟然是万龙坞的蛊师,他不是散修吗?”

%44
“嘿嘿,你没看电语宗挖出来的情报吗?这位火工龙头,早年触犯了门规,结果被万龙坞扫地出门。这一次参加炼蛊大会,就是要夺得名次,回归万龙坞的!”

%45
“原来如此。据说这位火工龙头,已经有了炼道宗师的境界!这样的人物,居然也要眼巴巴地请求回归万龙坞,中洲十大古派的号召力真是巨大啊……”

%46
而在仙鹤门这边,方正和同行的长老也在悄悄议论。

%47
方正问道:“不知道这场比试,我派的炎堂长老获胜的可能有多少?”

%48
方正炼蛊实力不行,好不容易经过集训,险险地通过四道试题,成功报名。结果却在第一轮。就惨遭淘汰了。

%49
好在这样的大比试,是最为公开的比试,不禁任何外人观看。

%50
方正又是仙鹤门长老,身份可谓高贵。这次炎堂长老参赛,他吃了对方这么的酒菜。于公于私都要过来观看,为炎堂长老打气的。

%51
方正这么一问,坐在他身旁的仙鹤门长老眉头又深皱一分,唉声叹气地道:“方正长老,你也知道咱们的炎堂长老,主修的是炎道。炼道不过兼修而已。而对方火工龙头,却是主修炼道,兼修炎道。这个差别就大了。”

%52
方正仍旧不太明白,又问:“即便如此,依照炎堂长老的炼蛊造诣。取得前三名,应该也是可以的。”

%53
“唉!”仙鹤门长老摇头苦笑,“方正长老你不了解,炼蛊大会过了第七场,之后的赛制都是单人获胜的残酷淘汰制。你眼前的这个偌大的驱邪派比试场,只有一位蛊师能够胜出,晋级下一场。”

%54
“什么?竟然是这样!”方正震惊了一下,几个呼吸之后。他反应过来,到底是挫折经历多了,心思灵变起来。“既然对手这么强大,那么炎堂长老为什么不去另外的比试场地呢?这炼蛊大会可没有禁止参赛者乱窜啊。”

%55
“方正长老,你这么想也是对的。毕竟能进一步,做些撤退也是战术上的胜利。但可惜的是,其他的比试地点也有炼蛊强者占据啊。而且……将炎堂长老安排在这里,也是仙鹤门高层的决定。”说到最后。仙鹤门长老故意压低了声音,悄悄地说。

%56
方正悚然一惊。他此时若还不明白,就是傻子了!

%57
“原来仙鹤门高层已经将炎堂长老。作为弃子,目的就是试探火工龙头的能力吗?”

%58
恍惚间,他又想到师傅天鹤上人,曾经跟他分析的话炎堂长老在仙鹤门被孤立,所以才请你喝酒,想和你结成政治同盟啊。

%59
“正是因为被孤立,所以才被选为弃子的吗……炎堂长老心中恐怕充满了不甘吧。若是我哪天被当做弃子呢?”

%60
一念及此,方正的脑海中忽然又浮现出他身处血池,浑身长满鲜血藤蔓的惨痛画面。

%61
呼!

%62
噩梦般的记忆再次浮现,方正不禁浑身一震,倒吸一口冷气。

%63
“你怎么了?”身旁的那位长老,关切地问道。

%64
“没什么。”方正伸手抹了把额头,一时间内,手掌上竟沾上一层薄薄的冷汗。

%65
他不敢多想,心中慌乱又彷徨。

%66
而这时,同门长老的声音又在他的耳畔响起来:“参赛者都一一入场,这场比试要开始了。”

%67
方正连忙抬头望去,只见一百多位蛊师,缓缓踏入场中。

%68
方正很快发现了仙鹤门的炎堂长老,后者面无表情,但目光中却是流露出愤慨和无奈。

%69
而一身赤袍的火工龙头,却是一脸倨傲模样,四下扫视,毫不掩饰对周围竞争者的不屑和鄙夷。

%70
“一群弱鸡,这场我赢定了。快点开始罢。”火工龙头不以为意,哈哈大笑,公然催促驱邪派的主持长老。

%71
周围炼道蛊师,却不敢反斥,摄于火工龙头的强大实力。就连仙鹤门的炎堂长老都捏紧双拳,一脸怒意,竟也是默不作声。

%72
“万龙坞必胜!必胜!!”

%73
“火工龙头大人必胜!”

%74
万龙坞的一群人,像是打了鸡血,兴奋地高呼起来。态势嚣张,周围的观者却只能静坐,默默看着这一切的发生。

%75
“可恶,真是太可恶!”方正到底是少年心性,口中咬牙,心中郁愤,“可恨我实力不足。真希望这个时候,能有个人出现,将这个火工龙头击败下去,把万龙坞这群人的嚣张气焰打下去啊。”

%76
仿佛是应和他心底的希望,一个身影悄然出现。起先并不引人注意,但当他踏上场地时,无数道目光旋即投注过来。

%77
黑袍。

%78
面具。

%79
五转魔修……

%80
是方源!

%81
他怎么来了?

%82
全场一静,旋即大片的哗然。

\end{this_body}

