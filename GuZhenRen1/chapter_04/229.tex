\newsection{危在顷刻}    %第二百三十节:危在顷刻

\begin{this_body}

方源再定睛一看时,发现自己已经置身在另一座云阁之中了。qiushu.cc [天火大道小说]

方源吐出一口浊气,这是有琅琊地灵帮忙,否则自己脱身恐怕就有危险了。

毕竟对手当中,有一位回风子。

此人号称是当今北原,移速第一人。就算是八转蛊仙,都赶不上他。曾经有过在药皇手中逃脱的惊人战绩!

不过危险,并不代表不可能。

方源手中有定仙游,只要争取三息时间,就可逃出生天。就算回风子速度再快,也不济事。

“本来就只剩下五座云阁,这次又连失两座云阁,时间却没拖延多久。对方已经袭来,这次我们三个一起出手!”琅琊地灵神色严肃,叮嘱道。

琅琊地灵说话的时候,方源也发现了,除了自己和地灵之外,云阁中还有一位墨人蛊仙。

方源打量这位墨人蛊仙一番。

他具有明显的墨人特征黑皮肤白头发,满脸皱纹,看起来已经颇为老迈。

“墨人城中不是说,只有一位蛊仙,就是墨人城城主墨坦桑吗?看来这位墨人蛊仙,应该就是墨人城中的底蕴了。平时隐藏得挺深,我的情报来源于黎山仙子,都没有得到他的丝毫信息。到了琅琊福地危机时刻,这才出现。”

方源心中猜测。

“墨人是异人,异人种族向来蛊仙甚少。墨人城能隐藏一位蛊仙,恐怕已经是极限了。不过话又说回来,墨人城方面,为了琅琊福地的安危。尽出两位蛊仙,也算是挺拼的。这么说来,墨坦桑对琅琊地灵挺有信心。”

方源见微知著,仅仅只是看到这位墨人蛊仙,就联想到了很多东西。

这位墨人蛊仙对方源只是微微点头。没有多说话,似乎不想和方源有过多的交流。

这也难怪。

方源大变模样,但也是人族蛊仙。

当今五域,整个天下,异人都被排挤打压,或者被捕猎训豢养。驯养成奴隶公开贩卖。

异人几乎是苟延残喘。

北原的这座墨人城,已经算是极为难能可贵的异人群落了。

也是有赖于世世代代,墨人城的城主都是蛊仙的缘故。

不是蛊仙庇护,怎么可能有墨人城?早就被各方大小势力联合攻下,瓜分奴隶了。

可以说。没有蛊仙,异人种族的生活状态,都是以小型、微型部落为主,分散在世界各地的旮旯角落里。

不是异人们不想组成中型、大型部族,而是部落规模一大,往往就是灭顶之灾,引来无数的势力捕猎奴隶。

这位年老的墨人蛊仙,似乎对人族抱有强烈的戒备。[看本书最新章节请到棉花糖小说网www.mianhuatang.cc]看到方源望过来,只是略微向方源点头,仍旧停留在原地。没有过来交流的意思。

方源也点点头,回应一下他。

身旁的琅琊地灵,紧皱双眉,忧心忡忡地望着窗外,忽道:“来的竟这么快!直扑我这座云阁,情况不妙。看来他们掌握了某种侦察手段,针对我的十二波云迷澜阵。”

方源连忙投去目光。但只见茫茫白雾。

他心知自己侦察杀招不得力,琅琊地灵应是所言不虚。

果然。几个呼吸之后,狂风骤起,排开浓浓的白雾,吹出一片清晰的空间。

两位敌方蛊仙,傲立半空。

一位身着黑袍,面皮极为好看,但因为身怀隐伤,导致神色中透出一股衰败。让人感觉,此人虽然倜傥潇洒,却身虚体弱,仿佛要命不久矣。

正是黑城。

还有一位,身着武装,狼背蜂腰,双眼绽射厉芒,也是方源的熟人七转蛊仙秦百胜!

方源顿时将主要的注意力,都集中在秦百胜的身上。

在方源的认知中:这个秦百胜极为强大,是中洲石磊一层的人物。若正要火并,方源自认为不是他的对手。

人要有自知之明。

方源能活这么久,原因之一就是很有自知之明。

他虽然掌握了力道杀招万我,暂时不缺仙元石,还有一套星道仙级杀招傍身,弥补了他的诸多短板。但真正战斗起来,恐怕只能和残阳老君、贺狼子、回风子、自在书生这等,勉强打个平手。

他之前虽杀了雪松子,但那是运气不错,计策生效了。

之前用星道的战力,对付贺狼子,虽然场面僵持住,但贺狼子的第三个仙道变化,却始终没有逼出来。换句话说,贺狼子仍旧有实力保留。

显然,地灵也知道秦百胜的厉害,眉头越皱越紧,口中嘀咕着:“可恶,就差这么一段时间而已。只要能拖延过去,老虎都会成病猫!”

他似乎有什么厉害的手段,但需要时间去酝酿。

可惜,敌人并不打算,留给琅琊地灵充分的时间。

眼下敌人已经逼近云阁,琅琊地灵只得硬着头皮抵御。

下一刻,地灵便转头,对角落里的那位墨人蛊仙道:“那个老墨啊,你先上场叫阵,注意要尽量拖延时间啊。”

“是,琅琊大人。”墨人蛊仙微微一礼,悠悠飞出云阁。

“方源,你随时准备出手。”琅琊地灵又转过头来,叮嘱道。

方源点头,心中则在暗思:“凭借自己的星道手段,还是能和回风子、贺狼子这等人周旋一二的。想要取胜基本上不可能,但拖延时间却可以。但是面对秦百胜的话,非得使出全力,单靠星道这边的修为是万万不成的。”

正想着的时候,陡然间从前方传出一阵极其强烈的压迫。

“这是什么?!”恐怖的压力,让方源和琅琊地灵面色骤变。

他们连忙望去,但视野中只有一片灿烂的金光。

金光散去之后,秦百胜傲立在半空中。身后的黑城惊骇地望着他的后背,像是看到了一个怪物。

而在秦百胜的手上,则提着一个脑袋。

正是刚刚墨人蛊仙的头颅!

至于这位蛊仙的身躯,却是刚刚坠落地面。

秦百胜竟然在这呼吸之间,斩杀了一位六转的蛊仙!

何其速也!

这是何等的战力!!

正面击杀了一位全身心都十分戒备的蛊仙。即便是异人,也过于恐怖了。

饶是见多识广的黑城,此刻也惊呆了。虽然他曾经亲眼目睹过,黑城一招压服贺狼子的场面。

但压服和斩杀蛊仙,是两个概念。

对面的蛊仙刚刚飞出来,才刚刚张口。想要说话。

秦百胜就动手了。

连让人张口说话的机会,都没有留给对方。

刚刚的那招究竟是什么?

在场的蛊仙和地灵,都没有看出分毫线索来。

“可怖!这秦百胜的战力,已经超越了石磊,完全和凤九歌一样的层次了。快快撤退!”方源对着琅琊地灵大吼。

琅琊地灵也同时反应过来。方源大吼的时候,他就一把抓住方源的胳膊,倏地一下,消失在原地。

“怎么回事?”另一处云阁中,墨坦桑镇守着,看到忽然出现的方源和琅琊地灵,感到意外。

方源脸色难看至极,他不知道秦百胜背后的隐秘。但刚刚的斩首,让方源印象极为深刻。

琅琊地灵满脸惊惶,四处打转:“这下可如何是好?”

他忽然身躯一震。抬头望向西北方向,目光似乎穿透了云阁,流露出惊恐的情绪:“他们过来了,直朝我们这里杀来。这,这可怎么办?!”

墨坦桑上前一步。

他看了一眼“陌生人”方源,果然没有认出方源的真面目。

墨坦桑也没有细究方源的身份。而是关切地向琅琊地灵询问。

琅琊地灵没有回应他,十分焦急。这样的状况,墨坦桑也是首次看见。

倒是方源开口。满怀忌惮,语气沉重地为墨坦桑说明:“那位墨人蛊仙死了,连一招都没撑过去……就被秦百胜斩首!这种战力,已经可以媲美八转,只有实力差距过大,才能出现这种直接碾压的战果!”

“什么?墨长老死了?!只是被秦百胜一招就杀了?”墨坦桑浑身一震,脸上也涌现出难以置信的震惊神色。

很快,他脸上的震惊退散,目光又变得呆愣。

墨人老蛊仙的死,对墨坦桑而言,是个沉重打击。

墨人城总共就两位蛊仙,如今只剩下一位墨坦桑了。损失之惨重,可见一斑。

“秦百胜怎么可能这般强大?”墨坦桑很快就满头的冷汗,他到底也是一位雄主,反应过来后,立即明白眼前只能依靠深不可测的琅琊地灵了。

他跨前一步,对琅琊地灵问道:“琅琊地灵大人,这可怎么办?我们拖延不住了!您的八转仙蛊屋炼炉,不能爆发威能,提前将这些蛊仙手中的仙蛊强行炼化吗?”

八转仙蛊屋……炼炉?

“这就是琅琊地灵的底牌吗?听这话音,这种仙蛊屋,居然能强行炼化他人的仙蛊?”方源听到这个秘辛,心头不禁一震。

一切恍然!

难怪琅琊地灵,能够抵御众多蛊仙的攻打入侵。难怪凤九歌,都要陨灭于此。

有这八转仙蛊屋炼炉,只要拖延一段时间,蛊仙手中的仙蛊就被地灵炼化。偏偏蛊仙在这里,只能用仙蛊以及仙道杀招。

若是仙蛊都被人炼化,那么蛊仙也只能任人宰割了。

难怪琅琊地灵,能够活捉生擒那么多的蛊仙。

八转仙蛊屋,能够隔着仙窍,炼化其他蛊仙掌控的仙蛊,这威能真是不可思议的强悍了!

不过想想,也不奇怪。

长毛老祖就是八十八角真阳楼的创造者,再创造出一座八转仙蛊屋炼炉,完全说得过去啊。

“要是能强行炼蛊,我早就动手了。怎么办?怎么办?时间不够啊!!”琅琊地灵满屋子乱转,抓狂地喊道。

墨坦桑也帮助抵御过两波攻势,他从未见过琅琊地灵这般慌张,一颗心不禁沉入谷底。

“难道我不仅玩坏了整个北原,连琅琊福地都被我影响到,要提前毁灭了吗?”方源亦是心情沉重。

ps:看了一下,5号欠一更,6号7号分别两更,欠大家五更。之前说过双更的,因为生病的缘故,导致了这种结果。虽然本月全勤没有了,但我也要咬牙补上。重拾节操!实在不愿再辜负一直默默支持我的诸位伙伴了……(未完待续)

\end{this_body}

