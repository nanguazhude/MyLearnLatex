\newsection{真意灌体}    %第三十八节:真意灌体

\begin{this_body}

%1
黎山仙子继续道,语气感慨:“关押狂蛮魔尊真意的这个传闻,据说来源于当今北原七转霸仙楚度。传说他在三百年前成就力道蛊仙时,就是在这处冰原之上渡劫。只是这么多年来,蛊仙本来就稀少难成。这新晋的蛊仙中,变化道稀少,力道更少得很。因此一直难以证明这个猜想。”

%2
太白云生冷哼一声,不再向方源疾驰靠近。

%3
虽然黎山仙子、黑楼兰欺瞒了他和方源,但毕竟落到了这样的天大好处,太白云生心中的怨气也就消散了大半。

%4
他回过头,用深沉的目光看向黑楼兰。黑楼兰被三气气茧包裹着,一切进展得都很顺利。气茧体积徐徐缩减,有条不紊。

%5
“此女野心真的很大,成为十绝真仙还不满足,还想师从狂蛮魔尊。也不怕自己实力不够,撑破了肚皮。说起来,这样冒险的风格倒是和方源很像,都是疯子!”太白云生心中暗叹。

%6
“果然如此,这些都是真意外显,啊哈哈哈!”方源大笑着,脸上神采飞扬,从半跪于地到缓缓站起。

%7
他再望面前的上古天马群,这一次他的目光已然变了。

%8
变得灼热万分,仿佛是憋了几千年的色狼,看到了绝世的美女。又仿佛是贪婪永不满足的盗贼,看到了世间最大的宝藏!

%9
“我晋升成力道蛊仙,但自己的力道底蕴终究还是浅薄,力道境界只是力道准宗师!境界的每一丝提升。都万分困难,要从日积月累的量变中引发出质变来。但现在,只要我打杀掉这些真意外显。就能迅速拔升力道境界,将其提升到真正的宗师境地!”

%10
五百年艰难打拼的记忆和经验,让方源无比清楚:提升境界的难度。

%11
境界,代表的是这对大道的理解!力道境界,就是蛊师对力道的理解。大概分为普通、准大师,大师级,准宗师级。宗师级,大宗师级。

%12
绝大多数蛊师,终其一生。都是普通级。准大师级,是那些天才之辈,或者年老成精的资深蛊师。

%13
大师级就已经万中无一!譬如刚刚过去的王庭之争,偌大的北原。人杰汇聚。也独独只有方源、太白云生两个飞行大师级。奴道大师,也就是江暴牙、杨破缨、马尊、常山阴(方源)、努尔图这五位。

%14
而论炼道大宗师,整个人族历史中,只有长毛老祖、天难老怪、空绝老仙三位!

%15
境界每提高一个层次,都是对战斗力天翻地覆的提升!

%16
皆因高境界,就是对天地法则的更深理解。能令蛊师炼制出更优良的新蛊,创造出更多的杀招,更能发挥出该流派最优势的地方。

%17
方源对各大流派的理解。最高的就是血道。在他前世五百年,他是声名赫赫的血道宗师!

%18
因此。他战力卓绝,创造出许多血道新蛊,这些新蛊在某些方面甚至超越血海老祖!

%19
他还研发了许多血道杀招,尽管这些都是凡道杀招,但威力绝伦。

%20
方源如今有仙道杀招万我,乃是压榨出全部潜力,师从自然,又借助智慧蛊开创的超绝杀招,奴力合流的新起点。

%21
但他即便有这样的杀招,论战斗力,也不是前世五百年巅峰的血道对手。

%22
方源单凭前世的血道杀招相互组合,就能压过仙道杀招万我一头。

%23
他甚至还凭借宗师境地,从无到有,创造出一些血道全新仙蛊的残方。当然这些残方的完善度,都很低就是了。

%24
这是蛊虫的世界,更是蛊师的世界。

%25
蛊虫养用炼,根本的主体还在于蛊师。

%26
正是一位位,一代代的蛊师推陈出新,不断地用丰富的创造力,去研发新的蛊虫,开辟新的流派,才有灿若繁花,涵盖方方面面的,万分精彩的蛊道!才有了群星璀璨,英雄辈出,浩瀚磅礴的人族历史。

%27
蛊是天地真精,人乃万物之灵!

%28
过去的先贤,创造出一个个的巅峰,纵然高山仰止,是后人难以超越的伟大成就。但总体而言,整个五域的综合实力都是在不断进步,越发深厚,超越太古、上古等等时期。

%29
“蛊师养蛊、用蛊、炼蛊,究极原本,皆是人对自然、对天地、对大道的探索和学习。什么仙侠世界、魔法世界、科学世界,都是一样,都在探索学习整个天地自然。只是根据世界法则不同,探索的方式就不太一样。”

%30
只有不断的探索,才能学习。只有学习,才能积累。只有积累,才得超脱!

%31
“而在这个世界里,衡量学习天地大道的阶段性成果,就是各道的境界!大道是一本无字天书,蛊师探索天地学习天地,需要通过蛊虫旁敲侧击,需要不断的总结、领悟、需要灵感,需要充足的体验和经历。但此刻,我得到真意灌体,就是站在巨人的肩膀看世界!是让狂蛮魔尊将无字天书,翻译成我能懂的文字,然后让我直接阅读。这是多么宽敞的捷径,千载难逢的机缘!省却我今后无数的苦功!”

%32
方源越想越是兴奋。

%33
“黑楼兰!你就是个疯子!居然想在师法自然的同时,还要师从狂蛮魔尊,得到真意灌体。哼,也不怕自己领悟不过来,手忙脚乱,最终连累自己升仙失败!不过……你的贪婪我很喜欢。你的这份厚礼,我就不客气的全收下了!”

%34
方源心中暗自呐喊,重利当前,他决定不给留下分毫。

%35
万我!万我!万我!

%36
他不断催动仙道杀招,头一次不计成本,疯狂地催动!

%37
无数的力道虚影,被他营造出来,顷刻间在仙窍中又形成力道虚影大军。规模比之前还要庞大得多。

%38
与此同时,组合仙道杀招的凡蛊大量损毁,这是天地反噬。也是方源疯狂催动的恶果。

%39
但方源毫不心疼,只要核心净魂仙蛊没事就行了。他此行做了充足战备,拥有大量的候补凡蛊可以替换。

%40
“来吧!”方源张开胸怀,从仙窍中喷出无数力道虚影。

%41
力道大军宛若洪流,倾泻天地,卷席山河。原本稀疏的军力,立即得到大力补充。很快就超过原先的极限,达到新的高度。

%42
大量,不。海量的力道虚影,充斥整片战场。

%43
“这至少有三十万!”这一幕,把太白云生、黎山仙子都看得呆了。

%44
上古天马群被力道虚影大军,死死地包围在半空。

%45
这一刻。两军剿杀达到白热化的程度。

%46
方源安步当车。走在战场下方的冰面上。

%47
他的脑海中,乐意疯狂生产,又迅速消耗。大军在他的指挥下,形成默契配合,一股股的军力,宛若一只只巨手,不计损耗地将一头头的四翼天马都扑打下来。

%48
随后,数十位力道虚影合力。将四翼天马强行摁倒在冰面上。

%49
再此之后,方源徐徐走来。抡起八只粗壮的巨臂,掀起如幕般拳影,狠狠打杀。

%50
在嘶鸣惨叫声中,一头头天马被打成粉碎。真意直灌而来,方源的脑海受到冲击震荡,,大道奥妙被方源迅速吸收到心底。他的动作僵硬了几个呼吸之后,旋即又恢复自如。

%51
就这样针对一只只上古天马,方源疯狂杀戮。

%52
“爽啊!”又得到一次真意灌体后,他兴奋地仰头嘶吼,目光如血,不可逼视。

%53
以数年,十数年为单位缓慢提升的力道境界,随着一次次真意灌体而迅速提升着。

%54
方源感到无比的快意和享受。

%55
太白云生不语,方源的疯狂让他心中发寒。黎山仙子那边也不由地沉默下来。

%56
天马群被消灭了一小半之后,地灾重新变化。天马群相互融合,变成三十三头巨蛇。

%57
这些巨蛇,浑身长满了黑色的硬甲,盾牌也似。防御力十分强大,蛇头大如房屋,落到地上,占据地利防守。

%58
“又变成盾蛇了?”方源狞笑一声,身躯未动,半空中的力道大军已像乌云狠狠盖压下来。

%59
占据了地面,这些盾蛇只要防守上方即可。不再像飞走半空中,四面八方头上脚下都是可以进攻的空挡。

%60
方源杀戮效率顿时骤降。

%61
好半天,他终于指挥大股军力,搬倒一头盾蛇。

%62
方源哈哈狂笑,飞扑上去。

%63
盾蛇伤痕累累,还待昂首起来,结果遭到方源八臂齐轰,四十五只群力蛊叠加的庞大巨力,直接将盾蛇蛇头打得稀巴烂。

%64
一股远超之前的庞大真意,冲击方源的脑海。

%65
失去蛇头的大半截蛇躯,忽然分散开来,化作一只只小蛇,转眼间形成一片狰狞滑腻的蛇海。

%66
方源体悟大道真意,身体动弹不得,一头栽进蛇海当中。

%67
蛇海疯狂攻击,将他庞大的身躯迅速淹没。

%68
方源这一次的感悟,又和之前不同。之前真意灌体,方源仿佛化身天马,从出生到死亡,从小到大,四蹄飞奔踏在坚实的地面上,飞翼扑扇翱翔在广袤的天空中……

%69
他感受着天马独特的身体构造,感受到种种力的作用,发力的巧妙。

%70
而现在,他化身成蛇,从破开蛇蛋钻出来,到浑沦吞枣般捕食一只只猎物。他身临其境地感受到蛇躯的蜿蜒,蛇躯缠绕的力量,蛇头张开吞咽食物的肌肉韵律,捕食时蛇头如闪电般出击的力量调动……

%71
不仅仅是力道境界的提升,而且还有变化道,甚至飞行大师造诣的拔升。

%72
这些体会延伸下去,就涉及到力道至理。这些至理,涵盖信息极其庞大,又难以用人们交流的语言去形容。因为大道至妙,妙不可言,语言和其相比太过简单、枯燥、浅薄,根本难以描绘出玄奇妙理!

\end{this_body}

