\newsection{黑楼兰升仙(中)}    %第三十五节:黑楼兰升仙(中)

\begin{this_body}

“有了小家子气蛊,就有了影响天灾地劫的手段。<-》”方源静立一旁,观察着。

狂风呼啸,大雪席卷,夹杂坚硬如铁的冰雹,接连不断地砸在方源的身上,方源纹丝不动,仙僵之躯不受分毫伤害。

砰砰砰……

黑楼兰不断洒出小家子气蛊,待这些上古气道蛊虫汲取天地二气达到极限,就立即回收。在这个过程,许多小家子气蛊径自爆开,汲取的天地二气又回到天地之间。

蛊师升仙,天地二气剧烈波动,不管是凡蛊还是仙蛊,都会受到反噬。

黑楼兰尽管准备了小家子气蛊,但充其量只能对天劫地灾稍作影响,还达不到控制的程度。

随着时间渐渐推移,不可避免的,天地二气越发浓郁,就要到产生天劫地灾的关口。

鬼哭神嚎的狂风,黑楼兰黑发飞舞,发梢和冰雪一起拍打在她的脸上。用蛊虫生成的衣服,显然比普通衣服更加耐用。她衣摆飘飞,遭到狂风撕扯,却仍旧坚韧不断。

方源暗暗觉得古怪:“都这个时候了,怎么仍旧不见天灾地劫生成?”

他双眼绽放奇光,催动蛊虫谨慎观察,漫天的风雪也无法阻挡他的目光。

方源提起十二分的戒备,天灾地劫,种类繁多,有的雄奇浩荡,有的诡异恐怖,真的说不准会来什么,蛊师只能拼运气,尽量准备。见招拆招。

呜呜呜\&

风声越加尖锐,几乎要刺破耳膜,方源听着一阵心烦气躁。

忽然。他猛地惊醒,强催侦察蛊,望进云层当去。顿时见到,一道紫色音磬成形,此刻缓缓转动,发出呜呜之音。

原来,天灾已经悄然生成!

呜呜之声越发高亢尖啸。混合在风,广布方圆上百里!

方源就算捂住耳朵,也阻挡不住这股声音。他虽然也准备了一些音道凡蛊。但音道造诣极低,他难以有效抵御。

他的心越加烦躁,心脏随着尖锐啸音急速跳动,浑身血液开始逆流。

另一边。黑楼兰闷哼一声。口鼻溢血,情况比方源还不堪。

方源眯起双眼,寒芒闪烁:“这是惊心音劫,音劫灌儿,蛊师听了,就会血液倒流,心脏越跳越快。时间一长,就会血液逆冲七窍。心脏轰然自爆。我是仙僵之躯,尸血冰冷。几乎不流动,所以受到的影响较小。但黑楼兰却是大力真武体,气血最是旺盛,恰被惊心音劫克制!必须摧毁音劫源头的紫色音磬!”

念及于此,方源再无犹豫,催动轻虚蝠翼,一飞冲天。

他迅速飞升,越靠近紫色音磬,他就越是心惊。距离劫云三百步时,他心脏跳动到极致的速度,甚至连浑身的筋肉肉都跟着一起跳动起来。

忽然,方源身形一滞,大口一张,吐出一口惨绿惨绿的尸血。

就在刚刚,他的心脏终于承受不住,彻底爆裂。

但方源乃是仙僵,心脏已经不是他的弱点,这样的伤势反而激起了他的凶性。

他哈哈怪笑,八臂齐扬,杀招猛地爆发。

冰钻星尘!

八道星团迅速成形,一齐向劫云打去。

它们钻进劫云,接连轰紫色音磬。音磬并不坚固,瞬间爆碎。惊心魔音顿止,方源压力骤消。

但下一刻,劫云滚滚,天气汇聚,一道紫色音磬转眼间又要形成。

方源眼冷芒频闪,天灾地劫怎么可能会这么容易,被一击即溃?这番变化,并不出他所料。

冰钻星尘!

他八臂聚拢一处,再次催动攻伐杀招,凝聚出一团老大的冰钻星尘。

深蓝色的星光,一颗颗星尘钻石般璀璨闪耀,不断相互乱碰疯撞。

方源轻喝一声,八臂齐齐用力一推,将星团推上劫云深处。

星团体型庞大,速度就变得缓慢,但轰碎再次形成的紫色音磬之后,仍旧有大股星团残留在原处,发出咔咔砰砰的声响,不断自爆。

天气凝聚,不断形成紫色音磬,又瞬间被星团磨灭。

星团以肉眼可见的速度迅速消融,但惊心魔音却是再没有传入方源耳。

方源成功压制了天劫,却仍旧没有放松警惕。十绝体晋升蛊仙,怎么可能如此简单?

他持续发出冰钻星尘,这杀招虽是凡道杀招,但威力着实不俗,再融合了都敏俊传承之后,可值四块仙元石。

冰钻星尘不断得到补充,紫色音磬在其不断生灭。

方源抽空低头,望了一眼下方的黑楼兰。

后者呼吸再次稳定,双眼仍旧紧闭。她神色坚定,凌空盘坐起来,岿然不动。白皙娇丽的面庞上,残留着风干的血痕。

但大力真武体恢复力惊人,称得上十绝第一,黑楼兰刚刚受的伤这个时间已经彻底痊愈。

她手动作不断,一手洒下小家子气蛊,一手则专门回收。她显然改变了战术策略,小家子气蛊多洒在地面上,很少飞向空。

由此可以推断,小家子气蛊的数量并没有多少,毕竟是上古气道蛊虫。就算是有蛊方在手,炼蛊的材料也十分难寻。

黑楼兰就算经过了长时间的充分准备,手积留下来的数目,也是不多。

方源刚抽回目光,就在这时,耳畔传来嘤嘤嘤的声音。

声音凄凄切切,宛若深宫怨妇哭泣,又仿佛少女思情郎的梦呓语,压低了声音,辗转悱恻,充满缠绵温柔,又有情仇哀怨。

方源听了这音,饶是仙僵之躯,也感到一阵阵的虚弱。浑身使不出力气似的感觉,英雄气短。陷入无边的温柔乡,一身壮志雄心难以舒展。原本坚实的拳脚,仿佛陷在棉花当。又好像大病初愈,虚不着力。

方源暗叫一身糟糕,催动侦察蛊抬头一望,果然见到紫色音磬不远处,,又有一道粉色音磬生出。

音磬不断震动,发出呜呜咽咽的声音。

这是靡靡音劫!

不待方源催生杀招。轰破粉色音磬,紧接着又有魔音贯耳。

咚咚咚!

声震如雷,又仿佛巨人敲鼓。每一声都似乎敲击在方源的脑海深处,方源脑海,意念顿时运转不开,被一声声炸雷声音震散震碎。

震念音劫!

仙僵原本就思维僵化。这天劫恰好克制方源。

方源顿感强烈的眩晕。无法思考,身躯剧烈摇晃,险些就从高空一头栽落下去。

危难关头,他仅凭一丝清明,催动乐山乐水蛊。

连续三颗青提仙元,灌进乐山乐水蛊,磅礴的乐意喷涌而出,顷刻间蔓延方源整个脑海。

方源刚刚稳住身形。咻!

忽然间,耳边传来尖锐音啸。急速逼近。

方源下意识伸手去挡。下一刻,他五根手指头直接被切断,切下的五个指头被风雪夹裹,立即消失在茫茫风雪当。

咻咻咻……

尖锐音啸连绵不绝,再次向方源逼近。

方源瞳孔猛缩,凝聚目光,连续转换三种侦察手段,终于隐约看见,风雪夹杂着一根根透明丝线,锋锐非凡,勾连成一张巨网,铺天盖地般地向方源罩下。

这些透明丝线,并非有形之质,而是全部由声音凝聚。

方源看得寒毛直竖:“这是飞刃音劫!发甲!轻虚蝠翼”

催动防御杀招,旋即又催移动杀招。

方源连续闪烁,黑甲块块崩碎,离开发甲主体后,碎块就化为蓬蓬的僵尸黑毛。

方源左右腾挪,终于寻到一个较为疏漏的缝隙,突破了音线巨网。

但透明音线源源不断,仍旧朝他打来。发甲又支持了三个呼吸,终于达到极限,轰然破碎开来。

方源冷哼一声,连忙再催防御杀招,形成新的发甲,勉强罩住身躯。

他左闪右躲,身形宛若鬼魅虚影,同时一团团的冰钻星尘暴雨般射上去。

星尘碰到音网,被尽数拦截,切碎成片片星辉。

趁着时机,方源的手指头迅速生长出来,爆碎的心脏也重新凝聚成形。

呜呜呜……

没有了星团持续补充,第一个紫色音磬终于生长出来。

鬼哭神嚎之音再次响起,方源浑身气血逆流,刚刚恢复好的心脏,又再次乱跳,走向自爆的末路。

方源目光凝重。

不管是冰钻星尘,轻虚蝠翼还是发甲,都已经被催使到极致境地,却难以抗衡危局!

这还只是黑楼兰的天劫,地灾因为小家子气蛊的压制,还没有彻底形成。

方源脑海,庞大的乐意迅速消耗,短短功夫,已经消耗了十分之一。

“我现在打出的战力,已经是转上等,结果都奈何不了这个天劫。十绝体升仙,天劫地灾果然恐怖,是寻常蛊仙的数倍!嗯?”

忽然间,方源面色一变,心陡沉下去。

他发现,在这劫云深处,又有四道音磬,颜色各异,缓缓成形。

刚刚的四道音磬,就已经让方源疲于应对。现在又生出四道来,方源唯有动用仙道杀招万我了。

太白云生在暗处看得咋舌不已:“看来十绝升仙的难度不是数倍,而是寻常蛊师升仙的至少十倍啊。”

他原本以为自己渡劫,已经很难了。没想到,黑楼兰的难度比他更高,而且高得多!

“糟糕了,这不是平常天劫,而是十大凶灾之一的八重魔音劫,八音齐发,浩荡八百八十里,天翻地覆,惊神灭仙!”远处,一直隐藏形迹的黎山仙子也侦察到八只音磬,认出门道,差点忍不住出手想帮。

“八重魔音劫……如果任凭八音齐响,必然威能超绝,此次渡劫提前失败,再无一丝成功希望。”方源目光冷然,局势越是困难,他的战意却越加昂扬。

啪啪啪。

整个过程,他使用的凡蛊,都在接连不断的爆炸。

天地反噬,令他损失了大量的凡蛊。

不过方源准备也相当充分,备用蛊虫仍旧绰绰有余,致使杀招从未断绝。

就在他准备催动万我的时候,下方的黑楼兰陡然睁开了双眼!(未完待续。。)

------------

第一千五百二十一章 金鹰

石碑硕大无比,遮天蔽日,金鹰身前的无数兵士被其一扫,全都变成肉饼,麻雀般从空中掉落。

“可恶!”

感受石碑散发的力量,金鹰知道此时继续攻击万力王的话,必然会被击中,虽不致死,也肯定不会好过,身受重伤是难免的。

一声长嘶,一柄奇特的兵器,横空出现,这件兵器似刀非刀,似剑非剑,看起来又有些像枪,虽然四不像,却威力无穷,右手紧握,全身青筋迸出,枪芒射出,璀璨如星,向疆域图迎了过来。

“金鹰我来助你!”

一侧准备偷袭聂云的暗枭,见后者当先出手,直对金鹰,目光一寒,一拳轰击。

他的拳头上面出现出一个山岳模样的铁甲,如同强大的利爪,还没来到跟前,朔风呼啸,气象万千。

“半步主宰强者?不愧是乾血王朝,还真够厉害的!”

这二人一出手,聂云顿时看来出来,眼神凝重。

之前以为独孤彦君是个巧合,凭借琉璃塔众女这才侥幸突破,个人实力在整个乾血王朝都能数得着,现在看来,还是太小看一个驰骋不知多少亿年的帝国了。

能将四大宗门压的抬不起头来,自然有其道理。

关于四大宗门,其他聂云并不知晓,单说归墟海,所谓的十大长老,连宗主级别都没达到,不过两千八百多条大道实力,宗主古雍。也只是宗主巅峰强者,距离半步主宰也还差一段很大距离。

当然,这些也可能只是明面上的力量。归墟海这样的大宗门,真正强大的应该是太上长老或者无上长老,而这种级别的强者他一个没见。

思绪在心中一闪而逝,大道圆环在体内疯狂旋转,眼神如冰,操控着疆域图,和对方的拳头、怪异兵器狠狠撞在一起。

轰隆!

狂暴的气浪以碰撞处为中心向四周扩散。强大的能量波撕开空间,蔓延出一条条漆黑的裂缝。

“啊……”

“救命……”

无数靠近战斗中心的兵士,连逃跑的机会都没有。被瞬间撕扯进入碎裂的空间,变成虚无。

他们这种强者的战斗,根本不是一些普通兵士能够插手的。

呼!

倒飞出去,聂云将疆域图收回体内。全身气血沸腾。脸色一红。

虽然疆域图威力无穷,但对方的实力太强了,二人都是半步主宰联手攻击,让他还是气血短时间内提不上来,嘴角发甜,身形不稳。

他感到力量虚浮,暗枭、金鹰也不好过,强大的反震力。让二人同时翻了个跟头,一个个面容煞白。

“这家伙古怪!”

稳住呼吸。暗枭、金鹰看向聂云眼中露出浓浓的忌惮之意。

本来他们以为遇到这小子,轻而易举就能斩杀,现在看到对方有食界蚁、诸多妖宠等手段,并且还有如此强大的实力,终于明白独孤彦君为何被杀了。

连他们二人联手,都狼狈不堪,一个独孤彦君又怎么可能是对手!

“兵士听令,组成大阵,防御外敌!”

明白眼前这人实力强劲,很难是对手,金鹰一声长啸,声动四方。

凭借他们二人想要击杀眼前这人很难,看来只能依靠兵士阵法取胜了。

“组成阵法,来不及了!”

喊声刚刚响起,金鹰就听到兵士上方一个大笑响起,一个人影出现在上空。

这个人影和不远处的聂云一模一样,甚至灵魂气息都完全相同,不用看就知道,必然是分身之类。

此时这个分身身体一晃,无数藤蔓从体内猛地蔓延而出,对下方的兵士冲了过去。

每一根藤蔓都有宗主级别实力,数百条藤蔓加起来,宛如一道强大的力量洪流,所到之处,兵士都不是一合之敌,纷纷变成干尸,从天空坠落。

“天心藤……糟了,这东西每吞噬一人,力量就会增大一分,连续吞噬这么多兵士,会蔓延出更多藤蔓,必须阻止……”

暗枭瞳孔一缩。

兵士组成阵法,必须有足够的人力和时间,突然多出的藤蔓,到处杀人,这些兵士还没将阵法形成,就被击杀,继续下去,只会越死越多,想要联成阵法,几乎不可能完成!

“天心藤需要强大的意念维持,那个必定是本尊,现在这个应该是个分身,金鹰你出手对付,我去拦住他的本尊!”

明白危急,暗枭吩咐转身就要冲过去,还没离开,巨大的石碑图再次飞了过来。

石图带着毁灭的力量无穷碾压,没留丝毫后手,暗枭还没走远,就被死死逼了回来。

“先把他这个分身灭了再说!”

被疆域图逼迫,暗枭有些喘不过气,脸色一沉,眼中杀机迸射。

“好!”金鹰也点了点头。

不管你实力如何强,分身就是分身,不可能太过厉害,再说,灵魂被分割,很难达到圆满境界,合力先将这个分身击杀,把眼前这个石碑图抢来,看你还能不能跳起来!

二人有了决定,兵器重新祭炼出来,半步主宰强者的强大力量,挥洒出来形成力量洪流,笔直向前狂涌。

“死吧!”

拳头一挺,暗枭拳头向前平伸,和金鹰掌心的怪异兵器再空中交织在一起,形成一条璀璨的光带,将周围的空间撕扯成一片真空地带,咆哮而来。

二人的联手很明显比刚才更加强大,不是简单一加一的问题,即便遇到半步主宰中三重的强者,恐怕也能瞬间击杀。

呼!

知道这道光芒的可怕,聂云露出凝重之态,并不硬抗,身体一晃,凤凰之翼从背后展开,整个人眨眼功夫跳跃不知多少公里,穿越了不知多少层空间,下一刻,已经出现在光带的上方。

石碑图再次从他掌心射出,借助天地威势,向下横劈。

“什么?怎么会这么快?难道这才是本尊,那个施展天心藤的是分身?”

看到聂云眨眼功夫逃过他们的联手,金鹰、暗枭对望了一眼,满是惊讶。

本以为这个是分身,天心藤是本尊,现在看来,根本就搞错了,如果是分身的话,不可能躲过他们这样强大的必杀一击。

总不可能有人拥有和本尊实力相同的分身吧!

“声东击西!”

惊讶一闪而逝,暗枭对金鹰点了点头,一股意念传递过去。

“一起动手!”

意念传递完,二人同时大喝,力量再次融合,粗大的光芒瞬间来到聂云跟前。

看他们的样子,很明显是先联手击杀他再说。

手掌再次一甩,疆域图碾压,将光芒挡在外面,聂云身体向前一纵,向暗枭抓了过来。

“哼!”

暗枭铁拳横立,猛地一扫,诸多大道形成一条条长河。

长河和聂云的手掌一碰,两股力量消散,二人再次后退。

“哈哈,我先杀了你这个分身再说!”

和他硬碰硬,暗枭已经知道了眼前这个必然是本尊,和一侧的金鹰对视了一眼,后者哈哈一笑,古怪兵器猛地和身体粘连在一起,整个人向天空一跃,变成一头巨大的神鹰。

刚才的古怪兵器居然是他的翅膀上最坚硬的羽毛!

嗖!

变回原形金鹰翅膀轻轻一闪,立刻向前窜出,速度之快令人咋舌,眨眼功夫就来到满是天心藤的聂云跟前,目光一闪,浮现出一道狠辣之意。

“死吧!”

一根尖锐的羽毛笔直射出,像是刺破空间眨眼功夫来到聂云跟前。

如果这个聂云是分身,如此犀利的攻击,必然接不住,被当场格杀。

这也是金鹰、暗枭故意施展的计策,一个拖住他的本尊,一个偷袭击杀分身。

分身被杀,本尊必然受到影响,到时候,再双管齐下,不愁这位聂云不死!(未完待续……)

\end{this_body}

