\newsection{大而化小压群魔}    %第一百五十二节:大而化小压群魔

\begin{this_body}



%1
原来,在太丘战场,自从方源等人走后,东方长凡渐渐压住内患,令战局有所起色。

%2
他老谋深算,方源走后不久,他就立即意识到了这个情况。他当即就猜到,这三位神秘的黑袍蛊仙极可能去攻击他的老巢。

%3
不过,他在碧潭福地中,早就有所布置,并不担心,反而乐见这些魔道蛊仙分兵,好让他能够安心从容地对付自在书生!

%4
但很快,他惊骇欲绝地发现,自己竟然和茅草屋失去了联系!

%5
这一惊,非同小可。

%6
茅草屋乃是东方部族的第一重宝,且被东方长凡郑重布置,就算是八转蛊仙也能支撑片刻。怎么会无声无息地,就和自己失去了联络?

%7
东方长凡立即改变战略,舍弃太丘这处战场,向这老巢方向急忙赶回救场。

%8
自在书生见东方长凡逃走,自然在身后紧追不舍。

%9
东方长凡还有最后的关键步骤,没有告诉残阳老君。后者也跟随着东方长凡,一路且战且走。

%10
两方一边飞行,一边在高空大战,战场迅速转移,浪费了不少时间后,终于是赶到碧潭福地。

%11
茅草屋已经不见踪影,碧潭福地也被打坏,形成破洞。

%12
见到魔道蛊仙们正在其中肆无忌惮地搜刮资源,东方长凡怒气勃发。他又见奴道蛊仙郄世民,居然正用独特手段,大肆捕捉东方一族的蛊师。

%13
东方长凡气得无数星念。冒出头顶,将头上的发冠都冲掉了下去。

%14
这是名副其实的怒发冲冠了。

%15
“魔道贼子,罪不可恕!”东方长凡手一指。不顾巨大的仙元损耗,再次催起底牌杀招万星飞萤。

%16
黑楼兰等人还在攻打着嫣然海,见无数飞星洒下,宛若雪花飘摇,星光烂漫,美不胜收。

%17
如此美景,却让众人警惕万分。连忙止住手脚,改变攻势。齐齐飞上高空,杀向东方长凡。

%18
此刻情形,却和太丘时有很大不同。

%19
在太丘战场,一干魔道蛊仙为了一份智道传承。激烈抢夺,相互竞争。偏偏东方长凡夺舍,这种智道传承还在他的脑子里,成功夺走传承的希望很小。

%20
但现在,广阔的碧潭福地中,无数的资源像是敞开怀抱,任由君取。

%21
重利摆在眼前,只要杀了东方长凡,就能继续抢掠。因而魔道蛊仙们一扫之前内斗羁绊,一起动手,神情振奋。颇有众志成城之感。

%22
魔道蛊仙们合力围攻,东方长凡脸色变得相当难看,感受到压力重重,左右遮挡,立即处于下风。

%23
“杀了他,整个碧潭福地就都是我们的!”半月蛮师叫嚣道。

%24
“除了这碧潭福地。还有东方老贼的智道传承,那可是北原当代第一的智道传承啊!”孔雀飞仙何若双眼放光。

%25
“就算智道传承我都可以放弃。老子只要他的夺舍之法!”皮水寒面容冷峻,手上不停,攻势宛若排山倒海。

%26
东方长凡连连爆退之后,终于在残阳老君的帮助下,稳住阵脚。

%27
被这样一压,他心中的愤怒积蓄到了极点。得到喘息之机,他仰头怒极反笑:“我要让你们统统死在这里!”

%28
话音刚落,整个碧潭福地发生剧烈的震荡。

%29
天摇地动!

%30
魔道蛊仙们骇然发现,自己手中的凡道杀招统统失去了效用。

%31
东方长凡担任东方一族多少年的太上大长老,对碧潭福地有许多掌控权力。虽然碧潭福地是公共福地,他不能操纵全部,但手头上的权利,已足够令碧潭福地排斥这些魔道蛊仙了。

%32
碧潭福地是东方部族的大本营,东方长凡在这里战斗,自然有地利。

%33
之前只是主人不在家,来了强盗。

%34
魔道蛊仙攻势顿时孱弱下来,东方长凡没有趁胜追击,反而循着空隙,飞身而下。

%35
他直接俯冲到嫣然海禁地,将幻境被破,好端端的花海桃林一片狼藉,桃太狼群更被斩杀得只剩下三两只,有几只狼尸还未来得及被收走,倒在桃树下,惨不忍睹。

%36
东方长凡看在眼里,心疼得要滴血。这多少年来的辛苦积累啊!

%37
他强忍痛楚,对下方大喝:“方兄,我护你族数百年,定下盟约,相互守望,不曾有一天的异变欺压。此时却是我族生死存亡之际,还需要你出手相助。”

%38
东方长凡的声音,在空中激荡。

%39
几个呼吸之后,一个尖细的声音,同样响彻天地,回应道:“正该如此。”

%40
说着,众人便见一道翠绿之光,从嫣然花海中电射而上,落到东方余亮的手中。

%41
翠绿光华消散,现出一座微型山峰。

%42
正是方寸山!

%43
“这就是方寸山了!”众魔目光顿炙。

%44
皮水寒、黑楼兰等人,神情却是微凝。

%45
听东方长凡和小人蛊仙的对话,原来方寸山并非是东方一族所有,而是小人族和东方部族达成协议,成为盟友。

%46
能和超级势力东方部族成为盟友,小人族自然有着相应的底蕴和实力了。

%47
方寸山在手,东方长凡朗声一笑,犀利的目光扫视一干魔修。

%48
他根本没有犹豫,心中的愤怒催促着他复仇!

%49
“今天就是你等的死期!”东方长凡一手持着方寸山,一手捏着指诀,操纵万星飞萤,真正杀向魔道众仙。

%50
无数星光绞动,掀起湛蓝色的滔天骇浪,他长袍翻飞,恍如神仙中人,气势暴涨无数,令群魔心中暗凛。

%51
天花黯淡!

%52
千解!

%53
冰龙锁!

%54
力道巨手!

%55
七转战力齐齐出手,四大仙道杀招,一起攻向东方长凡。

%56
凡道杀招已经试不出来,只有仙道杀招还有效果。

%57
“尽处余晖”残阳老君默默一念,分别为自己和东方长凡加持,随后他大袖一拂,甩出追命火。

%58
追命火袭上威胁最大的自在书生,后者被迫后退,大乱阵脚。

%59
自在书生气得大吼一声,好多次都是这样,追命火让他总是不得不将千解对准自己,用来阶位。东方长凡无数次挣扎在生死边缘,但就是不死!

%60
千解消散,但剩下的冰龙,力道巨手,天华,却是轰向东方长凡。

%61
东方长凡高举手中的方寸山,魔道蛊仙轰出的三大杀招,越是靠近方寸山,威力就越急剧下滑。

%62
结果具硕的冰龙化为了冰蛇,力道巨手缩水,不足原本的百分之一,天花黯淡的不是别人,反而而是自己,花瓣片片凋零。

%63
最终这些攻势,落在东方长凡面前,被他一挥长袖,轻松挡下。

%64
“仙道杀招大而化小!”皮水寒脸色阴沉如水,一口道破刚刚对方的手段。

%65
原来,这方寸山上的方姓小人族蛊仙,乃是律道蛊仙,掌握律道仙蛊“小”!

%66
他领导小人一族,和东方一族联盟,为本族求得生存空间。此刻出手,现出非比寻常的手段,竟有力压群魔之势。

%67
众魔纷纷皱起眉头。

%68
黑楼兰心中已有后悔,早知道方寸山实力这么强,她肯定不会死磕。众魔强攻嫣然海,虽然打破桃太狼群,但还未让方寸山使出这一杀招的程度。

%69
“事情有些棘手,要解决这个麻烦,就看自在书生你的了。”皮水寒回头,对自在书生大声道。

%70
自在书生冷哼一声:“关键时刻,除了靠我还能靠谁?”

%71
皮水寒眼皮子一抖,却没反驳。

%72
自在书生也是律道蛊仙,掌握着律道仙蛊“解”,同样有仙道杀招“千解”。

%73
对付律道,当然用律道,才最为有效。

%74
当即,众魔便以自在书生为首,激战东方长凡、残阳老君和小人蛊仙。

%75
群魔相继出手,为自在书生争取战机。

%76
自在书生抓住战机,使出千解,照上方寸山。

%77
千解对拼大而化小,竟被后者所克。

%78
“你那律道仙蛊,不过六转。我的‘小’蛊,却是七转层次,怎么能胜得了我?”方寸山上传出小人蛊仙咯咯的笑声,满满的都是得意。

%79
自在书生脸色阴沉,却无法反驳什么。他的底蕴和小人一族相比,还是浅薄了。

%80
情势对魔道一方,变得不利。

%81
东方长凡手持方寸山,又有残阳老君在一旁相护照顾,已然立于不败之地。只消等到万星飞萤杀招越战越多,他的战力就越提升暴涨。最终会将优势转变成胜势,从而战败群魔,夹胜强势回归北原蛊仙界。

%82
群魔苦思,却没有手段能反制大而化小。

%83
摆在他们面前唯一的战术,似乎就只剩下强大硬攻,以人数优势拼仙元消耗一途了。

%84
但魔道蛊仙独来独往,凡事都为自己考虑,都十分不愿损耗过大,得不偿失。

%85
正胶着间,忽然有魔道蛊仙周千怒吼:“该死,我的散文鲤鱼群被人收走了!”

%86
众魔循声望去,便见战场边缘,一处深潭上空方源正将潭水抽空,吸入自己的仙窍。

%87
这下子,就连黑楼兰都向方源愤怒传音:“方源,你这是干什么!”

%88
众魔本来就战斗辛苦,结果发现居然有人趁他们不备,偷偷采集资源去了。

%89
这就好像是同舟共济,众人都奋力划桨,让船前行。结果有同伴坐在船尾钓鱼,还他娘的钓来一头肥鱼,中饱私囊了!

\end{this_body}

