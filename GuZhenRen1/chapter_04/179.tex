\newsection{五德山上五德门}    %第一百七十九节:五德山上五德门

\begin{this_body}

这一天,方正听完另一位长老的公开讲授,走在回去的路上,却被一位叫做史宏的弟子拦住。

“方正长老。”史宏首先施了一礼。

方正还礼,心底还有些不好意思。因为史宏这位精英弟子,年龄比方正要大,原本是高出方正好几辈的师兄。

“请教方正长老,这次炼蛊大会的四大入会试题中,有一题是要炼制地藏花蛊。而炼制此蛊,倒数第三步,是需要动用上游草,搭配禅狮的鬃毛,再辅以炼蛊手法。但弟子到了这一步总是失败,不知道正确的炼蛊手法具体是什么?”

“这个……”方正一愣,语气迟疑。他主修奴道,控制飞鹤,战力在凡人蛊师中可谓出类拔萃。但对炼道完全不懂啊。

好在他有寄魂蚤,天鹤上人暗中传给他答案。

方正这才回答,道:“这个炼蛊手法,名为纷纷手,其目的是要将每一根上游草,和每一根禅狮鬃毛绞在一起。要炼制地藏花蛊,须得在三十个呼吸内内,将上百根草毛编织好。一旦超过这个时间,炼蛊的火焰就会将这些草、毛烤焦。因此你若不熟练这个手法,炼蛊会很容易失败。”

方正重复天鹤上人的话,越说越从容。说完这段,又接着讲述如何锻炼“纷纷手”这个炼蛊手法。

史宏神情微变,有些错愕,连忙道谢:“多谢方正长老指教,弟子受益匪浅。”

“还有什么不明白的地方吗?”方正微笑,过了一把教师的瘾。

“没有了。没有了。弟子告退!”史宏行了一礼,告辞。

两人分别后。方正走了一段路,忽然越走越慢。

他皱起眉头。醒悟过来:“这个史宏,不是来诚心讨教,倒像是故意来刁难我的。”

“呵呵,你如今也能看出来了。不错,不错。”天鹤上人笑嘻嘻地道。

方正便暗叫:“师傅,原来你早知道!”

“废话,你的来历非常容易打探,年龄有摆在那里,谁都知道你没修过炼道。史宏却故意来问你这么偏僻的问题。不是故意刁难是什么?”

说到这里,天鹤上人顿了一顿,又问方正:“你可知他为什么要刁难你?”

“为什么?”方正感觉莫名其妙。

天鹤上人朗笑一声,叙述缘由:“因为史宏喜欢女弟子伊月。而这位伊月,却是炎堂长老的女儿。炎堂长老势单力孤,在仙鹤门中遭受排挤,并不好过。他意欲将女儿许配给你,拉帮结派,和你这位仙鹤门史上最年轻的长老达成利益的联盟。要不然他会三番五次请你吃酒?还每次故意安排。让他的女儿和你坐在一起?”

“啊。”方正惊呼一声,这才明白过来,他的脑海中不由自主地浮现出伊月的美貌,坐在他的旁边。热情地为他夹菜。还向方正频频敬酒,喝几口小酒后,满脸红晕。美丽动人。

“想起来了,想起来了吗?哈哈哈。傻小子!”天鹤上人见方正愣在原地,很是开心。显然八卦精神。不分男女老幼,甚至不论生死啊。

方正无奈地叹了口气,摇摇头,厌烦地道:“又是算计,又是利益,我真的不喜欢。从今以后,炎堂长老再邀请我,我就推辞了吧。反正我也是长老,和他平级,拒绝也不算什么。”

“傻小子,这个世界里哪个组织,不是用利益维系的?真情是有,但正因为稀少,才见得可贵啊。”天鹤上人唏嘘感慨,“你别要回避这些,我劝你更不要拒绝炎堂长老的邀请。就算你不娶他的女儿,也不要恶了和炎堂长老的关系。因为现在的你,比炎堂长老还要更加势单力孤啊。”

“不说这个了,师傅,刚刚提到炼蛊大会,最近不管是弟子还是长老,也都在议论此事。这个炼蛊大会究竟是什么呀?”方正故意岔开话题。

“这个问题你问我最好,千万不要问别人。否则他们会拿看傻子的眼神看你的。我这就给你详细说说。这炼蛊大会可不一般,乃是中洲一百年才有一场的盛会。也就是说,没有特别的延寿手段,绝大多数蛊师一生中只能参加一次。”天鹤上人答道。

方正开口:“炼蛊大会,难道就是炼道蛊师参加的盛会吗?”

“不是这样的。蛊师修行,有三大方面,分别是养蛊、用蛊、炼蛊。炼蛊大会不是只有炼道蛊师才能参与的盛会,只要你在炼蛊方面有一技之长,或者有经验心得,都可以参加这项盛事。”

天鹤上人继续道:“这场盛事的规模,是空前的,绝对是天下第一的炼道盛会。每一届的炼道大会,都有数十万的蛊师参加,他们来自中洲大大小小的门派。甚至就连东海、西漠、南疆、北原的蛊师,都会出现。”

“南疆……”方正被勾起一缕情思,他不禁回想起了青茅山。

他旋即又问:“那么史宏刚刚提到的,所谓的四大入会试题,又是什么?”

天鹤上人知无不言言无不尽:“所谓四大试题,就相当于入场的资格。但凡蛊师,不管是谁,在修行中都会炼蛊。炼蛊大会奖励丰厚,吸引无数蛊师参加。若不设关卡,那滥竽充数,想碰碰运气的人就太多了。所以就有四大试题,来筛选出真正优秀的,有炼道造诣的蛊师。”

“也就是说,要参加炼蛊大会,就必须完成这四大试题。”方正恍然。

“呵呵呵。”天鹤上人笑道,“其实这四题,考察的都是炼蛊的基础技艺。有一定经历的蛊师,往往都能通过。”

方正赫然:“师傅,我就通过不了啊。”

“没有关系,在我的指导下。你这段时间疯狂集训,也能成功。”天鹤上人道。

“是吗?那我岂不是可以大开眼界了!”方正大喜。

中洲。五德山。

熙熙攘攘的人群,宛若河流。围绕着五德山。

五德山并不高,位于中洲东部,是中型门派五德门的门派驻地。

五德门在方圆三千里的范围内,算得上一个大势力。尤其是它的背景深厚,当代五德门门主乃是天莲派的长老。天莲派是中洲十大古派之一,超级势力,因此五德门虽然建立的时间并不长,但发展顺利,周围的老牌势力并不敢多排挤打压它。

方源伪装成一位凡人蛊师。此刻也混在人流当中,从山脚处向着五德山上缓步行去。

他一身黑袍,体型不高不矮,不胖不瘦。脸带面具,头带雨帽,帽檐很低,在阳光下投下的阴影,甚至遮住了方源的肩膀。

但这副装扮,在人流中却并不起眼。很多人的遮掩。比他还要过分。

随着人流缓缓前行,方源见到五德门的山门。

这座高大的门牌楼,有十六根巨柱,左右宽达三十多丈。比五层楼还要高。门匾上五德门三个大字,金光闪闪。门楼下,有六座石头狮子。威武不凡,很有气派。

过了山门。便是一条大道,用上佳的青玉石铺成的阶梯。宛若一条青河,悠然向上,攀上五德山。

大道旁,绿树成荫,葱茏青翠。时而有山风吹来,带来一丝凉意。

人群摩肩擦踵,人挨着人,人挤着人,形形色色的人物都有。

方源目光四扫,在他左前方,有一群长发飘飘的女蛊师,穿着统一的花裙,应该是来自统一门派。右手边,是一位白衣公子哥,手持着折扇不断扇风,胯下骑着一头花豹,眼睛则不断地瞄向那群女蛊师。

左手边,则是一对师徒,穿着寒酸,正在交谈。

“师傅,人好多啊!”徒弟感慨道。

师傅呵呵一笑:“人太多,千万别走散了。好徒弟,你的炼道天赋很高,这是你飞黄腾达的机会。四大入门试题你通过不难,但一定要取得好名次。只有这样,才能令他人刮目相看,争相招揽你去。”

徒弟傲然一笑:“师傅,您老就放心吧。这一次我一定能夺得头名。取得头名的奖励蛊虫,为您治病!”

师傅正欲开口,忽然身后传来一股巨力。

师傅被推搡在地,徒弟惊呼一声连忙赶去搀扶。

“让开!让开!”一群五大三粗的壮汉,气势嚣张地走过来。

在他们的身后,一位老蛊师目光阴鸠,惬意地坐在躺椅上,前前后后由四个人抬着走。

“飞霜阁大供奉安寒大人驾到,你们还不赶紧让路!”开道的壮汉高喝出声。

“快走,飞霜阁的人来了。不是我们能惹得起的。”

“飞霜阁这次出动的大供奉,来势汹汹,是想在五德门身上找回场子。”

“不错,上一次五德门和飞霜阁争夺泉眼,结果五德门险胜,飞霜阁损失惨重。”

众人议论纷纷,不愿惹到飞霜阁,皆让开道路。

“可恶,欺人太甚!”徒弟将师傅搀扶起来,咬牙切齿,就要上前理论。

“不要去。”师傅到底是老江湖,连忙拦下小徒弟。

“飞霜阁,是个什么势力?”方源前方,一位男蛊师询问身边的同伴。

得到答案后,蛊师不屑地冷笑:“哼,整个阁中只有三个五转蛊师,也不过如此。”

身边的同伴连忙相劝:“这里是中洲,不是东海,强龙不压地头蛇,多一事不如少一事。算了,算了。”

男蛊师想了想,最终冷哼一声,不待飞霜阁一行人来到,自己先挤到右前方的人群中去了。

“你怎么回事,不长眼睛啊,让开,快让开!”壮汉来到方源的身后,恶声恶气地吼着,伸出手来企图驱赶。

方源没有转身,充耳不闻。

“嗯?”壮汉脸上涌现怒色,但旋即脸色骤变,“五转修为?”

方源伪装的身份,就是五转蛊师,此时故意泄露出一丝气息来。

壮汉们惊疑不定,纷纷缩手。

竹椅上,飞霜阁的大供奉安寒,坐直身体,凝视方源的背影。

“五转气息……货真价实!这样的装扮,不是散修独行侠,就是魔道蛊师……”安寒双眼眯起,手一摆,下令道,“还不饶到前面去?”

壮汉们连忙领命,转过方向,绕过方源,朝前面去了。

在凡间,五转绝对是巅峰的存在,是势力的首脑。就算是仙鹤门的掌门,也只是五转修为。

“这群欺软怕硬的家伙!”小徒弟和师傅重新汇入人流,小徒弟盯着飞霜阁一群人的背影,愤愤不平。

随即,他又将好奇、探究,又略带崇拜的目光,投向方源。

右方的白衣公子,收起了折扇,下了花豹,前方的那群女蛊师故意停下,还有其他不少蛊师,主动来到方源身边,向他示好。

五转蛊师……这样的人物,要是能攀上交情……

道路旁边,负责维护秩序的五德门弟子也在连忙传信:“快去汇报上去,发现一位五转蛊师。”

面对这些人,方源只说了一个字“滚。”

语气平静,声音冰寒,充斥狠戾之气。

众人纷纷色变,一股凉意从心底升起,哪里还敢纠缠?

小徒弟也慌忙收回目光。

方源再度将气息收敛起来,周围空开一片,三步之内没有一人。(,!

\end{this_body}

