\newsection{旷世赌斗}    %第三百一十三节:旷世赌斗

\begin{this_body}

%1
“哈哈哈,终于大功告成了!”

%2
从地下石窟出来,方源的脸上充满了喜悦之色。

%3
这段时间内,他全力以赴,一门心思的推算仙道杀招,终于在今日得成正果。

%4
这道全新的杀招,是以吃力仙蛊、拔山仙蛊、挽澜仙蛊为核心,辅助三十三万多只凡蛊,形成一套繁芜至极,令人眼花缭乱的搭配。

%5
要催动这个杀招,方源必须要一心一意,并且至少得持续一个时辰,才能将整套蛊虫都催动起来。

%6
然后三天三夜不休,才能将自身一个大活人逐渐转变成仙僵,或是从仙僵转变活人。

%7
虽然一点都不如涅槃火方便,但方源做到这样一步,已经达到了极限。

%8
方源对这个结果已经很满意了。

%9
毕竟,方源身上的力道仙蛊就这么几只。能够将它们用上,多亏了智慧光晕。

%10
当然,还有涅槃火为参照版本。

%11
若是没有涅槃火,没有焚天魔女借来的这一套蛊虫供方源不断试手,方源绝不会这么快,就能得到力道版本的成果。

%12
还有方源的智道宗师境界。

%13
正是各方各面的原因,才造成了这一场不大不小的奇迹。

%14
“接下来,就是利用那些力道仙僵,还有奋力蛊,成就上等生死福地,重获新生了。”方源心中感慨万千。

%15
辛苦追寻了那么久,终于到了这一刻了。

%16
至于落魄谷,太白云生已经在不久前,修复了它。

%17
方源也试过一次,去进入落魄谷修行。

%18
果然不愧是和荡魂山齐名的魂修圣地!

%19
搭配荡魂山胆识蛊,方源的魂魄底蕴可谓一日千里。

%20
不过现在,方源知道最紧要的,还是重获新生。魂道修行可以先放置一边,不去管它。

%21
方源已经有些迫不及待。

%22
然而,就在他正准备开动的时候。忽然接到焚天魔女的来信。

%23
心中的内容,让他近在眼前的重生大计戛然而止。

%24
“八转大力真武仙僵?”当方源看到这一行字眼的时候,心脏随之砰然而动。

%25
焚天魔女给他的情报,正是关于南疆的那座无名山峰。

%26
这件事情实在太大了。事关仙蛊屋,造成的影响已经迅速波及整个南疆的蛊仙界。

%27
南疆当中,自然也有僵盟分部。

%28
因此,焚天魔女作为北原僵盟分部首领,很快就得到了相关的情报。

%29
“焚天魔女野心勃勃。她知道我有定仙游,就关注着五域的风吹草动。而她身为僵盟的高层,情报收集的能力,比黎山仙子还要强。”

%30
方源再次感到自己在情报方面的弱势。

%31
前世的时候,他有组织,有人脉,有成熟的血道蛊虫搭配,可以替代大部分常用的信道手段。

%32
今生重生,方源则主要借助自己前世的记忆,还有和他人联合。借助黎山仙子、琅琊地灵等等的情报共享。

%33
但随着越来越多的意外发生,方源已经逐渐察觉到,他的前世记忆并不那么可靠。

%34
历史的真相,往往隐藏在迷雾深处,而方源前世看到的,却大多是浮于表面的假象。

%35
就算他亲身经历的事情,也未必像他料想中的那样简单。

%36
很多事情的形成原因,是很复杂的。

%37
“前世我命途坎坷,一步步慢慢向上爬,人脉、资源都是点滴积累上来。所以并不存在短板。但重生之后,却因为屡屡得到机缘,实力飞速增长,乃至于跳跃式的暴涨。所以很多方面。就跟不上来了。”

%38
“看来我今后还要增加一些信道手段,增加收集情报的能力。短时间内可以依靠焚天魔女这些外人,但长久之计,还是要靠自己!”

%39
方源沉思着。

%40
失去情报,往往就意味着就丧失了先机。

%41
就像这一次,方源还不知情。都被蒙在鼓里,险些和这场机缘失之交臂。

%42
三天之后,方源孤身一人,回到南疆,来到无名山峰。

%43
焚天魔女还在阴流巨城,她在信中告诉方源,她忙于炼蛊,布置蛊仙大阵,脱不开身。但必要的时候,她定会出手。

%44
而黑楼兰、黎山仙子则在北原,利用落魄谷修魂,增长魂魄的底蕴。她们也对仙蛊屋抱有强烈的兴趣,但是北原蛊仙的气息,让她们难以行走在南疆。

%45
方源此次过来,主要是进一步打探情报。

%46
他虽然是北原成仙,但有见面似相识杀招遮掩,寻常蛊仙辨认不出。

%47
无名山峰上,每隔一段时间,幻影就会重现。

%48
而白凝冰已经不知所踪。

%49
方源观看了整整三遍,沉吟片刻,便试着动身,接近无名山峰。

%50
果然,就像焚天魔女信中所说的那样,这里成了禁仙绝境。方源越是深入其中,越发四肢无力,整个仙窍似被一股无形的重压排挤。

%51
方源不得不停下脚步,他知道再这样下去,走不了五六十步,他的仙窍就会因此彻底损毁了。

%52
这当然不行。

%53
他还指望着力道仙窍,能够形成生死仙窍呢。

%54
方源只好后退。

%55
这时,在他身后,传来一阵娇笑之声。

%56
方源瞳孔微缩,立即转身望去。

%57
只见一位美娇娘,一声粉红衣衫,青丝绾成发髻,肌肤娇嫩似雪,双眼媚如春水,俏生生地站在不远处,盯着方源。

%58
方源不敢大意。

%59
这位女子浑身洋溢着蛊仙气息,赫然是一位六转蛊仙。

%60
见方源看来,这位美娇娘笑着自我介绍道:“小女子李梅花,人称梅花婆婆。这位小郎君,好面生,不知道在哪里修行呢?”

%61
方源笑了笑,心道:“原来她就是梅花婆婆。”

%62
方源虽然没见过梅花婆婆的真容,但见过她的孙女儿,便是那魔道女蛊师狐魅儿。

%63
爱美是女人的天性。

%64
蛊仙的年龄和相貌之间,并没有固定的联系。

%65
此时的方源,一身青袍,袖子又宽又大,风轻轻一吹,青袍鼓荡起来,宛若战旗猎猎作响。

%66
方源伪装的形象,六转蛊仙气息,身材高瘦,中年模样,鼻梁高挺,一双眼睛十分细长,瞳孔转动间散发着丝丝碧芒。浑身阴气散发,一看就不好惹。虽不俊俏,却也别有一股风采。

%67
“原来是梅花婆婆,久仰久仰。在下盛鹰,不过是一位山野村夫。”方源回答道。

%68
“盛鹰……”李梅花将这名字记在心上,搜刮脑海也没记得起南疆还有这一号人物。

%69
不过,她也不奇怪。

%70
听方源介绍,就知道他是一位散修。

%71
南疆多山,龙蛇潜藏。有很多的散仙,都不露面,不为人所知。

%72
但是因为仙蛊屋的出现,吸引了一波波的蛊仙,陆续登场。

%73
方源假扮的盛鹰,不过是其中之一。

%74
这些天来,李梅花联络的蛊仙,大多都类似于方源。

%75
李梅花主动找自己搭讪,并且态度热情,这让方源暗生疑惑。

%76
方源正要直接开口,李梅花却主动道出缘由。

%77
方源这才恍然大悟。

%78
李梅花接着邀请方源同行,方源沉吟片刻,便点头答应下来。

%79
他跟着李梅花,一道出发,在千里之外的无头山上落脚。方源到的时候,此山上已经有不少魔道或者散仙。

%80
见到方源,大多数人投来好奇的目光,有的则显得阴沉或凶狠。

%81
因为摸不清方源的根底,暂时还没有蛊仙和方源搭话。

%82
倒是梅花婆婆人缘相当不错,刚一落下,就有人笑着道:“李梅花,又带来一个啊?”

%83
“哈哈哈,这一次我们魔散联合,和正道谈判,梅花婆婆立的功劳很大。”

%84
“哪里,哪里。各位盛赞了,小女子也是稍尽绵薄之力罢了。”李梅花笑着四处打招呼,纷纷搭话,游刃有余地处理着各方面的关系。

%85
“给大家介绍一下,这位兄台姓盛名鹰,是一位散仙。”应付好了,李梅花不忘向大家介绍方源。

%86
“原来是盛兄。”当即,就有魔道蛊仙拱手。

%87
“在下重禾子。”

%88
“我观阁下似乎修行的变化道,哦,鄙人姓蓝,名天鸿,也是一位散修。”

%89
……

%90
方源表现出一副不善交际,勉强应付的样子。

%91
短暂的热络之后,无头山上又恢复了方源来之前的平静。

%92
李梅花并未在山上久留,她还要去无名山峰处留守,临行前她手指着对面一座山,对方源道:“那座松尾山,就是正道蛊仙的营地。”

%93
其实不需要她指点,对面山峰上外溢的蛊仙气息,已经告诉了方源。

%94
和李梅花分别之后,方源就在无头山暂居。

%95
他鲜少出走,大多数时间都缩在山洞之中。毕竟方源是伪装的身份,见面似相识也不是无敌的。

%96
他耐心地守候了十多天,期间无头山、松尾山上陆续迎来正魔两道的蛊仙。

%97
大多数都是六转,每一位七转蛊仙的到来,都会引发一波轰动,最终,就连八转蛊仙也来了四位!

%98
一位魔道,一位散修,两位正道。

%99
刚好形成均势。

%100
八转蛊仙并不露面,双方蛊仙每天各推举一人,进行艰难谈判。

%101
谈判进行了七天七夜,终于大功告成。

%102
正魔两道,数十位蛊仙统一定下契约,围绕无名山峰,进行一场千年难得一见的大赌斗!

%103
赌斗中的胜者,将得到无名山峰下的机缘。其余蛊仙,在赌斗后三年之内,都不得向胜者出手。

\end{this_body}

