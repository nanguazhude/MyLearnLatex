\newsection{炼道杀招转金钟}    %第一百八十五节:炼道杀招转金钟

\begin{this_body}

“徒儿!!”岐山老人焦急无比。

郑山川转身面向岐山老人,清秀的面孔上,一对眼眸清澈见底:“师傅,徒儿不孝,这次要违背您的意志了。师傅,对方让我出题,我现在的优势这么大,没有理由不赌!若这种条件下,我还不赌,这事情必然会成为我的一块心病,将来我还有什么勇气再在炼道一途中继续前行呢?”

岐山老人听了这话,陷入沉默。

他在心中长叹:“徒儿啊,你涉世不深,不懂得人心险恶。对方既然能让步如此地步,必定是自信至极,有其他底牌手段的。你中计了呀!可惜,可恨,师傅我年老力单,局面已经成型,心有余力不足,阻止不得啊。”

表面上,岐山老人则伸出一双老手,把住郑山川的肩膀:“徒弟,师傅知道你的孝心。也罢,既然你有自己的主见,执意如此,师傅必定全力支持你。就用小心蛊出题,咱们不赌炼蛊成败,赌时间!谁要是越早炼出小心蛊,谁就是胜者!”

“师傅!”郑山川感动得双眼通红。

小心蛊是岐山老人这个传承的独门秘传,外界根本没有。岐山老人为了支持自己的徒弟,将这个珍贵的蛊方贡献出来,不惜让竞争对手方源观看。

他还提出,以炼蛊时间论输赢。方源第一次炼制小心蛊,肯定不熟悉,以时间论输赢。比论炼蛊成败更加具有压迫力。可见姜还是老的辣。

方源让对方完全出题,让步太大了,把他自己陷入到十分被动的局面里。

郑山川犹豫了一下,他觉得师傅这个时间标准,有点过于欺人。就算自己胜利了,也有些胜之不武。

不过他旋即又想到,这个事情事关师傅的安危。就算胜之不武,也要赢得绿曜蛊!

于是他转过身,对一旁的飞霜阁长老道出自己的考题。

飞霜阁连忙安排下去,大供奉安寒大喜,主动让出场地。

方源和郑山川一起下场。进行赌斗。

全场静悄悄一片。

双方面对面。相距数十步距离,盘坐在地上。

郑山川先将小心蛊的蛊方,交给方源观看。

很快,飞霜阁方面便宣布这场赌斗开始。

场下响起一片微微的议论声。

“小心蛊我听都没听说过,显然是独门蛊方。那个方源才刚刚看了几眼,赌斗就开始了。这也太赖皮了吧?”

“呵呵。这也怪方源太托大了。居然让对方完全出题,简直是脑袋缺根筋!”

“少年蛊师很不简单。是上一场的第二名,其实却有第一名的成绩。有很多人猜测,他已经有了炼道大师的境界。”

对这个说法,立即有人嗤之以鼻:“这怎么可能?他才多大年纪,就已经是炼道大师了?要是这样,我这么大年纪,岂不是活到狗身上去了?”

但随着比试进行下去,在众目睽睽之下,郑山川连换了近十种炼蛊手法,手法熟稔至极。给人眼花缭乱之感。

观看人群中,猜测郑山川炼蛊大师境界的呼声,越发高涨起来。

而方源还在研究蛊方,还未开始真正的炼蛊。

此消彼长之间,众人更加不看好方源。

“不妙。”岐山老人却皱起眉头,“此人心性竟然如此沉稳有加,是要看透每一步骤。才真正施行炼蛊。这是临危不乱,大家风范!幸好我是以时间为胜负标准,谁先炼出来谁就胜。小心蛊炼制过程繁琐,除非对方早就熟悉这个蛊方,练习的次数比小川还多。”

这当然不可能。

方源也是第一次看到这个小心蛊的蛊方。

小心蛊严格而言,是智道蛊虫,但用于炼道作用很大。方源猜测,这很可能就是郑山川这个传承的核心蛊。

历史上记载,山川堂的蛊师最擅长的,就是炼制那些有关键步骤,并且这些步骤都是成败攸关,悬于一线的蛊虫。

小心蛊最擅长的,就是辅助炼道蛊师,处理这些麻烦的。

方源将这蛊方反复看了好几遍,这才做到了然于心,融汇贯通。他是炼道准宗师的境界,要读透一份三转蛊方,并不困难。

事实上,他不仅观看蛊方,还很讨巧地观察对面郑山川的手法。

郑山川已经炼制到整个炼蛊过程的中段,前面的手法让方源都看在眼里。

就算是方源也不得不点头,按照郑山川如此年纪,做到这样的地步,实在是非常难得了。不过在方源这位炼道准宗师看来,他的这些手法还链接得不够娴熟,处理材料的时候也有瑕疵。

方源缓缓伸出双手,开始炼蛊。

“他终于开始炼蛊了。”场中响起一阵低微的嘲笑声。

“倒是不急不缓。”有人中肯地评价道。

方源正式开始炼蛊,也吸引了郑山川的目光。不过他只是瞟了一眼,就很快将注意力再次集中在自己的手上。

方源开始炼蛊,起先按部就班,惹得场外众人鄙夷,这明明是照搬郑山川的手法。

但不久后,方源做出改变,动用了不一样的手法,并且炼蛊速度越来越快。

场中的嘲弄声渐渐消散,众人目睹着方源闪电般的手法,无不震动。

“他已经连续动用了十多种炼蛊手法了!”

“千锤百炼的炼蛊功底,简直是深不可测。”

“他速度越来越快了,已经开始逼近郑山川……”

岐山老人眉头紧皱,暗自心惊:“这人速度快,并不可怕,可怕的是他仍旧快而不乱,手法从容稳定的叫人害怕。厉害,好厉害,这人的炼道境界恐怕已非大师级了!小川有压力了,这个时候千万不能乱,不能急。小心蛊炼制是越来越难的,小川你有优势,就要保持。对手强大,自乱阵脚,你将必输无疑!”

郑山川的确压力很大。

这很正常。跑步比赛中,领跑的那位往往是心理压力最大的。因此很多战术,都是在第二位、第三位跟跑,直到最后反超。

心理压力大,就会加重肉身的疲劳。尤其是郑山川目睹方源炼蛊的情景,看着他一步步逼近,心中不免就有慌乱。

“好强!他连续动用十多种炼蛊手法,并且还在动用。他的心神太强悍了,我在这方面比不上他。糟糕,他要追上来了……”郑山川到底是少年,方源的追赶,让他口干舌燥,心跳加速。

他目光陡然变得坚毅:“也罢,就让我动用杀手锏,将你远远抛下!”

郑山川下定决心,手中炼蛊速度骤然一缓,心神投入自家空窍,开始调动蛊虫。

“难道小川是要……”师徒连心,岐山老人看到这里,便立即明白郑山川的打算,顿时紧张起来。

“郑山川的炼蛊速度放缓了。”很快,场中也有人开始注意到这个变化。

“难道是接下来的步骤,十分困难,需要蓄势吗?”很多人都在猜测,他们不清楚小心蛊的蛊方,猜的当然不准。

就在这时,郑山川深吸一口气,眼中绽放金光,浑身笼罩金芒。

他似缓实快地伸出右手,手掌向下,五指张开,沉稳罩下。

霎时间,他眼中的,身上的金芒都流动起来,集中在他的右手上。

金光流转,幻化为一座小型金钟。

金钟缓缓旋转,将火团罩进去,旋转速度开始越来越快,并且爆发出一股吸力。

散落在地砖上的炼蛊材料,在这个吸力之下,被统统吸进金钟里去。

金钟只持续了六个呼吸,便徐徐消散。

但接下来,展现众人目前的,却是一只小心蛊的雏形了。

场外陡然沸腾起来。

“这是炼道杀招!”

“炼道杀招可比其他流派的杀招,要稀少很多。郑山川年纪轻轻,却是已经掌握了炼道杀招。”

“更加难能可贵的是,他居然催动成功了。此子天赋卓绝,前途不可限量啊。”

“方源危险了。他的确有实力,但过于托大。”

“哼,郑山川这边优势如此巨大,还要动用杀招,简直是为了胜利,无所不用其极。我倒是觉得,这场比试方源即便是失败了,也是虽败犹荣。”

郑山川吐出一口浊气,神色振奋至极:“我赢定了!我动用杀招,在瞬间碾磨了六份炼蛊材料,得以完成整个炼蛊过程中耗时最长的一个步骤。一下子就将方源远远地甩在身后!”

岐山老人心头一块巨石落地:“就算这方源也有相似的炼道杀招,但是第一次炼制小心蛊,经验为零。正常步骤中,要炼化这六种材料,都要一步接一步,小心控制火候的。冒然动用杀招加快步骤,根本就是自找死路。小川虽然弄险,但选择也不算错,这次是赢定了。”

倒是其他蛊师,对小心蛊方不熟悉,因此也不太了解状况。不过方源处境不妙,是任何人都能看得出来的。

这一步甩开来,简直是确定了胜败,郑山川稳定情绪,继续炼蛊。他的手法快速又稳定,又以稳定为主,不给方源任何机会。

方源不为所动,开始徐徐炼化六种材料,小火慢慢煎熬,将材料一一炼化。

这一步,难度其实不高,但的确是用时最长。

方源竟然一点都不着急。

\end{this_body}

