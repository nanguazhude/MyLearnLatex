\newsection{短暂的休整}    %第二百三十九节:短暂的休整

\begin{this_body}

半个月后。

方源的仙窍死地之中,两股意志纠缠在一起,死缠烂打。

一股意志,如星光点点,明灭闪烁。另一股意志,则虚幻不实,忽进忽退。

正是星意和假意。

这两股意志交战,已经持续了半炷香的时间。星意渐渐不敌假意,已呈败相。

但方源心念一动,又一股星意,顿时从天而降,作为生力军,加入战斗,立即改变战况。

最终,假意惨败,只剩下零星点点。

“方,方源,你……不得……好死……”假意凝聚不成任何形态,竭尽全力,才断断续续地发出诅咒。

这种弱者的咒骂,根本无关痛痒。

星意得胜,凝聚成方源的形象,看了一眼墨瑶假意之后,冷笑一声,冲天而起,当即离开了仙窍死地。

“意志交战,我已大有进步。如今一对一的话,虽然打不过墨瑶,每次都是战败,但墨瑶假意也不好过,最近几次都是惨胜。”

方源心中自我评估着的时候,星意进入他的脑海,将此次获取的情报呈现给方源。

方源原本已经不抱期待,但这一次的情报,却让他稍稍有些微的意外。

“春梦果△②长△②风△②文△②学,→↑.n★et树?”方源心头一动。

原来,墨瑶曾经在外行走时,意外地在一处无名村庄附近,碰到了一株春梦果树。

那个年代,梦道的概念都很薄弱,还没有梦境的外显。就算是天庭,对于梦境的探索。也浅薄无比。

墨瑶只是觉得这株树有些奇怪,她不认识。所以暗记在心。

她将这株春梦果树留在原地,没有动它。

春梦果树不是荒植,从外表看去普通无比,通常蛊仙们也不会发现。只是墨瑶动用的侦察杀招比较独特,发现了春梦果树的一些奇异地方。

不过她也只是微微好奇,毕竟不是仙级材料。后来因为其他事情,没有再去那个小村庄。

那株春梦果树,就一直放在那里。

沧海桑田,这么多年过去。也不知道那株树怎么样了。

也许已经被凡人砍伐,也许受到火灾烧毁,也许它还留在那里。

可能性太多了,说不准。

“墨瑶这个老妖婆,还真是难缠!这种记忆,她都压在最底层,直到掩藏不住了,才被我翻出来。不过按照这个架势,她知道的东西已经差不多被压榨尽了。”

方源按捺下动身去取春梦果树的心思。

事实上。之前他也得到过另外一份很有价值的线索。

是关于芙蓉石中乳的,这可是货真价实的仙材。

但后来他偷偷赶到那里时遗憾地发现,关键地点已经被灵缘斋秘密占据了。

方源思考了一番后,终究没有去动它。

所以对春梦果树。方源抱的希望并不大。而且要收取春梦果,非得用梦道的独特手法。这点方源还要多加准备。

“如果真的有春梦果,对我接下来炼制梦道凡蛊。将大有帮助。毕竟能用于梦道蛊虫炼制的蛊材,实在太稀少了。”

自从有了解梦仙道杀招之后。方源就算再忙,每天几乎都要抽出时间来。炼制梦道凡蛊。

若是有了春梦果,对他炼制梦道凡蛊,将大有帮助。保守估计,炼蛊效率至少能翻六倍!

虚弱至极的墨瑶假意,就留在仙窍死地中,让她慢慢恢复。

方源睁开双眼,离开床榻,来到荡魂行宫中的牢房。

古月方正就被关押在里面,他躺在石床上,目光无神呆愣地望着顶壁,一动不动。

仙鹤门将他抛弃,对他而言是个极其沉重的打击。

经过最初的歇斯底里的咆哮嚎哭之后,方正就成了现在这个样子。

方源开口说话。

哪怕是充满了嘲讽,或者贬低咒骂仙鹤门以及天鹤上人,方正都没有反应。

方源不以为意,说了几句之后,转身离开这里。

调教自己的这个弟弟,是需要时间的,不可能一蹴而就。方源早已预料到这一点,也从不缺乏耐心。

每隔一段时间,他都会主动来看看方正。

在荡魂行宫中,他完全可以在遥远的距离监控方正的一举一动。方源亲自去看的真正意义,在于他想让方正知道他来过这里。

方源离开这里,又转入其他牢房。

直至今日,方源的牢房中可关押了不少的囚犯。

除了墨瑶假意、古月方正之外,还有东方长凡的魂魄,羽人七转蛊仙郑灵的魂魄,最后还有在琅琊福地攻防战中,俘虏的雪松子魂魄。

东方长凡的魂魄,已经被方源彻底搜魂,价值只剩下魂魄本身。

而郑灵、雪松子的魂魄,却大有价值。

从琅琊福地回来的这些天里,方源一直都在不断地对这两人搜魂。

方源收获良多,大量的凡蛊蛊方,许多凡道杀招,一些仙道杀招,少数仙蛊蛊方,以及秘辛知识,许多宝藏、传承的线索。

但因为郑灵是风道蛊仙,雪松子是雪道蛊仙,因此并没有适合方源力道,或者星道的东西。

在方源拥有风道仙蛊、雪道仙蛊之前,这些只能为方源增添底蕴,夯实基础。

值得一提的是,雪松子的魂魄被动过手脚,最近的一段记忆,都被消磨掉了。因此方源并不知道,雪松子、黑城是如何和秦百胜等人走到一起的。

这种魂道手段,十分高深,让方源叹为观止,束手无策。

当然,郑灵、雪松子两人都未彻底搜魂完毕,接下来有什么收获,还是未知数。

至于琅琊地灵的奖励,的确十分丰厚。

方源不仅得到了之前就约定好的全力以赴蛊的六转仙蛊方,而且还在方源的坚持下,将如何搬迁福地的方法,也毫无保留地传授给了方源。

收获只有两项,看起来稀疏平常,但对方源而言价值都是极大。

全力以赴蛊仙蛊方就不多说了,单说搬迁福地。

不管是狐仙福地,还是星象福地,都急需转移。

不过此法耗资不菲,对福地本身也有巨大损伤,需用同时催动十二只仙蛊。琅琊地灵那边早已经在准备搬迁了。

搬完之后,他会将这些仙蛊借给方源,让方源搬家。

当然,这次借蛊的费用是相当高昂的。(未完待续……)

\end{this_body}

