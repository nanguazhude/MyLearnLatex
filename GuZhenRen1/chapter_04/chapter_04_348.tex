\newsection{炼化薄青的仙蛊}    %第三百四十九节:炼化薄青的仙蛊

\begin{this_body}



%1
其实,仙僵之躯,承载魂魄的极限,还要比普通的血肉之躯更高一些,只是少了滋养魂魄的功效而已。

%2
唯有鲜活的血肉之躯,才能滋养魂魄。死人尸躯之中,魂魄会越见枯萎。

%3
方源的魂魄能支撑到现在,胆识蛊起了最主要的作用。除此之外,他还动用了其他一些魂道手段,辅助滋养。

%4
虽然幽魂魔尊带给天下生灵重创,但也因他魂道鼎盛了一个时代,无数魂道杰出人物纷纷涌现,创造出了海量的魂道手段。

%5
尽管经历了乐土仙尊的时代,到此时,仍旧有大量的宝贵的魂道手段,流传下来,遗泽后人。

%6
“我并不修炼魂道,魂道蛊仙的身上,有魂道道痕,使得身体能承载更为强大的魂魄。”在意识到达到极限之后,方源便离开狐仙福地,来到星象福地。

%7
落魄谷就安放在这里。

%8
方源进入落魄谷。

%9
落魄谷原本被影宗改造,变得面目全非。但方源让太白云生出手之后,落魄谷恢复了原来的生态环境。

%10
落魄谷中地形复杂,宛若迷宫。《人祖传》中记载,当初人祖就受困在迷宫中,找寻不到出路。

%11
方源如今,效仿人祖,来到落魄谷里。

%12
他直接让魂魄离身。

%13
一阵灰白雾气,弥漫而来。

%14
这是大名鼎鼎的迷魂雾。

%15
方源的魂魄被迷魂雾笼罩,渐渐松散开来。

%16
雾气过后,又一阵阴风吹鼓过来。

%17
这是声名赫赫的落魄风。

%18
阴风如刀,方源的魂魄被风刃切割,造成无数的伤口。无以伦比的痛楚,让方源几乎要发狂。这种痛苦,根植于灵魂深处,比凌迟还要痛苦万倍!

%19
方源仰头怒吼,但他缺少了肉身,只能发出一声声尖锐刺耳的魂啸。

%20
支撑了片刻之后。方源艰难地意识到,自己再次达到了极限。

%21
他明智地选择撤退,魂魄归于仙僵尸躯。

%22
有了仙僵尸躯的保护,迷魂雾、落魄风的威力。顿时下降了一大截。

%23
方源得以安然退出落魄谷。

%24
检查一下自身的魂魄。

%25
经历过迷魂雾,落魄风的摧残之后,方源的魂魄瘦小了一大截,显然受创不小。

%26
再待下去,虽然还达不到魂飞魄散的严重程度。但奴隶住的羽民蛊仙周中。恐怕就要脱离掌控了。

%27
这是因为,魂魄是奴隶的基石。魂魄受创严重,奴隶的效果就会大打折扣,让被奴隶的一方,有了反抗脱离的可能。

%28
“也不知道当初,人祖究竟是如何在落魄谷中,待了那么长的时间。”

%29
方源感慨了一下,又回到荡魂山。

%30
他继续耗用胆识蛊。

%31
胆识蛊虽是凡蛊,但终究不愧是传说之物。用它治疗魂魄上的伤势,简直立竿见影。

%32
很快。方源的魂魄上的伤势,就彻底消失,全然不见,痊愈了!

%33
此时,方源再检查自身魂魄。

%34
很明显的,他就发现,自家魂魄比先前凝实了至少一倍!并且魂魄本身,就像是涂了一层油,散发着微微的幽光。

%35
再感应一下,因为奴隶羽民蛊仙周中而造成的魂魄负担。也随之小了三成。

%36
之前在落魄谷中,方源遭受的非人的折磨和痛楚,都是值得的。

%37
正是因为在落魄谷中,魂魄中的杂质被剔除。就像是一块铁被无数次锻打、冶炼,最终化为钢!魂魄的质量,立即上了一个台阶。

%38
而在荡魂山上,胆识蛊不断消耗,魂魄以一种匪夷所思的速度,不断壮大。数量上节节攀高。

%39
“落魄谷炼魂。荡魂山壮魂。质变、量变相互结合,魂魄底蕴一日千里。这种巨大的成功,来得是如此快速。难怪前世,黑楼兰、黎山仙子得到了落魄谷后,就整天往返在荡魂山和落魄谷之间,增长魂魄底蕴。她们是修炼出了快感!”

%40
方源叹息一声。

%41
他也想效仿黑楼兰、黎山仙子这样,但他要做的事情还有很多。摆在他眼前,就有最重要的一件事情,那就是义天山大战,仙蛊屋惊鸿乱斗台。

%42
毫无疑问,方源的实力越强,在关键时刻,他夺取惊鸿乱斗台的可能就越大。

%43
方源只能尽量地去思考每一种可能,去更合理地安排好每一件要务。他压榨出每一分每一秒的时间,争分夺秒地去修行!

%44
终于,方源得到仙蛊屋惊鸿乱斗台出世的消息。

%45
“怎么回事?我在义天山附近,布置了许多手段,居然被人悄无生息地拔出了。”方源在百忙之中,抽出身,亲自去南疆查探。

%46
但他没有发现什么有价值的情报。

%47
就如同前世,义天山上反复放映着一场古时的激战。

%48
激战中,额头刻印红莲的神秘蛊仙,最终借助仙蛊屋惊鸿乱斗台,将大力真武体,八转强敌艰难镇压。

%49
“这位蛊仙额头刻有红莲印记,和红莲魔尊有什么关系吗?我重生之时,已经失败,但是在光阴长河之中,却忽然出现了一道红莲,让我由败转胜,成功重生。光阴长河中的红莲,和这个神秘蛊仙之间,又有什么联系?”

%50
方源一头雾水,只好摇摇头,离开这里。

%51
他还有很多事情要去安排。

%52
惊鸿乱斗台虽然出世,但距离义天山大战还有一段时间。

%53
在这段时间里,方源至少要将奴隶羽民蛊仙周中的后遗症消除干净。

%54
通过荡魂山、落魄谷的往返修行,方源的魂魄底蕴一路暴涨,周中的魂魄却还在原地踏步。

%55
“将奴隶蛊仙的负担消弭干净,我就能一身轻松地潜入义天山,偷偷炼化仙蛊屋了。”

%56
肉身滋养魂魄,而魂魄饱具灵性,产生念头、意志、感情,促发思考。

%57
魂魄底蕴提升,对于方源炼化仙蛊屋惊鸿乱斗台,将有巨大的益处。

%58
“除了魂魄方面,若是我能将薄青的剑道仙蛊化为己用,对我的战力也是巨大的提升。那就更妙了!”

%59
全力以赴仙蛊虽然尝试了几次,但一直都未成功。

%60
到了此刻,方源的重生计划,已经到了最后关头。

%61
他毅然放弃了全力以赴仙蛊的炼制,不在这个坑里挣扎了。

%62
最后的这段时间,方源拼尽全力,做着准备。

%63
中洲。

%64
仙僵薄青离开落天河,徐徐上升。

%65
他的脚下,落天河面波涛汹涌,惊涛拍岸,卷起千堆雪。

%66
血水已经褪尽,无数河水中的猛兽,感受到薄青的气息,惊恐地四下游窜。

%67
而在落天河底,大量的蛊仙丧失了生命,就连天庭的八转蛊仙白沧水都没有幸免。

%68
“我的仙蛊丢失了不少,战力要打折扣了。”仙僵薄青开口,声音低沉,“接下来我们该如何行事?”

%69
此时,站在他的身旁的,有繁星洞天之主,影宗蓝正使,仙僵七星子,号称“血龙”的血道魔仙宋紫星,以及毛民炼道蛊仙余木蠢。

%70
不知道为什么,这个情况和前世产生了差别。

%71
前世的时候,七星子、宋紫星都死在薄青手中。但现在,却双双生还。

%72
或许,这是方源提前的行动,利用墨瑶假意,勾出薄青仙僵真身的缘故。

%73
宋紫星、余木蠢闭口不言,由仙僵七星子开口道:“接下来,我和宋紫星、余木蠢联手,施展仙道杀招,将我们四人一齐传送到中洲东北部去,讨伐天莲派!”

%74
他早就有了计划安排。

%75
前世,七星子、宋紫星死了,只剩下余木蠢一个人,无法发动这个强大的仙道杀招。

%76
但现在,却不一样。

%77
天庭中,监天塔主惊怒交加。

%78
他看着壁画上的以仙僵薄青为首的四位仙人,杀向中洲十大古派之一的天莲派,他再无法作壁上观。

%79
因为他监天塔主,就是天莲派出来的蛊仙。

%80
在天莲派中,他有大量的徒子徒孙。

%81
此刻,仙僵薄青苏醒,天莲派危在旦夕。唯有天庭蛊仙,才能施救!

%82
监天塔主飞出监天塔,重新召回炼九生,碧晨天:“落天河底惊变,白沧水已经阵亡,仙僵薄青复苏,如今杀向战仙宗。我们速速前往救援!”

%83
二仙分外震惊,监天塔主的一句话,包含的信息量实在太过巨大。

%84
“走!”

%85
二仙的震动,只是一瞬。

%86
旋即,他们就反应过来,三仙联袂走入天庭的传送蛊阵,下一刻,就来到天莲派的大本营。

%87
这个时候,薄青等人也才刚刚到来。在他们的身边,用于传送的仙道杀招的残余光辉,还没有彻底消散。

%88
两方蛊仙对视,战意在瞬间鼎沸。

%89
“怎么回事?”天莲派的蛊师、蛊仙们还在懵懂疑惑。

%90
战!

%91
一场八转级数的超级大战,陡然爆发……

%92
中洲,狐仙福地,地下石窟。

%93
“成功了!”方源看着眼前的蛊阵,眼中闪烁着欢喜和遗憾。

%94
这些天,他借助智慧光晕不断思考推算,得到了一个炼化薄青剑道仙蛊的强大蛊阵。

%95
这套蛊阵,竟然是以智慧光晕为主,利用墨瑶假意,辅助毒道仙蛊妇人心、智道仙蛊解谜,达到炼化薄青身上仙蛊的目的。

%96
“可惜的是,三天来,利用这个蛊阵,只炼化了一只仙蛊。偏偏我还不知道,这只仙蛊有什么作用。不能再等下去了,义天山大战已经开始,越早收取仙蛊屋越好。省得夜长梦多!”

\end{this_body}

