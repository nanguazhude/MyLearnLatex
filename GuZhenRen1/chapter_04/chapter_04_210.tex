\newsection{炼仙蛊}    %第二百一十节:炼仙蛊

\begin{this_body}



%1
中洲,真阳山脉。

%2
残阳如血,映照着连绵起伏的山丘。

%3
在这其中一处的险峻山峰上,一位毛民正在奋力攀爬。

%4
“师傅,师傅!等等我呀!”毛民一边攀爬,一边高喊道。

%5
和其他的毛民不同,这位毛民目光中透着一股灵性,说话更和寻常人族没有什么区别。

%6
他在崇山峻岭追逐着余木蠢的身影,从中洲炼蛊大会时,这位毛民早就关注余木蠢。

%7
当中洲炼蛊大会结束,余木蠢获得第一,毛民便追踪后者,一路跋涉。

%8
这位毛民,正身强力壮的年纪,但却根本追不上余木蠢。

%9
余木蠢在大会结束之后,就进入了真阳山脉。毛民紧随其后,一路叫喊,一路追逐。路上,他吃了无数的苦头,满身灰土,浑身伤口。

%10
“师傅,你等等我!”毛民登上山头,却瞪圆了眼睛,“怎么会?师傅刚刚还在山峰上,怎么一会儿工夫就到了山脚?他用的什么蛊虫?”

%11
眼看着好不容易拉近的距离,反而更远了。毛民心生焦急,连忙下山,结果他脚下踩空,一失足,顿时从山涧之间跌了下去。

%12
幸好有茂盛的树枝,为他抵消下冲的力量。

%13
饶是如此,等到他一路哀嚎,跌跌撞撞地滚到半山腰时,他已经头破血流,浑身多处骨折。

%14
身体上的剧痛,远不及心中的剧痛。

%15
“错过这次机会,我要再找到师傅,可就难了……”

%16
毛民泪流满面。挣扎欲起,却爬不起来。

%17
他仰头绝望呻吟:“大师。请您可怜可怜晚辈,留步啊!呜呜呜……”

%18
他的声音微弱至极。哀嚎之后却是嘤嘤的哭泣起来。

%19
但当他眼泪的时候,忽然浑身一震。

%20
他发现,不知不觉间,自己的眼前出现了一双脚。

%21
他顺着脚视线迅速上移,惊喜地发现余木蠢不知道何时返程,此刻就站在他的面前,俯视着他。

%22
毛民连忙跪在地上:“大师,大师,你终于肯见我了!”

%23
余木蠢叹息一声。声音低沉,饱含磁性:“本多一啊。几年前我只是随手指点你一下罢了,你我是没有师徒缘分的。”

%24
毛民本多一听到这里,顿时大哭起来。他不知道哪里来的力气,忽然一跃,抱住余木蠢的大腿:“师傅啊,师傅!就算您不肯承认是我的师傅,但是我受您指点,才脱离愚昧。冲破桎梏,使得自己的炼蛊达到了前所未有的境地。您虽然只是三言两语,但这份授业之恩,恩重于山。深于海。我的命运都由此改变,所以您在我心中永远是我的恩师!”

%25
余木蠢伸出大手,抚上毛民本多一毛茸茸的脑袋上:“呵呵呵。我当初指点你,也是看你比寻常毛民多一些机灵多变。心思通透。不过我不认你收你做徒弟,也是为你好。我是逃脱宿命之人。你和我牵扯上关系,有害无利。今后你也决不可四处宣扬,我曾经指点你的事情。否则,你必定死无葬身之地。切记,切记。”

%26
“徒儿记住了!”毛民本多一立即点头如捣蒜。

%27
余木蠢鼻腔轻哼:“嗯?”

%28
本多一连忙改口:“记住了,我记住了。余木蠢大师,请您收留我。我没有资格做您的徒弟,请让我做您的随从,做您的奴仆。我很勤劳的,也很机灵,您刚刚还夸奖过我呢。我一定会忠心耿耿,为您鞍前马后尽全力效劳的。”

%29
余木蠢摇头苦笑:“你啊,你啊。”

%30
“也罢。念你此行追我,连翻了三十几座山头,吃了无数苦头。这份向道的诚心和坚持,难能可贵。本多一,你要记住,你的这份机灵劲儿是你突出的优点。人乃万物之灵,你身为毛民,但身上的灵性已经不属于人族。但你不可以将此作为依仗,须知善泳者溺。今日你要记住自己的这份诚心和坚持。你起来吧。”余木蠢道。

%31
“大师,您不答应我,我就不起来!”本多一叫道,“哎哟。”

%32
忽然间,他双臂一空,原本双臂环绕,拥抱住的余木蠢,但此刻一股无形的力量排斥着他。让他只能仰望,看着余木蠢凭虚登空,一步步向高空走去。

%33
“本多一,你我缘分已尽,今日这场见面便是最后一面。现在我就教你最后一课,算是临别的礼物,你且看好了!”

%34
空中,传下来余木蠢的声音。

%35
本多一连忙仰头,期盼地望去……

%36
中洲,地渊,星象福地之中。

%37
夜色温柔,清风徐徐,方源仰头望去,福地的苍穹繁星点点,散发着晦暗的星光。

%38
万籁俱静。

%39
方源缓缓低头,将仰望天空的目光,慢慢收回,投到眼前的巨大乌龟壳上。

%40
这乌龟壳来历不凡,乃是一头荒兽龟的龟甲。

%41
此时,龟壳倒放,宛若一只巨碗。龟壳里面盛满了毒血。

%42
这些毒血,正是方源在狐仙福地渡劫时搜集的。此时为了炼制变形仙蛊,这些毒血也派上了用场。

%43
“炼制变形仙蛊,终于可以开始了。”方源吐出一口浊气。

%44
炼制变形仙蛊的事情,方源全程保密,没有告诉其他的任何人。诸如太白云生、黑楼兰、黎山仙子等都被蒙在鼓里,毫不知情。

%45
而星象福地,也早早关闭了门户。本身福地又处于地渊深层。在这个时间,这一层的地渊还不为人所知,安全的很。

%46
安全是很重要。

%47
炼制仙蛊本来就不容易,可谓困难重重。影响蛊仙炼蛊发挥的外在因素太多太多,偏偏一旦操做失误,价值连城的仙材就要打水漂,功亏一篑。

%48
因此,炼制仙蛊就更需要安稳平静的环境!

%49
虽然方源有一道成功道痕,可以在炼制六转仙蛊的时候,将炼蛊失败率归零。

%50
但万一在炼蛊的过程中,方源自己操作失误呢?

%51
因此,星象福地这种安静可靠的炼蛊环境,十分需要。

%52
否则,方源炼蛊失败,哭都不知道找什么地方哭去!

%53
虽说要开始,可方源在动手之前,还是先调息了半天。

%54
直至他感觉神魂饱满,达到巅峰状态,已经不能再完美一点时,他这才真正出手。

%55
他双眼神芒一闪,无数鬼火便在龟壳底部升腾。

%56
一时间,烧得方圆数里地阴火森森,温度陡降。

%57
龟壳中的毒血却是一片平静……

%58
中洲,真阳山脉。

%59
本多一刚刚抬头望去,便忽然怔住。

%60
一时间,他的双眼瞪得极大。

%61
他原以为,余木蠢给他上的最后一课,顶多传授给他一些独门的炼蛊手法或者强大的炼道杀招。

%62
但没想到,余木蠢立足半空,居然信手一挥,洒下一片恢宏的光。

%63
这光辉极为纯净耀眼,晃得余木蠢眼泪横流。

%64
他正感到头晕目眩之际,就听余木蠢缓缓地道:“本多一,最后一课你好睁大眼睛,能吸收多少就看你的造化了。好了,现在就且看我炼一只仙蛊!”

%65
仙蛊!!

%66
本多一心头剧震,嘴巴张得老大,下巴都差点要脱臼了。

%67
“炼,炼仙蛊?”他结结巴巴,万分不解,惊疑不定。

%68
怎么余木蠢大师一下子玩这么大?

%69
就算本多一不擅长炼蛊,也知道炼制仙蛊极不容易。在炼蛊之前,需要调节自身,将状态推上巅峰,同时心静神清,对仙蛊方研究推演已经无数遍。

%70
但余木蠢,却没有这样。

%71
他在炼蛊大会结束之后,就进入真阳山脉,跋涉在山中,根本没有好好休息过。

%72
而且在这里炼蛊?!

%73
真阳山脉,可是很凶险的地方。

%74
环境对于炼蛊的重要性,就算是三岁小孩都知道。

%75
现在本多一极为担心:万一炼蛊时的动静引来荒兽,上古荒兽怎么办?

%76
这其实还不是最麻烦的。

%77
比荒兽、上古荒兽更具有威胁的是蛊仙。

%78
因为宋紫星的关系,现在真阳山脉中有不少蛊仙,在这里搜寻,企图找到宋紫星的下落。一方面痛打落水狗,另一方面则可以完成中洲十大派暗中发下的重磅悬赏任务。

%79
余木蠢却不管本多一心中,到底有多少的担忧和紧张。

%80
他气度悠然,连连挥手。

%81
一时间,这片山峰光芒百丈,祥云瑞气层层浮现。

%82
旋风生起,满山的青草随风摇曳身姿,山花烂漫,吐露出大自然中最为淳朴的芳香。

%83
……

%84
中洲,星象福地。

%85
半个时辰已经过去了。

%86
巨大的乌龟壳里,毒血已经不断地沸腾,发出波的响声。

%87
毒血上空,一大片毒气缭绕,形成一股浓郁的漆黑毒雾。

%88
伴随着方源不断的加热,毒雾渐渐弥漫开来,将周围草地腐蚀得面无全非,草萎花枯。

%89
方源不管这些。

%90
他精心准备,早就在周围布置了一层层的蛊阵。这些毒气不会逸散出去,只会局限在周围空间里头。

%91
“千年苦贝一只。”方源凝视着沸腾的毒血,忽然向后方伸出一只仙僵怪臂,并且将巨爪摊开。

%92
“给,主人!”立即在下一刻,就有星象地灵头上托着千年苦贝,将其交到方源的手中。

%93
为了炼制变形仙蛊,星象地灵变成了方源的免费劳工。

%94
整个过程中,他都会给方源打下手。

\end{this_body}

