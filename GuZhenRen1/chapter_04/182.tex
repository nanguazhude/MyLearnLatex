\newsection{依旧第一}    %第一百八十二节:依旧第一

\begin{this_body}

中洲,东海岸。

水浪波涛,翻腾不休。寒气四溢,结成白色雾气,笼罩方圆数百里。

数百年前,不知从哪里飘来一块玄冰,靠在中洲东海岸。这块玄冰十分庞大,宛若小岛,寒气逼人,冰上聚集着不少的大量冰道野蛊,还有三四种寒性灌木树丛。

玄冰被人发现之后,在当时很快就引起了轰动。

生活在东海岸的蛊师们,纷纷猜测,这块玄冰的来历。

主流有两种说法,第一种玄冰既然从更东边漂流而来,很可能是来源于东海的冰流海域。此海域常年低温,冰寒彻骨,一股股冰流宛若长蟒龙蛇,在海底蜿蜒。一旦靠近水面,就会结成巨大冰块。

第二种说法,是指白天碎裂,从白天中掉落下来的冰块。起先应当体积更为巨大,宛若小型陆地。但是坠落的过程中,摩擦生热,不断消融,落到海水里又灭了火热,最终成为这块玄冰小岛。而这块玄冰周围的边缘光滑,毫无棱角,宛若蜡烛受热熔化似的,是这种说法的有力佐证。

这块玄冰靠在沙滩,不在动弹。

起初吸引了许多凡人蛊师,前来搜刮。这些近水楼台先得月的蛊师,无不发了一笔横财。玄冰之上的野蛊、寒性灌木都被搜刮一空。

此后几年,玄冰陆续迎来了许多蛊师,入驻其中,停留不走。

原来。玄冰小岛上虽然资源匮乏,但玄冰千载难化,寒气四溢,本身就是冰道蛊师上佳的修行之地。

中洲东海岸四季温暖,如春如夏。冰道蛊师难以修持。因此这方玄冰小岛,很是吸引了不少的冰道蛊师、水道蛊师。

有人的地方就有江湖,人数渐多,冰岛面积有限,又陆续有外人进来。为了维护自身既得的利益,冰岛上的蛊师们,便自发地结成联盟。将冰岛圈住起来。不在向外人开放。

由此,形成了一个不大不小的势力。

又经过数百年的发展,这个原本结构松散的散修联盟,因为其中涌现过几代雄心壮志的首领,渐渐凝实起来,组建了门派。又逐渐发展,门派壮大。形成了如今的一个大势力,在方圆数千里的范围内,鲜有势力能够抗衡它的。

这个门派的名字,便叫做飞霜阁。暗示门派根基的玄冰小岛,是从外部漂流而来的历史渊源。

此届中洲炼蛊大会的第二场比试,就在飞霜阁举办。

方源手持令牌,来到飞霜阁中。

这次的试题,是要炼制二转的鬼火蛊。题中规矩也改了,不再像之前的第一题以数量取胜,而是以时间为标准。

要求蛊师同时炼制出十只鬼火蛊。用时越短的蛊师,成绩便越高。题中罗列了一份标准,用时半炷香或者少于半炷香的,可得第一。先到先得,有人得了第一,后面再有得第一的,也只能名列第二。若第二位也有人。只能排第三。前三都有人,那只能名落孙山,得不到奖励。

这一次,头名的奖励是五张三转冰道蛊虫的蛊方。对于炼道蛊师,或者冰道蛊师,亦或者一方势力都有巨大的吸引力。

方源步入场中炼蛊时,前三名的位置都还空缺着。

“第二场的难度,要比第一场高出许多。十只鬼火蛊一只只炼制,并不难,用时很容易就少于半炷香。但难就难在同时炼制十只鬼火蛊。鬼火蛊隶属炎道、魂道,炼制颇为细琐繁杂,对心神消耗很大。长久炼制,很容易就要造成魂魄虚弱。这是魂道蛊虫的炼制特征之一。”

方源思考片刻,决定采用魂道、炎道相互结合的炼制手法,焚烧幽魂,形成假鬼火,用来炼制真鬼火。

飞霜阁是名门正派,方源焚烧的幽魂,当然得自己准备。飞霜阁是不会提供这种炼蛊材料的。

“邪魔外道!”见到方源采用此法,场外的安寒冷哼一声。

台上主持场面的飞霜阁阁主,则瞳孔一缩,心想:“五德门门主招揽此人,都被拒绝。这人又在大庭广众之下,灼烧幽魂,赫然是魔道行径,为了成事不择手段。飞霜阁就算招揽了此人,恐怕也是一个大麻烦,还是算了罢。”

参加炼蛊大会,除了不能动手打杀之外,蛊师是正是魔,都无所谓,并不禁止排斥魔道蛊师参加。

只要用的是炼道手段,切磋交流。也正因为如此大气,中洲炼蛊大会才越发兴盛。

不明内情的人,都赞叹中洲十大古派的胸襟,只有方源这种人才明白,这和不败传承有关。不败传承中的隐秘需求着,参加的蛊师越多越好。

只是正魔终究有别,方源利用魂魄炼蛊,有伤天和,是正道鄙视不容的凶残邪法。

围观的蛊师们,以正道为主,见到方源这样的魔头,内心大为厌恶反感,不禁都在希望方源此次炼制失败。

许是众志成城,他们的祈祷起到了作用,关键时刻,方源眼前的魂火陡然一爆,十个鬼火蛊已经快要成形,其中一只却是化为灰烬,从火焰中洒落下来,变为地砖上的一堆残灰。

众人见到此幕,双眼纷纷放光,有心直口快的正道蛊师,甚至脱口叫好。但旋即想起方源五转修为,便又闭上嘴巴,飞扬的神色却是掩盖不住。

安寒的脸上,也涌现出微微喜色。

方源带给他的压力极大,若按照方源之前的进展,已经到了炼制成功的最后几步。方源炼制速度很快,幽魂焚烧的效果十分显著。若跨越几步,炼制成功,用时必定少于半炷香,头名就是方源的囊中之物了。

当然,方源对这些奖励和名次,根本没有什么欲望。

安寒却不然,他首先是飞霜阁的大供奉,作为东道主,若是失败,在同门面前实在大掉脸面。其次在炼蛊大会中取得一场的头名,是炼道蛊师的极高荣誉。最后头名的奖励,也让安寒分外心动。

“好了,这样一来,他势必要重新炼蛊。炼出九只鬼火蛊,就算只差一个,也不合格。只有重新开始,再炼制十只,非得同时炼成,才能过关。”安寒勉强按捺下神色变幻,心中实则大喜。

“果然多行不义必自毙,就算是用生魂炼蛊,也要失败。”

“他是五德门第一场的头名,那就罢了。在我们飞霜阁,头名一定是安寒大人的!”

“邪魔外道,炼蛊失败,真是大快人心啊。”

围观的众人暗暗交流,当然场上有隔绝声音的蛊阵,并不会让场外的声音干扰到炼蛊的蛊师。

但就在众人暗喜的时候,方源手中的鬼火陡然一分,分化成两团。

两团一大一小,其中大团中包含九只鬼火蛊雏形。小团鬼火中,空无一物。

方源面色不变,深吸一口气,终于认真了一点起来。

刚刚那只鬼火蛊爆炸,并非他的失误,而是炼蛊本身就有失败概率。虽然二转蛊的失败概率并不高,方源也已经做到了尽善尽美,但碰到了这个概率,也是无可奈何。

眼下,他一手维持着大团鬼火不熄,另一手则向小团鬼火中连续投了三个魂魄,一猪魂,一羊魂,一人魂。

三个魂魄一同焚烧,小团鬼火陡然火焰大盛,赫然还发出呜咽惨叫的声音。

“他居然用人魂!”

“大庭广众之下,他连人魂都敢公然焚烧……这这这!”

“嘿嘿嘿,不愧是魔道五转级的存在,就是这样嚣张。”

飞霜阁的人脸色都难看起来。

他们是名门正派,方源在他们这里公开烧人魂炼蛊,完全不把飞霜阁放入眼里啊。但此时是炼蛊大会,飞霜阁就算再不爽,也不能阻止方源炼蛊。除非他们想开罪十大古派,不想让门派继续流传下去。

方源神色平静,主要注意力集中在小团鬼火中。

他不断投入炼蛊材料,调控鬼火忽大忽小,忽强忽弱。到了中间的关键一步,他向火中闪电般抛入一只丹火蛊,一只魂球蛊。

两只蛊虫在火中相互交融,旋即合二为一,成为鬼火蛊的雏形。

一些识货的蛊师,见到此景,不禁惊呼起来。

“连续投入两只蛊虫,这是炼道手法复投。”

“复投虽然是基础的炼道手法之一,但看此人使用时,熟练至极,简直是如呼吸般自然随意。这不是单凭天赋才情就能做到的,非得是无数次的练习苦修,才能达到这样的境地。”

“难道此人主修的就是炼道不成?”

“难说!他报名时,门派、流派都保密不言。他是货真价实的魔修,门派应该没有,流派可能就是炼道了。”

众人纷纷猜测。

当然,他们猜的一点都不准。

方源的主修流派,目前是力道,和炼道八竿子打不着。他还有官方的门派身份,隶属于中洲十大古派的仙鹤门。

若是让众人知道,他们眼前凶残的魔头,居然是仙鹤门中之人,不知道会有什么表情。

片刻之后,方源补救成功,将两团鬼火重新合二为一。

最后几个步骤下来,有条不紊,再无任何意外出现。

最终鬼火哗的一声,骤然散去,十只鬼火蛊齐飞而出。

场外息声,无一人开口。

飞霜阁长老脸色铁青,无奈地宣布道:“炼蛊成功,用时少于一炷香,此场第一名者方源。”

\end{this_body}

