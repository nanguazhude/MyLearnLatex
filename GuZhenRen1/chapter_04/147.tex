\newsection{心存异志图重宝}    %第一百四十七节:心存异志图重宝

\begin{this_body}

皮水寒冷哼一声,心中十分恼怒。

他是成名强者,方源如此叫喧,已经是许久都没有发生的事情了。

皮水寒自然不愿弱了气势,但就在他想动手,强势反击方源时,黎山仙子飞来,泄露出七转气息,黑楼兰则在一旁,不加掩盖地催动数只仙蛊。

这明显就是三人夹攻他的局面。

皮水寒眼角颤抖了一下,狠狠咬牙,郁闷收手。

他不得不收手。

因为此时他的身边,可没有自在书生这号人物可以联手。

他独单一人,对面却人多势众。

皮水寒找不回丢了面子,一边对付荒兽,一边冷哼,对方源道:“你们不是也跑了过来?你别太嚣张,真以为我软柿子容易捏不成?我是照顾大局,目光长远,不想和你这个莽夫斤斤计较。重利当前,和你纠缠没完没了,错失良机后想要懊悔就迟了。”

荒兽虽然凶猛勇悍,但皮水寒实力无疑十分强大,一面应战,一面还能出言,和方源等人交涉。

反观另一边,郄世民、半月蛮帅,就显得相形见绌了。

“呵呵。”黎山仙子笑起来,于空中站定,故意用沙哑的声音道,“此言有理,现在的确不是乱斗的时候。”

“为什么不杀退他们,独吞了碧潭福地?”方源故意不满地吼叫道。

黑楼兰配合默契,立即反驳:“时间不够了,你以为就我们想得到这一层?我们既然能来。旁人也不是傻子,恐怕陆续也会有人过来。而且东方长凡老谋深算。怎么可能不在老巢有所防备?”

黎山仙子也道:“就这样吧,我说过你多少次了。七号,你要耐住暴躁的性子。”

黎山仙子故意以教训的语气冷喝,方源哼了一声,仿佛听命于她。

这三人合作已有数次,如今有了默契,如此表演,没有丝毫破绽。

皮水寒心中不由更加忌惮。

那边郄世民,半月蛮师相互对望,均暗叫棘手。

同时心中惊疑不定:“这究竟是哪里冒出来的势力?”

“这样的人物。还只是排在七号位,不是头领?”

“皮兄,碧潭福地这么大,不是你我任何一方独吞得了的。咱们不妨暂时携手共同进退,时间有限得很,夜长梦多呀。”黎山仙子沉声道。

皮水寒哪里想和他们对打,见有台阶下,他挽留了些面子,脸色勉强地道:“也罢。就让你们参与好了。咱们先前定下了规矩,相互之间不能随意内斗,好东西一定有很多,谁先发现就归谁。相持不下的。可以交易。若交易不成,那就按实力说话!”

方源和黑楼兰对视一眼,均看到彼此眼中显露的笑意。

这是什么规矩?

跟没有制定规矩。根本毫无区别!

魔道蛊师之间,从来就缺乏信任。这样的规矩。充满了魔道风格,也毫无约束力。

“当然。之前皮水寒和郄世民等人商定的规矩,肯定不全是这些。估计有依照修为高低,分化多少战利品的条款。”方源心中嘀咕。

皮水寒是七转蛊仙,战力最强,他当然要维护自己的利益。

不过此一时彼一时,现在方源等人更加强势,皮水寒处于下风,自然不会蠢到提出这样的条款来了。

郄世民、半月蛮师呆在一边,早已懊悔不已。

怎么这么着急就出手?也不看看风头!

奴道蛊仙郄世民见双方谈妥,不得不硬着头皮,出声道:“还请阁下归还了我的三只荒鸟。”

方源嘎嘎大笑,嚣张地道:“这些中看不中用的货色,还给你就是。”

话音刚落,他便随手就放了这三头荒兽大鸟,仿佛不值一提,不屑一顾。

郄世民心中有气,憋的难受。

这荒兽大鸟的确战力孱弱,但胜在容易豢养控制。

奴兽仙蛊仅有一只,但控制荒兽的办法,却有许多。智道、水道、炎道等等,都有互通之处。

郄世民当然有手段没有使出来,一旦催动,便可使得荒兽大鸟战力上涨五成。

但方源一方表现得实在太凶猛了些,若是和他交手,也还好些。偏偏方源直接找皮水寒的麻烦去了,看都不看他们一眼。

两位魔道蛊仙不知道方源等人的底细,但却明白皮水寒的厉害。

皮水寒都受制,落入下风,他们俩怎敢动手?

于是,郄世民被诈唬住了,未战先怯,不敢真的动手。

方源的嘲讽,他也闷声不吭,默默地承受下来,不敢多说一句话。

实则,黎山仙子乃是信道,在七转中战力平平。黑楼兰失去万我,还没有真正意义上的仙道杀招,战力回落,充其量只能算是六转中等。

方源要和皮水寒对打,时间一久,势必也会落入下风。

依照实力对比的话,其实方源一方,要更弱一筹。

但上兵伐谋,其次伐交,其次伐兵,再次攻城。方源等人攻心为上,不战而屈人之兵。

力量体系可能因为每隔世界而有所差异,或者大相径庭。但哲学思想,却是通用的。

不论在那个世界,伐谋的道理,都是真知灼见。

只有脑袋秀逗的,才整天喊打喊杀。

只要稍有政治见解的都知道,战争不过是一种手段,是政治的延伸。而政治的根本,就是经济。

经济就是利益,换言之,一切都是利益之争。

对于方源而言,冷酷、嚣张、卑微、狂妄,都不过是面具而已,只是获取利益的某种工具手段。

两方人马商议一定,便暂时联手。碧潭福地中的荒兽。本就处于下风,无人指挥。如今魔道蛊仙一方。又来了生力军,很快就败北下来。死的死,伤的伤,逃的逃。

“哈哈哈,我们杀进去。”方源故意狂笑,率先飞进碧潭福地深处。

从高空鸟瞰,福地广袤的大地上,布满了大大小小的深潭。

这些深潭、小湖青碧交接,成千上万。宛若天空繁星,美不胜收。

福地中的地貌。十分凌乱,以每一个碧潭为中心点,向四面八方辐射。有的是草原,有的碧潭周围绿树成荫。有的深潭位于山谷之中,有的却镶嵌于沼泽腐地里。

“好福地!”方源不禁出声赞叹。

紧随其后的黑楼兰,默默不言,双眼放光。

黎山仙子已经来过这里,此时笑着介绍道:“这里的每一处碧潭,都是独一无二。有着不同水质,孕养无数水族,或者豢养蛊虫群,或者营造环境。储备相应的物资。东方长凡有独到的经营理念,他在位期间,陆续引入东海方面的资源。在碧潭福地中经营出成果,再放到北原蛊仙界贸易。不论在正道。还是魔道,都很受欢迎。就算是在宝黄天中。东方部族的鱼群、水草,都是出了名的。虽然不及不上东海方面的超级势力,但已算得上别有特色。”

听了这番话,黑楼兰顿有所悟,欲欲跃试:“这么说来,这里的每一处深潭,就是资源的囤积之处了。”

“哈哈哈,竟然有这么的晶髓水草,正适合我的福地啊,省了我好大一笔费用!”不远处,已经有一位魔道蛊仙,占据了一处深潭,发出欢喜的大笑声。

“我们必须分头行动,总是在一起,会引起他人怀疑的。”方源传音道。

“好,分头行动,才能搜刮更多。但要注意联络,相互照应。不提皮水寒这群人,东方长凡或许在这里布置了陷阱。我先走一步了!”黑楼兰方向一折,脱离了方源、黎山仙子,向左下方的一处深潭飞去。

“这孩子……”黎山仙子望着黑楼兰离去的背影,叹了一口气,对方源道,“我当初定下盟约,不能对东方一族出手的。这些资源我碰都不能碰,不过打个擦边球倒是可以。我为你们侦察四方,一有珍稀资源,就通知你们俩。”

“好。”方源回应得言简意赅。

他忽然瞥见一处深潭,潭水汪汪,仿佛水面上飘着一层油。潭口面积,也超过周围的深潭。

“我也去了!”方源身形一晃,飞电一般激射下去。

黎山仙子则飞向高空,很快身影便没入云端。

远处,皮水寒见方源三人分离,眼中阴芒一闪,冷哼一声后,转向深潭。

黎山仙子飞上云头,旋即就利用洞地蛊,只是向黑楼兰传去秘密消息:“快!碧潭福地中有两大重宝,一件是茅草屋,第二件则是方寸山。前者是六转仙蛊屋,攻防一体,擅于储藏。后者是《人祖传》中明确记载的传说之地,生存着异人之一的小人族。咱们不论得到那件,都对今后的修行有着巨大的帮助。”

黑楼兰立即回应:“我也早听说过这两件重宝。但小姨妈,你怎么确定这两件都在碧潭福地当中?”

黎山仙子笑着回答:“我和东方部族有过多次合作,对族内事务知道一些。茅草屋中存储着东方部族公共财物,就算是东方长凡也不能私自将其带走。至于方寸山,别看它小,却是货真价实的山峦,需要汲取地气,稳固山体,不会轻易带出去的。再说,东方长凡又和各正道定下盟约,有方寸山镇守这里,盟约的约束力量就是最大!”

“当然,这两件重宝也可能被秘密迁移走。不过可能不大,就算可能再小,我们也要去试一试。碧潭千千万万,茅草屋说不定还落在上一次的地方,你按照我的的指示,悄悄潜行过去。”(未完待续……)

\end{this_body}

