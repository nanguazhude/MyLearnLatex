\newsection{棋子白凝冰}    %第三百一十节:棋子白凝冰

\begin{this_body}

望着石亭中的砚石老人,白凝冰心中十分警惕。无弹窗,最喜欢这种网站了,一定要好评]

自从他被提拔成仙,顺便摆脱了困扰他多时的女儿身,他就彻底地意识到砚石老人的深不可测。

白凝冰的冷漠,是他的天性。但他对砚石老人的强硬,却只是一种试探。

甚至,他一心想要对付方源,恨不得吃肉喝血的复仇渴望,也是一种对自己的伪装,以及对砚石老人的试探。

虽然时间才过去几年,但白凝冰已经不再那么在意方源了。

蛊仙的境界,让他眼界开阔,看到了更多的东西,见识了更多的精彩。

他在意的是眼前的砚石老人,他不是没有暗中调查过,但影宗仍旧阴影深邃,隐藏在一片无底的黑暗当中。

他不知道砚石老人,为何如此帮助自己的原因,同时,白凝冰也不甘心,就这样被控制住。

他也不是没有想过,抽身离开。但一来,影宗费了那么大的工夫,是绝不会就这样让他轻松离开。二来,白凝冰看重精彩,更高于性命,敌人越是强大,过程越是凶险,反抗越是艰难,反而让他觉得没有虚度光阴,是一种对生命的享受。

与天斗,与地斗,与人斗,其乐无穷也!

白凝冰的心中,从小到大,都没有一样东西,那就是临危退缩!

此时,面对白凝冰的追问,砚石老人施施然从怀中取出一只仙蛊,并说道:“你只要按我说的去做,担保方源会回来南疆。”

白凝冰眼中冷芒四溢:“你好像对方源的行踪。了如指掌。可是当初我在升仙之时,追问过你。你却回答我不知道。”

砚石老人淡笑一声。

他没有面对白凝冰的诘问,而是直接道:“在那眉峰以南。离丘以东,距离石龙洞窟不远,有一座无名山峰。山高水长,青翠葱茏,独树一帜,别无支峰。你去那山峰顶上,催动此蛊。一旦成功,速速离开。此山将成为禁仙绝境,任何蛊仙进入此境。都会遭受致命杀机。”

白凝冰没有接过仙蛊,目光仍旧紧紧盯住砚石老人,冷冰冰地道:“你还没有回答我的问题。”

“我不需要回答你的问题。你只需要知道,做成这事,方源就会回来。当初你和影宗签订的盟约,内容不就是斩杀方源,才可脱离盟约,回复自由之身吗?”砚石老人淡淡而笑。

白凝冰沉默片刻,这才伸手一摄。将仙蛊拿到手中。

随后他冷哼一声,身形冲天而起,迅速消失在天边。

他一路兼程,两天之后。寻找到砚石老人交代之地。

“这座山普普通通,毫不出奇,为何要选择此地行事?”白凝冰心中疑惑。

其实。他心中还有很多猜测,甚至还想过砚石老人暗害自己的可能。

“不过。他耗费巨大代价,提升自己成仙。为什么现在要杀害自己?这种可能性很低。”

白凝冰摇摇头,他心中还有一个长久的疑惑:“为什么砚石老人要帮助自己,将自己提升成仙?那么多人,为什么要独独选择自己呢,难道就仅仅因为自己是北冥冰魄体吗?”

以前白凝冰是这样认为,但现在,他越发觉得真相没有这么简单。

接下来,白凝冰细细探查,没有在这无名山峰上,发现任何一个蛊阵的根基。

他心中疑惑又犹豫。

若这里真的是一个陷阱,那他自己傻乎乎地过来送死,岂不是要贻笑大方吗?

白凝冰忽然念头一动:“砚石老人不会无缘无故,叫我专门到这里来,催动仙蛊。他的用意神秘莫测,但却没有规定时间。我不妨拖延下来,对他进行一场试探。”

若是事关重大,砚石老人心急无比,自然要进行催促。到那时白凝冰扣着仙蛊,就能占据主动。

白凝冰按兵不动,没用动手,影宗一方果然有人焦急。

一位黑袍蛊仙来到石亭,粗声粗气地道:“砚石,那白凝冰带着仙蛊已经去了小半个月,却毫无动静。莫非是他携挟蛊潜逃,又或者遭了什么意外?”

砚石老人悠闲自得,盯着星盘棋局。

他眼前的星盘棋局,道痕密布,但比之之前白凝冰那会儿,纷杂的线路已经消散了一小半。

砚石老人一面掐指推算棋局,一面对黑袍蛊仙道:“白凝冰仍旧在那里,只是按捺不发。他早就心存疑虑,怀疑此行是对他不利的,有此反应也属正常。”

黑袍蛊仙见砚石老人一片悠然自得之色,不禁急道:“可是中洲方面,已经不能再拖了!如今,不仅是中洲十派蛊仙,就连许多散修、魔修,都进入落天河底探索。蓝副使他们最多只能支撑三天。三天一过,天庭就会反应过来,意识到这是个局,恐怕对我们的真正大计将有干扰。”

砚石老人点点头,深以为然地道:“你说的很有道理。事实上,现实已经和原计划出入很多,进展太过缓慢。”

“不如你去信催促白凝冰,让他尽快出手。若他不就范,我们就剥除他的仙位,将他重新打落凡尘!他在我们的帮助下成仙,不过只是假仙,没有必要让他继续这样嚣张下去!我们可以先出手,让他失去蛊仙的力量,重新成为凡人,重新恢复女儿身。这种心高气傲的后辈,不加以敲打,是用不顺当的。”黑袍蛊仙建议道。

“呵呵呵,你以为我为什么用他?”砚石老人笑起来,抬起宽大的袖口轻轻一拂,星盘棋局上立即少了一道星痕。

这一幕若是让方源看见,一定大为震惊。

如此举重若轻的手法,砚石老人的智道境界绝对是大宗师!

黑袍蛊仙罩着帽兜。看不清脸上神色,但语气中已经流露出浓重的疑惑:“砚石。你挑中白凝冰,难道不是看中她十绝体的战力吗?”

“非也。非也。”砚石老人长笑一声,“这只是他表面上的特殊。他真正特殊的地方,在于他也是逃脱宿命之人!”

“什么?他竟然也是?”黑袍显得大为吃惊。

砚石老人详细解释道:“我们的大计筹谋了十万年!真正的敌人,不是中洲天庭,也是其他四域,而是这苍天呐。所以我才将这最后关键一步,交到白凝冰的手中。他是逃脱宿命之人,从某种程度上,他也就脱离了天意的掌控。只有这样的人。才能有资格成为我们的棋子。”

“原来如此。”黑袍蛊仙恍然大悟,再无任何疑虑。

砚石老人继续道:“白凝冰是逃脱宿命之人,天意都无法将他完全掌控,将是我们的一柄利剑。先让他打头阵,而我们需要的是养精蓄锐,珍惜每一分力气,好应付天意的怒火。白凝冰从出发后,我就再不打算联系他。就让他自由发展吧。不过你放心,依照他的个性。他不会再等待多久的。

“既然如此,那我便继续休眠了。”黑袍蛊仙抱拳一下,旋即告退。

白凝冰在无名山峰上又等待了两天,期间频频遥望砚石老人的石亭方向。

“过了这么多天。砚石老人那边,都没有催促。从这点就可以看出他,他对此事并不着急。也对!若是他着急。他在交代我的时候,就会强调任务完成的时限了。”

白凝冰心中不免有些气馁。

这一次的试探。无疑是失败了。

他又将目光投向脚下的山峦,心中暗忖:“这些天来。我屡屡查探,根本没有发现任何蛊阵痕迹。唉!若是此次真有不利,那我也认栽了,我什么都查不出来,只能怪我本事不济!”

白凝冰叹息一声,身形飘飞,悬浮在高空,俯瞰这座平凡的山峦。

“去。”他心念一动,消耗仙元,将神秘仙蛊催动。

仙蛊逸散光辉,却躺在他的手中,一动不动。

白凝冰微微讶异,心道:“原来这仙蛊是个大肚汉。”

他继续催动,一颗颗的青提仙元陆续消耗,仙蛊的光辉越加旺盛,小巧的身躯开始微微颤抖起来。

片刻之后,白凝冰额头见汗,心生担忧:“我这已经数百颗仙元消耗了,怎么还是没有动静?原来这个任务的真正难度就在于此么!”

正当他惊疑不定之时,忽然手中仙蛊嗡鸣一声,化作一道光芒,飚射向天。

光芒越来越大,速度却越来越慢,达到极限高度之后,又向着底下的无名山峰,俯冲过去。

轰!

最终,它撞上山峰,爆发出堪比日月的刺眼光辉。

白凝冰连忙遮住双眼,同时急退。

此时,已经不需要他耗费仙元了。

但仙蛊是否已经成功催动,他还要再看看,才能确定。

于是他飞出一里之外后,就在半空中停住身形。

然后,他就看到一副难以置信的奇景!

剧烈的光辉中,无名山峰毫无损伤,山林之上出现了一重重的幻影虚像。

这是一道长河的一段河面,河水波涛翻滚,跌宕不休,充满了难以描述的自然妙韵,以及让人流连忘返的不可言述,不可表达的无尽道理。

“这般景象……难道是光阴长河?!”

白凝冰瞳孔陡然扩张,心中十分惊讶。

然后他便又看到,光阴长河的虚影渐渐消退,越变越淡。两道人影好似从河水中跃出一样,面目服饰都渐渐变得清晰起来。

这两个人在交战!

双方你来我往,战斗之猛烈,让白凝冰都为之心神震动。

其中一人,身形巨大,肌肉贲发,宛若猛虎蛟龙,横冲直撞,气势凶狂。赫然是一位八转蛊仙!

与之对战的另外一人,乃是七转蛊仙。但他明显动用了杀招,遮掩伪装,浑身上下笼罩在一层迷雾当中,看不清脸面,也辨不出身材。唯有一处地方,清晰可见,正是他的额头,绘有一朵红莲。(未完待续……)

\end{this_body}

