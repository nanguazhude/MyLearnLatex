\newsection{赤煞神舟}    %第一百一十三节:赤煞神舟

\begin{this_body}

%1
“公开拍卖,至此结束。?下面有一个时辰的休息时间,诸君请在拍卖场中活动。一个时辰之后,便展开自由交易。”

%2
秦百胜说完这话,便退下高台。

%3
他也是累了,一直主持到现在不说,尤其是拍卖运道真传精髓时,事关自家利益,在主持的过程中颇多用心算计。

%4
虽然着了痕迹,场中很多蛊仙都看得出来,但人为财死,重利当前,秦百胜也顾不得了。

%5
“这才拍卖大会举办完成,我便立即远遁,避世不出。暗中经营个上百年,消化了收获之后,再来行走天下。”秦百胜暗暗下定决心。

%6
可以说,他是本场拍卖大会,最大的赢家之一。收获的仙材可谓堆积如山,一只只金道仙蛊,正合他用。

%7
俗语曰:人为财死鸟为食亡。财帛动人心,别看蛊仙平素高高在上,只是对凡人资材看不上眼罢了。

%8
拍卖运道真传精髓,获利极丰,隐隐烫手。就连七转强者的秦百胜,也感到惴惴不安。

%9
大厅中,很快便议论纷纷。许多蛊仙从单间中走出来,混同进大厅中,相互攀谈交流起来。

%10
秦百胜特意空出的一个时辰的时间,自有用意。

%11
并非真的用来休息,而是为了让蛊仙们有相互沟通、交流的机会。

%12
从刚刚的拍卖当中,大家或多或少都会彼此手中的资源,有些了解。此时为了换取心仪之物。自然要提前打探交流一番。

%13
方源和太白云生,坐在密室中,并未动身。

%14
见面不相识。只是凡道杀招,效果有限,二人谨慎,不敢丝毫大意。

%15
没有得到任何一件运道真传中的精髓,方源心中有一团遗憾。尤其是马赵二人,涉及王庭大案的真相,方源处之而后快。但实力不足。也只能旁观,一点都插不进手去,无可奈何。

%16
好在手中的蛊仙俘虏。还未抛出,对自有交易环节,还有相当期待。

%17
一个时辰之后,秦百胜再次登场。宣布自有交易开始。

%18
为了公平。他故技重施,用蛊虫随意选出一位幸运儿,首先登场。

%19
这第一位蛊仙,取出许多仙材断浪花来,提出需求繁华土。

%20
断浪花是大江长河中,在浪涛中瞬间生灭的奇妙花朵。因为生长、凋零的时间极短,因此难以采摘。

%21
至于繁华土、钻土、青拳土……这些土壤都是通常蛊仙渴求之物。

%22
蛊仙们经营仙窍,靠仙窍产出营生修行。土壤肥沃一直都是个关键因素。繁华土,能栽培出各种花朵。钻土更易酝酿出金属矿物,青拳土则肥力凝聚成团,不轻易散去,适宜荒漠沙海等贫瘠的仙窍地貌。

%23
拥有繁华土的蛊仙,数量并不少。一时间纷纷开口,取出或多或少的土壤进行交易。

%24
站在高台上的卖家,手中居然有大量的断浪花。足足换取了三千多斤的繁华土,这才罢手,走下台去。

%25
拍卖和自由交易的区别,一目了然。

%26
拍卖是大宗买卖,门槛高。自由交易,通常则是散卖,门槛低。

%27
不过,这也不是绝对的。

%28
几位蛊仙过后,一位不愿露面的蛊仙,亮出一只仙蛊,由秦百胜代卖。

%29
这只仙蛊,正是之前流拍的一只,这次卖家大大降低了要求。

%30
但即便如此,也至少是要求一只仙蛊的。应和者寥寥无几,要求降价。卖方不肯,最终交易失败,这只仙蛊只得再次回到卖家手中。

%31
轮到方源时,已经过去了三十之数。

%32
“三十七号密室,贩卖一位蛊仙俘虏。”秦百胜一边说着,一边展示着眠云棺椁中的昏睡蛊仙。

%33
顿时,拍卖大厅中哗然一片。

%34
奴隶贸易,众仙早已熟悉。但将蛊仙当做奴隶贩卖,这样的大手笔,还是极为罕见的!很多人尚是第一次见到。

%35
一时间,蛊仙们纷纷交谈议论,神情各异。

%36
当秦百胜再次宣布时,大厅中的嘈杂声这才稍稍收敛下来。

%37
有人便询问详情。

%38
秦百胜介绍道:“这位蛊仙俘虏,某些人可能熟知,外号红玉散人。乃是一位六转炎道蛊仙。具体修为,还是一次天劫。他的仙窍福地情形,诸位可观。”

%39
说完,高台上浮现出一片变幻的光影,映照着红玉散人仙窍福地的状况。

%40
“这样的福地,有点贫瘠啊。”

%41
“是,福地经营不佳,难怪被俘虏活捉了。”

%42
“你们别忘了,俘虏他的蛊仙,想必会提前搜刮了一番的。”

%43
“对于我们炎道蛊仙而言,这位红玉散人的修为弱一些,不是更好么?”

%44
蛊仙们怦然心动。

%45
炎道蛊仙们,尤其看重红玉散人的仙窍福地。只要吞并了他的仙窍,就能免除一次天劫,许多地灾,凭空增长修为,福地底蕴猛地暴涨一大截!

%46
就算不是炎道蛊仙,大多数的蛊仙也想得到福地。

%47
哪怕流派不同,不能吞并,杀了他后占据这片无主福地,便可获得一片上佳的经营之地。

%48
经营得好,福地就是一个聚宝盆,不断地产出资源,支撑着蛊仙们的修行进益。

%49
不出秦百胜之前估算,红玉散人引发了哄抢热潮。

%50
方源提出的要求,繁杂无序。首先是一大堆的仙材,其次是力道、毒道、宙道、智道等等杀招,之后又是各种蛊方,蛊方越老越稀少便越好。最后还求购荒兽尸体,食道传承或其线索,智道传承或其线索,宙道传承或其线索。甚至还要求战骨车轮蛊,白骨战车种种配方,寿蛊等等。

%51
最终。红玉散人归属于五行大法师手中。

%52
五行大法师同修金木水火土五道,红玉散人的仙窍福地对他也是有效的。

%53
红玉散人卖出去之后,便轮到下一位。

%54
按照自由交易的规矩。每一轮中,蛊仙只有一次贩卖的机会。

%55
这也是为了公平。

%56
方源也不着急,耐心等候。如此七八轮之后,他便已然将手中蛊仙俘虏卖去大半。将琅琊地灵给予的材料清单,完成了不说,还有超余。不仅如此,方源还得到了一头荒兽蝙蝠的尸体。

%57
在这期间。他也发现,不仅仅只是自己一方贩卖蛊仙俘虏。

%58
雪胡老祖、百足天君都公然贩卖。东海蛊仙则买了三具仙僵之躯,惹得在场的阴六公、夜叉龙帅等人非常不满。最终这三具仙僵身躯。被药皇喜出望外地收入囊中。

%59
有了这个发现,方源悄悄放宽了心。

%60
能俘虏蛊仙的,不只是琅琊地灵之能。人外有人,天外有天。其余蛊仙当中更有能人。

%61
再加上这个期间。方源不断转换密室,大放迷雾,让人瞧不出这其中的大部分蛊仙俘虏,都来源于他一人之手。

%62
渐渐的,越来越多的蛊仙,不再参与交易,选择作壁上观。

%63
他们已经基本上达成了各自目的。

%64
剩下的蛊仙们,虽然数量越加稀少。但交易之物却也越加稀罕珍贵。

%65
第九轮之后,已经开始陆续有蛊仙离场。就在方源也以为手中的蛊仙俘虏,会按部就班地完成交易的时候,秦百胜忽然神情一变,收到了某个散修蛊仙的重大委托。

%66
他声调一扬,道:“下面交易的是仙蛊屋赤煞神舟的全部蛊方,以及搭配运用之法。卖方要求木道方面的,任意有助于修行之物。”

%67
“什么?赤煞神舟!?”

%68
“这种仙蛊屋,居然也拿出来卖?”

%69
“怎么现在才拿出来卖?拍卖大会已经进入尾声了!”

%70
“哈哈,提前离开的人一定后悔死了。”

%71
两处单间中,阴六公、夜叉龙帅脸色皆是铁青难看。

%72
僵盟意图铲除贩卖赤煞神舟的蛊仙,结果这位散修乖觉无比,根本不给他们下手的机会。来到拍卖大会之中,僵盟更不好动手,只好婉言相劝,迫不得已开出高价。

%73
然后这位散修许是被拍卖氛围感染,贪心不足,屡屡开出更高价码,狮子大开口,威胁僵盟。

%74
终于在方才,两方谈崩,散修蛊仙便立即将赤煞神舟放出。

%75
“我愿出仙元石六千块,赤练金三千斤,买下全部!”

%76
“仙农土!我手中有八千斤仙农土,全部给你,换取赤煞神舟。”

%77
“一群蠢货,没听到秦兄刚刚说的话吗?卖家要求木道资源。我这里有碧落草!”

%78
“碧落草算什么,我这里可有本性根!!”

%79
一时间,群情激动,尤其是各大超级势力的代表,几位八转大能亦掺和其中。

%80
僵盟两大首脑,尽管愤怒至极,此刻也不得不参与竞争。

%81
一时间,氛围缓和的自由交易,被搞得剑拔弩张,仿佛重回了拍卖环节。赤煞神舟的巨大吸引力可见一斑。

%82
“我愿付出一只木道仙蛊,交换赤煞神舟。”很快,超级势力蒙家的代表,喊出高价。

%83
“我袁家也有木道仙蛊,还附赠三百只真武鲤!”袁让尊也吼起来,赤煞神舟诱惑之大,甚至让他将最心爱的真武鲤都舍去,充当筹码。

%84
“若是让我僵盟收购此物,我阴六公以名誉发誓,必将此方再度出售。”阴六公寒声道。

%85
“老夫药皇可为见证。”随后,药皇发言,顿时加重了份量。

%86
“哼,本天君手中有智道解谜仙蛊一只,能推算一切传承线索。还有芝灵液三大酒缸!”百足天君立即拆台道。

%87
一时间,喊叫声不绝于耳,场面一片混乱。

%88
方源的心也陡然动了,双眼暴射出精芒!

%89
刚刚自己没有听错,应当就是解谜仙蛊?!

%90
ps:今晚只一更。

\end{this_body}

