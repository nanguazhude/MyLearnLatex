\newsection{三只仙蛊,皆有大用}    %第二百零三节:三只仙蛊,皆有大用

\begin{this_body}

整个星象福地形如巨碗,最中央的位置也是地势最低的位置。◎,.

在这里,是一片广阔似海的湖泊,命名为碎星湖。而万象星君平时修行之地,也设置在碎星湖底。

深入湖底,方源便发现一座水晶宫殿。规模不大,只有一座主殿,两座副殿。

万象星君的仙蛊,就保留在主殿当中。

方源在星象地灵的带领下,顺利地进入主殿。

一进主殿,他便看到一只飞舞的仙蛊。

这只仙蛊类似蜈蚣,修长如臂,晦暗宛如钨铁。它在主殿空旷的内部空间蜿蜒漂游,身上一道道的短横,不时地亮起蓝萤之光。

“这是六转仙蛊星痕,可以让主人您的肉身暂时增添六百道星道道痕。”星象地灵适时地介绍道。

“六百道?”方源吃了一惊。

他身上的力道道痕,也不过两百多道而已。这还是最近这段时间,利用吃力仙蛊,增长了四十几道。

这只星痕仙蛊,居然一下子能增长六百多道星痕,效果可谓优异了。

“两百多道道痕,便能增幅两成的威能。六百道,就是增幅六成威能。可惜是星道道痕,我却不是星道蛊仙。”

毫无疑问,这只星痕仙蛊是上佳的辅助仙蛊!

方源纵然是力道蛊仙,身怀力道道痕,一旦催动了星痕仙蛊,身上就暂时覆盖一层星道道痕,这都能让他扮演星道蛊仙了!

除了星痕之外,大殿中还有两只仙蛊。

一只宛若棱形水晶。拳头大小,静静地躺在桌案上。一动不动。

另外一只则宛如白嫩豆芽,充满了生机勃勃的气息。

经由星象地灵介绍。方源得知,前一只无色水晶模样的仙蛊,乃是星光仙蛊。后一只豆芽般的仙蛊,则名为星芽。

方源闻言,不禁大喜。

运气来了!

这两只仙蛊对方源的用处,比刚刚那只星痕仙蛊还要更大。

因为星光蛊是炼制智道星念蛊的必须蛊材啊。

现在有了星光仙蛊,完全可以效仿力气仙蛊炼制气囊蛊那样,更大规模,更快速地炼制星念蛊了。

有了星光仙蛊。就省去了收购星光蛊的麻烦步骤。

同时以仙蛊为基点,炼制凡蛊星念蛊,就是从上而下,失败概率比之前的炼蛊方法要小很多。危险性也会大减,如此一来,毛民的损失也将大大降低。

当然,随之而来产生了两个问题。

第一个问题,要利用星光仙蛊,还需方源改良星念蛊蛊方。

第二个问题。使用了星光仙蛊,就得消耗仙元了。

不过有着智慧蛊,又有完整的星念蛊蛊方,方源改良蛊方并不困难。只是时间问题。

至于第二个问题,若是之前的方源,的确是个阻碍。但如今方源成为星象福地之主。六大经济支柱,将来甚至有八项谋利的买卖。赚取的仙元石是当下的整整四倍!大量的仙元石,也使得方源有十足的底气。可以消耗仙元不断地催用星光仙蛊。

星光仙蛊给星念蛊的大规模累积,带来了巨大的帮助。而星芽仙蛊,也是很有用处。

方源曾经在万象星君手中,购买了一记凡道杀招,名为春星雨。

此招能灌溉草木,保证草木的茂盛长势。自从拥有了这招之后,方源就一直在陆陆续续的用。

狐仙福地的空中草原,能够维持当今的规模,这个凡道杀招起了巨大的作用。

频繁催动春星雨杀招,就会消耗大量的星芽凡蛊。

星芽凡蛊只有万象星君处有的卖。

方源曾经一度想摆脱万象星君的控制,自己研究出星芽凡蛊的炼制蛊方。然而即便有智慧蛊的帮助,方源的星道境界只是普通,没有达成所愿。

现在方源直接有了星芽仙蛊,今后催用春星雨,都随自己的心愿。他还可以将春星雨杀招提拔到仙级程度,以星芽仙蛊为核心,对星屑草的栽培起到更大的作用。

不过随后,方源却从星象地灵处得知:原来万象星君手中就有仙道杀招春星雨。

事实上,万象星君出售的凡道杀招春星雨,便是从他手中的原版春星雨删减而得的。万象星君向外兜售的星芽凡蛊,也是利用了星芽仙蛊炼成的。

这就难怪方源,很难从星芽凡蛊身上逆推出蛊方来了。

因为万象星君的炼蛊方法,完全是由上而下。

星痕、星光、星芽,这三只仙蛊,便是福地中遗留的全部仙蛊。

万象星君一共有四只仙蛊。可惜第四只仙蛊,在他和宋紫星的战斗中损毁了。

方源得到的,只有三只。

就是这三只,方源也已经十分满足了。

一般而言,整个五域中的六转蛊仙,只有少部分手中掌握仙蛊。大部分都还在用凡蛊,用凡道杀招。

只有六转中的精英分子,才掌握六转仙蛊。比如中洲十大古派中的六转蛊仙,又比如万象星君这种有奇遇的散修。

北原的情况比较特殊,因为八十八角真阳楼倒塌,使得北原仙蛊的普及率,要大大超出其他四域。

说起来仙蛊的普及率,中洲反而在五域中倒数第一。没办法,中洲的蛊仙数量最多。

当然,方源就更特殊了。

不算刚刚到手的这三只仙蛊,方源手头中已然有十四只仙蛊。这数量要说出去,恐怕没人会信。更别提这里,还包含九转仙蛊智慧!

一般而言,即便杀死了敌方蛊仙,也很难有仙蛊这样的战利品。

就像方源对付东方长凡,优势那么大,连东方长凡的魂魄都生擒活捉了。但东方长凡只是动了一个念头。所有的仙蛊都自爆毁灭,方源毛都没捞到一根。

万象星君是个特例。

因为他误入了星宿仙尊的梦境。没有任何反应,就稀里糊涂的死了。因此仙蛊才保留下来。最终便宜了方源。

这三只仙蛊,还都残留着万象星君的意志。不过有星象地灵的协助,要收服它们问题不大。

“主人,这是福地中收藏的仙道传承,请您过目。”地灵又献上一份传承,给方源过目。

原来,这是万象星君的第一份奇遇。

一份有关星道的蛊仙传承。

正是这份传承,指导散修万象星君一路修行,成为六转蛊仙。

万象星君的第二份奇遇。是发现了繁星洞天的入口,多次进入其中搜刮资源。

说起来他的运气挺不错。

万象星君主修星道,繁星洞天也是以星道为主的洞天,资源方面一脉相承,十分契合。

不像方源,方源是力道蛊仙,狐仙福地是奴道,荡魂山胆识蛊是魂道,这几个方面联系并不紧密。

可惜万象星君最终还是死了。一切都便宜了方源。

方源在星象福地中待了足足一天,又考察了福地中的一小群石人部落,以及那三头荒兽刺脊星龙鱼。

万象星君生前,千方百计地迁徙了这一小群石人。他没有胆识蛊。星象福地又并非石人生存的温床,养死了许多石人。好不容易才留下了这么一批。

三头荒兽刺脊星龙鱼则是从繁星洞天里捉来的,那里最不缺的就是荒兽。

之所以耗费这么大的力气和血本。就是因为星痕仙蛊的食物,便是刺脊星龙鱼的新鲜鱼肉。

而喂养星光仙蛊。则需要石人的身躯。

至于星芽仙蛊,食物是大量的暗星海带。这种海带在碎星湖底生长了很多。

这才是正统蛊仙的态势。

主修什么流派。仙蛊就是什么流派,仙窍就是以这种流派的道痕为主。如此一来,喂养仙蛊就可以自己负担,自己培育食料。

不像方源,力道仙窍死去,狐仙福地是奴道仙窍,手头上有力道、智道、魂道、运道仙蛊。喂养这些仙蛊,必须寻求外界帮助,每一次都要操劳心神,还不稳定。方源手头中的净魂仙蛊,至今都未喂饱呢。不管是狐仙福地、力道仙窍,都稀缺魂道道痕,就算方源想豢养白莲巨蚕蛊,也不可能。

星象福地认主,方源的实力膨胀整整一倍!

三只星道仙蛊,四大贸易,一份完整的星道传承,方源还需要大量的时间,才能慢慢消化。

他仍旧让星象地灵,在宝黄天持续散发认主的要求,伪造出星象福地还未认主的假象。

短时间之内,他也不打算将真相告知黑楼兰、太白云生。

能隐瞒多久,就隐瞒多久。

狐仙福地已经暴露,星象福地算是半暴露状态,这个位置黑楼兰知道。将来要是亡命天涯,这片基业就十分重要了。

至于狐仙福地中的各项资源,以及荡魂山,方源暂时也不打算搬迁。

刚刚布置好的,大量的资金已经投入下去,一旦搬迁到星象福地这边,黑楼兰何等精明,还不一清二楚?

接下来的日子里,方源继续参加中洲炼蛊大会。

受到大会影响,中洲一片全民炼蛊的热潮。

中洲十大古派联手,埋伏宋紫星失败的消息,并未传播出去,仅限蛊仙高层知晓。

十大古派脸面无光,宋紫星杳无音讯,叫许多他派的,或者散修蛊仙都看了笑话。不少蛊仙,纷纷行动,企图在真阳山脉中寻找到宋紫星留下的蛛丝马迹。

但怎么可能找得到?

方源前世,中洲十大派为了维护门派名誉,找了数百年都没找到。宋紫星躲藏功夫堪称绝顶,就算是方源也比不上他。

所以方源从一开始,就没有追杀宋紫星的念头。

终于,到了约定的日期,就是方源和凤金煌的赌斗之日。

虽然不是大比,却引来相当多的蛊师观看。

“凤金煌,你要赌斗,先让我看看赌资,赌资太差的话,这场赌斗就没必要开始了。”方源当众道。

能不赌斗就不赌斗,方源要的只是不败传承的前六名。

再说赌斗若输了,也不会淘汰双方,影响名次。

凤金煌早有准备,此刻传音道:“我的赌资必定让你心动不已。方源你已成仙僵,定然知道仙僵的苦。我手中却有一法,可以让你重获新生!”(未完待续……)

\end{this_body}

