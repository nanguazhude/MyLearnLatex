\newsection{境界暴涨}    %第二百六十六节:境界暴涨

\begin{this_body}

“我来帮你们!”方源解开手中的风结草后,主动走到最近的一位孩童面前。mianhuatang.cc [棉花糖小说网]

“嘿嘿,小崽子,你还想救其他人?可以!但是你……”兽人首领冷笑。

但他话还未说完,就被方源打断。

“知道了。救不了他们,我也不会走的,你们就吃了我吧!”方源挥挥手,很随意地道。

“呃……”兽人首领一噎,说不出话来。

周围的孩子们,则用看待英雄的目光,希冀且又崇拜地仰视方源。

“小崽子,你能救出所有人,我这个首领就不当了!”兽人首领阴笑连连。

但不一会儿,他笑不下去了。

方源根本不按套路出牌,解梦不断催动,速度惊人!

一个个被俘虏的孩子,都相继获救。

若让他仅凭自己拆解风结草,那连自救都达不到。

但运用解梦之后,难度便暴降到极点。

打个比方,就好像是玩麻将,别人都是一张张抓牌碰牌,方源是直接把牌变成自己想要的。

但这种*裸的作弊行径,落在周围人的眼中,却是方源手指如飞,不管是怎么样的风结草,在他手中转几圈之后,就以肉眼可见的速度呼呼削减,最终直达内心,种子被顺利地取出来。

易如反掌。

轻而易举。

这就是梦道杀招解梦的厉害之处!

孩童们双目闪烁着泪光。望着方源的身影,充斥着无尽的感激和崇拜。

兽人们张开大嘴,看得瞠目结舌。难以置信。

“全部得救了,我成功了。”方源语气平淡,对兽人首领说道。

半晌,孩童们才如梦方醒,爆发出冲天的欢呼声,有的喜极而泣,有的活蹦乱跳。

兽人首领无话可说。充血的双眼死死地瞪着方源。周围的兽人们或是低吼,或是张开獠牙。蠢蠢欲动。

方源却不担心兽人反悔。

一般而言,这种原始部族对于信仰,对于风俗和传统,遵守得都相当严格。

当然。凡事不能一概而论。

现实中,也存在着兽人族反悔的可能。

但是在这里,却是特殊的梦境。

并非现实!

果然,正像方源预料的那样,兽人们就算再愤怒和不甘,也只有看着到了嘴边的可口食物悠然离开。[求书网www.Qiushu.cc想看的书几乎都有啊,比一般的小说网站要稳定很多更新还快,全文字的没有广告。]

穿越幽暗的丛林,方源带领着一群颇具规模的孩童们,将兽人部落,将冲天的篝火都甩在身后。

孩子们簇拥着方源。一齐行走,渐渐悄无声息。

黑暗中,那一缕光明不断放大。最终再次充斥视野。

方源魂魄归体,睁开双眼。

定睛一瞧,方源的脸上浮现出诧异的神色。

他惊愕地看向身后的外显梦境,心中充满了疑惑:“怎么可能?我已经将所有的孩童都救了下来,已经做到极致。但是怎么还是被梦境甩了出来,没有进入第二幕呢?”

方源深深地皱起眉头。

“难道说。我之前的猜想有误?解开风结草,救下这些孩子。并非是通过梦境第一幕的正确法门?”

“还有什么我遗落的线索呢?”

方源凝神思索,绞尽脑汁。

梦境的探索,便是这般艰难。就算是方源拥有前世经验,有着领先这个时代的优势,但探索梦境时,也不好把握。

每一个梦境,都是独一无二的。

探索梦境的方法,也几乎个个不同,很难将经验总结起来。

方源现在,只能不断猜想,然后一个个的去尝试。只有坚持不懈、不厌其烦,才有将梦境打通的可能。

当然,方源前世也有大把的例子,诸如有人探索梦境,数十年而不得成效。

方源想得脑袋都有些生疼,新的猜想有一些,但连他自己都没有太多的信心。

“既然如此,还是先去他处,将藏经鼋先收服了。”

方源在这处梦境上面碰壁,也不执着。

这处星宿仙尊的梦境,虽然是他的主要目的,但是龙鱼、藏经鼋以及那株荒植,都是此行必得之物。

念头一转,方源索性暂时离开这里,前往他处。

一个时辰之后,凤金煌苍白的脸色,终于转为往昔的红润。

她睁开双眼。

清澈的眼眸中,透射出丝丝精芒。

“胆识蛊对魂魄方面的伤势治疗,帮助太大了,见效真快。”她心中一阵庆幸。

见到她安然无恙,同行的灵缘斋四位蛊师,纷纷松了一口气。

她们此行,就是陪太子读书的角色。若凤金煌有个三长两短,她们回门派去,日子必定不会好过。

凤金煌看到方源留下的荒兽鱼翅狼,目光微凝,又问了同行蛊师,得知自己疗伤期间,没有任何蛊师接近这里,心神微松。

最后,她检查一遍状况,确认已经彻底痊愈后,便站起身来,再次面对梦境。

“之前我用梦翼仙蛊逞能,收获炼道境界。但这一次面对的是仙尊梦境,难度天差地别,之前的错误,不能再犯了。咦?怎么梦境好像缩减了一些?”

凤金煌忽然神色一凝,目光中流露出一抹怀疑。

梦境一旦被成功探索,就会彻底消失。

方源虽然没有竞全功,但也的确收获了许多,智道境界暴涨。

俗语曰:有得必有失。

方源这边得了,梦境这边自然有失。

表现外在,就是梦境体积缩减。

“的确是缩减了一些,怎么会这样?”凤金煌再三确认之后。心中越发惊疑,“难道这就是仙尊梦境的特殊地方?会随着时间,而不断缩减?”

凤金煌没有猜测怀疑其他方面。

在这个时代。人们对于梦境的认知,还处于极其原始愚昧的时候。

而且凤金煌的心中,也深信着自己独有梦翼仙蛊,在梦道方面的先期优势。

尽管心中惊疑,凤金煌也无法进行验证。

最终她微微摇头,不管心中的疑惑和淡淡的不妥之感,重新进入梦境。

就在凤金煌进入梦境不久之后。方源通过鹤风扬留给他的情报,赶到藏经鼋的地点。

这头智道荒兽。体型巨大,宛若一座山峦。

此刻的藏经鼋,四肢、首尾都缩在壳内,大半的身躯都隐藏在山石中。

方源轻笑一声。身形如鹰,迅猛扑下。

轰轰轰……

剧烈的轰鸣,接连不断,宛若一阵阵的闷雷炸响。

方源和藏经鼋展开战斗。

一时间,土石翻飞,烟尘四起。

藏经鼋起初还尝试反击,但它身无仙蛊,本身又只是荒兽,不是上古荒兽。岂是七转战力的方源的对手?

几轮回合之后,藏经鼋就只能龟缩在壳内,被动挨打。

“这乌龟壳倒是厚实……”方源攻打一阵之后。见一时间竟然奈何不了藏经鼋,不禁发出无奈的笑声。

他当然没有尽全力,甚至连仙蛊都很少动用。

万我大手印,他在不少地方用过,能不暴露就不暴露。谁知道其他门派,会不会准备侦察仙蛊。带在身上?

方源需要尽可能的遮掩实力。

所以,藏经鼋缩头之后。方源一时间也不焦急。

他分心二用,一边故意将战斗打得轰轰烈烈,一边则品味梦境收获。

他这一细心品味,顿时有无数灵感,在心头涌现。

就像是眼前忽然被打开了一扇窗户,看到了屋外的辽阔风光。

这种感觉妙不可言,令方源沉醉。

“原先我的智道境界,普通的不能再普通,毕竟我在不久前才尝试智道修行。但现在,我的智道境界已经达到了准大师的地步。星宿仙尊的梦境,真的是了不得!”

这种进步程度,若是按照正常的修行,非得有数十年的点滴积累。

梦境虽然凶险,但一旦探索成功,哪怕是部分成功,收获也十分巨大,堪称一蹴而就,省去了方源大量的功夫。

“怎么打雷了?”

方源和藏经鼋的激战,陆续吸引了一批批的蛊师队伍前来侦察。

“方源?他怎么能够如此厉害?!”

“难怪门中长老千叮咛万嘱咐,要我们规避方源这个怪物!”

“这简直最作弊。对我们太不公平了,这种怪物怎么斗得过?”

最终,方源耗费一天一夜的时间,将藏经鼋打服,收入仙窍。期间不少蛊师被吸引过来,侦察一阵后,尽皆仓惶退走。

“第三十六次失败了……”

凤金煌疲惫地睁开双眼。

她自从受重创之后,便小心谨慎。可惜她没有解梦,风结草之前也从未接触过,整整三十六次,连一次成功拆解都没有。

成功自救,是探索梦境成功的最低标准。

拆解风结草失败,无法自救,那就是探索失败。

所以凤金煌每一次探索梦境,失败后魂魄的损伤,都远远超过方源。

她必须花大量的时间,进行休整和疗伤。

所以,过了两天一夜,她仅仅只是探索了三十六次。

“仙尊的梦境,竟然如此艰难。我有梦翼仙蛊,尚且如此。更别提其他人了!”凤金煌看着眼前的梦境,满脸都是苦涩。

她心中充斥着深深的挫败感,现在只要她一想到风结草的样子,这位天之骄女,就有一种想要呕吐的冲动!

此刻,方源秘密接近梦境。

他已经将藏经鼋、幽冥草都收入囊中,剩下是时间和精力,都将投入到仙尊梦境当中。

这一阵时间的苦思冥想,让方源有了新的猜想。

在凤金煌苦闷丧气的时候,方源带着自信的微笑,再次悄悄地投身梦境。(未完待续)<!--80txt.com-ouoou-->

\end{this_body}

