\newsection{山川堂的祖师爷}    %第一百八十四节:山川堂的祖师爷

\begin{this_body}

场上氛围越加热烈,少年蛊师郑山川和方源一时都都下不了台。\&\#65288;\&\#26825;\&\#33457;\&\#31958;\&\#23567;\&\#35828;\&\#32593;\&\#32;\&\#87;\&\#119;\&\#119;\&\#46;\&\#77;\&\#105;\&\#97;\&\#110;\&\#72;\&\#117;\&\#97;\&\#84;\&\#97;\&\#110;\&\#103;\&\#46;\&\#67;\&\#9

老师傅察觉到,这其中鼓动的蛊师,大多来自飞霜阁时,脸上的皱纹便又深了几分,满脸愁苦之色。

方源暗中咀嚼着“郑山川”这个名字,总觉得有点耳熟。但他身为仙僵,到底还是不如活人时思维灵敏,一时间想不起来。

方源便将脑海中的星念按下,催起忆念挖掘记忆。

这下终于想起来。

原来中洲东海岸,数百年后兴起了一个超级势力,名为山川堂。其祖师爷,便姓郑名山川。山川堂擅长炼道,梦境外显,梦道兴盛时,山川堂堂主赶上了时代潮流,成就了蛊仙。

不过在那个大时代,凡人成为蛊仙也不算奇闻。尤其是梦境出现,提供给普通蛊师海量的机缘。龙蛇隐于草莽,很多人欠缺的往往是一个机会。

恰好梦境外显,给了许多人奇遇。五域乱战,又给了大量底层的蛊师冒头的机会。

五域乱战的大世,各种人物争相涌现,如群星璀璨,有无边风流。十大古派都对中洲渐渐失去掌控力,许多新派崛起,其中有十个翘楚,号称十大新派,公然和古派抗衡,风头无两。

不像现在,中洲十大古派牢牢把握着中洲大局,就算是拥有蛊仙的超级势力,想要脱离十大古派,摆脱附庸的身份。都万分艰难。

“按照时间推测,难道我眼前的这个郑山川,就是山川堂的祖师爷不成?他既然能够都得五德门的第二,炼道大师级的境界是有的。小小年纪,就是炼道大师。果然是个天才。正符合郑山川的传记内容。名师出高徒,那么这老者就应当是岐山老人。”

“岐山老人受旁人暗算而中毒,一直饱受毒素侵害,郑山川一片孝心,千方百计为其师治病。是了,绿曜蛊就是解决热毒的上佳治疗蛊虫。可惜就连岐山老人也不知道,他身上所中之毒。是变化之毒。表面看上去是热毒。用了绿曜蛊之后,便会变成寒毒,毒性更加猛烈,至少还要严重一倍。不过最终,郑山川还是成功为他的师傅接了毒,他集齐绿曜蛊、蓝曜蛊、红曜蛊等等九种曜蛊,形成治疗杀招九曜。[www.mianhuatang.cc 超多好看小说]可谓无毒不克。后来他的血脉后辈,成就蛊仙的山川堂堂主,将这个杀招发扬光大,改良成仙道杀招玄光九曜,一举成为光道治疗大能,在五域乱战时都小有名气。”

方源用忆念,挖掘出记忆深处,已被时间的尘埃蒙上灰尘的情报。

他对周遭一切洞若观火,对安寒的阴谋心知肚明,隐晦地观察几下。便觉得眼前这对师徒极可能便是岐山老人和郑山川了。

便又用察运蛊观察二人。

飞霜阁是炼蛊大会的比试地点,本身就有严格的蛊虫布置,是严禁场外蛊师随意动用蛊虫的。

但方源乃是仙僵之躯,身上有力道道痕,仙窍自成一体,暗中动用察运仙蛊,这些凡人根本防备不了。

当即。方源便见这师徒两人的气运。郑山川的头上运气,宛若七彩虹光,光泽动容,夭矫不群。而岐山老人的气运,则宛若一件残破的玉盆,从玉盆中流淌出丝丝缕缕的灰色气运。

岐山老人的气运显然是糟糕的,但郑山川亲近他,师徒两人的气运已经相互连接起来。

七彩虹光似的气运,分出一部分,注入岐山老人的气运玉盆,大大冲淡了他的灰色气运。

方源感到有趣,暗想:“没料到我会在这里碰到了山川堂的祖师爷,我这要杀了他,恐怕数百年后东海岸这里,就不会有山川堂这个超级势力了。”

方源杀心一起,顿见师徒二人的气运,飘飘摇晃,宛若大风刮来,彩光晦暗,奄奄一息。

“不过我这样做,又有什么利益呢?北原已经被我玩坏了,其实也不是我的本意,都怪那墨瑶假意。可我到哪里讲理去?就留下这个郑山川一条命吧,这些人还是要尽量保留的,如此才能确保我重生的优势啊。”方源转念又一想。

他当初在北原,就是知道马鸿运,也没有去杀他。

这并非惧怕,而是魔道巨擘的雄心壮志。

前世之所以只是六转蛊仙,距离七转差上一步,大多还是因为机遇,早些年时蹉跎了岁月。

方源有着自信,就算这些人成长起来,甚至比前世记忆中取得更高的成就,方源也有能力与其周旋,从中谋利!乃至将这些人踩在脚下,杀出一条血路,奔向永生。

至于永生究竟存在还是不存在,那又有什么关系呢?

就算死在半途中,又有什么遗憾呢?

这条路本身,就是方源想要的生活,充满了挑战、趣味、甘苦。除此之外,美色、财富、权利都只是可以利用的工具而已。

方源这一想,对眼前这对师徒的杀意便烟消云散。

立时,他便见到郑山川、岐山老人的气运,又恢复如初,甚至还隐隐更兴盛一分。

“运道之妙,果然妙不可言啊。”方源看到这样一幕,心底很有感慨。

他最近对运道,又有了新的感悟运道并非万能,和本身实力修为匹配。

方源此时,一念之下,就能令眼前师徒气运陡变。这是因为他是仙僵,杀这两人如捏死两只蚂蚁一般轻易。

之前的马鸿运,很多人对付他,反而让他因祸得福。现在落到雪胡老祖手中,却翻腾不起来了。八转的鸿运齐天,也抵不上实力的巨大差距。或者说雪胡老祖本身为八转大能,本身实力能镇压得住。

还有为狐仙福地渡劫,方源动用了时济运。结果又因为连运过的关系,运道增幅划分四份,方源只留了一份。结果其余三人各有奇遇斩获,方源身上的效果却不明显。这也是因为方源实力太强,已经成仙。相同的好运要影响他,比影响那些凡人困难多了!

此时场中,撺掇起哄的人越来越多,人都有从众之心,此刻已经不需要飞霜阁的人在背后推波助澜了。

方源摆了摆手:“赌斗可以,但这些东西我还看不上眼。我若输了,我给你绿曜蛊。但你要是输了,就要做我的奴隶,终身听我差遣。”

“啊!徒儿万万不可。”岐山老人大惊失色,连忙阻拦。

众人交头接耳,都说这方源打的好主意,对方是个炼道天才,前途不可限量。若他赢了,岂不是可以收到一位炼道大师级的奴隶!

“前辈您这个要求,未免太过苛刻了吧?”郑山川就算再年轻,也不会轻易打这种赌。毕竟关乎他自身的前途,决定他一生的命运。

“你当然可以拒绝。但我明确地告诉你,你若不用你自己当做赌注,我是不会赌斗的。”方源冷笑三声,又道,“而且你以为绿曜蛊的数量很多吗?这可是治愈热毒的上佳之选啊。小子,树欲静而风不止,子欲养而亲不待。很多事情是不等人的,很多时候,机会就在你的面前,错过了往往会抱憾终身。”

“难道此人,竟然看出我师傅身中热毒?”郑山川心中一惊,方源“子欲养而亲不待”,“机会错过抱憾终身”的话,也深深地刺激他的心。

“徒弟,万万不要赌斗!咱们可以另想办法!!”岐山老人脸色沉重,他到底是老江湖,方源的话简直是像钓鱼的饵,用心阴险,专门钓郑山川这种涉世不深的少年的。

“赌斗是双方的事情,外人不得阻挠。”一位飞霜阁长老走了过来,脸色肃穆,郑重地对方源、郑山川二人道,“二位已经严重干扰了场中秩序,请问二位是否进行赌斗?若是赌斗,本门大供奉心胸宽广,愿意为二位让出赛场。请二位速速做出决断。”

岐山老人心中叫糟,刚要开口,方源又道:“小子,我这是念在你孝心可嘉,才给你这个机会。错过了,就算你今后再找我赌斗,我都不会答应你的。我也不以大欺小,再给你一次机会,让你完全出题。”

“什么,前辈你让我出题?!”郑山川意动了。

原来赌斗,是一方挑起,双方各出赌资。但这比试的内容,双方要协商妥当的。比如一个人擅长炼制某种蛊,另一方完全没炼过,这就太不公平了。

现在方源居然要让郑山川完全出题,也就是说,郑山川尽管拿自己最拿手的项目,来刁难方源,方源只有接受,没有任何改变的权利。

因而此话一说出口,周围蛊师们都是惊呼一片,暗道方源实在过于拿大,纷纷看好少年蛊师郑山川。

“徒儿,不要赌斗了,咱们的正事在大比上……”岐山老人仍旧劝道。尽管郑山川优势很大了,但这位师傅仍旧心中担忧,不想看到自己的爱徒有任何的不幸。

但郑山川到底是少年心性,有着热血。

“师傅的热毒,已经很严重了,再不治疗就晚了。对方这么托大,是看不起年纪轻轻的我!正好我最拿手的是小心蛊。这个蛊是我门中独有,三转级别,炼制步骤繁杂,要求手法多变,我练习了整整三年,付出无数辛酸汗水,才有八成的成功率。我就和他赌这个!”

想到这里,郑山川下定了决心,对方源道:“我赌!”(未完待续。)<!--80txt.com-ouoou-->

\end{this_body}

