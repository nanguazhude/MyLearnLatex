\newsection{群魔战退正道仙}    %第一百五十四节:群魔战退正道仙

\begin{this_body}

方源上前相帮,和黎山仙子合作,挡下残阳老君。

方源乃是仙僵,并不惧怕追命火,犹是如此,他仙窍中的力道虚影大军却是被火烧得十分迅猛。

激战片刻,方源不由暗暗心疼。

原来他催出力道巨手,并不多耗仙元。皆因力道巨手,乃是通过无数群力蛊,将力道虚影的力量结合起来,外显而出。

从根本上而言,每一只力道巨手,只是十万力道虚影在出手。

而力道虚影,早就耗费仙元凝练而出,存放在方源的仙窍之中了。

但现在追命火追溯本源,直接烧毁力道虚影。站至如今,方源先前产生积累的力道虚影大军,已经损耗得七七八八。此时要重新催生,却是要耗费大量青提仙元的。

“这样战下去,消耗太大,不行!仙子你先挡着,我有一计。”方源激战片刻后,忽然传音,同时抽身撤退。

黎山仙子听了这话,心中顿有一股骂人的冲动。

战到如今,还记挂仙元损耗?老娘我早就顾不上损耗了。一提到仙元损耗,她甚至有点不敢去想。

攻打嫣然海时,为了争取时间,她就开始连连催发仙道杀招,现在又和残阳老君交战,损耗更剧。

方源这时候还记挂仙元损耗?一看就是你捞得最多!

看到方源抽身而退,众魔神色各异。纷纷骚动。

“他怎么又飞了下去?”自在书生忙问黎山仙子,他现在可以确定黎山仙子的真正身份。方源根本不鸟自己,他直接找黎山仙子,这无疑更有效。

黎山仙子心中有气,一边激战。一边硬邦邦地回道:“他说激战下去,仙元耗费太大,有一计策需要实施。”

还记挂仙元损耗?我都不知道损耗多少了!

众魔气得怒火升腾,有的双眼通红,有的鼻孔张大,喘着粗气。

自在书生气得身躯都在颤抖。

说起来,即便是黎山仙子等人。也有所收刮。整个魔道群体。就属于他自在书生出力做多,得到的最少。根本就没搜刮到一笔资源!

东方长凡立即感到压力剧增,险象环生。

方源之语,气得众魔下意识地将怒火发泄到他的身上。

一时间,东方长凡举着方寸山左遮右挡,身上受创数十处,狼狈不堪。险象环生。

方源之前所言,并非诳语。

他直朝一处深潭而去。

原来双方激战,不知不觉间战场转移到了另外一处。方源便发现在脚下深潭,隐藏着许多东方一族的族人,很显然,这是一个聚集点。

“东方长凡,我要让你亲眼看着,你的族人们在你眼前惨死!哈哈哈!”

方源的狂笑声中,一只力道巨手悍然轰下。

凡人的布置,怎么当得了仙道杀招?

瞬间撑起来的光罩。层层叠叠,但是在力道巨手摧枯拉朽似的重击下,根本就没有抵挡住一秒中,就彻底崩溃了。

力道巨手呈巴掌拍下,轰的一声巨响,地面深陷下去。

许多倒霉透顶的凡人蛊师,被顷刻间压成肉酱血泥。

边缘处的凡人蛊师。则被一股强烈的掌风掀飞,四散溃逃,大叫救命,毫无斗志。

力道巨手迅速升空,在地面上留下一块血糊糊的手印大坑。

“我的族人!我的亲人!我要保护他们!!”东方长凡见此,再次失控,声音变得清越高扬。

东方长凡心中叫糟,但已经阻挡不住。

这是东方余亮的意志!

最大的内患!!

众魔眼冒精芒,掀起一波凶猛的攻势。东方余亮不管不顾,合身投向下方,救援之心极为迫切。

“该死!该死!”死亡的危机逼近,方姓小人蛊仙连连喝骂,碍于盟约,只得飞起方寸山,悬浮在东方余亮的头顶,为他顶住攻击。

方寸山在众魔的攻势中,剧烈颤抖,无数微小碎石四下迸溅。

“东方长凡,我可被你害惨了!”小人蛊仙不甘又愤怒地大吼。

方源见此计果然有效,仰头大笑,故意放慢手速,追杀残余的东方族人,吸引东方余亮上钩。

但东方余亮飞在半途,又被镇压,让东方长凡再次夺回控制权。

东方长凡忽然发现,自己虽然又受几处重伤,但周围十分空疏,只有几位蛊仙紧追不舍,突围出去和残阳老君汇合大有希望!

原来刚刚东方余亮,不管生死,冒险冲出来,让方寸山以惨烈的代价,承担了攻势。

置之死地而后生,这一点出乎众魔意料,反而让他得到了机会。

“你们这群蠢货,老子好不容易给你们创造了战机,你们居然让人跑出来了?”方源指着众魔的鼻子大骂。

自在书生等人,心中又急又怒,哪有心思和方源计较,纷纷怒吼:

“追上去!”

“不要叫他跑了!!”

一时间,激战变成了追逐战。

东方长凡一力逃窜,企图突出包围,冲出去就是海阔天空,可惜众魔死死跟住,隐隐包围。

另一边残阳老君企图跟东方长凡汇合,却被黎山仙子掐住位置,堵住路线。

一时间,双方像是一群无头苍蝇,在空中胡乱飞舞。

杂乱无章的飞行轨迹,其实体现了蛊仙们扎实的飞行底蕴。这些蛊仙,至少都是飞行大师级!

凡人蛊师没有精力和时间,去修炼飞行,但蛊仙却有的是。

飞行又是一个十分实用的进退手段,基本上蛊仙都会大量练习,掌握精深。

“快,再杀点东方蛊师,让他失误!”自在书生大喝。

方源早就再做了,他用力道巨手,捏起一群群的东方蛊师,大声威胁,然后一一捏死,手段残忍凶暴,已经引得东方长凡失去控制了许多回。

但东方蛊师数量有限,东方长凡险死还生。

这时有人喝道:“郄世民,郄世民先前俘虏了好多的东方族人,想拿去当做奴隶贩卖!”

郄世民大怒,这是他辛苦搜刮的资源!

“郄世民!!”自在书生一边追击,一边匆忙呼喊。

“你且做牺牲,付出的我们在战后都会双倍补偿你!”皮水寒语气阴沉。

郄世民有心拒绝,但一瞥皮水寒已经杀红的双眼,他心头一颤:“好!”

他当即从仙窍中,拿出许多东方蛊师,在东方长凡的面前杀死,惹得其余意志愤怒至极,轮番亮相。

其余的东方蛊仙意志也就罢了,关键是东方余亮的意志。

他还是凡人五转,和族人维系更深一沉,每次他的意志冲破压制,都带给东方长凡、方寸山、残阳老君巨大的麻烦。

时间在激战中迅速流逝,黎山仙子终究阻拦不住残阳老君,让他们成功汇合。

但一众魔道蛊仙,靠着东方长凡频频失误,已经积累了巨大的优势。

战局已定!

方寸山被轰炸得裂痕满布,就算是在飞行中,也不断向下洒落山石。残阳老君为了照看东方长凡,硬挨了不知多少攻势,伤痕遍体,胡须都被鲜血染红。

这一仗,是他生平以来,打得最为郁闷的一仗。

战斗中,几番追问东方长凡最后的夺舍法门。但东方长凡哪里敢这时候告诉他?

一告诉他,残阳老君就跑了!

东方一族只是和仙鹤门合作而已。夺舍法门就是最大的合作筹码。

“撤退吧,此地已不可久留!”残阳老君叹息,向东方长凡传音。

“留得青山在不愁没柴烧啊……”小人蛊仙生怕东方长凡硬磕,连忙附和。

东方长凡苦笑:“二位以为我是那种不识时务之人吗?撤吧!”

他留恋地望了碧潭福地一眼,心中充斥着仇恨、怒吼,但也充满了决断!

一时的得失算不了什么,能舍能退方为真正英雄。

但就在这时,方源传来大笑:“又是一处东方部族的聚集点,东方长凡,你看好了,我要杀了!!”

“无耻的奸贼,胆敢如此欺我!”东方长凡出离了愤怒,万道飞星猛地轰下。

方源见其威势赫赫,笑了笑,他才不硬接,立即闪避。

但这万道飞星却不追他,而是转了一个弯后,直直落下,轰炸在东方一族的那处聚集点上。

躲藏着的东方族人,被顷刻间杀光,没有一人存活。

“这!”一时间,就算是魔道蛊仙,也被东方长凡的狠辣惊住。

“与其让你们威胁我,反不如趁着我能掌控身体时,壮士断腕!东方族人的英魂们,你们听着,我东方长凡一定会给你们报仇。今天的仇恨,我将以千百倍奉还给这些魔道贼子!”东方长凡流下眼泪,痛声哀嚎。

听到他的凄厉的哀嚎声,饶是胆大包天的魔道众仙,也不由地在心底感到冰寒。

“如果让这智道第一人逃出去,日后来对付我的话……”众魔心生忌惮,眼中纷纷涌现凶芒,杀机暴涨数倍。

激战到如今,双方都近乎油尽灯枯,消耗十分巨大。

起因只是利益之争,但双方都杀红了眼,停不了手,如今已经建立深沉的仇恨,不共戴天。

“走!”东方长凡忽然拔升而起,忽然打开碧潭福地的门户,飞逃出去。

\end{this_body}

