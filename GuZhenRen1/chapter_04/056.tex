\newsection{屠仙计方源让利}    %第五十六节:屠仙计方源让利

\begin{this_body}

虎魔上人的怒意接着道:“这些年来,我们十大古派处境都不好,可谓如履薄冰。23us我们就像堵在一个火山口上,火山口底下的岩浆迟早有一天,会喷发出来,对我们十大派造成剧烈的冲击!”

“三长老所言,皆是中洲大局大势。然则大势如此,又如何能改变呢?”太上二长老的意志叹息道。

虎魔上人意志又道:“钧天剑派的例子就在眼前。狐仙福地的一帮蛊仙,有可能是外域势力,但更有可能是本州散修的联盟。而今日袭击鹤风扬者,或许也是本州蛊仙势力。但若是这样,那就可怕了。”

太上二长老点头:“的确如此。袭击鹤风扬的这股势力,明显是不想狐仙福地继续掌握在方源这一方的手中。若是鹤风扬身死,我们仙鹤门就得为了维护名誉,强行对狐仙福地进攻。这虽然是十分明显的嫁祸之举,但我们仙鹤门却不得不吞下这个苦果。”

十大古派维系百万年的尊严,不得冒犯。若是连自家蛊仙身死都不能复仇,那名誉扫地的结果,就是人心浮动,人人都想来冒犯仙鹤门。仙鹤门四面烽火,将再不能掌握如此广阔的地域以及资源。届时仙鹤门越捉襟见肘,露出疲态,便越会引来更多贪婪的豺狼之徒。

这将是一个恶性循环。一旦落入其中,就很难脱身。

仙鹤门的三大太上长老,都是经历丰富。人生经验充足的老怪,早就将这个可怕的前景估算在心。

不仅是他们,中洲其余九大古派又怎么会没有明白人?

当今的中洲。实力膨胀,资源有限,就像是一个越来越大的炸药桶。而十大古派,就坐在炸药桶上,稍不留意,引爆开来,就会被炸得粉身碎骨。

“虎魔。你今日借题发挥,长篇大论,必有见解。不妨直接说来。”太上大长老直接开口。

虎魔怒意哈哈一笑。语出惊人:“我目前的决意,便是答应方源的条件,承认狐仙福地是为我派服用。同时彻查袭杀鹤风扬的蛊仙身份。一旦弄清楚这些散修的联合,我们便和其余九派携手。进行一场大范围的屠仙计划!”

“屠仙?!”太上二长老意志听闻。眼中发出振奋的光。

太上大长老的意志则闭上双眼,缓缓地道:“虎魔,我不瞒你,你的这个建议,早在很多年前,就由战仙宗的蛊仙石磊提出过,他们甚至还上书天庭,要求天庭出面。组织十大古派,实施这个大计。”

“哦。那个仙猴王?不错,我们十大古派虽然在历史上,也有过更名变动,但追根溯源都是正统,都是旁支主干之间的轮流变动,都和天庭有着千丝万缕的联系。十派联合屠仙,兹事体大,涉及无数关键利益。也只有天庭出面,才能服众,引领我们走向胜利!”虎魔怒意赞同地道。

“但天庭方面,却回绝了这个提议。并且言辞警告我们十大古派,不允许特意针对其余各派、以及散修蛊仙,不得随意屠杀他们,违者必定严惩。”太上大长老意志道。

“什么?”虎魔怒意惊异出声。

“这事情,我怎么不知道?”太上二长老也犯嘀咕。

太上大长老长叹一声,仰望苍穹:“天庭的意思,就是天意。天意不可违,天意不可测啊。”

……

鹤风扬再次来到狐仙福地。

方源正和苍郁仙子两人,行走在荡魂山上,观看着胆识蛊的长势。

太白云生远远站着,盯着苍郁仙子。

而那头魅蓝电影,则被八头荒兽围在一起,每隔一段时间,就有一头荒兽上去攻杀,实施车轮战法。

“鹤兄来的有些慢呐。”方源看到鹤风扬,语气淡淡地谈笑道。

鹤风扬先和苍郁仙子对视一眼,见她点头,暗示自己毫发无损,便转向方源道:“我派已答应,承认贵方为我派附庸,附庸关系无期限,随时可以脱离。同时建立胆识蛊贸易,具体价格是一百二十只胆识蛊一块仙元石。这是六转仙蛊诺言。”

仙蛊诺言,亦是信道蛊虫,和山盟蛊、海誓蛊效用类似。

当即,鹤风扬催动仙蛊发下诺言。

轮到方源,他亦接过仙蛊,检查无误,也消耗青提仙元,发下诺言。

诺言宛若一团黄金,双方相互交换,作为依凭。

苍郁仙子终于松了一口气,贸易建立起来,她终于安全了。

方源也松了一口气,他苦心谋划的局面终于达成。和仙鹤门死磕,是绝对不妥的。仙鹤门家大业大,真正惹毛了对方,狐仙福地肯定保不住。方源的经济,也再难支撑一场蛊仙级别的大激战。

但鹤风扬接下来的一句话,却又让他心神一提:“我在回去的路上,遭到三位蛊仙设伏阻杀,几乎命悬一线。对方明显是想嫁祸贵方,如今我派正在彻查,不知道贵方是否得罪了什么势力?”

“什么?居然在中洲,竟然有蛊仙胆敢袭杀十大派的蛊仙?”方源并不遮掩诧异之情。

他明明记得,这样的情况,还要在五域乱战持续了一段时间后,才会发生。

那时,五域遍地烽火,民不聊生,各方混战不休。

天庭的威仪也不管用,十大派蛊仙屡屡被暗算袭杀,有来自其余四域的报复,也有中洲散修蛊仙在混水摸鱼。

情况越来越混乱,方源甚至能带领一干魔道蛊仙,进攻天梯山的狐仙福地,最终将凤金煌杀死。

不知为何,方源在第一时间,就想起一年多前,在宝黄天中暗算他的那个神秘势力。

但他连这个势力叫什么都不知道。就算知道,也不会随意地对鹤风扬暴露这些情报。

于是他摇摇头:“我占据狐仙福地足不出户,怎么去得罪其他势力?我无意开罪鹤兄。但想想看的话,仙鹤门的确比我方要树大招风得多,或许是有什么势力本就要对付贵方,想借助我方来转移视线呢?”

鹤风扬没有得到情报,心中有些失望,却也不能强逼方源。

仙鹤门有自家顾忌,方源尽管只是仙僵。但背后势力,足够双方平等对话。

鹤风扬、苍郁仙子没有多留,方源打开狐仙福地的大门。送他们俩出去。

方源站在门内,与他们俩辞别。

看到这一幕的其余九大派蛊仙,立即察觉到,狐仙福地一事出现了变故。

“怎么回事?怎么有一头仙僵?”

“鹤风扬、苍郁仙子意气风发而来。出来的时候却有些神色不佳。看来是出现了问题!”

“难道他们没有攻打下来?这不太可能吧。”

蛊仙们一头雾水,纷纷猜测。

十大古派同出一源,皆是中洲霸主,这些年来少不了明争暗斗。狐仙福地之争,先是明争,用各派弟子较量,决定归属,结果被方源所趁。现在是暗斗。不能当面跳出来发问,以免坏了面皮。

蛊仙们不知道狐仙福地究竟发生了什么。都逡巡不去。

方源关闭门户,黑楼兰、黎山仙子二人带着笑颜,上前恭贺。

“方源,你叫我刮目相看。这一次你和仙鹤门合作,可谓不战而屈人之兵,实乃大胜。”黎山仙子交口称赞不已。

这是蛊仙向来最为欣赏的胜利。

没有战斗,没有巨大的投资,却带来丰厚的利润。

“前辈说的不妥。”方源朗声一笑,“不是我和仙鹤门合作,而是我们和仙鹤门合作。须知要炼成气囊蛊,还得需要黑楼兰的力气仙蛊催发的力气。关于胆识蛊的收益,你方四成,我方六成。”

黎山仙子闻言,眼中笑意不禁更浓。只要不傻,都能看出胆识蛊贸易的广阔前景。

她当然明白,方源此举是想彻底稳固两方联盟,将她们俩个绑在荡魂山。今后狐仙福地遭受进攻,她们将是必到的防守力量。

但重利当前,由不得黎山仙子不心动。

黑楼兰深深地看了方源一眼,道:“没有想到方源你有这一手,居然能够设想出气囊蛊。要炼三转气囊蛊,却需要六转力气蛊的参与。这蛊方算是相当奇特。”

这其中,当然有智慧蛊的功劳。

方源自从得到智慧蛊后,就一直设想,能否在无限灵感的帮助下,开采出胆识蛊。

这一次成功,也是他意外地从太白云生口中,得知黑楼兰掌握了力气仙蛊。

唯有用仙级的力量,才能包裹胆识蛊,令其无损地离开荡魂山表面。

方源转了话题:“要催生胆识蛊,还得需要大量的魂魄。这一点,就麻烦二位了。”

“北原好战,战乱频繁,兽群、部族迁徙流动,魂魄是少不了的,很容易就能收集。我们这便回去着手处理。”黑楼兰雷厉风行。

她升仙成功,对资源需求更大,尤其是仙元石短缺,总不能一直依赖黎山仙子。

依靠别人,不是黑楼兰的风格。

黎山仙子也赞同地点点头:“不错,荡魂山对魂魄没有太多要求。我们还可以向北原僵盟等势力收购,比亲自动手收集,自然要贵一点,但价格仍旧可以承受,还可以省下许多麻烦。”

“呵呵呵。”方源摆手,“二位且慢,不要着急,先随我一同,再去势压一位蛊仙。此仙就在狐仙福地当中。”

“哦?是谁?”太白云生疑问。

“居然还有其他蛊仙躲在福地里?难道是仙鹤门别有用心,故意留下的埋伏?”黑楼兰和黎山仙子对视一眼,也大感诧异。

------------

\end{this_body}

