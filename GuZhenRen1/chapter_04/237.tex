\newsection{痛哭的小人}    %第二百三十八节:痛哭的小人

\begin{this_body}

a hre=\&quot;\&quot;  \&gt; /b\&gt;

北原,某处隐秘之地,落魄谷。[求书网qiushu.cc更新快,网站页面清爽,广告少,无弹窗,最喜欢这种网站了,一定要好评]复制网址访问 ,: 。

‘阴’魂咆哮,天雷电闪。火光四‘射’,炸响连连。

一场大‘激’战,正在上演。

防守的一方,便是之前断然放弃琅琊福地,及时回援的影宗秦百胜、姜钰仙子、回风子、贺狼子等人。

仙道杀招魂压!

秦百胜低喝一声,冒着漫天的光和火,‘挺’身而出。

轰轰轰!

三声剧烈的爆响,秦百胜将来犯的三位蛊仙打退回去,气势勃发,给人一种一夫当关万夫莫开的感觉。

“我堂堂的陈振翅,居然会在一位蛊仙手中,连败三次,每一次都走不过一招?!”来自万龙坞的蛊仙陈振翅,艰难地停住身形,望着眼前的秦百胜,满脸惊怒。

“可恶……好不容易才凝聚出的一‘波’攻势,居然又被他轻而易举地瓦解了。”步飞烟咬牙,嘴角溢血。

“这个魂道杀招真的太强太强,没想到这个秦百胜真正的实力,居然这样强大!老算子的推算果然是对的,这个家伙很有可能就是捣毁八十八角真阳楼的凶手!”天聋老人心中暗道。

进攻落魄谷的,不是旁人,正是来自中洲十大古派的一行蛊仙。

自从在大雪山福地碰壁之后,凤九歌并无气馁,指挥得当,探知了许多隐秘线索。

经由老算子推算之后,众仙得出落魄谷的方位,便一齐赶来。

但落魄谷早就被影宗经营长久,有强大的防御蛊阵。中洲蛊仙们受此拖延,秦百胜等人则当断则断,立即回来支援。

因此,便形成现在的局面。

双方僵持不下。

中洲一方想要进攻落魄谷,而影宗一方则是严防死守大本营。

变化道香巫‘阴’雕狼!

风道亡风飞刃!

贺狼子、回风子见中洲蛊仙攻势受挫,立即抓住良机,展开犀利的反攻。

来自天妒楼的凌梅、傲雪两位仙子。不敌他们俩的锋芒,一时间只有节节败退,没有还手之力。

仙道杀招碧‘玉’歌!

关键时刻,凤九歌出手,音道杀招一出,不同凡响。

贺狼子、回风子受创爆退。

回风子退回防护蛊阵之中,连连吐了十几口的血。这些血液都成了幽绿的‘玉’‘色’。发生了质变。

贺狼子受创更大,他变化的香巫‘阴’雕狼大半个身躯,都被侵染成‘玉’石,一时间心中震惊无比:“这是什么仙道杀招?似乎克制变化道!我中了此招,居然无法转变回人形。看来只有将这伤势解决掉,才能继续变化了。”

“凤九歌!”秦百胜怒喝一声。声音响彻整个战场。

“秦百胜,我小看了你。上一次居然被你糊‘弄’过去了,你的演技叫我甘拜下风。这一次也是得亏我方有智道蛊仙,又不吝寿元损耗,方能推算成功,算出这个落魄谷。<strong>在线阅读天火大道Http://wWw.qiushu.cc/</strong>”凤九歌仍旧是那一身红白相间的长袍。

他身姿‘挺’拔,似枪似剑。此时说话。微微带笑,口气悠姿十足。

“推算成功个屁!我和八十八角真阳楼倒塌,一点关系都没有。不过你既然来了,就把命‘交’下来罢。”秦百胜说着,缓缓地闭上双眼,低下了头。

然后他双掌合十于‘胸’,右手在左掌上一抓。捏成拳头,高高举在头顶。

见到这样的姿态,中洲蛊仙们纷纷变‘色’,惊疑不定之间,众仙纷纷后退。

唯有凤九歌立于原处,好像是钉在天地之间,风吹雨打岿然不动。

他见到秦百胜这般姿态。不由双目一亮:“你这招,难道就是剑仙薄青的五指拳心剑不成?”

“不错,正是如此。”秦百胜开口答道,“你准备好受死了吗?”

凤九歌哈哈一笑。神‘色’略带出兴奋:“好好好,妙得很。剑仙薄青乃是我灵缘斋的前辈,他升仙失败,杀招遗藏也神秘消失。我们灵缘斋没有得到,反倒是北原的蛊仙得到了手中。不过我从‘门’派典籍中看过相关记载,这五指拳心剑仙道杀招,威力绝伦,锋锐无当。据说,当初剑仙薄青大人创下此招,是有感于天地宏伟,升仙困难,心中尽是以一己之力,抵抗天地的勇气和豪情。”

“所以这一招姿势相当奇特,用招之人面对天地,低头闭目。看似服软,实则是竭尽身心之力,酝酿出最犀利的反击。用拳头高举头顶,便可见剑仙薄青是多么的豪迈无双,正所谓我命由我不由天,剑在心中,矢志剑道。”

面对传说中的杀招,凤九歌一点都不紧张,反而侃侃而谈。

他身后的中洲蛊仙们,因此受到影响,心中的那丝惊惶纷纷散去,不由更加佩服凤九歌的心‘性’。

“哼,你知道的倒是不少,看来是预感到失败身亡的结果了。”秦百胜冷笑连连。

凤九歌摇摇头,朗笑一声:“这可巧得很。我自创一记音道杀招,灵感来源于某个时期。在修行的途中,我发现天地,面对天地,敬畏天地,感受自然浩瀚,察觉到自己的渺小和虚弱。这一招,叫做天地歌!天地是多么的宽广,而人是多么的渺小。此歌就是借助天地之力,以无以伦比的威势,镇压一切反抗!”

一方是天地浩瀚之威,顺其自然,雄浑万丈。一方是以己逆天,剑道决意,一往无前。

这两招,简直是针锋相对。

就是不知道,究竟是天地歌厉害一筹,还是五指拳心剑胜出一等?

一时间,整个战场的节奏都缓慢下来。

众仙的目光,都集中在凤九歌和秦百胜的身上。

甚至就连秦百胜的脸上,都涌现出一抹奇异的神‘色’。他仍旧闭着眼眸,口中出声:“哦?那这场对决真是有趣得很了。接好了,第一指!”

……

光,照耀着天庭。

亘古不灭。

炼道蛊阵,形成的巨大光影,盘踞在半空中,已经完全不刺眼,光辉尽数内敛。

“好。第一阶段已经结束,所有的材料都被处理了。接下来第二阶段,就是接引天意!”监天塔主视察了一番后,开口道。

“接引天意……”沧水仙子口中喃喃。

炼九生、碧晨天的脸上,也随之涌现出一抹凝重之‘色’。

监天塔主继续解释道:“所谓天意,就是天地的意志!人有意志,天地也有意志。人和天地相比。渺小如蚁,卑微如沙,根本不值一提。天意就是修复宿命蛊,最重要的仙材。接下来我们主持这座炼道蛊阵,将极其漫长艰辛。因为我等都要抵抗天意,天意浩‘荡’。千万不能被其摧垮,否则将受到极其严重的损伤,甚至死亡!在这个过程中,天庭也曾经损失过好几位八转蛊仙的。磨刀不误砍柴工,接下来先轮替休息片刻,整理状态。”

“好。”

……

中洲,狐仙福地。

一块巨大如耕牛的羊‘肉’。摆在方源的面前,血‘肉’淋漓。

方源坐着,伸出怪爪,抓捏一份,轻易撕扯开来,放入嘴中。

血盆大口不断咀嚼,片刻后,咕咚一声。吞咽下去。

一丝丝的血液,顺着尖锐的牙齿缝隙,漫溢出嘴角。方源的脸上却涌现出满足的幸福神‘色’。

这羊‘肉’可不普通,乃是巨角羊身上的‘肉’。

方源在北原时,活捉了一头力道荒兽巨角羊,此刻正是利用吃力仙蛊,吞食其‘肉’。增长自己身上的力道道痕。

“我终究是力道蛊仙,力道道痕才是根本啊。”

方源一边吃,一边回顾自己这次参加的琅琊福地攻防战。

此战中,方源首先检验了见面似相识杀招。瞒过了几乎所有人,效果叫方源满意。

其次,他又在实战中运用了星道杀招。

星云磨盘、星蛇索、六幻星身、位星移。

实战和平时的练习大大不同,方源经过这次实战,对这四道仙级杀招有了更深一层的领悟。

当然,万象星君的那份星道传承中,绝不仅有这四道杀招。

但方源目前,就只能用这四种。

因为他只有星痕、星光、星芽三只仙蛊。这三只仙蛊轮番运用,作为核心,才有了四个星道杀招。

万象星君曾有还有第四只星道仙蛊,但因战斗而毁。落到方源手中的,就只剩下三只。

“我原本还想,通过力道仙蛊,利用智慧光晕,推算出一些仙道杀招,补齐自身短板。但这些星道杀招都‘挺’不错,暂时能应付着用,就不用‘浪’费这个‘精’力和时间了。”

这一次的琅琊福地攻防战,让方源对自身战斗力又有了一层更加清晰的认知。

“算上星道杀招,我的战力已经稳定在六转巅峰程度。单算攻击方面的话,有万我在,可以媲美七转。不过要和老字辈的七转蛊仙作战,我还差得远,只能尽量周旋。至于秦百胜,就更无法企及,这是准八转,凤九歌一样的人物!”

这一次战斗,让方源真正认识到了秦百胜的强大。

正因为如此,整个攻防战,方源出的力都较少,基本上在划水。只有在最后关头,才施他展出万我大手印。

这种情况下,高调就是找死。

至始至终,方源都保留一份心神,时刻控制着定仙游。一旦事情不对,他就立即撤退。

“秦百胜究竟是什么人?真正的实力居然这样深厚!黑城、姜钰等人、回风子这些人,怎么搞在一起的?秦百胜似乎是他们的首领,黑城在队伍之中,黑楼兰要报仇恐怕是无望了。说起来,琅琊福地的水真的很深啊。”

方源的前世记忆中,琅琊福地可是抵挡了整整七‘波’攻势。

但是到现在的第四‘波’,就有一种抵挡不住的迹象了。

究竟历史的真相是什么?是不是方源自己带来的影响,改变了琅琊福地的处境?

呈现在方源面前的,是一团厚重的‘迷’雾。

“我现在就算是完美状态,面对秦百胜,尤其是那仙道杀招魂压,根本毫无还手之力。真是弱小。”

方源低头望着自己的手,他的手沾满羊血,血糊糊的一片。

和秦百胜相比,方源就仿佛是这巨角羊,只能任人宰割。

天地广博,自然浩瀚。知道的越多,就会发现自己越无知。力量越强大,就会发现自己越软弱。

《人祖传》中,有个记载很有趣。

森海轮回脱离了父亲人祖,只能留在平凡深渊里头。

她十分伤心,吃着果实也不再快乐。

她每天都以泪洗面,哭泣不止,最终哭得累了,渐渐睡着了。

在睡梦中,她‘迷’‘迷’糊糊地听到一些十分微小的声音,感到身上像是蚂蚁在爬动。

于是她睁开双眼苏醒,坐起身来,发现身上爬着一个小人。

这个小人因为森海轮回的动作,立足不稳,摔倒在地上。

“你是谁?天底下居然有你这么小的人?”森海轮回发现小人不足自己的手指头大,感到十分好奇,一时间忘了哭泣。

小人呆呆地望着眼前巨大的森海轮回,震惊过后,他仰头大哭。

“喂喂喂,小小的人啊,我都没有哭,你哭什么?”森海轮回十分不解。

小人一边哭,一边说道:“我是我们部族中体型最大的了,我常常因此而勇敢、骄傲、得意。今天我打算攀爬一座山,没想到这座山居然是一个人。天底下居然有你这么大的人,我还是头一次看到,忍不住就哭了!”

几乎每个探索成长的人,都会有这样的心理历程。

看到的越多,越明白自己的弱小。有时候会感叹天地的伟大,有时候会发现自己的目标是那么的遥远,自己要达到目标,有一段自己曾经没有认识到的漫长路程。于是心中‘迷’茫、失措、气馁、惊惶,甚至绝望。

小人见到森海轮回时的痛哭,就可以理解了。

“回想起来,前世的时候,我也曾经‘迷’茫,‘痛哭’过。山外有山,人外有人。总有比我强大的存在,永生的目标太过遥远高上,渺小的我,何以实现?”

方源望着自己血糊糊的手掌,一阵失神。

半晌,他忽的轻笑一声,‘露’出锋利的獠牙。

“还是太弱了。不管是前世还是今生,我都是那个想要去攀山的小人啊。”

“不过……自身的渺小,不是停止追寻伟大的借口。”

“只有懦弱和失败者才会四处寻找借口。”

“就算痛哭流涕,我也要继续攀山,这才是人生的乐趣所在啊。”

想到这里,方源伸出手掌,撕下一块血‘肉’,放入嘴中。

獠牙利齿狠狠咬下,在他的嘴边,溢出一缕鲜红的血。 小说<!--80txt.com-ouoou-->

------------

\end{this_body}

