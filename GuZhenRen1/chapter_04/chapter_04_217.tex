\newsection{设计羽民,羽飞的无力}    %第二百一十八节:设计羽民,羽飞的无力

\begin{this_body}



%1
羽民蛊仙郑灵的面色变得十分难看:“阁下以为吃定了我们吗?其实何必如此咄咄逼人呢。贵方经营仙窍着实不易,这漫空的资源若是因战而毁,岂不可惜吗?”

%2
“你这是在威胁我喽?”方源赤红的双眸,透露出凶光。

%3
“当然不是威胁,但我方说的也是事实,不是吗?”蛊仙周中出声道。

%4
方源心中大笑。

%5
若换做其他蛊仙被这样威胁,便要投鼠忌器,说不定真叫这两位羽民蛊仙得逞了。但偏偏他们遇到的是太白云生。

%6
太白云生手中的仙蛊江山如故,正是对威胁的最佳解决之道。

%7
两位羽民蛊仙不知道江山如故仙蛊的存在,所以才说如此的话。

%8
依照方源自己的意思,当然是留下两位羽民蛊仙的性命。不过此事却是太白福地渡劫,不是方源做主。而白云生此人,又偏偏是个老好人,性情方面有些仁慈优柔。

%9
所以方源没有再开口,而是看向太白云生。

%10
两位羽民蛊仙也望向太白云生,等待着正主的回应。

%11
凡人羽民们仰望高空,带着浓重的敬畏之色,看着天空中悬浮着的四位身影。

%12
之前,方源和羽民们交谈,声音恢弘,所以这[些凡人羽民都听得到。

%13
太白云生的决定,将极大地影响着这数万羽民的命运。

%14
有的羽民已经开始默默地祈祷,有的咬牙切齿,有的受伤痛哼。

%15
刚刚登上羽民王座的羽飞。则紧捏双拳,死命地盯着高空。他是草根出身。不像原王子丹羽接触到隐秘,他没有太多见识。这是首次看到蛊仙,震骇的同时,又有一股深深的无力之感。

%16
太白云生从这群羽民出现起,就陷入沉思考虑当中。

%17
羽民成灾,太白云生始料未及。

%18
这些智慧生命,可以沟通。若是太白云生放了这些羽民,那么地灾就算是渡过了。若是不放过,在自家仙窍福地中激斗,或许可以收编了这些羽民。但战斗的风险和损失,一时间难以估量。

%19
不管他如何选择,这都是个重大的决定。

%20
正如方源所料,太白云生有些优柔寡断,此时急切之间,也拿不定主意。

%21
见到方源望来,他也回望过去,以目光示意,征求方源的意见。

%22
“当然要战!”方源立即回应。斩钉截铁。

%23
“你有仙蛊江山如故,根本不惧怕仙窍福地被打坏。唯一损失的是这漫空的浮球茶草、天魁云丛等等资源。这些资源要损毁了,江山如故仙蛊是无法回复的。”

%24
“但你这次渡劫,没有将这些资源都搬出去。本来就是有承受损失的准备。”

%25
“你的主要财力支柱是江山如故仙蛊,这些资源收益很小,就算全部损失了。重头再来也无妨。”

%26
“这些羽民有数万,都是可以调教成奴隶。羽民天生有云道道痕。擅长飞行,羽民中的精锐几乎都是飞行大师!但这两位羽民蛊仙。绝对不能留。就算他们没有主修云道,杀了他们俩个,也能得到不少云道道痕,增添到你的福地里去。这不正是你想要的吗?”

%27
这种千载难逢的机遇!

%28
两位羽民蛊仙身上都有伤,似乎也没有手段逃脱这里。

%29
只要太白云生不开放门户,这群羽民简直就是瓮中之鳖。

%30
一旦将这两位羽民蛊仙杀了,他们的仙窍就落到了太白福地当中。这种情况是非常特殊少见的。

%31
失去的蛊仙,留下仙窍,仙窍要化为福地,就得汲取天地之气。

%32
就像当初,方源取走了万象星君的尸体,特意跑到地渊深处落下福地。结果福地种下之时,汲取大量的地气,少量的天气,地气抽空,剧烈波动,导致地渊都发生了一场剧烈的地震。

%33
若是羽民蛊仙战死,仙窍落到太白福地当中,要彻底化为福地,就要汲取天地之气。

%34
但太白福地和五域大天地不同,这里受到太白云生的掌控,天地二气不是你想抽取就能抽取的。

%35
仙窍不能彻底化为福地,会变成怎样呢?

%36
历史中有明确记载,这种情况下的仙窍,就会泯灭,里面的一切资源都会随着仙窍毁灭而毁灭。

%37
但是仙窍中蕴藏的各种道痕,却是保留下来,融入了外部天地。

%38
这里的外部天地,就是太白福地。

%39
前面已经提过,道痕是极其珍贵的,获取的成本是极高的。

%40
太白云生就算将福地打烂,但若能继承了这两位羽民蛊仙的所有道痕,那这笔生意绝对是稳赚不赔!

%41
方源暗中传音劝说,末了又添加一句:“老白,你该不会起了妇人之仁,将这到嘴的肥肉都要吐出去吧?”

%42
太白云生正色,下定了决心,传音回去:“这你放心!异人羽民,又不是纯正人族。非我族类,其心必异。那就战吧,杀了他们!而这些羽民俘虏,我必会善待。”

%43
商定妥当,他便欲动手,却反被方源暗中传音拦住:“且慢!难得有这么好的机会,岂是能说打就打?先让我再试探一下。”

%44
太白云生哪里不知道方源的狡诈,当即道:“那就有劳方源你了。”

%45
两人秘密交谈,只在很短功夫。

%46
方源望向两位羽民蛊仙,假意开口:“放过你们俩个,其实也不是不可以。但你们要留下这群羽民凡人,充当我们的奴隶!”

%47
两方蛊仙交谈的声音,没有隐瞒。方源的这番话,立即引来了地面上的数万羽民剧烈的骚动和强烈的抗议。

%48
“什么?居然要我们成为他们的奴隶?”

%49
“失去自由,就等若失去双翼!”

%50
“不,我就算死。也不会当人族的奴隶!”

%51
“别怕,我们也有蛊仙老祖宗。他们一定不会放弃我们这些后辈的。”

%52
“这绝不可能!”两位羽民蛊仙勃然变色,也是当即否决。

%53
方源却不意外。

%54
异人族群。比较纯正人族,往往更为团结。东方长凡坑害亲族,成就自己的事情,在异人中基本上不可能发生。

%55
人是万物之灵,灵性第一,心思多变。反而异人们灵性较次,心思单纯,坏心思歪歪肠子远没有人族的多。

%56
当然,这也和种族大势有关。

%57
当今。人族统治五域,最为势大。异人们生存十分艰难,被四处排挤,饱受人族打压,鲜有能掌控珍贵资源的异人族群。

%58
不论在哪一域,都有异人奴隶贩卖,且大有市场。

%59
在种族大势中,人族处于绝对的优势,而异人们则是绝对的弱势。

%60
处于绝对弱势的群体。若想生存下去,必然要更加抱紧成团,相互扶持,才可对抗强大的外部压力。

%61
因此。对方两位蛊仙不愿意放弃凡人羽民,且态度强硬,方源早有意料。并且深刻理解他们两仙的决心!

%62
方源连忙传音,太白云生便表现出一副被冒犯而生气的样子。照着方源暗中叮嘱的话重复道:“哼!你们这群羽民无故地闯进我方的福地,这就是入侵!你们不仅入侵。主动向我方出手,还威胁我方,又想要出去?哼,不付出点代价,怎么行?当然,你们两位也可以带着这些羽民打出去。只要你们本事够强,我方认栽!不然的话,老夫分分钟收拾你们!”

%63
两位羽民蛊仙,哪有这样的本事?

%64
若是有,他们又何必和方源用言语苦苦交涉呢?

%65
虽然他们还可以动用仙道杀招天随人愿。

%66
但如今再没有仙蛊屋羽圣城的防护,天随人愿杀招耗费大量仙元不说,又需要一段长时间的酝酿。

%67
在这种情况下,基本上是不可能复制之前逃脱的壮举的。

%68
眼看方源不肯松口,蛊仙周中只得再度重申,仙窍对蛊仙的重要意义。眼下若真要开战,对两方都不好。

%69
“若是我们两位,一味避战,专门轰击贵方的这片福地。到一定程度,福地毁灭破碎,引发大同风,那就不好收场了。”周中说道,威胁的意味十分明显。

%70
他们不知道江山如故的存在。

%71
有了江山如故,仙窍福地破坏哪里,就能修复哪里。绝对不会毁坏到引发大同风的程度。

%72
在方源的指点下,太白云生大怒:“你还敢威胁我?战就战!怕什么?真当老夫是孬种了?老夫分分钟就能收拾你!”一副人老心不老,性情火爆的好战样子。

%73
方源则表现出一副投鼠忌器的样子,但仍旧不松口,想要再协商。

%74
他这个样子,不由地让羽民两位蛊仙都平添了许多和平出走的希望。

%75
双方扯皮了良久,郑灵终于让步,提出可以向方源一方,补偿一笔不菲的仙元石。

%76
他们两个,周中扮红脸,郑灵扮白脸,一唱一和,倒是很有默契。

%77
方源望太白云生,太白云生便翻起大白眼:这笔仙元石哪及得上数万羽民的价值?打吧打吧,分分钟就能收拾了他俩!

%78
方源和太白云生的默契,也不输给对方两人。

%79
地面上,羽民们骚乱不止。气愤恼怒又悲哀无奈,这就是残酷的世界,弱者的悲哀。

%80
羽飞仰天大吼:“够了!我们就算死,也不会成为卑贱的奴隶!”

%81
这种被他人当做牲口讨论,当做谈判的交易筹码的感觉,真是糟糕透了。

%82
“你闭嘴!不要去无故地招惹蛊仙这样的存在。你是新的羽民王,就要为整个族群考虑。你这样嘲讽蛊仙,是想要把我族带入深渊吗?”丹羽怒骂。

%83
“可恶,可恶啊!”羽飞咬紧牙关,双拳捏得青筋暴起,却听从了手下败将丹羽的喝骂,陷入了沉默。

%84
对于羽飞而言,命运的起落来得太快,太突然了些。

%85
立志于成为羽民王的他,已经达到了他人生的理想。但曾经堂皇高耸的目标,在此刻看来,却显得渺小如尘埃。

%86
蛊仙对于凡人而言,真是太强太强了。

%87
就算再热血,再拼命,也无法超越那实力的鸿沟。

%88
ps:提前祝大家节日快乐!这个月每天都是双更,时常思虑枯竭熬到深夜。如果可以的话,还请大家都支持一下蛊真人吧。

\end{this_body}

