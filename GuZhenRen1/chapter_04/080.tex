\newsection{分赃}    %第八十节:分赃

\begin{this_body}

“万象星君?是他……”听到地灵自报家门,方源的脸上立即闪现一抹恍然之色。

“地灵,不知道你还记得我吗?”旋即方源淡淡而笑,“我可是向你的本体,购买过星屑草,以及春星雨杀招的人呐。”

星象童子目光一闪,重新打量了一番方源,神色也缓和了一丝:“原来是你啊,我有一点印象。”

“单靠地灵你一个人,却是无法顾及福地周全的,不知道你想不想投靠我们中的一个?”方源直接问道。

黑楼兰站在一旁,目光中微微流露出紧张之色,盯着悬浮着的星象童子。

星象童子点了点小脑袋,流露出一抹悲伤之色,道:“我要保全这片基业,的确需要记住他人之力。不过我也是有条件的。”

星象童子虽然来源于万象星君的遗留执念,但此刻已经和这片福地同化,是整个福地的意志代表。

如果没有其他蛊仙的帮助,福地将会经历一次次的天灾地劫,最后毁灭消亡。

任何生命都有求生的本能,就算地灵也不例外。

“请讲。”方源忙道。

星象童子眉头皱起,嘟起小嘴,很生气地道:“我的本体之所以陨落在第八星殿里,起因都是宋紫星。他让我的本体受伤,不得不暴露秘密,联合仙猴王石磊,探索洞天。想要成为这片福地的主人,就得将宋紫星的人头带过来给我!”

“宋紫星的人头?”黑楼兰听了这个条件。旋即苦笑一声。

即便她不是中洲之人,也早早听闻宋紫星的大名。

宋紫星早年是万龙坞弟子,得到血海真传。不愿上缴,选择了叛逃。万龙坞为了剿灭他,陆陆续续地派遣了八位蛊仙。

五位六转,三位七转,结果被宋紫星杀了四个,打残三个,败退一个。

万龙坞吃了个大亏。从此之后,再也不敢随随便便就向宋紫星出过手。

宋紫星如今逍遥法外,是中洲魔道顶级的强者。他有七转修为,血道又极为擅长战斗。

中洲世人公认,宋紫星的战斗力和石磊相差仿佛,比凤九歌差一筹。皆因多年前。宋紫星路遇凤九歌。凤九歌出了三招,宋紫星不敌,溃败而逃。

“好,我尽量试一试。”方源面色不变,答应下来。

“你们要快一点,尽全力!因为星象福地的地灾快要到了,如果我撑不住,福地毁灭。你们就算达成标准,也没用了。”星象童子叮嘱道。

方源目光一闪:“若是这样的话。我倒是可以首先帮助你抵御地灾。”

星象童子立时将脑袋摇得如同拨浪鼓:“我是不会让外来蛊仙,在星象福地中待着超过半刻钟的。人心叵测,只有我的主人才能帮助我渡劫。不久后,我还会通过宝黄天,将这个消息透露出去,让更多的人去对付那个可恶的宋紫星!”

黑楼兰心中顿时一沉,不过她城府很深,面色仍旧保持着平静。

她斟酌了一下措辞,这才道:“地灵,那你可得小心点。人心叵测,说不定会引起贪婪的豺狼,直接进攻你的福地!”

星象童子笑了一下:“你放心,我不是笨蛋,不会将这里的位置先告诉他们的。只有达成我的条件,才能进入星象福地。你们走吧,记得不要把福地的位置告诉其他人。”

地灵逐客,方源也没强硬逗留,顺势告辞,和黑楼兰两人通过福地门户,回到地渊。

他们俩个刚刚踏出地渊,身后的大门就乍然消失,彻底不见。

福地是一个小世界,虽是落在地中,汲取着地气,但只要门户不开,极其隐蔽。

方源、黑楼兰沉默不语,扫清周围一切痕迹之后,这才动用定仙游,回到狐仙福地。

周围安全了,两人首先要做的第一件事,就是分赃。

“这次进入繁星洞天,有惊无险,收获巨大。我现在有些后悔了,早知道如此,当初就不应该答应你四六分成的。”黑楼兰叹了一口气,将仙窍中的战利品,都拿出来。

“怎么这么多的荒兽尸体呀?这头大猪,好像是上古荒兽。咦,还有一头狮子荒兽没死呢。”一时间,小狐仙都被吸引过来,惊异地看着一地的荒兽。

赤莬神驹、飞熊、星魔蝠、板栗牦牛……一共七头荒兽尸体,还有一头黄玉狮子,濒临死亡,吊住一口气。

除此之外,就是气罡飞天猪,这头上古荒兽的尸体。还有数百斤的钻土,从钻熊的洞中挖出来的。以及最后的近十根神秘大树。

这些大树,结成树林,在刚刚进入繁星洞天时,曾经一度困住方源和黑楼兰。两人都看不出此树来历,因此采集了一些进入仙窍。

后来,有一头星荒犬杀来,破坏了二人的采集行动。

一头荒兽尸体,大约能卖四十块仙元石。而上古荒兽,卖价更高,是荒兽的十倍,也就是四百块仙元石!

这一听,上古荒兽的卖价,似乎高得有些不合理。

但事实上,上古荒兽难以斩杀,战力媲美七转蛊仙不说,身上道痕密布,一般的凡道杀招都起不了作用。就算是仙道杀招,也是够呛。

一般的仙道杀招,以六转仙蛊为核心的,对付上古荒兽,就像是用小匕首捅大象。虽然见效,但需要不断积累伤势。

只有用七转仙蛊组合的仙道杀招,才是对付上古荒兽的有力手段。

这样一来,涉及到仙道杀招,就要消耗仙元。斩杀上古荒兽的成本,就高了。往往需要消耗数百颗的青提仙元,或者小几十颗的红枣仙元。

如此一减。真正的纯收益往往只在一百块仙元石左右。

虽然比斩杀荒兽的纯收益,要多得多,但还得考虑到两个因素。第一。和上古荒兽激战的巨大风险。第二,不是任何一位蛊仙,都能对付上古荒兽的。普通的六转蛊仙,面对上古荒兽,常常会选择撤退躲避。

也就是石磊这样的七转蛊仙,而且是七转中的强者,才能够有自信能斩杀上古荒兽。

“七头荒兽尸体。总价二百八十块仙元石。濒死的黄玉狮子,姑且算作八十块仙元石。气罡飞天猪的尸体,四百块仙元石。数百斤钻土。大约价值三十块仙元石。这些神秘树木,不知道来历,暂且不算。如此一来,就是七百九十块仙元石了。”

黑楼兰算着算着。得到的结果。让她自己都有些咋舌起来。

将近八百块仙元石!

毫无疑问,这是一笔巨款横财!!

黑楼兰的呼吸,都微微急促起来。

自升仙以来,她都是一块一块仙元石,这样算计着开销的。现在手中忽然冒出了近八百块的仙元石!

方源比黑楼兰的情况要好点,凭借智慧蛊灵光无限,他和琅琊地灵做交易,手头上的仙元石都是十几块。小几十的流通。

但即便如此,面对眼前近八百块仙元石的财富。方源也是心生感慨。

就算是前世,他也没有过这样的经历。在他五百年前世,他手头上的仙元石最多的结余,也不过六十几块而已。

按照四六分成的话,方源能够得到四百七十四块仙元石。

“这次收获巨大,主要还是因为我们不劳而获,没有因为战斗消耗任何的仙元。纯粹是白捡的。”方源平息了心中的微澜,道。

“嘿嘿,那只蛮横的猴头估计要气坏了。他会不会来进攻狐仙福地?”黑楼兰有些担忧地问道,毕竟涉及到如此一笔利益。

“不会。”方源摇摇头,“对我们来说,这是一笔庞大的收益,但是对于石磊这样的七转蛊仙,手中的仙元石会有数千块!”

一般而言,六转蛊仙的手头上都会积余数百块仙元石的。普通的七转蛊仙,就要有上千的数量。

实力和财富,是成正比的。

方源、黑楼兰因为是新晋蛊仙,刚刚迈入蛊仙的世界,以全新的身份开始打拼而已,白手起家,一穷二白。

至于前世方源手头紧,一是因为血道不擅经营,长于战斗;二是因为五域乱战,战斗频繁,消耗很大。

方源接着分析道:“而且,中洲十大古派都是名门正派,比其他四域的几乎每一个超级势力,都要历史悠久。这些名门大派更需要秩序和稳定,更需要脸面和名誉。石磊肯定要报复我们的,但是却不会硬来,而是通过正道手段来打压我们。毕竟我们挂着仙鹤门的名头呢。”

黑楼兰点点头,认可方源的观点,她的脸上又浮现出一抹可惜之色:“遗憾的是,那株走肉树我们没有搞到手。它可是上古荒植,又是传说之物,卖价要比气罡飞天猪还要高。”

方源道:“我不这么认为,那个时候,比起走肉树,当然是星象福地更为重要得多了。”

“那么星象福地怎么分呢?”黑楼兰将美目投注到方源的身上,神色不禁严肃起来。涉及到巨大的利益,她当然不会好心到拱手相让。

这些荒兽、上古荒兽的尸体,容易分赃,因为可以估价。星象福地却是不好估价,因为仅用仙元石是无法表达出一块福地的真正价值的。

福地是一个聚宝盆,不仅能源源不断地产出资源,而且吞并之后,对蛊仙修为更有巨大增益。可以说,福地或者洞天,是一个蛊仙修行了一辈子的最大成果。

方源垂眉思索了一下,这才道:“星象福地的确不好瓜分,那就不妨这样,我们公平竞争。谁能达到地灵的条件,谁就获得星象福地。另外一方,必须接受结果,没有任何补偿。星象福地是我们两人夺取的,这场竞争只有我们两人可以参加,彼此之间不得恶意干扰。除去我们两人之外,哪怕是黎山仙子、太白云生要抢夺星象福地,我们都得合力制止。你看如何?”

黑楼兰目光一闪,点了点头:“如此甚好!”(未完待续……)

\end{this_body}

