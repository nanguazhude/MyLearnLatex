\newsection{赌斗凤金煌(中)}    %第二百零五节:赌斗凤金煌(中)

\begin{this_body}

“连续用了两个仙道杀招么……”

面具下,方源目光清洌如刀。[\&\#26825;\&\#33457;\&\#31958;\&\#23567;\&\#35828;\&\#32593;\&\#119;\&\#119;\&\#119;\&\#46;\&\#77;\&\#105;\&\#97;\&\#110;\&\#104;\&\#117;\&\#97;\&\#116;\&\#97;\&\#110;\&\#103;\&\#46;\&\#99;\&\#111;\&\#109;

但旋即,他又将目光重新投向手中的火焰。

“呼,好险啊!”场下,一位灵缘斋的女弟子吐出一口浊气。

“秦娟师姐,怎么危险了?我明明看到金煌大师姐连用了两个炼道杀招,一举赶超上来了啊。”坐在秦娟身边的另一位灵缘斋女弟子,有点婴儿肥的孙瑶,好奇地问道。

“孙瑶师妹,你要知道,这场赌斗和之前看到的不同,这是武斗!武斗中,可以让蛊师之间相互出手,干扰对方炼蛊。”秦娟耐心地解释道。

“啊,还能这样?我平常炼蛊的时候,就已经很艰难了。这要在炼蛊的时候,还要被干扰。天呐……”孙瑶捂住嘴,脸上一副不敢去想象的神情,“这武斗分明是折磨人,谁想出来的规则啊,也太刁钻了。”

“不。”秦娟却一脸赞同的样子,“抵抗干扰,也是炼道最重要的部分之一。我们在今后的修行中,也许就会碰到自己在炼蛊时,遭到敌人偷袭的情景。事实上,很多时候我们离开门派,外出完成门派任务时,就经常会因为各种原因蛊虫损耗,而不得不在野外炼新蛊应付局面。”

“同样的道理,我∴■长∴■风∴■文∴■学,■∽.n◇et们面对强敌,发现敌人正在炼蛊,我们也会选择偷袭强攻,乘敌不备,施展先手,打出立于自己的战斗局面。这就是武斗的现实意义。孙瑶师妹,你也知道。炼蛊的过程中,蛊师需要全神贯注。因为一个不小心。就会失败,受到反噬。但是在野外炼蛊。还需要留一些心神,关注自身的环境。”

“金煌大师姐,刚刚连用了两个炼道杀招,这就很冒险了。因为催动杀招的那一刻,心神自然要集中起来,转移到组构杀招的蛊虫身上去。若换做一般的蛊师,恐怕都会因为分心动用杀招,导致火焰失调,炼蛊失败。如果在这个时候。那方源魔头忽然出手,攻击金煌大师姐,恐怕会带给大师姐相当大的麻烦。”

听了秦娟师姐的解释,孙瑶这才恍然:“原来如此,这的确是相当的惊险啊。幸好那方源魔头没有出手。”

因为方源屡次在关键的时刻动用过血道炼蛊术,使得不待见他的灵缘斋弟子们,都私下称呼他为魔头。

不过在中洲十大古派中的魔头,其实很多。

比如仙鹤门中的三长老虎魔上人,早年前本就是魔道蛊仙。被仙鹤门招揽吸收。

这个方面做到极致的,反而是灵缘斋自己。

曾经的剑仙薄青,就是魔道中人,杀性十足。结果被灵缘斋那代的仙子墨瑶以情感化,将薄青吸收到灵缘斋中。

当今,白晴仙子和凤九歌的故事。也广为流传。凤九歌也是正儿八经的魔道出身。

中洲那么大,人杰地灵。每隔一段时间。总会有一些惊才艳艳的魔修、散修,横空出世。

中洲十大古派为了维护大局。保障自己的利益,将这些人物吸收进来,是维持局势的不二法门,上佳手段。

“没有攻过来吗?可惜了……”凤金煌心中暗叹一声,眼前燃烧着的两团火焰中的一朵,忽然间熄灭。

“怎么回事?”全场不由地掀起微微声澜。

“难道方源魔头已经出招,大师姐被他干扰,炼蛊失败了一半了吗?魔头太狡诈了,到底是什么攻击,我们居然都没看清楚。”孙瑶着急地叫道,言语间充满了对方源的厌恶。

“不,不对。主持这场赌斗的长老,没有宣布什么。这证明刚刚方源魔头并未出手。原来如此啊!”秦娟双眼一亮,忽然明白过来,“原来大师姐是做出连用两个杀招的假象,其实是一个陷阱,想引诱方源攻击她。方源一旦攻击,心神就会在一刹那间集中在自己的攻击手段上。如此一来,他就露出了破绽。大师姐反攻过去,得手的可能就很大了。”

“我晕,居然是这种情况?”孙瑶嫩嘟嘟的小嘴微微张开,惊叹于这场较量背后的智斗谋略。

秦娟脸上浮现出严肃之色:“刚刚我也被蒙在鼓里。如此一来,真相是大师姐设置了陷阱,方源那魔头却是看出来了,没有上当……孙瑶师妹,你好好看吧,这场比试将极为精彩。比试的双方,绝对是当今这一代最优秀的翘楚之争。”

“嗯!”孙瑶点头,一对大眼睛一瞬不瞬地盯住场上。

中洲十大古派一直都竞争激烈,不仅是蛊仙之争,还涉及到弟子的竞争。

这一代,中洲十大古派中最优秀的无疑是凤金煌。

但狐仙福地之争,方源突然出现,将凤金煌踩下山脚。其后种种内幕,孙瑶、秦娟这些弟子是接触不到的,就连仙鹤门自家的弟子,以及大部分的长老都以为,方源是仙鹤门暗中培养出来的天才弟子。

上一场凤金煌被方源击败,这一场赌斗就是凤金煌要找回场子,涉及到中洲十大古派最优秀的弟子之争。是灵缘斋、仙鹤门的隐形对撞。

所以场下的观众们,绝大多数都在探究一个问题究竟谁才是当代最优秀的年轻蛊师?

他们不知道的是:方源已经成为蛊仙,虽然只是仙僵,但早已不是他们能揣度的。凤金煌则拥有梦翼仙蛊,已经开始运用,并且尝到巨大的甜头。

外人只能看到最表面的东西,而比试的双方其实都对名誉之争,不大看重。凤金煌已经成熟了许多,明白了利益这两个字。方源就更不用说了。

比试在继续。

方源虽然看破了凤金煌的陷阱,没有上当。但凤金煌连用的两个杀招,其中的小如意手却是真的。并不是像“齐头并进”,那只是营造出的一种假象。

在这两只意志小手的辅助下,凤金煌的炼蛊速度仍旧超过方源一线。并且随着时间推移,双方的差距在逐渐拉大。

不过这个差距,并不明显。

方源保持着自己的节奏,并为动用任何的杀招,只是单纯以炼蛊手法,近乎完美地处理材料,操纵火焰。

炼蛊手法,不是炼道杀招。手法只是一种单纯的技巧,经验的总结。可以说是炼蛊的基本功。

“好强!就算是敌人,也不得不承认方源这个魔头,炼蛊的基本功真是太扎实了。就算是本派的一些长老,都做不到他这种程度。”秦娟目不转睛地关注良久,忽然叹息道。

“他掌握的炼蛊手法真的好多,目前为止,至少出现了三十多种。不过我觉得,还是咱们大师姐厉害。从一开始落后一点,后来就赶超上来了。现在不管方源魔头怎么追,都追不上。这个小手杀招真的好方便啊!”孙瑶羡慕道。

“不。”秦娟摇头,“大师姐这样的领先,也未必是好事。维持杀招是需要耗费真元的,同时也要损耗精神。反观方源魔头,只用基础的炼蛊手法,精神消耗少。长期下去,他的精神就会比大师姐更加饱满,对之后的炼蛊更有帮助。”

“啊,是这样!那大师姐不就危险了吗?秦娟师姐,听你这么一说,大师姐用了炼道杀招,本身还领先,反而处于劣势吗?虽然你说得很有道理,但怎么会这样?”孙瑶感到头晕。

“呵呵,孙瑶师妹,这就是炼蛊大会的精彩啊。你才刚刚开始修行,很多东西都接触不到,以后就慢慢了解了。”秦娟笑了笑,心中却是沉重。

“方源魔头真的太强大了。单凭炼蛊手法的连续使用,就紧紧咬住大师姐。大师姐即便用了炼道杀招,都拉不开差距!这样下去可不行啊……”

凤金煌感到了疲惫。

按道理,比试才开始不久,双方都在炼制一转蛊虫单窍火炭蛊的阶段。

依照凤金煌的能力,精神仍旧饱满,不应该感到疲惫才是。

但之前的估算是一回事,真正比试的时候,又是另外一回事。

场外的关注目光,凤金煌可以无视,她天资卓绝,又有蛊仙双亲,从小到大就沐浴在四面八方关注的目光中。可以说,凤金煌已经习惯了。

凤金煌的压力,主要来自于方源。

方源展现出扎实无比的基本功,仅凭炼蛊手法的连续使用,就咬住凤金煌,让她动用炼道杀招,都没有甩开方源。

方源之前的出色表现,强大的战绩,外界对他境界的高估,让凤金煌在比赛之前,就感到压力。

比试开始后,方源在后面不疾不徐的追赶,也让凤金煌感到压力。

这种压力,极大地加重了凤金煌的心神损耗。让凤金煌从内心深处,感到了丝丝的疲惫和不安。

“如此看来,我就算是有炼道准宗师的境界,也比不上方源的炼道造诣。怎么办才好?”凤金煌一边炼蛊,一边急速思索对策。

“现在我动用了炼道杀招,精神加速耗费。比试才刚刚开始,随着时间推移,方源在这方面的优势会更加明显。但若是我放弃使用炼道杀招,单凭炼蛊手法的对拼,也是比不过方源的。而我最大的优势,就是精心准备的大量炼道杀招啊!”(未完待续……)

\end{this_body}

