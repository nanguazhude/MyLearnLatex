\newsection{赌斗凤金煌(上)}    %第二百零四节:赌斗凤金煌(上)

\begin{this_body}

%1
“重获新生的法门?”方源双眼一眯,神情有了明显的变化。

%2
灵缘斋不愧是中洲十大古派之一,底蕴深厚,超乎想象。

%3
方源苦苦追寻的,摆脱仙僵身份,重获新生的法门,灵缘斋居然就有。

%4
此刻凤金煌提出这个当做赌资,简直是击中了方源的命门,由不得他拒绝!

%5
“呵呵呵。”方源笑了一声,收敛起激动的情绪,他看向凤金煌,“你说的这个方法,该不会是仙鹤门的夺舍法门吧?”

%6
“当然不是。”凤金煌淡淡一笑

%7
方源沉吟道:“你的这个赌资,的确让我心动。但是这样的赌资,和我的狐仙福地还是有差距的。”

%8
凤金煌道:“任何的事物,价值都是会变化的。方源,若是让你在狐仙福地和重获新生这两项中二选其一,你会选择哪个?”

%9
“当然是重获新生!”方源心中毫不犹豫,嘴上也毫不犹豫,“当然是狐仙福地。这是我最大的基业,就算我成了仙僵,也能靠此生存得很好。”

%10
凤金煌嘴角一扯,缓了口气,这才摇摇头道:“方源,你不会是如此短视之人。不过,我知道你不会拿狐仙福地做赌注,而我也不能强行要求你这样做。咱们谈点实际的。我若输了,便告诉你重获新生的法门。你放心,这个方法已经有成功的先例。你若输了,咱们就做一笔买卖。”

%11
方源看着凤金煌淡淡而笑,从容自信的样子,眯了眯样:“哦?什么买卖?”

%12
凤金煌便道:“你若输了,就和我灵缘斋达成交易。长期供给我派胆识蛊。价格可以商量,只要不离谱就行。但是你必须定期交货,量要足!你若缺少魂魄,我灵缘斋可以卖给你,价格比市面上还要低三成。但是这笔交易必须是最优先的。甚至还要优先于仙鹤门。你的货不管有多少,必须首先供应给我灵缘斋,然后再卖给其他方面。”

%13
方源听了这番话,不禁再次打量眼前的凤金煌,对眼前的这位少女真正刮目相看起来。

%14
没想到只是过了这么一段时间,对方就有如此程度的成长了。

%15
不愧是天才。

%16
不。应该说也是背后有能人,教育得好。

%17
凤金煌的这个要求,对于方源而言,实在太过宽容,甚至可以说是优渥。

%18
而胆识蛊对于凤金煌而言。也是急需之物。

%19
为什么呢?

%20
因为凤金煌动用梦翼仙蛊,探索梦境。就算有仙蛊辅助,魂魄方面也会疲惫,也会受伤。而胆识蛊便是天底下独一份的壮魂佳品。就连九转魔尊的幽魂魔尊,也曾经赞赏此物。

%21
凤金煌原本挑战方源,不过是少年心性,赌气之举。

%22
但此刻和方源谈赌资,却是着眼现在。实事求是,讲究实利。这才是成熟的举动。

%23
凤金煌若是要求方源,以狐仙福地或者荡魂山为赌注。方源肯定不会同意。

%24
而方源虽然挂着仙鹤门的名头,但他和仙鹤门之间的关系,十大古派都心知肚明。

%25
方源绝不会顾忌仙鹤门的名声,而不得不答应凤金煌赌斗。

%26
所以凤金煌就提出这个要求。

%27
对方源而言,将胆识蛊卖给灵缘斋,自然也有好处。

%28
第一个好处。就是扩大销售规模,能赚取更多的仙元石。谁会嫌弃自己的仙元石多呢?

%29
第二个好处。则是让方源和灵缘斋搭上关系。若是哪一天和仙鹤门闹翻,方源也好有个退路。

%30
事实上。方源之前答应凤金煌赌斗的要求,本意就是想和灵缘斋合作。

%31
凤金煌探索梦境次数多了,觉察出胆识蛊的重要价值,便想撬仙鹤门的墙角。而方源也乐意这样做,若真和灵缘斋达成交易,对于他处理和仙鹤门的关系,也是极为有利的。

%32
“不管是输是赢,这场赌斗都对我有利。反正赌斗的结果,和炼蛊大比并不相干。嗯?有点不对。凤金煌提出这个要求,也是要降低我的斗志啊。”

%33
方源忽然间灵光一闪,又明白了凤金煌此举的又一层用意。

%34
“那就开始吧,这一场比试,最终的胜利者只会是我!”方源眼中精芒暴涨,哈哈大笑。

%35
凤金煌目光一闪:“这可不一定。”

%36
赌资确定下来,双方又开始磋商此次赌斗的内容。

%37
凤金煌提出,以炼制浪迹江湖蛊为题目。浪迹江湖蛊乃是五转水道蛊虫,和三转的水迹蛊,四转的浪迹蛊,六转的浪迹天涯仙蛊是一个系列。

%38
方源当即否决。

%39
灵缘斋十分擅长水道蛊虫的培育豢养和运用,这点从近水楼台仙蛊屋就可看出。

%40
若是炼制浪迹江湖蛊,对凤金煌的优势就太大了一点。

%41
方源则提出炼制血走蛊。

%42
这是血道三转蛊,用于移动。

%43
凤金煌也直接否决。开玩笑,谁不知道方源擅长血道炼蛊术,要炼制血道蛊虫凤金煌可没有信心。

%44
双方磋商了片刻,终于决定以武斗的方式,炼制炎道五转蛊虫五窍火塔蛊。

%45
整场赌斗,不是私斗,而是和大会大比一样,是完全的公开斗。

%46
很快,同样的材料,出现在双方的面前。为了确保公平,双方必须按照给予的蛊方步骤,炼制蛊虫。

%47
哪一方能够率先成功地炼出五窍火塔蛊,哪一方便是获胜的一方。

%48
和之前的文斗不同,因为是武斗的关系,每到特定的阶段,炼蛊双方可以向对方施展一次攻击。

%49
在众人的注视之下,方源、凤金煌相对盘坐,各占据场地的一角,双方间距两百六十步的样子。

%50
比试开始,方源立即查看手中的炼蛊材料。

%51
他必须要做到对每一份材料。都十分了解。若是有些缺损残破的材料,必然会大大影响炼蛊。

%52
方源检查的很仔细,有时候大会方面,会故意提供一些有瑕疵的材料,甚至是做过手脚。将材料本身的瑕疵掩盖住的,让蛊师轻易之间无法察觉出来。

%53
若真的被蒙蔽,等到真正炼蛊的时候,就糟糕了。

%54
这项布置,很有现实意义。

%55
蛊师平素炼蛊,不可能总收集到完美的炼蛊材料。有时候因为材料存放的时间太久。或者运输中损耗,或者在战斗中抢夺,总会有一点的残损。或者在市场上买到以次充好的炼蛊材料,这都需要蛊师辨别。

%56
因而,查看炼蛊材料也是炼道造诣的基础。

%57
这一次炼制五窍火塔蛊。大会提供的材料多达上千种。没有一只蛊虫,都是花草石骨之流。

%58
方源检查完所有的材料,足足耗费了半个时辰。

%59
随后,他又查看大会提供的蛊方。

%60
一共五张蛊方,分别对应一转单窍火炭蛊、二转双窍火炉蛊、三转三窍火屋蛊、四转四窍火楼蛊,以及五转的五窍火塔蛊。

%61
这五种蛊虫,都是一个系列,一脉相承。

%62
要炼出五窍火塔蛊。首先得炼出数量众多的一转单窍火炭蛊。再以单窍火炭蛊为主要材料,炼制出双窍火炉蛊。从一转到二转,再从二转到三转。三转到四转,四转到五转,如此递进上去,最终炼出五窍火塔蛊。

%63
正因为步骤繁多,过程冗长,所以提供给比试双方的炼蛊材料。才多大上千种。

%64
当然,依照方源的智道手段。可以改良蛊方,节省整个炼蛊过程中的步骤。甚至可以直接以这些材料不断炼制。一步登天似的直接炼出五窍火塔蛊。

%65
不过方源稍微思索了一下,便放弃了改良蛊方的想法,而是决定一步步来。

%66
原因有很多,首先在大庭广众之下,他无法依靠智慧蛊。其次,推算改良蛊方消耗的时间很多,得不偿失。再者,这是武斗,双方可以出手五次,干扰彼此炼蛊。若是一步到位的蛊方,一旦被干扰,就前功尽弃了。还是一步步来的稳妥。

%67
方源在检查炼蛊材料的时候,凤金煌在查看蛊方。方源在查看蛊方的时候,凤金煌则在细心地检查炼蛊材料。

%68
双方顺序虽然不一致,但真正动手,开始炼制蛊虫的时间却几乎相同。

%69
“终于开始了。”

%70
“双方都花了大半个时辰的时间,来查看材料和蛊方,真沉得住气啊。”

%71
“这一场比试,关乎仙鹤门、灵缘斋的脸面,是两大古派的名誉之战!”

%72
场下观者们开始嘀咕起来。

%73
有的人支持凤金煌,有的人则看好方源。

%74
场下还形成两派,数目众多的灵缘斋弟子,还有仙鹤门的弟子,都在相互瞪眼较劲。仙鹤门的带队长老们,则脸色不大好看。方源将胆识蛊作为赌资,并没有汇报仙鹤门高层,独断专行,这让他们心情不快。

%75
起先,方源和凤金煌的炼蛊手法还是一模一样的。

%76
在这个过程中,方源依靠着娴熟的炼蛊手法,渐渐超越凤金煌那边,进展更快,慢慢拉开差距。

%77
毕竟方源有前世五百年的动手体验,凤金煌就算依靠仙蛊探索梦境,使得炼道境界暴涨,距离运用还得实践,缺乏锻炼。

%78
为了追赶上方源,凤金煌开始施展炼道杀招。

%79
齐头并进!

%80
她双手一抖,原本把持的一团火焰,陡然分成两半,形成两团火焰。

%81
炼道杀招小如意手!

%82
从凤金煌的脑海中飞出两只小手,全是意志构成,凝聚紧密,散发着玉的光泽。

%83
凤金煌的两手托住两团火焰,不断调控火焰猛烈和温度。而两只小如意手,则抓取材料不断投入火中去。

%84
“厉害!”

%85
“一连用了两个杀招。”

%86
“凤金煌是有备而来的,这下子速度要远远快过方源了。”

%87
场下响起一片嗡嗡的交流声。

\end{this_body}

