\newsection{天随人愿}    %第二百一十七节:天随人愿

\begin{this_body}

%1
轰隆隆……

%2
爆炸声已经持续了三天两夜。

%3
来自白海沙陀一行人的攻势,从开始就未停止过。

%4
无数的凡道杀招,夹杂其中的仙道杀招,屡屡轰击在仙蛊屋羽圣城上。

%5
羽圣城摇颤不止,城中军民已经死亡大半,绝望在蔓延。

%6
“失败已成定局,突围吧!”羽民蛊仙周中涩声提议道。

%7
“羽圣城已经被对方用手段定住,我们没有了这座仙蛊屋,如何突围?”另一位羽民蛊仙郑灵反问。

%8
羽圣城中有三位蛊仙,如今已有一位阵亡,只剩下七转蛊仙郑灵,六转蛊仙周中。

%9
“郑灵大人,事情危急,我不得不直说了。”周中一脸郑重的神色,“之前羽圣城之所以被困住,是因为大人您不愿舍弃这方天地,所以才让敌人有时间布置手段。现在大人你若再不壮士断腕,当机立断。恐怕你我性命都要交代在这里了!我不怕战死,但城中的数万无辜的族人,也要丧生。就算这群敌人绕了他们一命,也会将他们训练成奴隶,为那些人族做牛做马啊!”

%10
郑灵听了这番觐言,终于幡然醒悟。

%11
他望向周中,眼中闪烁着泪光:“你说得对,是我糊涂了!”

%12
周中劝慰道:“大人,你的决定我也能理解。毕竟如今是人族的天下,没有我们异人的生存空间。这片天地,是太古九天中绿天的碎片世界,与世隔绝,是世外桃源。轻易放弃不得。不过如今我们不放弃,也得放弃了。”

%13
郑灵皱起眉头:“如今羽圣城也被制住,我们要带着城中的子民逃出生天,只有一个办法。”

%14
周中立即表示支持:“没错,就用那个仙道杀招天随人愿吧!”

%15
于是半个时辰之后。仅剩下的羽民们都被召唤,集中在演武场周围。

%16
“我愿我族羽民得生存!”

%17
“我愿我族羽民保自由!”

%18
“我愿我族羽民有天地!”

%19
数万的羽民齐声长啸,在周中、郑灵的联手催动下,整个羽圣城开始闪烁出一股洁白的光。

%20
光芒起先只有五六尺的程度,但很快就节节攀高,达到七八丈的程度。

%21
“白海沙陀大人。对方似乎有大的行动。我们不进行破坏么?”女蛊仙唐嫣然问道。

%22
“无妨。这个情况我早已经料到。我要的只是这座仙蛊屋,他们是带不走的。”白海沙陀淡淡地解释一句,旋即又命令其他人,“攻势不要停,就维持现状即可。不要干扰对方。就让他们逃走罢。”

%23
北原,太白福地。

%24
天空湛蓝如水晶,大地荒野如黄石。

%25
在太白云生的陪同下,方源浏览着这片上佳的福地。

%26
太白福地,方源已经不是第一次来了。但由太白云生陪同,尚属首次。

%27
只有蛊仙将仙窍落在天地中时,蛊仙本体才会进入仙窍,主持局面。

%28
平时的时候。仙窍是蕴藏在蛊仙体内。等到将仙窍落到外界天地中时,仙窍则包含蛊仙。

%29
蛊仙和自己的仙窍,本是一体。在仙窍落在外界天地的这个时间段。蛊仙是无法脱离仙窍,走到外界去。

%30
太白福地本身就是上等福地,面积广阔,有七百余万亩。宙道资源极为丰厚,外界一天,这里便是三十三天。

%31
当然。太白福地落下之后,吸收外界天地气。和北原产生沟通联系。时间流速已经放缓,只有十二天。

%32
太白福地被太白云生经营得十分不错。

%33
不提漫空的浮球茶草。也不提茶草中的玉峰鸟群。在空中,已经有了大片的天魁云丛,云中生长着天花,翱翔着浮鹞。

%34
太白云生已经在开始着手建立第三项经济支柱。

%35
他的第一项经济来源,就是江山如故。为蛊仙修复福地,治理还原五域外界的天地。这是太白云生最大的进项,不仅可以增长财力,而且还能开拓人脉。基本上,可以和黎山仙子的山盟蛊相比了。

%36
太白云生的第二项经济来源,则是浮球茶草。每个季度,太白福地中会产出一笔规模庞大的浮球茶草。

%37
这些茶草,太白云生都卖到东海的市场里。由僵盟总部里面的鲨魔仙僵出面,太白云生已经挤进了那边的市场。不过这项买卖,竞争十分激烈,属于薄利多销。往往一个季度的收获,还比不上太白云生帮助其他蛊仙,修复仙窍的一次报酬。

%38
“我打算将运道凡蛊,当做第三项营生。浮鹞迁徙进来,已经有十多万的规模。浮鹞身上有寄生虫,大量的云道凡蛊就会在这些寄生的虫群里产生。不过接下来,还是要看这次的地灾,若是渡过顺利,并且为我福地增添云道道痕的话,那就最好了。”

%39
太白云生对方源阐述了他的期许。

%40
正因为如此,他的福地中的资源,都没有搬迁出去。

%41
就是为了“勾引”出相互关联的地灾,渡过地灾后,为太白福地增长云道道痕。

%42
福祸相生。

%43
之前方源的狐仙福地渡劫,基本上将所有的珍贵资源都搬迁出去,迎来了血道地灾,为狐仙福地增长了血道道痕。

%44
若方源不搬走这些资源,说不定地在就会改变,形成其他地灾。诸如利于幽火龙蟒的火道道痕,有益于长恨蛛的智道道痕,更倾向于龙鱼的水道道痕等等。

%45
这些道痕若是增添,对相关的资源大有好处。

%46
但此举也有弊端。

%47
地灾来临,这些宝贵的资源往往就会损失惨重。

%48
一句话,就是损失现在的利益,换取长久的利润。

%49
当然,天劫地灾并不一定会如人所愿。这只是一种可能。说不定方源没有搬走这些资源,仍旧是血海棠地灾。那样的话,他就损失惨重了。

%50
毕竟老天爷的意思,是无法揣摩通透的。

%51
现在太白云生没有搬走这些资源,就是打算勾出云道有关的地灾。

%52
他究竟能不能成功。方源也没有底气,只是安慰道:“这次渡劫,我已经将时济运仙道杀招,用在了你的身上。说不准,你就能得偿所愿了。”

%53
说起来,时济运这个仙道杀招。方源在这段时间里,也靠着智慧光晕,稍加改良了一番。

%54
方源现在,已经渐渐地体会到了智慧蛊的巨大作用。

%55
他虽然只能蹭用智慧光晕,但单凭这点。已经对他的修行产生极其巨大的推进力量!

%56
“嗯?”太白云生忽然有所感应,神色一动,“地灾来了。”

%57
整个太白云生开始微微的震动起来,大量的地气蓬勃而动,少量的天气垂挂而下。

%58
一股玄光陡然爆开,从玄光中传出万众一心的呼唤和祈祷之音。

%59
“我愿我族羽民得生存!”

%60
“我愿我族羽民保自由!”

%61
“我愿我族羽民有天地!”

%62
玄光消散,在福地的西部,陡然现出数万的羽民。

%63
“这是什么灾劫?”一时间。就连方源都有些愣神,他还从未见过这等奇异地灾!

%64
太白云生则陷入沉吟之中。

%65
“这里是什么地方?”羽民蛊仙周中连忙扫视周围,一脸警惕之色。

%66
动用了仙道杀招天随人愿之后。一股强大的力量就将羽圣城中的所有羽民,都挪移出去。

%67
留下一座空荡荡的羽圣城,白海沙陀眼光独到,没有费力气,就顺利地接收了这座仙蛊屋。

%68
看着方源和太白云生缓缓飞来,羽民蛊仙郑灵的神色骤变:“糟糕!这里是某个人族蛊仙的仙窍福地!我们被当做地灾。降临此处了。”

%69
周中的脸色也浮现出震惊之色:“怎么会这样?那我们岂不是仍旧要面对蛊仙敌人?”

%70
“对面只有两位六转,且其中一位还只是修为垫底的仙僵。你怕什么?”郑灵却是哈哈大笑。“杀了他们,我们就能夺得此地。天随人愿。这里虽然不如原先的绿天碎片小世界。但也足够我族生存繁衍了。动手!”

%71
郑灵说着,便飞射上来,距离方源、太白云生还有数千步的距离,他就双掌一推。

%72
呼!

%73
大风咆哮,一面厚实的风墙旋即形成,铺天盖地似的,朝方源、太白云生二人压去。

%74
这是仙道杀招,郑灵泄愤一击。他在白海沙陀面前找不回场子,一肚子的气就朝方源和太白云生身上发泄了。

%75
见到风墙压来,方源将太白云生拉到身后,发出一声不屑的冷哼:“雕虫小技尔。”

%76
万我第一式大手印!

%77
轰隆!

%78
力道大手印横冲直撞,将风墙撞得粉碎。并且余势不减,向着郑灵狠狠地碾压过去。

%79
郑灵、周中脸色剧变,看到方源的大手印,像是看见鬼似的。

%80
“什么?”

%81
“区区一位垫底的六转仙僵,居然有七转的战力!”

%82
两位羽民蛊仙顿时知道己方踢到了铁板,郑灵满嘴苦涩之意。若换做自己平常状态,也不惧方源这等对手。

%83
但现在,经过和白海沙陀等人激战,他战力下降得十分厉害。

%84
更别提六转的周中了,情况比他还要糟糕。

%85
“慢来,二位有话好说。在下刚刚莽撞出手,多有冒犯,还望海涵!”郑灵开口高喊,“我们是有苦衷的,进入贵福地,实属无奈。”

%86
他喊慢来,方源反而动作更快了。

%87
不仅如此,他还催出领悟五只力道巨手,群攻过去。

%88
“既然来了,就别想走了。”方源仰头大笑。

\end{this_body}

