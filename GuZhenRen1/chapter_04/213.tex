\newsection{炼蛊之后}    %第二百一十四节:炼蛊之后

\begin{this_body}

中洲,狐仙福地。[八零电子书wWw.80txt.COM]

仙蛊星光!

方源傲立在一座方源石巢的顶部,心念调动,星光仙蛊催起。

顿时一道星光巨柱,暴射而出,直贯进石巢当中。

在方源有意的操纵下,星光并无伤人之势,十分温和。

在这座石巢中,早已经做了许多蛊阵的布置。星光巨柱一射入其中,便被分化成一束束的湛蓝星光线,被接引到各层的第一号房间。

第一号房间中的毛民,轻车熟路地催动手中蛊虫,将射进来的星光吸纳,随后再进行处理。

处理的半成品,旋即被送入二号房间,在这里进行第二次加工。

一共经过七个房间之后,一只只的星念蛊终于炼成。

这些星念蛊又顺着管道,纷纷流入到石巢最底层的仓库当中。

方源催动一次星光,至少能让这座石巢中的毛民,不眠不休地炼蛊八九天!

用了星光仙蛊之后,星光就很巨量,方源已经不需要再像之前那样,先收购星光蛊,再以星光蛊为主材,炼出星念蛊。这样不仅省去了很多麻烦,而且成本也略微降低了一些。

虽然动用星光仙蛊,是需要耗费仙元的。但收购海量的星光蛊,也要被赚取一笔,收购成本其实挺高的。

改良后的星念蛊蛊方,也较为安全。炼蛊是个高风险的事情,不比战斗来得安全,很多毛民因为炼蛊而受伤,甚至丧失生命。

方源发了星光之后,却不离开,而是观察一番。

良久,他微微点头,心中较为满意。

现在的星念蛊方,已经被他改良过了,炼蛊时毛民的损失比之前大为减少。最关键的是,星念蛊的产量并不比之前差。

“但是改良已经到了极限,除非我今后星道境界提升,或者遇到更好的星念蛊蛊方。”

方源心知肚明,自己不是修的星道。星道的境界很可能一生就这样的水平了。或许以后探索梦境,能提升星道境界。

不过未来的事情,谁说得准呢?

“可惜这座石巢中,平均十个房间里,有六个是空余的。若是毛民充足的话,星念蛊的产量还能再提升一倍有余!”

方源心中暗暗可惜。txt小说下载80txt.com

这一次炼制变形仙蛊,他不仅耗去了从不败传承中得来的成功道痕,而且还有手头上几乎全部的仙材和仙元石。

导致现在的方源,堪称一穷二白。

因此毛民短缺的问题,暂时也解决不了。

他就算能抓取和猎捕野生的毛民,也要花费精力和时间训练他们,才能得到合格的毛民奴隶,才能驱使他们给自己炼蛊。

方源当然不知道如何将野生毛民训练成合格的奴隶。

最便捷的法子,就是从其他蛊仙手中收购毛民奴隶。遗憾的是,方源手中暂时没有余钱。

就连手中的仙元,都十分稀少,只有十余颗,要掐算着使用。

可以说,现在的方源处于十分虚弱的状态。就算他有万我杀招,又增添了三只星道蛊虫,得到了万象星君的几个星道仙级杀招。没有足够的仙元,这些杀招就无法催动。

有道是手中有粮,心中不慌。

对于蛊仙而言,所谓的资粮,最主要的就是仙元、仙元石了。

所以方源的当务之急,便是积攒仙元。

离开这座石巢,方源又接着巡视其他两座石巢。

在最近这段时间,这两座石巢都是由黑楼兰负责。

她催动仙蛊力气,为石巢中的无数毛民提供充沛的炼蛊材料。大量的气囊蛊被炼制出来后,装载胆识蛊,卖给中洲大小势力,卖到宝黄天中去。

这两座石巢中,基本上每一个房间,都坐着一位毛民。

炼制气囊蛊的风险,比炼制星念蛊还要高出很多。为什么石巢还能满员,这是因为方源将第三座石巢中的毛民,都抽调了一部分过来,填充这两座石巢。

方源现在的重点,是加大胆识蛊的贸易量。

在中洲炼蛊大会中,方源和凤金煌赌斗,当众定下协约。从今以后要贩卖给灵缘斋一笔庞大的胆识蛊。

所以胆识蛊需求更多,方源不得不加大气囊蛊的产出。

在一座石巢中,方源见到了黑楼兰。

黑楼兰正在阅览中洲的情报。

“中洲十大派秘密擒杀宋紫星的事情,你听说了么?”一见面,黑楼兰就问方源道。

方源淡淡一笑:“当然听说了。这个事情可以瞒得住凡人,却瞒不住蛊仙。就算中洲十大派尽力遮掩,但在中洲其他蛊仙绝不想此事默默无闻下去,如今他们都在看中洲十大古派的热闹呢。”

“唉!”黑楼兰深深地叹了口气,“如此良机,却被我们错过了。如果能趁机斩杀宋紫星,说不定星象福地就到手了。”

方源沉默了一下,这才道:“黑楼兰,你想得太美了。中洲不比北原,你我都是北原蛊仙,一旦动手气息流露,就会被人发觉。身份暴露的后果,必定严重到你我都无法承受。而且宋紫星此人十分狡猾,擅长逃窜,我们的侦察手段都不怎么样。中洲十大古派在真阳山脉里都未发现他的踪影,我们又能如何发现他呢?”

“你说的对。”黑楼兰又叹息一声,旋即闷闷不乐地回应道。

自从她被方源“打”醒之后,意识到了自己的不足。

这些天来,她日思夜想,就琢磨着如何经营,赚取资粮,支撑自己的修行,做到独立自主。

但可惜的是,这些营生都不容易。

黑楼兰要白手起家,从零做起,就要耗费许多时间,经历许多挫折,才能将项目建立起来。

但项目建立起来,就要面临市场竞争,究竟能赚多少,就看具体情况了。

黑楼兰考察了许多,又向询问黎山仙子询问请教,便知晓自己若真要从零做起发展项目,初期必定是要亏本的,中期难捱,后期若赚得少了,反而拖累自己。

没办法,她是十绝仙窍之一,时光流速很快,天劫地灾来临更猛。

黑楼兰调查之后,才发现:真正适合自己的,只有类似胆识蛊这种贸易,卖家市场,垄断生意,独一无二。又或者像方源在东方一族大本营中掠夺的几处经营项目。诸如幽火龙蟒、长恨蛛、龙鱼等,这些资源已经被东方一族经营得很好,方源只是接个手,选择卖到哪里罢了。

发现这点之后,黑楼兰对星象福地的觊觎就更深了。

只要得到星象福地,黑楼兰就能瞬间站稳脚跟!继承了万象星君的营生,她就能做到自给自足,推进本身的修行。

方源没有将自己已经成为星象福地主人的事情,告诉黑楼兰。

他喜欢留一手。

雪山盟约有时限,黑楼兰将来是敌是友还不好说。

当然,若是黑楼兰发觉这一点,主动问起方源。方源碍于雪山盟约不能说假话,虽然可以选择不答,但这样一来也会被黑楼兰轻易证实。

能瞒多久,就瞒多久。

这是方源的打算。

……

余木蠢炼成仙蛊之后,便一路跋涉,深入真阳山脉当中。

自从和本多一分别之后,他就动用了仙道杀招,遮掩了行迹。

这一天中午,他在一座不起眼的小山谷前,停下了脚步。

他取出一只信道蛊虫,随手抛入前方。

蛊虫乍然消失,像是石头没入水中,眼前空气荡漾丝丝涟漪,旋即涟漪扩大,撕开一道长长的口子,露出里面的血光。

余木蠢毫不犹豫,迈开脚步,从空间缝隙中踏了进去。

下一刻,他置身在一片陌生的空间里。

身后的空间缝隙旋即合闭,余木蠢眼前血光充斥视野,一片幽暗的毒气沼泽地,大量的尸骨遍及沼泽泥泞之中。

浓郁的血腥气味,扑鼻而来,余木蠢不禁皱了皱眉头,开口道:“动用了这么多的血祭,看来你的伤势很重啊,宋紫星。”

“咳咳咳,没办法,能逃得一命就已经算是幸事了。进来吧。”从沼泽深处,旋即传出宋紫星的声音。

看这个情形,身为毛民蛊仙的余木蠢,居然和中洲魔道蛊仙第一人的宋紫星有着很深的纠葛,双方十分熟稔,关系似乎极为亲近!

余木蠢轻车熟路,走进泥沼深处。这个地方,似乎他已经不是第一次来过,熟悉的很。

终于,在泥沼中央的血池中,他见到了宋紫星。

只剩下头颅和大半胸膛的宋紫星,原本全都没入血池中修养恢复。余木蠢到访,他这才从血水中探出脑袋来。

见到宋紫星,余木蠢脸色又沉重了一分:“你这样的伤势,恐怕难以参加接下来的大计了。”

“呵呵,无非是舍弃这身性命和修为罢了。”宋紫星神情淡然,语气也极为洒脱,他转移话题道,“那只律道仙蛊你炼出来了?”

“嗯,耗去了从不败传承中得来的成功道痕,再用我的自然炼蛊法,炼蛊成功是必然的。不过有点可惜了,若是给我充足的时间,我应该可以从这个成功道痕中研究出许多有价值的东西来。不败传承中的成功道痕,和普通成功道痕不同,极为贴近炼道。如果我能研究出它,我甚至可以自己布置不败传承!”

余木蠢摇摇头,叹息着。<!--80txt.com-ouoou-->

------------

\end{this_body}

