\newsection{见面似相识初成}    %第一百八十七节:见面似相识初成

\begin{this_body}

炼蛊大比的第二场结束之后,方源便回到狐仙福地。

福地中已经过去了一段时间,墨瑶意志在方源的布置下已经恢复了元气,可以再次搜意。

方源没有丝毫的怜悯之意,再次对墨瑶意志进行搜刮。

这一次,他获得了墨瑶掌握的不少炼蛊手法,甚至还有一记炼道杀招。

这股墨瑶假意掌握的记忆中,最多的就是有关炼蛊的内容。

在这方面,墨瑶是炼道宗师,方源不过是炼道准宗师。墨瑶又是灵缘斋的某代仙子,掌握的炼道手法,叫方源也有大开眼界之感。

方源学习之后,顿感受益匪浅。

寻常师傅教导徒弟,兴许还会藏私,留着底牌或者杀手锏,常言道:教会徒弟饿死师傅。

墨瑶意志如今就成了方源的“师傅”,还是那种毫无保留的倾囊相授。

方源收获很多,一边从墨瑶假意身上学习各种陌生的炼蛊手法,一边又复习自己掌握的东西。

他打算在炼蛊大会中展露头角,以全部的实力尽量争取最高的名次。

因为这关系到不败传承。

依照目前的局势,方源估算着,他在参加炼蛊大会期间还是较为安全的。但时?间一长,就不好说了。

八十八角真阳楼倒塌,这事情太大了,方源势单力孤,顶多是尽量拖延,要在北原、中洲大力侦查的情况下,将真相掩盖下去,这是不可能的事情。

若是不久后。就迎来五域大战,五域自顾不暇。方源还有可能浑水摸鱼过去。

但可惜的是,五域大战还在数百年之后。

中洲在十大古派的掌控之下。波澜不兴。北原大局,经过雪胡老祖和药皇、百足天君之战后,也趋于稳定。

“总有一天,我弄倒真阳楼的秘密,会被发现。到那时,中洲、北原都会通缉我,其余三域也会因为巨阳真传,四处抓捕我。狐仙福地再不是世外桃源,估计第一时间就会遭受强攻。所以在真相未必发现的这段时间内。是我最后的大力发展的时期。”

方源对自身情势,有着清晰的预判。

一旦真相曝光,方源就要展开四处逃亡的生活了。

所以,他必须尽可能地增长自己的实力,抓紧一切的机遇,没有机遇也要创造机遇。

眼下,中洲炼蛊大会就是一场巨大的机遇。

这才是第二场炼蛊比试,等到中后期,每一关的头名奖励就是仙元石、仙材、仙蛊方、仙道杀招。

若是最终大比。名次很靠前,就有资格得到不败传承的馈赠。

因此,就算惹人注目,高调一些。方源也顾不得了。

还有,他和凤金煌赌斗的消息,早已经流传出去。中洲十大古派的高层几乎都有耳闻。方源就算想低调,也低调不起来。

等到有一天。他弄倒真阳楼的真相披露出来,他的大名必定会名传五域。高调到极点!

到那时,必将是一场极其艰巨的考验。

整个天下都是敌人,无数蛊仙都要追杀方源,在他们看来,方源一定在巨阳真传中捞到了无数好处。

“必须趁着这个最后的发展时期,大量累积仙元石、仙材,尽全力提升战力,最关键的是摆脱仙僵之躯,重获新生!”

仙僵不能自产仙元,平时的修行中就十分拖累方源。看看那些僵盟中的大多数仙僵,任何一场战斗都要尽量避免,小心翼翼。

若在追杀逃亡的生涯中,方源还是仙僵,那压力就太大了。甚至会是导致败亡的主要因素。

于是在接下来的日子里,方源一边参加炼蛊大会,勤加练习炼蛊手法之外,另一边则积极推演,消耗大量的星念蛊,推算改良仙道杀招见面似相识。

这个杀招,是当务之急,也是改善局面的最好手段。其他事务,都比不上它,都被方源在那时放到了一旁。

第三场比试、第四场比试、第五场比试,方源都陆续通过。

他实力出众,准宗师的炼道境界,就算在蛊仙当中也是不常见的。五场比试,他连取五场第一,来势汹汹,声威赫赫。

方源的名号,已经开始四处传播。听过这个名字的,都知道这人是魔道蛊修,十分擅长炼蛊,炼蛊手法多变繁杂,功底相当深厚。

到了第六场比试的前一天,方源推算“见面似相识”仙道杀招,终于得到了第一阶段的成果。

“真的不容易啊!”方源从地下石窟中出来时,几乎要激动得流泪。

推算改良这个仙道杀招,过程分外困难。方源的智道、变化道境界太低,就像是小孩推石磨,步步维艰。

为了这个成果,方源在第四场比试的时候,就将从东方长凡那里得到的星念蛊全都消耗一空。好在他有石巢,有大量的毛民,早就在不断地炼蛊,囤积星念蛊。

方源这才有足够的资本,得到他可以使用的“见面似相识”。

之前也说过,仙蛊唯一,见面似相识的三大核心仙蛊,方源一只都没有(都在凤九歌手中)。

方源要重现见面似相识,只有换掉核心仙蛊,改变杀招。并且这个核心仙蛊,最好是方源手中就掌握的仙蛊。

如今,方源达到了这个目的。

推算改良后的“见面似相识”,比原版要寒碜很多,只有一只仙蛊为核心吃力仙蛊。

其他仙蛊,诸如定仙游、我力、铁冠鹰力蛊等等,都不适合。也可以说方源境界太低,无法利用上这些仙蛊。

方源能够将吃力仙蛊,当做核心,运作见面似相识,还是多亏了他力道宗师的雄浑境界。

以“吃力”仙蛊为核的“见面似相识”,效果也很有限。

杀招催动起来,并不能让方源改变容貌,只能让方源变换气息。

方源是北原蛊仙,一旦动手,气息泄露,身份也就随之暴露了。但若催动吃力核心的见面似相识,就能转变气息,将北原蛊仙气息变换成中洲、南疆、东海、西漠气息。若不是擅长侦察的蛊仙,很难发现这个伪装。

事实上,这就是残阳老君拥有的手段效果。

残阳老君帮助东方长凡,和北原一干魔仙大战,激烈非凡。但至始至终,他都表现出北原气息,没有流露出丝毫的中洲蛊仙气息。这显然是某一种仙道杀招,也是仙鹤门的深厚底蕴的体现。

不过方源草创的见面似相识,效果是要弱于残阳老君。

残阳老君面对大量的北原蛊仙,激烈对战,都没有暴露中洲气息。方源心知肚明,自己的这个见面似相识,是万万做不到这点的。

并且他的这个杀招,还有一个巨大的弊端,那就是使用代价太大。

每一次使用,都得损耗掉方源身上的数道力道道痕!

吃力仙蛊的效用,其实是让蛊仙吞食具有纯粹力道道痕的仙材,剥夺仙材上的天然力道道痕,将其增添到蛊仙自己的身上。

方源推算中尝到无数失败的苦果,最终苦思冥想,将主意打到自己的身上。

他身为力道仙僵,身上也是纯粹的力道道痕呐。至于他的仙窍中蕴含宇道道痕、宙道道痕等等,是不关肉身什么事的。再举个例子,狐仙福地中蕴含土道、宇道、宙道、奴道道痕,但狐仙生前的肉身,却是纯净的奴道道痕。狐仙死后,她身上的奴道道痕便都转嫁到福地中去了。

于是方源决定“吃”自己的肉。他以吃力仙蛊为核心,铺设大量凡蛊,不惜消耗力道道痕,最终做到伪装气息的程度。

尽管这个仙道杀招,效果不佳,远远达不到方源的需求,弊端也很大,但总归是能用的。

但对于能够得到这个成果,他还是十分庆幸。他一度以为,自己这次推算改良,将以彻底的失败告终。有这样的成果,他已然分外满意。

这其中最大的功劳,当然要归属智慧蛊。在智慧光晕下,方源灵感无限,居然硬生生的将力道仙蛊,当做核心,改造成功了见面似相识。

带着成功的喜悦,方源参加第六场比试,再次赢得第一。

回到狐仙福地,他稍作休息,便利用定仙游,赶往西漠。

西漠,怎渡丘。

狂风呼啸,黄沙漫天。

萧家的蛊师们,结成一个个的战阵,正和潜沙蛛群激烈交战。

这些潜沙蛛,一个个都有大象般的体积,浑身包裹着一层亮油油的甲壳,擅长在沙地表层穿梭。

一只只的潜沙蛛从沙地里冒出来,口器狰狞,扑向萧家的蛊师们。方源从高空鸟瞰,大略估算一下,这些潜沙蛛的数量,足有三万多头!

但萧家却是西漠的超级势力之一,萧家蛊师们战斗素养极高,又有常年对付潜沙蛛的经验。因此,整个战局反而是人类这一方,占据上风。

方源只是大略瞧了一眼,便将目光投向潜沙蛛群的身后。

那是一片巨大的沙丘。

沙丘之中,飓风卷席,昏天暗地,伴有无数的兽吼传出。

若定身仔细看,便可发现这些沙丘竟是在缓慢的移动!

\end{this_body}

