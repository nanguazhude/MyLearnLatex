\newsection{利大于弊时济运}    %第一百七十四节:利大于弊时济运

\begin{this_body}

%1
时济运杀招,虽然不用消耗自身寿命,又能在运道上有所增益,但却有两大弊端。

%2
第一大弊端,便是残魂反噬。因为仙道杀招的缘故,残魂十分凶恶,魂战凶险,偏偏又不能动用任何魂道手段,只能凭借自身魂魄硬磕。因此就算是魂道蛊仙,魂战之后,魂魄往往又会受伤惨重。

%3
第二大弊端,则是仙体经历时间,每一个月只能动用一次时济运。

%4
什么是仙体经历时间?

%5
福地、洞天和外界的时间流速不同。举个例子,外界一天,福地就过了十五天。蛊仙若在外界,仙体经历时间就是一天。若在福地中,仙体经历时间,则是十五天。

%6
蛊仙一生的寿命长短,至死活了多久,就是指仙体经历时间的总和。

%7
方源收起时运仙蛊,以及其他凡蛊,又下意识地抬头仰望高空。

%8
“可惜我没有察运仙蛊,不能观察到自身的气运变化,时济运杀招的效果到底如何,也就体会不到了。嗯?不对,我查看不到自己,但我可以查看其他人啊。”方源忽然灵机一动,想到了方法。

%9
他手头上有察运凡蛊,看不了自己这种仙僵,但别忘了,他已经和叶凡、韩立、洪易连运了。

%10
一经连运。四人的运气就像是四碗联通的水,水量保持着一致。

%11
方源这边的水忽然多了,立即就会平分四份,只留一份在自己的碗里,其余三份就流给了其他三人。

%12
方源观察不到自己。但可以观察其他三人。毕竟这三人,现在还都是凡人呢!

%13
南疆。

%14
茂密的山林中,叶凡宛若猿猴一般,在山地上迅速奔腾。

%15
他时而跳跃,跨过巨石,时而攀援,抓住长长的青藤。在高大的树木之间飞荡。

%16
他行动极其迅速。仿佛真的是猿猴附体。

%17
事实上,他的鼻翼、脸颊、手背等处,都长出了细细绒绒的猴毛。

%18
叶凡双眼也不是黑白的瞳孔,而变成金黄色。他紧盯着眼前的一点蓝光,目光渐显焦灼。

%19
“这蓝颜蛊还真是滑溜,如此难以抓捕。不行,我必须抓到它。它是变化道的三转蛊, 可以改变男性蛊师的容貌。我现在受到通缉追杀,这个蛊虫对我尤其重要!”

%20
忽然间,一阵奇痒袭上叶凡的内心深处。

%21
叶凡目光匆匆一瞥,看到手背上已经长满了浓厚的猴毛,心中顿时一沉:“糟糕!我强行催动杀招猿猴变,时间越长,猴毛就越多,过了时限,恐怕就要变成半人半猴的怪物了。可是我要解除这个杀招。速度必然暴降,只能任由这只野生的蓝颜蛊逃走了……”

%22
就在叶凡两难之际,天空迅速变暗,乌云覆盖苍穹,很快豆大的雨滴落下。十几个呼吸之后,形成倾盆暴雨。

%23
暴雨遮蔽视线,叶凡彻底丢失了勉强追踪的蓝光。

%24
“可恶。耗费了几天几夜的功夫,结果终究还是失败了。唉,没有专门抓捕野蛊的蛊虫,等若赤手空拳,真是难办啊!”

%25
叶凡置身在暴雨中,很快就被淋成了落汤鸡,他心中失望极了。

%26
就在这时,咔嚓一声,闪电劈下,正中山中大树。

%27
大树倒下,火焰升腾,迅速弥漫,竟然暴雨都扑灭不了。

%28
“这座桐油山上,到处都是桐油树,遇到火焰极易蔓延,难以扑灭。我得赶快离开这里,否则形成漫山大火,封锁山路,到那时再走就迟了。”

%29
叶凡连忙离开。

%30
一刻钟后,他站在山脚下,看着漫山的大火,以冲天之势熊熊燃烧,暴雨都浇灭不了。

%31
他叹了一口气,正要转身离开,忽然视野中一点蓝光摇摇晃晃,飞了过来。

%32
叶凡看清之后,眼珠子一瞪。

%33
竟然是那只蓝颜蛊!

%34
但这只蓝颜蛊上,有着浓郁的烟灰,显然受伤不轻。

%35
叶凡连忙赶过去,一把就抓住它。本来要炼化野生蛊虫,会受到蛊虫意志的顽强抵抗,但现在蛊虫虚弱不堪,反而叫叶凡在几个呼吸间,就将其炼化。

%36
“想不到我竟然这样得到了蓝颜蛊。”叶凡感慨不已,心中不由生出命运玄奇之感。

%37
这番惊喜,让他心满意足,正要离开此处,忽然他身体一僵,宛若石像般顿在原地。

%38
原来在蓝颜蛊之后,又有三只野生蛊虫,逃过火灾,奄奄一息地向他飞了过来……

%39
西漠。

%40
顽石滩上,大大小小的顽石,堆叠蔓延,形成方圆万里的天然石场。

%41
韩立大口大口地呼吸着,脚踩在乱石上,硌得生疼,仍旧疯狂的奔跑。

%42
“臭小子,把那三颗点头石给我放下!”在韩立身后,一位三转蛊师怒吼着,紧追不舍。

%43
韩立听了这话,跑得更快了。

%44
视野中,一堆高大的顽石群,越来越近。

%45
韩立的脸上,不由地涌现出喜色。

%46
这些顽石十分巨大粗壮,巨石之间缝隙或大或小,正是上佳的躲藏地点。

%47
追杀韩立的三转蛊师,则分外焦急。

%48
韩立只是少年,体型瘦弱,钻入缝隙中,很快就会脱离蛊师的视野。偏偏之前的激战,让三转蛊师体内的真元耗尽,就连三转移动蛊都催动不起来,因此只能徒步追捕。

%49
“该死,难道就让他就这么溜了?不行,我堂堂三转蛊师家老,对方不过是个凡人,让他溜了。这事情要传出去,我以后还怎么混?那三颗点头石,价值非凡,是炼制五转蛊的材料啊!这要卖出去的话……”

%50
三转蛊师心中万分不甘,狠狠咬牙。终于动用了底牌蛊虫。

%51
顿时他心中一空,寿命减少了两年,瞬间换来了两成的三转真元。

%52
他用一成的真元,催动移动蛊虫,速度暴涨,几个呼吸之后,就来到韩立的背后。

%53
他再用另一成真元。催发一道石拳。暴射出去。

%54
石拳下去,必定能将韩立击毙。

%55
但蛊师眼中精芒一闪,打消了这个想法。一方面他害怕打碎了韩立身上的点头石,另一方面蛊师恨极了韩立,想好好地虐杀他泄愤。

%56
于是这颗石拳,擦着韩立的右腿,直飞出去。撞击在高大粗壮的顽石群中,一连串的闷响,引得烟尘四起,碎石飞溅。

%57
韩立受到重创,右大腿直接骨折。

%58
但他心性坚韧,知道此时此刻性命悬于一线,怪叫一声,奋尽全力往前一扑。

%59
他体型瘦小,恰好钻近了石缝当中。

%60
三转蛊师心中大惊,他现在真元彻底耗尽。已经毫无办法,难道真叫眼前这个臭小子逃出生天去不成?

%61
但很快,蛊师又放下心来。

%62
原来韩立虽然跳进了石缝当中,但磕得头破血流,右大腿严重骨折,他头晕目眩,爬都爬不起来了。

%63
“臭小子。你还真能跑!跑啊,再跑啊!”三转蛊师满脸凶厉之色,慢慢踱步,逼近韩立。

%64
韩立心中绝望,下意识地向后蹭,但速度极慢。

%65
三转蛊师心中涌起无数折磨韩立的残酷手法,狞笑着钻进石缝。

%66
轰!

%67
忽然间,构成石缝的那几块高状的顽石崩塌下来,将三转蛊师一下子压在石头底下。

%68
三转蛊师的胸骨被压折,刺破内脏,挣扎了几下之后,当场死亡。

%69
留下韩立,楞在原处。

%70
中洲。

%71
众生书院。

%72
这一场的炼蛊大比,已经进行到最后的关口。

%73
十六位蛊师少年,盘坐在广场上,在众目睽睽之下,同时进行炼蛊。

%74
距离中洲炼蛊大会,已经越来越近了。众生书院虽然只是个小门派,但凡事中洲门派,都有参加炼蛊大会的名额。

%75
众生书院里,只有三个名额,分别留给门中的弟子、长老、客卿。

%76
现在,十六位蛊师弟子正在角逐唯一的名额。

%77
“这场比试已经快到关键时候了。”

%78
“不错,炼制红颜蛊有一道最后的关卡,考验蛊师心神的微妙操纵,手心必须合一,才能让火焰不会失控。”

%79
“看,曹宇已经先到了这关。他故意将炼蛊的速度放缓了,好让自己有充分的反应时间。”

%80
围观的弟子们小声谈论着,分析着场中的形势。

%81
片刻之后,又有两人步入最后关卡。

%82
“果然是谢兰和鲁文两人。”

%83
“他们两个和曹宇,算是我们书院此届公认的,在炼道上最后才华的三位精英弟子了。”

%84
“这场比试,最后的赢家,应该就在他们三人之中产生!”

%85
“也许还有黑马跳出来?”

%86
“呵呵,这不太可能。炼道的修行,需要大量的练习,绝非一蹴而就的。就算有人偷偷练习,在这个过程中也会耗费巨额的资源,采买炼蛊材料的过程就隐瞒不起来。”

%87
高台上,以院长为首,六位长老都坐着,观察着眼前的比试,杜绝任何作弊的行为。

%88
随着时间的流逝,又有第四位、第五位、第六位少年蛊师,纷纷踏入最后的关键一步。

%89
“基本上已经确定了,赢家就在曹宇、谢兰、鲁文三者之间产生。”

%90
“只要他们三个不出失误,就是前三名。其余蛊师虽然也陆续进入最后一步,但时间已经拉得太长了。”

%91
场外围观者们在议论,长老们也在暗中交流着。

%92
(小说《蛊真人》将在官方微信平台上有更多新鲜内容哦,同时还有100\%抽奖大礼送给大家!现在就开启微信,点击右上方“+”号“添加朋友”,搜索公众号“qdread”并关注,速度抓紧啦!)

\end{this_body}

