\newsection{愧疚的太白云生}    %第二节:愧疚的太白云生

\begin{this_body}

%1
修为停滞只是最主要的弊端,除此之外,还有其他小弊端。痛觉全无,是其一。思维僵化,是其二。

%2
这个世界上,智道早已经阐明人类思考的奥秘。

%3
人思考时,从脑海中产生一个个的念头。聪明人念头产生更快更多,蠢笨的人产生的念头则相对少而慢。

%4
这些念头相互碰撞、合并或者毁灭,最终得到一个或者几个全新的念头。这些新念头,就是思考的结果。

%5
脑海便是念头产生,生命思考的关键领域。而这个领域,是由两方面决定的。

%6
一个是**,一个是灵魂。

%7
人的灵魂如果寄居在野兽身上,那么这只“野兽”将会变得很聪明。这种聪明程度,远高于野兽,稍弱于人体本身。

%8
蛊是天地之精,人是万物之灵。人在万物中,是最聪明的。要达到这点,需要人的**,再搭配人的灵魂。

%9
现在方源的**已经彻底死亡,只有灵魂健在。因此脑海中,念头产生的数量,思考的速度就下降了一大截。

%10
剧烈的思考,会导致念头迅速减少。僵尸头脑生出念头较少,速度较慢,生出念头的速度跟不上消耗的速度。

%11
方源若是莽夫,也就罢了。偏偏他是个精于谋算,习惯思虑的野心家、阴谋家,转变成僵尸之后,他立即感到很不习惯,很不方便。

%12
“难怪大多数的僵尸,例如古月一代。会选择沉睡。沉睡的时候,思考得较少,念头损耗就少。头脑中就会慢慢积蓄更多的念头。等到战斗之时,就会剧烈思索,用到这些念头了。”方源心中升腾起一股明悟。

%13
这和巨阳意志选择沉睡,是一个道理。

%14
“思考得少了,就会显得笨拙、浅薄。想不到有一天,我古月方源也会变成一个笨蛋。呵呵。”方源心中自嘲了一句,收起泛滥的思绪。

%15
他对小狐仙道:“将我的师兄太白云生带进来吧。我要见见他。”

%16
小狐仙乖巧地答应一声,旋即消失在原地。

%17
它是地灵,在狐仙福地中可以随意挪移。

%18
十个呼吸之间。小狐仙便带着太白云生再次出现。

%19
“师弟,你……唉!这下可如何是好?”甫一看到方源,太白云生楞了一下,旋即双眼泛红。哽咽出声道。

%20
他对方源彻底变成六臂天尸的事实。已经有所了解。这也是方源之前特意关照小狐仙,可以给太白云生说的内容。

%21
方源哈哈一笑:“本来我晋升成力道蛊仙了,可惜现在变成了这个样子,没办法,还得继续叫你师兄。来!师兄,居所简陋,请随处找个石墩坐吧。”

%22
方源身处的这个山洞,正是当年白狐仙子在荡魂山中央挖出来。建设荡魂行宫的遗址。

%23
荡魂山被和稀泥仙蛊毁灭之后,方源在太白云生的帮助下。重新复原。

%24
复原之后的荡魂山,保留了这个山洞。

%25
但之前荡魂行宫中的金砖砌墙,银砖铺地,粉红帐幔,圆形大床,金花丝绸被褥,香炉风铃种种,却是一概皆无。

%26
六转仙蛊江山如故,能将山水恢复到一定时间之前的状态。荡魂行宫中的装饰家具,却非山水这个范畴里面。

%27
当然,方源也不想重现昔日的荡魂行宫。毕竟是白狐仙子的闺房,脂粉气息太重,不适合方源。

%28
山洞中没有家具,显得简陋无比。太白云生选择了最靠近方源的一张石墩坐下。

%29
现在,他的心中,对方源充满了感激、亲切,几乎是无条件的信任。

%30
皆因他和方源同生共死,北原一行经历了太多磨难,已见彼此间的赤诚之心。

%31
且不说在真传秘境中,方源两次帮助太白云生,毫不犹豫,眼睛都不眨一下。第一次挽回了江山如故仙蛊,第二次竟然抛弃人如故仙蛊,去挽救太白云生的性命。

%32
当时,太白云生差点感动得落下泪来。

%33
他有蛊仙传承,自然明白仙蛊对一个蛊仙的吸引力。方源抛弃仙蛊,去挽救他太白云生,足以证明方源的真心!

%34
后来,太白云生被黑楼兰俘虏。方源立即调转方向,首先抢回人如故,太白云生也十分认可方源的选择,心中大石落地,这是理智的决断。

%35
等到他苏醒时,他发现自己已然身处狐仙福地,脱离了危险。

%36
太白云生大喜,能侥幸捡的一命,自然是一件值得高兴的事情。但更让他惊喜的还在后头,见他苏醒,地灵小狐仙就将江山如故、人如故两大仙蛊,还给了他!

%37
太白云生生性仁慈软弱,对这两只仙蛊,有着深厚的感情。重宝失而复得,他当然开心不已。

%38
但等到他从小狐仙处打听到方源的近况时,他心头一震,高兴不起来了,心中充满了难过、愧疚、悲伤以及同情。

%39
因此他三番五次,要求见方源,企图尽自己全力救助自己的师弟。

%40
此刻他坐在石墩上,满脸哀愁,喟然长叹:“惭愧啊,被师弟你救回一条老命不说,见了面还要受师弟你的安慰、开解。”

%41
方源伸出一只手臂,拍拍太白云生的肩膀,用沙哑的嗓子朗笑道:“命运无常,十有八九不如意。人嘛,就要看得开些。我虽然成了僵尸,但至少还半死不活着,比死去的那些人,比巨阳意志要好得太多啦!而且在最后关头,还得了智慧蛊!已经赚大了,师兄你不必介怀,应该高兴才是。”

%42
方源离开之际,打开了星门。

%43
智慧蛊意识到这是通往外界的通道,求生的本能驱使它主动飞到了方源的面前。

%44
这点也在方源的意料当中。

%45
星门蛊是凡蛊,不能承担仙蛊之威。方源咬咬牙。将智慧蛊装入到自己的仙窍当中。

%46
饶是智慧蛊主动收敛了气息,方源的仙窍又便成死地,承受能力暴涨。通过星门蛊的时间不长,方源也差点吃不消。

%47
他回到狐仙福地的第一件事情,就是赶紧将智慧蛊放出来。

%48
狐仙福地也是六转仙窍,但种于天梯山,汲取中洲地气,十分稳固。和藏在蛊仙身中的仙窍,不可同日而语。

%49
关于智慧蛊的事情。太白云生也知道。

%50
“非常人行非常之事,师弟的本事为兄佩服得五体投地。但纵然是传说中的九转仙蛊,也治不好师弟你的僵尸体。不如让我出手。动用人如故仙蛊试试看!”太白云生情真意切,说到这里,已经主动站起身来,迫不及待。

%51
但方源劝阻了他。

%52
“师兄。你心底也十分清楚。这人如故仙蛊只能将人,还原一瞬之前的状态。虽有重生之能,但此刻时间已经过去这么久,不知多少亿的瞬间过去了,怎么可能将我还原回来?用了也是白用,何必浪费珍贵的青提仙元呢?”

%53
太白云生脸色灰败,情绪不稳。方源说完话,他忽然伸出手掌。狠狠地拍在自己的脸颊上!

%54
啪啪啪。

%55
连续五六声,太白云生竟然自己掌掴自己。

%56
“师兄。住手!你这是为何?”方源猝不及防也似,慌忙站起身来,伸出两只手臂,牢牢抓住太白云生的手。

%57
太白云生哪里及得上方源的力气,被禁锢之后,他却是痛哭流涕:“师弟,为兄对不住你啊,真的太对不住你了!”

%58
哭嚎着,膝盖一弯,居然要拜倒。

%59
方源连忙将他扶直,惊问:“师兄,何以至此?!”

%60
“师弟,若是我当初直接将人如故仙蛊借予你,你在关键时刻就能用得上,也就不会变成现在这个样子了!”太白云生痛哭流涕道。

%61
他生性仁厚,纵然曾经害过高扬、朱宰的性命,那也是人的求生本能。对于他而言,方源是他的救命恩人。又是他的小师弟,同属一个师门,但既是救命恩人,又是小师弟,却被自己的一时疏忽给害了。

%62
如果太白云生当时主动将蛊虫借给方源,也不至于方源落到如今的尴尬地步。

%63
太白云生苏醒之后,这个想法就一直萦绕在他的脑海当中,让他羞愧欲死,悔恨万分。

%64
此刻太白云生几乎瘫倒在地,伤心惭愧,全靠方源驾着他的两只胳膊。

%65
方源身高两丈,居高临下俯视着垂首痛哭的太白云生,他的眼底深处闪过一丝阴芒。

%66
“你能这么想,我很高兴呢……”他在心中一笑,口中却诚挚地道,“师兄,你不必如此难过。僵尸之体虽难复原,但我有很多细微。你可别忘了咱们有智慧蛊呢。”

%67
太白云生缓缓摇头,断断续续地道:“智、智慧蛊高达九转,它能跟你来到这里,完全是求生本能。师弟,你就算还是六转蛊仙,也万万炼化不了它,掌、掌控不住的!”

%68
“这点我当然明白。但就算如此,蹭一点它的智慧光晕,也能带给我莫大的好处。说起来,我这僵尸之躯,没有寿命,反而更适合接近智慧蛊呢!”方源伸出第三只手臂,轻轻地拍拍太白云生的后背,已示安慰。

%69
“再说,我还有墨瑶的意志。”

%70
“墨瑶?”太白云生疑惑。

%71
“这个是我在八十八角真阳楼中的另外收获,墨瑶可是曾经的炼道大师,中洲灵缘斋的某代仙子。”

%72
太白云生是地地道道的北原人,没有听过墨瑶这个名字,但是鼎鼎大名的中洲十大古派之一的灵缘斋,他还是听过的。

%73
“灵缘斋的某代仙子,又是炼道大师,看来这个墨瑶很不简单。”太白云生哀容稍缓。

%74
方源又笑一声,道:“还有最关键的,师兄你莫非忘了我们还有师傅呢。师傅他老人家,一定有办法。不瞒你说,师傅交给我探查八十八角真阳楼的任务后,留给我一只消耗蛊,专门用来通知汇报。两天前,我刚刚回到狐仙福地时。就已经将这蛊用了。”

%75
方源之前在八十八角真阳楼中,搜过太白云生的魂魄,知晓太白云生的一切经历和秘密。

%76
太白云生曾经遇到过一位神秘的老乞丐。从他手中得到了宙道蛊仙传承。

%77
方源就以此坑蒙拐骗,动用三寸不烂之舌,以假乱真的演技,在八十八角真阳楼中成功说服太白云生,让他相信自己和太白云生有同一个师傅,就是那个老乞丐。

%78
方源还随便给老乞丐起了一个紫山真君的名号。

%79
太白云生却对此深信不疑。

%80
现在方源旧事重提,太白云生的双眼顿时一亮:“原来师弟你有办法联系到师傅!这真是太好了。旁人没有法子,师傅他老人家神通广大,一定有办法的!”

%81
他心中燃起希望的火光。

%82
听他的语气。老乞丐在他的心目中,占据的份量极重,太白云生也对老乞丐充满了信心。

%83
如果此刻他知道真相,不知道会是什么表情。

%84
方源不露丝毫破绽。煞有介事地道:“师兄你稍安勿躁。我相信师傅来信,必然就在最近几天,我们耐心等候就是了。”

%85
太白云生点点头,方源祭出紫山真君的名头,终于让他心绪平复下来。

%86
他站直了身体,方源顺势将他手臂放开。

%87
太白云生后退一步,目光对上方源的赤红双眸,忽然举起手掌。抚摸在自己的心口,满脸肃容郑重其事地道:“师弟。你的恩情我铭记在心,永生不忘。我对长生天起誓,就算师傅没有办法,我太白云生竭尽毕生之力,也要将师弟你重新复活!”

%88
“哈哈哈!长生天乃是巨阳仙尊的洞天,我们刚刚拆了这老家伙的真阳楼,你对它发誓,未免太假了点吧!”方源仰头大笑,“师兄,我不会跟你客气的!咱们是自己人,我可对你有救命之恩呐。滴水之恩当涌泉相报,接下来你得好好报恩!”

%89
“你说。”太白云生立即回答,没有一丝犹豫。他早就决定要报恩,就算方源让他上刀山下火海,他也在所不辞!苏醒的这些天来,他的良心受到沉重的谴责,为方源付出、牺牲,会让他的良心好过一点。

%90
于是方源也肃容道:“这件事情,我早就耿耿于怀很久了。那就是咱们的辈分问题!我之前可是师傅的真传弟子,你现在升仙了,后来居上了,但我不服气!我本来也是力道蛊仙的,虽然现在成了僵尸,但早晚有一天我会变回去的。所以按照辈分排的话,我是师兄,你才是师弟啊。”

%91
“啊?”太白云生瞪眼,万万没料到方源郑重其事地让他报恩,结果却是这么一件小事。

%92
一股感动,从他的心中涌起。

%93
方源明显是不挟恩以报,但他太白云生岂是不知恩义的小人?

%94
太白云生心中感慨,伸出手掌,拍拍方源硬如山石的大腿,粗硬的腿毛还扎得他手掌有些疼。

%95
一丈等若地球上的三米三,方源身高两丈,就是一个六米过半的巨人。

%96
太白云生也只能拍到方源的大腿。

%97
他沉默了一下,然后抬起头来,望向方源的脸,谑笑道:“师弟啊,你想得到美!辈分这东西,怎么可以乱呢?除非师傅他老人家排序定位,否则我这个师兄,是当定了,哈哈哈!”

%98
太白云生也是一个骄傲之人,欠别人的恩情,怎么可以如此轻易偿还?

%99
“喂!”方源大叫一声,宣泄着自己的不满,“老白,你怎么是这种小人!你难道忘了刚刚说的话啦?我是绝对不会叫你师兄的!”

%100
“哈哈哈,师弟,我可以理解你,为兄不在意的。”太白云生笑着笑着,却是流出感动的泪水。

%101
多少年了,没有这样开怀大笑过。

%102
终于找到组织了。

%103
虽然和方源认识没多久,但他感觉很亲切,就好像是家里人一样……

%104
笑声回荡在山洞中,水晶山壁粉红光辉,此刻也显得温馨起来。

\end{this_body}

