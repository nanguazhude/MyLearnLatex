\newsection{终入地沟}    %第二百八十三节:终入地沟

\begin{this_body}

%1
地壳蜗牛生活在地沟的极深处,也只适合生存在那里。

%2
依照方源所说,要收集星夜黏涎,必须得深入地沟,寻找到地壳蜗牛,在在它身后即时作业,对你地壳蜗牛爬行之后,在地上残留的黏涎进行处理加工。

%3
加工完之后,还必须在很短的时间内,将星夜黏涎收集起来。

%4
这其中的难度,相当大,危险性也很高。

%5
地壳蜗牛,性情温顺,接近它并不困难,甚至可以站在它的背上,对它进行轻微的攻击,都不会让这种荒兽发怒。

%6
危险的地方在于,地沟中并不是只有地壳蜗牛一种生物。

%7
越是深入地沟,各种猛兽,诡异的植物,野生的蛊虫,就比比皆是。

%8
地沟深不可测,地壳蜗牛生存的那段地方,至少要距离地表上万里。在地壳蜗牛身边,生存着许许多多的强大猛兽,成群结队的荒兽,乃至上古荒兽都不少见。

%9
这些猛兽植株,才是最主要的威胁。

%10
“地沟我是一定要下的。既然焚天魔女大人要你们俩护卫我,那这次就看你们的了。你们死了,我也得活着,听明白了吗?”方源用一种理所当然的语气,命令道。

%11
雷雨楼主、玄阴医师二人恨得牙根发痒,却奈何方源不得。

%12
方源搬出焚天魔女出来,这两位仙僵还真的没有办法。

%13
两人商议了一番,玄阴医师阴沉着脸,对方源道:“兹事体大。我们要先向焚天魔女大人汇报,我们是坐不主的!”

%14
到头来。还是绕不过焚天魔女这道坎儿。

%15
不过,这也在方源的意料之中。

%16
于是。他一甩袖子,叫嚷起来:“那你们快去!”

%17
他并不怕焚天魔女的调查。

%18
事实上,星夜黏涎确有其事。

%19
这是方源近日来,改良了星念仙蛊方的成果之一。

%20
星夜黏涎为方源一手研发,用途广泛,潜力巨大。不仅能用在炼制星念仙蛊上面,还能用在其他星道仙蛊的炼制,是取材较为容易,但作用很大的六转仙材。

%21
“如果我有一个适宜地壳蜗牛生存的环境。大可以在这方面做文章,不断炼出星夜黏涎。这种仙材放到宝黄天中贩卖,独家一份,一定可以卖出大好价钱。短期之内的收益,足以媲美胆识蛊。”

%22
方源心中估量。

%23
但长期之后,星夜黏涎的加工手法,一定会被人勘破,甚至仿造。

%24
毕竟这种加工过的仙材,不比胆识蛊。在炼道、智道蛊仙的手中,十分容易就会被破解了。独家的垄断生意,做不长久。

%25
玄阴医师留下来,依旧陪伴在方源的身边。而雷雨楼主这个倒霉蛋。则怀着沉重不安的心情,前往阴流巨城某处的蛊阵,利用信道蛊阵。向焚天魔女传递消息去了。

%26
方源并不担心焚天魔女会阻挠。

%27
事实上,方源比较自信。相信焚天魔女会同意。

%28
毕竟炼成了星念仙蛊,对焚天魔女会大有帮助。方源还欠下焚天魔女一笔欠债。焚天魔女若不想方源陨落,说不定还会派遣其他仙僵来护卫。

%29
当然,护卫方源的仙僵,必定不会多。

%30
毕竟,现在北原僵盟分部的主要注意力,还在如何狙击大雪山福地外出的魔道蛊仙,阻挠他们收集炼蛊仙材,给雪胡老祖好看。

%31
“如果焚天魔女这样做了,更和我心意。毕竟深入地沟,多些护卫,也可省事安全。到了遗藏附近,再甩开他们。护卫人数不多,甩开他们也容易一些。”

%32
果然不出方源所料,最终焚天魔女同意了方源的想法,并且安排了五位仙僵,护卫方源。

%33
其中两位七转,三位六转,阵容强大,还要稍微出乎一些方源的估计,足见焚天魔女的诚意。

%34
三天之后,五位仙僵集齐。

%35
又等了两天,仙僵们筹借了一些仙蛊,专门应对此次任务。

%36
身为地头蛇,没有人比他们,对这道地沟更为熟悉了。

%37
做完充分的准备之后,一行人这才动身。

%38
这里值得一提的是,并不是北原才有地沟。事实上,五域都存在着地沟。

%39
大地裂开沟壑,有的地沟绵延千万里,有的深达数百万丈,深不见底,乃是蛊师世界中,最为雄伟壮观的自然奇观之一。

%40
僵盟这条地沟,历来被僵盟严格把关。焚天魔女不在北原的时候,阴六公、夜叉龙帅、黄泉翁这三位七转仙僵,至少会派遣一位,专门看守着地沟入口。

%41
地沟资源丰富,荒兽众多,当初阴流巨城之所以建造在这个地方,就是为了霸占地沟这条丰富的资源点。

%42
众所周知,北原的修行资源相较于其余四域而言,是贫瘠的。但地沟深入地下,空间广阔,资源十分丰富。唯一的弊端,就是里面的荒兽、荒植十分的凶猛勇悍,数量也层出不穷。

%43
一般能掌握地沟这种资源点的,皆是超级势力。

%44
北原僵盟分部如此,中洲有个驱邪派也是如此。驱邪派乃是超级势力,高层是三位蛊仙坐镇,掌控的地沟名为安祖地沟,在中洲十分闻名。

%45
傍晚时分,方源等人,穿越入口,深入地沟。

%46
方源筹谋已久,自然不要偷偷摸摸的潜行。他身边环绕着五位仙僵,可谓大摇大摆,光明正大。

%47
巨沟宛若恶兽的血盆大口,在浓郁的黑暗中静默地张开。

%48
方源等人投身其中,沟壑渐渐扩大,数十倍,上百倍。

%49
往下飞行了半柱香的时间,地沟内部已经广阔到,可以装下七八十个阴流巨城的地步。

%50
深幽黑暗,只有微光闪烁。

%51
两边的悬崖峭壁上,大部分都是光秃陡峭。只有些微的针刺植株,像是芝麻一般点缀在上面。

%52
这道地沟有多深。没有人知道。

%53
就算是掌控这里这么多年的仙僵们,也从未有人探查到地沟的真正底部。

%54
地沟中虽然资源丰富。但荒兽、上古荒兽数量之多,骇然听闻。北原僵盟分部,没有更多的实力进取,只能尽最大努力,掌控距离地表最近的五万丈的区域。

%55
在当今的天下,人族虽然已经屹立已久,是当之无愧的种族霸主。但是并不能涉足大自然的任何角落。

%56
和大自然的雄伟壮阔一比,人族也显出渺小。

%57
天地的奥秘,堪称无穷无尽。深藏不露。就算强如仙尊魔尊者,也不过在一两道上,洞悉到了天地规则的极致。

%58
人族距离穷尽天地奥秘,还有一段遥不可及的路程。

%59
一行人渐渐深入,地沟呈现出巨大的空间,反衬出方源等人身形的渺小。

%60
到了这里,已经距离地表,有三万丈的距离。山峦都填补不尽这里的地沟,无尽的黑暗。悄然吞噬一切。

%61
一干人等都很沉默。

%62
方源不动声色打量周围,用眼角余光,暗暗观察同行的仙僵。

%63
五位仙僵护卫,两位七转。三位六转。

%64
其中七转仙僵,分别是夜叉龙帅,鼋姥姥。

%65
三位六转中。包括之前的玄阴医师和雷雨楼主, 还有一位奴道仙僵。名为林大鸟。

%66
之前五天,方源早就刺探了他们一些情报。对他们并非一无所知。

%67
两位七转仙僵当中,首推夜叉龙帅。

%68
他是阴流巨城三大副首领之一,也是此行中战力最强的那个。值得一提的是,他能够依靠自家仙窍豢养大量的夜叉章鱼,夜叉龙帅的名头部分来源于此。

%69
仙僵的仙窍早已死亡,夜叉龙帅却仍旧有方法,豢养得住大量的夜叉章鱼,这有些让人费解,也让无数仙僵项目。

%70
另一位七转仙僵鼋姥姥,擅长防御。

%71
玄阴医师,从称号中就可看出,他是治疗蛊仙。他当然不能和太白云生相比,但对于治疗仙僵,却很有一手。

%72
雷雨楼主,擅长进攻,并且攻势浩大,气度磅礴。

%73
最后一位仙僵林大鸟,则控制飞禽,是一位奴道仙僵。

%74
从中就可看出,焚天魔女安排这些人护卫方源,是花了心思的。

%75
鼋姥姥可以护卫方源,玄阴医师能在关键时刻,为方源恢复伤势,拯救方源性命。万一遭遇荒兽群等,众仙不敌之时,就可让林大鸟出手,牺牲麾下鸟禽,为众仙断后,争取到足够的撤退时间。

%76
越是深入地下,地沟就越是宽阔。

%77
两边悬崖峭壁,中间是巨大的空间,鸟类能很好的在这里发挥。

%78
此时,一只只飞鸟,时不时从林大鸟的仙窍中飞出来,不断飞往四面八方。

%79
奴道就是方便,自从深入地沟,林大鸟便担负起侦察的责任。

%80
“遗藏就在那个方向,但是却不可直接前往,目的性太明确,会惹来怀疑的。”方源不动声色,耐心十足。

%81
又下了一段路程,林大鸟忽道:“有一头夜叉章鱼,在左前方,不过受了伤。”

%82
夜叉龙帅面色不变:“不用管它,我们绕道而行。”

%83
方源这六人,阵容颇强,一头六转的夜叉章鱼肯定比不上。

%84
不过此行,深入地底,寻觅到地壳蜗牛,采集仙材,才是真正的目的。

%85
到那时,战斗必定十分频繁,此时能节约仙元,便要多节约一分。

%86
地壳蜗牛在地沟的极深之处,爬行的速度似缓实快。居无定所,四海为家,踪迹难以探查。

%87
越是强大的猛兽,便具有越加顽固的领地意识。

%88
地沟的极深处,各种猛兽植株,盘踞一方,地盘林立。

%89
地壳蜗牛,却是一种罕见的可以跨越领地,而不引起领主们敌视和战斗的荒兽。

%90
也许是地壳蜗牛温顺的性情,已经让其他的强大生物,认识到它的无害了。

%91
但是方源等人,显然享受不到地壳蜗牛这样好的待遇。

%92
所以,战斗绝对是少不了的。

\end{this_body}

