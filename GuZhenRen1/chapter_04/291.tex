\newsection{方源谋划}    %第二百九十二节:方源谋划

\begin{this_body}

%1
既然得到了生死仙窍重生法门,方源自然想要做到最好。

%2
不管是前世今生,一直以来,修行资质都是方源心中的一抹遗憾。如今能将这个遗憾填补,方源乐意之极。

%3
更关键的是,十绝体的生死仙窍,实在是令人心动,辅助修行的优势太大了。

%4
然而,要取得大力真武仙僵,却是异常困难。

%5
首先十绝体,本身就极为罕见。就算出现,也十分不容易修行。

%6
修行到蛊仙层次的大力真武体,更少之又少。

%7
而在这个结果上,在转成仙僵,躯壳还遗留当代的,根本没有!

%8
方源查阅典籍,搜刮记忆,没有任何的线索。

%9
所以,黑楼兰就是他目前最容易得手,也最为实际的合适目标。

%10
也可以说是,唯一的目标了。

%11
然而,横亘在方源面前的麻烦,一点都不少。

%12
首先,黑楼兰本身就是枭雄人物,文武双全,在前世历史上都留下了浓墨重彩的一笔。这样的人物显然不好对付,能拼命也能算计,不是那么容易陷害的。

%13
其次,黑楼兰并非孤家寡人,她的身边还有黎山仙子这样的七转助力。她们俩感情深厚,根本分化不得。方源要对付黑楼兰,必然要解决黎山仙子。

%14
再其次,方源和黑楼兰、黎山仙子之间,还签订了盟约。

%15
最后,也是方源最苦恼的事情,黑楼兰知道王庭福地的真相。一旦处理不好。她就会将这个真相广为流传。到那时,方源就会被全天下的蛊仙追杀。巨阳仙尊遗藏的吸引力。足以让无数贪婪的蛊仙前仆后继,形成潮水海啸。淹没方源。

%16
总结而言,黑楼兰就像是一个刺猬,轻易动手,不仅伤不了她,还会被她轻易反噬。

%17
“黑楼兰啊,黑楼兰,若非你是大力真武体,能助我修行,我也不想打你的主意。”方源心中叹息。

%18
在接下来的日子里。方源做两手准备。

%19
一方面,他苦思冥想,想要设计出一个妥善的方法,来陷害黑楼兰。

%20
另一方面,他也在炼制星念仙蛊,通过信笺和黎山仙子交流,探讨如何违约。

%21
可是日子一天天过去,不管哪方面,都是进展缓慢。

%22
黑楼兰自从明白自己的缺陷之后。就沉下心来,重点经营自家仙窍。大雪山福地虽然不能再隐藏,她就隐匿在外,行踪不明。

%23
而方源回到阴流巨城后。又装模作样地下去地沟,将星夜黏涎采集全了。之后,的确采用了星夜黏涎当做辅料。炼制星念仙蛊,结果再一次失败。

%24
黎山仙子早年修行土道。无弹窗,最喜欢这种网站了,一定要好评]而后主修信道,兼修木道。她有山盟仙蛊,信道造诣肯定高于方源。之前她主动对付东方一族,算是违背了当初的契约,但她至今仍旧活得很好,可见她不仅擅长建立盟约,违约方面,也十分擅长。

%25
协约这种东西,就是信道的专长。打个比方,就好似一把锁,制约双方的枷锁。

%26
黎山仙子就仿佛一个锁匠,既然擅长打锁,必然也擅长解锁。

%27
方源和黎山仙子、黑楼兰之间,虽然建立了盟约,但时限将至。所以这道盟约并非大麻烦。

%28
方源向黎山仙子求教的理由是他需要解决加入僵盟的盟约。

%29
方源在东海,加入了僵盟总部,因此被僵盟盟约束缚。

%30
“如果黑楼兰解决不掉,那么我就只有退而求其次,收集大量的力道仙僵,成就上等福地。这个时候,就需要解除僵盟盟约,方便我在僵盟墓穴中盗尸。”

%31
谨慎如方源,自然不会把鸡蛋都放在一个篮子里,他总会做两手准备。

%32
于是这些天,方源经常向黎山仙子请教信道的手段。

%33
不过,收效甚微。

%34
一来,黎山仙子防范着方源。方源发展的太过于迅猛,早就让这位大雪山福地的三当家,心生忌惮了。尽管方源出价甚高,但她总是以奉命采集仙材为由,多番推脱。

%35
二来,方源对信道了解泛泛,造诣很浅,短时间内的确难以学到什么东西。

%36
三来,僵盟的盟约,岂是易于?指望方源学习一段时间,就能解决它,这根本是妄想!

%37
虽然没有在信道方面,有所进展,但通过和黎山仙子的交流,方源再结合北原僵盟这边的情况,倒是对僵盟和大雪山福地之间的斗争情况,有着清晰的了解。

%38
焚天魔女的战略的确是相当明智的。

%39
这些天来,她率领仙僵们四处出击,不断狙击大雪山福地的魔道蛊仙们。

%40
很多的魔道蛊仙,本来就运气不佳,采集珍稀仙材十分不易。如今再加上仙僵们捣乱,可谓雪上加霜,屡遭挫折。

%41
雪胡老祖无可奈何。

%42
原因无他,他虽是八转大能,却只身一人。

%43
焚天魔女一方,却有三位八转仙僵。

%44
若是焚天魔女携众强攻大雪山福地,依照雪胡老祖的战力,鹿死谁手尚未可知。

%45
但焚天魔女避重就轻,紧紧扣住雪胡老祖的弱点,这就让雪胡老祖的日子很难过了。

%46
他若是轻易出击,大雪山福地势必空虚。

%47
留守的蛊仙,修为最高,不过七转。二当家万寿娘子擅长炼道,三当家黎山仙子主修信道,战力都不高。

%48
若是雪胡老祖出击,被同级强者纠缠住,然后大雪山福地老巢被另外的八转仙僵抄了,那就要笑掉北原正道蛊仙们的大牙了!

%49
尤其是焚天魔女。

%50
这家伙底线甚低,当年连九天碎片世界,就能焚毁,破罐子破摔。

%51
大雪山福地不过是拼凑出来的,防御力还较正统福地更低一筹。雪胡老祖不得不坐镇防备。

%52
因此雪胡老祖虽然战力极强,但和僵盟交锋的过程中。一直处于下风,从未抬过头来。

%53
战力强并不代表一切。

%54
焚天魔女的老辣和算计。让雪胡老祖一直在吃闷亏,却无法做出有效的还击。

%55
方源别无他法,只好耐下心来,将主要的精力投放在炼制奋力蛊、失望蛊上面。

%56
他建立的方源石巢,已有一些规模。

%57
大量的毛民奴隶勤炼不辍,奋力蛊、失望蛊的数量迅速增长,但距离标准还有遥远距离。

%58
方源虽然得了那些珍稀的八转仙材,却没有急着出手贩卖,都保留下来。

%59
炼制星念仙蛊。有焚天魔女资助。方源每月盈利,除了用作炼制蛊虫外,主要都投放在喂养仙蛊,以及建设星象福地方面。

%60
他采购了大量的云土,命令星象地灵一步步建设。

%61
他打算在星象福地中,建设出一套立体循环的河流,充分利用空间,培育星屑草。

%62
这个前景非常广阔。

%63
如果达成,收益不下于胆识蛊买卖。

%64
但先期投入巨大。方源只能一步步来。

%65
他要对付黑楼兰,极有可能犯罪事实就会暴露。毕竟黑楼兰要将真相暴露出去,实在太简单了。虽然有盟约,但她要单方面违约的话。可是有黎山仙子帮助的,比方源优势大得多。

%66
一直以来,方源心中都怀有焦虑。

%67
中洲蛊仙的调查。不可能没有进展。

%68
也许明天,中洲蛊仙们就会围攻狐仙福地了。

%69
狐仙福地绝不可能守住。一旦丢失,星象福地便是方源的退路。

%70
狡兔三窟。更何况方源老魔?

%71
这一日,方源正催动炼道蛊阵,处理着某个仙材。

%72
仙僵死窍中,推杯换盏蛊蓦地发动,带来黎山仙子那边的来信。

%73
方源暗感奇怪,这封信竟是黎山仙子主动联系他的。

%74
展开一看,却是一封求援信。

%75
原来,黎山仙子竟是发现了黑城的行踪。

%76
黑城是黑楼兰的亲身父亲,也是黑楼兰为母报仇,必杀的复仇对象。

%77
黎山仙子来信,便是邀约方源一齐出力,斩杀黑城!

%78
黑城的处境很不妙。

%79
他身为超级势力黑家的当权者,却神秘失踪多时(参与百日大战),再次出现的时候,狼狈不堪,甚至遭受着来自黑家的秘密追杀。

%80
原来黑城为了保命,不惜舍弃了六转仙蛊屋黑牢。

%81
这可是黑家的实力象征,也是黑家的底牌,居然被他弄丢了。

%82
饶是黑城平时位高权重,这种情况下也要遭受本家的制裁!

%83
仙蛊屋黑牢可不是他的,组成的核心仙蛊分别来源于黑家四位太上长老。中洲蛊仙将黑牢瓜分后,将仙蛊带回门派,已有一只仙蛊被中洲蛊仙强行炼化。

%84
因此,黑家的太上长老们这才知道,黑城犯下了大错,需要严惩!

%85
黑城自然不想伏罪。

%86
他参加影宗的行动时,就和影宗方面订下盟约,不能向外泄露任何影宗的情报。

%87
这道盟约具有极强的约束力!

%88
黑城根本无法向本族解释,只得奔逃。

%89
黑家当然不肯善罢甘休,仙蛊屋事关重大,具有战略威慑的能力,必须要追讨回来。

%90
而线索就在黑城身上。

%91
所以,黑家派遣了蛊仙,出来追捕黑城。

%92
双方交战了数次,黑城惊险逃脱。动静渐大,被黎山仙子探知。她和黑楼兰一商议,都觉得黑城众叛亲离,是报仇雪恨的最佳时机。

%93
要杀一位蛊仙,自然要人多势众,更加保险。

%94
所以,黎山仙子立即来联系方源,算是将快到期限的盟约,最后大大地利用一把!

%95
不谈黎山仙子提出的报酬,的确优渥,方源想到自己不可告人的目的,或许眼前就是个良机。

%96
黑楼兰平素隐居不出,唯有激战混乱中,方有适合方源出手的机会。

%97
ps:祝大家中秋节快乐!不知不觉间,本书已经突破300万字了,回首一望,字里行间中都是逝去的青春,岁月的光辉,对生活的感悟和感动,激情和平淡。

%98
就这样,一起走下吧……

\end{this_body}

