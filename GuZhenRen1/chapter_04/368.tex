\newsection{深深阴谋}    %第三百六十九节:深深阴谋

\begin{this_body}

又是一片较大的画幕。

“绿,快走吧!”黄死死拽着绿的手臂,将他往外拖。

“不,我不走!还有救的,你看那个十尊仙胎蛊,还有气息残存!”绿大叫,挣扎不已,“放开我,那可是我毕生的心血啊!”

“你以为这场爆炸,是纯粹的意外吗?你看这些残留下来的八转仙材,你好好看看!”黄怒喝道。

绿浑身一震,眼中闪过清明之色,他认出来了,旋即满头冷汗,心有余悸地道:“这些仙材之中,竟然都充斥天意。好险,刚刚那一刻,我居然被天意影响了神智。就算布置了如此大阵,躲到了北原地沟之中,也被天意察觉了吗?”

“只有完整的天外之魔,才不会被天意察觉。我们被察觉,没有什么好气馁的。放弃吧,我们从头再来。”黄宽慰道。

绿无言,凝视着一片沙地,满脸沉重地点头。

……

“我已经告诉过你了,你的推断完全错误,红莲真传的事情和我毫不相干。”蛊仙大荔无奈地道。

对面的宙道蛊仙,面目都笼罩在一层雾中,看不分明。唯有额头的红莲刻痕,鲜艳欲滴。

他(她)冷笑一声:“我以红莲真传中的宙道手段推算,怎可能失误?红莲魔尊的下一道真传,就落在你的身上。”

大荔沉默片刻,终究道:“那还有什么好说的,战吧!”

说完,反迎上去。

“你终于不躲了!”神秘蛊仙同样战意昂扬。

激战良久。

蛊仙大荔不敌对手,被仙蛊屋惊鸿乱斗台镇压。

……

万劫太清空宇,终于结束。

魔尊幽魂毫无抵抗,被万千青鸟削得满身伤痕,整个体型竟被削减了三分之一。

与此同时,监天塔蠢蠢欲动,似乎也有挣脱虚化状态出来的趋势。

弥漫周围的灰色雾气,不知不觉间。已经消散了大半。

这表明:万劫灰忆,也过了大半。

但天空中的灾劫,还在继续!

又一波万劫,正在魔尊幽魂的头顶上蓄势。

似乎只要魔尊幽魂不灭。天意就不会放过他!

万劫阴油毒。

从天空中,垂下一滴滴粘稠的剧毒油浆。

这些油浆似乎滚烫无比,滴在魔尊幽魂的身上,立即嗤嗤有声,冒出紫嫣的气雾。

油浆从之前的伤口中渗透、流入。魔尊幽魂上被油浆沾染侵蚀的地方,都发生了剧烈的腐烂现象。

很快,这些剧毒油浆,从一滴滴变成一连串,一连串又变成无数缕,从高空中垂落下来,笼罩方圆数百里。

魔尊幽魂庞大的身躯,沐浴在剧毒油浆之中,以肉眼可见的速度被侵蚀腐烂,体格迅速萎缩下来。

但他始终不见任何动作。

底下的影无邪。满脸焦躁之色。

他想高喊,但奈何身中浩劫地陷。任何的动作,甚至是激动的想法,都会让他越陷越深。

“怎么搞的,你们两个都成这样子!”他低声嘀咕,又瞥向身旁的仙僵薄青。

从刚刚开始,仙僵薄青就沉默得仿佛一尊石像,一动不动,似乎哀莫大于心死,对外界的一切都无动于衷。

所幸。他们在魔尊幽魂的脚下。

被魔尊幽魂高耸入云的身躯遮挡,万劫目前还没有对影无邪、仙僵薄青造成什么危害。

数十万里之遥。

黑楼兰、太白云生相互扶助,极力奔逃。

在他们身后,一群南疆蛊仙追杀着。

五域蛊仙的本土意识。都相当强烈。就算是在最开明的东海,外来的蛊仙也并不好混,常常被东海蛊仙抱团欺压、排挤。

之前的交战,在黑楼兰差点打死一位南疆蛊仙之后,事情就变得越发不可收拾。

南疆蛊仙大怒,你北原的外人居然如此嚣张。欺人太甚,在南疆干架不说,还打伤我们的人。这是把我们南疆当做什么地方了?

所以追杀不辍,不仅如此,还呼朋引伴,又召来数位援手。

黑楼兰、太白云生本身就状态不佳,身上有伤。他们且战且退,在刻意的引导下,离义天山越来越远。

黑楼兰野心再大,也不敢再靠近那里。

那不是六转蛊仙的舞台。

甚至七转都不是,八转也只能沦为配角。

那是魔尊幽魂和天意的厮杀战场!

万劫阴油毒已经结束,现在魔尊幽魂被另一种万劫笼罩。

这万劫已经不为人所知,突破了历史记载的极限。

它无形无质,只有一股玄妙至极,不可捉摸的气息。

气息缠身,所到之处,引发魔尊幽魂身上一次次的爆炸。魔尊幽魂乃是魂魄,但凝如实质,宛若魂兽。

现在被气息引炸,魁梧如山的身躯已然消失,只剩下骨架一般,大块的魂魄已经被四下炸飞出去。

灰雾中,又呈现一幕。

砚石老人口吐鲜血,手握着天机蛊,口中呢喃:“原来方源的身上有着春秋蝉。难怪他能炼成定仙游,去往中洲的狐仙福地,夺取荡魂山。而后又捣毁王庭福地,弄塌八十八角真阳楼!”

“他既是天外之魔,又有春秋蝉,明显逃脱了宿命……呵呵呵,这会是我对付天意的最佳棋子。也罢,就让我来替你和太白云生遮掩,防止其他人推算出你来罢。”

……

秦百胜暗中窥视着太古墟蝠,还有天地间重重的灾劫。

这正是当时,东方长凡夺舍重生,从而渡劫时的一幕。

“东方长凡,不过是夺舍重生,却因为沾染了本体的气息,被天意察觉,降下如此灾祸。日后,本体要重生,恐怕灾劫必会更大!”

另一旁的炎煌雷泽仙僵点点头:“咱们可以走了,这场对天意的试探,已经结束。只是楚融怎么还没有回来?”

这时。姜钰仙子显现身形,神色糟糕:“楚融已死,回收凡草屋的行动失败!”

“怎么回事?”秦百胜微吃一惊,“楚融可是有炎道宗师的境界。七转修为。”

“是三茅魔仙的传人。”姜钰仙子道,“此人潜伏极深,手段狠辣,连我都差点无法回来。”

“三茅魔仙么……哼,也罢。就放弃凡草屋吧。大计启动在即。不能因为这座仙蛊屋,暴露了身份,惹来天意的提前打击。”秦百胜沉思片刻,这才道。

……

落魄谷。

影宗一行人,面对着中洲蛊仙,面色沉重。

秦百胜目光紧紧盯着凤九歌,叹道:“天意的打击已经来了。砚石为了保住方源这枚棋子,替他遮掩行踪。中洲蛊仙调查方源,结果顺藤摸瓜,反而调查到我们身上来了。”

“这下怎么办?”姜钰仙子问道。

“看来是非战不可了。”秦百胜咬着牙关道。

……

“咳咳咳……”砚石老人咳嗽不止。面色苍白,虚弱疲惫得仿佛下一口气,都有喘不过来的危险。

“这一次用了天机蛊,损耗了上百载的寿元。不过倒也物超所值,算出春秋蝉状态不佳,已是被用过。”

“看方源提前来到义天山踩点,很显然已经是知道义天山之战了。如此推断,本体已经失败过一次,所以借助夺得的那道红莲真传,让方源重生。回到过去,改变未来。呵,幸亏只是六转的春秋蝉,还不足为惧。”

“这样的话。方源无疑是一个极佳的标杆。看他接下来的表现,就能让我推算出更多的东西。”

“直接搜他的魂,会不会更直接干脆一点呢?会影响整个大局吗?”

……

“按照我的部署,薄青、七星子等人,已经尽量拖延了天庭的脚步。这个结果应该比上一世,要好得多吧。”

砚石老人作于凉亭之中。看着青山白雾,变幻莫测。

“僵盟方面,也布置妥当,可以作为挽救局面的后手。当然能够保留下来最好不过,唉,希望不会用到它吧。”

“用了新得的寿蛊,我的寿元还有二十多年。是时候死了。分魂归于本体,也是理所应当之事。不过在死之前,再为本体推算下逆天大计时的灾劫罢。”

半晌之后。

“咳咳咳。我居然没有死?这么说来,天地灾劫已经被我算尽了?不,是每一代的影宗智囊,前仆后继的推算,终于厚积薄发,达到了质变!”

“呵呵呵,真是有点讽刺。以天机仙蛊为核心,形成的智道杀招,竟能推算出天地灾劫的内容。然而这只仙蛊可是乐土所创。他生前千方百计地消除本体的影响,死后没有想到,居然会帮了影宗这个大忙。”

……

见到这一幕,天庭蛊仙们大惊失色。

原来,影宗竟然已经达到算出灾劫的地步!这样一来,魔尊幽魂在万劫中一动不动,就相当可疑了。

神秘万劫已经渐渐终止,最后一场万劫降临。

荆虬劫!

无数青绿的藤蔓,长满尖锐的倒刺,从天上,从地下,暴射而出,死死缠绕在魔尊幽魂的身上。

很快,上百万的藤蔓,将魔尊幽魂牢牢包裹,形成一个巨大的藤球。

它们死死缠绕,越缠越紧,并且扎根在魔尊幽魂的身躯上,竟然将魔尊的魂魄当做养分吸摄!

“不对劲!不对劲!”

“他是故意的,你们快看那座十绝大阵!”

天庭蛊仙目光转移,顿时如梦初醒。

原来,影宗早已算出灾劫的内容,所以针对性的做了布置。

魔尊幽魂故意不反抗,就是想借助万劫的重重力量,将自身的魂道底蕴,都融入大阵之中,当做大阵的养料。

“他究竟想炼出什么东西来?居然把自己的魔魂,都当做蛊材炼制!”

“不只是这样。万劫加身,其实是在他的身上,刻印了宇道、毒道、气道、木道种种道痕,破坏他的魂道道痕。之所以能重创魔尊幽魂,也是因为这些道痕,并不弱于他积累了十万年的魂道底蕴啊!”

嘶……

察觉到这一点,天庭蛊仙们不由地倒吸一口冷气。

如此看来,魔尊幽魂不仅将自己炼了,还利用万劫的力量,帮助自己达成图谋。

他究竟图谋的是什么东西?

十绝大阵,已经步入最后的关头。阴雾中,那个光球已经越发圆亮。

监天塔主的一颗心,顿时沉入谷底。

他脸上充斥坚毅铁血之色,沉声道:“不能再这样下去了。我们必须破坏魔尊幽魂的图谋!不管他炼制什么,必须尽快破坏了这道十绝大阵。”

“可是我们已经陷入虚化的状态。”

“不,我们还有一次反击的机会。这是只有历代监天塔主,前仆后继的钻研成果。一直以来,都严格保守的秘密!”监天塔主道。

天庭蛊仙们顿时又惊又喜:“那还不快快使用出来?”

监天塔主苦笑一声,将使用的代价说出来。

天庭蛊仙们陷入死一般的沉寂之中。

\end{this_body}

