\newsection{仙蛊宿命}    %第一百三十一节:仙蛊宿命

\begin{this_body}

%1
中洲,天上之天,天庭!

%2
银白的天空,一片光洁。

%3
无数白玉堆砌的宫殿,精致华美,空寂幽静。

%4
在宫殿群中,有一处苍老的,周身布满伤痕的洁白塔楼,如鹤立鸡群,高耸矗立,分外惹人注目。

%5
塔名监天,出自星宿仙尊之手,意为伫立此塔,监察天下!

%6
然而风流逸散,世事无常,星宿仙尊纵然才智比天,也逃脱不得寿元耗尽而消陨的结果。其后,又经历了三位魔尊的进攻,元莲仙尊、巨阳仙尊入主,幽魂魔尊制霸天下时,也曾有意向染指天庭,但不知为何打消了心中想法。

%7
三百万多万年!

%8
无数历史的痕迹,深深地印刻在监天塔上。

%9
或辉煌灿烂,或晦黯不堪,种种沧桑,已然和监天塔融为一体,化为一股稳重的威仪,宛若一尊古树,从远古生长至今,老而弥新。又仿佛铜鼎,见证着世事变迁,屹立不倒。

%10
监天塔主拄着拐杖,佝偻着背,一步步拾阶而上。

%11
他是八转蛊仙,澎湃的强者气息中却掺杂着一股强烈浓郁的老朽之气。

%12
他白发苍苍,身上皱纹如老树皮,一双昏花老眼,目光浑浊。

%13
他慢慢地抬起脚,或者说挪动,更适合一些。他就像是一只老虫,在漫长的阶梯上蹒跚,举步维艰。

%14
他一步步地走着。

%15
每走一步,脚下的白玉阶梯。都会闪耀出一抹微光,发出钟磬般清幽动听的声响。

%16
随之变化的,是老人身边的墙壁。

%17
墙壁上不断有光影变化。有时候是一团模糊的雾影,有时候是含义莫名的彩色线条,只有少数时候,墙壁上出现清晰的画面。

%18
老人关注着墙壁上的一幅幅画面。

%19
每当他登上一个阶梯,他体内的仙元就耗费一颗,与此同时,墙壁上的画面便起变幻。

%20
老人的脚步微顿。

%21
墙壁上一幅画面。生动地描绘着一片山谷。

%22
“落魄谷。”老人轻声喃喃,浑浊的双眼中,一抹精芒转瞬即逝。

%23
画面的中心。是两位蛊仙的对决。一位风道蛊仙,一位金道蛊仙。

%24
在画面的边角,则是几位蛊仙站立着,他们的目光投向中央。看着蛊仙的对战。

%25
画面持续变化。

%26
两位蛊仙之争。并不激烈,只交手了一两下,便停下手来。

%27
最终画面静止在这样的一幕风道蛊仙向金道蛊仙,缓缓地低下头颅。

%28
老人暗暗将这一幕记在心中。

%29
在通往塔顶的九万九千道阶梯,能够显现出如此清晰画面的,不过十几而已。

%30
监天塔主迈开步伐,继续拾阶往上。

%31
他看到深海之中,一群蛊仙。其中大多数都是仙僵,正进攻着一片福地。

%32
一位女仙。跪在沙地上,对一位蛊仙老者哀求着什么。

%33
一位少年蛊师,昏迷在床榻上。一只寄魂蚤,趴在他的额头,微微颤动着。

%34
他又看到一位蛊仙,白衣蓝瞳,在南疆的山林中静默行走。

%35
他还看到一片幽暗的沼泽中,庞大的血光萦绕覆盖,当中一位血道蛊仙修行。

%36
老人越看,脸色越是冰冷,浑浊的老眼中积蓄着越来越多的怒火。

%37
“这些人,都逃脱了宿命的制裁!”

%38
最终,他走完阶梯,踏上塔顶,一只仙蛊出现在他的眼前。

%39
九转仙蛊宿命!

%40
它形如蜘蛛,黑白两色,气息微弱,一道赤红的伤痕,几乎将它切成两半。

%41
老人看着此蛊,目光凝视许久,叹了一口长气。

%42
监天塔乃是九转仙蛊屋,可惜第一核心宿命,遭到致命的创伤,濒于毁灭。

%43
“红莲魔尊!”老人咬牙切齿,双目中涌现出深入骨髓的仇恨。

%44
造成宿命仙蛊如此创伤的,不是别人,正是历史上赫赫有名的红莲魔尊。

%45
红莲魔尊打坏宿命,也就打破了命运的枷锁,让天下众生都掌握自己的命运。不过,仙蛊宿命并未彻底毁灭。

%46
然而尽管有天庭众仙的力保,宿命所受的创伤,历经百万年,仍旧难以恢复。

%47
造成这一情况的,正是红莲魔尊的手笔。

%48
宿命仙蛊的创伤,不仅仅表现在蛊虫本身之上,还在于天下众生,一切逃脱宿命制裁的人。

%49
这些人的存在,本身就意味着宿命的崩坏。

%50
因而,恢复宿命仙蛊就得从两方面着手。一方面,是针对蛊虫身上的伤痕。另一方面,则是斩除天下逃脱宿命者,为宿命仙蛊扫清障碍。

%51
不管哪一方面,都是令人头疼的难题。

%52
尤其是第二方面,天下五域广袤博大,每时每刻都有逃脱宿命之人。天庭要铲除这些存在,十分困难。

%53
即便天庭乃是第一蛊仙组织,但也只能掌控中洲。其余四域,和中洲有着界壁相连,越是强者越难以突破界壁。

%54
五域各自的界壁,就像是一个个的保护层,将五域隔绝成相互的领域。

%55
起先,宿命仙蛊刚刚受创的时候,天庭铲除逃脱者,还颇见成效。但仅仅几年之后,无数的逃脱者宛若海面上的浪花,接连涌现,应接不暇。

%56
十几年后,这种情形在五域广阔泛滥。

%57
数十年后,出现仙僵的存在,该死之人,却仍旧生还,这是典型的逃脱宿命。这种情况,曾一度令整个天庭震怒惊惧。

%58
到如今,人族昌盛,乃至中洲都出现了仙僵分部。求生是人的本能,即便天庭也难以抵抗这般大势。

%59
恢复宿命仙蛊的目标,似乎越加遥遥无期,希望渺茫了。

%60
但天庭却从未放弃过。

%61
因为一代代的天庭蛊仙们,都牢记着天庭远古时代的威仪,上古时代的辉煌。

%62
而建立这层辉煌的最大基石,便是宿命仙蛊。

%63
星宿仙尊死前,布置的手段,正是以它为主。因而,才能力抗三位魔尊,而维护天庭不倒。

%64
彻底修复宿命仙蛊,就等于将天下众生的人生轨迹掌握于手。这也就意味着,天庭将重新高高在上,再次成就仙中至仙,万王之王!

%65
一切的努力,都会有所结果。

%66
宿命仙蛊在天庭蛊仙一代代的努力下,耗费不知多少代价,历经漫长岁月,一点点的恢复,慢慢积累,这才到达如今的程度虽然奄奄一息,但已经能够勉强使用。

%67
正因如此,监天塔的塔壁上才会出现种种画面。

%68
逃脱者和宿命仙蛊,是不共戴天的对立两方。但是逃脱宿命制裁的存在,实在太多太多了。监天塔的塔壁上,显现出来的,也不过是当中最强的,并且不在福地洞天当中,没有智道手段遮掩,最容易推算的存在。

%69
“快了,就快了。借助这一次的炼蛊大会,弥补宿命仙蛊的创伤,就会有质的突变。过往一切的投入,都是值得的。无数年的积累,将会物有所值,从而得偿所愿。炼蛊大会之后,宿命仙蛊就能展现出五成的威能!”

%70
监天塔主抚摸着宿命仙蛊,口中喃喃。

%71
他的神情渐渐柔和起来,心中的恨怒也得以暂时的平息。

%72
“不过,在此之前,还是要铲除这些逃脱宿命的罪孽头子!一切的生命,从降生之初,就已然决定了轨迹和结局。这是天地的期许,自然的规划,怎么容得你们逃脱逍遥?这本身就是不属于你们的生活。”

%73
想到这里,老人的心头又浮现出之前登梯时,牢牢记着的清晰画面。

%74
“在宿命恢复之前,就让我来替天行道,斩除这些窜得最高的杂草吧!”

%75
中洲,狐仙福地。

%76
智慧光晕充斥着地下洞窟。

%77
方源置身于此,紧紧闭着双眼。他的背后,是洞窟中最大的一株灵芝,乃是芝林中王,此刻已经长成肥敦敦的小树,芝叶宛若华盖。

%78
七彩的光辉,映照在方源的脸上。

%79
他呼吸沉稳平缓,脑海中恶念此起彼伏,翻滚如浪,相互之间不断碰撞。

%80
片刻后,方源便睁开双眼。

%81
“积攒了这么久的恶念蛊,就这样消耗光了。”他的眉头微微皱起,心中并不满意此次成果。

%82
他先是推算了拔山、挽澜两只力道仙蛊,如何添加到万我杀招中去。结果耗费了恶念蛊的一半存货,有了不到一成的进展。

%83
这进度太慢,方源便转而推演仙道杀招见面似相识。

%84
结果将恶念蛊完全耗光,进度只达一成多点。

%85
“到底还是变化道的境界,差了许多。”方源微叹一声。

%86
这时他心中忽然有感,仙窍中来了一只推杯换盏蛊。

%87
蛊中送来一份信笺。

%88
方源展开一看,信中只有四个字时机已到。

%89
方源眼中精芒一闪,便站起身来。

%90
来信的人,正是黎山仙子。

%91
本来方源打算着,去循着盗天魔尊的传承线索,找寻落魄谷。但不久前,黎山仙子处忽然传来一个消息,让方源打消了落魄谷的计划。

%92
这个消息,便是东方长凡的智道传承。

%93
东方长凡死后,立东方余亮为衣钵的继承人。但后者只是凡人蛊师,为了防止别有用途的蛊仙打这份智道传承的主意,东方长凡暂时没有将智道传承,直接交给东方余亮,而是布置在隐秘地点。

%94
黎山仙子交友广泛,善于探知情报。

%95
这次的情报,就是东方余亮已经秘密动身,前往未知地点,企图继承智道传承!

\end{this_body}

