\newsection{万星飞萤战群雄}    %第一百四十四节:万星飞萤战群雄

\begin{this_body}

“东方余亮?我没有看错吧?”

“他怎么成为了蛊仙?看来这天劫地灾,是因他而起了。qiushu.cc [天火大道小说]”

“立地成仙,这就是东方长凡的传承吗?唉,竟然给一个凡人抢夺去了!”

魔道蛊仙们看到东方长凡,大为震动,恍然之余又是懊丧。

“东方余亮”面无表情,双眼不见正常的瞳孔和眼白,而是覆盖着千万种琉璃之光,不断闪烁。

他用这样的目光,缓缓打量远处的魔道蛊仙们,然后低吟道:“就是你们,想要抢夺恩师留给我的传承?谁给你们的胆子,竟然来对付一个正道超级势力?”

东方长凡成功夺舍,但却并未暴露真相,而是以东方余亮的口吻出声。

魔道蛊仙们灰头土脸,本就一肚子怨气,此时听着东方余亮居然大言不惭,顿时脸色不愉,有的不屑冷哼,有的喊道:“区区小辈,刚刚晋升成仙,一步登天,就目中无人了!来,接我一锤!”

话音还未落下,众人就将一道身影,从身旁飚射而出。

这位魔道蛊仙,虎背熊腰,浑身套甲,胯下一只飞天猪,肥胖的四只猪蹄急促奔驰,载着主人气势汹汹地向东方余亮杀去。

正是卓战。

众仙一个激灵,顿时反应过来。

“这卓战奸猾,想捡便宜!”

“差点被刚刚的天劫地灾唬住了。东方余亮就算成仙,也不过是个嫩鸟。怕他什么?”

“活捉东方余亮,拷问出智道传承!!”

魔道蛊仙们纷纷出动,争相恐后地向“东方余亮”扑去。

自在书生也夹杂其中。

唯有方源等少数几人,没有动弹。仍旧停留在原地。

他还不知道东方余亮已经被东方长凡夺舍,而是忌惮着残阳老君。

“东方余亮出现,残阳老君又在哪里?”

残阳老君其实就站在“东方余亮”的身后,只是用了仙道手段,隐去身形。他手段高妙,魔道众仙竟然没有一个察觉。

不过他受伤不清,此时见到众仙扑来。向“东方余亮”传音:“需要我出手吗?”

“东方余亮”连忙否决:“你不必出手。让我独自杀退他们!”

残阳老君出手,万一暴露,北原蛊仙就会知道“东方余亮”已和中洲势力勾结。\&\#65288;\&\#26825;\&\#33457;\&\#31958;\&\#23567;\&\#35828;\&\#32593;\&\#32;\&\#87;\&\#119;\&\#119;\&\#46;\&\#77;\&\#105;\&\#97;\&\#110;\&\#72;\&\#117;\&\#97;\&\#84;\&\#97;\&\#110;\&\#103;\&\#46;\&\#67;\&\#9到那时,“东方余亮”不仅会被正道排斥,就连北原魔道也不会待见他。

再者,“东方余亮”也想用一己之力,重创来犯诸敌。只有如此。他才能东方八位蛊仙陨落的情形下,强势崛起,重掌东方部族,威震四方!

卓战领先一步,越冲越近,近到东方余亮都能看清他脸上的狞笑。

在他身后,十多位魔道蛊仙更是杀气腾腾。

东方余亮孤身一人,修为不过六转垫底,却是不慌不忙。

他的嘴角微微翘起,浮现出一丝运筹帷幄的笑容。

身后的残阳老君。饶有兴趣地看着他,倒是要看看凭他此刻修为,如何能杀退强敌。

只见东方余亮缓缓伸出右手臂,白袍宽袖,右手食指指天,一字一顿地朗声道:“万、星、飞、萤。”

来犯的众仙闻言,不禁气势一滞。脸色各异。

万星飞萤在北原蛊仙界几乎众所周知,乃是东方长凡的最强杀招。

东方长凡被公认为当代北原智道第一人,不是单凭他的推算,更主要的还是战力!

没有惊人的战力,怎么可能在战乱纷繁的北原站住脚跟?没有雄厚的战力,怎么能带领东方部族重新兴盛?

因此,论战力强悍,东方长凡生前还要稳稳地压过皮水寒、自在书生一筹。他可是正道超级势力的太上大长老!

而智道杀招万星飞萤,便是东方长凡战力的最佳象征。此招的赫赫威名,早已深入北原蛊仙之心。

“听他诈唬!万星飞萤涉及数万只蛊虫,是那么容易催动的吗?”卓战大吼。

“他不过蛊仙新人,刚刚升仙,才脱离凡人境地,就想一次性操纵这么多蛊虫?”陆青冥冷笑。

“异想天开!”天都神君手中拿捏着仙蛊,蓄势待发。

“哼,万星飞萤,当初我就是败在此招之下。如今我倒要看看你有几成威能?”自在书生的脸上怒色一闪。

“我们真的不动手?”黑楼兰怦然心动,但方源才是他们之中的最强战力,黑楼兰失去了我力仙蛊,战力较原先还稍弱一筹。

“再等等,情况有些不妙。”方源刚想将之前的推测说出,天变了。

明明刚刚乌云散尽,还一片晴空万里。此刻忽然夜幕降临,天地晦暗。

夜幕覆盖方圆近千里,无数的萤火虫似的的光点,在夜色中生成,旋即飞舞起来。

星光飘摇,浩荡烂漫!

“万星飞萤!!”众魔道蛊仙面色骤变,纷纷惊呼,陷入星萤的包裹当中。

星光连绵,映照在众人脸上。

自在书生脸色铁青,这是货真价实的万星飞萤,威能似乎比他记忆中还要强盛一筹!

但这怎么可能?

明明东方余亮,刚刚晋升成仙?就算有着东方长凡的星意,可以消耗东方长凡留下来的七转红枣仙元,但头一次催动万星飞萤,怎么会如此娴熟,仿佛千锤百炼的一般?

自在书生百思不得其解。

他身旁的魔道蛊仙们,更加狼狈。

周围的星点密密麻麻,从四面八方向他们围攻。

一点星光本身威胁不大,但成千上万的星光呢?十万。百万,千万的星光呢?

尤其是每一点星光内里,包含着一颗星念。

蛊仙们就算击破星光,星念蹦出来,就向蛊仙们的脑海中钻去。

蛊仙们纵然有防御手段。但如此星光无穷无尽,绵延不绝,必会守久而失。一旦让星念钻了空子,进入自家识海,那就不得了了。

轻则干扰自身思考,重则念头剿杀,本体行动呆滞。更甚者。本身的念头被星念蜂拥而入,完全压制。星念代替思考,蛊仙会暂时成为东方余亮的傀儡,受其操控!

“你干什么?!”卓战身旁,一位魔道蛊仙朝着他怒吼。

卓战面皮酱紫:“不好意思,我刚刚一击竟忽然改变了方向!”

“哼!你这货色,也来抢智道传承?竟然这么短时间。就被星念侵蚀了。真是成事不足败事有余!”魔道蛊仙险些就被拍中,气急败坏。

“你说什么?你敢再说一遍?!”卓战也怒了,朝着魔道蛊仙对吼。

“够了,还在内讧?当务之急,是赶紧脱离这里。这可是万星飞萤啊,时间待得越久,就越是危险。”有蛊仙急道。

旋即就有蛊仙反驳:“脱离?你怕什么?对方不顾区区六转垫底的蛊仙,我们这边多少人,还有自在书生大人,你居然害怕成这种样子?依我看。还是直接杀上去!”

众人进退犹豫,下意识地都将目光投向自在书生。

皮水寒已逃,方源又在边远,自在书生作为最高且唯一的七转战力,便是众人的领头羊。

这并非是众人都认可自在书生的领导,而是天塌下来,由高个子顶着。

自在书生就是他们中的“高个子”。跟着强者行事,才有更多的生存机会。

自在书生心中也有犹豫。

他是和东方长凡交过手的人,当初就败在了这招万星飞萤之下。

战败之后,他痛定思痛,汲取教训,更加关注东方长凡的一切情报,因而对万星飞萤的了解,比众人更为高深。

他知道,万星飞萤乃是智道杀招,威能超绝,由东方长凡一手创立。

此招的弊端,难以勘察。

自在书生到目前为止,也只知道:只要入了这招的范围,就受到算计。蛊仙自身的一思一念,都会形成星念。

这些星念在万星飞萤的影响下,形成点点星光,夹攻蛊仙。

蛊仙激战中,自然要心思万千,脑筋急速转动,便又会产生更多的星念。

星念越来越多,优势便愈来愈大。尤其是入阵的蛊仙越多,星念产生的也就越多。当星念侵蚀了蛊仙脑海,更会沦为东方余亮的傀儡,为其拼杀。

就算达不到此种程度,被星念影响到,就会失误。就像刚刚,卓战差点误伤他人。

因而,万星飞萤这招更擅长以一敌众,对付单个强敌,反而威力弱些。对付多人,却会借助彼此之力,相互牵制,战果往往更胜一筹。

东方长凡之所以创造此招,也是当初自己独领东方一族,强行崛起。本家却无多少蛊仙战力帮衬,往往要让东方长凡以一敌众。

不过,东方长凡自创造此招之后,就再也不惧群攻了。

“眼下这情景,却是东方长凡最想看到的。我方人数众多,又并不统一,相互猜忌。偏生强者又不多,七转战力只余我一位。真要战下去,反而相互之间成为掣肘。不仅如此,看刚刚的攻防,似乎还有一位火道蛊仙隐藏了行迹。”

自在书生心中迅速思考,周围的星光也随之点点浮现,明显比他人更加浓密。

如此多的顾虑,自在书生却反而没有退意,反生出激昂的战意!

“我败于东方长凡,乃是生平之辱。现在面对他的一个传人,还要后退,今后我如何面对自己?就算对方有一位蛊仙隐藏,我方也是人数众多,若不战而退,今后北原蛊仙谁人能看得起我?”(未完待续 。)<!--80txt.com-ouoou-->

\end{this_body}

