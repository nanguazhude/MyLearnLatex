\newsection{封印春秋蝉}    %第二十二节:封印春秋蝉

\begin{this_body}

半月之后,方源再次来到琅琊福地…

眼前云雾飘渺,纯白茫茫,云海翻腾中十二云阁屹立不动。又有数条大江大河,宛若游龙,行于云土之上。

琅琊福地在太白云生的治疗下,已经重塑过往盛景。

方源被接引着,进入一处云阁,见到了琅琊地灵。

琅琊地灵身上,仍旧覆盖着层层气道封印。不过原来是十八层封印,在墨人王的帮助下,已经削减了三层,剩下十五层。

墨人王就站在琅琊地灵的身边,见到方源他不敢托大,首先招呼道:“方源阁下,别来无恙乎?”

方源对他笑了笑,外龇的獠牙显得他的笑容十分狰狞。

他转向琅琊地灵,抛给他一只四转东窗蛊:“你看看。”

东窗蛊静静地悬浮在琅琊地灵的眼前。

琅琊地灵无法伸出双手,但方源主动借给他东窗蛊,使得他能够探入心神。

他查看一番后,满意地连连点头,交口称赞道:“不错,很不错!你把这三道仙蛊方都给彻底完善了!厉害,这思路简直绝了!!”琅琊地灵并不笨,念头一动,召来三十块仙元石:“这都是你的报酬!”

方源立即接过这些仙元石,放入自家仙窍。

这一次赚的,却没上次多。皆因这次推算仙蛊方,耗去了他整整六颗青提仙元。若非星门搭建好了,省去了他动用定仙游消耗的两颗青提仙元,他的成本会高达八颗青提仙元。

虽然这些仙蛊方,也都是九成以上的残方。但每道仙蛊秘方,却要具体分析。有些仙蛊秘方推算难度很大。难关过了一道,还有一道,思考的时候就会消耗更多的意志。

但尽管如此,这里面的利润还是丰厚至极。

墨人王用羡慕的眼神,看着双方完成交接,而后微笑道:“方源阁下,其实每次交易,未必都需要仙元石付账吧?琅琊福地这里。有许多宝黄天中都买不到的好东西,咱们可以以货易货。实不相瞒,我的墨人城中也收藏了一些货物,也许阁下会感兴趣。前不久,阁下的师兄太白云生大人,就曾买下我手中的一道杀招。”方源闻言一笑:“墨人王的提议甚好,我这边正好缺少三只小泽蛊,六只松岛蛊。不知道你那里有没有?”

这问题,问住了墨人王。他脸上浮现出尴尬之色:“惭愧,小泽、松岛之名,在下还是首次听闻。”

“这些都是上古蛊虫,小泽蛊能改变地貌,形成一小块沼泽。松岛蛊同样如此。它可以在海面上凝聚出一块浮岛,岛屿的中央是一颗参天大树,树根盘绕纠缠。凝聚固定土地。”琅琊地灵适时地解释道。

末了,他好奇地看向方源:“这些蛊虫,都已经被淘汰了。到如今,蛊仙们有更廉价的办法去改变福地地貌。你要这些蛊虫干什么?”

“我的目的你无须打听,听你的口气,琅琊福地中应当有这些蛊虫的吧?”方源笑了笑,神秘地道。

“当然!”琅琊地灵微微扬起头颅,傲然地道,“我每得到一道蛊方。都会炼制一些蛊虫进行收藏。小泽蛊、松岛蛊我都各有上百只的收藏品。可以卖给你。不过炼制这些蛊的材料,如今已经绝迹。你需要付出一块仙元石。……

物以稀为贵。小泽蛊、松岛蛊这两种蛊虫,连宝黄天中都没得卖。

方源没有丝毫犹豫,当即抛给琅琊地灵一块仙元石,得到三只小泽蛊,六只松岛蛊。

接着,他感兴趣地又问琅琊地灵:“奇怪了,你收藏了这么多蛊虫,怎么解决它们的喂养问题的?”

方源这些天,也在为喂养蛊虫的事情烦神,尤其是他手中的仙蛊。

琅琊地灵给出了他的答案:“凡蛊的喂养问题,很简单,你可以打造一个特殊环境,让它们都陷入沉眠之中。仙蛊就不能这样了,就算是沉眠,也要喂养。只是喂养的资源,比正常的时候少至少一半。”

这个答案多少让方源有点失望。

“能详细说说吗?”他道。

“相比你也去过八十八角真阳楼里的真传秘境吧?真传秘境就是一种特殊环境,能令蛊虫都陷入休眠,凡蛊不需要喂养,仙蛊的喂养费用也大大降低。”

方源皱起眉头:“但就算这样,真传秘境中的仙蛊也数量繁多,我不相信八十八角真阳楼每隔十年,都能搜刮到刚好满足喂养这些蛊虫的食料。”

“那是因为巨阳仙尊采用了食道的手段。食道是一个隐秘流派,专门研究解决喂养蛊虫、异人、野兽甚至人类的大难题。别问我食道的问题,我不知道任何消息。我老人家手中的仙蛊,都是老老实实地,按照最正统的方法喂养的。”

方源还不死心:“当初你的本体和巨阳仙尊合炼八十八角真阳楼,他动用的食道手段,你会不知道?”

琅琊地灵翻了个大大的白眼:“本体曾经和巨阳仙尊立下誓言,不透露有关食道手段的任何信息。这是一个大秘密,其实我也很想知道。但继承下来的记忆中,完全没有这一块的信息。”

方源点点头,他心知地灵正直,不会撒谎,只是随口一问而已。

虽然完成了交接,但方源没有急着离开,他这次来,还有另外一个目的。

他掏出一只星芽蛊,递到琅琊地灵的面前:“这只蛊你能否逆推出炼制蛊方呢?”

琅琊地灵没法伸出双手,只能用鼻子嗅了嗅这只星芽蛊,道:“这蛊虫明显被人有意动了手脚,蛊方难以逆推,至少得炼道宗师出手。单凭这一只三转星道新蛊,要去逆推蛊方,完全是不可能的事情!因为一只新蛊,完全不够研究过程中的消耗。除非你拥有四千只这样的蛊虫……”

方源一笑,伸出一只右手,摊开五指:“我有超过五千只的星芽蛊。”

星芽蛊是组成杀招春星雨的必要蛊虫,方源尝试用了一次春星雨,发现这个杀招的效果极佳。

他一次性买下这么多星芽蛊,就是打着逆推蛊方的想法。现在这个念头,更加强烈了。

“五千多只……当然能够逆推蛊方了。可惜,我老人家现在被封印住,根本出不了手啊。”琅琊地灵深深叹息一声。

方源跟着皱起眉头:“你这气道封印,还有十五层,该用多久才能彻底清除?”

“至少一年。”墨人王道。

“这么长时间?”

“没办法,这封印绝对是一个仙道杀招,越往后解除封印,就越加困难。”墨人王流露出苦涩的笑容。

长毛老祖是历史上的炼道三大宗师之一,名垂千史。琅琊地灵的炼道境界,至少炼道宗师。而方源只是炼道大师。

如果能请琅琊地灵出手,自然最好。但现在琅琊地灵被封印住,不能炼蛊,就不能逆推蛊方。

“其实,逆推蛊方也不是不可以。”墨人王提议道,“琅琊福地中不是豢养了很多的老毛民吗?这些老毛民,都有炼道大师的造诣。到时候,请他们为地灵打下手,地灵临场指导,不就可以了吗?”

琅琊地灵顿时对墨人王翻了个白眼:“外行人的话!炼道博大精深,很多手法、杀招都需要相应的智力、修为、天赋才情,还有大量的刻苦训练。你以为炼道大师数量多了,就能媲美一位炼道宗师吗?扯淡!”

墨人王被骂得一愣一愣的,只能呵呵地笑,不敢反驳什么。

“既然这样,那就先算了。等到地灵你解脱了封印,咱们再谈吧。”方源告辞,转身离去。

这个事情,并非他的当务之急。

“小子,就算我封印解除了,也不会白白帮你逆推的。到时候准备好三块仙元石吧!”琅琊地灵冲着方源的背影大喊。

方源头也不回,向后摆了摆手,随后踏入星门,消失不见。

“哼,臭屁的小子。”琅琊地灵望着他消失的背影,忍不住嘟囔了一句。

方源回到狐仙福地,便进入荡魂行宫,立即着手封印春秋蝉之事。

小泽蛊、松岛蛊到手,终于补齐了缺失的最后一块。

七天七夜之后,方源相继在自己的第一空窍,以及丹田附近的腹部,种下大量蛊虫。

这些蛊虫效用相互连接,相互增幅,形成一道强大的封印。一方面帮助第一空窍,抵御春秋蝉的压力。另一方面,将春秋蝉的部分气息,经过篡改后,散发到体外去。

第三方面,也是最关键的一点,封印之后,也不影响春秋蝉的使用。

“这封印足可以支撑四年时光。如此一来,春秋蝉撑破第一空窍的难题,也就暂时解决了。”方源查看无误之后,心中着实松了一口气。

这个难题的解决,让他身上肩负的重压,也为之一轻。

封印春秋蝉的过程,绝不轻松。毕竟是以凡蛊封印仙蛊,方源累得魂魄都黯淡了。

封印成功之后,方源倒头便睡,一连三天,都不想起床。)

\end{this_body}

