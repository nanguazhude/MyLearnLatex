\newsection{巨大利润!}    %第十七节:巨大利润!

\begin{this_body}

%1
“难道这个小子,居然真的是百年难得一见的智道天才蛊师?不,这样的才智,简直是惊天动地,何止百年难得一见,说是千年、万年都丝毫不为过啊!”

%2
这样一想,琅琊地灵顿时觉得方源的臭脾气,是很有存在的道理的。。

%3
“对,也只有这样有才华的人,才有这样的臭脾气。就像我的本体……”琅琊地灵想到这里,顿时觉得方源的臭脾气不那么讨厌了,反而心中生出一种惺惺相惜之情。

%4
狐仙福地那边,方源好整以暇。他虽然成了僵尸,思维僵化,运转不便,但欺负一个地灵还是绰绰有余的。

%5
如果有可能,方源当然不愿这么冒风险。他宁愿一次次,间隔较长时间,将仙蛊方交给琅琊地灵。而不是像现在这样,直接将这三道仙蛊方都交出去。

%6
这样做,太出风头,太高调,肯定会引来有些人的怀疑和猜忌。

%7
琅琊地灵纵然坦率,但他身边还有个好友墨人王,此异人可是货真价实的枭雄之辈。

%8
但方源没办法,形势所逼。

%9
他必须在短时间内,搞到大量的仙元石增强自己的实力,破开眼下的危局,同时尽可能地推迟自己暴露的时间。

%10
若是因为顾忌这里顾忌那里,而保留仙蛊方,最终导致战败身死,那方源找谁哭去?

%11
方源一笑,琅琊地灵那边没有回应,但他知道:这三道仙蛊方接连抛出。宛若三颗炸弹,已经把琅琊地灵炸得心驰神摇。此刻正是大谈条件的好时机!

%12
于是他开口:“琅琊地灵,不瞒你说。我需要仙元石,而你需要仙蛊方,我们今后的合作还会更多。建立星门,对你我都有好处。”

%13
“建星门……”这一次,琅琊地灵犹豫起来,没有立即回绝。

%14
方源展现出了他智道的超绝才华,这让琅琊地灵真正改观。对方源刮目相看。因此方源的建议,令地灵开始史无前例地重视起来。

%15
方源又劝道:“地灵,你以为咱们不建立星门。你的琅琊福地就安全吗?别忘了,你已经遭受了三波攻势,每一波攻势都比前一波要更强。毫无疑问,琅琊福地的位置已经被发现。你也已经被某个大势力盯上。咱们建立星门后。甚至可以进一步合作,关键时刻我可以通过星门,及时地来保护你。”

%16
“保护我……”琅琊地灵终于心动了。

%17
他想到了方源之前暴揍泥沼蟹的骄人战绩,的确是强大。

%18
而且这三波攻势,彻底搅乱了琅琊福地的平和盛景,让琅琊地灵产生了巨大的危机感。

%19
这时方源又问道:“你没有在宝黄天中,出售过星门蛊吧?”

%20
“目前还没有,我这里可不缺仙元石。”琅琊地灵道。

%21
方源哈哈一笑:“这就更好办了。星门蛊乃是创新之物。并非洞地蛊种在地上,容易被发现。每次开启星门。你都需要催动星门蛊,因此会很隐秘。就算预料到不妥,关键时刻你直接摧毁手中的星门蛊,不就行了?建立星门,根本不算什么威胁。”

%22
方源巧舌如簧,句句说在琅琊地灵的心坎里。琅琊地灵终于答应下来,道:“好吧,咱们两家建立星门。不过深入合作暂且不谈,就先进行推算仙蛊方的交易。”

%23
说完,地灵又附加一句:“另外星门蛊我出,我借给你用。”

%24
这样一来,他就更能掌控局面,他到底还是有些不放心。

%25
狐仙福地这边,方源则露出胜利者的微笑,八只拳头振奋握紧。

%26
成功了!

%27
如此一来,就得到了琅琊地灵的初步认同,沟通了琅琊福地,对今后交易大有好处。不仅能剩下一笔钱财,而且还有了另一条退路。

%28
除此之外,星门蛊方源也重新控制起来,防止它流入宝黄天市场,带给其他蛊仙巨大优势。

%29
星门蛊琅琊地灵出了,也省了方源一些麻烦。

%30
这样做,能带给琅琊地灵安全感,方源乐见其成。

%31
“和琅琊地灵搞好关系,很有必要。和他交易,不会被骗,又能搞到大量的仙元石。只要关系处理好了,不断加深,说不定今后还能从琅琊地灵那里,借一些仙蛊使用。”方源暗暗盘算。

%32
琅琊福地中,可藏有好多仙蛊呢,尤其是其中驭兽仙蛊、天元宝皇莲。

%33
借仙蛊的话,方源也省下了好大一笔养蛊费用。

%34
多么划算的事情啊!

%35
方源做一步,算十步。琅琊地灵也很高兴,他觉得自己运气真好,找到了一个万年难得一出的智道大天才,他手中的仙蛊残方都有望完善了。

%36
方源原本打算利用宝黄天,传送星门蛊,结果被谨慎的琅琊地灵拒绝。

%37
方源只好动用青提仙元,再跑一趟琅琊福地,带回星门蛊。

%38
太白云生仍旧留在那里,福地中还有几道长河需要他去修复。

%39
如此一来一回,消耗了方源两颗青提仙元。

%40
但他带回来一只星门蛊,三张新的仙蛊残方,同时还有三十块仙元石。

%41
太白云生要给方源的三块仙元石,方源没有收,仍旧留给了太白云生。毕竟太白云生也需要资源修行,他更强了,对方源的帮助比单纯的三块仙元石要更多。

%42
星门没有当即建立起来——方源手中的星萤蛊已经消耗一空了。

%43
没有星萤蛊散发出来的星光,两面星门都是无法开启的。

%44
方源总结了一下此次的交易。

%45
他为了推算仙蛊方,总共消耗了五颗青提仙元。先前独自从琅琊福地回来,消耗一颗青提仙元。为取得一只星门蛊,来回琅琊福地、狐仙福地,消耗两颗青提仙元。总共付出了八颗青提仙元的代价。

%46
而他获得的,是整整三十块仙元石。

%47
“我现在的青提仙元,只剩下九颗了。”方源成就中等福地,得到的青提仙元本来就少,这一次北原往返间,猛烈使用,青提仙元因而剧烈消耗。

%48
但方源并不焦急。

%49
他虽然成为僵尸,仙窍已死,不能再自产青提仙元。但是他可以从仙元石中,抽取石中的仙元,化为自己的青提仙元。

%50
当初,古月一代化为血鬼僵尸,也曾利用过元石来恢复真元。这两者之间的道理相同,只是层次不同罢了。

%51
也就是说,方源完全可以将这三十块仙元石,统统用掉,转为自己的青提仙元。一块仙元石一颗青提仙元来算,他的青提仙元将暴涨到三十九颗。

%52
“但仙元石可比青提仙元有价值多了,前者能作为硬通货币,后者只能个人使用。不到必要时刻,还是不急着将仙元石炼为青提仙元。只要和琅琊地灵的交易持续下去,仙元石会越来越多的。现在,我先沟通宝黄天看看。”

%53
这一次沟通宝黄天,方源是底气十足。

%54
仙元石购买力十分强大。当初方源用一块仙元石,就买下了一头万狼王,三万多庞大规模的狼群。

%55
三十块仙元石,足以称得上一笔巨款。

%56
前世五百年间,方源建立血翼魔教,身为魔道巨擘,手头里仙元石最多的时候,也不过六十四块。当然,前世的他花费也大。这个数字,是他消费和盈利相互抵消而得的结果。

%57
方源闭目静坐,将心神投入自家仙窍。

%58
仙窍一片死寂,灰天黑地,充裕着阵阵腐烂的臭气。

%59
墨瑶意志仍旧关押在里面,时刻受到方源的监视。原先移植过来的那株镜柳,早已经死亡,成为一段朽木,倒在地上。

%60
他念头稍动,仙窍中一只新的通天蛊,飘摇而起,升上半空,沟通宝黄天。

%61
方源念头再动,催发神念蛊,形成连续不绝的神念,顺着通天蛊的通道,探入宝黄天。

%62
他如今已成仙僵,再无须借助地灵小狐仙之手,完全可以亲自买卖了。

%63
宝黄天仍旧是那般气象,广阔的空间中,荡漾着一片柠檬般的黄光。没有山川树木,也没有鸟兽虫鱼,甚至连天地都没有,只有各式各样的珍宝货物。

%64
蛊虫、蛊方、杀招、兽群、异人、植被、矿脉、土壤、水、美酒等各种各样的物资,一团团静静地漂浮着,各自散发着五颜六色,或高或矮的宝光。

%65
宝光越盛越高,证明珍宝的价值越大。但宝光只是初步的估算,若是货物被蛊仙动过手脚,宝光也不一定能显现端倪。

%66
这是五域蛊仙共有的最大交易市场,允许讨价还价,抽取一定的手续费用。也考验蛊仙的眼力,若是买到假货,一般都只能自认倒霉,宝黄天并不负责赔偿。

%67
方源首先要买的,便是星萤蛊。

%68
这种蛊虫能绽放纯粹的湛蓝星光,是星门开通的前提。

%69
他神念搜索,很快就发现一大团的星萤蛊,足有上千只。

%70
而这团星萤蛊的卖家,方源也很熟悉,便是之前的万象星君。

%71
万象星君似乎是星道蛊仙,他不仅出售星萤蛊,而且还出售星镖蛊、星火蛊、流星天陨蛊、星河蛊等等。

%72
显而易见,万象星君的福地盛产星道凡蛊,他售卖这些蛊虫创收。

%73
即便不是星道蛊仙,也往往购买一些星道蛊虫,用来炼蛊,或者研究杀招等等。

\end{this_body}

