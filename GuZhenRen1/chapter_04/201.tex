\newsection{血魔解体,破钟得臂}    %第二百零一节:血魔解体,破钟得臂

\begin{this_body}

%1
“仙道杀招血魔解体……”方源凝望着金钟虚影内的断臂,口中喃喃。

%2
根据《宋紫星传》中记载,宋紫星反叛万龙坞之后,杀溃追兵,便潜藏行迹,苦心经营,迅猛发展。

%3
他目光长远,心知自身处境,便专门研究出了这个杀招。

%4
在此届中洲炼蛊大会时,十大派联手埋伏设计,企图围杀宋紫星,结果竟然被他逃跑。

%5
此役,血魔解体首次亮相,便惊艳中洲。宋紫星能够逃跑,大半的功劳都在这招身上。

%6
这个仙道杀招,能在短时内形成一个血肉分身。血肉分身速度极快,可以媲美血虹闪。而且又惟妙惟肖,极其逼真。哪怕是仙道侦察杀招,都不大可能能看出破绽来。

%7
然而,施展此法代价也是极大。

%8
需要用自身肉体的一部分,化为血肉假身。还要耗费大量的仙元,给予动力,维持血肉假身的存在。耗费的仙元越多,维持的时间就越长。

%9
蛊仙断肢重生的手段不少,血魔解体的肉身损耗只是表现现象。实则,是消耗了肉身上的血道道痕。

%10
正是因为这些血道道痕,才成就了血魔解体的玄妙威能。

%11
若非是用血道道痕作为代价,血魔解体形成的血肉假身,绝不会做到如此以假乱真的程度。

%12
寻常蛊仙受伤,比如方源哪一天受到攻击,大半个身体都没了。他身上的力道道痕,都会被仙窍回收。等到他的身体复原,力道道痕便再次依附肉身。

%13
蛊仙一旦死亡,浑身上下的道痕,也都会回收到仙窍中。就比如狐仙福地。

%14
宋紫星成功施展了血魔解体,就算断肢能重现生长出来。新生的肉身上也只会空白一片,原先的血道道痕都随着血肉假身而消耗光了。

%15
毫无疑问,道痕的损耗对蛊仙而言,是个极大的损失。

%16
因为道痕十分难得,历经天劫地灾,才会渐渐有积累。每一次天劫地灾,蛊仙往往都要冒着生命危险,乃至仙窍受损,苦心经营的资源毁于一旦。

%17
因此,蛊仙得到道痕的成本,是十分高昂的。

%18
从这点便可看出吃力仙蛊的价值。

%19
但方源可以肯定,宋紫星身上应当没有类似吃力的仙蛊。他损失掉了血道道痕,就得再渡灾劫,重新慢慢积累。

%20
蛊仙身上的道痕,能有效地增长仙蛊、仙道杀招的威力。这种增幅程度是恐怖的,八转级数能成百上千倍的增长。因而七转蛊仙,很少能斗得过八转的存在。

%21
《宋紫星传》中记载,宋紫星逃之夭夭,留得一命。最后仅剩下一个头颅和半个胸膛,其余肉身都被他断然舍弃,催使了血魔解体。

%22
这一役,中洲十大古派大失颜面,没有达到目标,反而被宋紫星戏耍一通,损兵折将。

%23
宋紫星损失也极其惨重,损失掉一头上古荒兽戾血龙蝠,肉身一大半的血道道痕损毁。这一役,使得他之后两百多年,都没有公开露面过。甚至一度杳无音讯,让人怀疑他已经死亡。

%24
不过等到他重出江湖之时,他不仅战力尽复,甚至还培育出了三头戾血龙蝠!

%25
伏击宋紫星这样重大的事件,方源自然知道得清清楚楚。

%26
他看着断臂,目光隐现灼热。

%27
这个断臂手掌,来源于宋紫星,是他本体的血肉。方源只要得到它,再依照炼蛊的手法,就能炼出宋紫星的头颅。

%28
而宋紫星的头颅,正是星象福地地灵认主的唯一条件!

%29
地灵都是认死理的存在,只要方源拿出宋紫星的头颅交给地灵,便可成为星象福地的主人。星象地灵是不会管宋紫星是否还活着的。

%30
它只是单纯的执念,再结合福地的天地伟力形成。

%31
方源凭此成为主人。就算将来某一天,宋紫星跑到地灵面前活蹦乱跳,地灵也不会背叛方源。

%32
因为认主的条件,方源已经确确实实的达到了。

%33
不过此时,方源要收服这个手臂,还有些麻烦。捏破金钟并不困难,然而金钟虚影一旦破开,手臂就会爆开,化为遮天的血雾。

%34
方源必须辅以其他手段,及时地镇压住断臂,否则就会竹篮打水一场空。

%35
事实上,方源对这个手臂的印象也颇为深刻。

%36
他现在的位置,位于真阳山脉边缘,鸳鸯城的附近。

%37
宋紫星的断臂被仙道杀招裹住,施展仙道杀招的蛊仙忙着追杀宋紫星,早已经将这个断臂遗忘脑后。随着时间流逝,一夜之后金钟消散,断臂就会爆成血雾。

%38
到那时,血雾笼罩方圆数百里,生灵涂炭,遗毒深远。

%39
之后数百年,鸳鸯城方面都会不断地发布任务,招揽贤才,深入血雾,斩杀血雾中不断孕育出来的血兽,防止血兽形成兽潮,冲击鸳鸯城池。

%40
前世方源辗转来到中洲时,就曾经在鸳鸯城生活过一段时间。为了生存,也接过血兽剿除的任务。

%41
从这个角度上来讲,方源这次来提前收服宋紫星的断臂,还是做了件好事。

%42
方源观察片刻,小心翼翼地施展几种凡道杀招试探。

%43
金钟虚影岿然不动,不愧是仙道杀招。

%44
方源又围绕着金钟虚影,不断转圈,从各个角度观察,试图寻找出合适方法。

%45
金钟虚影并不是纯粹静止,而是缓缓转动,里面断臂悬浮于空,臂膀手掌上还不时地闪烁一下鲜红的血光。

%46
方源转了十几个圈,心中已有定计。

%47
他正要出手,忽然双耳一动,听到许多声音。

%48
“爆炸……你们也听到了?”

%49
“……是……看到奇光……”

%50
“宝物……说不定……”

%51
声音断断续续,不断接近过来。

%52
方源顿时明白,这是附近的蛊师被吸引过来。

%53
这里位置处于真阳山脉边缘,荒兽鲜有涉足,又在鸳鸯城附近,有许多蛊师在这里活动。

%54
之前那蛊仙施展仙道杀招,金光冲霄而起,动静不小,因而吸引了许多蛊师前来查看。

%55
“一群蝼蚁。”方源冷哼一声,眼中闪过无情的光。

%56
他当即心念一动,从仙窍中冲出一个个的力道虚影。

%57
很快,这些力道虚影就成百上千,密密麻麻,围拢在方源的身边。

%58
方源每次行动,都会事先存备大量的力道虚影,储藏在仙窍内。因为施展万我,形成力道虚影大军是需要一定的时间的。

%59
“杀。”方源命令下去,近千只力道虚影顿时发动,急行而去,很快就都没入夜色当中。

%60
“什么东西?”

%61
“啊!”

%62
“结阵!快结阵!”

%63
“快跑……”

%64
惨叫声、惊吼声旋即传来,生死激斗在瞬间展开,仅仅僵持了几个呼吸的时间,蛊师们彻底溃败。

%65
力道虚影展开无情的追杀,蛊师们死伤无数,血腥气味很快随着风传到方源这边来。

%66
方源负手,凝望着眼前的金钟虚影,面具下脸色平静如水。

%67
他虽然想好了手段,此时却仍旧没有动手。

%68
欲速则不达。

%69
力道虚影虽然强悍凶猛,但却无法侦察,也没有增速追击的手段。说不定会有蛊师不死心,通过地遁或者隐身,再溜进来。

%70
不过,这些人都将死无葬身之地。

%71
皆因方源在此,已然催动了数十种侦察杀招,毫不停歇,一层层覆盖在这里。

%72
没有什么动静,可以逃得了他的查探。

%73
真元无限的好处,就是随意催动凡蛊,凡道杀招,永久不歇。

%74
而方源因为前世的积累,懂得的凡道杀招更加优异,甚至超出了这个时代。

%75
方源等了片刻,果然有三四位蛊师,潜了进来。

%76
方源操纵力道虚影,根据他的指点,一一围杀这些前来送死的蛊师。

%77
“不可能,怎么发现我的?”

%78
“饶,饶我一命!”

%79
“啊!”

%80
蛊师们无所遁形,均惨遭杀害。

%81
过了一会儿,周围再次安静下来。血腥气味越发浓重,附近稀疏的树林中连一声鸟鸣都没有,完全陷入死寂的状态。

%82
方源这才开始动手。

%83
万我第一式大手印!

%84
一只力道大手凭空而生,将金钟虚影握在手心。

%85
金钟虚影特别针对内在,进行镇压,外在的防御却并不强大。

%86
在增添了拔山、融合了挽澜之后,方源如今的力道大手印,可算是威力暴涨。只需紧紧一握,就可将金钟捏爆。

%87
但关键在于,捏爆之后,如何让宋紫星的断臂不在瞬间爆成血雾!

%88
好在方源不仅是力道宗师,更是血道宗师。他熟知的血道手段,不必宋紫星少!

%89
而且这一次,他也谋划了很久,准备相当的充分。

%90
当即,他用心布置,全神贯注。一连布下近百道血道杀招,紧密有序,相互呼应。更有大量蛊阵,铺设在周围,层层叠叠。

%91
铺设之后,他又细心检查,看看有没有布置失误的地方。

%92
检查出了一两处,进行修改之后,方源又第三次复查,第四次复查,第五次复查。

%93
事关重大,方源耐心十足。

%94
终于,他真正动手时,一切的变化都在他掌控之中。

%95
金钟虚影破碎,手臂刚要自爆,就被镇压,随即一层层的封印叠加上去。

%96
宋紫星断除这个手臂,施展血魔解体时,已然是强弩之末,为其付出的仙元恐怕都不到十颗。

%97
也是有这个主要原因,方源才有机会将这断臂搞到手。

\end{this_body}

