\newsection{最大赢家}    %第二百六十七节:最大赢家

\begin{this_body}

星宿仙尊的梦境,再一次在方源的眼前展现。

夜晚,冲天的篝火。

围绕着篝火,狂舞的兽人们,发出此起彼伏的野兽嘶吼。

这是一场狂欢。

狩猎之后的血腥庆典。

方源平淡地扫视周围。

他再度化身成一位人族孩童,被五花大绑着。和他相同遭遇的,还有许多的男童、女童。

“完蛋了,我们死定了!”

“呜呜呜……我不想被吃啊。”

孩子们嘤嘤哭泣,脸色无不绝望苍白。

方源眼中精芒一闪,忽然站起身来,高喊:“我要风结草!”

兽人族的庆典,正进行到热烈的地步。

方源的声音,突兀地插进来。

一时间,兽吼顿消,全部的兽人们都睁着血红的双眼,瞪过来。

巨大的压力,让人族孩童们都噤若寒蝉,瑟瑟发抖,有的甚至当场尿了裤子。

方源面无表情,又高声喊了一遍。

兽人族的首领怒哼一声,杀意澎湃地吩咐道:“给他!”

“小崽子,我会盯着你的。只要你不小心破坏了一点点草茎,我就用鼻子把你娇嫩的小身子,直接卷成肉酱骨渣。”一位象鼻兽人走了过来,将手中的风结草塞给方源,并且恶声恶气地威胁道。

方源轻轻一笑,望着手中的风结草一眼后,抬头道:“一个?这怎么够?这里有人族的多少俘虏,就给我多少风结草。我要解开所有的风结草,把他们统统都救下来!”

顿时。惊呼声迭传出来。

无论是兽人们,还是方源身边的孩童。都瞪大双眼,神情各异地看向方源。

短暂的震动之后。兽人们开始哄笑。

而刚刚涌起希冀振奋神色的孩童们,也都纷纷脸色委顿下来,换成担忧、绝望。

很快,所有的风结草都堆在方源的脚边。

整个风结草堆的高度,已然是方源此刻身高的数倍。

要用一己之力,在极其有限的时间内,将这些风结草都解开,这是不可能完成的任务。

除了方源之外,没有人相信他会成功。

众目睽睽之下。方源吐出一口浊气,开始着手拆解风结草。

解梦。

解梦。

解梦。

……

这个独特的仙道杀招,是以智道仙蛊解谜为核心,许多梦道凡蛊为辅助。

方源此时用来,收效极佳。

但解谜仙蛊虽然可以重复利用,但其余的梦道凡蛊,却会随着不断催用杀招而剧烈消耗。

幸亏方源之前,一直都没有懈怠。

基本上每天都会挤出时间,来进行自我梦境的探索。辛辛苦苦的一只只地炼制出梦道凡蛊。

这些梦道凡蛊积少成多,用在当下。正可谓:机遇都留给有准备的人。

“终于进入梦境的第二幕了!”方源暗自振奋。

这一次,他终于没有被梦境甩出去,得偿所愿地终于进入到了梦境第二幕之中。

原来。要想通过第一幕梦境,就必须将所有的孩童都解救下来。

方源虽然之前也尝试过,但事实上已经晚了一步。

在原有的梦境轨迹中。早就有一部分孩子被象鼻兽人杀死。

象鼻兽人听从兽人首领的吩咐,将这些孩童都带上来。但他故意失误。将树干砸在地上。

树干顺势碾死了许多捆绑的孩童们,孩子们的血肉遭到兽人们的哄抢。被尽数吞食。

这虽然不符合兽人族的传统,但也算是打个擦边球。

象鼻兽人乃是兽人部落里的著名勇士,兽人首领等人也都容忍了他的这个举动。

方源在收服藏经鼋、幽魂草的时候,想明白了这个关窍,将所有的人族孩童都救下,终于大功告成,进入了第二幕梦境。

“通过了第一幕,我的智道境界恐怕要达到大师级了!”

“之前频繁催动解梦杀招,梦道凡蛊消耗甚多,接下来必须小心节省。”

“可惜之前顺着墨瑶意志得到的情报,去探访那株春梦果树。结果在半年前,这株树刚被无知的凡人砍伐。否则依靠这株树,我现在手中的梦道凡蛊数量至少还能多十倍!”

“第一幕,应当是星宿仙尊童年时期的记忆,衍生出来的。也许在她一生中都是一个遗憾。我救下所有孩童,算是弥补了遗憾,所以成功通过了第一幕。”

“那么,接下来的第二幕,又会是什么情形呢?”

方源压住翻腾的思绪,打量周围梦境。

这是一座山。

寂寥的夜空中,闪烁着点点稀疏的星光。

仿佛是夏天时节,暖风徐徐地吹拂,带来山林中郁郁葱葱的林木清香。

山溪潺潺的声音,以及风过山林的沙沙响声,还有夜莺的歌唱声,混杂在一起,组成大自然无须用任何言语去修饰的天籁。

和之前第一幕的血腥、凶残、险恶相比,第二幕梦境简直是温柔如水,和风细雨。

“快走吧,你还楞着干什么?”

“只要我们在星盘棋局上,走过六步,我们就能得到仙尊大人的亲传!”

“成为仙尊大人的弟子,我们就能学会本领,成为强者!到那时,我们要为亲人们报仇,杀光那些该死的兽人。”

周围的孩童,见方源停下,一个个催促道。

方源顿时了然,有些明悟第二幕的通过条件。

“历史传闻中,也有记载。人族历史上的第一位九转蛊仙,号称元始仙尊。他洞悉部族制度的弊端,为了开创门派制度,以身作则。曾在多处地点布设星盘。只要在星盘棋局上,连走六步。就能成为他的亲传弟子。”

方源回忆着,随着周围同伴。一起攀登到峰巅之处。

在那里,他看到了著名的星盘棋局。

星盘棋局被刻在一块巨大的,表面平坦的巨石上。

石面上,一道道线路,或是横切,或是纵劈,或是斜插。

每当微风吹过时,这些线条上就会闪现一丝丝的湛蓝星芒。

待方源稍稍靠近,这些星线就直接透过他的双眸。在他的脑海中浮现而出。

与此同时,他的魂魄消耗程度,骤然剧烈了数十倍!

不少同伴,都停滞不前,有的甚至当场昏倒。

方源屏气凝神,终于走近巨石,缓缓地伸出手掌,贴在石面上。

轰的一声。

他双耳嗡鸣,幻觉大盛。

那丝丝星线。陡然扩张成一条条街道。而方源周身绽射星芒,已经成了一颗星棋子,正停顿在这些星路交织的网中。

星盘棋!

方源观察片刻,满头冷汗。

他考虑半天。犹豫地踏出一步。

下一刻,他的魂魄被甩出梦境,归于肉身。

魂魄重伤!

一步踏错。满盘皆失。

方源身躯摇晃,直欲栽倒。这一次魂魄受创。比之前凤金煌那次都要严重,几乎掉了他半条命!

“第二幕梦境。比第一幕要凶险百倍!”

方源的脸上闪过一抹震惊之色,旋即又平复下来。

“不过,我已经打通第一幕,收获极大。智道境界,果然已经达到了大师级别。”

“时间还充裕,不急。先用胆识蛊休养,再徐徐探索。”

就在方源盘坐疗伤时,隔着外显梦境的对面,凤金煌刚刚结束疗伤,睁眼站立,意图再做尝试。

“咦,怎么梦境又缩减了?”

“而且这次缩减了好多,将近三分之一了!”

凤金煌凤目瞪圆。

方源通过了第一幕,收获巨大,外显梦境的第一幕因此永久消失了。

这一部分的梦境,已经尽数化为养分,滋养了方源。

“难道说有人捷足先登,探索成功了?”凤金煌念头一转,自己先轻笑起来,“怎么可能?母亲借助门派之力,已经推算过,确认中洲中只有我拥有唯一的梦道仙蛊。看来这处梦境,真的很特别,会随着时间逐渐消融。”

“我得尽快了!”凤金煌眯起双眼,催动梦翼仙蛊。

梦翼扇动,带着她的魂魄,深入梦境。

一进来,凤金煌便大喜过望。

“这显然是更深一层的梦境,太好了,我终于不用面对那该死的风结草了!”这一刻,凤金煌在梦境中雀跃欢跳,开心极了。

然而片刻之后,凤金煌惨败而归。

残破的魂魄归体,她面如金纸。

“风结草没了,但却有更复杂的星盘棋……”她恨恨地看着眼前的梦境,外显成实体,散射着七彩绚丽的光彩,华贵且神秘。

憋屈和愤怒,充斥心胸,却无处发泄。

凤金煌曾经在梦境中收获许多好处,梦翼仙蛊在手,更让她生出一种“梦境也不过如此”的轻敌想法。

有史以来,她还从未栽过这么大的跟头。

面对这片梦境,她已经花费数天的时间,期间大大小小受伤数十次,结果一丁点的斩获都没有!

“星宿仙尊,我记住你了……”凤金煌咬牙切齿,随后双眼一翻,终于撑不住伤势,当场昏迷过去。

时间匆匆一晃,已是到了约定的时限。

所有进入碎片世界的蛊师,都走出来,回归到各自门派的蛊仙身边,进行汇报。

很快,众仙的目光都集中在方源和鹤风扬的身上。

这一场竞争,无疑是仙鹤门独占魁首,不仅吃了肉,甚至就连点肉汤都没有留给其他人。

这样的情形,方源自然不愿久留。

当场和鹤风扬完成交接之后,就带着龙鱼,离开了这处山谷。

方源闷声发财,他深知自己才是此行最大的赢家!i527

\end{this_body}

