\newsection{榜眼}    %第二百零九节:榜眼

\begin{this_body}

%1
“现在我可以确定,鹤风扬这是在向我示好。”方源产生了一股明悟。

%2
鹤风扬主动邀请方源,前往飞鹤山看看。

%3
飞鹤山是什么地方?

%4
是仙鹤门的山门所在。虽然象征意义,大过实际意义,但这话的内涵已经表达得十分清楚了。

%5
回到狐仙福地,方源琢磨鹤风扬的微妙态度,忽然无声地笑了。

%6
和凤金煌赌斗,方源和灵缘斋建立了贸易往来。这笔胆识蛊的买卖,优先度甚至高于仙鹤门。

%7
但因为借着中洲炼蛊大会的虎皮,又是众人见证,就算是仙鹤门也挑不出什么借口,去阻止这场意义不同寻常的交易。

%8
狐仙福地和灵缘斋勾搭上了,这让仙鹤门产生了一股危机意识。

%9
历史上,灵缘斋拉拢魔道蛊仙的例子是十派中最多的。诸如薄青、凤九歌等,都是深入人心的例子。

%10
在前不久,收复狐仙福地失败的前提下,方源又展现出极强的炼道造诣,仙鹤门终于决定对方源以怀柔拉拢为主。

%11
所以,鹤风扬对方源的态度,发生了转变。

%12
方正的事情,应当是真的。鹤风扬此举,毫无疑问,是对方源的一种示好。更以夺舍法门,对方源展开利诱。

%13
对于他而言,方源亢方正,天庭必定会攻破狐仙福地。其中的荡魂山,如果没有被方源提前毁掉,天庭肯定会收到自己手中。狐仙福地不能搬走,天庭可能也看不上。会将它留给十大古派中的一个。

%14
至于哪个古派,就该看天庭内部的派系竞争的结果了。

%15
很显然。仙鹤门上头的天庭蛊仙,势力较低。在竞争中夺回狐仙福地的可能不大。

%16
就算夺回来,没有了荡魂山,狐仙福地的价值也大大降低。

%17
不管怎么算,这有违鹤风扬、仙鹤门的利益。

%18
所以鹤风扬告知方源此事,也是对自己一方利益的维护。

%19
“现在我名义上是仙鹤门的附庸,但却在灵缘斋、仙鹤门之间左右逢源。这一下子,顿时让我的外部压力骤然减少了一半还不止啊。”

%20
方源感叹道。

%21
事实上,方源早就在谋算这样的情势了。

%22
他先前置之不理凤金煌的挑战,后来又主动沟通。答应挑战,为的是什么?

%23
就是想借助灵缘斋的势,去对抗仙鹤门的强势。

%24
方源达成了之前的谋划目标,但起关键作用的,却不是他。而是胆识蛊本身的诱惑力!

%25
方源抢夺了狐仙福地,导致凤金煌受到刺激,提前发现了梦翼仙蛊的真正作用。而正是因为凤金煌使用了梦翼,尝到了好处,才更加看重胆识蛊。若非如此。灵缘斋方面怎么可能让方源轻松借势?

%26
对方又不是傻子!

%27
没错,方源是故意打成平局的。为的就是展现一个温和的态度,给灵缘斋的人看。

%28
对于方源而言,因为这场赌斗。自己在中洲的处境发生了前所未有的改观,让他压力大减,能有更多的精力去筹谋其他大计。

%29
当然。一旦方源破坏八十八角真阳楼的秘密曝光,中洲局势必定骤然崩坏。方源将人人喊打,亡命天涯。

%30
对于凤金煌而言。她虽然在赌斗中受伤溃败,但却是最大的赢家。

%31
今后有了胆识蛊的辅助,她在梦翼仙蛊的帮助下,必将突飞猛进,以超越几乎所有蛊仙的速度,急速积累成长。

%32
“前世我杀了凤金煌,但是今生,在目前看来,反而是我成就了她啊。”对此,方源感慨很多很深,不足为外人道。

%33
“方源,事情已经办妥,没有其他的事情的话,我这就离开了。”太白云生过来告辞。

%34
他这些天,在北原四处行走,寻找合适的地点落下福地。

%35
本来已经找好了地方,就要落下福地了,却在关键时刻,被方源唤回来。

%36
“这次多谢你了。你选的地点很好,放心去调节天地二气吧。等到灾劫来临时,我必会来助你一臂之力。”方源对太白云生道。

%37
太白云生点点头,告辞了方源,利用星门去往北原。

%38
方源得了星象福地之后,星萤蛊已经储量极巨,星门的建立和开启再不像之前那么紧张了。

%39
方源来到密室,再次见到了古月方正。

%40
古月方正不是已经被鹤风扬杀死了吗?怎么还好端端的活着呢?

%41
原来,这只是方源安排的一场把戏。

%42
他故意将古月方正的手臂切断一条,安装狗肢,做出拷打侮辱的假象。

%43
实则将这个断臂留在这里,在鹤风扬杀死方正之后,被方源唤回此处的太白云生利用人如故仙蛊,耗费了一笔颇为可观的仙元,针对断臂,将方正复活。

%44
鹤风扬不疑有他,他也有局限性,不知道这个世间会有人如故这种仙蛊存在。

%45
在他看来,方正的肉身、魂魄都在自己眼前被彻底毁灭,方正确确实实已经死了。

%46
换做其他蛊仙,恐怕也会如此认为。

%47
这是常识。

%48
但这个世界很大,总会有各种稀奇古怪的仙蛊、仙道杀招等等。就像人如故,他们不知道,人都是有局限性的,就算是仙人也不例外。

%49
全知全能,就算是九转蛊仙,乃至人祖也做不到。做到这一步的,也许只有一个伟大的存在,那就是这个世界!

%50
方源将一只信道蛊虫,抛给方正,脸上露出嘲讽之意:“你好好看看,这就是你感恩戴德的仙鹤门,还有你那好心的师傅。如果不是我,你已经死了。”

%51
“不可能!我不会相信你的谎言!!”方正发出怒吼,但终究还是将这只信道蛊虫取到手中。

%52
这只蛊虫中,自然记录着鹤风扬杀死方正,以及和天鹤上人魂魄对话的景象。

%53
方正看完之后,立即将信道蛊虫捏成碎片,整个人陷入到无比的愤怒状态。他仰头嘶喊,面容扭曲,手指着方源,对他咆哮:“不可能!这都是假的,是你伪造的。我绝对不相信,我永远都不会相信!你骗我,你骗我!!除非我自己亲眼看到,亲耳听到,亲自得到师傅的回应!”

%54
“你会亲眼证实这一切的。不过……不是现在。”方源转身,离开密室。

%55
他关上牢门,密室中便重新陷入黑暗,方正却在黑暗中连连怒吼,疯狂地在有限的空间里踱步,对墙壁拳打脚踢,歇斯底里。

%56
处理了方正的事情之后,方源在中洲炼蛊大会上,一路凯旋高歌。

%57
他提前淘汰了真正的炼蛊强者,导致大会后期反而比中期容易。

%58
最终,凤金煌取得第六名的成绩。而方源进入最终的决赛。

%59
他迎来了此场大会最强的对手余木蠢。

%60
这一场,方源败得干净利落。

%61
即便比试的题目,和前世没有什么两样,方源占据先知先觉的巨大优势。

%62
但对方余木蠢却是货真价实的炼道准大宗师。

%63
这个境界,就恐怖了。

%64
就算是凤金煌有梦翼仙蛊,也不会轻易达到这等境界。流派的境界越是往后,晋升难度就以几何倍数提升。

%65
整个历史上,超越他的只有三人,就是炼道三老。

%66
整个北原,当今公认的只有四位能和他比较,就是北原炼道四能。

%67
方源深知,就算自己有着知晓试题的优势,也赢不了余木蠢。这是实力的差距。

%68
说起来,这个余木蠢和方源打扮还差不多,同样是一身黑袍,笼盖全身,脸上还带着宽大的面具。

%69
方源五百年前世,就是他夺得了第一。没有任何意外,他的实力是真正的深不可测。

%70
方源主动认输,被许多人指责。毕竟炼道境界很难具体表述出来,余木蠢的确很强,方源在世人眼中也是相当的强势啊。

%71
如此不战认输的行径,让大为期待最终决赛的观众们十分不满,一度流言喧嚣尘上,认为方源不战认输,这是大会黑幕!

%72
不管怎么说,获得第二的方源,达到了他原本的目的,身上印刻下了一道成功道痕。

%73
事后,仙鹤门等十大古派都接连派人,接触方源,企图收购他身上的道痕。

%74
这个成功道痕,是无法夺取和转移的。就连天庭都做了许多努力,可惜无一次成功。十大古派自然没有夺取的心思,只谈交易。

%75
反正方源有这么厉害的炼蛊手段(仍旧大大高估了方源),让他炼蛊。若炼制六转仙蛊,便会消耗掉这道成功道痕,换来的是零失败率。

%76
不管中洲十派开出何等高价,方源都一一婉拒。

%77
回到狐仙福地,他便做炼制变形仙蛊的最后准备。

%78
方源自从有了炼制变形仙蛊的计划,就一直在有意识地筹集炼蛊材料。

%79
因为有成功道痕加身,方源只需要筹集一份炼蛊材料即可。这让他的支出大为减少。

%80
饶是如此,等方源筹集妥当,从北原拍卖大会得来的众多仙材已经卖的卖,换得换。若非有炼道大会中得来的一些材料及时填补,否则还会有缺口。

%81
炼制仙蛊的难度,可见一斑。

%82
方源只是筹全一份材料,手中积累的资源储备,几乎为之一空。

%83
若非是有胆识蛊等贸易,若不是有成功道痕加身,方源绝不会行险,做出如此孤注一掷之举。

\end{this_body}

