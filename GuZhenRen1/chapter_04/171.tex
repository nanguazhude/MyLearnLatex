\newsection{要炼仙蛊无财力}    %第一百七十一节:要炼仙蛊无财力

\begin{this_body}

ps:万星飞萤这招要使出来,消耗的道痕至少是十六道,并且没有上限。万星飞萤囊括的范围越广,威力越大,耗费的道痕就越多。

不过也正是如此,才使得此招分外强悍。

道痕镶嵌空中,形成特殊战场,还很难被破坏。蛊仙进入其中,一思一绪,都会变成星念,从而形成万星飞萤。时间拖得越久,万星飞萤的威力就变强。这是将敌人的力量,巧妙地化作自己的力量,再攻击敌方。

对于敌方蛊仙而言,感觉就分外恶心了。仿佛是搬起石头砸自己的脚,拿自己的刀戳自己一样。

明白了万星飞萤的运转秘密之后,方源又明白一点。

难怪历史上东方长凡,基本上都以算计为主。真正到了不得已的时候,才会出手战斗,动用这招万星飞萤。

因为就算是东方长凡,北原当代第一的智道蛊仙,也不能将万星飞萤当做常规手段频繁使用它的消耗太大了。

道痕一旦消耗掉,当然不会自己回复。道痕的来源也很狭隘,绝大部分的蛊仙只能从渡劫中,获得大量道痕。

“这招万星飞萤,几乎和我现在的杀招万我不相上下。当然我的万我杀招。还有广阔的发展潜力,在这点万星飞萤是远远比不上的。我身上没有智道道痕,除非将来走智道这条路子。否则终其一生恐怕也使不出这个杀招了。”

当然除此之外,还有一个关键因素。

那就是这个万星飞萤杀招,需要三只智道仙蛊为核心。

方源是一只都没有。

东方长凡落败之时,就将所有的仙蛊都引爆了,没有半只留给方源。

这三只智道仙蛊中,方源最想得到的,便是星念仙蛊。

这份智道传承中。包含了仙蛊方十多张,其中就有星念仙蛊的蛊方。甚至还有历代继承者如何炼制出来的心得体会,以及许多的失败感想!

一旦有了星念仙蛊,方源就无需大废功夫,慢慢地囤积星念凡蛊了。

按照传承中的记载:一颗青提仙元消耗下去。经过星念仙蛊的作用,至少有十万星念产出。

当然,方源要是催用星念仙蛊,得到的星念数目一定会少于十万。

因为,历代的智道蛊仙继承者们,身上都有智道道痕,增幅着星念仙蛊的效果呢。就像方源有力道,使用大手印,足足能增长两成威能。

两成还算是少的。通常的蛊仙强者,身上道痕都能令仙蛊威力、仙道杀招的威力成倍的增长。一倍、两倍、三四倍,乃至十倍。百倍!

八转七转为什么差距这么大,道痕就是一项主要的原因。

修为越高,劫灾越恐怖,一旦渡过,身上获得的道痕就越多。

福兮祸所伏祸兮福所倚,换个角度来看。天劫地灾威力越强,也是某种好事。

可以说。一旦有了星念仙蛊,方源在推算方面,阻碍就大为减少了。再不用精打细算,克扣着星念小心翼翼地耗用了。

而且凭借方源此时的财力,他也能消耗得起了。

但可惜的是,方源能支撑得起使用仙蛊的消耗,却没有财力炼制仙蛊。

就算发了几笔财,他距离炼制仙蛊还是有很长一段差距的。

仙蛊唯一,乃是大道法则碎块,炼制仙蛊耗费的仙材、仙元石极多。

更关键的是,炼仙蛊,你不可能只准备一份材料,你得准备很多份,因为炼制仙蛊的成功率太低了!

东方长凡乃是一代枭雄,北原当代第一智道蛊仙,执掌超级势力多年,又暗中和仙鹤门交易,得到不少资助。就算如此,他手中的仙蛊在鼎盛时期,也只有八只。

八只当中,还有经血仙蛊、分影仙蛊这种旁门。智道传承中记载的十几张仙蛊方,他炼成的只有一小半。

像雪胡老祖、万寿娘子,前者现在是北原八转战力第一,后者是北原四大炼道蛊仙之一。算得上底蕴深厚,家大业大了。

但要炼制八转鸿运齐天仙蛊,即便已经有了马鸿运,可以省却很多过程很多资源,但这样筹措下来,仍旧要耗费掉几乎全部的家底,可谓孤注一掷!

方源现在的情况,想要炼制仙蛊,那是想都别想了,一想就纯粹是痴心妄想。

除了星念仙蛊之外,方源最看中的还有一记仙道杀招,名为星雾掩。

此招用处,能使得蛊仙身上常年笼罩住一层星光薄雾,遮蔽面容身形,让外人看不分明。

当然这还只是外观效果。

真正的主要效果,在于能遮掩天机,让其他智道蛊仙推算这位蛊仙时,朦朦胧胧,得到似是而非的答案。

这几乎和暗渡仙蛊的效能相差不多了。

暗渡仙蛊其实是忽略,叫人推算不出来。星雾掩杀招是模糊,推算困难,答案不准确。

这份智道仙级杀招,在整个智道传承怀中,属于较为偏门的那种。很多继承人都没有使用的经历,因而记载的使用经验很少。

因为智道蛊仙本来就稀少,智道蛊仙相互算计、破局智都的次数更少了。大多数情况下,就算用了,也不可能从敌方哪里获知到推算感受的。

不过对于方源而言,却是十分适合的。

他得到智道传承的目的之一,不就是为了抗衡其他智道蛊仙,大大拖延自己犯案的真相被挖掘出来么?

可惜的是。这仙道杀招需要用两只智道仙蛊,充当核心。

所以难题又再次归结到了一点方源没有炼制仙蛊的丰厚财力!

最终,方源纵览整个智道传承。发现他目前能够学习并加以利用,又正适合他自己的,就只有一块内容。

那就是怎么样对付敌方意志。

这块内容很多很丰厚,如何防御敌人的意志,如何归辨意志种类,如何控制敌人意志,万一被敌人意志控制又何如反控制。还有如何用意志攻击,意志大战的注意事项。如何侦察意志,意志如何治疗,怎样增幅意志,怎样削弱意志等等。

方源为什么要学习这部分内容呢?

因为他的手中。还关押着一股墨瑶意志!

墨瑶生前炼道造诣惊人,又是墨人,掌握许多墨人秘辛,又和薄青有着情缘,是灵缘斋的那代仙子,知道的历史秘辛自然价值极大。其中最令方源心动的,当然是她掌握在手中的有关红莲魔尊的传承线索。

方源的第一本命蛊春秋蝉,就是当年红莲魔尊的本命蛊!

于是这部分传承内容里,方源又重点关注学习如何搜意。

搜魂和搜意。两者完全是不一样的难度。

前者容易,宛若吃饭吃菜,人人都有一张嘴。方源不是魂道,照样能搜刮太白云生之魂。外行人都能插一脚。

而搜意就像是大厨吃菜,需要品出这道菜的主材,辅材,佐料,还得品出这道菜的制作步骤。这个难度就大了去。一句话,这是一个技术活儿。非得是智道专业人士出手。

于是,当太白云生回到狐仙福地的时候,方源正在荡魂行宫中,孜孜不倦地操练着数种的搜易手段。

他才刚刚学习,毫无前世经验可以帮衬,距离运用还有段距离。

“师弟,别来无恙乎?”太白云生微微带笑,在东海他压力甚大,如今面对方源,他才真正地放松下来。

患难见真情,对于他而言,有着方源的狐仙福地,就仿佛是家一样。

方源示意他座下,看他衣衫破烂的狼狈模样,有些意外地道:“你受了伤?”

太白云生苦笑,尤有余悸地道:“我现在不同初入东海,已经引人瞩目,不能直接通过星门回来,便去界壁装装样子。没想到还真有人跟踪,此行也算是彻底见识到了地潮的恐怖威能了。界壁虽然因为地潮而薄弱,但在界壁附近,无形潮力盘踞,防不胜防。我不过是在外围,就狠狠地吃了几次苦头。非得是仙道防御杀招才能有所效果,我的九云环在无形潮力面前根本是纸糊的一般。嗯,不谈这个了,听说师弟你此番大有斩获?”

“哈哈哈。”方源便笑起来。

他当然不会告诉太白云生,自己这个“假师弟”,杀了他的“真师兄”。

不过方源也只隐瞒了一些事情,大部分真相都告诉了太白云生。

太白云生听完,对于东方长凡这样的人物,也是唏嘘一阵。

他又和方源交流了自己的近况。

总体而言,太白云生在东海那边十分吃香。发展的真是不错,照着这样的趋势下去,几乎又是另一个黎山仙子了。

太白云生又谈及玉露福地。

鲨魔数次攻略,碰到无数难关,到了现在都未真正杀进玉露福地去,而是被外围的关卡所阻。

方源大有感慨:“玉露仙子当初可是八转蛊仙,师承乐土仙尊。玉露洞天跌落成玉露福地,但虎死威犹在,恐怕是这天底下最难攻伐的福地之一了。其实影响福地防御的因素,有很多。其一福地来源是否统一,其二有无地灵,其三仙元是否充沛,其四有无荒兽等等护卫,其五有无仙蛊,其六仙道杀招,其七有无仙蛊屋,其八有无蛊仙助阵等等。然而依我所见,福地攻防,归根结底只有一条,便是道痕多寡纯杂。”

太白云生听方源说话,原本只是点头,皆因前言都是老生常谈。但当他听到方源最后一句仿佛画龙点睛的话时,却是心中一动,脸上涌起感兴趣的神色,便开口道:“愿闻其详。”

方源却不答,反问太白云生一句:“老白你可知道‘人乃万物之灵,蛊是天地真精’这话?”

ps:今天4月7号,不知不觉间,双更已经一个星期了。真诚的感谢每一位投我月票的朋友们。感谢偽狐?、独步青霄、邓方槐、掌心扑火四位同学本月累计已过万的起点币打赏。感谢打赏的每一人,名单长,这里不细表,我都一一记在心中。蛊真人在这里向诸位鞠躬道谢!(小说《蛊真人》将在官方微信平台上有更多新鲜内容哦,同时还有100\%抽奖大礼送给大家!现在就开启微信,点击右上方“+”号“添加朋友”,搜索公众号“qdread”并关注,速度抓紧啦!)(未完待续)

\end{this_body}

