\newsection{三大鱼群斩获丰}    %第一百五十一节:三大鱼群斩获丰

\begin{this_body}



%1
扑通!

%2
荒兽龙鱼,重重地砸入水中,顿时在潭水中激起一股高达数丈的巨浪。

%3
方源一身黑袍,遮住面目身形,傲然站立在高空,身旁的力道巨手多达六只,围绕着他不断盘旋。

%4
荒兽鱼龙嘶吼一声,落入水中后,却不敢再战方源,反而往水中深处游窜。

%5
方源哈哈一笑:“你以为躲进水里,我就拿你没有办法了吗?仙蛊挽澜!”

%6
方源雄躯一震,伸出左右手掌,遥遥对准广阔如湖的深潭,慢慢用力虚抓。

%7
一股无形的巨力,将潭水抓取上来,仿佛抓住一匹厚实绵长的绢布。

%8
粗壮的水流,夹裹着了无数的普通龙鱼,迅速地灌入方源的仙窍之中。

%9
方源告辞了黑楼兰、黎山仙子之后,没有花费多久,就发现了这处深潭。

%10
这深潭是他有史以来,见到最为宽广的潭水,大如湖泊。东方部族特意开辟,在里面豢养了九十多万只龙鱼,更孕育出一头荒兽龙鱼。

%11
方源见猎心喜,出手搜刮。荒兽龙鱼挺身而战,结果被方源轮番打出力道巨手,轻松打败。

%12
龙鱼虽然占据一个龙字,却十分平庸。这只荒兽龙鱼,虽然是六转层次,但没有仙蛊傍身,几乎是荒兽中垫底的存在。

%13
但龙鱼的肉,却是应用范围最为广泛的食料。不仅是野兽、异人可以吃。而且还可以作为辅助食料,喂养蛊虫。

%14
在宝黄天中,龙鱼肉的买卖,一直都有蛊仙在做,每一笔的交易量都是极大。

%15
“好多的龙鱼。这一潭龙鱼,恐怕就算是东方一族,也要积累上百年。现在却便宜了我,哈哈哈。”方源心中欢喜,催动挽澜仙蛊一刻都不停歇。

%16
幸亏有挽澜仙蛊,否则他还真没有太好的办法,收取这笔丰厚的资源。

%17
轰轰轰……

%18
远处。遥遥传来蛊仙激战的声音。

%19
方源一边收取龙鱼。一边转头望了望,目光便有些深沉,心道:“看来是黑楼兰那边,对嫣然海进行强攻了。”

%20
他并不后悔放弃方寸山的决定。

%21
这世间之事,向来都是知易行难。大道理谁都懂,但真正要做时,又有几人能够达到?

%22
所以重利当前。真的要取舍,是很难的。

%23
尤其是方寸山,对于黎山仙子而言,具备极大的价值!

%24
黎山仙子虽然本命仙蛊是山盟蛊,是信道蛊仙。但观其作战,却多是木道手段。显然兼修了木道,仙窍的经营恐怕也是花草居多。

%25
《人祖传》中的记载,更增添了方寸山的吸引力。

%26
但要夺走方寸山,怎么可能会如此容易?简直太难了,希望十分渺茫。

%27
那位魔道蛊仙既然如此轻易地。就将这个消息告诉方源,那么他也会更轻易地告诉他人。说不定,已经有蛊仙更早知晓了。

%28
攻打嫣然海禁地,动静一定很大,怎么可能瞒得住皮水寒这些人?

%29
人多虽然力量大,但魔道蛊仙各自为政,什么作风性格。方源再清楚不过了。不久前太丘之战,若非魔道蛊仙之间相互扯皮内斗,东方长凡怎可能如此轻易地夺舍成功?

%30
要夺取方寸山,基本上是没有什么可能的。黎山仙子、黑楼兰难道不知道可能十分之小吗?

%31
只是这利益实在太大了,大到让蛊仙心动至极,哪怕仅有一丝丝的希望,都要拼命抓住!

%32
一声龙吟,打断方源的思绪。

%33
巨大的龙鱼从水面下陡然扑出,双眼通红,杀向方源。

%34
方源对此早有预料,四只力道巨手齐齐拍下。

%35
他这样抽取潭水,效率极高,不需多时,就能将潭水抽空。换言之龙鱼已经成了绝路上的野兽,此时若不反击,那就彻底迟了。

%36
但面对七转战力的方源,龙鱼的绝地反击,很快就遭受重创。

%37
十几个呼吸之后,伤痕累累的荒兽龙鱼,被方源的力道巨手抓入仙窍。

%38
又花费片刻功夫,这处巨大的深潭基本干涸,里面的龙鱼全数被方源收入囊中。

%39
为了收取这群龙鱼,足足耗费了四颗青提仙元,但此时方源眼中透出一股喜悦。

%40
自从他打进碧潭福地,斩获之物还有油水、荒兽鱼翅狼、幽火龙蟒、一些常规资源,以及一批长恨蛛群。

%41
要论价值排行,这批刚刚到手的龙鱼群,已经跃升第一,并甩开第二位的荒兽鱼翅狼一小截。

%42
这批龙鱼数量太大了,有近百万条。

%43
如此雄厚的根基,只要稍微用心培养,就是一条源源不断的盈利渠道。

%44
黑楼兰那边的激斗轰鸣声,仍旧远远传来,方源根本不去回望,朝着更远方飞射而去。

%45
不多时,他倏地停下身形,惊喜地看着脚下的一片茂密森林。

%46
森林遍布方圆数百里,森林的中央只有一个房屋大小的小水潭。

%47
整个森林,都是为了这个小水潭而栽种。

%48
准确的说,是小水潭中养着的一群奇异鱼种散文鲤。

%49
“散文鲤和真武鲤相对,并称为文武双鲤,想不到东方部族手中,居然豢养了十几条散文鲤!”方源带着欣喜之情,迅速扑下。

%50
散文鲤十分难以捕捉,且数量稀少,在东海都不多见。

%51
散文鲤平时静止在水中,随着水波荡漾。一旦甩动鱼尾,悠悠游动时,整个身体就会彻底化为一股文气,就算是蛊仙,没有针对的手段,也捕捉不得,只能望而惜叹。

%52
方源的到来,让散文鲤受到惊吓,顿时化为一片文气,氤氲交织,弥漫在水中。

%53
方源的嘴角浮现出自信的微笑,他的挽澜仙蛊可以操纵水流。散文鲤始终游不出潭水,对他而言,就是瓮中之鳖。

%54
但就在他要出手的时候,一位魔道蛊仙显出身形,升腾而起。

%55
“前辈请住手,这处资源已早被晚辈占据了。”这位魔道蛊仙身材削瘦,笑眯眯,只是六转蛊仙,但面对方源有恃无恐的样子。

%56
“你是何人?”方源神色不变。

%57
“晚辈周千。”魔道蛊仙微微拱手道。

%58
方源狞笑,恐吓道:“小辈周千,胆子挺大,居然敢阻挡老夫。以你的修为,还不配和老夫讲规矩。”

%59
魔道蛊仙周千却笑道:“不然,不然。晚辈当然打不过前辈,但晚辈却在飞遁上颇有心得建树,有两个仙道杀招。就算是自在书生、黑城、女仙乐瑶都追不上晚辈。前辈既想要散文鲤,不妨与晚辈做个交易。否则晚辈虽然打不过前辈,但却可以在顷刻间毁掉这片潭水的。”

%60
方源沉默,眼中凶芒烁烁,不住地打量周千。

%61
周千泰然自若,有底牌自然有底气。

%62
这也是擅长移动的蛊仙的优势,往往立于不败之地,可以和高修为高战力的对手从容周旋。

%63
方源虽然有万我杀招,但移动方面却比较薄弱。就算是有铁冠鹰力仙蛊,配合返实蝠翼,但并不是仙道杀招,和周千还有差距。

%64
“很好,周平,老夫记住你了。希望你一直留在这里,今后不要落在老夫的手上。”方源抛下一句狠话,转身即走。

%65
周平闻言,原本的笑容消失了,变声道:“前辈,何必如此?晚辈只是想做个买卖罢了。晚辈其实要求也不多的……”

%66
方源却没有再听周平多讲,身形如闪电般,迅速远离。

%67
周平皱起眉头,心情有些懊丧,凭白无故地得罪了一个实力强大的神秘蛊仙,让他感觉不太好。

%68
更关键的是,方源的话击中周平的要害

%69
他不可能一直停留在这里,看守这处散文鲤的。

%70
“可恨我没有手段,收取这群散文鲤啊!”周平望着深潭跺脚,十分郁闷。

%71
方源离开了周平,一路疾飞。

%72
足足飞了近千里,他再次顿足,俯瞰脚下深潭,颇有欢喜之情。

%73
这口深潭不大不小,里面也有鱼群。

%74
这种鱼有点类似于金鱼,鱼肚滚圆,两头尖小。但没有金鱼斑斓的颜色,大多是乳白色。也没有金鱼如裙摆般的鱼鳍、鱼尾,它们的鱼尾鱼鳍都很小巧精致。

%75
它们游动起来,不是左右前后,而仅仅只是上下沉浮,好像是水中的气球。

%76
这种鱼,方源也豢养过,名为气泡鱼。

%77
只是后来一段时间,方源过得艰难,手头极紧,不得不将养了一段时间的气泡鱼群,贩卖许多出去,换来资金救急。

%78
气泡鱼能吞吃普通虫子,一部分消化,一部分当做储备粮食,蕴藏在体内。

%79
这些虫子在鱼腹中,得到营养滋润,反而有很大几率,突破形成蛊虫。

%80
因此,气泡鱼有能增加虫群产蛊的奇能。在五域大战时期,成为各个蛊仙竞相追逐的货物。

%81
“这气泡鱼群规模不小,若论宝黄天的市价,已经有荒兽鱼翅狼的一半了!”方源左右查看一番,没有周平之类的魔道蛊仙出现,他立即出手,催动挽澜仙蛊,将这群气泡鱼连潭水一锅端。

%82
这支气泡鱼群便成了方源手中,价值第三的收获。

%83
“一群贼子!光天化日之下,敢来我东方一族抢劫!!”这时,东方长凡惊怒的吼声,陡然传遍整个碧潭福地。

%84
自在书生的爆喝,也旋即响起:“东方老贼,留下你的狗命!!”

\end{this_body}

