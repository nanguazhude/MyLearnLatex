\newsection{拍卖大会(中上)}    %第九十九节:拍卖大会(中上)

\begin{this_body}

%1
密室中,方源饶有兴趣地把玩着手中的蛊虫。

%2
这些凡蛊,自然是秦百胜所有,但已经主动借给密室主人,任由方源催用。

%3
他调动一股真元,试着灌注进去,密室中的环境顿时发生改变。

%4
“看来秦百胜的确用了心思。说是密室,却是一方小天地,比园林还大。又可动用这几只凡蛊,随意改变这里的环境,变出高山流水,亦或者说海洋湖泊,亭台轩榭等等。”

%5
随着方源的调动,密室周围的环境不断发生变化。时而青山葱茏,他置身在山峰上的凉亭中。时而大河滔滔,他处于江边楼船里。

%6
当然这些变化,都是幻影,做不得真。若要真的,如此顷刻间就能改天换地之举,必定是用仙蛊威能,要耗费不菲的仙元。

%7
方源又将手边的蛊虫,取到手中,投入心神观看浏览。

%8
这只东窗蛊中,记录着许多拍品,皆是珍稀宝物。方源看了,也有眼花缭乱之感。

%9
有些拍品后面,还附写了相应的条件。除了都可以用仙元石拍买之外,还可以利用一些物资换取。

%10
显然这些物资,都是物品主人想要寻求之物。

%11
方源对前面的内容一扫而过,重点关注后面的部分。

%12
这部分内容,列举了众多的仙蛊,五花八门,涉及多种流派。在这些仙蛊后面,均标注着换取信息。

%13
和之前的拍品不同,仙蛊唯一。再多的仙元石都不卖。这点几乎是市场的铁律,哪怕是到了五域乱战时期,也是如此。

%14
因此要想买下仙蛊。就得用另一只仙蛊作为代价。

%15
方源心神投入,很快就看到自己的一些仙蛊,也列举在上面。

%16
有平步青云、浪迹天涯、招灾、乐山乐水四大仙蛊。

%17
前两者,都是移动仙蛊,一只云道,一只水道,均和方源的力道不符。太白云生虽然兼修云道。但本身是宙道蛊仙,平步青云对他而言,也是不合用的。

%18
至于招灾仙蛊。高达七转。但效能可谓坑爹,简直是运道自杀利器。虽然方源也设想过一些用途,但若能换取其他仙蛊,更为划算。

%19
最后的乐山乐水仙蛊。能产生大量乐意。但方源已经准备大规模炼制恶念蛊。如此一来,就能取缔乐意的作用。不妨也放出来,看看能换到什么仙蛊。换不到的话,大不了再收回头。

%20
四大仙蛊之后,自然标注着方源要换取的仙蛊要求。

%21
值得一提的是,这四只仙蛊都是方源通过黎山仙子这条线,交代出去的。算是撇清了自己。

%22
虽然招灾等仙蛊,抛卖出去有些烫手。甚至会有麻烦。但若能换取合用的仙蛊,却是利大于弊的。

%23
错过这次良机。要再等的话,估计就要几百年后的五域乱战时期了。

%24
方源生性谨慎,但当断则断,该冒险的时候,绝不会怂缩不前。

%25
“除了这四只仙蛊,我还有春秋蝉、净魂、连运、妇人心,智慧蛊也勉强算是罢。”方源算了算,这剩下的五只仙蛊,他都需要,不会拿出来拍卖,甚至连任何的暴露机会都会尽量避免。

%26
春秋蝉,谁用谁知道,虽然有失败可能,但方源尽得好处,经验不少,已经欲罢不能。哪怕有智慧蛊在狐仙福地里面,春秋蝉仍旧是他的最大底牌。

%27
方源已经决定,第一空窍仍旧保留春秋蝉作为本命蛊。将来若有机会,就会选择宙道升仙。

%28
净魂仙蛊是仙道杀招万我的核心,虽然目前饥饿,不能催用,但也不会卖。

%29
连运仙蛊是弥补春秋蝉弊端的辅助仙蛊,方源切身体会到运道的厉害,这只仙蛊也不会放手。

%30
智慧蛊更是想都别想,一拿出来,就是滔天风云,杀身之祸。

%31
至于妇人心仙蛊,本来也是方源抛弃之物。但机缘巧合之下,方源领悟出毒气喷吐杀招,此刻妇人心仙蛊,已经内置于方源的右胸。

%32
它形如心脏,砰砰微跳,只是体型颇小,只有婴孩拳头大小。它通体紫黑,毒气萦绕,此刻已经联通方源的血管。随着方源的每次的呼吸,尸血的缓慢流通,而不断地渗出毒气,参与到方源的内部循环当中。

%33
从这点上看,明显可以看出智道杀招“包藏祸心”的影子。

%34
但妇人心周围的辅助凡蛊,却更近似一个逆炼的蛊阵。通过逆炼,将妇人心的力量提取出来。

%35
这点还归功于仙蛊妇人心的特殊性质养炼合一。

%36
此蛊需要用妇人心脏喂养,在喂饱了的基础上,数量越多,威能就越强。现在方源将这个过程逆转,提取出威能。

%37
人是万物之灵。

%38
妇人心虽然是消耗仙蛊,正常使用的话,一次性消耗掉了。但想到方法,搭配蛊虫,就能做到不断提取威能,重复利用。

%39
“毒气喷吐,只是仙道残招,今后还需要完善。有智慧蛊在,应当问题不大。”方源收敛思绪,将手中的东窗蛊放下,念头一动,面前一片透明,令方源能够直观拍卖场大厅。

%40
拍卖场分有大厅、单间、密室。

%41
后两者,方源是看不到的,只能看到大厅中的景象。

%42
此时,厅堂中的座位上,已经坐下了二十几位蛊仙,各自交谈着,内容不避人耳。

%43
“鹿老,许久不见,别来无恙乎。”

%44
“原来是青玄子,看样子你应该是渡过二次天劫了吧。”

%45
“惭愧,惭愧。在下底蕴不足,心生惧意,这些年来大多将福地种在北原外界。又耗费巨资,借了宙道仙蛊,大大拖缓了福地的光阴流速。现在仍旧还是一次天劫的修为。”

%46
“地灾已是难渡,天劫更是艰巨关卡。阁下老成持重,不轻言冒进,正是稳妥之道啊。”

%47
……

%48
“史悠彦,你也在这里?”

%49
“你这话说得太奇怪!你邬容能来,我为什么就不能来?”

%50
“哼!当日你夺蛊之仇,我必报之。你等着看好了。”

%51
“呵呵呵,我知道此仙蛊合你所用,但我已经决定拍卖,就看你今天有没有这个本事了!”

%52
……

%53
有的蛊仙在叙旧,套交情,有的却是剑拔弩张,火药味甚浓。

%54
这时,一位白袍老者昂首踏步,进入大厅。

%55
“哈哈,来了不少人呐。”白袍老者正是袁家太上大长老,他环视一圈,哈哈大笑起来。

%56
豪放的笑声,立即引来厅中之人的注意,议论声为之一低。

%57
方源也投去目光,这位袁家太上大长老乃是七转蛊仙,超级势力北原袁家的首脑,战力非凡,财力也非凡,是个强大的竞争对手。

%58
秦百胜一直站在门口迎宾,此时主动上前施礼道:“袁大人能够光临拍卖会,真是令我百胜福地蓬荜生辉。”

%59
“不要叫我大人,直接叫我名字袁让尊!”袁家太上大长老看到秦百胜,双眼一亮,态度亲热地拍拍后者的肩膀。

%60
早年,他和秦百胜对战过,后者的实力,已经得到他的认同。

%61
“不过,我最近演练枪术,又有了心得。什么时候,我们再切磋一次。”袁让尊忽然话锋一转,透露出武痴的本性。

%62
“袁前辈的道痕枪法,一直让晚辈记忆犹新。不知袁前辈,是想入座大厅,还是单间,或者密室?”秦百胜笑了笑,不借袁让尊的话头。

%63
“给我一个单间吧。”袁让尊也知道不是约战的时机,想了想后,决定下来。

%64
袁让尊前脚刚刚进入单间,一位俊朗书生,身后亦步亦趋地跟随着一位女蛊仙,也进入拍卖场。

%65
“自在书生也来了?”大厅中蛊仙们的目光,再次被吸引。

%66
“我早就听闻自在书生艳福不浅,身边有两位蛊仙充当侍女,名为红袖、添香,不知道是哪一个?”更多的男性蛊仙,则看向女蛊仙。

%67
“观其红裙如火,应当是红袖仙子。添香仙子,生有体香,若是来此,此刻大厅恐怕已经香气扑鼻了。”熟知内情的某个蛊仙津津乐道起来。

%68
方源脑海中,也迅速划过相关情报。

%69
这自在书生,乃是一位散修,来历有些曲折。他的祖上并非北原中人,而是从中洲流落的世家。

%70
中洲门派林立,家族竞争力不强,生存不下去,有条件的都会穿过界壁,来到其他四域生存。

%71
独在异乡为异客,自在书生的家族,本身被排挤得就实力大损,在北原艰难生存,每况愈下。勉强维持了几代之后,终于融入了北原,但也如垂死之人一般,病入膏肓。

%72
到了自在书生这一代,家族积重难返,彻底崩灭。自在书生带着两位侍女,逃得性命,艰难生存。困顿中,哪怕面临死亡的危机,自在书生也对两位侍女不离不弃,最终他得到机缘,成为蛊仙。

%73
成仙后,又不惜耗费财力精力,提携了两位侍女一同成仙。这在北原蛊仙界,一度成为美谈。

%74
不管是袁让尊、自在书生,还是其他蛊仙,都有着各自的精彩,都是自身传奇故事的主角。

%75
能够成为蛊师者,自然是人上之人。这场拍卖大会,还未开始,甚至连蛊仙都未全部进场,便已经是英豪连出,星光耀目了。

\end{this_body}

