\newsection{我要大义灭亲}    %第七节:我要大义灭亲

\begin{this_body}

“老白你提醒的是,只是我没有办法直接联系到琅琊地灵。。我上次用了两次机会,从琅琊地灵那老头手中得了两只仙蛊,几乎是被老头子赶出来的。呵呵呵,所以还没有方法直接联系到他。”方源苦笑起来。

“是这样……”太白云生脸上喜色一滞。

两个福地之间直接联系的方法,就是用洞地蛊,直接沟通两地。方源之前和仙鹤门交易石人,就是用此方法。

但方源显然和琅琊地灵的关系,还没有好到这种程度。

更关键的,狐仙福地在中洲,琅琊福地在北原,就算洞底蛊也不好使。要直接联系,只有星门蛊。

星门蛊是一对两只,可惜现在方源手中的星门蛊,只剩下其中一只。另一只,已经被大同风幕损毁了。

“看来我们只有回去北原,亲自见到琅琊地灵,才能展开合作了。”太白云生叹息一声。

“不,还有一个看运气的法子。”方源沉吟道,“那地灵老头酷爱炼蛊,自己又不能出去搜寻炼蛊材料,琅琊福地地貌单一,产出有限,因此他总会去宝黄天收购。咱们可以守株待兔,盯着宝黄天,等候琅琊老仙的神念。”

“对,还有这个法子!”太白云生双眼一亮。

“所以接下来,就要靠老白你帮忙了。”方源很自然地道。

“没问题,包在我身上。我这就回去,打开通天蛊。联通宝黄天!”太白云生拍了拍胸脯,他迫不及待,被地灵带回了住处。

他选择在福地西部。搭建了一处小屋。

那就是他的临时住处。

狐仙福地地貌,极为类似北原,太白云生住在这里,没有任何不惯,反而有一种安宁的家的感觉。

他在北原漂泊流荡了大半生,如今终于找到组织,定居下来。因此分外珍惜这处家园。

打发了太白云生,方源陷入沉思。

他早已经想到去与琅琊地灵合作。事实上,几天前他就叫小狐仙关注宝黄天。等候琅琊老仙的神念。

但这些天一直没有收获。

他现在思考的,不是琅琊地灵这件事,而是太白云生。

北原之行匆忙混乱,在狐仙福地的这些天。方源重点考察了太白云生。对他加深了很多认识。

“太白云生这个人,的确是个好人。能力有,野心没有。聪明程度只算是正常人,更多的是一生经验积累出的智慧。他没有大局观,难怪曾经是少族长,流浪了这么多年,名望有了,实力也有。却仍旧建立不出势力来。”方源在心中评价。

这样的人,最好是跟随自己左右。不能放任一方,领导能力堪忧。因为心性、理念不合,放得太远,不仅配合不了自己,甚至还可能会坏事。

这些天,方源和太白云生朝夕相处,关系更有进展。

几次秉烛夜谈,方源将自己重生的经历,都说出来,只是添加了一个不存在的“紫山真君”。

而太白云生也将自己的一生经历,也都基本上告知了方源。

好几次,尽管方源自己心中已有定计,也把太白云生叫过来商议事情。

看似浪费时间,却是方源御下之道。

其一,是考察太白云生才华和内心,发现他的关心的确情真意切,发自内心。

其二,毫不保留地告知他目前的困境,增加他的归属感、责任感。

其三,是对太白云生的无形打压。基本上每次的决策结果,都是方源的方法,否决太白云生的方案。这样的次数多了,潜移默化,让太白云生渐渐依赖方源,重视方源的意见,潜意识否定自己的想法。关键时刻,方源就能一语而决,一锤定音。不会出现双方意见相左,争执误事的情况。

方源不是黑楼兰,黑楼兰手中有六转奴隶蛊,可以直接奴隶蛊仙。

方源手中没有。但太白云生不是白凝冰,也不是黑楼兰。

最关键的是,方源有自己的手段,并且相信:就算没有奴隶仙蛊,照样可以收服太白云生,压榨出他的最大价值,为自己所用。

这是魔道巨擘的自信!

中洲,飞鹤山。

云海广阔,千鹤飞舞。

大山青葱,风卷松涛。方正坐在山崖上,凝神望着数千头铁喙飞鹤盘旋,不断按照他的心意变化阵型。时而俯冲,时而兵分两路包抄,时而化圆阵防守。

大风吹拂他的长发,明亮的双眸闪烁着坚定的光。

磨难使人成长。对方正而言,一年多前的狐仙福地之争,无疑是一次重大的挫折。

经过开解,他走出困境,越加刻苦用功,几乎是每天都在拼着命去修行!

这时,从方正的空窍中传来一道声音:“好!你指挥铁喙飞鹤,已经达到鹤随意动,阵随心转的地步。现在已经可以回去门中参加飞鹤指挥使的考核,冲击乙等评价。你有六成把握能够成功,成功之后,就能得到考核奖励——一只五转的五孔玉箫蛊。一旦你有此蛊在手,指挥鹤群,就能达到奴道准大师的地步了。”

这股声音的源头,来源于方正空窍中的一只寄魂蚤。

方正的师傅——天鹤上人的魂魄,就寄居在蛊虫里面。

“五转的五孔玉箫蛊?”眼中一亮,兴奋之色溢于言表。“是,师傅,我这就去。”

方正站起身来,心念一动,便有一只巨大的铁喙飞鹤王,飞到他的面前。

他纵身一跳,动作轻松地跳上一只鹤王背上。

鹤王引吭长啸,昂扬之意勃发。方正坐在鹤王背上。身边群鹤紧随盘绕,雪白一片,夹风裹云。飞向仙鹤门派驻地。

不一会儿,他便飞到丁字号白玉广场。

铁喙飞鹤纷纷落在广场上,方正刚刚双脚落地,便有仙鹤门弟子疾步而来,向他一礼道:“方正师兄。”

方正点点头:“这位师弟,我来参加飞鹤指挥使的考核。”

“方正师兄请跟我来。”仙鹤门弟子在前引路。

“你们快看,是方正师兄啊。他可是咱们此届精英弟子之首!”

“刚刚你们听到没有。方正师兄竟然是要参加飞鹤指挥使的考核了。”

“方正师兄是奴道天才,想不到已经达到奴道准大师的地步,真是厉害啊……”

路上。行人小声的议论,传入方正的耳中。

方正听了,不禁微微带起了笑容。

他如今已经完全长成,身材欣长。一头乌黑的长发披肩。双眼明亮清爽。虽然容貌普通了些,但气质昂扬,生机勃发。一身青白相间的长袍,温文尔雅,叫人看着就感觉舒服。

“方正师兄虽然强大,但他哥哥更加厉害,乃是夺得狐仙福地的超绝人物呢。”

“不错,据可靠消息。他哥哥方源,一直是被某位太上长老秘密培养的真传弟子。灵缘斋差点要夺得狐仙福地时。太上长老没有办法,只好派遣方源师兄出马,果然不鸣则已一鸣惊人,一锤定音,抢到了狐仙福地!”

“方正有个好哥哥啊,荡魂山上的胆识蛊可以壮大魂魄。我敢肯定,方正一定是用了胆识蛊。否则单靠他个人的努力,怎么可能在短短一年时间,就将修为提升到五转?而且还在奴道上进步神速,竟然要参加飞鹤指挥使的考核了!”

这些人的语气,又酸又涩,方正听在耳中,脸上的笑意消失了,缩在长袖中的双手暗暗握紧。

“哥哥!”方正眼中迅速闪过一丝阴霾。

自从方源以不可思议的方式,夺得了狐仙福地,仙鹤门就主动向外宣布,方源乃是仙鹤门弟子。

这一年多来,方正感觉自己就像活在从前,被方源盖压在下面。

不论他取得多么好的成绩,多么大的进步,旁人也只会在稍稍的赞叹之后,便提及到更加优秀的古月方源。

“平心静气,我的徒弟。”寄魂蚤中传出天鹤上人的声音。

天鹤上人的阅历,可比方正丰富多了。这些年朝夕相处,也让他对方正了解得十分透彻。

他安慰方正道:“这些年,你的努力我都看在眼里,你一点都不比你的哥哥差。别忘了,你现在已经是五转蛊师,奴道准大师!你哥哥现在估计也不是你的对手了。而且再过不久,方源就要灭亡了。这一次计划,门中将出动三位太上长老,务必回收狐仙福地,你哥哥绝无生还的希望。”

三位太上长老……就是三位蛊仙战力。

方正闻言双眸一亮,不禁舔了舔嘴唇,旋即目光又有些黯淡下去。

天鹤上人知道他在想什么,笑了一声道:“方正,你不要有任何的内疚感。你的哥哥已经步入歧途,走上了魔道。他竟然诛杀全部亲族,这是猪狗都不如的畜生行径!方源是地地道道的魔头,你不能对他抱有同情心。杀了他,是为天下苍生谋福利。你想想看,他若活着,不知还要祸害多少无辜的人。”

“的确,哥哥他双手沾满了血腥,是屠夫!刽子手!我要为我的亲族,为我的舅父舅母,为了沈翠儿报仇雪恨!”方正暗自呐喊,为自己鼓劲。

“对,就是这样。”天鹤上人显然很满意方正的态度,“想想看这些年,你在伏虎福地中的训练,多么的艰难惊险!外界一年,伏虎福地中就是八年!你成为五转蛊师,吃了这么多苦,遭了这么多罪,不就是为了回收狐仙福地的大计吗?千万不要掉链子,绝不能在这关键时刻失误,让大家对你失望啊。”

“你放心,师傅!我懂的,我一定竭尽所能,为门派贡献,斩奸除恶,大义灭亲!绝对不会让您失望!”方正斩钉截铁地保证道。

\end{this_body}

