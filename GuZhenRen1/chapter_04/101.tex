\newsection{拍卖大会(中下)}    %第一百零一节:拍卖大会(中下)

\begin{this_body}

时间的脚步稳定向前,进场的蛊仙渐渐变得稀少。(www.QiuShu.cc 求、书=‘网’小‘说’)又等了片刻,秦百胜关闭福地门户。

他走上大厅前面的高台,言简意赅地宣布道:“诸位,按照协约,时候已到,拍卖大会正式开始!”

“奇怪,黎山仙子和黑楼兰,怎么都没有进场?”太白云生那边,向方源传来信息。

“是啊,我也在疑惑。按照黎山仙子的秉性,进场之后,应当坐于大厅。但却一直不见其踪影。”方源回道。

“此次拍卖大会,分有大厅、单间、密室。若是选择密室,就会沿着秘密通道,直接进入,不会公开亮相的。但就算黎山仙子进入密室,也绝不会不和我们打一声招呼。你有没有发现,黑家的蛊仙并无一人到场。”太白云生意有所指地道。

方源沉吟道:“且先书信一封,过去问问情况。黎山仙子、黑楼兰和我们有雪山盟约,背叛我们的可能性不大。若是遭到黑城等人的袭杀,凭借她们俩的实力,也不至于一封求援信笺都不能发出。”

两人秘密商议期间,高台上已经搬出第一件拍品。

“众所周知,这是天魁云。铺展开来,可绵延两百多万亩。如此大量的天魁云,就算在宝黄天中也是少见!起拍价二十块仙元石,每次报价至少五块仙元石。请开始!”秦百胜开口道。

“天魁云,这正和我的太白福地所用。”太白云生的注意力,顿时被吸引过去。想也不想,操纵密室中准备的凡蛊竞价。

秦百胜心有所感,当即宣布道:“十一号密室。出价三十块仙元石。”

下一刻,大厅中的一位云道蛊仙开口:“三十五块仙元石。”

太白云生轻轻一笑,催动凡蛊,再次报价。他直接将价格再提十块,升到四十五块仙元石。

四十五块仙元石,已经可以买到一只荒兽尸体了。

先前报价的云道蛊仙,咬了咬牙。犹豫了一番,没有再开口。

地下有地魁兽,天上则有天魁兽。在太古九天中。基本上都有天魁兽群或多或少的生存着。

所谓天魁云,便是天魁兽群的呼吸气息,在日积月累的过程中,慢慢凝聚形成。

此云能营造环境。令仙窍中云道蛊虫的产量大增。若有一支天魁兽群生活在云中。那就更妙,处理得当,便能有源源不绝的天魁云了。

不过,完整的太古九天,只有强大的个别七转,或者八转蛊仙才有资格一探。

每一只天魁兽至少是异兽王,媲美普通兽皇。而每一支天魁兽群中的王,基本上都是荒兽级天魁。

“四十五块仙元石。第一次。”秦百胜见无人再报价,便开口喊道。

“我的福地中。已经有漫天的浮球茶草。这个月底,就会有第一批丰收。接下来就是打算引进玉蜂鸟群了。若是再添加天魁云进去,就有了基石,能够顺利引进天花和浮鹞。师弟你虽然比我晚成仙,但早就经营狐仙福地,有经验。你帮我参谋参谋,觉得我这个设想如何?”太白云生笑呵呵地对方源传音道。

台上,秦百胜再次道:“四十五块仙元石,第二次。”

方源思索了一下,回答太白云生:“天花的确是依附在云中生长,浮鹞身上通常寄生云道蛊虫。老白你是想主修宙道,兼修云道了?”

“正是如此。我宙道长于治疗,可以辅修云道,弥补其他短板。”太白云生直言不讳。

他本身就擅长云道移动,造诣深达飞行大师。后来又得到北原墨人王墨坦桑交好,低价卖给他九云环杀招,更使得太白云生对云道产生浓厚兴趣。他会做此选择,并不让方源感到突兀。

这时,秦百胜第三次喊道:“四十五块仙元石,拍卖两百多万亩的天魁云,第三……”

“慢着,五十五块仙元石。”眼看大局将定,忽然间又一位蛊仙开口。

他的声音,是从十七号单间传出来的。

“呃。”太白云生没料到有如此变故,连忙再次加价,“六十块仙元石!”

这个价格,终于打消了在场所有人的斗志,最终令太白云生如愿以偿。

“下面是第二件拍品,一支珍稀的黑魂羚群。兽群规模达三十二万,拥有五头兽皇。底价十块仙元石,每次报价至少四块仙元石。若用垂虹膏换取,将有价格优惠。”秦百胜道。

黑魂羚在北原外界已经几乎看不见,这支兽群规模众多,应当是某位蛊仙仙窍中豢养。且寄生在羚羊身上的蛊虫,多为魂道蛊虫,因此特别吸引魂道蛊仙。

出售这支黑魂羚的蛊仙,需要炼蛊佳材垂虹膏。因此可用仙元石,亦可用垂虹膏竞价。

一轮竞价之后,这支黑魂羚被某位蛊仙,以二十八块仙元石购下。

“第三节拍品,一套源自中古时代的蛊方,名为寸光阴。从一转到五转皆有,关键的一点,此蛊方已经被改良。一概炼蛊材料,都在当今常见,可以寻得。底价是二十二块仙元石,还可用荒兽六头大蛇的蛇皮抵价。”秦百胜喊道。

“二十三块仙元石。”

“才加价一块,也不嫌丢人。三十块仙元石!”

“我出三十二块。”

这一次竞价的蛊仙,明显比前两次要多。

寸光阴是宙道蛊虫,不能对人施展,却可用于其他对象,广泛用于炼蛊。譬如要炼一只蛊虫,需要三天三夜。这时消耗寸光阴蛊,迅速达到三天三夜的时间要求,加速了蛊虫的酝酿和变化。而蛊师真正耗费的时间,也许只是短短片刻。

方源曾经得到过上古蛊虫群力蛊的蛊方。市面上只价值两块仙元石。寸光阴只是中古时代的蛊方,但有一整套,从一转的一寸光阴。到五转的五寸光阴皆全。

最关键的是,这套蛊方已经被蛊仙改良!所用的炼蛊材料均符合当下情况,不难搞到。不像原来的蛊方,很多炼蛊材料稀缺无比,不是价格高昂,便是已然灭绝。

这一点,决定了价格一路暴涨。底价是二十二块仙元石,很快就抬高到四十八块。

太白云生也参加了这次竞价,此时开口报价:“五十八块仙元石!”

一下子。将价格抬高十块,意图吓退竞争者。

太白云生富有,甚至比方源更甚。

他有江山如故、人如故两大宙道治疗仙蛊,在东海很是吃香。尤其是这次探索玉露福地。虽然失败。但用人如故救下某位蛊仙。这位蛊仙报答他的酬金,就是百十来块仙元石,并亲口承诺欠下太白云生一份情,来日再做报答。

最关键一点,太白云生的福地是健康的,每隔一段时间,就产生青提仙元。他青提仙元越积越多,都用不掉。

他的太白福地。时光流速高达外界33倍,和黑楼兰的特等福地都差不多了。

这是因为他是上等福地。而且是宙道蛊仙。在宙道资源方面,比较突出。

外界一天,太白福地中就是一个月。理论上,每个月福地都会产生青提仙元。再相同的外界时间,太白云生得到的青提仙元将是普通六转蛊仙的数倍,甚至十多倍!

而反观方源,虽然有胆识蛊买卖,但各种消耗,尤其是不断地炼化仙元石,换取青提仙元。

没办法,他的仙窍已死,无法自产仙元,每隔一段时日还会崩解萎缩,对他是极大的拖累。

太白云生的报价,的确吓退了大部分的竞价蛊仙。但仍旧有一批,仍旧在坚持着。

一块两块的这样报价,价格慢慢涨到六十六块仙元石。

“六十六块仙元石,就为了买下一套凡道蛊虫的蛊方!”大厅中有些低层蛊仙,不免咋舌。

“不算贵。寸光阴虽然不能用于自身,但能节省时间,浓短炼蛊步骤,有广泛的利用空间。要做长远打算,尤其对那些兼修炼道的蛊仙们,日积月累下来,能够用寸光阴,节省到六十六块仙元石的利润。”有蛊仙平静地分析着。

“这个价格,已算是到顶了。再往上提价,就有些不智。除非本身有特殊需求。”另一位蛊仙也附和着分析。

“六十六块仙元石,第一次。”秦百胜开始喊道。

这时方源传音太白云生:“老白,你我各出一半仙元石,合力竞争如何?”

太白云生大喜:“师弟你有心出手,那就再好不过。”

于是他继续暴降,将价格提到六十八。只有一位蛊仙继续竞价,加了一块仙元石。

最终方源和太白云生,两人合力,以七十块仙元石的不菲价钱,拍下寸光阴。

时间缓缓流逝,一项项的宝物接连亮相。

拍卖大厅的竞争,越加激烈,氛围变得火热。许多蛊仙失去仪态,争得面红耳赤。尽管有着理智和冷静,但重利当前,却是不得不争个脸红脖子粗。

完成了数十项拍卖之后,一些底层蛊仙已经无力再战,有的满头大汗,有的脸色苍白,有的更是近乎虚脱。

竞价比的不仅是财力,更是心理上的暗斗交锋。

渐渐的,六转蛊仙的竞价者越来越少。

六转蛊仙通常有数百块仙元石积蓄,七转蛊仙手头上,则有数千块,上万块。到了八转,数万块一定是有。甚至积年老修手中,更多达数十万块仙元石。

修为越强,占据的资源就越多。普通六转,底子太薄,幸亏还有物资可以抵价,否则大多数都已经彻底出局了。

方源的手中,也只是小几百的仙元石。间或出手,鲜有成功,多数失败,一直耐心等待着。

终于,他听到秦百胜宣布:“下面进行仙蛊的拍卖。”

“重头戏来了!”方源双眼骤亮。

不止他一人,其余蛊仙亦都是强振精神。

之前的拍卖,只能算是前奏。真正的大菜,终于上场。

换仙蛊!(未完待续。)

\end{this_body}

