\newsection{小九}    %第五十五节:小九

\begin{this_body}

%1
眼看着匕首状的仙蛊,就要刺中自己,鹤风扬心中,愤怒、惊恐、茫然、不甘等各种情绪如岩浆沸腾翻滚,大大影响着他的行动。

%2
“难道我今日就要命丧于此?”鹤风扬的心中忍不住冒出这个念头。

%3
但就在这个危急时分,他脚下的荒兽九宫鹤,忽然引吭清啸,随后猛地一翻身。

%4
这样一来,鹤风扬就被掀飞下去,而九宫鹤则挡在暗杀的蛊仙面前。

%5
噗嗤。

%6
一声轻响,匕首状的仙蛊,狠狠地插进九宫鹤的胸脯之中。

%7
“嗯?”第三位蛊仙的脸上,闪现出一抹惊讶之色。

%8
他原以为,干扰了鹤风扬,就能影响到九宫鹤。毕竟鹤风扬是奴道蛊仙,奴役着九宫鹤。影响了他,九宫鹤也就不足为虑。

%9
他动用数个情道杀招,严重干扰着鹤风扬,却没料到最后关头,这九宫鹤竟会自动护主。

%10
“小九!”鹤风扬坠落下去,看到九宫鹤遭殃,立即泪流满面,痛声嘶吼。

%11
这九宫鹤他精心培育,从未用过任何奴役它的法子。

%12
在他年轻的时候,外出任务,惨遭追杀,命悬一线,结果意外地碰到同样重伤的九宫鹤。

%13
经过一系列的巧合和意外之后,他和九宫鹤相互扶持,走出困境。

%14
九宫鹤濒死,鹤风扬带着它回到门派,倾尽家财,勉强吊住九宫鹤的一丝气息。

%15
之后数十年,鹤风扬出生入死。赚取钱财,一点一滴的救治九宫鹤,将它慢慢地从死亡边缘拉回来。

%16
由此人鹤之间。建立了深厚感情,比兄弟还亲密。

%17
又后来,一次变故,五转蛊师的鹤风扬任务失败,身受重伤,身上的资源只够救他一人。救仙鹤他就死,救他仙鹤就没有继续残喘的生机。

%18
如此抉择关头。鹤风扬思考了三天三夜,终究决定牺牲自己,救仙鹤!

%19
此番举措。引得太上三长老虎魔上人感叹,由他出手相助,救下鹤风扬和九宫鹤。

%20
后来鹤风扬不负期待,成就蛊仙。就依附于虎魔上人一系。

%21
九宫鹤发出凄厉的哀鸣。庞大修长的身躯罩上一层灰光。在灰光中,它体型迅速缩小,竟是逆生长,从成年不断回溯,变成青年,再变幼年。

%22
这是仙蛊的威能!

%23
匕首状的仙蛊,竟是一只宙道仙蛊,能让目标的身体回复到更加年轻的状态。

%24
“撤吧。不要辜负了你那朋友的一番心意。”仙窍中。虎魔怒意长叹道。

%25
“小九!”鹤风扬怒吼,不顾虎魔怒意的劝阻。返身扑上。

%26
他浑身衣摆鼓动翻飞,化为一件鹤羽大氅,他双目瞪圆,绽放血光,修长的绿眉飞舞起来,灵动如蛇似龙。

%27
神秘蛊仙面露诧异之色。

%28
刚刚鹤风扬明明已经可以逃遁,没想到他竟然返身杀来。

%29
他可是奴道蛊仙啊!

%30
鹤风扬举起双掌,狠狠一推,打出澎湃的碧绿雷光。

%31
神秘蛊仙冷哼一声,举起匕首,悍然对撞上去,同时催动情道杀招,以期影响鹤风扬的情绪。

%32
但鹤风扬心中充满了愤怒,竟然影响不住。

%33
轰!

%34
爆响声中,神秘蛊仙倒飞出去。鹤风扬身形巨颤,吐出一口鲜血,强行振作,将幼年九宫鹤捞到手中。

%35
“小九!”鹤风扬怀抱九宫鹤,掉头飞遁,再无逗留之意。

%36
九宫鹤已变成白鹅大小,被鹤风扬抱在怀中,轻轻鸣叫。

%37
留下情道蛊仙停留在远处,这一会儿功夫,就已经追之不及。

%38
其余两位神秘蛊仙,撤销魂狩战场,赶了过来。

%39
“唉,到底是让他给跑了!”风道蛊仙遗憾长叹。

%40
“这家伙明明是奴道蛊仙,居然近身战斗也不弱。刚刚那道碧雷杀招,有些似是而非,让我仿佛想起什么……”情道蛊仙道。

%41
“哼,若非仓促之间调集不了人手,还怕拿不下他?我们再追上去的话,或许就能……”毒道蛊仙语气不甘。

%42
“对方有拓宇仙蛊,魂狩战场困不住他。我们赶紧走吧,此地不宜久留。”风道蛊仙很冷静。

%43
“不错,这次埋伏狙杀,不过是临机设想的行动。万万不能因小失大,暴露身份,影响主上的大计。我们撤!”情道蛊仙回过神来,轻喝道。

%44
他的地位,似是比其他两位神秘蛊仙还要高些,有着一语定音的作用。

%45
“这便去了。”风道蛊仙长遁空中,很快身形就成了一个小黑点。

%46
“哼!”毒道蛊仙落于地面,直接地遁。

%47
情道蛊仙又停顿了一会儿,见两人彻底离去,身形在半空中悄然消失,仿佛从未出现过。

%48
鹤风扬一路疾飞,回到飞鹤山,也不停留,直接进入伏虎福地。

%49
虎魔上人正站在一座巨大的石坑边上,看着坑中的上万石人劳作。

%50
“晚辈办事不利,有负大人所托。”鹤风扬见到虎魔上人,躬身一礼,他身上带伤,一脸惭愧。

%51
“你受伤了?且先去自行治疗。”虎魔上人不问鹤风扬任何事情,只是一招手,那股怒意就从鹤风扬的仙窍中飞出来,钻入他的脑海当中。

%52
只一瞬间,他就明白了鹤风扬此番经历的一切。

%53
他目光闪了闪,旋即催动怒意蛊,生出又一股怒意。

%54
怒意飞遁长空,出了伏虎福地,只往飞鹤山巅峰飞去。到了山巅的议事堂中,留守的三股意志纷纷有感望来。

%55
这三股意志,分别来自太上大长老、二长老,以及三长老虎魔上人,负责监察门派,同时处理一些掌门处理不了的小事。

%56
而大事则整理起来,每隔一段时间,再召集诸位蛊仙进行共同的商议。

%57
飞来的虎魔怒意,降落到堂中,先和之前的意志合为一体,然后诉说了鹤风扬之事。

%58
大长老、二长老的意志听后,纷纷陷入沉默,急速思考,与此同时,它们的身形以肉眼所见的速度,不断缩小。

%59
缩小了近一半后,太上二长老的意志道:“狐仙福地之事,就由虎魔上人你亲自定夺吧。”

%60
太上大长老意志接道:“鹤风扬被埋伏,遭受阻杀一事,比狐仙福地更为严重。凶手是在何处动手?”

%61
说着,一只蛊虫从堂中飞出,化为一片光影地形图。

%62
“这里。”虎魔上人的怒意指点了位置。

%63
太上大长老意志点头,沉声道:“此处位置,设伏绝佳。前后不靠,是最大限度的隐蔽。看来凶手对中洲地形甚为熟知。”

%64
“彻查!什么时候,居然敢有人对我们十大古派的蛊仙动杀手了?”太上二长老意志怒然低喝道。

%65
“我所虑者,正在于此。两位长老,以为这些蛊仙来自何处?”虎魔上人怒意道。

%66
太上大长老、二长老的意志相互对视一眼,均流露出沉重之色。

%67
“三长老担心,这些蛊仙其实就是中洲蛊仙?”太上大长老低语。

%68
虎魔怒意开始侃侃而谈:“不错。我们十大古派,掌控中洲已经多少年了?这些年来,情势越发艰难,是因为什么?咱们中洲和其他四域不同,以元始仙尊为首,早在远古时期就已经改革,建立门派制度。远古时代,门派式微稀少。上古时代,家族势力远超门派势力。中古时代,门派和家族两大势力剧烈摩擦,并驾齐驱。近古时代,家族示弱,门派昌盛。到了现在,中洲门派林立,家族几乎不存。”

%69
“家族制度,只选取家族子弟为蛊师。门派制度,却能令凡人踏上蛊修之路。我们中洲的历史,就是两大制度的争斗史。经过三百多万的抗争、演变、积累,受益于门派制度,中洲的蛊仙的数量,远超其余四域,越来越多,已经快要脱离我们十大派的掌控了。”

%70
“我们十大派加起来的蛊仙有多少?其余的中洲蛊仙数量,是我们的数倍!这些小派的蛊仙,魔道的蛊仙,散修蛊仙,要继续修行,就需要更多的修行资源。冲突是不可避免的,因为我们十大派掌握着中洲八成的修行资源。”

%71
“钧天剑派的例子,难道之前没有过吗?只是我们十大古派暗中打压下去了。但这些年来,新的门派仍旧层出不穷,新的思想,新的流派屡见不鲜,凡人中不凡天才、鬼才、怪才不断涌出,而我们十大古派却位置有限,尽管每年都招揽天资出众之辈,但中洲实在是太大了。”

%72
说到这里,虎魔上人的怒意叹息一声。

%73
两大长老一直保持沉默。

%74
十大古派掌握着中洲现有的八成资源,但分摊到每个人,每个蛊仙的身上,又有多少呢?

%75
蛊仙修为越高,经营福地就越需要更多的资源。尤其是天灾地劫也越强,每一次渡劫之后,损失重大,就要更多的投入去修复,去更上一层楼。

%76
十大古派的蛊仙们,是不可能将自己的资源,分给其他人的。

%77
过去,十大派之外虽然门派不少,但蛊仙数量不多。仙凡有别,凡人和仙人之间有着巨大的战力鸿沟,因此可以镇压。

%78
但现在,十大派之外的蛊仙数量越来越多,虽然这些蛊仙通常战力都不怎么样,但庞大的基数,给十大古派造成冲击。

%79
钧天剑派就是一个很好的例子,这个大型门派中原本有两个蛊仙,是仙鹤门的附庸。现在增添了第三位蛊仙,便立即开始计划脱离仙鹤门,企图自立了。

\end{this_body}

