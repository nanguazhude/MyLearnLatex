\newsection{惊鸿乱斗台}    %第三百一十一节:惊鸿乱斗台

\begin{this_body}

双方激战,打得天翻地覆,山河迸溅。

即便是虚影变化,不干涉现实,也叫人动容不已。

战斗渐渐白热化,八转蛊仙壮汉占据上风,仰头狂笑:“吾乃是八转蛊仙,又是大力真武体,你区区七转,居然痴心妄想,想要挑战我的威名?!”

那额头绘有红莲的神秘蛊仙,呵呵一笑,发出中性的声音:“武斗天王,今日你必败无疑。接我这一招!”

话音刚落,七转神秘蛊仙就右手一扬,飞出一座高台。

下一刻,八转蛊仙武斗天王脸上的轻蔑之色,荡然无存,惊呼道:“这竟然是仙蛊屋!这是什么仙蛊屋?”

七转蛊仙傲然一笑:“好教你知,这便是我独创的仙蛊屋惊鸿乱斗台!”

“惊鸿乱斗台?”白凝冰见此,皱起眉头,就他所知,当下的南疆蛊仙界根本没有这么一座仙蛊屋。

惊鸿乱斗台的威能,十分玄奇。

不管武斗天王,发出何种攻势,都尽数被惊鸿乱斗台吸摄封印。

这还不算完。

这是攻势被封印起来后,还能被七转蛊仙利用,尽数打回去!

武斗天王攻势凌厉无比,结果大多数反而被他自己承受了去。

影像不停变化,战况愈演愈烈。

最终在惊鸿乱斗台的帮助下,七转神秘蛊仙战胜了武斗天王,完成了七转胜八转的旷世壮举!

但同时,七转神秘蛊仙也付出了惨重的代价。

武斗天王最后反击,不惜化为八转仙僵。几乎拥有了不死之身。七转神秘蛊仙实在取不了他的性命,只好舍弃这座惊鸿乱斗台。将武斗天王镇压在地底深处。

为了战胜武斗天王,七转神秘蛊仙赔上了一座仙蛊屋。

战罢。他望着埋藏在地底的仙蛊屋,叹息一声,一飞冲天而去。

没有了蛊仙的身影,但虚影幻象却仍旧在变化不停。

蛊仙交战的战场,重归平静。而后风调雨顺,草木生长。又有电闪雷鸣,洪水袭来。或是流星陨落,大火焚烧。原本平坦的地势,渐渐隆起。慢慢增高。先是土丘,随后成为山峰。

偶尔有人物闯进画面,动作极快,不管是自然气象,还是人物猛兽的动作,像是快进的电影。

这山峰也在迅速变化,时而崩塌,又时而缓慢成形。最终在幻影中,这座山峰渐渐长成无名山峰的模样。

当幻影彻底消失。天地又恢复本来面貌。

白凝冰眼冒奇光,他看出来了:砚石老人交给他的仙蛊,应当是宙道侦查仙蛊!

它钻破时间的缝隙,将光阴长河上游的某段“过去”情景。传送到下游的“现在”来。

曾经两位蛊仙大战的战场,就是这座平凡得毫不起眼的山峰。

“也就是说,这里埋藏着一座仙蛊屋!”白凝冰紧紧盯住山脚下。心潮澎湃。

他当然知道仙蛊屋的价值。

拥有蛊仙的势力,才能算是超级势力。但这些超级势力当中。只有当中的强大者,才能拥有一座或多座仙蛊屋!

“若我能掌控这座惊鸿乱斗台。必将能带给我巨大帮助,让我更加从容地对付影宗,脱离影宗!糟糕,这仙蛊的气息四溢散发,我催动仙蛊的动静也不小,一定会惹来其他蛊仙的注意。时不我待,必须抓紧时间!”

白凝冰刚想要飞下去,钻破土地,深入地下,取得仙蛊屋,但他又犹豫了。

他想到了砚石老人。

他若下去,会不会就糟了砚石老人的算计呢?

或许砚石老人早已经算到他白凝冰的反应,就想要他深入地下,探索地底深处,取得惊鸿乱斗台。

“若是这样,他必有后手……螳螂捕蝉黄雀在后,我若盲目施为,只会做无用功,平白被人算计利用!”

白凝冰目光深邃,他决定先查明周围环境,再决定行动。

然而让他没有预料的是,他才刚刚接近无名山峰一段距离,就感到一股强烈的虚弱感,他身上的蛊仙气息迅速衰落下去,甚至念头都开始调动不起来。

白凝冰骇然后退,他脑海中闪现出砚石老人临行前的交代。

“你去那山峰顶上,催动此蛊。一旦成功,速速离开。此山将成为禁仙绝境,任何蛊仙进入此境,都会遭受致命杀机。”

白凝冰皱眉思索,难道这就是禁仙绝境?

若是这里真的成了禁仙绝境,那他该如何夺取惊鸿乱斗台?

这时,山峰上幻象虚影又再度成形,化为先前的一幕幕。

白凝冰耐心细看,不免猜测:难道说这里的幻影变化中,就藏有收取仙蛊屋的线索不成?

然而幻象虚影变化完毕,又重新消弭,白凝冰看不出来任何的线索。

之后他发现:原来每隔一个时辰,就会重新出现这个幻象虚影。

每一次光影变幻,都是呈现的相同的内容,别无二致。

他还发现:所谓的禁仙绝境正在扩张。

起先,他还能接近无名山峰三里之外。现在,他在距离无名山峰的十里的地方,就感到浑身疲惫,虚弱不堪。一股冥冥中的强烈直觉告诉他,若是他强硬闯进去,一定会身死道消,不会有第二个结果。

眼看着巨大的机缘要与自己失之交臂,越来越远,白凝冰心中当然焦急。

他在这里呆了三天,期间试着掩盖这股不断出现的变幻光影。

他已经尽量清理了仙蛊的气息,但这些却难以遮盖这些幻象。

砚石老人虽然助他升仙,但却没有给与任何仙蛊。

这现象是之前的宙道侦查仙蛊造成的,白凝冰动用凡蛊,当然遮盖不住。

当然,他若是有优秀的凡道杀招,也还有暂时遮掩的可能性。但他的底蕴很浅薄,绝没有如此手段。

“如果方源在这里,他会怎么做?”白凝冰苦恼之际,心头闪过方源的身影。

尽管不在意方源了,但不可否认方源的狠辣和狡诈,带给他深刻的印象,潜移默化的影响了他。

想着想着,白凝冰悚然一惊。

他猛地意识到:这仙蛊是砚石老人给的,这应当是影宗方面的计划。

影宗要干什么?

不秘密地夺取惊鸿乱斗台,闷声发大财,而将这个宣扬出来,唯恐全天下不知道的样子……

“涉及上古大战,涉及仙蛊屋惊鸿乱斗台,恐怕整个南疆都要震动。不管是正道蛊仙,还是魔道,或者散仙,都要为之疯狂。为了争夺仙蛊屋,他们不会退让,必定会搅荡风云,掀起血雨腥风。难道影宗的目的,就是在南疆人为地发动一场浩劫吗?但这对他们而言,又有什么利益呢?”

白凝冰猛地一拍额头,神色懊恼,目光重现清明。

“我这些天是怎么回事?!居然会被贪婪蒙蔽了心智。这不是我可以得到手的东西,这座无名的山峰,也会在不久之后,成为整个南疆的漩涡中心,将几乎整个南疆的蛊仙都吸摄进来。我不过是个六转蛊仙,手段缺乏。此地不可久留,还是先走为妙。对,回去试探砚石老人,问问他究竟想干什么。难道说他想靠此,将方源吸引过来?”

白凝冰赶了回去,但石亭中砚石老人早已经不在。

他前往影宗福地,却唤不开福地的门户,动用信道凡蛊也联络不上,似乎一下子整个影宗,包括砚石老人都彻底消失了。

白凝冰心中的疑云,越加浓重,一种被抛弃的感觉,让他隐隐觉得很不妙。

中洲,落天河。

此刻,落天河的源头河底,充斥着光明。

来自中洲的无数蛊仙,凝望着河底唯一的光源,心中都是一片火热。

在这光明之中,有一具仙僵闭目悬浮。他的相貌还一如生前,似乎充满生机,剑眉高鼻,一头碧绿长发,脊椎挺拔如剑,正是薄青!

“想不到薄青,转化成仙僵。”

“当年和他一起渡劫的还有灵缘斋仙子墨瑶。但此时此地,只见薄青,不见墨瑶,看来墨瑶已经烟消云散。”

“根据门派中的记载,那一场天劫震撼中洲,恐怖绝伦。薄青实力如此强大,竟然还能保留,真是叫人不可思议,他究竟是有多强?”

众仙议论纷纷。

“尸骸保存这么完整,不知道在其中隐藏了许多秘密。只要得到它,就能破悉薄青当年修行的奥秘。就算不留下一份剑道真传,单单这个研究,也能够大有斩获。说不定能再为门派增加一项真传呢。”

这个想法,大多来自十大古派的蛊仙。他们目光长远,考虑门派大局。

而类似剑一生这样的散仙魔修,则把主要注意力放在仙僵薄青携带的仙蛊上。

“他的身上,还有许多仙蛊的气息。难以想象,居然都保存下来了!这都是亚仙尊的剑道仙蛊啊,只要得到一只,我就满足了!”

中洲正道荣昌,魔修都被狠狠打压,处境是五域中最凄惨的。

这些魔修、散修,都不敢打仙僵薄青的主意,只想捞一笔,快速闪人。

其实在此之前,众仙深入落天河的过程中,多多少少都有些收获。

剑光斩杀的碎尸残肉,不在少数。只要细心,就能有发现。

但现在薄青的遗产,实在是太让人动心了。

人为财死鸟为食亡,既然亲眼见到了,谁想现在就走?

\end{this_body}

