\newsection{盗取薄青仙蛊}    %第三百四十六节:盗取薄青仙蛊

\begin{this_body}

%1
“这个是太古荒兽万目大明牛?!”方源望着落天河中,一头正在沉没的牛头将他的目光牢牢吸引。

%2
他连忙赶过去,施展手段,将这个牛头打捞上来。

%3
太古荒兽,乃是可匹敌八转蛊仙的强大生命。

%4
万目大明牛,更是体格庞大。只是这个牛头,就有小山般大小。

%5
并且非常沉重。

%6
方源也是拔过落魄谷的,然而单单这个牛头,居然比落魄谷还要沉重。

%7
“这牛头大部分的重量,都在牛角上。好家伙,这牛角已经被剑光劈断了,只剩下一半,居然还这么重!”

%8
方源暗自咂舌。

%9
这牛角可是八转级数的仙材,充斥金道、土道道痕。

%10
而这牛皮,也十分稀罕,水道道痕极其浓郁。可惜的是,方源只有牛头部分的牛皮。若是牛身上的万目斑纹牛皮,不仅有水道道痕,还有光道道痕,用来制作侦查仙蛊极佳!

%11
不过牛头上的一对牛眼,却是万目大明牛的精华所在。

%12
这对牛眼,漆黑深邃,拥有极其丰富的暗道道痕。方源得手的时候,里面还寄托着好多暗道野生凡蛊,五转居多,还没有来得及逃窜。

%13
万目大明牛的这双眼睛,并不能目视。

%14
这双眼睛原本就是瞎的。

%15
万目大明牛真正的看东西的器官,是它浑身厚实牛皮上的眼瞳斑纹。

%16
这很奇妙。

%17
像这种八转的生命,浑身的道痕已经形成奇妙的生命本能。

%18
又比如万里口蚯,这也是太古荒兽,拥有极其强大的断肢重生的能力。而赋予它这种能力的,真是它身上蕴含的奇妙道痕。唯有把这些道痕破坏,才能真正杀死万里口蚯。

%19
不过万里口蚯,是深藏在地沟或者地渊极深处的生命。落天河中并不存在。

%20
落天河中的生命,都以水道为主,其余土道、金道、风道、光道、暗道为辅。炎道数量很少。

%21
“嗯?这是上古荒兽冰瀑神猿的右臂。”

%22
“这是上古荒兽雪泥鳄的鳄尾。”

%23
“这个是什么?不管了先收起来再说。”

%24
方源早有准备,迅速搜刮。

%25
这些都是残肢碎体。有些方源能认出名堂,有些则不能。

%26
上古荒兽、荒植数量居多,太古级别的比较少。但这种八转仙材,哪怕体积很小。也是价值极为重大的。

%27
方源每一次的发现,心中都不由一阵欢喜。

%28
“好家伙!这个好像是生死鲲鹏啊!”方源惊叹。

%29
当他快将河面上的漂浮的仙材,都收刮完毕之时,他发现了第十九个八转仙材。

%30
这是一头楼船大小的灰鱼。

%31
它没有鱼鳍,而是在鱼鳍部位。有着羽毛翅膀的道痕纹理。

%32
“生死鲲鹏是律道太古荒兽,它极其特别。它的身上充斥生、死道痕,有两个生命形态。一个是鱼,一个是鸟。它作为鱼,生活八千年,死后化为飞鸟,再生活八千年,这才真正死亡。这头生死鲲鹏,看来已经快要化鸟的样子。恐怕已有七千多岁了。”方源感慨道。

%33
人虽然是万物之灵,但寿元很短。尤其是太古荒兽、上古荒兽。动辄数千年,上万年的生命。

%34
更别提那些草木了。

%35
这当中最著名的例子,就是北原的那座枯木山。

%36
其实不是山,而是一株地老木。

%37
这株大树已经有一百多万年的历史,长得极其巨大,所以称之为山!

%38
相比较它,就算寿元最长的元始仙尊,也不过两万五千岁。寿命最短的红莲魔尊,只有三千岁,连地老木树龄的一个零头都没有!

%39
这头生死鲲鹏。尸体相对保存完整,只有鱼头处,被剑光切去小半,整个鱼身都还在。

%40
再加上它身具生死道痕。这种特殊的律道道痕,在整个天地间都很稀有。因此价值更大。

%41
“这头生死鲲鹏,恐怕是我此行最大的收获了。”将生死鲲鹏收入仙窍,方源不禁在心中感叹。

%42
此时,落天河中血水沸腾。

%43
大量的猛兽,甚至一些太古级别的水草。都在争抢这些残尸碎肢。

%44
方源战力不足,也只能凭着眼疾手快,收刮河面上漂浮而出的这些仙材。

%45
不过时不时的,还有一些仙材翻腾出来。

%46
这些仙材,大部分都被方源收取。有一些,摄于水中猛兽之威,方源明智的没有和这些猛兽抢食。

%47
渐渐的,河面下,已经乱了套。

%48
这些残肢碎肉,对于这些生物而言,也是绝佳的食粮。

%49
食粮渐少,它们之间也开始发生血腥的厮杀和争斗。

%50
有时候,一些阵亡的上古荒兽的尸体,会翻到河面上。方源便趁机捞取便宜,还真叫他又发了一笔。

%51
随着时间推移,方源的收获越来越少,捡取仙材的风险也越来越大。

%52
他有些犹豫不绝,一方面想见好就收,就此退走,另一方面又想探探河底去。

%53
毕竟,星宿仙尊的诗词第二句,似乎就是指的薄青仙僵。

%54
但是这一切,都只是他的猜测,他并不太肯定。

%55
就在这时,星宿仙尊的歌声居然再次萦绕在他的耳畔。

%56
歌声寥落,英雄落魄,难挡命途多舛。

%57
折剑沉沙,千古兴亡,不尽天河滚荡。

%58
忧愁……

%59
幽夜漫漫魂梦长,问何处安乡?

%60
物换心移几春秋,唯天意苍茫。

%61
歌声渐消,第二股神秘信息,再度流淌在方源的心头。

%62
“剑仙薄青、墨瑶残魂!”方源震惊,眼中精芒一阵爆闪。

%63
“冒一次险又何妨?只有我得到剑仙薄青的剑道仙蛊,就算无法发挥真正威能,也可算是得到一座准仙蛊屋了。”

%64
方源十分果断,他立即离开这片河面,选择远处,钻入河水当中。

%65
因为剑光冲刷过一遍,又因为几乎全部的凶猛生命,都集中在那段河域争抢,因此方源潜水下去时,十分顺利。

%66
甚至,他还捡到了不少天然仙材。

%67
诸如雪泥,难化水,麒麟冰等等。

%68
来到神秘信息指点他的地方,方源在河水中缓缓停住身形。

%69
他将墨瑶假意提取出来。

%70
“方源,你想干什……”墨瑶假意起先大怒,但很快就被方源的手段弄得疲软,毫无防备,任凭方源施为。

%71
“明明当中,似乎自有天意。墨瑶,你这段假意害我弄塌了八十八角真阳楼,今天我就要在这里如数讨回。”

%72
墨瑶假意被方源拘于手中,缓缓消散,浓烈的气息散发出来。

%73
感应到这股气息,一道光柱乍然出现,无以伦比的剑气冲天而起。

%74
方源惊得差点爆退,但剑气并未波及他,只是肃清周围,反而对他很温和。

%75
在光柱之中,剑仙薄青的仙僵尸躯,缓缓浮现。

%76
方源不由地屏住呼吸,双目紧盯着薄青仙僵。

%77
后者没有睁开双眼,仍旧双目紧闭。这表明尸躯里面的墨瑶残魂,还在迷糊的状态当中。

%78
这股残魂只是感应到自己熟悉的气息,被方源的手段勾引,提前现身而已。

%79
他的周身,都被剑光笼罩,方源只能接近到三步之远。

%80
这个距离,已经就是极限了。

%81
再近的话,方源就遮掩不住。毕竟他此时所依赖的,只是墨瑶当年残存的一股假意罢了。

%82
并且这股假意,已经被方源多次折磨削弱。

%83
“基本上,现在其他的手段都不能动用。唯有利用墨瑶意志,伪装气息,将仙僵薄青身上的仙蛊给主动吸引出来。”

%84
方源小心翼翼施为,唯恐动作稍大,惊醒了对方。

%85
其实,最釜底抽薪的手段,是将仙僵薄青身上的八转仙元,统统收起来。

%86
如此一来,就算他仙蛊再多,再厉害,也是无本之源。

%87
不过,也不排除有梦翼仙蛊这等特殊的存在。

%88
仙僵薄青已经死了很久很久,他的仙窍死地已经彻底消散了。若非紧紧包裹全身的凝重剑光,方源甚至可以将其活捉。

%89
但现实并不如方源所愿。

%90
厚重的剑光,保护着仙僵薄青,稍有冒犯,就是雷霆般摧枯拉朽的剑光轰击。

%91
按照心中流转的神秘信息,方源依法施为,很快就收到成效,钓出一只仙蛊。

%92
这只仙蛊,像一只鹅蛋,蛋壳半透明,从外面望去,可见里面氤氲的碧绿光气。离得近了,方源还感到一阵刺骨的寒意,从这只仙蛊中散发出来。

%93
方源心中大喜。

%94
他虽然不知道这是什么仙蛊,有什么作用,但仙蛊强盛的气息,告诉他这是一只七转仙蛊!

%95
薄青当年是怎样的人物?就算再历史上,也可称之为尊者之下战力第一。他虽然达不到无敌天下的程度,但盖压中洲,还是轻轻松松的。

%96
他所用的仙蛊,恐怕都没有六转的。最少是七转,核心剑道仙蛊必定是八转级数。若非如此,他不会达到历史记载中的恐怖战力。

%97
方源用墨瑶气息很是谨慎地包裹着这只仙蛊,然后动用智道手段,将其一层层封印。

%98
这些封印,其实很脆弱,只有欺骗作用。

%99
就是欺骗这只仙蛊中的意识,让仙蛊中的意识认为:它还在主人的手中,没有落入他人之手。

%100
这只仙蛊,方源暂时还不能动,先收入仙窍。

%101
随即,他如法炮制,又针对薄青,施展手段。

%102
这一次,过了好一会儿,第二只仙蛊也被方源吸引出来。

\end{this_body}

