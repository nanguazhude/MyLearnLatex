\newsection{分明大势来交易}    %第一百六十五节:分明大势来交易

\begin{this_body}

%1
ps:想听到更多你们的声音,想收到更多你们的建议,现在就搜索微信公众号“qdread”并加关注,给《蛊真人》更多支持!

%2
雪胡老祖以一敌二,击败药皇、百足天君联手。

%3
此战震荡北原!消息如飓风一般,将北原整个的蛊仙界刮得凌乱不堪。

%4
并且,在中洲蛊仙们的亲眼见证之下,消息第一时间传入中洲,惊动中洲十大古派的诸多首脑,纷纷将雪胡老祖的名字,记在内心最深处。

%5
几天后。

%6
中洲,狐仙福地。

%7
黎山仙子、黑楼兰造访,带来第一手的情报。

%8
方源手持这一只信道凡蛊,闭上双眼,宛若置身在大雪山福地,借用了黎山仙子的视角观看。

%9
天空极顶,正重现着三位八转大能相互切磋的战斗景象。

%10
从始到终,大气磅礴,叫他看得心驰神摇。

%11
虽然间隔很远,交战三方远在白天深处,但遥遥观看,也可从中感知到八转的浩瀚威能。

%12
蛊仙中,转身越高,差距越大。六转七转之间,尚且还有精英,可以偶尔力战七转。

%13
七转八转之间,相互差距就有天壤之别,云泥之差了。能以七转修为力战八转,整个人族历史上都是少见!

%14
到了九转,以一人之力横扫天下,无人可挡,举世无敌,天地第一!八转蛊仙根本不是九转的对手。皆是大败亏输,从未有过例外。

%15
将这景象一连看了三遍,方源便将蛊虫还给黎山仙子,睁开的双眼中已是一片平静。

%16
雪胡老祖的强势,方源早有心理预料。在前世记忆中就叫中洲狠狠此了几次大败仗,甚至杀了不少中洲八转。

%17
这次亲眼目睹之后,方源对北原局势的理解,则又更加深刻了一分。

%18
“北原正道,有两位八转蛊仙,分别是药皇和凤仙太子。论威望,药皇是当之无愧的正道第一。凤仙太子却和中洲有瓜葛。是中洲埋在北原的最大暗间。”

%19
“八十八角真阳楼倒塌,王庭福地被毁,导致北原动荡不安,巨阳仙尊苦心孤诣布置的格局,土崩瓦解。秩序崩坏,又因争夺仙蛊,正魔激战。新仇旧恨矛盾重重。虽然秦百胜举办了拍卖大会,但众仙只是忙着消化战果,短时间消停,矛盾却越积越深。”

%20
“王庭福地毁灭,北原正道损害最大,魔道却是受益匪浅。一直以来,巨阳仙尊的布置,都是为了他家天下的野望。北原正道皆是黄金血脉,打压得魔道、散修,几乎抬不起头来。现在八十八角真阳楼一倒。魔道、散修的头顶上就少了一座大山。不管是前世记忆里,还是在今生,百足天君都是在王庭福地毁灭之后,才建立超级势力。他是八转蛊仙,却不是黄金血脉,都要如此,无非就是害怕各大黄金超级势力的联合打压。”

%21
“百足天君要建立正道势力。自然要和药皇等人打好关系,和魔道划清界限,表明阵营立场。因此药皇上门索要方寸山,就主动去了。但他没有想到,雪胡老祖如此凶猛,居然将他和药皇打得溃败。”

%22
“对于雪胡老祖而言,要炼制八转鸿运齐天仙蛊,实在是树大招风,惹来无数觊觎。各方正道都不想他成事。尤其是八十八角真阳楼倒塌,正道势力对魔道的猜疑和忌惮更重,唯恐这些外人坐大。”

%23
“但偏偏雪胡老祖需要很多资源,才能炼出鸿运齐天仙蛊。他广派麾下的魔道蛊仙,四处采集炼蛊仙材,发生的摩擦碰撞不少,今后这种摩擦碰撞必然会更多,产生更多的矛盾和阻碍。雪胡老祖此行强势出手,最大的动机,应当是震慑北原蛊仙界,为炼得鸿运齐天仙蛊清理未来路上的障碍。”

%24
“北原局势原本波云诡谲,暗流汹涌,仿佛是即将喷发的火山。现在雪胡老祖击败药皇、百足天君联手,影响十分重大。简直是一扫尘埃,直接跳过正魔大规模交战的步骤,立即将魔道带领到强势地位,盖压正道一头。接下来的日子里,北原蛊仙界魔道嚣张,正道低头。这就是这场交战的余威,接下来北原的大势!”

%25
方源年老成精,看得分明。

%26
六转,虽然超凡脱俗了,只是蛊仙中的最底层。

%27
七转,才是中坚力量。

%28
到了八转,通常都不轻举妄动,平日里的一举一动,甚至无心之语,都会波及开来,带来庞大影响。

%29
什么是大势,每一位八转蛊仙就是大势。

%30
“六转、七转,通常只能顺应大势。什么时候,我才能成为这样的大势?酿造这样的大势?”方源不免心生向往。

%31
“好了,方源。我们这次来,主要是给你带来你想要的东西,达成此次交易。现在,都交接给你。”黎山仙子这时开口。

%32
她给方源带来了一群白莲巨蚕蛊,还有大量的星念蛊。

%33
原来大雪山方面,之所以知道东方长凡死后福地的具体位置,都是方源提供的。

%34
东方长凡夺舍重生,刚刚成仙不久,仙窍就算再上等,里面的资源几乎为零。东方长凡死后,形成的福地,却是紧密统合,天地一体。防御程度,不是那种公共福地可比的。

%35
如果形成地灵,极可能是对方源的复仇执念,恨透了方源。再算上那些调查方源的幕后势力,方源不想因为净魂仙蛊,而陷自己于危险境地。

%36
于是,他便通过黎山仙子,借助大雪山势力,来帮助他达成自己的目标。

%37
一个福地的价值十分巨大的,方源和黎山仙子达成约定。方源提供情报,告知福地具体位置,将这块福地直接让给黎山仙子。而自己需要获得的是里面的白莲巨蚕蛊,以及普通的星念凡蛊。

%38
至于那些仙蛊,方源已经从东方长凡魂魄处得知。都已经被东方长凡一念自毁了。

%39
方源得到白莲巨蚕蛊,按捺住心中欢喜。待看到星念蛊居然如此之多,多达五万只时,心中的喜悦再也遮掩不住,从眼中流露出来。

%40
这五万只星念蛊,来的太及时了!

%41
真是一场及时雨。

%42
黎山仙子察言观色,故意咳嗽两声。开口道:“恭喜你了方源。这一次东方一族覆灭,你斩获最多。竟然得到了东方长凡的魂魄,这可不仅仅只是一套完整的智道传承。东方长凡掌握的运营机密,就是极为巨大的价值。更何况,你在碧潭福地中,本就搜刮了不少!”

%43
黎山仙子说着,毫不掩饰艳羡之情。

%44
因为雪山盟约还未到时限。盟友之间讲真话,方源对于擒拿东方长凡魂魄这事,并未做任何隐瞒。

%45
其实也隐瞒不了。

%46
很多蛛丝马迹都能推断出真相,方源索性在事后,就直接通知了黎山仙子、黑楼兰,不加隐瞒。

%47
此刻他道:“仙子,你得到了方寸山,那可是《人祖传》的记载之物,价值可不输给我。”

%48
“不必说这种奉承之语了,你有荡魂山。看不上方寸山。呵呵,大家心知肚明,你才是此次最大的赢家。再说这座方寸山……”黎山仙子苦笑起来,“我正要因此拜托你,向你求助来了。”

%49
方源面容一肃:“哦,愿闻其详。”

%50
原来,黎山仙子和东方一族签订过盟约。算作盟友,不能相互攻击。但黎山仙子为了抢夺方寸山,在这个过程中却是违背了这个誓言。

%51
黎山仙子的信道虽然半路出家,但为防备无意间或者不得不违背誓言的情况,早年前就请人下过一道宙道仙级杀招。

%52
这宙道杀招的效果,是令黎山仙子万一违反了山盟蛊签订的盟约,那么盟约反噬的伤害将暂缓一段时间之后,才会爆发。

%53
方寸山小人族和东方长凡只是盟友关系,不是真正意义上的东方部族之物。即便如此,违约的反噬还是相当恐怖,黎山仙子本身就受了重伤,要抵抗这股反噬,信心十分不足。

%54
于是,黑楼兰便替黎山仙子想到一个妙法。

%55
当初,黎山仙子和东方长凡盟誓,动用了山盟蛊,誓约正是应在方寸山上。

%56
若是方寸山毁灭,誓约无疑就失效了,这样一来自然就没有反噬伤害了。

%57
方寸山若毁掉,黎山仙子、黑楼兰之前的拼死冒险,所做的努力就成了竹篮打水一场空,这自然是不行的。

%58
黑楼兰此法是想:方寸山毁灭之后,等到宙道杀招效果消失,违约的反噬发作之后,再请太白云生出手,运用江山如故,重建方寸山。

%59
如此一来,方寸山恢复从前状态,虽然黎山仙子和东方一族还有盟约,但今后只要不触犯,也就没事了。

%60
黎山仙子巧妙避过违约的反噬伤害,还得到了修复好的方寸山。至于为什么要求助方源,显然是因为黑楼兰、黎山仙子知道,方源和太白云生之间,当以方源为首。

%61
方源一笑,赞同道:“这的确是一个妙计,我愿意撮合此事。至于相关报酬,我们可以详谈。”

%62
黎山仙子见方源这个笑容,心头微微一沉,暗道:“来了。”

%63
她是深知方源面厚心黑的,眼下明显是自己要被狠宰一顿的节奏,不由苦笑点头。

%64
一旁的黑楼兰却似有所准备,这时开口提到:“方源,如果我没料错,你这狐仙福地的地灾快要降临了吧?你看这样,我留下来帮助你守卫狐仙福地,对抗天劫。就当做太白云生恢复方寸山的报酬了,怎么样?”(我的小说《蛊真人》将在官方微信平台上有更多新鲜内容哦,同时还有100\%抽奖大礼送给大家!现在就开启微信,点击右上方“+”号“添加朋友”,搜索公众号“qdread”并关注,速度抓紧啦!)

\end{this_body}

