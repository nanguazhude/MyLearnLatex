\newsection{再见智慧蛊}    %第三节:再见智慧蛊

\begin{this_body}

%1
山洞中只剩下方源一人,太白云生已经离开。。。

%2
他坐在石凳上,八只巨大的怪臂有的自然垂下,有的背在后面,有的怀抱在胸,再配合他高大的身躯,只是坐着,就显得渊渟岳峙,令人望而心悸。

%3
如何处置太白云生,方源经过了深思熟虑。

%4
方源只相信自己,从不真正相信别人。

%5
若杀了太白云生,他便会得到人如故仙蛊、江山如故仙蛊,以及宙道六转仙窍。但这样利用,却不是利益最大化。

%6
不得不承认,太白云生是个很好用的棋子。

%7
这个“好用”体现在两个方面。

%8
第一个方面,他有用。他未成仙时,就是名满北原的治疗大师。如今成就了六转宙道,福地更是上等福地,别忘了,他本身还是飞行大师。利用价值很大。

%9
第二个方面,他很好掌控。方源曾对他搜魂,因此知根知底,了解透彻。太白云生是个老好人,有良心,知恩图报。方源对他有恩,在真阳楼中三番五次地帮助过他,尤其是最后将昏死的他带离大同风幕,又将两大仙蛊还给他,已经得到了太白云生的彻底信任。

%10
这点从刚刚太白云生选择座位时,就可以看出来。

%11
山洞中的石墩并不少。方源形象丑恶,身材高大,靠近方源的常人都会感到压力,会下意识选择回避。但太白云生却却偏偏选择了,离方源最近的位置坐下。

%12
这说明。在他心中,方源已经成了最亲近的人之一。一点都不担心方源会加害于他。

%13
太白云生曾经害过高扬、朱宰的性命。这恰是最关键的一点!

%14
求生的本能在一时间,占据了上风。这是人之常情。事后,太白云生心怀愧疚,念念不忘,形容憔悴,饱受折磨。在方源陷害之下,他几近入魔,冲动升仙。起初的目的就是为了寻死。

%15
因为愧疚而寻死觅活,可见他的良心。

%16
机缘巧合之下,太白云生得到了巨阳意志的帮助。成功登仙,而后陷入昏迷。

%17
方源利用琉璃楼主令,深入真阳楼某处关卡,将他唤醒。在这关键的时刻。告诉他师门的“秘密”。这恰恰击中了太白云生的心结软肋。

%18
太白云生否定自己。因此痛苦万分,感到万分迷茫。

%19
就在这个时候,方源恰好出现,给了他门派组织,给了他归属感,给了他安全感和温暖。之后出入他的宙道仙窍,大大咧咧地将仙蛊交给他保管,就是给他初步的信任。

%20
之后。在真阳楼中共患难,生的一切。让太白云生彻底相信方源。从某种意义上来讲,还要感谢墨瑶干扰了方源的思想,否则怎么可能得到太白云生的如此信任呢?

%21
“太白云生害了高扬、朱宰二人的性命,此事大大违背了他的价值观,一直心怀愧疚。他觉得自己不是这样的人,却偏偏做了这样的事情。现在我是他的恩人,他亏欠我,一定会百倍千倍地偿还我。这不仅仅是因为他知恩图报,而且给了他重新证明自己是个好人的机会!”方源心中冷静地分析着。

%22
坏人当然不会知恩图报,甚至会恩将仇报。太白云生今后为方源牺牲越多,越证明他是好人。如此一来,就能抵消他对高扬、朱宰二人的愧疚。这就涉及到地球上心理学中的平衡和补偿心理。

%23
从某种意义上,方源给太白云重新证明自己的机会,重新生活下去的理由。

%24
方源对此知之甚详,因此救活了太白云生,还给了他人如故和江山如故两大仙蛊。

%25
“我手中的仙蛊太多了,喂养的费用将十分高昂。这个时候,将人如故、江山如故抛给太白云生,无疑更好。同时,仙窍已死,再不能产仙元。我的青提仙元只有十九颗,越用越少。虽然还有些白狐仙元可用,但能够利用到太白云生的青提仙元,何乐而不为呢?”

%26
一想到仙蛊的喂养,方源就头疼起来。

%27
养蛊虫如养情妇,蛊师常常养不起,用不起。自从方源有了狐仙福地之后,喂养凡蛊不成问题。但如今他有了这么多的仙蛊,狐仙福地也不能支撑这样的高价费用。

%28
这真是幸福的烦恼!

%29
很多蛊仙手中没有一只仙蛊,苦苦追寻。方源前世也在为一只春秋蝉挣扎,辛苦炼成后,就被围剿自爆。

%30
现在的方源,则在为如何喂养这么多仙蛊,而伤透脑筋。

%31
方源一个人静静地坐着,试图从前世记忆中,寻找灵感,尝试了一会儿,他却主动放弃。

%32
“唉,变成僵尸之后,思维果真僵化了许多。刚刚和太白云生对话,就感到思考缓慢拖拉。现在思考问题,根本跟不上原来的节奏。”

%33
他唤出地灵小狐仙:“智慧蛊的情况如何?它现在又在何处?”

%34
小狐仙乃是福地地灵,福地的一切变化、情况,基本上都在她的掌握当中。

%35
“主人,智慧蛊被你带进来后,深入地底,找到了一处石人洞穴,目前似乎正在休眠。”小狐仙闭上双眼,稍稍感应了一下后,旋即睁开水灵灵的大眼睛答道。

%36
方源点点头,命令道:“带我去吧。”

%37
智慧蛊休眠的这处地底洞穴空间颇大,方源被小狐仙挪移进来,两丈的身躯也能活动自由。

%38
察觉到方源的到来,智慧蛊上闪烁了一下五彩的华光。

%39
小狐狸坐在方源的肩膀上,好奇地直盯着智慧蛊。

%40
这可是九转智慧蛊啊!

%41
虽然小狐仙可以感应福地的每一处地方,但如此近距离接触,还是少数。

%42
“智慧蛊啊,我带你逃脱陷阱,救了你的性命。按照我们之前达成的协议,该是你履行一部分内容的时候了。”灰暗的山洞中,回荡起方源干涉的声音。

%43
智慧蛊静悄悄的,几息之后,它慢慢地悬浮,停在半空当中。

%44
同时,它撑起智慧光晕。

%45
“好漂亮啊!”小狐仙双眼熠熠闪光,当即欢笑一声。

%46
球形的智慧光晕,不断闪烁着五颜六色,将方源和小狐仙包裹进去。

%47
方源顿时感觉到念头迅碰撞,思考如电,以一种极快的度得到思考的结果。

%48
“这种感觉……好舒服呀。”小狐仙缓缓瞪大双眼,呆愣呆愣,沉浸在智慧暴涨的快感中。

%49
但几个呼吸之后,她的身形开始渐渐变淡。

%50
“小狐仙,你是执念所化,虽然结合了天地之力,但仍旧受到智慧蛊的克制。这里你不可久待,回去吧。”方源伸出一根手指头,对着小狐狸的脑袋轻轻一点,唤醒了她。

%51
小狐仙捂住小脑袋瓜,白皙的脸蛋上升腾起两朵兴奋的红晕。

%52
“主人,主人,原来人家可以变得这么聪明。好厉害的智慧蛊啊!”她叫起来,像是小朋友现了新玩具似的,十分开心。

%53
“主人最厉害了,连智慧蛊都能弄到!就让人家再待一下会儿,好不好嘛?”她竟然开始拍方源的马屁。

%54
方源哈哈一笑:“看来智慧蛊开动了你的小脑筋,赶紧出去。”说着,伸出两根手指,轻轻按了按小狐仙的脑袋。

%55
小狐仙那可爱的小脑袋瓜儿,便被方源的手指一压一点,一压一点。

%56
小狐仙顿时微微皱起琼鼻,甩了甩脑袋:“主人,那我走啦。”

%57
她不敢违逆方源的话,下一刻闪烁消失。

%58
地底洞穴平静下来,方源也不盘坐,反正僵尸之躯坚硬厚实,站着和坐着的感觉没有分别。

%59
他直接站在光晕当中,先第一件事情,是观察仙窍中被镇压的墨瑶意志。

%60
墨瑶意志没有受到智慧光晕的任何影响。

%61
尽管之前在大同风幕下已经观察过,但现在方源再次确认,这才将心放下。

%62
“仙窍既成,就成了一方小世界。仙窍之壁,就是天地之壁,隔绝内外,外力都几乎影响不到里面去。”

%63
方源这才闭上通红的双眼,开始静静思考感悟。

%64
他已经成为僵尸,脑海中念头的生产度下降了几个层次,思维也僵化。而智慧光晕,却是让念头迅攒动,极碰撞,在短时内消耗大量念头产生思考结果。

%65
若是寻常僵尸,置身其中,恐怕顷刻间聪明一下,随后就要变成白痴。

%66
但方源的脑海中,却还有意志。

%67
这些意志,都是由智道蛊虫产生,也可以思考。

%68
方源之前为了对付墨瑶,钻研智道,购买了特意蛊、刻意蛊、锐意蛊等等蛊虫。这些意志在北原之行的最后关头,也帮了他不少忙。

%69
现在方源的头脑,已经跟不上,只能用这些意志代替。

%70
他真元无限,意志也几乎无限。在智慧之光的笼罩下,原本稳定的一股股意志,立即像是见到阳光迅消融的积雪一般,越缩越小。

%71
随之而来的,是一个个的灵感闪现,尘封的记忆变得清晰。只要方源集中精神回忆,这些繁芜庞大的记忆一角,就会栩栩如生地重现在脑海中。

%72
在这一刻,世界都仿佛变得光明起来。

%73
虽然还未思考出真正的结果,但是方源已经有了许多解决问题的方案。只要遵循着这些道路走下去,方源坚信自己必定能找到最合适的解决问题的办法!

\end{this_body}

