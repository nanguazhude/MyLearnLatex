\newsection{不甘平凡一类人}    %第一百五十八节:不甘平凡一类人

\begin{this_body}

《人祖传》第三章,第十八节。

人祖踏上独属于自己的人生之路,脱离了生死门,却落魄在平凡深渊之中。

他徘徊在渊底,迷茫地前行,过了很久很久的时间,想走出平凡深渊,却始终找不到出路。

在苦闷中,思想蛊指点他迷津:“人啊人啊,你循着其他存在踩踏出来的脚印行走,这固然会轻松安全,但你要想不平凡,想走出平凡深渊,一味遵循别人的路,恐怕机会不大。你既然走了自己的人生路,一切的难关,就得你自己闯,就得创新,就得有自己的脚印。”

人祖得到指点后,恍然大悟,十分欢喜,便照着思想蛊建议的去做。

他脱离地面上原本存在的脚印,开始涉足没有较硬的地面。

平凡深渊的地底,很不好走。

有些地方是泥泞沼泽,十分容易泥足深陷,而且恶臭熏人。有些地方是荆棘满布,尖刺密密麻麻,人祖被刺得伤痕遍体。还有的地底,埋藏着刃蛊。人祖踩在上面,脚底就被尖锐的刃边割伤,伤口宽大,血液横流,走起路来,痛到心脏。

脚底伤口传来的疼痛,让人祖决定轻轻的走。

但走着走着,人祖在平凡深渊中迷了路,很多时候,他会重复之前的路,做无用功。

人祖渐渐发现了这个问题,十分困扰。

怎么样才能在深渊中。不迷路呢?

思想蛊便告诉他:“人祖你要不想迷路,完全可以自己做到。你害怕痛,轻轻的走,踩在地上的脚印太轻太浅。这平凡深渊中刮着平常风,掀起凡俗土。灰尘落下,很容易就把你的脚印痕迹掩盖了。你要想不迷路,就要留下深深的脚印。你明白我的意思吗?”

人祖点头,说明白了。

于是他开始重重地行走,每一步都用力踩在地面上,踩实在了,踩出深深的脚印。

这样一来。他每走一段路。都在上面形成深深的痕迹,十分清晰。人祖只要看到这些脚印,就明白这些路他已经探索过,路上没有走出平凡深渊的出口。

但好景不长,时间久了,再深的脚印都会被风尘渐渐掩埋。

人祖为此苦恼不已,求教思想蛊。

思想蛊再次提出建议:“人啊。你虽然走出深深的脚印,但却故意规避了那些荆棘和利刃。每次碰到这些地方,你就绕开来走。这是不行的。你不能奢望你不平凡的同时,又是舒适的。”

人祖得到启示,咬咬牙,便故意走上遍布荆棘,埋藏利刃的路。

每走一步,他都踩出深深的脚印,不管痛楚多寡,不管伤口深浅。[\&\#26825;\&\#33457;\&\#31958;\&\#23567;\&\#35828;\&\#32593;\&\#77;\&\#105;\&\#97;\&\#110;\&\#104;\&\#117;\&\#97;\&\#116;\&\#97;\&\#110;\&\#103;\&\#46;\&\#99;\&\#99;更新快,网站页面清爽,广告少,

他的汗水和血液。顺着他的脚底,被深深地踩在平凡的泥土中。

当他抬起脚,往前走时,他留下的脚印中,冒出一棵的小草。

草的名字,叫做成就。

每一个脚印中,都生长出一棵小草。

一棵棵的小草。风吹不倒,尘埋不了,顽强生长,比脚印保留的时间要长得多。

“这样一来,我就不怕迷路了。”人祖十分开心,咬着牙,顶着疼痛,顽强地走在充满刀刃荆棘的路上,不怕流血流汗。

他越走越远,不再迷路,也不会在原路徘徊绕圈,他涉足到以前没有走到的地方。

他用汗水和血液种下的草,也越来越茂盛,越来越高。

渐渐的,从脚印中生长出来的,也不再是成就小草,而是成了成就小树。

随着时间推移,小树渐渐成为大树,树叶茂盛,郁郁葱葱,甚至长出了果实。

人祖走得累了,就躺倒在树荫下休息,摘下香甜多汁的树果品尝果腹。

随着时间推移,他几乎走遍了平凡深渊的每个角落,他走过的地方形成一大片的森林。

人祖看着身后的这些森林感到幸福和快乐,但是当森林蔓延了整个平凡深渊,人祖仍旧找不到脱离平凡深渊的出路。

他心中焦躁失望。

他摘下一颗树果,放入口中,树果不再香甜可口,反而苦涩难咽。

人祖感到很奇怪,他查找原因,很快发现:原来不知不觉间,他的身体里又长出第二个心。

这颗心,叫做不甘。

顾名思义,品尝任何东西,都不会尝到甘甜。

人祖吃着苦果,看到漫无边际的森林,再也感觉不到快乐和幸福。

这时,他原本的另一颗心,孤独之心中,传出自己蛊的声音:“人啊,我替你想到了走出深渊的方法。你可以种出一棵高耸伟大的成就树。只要这棵树高过平凡深渊,你就可以顺着树枝干攀升上去,脱离这里了。”

人祖一想,双眼骤亮:“是啊,这的确是个好办法。”

但旋即又很苦恼:“我又该如何,种出伟大到高出平凡深渊的大树呢?”

自己蛊:“你用足底的血,种出这片森林,这些都是平常的树。你用心中的血,应该就能种出伟大的树来。埋在平凡深渊地面下的,有许多的刃,你不妨将这些刃插在心头,滴下心血,去浇灌出树来试试。”

人祖便照着自己蛊提出的办法尝试。

刃插在心口,传来剧烈的痛楚。

这种痛,是身上伤口的千百倍!

然而浇灌出来的树,果然又高又大,超出原先的树一大截。

人祖痛苦却又欢喜,继续往心口插上更多的刃,滴下更多的心血。

他插的刃越多,心血流淌的也越多,种出的树木越来越高大。

但就算最高大的树。也不过抵到平凡深渊的一半高度。

人祖继续坚持,希望蛊一直伴随着他。

当他种出的大树的树冠,和平凡深渊几乎一样平齐的时候,大树的树干陡然裂开,从里面蹦出一个女儿。

“父亲。父亲!”女儿投入人祖的怀中,十分亲爱。

这是人祖的四女儿,名为森海轮回。

人祖也十分欢喜,抱着女儿玩耍逗乐。森海轮回饿了,就为她摘取树果,喂她吃。

“好甜,好甜。”森海轮回十分喜欢吃树果。长得白白胖胖。

她整天嬉戏在森林中。感到十分幸福快乐。

人祖仍旧向往着走出平凡深渊,森海轮回屡次劝说:“父亲啊,你何必这么劳累呢?待在这里多好,有大树为我们遮挡烈日,有甘甜多汁的树果果腹,我们可以在这里嬉戏玩耍,一直终老都会很安逸。”

人祖摇头。态度很坚决,种出了更加伟大的树,树冠的枝条彻底探出了平凡深渊。

森海轮回抽泣,拉着人祖的手,哀求:“父亲,你不要丢下我。我不会爬树,你走了,我一个人留在这里,又不会种树。树果有限,迟早有一天我会饿死的。”

人祖说:“我怎么会丢弃你呢?你是我的女儿。我会背着你,一起爬上去。”

于是父女俩开始攀爬大树。

越爬越高,人祖也越来越累。森海轮回是个沉重的负担,就算没有她,人祖爬上树冠也十分危险,更何况要带上一个完全不会爬树的人呢?

更麻烦的是,大树再咯吱作响。摇摇欲坠。

思想蛊告诫人祖:“不妙了,人啊,你背着你的女儿,要爬出平凡深渊是想当然的事情。这是你的成就之树,难以让他人脱离平凡。就算是你的女儿,也不例外。”

人祖摇摇头:“我不想放弃。”

希望蛊也劝道:“将她放下来吧,不然你根本爬不出去。你虽然有孤独之心,不甘之心,但种下这么多的树,你的心血已经干涸了。这是你最后的希望!但你看,你脚下的这棵大树已经就要倒了!”

人祖摆摆手:“我还想试试。”

自己蛊见人祖一意孤行,其他蛊虫劝说都失败,直接飞出来。

“啊,真是急死我了!”自己蛊根本就没有和人祖商量,直接一口咬在森海轮回的手上。

森海轮回喊痛,十分愤怒,伸出手来想要拍死自己蛊。

但如此一来,她就松开了手,加上大树剧烈摇晃,她从人祖的背上摔落下去,一路跌跌撞撞,在无数枝叶的缓冲下,最终屁股着地,摔得龇牙咧嘴,痛得哇哇大哭。

“女儿!”人祖呐喊,想要下去。

“来不及了,大树就要倒了!”自己蛊在人祖背后一推,人祖下意识迈开大步,一下子跨出,走出了平凡深渊!

大树轰然倒下。

人祖趴在悬崖边上,失去了回去的路,无可奈何地大吼:“女儿,我一定会回来救你的。”

森海轮回呜呜的哭,十分悲伤无助:“父亲,你怎么这么狠心丢下我,留下我一个人在这里生活!我好怕!”

人祖听着哭声,简直肝肠寸断,连忙在平凡深渊四周打转,却找不到任何进去的路。

“没有用的。”自己蛊道,“你的成就之树,即便倒掉,也足以证明你的伟大。伟大的人,根本就不平凡。不平凡的人,怎么可能进入平凡深渊呢?”

……

狐仙福地,荡魂行宫。

方源将手中的《人祖传》合上,叹了一口气,神色复杂。

水往低处流,人往高处走。不甘平凡,乃是人之常情。东方长凡如此,方源亦如此。

“从这一角度而言,你我都是同一种人呐。”方源淡淡说道。

在他的面前,东方长凡的魂魄被拘束着,此刻一脸冷笑,张口说话因为失去了肉体,他发不出声音,但是有魂力的波动,被方源尽数捕捉。

只听这位北原当代第一智道蛊仙说道:“你要想搜魂,尽管来!不过想得到我的传承,呵呵,你还想的太简单了。”

ps:月初,请大家多支持,求月票,求推荐票!(想知道《蛊真人》更多精彩动态吗?现在就开启微信,点击右上方“+”号,选择添加朋友中添加公众号,搜索“zhongenang”,关注公众号,再也不会错过每次更新!qdbook)(未完待续。)

\end{this_body}

