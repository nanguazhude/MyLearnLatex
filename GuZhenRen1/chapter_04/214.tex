\newsection{提升全力以赴蛊}    %第二百一十五节:提升全力以赴蛊

\begin{this_body}

%1
听到余木蠢叹气,宋紫星也叹道:“最关键的就是时间。我若有充分的时间,也能痊愈,恢复战力。可惜时间太紧了,原本预估还有十年的时间,没想到必须就要发动了!”

%2
“这些年来,我们做的准备也不少。但凡计划不如变化,宿命仙蛊既然没有彻底毁去,发生这些变故也很正常,不是吗?”余木蠢道。

%3
就连方源这个从前世五百年重生过来的人,都不知道宿命的存在。余木蠢却知道,并且宋紫星也一脸了然的样子。

%4
“哦,对了。”余木蠢想到哪里,说到哪里,“来之前,遇到了一位不错的毛民小辈,我已经将我的炼道传承交给他了。”

%5
作传和传承,是蛊师世界的文化特征。

%6
就好像是中国古代,千方百计也要认祖归宗。人中豪杰英雄,都要博得个青史留名。人死后入土下葬时,会有大量的金银珠宝陪葬,甚至有奴婢妻妾殉葬。日本战国时期,武将上战场时都要穿着华丽的铠甲,丝毫不顾吸引火力的危险。

%7
还有埃及、罗马,从古代到现代,都有类似的文化情结,虽然各有差异,但归根结底都是同一个出发点那就是印证自己的存在。

%8
向这个世界,向其他人证明,我曾经存在过,活过。就算死了,这个世界上也曾经存在过这么一个人!

%9
所以作传,就不难理解。有的人在生前就作传,给自己作传。有的则是死后,由后辈给前辈作传。还有的并无后人,却是由敌人给他作传!这种情况下的人物传记。反而更实事求是,描绘生动具体。敌人往往比亲人要更了解你。由敌人作传,向来不吝赞美之词,往往会成为一时美谈。

%10
而传承也是如此。

%11
作传只是形容一个人的生命轨迹,而传承却是一个人立足现世的力量根基。是他(她)对大自然的理解。对天地的思考,是对真理的总结。

%12
就像是中国古代,就算是手艺人在死前,也不愿留下遗憾,千方百计地寻找一个传人,不愿祖祖辈辈传承下来的手艺丢失了去。

%13
手艺。传承,是对本身价值的肯定,是对自己劳动成功的珍惜。留下传承,就算自己死了,也能在世间留下属于自己的印记。

%14
余木蠢留下传承。就有一种临终托孤的不祥意味。

%15
什么人要留下传承?

%16
好好活着,前途光明的时候,没有人会留传承。

%17
好端端的,留下传承干什么?教会徒弟,饿死师傅。人心叵测,稍有不察,恐怕就要多出一个竞争者。

%18
往往只有人弥留之际,意识到死亡来临时。才会留下传承。

%19
比如花酒行者,比如血海老祖,比如东方长凡。

%20
余木蠢留下传承。显然是要参加大计,对自己活下来的可能并无丝毫信心。

%21
是什么样的大计划,能令这位深不可测的炼道蛊仙强者,也无生还的希望?又是什么样的动力,能让他如此心甘情愿?

%22
甚至就连宋紫星,刚刚逃出生天。好不容易逃得一命的中洲魔道第一人,为了这项大计。也甘愿牺牲!

%23
还有一点,余木蠢居然在宋紫星的面前。亲自说出自己选择传承的继承者。这点也很不正常!

%24
若是宋紫星心生歹意,悄悄地前去截胡,会很容易就将余木蠢的传承抢到手中,壮大他自己。

%25
而对余木蠢而言,自身的底细也被泄露,便会受到宋紫星的针对和克制。

%26
余木蠢能够将传承这样的,如此珍秘的信息,告诉宋紫星,显然对后者极为信任。

%27
这种信任的程度,可称得上掏心挖肺了!

%28
宋紫星叛逃万龙坞,声名狼藉,竟然能够获得余木蠢如此信任?是他的人格魅力,还是其他原因呢?

%29
“是要做些后手准备了。”听到余木蠢留下了传承,宋紫星脸上一片淡然,似乎并未起什么抢夺之心。

%30
“我也给你看样东西。”宋紫星神秘地一笑。

%31
下一刻,血池中掀起一阵波澜,从血池底部缓缓浮出一个血胎。

%32
血胎有一匹小马驹大小,胎膜中隐约有一个人形模样,宛若婴儿蜷缩着身躯,似乎在沉眠。

%33
余木蠢看到这个血胎,顿时瞳孔一缩,有些难以置信地道:“这,这……你居然将血魔解体发展到了这种程度?!”

%34
宋紫星笑了三声,流露出得意之情:“虽然一直没有搜集到完整的血神子仙蛊方,但好歹也收录了一些血神子残方。这个血胎就是我依照血神子残方,还有北原那边的血道方法,再结合我的仙道杀招血魔解体形成的。”

%35
“血魔解体之后的假身,需要时刻耗费仙元,维持存在,有着时间的限制。但这个血胎孕育出来的假身,不耗费仙元,至少能维持两三百年。而且本身能够修行,有独立的思维,还能修行,一步步提升,最后升仙。容貌方面,当然和我一模一样。”

%36
“你是想打算?”余木蠢露出一丝了然之色。

%37
“不错。我死了不要紧,关键是要大计功成。若是大计失败,我方就得考虑失败后的方针。宋紫星需要活下来。就算我牺牲了,血胎中还会孕育出第二个宋紫星来的。”

%38
余木蠢凝视着血胎,露出感兴趣的神色:“你这个血胎,能让我研究研究吗?我好像有了一股全新的灵感!”

%39
“当然,你尽管研究。不过要注意,血胎不能离开这个血池,一瞬间的时间都不行。”

%40
“嗯,我知道分寸,放心吧。”

%41
中洲,狐仙福地。

%42
看着眼前的洞地蛊,方源吐出一口浊气。

%43
这是狐仙福地的第二座洞地蛊。第一座沟通了仙鹤门,这一座则联络了灵缘斋那边。

%44
就在刚刚,和灵缘斋的第一笔交易达成。大量的气囊蛊装载着胆识蛊,交到灵缘斋那边。而方源则收获了一百八十块的仙元石,顿时解决了他的燃眉之急。

%45
说起来,方源和凤金煌之争,真正有价值的不是凤金煌提供的仙僵新生的法门,而是和灵缘斋的合作。

%46
凤金煌之所以拿出重获新生之法,当做赌资。就是看在方源已经是仙僵,难以利用这个法门。

%47
不过她并不知道,方源还有第一凡窍。想要坑方源,却没有完全坑住。

%48
有了这一百八十块仙元石,方源毫不犹豫地将其一半,都转化为青提仙元。如此一来,他就渡过了虚弱期,战力重新恢复过来。

%49
一天之后,他和仙鹤门再次交易,同样收获一笔不菲的仙元石。

%50
仙鹤门方面态度比较之前,温和很多。

%51
显然也顾及着:若是太过逼压方源得狠了,导致方源倒向灵缘斋一方的可能。

%52
同时,仙鹤门还想收购方源身上的成功道痕,开价很高,请方源出手炼制某种六转仙蛊。仙蛊方免费提供给方源阅览,仙蛊材料全都由仙鹤门负责。

%53
就算是中洲十大古派,对仙蛊的需求也必然是永无止境的。

%54
对这个要求,方源尽量推脱,也不说早已经用掉的实情,而只是回应要考虑一段时间。

%55
仙鹤门有求于方源,自然态度就更要缓和一些。

%56
方源做成这两笔生意,便立即用手头上的仙元石,在宝黄天中收购了一些毛民奴隶,补充炼蛊失败时的损耗。

%57
毛民奴隶的价格,可不便宜。

%58
方源手中的仙元石刚刚才有结余,如此一来,又见了底。

%59
好在方源手中的青提仙元比较充裕,比之前情形要安全许多。

%60
但就算买下这群毛民奴隶,方源也只是补充了之前的损耗,并不能做到扩充下一座石巢的规模。

%61
因为和灵缘斋的贸易,对胆识蛊的订单又增添了一大笔。

%62
方源动用两座石巢,已然不能满足如今的市场需求。他打算先积攒一笔仙元石,建立第四座石巢,并且配备齐全的毛民奴隶。

%63
这笔资金缺口并不小。方源的第三座石巢,还是黑楼兰提供的资金。

%64
不过有星象福地在暗中帮衬,万象星君的几大蛊虫贸易,也在陆续地帮助方源积攒仙元石。

%65
这使得方源积累资金的时间,大为缩减。

%66
方源一面积攒仙元石,一面利用智慧光晕,进行推算。

%67
狐仙福地时间的一个多月之后,方源成功地将变形仙蛊,完美地融合到见面似相识当中。并且凭借力道宗师级的境界,成功地推算出了五转的全力以赴蛊蛊方。

%68
和上古蛊方不同,方源推算出来的蛊方,尽量都取用了现在容易收购的炼蛊材料。

%69
全力以赴蛊是方源第二仙窍的本命蛊,一直以来都只是四转,转数大大落后于方源本身。

%70
有了这道蛊方之后,方源便有了另一项任务,那就是升炼全力以赴蛊。

%71
失败了一次之后,全力以赴蛊提升到五转级数。

%72
“若是能将全力以赴蛊提炼到六转,那对杀招万我的提升,将是巨大的,仅次于我力仙蛊。可惜这种事情,已经超出了我的能力范围。”

%73
有了这次成功的经验,方源又推算小家子气蛊的蛊方。

%74
结果惨败!

%75
方源的气道境界只是普通,连准大师都算不上。有此结果也十分正常。

\end{this_body}

