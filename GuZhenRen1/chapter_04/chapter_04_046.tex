\newsection{沙鸥死蛋}    %第四十六节:沙鸥死蛋

\begin{this_body}

%1
落日的余晖,洒在粥稀绿洲之上。

%2
往日里袅袅的炊烟,在今日却是被浓烟、火焰所取代。

%3
宁静安详荡然无存,房屋建筑塌毁倒塌,化为一片片的废墟。废墟中,街道上,湖畔,树下,血泊接连,横尸遍地。

%4
方源重新又回到家主阁,站在一堆碎瓦破砖上,他满意地将毒道仙蛊妇人心收回仙窍。

%5
屠杀了大半天,终将妇人心喂饱了。

%6
妇人心养炼合一,再继续喂下去,妇人心就进入炼蛊过程中。吃下的妇人心脏越多,妇人心的威力就越强。

%7
不过方源,暂时没有炼蛊的想法。

%8
以他一人之力,屠杀凡人比屠猪狗还要容易,但凡人那么多,终于还是浪费了大半天时间。

%9
况且,妇人心是一次性的消耗仙蛊。威力再大,用一次就没有了。投资前景并不大。除非方源手中有妇人心的仙蛊方。

%10
站在家主阁的废墟上,方源仙窍中数百只蛊虫一同催动起来,进行全方位的侦查。

%11
两个呼吸之后,方源就找到了一处地下入口。

%12
他走到入口附近,大脚一踏,就将中空的地面踏毁,露出一个入口。方源便一路强拆下去,进入地道数百步后,终于空间宽敞起来,足够方源弯着腰行走。

%13
须臾,一道巨大的石门。出现在方源的面前。

%14
门头上,刻有大字:“族库重地,闲人免进”。

%15
方源冷笑一声。直接轰破石门,立时数百道金色光刀,斩击在的方源身上。同时尖锐的警报声响起。这显然是兰家防御偷盗者的手段,但可惜的是,连方源发甲上的一根倒刺都没有损毁。

%16
方源进入石门,看到一大堆的元石,方源粗略估计一下。有十万多块。

%17
若是方源还是凡人,这些元石将是一笔巨大的财富,可惜方源如今已经成仙。真元无限,对元石的需求已经下降到最低点。

%18
不过,他还是将元石都收入仙窍,纯粹是顺手而为。

%19
不一会儿。他打破第二道石门。进入其中。

%20
他看见一大堆的炼蛊材料,分门别类地摆放着。都是寻常材料,虽然方源也有需要,但这些材料数量不多,方源真正要炼蛊什么的,需求量一定很大。仍旧需要蛊仙这等存在进行大批的收购。接下来的第三道、第四道、第五道石门内。都是炼蛊材料,只是种类不同。存放的要求也不同。方源将这些统统收如囊中。

%21
到了第六道石门,他终于发现了蛊虫。

%22
大量的凡蛊,存放在这里。这是兰家的蛊虫秘库,一族的底蕴所在。

%23
方源全都收走,以他丰富的记忆和老道的眼界,这些凡蛊都认得,皆是可有可无的货色。

%24
但对于凡人蛊师来讲,这些蛊虫中不乏珍稀蛊虫,也有五转蛊、四转蛊。得到其中一只,兴许整个生活都能得到改善,甚至人生轨迹都会发生转折。

%25
这就好像是方源在青茅山上得到酒虫;在白骨山得到骨肉相连蛊,在商家城得到全力以赴蛊一样。

%26
这样的蛊虫秘库,有三座。元石库藏,有五个。

%27
除此之外,还有库中库,隐秘小库,都没有逃出方源的搜索。

%28
一些秘库,里面已经狼藉一片,还有蛊师的尸体,打斗的痕迹。显然在方源屠杀地面生命的时候,一些利欲熏心的蛊师知道兰家完蛋了,便闯入这里,抢夺一切可以利用的资源。

%29
他们在这里发生火并,抢夺能够带走的宝物,还有一部分人则死于秘库的防御蛊虫。

%30
方源平静无波,走过这些秘库,顺着一条主要地道往地下深入。

%31
沿途中,蛊师的尸体越来越多,除此之外还有沙鸥的尸体。

%32
这些沙鸥,仿佛鸵鸟和海鸥的结合体,腿脚粗壮,肌肉发达,同时双翼宽大,可以利用气流翱翔天空。

%33
沙鸥可以在沙漠中急速奔跑,也可以在载人飞行,吃的食物也只是清水和草,非常容易喂养,性情也温和,是西漠蛊师最常用的代步之物。

%34
唯一美中不足的是,沙鸥生育力比较低。十颗沙鸥蛋,常常只有三四颗可以成功孵育出健全的沙鸥。

%35
不过,这个兰家豢养的沙鸥,数量很多,超过了同等势力。方源大杀四方时,杀了不少的沙鸥。他猜测,兰家手中也许掌握了某种孵育沙鸥的独到手段。

%36
如果真的有,那么这将是整个兰家中,方源唯一稍稍入眼的东西。

%37
方源又往下走了数千步,见到的蛊师尸体越来越多。终于在某处地段,发现了大量的蛊师尸体,以及许多沙鸥也死在这里。

%38
“看来进入这里的蛊师们,遭遇到了沙鸥的阻截,一场激斗后,最终全数丧命于此。他们冒着风险来到这里,很显然有重大的利益在诱惑着他们。”方源分析着。

%39
欧欧欧……

%40
又行了数百步后,方源遭遇到一波沙鸥的攻击。

%41
沙鸥有上百只,当中的沙鸥百兽王还带着新伤,应该在不久前和蛊师们交过手。

%42
区区兽群,怎么是方源的对手?几个呼吸之后,方源杀掉这些沙鸥,继续前行。

%43
地道明显往下延伸,深度让方源都有些意外。

%44
解决了十几波的沙鸥袭击,方源走出地道,立即视野开阔,来到了一处地底大洞窟。

%45
这洞窟之大,仿佛一座广场。

%46
洞窟中央,有一座巨大的石台。石台上竖着一只巨大的鸟蛋,足有房屋大小,通体沙黄色泽,粗糙黯淡。

%47
石台周围,铺着一层厚厚的黄沙。

%48
黄沙细软无比,微微温热,踩在脚下,如踩着棉花团一般。

%49
一只只的鸟蛋,就铺在黄沙中,微微陷进去。这些鸟蛋有的大,有的小,还有破碎的蛋壳。

%50
“这些鸟蛋,都是沙鸥蛋。这里就是兰家的沙鸥孵育之地了。”方源走进洞窟。

%51
为了保护家园和鸟蛋,大量的沙鸥暴动而起,向方源围剿过去。

%52
方源无动于衷,一挥手掌,先是洒出一片风刃,将沙鸥如割麦一般割倒大半,又洒出一抹冰霜,冻住另一半。

%53
幸存的沙鸥,数量稀少,再不复刚刚的气势汹汹,转而惊惶逃窜。

%54
方源也没有兴致去追杀个干净,距离石台越来越近,他的目光渐渐被石台上的巨蛋吸引,流露出越加喜悦的光。

%55
“这个蛋,难不成就是天地沙鸥的鸟蛋?”

%56
方源踏上石台,来到巨蛋面前。他伸手抚摸着粗糙的蛋壳,蛋壳上裂痕满布,同时还有数个小洞,不断地从裂缝间,从小洞内往外流出透明的蛋清。

%57
这些蛋清落在石台上,顺着石台表面刻着的沟渠,流通到周围的黄沙下面。

%58
这是上古荒兽天地沙鸥的一颗鸟蛋。

%59
鸟蛋中孕育的生命,早已经夭折。蛋清都往外泄露,很明显这是一颗死蛋。

%60
沙鸥孵育率低下,对于相同血脉的天地沙鸥,孵育率就更低了。但毕竟是上古荒兽的生命精华,只是如此粗陋肤浅地利用蛋清,就大大提高了沙鸥鸟蛋的孵育率。

%61
方源没有得到猜测中的孵育手段,但却得到了一只上古荒禽的死蛋。

%62
这是个意外的惊喜。

%63
“果然连运之后,就有际遇了吗?屠杀了一个中小型部族,就收获了对自己有帮助的东西。”方源心生感慨。

%64
他的连运仙蛊,需要上古荒兽天地沙鸥栖息地的沙土喂养。

%65
这沙土名为沙鸥土,是普通泥沙受到天地沙鸥气血感染,日积月累形成的一种特殊资源。

%66
方源原本要获取沙土,就得从别人手中收购,或者寻找到天地沙鸥的巢穴,冒着巨大风险偷偷潜入,收取沙鸥土。

%67
但现在他有了天地沙鸥的死蛋,完全可以利用这颗蛋中的生命精华,将普通的沙土转化为沙鸥土。

%68
“要喂饱连运仙蛊,需要万斤沙鸥土。这颗死蛋中,生命精华还很充足,应该够用。就算不够用,也能帮助我支撑过这段困难时期。连运仙蛊也不是一定要喂饱,半饱不饥的状态也能马马虎虎撑过去。毕竟接下来一段时间,没有合适的目标,连运仙蛊很可能闲置不用。”

%69
怀着喜悦的心情,方源将这颗死蛋收起。

%70
回到狐仙福地后,他立即着手布置。他前世乃是血道宗师,如今又是力道宗师,这两大流派都是跟生命、气血打交道的。

%71
他很快就设想出一个法子,利用六百多只凡蛊,组成蛊阵,抽取死蛋中的生命精华,形成庞大的气血。

%72
方源又在智慧蛊的帮助下,改良这个方法,将气血浓缩成嫩黄色的汁液。

%73
普通的沙石,在这汁液中浸泡个一天一夜,就能转变成沙鸥土。

%74
三天之后,方源得到充足的沙鸥土,将连运仙蛊喂饱。

%75
如此一来,他的手中就只剩下净魂仙蛊的喂养问题了。

%76
喂养净魂仙蛊,需要上万头白莲巨蚕蛊的血肉,然而白莲巨蚕蛊却是十分罕见。十万年前,近古时代,魂道荣昌,这种蛊虫曾经风靡一时。

%77
可惜到了现在,就算是宝黄天中已经很少见到白莲巨蚕蛊了。

%78
方源以八臂仙人的名义,在宝黄天中求购这种蛊虫以及蛊方,一直都没有得到他人的回应。

\end{this_body}

