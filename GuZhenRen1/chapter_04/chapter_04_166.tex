\newsection{时济运恺昼夜子}    %第一百六十六节:时济运恺昼夜子

\begin{this_body}

%1
方源是多么精明的人物,立即摇头:“这两件事,价值完全不等。对于你们而言,太白云生的江山如故必不可少。在我们看来,抵御地灾却可独立完成。而且,大雪山福地中时刻由雪胡老祖镇压着,你黑楼兰没有加入大雪山福地,只是个外人,可以说那里你已经呆不下去。你借助抵御地灾的由头,正好可以暂居我狐仙福地。我还没有收你房租,你倒好,还想凭此偿付太白云生出手的代价!”

%2
方源一下子便戳破黑楼兰的尴尬处境,黑楼兰脸皮极厚,哈哈一笑,仍旧和方源力争。

%3
双方言辞交锋,半晌后终于达成一致:黑楼兰这段时间,就留在方源这里,无偿贡献力气。这段时间的胆识蛊收益,都没有黑楼兰的份。同时,狐仙福地面临地灾,黑楼兰也得全力抽手,帮助方源一同抵御。

%4
这场交易完成,黎山仙子却仍旧不急着走,和方源谈到仙材。

%5
方源在拍卖大会中获取了无数仙材,黎山仙子想要收购,好应付雪胡老祖交代下来的资源收集的任务。

%6
黎山仙子得到方寸山,雪胡老祖只是借来一观,并没有贪墨她的。

%7
雪胡老祖战败两位八转联手后。大雪山福地地位立即水涨船高,在蛊仙界的声威更是达到前所未有的巅峰。

%8
雪胡老祖召见黎山仙子,以为她出手挡下药皇,护住方寸山为由,要求黎山仙子加强贡献。交代了黎山仙子许多收集仙材的任务。

%9
黎山仙子处境尴尬,自己新得了方寸山,虽然请得太白云生出手,恢复方寸山,剩下一大笔费用。但自身却仍受着重伤。

%10
原本她靠着山盟蛊,有许多积蓄,身家富有。但因为大力资助黑楼兰。导致积蓄迅速消耗。已呈现干涸之状。

%11
独自修仙,耗费资粮就是无数,已经是大不易。很多蛊仙没有经济支柱,入不敷出,只能为他人卖力,东奔西走。

%12
就算有经济支柱,买卖总会有赚有赔。类似山盟立誓、胆识蛊的买卖。都是稀少唯一,怎么可能普及?

%13
关键还有天灾地劫,一场下来,许多经营、努力都化作泡影。一场场的天劫地灾,蛊仙们就算撑过去,看到狼藉不堪的仙窍,通常都会有一段时间的心灰意冷。

%14
黎山仙子魄力很大,支持自己之外,还大力扶持黑楼兰。黑楼兰没有经济来源,两人修行的庞大需求。都压在她一人身上。就算是魄力再大,财力也终究开始捉襟见肘了。

%15
但黎山仙子除去仙元石、仙材这等实物,手头上还有不少其他东西。

%16
许多珍贵的信息,价值就很超群。

%17
黎山仙子人脉广泛,就是最优秀的情报收集者。

%18
方源手头上的仙材,本来是想参加炼蛊大会用的。但分量比较足,卖给黎山仙子一些。也无不可。

%19
方源点头同意,但要求黎山仙子手中的一个仙道杀招。这个仙道杀招,他早就觊觎良久,如今正是敲诈的好机会,方源自然要紧紧把握住。

%20
黎山仙子听到方源的要求,心道对方果然提了这个,一时间满嘴都是苦涩。

%21
但形势比人强,黎山仙子就算再不愿意,为之奈何呢?

%22
时济运。

%23
这便是方源从黎山仙子手中,索要得来的仙道杀招的名字。

%24
此仙道杀招,是以时运仙蛊为核心,掠夺他人寿元,获得自身的暂时的运道增幅。

%25
方源从黎山仙子处,得到时运仙蛊。曾想利用寸光阴蛊取代寿元,和时运仙蛊进行配合,可惜结果失败了。

%26
寸光阴和时运仙蛊,虽然可以搭配起来,但前者消耗实在太过猛烈。后来方源便想推算出仙道杀招,使得寸光阴蛊的消耗不那么剧烈。

%27
但又失败了。

%28
他虽然有智慧蛊的光晕可以利用,灵感无限,但脑海中念头太少,不够消耗。恶念蛊、忆念蛊一直在炼,却始终不够消耗,存货稀少。

%29
就算有足够的念头,方源的宙道境界也是不足。想得到成功的果实,希望分外渺茫。

%30
但方源之前就知道,黎山仙子手中有着这样一个仙道杀招,以时运仙蛊为核心。若运作得好,是一个不输给山盟蛊的经济支柱。

%31
于是这次方源趁着这个千载难逢的好机会,直接将这个仙道杀招索要了过来。

%32
得到手后,他立即查看一番,心中有些奇异,便问黎山仙子:“你这杀招是从哪里得来的?”

%33
“这就不方便透露了。”黎山仙子冷着脸,心情很不好。方源这是趁人之危,但偏偏这次她有求于人,只得妥协。

%34
“这杀招的构思让我感到熟悉,难道是北原散修昼夜子?此人在五域大战时,大放过光彩。他本身是位不问世事的隐居之士,结果北原和中洲蛊仙之间爆发大战,战斗波及了他的住处,使得他愤然出手,转瞬之间就将在场的三位蛊仙擒拿活捉。北原方面请他出山相助,他不肯,但碍于阵营,便给北原的蛊仙种下宙道杀招,用来护身保命。”

%35
那个杀招的名字很怪,而且很短,只有一个字。不久后就传遍五域,方源前世更是如雷贯耳,叫做恺。

%36
杀招效果是专门能拖延伤害,留出一段时间让蛊仙寻求方法救治。

%37
比方说某个蛊仙身上被种下“恺”,又受到了致命打击,但在当时伤害却不立即发作,而是延迟一段时间。才会爆发。

%38
如此一来,就给了蛊仙大量的机会,去寻求帮助。在这个期间,蛊仙可以逃窜,得到喘息之机。自己救下自己。也可以飞回战友身边,得到他人援助。就算是必死,也可以将自身的修行资源交给家族后辈,福地落在隐蔽地点,不白白资敌。

%39
北原蛊仙虽然战力不俗,但数量少于中洲方面。中洲又先发制人,布置暗间。先杀了一批北原蛊仙。导致北原方面一直被压入下风,节节溃败。

%40
但“恺”的出现,却令北原蛊仙伤亡骤降,很大程度上影响了战局。最终导致中洲、北原两方僵持,令中洲速攻北原,拿下此域的战略计划破灭。

%41
“若真是昼夜子,那黎山仙子能延迟违誓伤害的仙道杀招。恐怕就是恺了。还真不可小看黎山仙子的人脉关系。居然能请得这位散修出手,不知道又付出什么代价?”方源越发觉得这种可能真的不小。

%42
蛊仙当中,有很多隐藏足迹,不显行迹,不参战,不集会,独自隐居着修行。

%43
很多人即便获得许多惊人的成就,却直至死亡陨落,也不为世人所知。比如繁星洞天之主,在世时隐藏得很深。声名不显,直到繁星洞天显现,这才暴露出他的高深修为。

%44
这类蛊仙数量其实还不少。六转、七转,乃至八转都有。至少方源知道,北原的八转蛊仙要超过五位,只是隐藏着而已。

%45
方源旋即又想到紫山真君,这个神秘的蛊仙就算在五域乱战中。也没有一丝行迹暴露出来。

%46
他不禁心生感慨:“江河万万,天下豪雄强者何其多哉!”

%47
黎山仙子至始至终,都没有透露可能的昼夜子的消息。这场交易谈妥,不管付出多少代价,她终究是达成了原先的目的。

%48
黎山仙子带着一批仙材,回去了大雪山福地。在接下来的一段时间内,她重点是养伤,解决违背盟约的反噬,以及四处如何收集豢养小人族的方法秘诀,以期在最快的时间内获得收益。

%49
而黑楼兰,则留了下来。

%50
方源领着她一路疾飞,不一会儿,便来到目的地。

%51
两人从高空俯瞰,便见宽阔的大地上,青草葱茏,矗立着三个巨无霸似的黑灰建筑物。

%52
这便是狐仙福地中的三座石巢。

%53
石巢中有无数房间里,每个房间至少一位毛民,正在不断炼蛊,热火朝天,有条不紊。这是借助地球上工业流水线的思想,打造出来的大型炼蛊基地。

%54
将一只蛊虫的炼制步骤,划分为数十个,甚至上百个阶段,让一位位毛民只负责其中一个阶段的炼制。不仅能保密蛊方,而且还可以提高熟练度,进而提高炼蛊的成功率以及产量。

%55
在石巢底部,一只只的气囊蛊接连产出。

%56
当然,石巢中也时不时的发生爆响,爆炸或大或小,有的还形成火灾,毛民全身附着火焰,在火焰中哀嚎惨叫。

%57
但他的惨叫声,绝不会传出去,干扰其他毛民炼蛊。

%58
每个房间都有隔绝声音的布置,不仅是防止干扰,也是阻碍毛民们的交流,增强掌控力度,让他们只能乖乖炼蛊。

%59
炼蛊当然也有风险,甚至风险并不下于作战。

%60
不过一旦发生了类似爆炸的意外,房间中就会立即飞出蛊虫,喷水灭火,救治伤口等等。

%61
炼蛊失败,受到反噬伤害的毛民,能够保下一命,已经算是幸运的。

%62
有些失败的代价,是当场死亡。或者脑萎缩,精神崩溃,手足消残,再也炼制不了蛊虫。

%63
伤势轻的,方源还救治。若伤势重,成为植物人,治疗代价太大,方源就直接舍弃,给他们安乐死。

%64
(我的小说《蛊真人》将在官方微信平台上有更多新鲜内容哦,同时还有100\%抽奖大礼送给大家!现在就开启微信,点击右上方“+”号“添加朋友”,搜索公众号“qdread”并关注,速度抓紧啦!)

\end{this_body}

