\newsection{和宋家交涉}    %第二百五十九节:和宋家交涉

\begin{this_body}



%1
和之前方源第一次加入北原僵盟相比,方源这次加入东海僵盟,反而更宽松一些。

%2
原本,方源还有些担心身份暴露,毕竟东海僵盟中可有好几位八转仙僵坐镇。

%3
为此,方源在之前狐仙福地里时,就利用智慧光晕,不惜耗费大量星念,将星痕仙蛊也融入到见面似相识杀招当中。

%4
有着吃力、变形、星痕三大仙蛊,充当核心之后,方源转变成星道蛊仙或者星道荒兽,更加逼真了。

%5
但此次加入东海僵盟,方源并没有见到任何一位八转蛊仙,就算是对自己星象子的身份审查,也只是走了个过场。

%6
“这其中的原因,除了八转仙僵各有要务之外,鲨魔在东海僵盟的地位也不低,活人蛊仙并不能真正深入僵盟高层,最主要的恐怕还是东海僵盟势力强大,有着充足的底气和巨大的心理优势吧。”方源心底下暗暗分析。

%7
七天之后。

%8
蜂海。

%9
无数的蜂群,从海水中飞出来,再落到海面上的船花里。

%10
在东海的五月,蜂海海面都会随着波涛,流淌过来无数的船花。

%11
这些花朵五颜六色,大如小船,交织紧密,宛若一幅巨大的缤纷画卷,从花海向四周蔓延席卷。

%12
蜂海,就是一处和花海接壤的海域。

%13
蜂海中生存着大量的水蜂,这种特殊的蜂群,只有短暂的飞行能力,更擅长的是在海水中穿梭游动。

%14
每年的五月,当数不清的船花顺着海流。蔓延到蜂海海域的时候,就是水蜂群一年仅有一次的采蜜良机。

%15
就在这海面上空。方源、鲨魔等人摆开阵势,和宋家的蛊仙进行了一场言辞激烈的交涉。

%16
最终。耗费两柱香的时间,双方终于谈妥,达成了协议。

%17
宋亦诗等人,将不再追究星象子的过错。星象子虽然是无心之失,但毕竟犯错,不仅要公开道歉,而且还要一次性赔偿宋亦诗六百块仙元石。星象子暂无能力一次偿还,可最多分六次还清,第一次赔偿一百块仙元石。第二次须赔付一百二十块仙元石。第三次一百四十块仙元石……

%18
按照协议的内容,若是分批赔偿,方源无疑要亏损很多。

%19
所以,协议达成之后,方源向太白云生、鲨魔等人各借了一笔仙元石,凑成六百块,当场交给了宋亦诗。

%20
宋亦诗铁青着脸,接过这些仙元石,恶狠狠地瞪着方源。语气生硬冰冷地道:“道歉!”

%21
方源是何等能屈能伸之人,满脸苦笑,诚恳地道歉:“在下无心冒犯了亦诗仙子,唐突佳人。罪该万死,还请见谅见谅!”

%22
“那你怎么不去死?”宋亦诗呛声道。

%23
方源又苦笑:“若是窥见亦诗仙子半点容颜,在下活该千刀万剐。但仙子你也明白。当时温泉中雾气浓郁,您又施展了隔绝内外的仙道杀招。我怎么可能看得清呢?”

%24
宋亦诗冷哼一声,当时的情况她记得清清楚楚。在海底火山她根本毫无戒备之心,哪里布置了什么仙道杀招。就算是有些薄雾,宋亦诗也是提前将其清除掉的,她不喜欢视线被遮挡。

%25
但方源这番胡诌,宋亦诗却没有反驳。

%26
就算眼前的这个“猥琐至极”的老头子,真的看光了她,宋亦诗也得矢口否认,绝不承认。

%27
宋家的两位蛊仙,则神情稍缓,甚至暗地里都有些同情眼前的星象子了。方源之前解释的很清楚,事实上,这件事情闹出来后,宋家也早就将李家村那边探寻清楚了。

%28
这道海底潜流,也是最近几年,才悄然形成的。再加上宋亦诗也对海底火山,没有仔细检查,才造成的疏漏。

%29
星象子只看了一眼,并且因为朦胧雾气,什么都没有看到,就得赔偿六百块仙元石,这代价未免也太大了点。

%30
说到底,还是宋家乃是超级势力,家大业大。星象子若是冒犯了别人,也无所谓,关键居然犯到了宋亦诗的手上,宋亦诗不足为意,但她的背后却是宋家的八转太上大长老宋启元!

%31
若非鲨魔代表东海僵盟主持这场交涉,若非东海僵盟势力比宋家还要更大一些,方源要摆平这件事情,就绝非付出六百块仙元石这么简单了。

%32
“星象子,这件事情虽然完了,但你最好别再出现在我的面前!”宋亦诗抛下这句狠话后,毅然飞走。

%33
倒是宋家的其他两位蛊仙,态度缓和下来,不失礼数地拱手道:“鲨魔大人、苏白曼大人,告辞了。”

%34
“三位慢走。”鲨魔还礼道。

%35
至此,这场风波终于平息。

%36
鲨魔再次邀请方源、太白云生,前往鲨海做客。

%37
但方源婉拒,他告诉鲨魔,想看看有东海僵盟中什么报酬丰富的任务,可以发挥自己的长处,赚取仙元石还债。

%38
鲨魔察觉到方源的“失落”,安慰他几句,说僵盟结构松散,并不会太束缚自由。

%39
方源演技十分到位,真的让鲨魔认为是一位散修,最注重人身自由。

%40
既然方源拒绝,鲨魔、苏白曼也没有再强求。这对夫妇也需要时间,去好好准备下一次的玉露福地的攻略之法。

%41
和鲨魔等人分别,方源真的接取一些东海僵盟的推算任务。他生性谨慎,做戏也要做全套,绝不会在这些细枝末节上留下破绽。

%42
方源有任务令牌,还有用来通讯的信道凡蛊,接取任务方便得很,并不需要亲自前往不死国福地。

%43
事实上,方源也不打算再亲自进入不死国福地了。毕竟那里有八转仙僵坐镇,在这种存在面前,方源对自己手上的这套见面似相识杀招。也是信心不足。

%44
“虽然耗费了六百块仙元石,但只要能顺利进入北原僵盟。也是值了。接下来就是耐心等待东海僵盟那边,出现一些和北原相关的任务。比如北原特有资源的收集任务。或者视察北原僵盟分部的任务。”方源暗自谋划。

%45
方源的计划进行的十分顺利,到了这一步,他需要的就是等待了。

%46
接下来,方源利用定仙游,秘密回到狐仙福地。太白云生和他分道扬镳,利用江山如故开道,在东海蛊仙界四处交好。

%47
利用智慧光晕,方源只花了很短的时间,就将手头上东海僵盟的推算任务。统统完成。

%48
若换做以前,这些推算任务,非得耗费方源个把月的时间,但现在方源只是用了几天功夫。

%49
方源渐渐已经感觉到,自己在智道方面的突飞猛进。

%50
前世五百年的蹉跎、折磨、辉煌、跌宕,将他打造成一位擅长谋划、算计的人,因此方源的性情对智道的修行十分契合。

%51
再加上极为优异的智道传承,还有九转智慧蛊,方源在智道上能不突飞猛进吗?

%52
完成这些任务之后。方源并未立即交付上去。这种速度未免太过惊人,和他表现出来的智道造诣十分不符。

%53
至于欠下太白云生、鲨魔二位的仙元石,方源也未放在心上。

%54
太白云生的欠债,方源甚至都可以不还。鲨魔的欠债。方源再过几天就有能力轻松偿还。

%55
胆识蛊供不应求,依旧火爆,其余生意也是正常进行。

%56
虽然因为之前搬迁了星象福地。导致星象福地中的四大资源,多有受损。但方源每个月的盈利,仍旧接近一千仙元石!

%57
不过尽管如此。方源仍旧缺钱。

%58
不管是星象福地的大规模改建,还是星念仙蛊的炼制计划,都需要资金,大量的资金!

%59
除此之外,还有仙蛊的喂养,方源手头上大量的仙蛊需要喂养。若是真的让仙蛊饿死,方源就损失太大了。

%60
寻求寿蛊、调教方正、审问俘虏、建设第四座方源石巢、改进仙道杀招、为下一次攻略玉露福地进修智道手段、收购蛊虫、收购力道荒兽荒植、关于毛民的豢养心得经验、春梦果树的探查……这些事情,简直千头万绪,都需要方源处理。

%61
尽管仙僵身躯不眠不休,方源仍旧感觉时间很不够用。

%62
方源对自己的处境,十分清楚。说不定哪天,他弄倒八十八角真阳楼的事情,就被探查出来,彻底暴露。

%63
所以,他必须趁着最后的时机,尽可能的发展壮大,消除自身的弊端缺点。

%64
方源恨不得将时间掰开,甚至将自己劈成两半来用。埋头忙碌的间隙,方源时刻关注着外部局势。

%65
西漠、南疆方源关注较少。

%66
中洲方面,还处在炼蛊大会的余波影响当中。

%67
北原方面,在雪胡老祖战胜药皇、百足天君之后,北原局势显得十分平静。但方源历经琅琊攻防战,也知道残阳老君这群中洲的调查者,所以北原局势越平静,方源越感到里面暗藏的漩涡激流。

%68
东海方面,目前最引人瞩目的,便是宋家等超级势力争夺登天野的霸权。

%69
几天之后。

%70
西漠。

%71
方源化作另外容貌,和萧家蛊仙进行贸易交接。

%72
“这里是这次的货,你们查查。”方源将带来的长恨蛛群抛出来。

%73
这次萧家蛊仙来的是萧十让,此人性情沉稳,深有谋略,和萧虎痴合称为萧家文武双壁。

%74
萧十让检查了一遍之后,满意地点点头:“都没有问题。接着!”

%75
他抛给方源一笔仙元石。

%76
但和往常不同,除了仙元石之外,还有一块令牌,一只信道凡蛊。

%77
这是萧家的交易令!(,!

%78
ps:前段时间实在太过劳累,现在总算缓过了一口气,明天开始正常两更,还有补更的事情。有谁告诉我,这个月欠大家多少更?

\end{this_body}

