\newsection{血幕挡却一干仙}    %第一百四十节:血幕挡却一干仙

\begin{this_body}

%1
“什么?”东方万休瞪大双眼。

%2
“这绝不可能!”东方一空断然否决。

%3
“怎么可能是东方长凡大人?大家小心,这极可能是幻象!是对方破坏不了大阵,要让我们自乱阵脚!!”

%4
惊变之下,东方一族的蛊仙们,都是难以置信,不接受这样的事实。

%5
残阳老君不禁嗤笑:“让你们自乱阵脚?好笑得很,你们都一个个动弹不得,连自乱阵脚都资格都没有了。你们已经成为砧板上的鱼肉,早就身不由己,可笑你们还看不清这事实。呵呵呵。”

%6
这番话像是惊天霹雳,狠狠地劈在东方一族的蛊仙心中。

%7
众人纷纷望向东方长凡的星意。

%8
东方万休大吼:“长凡大人,这一定不是真的!”

%9
东方一空质问:“长凡大人,你倒是说句话啊!”

%10
“长凡大人,东方长凡!你究竟意欲何为?!”

%11
无数的质询声,嘶吼声,回荡在这座大殿当中。

%12
星意轻笑着,好似云淡风轻。他只是一股意志,但凝聚成生前本体的形象,身躯挺拔,面冠如玉,目蕴神光,大有本体生前运筹帷幄的卓越风姿。

%13
但此时此刻,此情此景,东方一族的蛊仙们看到星意这般的风姿神态,却叫众人心中发凉发冷。

%14
虚化大阵再起变化,先前凝聚成一团的庞大威能,终于在此刻,猛地喷发出去。

%15
好似一阵狂浪巅涛。血光四射间,形成一道厚实的血光大幕,将墟蝠尸体的主要位置,都覆盖进来。

%16
看到这一幕,东方万休等人终于脸色狂变。

%17
原本估计,是用来打击外敌的力量,真实用途却是防御。

%18
冰冷的事实摆在众人的面前,戳破了星意之前的谎言。

%19
“东方长凡,你为什么这么做?为什么要陷害我们!?”东方万休脸色狰狞,充满了痛楚。悲愤咆哮。

%20
“还能为了什么?生命才是一切的基础。谁想死?谁不想活?只要有活的一丝机会,谁能抵挡得了这样的诱惑?”残阳老君冷笑连连,双眼则紧盯着虚化大阵,一眨不眨。

%21
看着虚化大阵运转正常。残阳老君不禁贪婪地深吸一口气:“这夺舍法门。未免代价太过高昂。难道每次催动。都需要牺牲蛊仙吗?”

%22
若真是如此,这夺舍之法就推广不开了,开启的条件实在太高。

%23
东方长凡的星意缓缓摇头:“老君你无须担忧。当年我在八十八角真阳楼中。得到夺舍之法。里面内容可以夺舍凡人,也夺舍蛊仙。但若夺舍蛊仙,就必须修为超过被夺舍的目标。而且,夺舍之后,仙窍便是目标的仙窍。这等若鸠占鹊巢,大有弊端。我生前乃是智道蛊仙,但却没有合适的智道蛊仙作为目标。因而这么多年来,我竭尽心思推演谋划,最终得到这处大阵。”

%24
“这个大阵,是基于夺舍法门研发出来。可以让我夺舍凡人的同时,还能抽调出同一血脉的蛊仙本源,使得我夺舍之后,就能立地成仙,生成全新的智道仙窍,免去了夺舍凡人后,修为衰弱的危险期。”

%25
听得星意解释,残阳老君不仅转过目光,打量起一旁的星意,暗自心惊:“这东方长凡隐藏得很深,不仅智道堪称北原当代第一,竟然还能利用宇道道痕,驱使虚兽大军。现在看来,他的血道造诣也不低!”

%26
星意不知道残阳老君的内心活动,仍旧自顾自地说着:“你可别小看这座大阵。有这八位蛊仙牺牲自己,来供给给我,可谓资源充足!我新生成的仙窍,绝对会是上等中的上等。同时还有一个妙处,便是仙窍福地中自行产生花鸟鱼虫等等资源。仿佛是苦心经营了很久,这可就省去我几百年的苦功。因而我夺舍之后,不仅修为全复,而且资源充足,更胜生前!残阳老君,你可好好观看,尽管随时提问。我每进行一步骤,便将内容精髓都说给你听。”

%27
残阳老君的脸上,迅速闪过一抹灼热。

%28
他哈哈一笑,深深地看向东方长凡的星意:“你真是好算计,想让我为你保驾护航。我驻足于此,就要为你抵挡外界的那些魔道蛊仙。而且你立地成仙,必有天劫地灾,到时候我就在你身边,不想出手也得出手了。”

%29
东方长凡的星意也是一笑,丝毫没有图谋被戳破的尴尬。他目视残阳老君,声音低沉地道:“这世间之事,要想得到,就得付出。哪有不劳而获的东西呢?尤其是得到的越多,风险就往往越大。老君何意,可迅速定夺了。”

%30
残阳老君嘿了一声,心中不禁暗暗佩服东方长凡,想这此人,真是能够取舍。

%31
当初东方长凡得到夺舍法门,也在八十八角真阳楼的真传秘境当汇总,察觉到中洲十大派的布置。

%32
其时,仙鹤门正往东方部族中渗透。东方长凡升仙后,成为部族高层,暗中发现了这个真相。

%33
但他却没揭露,反而顺着这条线,暗中和仙鹤门展开合作。

%34
他用夺舍之法,和仙鹤门交易。

%35
寿蛊难寻,但凡蛊仙都会对任何的延寿法门趋之若鹜。更何况这夺舍之法,来源于巨阳仙尊之手呢!

%36
仙鹤门纵为中洲十大古派之一,历史悠久,底蕴深厚,也难以挡住夺舍法门的诱惑。

%37
双方各有需求,东方长凡一步步放出法门内容,也因此获得了仙鹤门中输送的资源。

%38
他靠着这些资源,迅速攀升,屡屡渡劫成功,更提携晚辈,四下谋算,将衰落不堪的东方一族重振辉煌!

%39
和东方长凡这样的人合作,残阳老君虽然战力强大,但心中从未放松,一直保留着十分的警惕。

%40
他没有思考多久,点点头,应承下来。

%41
东方长凡的星意大笑一声:“好,我先给你讲解这第一步……”

%42
残阳老君连忙竖起双耳,全身倾听。

%43
虽然明知道是东方长凡在利用自己,但残阳老君是带着仙鹤门的意志前来,也算是一种身不由己。

%44
再想想这个任务完成后的门派丰厚奖励,残阳老君本身也并非不愿的。

%45
东方一族的八位蛊仙,早已经破口大骂。

%46
东方余亮则昏迷过去,双目紧闭,一动不动。

%47
而在外界,方源看着下方,视野中又是另一种景象。

%48
只见一道血幕宽大无比,将山峰一样的墟蝠尸体牢牢遮盖。血幕厚实无比,血光充盈,蕴藏的磅礴威能叫人望而心悸。

%49
魔道蛊仙们纷纷转移注意力,不再相互内斗纠缠。

%50
其中一些人稍试身手,打击血幕,结果就像是小石头投入大湖,血幕表面只是泛起微微的涟漪,岿然不动,稳固如山。

%51
“好强的防御!”

%52
“这究竟是什么蛊阵?我不仅没有见过,甚至连听说都没有。”

%53
“东方长凡留下的后手,还真是层出不穷”

%54
一干魔道蛊仙纷纷猜测。

%55
方源也皱起眉头,他即便是血道大能,又有前世记忆经验,但这个蛊阵他也没有见过。

%56
“管他是什么蛊阵,让我一锤子敲碎它!”蛊仙卓战大吼着,杀下去。

%57
他骑着一头荒兽飞猪,身罩铠甲,手拿大锤,虎背熊腰,心焦气躁,却又欺软怕硬。

%58
众人目光纷纷集中在他的身上。

%59
卓战大出风头,心中得意,脸色更加凶狠。

%60
但几个呼吸之后,他凶狠的神色荡然无存,已被惊骇震恐所取代。

%61
原来他接近血幕,体内的血液就不受控制,不断地向体外喷涌。

%62
蛊仙号称为仙,但大部分身体,也是保留着人类正常的运转规律。血液是支撑身躯活动,必不可少的关键之物。

%63
卓战距离血幕还有老远的距离,身上的鲜血就喷洒出一大半了。

%64
若他继续冲下去,还未杀得上血幕,血液就已经喷光,自己便成为活生生地抽成干尸一具了。

%65
“好个奇诡的大阵!”

%66
“这是经血蛊的效果,能够让人放血。”

%67
“经血蛊是血海老祖的血道真传之一,没想到东方长凡获得的传闻不但是真的,而且他还将这经血蛊,提升到了仙级!”

%68
魔道蛊仙们纷纷动容。

%69
陆续有几位蛊仙出手,尝试了各种手段,接连狼狈败下。

%70
众人这才意识到,除非有正好针对的防御方法,否则绝难冲破这道血幕。

%71
方源紧紧皱起眉头。他自己纵然前世是位血道强者,但手头上却无任何血道仙蛊,他走的力道。巧妇难为无米之炊,他对这个大阵也是无法可想。

%72
局势棘手,众人心中焦急。东方余亮进去了这么久,是不是已经快要彻底获得传承了?

%73
“如今之计,只有动用远战杀招了!诸位一起出手,先坏了东方长凡的布置,在相互角逐。否则众仙眼下,却被一个凡人夺得传承。这事情要真的发生,我自在书生都没有脸面在北原混下去了!”自在书生陡然大吼。

%74
刚刚吼完,他便出手。

%75
魔道蛊仙们纷纷应和。

%76
一时间,水火相交,电轮飚射,璀璨绚烂,无数狂轰滥炸打的血幕层层削减。

%77
众人喜色刚刚浮现脸庞,却又很快凝重起来。

%78
血幕不断削减,又不断生成。尤其是越到后面,魔道蛊仙们的合力攻击,越显得乏力。

\end{this_body}

