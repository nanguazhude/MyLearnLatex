\newsection{墨人王}    %第十三节:墨人王

\begin{this_body}

琅琊地灵心疼地看着自己的泥沼蟹。。。

这头泥沼蟹,瘫倒在地上,爬不起身来。九对螯足被卸了七七八八,甚至就连最大的钢钳螯足,都被打碎一只。

这样的战果,让墨人蛊仙,以及太白云生都心底发凉。

方源打出上万力道虚影,所剩三千有余。他好整以暇,将剩余的虚影收入仙窍。

这虚影只能存一段时间,过了时限,拳气就会消散,虚影便随之消失。但能利用方源便利用,不会有一丝浪费。

他自从北原回归狐仙福地,两袖清风,一块仙元石都没有,因此就再也没有补充过大量的凡蛊。

他刚开始交战时已经试探出:单凭八臂仙僵之躯,难以对一身硬甲的泥沼蟹造成什么威胁。

因此,唯一的选择就是杀招万我。

这杀招的核心,乃是方源手中的一只魂道仙蛊,因此需要消耗青提仙元。

之前情况不明,方源毅然舍弃一颗青提仙元,掌控局面。

杀招万我,不愧是奴道合流,威力果然非同凡响。曾经带给方源巨大麻烦,令方源费尽浑身力气才堪堪应付的泥沼蟹,在万我之下,打得连头都抬不起来,一直处于下风,就没有翻身的希望。

“臭小子,跑到我家不说,还打伤了我的荒兽。你该拿什么赔偿我?!”琅琊地灵对方源吹胡子瞪眼,怒气冲冲。

太白云生则站在方源身后。饶有兴趣地看着这个地灵。

琅琊地灵身形瘦长,白发如雪,胡须垂胸。面目似婴儿般红润,身穿一身宽大的衣袍,两袖飘飘浮动。若非此刻怒气勃发,对方源瞪圆了双眼,否则会卖相无疑更佳,更为仙风道骨。

方源早已经熟悉琅琊地灵,他盯着琅琊地灵身上的五花大绑。皱了皱眉头:“你这是中了封禁,难怪我在宝黄天一直等不到你的神念。按照北原时间,我们也不过大半年未见。你怎么搞成这种样子?”

琅琊地灵眼睛瞪得更大了,立即反唇相讥道:“你又怎么搞成这种样子?人不人,鬼不鬼的!晋升成仙,又堕落成僵。嘿嘿。你应该寿元很充裕的啊。”

“哼,琅琊地灵,你活了这么长年岁,连最简单的取舍之道都不懂么?有舍才有得,我若不转化成僵,又怎么能再次出现在你的面前,又如何掌握如此强大的战力呢?倒是你,越活越回去了。现在都不能炼蛊了吧,”真是给你的本体丢脸呐。方源冷笑。故意激将地灵道。

琅琊地灵被说到痛处,气得当场跳脚。

不久前,琅琊福地遭受神秘势力的进攻,琅琊地灵虽然击退了来犯强敌,但本身却遭受了气道杀招封印。

因此,他才请来他的至交好友,来未他解除封印。

琅琊地灵对方源破口大骂:“臭小子,你还好意思说我?真阳楼倒塌,一定是你做的吧!嘿嘿,那么多人死了,两大超级势力的族人都丧命,整个北原的蛊仙都在找你这个罪魁祸首。你现在已经成了人人喊打的过街老鼠,日子很不好过吧!”

墨人蛊仙、太白云生齐齐变色。

墨人蛊仙瞳孔微缩,心中升腾起一股强烈的冲动,要去捂住地灵的嘴巴。

但地灵已经脱口而出,一切都晚了。

“这种事情,怎么能说出口?不怕被眼前二人杀人灭口吗?!”墨人蛊仙不由地心惊胆战起来,对面二人居然就是搅得北原天翻地覆,连巨阳仙尊布置都能破坏的凶犯!太危险了!这个局面要糟!

下一刻,方源和太白云生二人,皆目光冰冷地看向墨人蛊仙。

绕是墨人蛊仙平素时位高权重,实力不凡,此刻被这两大凶人盯住,也感觉心中发凉。

“琅琊地灵,怎么不给我们介绍一下这位呢?”方源呵呵轻笑两声,他声音沙哑难听至极,让人听着很不舒服。

不想琅琊地灵回答,谁知道心直口快的地灵会说出什么话来?

因此,墨人蛊仙硬着头皮抢先站出来,道:“鄙人墨坦桑,身居墨人城,乃是北原墨人之王。”

太白云生眉头一挑,没想到眼前这位颇有来头,立即对墨人王刮目相看起来。

当今五域,皆是人族的天下。异人在夹缝中生存,很多都被当做奴隶蓄养买卖,生活相当困难。

但北原中,墨人却是异人当中处境最好的一支。

很多异人居无定所,只能流浪落魄。而墨人却在北原建起了城池,拥有墨人蛊仙三位。

眼前的墨坦桑,便是墨人城主。在他的领导之下,墨人能抵挡住各反面的压力,抵御无数蛊仙贪婪的目光,维持着墨人的生存,这很不容易,足见墨人王的才华和手段。

“墨人王墨坦桑……”方源则在心中嘀咕一声。

这个名字,他有印象。

前世五域乱战,墨人王趁着人族内斗,无暇顾及他时,抓住机遇,积极发展,将墨人势力大大扩张。

待北原人族势力想要打压他时,他居然不顾王者威仪,主动投靠了刘家。对刘家太上大长老行奴仆之礼,以奴仆身份自居。

刘家乃超级势力之一,因此力保墨人势力。墨人势力在这一层保护伞之下,稳步发展。

而后刘家衰败,墨人王立即舍弃刘家,和马鸿运平等合作。

方源自爆之前,墨人已经有用城池数百座,占据北原三分之一的江山。

从此便可看出,墨人王墨坦桑乃是一位十足的雄主。不仅眼光独到,果断敢行,而且能屈能伸,不可小视。

想到这里。方源轻赞一声:“原来是墨人之王,果然有一股威仪,超出寻常。”

“岂敢。岂敢。”墨人王连忙谦虚。

眼前之人,可是破坏八十八角真阳楼的主犯,这样的危险人物,他心中警惕至极。

他主动解释道:“我和琅琊地灵是多年的至交好友。事实上,墨人城一直都和琅琊福地,关系紧密。我们墨人城最擅气道,恰好琅琊地灵中了气道封印。因此赶来帮忙。阁下是行于天下的龙蟒。搅动风云,动乱了整个北原。如此风采,叫墨某不得不佩服。我们墨人城。一直受到黄金部族的联合打压。昔年,巨阳仙尊更是过分要求墨人城,进贡了无数墨人女子。这样说起来,你们破坏了王庭福地。也算是帮助我们墨人报了仇。你们又是地灵的朋友。那么就是我墨坦桑的朋友。若是今后有计划的话,还请去墨人城做客。”

不愧是墨人之主,口才了得,一番话就将自己示好的意图完全展现,尤其还不卑不亢,十分难得。

“谁跟这个臭小子是朋友?”琅琊地灵不满地叫出声来。

不过,刚刚听到方源赞美他的朋友,他心底高兴。之前的怒气倒是削弱下去了。

方源对墨人王点点头,饱含深意地回道:“有机会。我一定去墨人城见识一番。”

说完,他又看向琅琊地灵,不买地灵的账,继续激将道:“地灵,我就算不是你的朋友,你也得欢迎我,为我炼蛊。你忘了,我还有一次叫你出手炼蛊的机会!我要让你炼蛊,你不炼也得炼!”

琅琊地灵毫无城府,一点就着,怒气再度升腾起来。

平时,也有蛊仙请他炼蛊。都是态度客客气气,甚至讨好谄媚。

他什么时候,受到方源这样的气来?

但偏偏方源说的是事实,他尚有最后一次炼蛊机会,琅琊地灵乃是长毛老祖执念所化。本体当年答应下来的约定,他必须得遵守。

“小贼可恶,气煞我也,气煞我也!”琅琊地灵气得脸红脖子粗,哇哇大叫。

忽然,他又喜笑颜开起来:“啊哈哈哈,我被困了,这气道封印太麻烦,足有十七八层,层层封印。刚刚墨人王不过解开了第一层而已!哈哈哈,我现在真的不能帮你炼蛊了,我出不来手,真棒!”

被封印以来,他无聊死了,不能炼蛊,最大的爱好被剥夺了。

但这一次,他反倒觉得自己这样子很开心。

都是方源气得。

看着这奇葩的琅琊地灵又叫又笑,老顽童一般活宝的样子,太白云生觉得此行不虚,算是大开眼界了。

墨人王沉默不语。他家大业大,不愿轻易得罪方源。

方源咳嗽一声,肃容道:“好了,说正经事罢。既然你不能炼蛊,那就算了。我这次来,还有一件事情,我要和你做一笔交易。”

“交易,什么交易?”琅琊地灵问道。

墨人王心中立即警惕起来,琅琊地灵虽然智商颇高,但性情率真,难保不被人坑骗。自己身为地灵的好友,若是方源真的图谋不轨,他就要站出来为地灵侦破对方的诡计。

“交易内容很简单,你应该还记得我是智道蛊师的事情吧?我愿意为你推算仙蛊方,而你支出仙元石作为报酬。这是一场双赢的交易啊。”方源道。

“推算蛊方?”琅琊地灵瞪大双眼,愣了愣后,再度大笑起来。

他笑得前仰后合,若是双手不被绑着,估计早就笑得拍大腿了。

琅琊地灵嘲笑道:“方源啊方源,你现在都成了僵尸,居然还想推算蛊方?还是仙蛊方?我老人家劝你趁早绝了这个念想!”

墨人王则道:“若是仙蛊方,即便是残方,也是价值极大。交给阁下推算,若是阁下推算不出,那么仙蛊方的内容岂不是也被阁下所知吗?”

琅琊地灵得到提醒,顿时炸毛喊道:“好小子,你居然想骗我老人家的仙蛊方!”

方源早有准备,此时哈哈一笑,伸出手来,怪爪一摊,露出一只仙蛊虫:“你们看,这是什么?”

\end{this_body}

