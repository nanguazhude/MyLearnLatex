\newsection{九龙护棺运}    %第三十二节 九龙护棺运

\begin{this_body}

繁星如钻石,点缀夜空。

江风徐徐吹来,偶尔传来的兽吼鸟鸣,更显得周围越加静谧。

方源耐心等候,但这一夜却没有等到叶凡的出现。

随着时间推移,第二天,第三天,叶凡始终没有出现。

“难道《叶凡传》记载有误?”方源忍不住思考,但很快他又自己否定了这个猜想,“不,江边的蛊师遗藏还在,叶凡并没有取走。再等等看……”

方源并不知道,就在千里之外,叶凡遭遇到了麻烦。

呼呼呼……

叶凡喘着粗气,心惊肉跳地看着洞口处趴着的巨大兽影。

“该死的,怎么会这么倒霉?刚刚被家族驱逐,躲到一个山洞里借宿,结果醒来的时候洞口就被封了!”叶凡心里咒骂,紧张而又无奈。

这山洞只有一个出口,偏偏被一头猛兽堵住了唯一的出口。

叶凡没有挖地钻洞的手段,他实力还很低微,偏偏这头猛兽体格庞大,竟是一头兽皇!

“这头兽中之皇,为什么独身一人,身边没有兽群护卫?难道说,它是一头老皇,被兽群中的新皇驱逐出去的?”叶凡一面紧张地注视着眼前的猛兽,一面心中迅速分析。

随着他持续观察,他很快发现这头兽皇外强中干。

这头犬形兽皇,浑身伤痕累累,体格不大,趴在地上,眼皮耸搭着有气无力。

它雪白的皮毛上,间或长有斑斓的花纹,好似雪地中的粉色花瓣;

“嘤嘤嘤……”叶凡仔细倾听,便听出犬皇正发出轻轻的呜咽声。声音分外孱弱。

再看它干瘪的,凸显出肋骨痕迹的肚皮,叶凡终于明白过来:“这虽然是一头兽皇,但它饿极了,似乎已经毫无战斗力了。”

得到这个结论。叶凡吐出一口浊气,放松的同时,心中又不免怜悯。

同是天涯沦落人,眼前的兽皇和他的处境是如此的相似。

叶凡小心翼翼地靠近犬形兽皇,兽皇没有任何反应,任由他接近。

叶凡大气都不敢喘。蹲在兽皇的身旁,缓缓探出手来,轻轻地搭在兽皇的额头。

蓬松的毛皮,让叶凡感受到柔软和舒适,同时还有湿漉和滚烫。

这头兽皇发烧了。浑身冒汗,打湿了皮毛,身体机能降至谷底。

“兽皇啊,兽皇,你快要死了。说不定哪天,我也和你一样。不过碰到我,也算你的运气。谁叫我曾经是寨子中最有名的兽医呢?”叶凡喃喃自语,心中同情心大起。开始治疗这头犬兽皇。

他催动蛊虫,治疗兽皇的伤势,缓解它的病情。然后。又分出宝贵的食物和水,一点点地喂给兽皇,让它慢慢地恢复体能。

叶凡再不急着赶路,这样在这个山洞连续待了七天,这头犬形兽皇终于好了起来。

它虽然还比较虚弱,身上一只野蛊都没有。连一头千兽王都打不过。但好歹已经脱离了危险,不再发热。并且能够自由奔跑,跟在叶凡的脚边撒欢。

叶凡救下了它。它便把叶凡当做了最亲近的人。每一次叶凡回来山洞,带回食物和水,它便主动迎接过去,绕着叶凡的双腿,欢快摇晃着尾巴。

到了后来,它恢复了一些战力,便自发跟随叶凡狩猎,帮助叶凡获取食物。

一人一狗,很快建立起了深厚的友谊。

当叶凡决定离开山洞,向远方跋涉时,犬形兽皇也选择了跟随。

“已经是第八天了,为什么叶凡还没有出现?”碧龙江畔,方源已经等得心焦。

还有五六天,就是黑楼兰渡劫之期。

届时,方源要在她身边护卫,不仅要帮助她分担天灾地劫,而且有可能还要面对七转蛊仙黑城和六转雪松子的联合进攻。

要估算一个蛊仙的战力,要考虑的因素有很多,但主要因素只有四个――仙元、杀招、仙蛊以及蛊仙个人的战斗造诣。

北原之行时,刚刚晋升蛊仙的太白云生,战力属于六转垫底。

太白云生手中有仙元,也有治疗仙蛊,但没有杀招,凡蛊也不充足。最关键的是个人战斗造诣很差;

因此,被大力真武体的黑楼兰压着打。

现在的方源仙元稀少,仙蛊虽多,但几乎都等待喂养,难堪催用。但他拥有杀招冰钻星尘、轻虚蝠翼以及发甲。正常发挥时,战力达到六转上等的层次。如果催动仙道杀招万我,则战力立即飙升到六转巅峰。

方源战胜打跑的西漠蛊仙肥娘子,战力也属于六转上等。拥有三种威力不俗的凡道杀招,尤其是最后落跑时的移动杀招,方源也追赶不上。

但方源本身是仙僵之躯,和轻虚蝠翼、发甲杀招相互之间紧密搭配,关键更在于方源丰富狠辣的战斗才华,因此把肥娘子打得斗志全消,只能逃跑。

雪松子的战力,也应当是六转上等。

他是蛊仙中的富翁,虽然资助马家失败,亏了一大笔,但仍旧还有底蕴。他的仙元一定不缺。按照黎山仙子的情报,他早年就收购了许多杀招。作为魔道出生,本来就注重战斗力,战斗造诣也不比寻常。

雪松子毕竟是老牌蛊仙,若方源不动用万我、仙蛊和他交手,孰胜孰败还须打过一场才会知道。

而黑城的战力,则是七转中等!

黑城比雪松子、黑柏的资格更老,七转福地早已步入正轨,产出的仙元更是红枣仙元,比青提仙元要更高一个档次。

他是黄金部族的蛊仙,杀招也不会缺乏。拥有仙蛊暗箭,已经是很久以前的事情。根据黎山仙子的情报,数月前黑城在北原某处似乎争抢到了一只仙蛊。但究竟是什么,黑城方面还没有主动暴露。

论及黑城的生平战绩。也是不俗,和一些正道蛊仙公然切磋过,也和魔道蛊仙血战到底,斩杀过好几位魔道蛊仙。

黑城不缺乏仙元、杀招,个人战斗造诣也是不俗。惟独仙蛊数量稀少,若是有一两只用于战斗的蛊,战力评估将升至七转上等。

方源虽然和黑城交过手,但战斗时间极短,方源并无正面交锋,而是以带着黑楼兰撤退为先。

平心而论。方源虽然仙蛊众多,但没有一只仙蛊专门用来攻防。暗箭袭来,方源只能躲闪,不能硬抗。幸好净魂仙蛊担当万我杀招核心蛊后,令方源有了仙道杀招。

正因为这张底牌。方源才有了和黑城正面交锋的一战之力。

方源对敌我双方的战力对比,心底一清二楚。黑楼兰渡劫将至,方源压力很大。

“叶凡迟迟未至,我为黑楼兰渡劫而准备的时间就越少。也许我应该放弃等待,回到狐仙福地抓紧时间战备,尽量地提升自己的战斗力。”

连续等了这么多天,方源心中不免有些微微动摇。

尤其是他想到,前两次连运。都会出现意外,导致连运失败。韩立的时候,是肥娘子出现。方源不得不打了一场,展现出凶狠之后,才把肥娘子惊跑。洪易的时候,是黑楼兰被追杀,方源被迫放下手中事物,赶去救援。

“难道这一次我企图连运叶凡;

。也出现了意外?不是出在我的身上,而是出现在叶凡的身上?”

方源的猜测。正好猜中了真相。

运气是不断变化的,宛若潮水。时涨时落。

大难不死必有后福,方源从大同风幕下逃出生天,原本的黑棺气运已然消散大半。

而后,他连运韩立、洪易,都是精挑细选的强运之人,只比马鸿运差。方源的运气因此大为改观。

这一次他连运叶凡,双方隐约的气运之争,已经再不是方源遇到麻烦,反而是叶凡的气运隐隐避让,带给叶凡一个躲避危难的机会。

叶凡抓住了这个机会,拖延了七天七夜,并成功收服了一头兽皇。

但可惜的是,当他接近碧龙江边时,方源只是心生动摇,还并未动身离去。

“咦?叶凡身边的这头犬皇,不就是我在三王福地中那头嘤鸣吗?居然没有死,还跟随了叶凡?”方源认出了犬皇,心生诧异。

这事情可在《叶凡传》中,并无记载的。

连运叶凡的过程,十分顺利,并无意外发生。或者也可以说,叶凡的迟到已经是一场意外了。

叶凡实力低微,至始至终都并未发现方源暗算他。再次验证了:好运坏运都非决定因素,只要实力足够,就能抵御厄运,抓住机遇。

叶凡的个人气运,也相当特别。

他的气运中,竟然也有一个棺椁。

不过,这个棺椁并非方源之前的黑沉棺椁,而是青铜色泽,古朴神秘。棺椁周围,还有九条盘旋的气运游龙守护着。

和方源气运相连之后,九条气运游龙立即缩减到四只,青铜棺椁也缩小到原来的一半大小。

方源对这个结果,相当满意。

叶凡气运缩减了一大半,但仍旧还有大量剩余。

“韩立、洪易、叶凡都分别和我连运,四人的运气已经均分。叶凡如此运气,同时也是我的运气程度。韩立的运气、洪易的运气也是同等浓郁了。”

一番辛苦到此地步,方源总算解决了春秋蝉带来的第二弊端。

不仅再没有之前的倒霉运势,而且还超出寻常,勉强算是一位强运之人了。

“可惜这人选,真的不好找。记忆中虽然还有几位,但此时都未出生呢。”方源暗暗可惜,气运浓烈且一直保持的目标数量稀少。这个阶段,方源也只能找到这三位。

至此,连运之事告一段落,方源看着叶凡不断接近江边遗藏,不由微微一笑。

碧光闪过,他悄然消失。(未完待续);

------------

\end{this_body}

