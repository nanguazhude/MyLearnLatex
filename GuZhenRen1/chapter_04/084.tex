\newsection{承认自己的平凡}    %第八十四节:承认自己的平凡

\begin{this_body}

%1
因为有前一次的炼制经验,这一次方源凝练荒兽蝠翼,却是驾轻就熟。**

%2
狐仙福地时间的大半个月后,他就将这对荒兽蝠翼,移植到自己的后背。按照之前的优良传统,方源仍旧保存了蝠翼部分的痛觉。

%3
这天,方源站在荡魂山的山巅,开始试验杀招。

%4
百里眼。

%5
他心念一动,赤红的双眼目光骤然大盛。环顾四周,一百里之内的任何草地、泥石,皆秋毫毕现,宛若放在面前。一百里之外,视野就变得模糊,但仍旧比之前肉眼所观,要清晰很多。

%6
当达到四百里之后,视线便达到极限,所见一片模糊,只能分辨一些色彩、光影。

%7
“百里眼是千里眼的基础,后者是仙道杀招,却是需要相关的仙蛊,才能施展的。”方源满意地点点头,这个杀招价值六块仙元石,的确效果十分出色。

%8
只是视线不能破隐,不能穿透,一旦被障碍物阻挡的视线,就看不到背后的东西。这是百里眼杀招的弊端。

%9
方源将优缺点记在心头,轻轻一振背后双翼,身体飞上天空。

%10
他一边催动着百里眼杀招不停,一边飞翔遨游,视察天上地上,体会着清晰的视野。

%11
狮毛甲。

%12
他心念一动,又催起防御杀招。

%13
很快,他的身上就笼罩了一层黄铜甲胄,狮口头盔。就连背后的双翼,都覆盖上一层坚厚的甲片。防御大增。

%14
但随之而来的,却是速度变慢了。

%15
“狮毛甲和返实蝠翼,相互之间并不能完美搭配。一旦催动狮毛甲。返实蝠翼的速度就降低。若不用,速度虽然快,但防御方面却是兼顾不了。”方源皱起眉头。

%16
这个问题,已经超出了他的能力范围。

%17
能够改良出狮毛甲这等移动杀招,平衡三道流派的道痕,已经是变化道大师境界的极限。

%18
还要再兼顾狮毛甲,这就太难为方源了。

%19
若他是变化道宗师级人物。还有可能。但要达成宗师这一层次,通常都需要大量的时间积累底蕴。这个时间不是数十年,而是上百年。数百年。

%20
方源目前境界最高的,是血道和力道。

%21
血道宗师境界,是方源前世五百年历练,当中的两三百年的深厚积累。无弹窗,最喜欢这种网站了,一定要好评]从这一点就可看出。境界提升之难。积累之难。

%22
而力道也是宗师境界,除了一小部分是前世的积累之外,更多的机缘是在方源得到了狂蛮魔尊的真意灌输。否则要积累到宗师境界,也得花费两三百年的长时间积累。

%23
到达蛊仙这一层次,已经有实力搜寻并且把握寿蛊,或者有其他种种手段来延寿。寻常凡人,寿命为一百岁。但蛊仙只要经营有加,都是数百岁。甚至上千岁。

%24
悠长的寿命和时间,使得积累这个词的意义更加重大。

%25
方源、黑楼兰为什么打不过仙猴王石磊?

%26
因为石磊成仙之后。背靠战仙宗,积累了近三百年!方源、黑楼兰不过才刚刚升仙,如何比得上人家三百年的积累?

%27
打不过是正常的!

%28
那么,方源、黑楼兰为何刚刚成仙,战力却已然在六转中属于中上等了呢?

%29
方源靠的是八十八角真阳楼的前贤资助,黑楼兰更多的依靠母亲的遗产,以及小姨妈黎山仙子的帮助。

%30
换句话来讲,他们俩个是借助了前贤先人的资助和积累!

%31
“要想实力突飞猛进,单靠自己摸索前行,无疑是漫长的,效率很低。前世我走血道,花费了两百多年,才自我参悟出十多个凡道杀招来,创造出数百只血道新蛊。这个教训,自然要吸取。”

%32
“这世界,能人辈出,天才俊杰比比皆是。漫漫历史长河中,到处都是闪烁着的明星、巅峰。要想脱颖而出,勇猛精进,唯有利用前贤遗藏和积累,将他们的精华都吸收到自己的身上。正所谓海纳百川,有容乃大,采百家之众长,方可领袖群雄,乃至凌驾众生之上,达成无上之伟迹。”

%33
方源飞在空中,目光闪烁,心生感慨。

%34
人往往在年轻的时候,常常自命不凡,总觉得自己是舞台的主角。等经历了一些事情后,才会明白和接受:很多其他人的优秀,并不亚于自己。

%35
正所谓:人外有人,天外有天。

%36
一个人的寿命,终究是有限的,宛若一艘孤舟。而这个天地的奥秘道理,就相当于一片浩瀚无垠的汪洋。

%37
古往今来,能够在某条道理上达到天地巅峰的人物,也不过只有十位。但就算是他们,也不能面面俱到,门门巅峰。因此巨阳仙尊、盗天魔尊,也要在炼蛊方面求助于长毛老祖。

%38
《人祖传》中,也有记载,说

%39
人祖踏上他的人生之路,周围一片黑暗,脚下一片肮脏的泥泞。

%40
人祖便问自己蛊:“这究竟是哪里?”

%41
自己蛊就道:“这里是平凡深渊最底层的平凡泥潭。”

%42
平凡深渊人祖是知道的,他的大儿子太日阳莽,就曾经被困在深渊过。

%43
人祖不由振奋地道:“这既然是平凡深渊的底部,那岂不是说我已经脱离了生死门,重新回到了世间?岂不是说我已经重生了?”

%44
“也可以这么说。”自己蛊道。

%45
“但这怎么可能?我还没有通过落魄谷,没有闯过逆流河呢。”人祖难以置信地道。

%46
自己蛊:“人啊,你要知道,你说的那条是宿命蛊走过的路。而你已经走上了新路,这是一条全新的路,是由你开辟的。这条路走向哪里,全按照你的心意。我早就说过。路就在你的脚下,只要你想走。”

%47
人祖恍然大悟:“原来是这样啊。”

%48
旋即又疑惑:“那我怎么走到了这里来呢?”

%49
自己蛊道:“人,生来就平凡。虽然是万物之灵。但没有猛虎的爪牙,不如草叶可以汲取大地的营养,没有**的变化。你来到平凡的泥潭,又有什么奇怪的呢?不只是你,你仔细看看脚下的泥泞,你就会发现有很多的脚印痕迹。”

%50
人祖低头,非常趋近地面之后。这才看清楚,果然如自己蛊所言,平凡泥潭上布满了各种痕迹。有野兽的爪印。也有草木扎根的根基,有水液流动的痕迹,也有石头滚过的痕迹。

%51
“怎么会有这么多的痕迹?”人祖好奇地问道。

%52
自己蛊便答:“这里是平凡深渊,万物生灵会因为各种原因。进入这里。万物都是平凡的。不过绝大多数的存在。一生都陷入平凡深渊里去了。只有少部分的存在,通过自己的努力和跋涉,走出了平凡深渊。”

%53
“我可不想陷在这里一生。这里什么都没有,臭气熏天,我也要走出去。”人祖皱眉道。

%54
自己蛊哈哈大笑:“你觉得平凡是深渊,那它就是深渊。但若觉得平凡是天堂,所以它就是天堂。你既然不想留在这里,那就走吧。用你的双脚一步步走出平凡泥潭,成为不平凡。”

%55
人祖走出一步。

%56
忽然身体一歪。他的前脚深深地陷进泥潭当中了。

%57
泥潭里可不好走,深一步浅一步。表面上看起来都差不多的路面,有的比较凝实,有的比较松软。

%58
人祖走了几步之后,忽然眼前一亮,想到了窍门。

%59
他专往印有痕迹爪印的地方走,这些地方既然能够留下痕迹,正说明土质比较凝实。

%60
于是人祖走路,变得十分顺利,和之前相比,简直是健步如飞。

%61
他有感而发道:“原来在平凡的泥潭中,踏着前者的脚步走,速度比一个人摸索要更快啊。”

%62
……

%63
试验了返实蝠翼和狮毛甲后,方源从高空降下,随意地落在一处草原缓坡上。

%64
见面不相识。

%65
方源催动这个新得来的凡道杀招。

%66
组合杀招的蛊虫,基本上都是由狐仙地灵从宝黄天中收购来的。少部分则是自家的毛民所炼。

%67
在杀招的作用下,方源蝠翼收起,八只粗壮的手臂也只剩下一对。口中的獠牙大为收敛,赤红的眼珠子也变成了微红。

%68
最终,方源高达两丈的身躯,也缩小成正常人的形态,面目大变。

%69
只是仙僵之躯,仍旧是仙僵之躯,并未遮掩住。

%70
尽管如此,方源也不禁交口称赞起来:“好厉害!这只是凡道杀招而已,却将我这仙体进行如此大的改变。不愧是盗天魔尊的手笔。”

%71
方源可以想见,若他此时还是个凡人,完全可以变化任何模样,甚至是由男变女,都惟妙惟肖。

%72
而他现在是仙体,身上充满了力道道痕。作为凡道杀招,也能改变这么多,很不容易。

%73
关键这种改变,不是幻象,而是真的**的变化,甚至还涉及到气息的微妙变化。

%74
总而言之,就算是方源此刻站在太白云生的面前,短时间内,他也认不出这位仙僵,就是自己的师弟。

%75
“主人,你样子变好看了呢。哦,这封信是刚刚来的。”就在这时,地灵小狐仙闪现过来,手中高举一只信蛊,递给方源。

%76
“哦?报信青鸟蛊……灵缘斋方面终于有回应了?”方源目光一闪,取来信蛊,探入心神。

%77
不一会儿功夫,他抽回心神,脸上闪过一抹思索的神色:“凤金煌居然要我参加炼道大会,比拼炼道?我记得她早年,对炼道根本就没有兴趣,这方面的底蕴堪比一张白纸。她怎么有这么强烈的自信,要和我赌斗炼蛊?难道说,她已经发现了梦翼仙蛊的真正作用了?”

%78
方源心中一沉。

%79
“梦翼,梦道……”方源目光如烟云,想起了前世的种种。

%80
前世他为什么要伙同一干魔道蛊仙,千方百计地进攻狐仙福地呢?就是为了斩杀凤金煌,夺得梦翼仙蛊啊!

\end{this_body}

