\newsection{方源智退回风子}    %第三百四十四节:方源智退回风子

\begin{this_body}

回风子又惊又怒。

他虽然投降的并不甘愿,也是迫于形势。但既然投靠了中洲势力,他也只能暂时硬着头皮去为中洲做事。

中洲蛊仙既然能将他放到这边,命令他看守落魄谷,自然在他身上布置了许多手段。

回风子受制于人,为了自身安全着想,必须站出来,阻止方源收取落魄谷。

但是当他见到方源的时候,惊怒交加的回风子,忽然浑身一颤,惊惧地低呼一声:“凤九歌!”

方源淡淡地瞥了他一眼,神色平淡得很。

中洲蛊仙既然已经全数离开北原,回归中洲。那么自然要在北原,布置一些后手。精明如方源,岂会没有考虑到这一层?

所以,他在收取落魄谷之前,就变作凤九歌的模样。

“原来中洲蛊仙安排了回风子在这里……看来中洲手段不俗,让这位‘北原第一速’也要违心维护中洲利益。并且,看他这个神情,刚刚没有看到凤九歌出去?难道凤九歌还在这谷中?不,不太可能。百日大战,影宗蛊仙已经突围,凤九歌知道这里是一处险境,他伤势沉重至极,虽说是要报还我的救命恩情,但的确是急于脱困。真传空间中,都没有和我说一句话。”

防人之心不可无。

这点,凤九歌做的极为明智。

毕竟方源不是赵怜云,前世赵怜云出现的时候。凤九歌主动出声指示。这一世,轮到方源出场,凤九歌就闭口不言。

还别说。

若方源提前知晓凤九歌的存在,很可能就动手了!

杀死一个凤九歌,就算得不到他的仙蛊。得到他的魂魄,搜搜魂,也是一笔极大的财富啊。

更何况,若是活捉了凤九歌,不管是勒索灵缘斋,还是灵缘斋或者凤九歌本身的敌对势力,都是一笔大买卖。

“凤九歌伤势极重。深具防范之心!他没有和我说一句话。或许离开落魄谷后,他也发现了回风子,只是没有招呼他。这其中的原因,一来是他不知道回风子投降,二来就算他知道回风子投降中洲了,也怕十大古派之间的内斗。毕竟人心叵测,而凤九歌此时的状态。已经无法冒险。”

方源表面上不动声色,但暗中早已经心思电转。短短几个呼吸的时间,他的脑海中不知道转过了多少的念头。

“是真的!”回风子暗中稍稍动用手段,鉴定真伪之后,他顿时松了一大口气。

凤九歌可是中洲蛊仙的首脑,若是他出手收取落魄谷,那回风子就没有理由去阻止了。

“属下见过凤九歌大人。”回风子老老实实,一脸恭敬地飞上前来,对方源躬身行礼。

“嗯。”方源点头应了一声,“发生的事情。我已经了解了。你为我护法,收取落魄谷不容外力干扰。”

短短一句话,说的不多。但已经提前解释了一些问题,防止回风子产生更多的怀疑。

“是,大人!”回风子连忙点头。

就这样,在回风子尽忠职守的守护下,方源成功地收取了落魄谷。

“这里的落魄谷。虽然被收取,但还遗留下不少仙窍福地。你仍旧在这里看守,不得有误。”临走之前,方源还严肃地关照回风子。

回风子不疑有他,连忙应是。

方源施施然飞离此地。

“凤九歌既然已经脱困而出,证明秦百胜已经败亡了。”回风子看着他的背影,心中对凤九歌的强大感慨不已。

对这个战果,回风子也并不吃惊。

毕竟他也知道,秦百胜重伤的事实。

回风子忽然想到:“不好。中洲十大古派相互钳制,内部也有竞争。凤九歌只是灵缘斋中的一员,他收取了落魄谷,我没有理由阻止,但也必须将这消息传过去,让其他九派得知才是。”

念及于此,他没有迟疑,连忙催动信道手段。

凤仙太子很快就接到他的传信。

“哦?凤九歌还活着?他不仅出来了,而且还收取了落魄谷!”凤仙太子喜不自禁,哈哈一笑。

再说,方源远离回风子后,一路飞驰,相对安全之后,和太白云生、羽民蛊仙周中汇合。

这两人早已经被方源安排在远处,已备不时之需。

同时,他一直保持着和黑楼兰、黎山仙子的通信畅通,一旦有什么危机,就会通知她们支援。

再加上黑楼兰和焚天魔女的关系,就算八转蛊仙出现,方源也有勉强应付的可能。

当然!

此时的情况,再好不过。

方源独自一人,就成功收取了落魄谷。

而黑楼兰可不像前世,她没有黑城的魂魄,无法得知落魄谷的存在。

方源打算闷头发大财,不将落魄谷的事情告知对方。

黎山仙子曾经打压过方源,方源更不可能将落魄谷的事情,告知她们。

她们一旦得知,依照焚天魔女的强势霸道,必然要对落魄谷有所要求。

“就算将来,我要用落魄谷和她们进行交易。也得是我魂魄大成的时候……”

方源回到星象福地,将落魄谷安置在这里。

狐仙福地中,黑楼兰也时常过去。她要用力气仙蛊,辅助毛民炼制气囊蛊的。所以,将落魄谷放置在狐仙福地之中,并不太明智。

落魄谷的事情告一段落,但方源的心境却久久不平。

落魄谷的收获,虽然远超他原先的预料,但过程中却大大超出了方源对局面的掌控。

丧失对局面的掌控。这种感觉,让方源十分讨厌,并且感到阵阵的不安。

盗天真传中,方源收获了鬼不觉。

方源探查全身上下,发现魂魄上似乎被包裹了一层道痕。

但是鬼不觉的真正用处是什么。方源还没有搞清楚。

好在他已经加入了琅琊派。

于是,方源便亲自去琅琊福地,询问盗天魔尊的事情。

琅琊地灵没有隐瞒,也没有提任何要求,直接告诉方源许多情报。

“盗天魔尊的传承有很多,但真传只有十道。你之前得到的传承,让我承诺炼蛊三次。并不是真传。每一道盗天真传。都是举世无双的珍宝,是盗天魔尊一身修为的精华所在。鸡犬得到,都能升天!不过每一道真传的标准,都只有一个。那就是继承者必须是天外之魔!”

“传闻中,盗天魔尊拥有两大防御杀招,称之为神不知鬼不觉。神不知的用途,便是屏蔽任何念头、意志、情感的推算。至于鬼不觉。一直很神秘,我知道的也不多,似乎是和魂魄有关。咦?你突然对盗天魔尊的这些情报这么感兴趣?为什么?难道你得到了什么盗天真传的线索么?”

方源哈哈一笑:“你猜的不错。”

“神不知、鬼不觉……”琅琊地灵感慨万分,“我若是有神不知在手,琅琊福地就会成为真正的乐土,再也不受天灾地劫的骚扰。可惜的是,盗天魔尊当年安排了十道真传,秘密非凡,至今我都没有听过,有什么人获得他的真传。他的真传都是成双入对。神不知、鬼不觉就是一对。继承其中一个,就有另外一个的线索。方源,你若能取得神不知,贡献给琅琊派,我必将立你为太上二长老。在整个琅琊派中,你只屈居我一人之下,拥有极大的权柄!并且。你还可以在我的库藏中,免费获得一只七转仙蛊。啊,不,至少两只七转仙蛊!!”

方源点点头,琅琊地灵的性情改变,对他的确有不少好处。

至少,换做之前的琅琊地灵,个性保守,绝不会许诺方源这样的好处。

“不过,我若是得到了神不知,那可是盗天魔尊的真传!当初盗天魔尊四处偷盗,无人可算出他的位置,不就是因为神不知、鬼不觉吗?这种传承,可比两只七转仙蛊,要珍贵无数倍啊。”方源眼中泛着笑意。

琅琊地灵神情一囧,哈哈一笑,自己化解了尴尬:“你说的不错。其实我的话还未说话,不仅是两只七转仙蛊,还有无数的仙蛊方,乃至仙道杀招,各种秘闻,传承线索等等,你都可以查询!怎么样,这个奖励可足够了?”

“嗯……这还差不多。等我得到神不知,我会回来的。”方源得到了自己想要的情报,不留恋琅琊福地,转身离开。

琅琊地灵瞪着双眼,看着方源的背影,嘀咕着:“这个家伙……搞不好真的有什么线索了!如果得到了神不知,我还担心什么?哇哈哈!就算给他当上太上二长老,其余的毛民蛊仙也都只听我的,哇哈哈!哎呀,一不小心,又把心里话说出来了啊!”

琅琊地灵捂住嘴,眼巴巴地望着方源。

方源早已经见怪不怪,肚中暗笑。

换做其他势力组织,他兴许还会防备和小心。但加入琅琊派,有这么一个首脑,简直是再轻松不过了。

“你说的话,我都听到了。”方源头也不回,向琅琊地灵扬扬手背,催动定仙游,离开了琅琊福地。

留下琅琊地灵一脸懊丧,恨得直跺脚,口中嘟囔:“该死!该死!”

回到狐仙福地,方源目光闪烁不定,陷入沉思之中。

“我是穿越之人,肉身虽然土生土长,但魂魄却是天外来客。因此我是天外之魔,能有资格继承盗天真传。”

“之前,盗天魔尊留下的话中,似乎也暗示了他也是天外之魔。”

“盗天真传继承起来,说难很难,说容易也相当容易。关键只有一点,那就是天外之魔的身份。难怪盗天魔尊设置这处真传,广撒线索。我之前还有些奇怪,为什么开门蛊、关门蛊的线索,这么容易搞到手。”

“但我得到鬼不觉之后,也没有神不知的线索。这样说来,神不知岂不是已经被他人捷足先登了?”

ps:今天两更。一更保底,第二更是10月份月票破1700的加更。月初了,求一下保底月票。老规矩,100月票加更一张!上个月十分感谢大家的支持,让我更加热情,更加努力地来创作。在这里,鞠躬感谢诸位对本书的大力支持和贡献!

\end{this_body}

