\newsection{黑楼兰升仙(下)}    %第三十六节:黑楼兰升仙(下)

\begin{this_body}

%1
从黑楼兰的眼中,猛地喷发出耀眼夺目的光辉。

%2
光辉凝练万分,形成有如实质的光柱,贯穿劫云,直射两道音磬!

%3
轰的一声齐响,音磬炸碎,彻底消弭,再也无法生成。

%4
光柱一闪即逝,黑楼兰旋即闭上双眼,整个过程极为短暂,只持续了一息时间。

%5
这是排难之光!

%6
黑楼兰动用了仙蛊排难!

%7
仙蛊之威,果然非同凡响,和方源凡道杀招冰钻星尘,形成鲜明的差距对比。

%8
黑楼兰平息了一会儿,又睁开双眼。

%9
两道排难光柱瞬闪即逝,又灭两道音磬。

%10
方源暗自震动:“这样的威力,绝非排难仙蛊的简单运用,而是配合了凡蛊。通过目光打击,使得排难之光更加凝实,更加精准。换句话说,这是黑楼兰已经研发了以仙蛊排难蛊为核心的仙道杀招!”

%11
黑楼兰天资卓绝,得到排难蛊没多久,就研发了仙道杀招?

%12
这让方源感到匪夷所思,但眼前的事实却胜于任何雄辩!

%13
这个时候,黑楼兰再睁眼,再灭两道音磬。

%14
但这一次她闭上闭眼,却是从眼角出留下两股鲜血。她睫毛震颤,娇躯微抖。手掌中,紧紧捏住的排难蛊,更是出现裂痕,发出丝丝的哀鸣。

%15
这个仓促间研发出来的仙道杀招,并不完善,有着很强的弊端。

%16
“还剩下两件音磬!”黑楼兰不管不顾,狠狠咬牙。再一次强行睁开双眼,一鼓作气地将天灾的最后两道音斗消灭。

%17
噗、噗。

%18
她的亮丽眼眸在这一刻崩碎爆裂,随之两道血泉。从她空洞的眼窝喷涌而出。

%19
砰的一声,她手中的排难仙蛊也自爆破碎,碎片化作点点光辉,从她手指缝隙泄露出来,旋即彻底消失。

%20
黑楼兰彻底失明!

%21
她虽然是大力真武体,恢复力惊人。但眼珠伤势,却是强行催动仙蛊排难。运道道痕残留在眼窝上,让她无法复原。

%22
六转的排难仙蛊,也因此损毁。成为绝唱。

%23
黑楼兰决绝无比,有枭雄之心,付出惨重代价,却直接消灭了天灾。如今只剩下地劫!

%24
这一刻。方源也不由在心中暗赞一声:“果断!”

%25
十绝升仙,天劫地灾极为恐怖。更糟糕的是,或许还会有强敌扰乱。巨大的生死存亡压力,让黑楼兰不得不舍弃排难仙蛊,以及自己的一对眼珠。

%26
这是常人难以做到的决断。

%27
如此舍弃,令黑楼兰的枭雄胸怀,惊人气度一展无遗。

%28
天气垂下,如一道清辉瀑布。地气上涌。如黄金喷泉。黑楼兰身上积蓄的深厚底蕴,也化为一股庞大的人气。猛地爆发开来。

%29
“这就是十绝体的人气?”太白云生在远处看得目瞪口呆。

%30
黑楼兰一人的人气,比方源和太白云生的人气总和,还要再庞大数倍上去。

%31
十绝体的人气,向来充裕无比,这其中十绝空窍占据主要因素。

%32
尽管有了心理准备,仍旧还是让观者震惊。

%33
三气相距,很快形成巨大气茧,将黑楼兰包裹其中。

%34
到此处,她便进入了第二阶段——纳气。

%35
黑楼兰全心全意平衡三气,小家子气蛊尽数收回,短时间内她无法抽身对敌。

%36
咔嚓嚓……

%37
冰川崩裂,天地震动。

%38
就在这时,一声巨吼宛若旱雷陡然炸响,从冰川深处跳出一头巨大暴猿。

%39
轰的一声,暴猿落地,踩出两个深坑,深坑周边轻脆的冰片四处迸溅。

%40
暴猿拍动胸膛,仰头咆哮,声音响遏行云,翻腾不休的劫云都因此一滞。

%41
它浑身如雪,双目赤红如血,根根猴毛如道道冰刺,直刺向天。它有数百丈之高,大如山岳,气势滔天,凶恶绝伦。方源比之,仿若猫面前的一只小小甲虫。

%42
“地劫产生了!竟然是上古荒兽劫!这可是七转蛊仙才能遭遇的地劫啊。”太白云生一颗心直往下沉。

%43
黎山仙子心中则不断回荡着一个名字:“冰瀑神猿!”

%44
方源面沉如水,冰瀑神猿巨大的阴影投射下来,这是七转战力。如果身上还寄居仙蛊的话,那就更加糟糕,与其对战势必难上加难

%45
黑楼兰全心平衡三气,黑城并未现身,黎山仙子、太白云生潜伏远处,形势险恶万分,方源双眼中凶芒一闪即逝,双掌在胸前一拍,悍然提前启动底牌。

%46
仙道杀招——万我!

%47
拳气喷涌,磅礴浩瀚,如江河倾泻,大海狂澜。

%48
几个呼吸功夫,上万个方源力道虚影,现身亮相,围绕在方源本体周围,四面八方,天上地下,结成一股圆球阵势。

%49
为了防止战况太过激烈,而导致时间不够,方源早在来临北原之前,就数次催动仙道杀招万我。

%50
形成的力道虚影大军,他都存储在自家仙窍当中。力道虚影可以维持一段时间,又非生灵,不受方源仙窍的死气侵害。

%51
方源此刻放出一部分来,立即扳回气势。

%52
冰瀑神猿原本蠢蠢欲动,想要直接冲杀过来,现在看到满天悬停的力道虚影大军,它立刻张开大口,露出尖锐的利齿獠牙,浑身肌肉发紧,选择了对峙。

%53
风雪呼啸,方源神情冷漠,身躯如石,伸手朝着冰瀑神猿缓缓一指。

%54
立时,三军齐动,喊杀声震天作响。

%55
数万的力道虚影大军,宛若滔滔洪水,掀起惊涛骇浪,向冰瀑神猿卷席而去。

%56
“如此威势!”太白云生看得心驰神摇。

%57
他不是第一次目睹杀招万我了,早在琅琊福地时。方源痛殴荒兽泥沼蟹时,就见过一次。

%58
但那一次,方源只是消耗了一颗青提仙元。打出万道虚影。

%59
而现在,却是数万大军齐齐扑杀,军势蔓延开去,自有一股铺天盖地的浩荡之气。

%60
大军连绵,冲向冰瀑神猿,途中又陡然划分四部。

%61
一部绕左,一部绕右。一部绕上,最后一部正面直冲。

%62
冰瀑神猿激发凶性,不退反进。向前一冲。方源力道虚影大军,如臂使指,在他的心念调动之下,轰然铺散开来。宛若一张巨网。将冰瀑神猿罩住。

%63
“杀!杀!杀!”力道虚影们咆哮连连,宛若只只蚂蚁,向冰瀑神猿发动围攻。

%64
冰瀑神猿左冲右突,打灭无数力道虚影,但始终冲破不了方源大军的封锁。

%65
“天底下,竟然还有这等厉害的杀招!再加上奴道大师的造诣,难怪方源被小兰如此重视了。”看到了这样一幕,黎山仙子惊喜连连。

%66
一个个力道虚影。不断地从仙窍中飞出来。方源仙窍中有十多万军力,这才出来了八万而已。

%67
这些力道虚影。最普通的只有双臂,其次是四臂、六臂。手臂越多,战力就越强。

%68
冰瀑神猿暴躁怒吼,奋起鏖战,巨大的身躯,凶猛的怪力,以及满身的冰刺,让力道虚影们挨着便伤,擦着便死。

%69
但崩散的拳气,损耗一部分,残留的却会重新凝聚起来,再度凝聚成力道虚影。

%70
这一变化,大大延缓了方源军力的削减损耗的速度。

%71
整个过程中,方源脑海中的乐意剧烈消耗。指挥数万大军,消耗乐意的速度极为恐怖,以至于短短几十个呼吸的时间内,方源催动乐山乐意仙蛊的次数已经接近了十。

%72
“奇怪,这头冰瀑神猿,虽然是上古荒兽,但身上不仅没有仙蛊,而且连一只凡蛊都没有!这是地灾从哪里挪移来的奇葩?”

%73
方源看出冰瀑神猿的虚实,没有仙蛊,甚至没有凡蛊,让这头上古荒兽的威胁大为降低。

%74
“既然如此,那是时候解决你了!”方源嘴角一咧,眼中凶芒暴涨,他亲自上场,轰然扑上。

%75
背后轻虚蝠翼连续闪烁,方源速度暴涨,宛若一颗流星。声威凶赫至极,沿途一阵阵的音爆,几息时间他便越过漫长距离,如魔神降临,轰至冰瀑神猿的头顶。

%76
“来吧,群力加持!”方源在心中兴奋地嘶吼。

%77
顿时,仙窍中四十五只群力蛊,一齐发动。

%78
群力蛊已经被方源完整地纳入万我体系当中,此刻催动起来,立即引发了万我的一种新变化!

%79
数百位力道虚影,砰砰消散,他们的力量在群力蛊的威能下,全数集中到方源的身上。

%80
方源高举一只右臂,照着冰瀑神猿的脑壳,狠狠砸下。

%81
轰的一声。

%82
他的整个前臂,刚猛无俦地砸在神猿的脑门上。

%83
巨大的反震之力,让方源的前臂直接炸裂开来,骨骼碎成渣滓,血肉化为糜粉。

%84
而冰瀑神猿的脑门,出现一个巨大的凹洞。旋即凹洞处裂痕闪电般蔓延下去,殃及整个猿首。

%85
冰瀑神猿的整个脑袋,都轰然爆碎!

%86
看到这一幕,不管是黎山仙子还是太白云生,都吓得眼珠子都要瞪掉下来。

%87
就连方源都大为吃惊:“怎么回事?居然整个猿首都被打爆了?不应该啊,这可是上古荒兽!”

%88
万我杀招是他自己独立完善的,群力蛊是他亲自添加进去的,自然也清楚群力加持后的战力。顶多打得冰瀑神猿头破血流,不至于整个猿首都被击爆。

%89
“小心!”正在这时,太白云生终于忍不住现身,大声示警。

%90
方源回首一望,瞳孔剧烈收缩。

%91
竟然在方源没有察觉的情况下,冰瀑神猿的巨拳,悄无声息地接近,如毒蛇出击,闪电般砸下。

%92
方源躲闪不及,被巨拳打个正着!

\end{this_body}

