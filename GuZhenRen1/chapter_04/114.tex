\newsection{三方互赢}    %第一百一十四节:三方互赢

\begin{this_body}

“解谜仙蛊……解谜仙蛊……”方源口中喃喃不断,手指头敲击在扶手上面,发出咚咚的响声。

大厅中,喊价声不绝于耳。在整场拍卖大会的最后时刻,竟然杀出了赤煞神舟这样的重宝,这是出乎意料的惊喜。

普通的散修,独行者就不用想了,唯有超级势力的七转蛊仙,以及八转大能才有一争之力。

尤其是超级势力,仙蛊屋对于超级势力的诱惑,是绝大的。

价格以极快的速度,猛烈攀升。虽然是自由交易的环节,但此刻众人却是为了争夺,下意识地激烈竞争,采纳了竞拍的规矩。

秦百胜苦笑。

规矩是他定的,但看到这么些强者嘶吼叫喊,他明智地选择作壁上观。现在去强调什么交易规矩,就是将这些巨头统统得罪一遍。

“看到没有,这就是赤煞神舟的吸引力,你们僵盟实在太抠门了!”卖方借势嚣张,对僵盟中的两位首脑传去信息。

夜叉龙帅整张脸都黑了。

阴六公则是唉声叹气,现在的这个价格,已经高出僵盟开价一大截。这还不算完,竞价还在节节上窜。

阴六公也是明白人,心知到了此刻,这赤煞神舟已算和僵盟无缘了。不晓得首领回来,会怎么批判处理办事无能的自己。

方源一边听着报价,一边则在静心思量。

“我手中的仙蛊俘虏,只剩下四人。其中一位吴浩。号称奔雷手,乃是俘虏中唯一的七转蛊仙。还有三位,就是花海三仙。属于木道。”

很显然,用这四位蛊仙俘虏,想要去换取百足天君手中的解谜仙蛊,成功的可能不高。

百足天君乃是宇道蛊仙,流派不和,这四位蛊仙俘虏的吸引力就骤降了。

关键还有一点,仙蛊唯一。解谜仙蛊又极为实用。若非是赤煞神舟勾引,百足天君才不会将此等好蛊暴露出来呢。

“我若强换,恐怕不妥。刚刚换得的力道仙蛊。百足天君看不上。若用招灾仙蛊,以七转换六转,却是极大的亏本买卖。至于定仙游、连运这些仙蛊,对我而言缺一不可。是不能拿出去做交易的。”

但若得不到解谜仙蛊。方源又不甘心。

错过这次机会,错过这样的氛围,以后哪怕资本充足,想要再交易也就难了。

解谜仙蛊本身,并不是方源必需之物。但方源前世记忆中,却有一法,可用此蛊充当核心,再结合其他稀罕辅蛊。酿造成一记梦道杀招!

梦道的仙级杀招!

在这个时候,三尊说流传五域。涉及大梦仙尊,基本上各大超级势力、八转巨头,都在研究梦道。

研究成果虽然肤浅,但如今已经有梦境外显。

梦道的仙蛊,也已经现世两只。

一只在凤金煌手中,一只为毒蝎娘子掌控。说起来,还是后者更加可俱。凤金煌还只是凡人小丫头,毒蝎娘子却是坐镇福地,陷入沉睡,不断探索梦境。

谁都知道,梦道是未来的潮流走向。因此敝帚自珍,纵然研炼出许多梦道凡蛊,也从不流通出去,当做自家最大的秘密。

能够成为蛊仙的,基本上都是人中龙凤,自然有着深思远见,精明得很。

因此即便是这样的拍卖大会,也从未见到什么有关梦道的宝物出卖。

方源当然也知道许多梦道凡蛊的蛊方。但凡蛊对探索梦境而言,帮助并不大。要不然各大超级势力,早就开始大规模地渗透梦境了。

真正给力的,是梦道仙蛊。

奇妙的是,不管是现在的两只,还是方源记忆中后来出现的梦道仙蛊,俱都落在女性手中。

或许还是应了三尊说,两男一女三仙尊。以此推测,大梦仙尊就是女仙尊。

短时间内,方源对梦道仙蛊是求不成的。但是若有梦道仙级杀招,却是可以替代梦道仙蛊的作用!

蛊是天地真精,人是万物之灵。

一句话用蛊,也得看谁用。没有最强的蛊虫,只有最强大的蛊师。天才蛊师常常能将不同的蛊虫组合起来,从而更具威能妙用。

就譬如方源自己,之前也没有力道仙蛊,仅凭魂道的净魂仙蛊为核心,再结合其他凡蛊,就能形成仙道杀招万我,开创奴力合流的起点。

这记梦道杀招也是如此,核心仙蛊是智道解谜,但使用出来,却是梦道的杀招。

“我若能得到此蛊,便可采集梦道蛊材,一一炼出凡蛊。虽然耗费时间精力,但组合杀招成功,收益将是千倍万倍!大大弥补我前世的遗憾,成为探索梦境的先驱,走在整个天下或大或小的势力、大多数的巨头前面!”

方源想的口干舌燥。

他略作估量,在心中评价这梦道杀招的价值,绝不亚于智慧蛊。仅次于春秋蝉、狐仙福地。高过全力以赴蛊。

例数方源前世今生的机缘,稳坐第一的是春秋蝉。尽管只是六转,远不及智慧蛊的九转,但令方源重生再来过。纵然有失败率,但每次都成功了不是?只要失败一次,它就是最糟糕的机缘。但它成功了,那就是当之无愧的第一。

狐仙福地是第二。让方源在凡人阶段,就有蛊仙的基业,站在蛊仙的肩膀上。成为仙僵之后,更是大大弥补仙僵方面的不足。正是因为有了此地,才令方源有了充裕资本。蛊仙修行从来都不是一味的打打杀杀,经营供给才是最大基石。

智慧蛊是第三。蹭用智慧光晕,带给方源无限灵感,以迅猛极速将方源带出困境泥潭。但归根结底,还得落于福地、人文的基础上。或许今后真正执掌。能带给方源惊喜。但现在只能排第三。

至于第四,便是全力以赴蛊。此蛊虽然只是四转凡蛊,但得到的时机却是极好。乃是方源重生以后的第一块跳板。没有全力以赴蛊带动,就没有方源的出色战力。没有出色战力,就会被淘汰,兴许都参加不了三王传承。

至于梦道杀招,若能得到,方源善加利用的话,就能和现阶段的智慧蛊相当。

过了这个时期。梦道杀招的价值就不断降低。若是在梦境全面外显,梦道全面勃发时,再得到这个杀招。价值就更加一落千丈了。

评价机缘的价值,主要还是看时机。

春秋蝉的价值大,但如果等到方源成就了魔尊,天下无敌了。还有什么用呢?

狐仙福地是好。但方源若是成为蛊仙,拥有自家鲜活福地,再得到狐仙福地,价值就不像现在这样高了。

同样的道理由,也适合智慧蛊、全力以赴蛊,以及这记梦道杀招。

“但要得到这记梦道杀招,核心仙蛊解谜,必须争取到手。否则一切都是空谈。”方源再看拍卖形势。

此刻,竞争者已是寥寥无几。

凤九歌、雪胡老祖二人已经被刷下。这两人花费巨额,买下马赵二人,伤筋动骨,此刻却已无力与他人争锋。

药皇也遗憾退出。他之前收购了一些炼道仙蛊,同时又是吃下蛊仙俘虏的大户。再加上炼道本身就极耗资源,种种因素让他被挤出竞争行列。

五行**师退出更早,他在拍卖会中收购了许多主流流派的仙蛊,本身又是散修。

最后剩下的己方,是百足天君、东海八转蛊仙,以及超级势力三家,分别为:袁家、慕容家、单于家。

很快,单于家也退出竞争。袁家袁让尊尽管叫嚣着,扬言要把全部的真武鲤都卖光,但最终也下不了这个决心。毕竟袁家早有了一栋自己的仙蛊屋了。

如此,便只剩下慕容部族、东海蛊仙、百足天君三人争锋。

“慕容尽孝,我身为教主,教中却没有一件仙蛊屋镇压,这大大不妥。今日你方若能罢手,我必有厚报。”密室中,东海蛊仙劝说道。

慕容尽孝摇了摇头,行了一礼,笑道:“大人,我们慕容家的仙蛊屋缺失了一只核心仙蛊,这是北原众所周知的事情。事关家族大利,请恕在下不敬,也要尽力争夺一番,不管成败,方可甘心啊。”

东海蛊仙幽幽一叹,不再劝说,而是再次加价。

“看来我的希望,也不是没有!”现在这价格已经加到两只仙蛊,远远超过了四个蛊仙俘虏的价值,但方源目光却越发活络。

“赤煞神舟虽好,但没有一只核心仙蛊出卖。三方所出的仙蛊,都是六转蛊,且偏僻难用。两只仙蛊已经是极限,三方报价已经在磨蹭,每一次报价的幅度越来越低。毕竟就算得到这赤煞神舟,包含完整的仙蛊方,也需要耗费巨额进行炼蛊。”

“卖方明显是木道蛊修,要求一切有助于修为精进之物。这六只仙蛊,只有一只是木道,来源慕容家族,却是侦察仙蛊。呵呵,妙哉。我手上的蛊仙俘虏奔雷手吴浩也还罢了,花海三仙却都是木道蛊仙呢。如今这局面,我或许可以借势成计。”

方源雷厉风行,立即通过秦百胜,联系上百足天君。

“如今三方已成僵局,我没有木道仙蛊,目前而言慕容家族希望最大。或许可以借此,来打破僵局呢?”

百足天君不禁心动,主动改变报价,他先将解谜仙蛊撤下,随后再将花海三仙报上去。

卖方眼前一亮,犹豫再三之后,终究选择了百足天君。

他也是聪明人。

对他而言,那两方虽然多出一只仙蛊,其中一方甚至有木道仙蛊一只。但百足天君报价中,一有移动仙蛊,二有木道三仙窍,正适合自己吞并,暴涨修为!

他卖掉赤煞神舟,已经公然得罪了僵盟。此时所虑,无非是保命和发展两项。

适合自己的,才是价值最大的。

百足天君对花海三仙兴趣不大,但对卖方而言,却是急需。

而对于百足天君,左右也失去不了什么,仍旧是付出两只仙蛊的代价。既然已经决定舍弃了解谜仙蛊,卖给谁不是卖呢?这下卖给了方源,反而使得自己打破了僵局,赢得了赤煞神舟。

至于方源,终于得到了解谜仙蛊。

“这是三方的互赢啊,哈哈哈。”他开怀大笑起来。

ps:今天也一更,最近状态实在不佳。身临困窘,心中有结,实在是惭愧至极……

\end{this_body}

