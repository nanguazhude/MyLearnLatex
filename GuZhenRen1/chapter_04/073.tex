\newsection{尸龙}    %第七十三节:尸龙

\begin{this_body}

走肉树乃是传说中的植物,记载于《人祖传》中。

《人祖传》第三章第十六节有曰:

人祖所生第三子——北冥冰魄,因为意外蛊复生,流落在外。遇到古月阴荒,赋予二姐全新的人生意义。

而后,在智慧蛊的指点下,姐弟俩结伴来到蓝海,寻求能复活父亲人祖的生命宝石。

结果古月阴荒察觉到生命的奥秘,主动牺牲了自己,将自己还原成一块残缺的生命宝石。

北冥冰魄怀揣着生命宝石,企图回到父亲的身边,救活人祖。

但他进入生死门却无比困难。

生死门有两条路,一条生路,一条死路。不管哪条路上,都有众多的忧患蛊。

而要克服忧患蛊的干扰,深入生死门,就得需要勇气蛊的帮助。

然后勇气蛊,已经被人祖带进去,此刻在人祖的身上。北冥冰魄没有勇气,无法进入生死门,一时无法,只好使出流浪,寻找解决的办法。

他走过高山,淌过流水,在野兽的追逐下逃命,在寒冷中忍饥挨饿,吃了很多的苦。

北冥冰魄心怀深深的愧疚和焦虑。

他愧疚的是,他赋予古月阴荒的人生意义,结果导致古月阴荒牺牲自己,变成了生命宝石。北冥冰魄等若是害死他二姐姐的凶手。

他焦急的是,虽然有生命宝石,但却无法回到父亲的身边。他担心,时间一长,恐怕父亲就要在落魄谷中魂飞魄散,而古月阴荒也会白白牺牲。

他走着走着。头顶上紫色的天空,慢慢消去,变成了黑色。

原来太古九天——白赤橙黄绿青蓝紫黑,相互轮换,每隔一段时间。就有其中一天覆盖五大域。

而太古时代的天空,和地面是相连的。

天和地相互交接,人可以从地上走上天,也可以从天上走下地。

北冥冰魄漫无目的,不知不觉间,离开了地面。走上了黑天的深处。

黑天中一片黑暗,但不死寂,有大量的生命生活着。

北冥冰魄在黑暗中,找不到回头的路,更加焦急、惶恐和迷茫。

他不辨方向地走啊走。不知跌了多少跤,摔了多少次跟头。忽然间,他看到前方竟然有一团火焰。

这是整个黑天里唯一的亮光。

北冥冰魄大喜,连忙向这团火焰走去。

他走近这团火焰,发现原来是一只蛊。

北冥冰魄很好奇,问道:“你是什么蛊啊,居然能在黑天中发出亮光。你帮帮我吧,我要走出黑天。回到地面上去,我有很重要的事情要去做。”

蛊的语气很虚弱,对北冥冰魄道:“我的名字叫做火。你又是谁啊?”

北冥冰魄就说:“我是人,名字叫做北冥冰魄。你能帮帮我吗?有了你的亮光,我就能看清楚路,回到地上去。”

蛊叹气道:“原来你就是人啊,我听说过一个人的大名,他叫做太日阳莽。”

原来太日阳莽。拥有名声蛊,名声广传天下。几乎没有不知道他的。

北冥冰魄高兴地说:“对对对,太日阳莽就是我的大哥。”

火蛊又道:“我可以帮你。不过在帮你之前。你得先帮我。我太饿了,饿得快要死了。你帮我找点食物回来吧。”

北冥冰魄就问:“火蛊啊,你吃什么呢?这黑漆漆的一片,我又如何才能准确的找得到呢?”

火蛊道:“我是天底下最不挑食的蛊虫之一了,几乎什么都吃。你喂我什么,我就吃什么。”

北冥冰魄四处摸索,拾取了一些树枝,丢给了火蛊。

火蛊吃了之后,立即明亮起来,散发出更多的温暖,个头也长大了,从一个拳头大小,变成脸盆大笑。

它很开心:“能给我再多一点的食物吗?”

北冥冰魄哦了一声,又四下摸索,拾取了一堆石子,丢给火蛊。

火蛊吃了半天,叹气道:“哎,我饿得太狠了,牙口不好,以前啃得动的东西,现在啃不动了。你给我搬一点容易消化的来吧。”

北冥冰魄想了一下:“不如这样吧,火蛊,你跟着我,照亮我的路。路上你看到什么好吃的,我都喂给你,怎么样?”

火蛊答应下来,便缩成一小团,让北冥冰魄托在手中。

就这样,北冥冰魄踏上归程,路上遇到的东西,他都喂给火蛊。

许多次下来,火蛊越来越大。

这一天,北冥冰魄停下来休息,火蛊在他面前吃着树枝,摇曳生姿。

忽然有无数的脚步声传来,北冥冰魄望去,只见好大一群树,在火光的照耀下,影影绰绰,向他这边奔来。

不仅是树,还有无数的野兽和虫群。

把北冥冰魄吓得,立即拿起火蛊就跑。

在他身后的树木、野兽和虫群就追。北冥冰魄跑到哪里,他身后的追兵就追到哪里。

追兵们在大喊:“前面那谁,你别跑了。”

“我们不想害你,只想在火光下逃命。”

“梦境就要追来了,你行行好,让我们借着火光保住小命吧。”

北冥冰魄气喘吁吁,跑不动了,眼看就要被追上来。这时火蛊对他道:“人啊,别害怕,这些追兵就是我的食物,我来帮你。你把我抛过去就可以了。”

北冥冰魄情势所逼,只能听信火蛊的话,将它用力向后抛去。

火蛊首先遇到的,是一片疾奔而来的数量。

这些树,是走肉树。树的枝干,都是肉质的,仿佛章鱼的触脚。

火蛊将这大片树燃烧起来,树干烧成灰烬,树枝烧成一片片的熟肉。阵阵肉香吸引着北冥冰魄,他捡起来大吃一口。喜上眉梢连说好吃。

……

黑楼兰微瞪双眼,一眨不眨地望着断成两截的树干以及触手般的树枝,言语中带着不可思议的语气:“这就是走肉树?”

“如假包换。”方源点点头,目光同样灼热,“太古九天如今只剩下黑天、白天。走肉树是黑天中的生命,要获得比其他天的太古生命要容易得多,至少黑天还健在。七星子是八转蛊仙,已有能力探索黑天,很可能这株走肉树,就是他的探索所得。”

“这棵走肉树。气息澎湃,至少是七转战力!更妙的是,它还是罕见的力道植株。”黑楼兰舔了舔嘴唇,语气兴奋。

方源同样心中喜悦。

走肉树是力道植株,树枝、树干、树叶。都是上佳的炼制力道蛊虫的材料。方源、黑楼兰都是力道蛊仙,对它都有需求。

但当二人正要潜行过去,将走肉树取走时,忽然高空中传来一声雷鸣般的炸响。

这响动史无前例,震动苍穹,一波无形的音浪,卷席周遭。

第八星殿狠狠颤抖了一下,差点散架。大量的星点逸散,接近于半毁状态。

吼——!

一声龙吼随即传来,不逊色之前的炸响。响彻天地。

然后方源、黑楼兰二人就望见石磊以及万象星君二人,飞退出星殿。

从星殿中则飞出一头巨龙。

这龙长达百丈,阴气森森,雄健威武。头上龙角如白银,龙眼半睁不睁,龙鳞片片如灰石。八只龙爪狰狞可怖,磅礴的苍白色尸气如云如雾。缠绕它的全身。

“这难道是……尸龙?”黑楼兰失声道破此龙跟脚。

尸龙却不是太古生物。

而是自宿命蛊被红莲魔尊打坏,天地间魂魄不被生死门吸走。便形成僵尸。

尸龙就是飞天的长龙死后,魂魄不离体,从而尸变,形成的龙僵。

“这头尸龙有八爪,生前应当是太古荒兽,战力高达八转。如今死亡,成为尸龙,战力是七转巅峰,难怪石磊会被打退出来。”方源分析道。

高空中,尸龙咆哮,身躯扭摆,迅速飞向石磊。

龙口一张,吐出一团澎湃的苍白尸火。

龙炎乃是巨龙的本体攻击,天赋战能,就相当于牛的角,熊的掌,蜜蜂的刺一样。

石磊眼中闪过忌惮的光,不敢硬接。他向左飞速移动,刚刚躲过尸火,就看到一大截龙尾充斥他的眼帘。

石磊骤然瞪大双眼,将双臂竖起,护在胸口。

躲不了了!

在方源、万象星君、黑楼兰的视野中,就看到尸龙优雅遨游,转动庞大的身躯,龙尾一摆,正中刚刚闪过尸火的石磊。

砰的一声,石磊如炮弹一般,被狠狠地抽飞下去。

砸在一处半山腰上,深深地锲进山石当中。整个山峰剧烈颤抖,土石翻滚,砸落下来,很快就将他掩埋。

万象星君倒抽一口冷气,连忙飞退不止。尸龙之威竟强悍如斯,他还是首次看到仙猴王石磊如此狼狈。

但旋即,他便听到石堆下石磊的怒吼:“泥鳅,你成功地激怒了我,吃我这一记仙道杀招吧。石破天惊!”

天空乍然破开,一座小山狠狠地压向尸龙。

黑楼兰看得眼皮一跳:“这仙道杀招发动时机竟然如此之快?”

一般而言,威力越强的仙道杀招,需要酝酿的时间就越长。但石磊的这记石破天惊,却是大违常理,不仅威力强,而且发动迅速。

即便是尸龙,都躲闪不及。

但尸龙似乎从未想过闪避,望着砸下来的小山,龙眸中似乎闪过一丝不屑之光。

轰的巨响,尸龙昂首,撞碎小山。

石磊的这记仙道杀招,居然没见丝毫效果。

“好,就是这样的对手才够劲!”石磊又惊又喜,轰的一声冲上天空,一头撞向尸龙。

一人一龙在天空中纠缠,打得风云激荡,天地变色。

方源瞳孔微缩:“这头尸龙大不简单,似乎身躯曾经被仙道防御蛊虫施加过影响。”

“现在该怎么办?”黑楼兰为难道,“在他们的眼皮子底下,我们不可能将走肉树搞到手啊。”(未完待续)

\end{this_body}

