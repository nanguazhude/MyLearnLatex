\newsection{只手拿龙鱼}    %第二百六十二节:只手拿龙鱼

\begin{this_body}



%1
“该死的古魂门,居然为了阻止我们,不惜将荒兽龙鱼惊动起来!”

%2
此时,湖心小岛上,天妒楼的一群人被困住,阻碍了去路。

%3
在他们的眼前,湖水波涛翻滚,掀起滔滔巨浪,水势浩大险恶。

%4
天妒楼五位蛊师,各个眉头拧成疙瘩,望着湖面,心中气愤又无奈。

%5
通过湖水,他们甚至能隐隐看到,在湖水深处,有一个庞大的怪物身躯,疯魔似的游荡。

%6
鱼尾的每一次甩动,都会绞出汹涌的暗流,掀起滔天的巨浪。

%7
甚至有时候,硕大坚固的鱼头,撞击到小岛根部。

%8
轰、轰、轰!

%9
龙鱼每一次撞击,都会爆发出剧烈的轰鸣,众人脚下的小岛随之一次次颤抖。

%10
天妒楼的一行人,被困在小岛上,看着眼前的巨浪,面色发白,束手无策,无可奈何。

%11
荒兽龙鱼本来性情温顺,很少发狂。

%12
但是之前的古魂门的蛊师队伍,动用了手段,让这头荒兽龙鱼发疯发狂,陷入疯魔愤怒的状态,完全失去了理智和本性。

%13
天妒楼的这些蛊师,虽有踩水等等跋山涉水的能力,但到底是凡人,真元有限。在如今这种情况下,简直是天堑横堵根本过不去。

%14
“古魂门居然能影响荒兽龙鱼,他们一定是动用了门派给予的仙道手段!我们天妒楼,同样也是十大古派之一,绝不可能输给古魂门。朱宁长老,我们也动用仙道手段吧!”

%15
正一筹莫展之际。天妒楼的一位蛊师,忍不住开口建议道。

%16
听到仙人手段。其余蛊师纷纷神色一振,双眼放光。

%17
掌握手段的朱宁长老。却是皱着眉头,苦叹一声:“若能动用仙人手段,我早就动手了,哪里会等到现在?”

%18
他接着向诸位述说苦衷。

%19
原来,朱宁此行之前,被师门的太上长老交托了一只仙蛊,还有若干仙元。

%20
可惜这仙蛊,用来防御尚可,若用在此处。要解决掉龙鱼的问题,却不应景。

%21
“师门交给了一只仙蛊?”

%22
“仙蛊到底是什么样子的,我还从未见过呢!”

%23
“快拿出来给我们开开眼界吧。”

%24
朱宁的坦白,引得众人强烈的好奇。一时间他们都忘了眼前的难题,都想要一睹仙蛊的风采。

%25
朱宁再次苦笑:“不瞒诸位,这仙蛊我也只是匆匆一瞥,就被太上长老种在体内。如今就算我,也不知道仙蛊究竟在身体的哪一处,更谈不上取出了。”

%26
众人大为失望。唉声叹气。

%27
唯有魏无伤双眼绽射亮光,向往地道:“若是哪一天我能有属于自己的仙蛊,那就妙了!”

%28
其余四位老蛊师,看着魏无伤的样子。不是摇头,就是苦笑。

%29
魏无伤还太年轻。

%30
年轻人喜欢做梦,是正常的。

%31
但只有经历多了。譬如这四位老蛊师,就会明白现实和梦想的差距。

%32
谁不想拥有自己的仙蛊。谁不想成为高高在上的仙人?

%33
但仙凡之别,犹如天堑。残酷的现实。消损了无数人的青春,打折了无数人的梦。

%34
真正能成仙的,能有几人?

%35
头领朱宁叹息一声:“当下也没有什么好办法,只有等了。等到龙鱼重新平静下来,我们才能启程。”

%36
“可是这样一来,好东西都要被其他九大派提前抢走了。留给我们的,只会是残羹冷炙!”有蛊师担忧地道。

%37
“那还能有什么办法?”朱宁摇头,但也不忘提振士气,“不过此事也并非完全是坏事。就让他们先去争夺激战去吧,我们可以趁机养精蓄锐。仙蛊虽然威力滔天,但需要相应的仙元来催动。仙元有限,就让他们彼此内耗,等到后期,我们保留下来的实力反而更强呢。”

%38
此言一出,众人脸色都稍缓,丧失的心气劲头也恢复了一些。

%39
“不过,有一点我得郑重其事地提醒你们。”朱宁脸色严肃起来,“就算我们保留的实力再强,有两方人我们尽量都不要招惹。一方是灵缘斋的凤金煌,另一方则是仙鹤门的方源。”

%40
“为什么?凤金煌、方源虽然是我这一辈的佼佼者,但他们也只是一个人而已。如今我们十大派都各有五人,各自的仙人手段又神秘莫测,为什么要怕他们两个?”魏无伤不太清楚事情的真相,有些不服气地反问道。

%41
“那是因为你不太了解内幕。”朱宁目光凝重,扫视周围四位蛊师,沉声地道,“我也是出发前,才从门派那里得知的惊人消息。凤金煌出生高贵无比,她的父母双亲都是蛊仙,而且是仙人中极为强悍的存在。你们说,这样的人身上,仙人手段会有多惊人?”

%42
“竟是这样!”魏无伤吃惊不已。

%43
其余人等,亦是流露出震撼的神色。

%44
朱宁再接着道:“然而比起凤金煌而言,那个方源更加可怕!”

%45
“难道他的身世,比凤金煌还要高贵么?”众人惊疑。

%46
朱宁皱起眉头:“具体的情况,我也不太清楚。但这话是门派特意关照我的,方源比凤金煌还要可怕得多,遇到他就避退三舍,千万不要与他争锋。”

%47
“方源虽然是我辈第一天才,但也不至于如此恐怖吧?”魏无伤语气怀疑。

%48
“难道他的身上,有仙鹤门交给他的仙人手段,是我们十派中最强的?”其余蛊师则尝试分析。

%49
朱宁摇头:“我只知道,仙鹤门这一次只派遣了方源一人过来。你们想想看,这意味着什么?”

%50
“看来仙鹤门对方源很有信心,觉得单凭他一人之力,就能对付我们所有人!”

%51
“狂妄!居然如此看不起我们……”

%52
“仙鹤门不是傻瓜。门派也关照我们避让方源,看来是有一定原因的。我觉得还是照门派叮嘱的去做吧。”

%53
“也不一定吧。俗话说。双拳难敌四手。如果方源最强,势必就会引起其余门派联手对抗。结果还不好说呢。”

%54
蛊师们议论纷纷。

%55
你一言。我一语。

%56
有人稳妥保守,有人心怀不忿,有人打起联合其他门派的主意。

%57
就在这时,一道尖锐的音啸声,从上空忽然传来。

%58
众人皱起眉头,魏无伤甚至捂住双耳。

%59
五位蛊师循声抬头望去,便见苍穹之上,有一道身影划破长空而来。

%60
这人影速度极快,直接刺破空气。拉出长长的音啸之声。

%61
一个呼吸之后,就从远处,疾飞过来。

%62
到了小岛上空,身影倏地停下。由极动转为极静,给天妒楼的五位蛊师极为突兀的感觉。

%63
“什么人?居然在如此高空飞行!”

%64
“速度好快,这是人是鬼?”

%65
“观其隐约体貌,好像是……方源?”

%66
众人惊愕震撼,下意识地张开嘴巴。

%67
呼呼的大风,旋即刮来。灌进他们的嘴里,甚至逼得他们双眼闭起。

%68
这风便是方源疾飞而来,冲击空气,形成的大风。

%69
方源凭空而立。目光扫了一扫。

%70
巨浪、小岛、龙鱼、蛊师,周围的情况,都在一瞬间被他掌控。

%71
他旋即将注意力。集中在了龙鱼身上。

%72
小岛上被困的五位蛊师,根本没有令他关注的价值。

%73
大风只是一阵。来得快,去得也快。

%74
五位蛊师连忙睁开双眼。仰望着空中的方源,十分紧张。

%75
“看样子,这龙鱼狂暴,是中了某种魂道或者智道的手段。”方源一边心中猜测,一边照准方向,伸出手掌,缓缓向下虚抓。

%76
轰!

%77
在众人惊骇欲绝的目光中,一道力道巨手,破空而出,势大力沉地砸进湖水当中。

%78
但奇异的是,力道巨手轰入湖水当中,却为造成惊天的浪涛。

%79
反而巨手过处,水流自动分开,仿佛是湖水主动配合方源一样。

%80
这就是力道仙蛊挽澜,融入力道巨手之中的成效了。

%81
湖水没有成为方源捕捉龙鱼的阻力,反而成为了一股助力。

%82
再加上龙鱼狂暴,失去理智,没有逃生的举动,反而让方源更加容易捕捉。

%83
吼!

%84
龙鱼张开嘴巴,发出如龙般的嘶吼。

%85
鱼嘴两旁,长大两丈的修长龙须,如长鞭疯狂抽动。

%86
整个鱼身剧烈挣扎,但力道巨手却稳如泰山,仿佛钢铁浇筑,岿然不动。

%87
在方源的意志下,力道巨手一击即中,将狂暴的龙鱼捕捉上来,提出湖面。

%88
天妒楼的五位蛊师,此时尽皆呆滞,惊骇欲绝地目睹着方源将龙鱼捕捉之后,又装入自家仙窍的整个过程。

%89
收了这荒兽龙鱼,方源还不满足,又将主意打到其他普通龙鱼身上。

%90
龙鱼是群居动物,这湖中到处都是龙鱼的身影。

%91
这一次,方源直接催动挽澜仙蛊,将一团团的湖水凭空抽起,夹裹着无数龙鱼,进入自家仙窍。

%92
天妒楼一行人呆若木鸡。

%93
这样的一幕,相信他们必定终身难忘。

%94
直至方源收刮殆尽,飞离这里好一会儿,这些蛊师才纷纷惊醒。

%95
一个个冷汗淋漓,魏无伤甚至瘫软,直接一屁股坐倒在地上。

%96
他们终于明白,门派叮嘱他们的用意,以及背后的无奈。

%97
“这样的人物,已经绝非我们能够对付的。”

%98
“天呐,我刚刚还在企图联合他人,对付方源?!”

%99
“这样的滔天威势,他是不是已经成仙了?”

%100
众人心惊胆战。

%101
龙鱼已去,他们已经可以启程。

%102
但方才的一幕,实在过于震骇人心,明明没有任何剧烈的运动,只是观看而已,但天妒楼的这群蛊师都在大口喘息着,感到身心上的极度疲累。

%103
(ps:生活充满意外,如何面对命运的不仁慈,恐怕是人生的永恒课题了。有时候自我期望和现实,是不相符的。这就是人生的不如意吧。七月份的更新不敢保证什么了,但总归会被六月份好的,最近这一段时间,我的生活重心都不会在写作方面,我会尽力而为。也会一直坚持下去,绝不会太监,蛊真人也从未太监过任何作品。对于大家的支持,实在是铭感五内,感激不尽啊。)

\end{this_body}

