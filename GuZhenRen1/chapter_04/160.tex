\newsection{三取宝地灾将临}    %第一百六十节:三取宝地灾将临

\begin{this_body}

方源脸上的喜色,尽数流露出来,不加遮掩。<strong>求书网Http://wWw.qiushu.cc/</strong>

“虚化蛊阵,是东方长凡仅凭一些古代文献,自己推算琢磨出来的东西。他的虚道境界,应该有大师级。能够改造太古墟蝠尸体,形成尸山堡垒,宇道境界也有大师级。真是不简单!”

方源对如何控制虚兽大军,尤为感兴趣。

几头荒级虚兽,虽然只是六转,但凭借本身虚化之能,就能纠缠住七转蛊仙。

方源不免就想:若是自己手中有一批如此大军,对战力提升程度必然是巨大的。

细细琢磨之后,他却发现控制虚兽大军的方法,应用范围并不广。

首要条件,就是控制改造太古墟蝠的尸体,形成尸山。然后再以尸山为大本营,用太古墟蝠的身份,才能驱使这些虚兽大军。

方源又是一声冷哼:“这东方长凡还想陷害我!他知道我可以拔山,便想用墟蝠尸山诱惑我去冒险。自从墟蝠尸山被天劫地灾摧毁了大半,威慑百兽的气息也随之散去,那里已经成了众多荒兽、上古荒兽啃噬墟蝠尸体的盛宴。我要去拔山,就是和这些荒兽,上古荒兽抢食,智者不取!”

方源休息片刻,第三次次对东方长凡搜魂。

东方长凡对方源充满了愤怒和仇恨,就是眼前之人,让他一切图谋都化为空,所有王图霸业都消散,所有的努力都化为泡影。宛若梦一场。

他纵然深谋远虑,智计百出,但沦落至此,已经没有可以依仗之物。双方差距太大,仅有的这两条计策。其实十分粗陋,旨在利用人性中的贪婪,引诱对手犯错。

这和东方长凡生前的布置谋算,完全是天地之差。

但东方长凡只剩下魂魄,大失水准也是可以理解。

方源没有中计,休息妥当之后,第三次的搜魂。终于让他搜出了他最想得到的东西东方长凡的智道传承!

叫方源意想不到的是。东方长凡的这份智道传承,竟然和星宿仙尊有些瓜葛。

当然,它不是星宿仙尊本人留下的传承。

若是仙尊本人而留,东方长凡必然不至于落到如此下场。

原来留下这道传承的,乃是古代中洲天庭的一位八转星道蛊仙。他善于寻幽探秘,得到一些星宿仙尊的秘密,揣摩星宿仙尊的本领。[棉花糖小说网Mianhuatang.cc更新快,网站页面清爽,广告少,无弹窗,最喜欢这种网站了,一定要好评]模仿而得。

这位蛊仙,本是星道蛊仙,但揣摩的却是智道。因而这道智道传承,基石就是星念蛊,由星入智,和星道十分贴近。甚至就连星念蛊的炼制,都是以星光蛊为主材。

方源越深入了解,心中的惊喜便越大一分,深感自己此行冒险物超所值!

原来这道智道传承,没有别的长处。却最擅长推算推演。

东方长凡正是凭此,推演出虚化大阵,控制虚兽大军等种种手段。

“这智道传承真是全面,我都有一种改修智道的冲动了。”方源叹息。

这笔收获,实在太丰厚了。

一道完整的传承,不管是哪个流派,必然要涉及攻防、医疗、转移、侦察等各个方面。

这些方面。也许不会太过突出,但总归是没有短板的。不会被人轻易地克制。

方源重生至今,修行的力道,都是东拼西凑而来,或者是自我改良,不是先贤的传承。

这道传承的实际价值,和方源五百年前世得到的血道真传,是一样的。

方源前世,正是因为血道真传,才有了平台基石,得以崛起。

这份智道传承的价值,可想而知。

甚至因为血海老祖还不是八转蛊仙,又分七道传承,所以方源手中的这份传承,价值还要更大。

“可惜啊,我得到这份传承的时机不对,迟了一点。我已经走上了力道的路子,就算还有第一空窍,却沦为仙僵,空窍已死,无能进修。”

不过方源转念又一想。

若他真的提前修行了这份智道传承,未必能创造出仙道杀招万我。

若算上仙道杀招万我的话,自己的力道前景之广阔,无疑胜过这份智道传承带给自己的未来。

“而且正是因为我有了力道战力,才能杀得东方长凡,夺取这份智道传承。没有前因,哪里来的后果呢?”

方源摇摇头,将心中的杂乱思绪排除出去。

这份智道传承,来的太是时候了。甚至说,极其适合方源。

为什么这么说呢?

原因只有一个,那便是九转智慧蛊。

狐仙福地中,藏着智慧蛊。方源沐浴在智慧光晕中,灵感无限,推算出许多仙蛊方、仙道杀招等等。

但此举有一个最大的弊端,那就是脑海中的念头,消耗太快!

以前有乐山乐水仙蛊的时候,消耗珍贵的青提仙元,换来海量乐意,支撑消耗。拍卖大会之后,乐山乐水仙蛊被卖出去,虽然赚了一笔,但恶念蛊、忆念蛊都经不起剧烈的消耗。

就算是方源专门打造了一座石巢,养了大量的毛民,几乎不眠不休的赶制炼蛊,以最快的速度囤积,也抵不上一次消耗。

这是制约方源推算东西的最大关卡。

现在好了,有了这份智道传承,方源就可以用更少的代价,更高的效率,去推算种种事物。

“地灵何在?”方源轻轻跺脚。

“主人,人家在这里。”地灵小狐仙嗖的一下,陡然出现在方源的面前。

她的小脸蛋红扑扑,嫩嘟嘟的,雪白的狐尾毛茸茸,得到方源的召唤,尾巴在后面开心晃着。

方源沉吟道:“即刻停下第二石巢中关于恶念蛊的炼制,改炼星念蛊。这是蛊方,你去安排一下。”

“明白,主人。”

“我带来的各类资源,你都已经安排妥当了吗?”方源又问。

他一回到狐仙福地,就将仙窍中的这些资源,都倒腾出来,布置在狐仙福地中,具体工作则让地灵小狐仙去落实。

这是因为他的仙窍中,充斥着死气,暂时储存还好,长久摆放就会死气侵染,导致大部分的资源价值大幅度下降。

“已经统统弄好了,主人。气泡鱼群放置在福地东面,和之前的气泡鱼汇在一起。鱼群数量暴涨,星萤蛊就在鱼群的上面,接下来的时间里,一定产量激增!”

“至于那近百万的龙鱼群,则暂时摆放在其他的小湖中。过一段时间,就调遣石人挖开一座大湖,让龙鱼群聚集在一起,这样就能繁衍出更多的小鱼啦。”

“散文鲤,人家专门挖了一个小潭,隐秘存放着。”

“幽火龙蟒剔除死去的部分,还有一千多条,现在已经和咱家原有的那几条,合并在一起,占据了福地西南方向好大一份地盘呢。”

“哦,还有油水,以及一些其他的常规资源,暂时摆放在荡魂行宫的储藏室里了。长恨蜘蛛群,则放在福地北面,那里地形狭长,人家费了好大劲,才安置好的呢。”

小狐仙一副翘首以盼,邀功的神色。

方源笑了笑,连忙夸奖道:“啊,我们家的小狐仙最能干了。”

小狐仙得到夸奖,圆溜溜的大眼睛立即笑得弯成月牙状,雪白的大尾巴迅速晃动着。一时间,得意忘形,轻轻一蹦,就蹦到方源的脚边,一下抱住方源粗壮的小腿。

她用滑嫩的脸蛋紧挨着方源的小腿,随后又仰起头,一双大眼睛水汪汪地往上瞧着方源,满脸都是兴奋的红晕,崇拜地道:“主人,你好厉害啊,这一次带来这么多的好东西。我们可以赚大钱了,狐仙福地从未有过这么富裕的时刻呢!”

方源哈哈大笑,伸出一只怪臂,像拎小猫似的,将小狐仙拎起来,放在自己宽厚的肩膀上:“放心吧,好日子还在后头。”

小狐仙点头如捣蒜,对方源的本事十分相信,不过又有些担忧地提醒道:“主人,主人,我得汇报给你,你这次带回来的那头荒兽鱼翅狼,并不太好,总是躲在水潭里面,不主动捕食,有时候还嘤嘤的哭泣。”

方源嗯了一声,不以为意:“没有关系,这是因为它受到的打击有点大。我有喂养狼群的经验,过段时间它就会渐渐回复了。”

“主人,还有一件事情,我得提醒你。你可千万别忘了,咱们家就快要有地灾降临了。”小狐仙又道。

方源微微皱起眉头:“你放心,我一直记得,这种事情我怎么敢忘记呢?”

狐仙福地和外界的时间流速比,为一比五。十年一次地灾,而外界的中洲、北原已经过去了一年多。

外界的两年时间,就是狐仙福地的十年。

上一次地灾,是荒兽泥沼蟹,携带着仙蛊和稀泥。这一次地灾,又会是什么呢?

“不过比起上一次,我只是凡人蛊师。这一次地灾,我却已经是六转蛊仙,身怀许多仙蛊,有仙道杀招,更有盟友,又新得了珍贵的智道传承。要想渡过地灾,把握是很大的。”方源并没有太担心,他的底牌有很多,身家十分厚实,比起上一次地灾简直天壤之别,所以相当有自信。(想知道《蛊真人》更多精彩动态吗?现在就开启微信,点击右上方“+”号,选择添加朋友中添加公众号,搜索“zhongenang”,关注公众号,再也不会错过每次更新!read2002)(未完待续。)<!--80txt.com-ouoou-->

\end{this_body}

