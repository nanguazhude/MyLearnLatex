\newsection{给你青浦茶}    %第八十一节:给你青浦茶

\begin{this_body}

中洲,白晴福地。

明媚的阳光,照映这片小天地。

山溪汩汩而流,参天古木,绿树成荫。微风吹拂下,阳光透过树叶缝隙洒下的碎屑光斑,也随之摇曳生姿。

凤金煌直接躺在绿草地上,背靠着大树,随着清风,她深呼吸一口气,清新的空气中夹杂着一丝好闻的青草香气。

一本《人祖传》,装裱精美,此时静静地躺在凤金煌的手中,已是翻到了最后几页。

凤金煌明眸如水,盯着书页,一眨不眨。

虽然有关人祖的故事,她从小就听说过,但现在因为有了梦境中的奇妙经历,再带着探究的心思去复读《人祖传》时,凤金煌又有了许多不同的全新感受。

《人祖传》的故事最后写道人祖十子相继死亡,就连人祖自己也要老死,即将走完了他的人生路。在临死之前,人祖取来十子尸体,又牺牲自己,一起投身衍化蛊,被它吃下肚子里去。

大量的梦境,也循着食物的气味,紧紧跟随,钻进了衍化蛊的嘴里。

衍化蛊撑破肚皮,爆炸开来,无数的生命之光落于大地,形成了第一批的凡人。

“因为有梦境参与衍化,所以新生的凡人,都会在夜晚熟睡的过程中做梦。人们常常沉浸于梦中,像北冥冰魄一样,无法自拔。”凤金煌一边思索,一边伸出白皙的柔荑之手,将《人祖传》轻轻合上。随后站起身来。

“娘亲,送我回小楼吧,我有些累了。”她轻声呼唤道。

“好。”下一刻。凤金煌的耳畔就传来白晴仙子温柔的回应。

凤金煌陡然消失在原地,再出现时,已是在锦绣小楼中的自家闺房。

闺房中,白晴仙子已在等候。

凤金煌不由双眼一亮,欣喜地道:“咦,娘亲,你今天怎么有空。怎么没有去调和地气呀?”

蛊师到达六转,成为蛊仙,空窍汲取天地人三气。升华为仙窍,便是一方小天地。

蛊仙经营仙窍,增长修为。但仙窍天地中承载的东西越多,渐渐的。仙窍中的天地之气就不够用了。

就好像是。悉心照料的盆栽,随着盆中的树木长高长大,原本的盆子就显小了,土壤就嫌贫瘠了。

这个时候,蛊仙就要将福地落到地上,汲取地气,或者将洞天种上天空,汲取天气。

天地二气汲取进来。将稳固福地洞天,增加仙窍底蕴。修为稳定下来。才能更快更稳地继续修行下去。

白晴仙子目前的情况,就是经营了福地一段时间,已经不太稳固。她将福地种在地上,自身便居于其中,每天都在汲取并调和中洲大域的地气。

白晴仙子用慈爱的目光,看着可爱的女儿,夸奖道:“煌儿,看来你这一次梦境遭遇,大大磨砺了你的心境。若是以往,你早就吵着闹着,要我陪你了。但这一次,我调和地气三天三夜都没陪你,你就静静看书,默默炼蛊,心性比以前的确成熟了不少,娘我很欣慰。”

“那是!”凤金煌展开笑颜,露出洁白贝齿,她头微微一扬,头顶上的凤冠叮铃作响,她骄傲地道,“我是谁呀?我可是爹和娘的亲生骨肉,怎么可以给你们二老丢脸呢!”

她的笑容感染了白晴仙子,后者也笑起来:“自从第一批凡人诞生,便可做梦。几乎没有人不做梦的,多少年来,无数的天才鬼才,甚至仙尊魔尊,都有自己的梦境。这些梦境凝结一体,形成史无前例的庞大秘境。煌儿,你的梦翼仙蛊,正是可以自由进入梦境,而不被梦境困扰的移动仙蛊!这是你的绝大机缘,你一定要好好把握。”

“我晓得了,娘。这些天来,我一直在尝试炼蛊。我发现自己的炼道境界,已经达到了准大师的程度。梦境的好处真的太大了,我对炼道一直兴趣缺缺,平常也没有练习过。但经过梦境一游,我的境界竟然突飞猛进到这种程度,把门派中的好些人,都甩在了身后去了。”凤金煌很是感慨地道。

“你在空绝老仙的梦中一游,无意中获取了对方的些许炼道真意。不过这也是空绝老仙本来就是炼道大宗师,所以让你占了一个便宜。梦境繁多芜杂,你下次未必能如此好运,遇到空绝老仙这等人物的梦境了。”白晴仙子分析道。

凤金煌却是不以为意,嬉笑着道:“遇不到,就遇不到呗。人家本来就对炼道没有兴趣,下次遇到仙尊、魔尊的梦境,那就可以了。唉,不过可惜,我走的是金道,历代十大尊者却没有修行这个流派的。”

白晴仙子心头微微一跳,肃容道:“煌儿,切勿掉以轻心。你这次有惊无险,是你运气好,遇到的梦境并不险恶。你要真的遇到尊者之梦,第一时间就要退出去。尊者的力量,不是你能想象的。”

凤金煌吐了吐舌头,连忙乖巧地道:“是,娘,孩儿知道了。”

白晴仙子神色缓了缓,掏出一只信蛊:“这是你爹寄来的家书,他已经成功地突破界壁,进入北原后,找到了一处栖息之所。他现在还不知道你已经苏醒的事情呢。不过既然他已经来信,我们就用这只信蛊回信过去,你也说几句话,好让你爹放下心事,全力应付这次深入北原的任务。”

“爹的来信?”凤金煌双眼骤亮,小跳过去,将白晴仙子手中的信蛊夺在手中,探入心神查看。

结果,却看不了。

凤金煌嘟起嘴:“原来是爹的那只信道仙蛊报信青鸟,我还是凡人,看不了啊。”

白晴仙子笑着摸摸她的脑袋:“你爹此刻人在北原。不用信道仙蛊,怎么可以穿透界壁呢?你着什么急,刚刚才夸奖过你稳重了一点呢。”

说着。白晴仙子便向凤金煌的脑海中,灌注了一股意志。

凤金煌驾轻就熟,脑海中念头泛起,和意志交汇,转瞬间就“看”到了凤九歌的家书内容。

“爹……”凤金煌看到凤九歌在信中,对自己的问询、在意,不由感动得双眼泛红。

“真想立即见到爹爹。让他看一看我如今的炼道修为。他可是一直都对我的炼道境界耿耿于怀呢。”凤金煌口中喃喃。

“放心,娘会在信中告诉你爹的。”

“不要说,不要说。我要给爹一个惊喜。”

“也好。”白晴仙子接着,又取出一只信蛊,仍旧是报信青鸟蛊。但却是一只五转凡蛊。

“这个也是给你的信,来自方源。早先就送过来了。你先看看罢。”白晴仙子道。

“方源!”凤金煌听到这个名字。立即恨得咬牙切齿,剑眉倒竖,绝美的脸上浮现出恼恨的羞红。

她一把夺来信蛊,捏在手中查看。

方源在信中书写的内容不少,开头先恭维了灵缘斋很多话,言语间似乎暗示有投靠迹象,随后答应下凤金煌的挑战,并为这么晚才回信而致歉。

信中的末尾。是这么一句话在下现身居狐仙福地,敞开大门。恭临凤金煌仙子移驾。又久闻灵缘斋青浦茶爽口之名。若是切磋之余,还能得到品茗之乐,自然不胜欢喜。

“哼!他很有闲情雅致,还想喝青浦茶?还想得到品茗之乐?恶贼!淫贼!我恨不得把你大卸八块,拆了骨头去喂狗!”凤金煌双眼似在喷火,手上不禁用力,将报信青鸟蛊捏得嘎吱地响。

白晴仙子连忙将这只信蛊救下:“煌儿,这也是接下来娘亲要给你详细说的。不得不说,方源再一次出乎我们的意料,他已经今非昔比了。你要向他切磋,只有败,没有胜。”

“什么?”凤金煌瞪大双眼,诧异地看向白晴仙子。

白晴仙子便将方源已经成为仙僵,并且背后势力神秘雄厚,甚至打退了仙鹤门回收狐仙福地的攻击。

“仙僵?他居然成了仙僵?”凤金煌喃喃自语,若非说这话的是自己的亲娘,恐怕她怎么也不会相信。

“这才多长的时间,他居然已经升仙了?他和我的年龄差不多大呀,怎么可能有如此深厚的底蕴,冲刺蛊仙境界的呢?”凤金煌极为疑惑。

白晴仙子答道:“关于他忽然成为仙僵,我们十大古派也有许多推测。提升底蕴,虽然困难,但并非毫无办法。眼前的例子就有一个,那就是梦境。除此之外,我们猜测他背后的势力,恐怕拥有一种特别的升仙法门,能够将蛊师晋升成仙僵。”

“但是成为仙僵,修为停滞,就再也不能继续加深了。”凤金煌皱眉道。

“的确是这样。但你想想,一个五转凡人蛊师,和六转仙僵比较起来,当然是后者战力更强盛得多了。”白晴仙子道。

凤金煌脸上浮现出不屑的神情:“不管他是急功急利,还是被迫如此选择,他在将来绝对不会是我的对手。他现在急匆匆地来挑战我,恐怕就是想趁着这段有限的时间,多欺负我一下,等到将来就该轮到我欺负他了。这个胆小如鼠,卑鄙无耻的家伙!”

“煌儿,不要被你心中的仇恨怒吼蒙蔽了眼睛。”白晴仙子神情变得严肃起来,“方源此子并不简单,这份挑战信是写在仙鹤门进攻之前,真正的意图是想和我灵缘斋搭上关系,是驱虎逐狼之计。他现在坐拥荡魂山,和我们十大古派都要贸易往来。名传中洲,几乎每个蛊仙都知道他的存在。”

“娘,我错了,的确有点激动。”凤金煌咬了咬牙,努力平静。

“现在这封信你该如何回应呢?置之不理,或是回信,都由你来处理。”白晴仙子故意考验道。

凤金煌不假思索地道:“当然要回信过去!既然这份挑战信是我下的,他毕竟也回应过来了。若是我因为他仙僵的身份,避而不战,倒显得我软弱了。”

“不过……”她眼珠子一转,“敌强我弱,傻子才会和他硬碰硬呢。哼,正巧不久后,咱们中洲的炼道大会就要开始。我就用炼道,在众目睽睽之下,将他击败!先让我出一口恶气再说。娘亲,你看我这样处理行么?”

白晴仙子满意地点点头:“此举甚好。”

凤金煌咬着虎牙,眼中神芒绽射:“接下来的这些天里,我就用梦翼仙蛊,专找炼道蛊师的梦,迅速积累炼道底蕴。我正好利用他这个对手,来磨砺我自己,促进我的修为。”

白晴仙子笑起来,摸摸凤金煌的脑袋:“这才是我的好女儿。”

凤金煌眯起双眼,一边回信,一边冷笑着:“你想要附庸风雅,想喝青浦茶,可以,我直接送给你五六十罐!但是你要和我切磋,没有彩头可不行。我要让你输了面子,又输里子!”(未完待续……)

\end{this_body}

