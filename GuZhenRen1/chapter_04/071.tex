\newsection{捡起一头又一头,大发横财好运来}    %第七十一节:捡起一头又一头,大发横财好运来

\begin{this_body}

黑楼兰晋升成仙之后,就受到黎山仙子的教导,手把手的指点,包括各个方面,可以说毫无保留。

因此黑楼兰在侦察方面的手段,早已经不属于寻常蛊仙。

“但方源的底蕴之深,显然已经大大领先于当今大多数的蛊仙。我虽然有仙蛊数只,但我能像方源这样,独立炼制新蛊,自创改良杀招吗?”

黑楼兰在心中暗自摇头。

她的力道杀招,是她娘给她的。

而炼制新蛊,就要开创新蛊方。这个过程,不仅需要资源,充分的底蕴境界,而且还需要灵感和运气!

黑楼兰是蛊仙不假,但到底太年轻,并不是炼道大师,甚至并不擅长炼道。而这些底蕴之类的东西,都需要长久的时间和操作实践的慢慢积累。

这时方源大踏步地走了过来,他沉吟道:“总结所有线索,我差不多知道对方的身份了。”

“哦?”

“一个是星道六转蛊仙,从未出手过,比较神秘。另外一个却是土道兼修变化道,气息张扬,通过蛊虫还原了他的一丝声音,十有六七会是中洲战仙宗的石磊。”方源道。

“是那个仙猴王?”黑楼兰和方源之间,大部分的情报都共享了,她自然清楚石磊是何等人物。

她微微皱起眉头,不禁感到有些麻烦。

石磊乃是七转蛊仙,战力脱俗得很。黑楼兰纵然有数只力道仙蛊,但石磊也有土道仙蛊。而且早就有了,甚至创造出了独属的仙道杀招。不仅如此,他还兼修变化道。战斗才情可谓万中无一。

黑楼兰拥有仙蛊我力,用它催动之前的凡道杀招,形成力道虚影巨人,勉勉强强算得上一手仙道杀招。

一直以来,黑楼兰都在试图改进完善,但一直进展缓慢。

“如果方源你动用万我杀招,和我联手。说不定能抵住仙猴王。但那个神秘的星道蛊仙,却无人对付。”黑楼兰沉吟道。

方源苦笑一声:“我还未告诉你,因为核心仙蛊饥饿。万我杀招已经不能动用了。我所能用的,只有凡道杀招。”

黑楼兰深深地看了一眼方源,而后双眼眯起:“这么说来,我们应当尽量避免和对方硬抗了。”

吼吼、吼——!

高空中。传来一声声的嘶吼咆哮。

方源仰头望了一眼:“两强相争必有伤亡。咱们见机行事就行。我们的战力虽然比他们要低,但若有机会,也未必不能斩杀了他们两个。就算事有不济,我手中还有定仙游,硬抗他们几下,争取到定仙游发动的时间,还是可以的。”

黑楼兰立即赞同地点点头,方源的话大和她的胃口。

两人决定。先将眼前的便宜吞下肚中。

绝影豹的尸体,乃是一笔可观的财富。虽然这周围。有大量的侦察蛊虫,但却难不倒方源这个老魔。

方源有着领先半个时代的小手段,依次动手,或是封印,或是哄骗。片刻之后,在不引起万象星君注意的前提下,将绝影豹的尸体搞到了手中。

“这绝影豹的皮毛,可以炼制多种防御蛊,骨骼可以炼成附影蛊、影踪蛊等,眼珠子则是暗视蛊、黑雾蛊等等的上佳材料。可惜了,它的血液流逝殆尽,否则要更加值钱。”黑楼兰将绝影豹的尸体,塞进自己的仙窍,眼中散发着欣喜的光。

“对方似乎是故意放血,看来高空中的那处星殿,的确是重要之地,让他们都暂且将荒兽尸体都放置一边不管了。”方源目光中流露出思索的意味。

他的仙窍中充斥死气,不利于保持尸体,因此干脆将绝影豹的尸体放到黑楼兰手中。

两人都是胆大,且又无法无天之辈。

纵然对手强大,也敢继续谋算。

接下来,他们俩个按照之前的计划,偷偷潜行,仍旧想尽量接近高空中的第八星殿。

但在途中,他们很快又有新的发现。

“前方的侦查蛊又明显多起来了,难道说……”黑楼兰目光中流露出期待之情。

不一会儿,两人避过所有的侦察蛊虫,穿过一个山隘,来到一处山谷中。

山谷里,躺着一具小山的荒兽尸体。

它形如马驹,但体型巨大,四蹄粗大,漆黑如铁,浑身一片赤红,肌肉结实,身躯雄健。

它头颈上的鬃毛和马尾,都十分茂盛,呈现金黄之色。

一股股血液,从它的身上一个个圆洞般的伤口中,流淌出来。血液滚烫,灼热的气息将附近的绿草树木,都熏得枯黄。

方源、黑楼兰渐渐走近,便越加感到一股夏日般的炎热。

“这是荒兽赤莬神驹,身上寄生大量炎道野蛊,能踏火而行,血液炙热,是上佳的炎道蛊虫的炼制材料,在宝黄天中都很稀罕。”

黑楼兰看着马血白白流淌,不由大为心疼。

二人忙活一阵,再次将赤莬神驹的尸体搞到手中,万象星君却仍旧被蒙在鼓里。

离开这处山谷,两人潜行一阵,方源放缓脚步。

“我们的目的地,在正前方。但是左手边,却似乎有大量的侦察蛊,要不要去看看?”方源看向黑楼兰,传念道。

“当然。”黑楼兰的回答十分干脆。

二人来到被大量侦察蛊包围的地点,果然又见到一只荒兽尸体。

这是一头黄玉狮子。

它体型如象,躺在地上,眼帘几乎阖上,只留下一丝缝隙。大量的黄色血液,顺着它肚腹上的伤口,流淌出来。

方源和黑楼兰悄悄接近,这头黄玉狮子竟然雄躯微微颤抖起来,奋力撑开眼皮,血口缓缓张开,发出微不足道的吼声,似乎还要再战的样子。

“黄玉狮子果然性情悍勇。能够将这头黄玉狮子伤成这样,看来石磊的变化道战力,比我料想的还更要出色。”黑楼兰赞叹道。

方源则是大喜过望。

他连忙出手,在众多侦查蛊环伺的情况下,将黄玉狮子偷过来。

他一拍手,大量的治疗蛊虫从仙窍中飞出,落到黄玉狮子身上,迅速稳定了它的伤势,吊住了它最后一口气。

“你打算救它?”黑楼兰微表诧异。

方源点点头:“把它救活,比我们动手活捉一头荒兽要容易上百倍,何乐而不为呢?这头荒兽也先放你那里,小心保护,算在我的那份里。”

这样一来,只要事情进展顺利,方源答应琅琊地灵的任务,也算完成了。

黑楼兰将黄玉狮子收入仙窍,目光闪了闪,像是下定了决心,开口道:“我们碰到的三头荒兽,都是被故意放血。很可能这就是石磊一方,必须要完成的某个步骤。或许高空中的星殿,就是因此显现出来。毕竟我们刚刚在青星空间中,天空中还是空无一物的。你猜除了这三处之外,还会有多少荒兽尸体呢?”

方源嘿然一笑:“你就算不提,我也想说了。不错,按照这种情势来看,荒兽尸体不在少数。一份荒兽尸体若卖得好,能得四十块左右的仙元石。我们捡取这些荒兽尸体,根本就是不劳而获。收益巨大,成本若无。双鸟在林不如一鸟在手,更何况我们还未必能等到机会。还是先将这些肥肉一一吞进嘴里,较为妥当。”

两人不谋而合,立即改变策略,开始有意搜索荒兽尸体。

他们不能大张旗鼓地运用侦察蛊虫,但线索却也相当明显。

万象星君原来打算,先留着这些荒兽尸体,等攻克第八星殿,若有时间再来处理。

这些荒兽气息残留,虎倒威犹在,一时间也不担心被其他野兽啃噬。

石磊负责战斗,万象星君就负责布置侦察蛊虫。他当然在荒兽尸体附近,会布置更多的侦察蛊。

但他怎么也不会想到,这些侦察蛊虫较多的聚集在一起,就是最明显的线索,是方源、黑楼兰二人的指路明灯。

“这是板栗牦牛啊,肉质如板栗,完全可以生吃,性情温和,是一种奇特的荒兽。嗯,收起来,收起来。”黑楼兰欣喜笑道。

“哎哟,有一头黄金穿山甲。可惜皮甲残破得很,不然能卖出更高的价格。”方源有些可惜地道。

“飞熊,居然有一头飞熊的尸体。好,太好了。这具尸体我要了!”黑楼兰喜出望外。

“嗯?星魔蝠!我就说这片腐烂的毒气沼泽有些眼熟啊。可惜这对蝠翅被撕烂了许多,这种情况移植的话,会有些影响移动速度的。不知道能不能先修复起来。”方源将星魔蝠的尸体也偷盗手中。

他有些遗憾,不过很快释然。

毕竟不是方源自己出手,石磊哪会特意照顾方源的计划。自己这样不劳而获,已经是捡了大便宜。

这样的好运道,真是许久都没碰到过了。

二人总共收获了近十头荒兽尸体,大发横财。之后二人又详加搜索,确定没有任何荒兽尸体遗留,这才赶到第八星殿的正下方去。

哪知刚刚赶到,就有一头小山般的巨影,重重地坠落下来。

轰的一声,砸在地上,山石飞裂崩解,烟尘四起。

“是上古荒兽气罡飞天猪!”

“它死了!这可是上古荒兽的尸体。”

方源和黑楼兰迅速对视一眼,两人神念交流,均是怦然心动!

\end{this_body}

