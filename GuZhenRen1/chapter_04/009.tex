\newsection{破局难黑楼兰}    %第九节:破局难黑楼兰

\begin{this_body}

%1
。

%2
地底洞穴,智慧之光明晦不定。

%3
方源身处其中,静默而立。一直等到脑海中的意志消耗将尽,他这才睁开双眼,退出智慧光晕。

%4
方才的推算,已经有了成果。但方源却也因此,心中沉重。

%5
“要封印春秋蝉,所需凡蛊多达六千多只,其中五转蛊虫就将近一半,珍稀蛊虫多达两千,还涉及大约八百多只的古代蛊虫。这些古代蛊虫,有些濒临灭绝,有些或许已经灭绝。初步估算,若真的采取此法,需要仙元石二十三颗左右。”

%6
方源穷得叮当响,二十三颗仙元石对现在的他而言,是一个难以企及的数字。

%7
但方源必须硬着头皮,去达成此事。

%8
皆因,春秋蝉恢复速度越来越快,对第一空窍施压日趋严重。

%9
幸好方源转为僵尸之躯,第一空窍成为死窍,才能支撑到现在。若是先前的空窍,恐怕早就被撑破了。

%10
“按照现在的情况估计,我还有两个多月的时间,可以筹划。超过这个时限,就算是第一死窍也要被撑坏。”

%11
方源心中颇为沉重。

%12
他现在是活死人,第一空窍就算毁掉,他也不会死。但失去第一空窍,无疑大损他未来的修行潜力,这是难以估量的惨重损失。

%13
方源处境艰难,要在短时间内,赚取二十三颗仙元石,唯一可行的方法,就是售卖仙蛊方。

%14
回到荡魂行宫,方源便召见了太白云生。

%15
“这些天,我时刻不停地关注宝黄天,几乎不眠不休。但琅琊地灵的神念,至始至终都没有出现过。”

%16
太白云生的答案令方源失望。

%17
琅琊地灵催用神念蛊,在宝黄天中自称“琅琊老仙”,在宝黄天中相当资深和活跃。如今却一直没有露面,有点音信全无的意味。

%18
方源敏锐地感觉到,琅琊地灵那边可能出现了问题。他想到前世,琅琊福地先后遭受过七波攻势。

%19
“我之前在北原,已经有两波强敌攻进琅琊福地。难道说,第三波攻势已经发生,使得琅琊地灵无暇炼蛊,疲于应对吗?”

%20
关于这点,前世的记忆也帮助不了方源。

%21
他只记得有七波攻势,并不没有记住这七波攻势发生的详细时间。

%22
就算记得详细时间,这个世界已经被他影响、改变得十分巨大,说不定层层影响后,也会让攻势提前。

%23
“这些天,我可算是大开眼界了……一块万载玄冰,就要半块仙元石!一斤飓风山椒,售价一块仙元石。璎珞神珠是什么玩意?居然十颗便能要价半块仙元石。还有那个什么螺纹白丝铁更加夸张,小拇指那么点大,居然就要三块仙元石!天呐,我彻底发现,我真的是个穷光蛋呐。”

%24
太白云生谈到这些天的收获,嘴上喋喋不休,大发感慨。

%25
这些天,方源和他朝夕相处,发现他渐渐暴漏出了一个毛病,那就是嘴太穷,罗里吧嗦。

%26
方源递给他一只信蛊,打断了他的唠叨:“这是仙鹤门的来信,你看看。”

%27
太白云生投入心神,眉头拧成一个疙瘩。

%28
仙鹤门的再次来信,语气更加严苛猛烈,攻击狐仙福地的意图已然分外明显。仙鹤门甚至以胜利者的口吻,高高在上,要求方源投降。

%29
“这些家伙,看来是想荡魂山想得疯了。若换做我,至少得先稳住你,暗中实施进攻计划。他们这样做,就不怕我们事先有所防范吗?”太白云生神情不悦地道。

%30
“这就是仙鹤门的强势和底气吧,他们可是中洲十大古派之一,底蕴深厚无比,矗立在飞鹤山已经有数十万年的时间。除此之外,我想他们也有逼迫我,以及试探我的背景的心思。”方源分析道。

%31
“是啊,仙鹤门比北原的任一超级势力,还要强盛三分。我们势单力孤,他们就是一个庞然大物,越加咄咄逼人。在他们面前,我们要保住狐仙福地,恐怕很难……”太白云生斟酌着措辞,小心翼翼地看了方源几眼。

%32
他已有退意,想放弃狐仙福地远走,只是没有明说。

%33
方源自然知道他的心思,太白云生是治疗蛊师,性格仁慈亦可以说是软弱,严重缺乏抗争精神。

%34
方源没有任何责怪太白云生的意思。

%35
事实上,方源也曾想过主动撤离,放弃狐仙福地的可能。

%36
毕竟荡魂山代表的利益太大,仙鹤门等诸多超级势力,必然得之而后快。

%37
但那是万不得已的时候。

%38
只要还有希望,只要还有利益可言,方源就绝不会轻言放弃。

%39
所以方源安慰太白云生道:“仙鹤门虽是中洲十大古派之一,超级势力、庞然大物,但他们也有他们的麻烦。他们能腾出多少手段,专门来对付我们?我敢担保,绝对没有你想象的那么夸张,尤其是在他们大大低估我们真正战力的情况之下。”

%40
生活不易,仙鹤门也有仙鹤门的难处。

%41
当今中洲,新派层出不穷,不断带给十大古派巨大的冲击。

%42
十大古派占据中洲最主要,最精华的修行资源,自然要承受来自四面八方的新锐冲击。

%43
历史上,仙鹤门有过好几次鼎盛时期,但今时今日显然和鼎盛时期,差距甚远。

%44
仙鹤门已经有很多年没有扩张过了——这就说明,他们也有他们的桎梏。

%45
“既然师弟你执意和仙鹤门打一场硬仗,那我就只有舍命陪君子了。你放心,不管情况多么艰险,我都会支持你到最后一刻的。”太白云生拍着胸脯,起誓道。

%46
方源点点头:“也请老白你放心,我古月方源也不是偏执之人。事不可为,见机不妙,我一定会主动撤退。打退仙鹤门的第一波攻势,也许并不困难。但关键的难点,不在于此处。仙鹤门家大业大,咱们打退他们,很快他们就会再度来攻。这样的次数越多,仙鹤门的攻势就越强大,就越重视我们。要解决这个难题,还得依凭大势,合纵连横,借助其他古派之势。”

%47
太白云生听得连连点头,他十分赞同方源的话:“不过,师弟你早就给灵缘斋去信,但他们至今都未有回音。我们的信蛊是不是被仙鹤门截了,要不要再多发几封过去?”

%48
方源摇摇头,他联系灵缘斋的信蛊,并非出自他手,而是凤九歌所创的五转报信青鸟蛊。

%49
当初,方源击败凤金煌,得到狐仙福地。凤金煌不忿,遣发了挑战信,就是用的这只报信青鸟蛊。

%50
但方源没有回应挑战一事,而是直接将这只信蛊扣押下来。

%51
他曾推测,这只信蛊的主人极有可能就是凤九歌。皆因当时的凤金煌,修为还不到五转。

%52
除了凤九歌之外,还有可能是他的妻子,凤金煌的母亲——白晴仙子。

%53
不管主人是谁,这只信蛊早已经借给了方源,原本的用意是让方源回应挑战信的。

%54
这一次,联系灵缘斋,方源就用的这只信蛊。

%55
方源原本对此信,是信心十足。但灵缘斋方面久未回应,方源心中也有点不大托底了。

%56
按照常理推测:报信青鸟蛊乃是凤九歌独创,一定程度上代表着这位强人的意志和颜面。宁愿冒犯他的威仪,还要去拦截青鸟蛊的可能性并不高。

%57
但这种可能性,也不能被排除。

%58
因此方源点点头,采纳了太白云生的建议,道:“也好,再等七天。若是等不到回应,就再去封信。”

%59
日子一天天过去,狐仙福地已成困局。

%60
外有强敌逼近,内则严重缺乏修行资源。纵然有利用僵尸躯体、智慧蛊,推算并贩卖仙蛊方的上等取财之道,但很遗憾的是,方源没有关键性的启动资本,只能暂时搁置。

%61
琅琊地灵,灵缘斋,都是突破困境的钥匙。

%62
但很可惜,这两把钥匙都杳无音讯,不知道何时才能得到,亦或者终其一生也得不到。

%63
命运的刁难,前路的未知、迷茫,再一次笼罩方源。

%64
“看来我的运气,仍旧不太好啊。”方源自嘲。

%65
没办法,有春秋蝉在身,他的运气会越来越差。

%66
他试着向墨瑶询问完善度高一点的仙蛊残方,但墨瑶却推说不知。

%67
她寄居在方源身上,一起度过北原最后决战,知道方源拥有智慧蛊,因此不难推测方源的意图。

%68
墨瑶是炼道宗师,她极有可能知道符合方源标准的仙蛊残方。但她不说,方源智道手段缺乏,也奈何不了她。

%69
就在这样的情况下,黑楼兰忽然来到了荡魂山。

%70
方源失去了一只星门蛊,已经无法独自回去北原。黑楼兰能够来到荡魂山,当然是用的定仙游。

%71
至于她为什么知道这里的景象,那是方源曾主动传给了她一股意志。

%72
这本是约定的计划,但方源仍旧有些意外:“如果我没有算错,距离我们约定的时限,还有半个月之久吧?”

%73
“情况有变,我的朋友。”黑楼兰语气低沉,流露出一丝焦急的情绪。

%74
能让这位枭雄焦急,看来北原那边的情况,的确不容乐观。

%75
但旋即,她也意识到不妥,将情绪尽数收敛,抱臂冷笑起来:“当然,我还有一个目的,就是打你一个措不及防。毕竟这里应该就是你的大本营,万一你安排陷阱设计我,我提前来此,兴许便可打乱你的计划!”

%76
方源哈哈一笑,显得从容淡定,仿佛困扰着他的难题一概都不存在:“北原之战,你我已经是一条绳上的蚂蚱,合则两利,分则两害。不过要我方全力合作的话,你之前的条件还不太够!”

\end{this_body}

