\newsection{战斗只是修行的一部分}    %第四十四节:战斗只是修行的一部分

\begin{this_body}

地底洞窟中,充斥着五彩的智慧光晕。

矮小的菇林,四处生长。智慧蛊就趴在菇林最中央的菇王身上,一动不动。

方源坐在不远处,借助智慧之光,来不断推演。

良久,他吐出一口浊气:“第二份仙蛊方,也完成了。”

他看了下自己的青提仙元,本来就不多,只有二十几颗。现在连续推算完成两个八成残方,只剩下十一颗。

“第三道仙蛊残方,完善了八成六,十一颗青提仙元应该足够了。”

方源再看脑海中的乐山乐意仙蛊,却是熠熠生辉,毫无异样。

距离黑楼兰渡劫,已经过去了不少时日。因为激战中频频催动,净魂仙蛊已经饿得虚弱,乐山乐意仙蛊也催动了许多次,却一直状态良好,没有喊饿。

方源起初也感到奇怪,后来问了仙窍中的墨瑶意志,却是明白了缘由。

原来,墨瑶假意需要不时的补充。

墨瑶没有假意蛊,是用了许多蛊虫,将乐意转化成了假意。

因此喂养的过程中,第一个重点照顾的就是乐山乐意仙蛊。乐山乐意被墨瑶假意充分喂养,并非方源之前猜测的饿一顿饱一顿,因此即便方源这段时间频繁催动,距离下一次喂养乐山乐意仙蛊还有一段时日。

这显然是一件好事。

要知道,饥饿的仙蛊强行催用,会导致仙蛊的死亡。

净魂仙蛊已经喊饿。以至于方源目前使用万我杀招的风险剧增。也许某次催动这个仙道杀招,净魂仙蛊就会支撑不住而毁灭。

推算仙蛊方,是方源目前最主要的营生手段。若乐山乐意仙蛊罢工。方源无疑就要再度陷入困局之中。

方源再次催动乐山乐意仙蛊,开始推算第三道仙蛊残方。

这一次,他得到魔尊真意灌体,对推算仙蛊残方也有不小帮助。

皆因真意灌体,是汲取狂蛮魔尊对修行的理解,直接增长方源的底蕴,提升方源的修行境界。

而推算蛊方。是以自身的底蕴为基础,底蕴越强,推算效率就越高。

虽然主要是以炼道境界为主。但考虑的因素众多广博,也会涉及到力道、变化道等等方面。

两天之后,方源推算完毕,得到三张完整的仙蛊方。

他旋即就带着这些仙蛊方。来到琅琊福地。和琅琊地灵完成交接。

不仅将欠债还清,刨除青提仙元的损耗,还赚取十一块仙元石的纯利润。

一切又走上正轨。

八天之后,方源再次将三张十成仙蛊方,交给琅琊地灵,完成第五笔交易。

至此,他的仙元石回复道三十六块,青提仙元则有二十七颗。

方源这才稍稍放松下来。

和黑城一战。他险些财政赤字,经济崩溃。

按照地球上的经济术语。只差一点就要引发现金流的断裂。

“仙道杀招万我虽然威力不俗,但每次催用都要至少消耗一颗青提仙元。使用次数一多,消耗就大了。今后须得谨慎使用,否则越打越穷,虽胜犹败。”

方源在心中,暗暗警示自己。

蛊仙修行,要考虑各个方面,并非仅仅只是战斗的胜利。

就好像地球上刘邦和项羽的例子。刘邦战不过项羽,屡战屡败,项羽屡战屡胜。但刘邦兼顾经济,拥有安稳的大后方。因此尽管每次战斗失败,但都有重新崛起的资本,因此能屡败屡战。

反观项羽,虽然屡战屡胜,却经营无方,最终随着时间推移,交手次数的增多,刘邦和他的差距越缩越小。最终刘邦一战而胜,就是彻底的胜利。

对于蛊仙的修行来讲,也是如此道理。

战斗,只是修行的一部分。

战斗的胜败,并不那么重要,重要的是胜败背后的收益。

一切都要往前看,不能鼠目寸光。

方源虽然打得黑城、雪松子满天乱窜,但他投入的青提仙元太多,反观黑城、雪松子消耗的仙元,比方源要少得多。

方源打得险些经济崩溃,黑城、雪松子却是状态良好。若是现在再让方源和他们打一场,方源的战力绝对要下降一两个档次。这样交手的次数越多,孰胜孰败,一目了然。

黑城拥有六转仙蛊屋黑牢,难道不可以拼命吗?不是不可以,只是审时度势。

能成就蛊仙的,基本上都不会是热血冲脑的莽夫,每一场战斗都会算计得失。

当然,若非有魔尊意志灌体,方源也不会消耗这么多的青提仙元。

“青提仙元没了,可以再炼化仙元石。但是魔尊真意,却是难得的机缘。如果按部就班地积累,至少要数十年光阴。”方源心里盘算得分明。

等到方源将手中的仙蛊残方,又完善了两道时,太白云生终于从大雪山回来。

“你没有见到特等福地的气象,唉,我是亲眼所见了。占地千万余亩,福地的时光流速是外界的三十八倍!生机勃勃,潜力广大,令我都要忍不住嫉妒啊。我觉得这一次渡劫之战,最大的赢家便是黑楼兰!”太白云生回来,口中喋喋不休,感慨万千。

黑楼兰渡劫,请太白云生修复特等福地。因此太白云生有幸进入楼兰福地,可谓大开眼界。

“特等福地就是如此,毕竟是十绝体成仙,如履薄冰,死中求生。相比这个,我更关心黑楼兰这次渡劫升仙,在第三步炼出了几只仙蛊。”方源问道。

“她是用飞熊之力仙蛊炸出的仙窍,这只力道仙蛊是巨阳意志给她的。飞熊之力蛊现在是她的本命蛊。她还将原来的核心我力蛊,提升到六转层次。除了这些,她还得到一只六转力气蛊。”太白云生答道。

太白云生升仙时。凭借残留的天地二气,得到人如故、江山如故两大仙蛊。

黑楼兰残留的天地二气,是不是比太白云生要多,外人无从得知。但方源知道,她有不少的小家子气蛊,这些蛊虫中存储着许多天地之气。

方源评估黑楼兰的家底:“这样说来,黑楼兰手中可以确定的仙蛊已经有四只。分别是飞熊之力、我力、力气以及奴隶仙蛊。原本的七转排难蛊。已经在渡劫时毁掉。但别忘了,她曾经打爆了飞熊虚像,并将飞熊虚像蛊夺走镇压。从此之后。我就和飞熊虚像仙蛊失去了联系,也不知道这只仙蛊是被无相手夺去了,还是仍旧在她的手中!”

黑楼兰实力增长得很快,至少有四只仙蛊。如今成为蛊仙。更是一飞冲天,前途光明灿烂。

而且,她和黎山仙子关系紧密,得到黎山仙子的帮助,直接跨越新晋蛊仙的起步期。黎山仙子不缺仙元石,大量的资源投入下去,楼兰福地将会得到迅速的发展。

“拥有特等福地的黑楼兰,今后的修行速度将大大地超过我。我有上等福地。跟其他蛊仙相比,已经是快跑了。但黑楼兰却仿佛是插上了翅膀飞翔。咱们虽然和她们联盟,但联盟也是有时限。超过时限,就要小心了。黎山仙子交游广阔,黑楼兰更是枭雄啊。”太白云生语气担忧。

身为天才的黑楼兰的迅速成长,带给了太白云生这个老一辈巨大的心理压力。

而方源更惨。

他的修为停滞了,只要他一日没有摆脱仙僵的身份,他的修为就难以有所寸进。

“只要黑城一天不死,马鸿运的事情没有解决,我们和黑楼兰就有合作的基础。你此去东海,切勿小心。鲨魔非易与之辈,需要支援,就用推杯换盏蛊即可。”方源对太白云生关照道。

太白云生在狐仙福地休整了两天,便奔赴东海。

他身上中了死期将至仙蛊,答应东海仙僵鲨魔一起探索玉露福地,不得不赶紧启程。

方源则留在狐仙福地,一面推算蛊方,赚取仙元石,一面着重处理仙蛊喂养之事。

待他完成第六笔交易后,仙元石数量上涨到六十多块。

手中的妇人心、连运也出现虚弱的现象,显然是因为饥饿所致。连同之前的净魂仙蛊,方源有三只仙蛊需要喂养。

净魂仙蛊喂养,需要上万头的白莲巨蚕蛊。连运仙蛊,需要上古荒兽天地沙鸥栖息地的沙土万斤。毒道仙蛊妇人心,要用妇人的心脏去喂养。

前两者没有头绪,方源便决定先易后难。

西漠。

沙丘连绵,伸展着大地宽广舒缓的脉搏。

烈日高空悬挂,空气炙热,仿佛要将人蒸熟一般。

一只大型商队,迤逦而行。

驼铃声声,车辕作响。

随行的蛊师们,间或催动各种蛊虫,有的降温,有的鼓风,有的制造清水,有的侦察方向。

“什么人?!”商队首领爆喝一声。在商队的面前,忽然出现一只狰狞的怪物。

这怪物高达两丈,肌肉贲发,身材魁梧。生有八臂,青面獠牙,双目赤红似血。让人一看,心底便是一紧,感到一股无形的压力。

商队不安地骚动起来。

“两个五转,十七个四转,四十几位三转……倒也算得上雄厚。”怪物开口,声音沙哑难听极了。

“阁下是哪位朋友?我乃是莫家商队首领莫言。”商队里五转蛊师中的一位,神色紧张,抱拳问道。

方源狰狞一笑,八只利爪倏地张开,超级势力莫家的名头在他身上可不管用。

他宛若魔神降临,一头扎入商队当中,掀起腥风血雨。

惨嚎声、惊叫声、哭泣声、求饶声纠结在一起,持续了一会儿后,渐息下去。

\end{this_body}

