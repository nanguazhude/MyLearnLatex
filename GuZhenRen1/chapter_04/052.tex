\newsection{三转气囊蛊}    %第五十二节:三转气囊蛊

\begin{this_body}

%1
八头荒兽奔近,将鹤风扬、苍郁仙子,团团包围起来。

%2
“都是真的。”苍郁仙子拥有极其优秀的侦察杀招,察看了一番后,她娇美的脸蛋上又惨白一分。

%3
鹤风扬环顾一周,只见——

%4
东方站着一只冰刺神猿,体型只差荡魂山少许,霜气四溢,一路奔来,在沿途留下一道洁白的冰霜路径。它浑身皮毛如一道道冰锥,布满体表,带给二仙一阵压抑感受。它的双眼,眼白是碧绿色的,瞳仁则是霜青色泽。这表明这头冰刺神猿体内的血脉浓郁。若是眼珠子都是霜青色泽,那就是上古荒兽冰瀑神猿,战力可媲美七转蛊仙。

%5
而在西方则悬空漂浮着一头凤羽熔岩鳄。这头鳄鱼,体型只是寻常鳄鱼的两三倍,比不上冰刺神猿那般巨大,但气息澎湃。每一次呼吸,都能引发炎炙的热浪。它鳄甲深厚,呈现棕红色,修长的鳄尾长达两丈有余,上面没有鳞甲,取而代之的是凤羽,绚丽多姿,宛若火焰。

%6
北方傲立着一头金砂乌骓。此马巨大,个头能达到冰刺神猿的胸膛。它生有六蹄,宛若金属雕琢,肌肉贲发,身姿矫健如龙,浑身闪烁着暗金色泽。唯有六只马蹄,乌黑深沉。

%7
南方则盘踞着一头青龙藤。这藤不是野兽,而是植株。藤蔓凝结成一头长龙模样,根须扎在泥土中时,防御力、恢复力都极其惊人,最擅长持久战。若是抽出根须。便能宛若一条青龙,飞翔长空。

%8
东南方向,有一头泥沼蟹。它是泥沼地里的君王。山一样的雄阔身躯,此刻撑起身体,高度能达到荡魂山的四分之一。它的双眼退化至无,厚实的甲壳让鹤风扬有种面对乌龟壳的无力感觉。一共十九对螯足,尤其是第一螯足,轻轻一夹,就能断山石。剪蛟龙,就连方源的仙僵之躯,都不敢尝试这对螯足的威能。

%9
东北方。蹲坐着一头桃太狼。此狼体型最小,宛若刚出生的小狗。它看上去极其憨厚可爱,圆滚滚的身躯,肉嘟嘟的爪子。粉嫩嫩的舌头。黑漆漆的圆眼睛。看上去人畜无害,但鹤风扬却是瞳孔一缩,要让他选择突围方向,他会首先否决掉这一边!

%10
西北方,盘旋着一只铁冠鹰。此鹰盘旋上空,悍勇之气逼人。它一对鹰眼紧紧锁住鹤风扬、苍郁仙子,鹰翅宽大厚重,根根鹰羽可当做利箭喷射。鹰爪坚硬狰狞,一把拿捏下去。可将岩石捏碎,将龙虎撕扯。

%11
最后西南方,一头地魁荒兽站着。它人身蛇尾,面若蝙蝠,朝天鼻子,双耳招风,浑身漆黑,生有肉甲。胸膛前后左右,长有五六十根肉鞭,根根长达六丈有余。肉鞭表面,还生有吸盘,一旦被缠绕上去,万难脱身。肉鞭前端,还有菊花般的喷孔,关键时刻喷射出乳白色的液体,液体具有极强的腐蚀性,且藏有极多的寄生小虫,可直接钻入人的毛孔当中,进入身体,实施破坏。

%12
八头荒兽,皆是方源从琅琊地灵处借来的。

%13
准确的说,应当是方源直接借走了琅琊地灵的七转驭兽蛊。此仙蛊可控天下任何的野兽、异兽、万兽王、兽皇,以及荒兽、上古荒兽。

%14
荒兽可战六转蛊仙,尤其是身上寄生的野蛊未知,更为难缠。

%15
八头荒兽,包围着鹤风扬、苍郁仙子两人,虎视眈眈。

%16
二人皆沉默不语,再没有之前的悠然神态。

%17
鹤风扬又将凝重的目光,转向荡魂山之巅,比起八头荒兽,真正的威胁还在于方源等人。

%18
皆因,荒兽几乎不能使用杀招,但是蛊仙却有充足的智慧,可以使用、研发杀招。

%19
鹤风扬一颗心沉入谷底,强自冷静,对着方源喝道:“你们究竟是谁?”

%20
方源等四人,除了方源之外,其余三人皆戴着面具。黑楼兰戴着黑熊面具,黎山仙子戴着青鸟面具,太白云生则带着鹿首面具。三者身上都笼罩着一层光,遮掩了真面目,隔绝了鹤风扬、黎山仙子的探查。

%21
方源呵呵一笑:“在问我们的身份之前,二位是不是应该先自报家门呢?”

%22
鹤风扬沉默了一下,和黎山仙子对视一眼,均察觉到彼此眼中的苦涩。

%23
他们原以为此行手到擒来,最坏的打算,也设想到了对方拥有一位蛊仙战力。但绝对没有想到,对方的势力会如此之强!

%24
因此,他们之前态度傲慢,一味劝降,并未自报家门。

%25
“鄙人鹤风扬,仙鹤门六转奴道蛊仙。”

%26
“小女子苍郁,仙鹤门六转水道蛊仙。”

%27
二仙相继开口,态度再没有之前的高高在上,分别自称鄙人和小女子了。

%28
识时务者为俊杰也,能够成就蛊仙的人,都是俊杰中的俊杰。如今方源强势逼迫,鹤风扬、苍郁二人果断换了态度,低下了高昂的头颅。

%29
虽然蛊仙击败容易,难以斩杀,鹤风扬手中又有拓宇仙蛊,可以破坏狐仙福地,直接离开。但他们不知道方源的底细,方源手中若是有针对他逃跑的仙蛊,那就糟糕了。

%30
“原来是鹤兄,苍郁仙子。”方源语气一直都很客气,“在下方源,欢迎二位来我福地做客。”

%31
这话听在鹤风扬、苍郁二人耳中,却有些刺耳,让人感觉到嘲讽的意味。

%32
鹤风扬不由地一阵气堵。

%33
他明明向问的是这些人的真正身份,但方源装糊涂,答非所问,报上自己的名字。

%34
他的名字,现在中洲十大派谁不知道?

%35
方源故意回避这个话题,鹤风扬处于弱势,也不敢强问。

%36
“不知二位,所为何来?”方源又问道。

%37
鹤风扬沉默。心中直骂娘:“我们还能为什么来?就是来夺取狐仙福地,抢夺荡魂山的,这臭小子明知故问!”

%38
苍郁仙子冷哼一声:“阁下何必如此阴阳怪气。冷嘲热讽呢?阁下的势力超乎所有人的想象,什么时候中洲已经出现了这样的势力?今日我们二人认栽了,毕竟是阁下运筹帷幄,技高一筹。接下来是个什么章程?若要战,那就开战罢。何必废话呢?我二人奉陪到底,大不了血溅五尺,殒命于此。”

%39
这番话不禁让方源多看了苍郁仙子一眼。旋即他笑道:“仙子勇烈,在下心中佩服得紧。二位的来意,我大致能猜测得出。应该就是荡魂山上的胆识蛊吧。”

%40
“不错。”鹤风扬坦言承认。事到如今,他不得不承认,自己辛苦谋划,准备一年多的收复狐仙福地的计划。已经彻底失败了。

%41
但与此同时。他心中也不免生出一些希望:“方源若要开战,早就战了。但却说到现在,显然是忌惮仙鹤门之势,只要我利用得好,说不定可以从容脱身。”

%42
方源忽然击掌而笑:“二位来的正是时候啊,胆识蛊我也早就想卖了。二位且看这只蛊如何?”

%43
说着,他手中飞出一只三转蛊虫。

%44
蛊虫在六位蛊仙,八大荒兽。以及一位魅蓝电影的注视下,缓缓飞过一段距离。落到鹤风扬的手中。

%45
鹤风扬不明所以,担心方源阴谋算计,小心翼翼地接过这只凡蛊。

%46
这只三转蛊虫,形如瓢虫。有海碗大小,瓢虫头部极小,肚腹却大,占据总体积的九成九分九。

%47
它的肚腹圆滚滚,宛若灯泡,半透明状,背上没有甲壳,也没有翅膀。

%48
在这半透明的肚腹中,还藏有一只蛊。

%49
“蛊中蛊?”苍郁仙子轻声喃喃。

%50
很明显,这只三转凡蛊,乃是一只存储蛊虫,专门用来储藏蛊虫。这种存储蛊虫也十分常见,就比如地藏花,就是同一类型。

%51
但这只肚腹中的蛊虫,却让鹤风扬的目光,像是沾了胶水一般,死死地盯着。随后他游移不定地道:“这里面的……难道是胆识蛊?”

%52
方源朗笑一声:“正是如此!”

%53
“什么?”苍郁仙子秀眸一瞪,十分惊异。

%54
她虽然对胆识蛊知之甚少,但也知道一点,那便是胆识蛊不能离开荡魂山,一旦离开荡魂山,就要消失。

%55
因此之前,方源不愿交易胆识蛊。因为旁人要使用胆识蛊,就得亲自来到荡魂山。而任由这么多外人进入狐仙福地,对方源的生命无疑是个巨大威胁。

%56
更因此,中洲十大古派都要抢夺荡魂山。因为胆识蛊不能离开荡魂山,掌控了荡魂山,就能掌控胆识蛊。

%57
但现在事实摆在眼前,胆识蛊居然能被储藏起来,离开荡魂山。

%58
苍郁仙子旋即认识到这只不起眼的瓢虫形状的蛊虫,具有多么重大的意义,一时间她的目光也紧紧粘住。

%59
“这只蛊虫,叫做什么名字?”鹤风扬抬起头,望向方源,语气微颤地问道。

%60
实际上,方源还未起好名字,此时随意一想,便道:“气囊蛊。”

%61
鹤风扬沉默。

%62
一瞬间,他联想到了很多东西。

%63
仙鹤门高层,也不是没有做过此类的尝试。当时参与的蛊仙很多,但结果都失败了。

%64
现在,成果就被鹤风扬拿捏在手中。

%65
鹤风扬手掌都在微颤。

%66
“一旦胆识蛊能够离开荡魂山,它就能贩卖到各个地方去,宝黄天、海市福地、中洲、南疆、北原……这是多么庞大的收益?”

%67
“就连我们仙鹤门都研究不出来的东西,方源背后的势力却成功了。他们到底是什么来头?也许他们早就预谋很久,千方百计地抢夺狐仙福地,为的就是贩卖胆识蛊!”

%68
“方源他没有进攻,难道是想和我们仙鹤门……”

%69
鹤风扬正思考着,方源已经开口道:“既然二位对我的胆识蛊情有独钟,不妨我们做做这方面的贸易如何?”

%70
鹤风扬怦然心动!

\end{this_body}

