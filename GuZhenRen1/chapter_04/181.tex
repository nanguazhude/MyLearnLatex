\newsection{红莲魔尊传承线索}    %第一百八十一节:红莲魔尊传承线索

\begin{this_body}

方源灵机一动,想到了一招妙法。

蛊方已经给出了,但蛊师炼蛊却可以采用各种方法。比如融合两种材料,用火焰灼烧融化一体,是一种方法。用水液融化,在混合在一起,则是另一种方法。

两种方法不同,但相同的结果都能达到。

一炷香之后,方源的名字,登上了第一名,绿曜蛊收获在手。

这水火炼法方源练习的不多,不过水光蛊炼制难度太低,就算方源手法生疏,也是炼出了一百七十七只。刚好比第一名的标准,多了一只。

方源如此优异的表现,让五德门门主亲自过来,为之前的那位长老不好的态度,来打招呼。之后,这位门主又隐隐透出招揽的意思,语气却极为婉转。

方源直接拒绝,这位门主也不着恼,反而态度更加客气,亲自将方源送出门去,看到这样一幕的路人们纷纷投来惊异的目光。

能够夺得第一名,已经证明炼道造诣的不凡。

不管是什么修为,都会得到各大势力的热情招揽。单靠炼道这一点,方源在凡间就有立足根基,几乎走到哪里都能吃香的喝辣的。

更何况方源还有五转的修为呢?

所以五德门门主,丝毫不敢得罪\%方源,态度十分客气。显然这位门主,不止是背景深厚,自己本身具备犀利准确的眼光,并且能屈能伸,难怪能一手创下五德门这样的基业。

不过他怎么也不会想到,方源乃是仙人!

和数量庞巨似海的凡人相比。仙人还是太少太少。

五德门门主是多虑了,依照方源的性格。就算那位长老说的话再难听,方源也会无动于衷。

只要不阻碍他求道求永生的路。不管是辱骂,亦或者赞美,都只是嘴皮子上的功夫,方源丝毫不会放在心上。

距离第二场比试,还有一段时间。

方源离开五德门之后,便回到狐仙福地。

他在中洲一天的功夫,狐仙福地中已经过去了三天两夜。

“主人,主人,你之前要的东西。人家已经给你准备好啦。”方源一回到,地灵小狐仙就出现,带给他一个好消息,满脸“快来夸奖人家”的小样子。

“很好。”方源摸摸小狐仙的脑袋瓜,小狐仙十分开心,笑得双眼都眯起来。

方源让地灵准备的,都是一些凡道蛊虫,有的从宝黄天中购买,有些则是依靠石巢中的毛民炼制而得。

拿着这些凡蛊。方源便催出一小股星意。星意从脑海中飞出,便入了自家仙窍。

在他的仙窍中,一直关押着墨瑶的那股假意。

“方源?你这是什么意志?”看到方源星意到来,墨瑶假意立即感到不妙。

方源星意一笑。什么废话也不说,直接扑了上去。

墨瑶假意躲闪不及,哗啦一下。两股意志撞在一起,混淆纠缠在一起。

墨瑶假意发出惊叫。想要脱离,一直后撤。

“哪里逃?”方源心中冷笑。念头一动,飞出几只蛊虫,立即组成一记凡道杀招,将墨瑶假意定在原地。

“方源,你要三思!寻常的智道手段,可制不住我!!撕破脸皮,对大家都没有好处!”墨瑶假意被定住,又被方源星意不断冲击,居然还能说话。

不仅如此,墨瑶假意如流水般急速流窜,竟将方源星意压入下风,有重新汇聚成人形的趋势。

由此可见,这场意志交战,墨瑶假意占据上风,经验老道,远超方源。

但方源见此,却不慌不忙,念头下又飞出几只蛊虫,组成凡道杀招。

这杀招,化为一道五角星钻,一下子打过去,将墨瑶假意直接打散。

方源星意趁势反扑。

墨瑶假意不再说话,连忙作战,企图维系占据。

但这时候,方源的第三道杀招配合使出,形成星光漩涡,正中墨瑶假意。

墨瑶假意对战局的掌控彻底消失,无可奈何地和方源星意一起,随着星光漩涡搅合翻腾。

水.乳交融中,两股意志急速交流起来。

墨瑶假意掌握的无数的记忆碎片,被方源星意获知。与此同时,方源星意中的记忆,也被墨瑶假意获知。

在交流中,墨瑶尖叫:“方源,你太愚蠢了!太自大了!大不了咱们一拍两散。”

说着,墨瑶假意迅速消减,竟然施行自杀,企图自我毁灭。

方源呵呵一笑,他早已经将这招算计在内,念头轻轻一动,一道星芒光柱,从天而降,照住星光漩涡。

漩涡中,墨瑶假意主动自杀,假意不断削减。但在星光的照射下,假意又不断产生。

甚至方源的这股星意,也主动转化成假意,居然补充过去,让墨瑶假意尾大不掉,保持一定规模,连自杀都不行。

“墨瑶,你太天真了。你若之前,认输自裁,我只能眼睁睁地看着,没有任何办法。但现在,我让你死你就得死,我让你生你想死也不行。乖乖地将你所知道的,都吐出来吧!”方源朗声一笑。

墨瑶假意心如死灰,方源竟然如此快速,便掌握了制服她的手段。这远远超出她的意料。如此一来,墨瑶假意知道自己没有了利用价值,十分清楚自己再无幸免遇难的可能。

就像方源说的那样,若是之前,她立即自杀,方源什么都捞不到。

但世间万物,只要存在着的,都有生存的本能。蝼蚁尚且偷生,只要有一线希望,就算是墨瑶意志,也没有自杀陨灭的打算。

片刻之后,仙窍中,战斗结束,方源星意凯旋,满载而归。

留下的墨瑶意志,斗志崩溃,连正常的人形形象都凝聚不起来。方源又布置手段,不断温养墨瑶假意,以备他下一次搜意。同时生性谨慎的方源,又留下许多凡道蛊虫,禁锢住墨瑶意志,防备她再次自杀。

和墨瑶纠缠的方源星意,彻底脱离开来,体积只剩下原先三成不到。足足有七成,都在刚刚损耗掉了。

方源星意一飞冲天,离开仙窍,一路往上,直贯进方源的脑海当中。

方源闭上双眼,查阅星意带来的情报内容。

搜意和搜魂的不同。

搜魂是直接搜刮魂魄,读取一切记忆。搜意却需要动用自身意志,进行意志间的强行交流。这个过程中,还得辅以各种手段,才可让对方意志乖乖就范。

不精通对付意志的手段,很可能竹篮打水一场空。对付不同种类的意志,动用的手段又不一样。幸亏来自东方长凡的智道传承,十分全面。方源掌握之后,之前难以下手的墨瑶意志,就是砧板上的鱼肉了。

每一次搜意之后,需要温养目标意志一段时间。意志比魂魄还要脆弱。

值得一提的是,意志承载的记忆,比魂魄要少得多,并且大多都是记忆的碎片。

不过方源最想要得到的,是有关红莲魔尊的传承线索。这一次方源搜意,他的这个首要目的,也达到了。

“红莲魔尊,宙道传承……”良久,方源缓缓地睁开双眼,沉吟起来。

红莲魔尊的传承,就设置在光阴长河之中。传承中有什么宝物,墨瑶也不知道。她只知道,要继承这道举世无双的魔尊传承,首先必要的条件,就是拥有春秋蝉。

不仅如此,还需要蛊仙自爆之后,用春秋蝉载着意志,进入光阴长河。在光阴长河中,寻找到一座石莲岛。

只有到了石莲岛,才能见到留存在那里的红莲意志。

“红莲魔尊的传承,要得到它,真是艰难无比啊!”方源暗叹。

这难度也太大了点。

六转春秋蝉,一旦催动起来,蛊师自爆,只有一定的成功可能。想要见到红莲意志,就要冒着自杀的风险。

就算蛊师成功进入光阴长河,那个什么石莲岛究竟在哪里呀?

方源也不是没有进去过光阴长河,虽然有着两三次宝贵经验,但从未发现过什么石莲岛。

也就是说,不仅要蛊师自爆,利用春秋蝉装载意志,进入光阴长河。还要驾驭春秋蝉,在长河中遨游寻觅到一座石莲岛。

方源可做不到这点!

他现在使用春秋蝉,就相当于一次单程航行,并不能控制方向和航线。

“要控制春秋蝉,做到这样的程度,还得有别的辅助手段。可惜我没有这样的手段啊。”

这些手段,无疑是宙道手段。但对于这些方面,方源一片迷惘,几乎是空白一片。

就目前而言,方源想要着手继承红莲魔尊的遗藏,还有很长的一段路要走。需要大量的充分准备。

红莲魔尊传承安置在光阴长河当中,想要取得实在是危机四伏的大冒险。

方源只能暂时红莲魔尊传承搁置在一旁,接下来的一段时间里,他缩在狐仙福地中,一面等待炼蛊大比的第二场,另一面则借助智慧光晕,推算仙道杀招见面似相识。

他从北原拍卖大会中得到的见面似相识,有五成的完整度,并且三大核心仙蛊是什么,都没有掩盖。

问题在于,这三只核心仙蛊方源并未拥有,都在别人的手中。仙蛊唯一,方源要重现见面似相识杀招,就得改变,将原有的核心仙蛊替换成别的仙蛊。最好是他手中目前已经拥有的仙蛊。

这个难度是很高的,因为涉及到变化道、智道,偏偏方源在这两方面的境界并不高。

\end{this_body}

