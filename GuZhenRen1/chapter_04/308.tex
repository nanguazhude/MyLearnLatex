\newsection{风云滚荡}    %第三百零九节:风云滚荡

\begin{this_body}

%1
------------

%2
“终于停止了。”白晴仙子望着平静下来的落天河面,脸上残留着一抹苍白,心中犹有余悸。

%3
剑光纵横飞射时,她是距离事发地点最近的蛊仙,亲眼目睹了这一惊天动地的景象。

%4
此时此刻,落天河上满是尸体。

%5
大量的上古荒兽,太古荒兽,残缺不全的身体部分,在河水中载沉载浮。

%6
原本滚滚浩白的天河水,已经被染成了血红。

%7
落天河虽是从天上而来,但河水中充满了生机。无数猛兽潜伏繁衍,大量的水生植物生长其中,还有无数暗流、漩涡等等天然陷阱,就算是蛊仙陷入这些陷阱当中,也会有致命危险。

%8
白晴仙子又守候片刻,再没有剑光发出。

%9
她赶忙跑到河中,捡起水中的猛兽尸躯,或者是上古荒植、太古荒植的枝叶根系。

%10
这让她着实大发了一笔横财!

%11
财富之巨,就算是她白晴仙子也不禁大喜。

%12
不过好景不长,许多上古荒兽开始露头,白晴仙子甚至还隐约透过河水,见到一些太古荒兽的朦胧影子。

%13
这些猛兽都被浓郁的血水激发起了凶性。

%14
被剑光四分五裂的血肉,对于这些猛兽而言,都是上佳的血食。

%15
越来越多的猛兽,被血肉勾引过来,开始了一场激烈的争夺。

%16
它们在河面上疯狂撕咬,相互抢食,激起滔天的水花浪潮,气势好不骇人。

%17
白晴仙子不得不撤退,她看着剩下大半的血肉,心中充满了遗憾。这些可都是上佳的炼蛊仙材啊!

%18
“剑光忽然喷射,落天河中到底发生了什么?我究竟该不该下去探明真相呢?”

%19
白晴仙子心中牵挂着凤金煌。这让她有些犹豫。

%20
但在她心中,始终有一种感觉,在告诉她。薄青的线索应当就在这落天河底。

%21
白晴仙子咬咬牙,身形化作一道白虹。一下子扎进落天河水当中。

%22
从落天河的源头,四下飞射出来的无匹剑光,打击范围覆盖了整个中洲。

%23
不管是中洲中部的蜈蚣峡谷,还是中洲东边的飞霜阁,甚至是高高在上的天庭,都被剑光惊扰。

%24
一时间,中洲为之震动。

%25
天庭。

%26
监天塔主满脸铁青之色,他手中动作毫不停歇。反应过来后,他就开始积极地修复监天塔。

%27
碧晨天、白沧水、炼九生,已经被他传讯,紧急召回。

%28
此刻,他们都正在紧急赶回的路上。

%29
剑光造成的创伤,非常麻烦,剑道道痕似乎有侵蚀性。剩下的监天塔虽然屹立不倒,但是却难以持久。

%30
若是任由剑道道痕侵袭下去,势必会酿造成更为惨重的损失。

%31
“虽然这一次修复宿命仙蛊大获成功,但临到使用的关口。监天塔却被斩断了。这是一个巧合吗?还是……”

%32
监天塔主心中有一股不妙的预感。

%33
片刻之后,碧晨天首先赶回来,随后是白沧水。最后才是炼九生。

%34
炼九生中途耽搁了一下,但也因此带来了具体情报:“查出来了,剑光来自于落天河的源头。现在整个中洲都闹翻了。”

%35
“落天河源头,那不是薄青陨落的地方吗?”白沧水惊呼一声。

%36
提起薄青,天庭四仙心情都有些复杂。

%37
当年他们不是没有招揽过薄青,但可惜薄青和他们的理念不合,连天庭的大门都没有通过去。

%38
“薄青……”监天塔主目光深沉,“调查,必须要调查清楚。沧水仙子。你擅长水道,就劳烦你跑一趟吧。我们三位则留在这里。全力修复监天塔。不知道为什么,我心中总有一股不安之感。”

%39
其余三仙均神色一凛。其中碧晨天道:“监天塔主你掌握监天塔和宿命蛊,这份不安不容小觑,我们速度行动!”

%40
中洲,战仙宗。

%41
“落天河源头惊变,疑似薄青的剑道传承出世!金烈阳,你便去探明真相。六转蛊仙的名额给你两个,随你挑选。”战仙宗太上大长老的意志,漂浮在半空中。

%42
蛊仙金烈阳正置身火焰之中,他身材魁梧,一头金发,火焰瞳眸,七转气息强势无比。

%43
听到这话后,他却冷哼一声,丝毫不买太上大长老的脸面:“这事情应该去找石磊去,他可是仙猴王,凤九歌已经不在了,他已经是中洲七转第一人了!”

%44
太上大长老的意志微微一笑:“金烈阳,石磊虽强,但偌大的战仙宗,也不是他一个人的。他这一次探索繁星洞天,功劳已经足够多了,难道你还想让他再立大功吗?”

%45
金烈阳脑筋一转,陡然间想通了。

%46
他从炙热的火中一跃而出,抱拳拱手道:“嘿!太上大长老这话说的是。本来我还想先身上的炎道道痕,增长十七八条再说。既然如此,我即刻动身!”

%47
太上大长老的意志点点头,叮嘱道:“速去速回。”

%48
风云府。

%49
“大师兄,你出关了?”蛊仙洪赤明看到眼前一人,十分欣喜,连忙行礼。

%50
这人一身白袍,腰系玉带,身材修长,文质彬彬,此刻含笑点头:“我是因落天河的变故而出关。师弟,你参加过百日大战,亲眼目睹过五指拳心剑。这一次为兄受命,前往调查落天河真相,还请你出手相助。”

%51
洪赤明深吸一口气,诚挚地道:“大师兄,你这话太见外了。当年若不是大师兄你出手相助,怎有赤明今日呢?洪赤明愿效犬马之劳!”

%52
灵缘斋,议事堂。

%53
“落天河源头惊变,我们该派遣何人前去调查?”太上大长老不在,由太上二长老亲自主持。

%54
“二长老有所不知,白晴仙子已经于前些日子,独生前往落天河源头探索去了。”李君影的意志便道。

%55
“哦?难不成她发现了什么线索?”有蛊仙不免联想起来。

%56
太上二长老则皱起眉头:“白晴仙子只有六转巅峰修为,不足以应付这个局面。而门派却正是战力吃紧的时候,这个时候该派遣谁去支援白晴呢?”

%57
徐浩意志哈哈一笑:“启禀太上二长老,白晴仙子的修为实则已经晋升七转,只是她偷偷隐瞒,并未上报。”

%58
“哦?竟有此事?我记得她距离七转的第一灾劫,还有一段时间的呀。”

%59
徐浩从容答道:“此事确凿无疑。凤九歌前往北原之前,就借用宙道仙蛊,秘密帮助白晴仙子提升了修为。”

%60
“竟是这样。”太上二长老沉吟一番,才道,“白晴仙子成为七转蛊仙,是我灵缘斋之幸事。待她此次回来,就将无量峰赏赐给她。另外开启库藏,允许她从中选取一只七转仙蛊。至于这次落天河之事,就不再派遣援兵,只传讯于她,说明门派的意思。”

%61
“是,谨遵太上二长老之命。”

%62
天梯山。

%63
“如此的剑光,薄青,薄青……剑劈五洲亚仙尊,为情所系幸苍生……说不定这就是他的剑道真传出世了!”

%64
剑道蛊师剑一生,抬头眺望天际,脸上尽是激动和神往之色。

%65
“这一次大变,落天河源头肯定是蛊仙集结,或许能有浑水摸鱼的机会啊。中洲十大古派虽强,但他们吃肉,我喝口汤总是可以的吧?”

%66
念及于此,剑一生终于下定决心,飞出天梯山,投向西北的天边。

%67
不管是以中洲十大古派为首的正道,还是魔道,或者类似剑一生的散仙,都各有举动。

%68
无数道目光,隔空眺望,纷纷集中在了落天河的源头处。

%69
薄青,亚仙尊,古往今来尊者之下第一人,这些名头,这些尘封已久的记忆,又鲜活起来。

%70
无数道肆虐整个中洲的剑光,提醒世人,薄青的强大。

%71
一道流言不知道地从什么地方传出来的,却是越传越广,牵动众人的神经。

%72
“这一次剑光散射,是薄青陨落前留下的后手。他是想提醒世人,他的剑道真传出世,他虽然身陨,但也不想真传所托非人呐。他要借此良机,择选出最适合的继承人!”

%73
于是因为剑光之威的震惊,渐渐转化成贪婪的欲火。

%74
中洲蛊仙界震荡!

%75
无数的蛊仙,甚至是隐姓埋名上百年的老怪,都开始赶往落天河。

%76
中洲风云动荡,一场囊括整个中洲的巨大纷争,就要在落天河源头展开。

%77
而在南疆,却是风平浪静。

%78
青山葱茏,松涛阵阵。

%79
山峰处,一座石亭。

%80
砚石老人坐在亭中,正着手下棋。

%81
他面前摆着一块巨石,平坦的石面上星道道痕纵横,或是横切,或是纵劈,或是斜插,组成一团令人眼花缭乱的线图。

%82
砚石老人在下的,赫然便是名垂青史的星盘棋局。

%83
一道身影,从天而降,霎时间石亭中寒气四溢。

%84
“你找我?”白凝冰冷漠地盯着砚石老人的后背。

%85
砚石老人慢慢地转过身子,面向白凝冰。

%86
白凝冰继续冷声道:“你虽然助我成仙,但我可不是你轻易呼来唤去的手下。”

%87
砚石老人微微一笑,轻声道:“你知道吗?方源就要回来了。”

%88
方源!

%89
白凝冰眼光骤亮,旋即双眼微微眯起,遮掩住眼中的厉芒。

%90
他冷哼一声:“请你把话说明白一些。”

\end{this_body}

