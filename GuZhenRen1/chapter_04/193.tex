\newsection{当众擒拿方正}    %第一百九十三节:当众擒拿方正

\begin{this_body}

方源前世,当方源将主意打到炼骨大会上时,上一届的炼蛊大会才过去十几年。

为了参加炼蛊大会,前世的方源便提前八十多年就开始准备。

每天他都抽出时间,勤加练习,日夜不缀。

在这个练习的过程中,方源拿出前前世,地球上的高考少年面对题山库海的精神。将历届的大会题目全部收集起来,一边照此闷头苦练,一边企图从中中阶出出题者的用意。

他十分容易就搞到全部的题目。在中洲还真有人专门贩卖这些信息,并且这种商货,一直都比较畅销。可见聪明人比比皆是。

回想起那段暗无天日,闷头苦练的日子,真有些不堪回事。不过说起来,方源真正炼道的奠基,也正是那段时间。没有那段时间,不可能有他现在炼道准宗师的境界。

所以方源不仅记得这届的试题内容,还记得之前好多届,更记得未来的几届!

有这么大的优势,尽管炼道境界不如火工龙头,又怎样呢?

当然,此法也有一个弊端。

那就是蝴蝶效用。

方源重生,改变了很多事物发展的原本轨迹。点燃一棵树木,兴许火焰就会燃烧整片森林。

善泳者溺,善骑者堕。方源依仗的优势,会不会因为今生题目的改变,而成为自身失败?

这完全是有可能的。

但只是有可能而已。

按照方源的推算,这种可能性不大。

方源决定冒险。

他赌赢了,题目内容并未有所改变!

时间以稳定的步伐,不疾不徐地向前流逝。

激烈的炼蛊场面,渐渐趋势明朗。

方源一马当先,领先第二位的火工龙头老大一截。他浑身上下散发着自信十足的气势,有条不紊地炼蛊,始终保持着巨大的领先优势,让身后拼命追赶的竞争者们渐生绝望。

场外的蛊师渐渐有人看出端倪,不禁出声赞道:“方源竟然如此才思敏捷。他根本就没有多想,居然就设计出这个炼蛊方案。了不起啊,他的这个方案不仅能够保障炼蛊速度,步骤减少,而且更有容错率。就算失败几次,,多余的材料还能让他再尝试。”

这是当然的。无弹窗,最喜欢这种网站了,一定要好评]

方源采取的这个方案,是前世这一届炼蛊大会后,无数炼道蛊师不断回味,不断尝试,最终总结出来的最优秀的炼蛊方案。

更多的人则感慨方源的炼蛊手法:“方源的炼蛊手法,实在太突出了。如此数量,简直像是天赋本能一样。这绝对是要千锤百炼的!此次炼制缄默蛊,涉及很多平时很少用到的手法,他居然都能如此得心应手。更难以想象那些常用手段了。”

这也是当然的。

方源为了这第八场比试,准备了好久。

在狐仙福地里,特意作了集训,专门炼制缄默蛊。一整套的流程,翻来覆去,练习了至少两百遍!

“更关键的是这他浑身上下,由内而外散发出来的自信气度。他笃信自己的成功,好像炼蛊失败的概率,从不存在他的脑海中一样。这就是炼道大匠的气质,历史上呵呵有名的炼道三老,就是如此。”

“是啊,看他炼蛊,就像看一场赏心夺目的表演一样!”

“他的内心是多么强大啊,仿佛胜利已经被他揣进口袋里了。”

“优势太明显了,即便方源炼蛊失败一两次,他都能和第二位的火工龙头抢夺最终的胜利。”

在场下场上的无数双目光的见证下,方源无惊无险,以极大的优势取得此场的胜利。

整个过程中,方源都牢牢盖压其他竞争者。哪怕是火工龙头,也被他甩出一大截来。

“火工龙头大人失败了么……”

“这就淘汰了?”

万龙坞众人都有些不敢相信眼前的结果,失神喃喃。

“方源,方源!”仙鹤门众人起身呼唤,忘情欢呼!

唯有方正仍旧坐着,他望着场上的哥哥,黑袍身影已经笼罩住他的整个身心。

“恐怕我这一生,报仇都无望了!”这一刻,方正的灰心丧气达到了极点,已经兴不起一点点的斗志。

虽然有着深仇大恨,但方正此时望着方源,却感觉浑身疲软,无力回天!

火工龙头瞪着一对牛眼,死死地盯着方源。

他不服!他不甘心!

他的最强杀招都没有使用出来呢。这该死的题目,让他束手束脚。

但他也不得不承认方源的才思敏捷。

“居然在这么短的时间内,就设想出如此优异的炼蛊方案。方源这个家伙,究竟到底是不是仙僵?情报会不会有误?”

仙僵弱于思考,方源“思考”出来的成果,让火工龙头这位正常的蛊仙,都感到自愧不如。一时间,火工龙头都开始怀疑方源仙僵的身份了。

比试已经结束,方源缓步而行,慢慢地走下场。

他没有再挑衅羞辱火工龙头,这不禁让后者暗暗松了一口气。

既然已经得胜了,挑衅羞辱对方已经缺乏意义。万一激得火工龙头主动向方源挑战,要赌斗方源,那方源岂不是搬起石头砸自己的脚了么。

最终,火工龙头带着万龙坞一干人等,灰溜溜地走了。

方源从驱邪派接过胜者奖励,这是一大批的仙材,种类繁多,但每一种的数量都较为稀少。

从第八场开始,每一次的奖励都远超前面七场,有大量的仙材、仙蛊方等等。

若非如此,方源前世怎么可能将中洲炼蛊大会,当做一种采集仙材的途径呢?

方源收起这些仙材,拒绝了驱邪派方面的留客邀请,一路下来,渐渐走到驱邪派的山门所在。

山门口,仙鹤门一群人早已在那里等待。

“方源大人!”见到方源到来,炎堂长老为首的仙鹤门两位长老,一起恭迎。

方正十分无奈,同样身为长老,他也出列,但站在两位长老身后,面无表情,也不主动打招呼。

“诸位同门,你们都好。”面目下,旋即传来方源温和的声音。

这让仙鹤门一群人十分激动。

方源之前的张狂,让他们都觉得方源很难相处。但现在方源声音温和,立即让他们觉得方源对同门的态度,就是和对外人不一样。

方源首先看向炎堂长老。

在刚刚的比试中,这位长老也同样参加。

方源随意指点道:“嗯,你的炼蛊造诣不浅,但设想的方案却有几大弊端……这些弊端应该如此处理,比如第一处……”

炎堂长老听完之后,更增对方源的钦佩,连忙道谢。

他身后的弟子们,一位位双眼发亮,充斥敬仰和赞叹。

方源这时将目光转向方正,声音陡然转冷,一副恨铁不成钢的语气道:“我的弟弟,你这是什么表情?是害怕我责骂你吗?你的表现一如既往的令我失望啊。居然连炼蛊大会第一场都没通过,简直是丢我的脸!”

气氛顿时凝结,弟子们暗暗吐舌头,均想:方源这个哥哥对弟弟好凶!

仙鹤门的两位长老则有些恍然,看来这兄弟俩之间,似乎关系比较冷淡啊。

方正脸上浮现出仇恨之色,冷哼一声,正要说话。

方源却大手一挥,竟然当众动手!

方正是五转蛊师,已经傲立凡俗巅峰了。但他一来是奴道蛊师,大半战力都落在鹤群身上,二来也绝对没想到方源会不顾场合,在众目睽睽之下,当众向他出手!

方源这一出手,蓄谋已久。

动用的不仅是好几个凡道杀招,而且其中有来自东方长凡的智道杀招。

仙鹤门同行愣神之际,方源已然得手,隔着一段距离,将方正禁锢在原地。

“方正,你太令我失望了。这样的你还在外面行走,为什么不静心潜修呢!如此实力还在外面招摇,成何体统?双亲早丧,我这个做哥哥的就要起到父亲的职责,好好教导你!你既然对炼道有兴趣,那么从今天起,你就跟随在我的身边,我传授给你我的炼道心得。你必须埋头苦修,直到我满意为止。”方源轻喝道。

“我不要你管!我已经贵为仙鹤门长老,已经长大了!!”方正大吼,一副不服管教,年少叛逆的模样。

实则,他心中一片冰寒。

他已经不受自己的控制,叫喊出来的话语,根本不是自己想说的话!

这就是来自东方长凡的智道杀招,直接控制方正的脑海,产生虚假的念头,让方正的身体接收过去,按照念头行动。

这和奴道蛊师驾驭猛兽,有异曲同工之妙。但不同的是,奴道蛊师大多只能掌控一种兽群,野兽念头简单。而操纵人类,人的念头多变复杂,十分困难。

方源是六转仙僵,对付方正这位五转蛊师,还得动用数种杀招,搭配起来,才有了现在一举擒拿的效果。

“不!快救救我,谁来救救我!师傅,师傅!”方正内心焦急大吼。

但寄魂蚤毫无回应。

方源早就切断了方正和自身空窍的联系。

方正死死地盯着方源,睚眦欲裂。他万万没想到方源居然这么疯狂,敢在这里直接出手。这简直是丧心病狂,根本不把仙鹤门放在眼里!

\end{this_body}

