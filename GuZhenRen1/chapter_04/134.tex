\newsection{百假当中有一真}    %第一百三十四节:百假当中有一真

\begin{this_body}

方源大手印一出,效果惊人,立即引得许多蛊仙,纷纷投来惊异的目光。

事实上,这种囤积数量,再一次爆发的杀招设计思路,十分普遍常见。

就好像是用最普通的丹火蛊,发出一道火球。怎么样才能威力更大?绝大多数的正常人,都会这样想:将一颗颗火球积累起来,等到对战的时候,一同发出。

如此,数量增多,招数的威力就得到的增长。

这是最平常,最简单不过的思路。

但万事,向来是想的容易,真正做到并不容易。

提前发出的火球,该怎么样存储下来?就算积累了很多火球,如何才能将这些火球一次性的发射出去?即便发射出去,该怎么样让火球有准头,而不是胡乱盲目的烂射?

真正解决了这些难题之后,蛊师往往才会拥有,一道真正意义上的杀招。

所以说,蛊虫养用炼,每个方面,都是博大精深。尤其用蛊方面,更需要创新精神。就算开发出杀招,也因为个人的差异,每一个杀招自然都有优劣长短之分。

方源构思出来的仙道杀招万我,可谓是打破藩篱,使得奴力两道合流,另辟蹊径的伟大成就。

他本身有着力道基础,又在第二空窍升仙时问询天地,因此有着深厚积累。

之后,他做出巨大牺牲,冒着九死一生的危险,依靠九转智慧蛊。脑海中灵感无限,这才构思出仙道杀招万我。

整个过程,方源起着最关键的作用。但周围的环境。智慧蛊、升仙、王庭福地覆灭,都有着独一无二的促进。

达成奴力合流的仙道杀招万我,从设计思路这个根本上,就超越了世间绝大多数的仙道杀招!

凭借这个起点,方源若在日后发扬光大,逐渐完善,甚至能创造出一条崭新的流派。

因此。虽然目前万我杀招还只是起步阶段,但潜力绝对惊人。

奴道,是消耗战术的极致。统领千军万马,炮灰战术,以绝对的数量淹没敌人。

力道,逞个人英雄。千般巨力加持一己之身。神挡杀神。佛挡杀佛,匹夫一怒,血溅五尺之地,这个范围内,人可敌国!

两大极致,融为一体,造就杀招万我。

方源可以用一人之力,催发千万力道虚影。指挥如意。与那黑城作战时,力道虚影大军让黑城、雪松子避退。无可奈何。

这是万我杀招中,力道支撑奴道的体现。

如今大手印,却是恰恰相反,是奴道转化力道的体现。

而造成这个关键效果的,正是群力蛊。正是这些上古蛊虫,最后投入杀招光团,调动了力道虚影的力量,加持在方源一人身上。

还有一个关键,便是我力仙蛊。

若换做之前的净魂仙蛊,是无法催出大手印来的。

所谓的大手印,不过是力道巨人虚影的一只手。

黑楼兰曾经掌握我力仙蛊,她的杀招最强变化,便是一位力道巨人虚影。

方源利用仙道杀招解梦,救醒黑楼兰后,获得了她的力道真意。这些真意,便是大手印最后的一份催化剂。

如此种种,才有了现在的大手印。

我力仙蛊不愧是最适合仙道杀招万我的第一核心。有了它的参与,使得万我发展更进一步,真正做到了奴力两道之间的相互转化支撑。

方源用这一招,而不是将仙窍中的力道虚影大军全部放出,自然也有思量。

这些力道虚影,都酷似他的相貌。如今他犯下泼天大案,却需要隐匿行迹。

尤其是拍卖大会之后,他的身份进一步曝光。而见面似相似杀招,还未利用智慧蛊勘破。

大手印这招,正适合此时的方源威力惊人,却又不会暴露了他的身份。

巨大的拳头,继续向着卓战轰去。

卓战惊出一身冷汗,撤退的速度飞快,和大手印越来越远。

嗖的一声,刚刚被禁锢的仙蛊雷梭,钻破大手印,投向其主人天都神君。

方源丝毫没有意外。

大手印这招的设计思路方案,的确有无相手在暗暗影响。

但两者区别甚大。

大手印是力道杀招,无相手却是偷道的极致。无相手能捕捉仙蛊,大手印顶多暂时禁锢,已算是做得不错。

方源一招发出,缓缓收势,几个呼吸的时间,大手印在半空中逐渐消失。

整个发招的过程,他的右手臂嘎吱作响。饶是力道仙僵之体,也有难以支撑之像。

现在收招之后,原本坚实有力的手臂,一时间竟然瘫软下来。皮肤之下的肌腱大筋,几乎全断。臂骨裂痕满布,血管崩裂,和肌肉混乱成一锅粥。

大手印是从万千力道虚影身上,汲取了一份份的力量,凝聚在方源一处。

凝聚的力量过于庞大,方源就像是承载的基石。

“我这次催出的大手印,不过调集了两万力道虚影的力量。但大手印的威能,暂时也只能这般强度。大手印可以提升,但我本身已达到极限。”

方源心中遗憾地想着。

他的手臂,慢慢垂下,被宽大的黑袖覆盖。

短短时间,凭借着仙僵之躯的强大恢复能力,他的手臂已经痊愈。

饶是如此,大手印的威力也足够惊人,吓得卓战满头冷汗,连退数里。望着方源的目光,带着惊惧。

周围的蛊仙,纷纷下意识地拉开和方源的距离。

黑楼兰、黎山仙子笼罩在黑袍帽兜之下的面孔,亦是泛出奇异复杂之色。

大手印的威力。让她们俩都暗暗心惊警惕。

但此刻方源作为盟友,站在自己这边,两人又不免庆幸开心。

这两种心情混杂在一起。自然滋味难言,情绪复杂了。

尤其是黑楼兰,察觉到大手印中带着自家杀招的味道,更生出一股微微的挫败之感。

远处,自在书生云逼退天都神君,抓向雷梭仙蛊。

天都神君大急,连忙施展手段。这才成功夺回雷梭仙蛊。

自在书生没有达成目的,也不气馁,云淡风轻的一笑。调转视线,投向方源:“真是厉害的仙道杀招。”

在他的眼瞳深处,闪过一抹郑重。

“这是名副其实的七转战力!”陆青冥等人斩杀了墟蝠,刚刚回到太丘外围。看到这一幕。神情皆是忌惮。

“这又是哪一位?”皮水寒则在推测。大手印刚刚第一次亮相,当然不是为人熟知的仙道杀招。

“这是力道仙级杀招,难道来人是霸仙楚度?”许多蛊仙不可避免地猜测着。

力道蛊仙,能有如此战力的,北原也就这么一位。

但旋即,众人又都在心中摇头。以楚度的行事风格,绝不会如此隐藏行迹的。

天都神界夺回雷梭仙蛊,匆匆扫视一圈。

对于自在书生。他是不想再交手了,现在看来方源这面。也是铁板,而游地三英也赶了回来。

他顿时心知单凭自己实力,已经无法捡便宜,脸色变得相当难看。

这些说来话长,但时间极短。

方源发出大手印,缓缓收招,手臂自然垂下,在无数目光的注视下,复又回归到黎山仙子、黑楼兰的身边。

他故意低头,没有发出任何声响,仿佛刚刚做的微不足道。

黎山仙子走在前头,他走在黎山仙子的身后,和黑楼兰并齐。

看到这样一幕,众人又不免惊疑。

方源等三人,都是黑袍罩体,服饰一致,神秘叵测。看方源甘居尾翼之态,众蛊仙不免猜测方源身旁的黑楼兰,恐怕有着和方源一般的战力。而位居之前,一副头领姿态的黎山仙子,战力恐怕还要高于方源。

方源深谙人心,这个简单微小的动作,落在许多蛊仙的眼中,却比刚刚的大手印,还要让他们惊悚!

尽管众仙亦知道方源有故弄玄虚的可能,但事实上当方源三人成功落到地面,接触传送蛊阵,捕捉东方余亮等人的线索时,仍旧没有人再出手阻止。

众仙默认了方源一行人的强大。

当然,更多的原因是智道传承连影子还未见着。众位蛊仙也都精明,犯不着现在打打杀杀。之前的交手,也是以试探阻挠为主,并未真正展露底牌,也未全力攻伐。

方源三人,一同捕捉线索。

几个呼吸之后,都是惊异出声。

“好一个传送蛊阵!”黎山仙子的声音故意变得沙哑。

“不愧是东方长凡,故意留下蛊阵,误导我们……”黑楼兰续道。

“这其中的线索,定有正确的答案,但能找到东方余亮,却需要运气了。”方源接着道。

“三位这是何意?”一位蛊仙不明所以,高声问道。

方源三人也不答话,忽然拔地而起,升上高空。他们离开太丘,化作三个方向,迅速消失在众人视野当中。

随后,自在书生也落到地上。

他稍微查探一番,便知方源等人所言不虚。

原来,东方长凡在传送蛊阵中做了手脚。他没有刻意布置手段摧毁蛊阵,因为这种蛊阵本就传送凡人蛊师可以,传送仙人不成。

摧毁了蛊阵,自有蛊仙能人,能够捕捉到一丝一毫的细微线索,顺藤摸瓜,找到东方余亮。毕竟东方长凡善于算计,并不擅长传送蛊阵的铺设。

而这般故意留下线索,将正确线索隐藏其中,却是叫人真假难辨。

如此一来,引得蛊仙们分兵四处,就算有人找到了东方余亮,也大大减轻了后者的压力。

先前方源三人,是选择了最可能的三个方向。

自在书生略微思量,选择了第四个方向,迅速离开太丘。

在他之后,众蛊仙一一降临,得到线索,向四面八方赶去。

最后,只剩下东方部族的蛊仙们。

“这些蛊仙真够狡猾,居然没有一人摧毁了传送蛊阵。”

“就算摧毁了蛊阵,也会有线索留下。这些魔道蛊仙能走到今天,自然不会做这样吃力不讨好的事情。”

“是啊,他们还盼望着哪怕自己这一路失败,也有人摸到正确线索。到时候闹出动静,他们再赶过去争夺的打算呢。”

“好了,我们也下去吧,看看有什么线索。”(未完待续……)

\end{this_body}

