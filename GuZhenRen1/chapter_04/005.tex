\newsection{问询墨瑶意志}    %第五节:问询墨瑶意志

\begin{this_body}

%1
“唉,我已经很多年没有见到镜柳了。”墨瑶意志口中感叹,抚摸着眼前唯一一株镜柳,满怀感慨,似乎回忆起了昔年往事。

%2
而方源的意志,则站在她的身旁,静静地看着,没有开口打扰。

%3
这片天地,晦暗腐朽,正是方源已死的仙窍福地。

%4
方源自哄骗了墨瑶意志进入自家仙窍,便将其镇压住。

%5
墨瑶乃是炼道宗师,曾经的传奇蛊仙。在中洲历史上,留下了浓墨重彩的一笔。

%6
她的意志,自然大有价值。

%7
虽然方源被墨瑶意志陷害,在北原之行时几乎九死一生,但方源并不痛恨墨瑶意志。反而很是欣赏,甚至赞叹。

%8
换做他来施计,未必能比墨瑶意志做得更好。

%9
他原本准备拷问墨瑶,结果墨瑶意志却是非常配合。对于方源在炼道上的提问,不仅知无不言,而且还延伸类比,叫方源大有收益。

%10
但是当方源询问有关灵缘斋的情报时,墨瑶意志却提出了一个要求:“我离开中洲太多年来,这些年,我越来越想念那里的人和物。请给我寻来一株镜柳罢,让我稍稍抒心中郁结的思乡之情。”

%11
镜柳只是普通植物,寻常柳树大小,但柳叶似一片片镜片。这种柳树的叶子,正是乞丐蛾的食物。

%12
之前石人贸易,方源从仙鹤门处收购了一批镜柳,如今栽种在狐仙福地。

%13
于是他随手选取一株,放进自家仙窍。

%14
“想当年。我的本体和薄青哥哥就在这镜柳下第一次相识。红尘滚滚,风流成空……”墨瑶意志感叹一声。

%15
旋即,她转过头来。看向方源意志:“谢谢你满足了我这个小小的要求。你既然想知道灵缘斋的事情,那我就告诉你罢。只是我知道的情报,早已经过时,你要注意这一点。”

%16
方源意志点点头。

%17
墨瑶意志站在镜柳下开口,说出了很多灵缘斋的内幕情报。

%18
方源一边聆听,一边和前世五百年经历做比较,对灵缘斋的了解大大加深。

%19
“你这仙窍已死。镜柳种在这里,不出三天,就会被死气侵蚀。化为枯死朽木。可惜了……”说完灵缘斋的情报,墨瑶又叹息一声。

%20
只是不知道她口中的“可惜”,是说的镜柳,还是方源的仙窍。

%21
“下一次来。请带给我一杯青浦茶好么?这是灵缘斋特有的茶。并不珍稀。我虽然品不出茶味,但却想再看一眼。”墨瑶道。

%22
方源意志冷哼一声:“看起来你毫无作为俘虏的自觉吗?居然要这要那,你该不会还在打什么主意吧?”

%23
墨瑶意志妩媚一笑:“方源,你太谨慎啦。我的计谋自被你揭穿,又被你哄骗到仙窍中镇压,已经是砧板上的鱼肉,每时每刻都受你监视,任你拿捏处置。就算打了什么主意。又有什么办法实施呢?”

%24
“不过,咱们打开天窗说亮话。我虽然彻底失败。无法翻身,但你若用强,恐怕得到的东西会十分残缺,毕竟搜魂简单,搜意困难。你若是智道宗师还行,可惜你的智道底蕴等若没有,都是从市面上胡乱收购的蛊虫。智道和其他流派可不一样,神秘艰深,你现在还未入门呢。”

%25
墨瑶意志批评方源起来,可是毫不客气。皆因她知道方源心胸宽广,乃不世枭魔。

%26
果然下一刻,方源意志哈哈一笑:“你说的不错。我智道造诣是浅薄,但也亏你主动配合,今后你就在我的仙窍中生活吧。青浦茶我会带给你的。”

%27
说完,意志冲天而起。

%28
仙窍恰好打开一丝缝隙,任由这股意志一路飞升,到达方源的脑海。

%29
转瞬之间,方源就洞悉了这股意志在仙窍中收获到的所有情报。

%30
他的确缺乏手段,去正确地搜索意志。

%31
搜魂容易,搜意艰难,这很正常。

%32
若非如此,智道蛊师常用意志对敌,这些意志一旦被擒,就是授人以柄,把自己的秘密坦露给敌人看了。

%33
正是因为搜意艰难,才会被智道蛊师用来对敌。

%34
至于墨瑶……

%35
“的确是个人物,难怪能青史留名啊。”方源暗中也赞叹。

%36
墨瑶意志的计谋被识破,彻底失败了。连自毁都不可能,被方源关押在仙窍中,死死镇压。

%37
但就这样的绝境,她仍旧没有放弃。

%38
她主动配合方源,为方源解惑,就是展现自己的价值,尽量地拖延时间。

%39
万一哪天方源遇到变故,仙窍破裂,或者被人擒拿,或者战死沙场,墨瑶意志就有了逃生的希望。

%40
墨瑶意志毫不气馁,依旧坚持的气度,让方源也暗自钦佩。

%41
当然,方源若是有正确手法,能够搜意,一定在第一时间动手。只是现在,墨瑶意志的主动配合,更符合他的利益。

%42
“若是真的能寻到一个智道传承,就好办多了。可惜智道神秘艰深,现世的蛊师、蛊仙数量极少。记忆中更没有什么无主的智道传承。”

%43
之前方源还屡次在宝黄天中,收购一些零散的智道蛊虫。算是智道方面的些微进展。

%44
可惜现在,他已然破产,购买力降至谷底,连这些细微的进展都没有了。

%45
至于红莲魔尊的传承信息,他没有问。因为墨瑶意志知道:方源暂时还没有手段来对付她。

%46
双方都是聪明人,强问是毫无结果的。

%47
“主人,仙鹤门的回信。”这时,小狐仙出现在身边,手捧着一只五转信蛊。

%48
仙鹤门忽然终止了石人贸易,不久前方源试探性地出一封信笺询问。

%49
仙鹤门回信得很快。

%50
方源信手取来信蛊,心神探入。

%51
信笺内容措辞强硬。要求方源即日开放胆识蛊贸易,否则仙鹤门将宣布方源为门派叛徒,调动蛊仙攻打狐仙福地!

%52
一年多前。方源动用定仙游,在众目睽睽之下,抢夺了狐仙福地,等若是虎口抢食。

%53
仙鹤门立即反应过来,借助方源和方正容貌相似的特征,对外宣布方源其实是仙鹤门秘密培养的弟子。其他九大派这才承认仙鹤门的胜利。

%54
仙鹤门打的主意,无非是先将其他九大竞争者排除出去。然后自己一人暗自对付方源,企图私自占据狐仙福地。

%55
方源当时乐见其成。

%56
毕竟对付一个仙鹤门,和对付十大门派是两个级别的难度。

%57
但这样一来。方源就等若默认了仙鹤门的谎言——自己是仙鹤门的弟子。

%58
“狐仙福地坐落在天梯山上,就算是十大派也不能随意攻伐,会惹来中洲众怒。但如果他们将我宣布为门派叛徒的话,再来大举进犯。就占据大义了。”方源阴沉着脸。沉思着。

%59
但他并不惊惶,这一层算计,他早在一年多前,就已经料到。

%60
只是没想到,仙鹤门的耐性这么快就被耗尽了。

%61
方源笑了一笑,对小狐仙道:“将太白云生请来。”

%62
太白云生得知这个消息,脸上愁云满布:“这可如何是好?仙鹤门乃中洲十大古派之一,势力之强。还要胜过北原的级部族。他们真要来攻打,单凭我们两个是绝对挡不住的!”

%63
方源便笑道:“老白啊老白。我叫你来,自然是向你请教这个难题。你见了面,倒反问我起来了。”

%64
太白云生啊了一声,脸色有些惭愧,他皱起眉头,尽力思考。终于想到一个办法:“我们不是和黑楼兰结盟了吗?我们可以请她帮助我们啊!”

%65
显然,黑楼兰的强盛战力,已经深深地刻在太白云生的内心深处了。

%66
“黑楼兰虽然是大力真武体,但就算她成功晋升蛊仙,单凭我们三人合力,也难以抵抗偌大的仙鹤门啊。”方源摇头叹息。

%67
“这可如何是好?这可如何是好?”太白云生来回踱步,苦思冥想。

%68
方源看着他,笑而不语。

%69
忽然太白云生停下脚步,一拍脑袋:“我想到了!恩师不是叫师弟你加入僵盟吗?我们完全可以加入僵盟,利用僵盟的势力来威胁仙鹤门啊。”

%70
但方源又摇头,道:“僵盟结构松散,中洲的僵盟分部远远不如仙鹤门,毫无威胁可言。而且,我若是借力僵盟,我就要出让利益。恐怕不是仙鹤门,而是僵盟第一个成为受益者了。”

%71
“这也不行啊,那我们该怎么办?”太白云生抬头看向方源,神情苦闷忧虑。

%72
但当他看到方源嘴角的笑意时,他愣了愣,终于反应过来,手指着方源笑骂道:“好你个师弟,竟然诓骗师兄,看师兄的好戏!你这样胸有成竹,显然是早有良谋啊。”

%73
“哈哈哈,老白你法眼如炬,终究骗不过你。”方源大笑三声,坦言承认。

%74
“你有什么好方法,还不赶紧说来?”

%75
方源便道:“这事情原委,我之前已经告诉你了。仙鹤门将我认作门派弟子,是因为他们害怕其他九大派插手。这是他们的软肋。我正可以利用这点,进行突破。”

%76
“具体该怎么做呢?”

%77
方源也不瞒他:“我已打算,向灵缘斋递交一封信,先初步建立联系。”

%78
之前方源向墨瑶打听灵缘斋的情报,便是用在此处。

%79
太白云生又担忧起来:“现在才建立联系,又单靠一封信,会不会太晚?而且十大派相互竞争,又相互合作。灵缘斋会不会拒绝你的信,将信交给仙鹤门示好呢?”

%80
“当然不会,因为我的这封信……会很特别。”方源显得自信十足。

\end{this_body}

