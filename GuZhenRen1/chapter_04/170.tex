\newsection{此回福地须假戏}    %第一百七十节:此回福地须假戏

\begin{this_body}

ps:想听到更多你们的声音,想收到更多你们的建议,现在就搜索微信公众号“qdread”并加关注,给《蛊真人》更多支持!

蛊仙穿透界壁,本来千辛万苦,分外艰难。\&\#26825;\&\#33457;\&\#31958;\&\#23567;\&\#35828;\&\#32593;\&\#119;\&\#119;\&\#119;\&\#46;\&\#109;\&\#105;\&\#97;\&\#110;\&\#104;\&\#117;\&\#97;\&\#116;\&\#97;\&\#110;\&\#103;\&\#46;\&\#99;\&\#99;]

但每年到了这个时间,因为地潮,东海的界壁变得孱弱虚薄,都会有中洲、北原、南疆的蛊仙,趁机进入东海。

当然,也有少部分的东海蛊仙,趁机进入中洲、北原、南疆。

所以,中洲、北原、南疆的蛊仙,来到东海并不是一件奇怪的事情。

唯有西漠的蛊仙,才是在东海真正罕见的。因为西漠和东海并不接壤,中间隔着中洲。西漠蛊仙一旦出现在东海的话,肯定吸引眼球。

天下五域,中洲、北原、南疆、西漠、东海。其中中洲综合实力最强,蛊师门派无数。南疆十万大山,山峦叠嶂。北原草地平坦无垠,部族流动性最强,最为好战。

而东海,则是资源第一丰富。广袤的海域,深邃的海水中,藏着无数的鱼虾海兽,无数的矿产珍石。

中洲、北原、南疆的蛊仙来到东海,大多都是奔着这些资源而来,情况普遍。尤其是北原,久战之地,资源最为贫瘠,北原蛊仙来到东海寻求发展机遇的,或避风头的,还不在少数。

太白云生继续道:“托二位的福,我来东海发展的挺不错,因而仙窍渐渐不稳。地潮来临。我正要趁此良机,回去北原,落下福地,稳固仙窍,迎接地灾。”

“是这样啊。”鲨魔点点头。

“我们早该想到了。太白先生此去,可要注意安全。或许让我们夫妇护送一程?”苏白曼微笑着提议。

太白云生摆手:“不必了,不必了。”

鲨魔夫妇见太白云生拒绝,便不再提。

尽管因为地潮,导致界壁薄弱。但薄弱之处,却也有无形的潮力盘踞,宛若航道中的无数暗礁旋流。还是很危险的。通过那里。必然要有一份正确的路线图。

这种线路图十分珍贵,有智道蛊仙推算还好,若没有智道手段,就只能凭借自身不断试探闯荡。

若是让鲨魔夫妇护送的话,那么太白云生手中掌握的线路图岂不就是泄露了吗?

因此,鲨魔夫妇见太白云生拒绝,也不方便强求。[www.mianhuatang.cc 超多好看小说]

“北原处理妥当。我便会趁着下一次地潮回来。说不定还会劝说几位好友,也过来闯荡。”太白云生又开口,为接下来方源的出场打下伏笔。

“好!太白老弟的朋友,就是我们夫妇的朋友,我们当然十分欢迎的!”鲨魔哈哈一笑,十分豪爽热情。

太白云生这才告辞。

望着太白云生离开的背影已经缩成蚂蚁大小,苏白曼这才皱起眉头,向鲨魔传音:“界壁虽然薄弱,但地潮恐怖,稍有大意。被无形的潮力拍中,伤筋动骨只是小事,横死当场都有可能。你看太白云生他这么从容,必然对其手中的路线图十分自信,咱们真的不跟踪过去吗?有了这路线图,我们也能进入北原,就算我们用不着。也能上缴给僵盟,换取大量贡献分,减少此次的损失。”

鲨魔摆手,态度坚决地拒绝道:“不必了。路线图也不是恒定不变的,几次地潮之后,无形的潮力就会产生巨大的变化,到那时旧的路线图就失去了价值。我们的主要目的,还是为了和他合作,利用人如故仙蛊,达成重生的目的。这是长远大计,可不能为了眼前小利,而恶了我们和他好不容易建立起来的良好关系。”

“好吧,我听你的。”苏白曼立即打消了自己的想法,她眸光温柔地看向自己的夫君,僵尸的面容却有些恐怖。

鲨魔毫不介意,深情地将苏白曼揽在怀中,让她偎依在自己的胸膛上,并柔声安慰道:“我知道夫人担忧的什么。不错,攻略玉露福地的任务是我们领的。此次动用了玄冰屋,我们损失很大。但没有关系,活在这世间,谁能没有挫折和失败呢?我们还有底子,可以支撑。现在又看到重生的希望,不要怕,不管将来如何,有多艰难,我会一直守护在你的身边。”

“夫君……”

东海碧波万里,天空晴朗,远处海天一色,时而有白鸥只只飞舞。

太白云生一路飞行,划破长空。

他身边萦绕着云雾,飞行得很快,但却悄然无声。当然,这只是凡道飞行杀招,和方源的三对返实蝠翼,并不能相比。一旦方源身上的铁冠鹰力蛊爆发出来,更能在一瞬之间将其甩得老远。

不过这个杀招,还有个优点,就是云雾包裹自身,凡人看不出来。

这不是五域大战时期,蛊仙和凡人之间还是有鸿沟。大部分凡人蛊师都只听闻仙人,从没亲眼见过,有很多人甚至只以为仙人只是个飘渺的传说。

太白云生毕竟是北原蛊仙,在东海而言,只是外来户,因此行事低调。

起先受到方源的指点,他在海市福地接下救治海域的任务,闯出点名气之后,就和鲨魔等人接触。每一次攻略玉露福地,基本上都有他。鲨魔夫妇待他甚厚,当然是图他的人如故。鲨魔手中掌握着宙锚仙蛊,和人如故配合起来,就是重生的希望。

当然,鲨魔夫妇也不是没想过,抢夺仙蛊人如故。

但除非是有盗天魔尊般的手段,否则要想抢仙蛊,几乎是不可能的事情。

东方长凡那么虚弱,方源占据的优势那么大,到头来,东方长凡一念之下就是自爆,让方源最终一只仙蛊战利品都没有得到。抢夺仙蛊的难度。便就可见一斑了。

太白云生飞了半天功夫,终于见得在海平面上现出一块“灰斑”。

随着距离越加缩短,“灰斑”在他的视野中渐渐扩大,原来是一座小型海岛。

这座小岛上没有一口元泉,植被稀少。基本上都是光秃秃的灰褐岩石,就连海鸟都很少光顾。在东海这种资源丰富的地方,属于穷山恶水了。

不过太白云生乃是蛊仙,也不在意,只是当做临时居所。

一座孤岛,穷山恶水,自然无人觊觎。少了很多关注和麻烦。

太白云生从高空中徐徐飞降下去。进入海岛。

他首先检查埋设下来的蛊阵,看看有无变化。

海岛上还是有些蛊阵的,但都是他的粗浅的布置,简单的监控是一层意思。若是碰到其他蛊仙,蛊阵便昭示此地并非无主之地,则是第二层意思了。

检查了一番之后,太白云生手中便多了近十只蛊虫。

这些都是信道蛊虫。承载着信息。

原来,太白云生虽然行事低调,但身为宙道治疗蛊仙,手中拥有的仙蛊,实在是太过优秀,宛若囊中之锥,不可久藏。

这些信道蛊虫,来源于各方之手,有的是超级势力,有的是魔道中人。有的是仙僵,有的是正道散修。大部分都是邀请太白云生出手,救治某些事物。

少部分不是请太白云生当即出手,也是想结下善缘。万一将来有个什么不测,这个善缘不就有用武之地了吗?

太白云生本就是老好人的性情,又有两大宙道仙蛊,在东海很受欢迎。

太白云生将这些信道蛊虫揽在手中。步入洞窟。

这洞窟不过是他临时开辟出来,并没有花费多少心血,因而显得十分粗陋简朴。

太白云生首先将这些信道蛊虫,一一回复,婉拒了各方邀请,然后便陷入沉思。

“我手中虽然掌握着星门,可以很方便地回到狐仙福地。但此时情况,却不能直接开启星门了,这样做是要坏事的。”

太白云生现在算得上是一方名人,举动都受到很多人的暗中关注。

因此他此番要回到狐仙福地,比以往更加麻烦。还真得要先去界壁附近假戏真做,再趁着无人关注,最后动用星门。

狐仙福地。

方源这一研究智道传承,便研究了三天三夜。

脑海中的念头不知道消耗了多少,便有了许多心得。

“东方长凡才华横溢,这份智道传承本以推算事物为主,不擅长对战。但他设想出来的仙道杀招万星飞萤,却是独树一帜,恰恰利用了传承的推算优势,几乎将其完美地转化攻势。纵览整个传承,万星飞萤乃是当之无愧的第一攻伐大杀招了!”

这是方源心中最大的感慨。

他的性情又和太白云生不同,带着浓厚的侵略性。

首先关注的,便是传承中的进攻手段。

这万星飞萤又如何引得方源如此感慨呢?

原来这招根基,仍旧在于星念蛊。每一只星光飞萤,都是包裹着一颗星念。

施展这招,耗费惊人。

不止是仙元,还要消耗蛊仙身上的智道道痕。

蛊仙身上怀有道痕,方源身上力道道痕,炎道蛊仙身上便是炎道道痕。东方长凡身上自然是智道道痕,每次施展这招,都要损耗至少十六条的智道道痕。

须知道痕一物,分外难得。

蛊仙要得道痕,除去稀罕独到手段之外,最正统最主流的就是渡天劫地灾。每渡过一次,就能在身上落得道痕。

道痕多寡,也视天劫地灾的强弱而定。天劫地灾越强,道痕便越多。(小说《蛊真人》将在官方微信平台上有更多新鲜内容哦,同时还有100\%抽奖大礼送给大家!现在就开启微信,点击右上方“+”号“添加朋友”,搜索公众号“qdread”并关注,速度抓紧啦!)(未完待续。)

\end{this_body}

