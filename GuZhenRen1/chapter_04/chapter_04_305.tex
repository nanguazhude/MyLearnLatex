\newsection{蓝}    %第三百零六节:蓝

\begin{this_body}

%1
中洲,灵缘斋,议事堂。

%2
门派中的十五位蛊仙,有超过半数,都是真身降临。

%3
这是相当罕见的一幕。

%4
门派议事每隔一段时间就会有,蛊仙们只需要留下一段意志或者感情,进行商议即可。

%5
但现在,却有八位蛊仙亲自参加这次的议事,其中就包括灵缘斋唯二的两位八转蛊仙。

%6
造成这个现象的原因,就是凤九歌的失踪。

%7
灵缘斋的蛊仙们都感到了未来的动荡和不安。

%8
“这一次,将大家召集过来,是有一个重要的消息要宣布。”灵缘斋太上大长老,端坐主位,目光缓缓扫视四周,声音低沉。

%9
灵缘斋和其他九派,有一个明显的区别。

%10
那就是女仙众多。

%11
灵缘斋中的女蛊仙有十位,男蛊仙只有五位。

%12
灵缘斋的太上大长老、太上二长老都是八转女仙。

%13
此刻,堂中众仙的目光都集中在太上大长老的脸上。

%14
太上大长老面无表情,不过她身旁的太上二长老,却是阴沉着脸。

%15
蛊仙们察言观色,都产生了不妙的预感。

%16
果然,太上大长老接下来的一句话,好像是一块沉重的巨石,砸在众人的心坎上。

%17
“已经确定,凤九歌陨落在北原。他死于大同风中,没有任何遗物,只有两个血字的留言。”

%18
众仙心中皆是一沉。

%19
亲自参加此次议事的白晴仙子,更是脑袋猛地眩晕,脸色倏地惨白一片。

%20
哪怕有心理准备。但真正听到这个噩耗的时候,众人还是有些不可思议。

%21
强大如凤九歌这样。居然折损在了北原。相反比他实力低的那些蛊仙,却大多生还。回归了门派。

%22
说实在话,凤九歌起行时,没有人会料到竟是这样的结果。

%23
长久以来,凤九歌强大的形象,已经深入人心。他是灵缘斋的招牌,甚至成了一个象征。

%24
如今他一死,众仙心中都有些失落和空虚。

%25
哪怕是倒凤派系的徐浩、李君影,都有这样的感觉。

%26
太上大长老继续道:“你们面前的信道蛊虫中,就有此事的详细记录。都先看看罢。”

%27
蛊仙们纷纷向蛊虫中,探入心神。

%28
“唉,凤九歌大人死在大同风中,也不算辱没他的名声了。”良久,一位蛊仙出声,打破了堂中的沉默。

%29
白晴仙子紧闭双眼,身躯都在微微颤抖。强烈的悲伤和痛苦,宛若汹涌的海啸,将她吞没。

%30
她是如此的深爱着凤九歌。同样的,凤九歌也是如此爱她。

%31
她的脑海中,浮现出临行送别时的画面。没想到那竟然是她看夫君的最后一眼!

%32
命运弄人。

%33
如今,我生你却死。我身在中洲,你却陨落于北原!

%34
白晴仙子不敢睁眼,她怕睁开双眼。眼泪就会止不住地滚落下来。

%35
她努力去想自己的女儿凤金煌,在心中不断地告诫自己:“白晴啊。白晴,你要坚强。这个时候。绝不能让别人看到你软弱的样子!”

%36
她深呼吸几口气,慢慢的睁开双眼。她的眼中,已经湿润,瞳孔上带着血丝。

%37
此时,堂中众仙已经谈论到凤九歌最后的血字遗言。

%38
“凤九歌临死之前,在手中书写了‘薄青’二字,究竟想表达什么意思?”

%39
“在我看来,这个线索相当重要。想必是凤九歌在死亡的巨大压力之下,猜测领悟到了什么。可惜他见到赵怜云时,已经油尽灯枯,没有力气再多说什么,便给我们留下关键的线索。”

%40
“凤九歌在北原调查八十八角真阳楼倒塌的真相,和薄青有什么关系?”

%41
“凤九歌、薄青这两个人很相似。当薄青比凤九歌要强大得多,他是当年的中洲巅峰,就连天庭蛊仙都要退让垂首。那个时候,是我们灵缘斋最辉煌的时代!当时,很多人都看好他,认为他能够成为剑道仙尊。可惜最终,他还是失败了。”

%42
“薄青的情报,我们也知晓的。我只想知道,凤九歌为什么在临死前,独独留下薄青二字?他究竟想要表达什么?”

%43
堂中沉默了一下,一位蛊仙开口道:“诸位忘了吗?在之前的情报中,凤九歌的对手秦百胜,就施展出五指拳心剑。而这个杀招,就是薄青所创,也是他的招牌杀招。”

%44
“凤九歌的意思,是想说八十八角真阳楼倒塌,和薄青有关系?”

%45
“依我推测,他应该是觉得,对方居然掌握了五指拳心剑,多多少少会和薄青有所联系吧。而薄青乃是我灵缘斋的蛊仙,我们调查起来会更有优势,这是个重要的线索。”

%46
“的确如此。薄青当年彻底陨落在天劫之下,连骨灰都没有。他的杀招,怎么会被一位北原蛊仙掌握呢?”

%47
蛊仙们又议论了一阵,众说纷纭,但都没有什么靠谱的说法。

%48
太上大长老缓缓地抬起手,她的这个动作让堂中迅速安静下来。

%49
“不管如何,薄青的事情需要调查。这项任务就交给你负责吧,白晴。”

%50
听到太上大长老忽然点自己的名字,白晴仙子立即转头,看向太上大长老,连忙应下。

%51
这是她夫君死前留下的遗言!

%52
白晴仙子定然会竭尽全力,去调查这个线索,追溯真相。

%53
“凤九歌的死讯,只限我们知晓,尽量隐瞒。谁若透露出去,就以背叛门派论处!”太上大长老冷喝一声,“接下来,我们着重商议一下门派在中洲各地的部署。”

%54
凤九歌的死,带给灵缘斋的影响是很恶劣的。

%55
灵缘斋虽有两位八转蛊仙,但不到门派存亡关头。这两位八转是不会轻易动手的。

%56
这其中的原因有很多。

%57
其一,八转这个修为。已经注定蛊仙如履薄冰,战战兢兢。全心全意应对灾劫。稍有大意,在争斗中折损实力,就会陨落在恐怖的浩荡灾劫之中。

%58
其二,中洲十大古派拥有同一个源头,那就是天庭。天庭既在,十大古派的争斗,就永远不会升级,不需要八转蛊仙出手。

%59
所以中洲蛊仙界,或者说整个五域的蛊仙界中。真正活跃的还是那些七转、六转蛊仙。

%60
而在七转中,无敌中洲的凤九歌,对于门派的重要性,就不言而喻了。

%61
正是因为有他的存在,灵缘斋的版图才扩张到这么大的地步,才占据了无数珍稀的资源点。

%62
凤九歌一去,灵缘斋对其余实力的震慑力大为降低。门派掌握的种种资源,就像是一块块鲜美的肥肉,勾来无数贪婪的目光。

%63
“玄武山脉资源丰富。是仙材宝库,至少需要一位七转蛊仙坐镇。”

%64
“金沙窟的开采,已经到了关键时刻。我们投入了大量的资金和精力,正是要收获的时节。不应该放弃。”

%65
“轮回战场才是重中之重啊……”

%66
蛊仙们都感到头疼,版图太大,但蛊仙战力就这么多。就算把他们掰成两半来使。也是不够的。

%67
直到此刻,他们才真正意识到。凤九歌的威望,对外界的威慑。是多么的强大。

%68
白晴仙子默不作声。

%69
蛊仙们讨论着这些仙材、资源,喊叫不休,再不提起凤九歌。

%70
好像凤九歌的最后存在,就是刚刚的那段对“薄青”二字遗言的讨论。

%71
白晴仙子不免有许多悲凉之情:“夫君啊夫君,你为门派做了这么多贡献,到头来,这些人转眼就将你抛之脑后了。”

%72
整个议事的过程中,白晴仙子都不在状态。

%73
众仙看在眼中,也都理解她,一贯严厉的太上大长老都没有批评什么。

%74
唯有当众仙提到赵怜云时,白晴仙子双眼一亮,对此密切关注。

%75
凤九歌若在,凤金煌几乎可以肯定,就是灵缘斋的当代仙子了。但如今凤九歌一去,赵怜云横空出世,成了凤金煌的巨大威胁。

%76
白晴仙子当然更爱护自家儿女,对赵怜云之事就很上心。

%77
只听蛊仙们不断讨论:

%78
“赵怜云一连继承了两个盗天真传,她有什么变化吗?”。

%79
“天外之魔不可信任!”

%80
“神不知、鬼不觉,这两道顶级的仙道防御杀招,我们还在研究当中……就以目前来看,高深莫测!这两道仙道杀招,已经在赵怜云的魂魄上形成了两层密实的道痕丝衣。这种奇妙的道痕运转,我还从未见过!”

%81
“这两层道痕丝衣,会一直保护着赵怜云。我们用了许多法子,尝试推算她,都没有效果。我们堂堂蛊仙,推算一个凡人,都推算不出来。若不是我亲身经历,绝不会相信。”

%82
“你们发现没有?神不知、鬼不觉是顶级的防御杀招,居然不耗损仙元。就好像是鸿运齐天蛊一样。尊者的境界,真是难以理解啊。”

%83
“隔绝念头、意志、情感的推算,这是神不知的防御效果。那么鬼不觉呢?”

%84
“还不太清楚,正在多方尝试之中。哦,整个过程赵怜云都很配合,我觉得她虽然是天外之魔,但很识时务,完全可以培养。”

%85
“这个小家伙一心想要救自己的情郎呢。呵呵,可惜根据情报,雪胡老祖似乎已经筹集到了足够的仙材,不久之后就要正式炼蛊了。”

%86
最后,太上大长老为此事总结:“继续研究,另一方面,也要多加培养赵怜云对我派的归属感。我很期待她的未来!”

%87
与此同时,在一处无名的森林中。

%88
七星子仙僵正透过一面光镜,和他人通信。

%89
光镜中是一道模糊的老者身影。

%90
他徐徐开口道:“所有的布置,都已经基本到位。但天庭已经将完成了这一次的宿命蛊修复,你那边必须先行启动。”

%91
“明白。”七星子仙僵沉声答道。

%92
“要小心,蓝副使。”光镜中的身影又道。

%93
七星子没有再说话,他停止催动这个信道仙级杀招,光镜骤然消失。

%94
随后,他头也不回,转身而去,身影迅速没入森林之中。

\end{this_body}

