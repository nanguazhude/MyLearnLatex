\newsection{仙蛊易主}    %第一百二十九节:仙蛊易主

\begin{this_body}

%1
十几天后,狐仙福地。

%2
时运仙蛊,静静地悬浮在方源的头顶上,并且不断地游走着。

%3
时运仙蛊宛若一截玉石蜈蚣,通体浅绿,此时在方源头顶的一小片空间中,缓缓蜿蜒游荡,散发着如水般的光泽。

%4
方源一边向其灌输着青提仙元,一边视察自家仙窍内的蛊虫寸光阴。

%5
之前,他和琅琊福地达成协议,委托琅琊福地中的毛民炼制大批的寸光阴。琅琊福地的时光流速比狐仙福地更快,里面的毛民质量之好,可以算是天下第一等。

%6
因此,在方源付出仙材报酬之后,大批的寸光阴蛊虫,宛若一座小巧的金山,堆积在方源的仙窍当中。

%7
一寸光阴一寸金。

%8
寸光阴蛊只是凡蛊,形体宛若一根金条。寸光阴转数越高,金条便越粗越大。

%9
然而此时,随着方源催起时运仙蛊,寸光阴蛊虫损耗极剧。原本的金山,在几个呼吸的功夫,就彻底荡然无存,消散一空。

%10
没有了寸光阴的支持,头顶上的时运仙蛊,倏地停下。

%11
方源遗憾一叹:“看来用寸光阴替代寿命,来催动时运仙蛊,还显得过于勉强啊。才催动这么一点时间,运道上的获益近乎于无。”

%12
就像黎山仙子之前所言,催动时运仙蛊,消耗的是仙元还有蛊仙本身的寿命,换取运道上的增幅。

%13
但方源此时。寿命无多,怎么可以轻易浪费?

%14
他在王庭福地大战的最后阶段,被迫转化为仙僵之躯时,已经老态龙钟。他自身也没有用过寿蛊,也就是说平均一百岁的寿命。如今只剩下十多年或者十年以下。

%15
若是有朝一日,他成功地摆脱仙僵之躯,恢复健康人身,所剩寿命也是不会变化。

%16
“也就是说,哪一天我恢复过来,身躯便会变得老迈沧桑,这倒是个麻烦。除非在此之前。我搜寻到寿蛊。用在自己身上。”

%17
方源想到这里,不禁微微皱眉。

%18
寿蛊难寻得很,且又十分珍稀。

%19
通常蛊仙得到手中,从不拿出交易。即便是不久前结束的北原拍卖大会,也没有一只寿蛊抛出来卖。

%20
寿蛊只是凡蛊,但产出条件不定,蛊仙搜索寿蛊。几乎完全是撞运气。因此很多情况下,寿蛊往往被凡人蛊师所得,蛊仙无法掌控。

%21
方源前世能够找到寿蛊,增添了数百年寿命,也是此类情形。

%22
“就目前为止,我所看到的搜寻寿蛊的最好方法,便是八十八角真阳楼。可惜我在此座仙蛊屋中,也只得到一只寿蛊,还给太白云生用了。要重现八十八角真阳楼,基本上是不可能了。”

%23
方源念头一动。时运仙蛊便化作一道流光,顷刻间钻入他的仙窍。

%24
他的仙窍虽然已成死地,每隔一段时间崩解一块,但对于蛊虫而言,还是显得空间广阔。

%25
时运仙蛊最终轻飘飘地落在另一处“金山”边上。

%26
这二座“金山”,当然还是寸光阴蛊虫堆起来的。

%27
方源从琅琊地灵那边,得到大量的寸光阴蛊虫。堆成两座小山。其中一座已经在刚刚,消耗光了。

%28
还剩下一座,就是这座。

%29
虽然刚刚试验的效果,很不理想,但同样带给方源不少珍贵有用的灵感。

%30
“刚刚只是寸光阴和时运仙蛊两者搭配,若是增添其他蛊虫辅助,甚至形成仙道杀招,说不定能大大降低寸光阴的剧烈消耗,从而使得效果理想。可惜我虽然可以借助智慧光晕,灵感无限。但这种推演涉及到宙道境界,却是我的空白短板。”

%31
这方面受挫,方源只好暂时收起心思。

%32
梦道凡蛊仍有缺口,方源没有丝毫放松偷闲的意思。缓了几口气候,他仍旧盘坐在床榻上,缓缓闭上双眼,这便沉入梦境。

%33
一连渡过三个梦境,炼成了两只梦道凡蛊,均存入脑海之中。

%34
这样的成功率,已经和最初形成鲜明的对比,有着很大的提高。不过在第三次炼蛊时,方源遭受失败,因而魂魄受了些损伤。

%35
梦境本身就有风险,更何况在梦中炼蛊呢?

%36
“稍不留神,就踏中了梦中陷阱。幸好我有胆识蛊可以壮魂。”方源心中庆幸,睁开双眼,出了荡魂行宫。

%37
荡魂山上,胆识蛊俯拾即是。

%38
方源用了几颗,便将魂魄损伤治愈。

%39
正当他要赶回去,继续炼梦道凡蛊的时候,忽然神色一动,一份信笺通过推杯换盏蛊,传入他的仙窍。

%40
方源浏览一番后,脸上闪过一抹喜色。

%41
当即打消了继续炼蛊的计划,在荡魂山上选择了一处风景宜人之地,就地盘坐下来,取出一块象腿,催动火蛊进行烧烤。

%42
片刻后,一道身影从东南方向直射而来。

%43
来者是一位女仙,黑发黑瞳,绝美神姿,逸散凌人霸意,正是黑楼兰。

%44
黑楼兰飞到半空,徐徐降落到方源身边,脸上冷漠。她瞥了瞥方源手中的象腿,立即认出这是荒兽旱象身上的骨肉,不由讥讽道:“原来你手中仙材,已经多到可以这样糟蹋了么?”

%45
方源哈哈一笑,抬眼看了看黑楼兰,又将目光落在象腿上。

%46
这么一会儿工夫,火力充足的缘故,象腿已经烧烤半熟,逸散出浓郁肉香。

%47
“楼兰仙子别来无恙,请坐请坐。”方源客气地道。

%48
黑楼兰冷哼一声,仍旧站着,无动于衷,隐藏着怒意。

%49
方源心中好笑:对方虽然是霸主枭雄,但到底还是个女人啊。不过其实设身处地想想。若是别人索要自己的春秋蝉,自己给不给?怕不是早就翻脸,直接干上了吧?

%50
想到这里,方源理解的同时,又暗暗警醒自己。

%51
黑楼兰的错误就在眼前。不可不察。

%52
黑楼兰一心复仇,冒险渡劫,增长修为,结果一着不慎,差点身死。如今终于熬不过,将本命仙蛊我力送了过来。

%53
到了这个阶段,切不可轻敌冒进。蛊仙灾劫。岂是那么容易渡的?哪怕在才华横溢。心高气傲,也得认清现实。黑楼兰初生牛犊不怕虎,一味勇猛精进,这下吃了苦头,以后应该就会收敛了。

%54
“不过也亏了她,不然我哪里能收获我力仙蛊呢?”方源想到这里,心里一笑。将象腿送上嘴边,张开大口撕咬一块,吃的有滋有味。

%55
黑楼兰见此一幕,冷笑连连:“你已经成为僵尸,早已经没了味觉。如今恢复人身无望,只有借此怀念曾经的岁月么?”

%56
方源慢条斯理,悠然地道:“楼兰仙子此来,不是专程来看着我吃肉的吧?”

%57
黑楼兰面沉如水,语气中隐含怒恨,意有所指地道:“你这般贪吃。不怕被撑死吗?”

%58
“你放心,我胃口好。撑死好过饿死。”

%59
“哼!”黑楼兰气得狠狠咬牙,努力压抑心中怒意,缓了缓语气道,“其实换换口味,又如何不可?这样一来,大家都有肉吃。命运无常。谁没有落魄的时候呢?”

%60
方源知道她的意思。

%61
黑楼兰是想用力气仙蛊,替代我力,当做报酬交给方源。但方源绝不会答应。

%62
他大口吞咽下嘴里的食物,深深地叹了一口气,感慨道:“正是因为命运无常,我才要千方百计地变强,好应付将来落魄的处境。黑楼兰你是懂我的,我意已决,不会更改。”

%63
黑楼兰双眼微微眯起,方源是这样的人,她同样也是这一类人。她刚要开口说话,忽然瞳孔一缩。

%64
只见方源身上,猛地有一道光线,缓缓浮现而出。

%65
这道光线,横亘在他的胸口,起先只是虚影,随即越来越亮,凝练成实,深深地刻印在方源的身上。

%66
“力道道痕?!”黑楼兰不禁微微变色。

%67
就在刚刚,她亲眼目睹了方源增添了一道力道道痕。

%68
她很快镇定下来,联想到了答案,眼睛隐隐抽搐了几下,语气都有些干涩:“原来北原拍卖大会中出现的吃力仙蛊,就在你的手中。怎么好东西都落到了你的手里?”

%69
吃力仙蛊,对于其他蛊仙而言,应用价值很低,甚至不敢多用。

%70
但对力道蛊仙,却是名副其实的宝贝。

%71
尤其是黑楼兰这种情况,修行遇到瓶颈,不能通过渡劫大幅度增长。这时,毫无运用风险的吃力仙蛊就更显得举足轻重。

%72
方源没有搭话,而是对准手中的骨头下口。

%73
象腿上的肉已经被他吃完,但骨头也蕴含着天然的力道道痕。方源牙齿尖锐,象腿骨虽然坚硬,也在他的利齿下被咬成骨渣,大口吞入腹中。

%74
须臾之后,方源的身上又增长一道力道道痕。

%75
他根本不怕黑楼兰反悔。

%76
皆因签订救人协约时,就已经规定了时限。黑楼兰卡着最后时限过来,只是想尽最后一份努力。

%77
但这事情,从一开始,方源就立于不败之地。

%78
一旦违背协约,黎山仙子就会应誓而亡。黑楼兰怎么可能看着这样的事情发生?

%79
“来,坐吧,我请你吃肉。”方源从仙窍中,又掏出一只力道荒兽的大腿。

%80
烤熟之后,他将大骨头递给黑楼兰。同时念头一动,吃力仙蛊就飞到了黑楼兰的手中。

%81
黑楼兰毫不客气地接过,吞下肉和骨头,然后催动吃力仙蛊辅助消化。

%82
片刻后,她的身上也增添了一道力道道痕。

%83
虽然和渡劫的收获相比,千分之一都不到。但此法胜在毫无风险,且能不断积累。

%84
吃到一半,黑楼兰深深地叹口气,将我力仙蛊交到了方源手中。

%85
两人当场完成交割,我力仙蛊至此易主!

\end{this_body}

