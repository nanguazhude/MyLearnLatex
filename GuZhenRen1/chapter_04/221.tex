\newsection{自由蛊}    %第二百二十二节:自由蛊

\begin{this_body}

ps:看《蛊真人》背后的独家故事,听你们对小说的更多建议,关注公众号(微信添加朋友添加公众号输入dd即可),悄悄告诉我吧!

《人祖传》中记载,人祖走在自己的人生之路上,和毛民的首领分别之后,他在绿天中遇到了的羽民。

人祖看到这些羽民,十分高兴。

因为羽民的背后生有双翼,羽民听得懂人祖说的话,双方可以沟通。

人祖便向羽民们请求:“羽民啊,请你们帮助我。我的女儿落在了平凡的深渊里面,出不来。请你们飞下去,把我的女儿带出来,让我们父女团聚吧。”

羽民们哈哈大笑:“人啊,你怎么可以命令我们。我们羽民是天地下最自由的生命,谁也不能命令我们,不能拘束我们。你让我们听你的请求,按照你的意愿行事,这是不可能的。”

不管人祖如何劝说,甚至哀求,羽民们都自顾自地飞翔,享受自由自在的快乐。

人祖说得嘴皮子都发干,但羽民们只当他是一个小丑,看笑话,并且嘲笑人祖。

“看啊,这是一个多么可怜的人啊。”

“就算是他是天底下最有灵性的生命,又能如何呢?”

~长~风~文~学,∽→◆t

“他没有翅膀,只能徒步行走,真是可悲。幸亏我不是人,而是羽民。你们看,我飞得多矫健。”

……

人祖的孤独之心中,自己蛊听到这些话,越来越生气。终于憋不住,主动跳了出来。

“你们都给我下来吧。”自己蛊发威。一下子就将所有的羽民,从天空中都活捉了下来。

人祖看到这个景象。不由地大吃一惊:“自己蛊,你怎么变得如此厉害?”

自己蛊不无骄傲地道:“那是当然的。我啃下了一口力量蛊,有了自己的力量,具有很强的威能了。而且使用这些力量,不需要任何的代价。人啊,你得知道:只有自己的力量,才是最可靠的。而没有力量的自由,都是虚假的。”

羽民们被擒拿活捉,再也飞不上天空。被压在地面上,动弹不得。

羽民们气得纷纷张口大骂人祖和自己蛊。

人祖叹了一口气:“羽民啊,我不是有意要来冒犯你们的。请原谅我的莽撞。我只想请你们来救我的女儿。等我们父女团聚之后,我一定会酬谢你们的。”

“这不可能!我们羽民最自由!”

“我们就算身体上失去了自由,心灵仍旧是自由的。”

“不错,不错!”

“残暴的人啊,你的意志不能强加在任何一位羽民身上。”

羽民们皆大喊大叫,一点妥协的意思都没有。

人祖苦劝了三天三夜,都没有效果。无奈之下,他只有将这些羽民都重新放生。

自己蛊不忿地道:“人呐,你怎么就这样轻易地放走了他们?你难道不想救出你的四女儿了吗?”

人祖却很有信心地答道:“我已经看出来了,这些羽民咱们不能硬来。这些天和羽民们相处。我已经发现,这些羽民虽然自由自在,但没有巢穴可以躲避风雨。没有足够的食物果腹。想要让他们帮忙的话,就得让他们自愿行动!”

于是人祖在羽民的附近。搭建了房屋,每天依靠自己蛊。采集很多很多的野果,猎杀许多许多的兽肉。

羽民们很快发现,房屋的安全和温暖。尤其在狂风暴雨的天气里,羽民们只能躲在树丛中,瑟瑟发抖,忍饥挨饿。而人祖却窝在房间里,享受着温暖的壁炉和丰盛的食物。

在一个寒冷的雪夜,一些羽民,悄悄地跑到人祖房屋的屋檐下,躲避风雪的同时,贪婪地吸收着从门缝中泄露出来的温暖。

人祖便戴起态度蛊,主动打开房门,用十分热情的态度,邀请这些羽民进屋,与他共享温暖和食物。

这样的次数多了,前来人祖这里的羽民也越来越多。

人祖十分好客,每天都招待羽民,甚至将最靠近壁炉的位置,都让给羽民。羽民不管吃多少食物,都尽管让他们开吃。

羽民们渐渐习惯了这样的生活,终于有一天,人祖见时机成熟,便摘下态度蛊,露出真正的冷漠表情。

他将屋子的大门紧闭,将食物也都收起来,不再无偿地供给羽民。

羽民们猝不及防,都慌了神。

他们已经习惯了房屋的安全和温暖,习惯有充沛的食物,很少飞到空中去,更很少去狩猎。他们的食物也积存的很少很少。有很多羽民,甚至胖得飞不动了。

人祖这么做,羽民们也拿他没有办法。他们打不过拥有自己蛊的人祖。

很快,羽民们都面临着要被饿死或者冻死的结局。

几天后,不少羽民都死了。

人祖左等右等,也不见羽民主动向他妥协,十分心焦。

终于,羽民们都死了大半,人祖不得不旧事重提:“羽民啊,只要你们扇动双翼,飞下平凡深渊,将我的女儿救上来,我便给你们充足的食物,还有温暖的房屋。”

哪知剩下的羽民,都摇头拒绝。

最终他们都死了。

从这些羽民的尸体上,飞出许多的小虫。它们就好像是粒粒微小的光点,闪烁着五颜六色。人祖伸手要捉,却怎么也捉不住。

“没有用的。”思想蛊这时告诉他,“野生的自由蛊是捉不住的。这些羽民从一出生起就追逐自由,可惜只有到了死后,才能得到自由和解脱。”

中洲,狐仙福地,荡魂行宫。

方源缓缓抽回自己的怪爪,羽民蛊仙郑灵的魂魄。则萎靡不堪,瘫软如泥地悬浮在空中。

蛊仙周中。因为主动违反协约,已经身灭魂散。

至于先死一步的郑灵。反倒是因为只受到方源的毒道、力道杀招,肉身毁灭,魂魄保留了下来。

他的魂魄,在当场就被方源暗中收入仙窍里了。

自古以来,魂道、智道、奴道本就息息相关,方源自从得到了东方长凡的智道传承,触类旁通之下,在魂道、奴道上的造诣也提升了不少。

尤其是方源依赖智慧光晕,三番五次地改良仙道杀招万我。不仅是陆续增加了几只核心仙蛊,而且还有不少的魂道凡蛊。

搭配起来运用,使得力道巨手打杀敌人之后,重伤的魂魄就会被方源摄拿过去。

如果旁人稍不留神,都会以为魂魄也被方源一同摧毁了。

但其实,方源却是暗暗扣下了魂魄。

这点,就连太白云生都没有发觉。因为他没有仙级的侦察杀招。

方源也没有告诉他。

蛊仙魂魄的价值,可远比蛊仙意志、凡人魂魄要高得多。

方源第一次搜魂刚刚结束。

在这次搜魂中,他得知了这群羽民的来路。

“来自西漠的羽民部落。世代生活在绿天碎块世界之中。难怪有如此古风,就算自杀,也不低头为奴。不过世外桃源的生活,丰富的生存资源。单一的种族,没有强大的外敌,却是将这些羽民的心智都腐蚀了。”

“白海沙陀……到底是什么人呢?”

“居然能纠集这么多的人。围攻羽圣城。并且手段之高,超乎想象。居然能限制住一座仙蛊屋。使得羽民们只能狼狈逃窜!”

方源搜刮肚肠,也未从记忆中找寻出白海沙陀的丝毫信息。

最终。方源只得感慨:“西漠也是藏龙卧虎啊。”

总结下来,这一次太白福地渡劫,收益很大。

太白云生得到两个羽民蛊仙的毕生道痕,一位六转,一位七转,可以说是最大的获利者。

而方源的收获也不小。

这群羽民来头很大,又有羽圣城,追根溯源的话,还要牵扯到近古前期,幽魂魔尊的时代。

作为羽圣城的领袖,七转蛊仙郑灵掌握的秘密、仙道杀招、仙道蛊方等等,是极其丰富的。完整的蛊仙传承,都有很大可能。

绿天碎块世界,与世隔绝了这么多年,羽圣城堪称羽民的圣地,五域天地中极可能的最大一块羽民聚落。

因为和外界交流很少,绿天碎块世界中又有丰富的资源,这群羽民蛊仙掌握的蛊仙手段,都是古意盎然,有着近古时代的鲜明特色。

方源甚至都有期待若是运气好的话,说不定能在郑灵魂魄中搜刮出力道的传承。

唯一可惜的是那些羽民凡人。

方源还打算,将这些羽民驯养成奴隶,帮助他在西漠开拓市场,甚至建立商队。

羽民在西漠,是上佳的奴隶。超级势力、大型势力,都有组建羽民为主的商队。

方源不可能只和萧家做买卖,初期他的确要靠萧家这块跳板,平稳地进入西漠市场。但长期只和萧家合作,这会让方源很吃亏的。

不过这些凡人羽民,死了也就死了。看他们的心性,是野惯了的,驯养他们很难。反而不如直接购买羽民奴隶来的方便快捷。

方源不是一位纠结过去的人,很快他就将此事抛之脑后。

现在他已经拥有了变形仙蛊,还有融合了变形仙蛊的仙道杀招见面似相识。方源本打算变作另外一副模样,前往北原僵盟,再继续之前受挫的计划。

但联系黎山仙子之后,方源从得到的情报分析,情况却不容乐观。

自从沙黄身份被凤九歌调查之后,北原僵盟对陌生仙僵加入僵盟的审核,提升了许多层级。方源要混入北原僵盟,再也不会如之前那么轻松。

而且凤九歌、残阳老君等一行人的潜藏,更让方源嗅出一股风雨欲来的危机感。

方源犹豫起来,在这种情况下,自己是否仍旧按照原计划混入北原僵盟呢?

ps:节假日应酬有点多,不好意思了。(天上掉馅饼的好活动,炫酷手机等你拿!关注起\~{}點/中文网公众号(微信添加朋友添加公众号输入dd即可),马上参加!人人有奖,现在立刻关注dd微信公众号!)(未完待续……)

\end{this_body}

