\newsection{顺其自然}    %第三百三十一节:顺其自然

\begin{this_body}

%1
方源有的是耐心。

%2
帮助太白云生成功渡劫之后,方源一边炼制梦道凡蛊,采摘春梦果实,另一边则大加经营胆识蛊、幽火龙蟒等等买卖。

%3
很快,他每月的盈利,就大大超过前世同期,竟然突破了两千大关!

%4
这其中有很多原因。

%5
一来,方源前世需要耗费时间,多番打探清楚,才谨慎地选择出合适的贸易对象。尤其是西漠这一块地方。

%6
但今生,由于保留了经验,方源轻车熟路就联系上了买方,省了许多步骤和时间,将盈利时间大大提前。

%7
二来,方源石巢中的毛民们,被解放出来,都来炼制胆识蛊。而不像前世,还要炼制星念蛊等等非盈利性质的蛊虫。

%8
三来,方源没有想要搬迁星象福地的计划,他已经和黑楼兰等人达成了新的盟约。将来或许还要迁徙,但现在并不急需。

%9
四来,有着羽民们组成商队,进行贸易,导致方源在西漠方面,很快就见到许多不菲的成效。

%10
羽民的作用,开始显现出来。

%11
但方源只是浅尝辄止,并没有将这数万的羽民尽数派遣出去。

%12
在这一块,他有一个长久的计划,他要借助这些人口,将羽民大量繁衍,将来或许还可以做羽民奴隶的贸易。

%13
在奴隶贸易方面,羽民虽然比不上毛民,但比石人强多了。尤其是西漠势力。每年牺牲的羽民不在少数,对羽民奴隶的需求在市场中一直居高不下。

%14
方源耐心发展,积累仙元石的同时,也不忘积累炼蛊仙材。

%15
时间匆匆,琅琊福地再次被强敌入侵。

%16
琅琊地灵心知敌势汹汹。只得求援,往方源、太白云生两边都送去了求援信。

%17
这一世,方源精心策划,太白云生的仙窍几乎没有多大的损失,接到信后,两人便一起来到琅琊福地。

%18
琅琊福地当中,充斥着无边的茫茫云雾。伸手不见五指。上不见天。下不见地。

%19
“这就是十二波云迷澜阵。”琅琊地灵用低沉的声音介绍着,“而这些人就是来犯之敌。”

%20
地灵挥袖一扫,方源、太白云生二人面前,就浮现出几个画面。

%21
秦百胜、姜钰、黑城、雪松子、贺狼子,还有回风子,和炎煌雷泽仙僵,俱在其上。

%22
“居然是他们!”方源故作惊讶之色。

%23
“你认识这些人?”琅琊地灵旋即问道。

%24
方源点头。神情严肃:“这些人实力极强,为首的秦百胜虽然只有七转修为,但却可敌八转蛊仙。”

%25
琅琊地灵顿时吓了一跳:“这么厉害?!”

%26
方源又指着其他人道:“这位姜钰仙子,有暗渡仙蛊。这个是当今北原,超级势力黑家的当权蛊仙长老黑城,他的手中可掌握着一座仙蛊屋!还有雪松子,你别看他只有六转修为,但却是大雪山福地中的成员,背后的靠山可是八转雪胡老祖。还有这位贺狼子,变化道蛊仙。战力极强,当初雪胡老祖想要招揽他,他都不买账。这位回风子,乃是风道大能,风道境界深不可测,是当今北原蛊仙界公认的速度第一人,曾经在多位八转蛊仙手中逃走。”

%27
方源每说一句。琅琊地灵的心情,就沉重一分。

%28
方源话说完,琅琊地灵的脸色已经凝重无比。他手指着神秘黑袍蛊仙:“没想到敌人如此强大。那么这位蛊仙,又是何方神圣?”

%29
方源当然知道他的跟脚,但此刻脸上却流露出疑惑的神色。

%30
他沉吟道:“此人遮挡面目,又有仙道杀招加身,似乎神秘莫测。待我来算一算罢。”

%31
说着,他掐指便算,浑身星光泛滥,流露出仙蛊气息,煞有介事的样子。

%32
这一下,顿时将琅琊地灵蒙住,他不禁心想:“我虽然以前,曾经多次找到方源,让他为我推算仙蛊方。但这还是第一次,亲眼见到他推算。他所用何种仙蛊,我不太清楚,不过看这样子,好像他的智道境界并不低!”

%33
方源身上的星芒渐渐消散,他脸色沉重,眼冒精光:“好家伙,这个神秘蛊仙也大有来历,居然是一位仙僵,而且还是十绝体之一的炎煌雷泽体!”

%34
“你这么快就推算出来了?”琅琊地灵颇为惊异。

%35
一旁,太白云生呵呵笑道:“那是当然。方源的境界,可是智道宗师。”

%36
“智道宗师?!”琅琊地灵心头一震,再次打量方源一眼,带着刮目相看的目光。

%37
方源神色平静,声音低沉地对琅琊地灵道:“敌方如此势大,恐怕此次危险了。”

%38
琅琊地灵呵呵一笑:“不妨事。只要坚持一段时间,再强的蛊仙我都能生擒活捉了去。”

%39
“哦?”方源微微一扬眉头,故作好奇地对琅琊地灵试探道,“看来你这边还有强援?”

%40
琅琊地灵脸上带着得意的微笑,本不想多说什么,但忽然又改变了主意,反过来试探方源:“你小小年纪,居然是智道宗师?说实话,真是出乎我的意料。你不妨算一算,我这一次对敌的把握,在哪里?”

%41
方源当然知道,琅琊地灵的自信,在于八转仙蛊屋炼炉。

%42
但接下来,他故作推算,反装作一副遭遇困难的样子。

%43
最终,他承认失败道:“我推算不出。实不相瞒,以前我也推算过琅琊福地,但都一无所获。这一次,也不例外。”

%44
“哈哈哈。”琅琊地灵仰头大笑,他得意地拍拍方源的肩膀,却没有再多说什么。

%45
仙蛊屋攻防一体,八转炼炉自然也是如此。

%46
这种防御的手段。绝不仅仅只是防护攻击,还有抵御他人推算的能力。

%47
方源故作失败,这在琅琊地灵的意料之中。

%48
在他想来,这是应该的。因为就算方源再出色,也终究只有六转修为。仙蛊再玄妙,也抵不上八转的仙蛊屋炼炉啊。

%49
“推算不出来,是肯定的事情。但我若推算出来,恐怕要引起琅琊地灵的忌惮了。”方源心中冷笑。

%50
地灵虽然偏执或者单纯,但不是无智。

%51
尤其是琅琊地灵本体长毛老祖,在这位传奇人物的生前,有一位至交好友一言仙。此人就是智道蛊仙。作三尊说,在历史上同样大名鼎鼎。

%52
所以琅琊地灵,很明白一位智道蛊仙的能力范围。

%53
方源若是推算出什么来,已经明显超过了他六转智道蛊仙的能力范围,这必然会引起琅琊地灵的警觉和猜疑。

%54
他会这样想:“仅凭方源,怎么可能推算出来?很明显,他的身后还有能人!而这个大能十分厉害。居然能破开我家炼炉的防护,将福地中的布置都推算出来!”

%55
这么一想,琅琊地灵自然十分警惕,对方源防范有加。

%56
要知道,琅琊地灵可是长毛老祖的执念所成。而长毛老祖乃是毛民,对人族,乃至对其他异族,都有防范之心。这点从前世,琅琊地灵转变出来的性情和言语,就可以看出来了。

%57
方源之所以能够接近琅琊地灵。主要原因还是因为,他继承了盗天魔尊的某个传承。

%58
碍于本体和盗天魔尊的约定,琅琊地灵必须帮助方源三次,为他炼制蛊虫。

%59
若非如此,方源怎可能以人族的身份,这样接近琅琊地灵呢?

%60
方源虽然有星念仙蛊、全力以赴仙蛊等需要炼制,但一直以来。也都不想耗费最后一次炼蛊机会,就是为了时刻接近琅琊地灵。

%61
琅琊地灵在这片福地中,生活了无数年。和外界的往来,除了宝黄天的买卖仙材之外,唯有墨坦桑这一条线。

%62
他知道怀璧其罪的道理,所以一直都很低调。

%63
同时也对其他人,都抱有一份天然的不信任的态度。

%64
方源之前实力不强,琅琊地灵还不会太过忌惮和猜忌,但当他渐渐展现出超凡的一面时,琅琊地灵自然会有警惕和不安。

%65
卧榻之上岂容他人鼾睡?

%66
以前的方源只是个小白兔,现在展现出智道宗师修为的方源,可称得上一条毒蛇了。

%67
接下来,事情的发展一如前世。

%68
方源被琅琊地灵安排,镇守一座云阁。

%69
不过在此之前,方源却提出召集黑楼兰、黎山仙子参战的建议。

%70
太白云生也在一旁作保。

%71
但琅琊地灵还是拒绝了。

%72
方源对战雪松子,这一次却不采纳上一世的计策,没有偷袭雪松子,而是光明正大地去打。

%73
虽然比前世的声势更加烜赫,但战果却不大。

%74
方源只是和雪松子纠缠僵持,不久后影宗赶来支援,方源就主动撤离了。

%75
杀了雪松子,也没有什么益处。他身上有仙蛊,也有寿蛊,但都难以抢夺。

%76
反倒不如放了他,让他不久后对付中洲蛊仙去。

%77
方源当然也曾想过,提醒琅琊地灵,提前安排,利用炼炉将影宗一行人拿下。但仔细思考之后,他否决了这个想法。

%78
首先,他要解决了影宗一方,风险很大。秦百胜的战力极强,方源毫无自信。影宗又和僵盟似乎有着隐秘联系,这里的水很深。

%79
其次,就算解决了影宗,方源的收获也不会理想。炼炉的作用最大,若是胜利,琅琊地灵必然得到最多的战利品。影宗据点落魄谷,会被中洲蛊仙们造访,方源也得不到落魄谷。

%80
再次,方源还需要北原的影宗众人,抵挡中洲蛊仙。虽然不知道中洲蛊仙明明调查方源,却怎么找上影宗的,但这是方源最喜欢看到的局面。若是杀了影宗,等若给中洲蛊仙扫清障碍,万一中洲蛊仙找上方源,方源又该如何是好?

%81
最后,方源从炎煌雷泽体上,猜测出义天山大战的幕后黑手,很可能就是影宗。若是杀了他们,势必会影响巨大,导致义天山大战时,出现前世未有的意外。意外多了,说不定就会影响到方源的布局,让他针对性的准备毫无用处。

%82
所以,琅琊福地攻防战,方源的打算是顺其自然。

%83
改变这一战的结果,不仅风险巨大,而且弊大于利,智者不取也!

\end{this_body}

