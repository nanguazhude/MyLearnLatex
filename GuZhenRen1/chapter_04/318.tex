\newsection{困众仙灾劫如雨}    %第三百一十九节:困众仙灾劫如雨

\begin{this_body}

“咦,怎么回事?”这一刻,盘踞在义天山万里之外的南疆蛊仙们,纷纷抬头,望向天空。

甚至就连义天山上的凡人蛊师们,也都感到了一股莫名的惊悸。

隆隆隆隆……

只听从极高的天际,传来隐约的轰鸣声。

几个呼吸之后,这股轰鸣声由小变大,仿佛有数百人在远处擂鼓。

方源自从登上义天山,就一直在争分夺秒,埋头苦练,积极转化战意,但此刻天上的动静越来越大,他也不得不暂时停下手头上的工作,昂首仰望。

便见万丈的高空,忽然亮起一连串的白色光点。

这些光点数量极多,密密麻麻,正朝着义天山,浩浩荡荡,飞速坠落。

“这是千珠光劫。”有蛊仙惊呼一声。

“不对,千珠光劫,只有千余光珠。但眼前这灾劫,包含的光珠至少有十万!”

“怎么会有灾劫产生?”很多蛊仙都感到莫名其妙。

“难道说有人在渡劫?”大多数蛊仙都在第一时间,将目光投向义天山。

方源心头一跳,提起十二分戒备。

很快,他就冷汗涔涔,感到一**凝如实质的侦测压力,扫过自己全身。

他虽然有见面似相识,但这个仙道杀招的核心仙蛊,只是六转级数。能骗过六转蛊仙,但是面对七转、八转的侦查,暴露的风险还是很大的。

方源提心吊胆,结果让他松了一口气。

他没有被蛊仙发现。

“居然没有找到?”蛊仙们诧异。

他们搜索义天山附近,没有发现任何一个渡劫的五转蛊师。

正当他们要继续搜索。施展更强的侦查手段的时候,千珠光劫已落到众仙头上。

“以防万一。不要让天劫影响了义天山!”

“不错,我们一齐出手。挡住天劫。”

一些蛊仙心中猜测,这灾劫是否因惊鸿乱斗台而起?

南疆蛊仙们很快,便达成了一致。

按照赌约,若有威胁者出现,参与赌斗的蛊仙们都要一起抵抗,维护自己的利益。

千珠光劫虽多,但占据这里的南疆蛊仙人多势众。其中包括四位八转,九位七转,十多位六转。

众仙纷纷出手。很快就将千珠光劫消灭,没有一颗光珠落到地上。

甚至蛊仙中的有心者,还施展手段,牵引了重重云层,遮住义天山的上空,防止凡人蛊师们察觉到异象。

“呵呵呵。”八转散仙彭世龙抚摸胡须,笑道,“不管是谁,能够在这个节骨眼上渡劫升仙。很有想法。”

“找到此人!在场的诸位,都是南疆蛊仙精英,还怕找不到一个渡劫的凡人吗?”七转蛊仙叶倾堂冷笑着,被别人当做挡箭牌。成其渡劫的工具,让这位孤高的蛊仙十分不爽。

“渡劫的凡人,是哪位的后人?直接说出来吧。免得找到了场面不好看。”蛊仙王凯呵呵冷笑道。

众仙面面相觑一阵子,却没有人站出来。

八转蛊仙任海洋的脸色阴沉下来。冷喝出声:“不识好歹!现在还想隐瞒,把我们诸位当傻子耍吗?”

“注意了。又有灾劫降下了。”忽然,一位蛊仙出声提醒道。

高空中一片黯淡无关,像是被巨人用笔,抹上一大块的浓郁青黛。

温度陡降,寒气四溢。

咻咻咻!

无数冰霜,宛若利刃,飞速盘旋,以倾盆之势向下倒灌。

叶倾堂声音有些凝重,低喝道:“是玄阴飞霜。”

“究竟是谁在渡劫,居然能引出十大凶灾!”

“不对劲,就算是十大凶灾,这灾劫的规模也未免太大了,远超常规。”

“来了!”

“挡住!”

轰隆隆,电芒激射,火焰翻飞,尽数抵消飞霜。

玄阴飞霜可不是盖的,列为十大凶灾之一。

此灾过后,许多蛊仙蛊仙都受了伤。其中一位蛊仙瓜老,更是重伤,他不禁萌生退意。

“奇怪!难道这是惊鸿乱斗台要出世的缘故吗?老天这才降下灾劫来?不行了,老子手中可没有仙蛊,不能在这里傻傻的硬挨灾劫,先脱离此处险境再说!”

念及于此,瓜老顿时化作一道华光,激射出去。

“这家伙跑的到快!”叶倾堂冷笑,背负双手,傲然悬浮在原处空中。玄阴飞霜虽强,但还不至于让他难堪。

但更多的六转蛊仙,开始效仿瓜老。

这些人大多都是底层蛊仙,没有仙蛊傍身,基本上身上都带着伤。

然而下一刻,瓜老的身影,忽然出现在蛊仙人群中的最中央。

他怔住了,自己怎么会出现在这里?

旋即,更多的六转蛊仙,也都被传送回来,出现在瓜老的身边。

“有埋伏!外围有人铺设了一个巨型的宇道蛊阵!”很快,蛊仙中有人反应过来,惊呼出声。

八转气息爆发出来,彭世龙眼中射出丈许厉芒,四下扫射:“是谁,出来!”

“呵呵呵……”在砚石老人的阴笑声中,十余位黑袍蛊仙,显现身形,将南疆蛊仙们团团包围。

砚石老人居于后方,手指苍穹:“给诸位一个忠告,第三波灾劫已经来了。”

众人抬头望去,只见空中飞来一片乌央央的狮群。

荒兽气宗狮!

这种狮子,身怀气道道痕,天生能够飞翔。

气宗狮的数量,竟多达六千余头,六转蛊仙们的脸色为之骤白。

八转蛊仙贾谊怒吼:“雕虫小技!都给我滚!!”

他一挥长袖,狂风呼啸天地。

一道道巨大的风镰,成千上万。朝着四面八法激射。

狮群哀嚎,遭受风镰重创。无数气宗狮当即惨死。被风镰碎尸万段,天空中下起了一场磅礴的血雨。其中夹杂着无数的骨头和碎肉。

“风狂贾谊,果然是好威风!”砚石老人淡淡笑道,不吝赞赏。

贾谊却皱起眉头。他发出的大部分的风镰,都刮向砚石老人。但风镰被蛊阵遮挡,全数消散,而蛊阵只是荡起涟漪,没有任何损毁,可见防御之强,超出想象。

“这些人究竟是谁。意欲何为?”

“暗算我等,简直是胆大包天,不想活了。”

“尔等竟然敢和整个南疆蛊仙界作对!”

蛊仙们冷喝怒骂,或是心中猜疑。

时间不容许他们等到影宗方面的回应,天劫又至!

嗷!

恢弘的龙鸣声,响彻天地。

一道火烧云,覆盖方圆万里,至上而下,盖压过来。

“四炎云盖劫!”有人认出来。大喊一声。

“小心,这四炎云盖劫我渡过,一共四层。一层比一层猛。”一位炎道蛊仙提醒道。

“挡住它!”

“还有龙吟之声,这是龙吟劫、四炎云盖劫。双劫齐至!”

“护住周围蛊仙,保存有生力量!”

蛊仙们纷纷怒喝,一齐联手渡劫。

“正常的四炎云盖劫。最多只有方圆百里,怎么会有这么大的火烧云?”义天山上。方源也颇为震惊地望着。

浮云能遮挡凡人蛊师的视野,但挡不住方源的目光。

他感到十分不妙。

“我好像步入了一个巨大的陷阱当中。这些黑袍蛊仙。似乎和黑楼兰所说的影宗蛊仙,十分相似。他们居然敢算计这么多的南疆蛊仙,究竟想干什么?!看来这天劫地灾,不是有人渡劫,而是他们搞的鬼。”

以砚石老人为首的影宗众仙,搭建巨型蛊阵,将南疆蛊仙统统困住。

南疆蛊仙们想要脱离此地,攻打蛊阵,不见成效,只能硬挨灾劫轰击。

随着时间推移,灾劫的威力,越发恐怖,很快七转蛊仙都感到支撑不住。

又三波之后,八转蛊仙应付起来,也变得十分吃力,各个脸色难看。

他们不是没有尝试,但影宗蛊阵面前都铩羽而归。

“你们挡住这一波天劫,让我来!”彭世龙下定决心,催发压箱底的杀招。

气势恢宏,光波扩散,但影宗蛊阵只是震荡了三番,便再次固定下来。

“这是什么蛊阵?!”南疆众仙见此,心都沉入谷底,难以遮挡心中的忧虑情绪。

天劫一**接连到来,彼此间隔的时间越来越短。不同种类的灾劫,纷至沓来,就像是倾盆暴雨,浩荡不绝。

蛊仙们渐渐攻少防多,丧失了主动。

“顶住!”贾谊呐喊,“我们还有机会!灾劫浩荡绵绵,针对我们,也殃及周围的蛊阵。”

众仙闻言,皆心头一振。

尽管已经有六转蛊仙陨落,但剩下的仙人们都没有放弃。

他们在苦苦支撑,心中还保留希望。

一旦蛊阵被灾劫破坏,那么他们就能逃出生天。

场中的局势,都落在砚石老人的眼中,对种种变化,他都洞若观火。

“无邪。”他轻声呼唤。

在他身后,站着一位青年男子。

他有一头黑色的波浪长发,一直披到肩头。他的眼瞳闪烁着无数光彩,五颜六色混杂在一起,形成漩涡,不断缓缓旋转。

他的鼻梁高挺,嘴唇线条宛若刀刻,若是抿着,尽显冷酷邪魅的风姿。但此刻他正专心地仰望高空,嘴角大大的咧开,笑容满布脸面,露出白色的牙齿,竟显得有些……傻。

“无邪。”见身后没有回应,砚石老人又叫一声。

“啊,你叫我啊,我在看烟花呢。”青年蛊仙反应过来,笑着回答道。

砚石老人苦笑一声:“这可不是简单的烟花,这是灾劫,虽然美丽,但蕴藏着致命的威胁。这些,都是天意的手笔,它企图毁灭我等,是我们最大的敌人。”

青年蛊仙脸上顿时流露出认真的神色:“哦,是这样啊。那叫天意出来,让我把他干翻!”(未完待续……)

\end{this_body}

