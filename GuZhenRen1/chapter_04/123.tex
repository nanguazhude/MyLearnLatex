\newsection{沉默和叹息}    %第一百二十三节:沉默和叹息

\begin{this_body}

轰!

剧烈的爆炸声中,贺狼子仿佛一滩肉泥,摊在巨坑中心。

他奋力挣扎,想要爬起来,但身体宛若被一座无形的大山镇压着,千万钧的重量压得他动弹不得。

贺狼子双目赤红,面容扭曲,嘶吼连连。

秦百胜双手自然垂下,面无表情地站在巨坑边缘,看着挣扎不止的贺狼子。

“一招,只是一招,就让贺狼子无还手之力!”雪松子的眼中,是一片惊骇之色。

黑城眯起双眼,掩下心中的剧烈波动,急速思索着:“这是什么仙道杀招?一用之下,居然令贺狼子连任何一个杀招,都动用不了!贺狼子动弹不得,但肉体却毫无受压迫的征兆。这应该是魂道的杀招……”

至于姜钰仙子,则一脸见怪不怪的样子。

而神秘的黑袍蛊仙,整个脸面都笼罩在帽兜中,不见其神色。

“有,有种的,让我动用杀招!先手突袭我,算什么本事?我不服!”贺狼子几乎咬碎钢牙,趴在坑中,仰望着高处的秦百胜,艰难而又愤恨出声。

秦百胜不屑地冷哼一声,俯视着贺狼子,扯起嘴角:“先下手为强,这个道理你身为魔道蛊仙都不明白?看样子还是高估了你,真正的战斗中,谁管你服不服?只要克敌制胜,就是本事!”

说到这里,秦百胜顿了顿,语气如冰:“如今,我为刀俎你为鱼肉,任我宰割。你服也得服,不服也得服。给你三息时间,臣服于我,否则我便当场杀了你。”

贺狼子没有犹豫:“我服!”

秦百胜哈哈一笑,放开禁锢。

贺狼子但觉浑身轻松,立即狞笑一声,猛地发出仙道杀招。

霎时间,刺目的光辉绚烂绽射。光芒来得快,去的也快,迅速消散后,巨坑中现出一头狰狞巨狼。

巨狼张开血盆大口,立即扑向秦百胜。

秦百胜站在坑边,渺小的身形和巨狼形成鲜明的对比。

巨狼攻击未至,就已经掀起一阵巨大的腥风。狂风吹得几位蛊仙的长袍猎猎作响。

贺狼子化身巨狼,挟愤而攻,声势浩荡无比。雪松子、黑城连忙后退,免得殃及池鱼。

秦百胜位置最近,见着巨狼杀来,却是一动不动,嘴角浮现出冷讽之意,显然贺狼子的袭杀并未令他意外。

轰!

再一声巨大的轰鸣,响彻众人的耳畔。

巨狼呜咽一声,从半空中骤然衰落下去,四爪趴在地上,巨大压力死死地镇压着狼躯。

贺狼子一如之前的模样,再一次动弹不得。

硕大的狼眸中,流露出难以置信的惊骇,他望着秦百胜大叫起来:“又是这一招!这是什么招数?”

“告诉你也没有什么大不了的。”秦百胜呵呵一笑,“这一记仙道杀招,名为魂压。以我的魂魄底蕴,直接碾压你的魂魄。你肉体虽强,但此招却直接针对你的魂魄。你没有克制魂道的仙道变化,不管变作其他任何猛兽,也不会是我的对手。贺狼子,现在我给你最后一次机会。臣服于我,或者……去死。”

面对如斯强势的秦百胜,贺狼子陷入沉默。

黑城、雪松子对视一眼,均大感不妙。

……

“沙黄……”凤九歌望着远离而去的夜叉龙帅等人,口中喃喃不止。

“这次搞来的情报,根本没有实质性的作用嘛。”凤九歌身旁,中洲蛊仙洪赤明不满地嘟囔着。

“不,得到的情报已经很多了。”凤九歌呵呵一笑,“首先这个仙僵沙黄,能够顺利地加入僵盟,证明他是北原蛊仙。或许不是土生土长的北原人,但也一定在北原升仙。其次,他既做伪装,证明不能以真面目示人。最后他的背后有大能或者势力撑腰。这股势力极可能便是八十八角真阳楼大案的罪魁祸首!”

“九歌大人,言之有理。”对于凤九歌的推测,其余三仙均点头赞同。

“接下来我们该如何行动?”

凤九歌思量一阵,这才道:“我们先和老算子他们会和,将得到的情报告知于他,方便他的进一步推算。”

……

大雪山福地,第一支峰。

“怎么?软玉沙还没有筹集全吗?”雪胡老祖捏着单子,不悦地质问道。

他质问的对象,乃是大雪山第四支峰的厉鹏王。

凶威在外,桀骜不驯的厉鹏王,此刻低垂着头,恭声道:“老祖息怒,属下遭遇了一群天魁兽,这才不得不终止了采集。这次再去白天,一定能够功成。”

“嗯,你就去准备罢。不是我针对你,而是第一轮采集仙材,你便出现了这样的差池。以后怎么服众,坐得稳第四交椅?你是我一手扶持上来的,你这一轮表现最差,叫其余峰主又如何看我呢?”雪胡老祖又点了几句,这才挥袖,让不停告罪的厉鹏王离开大殿。

“厉鹏王此次也是运气欠佳,遭遇了天魁兽群。他身受重伤,却仍旧带来一部分软玉沙,已属不易。”厉鹏王走后,从殿后转出一位女仙。

整个大雪山福地中,也就这位女仙,能够这般语气与雪胡老祖说话。

雪胡老祖转过视线,看向女仙,面容柔和了几分:“娘子,你是不知道此中关窍,所以才觉得我对于这些峰主过于苛责了。我搜了马鸿运的魂魄,得知了这小子的一切经历。鸿运齐天蛊的威能,实在是可畏可怖啊。你以为厉鹏王此番,只是巧合吗?绝非如此。不仅是他,其余峰主也或多或少,在采集仙材中,有着变故。这一切,都是鸿运齐天蛊在暗中影响着我们。任何对其宿主不利的举动,都会引来鸿运的反击,镇压我等的运气。”

“竟有这种事情?”女仙万寿娘子奇道。

“按照运道的理论,我等身为蛊仙,自有超出平常的气运护身。但鸿运齐天蛊乃是运中帝皇,因此我们也会被影响。时间拖得越久,就越会生出无数变故。最终炼蛊失败不说,甚至还会惹来巨大祸端。”雪胡老祖详细解释道。

身为第二峰主的万寿娘子,这才了然,思索了一番,皱起眉头道:“如此一来,岂不是对我炼蛊,大有妨碍?”

万寿娘子乃是北原四大炼道蛊仙之一,按照雪胡老祖的计划,就是让她最后操刀,以马鸿运为主材,炼出鸿运齐天蛊。

雪胡老祖点点头:“所以咱们这一次炼蛊,不仅要多准备几份炼蛊仙材,而且还要借助其他运道仙蛊,护住自身气运。好在有马鸿运这个主材,比之无中生有地炼出鸿运齐天蛊,节省了不知多少的仙材了。”

“即便如此,我们要准备的仙材也过多了,几乎要超出整个大雪山的承受能力。这一次炼蛊,实是重大,几乎将我们夫妇大半辈子的积累,都消耗一空。”万寿娘子眉头愁云一片。

雪胡老祖笑了笑,揽住她的腰,劝慰道:“娘子是担心最终炼蛊失败?”

万寿娘子点点头,凝望自家夫君:“就算是炼道大宗师,也会失败。更何况我呢?”

“哈哈哈,娘子放宽心尽管去炼,不管成败,我都接受着,绝无一丝怨愤之心。我能达到今天这般修为,已经榨干了一切潜能才华,对接下来的灾劫越来越没有信心。鸿运齐天蛊就是我破局的希望,唉,但愿一切都能来得及。”雪胡老祖悠悠一叹。

……

“方源,去把那箱货物搬下来。”商队管事指着一个箱子,厉声大喝道。

“是。”方源答应一声,连忙爬上黑皮肥甲虫,将上面最大的那件木箱搬下来。

“这小子居然还有力道修为,我竟看走眼了。”看着方源轻轻松松地完成任务,商队管事眼中阴芒一闪,“不成,我答应了二公子的事情,不能这样失败!”

想到这里,商队管事挥起手中长鞭,照着方源的后背狠狠一抽。

啪的一声。

方源后背衣服瞬间被抽破,一道深深的鞭痕,印在方源的背上。

剧烈的疼痛,袭上心头,方源被抽倒在地,淋漓鲜血很快从伤口处溢出。

前世混迹商队,因为拒绝了某位公子的招揽,因此受到屡屡打压、羞辱的记忆,又重新鲜活起来。

一股愤怒之情,在方源的心中升腾而起,但很快又被方源按捺下去。

“磨磨蹭蹭的做什么,动作给我再快一点!”商队管事纯粹是没事找茬,痛骂方源。

方源修为高达三转,但此时却不用。

正所谓利刃在手,杀心自起,这就是梦境中陷阱,故意引他攻击,好激发更多的愤怒。

好不容易捱过一天,受到管事多番刁难,方源进入临时搭建的帐篷,继续炼制梦道凡蛊。

“在这梦境中已经过了五个场景,这次终于快要成了。”方源望着渐渐成形的蛊虫,心中欣慰。

但哪知半夜时分,正在炼蛊紧要关头,忽然兽群冲击商队的临时驻地。方源尽管布置了防御手段,却终究抵不过兽群的浩荡冲锋,最终功亏一篑。

方源睁开双眼,脱离梦境,仍旧是身处荡魂行宫。

查看仙窍,发现黎山仙子送来了消息。

却是说明她那边的处境,雪胡老祖命令各峰主外出采集仙材。黎山仙子离开大雪山福地,留着黑楼兰一人并不安全。因此,黎山仙子催促方源尽一切可能快速,另外告知方源北原僵盟出现变故,有人调查仙僵沙黄的身份。

“已经查到如此地步了吗?”昏暗的房中,方源皱起眉头,幽幽叹息一声。

\end{this_body}

