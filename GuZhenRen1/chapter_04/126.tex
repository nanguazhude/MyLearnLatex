\newsection{梦道先驱}    %第一百二十六节:梦道先驱

\begin{this_body}

西漠。求书网小说qiushu.cc

“噗!”蛊仙唐妙陡然娇躯一震,睁开双眼,吐出一口逆血。

“又失败了。”唐妙脸色惨白如纸,比起身上的伤势,她更对此次探索梦境的失败结果而绝望。

唐妙魂魄重创,此时头昏眼花,身躯沉重,心中哀伤:“哥哥,小妹无能,救不出你!甚至此次探索梦境,都没有找不到你。”

她缓缓闭上双眼,凄苦的泪水滑落脸颊,形成两道泪痕。

片刻后,闻讯而急忙赶来的唐家蛊仙们,来到密室,站在唐妙的面前。

唐家太上大长老唐阳皱起眉头,他看着床榻上盘坐着的唐妙,语气沉重:“没有想到,尽管这一次大费周章,向萧家借来了七转情蛊。结果闯荡梦境,仍旧遭受了惨败。”

“唉,梦境本身就奇妙诡异,更何况妙探索的乃是盗天魔尊的梦境。”蛊仙唐烂柯叹息着。

三尊说早已经流传天下,关乎大梦仙尊,不仅是中洲,其余四域的超级势力,大多数都在探索梦境,积累经验,研发相关蛊虫。

身为西漠超级势力之一的唐家,更在许多年前,意外地发现了盗天魔尊的梦境。

这个发现,极大的激发了唐家研究梦道的热情。

为了探索这道梦境,素有唐家第一天才之名的唐家蛊仙唐方明主动接过这个家族重任,结果屡屡受挫不说,还不慎迷失在梦境当中。

他的亲妹妹唐妙。为了救醒哥哥,屡次尝试,皆遭失败。

这一次失败。更是惨重。

为了此次梦境探索,唐家耗费巨资,借得情道仙蛊,寄托着希望。没想到结果让人感觉到一股冰冷。

“仙蛊虽有作用,能够镇压我的情感,不被梦境轻易勾动。但也因此,仿佛成了局外人。似乎一直被演化中的梦境排斥在核心之外。这一次入梦,我甚至连哥哥的身份,都没有发现。”一直盘坐在床榻上。默默疗伤的唐妙,语气沉重。

“唉,看来探索梦境,还真得梦道蛊虫。可惜我唐家探索梦境已经六十多年。创出的梦道凡蛊不过十几种。且都是三转以下。其中又有大半,作用有限至极。算起来,真正适用于梦境探索的,只有四五只。”

万事开头难,尤其是这起步阶段,更是分外艰难。先驱者往往茫然,因此每前进微微一小步都是巨大的成功。

对于梦道,大家都没有经验。[看本书最新章节请到都是一片空白。待渡过这个艰难的起步时期,渐渐积累了宝贵的经验。优势就会如滚雪球一般越加壮大。

等到中期后期,优势更加明显,梦道探索的难度就会大幅度降低。方源前世记忆中,就有这么一段时期,几乎每一天都有梦道蛊虫被创新出来,种类繁多宛若夜空星辰。那是一个烈火烹油,鲜花着锦的梦道盛世。

人族历史上,每每这样的盛世,必有绝世大能应运诞生。

按照三尊说之言,这个大能应当便是大梦仙尊。只可惜方源没有等到大梦仙尊的出现。

说起来,如今这个起步时期,唐家已经走在时代的前列。就算是中洲十大古派之中排列,也是名列前茅。

不过时代的局限,仍旧牢牢桎梏着他们。就相当于瞎子摸象,不着全局,只能一步步艰难摸索,呕心沥血,才能得到些微成果十几种的梦道凡蛊。

然而这些成果中,真正实用的性价比高的梦道凡蛊,又少之又少。

拿地球上的例子解释,就仿佛是实验室制品,距离军用尚有一段距离,距离民用则有更长距离了。

尤其这种创新、探索,不是任何一位蛊仙都能胜任还需要天赋和才情。

在这方面,循规蹈矩的蛊仙只能算是庸才,没有创新精神,甚至连某些凡人蛊师都不如。

唐家的蛊仙数量也不少,但也就唐方明适合做这种事情。这十几种梦道凡蛊,大部分都是他创造的。可以说,他是唐家探索梦境的主力大将!

遗憾的是,如今的唐方明陷入梦境不能自拔。如此一来,唐家的探索进度便几乎停滞不前。

唐方明的无伤,有着家族蛊仙的手段保护。但限于梦境中的魂魄,已经脱离了肉身,不见踪影。只要陷于梦境,魂魄就会受到不断的削弱,直至彻底衰弱消亡。

唐家蛊仙尤其是唐妙,为此心急如焚,但却有心无力。

“我哥现在的情形如何了?”稳定了伤势之后,唐妙开口问道。

几位太上长老们对视一眼,其中一位艰难地道:“自有我们担保,但市面上几乎没有胆识蛊。我们……已经买不到了。”

唐妙立即紧皱眉头,急问道:“这是怎么回事?”

先前那位开口的蛊仙,再次答道:“你也知道,贩卖胆识蛊的卖家其实只有一家,做的是垄断生意。但之前不久,这位卖家忽然断货,不知道什么缘故。市场上,虽然也有胆识蛊流通着,但源头不卖,就像是无源之水。如今已经干涸,市面上远不止我们需求。”

胆识蛊是壮魂的第一手段,毫无任何后遗症,更关键的一点是见效奇快。

幽魂魔尊曾评价过,此法为天下第一壮魂法,说的无错。

对于唐方明而言,胆识蛊更是救命手段。

“没有胆识蛊的话,那哥哥……”唐妙心中不由焦急万分,语气中流露出一丝惊慌之音,复又抱着万分之一的希望,问道,“怎么就忽然不卖了?市面上难道就没有丝毫流通?”

这次换做唐家太上大长老缓缓开口:“我们也曾联系卖家,许以重利。但对方毫无回应。最近这些天吊着方明的性命,全靠我族大力收购市面长残留的胆识蛊。勉力支撑到现在,终是胆识蛊数量越来越少。如今近乎于无了。”

唐妙闻言,一颗心沉入谷底,呼吸也不由地紊乱。

普通的壮魂手段,唐家不是没有。之前胆识蛊买卖没有在宝黄天出现时,他们就是靠着这些手段,吊住唐方明的性命。

但如今,唐方明陷入梦境太深。魂魄衰弱的速度太快。这些手段已经没用了,恢复的程度跟不上他衰弱的脚步。

没有了胆识蛊,唐方明立即面临死亡的险境。

这可如何是好?

唐妙咬牙。紧皱眉头,深深地盯着眼前的蛊仙们:“我不管!若非当初你们不将这任务授给哥哥,哥哥也不会出现这事。现在哥哥处境这样,你们就这样袖手旁观?”

“话不能这么说。什么叫袖手旁观?这些年来我们唐家上下。为了吊住方明的性命,尽了多少的努力,付出多少的代价,你也是当事者,都一一看在眼里的……”太上三长老反驳道。他例举了无数事例,并非空口白牙,的确是事实确凿,有理有据。

“我不管!”唐妙的眉头皱到极致。青黛色的眼眸中此刻酝酿着一股气愤,仿佛是表面平静。内里汹涌的深潭。

“三长老,不必再说了。”太上大长老伸出手,阻拦道。

三长老住口不说,太上大长老望着眼前倔强愤怒的唐妙,便深深地叹了口气。

唐方明、唐妙兄妹情深,唐家这些蛊仙都心知肚明。

这对亲兄妹幼年时,曾经因为唐家高层夺权,被无情抛弃。兄妹俩历经困苦,艰难存活下来,在屡次的奇缘眷顾之下,哥哥唐方明竟修为蛊仙,妹妹唐妙亦成为五转巅峰的蛊师。

两人为报仇雪恨,杀回唐家。

唐家的蛊仙们出面,和唐方明协商。最终唐家主动牺牲了一干凡人高层,换取了唐方明的回归。

和一位蛊仙相比,就算牺牲再多的凡人又算得了什么?

唐方明审时度势,加入唐家,靠着唐家助力,又提携妹妹唐妙,帮助她成为蛊仙。

当唐妙终究不如哥哥那般胸襟,对唐家仍旧含着偏见,隐藏着怨愤。

此时在哥哥命不保夕的情形下,对她劝说什么都是无用的。

于是太上大长老一脸浓郁的哀色,对唐妙道:“我和诸位太上长老商议了三天三夜,对于方明,如今只有三策。”

“哪三策?”唐妙面色微微一缓,立即问道。

“上策,是寻找一头上古魂兽,斩杀之。将其魂魄种入方明体内,再辅以手段杀招,以魂灌魂,令方明渡过眼前难关。”

“中策,是利用族中保留的那只血道仙蛊,实施感应之法,强行召回方明的魂魄。”

“至于下策,则是寻求那位散修白海沙陀的帮助。”

太上大长老缓缓说完,脸上悲苦神色越加浓郁。

唐妙静静地听完,陷入良久的沉默当中。

好一会儿,她忽然冷笑一声,抬起双眸,逐客道:“此事我需要考虑,诸位长老请回罢。”

蛊仙们一一离开,只剩下唐妙一人。

她再轻声一笑,双眸里滑落泪珠,嘴角上尽是嘲讽冷意。

“什么上策?一旦用了,就算成功,也会魂魄相融,变得人不人,兽不兽。至于中策,成功的可能不足三成。就算没有失败,召回的魂魄也是极为残缺,说不定哥哥连我这个妹妹都不认得了。唯有下策还有希望!那白海沙陀虽是散修,但梦道造诣惊人,就算是哥哥也多次私下赞佩。然而请他出手,需要极大代价。这些老东西这么说,无非是惦念着我兄妹手中掌握的乐土仙尊的传承!的确,也只有这道传承,才能请的动白海沙陀罢。”

想到这里,唐妙伸手擦干脸上的泪痕。

“哥哥,我不会放弃的。就算是牺牲了这道传承,我也要将你救醒。”

唐妙盘坐在床榻上,昏暗的光线中,人却如花般亮丽。(未完待续。)

\end{this_body}

