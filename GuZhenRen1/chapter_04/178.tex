\newsection{方正苏醒}    %第一百七十八节:方正苏醒

\begin{this_body}

%1
在狐仙福地的南部,少部分的石人仍旧沉眠在地底,因为有大量血毒棠生灭在石人地道中的缘故,这些石人的地下家园,将在数十年后产生血石矿藏。隐藏智慧蛊的地下石窟,则利用江山如故进行了还原,芝林萎靡不振,但无需治疗,它们会自行排毒,甚至还有一定几率,结成血芝林。

%2
在福地北部,地域狭小,设置成了长恨蛛群的豢养基地。

%3
在福地中央,仍旧是荡魂山。荡魂行宫中关押着东方长凡的魂魄,残留的数十位东方俘虏。仓库中则存放了大量的仙材。上万斤的油水,天地沙鸥的巨大死蛋,大量的气囊蛊存货。荡魂山附近,则有三座方源石巢,里面生活着许多毛民奴隶,不停地在炼蛊。至于之前的那一小批镜柳林,却是损毁了,生机丧尽。以镜柳柳叶为食的乞丐蛾群,已无利用价值,被方源抛售出去。

%4
如此一来,幽火龙蟒、长恨蛛、龙鱼这三大资源,都真正建设起来。不像之前的单纯存放,这样发展下去,三大资源可以一边豢养繁衍,一边售卖获利,已经是步入正轨,

%5
只是目前这三大资源都在囤积存货,还不能急着挤进市场。

%6
荡魂行宫的仓库中,还有大量的气囊蛊。但方源也不能一下子抛入市场去大赚一笔,仙鹤门还在盯着狐仙福地。这样的风头还是少出为妙,徐徐出售,才能降低风险,细水长流。

%7
狐仙福地建设完毕。太白云生便在第二天向方源告辞。

%8
这期间,他出了不少力气。没有他的帮衬。方源要完成这么大的工作量,至少还需要半个月。

%9
甚至他不仅出了力气。还借了一笔仙元石给方源。

%10
方源本来仙元石就不多,前期改造建设,资金投入缺口很大,太白云生主动送给方源一笔仙元石,弥补了这个缺口。

%11
说起来,如今签订雪山盟约的四人中,反属太白云生财力最厚了。

%12
他在东海发展的很好,仙窍中移进了大量资源,使得仙窍都显得底蕴不足了。地气稀薄,有不稳趋向。这一次太白云生回到狐仙福地,不仅是为了帮助方源,还有一个目的,就是回到北原,落下福地,汲取地气,同时迎击地灾。

%13
不过地灾降临的时间还早,汲取地气也需要一段时间。到地灾来临时。方源也必然会出手相助太白云生。

%14
太白云生离开了狐仙福地,黑楼兰却仍旧留守这里。

%15
方源很快也离开了狐仙福地,去往中洲。

%16
没有别的原因中洲炼蛊大会即将召开!

%17
中洲,飞鹤山。仙鹤门。

%18
云雾缭绕着这座悬空巨山,群鹤飞舞在山林松涛之间。

%19
仙鹤门的一处地下密室,潮湿阴暗。石壁上青苔片片。房中除了一张石床,再无其他家具。

%20
石床上。躺着一位青年蛊师,脸色苍白如纸。身上毫无伤势,却是沉迷不醒。

%21
正是古月方正。

%22
一只蛊虫,趴在方正的胸口处,一动不动。

%23
忽然间,石室中微光一闪,旋即出现一位蛊仙。

%24
他大袖飘飘,面冠如玉,少年模样,最为引人瞩目的是他的眉毛。他的眉毛碧绿修长,眉尾一直垂到腰间。眉毛下,目光幽幽。

%25
“鹤风扬大人!”趴在方正胸口的那只蛊虫,立即振翅飞舞起来,激动地叫破来者的身份。

%26
这蛊虫正是寄存着天鹤上人魂魄的寄魂蚤。

%27
“大人,请您出手,治疗方正的魂魄,让我的徒弟苏醒吧。”天鹤上人道。

%28
“天鹤,你还是这个答案么……”鹤风扬叹了口气,心中有些失望。他更愿意天鹤上人夺舍重生,为什么呢?

%29
因为天鹤上人,本来就是他鹤风扬的下属,为鹤风扬办了不少事,知根知底,是个人才。反观古月方正,实在太过年轻,又有被迫牺牲的经历,万一对仙鹤门心存芥蒂,那就更不美了。

%30
“天鹤,你知不知道,我们仙鹤门这次得到的夺舍法门,堪称完美,极为优异。你若之前夺舍,兴许**魂魄不适应,会留下后遗症。但现在夺舍,仿佛生来就是如此,魂肉交融,相处融洽。一点弊端都不会有的。这是你的运气啊,你可以重生,再次重拾你曾经拥有的辉煌。机会上天已经赋予给你了,你为什么不把握住呢?”鹤风扬劝道。

%31
“大人……”寄魂蚤簌簌而动,声音低沉,“请原谅我的执迷不悟吧。若是真的夺舍了,我恐怕连自己都面对不了。方正,他的是我的徒弟。就像是曾经的我,这是属于他的未来。如果我夺舍了,我会面对不了自己的内心。请成全我吧,大人,念在我多年服侍您的份上。”

%32
“你!”鹤风扬脸上涌现出一丝怒气。

%33
他最近过得并不如意,为了救治九宫鹤小九,鹤风扬四处求人,耗费大量钱财,却收效甚微。

%34
现在,就连昔日的下属,都要一意孤行,不听他的劝告,不让他如意!

%35
这夺舍之法,虽然强大,但也有前提。不能贸然夺舍,最好在夺舍之前,就让魂魄和要夺舍的**长时间接触。

%36
天鹤上人当初打算夺舍方正,就将寄魂蚤寄托在方正的空窍中,时间过去了这么久,夺舍的条件已然成熟,但天鹤上人却改变了主意。

%37
也就意味着,天鹤上人若还要夺舍他人,短时间内就不可行了。还需要一段时间,接触肉身。

%38
至于东方长凡夺舍东方余亮,也是暗中将魂魄的一部分,存放在东方余亮的体内,手段巧妙,东方余亮没有发觉到任何的不妥。

%39
鹤风扬这些天过的颇不顺心,心烦气躁,当即就想喝斥天鹤上人。

%40
但心中的话却始终说不出口,因为鹤风扬不禁从天鹤上人的身上联想到自己。

%41
他曾经不也和天鹤上人一样么?

%42
为了救下小九,一意孤行,甚至愿意牺牲自己。当初的自己面临绝境,是仙鹤门太上三长老虎魔上人伸出援手,帮助了他,才有现在的鹤风扬。

%43
想到这里,鹤风扬长叹一声:“也罢,也罢。既然你执意如此,我鹤风扬就成全你吧。这是念在你昔日为我办事,补偿你的。将来你若想再夺舍,再请我出手,就不是无偿的了。”

%44
说完,鹤风扬挥起大袖,往古月方正脸上轻拂过去。

%45
这个动作结束之后,鹤风扬便骤然消失在原地。

%46
几个呼吸之后,古月方正魂魄凝实如初,从昏迷中苏醒过来。

%47
“你醒了,徒弟!”寄魂蚤激动得在他眼前飞舞。

%48
古月方正懵懵懂懂:“师,师傅,我们这是在哪里呀?”

%49
“当然是在仙鹤门啊,笨蛋徒弟。”天鹤上人大笑道。

%50
“我怎么会在这里?”古月方正双眼仍旧失神,口中喃喃,忽然一个寒颤,脸上浮现出惊恐之色。

%51
他想到了那个血池,攻打狐仙福地,自己身上却生长出血藤的恐怖回忆。

%52
“啊!”古月方正发出一声惊叫,下意识后退。

%53
结果后脑勺撞在坚硬的石壁上,他仰头而倒,当场昏了过去。

%54
天鹤上人:“……”

%55
三天之后。

%56
仙鹤门正式宣布,古月方正苏醒的消息。

%57
再次出现在一干弟子眼前的古月方正,却悍然拥有了五转修为。

%58
清晨的早课上,仙鹤门掌门当众任命古月方正,为仙鹤门长老。

%59
一时间,门派震动!

%60
昔日方正的竞争对手,诸如孙元化等人目瞪口呆,弟子们、精英弟子们哗然一片,长老们议论纷纷。

%61
仙鹤门的蛊仙,为了方便攻略狐仙福地,将方正的修为强行拔升到五转程度。弟子的身份,已经再不适合方正了。

%62
新的地位,新的身份,让古月方正刚刚适应时,大有手足无措之感。

%63
他走在路上,遇到每个弟子,都得向他行礼问好,这其中有很多年龄比他还大的精英弟子。

%64
有时候,见到孙元化这些曾经的竞争对手,他们也是毕恭毕敬。许多美貌的女弟子们常常围绕着方正,有的水汪汪地看着方正,有的甜甜地叫喊方正长老。

%65
和其他门派长老相比,方源年轻得过分,可谓独树一帜,传奇般的经历吸引了许多女弟子的关注,甚至是倾心暗恋。

%66
即便是方正本人,有时候看到其他长老,都还会忘记自己的身份,向这些长老们行弟子之礼。搞的每次场面都有些尴尬。

%67
幸好方正身边,还有天鹤上人辅佐帮助。仙鹤门的掌门,也时不时地,召见方正,问他有没有生活上的困难,嘘寒问暖,让古月方正颇有受宠若惊之感。

%68
更令古月方正欣喜不已的,是门派针对长老的优厚待遇,比弟子们的待遇要高出数十倍。

%69
这些天鹤上人看在眼里,隐隐明白用意。

%70
之前,门派高层故意牺牲方正,拿他的命去增大攻略狐仙福地的可能。此时加以笼络,便是想恢复方正对门派的忠诚。

%71
方正到底不是方源,年龄太小,涉世不深,对这层用意毫无所觉。

%72
事实上他对仙鹤门从未有过怨言,这样一来,成为长老之后,他对仙鹤门有了更多的归属感。

%73
对于曾经攻略狐仙福地的恐怖记忆,他都下意识地回避,不去想它。只是有时候深夜,梦到血池中的景象,回忆起曾经的痛楚,他都会满头冷汗地惊醒过来。

\end{this_body}

