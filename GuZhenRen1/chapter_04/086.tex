\newsection{终于有了智道传承}    %第八十六节:终于有了智道传承

\begin{this_body}

纵然有胆识蛊的生意,但时日太短,方源手头的仙元石并未积累丰厚。

再加上这期间,炼蛊,凝练蝠翼等等不断的消耗,更使得积累的进展缓慢。

拍卖大会、炼道大会两大盛事,将会接踵而至。尤其是前者,已近在眼前。这个时候从哪里弄得到仙元石巨款,唯有琅琊地灵手中的蛊仙福地。

“可惜我的仙窍福地已死,无法吞并其他福地。否则第一个要买下福地的人,就是我啊!”方源心中暗叹。

仙窍是一个个的小世界,可以相会并吞,但也有讲究。

首先,死窍不能吞并,只能不断崩解溃散,直至消散至无。

其次,仙窍之间不能以小吞大。

最后,要吞并六转福地,需要大师级境界,如此才能调理整合两个不同的世界合而为一。要吞并七转福地,需要宗师级境界。要吞并八转福地,则要求大宗师级。

也就是说,方源若要吞并其他仙窍,首先得是重新转生,仙窍成为活地。其次他只可以吞并血道、力道、炼道、奴道、变化道蛊仙的福地。血道、力道虽然是大宗师级境界,但方源六转修为,也只能吞并六转福地。

即便有如此限制,能够吞并,蛊仙们绝不手软。

皆因仙窍乃是蛊仙修行之根基,吞并福地,是夺他人根基,使得自身修为暴涨的无上捷径。

只要是蛊仙,都会清楚福地的价值。关于福地的拍卖,一定会在拍卖大会中掀起一波巨大的高潮。

接下来的日子,方源一边炼制群力蛊、小家子气蛊,一边利用智慧蛊光晕,试着改良变化道杀招见面不相识。

与此同时,宝黄天中的荒兽尸体,仍旧在售卖。

狐仙福地时间过了小半个月,方源手中的唯一一头七转兽尸气罡飞天猪,成功卖了出去。

方源获得了一份他很久以前,就朝思暮想的智道传承。

“恶念蛊……”几天后,方源手里捏着一只五转凡蛊,神情复杂。

这恶意蛊,就是他按照智道传承中的蛊方,亲手炼制出来的。

它体型颇大,有脸盆大小,形似蜘蛛,但八只触脚全如铁针,脚端尖锐至极。蜘蛛身上,长满坚硬的倒刺绒毛。若非方源仙僵之躯,单凭普通的血肉凡躯,摸一下此蛊,就能扎出血来。

方源灌注真元,顿时一个个恶念,在脑海中产生。

这些恶念,仿佛球泡,但表面也是生有倒刺。方源试着用这些恶念思考,顿时颗颗恶念相互碰撞在一起。

它们碰撞的情形,也十分奇特。

球泡表面的倒刺,一个个勾连起来,随后或是泯灭,或是融合成更大的念头。

很快,这些恶念消耗一空。

方源睁开双眼,目光中微微流露出欣喜之色。

“原来哪怕是智道蛊仙,推算的主要手段仍旧是用的念头。只要用着恶念思考阴谋诡计,算计别人的计划,都会效率倍增。一个普通的恶念,能相当数个,甚至十多个的乐念、星念、空念。”

方源先前用乐山乐意仙蛊,催生出海量乐意,辅助思考。其实却是走入误区。

意志虽然是由念头组成的,但真正的用途,并非是用来思考,而是分离本体之后,自主行动。

蛊仙们制造出一股股的意志,进入宝黄天,帮助售卖货物。巨阳仙尊留下巨阳意志,看守八十八角真阳楼,这才是意志的正确用法。

打个比方,念头好比菜刀,意志仿佛长柄大刀。思考问题宛如切菜,切菜当然是菜刀好使,谁家厨子用长柄大刀切菜的?

隔行如隔山,智道又是最为神秘的流派,方源先前不知道这点,动用乐意思考问题,看似效果不错,其实多有损耗,反而不如用乐念。

“我现在有了恶念蛊、恶意蛊的蛊方,完全可以大批量的炼制这两种凡蛊。我已成仙,真元无限。催生出大量的恶念来,物廉价美。消耗青提仙元,催动乐山乐意仙蛊,代价太昂贵了。”

以前,方源手头上只有数量不多的智道凡蛊,都是从宝黄天中买来的。

用这些凡蛊催生念头,念头数量太少,且种类不同,难以支撑推算仙蛊方时的剧烈消耗。

因此,才用的乐山乐意仙蛊。

现在方源有了智道凡蛊的蛊方,首选之法当然就是大规模制造这种凡蛊。一只凡蛊产生出的念头稀少,但凡蛊数量上去之后,念头的规模就变得很可观了。

使用乐山乐意仙蛊,不仅代价高昂,而且大材小用。

“可惜的是,这智道传承出现的有点晚啊。我和琅琊地灵的仙蛊方交易,已经步入了另一个阶段。若是早先出现,我必定能节省出大量的青提仙元。青提仙元节省下来,成本就骤然下降,赚取的每一笔利润就大大提高,绝非之前的二十六块左右的利润了。”

方源想到这里,心中也有一点遗憾。

但没办法,生活就是这样。

不可能凡事都恰到好处的出现,让你称心如意的。

总会有不圆满,总会有不如意,总会有不及时。

就像方源得来的这道智道传承,也不完整,相当残缺。

总的来说,方源能够拥有恶念蛊、恶意蛊的凡蛊蛊方,已经是一件幸运的事情了。除了这两道蛊方之外,还有一个智道杀招,名为包藏祸心。

卖掉蛊方的那位蛊仙,似乎对气罡天地猪的尸体,有某种迫切的需求,手头似乎也有些紧。

若非如此,就算是上古荒兽的尸体,价值数百块仙元石,对方也不会转卖掉智道传承的。

哪怕是这等残缺的智道传承,在市面上也极为稀缺。

于是,接下来的日子,方源安心在狐仙福地中,不断赶炼恶念蛊。

“恶念凡蛊虽说不是消耗蛊,但毕竟是凡蛊,也有着极限。而我真元无限,催动得次数多了,时间久了,这些恶念蛊总会损坏。”

抱着这样的想法,方源觉得他今后对恶念蛊的需求,将会一直保持在一定的程度。

而但凡炼蛊,都似乎较为耗费时间的。尤其是蛊虫转数越高,耗费的时间就越长。很多蛊仙为了炼制一只仙蛊,常常是数十年,上百年,甚至数百年。

这样想着,方源就决定再投入大笔资金,去收购毛民。

这段时间内,他又贩卖出去几批胆识蛊,每一批都有四十八块仙元石左右的高额利润。

他在手中只保留了五十块仙元石,将其余大部分的仙元石,都用来大肆收购毛民。一度抬高了宝皇天中毛民的价格。

方源最想要的,当然是琅琊福地中的毛民。这些毛民都培养出色,比宝黄天中最好的货色都要高出几个档次,可惜他问了琅琊地灵之后,遭到严词拒绝。不仅如此,还被琅琊地灵痛骂了一顿,说方源不够朋友。

方源虽然被骂了,却很开心。

琅琊地灵骂他不够朋友,说明琅琊地灵已经将他当做朋友,尽管这份友谊,远不如琅琊地灵和墨人王墨坦桑之间。

不过墨坦桑和琅琊地灵的情谊,不是这一代,而是追溯到很久很久之前墨人城的某代城主。无数代下来,墨人城一直暗中和琅琊地灵保持着联系。这份情谊,是时间积累的厚度。

“琅琊地灵新近解放,又得了那么多完善的仙蛊蛊方。按照他的脾气秉性,肯定要炼仙蛊的。而炼制仙蛊,消耗的将是庞大无比的资源,就算是富得流油的琅琊福地,也支撑不住。宝黄天中可进不了蛊仙。过了拍卖大会,就在没有像样的机会。他要贩卖蛊仙福地,对他而言,也是迫在眉睫的事情。”

方源知道,琅琊地灵已经暗中焦急。只待他绷不住,他就会主动来找方源。

琅琊地灵的关系网太简陋了,只有一个墨坦桑。

但墨坦桑是有墨人城这个家业的,又是异人,顾虑重重,怎么可能去出这样的风头,卖这些蛊仙福地?

异人的地位,低于正统人族。异人蛊仙也是如此。

看到异人蛊仙俘虏了人族蛊仙贩卖,必然会引发人族蛊仙同仇敌忾之心的。

也就是说,除了找方源之外,琅琊地灵没有其他门路。

这其实也不怪他,毕竟是地灵,出不去福地。且琅琊福地乃是货真价实的宝地,财帛动人心,福地又可以被吞并,减少和蛊仙的接触,也是琅琊地灵的自保之举。

琅琊地灵暗中焦急,方源就更不急了。

接下来的两个月内,方源在狐仙福地中,又建造出了一座方源石巢,并且将石巢当中,所有的房间都用毛民充满。

这些毛民,都较为擅长炼蛊,是宝黄天市场里中上等的货色。

上等毛民奴隶,方源虽然也买得起,但数量就少很多了。方源买下这批毛民,不仅要注重质量,还要兼顾数量。

有了第二座石巢,气囊蛊的产量就会暴涨,带动胆识蛊的产量增加,方源每一次交易得到的仙元石就会更多。

不过方源暂时没有命令全部的毛民,都去炼制气囊蛊。

第二座石巢,方源专门用来炼制恶念蛊!

\end{this_body}

