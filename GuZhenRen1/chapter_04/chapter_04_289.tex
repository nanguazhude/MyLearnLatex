\newsection{顺利回城}    %第二百九十节:顺利回城

\begin{this_body}

%1
一声嗡鸣。

%2
夜叉章鱼的地巢中的某个角落,翠绿的光辉陡然闪耀,周围的黑暗随之消退。

%3
翠光一闪即逝,下一刻,原本空无一物的地方,出现了一个人影。

%4
正是刚刚从蛊阵空间中,脱身而出的方源。

%5
他利用的正是定仙游仙蛊。

%6
而这个地方,正是他之前探索夜叉章鱼的地下巢穴时,特意留下的布置。

%7
他在这里布置了不少凡蛊,组成了一个简陋的蛊阵。

%8
没办法,方源的阵道造诣一点都不高,只能布置这种程度的蛊阵。

%9
蛊阵只能掩盖大部分的动静,但定仙游毕竟是仙蛊,随着方源的降临,这座凡蛊蛊阵立即崩解,仍旧有气息泄露出来。

%10
方源连忙动手施为,尽量将这些痕迹清除。

%11
但很快,一声荒兽嘶吼,地洞中也传来夜叉章鱼奔行而来的动静。

%12
“这附近的夜叉章鱼,已经回来了?”

%13
方源眼中精芒一闪。

%14
他加紧动作,在夜叉章鱼赶来之前,清除了仙蛊气息后,迅速抽身离开。

%15
夜叉章鱼感受到仙蛊的气息,因此躁动而来,但赶来这里的时候,已经是干干净净。它简单搜寻了两遍,脸上的犹疑之色尽数消退,复又折返回去。

%16
若是蛊仙,就绝对不会这么好骗。

%17
荒兽智慧欠佳,才会被方源如此轻易应付。

%18
不一会儿。回到口蚯处的战场,这里已是一片狼藉。

%19
大部分的夜叉章鱼都已经撤退,而北原仙僵们也不见了踪影。

%20
一道巨大的沟壑。像是地沟峭壁上一道狰狞的伤口,横霸在方源的视野之中。

%21
方源嘴角微微翘起,发出无声的微笑。

%22
他知道,这到沟壑是口蚯钻出来的。

%23
口蚯是一种地行生物。

%24
它常年埋藏在之中,只有捕食的时候,才会猛地钻出来。在刹那间将猎物吞入腹中之后,它就会闪电般地缩到土中。一动不动。

%25
与此同时,它的身躯会缩小到一个惊人的程度。简直是从几百斤的大胖子,立即瘦成竹竿。

%26
这个时候,它体内厚实的肉壁,会紧紧地挤压在一起。肉壁上生长的独特器官无数的钢骨利齿。不断旋转,将猎物绞碎成血水骨渣。

%27
“这条口蚯吞食了我,但我在它体内立即动用了定仙游逃离。它捕猎失败,空腹之下,必定暴怒,和北原仙僵们展开大战。口蚯不会是仙僵们合力的对手,它没有智慧,必然是落败的结局。不过这种野兽逃生的本能,也是十分强大的。一旦受创达到一个限度。它就会重新钻入土中,迅速逃离战场。”

%28
方源眼中幽芒一闪即逝。

%29
若不是这样,他也不会选择利用口蚯。逃离众仙僵的视线。

%30
之前他在阴流巨城中收集到的情报,也显示着夜叉龙帅等人,并没有有效的手段,来克制钻土的口蚯。

%31
他们的侦查手段,也难以逾越口蚯身上丰富的土道道痕,所以方源才敢在它的体内动用定仙游仙蛊。

%32
因此。北原仙僵虽然人数众多,战力也强。但短时间内未必能把口蚯怎么样。

%33
尤其是口蚯无心恋战,一味地钻土逃生。

%34
战力并不意味着一切。

%35
像雪胡老祖,修为高达八转,能以一敌二,战退药皇和百足天君联手,战力算得上非常强悍了。但为了铺设蛊阵,还得七次邀请孙名录呢。

%36
就算是九转尊者级,也是如此。

%37
蛊仙历史上,巨阳仙尊、盗天魔尊,先后请长毛老祖出手炼蛊。

%38
蛊修流派百花齐放,精彩纷呈,也意味着术业有专攻,就算将其中一道修行到了九转级数,其余方面也大多是隔行如隔山。

%39
毕竟精力是有限的。

%40
除非拥有极其多的时间,或者相当程度的奇遇。

%41
口蚯钻地潜逃,北原仙僵们穷追不舍,因此战场的痕迹就一路延伸出去。

%42
在这一路上,很多地方都可以看到:被猛烈轰炸的深坑,沟壑中不少的血液,应当是口蚯之血了。除此之外,还有许多被殃及池鱼的地沟生物的碎尸。

%43
口蚯为了活命,也是拼了,无形中为方源遮掩行踪帮了大忙。

%44
一方逃,一方追,形成了这道漫长的战场。

%45
看着激战的痕迹一直没入视野的尽头,方源轻轻一笑。

%46
“口蚯逃命,必然慌不择路,将北原仙僵们带到其他的猛兽的领地中去。很好,场面越是混乱,就越有利于我。这正是我之前料想的情形之一。”

%47
旋即,方源顺着激战的痕迹,迅速飞行,身形很快就没入黑暗之中。

%48
北原仙僵们垂头丧气地缓缓飞行着。

%49
他们伤势都不轻,狼狈不堪,神情灰败。

%50
“这可如何使好?星象子丢了性命,让我们如何回去向焚天魔女大人交代呢?”

%51
“唉,我们已经尽力了。一路追杀口蚯,先后穿过了五块猛兽的领地,其中有三块都是由上古荒兽盘踞着!要知道这可是在地沟!我还从未有今天如此疯狂过。”

%52
“可惜最后功亏一篑,那头口蚯落入酸沼苔藓当中,彻底溶解了。星象子毫无幸理,我们总不能为了他,直接搭上性命吧。”

%53
“要我说啊,还是星象子太不小心了。在地沟中,居然如此麻痹大意,简直是自寻死路!”

%54
众仙僵低声交谈着,唯有领头的夜叉龙帅沉默不语。

%55
他在这群人中地位最高,也是首脑,星象子丧命,他的责任最大。

%56
一想到回去要面对焚天魔女。夜叉龙帅心中就极为沉重。

%57
“诸位,终于见到你们了。”方源化作星象子,出现在仙僵们的面前。

%58
“嗯?”一下子。北原仙僵们都全都楞了。

%59
“星象子!?”玄阴医师声调一扬,目光中满满都是惊喜。

%60
其余仙僵不外如是,就连雷雨楼主,此时看着方源,也觉得眼前这老头十分顺眼!

%61
“你怎么逃出来的?”旋即就有仙僵发问。

%62
这个问题,也是他们所有人的疑惑。

%63
数道目光,都集中在方源的脸上。

%64
方源早有准备。施施然说出理由:“惭愧!我落入口蚯肚中后,四周肉壁就推压过来。无数利齿紧随其后。情急之下,我连忙动用仙蛊,催出仙道杀招,将自身凝于星冰之中。这是我最强大的防御。抵御利齿不成问题。只是有不少弊端,首先不能动弹,不仅是肉身,就连脑海中的念头,都一动不动,因此不能思考。星冰的存在时间只有片刻,不过我相信诸位会伸出援手。只是等到星冰消融,我恢复自由的时候,却发现自己被埋藏在碎石堆下!”

%65
“原来如此!”玄阴医师恍然大悟。

%66
“一定是我们追杀口蚯途中。混战之下,将口蚯的身躯打破,星象子就顺着破洞漏了出来。”

%67
“好险。好险!”

%68
“星象子,你也算是福大命大了。”

%69
众仙僵不疑有他。

%70
夜叉龙帅也着实松了一口气:“如此便好,还要继续搜集星夜黏涎吗?”

%71
方源面色微变,露出动摇犹豫的神色,道:“还是算了!这次运气不好,先回去整顿。休整之后再来吧。”

%72
几位仙僵对视一眼。纷纷暗笑星象子的胆小。

%73
不过如此激战,他们的自身状态也不好。因此都同意先回去阴流巨城,进行休整。

%74
至于缺少的星夜黏涎,只能留在以后再说了。

%75
几天后。

%76
阴流巨城。

%77
密室中,炼道蛊阵正在运转,散发着灼灼的光辉。

%78
光辉映照在方源的脸上,但此时的他,却显得心不在焉。

%79
“可行,这是可行的!”方源心中念道。

%80
他表面上正在利用蛊阵,处理炼蛊材料,实际上则在自家的死亡仙窍中,操控各种蛊虫,进行试验。

%81
谨慎如他,虽然在地沟中得到了重生法门,但毕竟来路神秘,方源自然要先试验一番,验明真假。

%82
试验的结果,无不证明,这道重生法门确实可行!

%83
方源心中,既激动又感慨。

%84
回顾一下,方源为了收集重生之法,从王庭福地覆灭开始,就辛辛苦苦,辗转反侧。

%85
他首先收集到宙道重生之法用人如故仙蛊,搭配宙锚仙蛊,进行重生。

%86
但此法虽然前景广阔,但只是一个设想,要完成它,需要和鲨魔等人合作,投入太多,耗时太长。

%87
之后,方源又从东方长凡的魂魄中,拷问出夺舍之法,利用这个方法也可以达到摆脱仙僵之躯的目的。

%88
但若用这个方法,且不说肉身和魂魄能否相互适应,肉身资质如何,单说方源辛辛苦苦得到的第二仙窍,就打了水漂。

%89
再然后,方源又在中洲炼蛊大会中,从凤金煌口中,得知十绝体升仙,可以摆脱仙僵身份。

%90
这个法子可行性更低,最适合凡人蛊师。方源已经成为仙僵,若用第一凡窍升仙,本身又并非十绝体,资质不足。虽然有血颅蛊,但身负道痕,难以用凡蛊提升。

%91
最后,在繁星洞天的碎片世界中,凤金煌还告知方源一个重生之法。

%92
利用变化道的原理,结合永固仙蛊,达成永久性的变形,同样可以重生。

%93
不过永固仙蛊怎么炼制?方源毫不知情。

%94
若是他人掌握永固仙蛊,那就更糟糕了。

%95
而且变化道,方源也不是很擅长的。

%96
这些方法,虽然都能成功,但各有弊端,达成困难。

%97
不过皇天不负有心人,方源奋斗不息,历经辛苦挫折,终于在北原僵盟的地沟中,寻觅到了最合心意的方法!

%98
ps:请大家关注接下来的单章,本人有话说。

\end{this_body}
\newsectionindepend{开单章,求支持并征集龙套!}
\begin{this_body}
%99
亲爱的读者朋友们,最近的稳定蛊,好用不?

%100
本人为了炼出此蛊,堪称呕心沥血了,已经达到了现下生活情况的极限。

%101
希望大家,也能在本书中,收获更多的阅读快乐。

%102
行文至此,相信很多读者都看出来了。

%103
没错!

%104
这一大卷,终于要到结尾了。

%105
按照本书的行文架构,每一卷的结尾,必然是惊心动魄的大高潮了。

%106
第一卷,方源翻盘,战胜五转,利用血颅蛊提升资质。

%107
第二卷,三叉山炼出定仙游,收获狐仙福地。

%108
第三卷,八十八角真阳楼倒塌,成就蛊仙,拥有万我和智慧蛊。

%109
第四卷是什么呢?

%110
容我卖个关子。

%111
请大家耐心等候。

%112
这一卷同样是精心设计,我对此自信十足!

%113
希望大家多多支持我。

%114
尤其是在这严打和谐的关头……

%115
唉,能写到今天这种程度,有两个原因。

%116
一个是这本书是我一直以来的梦想!第二个始终有着你们,不离不弃的死忠粉!

%117
你们不仅可爱,而且宽容大度,很少有读者能像你们这样。

%118
有你们的支持,我感到相当的幸福。

%119
第二个重点,就是征集龙套。

%120
我会在起点书评板块,置顶一个全新的龙套征集帖,名为“第四卷《蛊真人》龙套征集”,大家按照格式跟帖。

%121
之前很多读者来求龙套,我满足了一些人的愿望,但还有一部分人,因为断更、遗忘等等原因,没有满足到。

%122
所以,这一次有意向的,请都来跟帖吧。

%123
之前向我求龙套的,我基本上都忘了。都以这次跟帖为准啊!

\end{this_body}

