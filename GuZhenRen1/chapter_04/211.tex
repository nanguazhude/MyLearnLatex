\newsection{自然炼蛊法}    %第二百一十二节:自然炼蛊法

\begin{this_body}

苦贝藏于深水之中。[八零电子书wWw.80txt.COM]

吞吃水中的沙石,能将沙石溶解,化为苦水。有人撬开它的贝壳,得到这种苦水,用来酿酒。酿造出的苦贝酒,口感又苦又香,十分独特。

曾经方源获得过苦贝,而后撬开苦贝,用苦水炼酒,有用苦酒炼成了酒虫。

苦贝已经比较稀罕,千年苦贝就更加少见,是炼制仙蛊的佳材。

方源手中苦贝,便是千年苦贝。贝壳漆黑无比,又有一圈圈的白色纹路,仿佛树木年轮。黑白相间,分外显眼。

方源双目紧紧盯着龟壳下的火焰,至于手中的千年苦贝,他看也不看,就将其扔了进去。

炼制仙蛊,事关重大,自然要检查一切的仙材。

但方源早就事先检查了很多遍,可以确保材料方面万无一失。

因此方源现在炼蛊,不用一边炼蛊一边分心再去检查仙材,而是专注于火候的大小。

千年苦贝扔进去之后,毒血顿时不再沸腾,但毒烟毒气却滚滚而动,仿佛一条黑蟒在里面翻滚。

一股浓烈的恶臭,旋即产生。

方源一动不动,置身在恶臭当中,细心品闻。

他虽然身为仙僵,本身是闻不到任何味道的。但√▼长√▼风√▼文√▼学,¢▲.n≤et有侦察杀招辅助,却可以替代嗅觉。

这恶臭难以形容,糟糕至极,让方源闻了都头晕,想吐。

但方源必须坚持,因为这个恶臭也是检验炼蛊是否顺利,千年苦贝是否全部融化的标志之一。

很快。方源扔进去的千年苦贝,就被完全消融。黑色毒雾停止滚动。毒血又重新开始沸腾起来。

方源便接着,向里面扔进第二只千年苦贝。第三只千年苦贝……

总共溶解了十二只苦贝之后,毒雾浓郁到了极点,那强烈的恶臭反而开始透露出丝丝清香。

方源脸上的神色越加郑重。

……

“孪生冰心!!”毛民本多一大叫道。

他虽然只是五转蛊师,但炼道境界相当不凡,远超寻常。因而眼界宽广,余木蠢刚刚取出仙材,他就道破这个仙材的名号。[八零电子书wWw.80txt.com]

本多一一脸震惊的神色。

小拇指大的冰心已经是仙材,但是余木蠢此时取出来的冰心,却有脸盆一样大!

不仅是大。关键是这块冰心,外表特殊,是由两颗心型的白冰连在一起。

这就是孪生冰心,比寻常的冰心要珍贵一百倍!

普通的冰心,只是六转仙材,可以用做炼制六转仙蛊。孪生冰心,却是炼制七转仙蛊的仙材。

“这么珍贵的仙材,都是第一个拿出来。显然孪生冰心并不是炼蛊的主料,只是辅料。大师真是厉害。不知道他想炼成什么蛊。”本多一眼闪烁着崇拜的光。

同时,他也稍稍放下心来。

连这么珍贵的都用了,可见余木蠢大师是有准备的,不是随兴所至。

但本多一又想到:余木蠢居然在这里。在这个野外,大庭广众之下炼制珍贵的仙蛊。这举动,是该说余木蠢他脑袋缺根弦呢?还是艺高人胆大?

实在让本多一无法评论。

周围拂面的清风。越吹越大,很快大风呼啸。形成一个个的龙卷风。

龙卷风相互间隔,彼此并不干扰。汇成一片龙卷风林。

余木蠢位于半空,在龙卷风林之间,大风吹鼓,将他的青铜面具吹飞,露出一张毛发浓密的老脸。

余木蠢哈哈一笑,身躯一震,包裹全身的黑袍陡然震碎,碎片很快被风吹走,露出余木蠢健壮的躯干,浑身的棕色绒毛。

原来他真正的身份,竟然和本多一相同,也是一位毛民。

难怪他曾经指点了本多一。

毛民是异人,和人族是有所区别的。准确的说,是两种不同的种族。

余木蠢的这个身份,连方源都不知道。

像是了解本多一心中的担忧,余木蠢一边关注着手头上的工作,一边开口道:“你这个痴儿!咱们毛民为什么有炼蛊的天赋?就是因为毛民从一出生起,身上就有炼道道痕。甚至,炼道的产生就起源于毛民。论炼蛊历史,我们毛民源远流长,底蕴积淀是人族的数十倍!只是人族势大,剿灭了许多毛民,丧失了无数珍贵的炼道心得。如今人族的炼蛊思想反过来影响毛民,这里面多有错误。”

余木蠢继续道:“比如人族炼蛊,大多要强调安静,寻求一个隔绝内外的独立环境。这固然更加安全,提升炼蛊的成功可能。但却只注重眼前,不计较长远。”

“蛊师养用炼,养蛊是熟悉材料和蛊虫之间的联系,为什么这只蛊虫需要喂养这样的草料?蛊虫和食料之间,就蕴藏着无数的蛊方。至于,用蛊是体会蛊虫身上的道痕。炼蛊则更加关键,是对道痕的处理。在炼蛊时,其实更应该置身于大自然中!因为这个大天地,才是拥有道痕最多的地方。就好像是母体之于婴孩,天地才最好的炼蛊地点。若是能利用天地间的道痕,就能在某种意义上,借助天地之力帮助我们炼蛊!”

“借助天地之力,帮助我们炼蛊?!”本多一瞪大双眼,瞠目结舌。

余木蠢语出惊人死不休,这个理论本多一还是头一次听说,带给他振聋发聩之感。

余木蠢哈哈一笑,双眼中爆闪出一阵精芒:“本多一,你学习人族的炼蛊法,固然强盛,却已经走入了歧途。它山之石可以攻玉,但人族的东西到底不是你的。你要想再进一步,冲击炼道的大宗师境界,就要返本归元,回归毛民我族的炼蛊之道。”

“我愿意!我愿意!余木蠢大师,请你大发慈悲,教我毛民本族的炼蛊法吧!”本多一兴奋大叫,磕头不止。

余木蠢朗笑一声:“你不要急着做决定,这其中大有风险,很可能不是造就你,而是害了你。想我三岁起开始喂养蛊虫,十岁只须触摸蛊虫,无须指点,就可得知陌生蛊虫需要什么样的食料,该如何喂养。十六岁,用人族之法炼出五转蛊,成功十之**。但这其实是走了弯路。二十二岁我意识到这一点,决定重起炉灶,置身自然进行炼蛊。到了一百三十八岁,自然炼蛊法终有小成。如今我二百四十六岁,自然炼蛊法大成。可感悟自然,上知天文气象下知地理脉络,选择最适合之地,用天地间暗藏的道痕,为自己炼蛊谋求巨大的助力。如今,我炼六转仙蛊,成功十中有四。炼七转,二十有一。八转仙蛊,碍于修为,至今也未尝试。”

这一番话,本多一听得心头剧震。

余木蠢炼蛊的成功率,实在太高了。

通常而言,六转仙蛊的成功率,连百分之一都不到。

七转仙蛊,成功率是千分之一。八转则是万分之一。

而余木蠢炼六转仙蛊,十中有四。这就是四成的成功率。而炼制七转仙蛊,平均二十次就可成功一次。

这种成功率说出去,必定震惊整个天下!

这就是自然炼蛊法的厉害之处。

然而本多一也听出来了。

要炼成这个强大的法门,首先要讲究才情天赋。余木蠢才情天资非常出色。三岁喂养蛊虫,十岁触摸陌生蛊虫,就可得知该蛊吃什么。可见他已经明悟出,蛊虫负载的道痕碎片,和天地万物之间的联系。

余木蠢十六岁时,就可用人族之法炼出五转蛊。二十二岁就意识到这不是自己的路,于此毅然舍弃一身本事,重新修行。

从这点又可看出,余木蠢不仅有天资,而且还有野心和壮志。

其次,要炼成自然炼蛊法,还得需要大量的时间积累,大量的资源消耗。

余木蠢到一百三十八岁,才有小成。二百四六岁,才大成。

这其中必然有海量的练习,耗费的资源,难以想象的恐怖。

才情天赋首先抛开不谈,单单资源损耗这点,依照本多一这种势单力孤的情况,根本就不现实。

蛊仙修行,需要资源。炼道蛊师更加需要。炼蛊,没有大量练习,没有足够资源,就算是蛊仙,是玩不转的。

蛊仙中大部分的都是飞行大师,但只有少数,才是炼道大师。

方源没有在炼道上有更高成就,只成为了炼道准宗师,除了天赋有限,主要精力不在此道上之外,也有受制于资源匮乏的原因。

余木蠢见龙卷风林开始平稳,便开始真正炼蛊。

他立足于半空,毛发随风舞摆,从自家仙窍中或抓拿,或提捏出一份份的仙材。

他将这些仙材,有选择地抛入到一个个的龙卷风中。

龙卷犀利的风刃,将仙材迅速切磨成粉碎。

本多一看得都呆了。

按常理,炼制蛊虫每一份材料都要精确份量,投放的时间,处理仙材的火候。

但余木蠢却是十分随意,不谈恢弘的龙卷风林,他处理仙材的手法可谓粗陋不堪,根本不关注每种仙材的份量。就像是一个邋遢的厨师,炒个菜,这边随意放点油,那边根据感觉放点盐。

若是旁人用这手法,本多一一定嗤之以鼻。

但余木蠢此时使来,却有一种说不出来的流畅和天然。

他的每一个举动,普普通通,却都透出一股难以言述的美感,切合自然,蕴藏深邃的妙韵。

一时间,本多一仿佛石化,双眼瞪得溜圆。

他看呆了!(未完待续……)<!--80txt.com-ouoou-->

\end{this_body}

