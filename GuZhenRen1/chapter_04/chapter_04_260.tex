\newsection{修为低也有好处}    %第二百六十一节:修为低也有好处

\begin{this_body}

%1
方源展目望去,便见眼前的湖泊波光粼粼,湖面宽阔,蔚蓝多姿。

%2
他进入这片繁星洞天碎片世界之后,就置身在这片蓝湖之中。

%3
蓝湖面积很大,远处湖中心有几座小岛,郁郁葱葱。

%4
再望得更远,蓝湖的湖岸处,有一片叠嶂的山峦,宛若一抹青黛,与天际温柔交融。

%5
方源此时站在水面上,像是站着平地一般。

%6
换做凡人蛊师的角度,这一手立在水面上的手段,可俊得很。但如今这种凡俗的东西,对于方源而言,已经微不足道。

%7
再深入这片碎块洞天之前,方源最后朝身后望了一眼。

%8
只见背后,原本的一汪蓝湖突兀地断裂了,好像是矗立着一片磨沙镜面,将这片湖面直接斩断。

%9
方源目光透过镜子,可以看到外面,真是中洲的那座无名小山谷里还有山谷中那些蛊仙的身影,似乎也正用各种莫名复杂的目光,望着方源。

%10
方源心中呵呵一笑,旋即转身,踩爆脚下的湖水,激飞而去。

%11
中洲十大古派的蛊仙们不知道,但事实上,方源已经不是第一次,进入这个繁星洞天了。

%12
繁星洞天十分特殊,拥有七个子空间,相互之间隔绝开来,并不像通常的洞天一样。

%13
很明显,繁星洞天中的宇道道痕,一定十分的多,并且相当奇特。

%14
繁星洞天中有很多的荒兽,连上古荒兽都有,镇守第八星殿的尸龙、走肉树等等。让方源至今印象深刻。

%15
“不过经过激战的消耗,还有战仙宗的搜刮。还有繁星洞天破碎时的损失,这些洞天碎片世界中的荒兽、上古荒兽已经很少了。目前这片洞天当中。只有两头荒兽一株荒植。”

%16
方源一边在空中疾飞,一边在脑海中回忆着鹤风扬带给他的情报。

%17
这两头荒兽一株荒植,分别是一头荒兽龙鱼,一头藏经鼋,一株幽冥草。

%18
龙鱼,只要是蛊仙都知道,是喂养仙蛊十分常用的食物。绝大多数的凡蛊,若是特定的食料不够,可以利用龙鱼的肉。替代缺失的部分,喂养仙蛊。当然,仙蛊唯一,食料稀奇古怪,龙鱼的作用就下降了许多。就算是能够替代一部分食料的,也只能是荒兽级的龙鱼肉。

%19
“荒兽当中战力最弱的,几乎就是荒兽龙鱼了。根据传闻,本来天地间并不存在龙鱼这一物种,它是食道流派的创造之物。”

%20
而藏经鼋。这是相当罕见的智道荒兽。它的习性十分奇特,从成年之后,就固定在某个地方,一动不动。天长日久。风吹雨打,藏经鼋的浑身都会长满青苔,结成灰石。变成一座巨大的假山。加之它又擅长收敛气息,善于藏匿形迹。很不容易发现。

%21
只有当它预感到巨大的灾难的时候,它才会搬迁。一路不停,直到寻找到了中意的栖息之地。然后,它又会停下来,继续缩到壳中,靠着风雨霜雪等等维生。

%22
这种荒兽,若能收为己用,比斩杀了它更为有用。

%23
藏经鼋是纯粹的智道荒兽,天生贴近智道,乃是智道中灵兽。有着天然的能力,能够寻找到安全隐秘的地方生存。

%24
很多智道蛊仙,在推算难题的时候,就会利用藏经鼋,帮助自己。

%25
方源继承的东方长凡的智道传承中,也有一种手段,是利用刚刚斩杀的藏经鼋的壳,辅助自己推算。

%26
很明显,将这头藏经鼋收服起来,对蛊仙更有利。加之它一动不动,防御很强,所以战仙宗就没有急着动手。

%27
而幽冥草也差不多,它扎根在地中,无法移动,只要不靠近它,威胁很低。

%28
有一株幽冥草,尤其对魂道蛊仙相当有利。因为幽冥草的身上,蕴含着大量的魂道道痕。时间一久,就能影响周围环境。幽冥草附近方圆数百里的地方,都会渐渐地变成魂道蛊虫豢养基地。

%29
幽冥草对于改造福地洞天的环境,有着相当厉害的效果。

%30
同样的,战仙宗不方便收取这株幽冥草,所以才留到今天。结果没想到,七星子气性很大,宁愿玉石俱焚,将繁星洞天爆碎坠落。

%31
这个局面远远超出了战仙宗的掌控极限,这片繁星洞天的碎片世界,就在中洲十大古派的磋商之下,以商议好的约定,对里面的资源进行采集。

%32
鹤风扬代表仙鹤门,在进入洞天碎片世界之前,就关照了方源几个任务。

%33
其中有三个,就是关于龙鱼、幽冥草以及藏经鼋的。

%34
“若是你能够将这三者都生擒活捉过来,是最好的。按照约定,你能得到其中价值三分之一的部分。”鹤风扬如此明确地告知方源。

%35
方源虽然表面上是仙鹤门的附庸,但事实并非如此,鹤风扬乃是当事人,最清楚不过了。

%36
所以他也非常明白,要调动方源的积极性,只有利诱。

%37
虽然方源只是一个六转垫底修为的仙僵,但面对这三者,优势很大。

%38
因为方源手中,不仅有自己的仙蛊,而且还有仙鹤门特意借给他的仙蛊。

%39
有仙蛊,和没有仙蛊之间,战力完全是两个概念。

%40
虽然龙鱼、藏经鼋、幽冥草都是本身就很强大的生命,但它们身上是没有仙蛊的。

%41
若是有仙蛊,早就因为怀璧之罪,吸引蛊仙出手,不是七星子仙僵,就是战仙宗的那帮人,怎么可能还留存在这里呢?

%42
一般来讲,福地洞天当中,基本上没有野生仙蛊。

%43
因为野生仙蛊一旦产生,就会立即被蛊仙取走,自己用了。

%44
毕竟野生仙蛊放在那里,不仅对福地的经营无益,而且还会导致蛊仙对洞天福地的掌控不足。万一将来荒兽暴动。就麻烦了。

%45
而且,繁星洞天主人七星子被方源早早惊醒。又是繁星洞天之主,对洞天中有没有野生仙蛊。哪里会不清楚呢?

%46
就算七星子被纠缠得分身乏术,繁星洞天又被战仙宗攻略,搜地三尺。除非是极其特殊的仙蛊,能够隐藏自身,否则不会存于现在。

%47
“按照路线图,龙鱼就在这个片蓝湖当中,靠近湖心中央小岛的附近。”

%48
方源飞在空中,目光微闪,龙鱼、幽冥草和藏经鼋这三者中。他最感兴趣的,就是荒兽龙鱼。

%49
当然,龙鱼是群居生命。

%50
荒兽龙鱼的身边,还有大量的平凡龙鱼。

%51
这些,对于方源来讲,就是一笔不小的诱人财富。

%52
“想不到我这么快,就有第二头荒兽龙鱼了。嗯,既然能够取走收获的三分之一,那就选择这群龙鱼吧。”方源想到这里。心情也不免愉悦起来。

%53
而此时,在蓝湖湖心的小岛上,一队来自天妒楼的蛊师,正望着眼前的湖面。陷入深深的烦恼当中。

%54
这队蛊师有五人。几乎个个都是五转修为,只有一位少年蛊师,修为最低。但也有四转巅峰的修为了。

%55
这少年姓魏名无伤,乃是天妒楼此代的精英弟子之首。

%56
因为他爱慕碧莲小仙子。曾经还和方正切磋过,可惜败北。方正主动给台阶。认作平手。魏无伤保住了脸面,对方正有一些感激之情,认为自己欠方正一个人情。

%57
他原本只是四转中等,但是因为此次资源的争夺,天妒楼就专门出手,将他的修为抬到四转巅峰。

%58
一方面是锻炼门派中的后起之秀,未来希望,另一方面也是人手有些短缺。

%59
这一次,围绕着繁星洞天的各处碎片,中洲十大古派都展开了激烈的争夺。

%60
七星子十分阴险,将繁星洞天自爆,炸碎得很有分寸。碎片世界让蛊仙们几乎都无法进入,只能依赖门派中的弟子们进行探索。

%61
前文已经讲过,这些洞天碎片世界,道痕稀疏,像是破陋不堪,摇摇欲坠的房屋,只能持续一段时间。

%62
而蛊仙身怀道痕,就相当于巨人。强行进入这些碎片世界,便如同巨人硬挤进小屋中,得到的结果只能是小屋破碎,里面的资源随之毁于一旦。

%63
前世,方源成了血道蛊仙,建立血翼魔教。等到繁星洞天破碎坠陨时,他也是指挥魔教中的凡人蛊师,代替他进入碎片世界,搜刮资源。

%64
但现在,方源为什么能进去呢?

%65
说起来,还有些哭笑不得。

%66
因为方源在蛊仙当中,修行垫底,浑身力道道痕就没有多少,还因为见面似相识,不断地损失力道道痕。

%67
见面似相似的核心仙蛊之一,便是吃力仙蛊。逆用吃力仙蛊,消耗力道道痕,从而改变蛊仙地域气息。

%68
虽然也有巨角羊这等纯粹的力道荒兽,它的血肉让方源增长力道道痕。但因为最近这段时间的频繁伪装,加减抵消之后,增长的力道道痕也有限。

%69
所以方源进入这里,而其他蛊仙却是不行。

%70
他刚刚好达到了碎片世界的容纳上限,成为蛊仙的,修为方面像他这般凄惨的,还真的很少。

%71
关键是他的身份,让仙鹤门能够利用起来。

%72
这是十大古派的搜刮和竞争,若方源没有仙鹤门附庸的这层身负,作为外来蛊仙,是万万没有可能被十大古派承认的。

%73
这就是正道。

%74
讲究规矩。

%75
不像魔道,看谁的拳头大,硬碰硬或者阴狠险毒。

%76
魔道最大的规矩,就是不讲规矩。

%77
ps:ps:今天是六一儿童节,祝大朋友、小朋友们都节日快乐!看到很多暖人心的书评和帖子,唉,怎么说呢,真是惭愧得很……谢谢大家支持,这是本月的第一更,14点,本月都尽量在这个点更新了。下个月若有变动,会另行通知的。

\end{this_body}

