\newsection{赌斗凤金煌(终)}    %第二百零七节:赌斗凤金煌(终)

\begin{this_body}

两团鬼火一前一后,忽左忽右,时远时近,毫无规律,凤金煌防备这两团火焰,还要分出心神去炼蛊。

很快,凤金煌就支持不住了。

嘭。

一声微小的爆炸,她操纵的其中一团炼蛊火焰瞬间毁灭,里面的蛊虫都已经炼出了雏形,可惜功亏一篑!

“大,大师姐失误了。好可惜啊,那只蛊虫只差最后三个步骤了!”孙瑶气得跳脚。

秦娟脸色也沉下来:“这是赌斗以来的第一次失误,大师姐虽然底蕴很深,但分心关注这么多事,果然还是太勉强了一点。”

“那怎么办?”孙瑶求教地望过来。

秦娟沉吟道:“若是我,此时就会主动撤销一个炼道杀招。因为优势很大,只要维持住这个优势,自己不再失误,着急的应该是对方。不过撤去杀招,和催动杀招一样,因为产生了变化,所以会更加注意,如此便会形成瞬间的破绽。”

此时场中,凤金煌也在闪烁着同样的念头:“要撤销掉一个炼道杀招吗?方源就是这样的企图吗?”

“不,我还有办法。就是第三次出手,动用音波攻势,一下子打掉这两团鬼火!”凤金煌这种天才向来都是骨子里都有骄傲,不愿意认输。

凤金煌想到这里,双眼一亮,但当她的目光投向方源时,心中顿时一凉。

方源带着狰狞的面具,面具的空洞处他的目光冷冽如冰,似乎藏着丝丝嘲讽的笑意。

“不对!”目光对视之下。凤金煌心头一震,“这是方源的陷阱。他早就等着了!一旦我第三次进攻。方源就会抓住这个时机,瞬间出手。到那时我的心神已经超过极限。必定难以顾全局面,甚至都来不及防御,会让他击中!”

“可恶……”凤金煌咬牙切齿,一滴冷汗从额头滑向眉角,又顺着脸颊滑落下去。

“这样下去不行,我得想想办法,该怎么办呢?好好想想,快想想!”

嘭。

在凤金煌思考的过程中,她掌控的火团再次爆炸了一个。

这次爆炸。比第一次要严重许多。

周围的火团都受到影响,凤金煌连忙摒弃脑海中杂乱思绪,极力稳住其余的火团。

“啊!大师姐又失误了,怎么会这样?不好,大师姐受伤了!!”孙瑶抱头惊呼。

“方源魔头带给她的压力太大了,大师姐急思对策,一不留意,炼蛊方面的关注减少,便产生了失误。炼蛊失败自然要承受反噬。好在她受的伤并不严重。”秦娟语气也流露出紧张之情,她吐出一口浊气,又道,“不过。这也是一件好事。”

孙瑶不明所以,气愤地嘟囔道:“秦娟师姐,你这是什么话?大师姐炼蛊失误。又受了伤,为什么是件好事?”

“因为她手中的火团熄灭了两个。心神反而解放了一部分。现在的她,就有了余力。可以从容应付当前的局面了。”秦娟分析道。

孙瑶一拍脑门,恍然大悟:“对喔!”

“呵呵呵。”方源又开始笑。

凤金煌现在听到方源的笑,顿时心中就一沉。

她抬起头,果然便看见方源取出第三只蛊虫。方源仍旧道:“你看好了,这是一只三转鬼火蛊。”

于是下一刻,凤金煌身边同时飞舞盘旋着三团鬼火。

“可恶,可恶!”孙瑶气得跳脚,“这个魔头太无耻了,大师姐刚刚缓过神来,他就又飞出一团鬼火。他还想逼大师姐失误!”

“的确是这样。”秦娟一脸沉重,“如此一来,压力剧增。恐怕大师姐要将仅剩下的心神,都集中在防守方面了。”

孙瑶气得开始磨牙,下意识地挥动拳头:“这个魔头不好好炼蛊,竟然把主要心神都耗在鬼火上,这是典型的不务正业!这种人太坏了,真恨不得把他揍得满头包!!”

“可恶,怎么会这样……”凤金煌陷入巨大的困境当中,她明明领先,但当方源出手时,她就处处被动。

凤金煌心知不妙,因为赌斗的节奏已经全部落入方源的掌控之中。凤金煌想要反击,却偏偏能力不足。

那三团鬼火,就在她周围飘啊飘,有时候甚至忽然突破三尺距离,然后又忽的撤走,摆明了是戏耍挑逗凤金煌。

凤金煌到底只是少女,被方源撩拨得一肚子火气,但偏偏发作不得。

她想要除掉这三团讨厌的鬼火,就必定要冒险!

但理智告诉凤金煌,她没必要冒险啊。

因为她就算失误了两次,也领先方源一大截呢。

只要保持住领先的优势,胜利就是她的。

然而当凤金煌开始炼制四窍火楼蛊,不一会儿,方源竟然也开始炼制四窍火楼蛊了。

这个发现,顿时让凤金煌悚然一惊:“什么时候,方源他竟然拉近了和我差距?这不可能,他明明连一个炼道杀招都没有用呢!”

但事实正是如此,方源仅凭炼道手法不断轮换,利用纯粹的基本功,将差距一点点的缩小,不知不觉间就赶到了凤金煌的后头!

场下的蛊师们,也终于发现了这点。

“难以置信,方源这个魔头居然赶上来了!”灵缘斋的女弟子们惊叫。

“竟然仅凭炼蛊手法,就要战胜炼道杀招么……方源的实力果然要远超凤金煌啊。”这是高估方源的人,现在仍旧在高估。

“怎么可能呢?方源一定作弊了,谁都知道炼道杀招的厉害,他没有动用杀招,就追上来。这不可能!”孙瑶很是不满,“主持长老是什么眼光,这都不停止比试。调查方源这个魔头吗?”

身旁的秦娟师姐却是深深地叹了口气:“安静点,师妹!炼蛊手法当然比不上炼道杀招。但是在这场赌斗中,却是可以战胜杀招的。”

“啊。师姐你这话是什么意思?”

“因为取胜的关键因素,并非是手段,而是两方的蛊师本身啊。炼蛊进行到现在,已经到了中后期。方源的精神还很富足饱满,而大师姐的心神却一直全力施为,一有放松余留,方源就会动用鬼火威慑,让大师姐疲于应付。大师姐已经很累了,没有心神的操纵。再厉害的炼道杀招也只是辅助,炼蛊速度自然要慢下来。”

说到这里,秦娟叹了口气:“这一切都是方源的计策。大师姐提出炼制五窍火塔蛊,他答应的很干脆。但蛊方,炼蛊材料都是他提出来的。这场消耗战一开始,他就故意落后,保留心神。到了中期,他出手进攻,虽然落后。却掌握了比试的主动。大师姐因为念及优势,不想冒险,所以一拖再拖。却不知这正是方源的阴谋啊。现在到了后期,大师姐太累了。速度越来越慢,很快就会被方源赶超的。”

“什么?!”孙瑶担心地望去。

很快,她的担忧就变为了现实。

方源的进展慢慢和凤金煌齐平。然后又慢慢地赶超过她。

“难道,难道大师姐要输了吗?”孙瑶双眼通红。不禁泛出泪光,“明明已经这么努力。明明已经到了最后,大师姐甚至还受了伤。就这样让这个卑劣无耻的魔头赢了吗?我不甘心啊!”

“我也不甘心。”秦娟深深地叹了一口气,“不过大师姐还有机会。”

“还有机会?”孙瑶旋即扭头,惊喜地望着秦娟师姐。

秦娟点了点头:“没错,你别忘了,大师姐手中还有三次进攻的机会呢。三转攻击一次,四转攻击一次,五转攻击一次。每一次攻击都会更强,这就是翻盘的手段!这场比试还未结束,一切就都有希望。”

“是的,我们还有希望!大师姐加油!!”孙瑶双眼骤亮,忽然高声大喊。

她的喊声很快引来一片附和。

强烈的门派荣誉感,归属感,让场下数百位灵缘斋的弟子们一起高喊,为凤金煌打气。

虽然声音传不过来,但凤金煌却感受得到。

“这群家伙……我不能输,我要赢!是时候动用杀手锏了。炼道杀招墨化!”

凤金煌心中大喝一声,同时抬手就是连续三记攻击,打向方源。

凤金煌孤注一掷,不出手则已,一出手就将仅剩下的进攻机会,全部用掉!

“雕虫小技。”方源将这些攻击全部防住,同时发动三团鬼火尽量干扰。

干扰的效果并不太出色,凤金煌将这三团鬼火扑灭,手中的炼道杀招已接近完成。

“哼,曝光蛊去吧。”方源心中冷笑一声,屈指一弹,探出一只四转蛊虫。

与此同时,他又打出五转的攻势,后发先至。

凤金煌早有准备,挡住五转攻势,但曝光蛊却期近她的身边,陡然自爆。

爆炸一点都不剧烈,但光亮十分刺眼。

“怎么可能?他如何知道墨化杀招的缺陷?!”在凤金煌难以置信的神色下,墨化杀招被彻底打断。

杀招催动不成,强烈的反噬立即让凤金煌五脏六腑一抽,体内大量出血!

剧烈的痛楚让她身心剧震,再无法操纵火团。整个炼蛊彻底失败,又是一股巨大的反噬伤害,在她身上爆发。

凤金煌仰头张口,喷出一口鲜血,旋即倒在地上,当场昏死过去。

场下顿时乱成一团。

灵缘斋的弟子们想要上台,但都被蛊阵挡住。

“不得干扰秩序,违者必究!”主持长老喝道,维稳之后,他又将目光投向方源。

一时间,几乎场下全部的观众,都看向方源。

凤金煌已经昏迷,就算苏醒过来,重新炼蛊,手头上的材料也远远不够她炼制的了。

现在方源已经立于不败之地,只要他徐徐炼出五窍火塔蛊,便能收获胜利。

不过就在方源快要成功的时候,他手中的蛊虫雏形忽然发生了自爆。

方源满脸愕然:“这?!”(未完待续

\end{this_body}

