\newsection{棋子棋手}    %第三百五十一节:棋子棋手

\begin{this_body}

“不过目前来看,我开了一个好头。我转化的战意,是当中最多的。只要接下来保持住这个优势,仙蛊屋就是我的了。”

萧家太上家老暗暗为自己打气的同时,心里也清楚:如何维持当下的优势,才是最大的困难。

按照赌注,第二位就是武家蛊仙武当芷的棋子登场了。

这位武家的女仙也是很精明的人物。她宁愿牺牲一些修为,也要让选中的棋子,提前进入义天山。

“希望接下来,萧山不要让我失望。”

萧家老祖的这个愿望,恰恰是其他南疆蛊仙不愿意看到的。

如今萧家老祖已经建立了前期优势,南疆蛊仙们更希望看到萧山被击败,甚至被击杀,从而打断萧家老祖的优势,好让自己有机可乘。

武神通来到义天山附近。

“诸位,这一次我的肩头担当着家族的重任。铲除魔窟,在所不惜!”他手指着义天山,神情肃穆,战意勃发。

在他身边,有着不少蛊师。

其中有三位修为最高,皆是四转蛊师。

一位是是武家家老,贴身护卫武神通。另外两位则是依附武家的山寨族长。

武神通本人是四转巅峰的奴道蛊师,他身材单薄,脸色苍白,时不时的咳嗽几声,就好像是一个病书生,连山风吹久了都要禁受不住。

不过众人都不敢对他有丝毫的轻视,皆因他修行奴道。这个流派,向来能以一敌众。

“义天山上,最强者当属萧家的原族长萧山。在他之下,就是孙胖虎、周星星两位魔道蛊师。这三人都是五转蛊师。依我之见,不如稍缓动手,再请些正道好手压阵,如此不仅更加稳妥,而且能一战而下,不让这些魔头逃脱。”其中一位四转族长建议道。

武神通脸色一沉。他也想这么干,但此次的任务,却是家族强制他来执行,又规定了时限。如此苛刻和匆忙。让武神通都不禁怀疑,自己这一次是不是成了家族政治斗争的牺牲品。

他竭尽全力,才依靠自己的人脉,招揽了身边的这些人手。

义天山的这个真相,武神通至死都不会明白。平时高高在上的蛊师大人们。都被蛊仙们充当赌博的工具。他当然更不会知道,此时此刻,不知道多少的南疆蛊仙,都将期待的目光投注在他的身上。

“尔等不必相劝。只要诸位护住我,就算他们有五转战力,于兽群当中又能支撑几时?一旦他们真元耗尽,斩杀五转的功绩,就要落到诸位的身上了。况且我既然亲自出手,自然是有把握,难道我会自找死路不成吗?”

武神通也是有手腕的人。短短一句话,就打消了众人的迟疑,振奋了士气。

兵贵神速,打的就是突袭效果。

片刻后,漫山遍野的兽群,就在武神通的指挥下,直接冲上义天山。

此时的义天寨,还在建设当中。

魔道蛊师们一片慌乱,他们单打独斗惯了,尽管萧山等人竭力组织。短时间内也没有见到什么成效。

萧山心急如焚,暗想:“义天寨刚刚建立,还差一小半,方能竣工。这是我正魔第一次交锋。义天寨就如同一面旗帜,不能倒下!一旦倒下,我方士气必定衰落,就好像是挨了当头一棒,抬不起头来。丧失了威风气势,今后谁还能来主动投靠我?”

想到这里。萧山连忙下令,命众魔道蛊师死守义天寨。

萧山的思量,的确很有见地,但他大大高估了魔道蛊师的配合能力。

魔道蛊师若是和正道单打独斗,往往胜多败少。但若是人数多了,相同的人数对决,通常是正道胜利居多。

没有配合,魔道蛊师就是乌合之众。

面对扑来的庞大兽群,他们又不据险防守,立即落入到武神通最想看到的局面。

起先,大量的野兽惨死在魔道蛊师们的攻势之下。

但很快,魔道蛊师们铺天盖地的攻势狂潮,就衰落稀疏下来。毕竟凡人蛊师的真元,是相当有限的。

野兽接连冲破火力的封锁,扑杀魔道蛊师。

魔道蛊师的伤亡越来越大,局势向正道一方迅速倾斜。

“打得好。”

“妙啊……”

南疆蛊仙们遥视战场,目光中都带着欢喜之色。

萧家老祖面沉如水,死死盯着萧山。

萧山也是久经沙场之辈,他心知绝不能让着局势这般持续下去,连忙大吼:“五转、四转的好汉们,都随我,杀下山去,斩了奴道蛊师!其余人等,且战且退。”

此时情形,魔道一方可谓损失惨重。正道一方,虽然蛊师稀少,却未损失一人。

谁都知道单独一个人,绝不会讨得了好。唯有相互依赖,方能杀出一条血路,还有逃生的可能。

萧山的话,很快得到众人的响应。

不同的是,四转、五转的蛊师们,都暗暗带着欢喜,集结在萧山的身边。而留下来的三转、二转的蛊师们,则各个脸色苍白。

突围的人,实力强劲,就算再不济,突围到外围去,也可独自逃窜。

但留在山中的蛊师,却被兽群重重包围,上天无路入地无门,只有等到救援一途。

唯有方源一人,虽然表面惶急凶狠,但心中却是一片淡定。

就算是兽群再扩张百倍,也对方源构成不了什么威胁。况且他是重生之人,知道事情的发展是怎样的。

果然接下来的发展,和上一世差不多一样。

这一场突击战,魔道蛊师们逆着兽群前行,很多蛊师在途中惨烈战死。

但是最终,众人还是杀到了武神通的面前。

萧山、孙胖虎、周星星三位五转蛊师,已经被逼上绝境,强攻正道一方的防线。

一番惨烈的生死搏杀之后,正道阵亡了两位四转蛊师,武神通重伤败退。

武家四转家老拼死拦截,关键时刻,又有一群飞鸟及时增援,魔道一方想要直接杀掉武神通的谋划,没有得逞。

只能无奈地撤回到义天山上去。

当夜,萧山在某个山洞中,召集了魔道剩下的残兵败将。

他浑身浴血,双眼通红,声音嘶哑地喊道:“武神通未死,实乃心腹之患。只要他在一日,我们就要面对兽群悍不畏死的冲锋。我们必须杀了他,否则义天寨就永远建不起来。”

萧山说完这番话,却是应者寥寥。

魔道蛊师新败,士气低沉得很。

其中一位三转蛊师,垂头丧气地道:“大头领,我们还是撤吧。正道势大,我们打不过,也很自然。留得青山在不愁没柴烧啊。咱们先离开这处险地,日后换了另外一座山,重新再建义天寨,也是可以的。”

此话刚刚说完,萧山眼中厉芒激射,身形暴起,陡然出手。

他手起刀落,将说话的魔道蛊师当场杀了,口中厉喝:“此人动摇军心,死不足惜!诸位谁敢言退,就是他这番下场!”

孙胖虎、周星星立即站起身来,走到萧山的两旁,对众人虎视眈眈。

众魔道蛊师被萧山气势所摄,纷纷开口,愿意死战。

萧山脸色稍缓:“我也知,诸位不易,身上都有不轻的伤势。但伤势再重,总比今天牺牲的同道兄弟们要好吧?今晚诸位都在这洞中休息,明日一早,我们就集齐所有人的力量,杀下山去,不杀了武神通,绝不罢休!”

众人连忙应是,方源也夹杂其中,他身上的伤口还在流血,当然这只是伪装,不足挂齿。

夜色渐渐浓重,山洞不大,魔道蛊师们睡觉的空间,也不宽裕。

这是萧山故意选择的山洞,方便大家相互监视。就算是有人要大小便,也都必须在山洞内部解决。

很快,山洞中就充斥着血腥气,汗臭味,还有大小便的腥臊之气。

魔道蛊仙们辗转反侧,心中挂念着明日凶多吉少的大战,更加睡不着。

唯有一人呼呼大睡,正是方源。

他打呼噜的声音,萦绕着整个山洞。

萧山闭眼假寐,听到这个声音,缓缓睁开双眼,看到方源之后,轻轻一笑,大声地道:“这个没心肝的夯货。”

他的声音,吸引了众人的目光。

萧山又道:“诸位放心,明日之战,我有十足的把握!那武神通已经被我等重伤,明日必死无疑。我萧山发誓,绝不临阵脱逃,违背誓言,天诛地灭,人神共愤!”

众魔头心气一振,都佩服萧山的壮烈情怀。

却不知道,萧山一心想要收服体内的仙蛊,得到萧家老祖的认可。不到山穷水尽之时,他是万万不会撤离这里的。

萧山将洞中的这些魔道蛊师,当做自己的棋子。

而他本身,则是萧家老祖的棋子,他却不自知。

在这个夜晚,身为棋手的萧家老祖同样焦躁担心,忧愁不安。

今天义天山一战,得出战果之后,他就离开自己的住处,来到某个山巅。在山风中,站了许久了。

“师尊,萧家老祖求见,你却闭门不见。他可是七转蛊仙,已经在洞外站了一个多时辰了。再拖下去,恐怕不好吧?”陆钻风小心翼翼地觐言道。(未完待续。)

\end{this_body}

