\newsection{威凌天地}    %第三百零八节:威凌天地

\begin{this_body}

%1
中洲,蜈蚣峡谷。

%2
一场惊心动魄的追逐,正在峡谷中狭窄的山路中上演。

%3
“快跑,快跑啊!”一个小人,趴在洪易的脑袋上,焦急的大叫。

%4
“疼、疼、疼!你别揪我的头发呀。”洪易疼的大叫。

%5
小人却置若罔闻,一双小手下意识地将洪易的头发紧紧攥住,转过头往后一看。

%6
“妈呀!它已经追上来了,你快点啊,再不快点,我们都要被它吃了呀。”小人惊惶大叫。

%7
“我也想快啊……”洪易咬牙切齿,他已经竭尽全力催动移动蛊虫了。可惜他的修为太低微,只有二转高阶。

%8
当然,这个修为放在洪易这个年龄,已经十分罕见。

%9
自从炼蛊大会之后,洪易也屡获一些小巧机缘,所以修行速度远超常人。

%10
但就现在这个生死危机而言,二转高阶的修为,面对身后追杀而来的五转野生金角蜈蚣,却是不够看的。

%11
这头蜈蚣,体型庞大,好似巨蟒。头生有角,坚硬锋锐。它行走在山道上,上百对的蜈足轮番踩踏,扭动身躯,行进速度非常的快。

%12
“完蛋了!它已经跟到你屁股后面了。”小人吓得惨无人色,几乎魂飞魄散。

%13
“拼了!!”洪易也感觉到,金角蜈蚣近在咫尺,他惊的寒毛炸立,无奈之下,只有催动唯一一个杀招。

%14
这个杀招,恰好是移动杀招。

%15
可是洪易得到手的时间,还不久。演练次数的还不多。

%16
催动的时候,失败和成功的概率也是对半分。

%17
杀招。必须至少两个蛊虫才能组合形成。蛊虫越多,运行步骤越复杂。催动杀招的难度就越高,但往往威能也越强。

%18
洪易得手的这个移动杀招,就是由近十只凡蛊组成的。

%19
不要拿蛊师和蛊仙相比,这个数量对于蛊师而言,已经是相当的多。

%20
杀招并不是那么容易训练的。

%21
有时候催动杀招失败,还会损伤自己或者蛊虫。

%22
因此,平时的时候,洪易每一次练习这个杀招,都是很小心翼翼的。

%23
但这个时候。生死存亡的危急关头,他已经全然顾不得了。

%24
按照原来的速度,根本摆脱不了身后的金角蜈蚣,他绝对是死。

%25
眼前唯一的出路,也就只有去拼一下!

%26
“给我催动起来,一定要成功啊!”洪易在心头呐喊。

%27
但愿望是美好的,事实却是残酷的。

%28
催动杀招需要专心致志,一些杀招,甚至需要特别安稳的环境。不能受到任何干扰。

%29
方源使用杀招,很容易就上手,那是因为他有前世五百年的经验打底。而洪易却绝对是新手。

%30
这个时候,他身处险境。一方面要看前面的路,山路上多是石头,坑坑洼洼。他疾速奔跑,万一跌倒下来。那就全完了。

%31
另一方面,洪易背后金角蜈蚣。紧随其后。那巨大的动静,始终萦绕耳畔,死亡的气息几乎要笼罩全身,洪易能不去在意吗?

%32
这种情况下,还能够不在意的,保持心中的绝对冷静,都是久经沙场,置生死于不顾的老将。洪易以后能达到这个境界,但现在的他还太年轻。

%33
“追上来了,追上来了!”头顶上的小人在大叫,惊恐地看着恐怖的金角蜈蚣,一路奔腾,气势汹汹地渐渐拉近距离。

%34
蜈蚣体型庞大厚重,数百对足肢踏地,虎虎生风,狰狞的口器大张,涎水四溅,照准洪易的后背,一口咬下去!

%35
小人啊的一声惨叫,吓得紧闭双眼。

%36
他虽然背生双翼,但都在此前折损,根本飞不起来。

%37
这一次进入山洞冒险,他成功地盗取五转蛊材百花凝露。但这样做的代价还有一个,就是被看守凝露的金角蜈蚣深深仇恨。因此小人根本逃不出去,只能和洪易在一起,借助他的速度苟延残喘。

%38
然而,小人意料中的疼痛并没有到来。

%39
他睁开双眼,露出不可思议的神情。

%40
不知道为何,洪易居然逃脱了金角蜈蚣的撕咬。

%41
但下一刻,金角蜈蚣又追了上来。

%42
小人心惊肉跳,却渐渐看明白。

%43
原来,金角蜈蚣每一次要咬洪易,下意识的都会有个昂首的动作。

%44
这个动作,让它的脑袋离地至少八尺的高度,连带着半丈的身躯,也都离开地面。然后再落下来。

%45
当蜈蚣昂首的时候,它的许多足肢都离地而起,速度骤降。

%46
而这个时候,洪易却仍旧在奔跑,速度还不变。

%47
因此,金角蜈蚣的每一次吞咬,都被洪易摆脱。

%48
“这个笨蛋!”小人哈哈大笑,心中全是劫后余生的狂喜。

%49
人才是万物之灵,野生蛊虫智慧极其有限,一举一动都遵循本能。

%50
“我好心带你跑路,你居然还骂我笨蛋!”洪易却不满意了。

%51
“没骂你,我说的是这头金角蜈蚣,你这个笨蛋!”小人旋即大叫。

%52
洪易心中却很有怨气:“你才是笨蛋。都说了蜈蚣沉睡着呢,不要害怕,偷了百花凝露就走,你偏偏要嚎一嗓子!”

%53
小人脸颊鼓起,眼中的愧疚一闪即逝,脸上露出羞恼的神色。

%54
他张口想要反驳,但下一刻他鼓瞪双眼,满脸惊骇。

%55
原来,背后的金角蜈蚣见屡次都咬不中洪易,就换了另一种攻击方式,那就是它头上的金角。

%56
这当然不是它变得聪明了。

%57
而是在漫长的岁月里,金角蜈蚣总会遇到一些体型庞大,一口吞之不下的猎物。这种情况下,它一般都会采用额头的金角,将猎物分割切碎。

%58
张口吞噬是金角蜈蚣的本能。顶头上的金角也是本能。

%59
金角蜈蚣恶狠狠地扑来,头低下。金角前突,很快就接近了洪易的……呃。屁股。

%60
没办法,就是这个高度。

%61
“笨蛋,快跑啊!”小人目睹了这个险情,吓得哆哆嗦嗦,浑身颤抖,手中的头发都有点抓不稳了。

%62
“你还骂我笨蛋哦!”洪易大叫着,忽然声调一扬,就好像是公鸡,忽然被抓住了嗓子。那个惊慌失措。那个猝不及防,当然,还有痛彻心扉。

%63
金角已经刺入洪易的屁股当中。

%64
小人心中一片绝望,心想这次绝逼完蛋了啊!

%65
但就在这时,洪易的速度居然猛地提高了一大截。

%66
他一下子就窜了出去!

%67
原来,在他惨叫的时候,剧痛袭来,却让他福至心灵,一下子就将杀招完整地使了出来。

%68
洪易因此脱离了险境。

%69
“流血了。流血了。”小人看着洪易背后,惶急大叫。

%70
原来蜈蚣的金角,是本来插在洪易的屁股上的。但现在洪易哧溜一下,窜了出去。双方分离开来。

%71
就好像是一柄匕首从洪易的屁股上拔了出来,伤口没了阻碍,自然向外流血了。

%72
“屁股好疼!哎哟哟。疼死我了!!”这一刻,洪易的脑海中充斥着类似的强烈念头。

%73
他下意识地捂住屁股。捂住伤口。

%74
然后,自然而然的。他因为伤口的疼痛,导致心神分散。

%75
刚刚催起来的移动杀招崩解,速度又骤降下来。

%76
可他背后的金角蜈蚣,还在追赶。

%77
又一记金角,刺中洪易的另一半屁股。

%78
“哦!”洪易又一声惊嚎。

%79
然后,相同的一幕又发生了。

%80
惨烈的攻击,突然起来的剧痛,让洪易注意力在一刹那间高度集中,再次使出了杀招。

%81
他又奔跑出去。

%82
“喷血了,喷血了!”小人急得大叫。

%83
没办法,两个伤口都很深,加上洪易又在剧烈运动,血不喷涌才怪呢!

%84
洪易捂住屁股飞奔,背后金角蜈蚣紧追不舍,如此情形让他骑虎难下。

%85
“糟糕,真元不足了!”忽然间,洪易的脸色惨白一片。

%86
蛊师的真元本来就稀少,很不耐用。而杀招同时催动许多蛊虫,消耗又大。奔跑这么长时间,洪易已经到达了极限。

%87
死亡来临,洪易被逼上了绝路,没有半分生机。

%88
“我们要死了吗?我们要死了吗?蜈蚣大爷,别吃我啊,我身子这么小,肉也少,你吃他,你吃这个笨蛋。”小人吓得瘫软在洪易的头发中,口中念念叨叨。

%89
这个时候,洪易心中也空空荡荡,完全已经没有闲情向小人计较。

%90
然后,就在这时,一道惊天动地的剑光射来!

%91
一刹那间,天地骤白,万物失音!

%92
异变没有让洪易停下奔跑,他又跑一段后,终于意识到不妥。

%93
回头一望,他惊呆了。

%94
五转金角蜈蚣蛊,已经没了,彻底消失不见了。

%95
和它一起消失的,还有大半个蜈蚣峡谷!

%96
呈现在眼前的是,一条巨大的深沟。深沟边缘,光滑平坦,像是镜面一样。

%97
“这,这是怎么回事?”洪易手足无措,难以置信。

%98
“一道,一道剑光……”小人已被吓傻。

%99
中洲东海岸。

%100
水浪波涛,翻腾不休。寒气四溢,笼罩八方。

%101
“孩子们,我们到了。这就是玄冰岛,飞霜阁的门派驻地,也是你们今后生活,出人投地,改变命运的地方。”领头蛊师手指着远处的冰岛,满怀骄傲地介绍道。

%102
少年们纷纷仰起小脸,向往地看过去。

%103
有的双眼熠熠生辉,有的神色激动无比。

%104
飞霜阁,虽然没有蛊仙坐镇,不是超级势力。但是在这方圆数千里范围,却是土霸王,唯一能和其媲美的,只有五德门。

%105
前不久,中洲炼蛊大会的第二场比试,飞霜阁就是其中之一的举办点。

%106
这场举报,也让飞霜阁的威名更加远播四方八面。

%107
“这一次,带来的好苗子,可能是十年来资质最高的一批了。这都是门派的未来啊,好好培养,一定能壮大门派。”领头蛊师看着这些孩子,内心也很澎湃。

%108
领头蛊师开口,高声地道:“孩子们,你们能成为飞霜阁的弟子,是你们的幸运。现在我来给你们讲一讲……”

%109
刷!

%110
就在这时,剑光闪过,让众人眼前一花,下意识地闭眼。

%111
等到他们睁开双眼时,所有都呆立住,嘴巴张得老大,一动不动,一个个仿佛表情夸张的石像。

%112
壮阔的玄冰岛已经不翼而飞,一道修长巨大的沟壑,横霸在视野当中。

%113
就连海水都被劈开!

%114
海面上一道五六里路的中空地带,残留的剑道道痕横霸于此,隔绝了水道道痕,以至于两边的海水都无法倒灌进去。

%115
一时间,形成一幅诡异霸绝的奇景!

%116
天庭。

%117
监天塔主手拄着拐杖,颤颤巍巍地站在监天塔下。

%118
他另一手则拿捏着宿命蛊,枯槁的老手慢慢地抚摸着宿命蛊。

%119
这一次,修复宿命蛊大获成功,监天塔主的心中满是激动和喜悦。

%120
和他一道修复的白沧水、炼九生,碧晨天三人,都离开了天庭,下凡去了。

%121
天庭中的蛊仙,几乎都是采用的沉眠延寿法。好不容易醒来一次,自然要争分夺秒,下去各自门派,处理一些私事,照看一下门派或者血脉后裔。

%122
天庭管辖着中洲十大古派,但同时天庭的蛊仙,也大多从中洲十大古派中抽选而来。

%123
若用门派的结构来看,天庭就相当于上宗,而十大古派就相当于下宗。

%124
不是所有的蛊仙,都有机会进入天庭。

%125
除了流派造诣深厚,战力强大,修为至少要达到八转之外,要进入天庭,还有一个重要的标准。

%126
那就是价值观!

%127
天庭。

%128
何谓天庭?

%129
星宿仙尊在三百万年前,就已经明确地阐述过,那就是

%130
顺应天意,替天行道!

%131
“这一次修复,宿命蛊能有五成威能。再用监天塔,就能发现更多的逃脱宿命之人。将这些人铲除,就更能帮助宿命蛊复原。一切都将踏入良性循环,这样下去,天庭将重振往昔的辉煌!”

%132
监天塔主心潮澎湃,正要举足,再入监天塔。

%133
轰隆!

%134
一道剑光,霹雳一般,飞射而来。然后以迅雷不及掩耳之势,划过监天塔。

%135
监天塔主张大嘴巴,瞳孔缩成针尖大小。

%136
他浑身僵硬无比,亲眼目睹着监天塔的上半部分的一小截,慢慢倾斜,然后轰隆倒地,砸在白玉砖石之上。

%137
“监天塔!!”愣了一愣后,监天塔主惊吼出声。

%138
ps:这章4000字,今天两更,其中是补上昨天欠下来的800月票加更。刚刚看来一眼月票,突破900,所以明天还是两更。感谢大家的支持!这一大卷也正式进入了末尾,**将临!

\end{this_body}

