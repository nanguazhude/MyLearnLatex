\newsection{故意被发现}    %第二百四十六节:故意被发现

\begin{this_body}

秘密筹谋了小半个月后,方源来到一座小岛之上。[看本书最新章节请到

东海多岛,方源踏足的这种小岛,在东海有数百万,乃至上千万。

绝大多数的小岛,都是荒岛。不过这座小岛却有着人烟。

岛上有三处渔村,最高修为者,只有三转。

方源隐去身形,顶着烈日,若无其事地在其中一处渔村里闲逛。

周围的人,绝大多数都是凡夫俗子。偶尔有一两位蛊师,一转或者二转的,都会引来凡人们敬畏的目光。

方源此事掌握的手段,已经和南疆那会儿,有天差地别的提升。

他在村中逛了两三圈,周围的人都无法发现他。

“这里应该就是李家村了。”方源观察良久,又结合前世记忆,认定了这处地方。

现在这个小渔村,当然还不是李家村。

但是前世五百年,这里出了一位传奇色彩极为浓重的蛊仙,名为李逍遥。

这处小渔村,便他而改名,被称之为李家村。

当然,现在李逍遥还未出世。按照前世,李逍遥的诞生还要往后两百年开外。

方源确认了这个小渔村后,便走出村去,来到海边悬崖。

悬崖上洞窟无数,有成百上千的海鸟栖息着,这些具有很强的领土意识。一旦有蛊师,或者大型生物进入这里,让海鸟们感到威胁,就会引发群鸟的强袭。

因此这里,就算是村中的人,很少踏足这里。偶尔有蛊师进入,采摘一些草药。或者偷一些鸟蛋。

这些鸟群成分复杂,相当于十多个百兽群的集合,并非是同一种鸟。

在这个小小的海岛上,拥有蛊师的人族,并非出于霸主地位。

“按照《李逍遥传》的记载。童年时期的李逍遥,常常被大龄的孩童欺侮,不得不经常躲到这里来。他在这里,意外地发现了一个隐秘的小洞窟,他将这处小洞窟当做他小小的秘密基地。”

这悬崖上,凹坑密布。洞窟大大小小,数不胜数。

要寻找李逍遥的那个洞窟,自然颇有难度。

“不过,既然有鸟群霸占这片地方,李逍遥不可能深入内部。他的洞窟应该就在外围。”

方源准备充足。搜寻了片刻,初步筛选出上百个目标。

等到正午的烈日,在西边徐徐落下,方源终于找到了真正的目标。

这个洞窟并不大,依照方源此时星象子的身形,还进入不了。更不要说方源原形八臂仙僵的身躯了。

好在有变形仙蛊。

方源又将身形缩小,钻了进去。

洞窟湿润,傍晚时分。透着些微余晖。在狭小的洞中行进了一小段后,方源来到这处洞窟的最深处。这里的空间,倒是颇为宽阔。

悬崖外的海风。不断地顺着洞壁周围的小洞口,灌入进来,发出呜呜呜的风声。

方源不由地想到《李逍遥传》中的记载,里面内容述说李逍遥童年时的的困顿,他时常被欺负,躲到这个洞窟里。一边小声的抽泣,一边听着这呜咽的风声。

这种情形。一直持续了很久。直到有一天,命运忽然转过了一个角。李逍遥的传奇便因此开始。

这个命运的转角,就是洞壁角落的一处小洞口。

李逍遥一次躲进来,靠着角落休息。结果小洞忽然坍塌,李逍遥一路跌落,直接落入悬崖内部深处,然后被里面的海流卷走。

方源此行,就是要来复制前世李逍遥的奇遇。

这处悬崖,宛若蚁穴,内部像是被钻透了一般,充斥各种各样的洞窟和密密麻麻的通道。

三个时辰之后,方源探知到正确的路径。

他打破这边的洞口,整个人钻了进去。

他进入悬崖内部,迅速落下。

里面一片黑暗、潮湿,生存着蛇蛙等等小型生物,最大体积的是一种壁虎,有猫一样的体积。

这些都不是方源的阻碍。

他很快到了悬崖底部,轰隆隆的水波声响,越来越大。

在悬崖底部,竟然有一片巨大的海水漩涡。

这是东海特有的潜流现象。

漩涡中的海水,飞速旋转,水汽充盈,好像是怪物张开巨口。

方源心念一动,飞出几只蛊虫,停留在这里,作为记号。

然后他一头扎进隆隆作响的漩涡当中。

一进入海水漩涡里,一股无形的巨力,就包裹着方源。饶是方源力气十足,此刻被海水夹裹着,也有些身不由己之感。

方源并不挣扎,任由这股海底潜流,将自己送往他处。

这股潜流,宛若无形的龙蟒,横贯海底。海底深处的鱼群,都会远离潜流。就算是海草海树,也不在潜流的位置生长。

因为根本生长不了,经受不住海底潜流的长期冲刷。

方源顺着这股潜流,急速前行。他此时的速度极快,甚至比一些常规的仙道移动杀招,都要快一些!

这就是海底潜流的特性。

在东海,有一种珍贵的海图。

海图的内容,不是海岛,也不是海平面,而是海底的潜流。

一股股海底潜流,纵横盘踞在海底,是纯天然的交通资源。

在东海,一些超级势力,都有深海商队。这些商队利用蛊虫,钻入海底,再借助一股股的潜流,迅速移动,在很短的时间内,跨越数万里,乃至数十万里,进行买卖。

约莫一炷香的时间,方源从这股海底潜流的一端起点,到了另一端终点。

终点处,和起点相似,仍旧是一处巨大的海水漩涡。

方源从漩涡中蹦出来,入眼的是一片火红。

炙热的空气。扑面而来。

这是一座海底火山。

方源此时,身处在这座海底火山的半山腰。

火山中空,内部并无海水灌入。

在火山底部,是炽热浓稠的岩浆,从海沟深处涌出来。大量的生命。栖息于此。

而在火山顶部,则是天然的海底温泉。

方源此行的目的地,就是火山顶部的海底温泉。

不过方源估算了一下,时间还不到,他便决定先向下探测一番,打磨时光。

越往下。越是炙热。

眼前一片赤红,一种火蜥蜴进入方源的侦察范围。

这种蜥蜴,满身光滑的鳞甲,舌头伸长弹射,捕捉飞虫当做食物。

这种飞虫。竟然来源于火山底部,缓缓流动着的岩浆河。

岩浆河中,一个个的气泡鼓起、爆裂,从中飞出一只只的小虫子。

火蜥蜴就趴在岩浆河的两侧,见到小虫子飞出来,舌头就如闪电般射出,又快又准地将这些小虫子捕捉到,然后在下一秒内舌头带着食物。缩回口腔。

这种独特的生态,让方源驻足看了好一会儿。

大自然的玄妙神奇,此时展现在方源的眼前。

方源观察了一会火蜥蜴之后。便将注意力集中在飞虫上。

这种飞虫相当奇特,从岩浆河里孕育而出,能够忍耐极高的温度。

观察久了,方源又有新发现:火蜥蜴捕食时并非十拿九稳,有时候还会连续碰到失败。

一部分的飞虫,似乎是在火蜥蜴带来的生死存亡的压力下。激发了某种玄妙的潜质。

这些飞虫,在被捕食的极短过程中。陡然变成蛊虫。速度激增,躲过火蜥蜴的捕猎。

这是一种进化。

就好像是鲤鱼跳龙门。从鱼变成了龙,生命的本质得到了升华。

这些飞虫进化成蛊虫之后,又有一部分主动落到火蜥蜴的身上,成为它们的寄生蛊,依托着火蜥蜴生存。

不过这些都是火道凡蛊,对于方源而言,没有多少价值。

现在能入方源眼界的,是这座海底火山。拥有它,就就一道生产线,可以长期,并且大量地生产火道、水道蛊虫。

凡道蛊虫只有数量上去了,对于蛊仙而言,才有价值。

最后,方源观察了一下这个岩浆河。

这道岩浆河里还有生物,是一种鱼类。

方源不打算深究,也不想自找麻烦地跳进熔岩河里去畅游一番。

岩浆河的温度很高,就算是方源要进入这里,也得催动仙道杀招进行防御。也就意味着,需要耗费珍贵的仙元。

自然是伟大的,有很多的地方环境恶劣,就算是蛊仙也无法长期涉足。

方源一直估算着时间,感觉差不多了,他便从火山底一路向上,过了半山腰后,方源渐渐逼近火山顶部。

到了这里,他开始小心翼翼,主动隐藏起形迹。

年轻貌美的宋亦诗,是东海蛊仙中的六大美人之一。

今天,宋亦诗的心情不太好。

她又被纠缠了。

这些来自各处的男性蛊仙,总是像是烦人的苍蝇,围绕在宋亦诗周围,嗡嗡嗡的吵个不停,赶都赶不走。

好不容易摆脱了这些追求者后,宋亦诗来到自己的私人海域。

这片海域,名为诗情海,盛产一种四转智道蛊虫诗情蛊。这种蛊虫可以用于存储情感,在凡人蛊师眼里,分外珍稀可贵。

宋亦诗深入海底,来到自己的海底火山行宫。

这座海底火山,位于诗情海中,自然也是宋亦诗的私人领土了。

宋亦诗在火山顶端,建造了一处精致华美的小型宫殿。她十分喜爱宫殿里的温泉,每隔一段时间,都会来这里泡澡。

她褪去衣衫,露出精美绝伦的白皙娇躯。她的肌肤白嫩得如同刚剥了壳的鸡蛋,身躯娇柔,红唇靓眼,长长的头发披散下来,能及腰际。

她进入温泉里,闭眼享受。

很快,她的心情好转起来,开始轻轻地哼起歌来。

她却不知道,方源早已经潜藏在附近,观察着她的一举一动。

“差不多了。”方源目光幽幽,默默算计,故意露出一丝气息。

“什么人?!”宋亦诗花容失色,低喝一声,双目如电,向方源藏身之处怒视而来。(未完待续)

\end{this_body}

