\newsection{双极海峡宋甲丹}    %第二百五十一节:双极海峡宋甲丹

\begin{this_body}

------------

东海,宋家大本营,重水福地。(www.QiuShu.cc 求、书=‘网’小‘说’)

“爷爷,你可要为人家出气!”宋亦诗风风火火地冲进书房,对八转蛊仙宋启元跺脚撒娇。

宋启元和一干七转蛊仙长老们,正在商量着家族要务。

但宋亦诗就这么直接闯进来。

宋启元缓缓地转过头来,脸上毫无生气的意思,反而满是慈祥的笑容:“我的乖孙女,要爷爷怎么为你出气啊?”

宋亦诗并不奇怪,自己的爷爷知道这件事情。事实上,如果宋启元不知道整件事情,那才奇怪。

这位年轻的女蛊仙在来之前,早已经想好了答案,她咬牙切齿地道:“爷爷你直接出手,把他活捉过来,然后让我处理。”

同时心里补充:“我要先将这老贼的眼珠子挖下来,然后将他抽经扒皮,剔骨腐肉,让他痛不欲生,让他懊悔自己做过的一切!”

宋启元点点头,没有丝毫的犹豫,答应下来:“好方法。不过……”

声音拖长,宋启元话锋轻轻一转,脸上露出为难之色:“爷爷现在手头上,有点事情。这段时间脱不开身啊,不如我让你宋夏麒叔叔帮你可好?”

宋亦诗眼前一亮,忘了自己的荣辱,联想起了什么,不由兴奋地\%道:“爷爷,是有关登天野的事情吗?”

宋启元也不瞒她,坦言承认道:“不错!登天野的具体位置,几乎已经成了东海蛊仙界公开的秘密。在登天野里,埋藏着天难老怪的传承。这一次爷爷要亲自率领家族蛊仙。和蔡家、若来家抗衡,尽力将天难老怪的传承夺到手。”

天难老怪。乃是远古时代的炼道大宗师,历史上和空绝老仙。长毛老祖二人齐名。

根据历史记载,天难老怪性情怪癖,因为妄图炼化太古九天,失败身陨。

宋亦诗虽然被宠溺,但识得大体,忙道:“爷爷,你尽管去吧,把那两家打得落花流水就是!人家的事情,你不用操心。不过只有宋夏麒叔叔还不够。我还要宋甲丹叔叔出手一次,为我测算推演一下那老贼子的方位。”

“哈哈,乖孙女。好,你拿着这个手令,去往双极海峡,找你的宋甲丹叔叔去罢。”说着,宋启元抛出一块令牌。

宋亦诗连忙接过,带着欢喜的神色,转身就走。一个招呼也未打。

双极海峡,常年笼罩在一层浓郁的迷雾当中。

半空中,宋亦诗悬停着,面对眼前的迷雾。txt下载80txt.com却是有些犯难。不由问向身旁一位中年蛊仙:“夏麒叔叔,我怎么进去啊?”

宋夏麒一身精悍之气,闻言淡笑:“老爷子不是交给你一块令牌吗?小亦诗。你只需要亮出令牌,你的甲丹叔叔自有感应。”

宋亦诗便高举手中的令牌。

不一会儿。眼前的迷雾开始缓缓旋转,显现出一个狭小的通道。直通双极海峡内部。

宋亦诗、宋夏麒二人便顺着通道,踏足双极海峡,见到了宋甲丹。

宋甲丹乃是宋家专门培养出的智道蛊仙。

他常年盘坐在双极海峡的悬崖之上,上半身已经衰老不堪,下半身则已经石化,和脚下的双极海峡似乎连为一体。

“甲丹叔!”宋亦诗叫道。

从出生起,她见过蛊仙宋甲丹只有数面,但并不妨碍她对宋甲丹的崇拜敬仰之情。

宋甲丹为了家族,甘愿牺牲自己的自由。自成为智道蛊仙起,就盘坐在这里,借助特殊的智道传承,与天地交融。因而宋甲丹的推算谋划之功,十分了得。每每宋家有重大行动,都会请他出手,为行动占卜推算。

在当今的东海蛊仙界中,他是公认的最出色的三位智道蛊仙之一。

有诗言证:双极盘甲丹,南宫藏华安,还有龙首龟,厄海中往还。

宋甲丹已经睁开双眼,面无表情地看着宋亦诗、宋夏麒二人。

他很想微笑,但脸面早已如石雕,肌肉僵硬得面无表情,就算是张开口,发声也极为艰难缓慢:“小诗,夏麒表弟,好久不见。”

“甲丹叔,我这次来,是要请你推算一个敌人的具体位置。喏,这是他的影像,还有我收集到的气息。”宋亦诗主动递过来两只蛊虫。

“先让我看看令牌。”宋甲丹缓缓地道。他虽然认识宋亦诗、宋夏麒,但铁面无私,要其推算,必须先有宋启元的令牌。

令牌确认无误之后,宋甲丹缓缓地张开双唇,发出一缕吸力,将宋亦诗手中的两只蛊虫都吸入他的口中。

他缓缓地闭上双眼,进行推算。

良久,他睁开双眼,眼中流露出疑惑之情:“我算不出来。”

“什么?”一时间,不管是满怀期待的宋夏麒,还是心中笃定的宋亦诗,都惊讶无比。

“叔叔,你是当今东海最强的三位智道蛊仙之一,怎么可能算不出来?”宋亦诗叫道。

“小诗,稍安勿躁。”宋夏麒脸色严肃,“甲丹表兄,对方只是一个来自北原的六转散修,依你的智道造诣,也算不出来吗?”

宋甲丹仍旧是面无表情,声音低沉缓慢:“我算不出来有什么奇怪?我不过七转修为,虽然有些虚名,但那都是好事者的无聊恭维。至始至终,我都不过是伟大天地中的小虫,只是辨识一些风雨而来的先兆。和天地世界相比,我卑渺如蚁,和世间豪雄英杰相比,我偏安一隅。”

宋夏麒苦笑:“表兄,你还是如此谦逊。”

宋甲丹:“不是我谦逊,而是我无时无刻地融入天地,每当我在这条智道的路上前进一步,我就能越发看到更玄妙的美景。天地的奥秘我知道的越多,就越明白我自己的渺小。所以这不是谦逊。而是实事求是,自知之明。”

宋亦诗不甘地道:“甲丹叔你就算算不出来。那你也是最聪明的人,你仍旧得给我一些好建议。”

宋甲丹沉默了片刻。这才道:“我算不出来,大抵上有两个原因。第一个原因,是对方不在东海。第二个原因,对方有智道手段,遮蔽我的推算。或者有其他的智道蛊仙帮助他,抵消了我的推算。”

宋亦诗和宋夏麒面面相觑。

这两个原因,都有些匪夷所思。

第一个原因,星象子这老贼不在东海,又能在哪里呢?如今地潮已经结束。难不成他还能回到北原去?

第二个原因,星象子怎么会有抗衡宋甲丹的智道造诣?若有人帮助他,那这个人至少和宋甲丹一个级别?难道是南宫家的蛊仙华安,或者是厄海中的龙首龟仙人出的手?

“甲丹叔,你再算一次啊。也许你没算准呢!”宋亦诗瘪嘴要求道。

宋甲丹直接拒绝:“不行。家族有重大行动,此次要争夺天难老怪传承。我要养精蓄锐,保留手段,几天后为这次行动推算。”

“原来甲丹叔你没有尽力啊!”宋亦诗听出话外之音,不满地道。

“好了。你们可以出去了。”宋甲丹缓缓地闭上了双眼。

宋亦诗被晾在一旁,她咬了咬牙,恨恨地一跺脚,无奈地转身离开。

抢占登天野的霸权。争夺天难老怪的传承,乃是家族发展的百年大计。而追捕星象子,只是自己的私事小事。

宋亦诗虽受宠溺。但也顾全大局,心系家族。绝非不知轻重之人。

中洲,狐仙福地。地下石窟。

方源沐浴在智慧光晕当中,双目紧闭,脑海中星念争相闪耀,默默推算着某些重要的事物。

上一次,他就是借助智慧光晕,连续推算出了仙道杀招星火遁、外映星念。

因此,方源现在掌握的星道仙级杀招已经多达六种。

星云磨盘、星蛇索、六幻星身、位星移、星火遁、外映星念。

其中星火遁,是移动杀招,在宋亦诗的相关行动中,大放光彩。而外映星念,则是方源勘破仙僵沙南江行迹的仙道侦察杀招。

这两个杀招的前身,都只是残招。是万象星君的星道传承中的先贤们,灵光一现记录下来的杀招设想,或者推算过程中遭遇失败,无法再继续进行完善的杀招。

方源的星道境界只是普通,若是寻常情况下,这些残招对他而言,只是参考价值,看看罢了。

但他有了九转智慧仙蛊之后,蹭用智慧光晕,灵感无限,硬是克服了星道境界方面的短板,顽强地将杀招完善出来。

“可惜,短时间内,我完善出星火遁、外映星念两招已经是极限了。毕竟我的星道仙蛊也只有三只。当然更主要的原因,是我的星道境界太低了。嗯?”

就在这时,方源仙窍里有推杯换盏蛊出现。

蛊中装载的消息,来源于琅琊地灵。

“琅琊地灵终于完成了琅琊福地的搬迁,可以将他那一套仙蛊借给我了!”

方源大喜。

他这一次,专门从东海回到中洲,就是为了等候琅琊地灵的这个消息,然后搬迁星象福地。

在方源的计划中,星象福地相当关键。

一旦他在北原犯罪的真相暴露,他就要亡命天涯。

到那时狐仙福地,必定凶多吉少。若他还是仙僵,仙元无法自产,那么星象福地就是他唯一的秘密基地了。

然而星象福地的位置,黑楼兰是知道的。黑楼兰知道,代表着黎山仙子也会知晓。

方源生性谨慎,既然如今已经成为星象福地之主,自然要将福地搬迁到一块只有他自己知晓的地方。(未完待续……)

ps:22点第二更!

\end{this_body}

