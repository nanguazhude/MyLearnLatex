\newsection{炼制群力蛊}    %第三十三节:炼制群力蛊

\begin{this_body}

黑楼兰渡劫,让方源隐隐预感到会有一场大战发生。

黑城乃是资深七转蛊仙,战力超出方源。雪松子坐拥大雪山第七支峰,也绝非易与之辈。

即便没有黑城、雪松子二人搅局,大力真武体渡劫,灾劫之重也要远超常人。

回到狐仙福地之后,方源就立即着手,尽全力战备。

他要在短时间内尽量地提升战力,就应该从四个方面着手:仙元、杀招、仙蛊以及个人战斗造诣。

个人战斗造诣方源已经很强,很难在短时间内提升。

要提升战力,就只能在其他三方面着手了。

方源首先查看自己手中的资本。

因为连运叶凡,方源去了南疆,这一个来回,青提仙元消耗两颗,剩下十七颗。

很显然,这个数量远远不够大战消耗。按照方源前世记忆,一场大战,消耗仙元大几十颗很平常,有时候甚至上百颗。

方源的仙蛊数量又多,因此急需补充青提仙元。

而要补充青提仙元,对于方源来讲,只有一个方法就是炼化仙元石。而方源手中的仙元石,也只剩下十一块半。

方源钻入地底洞穴,再次见到智慧蛊。

“智慧蛊,我这次给你带了点好玩的东西。”他开口打招呼。

尽管《人祖传》中,智慧蛊可以和人祖正常交流。但传说故事,也不是没有夸张的修辞手法。

方源也揣摩不准。

不管怎么说。智慧蛊仍旧保持着沉默,倒是智慧光晕闪烁了几下。

方源微微一笑。

依他之前和智慧蛊打交道的经历,看到智慧光晕闪烁。便知道智慧蛊应承下来。

于是,他就将顺手牵羊来的灵芝王心,当场栽种下去。

只是几个呼吸的功夫,地面就破开,一颗颗的蘑菇状的小灵芝,便以肉眼可见的速度迅速生长、壮大,最终长成板凳、桌椅的高度。

没办法。狐仙福地的土质,和众生书院山谷的土质并不一样,不适合菇林的生长。

这已然是菇林生长的极限了。

方源也不可惜。本来就是随手而为,他顺手坐在一株灵芝上。

灵芝狠狠颤抖了几下,居然能承受住方源仙僵之躯的庞大重量。这个洞窟原本空无一物,但现在却不再单调。智慧蛊仿佛一个新奇的婴孩。在矮小的菇林丛中不断盘旋飞舞,起初是小心翼翼,不过很快,它飞行的轨迹透露出欢快的情绪,像是小孩子发现了新玩具似的。

“智慧蛊啊,该办正事了。”方源看了一会儿,开口道。

智慧蛊便停在一株灵芝的顶部,再次扩散出智慧光晕。将方源笼罩在内。

方源默默催动乐山乐水蛊,开始推算仙蛊残方。

七日之后。他彻底成功,再一次完善了三道仙蛊残方。利用星门,他和琅琊地灵完成交接,获得仙元石四十块。

按照之前的定价,完善九成仙蛊残方,得十块仙元石。完善八成仙蛊残方,得二十块仙元石。

方源这一次完成的仙蛊残方,有两道是九成残方,一道是八成九的残方。

琅琊地灵手中的仙蛊残方,并非件件都是九成残方,更多的残方完善度都比较低。

方源每次得到的仙元石虽然多了,但推算完善度低的仙蛊残方,消耗的青提仙元也多。总得来说,每次和琅琊地灵的交易,赚取的净利润在二十六块仙元石左右。

回到狐仙福地,地灵小狐仙传来了好消息:宝皇天中似乎有蛊仙意外寻找到了一只六头大蛇的遗骸,目前已经将遗骸拾掇好,分门别类地挂在宝黄天中售卖。

方源闻言,立即亲自催动神念蛊,沟通宝黄天,以八臂仙人的身份接触售卖的蛊仙。

他手中的招灾仙蛊,正需要荒兽六头大蛇的黑色血液喂养。

但方源看了之后,却发现这摊上有皮,有肉,有骨,有眼珠,有蛇筋,却偏偏没有六头大蛇的黑血售卖。

“我意外发现这具荒兽尸骸时,已经距离六头大蛇死亡的时间,过去了近百年。因此大蛇的黑血都流尽了。不好意思。”售卖的蛊仙如此回答。

方源辨别了一下六头大蛇的尸骨,发现蛊仙所言非虚,的确是如他所说。

不过方源并不气馁,仍旧决定将这具六头大蛇的骨骼买下。

六头大蛇乃是荒兽,战力媲美六转蛊仙,一身是宝,皆能用来炼蛊。

方源大手笔,直接买下整个骨骼,花了三块仙元石。

骨骼中暗藏骨髓,能够生血。方源前世乃是血道高手,最擅长的就是提取血液。有这副骨架,招灾仙蛊的食料就有着落了。

买下尸骨之后,方源又找到仙猴王石磊在宝黄天中的摊子。

不出所料,这里神念频动,生意火爆得很,都是冲着石猴毫毛这一炼蛊材料而来。

石磊没有亲自关注宝黄天,而是留下一段傲意看守。

智道传承难得,但智道蛊虫在宝黄天中,并不少见。蛊仙几乎人手一只,主要都是为了在宝黄天中买卖方便。要不然方源当初,也不会在这里买到那么多的智道蛊虫。

“你这石猴毫毛,如何售卖?”方源催谷神念询问石磊的意志。

石磊意志语气生硬:“一百根毫毛,一块仙元石!”

饶是方源见多识广,也被这价格吓了一跳:“怎么如此昂贵?”

石磊傲意把头一昂:“嫌贵可以去别的地方买啊。”

方源心中冷哼一声,这石猴毫毛只被石磊重新研发出来。别的地方哪有售卖?这就是做垄断生意的好处,价格随便定。偏偏价格这么高,还有许多蛊仙买。

哪个蛊仙手中没有蛊方。仙蛊残方!也许恰巧有那么一两张,就需要这个石猴毫毛作为材料。

还有一部分蛊仙,则打着另外的主意。

他们买下石猴毫毛,也想尝试一下研发出如何生产之秘。如果自家福地能够生产,那么这样的暴利,自己也能分得一杯羹了。

“别说我没告诉你啊,如果你想研究这石猴毫毛如何生产。至少得买千根以上的毫毛。少了的话,是研究不出什么名堂的。”石磊傲意也知道蛊仙打算,竟直接坦言叫卖。丝毫不担心石猴毫毛的生产之秘被旁人研究了去。

方源试着还价,但石磊傲意寸步不让,不耐烦地道:“走开,走开。不想买就别捣乱!有的是人买!一个穷光蛋。也想买上古的炼蛊材料。”

方源冷哼一声。

仙猴王石磊乃是七转蛊仙,脾气火爆高傲,不仅将福地经营得很好,而且战力卓绝。

他自命不凡,但也的确有自命不凡的资本。

五域乱战时,他雄霸一方。后来凤九歌进攻琅琊福地身陨,他再无人压制,势力膨胀极速。以至于中洲十大派都不能钳制他。

他桀骜不驯,不甘人下。捣乱中洲秩序,甚至公开叫嚣,要攻上仙庭!

方源现在大战在即,急需石猴毫毛炼出群力蛊,因此不管石磊傲意多么鼻孔朝天,也只能捏着鼻子买下石猴毫毛。

最终,他化去五块仙元,买下五百根石猴毫毛。

石猴毫毛在手,方源着手炼制群力蛊。

群力蛊早已绝迹,隶属力道,能集群体之力于一身。若组合进杀招万我,将会带给方源战力上的大提升。

方源并非要炼出一只五转群力蛊,而是越多越好。这个时候,方源他收购的老毛民,终于发挥了关键作用。

这些老毛民活不了多久,但各个都是炼蛊好手。方源将群力蛊分解步骤,将不重要的,成功率高且又繁琐的步骤,都交给他们代劳。自己这个炼蛊大师,则专门进行关键步骤。

连续三天三夜,不眠不休,方源得到了总共四十五只五转群力蛊。

蛊虫难炼,越是高转,越是容易失败。若次次成功,五百根石猴毫毛就是五百只群力蛊。

但方源消耗了大量的炼蛊材料,将石猴毫毛消耗一空,最终只得到四十五只。

就这种成功率,也是群力蛊方比较好练,成功率较高的结果了。

当然,方源也可以动用智慧光晕,改善群力蛊的蛊方,或者寻找到石猴毫毛的替代材料。

但方源现在,却没有这个时间去慢慢推算。

就算有时间,他也未必有炼道底蕴。毕竟这是上古力道的蛊方,和现今炼道体系并不相同。

即便他有底蕴,有时间,也得算计支出和收益。别忘了,整个推算的过程中,他还需要消耗青提仙元。而这些青提仙元的数量,他无法估计。

炼出群力蛊后,方源又马不停蹄,利用智慧光晕,思考杀招万我。这花费了他一天时间,终于将四十五只群力蛊,完美地融汇到杀招万我的体系当中。

炼蛊期间,小狐仙领命,终于在狐仙福地的西部,完成了血池的建设。

血池如湖,十分庞大。湖中央,六头大蛇的尸骨载沉载浮,刺鼻的血腥气味弥漫方圆百里。池子中,还有大量蛊虫,形成一个蛊阵,用来不断地消耗尸骨来营造黑血。

其中蛊阵的核心,就是方源得自北原的那只五转战骨车轮蛊。

狐仙福地的时间流速,和中洲外界有着五倍之差。正因为如此,方源才能在北原时间只剩下五六天的情况下,尽量准备。

时间推移,越来越临近黑楼兰渡劫之日。

太白云生通过星门,回到狐仙福地。

他带给方源一个惊喜。

------------

\end{this_body}

