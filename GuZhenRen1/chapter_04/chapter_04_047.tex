\newsection{宣布叛逆}    %第四十七节:宣布叛逆

\begin{this_body}

%1
“夫君,煌儿这个样子,已经整整三个多月了。我们该怎么办?如果她出现了一点意外,我也不想活了。”白晴仙子看着蒲团上的凤金煌,低声哭泣着。

%2
她是堂堂的灵缘斋蛊仙,往日里风姿绰约,高贵雅致。但此刻事关自己的爱女,她失了分寸,再无往日里的悠然仪态。

%3
而在她的面前,凤金煌盘坐在蒲团上,一副打磨空窍的姿态。

%4
她头戴凤冠,金眉修长入鬓,眉心一点红痣,容貌结合了其母白晴仙子的柔美,又掺杂着其父凤九歌的英伟,显得端庄雍容,且又英美无俦。

%5
她天资脱俗,才情绝佳,盖压十大古派中的同龄人。若非方源破坏了她的登顶,此刻的凤金煌已经是狐仙福地的主人。

%6
她的肌肤若雪,双眼闭合,呼吸均匀绵长,仿佛在酣睡。

%7
在三个多月前,凤金煌像往常一样打坐,温养空窍窍壁。但这场平凡普通的修行中,却发生了意外,她从此沉眠不醒,任凭外人用什么方法,都唤不醒。

%8
自从发现了凤金煌陷入古怪的状态后,凤九歌、白晴二人就想法设法拯救爱女,可惜收效甚微。

%9
他们请来灵缘斋的太上长老们,试了各种法子,却也都束手无策。

%10
“夫君,你说我们的煌儿会不会一睡不醒?再睁不开眼看我们?”白晴仙子想到恐怖之处,不禁悲从中来。

%11
凤九歌心中长叹一声,伸出手臂环抱住自己的爱妻。嘴上宽慰道:“晴儿,你无须担心。太上大长老不是说了吗?我们的女儿发现了梦翼仙蛊的真正用途,陷入空明无想的境界。整个魂魄都进入到了某种玄妙无端的境地中去了。她并没有失去生命,仍旧活得好好的。我们应该相信她,说不定这对她而言会是一场奇遇。”

%12
凤九歌、白晴夫妇虽然没有唤醒凤金煌,但尝试并非没有成果,他们已经探明造成凤金煌沉睡不醒的原因,就是梦翼仙蛊!

%13
白晴仙子偎依在丈夫的怀中,感受到凤九歌坚实有力的胸膛。心中稍安,她纠结道:“煌儿三岁时,梦翼仙蛊主动来投。我的心中就一直惴惴不安。这仙蛊我们从未见过,翻遍了灵缘斋的所有典籍,也没有任何情报。之前我们试探出来的效果,是消耗魂魄。产生绚烂羽翅。现在看来。恐怕不是梦翼仙蛊真正的作用。”

%14
凤九歌点点头,认可地道:“你的分析很有道理。蛊师养用炼,得到不知根底的蛊虫,往往会在不断的运用中有新的发现,最终将蛊虫的底细摸透。就算是知名的蛊虫,也存在某些鲜为人知的地方。而蛊师、蛊仙们得到的宝贵的经验,向来都是敝帚自珍,以防他人窥破自家虚实底细。煌儿将来苏醒。我想一定会大有收获,懂得真正运用梦翼仙蛊的方法。我这次远去北原。家里就靠你了。煌儿苏醒后,你就立即通知我。”

%15
中洲十大古派布局无数年,殚精竭虑,耗费无数人力和资源,苦心孤诣,艰难布置,这才在八十八角真阳楼中有所成果。

%16
结果方源按照前世记忆,提前摘了果实,将这些布置化为己用。

%17
最终,王庭福地毁灭,八十八角真阳楼倒塌。消息传来,对于预谋已久的中洲十大古派来讲,无疑是当头棒喝,惊天的噩耗。

%18
作为此项计划的领头羊——灵缘斋,更是受到来自各方面的压力,有来自其余九大古派的,也有灵缘斋内部的。

%19
“图谋八十八角真阳楼的计划,是从墨瑶仙子遗留的资料开始。这么多年来,一直都是我派领头,并掌握最关键的手段。这一次出现了这么大的变故,我们不仅要洗清嫌疑,同时更要查明真相!作为本派第一战力,我虽然很想留在女儿身边,但遗憾的是其余长老都属意我去调查。”凤九歌叹息道。

%20
“夫君此去北原,一切都要小心。见机不妙,以撤退为先。毕竟那里不是中洲,出了事情没有自己人帮衬。不要过于信任同行的其他九派蛊仙,你当年大战十大派,连战连捷,无一败绩,令那么多人颜面无存。这些年,你又是公认的十派第一强者,盖压他们一头。灵缘斋若是失去了你,他们兴许都会松一口气。”

%21
“你放心,为了你和女儿,我会小心的。”凤九歌抚摸着白晴仙子的面颊,在她的唇边轻轻一吻。

%22
唇分后,他掏出一只五转报信青鸟蛊。

%23
“这是从狐仙福地的来信,叫方源的那个臭小子,答应了煌儿的挑战。”凤九歌道。

%24
“是他?煌儿自出生,唯一在方源手上栽了个大跟头。煌儿归来之后,愤愤不平,矢志报仇,之前让夫君给去了一份挑战信,一直都没得到回应。没想到,他现在忽然回信,是有什么目的?”白晴仙子对方源的印象很深刻。

%25
凤九歌目光沉凝:“自从煌儿失败之后,我们就详细调查了方源这个小子。他来头不小,背后有神秘势力撑腰,居然敢在十大古派的眼皮子底下,火中取栗,将狐仙福地夺到手里。偏偏仙鹤门自作聪明,主动承认他是仙鹤门人,搞的我们都没法下手。”

%26
“这样一来,仙鹤门便能排除我们九派,自己一方独自回收狐仙福地了。换做是我,恐怕也会这样做的。”白晴仙子道。

%27
“中洲时间已经过去了一年多了,连北原的真阳楼都倒了,可笑的是,仙鹤门仍旧没有得逞,狐仙福地还在这个臭小子的手中。简直是个笑话。”凤九歌不屑地冷笑一声。

%28
“也不能这么说。最近这一年,仙鹤门先是为了捕捉上古荒兽灵犀耗费精力,而后门中多位蛊仙又陷于轮回战场,现在辖区内的钧天剑派又想独立,使得仙鹤门上下应接不暇,焦头烂额。再加上我们九大派,得悉真相之后,多次暗中施压,间接阻挠了他们对狐仙福地用兵。因此狐仙福地,才没有陷落。”白晴仙子道。

%29
凤九歌微微摇头:“方源背后的势力,一直笼罩在迷雾中,还未探明。昨日仙鹤门公然宣布方源因为得到狐仙福地而起了异心,将其视为门派叛徒。显然怀柔计策没有起到效果。如此看来仙鹤门针对狐仙福地的大动作,就在近几日了。嘿,我倒要看看,究竟是什么势力,能够和我们十大派叫板。”

%30
白晴仙子目光一闪,忽然悟到了什么,道:“我知道这个方源的用意啦。他不是仙鹤门人,似乎也察觉到了仙鹤门要对付他。因此回了信,想打通对外的交际,利用我们灵缘斋来制衡仙鹤门。”

%31
“但是现在仙鹤门占据着大义,方源是仙鹤门叛逆。仙鹤门可以公然进攻狐仙福地,而我们却没办法掺和此事。”凤九歌道。

%32
白晴仙子有些不甘心:“真的没办法掺和此事吗?若是仙鹤门得到了荡魂山,恐怕整个门派的底蕴就要上升几个档次了。”

%33
“名门正派,自然要有法度规矩。毕竟中洲那么多的其他门派都盯着呢,若是十大派首先坏了规矩,今后还怎么号令一方呢?虽然仙鹤门欺骗在先,但九派的确被一时蒙骗,谁叫仙鹤门中有个方正,正巧可以利用呢。”

%34
“既然仙鹤门胜出一筹,那就得承认。至少这一次,我们是没法插手的。不过若此次仙鹤门攻打狐仙福地失败,我们九派就有插手的机会了。毕竟这个局面,大家都心知肚明,只是碍于规矩和颜面,都在装糊涂而已。仙鹤门若此次失败,还想独吞狐仙福地,我们九派是绝不会同意的。”凤九歌分析道。

%35
他顿了顿,又继续道:“前往北原之前,我打算将我素蛊,送给仙鹤门使用。”

%36
“啊?夫君这是何意?”白晴仙子微微吃了一惊。

%37
凤九歌运筹帷幄地道:“方源这一方实力不明,究竟能不能护住狐仙福地,还在两可之间。若他胜仙鹤门败,那我们十大派就一齐出力,瓜分狐仙福地。若他败仙鹤门胜,我们正可以借助我素蛊这个由头,在荡魂山上分割一块好处。”

%38
狐仙福地,荡魂行宫。

%39
“宣布我为门派叛徒,这么说来,仙鹤门攻打狐仙福地,已经近在眼前了。”方源看着手中的情报,心头沉重。

%40
净魂仙蛊还未喂饱,在这种情况下,催动仙道杀招万我,风险很大。

%41
搞不好,就会彻底失去净魂仙蛊。

%42
这样一来,没有万我,方源就失去了最大的底牌。

%43
“仙鹤门提前宣布我为叛逆,却不立即攻打。显然是摆明车马,想诱我背后的势力出手,届时好生较量一番。这的确是十大派的作风和底气。”

%44
方源稍微想想就知道,仙鹤门这一次的攻势必定极为凌厉!

%45
对他而言,现在最紧要的就是找到白莲巨蚕蛊,将净魂仙蛊喂个饱。

%46
同时,他还需要筹集资金,储备足够多的青提仙元。

%47
绝大多数的仙蛊,都需要仙元才能催动。

%48
没有仙元,纵有再多的仙蛊,也发挥不出威能来。

%49
战争打的是什么?钱!

%50
放在这个世界中,蛊仙战斗打的是什么?是仙元!是仙元石!

%51
“唉,这一次麻烦不小。好不容易复苏的经济,恐怕又要遭受沉重的打击。接连两场大战,将令我浅薄的经济彻底崩溃。”

%52
ps:万分抱歉,不好意思,我忘了上传了。原以为中午已经将两章都上传了,要睡觉了,看了网页一眼,这才发现出现了这个重大的失误。给诸位造成的麻烦,十分过意不去,鞠躬致歉!

\end{this_body}

