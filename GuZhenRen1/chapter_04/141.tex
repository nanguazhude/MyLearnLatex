\newsection{攻防较技孰更强}    %第一百四十一节:攻防较技孰更强

\begin{this_body}

血幕牢牢罩着墟蝠尸体,顶住魔道蛊仙们的狂轰滥炸。[八零电子书wWw.80txt.COM]

未料到血幕这般结实,情急之下,皮水寒大喊一声:“都闪开,让我来!”

他是七转蛊仙,在魔道中早就名声广传。众人纷纷飞离,给他让开空间。

水道、冰道杀招流川!

皮水寒仙窍中,上万只蛊虫调动起来,七转的红枣仙元一颗颗迅速消耗。

哗啦!

无边的浪潮,从皮水寒周围数百步的空中,凭空激射。

惊涛恶浪,宛若成千上万条逆世狂龙,咆哮着,怒吼着,带着毁天灭地的气势,倾泻而下!

巨浪滔滔,还在空中坠落的过程中,又起变化。

咔嚓嚓的声音,响彻众人耳畔。巨浪由内而外,结成坚冰。几个呼吸之间,滔天巨浪化为一道庞巨的冰川,宛若一头盖世的巨鲸镇压而下,又仿佛是天柱倾覆。

冰冷的寒气,令人忍不住打寒颤。

恐怖的攻势,让人不自觉地屏住呼吸。

大殿中,东方长凡的星意,一扫之前的云淡风轻之态,全身贯注,将防御力量催至最大。

之前魔道众仙虽是狂轰滥炸,但步调不一,数量虽多,质量不行。如今皮水寒动了底牌手段,出自他一人之手,力量融汇统一,比之前的狂轰滥炸,至少要强大十倍。

轰隆隆……

冰川从天而降。重重地砸在血幕之上。

血幕被巨大的重量,压得形变,但终究还是死死地抵挡住冰川的碾压,寒气的侵蚀。

仙道杀招流川虽强,但东方长凡的血幕。却是抽取了八位蛊仙的仙窍本源,又结合地利,早做了布置。皮水寒一人之力,并未打破血幕。

但此杀招的效果仍旧惊人。

众位魔道蛊仙俯瞰下去,地面上以墟蝠尸体为中心,方圆数百里都是冰川。

死去的太古墟蝠,本就庞大若山。此时在冰川的盖压下。仿佛盖上一层厚厚的冰甲。更显得庞大。

半透明的冰山中,裹着一只鲜红的血幕巨蛋。

刺骨的寒气,还在不断四溢,将最边缘的树木草地染上冰霜,扩大着地界。

一时间,众人纷纷向皮水寒投去惊异的目光。<strong>最新章节全文阅读www.QiuSHU.cc</strong>

虽然没有打破血幕,但这杀招之威。仍旧叫人印象深刻。

“就算是我八只力道巨手齐齐出动,正面也挡不住这个杀招。”方源目光闪烁着。

这才是老牌七转蛊仙真正强大的一面。

相比较而言,方源还是有很大差距。

之前方源以一敌二,抵挡自在书生和皮水寒,也是因为后两者并未动用底牌手段。

蛊仙都有理智,大家都是为了智道传承而来,彼此之见,又没有什么深仇大恨,打生打死才是脑袋有病。

但如今,众仙都意识到不妙。这才有皮水寒调动压箱底的底牌,迸发全力,血拼一记!

“我虽然依仗着仙道杀招万我,有了七转战力,但也只是勉强够得上这个标准。真要和皮水寒这等人物死战,失败的极可能是我。不过真要面对此招,我也不会力拼。应当避其锋芒,迂回游斗。”

仙道杀招冰川之强,让方源看清了皮水寒的一部分底细,同时也清醒地意识到,他自己和皮水寒的差距。

一时间,方源心中升腾起一股焦躁的情绪。

蛊仙本身的修为,是影响战力的主要因素。皮水寒的强大,有一部分原因,是消耗了七转红枣仙元。方源此生已经极为优异,拖累他的乃是仙僵之身。他真正的修为,只是六转垫底,连一次天劫都没有渡过。

“修为才是根本,是基石。眼下仙僵之躯虽然强大,但却没有向前大进步的空间。究竟有什么法子,才能令我回复人身?”

“皮兄好大的威风!也让我来试试身手罢。”这时,自在书生朗笑一声,向皮水寒飞来。

皮水寒冷哼一声,面色不大好看。他劳师动众,结果仍旧没有打破血幕,见自在书生出手,他缓缓让开最中央的位置。

众目睽睽之下,自在书生飞临到墟蝠尸体的正上方。

他早已经酝酿,陡然双眼圆瞪,眼眸全白,射出两道稀薄的淡淡白光。

此招近乎悄无声息,论气魄和之前的冰川完全不能相比。

不知情的众仙,看得有些莫名其妙。

皮水寒却是脸色微变。

但见两道淡白目光,照射之下,冰川顷刻消融,血幕剧烈震颤,竟有不敌之象。

“嗯?这就是自在书生?真是好一招千解!”大殿中,残阳老君却是认出了此招。

中洲十大古派,为了八十八角真阳楼,渗透北原多年。残阳老君既然冒险前来,自然对北原情报大有深究。

自在书生的跟脚,就是中洲陈家。

中洲陈家擅长律道,但中洲门派制度盛行,没有家族的生存空间。被排挤之后,陈家便转移到北原。

可惜,北原正道却是黄金血脉的天下。陈家虽是有蛊仙护佑的超级势力,但被黄金血脉势力暗中排斥,终于还是衰弱凋零。

到了自在书生此代,便只剩下他一个翘楚,家族已经不存,只剩下他一个人单打独斗。

“陈家擅长律道,家族中的蛊仙,六转能使杀招百解,七转能施展千解,到了八转施展万解。威能绝伦,能攻能守,有近乎化解一切的奥妙。当初在中洲,不知多少门派蛊仙,惨败在万解之下,多少人吃了这招的亏!”残阳老君语气唏嘘,感慨起来。

“你说了这么多废话。还不出手?血幕和自在书生对耗极大,这些都是珍贵的仙窍本源!”东方长凡的星意催促出声。

“哈哈。”残阳老君负手站立,不见其出战对敌,“自在书生只是一人,你这边却是八位联手。他又不知道此中虚实。且看我稍施手段。”

说完,他便散出自家的七转气息。

气息暴露出来,顿时让天空中的魔道蛊仙们一惊。

“对方居然有七转蛊仙?”

“东方家的蛊仙们,不都是六转的么?难道有谁,秘密进入了七转,却一直秘而不宣?”

“又或者东方部族请来了什么强援?别忘了,他们可是黄金血脉。正道的超级势力!”

众仙惊疑不定。

自在书生虽然面色不变。心中也不由地生出动摇。

原来,他这杀招虽然威力惊人,来头极大,但消耗红枣仙元十分剧烈。

对方既有七转蛊仙,那么显然可以和他对拼。自己这边虽然人多势众,却各个有着打算。自在书生心中不免思量:自己可不能现在就无脑冲锋,大损仙元。否则到真正血拼之时。却后力不继,给他人做了嫁衣裳。

想到这里,自在书生便停下杀招。

“你看,这不就停了?”残阳老君哈哈一笑,稍显得意之色。

他只是稍漏气息,便令自在书生收手。论耍弄智谋,残阳老君虽然不及东方长凡,但凭借丰厚的人生经验,也可深达人心,足以借助形势实施谋算。

“时间拖得越久。对那东方余亮就越是有利。这血幕厚实无比,要想破它,还得我们动用底牌,轮番轰炸。”自在书生一边说着,一边将目光投向方源这处。

在场的游地三英失去了韩东,陆青冥和苏光二人,只是六转。除去自在书生、皮水寒之外。展露七转战力,就还剩下方源一人。

皮水寒也看向方源。

这些魔道蛊仙强者,自然不会让方源轻松地讨便宜。

“请二位仙子护住我的安危,接下来我要全力出手了。”方源暗中传音,得到肯定的答复之后,当即虚空盘坐。

八只力道巨手,飞向血幕。

尸山血幕的外围,已经被一层冰川盖住,仿佛一座含馅的大冰山。

方源不管不顾,调动八只力道巨手,分别位于八个方位,展现巨掌,掌心向内,紧紧贴住冰山表面。五根手指头,则狠狠地抠进冰山之中。

“起!”方源断喝一声,顿时一股隆隆之音,传入众人耳中。

起先这股声音,只是微不可闻。

但很快,声音越来越大,隆隆震响,仿佛闷雷阵阵,绵绵不绝。

大地震颤,一道道裂缝巨沟,出现在冰山的周围。

在众人惊异的目光中,只见冰山包裹着血幕,竟然缓缓地抬升上来,有脱离地面的趋势!

大殿中地动山摇,残阳老君不免失声:“这是?”

星意也变了脸色:“不妙。我这太古墟蝠尸体,早已经被我布置改造,为了岿然稳固,和脚下大地连绵一体,结成一座尸山。对方似有能拔取山气之能,偏偏克制此点!残阳,你速速出手!”

东方长凡的星意,急得差点形体崩散。

他的本体的确能推会算,但人力心智,总有极限。布置是死的,最算再面面俱到,也会有意外。

方源的拔山仙蛊,就是意外。

仙蛊能针对任何山体,东方长凡生前做了布置,将太古墟蝠的尸体和周围地势链接一体,形成一座名副其实的尸骨之山。

这点布置,却反而成了缺陷,正被拔山仙蛊所克!

“居然能把这座山直接连根拔起?”皮水寒死死盯着,目光不断闪烁。

“他到底是哪个人物?!”自在书生忍住心中惊疑,再度施展杀招千解,照着血幕望下去。

两相夹攻,轮到东方长凡一方,情势不妙起来。

残阳老祖嘎嘎大笑,终于在此刻出手:“且瞧我的追命火。”(未完待续 。)<!--80txt.com-ouoou-->

------------

\end{this_body}

