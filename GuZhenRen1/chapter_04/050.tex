\newsection{进攻狐仙福地(中)}    %第五十节:进攻狐仙福地(中)

\begin{this_body}

天鹤上人犹豫!

一方面他受命于鹤风扬,身负艰巨任务。理智告诉他,为了仙鹤门的大局,牺牲方正这个凡人小子,是最正确的选择。

但另一方面,他的感情阻止他。

人非草木孰能无情?

和方正的朝夕相处,看着他一步步成长,看着他有时候犯糊涂,看着他为复仇而努力刻苦地去修行,像极了曾经的自己……

天鹤上人不忍!

“方正,你这个臭小子,教了你多少遍了,这是最基本的炼道手法啊!”

“对不起,师傅。”方正手掌冒烟,掌心中的蛊虫已经成了一堆焦炭。

……

“方正,给我练习,练习,再练习。我堂堂天鹤上人的徒弟,居然连飞鹤的箭矢阵型都排布不出来,这要是传出去,让我的老脸往哪里搁?!”天鹤上人气得怒吼。

“对不起,师傅。我一定努力……呃!”方正越是紧张,越容易出错。

天空中两股鹤群,在他的失误下,直接飞撞到一起。

一时间,飞鹤哀鸣惨叫,骨折声不断,像是饺子下锅一般,往下坠落。看到这幅景象,方正空窍中的寄魂蚤都僵滞住了,旋即暴跳如雷,“你这个大笨蛋!”

……

“师傅,我错了,对不起。”关在禁闭室中,方正面对着墙壁,对天鹤上人道歉道。

“傻小子,说什么对不起?打得好!”

“啊?师傅,不是门规中禁止同门斗殴吗?”

“哼,张南的师傅是玄机子。当年师傅就看玄机子不爽,将他揍成了猪头。关在这里,比你要久呢。老夫我虽然已经死了,但你作为我的徒弟,怎么可以这样被人欺负!”天鹤上人嗤笑道。

“师傅……”方正哽咽。双眼泛红,眼眶中滚动泪水。

“白痴,哭什么哭。男儿有泪不轻弹!”天鹤上人教训道。

“是的,师傅。对不起,师傅!”

……

在这个关键的时刻,时间仿佛被拉得无限漫长。

往昔的一幕幕。浮现在天鹤上人的心头。一声声的师傅回响在他的耳畔。

天鹤上人大吼,吼声振聋发聩:“方正,你要加油!你忘记了方源带给你的痛苦了吗?忘记了初次来到仙鹤门,被周围同门欺负得头破血流了吗?忘记了家族中的冤魂,忘记你的舅父舅母惨死在方源的手中了吗?你的仇恨。你的努力,就看这一刻!你不能失败,你不能放弃!”“我一定要感应到,一定能感应到。”方正听到天鹤上人的吼声,心气劲儿为之一提。

但是,在他的感应中,仍旧是漫漫无边的黑暗。

不管他怎么努力,如何用功。都察觉不到任何的迹象。

“为什么啊?为什么!”方正心境动荡,开始紊乱,记忆最深处的那些不堪回首的往事。又一幕幕浮现心头。

从小到大,方源带给方正的阴影,仿佛和感应中的黑暗同化为一体,带给方正无限的压抑,一切都黯淡无光。

方正仿佛成了极微小,极微小的一点。置身在这广袤的黑暗中。

彷徨、无措、孤寂、无奈种种负面情绪,充斥他的心头。

“对不起。师傅,我……失败了。”方正留下泪来。身体达到极限,心境也接近崩溃。

“不!你不能失败,绝对不允许!”天鹤上人也急了,一声声呐喊。

但方正渐渐地听不到了,他即将陷入昏迷,就好像之前的训练一样。

“该死,该死的!”天鹤上人在心中咒骂,这一刻他想到自己的夺舍计划,想到任务失败后,鹤风扬回来时对他的惩罚。

“方正,你这个家伙,枉费我倾尽心力地来栽培你。结果到头来,你却仍旧无法那个男人的阴影!既然如此,那就让我来帮助你一把!!”

天鹤上人念头闪烁,终于说服了自己。情势也逼迫他,不得不这样选择。

他掀开那张底牌

瞬时,整个血池开始散发出明亮的光,一扫之前的昏暗。

方正的身上,血池周围暗藏的蛊虫,接连催使起来。

“啊――!”方正像是触电一般,身躯巨颤,头猛地往上仰,双臂张开,双手捏拳,指甲直接陷入肉里。

无以伦比的痛苦,袭上他的心头,让他大翻白眼。

几乎在下一个呼吸,方正的吼叫声戛然而止,他就失去了清醒的意识,但在蛊虫的作用下,他仍旧在催动败血妖花蛊,仍旧在利用血感应蛊,感应方源的存在。

血池沸腾,咕咕作响。

咯吱吱……败血妖花急速生长,发出令人毛骨悚然的声音。

方正保持着仰头张臂的姿势,宛若雕塑,一动不动。

原本的妖花藤蔓,和针一样细小。但现在这些藤蔓长得粗壮,至少比手指头还要粗。最大的一根藤蔓,从方正的喉咙深处生长出来,宛若一条巨蟒,钻出他的嘴巴,向上生长。

除此之外,他的耳朵中,鼻孔中,也冒出藤蔓。

很快,他的肌肤孔隙都钻着藤蔓,方正彻底成了妖花的养料,皮开肉绽,面目全非,仿佛是修剪得当的花草塑像。

“臭小子……”看到方正变成这般凄惨的模样,天鹤上人原先的焦急紧张,都化为乌有。他感到心中空落落的,虚不着底,十分难受。

很快,这种难受的感觉转化为沉郁的愧疚,充斥天鹤上人的心头。

“臭小子,师傅我……对不起你啊!”藏在寄魂蚤中的天鹤上人的魂魄,此刻竟也流淌下点点会会的魂泪。

“嗯?鹤风扬的布置启用了啊。”伏虎福地深处,仙鹤门太上三长老缓缓睁开双眼,口中喃喃。

他想到了鹤风扬临走前。特意拜托他做的事情。

于是,三长老虎魔上人消耗仙元,驱使一只仙蛊,化作一道惊鸿飞出大殿。

这仙蛊,也不是虎魔上人的。而是鹤风扬从桑心夫人手中借来,乃是六转信道仙蛊,名为――同感。

十来个呼吸的功夫,仙蛊就降临道血池上空,方正的头顶。

“这是――仙蛊?!”天鹤上人瞪大双眼,心中震动。

同感仙蛊绽放出一道灰白色的光柱。光柱如烟似雾,笼罩住方正。几息之后,原本灰白的光柱,被方正染成血红色。

“这是……哪里?”方正的残留意识,环顾四周。

四周是一片的黑暗。不论他往哪里走,走了多少不,四周仍旧是黑暗,深邃广袤。

但就在这时,一道血红色的光点,出现在方正的前方。

“啊?那是……”方正试着走过去,随着距离的接近,他从这道血色光点中。察觉到了一股熟悉的气息,“这是哥哥,不。方源的气息!师傅,师傅,你听到没有,我成功了!我终于感应到了!”

方正激动万分,试着开口,却说不出话来。

他的嘴张的老大。几乎变形,被粗壮的妖花藤蔓撑得。但是他的眼角处。却流下了一滴喜悦的泪珠。

察觉到这颗泪珠,天鹤上人的魂魄狠狠一颤。陷入到死一般的沉默当中。

方正接近血色光点,忽然光点爆发出极强的吸力,将猝不及防的方正残留意识,全数吸扯了进去。

在这一瞬间,狐仙福地中的方源,轻咦一声,缓缓睁开双眼。

他感受到了一处血池,感受到了方正的身体,感受到了繁盛妖冶的败血妖花,感受到了空窍中的寄魂蚤,感受到了头顶上空的信道仙蛊同感……

“原来是这样啊。”方源冷笑一声,“我可爱的弟弟,你终于要死了吗?不,用正道的话讲,是为集体牺牲小我了吗?呵呵呵。”

“怎么回事?”一旁坐着的太白云生,见方源忽然出声,连忙问道。

在他的另一边,坐着黑楼兰、黎山仙子,同样向方源投来询问的目光。

“这一次真是巧了,原本想谈关于胆识蛊的生意,结果仙鹤门攻打上门。”方源笑了一声,特意垂下眼帘,挥手道,“对方利用我的亲弟弟,又用了仙蛊同感,诸位且先离开荡魂行宫,埋伏下去。详情我师兄会为二位解答。”

仙蛊同感,能令双方感受到彼此的情况。

这一刻,方正、方源通过血感应、同感仙蛊,感受到了彼此的状况。

当然,因为双方的实力差距,方源能洞察方正的一切,而方正只能通过方源的双眼,观看事物。方正感知有限,甚至连方源的空窍、仙窍的方位都无法察觉。

黑楼兰、黎山仙子对视一眼,均看出彼此凝重的心情。

碍于大雪山盟约,她们俩不好逃避,只能硬着头皮和方源并肩战斗。

天梯山上,盘坐在白云蒲团蛊上的鹤风扬,忽然睁开双眼。

“时机成熟了!”他眼中精芒暴射,当即取出一只仙蛊。

“拓宇仙蛊。”苍郁仙子听到动静,看过来,口中轻呼一声。

这仙蛊仙鹤门太上二长老之物,能够扩宽福地空间,此时被鹤风扬借来,却是当做仙道杀招的核心来用。

仙道杀招――破门而入!

十多颗青提仙元在一瞬间消耗,庞大的气息升腾起来,空气荡起半透明的涟漪。

咔嚓嚓……

仿佛玻璃碎掉的声音,在鹤风扬、苍郁仙子二人面前,空间破碎,露出狐仙福地的一角地貌。

鹤风扬施施然收起拓宇仙蛊,和苍郁仙子对视一眼,风度翩翩地邀请道:“仙子,请。”

苍郁仙子娇笑一声,迈开秀步,正要跨入狐仙福地。

忽然间,一道电光向她猛烈袭来!

------------

\end{this_body}

