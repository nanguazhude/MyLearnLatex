\newsection{冰钻星尘}    %第二十一节:冰钻星尘

\begin{this_body}

星萤蛊宛若蜂群一般,盘旋在半空中,绽放着湛蓝星芒。?

星光受到一股无形之力的牵引,最终汇聚到一扇星门之中。

这是一道巨大的圆形拱门,完全是由星光构成,灿烂如梦,优雅如幻。

随着时间的推移,星萤蛊群中不断减员,死去的星萤蛊从半空中无力地坠落到地上。

一只脚从星门中踏进狐仙福地,旋即整个人都走出星门。

老者是个老人,雪发苍苍,皱纹如沟,双眼沧桑却又明亮,正是刚刚从琅琊福地归来的太白云生。

“又回来了啊……”太白云生风尘仆仆,望了望这狐仙福地中的天和地,不由感叹一声。

他虽然在这里生活的时间,一点都不长。但他却觉得,这里就是他的家。

他对这里,有一种归属感。

这时,方源从高空中飞过来:“哈哈哈,老白你回来的正好,上来,看看我刚推算出来的杀招。”

太白云生看到这个熟悉的身影,当即朗笑一声:“哦,是吗?正巧我在墨人王那里,买下了一道云道防御杀招。咱们师兄弟两个,不妨就切磋一下。”方源听了一挑眉头,伸出一只右手,掌心向上,五指平摊。

一团星光很快在他的掌心中酝酿而成。

“咦?难道你这杀招,是星道杀招?”太白云生看到这团星光,神情微讶。

他飞到高空,身边浮现出一圈白云。

太白云生处在白云圆圈中心,白云像是一个巨大的腰带,紧随着太白云生而动。

方源缓缓屈起五个手指,将星光轻轻地握在手心。

“老白,你可接好了。”他低啸一声,屈指一弹。顿时一点星光,电射而出。

星点体型微小,速度极快,在空中迅速划过,向着太白云生电射而去。

“速度不错。”太白云生点头评价,这么远的距离,再加上全神戒备,他本有机会躲闪。但此刻双方切磋。他有心试验一下自己的云道杀招,于是站立不动,硬抗星点。

星点接近太白云生之际,白色云圈忽然转动起来,速度极快!

云环成功地挡下星点。

太白云生停下云环,凝神看去,只见云环挡下星点的地方,已然变成蓝色,同时寒气四溢。冻结了一层小小的冰霜。方源嘴角裂开,露出参差不齐的尖牙:“我也同样如此。看招!”

他连续弹指,嗖嗖嗖。无数道星点转瞬间打出。

太白云生停驻不动,周身一道云环,迅速转动。速度之快,几乎已经绕成一个白色球影。

方源的星点,被这道云环尽数挡下。

方源攻势不停,眼中露出感兴趣的光。原本洁白的云环,渐渐地被星点染成蓝色,速度也越来越慢。

“你这杀招冻气十足,还有迟缓敌手的作用。”太白云生淡然评价,“我的杀招也有一个变化。”

说着,他继续催动杀招。

几乎被冻成蓝色冰柱的云环。陡然爆炸。砰的一声,蓝色冻气成圆状。向四周迅速蔓延。

太白云生略带喜色,自夸道:“我这云环,能汲取对手的攻势,渐渐储藏起来,到了储藏极限或者我催动杀招后续变化,就能引发云环自爆,让敌手尝尝他自家攻击的滋味!师弟,看来我这杀招,比你推算出来的要强啊。……

方源呵呵一笑:“老白,你这么说,还为时过早啊。”

话音刚落,他右手一抛,直接将手中的星团打向太白云生。

太白云生心中一凛,连忙催动杀招,顿时五道云环凝聚形成。以他为圆形,从内而外,云环大小依次递增,五道云环形成五个同心圆,将太白云生牢牢护在中央。

星团打将过来,五云环绕着共同的圆心迅速旋转,残影缭绕,形成一个白球虚影。

太白云生哈哈一笑,他身处其中,感觉到强烈的安全感。

但就在这时,星团轰然爆炸。

叮叮叮、当当当……

大量的星点,四处漫射,相互撞击。爆炸将太白云生的身影完全罩住,里面寒气四溢。

“这一下子的爆炸威力,远超星点的简单叠加。”太白云生面色微变,五云环迅速浸透成蓝色,自转速度暴降。

“再来!”方源八臂高举,八只大手宛若怪爪,齐齐虚抓。

下一刻,一团团的星光,在他手中形成。

太白云生见此,瞳孔一缩,再不敢托大,立即催动蛊虫,迅速转移。

方源站住不动,八臂依次投射,星团疾驰,直朝太白云生炸去。

轰轰轰……

爆炸声不绝于耳,太白云生狼狈不堪,左闪右躲。五云环依次达到极限,相继自爆,自爆的威力的确带给他不少帮助。

太白云生迅速填补五云环,任凭方源轰炸,却是始终维持着自身防线。

“可以了。”方源忽然停手。

太白云生气喘吁吁,缓缓飞行到方源身边:“这杀招叫什么名字?你用了几成威力?”

方源很不客气地答道:“我将其命名为冰钻星尘,是我的独创。全力催动时,可打出六次打出水缸大小的星团,不过星团越大,速度就慢了。因为这里面的星尘越多,相互碰撞,内耗了许多速度。”

“这么说来,你刚刚连五成的威力都没有用到!你这杀招威力怎么这么大?虽然不及仙道杀招,但已经远超大多数的凡道杀招了。”太白云生惊叹。

方源哈哈一笑:“我在王庭福地时,搜刮到一份传承,来自北原著名诗人都敏俊。他另辟蹊径,创造出一种全新的星蛊。这种星蛊,能增长星道蛊虫的攻击威力,等若削减版的功倍蛊。我这次推算杀招时,就将这种星蛊也融汇进去。因此威力才这样大。”

这正是方源选取星道攻伐杀招的原因。

太白云生这才恍然大悟:“原来如此。我说嘛,我这防御杀招,名为九云环,全力催发时,能一共存在九道云环。不仅防御厉害,而且能随时补充,还能将敌人的攻势反击回去。这可花了我整整两块仙元石!”

方源点点头:“你这杀招绝对值这个价,甚至在宝黄天。会卖到两块半的仙元石。墨人王卖你了一个人情啊。”

太白云生便笑:“这我心里明白,他是看中了我的江山如故仙蛊。他也是蛊仙,也要渡劫,将来福地受损,肯定也想请我去修复!江山如故蛊就是我今后的生财之道了。”

“老白,你这生意今后做大了,一定会受到许多蛊仙的欢迎的。来,咱们边飞边说。”

方源和太白云生徐徐飞向荡魂山。

太白云生往后回望了一眼,只见刚刚的战场上空。还遗留着数十朵湛蓝星云。从这些星云内部,不断传来噼啪的声音,显是星点相互碰撞造成的。

星云周围寒气四溢,随着时间越变越淡。

“师弟,你很了不起啊。你这杀招冰钻星尘,要卖到宝黄天去。至少得卖个四块仙元石。”太白云生感叹道。

“若非有智慧蛊,我怎么会有无穷无尽的灵感呢?”方源淡然答道。

其实单靠智慧蛊,若没有星罗小仙提供的冰道杀招寒纱。小杀招六九星尘,方源也无法推算出冰钻星尘这个杀招来。

星罗小仙崛起之前,一穷二白,借助五域大战的东风,凭借自创的寒冰星尘杀招,逐渐积累财富,崭露头角,直至名噪一时。

方源抱着尝试的态度,从现在的她身上。买到了寒冰星尘杀招的前身。

他看到寒纱和六九星尘的时候。就猜测到这两个杀招,就是星罗小新自创杀招的基石。

他花去半空仙元石买到手后。就赶往智慧蛊身边,靠着智慧光晕,不断推算。

智慧光晕带给他无限灵感,方源又是重生之人,亲眼见过寒冰星尘的威力,这样一来他就有了最关键的方向,而并非漫无目的去推算。

最终,他结合都敏俊的传承,推算出冰钻星尘杀招。

这个杀招,显然有别于寒冰星尘,比寒冰星尘笼罩的范围小,但威力无疑更大。

可以说,靠着都敏俊的传承之功,已经稍微超出了寒冰星尘一些。

“九转智慧蛊,果然非同凡响。真的难以想象,它全力催动时的景象。说实在话,我现在都有点羡慕你了。你转成僵尸,居然能有这样的好处!”太白云生羡慕地看向方源。

方源摇摇头:“成为僵尸,弊端很多。首先我得感谢红莲仙尊,其次我能做到这一步,也是机缘巧合。若非当时情况特殊,和智慧蛊达成了协议,否则我怎么会蹭得到智慧光晕呢?”

“是啊,若非有红莲仙尊打破宿命蛊,我们人死后,魂魄就会被生死门吸走,永远陷入沉迷死境。哪里会弥留在世上,形成僵尸呢!”太白云生长长一叹。

“好了,我临走前叮嘱你,要你向琅琊地灵旁敲侧击出运道的情报。不知道你这次回来,有什么收获?”方源问道。

自从他知道,当初长毛老祖和巨阳仙尊论道长达七天七夜,他就动了这个心思。

琅琊地灵可比墨瑶意志,容易对付多了。

但太白云生却摇摇头,嘴角露出古怪的笑意:“我修复了几条江河,就被琅琊地灵赶也似的催促走了。他要吃药舂碎片,忙得很呢。”

“吃药舂碎片?”)

------------

\end{this_body}

