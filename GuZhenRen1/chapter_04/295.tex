\newsection{恐惧最可怕}    %第二百九十六节:恐惧最可怕

\begin{this_body}

%1
------------

%2
中洲,灵缘斋。

%3
山峰林立,云雾缥缈。竹林深处,一道瀑布宛若白丝绸一般,悬挂而下。

%4
凤金煌静静地坐在一株松树的枝桠上,默默地看着眼前的瀑布,泪水无声落下。

%5
这段时间,中洲派遣去北原调查的蛊仙们,已经回来。

%6
但是凤九歌,却没了音讯。

%7
对于凤金煌而言,自己的亲生父亲音讯全无,陷落在外域,自然是凶多吉少。

%8
这些天来,凤金煌茶不思饭不想,修行全无状态,以泪抹面,日渐消瘦。

%9
生死。

%10
这个沉重的字眼,在出人意料的时节,狠狠地撞击在凤金煌脆弱的心扉之上。

%11
凤金煌也不是没亲眼目睹过死亡。

%12
只是当自己的亲人,面临如此处境时,她失去了一切的淡然和所有的平静。

%13
但她终究是坚强的。

%14
噩耗并没有完全打垮她,只有在没人的角落里,她才偷偷地抹眼泪。

%15
表面上,她一如既往的修行,但她全然不知自己最近修行的内容是什么。

%16
一道身影,仿佛水墨画,由淡转浓,悄无声息地降临到凤金煌的身后。

%17
“女儿。”身后传来熟悉的呼唤。

%18
凤金煌转过头,看到来人,正是她的母亲白晴仙子。

%19
“娘!”凤金煌再忍不住,一头扎进白晴仙子的怀抱当中,高声抽泣起来。

%20
白晴仙子好一阵安慰。凤金煌这才渐渐止住抽泣。

%21
“娘,爹那么厉害,一定没有事的,对不对?”凤金煌仰起脸,满是希冀期盼地望着自己的母亲。

%22
但白晴仙子却没有在此事上。直接安慰她,而是摇头道:“就算再强的九转蛊仙,都有灭亡的那一天。更何况你父亲呢?人总归是要死的,煌儿,别哭了,让娘来给你讲一段故事吧。”

%23
这是《人祖传》上的故事。

%24
话说人祖想要依靠羽民的能力,去救援陷落在平凡深渊里的女儿。

%25
然而羽民的自由。是不会被束缚的。

%26
人祖想了计谋。施展失败,羽民们宁愿死亡,也不愿违背了自由。

%27
人祖陷入迷茫之中。

%28
他找不到好的方法,来拯救自己的儿女。

%29
大儿子太日阳莽如此,女儿森海轮回也一样。

%30
这时候,人祖心中的自己蛊开口道:“人啊,你想救你的儿子太日阳莽。我有方法。”

%31
人祖想能救一个是一个,连忙问道:“哦?什么方法?”

%32
自己蛊笑道:“天下万物都会死亡,这是因为宿命蛊进入生死门,拜访公平蛊而留下的轨迹。人呐,你进入生死门,重走生死路,只要你不走在宿命的轨迹上,踏出独属于自己的路来。当你走进生死门,再走出去,形成一条崭新的路。这样就算成功了一大半。”

%33
“然后,你只要将你的儿子太日阳莽,带上你所走的道路,脱离生死门,就能回到人世间,太阳普照的地方。你的儿子太日阳莽就能脱离死亡,重获新生了。”

%34
人祖听了自己蛊说的方法。有些犹豫不决,但终究没有更好的法子。

%35
于是,他便决定先让森海轮回在平凡深渊里待一会儿,先按照自己蛊传授的方法,去救大儿子太日阳莽。

%36
人祖向生死门进发,走着走着,有一天碰到一个兽人。

%37
这个兽人十分的强壮,身上的肌肉如块块石磊,嘴里的獠牙比刀剑还锋利。他迈着巨大的步伐,在荒野里狂奔着,哀嚎着:“别过来,别过来!我怕!”

%38
人祖感到很奇怪,便问:“兽人啊,你怕什么?”

%39
兽人说:“我怕自己的影子,它始终跟着我,我怎么也甩不脱。我怕的只能四处乱跑,又累又渴又饿,我快要不行了!”

%40
人祖感到好笑:“兽人啊,你有如此强健的体魄,却怕无害的影子,你生的是一颗胆怯之心吗?这有什么好怕的?”

%41
这个时候,一只蛊虫从兽人的心中,钻了出来,朝着人祖大笑:“人啊,别大言不惭。你不感觉到害怕,是因为没有碰到我恐惧蛊,嘎嘎嘎嘎嘎。”

%42
“恐惧蛊?”人祖后退一步,面色变化。

%43
恐惧蛊一出现,人祖的心中就滋生出恐惧的情绪。

%44
他感到了害怕。

%45
恐惧蛊更加嚣张的笑起来,然后对兽人道:“暂且就放过你吧,小兽人,你这个可怜虫。”

%46
兽人解脱了,立即瘫倒在地上,喜极而泣。

%47
而恐惧蛊又转过头来,面对人祖:“人啊,你居然敢小看我恐惧蛊,现在我就要让你饱受恐惧的折磨!”

%48
说着,恐惧蛊就嗖的一下,直接钻进了人祖的心中。

%49
人祖感到无边的恐惧。

%50
害怕这个,又害怕那个。

%51
恐惧蛊让他害怕风,每一次风挂起来,人祖就惊惶大叫。

%52
恐惧蛊又让他害怕阳光,人祖只好在夜里赶路,经常迷路,白天的时候就钻进山洞中,或者浓密的树荫下潜藏。

%53
恐惧蛊还让人祖害怕树叶,于是人祖远离了丛林,任何一棵树,都能让他尖叫。

%54
恐惧蛊又让人祖害怕蛇,结果人祖连自己编织的草绳,都丢弃不用了。

%55
之后,恐惧蛊让人祖害怕雨。

%56
每当下雨的时候,人祖只能龟缩起来,胆怯地望着天空绵绵雨滴,害怕惊恐至极。

%57
人祖原本想要去往生死门,却身中恐惧蛊之后,举步维艰,根本走不远。

%58
当恐惧蛊了解到人祖的目的时,它又让人祖害怕死亡。

%59
人祖不敢再向生死门进发了。

%60
因为进入生死门。就是从生走向死。

%61
人祖害怕自己会死,只能停留在原地。

%62
自己蛊叹息道:“人啊,其实死亡并不可怕,真正可怕的是你心中的恐惧啊。”

%63
“没错!”恐惧蛊听了这话,骄傲地道。“只有我恐惧本身,才是最值得害怕的!”

%64
白晴仙子说完这个故事,怀里的凤金煌久久无声。

%65
白晴仙子怜爱地看着怀中的女儿,又道:“煌儿,不管结果如何,请你坚强起来,直面死亡!死并不可怕。每个人都会死。就是是九转蛊仙也不能免除。你的父亲也许死了,也许没有死。总有一天,我会死,你也会直面死亡。千万,不要被你的心中的恐惧击倒。”

%66
凤金煌娇躯一颤。

%67
她轻轻一挣,从母亲温暖的怀抱中挣脱出来。

%68
她的眼中还噙着泪花,此时此刻却闪现着坚强之色。

%69
她看着白晴仙子。微微咬牙:“娘,我明白了!我要去修行,我心中不再有恐惧,不管爹如何,我都不再惧怕,我要面对它,面对任何可能发生的事情。我是凤金煌,怎么可以给爹娘丢脸?”

%70
“呵呵呵,真是好孩子。”白晴仙子掩藏住眼里的忧愁,脸上则露出欣慰的笑容。

%71
其实她心里也慌张。

%72
凤九歌的失踪。影响非常之大。

%73
灵缘斋招揽了凤九歌,成为这一代中洲十大派中的魁首。其余九派,都或多或少遭受着打压。

%74
这就是个人修行力量体系社会的特色。

%75
将个人的影响力,扩张至最大。

%76
如今凤九歌不在,灵缘斋威势一落千丈,其余九派纷纷昂首,蠢蠢欲动。中洲暗流汹涌,将来必定会产生剧烈的动荡。

%77
门派之外,是如此。

%78
门派之内,白晴仙子的日子同样不好过。

%79
有人的地方,就有江湖。

%80
有门派,就有内斗。

%81
凤九歌的消失,让常年被打压在最底层的那一伙势力,陡然抬起头来。

%82
凤九歌真的太强势了,以至于白晴仙子都快忘了,门派中反对自己的大有人在。

%83
这些天来,这些人齐齐发力,四处排挤白晴仙子。

%84
白晴仙子深爱着凤九歌,当然想动身北原,前往救援。但她苦苦忍耐,按捺住这一股强烈的冲动。

%85
她有孩子,凤九歌如此强大,都音讯全无,她更不能轻举妄动。

%86
一旦她自己也去了,凤金煌怎么办?

%87
“她还只是个孩子!”这是白晴仙子的心声。

%88
几乎每一个父母眼中,自己的孩子永远是孩子。

%89
北原。

%90
黑城的无头尸体,还躺在烂泥地上。

%91
他的魂魄哀嚎着,在黑楼兰的手上挣脱不得。

%92
黑楼兰杀了黑城,踩爆了自己亲生父亲的脑袋,还不过瘾。现下,黑城的魂魄也被拘拿,将来必定会饱受黑楼兰的折磨和拷问。

%93
而焚天魔女则蹲到地上,催起一记仙道杀招。

%94
一只火红的小手,凭空出现,抓向黑城的腹部。

%95
火红小手十分轻易地融进黑城的尸体当中,捣鼓了一阵之后,再飞出来时,手中已经拿捏着一颗珠子。

%96
“这是黑城的仙窍,我将它暂时取出来。小兰,给你。它只能维持七天七夜的时间,时限一到,我的仙道杀招就会崩解,仙窍就会融入天地,形成福地。只是可惜仙窍中的蛊虫,不论仙凡,都已经全数毁去了。”

%97
焚天魔女说着,将这颗火焰珠子交到了黑楼兰的手中。

%98
黑楼兰沉默地接过。

%99
方源、黎山仙子都为焚天魔女的手段暗暗吃惊。

%100
焚天魔女笑道:“你们不要用这种眼光看我。呵呵呵,这种取窍之法,并非是我的本事。而是我在东海时,意外发现了空绝老仙的传承,学得了他的取窍法门,然后创出了这个炎道仙级杀招。”

%101
焚天魔女是极其罕见的大宗师。

%102
到了这个境界,很容易就触类旁通,手段极其全面、丰富。

%103
“好了,接下来该谈谈你的事情了,方源。”黎山仙子面向方源,笑容中有着冷意。

%104
ps:大家国庆快乐!感谢大家的月票支持,截止今天中午,月票超百,所以加更一张。今天双更,都在晚8点。是这样的,计算加更,都看当天中午12点时的月票数。这样我就有个明确的答案,如果看晚上的月票,可能会有浮动。总之不会欠更的,敬请放心。

\end{this_body}

