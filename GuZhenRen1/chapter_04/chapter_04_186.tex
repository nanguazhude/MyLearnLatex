\newsection{三件事情的约定}    %第一百八十六节:三件事情的约定

\begin{this_body}



%1
时间流逝,郑山川终于进行最后的几步,火团中蛊虫雏形已经越发清晰。

%2
保险起见,他再看一眼方源,只见对方刚刚结束那一步骤,将六种才俩炼化融汇一体。

%3
到此步,郑山川也不得不佩服方源的稳重:“可惜,这场胜利终究是我的。绿曜蛊,我拿定了!!”

%4
郑山川涌现出自信的光芒,但就在他收回目光的那一刻,方源忽然大笑:“哈哈哈,郑山川你败了,看我的炼道杀招!”

%5
“什么?”郑山川心头一颤,连忙又将目光投向方源。

%6
只见方源取出一只蛊虫,在他眼前捏碎。

%7
一瞬间,郑山川瞳眸缩成针尖大小,几乎要跳起来。

%8
他对着方源惊怒大吼:“你!你居然捏碎了绿曜蛊!!”

%9
话音刚落,他手中的火团不稳动荡,达到临界点,陡然发生爆炸。

%10
爆炸的威力并不大,但爆炸之后,烟灰将郑山川清秀的脸庞涂成黑色,仿佛煤炭,极为狼狈。

%11
郑山川愣在当场,宛如石像。

%12
如此惊变,只在兔起鹊落之间!

%13
场外众人静默了几个呼吸,这才反应过来,皆是哗然,声音嘈杂,掀起一股巨大的声浪。

%14
“好个心狠手辣的魔道蛊师!”

%15
“居然直接将绿曜蛊捏碎,让郑山川心境失守,炼蛊失败,功亏一篑了!”

%16
“这个方源的城府太深了,一下子就将局面彻底地翻转过来。郑山川失败,就得重新炼制。而方源的优势太大太大。就算是郑山川重演刚刚的炼蛊过程,估计也追不上方源了。”

%17
“更关键的是。他只是捏碎了自己的一只蛊,根本无法判他恶意干扰对方!”

%18
“糟糕。糟糕!他居然会如此做,竟然成了这样!”岐山老人这一刻也慌了神。

%19
郑山川受到的打击太大,太突然,好半天才缓过神来。

%20
等到他重拾心境,重头开始炼蛊时,方源已经步入最后步骤。

%21
这场赌斗再无悬念,最终以方源的获胜而告终。

%22
“精彩,实在是太精彩了。”众人事不关己,兴奋地评价着。

%23
“郑山川还是吃了年轻的亏。这下子把自己的一生都葬送了。唉……”有人为之可惜。

%24
当然也有人开心不已,比如安寒,不过他身为大供奉,名门正派,只能心里偷着乐,表面上则是唏嘘不已,对郑山川的遭遇表示同情。

%25
赌斗已经结束,方源缓缓站起,郑山川却呆呆地望着手中静静燃烧的火焰。

%26
他输了!

%27
把自己的一生。大好的前途,都在一场赌斗中输给了一位神秘的魔道蛊师。

%28
怎么会这样?该怎么办?

%29
郑山川顿时觉得前途一片晦暗,茫然至极。

%30
“前辈,请您开恩!我师徒二人有眼不识泰山。冒犯了前辈。请前辈大人大量,念在小徒还年轻,放过他吧。老朽愿意以身代之。更愿意奉上全部积蓄,换取小徒的自由。还请大人仁慈。手下留情,手下留情啊!”

%31
岐山老人忽的跪在地上。向方源磕头不休。

%32
老人的头撞在地砖上,发出砰砰的响声,不一会儿便是血流满脸。

%33
“师傅,师傅!”郑山川惊醒,连忙跑过去,搀扶岐山老人。

%34
但岐山老人此刻执拗无比,将自家爱徒一把推开,用尽全力磕头,哀嚎苦求。

%35
“师傅……”郑山川泪流满面,啪的一声,跪在地上,却是面向岐山老人。

%36
旁人观之,无不动容。

%37
飞霜阁阁主面露同情不忍之色。

%38
安寒心中冷笑不已。

%39
但是赌斗已成定局,众目睽睽之下,岂容悔改?除非方源肯放过他们,可是方源可是心狠手辣的魔道蛊修,并且郑山川年纪轻轻,却有炼道大师的造诣。方源怎么可能放过?

%40
就算是正道蛊师们,此刻换位思考,也没有放郑山川一马的心思。

%41
炼蛊大会狰狞残酷的一面,陡然展现在众人的面前。

%42
但下一刻,方源却对岐山老人道:“放过他,也不是不可以。”

%43
安寒冷笑顿止,场中为之一静。

%44
“什么?”郑山川呆愣。

%45
岐山老人最先反应过来,大喜过望:“谢前辈大人大量,谢前辈包容海涵!”

%46
方源呵呵一笑,语调很温和:“与人为善,就是与己为善嘛。”

%47
众人听了,都是一脸见鬼的神情。

%48
这话说的也太假了吧?刚刚是谁捏碎了绿曜蛊,让人家一个小年轻败北的啊?

%49
安寒神色惊愕,心中着急大吼:“喂,你还是不是魔道蛊师啊?快给老子心肠狠起来啊!你说的什么话,一点都不冷酷,也太丢魔道的脸面了吧?!”

%50
方源旋即话锋一转:“不过,你们俩既然冒犯了我,还是要受到惩罚的。”

%51
众人这才释然。

%52
安寒放下心来,心中高兴狂吼:“哦,原来如此,是想玩弄这对师徒的感情啊。哈哈哈,这才对嘛,这才是魔道风范嘛!”

%53
郑山川神色一紧。

%54
岐山老人又开始磕头了,口中连呼:“请大人怜悯,请大人怜悯!”

%55
“也罢,我也不叫你终身为奴了,只让你替我做三件事情。不过你现在似乎还没有资格,替我做事。至于这三件事情是什么,呵呵,我其实也没有想好。也许我终其一生,都不会让你做事。也许我下一刻,就让你连做三件事。这些都看我的心情了。”方源徐徐地道。

%56
他声音不高,但全场都十分安静,都在听他说话。

%57
岐山老人松了一口气,涕泪交流,高呼:“谢大人手下留情!”

%58
郑山川也叩首:“晚辈拜谢前辈。”

%59
只做三件事情和终生为奴相比。前者无疑令人接受要更容易得多。中洲还是以正道为主的,郑山川若是将来顶着一个魔道蛊师的奴隶身份。就算再有炼蛊的才情才华,也出不了头了。

%60
而做三件事情。只是一个约定。正道、魔道蛊师之间的约定,不算什么。郑山川的前途,仍旧是光明一片。

%61
看着脚下的师徒感恩戴德的样子,方源哈哈大笑,命令道:“你们两位咬破舌尖,将舌尖血都交给我一份。”

%62
岐山老人、郑山川对望一眼,心里惴惴不安地咬破舌尖,将血交给方源。

%63
方源伸出右手,五指虚握合拢。将两份血都捏在手心中。

%64
随后他缓缓动手,五指碾磨,竟然从指缝间逸散出一股黑烟。

%65
呛人的黑烟散去之后,方源手掌摊开,现出两只蛊虫。

%66
岐山老人、郑山川双眼瞪圆,这两只蛊虫赫然是一对四转蛊。

%67
全场惊叹。

%68
方源的实力居然如此强大,在几个呼吸之间,炼出了一对四转蛊。

%69
“这是血道炼法!居然有如此惊人的血道炼蛊法!”飞霜阁的高层们都是脸色铁青。

%70
就算是场外的魔道蛊师们,也流露出十分忌惮的目光。

%71
血道臭名昭著。比之魂道焚烧人魂还要恶劣!正道中人人喊打,甚至连魔道本身都多不容许。

%72
“这对四转蛊,能够相互感应。你们拿一只,我拿一只。将来若有事情要你们做。我会遣人带着这只血道蛊虫,当做信物过来的。两只血道蛊虫,能够相互感应。辨别真伪。就算将来你们死了,这三件事情的约定。也要延续到你们的子孙、徒弟等后辈身上。你们可愿意?若是不愿,就当我的奴隶吧。”方源道。

%73
“愿意。愿意。”岐山老人忙不迭地答应。

%74
方源手掌一抖,便将其中一只血道蛊虫,抛给了岐山老人。

%75
又抛给郑山川另外一只蛊。

%76
郑山川接到这只蛊时,整个人都傻了,结结巴巴地道:“这,这是绿曜蛊?”

%77
方源从容不迫地收起另一只血道蛊虫,哈哈大笑:“你该不会真以为,我捏死了绿曜蛊吧?”

%78
笑着,他留下一群呆滞的蛊师,转身离开。

%79
郑山川傻傻地望着方源离开,岐山老人则小心翼翼地收起血道蛊虫。他可不敢违背这个约定。皆因作证的,不仅是飞霜阁,而且还是背后的整个炼蛊大会。

%80
所有的比斗,包括这场比斗,都将记录在案。中洲十大古派,乃至参加炼蛊大会的全部势力,都为其作证,不容反悔。

%81
“师傅,你头上的伤要赶紧治疗。”郑山川站起身,担忧地搀扶起岐山老人。

%82
“走,快走。此地不可久留!”岐山老人被方源的手段震慑,真的是心惊胆寒了,连忙催促自家爱徒离开这处是非之地。

%83
经此变故,郑山川直接退赛,不再参加接下来的所有比试。

%84
不过他得到绿曜蛊,已经达到了原先的目的。

%85
事后,岐山老人借助此事,好好地给郑山川上了一课,向他阐述方源手段如何狠辣,简直是摸透了人性人心,又向郑山川点明飞霜阁大供奉安寒,如何嫉贤妒能,在暗中推波助澜的。

%86
郑山川经此一事,成熟了一大截,终于认识到自己的肤浅,以及江湖的险恶。之后的大半生,他一边辗转各地为岐山老人疗伤,一边提升炼蛊造诣,低调做人。五十三岁,他为岐山老人治好了毒伤。八十八岁时,他不声不响地达到了炼道宗师的成就。晚年的时候,他重新回到中洲东海岸,开创了一个小门派,取了自己姓名中的两个字,就叫做山川堂。

%87
他寿一百五十载,但终其一生再没有参加过炼蛊大会。

%88
至于那三件事情的约定,就是后话了。

\end{this_body}

