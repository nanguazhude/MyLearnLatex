\newsection{重点建设星象福地}    %第二百五十四节:重点建设星象福地

\begin{this_body}

%1
方源进入星象福地。

%2
星象福地经过这一顿搬迁,天地二气动荡,一场打地震随即在星象福地中产生。

%3
直到星象福地再次落下,才汲取了源源不断的地气,重新稳固下来,地震渐渐停歇。

%4
此时,方源放眼望去,只见广袤大地,裂痕满布。火焰燃烧,黑烟袅袅缭绕。

%5
方源心微微一沉,深深叹气。

%6
这是没有办法的事情。

%7
因为要仙窍移植,不能干扰到两套仙蛊阵的运转。所以,就算方源有仙蛊,有手段,也不能对星象福地进行保护和防御。

%8
这一点,琅琊地灵也曾经关照过方源,琅琊福地的损失,比方源的星象福地还要大。

%9
所以一般而言,福地是不会搬迁的。

%10
“地灵何在?”

%11
方源唤出地灵。

%12
星象地灵双眼通红,有点搞不懂主人为什么要搬家。

%13
在星象地灵的带领下,方源开始巡查星象福地。

%14
首先要查看的,当然是星象福地中的四大经济支柱:箭竹林、碎星湖、陨石天坑以及星屑草场。

%15
碎星湖中的湖水,已经顺着一条巨大的裂缝流淌出去,湖底众多的海藻趴在地上,最底部还残留了一些湖水,总算是没有彻底干涸。

%16
江山如故!

%17
方源取出仙蛊,将仙元灌注进去。

%18
江山如故仙蛊迸发出白色华光,扫过碎星湖一圈。

%19
顿时碎星湖中的土地弥合,湖水充盈。恢复到了先前的状态。只是海藻损失了不少,未来一段时间。对星河蛊的产出会有一些影响。

%20
湖水中空无一物,只有海藻海草在飘荡。原本的湖底宫殿。也已经彻底损毁。这种人造建筑,是江山如故仙蛊不能影响的事物。

%21
不过在这湖水里面,原本的鱼群、虫群、蛊群,并非丧生在地震当中。而是很早之前,就被方源提前搬迁到了狐仙福地里了。

%22
方源的仙窍充斥死地,活物存放在里面不能太久。

%23
之后,方源又去箭竹林、陨石天坑、星屑草场巡查。

%24
出乎方源的意料,箭竹林居然毫发无损。  在箭竹林的旁边,有两道漫长的沟壑裂缝。几乎和箭竹林擦肩而过,险之又险。

%25
箭竹林是星象福地中最大的豢养之地,方源之前已经将其中绝大多数的资源,都收集起来,存放到了狐仙福地里。

%26
没想到箭竹林居然躲过了这一劫,算是意外之喜了。

%27
而陨石群坑和几片星屑草场,就没有这么幸运了。陨石群坑损毁严重,几乎面目全非。而星屑草场更是因为地火升腾,一些都燃烧成草灰。放眼望去,一片灰烬。

%28
在江山如故的修复下,陨石群坑彻底恢复。星屑草场的地貌也全面复原。

%29
“江山如故仙蛊真的很好用,有了它。我的损失至少缩减了八成!又因为之前将大部分资源都调出去,原本伤筋动骨的星象福地,只是落了个轻微擦伤。”

%30
方源心中带着概况。陆续走了许多地方,利用江山如故修复了星象福地的地貌。耗费了不少青提仙元。

%31
也不是所有的地方都修复,方源故意留了一些裂缝。

%32
他突发灵感:“也许我可以趁着这个机会。将这些地面裂缝,扩张固定成河道,将来在星象福地中形成一个遍及全福地的河湖水系。不过星象福地地貌特殊,具体该如何实施,还得事先利用智慧光晕,思虑周详了。”

%33
接下来,方源利用定仙游,去往琅琊福地,将借来的仙蛊都还给了琅琊地灵。

%34
琅琊地灵将抵押的仙蛊,也都还给方源,并在临走前关照:“若是有搬迁需要,还可以来找我。”

%35
不得不说,他这一次赚大了,短短一个月不到,就有两千块仙元石入账。

%36
琅琊地灵又道:“你若加入我琅琊派,成为太上长老之一,将来搬迁福地,借蛊费用可以大大优惠!”

%37
“我再考虑考虑。”方源敷衍过去。

%38
他只想在琅琊地灵身上,榨取利益,从未想过和这些毛民绑在一条战车上。

%39
一旦加入,就无法抽身,失去自由。这种愚蠢的事情,他可不会干。

%40
随后,方源又利用定仙游,来到太白云生的仙窍里,将江山如故仙蛊还给太白云生。

%41
在仙窍中,方源向太白云生打听东海的情报。

%42
太白云生便告诉他,宋亦诗等人正在满世界的搜寻他,并且已经有人因此找上了太白云生。

%43
“按照看来再过一段时间,你我之间的关系也会彻底曝光了。”方源沉吟道。

%44
“我建议你现在直接联络鲨魔,加入东海僵盟。然后依靠僵盟,解决这个事情。”太白云生道。

%45
方源却不着急,他没有被外在压力动摇计划,摇摇头:“时机还差了一点,距离下一次攻略玉露福地还有十多天,再等等看。”

%46
狐仙福地,数十天后。

%47
地下洞窟。

%48
沐浴在智慧光晕中的方源,缓缓地睁开双眼,吐出一口浊气。

%49
“还是失败了么。”他口中喃喃,慢慢站起身来。

%50
他原先是坐着的,直接坐在一株矮小的芝林上面,将其当做板凳。

%51
不过这种低矮的芝林,已经很稀少了。

%52
此时的地下洞窟中长满了芝林。这些芝林已经和原先的大不一样,不仅又高又壮,而且通体血红色泽。原来是因为狐仙福地的第七场血海棠地灾之后,撞到了那份几率,从普通的芝林,转变成了血芝林。

%53
狐仙福地的土壤,原本并不适合芝林的生存。但是转变成血芝林后,却是如鱼得水。

%54
不过这片血芝林要慢慢扩张下去,挤开泥石,渐渐形成广袤的地下森林,至少需要一两百年的时间。

%55
方源对这片血芝林又没有什么期待,所以也不会在这上面投资。

%56
这些天来,方源不眠不休,大力建设星象福地。

%57
一旦身份暴露,世人皆知他就是八十八角真阳楼倒塌的罪魁祸首,狐仙福地必定要舍弃的。星象福地就成了唯一的基地了。

%58
若在那时,方源还没有挣脱仙僵身份,回复人生,星象福地对于方源的价值和意义将更为重大。

%59
所以,方源的重心已经从狐仙福地,转移到了星象福地。

%60
关于星象福地的建设规划,方源早在数十天前,就利用智慧光晕,琢磨好了。

%61
按照他之前的灵感,在星象福地上铺设出一套巨大的河流水系。推算之后的这个设想,却更加宏大。整个水系河流,不仅是地面上,而且囊括天空,形成一套巨大的立体河流。

%62
这正好利用到了星象福地特有的地形。

%63
星象福地外凸内凹,外边地势高,越到中央地势越低,呈现一个碗状。

%64
若仅仅只是地面上铺设河流水系,就会造成水流往福地中央堆积。但若是立体河流,就可将中央的水抽取到天空中,再流淌过后,落入福地边缘,形成水流的壮阔大循环。

%65
至于在空中,如何承载河流。方源已经推算出了一个最为物廉价美的方法,那就是云土。

%66
云土也被琅琊地灵利用,在琅琊福地中硬生生建设出了一个云盖大陆。

%67
方源打算用云土,建造出承载河流的条条河道,河道两岸肥沃的云土,将栽种无数的星屑草。

%68
星屑草最佳的土壤,就是云土。万象星君栽种在福地黑泥中,也是迫不得已,他没有财力收购大规模的云土。但是星象福地的环境,十分适合星屑草的生长。导致星象福地中星屑草的长势,比狐仙福地中要好得多!

%69
方源的设想是庞大而美好的,但是现实还是颇具骨感。

%70
方源的资金缺口非常大,刚刚积累的两千仙元石,都给了琅琊地灵。所以这些天,他只能向将星象福地中的四大资源点,都修整好。

%71
箭竹林在大量月井水的灌溉下,已经重现薄雾笼罩的最佳状态。

%72
陨石群坑和碎星湖,都完全恢复。

%73
星屑草场麻烦一点,但经过方源几番春星雨杀招的灌溉,也变得长势可人。

%74
自从继承了星道传承,拥有了星芽仙蛊之后,春星雨杀招已经转变成了仙道杀招,效果比之前好了数十倍!

%75
至于狐仙福地中的龙鱼、幽火龙蟒、长恨蛛、荡魂山,暂时都未搬迁。

%76
一来,需要前期的建设,将河流水系初步贯通后,搬迁这些龙鱼等等最好不过。

%77
二来,星象福地面临着天劫!

%78
万象星君为什么这么着急探索繁星洞天呢?

%79
就是因为天劫逼近。

%80
他是六转一阶的蛊仙,面临的是第二场天劫。

%81
因为他和宋紫星一战,伤了根本,失去了渡劫的信心,只好不惜将繁星洞天的秘密暴露出来,拉石磊来共同探险繁星洞天。

%82
方源现在还在犹豫,是否在天劫来临前,将这些资源都搬迁过去。

%83
从地下洞窟中出来,方源回到荡魂行宫。

%84
被他一直关押着的古月方正,像是认命了一般,双眼无光,瘫坐在牢房中,一动不动。

%85
显然,仙鹤门杀害他的真相,师傅天鹤上人原先要夺舍他的内幕,对他而言是个十分沉重的打击。

%86
按照惯例,方源站在他面前,说了几句话。

%87
古月方正一动不动,神色麻木,似乎根本没有听进去。

\end{this_body}

