\newsection{提前加入义天山}    %第三百五十节:提前加入义天山

\begin{this_body}

%1
南疆,义天山。

%2
方源赤着双脚,大步前行。

%3
他的形态容貌,已然大变。浑身肌肉贲发,袒露胸膛,头发、胡须、胸毛乃至腿脚上的毛发都分外浓密。

%4
这些毛发都是土黄之色,一身衣服近乎破衣烂衫。

%5
脸上颧骨凸出,手指头、脚趾头的关节处,骨节也超越常人的粗大。

%6
鼻梁又短又矮,两个大鼻孔,向外冲着,里面的鼻毛同样浓密,争先恐后一般,从鼻孔内向外喷涌似的。

%7
别看这副尊荣土不拉几,丑陋不堪,却是方源结合前世记忆,精挑细选而来。

%8
方源上一世,这副尊荣的原主人,就在这个时候加入了义天山。

%9
但方源重生之后,就秘密动手,将这人悄然斩杀。

%10
方源顶替他,来到义天山。

%11
朝阳刚刚升起,微有薄雾。

%12
义天山还很清静,山林间时不时地传出鸟鸣之声。

%13
这个时候的正魔大战,连第一波交锋都未开始。

%14
前世方源伪装身份,加入义天山时,已经是正魔交锋如火如荼的时候,连山脚下都有魔道的喽啰把守。

%15
但这一世,方源提前了数月,就现身于此。

%16
这个时候,萧山才刚刚被萧家流放,他和周星星、孙胖虎二人一起,创建义天寨不久。

%17
方源走过山脚,都未见着什么人烟。

%18
攀越了半山腰之后,他终于透过郁郁葱葱的树林间隙,目睹到义天寨的一丝景貌。

%19
义天寨还在建设当中,并未建成呢。

%20
“此人是谁?会是哪位蛊仙的棋子?”

%21
“好像不是纯正的人族,似乎有毛民的血统。”

%22
“哼,这个杂种似的东西,恐怕不会有人来选作棋子吧?呵呵呵。”

%23
远处,南疆的蛊仙们就方源相互议论着。

%24
方源安步当车,心底自信而又从容。

%25
前世,他动用见面曾相识。都未曾被这些南疆蛊仙们识破。今生,他是采用了以态度蛊为核心的仙道杀招见面曾相识,这要是被识破,那就真见鬼了!

%26
他昂首阔步。朝山峰走去。

%27
片刻之后,他接近义天寨的时候,终于被人拦下。

%28
来者是一位三转魔道蛊师。但方源伪装的这个人物,也是三转蛊师,所以他不敢大意。凝声问道:“你是何人?”

%29
方源便抱拳,粗鲁地道:“俺就是黄沙,听说了萧大侠的事情,就来投奔你们了。”

%30
对面的三转魔道蛊师,身躯一震。

%31
倒并非被“黄沙”的名头震到了,而是方源的嗓门太大。

%32
“好了,低点声,扯什么嗓子。既然你识得大头领的名号,那就随我去拜见大头领罢。”三转蛊师掏掏耳朵,转身就走。

%33
方源嘿了一声。埋头赶上。

%34
他体格高于常人,步伐很大,三步并两步,就赶超了前者。

%35
前面的魔道蛊师顿时不悦,伸出手来,将方源拉住:“你跑这么快干什么?你要加入义天寨,就要懂规矩。知道不?先来后到,我的排位可比你高!跟我后面走!”

%36
“哦,哦。”方源连忙点头,一副体格粗大。脑子很小的样子。

%37
“这处的建筑,一定要重点下功夫建设。将来若是有人攻打过来,我们就要依凭这里据守。至少要铺设上百道的铁蛇藤蛊。”萧山手指着某处,关照身旁的蛊师。

%38
这时。听到有人高喊:“大头领,你的威名遍布四海,传遍南疆!这不,又有一个好汉加入我们来了。”

%39
萧山闻声,心头一喜,转身看去。看到方源。

%40
心中的喜悦顿时消散了些,微微失望。不过同时,他的脸色却浮现出明显的欣喜的笑容。

%41
他快步地走上前去,拍拍方源的肩膀:“好个雄壮的汉子!”

%42
方源哈哈大笑,抱拳道:“你就是萧大侠吗?我是来投奔你的,你带种!居然敢和那些正道对着干!”

%43
说着,方源对萧山直接竖起大拇指,又道:“就凭这个,俺就服你,愿意跟你,但你每天三顿饭,都要管俺饱。”

%44
萧山见方源粗鲁不堪,心中的失望更盛。

%45
不过他表面上,丝毫不表现出来,夸奖方源几句,然后当场安排他一个任务。

%46
方源走后,萧山又招来周星星,问道:“这个黄沙,是什么来路?我却不太清楚。”

%47
周星星想了想,笑道:“大哥你是什么层次的人物,平日里云来雾去,这种小人物如何入你的眼界?此人小弟恰巧知道,他父亲是人,母亲是毛民,出身时就是个沙山上采沙石的奴隶。结果在沙山上,意外得了一道传承,成为蛊师。后来一段时间,和一位水道蛊师占据一段江面,并称为‘白沙二将’。后来被铁家蛊师击败,白将丧命,这个黄沙却是逃出去,下落不明。没想到他也仰慕大哥你的威名,赶来加入义天寨!”

%48
“原来如此。我记起来了,的确曾经有段时间,有过‘白沙二将’袭击铁家商船的消息。”萧山点点头,心里对方源的期待,降至谷底。

%49
连铁家商船都敢动,可见这白沙二将,都是鲁莽粗犷,不动脑子。

%50
再加上这个黄沙的身上,居然还有毛民的血统,更加让萧山看不起。

%51
诸如毛民、羽民、雪人这些异人,都是人族的奴隶,人族蛊师又岂会平等对待这些人?

%52
若方源的修为,有个四转、五转,说不定萧山还能打破成见,重视方源。但方源伪装的这个黄沙,只是三转。

%53
虽然三转的地位,已经高于一般的蛊师了。

%54
但在这个义天寨中,却并不起眼。

%55
所以萧山很快,就将这个黄沙抛之脑后了。

%56
方源在工地里干活。

%57
“如今,我已经加入了义天山。又被萧山打发过来,建设义天寨,可见此人并未将我放在眼里。很好,这正是我想要达到的目的。目前为止,一切都很顺利。”

%58
方源若是修为更高,被萧山重视,接下来的正魔大战,就必定有重任缠身。不去执行,肯定不行,会露出破绽。去执行的话,又浪费时间。

%59
若是修为低一点,则会被沦为炮灰,被安排在前线,吸引火力,浪费敌人的真元。几次正魔交锋之后,就会牺牲了。

%60
唯有三转修为,高不高,低不低,算是小头目。

%61
不会被委以重任,战场上活下来,也不会叫人感到奇怪。

%62
义天山上的禁仙绝境,只是针对仙窍,并不针对仙蛊。

%63
前世这个时候,方源还在推演法门,如何封印仙窍。但今生,他直接掌握了此法,提前数月,加入义天寨。

%64
他虽然仙窍被封印,但身躯还是仙僵。

%65
白天,在工地上,方源干些体力活,轻轻松松。晚上,其他人都呼呼大睡,他仍旧精神矍铄,就暗中炼化山底下的惊鸿乱斗台。

%66
惊鸿乱斗台镇压着大力真武仙僵,牢不可破。

%67
这头八转大力真武仙僵,毫无魂魄的波动。看样子,就像是魂魄彻底消亡了。

%68
但方源结合前世记忆,却不敢丝毫大意。

%69
他没有对大力真武仙僵有任何试探之举,而是一门心思,转化战意。

%70
参与这场旷世赌斗的南疆蛊仙们,要将这些战意转为自己所有,须得借助这些凡人蛊师棋子,让他们不断生死激战,才能以战意呼应战意。将仙蛊屋中的纯净战意,逐渐转化成该蛊师的个人战意。

%71
但方源亲身降临,又是智道宗师,就无需如此了。

%72
他就算不去激斗,也可以激增脑海中的战意,将仙蛊屋中的战意不断转化,效率是寻常蛊仙的数倍。

%73
只要方源将这些仙蛊屋中的战意,都转化成自家的战意。那么他就会完成当年八转仙僵未完成的最后一步,也就是炼化仙蛊屋,成为仙蛊屋的真正主人!

%74
所以,方源要在这些南疆蛊仙的虎口中,成功夺取惊鸿乱斗台,还有一个关隘。

%75
就是到了最后关头,仙蛊屋中的纯净战意,都被众人瓜分转化。

%76
但成功者只有一人。

%77
所以这些战意之间,还要相互角逐,进行一场惨烈的激斗。

%78
这场激斗,任何的失败者,都会烟消云散,成功者有也只会有唯一的一位。

%79
“我是智道宗师,战意的比拼,我有巨大优势。况且接下来,我必定会一马当先,以超出其他人数倍的速度转化战意。战意转化越多,将来战意决战,我就越有优势。只要按照计划,我夺得仙蛊屋,已是毫无悬念!就怕出什么意外……”

%80
方源心中思量的同时,其余的南疆蛊仙们也是各怀心思。

%81
萧家的太上家老,时刻保持着对义天山战场的高度关注。

%82
他上一次渡劫,勉强生还。而下场灾劫已经逼近,本来他已经毫无希望,但仙蛊屋惊鸿乱斗台的出现,却让这位蛊仙在黑暗中看到一丝光明。

%83
所以,这一次旷世的赌斗,他孤注一掷。

%84
几乎将所有的身家,都押了进去。

%85
他的赌注位居众仙之首,按照赌斗的规矩,他选择的棋子之一萧山,就成了第一个登上义天山的蛊师。

%86
萧家太上大长老,也成了南疆蛊仙当中,最先转化仙蛊屋中战意的第一人。

%87
“这一次仙蛊屋的争夺,我必须成功,不能失败!”

%88
“若是没有仙蛊屋,接下来的这场灾劫,我万无幸理。”

\end{this_body}

