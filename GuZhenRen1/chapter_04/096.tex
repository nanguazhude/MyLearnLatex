\newsection{仙道残招}    %第九十六节:仙道残招

\begin{this_body}

%1
方源打算将十三具眠云棺椁,直接一同带回了狐仙福地。

%2
因为签订了山盟,琅琊地灵不仅没有阻挠,而且还主动附赠了方源八个凡道杀招。

%3
“这些杀招,就是解除仙道杀招眠云棺椁的‘钥匙’。记得要按照我给你的这个顺序施展,如此方能成功解除棺椁,否则的话,会连同里面的蛊仙俘虏一同毁灭。”方源临走前,琅琊地灵叮嘱道。

%4
方源点点头,将这些凡道杀招牢记。这些杀招连名字都没有,唯一的作用,就是充当眠云棺椁的打开钥匙。

%5
不过对于云道蛊仙而言,或许也有些借鉴价值。

%6
太白云生就兼修云道,方源当即就打算将这些凡道杀招,交给太白云生看看。

%7
方源带着十三位蛊仙俘虏,顺利地回到狐仙福地。

%8
他先是招来狐仙地灵,问了一些最新的情报。

%9
狐仙福地一切安好,方源之前关照过,狐仙地灵一直在稳步操作。一步步地将福地中的石人,贩卖出去,同时又引进一些狐群。

%10
狐仙本尊,便是操作狐群作战的奴道蛊仙。她的仙窍狐仙福地,也最适合的豢养狐群。按照这种路子走下去,豢养出荒兽级别的狐群,收益巨大。比方源卖出两三批的胆识蛊,赚取的利润还要多得多。

%11
不过此法,见效很慢,需要长久的坚持。

%12
狐仙是散修,当时处境不堪。几次地灾之后,经济崩坏,最终损失得不到补充。战力跟着下降,死于地灾魅蓝电影之手。

%13
蛊仙仙窍,每隔一段时间的灾劫,就像是一道道难关,为五域大世界牢牢限制着蛊仙的数量。

%14
还有一道巨大枷锁,就是寿元的限制。

%15
狐仙当时,有荡魂山在手。自然就想利用这个独一无二的资源,来谋利经营。

%16
她这个做法,也是明智之举。虽然豢养狐群。是狐仙福地的经营正道,但见效太慢了。时间越长,发生的意外就会越多。况且狐仙处境每况愈下,已经不容许她慢慢发展。

%17
然而她手中并无智慧蛊。想不出气囊蛊的类似法子。只能利用胆识蛊,间接地增长石人的规模,再贩卖石人。

%18
这个法子,方源也曾经用过,但收益不多,十分麻烦。石人市场早已经饱和,哪里有直接贩卖胆识蛊来的既轻松又火爆?

%19
最终,狐仙失败。死于地灾。她留下的财富,便宜了方源。在关键时刻。提供给方源难以替代的巨大帮助。

%20
狐仙福地可谓是方源的起家之姿,龙兴之地。前世方源因为袭杀凤金煌,毁掉了狐仙福地。今生,则成了狐仙福地的主人。这前后对比也算是奇妙。

%21
现在,方源既然能直接贩卖胆识蛊了,手中又有余钱,再不拮据,自然就要重新利用狐仙福地,走上就连原来主人狐仙都没来得及走的“正道”,慢慢积蓄狐群,以期豢养出荒兽狐狸来。

%22
这项经营,是长久之计,方源也不着急,更没有过多期待当中的利润。

%23
关注了狐仙福地的情况之后,方源又重点关注了仙鹤门、战仙宗的最新动向。

%24
狐仙福地名义上,是仙鹤门的附庸势力,仙鹤门还是要多关注一些的。

%25
至于仙鹤门中的亲弟弟古月方正,方源早已经抛之脑后。

%26
区区一个凡人蛊师,哪怕是五转,又能如何?就相当于曾经在青茅山上,只要不挡住我的路,踩都不屑踩。

%27
没空。

%28
凡人如蚁,耗费精力时间去踩,也没有必要。

%29
“我那个亲弟弟,恐怕已经死了吧?”方源心中闪过这样的猜测。

%30
之前,方正借助同感仙蛊,和方源达成沟通。令方源单方面获知了更多的信息,知道方正的处境,“看”到了方正置身血池,浑身长满妖花的惨象。

%31
最近这段时间,仙鹤门的情况马马虎虎。他们顺利解决了钧天剑派的问题。

%32
钧天剑派新出了第三位蛊仙,准备摆脱仙鹤门的附庸身份。仙鹤门乃是名门正派,纵然实力远远超过钧天剑派,但正道规矩在,无法强行动手,只好按照招揽钧天剑派的蛊仙。

%33
最终他们成功地招揽了钧天剑派的蛊仙,就是第三位新升之人。

%34
钧天剑派掌握的资源有限,供给两位蛊仙已是勉强,再添第三位,就更加捉襟见肘。

%35
新升的蛊仙被仙鹤门开出的条件诱惑,加入了仙鹤门。

%36
钧天剑派没有了关键人物,实力下降,再无底气脱离仙鹤门,只好续签了附庸契约。

%37
不过仙鹤门虽然稳定了西北局势,但是在轮回战场上,又遭受打压。镇守那里的蛊仙雷坦,因为身受重伤,不得不回归门派休养。

%38
轮回战场乃是世间公认的第一战场杀招,由乐土仙尊亲自布下。十大古派为占据此地,已经激烈竞争了无数年。

%39
仙鹤门的注意力,再次集中到轮回战场上面,短时间之内,无法对狐仙福地动什么歪脑筋了。

%40
而一直负责狐仙福地攻略的鹤风扬,最近也在忙着找寻方法,治疗他的爱兽九宫鹤小九。

%41
这无疑对方源是个好消息,能令其安心地参加拍卖大会,而无需太过顾虑中洲方面。

%42
方源又重点关注了一下战仙宗。

%43
从战仙宗的几个无关紧要的消息中,方源嗅到了不寻常的气息。

%44
这些消息,还能蒙骗住其他九大古派一段时间,但对于熟知内情的方源来讲,却是洞若观火。

%45
“看来战仙宗果然秘密调集了蛊仙战力,前去攻略繁星洞天了。”方源心道。

%46
他此时还不知道,自从他走后,仙猴王石磊一直被困在繁星洞天当中。战仙宗派遣过去的三位蛊仙援兵,也在进入繁星洞天的时候,遇到了细微阻碍。

%47
“这样一来,繁星洞天中的梦境,就暴露在战仙宗的眼里了。事关星宿仙尊的梦境,又和大梦仙尊的预言有所牵扯,战仙宗一定会偷偷地大力攻略此洞天。我得抓紧时间,重生的优势正在一步步的缩减。”

%48
方源心中又增添一份紧迫感。

%49
自从他因为胆识蛊买卖,和中洲各大小势力都混了脸熟,收集情报比之前容易了许多倍。

%50
不过这些情报,只能流于泛泛,还谈不上重大秘要。比不上从黎山仙子处,得来的北原情报的质量。

%51
方源对此,也没有过多在意。

%52
很多蛊仙都极其注重情报的收集,每年投入大量的资金,用来攫取关键情报,维护获取情报的渠道。

%53
但方源不需要。

%54
他为此省下一大笔开销。皆因他有重生优势,前世记忆中的许多情报,比这些蛊仙辛辛苦苦,耗费大代价得来的,要重要无数倍,详实无数倍。

%55
确定了狐仙福地周围的局势环境,方源放下心来。他又看了看太白云生的来信,又关注了一下黑楼兰方面的情况。

%56
太白云生目前正在和鲨魔等人,共同探索玉露福地。江山如故仙蛊,彻底修复了玉露福地的门户,叫太白云生收获了不少好感,成功融入了探索的队伍。

%57
至于黑楼兰,最近则在渡劫。

%58
蛊仙每隔一段时间,都会遇到灾劫。六转蛊仙每隔十年,就会遇到一次地灾。

%59
十年这个时间,不是五域时间,而是针对各个蛊仙仙窍的具体时间。

%60
因为每个仙窍,引入的光阴支流规模不同,和外界的时间流速比率也不尽相同。

%61
黑楼兰是大力真武体升仙,成就特等福地,时间流速比率极高,比身为宙道蛊仙的太白云生还要高出一些,达到一比三十八。

%62
也就是说,五域时间一年,黑楼兰的福地就过了三十八年。其中要经过三次地灾,距离第四次只剩下两年时间。

%63
距离王庭福地毁灭,五域时间已经过去了大半年。

%64
黑楼兰的福地,已经渡过了两次地灾,如今正在渡第三次。

%65
每一次渡劫,都是一道难关。不过黑楼兰有黎山仙子帮衬,又有数只仙蛊在手,若是遇到难题,想来会像方源求援。现在那边没有消息,那就是安好的意思。

%66
确定了身边之人的情况后,方源再无顾及,重新投入到自身的谋划中来。

%67
他算了算,距离拍卖大会,只剩下半个月不到的时间。放到狐仙福地中,因为时间流速有五倍差距,就是两个半月少一点。

%68
胆识蛊买卖,还是要继续的。

%69
继续的同时,方源打算利用智慧光晕,尝试一下之前那个冒出来的灵感。

%70
时间匆匆,狐仙福地时间的两个月之后。

%71
方源带着一脸的满意之色,从地底洞窟中出来。

%72
他此番利用智慧光晕推算,大有收获。成功地融合了毒气吐纳以及包藏祸心的部分内容,从而形成一记全新的仙道残招,方源取名为毒气喷吐。

%73
以仙蛊妇人心为核心,自然冠名为“仙道”。但杀招并不完全,每次催动起来,都会有大量的毒气残留在体内,造成残毒的后遗症。因此只能算是“残招”。

%74
“凭我仙僵身躯,能够连续催动三次毒气喷吐。超过三次,身上的残留毒气就会将我的魂魄都毒杀。两道杀招,分别就是智道、毒道的境界具现,是坚实的基础。而智慧光晕带来的无相灵感,则让我充分利用这些基础,开出花来。”

%75
方源蹭用智慧光晕的次数,已经很多次了。

%76
用的次数越多,他就越感觉到智慧光晕带来的便利,越发感慨智慧蛊的无上威能妙用。

\end{this_body}

