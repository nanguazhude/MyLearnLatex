\newsection{僵持}    %第一百二十八节:僵持

\begin{this_body}

这件事,给方源提了一个醒。txt全集下载www.80txt.com

即便是地灵,也不容小觑,也会随着时间不断成长,增长智慧或者经验。

自从琅琊地灵碰见方源之后,很多次都被方源讨得便宜,尤其是这次拍卖大会,六四分的利益抽成,更是琅琊地灵吃了一记大亏。

俗话说吃亏是福,这点体现在琅琊地灵的身上,便是变得聪明了。

琅琊地灵痛定思痛,重新反思他和方源之间的关系,终于认定彼此之间的优势弱点,从而卡住蛊方这块儿,企图重新在和方源的合作中占据上风。

到底是长毛老祖死后,遗留的执念所变。方源这次还真的被琅琊地灵卡住了脖子,难受得很。

偏偏方源还不能拿琅琊福地怎么样。

便宜不是那么好占的,方源在拍卖会上贩卖蛊仙奴隶,获取了那么大的好处,现在后遗症终于爆发出来,满嘴都是苦味。

缓缓降落到荡魂山之巅,方源收起背上的三对蝠翼,一时间站在原地,遥望远方,若有所思。

“全力以赴蛊还只是小小挫折,毕竟不是仙蛊方,因而就算得不到,凭我的力道境界和智慧蛊的帮助,也能推导出凡蛊蛊方来。只是耗费时间、精力很多,这个时节暗流汹涌,一门心思推算蛊方,对战力提升不大,此举很不妥当。”

其实就算方源手中,有一道十成的全力以赴蛊仙蛊方,方源也无暇炼蛊。

真正的麻烦,在黑楼兰的身上。

方源救醒她的报酬,便是我力仙蛊。

黑楼兰从黎山仙子处得知之后,当场气得从床榻上跳起来。我力仙蛊乃是她的本命仙蛊,更是她的杀招我力虚影的第一核心!

方源索要此蛊,简直就是挖她的根基。失去我力仙蛊,黑楼兰的战力将一落千丈。不仅如此,涉及我力蛊,更是她的母亲苏仙儿留给黑楼兰的纪念之物。对黑楼兰而言,是感情上万万不能割舍之物。

但方源和她、黎山仙子有着雪山盟约,当初救助黑楼兰之前,黎山仙子也代表黑楼兰和方源签下协约,不能违反。

除非黑楼兰狠下心来,不顾及黎山仙子的性命,死命扣下我力仙蛊。这种情况下,方源也拿她没有办法。毕竟当初签下合约,不是黑楼兰她本人。

但黑楼兰又怎么忍心黎山仙子出事?

黎山仙子和她母亲苏仙儿是亲姐妹,黑楼兰还是凡人时,就受着黎山仙子的照顾。待她成仙之后,黎山仙子更是不计成本,大力培养黑楼兰,视如己出。

黑楼兰矢志复仇的同时,对黎山仙子这位亲人,却是饱含敬爱感激之情。

方源也正是拿捏着这一重点,才和黎山仙子达成协约,不担心事后黑楼兰反悔。

然而事关本命仙蛊,黑楼兰自然不肯轻易认命。

自从她得知情况之后,十数次和方源谈判,威逼咆哮、软语请求、陈说利弊,以及拿大局要挟,用尽种种手段,都是想挽回我力仙蛊。

但方源怎么可能松口?

我力仙蛊对于黑楼兰而言是第一核心,对于方源而言,几乎同样如此。

方源的杀手锏,是仙道杀招万我。我力仙蛊也是最为契合的,不像是拔山、挽澜还需要推算改善仙道杀招。

黑楼兰凭借升仙,提炼我力仙蛊。方源当初得到这个消息时,不知道多么遗憾和惋惜。

仙蛊唯一,黑楼兰手中有此蛊,又身为盟友,等若直接断绝了方源获取的可能。

气运无常,天可怜见,黑楼兰一门心思复仇,拼尽全力提升修为,结果大意之下,陷入梦境,终于给方源逮到良机。

方源腹黑阴险,像是一头闻到血腥气味的鲨鱼,当即一口狠狠咬下。

就算是盟友,又如何?自身的实力才是最重要的。别人都靠不住,唯有自己最为可靠。

方源态度坚决,黑楼兰屡屡尝试,皆吃瘪失败。气极之下,便故意扣着方源的我力仙蛊不给,同时不再出面为方源催动力气仙蛊炼制气囊蛊。

黑楼兰目光毒辣,此举真的带给方源很大压力。

黑楼兰不想催动力气仙蛊,方源就没有气囊蛊。没有气囊蛊,便不能向外出售胆识蛊。

方源之前为了疗伤,停止了胆识蛊买卖,专门供给自身。没有了这项进益,两座石巢又不断炼蛊,如今手中资金几乎干涸,仙元石已经跌落一百关口。

要想突破黑楼兰的限制,方源可以凭借智慧蛊,推算出另外的气囊蛊蛊方。

这方面,又有不小的麻烦。

方源在拍卖会上,已经将乐山乐水仙蛊卖了出去。用来替代的恶念蛊,数量上还不够多。尽管有一整座石巢的毛民,日夜不停地赶炼恶念蛊,但奈何智慧光晕之下,念头消耗的速度实在太快。

这情况方源也早有预料,原先的解决方法就是扩大生产规模。

但扩大规模,建造第三座,甚至第四座炼蛊石巢,需要资金。毛民,尤其是擅长炼蛊的毛民奴隶,要价更高。

方源资金不足,难以扩大生产。难题兜兜转转,又绕了回来,好似打了一个死结。

方源便找琅琊地灵,想用仙材利诱,令琅琊福地中的毛民出力,炼制恶念蛊。

但琅琊地灵竟然再次拒绝。

提出的要求,仍旧是杀上太古紫天,寻找其中一种六转毒花,并搜集花瓣。

看来这种仙材,对于琅琊地灵分外重要。

于是方源这条路也走不通了。

气囊蛊炼制受阻,就没有胆识蛊贸易。胆识蛊买卖,是方源目前的经济支柱。没有资金,就没有恶念蛊,方源也就无法利用智慧光晕。

这一切都大大影响了方源消化成果,迅速提升战力的计划。

方源当然更加心知肚明,这是黑楼兰在逼迫他逼他放弃我力仙蛊,选择力气仙蛊。

“哼,黑楼兰……我倒要看看,谁能熬得过谁?”方源冷笑一声,收起思绪。他唤出狐仙地灵,再次回到荡魂行宫。

盘坐在床榻上,他缓缓闭上双眼,渐渐陷入梦境。

之前救醒黑楼兰,方源使用了仙道杀招解梦,损耗了不少梦道凡蛊。为了第二次能够催动解梦杀招,他仍旧需要炼制再去梦中,炼制梦道凡蛊。

与此同时,北原某处隐秘之所。

来自中洲十大古派之一古魂门的智道蛊仙老算子,从闭关的密室中步履蹒跚地走出来。

他一脸疲惫神色,全力推算真阳楼倒塌一案的凶手,使得他精力憔悴。两鬓之间平添许多白发,原本年轻的面庞上也多出许多沧桑的皱纹。

“如何?”赤洪明等仙见他出来,立即围拢过去,神情关切。

老算子当然知道,他们关心的不是自己,而是此次推算的成果。

这一次调查,得出许多的关键证据,尤其是搜魂赵怜云,重现了当初在王庭福地中发生的情形,为推演带来巨大帮助。

但老算子摇摇头,一脸落寞地道:“我造诣浅薄,让诸位失望了。”

“怎么会?”

“这还推算不出么,凶手大不简单,隐藏得实在太深了……”

“看来还得需要收集情报啊。”

众仙都是门派精英,失望了一阵后,迅速接受这个结果,开始设想接下来的行动计划。

一番议论之后,凤九歌开口道:“如今没有明确的目标,不妨先解散开来,各自行动。诸位从中洲来到北原一次,很不容易。相信大家身上,不仅有查明真相的主要任务,还有各自门派的私事。接下来这段时间,请诸位处理私事的时候,多多采集情报。”

凤九歌威信第一,没有人不信服。

关键是此时众人没有方向,也的确身有要事,便一一散去,暂时离开这里,处理自家门派的任务去了。

老算子和凤九歌留在最后。

“九歌大人。”老算子欲言又止。

事实上,他此次推算已经得到重大突破,但告知凤九歌之后,却被后者下令封锁消息。于是就有了刚刚,老算子向众仙谎称,推算失败的一幕。

凤九歌看着他,意味深长地一笑:“八十八角真阳楼倒塌,最有可能的凶手便是中洲十大派。如今你推算得出凶手便是中洲十派中人的结果,我便让他们自由行动,一旦有什么马脚漏出来,就可能查出个水落石出。”

“可是在下的推算结果,也有可能本就是错误的。”老算子低头道。

“放心吧。若真是十派中出了内奸,犯下如此大案,不管如何扫尾,一定会留下蛛丝马迹的。”凤九歌似宽慰似警告地道。

老算子抬起头,目光清澈有神,盯着凤九歌:“九歌大人瞒着其余蛊仙,就不怀疑在下的门派吗?”

凤九歌哈哈一笑,气度不凡地坦言道:“实话实说,我当然早已怀疑了。你自去吧,相信古魂门也有隐秘任务给你的罢。”

凤九歌的言语中,充满了掌控局面的自信。

老算子心中一凛,颔首道:“的确如此,那么在下这就去了。”

说完,身形如电,眨眼消失在天边。

只剩下凤九歌立在原地,双目幽幽,神色沉静,不知道在想些什么。

\end{this_body}

