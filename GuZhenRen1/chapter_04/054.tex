\newsection{局中局,致命的杀招}    %第五十四节:局中局,致命的杀招

\begin{this_body}

%1
战场中,激战继续。

%2
一大群铁喙飞鹤,各个浑身漆黑似铁,防御暴涨,无畏生死,向两位神秘蛊仙狠狠压来。

%3
神秘蛊仙之一,忽然张口尖啸,啸声高亢嘹亮,回荡一方。

%4
这是一记音道杀招!

%5
铁喙飞鹤群在啸声中,阵型大乱,原本密实的战阵变得稀疏,呈现出许多破绽。

%6
“走!”另一位神秘蛊仙见机,拉住另一人的胳膊,两人突然化作两团灰白的火焰,穿透铁喙飞鹤的战阵,向鹤风扬迅速袭来。

%7
对付奴道蛊仙,就不能与兽群太做激斗,擒贼先擒王才是最正确的战术。

%8
鹤风扬看到强敌扑来,不仅没有慌张,反而流露出一丝冷笑:“你们中计了。”

%9
他取出一只五孔玉箫蛊,凑在自己嘴边吹奏。看似青衣少年,实则万兽之君。衣摆无风而动,眼中爆闪神芒。

%10
奴道杀招——八面埋伏!

%11
数万只鹤群闻声而动,一群群暗藏隐形的飞鹤,暴露身形,将两位神秘蛊仙重重包围。

%12
鹤风扬故意留下一个兵力部署的空白,就是陷阱,等着两位对手入瓮。

%13
现在对手中计,他仍不罢休,箫声高扬上去,同时身上飞出三只五弦古筝蛊。

%14
箫声悠扬,筝音清促,化为又一道奴道杀招——阵如水!

%15
无数飞鹤向外疾飞,又有无数飞鹤向内靠拢。

%16
整个战阵变化宛若精密的齿轮相互咬合,规模庞大却又井然有序。飞鹤相互穿插,阵型宛若流水般变化。

%17
两位神秘蛊仙置身其中,看着鹤群眼花缭乱的变化。顿时感觉自己就像是磨盘上的黄豆,要被碾碎压磨!

%18
鹤风扬吹奏不停,稳具兽群中央,傲立荒鹤之背,冷眼看着鹤群和两大神秘强敌纠缠成一团。

%19
他心中思量:“自交手以来,这两人杀招用了无数,涉及音道、宇道、炎道、风道、光道、暗道、律道等等。乃是故意为之,不愿动用真正手段,以防暴露身份。”

%20
蛊师升仙。按照自身底蕴所在,以及投放何种蛊虫炸出仙窍,细分为各派蛊仙。有炎道蛊仙、风道蛊仙、智道蛊仙等等,成就各道仙体。

%21
仙体是蛊师升仙时。受到三气改造。剔除杂质,生命本质得到升华而成。

%22
风道仙体,有风的道痕、法理,因此使用风道蛊虫,效果更佳。炎道仙体,是受到关于火焰大道的洗涤升华,更亲近于炎道,使用炎道杀招效果脱俗。

%23
例如方源。虽是仙僵,但归根究底乃是力道蛊仙。本身便是力道仙体,使用其他流派的蛊虫、杀招,效果只是一般。

%24
企图阻杀鹤风扬的两位神秘蛊仙,并未使用出真正的手段,因此才在以二敌一的情况下,反被鹤风扬压着打。

%25
“二位还想遮掩身份到什么时候?再不动用真本事,可就要丧命于此了。”鹤风扬冷笑连连,声音响彻整片战场。

%26
两位神秘蛊仙喘息微促,鹤群攻势如滔滔江水,连绵不绝,让这二仙疲于应付。

%27
正如鹤风扬所讲,二人若是不动用真正手段,真的就被打杀。

%28
两人对视一眼,点了点头,终于下定决心。

%29
一位叹道:“不愧是名门正派。鹤羽飞仙之名,某家今日见识到了。”

%30
另一位则道:“不过我们就算不动用真本事,也能杀得掉你。鹤风扬,你且看看周围吧。”

%31
“嗯?”鹤风扬目光迅速扫视周围。

%32
原本迷雾重重,哀声连连的战场,此刻终起变化。

%33
一团团的浓雾中,凝聚出一只只的暗蓝魂兽。短短几个呼吸的功夫,魂兽数量暴涨,和鹤风扬的鹤群规模齐平。

%34
“这难道是……魂狩战场?”鹤风扬心中顿时一惊,脱口而出道。

%35
“呵呵呵,鹤羽飞仙,果然好见识。”神秘蛊仙笑道,声调不断变化,时而男声时而女音。

%36
鹤风扬脸色一沉。

%37
这魂狩战场杀招,赫赫恶名,十分难缠。在这战场中死去的生灵,魂魄会被战场抽取,形成魂兽,被杀招之主驱策。

%38
更关键的是,这战场杀招来头更大,乃是历史上十大尊者之一,杀性最重的幽魂魔尊,年轻时候所创的战场杀招。

%39
这两个神秘蛊仙,究竟什么来头,居然掌握了失传的战场杀招魂狩?!

%40
魂兽一出,立即和鹤群厮杀在一块。

%41
战局陡然变化。

%42
鹤风扬连忙指挥鹤群,换了阵型,抵御魂兽进攻。两位神秘蛊仙,压力骤减,冲出鹤群包围。

%43
魂兽群规模和鹤群相当,但却战不过鹤群,往往牺牲了十头魂兽,才能杀掉一只飞鹤。

%44
但鹤风扬心中却是越加警觉,魂兽被撕扯成碎片,这些碎片又能在魂狩战场中,重新凝绝成新的魂兽。

%45
而一头飞鹤死去后,魂魄就会被战场抽取,形成魂兽,成为对方的兵力。鹤风扬修为再高,也不可能做到毫无战损。一有战损,此消彼长之下,就是魂兽占优。

%46
时间拖得越长,鹤风扬的优势就越弱,最终鹤群就会被魂兽群彻底消灭。

%47
“鹤风扬,好教你知,刚刚我二人故意示弱,无非是拖延时间罢了。如今魂狩战场终于形成气候,你的奴道优势已经荡然无存。”

%48
“鹤羽飞仙,你今日就陨落于此吧。”

%49
两位神秘蛊仙联袂而至。

%50
风道杀招——风卷龙鞭!

%51
风鞭在手,长达百丈,遥遥一抽,沿途搅碎无数飞鹤。

%52
毒道杀招——蜂罗刺!

%53
数以千百计的针刺,宛若暴雨倾盆。盖压下来。毒性之猛烈,但凡被刺中的飞鹤,无一不立死当场。顷刻间骨消肉解。

%54
两位神秘蛊仙终于使出真正手段,原来一位是风道蛊仙,一位是毒道蛊仙。

%55
鹤群大片大片地被消灭,大量的魂魄被战场抽取,形成一批又一批的魂兽。

%56
“不好,对方有备而来,早已谋算妥当。此地不可久留。”鹤风扬见此,立即想到撤退。

%57
只有傻子,才会在魂狩战场中。跟着敌人死磕。

%58
“这魂狩战场迷雾重重,很容易迷失方向,极难脱离。但我此刻手中,却有拓宇仙蛊。脱离这里。应该不难。”

%59
鹤风扬念及于此,便用神念沟通仙窍中的一股意志。

%60
这意志形成一个魁梧男子,鹤风扬之前出得飞鹤山时,就藏身于他的仙窍中,并且一直通过蛊虫,观察外界情形。

%61
鹤风扬一路上的经历,从离开飞鹤山,进入狐仙福地。再到被方源等人围攻,鹤风扬离开狐仙福地。又陷入魂狩战场,被两大神秘蛊仙围攻。这股意志都一清二楚。

%62
鹤风扬对其请求道:“虎魔大人,还请你再用一次仙道杀招,破开战场,离开这里。”

%63
这股意志正是仙鹤门,太上三长老,虎魔上人的怒意。

%64
他盘坐在半空中,面无表情地睁开双眼:“你须知每用一次破门而入,就得耗去我本体的两颗红枣仙元。”

%65
鹤风扬连忙道:“晚辈知晓,这一次回去,一定相应补偿。”

%66
虎魔怒意点点头,不再说话,手掌一抓,抓起身边的两颗红枣仙元,抛给面前的拓宇仙蛊,同时周围大量的凡蛊,也升腾而起。

%67
酝酿片刻,鹤风扬手掌一展,从仙窍中挪出拓宇仙蛊。

%68
仙蛊笼罩着一层黄光,鹤风扬照着左近一打。

%69
黄光脱手而出,立即打坏魂狩战场,破开一个空洞。

%70
鹤风扬心中一喜,收起拓宇仙蛊,脚下九宫鹤清啸一声,双翅一振,载着他飞出魂狩战场。

%71
战场外,仍旧是万里晴空,湛蓝无云。

%72
“糟糕!他竟然有拓宇仙蛊,让他给跑了!”

%73
“鹤风扬,有种的你别跑,让我们大战三百回合!”

%74
身后传来两位神秘蛊仙的叫喊声,颇有些气急败坏的意味。

%75
鹤风扬朗笑一声,心中充满了得意之情:“就算是有魂狩战场,也困不住我。不管你们是谁,今日阻杀埋伏我的仇,我记下来了。别以为你们遮掩了容貌,我就查不出你们来。你们俩个动用了真正手段,届时我回到门派,大力排查,就不相信查不出你们的身份!嗯?不对!”

%76
忽然间,鹤风扬察觉到不妥,脸色骤变。

%77
他自己向来谨慎镇定,忍耐克己,怎么一脱离魂狩战场,心情却飞扬起来,仿佛打了胜仗一般?

%78
他此次收服狐仙福地失利,有负虎魔上人的栽培和期许,应该心情沉重才是。

%79
“糟糕!”一股强烈的警兆在他心头升起,鹤风扬骇然回首,只见一位身材矮小如童子的蛊仙,浑身笼罩一层暗光,手持匕首状的蛊虫,悄然飞扬在他的身后,距他已经不足一尺之地。

%80
第三位蛊仙!

%81
竟然有第三位蛊仙埋伏着。

%82
这才是真正的杀招,局中之局!趁着鹤风扬脱离魂狩战场,心神松懈的那一刻,实施致命的暗杀!

%83
鹤风扬惊怒交加,瞪大通红的双眼,怒气沸腾,一时间没有后退,反而心中有一股和来者拼个你死我活的冲动。

%84
“快躲!对方是情道蛊仙,可以影响你的情绪,干扰你的判断!”仙窍中,虎魔怒意急得大吼。

%85
“呵呵呵,来不及了。”第三位蛊仙扬起匕首,轻轻地刺向鹤风扬。

%86
匕首状的蛊虫,陡然爆发出强烈的仙蛊气息。

%87
鹤风扬睚眦欲裂,惊骇欲绝。

%88
对方掌握着杀伐仙蛊?!

%89
他手中只有一只拓宇仙蛊,不能带给他帮助。而他掌握的凡道防御杀招,绝防不住此等攻伐仙蛊!

\end{this_body}

