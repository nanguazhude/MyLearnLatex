\newsection{北原风波起,智道有传承}    %第四十一节:北原风波起,智道有传承

\begin{this_body}



%1
大雪山第三支峰,静室。

%2
窗外天气正好,晴空万里,空气冰爽,白雪高山,结合辽阔的碧空,汇成一幅宁静旷远的画面。

%3
静室中,茶香渐渐弥漫。

%4
方源半躺着,浑身伤痕累累,喘着粗气,一股股细小的尸血从他身上的大小伤口,流淌下来,落到地上,汇成一滩。

%5
但他毫不在乎,一点疼痛的感觉都没有,一直闭目养神着,已经快要一盏茶的功夫了。

%6
除他之外,静室中还有三人。

%7
太白云生站在方源身旁,双手翻飞,打出一道道的黑线,治疗着方源的伤势。

%8
黎山仙子坐在靠着窗口的老位置,心不在焉地煮茶,实则大半注意力都集中在身旁的黑楼兰身上。

%9
黑楼兰盘坐着,双目紧闭,面容肃穆,宛若一座玉像。

%10
升仙三步,碎窍、纳气、放蛊。她在冰原渡劫,完成了前两步。通过定仙游回到静室后,便着手进行第三步。

%11
凭借她的底蕴,炸出仙窍是板上钉钉的事情。现在估计正在新生的仙窍中炼制仙蛊,抵御仙窍中产生的天灾地劫。

%12
看样子,她是想独立走完这一步,没有让外人帮忙的意思。

%13
不过一旦事有不谐,黑楼兰也会立即求援,届时方源、黎山仙子、太白云生这三人都会迅速出手,进入黑楼兰的仙窍中抵抗天灾地劫。

%14
当新茶汩汩沸腾的时候,黑楼兰吐出一口浊气。缓缓睁开双眼。

%15
这个动作,立即吸引了其余三人的目光。

%16
黎山仙子立即停下动作,放着茶壶不管。关切地问道:“成功了?”

%17
黑楼兰点点头。

%18
黎山仙子深呼吸一口气,双眼泛红,脸上流露出兴奋、欣慰交杂的浓郁感情,微微哽咽道:“小兰,你是大力真武体,如今成为了绝仙之体,拥有特等福地。潜力超凡。这真的是守开云雾见青天,妹妹泉下有知的话,一定会很开心的!”

%19
寻常蛊师升仙。有上中下三等福地。

%20
十绝体升仙,冒着巨大危险,当然就有更大收益。一旦他们升仙成功,福地皆是特等福地!

%21
上等福地有七百万至九百万的地域。引动大型光阴支流。拥有超过三十颗的青提仙元,依靠仙窍中残留的天地二气,使得本命蛊、核心蛊升成仙蛊。

%22
而特等福地更加广袤,超过一千万亩地域,引动巨型光阴支流,青提仙元超过五十颗!不出意外,能收获至少两只仙蛊。

%23
黑楼兰却显得神色平静:“虽然是成为蛊仙了,但却没有杀掉那个老贼。”

%24
平淡的语气中。包含着深刻入骨的仇恨。

%25
“黑楼兰,恕我直言。要想杀掉黑城,凭我们四人现在联手,哪怕专门对付他一人,恐怕也希望渺茫。”太白云生咳嗽一声,开口道。

%26
“不错,要打败一个蛊仙容易,要斩杀掉却难。”方源也坐起身来,深叹一声,他身上伤势已经基本痊愈。

%27
他上一次和西漠蛊仙肥娘子对战,占据优势后,肥娘子跑了。

%28
这一次和黑城、雪松子交手,虽然万军横扫,也占据上风,却是围困不住蛊仙,让他们安然后撤。

%29
万我杀招虽强,但对方不跟你硬碰硬,你也没办法。

%30
蛊仙都是精明之辈,都知道避敌锋芒的道理。事实上,方源二战黑城,只能算是打个平手。

%31
方源占据上风不假,但却难以持久,若是时间一长,力道虚影自发消散,就轮到黑城的凌厉反攻了。

%32
仙道杀招的确营造出来巨大的优势,却难以在黑城的身上转化为彻底的胜势。

%33
所以当黑楼兰开口,要求撤退时,方源也没有反对。他见好就收,当即催动了定仙游,带着他们回到了这里。

%34
黑楼兰看向方源,微微一笑:“这一次能抵御强敌,令我渡劫成功,多亏了方源你的力量。”

%35
“既是盟友,自然要守望相助。”方源眼中闪过一抹精芒。

%36
这一次他帮助黑楼兰,下一次仙鹤门攻击狐仙福地,碍于盟约,他们也要帮助方源。

%37
双方是互利互惠。

%38
“按照当初的协约,这便是小家子气蛊的蛊方。另外本人借给贵方的三十块仙元石,也只需还来一半即可。”黎山仙子道。

%39
她现在知道方源的一些秉性,只有货真价实的利益,才符合方源的口味。

%40
方源当场接过小家子气蛊的蛊方,故作叹息到:“这一次大战,我虽胜犹败,损耗了六十多颗青提仙元。欠仙子的十五块仙元石,恐怕需要一段时间,才能够还上了。”

%41
黎山仙子因为黑楼兰升仙成功,心情很好,当即便道:“不急,不急,方源你什么时候手头宽裕方便,再还给我罢,不会收你任何利息。另外,你不是要我们代为打听智道传承的事情吗?”

%42
“难道你们有眉目了?”方源心中一喜。

%43
黎山仙子徐徐道:“是有一条线索。你也知道数月前,北原智道第一蛊仙东方长凡逝世,他为后继者东方余亮留下了一套完整的智道传承。但东方余亮只是凡人,难以保住蛊仙传承,尤其是这道传承还是智道传承。”

%44
“根据线报,这套智道传承被东方长凡布置在一处隐秘的地点,就连东方家的其余蛊仙都不知道具体位置。惟独东方长凡掌握了一点关键线索。”

%45
东方长凡虽然在临死前,和各大黄金部族、超级势力达成了互不攻伐的盟约。

%46
但却没有和魔道蛊仙达成协定。他的能量,还没大到黑白通吃的地步。

%47
东方长凡的智道传承,早就引起了许多魔道蛊仙的觊觎。甚至就连东方家的其他蛊仙,也不是没有想法。

%48
因此东方长凡才会布置这些手段,将自己的智道传承彻底隐藏,只为留给他认可的继承者东方余亮。

%49
方源不禁思量:“我虽然有智道仙蛊乐山乐水,但只能算是粗陋使用,根本发挥不出智道的精髓。宝黄天中虽然有智道蛊虫售卖,但从未卖过完整的智道传承。智道蛊虫易得,智道传承却是相当难得。如果我得到完整的智道传承,相信一定能用更少的意志念头,推算出更多的东西。”

%50
东方长凡生前,乃是北原公认的智道第一蛊仙。他的传承当然让方源心动。

%51
但横亘在方源眼前的,是超级势力东方部族这一庞然大物。

%52
东方家虽然失去了东方长凡,但仍旧有数位蛊仙维持局面。反观方源,刚刚激斗黑城,青提仙元消耗大半,难以再现万我大军的雄风。

%53
“如今,已经有多位魔道蛊仙暗中走到了一起,就是想图谋东方长凡的智道传承。当年东方长凡在世的时候,北原正魔两道都吃过他的亏,很多被他谋算,在他手里栽过跟头。据说,他的继承者东方余亮还有着强于东方长凡的天赋和才华。大多数的北原蛊仙,都不会想出现第二个东方长凡。”黎山仙子继续道。

%54
“几日前,一些魔道蛊仙已经主动找到我,希望利用我的山盟蛊,达成某个暂时的图谋智道传承的联盟。但我之前已经和东方长凡达成盟约,不能出手对付东方一族。方源你若有需要,我可以为你牵桥搭线,令你加入这个联盟。他们有更详细的情报来源,对你争夺传承,大有帮助。”

%55
方源听了这番话,没有拒绝,也没有立即答应:“让我先考虑考虑。”

%56
“这事情也不急。”黑楼兰开口道,“现在北原最大的事情,就是秦百胜俘虏了马鸿运、赵怜云二人,掌握了运道真传中最精华的部分。此事已经引动北原正魔两道绝大多数的蛊仙,甚至那些八转的积年老怪,也隐隐传出要出关的风声。”

%57
黑楼兰流露出一丝嘲讽之色:“呵呵,秦百胜虽是七转蛊仙,散修中的强者,木道战力强横,甚至还要稍稍强过黑城老贼。但这次运道真传实在太烫手了,几乎整个北原的蛊仙都要对付他。现在蛊仙们四处奔走,都在搞串联。秦百胜一人之力,纵然占据福地,也万难抵挡,日子很不好过呢。”

%58
方源眼中寒芒闪烁不定,沉吟道:“你我都是大同风幕下幸存之人。马鸿运、赵怜云都是知情者,当事人!虽然他们俩个提前脱离,但知晓我们的情报。对我们今后行走北原,是个阻碍。”

%59
自从王庭福地毁灭,八十八角真阳楼倒塌,无相手将众多仙蛊带到北原各处,整个北原的蛊师界就彻底乱了,动荡不安,大大小小的战斗屡次发生。

%60
后来,流落在野外的仙蛊大多皆被捕获,局面这才稍安。

%61
现在却又爆出秦百胜俘虏马赵二人的消息,北原当今的局势是暗潮涌动,围绕着秦百胜、运道真传渐渐掀起微澜。随着时间推移,一定会演变成殃及整个北原的巨大风波。

%62
方源比其他蛊仙,都更有去掺和一脚的理由。

%63
不过要投身到如此巨大的风波之中,依靠方源如今的实力,犹显薄弱。

%64
没有交谈多久,方源便做告辞,独自一人回归狐仙福地。

%65
而太白云生则留在这里,准备为黑楼兰治疗。黑楼兰渡过了仙蛊之劫,仙窍福地中一片狼藉,正需要太白云生的江山如故。

\end{this_body}

