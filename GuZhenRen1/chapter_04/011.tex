\newsection{雪山立盟}    %第十一节:雪山立盟

\begin{this_body}

窗外的雪,静静地下。。。

静室中,茶香弥漫。

朱红色的窗棂旁,坐着一位女子。

她身着北原女子特有的锦缎皮裙,皮裙上绣有紫红花蕾,边角银光灿烂。她头上戴着宝蓝色的丝带,丝带中央镶嵌着一颗洁白珍珠。

她低垂眼帘,睫毛浓密,轻声呼吸,双手欺霜赛雪,动作轻缓,正全神贯注地煮茶。

静室不大,只有她一人。但正茶几上,却摆着四个人的杯盏。

忽然,浓郁的碧光陡然出现在静室当中。

光芒消散之后,露出一位老者的身形。

老者身材高大,相貌奇古,鬓发苍苍如雪,满脸皱纹深皱。一双眼睛,饱经沧桑,温和坚韧,蕴藏着岁月积淀下来的人生智慧。

见到老者,煮茶的女子抬起头来,感兴趣地淡笑一声:“你便是太白云生罢。”

老者正是太白云生,他迅速扫视周围一眼,旋即对女子行礼道:“晚辈见过黎山仙子前辈。”

女子含笑点头。她正是北原蛊仙中的风云人物,七转蛊仙,骊山仙子。

她青春貌美,但实际年龄,却是比太白云生要大得多。

太白云生确认周围安全后,打开自家仙窍,旋即便有两道人影跃出。

一个化作黑楼兰,一个是八臂仙僵,高达两丈,青面獠牙,正是古月方源。

“小姨妈,我回来了。”黑楼兰主动坐到黎山仙子的身旁。神情仍旧清冷,但目光中却透露出一丝亲昵。

黎山仙子先是温柔地看了黑楼兰一眼,叹了一口气。随后目光转向方白二人:“我和小兰的关系,一直秘而不宣,甚至外人都不知道我们俩相互认识。今天她主动叫破,可见对二位客人是诚心合作。尤其是你,方源,最近这些天小兰多次向我提起了你。你做了好大的事,居然将八十八角真阳楼都弄塌了。”

方源哈哈一笑。用僵尸特有的沙哑喉咙道:“仙子谬赞,事情闹到这个地步,其实并非我愿。实话实说。我对黑楼兰的合作提议,一直抱有疑虑。但没想到,仙子你居然和黑楼兰的关系如此紧密。这样最好,有了仙子你的仙蛊山盟。我们的联合才会牢靠。”

不久前。在狐仙福地中,黑楼兰袒露了自己的复仇秘密。随后,她又告诉方源她和黎山仙子的关系。

方源既意外又不意外。

黑楼兰处于蛊仙黑城的严密监控之下,单靠自己努力绝难有如今的成就。除了她本身的努力之外,一定还有外力帮衬。

“二位请坐,这是刚沏好的雪油茶。”黎山仙子伸手示意,招呼方白二人坐下。

方源摆手,拒绝道:“还是先起誓建盟。茶水之后再喝也不迟。”

“方贤侄雷厉风行。”黎山仙子轻赞一句,旋即唤出一只仙蛊来。

这只仙蛊。乃是一头兜虫,体型粗壮,比成年人巴掌还大。浑身宛若灰色石质,头部长有一对巨大的钳子,背部并不光滑,宛若山壁嶙峋,足肢的关节处长了青苔似的点点斑纹。

黎山仙子适时地解释道:“此乃六转信道仙蛊,和海誓蛊齐名。只要选取一座高山发誓,只要这座高山健存,所发的誓言便不可违背。方贤侄,不知道你想要选取哪座高山起誓?”

方源微微一扬眉头,手指窗外,嘶哑地笑道:“还有什么地步,能比这座大山更好?”

太白云生不明就里,懵懂问道:“这是什么山?”

“此山名为大雪山。”黎山仙子笑着道。

“大雪山,好像从哪里听说过,等一等,莫非这里是北原魔道蛊仙的老巢――大雪山福地?!”太白云生失声惊呼。

“要不然你以为是哪里?”黑楼兰冷哂一声。

方源接着为太白云生介绍道:“仙凡有别,老白你刚刚晋升,北原蛊仙界的消息你也只是听我提过一些。这位黎山仙子,便是大雪山福地第三支峰的主人,你也可以称呼她为三当家。”

“三……当家。”太白云生瞪眼看向黎山仙子,万料不到如此温柔文静的女子,竟然是魔道蛊仙,而且还是北原最大的魔道老巢中的第三首脑!

……

“咳咳咳。”东方长凡躺在病榻之上,咳嗽不断。随着每一次咳嗽,他原本就苍白的脸色,就增添一分灰败。

“大人……”病榻旁站着一位青年美男子,面容悲戚哀伤。

他一身白衣,面冠如玉,双眼深邃,透着从容淡定的成熟气质,正是东方余亮。

“无须悲伤,亮儿,咳咳,生老病死乃是天道至理。”东方长凡说了这句话,喘息了几声,恢复了点力气,继续道,“你的天资比我还好,整个部族中我最看中的便是你。东方一族再度兴盛的重任,也只有你能挑得起。我东方长凡,不会看错的。”

“太上家老大人!”东方余亮双眼通红,哽咽无语。

眼前濒死的老人,是他的恩人!

他东方余亮十一岁时丧失双亲,不仅要维持生计,还要照顾六岁大的妹妹东方晴雨。

为了保命,他将双亲留下的遗产都被迫送人。

不过也正因为这一点,被东方长凡一系看中,不仅自己成为心腹,而且妹妹也得到了很好的照料。

之后,东方余亮甚至得到东方长凡的亲自指点。后者不顾多方阻挠,钦定东方余亮为本代族长。

东方余亮争夺王庭失利,回到家族受到多方打压排挤,又是东方长凡保护他,为他遮风挡雨,付出良多。

东方长凡越来越虚弱,几次张口都说不出话来,最终他开口,声音十分轻微:“手来。”

东方余亮便伸出手,握住老人的右手。

老人的手中,捏着一只蛊。

“这,这只蛊你拿着。”东方长凡脸上涌起红晕,回光返照带给他一股力气。

他紧紧盯着东方余亮,关照道:“虽然东方一族和其他正道部族都签订的盟约,但世事无常,难以预料。我死后,东方一族从盛转衰,你是我的传人,千万要小心。这只蛊一旦催动,便会带你去一处隐秘之地,那里有我为你准备的修行资源、升仙心得、部族秘史以及我一生智道修行的体悟。切记要以自身安全为先,不必急于一时。部族中,有……有魔道奸细。”

说完,东方长凡神情凝滞,脸上红晕淡去,目光中也终于失去最后一丝光彩。

“大人!!”东方余亮早已泪流满面,此刻不禁悲哭出声。

北原第一智道蛊仙,东方长凡,就此身亡。

消息传出,东方部族悲哭声三天三夜不绝。而收到消息的各大北原势力,正魔两道蛊仙,却是齐齐松了一口气。

东方长凡是一个传奇人物。

他刚刚出生时,东方家族已经是日薄西山,只有超级势力之名,已无超级势力之实。

东方长凡成就蛊仙,领袖部族,多方谋划,运用智道手段合纵连横,结强友,毙弱邻,更设计使得仇敌相互对掐,终令东方家族重振雄风。

东方部族能够崛起,大半功劳都在东方长凡一人身上。

但也正因如此,使得北原蛊仙们都意识到东方长凡的厉害!智道蛊仙要对付敌人,往往不用自己亲自动手,就能叫其好看。设计起来,丝丝入扣,一环接着一环,被算计者如沉入泥沼,就算意识到也无法自拔。

蛊仙们皆忌惮东方长凡,暗自达成默契。禁止对东方长凡出售寿蛊,甚至暗中破坏东方部族搜刮寿蛊的计划。

东方长凡算计了他人,也终被他人算计。

……

艳阳高照,月牙湖波光粼粼,时而有龙鱼跳出湖面。

湖畔堆砌着残雪,这是十年暴风雪灾的遗留。

方源破坏了巨阳仙尊的布置,大雪灾便灌溉到王庭福地,因此北原受灾情况,比往届小了许多倍。

如今王庭福地不在,真阳楼也摧毁了,北原再无十年大雪灾之说。

残雪在阳光下,慢慢消融。

雪中,已经有冒出头的青草。一块块的青色、白色,相互混杂着。

方源和太白云生联袂而行,路途中看到不少的水狼、形单影只的三角犀。原本这里有密集的马蹄树林,但现在大片的树林都已经被积雪压倒、冻死。

景色的剧变,带给方源一点小小的麻烦。

他正在寻找前往琅琊福地的通道――盗天魔尊布置的那片紫色石林。

和黑楼兰结盟,已经过去了三天。

小狐仙留守在狐仙福地,一直对宝黄天保持关注,但仍旧没有等到琅琊老仙的神念。

方源拿回了自己的定仙游,利用这只仙蛊,来到月牙湖畔。

为了避免引发不必要的误会,方源没有直接传送到琅琊福地。他打算再次利用盗天魔尊的布置,中规中矩地进去。

因为环境的变化,他无法直接传送到紫色石林。能够直接来到月牙湖畔,还是多亏了太白云生。

他在北原游荡的时候,曾经留宿在月牙湖边,挖出一个地洞,做了简单布置,生活了两三个月。

地洞没有坍塌,方源先将定仙游借给太白云生,随后钻入他的仙窍中,来到了这里。(未完待续。。)

------------

\end{this_body}

