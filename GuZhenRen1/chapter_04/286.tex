\newsection{惊喜收获}    %第二百八十七节:惊喜收获

\begin{this_body}

%1
咻!

%2
方源好似一支离弦之箭,在半空中拉开气浪,飞速前行。

%3
下方,是漫山遍野的仙材,但方源已是看都不看一眼。这些“死亡”的仙材,毫无价值可言。

%4
忽然,方源目光微微一凝。

%5
在地平线上,陡然出现了一头荒兽。

%6
这头荒兽大如小山,趴在地上,一动不动,赫然就是一头藏经鼋。

%7
方源对这种荒兽一点都不陌生,因为不久前,他就在中洲繁星洞天的碎片世界中,成功捕捉到一头。

%8
把飞行的速度降下来,方源身体轻轻一折,便飞向藏经鼋。

%9
离得近了,方源便发现这是一头已经死去多时的藏经鼋。

%10
它趴在地上,头部耸搭下来,有一小半的脑袋,几乎埋在了松软的紫沙之中。

%11
藏经鼋如此死状,就像是脑袋下的脊椎被敲断了,从四肢看去,明显可以发现它的皮肉无比松弛。

%12
但是它死亡之时,神色却很平静,甚至是安详,让方源不由地感到一丝诡异。

%13
“无数“死去”的仙材当中,躺着一头荒兽的尸体么……”

%14
方源试着采集藏经鼋的身体。

%15
片刻后,他心中一沉,看着龟壳的坚硬一角,在自家的手中再次化为紫沙,随风洒下。

%16
方源目光一阵闪烁,几个呼吸之后,他离开这里,继续飞空探索。

%17
很快,他又看到第二头荒兽。

%18
这是一头地魁。

%19
人身蛇尾,面若蝙蝠,朝天鼻子,双耳招风。并且浑身漆黑,生有肉甲。

%20
在地魁的胸膛前后左右,长有五六十根肉鞭,根根长达六丈有余。肉鞭表面,还生有无数吸盘器官,密密麻麻。

%21
每一条肉鞭的前端。还有菊花般的喷孔,关键时刻喷射出乳白色的液体,液体具有极强的腐蚀性,且藏有极多的寄生小虫。可直接钻入人的毛孔当中,进入身体,实施破坏。

%22
地魁的尸体栩栩如生,保存的相当完好。

%23
但当方源试图采集地魁身上的仙材之时,之前的一幕再次发生。

%24
方源不再感到意外。只是口中嘀咕着:“藏经鼋是中洲的荒兽,地魁则多在北原出现……”

%25
离开这里,继续探索。

%26
方源很快又有了新的发现。

%27
“居然是檀香圣象,而且还是三头!”方源微叹。

%28
檀香圣象比之前的藏经鼋、荒兽地魁,要高一个档次。

%29
因为它是上古荒兽,成年的檀香圣象一身战力,可敌七转蛊仙。

%30
在许久前,秦百胜组织的那场几乎涵盖北原蛊仙界的拍卖会上,就有人出价一个檀香圣象的巨蛋,引发了不小的轰动。

%31
当今五域。檀香巨象已经比较稀少了,很难看到一头。

%32
但此刻,在方源的眼前,却是三头檀香巨象的尸体。

%33
接下来,方源发现了更多的荒兽、上古荒兽的尸体。三三两两,躺在地上,周围簇拥着无数姹紫嫣红或者郁郁葱葱的仙材植株。

%34
啸月天狼、星荒犬、金砂乌骓、冰刺神猿、白蹄墨骊、九宫鹤、铁冠鹰、龙鱼、歧牙猪……天上飞的,地上跑的,水里游的,在这里都能看到。简直像是一场诡异的标本展览会。

%35
越向中央飞行,方源发现的尸躯就越多,数目惊人得很!

%36
北原特有的盘山羊王、南疆的凤羽熔岩鳄、中洲的钻熊、东海的烟华鼋,来自天下五域的荒兽、上古荒兽。堪称应有尽有。

%37
“嗯?这是一头墟蝠啊。”方源不禁微微放缓飞行速度。

%38
这头墟蝠是罕见的宇道荒兽,比不上之前方源在北原太丘时,见到的那头太古级墟蝠。

%39
“半月麒麟!”方源身形一滞。

%40
当今的北原僵盟分部,就有一头活着的半月麒麟,被整个北原僵盟当做宝贝疙瘩豢养着。

%41
“这头上古荒兽,好像是……天残犬?”方源再三确认后。肯定了这个答案。

%42
天残犬他也没有亲眼见过,只是在典籍中一睹过它的形态容貌。这种上古荒兽,可是白天才有的猛兽。

%43
见到越多,方源也开始麻木了。

%44
他的见识已经算得上博文广志,但很多上古荒兽、荒植,他竟然都不认识。

%45
“还有这里的空间,居然如此广阔。我飞行了这么久,速度飞快,居然还未见到蛊阵的边缘。”

%46
这一点,有点出乎方源的意料。

%47
他之前试图破解过这个巨型蛊阵,知道这是一个宇道为主的巨型蛊阵。

%48
方源估算过里面的空间大小,但现在看来,之前的估算完全低估了这个蛊阵的威能。

%49
“看来之前的推演、破解,还远远没有达到蛊阵的真正核心之处。”

%50
“这里究竟是何人的手笔?简直骇人听闻!如此多的仙材,简直是海量!难道是仙尊、魔尊的手笔?”

%51
方源暗暗猜测。

%52
就算是八转蛊仙,也没有这么大的能力,可以独自布置出这个来。

%53
因为五域界壁的存在,几乎隔绝了八转蛊仙的异域往来。

%54
这也是为什么,八十八角真阳楼倒塌这么大的案件,中洲十大古派只派遣了六转、七转的蛊仙去。

%55
当然,到了九转,天下无敌,睥睨一切。五域的界壁,也构不成阻碍了。

%56
“如果不是九转尊者,那布置这里的,必然是一个超级组织。这个组织横跨五域,只有如此,才能采集到五域中的这些仙材。”

%57
虽然有宝黄天,蛊仙可以通过宝黄天,进行交易。但如此的规模,不可能是宝黄天交易出来的。

%58
因为这样的交易,不仅规模庞大,而且连续不断,在蛊师历史上必定会有记载。

%59
但具方源所知,根本没有这样的交易记载。

%60
他虽然深入蛊阵,探索良久,但这个他向往已久的地方,覆盖的神秘迷雾反而越发浓郁厚重了。

%61
方源对这个地方了解不深。

%62
他前世去攻打过狐仙福地,但这个地方,他从未涉足。

%63
这只是他前世的听闻。

%64
北原僵盟忽然强盛起来。甚至压过东海的总部一头。北原僵盟的对外解释,就是这里的地沟遗藏。

%65
说是一位仙僵大能所留,有着仙僵重获新生的法门。依赖于此,北原僵盟中的许多成员。都转成活人,吸引了无数在野的仙僵投靠,北原僵盟的势力越发庞大。

%66
不过之后,好景不长,北原僵盟的强盛被马鸿运破坏。

%67
方源对这里并不熟悉。他看到的一切,让他产生了更强烈的好奇心。

%68
“我探索的方向应该是对的,因为见到的仙材的层次越来越高了。布置这里的,究竟是何方神圣?简直恐怖绝伦,恐怕天庭都没有这么大的手笔吧?”

%69
天庭是中洲的组织,牢牢霸占着中洲,在其余四域布置的暗线并不多。

%70
前世五百年动荡,什么牛鬼蛇神都冒出来,方源知道这点。

%71
应该说,每个超级组织都它自身的发展战略。天庭的战略。就牢牢掌控中洲,光明正大,至高无上。

%72
“如果布置这里的是一个组织,那么这个组织,就是横跨五域,隐藏暗处,悄无声息。嗯?那里是……”

%73
方源蓦地发现,在远方出现了一片紫。

%74
这是一片荒芜之地。

%75
全是紫色的细沙,没有之前漫山遍野的仙材。

%76
这片荒芜的沙地,呈现一个规则的超巨型圆圈。方圆至少有三千多亩!

%77
无风无声,死寂一片。

%78
方源心头暗惊:“这里便是整个蛊阵的最中央,曾经发生过一场恐怖的大爆炸!”

%79
正是这一场爆炸,顷刻间造成了三千多亩的荒地。曾经这里。应当也是仙材遍布,但现在这里空无一物。

%80
渐渐接近荒地的边缘,忽然沙地上一道亮光闪烁,引起了方源的注意。

%81
方源落下身子,竟发现沙地上,倒插着一片薄薄的翅膀。

%82
这不是鸟类的翅膀。因为它没有羽毛,半透明状,应当是蜻蜓、蝉之类的翼翅。

%83
这片类似蝉翅的东西,散发出来的气息,叫方源都暗吞口水。

%84
这是太古荒兽的气息!

%85
虽然只是残留,但货真价实,惊人无比!

%86
这个蝉翼已经残破不堪,但是在光线下,却如宝石般莹润。时不时的,还在充足的光线下,闪耀一抹暗金色的光辉。

%87
“这是金道的道痕之光,这片蝉翼的主体,在生前一定具有恐怖到极点的道痕底蕴!也只有堪称准九转的道痕积累,道痕的存在才会令肉眼都可见到。也就是发出道痕之光来!”

%88
“等一等,这个或许是……斑虎蜜蜂的蜂翅!”

%89
方源忽然灵光一闪,得到一个猜想。

%90
这个猜想,令他心神颤动。

%91
斑虎蜜蜂,这是《人祖传》中明确记载的生物。

%92
它们一个个都有花豹子一般大小,身上的花纹好似虎纹,黄金打底,黑斑点缀。实力强大无比,单单一只,太日阳莽也不是对手。而炼成的蜜酒甜而不腻,醇香可口,十分好喝,是天地间的绝对佳品。

%93
“如果是豹子大小的蜜蜂,这片蜂翅就恰好合适。”方源想到这里,渐渐苦笑起来。

%94
他也不确定。

%95
以往的见识,在这里显得狭隘不堪,浅薄得很。

%96
他信手将这片蜂翅,或者蝉翅的东西,拿起来。

%97
出乎他意料的一幕发生了!

%98
这片蝉翼离开了紫沙,居然完好无损,没有如之前的那些仙材一样化成紫沙。

%99
一时间,方源又惊又喜!

%100
这可是太古荒兽的身躯部分,是八转级数的炼蛊仙材!

%101
甚至,都能发出道痕光晕。

%102
因此更准确的说,这是准九转的仙蛊仙材啊!

%103
ps:蛊真人公众号上会推送人族传和发布其他一些东西,欢迎大家关注搜索蛊真人公众号,添加即可。

\end{this_body}

