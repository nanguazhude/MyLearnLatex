\newsection{炼制梦道凡蛊}    %第一百二十一节:炼制梦道凡蛊

\begin{this_body}

细雨飘扬,青茅山在雨中更显得苍翠葱茏。

方源静坐在屋中,面对着弟弟方正。

“哥哥,你为何还执迷不悟呢?舅舅舅母这些年来,含辛茹苦地养育我们,你却将他们告到族中长老会中去,这简直是忘恩负义啊!”方正昂首挺胸,站在方源的面前,义正言辞地斥责着。

方源面色平静,目光如冰,环视左右,心道:“这处梦境却也逼真。”

随后,他又视察自身,只见三转空窍中,藏着酒虫、全力以赴蛊等等。

方源心中便一定,心道:“这身修为相当于三王福地时候了。”

这时,面前的方正又激动地喊道:“现在距离召开长老审判,还有一段时间,哥哥你现在撤诉还来得及。你真要告上去,你的名声也就毁了。周围人都会看不起你,我也不会认你这个哥哥了!”

方源呵呵一笑,长身而起,向方正走去。

方正后退一步:“哥哥,你想干什么?”

见他这番模样,方源心中泛起一股厌恶之情。

“这是梦境迷障了。”方源心知肚明,忍耐着心中的厌恶,越过方正,走向屋门。

“哥哥!”方正猛地回头,拽住方源的手臂。

方源行进不得,回过头来,看着自己的弟弟,心中厌恶之情更加浓郁。有一股冲动泛起,真想一巴掌甩到方正的脸上,然后再扬长而去。

但方源越加冷静。轻轻地捉住方正的手臂,慢慢用力,试图摆脱他的纠缠。

然而随即。方正的另一只手也抓住方源。

方源无奈地叹了一口气,眼中却是厉芒闪过,猛地抬腿,将方正直接踹倒在地。

方正被踢倒在地,短时间内无法起身。

方源这才自由,转过头来,迈开大步。走出大门。

刚一出门口,眼前的景象便骤然变化。

却是来到了古月家族的议事大堂。

一位位家老,分坐两旁的位置上。族长则坐在中央的高位。俯瞰着方源和方正。

方源审视自己,这才发现,他正跪在地上。身旁的方正也跪着。

方源心中明镜一般,心知刚刚踹倒了方正。心中情绪勃发。使得梦境迷障更强。

刚才的第一场景,只有小屋一间,人物两位。现在这里的议事堂,比小屋更加宽敞。而且出现了近十位人物,只是各个脸面模糊,唯有方正清晰。

方源目光逡巡间,只是盯着族长看的时间多了一点,族长的脸面就开始渐渐变得清晰起来。

方源连忙转过目光。不再多看。

看的越多,心中记忆就给勾起。梦境就越加徐徐如生。更为可虑的乃是牵动心中情怀,一旦入情,就会陷入梦境迷障。

方源没有梦道仙蛊辅助,一旦陷入迷障当中,想要脱身就很艰难了。

“这梦中炼蛊材料,究竟在何处?难道说,不在这第二场景,还在更下面?”方源目光搜索,没有结果,正心中揣摩时,堂中族长发言。

随后,一位家老越众而出,向诸位宣布方源状告舅父舅母,侵吞父母遗产一事。

方正作为证人,当众为舅父舅母说话。

方源一一听在耳中,不由心中渐增厌恶,更有一丝愤懑之情,潜藏其中。

堂中家老们纷纷开口,言语明祥偏向方正一方,对方源很不待见。

方源情势危急,但心中仍旧冰雪般冷静。他仔细品味心中情怀,内心深处始终不屑一哂。

“宣被告一方进来。”这时,族长忽然开口道。

舅父舅母登场,带着义愤填膺的神色。一开口,便是数落方源,宣扬他平时不孝的举止态度。这纯粹便是诬告了,完全是子虚乌有之事,偏偏各个家老都信以为真,对方源冷眼冷笑。

“给你最后一次机会,你现在可有什么分辩言语?”末了,族长开口,问向方源。

方源冷笑一声,这是梦境中的陷阱,真要发言,勾动了心中情怀,就落入险境了。

于是他摇摇头,一言不发。

族长顿时变色,冷笑一声,手指着他:“你这果然理亏,无法辩驳了。我现在就宣布,将九叶生机草蛊,交给你的舅父舅母。”

九叶生机草栩栩如生,被族长从空窍中取出,当场交给舅父舅母。

方正连连磕头,感谢道:“谢谢族长大人,感谢诸位家老大人明察秋毫,还我舅父舅母清白名誉。”

方源心中升腾起一股淡淡的悲愤,但都被他忍住。

他目光炯炯,紧紧盯着九叶生机草。

梦中的炼蛊材料出现了,就是眼前的这只“九叶生机草”。

“真是阴险。落在关键之物上,便是千方百计地想要勾动我的情绪。我精神越集中在此梦材身上,就越会被勾动情怀,陷入梦境迷障。”

想到这里,方源猛地发动,真元催谷,灌注在数只蛊虫身上。

数只力道虚影爆发,将舅父舅母打飞出去。

他动若兔脱,一把抢过“九叶生机草”,随后根本就不停留,夺门而出。

“好贼子!”族长怒极反笑,首先追杀出去。

“居然以下犯上,简直胆大包天,该杀!”诸位家老们纷纷怒吼,紧随族长身后,向方源追去。

方源刚出了大门,眼前景象便发生翻天覆地的变化。

只见他身处一处山洞,周围石壁惨白如骨,让方源不禁回忆起南疆的白骨山。

因为灰骨才子的遗藏,方源和白凝冰合作,最终逃离白骨山,躲开了百家一族的追杀。这个经历让他印象深刻,刻印在内心深处。

此时回忆的念头一动,周围的山洞就越加清晰起来,甚至发生变化,变得和当初经历的白骨山洞一模一样。

方源心中警惕,连忙止住回忆。

这时,古月族长带着近十位家老,出现在方源的视野中,追杀过来。

方源连忙后退。

在梦境中一旦死亡,对魂魄大为有害。更有甚者,会因梦而亡。五百年前世时,不知道多少蛊师在探索梦境时,卷入梦境迷障,最终失去了生命。

依照方源此时的魂魄底蕴,死亡倒是不可能,但魂魄重伤是一定的。

魂魄损伤,方源也不惧怕,因为他做着天底下独一份的胆识蛊买卖。但受伤后退出梦境的话,他辛辛苦苦找到的梦材,却要消失了。如此前功尽弃,不是方源所愿。

方源一边向后逃窜,一边查看怀中的九叶生机草蛊。

草蛊栩栩如生,方源心中亦是感应不断,时时刻刻提醒着他这是炼制梦蛊的上佳材料。

方源左拐右拐,逃窜到一处小巧的洞中。当即抛下几只蛊虫警戒,紧接着就盘坐在地上,按照记忆中的蛊方,开始炼蛊。

刚炼到一半,方源就被追兵们发现,堵住了唯一的洞口。

方源无奈地叹了口气,催动数只力道兽影,强行突围而出。

激战当中,他硬生生忍住情绪,只是打退他们,没有杀死。

片刻之后,他又发现第二个洞口,再次入洞炼蛊。

这一次炼蛊时间更短,只完成了大体上的三成,就被方正发现。

“他在这里!”方正高声大喊。

方源冷笑一声,一脚将方正踢飞。

这时,一位家老已经赶来,突袭方源的后背。

在梦中的每一次受伤,都是对魂魄的重创。

方源陷入险境,连忙转身,爆发数头力道兽影,将袭击自己的家老打爆成肉酱。

“方源,你不得好死!”临时之前,这位家老呐喊,发出诅咒。原本模糊的脸面,陡然清晰起来,化为学堂家老的样子。

方源无奈地叹了一口气。

刚刚危险关头,他分心他顾,压制心中情绪的力度稍稍减弱,使得入梦更深一筹。

这也是他为什么始终留手,不打杀了追兵的缘故。

梦中场景源源不断,几乎可以说永无止境。方源就算杀死了追兵,也会有其他的追兵出现。就算不是追兵,也会变生出不一样的干扰掣肘。

且生死搏杀之时,更要全神贯注,使得情绪容易泄露出来。梦境种种,就是千方百计地勾动做梦者的情怀,使其不辨真实虚假,最终永远沉溺在梦境迷障中,不能自拔。

好在方源有着前世经验,虽然前世在梦道上的成就十分可怜,但也足够他面对这个梦境了。

摆脱了追兵之后,方源来到一处大厅,却是和他记忆中炼制骨肉团圆蛊的场景,一模一样。

在这里,方源终于炼成梦道蛊虫。

追兵杀来,方源哈哈一笑,自言自语道:“这就醒来罢。”

说完,眼前骤然化为乌有,一片黑暗。

方源缓缓睁开双眼,黑暗消散,露出荡魂行宫的景象,一如入梦之前。而他着盘坐在床榻上,手中虚捏着,保持着梦中的动作。

在梦中,他正捏着刚刚炼成的梦道凡蛊。此时回到现实,这只梦道凡蛊却是不翼而飞。

方源也不惊惶,而是向自家脑海中探去。

只见脑海中,一只梦道凡蛊紧紧地悬浮着,周围各种念头宛若气泡般,在飘扬消散。

“第一只梦道凡蛊,总算是成了。”方源欣慰一笑。

梦道有别于其他流派。凡蛊都没有形体,只能存于脑海。唯有梦道仙蛊,才能由虚返实。(未完待续。)

\end{this_body}

