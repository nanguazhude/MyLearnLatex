\newsection{好一桩大买卖}    %第八十七节:好一桩大买卖

\begin{this_body}

%1
如果要用恶念凡蛊,来替代乐山乐意仙蛊的话,那么前者的数量就必须很庞大。

%2
方源需要很多的恶念蛊,不只是数百只这么简单。他初步估算了一下,至少要达到两千只以上,同时催动,产生的恶念,才能在智慧光晕下勉强搭上够用标准的及格线。

%3
要强调的是,这两千只以上的恶念蛊,都必须是五转。

%4
方源手中的恶念蛊秘方,从一转到五转皆有。一转到五转的恶念蛊,都是产生恶念,效用相同。只是一次产生的念头数量不同,消耗的真元多寡也有巨大差距。

%5
“炼蛊不易,转数越高,炼制越难。尤其是五转恶念蛊,需要从一转到二转,二转到三转,如此层层升炼,却无一蹴而就的蛊方。这样算下来,成功率并不高,当中耗材将极其巨大。”

%6
方源初步打算,先炼制出两千只恶念蛊。至于恶意蛊,他暂时并不需要。他的脑海中,还残留了许多乐意呢。

%7
炼制两千只五转恶念蛊,无疑是一场规模巨大,耗时日久的炼蛊。

%8
若方源没有收购毛民,建立第二石巢,单凭第一石巢的毛民,只在炼制气囊蛊的闲暇之余炼制的话。要达到两千只五转蛊的目标,至少得有十年光阴。

%9
现在方源有了第二石巢,大量的毛民,这个时间就大为缩短,只有十六个月左右的样子。

%10
“这期间,我若可以将蛊方改良,设想出一个全新的五转蛊方,可以直接炼制五转恶念蛊的话。那么,十六个月的时间还会大大缩减。”方源心中想着。

%11
他现在的炼道境界,已经提升到了准宗师地步,但智道境界几乎是空白一片。

%12
不过方源仍旧具有信心。

%13
这股信心的来源,不是其他,正是智慧蛊。

%14
借助智慧光晕,他会有无限的灵感。就算境界欠缺一点,也不要紧,可以强行推算。

%15
“九转仙蛊之威,的确恐怖至极。”方源现在越来越深感九转智慧蛊的厉害之处。

%16
借用智慧光晕,他可以推算仙蛊方。而这些仙蛊方,是炼道准大宗师境界的琅琊地灵都位置困扰,无有进展的东西。

%17
方源还可以借助智慧光晕,推算仙道杀招。

%18
凶雷恶人从血神子残方中,参悟出仙道杀招雷神子,耗费了数年的时间。期间一直闭关,足不出户。

%19
换做方源,且拥有相同的雷道境界,借助智慧蛊的话,顶多个把月就能完成。

%20
黑楼兰目前正在努力地将凡道杀招我力虚影,推到仙级程度。可惜,哪怕她有凡道杀招的底子,又有力道境界,想要达成目的,也至少要有数年时间。

%21
换做方源的话,估计半个月就能搞定了。

%22
事实上,推算仙道杀招这种事情,方源已经早就在断断续续地做。他推算的仙道杀招,不是别的,真是万我。

%23
万我的原本核心仙蛊净魂,因为饥饿,不堪催用。方源推算万我的主要目的,就是替换掉净魂仙蛊,用其他蛊虫代替。

%24
但可惜的是,这只唯一的仙蛊一去,威力暴降。哪怕用再多的凡道蛊虫,也达不到原来的威力。

%25
纵观蛊师历史,古往今来唯有一个凡道杀招,攻伐之威,能够媲美仙蛊。

%26
那便是白骨战车!

%27
此杀招乃是由傲骨魔君沈桀骜所创,惊才绝伦,傲视古今。

%28
所以方源将荒兽尸体挂到宝黄天中贩卖,其中就有一条交易,就是用荒兽尸体换取战骨车轮蛊之类的五转凡蛊。

%29
战骨车轮蛊,便是组合白骨战车杀招的凡道蛊虫之一。方源手中仅有一件,是从北原,狼王常山阴的藏身战场中得来的。

%30
方源的才情还比不上沈桀骜,去掉净魂仙蛊的话,万我杀招威力根本不够看。

%31
接下来的半个月,方源都在努力改良恶念蛊的蛊方,除此之外,也没有放弃改善仙道杀招万我。

%32
在这期间,他留在宝黄天中的荒兽尸体,接连卖掉两头。

%33
换来了第二对荒兽蝠翼,以及数十块仙元石。

%34
方源本来不想,用荒兽尸体直接换来仙元石。但他手头上只剩下五十块仙元石,大部分的钱财都用来收购毛民。现在大规模炼制五转恶念蛊,需要的资源很是庞大。因此,不得不换来仙元石,勉强应付了开销。

%35
这样一来,方源挂在宝黄天中的荒兽尸体,只剩下三头。

%36
手中的仙元石,也只剩下二十几块。青提仙元,也因为不断推算的缘故,一连降至到三十六颗的程度。

%37
不过,好在方源改良恶念蛊方的计划,已有成效。

%38
全新的恶念蛊蛊方,虽然没有达到预期的程度,但已经减少了许多步骤。先前的五转恶念蛊,需要从一转开始,不断升炼。现在的五转恶念蛊,只需要两个步骤。第一步炼成半成品,第二步成品。

%39
虽然成功率没有提升,但是步骤减少,耗材就降低了许多,时间也节省下来。

%40
“按照这个进度,原本的十六个月,直接缩短一半。只要炼制八个月,我就有两千只五转恶念蛊了。不过我现在手头上的仙元石,却已经吃紧了。照这种程度,即便有胆识蛊的买卖支撑着,但也入不敷出。难道我需要暂缓一步,先让第二石巢炼制气囊蛊,提升胆识蛊的产量,先赚来仙元石支撑局面吗?”

%41
就在方源不得不如此考虑的时候,他等待已久的好消息,终于来到了。

%42
“臭小子,你赢了!我答应你的条件,咱们就四六分吧!”琅琊地灵恶声恶气,心情很是糟糕。

%43
方源因为大规模地炼制凡蛊缺钱,琅琊地灵要炼制仙蛊,消耗的资源比方源要高出百倍还不止,他更缺钱!

%44
不出方源之前所料,琅琊地灵没有其他渠道,只有方源这一条路可以走。

%45
所以,他不得不走,哪怕明知道这样一来极为吃亏。

%46
方源也不想得寸进尺,这样的利益瓜分,已经达到了地灵的极限。若是再不知好歹,惹怒了琅琊地灵,不仅不会交易,而且还连之前好不容易建立起来的关系,都会破坏掉。

%47
北原,百足洞天。

%48
大殿内灯火辉煌,美人歌舞不休,美酒佳肴接连呈于桌案。

%49
“吃吧,这道菜我个人非常喜欢,是用五转蛊虫金刚飞蜈所制,用炭火烤制,肉质鲜美焦嫩。”百足天君居于主位,呵呵淡笑,招待着殿中的两位蛊仙。

%50
两位蛊仙,不是他人,正是秦百胜、黎山仙子二位。

%51
虽是他人所做菜肴,但二仙却毫无顾忌。秦百胜伸手捏起一条长肉,张开大口,扔进嘴中。

%52
他大口咀嚼之后,满嘴皆是油光,竖起大拇指赞道:“果然美味!不错,相当不错。”

%53
百足天君哈哈一笑,又看向黎山仙子:“不知仙子感受如何呀?”

%54
黎山仙子吃香可比秦百胜文雅得多,小口咀嚼,品味之后,轻笑道:“肉质鲜美,不仅焦有味,更有嚼头,滑腻爽口,实乃人间胜品。”

%55
百足天君哈哈大笑,极为得意:“不瞒两位,我为了创造这道菜肴,苦思冥想,摸索了八天八夜,终于得到了成品。这是我今年,最满意的作品之一了。”

%56
百足天君修为高达八转,乃是散修。有一爱好兴趣,闻名北原的蛊仙界,就是喜欢烹饪美食,创新美食。

%57
“这次来,不仅成功邀请天君参加拍卖大会,而且还品味到这等佳肴,真是我等的幸运啊。”秦百胜适时开口道。

%58
“天君盛情款待,黎山铭记在心。如今已经签订了盟约,请恕黎山不能久留了。”黎山仙子接着道。

%59
“呵呵呵。”百足天君笑容满面,点点头道,“不错。北原的八转老人,就那么些个。你们俩个既然已经请了我,又请了药皇、五行大法师、雪胡老祖,独独剩下凤仙太子的话,就不好了。”

%60
“天君明鉴,我俩正要去请凤仙太子。”

%61
“去吧,去吧。把这场拍卖大会搞搞好,老夫很是期待呢。”

%62
由此,秦百胜、黎山仙子二人出了百足洞天,辨明方向,向东南方赶去。

%63
行到半路上的时候,忽然黎山仙子神色微微一变。却是仙窍中,方源通过推杯换盏蛊,传来消息。

%64
黎山仙子缓缓停下身形。

%65
秦百胜当即奇道:“仙子何故停留不前?”

%66
黎山仙子浅笑:“却有一人,需要向大人你引见。”

%67
“哦?”秦百胜起了兴趣。

%68
方源这时,钻出黎山仙子的仙窍,迅速扫视四周一眼后,看向秦百胜:“秦大人,您的威名在下是如雷贯耳啊。”

%69
秦百胜是七转蛊仙,战力超俗,是石磊一流的人物。此刻见方源只是一位仙僵,而且还是六转最低修为,刚刚升腾而起的兴趣便迅速消失。

%70
不过,念在方源能从黎山仙子的仙窍中直接钻出来,可见方源和黎山仙子关系不浅,秦百胜也没有态度倨傲,只是不咸不淡地点点头,淡淡笑道:“不敢当,阁下是?”

%71
“在下沙黄,手头上有项大买卖,要和秦大人您详谈。”方源笑道。

%72
“什么大买卖?”

%73
方源便说了贩卖蛊仙福地之事,秦百胜、黎山仙子皆连失色。

%74
在推杯换盏蛊的信中,方源并未交代清楚,只说自己也想要参加拍卖大会。

%75
黎山仙子吃惊不已:“沙黄,你真的有蛊仙福地要卖?”

%76
“哈哈,不是我有蛊仙福地。我也只是个跑腿的罢了。”方源笑道。他和黎山仙子签订了雪山盟约,必须说实话。

%77
不过这的确是实话,只是说了一半。他当然不会说,自己这个跑腿的,比正主赚得还要多。

%78
秦百胜这才心绪稍平,推测方源是某个大人物的麾下,北原有明面上的八转蛊仙四位,但根据传闻,也有一些老怪物似乎还活着。

%79
当即,秦百胜不重新打量了方源一眼,仍旧笑着,只是语气产生了微妙的变化:“的确是一桩大买卖。既然小兄弟看重我秦某,对这场拍卖大会有信心,那秦某又岂会拒之门外呢?”

%80
“多谢秦大人的通融了。”方源笑容更盛。

%81
“什么通融不通融的,说实在话,我高兴还来不及呢。蛊仙福地,啧啧,一定能够掀起拍卖的大高潮。哈哈,我举办的这场拍卖大会,将会是北原数百年来的屈指可数的盛会了!”

%82
秦百胜豪迈大笑几声,又态度亲和地拍拍方源这位六转小仙僵的肩膀:“说实在话,我是个粗人!秦大人这称呼忒古怪,小兄弟要看得起的话,念我痴长几岁,不妨就叫我一声老哥。”

%83
秦百胜已有七百多岁,确实比方源年长。

%84
方源自无不可,抱拳含笑道:“秦兄。”

%85
“哈哈,沙老弟。”秦百胜立即还礼道。

\end{this_body}

