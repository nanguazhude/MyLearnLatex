\newsection{星罗小仙}    %第十九节:星罗小仙

\begin{this_body}



%1
和平时期,蛊仙养蛊,对凡蛊需求量不高。购买只是次要手段,更主要的手段是福地自产。福地时间和外界时间有着流速差,外界一天,福地就许多天。外界一年,福地就过去很多年,每年都有产量。蛊仙们通常不会选择气泡鱼,因为价格太高。他们宁愿等一段时间。

%2
而战争时期,蛊仙之间交手次数多了,作为底牌的仙蛊并不常用,凡蛊作为寻常手段,消耗巨大。气泡鱼在五域大战中,也就成了各大蛊仙竞相追逐的物品。

%3
说起来,方源之前也买过气泡鱼,是为了星萤蛊的增产。

%4
可惜到了北原最后关头,他手中缺钱,不得不卖了几乎全部的狼群、狐群,大量气泡鱼以及毛民等等。

%5
方源本来有机会赚上一笔,但世事难料,结果反而因为卖得急,亏了一点本。

%6
现在狐仙福地中,气泡鱼少得很。

%7
“气泡鱼还是要买的,但却不是现在。过个几年,气泡海就会被修复好,气泡鱼的价格就会降低。到那时,我再投入仙元石,便可以买下一大群。也许我应该大量繁殖这种气泡鱼,等到五域大战时候,大发利市……”

%8
方源刚刚买下了一千多只星萤蛊,足以使用很长时间,不急切地需要气泡鱼群的增产。

%9
“阁下便是星罗小仙么?”片刻后,方源找到了他的目标。

%10
赶巧的是,他的这位目标也正在关注着宝黄天。

%11
“正是在下。前辈要买些什么?价格都好商量的。”星罗小仙的神念传来。显得十分热情,甚至还带点讨好的意味。

%12
方源看了下,她出售的这些货物。不仅量少,而且很常见,基本上无人会问津,显得寒酸窘困。

%13
她是蛊仙不假,但蛊仙也并非各个都富有。战斗、炼蛊、研究、渡劫,都会大大地消耗蛊仙的底蕴和财富。

%14
方源凭借前世记忆知道:这位星罗小仙,是一位才刚刚晋升的散修女蛊仙。底蕴浅薄,正处在事业起步的艰难时期。

%15
星罗小仙放在凡间,那绝对是顶天立地的大人物。引得凡人顶礼膜拜。但是放到宝黄天,各大蛊仙之间,却显得很不起眼。

%16
“谁又能知道,在五域大战时。这位星罗小仙逐渐绽放光彩。凭借杀招寒冰星尘起家,迅速积累战争财富,最终成为名噪一时的人物呢。”

%17
方源不禁感慨命运和时间的魔力。

%18
按照三尊说,五域大战,是大时代的开幕,这个时代将诞生大梦仙尊。

%19
每一个诞生九转蛊仙的大时代,都有一个共同点,那就是大战乱中。产生大量的新生蛊仙。

%20
新老交替,时代的浪潮激荡出一朵朵闪亮的浪花。流派之间。势力之间,个人之间的碰撞、交流,导致蛊道飞速发展,日新月异。

%21
方源就是其中之一。

%22
他以血道发家,成为中洲蛊仙。但可惜的是,只经历了这个大时代的前小半部分,他就不得不使用春秋蝉自爆了。

%23
“你这里有杀招卖么?”收拾情怀,方源问星罗小仙。

%24
“呃,我这里只有一个杀招。”星罗小仙有些不好意思,“这是晚辈偶然得到的冰道杀招,名为寒纱。售价半块仙元石,这可是五转杀招的最低价啊。”最后一句话生怕方源压价。

%25
方源看了一眼内容,不满意地沉吟道:“这个寒纱杀招,需要这么多的五转蛊,而且防御效果是大范围防护……”

%26
同时催动大量的五转蛊,就必须是蛊仙,凡人蛊师真元有限,基本上不可能了。

%27
防护范围大,就代表防御力要低于一般的单体防御杀招。而蛊仙往往是单人作战,除非是奴道。但奴道蛊仙,自有奴道防御手段,为什么偏要选冰道杀招呢?

%28
星罗小仙心中苦涩一笑,方源一语中的,正是因为这些原因,才令杀招寒纱极不好卖。

%29
但星罗小仙没有办法,她手中货物有限,又是散修,刚刚起步需要大量的修行资源。

%30
她虽然成功晋升蛊仙,但因渡劫两袖清风,大部分蛊虫都毁了。她设定了详细的计划,打算在自家福地中载种海量的抖抖树。她的福地环境,非常适合栽种抖抖树,但她缺少最关键的启动资金。

%31
她手头上有数十亿的元石,但却没有一块仙元石。

%32
抖抖树种的卖家,只有几个,不少需要换取特定的物资,要不就是仙元石。根本看不上星罗小仙手中的东西。

%33
听方源这么一说,星罗小仙的心中已经不抱有什么期望了。

%34
数月来,她已经习惯了失望。

%35
但下一刻方源却道:“我最近正在研究星道杀招,这件杀招寒纱,我有一点兴趣。但半块仙元石,实在太贵了!你这里若还有什么小杀招,比较新奇一点的,做个添头,也许我就花半块一起买下来。”

%36
“比较新奇的小杀招?”星罗小仙重新燃起希望,她立即想到自己的星道杀招“六九星尘”。

%37
“六九星尘”是她的独创,总用六只三转蛊,九只四转蛊,却可以打出超出五转蛊的攻击来。

%38
她曾经一度引以为傲,但后来眼界渐开,明白了天大地大,人上有人之后,便变得谦卑起来。

%39
星罗小仙只犹豫了一下,她心中对仙元石的渴望,便彻底地占据了上风。

%40
“晚辈这里的确有一个小杀招,只怕难入前辈法眼。”她小心翼翼地道。

%41
方源心中砰然一动,语气依旧平静:“哦?说说看。”

%42
看了六九星尘杀招后,方源按捺住心中喜悦,评价道:“嗯,杀招差强人意,但也许能带给我新的灵感。谁让我今天心情好,就照顾一下你这个新人罢。哈哈哈,这是半块仙元石!”

%43
“终于有半块了,还差半块!”星罗小仙大喜,分别时,她语气激动地祝福了方源许多好话。

%44
之后,方源花掉两块仙元石,买下大量的炼蛊材料、凡道蛊虫。

%45
又花去两块,引进一群老毛民。

%46
毛民在所有的奴隶价格中,几乎永远是最高的。

%47
这些老毛民年龄虽大,大几年后就会死去,但却都是炼蛊熟手。

%48
方源还打算炼几套星门蛊,但此蛊成功率很低,当年就连琅琊地灵出手,也多次失败。若让方源亲自出手,无疑会浪费他大量的时间,他准备将这些人物,都交给老毛民,让他们代炼。

%49
中洲,仙鹤门,伏虎福地。

%50
方正满怀严肃之色,步入地道。

%51
地道修葺得高大宽敞,每隔百步,都会有一只石人精英护卫,防卫可谓森严。

%52
这些石人,都被福地主人精心磨练,各个都至少是四转战力,尤擅土道蛊虫。

%53
而且每隔千步,都会有一位巨大石人,体格高大,是正常石人的三倍。如此宽阔的地道都显得狭窄,他们只能蹲在地上。

%54
不过石人天性就在地底挖洞生活、沉眠,对此这些石人都没有什么不适应的。

%55
每次方正经过这条地道,都会心生凛然。尤其是当他每每感受到这些巨大石人,身上洋溢着五转蛊师的澎湃气息。

%56
“咱们仙鹤门的实力,真的是太雄厚了!这些五转石人要统统放出去,估计要把其他门派都吓坏了。”在经过一个巨石人的身旁时,方正暗暗对寄魂蚤嘀咕道。

%57
寄魂蚤中寄生着天鹤上人的魂魄,道:“这倒不会。哪个门派当中没有深藏不漏的底蕴呢?尤其是其他九大古派,丝毫不弱于咱们仙鹤门。不过掌管伏虎福地的主人,乃是咱们门中的太上三长老虎魔上人。他培养的石人奴隶,堪称中州第一!在宝黄天中都独占鳌头,他的伟力和财富,非你我这些凡人能够想象。”

%58
方正不由心驰神往:“蛊仙究竟是什么样的境界啊,竟然能豢养这么多的石人,真是厉害!”

%59
天鹤上人答道:“蛊仙每一个都是天纵奇才的精英俊杰,人中龙凤,各有奇遇机缘。他们都是由凡人一步步修行上去的,方正你也已经是五转,也许将来有一天,你也能成为蛊仙呢!”

%60
方正摇头不止:“师傅,我可有自知之明。我的五转修为,大多都是门派用了舍利蛊堆砌上去的。门派精心栽培我,将资源朝我身上倾斜。我的五转,比不上师傅你一步步修行出来的五转。而且我现在觉得五转也不稀奇了,你看这地道中的巨石人至少有三十位。”

%61
“方正,你能记得仙鹤门对你栽培,有感恩之心,这很好。但你也不必妄自菲薄。人乃万物之灵,异人只是沾了了人字,你别看这些石人威武雄壮,实际上单对单,完全无法和门中的五转战力媲美,他们更擅长的是群体作战。”天鹤上人笑道。

%62
“是这样?”

%63
天鹤上人道:“这些石人很笨的,又赶巧碰到了一位富有的主人,才有这样的成就。你以为这些石人,能和门中的长老们相比么?若我生还,一个人就能对付十多位巨石人。好了,血池已经到了。”

%64
方正听到血池二字,浑身肌肉不由一阵发紧。

%65
他推开眼前的大门,进入一处洞穴。

\end{this_body}

