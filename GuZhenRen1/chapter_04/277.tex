\newsection{炼制星念仙蛊}    %第二百七十八节:炼制星念仙蛊

\begin{this_body}

狐仙福地,地下石窟。

血芝林已经蔓延生长,不知不觉间扩张到五十亩左右的范围。

在血芝林的中央,芝林散发的暗红色血光,此刻已被灿烂多姿的智慧光晕排开。

此刻,方源就沐浴在智慧光晕当中,闭目盘坐。

他神色平静,宛若雕塑,脑海中却是电光火石般,无数的念头此起彼伏,相互碰撞,激烈的宛若流星下雨,火山喷发。

一声闷哼。

方源身躯微震,缓缓地睁开双眼。

脑海中完全由念头组成的风暴,戛然而止。

他皱起眉头,沉吟不语。

两道血迹,从他的鼻孔间缓缓流下。

仙僵的血液,是惨绿之色,一丝温热也没有,十分冰冷。

方源满不在乎地将这两道鼻血擦干,心中评价:“看来我新创的这套蛊阵,还有不少缺陷。”

自从他成为智道宗师之后,就有了无数灵感。对于如何运用智慧蛊,更有了许多全新的想法。

这套蛊阵,就是他不久前独自创造的作品。

之前,他智道境界不足,只能直接运用智慧光晕。如今有了这套蛊阵,却是可以加剧脑海中念头碰撞的过程。

虽然不能提升推算思考的效果,但却节省了不少时间。

智慧蛊高达九转,方源区区六转仙僵,能蹭用智慧光晕,已经十分勉强。这套蛊阵。已然算得上一次难能可贵的突破。

并且这套蛊阵,已经不是最初的版本,而是经过了几次删改。

前几次,表现出了各种各样的缺陷,有一次蛊阵甚至在方源的体内直接爆炸。

这一次使用。方源承受蛊阵的反噬,受到了轻伤。

“蛊阵还需要修改……不过,这不是关键。现在的难题是星念仙蛊方的改良。”

方源眉头越皱越深。

许多天前,焚天魔女征召方源,方源就趁此良机,提出炼制星念仙蛊的要求。

双方达成协议,焚天魔女资助方源。十分大方爽气。炼制星念仙蛊的各种材料。她都主动替方源寻来,派遣卜单送过来。

这些仙材,十分珍稀。不仅花费甚大,而且若是方源亲自去搜集,肯定要耗费大量的时间。

唯有焚天魔女,依仗八转修为,以及在僵盟总部的地位。还有极深的底蕴,才能在如此短的时间内,帮助方源搜集到整整三份仙材。

方源万事俱备,自然着手炼蛊。

可是,他先后连续尝试了三次,尽皆失败。

对于失败,方源早有心理准备。

毕竟他这一次,没有炼蛊大会得来的成功道痕辅助,只能靠实力,拼运气。

“靠实力”可以理解。为什么还要“拼运气”?

就拿星念仙蛊而言,在炼制的过程中,有一个步骤,需要用到两份仙材。一份是苍炎白骨,一份是青蚨水莲。

前者是炎道仙材,后者是水道仙材。

到了这一步,需要将这两种仙材同时处理。水火交融,化为一份清澈见底的原液。

但这一步,十分困难。

苍炎白骨上有炎道道痕,青蚨水莲则是水道道痕。要处理成功,形成清澈见底的原液,必须是炎道道痕和水道道痕的数量,要相互一致。就算不一致,相差的也必须极小。这样才能水火抵消,若是一方的量多过一方,超出了容许的极限,形成的原液就是浑浊不堪的。

这种浑浊的原液,就不能使用。

偏偏原液,不可保存太久,需要随用随制。方源又没有宙道仙蛊,可以定住原液的时间。导致方源必须在炼制仙蛊的过程中,即时处理仙材,合成原液。

而道痕之间的差别,实在过于精微。就算是方源费劲心思和手段,也不能洞察全部,只能估计一个大概。

方源并非是炎道、水道的蛊仙,他已经做到了他查探能力的极限。

正因如此,仙蛊方中严格规定了青蚨水莲的年份,以及苍炎白骨的份量。

但即便是这样,同样年份的青蚨水莲,也会因为生长环境的优劣,水道道痕会有多寡之分。苍炎白骨若是受损,也会导致道痕的消褪。

所有的炼蛊材料,都经过方源的仔细审查,符合仙蛊方中的标准。

但真正实施起来,难免失败。

要炼出清澈见底的原液,不免要带运气成分。

而整个炼制仙蛊的过程中,类似这种水火交融的例子,比比皆是。

一次做好,两次成功,并不困难。困难的是,这些例子次次都要成功。

所以,炼制仙蛊的成功率极低。

很多蛊仙为了炼制仙蛊,倾家荡产,也没有成功。

仙蛊对于蛊仙的提升,是极其巨大的。

因为天劫地灾的存在,让蛊仙们对仙蛊的需求,更加强烈。

但翻开蛊修的历史,到处都是炼制仙蛊失败,血淋淋的教训,凄凉的下场。

当今,很多的六转蛊仙手中,都没有一只仙蛊。

蛊仙们宁愿捕捉野生仙蛊,也不愿意去冒险亲自动手炼制仙蛊。

当然,这其中还有一个原因,就是蛊仙之间的炼道造诣,也参差不齐。很多蛊仙都没有自信,去自己着手炼制仙蛊。

尤其是转数越高的仙蛊,蛊仙要炼制的自信就越缺乏。

想当初、巨阳、盗天两位尊者,都要请长毛老祖出手,帮助他们炼制仙蛊、仙蛊屋。

方源改良仙蛊方,就是改良的这些地方。

他成为智道宗师的同时,星道境界也暴涨到宗师境地。这让他有资本,去改良星念仙蛊方。

算上这一次推算,他已经将原本的仙蛊方,改良了许多地方。

类似水火交融的这种步骤,至少消去了十一处,全都替换成更加稳妥,更容易成功的方式。

“算算时间,焚天魔女就要送来第四次的炼蛊仙材了。”

有了焚天魔女的资助,彻底打消了方源之前的财力困窘。焚天魔女一心想要攻略玉露福地,大有方源不炼成星念仙蛊,誓不罢休的架势。

不过,她也不是好糊弄的。

若是方源想借着失败的名头,欺骗焚天魔女,哄骗炼蛊仙材,那是不可能的。

焚天魔女虽然答应资助方源,但之前也定下协约:资助过去的仙材,都是方源亏欠焚天魔女的。不管方源最终炼成仙蛊还是失败,方源都必须偿还欠款。并且时间拖得越久,焚天魔女还要收利息。若方源炼成了仙蛊,却不在一百年间将欠款偿清,那么焚天魔女就要收取方源身上的任一仙蛊,进行补偿。

这种条件,若是换做鲨魔,方源肯定不会答应,会提出更有利于自己的要求。

但面对强势霸道的焚天魔女,方源又一心想要尽快地炼出星念仙蛊,只好捏着鼻子认了。

所以,焚天魔女的资助绝不是那么好拿的。

方源失败了三次,是自己的亏损,已经欠下焚天魔女好大一笔外债。

焚天魔女的盘算,方源也略知一二。

就算将来方源还不起欠债,但表现出来的智道造诣仍在,焚天魔女便可利用此点,对方源进行要挟,让方源为她卖力。

毕竟,智道蛊仙十分稀少。像方源这等造诣的智道蛊仙,将是一个极好的手下。

若是方源成功炼出星念仙蛊,焚天魔女会更开心。因为她可以借此,攻破玉露福地了。

总之,焚天魔女稳赚不赔。

连续失败了三次,压力全在方源身上。

“依照我目前的情况,最多能够炼制七次。如今已经失败了三次,还剩下四次机会。炼蛊方面,我已经做到了极致。只能在运气方面,多做些努力。”

方源思量起来。

他手中有连运仙蛊,可以连接强运之人,来提升自己的运气。

但这一点,也有弊端。

方源虽有连运,但没有断运仙蛊搭配。

只能连,不能断。

若是连运的对象,倒霉了,走了霉运,也会拖累方源这边。

所以之前,方源才精心选取了一些目标。这些人物,运势极强,而且连绵不绝,长久不衰。

但这种人物,少之又少,整个五域,回顾前世,也剩下不了几个了。

方源心中盘算:“剩下的几个,大多数都还未降生。其中,无名海岛上的李逍遥,已经被宋亦诗杀了他的祖宗,今生根本无法出现了。最近的一个目标,就只有黑月仙子。”

黑月仙子十分了得。

乃是方源五百年前世,灵缘斋的当代仙子。代表着当前时代灵缘斋中最杰出的人物。

地位就和历史上的墨瑶一样,甚至还有超出。

这是因为方源五百年前世,乃是一个波澜壮阔的大时代。五域前所未有的乱战,梦境的出现,打破了旧有的势力格局,产生全新的蛊修流派梦道,一切都昭示着大梦仙尊的产生。

在这种大时代,英雄辈出,豪杰涌现,枭雄层出不穷,龙蛇翻腾,黑月仙子仍旧能代表中洲十大古派之一的灵缘斋,本身天赋才情,必定是五域一流。

并且,黑月仙子率领中洲蛊仙,攻打王庭福地,取得惊人成就,战力卓绝,手段高超。

自黑月仙子出道以来,一直越战越强,因祸得福,运势十分强悍。

的确是方源理想中的连运目标。

\end{this_body}

