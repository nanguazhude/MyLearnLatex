\newsection{取舍进退见真雄}    %第一百五十节:取舍进退见真雄

\begin{this_body}

%1
一共四只力道巨手,悬浮在半空中。♀♀,

%2
对面两位魔道蛊仙,不自禁地齐齐后退一步,方源带给他们的心理压力,实在太大了点。

%3
情势比人弱,为之奈何?

%4
魔道蛊仙脸色一阵变幻,终于还是认怂,将得到的长恨蛛都转交给方源。

%5
方源接手之后,立即飞走,再没有继续纠缠下去。

%6
两位魔道蛊仙顿时松了一口气。

%7
被敲诈的那位,脸上浮现出一片愤恨之色,他盯着天边方源渐渐远去的背影,咒骂道:“这个老魔头,我期待你死在那里!”

%8
“息声,息声,你怎么确保他没有侦听手段?小心他又回头找我们麻烦啊。”另一位魔道蛊仙连忙摆手劝阻道。

%9
两人对视一眼,齐齐一愣,旋即都迅速反应,电射出去,拉开距离。

%10
下一刻,两者激烈竞争,一言不发地向深潭出手,拼尽全力搜刮着里面的长恨蛛群。

%11
“方寸山,嫣红海么……”方源一路直飞,迅速向消息中提到的地点接近。

%12
他原本并未打算搜刮长恨蛛,因为他缺乏有效捕获长恨蛛的手段。

%13
长恨蛛群对他而言,价值不大,他已计划放弃。没想到一阵敲诈勒索,不仅节省时间,立即得到了一大批的长恨蛛,而且还获知了一个重要情报。

%14
方源私下揣测,觉得这个情报还是比较真实可靠的。

%15
小人,是异人中的一种。

%16
方寸山,就是小人生活的乐土。根基之地。这点在《人祖传》中,早有记载。

%17
小人背后生透明薄翼。通常只有手指头大小,以采摘花蜜或者汲取草木汁液为生。

%18
每一种异人。都有特定本领,比人族更强。例如毛民天生就擅长炼蛊,石人生活地底,擅长挖矿,甚至石人的身体上都会伴生各种矿石。雪民耐寒,墨人学识丰富……

%19
小人擅长栽培花草树木,有他们生活的地方,通常草木会非常茂盛,花朵长年盛开。四季不凋。

%20
“更关键的一点,是方寸山虽然体型微小,但却是货真价实的山峰,需要大量地气稳固山体。这等传奇名山,放在仙窍内,都是极大的负担。蛊仙不可能随时带着身上,更何况这方寸山也非东方部族中的某位蛊仙私人所有。”

%21
蛊仙的仙窍,藏于体内,里面的天地二气都是有限的。方寸山若长时间存放在这样的仙窍中。势必会大大地汲取地气,维持传奇山体,导致仙窍的天地二气失衡。

%22
因而,方寸山大部分时间。都是落在碧潭福地当中。

%23
为什么呢?

%24
因为碧潭福地常年种在地上,时刻汲取北原外界的,绵绵不尽的天地二气。根本不惧方寸山的负担。

%25
方源的荡魂山,同样也是如此情形。

%26
荡魂山需要汲取地气。否则山体就要崩溃倒塌。所幸狐仙福地种在天梯山上,时时刻刻汲取着天地二气。并不惧怕荡魂山的消耗。

%27
若放在黎山仙子、黑楼兰的仙窍内,四处移动,时间一长,就算八转蛊仙也吃不消,需要将仙窍种于外界,汲取外界二气,稳固仙窍小世界。

%28
一处看似普通的深潭边。

%29
“是方源?”

%30
“他怎么会过来这里?”

%31
黎山仙子、黑楼兰隐形匿迹,方源的出现,被她们察觉。

%32
二女起先私密交流,去寻找茅草屋。结果茅草屋,早已经不在原来的地点,但二女亦有重大收获在那处深潭中,搜寻到了一群冥鳝!

%33
之后她们辗转各处,终于发现了方寸山的踪迹。

%34
来到这里后,却被阻挡,正为此为难,相互商讨之际,方源出现在她们的警戒范围之内。

%35
黎山仙子还留有希望,对黑楼兰道:“方源侦察手段不强,你我隐身乃是仙道杀招,我们不主动现身,他肯定觉察不出。也许他只是路过而已。”

%36
黑楼兰吃过方源的大亏,眉头皱起,有不妙的预感:“唉,决不可小看方源这个家伙。”

%37
果然,下一刻,方源缓缓从高空中降落,迅速向她们接近,目标直指此处深潭。

%38
黎山仙子的幻想被打破,脸色一沉。

%39
方源来到情报所言之地。

%40
眼前的这片深潭,十分普通,毫不起眼。

%41
但方源仍旧略有期待。

%42
按照之前那位魔道蛊仙所言,这是一处幻境。里面有着幻界桃树,和荒兽等同,有营造幻境,保护树身的能力。

%43
只有蛊仙真正踏入其中,才可发现端倪。

%44
方源撑起防护,正待进入,这时两道身影,在不远处陡然现出。

%45
“方源且慢。”黎山仙子立即传音。

%46
黑楼兰沉默不语,站立在一旁。

%47
“是你们俩啊!”方源目光一闪,神情似笑非笑。

%48
语气略带嘲讽,身为盟友,居然有如此重大的线索,却不告知自己,这明显是想独吞好处。

%49
结果却被自己无意撞见。

%50
黑楼兰乃是女中枭雄,丝毫不见尴尬:“方源你来得刚刚好,我们正在为如何攻打这里犯愁呢。”

%51
黎山仙子是老资格的蛊仙,脸皮也很厚实,接着道:“没错。这里被东方一族命名为嫣然海,乃是碧潭福地中的禁地。主体是一片幻界桃林,桃树众多,幻境深重。树下,是草原花海。大片大片的席地草,宛若无边绿毯。深邃的花海,姹紫嫣红,花香若腻,数万株紫毒腐心花,正要开花。”

%52
“哦?”方源目光微微一凝。

%53
之前那位魔道蛊仙,包藏祸心,只说是有幻界桃树,却没有提到数量。至于席地草,紫毒腐心花。根本提都没有提。

%54
席地草本身很普通,但草的数量如此众多。肯定有席地草蛊。这种木道蛊,只有一个功效。就是能叫生物顷刻进入梦乡。

%55
紫毒腐心花中,往往会产出一种腐心蛊。此蛊无色无形,乃是一缕香气,让人闻之香气若醉,却腐蚀心脏。

%56
不过这些对于方源而言,只是麻烦而已。

%57
方源虽然没有涉猎毒道,且力道修行中也没有防毒手段,不过他本身就是仙僵之体,十分耐毒。他的身体中。还藏着一只毒道仙蛊妇人心呢。

%58
黎山仙子见方源虽然仔细聆听,表情却无动于衷,便又道:“以上之物,数量繁多,但对我等蛊仙而言,却只是有些麻烦。但事实上,幻界桃林中,栖息着一支桃太狼群,数量有近三十只。而这处嫣然海其实是一座蛊阵。最深处便坐落着方寸山。方寸山上有蛊仙镇守,气息不定,数量不明。”

%59
方源这才动容。

%60
桃太狼乃是荒兽,媲美六转蛊仙战力。一群桃太狼。就等若有一群蛊仙。

%61
方寸山是小人的大本营,山上的蛊仙估计就是小人蛊仙,不管数量多少。只要有一位就相当棘手。

%62
荒兽智力低弱,但异人蛊仙已然不输给正统人族蛊仙。

%63
有无智慧。威胁性完全是两个档次。

%64
方源心中恍然:难怪二女躲在外围。

%65
他又打量眼前二位,方源目光自然老辣。顿时发现黎山仙子、黑楼兰身上尽管做了处理,但仍旧可看出一些辛苦激战的痕迹。

%66
“黎山仙子乃是木道蛊仙,不用猜都知道,方寸山、小人族对她的吸引诱惑无比巨大。在嫣然海禁地这样的环境中,黎山仙子战力要超过平常状态。她们突进到花海深处,结果仍旧被打退回来。嫣然海防卫森严,还要大大远超我最乐观的估计。”

%67
方源不动声色,精明的脑海中则电光激闪,透过表象,思考出许多真相。

%68
“方源,你要单独行动,恐怕收效甚微。我们刚才也是趁着对方疏忽,才在桃太狼群中找到缝隙,偷偷潜入进去的。如今方寸山被惊动,里面肯定戒备森严。只有你我精诚合作,一起联手,才有可能吞下这块肥肉。”黑楼兰开口劝道。

%69
黎山仙子也旋即附和出声:“方源,我知道你有力道仙蛊拔山。咱们可以配合,方寸山我势在必得,你也不用担心,凭我的底蕴一定会给你满意的报酬。”

%70
方源没有立即答应,而是低头思索,陷入沉默。

%71
几个呼吸之后,他缓缓摇头,打破沉默,苦笑道:“这块肥肉之下,却是最硬的骨头。方寸山虽好,但时间有限,我并不想在此处死磕。我劝二位也赶紧打消不切实际的想法,外面的资源那么多,很容易就收取,何必冒险呢?”

%72
“什么?方源你居然想退出?”黎山仙子听到这话,感觉难以置信。

%73
黑楼兰皱起眉头,劝说道:“方源,我们既然发现了方寸山,那就说明此物就是与我等有缘。错过如此良机,今后你会懊悔的。”

%74
方源态度坚定:“这里防御森严,关键是有蛊仙镇守,还有一群桃太狼。要攻破这里,恐怕就算是全部的魔道蛊仙,也要耗时日久。东方长凡会给我们这个时间吗?就算东方长凡被绊住,其他的魔道蛊仙呢?”

%75
黎山仙子前进一步,张口:“方源,我承认你说的在理,但这可是方寸山呐。它的价值,恐怕整个碧潭福地都比之不上!”

%76
“哈哈哈,仙子,你不必再劝了。告辞!”方源哈哈一笑,迅速返身,划破长空,电射而去,竟是如此干净利落。

%77
留下二女仙,兀自不敢相信,甚至还有些犹疑:“这方源会不会是耍滑头,故意诓骗我们俩,好等到混乱时机来捡便宜?”

%78
“不太可能。我们可是签订了雪山盟约的,他要么不说,说出来的都应是真心话。”

%79
一时间,望着方源快要消失在天际的背影,二女的脸色都浮现出异色。

\end{this_body}

