\newsection{芝林、地马}    %第三十一节:芝林、地马

\begin{this_body}



%1
洪易早已决定,用过得去蛊对付对方的防御蛊。

%2
只要对方的防御蛊不是同时催动两只,洪易突袭之下,轻轻一击就能打中对手肉身,将其重创,进而击败。

%3
若对手有两只防御蛊同时催动,洪易就不急着进攻,采取消耗持久战术。对方同时催动两只防御蛊,真元消耗速度一定比洪易快。

%4
可以说,洪易有此蛊在手,优势极大,十有八九都能获得此次大比的第一。

%5
“只要得到第一,将娘的牌位迁回祖宗祠堂,便是名正言顺了。嗯?”

%6
忽然间,洪易眉头微皱,刹那间他感觉心头一空,仿佛什么珍贵重要的东西,被人偷走了似的。

%7
“难道是多年的愿望即将实现,让我患得患失起来了吗?我孑然一身,除去过得去蛊,等若一无所有。又能有什么东西,值得别人夺取的呢?”

%8
洪易摇摇头,苦涩地笑了一声,将刚刚的不妥感受遗忘到了一边。

%9
远处的山谷内,方源缓缓吐出一口浊气:“连运成功了。”

%10
这一次连运,他特意抓紧时间,在狐仙福地稍稍休整了一下,就立即赶了过来,因此没有再受到任何的意外干扰。

%11
方源动用察运蛊,一直看着洪易的运气。

%12
他的运气是乳白色,也很特别,给人饱满,蕴藏生机之感。运气凝聚成一团,形成一个书生坐卧读书的高大姿态。书生的面貌和洪易本身,有着三分相似。神采飞扬。

%13
但现在和方源连运之后,这个白色书生气运立即萎缩,连原来的三成大小都没有。

%14
而且原本书生锦缎一般流畅的衣袍。变得衣衫褴褛,四处补丁。仿佛是富家公子少爷,落魄成了穷人子,弟寒酸书生。面色上也不那么神采飞扬,反而脸庞削瘦,神态间充盈阴郁,仿佛怀才不遇;

%15
“看来连运这些目标。果然有好处。我之前和韩立连运,差点让他噎死。这一次和洪易连运,却没有给他带来死运。可见我本身的气运。的确改善了很多。”方源暗自欣慰。

%16
正当他要离开这里的时候,忽然心中微微一动。他发现了两个鬼鬼祟祟的蛊师,正偷偷地在这处山谷中潜行。

%17
方源之前为了连运,在众生书院的山谷中。布置了许多蛊虫。一方面遮掩连运时的仙蛊气息,另一方面也侦察周围环境,方便提前示警。

%18
可以说,整个山谷都在方源的监控之中。

%19
方源闭目感知。

%20
这两个鬼祟的蛊师,实力还不小,均是五转蛊师。

%21
他们在山谷的另一边,偷偷前行,并且小声交谈着。

%22
“范医掌门。你确定这处山谷地下,有芝林、地马?”

%23
“袁白谷主。错不了的,我是亲眼所见!你别着急,跟着我进入山洞,待会亲眼看到了,不就确信了吗?”

%24
“哼,这里可是众生书院的地盘。若是被众生书院的院长洪玄机发现,咱们俩可就吃不了兜着走了。”

%25
“洪玄机虽强,但侦察手段欠缺,要不然早就发现这近在咫尺的大好资源。我们有心算无心,怎么可能被他发现?”

%26
“唉,一旦被发现,我们两大门派头领一起夜探众生书院,名声脸面就全毁了。”

%27
……

%28
方源将这两人的对话,尽收耳中。

%29
“芝林,地马?”他心头一动。

%30
这芝林长在地底洞窟当中,往往幅员数十里。根根灵芝硕大如树,芝肉肥腻饱满,可以喂养蛊虫,可以当做炼蛊材料售卖。

%31
地马则算是芝林中的土生土长的异兽,一片芝林中往往只有两三头,以家庭组织在一起。地马之蹄,乃是五转遁地蛊的主材之一。地马之眼,可以炼制透视蛊。地马的尾毛,可以炼出烟尘蛊。

%32
地马可谓浑身是宝,难怪引起袁白谷主、范医掌门这两位五转蛊师的觊觎了。

%33
不过对于方源来讲,地马、芝林并无多少吸引力。

%34
地马是异兽,充其量相当于万兽王。地马一家,也不过两三头万兽王而已。芝林在宝黄天中卖的人很多,是普通货物,方源要买根本都不需要用仙元石。

%35
“不过对于众生书院来讲,地底下的大片芝林,却是牵扯极大利益的重要资源。可惜他们现在,还被蒙在鼓里。”

%36
方源想了想,还是远远地跟在两位五转蛊师后面,顺利地进入地下芝林。

%37
果然和他估计的一样,这是一片普通的芝林,幅员近十里地;

%38
。三头地马,正是一家三口,居于芝林当中。

%39
“好大一片芝林啊!”袁白掌门行走在芝林当中,连声感慨。

%40
“这是一块无穷的宝地,为什么不出现我的山谷中呢?唉!”范医谷主深情地抚摸着灵芝肥嫩的树干,语气嫉妒。

%41
方源绕过他们俩个,来到芝林的最中央。

%42
这里长着一株最大的灵芝,有两丈多高,直接顶着洞壁。灵芝肉叶像是一柄巨伞,覆盖老大的一个圆。

%43
这是灵芝王。

%44
方源低着头,来到灵芝王的跟前。

%45
灵芝王的身边,有三头地马护卫,同时还有噬金蚁群盘踞。蚁群中多有金道野蛊,灵芝王的身上还有野生的木道蛊虫。

%46
但这些都是凡物,方源毫不掩饰自己的仙僵气息,径直来到这里,不受丝毫阻挡。

%47
方源掏出一只怪爪,窥准位置,狠狠地抓破灵芝王的树干。

%48
灵芝王剧烈颤抖,带动头顶洞壁碎石坠落。地马一家被方源的仙僵之气所摄,不敢靠近这里,只能在远处不断呜咽凄鸣。

%49
方源掏了一会儿,抽回自己的僵尸怪爪。

%50
只见怪爪中。拿捏着一颗心脏。

%51
这心脏真是灵芝王的心,嫩嘟嘟,热烫烫。表面是乳白色,由纯粹的灵芝肉形成,散发着一丝肉香。

%52
“只要将这灵芝王心栽种下去,不久之后,便能形成一小片芝林了。”方源也算顺手牵羊。

%53
狐仙福地的土质,不适合栽种大规模的芝林。但存活这一小片,也并非难事。当然。也牵涉不到多少利益。

%54
方源纯粹是出于玩乐之心而已。

%55
至于这一大片芝林,还有地马,方源也看不上。他要迁移进狐仙福地。也会非常麻烦,需要的时候,直接在宝黄天收购最方便。

%56
失去了心脏,灵芝王也不会死。只是芝林不在蔓延生长。至少得休整十多年。

%57
“什么声音?”远处,袁白和范医迅速对视一眼。方源取走灵芝王心,造成的动静不小,而且还伴有地马的哀鸣。

%58
但是当袁白和范医二人,赶到灵芝王前时,方源已经消失无踪,离开了这里。

%59
留下的地马一家,将愤怒的情绪尽数发泄到袁白、范医二人身上。

%60
剧烈的激斗声音。很快惊动了众生书院。

%61
洪玄机率领一众蛊师,赶到这里。又惊又喜。

%62
袁白、范医二人则在心中大叹倒霉。

%63
一番打斗之后,洪玄机驱逐了袁白、范医二人;

%64
。对狂喜中的书院长老们下令:“芝林事关重大,从即日起,停止本门大比,同时召集外出任务的弟子、长老。接下来必将是艰难时期,袁白、范医等人一定会纠集更多门派向我们施压。我们一定要守住这处芝林,这是我们众生书院崛起的基石!”

%65
“是,院长大人!”众蛊师轰然应命。

%66
而另一边,熟睡中的洪易还被蒙在鼓里,并不知道他期盼的门派大比,已遭受变故而取消。

%67
他心中关于迁回其母牌位的梦想,实现之期又要往后大大推迟了。

%68
哗哗哗……

%69
水浪滔滔,湿润的风扑打在方源的脸上,但如此轻微的力量,没有带给方源任何的感触。

%70
这里已经不是中洲,而是南疆。

%71
眼前的这道大江,就是南疆第二江河碧龙江。

%72
南疆有三大江河,第一是赤龙江,第二是碧龙江,第三是黄龙江。

%73
赤龙江水猩红如血,碧龙江上绿波荡漾,黄龙江则是方源今生曾经漂流过的,黄褐沉沉,泥沙最多。

%74
方源手中把玩着灵芝王心,取走这颗心后,他没有直接动用定仙游离开。而是转移出去后,来到山谷的一个角落。催动定仙游前,又留下许多蛊虫。

%75
这些蛊虫将在他离开之后,接连自爆,消弭掉定仙游遗留下来的气息。真正做到神不知鬼不觉。

%76
现在,他在等一个人。

%77
这个人也是他的目标之一,拥有很大气运,奇遇连连,从卑微走向传奇。五域乱战时的七转蛊仙,和韩立、洪易一个层次的风云人物。

%78
但和后来屡次加入家族的韩立,或者加入门派的洪易不同,这个人从始至终都是散修。身边有几位亲朋好友,但从来都未加入某个势力。

%79
到了方源自爆前夕,曾有谣言纷起,说中洲天庭准备吸纳他。

%80
至于这个消息的真假,方源也无从查明。

%81
不过方源既然已经重生,这都并不重要了。

%82
“记忆中,这个人被小人排挤,家族驱逐之后,流浪到这里。在这江边,得到了一位四转蛊师的遗藏,正好足够他用。以此遗藏,他维持生计,抵御野外危险。进而一步步顽强成长,之后抓住大机缘,竟然得到玄黄母气蛊,又得天师传承,赌石成功十有八九,横扫南疆各大赌石场……”

%83
方源回忆着。

%84
他之所以知道这处地方,是来源于。每一份重大人物的传记,向来都是图文并茂。

%85
叶凡就是方源这一次的目标人物。

%86
而这里,就是叶凡的起步之地,他于今夜收获了第一份奇遇。

\end{this_body}

