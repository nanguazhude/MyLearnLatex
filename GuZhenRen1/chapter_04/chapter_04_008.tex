\newsection{东方长凡方寸山}    %第八节:东方长凡方寸山

\begin{this_body}

%1
智慧光晕,照得洞穴石壁五颜六色。

%2
方源静静地站立期间,良久,他睁开双眼,缓步退出光晕的范围。

%3
脑海中的意志,又几乎消耗一空。凡蛊催生出来的意志,有限得很,难以和智道仙蛊比拼。

%4
“不过我这次推算仙鹤门图谋失败,却非意志不足,而是线索不够。”方源在心中反思。

%5
这些天来,他多次蹭用智慧之光,也总结出了经验。

%6
智慧蛊和师法自然不同。

%7
智慧蛊是散发智慧之光,给予蛊师无穷无尽的灵感。但思考的结果,是基于使用者的底蕴,以及推算的证据因素。

%8
方源刚刚推算仙鹤门对狐仙福地的图谋,失败了,就是因为收集的证据太少。

%9
总结下来,使用者底蕴越深,收集的证据越多,推算就越能得到正确的结果。

%10
当然证据越多,需要参考的因素越多,消耗的念头、意志也会越大。尤其是推算仙蛊方,考虑的因素过多,消耗的意志便会极其巨大。

%11
而师法自然呢,它是蛊师升仙时的珍贵经历。

%12
在这个过程中,蛊师询问天地,天地就直接给出答案,不需要蛊师思考。

%13
拿地球上的数学难题举例,譬如要求解一个多元方程式。

%14
蛊师若用智慧蛊,就是念头极速消耗,灵感勃发,推导出许多解题途径。这些途径,有些是死胡同。有些只能推算出错误结果。少数则是正确途径,能推算出正确结果。

%15
而师法自然,则相当于直接给出正确答案。

%16
使用者得到正确答案。但不知道是怎么求解出来的。

%17
因此,师法自然时,不能询问天地太过高深的问题。方源不会直接询问如何永生,也许天地会给出答案,但他绝对理解不了。

%18
就好像是还未学数学的幼儿,看到了一个线性方程式,当然不会理解这个答案。

%19
要理解这个答案。幼儿得学习每一个数字,未知数,正负数等等。但这仅仅只是基础中的基础。

%20
对于蛊师而言。太过高深的问题,哪怕得到了答案,也无法理解,无法利用。

%21
就算接着师法自然。询问关于这个难题答案的方方面面。学习能帮助理解的基础,那这样一来,耗时太多,信息量也会极为庞大,脑海都可能承受不住。

%22
所以,最充分利用师法自然的方法,就是根据自身实际,稳扎根基。层层递进。

%23
“不过,智道当中。也有直接能令蛊仙直接得到答案的仙蛊。那便是天机蛊。蛊师无须念头碰撞,不用推导,就能够抓住一缕天机,直接得到答案。历史记载,此蛊乃是乐土仙尊开创,他曾经拥有过八转的天机仙蛊,死后天机仙蛊就消失无踪了,再没听说过相关的消息。”

%24
方源忽然想到了天机仙蛊。

%25
蛊虫效用单一,就算是九转智慧蛊也不例外。蛊虫转数越高,威能便越大,智慧蛊若真催动起来,效能一定恐怖。但就算再强,也达不到天机蛊的效果。

%26
“传闻乐土仙尊,是最能顺天应命的九转蛊师。天机蛊,便是他为了感应天地而创,也是他一直企图再次师法自然的尝试成果。若是我有天机蛊,就能和智慧蛊形成绝妙的配合。若是他人用天机蛊测我,恐怕我的一切秘密都会暴露……”

%27
天机蛊是令蛊师直接询问天地。

%28
只要是在天地中发生的事情,就没有天地不知道的。

%29
正所谓天知地知,你知我知。

%30
至于其他智道蛊虫,方源倒是暂时无忧。皆因大同风同化一切,方源在王庭福地的一切作案痕迹,都被抹除得干干净净。

%31
没有天机蛊,智道蛊师按部就班地去推算,就要收集证据,且证据越多越好。

%32
他们无法从王庭福地搜刮关键证据,就只能从王庭之争入手。如此一来,要推算出方源的真正身份,也不是不可能,只是概率小,需要的时间也多。

%33
北原,碧潭福地。

%34
从高空鸟瞰,福地中布满大大小小的深潭,青碧交接,成千上万。

%35
以单于部族的蛊仙童祖为首,北原正道的近十位蛊仙,从高空处悠然而落。

%36
“贵族的碧潭福地,果然别致清新。这万千碧潭,宛若天空繁星,美不胜收。”一位女蛊仙,此刻漫步长空,有感而发。她长发如披风,往后拖出两三丈之长,引人瞩目。

%37
“这碧潭中,皆是不同水质,孕养无数水族。东方部族的鱼群、水草,在宝黄天中都是出了名的。”一位男蛊仙,额头上睁着第三只竖眼,此刻眼蕴奇光,不断扫视碧潭。

%38
“慕容青丝、关神照二位大人谬赞,我族的碧潭福地只经营了六千多年,论盛景比不上慕容家的阴阳福地,论产生更不及关家的十荒福地。”在前领悟的蛊仙东方一空,微笑从容而答,带着东方部族特有的谦虚语调。

%39
碧潭福地,乃是超级势力,黄金部族东方家的大本营。

%40
一群蛊仙徐徐落地。

%41
这是万千碧潭中,并不起眼的一处。

%42
碧潭旁,建有一栋小茅屋。

%43
茅屋前,一位老者正坐于潭边垂钓。

%44
“这竟是冥鳝。”关神照额头竖眼一扫,看清深潭中的景象,失声喃喃。

%45
其余众人闻言,一些人的脸色也不禁微微动容起来。

%46
“东方先生,我们已经有十多年没有见了。想当年,我们共闯大雪山,并肩而战,时光匆匆,一晃即逝啊……”众人站定,修为最高的蛊仙童祖首先开口。

%47
“童祖大人,你仍旧是风采依旧。而老夫却是行将就木,不敢和您媲美。”垂钓老者缓缓站起,正是东方长凡,北原第一智道蛊仙。

%48
东方一空站到长凡的身旁,后者道:“今日东方家蓬荜生辉,能得诸位贵客光临寒舍,不胜荣幸。诸位,里面请。”

%49
东方长凡将众人引入茅屋。

%50
茅屋中空间广阔,有广场,有大殿。广场上,有十六座石像,排列似有序似无序,相距不一,神情微妙,仿佛暗含某种奇妙规律。

%51
“这就是东方家的六转仙蛊屋——凡草屋么?果然不同凡响。”当即便有人赞道。

%52
东方长凡将众人引入议事殿,一一坐定。北原正道蛊仙,各方势力代表,济济一堂。

%53
东方长凡坐在主位上,环视众人,微笑道:“这一次老夫主动邀请诸位同道前来,是想和诸位做一个交易。”

%54
关神照等人相互对视几眼,并不言语,仍旧是童祖开口回应道:“东方先生已经有数十年,不为他人测算。但在信中却说可以为我等破例,推算真阳楼倒塌真凶,或者其他方面。不知道这场交易,我们又要付出些什么?”

%55
东方长凡咳嗽几声,叹息道:“老夫寿元将尽,心知续命无望。但临走之前,心有牵挂。一为部族大业,二为后人子孙,希望和诸位做一场交易。老夫为你们每一家推算一次,老夫死后,请诸位同道分别与我东方家结盟。”

%56
“结盟?”

%57
“是的,详细的盟约老夫已经准备了大概,诸位可以浏览查看。有什么异议,便可当堂修改。”东方长凡取出两只东窗蛊,交给众蛊仙。

%58
蛊仙们接连浏览,结盟的条件十分宽松。东方长凡甚至没有要求盟友之间守望相助,只是要求在他死后,保东方部族五十年太平。期间,各方势力不得针对、打压、攻伐东方部族。

%59
众蛊仙不由地砰然心动。

%60
东方长凡乃是当代北原,公认第一的智道蛊仙。没有谁会比他更有可能,推算出真阳楼倒塌的真凶!

%61
蛊仙修行,谁不会遇到一些修行上的难题,或者蛊仙残方需要推导?有东方长凡出手,这些难题都有可能解决。

%62
更关键的是,王庭福地破灭,八十八角真阳楼倒塌之后,大量的蛊虫被无相拳带到北原各地,许多地方出现仙蛊气息,引得各方蛊仙竞相争抢。

%63
这个时候,只要让东方长凡推算仙蛊出现的位置,那么就很可能得到仙蛊!

%64
众人心动之余,不禁也在赞叹东方长凡的精明。

%65
若换做平时,东方长凡的付出就有些轻了。但现在,一次推算机会很可能代表着一只仙蛊。这让众蛊仙都为之心动不已。

%66
“好,东方先生,你的条件我们单于部族答应你了。”

%67
首先是童祖开口,其后黑家、慕容家代表,皆答应下来。

%68
这个情形,不出东方长凡的意料。他笑容更盛:“既如此,那就请黎山仙子出手。”

%69
“黎山仙子?”

%70
众人讶异间,便见一位六转女蛊仙从后堂走来,出现在众人面前。

%71
“黎山仙子,别来无恙乎?”当即就有人打招呼道。

%72
又有蛊仙笑道:“东方先生果然准备充分,连黎山仙子都请来了。只是这碧潭福地,可没有什么山川,如何让我们起誓啊?”

%73
这时,一直默不作声,站在旁边的蛊仙东方一空开口道:“诸位勿忧,且看这里。”

%74
说着,他展开手掌,掌心一阵流光溢彩,显露出一座微型山峰。

%75
“这难道是——方寸山?!”慕容青丝瞪大秀眸,首先反应过来,不可思议地叫道。

%76
“正是。”东方一空谦虚而笑,骨子里的骄傲却是掩藏不住。

%77
众蛊仙一片哗然。

\end{this_body}

