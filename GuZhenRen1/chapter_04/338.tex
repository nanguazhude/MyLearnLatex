\newsection{星雾笼身掩踪迹}    %第三百三十九节:星雾笼身掩踪迹

\begin{this_body}

东海,双极海峡。

波涛声透过厚重的浓雾,隐隐传入海峡之中。

宋甲丹盘坐在悬崖峭壁之上,闭目养神。

他的下半身,皆是灰色的石头。而上半身则是老者模样,皱纹丛生,苍老不堪。

忽然,他的眼皮子动了一下,感应到了宋家太上大长老,八转蛊仙宋启元的令牌。

宋甲丹念头一动,迷雾涌动,露出一段空白的甬道。

片刻之后,从甬道中,走来两个身影。

一男一女,皆是蛊仙。

女仙身着白衣,如莲花仙子,清纯动人,灵动的双眸顾盼间又透着一丝活泼气息。男仙则是中年模样,老成持重,走在女子身后。

“甲丹族叔!”还在远处,宋亦诗就高声叫道。

宋甲丹想露出一丝微笑,但终究不成,他已经不能动容,脸部肌肉已经完全僵死。

他的智道修行,十分特殊,能借助天意推算事和物,往往十分精准。

但有利有弊,弊端就是宋甲丹从此失去自由,下半身石化,已经和双极海峡融为一体。等到他寿元耗尽,他的整个身躯都会化为一块石头,再无生息。

最终,宋甲丹只能面无表情地低缓出声:“小诗,夏麒表弟,好久不见。”

来到双极海峡,拜访宋甲丹的正是宋家宋亦诗,以及宋夏麒。

“甲丹叔叔,这是爷爷给的令牌。您先看看。”宋亦诗首先将手中令牌,交到宋甲丹的面前。

宋甲丹查明无误之后,点点头,问道:“这一次,你们想算什么?”

宋夏麒开口道:“这一次,我们是为了登天野中的天难传承而来。”

宋甲丹沉吟道:“登天野中宇道道痕繁多杂乱,又随时变化。我不能亲临,就不能因地制宜,及时算出路径。不过若来家、蔡家。也无利害的智道蛊仙。相信你们凭借我的智道仙蛊,当能拔得头筹。”

宋亦诗轻笑道:“甲丹叔叔,登天野的事情有了新的变化。”

“哦?”

当即,宋亦诗就将焚天魔女、方源二人。闯入登天野的事情说了一遍。

宋甲丹立即领悟到了宋亦诗、宋夏麒的来意,恍然道:“原来你们是想雇佣这位智多星,帮助宋家取得天难传承。”

“不错。智多星并非僵盟中人,只是太白云生的好友,他是北原蛊仙。因为地潮来到东海。他既然能为僵盟做事,自然我们宋家也可雇佣他。他的智道造诣相当不错,若是我们宋家慢一步,被其他两家先得手,那情势就很被动了。”宋夏麒道。

“所以你们这次来,是想让我推算这位智多星的事情?”宋甲丹接着道。

宋亦诗连连点头:“真是什么都瞒不过叔叔你啊。不错!我们既要雇佣智多星,必然要对他知根知底些方好。毕竟他不是自家的族人。最关键的是,对方也是智道蛊仙,他若是耍滑头弄阴谋,我们宋家不仅要吃亏。而且还要丢面子。所以这次来,还请叔叔施展手段,看看能不能镇住这位北原智道蛊仙。”

宋甲丹垂下眼帘,眼中闪过一丝犹豫:“可是这智多星陈道,能破解了我为鲨魔施展的手段,又能在登天野中大放光彩,就说明他本身不仅有智道境界,而且智道仙蛊也不缺乏。这等智道强者,我若冒然推算他,恐怕有违我们智道蛊仙间的规矩。稍有不慎。还会引起对方的敌意,为家族惹上一个智道强敌。”

前世,方源伪装星象子,因为偷看了宋亦诗洗澡在先。所以宋甲丹师出有名。

这一世,方源伪装成智多星,并未留下什么把柄,更关键的是展现出高超的智道手段,让宋甲丹都生出了忌惮之心。

宋亦诗和宋夏麒对视一眼,前者不再开口。由后者道:“凡事总会有利有弊,只是个取舍。我们来之前,已经和太上大长老、二长老都商量过了,请甲丹表兄出手。”

宋甲丹点点头,他虽是智道蛊仙,但并非家族的决策者。

既是如此,他便再不迟疑。

酝酿片刻之后,就立即动手。

良久之后,宋甲丹缓缓地睁开双眼,眼中流露出一丝疲惫之色:“我算不出来。”

“什么?叔叔,你是当今东海最强的三位智道蛊仙之一,怎么可能算不出来?”

“难道这个陈道,在智道方面上,真的如此之强?已经和表兄能相提并论了?”

宋亦诗、宋夏麒都感到十分惊讶。

宋甲丹缓缓摇头:“人心有限,天意难测。或许这陈道真的很厉害,将我的推算手段尽数防住。也或许他并不在东海。”

“不在东海,还能在哪里?地潮已经消退,距离下一波地潮,还有很长一段时间呢。”宋亦诗十分疑惑。

“每一个仙窍,都是一个小天地,隔绝内外。我修为不过七转,推算不出,也不奇怪。”

“这样啊……”宋亦诗喃喃自语,有些丧气失望。

宋甲丹话锋一转,道:“我更在意的是登天野本身。此地能沟通白天、黑天,是战略要地,惹人觊觎。这一次,焚天魔女虽然撤离,但却没有杀住她的威风。如此,更会助长一些有心人的贪欲之炎。东海的散仙、魔修,当要注意。而其他正道超级势力虽然不好直接插手,但却也可以隔山打牛,搅乱登天野的局面。这才是目前潜伏着的最大危机。”

“表兄真知灼见,此言甚是有理!”宋夏麒面色一变,连连点头。

宋亦诗脸上也浮现一抹忧色。

“让我算智多星陈道,还不如谋算此事。你们放心,只要按我说的布置,定能打击来犯的散仙和魔修,甚至能透过他们的力量,削弱若来家、蔡家的实力。”宋甲丹自信十足地道。

……

中洲,狐仙福地。

“星雾掩!”

方源念头一动,催起仙道杀招。

顿时,一股朦胧的星光雾气,从他的第二仙窍中徐徐涌出。

很快。星雾笼罩住方源的全身,使他的面貌都模糊不清。外人只能透过星雾,看到雾中方源隐约的身形。

“推算改良的星雾掩,看来是成功了!”

方源检查一番之后。心中很是欢喜。

这招星雾掩,可以防备他人针对自己的推算。

方源离开登天野,和焚天魔女谈妥之后,就回到狐仙福地,利用智慧光晕。将原版的星雾掩进行了改良。

原版的星雾掩,核心仙蛊有两只,其中一只便是星念仙蛊。

改良之后的星雾掩,核心仙蛊也是两只。星念仙蛊仍在其中,还有一只则被方源替换为解谜仙蛊。

如今方源有智道宗师的境界,之前难以利用的智道仙蛊解谜,如今也开始发挥所用。

“我前不久的登天野之行,很快就会让智多星陈道的名号,传遍东海蛊仙界。届时,推算我的。寻觅关于智多星情报的蛊仙,一定很多。见面曾相识是变化道的杀招,可以伪装自己,但无法应付推算。如今有了星雾掩,我在东海就安全许多了。”方源心道。

当然,这星雾掩也是有防御极限的。

首先,它的防御只针对他人的推算,并不能防御地水风火,或者拳脚交击。

其次,它的核心只是两只六转仙蛊。若是七转智道蛊仙推算,也只能抵挡一会儿,就会溃散。

最后,星雾掩并不需要一直催动。催动此招产出的星雾。就是一层防护,只要星雾不消,就有防备推算之能。

“这层星雾当中,充斥星道、智道的碎杂道痕。正是这些无序的道痕,抵挡了外界的推算。就算是宋甲丹之流,推算我。星雾也可以抵挡片刻,让我警觉,并且有充分的时间,利用定仙游躲避到仙窍福地之中去。”

一进入仙窍,就是两个天地。即便是八转智道大能,想要推算,也几乎无可奈何。

先是战场杀招星魂,后是星雾掩,方源的两个短板,已然补齐。

接下来,就是着手炼制六转全力以赴仙蛊,同时,耐心等候焚天魔女那边的行动了。

“前世,焚天魔女已经排挤了鲨魔、苏白曼,可惜时机差错,只能前往北原。今生有我影响,大大推进了玉露福地的攻略进度,令她提前出手。时间应该是足够了。希望这次玉露福地的收获,不会像繁星洞天碎片中的梦境那般叫人失望。”

之前,尽管方源突破了第二层梦境,来到第三层,达到了前世唯有的进度。

不管是智道、星道境界,都略有提升,但仍旧还处于宗师级。

最关键的,是第三层梦境,分外古怪。

方源也百思不得其解。

第三层梦境自然消散,根本没有带给方源任何境界上的增益。

梦境中,星宿仙尊所吟的诗词,方源怎么回想,也记不起来。

时间一天天过去。

焚天魔女终于打点好东海僵盟的关系,和前世相差不多,她利用卜单,主动出击,将鲨魔、苏白曼排挤。

这个过程,方源并未参与其中。

等到焚天魔女攻略玉露福地之时,他才应邀出手。

正如他所料的,战场杀招按兵不动虽然奇妙,但无人主持,等若空阵。方源耗费了不菲的精力和仙元,突破最后这道关卡,彻底攻略了玉露福地。

“力道仙蛊只有一只,你不要想了,这是我给小兰的礼物。按照之前的约定,你可以拿取两只六转仙蛊。不过这只星道仙蛊,虽然只是六转,但却是十大奇蛊榜上之物,价值甚高。你若选它,就无法再选其他仙蛊。”焚天魔女对方源道。

ps:上一章有笔误,现已经将文中的“宋启元”改成“宋坤”。感谢大家的提醒!(未完待续。)

\end{this_body}

