\newsection{瓜老借钱}    %第三百一十六节:瓜老借钱

\begin{this_body}

萧山击退萧家追兵,将此峰命名为义天山后,便开始广邀魔道,招兵买马。

在他被家族逐出之前,他号称“义薄云天”,人脉十分广泛,正魔两道上都吃得开。

萧山本身性格豪爽,恩怨分明,重情重义。

另一方面则是萧家势弱,被武家等势力打压,必须左右逢源。要知道,有时候魔道蛊师也是得罪不起的,尤其是那些五转、四转的独行侠。

这些魔道强者,和正道强者不一样。

魔道强者孤家寡人,行事无所顾忌。正道强者却有道德、家族、名声等等羁绊手脚。

萧家这些年来,为了壮大家族,组建萧家商队。对这些浪迹天涯的魔道强者,比其他家族,还要更加小心对待。

这些年来,萧山面对这些魔道蛊师,能放就放,能拉拢就拉拢,必须打杀的就尽力尽快去做。有时候,他甚至救济一些散修,或者魔道蛊师。因此他虽然是正道族长,但在魔道上,却得到很多蛊师的认可。

这一切,都造就了萧山极强的号召力。

当他决定创建山寨,消息传出,立即得到了多方响应。

数天之后,就有数位魔道强者,风尘仆仆,主动来投。

再●♂长●♂风●♂文●♂£t加上南疆蛊仙们在暗中推波助澜,好将自己的棋子插入这个战场。

因此,越来越多的魔道蛊师、散修蛊师,都加入义天寨。

只是过了六七天,义天寨的成员就已经有三十多位。这种增长的速度就连萧山本人。都暗暗吃惊,大感意外。

人员一多。就形成组织。

任何的组织,都必须有地位之分。

蛊师的世界里。向来强者为尊。

在所有人中,萧山不仅实力最强,而且声威最高,被众人公举为义天寨寨主,稳坐第一交椅。

其余两人,孙胖虎、周星星分别坐第二,第三交椅。毕竟他们是五转强者,实力最强。

至于后来者,都是二转、三转居多。鲜少有四转。依照修为、强弱,简单安排了次序。

一处山洞中,方源正在和一位蛊仙喝酒。

这位魔道六转蛊仙,身材矮小,但脑袋硕大,是个老头儿,人称瓜老。

“来,盛鹰兄弟,喝喝看。我的瓜酒味道如何?”瓜老热情劝酒道。

方源心知瓜老的来意,不动声色地喝了一口,点点头,评价道:“不错。不错。”

虽然嘴上说不错,但方源神情平淡,一看就是敷衍。

瓜老察言观色。呵呵一笑,指向洞外:“不知道盛鹰兄弟。对此次南疆大赌,有什么看法?”

方源平静地道:“事关惊鸿乱斗台的归属。谁都不会让步。既然大家都参与了赌斗,订下赌约,投入巨量赌资,应该不会有人闹事破坏。”

以往而言,向来都是正道压过魔道一头。

不管是凡人蛊师,还是蛊仙界。

但这一次,却是个例外。

仙蛊屋出世,让南疆蛊仙界彻底动荡起来。为了争夺仙蛊屋,魔仙散仙们纷涌而出,结成大势,正道一方不得不妥协,定下了这场超级赌约。

赌约的内容,就是选择凡人棋子,参加义天山正魔大战,帮助幕后蛊仙炼化仙蛊屋。

但要参加这场大赌,还必须要有前提要求。

任何蛊仙,想要参加这场赌斗,都必须要下赌资。谁的赌资投下的越多,谁安排的棋子修为允许更高一些,加入赌局的时间就更早一些。

最终,这场大赌结束,所有的赌资都要重新划分。

按照各个蛊仙,转化仙蛊屋的战意多寡,进行排位。谁在仙蛊屋中的战意越多,获得的赌资价值就越大。

当然,炼化了超过五成战意,能够催动仙蛊屋的胜利者,获得的赌资最多。

但其他蛊仙,也能喝口肉汤。甚至还能以小博大,收获一笔横财。

这项规定,堪称神来妙笔!

正因如此,使得所有参与赌斗的蛊仙们,不管是正道、魔道还是散修,不管是有仇还是有怨,都下意识地联合起来,维护这场旷世赌斗。

谁若在途中有小动作或者赖皮,投注下去的赌资就彻底充公。

后来的蛊仙也可以加入,但必须参加这个赌斗。

而随着赌斗的进行,蛊仙们要继续加注,也是当然可以的。

而方源面前的魔仙瓜老,就是一位想要继续加注的人。

但他手头上却没有资金。

这些天,他四处借贷,为人所知。这一次,就特意找上方源。

方源表现得冷淡,瓜老不得不主动提及这场赌斗。

方源只是随意地评价了一下,瓜老眯起双眼,竖起大拇指,对方源夸张地笑道:“盛鹰兄弟,你说的好,太对了,可谓一针见血啊。这场赌斗,规矩比天还大。任何一人若坏了规矩,就等于和整个南疆的蛊仙界为敌!所以要想捞一笔,就必须老老实实地参加赌斗。”

“你看眼下,虽然赌局才刚刚开始,但萧家太上长老却是独占鳌头。萧山就是他的棋子,之前萧山和萧芒一战,就已经让萧家太上长老成了第一个,在仙蛊屋中转化了战意的人。只要萧山不死,他的优势会越来越大。”

方源瞥了瓜老一眼:“萧山是五转蛊师,又第一个进入赌场,萧家太上长老当然优势巨大。不过,他也为此付出了极大的代价。他虽然只是七转蛊仙,但投入下去的赌资却是最多的。就连其他的八转都比不了。”

瓜老连连点头,咧开嘴,露出一口黄牙。笑道:“是啊,越早将棋子投入义天山。就越有利。当然棋子的修为得有所保证,至少得是四转吧。不然战死在里面,就亏老本了。盛鹰兄弟,不瞒你说,我恨不得把自己都当做赌资投下去!但我现在偏偏手头紧,盛鹰老弟,你能不能暂时借我一些仙材。等到事成之后,我必定双倍还你!”

方源笑了笑:“可以。”

瓜老立即喜形于色,正要感谢,方源却道:“不过要定下契约。”

“这个是自然的。也是必须的。”瓜老连忙点头。

方源又道:“还要拿仙蛊当做抵押。”

瓜老脸上的喜色猛地僵住:“盛鹰兄弟,你这是开的哪门子玩笑?我若有仙蛊在手,何必来图谋这座仙蛊屋?”

方源也立即变色,满脸的冰冷,眼中暴射厉芒,沉声道:“瓜老你连一只仙蛊都没有,居然还想要图谋一座仙蛊屋?你不觉得自己太过异想天开,胃口太大了吗?!”

“盛鹰,你……”瓜老正要发作。但忽然间他感受到方源身上不再遮掩的仙蛊气息。

“他居然有仙蛊!”瓜老心中顿时一沉。

很多六转蛊仙手中,都是没有一只仙蛊的。

但方源却有。

这一点,出乎瓜老的意料之外。在他想来,但凡有仙蛊的蛊仙。都多多少少有一些名气的。

而盛鹰显然没有。

他把方源当做一个落魄的散仙,只在犄角旮旯里闷头苦修。

但方源有仙蛊,这让瓜老心生忌惮。

一有仙蛊。战力就往往提升一大截了。拥有仙蛊的盛鹰,不是瓜老所能拿捏的。

所以他发作不得。最终干笑一声,对方源抱拳道:“是我唐突冒昧了。盛鹰兄勿怪!”

方源冷哼一声:“你可以走了。”

“这就走,这就走,告辞!”瓜老满脸堆笑,后退几步,然后转身便走,再不敢纠缠下去。

当他走出山洞外,他的脸上笑容消失了,流露出恼羞成怒的神色。

他目光冰寒,在心中咆哮:“哼!有仙蛊果然是了不起!今天这个耻辱,我记住了。等到我有仙蛊的那一天,我一定会将今天所得,都还给你的!!”

山洞内,方源若有所思。

瓜老这件事,让他察觉到了许多底层蛊仙们的参赌心理。

这些人手中大多是没有仙蛊的,所以对仙蛊屋无比渴望。谁都知道一口吃不成胖子,但落到自己身上,谁不想一步登天,一夜暴富?

眼下的赌局,给了这些蛊仙十分良好的竞争环境。

换做通常情况,比如发现野生仙蛊,蛊仙们进行争夺。没有仙蛊的这些底层蛊仙,战力低下,怎么可能是其他蛊仙的对手呢?

但这场赌斗,考较的却非是蛊仙本身的战力,而是其他方面。

这一切,都让底层蛊仙们有了争夺胜利的希望。

瓜老这样的例子,并不在少数。

方源又想到萧家的太上长老。

据悉,这位七转正道蛊仙,正面临巨大的麻烦。他的上一次灾劫,只是惊险渡过,实力至今还未恢复。而下一次灾劫,威力更大,已经近在眼前。

而萧家只剩下他一位蛊仙,没有家族蛊仙的支援,他独木难支,几乎必死无疑。

但这一次赌斗,让他看到了生存的希望。

若是有仙蛊屋在手,他就能有相当大的把握,挨过这场灾劫。

所以,他这一次掏空老底子,孤注一掷地投下了最多的赌注。这些赌注让他有资格,选择两个棋子。

并且两个棋子的修为,都高达五转。其中的萧山,第一个投放赌场。另外一个萧芒,却还要等一段时间,才能上场。

萧山是魔道,萧芒是正道,不管正魔大战结果如何,过程怎样,萧家太上长老两头准备,十分妥当。两个五转的棋子,死亡的可能很小,随着时间流逝,萧家太上长老的优势会越来越大。

“这一次,我故意泄露出一只仙蛊的气息,足以让其他蛊仙忌惮,不会认为我是软弱可期的。”

“这一场赌斗,我要好好参加。赌斗中泄露出来的情报,十分宝贵,是我近距离接触南疆蛊仙界,了解南疆蛊仙势力格局的上佳机会!”

“至于这场赌局,或许我可以亲自出马……”想到这里,方源眼中精芒一闪即逝。(未完待续……)

\end{this_body}

