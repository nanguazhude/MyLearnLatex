\newsection{方源修魂}    %第三百四十八节:方源修魂

\begin{this_body}



%1
中洲,天庭。

%2
“监天塔主,放弃吧,这只仙蛊救不回来了。”碧晨天在旁劝道。

%3
监天塔主没有答话。

%4
此时,他的双手虚拢,散发出一股温暖的橙光。一只仙蛊几乎被劈成两半,被橙光吊着命,气息时强时弱,很不稳定。

%5
炼九生也是满脸忧色。

%6
他是炼道的大能,依他之见,这只仙蛊已经被判处了死刑。就算是炼九生亲自出手,就救不得。

%7
但监天塔主就执念一试。

%8
监天塔主执掌监天塔,乃是智道蛊仙。但凡到了他这种程度的蛊仙,就算专修一道,也是精通一道,而通晓百道。

%9
根本没有任何短板,能用智道模拟其他流派最擅长的地方,甚至是因为另辟蹊径,而超出一筹。

%10
眼下,炼九生对这只仙蛊没有办法,但监天塔主却有智道手段,有一丝的希望。

%11
只是他动用的这个手段,在碧晨天、炼九生看来,未免过于冒险了些。

%12
稍有失误,监天塔主就会受到反噬,从此一身道痕紊乱无序。受到重伤不说,三十年之内无法再催动任何仙蛊,否则伤势会越来越重。

%13
哧。

%14
忽然间,从仙蛊的身上,射出一道犀利的无形剑气。

%15
监天塔主甚至来不及反应,但幸亏这道无形剑气,只是擦过他的耳畔,切下一缕霜白的发丝。

%16
碧晨天吓了一小跳。

%17
“好了。”监天塔主的脸色稍缓。他吐出一口浊气,欣慰地道,“拔出了盘踞在蛊中的剑气,这只仙蛊总算挽救回来了。”

%18
自从,落天河中爆发出无数剑光。横扫中洲。

%19
一道剑光,犀利非凡,如同前世一般,竟然穿透了天庭,将监天塔切成了两半。

%20
监天塔主便紧急召回碧晨天、炼九生和白沧水三人。四仙合议之后,白沧水前往落天河底探明情况,而剩余三仙则趁着剑气侵蚀的监天塔。伤势没有更加恶化之时。赶忙进行修复。

%21
很快,监天塔主手中的那只仙蛊,在橙光的照耀之下,伤势以肉眼可见的速度,迅速复原。

%22
炼九生则双目放光,击掌而赞:“塔主你这手牺牲心中情感,修复仙蛊的智道手段。当真一绝!居然真的起死回生了,了不起。不过……你何必如此冒险呢?这只是一只六转仙蛊,我们天庭中库藏丰富,完全可以再行炼制。反之,你若受伤,三十年不能动手,必将是天庭的巨大损失。”

%23
监天塔主摇摇头,缓缓地道:“不知道为什么,最近一段时间,从炼蛊大会开始。我就感到不妥。好像有一种潜在的危机,正在某处酝酿,一旦令其发生,就会有十分恐怖的后果。我们刚刚修复了一次宿命蛊,能让它发挥出一半的威能。但我刚要进入监天塔,就有一道剑光斩来,好巧不巧。将监天塔劈成两半。这一切都是巧合吗?”

%24
碧晨天、炼九生相互对视一眼,两人的神情顿时凝重起来。

%25
“不瞒你们,近来我也有这种感觉。”碧晨天沉声道,“不过我的这种感觉,是在修复宿命蛊的时候,才开始的。”

%26
炼九生也道:“原来这不只是我一人的感应啊。我还以为,是我的徒孙后辈遭了劫难,让我冥冥之中有了感应。现在看来,恐怕不是如此简单的。塔主,你是智道蛊仙,执掌监天塔,感觉比我们强烈,恐怕真的有什么事情要发生。”

%27
监天塔主望着手中,正在修复的仙蛊:“所以,我才要甘冒风险,也要尽快地修复这只仙蛊。这虽然不是监天塔的核心仙蛊,但救活它,就是解决了此次监天塔伤势中的最大的一个难题。我们就能尽快地修复好监天塔,再利用它的威能,结合我们的感应,推算出真相。”

%28
“还有,薄青当年陨落在灾劫之下,为何现在忽然出现?两者之间,似乎有着某种紧密的联系。接下来,我会继续动用这个手段,修复其他仙蛊。在这个过程中,我必须全神贯注,一心一意。白沧水仙子一旦有什么回信,你们俩必须尽快通知我。”

%29
碧晨天、炼九生齐声应道:“明白。”

%30
仙僵薄青的事件,随着时间,不断发酵,形成波及中洲,囊括天庭的巨大影响力。

%31
就在中洲蛊仙们,都将目光,集中在落天河的时候。

%32
方源已经提前截取了最大的收获,在荡魂山上修行。

%33
此刻,他站在荡魂山巅,大手一挥,释放出数万的魂魄。

%34
这些魂魄多以兽魂为主,什么兔魂、羊魂、马魂等等,数量最多,狼魂、虎魂等猛兽魂魄稍小。也有植物的魂,而异人之魂、人魂的数量最少。

%35
其实,关于魂魄的买卖,一直都存在着。

%36
因为,这个世界上,修行魂道的蛊仙,也不在少数。

%37
魂道的蛊虫,其实从太古时期,就开始出现,但一直都没有形成系统。

%38
直到一个蛊仙的崛起,才有了魂道的创建和鼎盛。

%39
这个蛊仙就是人族历史上,十大尊者之一的幽魂魔尊。

%40
他是开创魂道的祖师爷,也是他单凭一己之力,将魂道携上历史的最巅峰。在他那个时代,十个蛊师当中,几乎有一半的人,都选择修行魂道。

%41
魂道的修行,魂魄是最重要的资源之一。

%42
炼制魂道蛊虫,大多都需要魂魄充当蛊材。

%43
幽魂魔尊为了炼制魂道蛊虫,疯狂杀戮,猎取魂魄。因此,他在历史上,也被评价为尊者之中杀性第一。

%44
而他带来魂道的荣昌,也使得他的那个时代,杀戮沸腾,世道黑暗,动荡不安。

%45
因为很多蛊师、蛊仙也修行魂道,自然也要大加屠戮,猎取魂魄,充当修行的资粮。

%46
也是在那个时代,魂魄的买卖就应运而生,并且大张旗鼓,堂而皇之。

%47
之后,幽魂逝去,乐土仙尊现世。

%48
这位生性仁慈的仙尊,与人不争,爱好和平。他的成长历程中,深知人民的苦难,清楚幽魂的遗害,所以当他成为仙尊,无敌天下之后,做的一件大事,就是打压魂道。

%49
魂道蛊师因此锐减,魂魄的买卖也从台面上,被迫转为台下。

%50
天下五域,开始休养生息,杀性消弭,动荡也渐渐平息。

%51
最让世人感怀的是,乐土仙尊并非以力量强逼,而是以德服人,以行动感化世人。

%52
因此直到此刻,就算是在宝黄天中,天底下最大最自由的市场,魂魄的买卖也十分罕见。

%53
方源的这些魂魄,是黎山仙子、黑楼兰他们搞来的。

%54
自从方源和黑楼兰她们合作,魂魄的事情,基本上就由她们负责,省去了方源很多麻烦。

%55
她们是北原的地头蛇,尤其是黎山仙子,乃是大雪山福地中的三把手。拥有山盟蛊,人脉宽广。

%56
这些魂魄,落到荡魂山上,立即遭殃。

%57
经过荡魂山的震荡和洗涤之后,魂魄都碎化为无数魂魄精粹,彻底消融在大山之中。

%58
不久后,因为这些魂魄的精粹,落魄山上就会产生出一批新的胆识蛊来。

%59
“通常而言,越强大的生命,魂魄的质量就越高。同时,灵性越高的生命,魂魄的质量也随之增长。这批魂魄的质量,只是一般。因为里面猛兽稀少,没有荒兽,异人、人族的魂魄最少。产生的胆识蛊数量,不会很多。”

%60
方源一边在心中估算着,一边开始在荡魂山中行走。

%61
这批魂魄,并非来源于黎山仙子、黑楼兰那边,而是方源自己搞到的。

%62
他现在已经和之前不同了。

%63
不需要借助黎山仙子,也不需要亲自出手,就能稳定的搞来魂魄。

%64
当然不是从宝黄天,而是从西漠。

%65
西漠是五域中,商业最发达的地方。方源在那里,早已不是初来乍到的人生地不熟,他已经和西漠萧家合作了一段时间,获得了他们一定程度上的认可,得到了关键的令牌。

%66
西漠萧家乃是超级势力,搞到魂魄,一点都不难。

%67
但如果魂魄买卖被曝光,萧家绝不会承认这个买卖的。魂魄买卖见不得光,而萧家毕竟是正道势力。

%68
方源从西漠买卖,得到魂魄,自然是为了防备黑楼兰、黎山仙子和焚天魔女了。

%69
虽然和她们签订了盟约,但并不代表从此之后,方源就对她们无需防备。

%70
防人之心方源有,害人之心方源多的是。

%71
方源忽然开始魂魄方面的修行,一定会惹来黑楼兰、黎山仙子她们的怀疑。一点蛛丝马迹,方源都不想留给她们。

%72
他暗中利用胆识蛊修行,一定会造成胆识蛊数量的剧减,为了弥补这个破绽,方源自己悄悄地投放魂魄,填补空漏之处。

%73
珍稀的胆识蛊,在方源的脚下,俯拾即得。

%74
方源也是这么做的。

%75
胆识蛊一旦脱离了荡魂山,就会立即毁灭。但若用方源发明的气囊蛊,却可包装起来,贩卖出去。

%76
消耗了大量的胆识蛊,方源的魂魄壮大到了极限,再盲目壮大下去,魂魄就会自行崩散。

%77
方源感到,就连仙僵之躯,都有些拥挤的错觉。

%78
好像是一个成年人,穿着他青年时候的,小了一号的衣服一样。

%79
ps:今天继续双更。

\end{this_body}

