\newsection{进攻狐仙福地(上)}    %第四十九节:进攻狐仙福地(上)

\begin{this_body}



%1
“但是残阳大人的去留,事关荡魂山、胆识蛊的归属啊。”鹤风扬还想争取一下。

%2
“凤九歌乃是公认的十大派第一战力,也是这一次北原队伍的首领。他的要求,我们无理由拒绝。门派的战力吃紧的情况,你是清楚的。我们就算强硬拒绝,相信其余九大派也会以不需要累赘拖累等等借口,联合施压。”

%3
听到太上大长老这么说,鹤风扬的脸色变得铁青。

%4
雷坦无声微笑,毫不掩饰自己幸灾乐祸的心理。

%5
鹤风扬气得肺都要炸了,勉强继续看下去,忽然他表情一僵,心中五味陈杂地道:“这……凤九歌居然主动将我素仙蛊,借给我用?而且他已经将我素仙蛊升上七转层次!六转的我素仙蛊,是消耗型蛊虫,只能用三次。但七转的我素蛊,却可以不断使用,只是需要的仙元要多得多。”

%6
“什么?我记得鹤风扬的我素蛊,只剩下最后一次的使用机会了。没想到凤九歌居然将我素蛊,合炼到了七转!难怪他索要了我素仙蛊,他是预谋的。”

%7
“不过他的运气可真好。仙蛊难炼,成功可能极小,我素蛊又并非他的本命蛊,他居然炼成了。”圭坜开口。

%8
树止戈则道:“灵缘斋家大业大,和天庭关系极为紧密,底蕴比咱们仙鹤门还要深厚得多。炼成七转仙蛊,并非一件奇怪的事情。”

%9
“有了我素蛊,就能不受福地压制,随意使用凡蛊了。”桑心夫人看向鹤风扬。

%10
“凤九歌好算计啊,一方面借了我素蛊,我派成功夺得狐仙福地后,方便他介入其中。另一方面又抽调了残阳老祖出去,事实上是降低了本次我派进攻狐仙福地的实力。”太上二长老分析着。

%11
太上三长老虎魔上人喝光碗里的羽化酒,将酒碗一把捏碎:“可恨!九大古派针对狐仙福地,暗中阻挠我派,已经不止一次了。鹤风扬!狐仙福地一事是你一手办理的,希望你这次能够成功,不给其他九派留下任何机会。狠狠地打他们的脸。我要让他们为了胆识蛊厚着脸皮来和我派合作!”

%12
“我明白了,三长老。”鹤风扬颔首。

%13
蛊仙们的商讨,持续了一天一夜,定下了仙鹤门数个月来的方向和举措。

%14
两天之后,方正推开石门,进入血池。

%15
“方正,多少次的艰苦训练,终于等到了现在这个时刻。今日你将为你的亲族报仇,为门派立功,为苍生造福。我们一定要成功!”空窍中,寄魂蚤颤抖着,传出天鹤上人的声音。

%16
“是的,师傅!一定会成功的!!”方源紧紧一握双拳,神色坚定无比。

%17
他走到血池中央,让滚烫的血液漫过他的大半个身体。

%18
随后,他深呼吸几口气,开始催动铁血蛊。血液发黑变沉,皮肤也因此转黑。

%19
紧接着,方正又同时催动血刃蛊。

%20
嗤嗤嗤……

%21
从他的浑身上下,暴射出数百道铁血飞刃,向四面八方激射。

%22
他的身上,形成数百道伤口,流出铁液般的血。

%23
强烈的快感袭上心头,方正咬着牙关,进行关键一步,催动了混血蛊……

%24
而与此同时,天梯山上,迎来了两位蛊仙。

%25
一位少年模样,温润如玉,一身绿袍,腰挂玉佩。最为引人瞩目的是他的一对眉毛,碧绿修长,且眉梢一直垂到腰际。正是仙鹤门六转蛊仙,人称鹤羽飞仙的鹤风扬。

%26
而另一位,则年轻貌美,身材窈窕有致,面若挑花,眼若秋波,一身粉蓝花裙,裙摆随风飘摇。亦是仙鹤门的一位六转蛊仙——苍郁仙子。

%27
二仙缓步踏上天梯山,来到狐仙福地的落点。

%28
在他们的眼前是一片青翠葱茏的树林,十分普通平凡。

%29
狐仙福地乃是另一个小世界,区别于中洲大天地。只要福地不存在漏洞,或者不敞开门扉,从外面看去,根本毫无异状。不知晓底细的话,就算是蛊仙擦肩而过,也察觉不到这里藏着一块福地。

%30
苍郁仙子环顾一周,青山苍翠,偶尔的鸟鸣更衬托出静谧,没有一个人影。

%31
但她却对鹤风扬笑道:“今天来的人,还真是不少,可有不少老面孔呢。”

%32
她和鹤风扬并未遮掩行迹,其实待会动静大了,也遮掩不住。因此从飞鹤山动身时,其他九大古派就得到线报。

%33
眼下,分别来自九大古派的九位蛊仙,早已经潜伏在天上地下,观看着鹤风扬、苍郁二人行动,收集着第一手的情报。

%34
鹤风扬一边挥洒蛊虫,一边称赞道:“久闻仙子有一道侦察杀招,名为微念,放在宝黄天中能卖出五块仙元石,果然效果上佳。”

%35
“飞仙大人谬赞。不过是昔年一次机缘,偶然从轮回战场所得。”苍郁仙子娇笑一声。

%36
“今日一战,还得赖以仙子出力。”鹤风扬态度十分温和。

%37
“飞仙大人客气了,小女子的些微手段,怎比得上堂堂鹤羽飞仙的万鹤齐翔。若是今日能有眼福,可就好了。”苍郁仙子啧啧赞叹道。

%38
鹤风扬便笑。

%39
谈话间,他洒出的凡蛊足有近万只,半盏茶的功夫后,终于布置妥当。

%40
“下面,就是静待时机了。仙子请。”鹤风扬一挥袖,飞出一团白云。

%41
白云离地只有一丈高,形成蒲团模样。

%42
鹤风扬一跃而上,盘坐下来。

%43
这是一只白云蒲团蛊,五转级数,能辅助蛊师修行,增加真元温养空窍的效果。效果虽然不大,但日积月累,却是可观。

%44
鹤风扬升仙之后,白云蒲团蛊对他失去了效用,只当做一个坐垫利用。

%45
白云蒲团很大,鹤风扬特意为苍郁仙子留下空间。

%46
但苍郁仙子瞧了一眼,却没有登上白云蒲团,而是屈指一弹,射出一只木道蛊虫。

%47
蛊虫钻入地中,几个呼吸的时间,一株桃树生长出来,枝条纠缠一起,形成一张桃花灿烂的舒适睡床。

%48
苍郁仙子嬉笑一声,登上桃树睡床,半躺半卧,闭眼假寐。

%49
与此同时,在伏虎福地中,方正已经进行到最后一步。

%50
他催动了败血妖花蛊,浑身血肉都成了妖花生长的养料和土壤。无数嬉笑的藤蔓,钻出他的皮肤,缠绕他全身上下。

%51
鲜艳妖冶的蓝色菊花,朵朵绽放,猛烈盛开。

%52
强烈的痛楚从全身各处传来,连铁血蛊都不起效果,方正咬紧牙关,面容扭曲,拼命坚持。

%53
“快用冷血蛊降温!”天鹤上人的魂魄时刻监察着方正的状态,不一会儿,便开口提醒道。

%54
方正强忍非人的痛楚,成功调动真元,催起冷血蛊。

%55
他浑身剧烈冷颤了好几下,终于是滚烫的鲜血温度下降,避免了自己被煮熟的悲惨下场。

%56
“用血感应蛊吧。”天鹤上人发出指示。

%57
方正将牙咬得都流出血迹,他艰难地催动了这只五转血道侦察蛊。

%58
此蛊能让他感应到,和他具有相同血脉的存在。且血脉越亲,越浓,感应的效果就佳。

%59
但这次不论方正如何感应,却都感应不到任何影像。

%60
“怎么会这样?你和方源乃是亲兄弟,血脉最浓最亲,而且败血妖花蛊又能增加血道蛊虫效能,怎么这么时间,你都感应不到?”方正已经痛到无法说话,但天鹤上人却在时刻感应着,良久不见效果,他变得紧张,不由地怪叫起来。

%61
“方正,坚持住!你训练了这么久,经历了非人的磨砺,就赌在这一次上。我派两大蛊仙已经出动,就等着你为他们指引方向!你千万要坚持住,不能有丝毫松懈之心。”天鹤上人语气急促地喊道。

%62
方正浑身剧烈地颤抖着,他已经没办法答话,只能用实际行动来表达对师傅的听从。

%63
“还是感应不到吗?奇怪!太奇怪了!方正你要顺着血池中的洞地蛊,将感应延伸出去。如果狐仙福地的那片洞地蛊没有被毁,那么这个洞地蛊就勾连着狐仙福地。它对你感应方源,将大有帮助!”天鹤上人不断提醒。

%64
但方正早已经顺着洞地蛊,去延伸感应,可惜收效甚微。

%65
对他而言,他仿佛置身在一片黑暗当中,去寻找的目标,就是黑暗中的那一点光明。

%66
但不管方正,还是天鹤上人都不清楚,方源身上的变故。

%67
方源便成了仙僵!

%68
他浑身的血液,已经从原本的健康鲜红,变成了碧绿冰冷的尸血。

%69
这无疑极大地增加了方正感应的难度。

%70
不管方正怎么努力,他都始终感应不到方源的存在!

%71
“可恶!可恶!”天鹤上人心里已经焦急得不得了,按照以往的训练经验,方正支持的时间有限。现在已经过去了九成时间,已经快要达到他的极限了。

%72
“难道,真的要动用那个底牌吗?”天鹤上人心中天人交战。

%73
为了确保此行成功,鹤风扬暗中在方正的体内,布置了一些血道蛊虫。再配合血池中布置的蛊虫,形成一张底牌,可以最大程度地防止意外。

%74
但是用了这个底牌,方正将彻底丧失神智,凭借惯性不惜生命地催动血感应蛊。方正的魂魄将会剧烈消耗,血肉迅速萎缩,死亡的几率十分的高!

\end{this_body}

