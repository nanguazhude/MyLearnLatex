\newsection{各自的牺牲}    %第三百六十节:各自的牺牲

\begin{this_body}

%1
在方源的操纵之下,惊鸿乱斗台再次爆发出刺目的红光。

%2
红芒长达数丈,宛若熊熊火焰燃烧。

%3
一股玄妙的力量,遍布火芒之中,但凡有砸落到仙蛊屋上的金刚劫珠,都被红芒卷席,封印进去。

%4
有巨阳仙尊遗留下的九转仙元,惊鸿乱斗台催发出此刻最强的实力。

%5
咔咔啦啦……

%6
整个仙蛊屋都在剧烈的颤抖,咔啦作响。

%7
吸收舍利金刚劫,还是很勉强。

%8
毕竟这可是浩劫!

%9
但终究,方源还是成功了。

%10
越来越多的金刚珠子,被他掌握,暂时封印在惊鸿乱斗台里。

%11
但方源也为此,付出了巨大的代价。

%12
“刚刚作战这么久,我撒下去的一把巨阳仙元,都未用到三成。吸收了上百颗金刚珠子,这些巨阳仙元就损耗光了!”

%13
巨阳仙元剧烈的损耗速度,让方源暗暗吃惊。

%14
但此时不是心疼的时候,他又再次挥洒出一把巨阳仙元。

%15
惊鸿乱斗台动力十足,在浩劫中不仅安然无恙,而且更卷席了不少金刚珠子进去,实力反而增强了不少。

%16
监天塔本身就是九转仙蛊屋,防御惊人不说,又有宿命仙蛊坐镇,本身对于灾劫就有针对防御之效。

%17
而方源的惊鸿乱斗台,也在巨阳仙元的支撑下,超常发挥,尽展宙道玄妙,实力不减反增。

%18
这一波浩劫之后,唯有影宗一方的羽圣城,受损最为严重。哪怕他们损失了巨量的愿力凡蛊,也比不上方源和天庭。

%19
舍利金刚劫。持续了半炷香的时间,这才戛然而止。

%20
没有让三方蛊仙有喘息之机。第二波灾劫再度降下。

%21
天空中亮起道道光圈,从光圈中奔腾出无数巨鹿。

%22
这些鹿,每一头都有象般大小,并且浑身半透明状,看似全由白光组成。

%23
太古荒兽天缘光鹿!

%24
这是太古荒兽形成的浩劫。每一头太古荒兽,都可战八转蛊仙。而现在扑下来的光鹿,足有二十二头。

%25
也就是说,这是二十二位八转战力!

%26
毫无疑问,就算是这些光鹿身上。没有任何的野生仙蛊,这波浩劫的威力,也比第一波要来得恐怖。

%27
光鹿群朝着十绝仙僵无生大阵奔腾而来,一时间,影宗诸仙都变了脸色。

%28
十绝大阵迅速地敞开口子,将一半的鹿群放了进来。

%29
“天缘光鹿有九色,对应太古九天。这些光鹿都是白色光鹿,看来应当是居于白天当中的。”

%30
“天缘光鹿原本性情温和,远古传说之中。常有载人上天的美妙传说。没想到在天意的影响之下,这些光鹿充斥战意和疯狂!”

%31
尽管天庭蛊仙不想掺和,但身在局中,无法躲避。也被波及。

%32
方源驾驭惊鸿乱斗台,边战边退。

%33
这些光鹿乃是活物,他无法动用仙蛊屋封印起来。

%34
惊鸿乱斗台能封印的攻势。都是地水风火之物。取宙道之奥义,惊鸿之名。由此而来。

%35
好在方源之前收缴了不少金刚珠子,此时打出去。逼退光鹿。

%36
第二波天劫,天庭的处境最好,方源次之,影宗却是压力最重,处境十分危急。

%37
关键时刻,是仙僵薄青站了出来。

%38
他催动五指拳心剑等等震古烁今的剑道杀招,威力沛不可挡,让人心悸。

%39
大半的天天缘光鹿都倒在剑光之下,命丧当场。纵然有野生仙蛊,也不济事。

%40
薄青尽管已经身死,但仅存下的仙僵之躯,充斥着难以想象的剑道道痕。有着这些道痕增幅,哪怕是相同的杀招,秦百胜使出和薄青使用,是两种截然不同的效果,有着云泥之别。

%41
仙僵薄青苦苦支撑,终于令影宗撑过第二波的浩劫。

%42
但天空中的阴云,已经稀薄无比,十位仙僵再藏不住身形,隐约可见。

%43
“这大阵已经被削弱到了谷底,冲出去!”方源断喝一声。

%44
惊鸿乱斗台向外喷涌金刚珠子,噼里啪啦,企图打穿大阵。

%45
另一方,监天塔主也看出这是个良机,落井下石,监天塔也发动起来。

%46
“怎么办啊?!”羽圣城内,影无邪双手抱头,束手无策。

%47
其余影宗蛊仙,亦都面沉如铁,氛围极其凝重。

%48
已经到了胜败的关头。

%49
十绝仙僵无生大阵建成无悔,不是成就是败,绝没有第三个下场。一旦大阵被打穿,必定会接连崩溃。

%50
“没有办法可想了,只能牺牲僵盟了!”七星子长叹一声,做出了决定。

%51
分布五域的僵盟,其实只是影宗的下宗。就如同中洲十大古派,对于天庭而言的意义。

%52
只是影宗神秘非凡,很多僵盟高层都未必清楚,隐藏在僵盟最深处的这个秘密。

%53
十绝仙僵无生大阵,乃是影宗耗尽心血,精心筹谋的最大计划。

%54
此刻,影宗众仙迫不得已,只能弃车保帅,牺牲僵盟,维护此次局面。

%55
后手发动,一道道的身影,从十绝大阵中闪现而出。

%56
正是来源于各处僵盟的仙僵。六转、七转、八转修为,不一而足,数量惊人。

%57
“这是?!”惊鸿乱斗台中,黑楼兰、太白云生等人瞪大双眼,一齐失声道。

%58
“哼!僵盟的背后,果然就是影宗。”监天塔主皱眉冷哼,杀意勃发,目光森寒,“来得好!把这些违逆天意的东西,一网打尽。”

%59
但出乎意料的是,这些闪现而来的仙僵,却无意识,宛若牵线的傀儡。

%60
还未等到监天塔主动手,他们竟然就牺牲自己,化作十绝大阵的一部分!

%61
“大姐!”黎山仙子哀嚎。

%62
在这些仙僵之中。她看到了焚天魔女的身影。

%63
但这位八转仙僵,北原僵盟分部的首脑。根本毫无神智,面无表情。

%64
在黎山仙子、黑楼兰的亲眼目睹之下。焚天魔女也随着这些仙僵,将自身身躯化为仙材,融入大阵之中。

%65
得了这些饱含道痕的仙僵之躯,还有无数仙蛊、仙窍,十绝大阵的阴霾浓雾,瞬间暴涨,超出原先的巅峰,达到史无前例的一种状态,几有遮天蔽日之感。

%66
黎山仙子当场流下悲伤之泪。

%67
黑楼兰沉默不语。

%68
方源紧皱眉头。他期待的八转援军,已经先他一步,惨死在他们的眼前。

%69
谁能想到僵盟、影宗是这样的关系?

%70
就算有人提前洞察了真相,但说出来,谁能相信?

%71
遍布五域,超级势力中的霸主,偌大的僵盟,居然只是影宗放在台面上的布置。

%72
“之前影宗发布强制任务,要大炼蛊阵。恐怕却是暗算僵盟成员啊。”太白云生恍然大悟,为影宗的险恶用心,感到心寒不已。

%73
方源脸色并不好看。

%74
“我之前,也猜测过影宗和僵盟之间的关系。但却没料想到会是这种情况!我前世的记忆,误导了我。毕竟僵盟可是在五域乱战的时期,都健在的超级势力啊……”

%75
“好大的手笔!几乎将五域的仙僵都杀个干净。充当仙材,填入这道十绝大阵之中。那人究竟想要炼出什么来?难道是传说中的永生之蛊?!”

%76
监天塔主同样脸色难看至极。

%77
他身后的天庭众仙。也都面泛忧愁之色。

%78
饶是他们见多识广,位高权重。也不禁被此刻影宗的惊人手笔所摄。

%79
从这一刻起,超级势力,蛊仙成员最多的超级势力,僵盟在五域彻底毁灭,烟消云散!

%80
咻咻咻!

%81
尖锐的声音,刺入众人耳膜。

%82
时间已到,第三波浩劫,随之降下。

%83
天意凶恶,不管影宗想要炼制什么东西,很明显,是绝对的逆天之物,已经惹来天地的愤怒。

%84
连天地都不允许它的创生!

%85
第三波浩劫,威力比前两次叠加起来都要更强。

%86
天空中,下起尖锐的雨。每一根雨丝,只有半寸来长,锋锐无比,大有洞穿世间一切的威势!

%87
浩劫绝刺雨。

%88
十绝大阵再遭重创,刚刚还在翻腾着的浓郁阴云,像是被一头闷棍狠狠打重,扬不起头来。

%89
方源故技重施,以巨阳仙元催动惊鸿乱斗台,吸摄射中自己的浩劫箭雨。

%90
只是这一次,他更能明显觉察到,应对的难度要大大超出前次。

%91
苦挨了片刻,惊鸿乱斗台就被洞穿多处,大量蛊虫陆续牺牲。

%92
“再这样下去,恐怕这座仙蛊屋只能支撑一刻的时间。”黑楼兰低喝,看向方源。

%93
方源也无良方。

%94
尽管他已经全力施为,令惊鸿乱斗台吸摄不少攻势,但仍旧来不及护卫周全。

%95
而他本身应付的这些箭雨,不过只是总体的一成而已。

%96
大多数的箭雨,都被影宗方面承担了过去。

%97
如此一想,就知道浩劫的恐怕之处。有这样的索命极难,八转蛊仙们的日子,绝不好过。

%98
第三波浩劫结束,羽圣城、惊鸿乱斗台都溃散分解,残存蛊虫被各自蛊仙收起。

%99
监天塔上,伤痕遍布,比之前更深了数倍不说,就连塔身一角都彻底崩塌。

%100
“一切都结束了。最终的胜利者,只能是我天庭一方!”监天塔主仰头大笑,催动监天塔的最强攻势。

%101
这是带着宿命威能的攻击。

%102
避无可避,挡无可挡。

%103
下一刻,监天塔主就看到:在监天塔的攻势之下,大阵破开,影宗蛊仙惨死无数,方源等人亦都难逃死亡!

\end{this_body}
\newsectionindepend{单章:假如天有不测风云}
\begin{this_body} \par
%104
最近压力比较大。

%105
可能很多朋友都知道,本人在起点写的书,除了正在写的这本《蛊真人》,其他都已经被禁了。

%106
起点上是看不了了。

%107
至于这本《蛊真人》的情况,老读者们都相当清楚,从一开始就被举报,到现在,我都记不得多少次了。

%108
其实《蛊真人》这本书,并没有什么值得举报的。

%109
和大多数的小说比起来,方源这个主角只是足够冷静,多一些理智而已。

%110
你要说他滥杀,他从未滥杀过。说他滥情,更不可能!要说反社会?反什么社会了?《蛊真人》中描述的是异世界!

%111
三年了。

%112
写这本书,已经三年了。

%113
三年前,我创作时,拥有的梦想,现在我仍旧拥有着。并没有忘却,更没有放弃。

%114
三年间,我遭受过不少生活上的坎坷和挫折,但这本书我始终坚持在写。金钱放到第二位。有人说,这是文青病。是,我承认,这本书是我的文青病,而且病入膏肓。

%115
三年后,我仍旧要坚持,甚至更努力。

%116
不管接下来的创作好不好,精彩不精彩,我都会尽我自己最大的努力。

%117
我在序中已经说过,《蛊真人》这本书是我圆梦的表达。

%118
感谢大家!

%119
这两个月来,本书成绩不错。让我更有动力,去创作。

%120
但更有动力的同时,我也有担心。

%121
但如果有一天,《蛊真人》被禁了,怎么办?

%122
我想我会继续更新下去。

%123
或许是在微信上。

%124
蛊真人的公众号,大家可以加一加。直接搜“蛊真人”,就能看到公众号了。

%125
很简单。

%126
也是一个保险。

%127
将来的事情,谁说的准?

%128
假如天有不测风云,我仍旧会尽力更新!

\end{this_body}

