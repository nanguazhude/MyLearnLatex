\newsection{一代天骄凤九歌!}    %第三百零三节:一代天骄凤九歌!

\begin{this_body}

%1
方源回到狐仙福地。

%2
焚天魔女的违约赔偿,让他很满意。

%3
首先是焚天魔女最近资助方源的仙材,都一笔勾销,无须方源做任何偿还。其次,焚天魔女还借给了方源三只炎道仙蛊,以及附送大量的凡蛊。

%4
这三只炎道仙蛊,加上无数凡蛊,正好可以组成炎道仙级杀招涅槃火。

%5
说起来,焚天魔女也算是比较倒霉的。

%6
她资助了方源炼制星念仙蛊,前后付出了大量的珍稀仙材。结果到头来,方源等于白炼,不需要偿还焚天魔女任何东西。

%7
尽管之前,那些仙材都被焚天魔女动过手脚,但方源也收获了大量的实践经验,对炼制星念仙蛊的整个过程,都十分熟悉了。

%8
因为盟约,方源早就知道仙道杀招涅槃火的具体内容。

%9
用什么仙蛊为核心,需要多少凡蛊辅助,蛊虫之间相互催动的次序是什么……这些内容,他都心知肚明。

%10
只是要用这个杀招,得有个前提必须消耗蛊仙身上的炎道道痕!

%11
这一点,焚天魔女巧妙的隐瞒了,算是盟约中的一处小陷阱。

%12
她是想借此,增强对方源的掌控力。

%13
毕竟方源身上没有炎道道痕,若是要用涅槃火重生,就得让焚天魔女亲自向他施展涅槃火。

%14
方源之前的打算,是将涅槃火这个仙道杀招改头换面,改换其他仙蛊为核心,却具备相同或者相似的效果。

%15
而新的核心仙蛊,最好是方源手中现有的仙蛊。

%16
这种改良难度相当的大,平常的智道蛊仙都没有这个自信,但方源却大有底气。

%17
原因就在于。他的狐仙福地中藏有智慧蛊。

%18
九转智慧蛊的威能效用,简直妙不可言!

%19
方源也必须要这么做。

%20
生死仙窍重生法,能让他重生。但每个仙窍只用用一次。重生之后,他就无法利用智慧蛊了。

%21
如果方源改良了涅槃火。能让他自由转变成仙僵和活人,智慧蛊就将继续带给方源难以估量的辅助作用。

%22
“焚天魔女手头紧,需要自掏腰包,炼制仙蛊,完成僵盟的强制任务。只好借出这三只炎道仙蛊。若非如此情境,就算我主动要求,她也绝不会如此轻易就借给我的。有了这三只炎道仙蛊,将给我的推算改良。带来极其巨大的方便!”

%23
方源对此很是庆幸。

%24
僵盟的蛊阵大计划,让他意外地成为了受益人之一。

%25
接下来的日子里,方源就整天整夜的沐浴在智慧光晕中,全力推算,专心致志改良涅槃火杀招。

%26
说起来,他已经筹集到了足够的力道仙僵,炼成了足够多的奋力凡蛊,还得到了一只六十年的寿蛊。

%27
三样俱全,他完全可以在下一刻,就启动生死仙窍重生法。成为活人蛊仙,得到上等生死仙窍福地。

%28
不过在有了涅槃火之后,方源就不满足于此。

%29
虽然因为焚天魔女插手。导致方源不能算计黑楼兰,得到特等福地。但却因缘巧妙,得到了涅槃火,顺利的话,可以在今后利用智慧蛊。

%30
这一得一失之间,究竟是赚是赔,方源一时间也无法计较清楚。

%31
就在方源改良仙道杀招的时候,赵怜云已经秘密踏足落魄谷。

%32
她是营救凤九歌的关键。

%33
而陪同她一行的灵缘斋救援队,却只有一人。

%34
凤仙太子!

%35
八转修为。足以应付任何挑战。

%36
表面上,他是北原正道蛊仙。是超级势力宫家的外姓太上长老。只是他的身上没有黄金血脉,又因为早年和宫家的某些恩怨。导致他常年独居在外。

%37
实际上,他却是灵缘斋的弟子。是中洲势力耗费巨大精力和代价,秘密安插在北原的最重量级的间谍,主要负责八十八角真阳楼。

%38
本来,凤仙太子是不能轻动的。一旦被发现什么蛛丝马迹,凤仙太子潜伏数百年,就成了无用功。

%39
但事关凤九歌就不一样了。凤九歌是灵缘斋的招牌,现在的他就已经盖压中洲,一旦晋升八转,凭他的才情和能力,假以时日,也会是八转中的强者。他的存活,就意味着更多的利益。他的生死,将改变中洲十大古派的现有平衡,乃至改变中洲整个蛊仙界的格局。

%40
所以,灵缘斋不惜冒险启动凤仙太子这个暗棋,竭尽全力施展救援!

%41
凤仙太子、赵怜云,一仙一凡,在落魄谷中漫步前行。

%42
“这位大人,我想请教您。如果我救出了凤九歌大人,是不是就能成为灵缘斋的当代仙子?”走在途中,赵怜云问道。

%43
凤仙太子戴着面具,身形模糊,好像笼罩着一层雾气,并未显露真容。

%44
他低声笑道:“灵缘斋的当代仙子,可不是这么容易就能当上的。不过你若能救出凤九歌,恐怕此事就十有*了。我听说,他的女儿凤金煌是竞争此位的第一人选。然而你一旦成为她的救父恩人,恐怕她会主动直接退出这场竞争。”

%45
“是这样啊……”赵怜云眼中闪烁出希望的光辉。

%46
她虽然有成年人的灵魂,但到了灵缘斋后,就十分配合,毫无藏私,极其乖巧。

%47
因为她已经深深地了解到了这个世界。

%48
没有力量,就算智谋再出众,也是无根的浮萍,风一刮,就会被吹走,随波逐流。

%49
她明白,自己想要救出马鸿运,就必须依靠灵缘斋。

%50
但真正想让灵缘斋去跨越进攻大雪山,并不现实。

%51
赵怜云已经了解了灵缘斋的门规,历代的仙子都会被门派全力培养。灵缘斋的历史上,这些仙子几乎都修成了蛊仙,很多都有强大的战力,掌握门派中的主要权柄。

%52
“要救出鸿运哥哥,还得要靠我自己。鸿运哥哥。你要支撑住啊!”赵怜云在心中暗暗为马鸿运祈祷。

%53
凤仙太子忽的顿住脚步:“找到了,就是这里。”

%54
借助灵缘斋送来的仙蛊,凤仙太子掌握了一门专门侦测盗天传承的仙道杀招。

%55
使用这个杀招。必须行速缓慢。

%56
凤仙太子带着赵怜云,走了一个多时辰。终于在偌大的落魄谷中,寻找到了盗天传承的具体方位。

%57
凤仙太子催动仙蛊,立即施为,一盏茶的功夫后,一道光门被打开。

%58
呼!

%59
光门刚刚打开,一股猛烈的风就迎面吹来。

%60
凤仙太子大惊:“这里面怎么会有大同风?!”

%61
他连忙催动得力的仙蛊,施展杀招,伸手一抓。

%62
大同风就被浓缩成球。被凤仙太子牢牢禁锢在手掌当中。

%63
但旋即,凤仙太子的手掌剧烈颤抖,连带着他的脸色都苍白起来:“快进去,我只能支撑片刻功夫!”

%64
赵怜云连忙哦了一声,她虽然看不清凤仙太子的脸色,但也能从他的语气中听出他的心中是多么的焦急。

%65
赵怜云踏足进去,又见到她熟悉的一幕。

%66
和之前的传承空间一样,这里也是空空荡荡。唯一多出来的,就是十几个龙卷大同风,像是一根根巨柱。在空间里缓慢移动。

%67
赵怜云心肝砰砰直跳。

%68
她虽然不认识大同风,但看刚刚那位蛊仙的表现,也知道大同风的威能。

%69
“你终于来了。快到我这里来。”一个声音,忽然传到赵怜云的耳中。

%70
赵怜云心中一惊,忍不住后退一步:“你是谁?”

%71
“我正是凤九歌,快,来我这里。”声音虚弱无比,像是油尽灯枯的老者回光返照。

%72
赵怜云犹豫着。

%73
她听出声音,竟是从其中一股龙卷大同风中传出来的。

%74
赵怜云还发现,空间中的所有龙卷大同风,都在围绕着最中央的那一股。缓缓移动,蓄势待发。仿佛是虎视眈眈的狮群。

%75
“放心吧,你是天外之魔。是得到盗天意志承认的人,这里的大同风对你无害。”

%76
凤九歌顿了一顿,又艰难地出声道:“你来到这里,不是为了解救我吗?我若是一直在这里,你也无法继承这道盗天传承。”

%77
赵怜云舔了舔干燥的嘴唇。

%78
她知道自己必须立即做出决定。

%79
时间可不等人。

%80
赵怜云小心翼翼地走近一道龙卷大同风,龙卷风的力量很是内敛,赵怜云站在风柱的旁边,也只是感到微风拂面。

%81
她颤巍巍地伸出小手,一咬牙将小手探入到龙卷风中。

%82
若是其他蛊仙见到这一幕,恐怕要情不自禁地惊呼。但赵怜云无知者无畏,并不太清楚大同风的威力,她选择相信凤九歌。

%83
果然如凤九歌所说的一样,凤仙太子都要头大的大同龙卷风,对于赵怜云一点伤害都没有。甚至当赵怜云的小手伸入风中后,这股龙卷大同风便以肉眼可见的速度,迅速颓靡缩小,几个呼吸之后,彻底消失。

%84
凤九歌又传来声音:“你看我说的没错吧?这里的传承空间,有高于大同风的力量,所以不会被大同风摧毁。你是天外之魔,得到了这片空间的认可。有这片空间保护你,你毫无危险可言。”

%85
赵怜云大喜:“凤九歌大人,那我现在就救你出来!”

%86
她冲进风柱群中,一路横冲直撞,所到之处,大同风尽数消失干净。

%87
最终,她到达中央,将困住凤九歌的那道最巨大的风柱,也轻易地消解掉。

%88
然后,她便看到凤九歌盘坐在半空中,离地十尺高度。脸色红润,衣袍一尘不染,但却紧闭双眼。

%89
“凤九歌大人……”赵怜云轻声呼唤。

%90
凤九歌缓缓地睁开双眼,目光温润似玉。

%91
他没有再开口说话,而是摊开自己的右手掌,给赵怜云观看。

%92
在他的右手掌中,用血写着两个字。

%93
赵怜云懵懂地看了看,又将目光投向凤九歌。

%94
凤九歌向她微微一笑,随后整个身躯宛若沙尘飞舞,蓦地崩解,当场消散。

%95
赵怜云呆立当场。

%96
一代天骄凤九歌,就此身陨!

\end{this_body}

