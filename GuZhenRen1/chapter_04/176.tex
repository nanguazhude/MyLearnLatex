\newsection{建福地楼兰醒悟}    %第一百七十六节:建福地楼兰醒悟

\begin{this_body}

狐仙福地的第七次地灾,算是渡过去了。(WWW.mianhuatang.CC 好看的小说

回过头来,再看这场地灾。

要说什么惊险,真的没有什么。但是这场地灾却直击方源一方的弱点,着实有些恶心人。

方源这边有三位蛊仙,但只能眼睁睁地看着地灾发生,然后在事后尽量弥补。

地灾结束,黑楼兰继续留在狐仙福地,这个地方对她而言,是目前最安全的。

在方源的安排下,她继续催动力气仙蛊,辅助石巢中的毛民奴隶,大量地炼制气囊蛊。

如今在仓库中,气囊蛊已经堆积如山。

气囊蛊算是一次性的消耗储藏蛊,专门搭配一只胆识蛊,然后贩卖出去。

方源在渡劫之前,压榨黑楼兰的价值,三个石巢疯狂炼蛊,制造了大量的气囊蛊。

如今气囊蛊数量众多,胆识蛊反而有点跟不上了。

方源便改变炼制方案,只让一座石巢炼制气囊蛊,其余两座改炼星念凡蛊。

这让黑楼兰气得牙根痒痒的。

因为之前的交易已经规定:在渡劫之前,黑楼兰出力帮助毛民们炼制气囊蛊。将来这些气囊蛊装载胆识蛊,贩卖出去的获利,全部归结于方源一人。

现在狐仙福地已经成功渡劫,今后胆识蛊的获利,便有黑楼兰的分成。

黑楼兰也缺钱呐。

准确地讲,像她这种刚刚晋升的蛊仙。几乎没有不缺钱的。

尤其是当下,黎山仙子的财力渐渐干涸,已经难以支撑黑楼兰的修行,黑楼兰对仙元石的渴求,更加旺盛强烈。

但就在她想要大展宏图。积极累积底蕴之时,方源却缩小了炼制气囊蛊的规模。

黑楼兰无奈。

三座石巢都是方源之物,他想要炼什么,就炼什么,根本不必看外人的脸色。

但黑楼兰又岂是轻易放弃的人?她找到方源,亲自交涉,希望方源至少能让两座石巢。<strong>txt电子书下载Http://wWw.80txt.com/</strong>都来炼制气囊蛊。从而推动胆识蛊贸易。

方源热情地接待她,然后动情地向她述说自己的难处:“楼兰仙子,不是我不想扩大生产啊。实在是让人头疼呢,你之前也瞧见了,炼制气囊蛊的过程中,毛民损失很多。我要补充毛民,成本就上来了。而星念蛊炼制起来。却是比较温和的,毛民损失较少。现在我手中的毛民,已经损失了至少三成。再损失下去,就得购买毛民奴隶。可是你也看到了,我的狐仙福地刚刚渡劫,损失很大,处处需要节省啊。”

黑楼兰听了,直皱眉头,心里十分膈应,“你放屁”这三个字差点就脱口而出。

黑楼兰就算不知道方源的全部底细。但也清楚得很:方源渡劫之前,将重要珍贵的资源都转移到太白云生的仙窍内,根本就没有什么重大的损失。

而且他在碧潭福地中,还抢了那么多的资源,又得到东方长凡的魂魄,可以说是太丘一役中最大的赢家之一了。

胆识蛊贸易一直都十分火爆,大部分的获利也都装进了方源的口袋里。

方源就算再没钱。买下一小群毛民奴隶,还是能买得起的。

方源向黑楼兰哭穷,见黑楼兰神情不悦,脸上的微笑仍旧不变,道:“其实依我看,我之前提议的计划,还是可行的。咱们四人筹措出一笔巨资,购买毛民豢养的心得经验,再收购大量毛民,自己培养。这种百年大计,从长远来看,是最为划算的。”

方源旧事重提,黑楼兰心中叹气。

方源的这个计划,其实对四人都有利。但第一次提的时候,黎山仙子为什么就一口反对呢?

因为这个计划一旦完成,方源才是最获利的一方!带给方源的帮助太大了。

四人当中,就属方源最需要毛民,他有石巢,更能充分利用毛民。方源的修行计划中,有大量的需要炼制的凡蛊。但对于黑楼兰、黎山仙子、太白云生,却是没有这个需求的。

这个计划成功,黑楼兰等人需要投入精力和时间,豢养毛民奴隶。最大的用途,是贩卖毛民奴隶。但他们的仙窍本来就不擅长豢养毛民啊,狐仙福地反而在这点上,比他们三个都强些。

方源不仅可以买毛民奴隶,豢养毛民的优势最大,而且能利用毛民炼蛊,贩卖大量凡蛊出去。

正因为这点,黎山仙子拒绝。

方源的发展太快了,黎山仙子这位老牌蛊仙也看不下去了。虽然是盟友,但大雪山盟约可是有时限的。

和方源合作了这么多次,黑楼兰、黎山仙子还不清楚方源是个心狠手辣、老奸巨猾的魔头?

黑楼兰心中叹着气,眉头深皱,不接方源的话,直接起身告辞。

方源带着笑,送她出去,望着她离去的背影喊道:“此事不急,楼兰仙子你可以多考虑考虑。”

黑楼兰飞在空中,听到这话心中冷哼:“考虑什么?我就不信你能忍耐得住?你只用一座石巢炼制气囊蛊,胆识蛊的贸易量就相应地骤减一半。如此一来,你方源每个月的进项也跟着大为缩水。你现在用钱的地方比我多得多,我不信你能忍耐得住!”

“嗯?不对!”几个呼吸之后,黑楼兰飞行速度忽然缓慢下来,“气囊蛊炼制少了,但方源手中还有一大批的存货啊。他完全可以用这批存货顶上,保持胆识蛊贸易量不变,甚至超出几倍都有这能力!”

黑楼兰越飞越慢,心中寒意越来越重:“这方源好深的算计,原来在这个地方等着我!他每个月的收益不会变,甚至还会上涨。我呢?仓库里的那些气囊蛊,都是我出力炼制的,但根据之前的交易规定,我是不能从中获利的。偏偏小姨妈那边,已经自顾不暇。方源这是掐准了这个时机,趁人之危啊!唉……我岂会在这个关键时候,再去拖累小姨妈呢?她帮助我的已经够多的了!”

黑楼兰心中一片冰凉,她回到石巢,看着脚下的三座宏伟建筑,忽然惊觉!

“和方源比起来,我还是有许多欠缺,相当的不足。”

“我是大力真武体,方源只是区区一位仙僵。但不知不觉间,我却已经被他拉下这么多。”

“我自以为城府深沉,能够算计,但方源简直深不可测,更加老谋深算!”

“我自以为战力出色,但方源的万我杀招,可以战七转!”

“我最大的缺点,就在于没有经济支柱。蛊仙修行不是一味的喊打喊杀,经营才是根基啊。在这个方面,我落后方源太多太多!”

念及于此,黑楼兰伸出手掌,啪啪啪,甩了自己几个巴掌。

“该醒了!”黑楼兰脸颊火辣辣地疼,但一对眸里目光如电,神情坚毅似铁,“我从小到大,就被当做族长继承人来培养。在凡人蛊师期间,不缺任何修行资源,因此不会自我经营,缺少这方面的意识。我还要谢谢方源,打痛了我,让我惊醒。果然最好的老师,便是自己的强敌!决定了,接下来的第一步目标,就是先让自己真正的独立!”

黑楼兰不愧是一流的人杰,能意识到自己的不足,又敢于承认,并做出改变。

她醒悟之后,再不去找方源交涉。她沉下心来,打算先寻找可以经营,并适合自己的仙窍,且易于盈利的项目。

如此一来,她觉得只有一座石巢炼制气囊蛊,也无所谓了。因为这样,她也就有了更多的时间,去学习积累,去思考适合自己的营生。

方源忙得头昏脑涨。

他先是将荡魂山,从太白云生的仙窍中重新取出来,仍旧放置到狐仙福地的中央。

随后,开始为龙鱼、气泡鱼挖掘全新的家园。

虽然毒血已经被抽干,表面被彻底腐蚀的土壤也被全部铲除,但剩余的土壤,仍旧有着淡红色。抓一把土壤捏紧了,就会从中挤出丝丝的血液来。

方源索性重新布局,这个工程量并不小,首先他得将原先的湖泊河塘,向四周挖开。随后他开始在土壤中,布置蛊虫,结成蛊阵。

在蛊阵的基础上,他购买大量的肥沃新泥,填埋进去。

新泥也分门别类。

豢养龙鱼的那处湖泊,方源采用的龙鳞土。这种土颗颗结粒,土质坚硬,远望的话宛若龙鳞层层叠叠。事实上,它是龙类荒兽、上古荒龙,甚至太古荒龙常年居所地方的泥土,沾染了龙气从而形成。

龙鱼在龙鳞土构造出来的湖泊中生活,便有更多繁衍生育的冲动。

喂养气泡鱼的湖泊,方源选择的是玉须土。这种土握在手中,十分冰凉,玉须土压制到极点,就会形成玉。寻常的玉须土壤,会形成一个个的气脉,宛若人的血管脉络,又仿佛千年老参的参须。这也是玉须土成名的根由。

气泡鱼亲近这种土壤,十分有利,能极大地延长气泡鱼的寿命,增添气泡鱼的活力。

正因为这两种土壤,龙鱼、气泡鱼的居所,被方源命名为龙鳞湖、玉须湖。

这两座湖位于狐仙福地东部,是面积第一、第二的大湖。

龙鳞湖的面积最大,里面生存着大量的龙鱼,甚至还有一头荒兽龙鱼。

玉须湖上空,有着大量厚实的云层,云层上栽种着大量的星屑草,形成空中草原。一群群的星萤虫群,以星屑草为食,在草原上飞舞缭绕,形成漫天的斑斓星光。(未完待续。)<!--80txt.com-ouoou-->

\end{this_body}

