\newsection{两方之争}    %第一百九十五节:两方之争

\begin{this_body}

%1
中洲炼蛊大会,第十二场。

%2
比试已经进行到了关键时刻,全场的焦点只有两人。

%3
一位身穿黑袍,身材魁梧雄厚,头戴着面具,操纵着眼前热焰,正是方源。

%4
另一位则一身素白长袍,双鬓如雪,眉心一道竖纹,乃是方槐。

%5
“这是两‘方’之争!”

%6
“方槐好生厉害,一连用了三个炼道杀招,居然将差距大大缩短了。”

%7
“即便如此,方源的优势还是有的。唉,他太厉害了,才思之敏捷可谓此届大会第一人!这一次炼蛊,他从一开始,就遥遥领先。”

%8
“还有变数呢。不管是方源还是方槐,手中都还有一次权利。只要动用了,就能让对方手中的仙材削除一份!”

%9
不管是场上场下,观战者们都在议论纷纷。

%10
这场比试进行到这里,其他的蛊师都已经遭到淘汰,只剩下方源和方槐二人。

%11
方源目光沉静,全神贯注地操控着面前的火焰。

%12
“方槐……”他甚至还有闲暇,瞥向右侧的方槐,观察对手。

%13
方槐同样是蛊仙,乃是前世的本届第三。他是木道蛊仙,纵观整个比试,也就这场题目最不利于他。

%14
方源便趁此良机,前来挑战,企图将他提前淘汰。

%15
但没想到方槐居然在中后期,忽然发力,一举连用三个炼道杀招,甚至其中有一道仙级连道杀招,一下子又拉近了和方源的差距。使得这场炼蛊的胜负结果,又再度悬念重生。

%16
“不愧是前世的第三。真是难缠……呃!”忽然间,方源神色一变。

%17
他眼前的火焰。扑腾两下,里面的蓝孔雀的头颅刚熔解了一半。就陡然化为灰烬,落在地砖上。

%18
“这?!”方源面色不变,心中却是猛地一沉。

%19
“糟糕,碰到失败的概率了!”他旋即反应过来,猛地将手抓向右侧,在那里还剩下最后一只蓝孔雀。

%20
但就在他斩杀蓝孔雀,取其脑袋投向火焰的时候,那边传来方槐的声音:“我要动用权力,剥夺方源手中的炼蛊材料蓝孔雀!”

%21
主持场面的长老。立即反应过来:“准。”

%22
话音刚落,方源身边的那只蓝孔雀就被摄取出来,离开方源身边。

%23
全场哗然!

%24
“原本蓝孔雀一人只有两只,方源要取孔雀脑袋炼蛊,结果失败了一次。本来还剩下第二只蓝孔雀,结果却被剥夺了这项材料!”

%25
“漂亮,真是漂亮的选择。方槐这个时机抓的太好了,这下方源麻烦了。”

%26
“不错,方源明明已经到了最后几步。但失去了关键材料,只能改变蛊方,绕一个大圈,再接近终点。反观方槐。他原本就和方源差距不大,这一次若能赶超,说不定就能将方源淘汰!”

%27
“方源要被淘汰了吗?!”

%28
意识到这一点。全场哗然。

%29
最近以来,方源主动出击。屡屡淘汰炼蛊强者,强势已经深入人心。

%30
外界对他的实力预估频频上涨。普遍认为方源已经有了炼道宗师的境界。

%31
甚至有少部分人,觉得方源展现出来的实力,已经超越了炼道宗师这一层次,达到炼道准大宗师境界。

%32
炼道宗师也就罢了,炼道准大宗师的这个评价,那就有些高得离谱!

%33
流派境界划分四等:普通、大师、宗师、大宗师。炼道大宗师历史上只有三位,号称为炼道三老,分别是长毛老祖、天难老怪、空绝老仙。炼道准大宗师是仅次于三老的境界,北原公认的只有四人达到,即为药皇、慕容雪祥、万寿娘子、妙狸夫人,这四人合称为当今北原炼道四能。

%34
在炼道造诣上,方源和药皇、慕容雪祥、万寿娘子、妙狸夫人这些人同级,这当然不可能。

%35
不过也难怪世人这样猜测。

%36
方源展现出来的实力,实在过于强大。对于题目的熟知,让他每一场都将第二名甩在身后。

%37
“常在河边走,哪能不湿鞋?这一次方源出击,想要淘汰方槐,终于踢到了铁板了。哈哈哈。”不少人幸灾乐祸起来。

%38
“不错,他太嚣张了。我早就看不惯他了,方槐干得好!”

%39
“这个方槐太卑鄙了,方源大人你要加油啊。”场中有不少的仙鹤门弟子,为方源摇旗呐喊。

%40
方源擒拿了方正回去,仙鹤门并无过激反应,好像是默认了这件事情。

%41
“方源,这一场我赢定了。”方槐朗声三笑,自信十足,似乎已经将胜利的果实收入囊中。

%42
“是吗?”方源冷笑,手掌凭空一抓,将身边的无头孔雀尸体,抓到手中。

%43
炼蛊材料中本有两只活的蓝孔雀,方源斩杀了其中一只,取其鸟首,结果遭到失败。第二只活孔雀却被剥夺。

%44
现在他手中的这头无头孔雀尸体,正是第一头遗留下来的。

%45
方源五指张开,一股浓郁的血色雾气,从掌心出喷涌而出。几个呼吸的功夫,雾气就将无头孔雀尸体牢牢包裹,让外界看不清里面的状况。

%46
“这又是邪恶的血炼之术!”

%47
“哼,方源动用血炼之法,已经不只一次了。为什么这样的人,会是仙鹤门的长老?!”

%48
“中洲十大古派之一的仙鹤门,也开始堕落了吗?”

%49
很多蛊师看到这一幕,纷纷叫嚷起来。

%50
甚至有人喊道:“这样的人,怎么没上诛魔榜?”

%51
方源堂而皇之的,一再动用血炼之法,让仙鹤门承受了很大的舆鹿力。

%52
“你们懂什么?血炼就血炼,只是手段而已。用之正则正,用之恶即恶!一切都看蛊师自己!”有仙鹤门的弟子不忿。大声反驳。

%53
但立即就有带队的长老,厉声呵斥他:“噤声。不要乱说话!”

%54
这位弟子只得低头,悻悻坐下。

%55
一会儿工夫。血雾渐渐散去,孔雀无头尸体,只消失了两成。但是却多了一只蓝孔雀的脑袋。

%56
“这是血道的炼蛊杀招,竟然如此玄妙,可以采集尸体,转变材料。”

%57
“这种材料的转换加工,其他流派也有,但这样的速率,也太快了吧?”

%58
众人的目光又被吸引过去。

%59
方源将这只孔雀脑袋。投放进火焰当中。在火焰的灼烧下,孔雀脑袋开始渐渐熔化。并且,熔化出来的一点一滴都汇入火焰底部的雏形蛊虫上去。

%60
片刻之后,鸟首全部熔解,汇入了蛊虫雏形里去,方源进入下一步。

%61
“大局已定了。”看到这一幕,有人叹息起来。

%62
“是啊,这最后几步毫无风险,若不出意外。方源已经赢了。”

%63
“刚刚那个意外,出现的可能本就很低。再出意外的可能,就更低了,几乎是不可能的事情。”

%64
“唉。没想到方槐也没阻止得了方源。”

%65
就在众人唉声叹气的时候,方槐心中却在大笑:“就是这样,就是这样。认为自己胜券在握吧。我剥夺你的蓝孔雀,拖延时间的目的已经达到。凭借我最后的炼道杀招。必将一举超越你,成为本场最终的胜者。方源。我还要多谢你。你主动送上门来,我击败你,就等于胜过你和你击败过的蛊师。哈哈哈,真是妙极!”

%66
就在这时,方源开口:“我要动用权利,剥夺对方的一项材料。”

%67
主持长老立即回应:“请说。”

%68
方槐微微一惊,旋即冷笑,心中暗道:“剥夺吧,尽管剥夺吧!”

%69
方源瞥了一眼方槐,目光平静如水:“剥夺他的炼蛊材料蛮牛瞳珠。”

%70
方槐嘴角的冷笑凝固!

%71
主持长老却是一愣。

%72
观众们大惑不解,牛瞳珠这种关键的材料,方槐不是早就用了,已经融入蛊虫雏形中去了吗?

%73
方源要剥夺对方,早就用过的炼蛊材料,这是规矩不允许的事情啊。

%74
但方源坚持己见,这次对方槐直接道:“方槐,你还不将你火中暗藏的蛮牛瞳珠拿出来?你这是要违反本场的比试规则吗?”

%75
方槐的脸上,充斥着难以置信的神情。

%76
他暗藏蛮牛瞳珠,乃是他的独门手段,竟然被方源察觉到了?蛮牛瞳珠是关键的炼蛊材料,方槐准备已久的用来确定胜利的炼道杀招,就是以此为主。

%77
没有了蛮牛瞳珠,方槐就是巧妇难为无米之炊,还拿什么来战胜方源?

%78
但是若不拿出来的话,也蒙蔽不了其他人。

%79
就算暂时蒙蔽住了,方源提出异议,炼蛊大会方面进行调查,方槐必将身败名裂!

%80
所以,方槐拿也得拿,不拿也得拿!

%81
在众人惊愕的目光中,他咬牙切齿地从火中,取出蛮牛瞳珠。

%82
“给你!”他气得将蛮牛瞳珠,扔在地上,然后他直接散去炼蛊火焰,转身就走。

%83
此举无疑是主动认输。

%84
全场恍然,又再度哗然。

%85
方源却不管这些声音,安安稳稳地将手中蛊虫炼成,取得最终的胜利。

%86
“终于胜了。如此一来,进入前六问题不大了。”方源沐浴在众人惊叹、赞赏、钦佩、嫉妒的目光中,面无表情,心中则吐出一口浊气,“幸好我有前世记忆,知道方槐的这一招杀手锏,不然还真让他赢了去!”

%87
他表面风光,但实则心理压力巨大。

%88
本身实力只是炼道准宗师,就算有重生的优势,淘汰这些强者也是很艰难的。

%89
火工龙头那种情况,只是特例。

%90
方源战胜这些人,绝非外界所看到的“云淡风轻”、“优势很大”。

\end{this_body}

