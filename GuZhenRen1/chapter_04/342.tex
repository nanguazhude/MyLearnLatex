\newsection{继承盗天真传!}    %第三百四十三节:继承盗天真传!

\begin{this_body}

就连当事人方源,也不知道为什么。[看本书最新章节请到棉花糖小说网www.mianhuatang.cc]

在星宿仙尊的歌声,再次萦绕耳畔的时候,他的心中就莫名其妙地多了一些信息。

这些信息告诉他,就在这落魄谷的某处,潜藏着盗天真传!

方源怦然心动。

同时,又很怀疑。

“这会不会是中洲蛊仙留下的一个陷阱呢?其实并没有什么盗天真传,而是一个智道手段,专门来影响潜入谷中的敌人的心智?”

方源迅速地冷静下来,开始检查浑身上下,还有周边环境。

星宿仙尊的歌声出现得突然,消失得也很快。

方源检查自己,没有发现任何的不妥。

又细察身边的环境,落魄谷中布置着许多的蛊阵。还有大量蛊阵的残骸。

不过,方源并没有发现有什么智道蛊阵。

他低头沉吟,左思右想,觉得此事虽然透着蹊跷,但是可信程度还是很高!

为什么呢?

一来是因为星宿仙尊的歌声,二来是信息中的内容指示方源,要进入盗天真传,就得用开门蛊。

“我已经是智道宗师,寻常的智道手段恐怕影响不了我。更别说中洲蛊仙当中,唯一的智道蛊仙老算子已经阵亡。其实证明真假的办法很简单,就是自己亲自去探一探。”

方源决定下来,小心翼翼地向目的地靠近。

接近的途中。他绕过四个蛊阵,还破解了三个蛊阵。

中洲蛊仙在落魄谷外围,布置了大量的蛊阵。这是因为,他们长期在外围攻打影宗。等到攻打进谷中,影宗蛊仙布置的蛊阵。都被中洲蛊仙们破坏拔出。

突围战之后,中洲蛊仙正式占据了此谷。而那个时候,激战已经结束,也没有必要多设置什么蛊阵了。

再加上方源日渐高深的智道造诣,这些蛊阵根本对他构不成阻碍。

片刻之后,他来到星宿仙尊歌声指示的所在之地。

方源察看了很久,并未急着催动开门蛊。

结果他没有发现任何的端倪!

这里非常的安全。至少方源费尽心思和手段。都没有探测出什么来。

再三确认之后,他双目一闪,终于下定了决心,催动了开门蛊。

前世的时候,方源也在落魄谷中,动用过开门蛊,那时根本没有任何效果。方源一直以为。盗天魔尊的布置,已经被影宗毁掉。

但今生这次,方源催动之后不久,就有异变发生。

一道光门,乍然出现半空处。

随后一道光柱,将严阵以待的方源照住。

方源心中一惊,想了想,任由光柱将他吸摄进去。

下一刻,方源进入了盗天真传空间。

这片空间,空空荡荡。十分广阔。

在他的身后,就是那道光门,并未因为他的进入而消失不见,仍旧维持着。

只是光门的体积,越来越小。

若是缩小到极限,恐怕会消失。

呼呼的风声从不远处传入耳中,方源挪移目光看去。顿时身躯微微一颤。

“大同风!”

他暗呼一声,目光十分忌惮。

他十分清楚大同风的威力,此风能同化一切,不断抓大自己。当初,王庭福地就是摧毁在大同风下。

此时,数十个大同风,形成龙卷巨柱,在真传空间中缓缓移动。

方源迅速后退,拉开自身和大同风的距离。

间距数百步之后,方源这才心中稍定,细细打量这片真传空间。

“想不到这里真的有一处空间!那股莫名其妙的信息,是真的吗?”方源谨慎,他还是对星宿仙尊的歌声提示,抱有疑虑。

他并没有急着去在这真传空间当中,找寻什么真传,而是将心念集中在关门蛊上。

“若是开门蛊能进去,那么关门蛊就能关闭关门?”

方源便催动关门蛊。

果然下一刻,他视野中的那道正在缩小的光门,就乍然消失。

方源眼中精芒一闪,又再次催动开门蛊。

在这只五转凡蛊的作用下,又一道全新的光门,在他的眼前骤然形成。

方源跨越这道光门,立即脱离了盗天真传空间,回到了落魄谷中。

“这么说来,那股信息应当是真的了。不,更准确的说,关于开门蛊、关门蛊这部分,应当是真的了。”方源心中不免期待起来。

他再次进入真传空间。

“按照信息中的内容,这些大同风虽然货真价实,但是对我而言,却是无害的。而我只有消除了这些大同风,才能引发真传的继承。”

方源小心翼翼地接近一道大同风,现出原形,伸展八只粗大的手臂。

他将其中一只手,慢慢地探入风卷巨柱之中。

忽然之间,低沉咆哮的大同风卷,就彻底消散。

方源双眸中精芒爆闪。

他口中呢喃:“难以置信!这大同风能同化一切,但是在这真传空当中,却受制如此。就算布置这里的蛊仙,不是盗天魔尊,单凭这手,也绝对是超绝的大能!”

虽然消散了一道大同风卷,但方源没有大意,每一次碰触风柱,都十分谨慎。稍有不对,他就会闪电般后撤。

每次移动一段距离,他就利用开门蛊、闭门蛊,将光门始终打开在最靠近自己的位置上。

如此一来,就算是有什么情况发生,来不及催动定仙游,方源还能以更快的速度,穿梭光门,回到外界去。

一道道大同风卷,在方源的努力下。接连消失。

一段时间之后,只剩下最中央的一道龙卷风柱了。

方源稍微打量了一下,就发现:这道龙卷风柱不仅具备最大的体型,而且旋转的速度也是最快的。

风声在方源的耳畔呼啸,隐隐约约的。在风柱中央似乎藏着什么东西。

“难道风柱当中的,就是盗天魔尊的真传?”方源心中不禁猜想。

他伸出手指,碰触这道墨绿色的风柱。

大同风骤然消失,现出一个人!

这个人,盘坐在半空中,身着红白大袍,泄露出一丝微弱的八转气息。

“凤九歌!”方源低呼一声。下意识就后退。

但凤九歌比他反应更快。他陡然睁开双眼,绽射出寸许的精芒。

刺眼的精芒在方源身上一扫之后,凤九歌化成一道虹光,以迅雷不及掩耳之势,穿过方源身畔,钻入光门之中!

整个过程,连眨眼的功夫都不到。

方源反应过来。第一个动作就是催动关门蛊。

光门闭合上,方源提在嗓子眼的心,才猛地落下来。

“凤九歌怎么会在这里?!难道他被困的地方,就是这处真传空间?他既然在此,那么其他人呢?秦百胜,还有傲雪、凌梅两位仙子呢?”

方源连忙探查,但真传空间中一片空荡,只有一只信道凡蛊,悬浮在方源的面前。

这只信道蛊虫,是凤九歌留下来的。

方源拾取之后。没有发现什么不妥,这才灌注神念进去查看。

“方源,原来你是天外之魔,我会为你保守这个秘密。要问我为什么知道你?因为我是凤金煌的父亲。今日你救我一命,来日我会还你一命。我走之后,真传空间只有你一人,盗天真传就会被你继承。后会有期罢!”

“凤九歌……”方源摸索着这只信蛊。神情复杂。

他叹了一口气,将这只信蛊揣入自家仙窍。

这时,整个真传空间开始暗淡下来。

原本光线明亮,须臾功夫,就化为一片黑暗。

在这黑暗中,只留下一片微光之地,正是方源立足的地方。

一个饱含磁性的男子声音,从黑暗中传来。

“远方的游子啊。”

“你和我一样,也是有家却不可归的可怜人。”

“我的本名叫做本杰孙,世人称呼我盗天魔尊。其实我只是一个想回家的流浪儿。”

“这个世界就像是个绝望的牢笼。”

“如果你也想回家,那么就接受我的这份馈赠吧。”

“我相信,它肯定能给你带来帮助。但是你要想归回家园,还得靠你自己的努力!”

“另外,这份馈赠的名字,叫做鬼不觉。”

下一刻,方源出现在落魄谷中。

他的浑身上下,都缭绕着一层灰色的光芒。

方源顾不得这层灰色的光辉,而是催动侦查和防御手段,警觉周围。

凤九歌没有出现。

“看来他身上的伤势,极其严重,甚至连半成的战力都没有了。若非如此,兴许在真传空间之中,他就对我动手了。”

“不过……此人倒的确恩怨分明。有恩必还,有仇必报。他死后的传记,也说明了这点。他说要回报我的救命之恩,恐怕……是真的。”

方源沉思着。

这样想的话,凤九歌没有对方源直接动手,或许并非因为战力不足,而是方源对他有救命之恩。

不管怎么说,方源这次赚大了。

对凤九歌施恩,这份巨大的人情,将来或许能够派上巨大用场。

得到鬼不觉,虽然方源还摸不清这究竟有什么用处,但毕竟是盗天真传啊!

除了这两者之外,还有这座落魄谷!

“凤九歌恐怕已经离开落魄谷了,现在就是收取此谷的最佳时机。”想到这里,方源决定动手。

“怎么回事?!”在谷外看守的回风子,被落魄谷的景象惊动,他双眼瞪得溜圆,感到难以置信,“竟然有人在拔落魄谷?!”(未完待续。)

\end{this_body}

