\newsection{方源前辈,我要和您赌斗!}    %第一百八十三节:方源前辈,我要和您赌斗!

\begin{this_body}



%1
众目睽睽之下,方源默默站起身来,缓缓踱步,走出场外。

%2
“好深厚的炼道造诣!居然一心二用,将之前的九只鬼火蛊雏形放缓炼制速度,又重新炼制最后一只鬼火蛊,速度极快。鬼火蛊炼制本身,就十二分耗费心神,他居然到了这一刻仍旧游刃有余,难道魂道或者智道方面也有修行?”安寒惊疑不定。

%3
“大供奉,大供奉?”安寒身边的弟子小声提醒道。

%4
“什么事?”安寒回过神来,不禁面上一热。原来自己不知不觉间,已经从座位上站起身来,如此失态,大损面皮。

%5
这时,飞霜阁阁主向安寒投来目光。

%6
炼道大会,自然要以炼道为手段,找回场子。

%7
安寒心头一跳,知道阁主的意思,只得硬着头皮下场。

%8
见到自家大供奉登场,场外很多飞霜阁的弟子,都轰然叫好,一扫刚刚的颓丧之色。

%9
就算是外人,见到声名在外的安寒出场比试,也不免投去关注看好的目光。

%10
人群中,却也有明眼人。

%11
此时一位老蛊师就小声地叹息道:“这个安寒,不妙了。”

%12
“为何呢,师傅。”在老蛊师身边,站着一位少年,疑惑不解。

%13
这对师徒,正是之前前往五德门报名,走在方源身旁时,被替安寒开道的壮汉们推搡倒地的那对散修。

%14
“那魔道蛊师方源。极为了得。炼道造诣深厚雄浑,刚刚那手何止是一心两用?要做到他这一点,为师也只有在人生巅峰的时候,才可完成。这其中的难度,外行人看不出来。只有越是内行的炼道蛊师,才越会明白。这安寒本事也是不俗的,但正因为如此,他更明白自己比不上魔修方源。你看他刚刚都失态地站起身来,可见心神已被方源所摄。”

%15
老蛊师侃侃而谈,当然声音很低:“以他如此的状态,想要炼蛊。恐怕一半的真实水准都发挥不出来。而且最关键的一点。方源既已经夺得头名,就算安寒再怎么努力,在这方面也改变不了。除非他另辟蹊径,用时远远小于方源,方能替飞霜阁挽回丢失的颜面。可惜他若是有这个才华,也不必蜗居在飞霜阁中这么多年了。”

%16
少年蛊师连连点头,赞同地道:“师傅。你说得对。这个飞霜阁的阁主,是个炼蛊的外行,此次派兵遣将,完全是一记昏招,急于求成。这就像师傅教导我的,炼蛊不能急于求成,不能心急。想来主持门派,也是一样的道理。”

%17
老蛊师听了这话,双眼一亮,欣慰地看向自己的这个唯一的徒弟:“不错。徒儿你果然天资聪颖,有慧根悟性。世间大道,其实殊途同归。你能从炼蛊的道理,影射其他方面,可见你炼道境界已然登堂入室,这第二次比试一定能大获成功。”

%18
少年蛊师微微一笑:“这些都是师傅您教导有功,毫不藏私。我若不是被师傅你捡来。早就饿死了。能有今天,是师傅您的栽培。徒儿原本就胸有成竹,只要动用十三种炼蛊手法,便可夺得第一。但现在头名已被那魔道蛊师夺走,师傅,就让徒儿来为您抢回那只绿曜蛊吧。”

%19
“什么?”老师傅闻言大惊失色,连忙阻挡,“徒儿,万万不可!”

%20
但已经迟了。

%21
少年蛊师初生牛犊不怕虎,已然高声呼喊:“方源前辈,小子斗胆,想跟您赌斗!”

%22
这番情景交代起来较长,其实时间只是过去一点。

%23
方源才刚刚下场,并未远离,因为还要和飞霜阁交接头名奖励。那边安寒起身,正要入场。

%24
少年蛊师一语,激起千层浪,立即将群众的注意力都吸引过去。

%25
“赌斗了,赌斗了!”

%26
“那个少年是谁,居然敢和那个五转大魔头赌斗?”

%27
“我知道他,是上一场五德门的第二,货真价实的炼道天才。”

%28
“乳虎不惧老狮威,这下有好戏看了!”

%29
老师傅面色惨然,已经阻止不及,立在原地手足无措,心中万分的担忧和悔恨:“自己只顾着传授徒弟炼道,忘了教导他世事艰难凶险。千不该,万不该啊!”

%30
而万众瞩目的少年蛊师,则昂首缓步,脱离人群,来到方源的面前,目光炯炯,神情坚毅。

%31
大供奉安寒刚刚一只脚,踏入场上,却发生了这样一件事情,不免心中气劲又是一沮。

%32
他原本是风云人物,一举一动都引人眼球,但今天的遭遇实在有点糟糕,自己出丑不算,还被公然忽视了,不禁又怒又恨。

%33
“赌斗?是谁要和我赌斗?”方源微微一笑,头颅微转,看了过去。

%34
中洲炼蛊大会,一百年才举办一次,是名冠五域的大盛事。按照赛事分,有大比试,小比试。

%35
五域大比试就是方源目前参加的,规模最大的比试。先是从中洲各处赛点,一关关闯下去,大量的蛊师被淘汰,最终只有很少的一撮人,才能参加最后的决赛。

%36
小比试则通常是地区举办,按照往届惯例,分为中洲小比试,南疆小比试,北原、东海、西漠等等。这是五大小比试,按照地域来源,划分出来。算是比较权威的。

%37
还有很多小比试,比如东海岸这边,会由飞霜阁、五德门等联合举办。在大比举办的同时,淘汰下来的蛊师们也会参加这样的东海岸小比试。

%38
数量最多的小比试,是一个个小圈子发起的。比如好友三五人,约定个时间,齐出奖励,或者干脆没有奖品,进行相互比试。

%39
若按照比试是否公开划分,又有公开斗,私下斗。公开斗,自然是可以围观的。不过大多数是提前发出邀请函,请有限的嘉宾观赏评定。私下斗,则是不允许围观,结果也不公布。

%40
大比试是最公开的公开斗,不限制任何人观看,但比试现场严禁观看者动用任何蛊虫。

%41
中洲炼蛊大会每一次举办,都会在中洲各处掀起全民炼蛊的潮流。举个例子,就仿佛是地球上的世界杯。每四年举办一次的全球性的足球盛会,通常这个时候,就算不踢足球的,都会耍耍足球。不关注足球的,也会多看几眼。

%42
放到这个世界,炼蛊是蛊师修行的三大方面之一,比之娱乐性的足球要更加重要。因而掀起的炼蛊风气,也更加宏大持久。这个期间,有关炼蛊的经验、心得都会得到巨大的交流,极大地促进炼道的发展,几乎让每一个参与其中的蛊师都因此受益。

%43
而要按照比斗的方式,炼蛊大会又可分为门派斗,赌斗,题斗,擂台斗等等。

%44
门派斗,是两个或多个门派,派遣相同数量的炼道蛊师,进行炼蛊的比试。通常而言,是因为利益纠纷,用这种方式分出上下,解决矛盾。

%45
题斗,是一方出难题,另一方解答。

%46
擂台斗,则是在某项炼蛊十分厉害的蛊师,摆下擂台,邀请各方炼道强者上台比试。先是摆擂者出一份奖品,每个挑战者也自带一份奖品。这种擂台斗,往往积累的奖励十分丰厚。

%47
而赌斗呢?就是方源此时遭遇的情况了。

%48
“是晚辈要赌斗。”少年蛊师走到方源的面前,不卑不亢。

%49
方源打量他一眼,见这少年斗志昂扬的样子,感觉有些意思,便道:“小子,你既然是要赌斗,那就拿出点东西,能入我法眼的来赌。否则我是不会赌的。”

%50
方源当然有权利拒绝赌斗,炼蛊大会中可以提出赌斗,但从未有“不能拒绝必须强制参加”这种不近人情的规定。

%51
少年蛊师郑重地点点头:“我这里有一张五转蛊虫的蛊方。前辈您是五转强者,这张蛊方正适合您。晚辈要赌的,则是您曾经赢得的四转绿曜蛊。”

%52
方源点点头:“绿曜蛊是四转珍稀蛊,价值和寻常的五转蛊相当。小子,你知道我的流派是什么吗?你确定你给出的蛊方是符合我流派的?”

%53
少年蛊师一愣:“前辈您不是炼道流派吗?”言下之意,他拿出来赌斗的蛊方,便是炼道蛊方。

%54
方源呵呵一笑:“我主修流派当然不是炼道。小子,你竟然只用一张蛊方,就想和我对赌一只四转珍稀蛊。你知不知道炼制五转蛊虫,一旦失败,即便是我等也会伤筋动骨。你这是害我,还是觉得我足够愚蠢,能答应这样的赌约呢?”

%55
少年蛊师张口欲言,喉结滚动几次,却说不出话,一时间僵立在原地,一动不动。

%56
他的师傅连忙赶过来,向方源深深一礼:“既然阁下拒绝了赌斗,那么我们就不赌了,不赌了。”

%57
“怎么又不赌了?”身边众人见没有好戏可看,立即嘘声一片。

%58
“五德门比试的第一名神秘魔道蛊修方源、第二名少年蛊师郑山川的龙争虎斗啊,居然因为赌资不够,这就看不起来了?来来来,咱们出手帮一帮,都出一点,石头都能汇成山嘛。”有好事者提议。

%59
“不饶大家出手,是小徒鲁莽了,谢过诸位,谢过诸位。”老师傅听了这话,心惊胆战,连忙四处作揖拱手,要搅了这局。

%60
那旁边的安寒眼中阴芒一闪,立即小声吩咐下去。很快,人群中就有人前头,许多人响应号召,纷纷出资,要促使这场赌斗进行下去。

\end{this_body}

