\newsection{暴利!}    %第五十八节:暴利!

\begin{this_body}



%1
中洲时间,成功敲诈了凶雷恶人的三天之后,仙鹤门对外宣布:承认方源为狐仙福地之主,狐仙福地为仙鹤门附庸势力,并且随时可以脱离。

%2
这个惊人的消息,宛若巨石投入湖面,立即激荡起连绵浪花。

%3
一时间,中洲蛊仙界为之侧目!

%4
先前,仙鹤门公开宣称,方源为门派叛徒,将受到征讨。这种事情,在中洲常常发生。很多蛊师,得到奇遇不愿门派瓜分自己的收益,或者受到其他门派招揽等等原因,便选择背叛原有门派。

%5
门派制度的凝聚力,的确比血脉亲缘团结在一起的家族制度,要薄弱一筹。

%6
门派当中,凝聚力最强的当属十大古派。但古派中出现叛徒,也不是没有过。

%7
曾经,古月一代就是夺得了一道血道真传,叛变了仙鹤门,远走南疆,隐姓埋名。

%8
近年来,原本是万龙坞的弟子宋紫星,也叛变门派。在战斗中成长,如今已经成就七转蛊仙,号称血龙,是中洲鼎鼎大名的魔道高手。

%9
因此之前,方源作为一位五转蛊师,叛变仙鹤门,只能算是一个不大不小的新闻。

%10
但现在,仙鹤门大张旗鼓,征讨方源,出乎常人意料,得到了这么一个结果。

%11
一时,不知多少人,原本以为方源作死的,听到这个消息后大跌眼镜。

%12
旋即,更多的内幕消息,从仙鹤门、凶雷恶人。以及方源自身传散出去。

%13
传说中的荡魂山,狐仙福地的庞大实力,神秘势力的崛起。凶雷恶人的仙道杀招,都成为吸引眼球的亮点。而在这当中,最引人瞩目的,还属于胆识蛊的贸易。

%14
胆识蛊居然可以离开荡魂山存储,从而进行交易买卖!这个消息中蕴藏的巨大利益,让各方势力为之心动。

%15
蛊仙们终于明白,仙鹤门为何如此宣布。原来是重利当前!

%16
为了胆识蛊,舍弃些虚名算什么。况且方源承认自家为仙鹤门附庸,也算照顾了这个古派的脸面。

%17
其余九大古派一面大骂仙鹤门无耻。一面纷纷和狐仙福地接洽。

%18
方源的胆识蛊还没有正式开始贩卖,收到的信道蛊虫却已然一大堆,并且还绵绵不绝。

%19
不管大小势力,方源都一一回信。先混个脸熟再说。

%20
以此为契机。方源终于登上中洲舞台。以仙僵资格,狐仙福地之主的身份,闯入各个蛊仙,大小势力的眼中。

%21
几天后,黑楼兰送来第一批魂魄,足有数十万之多。

%22
当然,大部分都是野兽魂魄。

%23
方源全数投放到荡魂山中,狐仙地灵开放荡魂山的全部威能。这些魂魄被震荡成碎片,无数的魂魄精髓洒在荡魂山表面。渐渐凝成胆识蛊。

%24
放着胆识蛊在这边生长,方源则紧锣密鼓的进行另一方面的准备。

%25
他命令手中全部毛民,赶炼基础蛊虫。

%26
连续七天七夜,三班轮倒,紧锣密鼓,炼制出大量蛊虫。

%27
又命石人部落出动人力,建筑石巢。

%28
大半个月后,前一批的魂魄,已经荡然无存,荡魂山上生出的大量胆识蛊。

%29
黑楼兰带着第二批魂魄,再次来到狐仙福地。

%30
方源便带她来到荡魂山附近的石巢。

%31
黑楼兰站在石巢边缘,看着下方的庞然大物,对方源的大手笔感到惊异。

%32
这座石巢,占地数百亩,挖至地下深逾百丈。

%33
石巢外围,是一个天坑圆洞,人为挖掘,直上直下。圆洞天坑半径长达一里,深则至少三里,宛若一个长体水杯。

%34
圆洞天坑中央,矗立着一根粗大庞巨的石柱。

%35
石柱半径有一百零五丈,这个长度比一里的一半还多一点,因此石柱和圆洞天坑之间,并不接壤,双方相距四十五丈,还有一片特意遗留的空间。

%36
石柱如蚁山,里面结构复杂,四通八达。

%37
不管是石柱的外壁,还是圆洞天坑的内部,都挖出无数房间。每一间房间里,都住着一位老毛民,里面都配备了基础的炼蛊工具和材料。

%38
“楼兰仙子请。”方源带着黑楼兰,飞到石柱的顶端,在最中央的地方站定。

%39
黑楼兰不断动用侦察蛊虫,发现不管这石柱,还是包裹石柱的天坑圆洞中,每隔一段距离,都布置了许许多多的蛊虫。

%40
这些布置十分繁琐,让黑楼兰看得眼花缭乱,只察觉出这些蛊虫并非随意布置,而是处处相连,交相影响,用意深远。

%41
“难道说……”黑楼兰心中正猜测时,方源开口,“仙子,我这边已经准备就绪,就等你用力气仙蛊了。”

%42
“哦?请指教。”

%43
“仙子只需催动力气仙蛊,令力气自由散发即可。”方源开口道。

%44
黑楼兰依言而行,磅礴的力气汇成一股股的气浪,四处逸散。力气越加浓郁,渐渐形成雾状,弥漫开来,笼罩了整个石巢。

%45
石巢中无数凡蛊相继催动,亮起璀璨各色的光点,仿佛夜空中的繁星。

%46
在这些蛊虫的合力之下,磅礴如雾海般的力气,被一丝丝,一股股地抽动,几个呼吸之后,全部的力气都充斥在石柱和圆坑之间。

%47
十几个呼吸后,这些力气不断流转,形成旋风状,包裹着石柱流窜。

%48
“炼蛊开始。”方源一声令下,房间中的毛民们,原本就整装待发,此刻统统睁开双眼,开始着手炼制蛊虫。

%49
每当他们需要力气作为炼蛊材料时,房间外急速流动的力气,就会自发地投入其中。

%50
这样的气象,让黑楼兰也赞不绝口地道:“方源你这石巢构造宏大。居然能容纳数千毛民同时炼蛊,真乃奇思妙想。”

%51
方源笑了笑:“我没有告诉你,我也是一位炼蛊大师么?”

%52
事实上。这石巢乃是他前世,五域乱战时期的发明。

%53
持续不断的大小战争,让蛊师、蛊虫都大量损耗,各大势力需要补充大量蛊虫,渐渐研究出石巢这种能够容纳许多蛊师,一同炼蛊的建筑。

%54
“我把它命名为方源石巢,专门大量炼制凡蛊的场所。我将气囊蛊的炼制。细分为十几个步骤。最上层房间中的毛民们,完成第一个步骤,就将半成品即刻送到下一层房间里去。下一层进一步加工后。又将半成品送入第三层。这样一来,各个毛民专精某一步骤,不仅防止蛊方泄露,而且还能更快熟练。增加炼蛊成功的可能。”

%55
方源毫不客气。将前世的他人发明占为己有。

%56
黑楼兰忍不住又赞叹一句:“这的确是大量炼制凡蛊的不二法门,从某种程度上,甚至可以降低炼蛊者的标准。毕竟他们只需要完成其中某个步骤就可以了。而方源你则从繁琐重复的炼制内容中抽出身来,只需控制材料即可。”

%57
“也有一些蛊方,需要炼蛊之人时刻操练不停。好在气囊蛊不需要这样。”方源道。

%58
黑楼兰点头道:“我原本还担心,气囊蛊数量太少,会影响到胆识蛊的贩卖。毕竟每只气囊蛊只能储藏一只胆识蛊。但现在亲眼目睹这个石巢,我放心了。这些胆识蛊。你打算怎么卖?”

%59
方源早有计划:“主要还是卖给中洲十大古派,和十大古派的洞地蛊。我都已经搭建好了。今后就通过洞地蛊交割钱货。剩下来的一小部分,则挂到宝黄天中去卖。”

%60
“不打算专门卖一点,给中洲其他的门派吗?”

%61
“不了。现在的中洲还是十大派为主,我们的身份、地位已经很敏感,不需要在此时候撩拨古派的神经。现在的安宁得之不易,我们只需要静静地赚取利润就可以了。”

%62
黑楼兰听到这里,又点点头,这一次她彻底的放心了。

%63
时光匆匆,狐仙福地中又过了半个多月。

%64
第一批的胆识蛊,已经贩卖出去,刨除成本,以及分给黑楼兰、黎山仙子的四成利润,方源获得了四十三块仙元石。

%65
狐仙福地时间一个月之后,第二批胆识蛊卖出去,方源的仙元石储备又增添了四十六块。

%66
第三批胆识蛊,则赚了四十七块。

%67
市场反馈极好,十大古派需求量每次都在增长,宝黄天中更是销售火爆,往往一两天后,就贩卖一空了。

%68
胆识蛊到底曾经是连幽魂魔尊都推崇的宝物,壮魂的效率极高,是普通手段的十多倍。

%69
事实上,不仅仅是魂道蛊师需要,奴道、炼道等相关流派,也需要魂魄的底蕴。

%70
蛊师升仙,魂魄也是人气是否浓郁的重要因素,谁也不会嫌自己的魂魄太强。

%71
方源开始体会到垄断生意的好处。

%72
他可以一手操控市场,掌控局面。他想卖给谁,就卖给谁,想卖多少,就卖多少。

%73
不过他没有胡乱提价,因为壮魂不仅有胆识蛊这个手段,价格太高,反而促使其他人去使用原先的手段。如此一来,反而让方源的收益下降。

%74
三批胆识蛊卖出去后,方源手头的仙元石囤积了一百多块。他开始购买大批的毛民,在石巢中开凿出更多的房间,扩大生产规模。

%75
毛民越多,炼蛊的熟练度有增高,使得方源每一批的胆识蛊贩卖,获利也跟着增多。

%76
狐仙福地时间五个月后,他的获利终于稳定下来,每一批的胆识蛊,他能赚到四十八快仙元石的纯利润。

%77
这是货真价实的暴利!

%78
更可怕的是,胆识蛊是一次性的消耗蛊,用了就没有了,而市场的需求似乎永无止境。

%79
黑楼兰、黎山仙子开始觉得自己是个拖累。

%80
因为他们负责收集魂魄。

%81
但如今,魂魄原料的数量,渐渐成为胆识蛊贸易的最主要限制。

\end{this_body}

