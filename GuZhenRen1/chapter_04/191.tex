\newsection{方正的噩梦}    %第一百九十一节:方正的噩梦

\begin{this_body}

方源的出现,出乎众人的意料。按照中洲炼蛊大会的赛制,到了第八场及其以后,每一场的比试地点,都只会有一位蛊师可以晋级。

因而渐渐形成惯例,整个中洲的比试地点都化为地盘,各个强大的炼道蛊师宛若猛兽盘踞一方,霸占地盘。

在没有逼不得已之前,没有哪一头猛兽提前跨界,去挑战另一头猛兽的。

这绝对是不明智的。

晋级越多,得到的荣耀也就越高,得到的奖励也越多,提前对决,简直是鹬蚌相争渔翁得利。

因此很多蛊师看到方源出现时,不免就开始猜测:“难道方源和火工龙头之间有什么私仇?”

主持这场比试的驱邪派长老,心中古怪,面无表情。

方源的登场,符合大会的规定。只要在规定的时间内,进入任何场地中比试,成绩便能得到认可。

因此,就算其他的有心人想要阻拦,众目睽睽之下,也是阻拦不住的。

“这位名叫方源?难道说,就是我们门派中的那位大名鼎鼎的狐仙福地之主?”仙鹤门中的弟子,也开始猜测起来。

“说起来,我派中的那位方源大人,还真是神秘。我至今都未见过他的真面目呢。”

∑,⊕.

“这位就是方源?不大可能吧。中洲那么大,同名的蛊师,又不是没有。”

“方源和凤金煌之约,已经广为流传。咱们仙鹤门有许多的免试名额,足以让方源养精蓄锐。到第十场之后参赛。”

仙鹤门弟子们议论纷纷,许多人悄悄地把目光集中在方正的脸色。

方正双唇紧抿。脸色显得苍白,双手缩在宽大的袖口中攥得紧紧。同为亲兄弟。方源一出现,一股直觉就告诉方正,那就是他的亲哥哥!

方正猝不及防。

童年的阴影,在这一刻,猛然降落,再次笼罩住他。像是一双黑手,紧紧地掐住他的脖子。

方正呼吸都感到困难。

自从那场昏迷中醒来后,他一直在极力回避的噩梦,又再度袭上心头!

万龙坞那帮人嚣张的呐喊声。助威声,也不禁一滞,渐渐低弱下去。

方源在之前展现出相当强大的炼道造诣,连夺七场第一,和火工龙头是一样的成绩。[\&\#26825;\&\#33457;\&\#31958;\&\#23567;\&\#35828;\&\#32593;\&\#77;\&\#105;\&\#97;\&\#110;\&\#104;\&\#117;\&\#97;\&\#116;\&\#97;\&\#110;\&\#103;\&\#46;\&\#99;\&\#99;更新快,网站页面清爽,广告少,

万龙坞的长老和弟子们,也不得不承认,方源是一个可怕的劲敌!

“听说这个方源,和咱们的火工龙头大人成绩一样,连续七场第一呢。”

“是不是就是和凤金煌赌斗的那位?”

“要真是那位的话。那可就是狐仙福地之主,富得流油啊。”

“那又怎样?哼,你们也不看看排名。火工龙头大人可是排在第七位的。他方源排了多少?不过是三十多位而已!”

“不错,他这次来。简直是自取其辱。我相信火工龙头大人,一定能击败他。”

虽然万龙坞的人都这样说,但其他观众却不这样看。

方源主动进攻。展现出了很强的侵略性。方源明显不是傻子,这一次跨界出击。很显然定有自信和底牌。

方源不按规矩出牌,也让火工龙头一阵惊愕、恼怒、疑虑。

于是火工龙头放声试探:“仙鹤门方源!你来我这里。是希望提前收获失败吗?”

火工龙头真的很想问:你跑过来干嘛啊?好好的在你那个地盘蹲着,安安稳稳晋级不行吗?真是脑袋抽了!

他嘴上责问的同时,心中还有些郁闷我跟你无冤无仇,你跑过来干扰我做甚?你还有凤金煌的赌斗呢。难道是因为我看着好欺负吗?

他的话一出口,场外哗然声顿时更大。

“什么,这个方源居然是仙鹤门中的人?”

“他不是魔道蛊修吗?”

“应该没错了,由火工龙头亲自确定,这还能有假?”

“特大消息啊,没想到居然是中洲十大古派的弟子。这真让人意料不到啊……但他为什么跑过来?仙鹤门提前和万龙坞杠上了吗?没听说过有什么风声矛盾啊。”

就连主持长老,也眼珠子瞪着,呆呆地看向方源。

一时间,方源成为众人目光的焦点。

“哥哥……”方正咬牙,脸色苍白如纸,不知不觉间竟然有一身的冷汗。他呼吸困难而又微弱,像是刚刚剧烈运动,简直要虚脱一般。

幸好此时众人的注意力,都集中在方源身上,没有人注意到他的狼狈。

空窍中,天鹤上人出声安慰,但收效也极小。

其他的仙鹤门人,都是目光暴涨,兴奋地盯住方源。有的人甚至不自觉地站起来。

方源从来没有正式的,在仙鹤门总部露过面。但是他的传说,已经不只一次,在整个仙鹤门广为流传。他神秘而又强大,关于他的小道消息有无数。

据说他是因为天资实在过于出色,一上山就被某位蛊仙收为亲传弟子了。

这当然是小道消息,纯粹的臆测,完全不靠谱。

但仙鹤门高层也无法辟谣,一旦说出真相,中洲十大古派的名誉还要不要了?

而且仙鹤门弟子们更愿意相信眼前看到的证据。

最大的证据就是方正!

方正是甲等资质,如今竟然有五转修为,都成门派长老了。由此可以推断,作为亲兄弟的哥哥,资质是多么的出色了。要不然怎么可能战胜凤金煌,夺得狐仙福地?

此时此刻,仙鹤门一行人都勾起脖子,目光像是胶水一般,牢牢地黏在方源的身上。

他们心中的疑惑和好奇,积累得都快要漫溢出胸膛了。

“真想看看方源大人的真面目啊。”

“应该和方正长老,长相类似罢。”

“唉,干嘛要带面具。偏偏这里作为比试地点,乱动用侦察蛊虫就会被驱逐出去。”

弟子们兴奋地交流着。

同行的那一位仙鹤门长老,终于发现方正的不妥,关切地问询道:“方正长老,你怎么了?有什么不舒服的地方吗?”

“没什么,没什么……”方正似乎惊了一下,连忙答道。

“怎么方正长老好似很惧怕方源,难道这对亲兄弟关系并不好?”仙鹤门长老眼底。闪过一抹异色。

方源的双眸隐藏在面具之后,他先是缓缓扫视四周,在方正的身上顿了顿,最后又徐徐落到火工龙头的身上。

他淡淡一笑,打破良久的沉默,声音显得有些沙哑,傲然向周围人宣布道:“没错,我就是仙鹤门的方源。”

“哈哈,真的是!”

“方源长老,加油!”

“方源方正两位长老,可是我门中的双杰啊。”

反应最激烈的,就是仙鹤门的这群人。一些弟子们几乎兴奋地跳起来,他们感到骄傲,有一股强烈的门派的荣誉感,在被万龙坞那群人打压之后,此时得到伸张,更有一种扬眉吐气的舒爽。

“得意什么?”

“就是!有什么好得意的……”

万龙坞那边很快传来反驳的声音。

方正眼中恨意一闪,咬牙切齿地对同伴道:“不要把我和我哥相提并论!”

这听在旁人耳中,却又换了另一层意味。

“听到没有,方正长老如此天资,都自认要弱于方源大人!”

“那是那是,传闻中方源可是蛊仙的种子,受到高层的全力栽培呢。”

仙鹤门弟子们“压低”声音交谈着,但声音其实一点都不低。

“你们……”方正听到这番话,脸色铁青,脑袋一阵强烈的眩晕。

万龙坞的脸色也变得更加难看。

火药味更加浓烈了几分。

万龙坞的一位长老冷哼,轻声诅咒道:“倒要看看你最后失败时,会是一副什么样子的落魄嘴脸。”

场外纷纷攘攘,场上其他的蛊师已经成了被人忽略的道具。方源继续凝视火工龙头,故意扬声道:“火工龙头,我来这里,自然为了击败你。击败那些无能之徒,实在太令我感到无趣了。你是个稍微像样点的对手,可以给我的晋级之路增添点光彩。所以拿出你真正的实力来吧,不要保留,我记得你最强的手段好像是那个什么疯神烈焰?拿出来对战吧,这样我击败你后,才能稍微添一点满足感。”

听到方源这番挑战的话,不管场上场外,蛊师们都有一股相同的感受嚣张,太嚣张了!

之前,众人已经觉得火工龙头张狂了,但现在看到方源,这才明白,原来“人外有人山外有山”这话说得一点都没错。看这个方源,这才是真正张狂的主儿!

火工龙头心中一惊:“他居然连我压箱底的手段,都知道?!这不可能,我还从未当众露过这手。他究竟是怎么知道的?”

惊疑之后,火工龙头怒气升腾,不甘示弱地大吼道:“好,既然你这小辈如此大言不惭,就让我来彻底地教训你。我要让你明白,什么才是真正的痛!”

同时他在心中怒吼:“我堂堂六转蛊仙,渡过两次天劫,只差一次就是七转,害怕你一个修为垫底的仙僵?小子太目中无人了,我要将你狠狠地踩在脚下,让你颜面尽失!你盲目挑战我,将是你此生最后悔的一件事情!!”

\end{this_body}

