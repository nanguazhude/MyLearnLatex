\newsection{仙蛊喂养的难题}    %第二十三节:仙蛊喂养的难题

\begin{this_body}

%1
三天之后,方源悠悠醒转。

%2
心中一片安宁,再无之前的疲惫和浮躁之感。

%3
“看来僵尸之躯还有一点比不上鲜活肉体,我的魂魄更容易疲惫虚弱,而且烦躁。”方源发现了僵尸之躯的另一个弊端。

%4
魂魄虚弱,可以用胆识蛊壮魂,这很容易解决。荡魂山上就长有不少的胆识蛊,这是壮魂的极品之法。

%5
但魂魄烦躁不安,就需要安魂。安魂最好的方法,是沉迷死境中的迷魂湖之水。这水记载于《人祖传》中,得太日阳莽赞曰:极品美酒。

%6
方源可没有迷魂湖之水,就算有,也不敢乱喝。他用的是老方法,也是见效明显的法子——睡觉。

%7
“我若是正常人身,只需要小憩休眠片刻,就能抚平魂魄的浮躁。但成了僵尸之后,就得连睡三天。魂道分壮魂、炼魂、安魂三面,但大多数魂道蛊师,都侧重前两个方面。现在想来,应该是睡眠本身的安魂效果就很上佳。不需要在这方面,过多投入。”

%8
经过这次封印春秋蝉,方源对魂道的认知又加深一层。接下来,他算了算近月来的账目。

%9
和琅琊地灵交易了两次,总共进账六十块仙元石。但他也用了不少。

%10
宝黄天中收购上千只星萤蛊,买下杀招春星雨、寒纱、六九星尘,买下大量炼蛊材料、大量凡蛊,买下一群老毛民。之后为了封印春秋蝉,买下小泽蛊、松岛蛊。总共花费二十块仙元石。

%11
现在方源手中,还余下三十块半的仙元石。

%12
但别看仙元石这么多,方源付出的代价还有自身的青提仙元。

%13
推算仙蛊方这事情。他只能亲力亲为,第一次和琅琊地灵交易之后,他的青提仙元只剩下九颗。第二次推算青提仙元,损耗了六颗青提仙元,优化春秋蝉的封印方案也耗去一颗,如今只剩下两颗青提仙元。

%14
优化春秋蝉的方案是很有必要的。之前估算的费用,是二十三块仙元石。优化之后。降至二十块。总的来说,还节省下两块仙元石。

%15
只剩下两颗青提仙元,这是很危险的事情。一旦遭遇不测。方源只能催动两次杀招万我,难以应对剧烈变化的形势,尤其是仙鹤门的进攻已经日趋逼近。方源便用掉十五块仙元石,转化为自家的青提仙元。

%16
如此一来。他的青提仙元上升到十七颗。而仙元石则降至十五块半的样子。

%17
现在春秋蝉撑破空窍的危机。已经暂时解除了。困局也因为和黑楼兰的联盟而打开。和琅琊地灵的交易,赚取到大量的仙元石,更是方源解决问题的力量之源。

%18
原本糟糕透顶的局面,已经舒缓了一些。宛若天昏地暗,层层乌云,但经过方源的不断努力后,渐渐打开缝隙,露出道道明亮温暖的阳光。

%19
现在摆在他面前的。是另一个难题,而且这个难题必须尽快解决。

%20
那就是喂养仙蛊。

%21
凡蛊的喂养。方源并不担忧。但仙蛊喂养,却是个日趋严重的大问题。

%22
总结下来,方源手中的仙蛊数量众多,有春秋蝉、定仙游、招灾、乐山乐水、浪迹天涯、平步青云、净魂、连运、妇人心。

%23
寻常的蛊仙,往往连一只仙蛊都没有。方源手中的仙蛊,多达九只。

%24
这是巨大的幸福,同时也是一个巨大的负担。

%25
春秋蝉还好,它以光阴长河的河水为食,也可以看做不用喂养。

%26
定仙游方源曾在中洲时间一年多前,喂过一颗青提仙元,可以支撑六年光阴。但中洲过来一年,并不代表定仙游也只过了一年。王庭福地的时间、方源仙窍的时间、狐仙福地的时间,都比中洲时间要快许多倍。

%27
而且六年这个数字,也只是理论数字。方源催动定仙游次数越多,这个时间就越会缩水。

%28
现在的定仙游,虽然还能支撑一段时间,但距离再次喂养也快了。

%29
剩下的仙蛊,都是麻烦。

%30
这些天方源查探到了关键信息。

%31
招灾仙蛊的食料,是荒兽六头大蛇的黑色血液。

%32
乐山乐水仙蛊,需要汲取大量的山气、水气,以及许多的笑石。

%33
浪迹天涯蛊,需要数万头的幽冥水母,以及数千头的深海闪电鳗鱼。

%34
平步青云蛊,则需要荒兽云龙的龙鳞,以及数万头风镰鸟的眼珠。

%35
净魂仙蛊的喂养,须得上万头白莲巨蚕蛊的血肉。

%36
连运仙蛊,则需要上古荒兽天地沙鸥栖息地的沙土万斤。

%37
至于毒道仙蛊妇人心,琅琊地灵也说过,要用妇人的心脏去喂养。此蛊养炼合一,喂养的妇人心脏越多,它的威能就越强。

%38
“这些蛊虫中,招灾仙蛊炼出来不久,早就应该喂养了,现在已经明显的虚弱。乐山乐水、浪迹天涯仙蛊,是仙蛊屋近水楼台的组成部分,这些年潜藏在王庭福地,也是饱一顿饿一顿。净魂、连运、妇人心、平步青云,都是在真传秘境中休眠。过去的这么多年里,巨阳意志只用一半的资源喂养。如今它们都脱离了真传秘境,彻底苏醒过来,现下都是饥饿状态,更需要进补。”

%39
“幸好我将平步青云,抵押给琅琊地灵。借他之手,来替我喂养,不怕被饿死。江山如故、人如故两只仙蛊,我也抛给了太白云生,让他去烦神。”方源心中思量着。

%40
饶是如此,压在他肩头的喂养重担,也绝不轻松。

%41
喂养仙蛊,不仅仅是仙元石的问题。

%42
这个世界不像地球,地球上是商业化社会,基本上只要有足够的钱,什么东西都能买到。

%43
但是蛊师世界,经济并不发达,很多垄断生意,无数的稀缺资源。仙元石的确是硬通货,但也不是万能的。

%44
就比方涉及到食料的荒兽六头大蛇、云龙,以及上古荒兽天地沙鸥,都是难于遇到,实力强大的蛊仙级战力。

%45
宝黄天中虽然也会贩卖荒兽身上的血肉、骨骼等等,但次数少,一放入市场就会迅速卖光。

%46
就算是地气、水气,也是提取不易。

%47
白莲巨蚕蛊、笑石等等,都是十分稀罕之物,宝黄天中也难得一见。

%48
反倒是妇人心的喂养,虽然麻烦一点,难度却低。

%49
关键的还有一点,那就是时间。

%50
方源没有充分的时间,去慢慢筹备这些食料。

%51
因此,他不得不在宝黄天中,公然收购这些稀有资源。尽管这会代价高昂,同时还会给智道蛊仙留下大量的推算线索。

%52
但方源现在顶着八臂仙人的名头,短时间之内,还不会有人联想到狐仙福地上去。

%53
就算被联系起来,方源也要这样做。

%54
毕竟仙蛊的存活,比被人怀疑要重要得多!

%55
当然,他早就询问过琅琊地灵、黑楼兰以及黎山仙子,可惜从他们那里,都没有得到理想的结果。

%56
琅琊地灵曾经拥有过笑石,但已经在炼蛊时消耗掉了。

%57
数年前,有蛊仙从东海归来,送过黎山仙子一群深海闪电鳗鱼。但也只有一百多条,只是用来观赏。后来黎山仙子疏于照料,导致这些鳗鱼死光。

%58
不过这个消息,给方源提了一个醒。

%59
东海……

%60
方源前世,在凡人时期就先后辗转五大域,流浪天下,曾在东海待过一段时间。

%61
东海资源之丰富,堪称五域第一,就算是中洲,也难以企及。

%62
而且这些年,东海动荡不安,可以从中渔利的机会也多,预期收益也大。

%63
方源若非荡魂山受损,太白云生最易得手的话,他的第一选择就是东海。

%64
又想到印象深刻的蓝墟福地、苍蓝龙鲸、彩虹岛等等,方源做下一个决定。

%65
他召来太白云生:“老白,最近修行近况如何?”

%66
太白云生微微一笑,絮絮叨叨地道:“好啊,进展得都很顺利。我之前给琅琊地灵修复云土地貌,赚了三块仙元石。又复原几条长江大河,又赚两块仙元石。之后买下防御杀招九云环,耗去两块。又在宝黄天中,收购了浮球茶草籽,花掉了一块。”

%67
“浮球茶草籽十分普通,你竟然花掉了一块仙元石,看来你购买的量极大。不过你的选择很正确,你的福地我也视察过,宙道资源远比宇道更多。浮球茶草生长缓慢,漂浮于空。正好让你扬长避短。过个几年,浮球茶草茂盛起来,你就可以搜割、贩卖。”方源评价道。

%68
“正是如此啊。”太白云生一拍大腿,“浮球茶草一定规模之后,我便打算迁徙一群玉蜂鸟,以此基础,建设福地。”

%69
方源看向太白云生,道:“玉蜂鸟是异兽,玉蜂鸟群的价格,可不便宜啊。”

%70
太白云生从容对答:“我已经想好了,今后我就用江山如故赚取仙元石,学习黎山仙子。我相信我的江山如故,不会属于黎山仙子的山盟蛊的。”

%71
方源点点头:“你拥有江山如故蛊,走到哪里,都会大受欢迎。”

%72
太白云生却叹了一口气:“可惜我现在不方便露面,王庭福地破灭了,我们是仅有的幸存者。现在已经被北原蛊仙,当成首要的怀疑对象。若是暴露的话,后果不堪设想。还有,作为关键目击者的马鸿运已经失踪,似乎是被某个蛊仙秘密俘虏,也许我们现在已经被某些蛊仙盯上了。”

%73
“所以我觉得,东海会是你的好去处。”直至此刻,方源才说出他的目的。

\end{this_body}

