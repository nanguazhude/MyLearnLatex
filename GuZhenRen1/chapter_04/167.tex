\newsection{毛民白蚕两难题}    %第一百六十七节:毛民白蚕两难题

\begin{this_body}

ps:想听到更多你们的声音,想收到更多你们的建议,现在就搜索微信公众号“qdread”并加关注,给《蛊真人》更多支持!

蛊是天地真精,凡蛊蕴含一丝道理法则,仙蛊则是大道碎片。

炼蛊,就是涉及到大道法则的交融、碰撞,比化学变化还要玄妙,比处理火药更要危险。

方源立在高空,观察片刻,便有十几位毛民炼蛊失败,受到伤害,其中几位当场直接死亡。

他不禁在心中一叹:“随着摊子铺大,毛民的损失问题,也日见凸显了。”

毛民奴隶,是所有奴隶中价格最贵的,甚至比人族蛊师还贵!

这有些不可思议,怎么人的价格,比毛民还要低贱?

其实很好理解人是万物之灵,人心多变,不好控制。异人却相对思维简单,方源能控制这么多毛民,夜以继日地炼蛊,每天只给他们有限的时间,除了吃饭睡觉,根本没有任何的娱乐生活。

若将这些毛民,换做人类蛊师,那几乎是不可能的。

人类蛊师长期从事这种高危密集劳动,没有休息没有娱乐,一定怨气沸天,心生反抗之意。

石巢中简单的布置,更抵挡不住人类的智慧。一定会被他们暗中破坏,相互之间迅速勾连。等到他们自以为时机成熟之后,就会反抗游兴,或者直接武力造反。

所以方源宁愿选择购买价格最高的毛民奴隶。也不去打人类蛊师奴隶的主意。

所以黎山仙子千方百计也要得到方寸山。不仅是因为方寸山上有稀有的炼蛊材料,而且还是小人族的大本营。控制了小人一族,能受益巨大。要不然,老奸巨猾的东方长凡也不会和小人族缔结盟约了。

“以前还不觉得毛民损耗惨重,那是因为规模小。现在有三座石巢,炼蛊的工作量又一直艰巨,因而就日益明显了。长久下去,不做改变的话,也不是个事儿。”方源心中急促思考。

这时,黑楼兰也建议道:“方源,你这毛民的损失有点大啊。一个月下来。恐怕得有上百位毛民丧命。仔细一算。叫人心疼。如果能减少这个数字,成本无形就降低了,胆识蛊贸易将会赚得更多。”

若她没有被方源拉进来,参与胆识蛊的买卖,她恐怕都不会说。

但现在胆识蛊贸易,也关乎黑楼兰的利益,所以她一发现问题。就提了出来。

方源点点头:“要解决这个难题,得从两个方面出发。第一就是改良蛊方,减少风险,更加细化步骤,让每一步都更加安全,成功率更高。第二个方面,就是自己豢养毛民部族,不需要去花费高价购买。”

黑楼兰摇摇头:“第一个方法太难了,蛊方本就千锤百炼,要在这样的基础上。改良蛊方,细分步骤,难难难!花费的精力,投入的实验资源,恐怕就不是一个小数目。”

她对改良蛊方,一点都不看好。

殊不知方源手头上有智慧蛊,虽然不能直接催用。但借助智慧光晕,却能灵感无限。对于方源而言,改良蛊方的难度就在于,他脑海中是否有足够多的念头可以消耗。

这点,方源自然不会说,不动声色地道:“第二个方法也绝不容易。我手中可没有任何的豢养毛民的经营之法。外面市场中所卖的,亦都是男性毛民,很少有女性和幼儿,就是为了防备其他蛊仙豢养。”

豢养毛民异人,可比培育猛兽,栽种植株难得多。

国难治,在于民多智。这句话虽然偏颇,但也有一定道理的。

智慧越多,就越难控制。毛民们会流窜,会钻漏洞,会集众闹事。

方源心知肚明:以他的底蕴能力,要豢养毛民,一定会亏得血本无归。这点前世记忆不带任何帮助。

他隐约记得,上上一世地球上的生活。他养金鱼,不勤换水,不供给氧气,又多喂食,三天两头就死一条,半个月后金鱼就全死光了。

豢养毛民,当然比养金鱼要难得多。毛民的习性,生活的环境等等需要注意的方面,更是繁多芜杂。

方源也养过石人,狐仙福地中石人部族一度还很壮大。

但那是因为有胆识蛊,可以令石人魂魄暴涨,迅速分裂,繁衍出下一代。

所以这是纯粹的作弊手段,真要让方源来养石人。

呵呵。

两三年之后,不管多少石人,估计都要死绝了!

当然,豢养毛民的心得经验,可以从其他蛊仙手中购买。当初方源培育星屑草,就是收购了一些栽种的粗浅心得。

不过这两者的价值,简直是判若云泥。要收购豢养毛民的心得经验,付出的代价将极为沉重。

就像是地球上收购吞并百强企业,付出的资金该有多大?方源不吃不喝不消耗,凭他的赚钱速度,也得积累好多年。

“我现在的库藏中,仙元石只有二三十块了。虽然得到龙鱼、长恨蛛、幽火龙蟒,又从东方长凡魂魄那里搜刮出了培养方法,但也要投资建设。就算建设了,也不能直接卖到宝黄天中去,还得转销他处。毕竟我还得多避风头呢。”方源心中苦笑。

一句话,前景是光明的,处境是尴尬的。

方源甩开脑海中的杂绪,对黑楼兰道:“好了,我已经关照了小狐仙,做了安排,就不送你下去了。接下来这段时间,你就吃住在石巢。我已经下了命令,如今三个石巢都炼制气囊蛊,你多多协助。等到地灾来临,需要你出手。我再唤你。”

黑楼兰一听,顿时感到有些不妙。听方源这语气,完全是要把她当做苦力使唤啊。

方源见她脸色有些难看,不做任何表示,只当做没看见。

不压榨出黑楼兰的最大价值。怎么会是自己的风格?

安排好了黑楼兰,方源便回到荡魂行宫。

打开暗门,来到密室,他再次看到东方长凡的魂魄。

魂魄萎靡至极,被牢牢禁锢于此,此时见到方源,也一动不动。

方源有意削弱。所以东方长凡的魂魄十分虚弱。方源不敢大意。对这老奸巨猾的东西,心中充满了戒备。

不过此刻要再次搜魂,却必须得先补给一下的。

当即方源捏碎几只气囊蛊,对东方长凡魂魄使用了胆识蛊。

魂魄迅速恢复,眨眼间就恢复了大半。

方源伸出怪爪,按住魂魄,进行搜魂。东方长凡一点都不反抗了。反正方源已经彻底搜过一遍,他对于方源而言,根本就没有任何的秘密了。

他仿佛彻底认命服输,灰心若死。

一介传奇人物,落到如此境地,也令人唏嘘。

方源却仍旧保持谨慎,搜魂结束后,他有查看密室中的蛊虫布置,防止意外的疏漏,不给东方长凡魂魄留下一丁点的希望。

这些动作。东方长凡的魂魄都看在眼里,心里简直是冰凉彻骨。

他想自己怎么落到这样的一个人手中?上天无路入地无门,绝境是路,毫无生机。

心中绝望的同时,反而还诡异地产生了一点欣赏之情。

在方源的身上,东方长凡仿佛看到了另一个自己!

方源离开密室,关好暗门。在门前又细心检查一番,确定没有差错,这才返回大殿。

他盘坐在床榻之上,脑海中星念翻腾,陷入沉思。

东方长凡已经再无秘密,其实基本上已经榨干。但方源还留着他,主要是为了图谋方寸山埋下一个伏笔。

目前,方寸山虽然是黎山仙子所有,碍于雪山盟约,方源根本无法抢夺。但盟约也不是不可以想办法,比方说黎山仙子身上的宙道杀招,可以拖延违誓伤害的来临。

再说,雪山盟约也有时限。

东方长凡和方寸山建立过盟约。从东方长凡那里,方源得知,这个盟约也是利用了山盟蛊,应在方寸山上。

盟约双方,并非只是东方长凡和小人蛊仙两者,而是整个东方部族和小人一族。

小人一族,黎山仙子肯定是要豢养,大加利用的。如此一来,东方长凡的魂魄,还有剩余价值,是一件利器。

只是这个利器,暂时不能利用,也许将来也根本没有利用的机会。但方源暗藏着,算是随意落下一子,有备无患。

他现在思考的,是白莲巨蚕蛊。

这一次他搜魂东方长凡,也是为了白莲巨蚕蛊。

净魂仙蛊一次喂养,需要数万头的白莲巨蚕蛊的血肉,但方源得到手中的,却根本达不到这个数量。

虽然之前也查探过,这次方源再次确认东方长凡得到白莲巨蚕蛊,只是一场意外收获。那处地方,是个天然洞窟,自然伟力巧妙构造,意外形成了适合白莲巨蚕蛊的生存环境。

并且,东方长凡将这些蛊虫都捕捉,没有任何遗漏。其中还用了一大部分,在虚化大阵中,起着关键作用。

用剩下的那部分白莲巨蚕蛊,现在则辗转到方源的手中。

现在败在方源面前的,有两个选择。

一个是借助手中有限的白莲巨蚕蛊,针对研究,进行试验,逆推蛊方,甚至逆推出白莲巨蚕蛊的培育方法。

此法若成,将来白莲巨蚕蛊源源不断,一劳永逸。但若失败,这些百莲巨蚕蛊都被试验研究消耗掉了,本来喂养数量就不够,剩下来的恐怕塞牙缝都不行,净魂仙蛊也就饿死了。

另一个选择,就是直接喂养。别管什么研究试验,长远打算了。但就算如此,白莲巨蚕蛊的数量还是不够,最多让净魂仙蛊脱离饿死的边缘,处于半饥不饱的状态。

方源原本还抱着一丝虚无缥缈的幻想,希望从东方长凡收获白莲巨蚕蛊的地方,缴获一些新的。

但这次搜魂,事实打破了他的幻想。

方源思索片刻后,睁开双眼,终于下定了决心。(我的小说《蛊真人》将在官方微信平台上有更多新鲜内容哦,同时还有100\%抽奖大礼送给大家!现在就开启微信,点击右上方“+”号“添加朋友”,搜索公众号“qdread”并关注,速度抓紧啦!)(未完待续。)

\end{this_body}

