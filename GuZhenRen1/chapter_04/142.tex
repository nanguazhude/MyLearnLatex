\newsection{追命炎威无可挡}    %第一百四十二节:追命炎威无可挡

\begin{this_body}

仙道杀招追命火!

残阳老君满脸肃穆,手掌缓缓展开,露出一点微弱的火。

这火焰摇摇曳曳,呈现昏黄之色,微弱无光,仿佛要随时熄灭。

残阳老君的双眸中,却是绽射精锐神芒,望着手掌中的这点火,他的脸上涌现出一抹自傲于世的神色。

这是他赖以成名,威震中洲的绝技!

“追命火,去罢。”残阳老君手掌轻轻一抖,追命火焰嗖的一声,急射而出。

星意连忙开出一道口子,让出通道。

追命火毫不起眼,出了血幕之后,一头扎进寒冷的冰山上去。

轰!

几乎刹那之间,巨大的昏黄火焰,冲天而起,剧烈灼烧着整座冰山。

半空中的魔道蛊仙们,几乎都被这异变吓了一跳。

但很快,他们的脸上涌现出古怪的神色。

皆因火焰虽然凶猛,但毫无热度,尤其是火中的冰山安然无恙,一丝融化的迹象都没有。

有些魔道蛊仙正要开口嘲笑,这时,七转魔道的蛊仙强者皮水寒,却是突然发出惨叫。

众人循声望去,眼前的一幕叫人大吃一惊。

皮水寒浑身笼罩火焰,受到剧烈的灼烧,痛得满脸扭曲狰狞。他浑身大冒寒气,水流激喷,宛若龙蛇缠绕自身。昏黄炎火却丝毫不受削弱,反而有越加旺盛的迹象。

“这到底是什么招数?”

“皮水寒究竟是如何中招的,为什么我丝毫都没有看清?”

“好诡异的火焰。大家快闪!”

魔道蛊仙们纷纷撤退,心有余悸地看着皮水寒在拼命地施展手段,企图扑面火焰,救治自己。

“这种火焰,难道是……追命火?不妙!”方源心头震动,意识到不妥,连忙收回八只力道巨手。

但已经来不及了。

火焰灼烧冰山,旋即便闪电般地弥漫上八只力道巨手。

力道巨手安然无恙,但方源仙窍中,却凭空生出昏黄火焰。在十六万的力道虚影身上静静燃烧!

这些力道虚影。正高举着手臂,正是他们提供了力量,才形成了外界的八只力道巨手。

他们沐浴在火焰中,却是安然无恙。

方源见此。心中却充斥一股寒意。

“这果然是那杀招追命火!”方源心知此招的厉害。连忙壮士断腕。念头一动,十六万力道虚影自行崩解。

没有了这些力道虚影,正在拔山的八只力道巨手也随之轰然溃散。

轰的一声巨响。墟蝠尸山重重落下,砸在地面上,掀起一波巨大的震荡,周围的地面大面积崩裂,一时间无数沙石溅射,烟尘滚滚。

方源拔山被打断,受到强烈的反噬!他胸口烦闷欲吐,头晕目眩,身躯摇晃,差点一头要从半空中栽倒下去。

黑楼兰见机不妙,连忙要上去扶他。

方源却如避蛇蝎,叫喊道:“别靠近我!”

说着,猛地后退,身上也如皮水寒一般,燃烧起昏黄之火。

与此同时,自在书生脸色骤变,他也中了此招,身上同样燃起熊熊火焰。

“这火道仙级杀招,居然能够气息感应,顺着我们发出的攻势,追根朔源,直达仙窍内部!”自在书生心中剧烈震荡。

这一刻,自在书生终于明白,为什么皮水寒会如此手忙脚乱。

皆因,他不仅是身体受到火焰的烧烤,而且仙窍中也弥漫大火。前者也就罢了,后者乃是蛊仙的大本营,根基所在,经过多少年的苦心经营,无数的修行资源存积在里面呢。

如今被这大火一烧,直接烧在中招者的心头,烧得他们心疼得要吐血。

皮水寒不仅要为自己身体灭火,更要在仙窍中施为,四处救火。但诡异的是,这火焰却是冰水难灭,非同寻常。

周围的魔道蛊仙们,纷纷和方源、自在书生、皮水寒三人拉开距离,唯恐沾染上这古怪诡异的火焰。

“怎么方源身上的火焰,比其他两人要弱小很多?”黑楼兰目光惊疑。

其余的魔道蛊仙们,旋即发现这种现象。顿时,方源在众人心中的形象,变得更加高深莫测了一些。

自在书生的目光,也在方源的身上顿了顿。

方源的情况,要比其他两人好得多。他身上的火焰,只是微弱的一层,仿佛油灯的火焰。而皮水寒、自在书生两人,却是灼灼燃烧,宛如大型的火炬。

短短几个呼吸,自在书生的身体已经被大面积烤焦,传出一股肉糊味道。

自在书生强自镇定,心中并无多少慌乱,皆因他手中还有一张底牌!

仙道杀招千解!

他狂催杀招,全白的眼眸中目光更盛。

但这一次却不是照射冰山血幕,而是将目光落在自己的身上,射进仙窍当中。

昏黄的火焰,在淡白的目光下,不断化解,迅速衰弱。

很快,他身体上的火焰彻底消失,仙窍中的火势也得到了控制。

千解的确好用,不仅可以对敌,还可以用来防守。攻防一体,威能又很脱俗。

须臾,方源也站了起来,身上的火焰消失无踪。

惟独皮水寒,尝试多种手段,火焰反而更盛。他浑身焦黑,浓郁的肉香味不断飘出,只得一边动用治疗蛊,治愈伤势,一边承受着火焰的炙烤。

他怒吼连连,痛在身上,更疼在心里。

火焰在他仙窍中弥漫,大量的资源被焚烧一空,偏偏却拿这股火势没有办法。

“怎么办?怎么办!”皮水寒急速思考,焦躁万分。

方源、自在书生不发一言,冷漠地看着。其余魔道蛊仙,位置更远,一些人的目光闪烁不定,蠢蠢欲动。

若是皮水寒因此重伤,虚弱无比,在场的这些魔道蛊仙们绝对会砰然心动,打上皮水寒的主意,落井下石。

忽然间,皮水寒大吼一声,身形猛地拔高,带着全身的火焰,仿佛一道火焰流星,向外激射而去。

这身火焰成了他巨大的麻烦,此地魔道蛊仙众多,不可久留。皮水寒撤退,实乃明智之举!

看着他迅速消失在天边,边缘的几位魔道蛊仙,也跟着鬼鬼祟祟地离开了这里,循着皮水寒离开的轨迹,悄悄消失。

宫殿中,残阳老君哈哈大笑,对着东方长凡的星意道:“如何?”

“果然不愧是追命火。”星意点点头,“此火有追根溯源之能,蛊仙常常将蛊虫放入仙窍之中。因此追根溯源,就会燃烧到他们的仙窍当中。此火又以生命为燃料,只要生命不息,就燃烧不止。因此要扑灭此火,非得动用特别手段,譬如千解之术。如若没有此等手段,那便要壮士断腕,主动割弃,或者任其燃烧,尽量收敛体内生机。只要生机耗尽,此焰便解。可惜这皮水寒未想破此点,不断治疗自己,反而引得生机旺盛,更加剧了火焰燃烧。”

星意说到这里,轻笑一声,又赞道:“残阳老君,你创下的这招,真是思维独到,另辟蹊径。此火只要烧中,便能令敌人方寸大乱。多烧几把,再富有的蛊仙也会底蕴大损,毫无斗志。”

听着星意侃侃而谈,对自己的手段赞赏有加,残阳老君的脸上,得意之色反而尽消,浮现出一抹阴沉之色。

残阳老君之所以号称“残阳”,追命火就是主因。

此火一烧,即便化解,也会令敌人灰头土脸。偏偏又能追根溯源,顺着攻势,烧到仙窍中去。

仙窍乃是蛊仙根本,几次一烧,大损根基。烧不死敌人,也能烧残了去。

因此残阳老君早年纵横中洲,威名赫赫。和他对战的蛊仙,看到他都头疼。

成就七转之后,残阳老君就走动得少了,论战力可算是仙鹤门前五!此番八十八角真阳楼倒塌,事关重大,仙鹤门才遣出他来进入北原,探查真凶。

追命火一出,果然烧得方源、自在书生束手,更烧得皮水寒狼狈而逃。残阳老君明意退敌,暗意也是在震慑东方长凡。

结果,东方长凡却打探得清楚,对追命火知之甚详,名为赞赏,实则是对残阳老君的反击。

因此,残阳老君能有好脸色,那就怪了。

“追命火……残阳老君……他怎么会在这里面?”方源悬浮在半空中,看着脚下血幕,心中疑惑。

他虽有前世记忆,但和全知全能还相差极多,东方长凡夺舍一事十分隐秘,方源并不知晓。

现在看到中洲蛊仙,和东方长凡的传承绞在一起,自然感到意外和困惑。

之前残阳老君虽然击退了他,但只是惊鸿一现,藏头露尾,根本没有暴露出身份来。

残阳老君身上,有着遮掩气息的秘法,动起手来,活脱脱的北原蛊仙的气息。魔道蛊仙们看在眼里,没有一位发现端倪,也算是残阳老君手段高妙。

“不妙。”方源心中暗叫糟糕,不禁生出退意。

他知道中洲势力秘密潜入北原,探查自己这个真凶,也知道有人找上北原僵盟,对自己沙黄身份有了怀疑。

“我只所以能解决追命火,一是瞧出了此招的来历,迅速决断。二是身为仙僵,乃是活死人,仙窍也是死地,生机渺茫,因此烧不起来。残阳老君经验丰富,恐怕此时已经看破我是仙僵之体了。我又如此隐藏行迹,恐怕更遭他的怀疑!”方源越是思考,心头越加沉重。

\end{this_body}

