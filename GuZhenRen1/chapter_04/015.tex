\newsection{收获颇丰}    %第十五节:收获颇丰

\begin{this_body}

顺利地和琅琊地灵达成交易,方源着实松了一口气。顶点小说23us

这场交易,来得实在太关键了,得来的利润能够帮助方源破开眼前的困局。

虽然方源的前世记忆中,也有不少获取仙元石的途径。但这些途径,基本都需要时间谋划,或者风险颇大。

和琅琊地灵交易,是唯一的风险不高,收益很高的选择。

思考了片刻后,方源将云道仙蛊平步青云,留在了琅琊福地,当做交易的抵押。

平步青云是移动蛊,方源不缺移动仙蛊,他早已有了浪迹天涯。后者适合水道蛊仙,在江湖河海的环境下催动效用更佳。而平步青云,则擅长升降,论正常的平行疾驰,其速度要逊于浪迹天涯一筹。

方源是个实用主义者,对他而言,浪迹天涯蛊无疑更加实用。

北原当晚,方源独自一人回到了狐仙福地,而太白云生则留在了琅琊福地。

对于此次出行,方源心底十分满意。

虽是消耗了一颗青提仙元,但那种情形下,不知道琅琊福地的真正状态,当以迅速击败泥沼蟹,掌控局面为先。

仙元的确珍贵,但该用的时候必须得用。不使用的仙元,什么价值都没有!

别说是一颗青提仙元,就算是八颗、十颗,该用的时候方源毫不犹豫。

他总结了一次,此次付出很少,收获却颇丰。

首先。定仙游拿回来了。

定仙游失而复得,的确是个大大的惊喜。有了定仙游之后,方源就能横跨界壁。足迹遍及五域。充分利用重生优势,四处布局,获取最大好处。

就算再不济,仙鹤门攻破了狐仙福地,方源和太白云生还能带着荡魂山、智慧蛊撤退。

进可攻,退可守。

定仙游虽是六转移动仙蛊,但却带给方源巨大的战略优势。

这是平步青云、浪迹天涯相加起来。也办不到的事情。

方源由此压力骤减。

其次,和黑楼兰、黎山仙子达成了雪山盟约。

黑楼兰乃是大力真武体,十绝之一。潜力庞大,一旦晋升成仙,战力非同小可。

黎山仙子则是北原风云人物,一方面她是大雪山福地第三首脑。另一方面她又交游广泛。和北原正道蛊仙多有往来。

正道蛊仙有败类、奇葩,魔道中更是形形色色。一切纷争都起源于利益,正魔两道并非总是打打杀杀。相互之间也有交易,甚至秘密的合作。

黎山仙子拥有山盟仙蛊,此蛊就是她广受欢迎的关键原因。

东方长凡和各大正道势力商讨建盟,需要山盟蛊。魔道蛊仙相互之间更加猜忌怀疑,能建立起大雪山福地这样的庞大组织,就更需要山盟蛊。

由山盟蛊建立起不可违背的守望盟约。是魔道蛊仙们相互信任的主要基石。

方源和她们俩结盟,一方面能得强援。另一方面也解决了实时情报不足的最大弊端。

黎山仙子交游如此广阔,要打探情报,比方源可要容易多了。

再其次,方源获知了四只仙蛊的具体情报。

他打破无相拳,抢得而来的,分别是净魂、平步青云、妇人心以及连云。

价值从低而高排序的话,首先是平步青云。对方源而言,此蛊实用价值最小,被抵押给琅琊地灵,算是充分利用。

毒道仙蛊妇人心,是一次性的消耗蛊,炼养合一,实用价值稍高一些。通常而言,这种一次性的消耗蛊都是威力很大的,譬如仙蛊和稀泥。

妇人心之上,则是仙蛊净魂。净魂能够炼魂,效果要远超方源之前的狼魂蛊。方源有胆识蛊,壮魂极易。在没有进入落魄谷之前,仙蛊净魂是最好的替代品。

除此之外,净魂还是杀招万我的核心蛊。这个价值就更大了。

方源比较了一下,他不动用杀招万我,战力要弱于普通的六转蛊仙。但一旦用了这招,战力疯狂暴涨,要站到六转蛊仙中的一流层次。

当然,比方源五百年前世的血道战力,还要更弱一筹。

方源五百年前世,是魔道巨擘,六转巅峰,离七转只差一步之遥。战力更是恐怖,牢牢占据六转顶层,有两次对战普通七转蛊仙而胜的骇人战绩。

最后是连运仙蛊。

这只运道仙蛊,是最出乎方源意料的大收获!其价值甚至还要超越净魂一点。

皆因净魂虽是万我核心,但却并非不可替代。

而连运仙蛊的重要意义,对于方源而言,是唯一的!

方源的第一本命蛊,是春秋蝉。这只仙蛊铸就了他,没有这只奇蛊,方源走不到今天。但同时,春秋蝉也一直深深拖累着他。

不管是针对空窍的压力,还是春秋蝉的弊端——时刻不停地削减使用者的运气,都带给方源极大的麻烦,甚至许多次死亡的危机。

若非方源能力足够,心狠手辣,手段也多,换做寻常人,早就死了。

春秋蝉是方源的底牌,同时也是他的最大心病。

现在用连运仙蛊的话,恰好能解决春秋蝉的气运弊端。

琅琊地灵提供了好多种连运仙蛊的用法,最赞赏的方式是用其他运道仙蛊,搭配连运仙蛊,使蛊师的气运链接名山大川,或者太古荒兽,洞天福地。这些存在都气运悠长,不易损毁。

但此法方源不打算取。

因为他办不到,他只有一只运道仙蛊,没有其他与之搭配。

方源想到的方法是:用连运仙蛊将他的运气,和其他人的运气链接在一起。

琅琊地灵不久曾说:人的气运,宛若潮水,涨落不定。和人运相连,须得时刻关注,否则容易拖累使用者自己。若是配合断运蛊的话,一来消耗珍贵的仙元,二来劳心劳累。

但这个方法的弊端,对方源而言,却是不存在的。

为什么呢?

因为他是用了春秋蝉的重生之人!

他知道哪些时代的风云儿、弄潮儿、幸运儿是谁,甚至知道他们今后的成就如何,他们又终结在哪个时间段。

“我完全可以凭借自己的前世记忆,找到这些未来的大人物,用连运仙蛊将自己的运气和他们沟通在一起。反正我有春秋蝉,运气一直都是糟糕透顶,链接运气之后,占便宜的总会是我。”

方源想到这里,更觉得定仙游带来了巨大的战略优势。

若没有定仙游,他就不可能纵横五域,哪有时间去长途跋涉,再一一找到那些现在或许还不起眼的小人物呢。

此行的最后收获,则是方源从琅琊福地带回来的三张仙蛊残方。

这些仙蛊残方,完善度都很高,皆是九成以上,其中一张甚至达到了九成九的地步!

稍微休整了一下,方源便来到地底洞穴之中。

推演仙蛊方,需要考虑的因素太多,凡蛊产生的意志完全不够消耗,方源毫不犹豫,催动了乐山乐水蛊。

他决定先易后难,将九成九的残方首先搞定!

与此同时,琅琊福地。

“琅琊地灵,你看我修复得如何?”太白云生背负双手,站在琅琊地灵以及墨人王的面前。

墨人王看着恢复原貌的浩荡云土,眼中不禁流露出惊叹之光。

之前,太白云生主动留下来,提出要为琅琊地灵修复福地时,他并不抱有期望。没想到太白云生做得这么好,效果超乎他的想象!

琅琊地灵好生打量了太白云生一番,叫道:“想不到你年纪轻轻,也很不一般啊。你是怎么做到的?”

太白云生苦笑一声,他年岁苍苍,发须雪白,居然还被他人说年纪轻,这种古怪感觉并不常有。

他拿出江山如故仙蛊,展露给二人观看,并详细介绍了一番。

墨人王不禁大为赞叹:“不想有此奇妙仙蛊。有了这仙蛊,就能修复福地的地貌和河流,大大减少天劫地灾带来的损失!这仙蛊价值极高!”

琅琊地灵也生出兴趣:“这只仙蛊我听说过,曾经是九悔仙人之物。九悔仙人继承了幽魂魔尊的某处真传,手中不仅有江山如故,还有一只人如故。你这只仙蛊我还没亲眼见过!你给我研究一下,它一定能带给我全新的灵感,我给你五块仙元石作为报酬。”

琅琊地灵家大业大,是个十足的富翁,并不缺仙元石。

“五块仙元石……”太白云生犹豫了一下,旋即摇头拒绝。

仙蛊本来就很私密,不轻易外显。

但太白云生为了今后营生,向黎山仙子学习,这才不得不暴露江山如故仙蛊。怎么可能将其交给琅琊地灵?说是研究,万一研究坏了该怎么办?

见太白云生没有答应,琅琊地灵显得很失望,但他还是如约付账:“你为我修复了全部的云土,按照交易,这是三块仙元石。”

太白云生却摆手拒绝道:“这三块仙元石我不收。待会方源交货时,劳烦地灵你都一起给他罢。”

琅琊地灵楞了一下,旋即哈哈大笑:“怎么?你该不会相信那个臭小子,真的能推算成功,完善蛊方吗?哈哈哈,你知道这些蛊方困扰了我多少年吗?尤其是其中有一张九成九的蛊方,当年连本体都被难住!”

\end{this_body}

