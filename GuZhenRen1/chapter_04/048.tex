\newsection{仙鹤门,蛊仙会}    %第四十八节:仙鹤门,蛊仙会

\begin{this_body}

中洲,飞鹤山。

山巅最高处,议事堂。

蛊仙们环坐一圈,仙气洋溢,正商议着仙鹤门中大事。

“灵犀的病情加重了,我需要采购大量的材料,炼出缓解病情的治疗蛊。但要彻底根治灵犀,我还得要大量的治疗蛊方,进行研究。初步估计需要仙元石三百六十块。”蛊仙树止戈朗声道。

当即,就有蛊仙皱眉反问他:“灵犀是上古荒兽,本身就极其健壮。可它的病情已经持续了大半年,怎么到现在还没有康复?”

“树止戈,你不会是故意拖延灵犀病情不治,想趁机捞取好处吧?”一位蛊仙大大咧咧地道。

树止戈闻言,顿时眉头竖起,咆哮起来:“樊西流,放你娘的臭屁!这还不是你们捕获它的时候,动用了毒道仙蛊吗?当时我早就劝说,不要动用它,不用动用它。结果你们为了省事,毒倒了灵犀。现在有了麻烦,却还需要老夫我来收拾!”

“你才放屁!你知道活捉一头灵犀有多难吗?站着说话不腰疼!”樊西流也不甘示弱,大吼道。

铛。

这时,一声钟响,清脆悠扬。

蛊仙们发出的声音,统统被钟声同化,因此钟声回荡,越加高亢。

树止戈、樊西流两人怎么开口,发出的声音都只会变成更加响亮的钟声。他们索性闭嘴,众人的目光,也纷纷投向钟声来源的方向。

那是最中央的主位。坐着一位老者,便是仙鹤门权势最盛,总揽大局的八转太上大长老。

仙鹤门太上大长老的声音。低沉缓和:“现在,就树止戈的提议,需要三百六十块仙元石,来收购蛊虫、蛊方,治疗灵犀,进行决议。”

“我不同意。三百六十块仙元石太多了,我上一次的预算也不过只要求了两百块仙元石。难不成治疗一只灵犀。比治疗蛊仙还要重要?”樊西流首先反对。

他旁边的一位蛊仙沉吟道:“灵犀的确重要,但并非紧要关头。当务之急,还是西北局势。我放弃此次决议。”

“关于钧天剑派将会重点讨论。现在只讨论治疗灵犀一事。”太上大长老插了一句。

“我反对。”第三位蛊仙言简意赅。

“在发表个人建议之前,我想听听桑心夫人的建议。我派之所以耗费精力,抓捕了上古荒兽灵犀,就是为了利用灵犀血。研发出更隐秘的信道蛊虫。不知道桑心夫人目前的进展如何?”第四位蛊仙则看向桑心夫人。

桑心夫人乃六转蛊仙。闻言一笑:“惭愧!目前收效甚微,皆因灵犀身上全是毒血,导致信蛊研究进展极为缓慢。”

第四位蛊仙点滴啊头:“这么说来,这方面的前景黯淡无光了。那么,我反对。”

在场的八位蛊仙,有五人反对。最终树止戈只争取到两百块仙元石。

这显然和他要求的三百六十块,有着不小的差距。树止戈冷哼一声,轻声嘟囔了几句。表示不满。

铛。

又一声钟响,不管树止戈满意还是不满意。太上大长老道:“下面进行,轮回战场的相关议事。”

雷坦从座位上站起身来,他体格魁梧雄壮,一头蓝发冲天:“我负责轮回战场事务,已经有三十多年。轮回战场的重要性,相信诸位都有深刻的了解。在今年年中,我派在轮回战场上大败亏输。虽然及时大力地支援过去,稳住了脚步,但目前境况仍旧糟糕。我需要一位六转蛊仙战力的支援,或者三头荒兽,并且这些荒兽中至少要有一头九宫鹤。”

鹤风扬立即反驳:“一位蛊仙战力?你知道仙鹤门现况吗?穆萧萧重伤修养,林三昧闭关,战力处处吃紧。你居然还好意思开口提出这样的要求?”

雷坦冷哼一声:“若是仙鹤门被彻底赶出轮回战场,你鹤风扬能够负责吗?”

鹤风扬冷笑:“轮回战场已经投入大量的资源,陷进去两位蛊仙,完全可以防守,维持稳定。我方增兵,解决不了问题,因为我们在轮回战场的竞争者是其他九派。我们支援,其他九派也会增援,甚至幅度更大!”

树止戈颔首:“同意。轮回战场的局面,已趋于稳定,目前应当保守策略,稳固地盘为先。”

其余众人望了望周围,依次发表自己的决议。

最终,雷坦得到两头荒兽的支持,他含恨看向鹤风扬,咬牙切齿。

铛。

“下面进行钧天剑派的讨论。”太上大长老道。

圭坜闻言,不禁神情一振,坐直了身体。

迎着众人的目光,他开口道:“钧天剑派附庸我仙鹤门,已经近千年历史。受到我们的托庇,经营日久,如今拥有第三位蛊仙。他们想要摆脱附庸,脱离我派。一旦成功,我派对西北的掌控力就要大减。仙鹤门的声誉也会受到极大的损伤。虽然没有明确的证据,但我推算几次,怀疑幕后的推手中含有万龙坞的影子。若是钧天剑派脱离我们后投靠了万龙坞,那么事情将更加糟糕。我们和万龙坞的地域接壤,钧天剑派周围的版图将随之并入万龙坞的势力范围内。”

“我们都知道此事的严重性。不需要你在这个方面详说。我直接问你好了,你打算策反钧天剑派中的某位蛊仙,究竟是哪一位?把握有多大,会不会是对方的计策?最重要的一点是我们需要付出多大的代价,才能拉拢策反一位蛊仙。”太上二长老问道。

钧天剑派,是一个大型势力,难以承担三位蛊仙的修行资源。

中洲目前的局势,是十大古派牢牢把持整个中洲。中洲大约六成的修行资源,都被十大古派瓜分。唯有十大古派,才能供养大量的蛊仙。

像钧天剑派。本身就是仙鹤门的附庸,每隔一段时间都需要向上交纳大量的资源。钧天剑派中的蛊仙,缺乏资源。从这点上看,若是许以重利,的确存在被策反拉拢的可能。

钧天剑派三位蛊仙,若去掉一个人,就只剩下两位。势力大降,就掀不起风浪了。

圭坜斟酌了一下措辞,这才道:“有句老话。叫做便宜无好货。要知道我们拉拢的,不是凡人,而是一位蛊仙。我允诺他会有和我们一样的待遇,毕竟这也是加入仙鹤门的标准。除此之外。还有一些资源。大约价值七十五块仙元石。”

顿了一顿,圭坜继续道:“依我个人浅见,在钧天剑派一事上,能不动用武力,就是好的。毕竟钧天剑派作为我派附庸的期限,确实已经到了。他们想要解除这层关系,脱离我们,算得上名正言顺。我们强行干涉。师出无名。钧天剑派又在边境线上,或许会给万龙坞带来介入的借口。”

至始至终。圭坜都没有透露出,他想要招揽的蛊仙究竟是哪一位。

太上二长老,却也没有追问。

在座的诸位蛊仙,都是仙鹤门中的太上长老,但彼此之间也有内斗。

有人的地方就有利益纠纷,有利益纠纷就有江湖。

圭坜要现在说出来,恐怕会生出不必要的波折。

钧天剑派一事,是仙鹤门上下都极为重视的大事。最终商议之后,圭坜的提议得到了通过。

铛。

一声钟响,太上大长老徐徐道:“下面商议的,是回收狐仙福地一事。鹤风扬你说说看吧。”

雷坦冷哼一声,立即向鹤风扬投去不怀好意的目光。

鹤风扬直接从座位上站起来,表露出自己的郑重态度:“诸位同道,狐仙福地是六转福地,虽然之前狐仙经营不佳,但却有荡魂山。众所周知,荡魂山的胆识蛊,会大幅度地提升门派的底蕴,即便对我们蛊仙也多有帮助。”

他话还未说完,蛊仙雷坦就开口发难:“鹤风扬!如今已经过去了一年多,怎么狐仙福地还没有得手?甚至一点进展都没有。你知不知道,其他九大派早就看我们的笑话了。”

鹤风扬早有准备,不疾不徐地道:“但凡成大事者,总要承受周围人的嘲笑讽刺,如果连这些都承受不来,又如何能有所成就呢?”

“我过来议事,不是来听讲大道理的。不管怎么说,狐仙福地尚未回收,这是不争的事实。”雷坦道。

“这一年多来,本人一直着手此事,紧锣密鼓,一刻都不曾松懈。几日前,我已经公开宣布方源是本门叛逆,不日讨伐。这些天来,我广布眼线,监控其余九大古派,暂时还没有发现哪些门派中有特殊反应。这次行动的关键人物已经训练妥当,只需要残阳老祖、苍郁仙子出手帮衬,便能夺回狐仙福地。”鹤风扬道。

“残阳老祖乃是七转战力,苍郁仙子在六转中亦是不俗,你好意思提出这个要求?你也是堂堂的蛊仙,对付一个凡人,却要纠结三位蛊仙战力。你杀鸡用牛刀,这就是你准备了一年多的成果吗?”雷坦大肆嘲笑道。

鹤风扬眼角抽搐了几下,强自压下心中火气,环顾四周,依旧冷静地道:“确保万无一失,我不得不这样要求。在座的诸位,想必都清楚情势。我派利用方正,成功蒙骗了九大古派,在名义上将狐仙福地占为己有。但自从九大派查明真相之后,早已经觊觎荡魂山,暗中多次阻挠。这一次行动,必须成功,不能失败,否则他们就要介入。到那时,情况就不是我派一方能够掌控得了的。”

听到这里,太上大长老轻叹一声:“他们已经阻挠了。鹤风扬,老夫这里有一个坏消息,还有一个好消息。”

鹤风扬心中一凛:“大长老,请讲。”

“罢了,你还是自己看吧。”大长老轻弹手指,飞出一只五转传信青鸟蛊。

鹤风扬接过手中,投入神念,看了开头,顿时脸色铁青:“什么人?居然点名要求残阳老祖大人前往北原去?”

在鹤风扬的计划中,残阳老祖是最高战力,他拥有攻伐仙蛊,是压阵的重量级人物。

但鹤风扬此行还未发动,就丧失了这员大将!

太上二长老道:“没有办法回绝。真阳楼无故倒塌,使得我们十派联合,颠覆真阳楼的大计提前夭折,多少年的努力都化为乌有,无数的投入都打了水漂。这事情太过于重大,必须组织一支队伍,前往北原查明真相。同时,北原地区各种仙蛊纷纷出现,因此北原之行,还有一个重要目的,就是尽量捕捉这些仙蛊。若派去的蛊仙战力低微,不仅什么好处都捞不到,甚至还会被排挤,沦为挡箭牌,处处吃亏。”

ps:明天是我爷爷逝世三周年,客人较多,需要招待。今晚刚刚回来,明天还要回老家,有许多风俗需要一一操办。所以今明两天都是一更,更新较迟。特此通知,造成的不便,请广大读者朋友们多多包涵。

\end{this_body}

