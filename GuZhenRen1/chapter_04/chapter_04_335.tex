\newsection{离奇梦神秘歌}    %第三百三十六节:离奇梦神秘歌

\begin{this_body}



%1
这里是第三层梦境。

%2
夜色温柔,天空中繁星稀稀点点。

%3
湖水一片平静,方源刚入梦境,就已经置身在湖心的亭中。

%4
他的目光,紧紧盯着亭中的女仙人,心中十分惊异。

%5
星宿仙尊!

%6
古往今来,十大尊者当中,只有红莲魔尊神秘莫测,容颜不可考证。其余诸人,却是皆有音貌流传。

%7
星宿仙尊乃是历史上的第二位尊者,是元始仙尊的徒弟,更是天庭的第二代仙王。

%8
她的丰功伟绩,她的音容相貌,自然流传甚广,所以方源一眼就认了出来。

%9
星宿仙尊身材高挑,穿着一身蓝色长裙,裙摆逶迤拖地。

%10
她有一头乌黑如瀑的长发,一直垂到腰际。

%11
她的睫毛浓密且长,一双眼眸宛若井中之月,平静中透着含蓄的灵性和神秘。

%12
她冰肌玉骨,肤白似雪,此时幽幽地瞧着方源,一双柔荑轻拂身前的古琴。

%13
“你终于来了。”她缓缓开口,声音轻柔,宛如这美妙的夜色。

%14
“有趣。在这梦境当中,我是扮演的何人?”方源看看自己的穿着打扮,却发现自己竟然是本来面貌!

%15
这让方源颇为惊讶。

%16
按照常理,不该如此。

%17
之前的第一层梦境,方源在这梦中是一位落难的孩童。第二层梦境,是寻仙问道的凡人少年。怎么到了这第三层梦境当中,自己反而回归本来的面目了?

%18
“我进入梦境之前,一直催动着仙道杀招见面曾相识。如今连这个杀招都失效了,莫非这个梦境之深,已经让我不知不觉间魂魄沉迷了吗?”

%19
方源心中念想至此,不由地加倍警惕。

%20
他刚刚探索前二层梦境时。心里十分清楚,这都是梦,不是真的。

%21
但若是魂魄彻底沉迷。就会丧失这个认知,觉得一切都是真的。就好像当初。黑楼兰渡劫陷入梦境之中,随着梦境不断循环,而她自己却始终无法自拔。

%22
“我在这梦中,发现自己本来面貌,这就是魂魄沉迷的一种征兆。看来这第三层梦境非同小可!”方源想到这里,更不敢大意,全神戒备。

%23
但面前的星宿仙尊,似乎看出了方源的心里想法。浅笑道:“你无须害怕,且听我一曲道来。”

%24
话音刚落,她嫩如青葱的十指频动。

%25
琴弦在她的手法下,发出悠扬的美妙之音,悠缓如泉,娓娓动听。

%26
随后,她亲启檀口,以古韵古法悠然而歌。

%27
方源只听她唱道:

%28
“歌声寥落,英雄落魄,难挡命途多舛。”

%29
“折剑沉沙。千古兴亡,不尽天河滚荡。”

%30
“忧愁……”

%31
“幽夜漫漫魂梦长,问何处安乡?”

%32
“物换心移几春秋。唯天意苍茫。”

%33
星宿仙尊在夜色中长歌,歌声如清澈的山泉,流淌到方源内心最深处。

%34
方源大皱眉头,乍然听闻,只觉得此歌似乎是一种预言,大有深意,意有所指,短时间内还不可捉摸。

%35
星宿仙尊轻歌完毕,身影徐徐消散。她的嘴角始终挂着那一丝神秘的微笑。

%36
方源全神贯注。提起十二分精神,但下一刻梦境乍然消散!

%37
“怎么回事?这片梦境怎么忽然消失了?!”梦境之外。正在养伤的凤金煌陡然睁开双眼,又惊又疑。

%38
方源及时的隐去身形。没有露馅。

%39
他悄悄离开此地,心中亦是惊异非常:“怎么会这样?第三层梦境我还未探索,居然就自动消散。究竟是什么梦,为何如此奇异?我即便是有五百年前世的经验,也从未听闻过有这样的梦境!”

%40
方源眉头紧锁,他隐隐觉得,这片梦境极其特殊,对他而言似乎有着十分重大的意义。

%41
“在梦中,星宿仙尊究竟唱的什么,她想要告诉我什么?”

%42
方源努力回忆,却什么都想不起来。

%43
就像是普通人做梦后苏醒,尽管知道自己做梦,但梦里的细节就是回忆不出。

%44
方源苦思冥想,甚至调动智道手段,但如何也想不起星宿仙尊所唱的内容,只有那悠缓的琴声,在他心头回荡。

%45
几乎与此同时,在南疆。

%46
影宗的大本营生死福地。

%47
生死福地中藏有生死门,而在这门前,也有一片梦境。

%48
砚石老人手持九只八转仙蛊,安步当车,步入梦境。

%49
刚刚进去,砚石老人就感受到凛冽如寒冬的杀机!

%50
“杀杀杀!”一位男子,一身黑袍,双目充血,披头散发,向他扑来。

%51
一时间,砚石老人竟然不能动弹分毫!

%52
若是有外人在场,见到这位黑袍男子,一定会惊呼出声:“幽魂魔尊!”

%53
下一刻,砚石老人被杀退出来,魂魄受创,归于肉身之后,当即就小吐了一口鲜血,满脸青白之色。

%54
幽魂魔尊的这片梦境,比方源探索的星宿仙尊梦境,要更加艰深困难。

%55
砚石老人只是进入梦境中的第一层,就被迫扮演梦中的角色。

%56
而这个角色,就是被幽魂魔尊斩杀的敌人。

%57
砚石老人尝试过无数次,都是刚刚入梦,就被幽魂魔尊杀掉,根本来不及反应。

%58
“不过已经将仙蛊留在梦中,接下里就是启动杀招的时候了。”砚石老人不顾伤势,枯朽如木的右手,颤巍巍地摊开来,露出掌心中的又一只仙蛊。

%59
这只仙蛊正是中洲余木蠢所炼,隶属律道,名为成真。

%60
成真仙蛊被砚石老人用仙元催动,立即化作一道银光,迅疾非常,飞射上空。

%61
它在空中,绕着梦境不断飞舞,拖出来的银色长尾,在半空中停留。并不消散。

%62
“起。”砚石老人心头默念,无数蛊虫飞出他的仙窍,包围住幽魂梦境。在半空中结成阵势。

%63
数以百千的凡蛊,尽皆五转。它们以成真仙蛊。还有滞留在梦境中的九只仙蛊为核心,组成一个神秘的仙道杀招。

%64
半个时辰之后,砚石老人面泛金紫之色,一生积累的仙元损耗七七八八,所有的蛊虫,都损害殆尽。

%65
只剩下一个银色的光茧,光茧十分巨大,将幽魂魔尊的这片梦境包裹得严严实实。

%66
砚石老人吐出一口浊气。身躯一晃,差点摔倒在地上。

%67
他疲惫万分,又带着欣慰之色,看着眼前的银白光茧,喃喃自语:“接下来,就等着孕育功成了。”

%68
数天之后,狐仙福地。

%69
方源哈哈大笑,从地下石窟中走出来。

%70
他成功了。

%71
在狐仙福地中耗费了一个多月的时间,终于创出了一记仙道战场杀招!

%72
此招名为星魂战场,以净魂、星痕、星芽、星光、星念为核心。爆发力虽然不强,但胜在绵绵不休,可消耗敌方极大的战力。

%73
“智慧光晕果然厉害。若是要我单独推算,恐怕三年时间都推不出来。”方源心中充满了感慨。

%74
别看这记星痕战场,连同仙蛊在内,只有六百多只蛊。

%75
但事实上,蛊虫之间的相互配合,运转的方法,简直繁杂至极。

%76
因此,方源要铺设这个星魂战场,至少需要半盏茶的功夫。

%77
这个时间并不长。和其他战场杀招比较起来,还在平均值之下。

%78
一般来讲。战场杀招的铺设都很耗时间。前世,黎山仙子对战黑家四老的青城纵横。使出仙道战场杀招山中梨园。也是在之前埋伏的时候,就开始准备了。

%79
“要论困敌之能,我的星魂战场不敌山中梨园多矣。但战场中消磨手段,堪称无穷无尽,这方面远比山中梨园要强。毕竟这个星魂战场,主要还是参考的战魂沙场。”

%80
有了星魂战场,方源就想找个合适的对手试一试。

%81
他很快就想到了那头一指流鲨。

%82
“一指流鲨是八转战力,身上又有三只宙道仙蛊,我的星魂战场可困不住它。更何况对付一指流鲨,必定要同时对付鲨魔和苏白曼。”

%83
方源旋即就打消了这个不切实际的想法。

%84
他当然也可以联合焚天魔女。

%85
按照这个时间段推算,黎山仙子已经和焚天魔女书信往来多次,初步达成了谅解和一些共识。

%86
方源此时仗着黑楼兰的盟友身份,前去联合焚天魔女,对付一指流鲨,大有可能。

%87
不过,方源不取此法。

%88
对付一指流鲨就算成功,在战利品上,方源也争不过焚天魔女,只能落到一小部分的利益。

%89
而联合焚天魔女,斩杀鲨魔、苏白曼,也不可能。

%90
虽然方源不是僵盟中人,但焚天魔女却是。碍于僵盟盟约,焚天魔女顶多做到打压鲨魔、苏白曼,还不能取其性命。

%91
“不过,焚天魔女还是要找的。我若不找她,黑楼兰、黎山仙子或许还会隐瞒焚天魔女的存在,将她当做奇兵来对付我。我主动找她,便绝了黑楼兰她们的这一层指望,也对焚天魔女进行威慑!”

%92
方源便又回到东海。

%93
东海的鲨魔夫妇,早就对方源望眼欲穿了。

%94
今生他们攻略玉露福地,因为方源,进展大大提前。但最后一关按兵不动,却难如登天。让鲨魔等人无从下手,只能看着仙蛊在眼前飞舞干叹气。

%95
鲨魔对方源的渴望,可想而知。

%96
但方源却不想理睬他。

%97
配合鲨魔,反不如帮助焚天魔女来侵吞玉露福地。

%98
焚天魔女和方源的关系,自然比鲨魔要紧密一些。方源能够得到的利益,也会增多一些。

\end{this_body}

