\newsection{寻常荒兽非我敌}    %第一百三十五节:寻常荒兽非我敌

\begin{this_body}

%1
“好了,我们也下去吧,看看有什么线索”东方万休一边说着,一边撤销了隐身手段,来到东方余亮消失的地面。无弹窗,最喜欢这种网站了,一定要好评]

%2
“太上大长老,不是被东方长凡大人临死前,特意关照过吗?”身边的家族蛊仙,就有很疑惑。

%3
“是啊,大长老特意召集我们一同前来,如今整个部族都没有留守的蛊仙战力。”

%4
“我还以为大长老手中,早就有线索了。”

%5
蛊仙们纷纷表示隐忧。

%6
东方万休长叹一口气,语气萧索:“别要叫我大长老,在我心中,只有东方长凡大人,才是我族的太上大长老。”

%7
“我等都知,万休大人被长凡老大人一手提携。不过如今,长凡大人已经仙逝,谁能没有一死呢?依我看来,还是得着眼现在。”

%8
“不错。如今事关长凡大人的智道传承,引来了这么多的魔道蛊仙。我族虽是超级势力之一,但稍有不慎,就有倾覆的危机。行事还需稳妥谨慎呐。”

%9
“若非万休大人坦言,这是长凡大人临死之前的布置,我们至少也得留出一位蛊仙战力,防备不测,保留我族火种。如今全部集结此处,很是危险。”

%10
蛊仙们紧跟在东方万休的身边,你一言,我一语地说道。

%11
东方万休摆手道:“诸位稍安勿躁,我手中虽然没有正确的线索,但却有获知正确线索的方法。”

%12
说完,他闭起双眼。静心探查传送蛊阵留下的气息线索。

%13
蛊阵气息多变,东方万休每一次探查,都是不一样。

%14
东方万休按照东方长凡临死之前给他的指点。将这些气息线索一一记住,不断排查。

%15
少顷,他缓缓睁开双眼,率先飞上半空,对其余人道:“跟我来。”

%16
……

%17
“这里是哪里?”东方余亮从昏迷中渐渐醒转,他睁开双眼,视野渐渐从模糊转为清晰。

%18
他头疼欲裂。事实上,不只是脑袋,整个身体都布满大小伤口。内脏出血,不少地方都发生了骨折。

%19
“我按照长凡大人留下来的地图,才走到一半路程,怎么会突然来到这里?”东方余亮心中疑惑。一边思索。一边立即着手治疗自己的伤势。

%20
凭他的五转巅峰的修为和手段,身上的伤势很快就得到了良好的治疗。

%21
“我记得我们三个,是在打坐,汲取元石中的元气时,传送蛊阵陡然启动,将我们传送至此的。难道说,东方长凡大人留给我的地图,其实也是假的。或者说后半部分是假的。真正的终点就在半路。”

%22
东方余亮觉得这个猜测比较靠谱。

%23
凭他对东方长凡的了解,东方长凡的性情还真的能做出这样风格的事情来。

%24
这样布置。也是为了防止碰到智道蛊仙出手。

%25
其他流派的蛊仙,投鼠忌器,害怕东方余亮这个线索断掉,又相互忌惮戒备,因而选择吊在后面,伺机而动。

%26
智道蛊仙则有搜索脑海,提取线索地图的能力,完全可以正确地提取线索。

%27
“谭武枫、东破空是和我一起传送的,但此刻却不见他们二人的踪影。也许就在附近,也说不定。我还是先找到他们,再来探索这片宫殿。”

%28
片刻后,东方余亮治好身上伤势,迈开步伐,开始四处搜索。

%29
反目望去,他发现这个地方,好像是一处宫殿群。而他自己则置身其中一座宫殿里头。

%30
宫殿华丽而又破败,俨然是一处占地相当庞大的废墟。

%31
东方余亮警惕戒备,但宫殿中空无一物,没有任何敌人出现。

%32
“谭武枫!”

%33
“公子!”

%34
走了片刻之后,东方余亮碰到了谭武枫,后者也在找寻他。

%35
二人相会不久,在宫殿的出口处,又发现了东破空。

%36
东破空还昏迷着,躺倒在地上。

%37
东方余亮和谭武枫合力将其救醒,三人团聚,自然大喜。

%38
自然仍旧是以东方余亮为首,他思考之后,决定道:“如果我所料不错,这里就应该是长凡大人布置的传承之地。第一座宫殿已经探查遍了,没有发现线索,我们去第二座宫殿里瞧瞧。”

%39
果然不出他的所料,三人到了第二座宫殿,在宫殿中央,迎来一场考验。

%40
这是一道疑难迷案。

%41
给出线索和证据,其中有真有假。又给出时间,需要东方余亮在有限的时间内,动用智道手段思考,不仅要排查出假的证据线索,而且还要推算出这个疑案的真相。

%42
东方余亮不愧是东方长凡选中的继承者,不到三分之一的时间,便闯过关卡,获得了一套五转智道蛊虫的丰厚奖励。

%43
东方余亮实力大增,进入第三座宫殿。

%44
这场考验,则是一道残缺蛊方,考验东方余亮,要求他在一炷香的时间内,推算出来。

%45
……

%46
轰!

%47
一只力道大手,庞若小山,沉重拍下。

%48
咩!

%49
巨角羊发出一声悲鸣,伤重难返,彻底被大手印压在地下。

%50
这头荒兽狠狠地挣扎几下,但大手印死死的压在它的身上,最终它力尽,颓然放弃了抵抗,不再动弹。

%51
方源见此情景,吐出一口浊气,眼中喜悦之光一闪而过。

%52
他循着线索,寻找过去,结果都是假的。到了记忆中的第五个线索,却是碰到了这头荒兽巨角羊。

%53
方源见猎心喜,和巨角羊展开一场激斗。

%54
自从和叶凡、洪易、韩立等人连运之后,方源运道不是缺陷。碰到这头巨角羊身上,并无棘手的蛊虫。

%55
大手印十分实用。屡屡发出,克制巨角羊,很快就将其压入下风。

%56
整个战斗。没有起什么波折。

%57
方源将优势不断积累,最终转化为彻底的胜势,将巨角羊生生活擒。

%58
“巨角羊乃是纯粹的力道荒兽,我将其豢养在福地中。每隔一段时间,便能割其羊肉,辅以吃力仙蛊,增添我身上的力道道痕。没想到居然能在这里。活捉一头巨角羊,倒是意外之喜了。”

%59
方源缓了缓气,仙僵伤势已然痊愈。

%60
他念头一动。大手便缓缓抓起巨角羊,送入他的仙窍之中。

%61
自从方源有了吃力仙蛊,就经常在宝黄天中采购力道食材,花费不菲。一次两次也就算了。长久下去。这笔巨额开支方源也要心疼。

%62
他早就打算,豢养一些力道荒兽,做长期投资。但荒兽买卖,宝黄天中也不多见,更别提方源只是单单购买纯粹的力道荒兽了。

%63
“敲诈出我力仙蛊,真是做对了。我的战力重新达到七转,甚至大手印,比之力道虚影大军。还更要实用一些。像是巨角羊这样的普通荒兽,已经不再是我的对手。哪怕是面对曾经毁掉荡魂山的泥沼蟹。我也可以将它正面击败。就算它的身上有和稀泥仙蛊,也能力保荡魂山的安全!”

%64
方源握了握自家的怪爪,享受着力量带来的掌握局面的快感。

%65
他好整以暇,并不着急。

%66
他明白局势:眼下狼群太多,猎物却只有一个。依他一人之力,将东方长凡的智道传承独自吞并,这种可能很低很低。

%67
“不出意外,这份智道传承即便到手,也是残缺。唉,不知什么时候,我才能获取一份真正完整的智道传承呢?”

%68
方源正想着,这时从仙窍中传过来黎山仙子的信息。

%69
黎山仙子的情报能力,的确是方源不能及的。原来已经有人,发现了东方长凡布置传承的真正地点,那边已经开火,展开了激战!

%70
正因如此,动静闹大了,吸引了越来越多的蛊仙前往。

%71
方源和黎山仙子、黑楼兰成功汇合,赶到那里时,已经有十多位蛊仙,徘徊在外围了。

%72
“啊,是他们。”

%73
“小心点,这三人神秘异常,有些深不可测。”

%74
“我刚刚从东南方向归来,远远看到那位动用大手印,将一头巨角羊荒兽硬生生活捉了!”

%75
魔道蛊仙最看中实力。

%76
方源展现出来的战力,让他们不得不忌惮。

%77
因此当他们缓缓飞来,半空中的这些蛊仙下意识地便让开足够的空间。

%78
“这就是传承地点?东方长凡倒是挖空心思了。”方源俯瞰脚下,旋即有感而发道。

%79
他的脚下,是一头墟蝠尸体。

%80
这头墟蝠大不简单,生前乃是太古荒兽,可战八转蛊仙。

%81
它是宇道荒兽,肉身尸体上绘有无数宇道道痕。这些道痕影响着这片空间,构造出一片虚幻的宫殿群。

%82
只要踏入范围,就会被扯进宫殿当中。除非是八转蛊仙,才有能力抗衡这股庞大的力量。

%83
这头墟蝠死在太丘深处,导致方圆数万里,都没有一头荒兽栖息。

%84
这是禁地中的禁地,凶中之凶。

%85
东方长凡也只是七转蛊仙,居然能将智道传承布置在这里,从这点也可见他不愧是北原智道当代第一,的确是非同凡俗!

%86
就在众仙犹疑在墟蝠尸体上空的时候,东方余亮已经一路凯歌,闯进第八座宫殿。

%87
“你终于来了,我所看中的后继者。”一股星意凝聚成东方长凡的模样,出现在东方余亮三人的面前。

%88
“老师!”东方余亮含泪拜倒。

%89
星意缓缓颔首,俯视东方余亮:“能够成功闯来,可见你的良才美质。这里不是最后一座宫殿,但已是传承的尽头。只要你完善这座蛊阵,并成功开启它,你才算是完全继承这份智道传承,获得我生前的智道仙蛊!”

%90
“老师,我一定尽力。”东方余亮的眼中涌现出强烈的斗志。

%91
“要快,时间已经不多了。”星意复又仰头,望向高空。

\end{this_body}

