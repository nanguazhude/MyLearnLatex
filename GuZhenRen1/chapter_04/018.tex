\newsection{八臂仙人春星雨}    %第十八节:八臂仙人春星雨

\begin{this_body}

%1
“你这团星萤蛊怎么卖?”方源神念探过去。?

%2
“请问阁下如何称呼,又需要多少星萤蛊呢?”万象星君人虽不在,但还遗留了一段意志在货物旁边。

%3
只见这段意志其卖相极佳,星光灿烂,煞是好看。方源智道底蕴浅薄,并没有认出跟脚来。

%4
“你可以称呼我为八臂仙人,这团星萤蛊我全都要了。”方源传念道。

%5
方源用自家福地沟通宝黄天,又换了一套神念蛊、通天蛊,因此万象星君意志并未认出方源就是曾经的“狐仙”,只当方源是个陌生人物。

%6
不过听方源的语气,像是一个大主顾,万象星君的意志顿时显得热切起来:“星萤蛊,乃是上古蛊虫,如今几乎绝迹。我这团星萤蛊,品相极佳,足有一千两百多只。这样吧,你出四块仙元石,我都卖给你。”

%7
方源哈哈一笑:“你在蒙我啊,别以为我不知道正常的卖价。我也是经人介绍,才找到你这里的。星萤蛊也不是只有你家售卖,还有瑶光仙子,还有帝渊等等,他们那里都可买到星萤蛊。”方源曾经命小狐仙,和这三位蛊仙都有过交易。不仅买了十万多只星萤虫群,还有星屑草籽,之后又分别买下了三人栽种星屑草的心得文书,花费不少。

%8
万象星君意志微微一惊,知道方源是个懂行的人,立即话锋一变:“呵呵呵,刚刚是句玩笑话,请八臂大仙阁下不要当真。”

%9
方源打断他的话,直接道:“我出两块仙元石,买你一千五百只星萤蛊。”

%10
万象星君意志立即拒绝,叫道:“阁下压价太狠,这万万卖不了的。”

%11
一番讨价还价,最终双方各退一步。以三块仙元,一千两百只星萤蛊的价格成交。

%12
方源买了星萤蛊,却未立即离开,而是问:“你这里可出售星道杀招么?”

%13
“阁下难道也是星道蛊仙?呵呵呵,当然出售了,你可以瞧瞧。我这里有三件星道杀招。”

%14
方源浏览一番。

%15
第一件星道杀招,为“灿烂星甲”,需要大量的星光蛊。少许克星蛊,一些星盾蛊等等组成。此乃防御杀招,催用出来,蛊仙身上就会形成一具全身甲胄,星光灿烂。第二件杀招,名为“春星雨”。需要大量的星雨蛊,少量春风蛊,少量星芽蛊等等,使用出来。会形成碧光小雨,窸窸窣窣,持续三天三夜。雨贵如油,能使作物营养充沛,增大产量。这是用来栽培植株,经营福地的杀招。

%16
第三件。则是螺旋钻星枪。顾名思意,这是用于攻伐的杀招。需要少量的螺旋骨枪蛊,一些星河蛊。无数星镖蛊,大量的流星蛊等。

%17
这鞋杀招的内容,很有意思。

%18
蛊虫的数量,都以大量、许多、一些、少许这些词来概括,并不写明清楚的数量。同时,也不详细涉及到如何构建杀招,按照什么次序催动蛊虫,什么蛊虫辅助什么蛊虫的关系。

%19
只说明大概情况,需要多少五转蛊虫。杀招的效果又是什么。

%20
只有当杀招真正被购买之后。这些内容才会真正的详细彻底。

%21
万象星君意志适时开价道:“杀招灿烂星甲,售价半块仙元石。春星雨杀招。乃是我本体独创,目前只有几位蛊仙购买,因此售价高些,需要一块半的仙元石。而螺旋钻星枪,则需要一块仙元石。……

%22
杀招依照稀缺程度,威力大小,后遗症,是否容易被针对,组合杀招的蛊虫等各种情况,价格也各有区别。但有一个共同点,那就是普遍都不便宜。

%23
在宝黄天中,通常凡道杀招的价格,还要高于相应层次的凡蛊秘方。但上升到六转层次,仙道杀招的价格,则要大大低于仙蛊方。

%24
灿烂星甲这个杀招,方源即便不是星道蛊仙也知道,这是烂大街的货色。半块仙元石,要价太高。

%25
方源原本打算购买一件星道的攻伐杀招,他记得一种凡道杀招,叫做寒冰星尘。这杀招打击范围广阔,组合的星道蛊虫十分常见,对敌效果也很出色,发出的星尘,有钻破防御的效果。就算钻破不了防御,黏在敌人身上,散发的寒气也会让敌人行动迟缓。

%26
至于,万象星君贩卖的这个螺旋钻星枪,方源还没有听过。

%27
这就表示,这件攻伐杀招的性价比不高。

%28
方源五百年前世,五域大混战,战争淘汰了腐朽的机制,催化了更多更好的新生事物。

%29
方源记不住螺旋钻星枪,说明它并不出色,甚至早早就被淘汰。被淘汰的杀招,一定弊端重重。

%30
倒是杀招春星雨,让方源眼前一亮。

%31
这算是一个意外的收获。

%32
方源想到星屑草,星屑草是星萤虫群的食物,也是家园。而星萤虫群越是壮大,便越能催生出星萤蛊来。方源对于星萤蛊的需求,必将持续很长时间。

%33
所以,他需要这个杀招。

%34
方源将这春星雨杀招,又反复看了两遍。

%35
他发现一个问题。

%36
组合春星雨杀招的蛊虫中,有一种星道蛊虫,名为星芽蛊,这是一种一次性消耗蛊。

%37
方源从未听说过这种蛊,这还是他第一次见到。

%38
方源琢磨了一下,很快明白:这是万象星君的算计。

%39
蛊仙购买了杀招春星雨,以后每次使用春星雨,都要消耗星芽蛊。而这种星芽蛊只有万象星君这里出售。

%40
万象星君将一次性的杀招交易,变成了一场持续不断的买卖。

%41
方源试探性地提道:“我出两块仙元石,买你的春星雨杀招,还有星芽蛊的蛊方。”

%42
星芽蛊不过三转蛊虫,这种凡蛊秘方,却价值半块仙元石,可以说价格已经极高了。但不出方源所料,万象星君意志直接拒绝,没有丝毫犹豫。他推脱为星芽蛊是其福地自然形成的,根本没有炼制蛊方。

%43
这个理由,倒也找的好。

%44
有些福地,经营好了,因为环境奇特,自然会产生新蛊。

%45
不仅是蛊仙能炼制、创造出新的蛊虫,大自然也有新蛊诞生。

%46
方源心知这十有八九只是说辞借口,但他不可能要求进去万象星君的福地验证。

%47
福地是蛊仙的大本营,最大的个人隐私。除非关系极好,相互十分信任,才会邀请蛊仙进入福地参观。

%48
方源见万象星君意志不愿,便又提出新的提议:“这样好了,我出两块仙元石,买下春星雨、灿烂星甲两道杀招,以及一些星芽蛊。这些星芽蛊的数量,至少能够我使用十次春星雨杀招。”

%49
万象星君意志听了,连连摇头:“六次杀招所消耗的星芽蛊总量,就价值半块仙元石。阁下只给两块仙元石,我却还要搭上灿烂星甲这个杀招。这是亏本买卖,我可不做。”

%50
方源便冷笑:“星芽蛊不过三转凡蛊,每次杀招消耗最多五百。六次杀招,也不过三、四千只。你居然卖半块仙元石?……

%51
“阁下有所不知,星芽蛊是我福地自然催生,产量很低。外界也买不到,是我万星福地的独特资源。”万象星君意志说的很客气,但言下之意,就是你爱买不买。你不买,在别的地方休想买到。

%52
这就是垄断生意的底气。

%53
方源笑了笑,继续讨价还价,对方寸步不让。

%54
在这个过程中,方源发现,对方的观念中,灿烂星甲的价值还是很重的。

%55
“看来,五域大战未至,灿烂星甲还没有被各大势力广为传播。杀招的价格,仍旧居高不下。”

%56
战争也是一种极有效率的交流方式。

%57
尤其在这个世界。

%58
和平时期,元石、仙元石自身具有极大价值,难以承担货币角色,经济并不发达。

%59
但到了战争时期,战功,门派贡献等等,取代了元石、仙元石的货币角色,反而促进了各反面的畸形繁荣。

%60
意识到这点之后,方源再不强求。最终他以两块仙元石,买下春星雨,以及足够十次春星雨消耗的星芽蛊群。

%61
达成这笔交易之后,方源继续搜索宝黄天。

%62
他发现宝黄天中,气泡鱼的价格,已经被蛊仙们炒到了原价的十倍多。

%63
方源打探了一下,情况果然和前世记忆一样。在今年年底,气泡海上两位蛊仙相斗,导致剧毒侵蚀整个气泡海,将其化为了绝境。

%64
物以稀为贵,气泡鱼产量降至谷底,价格自然而然就上去了。

%65
“五大域,各有界壁,隔绝内外,因此各成格局,相互之间影响很少。我重生之后,影响了南疆、中洲、北原。其中北原变化最大,已经一片混乱。中洲次之,南疆再次,毕竟南疆的改变只局限在凡人层次。至于西漠、东海,却还是依循老样子。”

%66
方源通过使用春秋蝉,总结出光阴长河的惯性和蝴蝶效应。这种不变和变化相互交织,正是方源对这个世界的时间法则的理解,越来越深刻的证明。

%67
至于气泡鱼,方源记得在今后的五域大战中,这种鱼群会更受蛊仙们欢迎,价格比现在的还要高。

%68
皆因,气泡鱼群能使虫群中凡蛊的形成率提升。)

\end{this_body}

