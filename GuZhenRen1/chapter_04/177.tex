\newsection{道痕新添底蕴增}    %第一百七十七节:道痕新添底蕴增

\begin{this_body}

星萤虫群,方源也是之前放进太白云生的仙窍中去的。<strong>最新章节全文阅读www.Mianhuatang.cc</strong>星屑草群也移走了一部分,现在完全合并,又恢复了原来的规模。

血毒棠化成的毒血,只是污染大地,对天空中的星屑草原却是没有丁点的伤害。

龙鳞湖和玉须湖遥遥相对,两大湖之间,以及周围,又如繁星绕月般,开垦了许多小湖。

这些小湖、河塘就没有动用特殊的土壤了。

龙鳞土、玉须土,都是炼蛊的材料。虽然达不到仙材的标准,但是量这么大,构造了两个大湖,也是让方源付出了不少代价。

这些小湖中,就生存着一些残留的水狼、鱼翅狼,方源压服的那只荒兽鱼翅狼,也在这里徘徊,饿了就捕捉小湖中的鱼群。

小湖河塘中,有海量的青玉鲫鱼。这些青玉鲫鱼生命力宛若小草般极其顽强,毒血都毒不死它们,反而使得其中一部分发生了变异,形成血玉鲫鱼。在市场中的价格,这种血玉鲫鱼反而比青玉鲫鱼高出几倍出来。

而在狐仙福地的西部,方源如法炮制,挖掘泥土,布置蛊阵,采取了火炭石,构造了一座大湖。

但这座湖中,却是没有一滴水的。反而热气蒸腾,火炭的碎石铺满湖底,风一吹,灰黑射的火炭石头上便掀起一阵赤红的光。

这个湖,方源命名为火炭湖,是幽火龙蟒次佳的繁衍之地。而最佳的方法。也就是东方部族的大本营碧潭福地里的布置。

这种最佳的布置方法,方源也从东方长凡的魂魄中得知,可谓一清二楚。

但可惜,方源的财力还不够支撑。尤其是前期投入的资金实在太多,方源深思熟虑之后。只能采取次佳的火炭石头,来营造出幽火龙蟒的巢穴。

将手中的幽火龙蟒,全部放置其中,便任其发展。方源又在远处,每隔一段距离,挖开小湖,却不铺设任何泥土。设下简单的蛊阵后。便直接将自己仙窍中摄取的毒血,一一倾倒进去。

这些毒血,在蛊阵的隔绝下,每年都只会向外渗透一小部分,浸染周围的土质。

整个狐仙福地的西部,方源挖掘了成百上千的小血湖。

为什么方源要这么做呢?

这就要论述到天灾地劫和福地的关系了。[www.qiushu.cc 超多好看小说]

之前也讲过,福兮祸所伏祸兮福所倚的道理。

福地的底蕴越深厚。福分越强,惹来的天劫地灾就越厉害,更具有针对性。

譬如狐仙福地的前一次地灾,荒兽泥沼蟹又携带了和稀泥仙蛊,针对的就是狐仙福地中最有价值的荡魂山。结果荡魂山化作一滩稀泥,经过方源的抢救,方才险死还生。

可见天道至公,讲究平衡。你有一方面特别占据优势,惹来天妒,天劫地灾就会来削弱你。当然。灾劫种类太多,这个规律也并不绝对,只能说可能性很大。

这就是“福兮祸所伏”的体现。

但泥沼蟹地灾渡过之后,事实上,狐仙福地中新添了大量的奴道道痕,以及少量的土道道痕。

奴道道痕增多,能让福地中豢养的生命。更加容易控制,繁衍更快。石人、毛民、狐群、狼群、鱼群等等,皆在此列。

这是狐仙福地最有优势的一面,毕竟原先的主人狐仙,就是奴道蛊仙。每一次渡劫,新添的道痕都是以奴道数量最多的。

除了奴道之外,还有少量的土道道痕。这些道痕,能让狐仙福地的土壤更加肥沃,大地更加深厚,更能汲取地气,承载荡魂山。

这些道痕,不是寻常手段能观察到的。而是融入狐仙福地,将这方小世界打造得更加全面,更加优秀。

而这便是“祸兮福所倚”的体现了。

中洲天地针对荡魂山,发动了泥沼蟹地灾。但渡过之后,狐仙福地却是有收获的,土道道痕的增长,反而使得荡魂山和狐仙福地的关系更加紧密。

而这一次地灾,是血毒棠的花海。方源私底下猜测,可能是中洲大天地针对他血道宗师境界所发。

撑过去之后,狐仙福地中再次新添了大量的道痕。其中,仍旧以奴道道痕数量最多,将狐仙福地本来的优势再次放大。同时,还新添了许多血道道痕。

血道道痕的出现,无疑改良了狐仙福地。

青玉鲫鱼转变成血玉鲫鱼,只是一个方面的表现罢了。在方源耗费力气,铲除了狐仙福地最上层的土壤之后,裸.露出来的下一层土壤中,已经冒出大量的细嫩草芽,以及小巧的花骨朵儿。

这些野草,称之为血镰草,完全长成后,形似镰刀般弯曲,背部较厚,外部则锋利如刃。

而这些野花,名为赤斧花。花瓣宛若一片片的斧头刀刃,花瓣都是白色的,但却能微微绽射出赤红的光。

不管是血镰草,还是赤斧花都是上佳的炼蛊材料,虽然不是仙材,但却可以用来炼制三转的凡蛊了。

遍及狐仙福地的血镰草、赤斧花,一旦成长起来,数量惊人,是一笔不小的财富。

而这笔财富,实际上就是这次地灾,带给狐仙福地的。

其实狐仙福地之前,也生长着不少花草,如蓝汪汪的剧毒蓝度草,马蹄形状的马蹄草,六片细长的叶子,如玉般细腻光泽的六神草。还有七彩缤纷的七宝小花,杯子形状,盛着奶茶般花汁的奶茶花。

但这些花草,顶多拿来炼制一、二转的凡蛊。价值要远远小于血镰草、赤斧花的。

毒血渗透土壤,将这些花草一部分的根系和种子,转变成血镰草籽和赤斧花种。最关键的,还是血道道痕的增添,改造了狐仙福地的大环境,使得血镰草和赤斧花有了生长的根本。

六转蛊仙,每十年一地灾,百年一天劫。历经三百年,总共渡过三十次地灾,三次天劫之后,便会晋升为七转蛊仙。

地灾的威力要弱于天劫,每一次的灾劫都是上天的艰难考验,充满了凶险,但也充满了机遇。

新增的道痕,会改良这片天地,能孕育更多,更有价值的事物。这些事物又会给蛊仙,提供修行的资粮。

狐仙最初在福地中,只能栽种了大量的普通花草。那时候,土壤浅薄,无水无风,若栽种蓝度草、马蹄草,都不会存活下来。

遭受几次地灾之后,花草数次凋零灭绝,狐仙损失惨重,却新添了木道道痕、土道道痕。

又因为火灾,有了火道道痕。经过水灾,有了水道道痕。

狐仙福地有了这些道痕,环境改良,狐仙这才能够移栽了蓝度草、马蹄草、六神草、奶茶花。

虽然有了这些机遇,福地也得到大幅度改良,但每次地灾,都会有损失,还要看蛊仙个人的经营能力。

狐仙屡遭削弱,实力不断下降,在渡第五次地灾的时候,终是死于魅蓝电影之下。

方源继承福地之后,勉强撑过第六次荒兽成灾的泥沼蟹。连同这次,狐仙福地已经渡过第七次地灾了。

这一次地灾之后,狐仙福地有了血道道痕,便了血镰草、赤斧花这等花草的生长环境。

老农都知道,什么样的土地,栽种什么样的作物,收成才最好。方源经营福地,当然更要因地制宜,扬长避短。他不让太白云生动用江山如故,反而自己还挖掘毒血湖泊,就是这个道理。

“事实上,不仅是这些花草,有了这些血道道痕,我可以豢养血纹狐。血纹狐的繁衍速度,将比之前的任何狐群,都要高出一大截。每年的产量都会比较客观,若再收购一些豢养血纹狐的经验心得,运气好的话,甚至可以挤入市场,卖点小钱了。”

不过血纹狐是生物链的中层,方源还要引进一些血纹兔,作为狐群的主食。

那血纹兔吃什么呢?自然要吃血镰草了。

所以方源大费周章,挖掘了这么多的毒血湖泊,就是将这些毒血当做营养,在可控的蛊阵下,渗透到泥土中,不断地催生出血镰草、赤斧花来。

当然为了保证血纹狐的活力,方源还打算引进一些毒须狼。之前的狐仙福地,不适合这些毒须狼生存。但现在有着毒血渗透,还是可以养活一小部分的毒须狼的。

毒须狼数量无须太多,只需要一小部分,对血纹狐形成鲶鱼效应即可。

方源忙得团团转,脚跟都要踢到后脑勺了。

忙碌中时间总是过得飞快。

似乎一转眼,狐仙福地中已经过去了大半个月。

方源改造狐仙福地的工作,终于告一段落。

狐仙福地焕然新生。

在东部,龙鳞湖、玉须湖交相辉映,周围是各个小湖河塘。天空之上,有星屑草原。湖中有龙鱼、气泡鱼、青鱼鲫鱼、血玉鲫鱼。湖边有少量的鱼翅狼群、水狼群,一些地皮猪。

在西部,湖泊数量暴涨,一举超越了东部。在西部中央,是火炭石湖霸占着。

每隔一段距离,就有一座血毒小湖。成百上千的血湖平均分布在福地西部,若从高空鸟瞰,景象壮观。这里不久之后,就会形成血镰草原、赤斧花海。目前生存着少量的血纹兔,少量的毒须狼,以及少量的血纹狐。但血纹狐将会不断从外界引进,数量将会越来越多。

值得一提的是,荒兽巨角羊也放养在这里,防止遭受东部荒兽鱼翅狼的捕猎。(未完待续。)<!--80txt.com-ouoou-->

\end{this_body}

