\newsection{星象福地}    %第七十九节:星象福地

\begin{this_body}

%1
“就是这里了。”方源环顾左右,满意地点点头。

%2
“这里是哪里?”黑楼兰从方源的仙窍中钻出来。

%3
二人运用定仙游,从繁星洞天撤退后,就直接来到了这里。

%4
黑楼兰环顾左右,发现这里疑是一处地底溶洞,或者山体中空的内部。光线稀少,一片晦暗,又有怪石嶙峋,湿气极大,不过空间倒是颇为广阔。

%5
“这里便是地渊了。”方源淡淡地回答道。

%6
“地渊?”黑楼兰向方源投来诧异的目光。

%7
她对中洲不陌生,知道这个地渊,乃是位于整个中洲极西,是一处广袤无比的地下世界。

%8
地渊分为数十层,每一层都至少有数亿亩的面积。空间十分广阔,且地下深幽,各种溶洞甬道,有的仿佛迷宫,有的积成巨大的地下湖泊,有的空旷如平原。

%9
地渊中,生活着成千上万种的生物,生机盎然,别致的生态明显区别于地表。

%10
“的确是个好地方,地渊中地气浓郁,会大大缩短福地落下的时间。”

%11
黑楼兰点点头,旋即话锋一转:“不过,在这地渊之上,就是古魂门的大本营。古魂门是十大古派之一,宛若庞然巨兽,横霸极西之地已经数千年。他们占据地渊,早已经将其作为禁脔之地。你要在这里落下福地,就像是在别人家的庭院里种花,不怕被发现吗?”

%12
方源哈哈一笑:“放心,地渊深幽无比。古魂门耗费了数千年光阴,也不过才探查清楚十八层,略微探索到第二十七层。从第二十八层到第三十六层,也只有古魂门中的蛊仙偶尔进出。只要我们种在四十层以下,哪怕动静再大,也不怕被发现。”

%13
“四十层?我怎么记得,古魂门探查的地渊,只有三十六层啊?”黑楼兰神情诧异。

%14
方源嘿然一笑:“那是因为他们无能,地渊深幽。超出世人想象。何止三十六层?”

%15
话说到这里,他不由地就想到前世。

%16
前世方源五转时候,从地渊中产生无穷兽潮。反攻地面,祸乱中洲。

%17
古魂门首当其冲,损失惨重,可谓元气大伤。

%18
兽潮冲出地渊之后。不断驱逐中洲地表的兽群。渐渐酿成规模空前,蔓延整个中洲的恐怖兽潮。

%19
那几年,中洲生灵涂炭,无数的小门派仿佛石头掉入洪水中,扑腾几朵浪花后,就烟消云散。

%20
十大古派领袖群雄,四处扑灭兽潮,忙得焦头烂额。

%21
耗费数年光阴。总算清除了中洲地表的兽潮,随后又广邀正魔两道的蛊师。一起深入地渊。

%22
方源也因此混进地渊,斩杀野兽,获取资源,供给修行。

%23
一层又一层的清剿,一层接着一层深入。深入到第三十六层之后,蛊师们发现了新的通道。

%24
继续往下深入,是更大的地底世界,无数兽群盘踞,荒兽、上古荒兽,各种险地密布。

%25
直到方源自爆,针对地渊的探索,也没有结束。只是发现的,就有一百零七层。

%26
古魂门现在占据的三十六层,不过是最接近地表的一小撮罢了。

%27
这其中的详情,方源自然不会和黑楼兰明说,他只是道:“跟着我来就行。”

%28
黑楼兰见方源自信十足,像是有把握的样子,也就跟着他一起深入地渊。

%29
他们进入的位置,只是第八层。一路向下,二人见到不少古魂门的蛊师。这些蛊师大多成群结队,有的在围猎野兽,有的在采集地底苔藓。

%30
越是深入,人烟就越稀少,修为则相应增加。

%31
从起先的二转蛊师,到三转,再到四转。

%32
这些凡人哪有什么能力,觉察出方源、黑楼兰的行迹?甚至在狭窄的小道中,和一些蛊师擦肩而过,这些凡人都毫无反应。

%33
下到三十层时,已经看不见凡人蛊师了。周围一片晦暗,视线大为限制,不过好在方源、黑楼兰皆是蛊仙境界,手头上有众多五花八门的侦查蛊。

%34
到了三十五层,有一位古魂门的六转蛊仙,正在被一头上古荒兽,四头荒兽撵着跑,好不狼狈。

%35
不过这位蛊仙倒是帮助了方源一个忙,方源、黑楼兰顺利通过荒兽的营盘,来到第三十六层。

%36
随后,方源按照前世的记忆,来到关键地点。

%37
布置好六座蛊阵之后,方源手指着脚下,对黑楼兰道:“我攻击不足,还要你出手,催动仙蛊,照着这处泥地打。”

%38
黑楼兰依言而行,勉强变化成力道虚影巨人,照着泥地狠狠捣了三拳。

%39
泥石翻飞,却从不飞溅到蛊阵外围去。响声如雷,但在蛊阵外门,却是一丝声音都没有的沉寂平静。

%40
泥地没有被砸通,只是一个深坑,深达七八丈的样子。

%41
“继续,快!”方源催促一声,首先跳下去。

%42
黑楼兰深呼吸一口气,猛地跳下,拳影翻飞。

%43
就这样连续捣击,巨坑越来越深,数百丈之后,终于砸通,形成一个漏洞。

%44
由此,方源、黑楼兰二人深入到了第三十七层。

%45
黑楼兰满身大汗,仰头回望,只见头顶上的洞口,在蛊阵的帮助下,迅速复原着。

%46
“这里是第三十六层,最薄弱的地段。即便如此,也有近千丈的厚度。”方源适时地解释道。

%47
“你是怎么发现这里的?”黑楼兰满肚子疑惑。

%48
方源嘿嘿笑了几声。碍于雪山盟约,他不能撒谎,只能保留不说。

%49
黑楼兰见他不答,冷哼一声,却也明智没有追问下去,

%50
进入三十七层之后,路途就明显的崎岖艰险得多。毕竟这里完全是原生态。方源、黑楼兰算是首次进入当中的外来探索者。

%51
等到两人好不容易,下达到四十层时,封印万象星君仙窍的时限也到了。

%52
“你倒是估算得挺准啊。”黑楼兰将万象星君的尸体放置在地上。深深地看了方源一眼。

%53
方源心念一动,从万象星君身上飞出数百只凡蛊,宛若一群蝗虫,嗡嗡地飞进方源的仙窍中去了。

%54
没有了最后一层压制,万象星君的身上,渐渐散发出明亮的星光。

%55
星光越来越盛,同时大地开始微微颤抖起来。

%56
几个呼吸之后。无数的地气浮现出来,仿佛地面上积了一层两三迟的灰尘。

%57
星光稳定下来,刺眼的光芒堪比夏日正午的太阳。

%58
一股无形的吸力陡然爆发。将周围的地气不断汲取。方源、黑楼兰已经远远退出去,和星光保持距离。

%59
“怎么回事?”

%60
“地震了,又地震了!”

%61
“快走啊,这一次地震幅度很强。再不走就要被活埋了。”

%62
身处地渊中的蛊师们抱头鼠窜。惊惶大叫,争先恐后地撤离地渊。

%63
他们并不奇怪。

%64
地渊结构并不稳定,时常发生地震,或者倒塌事故。也正是因为如此,古魂门虽然掌控此地,却一直以来都没有耗费精力重点探索开发。

%65
大量的石块,从头顶上砸下来,更危险的是那种枪尖钟乳石。只要被砸中,三转的防御蛊都不能抵挡。

%66
事实上。不仅是蛊师遭殃,生活在地渊中的生物,也是生灵涂炭,许多都被碎石砸死,或被活埋。

%67
方源为了落下福地,造成了至少十数万条生命的陨落。

%68
大约过了一盏茶的时间,星光缓缓停止吸收地气,形成一道四方大门。大门完全是由星光构造,悬浮于空。门头又有一个门匾,上书四个大字星象福地。

%69
黑楼兰吐出一口浊气:“终于成功了,看来我们的运气不错。动静虽大,却始终没有引起这里的野兽袭击。”

%70
“走吧。”方源一马当先,推开星门。

%71
黑楼兰紧随其后,二人终于进入星象福地。

%72
这里好像是深夜,天空漆黑,布满繁星点点。

%73
星辉灿烂,映照福地地表,可见度很高,并不晦暗。

%74
整个星象福地的地貌,就是一个巨大的盆地。中间是不断内凹的坡地,而福地的边缘,是连绵的山脉,围成一圈,仿佛围墙一般。

%75
“好多的星屑草!”方源看到脚下,竟然生长着的都是星屑草。

%76
明明星屑草要种在云土中,但眼前的这些星屑草却是直接栽种在黑色的泥地上,反而长势比方源在狐仙福地中的那一片,还要良好。

%77
方源赞叹道:“这就是星道福地的好处啊,栽种相关的植株,不仅更加容易,而且保证收成。”

%78
黑楼兰的目光,则被夜空中悠然飞行的刺脊星龙鱼吸引。

%79
就她看到的,就有三头刺脊星龙鱼。

%80
这些巨大的龙鱼,皆有有正常鲸鱼大小,却形似鲤鱼。背脊处,有骨刺长出体外,长长地延伸出去。

%81
它的鳞片都是汪蓝色泽,一双死鱼眼大如马车,星芒绽射。

%82
不过在方源、黑楼兰的敏锐观察下,很快就发现了一些明显的战斗痕迹。不远处,大片的草地还在燃烧着火焰,黑烟升腾。三头刺脊星龙鱼的身上,皆有伤痕。视野的尽头,那些连成一片的山脉,也呈现出不自然的缺口,仿佛被老鼠啃噬过的酪干。

%83
“看来这位蛊仙,曾经有过惨烈的战斗。”方源道。

%84
“不错,若非我的本体被宋紫星重创,不管是肉身,还是魂魄都有暗伤,我的本体是堂堂的万象星君,怎么可能轻易地陨落在第八星殿中呢?”一个男童陡然闪现在方源、黑楼兰的面前。

%85
他粉雕玉砌,身穿一个粉蓝色的小肚兜,小胳膊小腿白嫩如藕,悬浮在半空中,小腮帮子气鼓鼓的,气愤填膺。

\end{this_body}

