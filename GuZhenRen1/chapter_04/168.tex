\newsection{推算利器属星念}    %第一百六十八节:推算利器属星念

\begin{this_body}

ps:想听到更多你们的声音,想收到更多你们的建议,现在就搜索微信公众号“qdread”并加关注,给《蛊真人》更多支持!

“也罢!我还是稳妥起见,将这些都用来喂养。净魂仙蛊已经饿得狠了,虚弱不堪,再不喂养,拖延下去,就会饿死。至于下一次喂养,只能期待将来了。”

方源是魔道,喜欢冒险,但不是盲目冒险。

也要衡量收益和风险。

抱着侥幸心理,自以为运气不错,拼概率拼可能,往往让残酷的现实教训得头破血流、鼻青脸肿。

净魂仙蛊,就在方源的仙窍当中。

它只有成年人的拳头大小,宛若一只蝌蚪,通体灰白干瘪,趴在地上,一动不动。

方源念头一动,运用蛊虫的力量,将白莲巨蚕蛊剖开,取出血肉送到净魂仙蛊的面前。

净魂仙蛊身躯微微颤动,竟然没有一丝力气爬上去就食!

方源只得将净魂仙蛊放到血肉上。

净魂仙蛊躁动了一下,旋即平静下来,默默地一小口一小口地吞吸肉上的血液。

好一会儿,血液吸收入腹,终于让它有了一丝力气,口器张开的更大了一些,开始吞食白莲巨蚕蛊肥嫩的血肉。

它吃的越来越多,干瘪的身躯仿佛是充了气的气球,渐渐恢复了圆润饱满。

灰白的身躯也显现出了一丝晦暗的光泽。

当全部的白莲巨蚕蛊都被杀死,血肉供给净魂仙蛊吃完,净魂仙蛊没有之前的颓靡。已经可以在半空中随处飞游了。

它像是灰白蝌蚪,圆鼓鼓的脑袋如同吹了一半的小气球。尾巴在脑袋后不断甩动,游的慢悠悠的。宛若老年人,有气无力的样子。

它偶尔张开嘴,它的嘴巴占据脑袋的一半,这时候已经能够全部张开,发出支支吾吾的声音。

方源知道,这是它欲求不满,还想吃食。可惜白莲巨蚕蛊已经消耗光了,没有一只剩下。

虽然满足不了净魂仙蛊,但方源还是松了一口气。

至少净魂仙蛊暂时脱离了饿死的尴尬结果。方源之前已经打算舍弃它了。

现在它还活着,也算是皆大欢喜的结局。

更关键的,是争取到了宝贵的时间。将来的事情,谁也说不准,但的确充满了可能。

净魂仙蛊还未彻底吃饱,方源也就依旧留着它,尽量不去动用。

暂时解决了净魂仙蛊喂养的困难,方源松了一口气,却没有片刻的休息。而是抓紧一切时间,又赶到了地底洞窟。

地底洞窟当中,生长着一小片芝林。

最高的灵芝王,有两丈多高。直接顶着洞壁。灵芝肉叶像是一柄巨伞,覆盖左右。

方源来的时候,便发现九转的智慧蛊。正围绕着这株灵芝王飞舞,宛若玩耍的孩童。扇动着翅膀,在半空中忽上忽下。时而又撞击灵芝王。被后者充满弹性的芝体弹飞出去。

方源心中一动,他手头中已有不少仙蛊,但和这些仙蛊相比,九转智慧蛊似乎多出了许多灵性,显得十分与众不同。

再联想《人祖传》中,智慧蛊、力量蛊等等,甚至能和人祖对话!

当然,这可能是寓言的修辞夸张手法。

不过方源也知道,智慧蛊的确是可以沟通的。正是因为可以沟通,他才会在王庭福地毁灭的关键时刻,将智慧蛊带走,如今可以蹭用它的智慧之光。

见到方源来临,智慧蛊不再飞舞,而是停留在灵芝王的树干上。

等到方源寻到一个低矮粗壮的灵芝,当做板凳坐下后,智慧蛊便开始静静地散发出智慧光晕,笼罩住方源。

方源催动星念蛊。

星念蛊是一次性的消耗蛊,伴随着一只只星念蛊的损耗,方源的脑海中念头此起彼伏,源源不断地产生。

一时间,宽阔的脑海中宛若一颗颗的星光亮起,半晌后,变得夏夜夜空一般,无数的繁星交相辉映,美不胜收。

方源试着推算。

顿时,脑海中的星念激动起来,宛若往锅里倒入了一大盆的热水。

星斗横空,星光璀璨,一颗颗相互碰撞,又猛地弹开。

灵感无限,机变无双,种种答案,繁杂过程,在顷刻之间铸就!

方源旋即停止,睁开双眼,目光中流露出浓郁的喜色。

刚刚他只是小试牛刀,这份智道传承果然没有叫他失望,反而比设想中还要更妙一些。

“东方长凡的这份智道传承,果然十分擅长推算。星念耐用极了,初步估算的话,一个星念能抵三四个的恶念、忆念。”方源在心中暗暗比较。

智道中,有念、意、情三分。

只说念头,分门别类,五花八门,说是成千上万种,都不为过!

念头相互区分,又各有优劣。

比如恶念擅长算计别人,忆念则长于在记忆中挖掘内容。但这两者,用来推算事物,就很一般。星念不擅长算计别人,也不精通在记忆里挖掘内容,但用来推算,却分外好使!

一个星念能抵其他念头三四个,这个比例似乎还有点不起眼。但想想看,推算一次仙蛊方或者仙道杀招,需要耗费多少的念头呢?

这个基数放大之后,差距就极其巨大了。成百上千,成千上万,甚至数十万,乃至数百万的念头消耗下去,用星念来思考推算的优势,简直大到无边!

可以说,星念智道传承是最适合方源不过的了。

一直以来,方源都只恨念头不多!

智慧光晕消耗念头极其剧烈,再多的念头也能消耗得掉。

方源手头上关于推算的计划安排,本来就有很多。

第一是推算仙道杀招见面似相识。之前没有成功,只完成了一小部分。

第二是完善仙蛊方血神子。也没有成功。浅尝辄止。

第三是推算仙道杀招,让时运仙蛊和寸光阴凡蛊配合。减少寸光阴的消耗,增幅自身暂时的运气。这个先前方源尝试过,彻底失败了,原因是宙道境界不足。现在倒不用推算了,因为方源已经从黎山仙子手中,直接得到了更好的仙道杀招时济运。

第四是将拔山、挽澜两只力道仙蛊,融入到仙道杀招万我当中,增强万我之威。

还有第五个,改良气囊蛊等等的炼制蛊方。细分步骤,降低炼蛊风险,减少毛民的损耗。

方源深呼吸几次,渐渐平息心境。

见面似相识是必须要优先推算的,因为风头越来越紧,方源要在外行走,就得需要仙级的伪装,防止身份暴露。

改良气囊蛊等等蛊方,并非是当务之急。而且这项计划内容太多。将来方源大规模炼蛊,必定不止气囊蛊、恶念蛊、忆念蛊、星念蛊这些。

至于血神子,更只是长远计划,甚至不是必须之物。毕竟方源现在走的力道。打算修行宙道或者智道,没想往血道的路上发展。

换做先前,方源当然以推算仙道杀招见面似相识为主。但现在狐仙福地地灾在即,方源优先考虑的。却是将拔山仙蛊,融汇到万我杀招之中。

方源紧闭双眼。沉下心来,全力推算。

脑海中星辉竞相回应,无数星念沸沸腾腾。

时间流逝,半个时辰,两个时辰,半天,一天两天……

直至第三天午后,方源才睁开双眼。饶是仙僵之躯,也是满脸倦怠,疲惫不堪。

“终于是将拔山仙蛊,成功地融合到了杀招万我之中了。万我之威,暴涨一成。若是对付土道蛊仙,或者是什么山川陨石,效果更要拔升,多出两成威力!”

方源再检查星念蛊。

仙窍中的星念蛊,还剩下近四万只。

星念蛊原有五万多只,也就是说,这一次思考推算,只耗费了一万多只星念蛊。

星念蛊果然是耐用!

“当然,这个耐用是用来推算事物的。用星念来算计他人,思考阴谋,还是恶念效果好。在回忆方面,星念蛊也不如忆念蛊耐用了。”

不管是星念蛊、忆念蛊、恶念蛊,还得多炼,大量囤积。

此项大计,任重而道远。

算算时日,还不到狐仙福地的渡劫之期。但若是用这时间来推算仙道杀招见面似相识,时间又不够,推算不出什么名堂出来。

方源索性回到荡魂行宫,继续钻研智道传承。

这份仙级的智道传承,内容博大,包含数百张凡蛊蛊方,十多张仙蛊蛊方,数十种凡道杀招,近十种仙道杀招,涉及攻防、进退、储藏、治愈等方方面面。

除此之外,还有种种智道手法、技巧。比如说扩宽脑海空间的修行秘术,还有改造心脏,形成心窍的独到法门。

还包括天文地理常识,五域资源分布图,北原势力大概,以及介绍在修行中遭遇的种种危机险情,又如何化解等等。重点还有蛊师升仙的内容,对种种天劫地灾的介绍,以及如何应对。

当然还有其他流派,诸如炎道、金道、宇道、宙道等等的介绍,这些流派的蛊仙有什么优劣,有什么需求,如果对战通常该怎么对付。

还有经营之法的探讨,该如何经营蛊仙福地,使得自己生存下去,修为不断精进。

可以说,简直是一份蛊师修仙的百科全书,让人立足于现实,明辨是非难易,掌握生存对敌等等手段,修行壮大下去。

这份传承,并非是创始人那位天庭的星道蛊仙一人的功劳。而是接下里,每一个继承了这份传承,用以修行的蛊仙,不断地增添内容、改进内容。一代又一代,耗费精力心血,最终形成的修行宝典。

方源从中可以看到,很多中洲、北原的古代历史。也在仙道杀招的末尾,发现最后一任的继承者东方长凡新添的仙道杀招万星飞萤。(小说《蛊真人》将在官方微信平台上有更多新鲜内容哦,同时还有100\%抽奖大礼送给大家!现在就开启微信,点击右上方“+”号“添加朋友”,搜索公众号“qdread”并关注,速度抓紧啦!)(,!

\end{this_body}

