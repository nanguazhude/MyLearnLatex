\newsection{商谈大计遭碰壁}    %第一百七十三节:商谈大计遭碰壁

\begin{this_body}

ps:

得知太白云生回到了狐仙福地,黎山仙子便尽自己最快速度动身,赶到了狐仙福地。

如今大雪山那边,有雪胡老祖时刻镇场,太白云生作为外人,身份又十分敏感,已经不方便去那里。

因此,最好的场所,仍旧在狐仙福地。

由于太白云生已经有了救治荡魂山的经验,因而整个过程并无惊险,平平稳稳地渡过去了。

脸色苍白的黎山仙子,这才松了一口气,渡过这场劫难,她心中高悬的一块大石总算是落了地。

不过她身上的伤势,仍旧不轻,还需时日缓缓恢复。

“仙子且慢。”黎山仙子欲回大雪山福地之时,方源却挽留住她。

“有一项大买卖,不知道仙子有无兴趣?”方源旋即谈及自己的计划。

他是没有财力,去炼制仙蛊的。

但是并不代表,集合太白云生、方源、黑楼兰、黎山仙子四人,没有这等财力。

“如今姜钰仙子行踪不明,要抢夺暗渡仙蛊,我们又没有盗天魔尊的手段,十分艰难。我从东方长凡处得到一个仙道杀招,名为星雾掩,却是能混淆天机的,极其适合我们。”

方源的这番话,顿时打动了黎山仙子,以及一旁的黑楼兰。

不过当听到星雾掩杀招,需要两只仙蛊充当核心才能催发时,黎山仙子摇头苦笑:“要炼一只仙蛊已经超出了我们的极限,两只仙蛊彻底是不可能了。”

这次将方寸山还原,省下了黎山仙子一大笔费用,但她身上有伤,要请其他蛊仙出手治疗,还有雪胡老祖加派在她身上的沉重任务,之前一直在资助黑楼兰修仙,深厚的财力已然见底了。

她陷入了和方源一样的困窘境地。

方源无奈之下,又换了另一套融资方案。

他提出:集齐四人的财力,收购豢养毛民的心得经验,并且大量购买毛民奴隶。从而加大石巢规模,加大气囊蛊的产量,也就加大了胆识蛊的贸易。之后,四人之间再瓜分红利。

这个方案耗费的财力,自然要远远小于炼制仙蛊的。

但也遭到了黎山仙子的拒绝:“方源,你有没有想过仙鹤门的反应呢?你做胆识蛊的垄断生意,可以说是火爆,已经惹了许多觊觎。若是加大贸易规模,仙鹤门势必更加眼红,万一打破心中的平衡,再次出手来收复你的狐仙福地,你又该如何呢?我们的麻烦已经够多,多一事不如少一事啊。”

方源沉默了一下,这才道:“仙子说的是,这就作罢吧。”

黎山仙子回去大雪山福地,黑楼兰则继续留守在石巢中,为方源炼制气囊蛊。

方源提议的两项,都没有成功。他心怀不甘,回到荡魂行宫后,便又联系了琅琊地灵。

琅琊福地中的毛民,是方源见过质量最上乘的。就算是宝黄天这种高端市场中,都没有琅琊福地中的毛民优秀。

从这一点就可以轻易推测,琅琊地灵手中必然掌握着最优秀的豢养毛民的心得经验。

方源手中没有仙元石,但他掌握了智道传承,有不少仙蛊方,又有大量的仙材,正是琅琊地灵渴求的东西。

能不能用这些东西,和琅琊地灵交易呢?

方源抱着希望而去,结果碰了个鼻青脸肿。琅琊地灵态度坚决,宛若铁壁,丝毫寰转的余地都没有。

“想打毛民的主意?你想都别想!想要和我交易,只有一个方式,那就是完成任务!”琅琊地灵旧事重提,仍旧想要那什么六转毒花的花瓣。

方源这次有些意动,想了想,便问:“若是我为你取得了这些炼蛊仙材,你是否就能卖给我毛民奴隶,或者一部分的豢养经验心得?”

琅琊地灵一挥长袖,一口回绝:“这不可能,涉及到毛民的一切东西,都是非卖品,你趁早打消这个鬼主意!不过你若能收获我指定给你的仙材,我可以将全力以赴蛊的蛊方,交给你。事实上,我不仅拥有一转到五转的全力以赴凡蛊蛊方,还有六转的全力以赴仙蛊方!这不正是你想得到的东西吗?”

琅琊地灵抛出的这个诱饵,的确叫方源心动,又让方源有些暗怒。

琅琊地灵这是把方源当枪使唤,但要获得仙材花瓣,岂是那么容易的事情?

这里面风险太大,方源是要拿命去拼的。

常言道,吃一堑长一智。方源在琅琊地灵身上占了多少次便宜?

但北原拍卖大会那一次,地灵吃亏吃的实在太大,痛得让地灵醒悟惊觉。

所以琅琊地灵变得如此难缠,原因还在方源的身上。

付出了才有收获,反过来讲,有了收获必有代价。

方源不劳而获了那么多的仙材,代价就是和琅琊地灵的关系降至冰点。

偏偏地灵这种东西,是执念结合天地伟力所化,固执得很。虽然不会撒谎,但下定决心,绝不轻易改变。

所以一时间,方源也无法可想。琅琊福地方面,已经不能帮助他什么了,更别提向琅琊地灵融资这一说。

计划失败,方源只能长叹一口气,将精力主要集中在福地渡劫一事上。

两天后。

“万我大手印,起!”

方源催发仙道杀招,八只力道大手中四只抓住荡魂山的四角,往上提运,剩余四只则托住荡魂山的底部,往上抬。

荡魂山迅速拔高,很快就距离地面五六丈了。

太白云生立在高空,看到这一幕,一时间目瞪口呆。

石巢处,黑楼兰远远眺望,看着巨大的荡魂山浮在高空,不禁口中喃喃:“短短时间,他又变强了……”

方源这一次提拿荡魂山,比之前轻松了无数倍。不单单是因为八只力道大手齐出,主要的原因在于拔山仙蛊,已经彻底地融入到了万我杀招之中,充当了核心。

原本万我核心,是净魂仙蛊。而后净魂饥饿,不堪催用,便用我力代替。如今万我杀招的核心,已经有两个,一个是我力,一个是拔山。

“别愣神,接好了。”方源轻喝一声。

太白云生连忙打开自家仙窍,八只力道巨手轻轻巧巧地将荡魂山,放进他的仙窍福地中去。

“好了,接下来就由地灵领着你,去将福地某些资源都收集起来。”方源道。

太白云生点头,当即便由小狐仙领着,倏地消失在方源的眼前。

方源落到地面上。

没有了荡魂山,福地中央显得空空荡荡。

他就地盘坐,缓缓闭上双眼,心念调动时运仙蛊,旋即在他的仙窍死地中,无数凡蛊也跟着飞腾而起。

一颗颗青提仙元消耗下去,以时运仙蛊为核心,周围凡蛊为辅,无数蛊虫交汇成一片白金色的光雾。

光雾升腾飞行,划过灰色的天空,飞了数里之后,来到一群蛊师俘虏的身边。

这些俘虏,正是东方一族。

之前被方源关押在地下石牢中,如今提取出来,暂时安放在自己的仙窍里。

光雾覆盖下去,这些俘虏们纷纷惶恐大叫:“这是什么东西?”

他们心中生出不妙之感,想要躲闪,但苦于受制,动弹不得。

白金光雾萦绕在他们的身边,由慢到快,不断旋转,最终形成一团白金色的光芒漩涡。

白金光芒漩涡酝酿半晌,忽然轰的一声,爆发出一道白金色的光柱。

仙道杀招时济运!

光柱从方源的仙窍中直透而出,一路向上,顺着脊椎,从方源的头顶上暴射,直没苍穹。

几个呼吸之后,光柱乍然消失。

与此同时,方源仙窍中的白金光漩也消失得干干净净。腐朽的地面上,横七竖八地躺着东方部族俘虏的尸体。

这些尸体,都各个干枯老朽,原本是以青年为主,但现在却是老迈得不成样子。

时济运,就是这样的效果。将外人的寿元抢夺过来,给蛊仙自己增添暂时的运道。

“我不甘!”

“你杀死了我们,凶手,我们化作厉鬼来缠你!”

“你为了一己之私,屠戮我等无辜生灵,你是罪人,你是凶手!”

“啊啊啊,吃了他,吃了他。”

忽然之间,方源的身体内出现了无数凶恶残魂,围绕着方源的魂魄纠缠,狠狠地扑杀过来。

方源冷笑,高大的仙僵躯壳安稳不动,宛若山岳。

体内的魂魄却是悍然反击,和残魂凶悍对撞。

这些魂魄都十分残破,来源于凡人,按理说并不值得一提,但此时此刻方源的魂魄却被这些残魂压入下风!

但幸好它们没有理智,不会相互配合。

方源驱动魂魄,集中攻杀,付出惨重代价,最终将这些残魂消灭干净。

大战之后,方源魂魄仅剩下原先三成大小,黯淡无光,恍若风中残烛,奄奄一息。伤势可谓极重。

但幸好他有胆识蛊。

此刻捏碎一只只的气囊蛊,里面的胆识蛊也跟着破碎,化作这天下最上乘的资粮,恢复壮大他的魂魄。

片刻之后,方源的魂魄完全恢复,重新回到巅峰状态。

比较之前,他的魂魄还多了一份凝实,宛若铁块经过锻打,质地更加紧密。

“这残魂反噬,虽是时济运杀招的重大弊端,但魂战结束,当魂魄恢复过来后,就可发现一些凝练魂魄的好处。”方源洞若观火,明察秋毫。(小说《蛊真人》将在官方微信平台上有更多新鲜内容哦,同时还有100抽奖大礼送给大家!现在就开启微信,点击右上方“+”号“添加朋友”,搜索公众号“qdread”并关注,速度抓紧啦!)

\end{this_body}

