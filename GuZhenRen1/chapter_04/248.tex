\newsection{葛辟福地难认主}    %第二百四十九节:葛辟福地难认主

\begin{this_body}

三天之后,在东海的蛊仙界中,到处流传起一个颇有绯色的小道消息。想看的书几乎都有啊,比一般的小说网站要稳定很多更新还快,全文字的没有广告。]

正是关于东海六大美人之一的亦诗仙子,如何在沐浴期间,被某个老淫.贼处心积虑,玷污纯洁的谣言。

宋亦诗是宋启元的掌上明珠,风头很盛,追求者众多。此事又事关宋家的名誉,消息传出,立即引得众多蛊仙的关注。

很快,星象子北原蛊仙的身份,就被蛊仙们调查出来。

在宋家还未正式表态之前,宋亦诗的追求者们,怀着愤怒之情,就已经发起了对方源的大规模搜捕。

一处无名无主的贫瘠海域中,海底深处,海波晦暗,鱼群稀少。

一座柴门也似的福地门户,陡然出现,旋即吱呀一声打开。

一位身影,从中迈出。

正是星象子模样的方源。

方源临阵斩杀了姚葛辟,事后,他便带着姚葛辟的尸身远遁。

对于蛊仙的尸首,有两种用途。一种是落地,形成福地。另一种则是放入自家仙窍里面,汲取蛊仙一身的道痕。

方源没有做过多的考虑,直接选择了前者。

他的仙窍是死的,要将尸体放入他的死窍中,不仅无法获取道痕,而且还会造成死地崩溃,整个身躯都被撑爆。

这种事情,早已经有仙僵试验过,可惜都没有成功的例子。

方源先利用智道手段,暂时镇压住仙窍,拖延了一点时间。彻底甩脱了追兵之后,便用星门首先回到了太白云生的仙窍中。随后。才用定仙游将自己挪移到东海最东北的偏远位置。

最后,他深入海底。在无人察觉的情况下,将姚葛辟的仙窍种在海底深处。

仙窍形成福地时,自然引发了地气、天气的波动震荡。

引来了两只鲸鱼荒兽,但都被方源打发走了。

方源先是变作另外一个样子,进入姚葛辟的福地,见到了地灵。

姚葛辟是木道蛊仙,地灵较为奇特,形如一株大树,栽种在福地中央。不能移动丝毫。

大树地灵告知方源它的认主条件十万块仙元石,还有大量珍稀的仙材,七转、八转的不计其数,甚至还要求元莲仙尊的鲜血!

这要求太高太难,方源当然达不到,只好遗憾地退出这片福地。

“姚葛辟是散修,在追兵当中是最没有背景,又来头不大,所以被我斩杀。算是杀鸡儆猴。果然让我震慑住了宋亦诗一行人,轻易逃走。”

“姚葛辟的仙窍形成福地,地灵认主的条件,却和我这个杀人凶手没有丝毫关系。要求的是大量的修行资粮。可见姚葛辟心中,最大的执念就是对财富的渴望。但他的这份渴望,也太过于贪得无厌了些。”

方源刚刚踏出。福地门户就倏地闭合。

他回望一眼,看到门户完全闭合之后。眉头不禁轻轻皱起。

“若姚葛辟的执念和我相关,凭我丰富的血道手段。我倒好处理了。可惜是这个认主条件,要达到标准比登天还难呐。”

不说多达十万块仙元石,就是那些珍稀的仙材,极其难以筹集。除非方源打劫中洲十大古派,或者在天庭里仓库里大捞一笔,否则单凭他自己现今的状态,尽全力筹集,还要在运气好的情况,两三百年的时间是最保守的估计了。

更不要说,大树地灵的认主要求还有最后一项元莲仙尊的鲜血!

元莲仙尊是木道九转蛊修至尊,姚葛辟也是木道蛊仙,觊觎元莲仙尊的鲜血也不奇怪。

但元莲仙尊是何等人物?就算他有鲜血,在元莲派保留下来,那也是货真价实的九转仙材啊!

这种能够炼制九转仙蛊的极珍之物,肯定被重重保护看守,方源怎么可能得手?

“这姚葛辟恐怕是生前穷疯了,对财富的执念居然如此强烈。这样看来,要得到这片福地,正规的认主途径已经是一条绝路,剩下来的就只有强攻了!”

方源眼中泛起冰冷的光芒。

这地灵要是不认主,那就打灭地灵,将这片福地强行占据!

不是所有的福地,都有地灵的。有的蛊仙被杀死得极其彻底,一点执念都没留下来,死后形成的福地就是无地灵福地。

姚葛辟的福地有地灵,认主条件太过于苛刻,方源只好生起歹意,计划强行攻打。就像鲨魔一行人攻略玉露福地一样。

“若是能打下这片福地,我在东海也就有了一片落脚之所了。不过现在却不是处理这片福地的时机,还是先去和鲨魔他们汇合罢。”

方源洒下一群凡蛊,在他身边结成蛊阵。

随后,他催动定仙游,离开这里,去往太白云生的仙窍里。

原先的海底,定仙游逸散的气息,在方源布置的蛊阵绞磨之下,彻底消散。

而完成了这项收尾工作后,这套凡蛊蛊阵也自行崩溃了。

除非是极为厉害的追踪侦察手段,否则谁也不能在这片平凡的海底,发现这么一处隐藏的福地。

方源从太白云生的仙窍中出来,后者早已经启程动身,正飞行在浩瀚汪洋之上。

“师兄弟”两个一路交流,半炷香之后,来到玉露福地的上空。

在这里,却是空无一人。

两人又等候了片刻,以鲨魔为首的僵盟蛊仙,就联袂飞来。

“星象兄,最近风头正劲啊。”一见到方源,鲨魔就打趣道。

其余的蛊仙,看着方源也是目光奇异。

这三天来,方源可谓在东海蛊仙界,大出风头,已经薄有名声了。

方源报以苦笑。回道:“时运不济,时运不济啊。见笑了诸位。本来想去寻一落脚海岛,结果发现一道海底潜流。就想探寻一下,结果碰到了这么一桩烂事。”

“这可不算烂事,说是风流韵事,这才恰当啊。”卜单哈哈大笑,嘲讽道。

之前,方源好生教训了他一通。

但事后,卜单大出血,送了苏白曼重礼,这次继续参加福地攻略。态度上有些有恃无恐的样子。

方源看他一眼,笑了一声,没和这种人多计较。

他目光扫视,除了鲨魔、苏白曼、卜单之外,就只有一位熟面孔的仙僵沙南江,人数方面比前一次要少了三位。

但鲨魔却是自信满满的样子,显然来之前,他做了充分的准备。

众人深入海底,打开玉露福地门户。再次落入战场杀招冰雨冻土之中。

“这一次攻略,主要还是请星象子你出力,卜单他就作为辅助。”鲨魔比较客气地指示道。

上一场,方源表现十分出色。让鲨魔等人都认为星象子的智道造诣更高一筹。

“鲨魔大人放心,在下必定竭尽全力。”方源立即表示了坚定的态度。

卜单心中冷哼一声,尽管十分不满。但也有自知之明,当下抱臂旁观。目光阴郁。

再一次面对冰雨冻土,方源脸色严肃下来。开始推算。

冰雨冻土战场杀招,已经全部恢复,之前方源等人造成的破坏,都被一一修补完善。

方源并不意外。

之前,鲨魔等人攻略福地时,战场杀招都有自行修复的现象发生。

就是不知道这种现象,是玉露福地的地灵施为,还是玉露仙子布置的手段。

说起来,鲨魔等人攻略了福地这么久,一直都没有发现过玉露福地地灵。地灵究竟存在与否,还是个未解之谜。

因为有上一次的经验,方源进展颇快。

他自然没有将那只关键凡蛊毁灭,所以当战场攻势第一次发起时,威力没有丝毫增长。

冰针暴雨倾盆而下,鲨魔挺身而出,催动仙蛊。

仙蛊名为冰消,乃是冰道仙蛊,却能令坚冰消融化解。这只蛊就是鲨魔等人,专门为了冰雨冻土战场杀招,而耗费不菲,向僵盟中的某位成员特意借来的。

有了这只冰消仙蛊,果然极为克制冰雨冻土的攻势,就连荒兽级的雪怪,也遭受巨大压制。

鲨魔等人大感轻松,消耗的仙元也比之前减少了好几个档次。

卤水点豆腐,一物降一物。强大的冰雨冻土,碰到区区一只冰消仙蛊,却威力暴降。

这个蛊师世界,就是这样。

万物都是平衡的,强大、弱小都是相对而言。从未有无敌的蛊虫,只有无敌的蛊师。

有了冰消仙蛊在手,鲨魔等人就显得游刃有余了很多。

方源也是稳步推进。

他虽然是智道新手,但上一次有过宝贵经验,回去之后,也多加温故,自觉提升了不少,也设想了许多手段。

这一次,方源是有准备而来。

不得不说,东方长凡的这份智道传承优秀无比,方源靠着它,硬生生地打开局面。

万事开头难,打开了局面之后,方源的进展就越来越快,推算出来的蛊虫隐藏点越来越多。

不过方源也在这个过程中,充分感受到经验不足的缺点。

好在他是个老谋深算的人,这个时候就命令卜单出手。

卜单虽然智道手段浅显又稀少,但方源打开局面之后,他认清战场杀招的部分内容,拆除的手法熟练而又准确。

方源假意推算,暗中观察卜单的手法,学到了许多实用的经验,智道造诣飞速拔升。

最终结合二人之力,终于将冰雨冻土战场杀招彻底破解。

至此,冰雨战场总共发动了三十三次攻势,全赖鲨魔掌握冰消仙蛊,撑住场面。

仙元的巨额消耗,很大程度上冲淡了破解成功的喜悦。(未完待续……)

\end{this_body}

