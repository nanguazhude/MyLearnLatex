\newsection{旷世赌斗}    %第三百一十三节:旷世赌斗

\begin{this_body}

“哈哈哈,终于大功告成了!”

从地下石窟出来,方源的脸上充满了喜悦之色。

这段时间内,他全力以赴,一门心思的推算仙道杀招,终于在今日得成正果。

这道全新的杀招,是以吃力仙蛊、拔山仙蛊、挽澜仙蛊为核心,辅助三十三万多只凡蛊,形成一套繁芜至极,令人眼花缭乱的搭配。

要催动这个杀招,方源必须要一心一意,并且至少得持续一个时辰,才能将整套蛊虫都催动起来。

然后三天三夜不休,才能将自身一个大活人逐渐转变成仙僵,或是从仙僵转变活人。

虽然一点都不如涅槃火方便,但方源做到这样一步,已经达到了极限。

方源对这个结果已经很满意了。

毕竟,方源身上的力道仙蛊就这么几只。能够将它们用上,多亏了智慧光晕。

当然,还有涅槃火为参照版本。

若是没有涅槃火,没有焚天魔女借来的这一套蛊虫供方源不断试手,方源绝不会这么快,就能得到力道版本的成果。

还有方源的智道宗师境界。

正是各方各面的原因,才造成了这一场不大不小的奇迹。

“接下来,就是利用那些力道仙僵,还有奋力蛊,成就上等生死福地,重获新生了。”方源心中感慨万千。

辛苦追寻了那么久,终于到了这一刻了。

至于落魄谷,太白云生已经在不久前,修复了它。

方源也试过一次,去进入落魄谷修行。

果然不愧是和荡魂山齐名的魂修圣地!

搭配荡魂山胆识蛊,方源的魂魄底蕴可谓一日千里。

不过现在,方源知道最紧要的,还是重获新生。魂道修行可以先放置一边,不去管它。

方源已经有些迫不及待。

然而,就在他正准备开动的时候。忽然接到焚天魔女的来信。

心中的内容,让他近在眼前的重生大计戛然而止。

“八转大力真武仙僵?”当方源看到这一行字眼的时候,心脏随之砰然而动。

焚天魔女给他的情报,正是关于南疆的那座无名山峰。

这件事情实在太大了。事关仙蛊屋,造成的影响已经迅速波及整个南疆的蛊仙界。

南疆当中,自然也有僵盟分部。

因此,焚天魔女作为北原僵盟分部首领,很快就得到了相关的情报。

“焚天魔女野心勃勃。她知道我有定仙游,就关注着五域的风吹草动。而她身为僵盟的高层,情报收集的能力,比黎山仙子还要强。”

方源再次感到自己在情报方面的弱势。

前世的时候,他有组织,有人脉,有成熟的血道蛊虫搭配,可以替代大部分常用的信道手段。

今生重生,方源则主要借助自己前世的记忆,还有和他人联合。借助黎山仙子、琅琊地灵等等的情报共享。

但随着越来越多的意外发生,方源已经逐渐察觉到,他的前世记忆并不那么可靠。

历史的真相,往往隐藏在迷雾深处,而方源前世看到的,却大多是浮于表面的假象。

就算他亲身经历的事情,也未必像他料想中的那样简单。

很多事情的形成原因,是很复杂的。

“前世我命途坎坷,一步步慢慢向上爬,人脉、资源都是点滴积累上来。所以并不存在短板。但重生之后,却因为屡屡得到机缘,实力飞速增长,乃至于跳跃式的暴涨。所以很多方面。就跟不上来了。”

“看来我今后还要增加一些信道手段,增加收集情报的能力。短时间内可以依靠焚天魔女这些外人,但长久之计,还是要靠自己!”

方源沉思着。

失去情报,往往就意味着就丧失了先机。

就像这一次,方源还不知情。都被蒙在鼓里,险些和这场机缘失之交臂。

三天之后,方源孤身一人,回到南疆,来到无名山峰。

焚天魔女还在阴流巨城,她在信中告诉方源,她忙于炼蛊,布置蛊仙大阵,脱不开身。但必要的时候,她定会出手。

而黑楼兰、黎山仙子则在北原,利用落魄谷修魂,增长魂魄的底蕴。她们也对仙蛊屋抱有强烈的兴趣,但是北原蛊仙的气息,让她们难以行走在南疆。

方源此次过来,主要是进一步打探情报。

他虽然是北原成仙,但有见面似相识杀招遮掩,寻常蛊仙辨认不出。

无名山峰上,每隔一段时间,幻影就会重现。

而白凝冰已经不知所踪。

方源观看了整整三遍,沉吟片刻,便试着动身,接近无名山峰。

果然,就像焚天魔女信中所说的那样,这里成了禁仙绝境。方源越是深入其中,越发四肢无力,整个仙窍似被一股无形的重压排挤。

方源不得不停下脚步,他知道再这样下去,走不了五六十步,他的仙窍就会因此彻底损毁了。

这当然不行。

他还指望着力道仙窍,能够形成生死仙窍呢。

方源只好后退。

这时,在他身后,传来一阵娇笑之声。

方源瞳孔微缩,立即转身望去。

只见一位美娇娘,一声粉红衣衫,青丝绾成发髻,肌肤娇嫩似雪,双眼媚如春水,俏生生地站在不远处,盯着方源。

方源不敢大意。

这位女子浑身洋溢着蛊仙气息,赫然是一位六转蛊仙。

见方源看来,这位美娇娘笑着自我介绍道:“小女子李梅花,人称梅花婆婆。这位小郎君,好面生,不知道在哪里修行呢?”

方源笑了笑,心道:“原来她就是梅花婆婆。”

方源虽然没见过梅花婆婆的真容,但见过她的孙女儿,便是那魔道女蛊师狐魅儿。

爱美是女人的天性。

蛊仙的年龄和相貌之间,并没有固定的联系。

此时的方源,一身青袍,袖子又宽又大,风轻轻一吹,青袍鼓荡起来,宛若战旗猎猎作响。

方源伪装的形象,六转蛊仙气息,身材高瘦,中年模样,鼻梁高挺,一双眼睛十分细长,瞳孔转动间散发着丝丝碧芒。浑身阴气散发,一看就不好惹。虽不俊俏,却也别有一股风采。

“原来是梅花婆婆,久仰久仰。在下盛鹰,不过是一位山野村夫。”方源回答道。

“盛鹰……”李梅花将这名字记在心上,搜刮脑海也没记得起南疆还有这一号人物。

不过,她也不奇怪。

听方源介绍,就知道他是一位散修。

南疆多山,龙蛇潜藏。有很多的散仙,都不露面,不为人所知。

但是因为仙蛊屋的出现,吸引了一**的蛊仙,陆续登场。

方源假扮的盛鹰,不过是其中之一。

这些天来,李梅花联络的蛊仙,大多都类似于方源。

李梅花主动找自己搭讪,并且态度热情,这让方源暗生疑惑。

方源正要直接开口,李梅花却主动道出缘由。

方源这才恍然大悟。

李梅花接着邀请方源同行,方源沉吟片刻,便点头答应下来。

他跟着李梅花,一道出发,在千里之外的无头山上落脚。方源到的时候,此山上已经有不少魔道或者散仙。

见到方源,大多数人投来好奇的目光,有的则显得阴沉或凶狠。

因为摸不清方源的根底,暂时还没有蛊仙和方源搭话。

倒是梅花婆婆人缘相当不错,刚一落下,就有人笑着道:“李梅花,又带来一个啊?”

“哈哈哈,这一次我们魔散联合,和正道谈判,梅花婆婆立的功劳很大。”

“哪里,哪里。各位盛赞了,小女子也是稍尽绵薄之力罢了。”李梅花笑着四处打招呼,纷纷搭话,游刃有余地处理着各方面的关系。

“给大家介绍一下,这位兄台姓盛名鹰,是一位散仙。”应付好了,李梅花不忘向大家介绍方源。

“原来是盛兄。”当即,就有魔道蛊仙拱手。

“在下重禾子。”

“我观阁下似乎修行的变化道,哦,鄙人姓蓝,名天鸿,也是一位散修。”

……

方源表现出一副不善交际,勉强应付的样子。

短暂的热络之后,无头山上又恢复了方源来之前的平静。

李梅花并未在山上久留,她还要去无名山峰处留守,临行前她手指着对面一座山,对方源道:“那座松尾山,就是正道蛊仙的营地。”

其实不需要她指点,对面山峰上外溢的蛊仙气息,已经告诉了方源。

和李梅花分别之后,方源就在无头山暂居。

他鲜少出走,大多数时间都缩在山洞之中。毕竟方源是伪装的身份,见面似相识也不是无敌的。

他耐心地守候了十多天,期间无头山、松尾山上陆续迎来正魔两道的蛊仙。

大多数都是六转,每一位七转蛊仙的到来,都会引发一波轰动,最终,就连八转蛊仙也来了四位!

一位魔道,一位散修,两位正道。

刚好形成均势。

八转蛊仙并不露面,双方蛊仙每天各推举一人,进行艰难谈判。

谈判进行了七天七夜,终于大功告成。

正魔两道,数十位蛊仙统一定下契约,围绕无名山峰,进行一场千年难得一见的大赌斗!

赌斗中的胜者,将得到无名山峰下的机缘。其余蛊仙,在赌斗后三年之内,都不得向胜者出手。(未完待续。)

\end{this_body}

