\newsection{仙蛊变形}    %第一百九十四节:仙蛊变形

\begin{this_body}

方源不是已经和仙鹤门缔结了盟约,怎么敢向仙鹤门中之人动手?

方正比同行的长老、弟子们知道更多的内幕,当初鹤风扬和苍郁联手攻打狐仙福地失败,因为胆识蛊的利益达成共识,故而双方签订了盟约。

盟约中自然有一条内容,规定双方势力成员之间,不得互相攻伐。这也是方正自觉安全的原因之一。

方源缓缓收手,方正动弹不得,叫喊不得,没有一丝反抗之力,就被一股无形的力道提起来,主动送到方源的手上。

仙鹤门一行人看得目瞪口呆。

“好恐怖的实力!”

“方正长老是五转修为,居然没有任何的反抗之力,就这样被拿下了。”

“方正虽然是奴道蛊师,但也不至于这样吧?”

方源展现出难以置信的手段,一下子震慑住这些quled人。

方正内心大吼:“你们这群人站着看戏吗?快来救救我呀,他早已经不是我哥哥,他杀了舅父舅母,杀了全族的族人,他是魔头,他是个大魔头啊!”

与此同时,方正口中则“高喊”:“我不去你那里,我为什么要听你的?你又想管教我?我的人生我自己做主!我有我自己的目标,我的梦想你怎么能懂?我自己可以奋斗,我能超越你!”

活脱脱一个不服管束的叛逆少年。

方正听完自己的高喊,简直是心若死灰!

方源叹了一口气,对仙鹤门众人道:“这段时间。我这个不成器的弟弟,给你们添麻烦了。”

“不麻烦。不麻烦。”仙鹤门一群人连忙摆手道。

方源点点头,又道:“我将方正带到我身边。好好管教一段时间。这个事情我已经向掌门打过招呼了。你们回去通传一声便可。”

方源谎话连篇,掌门长什么样,方源都不知道。

两位长老,却是放下了最后一层犹豫。

方源的事情,仙鹤门为了维护门派名声,一直都在编织谎言。结果现在,方源利用这个谎言,将仙鹤门这群人骗的晕头转向。

方源一手提着方正,就这样施施然走了。

路旁的蛊师们看到这一幕。都很诧异。但既然人家仙鹤门都没说什么,他们又怎会去动手?并且动手的对象,又是众人熟知的炼道强者方源呢?

有驱邪派的蛊师,不放心,过去询问。

炎堂长老笑着解释:“不碍事,方源大人和方正长老是亲兄弟。现在是哥哥管教弟弟呢。”

驱邪派蛊师信服而去。

仙鹤门一群人一直目送着方源离开,很多弟子都对方正投去羡慕的目光。

有人有感而发地道:“方源大人虽然口气冰冷,实际上对自己的唯一的亲弟弟还是关爱有加的。”

“没错,有这样的炼道宗师悉心栽培。这是多么千载难逢的机遇啊。”

“你们说,方正上一次接受门派任务,回来之后就是成了五转长老。这一次被哥哥带走,回来之后会不会就成为炼道宗师了?”

“不至于吧。”

“这也太夸张了!”

“尽管如此,有这样的强者指点一下,绝对是受益无穷的啊。”

“唉。我怎么就没有这样的哥哥呢?”一位弟子的一句话,说到了众人心坎里去了。

如果此时他们知道。方源擒拿方正,不过是为了夺舍做个有备无患的准备。不知道会是什么样的精彩表情。

啪。

方源将方正扔到地上。

方正挣扎欲起,心中发出怒吼:“方源你这个魔头,你究竟想把我怎么样?!你和仙鹤门定下盟约,是不能杀我这个仙鹤门长老的。”

方源哈哈一笑:“我又没杀你,你激动什么。我的好弟弟,你就乖乖地待在这里罢。”

方正脸现惊恐之色,心中叫道:“方源不是我师傅天鹤上人,他居然也能知道我内心的话?”

方源傲然一笑,俯视着脚边的方正:“你那所谓的师傅,如今不过只是一个魂魄罢了。就算他生还,全盛时期也不过是五转蛊师。”

方源说着,手掌凭空一抓,就将方正浑身上下的所有凡蛊,全都摄取过来。

寄魂蚤在微弱的震颤,这是天鹤上人极力反抗。

但怎么反抗得起来?

若是在青茅山时,一百个方源都不是天鹤上人的对手。但如今,方源已经今非昔比,成了六转蛊仙。而天鹤上人却沦为一个魂魄。

“方源,你就算不杀我们。将我们擒拿俘虏,也犯了盟约的规定,你必会受到盟约的剧烈反噬!我劝你还是把我们放了,你堂堂的蛊仙,怎么和我们这等凡人为难?你就不怕这样做,恶了你和我仙鹤门的关系吗?别以为仙鹤门收拾不了你!中洲十大古派的底蕴,不是你能想象的。”天鹤上人在寄魂蚤中叫喊,他知道方源一定听得到。

方源仰头大笑:“哈哈哈,天鹤上人,我还要谢谢你。没有你攻打青茅山,我也不会因缘巧合,走到今天这一步。我抓了你们,仙鹤门又能如何?你既然知道不少秘辛,更应该明白中洲十大古派相互掣肘,互相牵制,一切都以利益说话。你说独一无二的胆识蛊和你们两个,孰轻孰重?”

天鹤上人哑然无语。

方源眼中阴芒一闪,调动无数蛊虫,转眼间形成魂道杀招:“现在就将你所知道的东西,都全部吐露出来吧。搜魂!”

一个时辰之后,方源走出这座地牢。身后的牢门,轰的一声,重重关闭。

方源一对赤眸,闪烁着光,口中喃喃:“天鹤上人,古月一代,血道真传……”

通过搜魂,他对天鹤上人和古月一代的恩怨,全盘得知。并且还知道了不少仙鹤门的隐秘。

“这个天鹤上人,是鹤风扬的得力下属,我抓了他,鹤风扬会不会过来要人?这倒有些麻烦。”

“呵呵,居然不想夺舍,想要成全方正。这个天鹤上人倒是有所坚持的人。”对此,方源不吝赞叹。

他虽然是魔头巨擘,但对于这种舍己为人,坚持自身原则的事情,还是可以用可观的角度,去看待和欣赏的。

一样米,养百样人。不同人活着,有不同的坚持。

对这种价值观不同的人,方源完全能够接受,甚至欣赏。

如果没有这等胸襟,他也不会有如今的这般成就。

“这样推测的话,残阳老君也将从东方长凡那里,得到的夺舍法门,传回了仙鹤门。”

天鹤上人的记忆,对方源而言,帮助并不大。

他只是凡人蛊师,涉及到的仙鹤门隐秘十分肤浅,方源能利用的也少之又少。

至于方正,方源当然也没放过,将其魂魄彻底搜索了一遍。

对于这个亲弟弟,方源打算一直关押。若是哪天真的不得已了,方源就只好出下策,夺舍方正,重新复活。

不过这样一来,他的第二空窍就丢了,浑身的力道道痕也消散。需要重新渡劫升仙。

损失很大。

第二空窍仙蛊,是动用了许多珍稀仙材炼制而成的。其中就包含寿蛊!

可以说,第二空窍仙蛊方源短时间内,是炼制不出来的。

寿蛊极其难寻,就算寻到了寿蛊,按照方源如今的情势,恐怕也是直接用了,增长自身的寿元。

“如果有可能的话,还是尽量不要夺舍方正的好。不过若是身份暴露,而自己还未重获新生,那就只能夺舍了。现在的目标,还是炼蛊大会。进入前六之后,便可受益于不败传承,得到一次炼蛊必成的良机!”

不败传承,是炼道传承。何谓不败?就是炼蛊绝不失败!

一旦获得前六位,就可被传送到一处隐秘福地。方源打算就在那里炼制仙蛊变形!

这是变化道的六转仙蛊。一旦拥有了它,方源的见面似相识就能有飞跃的提升,再不是现在只能遮掩北原地域气息的蛋疼版本。

至于变形仙蛊的仙蛊方,方源早就到手了。

这事情,还得感谢琅琊地灵。

在很久之前,方源刚刚得到智慧蛊时,就和琅琊地灵达成交易,帮助琅琊地灵完善仙蛊方。

方源因此获利,不仅摆脱了经济破产的困境,而且还获得了大量的仙蛊方。

而这变形仙蛊的仙蛊方,恰恰就是其中之一。

至于炼制变形仙蛊的仙材,方源也准备的差不多了。

其中将近一半的仙材,都是从北原拍卖大会中得来。缺少的另外一半,方源则请黑楼兰出手,利用她的福地沟通宝黄天。再用方源剩下来的其他仙材,或卖或换,进行多次交易,陆续筹集了不少。

只剩下的那一小份缺口,在方源第八场比试之后,也被得到的奖励仙材而弥补。

接下来的几场比试中,方源再三主动出击,利用重生的优势,将这些人一一击败。

一时间,方源风头无两,大大小小的势力,无数蛊仙都将目光集中在他的身上。

其他人都中规中矩,盘踞在自己的地盘上,进行安全的晋级。

唯有方源一人,独树一帜,特立独行,四处征伐,还偏偏每一场都能胜利!

\end{this_body}

