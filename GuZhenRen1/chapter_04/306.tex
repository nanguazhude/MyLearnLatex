\newsection{剑纵中洲}    %第三百零七节:剑纵中洲

\begin{this_body}

------------

大地已陷入沉眠。<strong>txt全集下载wWw.80txt.COM</strong>

月光朦胧,树影婆娑。

微风时而轻拂,山林中的鸟鸣,或者偶尔的野兽呼啸,仿佛是这片连绵山峦的梦呓。

山谷中的小屋,油灯燃烧着,光线暗淡。

凤金煌躺在床上,浑身虚弱无力,脸色惨白,没有一丝血色。

她望着坐在床边的白晴仙子,低声道歉道:“娘,我错了,我下次不会……”

白晴仙子一脸严肃:“煌儿,娘知道你心里想的是什么。你这么拼命的修行,这数十天来,已经将自己伤了五六次。娘在你修行之初,就告诫过你,修行讲究自然,讲究张弛有度。你这样下去,不仅欲速则不达,而且还会把自己搞垮,甚至有性命之忧!”

凤金煌垂下眼帘,声音低弱:“娘,对不起。”

“你身上已经被娘下了蛊,罚你七天七夜不得修行,活动范围只能在这座山谷里。饿了你就去采摘野果,渴了你就去喝山泉。你好好想想罢。”白晴仙子沉声道。

“娘,不要!”凤金煌大急。

但白晴仙子轻拂长袖,袖口划过凤金煌的脸颊。

顿时,一股猛烈的睡意,袭上心头。

眼皮子前所未有的沉重,无以伦比的疲惫之感,让凤金煌再说不出话来。

眨眼间,她便陷入了沉睡当中。

看着女儿虽然熟睡,却仍旧微微皱着眉头。白晴仙子心中既悲怜又忧愁。

凤九歌的死讯,白晴仙子还没有告知凤金煌。

一来,灵缘斋需要隐瞒死讯,尽量拖延时间,好做出部署。身为凡人的凤金煌。还没有资格知道。毕竟刺探凡人的情报,可比刺探蛊仙的情报,要容易许多。

二来,白晴仙子不知道该如何去说。

凤金煌为什么会这么拼命地修行?她正是想努力提高自己,让自己变得更强大,好去寻找父亲,救援父亲。

但白晴仙子又如何能将真相。残忍地告知凤金煌呢?

她不忍心。

“唉……”看着女儿的面庞。白晴仙子发出一声长长的叹息。

往日里她挺拔的身姿,高洁的白袍,此刻在昏暗的光线中,却显得瘦削虚弱,光泽黯哑。

这些天,她真的很累。

直到此刻,她才卸下伪装。真实的情感外露出来。

白晴仙子伸出手,轻柔地抚摸着凤金煌的脸颊。她目光中的慈爱,能将钢铁融化。

凤金煌的容颜,汲取了凤九歌和白晴仙子的优点。

在她的脸上,白晴仙子依稀能看到凤九歌的影子。

凤九歌已去,凤金煌就成了她唯一的人生寄托。

就这样无声地注视着,过了好一会儿,屋外天边已经半亮。<strong>小说txt下载HtTp://Www.80txt.Com/</strong>

黎明的光,让白晴仙子意识到时间的流逝,她必须启程了。

她慢慢地站起身。缓步走出屋外,将房门都小心关好。

随后,她深深地望了一眼山谷小屋,自言自语:“煌儿,等娘七天后回来,这个期间你要好好保重自己,好好平静你的心。”

白晴仙子并不担心凤金煌的安全。

毕竟这里是灵缘斋的腹地。

门派就算内斗。也有限度,不会对凤金煌出手的。

白晴仙子脚踏云雾,一路西北方向飞去。

她此行的目标,直指落天河的源头。

为什么要去那里?

原来,她接受门派任务,调查薄青这个线索。

薄青本就是灵缘斋的蛊仙,门派中有大量资料,供白晴仙子查询。

调查中,白晴仙子发现薄青的经历,其实和凤九歌极为相似。更叫她感兴趣的,是薄青的仙侣墨瑶。

墨瑶是那一代灵缘斋的仙子,和历代仙子不同,她是一名墨人,并非纯粹的人族。

但就是这样,薄青和墨瑶却是一见钟情。

“薄青虽然没有进出北原,但墨瑶却出入北原,甚至进入过王庭福地。难道说,八十八角真阳楼倒塌一案,和墨瑶有牵连?”

白晴仙子知道,墨瑶之所以冒险进入王庭福地,是为了帮助夫君薄青渡劫。

当年,八转巅峰的薄青渡劫,冲击九转境界。这件大事,不仅是在中洲,更在其他四域,引起广泛的关注。

可惜最后,薄青身亡,陨落在恐怖的灾劫之中。而墨瑶也伴随着他,一同陨落。

这个事情,其实中洲蛊仙都知道。

白晴仙子早年也有多次耳闻。

但现在,她回想起来,却是别有滋味,大有同命相连之感。

她甚至羡慕墨瑶,就算是死,也和自家的夫君一齐死。谁也没有辜负了谁,可谓伉俪情深,死得其所。

白晴仙子连续调查,殚精竭虑,进展并不大。她在门派中的典籍里发现了一些线索,都指向同一个地方落天河源头。

在其他方面难有突破的时候,白晴仙子便毅然决定去落天河亲自探索。

落天河的源头,就是当年薄青陨落之地。

灵缘斋距离落天河源头,有很长的一段距离。

白晴仙子若是单凭自身手段,飞过去至少要数年光阴。

她当然不会这么做,在临行前她已经做了充分的准备。

她一路疾飞,途中不断利用灵缘斋的蛊阵,进行传送,单一次传送就跨越数十万里的距离,大大节省了时间。

她并没有直取落天河源头方向,而是先转向万龙坞。

在万龙坞势力范围中,她借用对方的传送蛊阵进行赶路。这一点,她早就和万龙坞的蛊仙沟通好了。虽然付出代价不小。但毕竟节省了大量的时间。

一路顺着落天河而上,过了万龙坞的地盘,白晴仙子踏入战仙宗的势力范围。

同样的,她借助传送蛊阵,进行赶路。

一天一夜之后。风尘仆仆的白晴仙子终于距离落天河源头,只有千里之遥。

举目眺望,一道巨大的瀑布,从万丈高空垂落而下。

亿万顷的河水,势大力沉地砸在地表的河面上。

轰隆隆的水流声响,宛若雷霆连绵。

庞大的水汽,形成浓雾。笼罩方圆数百里。

何谓落天河?

这便是落天河。一道从天上落下的巨河。

和这道空中的巨河相比,白晴仙子渺小如蚁。

白晴仙子早年时候,和凤九歌结伴游历,也见过落天河。

这一次望着这条巨河,不免就想起了凤九歌,心生酸楚。

她还记得,当时凤九歌和她的谈话。

是关于落天河的形成原因。

凤九歌侃侃而谈地告诉白晴仙子。这里面有两种说法。

一说是落天河本就是灾劫本身,二说是薄青抵挡灾劫,剑光犀利至极,一下子用力过猛,洞穿了白天和黑天。

而当时,薄青渡劫是在白日里。

所以这道剑光,将白天洞穿了一个大洞。而又贯穿黑天后,只在黑天底部形成一个小洞。

正好这两个洞上,是浩瀚的天河。

天河便顺着这两个洞口,一齐灌下。砸在中洲的土地上,形成一片内陆海洋,同时河水奔腾开道,一路形成横贯中洲大陆的第一长河。

白晴仙子便问,这两种说法谁更靠谱一点?

凤九歌笑着答道:“我也不太清楚。不过第二种说法,却能解释水流的变化。在白天时候,落天河水势强大。八转蛊仙都无法抵御。而在黑夜,落天河水流缓慢下来,七转蛊仙可以勉强抵挡。”

自然之威,浩荡绝伦。就算是蛊仙,与之相比,也多显得人力渺小。

收拾情怀,白晴仙子目光 一定。

她此行,已经借来不少水道仙蛊,都是为了帮助她进入落天河。

但白天显然是不可能的,白晴仙子是七转蛊仙,唯一的机会就是在黑夜里,进入落天河探寻线索。

此时还是白天,落天河近在咫尺,白晴仙子便放缓速度,慢慢接近。

她打算等到夜幕降临,再深入河底。

时间流逝,太阳逐渐西下,天边火烧云层层叠叠,晚霞的光辉照耀在白晴仙子的脸上。

忽然,她猛地睁开双眼,惊疑不定地盯着远处落天河。

吼!

一声咆哮,水流喷发,一头巨大的猛兽,探出脑袋,在河面上露出冰山一角。

这是一颗巨大的牛头,牛角弯弯绕绕。

“太古荒兽万目大明牛!”白晴仙子认出这头猛兽的跟脚,不禁脸色一白。

落天河中并不安全。

除了水势磅礴之外,里面还生存着大量的上古荒兽,乃至太古荒兽都有。

万目大明牛,就是太古荒兽,生活在落天河河底的霸主之一。

“它通常不是在河底,脚踏泥沙,四处巡视地盘的吗?怎么会突然跑到水面上来?”白晴仙子心中疑惑,同时身形飞退,和这头万目大明牛保持距离。

太古荒兽,是八转战力。

白晴仙子可打不过这头巨牛。

况且她来此的主要目的,也不是为了狩猎。

白晴仙子满脸的谨慎,哀叹运气不好。万目大明牛的出现,让她的心里蒙上了一层阴影。

就在她要继续拉开和万目大明牛的距离时,一道剑光陡然从落天河底飞出。

宛若一道霹雳雷光,贯穿万目大明牛。

这头皮糙肉厚的太古荒兽,在刹那间,被剑光斩成两半。

无数的鲜血、内脏,随着喷涌而出,将附近一片河面都染成血红之色。

嗖!

还不待白晴仙子反应过来,又是一道剑光,飞射出来,斩在河边的堤岸上。

下一刻,坚硬的大地就像是脆弱的豆腐,被一刀劈开长长的口子。大量的河水争先恐地涌入进去。

河水漫溢出来,将方圆一里之地,都笼罩住,一片汪泽。

白晴仙子骇然不已。

怎么回事?

这到底是什么情况?

剑光如此威能,简直恐怖到了极点。太古荒兽在剑光面前,简直是比砍瓜切菜还容易。

嗖嗖嗖!

无数的剑光,从落天河的河底暴射而出。

像是放烟火一般,剑光四处散射,脱离落天河,然后速度极快地消失在白晴仙子的视野尽头。

剑纵中洲!

ps:来不及了,啊啊啊啊,今天本来要两更,但妻子不在家,我要带小孩。所以只能一更了。拖欠的一更,明天补上。(未完待续。)<!--80txt.com-ouoou-->

\end{this_body}

