\newsection{百日大战落幕}    %第二百七十四节:百日大战落幕

\begin{this_body}

%1
听到焚天魔女如此强势霸道地征召自己,方源丝毫没有生气。

%2
相反,他还有些欣喜。

%3
鲨魔、苏白曼夫妇,辛辛苦苦地攻略玉露福地,眼看着要有成果,结果被焚天魔女强行扣下,实在可怜。

%4
鲨魔、苏白曼此次付出了这么多,绝大多数的身家,都赔了进去。

%5
这次被焚天魔女摘了桃子,可谓伤筋动骨,血本无归。

%6
五百年前世,鲨魔、苏白曼陨落,鲨海被他人侵占,也许就有这一层关系?

%7
但方源并不反感,焚天魔女的行径。也丝毫不同情鲨魔、苏白曼的遭遇。

%8
若有焚天魔女的修为,换做他来,方源兴许还会更加过分。

%9
大鱼吃小鱼,这本身就是修行界的常态,赤裸裸、血淋淋、冰冷残酷。

%10
但其实但凡年龄大些,阅历多些的,都会明白这种常态,可以覆盖到大自然的每一个角落。

%11
哪怕是方源穿越前,在地球上的人类社会,也是如此。

%12
涉及利益,同室操戈,兄弟反目,这种情况太多了。

%13
只是多披了一张皮,显得冠冕堂皇。事实上表面道貌岸然,背地里龌龊不堪的人,比比皆是。

%14
但这并非过错。

%15
生存和繁殖,是任何生命的天性。

%16
只是焚天魔女强插这一手,给方源原来的计划,平添了巨大的变数。

%17
“焚天魔女修为高深,精明能算,比鲨魔、苏白曼要难对付百倍。但她家大业大。兴许能资助我炼出星念仙蛊?还有一桩好处:接近她,似乎更有机会让我重返北原僵盟。深入地沟,获取那份宝藏。”

%18
一个呼吸的时间。方源已经考虑得清清楚楚。

%19
焚天魔女的话音刚落,方源就旋即拱手,苦笑道:“焚天魔女大人有命,在下不敢不从。只是在下有苦衷。方才和鲨魔大人也提及过,要想破解了最后一层战场杀招,非得炼出一只六转智道仙蛊才可。”

%20
鲨魔冷哼一声。

%21
方源投靠的速度,也未免太快了点。

%22
苏白曼更是暗恨:看这星象子平日风度翩翩,道貌岸然,不想却是个软骨头。卑躬屈膝,风度丧尽,一丁点的节操都没有!

%23
方源哪管他们俩的想法,他们已经出局了。反而半途加入的方源,留了下来。

%24
见到方源直勾勾的眼神望来,焚天魔女摩挲着下巴,露出意味深长的笑容:“要炼制仙蛊吗……呵呵呵,这好办得很。我资助你就是了!”

%25
北原,落魄谷。

%26
又一场惊天动地的大激战。徐徐落下帷幕。

%27
作为战场的落魄谷,一片狼藉,碎石满地,像是地震之后。又被飓风肆虐。

%28
来自中洲的一行蛊仙,伤势不一,此刻都将目光集中在前面。凤九歌的身上。

%29
崇拜、忌惮、惊疑、深沉、凝重,复杂的情怀。不一而足。

%30
“凤九歌又胜了!”

%31
“他的对手秦百胜,纵然强悍。也难以抵挡凤九歌的凛冽攻势。”

%32
“这一场,已经是连胜第七场。秦百胜被凤九歌彻底压入下风。”

%33
“百日大战呐,足足百日,终于见到结局了。”

%34
“实话实说,秦百胜极为强大,可惜却碰到了更胜一筹的凤九歌!”

%35
算算时间,从中洲一行人算到落魄谷,并展开进攻,结果遭到影宗秦百胜等人的顽强防守,时间已经过去了三个月余。

%36
百日大战!

%37
这是货真价实的百日大战。

%38
通常情况下,蛊仙交战,很少有这么长的时间。

%39
再论规模,涵盖中洲、北原十多位蛊仙,尽皆精英高手。可以说,整个五域上千年来,都没有发生过这样的激战。

%40
若是公布出去,必定引起天下哗然,风云激荡,无数瞩目。

%41
但可惜的是,不管是秘密调查的中洲蛊仙,还是见不得光的影宗一方,都不愿意声张暴露。

%42
于是这场本该震动五域的大战,至始至终,都悄无声息地默默进行着。

%43
对于这一点,交战双方有着一致的默契。

%44
百日大战以来,关键人物至始至终,都在凤九歌和秦百胜的身上。

%45
起初两人打成平手,场面僵持。占据地利,施展了战场杀招亿万屠杀场的秦百胜,还要略微胜出一些。

%46
但后来,凤九歌越战越强,展现出惊人的天赋才情。

%47
双方渐渐扳平,然后凤九歌一点点地扩展优势。

%48
最终,也就在最近的几天,凤九歌经过长时间的激战,战斗能力更上一层楼,终于将秦百胜击败。

%49
击败一两次,不算什么。但连续七场,足以证明凤九歌已经牢牢占据上风。

%50
影宗众人,徐徐退往落魄谷的最深处。

%51
“穷寇莫追。”凤九歌傲立最前方,看着敌人隐没于谷中,语气平静。

%52
他神色憔悴,身上创伤足有十多处,嘴角、眼眉处都有淋漓血迹。

%53
原本洋溢的蛊仙气息,也低弱微薄到低谷。

%54
但这些,都难掩他熠熠生辉的目光。

%55
每一次和秦百胜的交手,都极为艰辛凶险。不过正是如此的艰苦卓绝,像是一块砥石,将凤九歌这块美玉琢磨得更加晶莹通透。

%56
凤九歌自从加入灵缘斋之后,就很少有机会,碰到如此激斗。

%57
灵缘斋本是中洲十大古派之一,凤九歌位高权重,威名鼎盛,不会有人想不开去寻他麻烦。

%58
这样连续百日的战斗,其他蛊仙会感到痛苦和艰难,但凤九歌却感受到久违的激情。

%59
这让他感觉,像是回到了魔道的精彩生活当中。

%60
只是,他的心中仍旧有一丝遗憾。

%61
“魔道……全力以赴地去战斗……可惜。这样的激情岁月,再也回不去了。”

%62
如今的凤九歌。已经是有家室的男人,不能轻易冒险。再加上身后的这些蛊仙。来自其余九大古派,心思难测。一旦凤九歌真的拼杀成重伤,他们有什么反应,还不好说呢。

%63
爱妻白晴临行前的叮嘱,凤九歌牢记心头。

%64
毕竟凤九歌当年,大杀四方,将中洲十大古派都打得脸面全无。加入灵缘斋后,长期以来,都将其余古派死死地压入下风。毫无翻身之力。

%65
凤九歌的话,其余众仙无不听从。

%66
百日大战下来,凤九歌的威望,已经深深地印刻在中洲一行人的心底。

%67
咳咳咳。

%68
秦百胜退入谷中,就不断咳嗽。

%69
他的脸色惨白如纸,每一次咳嗽,都带出一口鲜血。

%70
“你的伤势,越来越重了!”搀扶着他的姜钰仙子,满脸的忧愁。眉头紧锁。

%71
“凤九歌太卑鄙!”失去一臂的贺狼子,不忿地低吼,“他明面上和秦百胜大人一对一,实际上却仗着人多。做出边路冲锋的架势,威胁我们的防御,逼迫秦百胜大人分心防范。若非如此。秦大人你怎么会连败?”

%72
秦百胜却正色道:“不必说了,成王败寇。战败就是战败,不需要找什么理由。”

%73
“这个凤九歌。的确惊才艳艳,依靠自身才情,创造音道杀招,一路走到如今这样的地步。即便作为敌人,我也十分佩服。依靠自身优势,精心谋算,营造优势,正是战斗之道,他若不用,我反而看不起他。”

%74
说到这里,秦百胜又咳嗽两声,吐出一大滩温热的鲜血。

%75
他虚弱地喘息两声,缓缓扫视周围。

%76
百日大战,他的身边,只剩下回风子、贺狼子、姜钰、黑城、神秘黑袍蛊仙五人。

%77
原本有雪松子,可惜在进攻琅琊福地时,被方源斩了。

%78
从琅琊福地撤退之后,又加入了两位毛民蛊仙。两者尽皆在百日大战中,被中洲蛊仙先后斩杀。

%79
不过中洲蛊仙那边,损失还要惨重一些。

%80
古魂门的老算子,灵蝶谷、战仙宗的两位蛊仙,都已身陨。

%81
想要在秦百胜身上占便宜,怎么可能?

%82
就算是中洲十大古派,也要损失惨重。

%83
只是交战至此,最为关键的核心人物,中流砥柱的秦百胜,已经伤重难返,落魄谷的地利也被陆续抵消,残存的五位蛊仙无不心知肚明百日大战的结局已经注定,他们将是失败的一方!

%84
秦百胜强振精神:“我方虽然战败,但大局未失。凤九歌那边,也不是铁桶一块,终究不能全心全力地与我对拼,让我成功拖延了时间。再撑过最后一晚,谷中那温养的仙蛊,就彻底大功告成。明日,你们五位就突围出去,我来拖住中洲这群人。”

%85
“秦百胜大人!”众仙顿惊。

%86
秦百胜居然选择这样做,无疑是自寻死路。

%87
这真让人难以理解。

%88
设身处地思考,换做回风子、黑城、贺狼子中的任何一位,死到临头,都只会关照自己。哪里会舍己救人?

%89
或者说,这是一个陷阱?

%90
回风子、黑城、贺狼子阴暗的心思不断泛起,隐晦的目光扫视其余三人。

%91
秦百胜、姜钰、神秘黑袍蛊仙,才是影宗的真正核心。而他们三位,都是外援。

%92
一阵难言的沉默。

%93
众人步行到休憩地点。

%94
“就这样吧。你们抓紧时间休息,明天突围是否成功,就看诸位自己的表现了。”秦百胜虚弱地摆手,挥散众人。

%95
黑城、回风子、贺狼子却是心情沉重,目光不定。

%96
三人不约而同地走到一起,决定今晚抱团休整,防备影宗三人将自己卖了。

%97
结果到了深夜,三人忽然被秦百胜传讯召集。

\end{this_body}

