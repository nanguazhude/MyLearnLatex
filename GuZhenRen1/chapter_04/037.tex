\newsection{方源师从狂蛮魔尊!}    %第三十七节:方源师从狂蛮魔尊!

\begin{this_body}

砰的一声巨响,方源仿佛一只苍蝇被拍飞,猛地倒射回去。 章节更新最快闪电一般划过半空,随即狠狠地砸在坚硬的冰面上。

轰隆隆!

冰块狂溅,巨大的力量令方源宛若炮弹,一路贯穿冰层,足足数千步的距离,这才将将止住冲势。

方源浑身重伤,半躺在冰堆中,饶是仙僵之躯,也动弹不得。他全身上下宛若破烂的麻袋,碧绿的尸血缓缓流淌,伤口密布,骨头折断,有的嵌在皮肉里,有的直接突出在体外。八只手臂,只剩下三只。一只断掉离体的手臂,就落在距离方源不远处的冰道上。

这个长长的冰道,就是刚刚方源用自己的身躯,活生生地开凿出来的。

咳咳咳。

方源不断咳嗽,睁大双眼难以置信地看着,已经失去脑袋的冰瀑神猿,仍旧昂然站立着。

“这到底是什么鬼玩意?”

按照常理,就算是上古荒兽,脑袋被打爆了,也要立即倒下。但这头并铺上神猿,不仅没有倒下,而且还给方源一记势大力沉的反击。

方源回想起刚刚的那记巨拳,按照常理,如此磅礴的力道早应该打出音爆。但偏偏整个过程,都是悄无声息。

这绝非一头冰瀑神猿,能够做到的程度。皆因它的身上,毫无一只蛊虫迹象。

方源试着挪动身躯,仙僵之躯抖了抖,但仍旧躺在冰面上,努力告吹。

不过方源可以感受得到。自己的身躯正在迅速复原。

僵尸,是以死气代替生气。只要不在一瞬间被消灭,死气充盈。总会以一种诡异快绝的速度康复。

“方源!”太白云生疾飞过来,与此同时,一道道漆黑光线从他手中迸发,落在方源的身上。

这些漆黑光线,大有讲究,是太白云生特意为方源准备的,并不是寻常的治疗蛊虫。

寻常的治疗蛊虫。带来生机,刺激生机,但对于方源这样的仙僵反而不能用。

方源要增强死气。死气越强,他的复原速度就越快。所以,僵尸也别称为“活死人”。

在太白云生的帮助下,方源恢复速度飙升。

“很好。”方源狞笑一声。很快恢复了行动力。从冰面上站起身来。

砰的一声,对面的冰瀑神猿陡然发生惊人变化,竟然忽的自爆。

爆炸之后,一大群的天马飞在半空中,取代了冰瀑神猿原来的位置。

这些天马绝非一般,并非是现在的双翼天马。它们体型更大,是双翼天马的两三倍。在它们的背部,羽翼至少有两对。有的甚至多达三对。

这是上古天马,羽翼越多。战力越强。四翼天马,是兽皇级数。六翼天马则是荒兽。

在如今的五域,上古天马早已经绝迹。

而此刻,这些天马群中,绝大多数都是四翼天马,甚至还有几头,背生六翼!

天马群呼啸而过,和方源的力道虚影大军撞在一起。

厮杀展开,一时间方源的力道虚影大军节节败退,伤亡惨重。

“这不是兽灾!这到底是什么玩意?”太白云生疾飞,迅速接近方源,看到这一幕,他不禁瞪大双眼。

上古天马群和力道虚影大军,绞杀在一起,形成一个相当混乱的战局。

方源大为吃亏。

因为,原本力道虚影松散的阵型,是针对冰瀑神猿这样的巨型猛兽。现在换做了团结在一起的天马群,反而因为兵力分散,被天马群肆意切入。尽管力道虚影相互配合精妙,结成一个个的小战阵,但终究败多胜少,大量损耗。

屠杀在持续进行着,也不见任何尸体。

力道虚影是拳力凝成,也还罢了。关键是天马群,并非实体身躯,杀伤之后,皆化为乌有。

“难道说……”方源赤红的双眼,忽然亮起一丝精芒。他绞尽脑汁,在记忆中搜刮答案,终于脑海中灵光一闪,让他想到了一个飘渺的可能。

这个可能让他不禁回头,看向黑楼兰。

他没有看到黑楼兰,黑楼兰被一团庞大的三色气团包裹着,不见踪影。只要她消化掉这三色气团,成仙之后的潜力势必将极为深厚,前景将一片光明。

方源旋即回头,重面天马群。他的双眼眯成一条缝,笑了一笑,向后摆手,同时传音阻止赶来的太白云生:“老白,你退一边去,为我摄阵。”

随后,他狠狠一咬牙,再度冲锋。

他竟然一路冲杀,扎进天马群和力道虚影混乱的战团当中。

立时,无数的上古天马蜂拥而至,朝方源杀来。方源像是捅了一个马蜂窝,他催动发甲,竖起手臂看,护住脑袋。

看准目标后,他陡然大喝一声,飞扑上去,将一头受伤的四翼天马,直接从天空中强行摁到地面上。

砰的一声,人和马一起重重地摔到地上,冰面破碎,砸出一个深坑。

方源浑身皆是伤痕,多处骨折,皮开肉绽。刚刚修复好的身躯,再度破碎不堪。

在他冲进混乱战团,又摁倒一头四翼天马的过程中,不知道遭受了多少天马围攻。

这些天马,当然不会放过他。大量的天马,汇集成一股白流,从天空俯冲而下,目标直指方源。

方源一面拼尽全力,镇压住身体下不断剧烈挣扎的四翼天马。另一面,在他的脑海中意志滚滚如沸水,剧烈消耗。

天空中,两股力道虚影军势,相互配合,像是一道巨剪,将俯冲而下的天马群拦腰截断。

与此同时,方源的仙窍中。涌现出大量的力道虚影,很快蔓延开来,在方源的头顶结成圆阵。不计牺牲抵御住天马群的扑杀强袭。

方源手脚下的四翼天马,并不乖巧,一直剧烈嘶鸣,奋力挣扎。巨大力道叫人心惊,饶是方源八臂仙僵,力道蛊仙,也几乎按捺不住。

时间紧迫。机不可失。方源高举铁拳,狠狠砸下,立时轰爆马头。但无头天马仍旧挣扎不休,好似没有任何影响。

方源八臂齐捣,拳影重重,尽数轰在天马身躯上。

地下的坚厚冰面。支撑不住。块块碎裂。

寻常的四翼天马,相当于兽皇,媲美五转战力。但面对方源的凶威,就算是一身强力野蛊,也要被捣成肉泥。

这只奇怪的四翼天马,没有蛊虫在身,被方源打得粉碎,却是消散一空。没有留下任何的皮毛、碎骨或者血液。

但下一刻,方源眯着的双眼。陡然大睁。随着四翼天马被彻底击碎,一股无形的意念冲进他的脑海,他的眼中流露出巨大的震动和惊喜。

一时间,他半跪在破烂冰面上,动作停滞,宛若雕塑,任凭头顶上攻杀如火如荼,却是不闻不问,仿佛中了梦魇。

“方源!”太白云生看到此幕,担忧至极,再度飞驰过去。

“回来,不要打扰他。这对他来说,可是千载难逢的机遇!”黎山仙子却忽然传音。

太白云生把眼一眯,立即愤怒地回道:“看来你知道这灾劫是什么?黎山仙子!我们可是盟友,此行专门护卫黑楼兰渡劫的。你居然一直在欺骗我们!居心何在!”

“这不是欺骗,别忘了雪山盟约,若是欺骗了你,我早就应誓而亡了!这只是一个猜想,现在得到了证实而已。”黎山仙子连忙解释道。

这个关键的情报她只是没有说,隐瞒下来,并不代表欺骗。

就算是联合,也不可能将什么秘密都共享。这点方源也不会办到。

太白云生呼吸一滞,一面疾飞,一面暴躁地质问道:“那你现在还不告诉我,这究竟是什么狗屁猜想!”

黎山仙子吐出一口浊气,语速加快:“太白云生,这片冰原的来历你也知道。当初狂蛮魔尊大战,打坏北原一角,使得这里成为虚无。狂蛮魔尊为了弥补,便化身太古冰凰,口吐极寒玄冰,冻住这里,结成偌大的冰原。”

“说重点!”太白云生吼道。

黎山仙子理解太白云生的情绪,并未责怪,继续解释道:“众所周知,狂蛮魔尊乃是力道蛊尊,力道之祖,同时又开创了变化道。但凡任何一个九转尊者,皆是大道之子,最接近天地真理。不仅言出法随,一举一动蕴含真意,甚至目光流转中就有自然奥妙。他们就算身陨,自身的王牌杀招,也会刻印到天地中,形成天劫地灾的一种。”

太白云生不由想到了无相手:“是的,这个我早知道了。”

黎山仙子继续道:“当初狂蛮魔尊化身冰凰,口吐玄冰,凝造我们现在的冰原。你别忘了,冰凰正是变化道的杀招。因此作为冰原一手缔造者,狂蛮魔尊对力道、变化道的真意,就刻印在这处天地当中。在少数蛊仙之间,一直流传着一个猜想,或者更准确地说是一个传闻。只要是力道或者变化道蛊师,在冰原升仙,就会形成道痕共鸣,在天劫地灾中勾动出狂蛮魔尊对力道、变化道的真意。”

太白云生身躯一震:“那,那这么说,岂不是?!”

“不错。冰瀑神猿只是假象,上古天马也是如此,都只是力道、变化道的真意,借助地灾外显!完全击碎这些外显之物,蕴藏的小段真意就会勃发,主动灌体!真意是什么?是对天地、大道、法则的理解。方源刚刚击碎天马,便得到真意灌体。换句话说,方源正得到狂蛮魔尊的教导啊!”

黎山仙子说出惊人话语,同时也流露出浓郁的羡慕情绪。

得到一位九转尊者的传授,这是多么大的机缘!

如果说,三气融汇,是令蛊师感悟天地,师法自然。那么在这里渡劫,得到真意灌体,就是从师狂蛮魔尊!(未完待续……)

\end{this_body}

