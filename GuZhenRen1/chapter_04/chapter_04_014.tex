\newsection{辨认仙蛊}    %第十四节:辨认仙蛊

\begin{this_body}



%1
一时间,众人的目光,都被方源掌中的蛊虫吸引。

%2
蛊虫仙气洋溢,蕴藏着澎湃法则,毫无疑问,它是一只仙蛊。但究竟是什么仙蛊,众人皆是不知。

%3
唯有太白云生心中一动,想起了方源曾经对他透露过的事情。

%4
“这只蛊,这只蛊……”琅琊地灵看得双眼发光,“这只蛊我从未见过,这是一只新蛊啊。”

%5
他跳到方源的面前,鼻尖距离蛊虫只有两三寸的距离,随后他像是闻花一样,用劲嗅了嗅。

%6
然后他大叫起来:“啊,这是运道的气息,这是运道蛊虫。好小子,你弄塌了真阳楼,但看来你收获也不少啊。”

%7
方源立即大肆嘲笑道:“想不到堂堂的琅琊地灵,见识如此孤陋寡闻。这哪是什么新蛊,早在万年前,就已经现世了。好教你知,此蛊名为招灾,乃是中洲灵缘斋某代仙子墨瑶所创。”

%8
“墨瑶?”墨人王闻言,双眼一亮,“如果我记得没错,此女应是墨人历史上有数的传奇之一。她修行到七转地步,自身魅力出众,大名鼎鼎的剑仙薄青也拜倒在她的石榴裙下!”

%9
墨人王如数家珍,说话的时候,神态都带着些许狂热:“阁下勿怪,我从小便以墨瑶前辈为楷模榜样,因而对她的事迹如数家珍。”

%10
方源顿时心中有些好笑,没想到这个墨人王居然还是墨瑶仙子的粉丝。

%11
不过这时,却不是和墨人王拉家常的时候;

%12
。方源借此良机。嘲讽琅琊地灵。

%13
琅琊地灵被他言语羞辱得满面通红,恨不得现在地上裂开一条缝,让自己钻进去。

%14
他口中哼哼不已。却无法反驳方源什么。毕竟这只仙蛊,他的确不认识。

%15
方源阴阳怪气地嘲讽了好一会儿,见时机成熟,便拿回招灾仙蛊,又掏出一只仙蛊来。

%16
“琅琊地灵,你不是自诩见多识广么?我再给你一次机会,让你看看这是什么蛊虫?”

%17
琅琊地灵立即瞪大双眼。趋步向前猛看。

%18
几个呼吸之后,他呼出一口气,紧张的脸色全数转为得意:“这只仙蛊我当然认得。它名为净魂,乃是魂道蛊仙辅助修行之物。魂道修行,需要壮魂、炼魂、安魂。壮魂首选荡魂山胆识蛊,炼魂首选落魄谷中的迷魂雾、落魄风。安魂首选迷魂湖中安魂汤。这净魂仙蛊。能精炼蛊仙魂魄,将杂质魂魄剔除出体外。但若是使用过度,反会把精华魂魄也会剔除体外。”

%19
“哦?你竟然认得此蛊!”方源做吃惊状,心里则道,“原来这仙蛊叫做净魂,我之前称呼它为分魂,算是叫错了。”

%20
这只净魂仙蛊,便是杀招万我的核心蛊。

%21
杀招万我能叫方源产生本我力道虚影。这些力道虚影蕴藏方源的一丝魂魄,因此方源指挥起来。如臂使指,灵活机变。

%22
但也正因如此,动用杀招万我,会消耗方源魂魄底蕴。

%23
琅琊地灵见方源吃惊,脸上得意之情立即又浓郁了几分。

%24
但方源接着道:“这次算你运气好,恰巧认识这只仙蛊。”

%25
“喂,什么叫运气,这是我老人家的实力。”

%26
“你要是有实力,连招灾仙蛊都不知道?”

%27
“你!!”

%28
“罢了,你再看这只仙蛊,你一定不认识。”

%29
“你尽管拿来,让我瞧破它的跟脚!”琅琊地灵大叫。

%30
方源便又取出一蛊。

%31
琅琊地灵屏气凝神,小心辨认,几个呼吸后,大笑:“哈哈,这是运道仙蛊平步青云啊。”

%32
“平步青云?”太白云生插了一句。

%33
“不错。”琅琊地灵得意洋洋,摇头晃脑,“此蛊乃移动仙蛊,历史悠久。蛊仙用之跋山涉水,不在话下。更可贵之处在于,此蛊上下升降速度极快,远超同济。”

%34
“原来这只仙蛊,叫做平步青云。”方源心中咀嚼了几下,又似不甘心,再取一只仙蛊来。

%35
“地灵,你好像似乎可能有那么一点点实力。你看看这蛊再说;

%36
。”方源道。

%37
“什么叫好像似乎可能,实力那是一定有的!”琅琊地灵口中反驳,两眼发光,盯住方源手掌。

%38
几个呼吸之后,他脸色微变:“啊,原来是你。”

%39
“这是什么蛊?”太白云生适时再次开口。事情进行到这一步,他已经看出来方源的企图。

%40
方源虽然在真阳楼中,捕获了这些仙蛊,但却不知道来历,一直在揣摩当中。

%41
不得不说,要熟悉一只陌生仙蛊的效用,是一件麻烦的事情。

%42
最直接的方法,就是消耗仙元,催动仙蛊,直接看它们的效果。但此举弊端重重。

%43
一来,仙元珍贵。二来,仙蛊威力大,搞不好就会害了自己,譬如招灾蛊、春秋蝉,这类仙蛊万万不能盲目催动。三来,有些仙蛊使用方法大异寻常,就算催动了,没有合适的对象和环境,也看不出什么效果。

%44
智慧蛊虽然可以辅助方源,但这也是基于方源所知证据之上的推算。

%45
琅琊地灵不疑有他,卖弄学识道:“这只仙蛊歹毒又奇特,是一只一次性的消耗蛊。老夫虽然没有亲眼见过,但从史料记载中得知其形,当时就很关注,有特别印象。此蛊名为妇人心,和寻常仙蛊不同,它养炼合一。”

%46
方源立即被勾动兴趣。

%47
蛊虫分养、用、炼,三大方面。一般而言,三方面虽有牵扯,但相互独立。

%48
养炼合一的仙蛊,方源还没有见过。但养用合一的一种蛊虫,方源前世十分擅长,这便是恶名远播的五转蛊血滴子。

%49
那么这只养炼合一的妇人心。究竟是什么蛊呢?

%50
方源心中好奇,但不方便直接问这个问题。这时太白云生开口配合道:“居然能养炼合一,我活了这么久。还是头一次听说。”

%51
琅琊地灵瞥了太白云生一眼:“小子,你还年轻,活得不够久。蛊虫之多浩如烟海,千差万别。养用炼三方面,各个博大精深。常言道:最毒妇人心。这只仙蛊妇人心,乃是毒道仙蛊,毒性极其猛烈。要喂养它。就得需要妇人的心脏。喂养的过程,就是一次次不断微炼提升的过程。喂养给它的心脏越多,那么它的毒性就越强。”

%52
太白云生闻言色变:“需要妇人的心?如此歹毒。这是一只魔蛊!”

%53
“嗯,其实也可以用女异人的心代替,不过效果比人心要差一点。相同的毒性,需要更多的女异人之心。”琅琊地灵又回忆道。

%54
太白云生满脸严肃地道:“即便是异人。也是生灵。有父有母,有血有肉,有感情有仇恨,如何轻言牺牲?只为喂养一只魔蛊?”

%55
说到这里,太白云生不由担心地看向方源。

%56
方源面无表情,将仙蛊妇人心收起,随后又取出一只仙蛊:“琅琊地灵,我承认你的确有那么丁点本事了。不过大抵也就这种程度。比我还稍稍差点。哈哈哈,现在我手中的这只仙蛊;

%57
。你一定猜不出跟脚来!”

%58
方源显得十分自信,琅琊地灵的目光被新出现的仙蛊吸引。

%59
“这也是一只运道仙蛊。”他用鼻子嗅了嗅,脸色明显紧张起来。

%60
但几息之后,他满脸的紧张不翼而飞,吐出一口浊气道:“这只仙蛊老夫认识,哈哈,当年本体和巨阳仙尊探讨修行时,就见过这只仙蛊。”

%61
“哦?真有这么巧?”方源一脸不信地道。

%62
“当然!”琅琊地灵双眼喷火,方源不相信他的表情让他感到很不爽。

%63
为了证明自己,他知无不言言无不尽:“这只仙蛊,名为连运。效用十分奇特,能令蛊师的运气和其他人的运气,链接在一起。一方有好运气,就会分给另一方。同理,一方有坏运气,也得影响到另一方。”

%64
“你若是还有察运蛊,就可观察他人的无形运气。看到好运之人,就使用这只连运仙蛊。但这不算最好的使用方法。因为运气如潮水,涨落不定。你要这样用,还得需要另一只断运仙蛊搭配。”

%65
“更好的方法是,你可以用其他运道仙蛊搭配,将自己的运气和山川之运,长河之运,福地洞天之运,或者太古荒兽之运,链接在一起。”

%66
太白云生领悟了其中道理,颔首赞同道:“不错,名山大川矗立已久,福地洞天不易损毁,太古荒兽更是媲美八转蛊仙。这些东西的气运,可比人的气运靠谱多了。”

%67
琅琊地灵感叹道:“运道另辟蹊径,又直指天地大道深处。当初创立它的巨阳仙尊,惊才艳艳,超凡脱俗。当初巨阳仙尊和本体论道七天七夜,就算是本体,也对他佩服万分。”

%68
墨人王附和道:“历史上十位尊者,皆是无上的丰碑。就算我身为墨人,也不得不钦佩他们。”

%69
他也早早看出方源的图谋,就是想利用琅琊地灵,获知仙蛊信息。

%70
但他没有揭破这一层。

%71
方源没有危害琅琊地灵,墨人王犯不着为了这点,恶了他和方源的脆弱关系。

%72
方源对墨人王点点头,又看向琅琊地灵道:“咱们还是回到交易上来吧,我是很有诚意的,从未想过借机贪图你的仙蛊方。地灵,我为你推算仙蛊方之前,会将一只仙蛊抵押给你。到时候,我就算推算不出,又看了仙蛊方,你便可以扣下我的仙蛊。这样你总可以放心了吧。”

%73
仙蛊的价值,一般都远超仙蛊方。

%74
琅琊地灵和墨人王对视一眼,大叫起来:“臭小子,看来你真的对自己很自信啊。行,这交易我不冒风险,甚至还有的赚。丑话先说在前头,我最多给你三张仙蛊方。不论哪一张仙蛊方,你都得完整地推算出来,才算成功。你推算一张仙蛊方成功了,就交给我,我付给你仙元石,再给另外一张残方。”

%75
“好。”方源没有丝毫犹豫,立即答应下来。

%76
ps:不好意思,这章自动发布,设定时间的时候,习惯设定成20点了。;

\end{this_body}

