\newsection{推算蛊方}    %第六节:推算蛊方

\begin{this_body}

。

一道道灵光,在脑海中闪现。。。

无数的材料、蛊虫,组合成一道道的方案。

脑海中的意识,以极快的度消耗着。如此剧烈的消耗度,已经远远过了方源独立思考的极限。

“九转蛊虫的威能,竟然强悍如斯!这还是我没有炼化智慧蛊,只是稍稍蹭了它的一点智慧光晕而已。”方源满怀惊叹地退出智慧光晕。

他正在构思并完善和稀泥仙蛊的炼制秘方。

和之前思考局势或自身情况不同,完善仙蛊秘方需要参考的因素极多。一份相似的炼蛊材料,往往就有数十种,上百种。更别提各种蛊虫之间的相互组合了。

正因为如此,方源脑海中意志损耗度惊人。

几天内好不容易积蓄起来的意志,十几个呼吸的功夫,就几乎消耗一空了。

思考出来的成果,却不足以推进和稀泥蛊方的百分之一。

“但就这进度,若是我单独完成,不借助智慧光晕,至少得大半年的时间。”方源心中计算。

十几个呼吸,对比大半年光阴,足可见智慧蛊的浩荡威能。

不过,方源却不满意。

推算蛊方,同时要参考的因素太多,消耗的意志实在太多,太快。他积蓄了好几天的意志,就这样轻易地消耗光了。

得到的成果,却不足以将蛊方推进千分之一。

方源原本打算,靠着智慧蛊推算仙蛊蛊方。然后拿到宝黄天中贩卖获利。但现在遇到了麻烦。

“推算仙蛊方,需要极其庞大的意志。而凡蛊一次产生的意志又太少。我这些天催用凡蛊产生意志,几乎已经达到使用凡蛊的极限。若是过这个限度。凡蛊就会损害毁灭,除非用六转的乐山乐水蛊……”

方源思考了一会儿,决定试一试。

他消耗一颗青提仙元,催动了乐山乐水仙蛊。

哗!

仙蛊悬浮在他的头顶上,绽放出夺目的光辉。

光辉倾泻而下,宛若一道天河,灌注到方源的脑海里。

这股乐意的规模。是如此庞大。打个比方,五转智道凡蛊催的意志,宛若一捧水的话。那么这股乐意就相当于一条大河!

“不愧是仙蛊!”方源心中一喜。

乐意源源不绝。灌注进来。持续了好一会儿功夫,这才悠然而止。

方源原本宽敞的脑海,已经有三分之一,全被乐意被填满。

特意是金黄色。宛若沙硕。乐意则是一片嫩黄。

“也就是说。只要我连续催动三次乐山乐水蛊,我的脑海中就会被填满?难怪宝黄天的市面上,会有那些改大脑海的智道蛊虫。”从实践中方源产生了一丝明悟。

力道蛊虫,改造蛊师的**。相应的,智道蛊虫则着重改变人的脑海。

方源再次踏入智慧光晕,进行思考。

爽!

太爽了!

好像是没有桎梏,一切通明。遇到什么难题,都能在瞬间攻破它!区别只在于:难度越高的难题。消耗的意志也就越多。

海量的炼道材料被一一筛选,无数蛊虫之间闪电般地组合。再拆分。

方源大展手脚,感觉世间从未有难住自己的问题。

这种感觉真的太过美妙,以至于乐意完全干涸时,方源这才依依不舍地退出光晕,感觉时间短暂,仿佛十几个呼吸。

但事实上,时间已经足足过去了一盏茶的功夫。

“呼……”方源吐出一口浊气,晃了晃脑袋。

退出智慧光晕之后,他立即被打落原形,仍旧是那个思维僵化的僵尸。

巨大的落差,就好像是从天堂到了地狱。

方源深呼吸几口气,迅调整好情绪,检查这一次思考的成果。

和稀泥仙蛊秘方向前推进了百分之一。

和之前对比,这是巨大的进展!

方源点点头,随后却又摇了摇头。

他默然离开地底洞穴,回到荡魂山中行宫。

“师弟,情况怎么样?”太白云生已经在行宫中等候多时,见到方源后关切地问道。

他风尘仆仆,刚考察狐仙福地西部回来。

“坐罢。”方源招呼一声,几个大步就迈到主位,先一步坐下。

太白云生选择最近的位置坐下后,用问询的目光看向方源。

他知道方源借助智慧蛊推演仙蛊方的事情。不久前,他跟随方源来到地底洞穴,远远地看过智慧蛊一眼——他好不容易得到方源送给他的一只寿蛊,寿命有限,可不敢靠近这只传说中的蛊虫。

方源靠在宛若石碑的椅背上,双腿张开。一双手放在大腿上,一双手握于丹田,一双手把握着扶手,一双手怀抱在胸,大马金刀,渊渟岳峙,宛若猛犸静立,巨人雕塑。

他叹了一口气,道:“凡蛊不堪催用,我试着用了智道仙蛊乐山乐水。它产生乐意大河,在智慧光晕下,将蛊方向前推算了百分之一。”

太白云生闻言,激动得从座位上站起身来,喜形于色地道:“百分之一?果然不愧是九转的智慧蛊啊!这么说来,我们动用一百次,不就推算出一道仙蛊秘方了吗?有了这些仙蛊秘方,我们贩卖到宝黄天中,就是挖掘不竭的大宝藏啊!到那时,我们就有源源不断的仙元石!!想不到……师弟你变成僵尸,居然还有这等好处。”

这些天来,太白云生也在为修行资源愁。

他现,踏入蛊仙层次后,他迫切地需要更多的修行资源。

蛊师修行,是温养空窍。蛊仙修行,则是培养仙窍。

仙窍如何培养,众说纷纭。但宗旨不变,就是让仙窍小世界更加繁荣昌盛。

在仙窍中种植更多的植物,移进更多的动物,建立更加完善的生态循环,供养及培养出更多的蛊群……

仙窍越是繁荣,生机越旺盛,凝聚形成的仙元就越快越多。

这就是蛊仙的修行之道,进取之道。

因此宝黄天中,不仅贩卖蛊虫、蛊方,而且还卖兽群、植株、奇石、珍水等等。

看到太白云生激动的样子,方源满意地点点头:“看来我让你考察狐仙福地,熟悉蛊仙的修行之道,很有效果。你终于意识到修行资源的重要性了。不过很可惜,虽然贩卖仙蛊方是个很赚钱的买卖,但是我们暂时还不能进行。”

“啊,这是为什么?”太白云生疑惑。

“因为投入太大了。”方源叹气道,“要从无到有,推算一道六转仙蛊方,耗费的青提仙元要至少过一百。”

“师弟,你这一次不是将仙蛊方推进了百分之一吗?”

“那是因为我推算的是和稀泥仙蛊残方。我也告诉过你,之前我在宝黄天售卖和稀泥,收购了很多和稀泥仙蛊残方。总结了这些残方之后,才得到这件六成残方。”方源答道。

太白云生并不笨,立即明白了方源的意思。

有了智慧光晕,任何的难题都难不倒方源。

正常情况下,推算蛊方遇到难关,很难解决。运气好,灵感闪现,想到解决难关的法子。运气不好,就想不通,卡个数年,甚至数十年不得寸进的情况都有。

但方源有智慧蛊,推算蛊方不存在难关。

就算遇到方源的底蕴也难以想通的地方,那就多想。靠着智慧光晕思考,消耗更多的意志,一般都能攻克难关。

如此一来,仙蛊秘方是越往后推算越容易。

六成残方,已经定下大致方向,也让方源排除了大量的炼蛊因素。在这样的基础上,推进了百分之一。

也就是说,如果从零开始,从无到有,刚开始推算蛊方,会很艰难,连百分之一都不到。

方源继续道:“我的青提仙元只有十八颗,你的仙元有二十六颗,狐仙福地中积攒的白狐仙元有六十四颗。但是你和狐仙地灵,都不能置身于智慧光晕中。就算借出乐山乐水仙蛊给你们,你们催用出来的乐意,我又用不了。目前而言,最有希望的和稀泥残方,也需要将近四十颗青提仙元。可你别忘了,我们并不安全。北原蛊仙,中洲仙鹤门都想找我们的麻烦。若是智慧蛊被现,恐怕五域的蛊仙都会蜂拥而至地杀过来!”

“是啊,投入太大,而青提仙元得留下来防备强敌和意外。”太白云生点点头,长叹一声,看向地面,失望地接受了这个事实。

但旋即他又猛地抬起头,双目闪烁着光:“咦?我想到了一个办法!师弟,我记得你跟我说过琅琊地灵的事情。恩师提点你八十八角真阳楼的秘密,你向琅琊地灵求教时,假称是自己推算出来的。因此琅琊地灵认为你是智道蛊师,想跟你合作完善蛊方。”

“哦!”方源作恍然大悟状,脸上显示出恰到好处的兴奋之色,“是有这么一回事啊。哎呀,哎呀,果然变成僵尸,脑袋就不好使了,居然没想到这一层!老白你提醒的对啊,我手中最好的残方,就是六成的和稀泥蛊方。琅琊地灵手中,一定有很多残方,很可能有完善程度更高一点的残方。”

“师弟,你认识琅琊地灵,知道进入琅琊福地的方法。咱们完全可以和琅琊地灵合作嘛!”太白云生想到妙处,兴奋得朗笑三声。

(未完待续)

\end{this_body}

