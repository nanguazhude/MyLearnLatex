\newsection{放弃夺舍}    %第六十二节:放弃夺舍

\begin{this_body}



%1
飞鹤长唳,在天空中悠然飞翔,展现优美的身姿。

%2
清风徐徐,吹得飞鹤山上松涛阵阵。

%3
旭日高照,明媚的阳光却照不进这间幽暗的地下密室。

%4
地下密室,形制简陋,四周皆是石壁,除去中央的石床之外,空无一物。

%5
湿润的水汽,使得石头缝隙之间,长满了青苔。

%6
石床上,躺着一具破败腐烂的五转青年蛊师,正是古月方正。

%7
为了感应方源的位置,配合鹤风扬攻打狐仙福地,方正遭受天鹤上人的算计,败血妖花蛊等尽皆失控,导致他失血过多,魂魄濒临崩溃,而陷入昏死的状态中。

%8
“你要复活方正?”鹤风扬此时,站在石床旁边,看着眼前悬浮于空的寄魂蚤。

%9
寄魂蚤中储藏着方源师傅——天鹤上人的魂魄。

%10
此刻,从寄魂蚤中传出坚定的声音:“是的,大人,我愿意放弃夺舍的机会,请您出手,救活方正罢。”

%11
鹤风扬将目光投向昏死的方正,轻声道:“他的肉体伤势,看似恐怖,但并不要紧,不需要我出手,门派中都有大量的手段可以医治。关键麻烦的是他的魂魄。他此番强行感应,使得魂魄大损,距离彻底崩散只剩一步之遥。”

%12
“其实,目前的情况,正适合你夺舍啊。这具肉身,本身资质就是甲等,如今又有五转的空窍,魂魄又虚弱至极,你若夺舍。魂魄之争根本毫无悬念。夺舍本来就是你的计划,不是么?”

%13
天鹤上人苦笑一声。

%14
自从他动用手段,害了方正之后。天鹤上人的心底,就总是浮现出方正的模样,总是回忆起和方正相处的一幕幕。

%15
方正的单纯,对正义的执着,总让他想起自己年轻的时候。方正的复仇,又和他的经历多么相似。

%16
“属下原本的计划,的确是想要夺舍方正。但是……此刻属下的心中。却充满了不舍和愧疚。按理说,他是古月一代的血脉,但他口口声声唤我为师傅。他信任我,从未想过我会害他。他相信正义,就像从前的我。若是我害死他,夺舍成功。今后顶着他的躯壳活着。我又如何面对自己呢?”

%17
天鹤上人叹息道。

%18
鹤风扬沉默了一下,忽然扬起手,轻轻地抚上方正的额头。

%19
青绿色的华光温柔如水,从鹤风扬的手掌中流淌出来,迅速蔓延方正的浑身上下。

%20
青绿华光所到之处,浮肿消弭,伤口结痂,大大小小的伤势都以肉眼可见的速度。迅速复原。

%21
几个呼吸之后,方正的身体完全康复。呼吸平稳悠长,只是仍旧昏迷不醒。

%22
“我将他的肉身伤势全然治好,也维持住了他的魂魄。作为你曾经为我驱策的回报,我再给你一次机会,天鹤。你不需要急着放弃夺舍的机会,我给你一段时间再考虑考虑,你好好想想。”鹤风扬开口道。

%23
“大人……”

%24
鹤风扬抽回自己的手掌:“我们掌握的夺舍手段,并不健全,来源于北原。如今,我派的残阳老君已经随着凤九歌,进入北原,他身上肩负着数个门派任务。其中之一,就是收集更多资料,完善如今的夺舍手段。等到他回归门派,天鹤,你将有更好的夺舍机会。到那时,你已经经过深思熟虑,再给我答案罢。”

%25
“大人……”

%26
不待天鹤上人说完,鹤风扬一挥长袖,身形凭空消失。

%27
狐仙福地,地底洞窟。

%28
七彩变幻的智慧之光,映照在方源的脸上。

%29
方源眉头紧锁,脸色阴晴不定。

%30
噗。

%31
忽然,他猛地张口,吐出一口碧绿的尸血。与此同时,脑海中无数的乐意,跟着崩散瓦解。

%32
这尚是方源蹭用智慧光晕,第一次受伤。

%33
方源退出智慧光晕,找到一处矮小的肉芝,当做板凳又坐了下来。

%34
他一边伸手擦干嘴角的尸血,一边强忍头昏脑涨,查看自家的脑海。

%35
脑海动荡不定,各种念头刚刚产生,就自行碎裂崩溃。脑海的四周边界,更是隐现裂痕,这都是刚刚方源强行思考的反噬。

%36
这一刻,方源只消念头泛动,脑海中就会传来剧烈的疼痛。

%37
他索性闭上双眼,停止一切思绪,一动不动,宛若石像一般。

%38
良久,他缓缓睁开双眼,再次查看脑海。

%39
脑海中的震荡消失了,但边界四壁处,还是残留着伤纹。念头产生的速度,比之前要更加缓慢一分,但却不会自行碎裂了。

%40
也就是说,方源又能继续思考了。

%41
这一次受伤,毫无疑问是一个宝贵的经验,让方源学到很多东西。

%42
“智道思考,也有风险,也会受伤。就和力道修行一样,过量发力,就会导致身体肌肉拉伤,甚至肌腱崩裂。”

%43
“我这次思考星道难题,用了大量的乐意,借助智慧光晕带来的无限灵感,同时推算无数可能。终于超过脑海的承受极限,因此受伤。”

%44
“我原本还想借助智慧蛊的威能,参透出星芽蛊的蛊方。但现在看来,我的星道底蕴过于浅薄,无限的灵光,带来无数的可能和方向,每一个可能和方向,都需要推算。而推算的过程中,又会产生更多的选择。”

%45
星芽蛊,是万象星君的独门蛊虫。搭配大量的星雨蛊,少量春风蛊,以及其他种种,便可以组合成杀招——春星雨。

%46
春星雨的效用,能令很多植株增长。方源屡次三番,动用这个杀招,浇灌他的星屑草。

%47
但星芽蛊是一次性的消耗蛊虫,目前只能从万象星君处购买。万象星君因此赚取利润。

%48
方源原本想利用智慧光晕,逆推出星芽蛊的蛊方,但却没有成功,反而因此受伤。

%49
“星芽蛊虽然只是凡蛊,但本身就被万象星君做过手脚,很难逆推蛊方。除非我的星道境界,提升到大师级,否则很难成功。”方源心中升腾起一股明悟。

%50
星道大师境界,代表着星道方面的底蕴,能够大量减少推算中的可能和方向,帮助方源更快地得到正确答案。

%51
方源按照难度,再做推测:“既然逆推星芽蛊,已经需要星道大师境界。那么研究出一个全新的蛊方,用来炼制出星萤蛊,就更加困难,应该要需要星道宗师境界吧。而将都敏俊的星道传承改造,形成全新的星道蛊虫,可以从攻防、移动、治疗全方面增幅整个星道效果,那就更难。应当至少需要星道准大宗师的境界。”

%52
星萤蛊,是星门蛊开启的必要之物,多多益善。

%53
而都敏俊的传承中,一星半点蛊,二星辉映蛊,三星在天蛊,四星立方蛊,五星连珠蛊,可以增强星道蛊虫的攻击效果。若是流传出去的话,势必能令星蛊大热,对所有蛊师流派的格局都有微弱影响。

%54
方源原本的打算,是要逆推出星芽蛊、星萤蛊的蛊方,一是摆脱万象星君的钳制,二是多一个途径,得到更多的星萤蛊用于星门。

%55
而提升一星半点蛊,二星辉映蛊,三星在天蛊……一系列的星蛊效用,便能发展出一整套适合方源的全新星道杀招,提升他的战力。

%56
但实践之后,方源知道他想当然了。

%57
即便他有智慧光晕,有无限灵感,但他的炼道境界只是大师级,星道境界只是普通级,难以令他原先的打算落实。

%58
“看来我要提升战力,还得扬长避短,从自身优势出发。”方源审视自己,他现在是血道宗师、力道宗师、飞行准宗师、炼道大师、奴道大师以及变化道大师。

%59
他首先排除血道。

%60
血道蛊仙人人喊打,皆因血道修行祸害他人,令战力暴涨。正魔两道都不愿看到血道蛊仙这样的强烈威胁。

%61
唯有到达五域乱战时期,各域秩序大乱,血道才会冒头。

%62
方源虽然有血神子残方,但暂时也不想推算出完整蛊方,更没有财力炼制血神子仙蛊。

%63
而且他前世的血道传承,现在还未酝酿出来,不可提前收取。

%64
随后,方源又排除力道。

%65
力道曾经昌盛过一段时间,到了现在,已然式微。纵然有楚度这样的惊才艳艳之辈,也难以重现力道的辉煌。

%66
一个流派的昌盛强大,不仅仅是依靠天才蛊师蛊仙,而且还要依赖资源。

%67
到了现代,用于力道的资源绝大部分都消失殆尽,或者十分稀少。

%68
力道前景不佳,很多蛊方都失落,前景暗淡无光。方源即便是力道仙僵,也不想走这条路。

%69
他之所以选择力道,纯粹是条件便利,可以成为跳板,以后随便改变流派。后来成为力道仙僵,也是在八十八角真阳楼时,被逼无奈的选择。

%70
除非方源像黑楼兰一样,拥有数只力道仙蛊,干脆靠这些仙蛊拔升战力。但这种情况,对方源来讲并不实际。

%71
血道、力道不成,飞行准大师不过是项战斗造诣,炼道本就不擅战斗,方源剩余的选项,就是奴道和变化道。

%72
但奴道要求兽群。

%73
尤其是蛊仙这一层次,寻常野兽都拿不出手,兽群核心已经从异兽王,兽皇,上升到了荒兽。

%74
方源借助智慧光晕,苦思良久。

%75
又过三天,他在变化道的方向上,寻找了突破。

\end{this_body}

