\newsection{敲诈成功}    %第一百一十节:敲诈成功

\begin{this_body}

但就目前,方源和琅琊地灵的关系,却还未深入到可以转卖仙蛊方的时候。<strong>最新章节全文阅读qiushu.cc</strong>甚至就连凡道蛊方,琅琊地灵也敝帚自珍,坚决不卖。

对于炼蛊方面的一切种种,琅琊地灵都十分小气。方源之前愿意出高价,买他的毛民。这些异人奴隶,只要和炼蛊挂钩,琅琊地灵都当做宝贝。更何谈蛊方了。

但若方源得到吃力仙蛊,情形就有了极大改善。

吃力仙蛊,可以令蛊仙自生力道道痕。这样一来,得到的道痕,必定完美适合他的身体。方源现在修为停滞,完全可以靠此仙蛊,得到这方面的些许弥补。

方源为之心动。

可以说,只要是力道蛊仙,看到吃力仙蛊,应该不会有人不心动的!

“有这力道仙蛊,不怕对方不心动。”凤九歌安然稳坐,内心把握十足。

此蛊一出,顿时令周围的竞争者,相形见绌。只要不是傻子,都能看出吃力仙蛊的好。

“若我没有看出破绽,兴许这个时候也就换了。可惜啊……”方源却是冷笑连连。

这时,二十三号密室中,忽有人开口报价:“一只铁冠鹰力仙蛊,同时附加仙材法眼血十五滴。”

听到这个声音,方源嘴角笑意更浓。

之前,他用平步青云仙蛊,换来了一只铁冠鹰力蛊。但这个过程,并未暴露自己身份。方源是通过秦百胜,确认了买家。

因此铁冠鹰力蛊落到谁的手中。除了秦百胜之外,场中其他蛊仙并不知情。

秦百胜签订了盟约,自然也不会泄露分毫信息。

方源因此可以大胆运用。

至于二十三号密室中人。不是旁人,正是太白云生。

凤九歌为了乐山乐水仙蛊,提前做了准备,向凤仙太子求援。方源怎么可能不做相应准备呢?

他早就通过秦百胜,安排太白云生,暗中换了密室。

虽然太白云生在十一号密室时,并未暴露出和方源的身份。但换一间密室。无疑更佳。

太白云生报价之后,方源迟迟不反应。

时间渐渐流失,凤九歌暗暗皱眉。那边霸仙楚度忽然开口竞价。他也出了一只兽力仙蛊,同时增添一小份仙材,作为砝码。

楚度并不太需要乐山乐水仙蛊,当然有了也能用。他也是力道蛊仙。之所以放弃一只力道仙蛊。却是打着吃力仙蛊的主意。

方源呵呵一笑,立即洞悉了楚度的想法。

谁叫楚度暴露了自家身份,很容易陷入被动。

对于楚度掺和一脚,方源乐见其事,仍旧默不作声,任由这场竞拍继续进行下去。

在他授意之下,太白云生再次加码竞价。

凤九歌眉头皱得更深一点,不得已。也开口竞价。

他话音刚落,那边太白云生紧接着开口。气势咄咄逼人,显示出对此仙蛊的必得之心。

凤九歌心中不禁疑虑:难道这吃力蛊并不吸引十号密室中人?还是有贪心作祟,要等价码再高些。

凤九歌没有轻易入瓮,到底是一代英中之雄,他决定缓一缓。如此他可以更冷静地分辨局势,同时还能试探一番方源的心意。

幽兰剑师不再竞拍,似乎不舍吃力仙蛊,退出了这场竞争。

这种情况,立即引发了蛊仙们的注意。[看本书最新章节请到求书 小说网www.Qiushu.cC]

最得意的莫过于楚度,他再度发去信息,力劝凤九歌转卖吃力仙蛊给他。

凤九歌轻笑一声,再度拒绝。

另一边,方源嘱咐太白云生,一直不断加价。

这价格往上攀升,渐渐高昂,让参与其中的大部分蛊仙相继退出,纷纷言道这个价格过高。

最终,只剩下楚度和太白云生二人。

楚度此刻,却是难熬。他真正的目的,在于吃力仙蛊。但凤九歌那边丝毫不松口,似乎还想再竞争乐山乐水仙蛊。

楚度不敢懈怠,一直跟价。一方面是防止吃力仙蛊,落入他手,另一方面就算自己获得乐山乐水仙蛊,再拿来跟幽兰剑师换蛊,把握也是大的。

太白云生的心思,则很单纯。师弟嘱咐他怎么干,他就怎么干。哪怕自己拍得乐山乐水仙蛊,也没事!反正出的价,都会落到自己口袋里去。

凤九歌的策略,也简单。就是坐山观虎斗,等到竞争得差不多了,再出一把力气,将价格拔高一大截,一锤定音。

这三人之间,因为各自的意图,巧妙地形成了一个僵局。

局面不断僵持下去,又过了几轮之后,终于大厅中有位蛊仙率先惊醒,脱口道:“这乐山乐水仙蛊,怎般如此吃香?仙材一大摞,竞价已然如此之高!”

众人这才猛然意识到这一点。

“是啊,已经很高了。”

“一只能产乐意的智道仙蛊,却能拍出此等高价,比较罕见。”

“依我看,之所以这么高,还是因为楚度大人参与其中。宁愿舍弃力道仙蛊一只,也要得到乐山乐水,看来他定有他图。”

被众人议论的楚度,此刻状态却不怎么好。

他满脸忧愁之色,实在是自己手中财力薄弱。

“唉,早知如此,就应该缓一缓渡劫了。”来自太白云生的压力,让楚度感觉整个人都不好了。

“什么时候,我堂堂的霸仙,也落到这般窘迫田地?”楚度战力出色,但在这场拍卖大会中,比拼的却不是战力,而是财力。

太白云生在方源的授意下,再度提价。

楚度咬牙支撑。

如此又几轮过后,乐山乐水仙蛊的价格。在蛊仙们的关注下,缓缓上升。

秦百胜一直默默关注,看到此处时。也不禁在心中暗赞:“这位沙黄,身为仙僵,居然也是如此精明,不简单!不过楚度明显财力见底了,出价时已显犹豫。若是我,此刻便见好就收。”

秦百胜知道:方源这计策看起来简单,施行起来。却是大不易的。身在局中,要行此计,需要大大考验眼力。以及对他人心思的揣摩。稍一过火,就是竹篮打水一场空。

方源却丝毫不懂得什么叫做“见好就收”,一直授意太白云生加价,气势逼人。

楚度终于支撑不住。额头都隐现汗渍。

他目光穿透单间。望向二十三号密室的方向,心思里面这人,恐怕对乐山乐水仙蛊有特殊需求,否则绝不会出这般高价,且势如猛虎,咄咄逼人。

“也罢。”楚度叹息一声,随后默不作声,不再参与竞拍。

如此一来。只剩下太白云生一人。

就算如此,太白云生也再度报价。将价格又提升了一小截。

而幽兰剑师那边,却迟迟听不到声音,仿佛真的早就退出竞拍一样。

方源冷笑一声,告诉秦百胜自己的决定。

秦百胜听到这个消息,首先一愣,怎么自己人卖给自己人?旋即自己想通,兴许沙黄和白胜,关系并非看上去那边亲密。

他这样想,也能想通。但心底里总觉得似乎还有猫腻蹊跷。

秦百胜暗自摇头,身在台上,不容多想,开始宣布:“二十三号密室第一次。”

顿了一顿,他又道:“二十三号密室第二次。”

大厅中蛊仙们小声议论着。

“看来这场胜者,应该就是二十三号密室了。”

“不错。二十三号对乐山乐水十分在意,就算是霸仙退出,也仍旧独自加价,目的就是区别开来,增添差距。防止卖家一个念头,在最后关头,选择卖给霸仙。”

“这价格真的很高,仙材之多令我都要流口水了。想不到在最后关头,居然也能出现这样的高价,差不多可以排入高价前十之内了。”

大局已定,方源却紧张起来。

他双眼眯着,心中生出一丝动摇:“难道我之前猜测错了?”

“二十三号密室,第三……”秦百胜的喊到这里,忽然被人打断。

凤九歌开口道:“吃力仙蛊一只,在二十三号密室的基础上,再添赤练金八百斤!”

八百斤的赤练金,一下子将价格抬高上去。手笔比方才太白云生的独自提价,还要大出许多。

众皆哗然,没想到最后关头,竟然还有如此异变。

秦百胜也是楞了一下。

霸仙楚度面色一白,随后幽幽叹息。

方源哈哈大笑:“好个幽兰剑师,真是耐得住性子!”

二十三号密室中,旋即传出太白云生的声音,一副好事被坏,恼羞成怒的语气,再次提价。

凤九歌苦笑一声,他看了这么多时,心中已然相信二十三号密室中人,是真心竞价的。

“我此番运气,却是不好啊。想要买下的两只仙蛊,竟然都遇到势在必得的对手。关键这两个对手,皆是财力浑厚。”

“不,这拍卖大会并不禁止暗换密室。也有一种可能,这密室中人就是一号密室的人物。也许他也知道近水楼台的事情?毕竟墨瑶的确留下传承……更或者,是中洲其他九派不愿看到我灵缘斋……”

凤九歌真是聪明人,但聪明人就是想得太多。

“不管怎么说,为了近水楼台,我都要将这只乐山乐水仙蛊拿到手!”凤九歌很快心思一定,再次竞价。

他和太白云生你来我往,纠缠不休。

在蛊仙们越来越惊诧的目光下,乐山乐水仙蛊的价格节节攀升。很快就冲入前五高价之内。

随后,又以一种惊人的速度,攀升到第二高价。

第一高价,当然是凤九歌以二换一,买下方源的浪迹天涯仙蛊。

“难道还要我以二换一?”凤九歌面色冷峻,他现在也有了之前楚度的感受,充分体验到了什么叫做咄咄逼人的气势。

不管他怎么加价,二十三号密室中必定立即报价,且高出一筹。

凤九歌坐不住了,开始在密室中踱步。

若以二换一,能够成功,他咬咬牙,也就这么做了。但他心中总感觉不妥,但要让他真正说出什么不妥的地方来,他又说不出来。

“如果我再次以二换一,只怕有心人看出我的真正目的。对我接下来寻找水乳交融蛊,也大有妨碍啊。”

凤九歌目标直指近水楼台,但三大核心仙蛊,拍卖会上只出现两只。他当然也想将水乳交融蛊,搞到手里。

凤九歌心中的不妥,甚至让他感到隐隐的不安。

他重新坐到位置上,决定主动出击。他再次喊出声来,要求卖家再次给出新的需求,他将尽力满足。

方源故意停顿了一番,给外人一副我在思考的假象。

旋即,他通过秦百胜提出新的需求:一道完整的智道真传。

众人哗然,这价格太高,居然有人真的能说出口!

凤九歌愣住,智道真传他手中的确有,但却是残缺。卖家这样搞,难道是想流拍?

太白云生则喊道:“我手中有一道智道残缺传承。”

凤九歌紧随其后:“智道残缺传承,我手中也有。”

太白云生冷哼一声,又道:“请卖家再提新需求!”

蛊仙们面面相觑,自己没有听错,这价格竟然还要往上提?!

秦百胜沉默了一会儿,替方源发声道:“卖方要求伪装身份的杀招,效果更佳者得胜。”

凤九歌心中一紧,卖家如此说,无疑对这伪装杀招,更加看重一些。

太白云生很快开口:“我有一完善杀招,虽是凡道,却源自盗天魔尊,乃是见面不相识!”

见面不相识?盗天魔尊!

大厅中再掀微澜。

凤九歌额头见汗,双手紧握把手,目光急促闪烁,犹豫了片刻,也喊道:“我也有一杀招,却是仙道残招,源自盗天魔尊,名为见面似相识!”

众人惊呼。

方源振奋地从座位上站起身,眼中精芒绽射:“看来前世传闻并非空穴来风,灵缘斋中的确有仙道杀招见面似相识!”

灵缘斋好东西虽多,但许多标志性的,却是不好敲诈。一敲诈,对方立即恍然大悟。并且对方源而言,最合用的无非就是这个杀招。

就算前世,灵缘斋掌握见面似相识杀招,也是秘而不宣。在五域大战时期,派遣间谍,深入敌后,取得许多战果。

次数多了,敌我阵营里的蛊仙们不断分析,这才渐渐确定是见面似相识。

但灵缘斋始终不承认。

毕竟是十大古派之一,正道魁首,却用魔尊杀招,面子上过不去。

方源却不知道,这个仙道杀招实际上是凤九歌之物。前世凤九歌贡献给了门派,这才有了种种传闻猜测。

方源不知道这一点,也没有猜到对面就是凤九歌。但并不妨碍他敲诈出这个杀招来。

当然,凤九歌不会将完整的仙道杀招,都交给方源。

残缺的仙道杀招,就已经远远凌驾于凡道杀招见面不相识了。

方源当即确定,将乐山乐水仙蛊,卖给幽兰剑师!

如此一来,他不仅得到大批仙材,将清单完成了一小半,而且还得到了吃力仙蛊,一份残缺的智道传承,以及残缺的仙道杀招见面似相识。(未完待续。)

ps:有点晚,没办法,这章4200多字。不能拆分,不能断,写这么多,才够流畅。耐心的人,有福!<!--80txt.com-ouoou-->

------------

\end{this_body}

