\newsection{鹤风扬之谋}    %第二十节:鹤风扬之谋

\begin{this_body}

这处洞穴大若广场,弥漫层层血光,血腥气味浓郁扑鼻。?洞穴中央,人为挖开一个圆坑,里面血液滚滚,不断咕咕地冒着热气。

方正深吸一口气,轻车熟路地脱去衣服,赤身走入血池。滚烫的血液,一时间让他不断大口呼吸。

他适应了血温,在血池中央站定,血液漫过他的腰际,露出他的胸膛,和大半个手臂。

“开始吧。咱们一个个来,不要着急。先是铁血蛊。”天鹤上人提醒道。

方正缓缓闭起双眼,依言催动铁血蛊。

顿时,蛊虫的力量转变了他的血液。他原本鲜红的血液,开始发黑变沉,血液流速变得十分缓慢。

方正白皙的皮肤,也因而变得黑漆。

“维持铁血蛊,催动血刃蛊。”天鹤上人又道。

方正催动血刃蛊,身躯一震,顿时他的皮肤破裂开来,瞬间形成上百道的伤口,一大群血刃绽射出去,均是漆黑的铁血之刃!

随后伤口缓缓流出铁液一般的血。

方正面色冷酷,感受不到丁点的痛楚,反而有丝丝的强烈快感袭上心头。这是铁血蛊的效果,能将痛楚转变成快感,帮助蛊师更加适应战斗,能使一位懦弱的蛊师变成嗜战如狂的硬汉。这些血缓缓流入鲜红的血池中,渐渐将方正附近染黑。

“下面是关键的一步,用混血蛊。”天鹤上人再道。

方正咬了咬牙,催动混血蛊。

顿时在蛊虫的作用下,他浑身的上百道伤口爆发出猛烈的吸力。周围大量的血液,不断被伤口吸纳进去。

“啊……”方正低声呻吟,强烈的痛楚转变成强烈的快感。他狠狠咬牙,身躯颤抖不止。

随着血液被大量吸入体内。他浑身浮肿,成为一个畸形胖子。他的体型是原先的三倍有余,皮肤硬生生地被撑开,血管粗壮如蛇,在方正的体内扭动。

翩翩的浊世佳公子,转眼间变成了恶心丑陋的怪物。

“别忘记了正事,对抗这种快感。方正,快。应该催动败血妖花蛊了!”天鹤上人时刻监视着方正的情况,赶忙提醒道。

败血妖花蛊这个词,让方正心头一颤,他艰难地从强烈的快感中挣脱出来,大口喘息着,催动败血妖花蛊。

方正握紧双拳,浑身庞大的血液迅速败坏,在蛊虫的影响下,血液中生出一朵朵的藤蔓。藤蔓中生出一朵朵的花蕾,花蕾迅速绽放,形成妖冶的蓝色如菊的花朵。

“呃――!”方正咬紧牙关,痛苦低喊。他满脸发白,整个身体成为土壤。硬生生地长出这么多的妖花。这种强烈的痛楚,比孕妇分娩还要恐怖十倍以上!

就算是铁血蛊的效用,也抵不上这种痛楚。

方正痛得几乎要把一口白牙咬碎。他脸色发白。额头青筋暴起,神色狰狞可怖。

“快,你的全身血液滚烫,再这样下去,你的五脏六腑、全身皮肉会被自己的血液煮熟。快用冷血蛊。”天鹤上人的声音,透出丝丝紧张。

方正艰难地催动冷血蛊。

血液迅速冷却,他打了个寒颤,终于脱离了死亡的危机。

“可以了,妖花将大大增幅你手中的血道蛊虫的效果。你离成功只差一步之遥。用血感蛊!”天鹤上人语气急促。

“血感应……”方正低着头。痛得视野都开始模糊。他凭借多次训练的惯性,始终坚持。催动了这只血道侦察蛊虫。

几个呼吸之后,方正开口:“我,我感应……到了。在地底东南角,距离五千六百步。”

“很好,你又做到了!你已经快成功了,接下来,就是最后一步,你要用血痕蛊将其定位。”

“啊……”方正却在这个时候,开始了无意识的呼喊。他视野完全模糊,单薄的身躯摇摇欲坠,他拼尽全力,压榨出全部的生命潜力,试图催动血痕蛊。

他神志不清,明明五转空窍中真元还相当充足,但他只能调动一小部分。真元如溪水,缓缓灌注到血痕蛊中。

然而血痕蛊需要真元量十分庞大,方正坚持了八个呼吸,终于彻底崩溃,一头栽倒在血池里,当场昏死过去。

……

香炉烟气袅袅,这处静室并无窗口,显得昏暗无比。

鹤风扬身着白袍,系着黑腰带,大袖翩翩,盘坐在蒲团上。

他面如少年,温润如玉。眉毛碧绿修长,眉间一直垂到腰间。幽深的双眼盯着眼前的寄魂蚤。

寄魂蚤悬浮在半空中,天鹤上人的魂魄正汇报着方正的这次训练结果。

鹤风扬语气不满:“果然,尽管让他在伏虎福地生活,消耗了他八年生命,还用了许多舍利蛊,但终究是速成的五转蛊师。哼,不过同时催动六七只蛊虫,这点简单的事情,都不能做到。”

天鹤上人斟酌着词句道:“方正的确年轻,一心几用的基本功稍欠火候。但属下觉得,他十分努力,进步很快很大。记得第一次的时候,刚用混血蛊才一个呼吸,他就痛得昏死过去了。”

“所以,我才添了铁血蛊给他,结果第二次他就因为快感太强烈,大肆泄精,泄得自己直接昏过去了,只支撑了三个呼吸。”鹤风扬不悦地打断天鹤上人道。

天鹤上人连忙道:“太上长老大人,实话实说,就算是属下,在这种强烈的痛楚和快感中,也难以一心几用的。方正已经离成功不远了,他这一次坚持了八个呼吸。总共需要十个呼吸,就能催动血痕蛊,彻底成功。而他只差两个呼吸而已。只要我们坚持训练几次,再给我们一点时间,就能……”

“够了!再多训练几次?你还要训练多少次?时间,时间,你说说看,你的这个计划总共耗费了多少时间。多长时间了?年多了!你还没有一点起色,荡魂山仍旧没有掌握在我们的手里。你知不知道,门派中有多少不满的声音,底下又有多少的弟子,要求高层开放胆识蛊的贡献兑换呢?”

“属下办事不利,连累了太上长老大人,真是罪该万死!”天鹤上人见发怒,连忙告饶。

鹤风扬深呼吸几口气,挥袖:“你下去吧,距离门派大会只剩下一个月的时间。我要在参加门派大会之前,听到你训练成功,可以发动夺回狐仙福地的好消息!”

“可是大人,一个月的时间太短了,过度训练,方正会吃不消的。他的身体我们可以用蛊虫调理,但是强烈的快感和痛楚,会伤及魂魄,最终会令他的魂魄崩溃!”天鹤上人叫道。

鹤风扬呵呵一笑:“他的魂魄崩溃了,不正是你想见到的吗?这样一来,你正好可以夺取他的身体重新复活了。这也是你当初的计划,不是吗?……

说到这里,他的笑容越家温和:“这个方正,不过是一个速成的五转蛊师,心性幼稚得很,还很缺乏磨练,怎么及得上你天鹤呢?这一次利用他夺回狐仙福地之后,你就回来吧。唉,自从苏三、周武二人被宋紫星那魔头杀掉之后,我的身边,像你这样的得力下属就越来越少了。”

鹤风扬语气越温和,天鹤上人的心中却越是发寒,他用感激涕零的语气道:“得太上长老大人如此看重,属下一定报效大人,肝脑涂地!”

“很好,下去罢。”鹤风扬含笑,挥退天鹤上人。

静室中只剩下鹤风扬一人,他的笑容渐渐消失,取而代之的是满脸的凝重,目光中还有一些烦躁。

从方源夺取狐仙福地之后,他就一直负责此事,一年的时间过去了,门派施加给他的压力越来越大。尤其是他在门派中的对头――蛊仙雷坦,更多次在公开场合中嘲笑他办事无能。

一个月之后的门派大会,就是一道难关。如果鹤风扬还没有进展,会很难过。他仿佛已经听到了,雷坦对他发出的响彻全场的嘲笑声。

“不过用不了多久了……只要我此行成功,就能让雷坦这些小人都统统闭嘴。我就是门派这些年来,贡献最大的大功臣!这一次攻略狐仙福地,我会亲自出动。而且还邀请了苍郁仙子,残阳老君。苍郁仙子战力和我不相上下,掌握至少三道凡道杀招,残阳老君更是七转蛊仙,拥有攻伐仙蛊!”

“唯一的麻烦,就是地灵会禁用一切凡蛊。我没有仙蛊可用,难免束手束脚。唯一之法,就是用仙元对耗。所幸狐仙福地本来就贫瘠,经营不佳,积攒的青提仙元能有多少?我就不信方源能耗得过我方三大蛊仙合力!只要狐仙仙元耗光,地灵有心无力,再无法禁用凡蛊。到那时,哼!”

鹤风扬口中喃喃,渐渐抚平心中的烦躁。

与此同时,狐仙福地,地底洞窟。

方源施施然退出智慧光晕,脸上带着满意的笑容。

就在刚刚,寒冰星尘杀招,被他成功地推算出来。)

\end{this_body}

