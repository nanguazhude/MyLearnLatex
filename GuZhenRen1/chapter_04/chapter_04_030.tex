\newsection{仙蛊喊饿}    %第三十节:仙蛊喊饿

\begin{this_body}

%1
方源思考了一下,对黎山仙子道:“这份毒气吐纳杀招,还有上古力道蛊方我都要了。多少仙元石?”

%2
黎山仙子答道:“毒气吐纳需要两块仙元石,群力蛊的蛊方则也卖两块仙元石。一共四块仙元石。”

%3
方源点点头,认可了黎山仙子的报价。

%4
毒气吐纳杀招不俗,的确值两块仙元石的价。

%5
群力蛊虽然只是五转的凡蛊,但早已经绝迹。物以稀为贵,其蛊方高达两块仙元石,也很正常。

%6
黎山仙子给出的这三件东西,还属这个不起眼的群力蛊方对方源的吸引力最大。

%7
不过,报价虽然公允,但方源手头紧张,仍旧想要还一还价钱,尽量节省手中的仙元石。

%8
于是,他开口道:“这群力蛊方虽然稀罕,但却是上古蛊方,炼蛊材料涉及上古,到如今可不好找啊。”

%9
黎山仙子便笑:“方源啊,我也早料到这点。这群力蛊的蛊方,我本来是给小兰准备的。之所以现在卖给你,也是因为炼制这蛊的主要材料石猴毫毛,最近面世了。”

%10
黑楼兰附和道:“不错,最近宝黄天中的一位蛊仙石磊,人称仙猴王,在自家的福地中研生出了石猴毫毛。这项炼蛊材料,已经灭绝了无数年,却在他的手中得以重现,如今已经在宝黄天引发了不小的轰动。”言下之意,就是不接受还价。

%11
方源并不气馁:“石猴毫毛?这种上古炼蛊材料,现在虽然重现,又被仙猴王石磊一家垄断,价钱应当不便宜吧。”

%12
“有句老话说得好,便宜无好货。方源。我亲眼目睹了你的仙道杀招万我,威力真正惊人,直至现在也深入我心。明人不说暗话。群力蛊对方源你来讲,价值比对寻常蛊仙要更大得多。两块仙元石,已经卖得很便宜了。”黑楼兰说道,该争取的利益。她寸步不让。

%13
被对方明白了底细。的确不好。

%14
不过,黑楼兰说的也没错。总共四块仙元石的价格,并不高,很公道。

%15
黎山仙子没有赚方源的仙元石,但也没有亏本贩卖。

%16
“也罢。”方源审时度势,不再坚持。黑楼兰、黎山仙子可不是琅琊地灵,过分还价,就是侮辱对方的智慧。方源得到毒气吐纳杀招,以及上古群力蛊蛊方。还有四块仙元石。

%17
至于移动杀招孔升天,并不适合方源。

%18
孔升天打开全身毛孔,从毛孔中喷吐气流,带动身躯行进。虽然这杀招速度不俗,且转向极为灵活。

%19
但全身毛孔张开,却让方源防御力大降。

%20
同时,方源的防御杀招发甲,将浑身笼罩,没有一丝缝隙。若用孔升天,就不能用发甲。若是用发甲,孔升天也就无用了。

%21
两者并不兼容。

%22
综上原因,方源放弃了杀招孔升天。

%23
他手头上的仙元石增长到十九块半,但一来要转化成青提仙元,二来要担负起喂养仙蛊的巨大压力。

%24
方源手头仍旧很紧,不能随意乱花去购买相对无用的杀招。

%25
离开的时候,方源也没有奢侈地动用定仙游,而是走的星门。

%26
他在临走前,看了一眼黑楼兰的运气。

%27
黑楼兰是五转巅峰,她的运气能够被五转察运蛊感知。

%28
黑楼兰的运气,已经从对付马鸿运陷入低谷的状态,回复过来了。她的气运如一根巨柱,呈现青云之色,给人绵延悠长之感。

%29
“黑楼兰的气运,明显也是韩立、洪易这等级数,以此推测她此次升仙,成功的可能较大。不过凡事也不能单看运气,要不然的话,我的五百年前世中,为何黑楼兰却暴毙了?可见运气虽好,也得看自己能不能把握。现在看来,也许黑楼兰的暴毙,是黑城动的手脚?”

%30
黑楼兰运气虽强,但雪山盟约中有详细条约:不能恶意加害盟友。

%31
方源霉运缠身,若要和她连运,的确有一种加害的意向。

%32
再者方源根据前世记忆,也不太看好黑楼兰的未来。方源没有断运蛊,不能胡乱连运,必须得精心挑选目标,看准目标。

%33
回到狐仙福地,方源首先将仙元石提出八块来,转化成自己的青提仙元。

%34
他多次用定仙游,往回西漠、中洲、南海,和黑城交战又用了浪迹天涯仙蛊,总共损耗了六颗青提仙元。

%35
原本十七颗的青提仙元,便剩下十一颗。

%36
鉴于最近战斗频繁,方源炼化八块仙元石后,将青提仙元总数提升到十九颗。

%37
这样一来,方源手中的仙元石也暴降到十一块半。

%38
回顾和黑城的一战,方源心底满意。

%39
胜败的结果,对方源来讲,毫无关系。他不会单纯的因为失败,而气恼愤怒,感到羞辱。也不会因为胜利,而感到高兴骄傲。

%40
他更看重利益得失。

%41
此次参战,方源消耗了四颗青提仙元,但得到了黑楼兰、黎山仙子的更多信任,同时还有四块仙元石,杀招毒气吐纳,以及群力蛊蛊方。

%42
很明显,他赚了一笔。

%43
“不过,若说上损失,还得记上浪迹天涯仙蛊。”方源掏出仙蛊浪迹天涯。

%44
此时的仙蛊气息,却无往常那般稳定,透露出一股虚弱的感觉。

%45
蛊虫身上也没有健康的光泽,显得晦暗无神采。

%46
方源心中了然:“浪迹天涯仙蛊,本来在王庭福地里头,就是饿一顿饱一顿的。大同风幕下,墨瑶意志催动近水楼台激战,我得到后,也催动了这只仙蛊多次。它已经达到了某种底线,需要喂食了。再不喂食,又过分催用的话,恐怕就要被饿死。”

%47
浪迹天涯仙蛊,是方源手头上第一只“喊饿”的仙蛊。

%48
完全可以预见,随着时间推移,方源手中会有更多的仙蛊喊饿,需要喂养。

%49
然而,要喂养浪迹天涯蛊,需要数万头的幽冥水母,以及数千头的深海闪电鳗鱼。

%50
方源问询了地灵小狐仙。太白云生走后,地灵就接替了太白云生的工作,时常关注宝黄天的动向。

%51
但方源虽然在宝黄天中主动求购,但响应者寥寥。

%52
这不是地球上的商业社会,这里经济并不发达,不是想买什么就能买得到。

%53
方源又用推杯换盏蛊,传去信笺,寻问太白云生。

%54
太白云生带给他一个好消息:他已经来到海市福地,顺利接触到了某些蛊仙,并且在海市福地中看到有人售卖幽冥水母的。

%55
只是深海闪电鳗鱼,并不常见。蛊仙需要去深海捕捉,不仅危险,而且盛产闪电鳗鱼的海域,已经被东海僵盟控制。

%56
东海的僵盟,是五域僵盟的总部,势力强盛,比东海本土的超级势力还要高出一筹。

%57
因此东海仙僵,行事跋扈嚣张,封锁了数个大海域,将海量资源全数收敛到自己手中。

%58
如此一来,更保持、助长了他们的战力,令他们对外更加霸道。

%59
太白云生来信建议方源,直接加入东海僵盟,进而在僵盟内部收购深海闪电鳗鱼。这样一来不仅方便,而且僵盟内部的价格绝对比对外出售要低廉。

%60
方源虽然心动,但他是在北原升的仙。他的仙窍是有北原的天、地二气凝聚成的。

%61
五域之间的天地气,都各有差异。只需稍稍测试,便一目了然,根本隐藏不住来历。

%62
若以北原仙僵的身份,加入东海僵盟,也不是不行。但加入北原僵盟,却关乎方源的一项大计。

%63
这项大计预期收益很大,以至于方源就算不成为仙僵,原本也想冒着风险伪装身份,混入北原僵盟。

%64
现在他成了仙僵,加入北原僵盟无疑更加方便。

%65
“加入东海僵盟虽然方便购买闪电鳗鱼,但再给太白云生一段时间交际,也总能买到,顶多代价大点。尽量不催用浪迹天涯仙蛊,也能再支持一段时间。”方源不愿放弃大计,最终回绝了太白云生的建议。

%66
稍微在狐仙福地休整了一下,方源便再次运用定仙游,赶赴中洲。

%67
众生书院位于山谷,夜幕下,亮点灯光,显得静谧祥和。

%68
洪易盘坐在床榻上,结束了每日温养空窍的修行。他睁开双眼,眼眸炯炯发亮,流露出一股期待和振奋之色。

%69
“今日大比,我已经挤进了前十之列,并未显露真正实力。这是天助我也,让我恰巧轮空,被人当做了幸运儿。明天是十强赛,他们轻敌之下,不会料到我的真正战力。后天是决战,只要我动用此蛊,突破对方的防御完全不再话下。”

%70
想到这里,洪易掏出二转蛊虫过得去。这是宙道蛊虫,十分珍稀。可攻可守,能让对手的蛊虫回到催发之前。

%71
譬如,对手凝出一道火球,在过得去的威能下,蛊虫回到发动之前,火球便乍然消失。

%72
不过,过得去蛊对真元的消耗极大,凭借洪易上等空窍,九成多真元,也最多只能催用一次。

\end{this_body}

