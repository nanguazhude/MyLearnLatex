\newsection{求合作长凡主动}    %第一百六十一节:求合作长凡主动

\begin{this_body}

%1
方源又询问地灵,有关东方一族的蛊师俘虏如何安置的。

%2
方源虽然杀了东方晴雨等人,但手头上仍旧有一批,大约有一二百的东方族人残存着。

%3
地灵回答,她临时开辟了一个地牢,就在地底深处,位于一座石人部族附近。又问是否将这些俘虏甩卖掉呢?

%4
这些人,并不好控制利用。他们不会乖乖听话,要逼迫他们去炼蛊,像毛民一样工作,一个不慎,反而会被人一心求死,进行破坏,引发麻烦。

%5
方源摇摇头,吩咐道:“地灵已先将他们手中的凡蛊,都搜刮干净,关押起来。每天喂一点食物,就当养宠物了。这些俘虏他还有其他用处,倒不忙着急处理。”

%6
“是,主人。”

%7
“我手头上,还有多少仙元石?”方源又问。

%8
“有一百三十七块。”地灵立即回答。

%9
方源点点头,不知不觉间,他的仙元石数量又再度突破了一百的关卡。

%10
最大的原因,就是胆识蛊的贸易量增长了一倍,每一个月不算让利,总共有九十二块仙元石的纯利润。

%11
因此,突破一百关卡就变得容易了。

%12
在这期间,方源还嘱咐地灵时刻关注宝黄天,若有便宜的狐群,就用低价拿下,放养在狐仙福地中,彻底发扬狐仙福地的长处,增添底蕴。

%13
小狐仙出手了几次,买下了两三批狐群。若非如此,现在积累的仙元石还会更多。

%14
做胆识蛊这种垄断生意,供不应求,赚取仙元石的确容易。

%15
方源思考了一下,保留二十块仙元石以防不时之需。取出七十块用来转化青提仙元。夺取智道传承的一系列激战,让他原本的七十七颗青提仙元也消耗殆尽,急需补充。

%16
剩下的都交给地灵小狐仙,让她重点主持炼制星念蛊的工作。交给她的仙元石,都是用来大量采购炼蛊原料的。

%17
“油水也就算了,是我打算参加炼蛊大会前期用的。气泡鱼也不会贩卖,用来增加星萤蛊的产出。散文鲤价值珍贵。但数量稀少。要等待合适的买家。但幽火龙蟒、长恨蛛群、龙鱼都是数量庞大,可以经营的项目。尤其是龙鱼,直接可以贩卖。”方源心中初步思量着。

%18
这三者,出现的很是时候。

%19
一直以来,方源的经济支柱只有一个,那就是胆识蛊贸易。

%20
垄断的买卖,简直就是抱上了一个黄金大粗腿。

%21
但毕竟只有一个。胆识蛊产量有限,方源的仙元石一直掐着用,精打细算。虽然摆脱了起初成仙时的危险状态,已经立住脚跟,做到了收支平衡。但却没有余力更进一步。现在的方源根本就不敢炼制仙蛊,没有底蕴实力。

%22
接下来,方源要不断前行,要摆脱仙僵身份重生,要推算各种蛊方、杀招,要应对各种外敌。要兼修智道,方方面面,零零总总,一根经济支柱肯定满足不了他越加庞大的需求。

%23
“如果能打造出这三个经营项目,那我就有四条腿,仙元石收益足够满足我当前的需求,大力推动我在各个方面的迅速进步了!到底是超级势力。底蕴丰厚。我搜刮之后,立即使我的前景为之开阔。”

%24
方源感叹的同时,又升起搜魂东方长凡的念头。

%25
这些产业,都是东方一族的经营机密。比方说幽火龙蟒怎么能成群结队地在一起,极大地增长繁衍效率。又比如龙鱼怎么豢养,吃什么食物,什么环境更适合龙鱼,生病了会怎么样。

%26
要形成一个经济支柱,十分不易。里面涉及到很多东西,有着各自的诀窍奥秘。当初方源为了培育星屑草,还向几位星道蛊仙购买了心得经验,付出了一笔仙元石呢。

%27
身为东方一族的太上大长老,东方长凡肯定知晓东方一族的经营机密。

%28
他的魂魄就像是一个巨大的宝藏,其价值绝非只是一个智道传承可以囊括的。

%29
于是方源第四次搜魂,企图得到这些经营奥秘。

%30
但这一次,他得到的是东方长凡的夺舍法门。不仅是东方长凡年轻时,在真传秘境中得到的原始版本,而且还有东方长凡改良之后,抽取同族蛊仙仙窍本源,立地成仙的全部方法。

%31
“用这种方法,我倒是也可以重生。需要相同的血脉……嗯,我那弟弟是否还活着?他就是一个现成的夺舍目标。不过就算死了,也不要紧,我还可以用血道手段,借助女子身躯,生出下一代。不过这种方法,会惹来天地愤怒,天灾地劫威力恐怖,东方长凡落到这样的下场,最主要的原因就在于此。而且我夺舍他人,就算以后可以升仙,我的新身子中,却没有第二个空窍了。”

%32
第二空窍的价值很大,方源一时间还不想放弃。

%33
东方长凡魂魄,见方源意动,忽然一改之前的沉默,再次波动起来,“说”道:“你身为仙僵,想要获得新生太难太难。这样下去,你的修行就止步于此,永远只是这般修为了。夺舍之法,十分适合你!当然,我现在知道了,这个方法招惹天嫉地怒,十分凶险。不过我可以改良方法,遮蔽天机,还可以只夺舍凡人,减少天地的愤怒,降低灾劫的威力。”

%34
“你直接用这个夺舍法门,就是送死,我就是你最好的例子。但是我现在有了最宝贵的经验,能够再次改良法门,一定可以成功。事实上,若非你们,我已经彻底成功了!怎么样,我们可以合作。”

%35
东方长凡向方源提出合作。

%36
他察觉出方源想要重生的想法,事实上,没有一位仙僵不想重获新生的。

%37
不待方源开口,他又继续道:“你我合则两利,分则两害。你还担心什么?我区区一个魂魄,还能陷害你不成?”

%38
言语间,不断鼓动方源,又隐隐带着激将的意味。

%39
方源就笑:“真不愧是东方长凡啊,之前沉默,还以为你已经彻底认输。没想到你一见到有希望有机会,就立即跳出来把握。这点值得我去学习。”

%40
“哦?我早说过,你我其实是同一类人,有着野心和抱负。”魂魄再次波动。

%41
但方源话锋一转:“正因为我们是一类人,所以我才不会留着你啊。你放心,一旦将你的剩余价值榨干,我一定会打得你神魂俱灭,彻底消散在天地之中。我们这类人,还是越少越好。”

%42
东方长凡魂魄一愣,他没有料到方源对他的杀意如此深重和坚决,他再次剧烈波动,仍旧不想放弃,强笑道:“哈哈哈,你骗不了我的,你已经心动了!”

%43
方源点点头:“我当然心动。我相信你必定有能力改良这个夺舍法门,因为你有大师级的血道境界、虚道境界,有宗师级的智道境界,还有最宝贵的夺舍经验。我在境界上,和你差距极大,就算掌握了你的智道传承,也根本不可能改良这个法门。”

%44
用蛊的境界,分为四等。

%45
分别为:普通、大师、宗师、大宗师。

%46
大师级,就是将蛊虫的运用,上升到艺术的层次。非得有个人的充沛才情,再加不断培养锻炼积累,才有可能上升到这一境界。

%47
宗师级,就是将一种流派掌握到十分艰深的程度,根基深厚到凭此能触类旁通,用自己的流派手段,达到其他流派的特点优势。

%48
大宗师则更加厉害,是占据巅峰的人物。不要说世间,就是纵览整个人族历史,也十分罕见。

%49
东方长凡是智道宗师,飞行准宗师,虚道大师,宇道大师,魂道大师,奴道大师,血道大师。

%50
方源是血道宗师、力道宗师、炼道准宗师、飞行准宗师、奴道大师、变化道大师。至于魂道、宇道、宙道、梦道等等都是普通平常,虚道更是没有涉及过,一片空白。

%51
一般而言,蛊仙基本上都至少是飞行大师。因为飞行十分重要,就算凡人时期不掌握,成了蛊仙之后,一定都会大量练习,掌握飞行能力。

%52
别看方源有两道流派,都是宗师,一是前世五百年积累,二是今生利用重生优势,不断冒险,又机缘巧合,才能成就如斯。

%53
术业有专攻,方源的境界不足以推算改良夺舍法门,但东方长凡却是专业对口的。

%54
也正是因为这点,方源不能做到,东方长凡才能做到。所以,他方有底气劝说方源,请求合作。

%55
搜魂只能获得记忆、秘辛等等,却不能汲取境界。

%56
境界是个人独特思维,个人感情经历,个人用蛊的心得体会等等,总结起来,凝练提炼成精髓,是对天地大道的感悟。

%57
当今唯有一种方法,能够获取他人境界,那就是梦境的成功探索。

%58
方源虽然也知道,和东方长凡合作,会有不凡收获,但他更忌惮东方长凡这个人。和这种智道蛊仙合作,不知不觉间就会设计,被出卖,被算计。

%59
尤其是方源是八十八角真阳楼倒塌的真凶祸首,外部情势十分恶劣,若东方长凡知道,利用了这点,必定能陷方源于死无葬身之地。

\end{this_body}

