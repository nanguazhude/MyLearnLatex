\newsection{虎头蛇尾的攻防战}    %第二百三十七节:虎头蛇尾的攻防战

\begin{this_body}

%1
琅琊地灵脸色剧变,难以置信地看着眼前的这一幕。

%2
“毛十二,毛十三,你们俩怎么回事?”毛民蛊仙们又惊又怒,其中一位大声喝骂。

%3
“你们身为毛民,却帮助外敌,背叛自己的亲族!简直罪无可恕!!”

%4
“不可能的。毛十二、毛十三是福地里土生土长的毛民,怎么可能叛变?这一定是幻术,大家不要被眼前的假象给欺骗了!”

%5
有的毛民蛊仙,几乎都不相信。

%6
众人视线焦点的两位毛民蛊仙,却是一脸高傲的姿态,双双拱卫着秦百胜,一副不屑解释的傲慢。

%7
他们俩的叛变,对琅琊地灵一方,简直是一记致命的打击。

%8
尤其是那些毛民蛊仙,心都乱了,更加难以操纵战阵。

%9
黑城得到极其珍贵的喘息之机。

%10
秦百胜则在毛民叛徒的配合下,收取蛊虫的速度更快了一筹。

%11
琅琊地灵脸色黑的像锅底似的,十分难看。

%12
他将一口牙齿压得嘎嘣作响,从牙缝中挤出话:“杀,杀过去!杀死这毛民中的两个叛徒!就算他们叛变捣鬼,但炼炉仙蛊屋仍旧是我的意志,不会完全停止。它还有威能,还在运转!诸位,琅琊福地的生死存亡,就在你们的手中,我们已经没有了退路,给我杀!!”

%13
琅琊地灵竭尽心力鼓舞士气。

%14
毛民蛊仙们就算再天真,此刻也意识到了情势的危急,纷纷下定必死的决心,全力抗争。

%15
银色巨人气势滔天,重整旗鼓,再战仙蛊屋黑牢。

%16
几个回合下来,黑牢连连溃败。

%17
黑牢中,黑城一头长发已乱,双眼充斥血丝,头晕目眩,操纵黑牢却已渐到极限。

%18
原来,这上古战阵还有一个弊端。

%19
那就是想要让战阵发挥威力,需要参阵蛊仙提前进行大量的练习。而方源等人,尚是第一次组合成阵。

%20
这上古战阵天婆梭罗,虽然有众多蛊仙参与,但真正主攻的只有方源、墨坦桑、琅琊地灵三位。这就好像明明有两个拳头,但对战时只用其中的三根手指头戳人。毛民蛊仙们不擅战斗,大大拖累了方源等人,使得银色巨人显得蠢笨异常,很多战机明明已经出现,却把握不住。

%21
不过经过之前的一番激斗,众仙之间已经可以操纵熟练,并且培养出了一点点的默契。

%22
就是这点默契,使得银色巨人抓住了两次战机,将黑牢打得碎片飞舞。

%23
黑城不得不退。

%24
两位毛民蛊仙飞上前去,为秦百胜掩护。

%25
但银色巨人横冲直撞,脚步只是稍稍凝滞一分,整个身躯宛若山峦,向秦百胜撞去。

%26
秦百胜大吐一口鲜血,中断收取过程,躲避开来,整个身体向高空迅速拔升。

%27
“给我追!”琅琊地灵大声下令。

%28
银色巨人追杀秦百胜,但此时黑牢撞来,速度很快,角度刁钻。负责防守的几位毛民蛊仙没有及时反应,应对不及,银色巨人被撞了一个跟头。

%29
银色巨人狠狠反击,一巴掌将黑牢拍飞老远。

%30
但这时,从远方飚射出两道身影,是回风子、贺狼子赶来增援。

%31
与此同时,两位毛民蛊仙也仿佛苍蝇一般,围绕着如山般的银色巨人,一遍迅速移动,一边不断轰炸,竭尽全力进行干扰。

%32
四位蛊仙,一座六转仙蛊屋,和上古战阵天婆梭罗展开激战。

%33
随着时间推移,毛民一方众仙配合越加默契。银色巨人越战越强,哪怕对手拼命阻击,也难以阻挡银色巨人的脚步。

%34
属于上古战阵的真正威能,正一点点地释放出来。

%35
秦百胜不得不主动中断了三次,转移阵地躲避银色巨人的追杀,脸色变得十分衰败苍白。

%36
仙道杀招万我大手印!

%37
激战中,方源双眼精芒一闪,忽然出手。

%38
他低调了这么久,在这一刻终于亮出獠牙,使出杀手锏!

%39
银色巨人的四只手臂,高高抬起,猛地落下。

%40
轰轰轰轰!

%41
爆裂的巨响在顷刻产生,空气被挤爆,四只力道大手宛若山峦,势大力沉地撞过去。

%42
这一瞬间,战阵中的蛊仙们直接消耗了上百颗的仙元!

%43
受到上古战阵的增幅,力道巨手的威力得到史无前例的暴涨。

%44
“不妙!”回风子立即使出风遁,拼命逃窜出去。

%45
贺狼子变身的巨狼,也连忙转身挪移。只是擦了力道巨手的边,巨狼发出一声惊天的惨嚎,被撞到的身子骨彻底粉碎,那部分的血肉直接成了一滩烂泥。

%46
黑城微微一犹豫。

%47
两道力道大手一左一右,宛若两只大手合十,像是拍蚊子一样,将黑牢拍进手掌心中。

%48
轰隆!

%49
一声巨响,两只力道巨手崩散。

%50
黑色流星外形的仙蛊屋黑牢,摇摇晃晃,向场外飞去。

%51
仙蛊屋不是蚊子,不会被拍死。但黑牢此时的惨状,却叫目睹的蛊仙们都纷纷倒吸一口凉气。

%52
这座仙蛊屋,发生了剧烈的形变,原先是圆球,现在两侧被拍扁,一道巨大的裂缝几乎在黑牢上蔓延一圈。

%53
一股剧烈的浓烟,从这裂缝中冒出来,时而有一团团火光在黑烟中爆发。

%54
“是那个杀死雪松子的七转力道蛊仙!”回风子不愧是当今北原飞速第一人,躲避开来,回望战场,为这力道巨手心惊胆战。

%55
银色巨人中,毛民蛊仙们呆了呆,旋即爆发出热烈的欢呼声。

%56
琅琊地灵先是喜色浮上脸面,但很快目光一闪,想到了什么,看向不远处的方源。

%57
方源面色苍白,头上冷汗密布。他几乎都站不住,身躯摇晃,摇摇欲倒,表现出一副超常发挥后,承受严重反噬的样子。

%58
琅琊地灵心中怀疑顿时冰消瓦解,再次将注意力集中在秦百胜这个大敌身上。

%59
秦百胜已经不在原先位置,一只力道巨手杀过来,他不得不再次中断收取仙蛊的过程,进行躲避。

%60
这一次,他面如金纸,身上的伤势似乎累积起来,达到质变。

%61
在方源等外人看来,秦百胜气势不再,虚弱不堪。

%62
“这一次暂且就到这里,下一次我再来,可不是这样的结果了。哼!”秦百胜抛下一句狠话,率先飞退。他似乎伤的很重,放弃了炼炉仙蛊屋剩下的部分,开始撤离战场。

%63
琅琊地灵一声怒吼,展开追杀。

%64
秦百胜等人却准备充足得很,汇合之后,打破空间,逃离了琅琊福地。

%65
琅琊地灵却不能飞出这片福地,只能含恨不已,望着秦百胜一行人带着大半个炼炉,远走高飞。

%66
“为什么要撤?炼炉还没有彻底到手,这和我们之前的计划不符啊。”一行人在高空急速飞行,姜钰仙子向秦百胜秘密传音询问。

%67
秦百胜脸色凝重:“情况有变,落魄谷正遭受强敌攻打。必须回去!那只仙蛊正在孕养,即将成功。一旦失去,对整个大计会将有严重影响!”

%68
姜钰仙子心头一震,又问:“那琅琊福地这边怎么办?”

%69
“我建议先留着吧,那里的毛民价值很大。”这时,两位毛民蛊仙加入了讨论。

%70
“只要有仙道杀招魂穿在手,我们就能穿越进去,夺舍潜伏,再造几个内应完全不成问题的。”

%71
“可惜,时间不够,只积累了三位毛民蛊仙内应。如今我们两个已经暴露,想要再积累回来,至少要有三百年!”

%72
“虽然没有露出什么马脚,但上一任的琅琊地灵似乎察觉到了什么。因此平日里只让我们蛊仙炼蛊,根本不给我们自由出入的权利,以及试演战斗,提高战斗力的机会。”

%73
两位毛民蛊仙你一言我一语。

%74
秦百胜思考了一番,沉吟道:“也罢,虽然还有一个内应剩下,但当务之急,是护住落魄谷,直到那只仙蛊孕养而成。这只仙蛊,关乎我们的大计,绝不容有失!至于炼炉仙蛊屋,也抢了大半,可算是仙蛊残屋。没有达成原先的目标,不妨就将黑牢拿来充数罢。”

%75
“好!”

%76
“等到良机,就一齐动手。”

%77
影宗几人,纷纷用隐晦的目光打量了身旁的黑城一眼后,不再传音。

%78
“搬家,必须要搬家了!”而此刻,琅琊福地中,琅琊地灵口中嚷嚷,下定了决心。

%79
长毛老祖已经逝去,琅琊福地已经成无主福地,居然还可以搬迁?

%80
这和常理大大不符。

%81
“我自有手段,能够将整片福地搬到另外一个地方。只是搬迁过程中,天地二气动荡不定,代价很大。”琅琊地灵为方源解惑。

%82
方源心中暗暗感叹,长毛老祖底蕴之雄浑深厚,刷新了他的认知。

%83
“这一次能够退敌,你们二位出力甚多,尤其是方源你!”琅琊地灵又道。

%84
墨坦桑瞪大双眼,吃惊地看向方源,至此他才知道原来眼前此人,就是他认识的方源。

%85
方源毫无身份被揭破的尴尬,向墨坦桑轻轻一笑。

%86
“唉。”琅琊地灵发出一声叹息,“看来我大毛民一族崛起之路,还很长很艰难。路程是曲折的,前景是光明的。你们跟着我混,是绝对没有错的!这一次,我很满意你们俩的表现,我要大赏你们!!”

%87
旋即又道:“没有草料,再糟糕的劣马也驱使不动啊。就算是狗,也得给骨头,才能让它叫得欢。啊呀,怎么又把心里话说出来了……”

\end{this_body}

