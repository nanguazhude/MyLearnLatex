\newsection{夜叉章鱼和口蚯}    %第二百八十四节:夜叉章鱼和口蚯

\begin{this_body}

在沉默中,方源等仙僵,已经深入到地沟深处,距离地表有四万多丈的距离。

虽然尽量避免去战斗,但大大小小的战斗,也足以进行了十六场。

仙僵们或多或少,都身负了一些伤势。

“不能再往下深入了。”夜叉龙帅道,“这种深度,已经是地壳蜗牛的活动范围,我们往四周探索。”

北原僵盟分部,最多探查到五万多丈的深度。

现在距离这个极限,还有将近一万的差距,但越往下,凶险越多,荒兽林立,依照方源等人的阵容,应对起来,越加困难。

夜叉龙帅的提议,没有仙僵反对。

选择了一个方向,众人继续默默飞行。

也许是方源在临行之前,动用时济运,提升过一时的运道。总之,不久后,他就遇到了一头地壳蜗牛。

“很好。”方源做出大喜过望的样子。

“先别着急,侦查一下,看看周围潜伏着多少危险。”夜叉龙帅却很沉稳。

众仙各展手段,很快探查清楚,这一片领地,由一群规模不少的夜叉章鱼把守着。

“我要处理蜗牛黏涎,必定要发动星光,十有*会惊动这群夜叉章鱼啊。”方源露出为难之色,看向夜叉龙帅等人。

众人脸色皆是微微一沉。

“你怎么不早说?”雷雨楼主不满地道。

“兴许能有什么方法。可以遮蔽这股动静……”玄阴医师道。

方源摇头:“这不可能。我也想遮蔽动静,但这已经超过了我的极限。其中具体手法,我也不方便透露。总之。还是先铲除了这群夜叉章鱼最好。”

方源态度强硬,理由也十分充分。

其余仙僵面面相觑一番,最后夜叉龙帅只好妥协:“那就先打杀了这群夜叉章鱼,再由星象子出手。”

这一次,由夜叉龙帅亲自侦查。

盘踞在这里的章鱼群,有八头荒兽章鱼,还有一头章鱼王。有七转蛊仙的战力。

“林大鸟负责外围,调动鸟群。将整个章鱼群包围。”

“由鼋姥姥牵扯注意力,吸引它们的注意。”

“然后由我和雷雨楼主出手,杀伤这些章鱼。”

“至于玄阴医师你,站在外围策应吧。”

夜叉龙帅很快布置完毕。竟然没有方源什么事。

夜叉章鱼可谓遭受到了天降横祸,仙僵们偷偷接近,以有心算无心。因此,战斗甫一打响,夜叉章鱼群就受到凶狠的打击,陷入慌乱之中。

仙僵一方很快建立优势,处于上风。

片刻之后,战斗进入尾声,章鱼被斩杀将尽。只剩下两头。

这时,夜叉龙帅蓦地对方源道:“星象子,你也出手吧。”

方源目光一闪。悠然上场。

仙道杀招星蛇索。

他先将一头状态较好的夜叉章鱼困住,然后施展六幻星身,团团围住另外一头。

刷刷刷……

星光灿烂,方源攻势连绵不绝,很快将这一头本就奄奄一息的夜叉章鱼打死。

随后,他掉转枪头。对付剩下的最后一只。

夜叉章鱼半人半兽。上半身是一位雄壮的男子,肌肉贲发。皮肤漆黑,硬如钢铁。下半身则是数十条章鱼触手,取代了腿的作用。

这最后一头夜叉章鱼,状态较为完好,凶悍勇烈,一时间和方源激战,陷入胶着状态。

方源便用星云磨盘防守,等候良机,一次次出手,陆续将夜叉章鱼的触角铲除割断。

他在战斗的时候,其余仙僵都袖手旁观。

最多是林大鸟,操弄鸟群,将屡次想要逃走的夜叉章鱼纠缠住,让身后的方源跟上,继续打。

方源耐心十足,故意将夜叉章鱼重创,流淌出一股股漆黑冰冷的血液。

“可以了,停手吧。”最终,夜叉龙帅看不过去,亲自动手,生擒了这头夜叉章鱼。

之前的几头,都因为战斗不能留手,都打死了。

这头章鱼,夜叉龙帅生擒活捉,准备豢养在自家的仙窍里。

这一次出手,让其余仙僵明白了方源的战力。

方源只暴露出星道方面的手段,即便如此,战力也是六转中的强者。对方源一直不满,颇有微词的雷雨楼主,也沉默下来。

后者暗暗估量,若他和方源单打独斗,还未必是方源的对手。

蛊仙界强者为尊,雷雨楼主就算再不满,也不在表情上流露了。

接下来,方源便开始收集星夜黏涎。

他要求仙僵们后退数里,并且不得用侦查手段,来偷窃自己的独门手法。

仙僵们自然心中不愉快,暗骂方源小肚鸡肠,但也不得不答应下来。

引动星芒处理黏涎的动静,并不小。

很快,地壳蜗牛就停止了蠕动,整个身体都缩进壳里。

一感到危险,地壳蜗牛就会如此应变。

它的壳厚实无比,虽然只是荒兽,但叫绝大多数的六转蛊仙都为之无奈。就算是七转蛊仙,想要打破它的壳,也是要费一番手脚的。

并且地壳蜗牛体大如鲸,重比山岳,一缩进壳里,往往要躲个一二年时间。若是外在动静再大一些,这个时间将会呈倍增长。

若是将其搬走,只要脱离地壳蜗牛的活动范围,哪怕是仍旧在地沟中,稍稍高出五六里的距离,地壳蜗牛都会命丧当场。

如此一来,众仙僵就只好舍弃这只,转向他处。

地沟中,并不仅仅是夜叉章鱼。

还有蝙蝠荒兽。攀壁魔猿,地乳通草等等荒兽荒植。

接下来的旅途中,众人又陆续碰到三头地壳蜗牛。

其中两头。方源都善加利用,尽量采集了星夜黏涎。剩下一头,却是行走在上古荒兽赤炎蛇的领地中。

这种怪物浑身上下,充斥着暗道、炎道、土道道痕,极其难缠,又善于打洞,在地沟峭壁中穿梭。速度很快。

一旦打斗起来,动静根本遮掩不住。所以方源等人只好放弃。

在方源有意无意的引导之下,众仙僵渐渐接近方源的目的地。

“前方发现大量的夜叉章鱼!停止前进,好多头夜叉章鱼,初步估算足有三十头!”林大鸟忽的语气急促地汇报道。

“那就离开这里。掉转方向。”夜叉龙帅立即下令。

三十头荒兽,就是三十位六转蛊仙战力。如此庞大的族群规模,必定还有至少三头的上古级的夜叉章鱼统御。

方源虽然心中并不情愿,但无法公然反驳,只好暗暗将此处记下。

众仙僵方向一折,向东南飞行。

片刻后,发现了第四头地壳蜗牛。

这头蜗牛,看起来有些老迈。背着土黄色的壳,白色的软体在地沟峭壁上蠕动。速度很是缓慢。

“运道来了。”侦查之后,林大鸟笑了一声,“这片区域居然没有什么危险。星象子先生大可随意施展手段。”

方源点头不语,凝神观察。

忽的,他眼中精芒一闪,认出此地的凶险。

原来此地看似安全,实则充斥土道道痕,一旦过于接近。就会被强行吸附到峭壁之上。

体重越大,吸附力就越强。

“荒兽大多体型庞大。重量惊人。一旦陷落此地,必定动弹不得,活生生的饿死。地壳蜗牛重如山岳,这头蜗牛却居然能够在缓慢地行走。看来它不是老迈,而是上古荒兽级的地壳蜗牛!”

发现这点,方源心中便活泛起来。

他飞过去,故意被吸摄过去,发出一声惊叫。

“糟糕,是重土带!”

“明明五万丈之后,才有的凶险地貌,居然这里也有了。”

“小心!这种重土带里,往往还潜伏着口蚯!”

一时间,众仙僵惊呼连连。

距离重土带越近,方源就感到吸摄的力量越强,速度激增。

砰的一声,方源摔落到峭壁上,双腿深陷土壁之中,神情惊惶。

仙道杀招星火遁。

方源旋即使出这个移动杀招。

顿时,他身上燃起星火,飞跃而起。

“不要飞!”夜叉龙帅大吼。

但他说的时候,已经迟了。

轰的一声巨响,一头巨大的口蚯,猛地从土中窜出来。

嗖的一声,地壳蜗牛缩回壳中,一动不动。

大量的土石翻飞,口蚯张开巨大的口器,足以同时吞下四五头的地壳蜗牛。

方源大吼:“谁来救我!”

话音刚落,就被口蚯一口吞下。

“小心!口蚯体内全是利齿,尖锐无比!”

“坚持片刻,我们定会救你出来!!”

众仙僵慌了神,同时动手。

方源若有个闪失,他们怎么向焚天魔女交代?

攻势凌厉至极,一下子就把口蚯重新打进重土带。

这头奇异的上古荒兽,捕食的时候,身躯庞大无比。一旦吞下猎物,它的身躯就好像是漏了的气球,体积急速缩减。体内肉壁上的密密麻麻的利齿,就会不断旋转,将猎物切割绞碎成肉渣血水。

但方源一入口蚯肚中,就催发星道杀招,撑住肉壁收缩。

然后,立即催动定仙游仙蛊。

他之前暗暗记住了那边的景象,三息之后,定仙游将他带出险境,逃之夭夭。

而夜叉龙帅等人,还误以为方源仍旧陷落于口蚯之中,疯狂攻打。

方源远离了夜叉龙帅这些仙僵,但是定仙游的动静,也惊动了那群规模庞大的夜叉章鱼。

方源微微一笑,变作一头夜叉章鱼,疯狂攻击。

片刻之后,他成功勾引着愤怒的夜叉章鱼,带领他们杀回夜叉龙帅处。

夜叉龙帅等人,自然窥探不破方源的变化,还以为是夜叉章鱼内部争战,逐出最弱的族群成员。

他们大呼倒霉,但星象子陷落口蚯腹中,又不能飞离。

于是,夜叉章鱼、北原仙僵,还有口蚯之间的混战,徐徐拉开大幕。

ps:蛊真人公众号上会推送人族传和发布其他一些东西,欢迎大家关注搜索蛊真人公众号,添加即可.(未完待续)

\end{this_body}

