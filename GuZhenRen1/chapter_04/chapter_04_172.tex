\newsection{福地攻伐浅论道}    %第一百七十二节:福地攻伐浅论道

\begin{this_body}



%1
太白云生靠着椅背,脸上一愣,旋即笑骂:“师弟,你消遣我!这话我听得耳朵都起老茧了,从开始蛊师修行,便充斥耳畔。甚至就算是凡人,也是耳熟能详。”

%2
方源点头,语气中流露出一丝傲然,再次开口:“这话的源头,乃是人族历史上的第一位九转蛊仙,元始仙尊所言。也正是他开创了中洲天庭霸业,一直屹立至今,纵使沧海桑田,万载悠悠也是不倒。此言精辟到了极点,可谓蛊师修行的金言玉律,流传之广,在如今甚至还要超过《人祖传》。但在我看来,其实这话还不算精辟,我将其总结成一个字,便是道。”

%3
太白云生闻言,不禁坐直了身体,目光炯炯地看向方源,又说了一声:“愿闻其详。”

%4
方源便侃侃而谈道:“人是万物之灵,在自然万物之中,人最有灵性、悟性,可以感知天地,体悟大道。蛊是天地真精,蛊在万物当中,十分独特,凡蛊体内蕴藏着些微道理的残屑,仙蛊体内承载着大道的碎块。因而,堪称天地的精华。”

%5
“什么是道?水往低处流淌,人不吃食物就会饿死,青木因为雨水和肥沃的土壤而茁壮。这些就是道。蛊师借助蛊虫。凭空点火,逆流河水,迸发雷电,治愈创伤,这些也是道。”

%6
“人养蛊、用蛊、炼蛊。就是借助蛊虫,使用出天地的一部分威能,进行生存繁衍,最终体悟天地的道理法则。成为蛊仙之后,空窍化为仙窍,身上更刻印下道痕。可以说,蛊师养蛊便是养‘道’。蛊师用蛊便是用‘道’。蛊师炼蛊便是炼‘道’。蛊师万千流派、力道、炎道、宙道、毒道,都是先贤们前仆后继,于茫茫无知中,踩踏出来的修行之路。”

%7
说到这里,方源顿了顿,这才道:“所以一句话,蛊师修行。便是修道。”

%8
“蛊师修行便是修道……”太白云生坐在椅子上,皱起眉头,在口中咀嚼方源的话。

%9
他越是咀嚼,越是觉得方源的总结真是精辟万分。

%10
简直是直达本质,对他而言,竟有一种拨开云雾,见得青天之感!

%11
这时,方源又再次开口:“回过头来,再说蛊仙修行的根基仙窍。仙窍是福地,是洞天。是一方小世界,也蕴藏着道。当然这种道,不能和五域天地相提并论,充其量只是拥有道痕。”

%12
“就拿我的狐仙福地而言,它有基本的宙道道痕,所以才能引进光阴长河的支流,福地中有时间的概念。有光阴在流淌。它还有比较多的宇道道痕,所以福地范围较广,就算是失去了北部那一大块,也承载了诸多资源。地灵小狐仙还能随意瞬移。宇道道痕稀少的福地,往往范围较小,地灵却是不会瞬移的。它拥有最多的是奴道道痕,正是因为这些道痕的存在,才使得在狐仙福地中,豢养狐狸更加容易,驾驭相似的兽群,乃至荒兽更加有效。”

%13
太白云生连连点头,开口赞同道:“不错,是这个理儿!我的太白福地,宙道道痕最多,因而时间资源最为丰富,外界一天,我福地中便是三十三天。土道道痕最少,因此福地中无山,土壤也不肥沃。宇道道痕算是平常,结合起来,地域不宽广,天空却很高很阔。”

%14
他一边说着,一边眼中精芒闪烁,已有所悟。

%15
“所以我说,福地攻防,只看一条,便是福地蕴藏的道痕多寡纯杂。”方源继续道。

%16
“福地自成一界,为什么能压制外敌,能禁锢蛊虫?原因就在于福地中拥有大量的道痕,宛若一处池塘。凡蛊蕴藏道的残屑,就如同一滴水,一点火。一滴水入侵池塘,反被融汇。一点火入侵池塘,立即熄灭。所以一旦发动福地中的道痕,形成天地伟力,凡蛊基本上就没用了。”

%17
“为何仙蛊可以不受福地禁锢呢?因为仙蛊是法则碎片,已经自成一体,宛若石头。砸入池塘当中,尽管水波荡漾,却不能阻止石头在其中穿行。不过虽然不能禁锢,压制还是行的。仙蛊在福地中作战,威能效用会受到福地道痕的增幅和削弱。”

%18
“再谈蛊仙。蛊仙进入侵福地,也受道痕的反制。但蛊仙拥有仙体,和凡人有本质上的不同。何谓仙体,说通透一点,无非就是身上刻印了道痕!蛊仙入侵福地,福地的道痕越多,蛊仙就越受压制。相反,蛊仙身上的道痕越多,福地的压制就越少。”

%19
方源前世,成就血道蛊仙。很多太古九天的碎片世界,他都进入不了。因为这些碎片世界,道痕稀疏,比他身上的道痕要少很多。方源进入其中,宛若猛虎挤进兔笼,唯一的后果就是撑爆碎片世界,导致碎片世界道痕销毁,刮起大同风,吹毁同化碎片世界中的一切资源。

%20
又比如群魔和东方长凡,在碧潭福地大战。

%21
东方长凡发动地利,导致蛊仙的凡蛊无法使用,但却禁不了仙道杀招。

%22
这都是因为碧潭福地是公共福地,虽然道痕很多,但相当驳杂。东方长凡不能调动全部道痕,只能催动他掌控的那部分福地,因此威能便有限了。甚至都没能令飞行中的蛊仙坠落到地上。

%23
不可能所有的蛊仙,都用仙道杀招在飞行。

%24
但蛊仙身上也有道痕,抵御着福地道痕的压制。魔道蛊仙们针对东方长凡的凡道杀招没有了效果,但对于自身的凡级增幅。却是仍旧有效的。

%25
“真知灼见,真知灼见!”太白云生恍然,感叹道,“我明白了。公共福地通常防御薄弱,为什么?是因为它的形成。是无数蛊仙切分出私人福地的一小部分,相互拼凑而成。这种福地道痕驳杂,相互掣肘,往往不能将全部力量统一起来。宛若人的五指,只能单独行动,不能攥成一个拳头。”

%26
“来源统一的福地,道痕也就是一个整体。若是有地灵。便能消耗仙元。调动道痕,形成天地伟力。仙蛊是道理碎片,荒兽身上有道痕,蛊仙更不用说,仙蛊屋是许多仙蛊的结合,仙道杀招是将数种道痕集合在一起,形成多种效果。或者放大某种威能。”

%27
“所以不管是福地还是洞天,道痕越多越纯,通常便越难攻陷。地灵、仙元,都是让道痕发挥的辅助之物。想那王庭福地,若非地灵和八十八角真阳楼对战内斗,否则真的是固若金汤,怎么可能毁灭呢?”

%28
说完这番话,太白云生离开座位,站直了身躯,面向方源。缓缓一礼:“今日听方源你一番话,真是叫人眼前空明,一扫尘埃,所获良多。从今以后,我不敢再称呼你为师弟了。达者为师,你在修行路上,是走在我的前面的。”

%29
“呵呵呵。老白,今日不过是咱们随意交流。你也不必谦虚,这种道理相信你也多有体悟,只是没有我清晰罢了。”方源摆摆手,云淡风轻。

%30
太白云生满脸肃容:“不,这道理看似浅显,却提纲挈领。‘道’之一字,的确比‘人万物之灵,蛊是天地真精’此句精辟。我以前是懵懵懂懂,现在才觉出蛊师修行的本质。这样的道理,恐怕寻常的蛊仙,都领悟不得。”

%31
方源微微一笑。

%32
太白云生此言,倒并不夸张。

%33
事实上,他能明悟这个道理,也是多亏了穿越的优势。

%34
前世的前世,地球上科学盛行。什么力学原理,三角公式,加速度,密度比热容等等,不都是天地自然的道理吗?

%35
从这点来看,科学也是一种修行的一种方式。“给我一个支点,我能撬动地球。”这样的名言名句,放到蛊师世界里,无非是“给我一只蛊虫,我能发光发热。给我全部蛊虫,我能无所不能!”

%36
正是因为两个世界的对照,才使得方源有着不一样的角度,来观察这个世界,体悟到旁人难以体悟的道理。

%37
而在蛊师的世界,难道没有人领悟出这层道理吗?

%38
怎么可能!

%39
领悟这个道理的大有人在,元始仙尊就是其中之一。“人是万物之灵,蛊是天地真精”就是“道”的更加详细通俗的解释,元始仙尊说这话,是有教无类,传承流派,出发点不同罢了。

%40
只是能领悟这样的道理的,基本上都是大人物,占据巅峰的蛊仙,回顾自己一生的修行时,体悟总结而得。

%41
平常的蛊师,在红尘中打滚,通常的蛊仙,又要为灾劫殚精极虑,哪有时间精力去细想这些“无用”的东西?是否领悟这层道理,又不影响他们的生存。

%42
说到底,就是思维角度。

%43
但却不能小看这种思维角度。

%44
火药在中国古代,当做炼丹用途。流传出去,却形成了枪炮。华夏文明的璀璨不容置疑,但地球现代自然科学的盛行,西方文明贡献最大。

%45
看待自然万物的思维角度不同,就能形成两个不同的文明。

%46
文明之间,难以评价优劣,只能说擅长的东西不同。

%47
因此蛊师文明中的太白云生,听了方源这番话,感触深刻。站在他的角度,就感觉到方源的境界要远高于他。所以再不敢称呼方源为师弟。

%48
正所谓达者为师。

%49
再称呼方源为“师弟”,太白云生心中就会感觉惭愧。

%50
有一种情怀,叫做自愧不如。说的就是太白云生现在这个状态了。

%51
两人又交流一阵,太白云生便向方源告辞。

%52
虽然两人之前一直都有通信,但当面交流之后,一层隐约的隔膜已经烟消云散。

%53
方源两世为人,奴下手段也颇为了得。轻描淡写的一场“论道”,就让太白云生印象深刻,甘拜下风了。

%54
(小说《蛊真人》将在官方微信平台上有更多新鲜内容哦,同时还有100\%抽奖大礼送给大家!现在就开启微信,点击右上方“+”号“添加朋友”,搜索公众号“qdread”并关注,速度抓紧啦!)

\end{this_body}

