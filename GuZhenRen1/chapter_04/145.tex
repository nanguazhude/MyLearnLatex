\newsection{内患乱战明进退}    %第一百四十五节:内患乱战明进退

\begin{this_body}

思虑一定,自在书生气势一振,长啸起来:“战!对方区区一人,是孬种的就自己单独退走吧。5,▲.2≈3.♀”

“好,必跟随大人左右!”

“真不愧是自在书生大人,有您在,东方余亮算得了什么?”

众魔齐呼。

自在书生一甩长袖,杀招爆发,清空周围萦绕的星光,率先冲向东方余亮。

“来得好。”东方余亮丝毫不惧自在书生的七转威势,不退反进,深入星光之中。

一时间,星光沸腾,宛若星海掀澜,惊涛不绝,骇浪滚滚。

东方余亮成了众星之主,驾驭万千星光,掀起壮阔星澜,竟以一敌众!

这一阵好杀,直杀得星斗飘摇,光屑飞溅。

东方余亮起先寡不敌众,岌岌可危。但随着时间推移,星光数量不断暴涨,几乎盈盈一团。

星光遮蔽视野,干扰配合,星念屡屡侵蚀脑海,魔道蛊仙不免失误连连。战局越是混乱,东方余亮越是有利可图。

众仙被误伤了多次,渐渐不敢配合,趋于单打独斗,更大减东方余亮的压力。

唯有自在书生,纵横捭阖,无法可挡,是东方余亮的最大强敌。东方余亮始终避其锋芒,不敢与其正面交锋。

自在书生屡屡被其他魔道蛊仙阻挠,心中又气又怒:“这东方余亮滑不溜秋,年纪轻轻,简直比当初的东方长凡还要奸猾!他毫无破绽,仿佛从生下来就操练了这个杀招!东方余亮的传承,竟会如此神奇?”

不远处。残阳老君也为这万星飞萤的威力暗暗吃惊。他一直着隐藏形迹,静静疗伤。不时转动双目。幽幽地望着战场边缘的方源等人。

他对方源的身份隐隐有所猜测,静待方源跳入战场。

他并不长于追逐。但若借助万星飞萤,擒杀方源的把握就大为增加。

但方源始终站立不动,袖手旁观,叫残阳老君暗暗焦急,苦熬着耐心。

仙道杀招千解!

自在书生久战不下,终于动用压箱底的攻伐手段。

东方余亮轻笑一声,早就防着他这一招,身体一晃,灵动如蝶。躲过目光照射。

但就在这时,东方余亮身形猛地停滞,从他体内传来东方万休的怒吼声:“老贼,我要和你同归于尽!”

说着,“东方余亮”竟改变方向,合身向杀招千解撞去。

这异变,出乎所有人的意料。

千解之强,深入人心,“东方余亮”一头撞上去。简直就是自杀!

一时间,众人惊愕,忘了反应。

就连自在书生也是愣了愣,眼看着“东方余亮”撞上自己的杀招。

千解之光。照射在东方余亮的头顶上,立即将东方余亮的头发化解得一干二净,随后有消融脑壳。直探脑髓。

生死存亡之间,残阳老君大吼一声。再也顾不得疗伤,显出身形。出手相救。

他一手攻,一手守。

火焰奔腾,掀起热浪滔滔,冲向自在书生。面对七转强者,自在书生不得不应付,分了心神。

残阳老君顺势,将东方余亮拉了回来。

东方余亮的脑壳,刚刚被穿透,脑髓差点就受到极其严重的损伤。此时,无数的星念仿佛是喷泉一般,顺着脑壳上的破洞,向外边喷涌着。

被残阳老君救下后,东方余亮神情一定,却是东方长凡的魂魄重新掌握住了身体。

他的脸上也不免涌现出惊恐后怕的神色。

若是稍稍晚了一息,东方余亮的整个大脑,都会被化解了去。

东方长凡可以说是在鬼门关走了一遭,生死一线,没有残阳老君,恐怕就再次死亡了。

“你怎么搞的?!”残阳老君救下东方长凡,一边拉着他后退,一边犹有心悸,在后者耳边大吼。

关于夺舍之法,东方长凡只教了他大半部分,还有一小半,却是最后的关键步骤,都被扣押在东方长凡的手中。

因此,残阳老君绝不愿意看到东方长凡的死亡。

东方长凡有意的保留,算计到了残阳老君,此时收到了效果。

但虽说保住了自己的性命,东方长凡却是脸色铁青,相当难看。

这一切都是他仓促升仙,在雷劫下强撑过去的后遗症。

原来,东方长凡将东方部族的八位蛊仙诓骗,让虚化大阵成功抽调出八股仙窍本源。

但这本源,蕴藏着八位蛊仙的意志、气息,却是不能直接灌注到东方余亮的身上的。

因此,东方长凡将这些本源,通过经血仙蛊的作用,化为血幕。又利用魔道蛊仙的攻势火力,进行“锻打”,将本源中的八位蛊仙的意志、气息都当做杂质,排除毁灭。

经过这一过程,东方长凡再将纯净的仙窍本源,灌输到东方余亮体内,就没有后患了。

但天地灾劫实在太过恐怖,远远超出东方长凡的估计。结果导致险情发生,生死存亡的关键时刻,东方长凡不得不省去血幕锻打的步骤,将八位蛊仙的仙窍本源全力抽取,直接灌输到东方余亮的身体内。

虽然他渡过一劫,强撑过来,成为了蛊仙。但新生的仙窍当中,却是掺杂了八位蛊仙的意志,充斥着他们的气息。

这八位蛊仙的肉身、魂魄,在惊天动地的爆炸中,已经灰飞烟灭。但他们遭受背叛,对东方长凡的恨意,都通过意志、气息保存下来。

东方长凡竭尽心神,应战自在书生,一个不留神,隐患爆发,便遭受了强烈的反噬。

东方长凡虽有意料,但反噬爆发时,他这才发现自己先前的估计过于乐观。仙窍中积累的意志之强。远超他的预算。

因而,反噬下来。他差点主动撞死在千解之下。

东方长凡叹了一口气,刚要开口向残阳老君解释。忽然间,他的面部扭曲,现出狰狞仇恨之色。

残阳老君心中警兆大起,连忙松手爆退。

下一刻,东方余亮施展狠辣杀招,杀向救命恩人残阳老君。

幸亏残阳老君机警,东方余亮未果,打了个空。

“长凡老贼,你无耻自私。算计自家蛊仙性命,好为你一己夺舍重生!”东方余亮怒吼咆哮,却不是先前的语调,竟类似东方一空。

这样的惊变,叫众人摸不著头脑。

自在书生眼中精芒爆闪,再次杀来。他这次反应很及时,主动出击,杀招千解再次爆发,射向东方余亮。

但东方长凡再次掌控了肉身。险而又险地避开千解杀招。

然而好景不长,很快又有一位东方部族的蛊仙意志,强行冒出,掌控身躯。一边爆喝,一边主动寻死。

残阳老君只得再次出手,救东方长凡于水火之中。

战局再生变化。

原本潮起潮落。攻防有序的万星飞萤,一片纷乱。魔道蛊仙们压力骤减。纷纷朝东方余亮杀去。

其中,又以自在书生威胁最大。纵横在星光海洋之中,千解杀招无可阻挡,所向披靡。

但东方长凡却有残阳老君保护,后者乃是中洲古派中的强者,底蕴深厚,仙道杀招追命火又颇为棘手,挡住自在书生。

东方余亮时而被某位东方部族的蛊仙意志反噬,主动寻思,或攻击残阳老君,甚至自杀,时而又被东方长凡重新掌控,对战魔道蛊仙,企图挽救整个战局。

但这个战局,终究不可避免地陷入混战当中。

混乱的源头,就是东方余亮。

“这一切到底是怎么回事?”黎山仙子等三人,站在战场边缘,作壁上观,此时看得稀里糊涂。

局势变化,诡异迷离,出乎他们所料。

“看样子,似乎是东方长凡牺牲了家族蛊仙,夺舍了东方余亮,结果没有彻底成功。这老东西真是能算计啊!”黑楼兰发出冷笑。

“不管东方余亮是谁,他身边的那位炎道蛊仙,并非是北原中人,而是中洲仙鹤门的蛊仙强者,名为残阳老君。刚刚的交锋之后,他现在恐怕已经对我产生了怀疑。”方源揭破残阳老君的身份,秘密传音。

“什么?这下麻烦了。”黑楼兰紧皱双眉,她和方源就是一根绳上的蚂蚱,荣辱一体。方源暴露,她也就岌岌可危了。

“现在怎么办?难道要杀人灭口?”黎山仙子蠢蠢欲动。

“不!他也只是怀疑而已,只要我不彻底暴露,他也算不到我中洲的身份。现在战局极乱,我们若插手,恐怕难以脱身。再者东方长凡老谋深算,万星飞萤强大,我们就算动手,为了遮掩身份,也不敢尽力猛攻。而且残阳老君的实力,也是七转中的强者,不容轻辱。”方源目光闪烁,不断向身边二女传音。

他早有退意,此时局面虽然混乱,但依照他们的情况,却不能尽情出手。就算能发挥出最大战力,三位战力加起来的总和,也没有强到威慑全场,彻底掌生控死的层次。

“你的意思,是让我们退走?”黑楼兰却不大愿意,眼前混乱的战局,对她而言,仍旧是个机会。

她矢志复仇,但黑城强大,她又失去了仙蛊我力,战力下降。若不通过这些机会,尽快提升,那她何时才能报仇雪恨呢?

方源却笑:“我的意思是一退一进。你们有没有发现,这短短片刻,东方余亮已经失控了十多次,但真正算起来,只有八位蛊仙意志作祟。”

“你是说?”黑楼兰、黎山仙子闻弦而知雅意,两对美眸中骤然绽放出夺目的亮光。

“不错。东方一族总共有多少蛊仙?就算有什么隐藏的力量,东方一族的大本营中仍旧是有史以来最为虚弱的。想想看,那可是一方超级势力的积累……”方源语气悠悠,却说得黑楼兰、黎山仙子内心火热。

“那还等什么,走!抄了东方长凡的老巢去!”黑楼兰再无迟疑,一飞冲天,一脸兴奋,率先杀向东方一族的大本营。(未完待续请搜索,!

\end{this_body}

