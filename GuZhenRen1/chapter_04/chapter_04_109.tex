\newsection{吃力}    %第一百零九节:吃力

\begin{this_body}

%1
接下来的拍卖,像是约定好的一般,接连好几只运道仙蛊亮相。

%2
能够迅速消耗运气的走运蛊,能够封印运气的封运蛊,能转移运气的转运蛊,还有方源早已经打探到的断运仙蛊!

%3
可惜这些运道仙蛊的卖方,都有各自需求,卖方要求换取金木水火土等主流仙蛊。这正是方源的弱项,不满足这个基本条件,方源连参加竞拍的资格都没有。

%4
方源吃过运道的苦头,也尝过运道的甜头。他原本还打算,看看情况,是否能获得一些运道仙蛊,可惜一点机会都没有。

%5
更遗憾的是,他和断运仙蛊失之交臂。

%6
方源已经有了连运仙蛊,和断运仙蛊可以搭配。可惜机缘不够,无法得手。

%7
这场拍卖大会又不是专门给他举-办的,方源有所得,自然也会有所失。

%8
“下面这只仙蛊,是一只七转运道仙蛊,名为招灾……”终于轮到方源的招灾仙蛊上场,秦百胜徐徐介绍。

%9
七转的仙蛊,相当少见。拍卖大会中,绝大多数都是六转仙蛊。

%10
虽然只相差一转,但价值却相差极大。这是因为六转提升七转,成功率极低,炼制成本非常高昂。

%11
至于八转仙蛊,本次的拍卖大会根本就没有。

%12
大厅中的蛊仙们,听到是七转的仙蛊,都生出兴趣,投来观望的目光。

%13
但得知招灾仙蛊的具体用途后,蛊仙们纷纷摇头。

%14
招灾仙蛊是祸己而惠人。帮助他人渡劫,勾动灾劫驾临自身。

%15
舍己为人者,并非没有。但通常也要讲究利益交换。

%16
招灾仙蛊并非没有市场,一些大势力需要它,可以帮助自家成员。另外也有一些蛊仙心动,或许可以借助此蛊帮助他人,从而盈利。

%17
但方源要求七转力道仙蛊,整个拍卖大会中,却是无一人能够取出。

%18
方源见机不妙。连忙又通知秦百胜,宣布也可用七转宙道仙蛊换取,仍旧无人响应。

%19
“果然是招灾仙蛊。不知道这只仙蛊,是否也是来自十号密室的蛊仙?”凤九歌沉吟一番,暗中沟通秦百胜,要求和招灾仙蛊的主人通话。

%20
“幽兰仙子。果然忍耐不住了么……”方源冷哼一声。毫不犹豫地拒绝了这个请求。

%21
最终,因为无人竞价,招灾仙蛊流拍。

%22
方源对这个结果,其实早有心理准备。毕竟是要求七转仙蛊,还得是力道仙蛊,这样的仙蛊显然不多。最关键的是,招灾仙蛊和其他仙蛊相比较起来,并不那么吸引人。最终导致无人问津。

%23
方源的招灾仙蛊,并不是流拍的唯一仙蛊。

%24
最早之前。就有一只隐运仙蛊流拍,其后陆续有数只,也沦落相同下场。

%25
对于这些的遭遇流拍仙蛊,还有机会。

%26
在拍卖之后,这场大会还有一个环节,就是自由交易。

%27
到了自由交易的时候,还可以出手。

%28
时间不断流逝,一只只的仙蛊登台亮相,又被人买下。蛊仙们内心的期待,逐渐积累增高。

%29
清单中的仙蛊已经所剩无几,快要卖完。拍卖这一环节的压轴之物,不久后就要亮相,便是秦百胜缴获的运道真传之精髓!

%30
“下面这只仙蛊,名为乐山乐水蛊,智道六转。卖方的竞拍条件是一只六转力道仙蛊,防御仙蛊最佳。”秦百胜宣布道。

%31
只剩下最后几只仙蛊时,终于轮到方源的乐山乐水仙蛊粉墨登场。

%32
他这一次,仍旧是要求力道方面的仙蛊。

%33
对仙僵身躯,重获新生的需求,只字不提。也不提任何,关于白莲巨蚕蛊的事情。

%34
这两件事情,一旦提出来,再结合力道仙蛊,线索就太过于明显了。

%35
方源手中有定仙游,早已为人所知。

%36
就算方源不知道凤仙太子和灵缘斋的关系,也不会无脑到提出这样的需求。

%37
六转的乐山乐水仙蛊,换取六转的力道仙蛊。这个竞拍的门槛,就比之前的招灾仙蛊,要低很多。

%38
秦百胜话音刚落,就有好几位蛊仙,接连报价。

%39
力道式微,智道神秘却经久不衰。在大多数的蛊仙眼中,六转智道仙蛊比六转力道仙蛊,更有诱惑力。

%40
一时间,围绕乐山乐水仙蛊,拍卖大会的氛围也是颇为热闹。

%41
方源仔细听着报价,不动声色,他在等幽兰剑师出手。

%42
没过多久,“幽兰剑师”果然参与了竞拍:“我的手中有一只吃力仙蛊。此蛊既是力道,也是食道。乃是力道修行仙蛊,能通过食用力道方面的兽肉或者果蔬,令蛊仙身上的力道道痕自行生长,完全符合自身。”

%43
此言一出,几乎全程的蛊仙都为之侧目。

%44
“这是一只好仙蛊啊。”

%45
“对于力道蛊仙而言,的确难得!”

%46
“食道早已不存,比力道情况更糟。至少力道还有鼎盛时期,食道却是没有,如今更不见踪迹。只听闻对于养蛊,大有裨益。”

%47
听着大厅中蛊仙们议论之声,凤九歌脸上微微带笑。

%48
这只仙蛊,却不是来源中洲十大派,而是之前不久,凤九歌对方源的需求有所预料,因此求助了凤仙太子。

%49
他虽然手中力道仙蛊不少,但力道方面,也就之前的拔山、挽澜两只。

%50
凤仙太子手中,却是正好有这么一只。

%51
原来,八十八角真阳楼倒塌之后,一只无相手破空飞出,正巧落到凤仙太子的洞天附近。被凤仙太子亲手破开无相手,捉住了吃力仙蛊。

%52
此蛊对力道蛊仙大有裨益,但对凤仙太子而言,却是用处不大。凤九歌来信求援,凤仙太子作为同门,立即动用手段,将此蛊暗中送到凤九歌的福地中。

%53
“幽兰仙子,不知道你这仙蛊可否转让?本人愿意用另外一只力道仙蛊换取,并辅以高价仙材,并不会干扰你争夺乐山乐水仙蛊。”很快,楚度直接来信。

%54
“霸仙也心动了,不过也是难怪。不过他给的这蛊,虽然也是力道,但怎么能有吃力蛊吸引人?”凤九歌回了信去,拒绝楚度的请求。

%55
他笑容更加自信,十号密室中的神秘蛊仙既然屡次需求力道仙蛊,极有可能本身就是力道。就算不是,吃力仙蛊对于此人的诱惑,也绝对是惊人的。

%56
一切正如凤九歌所料,方源听闻吃力仙蛊的介绍,心中又惊又喜。

%57
这吃力仙蛊太适合他了!

%58
须知蛊仙修行,主要就是依靠仙窍渡劫。每一次渡劫成功,蛊仙身躯都会增加本道道痕。

%59
蛊仙身上的道痕越多,就越亲近天地,亲近大道法则。劫数越大,渡过劫数后,得到的道痕就越多越深刻。

%60
因此评论蛊仙修为,都冠以一次天劫、二次天劫的相关标准。

%61
道痕越多越深,使用相应仙蛊时,就能引发更深的共鸣,使得仙蛊威能增幅更巨。

%62
方源使用浪迹天涯仙蛊,身上的力道道痕,和浪迹天涯本身的水道道痕毫不关联,有所冲突。因而效果不佳,十成威力只能发出七八成来。

%63
但方源若用力道仙蛊,诸如拔山、挽澜,却能引发自身和仙蛊的共鸣,从而不仅能增添仙蛊威能,十成威能可以增长到十二三成!而且,共鸣的次数越深,方源通过使用仙蛊,就更能体会到力道方面的道理法则。体会越深,力道境界也会随之,潜移默化地积累,不断加深。有朝一日,量变引发质变。

%64
然而!

%65
方源的最大难题,就在于他现在是仙僵之躯,仙窍已死。每隔一段时间,福地自行崩解一小部分,并且毫无灾劫降临。

%66
天道平衡至公,降下灾劫的对象,都是有福的天地。福分越大,灾劫就越强。

%67
仙僵不渡劫,本是一件好事。仙窍既死,若是再加灾劫,基本上就没有仙僵什么活路了。

%68
但对于方源而言,没有灾劫,就没有道痕增加,就没有修为方面的进展。

%69
“当然,要增加道痕,也可以用铁冠鹰力蛊这样的仙蛊,在我的身上刻印道痕。但这样的兽力仙蛊,一来不大适合人体,二来数目稀少,效果不大。两三只兽力仙蛊刻印上去的道痕,还不如渡过一次地灾得多的道痕收获更多。”

%70
方源有使用兽力蛊虫的经验。

%71
兽力不大适合人体,使用不大灵便。有时候能打出兽力虚影,有时候不能。

%72
像方源得到的铁冠鹰力蛊,可以在他的身上刻印出铁冠鹰身上的一部分力道道痕。但这些道痕,并不完美适合方源肉身,毕竟人体和雄鹰结构不同。

%73
铁冠鹰的相关道痕,在方源的身上完全共鸣时,方源才能打出一头铁冠鹰的力量,同时爆发铁冠鹰力道兽影。

%74
除非他得到全力以赴仙蛊。

%75
全力以赴真正的效用,就是能时刻将人体上的力道道痕,完全共鸣。

%76
但方源的全力以赴蛊,还只是四转。虽然是本命蛊,但如何升仙,却是毫无头绪。全力以赴蛊的仙蛊方,似乎早已失落在漫漫历史长河当中了。

%77
对此,方源最大的希望,就是琅琊地灵。琅琊福地中的仙蛊方,有数千张,希望当中就有全力以赴蛊的仙蛊方。

%78
ps:待会还有第二章,在半个小时之后。

\end{this_body}

