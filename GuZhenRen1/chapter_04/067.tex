\newsection{星殿}    %第六十七节:星殿

\begin{this_body}

大半天下来,方源和黑楼兰几乎将这一片,都探查过了,见识的荒兽数量不少,但偏就没有星魔蝠。。

“你的情报源有点不可靠啊。”黑楼兰大大咧咧地坐在地上,一边疗伤,一边道。

她的左臂上,有一条长长的血淋淋伤口,从手背一直延伸到肘部。

这是之前,黑楼兰和方源遭遇到的一头荒兽飞剑鼠,而留下来的印记。

飞剑鼠体型小,速度极快,稍不留意,就被它的利爪所伤。

饶是黑楼兰乃大力真武体,力道绝仙,恢复力惊人,治疗这个伤口,也花费了许多心思。

这是因为飞剑鼠利爪割刺的伤口处,充斥着金道道痕,排斥黑楼兰仙体的力道道痕,抵制了大力仙体的恢复能力。

半晌之后,黑楼兰才将伤口消弭,结成一层粉红的薄薄肉茧。

她来到方源的身边,方源盘坐在地上,屈起一根尖锐如铁的食指,直接在一块平坦的山石上,勾勒出他们俩勘探出来的地形图。

这里的青峰,总共有数十座。大部分青峰上,都有一只荒兽坐镇,譬如星荒犬、飞剑鼠、钻熊之流。

没有荒兽的山峰上,也会有大量的兽群,以及海量的蛊虫结伴生活着。

很显然,七星子为了经营繁星洞天,投注了大量的心血,才有如此密集的荒兽生活在这块区域里。

但奇怪的是,没有上古荒兽。

上古荒兽媲美七转战力。按照常理,经营如此优秀的繁星洞天里,应当有上古荒兽才是。

这还只是古怪之一。

古怪之二。是天灵久久没有出现。

方源、黑楼兰二人四下探索,虽然遭遇了不少战斗,但都顺利脱身。

严格来说,他们两个就是入侵者,但这座繁星福地仿佛没有天灵主持,方源意料中的围剿根本就没有出现。

古怪之三,是方源死活都搜寻不到毒气沼泽。

按照前世记忆。他此刻踏足的地域,就是他前世派遣得力下属,进去探索搜刮的洞天碎块小世界。

原本就在那片古树林附近的毒气沼泽。方源怎么找都找不到。

“难道说,毒气沼泽是数百年之后才形成的地貌?我提前来到这里,因此才搜寻不到毒气沼泽的。若是这样的话,我要捕杀星魔蝠的计划就要落空了。”

方源盯着石面上的地形图。正思考着。忽然感到一股强光绽射而来。

他和黑楼兰同时抬起头,看到青绿的苍穹中,不知何时,浮现出无数星星点点。

这些星光,数量激增,越来越多。转眼间,就充斥整个天地,纷纷扬扬。宛若一场大雪。

“这是洞天才有的天象变化。”方源站起身来,目光戒备。撑起狮毛甲。

黑楼兰也催动防御杀招,心中警惕。

星光灿烂,映照万物,各大山峰中接连传出兽吼,或清冽或干脆,或绵长或嘶哑。

大风骤起,漫漫星点忽的都往一座山峰顶端汇聚。

一阵刺眼的蓝芒之后,星点消失,一座璀璨的宫殿,精致华丽,凭空出现在山巅之上。

“这座宫殿……”方源瞳孔一扩,宫殿形制让他感到分外眼熟。分明就是前世记忆中,碎块世界里四处洒落的断壁残垣。

只是曾经的断壁残垣,此刻却丝毫未损。

方源和黑楼兰对视一眼,后者猜测道:“是否是因为特定的时机到了,天象变化,才引发了宫殿出现?”

“或许是天灵故意设局,引我们入瓮?”方源眼中精芒闪烁。

两人只犹豫了一下,便都决定前往星殿一探。

与此同时,在繁星洞天的另一处地域。

一场激战,进入尾声。

“孽畜,吃本王一拳!”一具高大如山的石人爆喝一声,拳头直捣下去,霎时间卷动狂风,击爆空气。

荒兽飞熊躲闪不及,脑袋被巨大的石拳击中,轰的一声,趴在地上,砸出一个大坑。土石飞溅,烟尘四起,大地都震颤了一下。

石头巨人并不罢休,双手十指张开,高高举起,旋即猛地落下。

蓬的一声,双掌重重地落在飞熊雪白肥厚的身躯上。

飞熊应力弹了一下,连惨叫都没有发出。它早已经伤痕累累,刚刚一拳更被揍得脑壳破裂,陷入昏死状态。

仙道杀招——大地根!

以双掌为中心,地气剧烈翻滚,无数粗壮尖锐的地刺,宛若枪尖戟首从地面下撞出。

嗤嗤嗤……

荒兽飞熊被大大小小的数百根地刺,顷刻间洞穿。

它回光返照地扬起头颅,猛地睁开双眼,发出一道凄厉的惨嚎声后,身体僵硬,力气消散一空,脑袋重新坠落在地。

咚的一声,宛若敲击了一下大鼓,一股烟尘扬起复又落下。

浓烈滚烫的鲜血,顺着石柱向下蔓延,迅速染红地面。

飞熊死亡,再无动静。

石头巨人冷哼一声,忽的绽放出刺眼的光辉。光辉消散后,一位蛊仙怀抱双臂,倨傲地站立于半空当中。

他一头白色短发,一对金瞳,狼背蜂腰,劲装武服,散发着一股彪悍勇猛之气。

一道星芒射来,化为一位中年男子,头戴高冠,宽袖长袍,鼓掌赞叹道:“不愧是仙猴王大人,屠杀荒兽飞熊,只用了半盏茶的功夫。”

这白发金瞳男子,竟然就是战仙宗的七转蛊仙,仙猴王石磊。

石磊转动淡漠的金瞳,看向中年男子:“七座星殿已经显现,万象星君,你说的第八星殿,怎么还没有开?”

原来这中年男子,就是曾经多次贩卖给方源货物的万象星君。

万象星君朗声一笑:“仙猴王勿急。要出现第八星殿,唯有斩杀足够多的荒兽,让荒兽血液渗透大地,才能引动第八星殿出现。我们虽然已经斩杀了六头荒兽,但血液似乎还不足够浓郁。”

“到底还要再杀多少头荒兽?”石磊不耐烦地问道。

万象星君流露出思索的神色,想了想,沉吟道:“根据我在过去二十多年来,每年一度的探索,再靠我个人的推测,应该只要再斩杀一头荒兽即可了。”

“哼!你最好不要骗我。”石磊语气蛮横。

万象星君低下头来:“我区区一介散仙,给十个胆子也不敢骗仙猴王您呐。中洲时间每年,我都只能在这个时间段,强行闯入这处星道洞天。而且每次进入的时间都只有两天,超过两天,洞天缝隙就会弥合,我们就再也出不去了。时间如此有限,我怎么可能会骗您呢?我巴不得搜刮更多的资源呢,但在几年前我意外地看到了第八星殿,那殿气象森严,守护星殿的皆是上古荒兽,一定是这处洞天的中枢重地,很有可能天灵就在那里。”

石磊点点头,他相信万象星君的这番话。

他和万象星君认识,也有一段时间了,熟悉后者的性情。

“唉,若非这次我和宋紫星大战一场,损耗了大量仙元,福地渡劫又在眼前,急需大量仙元石转化仙元。我也不会将这个重大的秘密,告知于您了。”万象星君叹息道。

石磊语气一缓:“放心吧,我会遵守协议,此次夺得的战利品你八成我两成。并且,这处秘密我也不会告知第三个人的。以后每年,我们都来这里探索。”

目前进入繁星洞天,只有万象星君才有着独门手段,石磊还没有掌握。

“石磊大人信誉出众,这点在下放心得很。按照我往年的探索经验,不远处就有一头星魔蝠,比较容易斩杀。”

“好,咱们就去那里。老规矩,打起来的时候,你站到一边去,别妨碍我动手。”石磊雷厉风行,一边说着,一边就动身疾飞。

万象星君苦笑一声,急忙跟上去。

方源和黑楼兰二人,小心翼翼地进入星殿。

这星殿毫无防卫力量,殿中更是空无一人,唯有六口大井,坐落在大殿正中央。

这六口大井中,都有高涨的井水。

井水颜色各异,分别为赤、橙、黄、蓝、紫、白六色。井水晶莹通透,散发着微微洁光。

黑楼兰、方源均用侦察蛊虫,彻底探看了一番。

发现这井水、星殿,都是普普通通。这当然不可能。普通的宫殿,怎么可能凭空产生?普通的井水,怎么可能颜色各异,且散发微光?

唯一的解释,就是这处星殿和六口井水,皆是洞天的天象变化。唯有如此,黑楼兰、方源二人手中的凡蛊,层次不够,就什么都侦察不到了。

两人无从判别六井的玄妙,一时间陷入困局。

“这个颜色……”黑楼兰观察了一阵,不禁沉吟起来。

方源知道她想的是什么,记得二人刚刚来到繁星洞天死,就抬头望过天。看见天中,就有六道星影,皆大如满月,充斥视野。

这六道星影,正是赤、橙、黄、蓝、紫、白六色,分别对应眼前的这六口井水。

“这其中定有什么联系。”方源口中喃喃,催动存储凡蛊,往一口井中深入,企图捞取一些井水研究。

但没想,凡蛊一进入井水中,原本平静无波的井水就陡然旋转,产生漩涡,将凡蛊一下吞了进去。(未完待续。。)

\end{this_body}

