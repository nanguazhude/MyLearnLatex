\newsection{多重复合蛊阵}    %第二百八十八节:多重复合蛊阵

\begin{this_body}

方源按捺住兴奋之意。

手中仙材的品质之高,即便是他前世,也从未得过。

准九转的炼蛊仙材,道痕之多,逸散华光,甚至能肉眼观察出来。

方源炼蛊纵然不是宗师级数,但也明白一点,那就是仙材之所以能够炼成仙蛊,是因为仙材本身蕴含着丰富的道痕。

人是万物之灵,蛊是天地真精。

蛊是大道的载体,为什么能够有威能?就是因为小小蛊虫的身上,蕴含着更多的道痕。

道痕是天地大道的痕迹,正是因此,催动蛊虫才会有各种奇妙的威能。

蛊虫身上的道痕越多,转数就越高,威能就越奇妙,更博大。

炼成仙蛊,便是将无数凡蛊,无数仙材身上的道痕,提取出需要的部分,然后集于一处。

现下,方源手中的这块仙材拥有的道痕极多,从理论上,是足以炼制八转仙蛊的仙材!

甚至只要取其一角,或者一个小碎片,就能炼制六转仙蛊。

八转仙蛊身上的道痕,是六转仙蛊的数万倍。

“当然,这块疑似斑虎蜜蜂的蜂翅,主要蕴含的是金道道痕。我若要炼制金道仙蛊,这就很有用途。可惜目前,对我炼制星念仙蛊没有多少帮助。”

方源将这块珍贵的仙材,好好收了起来。

他现在暂时用不到,并不代表将来用不到。

就算将来用不到,如此品质的仙材,放入宝黄天中。必定是一场振荡人心的宝物,不愁卖不出去。

“进入这个蛊阵以来。终于是有所收获了。”

方源心里开始活泛,动了心思。

他没有急着赶赴中心。而是开始在这片紫沙荒地中搜寻。

果然,片刻之后,他飞在半空中又有发现!

在紫色沙地中,一片白色的纱娟,仿佛手帕,落入他的眼帘。

他连忙降落下来,近前观察,顿时喜悦拢上心头。

气息做不得假,这又是一块八转仙材!

只是品质比之前的蜂翅。稍有不如,没有产生道痕光晕。

方源拾到手中,果然这块仙材脱离了紫沙之后,也没有损毁。

“看来,到了这种程度的仙材,已经不是紫沙的力量可以同化分解的了。”

方源用手捏了捏,发现这块仙材质地十分柔软,可以随意揉捏,随意形变。

“原来这不是纱娟。而是一块……云?”

方源细细观察,大为意外。

这好像是云朵的碎片,但蕴藏着极其庞大的云道道痕。

“这莫非是……”方源忽然心中一动,催动蛊虫。张口一吐,喷出一口普通的水雾。

这块仙材,被水雾笼罩住。顿时轻飘飘的,脱离了方源的手掌。悬浮到空中,化为一片凤凰状的小巧白云。凤翼微微振动。白云悠然飞舞。

方源身躯一震,继而大喜:“果然!这是太古荒兽云凰的身体碎片啊!!”

方源心头频振。

这云凰并不是五域生物,而是太古九天之一白天,才有的奇妙物种。

它生来并非是凤凰血脉,而是白天中上万年都不毁的古老白云,被凤凰的气息或者血液沾染,渐渐由一朵白云,转化成的生命。

所以,它的身体非常特殊,并非是寻常的血肉,似云非云,似肉非肉。

方源得到的这块云凰碎片,恐怕连云凰本体真身的万分之一都不到,但其中蕴藏的云道道痕,至少可以当做六转云道仙蛊的炼制主材十多次!

云凰碎片的发现,让方源肯定了之前的猜测。

他继续搜索,片刻之后,他几乎绕了整个紫沙荒地一圈,收获很大。

清点一遍,他统共拾取了十八块这样的仙材,道痕种类涵盖了大部分的蛊道流派。其中三块仙材,蕴含丰富的星道、智道道痕,对炼制星念仙蛊的帮助非常大。

可以说,有了这三块仙材,方源完全可以舍弃焚天魔女,自行炼蛊了。

当然,因为仙材的改变,关于星念仙蛊方也要随之剧烈改动。不过方源不怕这点,他有智道光晕可以利用,本身又已经是智道宗师了,推算并不困难!

“我收集到的这些仙材,虽然珍贵无比,但只是各自本体的一部分而已。布置这里的蛊仙,居然采用如此多的珍贵仙材,难道是想炼制出九转仙蛊吗?”。

方源不免在心中猜测。

在收集仙材的过程中,他渐渐发现了一些端倪。

这个紫沙荒地中,曾经布置着一座十分高深的炼道蛊阵。

再结合这些碎片仙材来抗,很可能这里的蛊仙,图谋极大,想要炼出某种强大无比的仙蛊。

但可惜的是,炼蛊失败,发生了大爆炸。

不仅将这座炼道蛊阵瓦解,只剩下些微残骸,而且用于炼蛊的仙材,绝大多数都被炸毁,只剩下这些残片碎渣。

“不管是谁,又或者是哪一个组织布置的这里,他们极可能也因为爆炸身陨。若非如此,这座庞大的蛊阵,也不会年久失修。这里的仙材碎片,也不会随意散落四处,无人拾取。”

虽然是一些仙材在爆炸之后的残渣,但价值却是十分巨大。

涉及八转、准九转的仙材,就算是八转蛊仙也会怦然心动。

若是雪胡老祖知道方源手中,有着如此大量的珍贵仙材,保管他舍弃面皮,急吼吼地跑过来找方源的大麻烦。

雪胡老祖为了炼制八转鸿运仙蛊,虽然有马鸿运作为炼蛊主材,但对于其他辅助仙材的需求,已经到了**的程度。

若非如此,他也不会不要八转大能的风度,亲自去僵盟墓地偷取仙僵尸躯,替代其他仙材了。

想到这里,方源的担忧稍稍缓解了一些。

既然这个神秘而又强大的蛊仙,或者蛊仙组织,已经消亡,那么此行的风险无疑下降了一大截。

方源举目眺望,看着眼前的紫沙荒地,又有了新的感受。

“唉……漫漫光阴长河,不知掩盖了多少历史的真相,多少惊天动地的人物,就这样默默无闻地陨落了,从不为人所知。”

风流逸散,传奇不在。

收拾情怀,方源来到紫沙荒地的中央地带。

这是整片荒地中,唯一一处他还没有探索的地方了。

虽然有这么多的仙材收获,但方源还是有些失望前世传闻中,那个可以摆脱仙僵躯壳,重新活过来的法门,他到现在还没有发现呢?

方源心中有一股直觉:这片荒地的最中央,应当就是整个蛊阵的中心,若是这里再无所获,那么其他地方就更无可能了。

举目四望,紫沙荒地的中心,只有一片细碎的紫沙。

这里是大爆炸的中心,紫沙都比外围的,要更加细腻。

没有任何的仙材存在。

乍一眼看去,这里空无一物。

不过在方源各种侦查手段之下,紫沙地下深处的一些蛊阵残骸被陆续发现了。

“这里果然是整个空间的中枢!”

“好家伙,原来是多重复合蛊阵,如此精妙的布置,必定是阵道大宗师的手笔!”

方源查看之下,心中暗惊。

原来,整个空间是以炼道蛊阵为主的,而营造这片空间,包裹所有的宇道大阵,只是炼道蛊阵外的辅阵而已。

除了宇道蛊阵外,方源还发现了许多防御蛊阵,一些食道蛊阵,还有虚道蛊阵。

食道流派,一直以来都没有强盛过,到今天越发罕见。

方源对此也不了解,只听说食道最擅长的,是蛊虫的喂养。

这里的食道辅助蛊阵,应当就是喂养这些蛊虫的吧。毕竟蛊阵是需要无数蛊虫,相互组合在一起,才能形成的。

“这里还有两三座食道蛊阵,仍旧在运转。难以想象,大爆炸之后,这里仍旧还有蛊阵在运作。”

“正因为如此,最外层的宇道蛊阵,才运转至今吧。”

这个发现,算是替方源解了一个谜惑。

“咦?这是……虚道蛊阵?”继续观察,方源又有新发现。

这座虚道蛊阵,已经损毁,但残留了两三只凡蛊,埋藏在紫沙深处,一动不动,奄奄一息。只是依靠食道蛊阵苟延残喘。

方源对虚道更加不了解。

但太丘之战,东方长凡曾经在墟蝠尸体上有布置,也设置过一座虚化大阵。

若非如此,方源还不能立即认出这个蛊阵的跟脚。

“虚化大阵,就是将整个蛊阵虚化,避免外来的打击会破坏整个蛊阵。或许正是因为虚化大阵,方才有这些蛊阵在大爆炸中存活下来。”

“造诣深厚的炼道蛊仙,都有自己的炼道蛊阵,在炼道蛊阵上还设置许多辅助蛊阵。利用这些蛊阵,炼制蛊虫事半功倍,节省大量的精力和时间。譬如大雪山福地中就有一座精妙的炼道蛊阵,号称是北原魔道炼蛊第一阵。据说这个蛊阵,也是多重复合蛊阵,总共融合了十二道不同的蛊阵。”方源思绪发散。

他也能布置一些炼道蛊阵。

但这些蛊阵,都很粗浅,毕竟涉及到阵道修为。

方源的阵道境界浅薄得很。

人的精力是有限的,时间更加有限,方源在某些方面有所成就,在其他方面势必就泯然众人。

\end{this_body}

