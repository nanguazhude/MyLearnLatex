\newsection{稍喘息又起战危}    %第一百五十六节:稍喘息又起战危

\begin{this_body}

%1
击破不了假的身影,方源只能胡乱猜测,又做了几次毫无依据的选择。

%2
最终,夜幕降临,他眼睁睁地看着追击的东方长凡,身影渐趋虚无黯淡,最终彻底消散,露出里面的一团蛊虫。

%3
方源收取了这团蛊虫,居然还没有那只分影仙蛊,只是一些五转蛊虫,倒是沾染着仙蛊的浓郁气息。

%4
“跟丢了!就看皮水寒那边了,不过希望不大,极可能叫东方长凡逃生了去。唉,不愧是传奇人物,超级势力的首脑,底牌如此之多!当务之急,还是赶回碧潭福地,搜刮最后一笔。”

%5
方源收拾情怀,立即返身飞回去。

%6
一路上他联系黎山仙子、黑楼兰,却被回到仍旧在追击方寸山,并请方源出手援助,给予大量资源补贴。

%7
方源一时间陷入两难选择。

%8
一方面是碧潭福地,里面资源真的很丰厚,但时间久了,恐怕就要被其他魔道蛊仙搜刮一空了。另一方面是方寸山,若是能得到方寸山的话……

%9
他的心,不由动了动。

%10
夜幕彻底降临,寒气下沉,夜空中乌云密布,空气压抑。

%11
片刻之后,大风骤起,雨瓢泼而下,跨嚓嚓,电闪雷鸣。

%12
东方长凡盘坐在地上,面容苍白如纸,毫无一丝血色。刚刚夺舍来的身躯,受创数百处,有些地方的伤势严重,已然深可见骨,淋漓的血肉就暴露在空气中。

%13
东方长凡气息都有是微弱的,眉头紧锁:“不想此次夺舍,居然会出现这么大的变故。果然,逆天而行,是要遭受天地的怒火。幸好在最后关头,我舍弃了方寸山,引走一部分追兵。又舍弃仙蛊分影,用计逃出生天了。”

%14
一时间,东方长凡的心中,有后怕,有庆幸,有喜悦,有沉重,有悲痛,有仇恨。

%15
情绪十分复杂,宛若丝线搅成一团。

%16
“身上的伤势,还是小事,不足以致命。当务之急,还是要清除体内的内患。积攒的仙元,大部分都用于夺舍。之后激战已彻底消耗,意志之战,便是我击溃这九股意志的唯一选择了……”

%17
东方长凡思量一定,便彻底闭上双眼,整个人陷入到一种冥冥的状态中去。

%18
事实上,恶劣的情况也不允许他有其他的选择。

%19
他已经油尽灯枯,仿佛是一道长河,消耗到干涸可见河底的状态。

%20
能以这种状态,甩开追兵,逃出生天,真是险之又险。

%21
此时他若再犹豫一些,不进行意志大战的话,对这具身体的控制权就又会被夺走。

%22
反而不如趁着现在还有底蕴,还有余力,将种种余力都转化为意志大军,进行意志大战。

%23
意志作战极为凶险,无关乎修为,只看意志本身。

%24
但东方长凡此时已无余力,只有这样做了。

%25
他的意志,从识海中升腾而起,转瞬间调集搜刮一切底蕴,形成一股巨流。

%26
巨流从识海中奔腾而出,宛若九天而下的天河,从头颅高处,直灌而下。夹裹澎湃气势,朝着身体各个角落中的其他意志,倾泻绞杀过去。

%27
“战!”

%28
“一齐出手,老贼的意志比我们任何一人都强。”

%29
“必须联手,只有联手才有胜机!”

%30
东方部族的八位蛊仙意志,以及东方余亮意志,纷纷揭竿起义,冲击东方长凡的意志。

%31
总共十股意志,混合在一起,绞杀一团。

%32
不管哪一方,都是有进无退,退一步就是失败的深渊,死无葬身之地。

%33
所以一开始,便是死战!

%34
双方几乎势均力敌,但片刻后,东方长凡的意志,便落入了下风。

%35
和群魔战斗时,东方长凡只是强行压制,并未发现这些意志的难缠。此时真正动手时,才知晓这些意志,股股凝如坚钢,强硬如钻。

%36
一位位,都对东方长凡有深深的恨意!

%37
“东方长凡,你欺骗我们,背叛我们!”

%38
“为了一己私欲,就杀我们,成全你自己!!”

%39
“老谋深算,我们看错了你,都不过是你的棋子……”

%40
“你让我们死,你也别想好过!!”

%41
“同归于尽,同归于尽!”

%42
“就是你,害了我们,还害了整个东方部族。”

%43
“如今,碧潭福地必定遭受群魔搜刮,我族数千年的经营消散一空!”

%44
……

%45
这些意志们呐喊着,悲号着,汇集成一波波的潮水,从各个角落中反攻出去。

%46
东方长凡的意志节节败退,很快就退缩到识海中。

%47
识海是东方长凡最后的根据地,一旦被这些意志攻破,就算是东方长凡的魂魄健全,也无力回天,彻底失败。

%48
“好厉害,好厉害!这些意志夹杂着死志,被背叛的滔天愤恨,最重要的是被天劫影响,有了玄妙难言的增强变化……”

%49
这一刻,东方长凡感觉到天地的深深恶意。

%50
他逆天而行,死而重生,恐怖的反噬再次让他命垂一线,后退一步,便是失败的深渊。

%51
但东方长凡临危不乱。

%52
反而情况越是凶险,越激发了他的斗志。

%53
他这一生,都是逆流而行,从自己修仙,到带领家族崛起,克服多少的困难,顶住多少的压力,打退了多少的敌人。

%54
“这一次的敌人,即便是这方天地,即便我的身躯老死,我的魂魄萎靡,也不能让我的意志屈服啊!”

%55
东方长凡明白:意志的交战,千万不能气馁,不能松懈,不能有丝毫的动摇。

%56
他不仅不动摇,反而更坚定,斗志飞扬,使得意志战力暴涨。

%57
这就是他千锤百炼,坚韧不拔的精神。奋战抗争的斗志,已经融入了他的骨子里。

%58
一下子,意志反攻,从识海中蜂拥而下,又将阵线推了下去。

%59
双方以东方余亮的身体,作为战场。

%60
东方余亮的头颅,便是东方长凡意志的大本营。之前虽然被九股意志,几乎打回大本营。但如今却直贯而下,意志拼杀,阵线从头颅,顺着脖颈,一直推到胸口。

%61
蛊仙们的意志越加愤怒,凝结起来防御,宛若龟壳,牢牢占据胸口,不再移动。

%62
意志交战,陷入僵持。

%63
八股蛊仙意志,围拱着东方余亮的意志,宛若八块厚实的盾牌,保护着他。

%64
东方余亮的意志总量,正在节节上升。

%65
整个十股意志,只有东方长凡的意志、东方余亮意志,可以不断恢复。

%66
这是因为前者夺舍,魂魄寄居其中,又占据识海,可以源源不断地补充意志兵源。而后者乃是这具身体的原来主人,最为契合肉身。整个身体就是东方余亮的巢穴,只要肉身不死亡,意志就能缓缓恢复。

%67
这场凶险万分的意志大战,胜负关键就在于东方长凡、东方余亮这对师徒的意志上。

%68
相比较而言,八大蛊仙意志,虽然强硬厚实,却是无根之水,无本之源。

%69
他们看透了胜败的关键,便选择牺牲自己,保护东方长凡意志,让他不断壮大。

%70
“不妙,我故意留着身上的沉重伤势不医治,就是想拖累东方余亮的意志恢复。但现在对耗下去,我的意志不断损失,后继乏力,东方余亮的意志却在不断回升。”

%71
念及于此,东方长凡越加冷静,甚至冷漠。

%72
他知道任凭这样下去,只有死路一条。

%73
为今之计,只有置之死地而后生,用意志感染意志,从内部瓦解意志。

%74
这手段乃凶中至凶,险中最险。

%75
“但没有办法……那就来吧!”东方长凡心中大喝一声,所有的意志原本凝结在一起,此刻忽然崩散,化为一股江流冲刷过去,将对方的九股意志一下吞吐进去,牢牢包裹住。

%76
心志飘摇,斗志激荡,所有人的坚持在这一刻,形成最直接的碰撞。

%77
“杀身之仇,不共戴天!”蛊仙们意志在呐喊。

%78
“我东方一族,就毁灭在你的手中……”东方余亮怒火冲天。

%79
东方长凡的意志却是沉静如冰:“只要我生还,就能带领东方一族再度崛起。你们以为,我不心痛么……你们可还记得《人祖传》中的故事……人祖为了走出平凡深渊,就连女儿都能抛弃!”

%80
“怎么是他抛弃,明明是自己蛊捣乱!”东方余亮的意志反驳。

%81
东方长凡冷笑:“那是谁的自己蛊?那是人祖的自己蛊!那是人祖内心最深沉的想法!为了脱离平凡,抛妻弃子也是无妨,更何况区区家族?”

%82
“肮脏卑鄙的人,果然看什么都是肮脏卑鄙!”蛊仙们的意志齐声怒吼。

%83
“够了!没有我东方长凡,东方一族早就消亡或被吞并了,哪里有你们的今天?哪里有重振雄风的东方一族?我提携你们,教导你们,才有你们的现在!我种下了大树,这是我伟大之成就,而你们!不过是躲在树荫下,吃食我成就之果而已!”

%84
这番话简直是滚滚惊雷,一下子将九股纠结在一起的意志冲散。

%85
蛊仙们的意志,仿佛陷落在滔天巨浪中,只能随波逐流。

%86
生平的一幕幕,倒带一般,迅速闪现在东方长凡的心中。

%87
碧潭福地中,和群魔激烈交手,险象环生的战斗场景……

%88
天劫地灾下,顽强抵抗,也要夺舍……

%89
死亡的那一刻,怀着不熄的野心,视野中的天空渐渐昏暗……

%90
为各大正道蛊仙推算仙蛊位置,缔结盟约……

%91
阴森的大殿中,自己坐着,已经下定决心逆天而行,冷笑:“就算你们封锁寿蛊,又能如何?天要我死,我便逆天!”

\end{this_body}

