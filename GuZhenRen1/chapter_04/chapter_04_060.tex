\newsection{见面曾相识}    %第六十节:见面曾相识

\begin{this_body}

%1
“哦?”涉及到盗天魔尊,方源心中不由地生出兴趣,“难道就是传闻中的‘见面曾相识’?”

%2
关于“见面曾相识”这个仙道杀招,蛊师历史上有很多有趣的记载。

%3
盗天魔尊依靠见面曾相识,转变各种形态、面目,曾经以六转修为,哄骗了一位八转蛊仙邓左岩,使其误以为是至交好友万车震。结果邓左岩引狼入室,被盗天魔尊利用机会,偷了福地中的许多重宝。

%4
事发之后,邓左岩大吐三口鲜血,愤怒之下,找到万车震,大战三百回合,打得天崩地裂。

%5
万车震相当纳闷,拼命解释都不成功。最终冒险,不做防守躲避,让邓左岩打了三招。

%6
三招之后,万车震濒临死亡,终于使得邓左岩相信他们俩之间的友谊。

%7
邓左岩懊悔之下,不惜一切代价为万车震治疗伤势。不久后,查明真相,将盗天魔尊视为必杀仇敌。

%8
万车震伤重难返,邓左岩又要面临天灾地劫,不愿拖累好友,暗中离去。

%9
却不想盗天魔尊看准时机,再度伪装成他的面目,哄骗了邓左岩,躺在病榻上装病,硬生生敲诈了邓左岩无数修行资源。

%10
事后,盗天魔尊扬长而去。

%11
邓左岩底蕴大失,在之后不久的天灾地劫中丧命。万车震听到好友惨死,气得吐血三升。原本就伤势沉重的虚弱病体,终于一蹶不振件。几天后也命丧病榻。

%12
这是仙道杀招“见面曾相识”最显赫的战绩。

%13
它不是攻伐杀招,但却间接害死了两位八转蛊仙的性命。

%14
历史上,除了这一笔浓墨重彩的记录之外。还有无数事迹,记录着盗天魔尊曾经依靠此招,扮猪吃老虎,或者伪装高手坑蒙拐骗,又或者变成雪人大闹过墨人城,结果导致雪人和墨人的大战。

%15
琅琊地灵瞪了一眼方源,连连摇头:“见面曾相识?你也真敢想!那是仙道杀招。核心便是传说中的态度蛊,我就算给你了,你也用不了啊。不过我这个杀招。的确和见面曾相识有很大关联,可以说是见面曾相识的前身,只是凡道杀招,不过正好够你使用。”

%16
“哦?”方源双眼精芒一闪。

%17
见成功地勾动方源的兴趣。琅琊地灵嘿嘿一笑。继续道:“这个杀招,叫做见面不相识。不过你想要拿到它,就得帮我一个忙。”

%18
方源早知道天底下没有白吃的午餐,点点头:“你说。”

%19
琅琊地灵便说出了他的计划:“我的十二云阁中,本来都有一头荒兽驻守。结果遭到不明势力的蛊仙屡次进攻,当场战死两头,事后伤重难返,也病死了一头。现在只有九头荒兽了。你给我奴役一头战力出众的荒兽回来。我不仅按照市价,给你相应的仙元石。而且我还将这个凡道杀招‘见面不相识’给你。”

%20
方源心中微喜:自己屡次和琅琊地灵做推算仙蛊残方的交易,又用山盟蛊彻底解除怀疑,此次借还荒兽之后,终于让自己和琅琊地灵的关系,向前挺进一大步。

%21
若是以往,琅琊地灵绝不会拜托方源这种事情。

%22
方源想了想,便答应下来。

%23
而后,他又从琅琊地灵手中,得到三只新的仙蛊残方,告辞之后,回到狐仙福地。

%24
“师弟,你回来了,这次顺利吗?”太白云生就留守在狐仙福地中。

%25
如今八头荒兽已经归还,黑楼兰、黎山仙子不可能停留在狐仙福地里,方源进入琅琊福地,太白云生就是唯一留守的蛊仙战力了。

%26
“嗯,此行比较顺利。”方源说着,就将江山如故蛊、人如故两蛊,交给太白云生。

%27
“鲨魔那边催的很急,我这就去东海了。”太白云生接过两蛊,也不查看,急冲冲地就要走。

%28
方源也不拦他,只关照几句注意安全的话。

%29
总算一个人静下来,他开始思考接下来的路。

%30
“化解了仙鹤门进攻狐仙福地的重大危机,我总算能勉强站住脚步,以附庸的身份挤进正道阵营,又能贩卖胆识蛊,借此打开了一丝中洲的局面。”

%31
以前的方源,只是凡人,连和蛊仙平等对话的资格都没有。虽然得到狐仙福地,但就像是孩童抱着元石,走在强盗窝里。强盗们见到方源,想到的就是如何狐仙福地占为己有。

%32
现在的方源,成了仙僵,总算有些资格。又合纵连横,扯起大旗,暂时唬住了仙鹤门等各派势力。又借助胆识蛊贸易,形成利益关系。以前的孩童长大成青年了,见到强盗,开始散发元石。强盗们想抢,又忌惮青年反扑,磕坏了自己的牙口,于是接过这些元石,稍稍满足了一些胃口。

%33
再加上,这些强盗本身也在相互提防,也有牵制,精力分散,青年在这个强盗窝里,还是暂时安全的。

%34
但方源清楚的很,今后随着胆识蛊的买卖越做越大,利益越来越重,必会引来更加强大的打击和抢夺。

%35
尤其是五域乱战,各大势力都不会放过任何一个,能增加自身实力的东西。

%36
青年虽然散发了一些元石,满足了一些强盗的胃口。但随着他怀中元石越来越多,必定会引起再一次的争抢。

%37
所以,方源甘愿低头,成为仙鹤门名义上的附庸。这样一来,青年就成了小强盗,强盗之间不能随意劫掠,就得讲规矩。

%38
但规矩只是无形的约束,时间越久,利益越大,就越无力。

%39
而这个青年小强盗,也并非他展现出来的那般强势。

%40
“四位蛊仙,八头荒兽,只是我四处借力,拼凑出来的强大假象。和黑楼兰一方的雪山盟约,有时间期限。琅琊福地面临强敌,自身难保,八头荒兽也不是任何时候都能借来的。我方虽然逼退了仙鹤门,和十大古派都建立贸易,但既然已经浮出水面,这些门派一定会千方百计的,从各个方面调查我。他们现在没有动手,只是摸不准底细。等到摸透了我方的底细,就会再次动手。”方源心中不断思量。

%41
如果打个比方,形容形势。那么方源刚刚得到狐仙福地时,形势如暴雨倾盆,闪电交加,极为恶劣,只有舍身忘死,方能有一线生机。

%42
方源北原之行后,成为仙僵,回到狐仙福地,形势如乌云盖天遮地,大风卷席。前途一片黯淡,唯有精心谋划,小心翼翼,锱铢必较,抓住每一次机会,才能从泥潭困境中一步步艰难地爬出来。

%43
现在方源击退了仙鹤门,又开始贩卖胆识蛊,形势宛若狂风消散,乌云变淡,且掀开一角,露出蓝天。前途已见一抹光明,但乌云仍旧滚滚翻腾,稍有大意,仍会遮天蔽地,雷雨交加。就像是脱困之人,刚刚爬出泥泞,站在深潭的边缘,算是勉强站稳了脚跟。

%44
若后退一步,方源就要再次掉进泥潭,想要再爬出来,就得看奋力挣扎,还得看运气机缘。

%45
而向前迈步,方源则会越走越稳,真真正正踏上宽敞大道。纵有沿途的荆棘,也不至于像之前那般在生死间挣扎,苟延残喘,一步惊一步险了。

%46
“接下来这段时间,就要趁着好不容易挣来的安宁之期,向前迈步。积累底蕴,提升修为,越变越强,强到让周围势力越发不敢打我的主意。”

%47
方源细细想来,摆在他面前的,还有几大难题。

%48
近在眼前的第一难题,还是仙蛊喂养。

%49
净魂仙蛊首当其冲。

%50
琅琊地灵、黎山仙子处都没有白莲巨蚕蛊的线索,尽管在宝黄天中高价收购了一些,但也只能堪堪维持住净魂仙蛊,不让它饿死。

%51
要喂饱净魂仙蛊,需要上万头的白莲巨蚕蛊。这种蛊,就算是蛊仙手中有些收藏,但量也很少。

%52
除非方源得到蛊方,大量炼制,方能彻底解决这个难题。

%53
再看其他仙蛊,喂养也存在难题。

%54
虽然这一次勉强挨过了难关,但下一次喂养,却仍旧是个难题。

%55
六头大蛇的骨头已经彻底化为黑血,供给了招灾仙蛊。乐山乐水仙蛊的喂养问题,其实没有解决。只是之前墨瑶意志在上一顿喂饱了,才拖延下来的。

%56
浪迹天涯蛊,需要数万头的幽冥水母,以及数千头的深海闪电鳗鱼。前者虽然数量众多,但只要仙元石充足,不难收购。但后者却是市面上也较为稀少的。若非这次太白云生遇到鲨魔,又牺牲自己,以身犯险,还收购不到充足的量。

%57
平步青云蛊,放在琅琊地灵手中,倒是可以暂时不考虑。连云仙蛊需要的万斤沙鸥土,按照天地沙鸥的死蛋情况看,还可以再产三万多斤,还能支撑三次喂养。妇人心的喂养,比较麻烦。大屠杀这种事情做多了,就会引来正道蛊仙们的攻击。

%58
毕竟现在还不是五域乱战的时期。

%59
五域中,还是正道压住魔道,秩序占据上风。

%60
方源现在手头上,虽然仙元石多了,但喂养仙蛊,并非仙元石多就能够解决。

%61
其实喂养仙蛊,往往和蛊仙福地挂钩,和蛊仙修行是一脉相承的。

%62
ps:第一更提前点发,第二更要在21点左右了。

\end{this_body}

