\newsection{想想都有点小激动}    %第三百二十三节:想想都有点小激动

\begin{this_body}



%1
虽然光阴长河中的异变,只发生短短的时间,但却深深地印刻在方源的脑海中,留下深刻的印象。

%2
事关生死胜败,印象不深刻也不行啊!

%3
现在回想起来,方源的心中充满了疑惑。

%4
“鬼脸红莲,代表的是什么?为什么我之前几次使用春秋蝉,都没有碰到过这种情况?亦或者只有使用春秋蝉失败,才会引发?还是……”

%5
缓缓摇头,方源暂时不再思考这个神秘的难题。

%6
他一跃而起,脱离毒血,飞出龟壳。

%7
“身上的伤势真的不轻,而且唯一的成功道痕,也因为炼蛊失败而损毁了,真是可惜。”

%8
方源查看自身,很快就发现这次炼蛊失败,而带来的恶果。

%9
身躯中充斥裂纹,几乎濒临崩溃。这一身的伤势,必须修养数月,才能弥合。

%10
炼蛊本身就是一件风险不下于激斗的事情。

%11
更何况炼制仙蛊!

%12
方源这一次因为重生,导致炼蛊失败,遭受重创也是常理。

%13
更让方源遗憾的是,他辛辛苦苦参加炼蛊大会,好不容易挤进排名,得到的成功道痕,也因此毁掉了。

%14
这一点,让方源隐有体悟。

%15
“我之前之所以炼制星念仙蛊失败,不是因为我的手法问题,而是仙材之间本身的道痕相互交融,这个细微方面无法人为操纵。这也是所有仙蛊的成功率,如此低下的关键原因。”

%16
“成功道痕的最大作用,就是影响这个方面。让仙材本身的道痕交融,全部达到炼蛊的标准。所以当我开始炼制变化仙蛊的时候。成功道痕就开始损耗了,作为一种特殊的仙材。一直在辅助。”

%17
“上一世我没有出现什么意外,最终炼成了变化仙蛊,成功道痕居功至伟。但这一世,我因为重生,反而把这个十拿九稳的事情弄砸了。”

%18
想到这里,方源也有点哭笑不得。

%19
不是说有了成功道痕,不管怎么炼蛊都能成功。

%20
蛊师要作死,故意用错炼蛊手法,那铁定失败。

%21
成功道痕绝非万能之物。

%22
上一世。为什么方源要特意选择星象福地中炼制仙蛊?就是害怕被外力干扰,导致炼蛊失败。

%23
炼蛊一失败,仙材损耗、受伤还是小事,关键是成功道痕也会因此毁灭。

%24
方源重生回来,好死不死,就在炼制变形仙蛊的时间段,这就类似于“自己作死”了。

%25
没有办法,上一世炼制变形仙蛊是成功的,但这一世却失败了。

%26
“重生过来。虽有前知的优势,但落实到具体事务上,也不是无往不利啊。”

%27
“不过我这一次重生,优势很大。失去一个成功道痕。根本就不重要!”

%28
方源很快就收拾情怀,稳定心境。

%29
定仙游!

%30
方源催动仙蛊,回到狐仙福地。

%31
以前他很焦虑。但现在他知道了,距离中洲侦破八十八角真阳楼一案还有一段时间。他可以更加从容一点也无妨。

%32
虽有上一世的经验,但方源向来谋定而后动。急躁之人。绝不会走到他今天这样的程度。

%33
无数线索,芜杂繁乱,如何才能利用自己手头上现有的资源,做到利益的最大化?

%34
这是方源最需要考虑的主要问题。

%35
仙僵思维僵化,方源便蹭用智慧光晕,辅助思考。

%36
和上一世相比,他的智道境界已经达到宗师级!甚至,还设想了一套小蛊阵,可以更高效率地利用智慧光晕,节省大量时间。

%37
仅仅半天之后,方源思考完毕。

%38
仙僵不需要吃喝,虽然身上有伤,但方源要争分夺秒。

%39
定仙游!

%40
他立即催动这只仙蛊,首先秘密回到南疆。

%41
利用手段,尽量将仙蛊的气息消除大半,方源就悄悄接近义天山。

%42
义天山附近方圆万里,都被方源所熟悉。

%43
没办法,上一世义天山旷世赌约,方源参加了。随着时间,禁仙绝境的范围在扩大,方源也随着南疆蛊仙们,不断往外迁移。因此对义天山附近的风景,相当熟悉。

%44
义天山!

%45
此时此刻,却还只是一座无名山峰。

%46
方源来之前这里刚好下过一场雨,只见这座未来的义天山,高山秀丽,林麓幽深。彩虹散彩,日月摇光。

%47
方源远远眺望,小心潜行,满怀着期待。

%48
如果要问重生之后,最令方源心动的是什么?

%49
那么无疑就是义天山下,地底深处的仙蛊屋惊鸿乱斗台,以及被镇压的八转大力真武体仙僵。

%50
后者的重要意义,对于方源而言,就不需要多说了。

%51
“虽然上一世,这位八转大力真武仙僵居然自己蹦跶出来,摆脱了惊鸿乱斗台的镇压。但是只要我成功炼化惊鸿乱斗台,成为它的新主人,一样可以继续镇压他!”

%52
“如果我能得到惊鸿乱斗台的话……”方源想到这里,不禁舔了舔嘴唇。

%53
仙蛊屋可是非同小可!

%54
仙蛊屋、上古战阵,都是阵道流派的最高成就。

%55
其中上古战阵,已经被时代的浪潮渐渐淘汰,仙蛊屋已经成为主流。

%56
不论是南疆、北原,还是东海、西漠,亦或者中洲,但凡强大的超级势力,都至少有一座仙蛊屋镇压底蕴。就像黑家一样,黑牢就是黑家强盛的象征,也是极为强大的底牌!

%57
方源上一世,黑牢失去之后,为了夺回黑牢,就连黑家四大太上长老,都冒险出动,追捕黑城。

%58
黑牢还只是六转的仙蛊屋,黑家都要急忙追寻。

%59
仙蛊屋的重要性,可见一斑。

%60
而义天山地底深处的惊鸿乱斗台。不仅比黑牢还要更高一转,达到七转级别。并且历史上一位神秘七转蛊仙,甚至借助它的战力。硬生生地越阶战斗,成功镇压了一位八转强敌,还是大力真武体!

%61
惊鸿乱斗台的价值,比黑牢还要高出许多倍!

%62
“想那上一世中,鲨魔、苏白曼夫妇玄冰屋,这还是残破的仙蛊屋,便能屡次冲破战场杀招的阻隔,顺利脱身。这座惊鸿乱斗台,我若能得之。即便达不到纵横天下的程度。也能在将来,真相暴露,遭受中洲、北原等等追杀的时候,保存自身,震慑诸仙!”

%63
一座仙蛊屋,就是一座名副其实的战争堡垒。它没有短板,兼顾攻防、转移、治疗乃至修炼,等各个方面。

%64
它结构稳定,易于操纵。

%65
一座七转仙蛊屋。六转蛊仙也能驾驭起来。

%66
它有两个大缺陷,一个是喂养,第二个是仙元。

%67
仙蛊屋是由许多仙蛊,还有海量的凡蛊组成的。喂养这些蛊虫,必然十分麻烦,而且是巨大负担。

%68
但方源知道惊鸿乱斗台。似乎兼顾食道精髓,喂养的难题可以忽略不计。

%69
剩下来的。只是仙元问题。

%70
要催动一座仙蛊屋,消耗的仙元将是一个庞大的数字。仅凭方源个人能力。担负不起。

%71
上一世,鲨魔不得已催动玄冰残屋,众仙僵参与其中,分担仙元消耗,事后各个都肉痛不已。

%72
催动仙蛊屋的代价很大。

%73
因为消耗的仙元太多。

%74
方源现在一穷二白,他本来是孤注一掷,耗尽家底,勉强收集到了一次炼制变化仙蛊的仙材。

%75
而现在,他炼蛊失败,损失了大半的仙材,手头正紧。

%76
就算他有许多盈利的项目,比如胆识蛊贸易,月入上千的仙元石。但积攒个半年,也不够仙蛊屋消耗半分钟的。

%77
但这一切完全没有关系!

%78
方源自身的青提仙元很少,但他手中有大量的巨阳仙元!

%79
这笔收获,还要追溯到王庭福地。

%80
巨阳仙尊是九转级数,他留下来的仙元,乃是九转尊者才有的黄杏仙元。这种仙元的质量,要把青提仙元甩出数百条大街去,两者之间的差距,好比天和地。

%81
这笔巨阳仙元,方源自身利用不了,但仙蛊屋可以吸收啊。

%82
仙蛊屋可以吸收异种仙元,要不然鲨魔等仙僵,如何一齐出力,催动玄冰残屋的?

%83
这是仙蛊屋的基本特性集合众仙之力为一体!

%84
方源上一世,也想到这点。

%85
对他而言,巨阳仙元是无源之水无本之木,用了多少,就损失多少,再也无法补充。

%86
所以更加经济实惠的方法,就是纠集更多的蛊仙,充当推动仙蛊屋的劳力。

%87
可惜南疆的那些蛊仙,死的太快了。

%88
或者说,方源没有来得及。

%89
惊鸿乱斗台还未被任何一人炼化,却被大力真武仙僵提前破出。

%90
这是一个巨大的遗憾。

%91
“若是我有惊鸿乱斗台在手,就算是那道取我性命的光柱,也能挡下!”方源心怀自信。

%92
惊鸿乱斗台当然不是监天塔的对手,但是监天塔本身状态不佳,那道光柱经过十绝仙僵无生大阵的削减,威力也并不充沛。

%93
有了惊鸿乱斗台,虽然谈不上“一屋在手,天下我有”。但也足以称得上“一屋在手,进退自如,纵横自由”了。

%94
“一屋在手,天下我有”,八转的八十八角真阳楼可以称得上这一点,它擅长搜刮蛊虫,甚至仙蛊。八转的炼炉当然也能算得上。

%95
对比仙蛊屋惊鸿乱斗台,什么我力仙蛊、全力以赴仙蛊、星念仙蛊,都显得不那么重要了。

%96
只要掌握仙蛊屋,又有巨阳仙元,方源就能以六转垫底的修为,苦逼的仙僵身份,强战八转,直接走上人生巅峰!

%97
“想想都有点小激动呢。”方源小心探查,发现周围没有人烟,立即钻破地面,深入地底。

\end{this_body}

