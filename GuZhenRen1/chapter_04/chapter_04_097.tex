\newsection{喜闻一法,可脱僵身}    %第九十七节:喜闻一法,可脱僵身

\begin{this_body}



%1
几天之后,太白云生回到狐仙福地。

%2
方源与之相见,看到他眼蕴喜色,便道:“老白,看来你这一次东海之行,定有喜人的收获。”

%3
太白云生哈哈一笑:“师弟法眼无差,的确如此。不妨猜猜我这次有什么收获?”

%4
方源不由奇道:“玉露仙子乃是乐土仙尊座下的近侍女童,她是土道、水道兼修,擅长防御和治疗。你们攻打福地的阵容,我也知道。居然一次成功,直接将玉露福地攻打下来了?”

%5
事实上,方源之前并不太看好太白云生此行。

%6
太白云生神情一僵,叹息道:“原来师弟一直都不看好我呀,还真是叫师弟你说中了。此次我们攻略福地七天七夜,最终被地灵击败,损失惨重。一位蛊仙险些身陨,结果被我用人如故仙蛊救下。”

%7
“哦?”方源扬起眉头。他心知人如故仙蛊的价值,太白云生一暴露此蛊,定然在其他人的眼中价值迅速高涨。

%8
果然,太白云生接着道:“我的人如故仙蛊被得知之后,受到许多恭维。而后又接到鲨魔的盛情邀请,请我去往东海僵盟做客。我哪敢孤身一人,前往别人的大本营?便婉言谢绝。结果鲨魔等三位仙僵,在海市福地款待了我,酒席间让我得知了一个重大消息。”

%9
说到这里,太白云生顿了顿,止住话头,目光炯炯地看着方源。

%10
方源心中一笑,便顺着他的意,问询道:“呵呵呵,到底是何重大消息?值得让你来卖关子?”

%11
太白云生神情一肃。振奋地道:“是能让师弟你重获新生,脱离仙僵之体的办法!”

%12
“哦?”方源挑起眉头。

%13
他自己加入了北原僵盟,还未有所进展。不想太白云生,反而在方源之前,有所斩获。

%14
“是什么法子?”方源追问。

%15
太白云生便道:“鲨魔有一位双修道侣。名为苏白曼,也是仙僵之体。早年时候,苏白曼抓住一次机缘,击败强敌,获得了一只宙道仙蛊,称之为宙锚。此蛊能在光阴长河中。做下标记。本身无用,但若搭配其他宙道仙蛊,却有极好的效果。”

%16
方源听到这里,顿时明白太白云生所言。

%17
他双眼一亮,但旋即又沉寂下去:“老白。我懂你的意思了。宙锚仙蛊我早已听闻,若是搭配你的江山如故,可以令江山准确地回归到过去的某个时刻。若是配合你的人如故,理论上也可以将目标,还原成过去某个时刻的状态。不过这一切的前提,都需要宙锚仙蛊提前落下标记。”

%18
“不错。我的人如故仙蛊,只能还原到一瞬之前。但若搭配宙锚仙蛊,却有可能突破一瞬的时限。师弟。你或可凭此方法,摆脱仙僵身份了。”太白云生兴奋地道。

%19
太白云生所言不虚。

%20
蛊是天地真精,功用单一。各自不同。仙蛊唯一,威能虽猛,却有弊端限制。

%21
人乃万物之灵,最擅智慧和创造。将不同的蛊虫组合起来使用,就能提升效用,互补不足。

%22
“有宙锚仙蛊搭配。人如故仙蛊的确可以摆脱一瞬的时限。但是……”方源摇摇头,继续道。“两只仙蛊相互搭配催动,并非简单的一起催动。而是需要很多的凡道蛊虫辅助,甚至仙蛊调和,从而形成仙道杀招。宙锚和人如故搭配,的确在理论上是可行的。但你们没有相应的仙道杀招。想来鲨魔等仙僵盛情款待你,就是为了让你帮助他们,贡献出人如故,方便他们研究出仙道杀招来吧?”

%23
太白云生连连点头,赞道:“师弟你是聪明人,猜得很准,的确如此。”

%24
“可惜,可惜。”方源再度摇头。

%25
“可惜什么?这个法子,我觉得大有希望啊。”太白云生疑惑地看着方源。

%26
“老白,你要知道,我成为仙僵已经铸成铁一般的事实。也许鲨魔等人,早已经在成为仙僵之前,用过宙锚仙蛊,在光阴长河中落下了当时的标记。我却没有。因此就算这道杀招研究出来,对我也是无用的。”方源叹息道。

%27
太白云生便笑:“师弟,你担心的对。不过我还要告诉你一个好消息,那鲨魔和苏白曼,却也没有动用宙锚仙蛊,及时地种下标记。”

%28
“哦?”

%29
“当初,苏白曼击退强敌,虽然获得宙锚仙蛊,但不知它的用途。苏白曼当时重伤濒死,不得不转为仙僵。鲨魔得知之后,为了陪伴他的夫人,竟也在寿元充足的情况下,毅然转为仙僵。两人成为仙僵的数年之后,这才搞清楚宙锚仙蛊的来历和用法。”太白云生道。

%30
“是这样啊。”方源神情微动。

%31
得到神秘仙蛊,花费数年,乃至十多年才弄清来历和用法的情况,不在少数。

%32
当初方源在王庭福地中,也得到了许多仙蛊,不敢胡乱试验用法。

%33
幸亏有琅琊地灵在,靠着他,这才弄清楚。

%34
太白云生接着道:“当两人得知宙锚仙蛊的用法之后,大为懊悔,但此时早已经晚了。我亦有和师弟你同样的顾虑,在酒宴上直接相问。鲨魔等仙僵则又说了一个法子,提到了春秋蝉。”

%35
“哦?”方源神色微微好奇,又掺杂疑惑,“春秋蝉可是那只奇蛊榜上之物?传说中能逆流光阴长河,回到过去。可惜一直得不到证实。”

%36
太白云生不疑有他:“春秋蝉不靠谱,提到春秋蝉,只是举一个例子。鲨魔是想说,虽然错过了宙锚仙蛊的使用时机,但若有春秋蝉这类的仙蛊,能将蛊仙带到光阴长河面前。到那时,利用宙锚仙蛊在上游落下坐标,再回来使用人如故仙蛊。我认为这个法子,也是成功的可能。”

%37
方源微微合上眼帘,一时间陷入思索,没有开口。

%38
太白云生并不知道,眼前的方源,就是当今春秋蝉之主。不管是鲨魔,还是太白云生,都对春秋蝉并不了解,流于传闻流言一层。

%39
方源心想:“若是我能直接动用春秋蝉,早就动用了。何必到时再用宙锚仙蛊多此一举呢?不过此法,在理论上,的确可行。”

%40
方源和智慧蛊达成协议,算是某种程度上,掌握了智慧蛊。

%41
他能够从仙僵的泥潭中迅速崛起,在大半年内勇猛精进,迅速改变境况,其中一大半的功劳就落在智慧蛊的身上。

%42
方源若想用春秋蝉重生,摆脱仙僵麻烦,早就用了。且不说使用春秋蝉有失败的概率。

%43
只说北原之行,方源实力太弱,就算重生再来一遍,变数太多。方源要想获得更大的利益,连半成的把握都没有。

%44
和智慧蛊达成约定,是天时地利人和。那时的环境太特殊了!九转智慧蛊,就是方源此行的最大收获。没有得到智慧蛊,就谈不上更大的利益。

%45
“然而除去春秋蝉之外,的确有其他的宙道仙蛊。比如观往仙蛊、历历在目仙蛊,都可以让蛊仙立足现在,视察过去发生的事情。只是两者之间有差别。使用观往,能在地老木上,看到往昔繁华盛景。使用历历在目仙蛊,则是观察自身过去的经历。两大仙蛊合炼,就能形成八转仙蛊极往,能抽离出蛊仙意志,俯瞰光阴长河,观察过去发生的一切种种。”

%46
方源在心中推理:“若是有宙道侦察仙蛊极往,和宙锚仙蛊搭配,也许能形成宙道仙级杀招,立足现在,标记过去。之后,再用人如故仙蛊,就能还原身躯,重新复活,又不会失去智慧蛊。当然,极往仙蛊也只是一个例子。也许还有其他宙道仙蛊,更适合搭配宙锚呢。”

%47
方源思索良久,这才睁开双眼。

%48
“老白,多谢你了。”他诚挚地感谢道,“虽然这个法子,前景开阔,但耗费的精力、物力、时间,都不在少数,是项百年工程。但到底是一个希望。”

%49
“一家人,不说两家话。谢什么谢。”太白云生立即道。

%50
不过,对方源所言,老太白心中也有数,叹息一声,接着道:“正是如此,至少是个希望。虽然时间漫长,但这个期间,师弟你可以用僵盟盛产的黑油,延缓福地崩解。黑油虽然难以弄到手,但凭我和鲨魔的合作关系,倒也方便得很。”

%51
方源笑道:“你离开的这段时间,我已经用假冒身份沙黄,加入了北原僵盟。”

%52
“原来师弟你早有计划。”

%53
方源又接着道:“我要参加北原的拍卖大会,你也一起来吧。这是凡道杀招见面不相识,出自盗天魔尊的手笔。还有几道云道杀招,也给你看看,你不是兼修云道的么?”

%54
太白云生接过这些杀招,心中不由地涌出一股暖流:“我在念叨着师弟的同时,师弟也在为我着想啊!”

%55
方源又给他看了蛊仙俘虏,告诉他和琅琊地灵的此次交易。

%56
太白云生震惊不已,随后喜忧参半地道:“看来这次的拍卖大会,会是一场好戏!我跟你一起去,最好将黑楼兰、黎山仙子都叫上。”

%57
“她们必然会去。虽然在黑城面前暴露了联盟的关系,但目前还是不宜公然和她们同行的。”方源道。

%58
“师弟所言有理。”太白云生点点头。

\end{this_body}

