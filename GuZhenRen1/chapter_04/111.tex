\newsection{运道真传精髓}    %第一百一十一节:运道真传精髓

\begin{this_body}

双方暗中迅速交接,交易完成的时候,秦百胜已经在拍卖下一只仙蛊了。

凤九歌脸上面无表情,看着手中的乐山乐水仙蛊,目光有些闪烁。

他心中不妥之感,仍旧萦绕不散。

“费了连连波折,终究是将这只仙蛊得到手中。纵然花费不菲,但论及门派任务奖励,却足以弥补,甚至还有超出。”

“吃力仙蛊虽然珍贵,但只有对于力道蛊仙,才有用途。”

“就算牺牲了见面似相识,也只是给了部分。就算给了全部,唯一的核心仙蛊亦在我手,不怕对方运用自如。”

凤九歌心中反思几句,脸上这才渐渐缓和。

大厅中,不少蛊仙争相竞拍着仙蛊,方源也和凤九歌一样,不在台上,也是在检视所得。

吃力仙蛊在凤九歌主动放弃的情况下,方源轻松炼化,已经收入仙窍,并无任何不妥。

那份残缺的智道传承,比之方源先前的恶念传承,稍稍完整一些。有一套从一转到五转的忆念蛊的蛊方,一只追忆仙蛊的仙蛊方,以及数件凡道杀招。

“这忆念蛊也比较适合我。我的前世记忆,自己只能回忆大概。有此蛊辅助,应当能回忆得更加清晰一些。若是能炼成追忆仙蛊,那就更好了。”

方源查看了一番,满意地将这份智道传承收起。

短时间之内,他不打算尝试炼制仙蛊。炼制仙蛊成本极大。风险极高,方源手头中已然有不少仙蛊,追忆仙蛊也并非必要。只是锦上添花而已。

方源的当务之急,仍旧是摆脱仙僵身份,重获新生。

哪怕有吃力仙蛊,可以增添道痕,但也只是稍稍弥补,比不上正统渡劫。

渡劫虽然危险,但增添的道痕极多。若非如此。黑楼兰也不会拼命渡劫了。

方源重点检阅了见面似相识。

这是一份残缺的仙道杀招,初步推断,只有五成。

不过方源却也已经心满意足。

他知道。对方不可能将完整的仙道杀招,全数卖给他。能得到这样的成果,已经不错了。

尤其是这当中的内容中,并未掩盖关键的核心仙蛊。

“核心仙蛊。乃是杀招的根本。如今在被灵缘斋掌握。我虽然得不到,但却可以改造杀招。”

方源的打算本就是如此。

他有变化道的境界,更有智慧蛊的帮助,还有见面似相识的基础完整的见面不相识。

如今再添上这份五成内容的见面似相识杀招,把握已然不小。

“见面三相识,是盗天魔尊的经典之作。但前贤遗智,我借鉴即可。说到底都是人,旁人做得到。我为何就做不到?因此不必拾人牙慧。”方源这点气量还是有的。

片刻之后,秦百胜已经将最后的几只仙蛊拍卖。其中倒数第二只仙蛊。不幸地遭遇了流拍。

蛊仙们的期待之情,已经从各自的目光中喷薄而出。

众所周知,接下来,便是拍卖环节的重头戏拍卖运道真传精髓!

和秦百胜暗中沟通之后,方源和太白云生,都及时地改换了密室。

不止是他们两个,许多蛊仙也同样如此行事。

毕竟,如果接下来的运道真传精髓,价值巨大,许多蛊仙都希望别人不知道自己得手了什么仙蛊。

其实,允许随时可以改换密室,也算是拍卖大会的规矩漏洞之一。

这场拍卖大会,到底匆匆举办,应势而生,漏洞不少,远不如方源前世五域乱战时期的健全。

对于这点,秦百胜其实心知肚明,但也无可奈何。

他举办这场拍卖大会,是为了保全自身。规矩太多,就会得罪许多势力,许多人物。

对他而言,拍卖大会举办一次就足够了。他当然不愿因此,去得罪他人。

“下面竞拍的第一项运道真传精髓,是一套察运蛊蛊方。”秦百胜的话,刚刚开头,就引发了许多蛊仙的强烈关注。

察运蛊!

顾名思义,能够观察到气运的侦察蛊虫。

气运无形无质,用肉眼或者常规手段,都察觉不出。唯有用这察运蛊方可。

可以说,察运蛊是运道真传的最主要的基石之一。

尤其是当中的一些蛊仙,之前买下了一些运道仙蛊,诸如走运蛊、封运蛊、转运蛊、断运蛊等,更要配合察运蛊。

秦百胜将这个察运蛊的蛊方,特意放在最后,也是有心思的。

看着台下蠢蠢欲动的蛊仙们,秦百胜接着道:“这套察运蛊方,不仅包括了从一转到五转的凡道蛊方,而且还有七转察运仙蛊的仙道残方。”

“怎么,没有察运仙蛊吗?”百足天君问道。

秦百胜微笑道:“的确有一只察运仙蛊,乃是六转,如今在我手中,却是不想拍卖。不过我可以保证,除了这只仙蛊之外,有关察运蛊的一切,都在此刻的台上了。”

因为之前的盟约,秦百胜的话令人信服。

“不卖察运仙蛊?”当即,大厅中一位蛊仙扬眉,公开表示不满。

秦百胜冷哼一声,目光一转不转地盯着出声之人:“一只六转的察运仙蛊,我自问我还是能够保下的。郭荣你有意见?你若买下这只仙蛊,你能保得住吗?”

蛊仙郭荣脸色一变,不再说话。

他只是六转蛊仙,秦百胜却是七转强者,两者不可同日而语。

再扫视周围,郭荣不禁冷汗涔涔,暗骂自己糊涂!他想借众人之势,威逼秦百胜。但众仙却都是冷眼旁观。

的确如秦百胜所言。他虽然被迫举办拍卖大会,但一只六转察运仙蛊,还是有能力保住的。

这是实力所致。

经过了这个小插曲。秦百胜旋即道出竞拍要求:“因为此项拍卖,不涉及到仙蛊,因此可以用任意资源换取。不管是仙元石,还是清单上的仙材皆可。”

说着,他又抛出一份清单。

清单上的仙材,只有十多种,数量要求也不多。不像方源抛出来的清单那么吓人。

秦百胜并不擅长炼蛊。炼蛊消耗甚大,散修很少擅长此道,除非极有天赋之人。

方源揣测。这些仙材更大可能,是被用作仙蛊食料。

拍卖开始,气氛热烈至极。

因为不涉及仙蛊换卖,导致竞拍门槛甚低。一时间大多数蛊仙都掺和一脚。纷纷竞价。

大厅中,叫价声此起彼伏,声势之壮大,堪称拍卖以来的第一。

价格也在浓烈的氛围中,飞速攀升。

秦百胜含笑看着,既然举办这场拍卖大会,他当然安排出更有利于自己的流程。

方源也竞价了几次,旋即就被众人超越。

“察运蛊我本就有一只。可惜凡蛊无法对仙人有效。这套凡道蛊方,对我无益。秦百胜坦言自己有察运仙蛊。我就算得到七转残方。推算出来,仙蛊唯一,也不能炼制出察运仙蛊。”方源明智收手,不再出价。

他只是退出竞价的一员而已。

激烈的竞价,很快淘汰了一大批人。

最终剩下药皇、东海八转蛊仙、凤九歌、袁让尊等一些强者。

这些人物,都有一个共同点皆是各大势力的成员。

对于他们而言,凡道察运蛊方比七转残蛊方,更有吸引力。

有了凡道察运蛊,就可以观察凡人。有了它们,这些势力就有了一套全新的考察人才的标准。

一直以来,蛊师界都以资质为考察标准。

但除去资质之外,还有个人的才情、心性,以及机缘。

才情包含战斗天赋,悟性,炼道天赋种种,可以在修行的时候,渐渐崭露头角。

心性则也可以通过交际,熟悉洞察。

唯有机缘,却是难以揣摩测度。很多时候,往往资质不佳,才情不高的蛊师,因为机缘,一飞冲天或者意外崛起。这是常有的事情。尤其是这个世界,都有蛊师传承留下的文化传统。

这些势力,若有此蛊,无疑就能更加全面地考察人才,提拔人才。

那些个人散修,都退出竞拍。但对于大势力而言,吸引力却是极强,一个个都在较劲,毫不相让。

最终,这套凡蛊方卖出天价,好几位势力首脑,差点都要用仙蛊换取。

还是药皇一句话:“我以自身名誉担保,我先购之,必在接下来的自由交易中,向大家出售此套蛊方。”

药皇乃是北原五大八转之一,名望最高。又是正派领袖级人物,由他当面担保,众人这才罢手。让秦百胜暗恨不已,表面上却浮起微笑,不敢得罪药皇。

“接下来,拍卖第二项。一只运道七转仙蛊,名为逢凶化吉!此蛊顾名思义,能令人的厄运坏运,转化为好运吉运。”秦百胜朗声道。

这只仙蛊,和招灾仙蛊差不多。都是运道,都是七转,但反响却差天地别。

不少七转蛊仙,乃至八转大能纷纷竞拍。

有此仙蛊在手,根本无需什么察运蛊搭配,十分实用。不愧是运道真传的精髓。

围绕逢凶化吉,展开激烈的竞争,惨烈程度不下于战场上的血腥厮杀。

因为竞拍底价,是一只金道七转仙蛊,方源也只能作壁上观。

最终,这价格一路上涨,直接越过第一的浪迹天涯,第二的乐山乐水,后来居上,成就高价记录的首位。(未完待续。)

ps:通知一下:以后的两章更新都在20点左右。另,今晚的第二更要晚一些,大家可以明天上午看。

\end{this_body}

