\newsection{两大盛事}    %第八十五节:两大盛事

\begin{this_body}

%1
“想不到,只是四百多年前的今天,梦境就已经显现端倪。”方源捏着手中的报信青鸟蛊,心中叹了一口气。

%2
一时间,他再无试验杀招的兴致。

%3
方源前世,中洲首先掀起浩大的战争序幕,以一洲之地力压其他四域,闪电般扩张。随后四域联手,艰难地抵抗住了中洲的进攻。

%4
整个战局陷入僵持阶段,正当四域联手,企图反攻中洲的时候,梦境在五域各个地方外显。

%5
在这个世界上,自从人祖衍化出第一批人族,人类就开始做梦。每一次做梦,都为梦境增添一份版图,一份力量。

%6
天长日久,历经沧桑岁月,万千变化,人族涌现出无数英雄鬼杰,不断地扩充梦境。

%7
梦境越来越庞大,终于达到质变,外显出来。

%8
世人很快发现,尽管梦境环境极端复杂,凶险异常,但却蕴含着巨大的利益。从梦境中,后人可以获得先贤的底蕴,让各自修行的境界得以飞速的提升。

%9
境界的提升,动辄数十年,上百年,数百年!

%10
蛊师们因为梦境,而令境界迅速提升,从而又带动实力的暴涨,井喷似的涌现无数的人物。

%11
在此之前,资源大多为家族、门派把持,境界又难以提升,除非是惊才绝绝,否则一般蛊师难以出头。

%12
在梦境出现之后,境界暴涨,站在先人的肩膀上,蛊师们可以很容易地让无数的蛊虫、杀招得以重现,在此基础上创造出适合自身发展的新蛊、新蛊方、新杀招。

%13
那是一个波澜壮阔的超级大时代。

%14
许多不受待见的小势力,纷纷崛起。无数形单影只的散修,接连冒头。

%15
梦境的出现,冲击着固有的体系和格局。秩序丧失,巨大的利益,让世人为之狂热!

%16
原本四域联合,对抗中洲的计划,还未进行,就彻底搁置。

%17
各方势力,无数蛊师蛊仙,都将目光瞄准了梦境。

%18
为了争夺从各个地方外显的梦境,五域彼此攻伐,陷入彻彻底底的乱战当中。

%19
而为了更好的探索梦境,无数有识之士,或者天才鬼才,开始研发相关的蛊虫。很快,梦道这个崭新的流派,迅速建立起来,并飞速发展。

%20
八转智道传奇蛊仙一言仙的“三尊说”,再次被人翻出来。

%21
他预言:无数年后,历经三个大时代,会分别出现两男一女三位尊者。第一位是幽魂魔尊,第二位是乐土仙尊,第三位是大梦仙尊。而关于遁空蛊的难题,将会在大梦仙尊手中解决。

%22
幽魂魔尊、乐土仙尊已经出世,人们因此更加确信三尊说的可信。

%23
两者联系起来,很快就形成一个普遍的共识梦境就是成为大梦仙尊的契机,谁能占据整个梦境,谁就能借助前人的积累,成为史上超越诸贤者,仙王之王,尊上之尊的大梦仙尊!

%24
“我前世能够崛起,成就一方霸业,除了血道机缘之外,就是借助了五域乱战的契机。各大超级势力对局势的掌控力大为降低,又专注于梦境的探索,无暇顾及现实大局,才令我有自由发展的空间和时间。可惜,待我在现实中站稳脚跟,梦境这个全新的版图,已经被各大势力霸占瓜分。我虽然可以借助梦道凡蛊,进入梦境探索。但属于起步阶段,大大落后于他人。除非取得梦道仙蛊,才有后来追上的可能。”

%25
正因为这一点,方源才去伙同宋钟等魔道蛊仙,一道攻伐狐仙福地,杀了凤金煌。但可惜的是,却没有得到她的梦翼仙蛊。

%26
没有仙蛊,只有一些梦道的凡蛊,导致方源后来探索梦境,一直遭受到各大势力的驱赶打压,根本无法拥有什么重大的收获。

%27
方源前世继承了血道传承,一直遭受通缉和追杀。后来建立了血翼魔教之后,也为魔教殚精极虑,大大牵扯了精力和时间。一来二去,也就耽误了探索梦境的黄金时段。

%28
他重生之后,也在反省自己。这个决策性的失误,一大半是外力逼迫,另一小半则是他的决策失误。

%29
所以今生,他牢记这个教训,就算在血道上有底蕴,也没有在拿到那个血道关键传承前,冒然修行血道。

%30
血道蛊师人人喊打,不仅是正道通缉,就连魔道也是深为忌惮。唯有在五域乱战时期,秩序崩坏,血道才能冒头,别人无暇顾及,才有可能兴旺发达。

%31
“原来梦境早在四百多年前,就已经外显了。各大超级势力早有接触,因此到了三百多年后,梦境四处外显,这些势力就立即进驻梦境,霸占扩张。他们早有准备,我前世输给他们也不算冤枉。”

%32
方源前世这个时期,还是凡人,在蛊师底层摸爬滚打,被来自地球的记忆和经历束缚。直到后来很久之后,才算打破了思想的枷锁。

%33
有时候,穿越者的优势,也是劣势。

%34
这也导致了他的记忆情报,和真正发生的事实有所偏差。

%35
譬如繁星洞天就是这样。

%36
星宿仙尊就是七星子仙僵。方源前世所知,这位星宿仙僵是在繁星洞天碎块世界中捞取好处最多的幸运儿,但其实他就是繁星洞天的主人。

%37
他之所以自称为星宿仙僵,恐怕是由于星宿仙尊的一处梦境,外显在他的福地里面。使得七星子因而获得了星宿仙尊的某些积累。

%38
“七星子的事情,是原本的事实。但凤金煌若真的领悟出梦翼仙蛊的真正用途,却是比前世大为提前了。难道这是我夺取了狐仙福地,所带来的影响吗?”方源沉思着。

%39
“凤金煌领悟出梦翼仙蛊的真正用途,势必惊动灵缘斋,灵缘斋的动向必然又影响其他九大古派。从九大古派再辐射至整个中洲。天下没有不透风的墙,如此一来,世人探索梦境的进度,或许因为我而大大提前了。”

%40
方源想到这里,顿时有一种紧迫之感。

%41
梦境是一个全新的版图,凶险复杂,资源只有一种,却是无以伦比的丰厚。前世的错误,方源不会犯第二次。

%42
“然而现在这种情况,梦境并未大规模外显。外显的梦境,可以用梦道凡蛊配合探索。没有外显的梦境,只有用仙级梦道仙蛊才有可能。”

%43
对于仙级梦道仙蛊,方源可以如数家珍。

%44
但就目前而言,凤九歌、白晴仙子健在,梦翼仙蛊在凤金煌这个凡人手中,却是没有人敢来打主意。

%45
入梦游在北原蛊仙毒娘子的手中,她沉睡在紫毒福地中,乃是运用入梦游探索梦境的第一蛊仙,后世以独自散仙的身份,跻身成为梦境第三霸主。毒娘子是七转蛊仙,本来的战力比黑城还要强大,福地中的防守更是固若金汤。根本不好去打主意。

%46
至于其他梦道仙蛊,不是没有被创造出来,就是还未出世,踪迹不显。

%47
方源思索片刻,调动真元,抹去报信青鸟蛊中的内容,用念头重新填写。

%48
他答应了凤金煌的邀斗,在炼道大会上决一胜负。

%49
“凤金煌有梦翼仙蛊,有信心,我的炼道境界也已经从原本的大师级,晋升到了准宗师级,怎么会怕这个后期之辈?”

%50
方源心中冷哼。

%51
为琅琊地灵推算仙蛊方,是一次契机,让他的炼道境界在最近成功突破。

%52
“炼道大会,事关不败传承,如此盛事,就是凤金煌不邀斗,我也要主动前往去分一杯羹的。不过参加炼道大会,炼蛊的材料却需要自备。我要有所收获,从现在开始就要筹措材料了。”

%53
方源前世就有参加过炼道大会的经验。

%54
要想夺得名次,准备的材料就要又多且好,价值得有五百块仙元石以上。巧妇难为无米之炊,就算是有再优秀的蛊方,没有材料也难以为继。

%55
“除此之外,秦百胜联合黎山仙子举办的拍卖大会,我也要参加。”根据最新消息,这场拍卖大会,将会有交换仙蛊这个环节。

%56
仙蛊是买不到的,通常而言,只有交换一途。

%57
蛊仙得到仙蛊,不一定符合自己的流派。这个时候,往往和其他蛊仙进行交换,换取符合自己流派的仙蛊,才能施展出拔群的效果。

%58
就像方源拥有浪迹天涯仙蛊,这是水道仙蛊,方源使用的效果,比水道蛊仙使用要差上许多。

%59
这是因为浪迹天涯仙蛊是水道法则,而方源为力道蛊仙,身上是力道道痕,力道水道相互干扰,反而降低了仙蛊的正常威能。

%60
方源前世五域乱战时期,因为战斗频繁,导致此类的换蛊大会也举办得很多。但现在距离五域乱战还早着呢。

%61
这场盛况空前的北原拍卖大会,可能是近两百多年来,唯一的一次大规模换仙蛊的机会。方源自然要把握住。

%62
除此之外,他对运道传承也十分感兴趣。

%63
“拍卖大会、炼道大会,都需要大量的仙元石。我手头上的仙元石,只有近两百块,却是难以参与任何一场盛事。看来只有借助琅琊地灵售卖蛊仙福地的想法了。”

\end{this_body}

