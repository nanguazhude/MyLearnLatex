\newsection{试仙蛊}    %第一百二十节:试仙蛊

\begin{this_body}

狐仙福地。**

一座晶粉大山,屹立在狐仙福地中央,正是《人祖传》中有载荡魂山。

此时,方源悬浮在荡魂山的上空,化为原形,身高三丈,青面獠牙,虎背熊腰,八只怪臂狰狞可怖。

不过如此身形,和荡魂山一比,却宛若蝼蚁般渺小。

方源双目绽放神芒,紧紧盯着下方的荡魂山,忽然轻喝一声:“拔山!”

下一刻,一股无形的磅礴力量,从他体内迸发而出,迅速向荡魂山挪去。

方源张开八臂,呈现怀抱之状,仿佛抱着一个巨大的东西。他牙关紧咬,使出浑身气力,好像要搬举什么重物。

轰隆隆……

起先这股声音,只是微微响动。但转眼间,音量越来越大,整个小天地都开始轻轻震动。

发出声音的,正是荡魂山。

在方源的催动下,这座水晶大山,竟然开始缓缓拔升,离开了地面。

方源咬牙坚持,八只手臂肌肉高高鼓起,宛若块块厚石。

不只是手臂,还有他全身的肌肉,都被完全调动起来。身上咯吱作响,这是骨骼不堪重负的声音。

在他的努力下,荡魂山越拔越高,渐渐远离地面,达到五六寸的高度。

一旁,狐仙地灵看得小口微张,一脸吃惊之色。

“再起。”方源存心要好好试验一番,达到此步。仍旧没有停住。

他身上渐渐亮起一条条光线。

这些线,布满他的全身,并不紧密。看起来十分稀疏。

光线越来越亮,与此同时,方源拔起荡魂山的速度,也变得越来越快。

起先,山体只是一寸一寸的提高,如今幅度越大,一尺一尺地飞上天空。

方源一边催动着仙蛊拔山。一边抽空查看自身。

这些在他身上崭亮的光线,正是力道道痕。

力道道痕分布在他的全身,杂乱无序。有的相互纠缠,聚集在一块儿,有的则一根根,独立占据某片肌肤。

“我全力催动力道仙蛊拔山。身上的力道道痕和仙蛊形成了共鸣。仙蛊的威能没有改变。但是在我此时催动时候,却能多出近两成的力量!”

方源又坚持片刻,直到浑身骨骼都出现裂痕,荡魂山升上二三十丈的高度,这才将荡魂山缓缓放下。

巨大而且沉重的山体,落到地面上,发出闷雷般的嗡响,掀起滚滚烟尘。

方源大喘粗气。浑身酸软,肌肉也多有损伤。

不过几息之后。在仙僵之躯的强大恢复能力面前,不管是骨骼还是肌肉的伤势,都迅速痊愈。

方源气息渐稳,仍旧悬停在半空中,缓缓闭上双眼,精心回味刚刚的感受。

在使用力道仙蛊拔山之前,方源就进行了多种推衍。如今正在用了,体悟更加深刻,所谓实践出真知,自是此理。

“我自从成为力道蛊仙之后,身上的力道道痕就没有增加过。因此全力催动,力道道痕共鸣之下,只能增添近两成威能。”

“催用拔山仙蛊,会持续不断地消耗仙元。刚刚这段时间,已经耗费我了三颗青提仙元。”

“仙元消耗看似巨大,不过考虑到对象是荡魂山的话,耗费并不多。”

荡魂山是一处奇地,可以产生胆识蛊,整个九天五域都独一份。这样的重地宝地,就算是大雪山,都无法媲美。

方源凭此猜测,若是他拔去其他的普通山峦,应该比荡魂山要容易得多。

“好一个拔山。”片刻后,方源睁开双眼,轻声称赞,心底十分满意。

要知道,但凡山峦,都是不是轻易就能拔取搬运的。

树有树根,山有山根。随意拔取,断根就毁。

按照气道的论述,整个世界由天地二气构成、流转。二气交汇,跌宕不休,便构成诸般自然景观。

譬如风雨雷电,便是天气充足,夹杂屡屡地气。

再如山峰丘峦,便是地气凝聚,掺和少许天气。

毁灭永远比建设,要容易得多。

山川容易毁灭,不容易转移。山根和大地连绵一起,地气相互凝结勾连。宛若高楼的地基,若强用蛮力拔去,便会直接断去地气,毁灭山峰。

但拔山却有一股玄妙威能,直接断开地气联系,不损山体。落下之后,又能地气相连,这绝不是普通手段能够做到的。

“拔山、挽澜……接下来再试试这只力道仙蛊。”

方源离开荡魂山,靠着小狐仙之助,瞬间来到福地东面,悬浮在湖水面前。

狐仙福地的东面,分布着大大小小的湖泊。乃是当年,方源沟通东、北两地,以水灭火而成。期间,牺牲了无数石人性命。

如今这里的湖泊中,豢养着大量的青玉鲫鱼,许多气泡鱼,还有少数水狼群生活着。

湖泊上方,是阴云绵绵成一片。

阴云上,种植着大量的星屑草,形成空中草原。草原中,一股股的星萤虫群,四下飞舞。普通的虫群中,夹杂着比例较高的星萤蛊。

因此,狐仙福地东部,虽然阴云浓郁,但却星光灿烂。空中是星云草原,地面上是大大小小,星罗棋布的湖泊,波光粼粼,反射着星芒。

方源此时来到一处最大的湖泊上空,催起挽澜仙蛊。

同样是一股无形力量,深入湖泊当中。

方源缓缓举起双手,顿时哗啦一声,大量的湖水凭空被提了上来。

和拔取荡魂山相比,方源提取这里的湖水,简直容易得宛若拎取一张白纸。两者难度是天差地别。

湖水形状,仍旧和原先一样。并未改变。里面大量的鱼虾鳝鳖,在自由游动。

只是它们游不出边缘一股无形的力量,替代湖岸。包裹托住湖水。

在悬空的湖水地下,湖底彻底干涸,一丝水液都没剩下。露出河床,大量的水草软趴趴地堆在一起,还有石头,贝壳等等。

方源十指频动,无形的力量发生变化。巨大的湖水。宛若软泥,被方源任意搓扁揉捏,时而呈现长条面团状。时而如球体浑圆,时而是六面体绵绵工整。

变幻片刻后,方源忽的停下十指,脑海中念头频动。

湖水颤动不已。摆出各种奇形怪状。有时如鹿飞奔,有时如云翻腾……但不管如何变化,当中的鱼群等生灵,都始终被无形力量包裹着,掉不出来。

“好玩好玩!”一旁观看的地灵小狐仙,一双大眼睛炯炯发亮,雀跃欢笑。

太白云生也在旁观,看到此景。不禁喟然长叹:“移山倒海,真乃仙家气象。”

方源念头再动。庞大的湖水轻缓地落回湖中。方源抽回无形力量,停下挽澜仙蛊。这片湖水,还原本来面貌,只是水面上水波翻腾着,经久不息。

“这两只力道仙蛊,都是巧力。一个针对山峦,一个针对水波。善念一动,行云布雨,移山开路。恶念一动,山峦翻覆,巨浪滔天。”方源心中暗赞。

这便是蛊仙的战力。

面对凡人蛊师,直接碾压,宛若碾死蝼蚁一般。

“我得两蛊,战力立即暴涨。但要将其融入仙道杀招万我当中,却不是简单的事情。必定耗时耗力。不像是我力、力气仙蛊,拿到手中,几乎直接就可以和万我融合一体。因为万我的理念,本来就和后两者一致。”

构思杀招,并不简单。

雷道蛊仙凶雷恶人,闭关数年,才从血神子仙蛊方中借鉴,创造出雷神子。

黑楼兰已经有了凡道杀招我力虚影,有这样的基础,要将其改良成仙道杀招,也是步履艰难。一直在努力,至今都没有成果。

方源前世,创造血道杀招,哪一个不是殚精竭虑,耗费时日,艰苦努力,才有的成果?

如何将这两只力道仙蛊,融入万我当中,方源还没有丝毫的头绪。

但他并不担心,甚至满怀信心。

信心来源一部分,在于他的力道境界。更多的,是方源对九转智慧蛊有着无比信心。

沐浴在智慧之光中,便能灵感无限。这就是推演蛊方、仙蛊方、凡道杀招、仙道杀招等等的无上利器!

回顾此次拍卖大会,方源失去了浪迹天涯、平步青云、乐山乐意,而有了铁冠鹰力蛊、拔山、挽澜、吃力以及解谜。

蛊都是好蛊,关键看是否适合自己。

这进出之间,方源的战力暴涨了一大截。

“师弟,这次拍卖大会你收获极大。等到他完全消化了所获,你的战力还要再涨数倍!”太白云生徐徐飞上来,面含喜色,口中恭喜方源。

方源点头:“不错,我现在手中有了荒兽蝙蝠尸体,可以增添第三对蝠翼。吃力仙蛊、铁冠鹰力蛊都可以用于修行,增添自身力道道痕。道痕越多,我使用力道仙蛊就越加厉害。还有寸光阴、忆念蛊、恶念蛊等等都急需大量炼制。结合完整的见面不相识,残缺的见面似相识,构思出全新的伪装手段。种种事务,各个方面,都需要着手施行。”

接下来的日子,方源肯定要忙得昏天黑地,脱不开身。

但这是好现象。

拍卖大会是一次绝佳的机会,方源紧紧抓住,充分利用,使得各个方面都或多或少地打开了局面。

太白云生提醒道:“别忘了你刚和黎山仙子签下的盟约,首要事务,是炼制梦道凡蛊,组合成仙道杀招,救醒黑楼兰。”

“这点我当然明白,你自去东海吧。”

太白云生不禁笑道:“这些天,鲨魔那边已经催促了好几次。师弟,有什么事情,就通知我,我会立即赶回来的。”

方源皱了皱眉头:“要小心。我们在拍卖大会中暴露了许多东西,而且马赵二人也流入在外,这些都对我们不利!”

此时他还不知道,自己沙黄的仙僵身份,已经彻底暴露了。(未完待续……)

\end{this_body}

