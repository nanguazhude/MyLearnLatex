\newsection{梦翼和解梦}    %第二百六十五节:梦翼和解梦

\begin{this_body}

凤金煌一死,她的魂魄却立即飞上夜空,咯咯娇笑。

在她的魂魄背后,伸展着一对七彩绚烂的羽翼。

正是梦翼仙蛊催动的景象!

“我有梦翼仙蛊,能自由出入梦境。就算这一次失败了,大不了等到梦境下一次轮回,再重新开始好了。”凤金煌心中得意的冷哼。

她刚刚一番尝试,手中的风结草早就被她搞砸了,草根之间打结,比最初的时候还要复杂严重。

但当凤金煌刚要扇动翅膀飞离,夜空中就有一位鹰身兽人,从高处猛地扑下。

凤金煌发出惊恐的叫声,想要躲闪。

但往常迅疾的振翅动作,在这个梦境中却显得尤为凝滞。

凤金煌错估了星宿仙尊梦境的难缠程度,没有来得及脱离,魂魄就被鹰身兽人一爪抓爆!

仅仅这一击,就让凤金煌魂魄重创!

下一刻,凤金煌脱离梦境,魂魄归体。

噗!

她睁开双眼的同时,张口就喷出一口鲜血。

灵缘斋的同行蛊师们,大惊失色,慌忙赶过来支援。

“好厉害的梦境!刚刚的那一击,我若是承受第二下,绝对会直接死在梦里的。”凤金煌目光中尽是震惊和后怕,冷汗直流。

“大意了!我虽然有梦翼仙蛊,但面对仙尊的梦境,还远远谈不上安全。不过幸好,此行我准备了许多胆识蛊,带在身边。有了胆识蛊,我的魂魄能迅速复原。半天时间。就能彻底康复了。”凤金煌擦干额头上的冷汗,珍惜点滴时间。连忙开始疗伤休整。

梦境中,方源重新将目光投向手中的风结草。

刚刚凤金煌的表象。让方源收获了不少情报。

“这里的梦境规模一点都不大,这一幕也只是梦境的表层,但到底是仙尊之梦,即便是六转梦翼仙蛊,也无法做到立即脱离。现在看来,这一幕梦境的关键,还在于我手中的风结草。”

方源心中闪过一抹明悟。

观察完毕,他开始拆解手中的风结草。

先从表层开始,方源小心翼翼地剥离。

他感觉自己就像面对着。一个胡乱纠缠的复杂线团。他需要寻找线头,将其一根根抽出来。

破除到第三层时,方源便难以为继。

眼前的小洞,一团乱麻,根茎相互纠缠,无法动手拆除。

“不用蛮力破坏,拆解风结草的话,不仅要自信观察,更要脑力推算。还要有耐心,更重要的是充足的时间,以及一些运气。”

方源暗自叹了口气。

他手中的风结草,被他破开了一个小洞。但这个小洞口到了第三层时就被卡住了。

摆在他面前的一条路,就是退到前两层,发现其他的突破口。再继续进行拆解。

换做其他人,也只得如此。

不过有个例外。

方源扫视一圈。旋即就注意到,那位疑似星宿仙尊的女童。已经拆解到了第七层。

风结草在她的手中,体积已经削减了一大半。

“没有动用任何智道蛊虫,单凭自身的脑力,就做到如此程度吗?”方源暗吃一惊。

虽然他迟了一些时间,才开始对付手头上的风结草,但这双方之间的差距,也过大了。

方源眯起双眼,遮住眼中的精芒。

论真正的拆解能力,方源在所有的孩童当中,只不过处于中上层次。

论此时的进度,他更是落在中后段。

而时间已经过去了大半。

按照所有孩童的进度,真正有希望逃生的,恐怕只有那位疑似星宿仙尊的女童了。

不过方源却有着充足的自信,他相信自己不仅能够成功拆解,而且还能超越所有人,夺得第一。

带给他自信的,不是别的,正是仙级梦道杀招解梦!

解梦发动。

眼前纠结在一起的根茎,缓缓消融,不攻自破。

周围的兽人,很多都看在眼里,尤其是方源身后的一位蛇人,将整个过程都亲眼目睹。

但他们都没有说什么,更没有任何异动。

这就是杀招解梦的奇妙效果。

它不是单纯意义上的攻伐杀招,而是顺应梦境,而产生具体的变化。

比如方源在梦中进行战斗,解梦就是强大的攻击手段。梦境若是弱小,便可以直接将敌人分解。

又比如现在,梦境的规则要求方源解开风结草。那么此时催动解梦杀招的效果,就是帮助方源,拆解风结草。

这种拆解,是顺应规则的拆解,不是蛮力破解。

落到周围的孩童,以及时刻监视方源的兽人眼中,就是方源根本没有作弊,而是开动脑筋,徒手成功拆除了风结草的第三层。

所以,他们都没有动弹,就算是看中方源,想要一口吞食他的蛇人,也只是目光微微诧异,觉得方源能够攻破第三层,有些出乎意料。

解梦。

解梦。

解梦。

方源一鼓作气,没有停留。手中的风结草,很快就一路破解开来,达到最中心。

在风结草的中心,有着一把种子。

这是太古绿天中,植物的自保行为,也是借助大风自行繁衍的手段。

方源取出种子,站立起来,高举在手。

“他,他成功了!”

“这速度也太快了!!”

“求你,帮帮我,好吗?”

周围人群一阵躁动,就连那位疑似星宿仙尊的女童,都投来惊异的目光。

“你自由了,小东西,给你三天的时间逃命。快滚吧。”部落的兽人首领气哼哼的,一脚将方源踢飞。

方源幼小的身躯。在空中划过十几步的距离,这才砸落在地上。

让方源有些诧异的是。他挨了一脚,胸口虽然有些气闷,但身上却毫无伤势。

兽人首领虽然强横粗豪,但手脚上的功夫,却已经到达了刚柔并济的程度。

方源既然解开了风结草,那么兽人部落就真的放他走了。

方源望了那位女童一眼,后者已经重新将注意力集中在手中的风结草上。

方源稍微犹豫了一下,转身便走。

他远离篝火,走入黑暗的山林当中。

他走了仅仅十几步。黑暗中便绽放出一缕光明。

他越向前走,光明就越来越大,最终排开所有的黑暗,成就彻底的光明。

光明渐渐消散,显露出现实景象。

方源这才发现,自己不仅已经睁开双眼,而且肉身不知不觉间走出十几步,离开了外显梦境。

望着身后的梦境,方源紧皱眉头。心头非常疑惑:“怎么回事?这就莫名其妙地脱离梦境了?”

他细心感受了一番。

这一次探索梦境,并未失败。

自己的智道境界,陡然上升了一截。

“不应该啊!按照这团梦境的规模,至少要有三幕。我成功地渡过了第一幕。怎么没有顺势进入第二幕的梦境呢?”

方源站在原地,陷入沉吟。

外显梦境散发出的光辉,映照在他的脸上。

他离开梦境。是顺着原路返回。隔着梦境,在另一边。则是灵缘斋一行人,还有正紧急疗伤的凤金煌。

所以。方源仍旧没有暴露。

但如果搞不清脱离梦境的原因,难保下一次脱离梦境时,会走到灵缘斋那边的方向上去。

若是被外人发现方源探索梦境的事实,那就多少有些麻烦了。

良久,方源紧锁的眉头,渐渐舒展开来。

他再次走上前去,任由梦境将自己全身吞没。

眼前视野骤变,方源再次进入星宿仙尊的梦境。

暗夜、篝火、兽人、孩童、风结草,一样的情形,在重复发生。

凭借解梦,方源一路作弊,第一个解开风结草。

但是这一次,他没有直接离开,而是走到疑似星宿仙尊的女童旁边,想要帮助她解开了风结草。

兽人首领没有反对,而是道:“小崽子还要帮别人?哼哼。你帮可以,但是你要解不开,就要搭上自己的命!”

“看来我猜的没错。”方源大喜,毫无悬念地在时限之内,解开风结草,带着女童顺利逃生。

“怎么?还是脱离了?”方源诧异地回望身后。

第二次尝试的结果,仍旧失败。方源没有进入第二幕,和第一次一样,被梦境甩了出来。

梦境千奇百怪,规则不一。因为每一个梦境都是独特的,因此探索梦境的经验很难积累。即便方源有着前世经验,此中优势仍旧十分微弱。

“这一次,救下女童,智道境界的提升比上一次还要多些……”方源口中喃喃。

他继而陷入沉思。

照目前来看,有两个可能。

一个可能,是他救下的女童,并非是真正的星宿仙尊。

第二个可能,要救下更多的孩童,达到某个数量的标准之后,才能进入第二幕。

在第三次进入梦境之前,方源稍微休整了一下。

他取出胆识蛊,将有些虚弱的魂魄恢复如初。

虽然他成功脱离梦境,探索成功,但魂魄并非无损,仍旧在梦境中耗费了不少。

当然,方源的魂魄损耗,要比凤金煌受创的程度,好上许多倍。

前者是探索成功,后者是探索失败,两者不可同日而语。 ”第三次!“方源在心中为自己鼓劲,再次步入梦境。

梦中的一切在重演。

分发到风结草的那一刻,方源就连续催动解梦杀招,几个呼吸之后,他手中的风结草就被分解开来。

方源抓住草中心的种子,高举右手,大叫:“我成功了!”

一时间,不管是兽人,还是孩童,都鼓瞪双眼,吃惊地盯着方源。

原本嘈杂的场面,陷入一片诡异的寂静。

\end{this_body}

