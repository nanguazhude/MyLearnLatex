\newsection{新的盟约}    %第三百二十七节:新的盟约

\begin{this_body}

%1
和黑楼兰的谈判,前前后后持续了三个时辰。

%2
结束之后,饶是方源仙僵之躯,都感到疲惫不堪,可怜他的仙僵脑子本就不灵光,耗费了不知多少的念头。

%3
黑楼兰也好不到哪里去,她感觉:这场谈判比生死激战还要累人,从骨子里散发出一股虚弱之感,口干舌燥,脑袋发晕。

%4
谈妥之后,方源将太白云生也拉进来,和黑楼兰、黎山仙子签订了全新的盟约。

%5
方源交还我力仙蛊,换取黑楼兰的奴隶仙蛊,同时借用态度蛊一年时间。

%6
方源还将协助黑楼兰、黎山仙子,提供机会,斩杀黑城。

%7
星象福地这块,大家继续公平竞争。

%8
除此之外,方源还征得黑楼兰的同意,两人连运。

%9
而新的盟约,逻辑紧密全面,纵然黎山仙子是信道准大宗师,也要耗费数年才能解开。更关键的是,照顾到了焚天魔女。

%10
里面就有一项条款,可以添加更多的盟友进来,新进盟友必须遵守盟约。

%11
并且,双方不得借助外人之力,对付彼此。双方都有一次求援的机会,一旦动用,对方必须救援!事后的报酬,则视具体情况计算。

%12
盟约非常严格,对利益分配也做了详细规定。方源借助黑楼兰一方忌惮紫山真君的心理,还有前世的记忆,成功地达到了自己的目的。

%13
只会战斗的人。不过是莽夫而已。

%14
战斗并不能解决所有的问题。

%15
人是万物之灵,深具智慧。若拿这种智慧只用来战斗,是辜负了上天的恩赐。

%16
结盟之后,方源的处境大为改观。

%17
首先,黑楼兰就是黑月仙子。运道旺盛,经久不衰。方源和她成功连运,抵消了春秋蝉的弊端。

%18
其次,借得八转态度蛊。这只仙蛊可是当初盗天魔尊用的,是传奇仙道杀招见面曾相识的核心蛊虫。

%19
当初,盗天魔尊还只是六转蛊仙时,就利用这个杀招。先后哄骗两位八转蛊仙。获取大量重宝,引得两位八转相互误会,间接地害死他们。

%20
方源失去了变化仙蛊,但得到态度蛊,比变化仙蛊不知道好了多少倍!

%21
他改良的见面似相识杀招,效果将暴涨到惊人的程度。现在哄骗八转蛊仙,方源都深具信心!

%22
最后是焚天魔女。

%23
前世。焚天魔女这位八转蛊仙,设计对付方源。今生,方源和黎山仙子等人定下新盟约,算是提前堵住了焚天魔女的算计。

%24
焚天魔女对黑楼兰的愧疚和爱护,就是她最大的弱点。凭借这点,方源有心算无心,就能好好利用焚天魔女一番。

%25
就算焚天魔女不加入新盟约之中,当黑楼兰救援方源时,身处险境,焚天魔女能不出手吗?

%26
可以说。黑楼兰就是控制焚天魔女的最好棋子。

%27
“黑楼兰一方,已经不是我的阻碍。化敌为友,一进一出之间,我接下来的计划必定更加顺利。不过黑楼兰这些人,都十分精明,不好糊弄。她们现在对这份新盟约很满意,但渐渐的。就会发现我占了巨大的便宜。到那时,定然会不满,甚至对我不利。”

%28
方源也是占着上一世的记忆,才讨了便宜。他的心中,并没有放下对黑楼兰一方的警惕。

%29
“不过,等到将来我手中掌握了惊鸿乱斗台,大势已成,双方实力对比颠覆。就算黑楼兰一方再不满,也翻不出什么浪花来了。”

%30
方源不惜付出我力仙蛊,仍旧是为了仙蛊屋惊鸿乱斗台,还有大力真武仙僵。

%31
我力仙蛊在手,不过是增添了杀招万我的一式变招,充其量战力不过七转而已。

%32
但若有了仙蛊屋,方源连八转都不惧怕,从此游刃有余,进退有据。

%33
拿到态度仙蛊之后,方源将其放入自家仙窍,转身又去琅琊福地。

%34
琅琊地灵十分诧异,对方源道:“这么快,你就又来了?你搞到我想要的仙材了吗?”

%35
“没有。”方源摇摇头。

%36
琅琊地灵的脸色顿时阴沉如水,冷喝道:“方源,你这个臭小子,是把我老人家关照你的话,当做耳边风了吗?你上一次离开之前,我就警告过你,琅琊福地不是你想来就来的地方!你如果没有我想要的仙材,就不要来烦我。我是不会答应你任何要求的。”

%37
“是吗?”方源淡淡一笑,态度从容,“如果说,我想要动用那最后一次机会,要求你帮助我炼蛊呢?”

%38
“呃。”琅琊地灵一怔。

%39
方源抢夺了马鸿运的机缘,以盗天传承继承人的身份来到琅琊福地。

%40
因为长毛老祖和盗天魔尊之间的约定,琅琊地灵必须无偿地为方源炼制三次蛊虫。

%41
若是炼制凡蛊,琅琊地灵必须保证炼蛊成功。若是炼制仙蛊,琅琊地灵可以失败,但炼蛊的材料必须他自己掏腰包。

%42
方源先后用了两次机会。

%43
第一次是炼成了星门蛊,这只蛊虫虽然只是凡蛊,但给当时的方源相当巨大的帮助。

%44
第二次是炼成了第二空窍蛊,炼蛊期间,方源又饮下四种美酒,获得神游蛊。最终利用神游蛊,和琅琊地灵换取了和稀泥仙蛊。

%45
所以,方源手中还剩下最后一次炼蛊的机会。

%46
琅琊地灵正色道:“小子,你这次想要炼什么蛊?”

%47
方源哈哈大笑,拍拍琅琊地灵的肩膀:“我刚刚只是说‘如果’,没有真的想要耗费第三次机会,让你炼蛊。”

%48
琅琊地灵冷哼一声,一把打掉方源的手,还拍拍自己的肩膀。表情十分嫌弃。

%49
但终究没有再说什么话。

%50
方源只要不动用最后一次机会,他就可以随意出入琅琊福地。

%51
琅琊地灵知道:这是方源对自己的警告,甚至是敲打。但他无法否决这点,所以心中十分不爽。

%52
“如果你不想耗费最后一次机会,那就赶紧给老夫离开这里!”琅琊地灵不满地叫道。

%53
方源不以为意地耸耸肩:“我虽然没有搞到你上一次提到的仙材。但是我带来这个。你看看吧。”

%54
琅琊地灵扯起嘴角:“你小子还想糊弄我?哼,你以为仙材很好相互取代?我是不可能……”

%55
他话还未说完,就顿住了。

%56
他瞪大双眼,目光暴涨,死死地盯着方源手中的仙材。

%57
这仙材当然不简单,是方源从地沟蛊阵中拾取的八转仙材!

%58
琅琊地灵的眼睛,都看得直了。

%59
“这。这。这……好像是太古荒兽云凰的身体碎片!”琅琊地灵结结巴巴,难以置信地道。

%60
方源向他竖起大拇指:“不愧是琅琊地灵啊,的确如此。这块仙材我得到手后,也是费了一番功夫,才想出它的来历的。”

%61
说着,方源就将这块仙材,递给琅琊地灵。

%62
琅琊地灵毫不客气。闪电般出手,一把将仙材抓到手中,眼睛贴上去,细细查看。

%63
“好品质,好品质。”

%64
“果然是云凰碎片啊,这可是稀罕物!”

%65
“你小子走了什么狗屎运,居然搞到这么一块宝贝!”

%66
琅琊地灵口中呢喃不断,感慨万千,看得久了,他的目光都有些痴了。

%67
不管是哪位琅琊地灵。他们都是源自长毛老祖。

%68
长毛老祖最擅长炼蛊,也痴迷炼蛊。琅琊地灵也继承了这个性格。

%69
现在,琅琊地灵看到八转仙材,脑海中顿时思绪起伏。他苦苦追寻这等仙材而不得,若是有了这个,好多仙蛊方上的材料空缺,就弥补上来了。甚至有一两个仙蛊的炼蛊材料。就都收集全了,可以开炉炼蛊了!

%70
琅琊地灵已经顾不得一旁的方源,目光紧紧纠缠手中的云凰碎片,把这块仙材翻来覆去地查看,嘴角都差点流哈喇子了。

%71
八转仙材,就算是琅琊福地中,也收藏很少。

%72
为什么?

%73
很简单,不好采集!

%74
就像这块云凰碎片,它的本体可是太古荒兽云凰!这云凰战力极强,能够匹敌八转蛊仙。

%75
要收集这等仙材,通常而言,就要斩杀云凰。蛊仙本就稀少,八转蛊仙少之又少,只是那么一小撮人。又要考虑渡劫和自身安危,不会轻易开战。除非自己需求,流到市面上的八转仙材极少。

%76
这也是方源只拿了一两块仙材,放置到宝黄天中,就能很轻易地换取各种仙蛊食材的原因了。

%77
就算是琅琊地灵的本体,长毛老祖也没有太大的能力,亲自采集八转仙材。

%78
他之所以答应盗天魔尊、巨阳仙尊,为他们炼蛊,也是想要借助九转尊者的实力,帮助他采集八转仙材。

%79
方源看着琅琊地灵的表现,嘴角浮起淡笑,他又取出一块八转仙材,故意问道:“琅琊地灵,你帮我看看,这块仙材究竟是什么来历?”

%80
“啊!!”琅琊地灵抬头一看,顿时发出一声巨大的惊叫声。

%81
他的眼中,充满了狂喜之色:“我的天哪,这不是斑虎蜜蜂的蜂翅吗?看呐,多么美妙的道痕光晕。这只蜂翅上的金道道痕是如此的浓重,以至于我们用肉眼都能看到道痕之光了!!”

%82
“你这个臭小子,怎么能得到这等恩物!?简直气死人了,老夫我苦苦守候了多少年,手头上也只有二十几块啊。都不舍得用啊!!”

%83
方源哈哈大笑:“如果说,这都是我捡来的,你信不信?”

%84
说完,他就出手,琅琊地灵手中的云凰碎片,一下子抽了回来。

%85
琅琊地灵猝不及防,反应过来后,顿时哀嚎一声,简直是心头肉被人挖下的感觉。

%86
他跺脚叫道:“你干什么啊?!”

%87
“干什么?”方源一脸奇怪的样子,“这是我的东西。难道你还想强抢不成?”

%88
琅琊地灵浑身一僵,不再说话,对方源吹胡子瞪眼!

%89
ps:今天第三更,这样的生日真的让我很开心,很温暖。感谢大家的祝福,感谢吧务组带来的惊喜生日礼物,谢谢你们的十万币打赏!加更这章,希望大家喜欢。没有大家的支持,《蛊真人》这本书是走不到今天的。让我们相互陪伴,共同见证,越走越远吧!

\end{this_body}

