\newsection{拍卖大会(上)}    %第九十八节:拍卖大会(上)

\begin{this_body}

%1
青峰高耸入云,顶峰处,一片深潭静谧如镜,映照着天空缕缕白云。

%2
这里是真武福地,北原袁家的根基所在。

%3
袁姓,并不罕见,十分平常,在南疆、东海、西漠等等五域皆有。但在北原,袁姓却更为沉重,代表着的是超级势力,无上的权威。

%4
深潭上空,一位白袍老人舞动长枪,气息浩荡磅礴,掀起阵阵云岚,风动万卷松涛。

%5
老人白发,精神矍铄,身姿变化间,时而迅猛如电,时而静如磐石。长枪在其手中,游龙般伸展,灵韵十足。

%6
但不管他掀起多大的动静,这片青峰之巅的湖中,却是一直波澜不兴,平静如镜。

%7
湖水中,亦是生机勃勃,水草飘摇,无波自动。一大群的鲤鱼,透过水面,盯着白袍老者挥舞长枪,小口不断张合,竟然纷纷开口说话。

%8
“袁老头,舞得不错嘛,挺好看的。”

%9
“这一枪刺得不错,有点当年枪道人的影子。”

%10
“嗯,这一枪却是横扫过度,有点偏颇了。”

%11
听到鲤鱼这么一说,白袍老者倏地停下动作,落到潭边。

%12
他手中的长枪,陡然绽放白光,化作一蛊,没入他的长袍大袖之中。随后他向湖中鲤鱼施礼,虚心求教道:“请问刚刚的横扫,为何显得偏颇?”

%13
“你用力过猛了!”

%14
“太蛮横,失去了神韵。”

%15
“老头子年龄大。火气太旺,该吃点水草降降火。”

%16
鲤鱼们七嘴八舌,肆意评论着。

%17
这些鲤鱼。都非凡品,皮白如雪,鳞黑如墨。黑白交错分明,再无其他一些杂色。世人称之为真武鲤,乃是天地灵物,每一只都是异兽王。

%18
白袍老者沉吟一番,竖起手指。以手臂代长枪,试着演练起来,道:“这样可行吗?”

%19
“不行。”

%20
“错了。”

%21
“老头子。你好笨啊!”

%22
鲤鱼们张口闭口,痛批老者。

%23
这时,袁家蛊仙袁驰匆匆赶来,听到真武鲤的批评声。却已见怪不怪。他向白袍老者施了一礼。恭声道:“拜见太上大长老。”

%24
白袍老者正是袁家太上大长老,七转修为,巅峰战力。见到来人后,不悦地皱起眉头:“袁驰小辈,你怎么来了?老夫不是关照过吗,在我静心演武的时候,不要随意来打扰我。”

%25
“大人,您之前关照晚辈。要晚辈来提醒您,不要错过了拍卖大会。”袁驰执礼更恭。

%26
“哦?是有这么一回事。老夫差点忘了……这么说来,拍卖大会已经临近了?”袁家大长老拍拍额头道。

%27
袁驰苦笑一声:“大人,今天就是拍卖大会的举行之日。”

%28
“啊?时间过得这么快啊!小子,你提醒的好,待老夫回来之后,再给你奖赏。这边去也。”说着,袁家大长老一挥长袖,从镜湖中飞出一只仙蛊。

%29
此蛊名为人语,能将野兽之言,转为人话。

%30
人语仙蛊一去,真武鲤群的议论声,顿时戛然而止,只剩下嘴巴开合时,一波波的轻微响声。

%31
袁家大长老收起人语仙蛊,一飞冲天,化光而去。

%32
鲤群相互盘绕,似乎又议论了一番后,感到无趣,纷纷潜游下去。

%33
袁驰目视大长老的身影,彻底消失在视野中。

%34
“大长老还是老样子啊。”他感叹一声,朝着峰巅镜湖一挥长袖。

%35
顿时风起云卷,浓浓白云将这处袁家最机要的重地,重新掩盖。

%36
……

%37
“木兰桨子藕花乡,唱罢厅红晚气凉。”

%38
“烟外柳丝湖外水,山眉澹碧月眉黄。”

%39
书生摇头晃脑,手持书卷,坐卧于亭台之中,亭外荷花静放,晓月轻夜,微风拂面,暗香流动。

%40
一位侍女,侍奉在书生身旁。她眉目如画,皓腕如雪,正手执一小壶,往石桌上的酒杯中倒酒。

%41
书生读完诗词,一手拿着书卷,目光仍旧投在上面,另一手则信手一拿,正好拿取到酒杯。他看也不看,将酒水倒入口中,咂砸几下,嘟囔道:“酒水虽好,却不多已。正好趁着拍卖大会,向那人再讨要几坛来兮。红袖,此行你便随我去罢。让添香留在这里,看守福地。”

%42
“是,公子。”侍女微微一喜,婉声答应道。

%43
……

%44
铛……铛……铛……

%45
仙蛊天锣之音,响彻蛊仙贺狼子心中。

%46
他盘坐于地,睁开双眼,从七窍中缓缓流淌出血迹。

%47
“这仙蛊天锣,果然厉害。我只是催发了半成威能,就将我震得五脏移位,血脉倒流。即便我一直催动着顶级凡道防御杀招。”

%48
贺狼子狞笑一声,血液又缓缓回流,几个呼吸之后消失不见,仿佛他从未受过伤。

%49
“这样的威能,不愁仙蛊换不出去了。”他长身而起,向着绯红芦苇荡的方向,疾飞而去。

%50
……

%51
“这里便是绯红芦苇荡了?果然是一番好景色。”一位八转蛊仙高冠大氅,面容古朴,凌空俯瞰,扫视脚下的芦苇荡。

%52
但见芦苇丛生,水道如巷,错综复杂,宛若迷宫。

%53
这里的主角,不是那水鸭,亦并非鸥鹭,更非鱼类,而是绯红色的芦苇。

%54
万顷芦苇,以蓬勃的姿势跃出水面,似乎带着豪迈和侠气,浩浩荡荡,肆无忌惮铺展开来,以王者之姿,将生活在这里的生灵征服,占为己有。

%55
这位八转蛊仙观察片刻,有感而发地道:“关此景象,浩荡与锦绣并存,豪迈中又蕴藏沟壑。以景及人,可见这秦百胜粗中有细。豪放不羁,又有城府心算。难怪能搅动北原风云,做得这场拍卖大会。”

%56
“教主慧眼如炬。”他身旁的一位七转蛊仙。赞同地点点头,“秦百胜此举,翻手为云覆手为雨,是起死回生的妙招。我慕容尽孝亦佩服得紧。”

%57
“真是期待啊,这几乎囊括北原英豪的拍卖大会!”被称之为教主的八转蛊仙收回目光,语气略微兴奋起来。

%58
“哈哈哈,东海富饶。英雄辈出。教主能更是一方霸主,雄踞整个琉璃海,北原虽多人才。但能入你法眼的,恐怕也少。”慕容尽孝恭维着道。

%59
听的他话,这位神秘教主,竟然不是北原蛊仙。而是来自东海!

%60
……

%61
“秦兄。我们又见面了。”方源抱拳,向秦百胜打招呼。

%62
“哈哈哈,沙老弟,我可是期待你好久了。只是没有想到,你是直接从黎山仙子那边过来。”秦百胜十分热情,把住方源的双臂,仿佛真把方源当做自家兄弟看待了。

%63
方源和太白云生联袂同行,并未行走北原。而是进入黎山仙子的福地。通过洞地蛊,直接来到百胜福地。

%64
因为之前的合作。又签订山盟的关系,黎山仙子和秦百胜两人的福地中,已经架起一对洞地蛊,方便沟通。

%65
方源带着十三具眠云棺椁,就通过此途径,直接转进福地中来,再次见到秦百胜。

%66
“没办法。我身怀重宝,修为低微,只能通过此法前来,才能减少意外啊。”方源故意苦笑一声。

%67
“老弟你不要谦虚,你是真人不露相,深不可测。单凭你和黎山仙子这等关系,谁敢打老弟你的主意?”秦百胜语气恭维道。

%68
“人外有人,天外有天。不过哪天我若是有老哥你的战力,也就不怕了。”方源迅速回应,论交际之能,他可不输于秦百胜。

%69
“不知道这位是?”秦百胜转过目光,看向方源身边的太白云生。

%70
此时的太白云生,也动用了见面不相识,化作一位平凡老者,容貌身材以和之前大不一样。

%71
“这是我的好友蛊仙白胜。”方源大大方方地介绍道。

%72
“白胜……恕我见识简陋,居然还不知道咱们北原,还隐藏着你这样的人物啊。”秦百胜主动施礼道。

%73
“不敢当,不敢当。”太白云生连忙回礼。

%74
秦百胜将方源、太白云生两人,带到拍卖会场。

%75
“为了这次大会,我特意改造了福地,形成这片巨大的拍卖会场。”秦百胜介绍道。

%76
“果然非同一般,有一番雄奇气象。不愧是出自秦老哥之手。”方源笑眯眯的,好话张口即来。

%77
秦百胜露出一丝不安担忧之色,道:“没办法,这场拍卖大会几乎邀请了北原各大势力,无数强者。我负责此事,若拍卖会场太过寒酸,恐怕要得罪人的。老实说,这些天我压力很大,几乎睡不着觉。幸好有沙老弟你出手,蛊仙福地一定会引发哄抢,场面令人期待。不知道数量可多?”

%78
方源没有直接说出数字,只笑着说:“秦老哥,你放心就是了。”

%79
“哈哈,我当然放心。”秦百胜目光一闪,没有试探出来,索性转过话题,“我这拍卖会场中,有大厅,有单间,也有密室。大厅中可以随意交流,单间则给那些喜欢独处,脾气古怪的蛊仙们。密室则隐瞒蛊仙身份所用。这三种布置,都已经写入之前的盟约之中,沙老弟无须担忧我会在当中做什么手脚。不知道沙老弟,选什么位置?”

%80
“这三种位置,可以在拍卖的过程中随时调换么?”方源先问了一句。

%81
“当然可以。只是密室一旦有人占据,除非主人同意,否则不会进出任何外人。”秦百胜答道。

%82
“那我便选择一间密室吧。”方源道。

%83
“老朽也选一间密室。”太白云生随后道。

%84
秦百胜点点头:“前九号密室,都已经被人预定。就给你们十号,十一号密室如何?”

%85
方源、太白云生自无不可。两人分别进入密室不久,百胜福地开放门户,迎进众多蛊仙。

\end{this_body}

