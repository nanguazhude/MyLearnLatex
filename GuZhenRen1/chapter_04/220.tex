\newsection{羽民的自由}    %第二百二十一节:羽民的自由

\begin{this_body}

ps:看《蛊真人》背后的独家故事,听你们对小说的更多建议,关注公众号(微信添加朋友-添加公众号-输入qdread即可),悄悄告诉我吧!全场死一般的沉寂。

巨大的悲哀、惊恐和迷茫,笼罩住所有羽民的心。

周中已经是羽民当中,仅剩下来的蛊仙了。若是连他都不在了,成为了人族的奴隶,那么剩下来的这群羽民,又能如何生存呢?

在当今的五域,整个的天下,基本上都是人族的。

留给异人种族的生存空间,很小很小,并且会越来越小。

周中因为震惊而张大的嘴巴,渐渐闭合上。他仰望着方源,目光像是看一个怪物,问道:“你这仙僵的外形是假的吧?一头仙僵怎么可能这么会算计?”

方源微微一愣,没料到周中忽然平静下来。

方源顿时感到一丝不妙。

身旁的太白云生则怜悯地道:“放弃吧,周中。你虽然成为了奴隶,但我允诺你绝不亏待你,平时的时候也不会限制你的自由。只有在关键的时刻,才会让你出力。你身后的这些羽民,今后就在我的福地里生活吧。你放心,我绝不会苛刻他们,虐待他们。他们的每一份工作,我都会给予相应的报酬。唉……”

太白云生说完,深深叹息一声,老好人的性情有些发作了。

但周中却微微地摇了三下头。

方源嘴角的笑容消失,脸色转肃。

然后,他就看到蛊仙周中忽然转身,面向身后大群的羽民。鞠躬到底。

“大家。”周中的语调十分平静,却透露出一股决意。他的声音不大,但响彻众人耳畔。

他道:“对不住了大家,这个世界上只有蛊仙周中,只有羽民周中。绝不会有奴隶周中。真是惭愧啊,仅剩下的我也不能在守护大家了。诸位,再见!”

说完,他陡然张开他的翅膀。

羽民的翅膀,不如鹰翼之宽,不如雕翼之厚。芊芊细弱的样子。

“周中,你这是何必?快停手!”太白云生大惊失色,想要阻止。

但他怎么可能阻止得了一位蛊仙主动寻死?

周中猛地扑扇双翼,一飞冲天!

他冲上天空,口中大叫:“我周中!”

“是羽民!!”

“不做奴隶!!!”

这一刻。所有人的目光,都集中在他的身上。

他违反了协约。

他踩在地上,被方源算计成功,成为了方源一方的奴隶。

但周中不愿意,主动违抗。

他身上的信道仙级杀招,旋即爆发开来,强烈到难以违抗的反噬伤害,袭击周中的全身上下。

周中越飞越慢。双翼扑扇越飞越艰难。

他此时飞在空中,就像是一个老人,行将就木。腿脚蹒跚,却在攀爬险峻高峰一样。

他浑身上下,迅速结晶,很快整个人都转变成透明的玻璃水晶。

他双翼也都变成了水晶玻璃似的物质,再也扇不动了。

但是他的双眼,仍旧一直仰望着苍穹。看都不看方源和太白云生一眼,透着无尽的勇气和决绝。

然后他慢慢坠落。往地面坠落。

在坠落的过程中,他的整个身体开始崩散分解。

先是他的头颅。然后是胸膛,然后是双翼,随后是腹部,腿脚。

在坠落地面之前,他整个人化为一蓬碎裂的玻璃,琐屑的水晶碎片。

太白福地无风。

这些水晶玻璃的碎片,却仿佛是随风飘扬,越飘越碎,越飘越小,逐渐消散在空中。

“周中……”太白云生口中喃喃,双目失神。

“果然……这个家伙。”方源脸色铁青。

他没有让太白云生复活周中。一个连死都不惧怕,不愿当奴隶的奴隶,根本毫无价值。

而且周中是蛊仙,要复活他,仙元耗费不低。

就算复活过来,他身上的仙蛊也没有了,他的仙窍福地太白云生也不能吞并。反倒不如现在这样,让他直接死去。仙窍无法汲取天地之气,形成固定福地,只有泯灭。泯灭之后,仙窍上的周中全身的道痕,都将自行增添到太白福地中去。

周中的死,让整个羽民都陷入沉寂。

沉寂只持续了片刻,忽然新的羽民王羽飞高声呐喊:“我羽飞,也不想当奴隶。大家伙儿,你们还没看出来吗?这两个人族的蛊仙,是恶魔,早就计划着将我们一网打尽。他们是绝不会放过我们的。你们选新的羽民王吧。我先跟随周中老祖宗,先去一步了!”

说话,他当场引颈就戮!

“王!”羽民暴动,齐声怒吼,声震四野。

“不错,这世间没有充当奴隶的羽民,只有自由的羽民。”

“就算是死,我也是自由的。”

“自信之心跳动的一刻,就绝没有奴隶羽民。”

“拿我们的尸体,充作奴隶吧。”

羽民们或呐喊,或呼啸,或沉吟,或嘲讽。周中、羽飞的行径,感染了羽民,竟然在这一刻,他们纷纷选择自杀!

“糟糕,还不阻止他们?一群凡人,哼!太白云生你速速出手,动用人如故仙蛊,这笔财富不容有失。我要让他们求死也不能。”方源冷哼。

但太白云生却迟迟不见动静。

“太白云生,你干什么?”方源回首,顿时暗自一惊。

只见太白云生满脸挣扎犹豫之色,他艰难地对方源道:“方源,我的脑子里现在有两个声音。一个声音告诉我,应该理智,将这些羽民驯养为奴。另一个更大的声音却告诉我,放弃吧。这些羽民是真正的羽民不容折辱,甚至就连同情,都是对他们的侮辱!”

说到这里,太白云生的眼眶中,赫然浮现出了泪光!

“该死!”方源眼中的阴芒一闪即逝。隐晦地让人察觉不出。

若是将这些羽民驯养为奴,对于方源的西漠计划,也会大有帮助。但关键时刻,太白云生居然心软了。

方源心中不禁怒吼:“竖子不足与谋!!”

旋即,他伸出一只怪臂,陡然抓住太白云生的肩膀。

话到嘴中。却是另一番内容:“罢了,老白,你说的不错。这群羽民是《人祖传》中的真羽民,你不要犹豫了,就让他们为自由而就义吧。”

“方源……”太白云生脸上神情顿时松缓下来。感激地看向眼前的仙僵,又有些羞愧地道,“难为你设想出这个计策,最终却因为我……”

方源打断他:“你别说了。人都有自己的坚持吧。我虽然不赞同,但理解。对于太白云生你,我也会支持你的。”

“方源!”太白云生哽咽,几乎要落下泪来。

方源沉声道:“你知道为何这两位羽民蛊仙会中计吗?呵呵,你记不记得我从东方长凡处得到的智道传承?”

“你是说?”太白云生一愣。

方源感慨道:“智道的手段。真是防不胜防。幸亏世间的智道蛊仙一直都数量稀少。我现在已经有了一份完整的智道传承,这种手段不可不防。老白,你虽然渡过地灾。但还请你稍待片刻,不要急着去往东海。我要给你种下几种智道手段,用来应付其他的智道蛊仙。”

太白云生闻言,十分感动,向方源行礼:“那就有劳了!”

“哈哈,都是一家人。何必言谢呢。”方源摆手,不以为意。目光又投向地面。

短短功夫,地面上的羽民都自戮一空。竟然没有一人苟且偷生!

方源的脸上,闪过一丝动容。

羽民主动求死,让他想起了记忆深处的一个人。

准确的说,他也是一位羽民。

在方源还是凡人蛊师时,身为刺客的他,多番来刺杀方源。

比朋友更了解你的,往往是你的敌人。

方源记得,在一次艰难的战斗中,他发现了这位屡屡刺杀自己,纠缠不清的强大刺客的秘密。他并非纯正人族,而是是一个羽民!

“你是羽民?双翼都被斩断了?真是悲哀啊。”方源用言语打击道。

“呵呵呵,这对双翼是我自己斩断的。”羽民刺客邪魅地笑出声,“知道为什么吗?”

方源微微变色:“为什么?”

“啊,因为羽民村里的长老总是夸我,什么百年难得一出的羽民天才啊,什么将来羽民村的支柱啊,什么飞行准宗师啊什么的。真是烦死人了!长老总是告诫我,我是羽民,我属于村庄。我想要脱离村庄去看看世界,不仅是他劝阻我,就连全村的羽民都阻止我。哼,我知道他们是害怕我出去,泄露了这个村庄的位置,引来人族的奴隶捕猎队。所以呢,在有一天,我不胜其烦,将自己的双翼都斩断了。然后就在同一天,我把全村的羽民都杀了。”说到这里,这位羽民刺客的脸上,却是露出骄傲的,微微带笑的声音。

“什么?!”方源震惊。

羽民刺客无所谓地耸耸肩:“你也看过《人祖传》的吧,羽民啊,都是崇尚自由。我的自由之心呢,可能比平常的羽民要旺盛十几倍吧。羽民的身份束缚我,那我就斩掉双翼。从小长大的村子里,村民们约束我,那我就杀掉他们。因为这个世界上,没有人可以限制我的自由。”

方源脸色顿沉,心中涌起十二分的戒备,那个时候的他,还未入魔道,于是出声咒骂道:“你这个疯子!”

“哈哈哈。”羽民刺客大笑,“谢谢夸奖!”

末了又道:“其实你和我,是同一类人呢。”

“谁他妈的和你同类?!”方源年轻的脸庞浮现出愤怒之色,大吼着,狠狠地扑杀上去。

ps:抱歉了,这章更新迟到了二十分钟。其实早上六点起来就码了,本来已经写好,但忽然来了《人祖传》灵感,于是进行了大改。《人祖传》是我自己独自创作的异界神话,能来灵感非常的不容易,所以每一次这种情况,真人我都会非常珍惜。下一章就有《人祖传》,恳请大家多多的谅解和支持。(天上掉馅饼的好活动,炫酷手机等你拿!关注起點中文网公众号(微信添加朋友-添加公众号-输入qdread即可),马上参加!人人有奖,现在立刻关注qdread微信公众号!)

\end{this_body}

