\newsection{霸仙楚度}    %第一百零五节:霸仙楚度

\begin{this_body}

%1
寂静没有持续多久,拍卖场轰然作响。

%2
蛊仙们交相议论,大厅中许多蛊仙将目光直接投射在某个单间。

%3
这个单间,正是“幽兰仙子”所在之处。

%4
人们下意识地想看看,做出这个惊人决定的“幽兰剑师”此刻到底是什么表情。

%5
作为主持者的秦百胜也有些发怔,他询问道:“以两换一,幽兰仙子你是这个意思吗?”

%6
“正是如此。”凤九歌既然做下如此决定,此时没有任何的犹豫了。

%7
听到凤九歌再次的确定,大厅中的议论声倒是小了下去。

%8
凤九歌是音道蛊仙,为什么他的手中,为什么会有两只力道仙蛊?

%9
事实上,他手中的仙蛊,还远不止这两只。

%10
凤九歌率领十多位蛊仙,秘密来到北原,身怀重任。首要任务,是查明八十八角真阳楼倒塌,提前利用中洲十大古派布置的真凶。次要任务,是尽可能多的收集仙蛊。为了达成这个目的,来到北原的中洲蛊仙们准备了多种手段。

%11
其中之一,就是抱着换蛊的想法,从各自的门派中带了不少仙蛊过来。

%12
毕竟,蛊虫难以抢夺。蛊师一个念头,就能轻轻松松地令蛊虫瞬间自毁。仙蛊也是如此。

%13
这些带过来的仙蛊,大多都是冷僻流派。诸如力道、气道等等。

%14
北原的情报,一直掌握在十大古派的手中。秦百胜举办拍卖大会的事情,从未隐瞒。甚至刻意宣扬造势。凤九歌等人身在中洲时,就已然听说此事。

%15
可以说,这一次凤九歌潜入拍卖大会。并非临时起意,完全是早有预谋的行动。

%16
当今天下,力道不昌,早已衰弱。有条件,有野心的,打死也不会选择力道去修行。

%17
用这等蛊虫,去换取其他蛊虫。对于十大古派而言,也是一种胜利。

%18
东海蛊仙,虽然是八转大能。但真正比较起来,还是输给凤九歌的。

%19
毕竟凤九歌此行,是十大古派在背后授意,支持他。而东海蛊仙秘密潜入北原。参加此次拍卖大会。完全是个人行为。

%20
凤九歌有这种魄力,敢于二换一,但是他却没有。

%21
但是他又不甘心。

%22
于是做出最后努力,竞价道:“我再加……一颗檀香圣象的巨蛋!”

%23
大厅中,旋即又响起一片轻呼之声。

%24
“檀香圣象,乃是上古荒兽。抚养长大之后,可战七转蛊仙。”

%25
“关键就是从小培养,可以建立深厚的感情。背叛的几率小之又小。”

%26
“一头活的上古荒兽,价值远不止如此。处理得当的话。它就像一个取之不尽用之不竭的宝库。圣象的血液,可是上好的仙材。圣象的檀香象牙,更是可以取用后再度生长。”

%27
“这第一密室中的蛊仙,也算是另辟蹊径了。这应对真的不错,一只仙蛊和一头上古荒兽,两者比较,各有优劣,很难取舍。”

%28
“仙蛊唯一是不错,但上古荒兽也很有用处。尤其卖方似乎对炼道很有想法的样子,说不定就会动心,选择上古荒兽。”

%29
众人议论纷纷。

%30
到了这一步,凤九歌、东海蛊仙也都停止了竞价。

%31
现在的局面,不好说。

%32
凤九歌以二换一,气魄够大,赢面也大一些。但东海蛊仙应对得当,也有获胜的可能。

%33
拍卖场渐渐安静下来,所有人都在等待着方源做出决定。

%34
这场意料之外的拍卖**,究竟如何落幕,浪迹天涯仙蛊又花落谁家?

%35
众人心中都被激发了期待之情。

%36
方源考虑了一下,便做出决定,暗中告知了秦百胜。

%37
秦百胜点点头,当众宣布道:“卖家已做决定,将浪迹天涯仙蛊卖给幽兰仙子!”

%38
此话一出,大家尽皆松了一口气,纷纷评论。

%39
“幽兰仙子能够获胜,也是情理之中。”

%40
“到底是二换一啊。”

%41
“若是我选,我就选上古荒兽。毕竟两只仙蛊,仙元用度也大。”

%42
“这只浪迹天涯仙蛊,到底是哪个幸运儿的?居然卖出了如此高价,一只仙蛊换来了两只仙蛊!”

%43
“是啊,若非亲眼所见,打死我都不会相信此事。”

%44
“幽兰仙子,或者说背后的凤仙太子,一定别有所图。他们不可能是笨蛋,既然能出这么高的价,那么一定能带给他们更大的好处!”

%45
“只要有需求,就会有市场。能产生利润。需求越大,利润也就越大。”

%46
不止他们,方源也在揣测这位“幽兰剑师”,高价买下浪迹天涯仙蛊的用意。

%47
“凤仙太子……”东海蛊仙口中咀嚼着这个名字,脸上不免带着微微阴沉之色。

%48
拍卖大会继续下去,接下来轮到一只土道仙蛊。

%49
但气氛却是一落千丈,大部分蛊仙的心思明显还停留在方才,很多人仍旧在小声谈论着。

%50
究竟什么是财大气粗,众人终于是见到了,开了眼界。

%51
尽管也有很多人疑惑,凤仙太子乃是变化道蛊修,怎么忽然将对水道仙蛊产生了兴趣?

%52
但八转大能的心思,深不可测,远不是外人胡乱猜测的。

%53
原本一只主流流派的仙蛊,几轮竞价之后,迅速成交,让人颇有些灰溜溜的感觉。

%54
对此秦百胜也颇为无奈,只能暗暗苦笑。

%55
**之后,就是低迷。毕竟前番凤九歌、东海蛊仙两人竞价,气势逼人,众人的心气劲头都被削弱了很多。

%56
方源坐在密室中,一边把玩着两只力道仙蛊,一边耐心地关注着拍卖会继续下去。

%57
拍卖仙蛊的交易是当场进行的。

%58
那些仙材。方源都清点了,数量质量都无误。

%59
两只力道仙蛊,在凤九歌的主动配合下。已经被方源完全炼化,成了自身之物。

%60
同样的,浪迹天涯仙蛊也落在凤九歌的手中。

%61
“这就是浪迹天涯仙蛊……多少年了,终于再次回到了灵缘斋的手里。”凤九歌轻声一叹,也是颇为感怀。

%62
一只只仙蛊拍卖出去,经过一段时间的积蓄,拍卖场的气氛再度热烈起来。

%63
“下面是第三十四件拍品。力道仙蛊鼎力,这是一只罕见的力道防御仙蛊。催动起来,能形成力鼎。防护自身。卖家要求一只智道仙蛊。”秦百胜朗声道。

%64
大厅中的蛊仙们,却是兴趣缺缺。

%65
力道已经式微,很少有人修行了,因此市场很小。

%66
就在鼎力仙蛊的主人。暗中焦急的时候。一个沙哑的独特声音从单间中传出:“智道仙蛊儿女情长。”

%67
“这个声音,是霸仙楚度!”旋即,大厅中有人低呼出口。

%68
“不错,应该是他,他的声音就是如此沙哑。”

%69
“想不到霸仙也来了。”

%70
“这有什么奇怪的,楚度大人乃是有名的力道强者,秦百胜怎么会不邀请他呢?”

%71
“霸仙楚度……”单间中,凤九歌喝着清茶。双眼微亮。

%72
他来北原之前,早就熟知情报。霸仙楚度的情报。洋洋洒洒,内容繁多,凤九歌相当重视。

%73
楚度乃是当今罕见的力道蛊仙,不仅战力强盛,而且才情横溢,创造出崭新的力道小流派。

%74
力道是个很大的流派,曾经辉煌一时。如今在力道中,又细分为兽力虚影流、气象天地流、人力钧力流。

%75
兽力虚影流,就是牛力、虎力等等,方源不久前得到的力道仙蛊铁冠鹰力蛊,就属于此流派。

%76
气象天地流,是风力、水力、火力等等天地自然力。比如方源拍卖浪迹天涯仙蛊,得到的拔山、挽澜,就是此流派。

%77
人力钧力流,则是斤力蛊、十斤之力蛊、一钧之力蛊、十钧之力蛊、百钧之力蛊这些。

%78
这个小流派,就是霸仙楚度一个人独自开创,引领潮流,令老朽迟暮的力道重新焕发青春的光辉。

%79
现在,人力钧力流已经成为五域力道的主流,整个北原蛊师,大多数都有力道的底子,他们几乎都采用斤力蛊、十斤之力蛊、一钧之力蛊、十钧之力蛊等等。

%80
可以说,霸仙楚度一人,改变了力道的现状,影响了天下蛊修流派的格局。

%81
这样的人物,凤九歌当然要重视,而且是十分重视。

%82
“北原战乱频繁,也是英雄辈出之地。若是有机会,一定要好好见见这位霸仙。”凤九歌的嘴角处,浮现出一抹兴奋的笑意。

%83
会尽天下英雄,正是大好男儿当有的豪情!

%84
这时,一个声音传出:“乐山乐水仙蛊一只,再添云泥万斤。”

%85
噗。

%86
凤九歌将口中的茶水噗出,双眼瞪圆,再无方才的淡定之色。

%87
“乐山乐意仙蛊!”他轻呼一声,“该死,担心的事情还是发生了。”

%88
仙蛊可以用来拍卖,自然也可以当做筹码,交换其他仙蛊。若非如此,记录在案的仙蛊被卡着,不能流通,将大大不利于拍卖的进行。

%89
这个规矩,凤九歌当然了解,也最担心。

%90
担心什么,就来什么。

%91
方源对力道仙蛊有需求,也早就关注这只力道仙蛊了。此时抛出乐山乐水仙蛊,也是他原本的计划之一。

%92
“天柱风一道。”面对方源的竞争,霸仙楚度也提价道。

%93
天柱风,十分奇特,来源于绿天。形如巨柱,极高极长。天柱风刮起来,从不移动,就在原地刮吹。从外面看过去,就像是撑天巨柱一样。

%94
一道天柱风,价值很高,绝不输给凤九歌之前报价的一百五十片雷音叶。

%95
“霸仙楚度……”方源口中喃喃,有所觉悟,“看来你便是我此次最大的竞争对手了!”

%96
ps:我想描绘的是一个世界,拍卖大会是众星璀璨的。不止是主角,每个角色,都有各自闪亮的地方。真心希望大家能够喜欢!

\end{this_body}

