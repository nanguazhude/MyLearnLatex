\newsection{顽抗不抵搜魂多}    %第一百五十九节:顽抗不抵搜魂多

\begin{this_body}

“哦?你就这么有自信……”方源淡笑。

东方长凡的魂魄,已经成为自己手中的俘虏,求生不得求死不能,居然还有这底气。

方源却不以为意,施展搜魂手段。

一时间,荡魂行宫中一片昏暗,阴风四起,呜呜呜的鬼哭之音,不绝于耳。

惨淡诡异的旋风中,方源展露原形,高大如塔的庞大仙僵之躯,全身肌肤都呈现青黑之色。面目狰狞,獠牙外翻,双眼血红,八只手臂齐齐舒展,怪爪尖锐庞大,其中一只缓缓探向东方长凡的魂魄。

东方长凡魂魄目睹方源如此凶恶的形象,不免波动了一下,旋即又镇定下来,任由方源的怪爪抓住自己。

魂魄本身并非实体,但方源这一抓,牢牢将东方长凡的魂魄制住,使其动弹不得。

随后,他催动仙窍当中的蛊虫。

搜魂!

一股无形的力量,顿时从怪爪中透出,直接窜入东方长凡的魂魄当中。

东方长凡魂魄立即产生不断的颤抖,痛楚一阵阵地产生、袭来。

片刻之后,搜魂结束,方源收回怪爪,放开东方长凡的魂魄。

方源搜刮了许多信息,却是眉头微皱。

原来,他此番搜魂,得到的内容虽多,但却都是那些无关紧要之物。

东方长凡的魂魄黯淡松散,再无之前的光泽和凝实。不过却流露出得意的神情。再次波动,传达信息。

方源“听”在耳中,却是东方长凡魂魄嘿嘿笑道:“你现在明白了吧?我虽是智道蛊仙,但魂道、智道本来就相互亲近,我亦有涉猎魂道。改造了仙魂。如今,你这搜魂之法,不过是凡道手段。就算是有魂道仙蛊对付我,我也可以故意遮掩,让你得不到你想要的东西。”

顿了一顿,东方长凡又道:“现在,我没有了肉体。魂魄得不到滋养。你搜魂次数越多。我的魂魄就损耗越大。我经历丰富,故意遮掩,在我魂魄损耗到极限,彻底消散之前,你根本就难以搜刮到我的智道传承,或者我的夺舍法门。”

紧接着,他又话锋一转:“不过我们可以做一笔交易。我将我的智道传承、夺舍法门等等。都交给你。你帮助我重生,给我寻找一个肉身。放心,只要一个凡人蛊师即可。我既然败在你的手中,也是认栽了。咱们完全可以定下契约,黎山仙子和你一同作战,你不相信我,总该相信山盟蛊吧?”

方源哈哈一笑:“不愧是东方长凡啊,到了如此境地,仍旧没有放弃,还想着翻盘呢。可惜。我有壮魂的手段,完全可以慢慢搜你的魂。总有一天,我会得到我想要的东西。”

“壮魂手段?呵呵呵。”东方长凡魂魄笑道,“看来你是不太了解魂道,我为了夺舍,特意改造了魂魄。虽然增大了夺舍的成功可能,但也导致寻常的手法难以滋养治疗魂魄。除非是动用仙道治疗手段。我说的可不是假话,你尽管对我的魂魄试一试。”

“哦?是这样……”方源收敛起脸上的淡笑,取出气囊蛊。

在东方长凡的估算中,既然方源搜魂的手段,都只是凡级,想来治疗魂魄的手段也会普通。

等到他看到方源取出一只凡蛊时,更加笃定原先的判断。

但很快,方源捏碎气囊蛊,也一同捏碎了其中存储的胆识蛊后,东方长凡的魂魄立即受到滋养,魂魄重复凝实和光泽。

“这,这是什么蛊虫?只是凡蛊,却居然能治好我的魂魄伤势?”一时间,东方长凡魂魄剧烈波动,流露出掩盖不住的惊异之色。

方源见此,放下心来,挪移地笑道:“你猜。”

东方长凡到底是不平凡的人物,就算只剩下魂魄,也难掩他的智慧。

魂魄沉静,很快又再次波动:“凡道之物,唯有一种可以影响我的仙魂,那就是荡魂山胆识蛊。你刚刚捏碎的,应该是气囊蛊,从宝黄天中买了的?”

“聪明。不过你只猜对了一半,气囊蛊正是我的独创,胆识蛊更是我的买卖。”

“什么?就是你!”东方长凡魂魄也不由的失声,残酷的现实再次狠狠地打击了他。

魂魄不再波动,陷入死一般的沉默。

东方长凡心中充满了绝望,他知道这次真的完了,落到谁的手中不好,偏偏落到方源这等人物之手。

人为刀俎我为鱼肉,方源根本就是想杀就杀,想宰就宰。而东方长凡的魂魄就算在反抗,也难抵大局!

用胆识蛊将东方长凡的魂魄修复好后,方源继续搜魂。

虽然东方长凡极力抵挡反抗,方源得到的信息,毫无价值的占据绝大多数。

但是反抗之心再强烈,也招架不住次数繁多的搜魂呐。

有价值的东西,终究是一点一滴的被方源搜刮出来。

花费了大半天的时间,方源搜魂告一段落,将有价值的内容拼凑起来,便组成一份有关碧潭福地各处资源的详细情报。

“原来,碧潭福地中有三大蛊虫豢养之地。长恨蛛不过位列第二,规模最大的是大洞蟾!规模数量是长恨蛛的三倍,五转大洞蟾蛊非常多,东方部族却故意不取走,是想催生孕育出仙级的大洞蟾蛊吗?野心真的挺大。”

“碧潭福地的最东面,有一处深潭,貌不惊人,却通达地下深处。里面生活着一只荒兽马,竟是稀罕少见的万里芝马。这种马生机旺盛,用处极大。不仅浑身血肉皮毛都是炼蛊仙材,而且放养在地底,长期生活的地方。汗滴洒下后,都会渐渐长出芝林。一匹万里芝马,就意味着有无数不绝的芝林。”

“哦?竟然还有这样一处盐场。好家伙,用上古荒兽苍生水母镇守,凝析出盐分。形成一处巨大盐场,可以得到仙材苍玄盐。专门产出苍玄盐的盐场,真是大手笔!”

“碧潭福地也有天象变化。连续三天白天后,就是一天黑夜。黑夜降临后,碧潭福地里,某些地表就会裂开缝隙,露出地下的两大深潭。一个是玉潭。虽然不大。却充斥着仙材荒玉!另一个是牙潭,潭壁潭底都是用一只太古荒兽麒麟的牙打磨成的。月光照下去,就会在水中生出月牙结。这种仙材,仿佛牙齿,又柔软如水,表面月光闪耀不止,极为珍贵。”

“除此之外。还有上古荒兽飞霜羚,埋藏地下的仙材潜龙须,位于铜潭中的仙材风啸铜……”

东方长凡乃是东方一族的太上大长老,对于碧潭福地中的资源分布,自然十分清楚。

方源浏览这些情报,大为心动,几乎忍不住要前往碧潭福地好生搜刮一番。

但他又按捺住这股冲动。

北原当晚,他杀死了东方长凡,捉拿了他的魂魄,便转身去了碧潭福地。想要再次搜刮。结果却发现那里战火绵绵,到处都是激战。

正魔两道蛊仙,有数十人,正在里面疯狂抢掠,不断厮杀,场面十分火爆。

方源仙元消耗很多,又要隐形匿迹。再加上孤身一人,黎山仙子、黑楼兰都去追方寸山去了,还引起了残阳老君的怀疑,便明智地选择了撤退。

方源眼中精芒一闪即逝:“现在的碧潭福地,仿佛就是一块香喷喷的大肥肉,引得蛊仙群情沸腾,不断汇聚过去。还是算了,我虽有七转战力,移动、防御两块却是相形见绌,处境又很不好……嘿,这东方长凡还是在算计我呢。故意先将这些情报暴露给我,好让我去碧潭福地蹚浑水!”

方源此时处境糟糕,乍看安全,其实危机四伏。

他是八十八角真阳楼倒塌的元凶,罪魁祸首,目前正在遭受各方势力的积极调查。

沙黄是他的假身份之一,但拍卖大会之后,就旋即暴露。夜叉龙帅在凤九歌这帮人手上吃亏,也开始调查沙黄,导致方源现在根本不敢去僵盟。

方源此番谋求智道传承,也暴露了很多东西,留给敌人许多可以推算的证据。

“不过这次冒险,是必须的!我若坐以待毙,早晚暴露。只有奋起反抗,争分夺秒地壮大自己,才是应对的正途。此次出马,我竟得到了一份完整的智道传承。自从连运之后,我的运气一直不错,且抓住了机会。等我掌握这份智道传承之后,我应该就能对付智道推算。如此一来,我便能布局设法,大大的拖延暴露的时间,继续隐藏下去。”

方源忍住碧潭福地的诱惑,休息片刻。

搜魂对目标会造成损伤,需要治疗。对自身也会令魂魄浮躁不堪,这却不是胆识蛊可以治疗,而是需要休息睡眠,安定魂魄。

休息之后,方源再次搜魂。

东方长凡的魂魄见到方源仍旧在此,没有被引诱前往碧潭福地,越加感到冰凉寒冷。

方源这次搜魂,持续了很久。

东方长凡魂魄反抗更为激烈,方源搜魂只是凡道手段,难度剧增。

但方源耐心十足,一次不行,那就十次,百次。

长时间的搜魂之后,方源得到了东方长凡改造太古墟蝠尸体的记忆,不仅知道控制虚兽作战的奴道法门,而且还得到了如何将蛊阵虚化的秘密。

不管是哪一个,都是价值巨大!(想知道《蛊真人》更多精彩动态吗?现在就开启微信,点击右上方“+”号,选择添加朋友中添加公众号,搜索“zhongenang”,关注公众号,再也不会错过每次更新!qdbook)

\end{this_body}

