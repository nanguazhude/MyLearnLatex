\newsection{仙材的尸海}    %第二百八十六节:仙材的尸海

\begin{this_body}

%1
这样的巨型蛊阵,要破解开来,至少数月的时间。

%2
若是只要撕开一个口子,也需得一个月左右。

%3
方源一面手指如花般绽放,无数蛊虫四下飞舞,破解着眼前的蛊阵,一面双耳频频颤抖,侦查手段催至巅峰,夜叉章鱼王已经快要回来。

%4
时间不等人!

%5
争分夺秒!

%6
“就是这一击。”方源眼中突然神光一闪,眼前的空气中忽然散射出氤氲的光雾。

%7
透过光雾,他看到巨型蛊阵中的一角。

%8
虽然视线模糊,但依稀可以望见,巨型蛊阵中的景物景象。

%9
夜叉章鱼王已经归来。

%10
时间耗尽,必须要撤了。

%11
在这么短的时间内,窥视到蛊阵中的景色,能做到这样的程度,已经达到了方源的能力极限。

%12
如果方源没有达到宗师级的智道境界,还做不到这点。

%13
这就足够了。

%14
方源舒了一口气,立即返身撤退。

%15
上古荒兽夜叉章鱼王在地洞中,穿梭得好似地铁。周围滑腻的地沟黑油,让它如鱼得水。

%16
方源几乎和它擦肩而过。

%17
这头夜叉章鱼王回到自己的住处,优先扫视了一番。

%18
它感到有些不太对劲。

%19
方源之前拆解蛊阵,留下了许多异类气息,野兽的敏感让夜叉章鱼王察觉到了不对劲的地方。

%20
它深深地皱起眉头。十多根巨大,且柔韧的章鱼触角,灵活地带动他的身躯。在巢穴中不断逡巡。

%21
片刻之后,没有发觉任何可疑,方源之前留下的气息也渐渐冲淡,夜叉章鱼王这才慢慢地平息下来,重新趴回窝里。

%22
只是这一次,它不再沉眠,双眼微微闭上。但始终留着一条眼缝。

%23
时不时的,它的双耳都会随着地洞中的微风。颤抖一下,倾听着任何的声音。

%24
见此,方源的担忧才缓缓消散,小心翼翼地后退。直至撤回到章鱼地巢的上半部分。

%25
这头夜叉章鱼王,能留守在兽穴里,似乎地位最高。若是让它察觉到强烈的危险,说不定它就会高声嘶鸣,将在外争战的其他族群成员,都召唤回来。

%26
到那时,不提北原仙僵们会不会也生出疑心,被勾引过来,单说方源在巢穴中的行动。就会受到极大的限制。

%27
“就这处吧。”片刻后,方源在数条地洞中,精心挑选一个不起眼的小角落。

%28
他犹豫了一下。仍旧催起定仙游仙蛊。

%29
虽然那道巨型蛊阵,是前人布置,但刚刚方源破解之时,已经查明它已经成为无人蛊阵,年久失修。冒然闯入的危险性,应该不大。

%30
为了寻求摆脱仙僵的法门。这点风险是要冒的。

%31
毕竟时间拖得越久,方源处境越是危险。

%32
说不定下一刻。他弄塌八十八角真阳楼的事情就会曝光,到那时全天下都来追杀他。狐仙福地肯定会保不住,只能把星象福地当做唯一的家园。

%33
这种情形下,越早摆脱仙僵之躯,就越有利。毕竟一个生机饱满的仙窍世界,能够随身携带,比什么狐仙福地、星象福地都要安全,更方便。

%34
而且正常的蛊仙仙窍,每隔一段时间都能自产仙元,对方源的经济还有战力,都是十分有益的影响!

%35
仙蛊催动,翠芒绽放。

%36
下一刻,方源消失在原地,出现在巨型蛊阵当中。

%37
蛊阵之外,趴着夜叉章鱼王。但有蛊阵相隔,它一片懵懂无知。

%38
“终于进入这里了。”方源十二分戒备,迅速环视左右,这里果然是他刚刚探查到的那个角落。

%39
冒然闯进一个蛊阵,是很危险的事情,遭遇蛊阵的攻击也很正常。

%40
但方源却没有遭受攻击,这更说明蛊阵无人主持。

%41
他松了一口气,但仍旧警惕无比。

%42
无人主持的蛊阵,仍旧在运转。也许蛊阵内核之中,有着蛊仙留下来的意志。

%43
就像在八十八角真阳楼中,巨阳仙尊留下来一股意志。

%44
这是宇道为主的巨型蛊阵,专门营造出一片空间。

%45
这处广大宽阔的空间中,有紫沙铺成的大地,有蛋黄色的天空。

%46
紫色沙地似乎十分肥沃,密密麻麻的植物,充斥方源的视野,仿佛来到了热带雨林。

%47
“太史草、丰源桃、仙豆、血潮花……”方源匆匆扫视一眼,就看到数十种仙材。

%48
即便是稳重如他,也不禁心头震动。

%49
这些仙材植株,虽然大多都是六转,少部分是七转仙材,但密集程度着实惊人。

%50
“这些仙材,来源五域,并不局限于地沟乃至北原。创建这里的仙僵大能,究竟是什么来头?”

%51
要采集眼前如此庞大规模的仙材,得耗费多少人力物力?

%52
并且仙材可不是街边摊子上的大白菜,仙材的采集往往和争斗、血腥分开不了。收集者实力不行,是根本做不成的。

%53
别忘了五域界壁。

%54
实力强大,修为越高的蛊仙,穿透界壁就越不容易。

%55
“难道说……”方源心中一动,有所猜测。

%56
他原本以为,这是一位仙僵大能的遗藏,但现在看来,如此巨大的手笔,恐怕不是一个人能达到的,至少得是一个组织。

%57
这个组织还必须庞大。

%58
因为小型组织,两三位蛊仙,根本就没有这么多的资金和精力!

%59
再举目眺望,视线所及之处,无数的仙材层层叠叠,看得方源怦然心动。

%60
“如此巨大的财富,简直难以想象!足够我今后炼制仙蛊数十次。上百次!或许五百年前世,北原僵盟分部就是收获了这笔巨大的修行资源,才一跃而起。将东海僵盟总部都压下一头去。”

%61
这里的仙材几乎像是仓库,直接堆叠在一起。

%62
总价值都一时难以估算,方源只知道这个数字很庞大,很庞大。

%63
若是依靠自己目前月入两千仙元石的速度,去积累的话,花费上千年时间都未必有此成果。

%64
毕竟还要算上修行的损耗。

%65
蛊仙修行越高,灾劫就越是恐怖。

%66
“不过相比较无数仙材而言。这座巨大的蛊阵,价值更高!”方源眼中精芒闪烁不停。

%67
他的野心很大。见识很广,反应过来后,就意识到这座巨型蛊阵的珍贵!

%68
在这个蛊阵中,来自五域的无数仙材。相互和谐共存,互不干扰。

%69
“幸亏我没有着急硬拆了这座巨型蛊阵,否则后悔莫及。有了这个蛊阵,就能栽种无数种仙材,比仙窍还更加好用呢!”

%70
方源对布置这里的仙僵大能,不禁暗生敬佩之情。

%71
仙僵仙窍已死,这位前人居然能布置出如此蛊阵,一定程度上取代了仙窍的作用。

%72
“不过,这座蛊阵防御薄弱。也不能移动,远没有仙窍方便安全。嗯……这里的紫色沙土,好像也别有蹊跷。”

%73
方源俯下身来。伸手一抓,立即抓起大把的紫色沙土。

%74
这些沙土,十分细碎。一些顺着方源的指缝,挥洒下来,宛若水晶做成的颗粒,在坠落的过程中。纷纷扬扬,散发莹光。

%75
“但是我要找的东西。可以重获新生的法门,又在哪里?”方源将手中的紫沙完全抛掉,时刻惦念着主要目的。

%76
他不急着采集仙材,开始在半空中徐徐飞行,展开搜索。

%77
时间一分一秒渐渐流逝,方源脸上的喜色却慢慢消散,不妥当的感觉越发强烈。

%78
“不对劲!”他蓦地停下身形,悬浮在半空中,俯瞰着脚下茫茫多的仙材。

%79
再一次审视,他发现不自然的地方。

%80
“这里太安静了。一丝风,一点声音都没有,比我的仙僵死窍还要沉寂。”

%81
“而且很多地方,大有蹊跷。比如仙豆,需要生长在至清泉水之中,但在这里居然生长在紫沙里。还有丰源桃生长的地方,必须有蛇尾猴群伴生。每一种仙材,都有苛刻的生存环境。而这里只有仙材植株,连一个虫子都没有发现。”

%82
方源眼中闪烁一阵寒芒。

%83
他降下高度,再次落到地上,试着采集一株太史草。

%84
下一刻,方源身躯一震。

%85
太史草长在紫沙中时,安然无恙。但一被方源拔出来,就立即枯萎,转眼间化为一蓬细碎的沙土。

%86
而这些沙土,和方源脚下的紫沙,别无二致,一模一样!

%87
“怎么会这样?”方源眯了眯双眼,又试着采集其他仙材。

%88
结果无一例外,一旦仙材脱离沙土,就会化为紫沙。

%89
这些仙材都大有问题,已不可用!

%90
“古怪!古怪!!仙材之所以称之为仙材,是因为它们的身上,蕴藏着丰富的道痕。事关道痕陨灭,这些仙材枯萎,化为紫沙时,居然一点动静都没有,悄无声息……”

%91
一股寒意,从方源心中升腾起来。

%92
仙材如此,那么身怀道痕的仙僵之躯,也可以看做是仙材的一种,是不是也会被这片广袤的紫沙大地同化呢?

%93
再看这些无垠的仙材,茫茫的花草树木,方源就仿佛看到了一个个的死尸残骸!

%94
换个角度,这里就是仙材的尸海!

%95
心中不妙之感,越发强烈。

%96
方源立即升到半空,和紫沙保持距离。

%97
他又望了望这里的天空。

%98
“营造这里的巨型蛊阵一定大有古怪,我不接触紫沙,恐怕还不够。仙材如此变化,应当和整个蛊阵有关系。再没有弄懂这里之前,我还是尽快行动,争取找到摆脱仙僵的法门,然后迅速离开!”

%99
ps:刚刚得知贼道三痴病逝的消息,重新看了一遍他最后的章节,不禁潸然泪下。好希望这是个假消息。唉……人生在世,命运无常,还是多多珍惜彼此,多多珍惜眼前的幸福吧。愿贼道三痴在天堂中快乐无忧。还有,读者朋友们,我爱你们,愿我们一起走下去,走到生命的尽头……

\end{this_body}

