\newsection{再面泥沼蟹}    %第十二节:再面泥沼蟹

\begin{this_body}

%1
“找到了。”方源双眼一亮,驻足在一根紫色的石柱之前。

%2
太白云生循声望去,只见这根关键的石柱,却也普通平凡。唯独石柱根部,有一块石头,平滑如凳,看起来有些特别。

%3
“就是这里。”方源伸出粗壮的怪手,将石凳顶部的残雪拂开,再次确认。

%4
待太白云生钻入他的仙窍后,方源便坐在石凳上,动用蛊虫割开自己的手臂,将血液涂在这根紫色石柱上。

%5
他的血液,已经不是正常人的鲜红色,而是碧绿色,一点温度都没有,很冰寒。

%6
紫色石柱,曾经被盗天魔尊暗施了神秘蛊虫,很快就将碧绿尸血吸收的点滴不剩。

%7
方源的脑海中,迅速闪过马鸿运的身影。

%8
这份机缘,原本属于马鸿运,方源算是盗用之人。

%9
马鸿运、赵怜云二人被与运道真传带出大同风幕后,就遭受到北原正魔两道蛊仙的哄抢。当时的场面一片混乱。

%10
如今他们俩已经音讯全无,可以肯定是被某个蛊仙捉住。能否脱困,又遭受了什么,不为人知。

%11
比较起前世,这一世马鸿运惨多了。

%12
可见纵然有齐天的鸿运,也不是万能的。

%13
运气只能提供机会,能否充分利用和抗衡,还得看蛊师个人的实力和手段。

%14
现在马鸿运仍旧被疯狂搜寻,不知道是哪个蛊仙藏起了他们。运道真传的诱惑力,尤其对于北原蛊仙而言。十分巨大。

%15
“我的春秋蝉有削弱自身气运的弊端,若是能够利用运道真传弥补这个缺陷,那就最好不过了。重生以来。我已经吃够了坏运气的苦。”方源心中感叹一声。

%16
不仅是马鸿运怀璧有罪,更关键的是——北原一行,马鸿运也是目击证人、当事人,他很有可能将方源暴露出去。

%17
因此,马鸿运同样是黑楼兰的死地。

%18
追杀马鸿运,已经是雪山之盟的内容之一。

%19
再次进入琅琊福地,方源却没有像前一次那样。直接进入云阁中的静室。

%20
琅琊福地地貌独特,乃是白茫茫的浩荡云土。

%21
十二座楼阁,相互间隔。分别矗立在云土之上,号称十二云阁。

%22
但方源看到的,却是一片狼藉景象。

%23
肥沃的云土,到处都是坑洞。有些洞中结满了寒冰。有些坑里还冒着袅袅黑烟。

%24
太白云生从方源仙窍中出来。看到这一幕,立即脱口而出道:“这里不久前发生过一场大战。看来琅琊地灵,真的有麻烦。”

%25
方源没有说话,只是遥望远方的十二云阁。

%26
这十二座楼台,各有特色,有的仙鹤环绕,有的羽人寄居,有的彩霞漫空。有的檀香逸散。

%27
但现在这十二云阁,其中八座完好无缺。剩余三座却是被攻打得破损不堪。雕梁画栋,成了断壁残垣。

%28
“怎么回事?我们已经来到琅琊福地,怎么地灵还未出现?”太白云生眉头皱起,心中更加警觉。

%29
“先进去看看罢。”方源沉吟道。

%30
两人飞行过去,慢慢接近十二云阁。

%31
“快看那里,有一只荒兽尸体!”行了半道,太白云生忽然手指着某个方向,开口道。

%32
只见一只巨鱼,躺在云土上,一动不动。

%33
它有正常鲸鱼大小,却形似鲤鱼。背脊处,有骨刺长出体外,长长地延伸出去。

%34
它的鳞片都是汪蓝色泽,一双死鱼眼大如马车,残留着些许星芒。

%35
“这是荒兽刺脊星龙鱼。”方源道。

%36
他知道,琅琊地灵的手中有一只驭兽仙蛊,奴隶了十二头荒兽,分别隐藏在十二云阁之下。

%37
这头刺脊星龙鱼,是否就是十二荒兽之一?

%38
“咦?有人进来了。”灰暗的密室中,琅琊地灵忽然睁开双眼,感应到方源和太白云生的存在。

%39
在他面前,坐着一位蛊仙。他中年模样,颇有威势,外貌易于常人,黑肤白发,赫然是一位墨人蛊仙。

%40
“怎么?又来强敌了?”墨人蛊仙听到地灵这么说,顿时紧张起来。

%41
“原来是这个臭小子!呼,吓我一大跳。没事,这人我认识。大半年前,他来过我这里。”琅琊地灵全神贯注地感应后,吐出一口浊气。

%42
墨人蛊仙讶异地扬起眉头,没有想到除了他之外,还有人能来到琅琊福地做客。

%43
“既然来者是友非敌,那就好说。我们进入密室,已经三天三夜,再过片刻,就能揭开你身上的一层封禁。这个时候,万不能受到干扰打搅。”墨人蛊仙心头的巨石落下。

%44
“那也不能让他们接近云阁。老友你不知道,这小子狡诈无比,是盗天魔尊的传人,我吃过他的大亏。先让我调动荒兽,暂时阻住他过来。”琅琊地灵咬牙切齿地道。

%45
“我们进来这么久,琅琊地灵仍旧没有出现,看来这里真的发生了大事!进攻琅琊福地的,绝非一人所为。这样的势力可不好惹,我们还是尽快搜索战场,然后撤退为妙。”

%46
太白云生正说着的时候,一块巨大的“黄色金属山石”,从云泥中缓缓钻出来。

%47
“块状巨石”动作轻巧,拦截在二人面前。

%48
“荒兽!”太白云生如临大敌。

%49
荒兽没有眼睛,一对巨大的螯足,摆在它的最前方。这对狰狞的凶器,没有人敢怀疑它的威力。

%50
随后,另外的九对螯足,也相继从块状巨石的身体两侧,延伸而出。

%51
螯足深深地插在云土中,将荒兽山一样的雄阔身躯撑高起来。

%52
至此,太白云生终于认清这头荒兽的来历:“这是沼泽君王——泥沼蟹啊!”

%53
方源冷哼一声。他对泥沼蟹再清楚不过。今生他夺得狐仙福地后,渡劫时就遭遇了一头泥沼蟹。

%54
关键那头泥沼蟹的身上,竟然还带着和稀泥仙蛊。祸害了荡魂山。

%55
为了拯救荡魂山,方源才踏上北原,图谋太白云生的江山如故仙蛊。如今,方源冒着九死一生的危险,不仅成功救活了荡魂山,而且还把太白云生都招揽到身边,可谓人蛊两得。

%56
泥沼蟹拦住了二人去路。太白云生停下身形,问方源道:“现在该怎么办?”

%57
方源目光锁定眼前的荒兽,轻轻地吐出一个字:“打。”

%58
“你小心。”太白云生点头。迅速后退,和方源拉开距离。他是治疗蛊仙,一般情况下,并不亲自冒险。这也是符合他性情的一贯打法。

%59
方源悬停在空中。静静地看着泥沼蟹。随后,他深呼吸一口气,腰杆挺拔,八臂伸张。

%60
他浑身的肌肉贲发,宛若钢铁怪像。血红双眼,青面獠牙,更添狰狞可怖之气。

%61
下一刻,方源陡然发动。身形如流星一般,狠狠地朝泥沼蟹砸去。

%62
泥沼蟹体型庞大。却有非同一般的灵活,九对螯足迅速挪移,带动身躯往旁边躲闪。

%63
但方源却是飞行大师,看似难以转向的冲撞,忽然一折,砸在泥沼蟹的背上。

%64
轰!

%65
一声雷霆巨响,泥沼蟹被方源带来的冲击力压得身形一低。

%66
方源高达两丈的身躯,站立在一个凹坑当中。这就是他刚刚一击,造成的伤害。

%67
“够硬!”方源咧嘴一笑,回收自己的四只右拳。

%68
他的拳头,无一例外,都是皮开肉绽,露出了白骨。

%69
但方源早已经失去了痛觉。裂开的拳骨,正以肉眼可见的速度愈合。几个呼吸之后,他的四只右拳再度恢复如初。

%70
呼呼!

%71
两道恶风猛烈袭来。

%72
方源背后羽翼打开,脚下一踏,身形闪电般拔升而起,躲过恶风。

%73
锵锵两声随即传来,方源回首一看,只见泥沼蟹摆在面前的那对巨硕螯足,正以一种不可思议的延伸程度,伸到自己的背上来,凶猛地钳合在一起。

%74
刚刚方源若是躲闪得慢一点,恐怕就要遭殃了。

%75
荒兽的战力,媲美六转蛊仙。方源就算是防御暴涨的天尸躯壳,也不想品尝一下被钢钳夹住的滋味。

%76
“泥沼蟹一身甲壳,刚硬超凡,防御几乎毫无漏洞。我没有斩断、钻破的手段,唯有以力破力,以暴制暴,才是正途。”方源脑海中念头迅速碰撞。

%77
他身形继续拔升,往下俯看。短短几息功夫,泥沼蟹的身上就已经布满了大量的螃蟹。

%78
这些螃蟹,有的大如虎,凶猛无畏。有的螯足尖锐,如同钢针。有的肢节如八爪,速度奇快。

%79
这是泥沼蟹的特有能力——它能够随时随地自我受孕,生产出庞大的蟹军,忠诚无二地为自己驱使。

%80
“很好。”方源目光森冷,猛吸一口气,悍然催动杀招万我!

%81
一颗青提仙元消耗,首先催动了核心仙蛊。

%82
随后大量的蛊虫,以某种规律,也紧跟着相继催动起来。

%83
方源八拳齐出,遥遥打向底下的泥沼蟹。

%84
嘭嘭嘭……

%85
每一拳,都爆发出一团无形无色的拳气。拳影翻飞,拳气如暴雨,倾盆而下!

%86
泥沼蟹发出尖锐的惨叫,身形被绵绵不断的拳气,越压越低。

%87
万千拳气砸在泥沼蟹的背上,将它的硬壳砸的凹凸不平。崩散的拳气,又再度相互汇聚,形成方源力道虚影。

%88
十个呼吸左右,泥沼蟹的背上,就伫立了一支力道虚影大军。

%89
“这是……”太白云生为之瞠目,他尚是第一次亲眼看到这个杀招。

%90
力道虚影围剿泥沼蟹,螃蟹大军一触即溃,泥沼蟹很快支撑不住,连连惨叫,奋力抗争,却难掩败局。

%91
无论它剿杀击溃多少力道虚影,方源一刻不停的进攻,每时每刻都会有更多的力道虚影产生。

%92
“哎呀呀,我的荒兽!这臭小子使得什么杀招,我从未见过,威力居然这么强!这个杀招,绝对是仙道杀招!!”密室中,琅琊地灵感应无差,再无镇定,当即大呼小叫起来。

\end{this_body}

