\newsection{参加琅琊攻防战}    %第二百二十四节:参加琅琊攻防战

\begin{this_body}

ps:看《蛊真人》背后的独家故事,听你们对小说的更多建议,关注公众号(微信添加朋友-添加公众号-输入qdread即可),悄悄告诉我吧!方源得到变形仙蛊之后,又将其融入到见面似相识仙道杀招之中,如今已经可以变化任何事物。

但他要再次混入北原僵盟,情势却不大妙。

因为沙黄身份暴露,导致北原僵盟十分警惕,重复第一次的行动,是无法顺利混进僵盟的。

于是,方源便打起其他僵盟成员的主意。

只要秘密打杀其中一位,再变作这个倒霉鬼的样子,混入北原僵盟就是可行的。

方源在北原升仙,浑身都是北原气息。僵盟的大门可以阻挡住凤九歌,却阻挡不住他。

但这个秘密袭杀的事情,也有困难和风险。

首先杀掉一个僵盟成员,整个过程得悄悄进行,不能透露出任何风声。

要做到这点,就很难了。

方源此时有七转战力,要击败一位普通仙僵,十分容易。击杀仙僵,也不无可能。

但要秘密袭杀,却是分外困难的。

因为战斗的时候,动静是很大的,控制不住。仙僵若是想跑,方源就得追击,返实蝠翼杀招又不是那么可靠。仙僵若是用信道蛊虫求援,方源就得在激战中拿出足够多的又正确的手段,去剿除这些蛊虫。

归根结底,方源除了进攻端,实力超群之外。其他方面还是有很多短板。尤其是,方源缺少战场杀招。

战场杀招,可以短暂地隔绝天地,营造出一个独立的战斗环境。

在战场杀招之中,蛊仙们通常是不能用信道蛊虫求援。当然。此事不绝对,绝对的事情世间是不存在的。

而且优秀的战场杀招,局限了范围,因此又能防止敌人四处逃窜。

仙道战场杀招,有个别称,称之为“准福地”。一旦催出。便能临时复刻无数道痕,形成一片暂时性的独特战场。

这种仙道战场杀招,其实大部分的蛊仙都没有,只掌握在少数蛊仙的手中。

方源没有仙道战场杀招,要独自袭杀一位北原僵盟成员。就十分困难。

就连他前世也没有。因为他刚刚得到第一只仙蛊时,他就惨遭围攻,不得不自爆重生。

黑楼兰、太白云生也没有。

他们刚刚晋升成仙,就算仙窍再优秀,也得有个漫长的时间去积累。

但黎山仙子有。

梨园,就是她的战场杀招。

若是能将仙僵诱入梨园,那么就有很大可能,秘密斩杀了。

当然。就算做成这样,也有风险。

方源当然不会忘记残阳老君这群人,他们之中必有智道蛊仙。推算出方源再度混入僵盟的可能性。非常大。

一旦真相暴露,甚至稍稍有点风声传出,方源混入北原僵盟,就等若只身闯进龙潭虎穴。

所以,黎山仙子的帮助,是非常必要的。

但黎山仙子拒绝了方源的请求。正当方源要另寻他策之时,一份求援信传到了他的仙窍中。

自从北原拍卖大会之后。方源就很少和琅琊地灵联系了。

这次琅琊地灵主动联系方源,却是求援来了。

这大大出乎方源的意料。

在方源心目中。琅琊福地可是底蕴极其深厚的。

前世的记忆,琅琊福地这片无主福地,足足抵御了七波蛊仙的强烈进攻。并且在第七次,甚至凤九歌都牺牲在琅琊福地之中。

虽然琅琊福地明面上,只有十二云阁,云阁底下各有一头荒兽。也就是十二荒兽。

但只要不是傻子都会明白,这十二头荒兽不过只是最表层的防护力量,别忘了琅琊福地的原主人乃是长毛老祖。

他号称古往今来炼道第一仙,生平炼出三十八只仙蛊,不发七转、八转。这还只是正史统计出来的数字。若算上野史,传闻等等的话,这个仙蛊的数量足以突破一百大关!

所以方源早就推测,琅琊福地中真正的防守力量,在于长毛老祖遗留下来的诸多仙蛊。

而不是荒兽。

琅琊福地和其他福地有一点大大不同。就算没有主人,福地地灵也有源源不断的仙元,可以催用消耗。

这是因为,琅琊福地中有一只天元宝皇莲仙蛊,可以自产仙元。

正是天元宝皇莲的支撑,导致琅琊地灵可以大手笔地催动仙蛊,发挥出它们的独到威能。

“现在,琅琊地灵居然主动向我求援?他前世可以支撑到第七波,仔细算算,这一次不过是第四波攻势吧。难道说,因为我的缘故的,导致北原格局大变,影响太深。令琅琊福地遭受了史上最猛烈的侵略不成?”

方源脑中急速思考。

能让琅琊地灵的求援,这次的入侵者的实力可想而知,绝对是要超过方源的个人能力的。

但危险中,又常常蕴藏着宝贵的机遇。

琅琊福地积累深厚,富得流油,就连前世的天庭都要被勾动贪欲,强行打下这片福地。天庭虽然付出了重大伤亡,但也因此受益匪浅,就算牺牲了凤九歌也是大赚特赚。

和琅琊地灵合作,是有很多好处的。

这一点,从过去的交往中,就可明显的看出来。

方源靠着坑蒙拐骗,从琅琊地灵手中夺了多少好处。纵观他今生成仙之后,之所以能迅速发展到这一步,琅琊地灵方面的帮助是相当大的。尤其是在前中期那会儿,起步的艰难阶段。

可惜北原拍卖大会之后,方源的贪婪和索求无度,终于引起了地灵的反感。导致双方关系恶化,降至冰点。

方源曾经几度要求和地灵合作,但都被地灵以强硬的姿态回应,始终要求方源为其搜寻一种毒花花瓣的仙材。

“如果,这次我能支援地灵。必定能令双方关系缓和。”

“并且我有定仙游,只需要支撑三息时间,就能回归狐仙福地。就算对方实力再强,我争取三息的时间总该够吧?”

方源心里推测,虽然是琅琊地灵情急求援,但恐怕战况并不那么糟糕。

他始终觉得琅琊福地的底蕴。是相当浑厚的。前世能抵挡七波攻势,连凤九歌这一代强人都交代在里面。这一次不过是第四波攻势。

这样思前想后一番,方源决定支援琅琊地灵。

他不是像羽飞一样的鲁莽小子,浑身热血,朋友遭受侵犯。就会不计得失挺身而出。

他是方源,魔道蛊仙。

他讲究利益,冰冷可观地计算得失,盟友就是拿来利用的,彼此交情是什么玩意?

在方源心中,只是获取利益的一种特殊筹码罢了。

正巧这时,太白云生也来信询问。

琅琊地灵同样也给他,送去一份求援信。

事实上。太白云生曾经为地灵,修复过琅琊福地。论交情,太白云生和琅琊地灵之间的关系程度。比方源还要好得多。

方源便告知太白云生,自己的决定,得到了太白云生的肯定和赞扬。

方源即刻动身,利用星门,赶往琅琊福地。

太白云生虽然渡过了地灾,但要回收成仙窍。还需要一段时间,只能随后赶去。

“你来啦。方源小子。”琅琊地灵陡然现身,出现在方源的面前。

方源见他并不委顿。浑身无伤无缺,心中更加肯定之前的猜测,不由哈哈一笑:“这就是你对救命恩人的态度吗?”

“哼!你以为我老人家不知道吗?你这臭小子,无利不起早!要不是这次情况紧急,我老人家才不会请你帮忙。”琅琊地灵翻着白眼道。

话虽然这么说,但琅琊地灵的态度,已经比之前缓和很多了。

“这次击退来敌,我就将六转的全力以赴蛊仙蛊方,当做给你的报酬!”琅琊地灵又接着道。

方源嘿嘿一笑,却是微微摇头道:“报酬的事情,还是再说吧。咱们先将入侵的蛊仙击退。”

方源没有谈论报酬,而是忧心来犯之敌,更没有乘机要挟,这个表现顿时让琅琊地灵舒心很多,感觉看方源越看越顺眼了。

“跟我来。”琅琊地灵抓住方源的手臂,倏地瞬移,来到一座云阁的最高层。

可见琅琊福地中,有大量的宇道道痕。

否则类似星象福地这种,宇道道痕不多,星象地灵就只能飞行,没有瞬移的能力。

方源站在高处,往外望去,只见琅琊福地已经大变模样。

原先的琅琊福地,广袤的云土,晴空万里,十二云阁伫立其中,比邻相间。但此刻,方源眼前充斥着无边的茫茫云雾,伸手不见五指。

上不见天,下不见地。

琅琊地灵用低沉的声音介绍道:“这是琅琊福地遭遇外敌之后,发动的蛊阵十二波云迷澜阵。十二云阁,都是凡蛊屋,在此阵中充当阵眼。你看!这些人就是来犯之敌。”

地灵挥袖一扫,方源眼前便显出数片明镜。

镜中,各有一位蛊仙,在茫茫云雾中或缓步而行,或急性冲锋。

“这是!”方源见到这些蛊仙,瞳孔不由一缩。

秦百胜、姜钰、黑城、雪松子、贺狼子!

还有两位,一位青袍风道蛊仙,行进的速度最快。一位黑袍蛊仙,遮住脸面,神秘异常。

“你认识这些人?”琅琊地灵觉察出方源流露出的异色。

方源点头,神情严肃:“这些人实力不凡,尤其是领头的秦百胜,更是北原七转中的知名强者。”

“我知道。若非如此,我的十二波云迷澜阵,也不会被攻破了大半。现在只剩下五座。其中两座,已经由墨人城的两位墨人蛊仙镇守。你现在镇守这座,我亲自镇守另一座。还剩下一座,留给太白云生。”琅琊地灵分配道。

方源皱起眉头:“对方实力极强,云阁不过是凡蛊屋,要让我挡住这些人,还要保住云阁,相当困难啊!”(天上掉馅饼的好活动,炫酷手机等你拿!关注起點中文网公众号(微信添加朋友-添加公众号-输入qdread即可),马上参加!人人有奖,现在立刻关注qdread微信公众号!)(未完待续)

\end{this_body}

