\newsection{蓝}    %第三百零六节:蓝

\begin{this_body}

中洲,灵缘斋,议事堂。<a href="http://www.mianhuatang.cc" target="\_blank">棉花糖小说网WWW.Mianhuatang.CC</a>

门派中的十五位蛊仙,有超过半数,都是真身降临。

这是相当罕见的一幕。

门派议事每隔一段时间就会有,蛊仙们只需要留下一段意志或者感情,进行商议即可。

但现在,却有八位蛊仙亲自参加这次的议事,其中就包括灵缘斋唯二的两位八转蛊仙。

造成这个现象的原因,就是凤九歌的失踪。

灵缘斋的蛊仙们都感到了未来的动荡和不安。

“这一次,将大家召集过来,是有一个重要的消息要宣布。”灵缘斋太上大长老,端坐主位,目光缓缓扫视四周,声音低沉。

灵缘斋和其他九派,有一个明显的区别。

那就是女仙众多。

灵缘斋中的女蛊仙有十位,男蛊仙只有五位。

灵缘斋的太上大长老、太上二长老都是八转女仙。

此刻,堂中众仙的目光都集中在太上大长老的脸上。

太上大长老面无表情,不过她身旁的太上二长老,却是阴沉着脸。

蛊仙们察言观色,都产生了不妙的预感。

果然,太上大长老接下来的一句话,好像是一块沉重的巨石,砸在众人的心坎上。

“已经确定,凤九歌陨落在北原。他死于大同风中,没有任何遗物,只有两个血字的留言。”

众仙心中皆是一沉。

亲自参加此次议事的白晴仙子,更是脑袋猛地眩晕,脸色倏地惨白一片。

哪怕有心理准备。但真正听到这个噩耗的时候,众人还是有些不可思议。

强大如凤九歌这样。居然折损在了北原。相反比他实力低的那些蛊仙,却大多生还。回归了门派。

说实在话,凤九歌起行时,没有人会料到竟是这样的结果。

长久以来,凤九歌强大的形象,已经深入人心。他是灵缘斋的招牌,甚至成了一个象征。

如今他一死,众仙心中都有些失落和空虚。

哪怕是倒凤派系的徐浩、李君影,都有这样的感觉。

太上大长老继续道:“你们面前的信道蛊虫中,就有此事的详细记录。<strong>求书网WWW.Qiushu.cc</strong>都先看看罢。”

蛊仙们纷纷向蛊虫中,探入心神。

“唉,凤九歌大人死在大同风中,也不算辱没他的名声了。”良久,一位蛊仙出声,打破了堂中的沉默。

白晴仙子紧闭双眼,身躯都在微微颤抖。强烈的悲伤和痛苦,宛若汹涌的海啸,将她吞没。

她是如此的深爱着凤九歌。同样的,凤九歌也是如此爱她。

她的脑海中,浮现出临行送别时的画面。没想到那竟然是她看夫君的最后一眼!

命运弄人。

如今,我生你却死。我身在中洲,你却陨落于北原!

白晴仙子不敢睁眼,她怕睁开双眼。眼泪就会止不住地滚落下来。

她努力去想自己的女儿凤金煌,在心中不断地告诫自己:“白晴啊。白晴,你要坚强。这个时候。绝不能让别人看到你软弱的样子!”

她深呼吸几口气,慢慢的睁开双眼。她的眼中,已经湿润,瞳孔上带着血丝。

此时,堂中众仙已经谈论到凤九歌最后的血字遗言。

“凤九歌临死之前,在手中书写了‘薄青’二字,究竟想表达什么意思?”

“在我看来,这个线索相当重要。想必是凤九歌在死亡的巨大压力之下,猜测领悟到了什么。可惜他见到赵怜云时,已经油尽灯枯,没有力气再多说什么,便给我们留下关键的线索。”

“凤九歌在北原调查八十八角真阳楼倒塌的真相,和薄青有什么关系?”

“凤九歌、薄青这两个人很相似。当薄青比凤九歌要强大得多,他是当年的中洲巅峰,就连天庭蛊仙都要退让垂首。那个时候,是我们灵缘斋最辉煌的时代!当时,很多人都看好他,认为他能够成为剑道仙尊。可惜最终,他还是失败了。”

“薄青的情报,我们也知晓的。我只想知道,凤九歌为什么在临死前,独独留下薄青二字?他究竟想要表达什么?”

堂中沉默了一下,一位蛊仙开口道:“诸位忘了吗?在之前的情报中,凤九歌的对手秦百胜,就施展出五指拳心剑。而这个杀招,就是薄青所创,也是他的招牌杀招。”

“凤九歌的意思,是想说八十八角真阳楼倒塌,和薄青有关系?”

“依我推测,他应该是觉得,对方居然掌握了五指拳心剑,多多少少会和薄青有所联系吧。而薄青乃是我灵缘斋的蛊仙,我们调查起来会更有优势,这是个重要的线索。”

“的确如此。薄青当年彻底陨落在天劫之下,连骨灰都没有。他的杀招,怎么会被一位北原蛊仙掌握呢?”

蛊仙们又议论了一阵,众说纷纭,但都没有什么靠谱的说法。

太上大长老缓缓地抬起手,她的这个动作让堂中迅速安静下来。

“不管如何,薄青的事情需要调查。这项任务就交给你负责吧,白晴。”

听到太上大长老忽然点自己的名字,白晴仙子立即转头,看向太上大长老,连忙应下。

这是她夫君死前留下的遗言!

白晴仙子定然会竭尽全力,去调查这个线索,追溯真相。

“凤九歌的死讯,只限我们知晓,尽量隐瞒。谁若透露出去,就以背叛门派论处!”太上大长老冷喝一声,“接下来,我们着重商议一下门派在中洲各地的部署。”

凤九歌的死,带给灵缘斋的影响是很恶劣的。

灵缘斋虽有两位八转蛊仙,但不到门派存亡关头。这两位八转是不会轻易动手的。

这其中的原因有很多。

其一,八转这个修为。已经注定蛊仙如履薄冰,战战兢兢。全心全意应对灾劫。稍有大意,在争斗中折损实力,就会陨落在恐怖的浩荡灾劫之中。

其二,中洲十大古派拥有同一个源头,那就是天庭。天庭既在,十大古派的争斗,就永远不会升级,不需要八转蛊仙出手。

所以中洲蛊仙界,或者说整个五域的蛊仙界中。真正活跃的还是那些七转、六转蛊仙。

而在七转中,无敌中洲的凤九歌,对于门派的重要性,就不言而喻了。

正是因为有他的存在,灵缘斋的版图才扩张到这么大的地步,才占据了无数珍稀的资源点。

凤九歌一去,灵缘斋对其余实力的震慑力大为降低。门派掌握的种种资源,就像是一块块鲜美的肥肉,勾来无数贪婪的目光。

“玄武山脉资源丰富。是仙材宝库,至少需要一位七转蛊仙坐镇。”

“金沙窟的开采,已经到了关键时刻。我们投入了大量的资金和精力,正是要收获的时节。不应该放弃。”

“轮回战场才是重中之重啊……”

蛊仙们都感到头疼,版图太大,但蛊仙战力就这么多。就算把他们掰成两半来使。也是不够的。

直到此刻,他们才真正意识到。凤九歌的威望,对外界的威慑。是多么的强大。

白晴仙子默不作声。

蛊仙们讨论着这些仙材、资源,喊叫不休,再不提起凤九歌。

好像凤九歌的最后存在,就是刚刚的那段对“薄青”二字遗言的讨论。

白晴仙子不免有许多悲凉之情:“夫君啊夫君,你为门派做了这么多贡献,到头来,这些人转眼就将你抛之脑后了。”

整个议事的过程中,白晴仙子都不在状态。

众仙看在眼中,也都理解她,一贯严厉的太上大长老都没有批评什么。

唯有当众仙提到赵怜云时,白晴仙子双眼一亮,对此密切关注。

凤九歌若在,凤金煌几乎可以肯定,就是灵缘斋的当代仙子了。但如今凤九歌一去,赵怜云横空出世,成了凤金煌的巨大威胁。

白晴仙子当然更爱护自家儿女,对赵怜云之事就很上心。

只听蛊仙们不断讨论:

“赵怜云一连继承了两个盗天真传,她有什么变化吗?”。

“天外之魔不可信任!”

“神不知、鬼不觉,这两道顶级的仙道防御杀招,我们还在研究当中……就以目前来看,高深莫测!这两道仙道杀招,已经在赵怜云的魂魄上形成了两层密实的道痕丝衣。这种奇妙的道痕运转,我还从未见过!”

“这两层道痕丝衣,会一直保护着赵怜云。我们用了许多法子,尝试推算她,都没有效果。我们堂堂蛊仙,推算一个凡人,都推算不出来。若不是我亲身经历,绝不会相信。”

“你们发现没有?神不知、鬼不觉是顶级的防御杀招,居然不耗损仙元。就好像是鸿运齐天蛊一样。尊者的境界,真是难以理解啊。”

“隔绝念头、意志、情感的推算,这是神不知的防御效果。那么鬼不觉呢?”

“还不太清楚,正在多方尝试之中。哦,整个过程赵怜云都很配合,我觉得她虽然是天外之魔,但很识时务,完全可以培养。”

“这个小家伙一心想要救自己的情郎呢。呵呵,可惜根据情报,雪胡老祖似乎已经筹集到了足够的仙材,不久之后就要正式炼蛊了。”

最后,太上大长老为此事总结:“继续研究,另一方面,也要多加培养赵怜云对我派的归属感。我很期待她的未来!”

与此同时,在一处无名的森林中。

七星子仙僵正透过一面光镜,和他人通信。

光镜中是一道模糊的老者身影。

他徐徐开口道:“所有的布置,都已经基本到位。但天庭已经将完成了这一次的宿命蛊修复,你那边必须先行启动。”

“明白。”七星子仙僵沉声答道。

“要小心,蓝副使。”光镜中的身影又道。

七星子没有再说话,他停止催动这个信道仙级杀招,光镜骤然消失。

随后,他头也不回,转身而去,身影迅速没入森林之中。(未完待续……)

\end{this_body}

