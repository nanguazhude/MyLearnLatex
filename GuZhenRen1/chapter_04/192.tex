\newsection{方源的公然挑衅}    %第一百九十二节:方源的公然挑衅

\begin{this_body}

好似听到火工龙头内心的声音,方源继续嘲讽,甚至加大力度:“对,就是这眼神,就是这样的情绪,愤怒吧,气愤吧,如此才能拿出你全部的实力来和我拼。如此一来,这样的对手,才让我稍微可以有点儿兴趣。记住,你一定要拿出,不然你就是万龙坞的耻辱!哈哈哈。”

看着方源手指着火工龙头,如此态势嚣张的发言。

一时间,场上场下的蛊师们都说不出话来。

这样太嚣张了吧?!

万龙坞一行人咬牙切齿,他们嚣张,方源比他们更嚣张。

火工龙头气得都要哆嗦起来了,死死瞪着方源的双眼直欲喷火。

和外人想象的不同,方源一副豪迈的外在表现,实则内心却是十分警惕警醒的。

“和外界的消息不同,这火工龙头其实并没有主修炼道,而是转修炎道。他早年在万龙坞时,是炼道蛊师。也的确曾经被驱逐出派,但因为一场奇遇,反而使得他获得某位炎道蛊仙的传承。这段时间,他侥幸成为炎道蛊仙之后,又眼巴巴地投入万龙坞的怀抱。万龙坞自然不会放过一个蛊仙战力,其实早已暗中将他收入门中。只是正值炼蛊大会,看他炼蛊竟然也有宗师境界,便连忙叫他来参加炼蛊大会。企图抢占前几名,为门派抢夺更多的利益!”

“这火工龙头的炼道造诣超出我一大截,前世就是此届大会的第二名,炼道宗师境界不可轻悔。尤其是第十场武斗时,他一举力压同场炼蛊的三位准宗师。凶威赫赫。这样的敌人,必须提前打击。若是放过这次机会,按照前世记忆,他几乎就不可遏制了。”

不过火工龙头虽然已经是蛊仙,但没有声张。而是对凡人隐藏了身份。

他的蛊仙修为,当然瞒不住其他蛊仙,但大家心知肚明就行了,也没有人去戳破他。

蛊仙保持自身神秘感,对维护统治有着帮助。

还有一个原因:万一在炼蛊大会中输给了凡人,那就太丢脸了。

为什么蛊仙也会输给凡人呢?

凡人击败蛊仙这种例子。其实在历届的炼蛊大会中屡见不鲜。

因为炼蛊并非打打杀杀,它是个技术活。最主要的原因是仙蛊唯一,太太太难炼制成功了。

没有仙蛊作为标杆,蛊仙和凡人比试,就只有依靠炼制凡蛊了。

凡蛊大家都能炼制。很大程度上体现不出仙凡的差别。

所以一般而言,蛊仙们若参加炼蛊大会的,都不会自爆身份。

“开始吧。”方源对驱邪派长老催促一声。

“快开始!”火工龙头大吼,他已经迫不及待地要将这个嚣张的方源立即踩在脚下了。

驱邪派长老强忍二者的压迫,咬牙坚持道:“按照规矩时间未到,不好开赛。还有三息……三,二,一。好,催动隔绝蛊阵,关闭上场入口。开放试题!”

炼蛊大会前七场,因为参赛人数太多太多,只能轮流上场比试。

现在到了第八场,人数大幅度地减少。每个比试地点只举办一场,错过这个时机的蛊师,就只能认做自动放弃。

装载试题的蛊虫。也大有讲究。这种信道五转凡蛊,乃是天庭的手笔。有非凡的保密能力。

当中的试题,也不是存进去的。而是由天庭在那端即时管控。等到快要开赛时,这才将真正的试题输送过来。

试题一出,场内场外顿时寂静一片,众人无不屏气凝神,盯着试题,一时间场中静的针落可闻。

第八场试题:要求炼制一只缄默蛊,五转音道,成功炼制出来并且速度最快的人,为优胜者,可进入下一轮比试。其余蛊师,均作淘汰。

给予材料:仙魂草、忘忧石,四转冷漠蛊一只,三转三缄其口蛊六只,四转沉没蛊两只……不允许适用其他自带材料。

给予炼蛊蛊方:六张不同的缄默蛊蛊方,三张内容不一的冷漠蛊蛊方,以及一张勾连蛊蛊方。

众人无不大皱眉头。

这个试题的难度,要比第七场的暴涨近十倍!

不仅给予了十张不同的蛊方,这就要求蛊师眼光独到,能从中挑选自己适用的。而且提供的材料是有限制的,不允许适用自带材料,但偏偏这些材料都不满足任何一件蛊方中的要求。

也就是说,蛊师们必须逆炼冷漠蛊、三缄其口蛊或者沉没蛊,得到更多的炼蛊材料。利用这些材料,进行再度加工,最终炼成五转音道的缄默蛊。

又或者,参赛蛊师可以利用智道手段,将这些蛊方都综合起来,一起改良。依照提供的材料为基础,推算出符合条件的蛊方。这样也可以。

不过这两种方法,当然是第一种最为有效可取了。因而第二种需要极强的智道造诣,能够闯到这一步,同时也拥有如此深厚的智道造诣的蛊师,相当少见的。

更关键的是,这场比试还考虑时间。

哪一位能够第一个炼制出来,就能获得优胜,其他人均被淘汰。

如此一来,心理压力就很大了。尤其是到最后关口,很可能会有蛊师承受不住心理的重压,而导致失误。

火工龙头看到这一题,差点一口老血吐出来。

这是什么鬼题目啊

火工龙头差点要骂娘!

他实际上是炎道蛊仙,最擅长的也是炼制炎道蛊虫。这音道蛊虫显然不在他擅长的范围之内,而且其中几个关键的炼蛊材料的处理,竟然都不能用火炼之法!

这就相当尴尬了。

方源之前叫嚣,让火工龙头使出他的最强炼蛊杀招疯神烈焰。

火工龙头心中恼怒,也打算使用这个杀招,狠狠地羞辱方源一番。

但现在这题目,让火工龙头还怎么用疯神烈焰呀?烧着玩么?

这道题目,对他火工龙头简直克制极了。

打个比方,十成炼蛊造诣他能发挥出来的,只有八成。即便是蛊仙级炼蛊,也有擅长和不擅长的地方。

这道题目,恰巧正中火工龙头最不擅长的地方。

当然,炼道宗师境界不是虚的,火工龙头绝对是能成功地将蛊虫炼制出来的。

但关键是出现了方源这个意外。

方源真实的实力,足以威胁到火工龙头。同等级的强者,一分实力的降低,就可决定胜负,何况是火工龙头这样,削弱了整整两成!

“这个家伙,倒是挑了个好时机。运气还真他娘的好!不行,我一定要击败他!”火工龙头暗暗发誓,瞟了一眼方源。

下一刻,他的眼珠子鼓瞪起来,差点要掉在地上!

方源竟然已经出手,已经开始炼制了!

这怎么可能?

怎么可能这么快?

要知道,这题中给予的蛊方和材料,是不搭配的。蛊师要么逆炼蛊虫,改造材料,要么就综合蛊方,推算出新的。总之,最后要炼出五转缄默蛊出来。

所以,这就要求蛊师进行深度的思考,在自家的头脑中不断推演,考虑正炼、逆炼中可能出现的失败,从而综合许多方案,组织出一份最精炼的炼蛊方案。

这个思考的过程非常重要。

思考出来的方案,必须要风险低。风险太高,失败多了,材料耗费就增多,不够下一次炼蛊所需,又不能用自带的材料,那就只能认输了。

要求风险低的同时,还得要求时间消耗少。你耗费时间太多,别人早就完成了,你就算是炼制成功,也会被淘汰。

既要风险低,又要时间消耗少,这就很难了。

需要精心的思索,不断比较排列,从而才能选出最优的方案。

就算是火工龙头这样的炼道宗师级人物,初步估算下来,自己这番思考也得要半炷香的时间。

但,这这这方源,怎么一上来就炼蛊了?

“有没有搞错?”

“他真的就这样自信?!”

“愚蠢,工欲善其事必先利其器,应当沉下心来思考才是正确的。”

同场竞技的蛊师们,纷纷瞟了方源几眼,均集中注意力,开始深度思索。

方源一马当先,进度极快。

其他蛊师都没有开始炼蛊,整个场中只有他一人动手,自然而然,大多数的场外蛊师都将目光集中在他的身上。

赞叹声迭传。

“真是让人看了都要眼花缭乱的手法啊。”

“熟练,太熟练了!简直是深入骨髓的本能!”

“不过这样真的好吗?欲速则不达啊。”

方源当然知道欲速则不达的道理,但他还知道这道试题的内容。

在方源前世,他成就血道蛊仙之后,也曾打炼蛊大会的主意。

为什么呢?

因为在炼蛊大会期间,就算是魔道都可以公开参加,哪怕是通缉犯,也能堂而皇之报名。正道是不会追捕的。

那个时候,方源手中空无一只仙蛊,正努力筹措炼制仙蛊的仙材,整个过程真是艰难的欲仙欲死啊。

比较起来,今生虽然是仙僵之躯,但经济状况却好了无数倍。

方源为了搜集炼蛊仙材,几乎什么方法都想过了。他发现炼蛊大会,会是一场机遇,可以获得大量的仙材。(未完待续)

\end{this_body}

