\newsection{利益枷锁}    %第一百三十节:利益枷锁

\begin{this_body}

%1
方源收下我力仙蛊,神色不变,只是对黑楼兰道:“今后楼兰仙子若要借吃力仙蛊修行,就拿相应的一份力道仙材作为报酬罢。”

%2
黑楼兰点点头。

%3
这是应有之意。黑楼兰是枭雄,看见方源故意展示吃力仙蛊,便心知方源的意图。

%4
就是不愿破裂,进一步加深合作!

%5
方源强索我力仙蛊,简直就是在两人的联盟上,深深地割开一道血口,鲜血淋漓,深可见骨。

%6
不过,对比珍贵唯一的我力仙蛊而言,方源借出吃力仙蛊,更像是给黑楼兰一个台阶下。

%7
黑楼兰将手中腿骨吃掉,吐出一口浊气,重新看向方源时,脸上已全无一丝怒意。

%8
“走吧,去石巢。”黑楼兰起身,主动提出要帮助方源炼制气囊蛊。

%9
方源朗声一笑,看来黑楼兰的处境也不好过。她此次渡劫,险死还生,被方源救醒,福地中一定遭受了巨大损失。

%10
黎山仙子一直支持着黑楼兰的修行,不计成本。但此刻她也陷入麻烦。全因雪胡老祖急炼鸿运齐天蛊,将大雪山的魔道蛊仙们使唤得团团转,为了炼蛊仙材四处奔波。搜集仙材的任务相当繁重,黎山仙子也只能勉力而为。

%11
方源需要胆识蛊买卖继续下去,黑楼兰又何尝不这么想?

%12
之前方源让利给黑楼兰、黎山仙子,做法极为明智。到此处。终于见着成效。利益也可以形成一道枷锁,黑楼兰就算心中再怒恨,迫于大局和自身情势,也得和方源合作。

%13
当天,黑楼兰催发出大量力气。令气囊蛊的炼制成功重启。

%14
临走前,她又向方源提出合作意向。

%15
眼前,方源没有资金扩大生产规模,但黑楼兰愿意帮助方源出资,建立第三座石巢,补充足够多的毛民。但方源也要对她更多让利。

%16
方源当即应允。

%17
胆识蛊买卖的重新开始,像是一股激流。灌注在快要干涸的河道中。

%18
原本几乎停滞状态下的福地经营。终于又仿佛马车一般,巨大的车轮缓缓向前,滚滚而动了。

%19
数天后,第三座石巢,在方源的狐仙福地中成功建立。

%20
现在方源手中,有数种蛊虫需要大量炼制:气囊蛊、恶念蛊、忆念蛊以及寸光阴。

%21
三座石巢,是远远不能满足需求的。

%22
这便涉及到选择问题。如何才能让有限的资源,最有效的利用起来。

%23
最终,方源令一号、三号石巢,用来炼制气囊蛊。令二号石巢,炼制恶念蛊。

%24
他手头上有着大量的推演计划,因此必须要炼制恶念蛊或者忆念蛊。恶念擅长算计别人,忆念则长于在记忆中挖掘内容。但若用来思考推算,两者并无明显区别。

%25
至于将气囊蛊作为主要炼制内容,是因为方源发现,胆识蛊完全可以卖得更多。

%26
今时不同往日。

%27
胆识蛊的市场。也在慢慢地壮大成长。

%28
尤其是方源不久前,停下胆识蛊贸易,专门用来供给自己治疗伤势。此举一下子断了货源,导致市场上胆识蛊紧缺,结果胆识蛊的价格被炒得很高。

%29
这个现象,让方源明白,胆识蛊的市场已经被刺激起来了。他完全可以提价提量。

%30
他是这样想,也是如此做的。

%31
原先胆识蛊贸易,是一个月一批,净利润为四十八块仙元石。

%32
提价增量之后,不算给黑楼兰、黎山仙子的让利,落到方源手中的仙元石已达九十二块。几乎是在原来的基础上,翻了一番。

%33
“垄断贸易,获利就是如此方便啊。”方源自身境况因此得到了不小的改善,他感慨之余,也在暗暗警惕。

%34
因为胆识蛊的利益越来越大,前景更是越加光明,仙鹤门等其他方面,必然又会蠢蠢欲动。

%35
之前是因为方源展现出了强大的实力,仙鹤门又受到其他方面的制衡,抽不出更多的力量来对付方源。

%36
如今胆识蛊贸易利益增大,总会有一天,使得仙鹤门下定更大的决心,抽出更多的力量,再来尝试“收复”狐仙福地。

%37
匹夫无罪,怀璧其罪。

%38
不管是正道、魔道,都是利益的角逐者。之间的区别,也许就是正道的吃相更好看点。

%39
而蛊师的修行,或者更本质一点人的生存,向来都是一边维护自身利益,一边吞吃他人利益。

%40
方源心知肚明这点,也不否认他自己本身,和其他人,其他势力没有分别。在利益方面,都是一丘之貉!

%41
“要威慑仙鹤门,只有一个根本的解决途径。那就是不断地增长自身的战力,让行动者失败,让心动者震慑。”

%42
怀着这般的迫切情怀,方源在几天之后,成功地将我力仙蛊,融入到仙道杀招万我之中。

%43
我力仙蛊虽然极为契合万我杀招,但方源还是历经数天,微微调整了一些辅助凡蛊。试演之后,收到几乎完美的效果。

%44
“缺少了净魂,但有了我力,如此一来,我的战力就回归到六转一流的程度。甚至万我杀招比之前,还要更强一点。也许是时候探索那道盗天魔尊的传承了。”

%45
方源静极思动。

%46
拍卖会的成果,他消化了不少。

%47
万我重归仙道杀招之威,又有返实蝠翼搭配铁冠鹰力仙蛊,可以媲美仙道移动杀招。而推算见面似相识等等,需要海量的恶念蛊。狐仙福地中,只有一座石巢全力炼制,要囤积足够的数量规模,还需要一段时间。

%48
而如何解决仙僵问题,太白云生虽然在东海那边,有所进展,但鲨魔的方法成本太高,时期太长。

%49
北原僵盟方面,则因为凤九歌强问情报的缘故,导致沙黄身份不能再用,因此计划受阻。

%50
“说起来,我得到这个盗天传承,已经有不少时日了。如果能得到落魄谷,搭配荡魂山,极速增长我的魂魄底蕴,也不失一个增强战力的好方法。甚至将来,也可以做落魄风、迷魂雾的买卖,增长收益。”

%51
与此同时,落魄谷。

%52
一场谈判,已经进行到最关键的时刻。

%53
秦百胜嘴角浅笑,看着眼前的回风子。

%54
回风子一身青袍,低头思索片刻,忽然仰头大笑:“好你个秦百胜,枉我当你是至交好友,结果却把我引到这处陷阱中坑害我。凭你的实力,都不能攻破琅琊福地,莫非是要拿我当炮灰?”

%55
秦百胜目光一闪,淡淡地道:“凭阁下的风遁,进退皆随心意,名冠北原,岂有成为炮灰之理?”

%56
“就算我的风遁,乃是仙道杀招,北原移动第一,但也不能穿透福地洞天。秦百胜你的邀请,恕我不能答应。”回风子摇头拒绝。

%57
他是七转风道蛊仙,成名已久的人物。纵然有绝技傍身,也是心性谨慎。也许正因如此,才能活到今天,有如此成就。

%58
秦百胜点点头,回风子的回答没有出乎他的意料:“既然如此,那我只好强行出手,留住阁下。”

%59
“你要动手?”回风子神情一紧,双目绽放厉芒,“也好,就让我们好好较量一番,也利于打消你不切实际的想法。我回风子不是什么人都能随便左右的。”

%60
话音刚落,他便主动出击。

%61
轰!

%62
回风子猛地张开嘴巴,舌尖一弹,一道巨大的风刃,狠狠击向秦百胜。

%63
仙道杀招亡风飞刃!

%64
“来的好。”秦百胜一动不动,只是笑着。

%65
风刃击中不闪不避的秦百胜,刹那间,金芒大盛。

%66
金芒散去,秦百胜安然无恙。

%67
“这是什么防御杀招?”回风子顿时大吃一惊,震骇地问道,“我的亡风飞刃,居然斩不掉你一根毛发?!”

%68
秦百胜含笑不语。

%69
“这还怎么打?”回风子的眼角狠狠抽搐了几下,斗志降落谷底,退意顿生。

%70
仙道杀招风遁!

%71
他使出闻名北原的招牌手段,顷刻间,回风子化作一股清风,向着落魄谷上空飘摇而去。

%72
此招乃移动杀招,无数次令回风子脱离险境,让敌人惘然无奈地长叹。最著名的一次,是回风子凭借此招,逃离八转蛊仙药皇的追捕。

%73
但秦百胜眼睁睁地回风子撤退,却是停在原地,不动声色。

%74
轰隆隆……

%75
落魄谷的上空,闷雷阵阵,乌云滚滚。

%76
呼呼呼!

%77
大风刮起来,无数魑魅魍魉显出隐约形迹,数以千百万的恐怖数量,在乌云下方翻飞。

%78
回风子大吐一口鲜血,像是一头撞上铁壁,被撞出原身。

%79
“这是什么蛊阵?”他惊疑不定,无往而不利的风遁,居然突破不了这一方小世界。

%80
秦百胜含笑,却不回答回风子的问题,而是问道:“阁下是否再考虑考虑,加入我们,进攻琅琊福地呢?”

%81
这一次,回风子只思考了几个呼吸的时间,便落下身形:“秦百胜,到底是你略胜一筹。也罢,我便加入你的这次行动。”

%82
秦百胜呵呵一笑,也不意外。

%83
回风子是出了名的知进退,乃是识时务的俊杰。

%84
既然退不得,回风子最大的优势就几乎被消弭殆尽。再者,秦百胜提出的联盟条件也并不苛刻,回风子虽然心有些微怨气,但也只能无奈地加入了。

\end{this_body}

