\newsection{第一场炼蛊}    %第一百八十节:第一场炼蛊

\begin{this_body}

沿着大道上了山,就来到半山腰。此处开辟了一个宽阔平台,山壁上刻着三个大字“择徒台”。想来是每年考核入门弟子的资质,然后让长老们挑选弟子的场所了。

但此时,这里已经被粗略地改造。

地砖上布置了简单的蛊阵,隔开一块块地方。每一块,又都标注了数字。

每隔一段时间,前方的人流就走动一段。道路两旁,自有五德门的弟子发放着号牌。

方源领过一张号牌,上面标注着数字六十三。

走到择徒台上,寻找到六十三的位置,直接盘坐在地砖上。

片刻后,广场上人员坐定,一位五德门的长老高声宣布:“本届炼蛊大会入门四题,考核开始!”

下一刻,择徒台上的数百蛊师,纷纷出手,各展其能。

有的手掌上拘着一团火焰,是为火炼之法。有的口中喷出云烟,是为烟炼之法。有的放出数条冰蛇,冰蛇一起昂首,喷吐寒气,这是冰炼。

而方源手中,则是飞舞着数十点星光,宛若一群湛蓝的萤火虫,萦绕飞旋。

依照次序,方源按部就班地放入炼蛊材料。一刻钟之后,他便完成了全部试题,站起身来。

“好快的速度!”主持场面的五德门长老,早就得知方源五转蛊师的身份,一直在关注他。

尽管如此,当方源站起身时。这位长老的脸上仍旧流露出惊异之色。

方源完成的速度,实在太快了。期间没有一次失误,手法更是熟练至极。

“什么,居然已经有人完成了?!”方源的举动,让场中不少炼蛊的蛊师侧目。

“是那位神秘的五转蛊师……”老师傅位于场外。看到方源,瞳孔微缩,旋即他又将担忧的目光投向自己的徒弟,心中祈祷,“好徒儿,你可不能因此分神,被干扰了啊。”

好在他的那个小徒弟。一直埋首炼蛊。神情极为专注,没有发现方源这边的动静。

方源走上高台,高台上五德门长老立即离开座位,亲自检查方源手中的四只蛊虫。

“没有任何负面的状态,仿佛是喂饱了的蛊虫,状态稳定极了。”五德门长老暗自心惊,检查完毕之后。向方源拱手,温声道,“这位大人,请走这边上山。”

方源点点头,缓步走去,离开了择徒台。

“不可能吧?”

“有人居然通过了,用了不到三分之一的时间……”

“好强的炼道造诣!”

场外众人嘈杂议论,就连场内正在炼蛊的蛊师们,也有数位,因为太过惊讶。心境动荡,导致手头上的炼蛊失败。

方源此次参加炼蛊大会,无须遮掩身份。本身又是蛊仙,没有藏拙的必要。

他走上上山的路,不一会儿,就有一位五德门的长老迎接上来:“阁下请暂且留步……”

长老刚开口,方源就摆手。打断他的话:“我知道你的来意。我没有这个想法,区区五德门还容不下我。”

长老楞了一下,脸上怒色一闪。

他此番起来,就是带着门主的命令,试图招揽方源,邀请他成为门中客卿的。

事实上,五德门作为炼蛊大会的报名点之一,主要意图之一就是招揽人才。

本来方源的修为,表现出来就是五转,让五德门上下极为关注。方才展现出的炼道造诣,让五德门门主都为之惊异,这样的人才怎么可以放过?于是便命令长老,上前商谈。

结果这位长老还未道明来意,就被方源一口拒绝。

被方源无情拒绝,这位长老好不尴尬。

“你还有何事?”方源冷瞥一眼。

长老气得面皮紫涨,冷哼一声:“阁下居然看不起我五德门,未免太妄自尊大了点吧?”

“你是想和我赌斗?”方源声音冰冷。

长老怒色一滞,深深地望了方源一眼:“好好好,阁下自命不凡,想来一定能在炼蛊大会中取得头名啊。我五德门上下,必定拭目以待!”

说完,拂袖而走。

方源不以为意,继续往前,转过一处山崖,便见到木德殿。

这座大殿隐于山林之间,通体翠绿,和周遭草木相得益彰。

方源步入大殿,殿中有桌椅无数,还有五德门精英弟子严阵以待。见方源出现,其中一位弟子连忙迎上去:“这位前辈,请问您的名字,何门何派,主修的蛊师流派。”

方源沉声道:“姓方名源,门派保密,主修保密。”

“呃……好的,前辈,请您缴纳一百块元石的报名费。这笔费用将用于……”

这位弟子还未说完,方源就将准备好的元石丢了过去。

弟子手忙脚乱地接住,客气道:“请稍坐片刻。”

不一会儿功夫,方源便得到一块令牌。

令牌正面写着方源两个字,令牌后面则是一排小字:门派保密,流派保密,于五德门报名。末尾,还有报名的时间。

“这位前辈,请您收好这枚令牌。若是丢失,请尽快补办,否则是无法参加炼蛊大会的。请您继续往山上走,火德殿中有第一场炼蛊大比。前辈只有通过这场大比,才有资格参加第二场。”这位弟子解释的很有耐心。

方源拿着令牌,离开木德殿。

至此,报名才算结束。

中洲炼蛊大会表面上是由十大古派联合举办,暗地中则是有天庭默默扶持。

炼蛊大会规模盛大空前,报名点有数百个,遍及中洲各处。其中大多数。就是类似五德门这样的门派,实力比较雄厚,亲近十大古派,或者干脆是十大古派利益的代言人。

方源想要参加炼蛊大会,就得在这其中的任何一个报名点。当场炼蛊,通过入门四题,才有资格报名,获得令牌。

因为参加炼蛊大会的魔道蛊师也有很多,甚至还有其他四域的蛊师慕名而来,蛊师报名时是很随意的。姓名可以随便起,流派、门派不想说就不说。重点在于令牌本身。

还有一群比较特殊的蛊师。

这些人所属的门派或者本人。在前一届的炼蛊大会取得过名次,因而可以直接参加炼蛊大会。

比如洪易,就是这种情况。

他所属的众生书院,在百年前的炼蛊大会中,取得了靠后的名次。这次众生书院中便有三个名额,可以免试参加炼蛊大会。

值得一提的是,这种免试待遇只有中洲蛊师。以及中洲门派才有。

毕竟这场炼蛊大会,是由中洲十大古派举办,因而倾向于中洲本土。

当然,仙鹤门中有大量的免试名额。但是方源只是挂着个仙鹤门附庸的名号,真要去讨要这个名额,可不容易。而方正则是另一种情况。

门派的名额,一部分划分给弟子、精英弟子,另一部分则是划给众位长老们。

方正成为仙鹤门的长老,自然不能和弟子们争夺名额。而他自己本身炼道造诣十分低微,也抢不过那些长老。

参加炼蛊大会。仙鹤门自然要将名额,发放给那些炼蛊造诣深厚,能够撑起仙鹤门脸面的蛊师了。

因而,不管是方源还是方正,都得自己去报名。

方源拿着令牌,走上山,进入火德殿。

这是炼蛊大会的大比斗的第一场。

如果成绩太差。是没有资格参加第二场的。

方源步入其中,得知第一场的炼蛊内容。

“要在一炷香内,炼制出一转蛊虫水光蛊,数量至少有一百只,才算合格,可进入第二场考验。一百五十只,可得奖励十斤虚影花瓣。一百八十只,可得奖励无根水六盆。两百只,就有雷击木十根奖赏……两百四十只是第三名,两百六十可名列第二,两百七十六只将名列第一,得四转治疗蛊虫,擅长解热毒的绿曜蛊。”

方源心中思量起来。

这第一场炼蛊的难度,顿时就比前面的四大入门试题,骤然高了许多。

不过,应该难不住大多数的蛊师。

因为只要炼出一百只水光蛊,即可通关,获得继续下去的资格。但这些奖励,对于那些蛊师而言,的确是诱人。

第一位就是四转蛊虫绿曜蛊。

四转蛊啊,也就是大型势力中的长老、家老,乃至小型势力的首脑,往往才拥有的蛊虫。

对于方源而言,绿曜蛊可以在宝黄天中论群来买。

方源的层次太高,这些奖励在他看来,几乎没有任何的吸引力。让他觉得有意思的是,是题目带给他的挑战。

“炼蛊的基本材料已经限定好,可以无偿使用。水光蛊的蛊方也给出,可以随意查看。但就算是我全力出手,动用水炼之法,最多也只能炼制出两百五十多只。这个成绩只能排名第三。但在五德门这里,第三名已经被人夺走了。”

这种题目,也讲究先来后到。

一旦前三名被人夺走,就算后来人成绩再好,也是无用。

也就是说,方源要榜上有名,就得夺取第二名,或者第一名。

“其实要夺第一,也不是没有办法。首先,我可以增添其他名贵材料,增加蛊虫产出。水光蛊只是一转蛊,我动用二转、三转的炼蛊材料,必然能增产。不过这笔费用,得我自己出。其次,我可以改良蛊方,我有智道传承,稍微改良一下这种一转蛊方,不用智慧蛊,自己都能搞定。不过这样一来,岂不是没有挑战性了吗?乐趣也就丧失了。嗯……或许我可以挑战一下水火炼法。水光蛊是水道蛊虫,通常都用水炼,或者冰炼法。但我反其道而行之,动用火炼,或许能有奇效!”(未完待续。)

\end{this_body}

