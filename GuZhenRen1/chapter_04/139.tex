\newsection{图穷匕现真内奸}    %第一百三十九节:图穷匕现真内奸

\begin{this_body}

“好了,余亮,你还是去完善你的蛊阵,只有闯过这最后一关,你才能彻底地继承我本体的衣钵传承。至于你,残阳老君,今日就授首于此罢。”东方长凡的星意,缓缓地道。

残阳老君稍稍后退一步,冷傲地道:“哼,我虽然势单力孤,不过你们只是一群六转,再加上区区一股意志,也想取走老夫的性命?你们可别忘了,我那同伙就在你们当中。更还有外面的魔道蛊仙,如狼似虎,你那些虚兽抵挡不了多久的。”

他话是这样说,脸上已经满是戒备谨慎的神色。

活到他这般年岁,取得这样的成就,谨慎的性格优点是绝少不了的。

残阳老君的话也是事实,句句直至东方部族这一方的要害。

也正是因为如此,东方家的蛊仙,虽然人多,却没有鲁莽动手。

东方长凡的星意呵呵一笑:“远古时代,开创智道的星宿仙尊去世,能算尽身后百万年。令其后三位魔尊无功而返,保住天庭。我东方长凡,虽然远远及不上星宿仙尊,但是身后百年之事,还是能稍稍算得到。我既然布下此局,自有底牌手段。”

话音刚落,大殿中便浮现起一道虚幻的蛊阵。

无数的蛊虫,盘旋飞舞,像是万千蜘蛛丝线,交织密布在大殿当中。

和这个忽然出现的大阵相比,东方余亮完善的蛊阵,简直就是大人脚下的婴儿。

残阳老君见此。面色顿时难看起来:“这是虚阵!想不到你东方长凡,在虚道上也有如此深厚造诣,竟然能结成虚阵!”

通常的蛊阵,一旦暴露,就容易遭受打击破坏。

皆因组成蛊阵的蛊虫,都是实体。

而蛊虫本身十分脆弱。就算是仙级的春秋蝉,只消方源两个手指头轻轻一捏,就会破碎毁灭。

但这虚阵,乃是运用虚道的上等手段,将组成蛊阵的蛊虫都一个个虚化。却不妨碍蛊阵的运转。

虚兽之所以能阻挡方源等魔道蛊仙这么久。就是因为虚兽天生能够虚化。一旦虚化,什么招数都没有效果。

星意大大方方地展现出蛊阵,丝毫不怕残阳老君出手破坏,也是因为蛊阵虚化。残阳老君只能干瞪眼地看着。却无可奈何。

蛊阵渐渐运转起来。大殿中充满了刺眼的玄光。

“哈哈哈。”残阳老君却陡然大笑起来,“我对虚阵无法可想,那就直接杀死你们好了!”

“大家小心!”

“他要杀过来了!”

“要注意内奸。千万别大意。这是我族生死存亡的关键一战!”

“不错,只要闯过这一关,就是海阔天空。”

东方部族的蛊仙们,早已经严阵以待,全神贯注。

“诸位稍安勿躁,短时间内,他还杀不过来。”东方长凡的星意却是平淡安然。

下一刻,残阳老君打杀过来,动作却极为缓慢。

他发出的火焰,原本速度奇快,但此时却慢如蜗牛。

“这是?”残阳老君很快发现,并非自己慢了,而是自己面前的空间,蕴藏着肉眼难以观察的奥妙。

看似很短的距离,却是空间的浓缩。

“我耗费数十年,无数心血,设立此局。这处大殿看似寻常,和周围其他殿堂一般无二。其实我早就将墟蝠尸体中绝大多数的宇道道痕,都集中在此处。我们和残阳老君看似只有数十步的距离,其实真正的距离,远远超过千里。”星意缓缓地道。

“真不愧是东方长凡大人!”东方部族的蛊仙们这才释然,一个个双眼发亮,怀着无比的敬佩之色看向星意。

星意点点头:“现在才是关键。我实话告诉大家,这现出的大阵,乃是一个大杀阵。以我智道仙蛊为核,血道仙蛊为辅,一旦发动,威能绝大,有崩天裂地之威。尤其是掺杂了血道仙蛊后,对付蛊仙更有诡奇成效。只要蛊阵催动起来,不仅能杀掉残阳老君,而且还能杀崩上空的魔道蛊仙。此战之后,我东方一族必能名扬北原,令正魔两道,各大势力忌惮谨慎。待此战结束,你们将这个大阵搬回家族的福地里去。余亮若是通过我的传承,他便是第一执掌之人。”

东方部族的蛊仙,听到这番话,顿时振奋非凡。

反观残阳老君,脸上难掩惊疑不定之色。

星意继续道:“我本体生前,虽然和正道超级势力都结盟,定下盟约,但这个世道,只有自己真正掌握了力量,才能确保安全啊。你们都要牢牢记住,不要把无谓的希望,寄托在他人手中。”

东方家的蛊仙们纷纷点头,表示受教。

东方一空则急道:“那长凡大人,我族之中还有内奸。千防万防,家贼难防啊!”

“别急。”星意笑了一声,“我还未说完。凡事有利有弊,有得有失。这蛊阵威力奇大,但却有重大弊端。催发此阵,需要抽调蛊仙身上的仙窍本源。”

“什么?”许多东方蛊仙顿时失声。

仙窍本源,乃是仙窍的根基、根本。

一旦抽调出来,好不容易经营起来的仙窍,就会发展倒退,空间萎缩,元气稀薄,万物凋零。

而蛊仙不进则退,时间一到,仙窍就有着天劫地灾的重重磨难。

仙窍亏空,蛊仙实力就弱,稍有不慎就会遭灾而亡。

性命攸关,东方家的蛊仙们自然要失态。

但很快,就有一位蛊仙站了出来。

正是东方万休。

殿中众人只听他道:“有失才有得,长凡大人辛苦筹谋。好不容易设下此局。我东方万休为了我东方一族,做这些牺牲又有何妨呢?”

“好!不愧是万休大人,我东方一空也奉陪到底!”随后,东方一空也紧跟着站了出来。

星意淡淡含笑,沉默不语,看着其他人。

其他蛊仙相互对视,很快都是点头,决定为家族斩除内外强敌,而做出牺牲。

“不忙。”星意却摆手,慢悠悠地道。“我还要告诉诸位一点。一旦入了蛊阵。开始抽取仙窍本源,诸位就动弹不得,无法强行脱离了。尔等生死,皆操于我手。你们可仍旧愿意吗?若是不愿。自可提出。”

几位蛊仙面色再次微变。

东方万休眼蕴神光。四下乱扫。冷笑道:“内奸你还是站出来吧,再不出来,可就来不及了。”

星意的话。明显是说给内奸听的。

只要稍有思考能力的,都能想见:若真有内奸,听了这番话,心理压力一定巨大极了。一旦入阵,就生死由人。万一打杀残阳老君的时候,后者情急之下,暴露了自己,那简直是连还手之力都没有了!

若是内奸,心思肯定两样,不想对付残阳老君。也绝不肯将自家生死,交托在星意手中。

然后,几个呼吸之后,大殿中的东方蛊仙仍是没有一人跳出来。

“好。到此处,我万休也不得不佩服你了。难怪你能潜伏在我们当中,这么久这么深。”东方万休恨声道。

星意轻笑一声:“既然如此,那你们就入阵罢。”

虚化的蛊阵打开八个阵眼,东方一族的蛊仙们齐跃进去,分别占据一个方位。

蛊阵开始缓缓地运转起来,蛊仙们脸色纷纷变化,皆感觉仙窍震动,从大阵中传来有一股吸摄之力。

“放开心神,配合大阵,让其全力运转。”星意肃容道。

蛊仙们有些稍稍犹豫,但最终皆主动开放仙窍,立时仙窍本源被抽调而出。从八个阵眼的位置,流淌出去。

虚化大阵中,很快浮现一片淡淡的血光。

在血光的作用下,这八股不同的仙窍本源,在蛊阵中央融汇,竟然真的渐渐相融为一体。

“这大阵好生玄妙!”有蛊仙见此,忍不住赞叹出声。

“这是因为你等体内,都留着东方一族的血液。血脉相近,有着同一个源头。因而血脉相融,带动仙窍本源的融合。只有融合起来,才会将数量的优势发挥出来,迸发出强大的威能!”东方长凡的星意,缓缓解释道。

阵中八位蛊仙纷纷点头,东方万休眼放精芒,四下扫射,却仍旧看不出内奸露出什么破绽。

“无妨。”星意轻笑一声,“如此局势,内奸不管是谁,已入我局中,为我族出力。诸位放松,我要加大抽取仙窍本源了。”

“尽管抽吧!”

“好,待会杀崩这些可恶的魔道蛊仙!居然敢觊觎我东方家族!!”

虚阵光辉大盛,蛊仙们接连冷哼,脸色浓重,有的双眉皱起,感到难受和痛楚。

仙窍的本源,在虚阵中渐渐融汇成庞大的力量,仿佛一团水液,又如同蛰伏沉睡中的猛兽。任是谁,都能感受到这股力量的庞大,就连周围的空气都产生了微微涟漪,仿佛要承受不住。

东方部族的蛊仙们,有的满脸通红,有的身躯微微颤抖,有的脸现狰狞,俱都期待着这股力量大发神威。

但这股力量,在星意的操纵下,引而不发,蠢蠢欲动。

“不忙。”星意仰望天空,关注着半空中的混战。

阻挡魔道蛊仙们的虚兽,已遭屠戮,几乎一空。

星意就像是一头蛟龙,隐藏在江水深处,在等待最好的时机。

时机一到,蛟龙升天,就要翻江倒海。就看是哪个倒霉鬼冲在前头了。

就在这时,东方余亮猛地站起身来,大喊起来:“师傅,我做到了,我成功地完善了这个蛊阵,我达到了您的要求!”

星意迅速扫视一眼,脸上露出欣慰的笑容:“很好。余亮,你没有令我失望。你也入阵,站到中央的那个阵眼里。你虽然是凡人,但将来是要晋升成仙的。待会开战,你提前领略一番蛊仙的力量,对你未来的成长大有帮助。”

“是,师傅。我都听您的!”东方余亮双眼通红,感动得有些哽咽。

他纵身一跃,跳到半空中,随即在虚阵的力量下,站稳阵眼,悬浮在半空中。

半空中,在自在书生、皮水寒的严重干扰下,方源终于找到一丝机会。

眼前一片空白,毫无阻拦的力量,机会稍纵即逝,他立即紧紧抓住,舍身杀下!

七只力道巨手狠狠拍下,只余一只护卫自身。

“是时候了。”星意见此,轻声一叹。

东方部族的蛊仙们,俱都兴奋嗜血地望上天空。

下一刻,他们看到残阳老君冲天而起,挡住方源的突击,将后者重新打退回去。

“嗯?!”这变化,让东方一族的蛊仙们都是神情呆愣。

“不枉费我费劲心血,布下此局。夺舍重生,就在这一刻!”星意昂然奋发,大笑起来。

东方长凡的魂魄忽然出现,一头扎入虚阵中央,进入东方余亮的体内。

“哈哈哈,一群傻瓜!实话告诉你们,东方一族真正的内奸,就是你们所敬爱的长凡大人啊,哈哈哈哈。”残阳老君狂笑不止。(未完待续……)

\end{this_body}

