\newsection{巨祸焚木}    %第九十二节:巨祸焚木

\begin{this_body}

要铲除这些树人,的确要耗费海量仙元。

这些树人的数目,比第一次地灾时的雪怪还要多,并且它们恢复能力极强。方源已经看到不少的断枝,插在土壤中,眨眼间就能冒出新芽来。

除非用剑浪三叠这种杀招,彻底将这些树人泯灭,这才能消除后患。

否则这些树木生长起来,可是要耗费仙窍中大量的地气的。

但用剑浪三叠,或者飞剑仙蛊,铲除这些树人,是太过大材小用了!

“若我是炎道蛊仙,眼下局面就不是什么难题了。”方源正想到这里,忽然听到蓬的一声,从那些倒下的树人尸体中,忽然升腾起一股火焰。

火焰迅速燃烧,眨眼间,就蔓延开来,将无数树人笼罩!

火焰附着在这些树人身上,熊熊烈火,不断燃烧,将这些树人烤成焦炭,树叶消融,很快就死伤一片。

火焰得到树人的滋润,越加旺盛,苗头高得离地足有五六丈!

滚滚热浪,朝方源扑面而来。

方源愣住。

这是怎么回事?

很显然,这同样是地灾的变化。

但为什么地灾如此变化,居然自己燃烧自己。天意到底想要干什么?帮助方源解决麻烦吗?

方源迅速否定了这个念头。

开玩笑,天意怎么可能帮他?

吼!

冲天的火焰中,蓦然站起一个巨大的身影。

橘红色的火焰里,出现了一个高达十八丈的巨人!

“这是?!”方源瞳孔猛地缩成针尖大小,下意识地咬紧牙关,“巨祸焚木!”

这是一株上古荒植。

根据历史记载,太古赤天常有这样的巨祸焚木。

但这种上古荒植,在五域中很少见,但算不上绝迹。

巨祸焚木的前身,可以是天底下任何一种树木。当发生森林大火,烧死无数生命之后。就会有一定概率,形成巨祸焚木。

巨祸焚木浑身上下,都缭绕着炙热的火焰。它终年燃烧,让万物都难以接近。更厉害的是。这种巨祸焚木还会给靠近它的生灵,带来无数的厄运祸端。

“好个天意,真是阴险!显然我有****运仙蛊护身,让天意看破了这点,所以酝酿出巨祸焚木。坏我气运!”

方源感到巨大挑战。

换做正常情况:他若在野外,遇到巨祸焚木,必定明智远离。但现在不一样,这巨祸焚木直接出现在他的至尊仙窍当中,不仅抽取仙窍地气,而且还带来灾祸和厄运,抵消他的****运。若坐视不管,一段时间之后,方源就会厄运缠身,事事不顺。甚至大祸临头。至于炼制仙蛊什么的,想都别想了。

火焰冲天,烧得焚木噼啵作响。[棉花糖小说网www.Mianhuatang.cc想看的书几乎都有啊,比一般的小说网站要稳定很多更新还快,全文字的没有广告。]

焚木出现的位置很巧妙,就挨着荡魂山。

纵然方源撑起防护手段,但仍旧被高温炙烤得眉毛、头发都微微打卷。

“这才是最外围,就有这样的温度,更何况焚木的表面?”方源叹息一声,抽身而走。

他直接离开荡魂山。

反正他有江山如故,荡魂山只要被彻底毁灭,总能复原。

现在的难题是如何灭掉这株。差不多顶到荡魂山山腰的巨祸焚木!

剑痕索命!

剑浪三叠!

毒气喷吐!

方源接连试招,效果都不理想。毒气还未接近,就被炙热的火焰烤个干净。飞剑仙蛊倒是穿透整个焚木树干,又飞回方源手中。这一击。让整个巨祸焚木狠狠地颤抖了一下,但很快就恢复如常。

至于剑浪三叠,倒是扑灭了火焰,冲到焚木面前,削掉了很多枝叶。不过很快,漫天的火焰又重新恢复。焚木迅速复原,伤势在几个呼吸之后,就消失无踪。

方源眼中精芒爆闪,周身仙气狂涌而出,口中低呼:“力道大手印!”

轰隆一声巨响。

一只巨手凭空而出,排开空气,带着排山倒海的气势,狠狠地抓向焚木。

但焚木体型也很大,几乎是小半个荡魂山。

力道巨手猛地抓住巨祸焚木,想要将它提起来。

但方源努力了半天,却没有良效。

巨祸焚木的无数树根,宛若蛟蛇龙蟒,深深地扎进地中,凶猛地汲取地气,还依靠天空中飞舞的只只春晓翠鹂,不断补充生机。可谓是稳如泰山,岿然不动。

时间拖得长了,整个力道巨手居然在火焰中开始融化,很快就仿佛消融了一大半。

方源索性撤掉这记大手印,心思一转,又有一计。

轰隆!

第二只力道大手印飞出,这次却不是抓向巨祸焚木,而是荡魂山。

方源提不动巨祸焚木,但力道大手印中可是有拔山仙蛊的威能,所以能提起荡魂山。

呼!

风声骤急,荡魂山被方源直接拔到空中,直接朝着巨祸焚木扔去。

巨大的声响,整片大地都狠狠地颤抖了一下。

巨祸焚木被压在山底下,原本冲天的火焰也再无张狂气焰,只在山脚下伸出一头,围绕山脚,形成一个巨大的火圈。

方源悬浮于空,面无表情,俯瞰脚下战场。

巨祸焚木已经无法翻身,并且饱受荡魂山上的魂道道痕的折磨,但它生命力极其惊人和顽强,居然还在抵抗和挣扎。

炙热的火焰开始烧烤荡魂山,将一些部分的山石,都烧成了软泥。

要彻底除掉巨祸焚木,不是这么容易的。

方源心头沉重,放眼望去,视野中一片赤红。

火海!

火焰已经蔓延到全部的森林,还有草地。

果然,天意之前的建设,不过是为了巨祸焚木的铺垫而已。哪里有那么好的事?

方源鼻息微微粗重起来。

原来他看到,一株又一株的巨祸焚木,从火海各处成形,无数的树根深深地插进大地之中,汲取地气,壮大自身。

之前的那株,还未除掉。眼下又有了这么多!

不仅如此,这些巨祸焚木竟然也发出咆哮之声,一个个站起起来,形成巨大的焚木树人!

它们的目标只有一个。那就是高空中的方源。

轰轰轰!

荡魂山底下,第一头巨祸焚木也变化成树人,很不安分,企图推翻身上的巨山。

它的力气惊人,每一次挣扎。都让荡魂山猛烈的震荡起来,发出令人心惊肉跳的巨响。

一瞬间,情况恶劣到了无以复加的地步!

之前的雪怪,方源还可以暂且不管它,徐徐收拾。

但眼下这些巨祸焚木树人,可不能不管。它们虽然行动缓慢,但对仙窍的恶劣影响太大了。不仅时刻汲取地气,损耗至尊仙窍中的底蕴,而且还会坏了方源的气运!

该如何是好?

火光映照在方源的脸上,他原本平静的脸上。忽然浮现出一丝微笑:“幸好,我还有后手!”

与此同时,远在南疆的商心慈,也痴痴地看着眼前的灯火。火焰中,似乎出现了方源曾经的面容。

“黑煞哥哥……”商心慈心中呢喃。

“族长,族长大人?”耳畔传来轻声的呼唤。

“啊。”商心慈回过神来,看到身边的小兰、小蝶,还有卫夫人、周全等等下属。

“抱歉,我又走神了。”商心慈连忙道歉。

“小姐,这已经是你第三次走神了。”小兰嘀咕着。

“是我失礼了。实在抱歉。”商心慈连忙再道。

周全不动声色地和卫夫人对视一眼,旋即咳嗽几声:“那么接下来,我们还是继续刚刚的话题,如何收回商家那几位曾经的少族长。所控制的蛊师队伍。”

议事结束之后,众人离开大厅。

“小蝶,族长她最近是怎么了?怎么总是魂不守舍的样子?”周全问道。

小蝶一脸担忧之色:“我也不太清楚。似乎是族长得到上面召见,回来之后,就变成这样了。”

“上面?”周全面色一变,他现在也知道了商家蛊仙的存在。事关蛊仙,那么这事情就不是他能处理得了的。

叶凡终于忍耐不住:“我去找心慈小姐谈谈吧。或许我这个外人,能够有所收获呢。”

“叶公子可不算什么外人。”小兰笑道。

“那就有劳叶公子了。”卫夫人点点头,认可了这个提议。

片刻之后,叶凡来到商心慈的书房。

书房的门,虚掩着。

叶凡透过缝隙,看到商心慈双手拿着一张薄纸,细细查看。

她看得是如此入神,以至于叶凡故意放重脚步声,她都没有察觉。

叶凡正要咳嗽几声,提醒商心慈他的到来。

但旋即,他听到商心慈的低咛:“黑煞哥哥,你究竟是个什么样的人……”

叶凡心神剧震,他瞬间明白过来,商心慈手中的薄纸,不是别的,正是黑煞的通缉令。上面正是黑煞的画像!

叶凡一时间感觉心中空虚至极,原本强劲的四肢,都有点虚不着力。

他站在门外,犹豫了良久,终究咬咬牙,默默转身,悄悄离开。

至始至终,商心慈都没有发现他。

叶凡出了城主府,来到街头。这里是商量山内部,商家的重地,灯火辉煌,繁华似锦。

但叶凡却面色忧郁,他心中也在问一个问题黑煞,你究竟是个什么样的人?

北部冰原。

楚度背负双手,观赏着眼前茫茫冰原。

忽然,他雄躯一震:“来了!”

他接到了方源的传讯,连忙催动手中仙蛊。

七转仙蛊招灾!

下一刻,至尊福地门户大开。

先是一只只春晓翠鹂被无形的力量,通过门户,摄出福地。然后是一株又一株的巨祸焚木,被无法反抗的玄妙力量拽了出来!

它们不甘,它们愤怒。

天空中响起阵阵闷雷,天意也在恼火。

但没有用!

大约一个时辰过后,第三次地灾终结。

方源收起仙窍,出现在北部冰原之上。

楚度得到不少狂蛮真意灌注,心情激动得长啸出声:“长路漫漫几个秋,今朝才得青云佑。青云托我瞰江湖,天地方圆一览无。”

啸声激荡天地,响遏行云,霸仙豪气尽显。

“敢问阁下姓名?”楚度向方源一礼。

方源微笑,亦吟道:“曲折途穷天地窄,重重灾劫生死微。身如柳絮随飞扬,无论云泥意贯一。”

顿了一顿,他向楚度回了一礼,淡然道:“在下柳贯一。”(\~{}\^{}\~{})

\end{this_body}

