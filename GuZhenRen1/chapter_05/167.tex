\newsection{疯魔之约}    %第一百六十七节:疯魔之约

\begin{this_body}

数天之后。[看本书最新章节请到棉花糖小说网www.mianhuatang.cc]

北原,疯魔窟。

“柳贯一,我们等你很久啦!”不是仙主动来到第一层,接应方源和楚度。

三人一边交谈,一边往更深处前行。

方源发现,这一次的路线和他上一次前来时,又稍稍有了不同的变化。

“难道最近的魔音,很是频繁吗?”方源开口询问道。

“的确是越来越频繁了。所以我们前进的路线,也要跟随时常变动。魔音一起,引发道痕错乱。不仅会让荒兽疯狂,也会改变地形地貌。”不是仙解释道。

片刻之后,他们见到了胖山。

胖山从沉眠中醒来,睁开一丝眼缝:“又见到了你,楚兄。”

他和楚度的关系,似乎很好。

随后,他又看向方源:“柳兄,你的本事我已知晓,定下疯魔之约后,咱们再细聊。”

方源便问:“如何和你们定下疯魔之约?”

“那就要下去,进入倒数第三层。我行动不便,就不陪你们去了。”胖山语焉不详,解释了等于没解释。

“随我来吧。”不是仙继续带头领路。

三仙继续往下跋涉,越是深入,路线越是曲折变化,有时候明明直线距离很短,但偏偏要绕上一个巨大的圈。

“相信我,在这里绕远路,反而比走直线更加快捷。”不是仙笑着道。

楚度也点头,指点方源:“待会儿到了下去,可别太惊讶。”

方源不免心生出更多的好奇和期待。

半晌之后,通过一个洞窟,方源正式踏足疯魔窟的倒数第三层。

视野中,一片光华璀璨。

各种颜色,五彩缤纷,从石头上、泥土中、草木上莹莹放光。

方源倒吸一口冷气:“这,这都是道痕光晕!唯有准九转的仙材,才能绽放出这等道痕之光。这里的道痕竟然如此浓郁。每一片土壤,每一块石壁,岂不都是准九转的仙材了?!”

不是仙哈哈一笑:“柳兄这么说,也有道理。不过这些仙材道痕错乱。是不能用的。”

楚度看着方源吃惊的脸色,亦笑道:“当初我来到这里的时候,也和你一样吃惊不小呢。无弹窗,最喜欢这种网站了,一定要好评]”

不是仙微微收敛起脸上的笑容:“这里的道痕浓郁至极,已经不单单只是散发出道痕之光那么简单!而是形成了一种类似于战场杀招的现状。在这里行走,我们将举步维艰。柳兄。你不妨试试看?”

方源连忙摇头:“不是仙你既然这么说了,我自然信你。我就不用试了。”

不是仙望向楚度,微笑道:“柳兄你可比某人聪明多了,想当初,某位却是不信邪,硬要往里面冲。我苦劝不住,结果让某人吃了一个闷亏呢。”

楚度摸摸鼻子,对方源坦诚道:“不是仙说的就是我。那一次受伤,我足足养了三年才彻底恢复。切不可小看这片道痕之地啊,这里堪称是层层险阻。步步危机!教你一个窍门,尽量往自己修行的流派道痕上靠拢,那样行走跋涉,阻力最小不过。”

不是仙点头:“正是此理。我乃律道蛊仙,要尽量走有律道道痕的地方。”

“而我主修力道,却是要靠拢这里的力道道痕。所以接下来的路,我们帮不了你,要靠你自己走了,柳兄。”楚度关照道。

方源点点头。

不是仙率先迈步:“我先走,柳兄可以好好瞧瞧。”

方源便聚精会神观看。

不是仙每迈出一步。都是细心谨慎。起先,十步他走得轻松自如,动作也透着潇洒。但十步之后,就艰难起来。二十步后。动作开始僵硬,仿佛无形的压迫力量,重重地压在他的身上。三十步后,不是仙出现明显的汗渍。四十二步,他首次停下来,气喘吁吁。回首对方源道:“柳兄,你看清楚了吗?”

楚度哈哈大笑:“不是仙!你别装,你明明就是支撑不住了,还故意询问柳兄,遮掩自己的窘迫。”

不是仙被当众揭了短,羞怒地道:“好你个楚度,我倒要看看你能一口气走多少步路?能否有我的这般距离!”

楚度冷哼一声:“我怕你不成,你好好看着!”

话音未落,他就迈开大步,走入道痕之地。

起初,他的速度比不是仙还要快,龙行虎步,看得不是仙脸色一沉。

但很快,楚度忽然一步踏错,道痕之光在他脚下猛烈一爆,楚度闷哼一声,雄躯狠狠一颤,一口鲜血直接当场吐了出来。

遭此重创,他气势顿消,没有停歇,楚度强忍伤势,接下来一步一个脚印,速度却是慢如蜗牛。

最终,他走出了三十八九步,第四十步的时候,终于无力为继。

一口气泄了。

停了下来。

“终究还是比不过你。”楚度擦了擦嘴角的血迹,叹息一声。

不是仙已经足够惊讶,他瞪圆双眼,看着楚度:“好你个霸仙,比较之前,足足进步了这么多!惭愧,我久居疯魔窟,行走这里的次数比你多出数百次,占了这么大的便宜,也不过一口气才能走出四十二步而已。”

楚度嗤笑一声:“得了,别抬举我。我知道你没有动用全力,是想给我留面子吧?”

不是仙也不否认,嘿嘿一笑:“即便如此,你的进步也远超我的想象。你还真是个修行力道的天才,霸仙……盛名之下果无虚士。”

楚度朗声一笑,摆摆手:“好了,咱们别相互吹嘘了,接下来就欣赏柳兄的表现。”

两仙微微侧身,转过头来,目视方源。

方源点点头:“待会我露丑了,可别笑话我。”

“柳兄且放宽心。你初来乍到,主要是体会踏足这里的感受。别执着于什么步数。”不是仙说道。

楚度却笑道:“哈哈,你放心,柳兄,你的狼狈我一定会记在心中。我第一次可是一口气走了二十二步,你若不及我的话,我会笑话你一辈子的。哈哈!”

方源也跟着笑起来,氛围十分和谐。

但实际上,他的心中却是一片了然:“这两人是故意试探我呢。”

“虽然楚度举荐了我,但是对我探索这里的能力,还是没有把握,所以故意安排了这一着。”

“他们一唱一和,故意叫我单独行走,就是想探探我的底。”

“若是我表现太差,对他们没有什么帮助,说不定连疯魔之约都定不下来!”

事关无极魔尊,又关乎永生之秘,方源下定决心,定要走出一个好成绩,向楚度和不是仙展露出自己的实力来。

方源眼蕴神芒,深呼吸一口气,向前迈开一步,然后稳稳当当地落在地上。

他选择的这块地方,自然是积蓄着变化道痕。

下一刻,方源脸色微变。

“哈哈。”楚度察言观色,几乎在方源脸色变幻的同时,他笑起来,“柳兄,光看是没有用的。这下你感受到了其中的压力了吧?”

方源心中怪异无比,但脸色却丝毫不漏出什么破绽出来。

他没有回应楚度的话,而是闷声不吭,继续前行。

一步两步三步……就这样方源连续走了十几步。

方源心中的怪异感受,越发浓郁:“怎么回事?为什么我丝毫没有感受到任何的阻力?我已经走了十多步,步步向走在普通的平地上。”

这明显很不正常。

方源很快就猜想,发生这种情况恐怕是因为他的至尊仙体。

“楚度、不是仙他们,之所以承受压力,是因为道痕之间相互排斥。这里道痕众多,而蛊仙本身也蕴藏大量道痕。”

“所以尽管他们一步步,都走在满是律道或者力道的道痕上,但周围的道痕,近在咫尺,仍旧会排斥他们,压迫他们。”

“但我却不一样。这里的道痕,不是仙道杀招,没有杀伤能力,对我影响根本没有多少!”

“当然,这一切都只是我的猜测。真正的原因,或许不是这样。”

方源一边思考,一边行走,又走了好几步。

不是仙、楚度的脸色都发生了微微的变化。

“这个柳贯一,好强!”不是仙心中震惊不已。

“他已经快要走到二十步了,还如此云淡风轻,我当时第一次走,几乎是步步泣血,竭尽全力,狼狈至极!”楚度回忆起自己曾经的经历,和眼前方源的情况一对比,简直是严重打击他的自尊心。

一个念头不由自主地从他脑海中浮现出来:“难道说,这个柳贯一修行变化道,比我修行力道还要天才?境界还要更高不成?”

其实方源已经尽量伪装了。

他装得气喘吁吁,劳累不堪的样子。

但他的表现,主要是以刚刚楚度、不是仙的表现为参照的。他们两人已经行走了许多次,经验丰富无比,也在尽量保持风度。

方源虽然表现出了狼狈的样子,但也只是弱他们一点罢了。

最终,方源走到二十步,停下不走,笑道:“我和楚兄你走得差不多啊。”

楚度饱含深意地望着方源:“柳兄是藏拙了。”

方源苦笑道:“我是强撑,再走下去,我可就要当场吐血了!”

楚度的笑容比他更加苦涩:“不瞒柳兄,我第一次走的时候,到了十步开外就开始吐血了。”(\~{}\^{}\~{})<!--80txt.com-ouoou-->

------------

\end{this_body}

