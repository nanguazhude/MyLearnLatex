\newsection{方源的未来身}    %第七百三十八节:方源的未来身

\begin{this_body}

%1
方源再次登上石莲岛。

%2
更准确地说,是今生第一次,但上一世他已是登上过。

%3
一踏入石莲岛,方源就沉浸在红莲魔尊的记忆影像之中。

%4
这在他上一世就已经参阅过,今生只是重复,但方源依旧有些触动。

%5
“红莲魔尊算是最特殊的一位尊者了。他本有九转修为,但利用了春秋蝉重生,算是舍弃了九转尊位。”

%6
“但若是他不舍弃的话,他就无法回到过去,改变事实,弥补心中的悔恨和遗憾。”

%7
“所以,到了后来,他才开创出未来身,以及前有古人、后有来者这些杀招……”

%8
承载着红莲真传的石莲岛,本身就有隐形匿迹的能力,天庭要寻到极不容易。一直以来,都没有什么收获。

%9
上一世,就快要有收获的时候,还被方源破坏了,最终凤九歌不得不动用大同风杀招,摧毁了石莲岛。站在天庭角度来看,也蛮悲催的。

%10
方源踏足的这座石莲岛,一直静立在此处,有数百万年的历史。

%11
不知道为什么,这一次方源站在石莲岛上,看着岛外的滚滚河水,忽然又有了些许明悟。

%12
“这光阴长河起源之点,应当便是整个世界的启始之时。”

%13
“光阴长河的终结之点,自然是整个世界的毁灭之时。”

%14
“光阴长河的纵向,它的长度,是整个世界的经历。而光阴长河的横向,它的宽度,则对应着世间的每一份事物——花鸟鱼虫、人兽草木等等。”

%15
“很显然,世界越繁荣,存在的事物便越多,光阴长河就越是宽阔。世界越贫瘠荒芜,存在的事物就越少,光阴长河就越狭窄。”

%16
方源有着宙道准无上的境界,此时灵光乍现,明悟种种,又对重生更增了解。

%17
“我重生的本质,其实只是下游未来的意志,通过春秋蝉,遁入了上游,并且和上游某个时间的我融合一体。”

%18
“这个世界的时间真理,决定了世界唯一,并不存在任何的平行空间。当我重生时,下游仍旧在流淌。在我死后的将来,很显然就是天庭制霸,四域束手的格局。”

%19
“但现在我重生了,宛如一团墨水滴在了原本清澈的河水中。随着我改变越来越多的事物,让一些事物存在,让一些事物消失,这团墨水就会逐渐浸染更多的河水,将更长的河面染成别样的颜色。”

%20
“等到墨水浸染到天庭修复宿命蛊,举办炼蛊大会的时候,许多事物都产生了剧变,未来就改变了。”

%21
方源本是从地球上穿越过来,在地球上有一个名词,称之为蝴蝶效应。

%22
方源现在陡然明白,对于这个蛊世界而言,墨水效应更能准确地阐述重生后的影响。

%23
“当然,我能够轻易改变其他的事物,是因为宿命蛊并不完好,更关键的是我本来就是天外之魔,不受宿命的束缚。”

%24
“若是红莲魔尊重生到过去,很难改变其他事物的状态,总会应该各种缘由令事物回归本来面貌。”

%25
“换句话讲,天外之魔是大团的墨水,而本土蛊仙重生,不过是砸下的水波,即便会在短时间内引发涟漪,但很快在宿命蛊的约束下,事物都恢复到本该的状态,水波不兴。”

%26
想到这一点,方源不由地更加佩服红莲魔尊。

%27
这位特殊的存在,即便不是天外之魔,也硬生生地打开了局面。

%28
难以想象,他究竟在这当中付出了多少,又牺牲了多少!

%29
红莲岛上,留有大量的宙道仙材,其中八转级数的仙材占据绝大多数。

%30
红莲真意——能够让一位蛊仙的宙道境界飙升到准无上大宗师的境地。

%31
还有以春秋蝉为核心的各种仙道杀招,春秋必成就是其中之一。

%32
最后还有一招红莲魔尊特意留下的手段,名为未来身,能够让蛊仙拥有未来七转巅峰的状态。

%33
缺少的是悔蛊。

%34
这只八转仙蛊还留在龙鲸洞天之中,是被乐土仙尊带走的。

%35
方源上一世,是通过乐土仙尊的布置,首先拿到了悔蛊后,被直接传送到了石莲岛上。

%36
今生则是事先知晓了石莲岛在光阴长河中的位置,提前找上门来,将大部分的红莲真传先继承了。

%37
在时间上,无疑是大大的提前了。

%38
上一世,方源三入光阴长河。第一次被天庭的四大宙道仙蛊屋围追堵截,又被厉煌埋伏,痛失重点投资打造出的仙蛊屋雏形,暴露了所有底牌后,总算逃走了。

%39
第二次,方源依靠乐土仙尊的手段,获得悔蛊后,直接传送到了石莲岛上,尽收红莲真传。

%40
第三次,则是发觉另外一座石莲岛出现,天庭极可能得手。方源便联合冰塞川等人,对抗天庭,争夺红莲真传。最终凤九歌催动大风歌,将石莲岛摧毁,双方都没有讨得好处。

%41
总体而言,方源在光阴长河中的经历,可谓坎坷起伏,历经酸甜苦辣。

%42
这一世重生,借助脑海中的记忆,方源先把甜头给尝了!

%43
“没有悔蛊,暂时也不要紧,毕竟眼下的难关是天庭入侵琅琊福地,距离我大批升炼仙蛊还有不少时间。”

%44
“先利用好杀招未来身罢。”

%45
方源将影无邪等人都放出来,让他们在石莲岛上修行,一如前世,熟悉未来身。

%46
和上一世不一样,这一世方源的麾下妙音仙子、白兔姑娘都还活着,大大提高了未来身的利用率。

%47
并且他自己也享受了一把未来身!

%48
他上一世修为八转,未来身杀招对他没有效用,但这一世方源提前过来,只是七转修为,未来身对他当然有用了。

%49
“等等!”

%50
“或许这才是我来到石莲岛的正确时间点。”

%51
“当初红莲魔尊推算我,是当我七转修为的时候,就来到这里,因此留下未来身杀招帮助我。”

%52
“上一世我以为是红莲推算失误……但现在看来,难道是红莲魔尊已经考虑到了我重生后的状态?”

%53
“这样想来的话,这份红莲真传不是留给我上一世的,而是专门等候这一世的我?!”

%54
方源眼中精芒闪烁不定了好一阵。

%55
他把影无邪等人留下,自己独自离开了石莲岛。

%56
利用宙道杀招,他隐藏自己,继续潜游。

%57
他还记得第三次入河时的情景,对于另外一座石莲岛的位置,心中早有算计。

%58
此番入河,他的计划就是两座石莲岛!

%59
花费了一大段时间,历经了一些波折,方源搜遍了石莲岛可能出现的地点。

%60
但遗憾的是,他始终没有发现那一座石莲岛。

%61
“石莲岛本身能够转移,看来这座岛是飘忽不定的,和我登上的石莲岛情况不同。”

%62
无缘红莲真传,方源也不气馁。

%63
这种情况,也在他的估料之中。

%64
今古亭。

%65
四旬子遭遇了一个小麻烦。

%66
一头上古年兽正在攻击今古亭,今古亭绽射玄光,岿然不动。

%67
很快,在四旬子的合力出击下,这头上古年兽阵亡。

%68
旬果子欢呼一声:“太棒了,我们又有大量的宙道仙材了!”

%69
“是啊。”中旬子感慨不已,“光阴长河不愧是我宙道蛊仙的修行圣地,如同荡魂山、落魄谷对于魂道蛊仙一般。”

%70
下旬子附和着:“是啊,天庭派遣我等来镇守今古亭,虽然重点是防备方源这魔头,但同时对我等而言,也是绝佳的修行良机啊!捕获宙道仙材,对我们修行大有裨益。”

%71
上旬子微笑道:“不止如此,在今古亭中,我们还能观察上游、下游,看到过去和未来的许多景象。看到无数事物在时间的影响下发生的变化,对于我等参悟宙道,提升宙道境界有着极妙的帮助。”

%72
旬果子轻轻咬牙:“可惜宿命蛊坏了,并不完整。根据历史记载,在很久之前,宿命蛊完整无缺的时候,我们宙道的大能就能通过观察光阴长河的上下游,洞悉过去,观察未来,能够准确预知许多事情呢。”

%73
下旬子连连点头:“是啊。当初星宿仙尊似乎就曾经这么干过。她虽然不是宙道蛊仙,但是智道手段非凡,模拟出宙道威能,通过光阴支流,推算出未来的种种情形。她预知到绿龟七人众开创出历史上第一座蛊屋——龟房,便创造出了星宿棋盘。”

%74
四旬子缅怀了一阵宙道在历史上的辉煌,便又平息心境,一边休整,收拾上古年兽的尸躯,一边继续监测光阴长河。

%75
他们根本不知道:方源已经偷偷地来过光阴长河,提前获得了红莲真传,然后又偷偷地离开。

%76
方源原路返回,走出光阴长河,进入光阴支流,最后回到北原。

%77
他走出大阵之后,便催动定仙游仙蛊,直接去了南疆。

%78
这个时间段,方源怎么会有定仙游仙蛊?

%79
仙蛊唯一,定仙游仙蛊当然还在凤九歌那边。

%80
事实上,方源利用的是杀招未来身!

%81
这个宙道杀招非常奇妙,它在方源的至尊仙窍中凝聚出了一个人形光影。

%82
光影盘坐悬空,宛若光阴长河的水凝聚起来,迸射着五颜六色的缤纷色彩。

%83
光影面目如同方源,栩栩如生。

%84
光影也有至尊仙窍,但这仙窍中是方源未来的仙窍景象,最关键的是有许多蛊虫的虚影,比如定仙游!

%85
方源重生至此,影响较小,光阴长河的下游还未被墨水浸染。

%86
因此下游的情景,还是方源的上一世。

%87
未来身能够让蛊仙得到未来的七转最强姿态,方源上一世的七转最强姿态,便是五界山脉大战之前一些。

%88
那个时间,方源已将夏槎等南疆故乡追兵俘虏,大肆勒索南疆各大超级势力。

%89
为了对抗强敌,设计天庭,进一步谋求光阴长河中的红莲真传,方源便吞掉了夏槎的宙道仙窍,晋升八转,在五界山脉中做出渡劫的假象,布置了陷阱埋伏。

%90
琅琊守卫战中方源偷取了凤九歌的定仙游蛊。

%91
之后,方源攻略掠影福地,和池家对掐,最终暗地里和池曲由达成交易。

%92
再之后,方源铺设年流伏诛阵,将南疆追兵坑害俘虏,勒索南疆各大家族。

%93
然后,五界山脉大战。

%94
再然后,方源一入光阴长河,石莲岛的影子都没看到,最终被天庭追杀,狼狈逃生。

%95
因此,方源的未来身就是他在吞并夏槎仙窍,晋升八转之前。

%96
这个时间,他早就有了定仙游仙蛊了!

%97
利用未来身杀招,方源直接从北原来到了南疆。

%98
但他没有去往五界山脉。

%99
五界真传对他有大用,方源当然知晓这一点。但是现在开启五界真传并不明智。

%100
因为五界真传开启时,动静很大,要巧妙地摧毁五界山脉,方能得到陶铸意志的认可,将他引动出来。

%101
方源目前只是七转修为,战力虽有提升,但对抗八转很难。

%102
万一开启五界真传时,惹来什么八转蛊仙,或者仙蛊屋等等,那就尴尬了。

%103
为了保险起见,方源暂时将五界真传推后。

%104
他此番来到南疆,不是为了五界真传,而是来到一处普通的山谷中。

\end{this_body}

