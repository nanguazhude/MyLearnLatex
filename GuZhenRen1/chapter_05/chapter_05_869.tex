\newsection{途中异变}    %第八百七十三节:途中异变

\begin{this_body}

%1
数日后,东海的某处无名小岛。

%2
八位蛊仙终于到齐了。

%3
庙明神为首,身边鬼七爷、花蝶女仙、蜂将拥护着他。除此之外,就是曾落子、土头驮、童画以及方源了。

%4
“虽然我重生了,但似乎没有过多影响到庙明神这里呢。”方源当然是伪装成了楚瀛,不动声色地观察着。

%5
和上一世一样,庙明神筛选出来的人选,没有任何的变化。

%6
但方源自己却是和上一世截然不同了。

%7
他现在的处境,比上一世同期要好得多。

%8
上一世的这个时候,他并不能吞下气海洞天,只是将气海洞天、五相公共洞天的资源吞并,初步消化了一些。而九转仙道杀招天相,仍旧抵抗着方源的淬炼,没有炼成。

%9
最遗憾的是,上一世他图谋青鬼沙漠的计划彻底告吹,魂道修行严重受阻,几乎停滞不前。同时更导致胆识蛊产量根本提不上去。

%10
这一世重生直至现在,方源早就彻底吞并了气海洞天,达到八转修为,更化身气海老祖,捞取庞大的政治利益。天相杀招早就被他炼化,只是还没有寻到更多的洞天。

%11
青鬼沙漠一直在开发,魂道修行进步迅猛。胆识蛊的产量更不必说,超越了过往最高纪录不知多少倍。

%12
上一世,天庭霸占着光阴长河,方源奋力拼杀,才逃出生天。

%13
而这一世,则是方源霸占了光阴长河,天庭损失极其惨重,威望遭受狠狠重挫。偏偏天庭还只能硬着头皮,继续搭建仙蛊屋,往光阴长河增兵,绝不敢坐视方源霸占着光阴长河,再收取第二份红莲真传。

%14
不只是光阴长河攻守易位,方源在西漠、东海以及华文洞天、兽灾洞天都有布局,并都取得许多可喜的进展。

%15
方源重生之后,殚精竭虑,以五域两天为棋盘,四面开花,八方布子,带来的庞大收益不只是眼前,而是会在将来延绵不止。

%16
相似的处境则是——方源在宝黄天中的生意,遭受到了中洲、南联的联合抵制打压。

%17
对于这个情况,方源暂时也没有太好的解决办法。

%18
一来,他对南联的影响力的确是在逐步下滑的。毕竟他勒索的筹码越来越少,如今手中也只有第二空窍仙蛊,才对南疆的数位蛊仙有着强烈的吸引力。

%19
二来,不管是天庭还是南联,家大业大,底蕴深厚,和他们在宝黄天中拼价格,拼底蕴,方源是绝对拼不过的。

%20
方源的底蕴更多是集中在仙蛊、仙蛊屋、天地秘境上,修行资源都在积极转化为自身的实力。就比如胆识蛊,若是方源将胆识蛊放到宝黄天中,这种垄断生意定然是广受欢迎,天庭、南联怎么打压都不会有效果。

%21
但方源没有这么做。

%22
他选择将绝大多数的胆识蛊,都用于自己的魂道修行上去。

%23
他仍旧在千方百计地提升战斗力!

%24
这种选择,让他在宝黄天中的交锋,更加处于下风。

%25
“仙材的结余方面,我这一世甚至要比前世还要糟糕一些。毕竟之前几番大炼仙蛊,损失太多。这一世的麾下也增多了,上一世白兔姑娘、毛六、妙音仙子早就死了。这一世,他们不仅健在,我还吞并了四大异族。在至尊仙窍中也铺出了更大的摊子,开始大规模地豢养雪人、石人、毛民等等,以图染指人道。”

%26
方源收获的仙材,原比上一世还要多,别的不说,单单他吞并下了琅琊福地,就是一笔庞大的收益。

%27
但他这一世同样支出也多得多。运道仙蛊、木道仙蛊的大笔炼制,异族的栽培,还有收集十二种太古年兽,栽培各个分身,以及数场激战耗费了海量仙元、损失蛊虫无数……

%28
总的来讲,方源不管是在外部,还是内部处境,都比上一世好了许多。

%29
唯一可能弱势的地方,就在于仙材的库存并不多,某些种类的仙材甚至快要见底。

%30
这种情况下,方源根本不打算和天庭在宝黄天中死磕。

%31
天庭收购宙道仙材,明显是在阻击方源,态势比上一世更加疯狂,方源选择避退。

%32
在这方面,他肯定斗不过天庭,更别说天庭身边还有个南联。

%33
方源接下来的计划,就是取得悔蛊,然后大规模升炼自家的仙蛊。这和上一世是一样的。

%34
他现在手头上,积累了大量的六转仙蛊。一旦将这些六转仙蛊提升到七转,他的战力又会得到一个全面的大幅度飙升!

%35
单单煮运锅这方面,就很值得期待。

%36
“上一世我是缺少时间,又没有得到红莲真传,所以在苍蓝龙鲸的仙窍中匆匆而过,并未真正安心攻略。”

%37
“这一次,我不仅要谋夺悔蛊,更要全力尝试,看看能否收拢了那座功德碑,甚至将整个乐土真传、苍蓝龙鲸都纳入手里!”

%38
方源把时间算计得相当清楚。

%39
眼下距离中洲举办炼蛊大会,全面修复宿命蛊其实已经不远了。

%40
但没有关系。

%41
方源安排了战部渡在兽灾洞天,这个分身不负期待,进展稳定而又快速。相信等到方源出来,战部渡就能里应外合,配合他顺利吞并兽灾洞天。

%42
这就节省了方源一笔相当珍贵的时间,完全可以投放到苍蓝龙鲸当中。

%43
这是早期的投资,如今已经滚雪球样儿良性发展。不久的将来,方源必然会收获更多。

%44
庙明神环视一圈:“诸位,关于此处行动,我都已经向诸位详细说明了,并且给了一天时间让诸位多加考虑。此行我也是首次,因此风险难以评估,或许有身死道消的可能。诸位当中若有人选择放弃,请现在离去。我庙明神礼送,绝不阻拦,也发自真心的理解。”

%45
当然无人退出。

%46
事关乐土真传,哪怕流派不合,众仙也都抱有相当的期待。

%47
这些天来,他们都在设想,苍蓝龙鲸当中到底会有什么。

%48
殊不知在他们的身边,方源早就门清,已经设定了种种详实可靠的计划了。

%49
和上一世一样,群仙在出发之前,曾落子就出力,为彼此定下了盟约。

%50
而后众仙钻入海底,又被种下一个标记位置的手段。

%51
随后这才进入海底潜流,庙明神领头当先,时而改换路径。方源夹杂在群仙之中,紧随庙明神。

%52
他在前世苍蓝龙鲸出现的位置,早就有所布置,但一直都没有等到苍蓝龙鲸的出现。

%53
若是不跟紧庙明神,恐怕还真的会找不到苍蓝龙鲸。

%54
众仙一共八人,连续赶路数天。主要是利用海底潜流,没有海底潜流的话,也一直潜藏在海底悄然前行,非常低调。哪怕是遇到拦路的荒兽、上古荒兽,都以闪避为主。碰到什么天然资源,也都舍弃不取。

%55
一路上,群仙在庙明神的有意带动下,不断地交流,队伍气氛变得相当融洽和睦。

%56
庙明神也悄然成为了队伍中当之无愧的核心人物,左右逢源,他的交际手段的确是当世一流。

%57
群仙日夜兼程,毫无停留。

%58
庙明神也害怕任修平那里走漏了风声,所以是抓紧一切时间。

%59
忽然,他速度微微一缓,压抑着兴奋之情道:“我们接近了!”

%60
群仙精神顿振。

%61
方源眉头暗皱,和上一世一样,他早就在侦查庙明神,但仍旧看不清庙明神究竟是用何种方法探寻到苍蓝龙鲸的。

%62
就目前为止这个位置,和方源上一世的情况,完全不同。

%63
随着距离苍蓝龙鲸越近,群仙逐渐发现异常之处。

%64
海水变得越来越湍急,荒兽、上古荒兽数量开始激增,方源等人不得不一边前行,一边对抗,屠杀大量荒兽,杀出一条血路出来。

%65
群仙奋力拼杀,方源却面露异色。

%66
上一世他着了道,但这一次他是有过经历,再细细品味,发现点点细微的异象。

%67
“看来乐土仙尊的考验,已经在这海底开始了。若是我自己现在主动受死,会不会被直接传送到苍蓝龙鲸的仙窍里面去?”

%68
方源觉得这种可能极大,但他并不打算冒险。

%69
稳妥起见,他还是打算学习上一世,按部就班地“去死”。

%70
“什么人?!”就在这时,曾落子忽然冷喝一声,向着左后方打出一记杀招。

%71
杀招并无攻伐威能,几位蛊仙身影却随之显现。

%72
“不愧是曾落子,好一记信道侦查手段,妙得很。”为首的蛊仙微微带笑,话音落下,就带着一股威能,不仅将曾落子的侦查杀招破去,同时还隐隐向庙明神等人卷席而来。

%73
方源见到此人,微微一愕:“怎么沈从声跟随我们这里?”

%74
这可和上一世的情况大大不同。

%75
这位只说了一句话,就破掉曾落子侦查杀招的,自然就是沈家的太上大长老沈从声。

%76
这可是当今正道八转蛊仙之一!

%77
庙明神等人最高不过七转修为,根本无法和他抗衡。

%78
然而沈从声不仅来了,身边还跟着数位蛊仙,不只是沈家的好手,还有任修平这样的散修强者。

%79
方源神色惊骇,这当然是装的。他是本体亲至,别说是沈从声,就是东海四位正道八转齐至,方源也怡然不惧。

%80
庙明神等人却是真正的骇然了。

%81
对方实力太强,与之对抗根本没有任何胜算。

%82
如何是好?

\end{this_body}

