\newsection{一起来分析楚瀛}    %第八百七十七节:一起来分析楚瀛

\begin{this_body}

任务完成,沈从声默念几声回归,便又传送回到了接引岛。

功德碑就在他的身边。

他查看了一下碑面,在功德排行榜单上,果然见到了自己的名字:沈从声,九功德。

沈从声眉头微蹙。

完成一次任务,竟然只有九点的功德?

这个报酬太少,出乎沈从声的意料。

“按照这种进展,岂不是在我来之前,那个楚瀛已经接取了十个左右的任务了?”想到这里,沈从声眉头皱得更紧了。

他再看看那些奖励,心都有点凉了。

那些他看得上眼的仙蛊、仙材,任何一份功德都要过万啊!

“还有一点,仙蛊唯一,若是让别人抢先兑换了出去,我自然就没有机会了。”

“如今楚瀛领先这么多,得加快步伐,追上他才是!”

沈从声身为八转,都产生了一股紧迫感,其他人就更是如此了。

“大长老。”沈潭走过来,迎接沈从声。他虽然早些时候就完成了自己的任务,但并没有急着去完成第二个,而是选择在碑下等待。

事实上,群仙在第一次传送之前,都约定好了,完成任务后相互交流。

沈从声、沈潭、沈泣、沈笑、沈奈何、任修平、童画是一伙,庙明神、蜂将、花蝶女仙、鬼七爷、土头驮、曾落子是第二伙人马。

两方可谓泾渭分明。

很明显,沈从声一伙人要强势得太多。但在乐土这样的特殊环境,他们不方便对庙明神等人下手,只得看着他们在另一边团结一致。

“天无绝人之路!诸位,眼下就是绝世仙缘,只有抓住了,将来面对沈从声,才有更多的一丝可能。我们要一起努力,只有紧密团结,才能有希望。否则靠我们单打独斗,绝无生机。”庙明神鼓舞周围的蛊仙。

曾落子、土头驮等人点头称是。他们的压力远比沈从声要大,因此完成任务,拔升实力的心情更加急迫。

就算在龙鲸乐土是安全的,但三百天后他们传送出去,就得面对沈从声了。

“说说看,你们各自有什么发现?”庙明神开口。

众人一五一十地都说出自己的成果,没有丝毫的隐瞒,更带有自己的许多猜测。

每个人的发现都和沈从声的差不多,庙明神最后总结道:“乐土对我们的限制,其实可以看成是对我们的一种保护,至少暂时沈从声是无法向我们下毒手的。”

“我们所有人完成任务后,获得的功德,最多只是九,不超过十。看来每一个任务的报酬,都在一到十之间了。”

“但我有一种感觉,这座功德碑会在将来出现更多的变化。”

曾落子、土头驮相互对视一眼,曾落子犹豫了一下,还是开口问道:“庙兄,你既然能感应到苍蓝龙鲸的准确位置,再将我们带回来。难道这里对你是完全陌生的么?”

庙明神苦笑:“曾兄、土兄,若我说我能感应到苍蓝龙鲸的位置,也不知道是为什么,你们会信吗?数年前,关于苍蓝龙鲸的信息就开始莫名其妙地出现在我的脑海中,只是刚开始这些信息非常模糊破碎,直到最近才拼凑成完整,让我明白原来是苍蓝龙鲸在呼唤我。”

“若我真的知晓这些布置,掌握什么捷径,我会不用吗?等着承受沈从声的威胁吗?在外界遭受追杀的时候,我若知道其实是一片幻境,我早就主动寻死了。”

“二位,这都是我的实话,没有丝毫欺瞒!”

曾落子、土头驮看着庙明神诚挚的神情,均点头道:“我们信你。”

鬼七爷忽道:“说起来,我倒是想起一件事情。之前在海中遭受追杀,楚瀛眼看着被音刃袭身,他却是速度暴涨,主动撞上一只太古雷凰,随后被雷电劈死。”

“没错,我也看到了。”蜂将附和道。

“主动寻死,实在奇怪,看来楚瀛是真的知道一些内幕的。”土头驮眼中精芒爆闪。

之前,方源并不知道,死在沈从声的手中也会被接引进来。为了保险起见,他还是选择上一世的死法。

如此一来,就有了破绽。

就像之前沈从声的音刃,主动避开童画一样。

庙明神一伙人发现了楚瀛的破绽,另一边的沈从声等人,也是同样如此。

“庙明神。”这时,沈潭走在最前面,主动靠近庙明神等人。

而在沈潭的身后,则是沈从声、任修平一伙。

“诸位仙友不必紧张。”沈潭笑道,“我们此次不是来找诸位的麻烦,而是寻求合作的。”

“合作?你们的脸皮真是够厚的。”

“之前是谁在追杀我们?”

蜂将、花蝶女仙语气不忿。

庙明神则拦住他俩,看向沈从声。

沈从声微微点头,淡淡地道:“这正是我的意思。之前的确是我出手,要斩杀你们,但诸位设身处地想一想,换做你们是我,难道你们会放任我离开吗?”

“我不是不想杀了你们,但眼下情形如此,那就只有和你们合作了。”

“我以沈家的名誉保证,就算将来出了这个龙鲸乐土,也不会对你们下毒手,而是放任诸位离开。”

“而诸位则需要将你们所知的一切情报,都一五一十地告诉我。”

“沈家的名誉……”庙明神沉吟,他身边的蛊仙亦都神色微动。

沈从声乃是沈家太上大长老,正道人物自然要一言九鼎,如今他是用整个沈家的名誉来担保,可信度很高。

当然,最稳妥的方法,还是签订盟约。

在场的曾落子、沈奈何皆是信道七转蛊仙,但签订盟约的手段是用不了的,被这片乐土严格限制住了。

“你们好好考虑一下,这是你们唯一的机会。若是错过了,将来出了龙鲸乐土,可不要后悔。”沈潭笑着威胁。

庙明神等人越发沉默。

功德碑上已经说得明白,他们这些人只会在这里待上三百天,时限一到,就会被传送出去。到那时,说不定他们就要面对沈从声了。

沈从声的保证,当然很不保险,他完全可以反悔。

但事关自家生死,庙明神等人当然不会放过任何一个生还的可能性。

庙明神忽然一笑,点头道:“我接受了。我和你们合作,把我所知道的一切都告诉你们。”

沈从声也笑起来:“好,识时务者为俊杰。我会放你一命,说吧。”

庙明神苦笑:“其实也没有什么好说的。”

他将之前对土头驮、曾落子所说的,又讲了一遍。

沈潭大怒:“你说你其实什么都不知道?你骗鬼呢?还是认为我们沈家好糊弄?”

沈从声却是露出思索的神色。

他觉得:庙明神的话很可能是真的,谎话可以说,但行动往往欺骗不了别人。来这里的一路上,庙明神的种种表现都印在沈从声的心中。看这个样子,也不像是对龙鲸乐土有所了解的人。

鬼七爷插言道:“我们刚刚分析了一遍,却是觉得楚瀛大有可疑。搞不好,他真的知道一些内幕!”

群仙眼中精芒爆闪。

任修平冷笑:“这话还用说吗?那小子之前就主动寻死,来到这里,又率先苏醒,把接引岛上的仙材都搬空了。再看看功德碑,他的功德已经近百了,呃。”

任修平声音一滞。

就在这个时候,白光一闪,方源再次被传送回来。

他任务完成,碑上的功德真正破百,上升到一百零二。

群仙聚在一团,看着碑下的方源。刚好说到你,你就出现了啊。

方源也瞥了一眼群仙。哟,气氛挺好啊!

方源有些意外,但并不奇怪。

蛊仙都是万中挑一的人杰,这些蛊仙自然是心思剔透玲珑,知道分则有害,合则两利。只要沈从声放下架子,庙明神等人又有未来忧虑,勉强合作并不奇怪。

沈潭笑了笑,主动走过来,接近方源:“楚瀛兄弟。”

他主动打招呼,态度可比对待庙明神要热情多了。

就目前的情况来看,方源的价值也明显要比庙明神要高得多。

但,还未等他说出招揽的话……

“滚。”方源看了沈潭一眼,便直接转身。

沈潭脸上的笑容顿时僵住,旋即眼角开始抽搐,怒气升腾起来,正要大喝,肩膀却被一人拍住。

是沈从声!

沈潭立即偃旗息鼓,不敢当场发作,强忍怒气,瞪着方源。

“小兄弟。”沈从声呵呵一笑,“我们是可以合……”

“滚。”方源一个字,语气平淡,但又干脆利落地打断了沈从声的话。

全场惊寂。

沈从声双眼微瞪,一时间竟也被骂得愣住了。

这有多长时间了?从未有人这么骂过他了。

他可是沈家太上大长老,东海八转大能,谁敢这么当面骂他?

就算是平日里和宋启元这些存在,有什么罅隙龌龊,彼此之间也是保持着风度的。

但方源就是这么骂了。

平平淡淡,干干脆脆。

庙明神等人震惊不已地看着方源的背影。

“这可是八转蛊仙,沈家的太上大长老,你要不要这么莽啊!”

“你现在不服软,将来三百天后传送出去了,怎么办?”

一时间,众人也不知道该佩服楚瀛的勇气,还是耻笑他的愚蠢。

方源骂完之后,立即消失,却是再一次接取任务,传送了出去。

留下蛊仙们仍旧站着,看着空荡的功德碑发呆。

沈从声忽然轻笑一声,打破了沉默:“这个楚瀛倒是好胆色。老夫也盼着他出去之后,也能有这样的胆色了。”

他脸上笑着,眼眸中却是闪烁着愤怒的电芒,显然是动了杀机!

“庙明神,这个楚瀛究竟是何方人士,在何处落脚安家?”沈潭转身,喝问道。

他这样的态度咄咄逼人,让鬼七爷等人十分不忿,纷纷怒视。

庙明神长叹一声,摊开双手:“若我告诉你们,我也不清楚他的来历,你们相信吗?”

“呵呵。”沈潭嘴角抽动,干笑两声,脸上神情已经说明了一切。

你骗谁呢?

你不晓得楚瀛的跟脚,你会邀请他来探寻乐土真传?

“楚瀛啊,楚瀛,我已经快要被你害死了。”庙明神心头苦笑,但也不甘就范,把锅一甩就甩到任修平的头上。

“我之所以信任楚瀛,还不是因为任修平。任兄,你说两句吧,现在隐瞒下去,还有什么必要吗?”

庙明神一句话,顿时又让群仙的灼热目光都集中在了任修平的身上。

任修平心头一颤,承受着巨大压力,只得将自己所知的和盘托出。

群仙这才知道大致原委。

庙明神又解释道:“其实我和楚瀛相处的时间也不长,我也只是邀请他参加过一场拍卖会而已。当时,童画、土头驮他们也在场的。”

沈从声又将询问的目光转移到他们俩的身上。

童画、土头驮接连开口,说出自己所知的一切。

沈从声听完一整轮,眉头大皱。

情报虽多,但屁用都没有啊!知道的都已经知道的,不知道的还是不知道。

无迹可寻,无从下手。

“楚瀛这个人物,不简单。”沈从声心生预感,但仍旧自信,“他始终只是一位七转变化道蛊仙,绝不是我的对手。”

------------

\end{this_body}

