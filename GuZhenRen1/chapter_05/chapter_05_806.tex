\newsection{红花眷属}    %第八百零九节:红花眷属

\begin{this_body}

%1
“学生李小白,参见二位师长。”李小白走上台来,很有礼貌。

%2
申大才子微笑地看着他,点了点头。

%3
而私塾的姜先生则问道:“小白,可有诗词?”

%4
姜先生的眉宇间,有一丝隐约的担忧。这个李小白才智中庸,但愿他平时有着什么积累,否则依照他平时的水平,难免会令人失望。

%5
平日里叫自己失望,也就算了。但如今却是在申大才子面前……尤其是这场文斗的前中期,已是精彩绝伦,必定会扬名天下。这要在后期失色,天下人即便晓得姜先生的才学,也会暗中笑他不是一位明师。

%6
“老师,学生的确是有一首诗。”李小白当即答道,神情自信。

%7
“那就吟来一听。”姜先生略微放下心来。

%8
在这华文洞天,不比寻常地方,文风极其鼎盛。

%9
皆因只要作出一首好诗词,哪怕无人赞赏,天地都会来奖赏你!

%10
所以,作诗词学问非常重要,有关人们的吃喝拉撒,乃至地位、声望和未来。

%11
一般而言,绝大多数的文人都会平日积累,有一些好的词句,哪怕不成诗,但都压在箱底,留待将来补足。

%12
还有许多文人,虽然作得好诗词,但却不公布出来。为的就是隐藏实力,应付频繁至极的文斗。

%13
在这样的大环境下,李小白藏着诗词,并不叫人意外。

%14
李小白心中早有考量。

%15
“我记着的诗词极多,有大量的绝世篇章,在地球上名传千古,广为流传。经过漫长岁月的打磨,越发熠熠生辉。”

%16
“但若是用这些绝世篇章,大大不妥!”

%17
“这个洞天世界乃是信道环境,最擅长的就是收集情报。尤其是观测文气、才气,这样的手段几乎人人都会,只是有高有低。”

%18
“我本身文气、才气不足,魂穿之前更是一位普通的学生。冒然弄出绝世篇章,不仅不会让人信服,而且还会惹来怀疑,引发调查。”

%19
“即便自己事后说明,这并非自己所著,那也为时已晚。本身会被打上不诚的标签,在这个世界,会让人鄙视,乃至唾弃。”

%20
方源为了攻略华文洞天,早已经推算了许多方案,用来应对各种情况。

%21
李小白这个分身,本身文气、才气都不足够,并非方源本体的才情,只是带来少许。

%22
不过,李小白的记忆中,却是有着大量的著名诗词,除了地球上的,还有方源本身的著作。

%23
当即,李小白挑选了其中一首,徐徐吟诵而出。

%24
姜先生暗中吐出一口浊气,这首诗词对仗工整,言之有物,也勉强够的上今天的文斗了。

%25
“什么嘛,也不怎么样啊。”

%26
“李小白我和他同窗了三年,今天算是他超水准发挥了。哈哈。”

%27
“这不正好吗?有了他的这首诗来衬托,接下来的两位幸运儿有福了。”

%28
台下的学生们心思越发活泼起来。

%29
李小白吟诵出来的诗词,虽然符合他的身份,但在他的同窗里面,还有大把的人比他更为优秀。

%30
尤其是一些学生的肚中,暗中珍藏了不少诗词。

%31
这些诗词是他们平日里搜索枯肠,竭尽心智,雕琢而出的,用在其他场合有些大材小用,如今却是正为合适。

%32
当鼓声再起,红花开始传动时,这些学生的精英,无不蠢蠢欲动,双眼放光。

%33
鼓声停歇下来。

%34
绝大多数的学生失望了,怎么不是我呢?

%35
然后,他们的双眼红了,怎么又是他!

%36
红花正停在李小白的桌案上。

%37
“咦?”李小白也非常惊讶。

%38
“规矩不可废,还请你再上来一次吧。”申大才子摇头失笑。

%39
姜先生再次担忧起来,按照李小白的素养和底蕴,之前的发挥已经是超水准,这一次恐怕……

%40
但李小白又吟诵了一首诗。

%41
这首诗和刚刚那首的水平,完全相当。

%42
姜先生放下心来,打量李小白的眼神柔和了一些:“看来我这个学生,虽然才情不佳,但是平日里知道努力,所以积累了两首诗词。这两首诗词应当是他自己的创作,按照他的水准,能创作出来非常不容易了。”

%43
李小白没有丢了他的面子,这让姜先生很是欣慰。

%44
“最后一个人选了。”

%45
“前两个机会都给了李小白这货,唉,真是白瞎了!”

%46
“要是我上场,必定能教两位先生刮目相看的。但就是不传给我啊。”

%47
“第三次机会,不知会花落谁家?若不是我的话,那就更不要是那几位才好。”

%48
学生们都有点坐不住了,各自心思都泛起在心头,不再淡定从容。

%49
鼓声开始,学生们望着红花,很多人眼眶都红了。

%50
“传给我!传给我!”

%51
“唉,又走了……但愿鼓声能再持久一些,在轮到我的手中。”

%52
“咚!”

%53
鼓声陡然一振,猛地停止下来。

%54
申大才子睁开双眼,面带微笑:“这一次会是谁呢,呃。”

%55
他愣住了,旋即脸上流露出古怪的神色,手指着最后的幸运儿,哭笑不得地道:“怎么又是你这小子呢?”

%56
李小白拿着花,上了台,颇有些愁眉苦脸。

%57
他先是对台上的两位先生行礼,随后又抱拳作揖,对台下的诸多学生致歉:“诸位同窗,非是小白夺人所好,今日三次登台,对我而言,实乃惊吓多过惊喜。小白知晓,台下诸位同窗才学多胜过小白,小白却三次吟诗,实在是忏愧。”

%58
原本,他的这些同窗都又惊又怒,听了李小白这话,再看他的神情,心中的怒意也就纷纷消散了。

%59
不仅如此,更有一部分人同情起这位李小白来。

%60
毕竟,李小白的才情、文气不如自己,让他上台不就是受罪吗?

%61
“你这运气也真是了得了。罢了,吟诗吧。”姜先生笑骂一声。

%62
李小白便又吟诗了一首,仍旧是水准平平,不过最后一句诗,却是有了亮点。

%63
这个亮点虽小,但却拔高了整首诗的层次。

%64
不仅申大才子笑了笑,对此句赞了一声,就连姜先生都有些意外:“你这首诗的最后一句,倒是有了一些新意。你是如何构思出来的?”

%65
李小白苦笑道:“启禀先生,学生不敢相瞒,这首诗之前不过三句,最后一句乃是刚刚,学生情急之下,想了出来。没想到竟能入两位先生之眼,实在是叫学生也非常意外!”

%66
姜先生一愣,哈哈大笑。

%67
申大才子摇头:“你这个李小白,倒是实诚。”

%68
原来,他在刚刚就暗中动用手段,查探了李小白的文气、才气等诸多信息,发现皆是中庸普通的气象。

%69
“不过灵光一闪,有一些超过水准的发挥,也不是什么稀罕的事情。”申大才子没有任何的怀疑。

%70
姜先生则对李小白有点刮目相看了。

%71
他用一种全新的目光,不着痕迹地打量李小白,心想:“李小白虽然作诗平平,但却看得出他平日里刻苦努力,所以才有了三首诗词的积累。他还能临时发挥,灵光乍现,可见他还是有潜力的。但最叫我欣赏的,却是他第三次上台的一番话,他和同窗告饶致歉,这是他精通人事啊。人情练达即文章,这点真的不错。”

%72
“等等。”姜先生忽然又想到,“这场文斗必定名传天下,我的声名早就天下闻名,也就罢了。但李小白却也因此出名,本来若是三位学生也就罢了,偏偏三次机会都成就了他一个。如此一来,他就成了众矢之的。”

%73
“这一次,申兄和我斗平,短时间内是不会有人找我文斗了。但却不妨碍有些小人,找我的学生下手,来间接地打压我!”

%74
偏偏李小白又是个比较普通的学生。

%75
姜先生皱了皱眉头,当场就下了决心:“回去之后,一定要用力栽培李小白,暗中给他开小灶!此次文斗,天下人都会知道李小白就是我的学生。虽然我的学生有很多,但成名的就他一个啊。”

%76
“将来,只要提到李小白,就会必然提及我。我和他是绑在一块了,必须要把他栽培好了,否则的话……”

%77
姜先生心生重重压力,他决定不管花费多大的代价和精力,也要把李小白雕琢起来。哪怕他是一个朽木,也要雕得看起来像块玉!

%78
中洲,飞鹤山。

%79
顾六如缓缓停止了杀招,他擦了擦头上的汗渍,吐出一口浊气:“可以了,他的仙窍时间,已经被我调至极致,灾劫会非常频繁。”

%80
“有劳了。”一旁的秦鼎菱微笑,她看向昏睡中的古月方正,忽然轻咦一声。

%81
“怎么了?”顾六如疑惑地问道。

%82
秦鼎菱便道:“我正在动用察运的杀招,发现在你的杀招生效之后,方正的运气忽然又有了新的变化。他原本气运如山,青翠葱茏,方方正正,但现在却是忽然变成了,嗯……一个好似锅盖的怪模样。”

%83
顾六如更加疑惑:“这是什么预兆?”

%84
秦鼎菱摇头,也是苦恼:“短时间内,我也搞不清楚。”

%85
两个月后。

%86
一则消息,让方源本体罕见的停止了潜修。

%87
“终于查探出龙宫的具体位置了么。”

%88
“很好,正巧我的气道手段也都基本掌握了,不妨就先下手为强,抢在天庭之前,秘密夺了这座仙蛊屋!”

\end{this_body}

