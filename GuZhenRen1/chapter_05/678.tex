\newsection{风灭红莲}    %第六百八十节:风灭红莲

\begin{this_body}

“收!”龙公猛地闪现,凌驾于龙宫的屋顶。

巨大的吸力用来,八转仙蛊屋极力抗衡,一寸一寸地被拖向龙公的虚窍之中。

其余八转蛊仙看到这一幕,连忙奔赴过来,各种杀招一齐轰向龙公。

龙公叹息一声,身形忽然消失在原地。

龙宫失去了束缚,立即从海底暴射而出,再次飞上苍穹。

东海八转正要联袂追击,冷不防龙公忽然出现在半空中,挡住他们前进的路。

龙吼声再起,龙公身形如电,在众多东海八转蛊仙之间腾挪转折。激战的轰鸣爆响,宛若雷霆连绵不绝,亿万无辜生命遭殃,海面上掀起高达数十丈、上百丈的巨浪。

没有了凤金煌的拖累,龙公展现出惊人的战力。在场的八转蛊仙没有一人,能是他的对手。

不过,东海的这些蛊仙能够修行到八转,各个底蕴雄厚,有两把刷子。

虽然打不过龙公,呈现败象,但一时半会儿并无生命危险。他们之间虽然相互戒备,但面对龙公,能合作的时候都积极合作。

一时间,东海的正道、魔道、散修都联合一起,对付龙公。

龙公感到越来越大的压力。

他之前爆发,压制全场,是因为之前隐忍,并未大动干戈,因此东海八转们都被打了个措手不及。

随着战斗进行下去,龙公的手段渐渐被东海蛊仙们熟悉。他们都纷纷找到克制,或者应对的正确方法。

这个世界上,没有无敌的仙蛊,也没有无敌的仙道杀招,只有无敌的蛊仙。

任何的仙道杀招哪怕再繁杂神秘,都是有迹可循的,既有优势也有弱点。

龙公的确强大,在场的东海八转没有一个能是他的对手。但眼下的战况比较特殊,各方都是为了争夺仙蛊屋,谁也不想和龙公死磕。仙蛊屋不断飞行,战场四处转移,也根本容不下龙公来铺设战场,拘束对方。

龙爪狂卷,挥退诸多东海八转,龙公狠狠地冲撞到龙宫之上,然后伸出大手猛抓。

一瞬间,一对巨大的龙爪虚影,浮现而出,死死地扣住龙宫。

龙宫浑身剧震,将龙爪虚影震碎,又挣脱出来。

龙公叹息一声,这样的情况已经发生多次了。要收取龙宫极不容易,他先得厮杀一阵子,暂时击退所有竞争者后,获得一次良机来尝试收取龙宫。但龙宫本身乃是八转仙蛊屋,实力卓绝,收取难度很大。

龙公已经尝试了多次,都以失败告终。

“龙公太强了,怎么办?”

“当他真正动手之后,我们就没有任何机会靠近龙宫,更别谈收取龙宫!”

“只有龙公一人,屡屡出手,幸好这座仙蛊屋够强!”

“不能再这么下去了,龙宫的反抗之力越来越弱,迟早会被龙公收服的。”

东海八转蛊仙们各个带伤,相互密切交流,都想其他人出动底牌杀招,来对付龙公。

修行到八转层次,哪一个没有压箱底的手段?

这些仙道杀招威力绝对强大,或者神秘玄妙,但谁都不太甘愿暴露出来。谁也不敢保证,在如今这等复杂的战斗环境中,使用这些仙道杀招不会失败。

越强的杀招,反噬就越强。

龙公也扣着三气归来杀招,始终不用。

不是他不想用,而是动用此招风险极大,很可能就伤了自己。

他之前和紫山真君对战,交战之初就千方百计地试探,觉得对方没有特别针对的手段后,这才寻找到良机,施展出三气归来。

现在这种战况,龙公若是冒然施展三气归来,恰巧被敌人克制破除怎么办?

这些东海八转,人数很多,底蕴雄厚,鬼知道他们究竟藏着什么手段!

就像是搭载弓上的箭,悬而不射,令人心忧。蛊仙雪藏着杀招,往往也是一种强大的威慑。

光阴长河。

仙道杀招――夏扇!

方源仍旧是太古年猴模样,他施展杀招成功,双手向前虚抓。

一把巨大的长柄蒲扇顿时凭空出现,被他握在手中。

“冰塞川,闪开点!”

方源扭转腰腹,双臂肌肉贲发,猛地一扇。

呼!!飓风骤起,一路卷席向前,狂暴至极的飓风将三秋黄鹤台,乃至鲨流撬都卷入其中,牢牢束缚,飞出镇河锁莲大阵之外去。

但关键时刻,凤九歌、星野望都飞出仙蛊屋,扑向方源、冰塞川。

他们虽然不是修行宙道,但是镇河锁莲大阵并没有崩溃,在大阵之中宙道反而被压制,其余流派得到增强。

天庭派遣来的七转蛊仙,一个个都是精锐,这段时间他们通力合作,已经将镇河锁莲大阵修复得七七八八。虽然破洞创口还在,但已经缩小了许多。

“五行大法师,快打出你的洞玄光!”方源呼喝道。

年关门楼中立即传来五行大法师的声音:“短时间内我已无法催发第二记洞玄光了,并且这座大阵已经变化许多,再用之前的洞玄光效果不会有多好。”

仙道杀招――一曲之士!

凤九歌身躯一晃,晃出四位一曲之士,向方源围剿而来。

星野望着浑身绽射星芒,宛若刺猬一般,扑向冰塞川。

冰塞川和方源对视一眼,随后两人一飞冲天,舍弃各自的对手,轰击镇河锁莲大阵。

这无疑是最明智的战术。

只要镇河锁莲大阵崩溃,不管是凤九歌还是星野望,都得龟缩到仙蛊屋中去。

凤九歌、星野望连忙阻截,五行大法师操纵仙阵,仍旧困住清夜驾驭的恒舟。

激战中,镇河锁莲大阵摇晃的幅度越来越大,承受着方源和冰塞川的猛烈轰炸,裂痕越来越多。

不过,天庭一方被方源卷走的三秋黄鹤台、鲨流撬再次参战,方源和长生天一方逐渐被压入下风。

“方源,到现在你还不肯信任我吗?”局势不利,冰塞川急得大吼。

方源眼中精芒一闪,冷哼一声,主动靠近冰塞川。

冰塞川双臂一振,一道球形光膜将他自己和方源罩住。

这是他的仙道杀招,用于防御。但凡打在光膜上的事物,不管是生命还是杀招,都会静止住时间。

“轰破它!”凤九歌、星野望同时大喝。

三秋黄鹤台、鲨流撬亦爆发出攻伐杀招,齐齐轰击冰塞川的光膜。

但凡落到光膜中的杀招都因时间静止,而停滞住。一时间,整个光膜表面就像是积累了一层色彩斑斓的烟火,绚丽无比。

冰塞川咬紧牙关,身躯微颤,支撑着光膜。

他虽然是八转巅峰战力,但是在这大阵之中,宙道受到压制,同时承受两座仙蛊屋以及两位八转蛊仙的狂轰滥炸,十分吃力。

方源得此良机,立即散去太古年猴变化,催动出阎帝。

战斗至此,他这才有了时机,转换战斗方式。

凤九歌、星野望的战斗经验实在太丰富,根本没有留给方源转化的时机,经常被打断。

方源化身阎帝之后,立即开始催动杀招。

他的气势迅速升腾,从弱小变得强盛,又从强盛发展到恐怖慑人的程度。

凤九歌、星野望满脸凝重,警惕非常。

战至此刻,方源终于得空,成功地施展出了压箱底的手段。

仙道杀招――落魄印!

“注意!”冰塞川咬牙,猛地撤去防御杀招,冲在最前面。

没有了阻碍,之前停滞的攻势如潮水般,向冰塞川、方源狂卷而来。

冰塞川顶在前方,硬是为方源开辟出一条道路。

方源一边手持着落魄印,蓄势待发,一边顶住余波,冲向星野望。

星野望爆退,凤九歌绕向右方,从侧门攻向方源。

方源忽然冷冷一笑,身形猛地拔升飞高,落魄印飞射而出,正中镇河锁莲大阵。

大阵的运转陡然停住,随后内里数位七转蛊仙当场身陨。

冰塞川顿时感觉浑身压力骤然一松,哗哗哗,大量的光阴长河的河水顺着缺口,开始往阵内倒灌进来。

随后,咔嚓嚓的一连串的脆响,镇河锁莲大阵一块块地分崩离析。

五行大法师对此吃惊不已:“方源这记落魄印居然直扑大阵中枢,摧毁了核心仙蛊,他究竟是如何推算出来的?”

轰!

凤九歌击中方源。

方源的主要注意力都放在破阵上,只能硬挨凤九歌猛烈一击。

饶是阎帝状态,方源也连续吐了三口鲜血,大量的魂道蛊虫死亡,数只仙蛊受伤。

没有了大阵相助,战斗环境迅速改变,方源和冰塞川再次占据上风。

双方又缠斗片刻,镇河锁莲大阵只余下三分之一,剩余的七转蛊仙们拼命死撑,但距离大阵的全面崩溃只差一线,他们毫无希望。

“掩护我!”方源对冰塞川道。

冰塞川再次施展出球形光膜,罩住方源,方源趁机撤销阎帝,又施展出太古年兽变化。

“掩护我!”凤九歌同样呐喊。

星野望以及两座仙蛊屋都围绕在凤九歌的身边。

仙道杀招――大同风!

凤九歌冒巨大风险,成功地施展出天下第一风!

冰塞川额头冒汗,对方源喊道:“不好,我的防御杀招绝抵挡不住大同风,我们快撤!”

但出乎两人意料,大同风没有扑向防御光膜,而是袭击了石莲岛。

在方源惊愕的目光中,石莲岛被大同风吞没,然后一块块碎裂,被同化成风。

大同风越刮越大,谁要上岛,就要面对这场毁天灭地的飓风!

“凤九歌你!”冰塞川气急败坏。

方源的脸上亦是带着一丝铁青。

“我们得不到的,你们也别想得到。”星野望仰头狂笑。

------------

\end{this_body}

