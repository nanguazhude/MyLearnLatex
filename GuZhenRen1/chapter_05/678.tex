\newsection{不胜不负}    %第六百八十一节:不胜不负

\begin{this_body}

%1
大同风在光阴长河中肆虐。 .更新最快

%2
凤九歌催发大风歌的时候,会引出狂风呼啸的声音。但当真正的大同风刮起来的时候,非常平静,没有一点声音发出。

%3
悄无声息的大同风,却令在场的所有蛊仙们感到一股彻骨的冰冷。

%4
石莲岛不断地被大同风侵蚀,最终彻底消失,大同风越刮越大,起先规模就不小,摧毁了石莲岛之后,体积膨胀数倍,变得撑天拄地一般。

%5
不过,它在光阴长河中却是无法逞威,尽管有不少的河水被同化成风,但整条光阴长河仍旧是岿然不动,损失的河水比九牛一毛还要稀少。

%6
大同风虽然厉害,但是对于整条光阴长河而言,仍旧是显得过于渺小,仿佛是老牛身上的小虫。

%7
所有的蛊仙,以及仙蛊屋都小心翼翼地远离大同风,哪怕凤九歌也不例外。

%8
“石莲岛真的被摧毁了!”方源大感惊疑,他之前想要突入进岛,被石莲岛的防护力量排斥出去。没想到大同风下,石莲岛毫无反抗之力。

%9
“难道说,刚刚这座大阵起到了关键作用,压制住了石莲岛?还是石莲岛本身已经是强弩之末,抵挡不住大同风的威能?”

%10
方源望着镇河锁莲大阵,这座大阵仍旧在崩溃当中。

%11
凤九歌心情也十分沉重。

%12
他虽然摧毁了石莲岛,但在他看来,自己也是失败了。

%13
天庭为了寻找红莲真传,付出了难以想象的物力、人力,好不容易寻找到了一座石莲岛,用镇河锁莲大阵困住。没想到被方源联合长生天的力量,阻截住了。

%14
凤九歌不得不摧毁了石莲岛,让之前一切的努力都打了水漂。

%15
对于他的这番举动,谁也不好指责什么。

%16
因为战局很明显!

%17
镇河锁莲大阵吃了方源一记落魄印,崩溃是注定的了。大阵一破,凤九歌、星野望等人都只有乖乖地回到仙蛊屋中去。

%18
如此一来,天庭一方战力必会暴降。

%19
反观方源、冰塞川在光阴长河之中,却是如鱼得水,失去了大阵牵制,战力会再次上涨。

%20
敌强我弱,凤九歌没有信心拼斗下去,迟早石莲岛会被方源一方抢夺走。所以,趁着还有机会,凤九歌便壮士断腕,悍然将石莲岛摧毁。

%21
“好决断。”方源眼中厉芒一闪即逝,不吝赞叹。

%22
冰塞川长叹一声,脸上流露出一股颓丧之意。

%23
他还寄希望于红莲真传,想要从中获得毁灭宿命蛊的某种捷径,但现在方源空有春秋蝉这个钥匙,石莲岛却被凤九歌摧毁了。

%24
他大失所望,并不知道方源之前早已经暗中收获了一份真传!

%25
这个秘密,方源当然不会告知长生天,天庭也被蒙在鼓里。

%26
轰隆隆。

%27
一连串的轰鸣声中,残缺不堪的镇河锁莲大阵彻底崩溃,大部分的仙蛊都因此而毁,主持的蛊仙纷纷陨落。

%28
长河倒灌,激起惊涛骇浪,开始恢复光阴长河的旧观。

%29
“我们撤!”凤九歌咬了咬牙,开口道。

%30
战局对天庭相当不利。

%31
尽管他们拥有三位八转蛊仙,但在这种情况下,就算是龙公来了也不好使。

%32
凤九歌等人都不是宙道蛊仙,只有龟缩到仙蛊屋中去,靠着宙道仙蛊屋对战。

%33
偏偏这些宙道仙蛊屋的层次也只有七转,若是有一座八转级数的宙道仙蛊屋,天庭一方还有取胜的希望!

%34
天庭底蕴虽然雄厚至极,但也不是没有极限的。

%35
首先,黄史上人死后,宙道的八转蛊仙天庭是拿不出了。尽管组建出了四座宙道仙蛊屋,但也只是七转层次。想要在短时间内组建出一座八转宙道仙蛊屋,天庭也无能为力!

%36
其实紫薇仙子的布置,并没有什么问题。

%37
尽管方源收获了红莲真传,手中的杀招都改良了大半,大批仙蛊也升炼上去,战力上涨许多。但单靠方源一人,要抢夺红莲真传非常困难。

%38
天庭万万没有料到的是,长生天会忽然和方源联手。

%39
要知道方源摧毁了八十八角真阳楼,长生天和方源之间有着深仇大恨。这种仇恨不是简单的仇恨,方源不只是摧毁一座八转仙蛊屋那么简单,而是摧毁了巨阳仙尊的布置,黄金一族的荣耀象征。

%40
“长生天居然会主动放弃仇恨,和方源联手。这是冰塞川所带来的改变吗?”凤九歌回到三秋黄鹤台中,望着冰塞川,眉头紧锁。

%41
天庭的三座仙蛊屋在飞速后撤。

%42
方源狂啸一声,变作太古年猴,追杀出去。

%43
冰塞川和他并肩作战。

%44
两人攻势猛烈,对准天庭的三座仙蛊屋狂轰滥炸。

%45
天梯虽有三位八转,此刻却只能默默困守,硬挨对手的猛攻,一味奔逃。

%46
今古亭终于挣脱了冰塞川的杀招,从光阴长河中飞升而出,脱离了河面。

%47
三旬子都憋着一口气,想要给方源等人好看,结果却看到己方三座仙蛊屋仓皇而逃,镇河锁莲大阵彻底崩溃,主持大阵的七转蛊仙们几乎都死绝了,仅有的几位在光阴长河中挣扎,也离死不远。

%48
“快撤!”凤九歌传音。

%49
三旬子齐吞口水,连忙加入奔逃的队伍当中。

%50
方源和冰塞川,还有五行大法师驾驭的年关门楼,一路追杀。

%51
这四座仙蛊屋都很坚固,并且相互掩护,滑不溜手。

%52
方源和冰塞川却是越战越强,在这光阴长河当中,他们能发挥出远超正常的战斗力。

%53
“光阴支流就在附近,我们就快要逃出去了!”清夜呼喝,令天庭蛊仙们士气一振。

%54
“想走?没那么容易!”方源狞笑,猛地拉近距离。

%55
“你们撤,我们来阻敌!”三旬子带着觉悟,驾驭着残破不堪的今古亭,主动留了下来。

%56
在三旬子的舍命拖延之下,其余三座仙蛊屋顺着光阴支流,成功撤离出去。

%57
而今古亭被方源摧毁,三旬子也尽数陨落。

%58
没有留下任何一位八转蛊仙,方源心中满是遗憾,这种良机可是相当罕见的。

%59
方源不想追杀出去,在五域外界,凤九歌等人已经可以自由出手。

%60
他打扫战场,收拢仙窍,一一吞并。

%61
冰塞川、五行大法师却不出手,只是在一旁看着。

%62
这些阵亡的七转蛊仙,数量不少,因为没有加入天庭的资格,所有都有着实打实的七转仙窍。

%63
方源吞窍之后,收获很大。

%64
“走吧,我们去东海。”收拾了之后,方源主动提议。

%65
长生天一方在红莲真传争夺战中,展露出了充分的诚意,暂时会是可靠的盟友。

%66
双方的目的是一致的,都不想天庭彻底修复宿命蛊。

%67
天庭若有了宿命蛊,完全可以重现曾经的荣光,对其他四域都有巨大的威胁。

%68
在光阴长河中又停留了一会儿,方源和冰塞川终于确定,红莲真传真的被摧毁了。虽然他们没有夺得真传,但也阻止了天庭得手,算是不胜不负的结果。

%69
很快,方源等人便出了光阴长河,前往东海阻击龙公。

%70
哪知半路上却得到消息,东海那边竟然尘埃落定,得出了结果。

%71
这场龙宫的争夺战,最终的获胜者便是龙公!龙公本身战力卓绝,力压东海诸多八转,随后,天庭新晋八转蛊仙陈衣携带仙蛊屋支援。关键时刻,东海散仙阳骏忽然相助龙公,令龙公成功镇压住了八转仙蛊屋的反抗,将其收取。

%72
不过龙公本身也因此身负重伤。

%73
最后关头,东海八转中的几人,终于再不留手,施展出了压箱底的手段。

%74
可惜的是,为时已晚。龙公夺得仙蛊屋后,毫不恋战,抽身就走。

%75
“没想到这八转散仙阳骏,竟然被天庭招揽!”五行大法师摇头不已。他也曾经是散仙,但如今成为了长生天的一员。

%76
“未必是这样。说不定,阳骏本身就是天庭在东海的棋子。”方源想到了北原的凤仙太子,凤仙太子就是天庭一方安插在北原的八转蛊仙,不过身份一直没有暴露。

%77
方源看了冰塞川一眼,决定还是先将这个秘密隐瞒。

\end{this_body}

