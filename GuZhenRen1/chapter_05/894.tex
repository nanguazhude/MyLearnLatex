\newsection{大炼仙蛊}    %第八百九十七节:大炼仙蛊

\begin{this_body}

方源之前就有一个组建三座仙蛊屋的宏大计划。

如今,他已经成功搭建了万年斗飞车、煮运锅,还剩下最后一座仙蛊屋的目标没有完成。

虽然他在这个过程中,又收获了豆神宫、龙宫这些仙蛊屋。但这些仙蛊屋都被安置在分身手上,是方源布局天下的关键,不是镇压场面,就是以此为支点撬动大局,都有重要的作用。

最关键的是,豆神宫、龙宫并不符合方源的设想。

在方源的原计划当中,他还需要一座仙蛊屋以防守为主。

万年斗飞车擅长的是速度,弥补了方源在寻常腾挪方面的短板,同时也重点照顾到了光阴长河这块战略要地。

煮运锅是辅助,当初设计时秦鼎菱还未出世,方源还打算用它来欺负运道薄弱的天庭,扩大自己的优势。

剩下的第三座仙蛊屋,方源用于防守和疗伤。宿命修复大战极其激烈,真正参战后就很难有喘息之机。方源需要一个在战场上能够有效防御,在必要时给他争取宝贵的喘息之机的仙蛊屋。

这座仙蛊屋甚至能够对尊者手挡,有着一定的抵御作用!

“土道厚重,本来就擅长防御。我用这些土道仙蛊构建第三座仙蛊屋,可谓恰到好处。”方源思索了一番后,就下定了决心。

方源此时还不知道,吴帅分身那边通过探索梦境,已经有了土道仙蛊屋的搭建之法。

这座仙蛊屋名为安土重山堡,正好符合他的当下需求。

只是方源和分身之间,境界虽然互通,但是各自的记忆却需要相互沟通,方能获悉。

最终,方源接取的是一个整合两大海域的大型任务。

“你们谁要和我一道来?”方源没有急着去,而是转身询问。

众仙面面相觑。

“我。”庙明神率先反应过来,第一个走上前。

他身后的鬼七爷等人,自然趋附着跟上来。

“你们呢?”方源又问沈家蛊仙。

沈家蛊仙们呆了呆。

他们之前还和方源翻了脸,这么短的时间里难以将心理转换过来。

沈家蛊仙们不像庙明神一伙人,后者是尝到了和方源合作的好处。而他们和方源合作的第一个任务就出问题了。

沈笑摇摇头:“多谢大人美意,我们还是先看看吧。”

沈笑委婉拒绝了方源,对于他的这个意思,其他的沈家蛊仙也没有反对。

方源不以为意,又微笑着看向另外两人,开口道:“任修平仙友?童画仙友?你们的意思呢?”

任修平阴沉着脸,没有说话,只是摇了摇头。

童画则像是被吓了一跳,没有任何动身的想法。

对于他们俩的态度,方源也有预料。毕竟刚刚他还收了他们的仙材库存。

从这点上,亦可看出沈从声的手腕。

沈从声命令任修平、童画,将各自的仙材转交给方源,看似将关系搞僵,实际上却是令这两人不得不更加依附沈家。

毕竟,他们还希望着出了龙鲸乐土之后,沈家可要依约,填补他们的仙材损失呢。

同时,这也是沈从声敲打任修平、童画的一种手段。让他们明白,谁才是主,谁才是次。在龙鲸乐土中你要听我沈家的,出去了就更要听从!自上了我沈家的这条船,想要下来?呵呵,可就没那么容易了。

方源当然知道沈从声的这份图谋,只是没有揭破。

任修平、童画让沈从声收下,也无所谓,这不是方源的核心利益。

五域界壁还存在着,但各大超级势力都开始拉拢和积蓄力量。房家收纳了败军老鬼、鹰姬,沈家接纳任修平、童画有什么问题?长生天更过分,直接整合北原蛊仙界,要以黄金超级势力为主题,收编所有的北原散修、魔仙。

至于中洲,早就做到了这一点。中洲十大古派身为天庭下宗,牢牢掌控中洲大局。就算有一些蛊仙和超级势力,也基本上都依附着十大古派,受到他们的掌控、监察。

最终,方源带着庙明神一伙人,传送到了任务地点。

身悬高空,俯瞰而下。

众仙便见到这两片海域,一片蓝、一片绿,蓝海高耸,绿海低垂。大量的海水从蓝海垂落,滚落到绿海当中。

两股海水却不相容,在海域交接之地,蓝绿海水相互碰触,发出嗤嗤之声,还有海量的云雾升腾不休。

在两片海域相交的部分,鱼虫不存,生命罕见,几乎是一片绝地。

显然是生态方面出现了大问题。

方源叹息一声:“要整合两片海域,并不简单,要顾及到各方各面。我等不妨各自探索,搜集情报。之后再汇总讨论,商讨出一个方案来。”

庙明神点头:“此言在理。”

方源看了一眼庙明神,对他笑了笑。

虽然自己身份暴露,但庙明神居然如此配合自己。他能够迅速改变自己的态度,以下对上继续跟随方源,这番应变足可见他的器量。

群仙分散开来,四下飞行,各自侦查。

惟独方源留在原地,一边催动侦查手段,一边分出心神,探入至尊仙窍。

他此番最大的收获,就是得了悔蛊。

上一世,他依照悔池的方案,以悔蛊为核心,构建了炼道大阵。

这一世却不需要这么麻烦。方源吞并了整个琅琊福地,拥有了极端优异的长毛炼道大阵,他现在要做的只是将悔蛊增添到这个大阵中去。

大阵当然需要调整,而方源早已提前推算好了。

此刻只是需要他亲自坐镇,将悔蛊安插在空余的核心位置上即可。

“大人,方源乃是当世魔头,我们和他是不是走的太近了点?”侦查的过程中,鬼七爷主动飞到庙明神的身边,一脸担忧之色。

庙明神拍拍鬼七爷的肩膀:“老鬼,你其实更想问我为何用这种态度对待方源,几乎成了他的下属吧?”

鬼七爷不好意思地笑了笑:“大人明鉴。”

庙明神面容微肃,叹息一声:“我也是逼不得已啊。龙鲸乐土我是首次进来,却出现了意外。沈家发现了这片地方,你说,他们会放任这片宝地,任由我们今后再次前来探索吗?”

“依照超级世家的秉性,当然是要独吞!”鬼七爷不假思索地道。

“正是如此啊。”庙明神眼中精芒一闪即逝,“我们此番出去,必然会被沈家打压、逼降。逼降不了,就会剿杀。但现在我们除了依附沈家,还有一个选择。”

“大人是指……方源?我懂了,大人深思远虑,是我愚钝。”鬼七爷恍然大悟。

身为散仙,又知道龙鲸乐土这样一块宝地,将来沈家必会出手对付。一个任修平就是强敌,更何况超级势力沈家呢?

庙明神选择依附方源,就是想借助他的声威来自保,尽量保住散仙的自由身。

将来沈家对付他,就要顾及方源的感受。

实在不行,他就舍弃散修身份,依附沈家或者方源。

这就赢得了一个左右腾挪的空间。

不像任修平、童画两人,这么早依附沈家。没有外在压力的沈家,自然将他们吃得死死的。沈从声叫他们交出仙材来,他们能不交吗?敢不交吗?

等到群仙汇合到方源身边,方源已经将悔蛊安置妥当了。

至尊仙窍中,毛民蛊仙们早已聚集,磨刀霍霍,斗志高昂,打算大炼特炼一番。

方源的宙道分身则在整理规划仙材库藏,按照眼下的仙材,再对仙蛊方进行一些微调改良。

方源上一世需要自己亲历而为,这一世有了整个琅琊派做帮手,几乎不需要他插手了。

因此,节省出来大量的时间。

在宿命大战越来越近的情况下,这些时间显得极其宝贵!

------------

\end{this_body}

