\newsection{焰云巨柱}    %第八百九十八节:焰云巨柱

\begin{this_body}

至尊仙窍。

长毛炼道大阵嗡嗡作响,阵内火光冲天,一股股热浪四处冲击。

一只六转仙蛊沐浴在火光之中,悬浮半空,表面宛若蜡般逐渐融化。

“稳住!”琅琊地灵主阵,口中低呼,“开仓门,放仙材。”

主持另外阵眼的毛民蛊仙立即打开仓门。

于是,一团团狗屎从仓门中缓缓飞出。

这些狗屎都是仙材。

有的闪烁着星光,仿佛镶嵌着诸多蓝钻,乃是星荒犬的屎。

有的狗屎表面尖刺耸立,古怪狰狞,乃是天残犬的粪便。

也有的狗屎洁白无瑕,宛若白玉,则是玉像混白犬的排泄物。

星荒犬乃是荒兽,天残犬、玉像混白犬则是上古荒兽。

狗屎缓缓飞入大火之中,在火焰中迅速融化。有的化为烟雾,有的沦为液体。

大阵转动起来,三位毛民蛊仙接连出手,相互配合,处理这些狗屎仙材。

先是烟雾弥漫,附着到六转仙蛊周围,随后液体横流,汇聚一体。

琅琊地灵在炼蛊后期出手,稳定得可怕。

然而就在快要成功的时候,当最后一份仙材融汇进来时,意外忽然发生了。火焰陡然变色,内中的七转仙蛊雏形迅速崩溃。

琅琊地灵吃了一惊,连忙再催大阵。

悔蛊发挥作用,阵内火焰的颜色旋即又变了回来,不仅如此,七转仙蛊雏形也恢复如初。

所有的损失的只是最后那一份仙材而已。

琅琊毛民们迅速稳定情绪,继续炼蛊。

三炷香的时间后,阵内的火焰徐徐消散,原本的六转仙蛊成功升炼成了七转。

七转——狗屎运!

毛民蛊仙们十分激动,琅琊地灵也是满怀感慨。

原本他们的炼道实力,便傲视天下。现在有了悔蛊,简直是蛟龙飞空,优势扩大得一发不可收拾。

与此同时,龙鲸乐土。

方源盘坐在云端,和蜂将等人看着庙明神飞上飞下。

两大海域的生态,已经调整得差不多了。最后的难关是两大海域之间,还有一片破碎的空宇。

庙明神正在努力地打消这片空宇,只要抹平了它,两大海域便能真正和谐的统一在一起。

庙明神乃是宇道蛊仙,对付这片破碎空宇,正是他的拿手好戏。

方源等人也便乐得清闲。

“哦?狗屎运仙蛊也升炼成功了。进展不错。”方源察觉到了至尊仙窍中的动静。

趁着他本体进行功德任务的时候,至尊仙窍中已经升炼成功了三只仙蛊。

进展可以说是非常迅速!

寻常蛊仙炼蛊,都是以年为单位计算,全程战战兢兢、小心翼翼,风险极大。

方源此时的处境完全不同。

琅琊地灵、毛民蛊仙、长毛炼道大阵、悔蛊,以及丰富的仙材,都让他炼蛊时游刃有余。

升炼成功的三只仙蛊,都隶属于运道,是方源拆掉了煮运锅而得。

运道是当前的重点。

刚刚升炼成功的狗屎运仙蛊,更曾是巨阳仙尊的本命仙蛊,有着极其厉害的辅助能力。

只是之前一直都是六转层次,登不上层面。如今升到七转,终于可以勉强一用。

方源如今修为八转,对于仙蛊的要求也随之提高,需要合适的高转仙蛊。六转着实太低了,七转勉勉强强,八转才是合适恰当。

但方源要炼制八转仙蛊,实在是无力。投资太大,风险太高,很可能竹篮打水一场空。

毕竟他不是自在修行,外部环境十分恶劣,若是不阻止天庭修复宿命蛊,他的处境更将每况愈下。

眼下,七转仙蛊也勉强能用。

况且升炼出来的这些七转运道仙蛊,都还要再次组建成煮运锅的。七转的煮运锅也能够影响到八转了。

“咄!”庙明神忽然轻喝一声,打断了方源的思绪。

只见庙明神猛地喷涌出一股湛蓝光晕,光晕渲染弥漫,所至之处,破碎空宇缓缓抚平,空间一统,再无差异。

“大人好手段!”蜂将等人交口称赞。

做成了这一步,整个大型任务也宣告完结。

“庙仙友手法高超,我不及也。”方源也不吝夸赞。他当然也有宇道仙蛊,不乏宇道手段,更有其他流派的杀招能够模拟出宇道威能来。

但真正和庙明神比起来,就算方源达成眼前的效果,消耗也比庙明神多出许多。

毕竟术业有专攻,庙明神是专修此道,而方源在宇道上一直投入很少,进展微小。

庙明神连忙拱手:“大人谬赞,明神忏愧。”

方源又客气几句,众仙再次回到功德碑下。

这一次任务,方源仍旧拿了大头。庙明神其次。

方源的功德排名涨了上去,再次登顶。

庙明神等人看得已经没有了脾气。他们屡屡看到方源的排名上蹿下跳,像猴子似的。第一和倒数第一是这只猴子最常待的两个位置。

方源却是目光微凝。

到了这一步,其他蛊仙的功德也逐渐积累上来了。方源的功德若再跌下去,想要爬到第一名,可就不那么容易了。

继续接取任务,完成任务。

方源的功德不断上涨,稳步提高。

庙明神等人始终依附着他。方源运用全流派兼修的优势来解决主体,而他们替方源擦屁股。

这种合作方式让方源始终保持着榜首的位置,并且越加牢固。

庙明神一伙人也因此受益,功德排位渐渐甩开沈家蛊仙等一帮子人,成为第二梯队。

而第一梯队当然只有方源。

沈从声到底是没有求助方源,在规定的时限里完成了多个中型任务,重新回到了正轨。

他的运气不错,接到了一两个有关音道的任务。至于其他任务,虽然他专修音道,但也有不少手段能够模拟出其他流派的威能妙用。

沈从声到底是八转,修为和手段都放在这里呢。

当沈从声的排名升上中段后,方源的劫运坛再次组装完成。

七转劫运坛!

巨阳仙尊亲自创造的极品仙蛊屋,已经可以影响八转。

方源第一件要做的事情,就是查看自身的运势。

轰!

就听见耳畔似乎传来一声轰鸣,方源仰头,便见一道火云巨柱,熊熊燃烧,冲霄而起,简直要撑天踏地。

他的气运宛若巨柱,烈焰蒸腾,焰火血红。在巨柱表面,多多火焰的末梢,又带着一抹抹的黑意。

“我实力高达八转,战力可敌龙公,又拥有大量仙蛊、仙蛊屋、杀招,因而气运磅礴浩大至极。”

“焰火血红,代表杀戮。焰尾漆黑,意喻死劫。死劫的黑气沾染巨柱表面,向内浸染,说明死亡来源于外部。当是天庭无疑。”

“天庭现在还未修复宿命蛊,一旦修复成功,恐怕黑色死气要膨胀许多倍数,最终将我整个气运巨柱都彻底侵蚀。”

方源再细看,又将火焰巨柱当中,还有种种影像随着焰火燃烧而不断浮现。

时而是一座红莲,时而是一只白鹤,时而化为兽影,时而转为土壤……

方源正要继续深究,视野陡然模糊起来。

旋即,他身心狠狠一震,咽喉一甜,差点吐出一口鲜血。

方源迅速疗伤,再看头顶,只见一座煮运锅静静漂浮着,锅内漫漫气运,锅外一道巨大的裂缝,正在不断地向外渗透气运。

方源暗自苦笑。

他的气运实在太过强大,就算是七转的劫运坛,也只能勉强一窥。

窥探的时间久了,他自己不仅要遭受反噬,而且劫运坛也会损伤。

刚刚搭建好的劫运坛,还得回炉修复。裂缝巨大,伤势不轻,若是放任不管,裂痕会越发扩大,因此还得立即着手修复。

观察自家运势,令方源忧喜参半。

喜的是自身气运极佳,这也并不奇怪,本身他底蕴极其雄厚,其次更有运道造诣。

忧的是隐患已经显现而出,若不阻止宿命蛊的修复,方源的危机会越来越大。其次火云巨柱虽然燃烧剧烈,但本身也有短暂之像。

这类气运灼灼燃烧,虽然短时间内蛊仙几乎无往而不利,但并不长久。燃烧殆尽后,必定会有一段气运低迷的时期。

不过这是更远的未来了。

方源推算了一下,至少在中洲修复宿命蛊的时候,他的气运会保持这种状况,甚至还会进一步增强。

方源继续接取任务,至尊仙窍中诸多炼道蛊仙则开始升炼力道仙蛊。

功德碑上开始出现清除黑火的任务。

沈伤逃离悔哭海域,仍旧疯魔,无意识中祸害周围环境生态。

方源对这古怪的黑色火焰非常感兴趣,立即接取,前往探查。

结果仍旧毫无进展。

黑火非常顽固,哪怕只是小小一团,也能在杀招下不断变化形态,支撑很长一段时间。

剿灭黑火的任务陆续出现,被方源、沈从声等人联合一起,共同完成。

“你有没有发现,最近这种任务越加稀少了。”这一天,沈从声一边对付黑火,一边和方源交流。

“贵族先祖沈伤恐怕是要恢复神智了。按照这些任务的地点,我们也可推断出他的大致路线。”方源分析道。

“是啊。”沈从声脸上涌现一抹期待之色。

就在这时,天边忽然飞来一位八转蛊仙。

不是沈伤,又是何人?

“先祖,太好了,你终于恢复了神智。我已经和方源谈妥,联手共存了。”沈从声大喜,连忙迎接过去。

沈伤飞来,对方源点点头,他的神情颇为憔悴,眉头紧皱,语出惊人:“情况有些麻烦,我身上的问题不小。在解决问题之前,我需要重新封印我自己。方源你善于阵道,还请你出手,再建大阵,必要时将我封印。”

\end{this_body}

