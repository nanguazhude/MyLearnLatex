\newsection{黑凡四珍(下)}    %第一百零八节:黑凡四珍(下)

\begin{this_body}

炼化仙蛊,真元就不行了,得用仙元。

这时,方源抬头看了一眼黄钟天灵,想了想,还是直接耗用青提仙元,当场炼化这蛊。

黄钟天灵没有任何反应,光从表面,方源也看不出它什么心态。毕竟外形是一口钟,不是人形,兽形。

炼化的过程十分顺利,这只形如蚁后的七转仙蛊,没有任何反抗,反而十分配合。

不过到底还是七转,方源的青提仙元消耗了不少。

“眼下,我的青提仙元储备又薄弱了。”方源暗中提醒自己。

这一次他来继承黑凡真传,消耗了大量的青提仙元。主要是战斗中的耗费。尽管有万我这种仙道杀招,但其余的剑遁、飞剑等仙蛊,都是七转级数,剑道仙级杀招也是多以七转仙蛊为核心。

尽管战斗烈度不强,但青提仙元的消耗着实不小。

“看来这次回去,还得赶紧将仙元储备提起来。否则的话,心中没底啊。”

片刻之后,方源将这只七转仙蛊炼化成功,再次纳入自家仙窍。

光壁中的阴影,只剩下两团了。

最下面的一个阴影,因为体型太小,方源将其留在了最后。

他伸手探入光壁,将右上角处的阴影拿取出来。

这团阴影像是一个石块,但方源首次接触的时候,被刺激了一下。

这团阴影很是冰冷。

方源拿出来一看,果真是一块冰。

冰层的上半部分,封印了一只蛊虫。

很自然地,方源的注意力,都被其吸引过去。

虽然被冰块封印,一丝气息都没有。

但方源却双眼骤亮,认出来,这是仙蛊“年”。

年蛊!

它形如瓢虫,大拇指头般体型,全身都是黑色。但在背甲上却有金色的纹路。

金色纹路十分炫目,闪闪发光,纹路间描绘出一个野兽张口怒吼的画面,狰狞的利齿给人深刻印象。至于是什么野兽。却不分明。

“年蛊啊……”方源感慨一声。

这可是宙道仙蛊当中,十分出名的存在了。

方源对其印象很深。

因为它很独特。

蛊仙消耗仙材,可以对一只年蛊,进行不断的平炼。

升炼、逆炼是炼道的主流,平炼比较罕见。

能够平炼的蛊虫。也比较罕见。[www.qiushu.cc 超多好看小说]

年蛊就是其中之一。

凡级的年蛊,都是十年以下。有一年蛊,两年蛊,五年蛊等等。

年蛊上升到六转,就成为仙蛊,世间唯一。六转的年蛊,最少也是十年。

十年的仙级年蛊,可以用其他仙材加以平炼,炼成的年蛊可以成为二十年蛊,三十八年蛊等等。最高是九十九年蛊。

百年蛊就是七转程度,要从十年蛊炼成百年蛊,那就不是平炼,而是升炼。

升炼的难度和失败概率,是平炼的数十倍,甚至上百倍。

千年蛊是八转层级,按照这个理论,万年蛊就是九转程度了。不过,历史上从未出现过万年年蛊。

八转的千年蛊,倒是存在过的。

年蛊的运用。也很特别。

它会根据蛊仙使用的次数和量,不断削减年份。举个例子,五十年蛊,被用了一次之后。就会成为二十年蛊。

年蛊的使用,要注意分寸,不能太过。比如一千年蛊使用之后,成为九百年蛊,那就是七转仙蛊跌落成了六转。要再升炼上去,相当费事。又极具风险。

严格来讲,年蛊是一种消耗性的仙蛊。

只是它并非是至尊仙胎蛊这种一次性消耗掉的蛊虫。

它可以被消耗,也可以再补充。

这在蛊虫当中,是很奇特的。

“原来黑凡手中,掌握着仙蛊‘年’。年蛊有个好处,就是不愁喂养。因为它的食料和春秋蝉一样,都是光阴长河的河水。可是,我该如何炼化它?”方源有些犯难。

眼前的年蛊,正被一块几乎透明的寒冰封住。

方源打量这块寒冰,发现这冰块似乎也很奇特。

它起到的作用它类似于化石或者琥珀,将仙蛊年冰封起来,使其陷入沉眠当中,加以保管。

“为什么,黑凡要将年蛊封存起来?等等,这是?”在观察的过程中,方源又有了新发现。

他发现冰块的下端,似乎还封存了一股水流。

“这是水吗?”

方源摇晃几下,果然更加清晰地见到水流的样子,而且还听到微微的响动。

冰中封水,实在是有点古怪。

方源之前的注意力,都被水流上方的年蛊吸引了。相比较而言,水流并不起眼。

但此刻,方源再观察时,他发现:这冰块很大,年蛊只是占据了上面一点,而水流却占据下方大半。给人一种感觉,好像是这冰块封印的重点不是年蛊,而是这水流!

“如此郑重其事,难道这水流是超越八转的九转仙材不成?”方源心中冒出一个猜测。

九转仙材,可以用来直接炼制九转仙蛊!

世间少有,价值极大。

一份九转仙材如此封存,似乎也说得过去。

方源寻觅不到解开冰块的法子,只得将其暂时搁置,用右手拿着。左手再探入光壁。

这一次,他伸向最后一个阴影。

拿出来一看,是一只五转的信道凡蛊。

里面记载的内容,却正是黑凡真传,价值连天了!

首先,是对其他蛊虫的介绍。

方源第一眼,就看到了关于蚁球的说明。

“原来蚁球中央的七转仙蛊,就是宙道仙蛊以后!”

七转以后仙蛊。

随后的内容,又重点说明了以后仙蛊的喂养问题。

以后仙蛊的喂养,也比较特殊。

这只仙蛊什么都吃。

吃的方式也很特别。

它自我产出黑蚁凡蛊,派遣它们出去,啃噬各种仙材,然后将这些营养汇集到自己身上。

“无论有什么仙材,都可以喂养以后仙蛊么。原来重点不在于仙材,而在于保存这些黑蚁凡蛊。”

这些黑蚁凡蛊虽然坚硬,能啃噬仙材。进行消化,保存营养。

但到底还只是凡蛊级数而已。

一旦被破坏得多了,就会让仙蛊以后饿到。仙蛊以后虽然能产出黑蚁凡蛊,但一个月只能产出一只来。黑蚁凡蛊太少的话。喂不饱仙蛊以后,让它饿死,可就糟糕了。

毕竟这只以后仙蛊,是不能自己进食的。

方源有种开眼界的感觉,他顺着再看下去。看到了关于冰块的介绍。

冰块本身是一种全新的八转仙材,由冰忍仙人独创。后者受到黑凡的邀请,才出手,为其封印了黑凡的两只仙蛊。

一只仙蛊,就是年蛊,只是六转级数。这个方源已经认出来了。

另一只仙蛊,就是那股水流。

这就是黑凡真传的重中之重!

它名为似水流年,高达八转!

八转宙道仙蛊似水流年。

八转蛊仙少有,八转仙蛊更是稀有,价值极大。很多八转蛊仙。手头上都没有一只八转仙蛊,苦苦追寻而不可得。

黑凡作为历史传奇,带领黑家走上霸主地位的枭雄人物,生平也仅有一只八转仙蛊似水流年。

方源的情况,是极其特殊的。

目前为止,他只是六转蛊仙。但他的八转仙蛊已经多达三只了。

一只态度蛊,一只慧剑蛊,第三只便是眼前的似水流年。

寻常的蛊仙,都是嫌仙蛊太少。方源的问题,是仙蛊很多。开发程度不高。有些仙蛊,比如慧剑,他还捉摸不透呢。

这只八转仙蛊似水流年的作用,就是产生年蛊。

蛊仙消耗的仙元越多。产生的年蛊就越多,甚至可以产生仙级年蛊。

当然,仙蛊唯一是大前提!

如今,这只六转的年蛊健在,除非方源将它用掉,彻底消耗。才可能从似水流年中,催生出仙级的年蛊来。

若是仙蛊年存在着,方源耗费仙元得到的年蛊,只会是大量的凡级年蛊。

方源继续看下去。

似水流年仙蛊有两项好处。

第一项,是它使用的条件不高。哪怕就算是六转蛊仙,也能催动起来。只是青提仙元耗费下去,得到的年蛊,没有七转、八转仙元得到的多而已。

第二项,是似水流年仙蛊容易喂养。它的食料也是光阴长河的河水,和年蛊、春秋蝉一致。

这就解决了方源的一个大难题。

八转仙蛊的喂养,可是个沉重的负担!

不过,似水流年仙蛊还有一个弊端。

信道蛊虫中的记载,阐述得很明确。

似水流年仙蛊一旦解开封印,透露出仙蛊气息来,就会不定期地吸引年兽来袭。

年兽,一种五域九天中都相当稀有的奇特猛兽,但在光阴长河当中却很多,很是常见。根据战力划分,它们中有荒兽、上古荒兽、太古荒兽不等。

年兽的食物,就是年蛊。

光阴长河是宙道蛊虫的天堂,里面有海量的野生宙道蛊虫。

年兽就在其中,捕捉年蛊吞食。

级数越高的年蛊,越是吸引年兽的捕猎。

事实上,仙级年蛊也能吸引年兽来袭。哪怕把仙级年蛊放在仙窍中,也不顶用。

因为正常仙窍中,都有光阴长河的支流。年兽们闻到年蛊的气味后,就从长河中,进入支流里,直接出现在蛊仙的仙窍之中。

仙级年蛊就已经十分吸引年兽,能够无限产生年蛊的似水流年,对于年兽的诱惑,是仙级年蛊的成百上千倍!

正是这个原因,才使得黑凡布置真传时,特意地将年蛊和似水流年仙蛊,都封印起来,不露出丝毫的气息。

至此,蛊虫方面,方源都了解了。

重点是下面的内容。

更明确地说,是仙道杀招!(未完待续。)

\end{this_body}

