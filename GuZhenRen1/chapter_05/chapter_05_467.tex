\newsection{搜刮战利品}    %第四百六十八节:搜刮战利品

\begin{this_body}

%1
光辉一闪,上古剑蛟又变回人形。

%2
尤婵、秦百合已经死亡,从此之后,东海蛊仙界的六大仙子只剩下四位。

%3
方源辣手摧花,仍旧面不改色,他对准两处地方,施展仙道手段。

%4
很快,两个仙窍就被他取出来,暂时封印住。

%5
方源继承影宗遗藏,其中就包含了焚天魔女的取窍手段,还有其他诸多杀招,都能对仙窍有效。

%6
来到东海之前,方源就此推算成功,有了手段可以暂时封印仙窍,随身携带着。

%7
所以,方源并不忙着对这两个仙窍怎样,他先收起仙道战场杀招紫宸断命后,然后唤出影无邪、白凝冰等人。

%8
为确保万无一失,影宗成员等等方源都揣在至尊仙窍里面,一同带到了东海。

%9
上古战阵四通八达!

%10
这个杀招原本有着缺陷,效果并不出众,现在也被方源改良修复了许多。所以此刻一经催动,就立即带着方源和其他三仙,离开了龙鱼海域,直接降临到白鹤海域。

%11
白鹤海域上,风平浪静。清风徐徐,蓝天白云,海天一色之间,无数白鹤,或是翱翔,优雅地舒展身姿,或是栖息在小岛上,以做休憩。

%12
方源没有停留,直接扑向海底深处。

%13
而其余影宗蛊仙,则按照方源的嘱托,开始收取周围的白鹤。

%14
方源进入海底,目光犀利,很快就找到了一座仙蛊屋。

%15
原来他连龙鱼海域都有限放弃,第一时间赶到白鹤海域,就是为了秦百合的六转仙蛊屋百合宫。

%16
秦百合离去之时,并未将这座仙蛊屋随身带走。一方面是和平太久,麻痹大意,另一方面也是为了留下仙蛊屋,镇压白鹤海域,防备不测。

%17
方源直接接近百合宫。

%18
百合宫打开门户,居然主动地将方源迎接进去!

%19
原来,方源已是变作秦百合的模样,见面曾相识杀招效果绝伦,代替秦百合镇守仙蛊屋的不过是她的一团意志而已。

%20
方源顺利进入百合宫中,接收秦百合的意志。

%21
秦百合的意志不疑有他,进入方源的脑海中后,顿时发现不妥。

%22
但方源早有准备,以自家脑海为战场,将这股意志很快禁锢起来。然后他开始借助这股意志,反过来争取这座百合宫。

%23
饶是方源智道仙蛊、智道手段不少,也前后花费了一天一夜的时间,才将百合宫成功收入囊中。

%24
至此,方源终于有了一座仙蛊屋!

%25
“只是这座百合宫,并不完善,很是残缺,威能极低,只有两只六转仙蛊为核心,乃是秦百合专门为了承载她的后宫妃嫔所制。”

%26
百合宫并不可靠,本身十分脆弱,相比较黑家的仙蛊屋黑牢,远远不及。

%27
方源控制之后,将其内部构造一览无余,又全部记下运转奥妙后,直接拆掉,得到大量蛊虫,两只六转仙蛊,一只木道,一只居然是食道,名为吃香。

%28
这有点出乎方源的意料之外。

%29
尤婵、秦百合身上的蛊虫,不提仙蛊,哪怕是凡蛊,都不会留给方源了。

%30
所以,组建成百合宫的这些蛊虫,就是方源在蛊虫方面的全部收获。

%31
吃香仙蛊形如木盒,体格小巧,宛若成年食指大小和长宽。它的作用很是鸡肋,只是帮助蛊仙吃取香气,填饱肚皮。

%32
“不过这世界上,绝无废物的蛊虫,只有废物的蛊师。”方源拥有食道传承,心念一转,就想到许多利用吃香仙蛊的方法。

%33
这只仙蛊虽然单独使用,效果鸡肋,但是和其他蛊虫一旦产生了配合,发挥出来的作用绝对值得期待。

%34
目前为止,方源在食道方面只有一只仙蛊,名为小吃。这一次多了一只吃香,很大程度上弥补了食道仙蛊不足的问题。

%35
这是一个令方源颇感惊喜的收获。

%36
方源拆了百合宫,将里面的妃嫔美人,一概杀死,魂魄留下。

%37
回到白鹤海域,影宗群仙仍旧在抓紧时间,搜捕海域中的资源。

%38
白鹤成群结队,数量极多。

%39
普通白鹤成千上万,荒兽白鹤更是多达二十一头,还有上古白鹤也有四头。

%40
耗费了不少功夫,方源等人不仅将这些白鹤都收拢起来,还在海底深处搜刮了其他资源。海底的资源主要有两种,一种是太史铁,金道、智道道痕汇聚的七转仙材,还有一种是蛋晶石,这是蛋人死后存留下来的生命精粹。

%41
蛋晶石数量还很多,可见白鹤海域在许久之前,是一处蛋人的庞大族群的栖息地。

%42
方源将这些都收入囊中后,又通过四通八达杀招,回到龙鱼海域。

%43
龙鱼海域中还有大量的龙鱼,规模远比方源原先掌握的,要多得多。

%44
方源全都收入自家仙窍,这一笔横财,直接让他的龙鱼数量规模,直接膨胀了几十倍!

%45
尤婵不愧是龙鱼生意的第一巨头,这方面的底蕴极其雄厚,让方源都暗暗吃惊。

%46
掏空了龙鱼海域之后,方源遥望天际,仍旧不见有人杀来。

%47
“天庭方面始终没有动静……枉我还在战斗中故意拖延时间,想要引出天庭的援兵。”方源吐出一口浊气,目光冰冷。

%48
他此行准备充分,就算是面对八转蛊仙,也不惧怕。

%49
天庭方面,能猜测出方源的身份,从而通过帮助尤婵,来阻击方源。同样的,方源也可以推算出天庭在背后推波助澜。

%50
方源忽然杀到这里来,既是一场针对尤婵、秦百合的突袭,又是对天庭的一种试探。

%51
但此次试探之后,天庭方面居然毫无动静。

%52
这个结果让方源感到一丝意外。

%53
“天庭方面,居然没有派遣人手过来?是紫薇仙子估料不到,还是反应不及时?亦或者……天庭方面的人手并不是那么充足的。”

%54
方源目光闪动,他老奸巨猾,一下子就猜到了天庭方面的窘困之处。

%55
天庭方面人手的确比较缺乏。

%56
这点,其实方源早在西漠逃亡的路上,就隐隐感知到了。这一次只不过是更加确认!

%57
既然天庭蛊仙没有来,方源也不会傻傻地等待。动用四通八达离开龙鱼海域之后,他就在东海的某个无名海域,将尤婵、秦百合的仙窍种下去。

%58
一个七转水道仙窍,一个七转木道仙窍。

%59
趁着仙窍转化为福地的良机,方源分别进入其中。

%60
尤婵福地中,地灵认主的条件,直接是要杀死方源。方源相当狡猾,改变了容貌,伪装身份进入其中。

%61
仙道杀招血光镇灵!

%62
尤婵地灵被方源镇压,水道仙窍很快就被方源吞并。

%63
而百合福地中,地灵认主条件,却是和百合宫有关联。

%64
方源不管这些,再次用血光镇灵,将地灵镇压。

%65
不过木道福地,他却不够资格吞并,只能暂且遗留下来。反正关闭门户之后,外界很难察觉到这片福地,还是比较安全的。

%66
果然如方源所料的那样,两大仙窍福地中,不仅没有任何的蛊虫,甚至就连经营的修行资源都被尽数毁灭。

%67
尤婵、秦百合恨极了方源这个杀人凶手,不想留给他任何的东西。

%68
好在大同风不是谁都能随随便便引发的。

%69
做完这一切后,方源不愿在东海久待,直接离开了这里。

%70
他此次犯下命案,杀死了尤婵、秦百合,又攫取掏空了两大海域,相信很快就会暴露出来,引发东海蛊仙界的轩然大波。

%71
这不是夸张。

%72
不管是尤婵还是秦百合,都有着广泛的影响力。

%73
再加上蛊仙界承平已久,这一次大案,又是两位七转强者丧命,必定会引发一定程度上的恐慌,让东海蛊仙界人人自危。

%74
还会有大量的蛊仙,不管是出于什么目的,为尤婵、秦百合报仇也好,伸张正义也罢,贪图两人的遗产或者为自己树立名声,总之会有大量的蛊仙要来追查方源。

%75
风紧还不扯呼?

%76
方源有八转战力,虽然不惧,但也不想自己身陷在接连不断的麻烦之中啊。

\end{this_body}

