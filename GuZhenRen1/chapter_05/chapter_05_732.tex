\newsection{我心里有数}    %第七百三十五节:我心里有数

\begin{this_body}

%1
方源打量着偷生仙蛊。

%2
这只仙蛊形如蜜蜂,有成人拳头大小,身上斑纹金黑相间,通体笼罩着一层猩红的薄薄光晕。一旦飞舞运动起来,这层猩红光晕将会大盛。

%3
最惹人瞩目的是蜜蜂的尾针,非常尖锐纤细,并且长度远超蜜蜂本身,竟有成人前臂一样的长度。

%4
根据传闻,偷生仙蛊乃是盗天魔尊年轻时获得的第一只仙蛊,来历非凡。

%5
方源看着偷生仙蛊,不免思绪泛滥:“说起来,盗天虽然是魔尊,但其实性情平和,并不嗜杀,完全不像幽魂魔尊。不过也正是因为这样的性情,他方才开创出了偷道这个流派。”

%6
方源对盗天魔尊还是比较了解的。

%7
盗天魔尊乃是天外之魔,一心想要回去原来的世界,但性情平和,不好厮杀,又得要积累资源,不断修行,寻找到回家的可能。

%8
怎么办?

%9
面对这个难题,偷道——就是他给出的漂亮回答!

%10
十大尊者是人族历史上,最为杰出的人物。

%11
方源重生后,对这些尊者更感兴趣,更加重视。

%12
可以说,方源上一世围绕着修复宿命蛊而展开的旷世大战,尊者起到了举足轻重的作用,每一次尊者的出手都几乎要颠覆整个战局。

%13
尊者的无敌力量,尊者的深远谋算,都给方源留下了极为深刻的印象。

%14
忽然,方源脑海中灵光一闪。

%15
“无极、星宿、元始、元莲、巨阳、红莲、乐土,都用各种各样的方式,参与了五域旷世大乱战。”

%16
“剩下盗天、狂蛮、幽魂三位没有出手过。”

%17
“不,幽魂已被俘虏,只剩下一团魂魄,怎么可能还有动手的能力。我继承影宗真传,带领参与影宗参战,本身就可算作是他出手了。”

%18
“幽魂魔尊虽然强悍无比,杀性第一,死后又积蓄实力,达到全流派至少大宗师的恐怖境界,单凭境界,就可算是历史第一,超越所有尊者。”

%19
“但是他也得遵守最根本的规矩,没有仙元,没有蛊虫,只剩下一团残魂,巧妇难为无米炊,被天庭关押得死死。”

%20
“那么盗天、狂蛮两位魔尊,有没有料想过会有五域旷世大乱战?”

%21
“应该都能算到的吧。毕竟元始、无极都算出来并且留下了后手,这两位修为同等的九转尊者,又是后世之人,怎可能算不到?”

%22
“他们算到此战,那么是否留下了什么干预的手段?”

%23
“盗天或许一门心思想要回归,那么狂蛮呢?历史上,他也是攻上天庭的魔尊呢。他留下手段的可能很高啊。”

%24
“前世没有显露,但今生我是否能找寻到狂蛮留下的手段,加以利用呢?”

%25
方源想到这里,不禁眼前一亮,这是一个很棒的想法!

%26
面对天庭这等庞然大物,除非他成为九转尊者,否则就得借力。

%27
这时,一旁站着的琅琊地灵打断了方源的思绪:“方源,你如今得了八转偷生仙蛊,手头上的八转仙蛊好像已经高达五只了!啧啧,真是不可思议啊。你不过只是七转修为,竟然拥有了五只八转仙蛊!历史上你这样的例子,也是屈指可数的。”

%28
琅琊地灵啧啧称奇,感叹不已。

%29
方源此刻得了偷生蛊,身上还有态度蛊、慧剑蛊、似水流年、魂兽令,的确是五只仙蛊。

%30
这当然不算智慧蛊,因为智慧蛊高达九转!

%31
琅琊地灵对方源一直都很关注,所以对他的情况,还是比较了解的。

%32
方源笑了笑,却是不以为意。

%33
难道他还要告诉琅琊地灵,在未来他还要获得春、夏、大气、悔蛊这些八转仙蛊吗?

%34
说出来,会不会把琅琊地灵吓到?

%35
“只是这只偷生仙蛊,留有弊端。当年盗天魔尊留下来,却又害怕此蛊带给众生太大的厄难,因此只留下三次使用的机会。上一世琅琊守卫战中,三次机会都用光,杀死了雷鬼真君,随后偷生蛊就被陈衣利用因果神树杀招,直接摧毁了。”

%36
“这还不是偷生的全部缺陷。单纯利用偷生仙蛊,敌我不分,对付敌人的同时,又会危害自己。这就很尴尬了。”

%37
“所以得到偷生仙蛊后,我还得琢磨出一些杀招来,至少得改变这个弊端。”

%38
“若是有可能,解决掉偷生仙蛊的三次使用机会的限制,无疑就更好了。”

%39
方源脑海中诸多思绪一闪即逝。

%40
不过,他也清楚,偷生仙蛊的三次使用机会乃是盗天魔尊生前限制,自己要打破十分艰难,希望极其渺茫。

%41
想到这里,方源对琅琊地灵微微一笑:“太上大长老,这只偷生仙蛊似乎只有三次使用的机会啊?太上大长老可清楚是怎么回事吗?”

%42
琅琊地灵心中震惊。

%43
方源居然识破了!

%44
这怎可能?

%45
这可是盗天魔尊动的手脚,即便是自己都无法发觉。从外表上看,毫无端倪,一切正常。

%46
就算是前任地灵动用各种炼道手段,也无法发现盗天魔尊的限制。

%47
方源只是看了几眼,他就洞悉出来了?!

%48
琅琊地灵满脸震惊之色,脱口问道:“你是如何发觉的?”

%49
话一出口,琅琊地灵脸色又变了,因为这样一问,无疑是证实了方源的话。

%50
他咳嗽一声,神情尴尬地道:“不是我故意隐瞒你啊,我是打算说的。”

%51
地灵虽然不可以说谎,但也能动心思谋划。像眼前的黑毛地灵,野心勃勃,领导才能也挺不错。而他的前任白毛地灵,更擅长隐忍,把琅琊福地经营得风生水起。

%52
黑白地灵,各有所长。

%53
黑毛地灵“打算说出真相”,具体的是指“当方源发现猫腻之后说出真相”,并不算说谎。

%54
方源看着黑毛地灵,似笑非笑:“太上大长老啊,本来要替换来炼道真意,单靠这些仙蛊就不行。现在偷生蛊又是这种状态,更得添一些事物,来补足当中的差价了。”

%55
黑毛地灵犯了一个白眼,嘟哝一句:“这我当然知道。你说吧,还需要什么?”

%56
方源嘴角含笑:“我有一个提议。”

%57
他说出提议,大意便是和其他三族加深合作,用琅琊派的底蕴来换取三个异族手中的宙道仙蛊。

%58
琅琊地灵不免诧异:“还换宙道仙蛊?这么说来,方源你是想往宙道方面发展了?”

%59
方源点点头。

%60
这是必须的。

%61
现在的他,还没有铲除夏槎,缺少春夏秋冬四大宙道仙蛊,也没有从铁区中身上得到七转刃蛊,光阴飞刃杀招也用不了。

%62
方源身上的宙道仙蛊主要来源于两部分,主要部分是黑凡真传,其次是惊鸿乱斗台残留下的一些蛊虫。

%63
方源的炼道境界是准无上大宗师,他的宙道境界也同样如此!

%64
再加上眼下,红莲真传仍旧在光阴长河之中,方源还得闯荡光阴长河。

%65
当然需要宙道方面的强硬实力!

%66
琅琊地灵点点头:“那我就舍了这张老脸,和其他三族谈谈吧。方源啊,眼下的关键还是天庭。我们得赶紧将福地中的不利道痕铲除掉,然后尽快秘密迁徙才是啊。”

%67
方源连忙安慰焦急心忧的黑毛地灵:“我心里有数,放心吧。我收集这些宙道仙蛊,也是为了对付天庭。琅琊福地中的不利道痕,我也有了一些应对的方案,过段时间我要改良咱们的超级仙阵。”

%68
“你有数就行。”黑毛地灵拍拍方源的肩膀,走了。

%69
他行动效率极高,不久后就谈妥当了,将手中的宙道仙蛊交给方源。

%70
至此,算是琅琊派本身的库藏,加上从异人手中换取而来的,一共有五只宙道仙蛊,三只六转,两只七转。

%71
七转的有时隐仙蛊、时针仙蛊,六转的有仙蛊昨日、一旬、累年。

%72
对方源最有帮助的就是时隐仙蛊,此蛊可为蛊仙隐身,方源得之大喜。

%73
时针仙蛊乃是攻伐之蛊,催动出来,会发出细针刺杀强敌。

%74
其余三只六转,差强人意,顶多做为辅助仙蛊。

%75
一天后。

%76
大量的八转仙材砸下,炼出一团七彩漩涡。

%77
智慧蛊主动投入漩涡当中,像是闻到鱼腥味的猫。

%78
方源打开至尊仙窍,将七彩漩涡牵引到自己的福地里去。

%79
当初,琅琊地灵就是依凭这个手段,把智慧蛊从狐仙福地哄骗到琅琊福地中去的。

%80
方源上一世也用的这个方法。

%81
这一次,他虽然熟悉,但却装作陌生的样子,对琅琊地灵的手段赞叹不已。

%82
隐瞒自己再度重生的秘密,就要从这些细节上开始做起。

%83
黑毛地灵虽然被方源声声夸赞,但心中的失落之情早已流露出来。

%84
“唉,感觉心里空落落的。”黑毛地灵深深叹息,他可是对智慧蛊觊觎良久。没想到最终还是方源的,整个过程中,黑毛地灵还搭进去了大量的寿蛊。

%85
但没有办法啊,在天庭会第二次入侵的大势下,为了防止智慧蛊沦落到天庭手中,黑毛地灵再不愿意,也得捏着鼻子低头!

%86
“这份仙蛊方你好好看看,接下来我们就要炼出这只仙蛊来。”方源收回了智慧蛊后,转手递给琅琊地灵一份仙蛊方。

%87
琅琊地灵仔细一瞧:“万我仙蛊?”

%88
“不错。此蛊对我有大用处,能极大地提高我的战力,对付天庭很有帮助。”方源一边说着,一边走出这片山谷,“你们好好准备一下,我这就外出搜寻主要的四种仙材。”

%89
“哎等等,你不是要改良福地的大阵吗?”

%90
方源头也没回,只是摆了摆手:“放心,我心里有数得很。”

%91
琅琊地灵:“……”

\end{this_body}

