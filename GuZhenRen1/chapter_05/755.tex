\newsection{眼底幽芒}    %第七百五十八节:眼底幽芒

\begin{this_body}

南疆。

某处无名的山谷中,一道瀑布垂下,水声轰鸣,水雾弥漫。

在瀑布顶的悬崖边上,方源本体盘坐在一株松树枝丫之间,闭目沉思。

人道灵感来得十分突然,方源分外重视,本体和分身都放下手头上的事情,一起感悟。

此刻,在至尊仙窍当中。

一只大军已然成型。

大约有十万多的力道虚影,静静地站在草原上,一动不动,仿佛是雕塑石像。

他们每一个都相貌相同,酷似方源本体,相互间隔等距,整整齐齐,密密麻麻。

这是方源动用了万我仙蛊,催发而出的万我大军。

方源抓住那道灵感不放,这一次却是舍弃万我仙蛊,而是动用万我杀招。

不一会儿,又有大量的力道虚影成形,站成一团。

方源又想了想,第三次催动万我杀招。但这一次,他却不是动用仙级杀招,而是最初始的凡道杀招。

以四转的全力以赴蛊为核心,其余苦力蛊、借力蛊、自力更生蛊、炼精化神蛊、地力蛊、水力蛊、风力蛊、电力蛊、火力蛊、潜魂兽衣蛊、敛息蛊等为辅助。

最终形成万我杀招,分化本体魂魄,在一瞬间形成大量的力道虚影!

不过因为是凡级杀招,这些力道虚影远远比不上之前产出的虚影大军。

但这却是源头。

当初,在北原王庭之争时期,方源结合六臂天尸王、我力,参考魂道、智道、气道、奴道各大杀招所创,解决千古难题,达到奴力合流,开创出了此招。

之后,又在此招的基础上,先后发展出力道大手印、逆流护身印。前者在恰当的时间,为方源增添攻伐优势,一度是方源的强力手段。而后者则是方源以七转抗衡八转的关键底牌!

不管是万我杀招,还是力道大手印、逆流护身印,都非常的优异,乃是极品杀招中的极品。

现在,方源抓住一股人道灵感,追溯源头,再次改良最原始的万我凡道杀招。

方源本体和分身一齐推算,同时分身更是沐浴在智慧光晕之中!

杀招很快改成,方源第四次催动万我杀招。

这一次同样形成大量的力道虚影,然而和之前的三大团虚影不同,这一团的方源虚影虽然仍旧是静静的站着,但是脸上的神情却是各种各样,千奇百怪。

有的在哭,有的在笑,有的愁眉苦脸,有的喜笑颜开,有的杀气腾腾,有的一脸淡漠,有的闭目养神,宁静平和,有的双眼滴溜溜地转动,对一切都十分好奇,还有的吹鼓双腮,无聊透顶的模样……

方源的神念扫视着这些虚影,脑海中各种思绪仿佛电闪雷鸣。

沉思良久,方源和分身舍弃这几团虚影大军,再次催动智道手段,开始推算。

这一次推算,时间足足用了三个时辰。

推算成功后,方源得到一个繁复至极,规模比原先还要庞大百倍的凡级万我杀招。

他催动这个杀招。

因为这个杀招涉及的蛊虫太多,布置太过繁杂,饶是方源乃是八转蛊仙,也是过了好一会儿,才催动出来。

这一次产生的万我虚影,数量上要远远少于之前的任何一团,但他们却似乎是有了灵性。

这些万我虚影,有的在跑,有的在跳,有的盘坐在地上沉思,有的相互嬉戏打闹,更有的带着恨意和怒气,对周围的万我虚影出手攻击,乱成一团。

万我虚影间的混乱规模越来越大,很快,这团万我虚影彻底厮杀在一块。尽管有人不愿意掺和,但身不由己。

这些万我虚影也不是相等的强度,有的人强,有的人弱。

方源眼中精芒闪烁,他发现在这混战中,胜利的万我虚影会越战越强,而失败的万我虚影则越来越弱,甚至当初灭亡。

万我虚影越来越小,很快只剩下个位数的虚影在惨烈厮杀。

最终,只剩下一位虚影站在战场上,其余的万我虚影都被消灭。

这道虚影比之前,要强大数十倍,但仍旧局限在凡人层次。

虚影模样仍旧酷似方源,但却是满脸的仇恨和愤怒。他对不远处的方源宙道分身咆哮一声,显露出浓郁的杀机和恶意,却不敢来攻,而是扑向之前的几团万我虚影大军。

这些万我虚影都静静站立,没有方源的命令,一动不动,任由恨怒虚影肆意屠戮。

这道恨怒虚影屠戮片刻后,气息越来越强,竟突破了凡俗桎梏,晋升到六转蛊仙的层次!

看到这里,方源本体还有分身都露出欣喜之色。

方源分身轻轻一挥手,场中剩下的万我大军顿时涌动如潮,将恨怒虚影团团围住,发出决死的攻势。

恨怒虚影实力出众,虽然陷入万军重围当中,却是越战越勇。

他每击败或者撕碎一个虚影,他的力量就强大一分。

他似乎不知道什么叫做疲倦,仿佛是一个可以成长的战争机器,不断咆哮,怒吼连连,始终在战场中屹立不倒。

终于,当他从六转蛊仙的层次,突破到七转时,方源本体终于出手,催动杀招,亲自将这具力道虚影打杀了账。

方源最后一次催动出来的万我杀招,形成的力道虚影,都各有各的想法和性情,连他本体都操纵、掌控不住。

但是很奇妙的是,最后的万我虚影都具有极其恐怖的成长性!

那个满脸恨怒的虚影,最初的时候不过只是四转程度而已,最终竟然成为了七转蛊仙!

当然,这是气息和层次上的突破。

他本身的战力,并不那么强大,甚至还要弱于寻常的上古荒兽。

因为他的身体只是虚影,没有上古荒兽那么结实强健。他的智力也有限,并不能支持他调动蛊虫作战,只能凭借本能。

“不管是万我仙蛊,还是万我杀招,都要消耗我的魂魄,这本就是杀招的原理,因此也成为无法避免的弊端。”

“正是因为这些力道虚影来源一体,魂魄相同可以共融,因此存在着吞噬彼此壮大自身的可能性。”

“我最后催动的万我杀招,已经不再是单纯的奴力合流,而是更进一步的人道杀招了!”

方源宙道分身始终沐浴智慧光晕中,眼底深处精芒闪烁不定。

到了这一步,万我杀招就是彻彻底底的人道杀招,而万我仙蛊本就是人道的仙蛊。

方源上一世,在进攻中洲的时候,还在遗憾自己没有人道手段,在人道上没有建树。

其实他早就有了,只是当时考虑太多,身处大战,自己没有察觉罢了。

正所谓“横看成岭侧成峰,远近高低各不同。不识庐山真面目,只缘身在此山中。”

他对人道的探索早就在五百年前世就已经开始,只是他毫无这方面的意识罢了。

“人的一生,本就是对人道的探索。”

“除了万我仙蛊之外,我还有一只人道上的仙蛊。那就是——坚持蛊!”

“等等,不只是坚持蛊,我抢来的至尊仙胎蛊同样也是人道蛊虫!!”

翻开《人祖传》,希望蛊、虚荣蛊、胆识蛊、固执蛊、尊严蛊、自己蛊、忧患蛊……这些蛊虫,不都是人道的蛊虫吗?

“难怪《人祖传》被公认为人道真传,这里面的确蕴藏着深邃的人道奥妙。”方源心中感慨不已,直到此刻他看《人祖传》才算是登堂入室。

得到这个结果,方源走了近六百年的生命历程!

他得到的人道灵感,不是因为最近栽培了异人,而是整个生命历程中的积累达到了质变的关键点。

就像是一点点堆积起来的火药,碰上了一个微小的火星,轰的一声,产生了爆炸,令方源突破了原先的桎梏,在人道上达到了另一种更高的层次。

这是厚积薄发,也是水到渠成。

中洲,天庭。

中央大殿。

滔滔紫气逐渐平息下来,随后尽数回缩,最终收敛于紫薇仙子的脑海中。

紫薇仙子眉头轻皱:“方源这魔头究竟在酝酿什么阴谋?”

琅琊福地守卫战,天庭战败,紫薇仙子主动承担责任,迅速将有关方源的情报都无偿地泄露出去。

但是方源方面,却十分诡异,毫无动静。

按照常理而言,星投杀招暴露,陈衣、雷鬼真君战死,方源应当将战果四处张扬,好来打击天庭威望,引发其他四域对天庭的更多敌意。

但是方源并没有这么做。

紫薇仙子心中总萦绕着一股不安之情。这些天来,她都密切地关注方源动向,哪怕是一丝风吹草动,她都要全力了解,尽力推算。

方源在南疆干的好事,紫薇仙子已经知道。

但为何方源只和池家作对?他究竟有什么目的,难道真的是要劫掠池家的这三处资源吗?

之前的掠影地沟大阵被破坏,那个神秘蛊仙是不是方源呢?

“或许他是吞并了琅琊福地之后,底蕴大涨,想要炼蛊了。”

“又或者他已经再次动用春秋蝉,重生归来,利用重生的优势,来一步步占据先机!”

紫薇仙子眼中精芒闪烁不定。

方源的春秋蝉暴露之后,作为他的敌人都会考虑到春秋蝉这个因素。

紫薇仙子更是始终关注着这一点。

“一旦方源最近重生,那么春秋蝉必定需要修复的时间,这是铲除方源的好时机。”

“同时,我方之前准备的种种手段,恐怕已被方源获知了情报,需要改换作战的方法。”

方源是不是动用了春秋蝉重生过来,这一点非常关键!

若是有,紫薇仙子就要推倒之前绝大多数的战术,必须重新设计。

然而,紫薇仙子全力推算多次,哪怕是借助星宿棋盘,也不能确定方源是否最近重生过。

方源在掠影地沟处,即便攻破了大阵,也没有取走梦境。对付蒙屠等人时,更是隐秘周全。就连大盗鬼手,方源都不是直接用,而是借助超级大阵掩护,使得天庭联系不上算不尽这层身份。

他放出的烟雾弹,真的太妙了,暂时成功迷惑住了紫薇仙子。

在这一点上,方源的智道造诣带给他相当大的帮助。

这一次的重生和前几次都不相同,面对天庭这个大敌,方源必须时刻考虑到墨水效应。

“方源狡诈奸猾,留下的有价值的线索真的太少!”

紫薇仙子恨恨地想了想,便起身离开中央大殿,来到关押魔尊幽魂的地方。

搜魂!

魔尊幽魂抵挡一阵后力竭,一股股记忆被抽取出来。

“有关金龙鱼的豢养之秘?”紫薇仙子先是欢喜了一下,旋即还是失望。

若是之前,紫薇仙子还能打击方源的这道主要经济命脉,但是现在对方吞并了琅琊福地,里面种种的经营项目都被方源接盘。

紫薇仙子得到的豢养之法,不过是一个鸡肋。

“即便如此……也不能放过任何一丝打压方源的机会。”紫薇仙子想了想,咬牙做出决定。

她却不知道,在她沉思的时候,她的眼眸深处,有过一道幽暗的光陡然一闪即逝。

------------

\end{this_body}

