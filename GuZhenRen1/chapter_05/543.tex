\newsection{阎帝!}    %第五百四十四节:阎帝!

\begin{this_body}

“静眠电蟒”琅琊地灵怒吼一声。求书网小说qiushu.cc

下一刻,电光激闪,交织如水,很快凝聚成一头庞巨似龙的电蟒出来。

雷电是阳,蟒蛇是阴。雷电组成的巨蟒,蕴含阴阳,消去了雷电的狂暴,反而显出温和的样子。

因此,电蟒并不狰狞,反而透着优雅和可爱。

它蜿蜒游动,缓缓盘踞在一头大象模样的石头山。蟒身不断蹭动石头,将石头磨碎成最细微的粉末。

炼道上有公认的四大仙招,最能够处理仙材。静眠电蟒便是其中之一。

琅琊地灵源自长毛老祖,自然掌握得滚瓜烂熟。

这杀招的层次,因为动用了八转的炼道仙蛊,也因此高达八转。

没有八转层次的静眠电蟒,是不能处理这座八转层次的仙材象甲山。

这是信道的太古仙材。

琅琊地灵处理仙材,已经持续了数个时辰,在象甲山之前,已经有十多件八转仙材,相继处理掉,这些仙材林林种种,分门别类,都是信道。

静眠电蟒逐渐消散,象甲山也彻底被磨碎成粉末。这些粉末和其他仙材融汇成一团,在琅琊地灵的手段下,竟形成了一团不断流转的七彩漩涡。

“可以了,将仙阵散去罢。”琅琊地灵吐出一口浊气,疲累不堪地道。

当即有毛民蛊仙充当下手,立即将这处炼道仙阵散掉。

仙阵一去,顿时七彩漩涡的气息就扩散开来,远处的智慧蛊像是问到鱼腥味的猫咪,立即飞了过来,然后一头扎进七彩漩涡当中去,深深沉溺。

“原来智慧蛊还有这样的特性。”方源见此,感慨道。

“传闻中,炼制智慧蛊的蛊方,就包含大量的信道仙材。大自然中,野生的智慧蛊也是在充斥信道道痕的地方,才能完成自我的晋升。”琅琊地灵道。

这样一说,方源顿时了然。

至于智慧蛊的蛊方,不管是幽魂真传,还是琅琊真传,都没有收录。<strong>最新章节全文阅读www.QiuSHU.cc</strong>只有在天庭中才有所收藏。

方源不能操纵野生智慧蛊,但却可以遥控迁移这团七彩漩涡。

至于那四千年的寿蛊,也早就在炼道真意之后,就到位了。

“这次劳烦大长老你亲自出手,欠款我会陆续偿还的。”方源诚恳地对琅琊地灵道。

挪移智慧蛊的方法,虽然简单,但代价不菲得很。

种种八转仙材硬砸下去,才能造成这么一团临时的七彩漩涡。过不了多久,漩涡就会逐渐消散,除了哄骗智慧蛊挪移了位置,不会有任何收益。

这部分代价,当然是方源支付。

和琅琊地灵详细商量了之后,双方达成约定,可以让方源暂时欠着,不着急当下还清。毕竟方源刚刚历经大战,仙元见底,急需重新储备上来。

这也是方源目前地位提升的原因。换做之前,绝不会有如此丰厚的福利。

但现在,当他痛揍了雷鬼真君之后,和琅琊地灵交往谈判,已经占据主导地位。

“时局紧迫,我要立即闭关,用尽全力推算,以最快的速度找出脱离当下困境的方法。”方源道。

“好,好,那我就不打扰你了。”琅琊地灵顿时闪过一抹喜悦,脸上的疲惫之色都因此舒减了许多。

方源便回到自家云城。

说起来,凤九歌等人入侵琅琊福地,也抢走了好几座云城。方源的云城却是幸运地躲过这一劫,保留了下来。

当然,这或许也和方源本身强势的运道有关吧。

方源在云城中闭关,其他异人蛊仙都想要和他搭上关系,但方源一概不见,哪怕是雪民一族同样如此。

至尊仙窍中。

仙道杀招力转生死

“这招乃是力道杀招,并不太适合宙道分身,将来可以改良成宙道。”

方源的宙道分身,瞬间转变成仙僵,来到智慧蛊的面前。

片刻之后,智慧蛊爆发出一蓬智慧光晕出来,方源的宙道分身闭目盘坐,沐浴在光晕中,迅速思量。

凤九歌留在琅琊福地中的道痕,方源将其放置一般。

琅琊福地的安危和自己相比,当然是后者更加重要。

“当下,最紧要的是提升阎罗杀招。否则的话,一旦出了仙窍,就要撑起阎罗,耗费的仙元实在太过巨大。”

阎罗不提升,方源就太被动。

若是暴露位置后,继续被天庭推算,就会有更多的秘密被对方得知,这种情况是方源绝不想看到的。

之前方源就思考过,与其利用鬼官衣来抗衡天庭的推算,倒不如借助鬼不觉杀招的力量。

阎罗杀招,就是成功的尝试。

但现在,天庭方面掌握方源的线索过了许多,当下的阎罗已经保护不了方源。

这一场推算,足足进行了一个多月。

当然,智慧光晕在此期间,不是一直存在。

整个过程大体上是顺利的,因为方源在偷道境界上,乃是大宗师,炼道境界更是准无上,鬼不觉、鬼官衣等等杀招,也掌握得非常透彻,惟独魂道境界稍微差一些,拖了一些后腿。

最大的一个难关,是在中后期。不过方源将八转仙蛊魂兽令,添加进去之后,也就解决了。

阎罗杀招大变模样,方源将其重新命名为阎帝

阎帝杀招最主要的组成部分,有四个。一个是鬼不觉,一个是鬼官衣,一个是魂兽令,最后一个则是万我。

“阎帝虽然推算出来,但只能算是草创,接下来还是要试着催动,不断调整。”

“这件事,就让本体来做。至于分身,则继续推算如何处置凤九歌所留下的道痕。”

就这样,从方源回到这里,前前后后琅琊福地中过了小半个月。

方源终于出关。

分身的推算任务已经完成了,方源找到了解决凤九歌遗留道痕的方法。

他将这个方法,交给琅琊地灵。后者见了,顿时脸色骤变,大骂天庭阴险狡诈。

原来,这些道痕若是用炼道手法去驱除,反而助长这些道痕。洁身自好杀招也是无效。天庭在魂道上,不及幽魂魔尊,但炼道方面的底蕴,绝不弱于琅琊派。毕竟历史上三大无上大宗师,只有长毛老祖乃是毛民,其余两位却都是人族,他们的传承大多被天庭收刮囊括了。

方源乃是准无上,洞悉了这处暗算,并且提供了最简单的处理方法。

就是当琅琊福地挪移位置时,这些道痕就可借助方源的手法迅速除去。除此之外,其他的方法耗时以十年以上计算。

于是,问题又回到原点,那就是琅琊福地的迁徙。

“怎么办或许天庭就是故意如此布置,逼迫我们迁徙福地啊。”琅琊地灵忧愁得很。

方源也有这样的担忧。

天庭这段时间没有动作,绝不代表他们罢手放弃,极有可能已经有了安排,就等着琅琊福地自己暴露位置。

但若是不迁徙,天庭迟早也会找到位置。

“看来就只有先强行迁徙的准备了。这是万我仙蛊方,若是炼成,对我战力将有极大帮助。”方源说着,递给琅琊地灵一份蛊方。

这当然是他最近推算出来的成果。

凭借炼道准无上的境界,方源自信,这已经是世间最优良的方案。

“万我仙蛊”琅琊地灵看了一眼,顿时注意力就被吸引过去。他尽管执念不同,但到底还是源于长毛老祖。

“你这仙蛊方若炼出来,至少会是七转层次。不过,里面有几样仙材,琅琊库藏中也很是稀缺啊。就算是宝黄天,短时间内也绝然收购不到。”琅琊地灵苦涩地笑道。

“无妨,这些仙材我自会筹谋。”方源摆摆手,微微一笑,目光深幽。

有了全新的杀招阎帝在手,他正打算借此机会,试验此招实战威能未完待续。

------------

\end{this_body}

