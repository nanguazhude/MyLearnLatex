\newsection{五界大阵}    %第七百一十五节:五界大阵

\begin{this_body}

不败福地战场。 .更新最快

宛若天柱的飓风,卷起无边风浪,碾压着天庭大阵。

留守在外的凤仙太子、巴十八凝神盯着远处天与地的交界处。

在那里,两道光虹划破长空,逐渐显露出两位八转蛊仙的身影来。

正是厉煌和清夜。

“终于赶到了!”

“没错,还不晚。”

厉煌、清夜联袂杀至。

“方源那贼子究竟在哪里?”厉煌一双眼眸闪烁着橙红色的火光,不断侦查。

他对方源充满了怨念。

之前方源撤退时,故意选择了和不败福地相反的方向,把厉煌等追兵吸引到很远的地方。

而后,方源找到时机,动用定仙游杀招,来到了不败福地处,把厉煌等人远远抛在脑海。

中洲天庭可没有烽火台,厉煌和清夜只得不辞辛苦,重新赶赴不败福地。

“应当是已经入了大阵。”清夜回应,他的目光在大阵上一扫即过,然后就盯上了无限风。

“这杀招很强!对我方大阵威胁极大,要优先铲除掉。”

“这个武庸,有这样的杀招,还有送友风,可以的话,这一次就留下他的命!若让他回到南疆,必定是我天庭的心腹大患!”

厉煌、清夜迅速商讨后,方向一折,直扑无限风杀招。

凤仙太子、巴十八早已洞察,杀奔过来。

“去。”凤仙太子轻喝一声,大袖一挥,瞬间飞出数百团的火焰。

“你拿火来攻我?可笑。”厉煌嗤笑一声,不闪不避,身上阳莽背火衣熊熊燃烧。

凤仙太子的火焰射到厉煌身上,没有带给厉煌任何伤害,反倒是增添了阳莽背火衣的威势。

凤仙太子面色不变,对准厉煌伸手一指。

呼。

厉煌身上的火焰,陡然变作紫色。紫色的火焰渗透到橙红色的阳莽背火衣中,不断瓦解厉煌的防御杀招。

厉煌面色顿时一变。

凤仙太子似乎不想留给他应对的时间,飞身扑上。

两人在高空中交手,打得烈焰熊熊,炙烤天地,战场中气温急剧攀升。

巴十八将凤仙太子的战斗看在眼里,暗道:“哼,这凤仙太子还在做戏,藏得真深。”

方源这一方留着凤仙太子和巴十八,在阵外驻守。

凤仙太子的身份,巴十八早已通过武庸得知,只是暂且按捺不发。

和武庸、方源所料的一样,不到关键时刻,凤仙太子并不想暴露了身份。事实上若有可能,天庭宁愿付出一些牺牲,也要隐藏凤仙太子的身份。将来五域乱战,或可发挥更大的价值。

凤仙太子、厉煌均是炎道八转,两人交手,从天上打到地上,又从地上打上天。烧红了苍穹,河流干涸,大地龟裂,白云蒸发,无数生灵纷纷奔逃。

巴十八和清夜,则分别修行律道、暗道。两人交手虽无凤仙太子和厉煌那般气势磅礴,但更加凶险狠辣。

厉煌当然知道凤仙太子的真正身份,双方斗得动静再大,也只是演戏。

巴十八和清夜却是真正的仇敌。

巴十八连连怪啸,一道道玄光宛若刀剑,或砍或刺,打在清夜身上。

三连击!

四连击!

五连击!

……

每一道玄光都威能相当,但是打在清夜身上,却是伤害节节暴涨。

这是巴十八一生苦修而得的连击之术。

这种律道辅助杀招,能够使得其他某个仙道杀招,威力迅速递增。

前提是,巴十八一直在使用这记杀招,终究并不转换其他手段。

清夜被杀得节节败退,额头见汗:“不能再让他连击下去,一旦积累到十八连击,我的防御手段将无法抵挡。”

清夜催动仙道杀招,忽然身形涣散,化作一片夜幕。

天空中,仿佛滴下了大团的墨汁,不断扩散。

巴十八见此,顿时犹豫:“这怕是清夜的招牌杀招大夜弥天。”

两方的情报都做得很到位,巴十八和清夜对彼此都有一定的了解。

大夜弥天乃是仙道战场杀招,但比较特别。寻常的仙道战场,都是酝酿片刻,一次成型。而大夜弥天不是,它是不断扩张,范围逐渐变大。这一招理论上,是没有极限的。

只要给清夜足够的时间,他甚至能够将这个仙道战场杀招,扩散到整个白天!

巴十八向后飞退。

他思考了一下,还是选择暂避锋芒。

大夜弥天乃是仙道战场,巴十八若是杀进去,律道就要受到暗道环境的削弱。

当然,更主要的一个原因就是凤仙太子。

巴十八知道凤仙太子乃是内奸,若是他杀进仙道战场,外界看不见战场中发生的一切。天庭一方或许会将厉煌、凤仙太子都囊括到杀招中去,来偷偷地对付巴十八。

到那时,巴十八以一敌三,那就比较危险了。

两对八转的战斗,陷入短暂的僵持。

片刻之后,天际出现了数座仙蛊屋的身影。

首当其中的,便是寒螭庄、风满楼。

这些仙蛊屋原本护卫着帝君城,如今帝君城战事已完,仙蛊屋装载着诸多蛊仙,纷纷赶来支援。

寒螭庄冲杀在最前方,庄园周围忽然闪现淡蓝光影。

光影凝聚成龙形,无角的长龙正是螭龙!

淡蓝螭龙光影猛地张开大口,吐出无数冰雹,打向巴十八。

巴十八的速度顿时被大大延缓,他冷哼一声,舍弃清夜,施展杀招轰击寒螭庄。

轰轰轰!

寒螭庄不断挨打,震得墙角龟裂,瓦片横飞。

艰难的抵御中,第二道淡蓝螭龙光影凝聚出来,和头一道相互纠缠,盘绕在寒螭庄周围一圈。

寒螭庄因此防御大涨,并且淡蓝螭龙光影张开大口,吐出一片片破碎不堪的玄光。

“这座仙蛊屋竟有排泄我杀招伤害的能力!”巴十八面色微沉,见短时间内无法摧毁这座仙蛊屋,再纠缠下去,就要被大夜弥天杀招罩住,便明智后撤,继续拉开距离。

这个时候,风满楼也杀入战场。

它并没有协助清夜或者厉煌,而是直接没入天柱般的无限风杀招之中去。

“好风,好风,给我一点时间,我能让此风为我所用!”当代风满楼主哈哈大笑。

在风满楼的干预下,无限风威势开始逐渐递减。

一座又一座仙蛊屋加入战场,方源一方越发危急。

“坚持住,我们来了!”

关键时刻,天庭大阵忽然炸裂出几道缺口,数座仙蛊屋飞奔而出,有南疆的玉清滴风小竹楼、太宇寺、海角阁、无定府,还有北原的指印桥、猿魔堂、少阳楼。

有了这七座仙蛊屋支援,巴十八顿觉压力剧减。

双方混战,仙蛊屋纵横,杀招对撞,一次又一次炸出宛如烟火般的斑斓光影,一时间阵外战场又陷入僵持当中。

大阵内。

武庸等人面色微沉。

虽然他们及时地炸开大阵缝隙,送出仙蛊屋,暂时稳定住了外面的局势,但武庸等人手中的底牌也都提前支出。

局面越来越不利。

凤仙太子是一个麻烦,天庭战场也不能指望更多,一切就在于方源等人能否迅速破阵。

“照我们目前的速度,胜机越加渺茫。”方源叹息,又询问道,“之前陶铸的五界真传,是否可以针对大阵呢?”

当初,他在五界山脉故意设局,来坑害南疆和天庭一干蛊仙。结果在大战中,武庸因为破坏了整个五界山脉,意外地引出了陶铸意志,得到继承五界真传的契机。

方源那个时候不得不撤退,五界真传就落到了南疆诸仙的手中。

按照方源的推算,五界真传应当是可以模拟出五域界壁。而在五域界壁中,蛊仙受到压制,施展任何的仙道杀招都会遭受反噬。

不管是仙蛊屋,还是仙道大阵,本质上都是仙道杀招。

方源寻思着,若是能够借助五界真传中的手段,来对付天庭大阵,或许能有奇效。

武庸和池曲由对视一眼,前者道:“是我获得了五界真传,真传中的确有营造出五域界壁的手段。只不过……”

池曲由接道:“只不过这个手段,需要利用地脉,再布置出大阵方可施行。耗费功夫,十分麻烦。”

陶铸当年能够营造出五界山脉,正是借助地脉,然后利用了大阵。

方源撤离战场后,武庸获得了五界真传,却发现最主要的精髓是五界大阵。

武家在阵道方面并不擅长,武庸便和池曲由合作,因此池曲由早已知晓五界大阵的奥妙。

“难道就非得借助地脉,才能布置出五界大阵吗?”方源眉头紧皱,“等等,我听说不久前,侯家的蛊仙曾经在地沟中收获了一只野生的地脉仙蛊?若有此蛊,是否能替代地脉?”

诸仙听了这话,顿时精神一震。

武庸和池曲由再次对视一眼,均感叹方源的敏捷才思。

池曲由摇头:“实不相瞒,我早已向侯家交易,换来了地脉仙蛊。只是要取代地脉的作用,将地脉仙蛊安插进去,必然是要对五界大阵进行改良。我虽然一直在努力,进展不小,但距离成功还有很长的路。”

蛊仙改良一个仙道杀招,常常要以年为单位。

池曲由虽然阵道境界高超,但是智道手段不佳,所以虽有进展,但却不多。

方源眼中精芒闪烁:“这五界大阵恐怕就是我等破阵的契机!池曲由你一人不成,但如今还有我和玄极子,不妨一试。”<!--80txt.com-ouoou-->

\end{this_body}

