\newsection{局势}    %第七百一十四节:局势

\begin{this_body}

武庸、药皇等人都有些微的茫然。

九九连环不绝阵,这可是星宿仙尊布置下来,专门为了对付无极魔尊的大阵。根据传闻,当年的无极魔尊也是连破百阵之后,才脱困而出。

凭他们实力,如何能突破这等天堑绝关?

更让人心中焦躁不安的是,他们必须争分夺秒,时间可不等人,一旦宿命蛊彻底修复,就再无胜利的希望。

“怎么办?”百足天君皱起眉头。

“这是星宿仙尊手笔,难怪!难怪!”玄极子恍然大悟。

池曲由为难地道:“即便是集结天下的阵道蛊仙,一起来破这大阵,恐怕也败多胜少。更何况现在,我们的时间越来越少……”

这位阵道大能,开战以来屡屡发挥超绝作用的南联领袖之一,也打起了退堂鼓。

武庸忽然哈哈大笑。

“盟主何故发笑?难道是有了破阵的妙计?”池曲由疑惑问道。

武庸摇摇头:“我虽没有什么破阵妙计,但我已经看透天庭的虚实。诸位,你们大可不必紧张,依我看,面前的难关我们过于高估了。”

“为何?诸位不妨想想。”

“当初的九九连环不绝阵,乃是星宿仙尊亲手布置。又有天庭蛊仙入阵,抵抗无极魔尊。但终究没有抵挡得住,无极魔尊破阵而出。既是破阵,那么这座九九连环不绝阵就已经惨遭毁灭过一次了。”

“就算天庭底蕴雄厚,修复好了所有仙蛊。做最坏打算,我们现在面对的便是完整的,甚至改良过的九九连环不绝阵。但镇守的蛊仙,可不全是天庭成员了。甚至真正的天庭成员,也恐怕不多。帝君城战场有厉煌、清夜,天庭战场更有龙公、紫薇仙子、袁琼都、车尾、从严等被牵扯在内。中洲此番受到我四域齐攻,逼得他们在此设立九九大阵,正说明他们分兵之下,人手稀少,只能被动防御。”

武庸双眼熠熠生辉,一时间浑身上下散发出自信绝伦、运筹帷幄的风采。

他继续道:“最关键的,他们已经露出的虚弱的本质!因为之前,陈衣、白沧水出阵来战我南联诸仙,陈衣险些被我杀死。诸位,若是九九连环不绝大阵真的牢不可破,陈衣、白沧水何必以身犯险呢?定然是他们俩为了后方布置新的蛊阵,而拖延时间。”

“之前单单我们南联进攻,就已经逼迫得他们铤而走险。如今我们南联和诸位北原仙友联手,此事大有可为。九九连环不绝阵虽是传奇,但也要看何人进攻,何人驻守!”

“对啊,是这个理!”

“哈哈,没错,当务之急,还是加紧进攻。天庭防御并不是那么强大的。”

“惭愧,我竟被星宿仙尊的名头吓住了。”

武庸一番话说得诸仙士气大振,攻势不降反升,越加凛冽。

武庸微笑,心中却在暗叹。

在场诸仙都是人杰,被九九连环不绝阵的名头唬住,并非是他们智短,而是压力太大。

冰塞川将实情通知多人,凝成了一股绳,来强攻大阵,企图争夺最后一份胜机,但此举有利有弊,在场诸仙均知道情势危急。

稍不留意,己方大败亏输,就要被天庭反击。

这可是在中洲。

他们又没有定仙游杀招,就算有,天庭恐怕也有相应的反制手段。

所以,诸仙要撤退,恐怕很困难。

保险稳妥起见,当然是现在撤退。时间拖得越久,撤离中洲的难度就越高了。

“西漠、东海的蛊仙,估计已经开始撤退了吧。毕竟这两域一盘散沙,不像我南疆,还有北原,可以凝聚统筹绝大多数的正道势力。”

“现在就看我方是否能迅速攻破大阵,追截到最后一批成功道痕了。”

“唉!刚刚还寄希望于方源的梦道手段,来对付这种大阵。没想到天庭一方居然也掌握了纯梦求真体的手段。是因为魔尊幽魂被俘虏的缘故吗?”

“也不知道这大阵内,到底是谁出手,阻止了方源?”

武庸心中不断猜测,思考着大局。

九九连环不绝阵中,凤九歌一脸苍白,却面露微笑。

“煌儿,做得好!”他望着凤金煌,笑意满满,欣慰至极。

阻挡方源的,不是别人,正是凤九歌的女儿凤金煌!

凤金煌吐了吐舌头:“爹,女儿不敢居功,能够操纵仙蛊,催出仙道杀招,都有赖于整个大阵的帮助。”

“即便如此,能够对抗方源,使其劳而无功,凤金煌你已立下大功了。”白沧水笑道。

陈衣点头,望了望凤九歌,又看了看凤金煌:“真可谓是虎父无犬女啊。”

凤九歌救了他的命,凤金煌又如此惊才艳艳,这让陈衣心中对这对父女充满了感激和赞赏。

天庭。

一番猛攻已经落下帷幕。

龙公粗重的呼吸渐渐平息下来。

他望着面前虽然伤痕累累,但仍旧屹立不倒的七极荒都,不满地冷哼一声。

“究竟还是没有摧垮它。是因为身怀伤势的缘故么……”

按照常理,龙公在龙御上宾杀招的加持之下,战力迅猛暴涨,绝对能力克七极荒都。哪怕这座上古大阵,可谓历史上的最强姿态。

但牵扯龙公的,还是他身上的伤。

这些伤,主要是旧伤。是不久前,龙公强夺龙宫,被诸多东海蛊仙创造的战果。

“若是没有这些伤势,我已能摧垮了七极荒都了。是我估算有些偏差了。不过再过一段时间,它必定不能再阻挡我!”

龙公心中冷哼。

之前,他对身上的伤势满不在乎,但现在却出现了七极荒都。

高手之争,向来只争一线。

龙公虽然已经可以力压七极荒都,一改最初对抗时的被动,但却未有足够的优势,能够摧毁掉它。

一阵强势的爆发之后,龙公攻势减弱,开始游斗。

冰塞川等人总算是松了一口气,龙公带给他们的压力太大了。

紫薇仙子等人则是目光复杂。越来越多的天庭成员知道,龙公的时间真的不多了。他这是在爆发出生命中最后的辉煌!就像是烟火,绚烂那么一刹那,在此之后,天庭将痛失这位传奇大能!

忽然,紫薇仙子得到龙公传音询问。

“紫薇仙子,现在局面如何?”

紫薇仙子连忙回答。

“帝君城被破,凡人死伤惨重……厉煌、清夜都在赶往不败福地战场……藏龙窟那边,东海八转蛊仙激战,如火如荼……”

龙公口中喃喃一阵,继而对紫薇仙子道:“很好,我没有看错你,紫薇。有赖于你的排兵布阵,这三大战场都守住了。帝君城战场是出现了天灾意外,并非是你的错漏。”

龙公对紫薇仙子不吝赞赏之词。他知道,自己虽是风头无两,但都有赖于紫薇仙子这样的智道大能坐镇后方。

紫薇仙子则对龙公更加敬佩,她诚挚地道:“这还多亏了龙公大人您夺来了龙宫。谁能想到,龙宫中竟包含了八转梦道仙蛊如梦令,使得龙宫可以奴役八转蛊仙!有了张阴等人,大大缓解了我们兵力上的不足。更因为龙公大人不惜身负重伤,隐藏了张阴、容婆等人的身份。这才使得我们算计东海蛊仙成功,否则的话,让这些东海八转流窜到三大战场中去,麻烦就大了。”

龙公:“是啊,如梦令的存在也令我颇感意外。”

他不禁回想起当初东海之争的情景。

张阴等四人一齐出现时,就令龙公有些意外。毕竟这四位不是散修,就是魔仙,又都是八转层次,心高气傲,平日里也不走动,关系并不亲近,居然联合行动。

让龙公产生更多怀疑的是之后的战斗。

张阴等四位蛊仙在战斗中,居然相互信任得很!

真正让龙公确定怀疑的是在龙公斗败四大蛊仙之后,龙宫忽然飞走。张阴等四人去而复返,却不追击仙蛊屋龙宫,反而是留下来继续围攻龙公。

若他们各怀私利,真正贪图龙宫,怎可能吃力不讨好地一起留下来,对付龙公呢?

若是正常关系,龙公如此战力,这四人定会对龙宫穷追不舍。毕竟龙宫飞走,气势惊人,东海的八转蛊仙可还有其他人在。而对付龙公,更像是为了龙宫争取时间。(详情见本卷第672章)

龙公在当时便隐隐猜到了这四位蛊仙,和龙宫的真正关系。直到他强夺龙宫,瞬间炼化之后,并证实了自己的猜想。

因为早有准备,所以他并没有让张阴等人暴露。当时若是暴露,也未必留得下其余的东海蛊仙。摆在天庭面前的头等大事,还是彻底修复宿命蛊。

想了想,龙公叮嘱紫薇仙子:“藏龙窟那边,还是要重点关注,不可大意。界壁消失,五域彻底合一即将发生,到那时,五域地脉合一,又因为此番大战,人情激动,帝藏生的战力将达到一种恐怖的层次,和我也难分高下。”

对于帝藏生,龙公当然比紫薇仙子更加了解。

紫薇仙子立即明白了龙公的话中深意。

当今战局,可以说渐趋平稳。不败福地、天庭、藏龙窟战场,都在天庭的掌控之中。

不败福地战场,厉煌、清夜以及一干仙蛊屋,即将到达。

唯一的变故,就在于藏龙窟战场。

龙公并不担心龙宫易主,他将龙宫摆放在那里,本身就是一个陷阱。龙宫在设计之初,就规定了必须是龙人方能掌控。要绕过这一点,根本不可能,除非是拆了龙宫。寻常的变化道手段,也难有成效。

龙公担心的是龙宫不稳,把帝藏生放出来。

帝藏生对天庭,对龙公,都仇恨至极。如今又实力暴涨,必定会大肆报复。

龙公命不久矣,帝藏生却是不死之身。

这样绝强的战力,一旦被方源等人利用,那就有大麻烦了!<!--80txt.com-ouoou-->

------------

\end{this_body}

