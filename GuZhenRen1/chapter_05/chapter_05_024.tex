\newsection{是机缘还是麻烦?}    %第二十四节:是机缘还是麻烦?

\begin{this_body}

%1
一般而言,南疆、中洲、北原、东海、西漠这五域中的荒兽,难以跨越界壁,前往他域。

%2
这些荒兽、上古荒兽、太古荒兽,可以媲美蛊仙,相当于一域之精华,很难脱离抚养它们的那一域天地。

%3
五域界壁,对它们而言,就像是一个枷锁。

%4
相比较而言,反倒是普通野兽,兽王,异兽,更加自由些,可以出入界壁。

%5
但现在追杀方源的这群上古云兽,并非南疆抚育造就,而是来源于太古九天之一的白天。

%6
五域界壁,对它们来讲,都没有亲疏远近之分,一视同仁,没有任何的束缚或者排斥。

%7
或许也是因为不论哪一层太古九天,都是覆盖整个五域。

%8
这个特性,早已广为人知。

%9
之前,魔尊幽魂逆天炼蛊,天庭蛊仙之所以能这么快赶到南疆,就是利用监天塔,穿梭白天。

%10
再之前,方源的星门蛊,能够利用太古黑天中的星光做引,跨越五域。

%11
两者的本质,和现下这群上古云兽是相同的,都是利用的太古九天的这份特性。

%12
方源在界壁中急速穿梭。

%13
身后的这群上古云兽,方源也想管也管不过来,只能任由它们跟在身后。

%14
这群云兽都是一根筋,盯上了方源,不达目的誓不罢休。

%15
这个时候,方源不禁又怀念起定仙游的好处了。

%16
若是有定仙游。他定能干脆利落地摆脱这些麻烦。

%17
就算方源身上有伤,上古云兽追踪方源总得有个范围的极限。利用定仙游,超出这个极限即可。

%18
但方源身上的移动仙蛊,只有剑遁仙蛊一只。

%19
一味狂催剑遁仙蛊,虽然可以拉开距离。但付出的代价就太大了点。

%20
最关键的一点,上古云兽的追踪距离的极限方源并不清楚。万一这个距离极限很高,方源就算耗费了整个身家,估计还不够催动剑遁仙蛊的损耗。

%21
“琅琊地灵虽然单纯,不会撒谎,但他也一直在算计我呢。”

%22
“之前对付戚灾,就被他算计了一把。输送暗渡仙蛊借给我。可是耗费了我不菲代价。”

%23
“我若是急于摆脱上古云兽的追杀。恐怕还要被他趁火打劫!毕竟落魄谷、荡魂山、智慧蛊,实在是诱人!”

%24
眼前的界壁,骤然转为深蓝之色。

%25
原来方源已经过了南疆的瘴气界壁,正式进入到东海的苍水界壁。

%26
没有什么可说的,方源继续闷头疾飞。

%27
一路畅通无阻。

%28
界壁中没有任何生灵定居生存,也无其他艰难险阻。

%29
因为五域界壁本身,就是最大的阻碍了。

%30
又疾飞一阵。方源扭头回望,那群上古云兽仍旧在他身后,不断追赶。

%31
云兽没有固定的形态,腾飞之时,宛若流云滚动,洁白如霜雪,姿态曼妙。但落到方源眼中,却是实在讨厌。

%32
到此时,方源的琅琊派贡献已经完全烧光,开始倒欠琅琊地灵仙元石。

%33
“如果不是这些上古云兽。我何至于此?嗯?怎么回事?”

%34
飞行途中,方源忽然神色变幻起来,眼中精芒瞬间暴涨。

%35
他惊奇地发现,在这苍水界壁之中,自身的气息开始发生一种玄妙的改变。

%36
属于南疆蛊仙的气息,正在不断地减弱缩小,而另一股属于东海蛊仙的气息。却随之不断地壮大。

%37
若是方源此刻催动见面曾相识,他绝不会如此惊奇。

%38
关键是,他此刻并没有催动这个杀招。

%39
“难道说,我的这具全新的身躯,不仅可以自由地穿梭五域界壁,而且还可以不断转换气息,变成各域的蛊仙?”

%40
方源心中暗自猜测。

%41
当他正式飞出苍水界壁的那一刻,他浑身上下的南疆蛊仙气息彻底消散,真正转变成一位东海蛊仙。

%42
方源当然又惊又喜。

%43
“这至尊仙胎蛊真是玄妙,我现在根本不用见面曾相识,就能完美地融入到五域当中去,不会被本域蛊仙排斥。”

%44
“不过现在,我还是催起见面曾相识为妙。”

%45
方源回头望了望身后吊着的那群上古云兽,叹了一口气,老老实实地催起见面曾相识,变化成一副陌生相貌。

%46
他原本想要静悄悄地穿过东海,回到北原的。

%47
但现在身后吊着这么一群上古云兽,原先的想法自然落空了。

%48
如此招摇,并非方源所愿,但他也实在是没有办法。

%49
“不过,这里距离乱流海域,却是不远。或许我可以绕一个路,转折到乱流海域,借助地利,甩掉这群上古云兽?”

%50
方源脑海中,忽然泛起了一个念头。

%51
得益于他五百年前世的颠沛流离的人生经历,使得方源对五域的地形都了然于胸。

%52
思考了一会儿,方源还是选择谨慎稳妥,放弃了这个想法。

%53
乱流海域宛若迷宫,一旦不慎陷入进去之后,就会身不由己,把自己给坑了。

%54
传闻中,乱流海域乃是蛊仙大能生死激战,形成的战场。一场血战,许多蛊仙大能陨落,留下不少传承和遗藏。

%55
许多东海蛊仙,以及来自中洲、北原、南疆的蛊仙,常会来乱流海域探索,寻求机缘。

%56
可惜,方源五百年前世,就从未听说过有人从中获得巨大机缘的。反而时常听到:某某蛊仙在乱流海域中失踪。或者某某蛊仙在消失数个月或者数年之后,脱困而出,重新现身。

%57
乱流海域并不危险,但是因为特殊的地形,导致蛊仙常常被困。不得自由。

%58
方源身形似箭,在高空中划出一道笔直的光线。

%59
他没有选择乱流海域的方向,而是直朝最靠近北原的界壁飞去。

%60
剑遁仙蛊催起,渐渐将身后的上古云兽甩远。

%61
他早在许久之前,就咨询琅琊地灵。询问上古云兽的情报,但琅琊地灵所知不多。

%62
方源又派出数十股意志,在宝黄天中搜寻,企图寻找到上古云兽的追踪距离的极限,但一直没有可喜的音讯。

%63
没办法,方源只好先试着尝试,看看能不能甩开上古云兽。

%64
狠下心来。方源一直催动剑遁仙蛊。一刻不停!

%65
终于他将上古云兽甩出视野之外。

%66
方源停下剑遁仙蛊,动用凡道杀招,继续飞行。

%67
速度因此陡降。

%68
“付出如此代价,若能甩开就好了。”方源心中暗暗担忧。

%69
看不到上古云兽了,并不代表已经甩开了它们。

%70
这种媲美七转蛊仙的存在,追踪的方式,往往并非简单的视觉目光。就看它们的追踪距离的极限究竟是多少了。

%71
就在方源时不时担忧回望的时候。一道血芒,忽然从东南方向飞来。

%72
血光划破天际,察觉到方源之后,顿时速度微微一滞,然后向着方源迅速接近。

%73
“是一位六转血道蛊仙,似乎经历了一番激战!”方源察觉到来者的气息,顿时皱起眉头。

%74
不管来者是何意图,方源已经不想再惹上什么麻烦。

%75
他已经够麻烦的了。

%76
身后的那群上古云兽,还不知道有没有甩掉。还有两个月不到的时间,灾劫就要来临。

%77
于是。他立即转折方向,远离主动接近过来的血光。

%78
血光中的蛊仙察觉到方源的意图,连忙大叫:“仙友慢走!我在乱流海域寻得无上机缘,只要你能助我脱离此劫,我愿动用血誓仙蛊,发血誓,将这份无上机缘和你分享!”

%79
“乱流海域?无上机缘?还真有人从中获得过好处吗?”方源心中着实诧异了一下。

%80
就在这时。又有数位蛊仙从血光的后方出现,急逼而来,气息强烈,气势汹汹。

%81
他们似乎听见了,刚刚血道蛊仙对方源喊的话。

%82
一位位皆连叱骂道

%83
“血道魔头,危害苍生,人人得而诛之!”

%84
“一律和魔头同流合污者,杀无赦!”

%85
“前方仙友,拦下这位魔头,我汤家必有赏赐。”

%86
“无须如此。这魔头中了我刘青玉的仙道杀招,肯定跑不了了。前方来者,若识好歹,就赶紧给我滚!”

%87
一时间,方源被这些东海蛊仙又是劝说,又是警告,又是喝骂的。

%88
方源冷哼一声。

%89
他绝非怕事之人,但此刻他的当务之急,是要赶往北原琅琊福地,好应付仙窍灾劫。

%90
方向一折,方源直接朝血道蛊仙,反向撤离。

%91
血道蛊仙大急,他已经到达了自身极限,方源是他唯一的生存希望。

%92
方源一退,他就连忙也改变方向,迅速朝方源靠拢过去。

%93
“你若助我,我愿将这份机缘,都统统送于你!”他大喊道。

%94
方源冷笑,义无反顾地后撤。

%95
见此,东海蛊仙中有人大笑:“对!你这小子识时务得很,快给老子滚!”

%96
也有人喊道:“前方仙友,不妨出手相助我等,事后必有酬谢。”

%97
方源眉头皱起。

%98
自己只想悄悄地回归琅琊福地,怎么一路行来,麻烦事一桩接着一桩。自己不想沾染,但这些麻烦却主动来黏上他。

%99
“仙友,这就是传承的关键信息,你若得之,必将修为突飞猛进,成为仙上之仙啊!”血道蛊仙抛出一只信道蛊虫,飞给方源,企图诱惑他。

%100
方源冷喝:“滚开!”

%101
挥手一扫,发出一缕劲风,就将蛊虫击爆。

%102
与此同时,他身形陡然拔高,一飞冲天。

%103
血道魔修算计不到方源,见此顿时绝望无比,他忽然打出一道奇光,射向方源。

%104
“我已将传承的关键,交付此人。你们要取机缘,杀我无用!”血道魔修大喊一声,身形猛折,向着反方向急退。

%105
奇光速度极快,向方源逼来。

%106
方源冷哼一声,极不愿沾惹麻烦,低喝出声道:“老子有要事在身,都别来惹老子!”

%107
说着,他同时催起剑遁、剑光两只七转仙蛊。

%108
剑遁让他速度激增,剑光仙蛊则直接劈中奇光,将其粉碎。

%109
“啊,两只七转仙蛊!”

%110
“此人是谁?!”

%111
方源陡然爆发,众仙顿时大为惊诧,一时间脸色皆变。

\end{this_body}

