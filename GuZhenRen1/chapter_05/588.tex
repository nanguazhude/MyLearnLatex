\newsection{勒索的门道}    %第五百九十节:勒索的门道

\begin{this_body}

方源在忙什么,自然不言而喻。

俘虏了南疆群仙,方源已经牢牢掌握着主动权,占据的优势极其巨大。

如此价值连城的人质在手,敲诈勒索起来,堪称恣意妄为,随心所欲。

不过方源却是性情谨慎,并没有大肆出手。他第一波敲诈勒索的对象,只是三家。

夏家毫无疑问,是排在第一位的首先。太上大长老夏槎,又是夏家唯一的八转蛊仙,堪称夏家之根,太重要了。夏家根本损失不起,一旦失去,整个家族的势力就要萎缩七八成!

而翼家排在第二位。

因为方源这一次俘虏的翼家蛊仙,乃是翼扬。

这位蛊仙那是翼家七转中的最强者,专修宇道,被所有人看好,有着八转资质。更重要的是,他是翼家太上大长老的嫡亲血脉,是太上大长老刻意栽培的接班人。

方源五百年前世,五域乱战,南疆内斗不休。翼扬被陷害,就曾经落入到商家手中,翼家太上大长老不惜耗费巨资,从商家手中赎回翼扬。

所以,方源对翼家的赎金,也很是期待。

最后第三位,则是池家。

方源手中的池家蛊仙,乃是家族的太上二长老,位高权重!再加上池曲由之前就和方源交易过,获得梦道成果。有这样的交易先例,池家妥协也很自然。

方源深谙人心,他首先挑选出来勒索的对象,避开了仇恨最高的武家,也避开了最仇恨魔道的铁家。

只要这三家就范,接下来,方源敲诈起其他南疆正道势力,就容易得多。

毕竟有这样的先例在,正道的心气劲儿就没有那么大了。

至尊仙窍。

小绿天,仙道大阵。

仙道杀招大盗鬼手!

鬼手扑进翼扬的仙窍当中,旋即归来,手中握着一只七转仙蛊。

影无邪、阵灵都在一旁静静地瞧着,一脸麻木的神情。

他们已经见怪不怪了。

因为这段时间,方源搜刮出来的仙蛊,数量巨大!并且之前就将夏槎的两只八转宙道仙蛊春、夏,偷盗出来。有这样的震撼,接下来就显得不温不火。

为了炼化这些仙蛊,方源特意搭建了另一座仙阵。

这座仙阵,就是以曾经的智炼蛊阵为基础,加上许多智道仙蛊,改良出来的全新版本。

智炼蛊阵,是方源利用智慧光晕,辅助妇人心、解谜,炼化他人仙蛊。若有他人意志,效果更佳。方源就利用过墨瑶假意,帮助他炼化了从薄青仙僵那里,得来的一系列仙蛊,其中就包含了义天山大战,最至关重要的换魂仙蛊。

现在,方源失去了妇人心,仍旧以智慧光晕为核心,全新的智炼仙阵威能更加强大。

尤其是智慧蛊高达九转,压制这些七转仙蛊,非常容易。

再加上方源的阵道境界、智道手段,都有了长足进步,导致智炼蛊阵效果出色,成绩斐然。

片刻之后,方源就将新盗取出来的仙蛊炼化。

这是一个宇道仙蛊,好像是一个湛蓝玉球,长着一对雪白的羽翅。

此蛊方源认得,名为行空,有转移腾挪的良效。影宗真传中就记载了相关内容,比如此蛊最初是被人在一头上古荒兽级的天马身上,捕捉到的。

方源将这只行空蛊收起来。

宇道他并不在行,境界只是普通,并且身上的宇道道痕不多。

不过方源手中的真传太多,仙道杀招、仙蛊方堪称海量,运用仙蛊的方法从来不缺乏。

时至如今,他已经将南疆群仙中的大半蛊仙的仙蛊,都掏了出来。

至于仙元缺乏的问题,早已经解决了。

夏家之前送上来的一百万块仙元石,就像是天河倒悬,灌溉到将要干涸的田地上。

一百万块仙元石,就意味着一百万颗青提仙元,转化成红枣仙元,就是一万颗。这样的数目,已经大大超越了方源最高的红枣仙元储备记录。

方源接下来,又继续施展大盗鬼手,但都未从翼扬的身上,再掏出什么仙蛊来。

于是他便将此人重新放回到梦境之中,又拽出一人,正是池家的太上二长老,土道七转蛊仙。

方源毫不犹豫,再下毒手。

这些南疆蛊仙身上的仙蛊,他都要拿走,放入自家腰包。

至于人,还是要先放一些的。

敲诈勒索也可以看成是一种交易,只不过卖方很强势,买家很记恨无奈。而做买卖当然要讲究信誉,不放出一些人质出来,南疆正道如何相信方源,又如何能继续饱含希望地被勒索呢?

方源打算先放池家的太上二长老,再放翼家的翼扬。

池家毕竟是有过交易,将来方源还打算扶持池家,对抗天庭。当然了,这两家肯定是要被继续勒索,直至底线。

尽管培养出一位蛊仙,非常的不容易。但是人质的价值,都是一定的,都有着上限。

这些正道势力,屹立至今这么多年,也都不是傻子。

如果方源要求的太多,大大超出一个蛊仙应有的价值,或者买家的心理承受价位,他们是不会做亏本的买卖的。

总而言之,勒索是有技巧的。

数天后,又一批仙元石到手。

那是翼家的第一笔赎金,一百二十万仙元石。

轮价值,翼扬是完全抵不上夏槎的,但翼家本身是很富有,至少要超过夏家,所有手中现有的仙元石储备,就有更多。

一个超级势力的仙元石储备,或多或少,但基本都在一百万附近。

所以,方源第一笔敲诈仙元石,不仅是因为他自己本身稀缺,更主要的原因,是其他势力完全能够拿得出手。虽然是有一定难度,但难度很小。

先让他们就范,再接着用越来越苛刻的要求勒索他们,一刀刀让他们放血。方源深谙勒索的门道。

“只是这一次,我虽然俘虏了南疆群仙,但对天庭的试探,却没有得到任何的反应。天庭方面究竟在光阴长河中准备了什么呢?”

方源心中不断地思量。

他这一次建设年流伏诛大阵,重点自然是为了对付南疆群仙,但也有引诱天庭出手的意思。

天庭若是有伏兵,查探到这边的动静,极有可能顺着光阴支流前来,夹攻方源。

但至始至终,方源都没有等到天庭方面的伏兵出动。

方源计谋落空,不禁更添忌惮和猜疑。<!--80txt.com-ouoou-->

\end{this_body}

