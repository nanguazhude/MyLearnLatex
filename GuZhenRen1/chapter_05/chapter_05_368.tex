\newsection{这便是凤九歌!}    %第三百六十八节:这便是凤九歌!

\begin{this_body}

%1
一位中年蛊仙,出现在方源和武庸的面前。

%2
他一身红白相间的长袍,身姿挺拔,似枪如剑。剑眉入鬓,眼中蕴藏神芒,他嘴角微微泛出温柔的笑意,风流倜傥却又不失昂扬霸气。

%3
不过此刻,他的面色上却流露出一丝惆怅之色。

%4
正是名满天下的中洲蛊仙,灵缘斋的台柱子,音道七转,能力战八转的凤九歌!

%5
他怎么会出现在这里?

%6
武庸、方源免不了惊讶。

%7
凤九歌的相貌和事迹,早已经传遍了五域。因此虽然武庸只是第一次亲眼见到凤九歌,但他迅速认出了凤九歌此人。

%8
毕竟七转蛊仙,能够力战八转的,这是千年不出的绝代天骄!

%9
若非方源的另一个身份柳贯一,凤九歌恐怕仍旧要独领风骚。

%10
武庸脸上的惊异之色,迅速收敛起来,他对凤九歌冷笑:“中洲的蛊仙,居然还敢出现在我的面前。很好,很好!”

%11
梦境一战,天庭获利最多,将南疆正道的许多仙蛊都卷席带走。

%12
凤九歌此时,便承受此事的影响,惹来武庸的敌意。

%13
方源的目光,也凝聚在凤九歌的身上。

%14
这不是他第一次见到凤九歌了。

%15
五百年前世,他早已经将凤九歌的相貌,印刻在心底。没办法,凤九歌在五域乱战中爆发出来的光彩,甚至盖过寻常的八转蛊仙。可惜他最终战死在了琅琊福地当中。

%16
他的死,直接引发了轩然大波,让中洲动荡,更让其他四域欢欣鼓舞。

%17
而重生到了今世,方源也见过凤九歌。

%18
就在落魄谷中的那个盗天真传的奇异空间里,方源在阴差阳错之下,解救了凤九歌一命。

%19
本来他会因为秦百胜的自杀攻击,而陨落在盗天真传的奇异空间之中。但是方源重生之后,改变了这一切。

%20
在这个角度来看,虽然方源是无心无意,但他的确是凤九歌的救命恩人。

%21
但救命恩人又能如何?

%22
这个世界上,恩将仇报的事情还少吗?

%23
方源凝视着凤九歌,不敢有丝毫大意,冷喝出声:“凤九歌,你便是天庭安排,狙击我的人吗?”

%24
凤九歌望了一眼武庸,之后就将目光定格在方源的身上。

%25
再一次见到方源,还是这种情景之下,让凤九歌心中充满了感慨。

%26
方源的进步太快了!

%27
第一次,凤九歌知道方源的时候,后者才是刚刚抢夺了凤金煌的机缘,成为狐仙福地的新主人。

%28
凤九歌因为自己女儿的关系,知晓了方源的姓名。

%29
不过他一点都不放在心上。

%30
为什么?

%31
因为当时,方源不过是区区一介凡人。仙凡之辈,宛若天地云泥,双方根本就不是一个层次的生命。甚至可以这么说,能够惹来堂堂蛊仙凤九歌的一丝关注,这已经是身为凡人的方源的莫大荣耀。

%32
但之后的事情发展,出乎凤九歌的意料。

%33
方源做下大案,屡次为祸世间,他的身份曝光,天外之魔,拥有春秋蝉,还继承了巨阳真传等等,让世人惊叹。尤其是方源的进步过于神速,从凡到仙,再成为蛊仙中的强者,仿佛一蹴而就之事。

%34
多少蛊仙努力无数岁月,都达不到这样的成果。

%35
但似乎,方源就这样轻而易举地达到了。

%36
第一次真正见面的时候,凤九歌居然是被方源救了一命。

%37
方源顺势继承了盗天真传之一的鬼不觉,而后,义天山大战,方源更是在此中发挥了举足轻重的巨大作用。

%38
“没想到柳贯一也即是方源。大时代将临,英雄辈出,方源就是当前最闪耀的一颗了。”再一次见到方源,凤九歌用目光打量的同时,也在心中做出评价。

%39
与此同时,他开口,带着一丝苦笑道:“我从中洲一路赶来,只是为了追逐一直盗天魔尊曾用的仙蛊。不想无意撞见了此事。”

%40
“这么说,他不是天庭派遣的人?”方源顿时皱起眉头。

%41
依凭凤九歌此人的性情和行事风格,他没必要在这个关头哄骗方源。

%42
武庸却是不信,他冷笑起来:“没想到凤九歌也有虚伪的一面。我施展战场杀招之时,你若不想掺和此事,绝对能够轻松退走,不会被囊括到战场当中来。”

%43
凤九歌点头:“不错。我是主动进来,这是因为我欠方源一命。若是没有撞见,也还罢了,既然撞见了,便要出手相助。”

%44
“什么?”武庸惊愕。

%45
方源也听得呆了。

%46
武庸难以置信地反问道:“我刚刚没有听错?你欠他一命,所以你要助他?哈哈哈。”

%47
武庸大笑起来。

%48
凤九歌是中洲蛊仙,十大古派中人,谁都认为他将来必定成为天庭中的一员。

%49
他居然要救方源。

%50
方源是什么人?天外之魔,天庭诛魔榜上名列前茅的魔头贼子。

%51
凤九歌居然要出手助他?

%52
这简直是天大的笑话。

%53
不止是武庸大笑,就连凤九歌自己嘴角上的微笑也随之浓郁起来。

%54
武庸看见凤九歌的微笑,他的大笑声渐渐停止,脸色逐渐转为严肃。

%55
他知道了。

%56
凤九歌是认真的!

%57
这人简直是疯了!!

%58
一个正道蛊仙,要出手帮助一位魔道贼子。

%59
凤九歌绝非寻常的正道蛊仙,他是正道的巨星,尽管只有七转修为,但世人已经把他当做一位准八转的大能看待。他在中洲,乃至五域的声望,非常巨大。

%60
而方源同样不是普通的魔道中人。这些年来,整个天下就数他风头最劲!他是完整的天外之魔,拥有春秋蝉、逆流河……巨阳真传、盗天真传……,他在中洲十大派的眼皮子底下,抢走了狐仙福地。他在北原捣毁了八十八角真阳楼,他来到南疆,击败了魔尊幽魂的逆天大计,抢夺了胜利果实。最近他居然成功地混进了武家,梦境大战后,成为影宗之主!

%61
凤九歌要出手,帮助方源这样的人物?

%62
武庸也是正道蛊仙,这种事情,他设身处地想想,就算是方源救了自己一命,要让他出手帮助方源,该有多么巨大的压力!

%63
这种压力不仅是来自于他的阵营,还来自于他的妻儿,他的门派,中洲的天庭,甚至全天下的蛊仙。

%64
“不!凤九歌这人,和我不同。他不是正儿八经的正道出身,他早年的时候,就是魔道蛊仙啊。”武庸忽然想到一点。

%65
“魔道贼子,果然疯狂,难以理喻!”武庸心中愤愤。

%66
方源则是想起了凤九歌的生平之事。

%67
凤九歌早年的时候,只是一个很不起眼,一点都没有什么名气的隐修。

%68
他专修音道,矢志不渝,向其中投注生命。

%69
有一次,他在无名的小山谷中高歌,引来另外两位蛊仙的对唱。

%70
当时,正值夜晚,明月高悬,清风徐徐,吹得山谷中一座小湖波光粼粼。

%71
三位蛊仙唱和之间,时间飞速流逝,竟一连唱到了天明。

%72
三仙唱罢,纷纷大笑。却不照面,兴尽而归。

%73
几年之后,凤九歌才知晓这两位的姓名和来历。当时中洲蛊仙界盛传,这两人抢夺了黑天寺的一只七转仙蛊,是魔道蛊仙,正被中洲十大古派联手追捕。

%74
凤九歌便动身前往相助。

%75
那两位蛊仙本是走投无路,众叛亲离,见到凤九歌来援,既感动又奇怪,便问凤九歌为什么会这样做?

%76
凤九歌便道:“几年前,我与二位对歌,我唱明月,两位仙友一唱青山,一唱白湖。清风明月,自照人心。能够唱出这样的歌声,怎可能是贪图仙蛊的小人?我相信二位仙友。”

%77
两人闻言,感动得落下泪来。

%78
一人道:“仙友知我等。这只七转仙蛊,本是我二人从一处真传中继承得来,不想却被黑天寺诬告。黑天寺乃是正道十大古派之一,它说什么,由不得他人不信。”

%79
另一人则劝凤九歌道:“仙友,你我只是和歌一曲,素味平生。我二人惹上杀身之祸,已无幸免希望。仙友速走,现在还来得及。”

%80
凤九歌却摇头,执意要留下相助。

%81
两仙急道:“仙友若不走,恐怕也会被诬告为魔头了。”

%82
凤九歌便笑:“魔不魔,正不正,天地自有凤九歌。走不走,留不留,死生皆在我心头。”

%83
两仙既感且佩,皆是动容落泪。

%84
凤九歌所吟之诗,当时并不出名。但在随后,他痛击黑天寺的蛊仙,三番五次,次次皆赢,事情越闹越大。

%85
其余九大古派有所耳闻,纷纷声援黑天寺,带来巨大的舆论压力。

%86
凤九歌索性扬言,要挑战中洲英杰,行走天下。

%87
十派遣人单挑,接连败北,不得不联手抗敌。

%88
凤九歌昂然不惧,一路转战三千万里,忽然调转兵锋,直捣黄龙,将十大派闹得灰头土脸,一片混乱,无可奈何。

%89
最终,灵缘斋出手,当代仙子白晴,以情动人,感化了凤九歌,令其成为灵缘斋的成员。

%90
昔日,凤九歌可以为一次的对歌,去舍身帮助两位素昧平生的陌生蛊仙。反被诬告成魔道贼子。

%91
那么现在,凤九歌又为什么不能去助魔头方源?

%92
尤其是,方源还是他的救命恩人!

%93
曾经的凤九歌,仍旧是现在的凤九歌。

%94
方源忽然想明白了凤九歌之前的那声叹息。

%95
凤九歌早就想出手,帮助方源。但是又要考虑到妻女,以及自己的处境。恐怕当时,他打算暗中出手。

%96
不过可惜,他的行踪被武庸叫破了。

%97
这让凤九歌不得不做出抉择。

%98
就像他曾经做出的抉择一样。

%99
他做出了相同的抉择。

%100
魔不魔,正不正,天地自有凤九歌。

%101
走不走,留不留,死生皆在我心头。

%102
魔道、正道,这两种阵营,两种身份,都不能拘束我凤九歌!

%103
是走还是留,对于我凤九歌而言,走了就是我心之死,留下才是我心之生。

%104
凡事依凭本心。

%105
我就是我。

%106
我就是凤九歌。

%107
一直都是!

%108
ps:这章有点难写,做了两次删改。

\end{this_body}

