\newsection{四族大联盟}    %第一百二十五节:四族大联盟

\begin{this_body}

%1
又过了三日。√∟,

%2
冰原地下深处,地母坛中。

%3
四炷香,香烟缭绕,四方蛊仙各自站立一处,阵营分明,面色皆是肃穆。

%4
方源也身在其中,身边的人都是琅琊派的毛民蛊仙。

%5
悠扬的号角声吹响,随后又是鼓声轰鸣。

%6
石宗在鼓声中,踏前一大步,首先开口高呼道:“石人一族蛊仙石宗,愿加入四族联盟,从今以后,待四族为同胞,荣辱与共,福祸并存。”

%7
说完,他便调动仙元,引动了地母坛中的某个变化。

%8
缭绕在空中的烟气,纷纷涌到石宗的身边,盘旋了十几个呼吸之后,全都融入到了石宗的身体当中。

%9
“这是信道道痕加持上去了!看这形势,似有上古之风……”方源心中暗忖。

%10
本来是毛民一方,和雪人、石人的联合部族,商讨联合事宜。

%11
但事情出乎意料的顺利,双方越发契合。毛民一方,主动提议将墨人城也拉进来。

%12
雪人、石人的联合部族了解了一番后,都点头同意。

%13
长毛老祖和一言仙,乃是历史公知的知己好久。长毛老祖留下琅琊福地,而一言仙创建墨人城。墨人和毛民一直都有紧密联系。

%14
既然毛民一方做出担保,雪人、石人联合部族也就选择了相信。

%15
所以,又成了三方会谈,牵扯了四个异人种族。

%16
最终,三方达成一致,都愿意结成联盟。患难与共,今后互帮互助。

%17
于是就有了现在的这一幕。

%18
三方动用上古信道手段。正式结盟!

%19
一位位的蛊仙,轮流上去。接受信道道痕的加持。如此便算是四族大联盟的成员了。

%20
时间流逝,方源渐渐看出一些端倪。

%21
“这好像不是什么正统信道,而是以土道塑造烟灰,旁敲侧击,达到信道的效果。一如我的宙道仙级杀招百年好合。”

%22
想到这里,方源心中微微一沉。

%23
今后要破解这个盟约,却是麻烦了。因为从信道方面着手,都不太能够解决这事。很有可能,还要从土道上想办法。

%24
“方源长老。该你了。”这时,身旁的蛊仙毛十二,小声地发出善意的提醒。

%25
说起来,自从解决了落星犬一事,又得知方源掌握太古上极天鹰之后,毛民蛊仙们对于方源的态度发生了翻天覆地的转变,变得敬畏有加,和之前形成了鲜明的对比。

%26
方源点点头,毫不犹豫地走上前去。一如石人蛊仙石宗所做的那样,任凭香烟刻印在自己身上。

%27
他的一举一动,都牵引了场中群仙的目光。

%28
见到他真正加入进来,群仙们都隐约松了一口气。放下一层担心,雪儿脸上的笑容更见娇艳。

%29
殊不知,方源却已经在心底琢磨。该如何破解身上的各种盟约了。

%30
他当然不可能和这些异人们一条心。

%31
加入琅琊派,只是为了更好地利用琅琊福地的力量。

%32
同样的。加入这个四族大联盟,方源除了形势所逼之外。也是为了更好地借助大联盟的力量,来帮助自己修行。

%33
“真要让我和异人们一个阵营,对抗人族,那是不可能的!”

%34
“人族占据绝对的大势,就算四大异人种族联盟,又能怎样?”

%35
“不过,这股力量倒也不可小觑。我五百年前世,这四族有没有联合起来?”

%36
方源思绪飘忽,暗暗揣度。

%37
他并不知道这个问题的答案。

%38
五百年前世,他是中洲六转血道蛊仙。对于北原蛊仙界的事情,还了解得不够深入。尤其是这种消息内幕,恐怕就算是北原的蛊仙也未必知道。

%39
“但我知道,五百年前世,这些雪人、石人部族一直悄无声息,说不定被灭了,也说不定仍旧偷偷活着。而琅琊福地一直没有建立帮派,最终被天庭所灭。反倒是墨人一族,混得风生水起,情况最佳……”

%40
想到这里,方源的目光便瞄向墨坦桑。

%41
此人修为只是六转,修行的是气道。气道比力道的历史还要悠久,但同样属于日渐落寞的蛊修流派。

%42
他中年模样,黑富白发,不笑的时候,颇有威严。笑的时候,则给人十分可亲近的感觉。

%43
“这人才是枭雄啊。”方源心中感慨。

%44
他记得,五域乱战时期,就是眼前的这位墨坦桑,趁着人族内斗,无暇顾及他时,抓住机遇,积极发展,将墨人势力大大扩张。

%45
待北原人族势力想要打压他时,他居然不顾王者威仪,主动投靠了刘家。对刘家太上大长老行奴仆之礼,以奴仆身份自居。

%46
刘家乃超级势力之一,因此力保墨人势力。墨人势力在这一层保护伞之下,稳步发展。

%47
而后刘家衰败,墨人王立即舍弃刘家,和马鸿运平等合作。

%48
方源自爆之前,墨人已经有用城池数百座,占据北原三分之一的江山。这种规模和势力,想必方源现在说出来,也未必有人信。

%49
在这四族大联盟中,无疑是墨人一族最为弱小。

%50
但在五百年前世,反倒是墨人的情况最佳。

%51
墨坦桑能屈能伸,是墨人一族崛起的灵魂人物。同时,也有外部的环境。五域乱战,一切秩序都崩溃,机会和风险无处不在。

%52
似乎是察觉到了方源的目光,墨坦桑回头看向方源,立即咧嘴一笑,表达出自己的深切善意。

%53
这种善意是如此浓郁和明显,以至于他笑的时候,近乎点头哈腰,甚至有一些谄媚。

%54
方源心中冷哼一声,表面上也回以微笑。

%55
墨坦桑的神情微变,表达出一种得到大人物赏识和青睐的激动之情。

%56
若是旁人,此刻恐怕会感到一股优越感和飘飘然了。

%57
“还真是厉害。”方源自诩演技不俗,但眼前这位墨坦桑绝不弱于他。

%58
“世界之大,能人辈出,万不能小觑天下英杰了。”方源暗暗告诫自己,心中始终冰雪般冷静。

%59
北原的局势,可谓一波未平一波又起。

%60
黑家覆灭,百足家取而代之的风波,还未平息。雪胡老祖正式开始炼制鸿运齐天蛊,将整个北原蛊仙界的目光,都集中到了大雪山上。

%61
而在背地里,人族蛊仙不知道的角落,异人四大族毛民、雪人、石人、墨人,悄无声息结成了一致联盟。

%62
虽然异人处于绝对弱势的地位,但这个联盟却不可小觑。因为它拥有八转战力太古石龙。当然,在其他联盟成员的心中,还要再添上方源手中的太古上极天鹰。

%63
方源迫不得已,加入其中。

%64
不过让他感到稍微放心的是,这个四族大联盟中,并没有智道蛊仙存在。

%65
联盟成立之后,方源的损失显而易见。

%66
那就是今后,他不能在北部冰原这里渡劫了。

%67
这无疑对方源影响很大。

%68
但没有办法。

%69
在这里渡劫,消耗地气,是对石人居所的最大损害。这也是这些石人、雪人不惜代价,架设战场杀招灰云战傀也要消灭方源的原因。

%70
不过,除了损失之外,方源也有收益。

%71
和石人、雪人一族开通的贸易,让他每月的盈利更上一层楼。

%72
雪人也就罢了,关键是石人,对于胆识蛊的渴求,比人族还要高!

%73
为什么?

%74
回想当年,方源如何利用胆识蛊,增加狐仙福地石人部族人口的事情,就可知晓了。

%75
胆识蛊对于是石人而言,是壮大种族的最佳方式。

%76
除此之外,方源将不久前,从黑凡洞天中收获的一些资源,也转卖给了石人、雪人以及墨人三族,赚得瓢盆满钵,把一旁的毛六看得都要着急上火了。

%77
“方源的收获居然这么大!这可如何是好?今后如何遏制?!”

%78
离别的时候到了。

%79
“方源长老,你真的急着要走吗?留在这里做客一两天也好。”雪儿试图挽留方源。

%80
方源摇摇头,表面上温和地微笑着:“我们结成联盟,此事事关重大,我也得亲自回去向琅琊地灵汇报才是。你放心,我会回来的。咱们也算是不打不相识吧,这里给我的感觉相当好。”

%81
方源对雪人蛊仙的冰雪流派的仙蛊,还有杀招都十分感兴趣。

%82
这两者对于他而言,也十分适宜。

%83
毕竟,他现在身上的道痕,最多的就是冰雪道了。

%84
依照他和雪人一族现在的关系,还有八转战力的招牌在手,互换仙蛊完全可行。不过方源并不打算放弃自己手中的某只仙蛊。他还有更好的计划。

%85
比如,在将来的某天,某位异人蛊仙难于渡劫,需要他的宙道仙级杀招来延缓仙窍时间什么的。

%86
如此一来,方源就能趁机索要仙蛊了。

%87
当年这种事情,黑凡可干得不少,得到仙蛊的数量也相当可观。

%88
雪儿自然不知道眼前这位俊美少年,居心叵测,一肚子算计,她见方源执意要走,低下头,露出犹豫挣扎之色。

%89
但很快,她的眼中涌现出一股决意,猛地抬起头来,然后伸出手将一件东西,交到方源手中。

%90
“方源长老,这件东西就当雪人送给你的分别之礼吧。”雪儿强自镇定地说道,声音微微颤抖。

%91
方源接过一看,是一颗泪冰!

%92
ps:今天状态有点不行啊,主要是设想人祖传花费了许多时间,结果什么都没有,唉!这个神话故事完全是我个人草创,越写越艰难。22点第二更。

\end{this_body}

