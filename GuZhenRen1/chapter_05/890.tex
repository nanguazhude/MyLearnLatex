\newsection{沈从声的仙材}    %第八百九十四节:沈从声的仙材

\begin{this_body}

%1
“这黑火还能变化?”方源眼中精芒一闪,旋即摇头,银白水气仍旧气息古怪,不属于任何流派。

%2
银白水气似乎和黑火并无本质上的不同,沈伤在其笼罩下,仍旧发疯,两头太古年兽也在持续哀嚎。

%3
方源再射破晓飞剑,竟发现破晓剑的力量居然助长银白水气,令后者规模见涨。

%4
惊疑之下,方源立即转变流派,换成风刃狂飙而去。

%5
银白水气再次被削弱,只是衰减的幅度仍旧十分微小。

%6
攻击持续一阵后,银白水气陡然化为三团大旋风,罩住沈伤和两头太古年兽。

%7
方源的风刃射入旋风之中,开始再次助长后者威能。

%8
方源浅尝辄止,明智地开始转变手段。

%9
换做其他蛊仙或许早就无奈罢手,但方源却非如此。他兼修全流派,手上仙蛊之多惊世骇俗,拥有多个雄厚的流派境界,转换各大流派更是得心应手。

%10
古怪的黑火在方源的攻击下,不断变化,越来越小。

%11
那两头太古年兽首先尸骨无存,方源很快就只剩下沈伤这个目标了。

%12
“且慢,且慢!方源仙友,我有一笔天大的富贵赠于你。”沈从声再也看不下去,连忙出声呐喊。

%13
让他喜出望外的是,方源竟然真的罢手了。

%14
方源看着沈从声,一脸笑意:“沈家仙友,你是想让我放过你家的先祖吗?要知道这可是魔仙,乐土仙尊当年亲手镇压的魔仙。他现在的状态你也看到了,疯魔如兽,神志不清,和功德碑上所述的一致。你还想救他?”

%15
沈从声沉默,深深地望了光柱中被囚禁的沈伤一眼,下定了决心:“我相信这种疯狂,只是一时的。若是当我救出沈伤先祖,他就恰恰堕落疯狂,这未免也太过巧合了吧?”

%16
“那你又能付出什么样的代价呢?”方源问道。

%17
“你瞧瞧这个如何?”沈从声掏出一份仙材。

%18
这是一个袋子。

%19
袋子不大,似乎只能装下成年男子的一个拳头。但袋子表面却是布满了一个个的纹路,这些纹路好像宝石,又仿佛是蓝色的漩涡。

%20
仔细盯着看,就会察觉到这些蓝宝石般的漩涡,正在不断地旋转。

%21
但是当人回过神来,这些纹路还是纹路,似乎刚刚看到的一幕只是一个错觉而已。

%22
方源却是眉头微扬,轻笑出声:“原来是宝蓝狸的胃袋。”

%23
宝蓝狸乃是太古荒兽,常常周游在东海和太古蓝天之间。如今太古蓝天早已破碎,而宝蓝狸也绝迹。历史上,屠戮宝蓝狸最多的蛊仙就是空绝老仙。

%24
空绝老仙乃是炼道三位无上大宗师之一,和长毛老祖相媲美。能够得到他的觊觎和认可的宝蓝狸,自然不同寻常。

%25
别的不说,单单这个胃袋就是半个仙窍,它蕴含的是宇道道痕和水道道痕,内有乾坤,可藏万宝。蛊仙得之,可以将它融汇到自己的仙窍中,增长自家洞天的内部空间。

%26
正因为这个特性,导致宝蓝狸一直以来都被蛊仙们大肆捕杀。所以宝蓝狸的绝迹,也并奇怪了。

%27
宝蓝狸的胃袋,可以说是它一身最具价值之物。并且在当今宝蓝狸灭绝的情况下,这个胃袋的价值还要再上涨数筹!

%28
简而言之,这是一个极其罕见,价值很高的宝物。

%29
方源伸手,直接从沈从声的手中将这个胃袋取来。

%30
胃袋轻飘飘的,表面十分滑腻。

%31
方源检查了一番,没有发现动手脚的地方,便当这沈从声的面堂而皇之地将胃袋放入自家仙窍。

%32
然后,他笑了笑:“还有什么?”

%33
沈从声深呼吸一口气,吃力地掏出一个黑色的长棒子:“还有这个,接好了。”

%34
他吃力地一抛,将黑色棒子抛给方源。

%35
方源伸手接住,这棒子十分沉重,不过他亦有不少力道道痕在身,自然是轻轻松松。

%36
不像沈从声是一个纯粹的音道蛊仙。

%37
这根黑棒比方源还要高出一头,鹅蛋粗细,表面也十分粗糙,重量不轻,仿佛是粗制滥造的铁器。

%38
不过,方源却从中感受到一股浓郁的生命气息。

%39
这是一株太古荒植!

%40
但它没有树根,没有枝叶,只有唯一的树干。

%41
在树干的顶端,有着一颗颗的星纹,这些星纹有的五角,有的两角,排列紧凑,并且都散发着淡淡的星光。

%42
“落星棒子树。”方源开口道。

%43
沈从声点点头,赞叹道:“方源仙友博文广知,令人佩服。”

%44
这棵树的生长条件至今是一个谜团,并且从古至今都分外罕见,数量极其稀少。

%45
它有独特的天赋,能不断地吸引星辰,落到树干中,化为一颗颗的星纹。所以称之为落星。

%46
据传闻,元莲仙尊当年独游天下的一个动机,就是要收集这种树。

%47
落星棒子树的价值,不只是它本身。

%48
蛊仙将它栽种在仙窍中,可以利用它打造出一副星辰的运转体系。这是一个独特的生态,带来的收益每年递增,天长日久,年复一年,收益总量几乎无法估算。

%49
和宝蓝狸的胃袋一样,方源再次将落星棒子树塞入仙窍,神态之自然几乎让沈从声有了一种错觉——好像这棵珍稀的太古荒植本来就是方源的一样!

%50
“还有什么?”方源笑着道,“单凭这些可换不来一位八转大能呐。”

%51
沈从声咬咬牙,带着一丝肉疼之色:“我相信这份仙材,一定能够满足仙友了。”

%52
这一次,他取出来的是一团水。

%53
水被拘束一团,不断掀起波澜。水波不只是在水的表面,即便在水中内部也是升腾不休。

%54
虽然有无数波澜,但是却毫无杂乱。

%55
水波进退统一,这一刻一齐进,下一刻一齐退,堪称进退有序。

%56
方源见此,眼中精芒一闪,沉声道:“进退潮汐水。”

%57
没错,这就是天下三水之一的进退潮汐水。

%58
它是九转仙材!

%59
单是这一点,就远远超过宝蓝狸胃袋、落星棒子树总和。

%60
更别说它还有诸多妙用。

%61
“我这里一共有十二份进退潮汐水,都可以给你。”沈从声开出价码。

%62
方源不禁轻叹一声:“沈家不愧是东海的超级势力,资产雄厚,叫人刮目相看。”

%63
沈从声微笑:“仙友谬赞。”

%64
但下一刻,方源却道:“但这还不够。”

%65
“还不够?”沈从声瞳孔缩起,神情不佳。

%66
方源冷笑,又开始催动破晓剑,一剑一剑戳着沈伤。

%67
沈伤仍旧疯魔,不断吼叫。

%68
沈从声看着自家先祖被虐待,不禁咬牙切齿:“先祖发疯,并非是完好的八转蛊仙,这点你我都清楚!”

%69
“但正如你之前的期许,他应该还有不发疯的时候,不是么?”方源仍旧冷笑。

%70
“方源!”沈从声低啸。

%71
“怎么?”方源一扬手,大把的破晓飞剑攒射沈伤。

%72
沈从声差点要咬碎牙齿:“好吧,你还有什么要求,仙蛊么?”

%73
“我当然不会那么贪得无厌。”方源摇头,从容地道,“我要仙材,当然不是你给我的这三样。它们的品阶都太高了,我需要大量的七转仙材、六转仙材,种类越多越好,规模越大越好。”

%74
“沈家贵为东海超级势力之一,坐拥七大海域,中小海域成百上千,今后恐怕会有两位八转坐镇。这点小小的要求,应该不会拒绝吧?”

%75
“当然……不会。”沈从声面沉如水,“那么我们沈家又能得到什么?一个疯了的八转?并且他还身怀隐患。而这个隐患,能够让一向仁慈的乐土仙尊都想要剿灭!”

%76
沈从声绝对是一个合格的领袖。

%77
涉及家族利益,他理智得近乎冷漠。

%78
不得不说,沈伤的疯魔,让他的价值暴降到谷底。而他身怀的恐怖隐患,更让这份价值几乎沦为负数。

%79
沈家是需要八转,越多越好。但不是没有理智的八转!

%80
没有理智,敌我不分,这种八转只是负担,是别人的笑料,甚至还会因为胡乱杀戮,而令沈家遭殃。

%81
而沈伤身上的隐患,沈从声看在眼里,忧在心中。他也没有丝毫的把握,能够解决掉这个隐患。

%82
所以,当他拿出这三份价值连城的资源,来换取沈伤,展现而出的诚意是显而易见的。

%83
方源哈哈大笑,手指着沈伤:“他可是你的先祖,你费尽周折,想要救出来的人物。”

%84
说到这里,他顿了一顿:“当然,你所能得到的绝不只是这些。你还将获得我的……友谊。”

%85
“哦?”沈从声顿时露出感兴趣的神色,“怎么说?”

%86
“当今天下,地脉频动,五域一统,必是一个乱世。沈家虽在东海腹地,但也绝不是世外桃源。东海资源第一,财富动人心,即便是五域界壁存在的时候,都有源源不断的蛊仙前来这里,更何况将来呢?”

%87
方源侃侃而谈:“在这种情形下,正道、魔道的阵营之分,有什么用呢?沈家和我联手,必是强强合作。当然,这种合作的关系我们尽可以隐藏下去。贵族财力雄厚,而我则更加自在自由。”

%88
“中洲实力远超四域,将来攻袭东海,沈家是想投降,还是保持自立?就算是投降吧,东海的超级势力都会投降吗?所以,敌人从不会缺少,而如我这般的盟友却是稀罕至极。”

%89
“你我在此相遇,实是一场机缘。相争,对彼此都有害。相助,则是双赢。近在眼前的,就是乐土真传、沈伤,远的则是中洲东海,天下大势!”

%90
ps:今天中了糊涂蛊,码完后,居然忘记上传了。印象中总觉得是已经定时更新了的。

\end{this_body}

