\newsection{太天真}    %第六百五十一节:太天真

\begin{this_body}

%1
余音散尽,全场的鲛人们这才开始小声地交流起来。

%2
“这是什么曲子,我还是第一次听到,实在是太美妙了。”

%3
“你们发现没有,歌曲上佳,但谢晗沫却反而有失水准,操纵蛊虫配合起来,频频失误。”许多鲛人谈论这一点,脸上神情都很古怪。

%4
真正的原因,就是方源和谢晗沫,根本就没有好好的演练过。若是演练过一两遍,绝不会有这样多的微小失误。

%5
不过即便如此,能够在第一场合作中,就能够做到这种程度,谢晗沫本身的造诣已经足够惊艳。

%6
“如今,海浪抚平,天气转好,阴云消散,月光出现。不管如何谢晗沫失误多少,这效果摆在眼前。”

%7
“这种结果应当是打平了,甚至谢晗沫还略微占优。”

%8
“就看接下来的了。”

%9
鲛人们对接下来的海神祭,更加期待。

%10
大族老也看出了许多,此刻吐出一口浊气,心中的巨石缓缓放下。

%11
“这是怎么一回事?!”蛊屋中,寒潮族长咆哮出声。

%12
“看来你的计谋虽然成功了,但是却发生了意外。”步素莲眯起双眼,目光集中在方源的身上,她兴叹一声,道,“这个男人不简单,难怪能入谢晗沫的法眼。我熟知曲目,这个曲子恐怕是他的原创,由此可见,此人在音道造诣上非常出色。”

%13
寒潮族长立即表示怀疑:“这世间的歌曲千千万万,难以计数,你怎么确信这就是他的原创?”

%14
步素莲微微一笑,看了寒潮族长一眼,没有遮掩眼中的轻视:“你不懂。”

%15
寒潮族长脸色顿时更加阴沉:“步素莲,你会好好说话么?!”

%16
步素莲冷笑一声,没有再搭理寒潮族长,反而望着方源的眼眸中,熠熠生辉。

%17
别人会惧怕寒潮族长的势力,但是步素莲不会。

%18
这不仅是因为她是前任族老的遗孀,更因为她本身非凡的手腕和才情。

%19
但有一点,步素莲猜错了,方源根本没有什么创作,这本来就是前世地球上的曲子。

%20
“这曲风另辟蹊径,闻所未闻,定是你的原创。没想到方源你在音道上也有深造。”谢晗沫走下台时,对方源传音,语气中充满了赞赏和惊叹。

%21
方源苦笑:“过奖了,你也都看到了,我连琴蛊都是借来的。我可没有那么深的音道造诣。”

%22
“你不必自谦了。能够创作出这样的曲目,音道造诣已经脱俗,或许你转修音道会很有前途。”谢晗沫看向方源,眼眸发亮,神情恳切真挚。

%23
关于这点,方源早已预料。

%24
他不想解释,因为这不是重点,也解释不清。

%25
“现在的重点是接下来的两首歌。如果我所料不差的话……”方源欲言又止。

%26
他们两走下台,冬蕾便紧接着上台。

%27
方源和谢晗沫的表现虽然令她意外,但此刻她一点都不慌张,仍旧有着镇静的风范。

%28
她开始歌唱,动听的歌声陆续引来大大小小的鱼群。

%29
“果然。”方源冷笑。

%30
谢晗沫目光也变得越加冷冽。

%31
蓝鳞、赤鳞两位侍卫面面相觑,气得满脸发红:“这贱人竟然又抢唱我们准备的歌!”

%32
“不要紧,我还有曲子。”方源呵呵一笑,自信十足。

%33
冬蕾下台,又轮到他们俩上场。

%34
方源伴奏,谢晗沫轻歌曼舞。

%35
……

%36
明月几时有,把酒问青天。

%37
不知天上宫阙,今夕是何年。

%38
……

%39
词曲一出,顿时气氛改易,在场的鲛人们都沉醉地闭上了双眼。

%40
……

%41
我欲乘风归去,唯恐琼楼玉宇,高处不胜寒。

%42
起舞弄清影,何似在人间。

%43
转朱阁,低绮户,照无眠。

%44
……

%45
谢晗沫回想起担当圣女的时候,位高权重,却是孤家寡人,一时间心中感慨万千。

%46
我是想乘风归去,但这圣女的琼楼玉宇却将我束缚在内。寒意逼人,辗转难眠,何人能与我共舞?

%47
……

%48
不应有恨,何事长向别时圆。

%49
人有悲欢离合,月有阴晴圆缺。

%50
此事古难全,但愿人长久,千里共婵娟。

%51
……

%52
一曲唱罢,天地无声。

%53
鱼虾龟鳖大片大片地漂浮在海面上,俯拾即是。

%54
海鸟也飞舞盘旋,很多都是白日里活动的飞鸟,竟在休眠中被歌声吸引过来。

%55
优美的词,缓缓的曲,深入人心,令鲛人们无法自拔。

%56
谢晗沫看着方源,心想:“这是否是他为我作的曲呢?”

%57
她从这词曲中得到共鸣,得到劝慰,得到温暖。她如同明月般冰清玉洁,却受到外人的诬蔑,但如今心中的寒意和烦躁已经尽数消散,之前种种的流言蜚语再不能在心底留下痕迹。

%58
“他是知我的。”一瞬间,谢晗沫心中升腾起了一股玄妙的不可言喻的感动。

%59
结果出来,两相比较,又是方源、谢晗沫稍稍占优。

%60
冬蕾在台下脸色惨白。她深深的明白,若非谢晗沫和方源之间配合的并不到位,有着一些误差,恐怕她都没有比试第三场的资格了。

%61
“这个人绝对是一个威胁!”寒潮族长咬牙切齿,砰的一声,他的拳头狠狠地捣在蛊屋的窗棂上。

%62
“你终于看出来了。”步素莲淡淡地道,语气中藏着一丝冷讽。

%63
寒潮族长冷哼一声,没有心情和步素莲计较。

%64
他必须赶紧处理危局,因为按照眼前的局势再发展下去,第三首歌后,必然就是谢晗沫获胜了。

%65
“方源是吗?没想到竟然是个大麻烦!”

%66
“必须要将此人处理掉!”

%67
寒潮族长暗自发狠,同时又非常头疼。若在平时,他自然有大量的手段可以针对方源。但现在海神祭,众目睽睽之下,他动手的余地太小太小了。

%68
“怎么办?”寒潮族长急速思考,不知不觉间额头已满是冷汗。

%69
思考良久之后,寒潮族长终于出手。

%70
“方源,我就是寒潮族长!”他直接传音方源,因为这是他想到的最可能的法子。

%71
方源神色一动,没有回话。

%72
寒潮族长呵呵一笑:“你静静地听着也好。你是个聪明人,我知道的。但你还太年轻,总是会抱有不切实际的天真想法。”

%73
方源冷笑一声,暗中回道:“我这不是天真,而是一种理想,你不会明白的。”

%74
“所以你冒傻气啊,小子。别看你们俩站在台上,风光无两,但其实本质上只是棋子罢了。你看看在风中飘扬的旗帜,它的根本是旗柱。你要好好想想,你们的根本依靠是什么?”

%75
“大族老吗?你去打听打听,她是什么样的人。她有势力,明明可以出力压制住我们这一方,但是她却选择让你们来打先锋。谢晗沫彻查贪腐的时候,她出过什么力吗?她帮助你们了吗?她或许是提供了一些帮助,但请相信我,这只是她随手帮的小忙而已。”

%76
寒潮族长口才相当了得,他继续道:“好吧,就算退一万步,你们赢了,保住了圣女之位,又能怎样?你真的认为我会死?不,寒潮一族乃是当即圣庭中最大的部族。让我死,就是要令整个圣庭动荡,乃至崩解。大族老她绝没有这样的坚定意志,她只是想敲打我,让我不要那么过分。”

%77
“所以最后,就算你们查探清楚了,最终我仍旧会活着,继续当我的寒潮族长,顶多是拿出一些替罪羊来,做做样子,稍微收敛一下罢了。”

%78
“海神祭中,几乎每一位竞争圣女的鲛女背后,都有一方势力支持着她们。你以为这只是简简单单的圣女选拔吗?不,这是一个游戏,让我们这些高层以不伤元气的方式,来角逐出今后数十年的资源分配。拥有圣女的势力拿得多,没有圣女的势力就拿得少。”

%79
“方源,或许你会痛恨,厌恶我们这些暗幕下的势力。但你要明白,在这片海洋中,黑暗才是主宰。所谓的光明,有的只是海面上浅浅的一层,凡俗浅薄的人着迷于它的光鲜亮丽,以为光明就是海洋的真相,这就……太天真了。”

%80
方源沉默。

\end{this_body}

