\newsection{方源至}    %第五百一十六节:方源至

\begin{this_body}

%1
那些倒霉的蛊师,不幸被水柱砰中的,直接被冲爆成一团血肉骨渣。

%2
如此惊变让蛊师们惊骇欲绝,纷纷住手,脱离战场避祸。

%3
“怎么回事?”云层上的三位蛊仙,正要做出应对,就见到这些喷射出来的水柱,忽然在半空中合而为一,形成一个鱼身蛙腿虎头的上古水怪!

%4
水怪咆哮一声,扑向地下的凡人。

%5
它没有多少智慧,觉得是这些凡人蛊师惊扰了它的沉眠。

%6
巨大的蛙腿一踏地,就有无边的洪水从地底漫出,汹涌而来。

%7
但三仙都在同时出手,三片光罩,分别护住三家族人。

%8
“原来这烟罗暖玉田下,还有一片地底寒心湖,湖水中蕴养出了一头上古水怪。”侯耀面色凝重。

%9
商青青咬牙,看向铁面神:“还请铁面神大人主攻,我等二人辅助,将这头上古水怪拿下,防止它胡乱疯狂,破坏了这片烟罗暖玉田。”

%10
她和侯耀都只是六转蛊仙,对付上古水怪,有心无力,场中唯有铁面神一位七转蛊仙,自然要以他为主。

%11
铁面神点头,正要冲下云层,这时忽然一声尖锐的鹰啼,闯入三仙耳畔。

%12
下一刻,在场的所有人都震惊地看到,叶凡身边的那头小鹰,忽然膨胀起来,化为一头滔天巨鹰!

%13
上极天鹰!

%14
巨鹰尖啸一声,扑向上古水怪。

%15
上极天鹰得到神秘蛊仙陆畏因的搭救,也同时被他的手段禁锢,一直栖息在叶凡身边。

%16
而当叶凡遭遇到生死危机的时候,陆畏因的禁锢手段就会大大削弱,让上极天鹰能够爆发出一部分战斗威能。

%17
上极天鹰终究没有人那样的灵性,也未查探到周围潜伏着的蛊仙,它只感受到上古水怪的威胁,感到尊严受到了挑衅,便直接出手要教训这等狂妄之徒。

%18
感受到上极天鹰的汹汹气势,上古水怪立即怂了,咆哮声戛然而止,直接化作水流,顺着刚刚冲出来的洞口,迅速溜走。

%19
上古水怪一去,叶凡没有了生命危险,上极天鹰身上封印又起作用,让它又变回原状,一股力量更让它不得不飞回到叶凡的肩头去。

%20
叶凡身边,鸦雀无声,无数蛊仙都瞪圆了双眼,看着他。

%21
“叶,叶公子,你这头鹰儿,究竟是什么……”小蝶结结巴巴,完全被惊到了。

%22
叶凡摸摸鼻子:“若我说,我也是第一次见识到这头鹰的厉害,你们相信吗?”

%23
“太古荒兽!!!”不管是云层上的三位蛊仙,还是湖水中隐藏着的翼家两仙,一时间也都震惊得说不出话来。

%24
“怎么会搞成这样子?”尤其是那两位翼家蛊仙,他们只想放出上古水怪,没想到竟然惹出了一头太古荒兽。

%25
“这是好事啊,我们不用自己出手了。靠这头太古荒兽,就能除掉这三仙!”翼渔大喜。

%26
翼南门目光阴沉:“这事情不是那么简单的,你没看清楚吗?这头太古荒兽被人豢养,是刚刚那位争斗的少年的宠物!”

%27
“商家仙子,这头太古荒兽是怎么回事?”铁面神目光如电,猛地掉头,逼视商青青。

%28
商青青苦笑,之前她心神激荡,脸上神情已经露出破绽,就算没有露出破绽,在铁面神面前,也向来难以撒谎。

%29
商青青只有六转修为,此刻只好道:“此事我也不知,恐怕是此少年的机缘吧。”

%30
“什么样的机缘,能够让一届凡人豢养一头太古荒兽,当做宠物?”侯耀半是震惊半是语气叵测地道。

%31
铁面神冷笑一声:“此事重大,不可轻忽。相比起来,烟罗暖玉田已经不重要了!这少年恐怕是继承了某个仙道真传,但我南疆正道从未有过如此记载,能够令区区凡人豢养住一头太古荒兽。我要究查清楚,防止魔道再起!”

%32
商青青心中愤怒又无奈,苦笑一声道:“我相信铁面神大人能够将此事调查清楚,主持公道。”

%33
“鹰儿,鹰儿,原来你是这么厉害的啊!”叶凡惊叹连连,情不自禁地伸出手来,想要抚摸上极天鹰的脑袋。

%34
上极天鹰早就试着挣扎许多次,但都无可奈何身上的封印。次数多了,也有些任命。此刻回到叶凡肩头,自然气恼愤怒,但听到叶凡称赞,眉宇间便又浮现出一层高傲之色。见到叶凡的手伸过来,它立即伸出翅膀,一下子把叶凡的手扇开来。

%35
叶凡不由苦笑。

%36
小蝶瞪大双眼:“这头鹰真是傲气,不过它如此滔天的本事,也配得上它的傲气。叶公子,你是怎么……”

%37
话未说完,就被商心慈打断:“小蝶,这是叶凡公子的私密,不是你随便打探的。”

%38
“无妨,无妨。这头神鹰是我师傅捉的。”叶凡见事情已经败露,不愿意隐瞒商心慈。

%39
“你师父居然能收服这样的神鹰?!”小蝶声调一扬。

%40
商心慈也面现惊容,很自然地联想到了,叶凡的师父极可能便是一位强大的蛊仙!

%41
正交谈着,忽然上极天鹰面现惊恐之色,在叶凡的肩头躁动不安,好像老鼠见到了猫,羊碰见了狼一般。

%42
“怎么回事?”小蝶等人俱都瞪大双眼,这头高傲的神鹰忽然这样表现,实在是让人大跌眼镜。

%43
“我也不知道!”叶凡也惊呼起来,“这还是我第一次见到鹰儿,呃,如此的惊惶失措。”

%44
上极天鹰在叶凡的肩头站立不足,向着天空的某个方向尖啸。但这声声尖啸,再也不像之前那边威武逼人,反而透露出浓郁的惊惶,甚至是哀求,让人一眼就看出其中的恐惧,当然若仔细分辨,还有一丝的愤怒。

%45
“怎么回事?”一时间,不管是凡人蛊师,还是蛊仙们,都不由自主地顺着上极天鹰尖啸的方向望去。

%46
随后,众人便见天空中一道身影,仿佛飞剑贯穿天地而来。

%47
此时,天空中白云漫漫,被这道身影切分成两半。

%48
“那好像是一个人?”

%49
“仙人?!”

%50
蛊师们惊呼。

%51
“来者何人?”

%52
“这股气度,绝非寻常仙人啊……”

%53
蛊仙们也各自惊异,纷纷警惕起来。

%54
身影戛然而止,悬停在半空中,众人顿时见到这位蛊仙的真面目。

%55
只见他面如冠玉,鼻梁高挺,深邃的黑眸透射出一缕缕的漠然冷芒。一头长发,漆黑如瀑,笔直垂下,垂至腰间。身着白袍,大袖飘飞,英俊得近乎姣丽,气度慑人至极。

%56
一群凡人蛊师尽皆仰望。

%57
场中鸦雀无声。

%58
然后又听此人开口,语气淡淡:“我的鸟儿,原来是在这里。”

%59
ps:待会有个单章,希望大家都能看一看。

\end{this_body}

\newsectionindepend{我卖书不卖笑!}

\begin{this_body} %begin a body

%60
有一年发生饥荒,百姓没有粮食吃,只有挖草根,食观音土,许多百姓因此活活饿死。消息被迅速报到了皇宫中,皇帝高高坐着,听到奏报,大为不解。“善良”的皇帝便苦思冥想,想出了一个解决之道:“百姓肚子饿没米饭吃,为什么不去吃肉粥呢?”

%61
在我们的生活中,永远有那么一部分人,表面上看上去很是善良,很是能理解别人,其实不然。

%62
我最近这段时间,每天都带着媳妇跑去南京陪她看病治疗,虽然是小毛病,但治疗过程比较繁琐。早上坐火车去,下午坐火车回来。有时候火车还没有座位,可能要站着,一个多小时。或者是做汽车赶回家,时间更久。

%63
真的是很累,你们想象一下自己的长途旅行,想想自己坐长途汽车或者火车,再想想将这个过程,扩散到一个月,每天这样做。

%64
写作状态因此不佳。

%65
这个事情我早就打过招呼,但很多人仍旧不理解。或者说他们觉得自己理解了。

%66
几天前,我因为失误,断更了一天。事情的经过我已经解释了,也致歉了,的确是自己失误,其实已经存稿放在起点后台,只是定时更新没有设定好。正巧哪天晚上有急事,我很晚回家,发现有人就开始骂爹骂娘起来。

%67
当时一气之下,就索性没有更新。

%68
同时,我也很有疑惑,是什么原因导致这种现象的发生?我该如何应对这件事情?

%69
第二天我便宣布停更一百年,所有的更新都是意外。

%70
然后我开始一天一更,一天两更,一天三更。

%71
我发现,那些骂爹骂娘的人销声匿迹了,不再闹腾了。甚至书评区里还有些热闹和喧哗。表面上,这是更新的问题吧,但实质上呢?

%72
我宣布停更一百年,却连续更新这么多天,这是不是意外?有时候还多更,这个意外是不是比那天断更更大一些?

%73
为什么这些人不因为这些意外,而闹腾呢?而骂爹骂娘呢?

%74
有人会立即这么回答:“这是因为你这个作者更新了呀。”

%75
是啊,是更新了。但本质上呢?

%76
同样是意外,怎么这些人不能一概而论呢?

%77
这是因为,我做出来的事情,符合他们的利益而已!

%78
所以这部分的人,看起来很善良,很理解人,很支持我,其实不过是支持他们的利益罢了。这些“支持”,并不是真正的理解你作者的苦楚,只是像那个皇帝一样“善良”。

%79
这些天里,也有一部分的读者抱怨,说:蛊真人你这样做太过分,为了少部分人,殃及了所有人看书。我看你很不爽。

%80
我在这里回答:我怎么殃及所有人看书了?我说过停更一百年,我哪天停更了?

%81
这部分人也是为了自己的利益在抱怨,并不是真正地去试图理解我。

%82
当然。

%83
我知道这是人之常情,很正常的。我写书,你看书,为什么我要强行地要求你们去理解我呢?你一个作者,写书的而已,是我什么人?朋友?家人?呵呵。

%84
这是正常的情况,我当然是接受的。实际上,我也从不强行要求别人来理解我。

%85
但我不能接受的一点是,骂爹骂娘,脱口成脏!!!

%86
我写书卖书,凭什么让我爹娘挨骂?

%87
你在商场里卖东西,你就要让你爹娘挨骂吗?

%88
我写书,付出我大量的心血、时间,我付出劳动,我获得成果堂堂正正。我卖书,千字几分钱,一章看下来几毛钱,甚至还可能不如一根烟的钱。

%89
你开烟酒店卖烟,有人买烟,你告诉别人,今天断货明天可以来买,或者可以去别家买。这个时候,顾客大骂,恣意侮辱你和你的家人,你什么感受?

%90
你肯定不爽,我也不爽!

%91
我告诉,没有人会爽,所有人都会不爽!

%92
我不爽,我就要抒发出来。

%93
我写书为的是什么?是完成梦想,同时赚钱养家。

%94
是生活!

%95
不是生存。

%96
这个世界上,有的人生存,有的人生活。

%97
我是后者。

%98
我追求的是高质量的生活,舒心的生活。不是让你们辱骂的。

%99
我不是卖笑的,我只是一个写书卖书的。我不是什么偶像,你骂我得受着,担心掉粉。

%100
对于我而言,掉粉就掉粉呗,什么订阅、收藏,掉就掉呗。

%101
有什么关系?

%102
几年前,我写《蛊真人》书时,我就没想过赚钱。我要是想赚钱,不会这种更新状态,也压根不会写这种题材。

%103
这是因为我和别人不一样,我写书的最主要的目的,就是写书本身,赚钱是附带的,次要的。

%104
《蛊真人》之前,我写过许多本书,我曾经多次骄傲地宣布:每一本都完本,从未太监过!

%105
这是不争的事实。

%106
可能很多人还不了解,这个事实背后的真正含义。

%107
那就是——我不管写什么书,不管这书成绩如何,卖不卖座,我都将我心中的概念都写完整了。

%108
我要是写书,都奔着赚钱去,我会这么干么?

%109
《蛊真人》陆陆续续更新了好几年,直至最近才显露出一些成绩,让我有专职做这件事情的希望和可能。

%110
我本人也很感兴趣,因为这很可能是一直高质量的生活方式。

%111
但就算是做专职,难道我会改变初衷,专门奔着钱去写吗?

%112
我追求的是有质量的生活,钱是次要的。

%113
我对钱财的观念是,贫困杀人,富贵也能杀人。鲍鱼海参我能吃,稀饭白粥我也能吃的下去。

%114
我童年的时候,过的是苦日子,也不怕将来过苦日子。

%115
现在就算有了儿子,我也没有大多数家长的想法,要想方设法地给他过富日子。他若有才,清贫能养其志向,他自己就能赚钱,何必需要我的钱?他若无才,给他再多的钱,也会给他败光,他身上的缺点,会因为过多的钱财而放大。

%116
富贵是好事,但清贫也有利处。

%117
退一万步讲,将来真的写书不行了,实在养活不了自己和家人,我大不了出去找份工作呗?我有手有脚,正当壮年,难道不能养活自己和老婆孩子么?

%118
所以,我不会受这种窝囊气。

%119
我写书卖书,不是为了受气来的。

%120
对于这部分骂爹骂娘的读者,我在这里告诉你:我蛊真人的书,不卖给你们!

%121
送你一个字:滚!

%122
两个字:快滚!

%123
三个字:速度滚!

%124
四个字:麻溜地滚!

%125
五个字:灰溜溜地滚!

%126
(写完这一段,我认真地数了数,发现我的数学还是很好的。)

%127
或许有人会问,蛊真人你是不是太装比了?这样不好啊。你看看那些大神,他们都没有你这么装比啊。你何德何能装这个比呢?

%128
那么我告诉你:我平时不装比,也不擅长装比,但我今天一定要装这个比!

%129
一是因为别人在向我装比,我蛊真人向来投桃报李。我是卖书的,不是卖笑的。我的笑容是给朋友的!朋友来了,我给笑容给热情,敌人来了我举起刀枪和剑!

%130
二是我和许多作者都不同,因为我的追求,我的性情不太一样。我若心情不畅,写书怎么能畅快呢?我也就写不出好书来了。

%131
三是我今天装这个比,没有什么何德何能,就是穷横!

%132
另外我还想问你:那些骂爹骂娘的读者,又有何德何能,这样做呢?

%133
就因为他们买我的书,“支持”我,就可以辱骂我,辱骂我的家人吗?

%134
诸位,你们什么见过这样的一群人,一边说:我支持你呀,一边又骂你爹你妈的呢?

%135
这群人还有通病,就是觉得:我如此这般这样的“支持”你,我奉献很大啊,我吃亏很多呀,我骂你几句怎么了?我不也是为你好么!

%136
所以,这种“支持”在我眼里,它压根就不是支持!这种“支持”我不要也罢!

%137
我再重申一遍:我蛊真人写书,以前的书,现在的书,今后的书,都不卖给你们这种人。你们有种别看!

%138
其实我写的书也就这样,你们何必来看我的书呢?其他有好多优秀的书,都去看其他人的吧。

%139
我的这篇单章,也算是一张自白书,我相信对于我整个写作生涯,都有重要的意义。

%140
可能你们这部分人看到今天的单章,会很不爽很不爽。

%141
对不起,这不是我的主要目的,让你们不爽只是附带的。

%142
最后点个题吧——

%143
我蛊真人卖书不卖笑!

\end{this_body}

