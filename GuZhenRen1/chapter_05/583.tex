\newsection{春剪}    %第五百八十四节:春剪

\begin{this_body}

南疆群仙士气大振,影宗等人则是眼皮子跳动,面色微沉下来。

陆畏因不愧是乐土传人,举重若轻,一击之下,就将一头凶猛的太古年虎轻松擒拿。他虽然没有攻伐手段,但就这一手段,就深不可测,令人敬畏。

嘶嘶嘶……

一头太古年蛇,拖动漫长粗壮的蛇躯,吐着猩红的蛇信,钻出漩涡巨门,进入战场。

第三头太古年兽!

这一次,不管是陆畏因,还是夏槎的脸色,都变了。

因为从这头太古年蛇的身上,他们感知到了野生仙蛊的气息。这头太古年蛇,远比其他两头太古年兽更加危险。

然而更关键的是,照这种趋势下去,出现第四头、第五条太古年兽,也不会令人奇怪。

“这到底是什么大阵!”乔家蛊仙大呼,心中压力甚巨。

其余七转蛊仙心头均是沉重无比,再无之前的意气风发。

“我来。”关键时刻,陆畏因再次挺身而出,对上太古年蛇。

只是这一次,他谨慎了许多,再没有探索出太古年蛇身上究竟是何种仙蛊前,他不会冒险行事。

“杀啊!”南疆蛊仙们咆哮,杀意腾腾。

年兽们嘶吼,张牙舞爪。

南疆蛊仙们掀起腥风血雨,年兽死伤惨重,但漩涡巨门无法破坏,生生不息,一直在不断地向内输送年兽。

年兽延绵不绝,随着战斗不断持续,南疆蛊仙们的状态逐渐下降,开始有人负伤。

“年兽太多了。此阵必定经过大幅改良,吸引来的年兽远超凤九歌时对战的规模。”刘浩心中暗自震惊,他手中有着一些推算的成果,来源于天庭,但不知道该怎么递交出去。毕竟他的身份,并非阵道、宙道蛊仙。

“这该死的大阵,必须将它破坏,我们才有胜机!”

“其实只要破坏一定程度,让我们能够将此处情报传递出去,必定有大批正道仙友前来支援。”

南疆群仙亦知道如何应对,可惜手段不足,这座年流伏诛阵乃是方源宙道境界暴涨后,精心推算而出,岂会如此容易就让南疆群仙得逞?

“这宙道大阵,真是玄妙……”夏槎紧皱眉头。她自从破解了一处阵眼之后,短时间内居然找不到第二处阵眼。

“大阵浑然一体,毫无破绽可言。我在阵道方面建树太少了。或许我应该全力出手一次,使其大阵承受不住,造出破绽来!”夏槎眼中杀机萌动。

她是夏家的太上大长老,宙道大能,心中只有强者傲气,此刻被方源算计,拘束在此,心中渐渐不耐。

陆畏因却是看出夏槎心中所想,连忙劝道:“夏槎大人,切勿中了魔头奸计。方源狠辣狡诈,天庭都奈何不得。影宗掌握过惊鸿乱斗台,这座仙蛊屋虽然毁了,但残余了不少仙蛊,保留在他们手中。你若要全力出手,恐怕会正中方源下怀,为方源所用。”

惊鸿乱斗台最招牌的手段,就是能将对手的杀招封存起来,再催发出去。

陆畏因的劝诫,非常明智,令方源都忍不住微微扬眉。他虽然不是智道蛊仙,但推算的非常正确,这座年流伏诛阵几乎耗尽了方源所有的宙道仙蛊,黑凡真传中的仙蛊充当核心,惊鸿乱斗太的仙蛊也在全力辅佐。

夏槎听进劝告,深呼吸一口气,她冷哼一声,忽然看向第一头掺和战场的太古年鸡。

太古年鸡正被一群南疆蛊仙围攻。这些蛊仙均是七转强者,杀招犀利,但打在太古年鸡身上,却是威能不显,如隔靴搔痒。

太古年鸡媲美八转战力,皮糙肉厚,南疆群仙们奋力拼搏,也奈何不得。

但当夏槎的目光,投注到太古年鸡的身上时,这头年鸡顿时感受到强烈的威胁,昂起头来,撞破南疆群仙的包围圈,直接就向夏槎扑来。

“好孽畜。”夏槎怒极反笑,身上大量蛊虫气息澎湃喷涌,顷刻间酿成一记仙道杀招。

只见一柄剪刀,翠绿作色,体大如象,飞向太古年鸡的脖颈上,狠狠一剪。

太古年鸡却未感觉到任何痛楚,承受着一击,仿佛来到春天,暖风和煦,阳光明媚,青草花香,漂游扑鼻。

迷惘了一下,太古年鸡的脖颈上形成一道深切的伤口,大量的血液宛若喷泉喷射而出。

太古年鸡浑身一震,凶性激发,咯咯狂叫,速度暴涨,仍旧扑向夏槎。

夏槎不闪不避,被太古年鸡扑中,却化为泡影消失。

随后,那柄翠绿剪刀,又再次飞绕到太古年鸡的头顶上空,照准它的脖颈见机就剪。

夏槎消失不见,太古年鸡的注意力全被翠绿剪刀吸引,一时间两相争斗,奋不顾身。

“这便是夏槎的春剪,果然威能脱俗,犀利非凡。”方源隐于幕后,对战场洞若观火,见到这样的战况,心中一动,脑海中相关的情报就浮现而出。

夏槎有一套杀招,共有四招,春剪只是其中之一。毫无疑问,这是八转级别中的攻伐杀招,锋锐犀利,就连太古年鸡这样皮糙肉厚的存在,都抵挡不住剪刀锋芒。

方源暗暗羡慕。

这是他现今最为缺乏的东西。

幽魂真传中的确是有八转宙道杀招,但方源缺少相应的八转仙蛊。就算是经过改良,用了六七转的仙蛊替代,也达不到八转层次了。

方源手中唯一一只八转宙道仙蛊似水流年,在黑凡真传中,倒是有几个杀招,以它为核心。诸如年兽召来,流年不利等。

八转层次的年兽召来,可以召唤出太古年兽,为方源作战。

而流年不利以似水流年蛊为核心,则是以宙道模拟出运道效果。但两者都不是春剪这般,直接用来攻伐,效果立竿见影的杀招。

所以,方源只好将主意打到光阴飞刃上来。

咯咯咯!

太古年鸡并无野生仙蛊,终究是难敌夏槎的春剪,被剪得遍体鳞伤,发出惨叫。

夏槎前后只用这一招,就将一头太古年兽打成重伤,展现出她非凡的战力。

“不愧是当今夏家之主,权利第一人!这番实力,恐怕可以达得上天庭成员的标准了。”方源心中感叹。

眼看着这头太古年鸡就要惨死在夏槎手中,忽然大阵一动,太古年鸡骤然消失在原地。

“给我臣服罢。”幕后,方源对这头被削弱到极致的太古年鸡下手,凶猛霸道的奴道手段下去,太古年鸡不得不低下高昂的头颅,向方源表示臣服。

于是,方源将太古年鸡顺利收回麾下。

他一边借助年兽,伏击南疆追兵,一边又借助南疆追兵,来帮助他削弱年兽,然后奴役。

这手法早在西漠时,就在天庭追兵身上用过,算得上老套。

但老套不要紧,只要有效就可以了。

眼看着自己的战果就要到手,一头死亡的太古年鸡,几乎全身都是八转仙材,价值连城,没想到居然就消失了。

许多南疆蛊仙气得破口大骂,夏槎的面色也很阴沉,不过她旋即一笑:“你总算是露出了马脚,再多给你点太古年兽,又有何妨?”

原来,方源的大阵只是传输年兽,也还罢了。如今一将太古年鸡送走,大阵势必要进行不一样的运转,这就给夏槎提供了方便,让她察觉到了更多微妙之处。

果然不久之后,她就勘测出第三处阵眼,将其摧毁。

但南疆群仙的欢呼声,并没有持续多久,这阵眼一毁,又转变成了漩涡巨门,又多了一道年兽输入的渠道。

“怎么会这样?”南疆群仙傻眼。

------------

\end{this_body}

