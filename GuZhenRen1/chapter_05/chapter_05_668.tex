\newsection{怪兽肆虐}    %第六百七十节:怪兽肆虐

\begin{this_body}



%1
“一般而言,体型如此庞大的怪兽,应当是性情蛮横,势大力沉,绝无可能有着如此敏捷的速度的!”

%2
巨鹰勇士从深坑中站起身来,满头都是冷汗,忽然间灵感一闪,想到一种可能:“等一等,难道说刚刚那一次出手,是这头怪兽的战技?”

%3
他仰望方源,方源已经化身成牛头人身的巨怪,庞大的阴影笼罩着巨鹰勇士。

%4
巨鹰勇士的心中,再战的勇气却在徐徐恢复。

%5
“你有战技,我也有战技啊!你的战技我已经知道了,但是我的战技你这头蠢笨的大块头可不清楚啊。”

%6
“没有错,只要我把握住战机,我必然有胜利的可能!”

%7
想到这里,巨鹰勇士的眼中绽射出希望的光。

%8
“他站起来了,他站起来了!”

%9
“没有错,巨鹰勇士怎么可能会被一下子就打倒呢?”

%10
“巨鹰勇士上吧,我们相信你!”

%11
见到巨鹰勇士站起来,前一秒还死寂的城池,陡然爆发出冲天的欢呼声。

%12
一声鹰呖乍响。

%13
巨鹰勇士猛地一振背后双翼,再次拔升而起,飞上了高空。

%14
“你有战技,我也有战技!怪兽,见识一下,我的战技银白幻影!”巨鹰勇士大喊出声。

%15
下一刻,他身形陡然变幻,分裂出六头幻影,栩栩如生,放眼望去和真身毫无差别。一时间,真真假假的巨鹰勇士们盘旋在方源的头顶上空,将方源重重包围。

%16
城池里的声浪顿时又掀到另一种高度!

%17
“看呐,是巨鹰勇士的银白幻影战技啊!”

%18
“他终于施展出来了,你们看!那头大怪兽也似乎看懵了,不知道向哪一个下手了啊。”

%19
“胜利的机会就在眼前了,上啊,把那头大怪兽杀死吧。”

%20
“啊啊啊……巨鹰勇士真的是太帅了!”

%21
巨鹰勇士将人们的欢呼声、加油声都听在心底,这给他加倍的勇气和斗志。

%22
但他心中仍旧冷静:“我必须珍惜这次机会,定要一击必中,即便杀死不了这头怪兽,也要重创它。毕竟战技可是会消耗极大的体能的,根本不能连续多次催用!”

%23
“所以,即便是这头怪兽实力强劲,战技也不能频繁使用。我先用幻影来勾动它不断地手,耗尽它的体力,令它大败亏输,甚至说不定,体力所剩无几,我还能活捉它呢!”

%24
巨鹰勇士居然要活捉方源,方源虽然还侦查不到他的这个心理,但是巨鹰勇士的战术却一目了然。

%25
一位位巨鹰勇士围绕着方源,不断飞舞,时而有一头两头忽然俯冲下来,伸出尖锐的爪和喙,攻击方源。

%26
方源不闪不避,站立在原地,这些巨鹰勇士都是幻影,攻击没有什么实质,落到方源眼中,就都露馅了。

%27
方源好整以暇,心中暗笑:“这个仙窍洞天有点意思啊。”

%28
首先,凡人和蛊仙的蛊仙就有意思。

%29
很和谐。

%30
和五域外界完全不同,五域中凡人如蚁,蛊仙高高在上。而这里的蛊仙却是凡人的保护神是还十分热衷于这种保护的责任。

%31
从这一点上,方源仿佛看到了乐土的一些味道。

%32
不过这并不奇怪。

%33
每一个仙窍洞天,都自成世界。若是和五域外界的交流少了,就会形成不一样的社会结构和认知,独具风格。

%34
其次,这里蛊师、蛊仙的战斗和修行也有意思。

%35
比如眼前的方源的对手,就是和上古荒兽银白巨鹰合体,从而形成一种人兽形态的战斗形态。

%36
这显然是变化道的手段,优点非常明显。

%37
正常的蛊仙,需要消耗仙元,催动仙蛊或者仙道杀招来战斗。但是这里的蛊仙只需要和一头荒兽合体,就能拥有战技。

%38
所谓战技,就是一些荒兽、上古荒兽等拥有的天赋能力。就如同方源曾经拥有的上极天鹰,它就能洞穿仙窍自由出入。

%39
这种天赋能力,在这里被人们称之为战技,不需要消耗仙元,但是会极大地损耗体力,乃至付出其他一些代价,诸如身体的一部分或者寿命。

%40
不管是战技、仙蛊还是仙道杀招,追根溯源的话,本质是都是道。

%41
荒兽、上古荒兽等因为身上浓郁的道痕积累,以及特定的道痕分部,所以才会拥有天赋能力,也就是战技。

%42
仙蛊本身就是大道碎片,代表着天地法则中的某一点。

%43
仙道杀招更不用说了,往往采用少数仙蛊、大量凡蛊,就是道痕和大道碎片的相互组合,然后通过一定的奇妙步骤,达到某种更佳的威能效用。

%44
事实上,最早的仙道杀招就是蛊仙们参照了一些荒兽身上的道痕,这才渐渐模仿、摸索,从而发展出来的。

%45
“这里的蛊仙,和荒兽、上古荒兽合体之后,就能施展出荒兽的天赋能力。也就不需要辛辛苦苦炼制什么仙蛊,也省去了苦苦修炼仙道杀招的种种风险和代价。”

%46
这个优点可谓十分突出!

%47
因为仙蛊太难得了,仙蛊唯一,每一只仙蛊都是独一无二的。

%48
但是荒兽、上古荒兽却有很多,即便是上极天鹰这种太古荒兽也有成群成群的。

%49
拥有天赋能力的荒兽、上古荒兽,往往因为缺乏智慧和灵性,显得蠢笨。但是当它们和蛊仙合体之后,蛊仙拥有了它们的天赋能力,而它们也是拥有了蛊仙的智慧。两者结合起来,稳稳的共赢方式,巧妙地弥补了双方的不足之处!

%50
“但此法修行起来,却是不够灵活。打来打去,就只有一张牌。天赋能力太单调,很容易被克制,短板很多。”方源同时也看出了弊端。

%51
“不过总体而言,此法的优秀之处还是太大了,远远超过弊端。一旦推广出去,在五域必定广受欢迎!”

%52
“幸亏我前世没有此法,否则我的记忆都将面目前非。”

%53
因为此法简直是批量制造蛊仙战力的不二法门,虽然不能改变高层战力对决,但是却足够改变一定的局势。

%54
五域中还有很多的六转蛊仙,手中都没有一只仙蛊呢!

%55
而这些人恰恰又是每一域的蛊仙界中,数量最多的人群。

%56
试想一下,一域没有此法,一域掌握此法,一旦全面开战,单论中下层的蛊仙对拼,当然是拥有此法的地域胜算极大了。

%57
“合体其实并不是什么新鲜的手段,只是一直难以昌盛。最大的难点就在于道痕之间互斥。”

%58
“但是在这个洞天中,却有着手段可以将其他道痕转变成变化道痕!”

%59
“于是,蛊仙和荒兽之间就能随意地相互合体。说到底,这个合体的手段和兽灾手段是一致的。”

%60
推算到这里,方源更想将这个变化道的手段得到手了。

%61
“我还有正事要办,苍蝇就不要再嗡嗡嗡了。”心底嘀咕一句后,方源便催动宙道手段。

%62
他现在的宙道杀招,已经今非昔比了!

%63
巨鹰勇士还在试探,忽然震惊地发现,周围的时间便得缓慢无比。

%64
“这!?时间居然变得如此缓慢!怎么可能?!”巨鹰勇士彻底地惊骇住了。

%65
他其实对自己的速度比较有自信的,但是问题是时间被影响,他的速度特长根本无从发挥!

%66
“结束吧。”方源缓缓伸出手掌,十指张开,慢慢地探出,然后将巨鹰勇士的两只鹰翅稳稳抓住。

%67
“该死,该死!快动啊,快动啊!!”巨鹰勇士在心中咆哮,但是他拼尽全身力量去飞,也只是飞了极其微小的距离。

%68
噗嗤!

%69
下一刻,方源双臂猛地用力,一下子将巨鹰勇士的两只翅膀直接撕掉!

%70
“啊!”巨鹰勇士满脸扭曲,痛得仰天嘶吼。他背后的伤口处,鲜血宛若喷泉,疯狂地喷涌而出。

%71
“巨鹰勇士!!!”

%72
“不!”

%73
城池中无数人惊叫,恐惧和惊骇取代了之前的冲霄战意,有的人当场晕倒,有的人直接闭眼捂脸。

%74
方源又缓缓地将两只手盖在巨鹰勇士的头脑上。

%75
巨鹰勇士心头狂跳,在这一刻,他充分感受到了死亡的浓郁气息!

%76
“原来这才是它的战技吗?”

%77
“放缓时间,真是恐怖的战技啊。”

%78
“不过……就算我战死在这里,我也是有价值的。至少,我探查清楚了它的底细!”

%79
想到这里,巨鹰勇士蓦地大吼起来:“怪物!就算你杀死了我,又能怎样?还有众多的战兽勇士来围杀你的,你必死无……”

%80
啪!

%81
方源双手猛地向中间一用力,巨鹰勇士的脑袋就像是西瓜坠落到地上一样,破碎成渣,白的红的和头骨碎片搅和在一起。

%82
方源松开双手。

%83
扑通。

%84
没有头的巨鹰勇士尸体,砸到了地上,掀起一阵小小的烟尘。

%85
方源看向城池。

%86
他的体格太庞大,高耸的城墙也只抵他的膝盖下方。他面色平静,但牛头脸面却令无数人恐惧。白云在他的肩膀,本是晴朗的天气,明亮的阳光照射而下,他庞大的阴影深深地覆盖在所有人的心中。

%87
城池中人山人海,却无人发音。

%88
一座死城。

%89
方源迈开步伐,缓缓走向城池。

%90
下一刻,城池像是爆炸了一样,无数人开始抱头尖叫,开始疯狂地奔逃,像是无头的苍蝇。

%91
从高空俯视,方源的视野中这些凡人就像是一群堆在一起的蚂蚁,顺着城池的四个城门,开始分流溃散。

%92
方源并不搭理他们,直朝城池终于前去。

%93
砰。

%94
他的小腿骨撞在城墙上,一下子就令整面城墙崩溃。

%95
他走在城池的街道上,宽阔的街道就算扩张五六倍,也被方源的脚底板覆盖。

%96
一座座房屋像是小纸盒子,被方源撞烂踩扁,当然还有一些倒霉的凡人,都被方源直接踩成了肉酱血泥。

\end{this_body}

