\newsection{心思沉玉清滴风楼}    %第三百二十四节:心思沉玉清滴风楼

\begin{this_body}

%1
“怎么了?”乔丝柳不解。?(?〈[

%2
方源没有回答,他四处踱步,动用智道手段,陷入深深的思考当中。

%3
眼下局面非常复杂,幕后黑手不出,武庸生死未明,一切都笼罩在层层迷雾当中。但方源却被架在前方,不得不做出反应。

%4
身不由己。

%5
方源深深的感到,自己身不由己。

%6
武家蛊仙,身边的乔家蛊仙,还有南疆大势,幕后黑手,都让他难以继续待在级蛊阵当中。

%7
这就是身为正道的无奈。

%8
方源面临着一场艰难的考验。

%9
他绝不想就这样成为乔家、武家任意摆弄的棋子,他要从这复杂万分的局势中,保住自己的利益。

%10
但如何去做?

%11
武八重的用意和图谋,方源已经猜到。

%12
“或许我可以借助这一点?”想到这里,方源便决定将计就计。

%13
他停下脚步,面向乔丝柳:“既然让我统领全局,我便下令好了。”

%14
“遗海,这才是男儿的英武果断!”乔丝柳明眸骤亮,当即赞叹道。

%15
方源沉声继续道:“眼下幕后黑手隐匿不明,当务之急,是稳守最妥。我会下令武家各个蛊仙,放弃外围资源点,尽数回归武仪山,祭起仙蛊屋,以防不测。”

%16
乔丝柳很满意方源的这个举动。

%17
方源将自己的想法,有关武家的内政,提前告知她,很明显不把她当做外人看待。

%18
乔丝柳心中喜悦,沉吟道:“此法的确最为稳妥。不过,也是壮士断腕,一下子放弃防守这么多的资源点,恐怕会被周围的级势力顺势侵吞。”

%19
“那也没有办法。武家的力量,一直都分散得太开。眼下局势不明,只能缩起拳头,以防意外和挑战。”方源叹息道。

%20
“也好。我们走吧,离开这里,回归武仪山去。”乔丝柳认可了方源的这个应对,又开口道。

%21
方源摇头:“不去,不去。”

%22
“此行太过危险。”

%23
“连我兄长武庸都被埋伏,我俩区区七转,还不被那幕后黑手手到擒来?”

%24
“若这幕后黑手要对付武家,那么他们接下来的目标,很有可能就是我。我怎可能随意外出?归去武家的路程风险太大!”

%25
方源推托,理由充足。

%26
乔丝柳目光沉凝下来,微微点头:“此言有理!的确应该谨慎一些。若能得到仙蛊屋的接应,才为稳妥。”

%27
费尽口舌,终于稳住了乔丝柳,方源心底也是松了一口气。

%28
但他眉头还是紧锁。

%29
好不容易留在了级蛊阵之中,但这种情况,又能持续到什么时候呢?

%30
武庸居然生死不明,这种惊变,让熟知历史展的方源,也有措手不及之感。

%31
紫血先河阵中。

%32
血河滔滔,已经化为十几股。

%33
光线晦暗的环境中,无数的紫色念头,无穷无尽,暴射喷涌。

%34
铁面神大吼一声,身边环绕的铁砂宛若烟雾飞绕,形成滔天巨幕,遮护其身。

%35
但紫色念头忽然由实转虚,射进铁砂烟幕当中,毫无阻碍地向铁面神扑来。

%36
“可恶!”铁面神咬牙,对这虚道手段无能为力。

%37
不过就在危机之时,一道清风徐徐吹来。

%38
本身化虚的紫色念头,一个个如泡沫般破碎。

%39
“好险……”铁面神回望,向远处的武庸投去感激的目光。

%40
武庸周身,几乎是紫色念头形成的风暴,还有无数的血兽夹杂其中。

%41
即便战况如此激烈,武庸仍旧毫无伤势,还有余力照顾另外两处战场中的铁面神、乔志材。

%42
“这紫血先河阵居然囊括了虚道变化,不把这些墟蝠尸体清除,恐怕虚道变化将持续不尽!”乔志材大吼道。

%43
他必须大吼。

%44
在这里传音,根本不行。

%45
念头等智道手段的交流,更是无能为力。

%46
铁面神目光冰冷,宛若刀锋,他扫视一周,血河中有三四士头墟蝠尸,大部分只有荒兽级,但也有少量是上古荒兽,更有一头太古墟蝠的尸身,被两条血河共同托住。

%47
事实上,站至如今,铁面神也看出了一些端倪。

%48
这些墟蝠尸在血河中,不断地消融。尤其是每当紫色念头有了虚实变化的时候,这些墟蝠尸的消融度,就会加快几分。

%49
但是知道墟蝠尸的重要性又能如何?

%50
交战不久之后,那位八转蛊仙(紫山真君)就消失不见了。

%51
武庸为何束手束脚,不就是为了防备紫山真君的偷袭吗?

%52
“要让武庸大人驱除墟蝠尸体,非常不智,会被那位紫八转后制人。”

%53
“所以眼下局面,还是得我和乔志材联手,清除这些墟蝠尸体。”

%54
“若是紫大敌出手,武庸大人就能应对,保下我和乔志材的性命……”

%55
但如果保不住呢?

%56
或者说,即便能够保住二人性命,但是却要丧失对于这座蛊阵的打击机会。武庸会如何选择?

%57
铁面神迟迟不决,正是因为他们三者之间的信任很少。

%58
就在这个时候,一股成分复杂的庞大气息,从最中央的那片战场猛地升腾而起。

%59
大风呼啸,武庸大袖飘飞,仿佛风中君王。这一刻,他普通平凡的相貌,竟被衬托得威武至极。

%60
“你既然想要见识一下,我母亲留给我的八转仙蛊,那就让你看一看好了。”武庸冷然低喝。

%61
海量的紫色念头被排空,血河滔滔,竟有数股在风中生断流。

%62
乔志材又惊又喜。

%63
铁面神的铁面上,也被震动得掉下几许铁屑出来,他失声惊呼:“这是……仙蛊屋?!”

%64
武庸不动手则已,一动手竟如此惊天动地。

%65
他没有催动任何的仙道杀招,而是从仙窍中取出了一座仙蛊屋。

%66
这座仙蛊屋,一点都不辉煌雄阔,从外表而言,它只是一座吊脚竹楼。

%67
这种建筑,在南疆地域极其常见,在其他四域也不是没有,但往往只存在于山峦地形上。

%68
这座吊脚竹楼只有两层,几乎全是竹子构造,竹子上还生长着青色的竹叶,一颗颗露水,附着在竹叶上,青翠欲滴。

%69
武庸大袖飘飞,悠悠飞入竹楼的第二层。

%70
他坐在窗棂旁边,屈指轻弹,竹楼微微一震,屋檐上掉落下十多颗的青翠露水。

%71
刹那间,这些颗青翠露水,从吊脚竹楼上脱落,电射而出。

%72
所到之处,清风徐徐,平定一切。无论是紫色的念头,还是血红的潮水,都被清风化为乌有。

%73
“好厉害!”乔志材脱口称赞,青翠露水飞过来,为他轻松解围。

%74
“武家本来就有三座仙蛊屋,现在竟又增添了第四座!并且这座仙蛊屋,泄露出两只八转风道仙蛊的气息……武庸藏得可真是够深的。”铁面神心中震动不已。

%75
两位蛊仙被相继救出,飞入吊脚竹楼当中,来到武庸身边。

%76
“此是我母亲构造的仙蛊屋玉清滴风小竹楼。”武庸仍旧坐在窗棂上,一面望着外界的蛊阵,一面适时地解释道。

%77
“太棒了!有了这座玉清滴风小竹楼,我们直接立于不败之地。脱困而出,只是迟早的事情。”乔志材大喜过望。

%78
若是武庸运用武独秀的那两只八转仙蛊,催动什么仙道杀招,那乔志材将会非常担心。

%79
因为催动仙道杀招,会有失败的可能。越强大的杀招,催用失败后,反噬的伤害就越加可怕。

%80
偏偏武独秀身亡,武庸继承两只八转仙蛊的时日,并不长。

%81
尤其是武庸还是武家的太上大长老,身居高位,平素能有多少闲暇时间来练习?这是个大问题。

%82
但现在好了,这两只八转风道仙蛊,居然和其他蛊虫搭配,组成了一座仙蛊屋!

%83
众所周知,仙蛊屋虽然手段固定,但最是操纵简易,不担心反噬。

%84
铁面神的心绪比他要更加复杂一些,他暗想:“武家居然有了第四座仙蛊屋,而且竟是一座八转仙蛊屋,威力远前三者。每一座仙蛊屋,都可大大直接提升一个家族的综合实力。有八转仙蛊屋在手,配上一些蛊仙,只要仙元足够,就是武家第二位八转蛊仙啊。这武庸藏得太深了!”

%85
他不由地再次打量武庸,刮目相看。

%86
武家这段时间,并不容易,遭受了四面八方的刁难。作为武家仅存的八转蛊仙,武庸左遮右挡,调兵遣将,处境很是勉强尴尬。

%87
但只要他甩出这座仙蛊屋,必定能直接化解尴尬,重振武家声威,立即让武家再次稳稳地坐在南疆第一级势力的宝座上,把其他蠢蠢欲动的心思都镇压。

%88
然而,武庸并没有。

%89
他一直凭借自身力量,来应付武家的危局。他把这座仙蛊屋藏得很深很深,就算是武家蛊仙也不知晓。

%90
若不是他身陷于紫血先河阵中,谁都不会知晓,武庸手中居然拥有第四座仙蛊屋,而且还是八转仙蛊屋。

%91
“单凭这一手,武庸绝不在他的母亲武独秀之下!幸亏我铁家身居南疆东北,而武家则位于南疆西南,两者相距遥远。而且我铁家也没有参与之前的行动,没有刁难武家。”铁面神吐出一口浊气,感到庆幸。

%92
各大级势力,霸占一方,盘踞在南疆已经这么多年。

%93
相互之间,已经了解得很多,很透彻。彼此的实力和地盘都维持在一个平衡的,相互对峙的状态当中。

%94
武独秀一死,这个平衡被打破,才有了其余级家族默契联手,为难武家的情况生。

%95
但只要这座玉清滴风小竹楼一出,就能替代武独秀,成功镇压南疆大局。

%96
武庸的心思太过阴沉,他一直暗藏不出。完全可以预料,一旦当他觉得是时候祭出这座仙蛊屋的时候,必定是其余家族倒霉,被揍得七晕八素的时候!

\end{this_body}

