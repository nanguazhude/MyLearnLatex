\newsection{开端布阵}    %第八百六十六节:开端布阵

\begin{this_body}

东海。

颠簸海域之东,烧烤群岛之西的地方,有一片奇特浩大的海域——气海!

气海中没有点滴海水,而是无数的气流交织纠缠,形成亿万的气流漩涡。

在远古时代,气海在全天下蛊仙的心中,占据着无上的崇高地位。

那个时候,气道昌盛至极,乃是蛊修的主流,其余皆是偏门。

万般皆下品,惟有气道高。

气海的规模更是浩瀚,内有乾坤,无数气流穿梭其中,生灭不断。那个时候的气海,是五域第一的气道资源点。

时光匆匆,沧海桑田。

气道衰败,气海的规模也不断缩减。到了如今,气海的面积已经不足万分之一,并且里面的气流都沦为单纯的水气。唯有最中央的地方,还有着其他种类的一些仙材气流。

气海从万仙来朝,沦落为几乎无人问津。

它包含最多的是水气。

但水气,偌大的东海哪一片海域没有?

水气相当容易提取,蛊仙们何必千里迢迢来到气海搜集呢?

就算是谋求其他的仙材气流,进入气海中央探寻,实在是风险巨大。反不如直接在宝黄天中收购来得划算。

“此次我举办东海大宴,气海必将再次名扬五域。”方源本体端坐云团,俯瞰着脚下的气海海域。

从高空看下去,气海海域呈现淡蓝色,无数气流气浪翻腾滚滚,和周围接壤的波光粼粼的深蓝海水形成鲜明对比。

当初,南疆八转蛊仙气相开创出气海无量杀招,在某种程度上也是从这片气海中得到的灵感。

方源号称气海老祖,当然不是随意取的称号。背地里,早就有了充分的考量。

比如气海老祖隐居之地,究竟是在东海的哪一块?

方源将这个隐居地点,就定在了气海海域。

这里几乎无人问津,很少有蛊仙前来探寻,但却是和气海老祖专修的气道,极其匹配。

说出去的话,合情合理。

从气海中央,传来龙人分身吴帅的消息:“大阵已经布置妥当了。”

这场东海大宴,对方源而言,意义重大,因此准备得很多。

方源不仅将吴帅调遣过来,而且还在气海中央铺设大阵。

这大阵自然是气道大阵。方源宙道分身凭借气道大宗师境界、阵道宗师境界,以及智慧光晕,耗费一个多月的时间,方才推算而得。

气海中央,无数气流汹涌,时刻穿梭不断,环境相当险恶。

在这种地方布置仙阵,难度极高。

不过,还是难不倒方源。方源阵道造诣不缺,手头上资源也算是丰厚,再者搭建的大阵也没奔着九九连环不绝阵这样的层次去,而是很务实,追求性价比。

方源本体探去神念,检查了一遍后,确认无误,便开始闭目养神。

这场东海盛宴对他而言,意义重大。

他在以天下为棋盘,布置棋子,引导大势,然而这种事情不是那么容易的。

南疆方面,方源的影响力早已经走低,武庸正在积极串联,他政治手腕高超,方源也无法遏制。

西漠方面,虽然有房家崛起,但影响力还需要时间去发酵,同时,不管是天庭或者其余西漠势力,都有着不同心思。房睇长分身任重道远。

中洲有天庭,几乎水泼不进,铁桶一般。当年影宗安插下的僵盟,还有逆组织都已经被紫薇仙子剿灭干净。

北原,方源和楚度走得最近,但无关大局。虽然冰塞川还未苏醒,但长生天已经逐渐发力,这不是方源能够掺和的地方。

所以,东海这片地方就成了方源布局的重要一环。

东海乃是资源最为丰富的一域,若是成功的话,将来五域合一,方源就能凭此影响力,辐射其他四域。

半柱香之后,方源缓缓睁开双眼,凝望天边,口中呢喃:“来了。”

天边一座仙蛊屋飞来,速度似缓实快。

仙蛊屋悠然停歇下来,走出一位明眸少女,她身着长裙,裙摆上绘有粉色彩云团团。

“华家太上大长老华彩云,拜见气海老祖。”华彩云有八转修为,态度恭谨。

随后,从华家的仙蛊屋中,又走出几位华家蛊仙,纷纷拜见方源。

“约定的时间未至,华仙友提前来了,请坐吧。”方源伸手一展,右手边上就气流交汇,形成一个云团。

华彩云便将仙蛊屋收入仙窍,她自己入座,其余的华家蛊仙站立在她的身后。

“晚辈初次参见老祖,心怀激荡,这是华家的一点小心意,还请老祖收下。”华彩云说着,她身后的蛊仙便取出一物,双手呈上。

方源看去,只见此物乃是一个气团,宛若圆球,被华家蛊仙托在两手之中。

这气团中时而电闪雷鸣,时而瓢泼大雨,时而风和日丽,时而大雪纷飞。

方源眼中精芒一闪,辨认出来。

这是一团罕见的九转天气。

天气、地气乃是万物母气,本身是很寻常的。方源有时候坐落仙窍,大开仙窍门户,就是为了吞吸外界的天地二气,弥补至尊仙窍。他的仙窍里有许多天地秘境,这玩意实在耗损天地二气,对至尊仙窍而言也算得上是沉重负担。

当然,方源吞并了气相洞天之后,气道道痕暴涨,仙窍中自产了许多天气、地气,让这方面的负担减轻了不少。

但方源至尊仙窍吸收或者产生的这些天气,都是普通的蛊材,只是量极大,连仙材都算不上。

天气若是成为仙材,就会凝聚成一团,无法轻易分离。价值暴涨,普通天气是无法比较的。

仙材天气最简单的用法,就是直接打散它。

有不少蛊仙就这样直接用。

他们在宝黄天中收购这么一团仙材天气,在自家的仙窍中打散掉,从而形成雷电风霜等等天气变化。

这种天气会持续一段时间,给蛊仙的资源点或者仙窍造成一定的影响。

宝黄天中一般贩卖的,都是六转、七转层次的天气。八转仙材天气,已经是相当罕见。而九转天气是可遇不可求,宝黄天中上千年都难得一见。

华彩云将一份九转天气送给方源,一是可见她的心意,二来华家最擅长的就是云道,精通黑白两天的探索,展现出了华家的雄厚底蕴。

几乎与此同时,光阴长河。

四座仙蛊屋在长河上空飞驰。

今古亭四面透风,结构简朴,踏浪而行。

三秋黄鹤台清秀飘逸,橙黄的屋檐飞角,如鹤展翅。秋日的光晕,分成递进的三层,笼罩仙蛊屋表面。

鲨流撬雪白如玉,撬前有着七头巨鲨,森白锯齿,拖拽着巨撬,奔袭如飞。

刹那台高达八转,外围是一圈石梯,螺旋攀升,直至内围最高处的红砖青瓦阁楼。

四座仙蛊屋承载着天庭、中洲的众多蛊仙,都奔向一个目的地——红莲石岛!

今古亭中,一位八转蛊仙站着,望着亭外河水滔滔。

他有一身红白相间的长袍,身姿挺拔,似枪似剑。剑眉入鬓,眼中蕴藏神芒,含而不露。他气息温和,潇洒风流,此刻目光深远,心中参悟时刻不停。

此人正是凤九歌!

“我在光阴长河的这段时日,时刻观察长河,览天地兴衰,看古今起伏。命运就像波涛,时而高时而低,时而缓时而急。其中玄妙不可言说,只可歌之。”

凤九歌一直以来,都在参悟命运歌。

和上一世不同,方源这一世重生,凤九歌的命运就有了许多改变。

他在琅琊福地中,被方源击败,又被陈衣通过来因去果杀招送走,便得到了命运歌的灵感。

这比他上一世,要提前了太多。

有所感悟之后,凤九歌便来到光阴长河,紧紧抓住这份灵感,进行参悟。

光阴长河乃是宙道秘境,但凤九歌凭借今古亭,却可以洞察古往今来的某些人生,某些画面。

这对他参悟命运蛊具有极大的帮助。

如今,他的命运歌已然完成了六成,比上一世最后时中洲宿命大战,还要高出一成多来。

“我的命运歌,已经讲述了万千生命的命运。但尊者的命运,却都不能尽情描绘。”

“上一次在天庭,我浏览狂蛮血皮,就有一种捉摸不透的感觉。但愿这一次的石莲岛,能够我创作命运歌,带来机遇。”

“到了,就在这里。”

“石莲岛……红莲真传!”

四座仙蛊屋缓缓停歇下来,而在石莲岛的上空,恒舟早就悬浮在这里了。

恒舟就是第一个发现石莲岛的仙蛊屋,上面的蛊仙立即通知天庭,这才有了现在的五屋集合。

凤九歌目含期待地望去。

刹那台中的顾六如,则是神情复杂。

“终于是找到了,不枉费我天庭一代代人的努力。无数年来的付出,终于在此刻有了结果。”顾六如心中感叹不已。

他一生经历坎坷,下半身瘫痪,无法治愈,性情冷漠,但此刻却流露出种种情怀,有激动,有骄傲,有酸楚。

深呼吸一口气,顾六如已彻底平静下来。

他坐在轮椅上,眼中闪烁着寒光,冷静下令:“开始布镇河锁莲大阵。我领刹那台,凤九歌领今古亭在此坐镇。恒舟出发,去往支流河口,接引我方援兵。鲨流撬、三秋黄鹤台逡巡周围,一旦有什么风吹草动,务必立即向我汇报。”

“是。”群仙领命,立即行动起来,效率奇高。

------------

\end{this_body}

