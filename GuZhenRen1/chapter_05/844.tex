\newsection{再走第七层}    %第八百四十八节:再走第七层

\begin{this_body}

方源一席白衣,缓缓降落到疯魔窟中。

在地面,有两位蛊仙早已经站立良久,恭候多时。

一位蛊仙,面容白皙,黑色胡须飘飘,宽松大袖,双目锐亮似星,风度不凡,他号称不是仙,专修律道。

另一位蛊仙,则身材矮小,只及方源的膝盖,一双小眼睛精芒烁烁,乃是秘谋人,专修智道。

时至今日,方源已今非昔比,琅琊守卫战、光阴长河大战两次大战,都将天庭踩在脚下。期间,还把南疆正道的一大群蛊仙俘虏,其中包括两位八转!前不久,从东海蛊仙界还有消息传出,说方源借助了气海老祖之力,抵抗龙公,将仙蛊屋龙宫收取走了。

要论当代最强,风头最劲的魔道蛊仙,非方源莫属。

“两位仙友,我们又见面了。”方源落地后,首先施礼,面带微笑,态度很是客气。

秘谋人、不是仙连忙还礼。

不是仙笑道:“上次分别后,便听闻柳兄的真正身份便是古月方源,着实让我疯魔三怪又惊又喜。这次仙友能够到来,定是有了突破当前瓶颈的手段,我等早已望眼欲穿了。”

方源当初是以柳贯一的身份,缔结了疯魔之约,和楚度一样,成为疯魔窟中的一员。

所以,不是仙称呼方源为柳兄。

对于疯魔三怪而言,方源的面貌却是没有变化的,因为当初柳贯一的相貌,就是方源至尊仙体的模样。

“怎不见胖山仙友?”方源问道。

“他上一次探索第七层时,似有所悟,如今正在闭关。”秘谋人答道。

“是这样啊。”方源笑了笑,心中了然。

这是疯魔三怪在防备自己!

当初,他扮做柳贯一,战力并不强。如今方源提升迅猛,天下公认方源虽只有七转修为,但战力可敌八转,疯魔三怪自然要顾虑方源,防止他忽然出手,对他们三个下毒手。

毕竟方源乃是当代最大的魔头,性情凶残,别看他现在这样笑呵呵的模样,实在是杀人不眨眼,吃人不吐骨头的凶人。

疯魔三怪虽然久居疯魔窟,一门心思地苦修,但并不蠢笨。尤其是三怪之首的秘谋人,更是在方源五百年前世有过相当出彩的表现,让天庭都吃了大亏。

疯魔三怪忌惮方源,因此胖山并不出现,而是隐藏一边。

很有可能,他的闭关只是一个借口而已,以往万一方源丧心病狂,忽然对他们下手!只要三怪中始终少一个人,就没有被方源一网打尽的忧虑了。

“不仅如此,恐怕疯魔三怪还有着某些手段,可以启动。所以要始终保留一人在外。”方源心中不断推算着。

方源虽然签订了疯魔之约,但这个盟约早就被他解掉了。

疯魔三怪也有不在乎的杀招,也完全可以无视疯魔之约。

疯魔三怪忌惮方源,防备方源,方源则是不大想动手。

至少现在不太想动手。

“说起来,我这一次来到疯魔窟,已经是第三次了。希望会有所收获!”方源此行最主要的目标,还是无极真传。

不久的将来,天庭必定会为了修复宿命蛊,组织中洲炼蛊大会。

到那个时候,必将有一场浩大至极的五域蛊仙大战!

在这场大战中,方源尽管实力雄厚,但并不能掌控局面。任何一道尊者手段,就能轻而易举地颠覆原有局势。

这一点,方源在上一世已经看得非常明白了。

最佳的例子就是龙公。

龙公之强,无人能敌,但在尊者的手段下,却是根本不够看的。无极魔尊一出手,就将他五花大绑,动弹不得了好久。最终还是在元莲仙尊的手段下,龙公这才脱困,重拾自由之身。

方源之前也计划过:这一世他不仅要积极地增长自身实力,而且还要收集一些尊者真传,若是能引发出尊者手段出来,将是对自己摧毁宿命蛊的大计的最佳支持!

而最可靠的线索,便是这里——疯魔窟!

方源选取的时机,也恰到好处。

自己目前的战力已经加强了很多。表面虽然只是七转修为,但实际上已经晋升八转,就算疯魔三怪对自己不利,自己也能迅速反击,性命无虞。

而再过一段时间,长生天中的冰塞川就要苏醒。他的醒来,对于长生天,对于北原蛊仙界,都有极大的影响。

到那时,方源再进来北原,若是闹出什么大的动静,恐怕就要面对冰塞川领袖的长生天,难以收场了。

不是仙在前边引路,秘谋人则陪伴在方源身边和他交流。

方源随着二怪,深入疯魔窟。

疯魔窟共分有九层。

第一层,就是最外面的一层,也就是方源刚刚鸟瞰到的景象。

第二层,则是一片炙热无比的火岩石滩。

第三层,云雾缭绕,猛兽隐藏,还有雾都,十分凶险。

……

每一层都有大相径庭的环境,每隔一段时间,就有魔音从最深的第九层传出,辐射整个疯魔窟九层。

魔音期间,万物疯狂,展开厮杀,混乱不堪。

越是深入,魔音越是强悍。

不过依靠疯魔三怪推算出来的手段,却是能够有效地抵抗魔音一部分的威能。

方源运用的却不是这个手段了,他在来这里之前,他在原来的基础上进行改良,卓有成效,算是优化版本。

毕竟他现在的律道境界,也达到了大宗师的程度。

魔音之间,对于催使仙道杀招有极大的影响。因为魔音本身,就是混乱道痕,神智不清、陷入疯狂只是副作用而已。

三仙一路往下,路线弯弯绕绕,终于下达到了第七层,也就是倒数第三层。

当初方源抵达的终点,就是这里。

疯魔三怪同样止步于此,已经数百年了。正因为他们陷入瓶颈,无法突破,这才先后吸纳了楚度、方源这两人。

不过他们同样收获多多。

在他们探索的七层之中,零零散散有着无极魔尊的遗产。

疯魔三怪得到种种线索,苦心钻研,又进行推算,也大致算出最后一层的景象。虽然他们从未进去过。

方源的视野中,一片璀璨耀眼。

石头上、泥土中、草木上莹莹放光。各种颜色,五彩缤纷。

这是道痕之光。

当道痕浓郁到一定的程度,就会天然绽放出这种绚烂至极的光辉来。

在这第七层中,每一片土壤,每一块石壁,都是准九转的仙材!然而分外可惜的是,这些道痕都是错乱不堪,无法利用。

“方源仙友,你先请。”秘谋人谦让道。

“自从上次失败之后,我反省良多,有不少心得。”方源哈哈一笑,当先迈步,走在前头。

秘谋人和不是仙对视一眼,也纷纷进入,与方源并肩行走。

最近这段时间,方源声威震天动地,但真正他到底有多少实力?

疯魔三怪心中没底,眼下就是一个最佳的观测机会。秘谋人、不是仙皆是将注意力集中在方源的身上。

方源信步而走。

前十步,惬意悠然,

二十步,他轻轻松松。

三十步,他开始面色凝重。

四十步,他渐渐气喘吁吁。

五十步,他面露疲惫之色。

六十多步,方源停歇下来,喘了口气。

秘谋人看向不是仙,不是仙也同时看向秘谋人,两人都从对方的眼中看到了震惊之情。

要在这里行走,需要克服海量的道痕排斥。疯魔三怪最远的极限已是深入上万步,但他们是走走停停,断断续续。若是像方源这般一口气走下去,最多也就四五十步而已。

“可怕的是,方源走下了六十多步,明显是没有动用全力,还是游刃有余的!”

“单单这份成绩,已经超过了我等三人。但是他的极限究竟在哪里呢?”

不是仙、秘谋人脑海中念头纷起。

但事实上,方源拥有至尊仙体,根本不受任何道痕的排斥,所有的疲惫都是他伪装和扮演的。

方源宁愿表演一番,因为同样是震慑疯魔三怪,并且还能藏拙。

方源生性谨慎,喜欢深藏不露,能不展现的牌面绝不会轻易拿出来展现。

“真不愧是方源仙友啊。”

“是啊是啊,如此实力,令我等三人忏愧。”

方源眯起双眼:“二位不必自谦了,我重生多次,带有许多记忆。疯魔三怪名头响亮,都有各自的绝妙手段,我亦感佩服。”

不是仙、秘谋人又都心中一凛。

方源说的很有水准,你可以理解成是一种夸赞,但也可以理解成一种警告——我重生归来,熟知你们的手段,不要与我为敌!

三仙接着往前走。

越是深入,越不好走,压力重重。

“接下来的路程,依靠和各自流派的道痕会省力许多,方源仙友,咱们暂且分路而行吧。”不是仙道。

他选择向左前方走,那里的律道道痕比较多。而他本身就是专修律道。

不久后,秘谋人也告辞,他选择了一条智道道痕比较多的路径。

方源当然可以直线行走,但他没有这么做,而是选择了藏拙,仔细挑选着路径。

疯魔二仙将这一幕看在眼里,皆若有所思。

\end{this_body}

