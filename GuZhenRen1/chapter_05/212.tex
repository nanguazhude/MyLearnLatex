\newsection{中洲蛊仙集结}    %第二百一十三节:中洲蛊仙集结

\begin{this_body}

明天就是515,起点周年庆,福利最多的一天。除了礼包书包,这次的『515红包狂翻』肯定要看,红包哪有不抢的道理,定好闹钟昂\~{}

中洲。

某处山村。

轰隆隆!

巨大的轰鸣声,回荡在山间。

无数的泥石流,宛若雪崩一般,从山上往下滚落,浩浩荡荡。

“快逃命啊!”

“我要死了,我要死了……”

“爹,你在哪里,不要丢下孩儿啊!”

眼见天灾降临,原本静谧祥和的小山村,已经彻底炸开了锅,乱了套。

鸡飞狗跳,无数人四处奔逃,也有人瘫痪在地上,小儿啼哭,被母亲死死地抱在怀中,不少人已经放弃了抵抗。

在无人注意的情况下,一个身影浮现在高空当中。

这是一位七转蛊仙。

他身着丝绸蓝袍,长发披肩,身材并不魁梧,反而有些柔弱的样子。

此时,他皱起眉头,看着眼前的泥石流,口中呢喃:“这场泥石流,来得分外古怪。”

一般而言,泥石流之前,都会有罕见的暴雨,但最近这段时间,一直都是风和日丽。

事实上,这里方圆数万里的环境,都已经被这位七转蛊仙暗中调控,可谓风调雨顺,年年丰收。

砰。

山石忽然四处飞溅,从山中冒出了一个巨大的黄铜甲壳。

七转蓝袍蛊仙悬浮在空中,看得分外明显。

他微微皱起的眉头,此时舒展开来:“我道如何?原来是一头泥沼蟹。”

泥沼蟹乃是荒兽,它有山一般的雄阔身躯。它没有眼睛,或者说眼睛已经完全退化,它浑身都被甲壳包裹,防御上毫无漏洞。

它是荒兽中,泥沼里的君王。

十对螯足,刚硬超凡,尤其是第一对螯足。分外粗壮,轻轻一夹,就能断山石,剪蛟龙!其余的十八只螯足。即便比较瘦长纤细。但实际上,也都比百年古木还要粗壮。

七转蓝袍蛊仙见到这头泥沼蟹,双眼微微一亮,心中欢喜。

他是水道蛊仙,而这头泥沼蟹的身上。却是拥有土道、水道两种道痕。斩杀掉这头泥沼蟹,对于七转蓝袍蛊仙而言,就是一大堆的可用仙材。

“不过泥沼蟹只是荒兽,若是再加以精心的培养,或许能生长成为上古荒兽泥羹沼蟹。泥羹沼蟹之上,还有泥衣沼蟹。泥衣沼蟹乃是太古级荒兽,我倒是不用奢望了。我的仙窍福地,可培养不出这等太古荒兽。不过泥羹沼蟹,却是可以尝试一下。”

想到这里,七转蓝袍蛊仙便开始动手。

他从宽大的袖口中。伸出手臂。

他的皮肤很白,十根手指也是纤细修长。

他的十根手指晃动起来,仿佛莲花绽放,搅起一层层的斑斓光影。

这是他特有的操纵蛊虫的方法!

很快,他的身上就升腾起无数蛊虫的气息,有凡蛊气息,有仙蛊气息,一股股的气息相互叠加起来,形成四处散溢的复杂气场。

山中的泥沼蟹,虽然没有双眼。但凭借野兽的直觉,敏锐地感觉到了来自半空中威胁。

泥沼蟹开始往山体中缩去。

巨大的黄铜甲壳,很快就消失了一半。

但这个时候,七转蓝袍蛊仙已经酝酿完毕。仙道杀招催发而出!

哗啦啦……

澎湃汹涌的蓝色浪潮,凭空而生,浩浩荡荡地向泥沼蟹冲刷过去。

泥沼蟹身躯颇大,躲闪不及,被浪潮卷席。

但潮起潮落,泥沼蟹却岿然不动。它的身躯着实太过沉重。

七转蓝袍蛊仙的嘴角处,却绽放出了胜券在握的微笑,他的十根手指晃动更疾,一根根手指的残影,停留在空气中,让人眼花缭乱。

他的这记仙道杀招,并不平凡。

随着红枣仙元的不断消耗,原本浅蓝色的浪潮,变成了蔚蓝之色,浪潮澎湃浩瀚,比之前的规模扩大了三倍有余。

潮水在七转蓝袍蛊仙的操纵下,如臂使指,形成一片巨大的漩涡。

泥沼蟹深陷蓝潮的漩涡中心,终于抵挡不住,被漩涡卷席。

哗哗哗!

泥沼蟹庞大沉重的身躯,在浪潮漩涡当中,越转越疾,好像是浮萍一般,完全身不由己。

七转蓝袍蛊仙忽然双手一握,残留在空气中的无数手指光影,骤然消散,浪潮忽然高高涌起,形成滔天的海啸。

轰!

一声巨响,数万吨的水浪形成的海啸,重重地拍击在泥沼蟹的背上。

泥沼蟹的背壳非常坚硬,但被这股海啸一拍,原本平整的背壳表面立即凹陷下去了一大块。

泥沼蟹一动不动,被拍得当场昏死过去。

七转蓝袍蛊仙朗笑一声,右手轻轻展开,食指由下至上,轻轻一提。

顿时平复下来的浪潮中,涌起一股,宛若是喷泉一样,将泥沼蟹庞大的身躯,缓缓冲上空,升到七转蓝袍蛊仙的面前来。

蛊仙打开自家的仙窍门户,将这头泥沼蟹收入囊中。

“仙人!是仙人啊!!”

“谢谢仙人,仙人救了我们全村的性命。”

“这位仙人还将山中的蟹妖给击败了!”

这样浩大的战斗情景,早已将山村中的凡人们看得震惊万分。

直到战斗结束,他们这才反应过来,一个个狂喜呐喊,或者倒头便拜。

七转蓝袍蛊仙关闭仙窍门户,用淡淡的目光扫视了下方人群一眼,他轻轻一笑。

原来在他和泥沼蟹交手的过程中,他分成了一部分心神,操纵浪潮,将滚落下来的泥石流统统卷走了。

“不愧是中洲有名的水道蛊仙沐凌澜。”这时,一个声音从更高空的云层中传来。

七转蓝袍蛊仙沐凌澜十指气动,将漫溢山间的大水,悉数收起。

然后他飞入云层高空,见到了另一位蛊仙。

只见此仙身着白袍,国字脸,眉毛粗重。鼻梁高耸,一身正气凛然,不可侵犯的气质。

沐凌澜笑着一礼:“原来是施阁前辈到了。”

施阁还了一礼:“也是刚到,没想便有眼福。见到沐凌澜你轻取泥沼蟹的手段。”

沐凌澜摆摆手,谦虚道:“我的手段,在前辈面前,只是小道,不足挂齿。听闻施阁前辈。也已接到天庭命令,参加此次北原之战。”

施阁点点头:“没有错。沐凌澜你也在名单之中,我们不妨同行?”

沐凌澜满脸欣然之色:“能与前辈同行,是晚辈的荣幸。”

于是,双仙结伴而走。

留下一地的凡人,还久久地仰望天空,兀自喟叹。

沐凌澜、施阁一路交谈,倒不显得寂寞。

两仙虽然都是七转修为,但施阁成名已久,已经度过二次浩劫。而沐凌澜却是一次都未渡过。所以,主要是沐凌澜请教,施阁指点。

行了一段路程,施阁忽然降下云头,缓缓悬停在高空。

沐凌澜不解,此处并非集合地点。

施阁笑道:“惭愧,我有一子取名正义,最近刚刚渡劫,成就蛊仙。然而缺乏历练,还有许多小孩气。这一次北原之行。我打算将其带在身边,加以磨砺。”

“原来如此。”沐凌澜恍然,当即顺着施阁的目光,注视下去。

只见云层下方。有一小城。

城中房屋不少,其中一座酒楼当中,一位说书人正讲行侠仗义的民间故事。

“好!杀的好!”听客之中,一位少年郎,浓眉大眼,满目纯真。作农夫打扮。此时听到故事中的主角杀富济贫,不禁拍案叫好。

他的喊声突兀,音量又大,一下子震得窗棂都微微颤动。

说书人被吓了一跳,停顿下来。

周围的听客都不满地嚷嚷道:“叫什么啊?”

“忽然一大声,要吓死人哩。”

“吵什么吵,安静听书,不然赶你走啊,小泥腿子。”

少年郎满脸涨红,挠挠头,不好意思地四处作揖:“对不住,对不住了,各位。”

听客们见他立即认错,态度谦恭,嚷嚷声顿时减弱了许多,不再过多计较。

少年郎讪讪坐下,忽然神情一变,又腾的一下子站起来,把方桌和板凳都推到了一边,造成更大的响动。

“又干嘛啊,你小子!”

“臭小子,要找打!!”

听客们群起而攻,忽然间,少年郎浑身冒光,身形如箭,射破窗棂,飞上过来高空。

酒楼顿时沸腾,无数人惊呼大喊,一片嘈杂骚乱。

少年郎来到施阁、沐凌澜两仙面前,抱拳施礼,恭恭敬敬。

原来他正是施阁的儿子施正义。

三仙继续启程,几日后,来到一处山脉。

一座仙蛊屋已经悬停于此。

它是一座亭阁,小巧精致,悬梁之上挂着无数鸟笼,各种鸟类在里面叽叽喳喳。

正是天莲派的仙蛊屋揽雀阁。

施阁见此,微微点头:“早已听闻风声,此次攻打北原,将出动仙蛊屋。没想到是天莲派的揽雀阁,此阁擅收飞禽,移速极快,的确是上佳之选。”

沐凌澜附和道:“天莲派乃是元莲仙尊开创,拥有仙蛊屋最多,数量多达四座。天莲派能派遣出一座来,也是合乎常理之举。”

施正义疑惑地问道:“天莲派拥有揽雀阁、岳阳宫、天池,怎么还有第四座仙蛊屋?”

沐凌澜笑了笑:“小义有所不知,近些日子,天莲派已经开创出了第四座仙蛊屋,只是暂且秘而不宣罢了。”

施正义哦了一声,心想:“沐凌澜前辈出自灵蝶谷,此派最是擅长信道手段,他知晓一些秘闻,也不奇怪。只是天莲派一派就拥有四座仙蛊屋,着实有些吓人呐。”

ps:第二更会晚一些。

ps. 5.15「起点」下红包雨了!中午12点开始每个小时抢一轮,一大波515红包就看运气了。你们都去抢,抢来的起点币继续来订阅我的章节啊!

------------

明天加更!

app上,有一个“作家荣登封面”的活动,蛊真人目前排行第17。

这个位置,有点可惜,再进一个名次,就是质变的结果了。而且和前一名差距很小。

在这里,拉一下票。

大家能投一些赞赏票给我,就十分感谢啦。

这个活动,很多读者朋友都一直在鼎力支持我。

非常感动!

谢谢大家了。

这个活动快要截止了,让我们在最后关头拼一把。

不管结果如何,我明天都要加更的。

最近的这些章节,因为涉及到很多人物和情节,所以写起来非常艰难。

一直在酝酿高潮。

明天将是大雪山福地和中洲蛊仙之间的精彩对决,希望大家能够喜欢!

\end{this_body}

