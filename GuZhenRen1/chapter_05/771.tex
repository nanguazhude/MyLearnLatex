\newsection{吞吞吞!}    %第七百七十四节:吞吞吞!

\begin{this_body}

至尊仙窍。

杳无人烟的小黄天之中,一道身影电射而来。

身影猛地顿住,落在一群蛊仙面前,正是方源的宙道分身。

“拜见方源大人!”

“拜见宗主!”

群仙一齐施礼。

放目望去,除了影无邪已经继续坐镇青鬼沙漠,方源的麾下已是到齐。

白凝冰、黑楼兰、妙音仙子、白兔姑娘、雪儿、冰媛、石宗、石狮诚、墨坦桑等,仙气缭绕,人才各异,济济一堂。

这些人都参与了之前的年流伏诛埋伏大战,亲眼目睹了方源是如何把控战局,将南疆诸仙俘虏的。

因此,此刻众仙看向方源的目光,都远比战前要更加热切、崇敬。

尤其是那些异人蛊仙,垂眉顺目,态度恭谨,乖顺得仿佛猫狗。

“无须多礼。”方源瞥了他们一眼,便转身看向另一侧。

一座仙阵悬浮高空,不久前才经有这群蛊仙之手搭建起来。

当然,大阵的主体是方源亲自出手的,剩下来的一些细枝末节,方源则交给了这些下属。

方源的宙道分身细心观察一阵,满意地点点头,确认此阵状态一切良好。

这就是智炼仙阵,利用智慧光晕,辅助妇人心、解谜,炼化他人仙蛊。若有他人意志,效果更佳。

最初的时候,方源就利用过墨瑶假意,帮助他炼化了从薄青仙僵那里的仙蛊,其中就包含了换魂仙蛊。这只关键的仙蛊让方源成功翻盘,在义天山大战中虎口夺食,抢走了至尊仙胎蛊。

上一世,此阵经过方源一番改良,威能更甚。

这一世,方源只是微调了一些地方,拿来就用,因为威能足够了,所以没有改良的必要。

“好了,入阵吧。”方源吩咐道。

他身边的一部分蛊仙纷纷入阵,替代方源操纵大阵,维持着运转。

方源转身,看向身后。

身后,是一片迷离的梦境,仿佛是一大团巨大的彩色云朵。

一具纯梦求真体早已凝聚出来,这是方源上一世从凤金煌手中学来,是永久存在的完美纯梦求真体。

只是方源魂魄修为很弱,没有资本分出魂来,因而操纵纯梦求真体的只是方源的意志。

意志会随着消耗而不断缩减,直至消散全无。

唯有搭配魂魄,才能算得上是一具分身。

因此这具纯梦求真体,只能算半个梦道分身。

不过这已经足够眼下的需求了。

纯梦求真体钻入梦境之中,很快推出一位南疆蛊仙。

还未入阵,仍旧留在方源身边的蛊仙们神色微动,他们都认出了此人。

大战之后,他们也都没有闲着,积极搜寻这席俘虏的情报,因此对他们的跟脚知晓得很多。

“这是羊家的蛊仙羊枯啊。”

“他是当今南疆蛊仙界中,魂道巅峰的代表人物之一。”

“他曾经以一人之力,捣毁一群上古荒兽的巢穴,随后在太古荒兽的追杀下成功逃生。”

白兔姑娘等人在低声交流。

羊枯的上半身仍旧留在梦境中,腹部以下的部分露出来,方便方源本体施展杀招。

仙道杀招——大盗鬼手!

一记鬼手从天而降,迅速钻入铁区中的仙窍里。

找群仙的注视下,只是几个呼吸,鬼手就钻了出来。

它飞过方源的宙道分身,来到徐徐运转的智炼大阵上空,然后将五指摊开,将被盗取过来的仙蛊轻轻地抛入大阵。

这是一只七转魂道仙蛊!

仙蛊进入智炼大阵,立即被大阵包裹,陷入中央地带。

“全力催动!”大阵内的蛊仙连忙大喝。

智炼大阵顿时发出璀璨的金芒,在嗡嗡的嘈杂之音里,智慧光晕也集中在了被偷来的仙蛊上,迅速将其炼化。

大盗鬼手又飞回铁区中的仙窍,数息时间就又钻出,带出第二只仙蛊。

这只仙蛊却是摄魂蛊了。

方源宙道分身点头,有了这只仙蛊充当核心,摄魂杀招就能催用出来。

大盗鬼手将摄魂仙蛊抛入智炼大阵,又再次飞回羊枯的仙窍里。

接下来连续两次,它都盗出仙蛊,将羊枯掌握的仙蛊偷盗殆尽。

一些不明就里的异人蛊仙,看到这一幕,皆露出震惊之色。

难道方源已经将大盗鬼手杀招改良,使得它每一次都能偷盗出仙蛊来?

若是这样,那大盗鬼手杀招无疑比之前,还要恐怖千百倍!

大盗鬼手杀招的确经过了一番改良。

方源参悟偷生真传,有所成果,使得大盗鬼手能够定位,有针对性的进行偷取。

只要方源对目标非常了解,知道仙蛊的具体位置,就能做到次次偷取出仙蛊来。

偏偏方源对羊枯的仙窍,可谓了如指掌。因为上一世,他就盗蛊、搜魂、吞窍一整套上去,直至将羊枯的仙窍并入自身当中来。

短短片刻,方源就将羊枯的仙蛊扒光,在智炼大阵的作用下,迅速炼化,成为了自己之物。

方源宙道分身取走摄魂仙蛊,当即灌注仙元,和其他蛊虫一同交替催用,使出摄魂杀招。

这摄魂杀招也是经过改良,比较上一世威能更盛数筹。

方源狠狠一摄,便将羊枯的魂魄摄取出来。

羊枯作为魂道蛊仙,魂魄完全脱离梦境,当即就要苏醒。但方源早有准备,一记魂道杀招打上去,将其封存。

取走仙蛊,羊枯宛若猛虎拔牙。

摄魂而出,羊枯这头无牙的病虎,已是病入膏肓。

“最后一步,取窍!”方源轻喝一声,催动杀招,直接将羊枯肉身上的仙窍拔取而起。

仙窍被方源强行封印,形成一颗淡灰光团。

白凝冰等三位蛊仙见此缓缓飞近。

方源宙道分身谨慎地将这团灰光交到白凝冰的手上,并道:“去吧。”

一切都已推算过。

白凝冰等三仙双手捧着羊枯的仙窍,直接飞下,来到小南疆。

照准位置,白凝冰轻轻一抛,将灰白光团抛入某处山峰一侧。

旋即,天地二气剧烈涌动,灰光炸开,越变越大。灰光外围大地开裂,山峦平移,发出震耳欲聋的巨响。

灰色光影越变越大,最终彻底还原出羊枯的仙窍本貌。

“注意了。”白凝冰冷喝一声,身旁的两仙早已严阵以待。

嘎嘎嘎!

先是一大群的花翼白骨鸟,发出乌鸦般的叫喊,争先恐后地向外飞出。

随后,大地震动,一股兽潮从羊枯福地发源,向小南疆的地界蔓延而去。

白凝冰等人一脸郑重之色,毫无慌乱之情。

早在之前,方源就推算出了一切,并且安排了极其详细的应对计划。

三仙旋即行动。

白凝冰很快就驱散了兽潮。

而花翼白骨鸟群也被一位异人蛊仙成功牵引,带去其他方向。

羊枯福地原本自成一届,万物和谐共存,但如今并入小南疆后,原本被羊枯精心设计的平衡被打破许多,造成了一系列的混乱。

若是寻常蛊仙,必定手忙脚乱。

但方源却从容不迫,利用深厚的智道造诣以及蛊仙下属出力,他将局面维持住。

混乱只是一阵子,便徐徐消散了。

三仙留下一仙收拾残局,而白凝冰和一位异人蛊仙则重新飞向小黄天。

他们归去的途中,正看到又一队的蛊仙,同样是三人,捧着一个粉色光团,向小南疆落去。

大盗鬼手、智炼大阵、摄魂杀招、取窍手段……

盗蛊炼蛊、身魂分离、吞并仙窍、镇压混乱,方源将一切安排得井井有条,做得游刃有余。

主要是有上一世的经验,让方源对俘虏的情况十分了解。

仅仅数天后,方源就几乎将所有的俘虏都一网打尽,只留下巴十八、刘浩二人。

一条巨大的乌青墨石矿脉,被挪移到石人的栖息之地,就已经让石狮诚惊喜若狂。

小南疆中,直接多了数十座山。

一些山峦,更是达到南疆的名山级数。

内景雨山、铜印山、骨鸟峰……

石人一族的太上大长老石宗肚子里一个劲地感叹:“果然是杀人放火金腰带!”

因为是南疆蛊仙,所以一半的资源都集中在了小南疆里,另外一半的资源,都分布到了其他小四域以及小九天中。

比如小中洲,多出一片虚影花的花园,海量的仙农土铺设出了一片肥沃的平原。

又比如小北原中,多了一具天柱风。

还比如小东海中,有了一片全新的海域,名为浪花海域。

再比如小西漠中,栽种了十多株的扶桑树,这种树的每一根树枝都仙材。

最重要的一个收获,是夏槎洞天中的年华池。

上一世,方源不知道夏槎拥有年华池,自己费尽辛苦制造了一座。

有了两座年华池,并不怎么样,造成了资源的严重浪费。方源从南疆正道那里勒索来的资源,大部分都投入年华池的建设之中了。

这一世,方源直接把夏槎的年华池占为己有。

方源一直感到头疼的年兽、上古年兽、太古年兽,方源都将它们塞入到年华池中去。

群仙心潮起伏,经历了一波波的震撼后,他们已经麻木了。

方源巧取豪夺,几乎将南疆蛊仙的一生积累,毫不浪费地全数吞下。

“方源大人的境界真是全面,吞并多种流派的仙窍,轻松至极。”

“他的这片仙窍洞天更是伟大浩瀚,一连吞并这么多的仙窍,却还是如此广博稳定!”

“这、这简直是屠猪杀狗。这些蛊仙强者落到方源手中,就好像是一只只鸡落到屠夫手里,被一步步割喉、放血、拔毛、清洗、切割。真是……”

群仙咂舌,有些人都说不出话来。

方源这次干脆利落的吞窍行径,让他们心中生寒。

仿佛蛊仙落到方源眼前,就不是蛊仙了,而是鸡鸭猪狗,任凭宰割。

即便是八转蛊仙也不例外。

这带给群仙的震动,尤其是对那些异人蛊仙,远比之前埋伏大战成功还要强烈数倍!

------------

\end{this_body}

