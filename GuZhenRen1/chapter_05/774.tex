\newsection{小魔尊}    %第七百七十七节:小魔尊

\begin{this_body}

“原来天庭的八转蛊仙,都栽在了琅琊福地里!”

“难以置信,雷鬼真君可是传奇人物,当年抗衡过幽魂魔尊。而陈衣更是元莲派的上一任太上大长老,掌握着元莲真传啊。”

“这两大人物居然死在方源的手中?难以想象!”

“不只如此,方源这魔头根本没有消停,又祸害了南疆蛊仙界。南疆的夏槎、巴十八以及一干七转强者,都被他俘虏了。”

“他怎么会有这么强的实力?明明只是七转啊。”

“对付天庭,他动用的是琅琊福地的底蕴。而对付南疆,他借助了一座超级宙道大阵。”

“这里面蹊跷太多了,到现在我都难以置信。”

“应当是真的。天庭和南疆都没有反驳!”

“方源……这个魔头越来越可怕了。这才多久,他就伏杀了八转强者,更俘虏了一大群七转、八转。”

“哈哈哈,南疆正道的脸面丢的太大了。”

“也不能这么说吧,就连天庭都栽在方源的手里。”

“方源明明只有七转,居然手段如此超凡?!”

“我算是看明白了,方源这样的人物绝不能以常理推算他。”

“他是有魔尊的气象了!历代的魔尊年轻时,不都是像他这样吗?越级挑战是家常便饭,强杀八转显得轻而易举!”

“嗯,很有道理……”

“可惜将来只有大梦仙尊,而没有魔尊啊。这不是方源的时代。”

“这你就错了,天庭的宿命蛊可是被红莲魔尊破坏了的,未来什么样,说不准!”

“方源七转就已经如此凶恶,罪孽滔天,等到他八转那还了得?”

“不错。趁他现在还是七转,对他下杀手,这是铲除他的最后良机了。”

“难!天庭不都没有找到他的踪迹么……”

“方源不仅凶狠,而且狡诈。他是重生的天外之魔,手段太多了。寻常强者找他麻烦,就是找死。就算是八转蛊仙也要多加小心,防止落入他的算计之中。”

“但若真的杀了方源,那收获可就太大了。”

“是啊,方源手中掌握的尊者真传,还有无数的资源,春秋蝉、定仙游种种仙蛊还有数座天地秘境……完全媲美一个超级势力!”

五域蛊仙界一片哗然,方源成了无数蛊仙交流、讨论的中心。

蛊仙们对他又恨又惧又嫉妒,情绪相当复杂。

如果说,方源之前逃脱雷鬼真君的追杀,并且将雷鬼真君的胸骨挂在宝黄天中,是名传天下。

那么现在,手底下染了八转强者鲜血,还俘虏一大票蛊仙强者的方源,真的是名震仙界!

他之前靠着逆流护身印,勉强和八转蛊仙放对,称得上是七转第一人。

但现在,不只是六转、七转蛊仙,就连绝大多数的八转都对方源十分忌惮,没有一丝的轻视之意。

一个关于方源的最新称号,渐渐被蛊仙们达成共识——小魔尊!

这个称号代表着方源具有成就魔尊的潜质和气象,可以说是成长中的未来魔尊。

这个评价是相当的高。

人族历史,古往今来,蛊仙强者如群星闪耀,其中又有多少尊者?

前前后后,不过只有十位罢了。

现在,蛊仙界公认方源有着跻身尊者之位的希望!

当然,除了针对方源的讨论之外,天庭也没有逃过蛊仙们的争议。

方源暴露出来的星投杀招,让其他四域蛊仙无不震骇。

尤其是那些正道超级势力,简直是如坐针毡。

星投杀招具有极大的战略意义,能够让中洲和天庭秘密调遣数位蛊仙,突然发动,集中打击。

这一招让天庭牢牢占据主动。

其他四域的八转蛊仙,几乎都是一方首脑,总不可能时时刻刻抱团一起,来防备星投杀招吧?

如此一来,就给了天庭运用星投,集齐八转战力以多打少的良机。

天庭中,紫薇仙子面沉如水。

即便她早有预料,等事情真正发生,她也难免有怀心情。

紫薇仙子不想让方源好过,方源同样没有让天庭好过。

这场公开的惨烈的撕逼大战,双方都是失败者。

而要说遭受打击最大的人,很可能就是古月方正。

他刚刚晋升六转,初步掌握了蛊仙的伟力,正是信心上涨之时,忽然听到这样的噩耗!

方源居然能够坑害天庭,甚至连南疆正道势力都拿他没有办法!

方源只是七转修为,但违背常理,八转蛊仙都不是他的对手!!

“我……”古月方正第一次听到这个情报,心里有许多话想要涌出来,但又说不出来。

好半天,他才脸色灰白,自言自语:“我……我真是太天真了。”

他苦笑:“方源……这才是你我之间,真正的差距吗?”

这可怜的娃,还不知道方源已经晋升八转的事实。

事实上,全天下似乎都被方源骗住了。

不过这也难怪!

至尊仙窍的秘密,目前连紫薇仙子都不知晓。

方源若要晋升八转,必定会渡劫,不管成败如何,都会留下重要的线索。

而且他若真的成了八转,这速度未免也太快了!其他蛊仙兢兢业业、如履薄冰的悠长的修行年岁,岂不是都仿佛活到狗身上去了?

方源关注着舆论。

“居然有这么一天,我的风头盖过了天梯?”

方源有些哭笑不得。

似乎其他蛊仙更关注他远超过天庭,但他没有失望。

因为方源知道,其他四域的超级势力一定都在商讨着如何对付天庭!因为很早之前,就有先贤意识到了未来的变化,比如南疆的姚家。而现在地脉震动,大变就在眼前,绝大多数的超级势力都知道到了界壁的消失,五域乱战的未来。

“中洲炼蛊大会,天庭,咱们走着瞧!”

方源关注舆论,也是在探索墨水效应。

他知道,研究这方面的变化,对于他对命运,对万物,对世界的了解,都大有裨益。

“回顾一下,重生以来,就直接吞并了琅琊福地,这第一步走的太对了。”

“一步先,步步先。优势越来越大,仿佛是滚雪球一般,这就是重生的优势!”

“当然,也有我身为完整的天外之魔的便利。”

“更有红莲损坏宿命蛊的功劳!”

和上一世相比,好的方面比比皆是,但不利的墨水效应也有不少。

首先,南疆的律道八转强者巴十八被方源俘虏了,未来也会被吞窍,不得善终。

这就意味着,将来武庸率领的南联攻击中洲,就少了一位强大的臂助。

上一世的大战中,巴十八可是力抗天庭蛊仙清夜,乃是方源一方的中流砥柱之一。

而天庭方面,因为方源取走了五界真传,破坏了五界山脉,也就不存在五界山脉埋伏战。

天庭的成员君神光,上一世是在五界山脉大战中被方源俘虏的。这一世他仍旧自由自在。

除此之外,天庭还多了另外一位八转蛊仙,那就是凤仙太子。

“宝黄天中,我的龙鱼生意也是遭受了剧烈的打击,收益被压制到了谷底。”

上一世虽然也是如此,但这一世紫薇仙子被方源刺激到了,天庭打击的力度强了至少十倍!

龙鱼生意遭受重创的时期,也被大大地提前了。

“好在我吞并了这么多的仙窍,增长了大量的经济支柱。”

“不过上一世,紫薇仙子就和南疆正道合作,一起来针对我的这些经济命脉。”

“不晓得这一世,会不会提前?他们打击的力度会不会加重?就像龙鱼生意这样?”

方源揣测着,觉得可能性颇大。

还有一个可预见的有害影响,就是长生天。

上一世,长生天吞并了琅琊福地,千金买马骨,导致声誉大涨,顺利收编整个北原蛊仙界,让冰塞川等人奋而入侵天庭。

而这一世,不仅琅琊福地被方源吞了,长生天没有捞到手,而且凤仙太子也回归天庭,北原正道出了这么大一个内奸,让黄金血脉脸上无光,长生天更是难逃,遭受牵连。

外界吵的天翻地覆,都动摇不了方源。

他蹲在角落里闷头发展,时不时转移位置,让敌人始终捉摸不透。

魂道修行进展神速,但距离分魂,还有一段小距离。

方源的宙道分身主要负责推演,而他本体则苦练杀招!

夏槎的杀招他都很熟练,这是上一世的财富,重生跟着他带来。

方源重点修炼陶铸真传中的内容。

仙道杀招——五禁玄光气!

蛊虫盘旋,仙元灌溉,一道恢弘的光气喷涌而出。

它呈现紫黑色,烟气的形态更多于光,让人第一眼看了就联想到南疆的瘴气界壁。

正是五禁玄光气中的南禁紫黑瘴光气!

其余四种则分别是中禁圣白金光气、北禁甘草碧光气、西禁狂炎赤光气、东禁沧水蓝光气。

方源才刚刚上手,目前只能施展其中一种光气,还不能同时催发出两道来。

至于另外的五界大限阵,因为缺少地脉仙蛊,方源暂时不练。

按照方源的记忆,地脉仙蛊此刻还未出世呢。

并且就算有了地脉仙蛊,方源还要推算改良五界大限阵,增添阵灵、阵旗蛊进去。8910

\end{this_body}

