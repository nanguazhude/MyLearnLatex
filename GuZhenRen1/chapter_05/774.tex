\newsection{名震天下!}    %第七百七十六节:名震天下!

\begin{this_body}

一座高大的水晶山峰,矗立在小南疆中。

它通体粉红,散发着梦幻般的光辉,正是《人祖传》中记载,大名鼎鼎的天地秘境之一——荡魂山。

此刻的荡魂山,已经是几乎恢复了旧观。

并不是彻底修复的。

原因和上一世相同,方源还顾忌着天庭,唯恐对方有着类似定仙游杀招的手段。

当然,恢复到这种程度已经无比接近完整,荡魂山的胆识蛊产量下降得微乎其微,方源完全可以接受。

石人六转蛊仙石狮诚来到山前。

尽管他早已知晓方源手中掌握着胆识蛊,但当他亲眼仰望着这座传奇山峦,仍旧激动得不能自已,眼中迸射出无以伦比的激情!

“荡魂山、胆识蛊……”他口中呢喃。

胆识蛊对于石人一族的意义,非同小可。

石人一族都是雄性,身躯高大坚猛,对魂魄负担很重。一旦石人活动过盛,就会损害魂魄。因此石人一生中,会拿出八成的时间睡觉,用来休养他们的魂魄。

老石人死了之后,分散出来的魂魄和石头凝结,就会形成新的石人,继承老石人的部分记忆,以及一些重要的经验。或者老石人睡得久,魂魄底蕴积累到一定程度,主动割舍部分出来后,也能形成小石人。

这种独特的繁衍方式,让石人的魂魄强度成为了关键。

石人魂魄越强,就能繁衍出越多的石人。

然而石人属于异人,天生拥有土道道痕,和魂道互斥。能够增长魂魄强度的魂道手段,对于石人而言,能用的很少。

胆识蛊就不一样了,它是天地秘境所生,增长魂魄强度,虽是凡蛊对蛊仙都有大用。石人用了,效果非凡。当年方源刚刚执掌狐仙福地的时候,就大肆栽培过石人。

石狮诚看到荡魂山,就是看到了整个石人一族崛起,并走向强盛的通天坦途,他怎么能不激动?

他心潮起伏,越发觉得跟随方源,对于他自己对于石人一族,都是极其正确的选择。

石狮诚的表现都落到方源眼中。

对此,方源暗自满意。

此次搜刮胆识蛊,方源故意派遣石狮诚过来,就是为了利诱,让他感受到前途是无比的光明,只要紧紧跟随自己,好处是受用不尽的。

石狮诚终于心境稍稍平复下来,开始主持大阵。

方源有着阵道、魂道造诣,又不缺仙蛊,并且当下的魂道修行乃是重中之重,便干脆在荡魂山周围一圈,布置了一座仙级大阵。

石狮诚得到方源的认可后,能够操纵大阵。

大阵轰鸣,发出一道道灰蓝的光,光芒如雾似烟,很快就蔓延到荡魂山上。

荡魂山轻轻颤抖起来,粉色水晶山峦也发出蒙蒙之光,将灰蓝的光雾迅速浸染成粉色。

粉色顺着光雾,很快染遍整个大阵。

大阵中囚禁着大量的魂兽,还有魂核的库存。

魂核、魂兽在粉色光雾下,迅速消散,悄无声息。

大阵运转片刻,一颗颗的胆识蛊凝聚而出,很快就从个位数,积累到了四位数。

石狮诚的脸上顿时流露出震惊之色。

“这是何等的效率!”

“并且胆识蛊的数量还在增长,好快,就要突破上五位数了。”

这道魂道大阵,方源是以幽魂真传中的内容,再结合了自身手中的种种仙蛊,推算改良而成,取名叫做销魂万胆大阵。

一次产量,至少是一万颗的胆识蛊!

这绝对比单纯运用荡魂山,要高效得多。

当然,一次产出上万只胆识蛊的前提,是魂兽、魂核足够。

在这个方面,方源早有了准备。

那就是之前他的布局落子,他已经和房家合作,共同开拓西漠的青鬼沙漠。

这个途径获取魂兽,不仅量大,而且稳定。

哪怕方源对魂兽、魂核的需求是如此恐怖,青鬼沙漠这条线仍旧可以满足他,并且游刃有余。

这座沙漠太过广大,魂兽无数,这里便是魂道的超级资源点!

可惜依靠沙漠的,却是专修阵道的房家。

并且青鬼沙漠中藏着青仇,这是一个巨大的隐患。只要有它在一天,青鬼沙漠就藏有致命威胁,不是善地。

方源甚至没想过在宝黄天收购魂核,或者向南疆的羊家勒索。若是这样做,他就会泄露出一些重要的情报。

上一世,方源修好荡魂山,还要在很久以后。并且荡魂山修复好了,仙元不够,魂兽、魂核也严重不足,拖慢了太多的节奏。

这一世,方源不仅提前修好了荡魂山,并且魂兽、魂核方面的供应完美地跟上,两相搭配,魂道底蕴暴涨是板上钉钉的事情。

方源把魂魄遁进来,直接钻进销魂万胆大阵。

大阵运转,帮助他汲取胆识蛊,几乎一瞬间方源已经同时服用了上万颗的胆识蛊。

胆识蛊的效用仍旧那么明显,方源的魂魄吹气球般暴涨开去,短短几个呼吸,就涨成了一个小小巨人。

方源魂魄便飞出大阵,来到旁边不远处的落魄谷内。

迷魂雾笼罩他的魂魄,令魂魄松散开来。

落魄风吹割他的魂魄,带来无数的伤口和无以伦比的痛楚!

痛痛痛!

这种痛苦,根植于灵魂深处,比凌迟还要痛苦万倍,让方源几乎要发狂。

但他拼命忍受住,片刻后,他拖着残破不堪的魂魄,缓缓飞出来。

他又飞回魂道大阵,借助胆识蛊修复,随后又是体型暴涨。

胆识蛊就像是绝世完美的营养,而落魄谷的苦修,让方源的魂魄如同锻铁成钢,杂质被尽数剔除。

如此,三轮之后,方源的魂魄底蕴暴涨到了千万人魂!

把一旁操纵大阵的石狮诚,看得瞠目结舌,震惊至极,乃至不敢相信。

荡魂山、落魄谷乃是魂修圣地,这美名绝不是盖的。

方源暗松一口气,心道:“总算是缓过来了。”

之前他魂魄底蕴消耗得很快,差点就要逼得他动用其他魂道手段。

这些魂道手段虽然也很优异,但都有后遗症,绝没有荡魂山、落魄谷这般完美无瑕。

打个比方,这就好像是用其他手段增寿,和用寿蛊增寿的差距。

“我的老天呐……”石狮诚发出一声呻吟般的惊叹。

他终于反应过来,接受了这个梦幻般的事实。

方源魂魄底蕴像是嗑药般暴涨,几乎粉碎了石狮诚长久以来的世界观、人生观、价值观。

他的激动之情无法用语言来表达,手指头都在剧烈颤抖。

“只要依附方源大人,牢牢跟紧大人的脚步,我们石人一族将重新攀上世间巅峰!”

“有这样的两座魂道圣地,就是超越五域仙界的底蕴啊。”

石狮诚却不知晓,方源心里还尤不满足:“可惜,我没有安魂汤可用。否则的话,魂道修行的效率将更快更恐怖!”

智道修行三要素有念意情,魂道修行也有三要素,分别是壮魂、炼魂、安魂。

前两者的世间极致的修行方式,就是荡魂山、落魄谷。而安魂首选迷魂湖中的安魂汤,这同样在《人祖传》中有所记载。

魂魄修行到一定程度,会渐渐和肉身脱离,轻易间就能飘出魂魄(一部分)来,造成魂不守舍,魂不附体的修行事故。

这个时候,喝下安魂汤,能够完美解决这个弊端。

方源没有安魂汤,只有依靠其他手段弥补,这和前两者的修行方式完全不在一个档次上。

这也导致方源只能每隔一段时间,修行一次魂道。虽然效率已经是惊世骇俗,差点把石狮诚都吓傻了,但若是有了安魂汤,几乎是可以不间断地修行下去。

欲速则不达,方源停下只好暂时魂魄的修行。

宝黄天中传来动静。

天庭方面散布流言,说方源修为已经逼近八转,这几年时间很可能就是他渡劫,晋升八转的最后时间。

除此之外,方源俘虏了南疆蛊仙的情报,也被天庭散布出去。

方源冷笑:“这应当是紫薇仙子的手笔了。”

上一世,紫薇仙子就这么做过,不过是在之后的时间里。

这一世,方源步步为营,导致紫薇仙子承受更大的压力,所以提前动用了这个手段。

老实说,这个毒计真的很厉害。

紫薇仙子因此能轻松借力,让天下所有的蛊仙都忌惮方源,唯恐他真正晋升八转,从而围攻和针对。

上一世,五行山脉大战,方源就是将计就计,利用了蛊仙们的这个心理。

“若非我有至尊仙窍,很可能就遭受紫薇仙子的算计,落入她的节奏中。”

“现在她散发这个流言,应当是还未发现我至尊仙体的秘密了。”

“当然,也有可能是她已经知道,却装作不知,散布出来的消息,只是一套阴谋诡计的前序。”

方源开始运转智道手段,进行深思熟虑。

片刻后,他眼中冷芒电闪,做出了一个新的决定。

“既然如此,那我也不能坐以待毙,干脆也把你们的星投杀招,还有陈衣、雷鬼真君之死爆料出来罢。”

紫薇仙子借力,方源也在借力。

比起方源,天庭要更令四域蛊仙忌惮、恐慌。

上一世,天庭炼蛊能引出四域齐攻,方源提前爆料,有着极大的推动作用。

“这一次,就彻底彰显出我的实力来,反正南疆的事情是瞒不住的。”

“既然天下会因此震动,那就更干脆一点,让他们更怕我一些好了。”

“如此一来,也好方便我接下来勒索南疆正道。”

果不其然,方源和天庭再度公开撕逼,让无数蛊仙为之惊呼。

消息如飓风般,很快传遍整个蛊仙界。

天下震动!

记住手机版网址:m.

------------

\end{this_body}

