\newsection{大梦仙尊凤金煌}    %第四百二十一节:大梦仙尊凤金煌

\begin{this_body}



%1
北原,琅琊福地,炼蛊大厅。

%2
炼道蛊阵徐徐运转,原本幽蓝的冰焰,已经彻底化为一座冰山。

%3
方源站立在炼道仙阵的边缘处,举起双手,用掌心紧紧地贴着冰山的表面。

%4
他双目炯炯发亮,全神贯注地进行炼蛊,几乎全部心神都投入进去。

%5
在他的身上,各种蛊虫的气息时起时伏,有凡蛊气息,也有仙蛊气息。正是他运用着大量的炼道手法,乃至杀招。

%6
在方源的努力之下,冰山中仍旧有蓝色的冰焰,正在静静地燃烧。

%7
蛊材在火焰当中,不断地被提炼,杂质都炼化,只剩下最精华的存在。

%8
一只仙蛊的雏形,一点一滴地开始成形。

%9
一旁站着的毛六,双目闪烁着喜悦的光辉。

%10
这样的进展,已经超越了他之前的记录。

%11
这当然不是因为,方源的炼道造诣比毛六要深厚,而是因为运气。

%12
更主要的原因,是炼道的方法。

%13
“想不到本宗,也收录了远古时代炼道大能北洛的传承。”毛六心中感慨不已。

%14
北洛乃是名垂青史的炼道八转蛊仙,他最擅长冰炼之法。直至今天,他的冰炼造诣,都是炼道后辈仰望的高峰。

%15
方源此时用的,正是当年北洛的冰炼杀招。因此一举超越了毛六之前的记录,距离炼制净魂仙蛊,只差一步之遥。

%16
毛六虽然也是分魂,是影宗的一份子。

%17
但是当初,影宗利用魂穿杀招,将其安插在琅琊福地里,并没有给予他太多的修行记忆。

%18
因为这些记忆一旦多了,就非常容易在修行过程中,露出马脚来。

%19
而在琅琊福地当中,琅琊地灵几乎无所不知,能体察到福地中任何角落的每一个微小变化。

%20
现在方源成为影宗之主,继承了整个影宗的传承,蛊修的底蕴得到了爆炸式的增长。

%21
这种增长的程度,堪称翻天覆地。就连方源自己,都有些麻木。

%22
打个比方,若是以前,方源的底蕴算作一个湖泊,那么影宗的传承就如同一片海洋。

%23
即便是琅琊派的底蕴,都次于影宗。别看琅琊福地建立的时间非常的悠久,但是源头只是八转蛊仙长毛老祖。而影宗的创建者,却是屠戮天下的幽魂魔尊!

%24
所以,当方源深入了解到了净魂仙蛊的炼蛊步骤之后,他就从脑海中海量的传承中,选择出了最适合他炼蛊的方法——北洛冰炼法。

%25
“好。仙蛊雏形已经成型了一大半,成功近占咫尺了!”

%26
时间不断推移,毛六越发激动起来。

%27
他将目光又移动到方源的身上。

%28
“这个男人……”毛六的眼底闪过一抹复杂的光芒。

%29
“在炼道方面,底蕴很扎实,而且很有天赋。最关键的是,冰炼之法似乎特别适合他!”毛六在心中评价。

%30
炼道手法千千万万,火炼、冰炼、多人合炼、宙炼等等,有些人适合火炼法门,有些人天生和水炼方法极其切合。

%31
方源也发现,冰炼之法挺适合他。

%32
“看来闲暇之余,我也可以多多练习一下冰炼的法门。似乎比血炼还要切合我的性情呢。”

%33
一边想着,方源一边调动蛊虫,酝酿出一记仙道杀招。

%34
在他的仙窍当中,早已经存了大量的万我力道虚影分身。

%35
这些分身,在他炼蛊的时候,帮助他组建杀招,非常实用和便利。

%36
这记仙道杀招催出来,方源顿时感到一股温热,萦绕在浑身上下,抵御着外界扑面而来的刺骨寒意。

%37
强烈的寒意,不断地从冰山上传来。这是冰炼法门的弊端,蛊仙必须一边抵御寒冷,一边炼蛊。

%38
若是万我没有提升,方源此时必然捉襟见肘,难以兼顾各个方面。

%39
但有了全新万我,这道难关他轻轻松松地就跨越过去了。

%40
就在方源继续炼蛊的时候,远在中洲的凤金煌,攀上了一座山峰顶端。

%41
黎明时分,天空中还显着黑,晨星寥落。

%42
凤金煌哈气成雾,眺望远方,小脸上却满是愁容。

%43
曾经的她,少年不知愁滋味,有着蛊仙双亲庇护。如今却是品尝到了挫折,体会到了人间辛苦。

%44
尤其是最近,赵怜云成功地晋升成仙。

%45
这大涨灵缘斋中倒凤派系的声威,而凤九歌远在西漠,白晴仙子独木难支,让徐浩等人很是赢回了不少场面。

%46
凤金煌从中,体会到了以前没有的艰难和辛酸。门派在她的眼中,又有了全新的形象。

%47
“唉!”她盘坐在一块山石上,叹息一声。

%48
她又想到赵怜云。

%49
这个人后来居上,抢走了凤金煌一直当做奋斗目标——灵缘斋的仙子之位。如今又成功升仙,从此仙凡两别。

%50
曾经赵怜云跪求凤金煌的情景,仍历历在目。

%51
“赵怜云是天外之魔,方源也是天外之魔呢。为什么我总会输给天外之魔?”凤金煌思绪发散,从内心的最深处浮现出那至死难忘的一幕。

%52
就在那荡魂山的峰巅,她第一次见到方源的景象……

%53
凤金煌顿时脸面发热,她连忙甩头,将这些景象和记忆又压入内心最深处。

%54
“想这些有什么用?”

%55
“凤金煌啊凤金煌,你当发愤图强!你要更加努力,将来成就蛊仙,帮助爹娘才是。”

%56
凤金煌收拾心神,开始总结最近一次的梦境经历。

%57
那是炼道大能北洛的梦境。

%58
“通过这个梦境,我的炼道境界,已经突破到大宗师一级了。北洛前辈最擅长的就是冰炼之法,不如我试试看。”

%59
凤金煌伸开双手,十指频动,掀起层层叠叠的手指虚影。

%60
一股寒冰之气,顿时在她手心中盘旋飞绕,并且越加旺盛。

%61
凤金煌乃是炼道大宗师,境界卓绝,很快就有一只蛊虫,从寒冰气团中一跃而出。

%62
这是一只冰道的三转凡蛊,飞出来后,很快就落到凤金煌的肩头,乖巧至极,一动不动。

%63
随后,又有第二只,第三只蛊虫飞了出来。

%64
并且,转身也从三转,逐渐上升到四转、五转。

%65
凤金煌脸色涌现出一抹喜色,双眼逐渐放光,手上动作越来越快。

%66
但下一刻,寒冰之气猛地一炸,炼蛊失败,凤金煌惨遭反噬!

%67
“糟糕!”凤金煌心中一沉,面色发白,下意识紧闭双眼,咬牙承受着即将到来的伤害。

%68
但没有动静。

%69
“怎么回事?”凤金煌缓缓地睁开双眼。

%70
然后,她就见到一位高人,将一只手伸在她的眼前。那团原本即将爆炸开来,居然在他的手中,被凝聚成一个冰球。

%71
高人真的很高,凤金煌的头顶之略高于他的膝盖。

%72
他长发飘扬,身姿雄威,目光沧桑,蕴藏智慧。最惹凤金煌注意的,是他额头上,竟长着一对珊瑚龙角。

%73
“前辈是谁?谢谢你救了我。”凤金煌连忙行礼。

%74
高人微笑道:“炼道法门千千万万,冰炼法却是不适合你的,凤金煌。你的性情活泼明朗,适合金炼、火炼。至于冰炼,往往擅长之人都有隐忍坚毅的秉性。”

%75
说着,高人轻轻地一握拳,将掌心中的寒气都直接捏灭。

%76
凤金煌看得咂舌,她深知这团寒冰之气的威能,绝对能让一般的五转蛊师重伤。

%77
凤金煌双眼骤亮:“前辈你是蛊仙吗?你认识我的爹娘?”

%78
凤金煌也接触过不少蛊仙,因此并没有不自在的感觉。一般来讲,那些蛊仙都很和颜悦色,大多是仰慕凤九歌的威名。

%79
果然,高人点点头:“我当然知道你父母,不过我此次来,是专程为了你。”

%80
“为我?”凤金煌诧异。

%81
“不错。”高人继续道,“凤金煌啊,你会是未来的大梦仙尊,是注定入主天庭,领袖这个时代,惊艳天下,傲立蛊仙之巅的伟大人物!我乃是天庭蛊仙龙公,这一次来,是专门收你为徒,引领你登上九转尊者的宝座。”

%82
“啊?!”凤金煌张大嘴巴,呆呆地看着眼前的高人。

%83
龙公对着凤金煌微笑,神情祥和可亲。

%84
凤金煌却用“我非常怀疑你是骗子”的眼神,看着这位击败了幽魂的传奇人物。

%85
她并不认识龙公。

%86
龙公的形象,为中洲十大派的蛊仙熟知,但是凤金煌毕竟是凡人,隔着一层。

%87
仙凡之间,是两个世界。

%88
“大叔,你知不知道,古往今来,能够成为蛊尊的就只有十人而已。大梦仙尊的预言我是知道的,但凭什么我就是大梦仙尊?”凤金煌问道。

%89
“你不要怀疑你自己。”龙公笑着道,“你就是未来的大梦仙尊。在你幼年时期,就有梦道仙蛊来投,这就是征兆!你的一生,原本会一帆风顺,光彩夺目。遗憾的是,这中间出了一些岔子。”

%90
“那些不在宿命中的天外之魔,抢夺走了你的机缘。”

%91
“比如狐仙福地,又比如灵缘斋的当代仙子之位。”

%92
“不过这些,都不要紧。”

%93
“宿命仙蛊会在十年之后,彻底修复。你必将成为大梦仙尊,因为这一切都是命中注定!”

%94
龙公说到这里,神情非常郑重,目光坚定无比,仿佛说的就是铁一般的事实。

%95
任凭任何存在,都动摇不了,改变不了。

%96
凤金煌也不由地被感染了,一时间愣住,说不出话来。

\end{this_body}

