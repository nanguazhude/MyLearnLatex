\newsection{本命仙蛊狗屎运}    %第八节:本命仙蛊狗屎运

\begin{this_body}

“将狐仙、星象两福地搬迁,这后方也算是平定了。”完成这件大事,方源不由地吐出一口浊气,心中放松了好些。

他未虑胜,先虑败。此举已成,不管影无邪那边如何应对,方源的损失已然大减。

说来也是有趣,阴差阳错之下,他和影无邪的身份相当于对调了一番。

方源矢志永生,原本肉身对他而言,无异一具皮囊,取舍之间毫无留恋之情。他在乎的只有三个方面。

第一,是原本肉身上的蛊虫。

第二,是狐仙福地、星象福地中积累而得的修行资源。

第三,则是黑楼兰、太白云生、琅琊地灵、灵缘斋、仙鹤门等等关系。

方源望了望漆黑的夜空,山顶的风刮得他栖身的松树,树枝摇晃。

他心中暗忖:“按照这个时节,定仙游应当已经自毁。接下来,则是其他蛊虫接连自毁。不知道影无邪会做何应对?”

“这些蛊虫我取回的希望,已经微乎其微。狐仙福地、星象福地的资源已经被搬空。狐仙福地落于中洲蛊仙之手,而星象福地则藏在地渊深处。如今就只剩下第三方面,黑楼兰、太白云生等人了。”

方源细心思量。

这其中,中洲灵缘斋、仙鹤门的关系,已经宣告破裂。

方源捣毁八十八角真阳楼的真相,已经大白天下。天庭追究下来,肯定要找方源麻烦,影无邪顶着方源的肉身,从某种程度上而言,就是为方源挡灾了。

所以,这些方面不足为虑。

黑楼兰、太白云生,实是方源心忧之处。

黎山仙子、焚天魔女身亡,黑楼兰的利用价值大减,但她仍旧是十绝蛊仙,大力真武体。再加上五百年前世的灵缘斋当代仙子身份,身负强运,又是枭雄,绝对不可小视。

太白云生身上。虽然只有两只仙蛊,但每一只都是极品蛊虫,用处极大。

方源极其担忧,这两人会被影无邪利用了去,与自己为难。

所以。联系了狐仙福地、星象福地之后,方源就紧接着,试图通过宝黄天,沟通黑楼兰、太白云生二人。

他们在宝黄天中,也留有意志。

这本就是早早预留下来的沟通手段。

但方源沟通了这两人在宝黄天中的意志,却仍旧未真正联系到这两位。

只有当这两人,主动沟通了宝黄天,这些意志回归本体,进行了交流之后,才能获知真相。

没有联络得上。方源的眼中更显忧色。

无奈之下,他暂且放弃尝试,转而联系琅琊地灵。

这一次,几乎是立刻得到了回应!

“好小子,你总算是联系我了!!”琅琊地灵显得十分激动,大叫大嚷起来。

说实话,他答应方源帮助他搬迁时,主要还是顾念方源已经加入琅琊派的身份。

等到他发现荡魂山、落魄谷,尤其是智慧蛊之后,他彻底惊呆了!

用一个词来形容。那就是震撼。

他万万没有料到,方源的底蕴居然会如此深厚。

深厚到简直让他难以置信的程度。

听到琅琊地灵的叫声,方源抿嘴一笑,仿佛看到了一个浑身都是长毛。仿佛猩猩的地灵,在地上活蹦乱跳,手舞足蹈,表情夸张的样子。

琅琊地灵源自长毛老祖,十分特殊,乃是双执念所成。

现在的这个执念。富有侵略性,一心想让毛民当家做主,成为五域之主,将人族镇压下去。

比之前仙风道骨的那个,要更加急躁。

但不管哪个,只要是地灵,都不会说谎话骗人,都是直肠子。方源和地灵相处,远比和那些知人知面不知心的人族蛊仙,要轻松容易很多。

见方源不说话,那边的琅琊地灵果然忍耐不住,又继续叫道:“方源,琅琊派这次可帮了你的大忙。要不然你损失了这些,可不要一头撞死吗?你说,你拿什么回报我呀?”

顿了一下,方源又听到琅琊地灵道:“你小子赶紧把这些都献给门派吧。我绝不会亏待你的,你可是咱们的客卿太上长老!我会给你算上海量的门派贡献。足够你换取数不胜数的仙蛊方,还有仙蛊!”

方源既然请琅琊地灵出手,自然早有心理准备。任谁来见到荡魂山、落魄谷,还有智慧蛊这三样,都会动心。即便是琅琊地灵,也不能免俗。

琅琊地灵虽然是直肠子、一根筋,但并非无智死物。

当即,方源从容回应道:“我可是琅琊派的客卿太上长老,这些都是我的。大长老,你不会想要贪墨我的这些东西吧?”

“啊啊啊!”琅琊地灵揪着头发大喊大叫,“早知这样,我就不要你参加琅琊派了。这样,我就可以直接没收了你这些无上珍宝!该死,我又讲真话了。”

方源哈哈一笑。

当初,琅琊地灵既然是要方源加入琅琊派,自然是真心实意的。

现在双方已经定下约定,琅琊地灵想要反悔,已经晚了。至少单凭他自己现在的手段,想要接触当初的入派盟约,已经是有力不逮。

即便是他想反悔,也得准备一段时间,好解开双方的信道盟约。

就算如此,因为他不会撒谎,方源只要稍加试探,便可得出情报。

单凭这点,方源已经是将琅琊地灵吃得死死的。

不过,方源经过深思熟虑,觉得荡魂山、落魄谷乃至智慧蛊,还是留下琅琊福地为妙。

荡魂山、落魄谷乃是幽魂魔尊曾经掌控的两座宝地,万一定仙游没有毁掉,那岂不是代表着影无邪等人,随时随地可以回到这两个地方吗?

方源若将这两块宝地,置入自家仙窍,那就大大不妥,简直是开门揖盗,引狼入室了。

而智慧蛊,方源暂时也运用不了了。

他现在重新复生,有了新的肉身,不比仙僵那会儿了。

仙僵时候的方源,不惧怕智慧光晕。但此时,他若是置身在此光之中,寿命刷刷地削减,却是方源不能接受的代价。

所以,今后迫不得已,方源还是不动用智慧蛊为妙。

“自己此番得了至尊仙窍,却失去了智慧蛊的辅助,也算是有得有失了。不过,终有一天,我会找寻出以鲜活肉身,自由运用智慧蛊的方法。这段时间,不如就将智慧蛊交给琅琊地灵保管算了,也省的我操心此事。”

想到这里,方源便对琅琊地灵道:“太上大长老,你让我直接将这三宝献给门派,那是不可能的。不过咱们之间,完全可以合作。我将荡魂山、落魄谷,还有智慧蛊,借给门派使用。门派负责保管,如有意外,十倍偿还。除此之外,补偿我一些门派的贡献,就可以了。”

琅琊地灵闻言大喜,在那边手舞足蹈,叫嚷道:“你小子很识相啊。那就好,那就好。有了荡魂山、落魄谷,我就可以给那些孩儿们炼魂、壮魂。对他们炼蛊大有好处啊!还有智慧蛊,嘿嘿,这可是智道九转,当年星宿仙尊的本命蛊啊,啧啧啧!用它来推算仙蛊方,简直不要太妙。”

方源听地灵语气,他竟然知道如何运用智慧蛊,不由相问。

地灵也不是笨蛋,笑呵呵地回应:“你要知道如何运用智慧蛊的方法,就拿你的门派贡献来换吧。咱们琅琊派,可是公平得很!”

双方商量片刻,就合作之事讨论完毕。

方源又顺势谈及羽民周中,这位异人蛊仙纵然是方源的奴隶,但这项关系也不是不可以破解。太白云生、黑楼兰都知道周中,此人已是暴露。不如就先统领那些羽民,暂时留在琅琊福地。反正琅琊福地的空间,可是大得很。当然,无法和方源的至尊仙窍比较就是了。

至于原先那些方源石巢中的毛民奴隶,方源和琅琊地灵讨价还价之后,这些毛民仍旧隶属于方源所有,但必须抛开奴隶关系,只能算是方源在琅琊派的下属。

琅琊地灵因为出身关系,热爱毛民,鄙视其他种族,有着大毛民主义。自然不会让方源继续将这些毛民,当作奴隶使唤。

他没有因此和方源闹翻,还将这些毛民仍旧归属于方源管辖,已经分外难得。

其实这事情,方源也乐见其成。

这些毛民,已经都是自由身,归于琅琊福地。但只要方源调动得他们,让他们炼蛊,他们自当从命,和之前并无多少分别。

甚至,这些毛民会被琅琊地灵栽培,还会在福地中成家立业,更让他们激发炼蛊的热情,提升工作效率。

最后,方源将影无邪、影宗的事情,大略地告知了琅琊地灵。

琅琊地灵听得一愣一愣,方源说完之后,好半天才反应过来。

“原来你连红莲魔尊的本命仙蛊春秋蝉都有啊!”琅琊地灵再次叫道,“你小子到底走了什么狗屎运?不,你不会连巨阳仙尊的本命仙蛊狗屎运,都有吧?”

“原来巨阳仙尊的本命仙蛊,竟然是狗屎运?不是鸿运齐天仙蛊吗?”

琅琊地灵哼哼两声:“鸿运齐天那是他晚年所炼。早年的时候,就是狗屎运。我生前和他合作过,怎么会不知道?”

ps:状态不佳,这章还是迟到了。之前所欠的两更,还有要加更的,都记在账上,只能容我下个月慢慢补上了。十分抱歉!

\end{this_body}

