\newsection{菇人乐土暗宗师}    %第三百一十七节:菇人乐土暗宗师

\begin{this_body}



%1
南疆,超级梦境。

%2
“左夜灰?!”听闻眼前的巨大怪物,自己如此称呼,方源心头顿时猛震了一下。

%3
这是太古荒兽,来头极其神秘,年岁极其悠久,从一百多万年前,上古年代就已经出现。

%4
它每一次的出现,都伴随着滔天的杀戮,战力雄浑。若是论资排辈,北原出现的那头狗尾续命貂,也不过是三十万年前才崛起的后辈新秀而已。

%5
左夜灰、佑天光,这两个名词,曾经并列一块,镶嵌在人族的历史中,成为从上古时代、至中古时代,再到近古时代的梦魇。

%6
它们是太古荒兽,但又具人形,很少有人知晓它们的根由。

%7
每当它们出现,都会带来滔天的杀戮,不管是哪一域,都会掀起腥风血雨。

%8
即便是中洲天庭,也无法斩杀掉它们。

%9
它们非常狡猾,时常联手,每当有魔尊仙尊出世,就蛰伏不出,难以寻找到。

%10
庆幸的是,到了乐土仙尊的时代,这位历史上最为仁慈的仙尊决意为民除害,花费大量的时间和精力,终于找寻到了这两头太古传奇荒兽的隐身地点。

%11
一场大战,佑天光死亡,左夜灰重伤濒死之际,逃得一命,至此销声匿迹,生死不知。

%12
“左夜灰……我在梦境中,居然遭遇到了这头太古传奇荒兽?”方源强忍住心头震撼,开始向外攀爬。

%13
左夜灰大口吞食身边的蛊师尸骸,以它为圆心,周围的尸山血海滚滚而流,全都倾泻到它的血盆大口之中。

%14
仙道杀招!

%15
很明显,这是一个仙道杀招。

%16
单纯靠本身的咀嚼,根本无法有如此恐怖的进食速度。

%17
这可苦了方源。

%18
他身受重伤,行动不便,攀爬的速度远远不及左夜灰吞噬尸骸的速度,被身下的这些尸骸往中央带去。

%19
忽然间,左夜灰深吸一口气。

%20
顿时狂风大作,方源猝不及防,被狂风一卷,直接飞了出去,投进左夜灰的嘴里。

%21
左夜灰用力咀嚼,方源瞬间被尖锐的牙齿咬成了肉酱和血泥。

%22
“又死了!”方源带着魂魄上的伤,回到了现实。

%23
“这梦境怎么破?”这个问题,像是一块巨大的山石,阻挡着方源的前行。

%24
他陷入深深的苦恼当中。

%25
在被两位猪头兽人抬着的路程中,方源尝试过多种方法,都无法逃脱一命。

%26
现在他已经知道,这片梦境的发展大势,就是让他进入这个山谷当中去。

%27
但是这山谷中,比落到猪头兽人手中还要危险。

%28
因为这里,居然藏着一头传奇太古荒兽。

%29
太古荒兽能媲美八转蛊仙,但绝大多数因为智慧低下,并不为八转蛊仙所虑。

%30
但凡事皆有例外,在这太古荒兽当中,有那么一小撮,非比寻常。

%31
它们因为种种机缘,拥有着不下于人的智慧,它们学习蛊仙修行的法门,拥有仙窍,懂得自主地操纵仙蛊,甚至能熟练至极地催发出各种仙道杀招。

%32
它们威胁极大,每一头传奇太古荒兽,都是有名有姓。

%33
像之前的狗尾续命貂毛里球,中洲的孽龙帝藏生还有祸空,以及左夜灰、佑天光。这些都是传奇太古荒兽。

%34
这等存在,往往战力要超过大多数的八转蛊仙。

%35
这点从狗尾续命貂毛里球,斩杀了天庭两位八转蛊仙,就可以看出来。

%36
人是万物之灵,但本体孱弱,若不是能运用蛊虫,根本比不上其他生灵。不管是速度、力量、寿命、恢复力、视力、听力等等,都不是人族的强项。

%37
太古荒兽本身就拥有极其浓郁的道痕,它们的寿命天生就比人族高出很多,恢复力、速度、力量等等更是凌驾其上。一旦能够运用仙蛊和仙道杀招,掌握仙窍,拥有人一样的智慧,实力自然会比八转蛊仙高强。

%38
“难道这片梦境,是想让我在接下来,逃脱左夜灰的吞食吗?”

%39
方源再次进入梦中。

%40
一路解梦,他坠落到谷底的尸山之上,还剩下一口气。

%41
“好。”方源奋力,在左夜灰没有开口吸气之前,立即扒开尸体,向下钻去。

%42
左夜灰大口吸气,狂风骤起。

%43
尸山表面的尸骸,都被卷席而出。

%44
方源幸免于难,因为他钻入了尸骸之中。

%45
但好景不长,狂风持续不断,他再次被卷走,没入左夜灰的口中。

%46
“我叉!”方源的最后一幕,是看到尖锐如剑的层层牙齿,插在他的身上。

%47
……

%48
又进梦中。

%49
“我钻钻钻。”方源不断向下钻,并且他在这个过程中运用解梦杀招,专门寻得松动的缝隙,让他进展颇快。

%50
左夜灰大口吸气,持续了很长一段时间,狂风消失。

%51
方源也累得只剩下最后一口气,缩在尸骸之中。

%52
“终于撑过去了。”就在他暗自庆幸的时候,山谷上空悬浮的两位兽人蛊仙中的一位,却开口出声。

%53
“左夜灰,你好胃口。”

%54
乌黑巨人嗡声答道:“太少了,太少了!我要吃更多的人,吃了更多的人,我就能更接近人了。”

%55
“放心吧,这里还有呢。”兽人蛊仙笑着打开仙窍,顿时一大蓬的人族蛊师尸体,向下跌落,仿佛暴雨一样。

%56
“我咧个叉。”方源面色大变,忍不住出声咒骂。

%57
然后,他就被一大堆的尸体落下来,活生生的压死了。

%58
……

%59
再入梦中。

%60
方源钻入尸骸当中,但是这一次,他不再动用全力。

%61
狂风消失之后,他在最表面的一层尸骸当中。

%62
尸体如暴雨般落下,方源迅速钻出来,贴近山壁,寻找到一处凸出的山石,当做保护伞。

%63
尸体落下来后,他安然无恙。

%64
“接下来该如何是好?”方源望着眼前黑压压一片的尸体,心中急速思考。

%65
如今的他,已经被埋在尸山当中,距离最上层的尸体,至少得有两丈的距离。

%66
就在这时,他忽然感觉胸口微微一麻。

%67
方源被堵在尸山之中,看不分明,连忙用手抚摸。

%68
然后他只摸到一个光滑的尾巴,好像是蛇,又仿佛是蜈蚣。

%69
旋即,这个尾巴就缩进了他的胸膛当中去。

%70
“难道我又要死?”这个疑问刚刚从他心底生起,下一刻,他就被逐出梦境。

%71
……

%72
“刚刚那个,好像是一只蛊虫啊?”方源在现实中琢磨。

%73
“人族蛊师尸体这么多,好像是发生了大战,一些凡蛊没有搜刮出来,也很正常。蛊仙就算发现,也对这些蝇头小利不会感兴趣的。”

%74
“当然,这种尸山血海的极端环境中,也有可能抚育出野生蛊虫来,这并不稀奇。”

%75
方源敏锐地感觉到,这只蛊虫或许是他的契机。

%76
伤好之后,他再次进入梦境。

%77
炼化蛊虫,失败,死。

%78
我再炼,双方僵持住,这时左夜灰又开始进食,方源遭受干扰,僵持失败,死。

%79
终于成功炼化蛊虫了,却发现是一种用于进攻的两转蛊,方源忍不住破口大骂,随后,死。

%80
又死。

%81
再死。

%82
继续死。

%83
“这他娘的什么破梦?”一连串的意外和失败,有一次甚至是因为他蹦跶得太欢了,引起了两位兽人蛊仙的注意,结果对方一个小指头遥遥一按,方源就被一道奇光射中爆炸。

%84
绝望。

%85
在梦境中,方源实力太过于弱小。

%86
导致任何一场意外,或者有那么一丝的风吹草动,都是足以让他致命的巨大危机。

%87
“难道要放弃这个梦境吗?”这么一想,放弃的念头就持续不断了。

%88
方源在这个梦境中的投入,已经远超过往,但是希望极其渺茫,甚至根本看不清未来的方向。

%89
这个梦境的难度,超乎想象!

%90
“再看看。”

%91
“再坚持一次罢。”

%92
“又死了吗……该收手吗?”

%93
又是一连串的死亡,方源想放弃,但又有一些不甘,毕竟投入了这么多,更关键的是,这种梦境他前所未见,如果探索成功,对他而言,是一次巨大的经验积累!

%94
这一次放弃了,若是将来还遇到这种类型的梦境呢?

%95
咬牙坚持。

%96
直至最后一次。

%97
山谷的地面上,只剩下五六十个蛊师尸骸,方源挣扎了半天,终于保留一命,混在当中。

%98
蛊阵显现出来。

%99
方源这才发现,左夜灰被一座超级蛊阵困住,下半身没入土中,只有上半身才能稍稍动弹。

%100
“该死的四元阵,让我来破了你!”吃饱喝足之后,左夜灰猛地仰天咆哮,从它身上猛地爆发出一股灰暗之光。

%101
灰光瞬间覆盖了整个山谷。

%102
“糟糕,这是仙道杀招夜灰!我们快撤!”两位兽人蛊仙连忙升上高空。

%103
四元大阵发动,地水风火四道奇光猛地喷涌而出,抗衡灰暗之光,让它只局限在整个山谷当中。

%104
左夜灰发出不甘的咆哮和怒吼,但无济于事。

%105
“我咧个去,这叫我怎么躲?!”灰光无处不在,笼罩整个山谷,兽人蛊仙可以避退,但方源不可以。

%106
他眼睁睁地看着,周围的山石草木在灰光中,不断挥发消散,至于他自己,也不例外。

%107
方源陷入深深的绝望当中:“这该不会是无解的梦境吧?!这灰光根本无法躲啊!早知如此,我就该早早放弃,才算是明智之举!”

%108
带着懊恼之情,方源回到了现实当中。

%109
魂魄受创,这一次伤害,比前面任何一次,都要严重数倍。

%110
不过……

%111
让方源惊喜的是,他发现自己的暗道境界,居然一下子暴涨到了宗师级。

%112
“怎么回事?”方源连忙检查。

%113
他惊讶地发现,他探索的这片梦境已经消失无踪。

%114
“原来只要支撑到最后时刻,这个梦境就算作通过!我成功了!这应该算是求生类型的梦境吧,只要支撑到最后即可。”

%115
“这片梦境只有短短一幕,但是通过之后,居然一下子让我普通级数的暗道境界,一下子飙升到了宗师级!”

%116
“难道说,是因为左夜灰的存在吗?”

%117
种种疑惑,都无法得到解答。

%118
方源对于梦境,还是所知甚少。毕竟五百年前世,他并未在梦境探索这方面下苦功。

%119
几乎与此同时,白凝冰和黑楼兰两人也充斥着疑惑。

%120
“这是哪里?”黑楼兰望着这片鸟语花香,薄雾缭绕的世界,问道。

%121
“这里是菇人乐土。”一位异人蛊仙,从薄雾中显露出身形来。

%122
“菇人蛊仙?”白凝冰的龙瞳微微一缩。

%123
菇人,异人中的一种,他们大体似人,最显著的特征是脑壳宛若蘑菇顶,仿佛是一个天然的帽子,盖在头上。

%124
蘑菇的帽檐下,就是菇人的眉毛和眼睛,至于鼻子、耳朵等等,一个不缺。

%125
“菇人……乐土?”黑楼兰咀嚼着最后两个字,“难道说,这里被乐土仙尊……”

\end{this_body}

