\newsection{杀方源为民除害}    %第七百一十一节:杀方源为民除害

\begin{this_body}

%1
方源的成就,是他自己的努力成果,也有许多前人的资助。

%2
天庭、中洲同样如此,能够防御住方源和西漠诸仙的进攻,并不奇怪。这是实力和底蕴,是中洲、天庭无数代人的努力。

%3
这种努力,不管是现在还是曾经,都是无法否认的。

%4
“事不可为,只有撤了。”方源心道。

%5
厉煌、清夜一直努力纠缠着方源,不过依凭方源的手段,他自信还是能够突围的。

%6
对于帝君城,方源已无摧毁的想法。

%7
一来摧毁它,对大局无益,二来在中洲蛊仙的重重保护下,再想对帝君城出手,困难重重。

%8
但就在这时,忽然传来隆隆巨响。

%9
大地开裂,一道巨大的地沟从南至北,一路吞山噬林,速度极快!

%10
好巧不巧,帝君城就在地沟的前方。

%11
“糟糕!”

%12
“保护帝君城!”

%13
这一幕,出乎所有人的意料。

%14
中洲诸仙手忙脚乱,纷纷出手,要护持帝君城。

%15
但天威难测,地沟的成形并不普通,根源在于五域合并,地脉一统。这是何等的天地伟力,岂是一干蛊仙仓促之间就能应付的?

%16
于是,中洲的蛊仙就只能眼睁睁地看着,地沟一口吞下帝君城!

%17
帝君城无法飞行,极其庞大,地基一毁,它坠落到地沟深处,不断崩解。

%18
这座举世雄城中,有了着近百万的人口。蛊师无数,都是从各方汇聚而来的精英。尤其是此届大会,天庭为了抗衡方源的运势,集结了许多拥有很强气运的人物。

%19
叶凡、洪易就在此中,还有许多人和他们俩相差仿佛。

%20
“快救人!”厉煌大吼。

%21
“哈哈哈,想得美!”方源猛地出手。

%22
他此刻正是阎帝状态,施展魂道杀招极其顺手。

%23
方源已在坠落的百万人群中,见到了洪易和叶凡。这两人都是和连运的,难怪自己的运势一直被压制。

%24
仙道杀招落魄印!

%25
仙道杀招春剪!

%26
仙道杀招夏扇!

%27
方源先是施展了落魄印,随后变化成太古年猴,又施展两大宙道杀招。

%28
“不!”厉煌仓促之间,根本挡不住方源的阴毒攻势,他怒吼起来,却只能眼睁睁地看着无数生命惨死在方源的手中。

%29
这其中就包括参与最终炼道大比的那一小撮精英。

%30
叶凡、洪易自然无一幸免。

%31
“方源,你枉为八转蛊仙,居然屠戮凡人。你如此穷凶极恶,该杀,该杀!”清夜双眼通红,充斥血丝,夹裹无边怒气,杀向方源。

%32
方源这一击,让中洲损失惨重!

%33
清夜深知:这帝君城乃是中洲的一处中心重地,采用的人道手段布置,吸引和栽培出源源不断的人道精英。世代积累,人气非凡,堪称蛊仙的最大摇篮。

%34
要知道中洲乃是门派制度,和其他四域不同。帝君城中盛产人才,尤其是人道气运浓厚的人才。这些人只要获得一定的资源,顺利成长,成为蛊仙的可能性非常的大!

%35
将来五域乱战,蛊仙势必陨落频繁,而这些人就是中洲的储备,蛊仙的种子。

%36
方源这一击,几乎把全城的人都干掉,也干掉了中洲一小半的人才储备,极大地削减了中洲的战争潜力。

%37
这样惨重的损失,如何不让厉煌、清夜等人怒不可遏,恨不得将方源千刀万剐!

%38
洪易、叶凡一死,大量的人才灭亡,方源顿时有一种浑身轻松的错觉。

%39
他知道自己的运道开始占据优势了。

%40
最后看了一眼还在下坠的帝君城残骸,以及自己亲手造成的近百位尸躯,方源飞速后撤。

%41
厉煌、清夜以及诸多仙蛊屋,对他展开凶狠的追杀。

%42
“魔头,魔头!”

%43
“他竟几乎杀了全城的人,里面就有……我的家人!”

%44
“杀了他,为民除害啊!”

%45
中洲蛊仙们无不义愤填膺,杀气腾腾。

%46
方源狂笑:“你们拦得住我么?”

%47
的确拦不住。

%48
虽然方源暂时奈何不了中洲这些人,但这些仙蛊屋,以及八转蛊仙,即便能够围住西漠的仙蛊屋,也围不住方源这等战力。

%49
除非是有仙道蛊阵,或者是仙道战场杀招。

%50
但这仓促之间,怎么可能布置出来?

%51
其实就算布置出仙道蛊阵或者战场,除非迅速消灭方源,否则依靠方源的智道造诣和强大手段,也能脱困。

%52
之前周雄信之死,就是最好的证明。

%53
不仅拦不住,方源还有反攻之力。

%54
轰!

%55
一下重击,方源差点拆掉一座仙蛊屋。

%56
古月方正就在此屋之内,吓得浑身冷汗。这座仙蛊屋连忙后撤,紧急休整,其他仙蛊屋迅速填补空缺。

%57
方源遗憾地看了此屋一眼。他虽然不知道方正就在屋内,但心中有一种直觉告诉他,若是将此屋打掉,对他大有好处。

%58
方正的存在,对于他而言,就是一种制约和羁绊。

%59
“方源、方源!”这个时候,冰塞川再次联络。

%60
“冰塞川,你还好意思再联络我?今日之败,皆在于你!”方源表露出愤怒之情。

%61
其实总体而言,中洲天庭虽有底蕴,但和四域比较起来,仍旧处于下风。

%62
然而,中洲万众一心,紧密团结,而其他四域则行动散漫,各有私心,力量虽然庞大,却分散四处。

%63
事实上,四大域的实力都有保留。比如东海虽然出动了许多八转,但七转蛊仙以及仙蛊屋,出现的很少。又比如西漠方面,虽然出动了仙蛊屋和大量七转、六转,但八转蛊仙很少。许多强者,诸如习家的太上大长老,根本没有参与此战,一直留守在西漠。

%64
冰塞川苦叹:“我这边支撑不了多久,伤亡很惨重,但我们还有机会!那就在不败福地!”

%65
原来,天庭采用人道手段,举办了中洲炼蛊大会,目的就是给不败福地提供无数次失败,然后再从不败福地中抽取出成功道痕。

%66
现在,宿命蛊仍旧未彻底修复,还差最后一批成功道痕。

%67
中洲炼蛊大会虽然举办完成,最后一批成功道痕正在酝酿,然而想要传输到天庭之中,却需要比以往中洲炼蛊大会更长的时间。

%68
这是因为,如今的不败福地已成战场,被重重的仙阵包裹。

%69
这些仙阵虽然抵御住了南疆诸仙的猛攻,但却给成功道痕的输送造成了一定的困难。

%70
经过冰塞川的解释,方源心头一片雪亮:“看来不败福地就是我等最后的希望!”

%71
方源没有丝毫犹豫,他转折方向,悍然扑向不败福地。

%72
身后,厉煌等人以及一部分的仙蛊屋,对他紧追不舍。

%73
不败福地战场。

%74
无限风宛若天柱,缓缓横移,以碾压之势摧残着天庭一方的仙道大阵。

%75
仙道大阵迸射出璀璨的金光,极力抵挡。

%76
武庸等人面色凝重。

%77
八转蛊仙池曲由沉声道:“不可思议,对方的仙道大阵竟然也能被增幅!这漫天漂浮的白色光点,究竟是何种人道手段?”

%78
仙道大阵的本质,便是仙道杀招,是由蛊仙操纵。

%79
蛊仙受到人中豪杰的帮助,仙道大阵自然也威能倍增。

%80
武庸不惜自损底蕴,这才酝酿而出的强大杀招无限风,此刻虽然仍旧占据上风,但战果却比之前要小很多。

%81
“看来我们要迅速破阵,还得里应外合。外面就交给无限风,内里还须我等出手,寻找破绽了。”翼浩方凝神道。

%82
巴十八沉吟:“但此时不同以往,之前我们是突袭,打了天庭一个措手不及。如今时间已经过去良久,我们要时刻防备天庭可能的援兵。所以,必须在阵外还得留守一些人。”

%83
巴十八的话,得到所有人的赞同。

%84
毕竟无限风,虽然威能浩荡,但仍旧只是一记杀招而已。

%85
招是死的,人是活的。

%86
武庸心头沉重。

%87
没有迅速攻破大阵,令他突袭的计划破败,如今只能改为强攻。

%88
天庭一方虽然被连续攻破数道大阵,仙蛊或许有所损失,但是并未伤筋动骨。

%89
陈衣本来已无生还希望,没想到凤九歌这个变数,居然救下了他。

%90
武庸深知:时间拖得越久,对他们而言就越不利。分兵势在必行,一部分人入阵,一部分人在外强攻。然而这样一来,力量就无法统合,破解大阵的效率势必下降。万一天庭援兵出现,也更容易被一一攻破。

%91
武庸心中暗叹,尽管分兵弊端重重,但他必须要做。

%92
“若是我的军力再强盛几分就好了,可惜我虽然成为南联盟主,仍旧无法调动更多南疆的实力。能调集其他三位八转过来,已然是当下最好的成果了。”

%93
武庸深呼吸一口气,立即进行安排。

%94
池曲由阵道造诣第一,当然要入阵去。武庸是战力第一,也入阵,一方面保护池曲由,一方面在阵内进行强攻。

%95
而翼浩方、巴十八,一个修行变化道,一个修行律道,都可以留守在外。毕竟天庭的援兵或许阵容庞大,留守两位八转更加保险。

%96
就在南疆蛊仙商量着如何分兵时,忽然异象传来。

%97
武庸面色微变,看向北方,轻声道:“有人来了。”

%98
来的人还不少!

%99
领头三位,俱是八转。

%100
一位老者,拄着拐杖,慈祥和气,正是北原药家的前任太上大长老,如今长生天的南荒仙人药皇。

%101
另外一位,身着白袍,毛发浓密,身材高瘦,乃是北原百足家的太上大长老百足天君。

%102
第三位,高冠凤目,金色披风,风姿卓绝,则是宫家外姓家老凤仙太子。

%103
随后几位,亦都是人中豪杰,仙中精英。

%104
有霸仙楚度,蛤蟆妖仙努尔倩,枪神袁让尊,吞财童子廿二富等等。

%105
北原蛊仙向南疆诸仙迅速靠拢。

%106
“北原诸位,意欲何为?”武庸扬声问道。

%107
药皇呵呵笑道:“大敌当前,愿与诸位仙友联手,齐攻天庭大阵!”

%108
武庸哈哈大笑,天庭没有来援,反倒是自己这边有了援手。

\end{this_body}

