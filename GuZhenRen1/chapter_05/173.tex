\newsection{不跪}    %第一百七十三节:不跪

\begin{this_body}

正道众仙刚刚坐定不久,散魔一方也齐齐开至。无弹窗,最喜欢这种网站了,一定要好评]

只见一大朵灰云之上,霸仙楚度昂然站在首位,身后两方阵容。一方中有昊震、仇老五、李四春、汪大仙等。另一方则有百足忍、百足凌,还有黑家的一些熟面孔,四大太上长老亦包含此内。

正道众仙目光透过大敞的殿门,望过去。

扫过敌方群仙,更多的目光最后集中在楚度的身上。

几个呼吸后,便有蛊仙笑道:“此战我正道必胜无疑。什么楚门、百足家,如此阵容,也赶来冒犯我黄金一族?”

“哈哈,此言有理。”

“以我观之,唯有楚度一人,需要戒备一二。”

众仙纷纷言笑,大殿中氛围十分轻松。

反观另外一方,昊震、仇老五、百足忍等人,都是目光忧郁,脸带愁容。

看看场面就知道了。

正道一方,出动了三座仙蛊屋。

吒雷殿居右,金晓大殿居中,神光堂居左。

而楚度一方的脚下,却只是一片空空荡荡的灰云。孰优孰劣,一目了然。

正道群仙在大殿中,风不临日不晒,还有美酒佳肴。楚度一方却是个个站在风中,干看着。

正道蛊仙们谈笑宴宴,士气旺盛。楚度这边,则是众仙沉默,一言不发。

唯有楚度面色从容。

他智勇双全,城府深厚,此行之前便早已料到这种情景。当下不慌不忙,对身旁一位蛊仙投去目光。

此仙身材高瘦,怀抱双臂,白眉白发,神情冷傲至极。

楚度向他传音:“雪老弟,你久居深山,苦练手段,拥有惊天的造诣。却声名不显。此时正是扬名之际,我愿将血斗的第一战交予你。此战之后,我相信你不仅将名扬天下,更可能名垂青史。”

雪姓蛊仙闻言。双眼顿时爆发出骇人的精芒。

“上一次楚兄邀我出手,小弟我炼蛊不得脱身。这一次,定叫天下识我雪无痕的大名!”

雪姓蛊仙暗中回了楚度这么一句后,冲天而起,随后又飞落到两方中间。

他仍旧是怀抱双臂。一直沉默。

金晓大殿中,众仙手指雪无痕,纷纷笑言:“竟派遣个无名小辈,前来送死。”

廿二平之刚要站起身来出战,却被身旁的廿二一方伸手拦住:“稍安勿躁,对方不过是六转修为,无名之辈。你杀之,不足以威震众仙。此战不去也罢。”

“啊?”廿二平之想了想,的确是这么一回事,便不再起身。

“何人出战。拿下首阵?”主位上,宫婉婷面带微笑,问向众仙。

一位青年蛊仙当即站了出来:“在下耶律小金,愿意一战!”

宫婉婷犹豫了一下。

她心想:“霸仙楚度非寻常人也,此次调兵遣将,竟派遣一位无名之辈,定非凡俗。若己方稍稍大意,首阵失败,却是不美了。”

而耶律小金也是个年轻后辈,此次在长辈的陪同下。前来参加血战武斗。也是和对方一样,声名不显,没有闯荡出威名。

这时,耶律家的蛊仙耶律恢弘。笑道:“我家这位小辈,虽然只是六转,但却是个战斗天才,常常能在战斗中有神来之笔。就算是我家太上大长老,也赞不绝口呢。”

宫婉婷听耶律恢弘这么一说,也不好驳其颜面。便应允道:“那我们接下来,就看看耶律家的男儿的英武了。”

“耶律小金领命!”这位年轻的蛊仙转身即走。

迈过廿二家前时,他用饱含金芒的眸子,重重地扫视了廿二平之一眼。

“你!”廿二平之被这一激,差点跳起来。

但耶律小金已经走出殿门。

自从廿二平之在铁鹰福地一战扬名之后,让很多同样年轻的正道蛊仙并不服气,耶律小金就是其中一人。

这一次,他也是想要在血战武斗中展现神威,扬名立万!

血战武斗第一场,在双方注视之下,即将展开。

而与此同时,远在南疆。

“白相洞天的有关一切情报,我都告诉你了。当白相血脉的后代,成为蛊仙之后,这只白相仙蛇,就能引领你进入白相洞天。但白相洞天中,危机重重。因为巡天五相中的白相,正是魔道蛊仙,你此行务必小心谨慎。根据我影宗收集到的情报,白相此人生前说一不二,最反感有人忤逆他的意愿,行事十分霸道。唯有战力入他眼界的蛊仙,才能和其交谈交往。你此行进入洞天,不可一味强顶硬冲,大丈夫能屈能伸也。”

影无邪对白凝冰详细叮嘱道。

影宗方面对于五相赌约,早就暗中调查许久,想要往其中插一手。

但是时间和机缘,都让此项计划进展缓慢。其中主要的原因,自然是影宗上下全力以赴去炼制至尊仙胎蛊了。

影宗招揽白凝冰,也有一石二鸟的打算。一来,利用他逃脱宿命的身份。二来,则是为插足五相赌约做准备。

白凝冰冷哼一声:“我自有分寸。”

影无邪不以为杵,仍旧微笑道:“这就好。去吧。”

白凝冰默默取出白相仙蛇。

白相仙蛇体型修长,浑身雪鳞,身姿优雅。双目如玉,仿佛翡翠。头部两侧,长有一对长须,飘飞如仙衣绶带。

它是一只五转凡蛊,对白凝冰十分亲切,应该是白凝冰身上北冥冰魄体的气息吸引着它。

白凝冰开始催动手段,祭起虚窍。

虚窍虚虚幻幻,并不真切。但被祭起之后,便会由虚化实。

这种手段,乃是影宗偷师天庭,学了个五六成,弊端不小。

正因如此,白凝冰才无须渡劫,即可暂时拥有蛊仙的力量。所以,他只是一个假仙。

离开了影宗一段时间,没有对她身上的虚窍进行维护,导致她的虚窍已经失去了许多效用。使用具有时限。

平时的时候,白凝冰以凡身出动,因此转为女身。等到虚窍祭起来后,化为暂时的假仙。这时候身上的凡蛊效用被压下去,令白凝冰还原成男儿身。

时男时女,白凝冰自己也觉得挺尴尬。

不过在场群仙,都没有笑话什么。

太白云生是因为仁厚,黑楼兰枭雄性情。不在意这些,石奴则其实比较紧张,毕竟白凝冰此行,对于影宗接下来的行动影响很大。

白凝冰身上开始散发出蛊仙的气息,白相仙蛇感受到这一点,顿时浑身一个激灵,然后仰头长啸一声。

它的啸声,完全脱离了蛇类的嘶鸣,带出一种高亢威武的气息。

然后白相仙蛇主动飞到白凝冰的脚下,承托着他。向高空飞升而上。

黑楼兰、影无邪等人驻足原地,仰头观看。

他们不能去,不是白相血脉,去了会坏事。一切都只能依靠白凝冰自己。

好在影无邪为了增大白凝冰此行的成功率,在此之前,特意赶往玉壶山,将冰魄仙蛊取给了他傍身。

白凝冰这才在真正意义上,拥有了人生的第一只仙蛊。

此刻,他踩踏在白相仙蛇的背上,俯瞰下方。

但见青山座座。山雾重重,恍惚间,他的眼前浮现出白家寨的一幕。

那是他第一次遇见这只白相仙蛇……

元泉如沸水般汩汩翻腾。

忽然间,哗的一声。泉水如浪,向上涌起一块。

达到一定程度后,这块泉水分裂开来,四处飞溅。白相仙蛇飞出。

“拜见大仙!”白家族长激动地跪倒在地上,同时焦急地叮嘱,“白凝冰。你还不一起跪下。”

“我从不向一只蛊下跪!”白凝冰冷哼一声,身躯挺拔如枪。

尽管白相仙蛇蛊散发着一股飘渺冰寒的气势,隐藏着森森杀机。但白凝冰毫无畏惧,一双蓝眸直直地凝视着白相仙蛇蛊的蛇瞳……

“白家寨的家族秘典中记载,一旦蛊师得到承认,白相仙蛇蛊就会托着继承者,飞升上天,得到天空中的秘藏。”

“原来所谓的秘藏,就是白相的洞天。而能得到承认的条件,则是成为蛊仙啊。”

很久以前的一个疑惑,终于在此刻解开。

“不知不觉间,我已经走到这一步了。”白凝冰深吸一口气,冰蓝的双眸仰望上去,“成败无所谓了,呵呵呵,我只希望白相洞天之行不要太无聊,要足够的精彩才行啊!”

琅琊福地,云城密室。

方源吐出一口浊气。

他的手中,一只梦道凡蛊刚刚炼成,还似乎带着热乎的气息。

他的修为还在六转二次天劫的层次,暂时卡住,升不上去。因为记忆中的福地,已经差不多被他光顾遍了。剩下的虽然一些,但依凭他的境界,却是不能够吞并仙窍福地的。

自从他意识到梦境对他的这种修行方式,有着无以伦比的提升作用后,这些天来,他就一直在炼制梦道凡蛊。

这些凡蛊搭配解谜仙蛊之后,就能形成仙道杀招解梦。

有此杀招傍身,方源能轻易洞悉梦境,瓦解它们,使得自己的流派境界节节攀升。

解梦,提升境界,方源便能吞并更多的仙窍福地。

福地吞并之后,修为飞速攀升,战力暴涨。

战力暴涨之后,杀人更容易,获得更多的仙窍福地。

如此一来,就形成了良性循环。

别提什么杀人有违道德,影响名声什么的屁话,魔道蛊仙就是这么干脆直接!

方源从来不认为自己是个好人。

“血战武斗大会已经开始了么……正好参加几次,杀死几位蛊仙,夺其仙窍吞并了,踏上最后一步,真正成为七转蛊仙。”

“东海那边虽然有市井,井中有不少福地,但是太远了。而且去过一次,天意已经知晓。”

“还是继续炼制梦道凡蛊,血战武斗大会两方杀出真火来,再去也不迟。”

念及于此,方源便再次沉入梦境之中。

绿树葱茏,山路崎岖。

一个商队,在山路上艰难跋涉。

方源就是其中一员。

他到底没有退让,想要证明自己的能力,完全可以击败甲等资质的天才弟弟。

他也需要这样的证明。

但是古月族长没有给他这样的机会。

他亲自暗算方源,动了手脚,方源只能是一败涂地。

最终,他成了自不量力的代名词,遭受各方刁难和唾弃。

“丙等资质,完全没有什么未来。”

“就算战胜了古月方正又如何呢?他可是甲等,整个山寨的未来都是他的。或者说,他就是我们山寨的未来!”

“你这个做哥哥的真是一点度量都没有,居然还为难弟弟。”

成王败寇。

胜败已定,是非都因此颠倒,黑白也因此混淆。

方源近乎于被流放驱逐,只能参加商队,一边付出劳动,一边尽量继续修行。

“停下,停下,我累了。这马车实在太过颠簸,先休息一下。”一个年轻人的声音,从马车中传出来。

“可是大公子,我们距离下一个山寨还有很长的距离。一路上咱们已经休息三次了,再修行的话,恐怕天黑都走不出这个山。”商队总管站在马车外,点头哈腰。

啪。

一声脆响,一记电鞭瞬间抽到商队总管的身上,将后者狠狠抽飞。

“什么东西!”

“这商队是我家的,由我全权管理。你一个狗一样的老东西,也赶来管束老子?”

车帘被猛地掀开,从中走出一位面目狰狞的青年蛊师。

“老奴该死,老奴该死。”商队管家磕头不止。

整个商队都下来。

前面在问:“后面发生了什么?”

后面也在问:“前面出了什么事?”

察觉到无数的目光集中到了自己身上,青年蛊师的眉头顿时又皱起来,叫嚷道:“看什么看,一群好吃懒做的东西,再看小心老子把你们的眼睛挖出来!”

方源忙低下头。

“你,就是你!”青年蛊师忽然手指方源,“过来,跪下,给本公子当凳子。这是你的荣幸,本公子要下车休息一下。”

方源抬起头,紧抿双唇,定定地望着青年蛊师。

“我不跪!”

“什么?”青年蛊师有点难以置信,他差点以为自己幻听了。

“你说什么,你刚刚说什么?你不跪?!”他指着方源的手指,都在微微的颤抖。

他的表情很夸张,就好像是听到了一个笑话。

“哈哈,你不跪?!你一个奴隶似的蚂蚁,你还不跪?!”

然后,他又听到方源的话

“是的,我不跪!”

ps:这章4000字,本来想分开,但想了想,还是不分最好。待会还有一章,会更晚一些。亲爱的读者们,别等了,熬夜对身体有害。明天起来看也可以。(未完待续。)h:.147.247.73

\end{this_body}

