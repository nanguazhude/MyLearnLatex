\newsection{此蛊何名?}    %第一百零二节:此蛊何名?

\begin{this_body}

陈尺老仙缓缓摇头,颇有不动如山的意思:“再晾凉他又有何妨?此事不及。<strong>八零电子书HtTp://Www.80txt.COM/</strong>∏∈,”

陈立志担忧地道:“他是个明白人,我怕拖延太久,会惹恼他。把他逼走,逼到郑驮那边去的话,就糟糕了。”

“小志,勿忧。”陈尺微微带笑,分析道,“真传上的最后考验,要求黑凡洞天中的蛊仙,至少一半支持黑城,黑城方可继承成功。现在这黑凡洞天中,不算黑城他自己,还有九位。我们这边,就有四人,几乎占据一半。这就是一笔巨大的本钱,只要他获得我们的支持,几乎就赢了一半。只要再赢得一人认可,即能功成。他是不可能不考虑这点的。”

“就算他放弃我们这边,跑到其他蛊仙那里,也会碰壁的。现在局面不同了,他毕竟是初来乍到,要想获得支持,非得出让利益,总得付出,才能得到。况且三仙洞那边,嘿嘿,郑驮等人可是野心勃勃呢。”

陈立志不语,陈婉芸则道:“老祖宗,实不相瞒,婉芸一直有一个想法。若是黑城得不到半数以上的支持,岂不就是失败了?黑凡真传继续留在继仙山上,总有一天,我们的后人会登上山巅,夺得此物吧?这黑凡真传,我们世世代代守护着。就连黑凡老祖都承认我们有资格竞争,难道就这样让给外人吗?”

“黑城可不是外人,是我们的本家啊。”陈乐反驳道。

“乐儿,你可别忘了,就是这个本家。在外界享受繁华和自由。而我们这一脉,世世代代被囚禁在这里面。饱受煎熬!”陈婉芸正色道。

陈乐面色一苦,不再说话。

陈尺缓缓摇头:“芸儿的话。其实我早就考虑过了,此法不行。”

“一来,凡人闯荡继仙山,极其艰难。就算我们全心培养,要等到幸运儿出现,继承真传,该是什么岁月?几十年?还是几百年?谁都说不准。”

“二来,黑城虽是一人,但你们可别忘了。他的身后还有黑家本家,还有大量的黑家蛊仙呢。他若是失败了,只是他个人的失败。黑家的蛊仙必定是要前仆后继的。到那时,我们怎么阻挡其他人?难道我们要把本家都阻止在外吗?能挡得住吗?除了我们四位,郑驮那些人一定可靠吗?”

黑凡洞天的这些蛊仙,都不知道宝黄天,这么多年,一代代人,都没有办法沟通宝黄天。所以对黑家的近况。根本一无所知。方源撒谎不打草稿,拈手就来,把陈家四位蛊仙都哄骗得团团转。

陈尺一番反问,说得其余三仙纷纷变色。

陈乐附和道:“是啊。今天黑城公子还和我说,他和黑家那些蛊仙竞争,能脱颖而出。十分不易。到现在,他身上还有伤呢。”

陈尺点点头。接着分析道:“黑城此人能脱颖而出,自然不同凡俗。他本身就是七转修为。又有上极天鹰傍身,前景广大。我们助他成事,恰如雪中送炭。将来回归本家,能够在他的帮衬下,站稳脚跟,那该多好?”

“不过,要让我们助他成事,也不是凭白无故。他还得付出代价,不付出代价得到的东西,是不会珍惜的。他也不会对我们感恩戴德。”

“还是老祖考虑周到啊。”陈立志道。

“听老祖宗的绝不会错!”陈乐笑道。

陈尺幽幽叹息一声,望着三位蛊仙,动情地道:“我老了,寿命不多了。只盼着你们这些后辈,往后日子能好一些。将来我魂归生死门,也算安心了。”

“老祖宗,您可别说这样的话,您一定会活很久很久的。”陈乐眼眶泛红。

陈立志则更加务实:“老祖宗,您别忘了这世间还有寿蛊这样东西!要获得我们的支持,那黑城至少得拿出些东西来,寿蛊肯定是少不了的。[棉花糖小说网Mianhuatang.cc更新快,网站页面清爽,广告少,无弹窗,最喜欢这种网站了,一定要好评]”

陈尺眼底深处闪过一抹精光,他看向陈乐:“不管什么寿蛊,至少得让我们的乐儿能够得偿所愿,和情郎在一起才是。”

陈乐羞得满脸通红,腾的一下站起身来,跺脚:“老祖宗,您、您取笑人家!”

哈哈哈……

密室中,响起一片笑声。

与此同时,方源立足在居所的庭院中,仰头望着夜空。

这黑凡洞天,分化出明显的昼夜。白天长,黑夜短。

这是洞天才有的天象变化,福地一般是不会有的。

黑凡洞天中的夜空,没有一丝繁星。并且黑的也不彻底,细究起来,应当是深沉的碧色。

清风徐徐,方源背负双手,望着天空,脑海中则在总结种种情报。

这些天来,他和陈乐逢场作戏,从这雏儿口中打探到了许多珍贵的消息,对黑凡洞天,还有其他蛊仙都有了更加清晰、全面的认知。

不仅如此,他还对黑凡真传的考验,有了更深一层的想法。

“差不多了,再等两天,时机就成熟了。”方源心中暗道,眼眸中藏着一片冷冽的寒光。

两天时间,眨眼而过。

午后,明媚的光,照得宫殿越加明亮堂皇。

暖风拂面,鸟语花香,一片怡人景象。

陈尺的门前,传来方源的声音:“在下冒昧来访,还望仙友勿怪。”

“终究是耐不住来了。不过这耐心,也算是不错了。”陈尺并不意外。

整个宫殿群,就是一座巨大的凡蛊屋,方源一举一动都在他的掌控之中。

房门无人自开。

陈尺躺在床榻上,坐起上半身,虚弱地道:“贵客前来,恕老朽不能起身相迎。”

方源迈步走进来,站到床前,满脸忧色:“看来仙友受伤不轻啊。仙友是律道蛊仙。我苦思良久,想出一法。可为仙友减轻伤势。”

陈尺哪里真有伤势:“劳烦上仙挂怀,可惜老朽这伤势却不是寻常方法能解。”

方源笑道:“陈仙友有所不知。我这乃是仙道杀招,可是本家的招牌手段。虽然核心仙蛊都还留在族中,但我却知晓杀招内容。我将这杀招传于仙友,仙友可替换核心,或许能对伤势有所帮助。”

“这可如何使得?”陈尺连忙推迟。

方源温和地道:“我和陈仙友一见如故,这些天来又叨唠诸位。相传杀招,算是还礼。”

“上仙哪里的话,这礼也太重了。”陈尺继续推迟。

方源面色一变,语气转为忧愁:“区区仙道杀招。怎能表达我的全部心意?唉!说实在话,这黑凡真传的最后一重考验,实在是叫我为难啊。还请陈仙友指教。”

陈尺听了这话,双眼精芒一闪,明白方源的隐语。他此番是来利益交换,求得自己的支持。

陈尺下意识地坐直了身体,言道:“以老朽浅见,此事和炼蛊是一个道理。炼蛊的时候,需要火候。有时候火候要猛。有时候火候要缓,正是对照事情的轻重缓急。呵呵,惭愧!老朽对炼道兴趣盎然,有些胡言乱语。还望莫怪。”

不是胡言乱语,而是屁话废话一通!

你好好的一个律道蛊仙,对炼道感兴趣干什么?

不过方源心知此话内涵。当即笑道:“本家有无数炼道藏书,浩如烟海。只要我夺得真传,解放诸位。待陈仙友回归家族。这些藏书皆能翻阅。而且私人的炼道手段、蛊方也是不少。等到事成之后,送给仙友,又有何妨?”

陈尺点头,脸上露出一丝满意的神色。

但他旋即又道:“老朽爱好炼道,可谓不务正业。但我那侄孙陈立志,却是奴道蛊仙。他对上仙身边的上极天鹰,可是羡慕得不得了呢。”

方源面色一沉:“上极天鹰,仅此一头,却是不能相让的。不过家族当中,豢养着大量铁冠鹰,名传北原。我可担保,只要回归,便可给与每人一头。”

陈尺脸上喜色一闪即逝,感慨道:“当年黑凡老祖打压乔家,就是为了乔家的养鹰窍门。没想到这么多年,发展的规模这般大了。但是先祖获罪,我等罪民,回归家族,究竟是何处境,老朽甚是担忧啊!”

“不必担忧,我可担保,仙友等人回归家族,壮大整个黑家,必定受到欢迎。”方源顿了顿,又道,“而且获罪之事,已是好几代前的事情。待我获得真传,便赦免诸位罪行。诸位世代看守黑凡洞天,其实早已经攻大于过。在加上助我夺得真传的功劳……呵呵,我回归家族后,必定会为诸位公正执言!”

陈尺笑了笑:“上仙乃人中龙凤,天才俊杰,一言九鼎,有你这话,我就放下一半的担心了。”

“多谢仙友信任。其实大家都是黑家血脉,都是一家人。不过……外面的其余几位,还得看他们的表现了。”方源意有所指。

隐晦的意思,当然是:先到先得,谁先支持自己,将来回归本家获得的好处就越多。

陈尺点头,他这个老狐狸,当然听得出方源的言下之意。

他笑起来:“上仙的话,字字珠玑,极具精妙。尤其是这一家人,形容得妙啊。”

说着,他打量方源几眼,见方源面带微笑,心中便有更多底气,问道:“不知上仙觉得小乐如何?”

方源笑意稍敛,答道:“明眸皓齿,天真活泼,叫人喜爱。”

“实不相瞒啊,乐儿这孩子其实暗中爱慕上仙。唉,可惜啊!上仙您这样的才俊,岂可是她能高攀得上的。唉,我这个做长辈的,根本不敢有这样的奢望。只盼着乐儿这个苦命的孩子,能忘怀上仙,将来有自己的幸福才好。”陈尺长叹道。

方源微笑不再,皱起眉头,思考几下,然后满脸严肃:“不瞒陈仙友,其实在下也是欢喜陈乐,愿娶为妻子!”

若真是喜欢,应当是惊喜连连,脱口而出。绝不会是思考几下,满脸严肃之色。

但陈尺只当做看不见,也似乎想不到。

方源心中冷笑,越发看清此人。这陈尺口口声声为晚辈着想,但真要着想,岂会这般?直接是将陈乐牺牲,换取自身利益!

陈尺哈哈大笑,方源的回答,虽然勉强,但答案让他十分满意。

他得寸进尺,笑声一收,深深叹道:“唉,可惜老朽寿元不足,怕是看不到上仙和乐儿的婚事了。”

方源顿时明白,这老东西竟然是想索要寿蛊!他的脸色顿时阴沉下来,全无笑意,干巴巴地道:“哪里的话,依我看,陈仙友老当益壮,胃口大得很呢。”

陈尺老仙笑吟吟地和方源对视,隐含寸步不让的强硬之意:“让上仙见笑了。老朽啊,其实贪心的很,不止是想看到上仙和乐儿大婚,还想看到二位结合,生下的孩儿呢。说起来,这也是老朽的后辈啊。”

方源开始踱步,眉头紧锁,干脆直接道:“寿蛊难寻,珍贵至极,这个在下实难拿出。”

“此等天赐恩物,得来不易。”陈尺老仙点头,似乎早已料到方源有此回答,他继续道,“不过上仙不必担心。好教上仙知晓,这黑凡洞天中也产出寿蛊,皆被收缴起来。天灵懵懂,但黑凡老祖却有布置。如是老朽所料不差,黑凡真传中定有不少寿蛊。老朽……唉,只需三百寿元的寿蛊即可了。”

“三百寿元?!”方源怒视陈尺。

陈尺仍旧笑吟吟的样子。

方源恼羞成怒:“三百太多,寿蛊我都没有!只给一百。”

“一百五。”陈尺讨价还价,终于不装了。

方源又踱步几圈,咬牙道:“罢了,就舍你一百五!”

“成交!”陈尺一拍掌,大笑起来,连老脸都不顾了。

不过说起来,能增添百岁寿命,一些脸面又算得了什么?

陈尺看了方源一眼,脸上笑容更盛:“听乐儿说,上仙身怀仙蛊无数,不知可否给老朽开开眼界呢?”

方源一愣,旋即怒气冲天,对陈尺咬牙切齿。

这个老东西,胃口这么大。嘴上说是看看,其实是想要方源付出一只仙蛊,来换取他的支持!

陈尺见方源怒气勃发,心头一跳,但又想到机会难得,过了这个村就没有这个店了。难道今后回归了本家,还有机会吗?

于是,他忙道:“这是老朽最后一个请求,只要达成所愿,老朽等四位必定鼎力支持上仙您的。”

方源又开始踱步。

狠狠踱步,仿佛这脚下的地砖,和自己有不共戴天之仇。

陈尺察言观色,见方源脸上怒气渐渐消退,满眼思索之色,渐渐放下心来。

他心中窃喜:“此事成了!”

果然,片刻后,方源停下脚步,站到他的床前,掏出一只仙蛊。

“七转仙蛊!”陈尺低呼一声,又惊又喜。

“此乃剑道仙蛊,与我不太合用,否则绝不会便宜了你!”方源恨恨地道。

“此蛊何名?”陈尺双眼放光,看着方源将这只蛊递给自己。

“等等,还是换做这只仙蛊吧。”方源忽然又改了主意,将伸出的手缩回,另一只手则又从仙窍中掏出一蛊。

陈尺的目光,下意识地移动到方源的另一只手上。

说时迟,那时快。

那边剑道仙蛊陡然发动!

剑道杀招暗歧杀!!

陈尺呆愣住,眉间一道血痕渐渐扩大,旋即鲜血从额前脑后喷溅。

死!(未完待续。)

ps:  这章4000多字,是300的加更!感谢大家的支持,我一直在努力。<!--80txt.com-ouoou-->

------------

\end{this_body}

