\newsection{赵怜云升仙}    %第四百二十节:赵怜云升仙

\begin{this_body}

“哈哈哈,方源长老,你回来了!”琅琊地灵大笑,张开双臂,迈着大步,走向方源。[看本书最新章节请到棉花糖小说网www.mianhuatang.cc]

方源亦是大笑:“能够回到琅琊福地,就像是回到了家一般的感觉,真是安心多了。”

两人用力拥抱了一下,彼此的态度都非常热情。

“方源长老如今已经名震天下,可喜可贺啊。”琅琊地灵看着方源的目光,完全和之前不一样了。

方源如今已经有了八转战力,这是独属于他自身的战斗力量,足以让琅琊地灵重视。

并且,方源还逃脱了天庭的追杀,安然回到琅琊派中。

这份战绩,宣传出去,足以让世人震惊。

站在琅琊地灵的角度,他当然愿意看到,方源和人族第一势力天庭闹得水深火热,不同戴天。

方源越是遭受天下追杀,他越是高兴。

因为他是毛民,执念是让毛民称霸世间。

当然了,琅琊派的安危还是重中之重。若非确信方源浑身“干净”,没有留下让天庭追查过来的尾巴,琅琊地灵也不会这样迎接方源。

“方源长老,接下来有什么打算?”琅琊地灵一边走,一边问道。

方源没有犹豫,坦言道:“如今风头正紧,我打算偃旗息鼓,休养生息一阵子。我已经继承幽魂真传,成为影宗新主,只要时间充裕,我就不断地拔升修为和战力。天庭方面,已经是追查不到我了。说起来,真是惊险,幸亏有太上大长老你在背后,一力地支持我!”

琅琊地灵被说的有些不好意思,只好哈哈大笑。

他当然没有全力支持方源,只是按照他定下来的琅琊派门规,帮助方源炼蛊。

若是全力支持,他早就派遣毛民蛊仙前往接应方源去了。

两人一边走,一边交谈。

沿途见到不少毛民蛊师,他们纷纷主动行礼,皆是满目的崇敬之色。

琅琊派的发展,也很迅速。

自从琅琊地灵改变发展策略,掀起三大陆上的毛民国战,从中激励并提拔出优秀的毛民蛊师。

这些毛民精英,都被选中,带到琅琊福地高空处的云盖大陆上生活,并且指导他们进一步的修行。

各个琅琊派太上长老的壁画,在云盖大陆上的每一座云城中都有。[看本书最新章节请到求书 小说网www.Qiushu.cC]

而属于方源的云城中,更有高达数丈的,和他模样、身材一致的巨大石像。

因此,琅琊派的这些蛊师,对所有的蛊仙长老都十分熟悉。

交谈了片刻,方源便和琅琊地灵分别。

两人交谈甚欢,琅琊地灵对幽魂真传当然非常有兴趣,而方源则想要借助琅琊派的举派之力,帮助他炼制仙蛊。

在炼道方面,方源的实力虽然也有一些,但是和琅琊派相比,完全是小巫见大巫。

双方达成了很多约定,琅琊地灵满意离去。

方源则来到炼蛊大殿,见到毛六。

“原来是太上二长老光临,有失远迎。”毛六笑着道。

他形容枯槁,神色憔悴,满脸烟灰之色,非常狼狈。不过见到方源时,他的双眼透露出复杂之色。

有喜欢,有感慨,有落寞,有叹息。

命运真是玄奇,曾几何时,毛六恨不得方源立即去死!

如今,方源成为影宗之主,却是拯救本体幽魂的唯一希望。

既然方源能保住影无邪的性命,又和天庭不共戴天,仇恨根本无法化解,那么毛六也选择承认方源的身份。

“毛六见过宗主大人,请恕在下不能行全礼。”毛六暗中传音。

这里是琅琊福地,琅琊地灵纵览一切,如掌观纹。

“虚礼而已,我岂会在意这些?”方源传音过去,“还要多谢你的努力,炼出的仙蛊帮了我的大忙。天庭是我的死敌,总有一天,我会攻上天庭,尽全力挽救幽魂本体!”

毛六点点头,眼中闪过一抹异彩,然后转身领路:“请这边走。”

其实他也清楚,方源说的话,没有什么保证。方源若要轻易毁约,根本没有任何的惩罚和代价。

但人活着,总是需要一抹希望的。哪怕这抹希望,非常渺茫。

并且毛六也清楚,若不将方源招揽进来,依凭剩下来来的影宗蛊仙,根本不会是天庭的对手,很快就会被剿灭。

甚至,不需要天庭出手,因为利益,方源更可能率先对影宗残余行动。

紫山真君早已经明白今后的处境,为了保留最多的影宗势力,保留希望,他才将整个影宗托付给方源。

方源心知紫山真君的谋算,却不得不接受,并且心情还带着喜悦。

方源跟着毛六,走了一段路,来到了炼道仙阵的面前。

在那里,熊熊的火焰燃烧着,却散发出惊人的寒气。

寒意逼人,很快,就让方源的眉宇、发梢,都沾染了一层白霜。

“宗主大人,炼制净魂仙蛊,已经到了关键步骤。但这一步,却是令我失败了数次。我运气不如你,宗主大人更适合亲自出手,炼制最后几步。”毛六传音道。

随后,他开始口述,详细介绍炼蛊的步骤,以及当中的诸多要点,自家的种种经验和心得体会。

方源静心倾听,一字不落。

净魂仙蛊将是他接下来,大举修行的关键。

几乎于此同时,远在中洲的灵缘斋。

一位五转巅峰的女蛊师,也面临着她人生的关键时刻。

此时,她一身战甲,肤如凝脂,眸如晨星,弯弯的柳眉微微蹙起,屏气凝神,满脸肃容。

“可以开始了。”灵缘斋的智道蛊仙徐浩开口道。

而在另一边,还站着一位女仙,正是徐浩之妻,李君影。

赵怜云点点头,深呼吸一口气候,她开始升仙!

五转巅峰的空窍中,真元澎湃汹涌,仿佛惊涛骇浪,拼命地冲击四周的窍壁。

窍壁上很快出现丝丝的裂纹。

随着裂纹越来越多,一股无形中的天地威力,托着赵怜云的身体,开始缓缓升空。

一直升到半空中,离地十丈多高,赵怜云这才悬停住。

此刻,赵怜云双目紧闭,一门心思,操纵空窍中的真元,对窍壁发起一阵阵猛地的冲击。

她不知道,在外界,随着空窍窍壁裂纹越多,透露出来的气息越加浓郁,已经勾动了两股庞大的气。

天气汹涌,令天空乌云翻滚,闷雷声声。

地气浩荡,令地面烟尘滚滚,隆隆不绝。

天气、地气越加浓郁。

忽然啪的一声轻响,赵怜云的空窍彻底碎裂开来。

顿时,从她的身上,喷涌出一团极其浓郁的人气!

天气、地气、人气相互吸引,很快就汇聚起来,将赵怜云团团包裹在气流巨球的中心处。

“一定要支持下去啊。”李君影口中呢喃。

辅助赵怜云顺利升仙,不只是灵缘斋的门派任务,更关乎她和她的丈夫徐浩在门派中的利益。

徐浩、李君影乃是倒凤派系,和凤九歌一系,最不对付。

而赵怜云正是倒凤派系当中,最为关键的人物。

徐浩一脸的严肃和谨慎。

赵怜云升仙,和旁人不同。因为她有着神不知护身,从此屏蔽推算,天地灾劫不加身。

这种情况升仙,前所未有。就算当初的盗天魔尊,也是成为蛊仙之后,开创出了神不知的原始版本。

所以,徐浩非常小心,时刻关注着赵怜云的情况,一有不对,就会出手相助。

不过,此时此刻,尽管徐浩有着强烈的帮忙*,也是爱莫能助。

三气吸引,整个融汇调控的过程,非得由赵怜云亲自操纵,旁人是插手不得的。

好在升仙之前,赵怜云做了大量的准备和练习,有着蛊仙的指导和训练,她对升仙的全部过程早已经了如指掌,熟悉得不能再熟悉。

三气交汇的气团,愈来愈小。

这是可喜的变化。

证明赵怜云正成功地,将三气不断地压缩凝聚。

片刻之后,三气凝聚压缩到了极致,赵怜云猛地睁开双眼,轻喝一声,将自家的本命蛊投入到三气的最中心处。

轰隆!

一声巨响,宛若雷霆在耳畔突炸。

一瞬间,赵怜云陷入到完全失神的状态,双目愣愣无关。

下一刻,她清醒过来,连忙看向自己的空窍。

空窍早已不在,取而代之的是一座仙窍。

六转仙窍!

“我成功了!”赵怜云喜极而泣。(未完待续。)

------------

\end{this_body}

