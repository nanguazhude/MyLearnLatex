\newsection{红莲出世}    %第六百五十八节:红莲出世

\begin{this_body}

%1
乌云盖顶,暴雨倾盆。 .更新最快

%2
轰隆隆!

%3
电闪雷鸣,震荡天地。

%4
中洲,枫叶城的中心城主府中,枫叶城主站在一间屋外的走廊中,四处踱步。

%5
他国字脸,不威自怒。此刻,他眉头紧锁,心情就好比这天色,焦躁不安,忧心至极。

%6
忽而,他停下脚步,走到门前侧耳倾听。

%7
排除暴雨声、惊雷声,他听到屋里自家夫人的惨哼,还有几位产婆的叫声:“夫人,用力,再用力一点!”

%8
枫叶城主已经有五十多岁,深爱结发妻子,他是五转蛊师,方圆数千里战力第一,又位高权重,富有领袖魅力,受人拥戴。但唯一美中不足的是,膝下一直无子。

%9
九个月前,他欣喜若狂,因为他的爱妻终于怀了他的骨肉。

%10
他可算是老来得子,人生最大的缺憾得到了弥补。

%11
但尽管他请来了远近闻名的三位产婆,两位修为三转,一位修为高达四转,但在他妻子生产之日,仍旧遭遇到了出乎意料的困难。

%12
“怎么会这样?!”枫叶城主想发怒,却又无从发泄。

%13
他捏起双拳,往日里引以为傲的战力,在此刻却是毫无用处。

%14
尽管他有不少治疗蛊虫,但绝不擅长帮助妇人产子。

%15
“唉!”他只有按捺住心情,低下头,又开始在门前的走廊上四处踱步。

%16
“哇!哇哇……”忽然,他听到婴孩的哭叫。

%17
然后他又听到产婆在里面大笑:“生了,生了,是个男孩!夫人,你成功了!!”

%18
“我,我有孩子了!我的……儿子!”枫叶城主一愣,旋即狂喜,就要迫不及待地打开房门,冲进去看。

%19
就在这时,忽有异香从屋内传来。

%20
这异香沁人心脾,幽华不腻。随后,异香浓郁,凝聚成一道道的流光溢彩,像是一丝丝的水流,浮现在半空中,将整个房屋乃至方圆数百步都笼罩。

%21
“这是什么?!”枫叶城主停住脚步,非常惊愕。

%22
更令他震惊的还在下面,只见空气中的这些异彩流华,越来越多,香气也越发浓郁扑鼻。

%23
天空中雷雨骤息,乌云徐徐推散,阳光穿过乌云的间隙如柱照射而下,正有一道最为恢弘的光柱,不偏不倚照射在城主夫人的产房上。

%24
随后,流光溢彩凝聚起来,竟形成一朵房屋大小的赤色莲花,悬停在半空中,栩栩如生,久久不散。

%25
“天生异象!”枫叶城主震惊,房屋中他的夫人,三位产婆也呆立当场。

%26
与此同时,在九霄云外的高空,三位八转蛊仙站在云端,一直注视着枫叶城的城主府。他们将婴孩出生的异象,都看在眼中。

%27
“终于出世了。”站在中央的蛊仙眉公长叹一声。老年模样,慈眉善目,一双眉毛尤其奇特,仿佛两根小巧的深褐树根,从眉头延伸出去,一直垂至他的胸前。

%28
“此子果然不凡,乃是人道气运所钟,天生气运就如此浓郁,竟然凝聚成红莲华盖,肉眼可见!这绝对是仙尊种子!不枉费我天庭损耗三位智道蛊仙,推算出他来。”站在左侧的蛊仙铜公感慨不已。

%29
他是一位中年男子,体格雄壮至极,满脸肌肤好似黄铜,散发着冰冷的金属光泽。他站在云端,渊岳峙,仿佛是一座金刚塔,大有天地飘摇他不动,风雨雷电他无睹的绝世气概。

%30
眉公点点头:“历代仙尊魔尊,都有一个共同特征,那就是具备人道的气运。只不过有的人初期不显,乃是隐运、藏运。等到某个时期,这才猛地爆发出来。此子一出生就有如此异象,难怪天地都不容,降下灾劫来,要消灭他。这在历代尊者的记载中都是罕见的,此子若是栽培得当,必定会成为尊者中最惊才艳艳的存在!”

%31
说到这里,眉公忽而身躯一震,张口一吐,吐出一口鲜血来。

%32
“眉公。”铜公皱起眉头,“你我为了抵抗灾劫,都是受伤不浅啊。”

%33
“眉公大人!”右侧的蛊仙也流露出关切之情。他人族青年模样,一头紫发,身边龙形气息环绕不休。

%34
眉公看向青年蛊仙:“龙公,宿命蛊已有启示,你与此子有着极深的缘分,此子将来成就尊者之位,你正是此子的护道人!去吧,收他为徒,指引他,栽培他。你将成就他,他也将成就你。你们两人注定在人族的历史中熠熠生辉,万古不朽!”

%35
“是,我这便去了。”青年蛊仙龙公点头道。

%36
“去吧。”铜公眉头仍旧紧皱,“这一次回归天庭,我和眉公为了养伤,都将陷入最为长久的沉眠之中。天庭三公的时代将一去不复返,龙公你将领袖天庭,带领天庭,继续荣耀这大世界吧。”

%37
“是!”龙公按捺住心中激荡之情,脚踩云朵,徐徐而下。

%38
他大袖飘飞,一挥手,气流滚滚,浩瀚无边,龙吟声不绝于耳,引得天地瞩目,世人震惊。

%39
在这股无边的气势之下,他徐徐飞降下来,落到枫叶城主府中。

%40
包括枫叶城主在内,所有的侍卫都早已跪倒在地上,许多人都瑟瑟发抖。

%41
枫叶城主终究是有见识的人,知晓蛊仙的存在,也知道自己的实力根本难以抵得蛊仙的一根手指头,强自镇定上前拜见道:“在下洪铸,拜见上仙。不知上仙屈尊下凡,前来何事?”

%42
龙公微微一笑:“洪铸,你的儿子乃是气运之子,百万年方出一位的绝世天才,根骨绝佳,举世罕见。只要栽培得当,将来必成大器。但若缺乏引导,就有屠戮生灵,祸乱宇宙的危险。所以本仙见才心喜,也为人族苍生所计,特意来收此子为徒,将来好生栽培,悉心教导,使其成为正道领袖,为整个人族谋求福祉的蛊仙。”

%43
“啊……”听到这一番话,枫叶城主洪铸当然惊喜交加。

%44
他为龙公这番话感到震惊,不过刚刚的那番天地异象他也亲眼目睹了,心理其实有了一些准备。

%45
喜悦的当然是,自家儿子有了蛊仙师父,将来成就不可限量,绝对会超过自己这个父亲。

%46
自己虽然是一城之主,位高权重,但是和蛊仙比较起来,那便是九牛一毛,完全不值一提了。

%47
不过除了惊喜之外,洪铸还有一些失落。他老来得子,好不容易有了自己的儿子,没想到还未见上一眼,就要被上仙收做徒弟,今后恐怕相见的次数极为有限了。

%48
龙公看了一眼洪铸,立即知道他心中所想,当即宽慰道:“你放心,我今日来,只是说明此事,并非要立即将你儿子带走。等到他十岁时才是我领引他上山,教他本领的正确时机。”

%49
洪铸听到这番话,顿时激动不已,连忙感谢。

%50
这个时候,房门打开,城主夫人已经收拾妥当,抱着刚刚出生的婴孩,和三位产婆走出房门,连忙跪倒在地,拜见龙公。

%51
龙公的目光立即被婴孩吸引过去,他哈哈一笑,迈开一步,下一刻就到了城主夫人的面前。

%52
他小心翼翼将婴孩抱住,仔细端详。

%53
只见这婴儿一点都没有寻常婴孩出生时的丑陋模样。他满头乌发,唇红齿白,十分可爱。他的双眼明亮如星,熠熠生辉,皮肤白嫩,胖乎乎的脸蛋儿极其讨人喜欢。

%54
这时,悬停在半空中的红莲缓缓下降,不断缩小,在龙公的注视中,最终落到婴儿的额头,浓缩成一块九瓣红莲的胎记。

%55
龙公眼中闪过一丝震惊之色,他心想:“这气运凝如胎记,刻印眉间,这是前代的所有尊者都没有的现象。我的这个徒儿,将来成就尊者,恐怕也十分特殊啊!”

%56
想到这里,轻飘飘的孩童在龙公的怀抱中,他却感觉有一座山峦的重量。一种强烈的使命感、责任感,充斥龙公的心头。

%57
“不必跪着,都起身罢。”又端详片刻,龙公郑重地将婴儿还给城主夫人。

%58
他张口一吐,三股气息一闪即逝,一股落到婴孩身上,剩余两股则融入城主夫妻体内。

%59
婴儿立即沉沉睡去,面带微笑。

%60
而城主夫妇均感到一股无穷的力量,填满四肢百骸。

%61
城主夫人惊呼一声,刚刚生产的虚弱已经不翼而飞,整个人精神焕发,直接回到了年轻时候的状态。

%62
而城主洪铸则骇然发现,自己一身的旧伤都顷刻痊愈,空窍中的真元竟然突破常理,带有一丝碧青光泽。整个实力原本已经升无可升,此刻却好像拔升了一大块!

%63
洪铸不忙感应自己的变化,立即和夫人一同拜谢龙公。

%64
龙公摆手:“你们无须谢我,因为我还要你们的帮助。父母双亲和师父之间,并不能相互取代。我喜欢你们相亲相爱,更重要的是给予你们的儿子深切的关怀,让他感受到人世间的爱和温暖,这对他的正确成长有着极大的帮助。十年之后,我会再来的。”

%65
说完,龙公漂浮而上,徐徐升上高空。一路气流奔腾,龙吟声跌宕不休,宛若无形龙潮汹涌。

%66
满城的凡人们仰望高空,看着龙公远去,最终身影被云层掩盖。

%67
沉静片刻之后,这才轰的一声,满城哗然。

\end{this_body}

