\newsection{你骗我!}    %第六百一十五节:你骗我!

\begin{this_body}

%1
仙道杀招万蛟!

%2
嗷吼吼……

%3
一时间,万千蛟龙飞舞,银鳞似海,铺天盖地般杀向威猛老者。 ?

%4
明白了宙道大阵的底细之后,方源无奈舍弃了威能最强的宙道手段,但却可以动用其他手段。比较起来,方源的手段反而丰富起来。

%5
威猛老者冷哼一声,脸上毫无意外之色。

%6
他来此之前,就已经做好了充分的战斗准备,熟知方源的万蛟杀招。

%7
“泥鳅样的东西,也能奈何得住老夫?”威猛老者傲立长空,竟不闪不避,眼睁睁地看着万蛟奔袭过来。

%8
他身上的阳莽背火衣熊熊燃烧,将他全身都包裹起来,形成一团巨大的火球。

%9
万蛟杀到威猛老者的身边,忽然吼叫声怪异起来,一头头剑蛟充满醉意,胡乱飞舞,四下乱转,凶猛的攻势顿时崩溃。

%10
当每一头剑蛟的醉意,达到一定程度之后,竟蓬的一声,突兀地自燃起来。

%11
剑蛟化为一团团的火炬,猛烈燃烧之后,迅消散。

%12
场面上看,威猛老者一动不动,就将万蛟烧得灰飞烟灭。剑蛟规模极大,但是刚刚冲到威猛老者的附近就直接灭亡,根本伤不了老者的一根毫毛。

%13
“仙道杀招阳莽背火衣!”方源瞳孔微缩,彻底见识到此招的厉害。它攻防皆备,正是方源万蛟杀招的克星。

%14
“那就再尝尝我这一招落魄印吧。”方源心底咆哮。

%15
他早已利用见面曾相识,伪装成一头剑蛟,顺着蛟群大军,冲到威猛老者附近。

%16
然后,他就扮做醉意熏熏的样子,又接近一小段距离,猛地施展出落魄印。

%17
杀招刚刚催动,方源就露馅了。落魄印气势磅礴,万难遮掩。

%18
威猛老者目光电射,怡然不惧方源正在的落魄印,杀向方源。

%19
轰轰轰!

%20
老者对方源狂轰滥炸,皆被方源的逆流护身印守住。

%21
方源闷声不吭,抗住老者的好一阵轰击,千辛万苦终于将落魄印酝酿成功。

%22
威猛老者已经有了防备,对于方源而言,战机就更难把握。终于,他勉强射出落魄印,奇光擦着威猛老者的肩膀飞出去,只中了一小半!

%23
老者顿时感觉心中陡然一阵空虚,与此同时,阳莽背火衣如遇狂风,火势顿时削弱了四成!

%24
“这招厉害,不过还是不能突破老夫的阳莽背火衣啊!”威猛老者哈哈大笑,一边爆退,一边催动手段治疗自己。

%25
他度比方源更快,方源很难对他形成连绵不断的打击。

%26
老者得到机会喘息,暂时落入下风,但片刻之后,他的伤势大半好透,气势再出恢复巅峰,阳莽背火衣也是重新熊熊燃烧起来。

%27
落魄印威能恐怖,但威猛老者却是警觉非凡,再不给方源什么可趁之机。

%28
他的阳莽背火衣防御极其出色,方源苦战中终于感受到,曾经他的对手面对逆流护身印的心情。

%29
“这老东西的魂魄并非是他的弱点。”虽然只是擦中,但方源还是收获到了宝贵的情报。

%30
落魄印是针对敌人魂魄,但威猛老者乃是老牌八转蛊仙,底蕴雄厚,并无短板之处,魂魄上的防御可谓森严。

%31
方源估算出,就算是落魄印全中,也打不死威猛老者,绝无一锤定音的可能。

%32
落魄印往往要和引魂入梦进行搭配,才能对付得了天庭的八转蛊仙。

%33
但在开战之初,威猛老者就摧毁了方源的仙蛊屋雏形,影无邪等人现在还陷落在梦境中。

%34
若是仙蛊屋雏形健在,方源有影无邪的引魂入梦辅助,对付威猛老者,绝对会占据上风。

%35
但被老者埋伏突袭,摧毁了仙蛊屋雏形,方源不仅是损失惨重,更是失了先手,落入被动局面,一直都抢占不了上风。

%36
这样的局面,对方源当然是相当危险的。他是落入了埋伏当中,必须尽早突围出去。

%37
仙道杀招万蛟!

%38
方源继续催动万蛟,更多的剑蛟充斥整个大阵空间。

%39
威猛老者虽然牢牢占据上风,却从未有麻痹放松的一刻。

%40
他眼眸微微一凝,方源再次隐匿到蛟群里头。这个时候,他是要选择侦查手段,来看破方源的真身,还是先剿除了这些剑蛟?

%41
方源若是不除,剑蛟可谓无穷无尽,但威猛老者却有自知之明,他心中暗忖:“恐怕我的侦查手段,并不能识破见面曾相识。方源狡诈,恐怕心思不纯!”

%42
威猛老者虽然占据主动,言语神态都张狂粗豪,但他心中却一直都很谨慎冷静。

%43
他知道:他和方源这种层次的高手对决,双方的战斗经验都异常丰富,自己虽然因为先手,暂时领先,处于主动,但若一个应对失策,就会被方源改变局面。

%44
老者呼啸,以他为中心,火焰熊熊燃烧,迅蔓延,形成滔天的火海。

%45
火焰中剑蛟如蜡,迅消融。

%46
老者的这个抉择,令方源相当难受。

%47
方源犹豫了一下,旋即咬牙坚持战术,操纵着万蛟冲击大阵空间。

%48
掌控此处大阵的,自然有中洲蛊仙。见到万蛟冲阵,他们就欲出手应对,但下一刻威猛老者就传音过来:“都不要动!方源宙道、智道境界都很高,大阵变动得越多,就越容易被他看破。我来处理这些泥鳅,稍安勿躁。”

%49
老者不想给方源任何的可趁之机,火海灼烧剑蛟,威力惊人。

%50
方源却是冷笑一声,忽然出现在梦境面前。

%51
仙道杀招纯梦求真变!

%52
他迅将梦境收拢,再次将铺散成一团的梦境,转变成一尊纯梦求真体。

%53
“不好,这魔头好生狡诈!”老者面色一变,方源虚虚实实,灵活多变的战术,让他难以拿捏,捉摸不透。

%54
方源张开仙窍门户一丝,想要将宙道分身、影无邪魂魄、黑楼兰、白凝冰都塞进去。

%55
但就在此时,他的身后传来威猛老者愤怒的咆哮声:“魔头,你休想!”

%56
轰!

%57
一记杀招爆,形成山呼海啸般的狂暴巨响。

%58
方源头皮麻,不用转身去看,都知道此招必是老者开战以来最凶猛的攻势。

%59
“来不及了!”方源根本就没有太多的时间,一颗心心沉落谷底。

%60
关键时刻,他也拼尽全力!

%61
宙道杀招爆开来,令他附近的时间忽然变快。

%62
他将群仙抢进仙窍,几乎与此同时,他的后背就像是被一个巨人用拳轰中,巨大的力量把他像是一颗炮弹般打飞,逆流河瞬间损失一成半!冲撞之力令方源头昏脑涨,直接呕出一口鲜血。

%63
他只来得及救出大部分人,剩下妙音仙子无法救援。她还处在昏迷的状态,在漫天的火焰中被瞬间烧尽,连一丝的灰烬都没有残留。

%64
方源咆哮一声,一边催动万蛟,冲击周围,一边自己主动出手,轰击大阵。

%65
威猛老者紧张起来,对准方源狂轰滥炸。

%66
方源拿着逆流护身印硬撑,不管老者,只对大阵出手。

%67
“方源,你想冲破此阵,纯粹是妄想!”老者咆哮,火焰狠狠地灼烧方源全身。

%68
“哈哈哈,你这是临死前的挣扎。”老者扑到方源身后,双掌如巨碑,狠狠拍中方源,将方源打飞出去。

%69
方源被打得七窍溢血,狼狈不堪,但目光却仍旧如冰雪般冷漠。他继续猛攻这处宙道大阵。

%70
“没有用的!如此卓绝大阵,你还想突破?好好看看你的逆流河罢,它已经快要消散了。”威猛老者嘲讽,攻势更加频繁,杀意狂涌,疯狂至极。

%71
尽管逆流护身印能够逆反攻势,但威猛老者同样有阳莽背火衣。他硬生生承受着逆反回来的攻势,不断地削减逆流河。

%72
他说的并没有错,方源的逆流河历经大战,早就消耗了许多。此刻被他狂轰滥炸,河水暴降,方源笼罩全身的仙衣绶带越来越模糊不清。

%73
忽然,方源大吼一声,停下攻势,酝酿落魄印。

%74
老者眼眸顿缩,攻势一变,时刻防备着落魄印袭来。

%75
方源酝酿完毕,手掌一甩,却没有攻击老者,而是射向一处微妙的大阵边缘。

%76
“呕!”一声惊呼旋即传出,一位操纵大阵的七转蛊仙中了落魄印后,当场阵亡。

%77
“怎么会?!”所有的中洲蛊仙都大惊失色。

%78
方源居然在这么短的时间里,就洞察了其中一处阵眼,将里面的人击杀。

%79
落魄印对付八转蛊仙,都威能不凡,更何况对付这些七转蛊仙。

%80
“快,那处阵眼已失,变阵,进行弥补!”蛊仙大叫着。

%81
另一边,老者爆喝一声,再次杀向方源。

%82
方源用羸弱不堪的逆流护身印继续遮挡,同时对大阵真正动手。

%83
种种手段接连使出,天庭费尽心力铺设而成的宙道仙阵,节节溃败崩解,根本挡不住方源的破解。

%84
“怎么会这样?!”方源对大阵的解析程度,让所有人都感到深深的震惊。

%85
“快,挡住那里!”又有人惊吼起来。

%86
大阵终于出现了一处漏洞,和外界联通起来。

%87
方源朗笑一声:“走也。”

%88
“你走不了!”火焰呼啸而来,化为威猛老者。他死死地将那处漏洞护在自己的身后。

%89
其余中洲蛊仙见此,顿时大松了一口气。

%90
方源如箭矢一般,冲向老者,语气淡漠,杀意横流:“让开。”

%91
老者狞笑:“没有用的,你根本奈何不了我的阳莽背火衣!”

%92
双方距离迅缩短!

%93
中洲蛊仙们急忙变阵,有人叫道:“挡下方源,只需要五个呼吸,我们就能改变阵势!这是翻天覆地的变化,他要推算破解比之前还要艰难十倍!”

%94
老者嗤笑:“别说是五个呼吸,就算……嗯?”

%95
他的双眼忽然瞪圆,脸色僵滞,难以置信地看着方源杀招的起手。

%96
方源气势全数收敛,毫无一丝外溢。

%97
他双手呈掌,啪的一声,在胸膛处合十。然后他的右手五指贴着左手掌,像是从中捏取了什么珍贵的东西,收拢起来,最终捏成一个拳头。

%98
他的左掌停在胸口,而右拳则抬升到他的头顶上。

%99
“这、这一招是……”老者再无一丝笑意,眼眸中透露出紧张,甚至是一丝惊恐。

%100
没错,正是仙道杀招五指拳心剑!

%101
时间滑过,第一个呼吸。

%102
一指!

%103
方源面色冷漠,陡然翘起右手的大拇指。

%104
他高举在头顶的右拳拳心,猛地射出一道剑光。

%105
快!快!快!

%106
剑光之快,匪夷所思!

%107
刚刚出,就射中老者的额头眉心。

%108
老者狠狠一颤,全身的阳莽背火衣火势削弱了一半。

%109
第二个呼吸。

%110
二指!

%111
老者反应过来,大吼一声,阳莽背火衣再度熊熊燃烧起来。

%112
但剑光再中,阳莽背火衣又遭削弱。

%113
第三个呼吸。

%114
三指。

%115
老者脸色渐白,几乎要将一口牙咬碎,阳莽背火衣凶猛膨胀起来,宛若火球,之前只是车马体积,现在却是大如房屋。

%116
老者怒吼:“这才是我阳莽背火衣的巅峰状态!来吧,你攻不破我。”

%117
刚说完,剑光正中,阳莽背火衣像是一个鼓胀的气球被戳破,骤然间又回到了之前的状态。

%118
五指拳心剑越往后,威能越大。

%119
第四个呼吸,四指!

%120
老者拼死硬撑,挡下第四剑。他脸色惨白,倒退一大步,阳莽背火衣仿若风中残烛,只留下薄薄的一层。

%121
第五个呼吸。

%122
方源冷笑:“什么火衣,可笑。厉煌,你是在给自己披麻戴孝吧。结束吧,第五剑!”

%123
老者身心狂震,眼眸中被惊恐充满:“不,薄青,你休想杀死我!”

%124
火焰再燃,他分化成数十个人影,向四面八方暴射而去。

%125
方源不管不顾,一头冲出漏洞,逃到光阴长河中去。

%126
他刚刚飞出来,宙道大阵就变得面目前非,之前的漏洞迅消弭,再度成为一个密不可分的整体,牢牢将内部空间禁锢。

%127
但方源已经逃出去,宙道仙阵内是死一般的沉寂。

%128
操纵大阵的中洲蛊仙们,皆是目瞪口呆。

%129
火焰分身静静地消散,只留下威猛老者的本体。

%130
此刻的他脸上一阵青一阵白,愤怒、后怕、羞愧、仇恨、杀意种种情绪充斥他的内心,像是一团乱麻,将他的神情扭曲成狰狞的恐怖模样。

%131
忽然,老者仰头咆哮。

%132
“根、根本就没有第五剑!”

%133
“方源!!”

%134
“你骗我!!!”

%135
ps:心中忽然有了一个灵感,是关于威猛老者厉煌的故事。我会专门给他写一个小传,到微・信・公・众・号蛊真人上,不是今天就是明天,算是番外吧,可以辅助阅读。

\end{this_body}

