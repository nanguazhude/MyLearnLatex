\newsection{战罢}    %第一百一十七节:战罢

\begin{this_body}

%1
“这,这是什么?”

%2
“好大的一只鸡啊!”

%3
看到上古年兽忽然乱入战场,毛民蛊仙们都被惊到了。

%4
“根据黑凡真传的内容所述,年兽有十二种形态。这次能召唤到鸡年兽,也算是运气不错。”方源一边打量这头年兽,一边在心中评估。

%5
这头年兽,体型好似小山。

%6
双翼张开,在地面上投下巨大的阴影。

%7
它雄赳赳气昂昂,神威凛然。羽毛鲜艳灿烂,鸡爪锋利如刀。

%8
它迅速扫视战场一圈,然后盯着方源看。

%9
没有人比方源更吸引它,因为它从方源的身上,嗅到了年蛊的味道。

%10
年蛊!

%11
这可是年兽的食物。转数越高的年蛊,年兽就越喜欢。

%12
仙道杀招年兽召来,便是依此原理所创。创造者正是黑凡,早在炼制出似水流年仙蛊之前,他就已经创造出这个杀招。

%13
“这是你的。”方源笑了笑,忽然一挥手,洒下大量的凡级年蛊。

%14
鸡年兽大喜,扬起脖颈,张口一吸。

%15
顿时,庞大的气流形成,仿佛是龙吸水一样,眨眼间它就将半空中所有的凡级年蛊,都吸入腹中。

%16
“打杀了它,你就能获得更多。”方源伸出右手食指,指向那头关键的上古鹰犬。

%17
鸡年兽顿时目光犀利起来,闪电般转头,直视目标。

%18
上古鹰犬顿时皮毛乍起,露出锋利的獠牙,它从鸡年兽的身上,感受到了强大的威胁。

%19
上古鹰犬和鸡年兽比起来,体型显得较小,就算人立起来,也抵不过鸡年兽的一半高度。

%20
喔喔!

%21
鸡年兽一扇翅膀,冲向目标。它爆发出来的速度,居然比上古鹰犬还要更快。

%22
砰。

%23
两者狠狠地撞在一起,鸡年兽不过后退了七八十步的距离。而上古鹰犬则倒飞出去老远。

%24
鸡年兽发出一声长啸,对上古鹰犬展开追杀。

%25
上古鹰犬的确狡诈,吃了点亏之后,就再不跟鸡年兽对撞。而是左右游曳,游斗鸡年兽。

%26
方源看了一会儿,放下心来。

%27
这头鸡年兽身上并无仙蛊,但本身素质比上古鹰犬要高出很多。

%28
年兽,就算是在光阴长河当中。也十分少见的珍兽!

%29
当然,上古鹰犬若催动仙蛊,鸡年兽就会落入下风。一次两次无所谓,数量多了,鸡年兽也会落败。

%30
拥有仙蛊的一方,战斗力自然能得到极大的提高。

%31
“黑凡在真传中告诫,召唤过来的年兽,并非奴役,掌控力度不够。遇到太强的敌人,不会迎战。就算战斗起来。有时候也会中途撤下,向召唤的蛊仙索要更多的年蛊。”

%32
想到这里,方源便知道留给自己的时间不多。

%33
他没有选择和鸡年兽一起,夹击上古鹰犬。

%34
因为鸡年兽不受他的掌控,只能通过黑凡留下的方法勉强沟通,双方很难配合在一起。

%35
方源真身隐藏在万我大军中,悄悄接近一头荒兽鹰犬。

%36
鹰犬们还在空中,和万我大军纠缠。

%37
时不时有鹰犬,被方源的力道虚影给打下去。但旋即,这些鹰犬就又飞上高空。继续参战。

%38
力道虚影虽然数量很多,但是面对荒兽、上古荒兽,攻击能力显得比较弱小。只能起到骚扰和消耗的作用。

%39
力道大手印!

%40
忽然间,方源动手。力道大手印凭空而生,照准一头鹰犬的脑袋拍去。

%41
方源的位置极好,鹰犬猝不及防,被一巴掌拍中,当场惨嚎一声,昏了过去。

%42
鹰犬开始坠落。但方源又催动一记大手印,早就在下方等候。

%43
第二只大手印,一把抓住这头昏死过去的鹰犬,仔细地将它放到地上。

%44
而方源真身,则再次隐没在万我之中,消失不见。

%45
但很快,他又在战场的另一处出现,力道大手印奇袭,又拍昏一头鹰犬。

%46
然后,他如法炮制,将第二头鹰犬继续放到地上,和第一头搁置在一块儿。

%47
毛民蛊仙们都看傻了。

%48
“方源长老似乎是在……”

%49
“没错,他是打算生擒活捉了这批上古鹰犬!”

%50
“真是艺高人胆大啊。”

%51
毛民蛊仙们无法不感慨。这些他们视若性命杀手的鹰犬兽群,在方源看来,却是一笔主动投怀的财富。

%52
一个接一个的鹰犬,被方源算计,拍昏过去,叠累在一起。

%53
很快,这些昏死过去的鹰犬,就堆成一座山,景象颇为壮观。

%54
上古鹰犬看到这一幕,发出怒号。但它被鸡年兽牵制住,无法救援。

%55
鸡年兽也有些惨,浑身伤痕累累,似有退缩之意。

%56
毕竟黑凡乃是宙道蛊仙,能创出类似奴道效果的仙道杀招来,已经十分了不起了。

%57
“看来年兽召来这种杀招,打顺风战是比较理想的。若打逆风死战,效果就多有不佳。”方源暗中记下。仙道杀招还是在运用中,得到的体会更深一些。

%58
方源牢牢占据场上主动。

%59
鸡年兽和上古鹰犬的战斗情况,他始终分心留意。

%60
在鸡年兽退缩之前,方源将其余的鹰犬都解决掉。

%61
大手印拍昏的鹰犬有八只,其余的却是被拍死了。毕竟大手印并非专门的擒拿手段,方源能用到如此程度,已属不易。

%62
“死!”

%63
方源一飞冲天,冲上前去,和鸡年兽夹攻上古鹰犬。

%64
战了几个回合,果然是效果不好。

%65
方源便叫鸡年兽退下去,守护住那些昏迷的鹰犬。

%66
鸡年兽早已战意衰减,不过接到方源的指令后,它也没有立即行动,而是稍稍退了一段距离,望向方源,不断张嘴。

%67
方源了然,向它撒了一大把年蛊。

%68
鸡年兽饱餐一顿,这才听话,掉转身体,落到地面上。做起守护的工作。

%69
方源和上古鹰犬再次交手。

%70
但这时,情况又和之前不同了。

%71
上古鹰犬急于援救那些昏迷的同伴,左冲右突。

%72
“可叹。堂堂蛊仙,落到如此境地!居然真的以为自己是一头鹰犬了。”方源此时作战。较之之前,大为轻松。

%73
上古鹰犬速度飞快,又有仙蛊傍身,关键时刻,还能激发残存的战斗本能。催起曾经的仙道杀招。

%74
方源改变了战术。

%75
之前,他冲上去使用大手印,太过冒险。血染征袍的防护效果,在这头上古鹰犬面前,可不够看。

%76
方源便屡次动用剑道杀招,遥攻上古鹰犬。

%77
但上古鹰犬的防护手段上佳,具体不知道是什么仙蛊。估计这位变化道蛊仙身前,是有感鹰犬本身防护能力较低,所以才特意加强了这个方面。

%78
方源久攻不下,上古鹰犬也援救不来。鸡年兽打起保卫战,还是比较到位的。

%79
上古鹰犬身上的伤势,慢慢累积。

%80
“还真是难缠。大手印速度慢,打不中。剑道杀招能打中,但是效果不佳。毒气喷吐简直是浪费仙元。上古鹰犬本身就极为耐毒!”

%81
方源无奈。

%82
似乎,这位变化道蛊仙彻底转变成上古鹰犬之后,拥有了野兽般的直觉,再加上残留的战斗本能,变得更难对付。

%83
面对它,方源发现自己居然有些无计可施。

%84
变化道蛊仙的强势之处。就在于变化成功后,身体素质极高。就眼下这头上古鹰犬,单单本身的飞行速度,就可在短程内比拟剑遁了。更别说仙蛊运用起来之后。

%85
眼下。方源只好选择消耗战。

%86
等到这位上古鹰犬身上的仙元,消耗一空,仙蛊调动不起来,那就是他的胜利了。

%87
或者,等到它伤势累积到一定程度,出现破绽。让方源抓住!

%88
方源感到无奈,其余的观战者却不是这般感觉。

%89
毛民蛊仙们都看得双眼发直。

%90
实力!

%91
竞争残酷的蛊仙界里,最看重还是蛊仙的实力。

%92
方源展现出来的实力,让毛民们都感到震惊。

%93
即便是毛六也不例外。

%94
“这个家伙,短时间内,怎么战力增长得这么多?!他居然能召唤出年兽,这个手段他如何掌握的?明明只是外出了那么一段时间而已啊……”

%95
毛六感到心里沉甸甸的。

%96
方源进步的速度,让他十分压抑,几乎有些透不过气来。

%97
又战了片刻,上古鹰犬忽然拔空而起。

%98
“嗯?想逃!”方源心中咯噔一下,连忙追上去。

%99
上古鹰犬的伤势积累到了一定程度,不管是攻击力道,还是飞行速度,都有明显下滑。

%100
不管是蛊仙生前的战斗才华,还是野兽趋吉避凶的本能,都让它选择了撤退。

%101
至于那些昏迷过去的“同伴”,那也只好舍弃了。

%102
方源心中叫糟,这个情况他早有预料,只是发生的时机有点早过方源的预期。

%103
方源催动剑遁仙蛊追赶,但距离缩小的速度很慢。

%104
上古鹰犬亡命飞驰,速度惊人得很。

%105
仙道杀招过往来动!

%106
方源无法,只好催动这个杀招。

%107
成功了!

%108
上古鹰犬直接退回到之前的位置,方源赶上,双方再次激斗起来。

%109
战了几个回合,上古鹰犬再次跑路了。

%110
方源只好再追。

%111
过往来动!

%112
再次成功。

%113
双方又战做一团,毛民蛊仙们遥望。

%114
但很快,上古鹰犬舍弃方源,双翼疯狂的扇动,第三次撤退,态度比之前更坚决。

%115
方源无奈,他断不了对方的鹰翼。

%116
过往来动。

%117
可惜,第三次他催动失败,立即吐了一小口鲜血。

%118
几个呼吸的功夫,上古鹰犬已经飞到天边去了。

%119
方源只能收手。

%120
天意一直在关注,太丘危险,又隐隐有兽潮的雏形出现。

\end{this_body}

