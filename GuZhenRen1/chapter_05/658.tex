\newsection{我后悔了}    %第六百六十节:我后悔了

\begin{this_body}

“宿命……”龙公叹息一声,脸上涌现出深沉和复杂。??他走了几步,做到一块山石上,又拍拍身旁的位置,示意洪亭坐下。

洪亭也学着龙公,双腿盘坐下来。

“你看。”龙公手指着他正坐着的山石一脚。

洪亭连忙看去,只见这处山石的旁边有一小群蚂蚁,它们正排成一队,托举着食物,朝巢穴归还。

“这就是宿命。”龙公继续道。

“啊?”

“你再看。”龙公又手指向天。

洪亭仰望天空,只见天空中漂浮着朵朵白云,形状各异。

“这也是宿命。”龙公又道。

洪亭心中一动,若有所悟:“师父,你的意思是……”他说不下去,心中有所领悟,但一时间难以有语言表达而出。

“蚂蚁搬食,蜜蜂采蜜,清风吹拂,白云漂游,这时间万物都有各自行进的路线。在我们看来,它们或许毫无规律,但其实都是根据天地的大道规矩而动的。”

“你看日月,每一天日升月落、月升日落。你看人的生死,不管是什么样的人,哪怕是仙尊魔尊,也都要死,都是从生向死。”

“这一切的一切都是宿命。”

“我们每一个人,每一个生命,乃至每一块山石,每一滴水,每一团火,既然在这个世间存在,就必有存在的意义和价值。善恶也是如此,没有善,哪里来的恶?没有恶,又何必谈善?”

“你急着铲除薛屠刀,是没有看清楚他身上的价值。天地既然要让他存在,必然有他活着的理由。这就是宿命。宿命对世间的一切,都早有安排,只是这种安排我们都只能盲人摸象,看不清楚。”

“看不清楚很正常。天地的规律,宇宙间大道,哪怕仙人穷尽一生,都无法尽数洞悉。我们太弱小,而天地太广阔。我们应当敬畏天地,按照宿命的安排,各行其道,方能顺天应命,造福世间。”

“你的父母之所以结合,是宿命的安排,最大的意义就是将你降生于这个世界之中。”

“你将来会成为仙尊,这也是宿命的安排。你要接受它,一步步登上巅峰,最终领袖天庭,为人间正道贡献毕生。”

“而你师父我,此生最大的意义就是教导你,指引你,走上正确的道路。我就是你的……护道人。”

“你要相信宿命,认可宿命,它所做的一切安排都有它的道理。我们若是强行干涉、改变,必然会得到悲剧和悔恨。就像你提前要取走薛屠刀的性命,结果你杀了他没有?”

龙公摇头:“并没有。你虽然实力远于他,但你会遇到各种意料之外的情况,所以最终你不仅没有除掉他,反而还连累了无辜。”

“现在你再回想看看,若是你听从宿命的安排,在薛屠刀最虚弱不堪的时候,铲除掉他,你还会遇到这些意外吗?”

“现在我告诉你,薛屠刀仍旧存在的意义,就是开启这份蛊仙传承,为王先驱,把这份传承带给你啊。”

洪亭听得懵了,他一动不动,仿佛石雕。

两股清泪从他的眼眶中,无声地流淌下来,然后他哽咽道:“师父,我错了。”

“知错能改善莫大焉,其实你刚刚的错误选择也是宿命的安排。”龙公道。

“师父,此言何意?”

“我们都在宿命的安排之下,你以为你要打破宿命,其实你这份念头都是宿命的布置。你不必内疚,应当体悟宿命的用心。你以为你的错误选择,是毫无价值的吗?不对。”

“每一种错误对于少年的成长,通常都有着不可估量的价值。你若能从这个错误中汲取教训,认知到宿命的存在,接受和认可它,那么这个错误就体现出了价值。那座山村的毁灭,就有了毁灭的意义!”

说道这里,龙公深深地注视洪亭:“我的徒儿,你是个天资聪颖的好孩子、乖孩子,但是为师有一项担心,你太重感情。如今你已有五转修为,为师让你考虑选择何种道路升仙,恐怕你会选择智道吧?”

“师父,你法眼如炬,徒儿的确是这样想的,徒儿感觉智道非常适合自己。”洪亭实话实说道。

龙公缓缓摇头:“智道,讲究念、意、情。你太重感情,专研情感对你弊大于利。听为师的,选择宙道罢。当你纵观古今,种种兴衰荣辱,你就会明白一切的感情、风流,都会被时间冲刷个干净。为师从宿命蛊中得到了启示,宙道是最适合你的道。”

洪亭微微张口,想要说什么,但最终点头道:“徒儿谨遵师父教诲,就选择宙道了。”

龙公欣慰地点点头:“这就好。为师乃是你的护道人,指引你走上正确的道路,便是为师存在的意义啊。”

时光继续向前流淌。

洪亭选择宙道,在龙公的保驾护航之下,成功地渡过了升仙劫,成为一名宙道蛊仙。

……

“洪亭上仙,那魔头扬言要屠戮了整座城池,除非我将女儿嫁给他做妾。他可是高高在上的蛊仙,我们只是一群凡人。我们实在没有办法了,还请您念在我和您父亲的交情上,出手除魔啊!”一位老城主找上门来,跪在地上,请求洪亭出手。

洪亭认得出来,这位城主的城池和枫叶城很近,平时来往也很频繁,的确是交情很深的。

甚至就连这位城主的女儿,他也见过,童年的时候一起玩耍过。

“老人家,你快快起身。这个忙我一定会帮,只是……”洪亭停顿了一下,“时机还未到。”

老城主大喜过望:“既然上仙您答应了,老朽也就放心了。老朽相信上仙绝不会食言的!”

洪亭等候良机,终于等到那魔头蛊仙应当命绝的时刻。

他果断出手,轻轻松松地铲除了魔头。

只是,老城主跪在地上,望着满城的废墟和尸体,悲哀大哭:“这天杀的魔头,你终于死了!呜呜呜,我的乖女儿,我的城民们,你们安心地去吧,你们的仇已经报了!!”

……

“徒儿拜见师父。不知师父召唤徒儿,有什么事?”洪亭来到龙公面前。

“徒儿,为师察觉了宿命,飘花江将于此刻突大水,河流改道。为师令你你前往救人。且记,不得提前出手,必须是三天三夜之后,才能出手。”龙公仔细叮嘱道。

“是,师父。”

洪亭来到飘花江畔,看着河水蔓延,无数生灵流离失所,溺亡的浮尸飘在泛滥的水面上。

他强忍住自己的情绪,足足等待了三天三夜。结果他现,竟然不需要他出手,河水已经自行退去,许多地方已经露出泥泞的地面。

一道野生仙蛊的气息升腾而起,竟然就在洪亭的下方不远处。

近水楼台先得月,洪亭轻而易举收服了这只野生仙蛊:“好蛊虫,竟然是七转宙道蛊,正适合我用呢。”

嗷呜!

被野生仙蛊的气息,吸引来两头上古荒兽。

洪亭面色郑重,隐藏身形,待得这两头上古荒兽自相残杀,一死一伤之后,他这才出手,将这两头上古荒兽一网打尽。

“妙哉,妙哉。”望着战场的痕迹,洪亭恍然大悟。

“原来这就是宿命的巧妙安排。经过这两头上古荒兽的激战,飘花江又扩宽了数倍,同时河道还被上古荒兽的血液浸染,变得厚重凝实。从此之后,飘花江的水灾恐怕不会再有了。”

“还有飘花江畔的这些土地,浸泡过江水河沙,渗透了无数生灵的尸骸精血,将会变得极其肥沃。未来迁徙到这里生活的凡人们真的有福了。”

……

“牛头蛮魔,你放了我的父母,否则我会令你求生不得求死不能!”洪亭怒视牛头魔仙,双眼简直要喷火。

牛头魔仙一手一个,抓住洪亭双亲,哈哈大笑:“小子,让你这么嚣张,想除掉我?我成为蛊仙的时候,你还不知道在哪里吃奶呢!现在怕了吧?”

洪亭投鼠忌器,气得大叫。

牛头蛮魔连忙叫道:“你别过来,你别冲动!你不想要你双亲的性命了吗?你再过来,我就把你父母的头直接捏碎!”

“你想要怎么样?!”洪亭大吼。

牛头蛮魔狞笑一声:“好说,好说。只要你把我的寿蛊还给我,我就归还你的父母。若不然,我也没几天好活的了,就带着你父母一块死吧。”

洪亭顿时愣在当场。

按照宿命的安排,那两只野生的寿蛊应当就是牛头蛮魔之物。但洪亭念及父母双亲老迈不堪,将近生命极限,便瞒着龙公,将这两只寿蛊夺走。没想到竟被牛头蛮魔找上门来,祸及双亲!

一时间,洪亭脸色惨白,满头冷汗:“你那两只寿蛊我已经都用了。”

“我知道,给你父母用了嘛!”牛头蛮魔却不意外,“可是你是龙公徒弟,将来的仙尊,要入主天庭的。我不信天庭库藏中没有寿蛊。你取来三百年,不,三千年的寿蛊过来,我就放了你的双亲!”

“这……”洪亭顿时陷入两难之中。

最终,一番周折,他终于将牛头蛮魔打跑。但是他的双亲都伤重难返,即便用仙家手段,也治愈不好。

“父亲、母亲,是我害了你们啊!若不是我为双亲找寻寿蛊,按照他们的年龄,还有好几年可活的!”洪亭哭泣,跪在地上,无助至极。

洪铸却带着微笑:“我儿,生死有命富贵在天,人总归是要死的,这个世间是有长生,但谁能永生不死呢?没有!我们注定是要死的,不必为我们悲伤。恰恰相反,你应该为我们感到高兴。我们夫妇能有你这样的儿子,未来的仙尊,这是多么大的荣耀,将来必定因你而名垂青史。”

洪夫人跟着道:“我的儿,好好听你师父的话,做一个好人,要一身正气。”

二老双手互握,一同没了气息。

“爹、娘!”洪亭仰头嚎哭,声嘶力竭。

……

岁月在洪亭的眼眸中积淀辉煌,沧桑和成熟使得他越加富有男人的魅力。

青山苍翠,夕阳余晖。

在晚霞下,他和柳淑仙第一次见面。

双目四对,他们双方彼此都感受到了心灵的悸动,一种妙不可言的氛围迅蔓延二人四周。

一见钟情。

柳淑仙面露奇异之色,轻启双唇问道:“你是谁?”

洪亭却毫无意外之色,他微笑答道:“我是命中的夫君,我的名字叫做洪亭。”

柳淑仙惊愕:“你便是未来的仙尊?”

“不要意外,你我在这里相遇,都是命中的注定。”

爱情,令洪亭感到一种前所未有的满足和幸福!

他和柳淑仙双宿双飞,情投意合。就好像他们天生就是为对方而生的那样,就好像是锁和钥匙,是天生的一对。

他们游走五湖四海,他们在明月下畅谈欢饮。他们默契至极,一个眼神就能表达自己的内心,并且将全部的心意传达到对方的心头。他们是真正的神仙眷侣,一同生活了成百上千年,在修行的路上两人相互扶持,没有一次争吵,没有一次恼怒。

柳淑仙一路陪伴着洪亭,成为八转,成为八转巅峰,最终冲刺九转至尊的境界!

灾劫难以想象,但最终洪亭仍旧成功了。

龙公因此伤重,数位天庭蛊仙陨落,洪亭抱住柳淑仙渐冷的身体,泪水横流。

他紧紧地抱住柳淑仙,不住地道:“不要离开我,不要离开我,求你,我求你活下来!”

“没有用的,我身中灾劫,此刻能弥留一丝魂灵,看你最后一眼,已经是天大的幸运了。我又岂能奢求太多呢。”柳淑仙笑道。

“是我没用,是我没用!我来渡劫,却连累了你!”洪亭垂,泪水滚滚。

“不,洪亭。那样的灾劫,只有我这样的特殊体质,才能去阻挡。你们就算付出全部身家性命,也只有失败的结果。我能出生,拥有十绝体,和你相遇,都是宿命的安排。在你九死一生的那一刻,我忽然明白,我此生最大的意义就是保护你,替你阻挡灾劫,助你登临仙尊之位!现在……我做到了。”

“不,不!仙儿,我宁愿不要这仙尊之位,我只想你活着,我只要你活着啊!”洪亭无助地嘶吼着,泪流满面,浑身颤抖。

“万事万物皆有定数,都有各自的宿命。洪亭,你不能这么想,你要好好活下去,你的宿命就是成为仙尊,领袖天庭,将正道的光辉照耀五域……你知道么,我一直期待着这样的情景,期待着站在你的身侧,陪伴你无敌天下,造福世间。可惜,我看不见了……”

柳淑仙气息渐消,彻底死去。

洪亭垂,腰背深深弯弓,宛若老朽,浓重的阴影笼罩住他的面庞。

这一刻,他像是失去了全部的生命的气息。

他凝如石像,一动不动。

在他的仙窍中,残余的天地二气不断汇聚,依照他此刻的心境,炼成一只八转的仙蛊。

这只蛊形如蜈蚣,浑身苍白,仿佛纸质,蜈蚣有百足,但它的百足却被百须取代。每一根足须,都是半透明,悠悠飘荡,挠着人心,不断地撩起人内心最深处的后悔之情。

八转悔蛊!

“咳咳咳。”龙公吐出几口鲜血,勉强起身,来到洪亭的身边。

“柳淑仙死得其所,所以收起你的悲伤吧,徒儿。这都是宿命的安排。如今的你已经成为仙尊了,九转尊者,漫漫历史长河中都是屈指可数啊。你的命还很长,你的使命才刚刚开始,我将退位,天庭还有五域两天都需要你。洪亭,洪亭?”龙公轻声唤道。

洪亭缓缓地抬起头,他没有看向龙公,目光仍旧停留在柳淑仙冰冷的面庞上。

他轻声地道:“我后悔了。”

\end{this_body}

