\newsection{方源vs武庸}    %第一百九十八节:方源vs武庸

\begin{this_body}

方源缓缓迈开步伐,从乔家蛊仙之中,走了出来。∷,

一时间,全场无数道目光,都集中在了他的身上。

方源刹那间成了全场的焦点。

看到他的面貌,一些武家的太上长老,都不由地低呼:“真像啊。”

方源的相貌,不,武遗海的样貌,长得随他的亲身母亲武独秀。这一点,毫无疑问,是方源的优势。

武庸就不一样,他长的随其生父,眉眼耸搭,比较敦厚,天生就有一种不起眼的感觉。

但武遗海不一样。

他面容英武,体格高健,挺拔而立,宛若一柄战枪刺向天空。他虽然在东海生活,但浑身上下的气质,近乎武家蛊仙,尤其是他的鼻子,鼻翼宽大,鼻梁挺拔,一眼让别人看上去,就觉得此子非凡不俗,心性刚强,不容轻侮。

当然,这其中也有方源可以表现出来的原因,尤其是气质方面。

真正的武遗海,反而是没有这种武家儿郎的气质的。

八转仙道杀招见面曾相识!

盗天魔尊的招牌手段之一,在此刻重现江湖,一干蛊仙都被蒙蔽了双眼,觉察不出方源的猫腻。

事实上,就连武家的蛊仙们,也都是第一次见到武遗海。

不得不说,方源之前的准备和努力,得到回报。武家蛊仙们在第一眼看去,都涌起了不少好感。

因为武遗海长得像武独秀。

武独秀本身并不好看,女生男相,但她的相貌若是摆在一个男儿身上,绝对是顶天立地的豪雄汉子。

武遗海至少有七分相似。

武独秀去世,武家动荡,家族蛊仙们都留念武独秀生前的辉煌。

这人呐。得到的时候往往不觉得,不珍惜,但是一旦失去了,就倍感珍惜和可贵。

方源在这样的关头出现,恰恰弥补了众仙的这种心理。

当然,这也是方源算计到的结果。

“像。真像武独秀啊。”就连其他家族的蛊仙,都感叹起来。

方源敏锐地感觉到,这些集中在他身上的目光的潜在变化。

五百年前世,或许没有让他获取多高的修为成就,但是打熬了五百年,早已养出了他深不可测的处世经验,还有对周围蛊仙情绪变化的敏锐洞察力。

他感觉到武家蛊仙们的目光,温和起来,而武庸的目光。却隐含一些敌意。

武庸的确有些意外。

方源伪装成的“武遗海”在这个时候出现,有些打乱了他的计划和安排。

尤其是让武庸在意的是,方源和乔家蛊仙走到了一起,这让他心中升起了一股警惕。

其实,就内心最深处而言,武庸有些讨厌自己的这个弟弟。

虽然双方之间,只是第一次见面。

第一眼,武庸就不待见方源。

因为武遗海的相貌。太像他的母亲武独秀。

多少年了。

武庸跟随在武独秀的身边,在她的庇护下成长。

武独秀像是参天巨木。大树浓密的绿荫庇护着整个武家,当然也包括武庸。

武庸感觉自己总是低着头,大树的绿荫何尝不是一种阴影,笼罩他的人生?

他是武独秀的儿子,但别忘了,同时他也是一个男人。

男儿。天生就有追逐力量、权势的强烈**。

但至始至终,武庸的头顶上是武独秀,她的力量、她的手腕,尤其是她的身份,都是武庸无法反抗。只能低头的原因。

武独秀去了,武庸非常伤心和悲痛,但同时,也不可避免的,他还有一些轻松,甚至是……欢喜。

这种欢喜,恐怕就连武庸自己都不敢承认。

而现在,当武庸见到外表相似武独秀的武遗海时,他心中就不可避免地,翻涌起许多的厌恶的情绪出来。

方源察觉到武庸的恶意和敌意,尽管后者掩饰得很好。

方源心中清楚:“第一步亮相成功,但第二步和武家交涉,被武家正式接纳,才是关键。”

“而被武家接纳,最主要的就是要得到武庸,这位八转蛊仙,武家的太上大长老的承认!”

所以,当方源走到场中之后,他就主动向武庸深深一礼:“武遗海见过兄长。”

武庸表现得有些激动,又有些迟疑:“你就是我弟遗海吗?像,真像母亲啊,只是眼瞳却是蓝色,咱们南疆没有这种瞳眸呢。”

这句话很有深意。

武庸并没有直接承认方源的身份,反而似乎无意地点出“武遗海”东海蛊仙的身份。

方源沉声答道:“启禀兄长,我原本也是黑瞳,只是修行变化道,出了一些意外。这一次回来,是母亲安排。可惜的是,路上遭遇多次伏击,陪同我的张叔和冷兄弟,都为了护我逃脱,牺牲在了战场上。”

“张、冷二位蛊仙,虽只是我母亲的侍奉,但对武家贡献卓绝。传令下去,要将他们厚葬,依照我族外姓太上家老的规格。他们的后人,也要大力照顾,与我武家内部一视同仁。”武庸开口,吩咐左右。

太上家老们连忙应声。

方源闻言,心中却是冷笑。

这武庸不放过任何一个机会,展现自己的英明形象,但言语间,却从未正式接受方源的身份。说话避重就轻,只谈论张、冷两位蛊仙,却忽略了应该有的重点他的弟弟武遗海。

武庸避重就轻,方源却没有手足无措。

这点小困难,难不倒他。

方源接着一礼,道:“恳请大兄允许,我上前祭奠母亲大人!”

武庸不禁眉头一挑。

方源的话,合情合理,生为人子,母亲逝世,怎可能不去祭拜?

这个理由天经地义,尤其是以血脉关系维系的家族制度,更是如此。就算是武庸,也不好阻止。

一旦强行阻止,就会被人诟病,他刚刚辛辛苦苦,好不容易建立起来的新形象,就会彻底垮台。

但如果不阻止的话,让方源祭拜了,这就表明整个武家认可了方源的身份,承认他就是武家的一份子!

说实在的话,武庸并不愿意。

武遗海的相貌,让武庸反感。

武遗海的身份,让武庸头疼。

武遗海和乔家一起出场,让武庸警惕!

“最好让这个武遗海,从哪里来,滚回到哪里去。他不是东海蛊仙吗?就让他去东海发展好了,哪怕丢一些资源打发他走。”

这是武庸最想达到的结果。

不得不说,方源的表现出色,还有乔家蛊仙作为另外一种佐证,让武庸也没有怀疑武遗海的真实身份。

“等一等,若是我将这武遗海除掉呢?”

“之前不行,是因为母亲还在。但如今我已经是太上大长老。这里是我的地盘,假以时日,掌控家族,暗中布局,将他秘密处死,不就一了百了了吗?”

“就算他在东海,也是个麻烦啊。若不在东海,回到南疆,更是麻烦。”

“不妨现在答应他,让他祭拜就是了。待到日后,做些手脚,证明他是假的。除掉他,也是光明正大!”

想到这一点,武庸心中不由地泛起了凛冽的杀意。

他刚要张口,直接答应方源的要求,这时却有武家的太上长老出声道:“恕我直言,诸位是不是忘了,我们还未验明武遗海二公子的真正身份。不是我怀疑二公子,更不是怀疑乔家诸位同道,而是既然要认祖归宗,这个过场绝可不少!”

武庸闻言,不禁再次挑了挑眉头。

他深深地望了一眼发言的武家蛊仙,后者便是武家的太上三长老武樵。

方源则不动声色地和乔家太上大长老对视一眼,心中则道:“我联络乔家是正确的,只是没想到乔家已经对武家渗透得如此厉害,武樵可是太上三长老啊。”

乔家太上大长老,望着方源,不禁回想起之前两人密谋的情景……

“只要我在祭典上公然出现,武庸就会陷入被动。就算他是八转蛊仙,伸伸手脚,就能轻易碾死我等,又能如何?呵呵呵。”

方源笑着,继续道:“他是不会动手的,也不可能动手。因为他是武家的太上大长老,而不是散修或者魔道中人。届时,我再提出,要祭拜母亲大人。武庸能阻止我吗?”

“哈哈哈,妙,实在是妙!”乔家太上大长老竖起大拇指,“私下里认祖归宗,危险太大了。这样处理,就相当巧妙,就算武庸不愿意,又能如何?”

方源正色:“所以,我们一定要在祭典上,正大光明地进行认祖归宗。一旦错过了这个机会,可就难了。不仅风险会增加无数,甚至武庸根本不给我们这种机会。”

“放心吧。届时,我会开口,提出这个建议的。”乔家太上大长老笑呵呵地道。

“不不不。”方源却摇摇头,“你提出这个建议,很不合适。你是个外人,怎可以干涉武家的内政?”

乔家太上大长老陷入沉吟之中。

方源的话,深深地击中了他的内心。这正是困扰他的巨大难题。若非如此,他也不能和“武遗海”合作了。

方源背负双手,微昂头颅,看向窗外:“所以,我需要一个武家的太上家老,他能够在那种场合下,公然地提出这个意见,将事情导向到认祖归宗方面。”

乔家太上大长老有些为难:“这个……”(未完待续。)

\end{this_body}

