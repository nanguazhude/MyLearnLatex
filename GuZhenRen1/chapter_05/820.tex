\newsection{尊者真传当筹码}    %第八百二十四节:尊者真传当筹码

\begin{this_body}

%1
“尊者真传?”东海二仙瞪起了双眼。

%2
龙公咬牙:“敢问是何真传?”

%3
“幽魂真传。”方源干脆地道,“龙公仙友若是不信,我完全可以透露一二。”

%4
龙公沉默了一下,点头:“我信。”

%5
同时,他心中沉思:方源这魔头果然是好算计!魔尊幽魂已经被我天庭俘虏,假以时日,搜索出幽魂真传,完全不是问题。这份真传的价值对于方源已经很低,所以他拿出来交易。但是气海老祖没有魔尊幽魂这样的俘虏啊,为之心动,也是理所当然。

%6
方源继续道:“龙公仙友,若是想要我罢手旁观,难道也出得起尊者真传吗?”

%7
龙公再咬牙:“这是自然。”

%8
“真的?”方源眼里放光,“说心底话,我平生对元始仙尊是最为崇敬的。”

%9
龙公嘴角顿时有抽搐的迹象,不禁腹诽着:这老东西也是狡诈成精,说是对元始仙尊最为崇敬,为什么刚刚交手,招招狠辣,更是用了偷生,完全是想置老夫于死地!不过他专修气道,觊觎元始真传也是情理当中的事情,正可为我所用。

%10
“仙友不知,元始真传最大的一项便是天庭洞天,其余的内容其实十分稀少,又因为时代变迁,早就不合时宜了。”龙公故意拿捏道。

%11
方源皱起眉头,失望的情绪流露出来:“这么说,元始真传是拿不出了?那天庭能拿得出什么呢?星宿真传,还是元莲真传?”

%12
不管什么,都对方源有利。

%13
龙公呵呵一笑,又给方源些许希望,模糊地应道:“这些事情都好商量。”

%14
东海二仙面面相觑,暗中大叫糟糕,龙公家大业大,为了笼络气海老祖竟然不惜尊者真传!

%15
最麻烦的是,这些东西沈家、宋家可都拿不出的。

%16
龙公主持天庭,实在是财大气粗,直接拿资源、拿尊者真传来“砸”气海老祖,想要把气海老祖“砸倒”。

%17
这是天庭积累了无数岁月的底蕴,深厚得沈家、宋家两位当家都为之气沮。

%18
方源却是忽然发现,龙公其实有时候看起来还蛮顺眼的。对于他这种行事风格,方源背地里简直要竖起两个大拇指。

%19
老哥,讲究!

%20
当即,方源和龙公深谈下去。

%21
他反正是要拖延时间,给龙人分身那边争取机会。

%22
一旁的东海二仙,只能干看着,心中焦躁,却是无可奈何。

%23
方源当然不会忘记这两人,在谈话中时不时地和他们俩搭话,让这两人始终保持着希望。

%24
这种搭话的次数一长,龙公当然看出端倪,目光越加深幽。

%25
他心知:气海老祖是在利用这两位东海八转,为自己争取利益!但也无妨,只要气海老祖肯耐心谈判,就是好现象。

%26
“只是古月方正却是落入沈从声的手中,此事有点麻烦,我要将他赎回来才是。”龙公一面和方源交谈,一面心中琢磨着。

%27
方源也在琢磨着古月方正。

%28
之前,东海二仙就提到了古月方正,只是方源演技太好,毫不在意的样子,完全像是个局外人,一点破绽都没有。

%29
但方源心中也有计较:“我这便宜弟弟,不仅被天庭救走,而且还受到了栽培,成为了蛊仙?有意思!”

%30
“中洲天庭既然如此重视他,必有缘由。具体原因是什么我不知道,但没有关系,待会将他索要过来,直接杀了了账。”

%31
对于方正,方源早些年时是想炼制血神子,后来因为天庭施压,种种缘由身不由己,让方正在琅琊福地中放养了许久。

%32
结果意外发生,方正被凤九歌救走了!

%33
这虽然也谈不上失误,但方源此次既然发现,便决定弥补。

%34
方源对于方正,其实并不重视,但既然天庭重视了,那就解决掉。这方面方源一直都很晚稳妥。

%35
至于血道的发展,当时是在万不得已的情况下,为了提升自己的战力。但此一时彼一时,现在方源对于血道的发展兴趣,早已经很低了。

%36
只要诛魔榜不除,修行血道,就是将情报主动送到天庭,告诉他们自己在哪里。

%37
有时候,方源猜测,上一世自己修行血道是否也是天意的影响,挖了一个坑提前留给自己。

%38
幸而自己一直小心谨慎,没有贪图上一辈子的拿手流派,没有求一时之快。

%39
龙公越谈,精神越是振奋。

%40
他看得出来,方源也是有诚意,他对尊者真传非常感兴趣。

%41
“也是,这气海老祖实力高绝,放眼天下,也就尊者真传或许能入他的眼界了。东海居然潜藏着这样的人物,事后定要好好查探一番他的跟脚来历。”

%42
气海老祖其实就是方源,而方源不仅要从他的手中谋求天庭的尊者真传,还对古月方正动了杀念。

%43
龙公此刻若是知道这一点,保不齐要当场吐血。

%44
龙宫梦境。

%45
书房中气氛凝重。

%46
扮演吴帅的方源分身,脸色难看地望着父亲,语气干涉:“关于非议峰的归属,真的要判给那范极吗?明明是我大获全胜,他见缝插针。若非我手下留情,他早就死无全尸了。”

%47
龙人蛊仙缓缓摇头:“非议峰利益极大,牵扯很深,黑天寺那边毫不松口。更让人无奈的是,我们也联系了书道阁主、绿蚁居士,但这两位却都是推托,没有帮助我们出面施压。”

%48
“可恶!”吴帅捏起双拳,“我乃是绿蚁居士的关门弟子,如今受了委屈,这位当师父的却是不闻不问。”

%49
“更可恶的还是书道阁主!我是她的女婿,而范极更是祸害她女儿的人,她居然连一句话都不说。”

%50
龙人蛊仙叹息:“八转蛊仙岂是我们能影响的?这一切都是为父的错,为父当年太过想当然了。事实上,我们的意图绿蚁居士、书道阁主真的不知吗?只是故意装作糊涂罢了。”

%51
“所以说啊,父亲!”吴帅盯着龙人蛊仙,目中显现精芒,“唯有我族的八转蛊仙,才真正可靠。我们龙人一族能够依附人族,发展至今,所依仗的不都是龙公老祖宗的恩泽吗?”

%52
吴帅之父点头:“我儿,你说的是。但八转蛊仙真的太难成就了,我们龙人虽然在其他方面强于人族,但是灵性方面却是不足的。现如今我们的七转蛊仙有很多,但八转蛊仙却是除了老祖宗外,没有一人啊。”

%53
吴帅目光炯炯:“所以,我们更不能坐以待毙。外在的八转蛊仙不可靠,我族又暂时没有其他八转蛊仙,那么我们可以用仙蛊屋来顶替啊。”

%54
龙人蛊仙再次叹息一声,微微摇头:“仙蛊屋?这个想法我们早就有了,但一来要打造出七转仙蛊屋,资源方面实在是海量,令人望而却步。二来我们龙人一族势力见涨,已经惹来诸多非议,越来越多的人族蛊仙都盯着我们。龙人威胁论调是越来越多,这种情况下,我们若还要建造仙蛊屋,岂不是激得这些人直跳脚?”

%55
“父亲!”吴帅双眼微瞪,掷地有声地道,“难道因为区区资源、众人非议的缘由,就不去建设仙蛊屋了吗?”

%56
“这些缘由难道还不够吗?我儿啊,咱们龙人一族可是依附着人族的,这层关系是我们的依靠,不能鲁莽。若是我们真的要建设仙蛊屋,破坏了这层关系,别说是建设七转仙蛊屋的资源,就算是建设一座六转仙蛊屋,也恐怖没有希望。唉,这些年来,为父也算是深居高位。这各种排挤的滋味和门道,你是不了解的。”龙人蛊仙一个劲地叹息。

%57
吴帅心中一阵失望,据理力争道:“不管千难万难,这件事情都要去做。并且寻常的七转仙蛊屋,还远远不够,要造就要建造一座八转的仙蛊屋来。唯有如此,才能真正镇压我族的底蕴!”

%58
“八转仙蛊屋?哈哈,我儿,为父看到你如此志气,真是欣慰。但是饭要一口口的吃,路要一步步的走啊。唉,非议峰的事情咱们就算了吧。”龙人蛊仙摆手道。

%59
吴帅沉默半晌,点头:“也只有如此了。”

%60
一幕梦境接着一幕,方源分身想要拖延时间,期待本体拯救,但奈何本体也在拖延时间,期待分身成功收取龙宫。

%61
龙宫梦境仿佛是流沙,方源分身身不由己,逐渐沉溺当中,不能自拔。

%62
他仿佛化身成了当年的吴帅,经过过去曾经发生的一幕幕。

%63
吴帅自从非议峰一事后,终于认清现实。他的师父绿蚁居士不可靠,他的泰水书道阁主也不可靠,甚至他的父亲龙人蛊仙的软弱,也暴露在他的面前,还是不可靠。

%64
真正的可靠的是自己啊!

%65
吴帅有了这样的明悟,从此奋发图强,励精图治。

%66
他有卓绝的天资,又有充沛的才情,再加上刻苦勤奋,又有着人脉资源,不仅自身修为节节攀高,管理的一部分龙人部族也是日渐壮大,成为龙人一族中最显著的一道分支。

%67
当吴帅晋升七转的当天,他便成为了龙人一族的高层,最年轻的实权太上长老!

\end{this_body}

