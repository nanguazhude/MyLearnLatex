\newsection{盗天的底线}    %第四百八十节:盗天的底线

\begin{this_body}

沙枭给了少年盗天一份蛊方,但后者却是勃然变色,坚决不肯依照蛊方行事。

沙枭阴森森的声音传来:“孙子,不听爷爷的话了?你还想不想活命了?”

少年盗天却很坚持:“谁不想活着?但我活着,是有底线的。这种蛊方太过阴损,我坚决不做。沙枭,你不是要让我接近那处池塘吗?如今我已经闯入八强,有资格进入其中,你现在杀了我不觉得可惜吗?”

“哈哈哈。”沙枭大笑,“是有点可惜了。”

话音刚落,少年盗天瞳孔猛地一缩,只感觉一股强烈的痛楚袭遍全身。

痛痛痛!

刹那间,他差点要倒在地上打滚。

但他旋即就拼尽全力忍住,赶忙跑到一处无人的墙角,背靠在墙根处,咬牙切齿,忍耐再忍耐。

这时候沙枭的声音又传来:“你不过一个小小的棋子,闯入了八强,就真以为能和我谈条件?你有那么高的利用价值吗?哼,想当我孙子的人,不知道有多少!你能喊老夫一声爷爷,是你的荣幸,你知道吗?”

说到这里,沙枭顿了顿,又道:“现在叫我一声爷爷来听听。”

少年盗天满脸扭曲之色,双手死死握拳,十指指甲都嵌进肉中。

他浑身都在颤抖,冷汗直冒,这既是痛楚所致,也是他心中强烈的屈辱感和愤怒,难以发泄。

“爷、爷爷……”好半天,少年盗天这才扭着眉头,艰难地吐出声来。

“嗯,乖孙子。你声音太小了,你爷爷我年纪大了,还未听清楚呢。”沙枭阴测测大笑。

少年盗天顿时感觉身上的痛楚变轻了许多。

原来,沙枭动手,放缓了手段。

少年盗天眉头倒竖,眼中闪烁着死志,但他又想到自己的家,自己所爱的女子,他深呼吸一口气,强忍住,又唤道:“爷爷好!”

“哈哈哈,只要乖孙子听话,爷爷就什么都好。[\&\#26825;\&\#33457;\&\#31958;\&\#23567;\&\#35828;\&\#32593;\&\#119;\&\#119;\&\#119;\&\#46;\&\#77;\&\#105;\&\#97;\&\#110;\&\#104;\&\#117;\&\#97;\&\#116;\&\#97;\&\#110;\&\#103;\&\#46;\&\#99;\&\#111;\&\#109;”沙枭的笑声中充满了得意。

“现在,爷爷要让你炼蛊,你做得到吗?”

“做……做不到!”少年盗天痛得脸色惨白如纸,毫无血色,但却仍旧坚持自己的底线。

“混蛋!”沙枭大怒,猛地加重手段。

少年盗天呜咽一声,直接顺着墙角倒下去。

折磨了好一阵时间,沙枭见少年盗天始终不肯妥协,也感到颇为头疼:“臭小子,你还挺有正义感。可惜可惜,我见你打斗时干脆利落,还以为你是个可塑之才。没想到,你到现在还不理解‘人不为己天诛地灭’的道理。死个婴孩算什么?他不死就是你死,你死了什么财富、*都成了空。你好好想想。”

少年盗天已经蜷缩成了一个虾米,浑身都被汗水打湿,他几乎痛晕过去,此时此刻瘫倒在地上,一丝力气都没有。

不过,他的嘴角却是微微翘起:“沙枭,你死了这条心吧。我和你不同,我绝不把自己的快乐和幸福,建立在别人的痛楚之上。”

“可笑的坚持!这世间很多东西,都需要去争夺,去屠戮,你不争不杀,怎可能活得下去?你脑子里究竟怎么想的?”沙枭声音冰寒,不屑至极,“你再想想你为什么会被部族流放?呵呵,还不是因为部族资源太少,培养你觉得浪费资源,所以就用部族中的规矩,光明正大地将你谋杀了么?”

少年盗天眼中精芒一闪:“这个道理我现在明白了,不过别人做是别人做,那是别人的事。我若做了,那就是我的事。我绝不会这么做!”

“冥顽不灵!那老夫就杀了你!!”沙枭暴怒,终于到达极限,不过正当他要动手时,忽然一怔。

“我可怜的儿啊,你怎么就这么去了啊!”少年盗天此刻倒在墙角,忽然听到了房屋中女主人的哭嚎声。

随后,房屋中的声音又嘈杂起来。

有人跟着哭,也有人在怒吼大骂医师无能。

少年盗天和沙枭细细听了一阵,明白了缘由。

原来此处房屋主家,有一个婴孩,刚刚诞生不久,但先天不足,命垂一线。主人家就算花了重金请了一位三转蛊师出手,也未救得婴孩性命,终于在这一天婴孩丧命。

“呵呵,你这小子,气运似乎不俗嘛。这个婴孩足够炼蛊,已经死了,你去不去收尸用来炼蛊?”沙枭笑起来,杀意滚滚,“你若不肯答应,现在就死吧。老夫大不了再换一个棋子罢了。”

少年盗天沉默良久,这才开口:“我答应你。”

“哼,这才对。”沙枭说完,不再开口。

少年盗天缓了好半天劲,这才勉强站起身来。

经此一事,少年盗天彻底认识到,沙枭这个魔头性格强硬至极,说一不二,不能容忍任何的忤逆。

为了回家,少年盗天决定忍辱负重。

按照部族的惯例,小比暂停。

剩下的少年,几乎都足不出户,在家里接受长辈们的悉心指导,临阵磨枪。

接下来的几天里,少年盗天也着手收集蛊方中记载的蛊材。

收集的过程,虽然磕磕碰碰,但好歹是完成了。

那个婴孩的尸体,则是等到主家下葬之后,被少年盗天偷偷在夜里掘坟挖出来的。

距离小比再次开始,还有两天时间。

这天夜里,四下无人的偏僻小屋中,少年盗天开始炼蛊。

“这蛊方本来十分简练,但爷爷我念你这个孙子真元太少,便只得添加了许多步骤。爷爷用心良苦,孙子你这一次炼蛊,切勿失败!”沙枭的声音传来,他要临场指导少年盗天。

炼蛊正式开始。

在沙枭的指导下,少年盗天按部就班,开始逐一处理蛊材,积极炼蛊。

因为的确是第一次炼蛊,少年盗天在刚开始时,表现得极不适应,单单用火处理蛊材,这么一个最简单的步骤,都屡屡出差错,若非沙枭及时指点,让少年盗天改正过来,这一次炼蛊早就失败了。

沙枭气得破口大骂。

不过随着炼蛊进行下去,沙枭的咒骂次数越来越少。

“这就是昔日魔尊的天资吗?!”一直旁观的方源,也被震撼到了。

少年盗天展现出了他极其强大的炼蛊天赋,适应很快,从一开始磕磕绊绊,到中途站稳脚跟,然后就变得四平八稳,到了炼蛊后期步骤,他甚至由了一丝从容不迫的态势和气度。

“还剩下最后四步,你这蛊虫就能炼成了。”沙枭居然罕见地鼓励起少年盗天,“乖孙子,好好干,把这蛊虫炼成了,然后在小比中为你爷爷我涨脸!”

少年盗天深呼吸一口气,目光投注在身前的那具婴孩尸体上。

他的眼眸中深处,闪过一抹决断。

“虽然这个婴孩不是死于我手,但我若用他来炼蛊,于心何忍?反正炼蛊会容易失败的,我在这里失败了,沙枭也怪不到我头上来。”

“孩子,算我对不住你,你死了,也不让你安息。但我实在是没有办法,只能做到这里。祝愿你在这火焰中焚化,灵魂能升上天去。”

“若你能投胎转世,请你不要再回来。这个世界……很残酷!”

心中念叨着,少年盗天便步骤一错,顿时火焰暴涨,一下子吞没了婴孩尸体。

“火势太大了,快弄小下去。”沙枭惊道。

“我在努力!”少年盗天装作慌忙的样子,立即动手,却反让火势更旺了几分!

“好大的狗胆!你这是故意失败,与我作对!!”沙枭眼光狠辣,少年盗天的把戏没有骗得住他,一下子他就明白过来。(未完待续。)

\end{this_body}

