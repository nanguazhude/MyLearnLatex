\newsection{上古云兽,福极祸至?}    %第二十三节:上古云兽,福极祸至?

\begin{this_body}

%1
------------

%2
中洲,地渊。

%3
影无邪望着眼前的塌方,脸色难看至极。

%4
在他身后,是黑楼兰、太白云生,以及七转异人蛊仙石奴。

%5
四位蛊仙战力!

%6
其中,自然以石奴最强,黑楼兰次之,太白云生再次,影无邪还未脱离仙僵身份,只有借来的几只仙蛊,方源残留的仙蛊还未收复,目前战力垫底。

%7
黑楼兰已经被影无邪直接逼降,摆脱了和方源建立的盟约,成为影无邪的助臂。

%8
石奴对影无邪最为忠心耿耿。

%9
而太白云生,虽然并不知情,但他从未怀疑过影无邪。黑楼兰、影无邪联手哄骗,使得他一直被蒙在鼓里。

%10
“这是怎么回事?”太白云生看着眼前,不解发问。

%11
石奴沉声答道:“地渊中并不平静,时而地力汹涌澎湃,就会造成震荡。震荡有时候会引起超级塌方,就像我们现在看到的这样。”

%12
黑楼兰眼蕴异色:“这种塌方,未免也太过于恐怖。”

%13
石奴点点头:“是的,地渊中的每一次塌方,往往涵盖数十亿亩,造成数百万,上千万的生灵惨死。塌方之后,亿万斤的土石堆叠在一切,结合惨死生灵的怨恨、死气、精血,还会在其中形成无数荒级血兽、骨兽,更有血道、土道的野生仙蛊孕育而生。主人,我们接下来该如何行事?”

%14
影无邪凝视着眼前,半晌后,才缓缓吐出一口浊气。

%15
他心中郁郁。暗道:“这条路线,乃是影宗之前秘密经营。脱离地渊最为安全隐秘。我刚想要离开地渊,居然在这个节骨眼上。意外地发生了地渊塌方。偏巧,还发生在这里,正好切断了我们前进的道路。呵呵。”

%16
影无邪肚中冷笑。

%17
“这当是天意作怪了!”

%18
“本体逆天炼蛊,抗衡天意,导致天意剧烈损耗。但现在看这样,恐怕天意已经恢复大半。”

%19
“我身上还有春秋蝉,里面有不少天意存在。因此在天意感知中,宛若黑暗中的火炬!”

%20
“我若是侦查手段不济,没有故意缓了一步。现在恐怕要被困在塌方里面。或者贪图这里将要形成的野生仙蛊,恋栈不去,保不齐会被其他赶来探查的蛊仙发现。说不定我之前动用蛊阵,一番动静早已经被天庭方面有所察觉怀疑,开始四下搜查了。”

%21
影无邪得到红莲魔尊真传中的本体意志的指点,对天意了解甚深。

%22
天意也有极限,也会被剧烈损耗。但因为根基是五域天地,所以天意不会被彻底消灭,恢复能力也是极快。

%23
关于天意这点。方源还被蒙在鼓里,毫不知情,远远比不上影无邪。

%24
“天意如此难缠,很大程度上还得归功于星宿那个贱人!”

%25
“我这身躯相当平凡。必须要炼出定仙游,才能顺利穿梭五域。再将各地影宗残余收拢起来,形成一股可观的力量。然后尝试拯救本体。”

%26
念及于此,影无邪自然又想起了方源。

%27
心中再次翻涌起绵绵恨意!

%28
“当初设计至尊仙胎蛊方。本体就考虑到了五域界壁。因此,方源现在不仅可以跨越五域界壁。视若无睹,还可以自由转换气息,变成任意的一域蛊仙!”

%29
“不过……你的日子也别想好过。你虽然是完整的天外之魔,天意不能再影响你的思考,但天意却可以影响其他存在,布局除你。”

%30
“还有,将来你每一次渡劫,都是天意铲除你的最好机会。”

%31
“希望你能捱过来,等到我抽身铲除你的时候!”

%32
影无邪虽然恨极了方源,但他现在最要紧的,还是拯救本体残魂。

%33
如今他的本体残魂,陷落在梦境之中。

%34
众所周知,魂魄困于梦境,会被梦境逐渐侵蚀消磨。所以,留给影无邪的时间,也很紧迫。

%35
和拯救本体相比,对付方源,就是一件小事了。

%36
所以,影无邪等人现在最要紧的,就是赶炼出定仙游,收拢残留势力,再回到南疆,将超级梦境周围的南疆蛊仙封锁打破,营救出魔尊幽魂。

%37
南疆。

%38
仙蛊剑遁发动!

%39
嗖。

%40
方源身形如剑,锋锐至极,穿刺长空,在他身后拖出一道长长的白色云尾。

%41
光是动用剑遁仙蛊,自然没有这个显眼的云尾。

%42
问题是,方源此时身上伤势沉重。

%43
深可见骨的伤口上,凝结着浓郁的云道道痕,正不断地向外翻涌出腾腾的白色云气。

%44
方源的脸色也是苍白如纸,眉头紧锁。

%45
而在他身后,十里开外,一大群的上古云兽正朝着他追杀而来。

%46
造成方源一身伤势的,正是这些太古云兽。

%47
方源虽然有七转飞剑仙蛊,但对这种聚散如意,和泥怪仿佛的云兽,最是无奈不过。

%48
“据说薄青当年,创下仙道杀招万剑劫,就是以飞剑仙蛊为核心,以一化万,形成剑雨似的磅礴攻势。我回到琅琊福地之后,一定要寻找到类似的仙道杀招,再也不吃这样的亏!”类似的内容,早已经在方源脑海中翻腾了不知多少次。

%49
原来,方源自从发现新躯的秘密之后,临时起意,果断抛弃之前的计划,想要穿透界壁,赶回琅琊福地。

%50
但不久之后,他居然落进了一群太古云兽的攻击范围之内。

%51
这种云兽,不管在南疆,还是在其他四域,都相当罕见。用“绝迹”一词形容,也不夸张。

%52
但方源却偏偏碰到一群云兽,甚至还都是上古云兽,每一头战力都能媲美七转蛊仙。

%53
上古云兽不动的时候,宛若飘飘白云,根本难以察觉。

%54
方源尽管一直在动用侦查手段,但到底没有相关仙蛊,只是凡道杀招而已。因此一头撞进上古云兽的警戒范围。

%55
一番激战后,惊险逃脱,但上古云兽并不打算放过方源,紧追不舍。

%56
视野中,忽然从山峦中升腾出一线的紫黑之色。

%57
南疆的瘴气界壁,已经遥遥在望!

%58
方源大喜。

%59
速度却骤降下来。

%60
原来是他主动停下了剑遁仙蛊。

%61
这仙蛊催动起来,代价太高了。如今方源已经将琅琊派贡献,彻底耗尽。

%62
节约起见,方源只能时断时续地催动剑遁仙蛊,和身后的云兽群保持一个安全的距离。

%63
没有剑遁仙蛊相助,方源就催起凡道杀招。

%64
说实话,这种移动速度也很快,可惜不能和上古云兽群相比。

%65
片刻后,上古云兽群就已经大大拉近了距离,和方源一里不到。

%66
眼见云兽追近,方源被逼无奈,只好再次催动剑遁仙蛊。

%67
嗖!

%68
他速度再次激增,迅速甩开上古云兽群。

%69
紫黑色的瘴气界壁,渐渐占据了他的大半视野。

%70
终于,方源飞到南疆界壁面前,没有一点犹豫,他一头扎进其中。

%71
毫无阻碍!

%72
仿佛满眼的瘴气界壁,只是单纯的光影幻觉。

%73
方源心头大喜,振奋地双手捏拳:“果然!我的猜测是正确的。这至尊仙胎蛊真是奥妙至极!今后,我就算没有定仙游,也能在五域中自由穿梭了。”

%74
当然,方源想到的,还不只这些。

%75
“这一点,还能帮助我战斗。之前和戚灾一战,就是个绝佳的战例。今后我可以将蛊仙强敌,都引进界壁中,再出手斩杀!”

%76
但他的兴奋,只是持续了很短的时间。

%77
很快,他就再次眉头紧锁。

%78
因为他发现,身后的那群上古云兽,也穿梭界壁,针对他的追杀没有停息。

%79
原本的猜测被验证,方源不禁哀叹一声:“果然如此,我竟然如此倒霉啊!”

%80
“南疆中的云兽可称绝迹,就算有,也是一两头的荒级云兽。这种成群结队的上古云兽,只有白天、黑天中才有。”

%81
“看来这些上古云兽,洁白如雪,是专门从白天中飞下来的,目的就是为了繁衍下一代。”

%82
其实,云兽本来就是太古九天中的奇妙生物。五域中的云兽,也都源自太古九天。

%83
云兽有个习性。

%84
就是每隔一段时间,它们就会飞下太古九天,落到五域中,借助天空中的云朵来寄存云精。

%85
被云精寄存的云朵,只要不遭受强劲外力,就会经久不散。

%86
数百年之后,这些云朵就会转化为云兽。

%87
新生的云兽,每一头都有荒兽级数。它们刚一出手,就会遵循生命的本能,朝高空升去。

%88
穿透天罡气墙,回归到各自的源头太古九天中去。

%89
在这个过程中,云兽会被天罡气墙阻挡,会被沿途的猛兽猎杀,会被种种自然气象等等陷害。真正能回到太古九天中的云兽,往往百不存一。

%90
“我居然倒霉到,刚巧碰到繁衍后代的云兽群,这运道也真是……”方源对此相当无语。

%91
“按照道理,我已经和许多强运之人连运,就算肉身、魂魄各分一半,我的运气也不至于如此差劲啊。难道说,我得了至尊仙胎,消耗了海量运气,福极祸至了吗?”

%92
方源在瘴气界壁中疾飞,身后始终吊着一群上古云兽。

%93
方源撞破了上古云兽寄存云精的地方,为了后代打算,上古云兽势要消灭方源,绝不善罢干休。

%94
只要方源身上伤势不除,云兽就能继续追踪。

%95
偏偏这些上古云兽,还同样在五域界壁中来去自如!

\end{this_body}

