\newsection{崛起大计盗宿命}    %第八百二十七节:崛起大计盗宿命

\begin{this_body}

%1
吴帅积累了上百年,搭建出绝不寻常的七转仙蛊屋龙庭。

%2
随后,他和风云府公开赌斗,利用龙庭,在众目睽睽之下斩杀了松涛子,令他一战成名。

%3
庆功宴上,绿蚁居士、书道阁主的出席,证实了吴帅此战的影响——他已经拥有一些资格,能够和八转的存在平等对话了!

%4
庆功宴的当夜,吴帅就做了一个美梦。

%5
在这梦中,他实现了他所有的理想,他所有的愿意为之奋斗的目标。

%6
他成为龙人共主,拥有八转巅峰的实力,龙公主动退位让贤。

%7
他掌握龙庭,带领龙人一族自由生活,和人族分庭抗礼,平等相待。

%8
他将范极囚禁,并在书九灵的面前,将他凌迟处死,随后又要对书九灵动手。书道阁主赶来,一场大战之后,阁主败北,愿意用自己作为代价苦苦哀求。吴帅应允了她,只是将书九灵贬斥为奴,而书道阁主从此成了他的打手。

%9
他将绿蚁居士、易酒仙姑充为龙人太上客卿,其实他们本不愿意,但龙人势大,吴帅之强,都令他们不得不从。

%10
最令吴帅欣慰的是,他终将泰琴立为她唯一的龙后,并且两人之间儿女成群。这些儿女各个都有天资,都有才情,如群星闪耀。虽然幼小,但终有一天会成就日月般的伟岸。

%11
第二天,吴帅是带着微笑醒来。

%12
他爱恋地看着身旁熟睡的泰琴,为她温柔地捻好被子。

%13
然后,他动作轻缓地起床,开展公务。

%14
他和松涛子一战,不仅是自己的崛起,也对龙人一族,同样给予了极大的振奋和推动。

%15
龙人一族在这样的影响下,变得更加主动和强硬。

%16
之后数十年,龙人一族和人族之间的矛盾层出不穷,逐渐激化。

%17
龙公得知情况,召唤吴帅。

%18
吴帅千里迢迢飞往,被龙公严厉喝斥。

%19
这些年来,他就是龙人一族和人族之间矛盾的主要推手。

%20
随后,龙公又勒令他颁布罪己诏,退位让贤,舍弃南华岛岛主之位,同时更要将龙庭之名,改成龙宫。

%21
吴帅当然不愿意,但对于龙公,对于天庭,他的实力还是太过弱小。

%22
他脸色铁青,一肚子闷气,返回南华岛,不得不按照龙公的嘱咐照做。

%23
古凉前来拜访。

%24
吴帅第一次见到他的时候,是在他当年战胜松涛子的庆功宴上。

%25
数十年来,古凉一直都和南华岛龙人一族走的很近,双方交易很多次,合作都很愉快。

%26
“吴帅仙友,古某听闻南华岛惊变,阁下却是要下罪己诏,这是何故?可有什么地方,让古某帮助你的呢?”

%27
吴帅深深叹息:“古凉仙友,此次是龙公老祖宗勒令,我是不得不从,万般无奈,别无他法啊。”

%28
吴帅语气萧索:“不怕仙友笑话,当年我斩杀松涛子,满以为这是一个转折点,能够带领龙人一族崛起。但这数十年来,我却是像陷入泥沼地里,如何行动都收效甚微,四面八方都是压力汹涌。”

%29
古凉微笑着劝慰道:“仙友何必妄自菲薄呢?这些年来,仙友的努力我都看在眼里。若非仙友主持着南华岛,这座龙人一族旗帜般的势力早就被各大黑手吃干抹净了。毕竟与仙友为难的,可是十大古派,甚至还有天庭呢。”

%30
吴帅默然片刻,他胸膛里汹涌着愤恨的火焰,又充斥着无奈的悲怒,最后他只好仰天长叹:“龙人一族的崛起,谈何容易,简直是难比登天啊。”

%31
古凉仍旧面带微笑:“一步步来,当年天庭的崛起,不也是如此的吗?”

%32
吴帅惨然地道:“如今我按照老祖宗吩咐的这么一做,等若是将我数十年,乃至上百年的苦功毁于一旦。人族占据大势,大势啊。”

%33
古凉道:“我有一计,可解仙友的烦忧。”

%34
“哦?是何妙计,仙友快快说来,你我交情何必藏掖?”吴帅立即问道。

%35
“此计说来也很简单。仙友所虑,无非是此举太过打压龙人一族的士气。尤其是你和南华岛,皆是龙人一族心中的斗志战旗,龙公却是将其直接拔除。但若仙友这个时候晋升八转呢?”

%36
“晋升八转?我又怎么不想!但是最后一场劫难,我纵然有着龙庭,恐怕……”吴帅犹豫不决。

%37
“哈哈哈,吴仙友,你这些年来统领南华岛,的确身担重任,你明白自己,明白南华岛对于整个龙人一族的意义。但也因此顾虑重重,失去了往日里那份勇猛精进之意。我此次就是来助你一臂之力,帮衬你渡过最后一劫,成为真正的八转蛊仙!”古凉说着,身上的气息就发生了变化,从七转变成了八转。

%38
原来,他是一位八转蛊仙,但却一直伪装成七转。

%39
吴帅大感震惊:“古凉……前辈,你……”

%40
古凉抬起手,制止他道:“吴帅仙友,你我还是平辈论交得好。实不相瞒,我接近你也是有着自己的意图。中洲势大,而其他五域薄弱,我身为东海蛊仙,更愿意看到中洲内斗。但扶持你,不仅是处于此意,事实上也对你个人着实欣赏。说句心底的话,你我之间其实大有共通之处呢。”

%41
吴帅大喜:“这些年来,我也苦求八转蛊仙出手助我渡劫。然而就算是绿蚁居士、书道阁主都是推三阻四,今日仙友相助,真的是如雪中送炭。此情此恩,我吴帅必不敢忘!”

%42
下一幕的梦境,便是灾劫。

%43
但因为在之前的梦境中,吴帅和古凉走得很近,得到了古凉的帮衬。

%44
古凉实力非凡,有了他的帮助,吴帅成功地渡过了灾劫,有惊无险。

%45
吴帅真正成为八转蛊仙。

%46
虽然他按照龙公的吩咐将一切都做了,引得波澜汹涌,龙人一族人心惶惶。但是当他成就八转蛊仙的消息传出之后,风声又立即转向龙人一边,龙人一族士气大振,比之前还要高一上一层。

%47
吴帅亦成为龙人一族中的第二位八转,成就超越了父辈,在修为上面唯有龙公一人压着他。

%48
吴帅有了真正的八转修为,终于可以彻底和绿蚁居士、书道阁主平等对话。

%49
龙公亦隐隐对他改变态度,再不像之前那般随意呼唤苛责。

%50
就像是冲破了万千重围,吴帅直感觉压力一空,数十年来包围着南华岛的重重暗流,在他晋升八转之后,自行消散了大半。

%51
“八转的风景果然不同!”

%52
“从今以后,我便是世间蛊仙的巅峰了。”

%53
吴帅对古凉分外感激,和他的合作也不断加深。

%54
古凉告诉吴帅很多秘辛:“天庭的根基在于宿命蛊。历史上两大魔尊都曾攻上天庭,为何都失败告终?原因就在于宿命蛊他们毁不掉,在星宿合道的前提下,根本动摇不了天庭的根基。”

%55
“然而,星宿合道虽然维护了天庭正统,但是却惹怒天道。天道在于平衡,如今人族却是昌盛无比,毫无制约。这就大大有违真正的天意!”

%56
吴帅明白许多,在古凉身上获益匪浅。他虽然身居中洲,还有一层十大古派中太上长老的身份,但是对于天庭的真正秘密,还是不太了解,也没有了解的渠道。

%57
“吴帅仙友,你要真的带领了龙人崛起,单凭自身八转修为还是不成。事实上,就算是龙人一族再添数位八转蛊仙,也根本撼动不了天庭的位置。你们有八转,中洲天庭远比你们多得是!”

%58
“甚至,说句难听点的话,将来你们龙人一族若真的有一天,实力超过了天庭的心理底线,他们绝对会对你下手的。他们不是常说吗?非我族类其心必异!”

%59
吴帅仔细思索,此言大有道理,便请教古凉。

%60
古凉便告诉他:“龙人崛起,非同一般,还是得从源头下手!”

%61
“源头?仙友是指宿命蛊?”吴帅苦笑,“宿命蛊被天庭严加保管,难度太大了。”

%62
“难度的确很大,但希望还是有的。尤其是现在,可是百千万年都难逢的机遇。”古凉眼中精芒闪烁着。

%63
吴帅动容:“你是指那位……红莲?”

%64
“不错。原本天庭想要将红莲栽培成仙尊,但他却有着另外的想法,选择了叛逆的路。你们的老祖宗龙公一直为此事操劳、头疼,对你们的管教疏松太多。从这点上来看,你们龙人一族还要感谢有红莲为你们分担压力。”古凉道。

%65
“我们要借助红莲的力量?这是个好办法,但如何联络他呢?”吴帅又问,“红莲这个人的行踪实在是太神秘,捉乎不定!”

%66
“我们不能直接找寻红莲,就算和他见面也必定遭受他的拒绝,甚至是打击和毁灭。”古凉神色微变,满脸忌惮。

%67
他饱含深意地道:“红莲不管是仙尊还是魔尊,他到底是一位纯正的人族蛊仙啊。”

%68
“既然他不愿意合作,那我们如何借助他的力量呢?”吴帅问道。

%69
古凉哈哈一笑:“红莲既然想要复活他的亲人、情人,就要违背天道,违逆宿命。这种矛盾不可调和,也是因为这个原因,师徒反目,龙公始终无法令红莲回头。红莲他必然是要攻上天庭,毁掉宿命蛊的。待他攻打天庭的时候,正是我们浑水摸鱼之际。”

%70
“那红莲魔尊何时攻上天庭呢?”

%71
“这我就不知道了。很可能是无数年后,也很可能是在下一刻。”

%72
吴帅恍然:“我明白了,从今往后,我们就要多往天庭靠拢,多多出入天庭,了解内幕,刺探情报。等到红莲真正进攻的那一天,我们就利用这样千载难逢的机会,来尝试盗取宿命蛊!”

\end{this_body}

