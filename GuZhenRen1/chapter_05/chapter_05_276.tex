\newsection{光照菌}    %第二百七十六节:光照菌

\begin{this_body}

%1
东海。

%2
一片普通海域,一座无名小岛,荒凉贫瘠,只有怪石嶙峋。

%3
几位蛊仙从高空中,飘然而至。

%4
正是方源、花蝶女仙、蜂将等一行人。

%5
“没想到这样的交易会,只是选择在如此荒芜人烟的小岛上进行。”方源踏足这座海岛,心生感慨之色。

%6
“楚兄有所不知,历来如此大宗交易会,地点都会在几天前临时决定,防止有人图谋不轨。”方源身边的一位蛊仙答道。

%7
这位蛊仙男子不高不矮,不胖不瘦,容貌中上,五官较为普通,唯有宽阔鼻翼,让人印象比较深刻。

%8
此人便是名传东海的七转宇道蛊仙强者庙明神。

%9
若是单论面貌,此人毫不出众。但方源和他接触之后,虽然交流的时间一点都不长,却能感受到他的种种过人之处,叫人不可小觑。

%10
方源以楚瀛的身份,救下了花蝶女仙、蜂将之后,得知交易会的消息。

%11
其后,方源跟随花蝶女仙、蜂将二人,见到了庙明神。

%12
庙明神毫无犹豫,直接答应了方源想要参加交易会的请求:“仙友两次助我,对小蝶、小蜂有救命之恩。区区一场交易会,在下怎会拒绝?请一定赏光!”

%13
他是七转蛊仙强者,而楚瀛却是隐修,名声不显。但庙明神一点都不拿捏架子,非常亲切近人。言语客气,暖人心扉。

%14
所以,方源便和庙明神来到了此处,参加交易会。

%15
他们一行四人,刚刚踏足小岛,一位蛊仙便从怪石中钻出来。

%16
“原来是庙明神你来了。”一位头大身小,类比侏儒的蛊仙,嗡声招呼道。

%17
“哈哈,土头驮,好久不见!”庙明神热情洋溢,哈哈大笑,快步迎接上去。

%18
“来,我来给你们介绍一下。这位是东海隐修楚瀛,这位是咱们东海里难得的土道宗师土头驮。”庙明神又道。

%19
“见过土仙友。”方源也很客气,但神情木木,伪装成常年独自的隐修,不善交际的样子,毫无破绽。

%20
土头驮打量方源几眼,咧开嘴,露出一嘴的黄牙,笑起来:“看来你这是得到了庙明神的引荐,第一次参加这样的交易会吧?哈哈,想当年,我也是被庙明神引荐,才6续参加的。”

%21
土头驮表现的也很客气。

%22
他心想:“庙明神眼光无差,能被他引荐者,无不是人中之杰。这位楚瀛貌不惊人,声名不显,但却能被庙明神看重,自然不会简单。”

%23
心中一边想着,土头驮一边又道:“你们随我下来吧,我是第一个到的,闲来无事,先在岛下地中,开辟了一处空间。”

%24
“劳土兄费神了。其实这次,是我主导交易会……”庙明神立即恭维道。

%25
“哈哈。区区一个地下洞穴,费不了什么神。走!”土头驮见到庙明神,像是见到了老朋友,一把抓住庙明神的手臂,引领着他往地下走。

%26
他脚步所到之处,原本夯实的地面,就宛若水流一般,融化流转,化出一个宽敞的通道出来,直达海岛地底。

%27
土头驮、庙明神在前方走,方源、蜂将、花蝶女仙则走在身后。

%28
土头驮和庙明神一路交谈。

%29
庙明神交际手腕极其出众,说得土头驮连连大笑,笑声在地下嗡嗡回荡。

%30
并且,庙明神也没有冷落了方源,是不是转头,对方源说话。

%31
他说的这些话,看起来平淡无奇,但又自然而然,没有让方源觉得自己是独单一人,或者说是受到了冷落。

%32
地下洞穴的确平淡无奇,不过胜在宽敞。

%33
一处高台,台下数个石椅,皆是宽大。石椅上还雕刻着花纹,纹路如流水,非常流畅。体现出了土头驮深厚的美学修养,以及心思细腻的一面。

%34
方源细细一数,现这些石椅个数,只有六座。

%35
“看来这场交易会,原计划参加的只有六人。花蝶女仙和蜂将不算,一旦交易会开始,他们就会离开此处,在外把守了。”方源暗中猜测。

%36
“来了一位新人,那我就再添一把椅子。”土头驮轻轻一跺脚,顿时地面隆起一个土包,旋即土包如水,不断流转。

%37
眨眼间,就化为了另外一座石椅,仿佛是经过能工巧匠,精心雕琢的一样。

%38
方源等人便入座,等候其他蛊仙的到来。

%39
片刻之后,接连来了两位蛊仙。

%40
前者是女仙,孩童模样,扎着两个冲天辫子,眼睛溜圆,却沧桑成熟。

%41
她是童画。

%42
东海北端,有名的光道蛊仙,七转修为。

%43
后者则是男仙。鹰钩鼻子,目光阴鸠,穿着黑袍,乱披肩。

%44
他是乌马杨,暗道蛊仙,在七转蛊仙当中,战力颇为出众。一记仙道杀招暗箭大幕,曾经击败过三位七转蛊仙联手,此战让他在东海蛊仙界名噪一时。

%45
童画叽叽喳喳,乌马杨却是沉默寡言得很。

%46
双方攀谈不停,在庙明神的有意引导下,气氛渐渐融洽起来。

%47
一盏茶的功夫后,第三位蛊仙进入这处地下空间

%48
他是公良柏,智道蛊仙,兼修了一些木道手段。当然智道是重点,智道蛊仙数量稀少,尤其是像方源这种,继承了完整真传的智道蛊仙。

%49
而公良柏不仅是有完整的智道真传,而且还修炼有成,虽然比不上最出名的三位智道,但在散仙当中,却有广阔人脉,愿意为他人推算,以此收取报酬。

%50
双极盘甲丹,南宫藏华安,还有龙龟,厄海中往还。

%51
这诗说的就是当代东海蛊仙界中,公认最为厉害的三位智道蛊仙。

%52
宋甲丹、华安,都是正道中人,很少会为散仙出手。而龙龟仙人,却是在厄海中流连忘返,足迹缥缈,让人难以追寻。

%53
“公良柏仙友,这一次交易会后,咱们好好聊聊。我有些东西,想要你出手推算。”交谈中,庙明神道。

%54
“不会是那柳贯一吧?我最近可是接了不少仙友的请求,都是为此人而来。”公良柏笑道。

%55
“当然不是。”庙明神微笑否认。

%56
公良柏点点头:“那就好。说起来,我屡次推算,也有五次。结果都是失败,主要是线索太少,而且那柳贯一也有防备推算的手段。”

%57
方源在一旁,静静地听着,面不改色,心中囧然。

%58
这个话题,立即引起了其他蛊仙的兴趣。

%59
土头驮大声叫道:“现在这个柳贯一,真是老火了,彻底出名了!”

%60
“任凭是谁,只要能以七转修为,让八转蛊仙无可奈何,就足以名垂青史。”童画感慨道。

%61
“能以七转修为,抗衡八转,真叫人向往啊。”庙明神摇头苦笑,“可惜这等人物,我无缘结识。”

%62
方源就坐在他身旁的椅子上,听着这话,面无表情地开口:“我听说,能够以七转抗衡八转,主要是那柳贯一成为了逆流河主。这逆流河,乃是天地秘境之一,但有何大能,可以让八转存在都无可奈何?”

%63
于是,众仙的话题,便从柳贯一身上,挪移到了逆流河。

%64
方源不动声色地引导了话题的转变。

%65
这一次,几位蛊仙一直交谈到傍晚。

%66
但夜幕笼罩东海,月亮刚刚升起,便到了交易会开始的时候。

%67
另外的两位蛊仙,也都赶来。

%68
交易会正式开始。

%69
按照已经商定好的规矩,童画走上了高台。

%70
“这是一种荒植。我是在某处深海探险时,意外现的。那次在黑暗无关的深海当中,我忽然见到了前方的一点亮光。我寻光而行,足足前进了一炷香的功夫,才来到光亮之处。那里简直是一片光道的天堂。光明灿烂,灼照万里。”

%71
“我起先以为,这是珊瑚群出的光亮。但很快,我现珊瑚表面上都附着了一层细小的毛茸茸的绿菌。这种绿菌,乃是荒植,随处可见,我命名为光照菌,它本身蕴含光道道痕,能够逸散出强烈的光明。”

%72
童画徐徐介绍了自己现,并命名光照菌的经过。

%73
方源有些意外,他没想到,交易会刚刚开始,就出现了他想要的东西。

%74
光照菌!

\end{this_body}

