\newsection{参悟气墙}    %第八百四十节:参悟气墙

\begin{this_body}

%1
兽灾洞天,光阴支流。

%2
方源分身战部渡一次次地进入黄金光雾,将陷落在里面的蛊仙拯救出来。

%3
每一次他都要经历艰难的战斗,面对数量众多的凶兽,劳累奔波导致伤痕累累。他身边的同龄小伙伴们都渐渐接受了他的领导,因为从过往的战绩无一不证明战部渡决策的英明。

%4
“这一次有三支兽群,从东南、正北还有西南方向,向你们包围过来。”方源本体传音道。

%5
“哦,好的。我会带领他们突围的。对了,我身边的那个白袍青年,帮我把他铲除了。”方源分身战部渡回应着。

%6
“明白了。”方源本体身处在黄金光雾最为浓郁的地方,此刻好整以暇地半躺着,神情悠然。

%7
他念头转动,年兽大军顿时接连出发,向分身的小队伍杀去。

%8
而他大部分的注意力,则都集中在元始真传上。

%9
东海龙宫之争,方源不仅收获了龙宫,而且还从龙公那边哄骗到了元始真传!

%10
不过这道元始真传,显然只有前面一小半的部分。即便如此,方源看着也是连连赞叹。

%11
从这小半份真传中,他领略到了远古时代的风格,气道浩荡,大巧不工。也领略到了元始仙尊的才情,卓绝非凡,不愧是尊者,难以用常理来局限。

%12
“这份真传内容,拿到现在来用,仍旧相当实用。”

%13
“当然,这么多年,气道也在发展,推陈出新。我手中掌握的气海无量杀招,就是最好的证明。”

%14
这一小部分的元始气道真传,最主要的内容就是气墙杀招。

%15
起先,龙公并不愿意给,但方源强硬要求,毕竟气墙杀招乃是元始仙尊的招牌手段,名垂青史,无人不知无人不晓。

%16
龙公想要和气海老祖讲和,将方源态度极其坚决,便咬牙同意下来。

%17
“通过这个气墙杀招,我对天庭中的那道气墙更加了解了,争取破解了它!”方源眼冒精芒。

%18
上一世,天庭中出现的气墙,给长生天一方造成了巨大的麻烦。

%19
方源若是能破解了此招,必定能给天庭一记重击。

%20
但显然这个任务难度极高,毕竟是仙尊的手笔。唯一值得安慰的是,元始仙尊当初布置下来的气墙杀招,一直流传到了现在,充斥着远古时代的风格,早已经过时了。

%21
这一点或许是方源破解气墙的契机。

%22
当然破解不了,也毫不意外。

%23
还是那句话,没有无敌的仙蛊,只有无敌的蛊仙。没有任何一个杀招是完美无敌的,所有的杀招都能够被破解,但相同的杀招由不同的人催动出来,威能效用也是天差地别的。

%24
方源能否破解了元始仙尊亲自催发的气墙杀招,这还是个大大的疑问。

%25
或许,龙公也正是考虑到了这一点,这才大大方方地将气墙杀招的内容,交托给了方源。

%26
“同时,这小半份真传也是一个绝佳的诱饵,用来诱惑我加入天庭。”方源心中冷笑。

%27
若真有一位气海老祖,常年隐居东海,见到这份诱惑,必定难以把持。

%28
身为八转隐修这么多年,说明气海老祖对于权势并不热衷。同时他专修气道,而气道式微已久,只有天庭收藏了气道中价值最高的真传。

%29
但事实上,气海老祖只是方源的一个伪装。龙公若是知道自己亲手资敌,不知道会是什么感受。

%30
“火候差不多了。”方源一边研究气道真传,一边留着心意始终关注着十二生肖大阵。

%31
这么一会儿工夫,他已经配合分身战部渡,完成了又一场表演。

%32
战部渡身边的少年们,有几个牺牲,战死沙场。这些人中有一些是阻碍,方源已经在为下一步计划提前清扫障碍了。

%33
战部渡救下一位蛊仙,再次凯旋。

%34
黄金光雾中的战兽勇士,已经只剩下一两位了。其中就包含山崖城主。

%35
当战部渡救回山崖城主之后,方源就会收手,撤回十二生肖战阵。

%36
凡事过犹不及,方源栽培分身,已经达到了现阶段的目的。

%37
战部渡除了修为暴涨之外,最大的收获是人脉的扩展。这一役下来,兽灾洞天中几乎所有的蛊仙都受到了战部渡的恩惠。

%38
这种恩情还不是寻常小事,而是救命之恩!

%39
若换做外界,蛊仙们或许翻脸不认人。但在这里,能够成就战兽勇士的必定都要用充沛的正面意念。所以统统都是知恩图报,甚至不乏滴水之恩涌泉相报的人物。

%40
战部渡的资源绝不会缺,前途一片光明。而他也绝不会挟恩图报,反而更加谦逊有礼,不计较什么,充分展现出种种美德和人品。

%41
方源原本可以一网打尽,但这些战兽勇士都被他放过。

%42
上一世他攻略兽灾洞天,简单直接粗暴,杀害了许多战兽勇士。这一世重生归来,他修改了计划,换另外一种方式吞并,就是为了最大程度地保留整个兽灾洞天的价值。

%43
“这些兽灾洞天中的蛊仙,尽数收服下来,也是对我麾下实力的补充。”

%44
“将来中洲炼蛊大会时期,派遣他们出战,会收到一定成效的。”

%45
“当然,他们的手段太过肤浅,能打敌人一个猝不及防。一旦被摸清底细,这批蛊仙必定损失惨重。”

%46
兽灾洞天中的蛊师走了捷径,虽然战力容易获取,但是并不丰富,不是蛊修正统。

%47
就像现在,方源单单布置了一个十二生肖大阵,就将整个兽灾洞天中的蛊仙耍的团团转。即便是战兽王也看不出真相,仍旧以为是黄金光雾作祟,连十二个太古年兽的真容都没有看到过。

%48
单凭这种层次上的差距,方源就能轻轻松松地碾压了他们。

%49
就像上一世,方源斩杀了巨鹰勇士,化身牛头魔神,屠戮无算。而这些战兽勇士、战兽王,都一直认为方源乃是兽灾,拥有各种强大的天赋战技。从未想过蛊虫、杀招什么,他们的眼界太过短浅了。

%50
数天之后,方源收走十二生肖战阵,笼罩在光阴支流上的黄金光雾也随之消失。

%51
为了避免对方生疑,临走之前,方源还故意放了一大批的年兽群为患,像是黄金光雾最后的挣扎。

%52
经过一番艰苦战斗,战兽公会大获全胜,人人喜气洋洋。

%53
蛊仙们都被救出,开始四处活动,斩杀或者擒拿兽灾洞天中散落各地的强大年兽。

%54
黄金光雾消失了,但是兽灾洞天的时间流速仍旧受到方源杀招的影响,非常的快。

%55
战部渡成了人人称颂,蛊仙们感恩的对象。

%56
战兽王更是当众为他颁奖,鼓舞地道:“小渡啊,你是名副其实的小救星。你救了大家伙,也就是间接地拯救了整个世界!”

%57
战部渡连忙摆手,道谢,语气非常谦虚。

%58
从这一刻起,小救星的称号就安在了战部渡的身上了。

%59
接下来,他就是等待另一份“机缘”,好让他成就蛊仙。再然后,便是正大光明地接受考验,继承兽灾仙人的真传。

%60
方源本体离开光阴支流,又利用天相杀招视察了华文洞天那边。

%61
李小白作为方源的分身,虽然进展没有战部渡那么大,但是日子也是美滋滋的。

%62
自从上一次吟诗之后,李小白就被私塾的姜先生重点关注,私下里安排了不少的小灶,专门给他补习功课。

%63
李小白渐渐表现出潜力和才情,让姜先生惊喜之下,越发悉心教导。

%64
“按照这种程度,要接触到华文洞天的最高层次,还有一大段距离。”

%65
“可惜了。”

%66
“华文洞天和兽灾洞天的情况不一样。后者封闭,前者却有八转蛊仙,并且还是信道洞天,最擅长收集情报。我也不太好出手啊。”

%67
“至少光阴支流这条路不能用,再用十二生肖战阵,恐怕会被一眼看破。”

%68
方源决定等一等,等待李小白的发展。

%69
他当然可以顺着光阴支流,率领兽群大军,入侵华文洞天。但如此一来,他势必要和那位八转蛊仙对战。

%70
方源自然不惧,但也难以速胜。对于华文洞天而言,更容易被殃及,因而损失惨重。

%71
这一点,方源不愿意看到。

%72
若是强攻可行,他早就强攻了。

%73
华文洞天本身不能吞并,但里面的信道传承、各项资源,都是方源觊觎的目标。

%74
方源现在共有五大分身。

%75
宙道分身是最早的分身,目前为了利用智慧光晕,停留在六转。

%76
龙人分身眼下修为最高,有七转,同时掌控龙宫,实力也是最强。目前正在古族接触,也有不少可喜的成果。

%77
战部渡经过方源帮助,进展极大。李小白则要更多依靠自身努力,还未到发力之时。

%78
至于方源的假身份,那就更多了。

%79
这一天,房家方面主动联络方源。

%80
准确的说,不是古月方源,而是他的伪装身份——算不尽。

%81
“算不尽长老,我们需要你出力,辅助太上二长老,尽快地炼化豆神宫!”

%82
方源大喜,他加入房家,除了青鬼沙漠的开发,还有一层目的就是谋夺豆神宫。

%83
这座八转仙蛊屋可是元莲仙尊之物,威能极大,方源可是亲自领会过的。

%84
“而且我还有因果神树杀招在手,这或许是我夺取此宫的关键底牌了。”方源心道。

\end{this_body}

