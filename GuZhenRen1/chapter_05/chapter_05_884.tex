\newsection{应声虫}    %第八百八十八节:应声虫

\begin{this_body}

%1
阴天恶风,波涛汹涌。

%2
这里正是悔哭海域,而方源和庙明神等人已经深入此地多时。

%3
“我后悔!我不该杀死那么多人呐。”

%4
“我痛悔!我不应该出走追寻力量,我若是能停留在父母身边尽孝多好啊!”

%5
“我忏悔!我欺骗了所有人,藏匿了仙蛊,我让最信任我的朋友背锅,自己得到了好处!”

%6
一声声的哭泣,仿佛仙道杀招,向众仙的耳中钻去,干扰情感心绪。

%7
这里已经是悔哭海域的最中心。

%8
庙明神等人停留在高空,看着沉思中的方源,耐心地等待着。

%9
即便他们没有多少阵道底蕴,此刻也查探到海底中恢弘巨阵。

%10
这明显是巨阳仙尊的手笔,让庙明神等人光是看看,就感觉自身的渺小。

%11
方源上一世也是如此,他曾感叹此阵浩瀚得宛若星空,自己要参悟出大阵的全部奥妙,非得要有数十年的光阴。

%12
但如今,这个时间却是大大缩减。

%13
“大概要八年左右的时间全力参悟,我就能对这座大阵了若指掌了。”方源心道。

%14
造成这样进步,不是旁的,正是方源的土道境界。

%15
方源上一世的土道境界可没有宗师层次,而这座大阵恰恰是土道为主的超级仙阵。

%16
“悔蛊没有丝毫的动静。”方源盯着中央海底的某个地方。

%17
这个情况不算好,但方源并不意外。

%18
为了留在龙鲸乐土中,方源提前损毁了石莲岛。没有了石莲岛,悔蛊上的仙尊手段也就没有了目的地,如何发动?

%19
按照方源在石莲岛上得到的历史记忆,乐土仙尊曾经进入石莲岛,借走悔蛊。

%20
石莲岛上虽有红莲意志,也有红莲手段,但如何能比一位活生生的尊者?

%21
红莲意志妥协,乐土仙尊便成功借走悔蛊,还想要和红莲魔尊抢夺继承人。

%22
而这个继承人便是方源了。

%23
方源上一世选择了石莲岛,但这一世他提前取走了石莲岛上的一切,赶来龙鲸乐土,想法是要继承这道乐土真传。

%24
“上一世悔蛊异变,直接飞出来将我带走。这一世它毫无动静,看来需要我亲自动手了。”方源心中生出一股明悟。

%25
尊者手段绝妙浩大,方源上一世被悔蛊送上石莲岛,根本没有反抗之力。

%26
悔蛊没有发动,事实上是方源乐于见到的情况。

%27
细细参悟了几个时辰,方源收敛散漫的心神。

%28
他又沉思片刻,便对庙明神等人道:“要想除掉这里的魔仙,就得借助大阵。然而乐土仙尊当年布置的大阵,虽然威力绝伦,但只是镇压封印,毫无杀伐之能。我的想法是,在外布置辅阵,然后发动辅阵,渗透到大阵内部将魔仙拘杀。”

%29
对这个想法,庙明神等人皆连点头,大为赞同。

%30
要破解大阵,将魔仙放出来,必然是十分危险的事情。

%31
想想看就知道,这位神秘的魔仙和乐土仙尊是同时代的人物。乐土仙尊寿尽而去,这位魔仙却仍旧活着。

%32
要论寿命,已经高达十万年以上。

%33
谁能活得十万年?

%34
寿命最长的元始仙尊也不过两万五千岁。

%35
活的比尊者还久,单论这一点,这位魔仙就已经是可畏可怖了。

%36
“诸位不必太过高估阵内魔仙。”方源笑了笑,“这座大阵本身除了封印之外,也有延寿的法门,让这位魔仙沉眠入睡,寿命得到极大的延展。因此才能活过这么悠久的岁月。”

%37
庙明神等人的脸色顿时舒缓了一些。

%38
“乐土仙尊真的是慈悲为怀,不仅没有杀掉此人,镇压魔仙之后,还专门为他(她)延寿!”

%39
“难道乐土仙尊也掌握了天庭的延寿法门吗?按照天庭的沉眠延寿之法,别说是十万年,就算是百万年也是轻轻松松。”

%40
“唉,尊者的手段我们岂能揣度?”

%41
方源将目光转向庙明神,又沉声道:“但是我要布设辅阵,需要一位蛊仙镇压一处阵眼。当辅阵发动起来,蛊仙越多,阵眼越多,辅阵的威能就会越大。而我的计划中,至少需要八位蛊仙来镇压阵眼。”

%42
庙明神等人的感叹微微一缓。

%43
加上方源,在场的不过五位蛊仙而已。方源此言就是要再邀人进来,至少还需要三人。

%44
这就有点微妙了。

%45
因为即便加上曾落子、土头驮,还有一位空缺。

%46
而其他的蛊仙,除了沈家一伙之外,就是任修平、童画了。

%47
这明显和庙明神一伙人不对付。沈家欺压过他们,任修平更是生死仇敌,童画是叛徒。

%48
该请哪一位?

%49
鬼七爷、化蝶仙子和蜂将,纷纷面露难色。

%50
庙明神却哈哈一笑,神情诚挚地回应道:“楚兄,眼下重中之重就是乐土真传,机缘难得,当抓紧这样千载难逢的机遇!我们还有什么仇恨不能暂时放下的?要算账,出去可以慢慢的算啊。依我看,曾落子、土头驮可请,任修平、童画也可请,沈家蛊仙若是也能来,那更加保险。”

%51
说完,庙明神又附加一句:“当然,一切都凭楚兄你做主。”

%52
鬼七爷等三仙面色皆变,有人想要说什么,但既然庙明神都这么说了,最终也无话可说。

%53
方源哈哈大笑,庙明神不愧是庙明神!

%54
方源温声道:“庙兄深明大义,胸襟广阔,令我佩服。那就这么办吧,我先布下几座辅阵,请四位各镇守一处阵眼,催动辅阵,对海底大阵不断渗透。然后,我再邀请其他蛊仙加入进来。”

%55
庙明神再道:“我与土头驮、曾落子有旧,可以书信他们,帮助他们做出正确的决定。”

%56
方源颔首:“这最好不过。”

%57
最终,方源从庙明神那里得来数只信道凡蛊。

%58
庙明神不仅给土头驮、曾落子留了书信,就连仇敌任修平、童画以及沈家,也各有书信,帮助方源邀请他们。

%59
书信的内容,方源都看了几遍,没有什么蹊跷。信中庙明神表达了合作的诚意,表示彼此间的仇恨可以暂放之后,同时恭维方源的手段,力劝他人和方源展开合作。

%60
方源不由再高看庙明神一眼。

%61
事实上,庙明神的书信只是锦上添花,有无书信并不存在质变。

%62
方源本身就有很大的把握,成功邀请到更多人手。

%63
就算庙明神反对,方源也会邀来其他蛊仙。庙明神正是清晰地认知到自身并不是不可取代的,也为了继续合作的利益,选择委曲求全。这若不是人杰,谁才是?

%64
招揽曾落子、土头驮的过程,极端顺利。

%65
方源只是提了一句,他们就迫不及待地加入了,甚至怀着某种程度的激动。

%66
毕竟,功德榜上方源提携庙明神一伙人,连沈从声都压在下面的情形,众仙都看在眼里。

%67
方源又邀请童画,童画有些迟疑,表示要考虑一下。

%68
方源再邀请任修平,任修平想都不想,就答应下来,速度比曾落子、土头驮都要快得多。

%69
任修平还给方源带来了一个小小意外的讯息,沈家的蛊仙也想要参与进来。

%70
沈家蛊仙的首领便是八转蛊仙沈从声,没有他的同意,其他的沈家蛊仙绝不会胆大到投靠方源这一边。

%71
于是,方源很快就和沈从声碰面。

%72
“楚瀛仙友。”这一次沈从声的态度相当恳切,几乎把七转修为的楚瀛当做同等的八转来对待。

%73
“我之所以来到这里,是因为得到天庭的情报。前不久,龙公和气海老祖之争,相信楚瀛仙友也有所耳闻。此战中,沈某俘虏了古月方正,天庭为了赎回他,给予了沈某一项重要的情报。”

%74
“原来在这龙鲸乐土中,有一只八转仙蛊应声虫,此蛊对我沈从声,乃至沈家都有极大的作用,沈某多年苦寻而不得。”

%75
“沈家愿和楚瀛仙友合作。若是仙友在功德碑的奖励中,见到此蛊,还请高抬贵手。若是此蛊在任务中碰见,我沈家还想和仙友合作,捕捉此蛊。沈家必有重谢!”

%76
方源面色微变:“应声虫?此虫可是名列奇蛊榜前十,催发出声,但凡听到此声的存在,只要回应一声,都会沦为奴仆。而八转层次的应声虫,已可奴役八转蛊仙了。如此重宝,贵族苦求不得,想要与我合作,又能拿出什么诚意来呢?在这龙鲸乐土中,我们之间可是不能直接缔结盟约的。”

%77
沈从声哈哈大笑:“楚瀛仙友,沈某早有准备。此次合作,是压上我沈家的名誉,整个过程你都可用蛊虫记录下来。而这些东西,便是我沈家展现诚意的一小部分,还请楚瀛仙友笑纳。”

%78
说着,沈从声便取出一份庞大的仙材,当场交给方源。

%79
这项仙材价值之高,即便是方源,也不由心头多跳了一下。

%80
悔蛊就在眼前,而大规模升炼仙蛊所需要的仙材却是有不少缺口,这份来自沈从声的仙材正好弥补缺口,乃是方源当下急需之物。

%81
方源点点头,将这份仙材收下。

%82
沈从声又道:“听闻楚瀛仙友接下的任务里,有一座乐土仙尊亲自布置的超级蛊阵。沈某想随同前往,或许应声虫就在当中。就算不在,沈家蛊仙一同出力,也能为楚瀛仙友完成这项任务尽一些绵薄之力。”

%83
方源沉吟了一会儿,点点头:“有沈从声大人亲自出手相助,实在是在下的荣幸。”

\end{this_body}

