\newsection{月亮节的故事}    %第二百九十八节:月亮节的故事

\begin{this_body}



%1
“他奶奶的,有没有搞错!我的九回香,居然连一杯海水都比不过?”罗木子在心中咆哮,表面上则维持着风度。

%2
“可恶!这个武遗海偷奸耍滑,我们是献茶,他是献海水?这脸皮真是够厚啊!偏偏丝柳仙子居然还这么认可他,太气人了,太气人了!”轮飞石桌下的双拳,捏得都要青筋暴起。

%3
天露仙子忙打圆场:“月已高空,茶水既饮,却无诗词,总是不美。”

%4
方源故作不解:“难道真要吟诗作赋?”

%5
一副毫无准备,首次听闻的样子。

%6
罗木子、轮飞顿时双眼骤亮,这又是个机会啊!

%7
又是一个打击情敌的上佳机会。

%8
“前一次被你耍滑头,蒙骗糊弄过去了。这一次,我一定要把你的风头给打落下去,然后再狠狠地踩上几脚。”

%9
罗木子满心萦绕着这个念头,表面上却是微笑着,风度翩翩。

%10
轮飞同样如此,转着差不多的心思。

%11
他们却不知道,方源是什么人?

%12
比诗词?!

%13
我的天!

%14
这比班门弄斧还要严重。

%15
方源来自地球,古诗词一大堆,不乏名留青史的佳作,更有震古烁今的巨作。随便抛出一份,就能打得这两人颜面无存。

%16
“的确是要吟诗,这其中还有说头。”乔丝柳回应方源的话。

%17
“哦?愿闻其详。”方源便顺着话道。

%18
“这是一个流传在南疆的故事,也是月亮节的由来。”乔丝柳开始娓娓道来。

%19
很久很有以前,在南疆的某个山寨里。

%20
一位凡人小伙子爱上了一位蛊师老爷的女儿,而这位蛊师的女儿,也同样爱上了这个凡人小伙儿。

%21
小伙子鼓起勇气去提亲,结果遭到了蛊师老爷的拒绝。

%22
“你只是一个凡人,而我的女儿却是一名蛊师,将来前途广大,你怎么可能配得上我的女儿?你给我滚!”

%23
小伙子苦苦请求,蛊师老爷冷笑:“想让我把女儿嫁给你,你简直是痴心妄想!你区区一届凡人,连一杯茶都酿不出来?你有什么用处?”

%24
小伙子道:“不就是区区一杯茶吗?这有什么难的,我做出来,你就把女儿嫁给我吗?”

%25
蛊师老爷很头疼。

%26
他知道女儿非常爱眼前的小伙儿,强行拆散他们,会让女儿恨自己这个当父亲的。

%27
蛊师老爷便道:“你若是能做出一杯让我满意的茶来,我就愿意给你一次机会。”

%28
小伙子大喜,连忙答应下来:“老先生,我一定能做出来的。”

%29
蛊师女儿听了这件事情,非常担心:“我家世代以茶出名,你却要做出一份让我爹满意的茶来。你只是凡人,没有蛊师的能力,怎么可能做好一杯茶?”

%30
小伙子却道:“你放心吧。谁说凡人就不能做茶?我来告诉你三个道理。”

%31
“第一个道理,大鱼吃小鱼,小鱼吃虾米。”

%32
于是,小伙子来到溪水边,钓上来一只大鱼,破开渔腹,取出小鱼,又破开小鱼的独自,取出了其中的小虾米。

%33
“第二个道理,人要吃饭,也要拉屎。”

%34
于是,小伙子把小虾米吃掉,拉出了一摊屎来。

%35
“第三个道理,屎能让花草更茂盛。”

%36
于是,小伙子将屎埋在土壤中,果然花草茂盛起来,节节拔高。

%37
小伙子从中挑选出了一种花草,将它泡在了溪水,整个小溪就都变成了茶。

%38
蛊师老爷喝了一口这样的茶,好半天说不出话来。

%39
他的女儿叫道:“爹,你不会是想反悔吧?”

%40
蛊师老爷这才勉强点头:“小伙子,算你过了第一关。但你一个凡人,想要娶我的女儿,还是不可能的。你太粗鲁,没有才情,不能吟诗。”

%41
小伙子挠头,苦恼:“我虽然没有吟过诗,但我可以试一试。”

%42
蛊师老爷嗤笑一声:“就凭你?”

%43
小伙子反问道:“我为什么不可以?”

%44
蛊师老爷:“小伙子,不是你随便说几句,就是吟诗。我们蛊师吟诗,能让天地变色,能让人手舞足蹈。你能吗?”

%45
小伙子闷声道:“不试试看,怎么知道行不行?”

%46
“好,那你就试一试,别说我不给你机会。你要是失败了,你就给我滚蛋,不要再见我的女儿。”蛊师老爷说道。

%47
小伙子不得不答应,他开始四处乱走,想要吟诗。

%48
但他从来没吟过诗,毫无头绪。

%49
这个时候,他看到地面上的蚂蚁,又看到窗外的飞鸟和夕阳,忽然一拍脑袋。

%50
他开口叫道:“燕子低飞蛇过道,蚂蚁搬家雨就到。”

%51
南疆多雨,这个时候又正当春天。

%52
小伙子刚说完,天空就开始飘扬起了雨丝。

%53
蛊师老爷面色一变。

%54
小伙子又道:“一滴春雨一滴油,多下几场我们愁。”

%55
屋外的雨越下越大,天空变得阴沉沉的,不复之前明朗。

%56
蛊师老爷面色有些难看。

%57
小伙子抓头挠腮:“榆钱黄了种地忙,柳毛掉了乱撒稻。”

%58
到了这一步,小伙子卡住,只差最后一句,但死活憋不出来。

%59
“再给你最后一点时间。”蛊师老爷冷笑不止。

%60
小伙子双眼顿时一亮,指着蛊师老爷脱口道:“老爷收了万担粮,我们还是闹饥荒。”

%61
蛊师老爷顿时气得跳脚,一下子站起身来,把手中的茶杯都打掉。

%62
然后手指着小伙子,咆哮道:“你区区一个凡人,好大的狗胆子!”

%63
但是他的女儿却笑起来,拍手:“太棒了,这首诗让天地变色,更让爹你手舞足蹈啊。”

%64
蛊师老爷见女儿站在情郎一边,又气又怒,偏偏又无法反驳。

%65
“好好好,算你过了第二关,还有最后一关。你想娶我的女儿,聘礼呢?你能出得起让我满意的聘礼吗?”

%66
小伙子沮丧地低下头,他住着茅草屋,睡着草鞋,只有一身打满补丁的衣服。

%67
“我愿意用我所有的一切财产,来当做聘礼。”小伙子很认真。

%68
“你拿出来我看!”蛊师老爷道。

%69
小伙子便将蛊师老爷引领到他的住处,那个破旧的茅草屋中。

%70
小伙子对蛊师老爷道:“这些就是我全部的财产了。”

%71
“就凭这个到处都是漏洞的破屋子?”蛊师老爷手指了指,非常不屑。

%72
“就凭这个快断成两半的黄草席?”蛊师老爷一把将草席掀飞。

%73
“就凭这些当过凳子的破石头?”蛊师老爷狠狠一脚,将石头踢破。

%74
小伙子低下了头。

%75
蛊师老爷每说一句话,小伙子就把头垂低一些。

%76
等到蛊师老爷说完三句话,小伙子的头几乎要低到自己的胸口了。

%77
但就在时候,从被蛊师老爷踢破的石头中,忽然晃晃悠悠地飞出了一只好似月亮的蛊虫,亮闪闪的,非常漂亮。

%78
蛊师老爷惊呆了。

%79
小伙子也惊呆了,这块石头只是他随便从山脚下捡来的。

%80
蛊师老爷的女儿欢喜的大叫起来:“这只蛊虫绝对足够了,就充当聘礼吧。”

%81
蛊师老爷无法反驳,他说不出话来,最终他只好捏着鼻子,将自己的女儿嫁给了这个凡人小伙子。

%82
这个故事,方源早已经听过,不过的确有趣。

%83
整个故事,描述了一场凡人和蛊师之间的斗争,最终竟是以凡人的胜利为结局。

%84
小伙子勇闯三关,最终抱得美人归,还是一位女蛊师。蛊师老爷竟然至始至终,都没有动用武力。这都不符合现实逻辑,但是却显露出底层凡人的心中对美好生活的向往,以及对幸福生活的追求。

%85
小伙子这个形象很值得探究和分析。

%86
不管他说的三个道理,还是吟的诗,都是通俗易懂,非常浅显,甚至是粗陋的。

%87
但这些,却都是凡人在日常劳动中的经验总结。

%88
尤其是那首诗,虽然声韵并不和谐,但却从侧面披露出了凡人受到蛊师剥削的真相。

%89
“老爷收了万担粮,我们还是闹饥荒。”

%90
蛊师高高在上,凡人做牛做马。

%91
所以凡人心底,有和蛊师斗争的愿望。

%92
但是蛊师太强大了,凡人怎么可能战胜蛊师?所以故事中,在最后的第三关,凡人小伙子取胜的关键,居然是一块破石头。

%93
石头中出现了蛊虫,这很明显就是解石。

%94
解石这种活动,在南疆非常盛行。更准确的来讲,天下五域中,南疆解石第一,因为南疆多山,有极其雄厚的自然资本。

%95
凡人小伙子借助解石,最终战胜了蛊师老爷,虽然这块石头是他捡回来的,但是他更加依靠的是运气。

%96
这也显露了凡人心中的软弱和无奈。

%97
在他们看来,蛊师太强大了,他们是绝对不可能战胜蛊师的。如果能战胜的话,那一定是需要强烈的好运!

%98
这个民间故事,是以凡人为主角,极可能也是凡人所创。

%99
正因为如此,得到凡人的强烈共鸣,在相对闭塞的南疆,广为流传,不亚于《人祖传》。

%100
而在其他四域,却传播得相当有限,很少有人知晓。

%101
正是因为这个故事,才渐渐形成了月亮节的南疆节日。每当到了这一天,不管是凡人,还是蛊师,亦或者蛊仙,都会来赏月、品茶、吟诗、解石,来共同渡过这个节日。

%102
备注:哇,这章写了好久,耗费了好多脑细胞,今天晚上状态不错。这章我自己感觉挺满意的,因为我一直致力于塑造一个别样的完整的世界。大家觉得呢?

\end{this_body}

