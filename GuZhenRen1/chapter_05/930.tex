\newsection{一剑枭首!}    %第九百三十四节:一剑枭首!

\begin{this_body}

仙道杀招——五轮肃杀转!

五行大法师怒吼一声,双掌翻上,狠狠一推,推出一道五色圆环。

圆环旋转飞射上空,环心爆发出强劲的吸力,被它吸引进来的种种杀招、荒兽都被直接绞杀。

天庭的防线为之一乱。

“我们冲进去!”南疆的蛊仙立即抓住这个战机,催动八转仙蛊屋太宇寺见缝插针地冲进防线。

天庭成员玉珠子狠狠一咬牙,手指太宇寺,喝道:“去!”

在他身边漂浮着的玉珠,大半飞射而出,射中太宇寺,纷纷自爆。

轰轰轰……

玉珠自爆,炸响如雷,玉粉四散,附着在太宇寺上,竟叫这座八转仙蛊屋行动缓慢数倍。

得到这个机会,天庭成员迅速回援,费劲心思终于又将太宇寺逼出去。

然而太宇寺虽然被逼出去,另一边的武庸却再次出手。

仙道杀招——指风龙。

仙道杀招——乱弹刃!

武庸屈指连弹,叮叮咚咚地弹出数只碧墨小虫。

小虫飞在空中,身躯急速膨胀,一尺、五尺、一丈、五丈、十五丈!

几个呼吸的时间,它们就化为一头头二十二丈的凶恶风龙,张牙舞爪,扑入天庭防线。

随后,武庸眼中绿意一闪,条条风龙仰天咆哮一声,陡然爆裂,化作无边的翡翠风刃,四处激射。

风刃锋锐,四处飞射,切割一切,天庭主力只得咬牙坚守,硬抗杀招。

趁着他们自顾不暇,武庸趁机扭转身形,宛若一道清风,就要渗透到防线里去。

“休想!”一声娇斥,花瓣飘飞,万紫红挡住武庸。

她一直对强大的武庸保持着关注,不敢有丝毫大意。武庸一有异动,她就立即补防。

武庸叹息一声,只得作罢。

玉清滴风小竹楼中的南疆蛊仙则向一旁的东海蛊仙华彩云高喊:“快冲进去!”

原来之前万紫红和华彩云、青岳安交手,此刻万紫红补防去了,便留出了空档,任凭华彩云、青岳安插入防线。

但华彩云、青岳安迅速对视一眼,察觉彼此都没有突入之意后,都选择了原地作战,根本没有进犯的动向。

人意虽然影响了东海诸仙的决断,但是影响程度也有极限。此刻华彩云、青岳安已经暗感后悔,怎么头脑一热,赶到天庭决战了呢?

自己在中洲搜刮仙材,不是挺好,挺惬意的事情吗?

何必来这里打生打死!

他们俩个都不傻,谁第一个突入防线,必定会遭受天庭的猛烈反击,稍有不慎,就会身死道消。

正是看破了这一点,万紫红才大胆地补防,挡下武庸。

“东海蛊仙性情都是如此。”万紫红双眼精芒爆闪,对东海也有深厚的了解。正因如此,她才放心补防。

但就在下一刻,石淼忽然斜插进来,杀进天庭防线之中。

他也是东海蛊仙,天庭主力维持防线捉襟见肘,不得已的情况下都会将防线的破绽暴露在东海蛊仙面前。但天庭主力不会想到,石淼已经被龙宫控制,成为了龙将,战斗起来悍不畏死。

天庭主力们一个愣神,就让石淼顺利地冲破防线,直逼监天塔。眼看他就要杀进去,忽然一个身影猛地出现在石淼面前。

“你休想!”龙公的声音传入石淼的耳中。

石淼早就撑起防御,浑身上下皮肤如石皮,坚不可摧。但下一刻他就被龙公右手狠狠一插,脖颈被龙公死死抓住。

石淼顿时感到窒息,一股强大的力量从龙公的右手传来,令他脖颈上的石皮咔嚓龟裂。

“这是何等的力量!”石淼骇然,极力挣扎,却是无果。

眼看着他就要被龙公直接捏爆,忽然一道剑光袭来。

龙公极力躲闪,但剑光早有预料,算计到了这一点,仍旧射中了他的腰腹,像是一道光,迅速透体而过。

龙公差点被剑光齐腰斩断,肚腹上的伤口十分巨大,喷血如瀑,一时间止都止不住。

而剑光余势不减,又打在监天塔上。好在监天塔已经催起防御措施,不像之前被薄青剑光斩断那样猝不及防,塔身一阵白光闪耀,终究是艰难地将剑光挡了下来。

“方源!”龙公目光如电,甩头看向攻击来临的方向,那里正是方源再次变化而出的太古剑龙。

剑龙龙爪捏成拳头,刚刚的一招正是五指拳心剑!

这一杀招威力恐怖,龙公绝不会坐视方源随意进攻,当即就要甩飞石淼,再用龙门接近方源。

但就在这时,石淼忽然露出诡异的微笑,整个身躯忽然融化,迅速覆盖在龙公的身躯上。

这一变化令龙公猝不及防,而方源已然再度催动五指拳心剑!

第二剑、第三剑、第四剑!

每一道剑光都正中龙公,剑剑透体而过,然后余势不减,打得监天塔剧烈颤抖。

“龙公大人!!!”这一惊变,让天庭诸仙大惊失色。

龙公也万万没有料到,石淼居然甘愿舍弃自己的生命来牵制他!方源的五指拳心剑射中自己,石淼也跟着遭殃。

龙公此时身上有三大剑创,恐怖淋漓,但他还剩下一口气。

石淼却死得不能再死,还原出破烂的身躯,含笑坠落而亡。

“龙公,你的路到此为止了。”方源轻声地道,随后龙指一松,五指拳心剑第五剑!

剑光似乎要洞穿宇和宙,快得让任何人都阻拦不住。

天庭诸仙只能用惊骇欲绝的目光,眼睁睁地看着剑光闪电般射向龙公。

死亡来临之际,龙公却是双眼失神,猛地陷入回忆的漩涡之中。

幼年……

龙公趴在床边,哭喊着:“爹、娘,你们不要死,不要离开龙儿啊!”

“孩子。”龙公的父亲抚摸着他的脑袋,微微而笑,即便死亡临近,也带着洒脱的笑,“不要哭,更不要悲伤。我们是为了抵御帝藏生作乱而受了致命伤,这是我们身为天庭成员的责任,更是我们心甘情愿要去做的事情。”

“死亡是每一个生命的终点,关键是死得其所。”

“等你长大之后,你就会明白这个道理了。我的孩子。”龙公的母亲也道,“你会明白我们的牺牲究竟意味着什么。它虽然意味着分别,但更多的是荣耀和守护人族的责任。能力越大,责任越大。守卫族群,是我们以及无数前辈先贤前赴后继的伟业。等你长大之后,你也会继续走上我们的道路的。我们会在另一个世界,期待着你的成长。要努力和坚持哦。”

少年……

少年龙公气喘吁吁,被一个傀儡打倒,倒在地上。

一旁的蛊师哈哈一笑,拍拍手掌,赞赏道:“小龙,你已经做得足够好了。起来吧,休息一会儿。”

“不,我还能战斗!”少年龙公拼尽全力,摇摇晃晃地站立起来,“我要加倍努力,努力修行,不断提升。眼前的傀儡算什么?终有一天,我会成为和爹娘一样的人物,成为天庭的蛊仙,遵循他们的道路!”

“哈哈哈,好志气,你的未来真是令人期待呢。”蛊师由衷的赞叹道。

青年……

“这就是蛊仙的境界吗?”龙公渡过灾劫,成为六转蛊仙。

“了不起,小子,你这个年龄就已经成仙了。想想我在你这个年纪,还在争风吃醋呢,哈哈哈。”天庭蛊仙笑道。

“多谢您为我护法。”青年龙公向蛊仙鞠躬,态度十分诚恳真挚。

天庭蛊仙摆摆手:“你大概也明白了吧,你的身份很特殊。你是未来仙尊的护道人,所以你的修行旅途向来顺风顺水,就算有所挫折,也总是意味着更大的收获。”

“天庭需要你,未来的仙尊需要你,人族需要你,小子,成了仙不是你的终点,而是你的起点。万万不可麻痹大意,止步不前了啊。我们都需要你呢。”蛊仙向他眨了眨眼睛。

“是的,我一定会坚持下去,直至我老死的那一刻。为天庭,为人族,为未来奉献出我的每一分力气,每一滴血和汗!”龙公发誓道。

壮年……

“八转修为,难以想象……你这样的年龄竟有这样的成就!”铜公感慨道。

“或许得说,不愧是护道人吗?哈哈哈,有了你,天庭更加兴旺了啊。”眉公朗笑着道。

“二位前辈谬赞了。”壮年龙公谦虚道。

眉公和铜公对视一眼,前者道:“那么,我们的提议你考虑好了吗?成为我们当中的一员,领袖天庭,指引还未出世的仙尊成材。”

龙公点头,毫无犹豫:“是的,我早就准备好了!”

“哈哈哈,那么从今天起,你便是龙公了。”

“天庭三公之一,也是最重要的一位。天庭的重担主要靠你挑了。”

龙公缓缓点头,满脸肃容:“这是我一直追求的道路。是父母,是我成长的路途中无数前辈期待的结果,更是天庭的需要,人族的需要!对于这份责任——我,龙公,当仁不让!”

晚年……

“回头吧,洪亭,回头吧!”龙公望着红莲,呼唤着。

“师父,我辜负了你的教导和期待,对不起,但我一定要复活他们!”红莲没有转身,身影消失在夜色里。

龙公痛苦地闭上双眼,身躯摇摇欲坠。他多年来悉心教导的徒弟,居然会走上这样的道路。

“我将一位仙尊教导成了魔尊,我有罪啊!”龙公跪倒在铜公、眉公的面前。

“起来,快起来。”铜公、眉公双双拉起龙公的胳膊,纷纷开口安慰。

“要有信心啊,龙公,我们还有希望。”

“错不在你,龙公,多年来你的努力和付出,我们都看在眼里。是洪亭让我们失望,而不是你。”

“不。”龙公挣脱两人的搀扶,面容严肃至极,“是我教导出来的徒弟,如今他走上歧途,是我的错!也是我的罪!我一定偿还,我会拼尽我所有的力气去弥补这个错误!!为了天庭,为了人族,哪怕我付出所有,也要把红莲拉回正道!!!”

暮年……

龙公垂头,盘坐在大殿中,披头散发。

两行泪从他的脸颊滑落,悄无声息。

在他的面前,大殿冰冷的地砖上摆放着一具具他并不陌生的尸体,这些都是龙人的尸首。

心里的悲痛铺天盖地,呼啸成海,几乎要淹没龙公。

他原本享受着天伦之乐,但没想到暮年竟迎来如此打击。更悲哀的一点在于,让他的子孙死绝的凶手,正是他自己!

是他自己一手造成的惨状。

“是我的错,我当初就不应该……”龙公哽咽,缓缓抬头,目光似有千钧之重,缓缓扫视殿中的每一具尸首。伴随而来的是一声声欢声笑语,一幕幕爷孙怡乐的温馨回忆。

“你们,都没有错,我的孩子们!错的是我啊!”龙公佝偻着背,用拳头无力地捶打自己的胸口,砰砰的声音在空荡的大殿中回响。

“是我对你们照料疏忽,是我没有引导好你们,让你们走上了歧途。我错了,我不是一个好师父,也不是一个好先祖。但为了人族大业,为了天庭,我只有牺牲你们。”

“相信我,这样的结局绝不是我想要的。但我没有办法,我虽是龙人,但我的心一直都是人族之心。我不能放任你们扰乱人族大业,那是多少代人前仆后继,不计牺牲换来的成果。最重要的是,你们的根也在人族之中啊!”

“你们要怪就都来怪我吧,我不能祈求你们的宽恕,也不敢祈求你们的原谅。那就让所有的罪孽都由我来担负,都由我来偿还!”

“我的孩子们……我绝不忍心和你们分离。你们会和我融为一体,我们永不分离!作为你们的先祖,我愧对你们,没有给予你们美好的生活,安定的环境。但我也只能做到这一步了。”

“我龙公,不过只是个行将就木的老人罢了。”

现在……

第五剑向龙公飞速袭来,速度之迅猛,已近在龙公面前!

哧。

一声轻响,时间像是被放缓了无数倍。

龙公的头颅轻轻的,轻轻的飞离他的脖颈。

血液从脖颈处缓缓的,缓缓的喷涌而出。

他被方源一剑枭首!

\end{this_body}

