\newsection{暗里交锋}    %第三百六十三节:暗里交锋

\begin{this_body}

武庸也考虑到天庭的消息真假的这一层。

然而,紫薇仙子提供的证据很多,还有当时方源虽然变化了容颜,和黑白等仙,乘着上极天鹰离开战场,也被许多南疆蛊仙目睹。

武遗海的失踪太过突然和诡异,如果武遗海就是方源,那么一切的疑点就都能解释通顺。

“将武遗海的命牌蛊、魂灯蛊,都通过宝黄天,给我送过来。除此之外,还有一些仙蛊,武罚你去内库给我提出来。”武庸站在窗边,俯瞰着层层叠叠的梦境,暗中下了命令。

与此同时,他对远处的池曲由道:“池家太上大长老,还请你出手布阵,圈住此地,防止对方动用宇道手段逃窜。”

“哦?对方有这样的手段?”池曲由讶异了一下,旋即点点头,“我的确可以布置这样的蛊阵,干扰宇道运转,但我此时身上的仙蛊并不足够。”

“无妨。”武庸笑了笑,“此时此地,有我南疆广大正道仙友在此,还怕没有仙蛊用吗?”

没有正道蛊仙反驳武庸的话,都依据池曲由的要求,看看各家手头上都有什么仙蛊,可以能够凑成一套蛊阵。

实在没有,还可以借助宝黄天输送。

尽管价值不菲。

但自从方源开了这个由头之后,南疆这块儿的蛊仙已经开始效仿了。

武家向武庸输送蛊虫的事情,很快就被方源等人探知得到。毕竟宝黄天是非常开放的,任何的仙蛊交易绝对动静不小。

“这些仙蛊,几乎都是武家内库中的。看来是通过宝黄天输送啊。”

“还有我的命牌蛊、魂灯蛊!哼!”

方源顿时有了一股不妙的预感。

可惜他虽然明明知道这一点,但是却不能阻止。

宝黄天的交易,非常安全。方源曾经依仗这一点,但现在轮到他的敌人借助这个优势了。

很快,方源掌握的信道凡蛊,就再次接到了武庸的消息。

“武遗海,或许你应该称呼你为方源?我们可以谈一谈,做一笔交易。用你身上的仙蛊,来买你的性命。如何?”

方源瞳孔微微一缩:“看来这个秘密已经被天庭抖露出来了。刚刚的那些蛊虫,都是送到武庸手中的吧。直接说买我的命?看来他应当是有某种手段……”

念及于此,方源主动停下和其他蛊仙的配合。

“你们练习着,我要处理一点小麻烦。”抛下这句话,他立即闪身一旁,开始催动逆流护身印。

武庸的仙窍中,蛊虫纷纷飞起,如云层叠叠,几乎是遮天蔽日。

“这是南疆曾经的超级家族巫家的招牌杀招,曾经带给南疆蛊仙界灰暗和恐怖。我武家得之,也是秘而不宣,就是担忧被其他家族联合排挤。现在,方源你就给我好好尝一尝罢。”

武庸眼中寒芒陡然爆闪。

仙道杀招催动成功!

但是在他的手段下,气息收敛起来,就算是站在武庸身旁的乔志材都没有觉察出分毫来。

下一刻,武庸猛地色变,雄躯狠狠一震,吐出一大口的鲜血。

“武庸大人,你怎么了?”乔志材大惊失色,连忙跑过来搀扶住武庸。

“怎么会这样?!幸好我只是想要警告和试探,并未完全催发威能。”武庸心中充满了震惊,明明是对方源施展的杀招,他居然成了遭难的对象。

武庸轻轻用力,挣脱开乔志材的手。

他的神情迅速平静下来,咳嗽几声:“无妨。是紫血先河阵时留下的暗伤而已。”

乔志材的脸上,顿时浮现出感动和敬佩的神色:“武庸大人身上有伤,却心系弟弟,不顾伤势赶来救援,实在是我正道的榜样!”

武庸:“……”

他不动声色,沉默了一下,对乔志材道:“你很好。”

乔志材哪里料得到武庸说的是反话,还以为自己马屁拍得正好,心中顿时有些得意。

武庸之所以受伤,自然是因为方源身上的逆流护身印。

“逆流河减少了细微,看来武庸的确掌握着某种厉害手段,能够根据我的命牌蛊、魂灯蛊,来直接攻击我的肉身。可惜碰上了逆流护身印,尽数逆反了回去。”

方源吐出一口浊气,他又开始思考:“有点奇怪。武庸既然知晓我是方源,怎可能没有考虑我拥有逆流护身印?”

“难道说……他是不知道我还有一个身份,就是柳贯一。是天庭没有告知他?还是天庭本身也不清楚?”

方源想到梦境。

他这一次之所以暴露身份,都是因为外显的超级梦境,被天意侵蚀。

他多次进入梦境探索,魂魄入梦,没有见面曾相识以及暗渡仙蛊的伪装,自然被天意瞧出真身,得到位置所在。

如果是这样子的话,柳贯一的身份就没有暴露!

“糟糕。”方源想到这里,忽然心中又一沉,“我现在动用了逆流护身印,身上又中了侦查杀招,岂不是不打自招,告诉别人,我就是柳贯一?!不管天庭之前有没有瞧出我的这层身份,这一次我动用逆流护身印,必定是身份暴露了。”

这是一场暗里的小交锋。

别开生面。

涉及到方源、武庸还有天庭三方面。

武庸的手段反噬自身,但他伤势并不严重。方源虽然没有受伤,但是却暴露出了不少信心。天庭占据幕后,没有动用自身力量,却让武庸和方源死磕。

这是紫薇仙子的算计。

这位天庭的智道蛊仙,在掌握了关键线索和情报之后,立即展现出了她强大的策划谋算的能力,并且手腕非常的巧妙和老道。

武庸随后又传信方源,但不管什么,方源都不做理睬。

上古战阵四通八达的演练,终于收到了成效,距离成功越加接近。

不过方源此时却成了“拖累”。

他为了防备武庸的出手,必须时刻维系着逆流护身印,导致心神牵扯很多,很难将注意力集中在四通八达上面。

不过,方源故意放缓了节奏之后,终究还是一点一滴地催动起来。

璀璨的光辉,照耀周边。

光芒骤然消散,方源等人都消失在原地。

强烈的宇道波动,让围剿在梦境外围的南疆正道蛊仙们,都或多或少有所感觉。

“他们走了。或许还留下一两位余孽。”武庸面目表情,心中并不意外。

“可是我们的蛊阵还没有建好。他们居然有这样的手段?为什么不早早地用出来?应当是有什么严重的弊端。”乔志材推测。

池曲由则抚须道:“无妨。我的这座超级蛊阵虽然没有布置完整,但也起到了某些效果。我相信此时此刻的他们,应当已经分散开来了。”

许多南疆蛊仙懵懵懂懂,武庸则不吝夸赞道:“池家太上大长老果然好本事!”

武庸既然知道方源等人掌握着上古战阵四通八达,也预估到他们会用这种方法逃生,怎么可能不做出相应的应对呢?

早在之前,他就悄悄传音给池曲由,让他做出克制和针对。

池曲由乃是八转蛊仙,阵道大宗师,此时得到具体情报,有了具体的目标,自然能够做到正确的应对,便在此刻建功。

噗!

白凝冰刚刚踏足地面,就吐出一大口血,浑身委顿,脸色苍白得渗人。

“这不是目的地!”白凝冰龙瞳微微一缩。

她连忙联系其他人。

上古战阵四通八达算是被破了,四位蛊仙被分散出去,相距遥远。

除了方源之外,她们各个都有伤势。

“怎么办,宗主?”妙音仙子传讯来问。

方源犹豫。

是自己单独乘坐上极天鹰逃走?还是浪费时间,去重新集齐影宗群仙?

此时,纯梦求真体基本上都已经自爆成梦境,留在了掠影地沟。

还有少数几位,留在黑楼兰等人的仙窍当中,包括影无邪也在白兔姑娘的仙窍福地里。

梦境是被天意侵蚀的,纯梦求真体也因此被天意时刻关注着,留在身边就等若泄露行踪。

不过纯梦求真体在关键时刻,也能自爆成梦境,用来阻挡仙蛊屋。所以方源还是勉强留在身边。

自己身上的侦查杀招,一直都没有解除。又有命牌蛊、魂灯蛊落在武家手上,关键是武家居然有手段,可以凭此攻击方源。

情势可以说是非常的糟糕和危险!

任何一座仙蛊屋,方源都难以应对。上极天鹰并不可靠,虽然是八转战力,但是遭遇强敌,上极天鹰便不会冒着生命危险去死斗。方源对它的操纵,还有待提升。

暗渡仙蛊也缺失了,这就导致天意将时刻监控方源,然后从容布局。

“并且这些人的身上,说不定也被种下了侦查手段。”方源目光阴沉。

从掠影地沟出来,方源的下一个目的地便是西漠。

天下五域中,影宗在西漠的残余资源最多。因为影无邪、紫山真君等人,都在其余四域晃荡过,将能够搜刮的残余资源和仙蛊都搜刮到手。惟独西漠遗漏。

“单独前往西漠吗?”

“就这样将她们当做弃子,有些可惜。而且不解决掉我身上的侦查手段,始终会被他人掌控行踪。”

“琅琊派……”(未完待续。)

\end{this_body}

