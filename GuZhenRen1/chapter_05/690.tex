\newsection{武庸vs陈衣}    %第六百九十三节:武庸vs陈衣

\begin{this_body}

轰!

一声巨响,天地瞬变。

武庸一马当先,率领南联诸仙杀破第五阵,来到第六阵。

第六阵中,一片黑沉之色,一眼望不见边际。

南疆诸仙置身其中,体型渺小如蚁。

“这是宇道大阵!”一位罗家的蛊仙忽然开口。

他一副南疆传统服饰,脸皮青白,神色冷酷,乃是罗家大将,名传南疆,拥有七转巅峰战力。

他姓罗名然,专修宇道,造诣雄浑身后。

“不错,这的确是宇道大阵。”池曲由点头附言,“且让我先推算一番。”

群仙无人异议。

进入大阵之后,胡乱蛮干是不明智的。最正确稳妥之法,便是让阵道蛊仙推算出大阵运转的奥妙,再针对阵法破绽,如此一来才能事半功倍,受到良效。

池曲由推算了片刻后,淡淡地道:“这大阵已被我推算得七七八八,跟脚便是无空阵。”

罗然听闻,眉头微皱:“我虽不是阵道蛊修,但无空阵我早有听闻,乃至宇道中最经典的仙阵之一,在远古时代就被开创出来。开创者是谁,无人可知。此阵只有一项威能,便是扩宽距离,延长路线。进入阵中,一步距离可变作百步。”

池曲由笑了笑:“罗然仙友见识不凡,正是无空阵。不过此阵已经经过一番改良,里面暗藏着一层大阵变化。”

有人便问:“什么变化?”

池曲由摇了摇头:“我暂时还无法推算出来,只是隐约察觉。不过我们可以先行启程,一边赶路一边容我推算。按照目前的情况,我们可以随意前行,并无不妥之处。无空阵本身并无杀伐威能,只是困敌,拉大敌我距离,拖延时间。”

“给我一段时间,我便能推算出这层变化,进而查勘出大阵破绽。期间若是天庭提前动手,只要我们防御住,让我见识到是何种变化,更能助推我的推算。”

池曲由的建议很好,群仙当即启程,就选择了正前方,一同疾飞。

无空阵毫无光明,一片黑幽,深邃无比。

南联群仙将池曲由保护在最中央,武庸则始终行走在最前端。

“你们看,前方似有异物!”姚家蛊仙姚天择忽然开口。

群仙纷纷减缓速度,慢慢走近,仔细一瞧。

只见一个青白色的大萝卜,大如房屋,直愣愣地悬浮在空中,一动不动。

“这是什么东西?好像是一棵太古仙植!”姚天择疑问。

其余人纷纷皱眉,眼前之物并非寻常可见。

武庸沉吟不语,他脑海中有一抹灵光,似乎自己曾经在哪里见到过这种仙植的记载,但急切间想不起来。

“若我没有记错,这应当就是太玄冰芒萝卜,乃是太古时代的仙植,早已经绝迹了。没想到天庭居然还有保留。”乔志材开口道。

他是乔家的太上大长老,虽然只有七转修为,但专修木道。

武庸双眼顿时一亮:“没错,就是太玄冰芒萝卜。大家小心,切勿靠近。一旦距离过近,这个萝卜就能暴射出亿万冰芒,能将人每一寸肌肤、脏腑都刺穿,然后在瞬间被冻死。”

乔志材接着详细解释道:“是的。大家千万不要小看冰芒,这些冰芒都是太玄冰芒萝卜上的冰雪道痕所化。冰芒暴射之后,太玄冰芒萝卜再无丝毫道痕加身,沦为凡材。所以这样的一场冰芒暴射,威能不下于八转蛊仙的仙道杀招。”

群仙闻言,不由地往后稍退,提起十二分戒备。

但就在这时,异变陡生。

空气忽然荡漾起涟漪,大阵猛地转动,一股庞大的宇道力量骤然爆发在一位南疆蛊仙身上。

南疆蛊仙早已撑起防御手段,但宇道力量却不是攻破他的防御,而是将他整个人都挪移出去。

挪移的地点,自然不是别处,正是太玄冰芒萝卜跟前!

太玄冰芒萝卜好似小女孩受到了惊吓,一下子身体内缩,缩得只剩下三分之一的大小。

然后,下一刻,蓬!

亿万冰针,白茫茫一片,射向四面八方。

南疆蛊仙早已撑起最强防御手段,但只挡了半个呼吸,就被茫茫冰针射成了筛子。

随即,又冻结成冰,硬如石像。

南疆群仙纷纷动容。

无空阵忽然催发,将阵中蛊仙腾挪掉转,显然就是池曲由之前所说的变化。配合太古冰芒萝卜,可谓相得益彰。

武庸脸上闪过一抹青色。

他率领南联群仙,开战至今,这还是第一处减员,真正有人丧了命。

武庸心中急思,想要找到应对之法。

就在这时,异变再起。

空气中荡漾涟漪,一股无形无质的宇道力量,沛然难挡,狠狠地攥住另一位南联蛊仙。

“有我在,休想故技重施!”关键时刻,罗然轻啸,催动一只仙蛊。

仙蛊威能卷席四方,瞬间令周围空间凝固如冰。

南联群仙均感到一股强大的束缚,仿佛自己在这一刻被封印在了冰棺之中。

七转仙蛊镇宇!

一只七转仙蛊,显然对付不了这座天庭大阵,但也并非没有效果。

镇宇仙蛊为落难的南联蛊仙,争取到了关键的几秒钟,使得他顺利脱离原地。

大阵搅动的空间涟漪,宛如怪兽张口,却扑了个空,只能抱憾消失。

罗然随即解除镇宇仙蛊威能,南联群仙立即感到束缚尽去,再度自由。

“情况紧急,在下忽然出手,催发了镇宇仙蛊。此蛊攻势广泛,连累诸位仙友了。”罗然连忙致歉。

武庸哈哈大笑:“做得好!”

“这就是镇宇仙蛊?好蛊虫,正克此阵啊。”池曲由连连点头,满脸欣慰之色。

罗然微笑:“此蛊乃是我最近从地沟中收获,没成想能在这里立功。可见冥冥之中,有着天意。”

天地万物相生相克,从未有最强的仙蛊,只有无敌的蛊仙。

天庭组建的这座无空腾挪阵,有数只七转仙蛊,海量凡蛊组成,但是却被一只镇宇仙蛊克制。

有了镇宇仙蛊,南联群仙闯荡无空腾挪阵,再无什么顾虑。

又因为天庭催发大阵腾挪变化,使得池曲由推算大有进展。

片刻之后,池曲由便指出了大阵中的一处漏洞。

他对武庸道:“天庭此阵空阔无垠,从内而外去破,费心费力。不若盟主大人利用这处漏洞,勾动外界的无限风,从外而内攻破此阵,必定简易轻松啊。”

武庸顺着这处漏洞,轻松地感应到外界的无限风。

当即,狂风卷席,顺着漏洞,破开此阵。

南联蛊仙再次回到了外界。

连破六阵,只损一人,南联群仙士气如虹。

此番战况令主阵之人眉头紧皱,心情沉重。

此人一身青袍,仿佛青年书生,但目光沧桑,已有数千高龄。

不是旁人,正是木道八转大能,天莲派的前任太上大长老,当代元莲传人陈衣!

“不想这股敌势如此凶猛!如今我方还只剩三阵。”陈衣沉吟一番,对身旁之人道,“沧水仙友,不妨你我联手,出阵去战,为后方铺设大阵争取时间。”

他身旁的女仙白沧水亦是天庭蛊仙,亲自参与过上一届炼蛊大会时的宿命修复。她没有丝毫犹豫,直接答应下来。

两人步出大阵,武庸等人正对着第七阵狂轰滥炸,企图直接轰破此阵。

陈衣向南疆群仙拱手一礼:“南疆诸贤,天庭陈衣在此,不知何人敢与我独斗单打?”

武庸眼眸微微一缩,当即一甩手,放出玉清滴风小竹楼。

八转之争,七转蛊仙插不上手,但让他们进入仙蛊屋内,却可勉强有参战的资格。

巴家太上大长老巴十八哈哈大笑:“陈衣老儿,你从那乌龟壳中出来,可不太明智。就让我来会一会你吧。”

“慢。不可中了对方拖延的算计,我们一起动手!”武庸冷笑一声,下达命令。

若是武独秀在此,必然接受陈衣的挑战,一人单枪匹马上前去。

但武庸和武独秀是两种人,他这种枭雄怎可能放过群殴别人的机会?

当即,武庸一马当先,身后南疆数位八转,以及七转蛊仙驾驭着的玉清滴风小竹楼,一哄而上。

陈衣、白沧水二仙陷入围攻,却不慌乱。

白沧水一推双掌,立时苍白巨浪,铺天盖地一般压去。

武庸身形一晃,只在原处留个残影,忽然消失无踪。

巴十八面临巨浪,大吼一声,气势暴涨,一道玄奇光辉照住滔天大浪。

随后,翼浩方背后双翼猛地一扇,无数飞羽如箭,编织成漫天箭雨,洞穿大浪,射向白沧水、陈衣。

“来得好。”陈衣微微一笑,也不见其有什么动作,身边忽然显现无数大树幻景。

翼浩方的箭雨射进树林深处,只激起一阵树梢震动,大量树叶沙沙作响。

陈衣气势升腾,正要趁势反击,忽然神情大变。

武庸身形显露而出,竟就在陈衣身后。

“他竟有如此隐匿手段?!我察觉到的那个隐形之人,实是他留下的另一个幻影!”陈衣为武庸的手段感到震惊。

危急关头,他连回身转头的时间都没有。

然后,他就感到自己的肩膀被武庸轻轻一拍。

“你我一见如故,你却要走。分别在即,好友,让我送你一送。”武庸淡笑着道。

听到这番话,陈衣饶是数千年生涯,经历丰富,都骇得面无人色。

他在心中大吼:“这是仙道杀招送友风我天庭苦寻久久不得,竟是掌握在武庸之手!”

随后,陈衣失去对身体的掌控,慢慢悠悠地向前飘飞。

在飘飞的过程中,他的头发、他的衣摆、他的手脚都开始随风飘散。

“救我!”陈衣心中大吼,但却喊不出来。

他动弹不得,任何手段都用不了。

就算是因果神树杀招,都无法催动!8)<!--80txt.com-ouoou-->

------------

\end{this_body}

