\newsection{分身入梦}    %第八百八十二节:分身入梦

\begin{this_body}

见闻福地。

宫殿深深,广厦万千。树荫浓郁,夏蝉鸣唱。

久闭的宫门打开,沿途的羽民少女们跪倒在地上,她们恭迎这座宫殿的主人——夏槎。

夏槎的脸色并不好看,神情郁郁。

闭关良久,她尝试了各种办法,都没有使得自己从五转蛊师,晋升为六转。

所有的方法都用了,摆在她眼前的只剩下一条路——向方源购买六转层次的第二空窍仙蛊。

身为夏家的太上大长老,夏槎本有八转修为,但是光阴长河埋伏战,她被方源俘虏。虽然侥幸生还,但身上的八转仙窍已是被方源取走了。

方源放走夏槎之后,又趁机贩卖第二空窍蛊,换另外一种方式对南疆的超级势力进行敲诈勒索。

夏槎因此得到重修的机遇,她宛若溺水之人见到唯一的救命稻草,当然一把死死抓住。

夏槎现在已经重新修成五转,然而却卡在最关键的一步。

升仙!

她无数次升仙的尝试都失败了。

“唉。”夏槎叹一口气,彻底认清现实。

“给我准备清水洗漱,另外传出我出关的消息,并将太上二长老、三长老二人唤来。”夏槎吩咐身边的羽民侍女。

然而,足足过了半个时辰,夏槎都没有等到太上二长老、三长老,而是他们传回消息,说各有要事,不能前来。

“这两个老货!”夏槎心中咒骂,一股寒意还是袭上心头。

虽然她也清楚,早晚会有这么一天,但是当这一天真正来临的时候,她还是感到一股心冷。

想她曾经坐镇中枢,为夏家付出种种,任何政务都是一言以决,如今却只是一位五转的废人,开始要看其他人的脸色。

曾几何时,夏家的太上二长老、三长老唯她夏槎马首是瞻。甚至他们的崛起和如今的地位,都是夏槎曾经一力栽培而出。

“看来……这两个老货,也要自立门户了。”夏槎望着空荡荡的大殿,心中长叹。

她心中的愤怒并不多,更多的是一种无奈和悲凉。

说他们忘恩负义?并不准确。事实上,夏槎很理解这些夏家蛊仙的选择。

因为她的原因,夏家已经被方源勒索得太狠,库藏亏空巨大,苦心经营积累了这么多年,等若是给方源看管仓库。

夏槎就算要重修回来,也得从一步步开始,漫长的过程中消耗的资源必然也是一笔庞大数字。

将这些资源交给夏槎,还是留给自己?

夏家的蛊仙们经过初期的慌乱和彷徨,开始有了彼此间的默契。

夏槎的太上大家老之名,完全成了一个虚名。五转的修为根本登不上蛊仙的台面。

夏槎思索了半晌,这才取出一只信道凡蛊,交给羽民侍女,让她带给太上二长老。

结果到了傍晚,太上二长老的回信再次让夏槎暗怒不已。

“连太上大家老的名誉,都不值钱了么。”夏槎咬牙切齿。

原来她在信道凡蛊中,愿意和太上二长老完成一个交易,在这个交易中,夏槎愿意渡让自己的太上大长老之名,将位置传给二长老。二长老可以名正言顺地登上夏家最高的权利宝座,付出的代价是一笔修行资源。

但夏家太上二长老没有接受,婉言拒绝。

很明显,他已经和太上三长老达成了共识,准备架空夏槎,仍旧让她继续当她的太上大长老。

夏家的情况和巴家不同。

巴家的巴十八也是和夏槎一样遭遇,如今已经卸任太上大长老之位,改由新晋的八转蛊仙巴德继位。

而夏家的夏兆、夏沉渊这两位太上长老,只有七转修为,并且修行的乃是土道。

很多年前,夏槎刚刚执掌权柄时,还不是八转蛊仙,为了掌控夏家,她栽培了夏兆、夏沉渊二人,用他们来抗衡夏家的旧有势力。

如今夏槎修为跌落到五转,夏兆、夏沉渊却是位高权重,有了不一样的心思。

但在正道而言,对他们俩有着知遇之恩的夏槎,他们是不敢动的。他们宁愿将太上大长老的名头,放在夏槎的身上,也不愿意自己的名誉受到损失。

夏槎知道自己对夏家的影响力已经下滑到一个危险的地步,并且下滑的速度越来越快!

她必须要尽快晋升成仙,否则的话,她就真的要被彻底排斥在夏家的高层之外了。

然而,她没有办法挣脱这片困局,因为她手头上没有谈判和交易的资本。

夏槎的情况,并非特例,而是常态。

南疆正道的风向,已经改变了。

在这里起到最主要的作用的人物,便是武家的武庸。

他一边借助天庭,一边动用自身的手腕,在南疆正道中不断提高自己的声威和影响力。

夏槎、巴十八这些老牌八转的沦落,给了他一个千载难逢的良机。

尽管陆畏因出现,但他背后的道德乐土,不只是人族还包括大量的菇人。这在南疆正道中是天然的政治短板,被武庸死死抓住,充分利用。

陆畏因对武庸构成不了阻碍,武庸在南联中的影响力与日俱增。

南联内部已经有一个声音,越来越响,武庸就是最主要的倡导者。他倡导南联积极对抗方源,不要再接受他的勒索和敲诈。和方源这样的魔头妥协,只会助长他的嚣张气焰。

南疆正道势力都差不多收回了人质,也被方源敲诈勒索得伤筋动骨,南疆蛊仙们陆续用更积极的态度来响应武庸的号召。

武庸利用的也正是正道蛊仙们对方源仇恨之情。

月夜下,方源的纯梦求真体和池曲由并肩前行。

一场有关梦境的暗中交易,刚刚结束。

“池大人,就送到这里吧。”方源梦道分身笑着。

他是临时分身,由梦境所化,有蛊仙修为,达到时限之后就会自爆成一团梦境。

池曲由拱手叹息:“方源仙友,不是老朽刁难,实在是这个生意的风险变大了。现在风声这么紧,一旦事发,我们池家的名誉就全完了。”

纯梦求真分身心中冷哼,池曲由提出加价,已经不满足之前的价码。理由是南疆正道对付方源的力度越来越大,武庸团结到了越来越多的南疆正道。

这理由当然是狗屁,真正的原因就是池曲由借此加价,胃口大开,想要谋取更多的梦道成果。

方源想要梦境,但如今勒索南疆正道越加艰难,池家和方源的暗中交易又成了主要渠道。

池家掌握着梦境大阵,南疆独此一家,方源只有和池家合作,才能盗取当中的一部分梦境。

“这个事情,我会向本体说明的。”梦道分身丢下这句话,直接飞离。

池曲由留在原地,看着方源的梦道分身消失在天际,面色转冷,轻哼一声,转身离开。

他不是没想过对梦道分身下手。

但他没有梦道手段,能够对付梦道蛊仙。再者,他也忌惮周围很可能就有方源本体在埋伏着。

方源本体远在东海,但为了这次交易,吴帅驾驭着龙宫早就埋伏在附近。

池曲由若要动手,定然是讨不了什么好的。

纯梦求真分身很快就回到龙宫之中。

这是临时分身,有仙级修为,但到达时限之后,就会分崩离析,化为一团梦境。

“池家老儿的买卖做不得了,除非我们提高买价。”梦道分身冷声道,他打开仙窍,放出几个梦道分身,“这批梦境恐怕是最近我们能得手的最后一批了。”

这也是没办法的事情。

方源和池曲由之间,并没有什么盟约的约束。

池曲由临时加价,还借助了南疆大势。

“大敌在前,过了这个坎儿,区区池家算不得什么。”吴帅开口。

眼下,他还不能随便出手。

出手次数多了,恐怕就要让天庭察觉到吴帅分身和方源本体的真正关系。

梦里轻烟杀招还是要留给天庭,用在池曲由身上,未免有些杀鸡用了牛刀的嫌疑。

与其对付池家,反而不如直接进攻大阵更有效率。

事实上,方源本体逗留在东海,就算只凭吴帅这边的实力,对付池家也是可以的。

但是提前暴露出这些底牌,天庭只怕做梦都要笑醒。

小不忍则乱大谋。

况且让南疆保存实力,将来也能给天庭造成更多的麻烦。

直接消灭敌人,虽然干脆利落,但有时候并不能收获最大的利益。就比如方源放走了夏槎和巴十八等人,这些人已经开始在各自的家族遭受排挤和冷落。

方源会在将来暗中资助他们。

他们如今的处境,可以说是方源一手造成的。但是没有方源,他们也依靠不了其他人。

而方源则需要这些人作为自己的耳目,安插在南疆正道势力之间,令他对南疆始终保持着一定的影响力。

只是当下,方源还欠缺一些制约他们的手段。

关于这些,宙道分身已经在夜以继日地进行推衍。

砰砰砰。

一声声的闷响中,临时的纯梦求真体纷纷自爆开来,化为一团团的梦境相互融汇成一体。

这些都是土道梦境。

吴帅端坐殿中龙椅,遁出魂魄,开始入梦。

------------

\end{this_body}

