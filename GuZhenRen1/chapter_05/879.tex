\newsection{积土为山}    %第八百八十三节:积土为山

\begin{this_body}

%1
天地一片阴沉,乌云盖顶,万雷轰鸣。

%2
猩红的闪电,一次次闪烁在漆黑的滚滚烟云之中,黑云卷荡漩涡,从漩涡的中心降落下一颗颗的雷球。

%3
这些雷球形态奇异,宛若陶瓷制品,大如水缸,红光烁烁。

%4
浩劫——红瓷雷球劫!

%5
雷球砸落下来,爆发出清脆的炸响。

%6
地面上,一座仙蛊屋默默承受着雷球炸裂的伤害,大片的残屑不断飞舞,抛洒在半空中后,就还原成无数只蛊虫的尸体。

%7
仙蛊屋内,吴帅忍受着耳畔的雷声,先是检查自己。

%8
他发现自己成为了一个土道八转蛊仙,手指宽大,身材魁梧,一身战袍上已是布满累累伤痕。

%9
而在他不远处,有一位女仙委顿在地,脸色苍白,气息不仅微弱,而且还有些怪异。

%10
她正在催动着某个光道的治疗杀招,因此浑身上下笼罩着一层圣洁的白光。但在她的脸上,裸露在外的手臂、腿脚上,却显现出一道道墨线般的暗道道痕。

%11
这些暗道道痕是如此的浓郁,达到九转仙材的程度,形成了道痕光晕,蛊仙们只用肉眼就能直接看到。

%12
女仙的修为,明显也是八转。

%13
一位八转的光道蛊仙,居然身怀如此规模的暗道道痕,实在叫吴帅诧异。

%14
更诡异的是,女仙身上的暗道道痕还在变化当中。不仅是数量不断激增,而且位置和排列也在迅速变化。

%15
女仙闷哼一声,显然是杀招催动结束,她身上的光晕骤然稳定下来,暗道道痕的剧变也被暂时压制了。

%16
女仙浑身大汗,睁开虚弱地眼帘,看向吴帅,苦笑道:“我已经没有救了。这是我最强的治疗手段,但也只能暂时压制伤势。等到杀招时限一到,伤势会爆发出恐怖的威能。”

%17
“郑央啊,我的挚爱,我不能再陪伴你身边了。你要好好的活下去。”女仙说着,眼泪落下。

%18
“不,一定有什么办法,一定有的!你千万不要放弃希望,我的这座安土重山堡可是用尽了我手中的仙蛊,防御威能堪称古今第一。我们还有时间,你要继续努力,我会一直陪伴你左右的!”吴帅含泪大吼。

%19
这当然是梦境的自然演化,不受他本人的操纵。

%20
不过这段对话之后,吴帅就真正有了控制能力。

%21
“安土重山堡?没有听说过……”吴帅双眼精芒爆闪,从刚刚那段简短的对话中,不断分析,搜寻线索。

%22
他又检查自身仙窍。

%23
仙窍中存有不少的白荔仙元,还有大量的凡蛊,但没有仙蛊。

%24
正如之前郑央所说,他将所有的仙蛊都用于搭建这座八转仙蛊屋安土重山堡了。

%25
“也就是说,我最主要的凭借就是这座破损的仙蛊屋了?”吴帅顿时有了明悟。

%26
他开始尝试操纵这座仙蛊屋。

%27
安土重山堡的结构相当复杂,耗用了上亿的蛊虫,其中仙蛊就多达二十多只。

%28
吴帅刚开始接触,居然发现不了哪一只是当中的核心仙蛊!这些仙蛊之间位置巧妙,关系神秘,整个仙蛊屋的设计博大精深,宛若一道高耸雄伟的山脉,蔓延万里,连天接地,山脉里是一层层、一重重的矿脉,种类繁多,相互纠缠。

%29
吴帅就好比是一个凡人蛊师,要在这座山脉中开矿,一层层地挖掘出自己想要的矿物,搞清楚这些矿脉之间的联系。

%30
“单单了解这座仙蛊屋的难度就太大了!并且我还要靠这座仙蛊屋,来抵御红瓷雷球劫!”

%31
吴帅心中狠狠一沉,知道自己必败无疑了。

%32
果然,接下来梦境的发展一如他所料。

%33
红瓷雷球接连不断地轰炸,吴帅尝试着催动仙蛊屋,并且进行修补。

%34
但是无济于事。

%35
他对这座仙蛊屋的了解太少了,就算想要修补,也不知道从何修起。

%36
最终,红瓷雷球炸毁了仙蛊屋,连同吴帅和女仙一起,成了雷劫下的牺牲品。

%37
出了梦境,吴帅便立即治疗自己。

%38
这一次,方源本体探寻龙鲸乐土,临走之前早有充分的准备和安排。不仅是将梦道蛊虫交给了吴帅,同时还有大量的胆识蛊。

%39
有了胆识蛊,吴帅的伤势就不是什么问题。

%40
这样的梦境探索可是和当初探索龙宫梦境不同,吴帅有许多次可以重来的机会。

%41
第二次失败,第三次失败,第四次,第五次……

%42
八次之后,吴帅总算对这座安土重山堡有了一个较为清晰的认知。

%43
虽然没有彻底将这座仙蛊屋修补好,但是吴帅的效率已经抵得上红瓷雷球的轰炸。

%44
天空中黑云翻腾,红瓷雷球渐渐稀疏,直至彻底停息下来。

%45
但是黑云中,竟又开始酝酿第二场浩劫来。

%46
女仙叹息:“唉……都怪我没有听你的劝说啊,去探寻狂蛮魔尊的葬身之地,结果惹来这场诅咒,导致灾劫不断。”

%47
“是我活该。郑央,快走吧,我不愿拖累了你!”

%48
女仙的身上,光晕已经明显地暗淡下来。而那些暗道道痕则紧密排列起来,像是一片片缝纫伤口的黑色丝线,布满女仙的皮肤和脸面。

%49
相比之前,女仙气息又衰落了一大截,同时气息更加古怪,不只是光暗两道纠结,在她的脖颈上竟长出了喉结!

%50
吴帅还未来得及细细分辨女仙的情况,天空中开始下起了淅淅沥沥的小雨。

%51
雨水虽小,却是冰寒彻骨,有一种冻僵魂灵的冷。

%52
之前雷球轰炸的暴躁和高温,迅速消散无踪。

%53
呜呜呜……

%54
凄厉的鬼嚎声由小渐大,迅速冲天彻底。

%55
一根根的灰色丝线,从天空的黑云中垂钓下来,向吴帅迅速延伸过来。

%56
吴帅心头一沉,辨认出此等灾劫来:“这是浩劫——鬼灵冰蚕劫!一场浩劫紧接另一场,看样子要绵绵不绝啊。”

%57
吴帅只好硬着头皮,抵御这些蚕丝。

%58
红瓷雷球爆裂狂猛,而鬼灵蚕丝却是阴柔刁钻。

%59
吴帅苦无攻伐手段,凭借的仙蛊屋安土重山堡也是一知半解,至少他现在还未发现什么攻伐手段。

%60
因此,吴帅只有眼睁睁地看着这些蚕丝,伸到面前来,不断地顺着仙蛊屋的缺口和缝隙,渗透进来。

%61
吴帅出手抵御,但是安土重山堡已是被蚕丝彻底渗透,支撑了半晌后,轰然崩溃。

%62
蚕丝包裹住女仙,也对吴帅下手,将这两人裹成结结实实的白色蚕茧。

%63
下一刻,吴帅被逐出梦境。

%64
这一次,他惨遭重创,眼前一阵阵发黑,喉结滚动,一阵阵干呕,几欲昏迷。

%65
“在鬼灵蚕丝劫这个环节下失败,受到的创伤比之前还要严重十倍!”吴帅苦笑,如此伤势,只有休养生息,暂缓攻略了。

%66
这等伤势放在方源本体身上,只能算是轻伤,并不严重,甚至还能继续探索梦境。

%67
但在吴帅分身这里却不一样。

%68
吴帅分身的魂魄底蕴,虽然有三千万人魂,之后虽然在东海修行也有所提升,但和本体是无法比较的。

%69
吴帅身为龙人,走的是奴道,魂道修行只是辅助。魂魄底蕴太强,就是魂道道痕浓郁,反而会限制奴道上的发展。

%70
而方源本体因为至尊仙体,异种道痕不互斥,所以魂道等种种流派的修行并不受限。

%71
吴帅休整良久,方才再次出动。

%72
一次次的失败,又一次次发起挑战,吴帅处境十分艰难。

%73
他没有宙道分身推算,虽然掌握龙宫,但梦里轻烟杀招却对探索梦境没有什么帮助。

%74
最值得依赖的手段,除了解梦杀招之外,就是胆识蛊为后备靠山的恢复能力了。

%75
一步步推进下去,吴帅终于有了阶段性的成果。

%76
安土重山堡被他探查出了大概,吴帅总算是掌握了这座仙蛊屋的三记杀招。

%77
第一个叫做积土为山,第二个名为入土为安,一个称之为卷土重来。

%78
叫吴帅大开眼界的,便是第一记杀招积土为山。

%79
这杀招十分奇特,不用于攻伐,也不是腾挪、治疗等等威能。它是土道杀招,却有阵道效果,在这个杀招下,蛊仙能够随意堆叠土道蛊虫,使得它们自行组成仙蛊屋。即便催动杀招的蛊仙的阵道境界一片空白!

%80
正是因为这个杀招,才有了这座安土重山堡。

%81
掌握了这个杀招,也就掌握了安土重山堡这座仙蛊屋的根!

%82
难怪之前吴帅找不到核心仙蛊,因为这座仙蛊屋的核心不是仙蛊,而是这记仙道杀招积土为山!

%83
“所以,我只需要催动这个杀招,然后直接往仙蛊屋内狂撒土道蛊虫,就能迅速修补仙蛊屋了。”

%84
明白这一点后,吴帅顶着鬼灵蚕丝劫硬是将安土重山堡修好。

%85
轰!

%86
就在修好的那一刹那,女仙陡然自爆,爆炸的威力竟比浩劫还要恐怖,不仅将吴帅瞬间摧毁,而且还将整座仙蛊屋都撕扯成无数碎片。

%87
“所以,我还得照顾好这个女仙人。”明白这个要点后,吴帅再次进入梦境。

%88
仙道杀招——入土为安!

%89
动用这记手段,女仙遭受良性封印,伤势得到镇压,并且持续接受仙蛊屋的治疗。

%90
当吴帅彻底修补好仙蛊屋,第三场浩劫出现了。

%91
天空中坠下紫色的星辰,这是毒道、星道合流的灾劫——腐毒紫星劫。

%92
紫色星辰不断堕落,落到仙蛊屋上,腐蚀出一个个的坑洞。

%93
吴帅全力修补,苦苦支撑,一次次失败却总不见这场浩劫停歇。

%94
“第三场浩劫似乎无穷无尽,这很不正常。难道说?”失败多次之后,吴帅心中闪过一道灵光。

%95
他尝试着将女仙的封印打开,没有了入土为安杀招镇压,女仙再次到了濒死自爆的边缘。

%96
“郑央,若有来世,我还愿与你相遇。再见了,我的道侣。”

%97
在吴帅艰难抵御浩劫的时候,她忽然顺着仙蛊屋的破洞,电射出去。

%98
轰!

%99
她陡然自爆,化为一团漆黑的火焰,灼灼燃烧,气息十分恐怖。

%100
天道仿佛被激怒,一时间亿万的紫色星辰攒射而下,黑火并不庞大,但遭受这些毒星轰击,却是坚韧无比,体积虽然在缩小,但程度微乎其微。

%101
见这场浩劫不行,苍穹中黑云再变,无数雷霆电蛇猛烈劈下,黑火在一瞬间就被上万道雷霆轰击。

%102
黑火却像是吃了大补药,体型膨胀起来,很快就大若小山。

%103
只是黑火的形态彻底发生了转变,变成了一片白金色的水雾。

%104
天上的浩劫不断变化,轰击白金水雾,水雾也不断变化,时而转为轻风,时而变作铁石,时而化为花草鸟兽。

%105
“这女仙到底身上潜伏着什么。如此众多的浩劫,居然几乎奈何不了!”吴帅心头震惊。

%106
他敏锐地感觉到,这场写实的梦境中记载着某个秘辛。

%107
这个秘辛恐怕是关乎尊者死因,关乎天地间最深处的那一层奥秘。

%108
女仙死了,但吴帅的梦境还在继续。

%109
他必须要在这恐怖的余波中生还下来,依靠的仍旧只有安土重山堡。

%110
最终,当无数场浩劫轮番降临,一丝丝地终于将那个诡异的黑火彻底消弭。

%111
最后一点的残渣,都在浩劫的针对下,顽强支撑了大半个时辰。

%112
“这鬼东西到底是什么?”梦境消解,吴帅的心中仍旧带着震惊。

%113
他付出这么多,收获也是惊人的。

%114
几乎是在梦境消解的同时,远在龙鲸乐土的方源本体眼中骤亮。

%115
“哦,晋升为土道宗师了啊。”

\end{this_body}

