\newsection{自爱仙蛊}    %第三百八十三节:自爱仙蛊

\begin{this_body}

北原,琅琊福地。qiushu.cc [天火大道小说]

甲字号炼蛊大厅之中,持续了数月之久的炼蛊,已经到了最关键的时刻。

琅琊地灵亲自站在一座炼道超级蛊阵的边缘,双目紧紧地注视着蛊阵中燃烧着的熊熊烈火。

这火焰呈现冰蓝之色,一点也不炙热,反而非常寒冷。

即便是超级蛊阵,也难以将这剧烈燃烧的冰焰之威,彻底隔绝起来。

琅琊地灵站在蛊阵边上,整个眉宇、毛发都染上了一层薄薄的幽蓝冰霜。

但他不管不顾,全神贯注地注视着火焰。

更准确的说,是火焰中央的那团玄冰。

玄冰在不断地熔解,又在不断地生成。最初的玄冰,棱角分明,如今它在琅琊地灵亲手炼制之下,已经被渐渐地磨平了棱角,有了一丝浑圆之感。

“扇门风五缕。”琅琊地灵忽然开口道。

为他打下手的,正是毛民蛊仙毛六。他闻言连忙动手,从仙材库存中,运用特殊的手段,提取出五缕扇门风。

这扇门风,乃是七转仙材。

非常奇特。

生长的地方,不在深山老林,更非大泽苍穹,而是在凡人家的门板上。

每当有这样的仙材产生,凡人的门板就闭合不了,只能不断地开合。

这种奇怪的现象,在很久之前,就引起了蛊师们的注意,继而引发了蛊仙出手收取。

只是刚开始时,扇门风这等仙材,蛊仙们都利用不了,不了解它。如今却是早已有了纯熟的手法,以及充分利用的方案。

五缕扇门风被琅琊地灵小心翼翼地,送去冰焰中心。

很快,五缕清风就围绕着玄冰,不断旋转,像是雕琢玉石一样的五只巧手,加剧了玄冰的变化。

“不好。”好景不长,琅琊地灵忽然变了脸色,惊呼出声。

毛六心中顿时一突,连忙望去。

只见那冰焰陡然熄灭,那块玄冰直接破碎,连同那五缕扇门风,一切都化为乌有。

“啊啊啊!又失败了呀!!!”琅琊地灵跺脚,大吼,十分气急败坏的样子。

毛六深深叹息。

太可惜了。

真的已经到了最后的几个步骤。

但失败了。

之前数月的努力心血,都直接打了水漂,做了无用功。

“若是利用毛民天地流,绝不会如此结果!”琅琊地灵冷哼。<strong>最新章节全文阅读www.QiuSHU.cc</strong>

毛六连忙提醒道:“可是太上大长老,按照门规,这既然是方源长老耗费大量贡献发布的任务,我们也得按照他的要求来啊。”

“唉!这方源脑子糊掉了,怎么一心想着用人族的炼蛊法门。他难道是存心要让我难堪吗?!”炼蛊失败,让琅琊地灵的心情很是糟糕。

毛六连忙为方源说好话:“方源长老一直都神神秘秘,我虽然和他不对付,平常也看不惯他。不过呢,这一次他似乎动用了全部资本,不惜耗费如此巨量的门派贡献,要来炼制仙蛊。他要炼制的仙蛊,多达三只。如果是存心想让太上大长老您难堪,不至于耗费这么大的代价吧?”

“嗨!我也只是随口一说罢了。”琅琊地灵垂头丧气,摆摆手,“先休息,休息一下。这已经是第五次失败了。而且,方源的门派贡献也不太够,你联系一下他吧,告诉他炼蛊的结果,若是他还想再炼,就必须有足够的门派贡献!”

“是。”毛六应声。

方源如一道利箭,飞翔在高空中。

沙漠的广阔壮美,他无心欣赏,此时此刻,他的心思还在琢磨着刚刚和凤九歌的交手。

那处沙流漩涡中的光阴支流,是影宗暗地里的修行资源,记录在影宗地图之上。

“紫山真君临终之前,特意在遗嘱中交代我,一定要尽快地去往光阴长河当中,和鬼脸红莲中的幽魂意志接洽。”

虽然方源继承了紫山真君的遗藏,掌握了大量的杀招、仙蛊方、秘闻等等。

但是影宗的整个财富,他还只是继承了大半而已。

而光阴长河中的鬼脸红莲当中,存在着的幽魂意志,则掌握着影宗的几乎全部修行内容。就连幽魂魔尊的真传,它都拥有着一部分。

影宗存在十万年,幽魂魔尊生前更是纵横天下,屠戮苍生,令万物齐喑。整个影宗虽然不再,但是掌握的修行内容,却是难以想象的丰富和浩瀚。

就比如仙蛊屋吧。

方源目前,只掌握了十二座仙蛊屋的完整全面的建设内容。但实际上,魔尊幽魂和影宗掌握着的,绝不止十二这个数目。

这些其余的修行内容,都得需要方源前往鬼脸红莲当中,亲自接收。

“而且红莲真传,也在光阴长河当中。我有着春秋蝉,就是掌握着真传的钥匙。”

方源肯定是要前往光阴长河当中的。

五域外界的光阴支流,就是他进出的门户。

虽然为了铲除天庭的追兵,方源损失了其中一道,但是没有关系,在影宗的记录当中,光阴支流可是还有五六道呢。

“不过,目前西漠当中,影宗掌握的光阴支流,就只剩下了最后一道。”

“保险起见,我还是先前往那两个资源点。在没有十足的把握之前,不要轻易前往那里了。”

方源暗中做出了决定。

现在他人在西漠,西漠和中洲、北原、南疆分别接壤。

但是这三域,方源都去不成了。

中洲有天庭、十大古派,是方源的禁地。

北原有长生天,到处通缉方源。

南疆的正道蛊仙们,不久前还追杀围剿过影宗呢。

西漠是最佳的潜藏地点。

至于东海,影宗方面只掌握了一道光阴支流的位置。但是很不巧,这道光阴支流已经被庙明神搜走了。

说起来,这事情还怨方源。因为就是方源发现之后,告知庙明神一伙儿的。

曾经,紫山真君为了解情况,也借助光阴支流,去了一次鬼脸红莲。他在东海时,找不到那条光阴支流,只好到了南疆,才如愿以偿。

“我现在最大的麻烦,就是身上有着侦查杀招,位置始终暴露在天庭蛊仙的眼中。”

“同时没有暗渡仙蛊傍身,天意也在时刻布局,企图剿杀了我。”

“关键是缺乏仙蛊,我手中的杀招海量,就是独缺核心仙蛊。有了核心仙蛊,就能解除侦查杀招,也能再度蒙蔽天意!”

正思考着,方源接到了毛六的来信。

方源将心神探入到这只信蛊当中,里面的内容让他皱起眉头。

炼蛊再一次失败了。

真是该死!

虽然他陷害了凤九歌,但天庭方面绝不会只派遣他来追杀自己。

天庭一定还有后手,只是时间早晚而已。

“这只智道仙蛊自爱仙蛊,一下子要炼出七转层次来,的确颇有难度。最近几次炼蛊,我都全程参与,琅琊地灵尽心尽责,绝没有敷衍了事的成分。”毛六如此说道。

方源没有怀疑他的话。

琅琊地灵的单纯和诚信,方源是相信的。

炼制不出来,也很正常。

毕竟是七转仙蛊,成功的可能很低。

但这只自爱仙蛊,方源必须炼出来。而且要速度快。

有了它,方源才能运用紫山真君遗藏中的一道杀招,解除自己身上的侦查杀招。

这是完全可以的。

方源对它保有充分的信心。

“炼,必须接着炼,哪怕是倾家荡产,也在所不惜。”方源咬牙,暗自发狠。

他现在获取琅琊门派贡献,再轻松不过。因为紫山真君的遗藏,丰富无比。

“不过……”

“我也不能无动于衷了。”

“这一次陷害凤九歌成功,是出其不意,动用了八转仙蛊似水流年。这是天庭暂时得不到的情报。”

“如今这张底牌已经暴露,我必须抓紧时间!”

“看来,是时候动用那个方法了。”

方源眼眸中闪过一抹决意。

他回信道:这一次炼蛊,请毛六主持大局,让琅琊地灵打下手。

毛六接到方源回来的信蛊,看到这部分内容,不禁感到相当的诧异。

他自问自己的炼蛊水准,并不如琅琊地灵,琅琊地灵也没有徇私舞弊,但为何方源偏偏要让他来主持炼蛊大局呢?

光阴长河之中。

隆隆的水声,不绝于耳。

凤九歌被滔滔巨浪,卷进长河之中。

“这里就是光阴的长河吗?”他竭力稳住身形,立即感到无比的宙道威能,不断地侵蚀自己的仙躯。

强。

强大无比!

凤九歌的仙躯防护,也是极为不俗的。

但在这短短一瞬间,他就感到自己正被周围的环境极力排斥。

凤九歌十分明白,这是因为他身上满是音道的道痕,和这里的宙道道痕格格不入。

“不愧是光阴的长河,必须尽快出去!”凤九歌想要退走,但是回望一眼,之前的光阴支流已经毁灭,留下的只是一层薄薄的光阴斑斓。

这层斑斓非常的巨大,但是对于凤九歌而言,却再不是什么出路。

“这下麻烦了。我不是宙道蛊仙,单靠光阴斑斓是出不去的。该如何是好?”

正彷徨着,凤九歌的耳畔,陡然一炸。

虎吼声起,一头巨大的虎形年兽,从河底钻出来。

砰的一声巨响,河水四溅。

凤九歌瞳孔缩成针尖大小,他的身躯和这头虎形年兽比较起来,宛若老牛身旁的蚊子。

“是太古年兽!”

死亡的阴影,笼罩凤九歌。(未完待续。)

------------

\end{this_body}

