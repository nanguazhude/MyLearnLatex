\newsection{紫薇控局}    %第三百八十四节:紫薇控局

\begin{this_body}

叮叮咚咚……

悦耳动听的脆响声不绝于耳,凤九歌在这一刻拼尽全力。

仙道杀招——碧玉歌。

歌声笼罩方圆数百步距离,太古虎形年兽的虎爪被染上一层碧色。

但它的虎爪太过于庞大,仍旧轰击过来。

凤九歌咬牙,连忙躲闪。

千钧一发之际,他几乎和年兽虎爪擦肩而过。即便没有被击中,恐怖的巨力带动的风浪,也是狠狠地袭上凤九歌。

凤九歌就像是一个断了线的风筝,被吹飞出去。

噗。

他高高抛起,在光阴长河的上空划出一道弧线,吐出一蓬鲜血。

“不行,在这里我的战斗力受到了极大的压制。”

“光阴长河中充斥宙道道痕,乃是全世界宙道道痕最多的地方。道痕互斥,我的音道杀招效果不及原先的一成!”

凤九歌强忍痛楚,立即催动杀招,进行疗伤。

在外界,几乎无往不利的治疗手段,在这里却显得差强人意。

不仅是治疗手段,凤九歌的移动杀招,也变得速度缓慢。他的碧玉歌原本威效极强,但是虎形年兽受伤,只是微微一震,就将虎爪上的那层薄薄的碧绿玉石皮彻底震碎。

显而易见,碧玉歌的效果,也被削弱到了谷底。

凤九歌只是七转修为,之所以拥有了八转战力,乃是因为他一身音道道痕可以媲美八转蛊仙了。

但是,当他被方源设计,卷入到光阴长河当中后,他身上了的最强优势,反而成了他最大的弱点!

音道道痕和这里的宙道道痕相比,规模数量都差得太多太多。

这也是为什么,方源非常有自信,凤九歌落入光阴长河中会凶多吉少了。

太古虎形年兽伸出鲜红的舌头,****了一下右爪上伤口,然后目露凶芒,再次向凤九歌冲杀过来。

它本来实力就强,生长在光阴长河之中,一身宙道道痕和光阴长河相得益彰,主场优势极其明显。

年兽没有九转的存在,太古年兽已然是年兽中的霸主。

虎吼声起,风浪滔天。

巨大的虎形年兽,奔腾猛跃,向凤九歌展开最凶猛的袭击。

凤九歌拼尽全力招架。

但很快,他就再次受伤。

他完全落入下风,没有任何胜利的希望。想要求生,只有找寻到一条光阴支流,然后撤离出去。

但是这个希望,也是非常的渺茫。

因为光阴支流要碰到,并不容易。就算碰到了一条支流,未必有这样的规模,能让凤九歌出去。

凤九歌本身只是音道蛊仙,并非宙道,更没有专门的手段,来逃离光阴河流。

“难道,我就要战死在这里了吗?”凤九歌心中闪过这样的念头。

不过就在这个时候,他的耳畔忽然响起了一声女仙的低语:“用那只八转仙蛊。”

“八转仙蛊——命甲吗?”凤九歌迟疑了一下。

天庭蛊仙交予他这只仙蛊,明显是让他护住性命的。

这是一只八转防御仙蛊。

但是凤九歌一直对此心存疑惑。

为什么呢?

他本身就不存在防御方面的短板,就算是面对武庸,他的防御也能够顶上一段时间。

天庭将这只仙蛊借给他,反而不如借给他其他仙蛊,用来攻击或者移动。

毕竟,凤九歌在挪移遁走方面,要逊色一些。而他的七大歌曲,虽然威效绝伦,但都是杀招,远不如直接动用一只八转仙蛊进攻,来得干脆利落。

凤九歌接到这只仙蛊之后,也曾经尝试催动过。

命甲蛊的效果的确十分强大,但是吞噬仙元的程度,也分外骇人。即便是凤九歌这样的人物,也大感吃不消。

“若要延迟时间的话,我应当运用自己的防御手段苦捱,再不断地治疗自己。若是用命甲仙蛊的话,恐怕仙元还不够支撑五十息的功夫。”

凤九歌心中急速盘算。

这时,太古年兽咆哮着,已经杀到他的面前。

凤九歌一咬牙,关键时刻,他还是选择听从那个神秘女音的指点。

八转仙蛊——命甲!

一时间,红枣仙元剧烈损耗,凤九歌的全身浮现出一层薄薄的光甲。

但就是这层光甲,在太古年兽的狂暴凶猛的攻势下,岿然不动,护住凤九歌周全。

凤九歌虽未受伤,但是在太古年兽的猛烈拍击之下,不可避免地撞入光阴长河当中。

一落入河水深处,无穷无尽的光阴河水,从四面八方,不断地挤压、冲刷凤九歌。

但命甲表现得非常出色,仍旧毫不动摇,只是表面上出现了一些细微的裂痕。

这已经非常了不起了。

毕竟它只是一只八转仙蛊。

而光阴长河这等天地秘境,道痕的规模往往可是和九转仙蛊相提并论的。

太古年兽顺着河水,再次向凤九歌追杀而来。

凤九歌暗道一声:“苦也!”

他奋力挣扎,想要飞升上去,脱离光阴长河。

但太古年兽也有相当的智慧,不让凤九歌实现他的图谋。

凤九歌被阻挠在河流当中,不仅受到太古年兽的猛攻戏耍,而且还时刻被光阴河水冲刷排挤。

情势对凤九歌越来越不利,他已经完全丧失了主动,再这样下去,仙元消耗光了,就是他凤九歌的死期。

“难道我刚刚的选择是错误的?那道女声只是一个陷阱不成?”

就在凤九歌疑惑的时候,一股龙卷风般的流水,忽然袭来,将他猛地卷住。

凤九歌猝不及防,正要挣扎,忽然一道声音传来:“不要挣扎啊,凤九歌,我乃天庭蛊仙,特来援助于你,你可以称呼我为——黄史上人!”

说着,水流龙卷猛地扩张,体积膨胀成之前的数十倍。

无数的气泡喷涌,遮蔽了太古年兽的视线。

凤九歌不敢撤销命甲,但放任自己,顺应水流龙卷的力量,重新弹飞,来到了光阴长河的河面上空。

在那里,早有一位黄衣光头的蛊仙等候着他。

天庭。

紫薇仙子吐出一口浊气。

在她的面前,星宿棋盘正如实地将凤九歌被救的一幕,清晰地演绎出来。

那个在关键时刻,提醒凤九歌的声音,自然是她紫薇仙子。

八转命甲仙蛊,起到的真正作用,乃是给黄史上人提供一个清晰的线索,让他能够快速准确地找寻到凤九歌。

光阴长河是很难进入的。

蛊仙进入这里,远比进入黑天、白天难度更大。

因为八转蛊仙本就数量稀少,宙道八转蛊仙又只是八转蛊仙当中的一小部分。

不过即便如此,偌大的天庭中定然是有八转宙道蛊仙的。

黄史上人,正是紫薇仙子布局的一部分。

原本紫薇仙子的打算,是让凤九歌逼迫方源,方源主动或是受于压迫,进入光阴长河,寻找红莲真传。

这个时候,就由宙道八转大能黄史上人出手,实施狙杀。

但是紫薇仙子没有料到,方源居然会舍弃一条光阴支流,布下了陷阱,险些坑害了凤九歌的性命。

“方源这头天外之魔,真是狡诈至极。”

“不过经此一战,黑凡真传的底子也差不多测探出来了。”

“他虽然舍弃了一条光阴支流,当做陷阱布置出来,反而更让我坚定了之前的猜测——方源他一定会进入光阴长河,取走红莲真传!”

“依照方源的谨慎心情,如此看来,恐怕西漠上,他还掌握着至少一处的光阴支流的位置。”

“不妨事。我的侦查杀招仍在,方源的位置始终是暴露的。”

凤九歌保住了性命,紫薇仙子便再次催动星宿棋盘,让她看到另外一处的追击战。

方源的那头上极天鹰已经身受重伤,正在黑天中飞腾。

两位刚刚晋升的天庭蛊仙,在后方追击着,他们身上也有些伤势,但都很轻微。

上极天鹰虽然强势,但是面对两位八转蛊仙,仙道杀招层出不穷,又能相互配合的对手时,自然是敌不过的。

感到了死亡的威胁后,上极天鹰在求生本能的驱动下,立即飙飞,进入了黑天,妄图甩掉身后的强敌。

可惜它的这一行动,早已经在两位大敌的料想中。

轰!

一记杀招,狠狠地击中上极天鹰。

上极天鹰发出惨烈的尖啸,差一点就栽倒下去,但很快它振奋双翅,又努力地攀升上来。

鹰血喷涌,在黑天中划出一道长长的血线。

太古荒兽原本自愈能力很强,但是天庭蛊仙的杀招,显然非同一般。

“大局已定了。”紫薇仙子双目一闪。

尽管上极天鹰还在挣扎,但战况非常明显,两位天庭蛊仙绝不可能失败。

尤其是其中一位,还用着仙道杀招,将自己吊在上极天鹰的身后。

上极天鹰飞的越快,他的速度也就越快。上极天鹰飞到哪里,他也能跟到哪里,并且毫不费力。

上极天鹰虽有智慧,但怎么可能有人族聪慧?

就算是人族蛊仙,也难以在战斗中,破解仙道杀招。

两位天庭蛊仙担忧上极天鹰临时的凶悍反扑,另一方面也是为了活捉这头上极天鹰,所以不断地给它放血,不断地削弱它的战斗力。

上极天鹰似乎也感到了自己灰暗的命运,但它仍旧没有放弃,强振双翼,冲上高空。

穿过黑天,它来到了白天当中。

天罡气墙对于它而言,根本不算什么阻碍。

但是对于天庭蛊仙而言,却不一样了。

两位天庭蛊仙要穿越天庭气墙,可不太容易。不过他们也早有准备,紫薇仙子怎可能不配给他们仙蛊屋?

“你留在这一侧,我去追击,防止上极天鹰往回飞。”两位蛊仙配合起来,也是滴水不漏,将上极天鹰任何逃脱的可能,都尽数封死。

“虽然有些意外,但也在我的料想当中。不管是人,还是鹰,亦或者红莲真传,都不会逃脱我的手掌。”天庭中,紫薇仙子双眼微微眯起,尽显智道蛊仙运筹帷幄,决胜千里的风范。(未完待续。)

\end{this_body}

