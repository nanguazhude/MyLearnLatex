\newsection{影无邪的图谋}    %第二百二十八节:影无邪的图谋

\begin{this_body}

镇运天宫。

“这里没有。”

“这里也没有!”

“古怪至极,这群中洲蛊仙怎么会突然间消失得无影无踪?”

南荒仙人再无之前的运筹帷幄,脸上的从容淡定也消失不见,额头已经渗出密密的一层细汗。

他不断操纵镇运天空,查找各处方位,但如何也不能查找到中洲一行蛊仙的位置出来。

药皇也有些懵,急忙道:“南荒大人,需要我做些什么吗?”

这时,巨阳仙僵的声音传入两人耳畔:“这是无间道。”

南荒仙人顿时更惊:“难道中洲蛊仙掌握了盗天魔尊的这记仙道杀招不成?”

但巨阳仙僵旋即否决:“并非如此,而是他们利用了盗天魔尊曾经遗留下来的无间道。”

无间道。

宇道仙级杀招。

曾经,盗天魔尊为了寻找回家的路,在整个世界游历。

毫无疑问,他是所有尊者当中,游历地方最多的尊者。盗天魔尊的足迹,几乎遍布五域两天。

盗天魔尊苦于个人太过渺小,而蛊师世界太过宽广,创造出了无间道。

这个仙道杀招,能让他大大缩短路程,加快他探索的进度。

盗天魔尊早已经失踪,下落不明,但是他遗留在五域两天里的无间道杀招,有一部分仍旧经久不衰,大有亘古长存的架势。

中洲十大古派方面,对于盗天魔尊的探究,一直没有停止过。

因为盗天魔尊和其他尊者不同,他是天外之魔。从这一点,天庭方面就绝不会轻易放过他。

灵缘斋能够掌握盗天真传之一,让赵怜云顺利继承。就是天庭方面探究盗天魔尊的成果之一。

中洲蛊仙们发现了这处无间道。并且有法子利用无间道,缩短自己的路程。

“那么这个无间道杀招,它究竟指向哪里?”南荒仙人反应过来。立即意识到了关键。

巨阳仙僵答道:“我也无法探查。这毕竟是盗天魔尊的手笔。”

巨阳仙尊在世时,就在盗天魔尊的遗迹中。吃过亏。巨阳仙尊乃是运道的开创者,但对于盗天魔尊,他似乎始终缺乏强有力的应对手段。

南荒仙人沉默。

药皇傻眼。

这该如何是好?

巨阳仙僵答道:“既然他们利用了无间道,那我们就只有守候他们再次出现了。”

药皇问道:“这么说来,中洲蛊仙们可以出现在任何的地方?若是他们直接出现在北原当中,那该如何是好?”

中洲蛊仙虽然狼狈不堪,战力也折损了许多,但阵容依旧非常强大!

三座仙蛊屋、三位八转蛊仙。不算那些六转七转蛊仙,还有一只九转爱情仙蛊!

这种庞大的阵容,若是出现在一个超级势力黄金家族的大本营中,倒霉的也绝非中洲蛊仙,而是黄金家族!

“若是这群中洲蛊仙,忽然出现在我药家附近的话……”念及于此,药皇心里顿时惴惴不安起来。

“还请仙祖指点迷津。”南荒仙人求教。

巨阳仙僵却是呵呵一笑:“儿孙自有儿孙福,福患相依并存,自可相互转换。此是北原之灾,亦可转化为北原之福。我的本体早已死去。这已经不是我的时代。尽自己的努力去做吧,不管成功还是失败。如何将灾劫转为福禄,还要看尔等自己。”

巨阳仙僵说完这话。就再无声息传来。

南荒仙人和药皇俩俩对视,都是无可奈何。

南疆,超级蛊阵。

方源手中拿捏着一份普通的蛊材。

他随手一抛,将这份蛊材抛入到眼前的火炉当中。

火炉中火焰燃烧,却散发出阵阵寒意。

火炉中的火,也很特别,不是寻常的橘红火焰,而是苍白中带一点蓝,这是冷焰。

须臾。冷焰骤然熄灭,悄无声息。令人猝不及防。

从火炉中飞出一只蛊虫来。

这只蛊虫,好似孑孓。身体细长,褐色中带着一点绿,****宽大。飘向方源的时候,身体一屈一伸,比较独特。

这是一只力道蛊虫,当然只是凡蛊。

但它的名字,却是挽澜,从这点看,它当然与众不同。

挽澜,方源早有一只,那是六转力道仙蛊。现在这只,则是力道凡蛊。

挽澜蛊能催发出一股玄妙力量,让蛊师或者蛊仙操纵河水江流等等。这只挽澜凡蛊的效用,自然远远不如六转挽澜仙蛊。

但方源却很欢喜。

为什么呢?

一者,仙蛊唯一。若是方源同时又两个仙道杀招,都需要用挽澜仙蛊,那该怎么办?如今有了凡蛊挽澜,方源就可以用大量的凡蛊,暂时取代挽澜仙蛊的作用。所以,一位蛊仙纵然有了某只仙蛊,但并不代表他对相应的凡蛊就没有需求了。

二者,方源此次尝试能够成功,是因为水道境界提升到了宗师级数。

挽澜蛊虽然是力道蛊虫,但当中却还牵扯到水道。一直以来,方源的力道虽然是宗师境界,但却从未能推演出挽澜的凡蛊方。

现在水道境界晋升成了宗师级,他几乎没有推算,就凭着感觉,将凡蛊方描绘了出来。

一切都是水到渠成的事情。

“水道宗师。”方源开始从中感受到方方面面的妙处。

不只是蛊方,还有杀招,方源前世五百年也有不少水道积累,比如水道的杀招或者水道的蛊方。

现在当他回忆起这些方面的内容的时候,他就有了一种全新的感受,更加深刻的认知。

这种感觉有些妙不可言。

就好像是之前他站在塔下,看着塔身的砖石。现在他忽然间上升到了塔顶,所见的风景依旧今非昔比。

“虽然境界的拔升暴涨,没有客观的物质来衡量。但是我的实力却增长很多。”

“尤其是我这种,其他流派境界已经有五个,都是宗师境界。这些宗师境界的流派相互结合起来。相互影响和提升,就会形成更加巨大的收益。”

“难怪五百年前世。那么多人对梦境趋之若鹜。”

方源回忆起来,那个时候的他,还在将主要精力放在血道的修行上。

当然,这也是有原因的。

一来,方源那个时候的处境并不算好,外在一直强敌,方源需要战斗力量。

二来,方源对梦道一无所知。更加没有类似现在解梦杀招这样的探梦手段,冒然去梦境探险,反而会得不偿失。

三来,方源现在回忆,这种恐怕还有天意作祟的成分。

正当方源想要好好感受水道宗师境界,带给各个方面的提升时,武家那边终于来了一位七转蛊仙。

这位蛊仙带来武庸的调令,当然,还有方源向武家商借的仙蛊。

方源很快就完成交接,离开了超级蛊阵。

他一路向东北方向飞行。马不停蹄。

这一次,他对武庸所说的借口,便是有不得已的事情。需要独自回去东海一趟。

任何人都有自己的秘密,若是寻常武家蛊仙,武庸定然要深究。但对于武遗海而言,武庸就算是想问,恐怕也问不出什么。所以武庸很明智,没有追问。

方源进入界壁,来到东海,他寻得一处无名海域,落下仙窍。暂时休整。

期间,他在至尊仙窍中不断演练杀招。熟悉新得的仙蛊。

越到关键时刻,方源反而越加沉稳。他决定这一次定要斩杀了影无邪。免除心腹大患,所以尽量多做准备。

当方源觉得时机成熟,他便立即收起了仙窍,准备动身,实施他酝酿已久的追杀大计。

气运交感!

方源催动这记仙道杀招,出乎他意料的是,他发现影无邪等人居然都离开了南疆,又回到了东海。

“好,在东海里,更适合我动手。”方源没有再犹豫,一飞冲天,兵锋直指目标。

影无邪等人在海面上疾飞。

白凝冰和他们并肩而行。

他们的关系已经发生了改变,成为了地位平等的盟友。

原来,在数天前,影无邪拿出阳蛊,让白凝冰暂息战意。

影无邪借此良机,与白凝冰商讨合作。

白凝冰却也今非昔比,表示宁愿不要阳蛊,也要置影无邪等人于死地。

影无邪巧舌如簧,针砭时弊,终于说服白凝冰与己方达成新的联盟。

白凝冰得到了阳蛊,也知道了方源拥有上极天鹰,是一个巨大的威胁。

这个时候,留在白相洞天中明显不智。

因为上极天鹰存在,方源可以借此,轻易地进入洞天。若是白凝冰等人留在白相洞天当中,反而成为方源的线索,让方源轻易找到白相洞天。

此时,白凝冰望着蔚蓝的海面,皱起眉头,秘密传音:“影无邪,你说你有对付方源的法子,现在咱们跑到了东海,仍旧一直都在赶路。究竟有什么方法,你直说吧。”

影无邪笑了笑:“要对付八转,就需要同样对等的八转存在。东海不是我们的目的地,北原才是。”

“这么说来,影宗在北原的残余势力中,竟有八转的存在?哼,你骗鬼呢。”白凝冰不屑地传音回道。

若是有八转战力,怎么在义天山时,没有动用?

影无邪苦笑:“那是因为时机不对。”

“这一次时机就对了?”

“这次时机很好,北原和中洲两方,围绕鸿运齐天仙蛊,正要展开厮杀。我们这一次去北原,将直奔漩涡的中心大雪山福地!夺取鸿运齐天,借助八转战力,反杀了方源,夺回至尊仙体!”影无邪说到这里,忽然一笑,反看身后。

“嘿,他要来了。”

ps:今天就一更,本来是打算这周的哪天加更的,但是这周写作状态并不好。所以,挪到25号加更,回馈大家。另外创建了一个新的微・信公众号,大家搜“作者蛊真人”,就可以关注了。今后的重心会逐渐移到“作者蛊真人”这个新号上面,会发布一些关于我本人生活方面的一些内容,让大家更多的认识我。还有本书的一些番・外吧,比如红莲魔尊的一些故事等等。关于q・q群,会在近期进行一次整合。我们的q・q书友群实在是太多了。我根本看不过来,我需要整合统一。群的名称要改变,有些群要合并。这一切,都是为了能更好地和读者朋友们沟通,收集大家的意见,今后还会做一些互动,更方便地回馈一直支持我的诸位兄弟姐妹们!(未完待续。)

\end{this_body}

