\newsection{煮运锅}    %第八百零七节:煮运锅

\begin{this_body}

%1
太古白天。

%2
方源驾驭着随意祥云,在空中疾驰。

%3
天相白鹤就跟随着他,不断盘旋飞舞。

%4
随意祥云杀招,是方源当初从黑凡洞天攻略中缴获而得的战利品。后来经过了他的一些改良,用于移动。

%5
但祥云的方向,并不随着方源的意志而动,而是遵循着运道的奥妙。

%6
祥云所行的路径,必定是运气颇佳的路线。

%7
随意祥云和天相杀招相互搭配,相得益彰,方源在上一世就从中获益,接连发现了华文洞天、兽灾洞天。

%8
前不久,方源已经魂穿两天,提前在华文洞天、兽灾洞天落子。

%9
在此之后,方源一边静候棋子发展,一边继续潜修,积累实力。

%10
他的气道杀招,还在不断地熟练当中,已经有了和八转蛊仙实战的能力。不过,作为底牌的气海无量杀招,方源还是有些陌生。

%11
除了锻炼杀招,方源本体还在太古白天中逡巡,借助随意祥云和天相杀招,企图寻找到更多的洞天。

%12
目前为止,方源还一无所获。

%13
“天相杀招侦查的范围虽广,但是相较于太古白天,天相杀招的范围还是显得渺小了。”

%14
“不过这一路行来,我确实是收获颇丰,八转仙材斩获许多。”

%15
“这的确是最佳路径,一路上都是好运连连,伴随着诸多惊喜!”

%16
“嗯?宙道分身又有推算的成果了?”

%17
方源眼中精芒骤闪。

%18
他本体苦修,暂无成果,宙道分身则经过了这段时间的积累,得到了又一个可喜的成就。

%19
“又一座仙蛊屋推算完成了。好得很!”

%20
方源心中大喜,当即就分出大部分的心神,投入到自家仙窍中,开始了组装。

%21
太古白天中,其实十分危险,太古荒兽横行,时常还会遇到太古荒兽群。寻常八转蛊仙在这里探索,都是小心谨慎。一些七转中的罕见强者,有时候也会来太古白天探索,搜刮修行资源,无不是战战兢兢。

%22
不过,方源有着随意祥云,一路上好运连连。更关键的是有天相护航,任何危险都能提前发现,让方源从容应对,轻松躲避。

%23
铁壁蛊。

%24
这是一只六转的金道蛊虫,它有成人拳头大小,十分沉重,形如独角仙,但不管是头部还是背部,都是方方正正,线条笔直,几乎没有弧线弯角。

%25
显然,铁壁蛊是一只用于防御的蛊虫,来自琅琊派库藏,现在被方源用来组建第二座仙蛊屋。

%26
铁壁蛊第一个催动起来,随后就是大批的凡蛊。

%27
这些凡蛊种类繁多,多达三千多只,其中以金道、炎道的凡蛊最多,几乎都是五转蛊。

%28
铁壁蛊镇守中央一点,其余的凡蛊围绕着它盘旋飞舞。

%29
一股黑光仿佛墨水般,从凡蛊之间荡漾而出,迅速蔓延,将所有的蛊虫都包裹进去,形成一个巨大的黑色光球。

%30
黑色光球悬挂在半空中,表面凹凸不定,内里的蛊虫飞旋速度越来越快。

%31
达到一定的速度之后,黑色光球的表面平滑无比,再无凹凸,变得十分的圆润。

%32
这个时候,方源便开始催动炎道仙蛊。

%33
他陆续催动三只炎道仙蛊,还有大量凡蛊,形成一股火焰。

%34
火焰如猩红的杂草,缠绕在黑色光球表面,不断地灼烧。

%35
一股诱人的香气,旋即扩散开来,让凡人闻了能瞬间沉溺其中,生出无限渴望,无限遐想,不能自拔。

%36
如此灼烧了好几个时辰,黑色光球被烧得金碧辉煌,成了金色的光球,之前的黑色仿佛是锅底的灰,被拂拭得一干二净。

%37
方源一面维持金色光球,一面停止催动火焰,然后就调动水道仙蛊、智道仙蛊,凝聚出一股清泉。

%38
清泉浇灌在金色光球上,发出嗤嗤的声音,好像是灼烧后的铁球被浇上一层冷水。

%39
但是,没有一丝气雾产生,清泉全都融入到了金色光球之中。

%40
金色光球旋转的速度缓慢下来,里面的蛊虫仿佛是一颗颗的金色微型流星,不断旋转飞舞。

%41
它们的轨迹玄妙繁杂,让人看得眼花缭乱,偏偏从未有一起相互碰撞的事故发生。

%42
这个时候,方源掏出了第二只辅助仙蛊。

%43
食道仙蛊——煮。

%44
它外形是一只田鳖,体积比较大,有脸盆大小。三角头部,短触角,触角下是两只突出来的复眼,闪烁着金色的光。

%45
它的身体扁阔,总体呈现椭圆形状,暗红色,用手抚摸,仿佛砂土质地,十分温热。

%46
它有三对触脚,最前面的一对最为强壮,张开来时,仿佛是一个厚实的铁钳。

%47
仙蛊煮,却是方源敲诈了南疆正道而得。

%48
时至如今,方源基本上已经将南疆群仙的肉身、魂魄都归还过去,开始和南疆各大家族互换仙蛊。

%49
方源费了半个时辰,这才将仙蛊煮安置到了金色光球内。

%50
金色光球的体积因此膨胀了一倍还多。

%51
方源本体已是出了一身的汗,这是第一处难点,他总算是通过了。

%52
搭建仙蛊屋可不容易,比催动通常的杀招,要困难许多倍。若是搭建失败,蛊仙反噬也就算了,参与搭建的蛊虫还会受损,乃至毁灭。

%53
所以,方源都是小心翼翼,谨慎无比。

%54
就这样继续下去,方源将狗屎运、气运、察运、连运、时运等诸多运道仙蛊,都融入其中。

%55
他的运道境界是大师级,阵道境界是宗师级,都已经产生了直觉。

%56
所以,搭建的过程中,一旦有什么不妙的感觉,方源就立即警惕起来,及时作出防范。

%57
最关键的是他搭建仙蛊屋的方法,是宙道分身通过智慧光晕推算而来,十分优良。

%58
至尊仙窍的时间过了大半个月,方源大功告成。

%59
金色光球猛地一炸,一座崭新的仙蛊屋亮相。

%60
和之前的万年斗飞车不同,这座仙蛊屋体积颇小,和寻常水盆差不多大。

%61
它浑身金黄之色,笼罩一层洁白光晕,贵不可言。

%62
它就是一口锅,锅边厚实,有大拇指头的厚度,锅口大张,锅里空无一物。

%63
在锅外面,有着八条雕龙,其中六条栩栩如生,剩下两条模糊粗糙。

%64
这八条雕龙,龙尾齐聚锅底,相互缠绕,又延伸下去,形成八爪的支架。

%65
八条龙尾,又分别对准东、南、西、北、东北、东南、西北、西南八个方位。

%66
每一条雕龙的龙爪都扣在锅外表面,龙身蜿蜒向上,龙头搭在锅边,朝着锅内。龙眼紧闭,仿佛在沉睡。

%67
六转仙蛊屋——煮运锅!

%68
虽然其中应用到的有七转仙蛊,比如食道仙蛊煮。但核心的运道仙蛊,皆是六转,所以这座仙蛊屋的品级有点低。

%69
这座仙蛊屋记载于巨阳己运真传之中,又经过方源宙道分身的改良。很显然,这是一座运道的仙蛊屋。

%70
它不能承载蛊仙作战,体积很小,另有妙用。

%71
比较起来的话,和星宿棋盘的确有些相似。

%72
方源当即灌输仙元进去,煮运锅嗡嗡作响,立即化为一道金光,扑到方源头顶上空去,随即隐没不见。

%73
即便是方源,单凭肉眼也看不到这座仙蛊屋,只能凭借暗里的感应确认它的存在。

%74
当然要想“看见”,也很简单。方源心念一动,仙元继续灌输,调动煮运锅的一项杀招。

%75
察运!

%76
方源仰视的视野中顿时有了变化,他看到了自己的气运,不断地流淌到煮运锅中。

%77
不一会儿功夫,煮运锅里的气运就已经蓄满。

%78
“察运仙蛊只有六转,而我已有八转修为,单靠这只仙蛊无法查看我的运气。”

%79
“煮运锅也只是六转层次,所以我只能看到锅里的气运,不能看到我的气运全貌。”

%80
虽是如此,锅外却还有一些气运,方源可以看到。

%81
一股河流般的气运,水面熠熠生辉,好像是微型的光阴长河,和方源最为紧密相连。河水一头联通锅内的气运,另一头向外延伸,没入虚空。

%82
这是方源的宙道分身的气运。

%83
宙道分身的运气相当稳定,但有一丝黑气逐渐生成,不断产生。这是春秋蝉的弊端,容易坏运。方源每隔一段时间,就要对此施展运道手段。

%84
还有一团运气,仿佛是一层粉色雾气,漂浮在锅边,绕着锅旋转。

%85
这是方源的梦道分身的气运象征。

%86
粉色雾气缥缈虚弱,不断变幻,仿佛一层微弱的光影。

%87
只有梦道分身晋升六转之后,才能和宙道分身的气运景象媲美。

%88
除此之外,还有龙人分身的气运,仿佛紫色小龙,在锅边游荡。方源在兽灾洞天中的分身,像是一头雀鸟,灰不溜秋。还有方源在华文洞天中的分身,气运如同花苞,萎靡不振。

%89
“这两个小分身的气运,都有些弱啊。看来之前,灰色魂球上的运气损耗很大。而我本体和分身之间,到底是隔着一层世界,有着阻碍。”

%90
“既然如此……”方源心念一点,催起煮运锅。

%91
煮运锅中温度迅速拔升,气运迅速沸腾。

%92
锅边的两条雕龙缓缓张开双眼,同时张口,对准锅内吸气。

%93
顿时,两股气运就被雕龙吸入肚中。

%94
雕龙活动开来,掉转龙首,分别对准锅外的灰雀气运、花苞气运,张口喷吐。

%95
两股气运喷射出去,汇入到灰雀和花苞气运之中。

%96
转瞬间,灰雀轻啼一声,变成一只尖嘴的鹰隼,振翅疾飞。而花苞气运则像是久旱逢甘霖,昂起头颅,花苞绽放,绚烂锦绣!

\end{this_body}

