\newsection{追寻——自由!}    %第九百三十九节:追寻——自由!

\begin{this_body}

%1
方源浑身浴血,伤痕累累,他冷哼一声,猛地站起,抬眼便望见伫立在自己面前的绣楼。

%2
绣楼本是凡物,乃是元始仙尊留给徒弟星宿的陪嫁的东西。只是按照当时的风俗,其实象征意义更多一点。但后来星宿成尊,便着手将绣楼转变成了仙蛊屋,便有了极其玄妙的威能。

%3
一百多万年前,狂蛮魔尊入侵天庭。他一路直闯,先后经过仙帝苑、蕴空阁、须弥泽、恒沙洞、百万天王画廊、绣楼、中央大殿,最终止步于监天塔。

%4
狂蛮魔尊在绣楼前吃了亏,绣楼施展出压箱底手段——道绣,七根绣花针,如灵雀飞舞,翩若惊鸿。狂蛮魔尊不得不舍弃三层血皮,这才闯关过去。

%5
自此之后,这三层血皮就被无数道痕充当丝线,缝在了半空中,一动不动。

%6
绣楼自那以后,损伤无法复原。而后又在义天山启用对付魔尊幽魂,新的战损令它雪上加霜。前因后果导致绣楼残损破败,此刻难堪大用。

%7
而在绣楼的上空,不仅是方源,包括龙公在内的众多蛊仙,都看到了三张血皮的异动!

%8
它们就像三面血红的旗帜,在狂风中挥舞。

%9
左面的旗帜上描绘着一只鸟儿,腹下六足,却没有翅膀。

%10
中间的血旗上有一头野兽的纹路越发清晰,它张开大口,却没有利齿。

%11
右边的血皮上则有一只鱼,越发栩栩如生,但很明显,它没有鱼鳃。

%12
血皮猎猎作响,像是上古的狂风穿越了一百多万年,呼啸在世人的耳畔。又仿佛是万军激战,金戈铁马的碰撞和呐喊,浩然激荡!

%13
血旗飘拂的幅度越来越大,它们如同忍耐万年终要喷发的火山,又好像是蓄势已久,即将猎食出击的猛兽!

%14
嗷吼——!

%15
猛兽咆哮。

%16
它们狂暴地挣断充当束缚它们的道痕丝线,然后化作三团血光,落到方源的周围。

%17
三声或古怪或高亢的兽鸣声中,血光尽数消退,露出三头庞然巨兽。

%18
一头黄色怪鸟,身躯如小山,六足粗壮,鸟喙又硬又长,闪闪发光,并无双翼。

%19
一头湛蓝的豹子,大腹便便,趴在地上,哈欠连天,张开的嘴巴中一颗利齿都没有。

%20
还有一头鱼,鳞甲碧绿,悬浮半空,鱼头高耸,鱼嘴紧闭,鱼眼两侧没有丝毫鱼鳃的痕迹。大鱼一动不动,仿佛玉石雕塑。

%21
群仙震动,就连龙公也神色微凝,暂缓进攻的脚步。

%22
“狂蛮魔尊留下的三张血皮发生了异变!”

%23
“这三头怪物气势竟如此惊人。”

%24
“是方源引动出来的尊者手段吗?”

%25
“等等,这三头怪兽,怎么好像是《人祖传》中记载的那三头呢?”

%26
《人祖传》第四章记载着这样的一个故事——

%27
人祖在苍茫的大地上孤独游荡,披头散发,失魂落魄,时而哭嚎,时而呆滞地坐着,时而痴痴的傻笑。

%28
宿命蛊的玩弄,让他和儿女分离,失去财富蛊,将人祖逼疯。

%29
“我是谁?我在哪里?我要做什么?”人祖茫然又疯狂。

%30
一天上午,一群鸟从人祖身边飞奔而过。

%31
这群鸟都没有翅膀,六只脚在地上轮流奔跑,卷起漫天的烟尘。

%32
人祖看到这些鸟儿,欢喜地跳起来。

%33
“原来我是鸟啊!”他也撒开腿狂奔,汇入鸟群之中。

%34
鸟儿纷纷对人祖发出怪吼:“你是人,你用两条腿走路,你不是鸟。你走开,不要干扰我们,我们正在追逐自由蛊,我们要把我们的自由找回来。”

%35
人祖便问:“你们为什么要寻找自由蛊呢?”

%36
鸟儿们语气沉重:“我们曾经拥有过自由蛊,但我们没有意识到。当我们失去了它,才发现我们已经没有了双翼,再不能飞翔。当我们重新获得自由,我们才可以展翅高飞了。”

%37
人祖大悟:“我明白了,人也得有自由。人如果没有了自由,就好像是鸟儿失去了翅膀。”

%38
“没错!我记起来了!”人祖一拍巴掌,大笑起来,“我也要寻求自由,摆脱宿命的束缚,以后想去哪里就去哪里,想和谁在一起就永不分离。”

%39
鸟儿纷纷诧异地盯着人祖看:“人啊,你怎么能有这样的非分之想呢?”

%40
“你看看我们,鸟儿没有翅膀怎么可以呢?所以我们追寻自由是一种本分。”

%41
“而你们人的一生注定是要孤独的,所有的欢聚的结果都会是分离。人啊,你要追寻自由,也要恪守你的本分,可不要胡思乱想。”

%42
人祖摸摸头,神情疑惑:“是这样的吗?”

%43
鸟群最后留下一句话:“人啊,让我们给你一个忠告吧。将来你若是得到了自由,千万要懂得珍惜,不要像我们一样轻易松手。千万不要放自由蛊飞走,不然你会后悔的。”

%44
人祖和鸟群分别,渐渐的又忘记了自己的身份和追求。

%45
一天中午,一群奔走的蓝豹路过他的身边。

%46
疯了的人祖见到这群蓝豹,非常开心,大叫起来:“原来我是豹子啊。”

%47
人祖冲进豹群当中。

%48
但豹子们都将他排挤出去,纷纷大叫:“你是人,你可不是豹子。你用两条腿走路,而我们是四条腿。你离开,不要影响我们,我们正在追逐自由蛊,我们要把我们的自由找回来。”

%49
人祖听了便问:“你们为什么要寻找自由蛊呢?”

%50
蓝豹们神色忧郁:“唉,我们曾经拥有过自由蛊,但我们没有意识到。当我们失去了它,我们才发现自己没有了利齿,再不能撕扯嚼碎猎物了。当我们重新获得自由,我们才可以愉快地进食。”

%51
人祖大悟:“我明白了,人也得有自由。人如果没有了自由,就好像是野兽没有了牙齿。”

%52
“没错!”人祖一拍巴掌,大笑起来,“我要得到自由,摆脱宿命的束缚,拥有数不尽的美酒佳肴,花不尽的财富,还有各种各样暖和又漂亮的衣服。”

%53
蓝豹们愣了愣,哈哈大笑,嘲讽人祖道:“人啊,你怎么能有这样的非分之想呢?”

%54
“你瞧瞧我们,猛兽没有爪牙像话吗?所以我们追寻自由是一种本分。”

%55
“而你们人的一生注定是要两手空空而来,两手空空而去。人啊,你要追寻自由,也要恪守你的本分,可不该胡思乱想。”

%56
人祖挠挠头,神情怏怏:“是这样的吗?”

%57
豹群最后留下一句话:“人啊,让我们给你一个忠告吧。将来你若是得到了自由,千万要懂得珍惜,不要像我们一样轻易松手。千万不要放自由蛊飞走,不然你会后悔的。”

%58
人祖和豹群分别,渐渐的又忘记了自己的身份和追求。

%59
一天晚上,一群鱼游过他的身边。

%60
人祖见到鱼群,非常开心,大叫起来:“原来我是鱼啊。”

%61
人祖冲进鱼群当中,想和它们一起畅游。

%62
鱼群们一阵骚乱,将人祖排斥出去,纷纷叫嚷:“你是人啊,你可不是鱼。你有两条腿,而我们都没有腿。你快走,不要麻烦我们,我们正在追逐自由蛊,我们要把我们的自由找回来!”

%63
人祖听了便问:“你们为什么要寻找自由蛊呢?”

%64
鱼群们唉声叹气:“我们曾经拥有过自由蛊,但我们没有意识到。当我们失去了它,我们才发现自己没有了鱼鳃,再不能在水里呼吸。当我们重新获得自由,我们才可以在水里随意畅游。”

%65
人祖大悟:“我明白了,人也得有自由。人如果没有了自由,就好像是鱼没有了腮,不能呼吸。”

%66
“没错!”人祖一拍巴掌,大笑起来,“我要得到自由,摆脱宿命的束缚,我要自由自在的呼吸,永远存在下去,我要永生!”

%67
鱼群纷纷冷笑:“人啊,你怎么能有这样的非分之想呢?”

%68
“你瞧瞧我们,鱼鳃是鱼必须要有的,所以我们追寻自由是一种本分。”

%69
“而你们人的一生注定和永生无缘,将会生老病死。人啊,你要追寻自由,也要恪守你的本分,可不能胡思乱想。”

%70
人祖皱皱眉,神情厌烦:“是这样的吗?”

%71
鱼群最后留下一句话:“人啊,让我们给你一个忠告吧。将来你若是得到了自由,千万要懂得珍惜,不要像我们一样轻易松手。千万不要放自由蛊飞走,不然你会后悔的。”

%72
人祖和鱼群分别,渐渐的忘记了鸟群、豹群、鱼群关照他的话。

%73
“我是人,我要追求自由!”

%74
“我要摆脱宿命的束缚,和爱的人永不分离,生活富足享乐,还要永远活着。”

%75
许多路过的生灵听到人祖的话,纷纷摇头,主动远离人祖。

%76
“快走,他就是人祖,又在说胡话了。”

%77
“他已经彻底疯了。”

%78
“他怎么敢这样想?”

%79
一天,自由蛊从路的那头主动飞向人祖。

%80
人祖大喜,一把抓住了它。

%81
“自由啊,我终于得到自由了。”人族非常开心,又感到疑惑,就问自由蛊:“真是奇怪,失翼的鸟群追逐你,无牙的猛兽追寻你,缺腮的鱼群追求你,你却主动向我飞来,这是怎么一回事呢?”

%82
自由蛊便道:“我当然不是向你飞来,人啊,你曾经企图用态度欺骗我,用爱情束缚我,用财富收买我。我厌恶你,并且恨你!我之所以主动飞来,纯粹是被你身上的思想蛊吸引而已。”

%83
思想蛊从人祖的身上浮现出来,笑着解释道:“那是因为人祖你疯了,你整天胡思乱想,一个人妄图永不分离,奢望衣食无忧,渴求永生不老。这不是疯子是什么?”

%84
自由蛊叹息一声:“思想的自由便是最大的自由。就是这些胡思乱想,能让我壮大自己。人祖啊,虽然我被你抓住,但我绝不会为你效力。你快给我松手!”

%85
人祖摇头,捏的更紧了:“自由蛊,我是不会放手的。”

%86
自由蛊冷笑:“那你可准备好了,别被压趴下。”

%87
话音刚落,责任蛊就飞了过来,压在人祖的肩头。

%88
“好重,好重啊!”人祖被压得几乎直不起腰来。

%89
思想蛊感叹道:“自由和责任相随,人祖啊,你要得到自由,就得担负责任。至少,你得为你自己负责。”

%90
人祖咬牙坚持,汗如雨下,很快就跪在了地上。

%91
他又看到了蛛丝。

%92
宿命蛊的蛛丝缠绕他的全身,人祖担负责任的重担就已经很勉强了,根本没有力气去挣脱蛛丝的束缚。

%93
宿命蛊的蛛丝越收越紧,把人祖全身上下都勒出血痕来。

%94
人祖大叫:“这是怎么回事?”

%95
思想蛊解释道:“人啊,你越是自由,就越会感受到你是受限制的。”

%96
自由蛊笑道:“快松手吧,你抓着我时间越久,蛛丝就缠得越多,绕得越紧,甚至直接将你勒死!”

%97
人祖摇头:“不,我绝不会放手,自由蛊啊,我要得到你!”

%98
无数的蛛丝嵌进人祖的皮肉中,人祖痛嚎嘶吼,在地上打滚,但就是不松手。

%99
“哈哈哈!”人祖又开始了傻笑,“我感觉到在遥远的地方有一只蛊。单单这份感觉,就让我感到快乐和满足。”

%100
思想蛊坦诚道:“那是理所当然的。谁能得到自由蛊,谁就可以感受到幸福蛊的位置。”

%101
人祖咬牙坚持,时而痛得大哭,时而乐得大笑。蛛丝紧紧收束,缠在他的骨头上,把骨头都勒出裂纹,但人祖一直都死死的不松手。

%102
最终,他痛得昏死过去。

%103
也不知道过了多久,他悠悠醒来。

%104
宿命蛊的蛛丝不再收紧了,责任蛊也不再施加更重的压力,思想蛊消失不见。

%105
“等等,自由蛊呢?”人祖感受不到自由蛊的存在,他慌了,连忙打开双手。

%106
刚露出一条缝隙,自由蛊就忽的飞了出去,离开了人祖。

%107
人们拥有自由的时候,往往意识不到有它。等到失去了,才会猛然发现。

%108
人祖看到自由蛊飞走,惊呆了,又想起鸟群、豹群、鱼群最后关照他的话,懊悔非常。

%109
他痛苦地撕扯自己的头发,满地打滚。

%110
“我还不如去死啊。”人祖痛不欲生,“我宁愿失去爱情,失去生命,也不想失去自由啊!”

\end{this_body}

