\newsection{苦逼少年}    %第四百七十二节:苦逼少年

\begin{this_body}

%1
影无邪、黑楼兰、白兔、妙音仙子都从方源的至尊仙窍中出来。

%2
“接下来,我便要亲自探索梦境了。”方源微微侧身,对唐方明道。

%3
唐方明心中一震,旋即了解方源的话外之意,苦笑一声:“请诸位仙子随我来。”

%4
这座仙道蛊阵,遮护盗天梦境,本是唐方明操控。但接下来,唐方明却是主动割舍了一部分权力,交予影宗蛊仙,让她们和自己一同掌控这座仙阵。

%5
这本是方源和唐家盟约的内容之一。

%6
方源探索梦境的时候,魂魄本体遁出,进入梦境之中,肉身缺乏防护。

%7
这个时候,自然是比较危险的。

%8
所以,方源带出了影宗一众蛊仙,届时她们会为方源掩护。

%9
这座仙道蛊阵,自然不能放任,完全交给外人掌控,所以影宗蛊仙插手,分割了这座仙阵的控制权力。

%10
如此一来,不管是影宗成员,还是唐家方面,就能相互制衡,任何一方想要对方源肉身图谋不轨,都很难办。

%11
方源做事,几乎是滴水不漏。

%12
等到仙阵大半落入影宗掌控之后,他这才施施然遁出本体魂魄,直接钻入盗天梦境当中。

%13
唐方明双眼一眨不眨,聚精会神地瞧着。

%14
“这可是真正关键的手段了!”他甚至呼吸都有些急促起来,脑海中全是想要一探究竟的念头。

%15
可惜光是这样看,怎么可能看出方源探索梦道的关键技艺呢?

%16
“这是……哪里?”方源进入梦中,视野骤变。

%17
他发现自己化身成了一个小小少年,居住在一个破旧的帐篷之中。

%18
帐篷里摆设非常简陋,只有一个破烂不堪的毛毯。

%19
风一吹,帐篷呼呼作响,一股凉意直接倒灌进来,方源视线不由自主地转移,看到了帐篷边角的一个漏洞。

%20
就连帐篷都是破的。

%21
“可恶的家伙,揍了我一顿不说,还将我的帐篷划破!”方源化身的少年,咬牙切齿,暗含愤怒地自语道。

%22
然后,这个少年就低下头,查看自己身上的伤势。

%23
方源因此视线转移,看到“自己”身上破衣烂衫,非常穷困潦倒,胸膛胳膊上更是青一块紫一块。

%24
少年伸手抚摸几处伤口,便有一股股的阵痛感,传到方源的心头。

%25
“想我堂堂本杰孙,莫名其妙来到这个世界,重新成长,居然落到个被一帮少年欺凌的地步。也真是够了!”

%26
“这个世界也是够了,人们控制蛊虫来获得各种诡异的力量,简直像是一场噩梦。”

%27
“唉!若真的只是一场梦,为什么过了十几年,我都未有醒来?”

%28
方源听着少年喃喃自语,顿时心头一震。

%29
他是知道不少秘辛的,盗天魔尊乃是天外之魔,和他一样,从另外的世界穿越过来。

%30
“盗天魔尊的本名,就是本杰孙。这么说来,我进入梦中,暂时化身成盗天魔尊本人了!只是现在他还只是凡人少年,并未开启蛊修的旅程呢。”方源心中顿时了然。

%31
这时,少年盗天又开口自语道:“不过好在今夜就是族群中开启圣地的日子。只要进入了圣地当中,我就能开启空窍,之后就能修行,掌握蛊师的力量。”

%32
“唉!但愿这种力量,能够让我脱离这里,再次回到我的家园去!”

%33
少年盗天说到这里,便忍着痛,龇牙咧嘴地站起身来,步履蹒跚地掀开帐篷门帘,来到外界。

%34
方源仿佛是寄存在少年盗天身上的一股意志,他只有旁观的权利,无法操纵少年盗天。

%35
这种情况对于方源而言,也是嫌少见到。

%36
再尝试了全部手段,仍旧没有办法后,方源就只能继续旁观下去,任由梦境不断发展了。

%37
少年盗天离开自家帐篷,来到外面。

%38
顿时,一副月下绿洲图,映入方源的视野当中。

%39
夜幕下,圆月高悬,洁白的月霜笼罩整片绿洲。

%40
这片绿洲很小,中间是一个池塘,池塘周围驻扎着许多的帐篷。

%41
这些帐篷有大有小,颜色各异,大多数都是灰色、白色,少部分则是黄色、金色、紫色。

%42
这些颜色的帐篷,一般都很大,显示着帐篷主人高人一等的社会地位。

%43
少年盗天羡慕地扫视一圈,又扭过头来,看了一眼自家的帐篷。

%44
他的帐篷,又小又丑,黑色的污渍布满帐篷表面,还有破洞,呼呼地往里灌着夜间的凉风。

%45
少年盗天皱起眉头,眼中傲意一闪即逝,冷哼道:“今夜过去,我就再不必住这种破陋的住处了!”

%46
说完,他抬脚便走,向着帐篷的中央走去。

%47
一路上,不时有少年从各自的帐篷中钻出来。

%48
他们神色肃穆,不敢开口说一句话。

%49
今夜是他们生命中最重要,最庄严,最神圣的时刻,按照西漠的习俗传统,今夜的少年们都要怀着敬畏之心,惜字如金。

%50
谁要是说话太多,或者雀跃,或者哭嚎,都是失去风姿仪态,会遭受族中的严厉惩处,严重的情况下,甚至会遭受流放驱逐。

%51
一旦流放驱逐,只是凡人的少年,基本上都是死路一条。

%52
惩罚之重,可见一斑。

%53
越来越多的少年,都向绿洲的中央走去,渐渐汇集成一股人流。

%54
路途中,少年盗天自然碰到了痛揍他的那帮同龄人。

%55
方源看去,只见他们个个人高马大,体格超出周围的少年一圈不说,身上的衣饰都是显贵一筹,表明各自的背景均是不俗。

%56
这些人也看到了少年盗天,虽然不敢说话,但是恶狠狠的挑衅目光,却是展露无疑。

%57
少年盗天冷哼一声,毫无畏惧。

%58
双方一边走,一边怒目互视,来到绿洲中心的池塘。

%59
池塘旁生长着茂密的芦苇,白色的芦苇开着花,在夜风中悠然摇曳,月光如清水挥洒,一只只野生的希望蛊在芦苇群中,影影倬倬,不断闪现,数量相当的多。

%60
这不仅让方源回想起,他曾经在青茅山上,经历过的开窍场面。

%61
虽然西漠的习俗和南疆不同,但本质上都是利用希望蛊开窍,可以说是大同小异。

%62
一位位少年走进芦苇丛中,惊飞起一只只的野生希望蛊。

%63
他们相继开出了空窍,各有忧愁悲喜。对于大多数的凡人而言,空窍的资质关系着他们一生的成就。

%64
不过和南疆热闹的氛围不同,西漠这里至始至终,氛围都相当凝重。

%65
就算是有人大悲大喜,都要死死克制住,让面容扭曲,让眼泪无声流淌,硬是不出声。

%66
很快,就轮到了少年盗天。

%67
他早已迫不及待,一经允许立即奔进了芦苇从中。

%68
然而,他测试出来的资质,却是最差的丁等。丁等资质的蛊师,空窍中只能存储两到三成的真元,大多数蛊师只能是一转修为,很少能够到达二转的。

%69
这种资质的蛊师,简直是毫无潜力和未来可言。

%70
拥有丁等资质的蛊师,基本上可断定,一生都将是蛊师界的垫底人物了。

%71
“怎么会!我怎么可能会是丁等资质?”少年盗天当场大叫。

%72
“闭嘴!”监控局面的族中蛊师立即出手,将少年盗天拘拿下来,封住他的嘴。

%73
少年盗天极力挣扎,族中蛊师冷哼一声,竖起手掌狠狠一劈。

%74
少年盗天立即昏迷。

%75
方源的视野也旋即陷入到一片黑暗之中。

%76
他发现自己什么都做不了,就算是解梦杀招,都没有什么效果。

%77
“越是庞大的梦境,就拥有越加强大的约束力。不过,解梦杀招没有效果,应当是我身处的这片梦境比较特殊。”

%78
方源心中估料。

%79
他尝试了一圈,没有任何成效,无奈之下,只好耐心等候。

%80
ps:还有第二更,在10点。

\end{this_body}

