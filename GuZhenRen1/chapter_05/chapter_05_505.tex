\newsection{万生春雷}    %第五百零六节:万生春雷

\begin{this_body}



%1
得益于方源的出色发挥,豆神宫外围的最后两头太古魂兽,都被方源夺走,收入麾下。

%2
至尊仙窍当中,失去了两头太古魂兽,转眼间又新添了两头。一头是牛身豹尾,一头是独眼双手的巨人。

%3
“这算不尽不愧是智道蛊仙,我们房家世代筹谋,如今豆神宫还未夺取,这个外人反而占尽了便宜。不仅得到我族的一只七转仙蛊,而且还有收走了两头太古魂兽,拥有了八转战力!”

%4
房家蛊仙们神色各异,方源的收获太过丰盛,即便是太上三长老房化生心中,都有疙瘩。其他人也有妒火暗生。

%5
不过,这也是人之常情。

%6
房家出力最多,但是到目前为止,反而是方源这个恰逢其会的外人,收获最大,房家还一直在付出。也难怪房家的蛊仙心中不平衡了。

%7
“这算不尽也是了得,居然有收服太古魂兽的手段!不过,他能纵横青鬼沙漠,收服众多魂兽,组成魂兽大军,这点也不奇怪。只是……他要走大盗仙蛊做什么?难道说,他还身怀偷道的传承不成?”

%8
“这大盗仙蛊乃是偷道的核心仙蛊之一,乃是盗天魔尊亲自开创。我房家先祖也是在一场意外中,从一位投靠而来的散仙手中得到。哼,我房家的东西岂有这么容易拿到手的?不压榨出这算不尽的劳力,岂能罢休?”

%9
房家太上二长老房睇长心中不断思量,他正要开口,却见到方源在收服了这两头太古魂兽之后,就直接闭目,随地盘坐,一副受伤很深,急需疗养的架势。

%10
“嗯?!算不尽仙友?”房睇长开口呼唤。

%11
方源不语。

%12
“算不尽仙友伤势如何?”房睇长双眼闪烁寒芒。

%13
方源沉默。

%14
“看来算不尽仙友伤势很重,不若我来出手为仙友疗伤。”房睇长继续开口。

%15
方源这才缓缓睁开眼帘,用沙哑的声音,疲惫至极道:“仙友无须挂怀,我无碍,能镇压住这两头太古魂兽。当务之急,还是攻入豆神宫。这豆神宫前后变化极其明显,拖延时间,恐怕所图不小。仙友若是任由拖延时间,恐怕会有变故发生啊。”

%16
“既然已经落入我房家的桃花迷林中来,还怕有什么变故。”一位房家蛊仙不屑地道。

%17
房睇长眼中寒芒爆闪,望着方源。

%18
方源又闭上双眼。

%19
“这算不尽好厚的脸皮,原本还想驱策他强攻豆神宫,试探危险,没想到他居然如此表现。可惜可惜,这盟约中并没有条约,让他就范。当下,还是要以大局为重。这豆神宫乃是我族世代筹谋的目标,不能因小失大。”房睇长饶是智道大宗师,也只得硬生生咽下了这口气。

%20
他继续下令,三大仙蛊屋中又再次消耗大量的仙元,无数花瓣飘零飞舞,宛若疾风暴雨。

%21
花雨中,又有大量的花间风流士成形,被房家蛊仙一一操纵,向豆神宫袭去。

%22
“你们去,该是你们表现忠心的时候了。”豆神宫中,陈衣满脸冷漠,对败军老鬼、鹰姬下令。

%23
后两者相互对视一眼,均十分担忧、为难。

%24
鹰姬脸色难看:“依我二人能力,出去一战,就是送死。还请陈衣大人明鉴!”

%25
陈衣呵呵一笑,目光冷酷:“你们若是违抗我的命令,就是立即受死。出去抵挡的话,我还有支援,还有一丝生机和希望。只要拖延得利,让我收服了这头太古传奇魂兽,我还会大大奖赏你们。我的身份你们也了解,也不屑虚言哄骗你们。给你们俩三息时间,做出决定。”

%26
陈衣相当强势,败军老鬼、鹰姬只得苦涩至极地领命,出了豆神宫,迎击众多的花间风流士。

%27
轰轰轰!

%28
双方交火,围绕着豆神宫,激战成一团。

%29
败军老鬼、鹰姬都是七转魔道蛊仙,远比太古魂兽要有智慧得多。尤其是他们两位成为魔修已经许久,都受到过正道势力的追上,所以对保命的功夫都很擅长。

%30
败军老鬼和鹰姬虽然被压入下风,但是韧性却是很强,居然牵制住了这些花间风流士。

%31
不过,随着时间推移,败军老鬼、鹰姬伤势渐多,情势变得岌岌可危起来。

%32
陈衣一心多用,一边对豆神宫加强影响,一边强行镇压青仇,一边还观战,他心中暗赞:“这一招精妙得很,能够灌输仙道杀招进去,令仙蛊屋中的蛊仙充分发挥出自己的独到手段,令仙蛊屋变得和上古战阵一般灵活,不简单!”

%33
念及于此,他伸手一挥,顿时豆神宫轻轻一震,发出两道玄光,正中败军老鬼和鹰姬身上。

%34
玄光流转,很快就没入败军老鬼和鹰姬的身体里面。

%35
两人俱都一惊,连忙查看自身,却没有发现任何端倪。

%36
这时房家的攻势又来,两位无暇多想,硬着头皮迎战。

%37
战斗中,两人很快发现了不一样的地方。

%38
当他们再次受伤的时候,身上都有玄光一闪,所受到的这些伤害立即就被转移出去,不会影响本身。

%39
“此乃豆神宫的手段,仙道杀招嫁木,尔等尽管施为,所受杀招均由豆神宫承担。”这个时候,陈衣的声音适时地在两人耳边响起。

%40
败军老鬼、鹰姬心中又惊又喜。

%41
喜的是自己还有利用价值,陈衣并没有纯粹将他们当做炮灰使用。现在出手支援他们俩个,使得他们又能够支撑更久。

%42
惊的则是陈衣这才来到豆神宫这么短的时间,居然就能掌控豆神宫,催发出第二种手段,可见豆神宫正逐步地落入此人掌控当中!

%43
“咦?”房睇长很快就察觉到了这种变化。

%44
败军老鬼、鹰姬不管受到何种杀招,身上都是玄光一闪,就若无其事,安然无恙。

%45
“这是什么杀招?如此玄妙!不仅是攻伐招数,就连一些宙道、宇道的辅助杀招,都不能影响他们两个。”

%46
房睇长心中疑虑渐重,忽然命令房家蛊仙:“去施展杀招,为那二人疗伤。”

%47
房家蛊仙为之一愣,怎么向敌人伸出援手?

%48
但房睇长威望很高,无人反对质疑,立即就有人行动起来。

%49
治疗手段接连打在败军老鬼、鹰姬的身上,也毫无作用。

%50
“哦?看来这临阵指挥的房家蛊仙,是一个能人呢。”陈衣双眼精芒一闪,立即看出这是房睇长在故意试探嫁木杀招的底细。

%51
“就连治疗杀招都不能够起到效果……”房睇长立即动用手段,推算片刻,又下令道,“不管这两人,全力轰击豆神宫。”

%52
众仙领命,花间风流士接连自爆,大量的攻伐杀招,有的威力非凡,有的效果奇妙,都打在豆神宫中。

%53
豆神宫被打得摇晃不已,虽然核心还未撼动,但大量的凡蛊不断损失。

%54
陈衣暗暗点头。

%55
房睇长的对策十分明智,已经算出败军老鬼、鹰姬和豆神宫的关系。打击这两个蛊仙,任何杀招都有可能被躲闪或者抵御,但豆神宫却是不动的,打在它的身上,它只能默默承受。

%56
并且房家的这些杀招,只要是通过花间风流士施展出来,其威能妙用还能得到桃花迷林的增幅!

%57
若非如此,也不能轻易地撼动豆神宫。

%58
陈衣微微皱眉,豆神宫他只能影响,还未掌控。甚至,在某些方面,他还不如青仇。

%59
青仇也能影响豆神宫,驱使这座仙蛊屋飞行转移。

%60
但现在这部分权利,被青仇霸占,陈衣根本无法驱使豆神宫腾挪,所以豆神宫只能一动不动地坐落在原地。

%61
陈衣可以杀死青仇,将这部分权利夺到自己手中来,如此就能令豆神宫飞行。

%62
但这样一来,青仇就不能留下性命,大大影响了陈衣今后的修行计划。

%63
因此,陈衣现在是千方百计地想要强行逼降青仇,逼降成功,不仅青仇能够为他效力,而且还能掌控豆神宫腾挪飞移,不再像现在这般被动。

%64
房家一改战术,顿时占据上风。

%65
那败军老鬼、鹰姬反而成了无人关注的角色。

%66
这两人也是别有心思,本来就是被陈衣强行威逼,现在房家蛊仙不关注他们,他们也乐见如此。因此也保留力气,没有全力出手干扰对抗。

%67
豆神宫硬是挨了房家狂轰滥炸半盏茶的功夫,陈衣紧锁的眉头忽然一展。

%68
这段时间又让他对豆神宫影响更深,能够施展出另一手段。

%69
“可惜不是防护手段,不过此招施展出来,也能改善局势了。”陈衣心道。

%70
仙道杀招——万生春雷!

%71
下一刻,豆神宫狠狠一震,无数电光如蛇如龙,缭绕在大殿周身。电光凝聚,很快形成一颗颗的碧绿雷球,向大殿周围四处轰炸。

%72
轰隆隆……

%73
一连串的爆响声中,花间风流士只有少数成功逃离,其余尽皆折损。

%74
“这杀招!”房睇长目光一凝,脸色史无前例地沉重起来。

%75
只见这春雷并不只是炸掉了花间风流士,而是还落到地上,落到空中,将无数桃林炸碎。

%76
这些桃林本身乃是仙道战场杀招桃花迷林所化,正常而言,若是破碎也能被桃花迷林杀招迅速修补还原。

%77
但经过这些春雷炸毁的地方,却是形成一片碧绿的草地。

%78
并且一棵棵笔直的松柏,取代了桃林,将根深深地扎进地里去。

%79
每一颗春雷炸毁的地方,就有一棵松柏生长出来,渐渐地就形成了一片稀疏的小树林。

%80
虽然这小树林的范围,完全不能和桃花迷林相提并论,但是却像是海边的礁石,顽固地抵挡着桃花迷林的复原。

%81
ps:今天一更。

\end{this_body}

