\newsection{菟舔}    %第三百三十三节:菟舔

\begin{this_body}

%1
方源心头顿时一震。

%2
“这是什么杀招?!”

%3
“卜卦龟最硬的地方,就是它的龟壳,但是居然承受不住这招的伤害。”

%4
“即便我这至尊仙体,有着道痕之间不互斥的特性。依照太古荒兽的皮糙肉厚,也竟然落得如此结果?”

%5
黑菟战力之强,出乎方源意料之外。她的这个手段,十分诡异,竟然连卜卦龟的防御,都有点撑不住。

%6
想到这里,方源再不敢保留,连忙调动七转仙蛊防备。

%7
这只律道仙蛊,已经被方源成功地添加进了杀招刚背之内。

%8
当即,原本的刚背杀招停住,全新的刚背杀招徐徐催动起来。

%9
呼。

%10
一阵风凭空产生,卜卦龟气势大变。

%11
全新的刚背杀招,采取了大量的宙道凡蛊,这些蛊虫带来的效果,就是削除了防备仙蛊本身单独使用时的十息弊端。

%12
因此,刚背杀招一旦催发成功,防备仙蛊就能立即起效。

%13
小龟壳念头猛地膨胀一圈,变得更加硕大,与此同时卜卦龟本体,开始闪烁如铁般的光晕,颜色也漆黑深重了一分。

%14
黑菟见此,面容微微一变,伸出舌头再舔。

%15
双方距离遥远,但隔着这么长的距离,方源仍旧遭受到了攻击。

%16
龟壳上再度出现了一个坑洞,但是这一次,坑洞只有成年人的拳头大伤口浅薄,并无黑烟缭绕升腾。

%17
另外之前的那道伤口,黑色的烟气也已经消散,伤口扩散的速度大大减缓。

%18
黑菟冷哼一声,很不甘心,口中低喃:“既然如此,那我就攻击这里!”

%19
下一刻,方源只感觉自己的龟头骤然一麻。

%20
随即,剧痛袭来,并且还有酥麻之感,夹杂其中。

%21
伤势不轻。

%22
方源连忙将龟头缩进壳中去。

%23
“该死!”黑菟见到这一幕,顿时气得双眼要喷火。

%24
方源很赖皮,不仅是龟头,他还将四肢和尾巴,都缩进壳中去。

%25
只留下小龟壳念头,在不断地旋转,而且数量不断增长,又将防守的强度和范围,提升一层上去。

%26
黑菟连续舔龟壳,但收效一点都不大。

%27
她陷入困境,终于拿方源没有办法。

%28
武辽一边疗伤,一边观战,看到这一幕,很很地汗了一下。他不禁心想:“就这样,武遗海大人要给武安报仇雪恨,该等到何年何月?”

%29
巴德冷哼一声,也有些不满。

%30
别的缺口,打得水深火热,方源这里却是慢条斯理,不温不火。

%31
方源龟缩起来,非常惬意。

%32
他迅速思考,琢磨黑菟的这记杀招。

%33
任何的杀招,往往都有指向的能力。就比方说一个火球杀招,火球催发出来,凝聚在手心中,最终要飞出去,才能有杀伤效果。

%34
但是飞到哪里,哪个方向,哪个目标,就需要杀招中包含相应指向的能力。

%35
只有指向哪里,火球飞到哪里。

%36
指向能力一般的话,火球就是直线飞行。脱离手心之后,就不能操纵和修改它的飞行方向了。

%37
若是指向能力强一些,火球可以曲线飞行。脱离手心之后,还能略微地调整一下它的飞行角度。

%38
更强一些,火球可以自行追踪目标。

%39
“黑菟的这记杀招,一定也有指向的能力。”

%40
“它虽然不像火球那样明显,攻击发出后,无形无质。但是缺乏指向,是不可能的。因为它显然不是范围杀伤,而是集中一点的伤害。”

%41
“那么它是如何指向的呢?是通过视觉吗?”

%42
想到这里,方源立即分心他用,催动蛊虫,卜卦龟的身边迅速出现一股浓雾,将卜卦龟牢牢笼罩其中。

%43
“这个武遗海,又想搞什么鬼?哼。”黑菟怀疑方源要使出其他仙道杀招,立即伸出舌头,遥遥对准方源,舔了一下。

%44
方源再次中招。

%45
“不是视觉吗?难道是气味?”

%46
方源再用手段,雾气陡然间变得极臭无比。

%47
离得比较近的武辽,闻到气味,顿时面容扭曲,连忙催动蛊虫,进行内呼吸。

%48
臭气飘散在空气中,很快逸散开来,惹得其余战团纷纷投来目光。

%49
“这个武遗海,居然在尝试破解对方的杀招。”巴德双眼精芒一闪,看出了方源的目的。

%50
他冷哼一声,心中更加不满。

%51
一般当场破解敌手杀招的可能性很要不断地试验,就算勘破原理,也未必有相应的仙蛊或者手段来实施。

%52
尝试破解杀招,一般是一对一的长久僵持作战。

%53
“现在这种紧急的情况,他居然还有闲情逸致来破解杀招?哼!有手段,直接把对方强杀了才是正理!”巴圈风却是直接将心中的不满说出来。

%54
巴德充耳不闻,此时此刻,不是闹矛盾内讧的时刻。

%55
黑菟犹豫了一下,再次伸出舌头,对准方源舔一下。

%56
方源的背壳上再次出现了一个坑洞伤口。

%57
方源忽然灵光一闪:“我明白了,很可能是味觉的指向啊。”

%58
一般而言,味觉需要舌头亲密接触来体验。但是蛊仙的舌头,在仙蛊或者杀招的效用下,隔空尝物,也并非难事。

%59
“每次发动,黑菟都要伸出一下舌头。”

%60
“她和影宗搅在一起,影宗则掌握着食道真传。也只有如此偏门的流派,才能对我这般防御深厚的卜卦龟变化,有着奇效。”

%61
“试验一下。”

%62
卜卦龟慢慢的通体发红,尤其是背壳,仿佛是烧红了的锅底。

%63
黑菟伸出舌头一舔,立即抽了一口冷气:“嘶好烫!”

%64
卜卦龟的红色消散下去,空气中飘出一股怪味。

%65
黑菟伸出舌头一舔,差点被齁得呕吐出来,眉头紧皱:“好咸!”

%66
卜卦龟再生微妙变化。

%67
黑菟终于不再尝试。她看出来方源的用意:“这家伙!居然在尝试破解我的杀招,而不是想要催动什么仙道杀招。可恨!!”

%68
黑菟之前觉得:卜卦龟身上的这些变化,很有可能是方源催动仙道杀招的前奏。

%69
所以,她三番五次进攻,企图破坏和干扰方源催动杀招,一旦干扰成功,反噬伤害定能让方源受伤。那么她对这个乌龟壳障碍,就有跨越的希望了。

%70
“这种情况下,他居然有闲工夫来尝试破解我的杀招?!”方源的举动,有违常理,毕竟此时战场争分夺秒,谁有如此闲情逸致。

%71
不过也正是因为这一点,才让黑菟中招。

%72
意识到这一点后,黑菟立即停用这一杀招,而是改为其他手段遥攻。

%73
她左右挥拳,黑青拳劲穿梭在烦杂的小龟壳念头中,有的被小龟壳念头联合剿灭,也有的击破沿途的小龟壳念头,打在卜卦龟的龟背上。

%74
虽然伤害微乎其微,但每一次拳劲集中,方源整个身躯都忍不住颤抖一下。显然,这个仙道杀招中掺和进了律道的蛊虫。黑菟如此干扰方源,防备他催动仙道杀招,的确是好手段。

%75
“的确是味觉么。”方源的龟头,始终缩在壳里。

%76
尽管尝试有了结果,但是距离破解黑菟的这一杀招,还有极长的距离。

%77
方源却不急躁。

%78
黑菟已经不再使用这一招,单纯的拳劲干扰,对于方源而言,无关痛痒。

%79
因为武遗海的这个身份,还有手段,明显都是攻弱守强。

%80
黑菟防备方源的攻伐杀招,其实是多虑了。方源的确是强大的攻伐手段,但是碍于身份,他不能用。一旦用出来,他就会惹来极大的怀疑,身份距离曝光就不远了。

%81
于是这处战斗的节奏,再次缓慢下来。

%82
方源不想进攻,或者说拿不出有效的手段来进攻,而黑菟则攻不破方源这个缩头乌龟。

%83
这么一段时间过去,方源身边环绕着的小龟壳,已经上涨到了十万的规模。

%84
黑菟感到自己无从下手,庞巨的小龟壳,数量惊人,已经变成战线中惹眼瞩目的景象。

%85
巴德瞧着方源的目光越加冰冷,他终于忍不住传音:“武遗海,我听闻你还有其他两项变化。此时此刻,你还留手存力做什么?赶紧杀掉你的对手,支援其他仙友!或者说,你和这位白兔真的有什么关系?你下不去手?”

%86
方源听到巴德的传音,理都不理他,就当没听到。

%87
巴德久久得到回应,气得暗暗咬牙,瞧着方源的目光越加阴冷。

%88
方源在琢磨自己的处境。

%89
“影宗肯定是要营救幽魂本体,所以强攻这里。”

%90
“他们控制了左夜灰,但控制程度应该有限,还有一位八转蛊仙紫山真君没有现身。”

%91
“正道这边拥有超级蛊阵,可以守护一段时间。只要拖延一定的时间,必然会有八方的支援,四面的援军。”

%92
“而我拥有逆流护身印,可抗八转。实力不上不下。”

%93
“紫山真君乃是智道蛊仙,怎可能没有后手?正道的这些蛊仙,也远远没有到达山穷水尽的地步。”

%94
“我还是隐藏身份,等到时机浑水摸鱼。说不定这一次,我真的能摸到一头大鱼!”

%95
方源越是琢磨,越发现自己大有机会。

%96
他有这样的实力,但必须等到恰当的时机。

%97
影宗是他的心腹大患,他一直想要追杀,将其除尽。可惜方源追杀的次数也不少了,都没有成功。不是他不努力,也并非他态度不坚决,而是影宗这群人的确是人中龙凤,各个精英。

%98
但现在,无疑是个绝佳的机会。

%99
ps:第七更。未完待续。

\end{this_body}
\newsectionindepend{今天的爆更献给一直支持和我的朋友们!}
\begin{this_body} \par
%100
首先修改了一个bug。

%101
黑莬,是错误的,是bug。

%102
现统一将“黑莬”修改成“黑菟”。

%103
黑菟,才是我一直想要为白兔姑娘的另一个性格而取的名字。

%104
莬是一种植物,菟是虎的意思。吕布的赤兔马,大家应该都很熟悉。为什么这么强悍的一匹坐骑,要起赤兔这个名字呢?

%105
其实有一种说法是这样的,原来并不叫赤兔马,而是赤菟马。兔是兔子的意思,菟则是虎的意思。原本吕布的坐骑,应当是赤菟,也就是赤虎的意思。因为通假字或者其他原因,渐渐成了赤兔马。

%106
我个人比较倾向于这种说法的。

%107
所以,黑菟的意思,就是黑虎,恰恰好和之前的“白兔”,形成鲜明的性格对比。她的招牌杀招,大家今天也看到了,菟舔。

%108
实话告诉大家,这是食道杀招。

%109
为什么要取这个名字?

%110
菟舔,和虎噬对应,但没有后者的霸气,而是多了小巧和阴损。非常适合黑菟这个角色。

%111
然后说一下今天爆更的原因。

%112
为什么爆更呢?

%113
回馈一直支持我的读者朋友们!

%114
自从我上个月转专职以来,陷入了一个困境。

%115
是心理疲劳、烦躁。

%116
因为整天把自己关在家里,苦思冥想情节什么的,一天到晚,从早晨到晚上,时间久了,很抑郁,很压抑。

%117
状态也越来越差。

%118
我意识到这样子是不行的,只能毁了自己。所以,我努力调整自己,多接触社会,多在外走走,渐渐的调整过来。

%119
在这段时间里,有些人质疑我,否定我,嘲讽我其实我是无所谓的,反正这么多年,已经习惯了。但是骂我身边一直支持我的朋友们,这点让我忍受不了。

%120
我在一个书评里,打过这样一个比喻:我们当中的一些读者,就像和我是小学校友,有四五年的交情,有两三年的交情。不管这书如何如何,一直相互扶持,一路走来。哪怕我的状态再不好,哪怕蛊真人写的再烂,从来都是在一起。就好像我上过的第一座小学,虽然已经被废弃了,被拆了,但是我们这些小学校友,往往路过的时候,会看上一眼,这是纯粹下意识的行为。或许还会趴在小学大铁门前观望里面的荒凉。

%121
是的,我说的就是情怀。

%122
有人或许反感这个词,要不要这样俗啊?

%123
那我要说,是你们不了解我们的这种情怀。不了解我们曾经面对各种责难、嘲讽时,拼力斗争的时候。

%124
当然,我很欢迎新朋友的加入,我很愿意让咱们这个和谐团结的集体更发展壮大。

%125
今天的更新献给你们,我的朋友们。

%126
我知道的,你们一直都在!

%127
之后还有一更。

\end{this_body}

