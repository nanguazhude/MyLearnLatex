\newsection{勒索梦境}    %第七百九十四节:勒索梦境

\begin{this_body}



%1
整个气道梦境的第三幕,方源完全沦为了看客。

%2
不过,当这第三幕完成,整个气道梦境也就消散无踪,方源探索全部完成。

%3
“这梦境前难后易,前面两幕需要不断的失败,积累经验,就算是我有解梦杀招,也不太好用。”

%4
“但到了最后一幕,根本不需要我出手,就直接通过了。”

%5
这是方源沦为看客,最为舒服的一次。若是次次都这么“看”过去,那探索梦境该是一件多么美妙的事情。

%6
“这个梦境的时代,不是远古,就是上古时代。”

%7
“天庭……”

%8
方源的目光也有些复杂。

%9
纵然他是天庭的死敌,也不得不承认天庭的伟大之处。

%10
旋即,他的目光就只剩下一片冰寒:“就算再是伟大,挡住了我的路,我也只有竭尽全力将你铲除了。”

%11
这片梦境探索成功,方源成功地晋升为气道宗师!

%12
进步的程度有点大,毕竟方源之前的气道境界,完全是一穷二白,普普通通呢。

%13
“难怪那梦中的卫玉书,能够获得墨人王女的不断恩宠。想必这份在气道上的才情天资,也是其中的主要缘由吧。”

%14
此次探索梦境,方源最大的体会,反而不是梦境,也不是天庭,而是自己的魂魄。

%15
荒魂!

%16
成就荒魂之后,探索梦境更加容易了。

%17
换做人魂层次,这一次探索时带来的创伤,必然重得多。虽然有胆识蛊可以恢复,但必将十分连累探索效率。

%18
方源脑海中忽然灵光一闪:“或许,将来我利用纯梦求真体探索梦境,魂魄受创的程度会比现在还要少。”

%19
就像方源在落魄谷中修行,吹拂落魄风。一旦方源魂魄遁入肉身当中,落魄风的威能就会急剧下滑。

%20
方源若是用纯梦求真体带着魂魄,进入梦境中探索,魂魄应当也会受到肉身的严密保护。

%21
寻常的肉身,是无法阻挡梦境的力量。但纯梦求真体是不一样的。

%22
可惜的是,如何运用纯梦求真体入梦,方源还不知道正确的方法。

%23
如果是直接将纯梦求真体派遣到梦境当中,纯梦求真体将进出自如,不受到梦境的干扰。但同时也无法沉浸在梦中的内容里,也就无从谈起探索梦境了。

%24
“接下来,就探索这片冰雪道的梦境。”方源转移目光。

%25
方源的至尊仙窍中,已经储藏了不少梦境。

%26
刚刚的气道梦境是规模最为庞大的,已经消散。这片有关冰雪道的梦境,则顶替上来,成为最大规模的梦境。

%27
除了这片梦境之外,还有一些梦境,有气道、剑道、木道等等,零零散散。

%28
“我有荒魂,探索梦境的效率大大提升,这些梦境库藏,之前还觉得不错,现在却是有些少了。”

%29
“并且,梦境分门别类,零散琐碎。呵呵,这是池曲由动的小心思啊。”

%30
方源心中雪亮。

%31
池曲由虽然和方源不断交易,但大阵他能够暗中掌控,又因为得到了不少梦道成果,已经能辨别梦境的大致种类。

%32
所以,他和方源交易,都是刻意控制,尽量将不同流派的梦境零零碎碎地交易给方源。

%33
方源真正想要的人道梦境,一个都没有。气道、冰雪道也不多。

%34
但这些梦境都对方源有用。

%35
哪怕是宙道梦境、炼道梦境,只要不是前世探索过的梦境。

%36
毕竟,方源的境界最高也不过是准无上,还是有着上升的空间。

%37
方源来者不拒,这些梦境都能提升他的境界。而且将全流派都晋升上去,再无短板,也是极好的事情,将来吞并仙窍就不用担心这方面的制约了。

%38
在方源探索梦境的期间,毛民蛊仙们有了成果。

%39
“方源大人,我们将这只仙蛊炼成了。”毛六送上一只仙蛊。

%40
这只仙蛊像是水滴,闪烁着蓝钻般的璀璨光辉,摸在手中,表面滑润冰凉。

%41
这是宙锚仙蛊,六转层次。

%42
方源虽然在这段时间,主要大炼运道仙蛊,但宙锚仙蛊这样的蛊虫,还是在炼蛊的计划中的。

%43
方源的仙蛊数量虽然破百,但他养得起,有的是资源,财大气粗,所以仙蛊是多多益善。

%44
宙锚仙蛊他早年就接触到,曾经是东海仙僵苏白曼的仙蛊,能够标记时间点。

%45
后来苏白曼死了,宙锚仙蛊就被影宗获取,但在义天山大战中损耗掉了。

%46
这一次,被方源命令毛民蛊仙,重新炼制了出来。

%47
宙锚仙蛊搭配春秋蝉,可以形成杀招,令方源重生在特定的时间点上。

%48
这个杀招价值极高。

%49
就拿眼前而言,方源完全可以标记现在这个时间点,然后不断地探索梦境,什么也不管。到了一定阶段,他利用这个杀招,回到过去,重新来到现今这个时间点,再探索别的梦境。

%50
流派的境界是可以累计的!

%51
方源可以凭借此招,来不断地积累境界。

%52
“可惜的是,若用这个杀招,就不能用春秋必成了啊。没有百分百的成功,就有失败的概率,一旦撞上,就等若自杀,要被世人笑死的。”

%53
方源虽然在宙道境界高达准大宗师,又有智慧光晕,但这里的障碍并非来源于他自身。

%54
“如我所料不错,宿命蛊虽伤而不灭,正是它制约了宙锚仙蛊融入春秋必成的杀招成效。”

%55
利用春秋蝉重生,是钻了这个世界的漏洞。因为这会扰乱事实,改变过去,影响平衡,天意并不允许有这样的事情发生。

%56
对于方源现在来讲,单纯用宙锚和春秋蝉配合,杀招刚好能钻过这个漏洞。或者用春秋必成,也能钻进这个漏洞。

%57
但若是两者叠加起来,漏洞就嫌小了。

%58
“这样看来,只有等到宿命蛊完全毁灭,宙锚融合到春秋必成当中去,方才有效了。”

%59
红莲真传中的种种杀招,也是如此。杀招之间就算融合贯通,使得威能更大,但实际运用起来,反而没有好的结果。

%60
龙人分身已经渡过了一次天劫,还有十几次的地灾,修为暴涨连连。

%61
方源为了尽快提升他的修为,动用了宙道的杀招,把龙人分身仙窍的时间流速加快到了极致。

%62
每一次灾劫到来之前,方源都用手段来算清灾劫的内容,因此有备无患。

%63
在方源强劲的实力下,这些天劫地灾完全没有任何威胁,成为一场场的烟火仪式,给方源的龙人分身一次次赠送道痕。

%64
方源有宙道手段,关键是又有石洞天机杀招,使得天意降下灾劫,也无法出奇制胜。在他如此雄厚的实力帮衬下,这些灾劫的威能根本不够看的。

%65
方源如今不仅能够提拔蛊师的修为,而且还连蛊仙修为都能迅速拔高。

%66
不过,此法并不值得提倡。

%67
因为成本高得很。

%68
加快仙窍时间流速的宙道杀招,就高达八转层次,消耗大量八转仙元。

%69
石洞天机杀招也不是一般的智道蛊仙能够催用的。

%70
再加上每一次渡劫的付出,种种累加起来,却只产出一个六转或者七转修为的龙人分身。

%71
投入和产出完全不成正比。

%72
若不是为了图谋龙宫,方源也不大愿意这样打造龙人分身。

%73
南疆的烽火台,铺设得如火如荼。而南疆正道的蛊仙也开始吵闹不休,要求方源先行释放一些人质,以示他的诚意。

%74
于是,方源将翼扬放出来。

%75
翼家得了翼扬后傻眼。

%76
因为方源只释放了翼扬的肉身,他的魂魄,他的仙窍都没了!

%77
光是肉身,不过是个尸体,有什么用?!

%78
南疆正道怒恨交加,又拿方源无可奈何。

%79
方源劝说他们:肉身只是第一步,展现我的诚意,接下来该你们展露诚意了。

%80
南疆正道均感到前景一片晦暗:方源若是将人质这要拆开来勒索,这可要勒索到何等地步,猴年马月?

%81
于是,又一阵叫嚣。

%82
方源便改变态度,变得强硬冷漠,单独威胁他们每个家:真要我释放人质,我就释放你们的敌对势力。并且对你家的蛊仙如何如何侮辱,保管他(她)今后抬不起头来做人。

%83
这事情方源上一世就干过,这一次干轻轻松松,十分熟悉。

%84
南疆正道可耻的怂了。

%85
没有办法,尤其是巴家、夏家这对苦逼的家族。他们的太上大长老都被方源俘虏了。

%86
其他家族损失的也是他们的重要战力,若不营救,就要丧失人心。

%87
方源承受要求南疆正道,将梦境当做筹码,交给方源。

%88
南疆正道起先极不愿意,毕竟梦境可是代表着未来!

%89
但方源却保证:自己要求不多,绝不会掏空这里,甚至会将大半仍旧留给南疆正道。

%90
为了表示更多的诚意,他愿意在接下来每一笔梦境的交付过程中,释放一位位蛊仙俘虏的肉身。

%91
南疆正道无法可想,只得接受了这个方案。

%92
池曲由心中则有些发冷,他知道这是方源的反击,还有警告。

%93
于是,对于南疆正道的勒索,进行到了下一个的层次。

%94
方源没有失信于人,每得到一批梦境,他就释放一位人质的肉身。这些梦境可比他和池曲由暗中交易,要大得多。并且种类上,也由方源精挑细选。

%95
紫薇仙子焦急不已,多次施展手腕,企图破坏这样的交易,但均没有成功。

%96
“是时候再来一次大战了。”方源算了算时间,驾驭万年斗飞车,再次进入光阴长河。

%97
他主动拉开第二次光阴长河大战,却不知道已有一位宙道大能顾六如在等着他。

\end{this_body}

