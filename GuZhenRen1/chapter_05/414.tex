\newsection{黄史阵亡}    %第四百一十五节:黄史阵亡

\begin{this_body}

轰隆!

巨浪绕过太古年猴,从四面八方围拢,直接拍击黄史上人。

黄史上人一颗心,在这刹那间冰凉彻底。

他乃是天庭蛊仙,八转中的强者精英,在这一瞬间终于明白过来。

“原来方源等人,竟然真的可用操纵这片河段?!”

他心中震惊至极。

这代表着,之前太古年猴困在刀剑河域中,明显是一个陷阱。而他自己一头扎进陷阱当中,之前的战斗他都被蒙在鼓里,落入算计还不自知!

这对于黄史上人而言,简直是一种侮辱!

“但是怎么可能?方源居然能操纵一段光阴河流,他虽然继承了黑凡真传,但也不至于如此吧!这可是紫薇大人推算的结果。”

震惊的情绪稍稍缓解,黄史上人的心头又涌出一股股疑惑。

在巨浪扑击的过程中,他的脑海中念头可谓千回百转。

“这恐怕是红莲真传的威能了。”

“也只有红莲魔尊这般的人物,才有这样的手段,可以掌控一段光阴长河。”

眼见着巨浪滔天,已经近在眼前,黄史上人脸上涌现出坚毅之色。

“也罢!”

“就让你们见识一下我黄史上人真正的厉害!”

“啊――!”黄史上人忽的仰天咆哮,一时间身上无数的蛊虫气息喷涌而出,一道道黄色的光晕,在他的身上不断地****闪耀。

他光头锃亮,闪闪发光。

头顶上空的土黄巨石猛地膨胀起来,变得宛若一座小山,镇压光阴河水。

他气息变得深有莫测,一双眼睛注视之处,光阴浪涛纷纷恢复原状。

与此同时,从他的身上跳跃出无数光阴斑斓,纷纷向白凝冰等人飘去,速度似缓实快!

黄史上人心头惊怒,又身处绝境,爆发出了十成战力!

“好厉害!一瞬间,动用了至少四记仙道杀招。”方源见到这样一幕,也不由地心头一跳。

黄史上人既然能得到紫薇仙子的信任,被独自派遣到光阴长河当中,狙击方源,自然有着他的实力。

此刻,他爆发出全部战力,立即抵挡住了光阴浪涛,维护自身安全之外,还施展辣手,向白凝冰等人攻去。

白凝冰等人尽皆变色,她们被困在黄褐烟尘当中,无法脱身。

就连太古年猴就加剧挣扎,咆哮声中显露出恐慌之情。

这光阴斑斓的威能,虽然随机莫测,但其恐怖世人皆知。就连凤九歌都不敢轻易碰触,更何况是白凝冰等人呢?

“他这是要逼我现身了。”方源口中呢喃,双眼目光投向一旁的幽魂意志。

他让白凝冰这些人作为诱饵,若是此时牺牲,对他而言,是巨大损失。

不过,其中夹杂着幽魂分魂的影无邪,方源故意安排,自然是怕幽魂意志有其他想法。

幽魂意志哈哈大笑:“不用担心,方源啊,这座石莲岛可是由红莲魔尊亲手建造,黄史上人虽然抵御得住光阴的河水浪涛,但是这其中可是蕴藏着刀剑两道的绝世杀招呢。”

“你看,这一招便是当初的刀九郎的最强攻伐杀招,名为九九轮回刀锋。一旦施展出来,绵绵不休,生生不息,刀光被破坏一层,还有一层,一共九层刀光,每一层皆可屠戮鬼神。刀光斩下之后,又会回环往复,再次攻击,一层刀光总共会攻击九下,九层刀光一起便是九九八十一下。此招极其厉害,刀九郎一生当中也只施展过三次,每施展一次,就要耗费他百年寿元!而此时我勾动的这一招,正是他和习渊交手时施展出来的,生平最巅峰的杰作!”

随着幽魂意志的解说,方源便看见那被镇压下去的光阴巨浪中,果然迸射出九层刀光。

刀光绚烂无比,好似太阳碎片,璀璨至极,即便是方源透过影像观看,也不由地眯起双眼,眼珠子感到一阵阵的刺痛。

他身居战场之外,已然如此,更何况战场当中的众仙。

见到这层刀光,黄史上人面色狂变,惊骇欲绝。

他在这九层刀光中,感受到了强烈的生命威胁。这不是说笑,刀九郎若是中洲十大派出身,绝对能进入天庭。

他曾经是西漠一域的翘楚,八转风云人物,傲立巅峰的刀道传奇。

他施展出来的这一击,是他一生的战斗巅峰杰作,是要和习渊拼命的杀招。

黄史上人惊骇绝伦,再也顾不得什么水浪、太古年猴或者白凝冰。

他竭力嘶吼,使出压箱底的手段,拼死防御。

轰轰轰……

巨响声连绵不绝,层层刀光不断轰击在他的身上。刺眼的刀光一层又一层,连绵不绝,围绕着可怜的黄史上人不断地斩杀,交织成一大团白炙的光团。

而在这光团的最中心,黄史上人缩进头顶的黄褐巨石当中,拼命抵挡。

刀光的轰击持续了片刻,逐渐消散。

黄史上人周围的巨石已经彻底消失,他整个人憔悴不堪,脸上毫无血色,一片惨白,身上更是出现巨大的伤口,遍及胸膛和后背。

伤口处并没有流血,但这恰恰是糟糕的表现,因为伤口上尽是刀道的道痕,不断爆发刀气,侵蚀黄史上人的五脏六腑。

“哈哈哈,看,他已经不行了。”幽魂意志大笑。

“那么再来这一招!”幽魂意志眉眼翘起,手舞足蹈,“这一招可是习渊的王牌手段,哈哈,不信你不死!”

说着,渐渐消散的光阴巨浪中,便升腾起一股强烈的剑气。

这剑气如渊如山,深不可测,有着无以伦比的威能,能埋葬世间万物!

黄史上人瞳孔缩成针尖大小,脑海中只有一个念头――逃!

但下一刻,他发现自己竟是被这股剑气锁住,在半空中悬浮动弹不得。

“这,这是习渊当年的杀招剑葬渊!糟糕,糟糕至极!!”

黄史上人心脏乱跳,已经惊慌失措。

他抵御住九九轮回刀锋,已经耗尽了手段,此时此刻状态低迷,防御力量也跌落谷底。

“难道我今天就要陨落在这里不成?”黄史上人预感到了死亡的气息。

就在这个时候,石莲岛上,幽魂意志却叫出声来:“糟糕,石莲岛支撑不下去了。我们若要执意杀掉黄史上人,这座石莲岛底蕴将被耗干。如果放掉黄史,还能存在数年。”

“杀了他。”方源目光幽幽,毫不犹豫地吐出这三个字。

“明白了!”幽魂意志应答一声,再不说话。

黄史上人剧烈挣扎,但却始终无法摆脱剑气的束缚。

他的脸色狰狞恐怖,心中不断嘶吼。

“没想到我堂堂黄史,竟然死在这里。”

“我牺牲并不要紧,却是害得天庭计划失败,让方源这个天外之魔继续逍遥。”

“我不甘心……啊!我不甘心啊……”

黄史上人咆哮连连,忽然声音戛然而止。

然后,他的生机彻底消失,那股剑气仿佛是黑暗中的怪兽,将他的生机全数吞并下去。

八转宙道,天庭蛊仙,黄史上人阵亡!(未完待续。)

\end{this_body}

