\newsection{东海追逃战(中)}    %第二十六节:东海追逃战(中)

\begin{this_body}



%1
追,还是不追?

%2
东海众仙面面相觑。

%3
最先反应过来的,是七转蛊仙汤诵。

%4
“休走!”他身形如电,追赶方源,口中大叫。

%5
方源冷哼:“得寸进尺之徒!莫要以为我怕了你,若非身负重任,我必定将你斩于阵前。”

%6
汤诵闻言更怒:“放屁!”

%7
他手中的仙道杀招已经差不多能催发出来了,就要挽回脸面,偏偏这个时候,方源跑了。

%8
他居然跑了!

%9
汤诵当然不愿善罢甘休,一路猛追不舍。

%10
他的行动,带动了迷茫中的东海蛊仙们,众仙也纷纷追赶过来。

%11
方源心头顿沉,表面却故意大叫:“不怕死的,就都过来吧,哈哈哈。”

%12
汤诵紧接着喊道:“怕死的都是孬种!追,传承印记就在他的手中,千万不能让他和其他蛊仙汇合。只有擒拿了此人,我们才有机会瓜分了大能的传承。”

%13
重利当前,东海蛊仙闻言,顿时飞得更快了些。

%14
方源嘀咕一声“不算太笨”,忽然伸手往后一指,飞剑仙蛊电射而出,直朝汤诵打去。

%15
汤诵吃了一惊,连忙躲闪。这一分神,刚刚要发出的仙道杀招,顿时破坏,若要催动,还得从头再来。

%16
汤诵气得想要跺脚。

%17
七转蛊仙刘青玉则越过他,追向方源。

%18
方源看来一眼。挑拨离间道:“刘青玉,传承印记明明就在你的手中,你却来杀我。好演技!”

%19
刘青玉大怒:“有种的别跑!”

%20
“面对你们,我何须逃跑,凭白丢了家族的颜面!”方源高呼。“有种的你们继续追,我倒要看看最后谁更倒霉些!哈哈哈。”

%21
方源有恃无恐的样子,被一群东海蛊仙追赶,居然还这么嚣张。

%22
反而东海蛊仙们,各个心中的迟疑之情越发深厚。

%23
且不说关键传承印记,到底在谁那里?就说方源之前的说辞,可信度很高。

%24
若真是如此。前方就有方源的援兵了。而且要围猎这么一大群的上古云兽。恐怕早就布置了仙级战场杀招,甚至调动了仙蛊屋,都有可能!

%25
“若是陷入仙级战场杀招,可就难办了……”

%26
汤诵他是超级势力汤家的一员,或许不惧,但这些东海蛊仙中,却是以散修居多。

%27
这些散仙也混得不太如意。要不然,他们也不会前往乱流海域,搜寻什么机缘了。

%28
毕竟,蛊仙的时间也是很宝贵的。

%29
先是一位蛊仙,慢慢落入队尾,随后渐渐脱离了追赶的大部队。

%30
随后,越来越多的蛊仙开始效仿。

%31
蛊仙都不笨,都很精明。

%32
落后的蛊仙,当然不愿意就这样放过方源,但也害怕落入仙道战场杀招里面。不得脱身。所以都打着让前面的蛊仙探路的盘算。

%33
“这群家伙……不足以成大事!”距离方源最近的刘青玉,察觉到了这个情况,肚中暗骂,行动上也显出一丝迟疑。

%34
他是散修,早年有过仙缘,但身上虽有仙蛊,却都是六转。若是面对超级家族的仙道战场杀招。恐怕脱身不易。

%35
哪怕是汤诵,也心怀顾忌,留出几分力气和注意力,好来应对可能出现的对方援兵。

%36
于是,形成了一个颇为古怪的追赶画面。

%37
方源在前方一路疾飞,时而还大骂怒喝,处于弱势,却态度嚣张。

%38
而他身后一大群的东海蛊仙,明明人多势众,却反而偃旗息鼓,气势上处于下风。

%39
方源大多数时间,只以凡道杀招飞行,身后的这群东海蛊仙,也用凡道杀招,吊在后面。

%40
方源心中,远没有外表那么自信从容:“真是麻烦!现在这个情形,只有逃到界壁里去,借助地利,才能真正甩掉这些追兵。”

%41
“刘青玉……”方源将这个名字记在心头。

%42
他明明没有接收到任何的传承印记,或许是被他击毁了,但也很有可能,是之前的血道魔仙故意掩饰的手段,真正的传承印记已经被刘青玉得到手。

%43
若是这样一来,这刘青玉的演技几乎可以和方源媲美,是个城府极深的对手。

%44
万丈云层的高空。

%45
嗖嗖嗖!

%46
以方源为首,一群东海蛊仙们身形如电,划破长空,声势浩大。

%47
汤诵暗中酝酿的仙道杀招,已可施展。

%48
他按捺住兴奋之意,加速上来,越过刘青玉,向方源逼近。

%49
这个古怪的举动,立即引起了方源的警惕。

%50
他连忙催动剑遁仙蛊,速度陡增,将汤诵甩下。

%51
“该死!”汤诵暗骂一声,他这七转仙道杀招距离越近,威能越佳。现在他和方源的这般距离,却是没有生擒活捉他的可能。

%52
方源一催动剑遁仙蛊,速度激增,顿时让身后的蛊仙叫苦不迭。

%53
三位七转蛊仙还好些,可以利用六转仙蛊,或者仙道杀招,维持距离。

%54
但那些普通的六转蛊仙,蛊仙界的底层,却是连一只六转仙蛊都没有,很快就被落下。

%55
眼看方源渐渐飞远,七转蛊仙周礼急道:“我有一法,可携带二位一同飞行,速度颇佳。只是心神凝注,难以再分神,还望二位仙友出手牵制。”

%56
刘青玉、汤诵对视一眼,纷纷点头应允下来。

%57
汤诵答应得最是爽快。

%58
这些人之前就已经达成协议,订下盟约,彼此间有信任的基础。

%59
周礼便长啸一声,身边突然翻涌起无数波浪虚影,波浪虚影承载着他,让他速度激增。

%60
仙道杀招!

%61
方源目光一凝。刘青玉、汤诵却是叫好,主动踏入波浪虚影之中,任由脚下的波浪载着他们疾飞。

%62
两人因此得到余暇,开始出手。

%63
方源顿落险境。

%64
之前,追逃的过程中。身后的蛊仙们也不是没有出手过。但一来方源时常飞出攻伐仙蛊干扰,二来飞行躲闪,灵活如鳅,滑不留手,这才让身后蛊仙没有得逞。

%65
但现在却大为不同,二位蛊仙抽出手来,可以催动更加精妙的手段。更具有威胁。

%66
方源压力骤涨。片刻之后,就身上负伤。

%67
“二位做得好,这人底细也就这样,不过是有两只七转仙蛊而已。其余手段还是凡道杀招。”周礼道。

%68
“他身上原本就受了伤,恐怕是那群上古云兽造成的。”刘青玉阴测测地开口。

%69
“他虽有七转仙蛊,到底还是六转蛊仙,能有多少青提仙元可耗的?”汤诵说着。自己都笑起来。

%70
只要以此发展下去,他们胜券在握。

%71
唯一的顾忌,就是方源背后子虚乌有的家族势力和埋伏。

%72
所以他们仍有保留,并未下死手。

%73
方源情势危急。

%74
如果任由情况这样发展下去,他恐怕就要在今日陨落。

%75
蛊仙灵性十足,智慧出色,比呆呆的上古云兽群要难对付得多。

%76
若真的硬拼,这些东海蛊仙绝不是这群上古云兽的对手。但他们带给方源的威胁,却比上古云兽要大很多。

%77
方源从未指望过,这两者掐架。相互干扰。

%78
蛊仙们都不笨,斩杀一头云兽,他们都要思量一番前后收益,更何况这群抱团的上古云兽?

%79
“这样的话,就只有提前借助五域界壁。就算有所暴露,也没有办法了。”方源方向微微一折,朝着最近的界壁飞去。

%80
身后的东海蛊仙们旋即跟上。

%81
自然还有那群上古云兽。

%82
很快。界壁就出现在方源的视野之中。

%83
界壁就像是一圈栅栏,将整个东海圈住。方源从界壁中出来,本就距离其他的界壁很近。只是当初,他为了赶时间,想直线飞跃一段距离。

%84
若紧贴着界壁行进,会绕很长一段路程。

%85
但谁能料到,好端端的赶路,却是祸从天降呢?

%86
“不好,他要进入界壁之中!”刘青玉窥破方源的意图,立即叫道。

%87
“好打算。我们是七转蛊仙,进入界壁之中,比他要受更多的限制。”周礼接道。

%88
“快快出手,不要让他得逞。”汤诵也急了。

%89
二仙这番出手,又不一样。

%90
攻势凶急,方源难以招架,身上伤势越加严重,有些伤口更是深可见骨。

%91
“呵呵呵。”但他却大笑,“你们真敢动手,很好,汤家!我若死了,坏了家族大事,本族是不会放过你们的。”

%92
汤诵心中一凛:“你口口声声说是家族中人,欲盖弥彰!我熟知东海正道势力,怎从来不知有你这号人物?”

%93
方源再次大笑,充满恨意地道:“汤诵你何必激将我?我是不会上你的当的。我族行此大事,自然要隐秘,担忧他人破坏,所以才派遣我出手。你就算杀了我,我也不会告诉你们我的来历。我还会自爆魂魄,让你们连搜魂都不可行。”

%94
方源的话,让三位蛊仙心中一沉。

%95
什么样的敌人最可怕?

%96
对于这个问题的答案,众说纷纭。但其中很多一部分人,都会认为:躲藏于暗处的敌人最可怕不过。

%97
敌暗我明,你不知道敌人究竟在哪里,采取什么行动对付自己。

%98
这正是三位七转蛊仙所担忧的事情。

%99
至始至终,他们都未怀疑过方源的谎言。

%100
皆因,这上古云兽实在过于稀罕,好端端的怎么会被这么一大群的上古云兽追杀呢?必然是有所图谋的。

%101
既然有所图谋,图谋一方就自信有实力,能吃下这群上古云兽。

%102
在东海,还有什么样的组织,能吃下这群上古云兽呢?

%103
很显然,唯有类似宋家这类的超级势力!

%104
方源一番话,终于让三位七转蛊仙攻势减缓,让他得到宝贵的喘息之机。

%105
眼看界壁已近,方源逃生的希望就在眼前,这时蛊仙周礼忽然出手!

\end{this_body}

