\newsection{红莲真传}    %第六百六十二节:红莲真传

\begin{this_body}



%1
大河澎湃,从来处来,往去住去。 .更新最快

%2
方源站在红莲石岛的边缘,望着眼前的光阴长河,脸色平静,眼眸幽幽。

%3
光阴长河可以说是最为特殊的天地秘境。

%4
因为它贯穿整个天地,没有它的存在和运转,整个蛊师世界就会变成一片静止的画面,不存在任何的变化和生机。

%5
方源夺了荡魂山、落魄谷,这两座天地秘境就从此归他一人所有,但他绝无可能夺走整个光阴长河。

%6
这座天地秘境是全天下万物生灵所共有的。

%7
每一头野兽,每一株草木,每一块土石,每一份水汽,都享受着光阴长河带来的利益。

%8
即便是蛊仙,最多也只是拥有一道光阴支流在仙窍内罢了。

%9
光阴长河是全天下共享的财富。

%10
每一滴的光阴之水,都是苍白透明。亿兆兆的水滴在相互撞击、纠缠、旋转之中,迸发出最灿烂炫目的流光溢彩。因此整个光阴长河的河面,始终笼罩着一层琐碎的,不断跳跃变幻的七彩光晕。

%11
这里是全世界宙道道痕最浓郁的地方,是宙道最大的温床。

%12
方源站在岛边,只是看了一小会,就见到了数头年兽,一头一指流鲨,大量的宙道野生蛊虫。

%13
光阴长河生机勃勃。

%14
自从方源来到这座红莲石岛,已经过去了数日。

%15
这期间,他已是将红莲真意彻底吞食消化,宙道境界飙升暴涨,上升到匪夷所思的超绝高度宙道准无上大宗师!

%16
拥有这样的境界,方源再看光阴长河,就有了许许多多的全新感触。这些感触并不流于肤浅,而是越加深入光阴长河的本质,使得方源越看越是入迷。大道的极致奥妙在他眼前,无声无息地掀开了神秘面纱的一角,使得他开始洞悉这个世界,明白这个世界的运转之理。

%17
虽然他之前,也有着炼道准无上的境界,但当时的感触远没有现在这么深。毕竟方源现在可是面对着天地秘境光阴长河。

%18
准无上宙道境界,带给方源全方位的提升。

%19
比方说光阴飞刃这记神奇的仙道杀招,方源完全有能力洞悉它的本质,从而改变其中的蛊虫,还原出杀招威能。

%20
又比如夏槎的宙道仙蛊和仙道杀招,方源可以进行改良,使得它们更适合自己。

%21
最大的提升还在于宙道仙蛊方。宙道、炼道都是准无上,方源几乎可以改良世间任何一件宙道的仙蛊方,对于宙道仙蛊的炼制把握暴涨。

%22
毫无疑问,红莲真意是价值最大的宝藏。排在它下面的是八转仙蛊悔。

%23
不提这只蛊虫的传奇性质,单凭它本身转数高达八转,就已经价值非凡。

%24
方源得了这只蛊虫,手中掌握的八转蛊虫已经多达五只。分别是态度蛊、慧剑蛊、似水流年蛊、魂兽令、悔蛊。

%25
有了悔蛊,方源搭建悔池就毫无关卡可言了。

%26
当然,难度还是有的,只是最大的难关因为掌握了悔蛊而荡然无存。红莲魔尊当初布置这种真传时,也推算到了方源的情况,因此还准备了大量的宙道仙材。

%27
有悔蛊、还有宙道仙材,方源只需要按部就班,耗费时间,就能建设出悔池。甚至,更优异于悔池的仙蛊屋来。

%28
如此一来,方源就能够大肆炼蛊。一方面,他能够炼制出更多的宙道仙蛊,另一方面,则是将身边的仙蛊转数提升上去。

%29
红莲魔尊留下的这些宙道仙材,转数都极高,最低的已经高达七转,大量的八转仙材琳琅满目。

%30
天庭方面若是看到,保不齐要大吐一口鲜血。他们之前千方百计地干扰方源,阻止方源收购宙道仙材,不惜为此耗费巨大代价。这的确给方源造成了相当大的麻烦,但现在他们的这一举动都沦为了无用之功。

%31
红莲真意、八转仙蛊悔、大量宙道仙材之外,还有仙道杀招。

%32
其中最大的收获,是以春秋蝉为核心的种种仙道杀招。最令方源在意的是春秋必成杀招,此招辅助数只仙蛊,能够令春秋蝉催动起来后,成功率达到百分之百!

%33
单纯用春秋蝉,失败的概率极大,并且还有运气衰落的弊端。但春秋必成这记仙道杀招,虽然没有改变弊端,但却将失败概率压制到零。

%34
如此一来,春秋蝉的实用性得到了极大提升!方源已下定决心,要将这个杀招筹备出来。

%35
几乎所有的仙道杀招,都只是一股讯息,需要方源自行筹备。但有一记宙道杀招,是红莲魔尊特意施展出来,留给方源的。所以历经一百多万年,曾经是九转层次,如今威能已是大减,但仍旧有着效用。

%36
这记宙道杀招名为未来身!

%37
此招依凭光阴长河和红莲石岛,能够将蛊仙未来的某个状态暂时借过来用。

%38
比如说,一个蛊师只有五转,未来他渡劫成功,有六转修为。在未来身杀招的作用下,他现在就能借用自己未来的蛊仙身躯,暂时提升到蛊仙的修为。

%39
此招之玄妙恐怖,令方源也咋舌不已。

%40
但很可惜,此招已经衰减成七转层次,对方源已经无效了。

%41
在红莲魔尊的推算中,方源找寻到这座石岛时,修为就是七转。但红莲魔尊毕竟不是智道仙尊,就算是智道仙尊,也有漏算的时候。尤其是宿命蛊受伤,运道横空出世,世间一切都变得越加难以预测。

%42
“我的修为已是八转,这记未来身杀招,只有留给他身边的下属了。”

%43
“这倒是能弥补他们实力不足的缺陷,能够有资格上场。”

%44
“可惜之前,我损失了不少人手。”

%45
以上便是红莲真传的全部内容,不愧是红莲魔尊布置下来的真传,极其丰厚。但是方源却仍旧有些失望。

%46
他最最期待的东西,是某种方法或者捷径,可以令他取巧地摧毁宿命蛊。

%47
但是红莲魔尊并没有留下。

%48
甚至红莲魔尊留下的一段记忆中,也只是描述了他成就仙尊而后悔的人生经历。在他之后怎样炼出春秋蝉,怎么攻上天庭伤害宿命蛊,都毫无记载。

%49
成尊之后,记忆就一下子跳跃到了红莲魔尊布置红莲石岛上来。这样布置,有什么深意?

%50
方源在吞噬红莲真意之前,就此询问过。

%51
红莲真意给出的回答是,你想要摧毁宿命蛊,没有任何捷径。就算当初红莲魔尊利用过某种捷径,如今也肯定是没有了的。不要小看天庭。任何某种捷径若是存在,就要小心,这应当便是天庭布置下来的陷阱!

%52
至于红莲魔尊为什么不将更多的记忆留下来,红莲真意也没有答案。他只是推测地回答:“不想给方源你任何的指引,因为这极容易导致误会。是否摧毁宿命蛊,都是你自己的选择。”

%53
方源又问:其他的红莲石岛在哪里。

%54
红莲真意也不知道。

%55
不过这个回答,倒在方源意料之中。

%56
若是每一座红莲石岛间,能够相互标记位置,那么早就被幽魂魔尊或者天庭一锅端了去。红莲魔尊就是为了防止这种情况发生,所以特意将每一座红莲石岛都单独布置,相互之间不能感应。

%57
“宿命蛊是绝对要摧毁的!”

%58
有无宿命蛊的天庭,完全是两个概念。若是宿命蛊真的被天庭修复成功,虽然方源不受宿命束缚,但世间其他人和物都却受到影响,甚至操控。

%59
届时,方源对付的敌人,就是除他之外的所有人和物!

%60
当然这话有点夸张,但性质却是绝对严重的。

%61
还有一点更为关键,宿命蛊的存在是方源追求永生的最大妨碍。

%62
这点显而易见,宿命蛊不允许死而复生,规定生者定死。

%63
所以,不管如何,方源也要摧毁了宿命蛊。

%64
恐怕红莲魔尊早已是料准了这一点,所以毫不担忧方源的动机。

\end{this_body}

