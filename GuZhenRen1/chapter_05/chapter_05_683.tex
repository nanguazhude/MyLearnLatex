\newsection{同仇敌忾}    %第六百八十六节:同仇敌忾

\begin{this_body}



%1
“囚谷的南华荆,遭到神秘蛊仙的偷采,损失大半。”

%2
“红河滩地上的仙阵被破坏,里面的赤火石被三位蛊仙联手劫掠一空。”

%3
“有人想要盗取落天河中段的混元沙,被天妒楼特意留下来的一记仙道杀招击退。”

%4
……

%5
紫薇仙子坐镇天庭,纵观中洲全局,一个个消息传来,令她眉头越皱越紧。

%6
她心中气愤。

%7
偌大的中洲,物产丰富,人文荟萃,乃是天庭领袖中洲十大古派,历经沧桑岁月,无数日夜,兢兢业业营造出来。

%8
如今,却是被各方宵小偷袭盗掠,一片乌烟瘴气。

%9
而龙公却仍旧端坐主位,听着紫薇仙子汇报的一条条糟糕消息,面色一片平静。

%10
“无妨,损失些许资源算得了什么?都是癣疥之疾罢了。”龙公说着,甚至微微闭上了双眼,手托着腮帮,开始休憩假寐。

%11
紫薇仙子深呼吸一口气,告诉自己要冷静,但下一刻她双眼猛地瞪大,惊怒出声:“方源已经逃脱,我方三位中洲八转人人负伤,周雄信前辈……已被方源斩杀!”

%12
“嗯?”龙公眼中厉芒一闪即逝,他不再假寐,缓缓坐直身躯,口中沉吟道,“周雄信这四人是我们专门为方源精心布置的,其中周雄信实力最强,居然会被方源所杀。而其他人却只是受伤撤退了,这其中必有内情。详细战报呢?”

%13
“在这里。”紫薇仙子立即递给龙公一只信道蛊虫。

%14
战报非常详细,因为除去周雄信之外,其余的中洲三位八转都还生还着。

%15
看了战报,大殿中一片沉默。

%16
良久,紫薇仙子开口:“我原本以为,方源是依靠落魄印杀死了周雄信前辈,没想到居然是动用的春剪杀招。”

%17
“根据这份战报,春剪杀招再次得到了改良。最重要的是,当方源变化成太古年猴时,一身战力已是突破到了八转巅峰!”

%18
说到这里,紫薇仙子不禁深深叹息一声。

%19
她在很久之前,就知道方源绝对是个威胁。若是放任他自由成长,必定会成为心腹大患。

%20
紫薇仙子心中早有预料,只是她万万没想到,这一天竟然来得这么快!

%21
“方源成为蛊仙,这才多久?似乎转眼间,他就成了八转。又一眨眼的功夫,他竟然已有八转巅峰的战力。即便是我,本身战力也不过是八转高阶罢了。”

%22
一时间,紫薇仙子竟有些心灰意冷。

%23
龙公看了紫薇仙子一眼,声调仍旧低沉缓慢:“紫薇,你不可妄自菲薄。方源崛起迅猛,除了他本身的天赋才情之外,还有各方的助力。他拥有春秋蝉,乃是红莲魔尊属意之人。又得到幽魂真传,继承了影宗宗主之位。他还拥有鬼不觉杀招,那是盗天真传。别忘了,八十八角真阳楼倒塌,方源又获得了巨阳仙尊的多少好处?”

%24
“我们对方源已经足够重视,你针对他的举措并无不对。只是我方屡屡失败,除了方源这个因素之外,还有他背后的种种影响。我们不是单纯地要铲除方源,而是在他身边环绕着的一股股力量做斗争。”

%25
“这些人中不乏仙尊、魔尊,以至于令方源成长到了今天这种地步。”

%26
紫薇仙子闻言面露惭色:“是我乍闻噩耗,心境不稳,多谢龙公大人宽慰。”

%27
龙公淡淡一笑:“我这不是在宽慰你,只是在陈述一个事实罢了。不过,方源虽有八转巅峰战力,但只要不是九转境界,便无力掀翻我天庭大局。周雄信之死,战报中已说得明白。是因为方源忽起仙阵,破坏了流言笼。”

%28
“方源狡诈阴险,故意拖延,使得流言笼因为中洲民意加持变得越来越强,大大超过原本的威能。方源起阵破坏了流言笼后,周雄信遭受到的反噬也变得极强,远超他承受极限,当场就重伤。”

%29
“其余三位八转,虽然第一时间出手相救,但方源早已准备,利用大阵干扰支援。忽然化身上古年猴,战力飙升至八转巅峰,令我方三仙措手不及,悍然击杀周雄信。”

%30
“方源一直都是示敌以弱,又借助流言笼,反过来坑害我方蛊仙。周雄信虽然谨慎戒备,但仍旧着了方源的道儿。”

%31
“其实,若是双方明明白白交手,方源就算是八转巅峰战力,也难以杀死周雄信。随后的战况就可证明这一点,我方三位八转相互扶持,一力撤退,方源也无法留下他们。”

%32
紫薇仙子连连点头,龙公的分析也是她心中所想,她非常赞同:“没有最强的仙蛊,只有最强的蛊仙。方源羽翼丰满,实力强悍,更可怕的是他狡诈阴险,又凶狠毒辣,常常能打出超越寻常的战绩。如今他已经是八转巅峰战力,我们之前的布置已经被他打乱,是不是将厉煌派遣出去,来对付方源呢?”

%33
龙公摇头:“厉煌仍旧要驻守,维护炼蛊大会。修复宿命,是大局中心,期间不得向外抽调任何人手。若是我们抽调厉煌去对付方源,且不说方源拥有定仙游。就算他的挪移手段失效,被厉煌纠缠,这也正是方源乐于见到的。方源要对付我们,破坏宿命蛊的修复,难道其他人就不想吗?眼下不可自乱阵脚。”

%34
“是,龙公大人。”紫薇仙子心境已经平静下来。

%35
龙公虽不是智道蛊仙,但他主持天庭,坐镇中枢,气度超凡,宛若擎天巨柱,稳稳把持局面。

%36
别说方源有了八转巅峰战力,就算他是准仙尊,和龙公一样的身手,又能如何?

%37
龙公相信:凭借天庭的底蕴,必定能确保宿命蛊的修复。方源他就算蹦跶再欢,只要不是九转尊者,就翻不了盘!

%38
龙公最关注的还是炼蛊大会的进展。

%39
他望向大殿中的另外一人,询问道:“正元大人,不知同仇敌忾杀招施展的如何了?”

%40
被询问的是一位老人,他脸上有着许多老年斑,皱纹丛生,浑身干瘦如柴,活脱脱就是一个行将就木的老人。

%41
他坐在一个石凳上,因为弯腰驼背,仿佛整个人都蜷缩起来。他的呼吸很轻很轻,气息之微弱,就算是下一刻死了也不奇怪。

%42
但他辈分比龙公还高,资格很老,修为只有七转,但却是天庭成员。

%43
他便是正元老人。

%44
因为宿命蛊的修复,他从仙墓深处苏醒过来。

%45
一般而言,天庭正式招收的都是八转蛊仙,并且还都是八转中的精英、强者。但正元老人却是七转修为,却得到天庭的承认,被招收进来。

%46
这其中最主要的原因,是正元老人选择修行的流派比较特殊——人道!

%47
有关人道的概念,很早很早就被提出来了。

%48
它的来源,就是无人不知无人不晓的人祖了。

%49
人祖留下的《人祖传》可以说是,古往今来人族历史上,规模最大、立时最久,最为其奇特的传承!

%50
《人祖传》中蕴藏着人道的奥妙,许多惊才艳艳的人物,从中阅览,得到一鳞半甲的成果。往往这些收获,效用绝妙,威能惊人。

%51
天庭一向自认为是人族领袖,人道正统,天庭积累下来的种种人道手段,正是龙公最大的依仗之一。

%52
正元老人气息微弱地答道:“当年,异人十八魔仙围剿元始仙尊,仙尊无人可敌。十八魔仙惨败亏输,却不甘心,用妖言迷惑仙尊。元始仙尊困扰,枯坐在杀阵中三天三夜,思绪纠结,不能自已。阅览《人祖传》后,仙尊看破心中迷障,一连创造出芸芸众生、万众一心、同仇敌忾等诸多人道杀招。”

%53
“老朽不过是一枚钥匙,催动元始仙尊遗留下来的手段而已,当不得‘大人’二字,龙公你直呼我名便好。”

%54
正元老人非常谦逊:“如今,芸芸众生杀招早已催动,同仇敌忾杀招也覆盖了整个中洲,接下来只要人意积蓄到一定程度,便能施展万众一心杀招了。”

%55
“好。”龙公笑了一声,“这就好,有了同仇敌忾,中洲万民同心,都被激起义愤和仇恨,炼蛊大会绝对能进行下去了。”

%56
中洲,法身门。

%57
这个门派乃是风云府太上长老吴发开创,同时也是此届中洲炼蛊大会中的比试场地。

%58
海选之后,能进入这里比试的,都是炼蛊方面的好手。

%59
广场中有数千位蛊师,每人占据一片地方,各自炼蛊。

%60
整个广场都被笼罩着一层厚实的光膜,七转蛊仙强者吴发亲自坐镇,保护着这一切。

%61
时间已经过去一半,不是蛊师都已经失败。

%62
有的人满脸乌黑,有的人则更惨,双手被炸毁,胸膛被腐蚀出血洞……

%63
炼蛊是一个很危险的过程,稍不留意,就会身遭恶难。

%64
又一个蛊师在炼蛊时发生意外,整个人猛地昏倒在地。

%65
铺设在广场地砖下的大阵立即运转,将这个蛊师挪移到场外,几位治疗蛊师迅速围拢过来,紧急救治。

%66
原来这人是处理蛊材时失误,被蛊材中的毒素侵入体内,因此昏死。

%67
很快,他就被救治得当,苏醒过来,捡回来一命。

%68
“你虽然苏醒,但四肢酸软,头脑不清,是不能再参加大比了。”治疗蛊师为他惋惜,这人实力不弱,之前被许多人看好。

%69
这人却笑道:“失败又如何?按照上仙们的话,就算是在炼蛊大会中失败了,也对中洲大局有利!对正道有益!那些魔仙千方百计要来破坏我们的炼蛊大会,杀死多少无辜的性命!我们虽然身为凡人,却绝不能放弃抗争。参加炼蛊大会,让炼蛊大会进行下去,就是对他们最好的复仇!”

%70
他的语气虽然虚弱无力,但是内容却铿锵激越,如刀剑碰撞。

%71
“好汉子!”身旁的治疗蛊师对他竖起大拇指,“我也是这么想的,那群杀人凶杀枉为仙人,毫无怜悯和仁心。我们要抗争到底,绝不能让他们如愿!”

%72
就在这时,场中有了变化。

%73
两位年轻蛊师同时起身,手里托着蛊虫。

%74
其中一人正是洪易,他喊道:“我成功了!”

%75
“好,二位同时成功,共为此场榜首头名。按照我先前所讲,二位下友都可获得我专门留下的小小奖品。”广场中传出蛊仙吴发的声音,顿时引得众人大为艳羡。

%76
两位年轻蛊师同时看了一眼对方,他们之间距离不远不近,一个在广场中央地带,一个在东北角落。

%77
“你们二人通报姓名吧。”吴发又道,他并不现身,但早已详细观察这两位优胜者,越看越是喜欢,两位年轻蛊师都是容貌俊美,修为十分不俗。

%78
“在下洪易。”洪易答道。

%79
“在下叶凡。”另外的年轻人也跟着开口。

%80
ps:最近每一章的字数都有点多,所以更新常常要到20点后面。另外《蛊真人》的繁体出版第一册终于寄到我手里来了,最近会做一个赠书活动,感谢朋友们的支持,后续会有详细通知。

\end{this_body}

