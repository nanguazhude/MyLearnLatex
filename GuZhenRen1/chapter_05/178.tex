\newsection{大甩卖}    %第一百七十八节:大甩卖

\begin{this_body}

方源点头,表示认可。mianhuatang.cc [棉花糖小说网]

这个价格很公允,因为胆识蛊的缘故,抬高了蛊仙魂魄的价值。

自从四族大联盟之后,北部冰原地下的石人一族,对于胆识蛊的需求非常迫切,而且需求量相当的庞大。

但是胆识蛊一直在宝黄天中,有着贩卖的规模。这个规模还不能缩减,毕竟要为长远打算。

目前为止,虽然石人一族和琅琊派关系紧密,但是胆识蛊的大部分产量,都是销售到宝黄天中的。

因为胆识蛊这种垄断的独家贸易,能够换取的不仅仅是仙元石,还能直接换取众多的仙材,以及其他各种资源。

位于资源贫瘠的北部冰原,石人一族能够提供的回报,是满足不了琅琊派的发展需求。

琅琊地灵为了整个门派的利益,是不会将石人一族作为主要的贸易对象。

石人一族对此颇有微词,但也奈何不了琅琊派,十分无奈。

一句话,胆识蛊的产量,远远满足不了市场的需求。尤其是南疆那边,因为正道蛊仙频繁地探索超级梦境,对于宝黄天市场上的胆识蛊索求无度。

甚至,还形成了胆识蛊的黑市。

一些胆识蛊,被人转手卖出更高的价。

琅琊派方面屡次抬高胆识蛊的价格,销售火爆非常,整个市场就是个怪物的血盆大口,琅琊派已经拼尽全力生产胆识蛊,但根本无法满足这头怪物的恐怖胃口。

在方源所有的收益进项中,胆识蛊的贸易利润,已经遥遥领先,成为一枝独秀。

方源能够屡次利用七转仙蛊战斗,青提仙元的储备每隔一段时间,总是能轻松补足,胆识蛊在这方面堪称最大功臣。

琅琊地灵收起这只蛊仙魂魄,他口中道:“人是万物之灵,有了这个蛊仙魂魄,的确能在下一次,极大地提高胆识蛊的产量。可惜,蛊仙魂魄可遇不可求。将来这方面的主要投入,还是在太丘上。方源啊,你有空的话,多去几次太丘。荒兽、荒植的魂魄虽然远远不如蛊仙之魂,但胜在量多,来源稳定。有你出力,太丘方面的开发会进展更快的。”

琅琊地灵很看好方源的能力。

但方源却道:“我最近要前往参加血战武斗大会,恐怕无暇分身。这是第二个蛊仙魂魄,还请太上大长老你验验成色。”

“第二个啊。”琅琊地灵扬了扬眉头,但也没有什么奇怪的。

他知道方源的一些底细,对方源卖出一些蛊仙魂魄,并不让他惊异。[棉花糖小说网www.Mianhuatang.cc想看的书几乎都有啊,比一般的小说网站要稳定很多更新还快,全文字的没有广告。]

“这是一头羽民蛊仙的魂魄呀。”琅琊地灵查看了一下,有些意外。异人被欺压,异人蛊仙特别少见。没想到方源居然手中有一位羽民蛊仙的魂魄。

“这是你杀的?”琅琊地灵随口问道。

方源笑了笑:“不错。”

这个蛊仙魂魄正是羽民郑灵,之前的第一个魂魄则是雪松子。

琅琊派乃是毛民的门派,对于一个陌生羽民蛊仙的魂魄,琅琊地灵决定收下,没有深究下去的意思。

因此,方源再次上涨了数百贡献。

紧接着,他又取出第三个蛊仙魂魄。

琅琊地灵笑了:“看来你今天是打算大抛售。行了,把你想要贡献上来的蛊仙魂魄,都取出来吧。”

“也好。”方源开始取出一个个的蛊仙魂魄。

他手中的蛊仙魂魄,真的不少。

之前,就有东方长凡、雪松子、郑灵等人。后来在黑凡洞天中,杀了一批。最后在乱流海域,收获更丰。

当方源一连取出三个蛊仙魂魄的时候,琅琊地灵笑了笑。

当方源取出六个的时候,琅琊地灵吃惊了,在座位上,下意识地挺直了上半身,前趋过来看。

当方源取出第十个蛊仙魂魄时,琅琊地灵的脸色变了,他手指着方源:“你怎么杀了这么多人?”

方源笑了笑,道:“也是机缘巧合吧。”

接着,他又继续往外掏出蛊仙魂魄。

最终,琅琊地灵看到满满一地的蛊仙魂魄,感到十分震惊。

“我终于明白,为什么你在人族那里混不下去,要投靠我们异人了!”琅琊地灵咋舌不已。

“你这才六转修为啊!”琅琊地灵知道方源的底子,因为见面曾相识就是他给予方源的。

“你升仙才多久?期间还为了获得新生,而奔波辗转。你杀了这么多人,简直就像是幽魂魔尊年轻的时候。”

方源听到幽魂魔尊这个名字,苦笑一声:“这些魂魄都贡献给门派了,除此之外,我还有大量的修行经验,各流派的蛊方、仙蛊方、杀招、仙道杀招。甚至是战场杀招,还有各域的最新地图,方便将来琅琊派的这些同门们外出行走。”

方源来一个大甩卖!

虽然琅琊福地的历史极其悠久,库藏丰富无比,就连天庭都觊觎有加,影宗更是念念不忘。

但方源杀了这么多人,搜魂搜出来的东西,也极其丰富。

琅琊福地的库藏,大多都是当年,长毛老祖为他人炼蛊,得来的报酬。或者是之后琅琊地灵,通过宝黄天,陆续收购而得。

这些库藏中,前者有些久远了,长毛老祖生前可是中古时代,三十万年前的人物。

而后者呢,能在宝黄天中往外贩卖的,可想而知,里面精品是稀少的,虽然也有罕见之物。

至于方源贡献上来的这些东西,很多都是蛊仙安身立命之本,这些东西蛊仙们都是珍惜若命的。比过去的东西先进不说,也不会凭白无故地拿到宝黄天卖去。

对此,琅琊地灵来者不拒。

他和上一任琅琊地灵不同,上一任琅琊地灵只对炼蛊方面感兴趣,而他却是野心勃勃,创建琅琊派,一心想要毛民制霸,重返太古、远古时代的荣光。

琅琊派目前,是以炼道为主,毛民蛊仙中,大多兼修变化道。

琅琊派要成长为一个大派,完成琅琊地灵的野望,必定要兼收并取,拥有各个流派的蛊仙。比如北原的各大黄金家族超级势力,中洲的十大古派等等。

方源的门派贡献蹭蹭暴涨。

另一处云城中。

毛十二正在炼蛊。

“糟糕,有了个失误!”忽然间,他面色大变,操纵的火焰忽然间虚弱下去。

他连忙催动蛊虫,口吐鲜血,喷到火焰之中。

呼。

火焰再次熊熊燃烧,毛十二满脸苍白,身躯颤抖不已。他刚刚受到炼蛊失误导致的反噬伤害,七窍都在缓缓外溢鲜血,将他浑身的毛发,都染红打湿。

他双目圆瞪,充斥血丝,死死地盯着眼前的火焰。

片刻之后,熊熊燃烧的火焰陡然一爆,乍然消失。

炼道蛊阵开启,露出里面上百只的我意蛊。

被上一任琅琊地灵培养出来的毛民蛊仙,虽然战斗力十分弱小,但是普遍炼蛊能力极其强大。

这些我意蛊虽然只是凡蛊,但却也高达五转。毛十二一下子就能炼出上百只,这样的我意蛊出来,显露出了他非凡的炼道造诣。

毛十二暗道一声好险,他心中庆幸不已:“幸好我刚刚自残身躯,弥补了失误。否则的话,这些我意蛊毁灭了,不仅让我数天里为炼蛊付出的时间精力,都打了水漂,而且我还要为此赔偿三倍的蛊材。”

方源为了加大我意蛊的产量,不仅用自己的门派贡献,吸引其他毛民蛊仙为他炼制我意蛊,而且他还财大气粗地自掏腰包,为他们免费提供仙材,让这些毛民蛊仙利用仙材炼制凡蛊。

毛民蛊仙们现在对方源的感觉,不再之前那样鄙夷和反感了。

方源拥有八转战力,让这些毛民蛊仙都很崇敬。暴露出来的这些强大财力,更让毛民蛊仙们感受到了自己和方源的差距,越加敬服。

强烈的眩晕感觉,让毛十二头昏脑涨。

“连续炼了十一次我意蛊,的确对精神消耗太大。刚刚的失误,就是对我敲响的警钟啊。是时候休息了。然后尽快把伤养好,再接取一些我意蛊的炼制任务,赚取更多的仙元石。”

毛十二想到仙元石,顿时满脸苦涩,深深地叹息一声。

“荒兽不好养啊。”

“这些荒兽都是大肚汉,我每个月都要喂养它们,负担好重!”

“更关键的是,奴兽仙蛊只算是门派借给我的,每一次借用,都要按照时间长短,支付门派贡献。”

“为了长远的打算,我肯定是要买下这只奴兽仙蛊的。可是六转奴兽仙蛊,需要数万的门派贡献……”

毛十二掏出一只信道蛊虫。

这只蛊虫唯一的作用,就是时刻显示琅琊派中的各位蛊仙的门派贡献,并以此分出名次。

简单而来,就是琅琊门派贡献榜。

毛十二将心神灌注进去,轻车熟路地在榜单的上半部分,看到了他的名字。

然后在他的名字后面,显示着一个数字。

“六百多的门派贡献,距离目标还是那么遥远啊!”毛十二苦笑一声,感觉长路太漫漫。

“不过,其他毛民蛊仙也和我差不多,甚至比我还要不如。”

“我的优势,在于我兼修了奴道,如今已经有数头荒兽傍身。得到更多的门派贡献,比其他人要容易得多。”(未完待续 \~{}\^{}\~{}。)<!--80txt.com-ouoou-->

\end{this_body}

