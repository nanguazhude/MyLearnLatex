\newsection{黑凡四珍(上)}    %第一百零七节:黑凡四珍(上)

\begin{this_body}

方源将冯军的尸躯收起来.

连同之前的蛊仙尸体,方源收集的已经多达九具!还有蛊仙魂魄,也有相应的九人。<strong>小说txt下载HtTp://Www.80txt.Com/</strong>

“如此收获,的确少见。换做外界的蛊仙,就算死了,恐怕也要自爆。黑凡洞天的蛊仙中,也就冯军一人勉强能看看。此人也是能屈能伸的聪明人,可惜被我占据先手之后,局势就被牢牢掌控住,他根本翻不了身。最后求饶,也是无奈之举。可惜是碰上了我。”

方源收起心中的感慨。

大局已定。

收获的时候到了!

他转过身,重新走进石亭,抬头看向横梁上悬挂着的黄钟天灵,开口道:“如今黑凡洞天中只有我一位蛊仙,我支持我自己,来继承这黑凡真传。”

黄钟天灵静默了一会儿,这才晃荡一声,发出一声动听的钟鸣。

随后,石亭中的石碑开始绽放出纯白的光辉。

白光并不刺眼,很是柔和。

光明照耀中,石碑变得越加透明,化成了一道光壁。

光壁内里,隐约有几个模糊的影子。

而光壁的表面,则浮现出几行字。

方源凝目察看,这几行字阐述的是:石碑中的模糊黑影,便是真传的内容。继承者只需要将手伸入其中,从中拿取即可。

最后一行字,则是黑凡的叮嘱黑家后人,继承传承,望你广大门楣,不堕黑家之名!

方源看完,不由叹息一声。

黑凡老祖布置这道真传,可谓苦心积虑,设想了方方面面。

可惜人死如灯灭,纵然是如此传奇人物,千算万算,也预料不尽身后之事。

曾经强势无比的黑家如今已经覆灭,而黑凡的真传。现在却被方源这样的一个外人夺走。

世事变迁,沧海桑田。兴衰荣辱,过往交替。

方源收拾情怀,开始按照光壁上的描述。开始行动。

他缓缓伸出左手,慢慢探向光壁。

白色的光辉又盛一分,照在他的左臂上,将方源的左臂也照成透明。

方源从外面看,就可清晰地看到自己的皮肉、白骨。还有血管中流淌的鲜红血液。

一股小小的阻力,从前方传来。

方源的左手,碰触到了光壁,感觉就像是触到了一层薄薄的冰壁一样。

同时,他注意到:白色的光芒照在血液上,让他的血液都起了反应。血液微微加速,颜色也从鲜红色忽然变成赤红,但很快又复原如初。

随后左手前方的光壁,忽然变得柔软起来,像是化为水液一般方源的左手。mianhuatang.cc [棉花糖小说网]得以顺利伸到石碑之中,就好像探入水中一般。

方源心中却盘绕着一阵寒意,暗道:“好险!这光壁还是一道测验,刚刚就是测验了我的血液。幸亏我准备充分,不辞辛劳,将血本仙蛊等融汇到见面曾相识当中,改良了这个杀招,使得自己的血液都变化伪装。否则的话,就要功亏一篑了!这黑凡还真是难对付。”

方源平复心境,开始打量眼前的光壁。

从外面就能明显地看出。光壁里有四团阴影,随意分布着。

最大的一团阴影,浑圆如球,有脸盆大小。占据光壁的最中央。

右上角有一团阴影,棱角分明,好似石块。

光壁左边一块,也是阴影,仿佛丝线纠缠。

还有一个点,在最下方。阴影最小,宛若手指头。

方源辛苦了这么久,终于解除到了大名鼎鼎的黑凡真传,但好像真传中包含的东西并不多的样子。

由于站位的原因,方源的左手,正是探进光壁左边,所以他直接伸手,向最近的位置阴影抓去。

他一把抓住这团仿佛丝线纠缠的阴影。

阴影很有质感,一片粗糙,给方源的感觉,像是老树的树枝。

然后方源轻轻抽手,将这团神秘的阴影拉出来。

一出光壁,顿时气息外溢出来。

很明显,这些是蛊虫。

这些蛊虫纠缠在一起,外形上十分相似,仿佛是参须,又如老树根。

都不是仙蛊。

只是凡蛊而已。

但即便如此,方源却是双眼放出精光,一时间连呼吸都变得有些粗重!

自古仙凡有别。

仙蛊唯一,独一份儿。凡蛊比不上仙蛊,但这其中却有一个最大的例外。

那就是寿蛊!

没错,方源抽出来的都是寿蛊。

这在意料之中,但也在意料之外。

之前,陈尺老仙在向方源索要好处时,就提到了寿蛊。所以,方源也有心理预估。但他没有想到的是,这里竟有这么多的寿蛊。

“这些寿蛊似乎有点多!”

方源开始炼化寿蛊。

凡蛊用真元即可炼化。

这些寿蛊虽然不是他的,但方源是通过考验,被认可的真传继承人。

在白色光辉的照耀下,方源很快就顺利地炼化了这些寿蛊。

他这才得到结论:“这些寿蛊加起来,共能增添七百二十的阳寿!”

大收获!

陈尺老仙原本估计,会有三百年寿蛊,但现在方源一看,方知是陈尺老仙估算得少了。

黑凡洞天中真正积累的寿蛊,是其两倍还不止!

“这笔寿蛊价值很大。”

“我可以留给自己用,谁会嫌自己的寿命长?”

“但我本身还有八十多年阳寿,寿命还比较充足。这些寿蛊就算我不用,也可以拿出来交易。”

蛊仙之间的交易,仙元石只是基本货币,寿蛊的价值,要远远大于仙元石。没有蛊仙不需要寿蛊的!很多高端的交易,仙元石已经不顶用了,蛊仙之间只认可寿蛊。

寿蛊是绝对的硬通货!

这么思考的时候,寿蛊已经全部炼化,为方源之物了。

方源将手上满满一把的寿蛊,都收入至尊仙窍,妥善保存起来。

这些寿蛊,他暂时还不打算用。

“其实黑凡洞天的建立。已经有好长一段时间了。积累这么多的寿蛊,细想起来,并不是什么奇怪的事情。”

“我的至尊仙窍,什么时候。能产出寿蛊呢?”

方源有些神往。

但他也知道,自己距离这个目标,还差得很远。

他现在虽然日进斗金,每月盈利很多,但那都是之前的积累。经营仙窍。还在基础建设阶段。只有当他将资源都建设起来,满足长期喂养手中全部仙蛊之后,才算是基础建设阶段完成了。

方源忽然又想到了琅琊福地。

“比起琅琊福地,黑凡洞天存在的时间,就不值一提了。长毛老祖可是三十万年前,中古时代的传奇人物!”

“黑凡洞天中的寿蛊,就有七百多年。琅琊福地中积累的寿蛊,能有多少?”

这么一想,方源眼中精芒直射。

“琅琊福地中的寿蛊,肯定是极多的。难怪琅琊地灵可以拿出一些。用来喂养智慧蛊了!”

方源忽然间很理解天庭了。

在他五百年前世,天庭进攻琅琊福地,哪怕是牺牲了凤九歌,也在所不惜。

恐怕寿蛊就是其中的主要原因。

想想看,天庭中的那帮老不死的,都是寿命短缺的人物,终日里只能是通过沉眠的方式,来苟延残喘。可想而知,这些寿蛊对于他们的吸引力,会强大到何种程度!

方源的目光。又转向光壁,停在正中央。

这里的一块阴影,体型最大。

“而且是在正中央,这是否预示着。这个便是黑凡真传中最贵重的珍宝?”方源怦然心动。

他将左手伸向光壁中央,很快,他就触碰到了那团阴影。

感觉很冰凉,但阴影的表面并不光滑,凹凸有致,像是大团的芝麻。

方源试着拿取。

很重!

依照常人的力量。根本拿不起来。

方源便催动力道凡蛊,增添力量之后,这才单手将它拿了出来。

见到这团阴影的真面目之后,方源的脸上微现诧异之色。

这不是一个整体,准确的来说,这是无数个体的结合。

这是一团蚁球。

黑色的蚁球,有脸盆大小,十分沉重。无数的黑色蚂蚁,相互抱结成团,密密麻麻,十分紧实。

这些蚂蚁当然不是普通的蚂蚁,都是凡蛊。

前一个是凡蛊,也就算了,因为是寿蛊,在很多蛊仙的眼中,价值比仙蛊还要大些。

但这个还是凡蛊?

方源纳闷,同时他还有些奇怪。

因为凭他的眼力劲,竟然也认不出来这究竟是什么蛊虫。

不管这些,方源觉得,毕竟是黑凡真传中的内容,就算是凡蛊,也差不到哪里去。

索性,他直接开始炼化。

凡蛊用真元炼化即可,方源乃是蛊仙,真元堪称无限。

一层层的黑色蚂蚁被他炼尘,化为己用。

在方源的意念调动之下,蚂蚁群分散开来,渐渐露出了蚁球中心。

“嗯?”方源忽然间神色微变,目光顿时变得犀利起来。因为他感受到了仙蛊的气息。

原来这蚁球中央,还藏有好货!有一只仙蛊!

普通的蚂蚁凡蛊终于被方源炼化干净,全部散落在地上。方源左手中央,就只剩下了一只蛊虫。

仙蛊!

此蛊比起其他黑蚁凡蛊而言,体格庞大了数倍。也是蚂蚁形状,只是触角短,胸足小,腹部极大,给人肥嘟嘟的感觉。

“这莫非就是蚁群中的蚁后?”方源胡乱猜测。

这只仙蛊气息洋溢,转数还不低,乃是七转仙蛊。

只是究竟是什么玩意,方源也不清楚。

他同样认不出来。

不管了,先炼化再说。(未完待续。)<!--80txt.com-ouoou-->

\end{this_body}

