\newsection{乐土传人}    %第五百三十节:乐土传人

\begin{this_body}



%1
即便是凤九歌,面对智慧蛊,也不由地心驰神摇。

%2
这可是智慧蛊!

%3
《人祖传》中记载,曾经是星宿仙尊的本命仙蛊,其名无人不知无人不晓,没有想到居然在今天,在这样的环境下,突兀地出现在凤九歌的眼前!

%4
“琅琊福地中竟然藏有智慧蛊!”凤九歌心中剧烈震荡,一时间不能自已。

%5
不过旋即,他强自冷静下来。

%6
他眼中精芒暴涨,仔细端详眼前的智慧蛊,他吃惊地发现这只九转智慧蛊赫然是野生仙蛊!

%7
一个念头在瞬间,霸占凤九歌的脑海,那就是把这只九转智慧蛊带走!

%8
仙道杀招得宝歌。

%9
他催动杀招,力度前所未有,迸发出八转级数的威能。

%10
得宝歌能够帮助凤九歌,直接炼化野生仙蛊,歌声亲切温暖,充满了祥和气息。

%11
在歌声中,原本在树枝丛中跳跃飞舞的智慧蛊,安静了下来。

%12
“九转智慧蛊?!”操纵超级仙阵的异人蛊仙也震惊了。

%13
智慧蛊原本隐藏在这片山谷之中,被琅琊地灵隐藏,三族的异人蛊仙们终究是外人,灭有发现这一处地方。

%14
但是现在,这片地方被凤九歌破坏,智慧蛊暴露出来,令三族异人蛊仙都察觉到。

%15
“绝不能让他收服了智慧蛊!快动手,干扰他!”琅琊地灵急得大吼,连忙下令。

%16
随后,他又补充道:“但是你们要注意,千万不要毁了智慧蛊啊!”

%17
不需要他提醒,三族异人蛊仙谁不知道这只九转仙蛊的重大价值?

%18
但是此时此刻,要干扰凤九歌,又不能牵连到近在咫尺的智慧蛊,把握其中分寸,对于刚刚熟悉仙阵的这些异人蛊仙,就相当有难度了。

%19
异人蛊仙们只好硬着头皮,操纵仙阵,迸发种种攻势,袭向凤九歌。

%20
凤九歌低喝一声,催动出一曲之士杀招。

%21
得宝曲士飞向智慧蛊,而其余曲士则帮助凤九歌,抵挡或者拦截仙阵的攻潮。

%22
异人蛊仙们有些出手重了,攻势危急智慧蛊,凤九歌还主动抵挡,维护智慧蛊。

%23
智慧蛊察觉到危险,求生的本能占据上风,飞身逃跑。

%24
得宝曲士在后面紧追不舍,却没有丝毫效果。

%25
凤九歌见此,一颗心顿时沉入谷底。智慧蛊高达九转,他的得宝歌虽然厉害,但终究只是八转而已,难以对智慧蛊有效果。

%26
“得不到手,那就毁掉它?”这个念头刚刚从凤九歌的心头升腾起来,就旋即被他否决。

%27
这可是九转智慧蛊,就这样摧毁掉它,未免太可惜了。

%28
而且,这只仙蛊曾经乃是星宿仙尊之物,若是能得到手,对于凤九歌、中洲十大古派,乃至天庭,都有极其巨大的益处。

%29
凤九歌不甘心,他不想就这样放弃。

%30
“别走。”凤九歌飞身扑向智慧蛊。

%31
智慧蛊速度很快,但凤九歌紧追不舍。

%32
智慧蛊感受到凤九歌的威胁,忽然间嗡的一声,迸发出一股强烈的光晕。

%33
智慧光晕。

%34
凤九歌不可避免地笼罩在智慧光晕之中,一瞬间,他脑海中无数念头此起彼伏,智慧升华,目光变得犀利至极。

%35
以往修行的种种难题、疑惑,都在这一瞬间,得到大量的解答,虽然不至于立即得到答案,但是许多关隘和瓶颈,就这样轻轻松松地突破了过去。

%36
“糟糕!”凤九歌终究是传奇人物,沉浸在智慧光晕中几个呼吸后,猛地意识到不妥。他咬破舌尖,强迫自己清醒,然后身形爆退。

%37
脱离了智慧光晕之后,他心有余悸地望着智慧蛊渐渐飞走。

%38
“好厉害的智慧蛊,短短功夫,我的寿命就削减了这么多!”凤九歌的额前,一缕青丝变作了白发。

%39
“我此次来,俘虏古月方正,收取荡魂山,都是有着紫薇大人特意安排下来的手段。但是万万没料到,这里居然藏有九转智慧蛊!我已经竭尽所能,但单凭我一个人,根本无法将此蛊带走。琅琊福地方面恐怕也没法炼化这只智慧蛊,否则的话,他们早就对它下手了!”

%40
凤九歌脑海中思绪如电。

%41
仙蛊唯一。

%42
一只野生的仙蛊,对于任何一个蛊仙,吸引力都是十分巨大的。

%43
更何况摆在凤九歌眼前的,是一只高达九转的野生仙蛊。

%44
其中,更加关键的一点是,这只仙蛊居然就是智慧蛊!

%45
“若是智慧蛊被他们炼化,抢夺它是几乎不可能的事情。但它现在还只是野生的状态啊。我必须回去,将这个重大的情报,禀告门派和天庭。下一次进攻琅琊福地,一定要将此蛊夺到手!”

%46
凤九歌深呼吸一口气,恋恋不舍地望了视野中,已经缩成芝麻一点的智慧蛊,然后他催动仙道杀招。

%47
这记以七转定仙游为核心的仙道杀招,名为“人来人往”,乃是宇道大能梁凉开创。凤九歌并未演练纯熟,但这一招却有不同寻常的奥妙。

%48
一旦催动起来,不仅能送人走,去往天底下任何一处地点,而且还能再送回来。前半部分最为艰难,后半部分却是简易至极。

%49
梁凉意志已经将此招催动成功,凤九歌此时只是顺势而为,完成后半部分而已。

%50
刷。

%51
一声轻响,凤九歌便消失在了原地。

%52
几个呼吸之后,琅琊地灵驱动银色巨人,赶到此处。

%53
“可恨,凤九歌你居然跑了!中洲十大古派,我绝不会饶过你们!”琅琊地灵望着满目苍夷的琅琊福地,怒不可遏,仰头咆哮。

%54
南疆,仙道战场杀招之中。

%55
陆畏因深深地叹了一口气。

%56
三生三世已过,方源并没有被他渡化感召。

%57
“我错了!”

%58
“我原以为这方源,乃是心中无情无义,或者充斥憎恨愤怒,才有危害苍生的行为。”

%59
“但实际上,他有着丰富的感情,无一缺失。虽然追求力量,但却毫不偏执。”

%60
“真正促使他这样做的,是他的追求。这种追求,已经深深地根植在他内心最深处,不要说三生三世,就是百生百世,也不能摧毁。”

%61
“这种人是根本无法渡化的啊。”

%62
“师傅……”叶凡轻声呢喃。

%63
这个时候,不只是他,他身边的商心慈等人,还有铁面神等蛊仙,也纷纷睁开双眼,苏醒过来。

%64
三生三世中一幕幕,让大多数人都神情愣愣,都在回味当中的玄妙感受。

%65
陆畏因目光扫视一圈,在商心慈的身上微微停顿了一下。

%66
商心慈眼眶泛红,险些留下泪来,悲伤之情溢于言表。她在三生三世中和方源都有感情纠葛。

%67
陆畏因暗暗点头,对众人道:“我失败了,此魔头不可遏制,即将打破杀招,脱困而出,我们快走。”

%68
叶凡相当吃惊:“师父,我们明明还很有优势啊。”

%69
陆畏因苦笑:“为师其实早已经尽了全力。都快走吧。”

%70
凡人蛊师们都很疑惑,铁面神以及翼家两位蛊仙,却是心生猜测。

%71
铁面神直接道:“前辈姓陆,又擅长土道,莫非前辈便是乐土真传的传人吗?”

%72
陆畏因点头:“不错,正是陆某人。”

%73
“果然是这样。”翼家蛊仙兴叹,重新行礼,参拜陆畏因。

%74
同时,他们对陆畏因的话,再没有什么疑虑。

%75
乐土仙尊没有一招杀伐手段,他留下的乐土真传也是如此。陆畏因擅长困敌,感化敌人,这正是乐土真传最为鲜明的特色。

%76
纵观十大尊者,乐土仙尊性情最为仁厚慈悲,是他一手将幽魂魔尊对天下的伤害抚平,拨乱反正。

%77
但不知为什么,这样的仙尊居然没有选择天庭,最终他留在了南疆,并将乐土真传也留在这片土地上。

%78
ps:跨年喽,祝亲爱的朋友们元旦快乐!《蛊真人》又陪伴大家走了一年,期间有不少的波折,生活上的事情总是充满意外,不尽人意。挫折和失败,总会以一种令人意外的方式猛地到来,打得我们猝不及防。但这就是人生啊。而当我们克服困难,忍受失败和苦痛,继续坚持走下去,一定能收获成功和荣耀。2017,让方源和我们一起克服困难和挫折,迈向成功吧。加油!

\end{this_body}

