\newsection{如此牺牲为那般}    %第八百二十八节:如此牺牲为那般

\begin{this_body}



%1
吴帅晋升八转,终于突围,压力骤降。不仅如此,他还找寻到了战胜天庭的唯一希望,那就是借助红莲的力量。

%2
红莲虽然并未成尊,但是却早已注定成为未来的尊者。这虽然只是一个猜测,但蛊仙界的很多人都倾向于这个猜测。若非如此,为何龙公以及天庭都要千方百计地挽留他?当初龙公降临,主动收徒就是一个明显至极的信号。

%3
和古凉沟通之后,吴帅决定改变策略,他偃旗息鼓,再不像之前那般作为。

%4
原本他晋升八转,龙人一族万分期待,终于有了一位可以为他们撑腰的八转大能了。

%5
然而吴帅却开始亲近天庭以及中洲十大古派,这让无数的龙人不理解。

%6
吴帅做出大量的努力,一副幡然悔悟的模样,渐渐赢得十大古派的好感和信任。他刻意接近天庭,千方百计地打探天庭的情报。

%7
甚至,他还一度出资,为天庭打造仙蛊屋。

%8
龙人和人族之间的矛盾,他也改变立场,打压龙人,维护人族利益。

%9
近百年过去,吴帅得到许多成功,但始终没有令天庭信任。

%10
“这该怎么办呢?”吴帅苦恼,和古凉商量。

%11
古凉道:“你是龙人,天庭向来对异族抱有绝对的警惕,天庭从骨子里就看不起异人。我们唯一的希望还在于龙公。该下一剂猛药了。”

%12
“那如何取信龙公呢?老祖宗对我可是……”吴帅摇头苦笑。

%13
古凉深深地看着吴帅:“就看仙友能否做出牺牲了。”

%14
吴帅毫不犹豫,当即答道:“我为龙人一族谋算,些许的个人牺牲算得了什么。”

%15
“那就好。”古凉拍掌,“也唯有你这样的人,才能成就大业啊。”

%16
古凉便献计,吴帅顿时犹豫起来。皆因古凉却是叫吴帅主动牺牲黄维!

%17
黄维已经是七转的龙人蛊仙,不仅在少年时代就开始追随吴帅,而且他一直跟随着吴帅,苦心经营南华岛,对吴帅忠心耿耿,劳苦功高,绝没有任何一点背叛的可能。

%18
古凉劝说道:“世人皆知黄维和仙友你之间的关系。我们可以设计,让你做出抉择。你若维护人族利益,牺牲了黄维,必定能教龙公改变过往的观念!”

%19
“但是……你是要让我屠杀功臣,并且还是最信任我,最崇敬我的人……”吴帅咬牙,仍旧犹豫不决。

%20
“我相信,黄维若知道自己牺牲的意义,他一定会选择牺牲的。不是吗?”古凉一句话大大打消了吴帅的犹豫。

%21
吴帅叹息一声,眼眶泛红:“黄维知我,我知黄维,若他明白我的良苦用心,他一定甘愿牺牲。”

%22
“不可!”古凉却连忙摆手,“此计关乎甚大,越少人知道越好。仅限你我是最为稳妥的。仙友啊,你要明白,就算是死人也能探查出大把的证据。我们万万不能让黄维知道内幕。”

%23
“这……”吴帅闭上双眼,摇头不已,“让我想想,让我想想……”

%24
他犹豫了大半年,终于还是下定过来决心。

%25
和黄维一个人比起来,龙人一族的利益才是更大的。

%26
吴帅和古凉便暗中设计,制造矛盾,斩杀人族蛊仙,制造确凿证据,诬陷黄维。

%27
事发后,立即引起中洲广泛关注。

%28
黄维大叫冤枉:“这根本不是我做的,兄长知我,我绝不会如此干!”

%29
吴帅一开始力保黄维,但随着陆续有更多的证据披露出来,他不得不“大义灭亲”,痛斩黄维。

%30
此事引发轩然大波,龙人对吴帅的意见非常大,开始有龙人一族主动从南华岛外迁。

%31
中洲十大古派对于吴帅普遍有了赞誉,就连龙公知情之后,也颔首肯定吴帅:“小八这些年来醒悟良多,令人欣慰。”

%32
吴帅由此成功取信了龙公。

%33
不过,他的心中却是空了一块。

%34
黄维在行斩之前的遗言,深深地印刻在他的心底,让他一辈子都无法忘怀

%35
“兄长,我黄维从不后悔跟随您!您要斩我,不管什么理由,我都认!但是兄长啊,我真的没有做过这样的事情。您要小心,千万要小心,那些人族蛊仙麻痹了您啊。这些年来他们对你没有动手,实际上是改变了策略,他们现在诬陷我,就是想除掉我,逐步剪除你的羽翼啊。”

%36
“兄长,你向我叙说的龙人一族的未来大业,我一直牢记在心。我死了不要紧,还请您一如既往地做下去。龙人一族的未来,缺少不了您啊!”

%37
黄维拜倒在地上,哭诉着,一声声令吴帅心头滴血。

%38
吴帅几乎要忍不住,告知黄维一切的秘密。陷害他的正是他最敬爱的兄长!

%39
甚至有那么一刻,吴帅心中有一种强烈的冲动,他想跪倒在地祈求黄维的原谅,但他终究没有这么做。

%40
他冷哼一声:“黄维,你贪赃枉法,故意杀人,死到临头还不认罪。唉,枉我多年来这么信任你。你真的是令我太伤心,太失望了。”

%41
话只是说了一半,吴帅已经落下泪来,声音哽咽。

%42
他再不敢看黄维,转身就走。

%43
“兄长!”黄维跪倒在牢房冰冷的地上,看着吴帅的背影大叫,“就让我死之前,再叫您一声兄长!!别忘了龙人的大业啊。”

%44
吴帅脚步更快,仿佛落荒而逃。

%45
黄维行刑,大量的蛊仙前来观看。吴帅没有去。

%46
当天夜里,他做了一个梦。

%47
在梦中,他见到黄维。黄维指着他的鼻子,对他大声咒骂。

%48
吴帅便告诉黄维实情。

%49
黄维听完之后,大哭起来,拜倒在地上,说:“兄长,你良苦用心,是小弟无知愚昧,错怪了你啊。你为了龙人一族的大业,不惜背负骂名,你这样牺牲,我也愿意牺牲。”

%50
吴帅连忙将黄维拉起来,搂住他的肩膀:“贤弟,你能知我就好!”

%51
梦醒后,吴帅带着微笑。

%52
他并没有直接起床,而是睁着双眼,看着帐幔。

%53
他嘴角的微笑未减,但眼眶中却是淌下泪来。

%54
黄维之死有许多影响。南华岛人心散乱,士气跌至谷底,龙人普遍对吴帅十分失望。

%55
不过吴帅却是能够偶尔出入天庭了。

%56
这是巨大的突破。

%57
一次,天庭决定开始搭建东天门。

%58
吴帅抓住这次千载难逢的良机,向龙公汇报,自愿将龙宫贡献出来。

%59
这么多年来,龙宫仍旧是一座七转仙蛊屋。

%60
吴帅此举经过他的深思熟虑:损失一座龙宫,无伤大雅,将来还可以重建。最关键的还是宿命蛊!

%61
龙公的反应,也没有出乎吴帅的估计。龙公道:“小八,你有这份心很好,令爷爷我心中十分宽慰。但中天门乃是信道仙蛊屋,你的龙宫却是奴道,并不相关。况且我天庭如此底蕴,还不需要压榨子孙的财物。”

%62
吴帅叹息:“爷爷,我早年时候年轻气盛,做了许多荒唐的事情。如今悔悟过来,便想着怎样去弥补自己的过错。还请爷爷给我这样的机会,我绝不会令您老失望!”

%63
龙公哈哈一笑:“那就这样吧,中天门就由你和你的父亲来负责督造。”

%64
吴帅大喜过望:“孙儿拜谢爷爷。”

%65
要筹建东天门,需要的资源很多,古凉主动前来资助。

%66
他对吴帅道:“天庭果然狼子野心,居然想要开设东天门,直接连通洞天。不过这也真的是一个千载难逢的机会!我们完全可以在这东天门下手,留下暗门,不断刺探天庭的内部情报。”

%67
吴帅加倍努力,为东天门的筹建而四处奔走。

%68
龙人一族对他意见更大,很多龙人开始斥责吴帅,说他是龙人一族的叛徒。

%69
这些龙人之前对吴帅有多么大的期待,此刻就有多么浓烈的失望和痛恨。

%70
甚至吴帅的父亲,也对他非常冷淡。

%71
吴帅忍受着这些误解,竭尽心智要完成东天门。

%72
在东天门中留下暗门可不容易,尤其是要在天庭的监控之下。

%73
就在他万分努力,将有一些小成果的时候,意外发生了。

%74
他的父亲从天庭回来,重伤垂死,临死之前叫来吴帅。

%75
“我接触到了宿命蛊,我儿!我得到了宿命蛊的启示,我们龙人当兴,龙人当兴啊!!”吴帅之父非常的激动,死死地抓住吴帅的手。

%76
“什么?!”吴帅大为震撼,一时间瞠目结舌,“这究竟是怎么一回事?父亲!”

%77
他的父亲便详细述说了经过:“我也是机缘巧合,虽然最后我被发现,但幸好我机智,自己故意搭建东天门失败,因此反噬重伤,借助这个理由,立即出逃天庭,来到你这里了。”

%78
“龙人当兴!我儿,我们龙人一族本来就当是天命所归,要取代人族的啊!这是宿命蛊的启示,绝对可信!!”吴帅之父双目放光,“你知道这意味着什么吗?快快治好我,将这个消息公布出去。龙人一族的蛊仙们不知道会兴奋成什么样子呢!”

%79
“父亲,你安心修养,我这就去办。”吴帅也非常激动,连忙安置好他的父亲后,走出房门。

%80
没有多远,他就撞见了古凉。

%81
古凉得知吴帅之父重伤归来,便前来拜见慰问,见到吴帅面色犹豫挣扎,便询问缘由。

%82
吴帅和古凉密切合作,已经得知古凉的真正身份,他并非人族,而是一位异人蛊仙!

%83
吴帅沉默半晌,终究还是将实情托出。

%84
古凉亦同样震惊,反应过来后,他顿时明白吴帅的犹豫是什么。

%85
“吴帅仙友,你没有冒然召集龙人,公布这个消息是正确的。一旦天庭知道这个秘密泄露,必然会对龙人一族展开杀戮和清洗!”

%86
“但是,我们要继续隐瞒天庭,让他们仍然认为这个秘密无人知晓,单凭这一点还是不够的。”

%87
“你的父亲举止太过异常,接触过宿命蛊这个事实,也是有迹可循。接下来必定会有天庭成员前来探查情报。”

%88
“古凉仙友知我,我所虑正是如此!”吴帅叹息,“当下该如何是好呢?”

%89
古凉犹豫了一下,还是开了口:“仙友,其实这个想法你不是已经想到了吗?当下之计,只有将你父牺牲,做出一副他刚刚回来,就重伤死亡的假象,我们才能逃过这场灭族之危啊!”

%90
吴帅面色惨白,蹬蹬倒退两步,摇头不止:“你是要让我弑父?!这绝不可行!”

%91
“不可行?若是天庭蛊仙探知了消息,你们龙人一族极可能全部屠诛啊。”古凉叹息,“我现在总算明白了,为何你们的老祖宗龙公会对你们这一族是如此的态度。宿命蛊中的龙人当兴的讯息,恐怕天庭早已经知晓了。”

\end{this_body}

