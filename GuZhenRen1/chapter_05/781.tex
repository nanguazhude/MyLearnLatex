\newsection{杂念丛生}    %第七百八十四节:杂念丛生

\begin{this_body}

%1
太古魂兽可是八转战力,并且同时出现四头。

%2
一时间,万家三仙的心中都升腾起一股荒谬之感。

%3
方源命四大太古荒兽拱卫自身。他一身黑袍,背负双手,站在其中一头太古魂兽的背上,不屑地道:“就凭你们三个,还想来算计我?”

%4
万良翰眼角抽搐。

%5
万逍脸色铁青。

%6
万豪光的心更是狠狠一沉。

%7
算不尽的实力,大大超出了他们的料想,他们纷纷意识到:算不尽或许也猜到这是一个陷阱,但他还是来了。因为有这样的底牌,任凭谁都会自信!

%8
万家三仙已是骑虎难下。

%9
“稳住!他到底是智道蛊仙,操纵四头太古魂兽,一定十分勉强。”

%10
“不错,说不定这四头太古魂兽只是虚有其表。”

%11
“就算是真的,我倒要看看智道和魂道是如何配合的!”

%12
三仙稳住士气,再向方源攻来。

%13
方源并不进攻,纯粹防守。

%14
因此从场面上来看,万家三仙占据了上风。

%15
那万逍主持仙道战场,同时又催发杀招,一道道风枪投下,几乎每一击都能重创乃至杀死一位上古魂兽。

%16
万豪光浑身笼罩着厚实的光铠,在魂兽大军中横冲直撞,屠戮无算。

%17
智道蛊仙万良翰则指挥战斗,同时自己也出手,不断推算魂兽大军中方源的真正位置。

%18
久战之下,万家三仙感觉越来越难受。因为他们杀得多,而方源放出来的魂兽更多。

%19
这些魂兽团结一起,将方源从四面八方,头顶脚下都守卫住,组成一道道的严密的防线。

%20
这种情况下,魂兽大军如此密集,万家三仙最好的针对之法,就是用大范围的杀招轰袭。

%21
然而每一次有这样的招数,方源就出手破坏。即便是有漏网之鱼,也无伤大局。

%22
大范围的杀招往往无效,万家三仙只好更多地运用针对单独个体的杀招,袭杀一只只的魂兽。

%23
他们采取斩首战术,企图打开一条通路,斩杀方源。

%24
万家三仙配合默契,每一个人战斗都相当高超,他们合作无间,三仙相互助益,战力非常强大。换做七转巅峰的强者,也要处于困境当中。

%25
遗憾的是,他们碰到的是方源这个怪物。

%26
四大太古魂兽组成最核心的防线,几番交手,这道防线的强大就让万家三仙的希望越发渺茫起来。

%27
“可恶,机会太少了啊。”万豪光不满地低吼,他刚刚找到了一丝缝隙,可以向最中央的防线突进,但随即许多上古魂兽就填补上来,将这个缝隙填满。

%28
“这里的魂兽太过密集了。算不尽居然能指挥这么强的魂兽大军,并且更可恶的是,这样的实力明明可以尝试进攻,来破坏我等的仙道战场,他偏偏选择严防死守!”万逍嘟囔,面色十分阴沉。

%29
万良翰的脸色同样也很难看:“是我小瞧了这个算不尽。他若是选择进攻,我们会有更多的机会。但他偏偏选择防守,恐怕此刻房家的援兵已经在路上了。”

%30
“怎么办?”万逍不禁看向万良翰。

%31
对他们而言,眼下这个局面很是尴尬。

%32
原本的计划,是他们三仙合力,将算不尽迅速拿下。哪怕算不尽是七转巅峰战力,这个计划也有极大的成功可能。

%33
然而,当方源一下子掏出四头太古魂兽的那一刻,万家三仙的计划就动摇了。

%34
当方源选择严防死守的战术,万家的这个计划已是化为泡影。

%35
身为智道蛊仙的万良翰,虽然境界上没有房睇长那么高,但也非常敏锐,此刻已心生退意。

%36
然而,此刻要主动退,可不容易。

%37
仙道战场铺设起来,若要主动撤销,必定会留下大大的破绽。算不尽还是智道蛊仙,怎么可能放弃这样的战机?

%38
况且,现在的局面,并非完全丧失了机会。万家三仙当然还有底牌未出。

%39
万良翰罕见的犹豫不决起来。

%40
方源从容淡定得很。

%41
他的真正实力若是暴露出来,恐怕要把这万家三仙吓得当场求饶。

%42
不过,他并不想真正出手。

%43
万良翰肯定是要死的,就是这位七转智道屡屡推算他,让他不胜其烦。

%44
至于其余两人,方源倒更愿意看到他们活着。

%45
当初伪装算不尽的时候,方源就特意为了这个身份,准备了一套智道的仙蛊和手段。如今这套手段更是丰富起来。

%46
方源暗中催起仙蛊杂念,酝酿出一记仙道杀招来。

%47
万豪光的视野中,充斥着各种魂兽。

%48
在外面看,魂兽大军仿佛是滔滔墨水,而万豪光浑厚坚实的白光铠甲宛若一个光点,淹没在了漆黑的墨水之中。

%49
万豪光身陷重围,但却没有丝毫的胆怯和躲闪。

%50
向前,向前,向前!

%51
他像是一柄利剑,刺穿紧密团结在一起的魂兽大军,最终要刺在方源的身上。然而,他屡屡突进都无功而返,绝大多数的情况是在中途就丧失了冲劲,鲜有的几次已经快要成功,结果被四头太古魂兽挡下。

%52
“前方有六头上古魂兽,体格魁梧,排成了两排。”

%53
“后方有两头上古魂兽,行动十分迅捷,一直在试图纠缠我。”

%54
“我不能停,一旦停下,被它们缠住,此次冲锋就要宣告失败了。”

%55
“但我也不能直接冲上去,前面的上古魂兽显然是算不尽刻意操纵的城墙,十分厚实。”

%56
“唯有改道了。”

%57
“不错!左前方的上古魂兽仿佛螳螂模样,这种外形定是攻势凶猛,但防御脆弱。”

%58
“就冲这一点!!”

%59
万豪光虽是光道蛊仙,但拥有一记奇妙的光道杀招,能够让他的脑海中产生明光念头。

%60
用这种念头来思考,速度极快。

%61
正是因为有它在,所以万豪光总是能在瞬息之间,思考清楚,在万军丛中选择出最正确的冲锋路径。

%62
万豪光冲杀过去,虽然挨了螳螂外形的上古魂兽一下重击,但却让他逃出了前后夹击的窘境。

%63
“哈哈,成了!”

%64
“我还能继续冲锋!”

%65
“现在这种情况,再没有撤退之前,就还有希望。”

%66
“我是主攻点,我必须在正面带给算不尽巨大的压力。再配合万逍、万良翰二位的手段,创造出战机,袭杀算不尽。”

%67
“其实,刚刚很凶险,若不是恰好有一头防御薄弱的螳螂魂兽,我还不知道怎么办呢。”

%68
“若是算不尽专修魂道或者奴道,说不定刚刚已经没有希望了。但算不尽是智道,目前根本没有展现出配合魂兽的杀招,他应该是没有这种手段的。”

%69
“不过,单单他能操纵四头太古魂兽,就相当强悍了,这是妥妥的七转巅峰的强者!若是他有着配合魂兽的手段,恐怕和八转蛊仙都能对抗几个回合。”

%70
“若是能斩杀掉这样的对手,我回归家族必然是大功一件。如此赫赫战绩,今后遇到七巧姑娘,也能吹嘘一番……”

%71
“等等!现在是战斗时刻,我怎能分心,想到儿女私情!”

%72
“说起来,七巧姑娘真的是可爱动人。她虽不是那种动人心魄的美艳,但却贤惠懂事,最是理想的仙侣人选了。”

%73
“只是她为什么总不接受我的爱意?或许,我应该转变追求的方式……有时候送一两个小礼物,也未必不可以。”

%74
“这种礼物太贵重,就失去了情趣。嗯……我用草编出一个玩具娃娃,送给她,看看能否逗她开心?”

%75
“这是个好办法!”

%76
“除了草娃娃,我还可以编出草兔子、草帽子、草螳螂呢。”

%77
“说起来刚刚突破的那头螳螂魂兽,外形倒是酷似我儿童时代常常捕捉到的螳螂。”

%78
……

%79
砰!

%80
一声巨响,万豪光被魂兽合围。

%81
从前后左右的重击,几乎同时轰在他的光铠上。

%82
万豪光思维胡乱运转,再不能专注于战场中,使得他遭此狠手,光铠剧烈形变,他整个人也懵了,头晕眼花。

%83
在高空中,万逍见到万豪光遭遇危机,连忙想要催动仙道杀招为他解围。

%84
但这个时候,万良翰却突然喊道:“不要!”

%85
万逍的反应,不知道为什么比平常要慢个半拍,他已是催发了仙道杀招。

%86
下一刻,杀招催动失败,狂风逆袭,万逍的手被割得血肉模糊。

%87
他本人更是如遭电击,鼻腔喷血,脸色煞白。

%88
“哦……到底是智道蛊仙,终于反应过来了。”方源在魂兽大军的重重保护之下,淡淡微笑。

%89
他刚刚催发的杀招名为杂念重生,能够让一定范围内的敌人脑海中不断产生杂念。

%90
起初的时候,杀招效果不太明显。但拖延得越久,敌人脑海中的杂念就越多,最终干扰他们的正常思维。

%91
万豪光被杂念影响,又距离方源最近,于是在战场中分神,想到了其他东西,惨遭合围。

%92
万逍被此招干扰,在催动仙道杀招的关键时刻,忽然有杂念干扰,杀招催动失败,受伤不轻。

%93
万良翰毕竟是智道蛊仙,察觉到了不妥之处,他终于下定决心:“这算不尽好生阴险,竟然有这样的诡异路数。我们毕竟不晓得他的底细,没有防备,这才吃了一个大亏。今日良机已失,还是回去思谋出破解此招的方法,来日再找这厮麻烦!”

%94
正想着,呼啦!

%95
万家三仙就将魂兽大军猛地一动,仿佛积蓄多时的大坝开闸,无数的魂兽宛若泄洪一般,掀起惊天动地的军势,杀向前来!

\end{this_body}

