\newsection{南联}    %第六百零二节:南联

\begin{this_body}

%1
天庭。

%2
大阵发出阵阵的嗡鸣之声,刺眼的光辉宛若根根尖针,深深地插在魔尊幽魂的身体上。

%3
尽管他缩成一团,尽全力抵抗,但仍旧难以抵挡大阵的威能。

%4
紫薇仙子双目微闭,一边全神贯注地操纵大阵,一边凝神接收着从魔尊幽魂身上搜刮出来的种种情报。

%5
直到大阵发出不堪重负的声音,紫薇仙子这才意犹未尽地停下来,深深地看了魔尊幽魂一眼。

%6
即便是死敌,紫薇仙子也不得不佩服魔尊幽魂的顽强意志。

%7
明明已是身陷绝境,一丝希望都没有,但他仍旧顽抗到底。不愧是古往今来,纵横天下的蛊尊之一。

%8
紫薇仙子没有多说什么,离开此处。

%9
回到自己的大殿,她立即开始全力推算。借助星宿棋盘,使得她的推算能力臻至世间的最巅峰。

%10
海量的念头在她的脑海中奔腾起伏,宛若亿万繁星在急速地闪烁。她的念头不断地碰撞,乃至于都碰撞出了脑海,飞到身体之外,化为无数的紫色光点,笼罩着紫薇仙子全身上下,很快就蔓延开来,充斥整个雄阔的大殿。

%11
良久,紫色的念头渐渐收拢、消散,最终平静下来,紫薇仙子深呼吸一口气,收了手段,缓缓睁开双眼。

%12
她额头已有一层细密的薄汗,整个人的精神萎靡不振,眉头紧蹙。

%13
“方源你究竟会在哪里渡劫?”紫薇仙子口中呢喃。

%14
她不辞辛苦,屡屡从魔尊幽魂身上艰难搜魂,就是要收集到影宗的内部情报。尤其是那些残留在五域中的布置,最是让紫薇仙子在意。

%15
因为之前,方源就是靠着影宗的布置,才将南疆正道的通缉队伍一网打尽。方源若是渡劫,极可能再去依靠影宗的这些布置,没有道理他放着这些不去借力。

%16
紫薇仙子一直在紧密地关注着方源的修为进度,自从琅琊攻防大战之后,她就推算明白,方源已是离八转程度不远了。

%17
七转、八转完全是两个不同的层次!

%18
九转不出,八转就是世间的巅峰。

%19
一旦当方源达到八转修为,那他就真的难以遏制,成为天庭的巨大麻烦。

%20
“方源最后一场浩劫,乃是关键。若是能在这里扼杀了他,对于天庭,对于天下苍生,都有巨大的利处。”

%21
紫薇仙子不放过任何一个剿杀方源的机会,但是当她去做的时候,她发现此事非常困难。

%22
因为她并不清楚方源渡劫的具体时间、具体位置。

%23
方源本人有着阎帝杀招,防备死死,根本不会让她有一丝可趁之机,完全无法推算。

%24
方源拥有非常丰富的宙道手段,改变仙窍的时间流速十分方便,所以到底是什么时候才去渡最后的浩劫,也无从估计。

%25
方源还掌握着定仙游,全天下都是他可以渡劫的地方。这叫紫薇仙子怎么去布置?

%26
她最多是可以将黑天、白天排除在外。

%27
因为福地渡劫,吸收的地气要更多一些,所以常常在五域之中。

%28
当然,黑白两天的可能性比较小。但也不是没有,毕竟有那么几处地方,地气充沛至极,十分特殊。

%29
但剩下的五域呢?

%30
不像是光阴长河,说到底这只是一片天地秘境,所以紫薇仙子可以布置仙蛊屋,严防死守。

%31
落到五域这么大的范围,千百个仙蛊屋都不够用的。况且天庭方面,也绝没有足够的人手来分派到五域各处,来防止方源渡劫。

%32
紫薇仙子叹息一声,她也没有什么好办法。

%33
如今之计,她只有拼命搜刮魔尊幽魂的魂魄,借助这些可贵的线索,来推测方源可能渡劫的地点。另一方面,她积极沟通五域正道势力,把方源要渡劫的消息放出来。天庭对其他四域鞭长莫及,唯有借力,才能希望。

%34
方源要渡劫的消息,很快就传遍了五域。

%35
不仅是各域的正道家族、门派,就连魔道、散仙都有耳闻。

%36
时至今日,方源已经名传天下,震动五域。他虽然只有七转修为,但战绩彪悍至极,已经被公认为八转之下的第一人。蛊仙界公认,就连凤九歌都不如他,他是近百年来魔道最强新星,成长速度让人瞠目结舌,蛊仙界刚刚反应过来,他就已经是一位魔道巨擘,凶威赫赫。

%37
所以,方源要渡劫的消息,在五域蛊仙界很快掀起一顿热议狂潮。

%38
“方源魔头若是要渡过最后一场浩劫,那就成为八转蛊仙,这还了得?!”

%39
“魔涨正消,为之奈何呀。”

%40
“此人非同一般,掌握春秋蝉,乃是重生之人,又获得诸多蛊尊的遗泽。一场浩劫而已,难不倒他。”

%41
“我倒不这么看。正所谓,多行不义必自毙,此魔头嚣张至极,想要渡劫,先问问天庭答应不答应,全天下的正道答应不答应了。”

%42
有人看好方源,也有人不看好。有一点是很明显的,那就是绝大多数的蛊仙都不想看到方源成功渡劫。

%43
“这个魔头,不过六七转就祸乱天下,若是成了八转,岂不是要翻天覆地?”

%44
“方源在其他地方渡劫,我不管。但若是在南疆渡劫,我铁家决不答应!”

%45
在蛊仙界议论的同时,南疆的正道势力也积极碰面,不断交流沟通,探讨着如何对付方源。

%46
尽管他们各家的手中,都还有蛊仙人质落在方源的手里。但若真的能趁机铲除掉方源这个大祸害,南疆正道势力还是非常热衷积极的。

%47
原因很简单,还是利益两个字!

%48
放着方源不管,将来他越来越强,威胁越来越大,始终是一片阴云,笼罩在南疆正道的心头。说不定哪一天,就有哪家势力遭殃。

%49
更关键的是,方源手中的仙蛊、仙蛊方、传承,各种修行资源非常丰富,他本身就是一个移动的人形宝山。铲除了他,说不定就能有巨大的收获。哪怕是仙窍都毁了,只要方源的魂魄残存一片,这些正道势力就能通过搜魂,获得某个尊者的残缺传承。单单这一点,就足够让这些正道势力趋之若鹜了。

%50
武庸沉吟道:“我们南疆乃是影宗的总部所在,这里定然是有大量的布置。之前方源能俘虏众多同道仙友,就是借助了其中一处布置。方源有很大可能,会在南疆渡劫!”

%51
武庸说这话之前,就在私底下得到了紫薇仙子的指点,知道了数处影宗在南疆各地的残存布置。

%52
尽管武庸心头雪亮,明白这是紫薇仙子想要借刀杀人。但武庸被这样利用,也是心甘情愿。因为方源伪装武遗海,将他狠狠戏耍过,方源活着,就是对整个武家,对他武庸的最大羞辱!

%53
“武庸大人分析的极有道理。”乔家蛊仙立即附和道,“方源劫持了我们各家的蛊仙,他又是影宗的继承人,当代宗主,定有极其了得的魂道手段。搜魂下来,他必定对我南疆正道了若指掌。”

%54
“其次,我南疆近些年来经历种种大变,不管是散仙、魔道,尤其是我正道,损失相当惨重。方源魔头若在此地渡劫,所受的阻力将是最小。”

%55
“就算方源是在我南疆渡劫,我们该如何防范呢?南疆这么大,他躲在哪个旮旯里渡劫,我们谁能知道?”一位蛊仙问。

%56
“这点是可以查探清楚的。”夏家蛊仙立即回答,“一旦他落下仙窍,就和天地沟通。这种状态从渡劫开始,一直到他渡劫成功,收起仙窍后,才能结束。这个时候,他就不再是隐藏在阴影之中,而是有迹可循了。不需要其他人帮助,光靠我们夏家的这些智道蛊仙,借助大阵,就能合力推算出他的具体方位来。”

%57
“当然了,若是更多的智道仙友出手相助,我们推算方源的把握将更大一些,也会更快一些。”另一位夏家蛊仙接着开口,“其实真正的难处,并不在于推算,而是当我们知晓方源渡劫的方位时,我们该怎么行动,才能及时地赶到那里,这里还有一个前提,那就是我们必须集结了相当数量的蛊仙,能够对付方源。”

%58
面对这个问题,南疆蛊仙陷入了沉默。

%59
从古到今,南疆的正道势力都是山头林立,谁也不服谁。而此时要铲除方源的话,就需要这些正道势力同心协力,一起行动。

%60
这几乎算得上是一种史无前例的事情了。

%61
其实,方源如今的战力可敌八转,但到底是孤单一人。大多数的正道势力,单单一家,就能让渡劫中的方源吃不了兜着走,渡劫失败,乃至身死道消。

%62
因为这些正道家族,不是拥有八转蛊仙,就是拥有着仙蛊屋,可抗八转。它们通常都是经营了千年万年,底蕴深厚无比。相比较起来,方源崛起的时间还是太短了。

%63
但是单单让一个超级家族来全力对付方源,可能吗?

%64
不可能。

%65
这是一片丛林,各个正道家族都是猛兽,割据一方。让这头猛兽离开自己的地盘,去对付另外一头猛兽,哪怕这头猛兽还未彻底成长起来,但到底是有着爪牙的。

%66
风险太大。

%67
就算除掉了对方,离开地盘的猛兽极有可能受伤,一旦受伤过于严重,那么引来的就是其他猛兽的入侵了。

%68
当然,入侵是本质,表面上还会披上一层大义凛然的皮。总会有各种借口的,南疆正道之间矛盾层层,这么多年积累下来,还愁找不到一个借口?

%69
就算实在找不到,那就“帮助”吧,比如说主动帮助“建立大南疆共荣圈”。你看,我多热心助人呐。你好意思拒绝吗?你想拒绝,你有这种力量吗?

%70
所以,要对付方源,真正的难点,其实不在于方源本身。

%71
方源很强吗?

%72
的确,能以七转修为力战八转,在历史中都十分罕见。

%73
但和一个超级势力比较起来呢?

%74
方源并不强,综合实力是比不上的。硬碰硬,方源会碰的头破血流。

%75
一些南疆蛊仙忍不住苦笑起来,因为他们意识到:眼前情况实在是有些讽刺,真正束缚他们的,就是他们彼此。他们相互忌惮,他们相互牵制,他们有庞大的地盘,他们彼此防备,所以他们需要分兵驻守。

%76
沉默良久,铁家的蛊仙忽然发声:“其实有一个方法,可以解决部分难题。众所周知,我铁家有一座仙蛊屋烽火台。”

%77
南疆蛊仙继续沉默,但沉默没有多久,一个声音忽然传出:“我同意铁家建立烽火台。”

%78
众仙顿时大吃一惊,因为此人不是别人,正是武家的太上大长老武庸!

%79
就连刚刚开口的那位铁家六转蛊仙都很诧异,武庸居然支持铁家建设烽火台?

%80
众仙还在震惊当中,武庸再度开口,他声音低缓而又有力,一字一字宛若巨石落在众仙的心头。

%81
他说到:“是时候要做出改变了。你们还没有察觉到吗?大时代的浪潮已经来临,五域的界壁正开始缓缓消散。如果我们不做出改变,那么我们就要淹没在时代的浪潮当中。”

%82
群仙沉默。

%83
第二人站了出来,他是巴家的太上大长老,一直想要谋夺武家的第一势力的地位。

%84
他罕见地开口了,语不惊人死不休:“我巴家支持武庸。我们要向前看,诸位,中洲有天庭,北原有长生天,我南疆有什么?”

%85
他顿了顿,然后说出一番饱含哲理的话来:“群山虽多,但一个个的山头,总会被他人依次征服过去的。所以,我支持武庸,同样也支持铁家铺设烽火台。”

%86
他说完后不久,池家太上大长老池曲由也紧接着发言,表示支持。

%87
随后,罗家、侯家、柴家等各个位高权重的蛊仙,都依次发言,表达出强烈的联合意愿。

%88
南疆群仙们忽然在震惊中明白,此次的议事意义之重大,绝对会被载入南疆的史册!

%89
从未有过的情况出现了,南疆的正道势力要联手了。这绝非是以往冠冕堂皇的宣传语言,而是真正的紧密合作!

%90
气氛前所未有地热烈起来,南疆蛊仙们心潮澎湃,均有一种参与历史时刻的激动!

%91
“就让方源成为我们南疆的第一个祭品,然后就是……其他的四域!”武庸最后结语,毫不掩饰他心中的野望。

\end{this_body}

