\newsection{神秘的羊皮地图}    %第四百七十四节:神秘的羊皮地图

\begin{this_body}

仔细检查了一番,方源并没有发现什么陷阱。

他便慢慢接近那具干尸,约莫距离干尸只有五六步的时候,异变忽生。

方源忍不住在肚中咒骂一声,因为他又再一次失去了对少年盗天的控制,重新成为旁观者。

少年盗天口中呢喃着:“这人困死在井底,难道是被那群野兽围住,无法逃生吗?”

然后,他来到干尸的身边,先是默哀了片刻,这才开始动手搜索尸身。

接下来并没有任何的意外发生,干尸很普通,没有什么坑人的陷阱。

少年盗天在搜尸的过程中,发现这具干尸生前应该是蛊师,并且转数不低,身份高贵。

他并没有什么蛊虫留下,但身上穿着的衣裳内衬,却是缝制了一块羊皮地图。

少年盗天什么都没有收获,除了这块羊皮地图。

在这地底空洞中,光线相当暗淡,少年盗天只看到地图上一条条模糊的线,并不分明。

他将地图揣入怀中,又仔细搜寻周围,但并没有什么其他发现。

不过少年盗天却已经很满足。

因为这里有着沙漠中最为珍贵的水源。

他倒也小心,先是检查了水质,发现没有问题之后,这才小心翼翼地喝了一口。

这水来源于地下深处,似乎是干尸生前费了好大力气,才挖掘引来的。

少年盗天喝了一口水,顿时感到一阵冷冽的清爽,原本之前喝血的口腔中充斥的血腥气息,也因此淡了许多。

少年盗天喉结滚动了一下,这一口水勾动出了他心中一股强烈的冲动,紧接着他直接俯下身来,将脸埋入暗泉。

咕咚咕咚,喝了好几大口,他这才猛地抬起头来,水滴四下飞溅。

一屁股坐在地上,少年盗天没有说话,双手撑着地面,仰头闭目,半晌后,满足的长叹一声。

休息了一阵,他伸出手来,一抹脸上的水渍,站起身,重新回到了枯井底部去。\&\#26825;\&\#33457;\&\#31958;\&\#23567;\&\#35828;\&\#32593;\&\#119;\&\#119;\&\#119;\&\#46;\&\#109;\&\#105;\&\#97;\&\#110;\&\#104;\&\#117;\&\#97;\&\#116;\&\#97;\&\#110;\&\#103;\&\#46;\&\#99;\&\#99;]

少年盗天旋即打了个哆嗦,枯井底部远比那处地下空洞要寒冷得多。

不过少年盗天有必须来到这里的理由。

他仰头望了望井口,虽然他搭了几层兽皮覆盖,但夜风很大,居然已经吹飞了大半兽皮,露出一个巨大的洞口。

从洞口中,少年盗天看到漆黑的夜幕中,那些悬挂在高空的繁星。

他叹了一口气,从身上取出一些兽皮,还有砍伐的木材,很快就搭建出篝火的雏形。随后,他钻木取火,努力了一盏茶的功夫后,成功地点燃了篝火。

火烟并不大,顺着井口飘走,火光带来的温暖袭来,逐渐驱逐少年盗天身上的寒冷。

少年盗天先是烤了几块兽肉,烤熟之后,吞咽下肚。

吃了熟食,饱腹的感觉,让他不禁昏昏欲睡。

但少年盗天强忍困意,就着微弱的火光,把刚刚得来的羊皮地图拿到手中观看。

“这羊皮地图的历史,看来已经很久了。”

“咦?它重点描绘的地方,不就是我那族群生存的小绿洲吗?”

少年盗天吃了一惊。

在这羊皮地图上,小绿洲被重点标注,旁边还有几行西漠的文字。

文字颇小,似乎因为年岁太久,绝大多数都已经模糊不清,只剩下开头部分的几个字,断断续续,最为清晰。

少年盗天勉强辨认,艰难地读出声来:“葬仙之地……不详……诅咒……”

“古怪!”琢磨良久,他都没有再琢磨出更多的线索,眉头不禁拧紧。

“我来到这个世界,已经有十几年,也听闻过族中老人说过仙的故事。不过这些都是传说,世间真的有仙吗?”

“这似乎并非不可能。看看蛊师,就知道这个世界足够奇妙诡异,什么都可能存在。”

“葬仙之地……难道说,我生活的家园还埋葬了一个仙人不成?”

“但为什么又有诅咒?还说什么不详?”

“还有这个羊皮地图也很古怪。这个世界里,存在信道的蛊虫,可以记录地图,储藏讯息。那个干尸生前应该是一个强大的蛊师,为什么他不用蛊虫记录这份地图,偏偏要用这份羊皮地图呢?”

“这羊皮地图被他缝进衣服的内衬当中,若非衣服枯朽破烂不堪,我也不会轻易发现。”

“如此郑重地收藏起来,这份羊皮地图应当很重要。但这样也未免太不保险,远远不如将地图记录在一只信道蛊虫中安全啊。”

少年盗天口中呢喃,眼中不断地闪烁着思索的光。

方源始终旁观,少年盗天的疑惑,也早就被他想到,这些都是疑点,尤其是羊皮地图的存在。

“暂时不管这些了。这个羊皮地图上,也标明了这处枯井的方位。我顺着这个地图的指示,就能回到部族中去了。”

尽管少年盗天对这个部族一点都没有归属感,但他也明白,单靠自己一个人在这茫茫沙漠中生活,完全是不可能。

且不说沙漠中危机四伏,谁也不清楚生死存亡的威胁会什么时候突然袭来,单说食物,少年盗天手中的食物也是有限的。

这片小绿洲底蕴太薄弱了。

眼皮子有如山重,少年盗天的视野迅速模糊,很快就陷入了沉睡当中。

他实在是太累了,不仅是身体上的疲惫,心灵上也是饱受起伏波折。

他这一睡,方源的视野中也就一片黑暗。

在这片黑暗当中,方源感到梦境消融的力量明显拔升了好几倍,他的魂魄再一次开始迅速消融。

好在方源已经有了前一次的经验,当下默默忍受,耐心等候黑暗的消失。

梦境的时间不好估量,黑暗消去后,方源的魂魄底蕴足足被削减了一半!

视野再次清晰起来,方源惊讶地发现,少年盗天被五花大绑,倒在枯井底部。

而他的面前,则站着一位面色阴沉的蛊师老者。

这位老蛊师满脸皱纹,头发全白,岁数已经很大,此刻他枯瘦如柴的手上却是拿着那张古怪的羊皮地图。

他用手不断地抚摸着这张羊皮地图,眼神中流露出狂热、深情、贪婪等等复杂的情绪。

这种眼神,让少年盗天头皮发麻,感觉自己遇到了变态。

“你是谁?为什么绑我?”少年盗天问道。

他感觉很委屈,一觉醒来就被绑住了,沦为阶下囚。

“你应该庆幸,我沙枭没有杀了你,你还有命在。”老人开口,声音相当的沙哑难听。

他小心翼翼地将羊皮地图收入怀中,目光投向少年盗天:“小子,你是一达部族的人?”

少年盗天没有吭声。

因为他在瞬间明白了一个道理,蛊师老者一定知道这张羊皮地图的价值和秘密,他之所以没有杀死自己,而是将自己绑起来,应该是想利用自己。

蛊师老者见少年盗天没有应声,顿时脸色再沉一分,他轻飘飘跨前几步,仿佛鬼魅一般,毫无声息地站到少年盗天的面前。

“小子,别以为你不说话,我就不知道你肚子里转着什么主意。嘿,你大概是没听闻过我沙枭的名头,现在就让你稍稍尝尝我的厉害。”

蛊师老者阴森森一笑,忽然抬起脚尖,狠狠一踢。

脚尖正中少年盗天的心口,一瞬间,剧烈的疼痛差点让他直接背过气去。

少年盗天仿佛是破麻袋般被轻易踢飞,旋即又撞到井壁上,扑通一声,摔倒地上。

但苦难才刚刚开始,少年盗天开始嘶声大叫起来。

因为蛊师老者的这一脚,并不简单,少年盗天感到全身皮肤都酸麻无比,与此同时,肌肉内脏仿佛是遭受万千冰针,不断地攒刺。

剧烈的痛楚,让少年盗天相当难以忍受。很快就痛得涕泪交流,身子蜷缩起来,宛如龙虾。

“小子,知道我的厉害了吧。”蛊师老者阴沉的笑起来。

他心中欢愉。

少年盗天的惨嚎,让他听着,有一种美妙的满足感。

ps:十点还有第二更。(未完待续。)

\end{this_body}

