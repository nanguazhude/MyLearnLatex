\newsection{五成!}    %第五百五十一节:五成!

\begin{this_body}

宝黄天。mianhuatang.cc [棉花糖小说网]

一大团的透明火焰,在静静地燃烧着,散发出的宝光厚重恢弘。

火焰一经上市,就吸引了许多蛊仙的注意。

大量的意志,还有少许神念,都被吸引过来。

黑天寺的蛊仙张继淡淡而笑,他正是这团透明火焰的主人。

“没想到这一次,外出探险,居然收获了这么一批稀罕的七转仙材。贩卖出去后,正可补足我的仙元储备啊。”

他一面思考,一面回应其他人的问价。

他开的价格,只是比往常稍微高了那么一点点,因此很多人都有购买的想法。

“这些货我都要了。”其中一道神念,直接联系上张继,语气豪爽,直接开出了高价。

张继怦然心动,有些难以置信:“这些你都要?这价钱……”

“来交易吧。”对方非常干脆,直接送来了大量的仙元石。

“好!”见到这一批巨量的仙元石,张继当即双眼放光,立即答应下来。

交易很快完成,关注此处的蛊仙,都有些猝不及防。

“不要着急,我这里还有货呢!”张继的神念不断波动,连忙宣扬,又掏出一团透明火焰,放进了宝黄天中。

只是这团透明火焰,在总量上要比卖出的那团,少上许多。

因为第一笔买卖,激发起了其他蛊仙购买的欲望,毕竟这种透明火焰,在市面上的确是罕见。换做平常,只能持续收购,往往每一次收购也只是拳头大小,脸盆大小的一团。哪里像现在这样,透明火焰大如车马!过了这个村,可就没有这个店了。

“不过这一次,我要抬价。”张继嘴角的笑意越加浓郁。他对透明火焰的所知,并不多。这一次也是第一次贩卖,行情不太了解,其他蛊仙的购买热情,超出了他的想象。

价格被他提得越来越高,但周围蛊仙很少退出竞争,开价也越来越高。

“糟糕,我刚刚似乎卖错了啊!”张继见到这样的情景,心中便有些暗暗后悔,同时也钦佩那个收购者的决断。<strong>棉花糖小说网www.MianHuaTang.cc</strong>

张继并非无智,他将所有的透明火焰,分成了三份。第一份是来试水,现在贩卖的才是第二份而已。

随着交易进行得越多,他就这个市场行情将把握得越加精准。

“浮生火!”方源望着至尊仙窍中的这团透明火焰,心情愉悦。

正是他抢先出手,收购了张继的货。

这种浮生火,是透明色泽,并且燃烧着,静静无声,毫无温度。但当它烧死一个生命之后,这个生命中的往昔种种,就会化为影像,在浮生火中不断重现。那个时候,浮生火就会散发出五颜六色的光泽,不再是透明。

浮生火的保管,需要特别的方法。

张继却是意外获得大量浮生火,没有这方面的手段。

方源得手之后,立即施展手段,将浮生火好生保存。这手段却不是幽魂真传中的内容,而是方源利用幽魂真传交换来的琅琊传承。

“我原本还想长途跋涉,找其他人的麻烦,谋夺这浮生火。没想到宝黄天中,忽然就有人来卖,并且大量出货。这次的运气真是不错!”

方源笑了笑。

想要收购浮生火的蛊仙,必定有很多。

但蛊仙意志,往往做不了什么主张,只能依照蛊仙遗留下来的命令行事。

而要一下子拿出那么多的仙元石,也不是寻常蛊仙能够做到的。

就算有资金在手,这些围观的蛊仙也想讨价还价,毕竟他们需求的量往往并不大。

如此种种因素,才叫方源觑得便宜,立即下手,占了先机!

至于风险?

虽然这张继乃是黑天寺的蛊仙,天庭的人,但他不知道购买者是方源,方源也不知道他的身份。

宝黄天的买卖,向来安全得很。若是天庭能依靠这个,算计到方源,连宝黄天都能渗透到这种程度,那方源干脆直接投降好了。

总的来说,这种可能性相当的小。

饮刃酒、刃蛊、正反狙神针、浮生火,这四样仙材基本上都收集全了,刃蛊还需要再收购一点,宝黄天完全可以满足,实在不行,西漠萧家那边也是一个渠道。

在外停留也就没有意义,方源便启程,赶回琅琊福地。

仍旧是通过传送仙阵,方源回到琅琊福地,过程非常顺利,无惊无险。

方源回来的时间,比计划中要早许多,这让琅琊地灵十分高兴。

“既然你回来了,那我就着手准备炼制了。”琅琊地灵非常积极,天庭施加在他心中的压力实在太大了。

“这一次,还要多谢太上大长老你,还有其他三族蛊仙,借给我这么多的仙元石。”方源道谢。

没有他们资助,方源怎可能有这样庞大的资金?

“无须客气!我们都是自己人,你强大起来,就能令琅琊福地更加安全。琅琊福地若是危险,他们受制于盟约也跑不掉!”琅琊地灵的回应非常直接坦诚。

的确,现在的事实,就是琅琊派、雪民、石人、墨人城都是一根绳子上的蚂蚱。谁若遭殃,其他人都无法逃脱。

方源便将这四份仙材,交给琅琊派处理。

这些仙材的处理,至少得要半个月的时间,当中过程非常繁琐冗杂。这还是整个琅琊派的毛民蛊仙们,全力以赴的结果。

从这一点,就可看出万我仙蛊方的优异!

方源成为炼道准无上,设计出来的万我仙蛊方,非常优异,堪称世间极限。几乎是已经到了无法改良的地步。

炼蛊的关键步骤,浓缩删减到了极致,只有三个。其他步骤则多达一万多步,大部分都是处理仙材。

方源将仙材处理到最佳的状态,将绝大多数的苦功都耗费在这里,哪怕仙材的消耗量大增,也在所不惜。

因为这些步骤,是蛊仙们可以掌控的。而那些炼蛊的关键步骤,就只能任凭道痕之间的碰撞和纠结,不受蛊仙掌控。

通过这样的方法,再结合琅琊派的炼蛊实力,方源的这道万我仙蛊方成功率高达五成!

也就是说,方源有一半的机会,能够炼制成功,另一半则是失败。

五成的成功率,实在太过恐怖。

绝大多数蛊仙炼蛊,若是有个百分之一的成功率,他们就偷笑了。

不过,这主要也是因为他是炼道准无上!

世间能有多少炼道准无上?纵观古今,方源这样的人物,真的是屈指可数。超过他这层境界的,也只有三人罢了。

而世间的蛊仙,兼修的极少,专修炼道的又有多少呢?

换个角度看,大多数蛊仙炼制仙蛊,其实都相当于门外汉。炼蛊的成功率若能提升上来,那就见鬼了!

其实,五成的成功率,就连方源推算出来时,也有些不敢相信。

不过,他旋即想到长毛老祖。

长毛老祖一生当中,炼制了许多仙蛊,当中不乏七转,甚至八转。他是炼道无上大宗师,若是没有这样的成功率,巨阳仙尊、盗天魔尊能找他帮忙吗?

“还有就是,我借助琅琊派举派的实力,来炼制这蛊。当今天下,炼蛊造诣,恐怕琅琊派乃是当之无愧的第一吧。”

“除此之外,第三大原因就是我的运势强盛。”

首先是炼道准无上的境界,其次是天下第一炼蛊势力全力出手相助,第三是继承了巨阳真传后的强盛气运。

三大原因,一齐促成了这次万我仙蛊炼制的五成成功率!

“终于有这么一天,我摆脱了炼制仙蛊的魔咒啊。”方源感慨万千,有一种泪流满面的冲动。

他知道,最主要的原因,还是他成就了炼道准无上境界!

其他两方面缘由,虽然也有帮助,但其实在之前炼蛊时,也多多少少都具备。五成成功率,主要还是炼道准无上境界带来的。

“这还是我推算、改良仙蛊方而已。若是将来,我补齐炼蛊方面的基本功,由我亲自主导炼蛊,成功率还会提升许多!”

“不过可惜,我没有这样的充裕时间啊……”

------------

\end{this_body}

