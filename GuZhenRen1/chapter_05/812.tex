\newsection{操蚁对决}    %第八百一十五节:操蚁对决

\begin{this_body}

方源立即动用全力,去奴役这些蚂蚁。

泰琴同样如此。

这些蚂蚁乃是绿蚁居士发明,他择徒的比试内容,自然就是比拼双方操蚁的能耐。

方源驾驭了三十八头工蚁,六头兵蚁后,立即心念一动,挥动全部军势,朝泰琴杀奔过去。

泰琴也有三十多头工蚁,五头兵蚁,她采取的战术和方源不同,乃是运用工蚁铸就蚁巢,兵蚁护卫。

方源兵力略处小优,战术也非常激进,令场中少年们纷纷惊呼起来。

泰琴大皱眉头,一阵慌乱:“这龙人少年吴帅,怎么像变了个人似的?如此激进!”

她连忙调遣兵力,先从蚁巢中调派工蚁出动,随后在外的兵蚁收缩回防,准备依托还未建成的工事,严防死守。

很快,两军交接,展开惨烈厮杀。

“没想到一上来,就出现这样的局面。”

“吴帅战术一扫之前,此举实在太过冒险了。”

“没错,只要泰琴防住这一波攻势,那她占据的优势就大了。吴帅没有建造蚁巢,泰琴却有,这是层次上的差别。”

“现在就看双方的细微操蚁之术。因为是大前期,一两头蚂蚁恐怕都要倾斜胜利的天平了。”

“吴帅虽强,但泰琴也造诣深厚,双方半斤对八两。依我看,泰琴的胜面极大,吴帅怎么会如此莽撞呢?”

然而片刻后,双方却是暂时分出胜负。

泰琴损失较多,缓缓后撤,方源一方趁势压上。

“怎么可能?吴帅竟然明显胜过泰琴?泰琴失误了?”

“并没有!泰琴是正常水准的发挥,但吴帅却是超出水准。是他动用了什么奴道手段,还是原本就是这样的水准?”

“难怪他有信心直接冲杀上来。”

泰琴脸色凝重,后撤的过程中,又被方源趁势杀死多头蚂蚁。

龟缩到未完成的蚁巢当中时,她手中的军力只剩下十二头,非常凄惨。

当然,方源也只是稍微好一些而已。

蚁巢通道只有几个,泰琴对自家的蚁巢自然了若指掌,在要地部署兵力,同时又分出心念,奴役在外的蚂蚁。

方源念头一转,却未急着进攻,眼下困敌之势已成,敌人如瓮中之鳖。

方源对蚁巢布局并不了解,若冒然进攻,自己手中就这么一点兵力,反而不美。

于是他也分兵把守各处道口,将蚁巢团团围住,随后也催动奴道杀招,不断奴役在外的蚂蚁。

蚂蚁不断从地底冒出,最初是工蚁,随后是兵蚁,现在已经开始出现箭蚁。

这种箭蚁能从口器中,喷射出尖刺。喷射出三根尖刺之后,箭蚁就会虚弱至极,连工蚁都打不过。

即便如此,箭蚁也是一种极强的兵种,往往能以少胜多。若是成就一定规模,在大战中一齐射箭,更具威胁。

“箭蚁乃是双方必争之物了。”

“没错,泰琴若有箭蚁,完全可以以少换多,将通道门口的敌方蚂蚁全部射杀。这样一来,内外兵力汇合,就能反杀而出。”

“吴帅虽处上风,但并不稳固,很容易就会翻盘。”

“可惜了,若是此刻没有出现箭蚁,而是甲蚁,那吴帅必定能凭借甲蚁攻坚,直捣黄龙,进而摧毁对方蚁巢。”

少年们议论纷纷,交头接耳。

双方之争,乃是时下最为风行的操蚁戏。操蚁戏开端,固定出现工蚁、兵蚁,但随后的蚂蚁种类却是随意涌现。这些新出现的蚂蚁,不仅考较比斗双方的奴役造诣,而且还考察争斗蛊师对于蚂蚁的了解,对于整个战局的洞悉程度。

方源和泰琴全力争夺,方源仍旧占据优势。

少年们尽皆侧目,有些难以置信。

“这片场地都布置了蛊阵,双方相对而坐,越靠近自己的蚂蚁越容易奴役。泰琴奴役的蚂蚁是自己身边,而吴帅却是远至极端,怎么可能还被他占去了优势?”

“难道这就是吴帅的真正实力?他到底是有多强?”

“我们之前一起联合,来削弱他,没想到他竟然还有如此实力。他就不感觉到累吗?”

若是真的吴帅,或许会感觉到累,但方源不是。

方源刚进入这幕梦境,状态良好,尤其是他的魂魄,乃是当初本体分魂所化。

之后,虽然完全转变成了龙人魂魄,并没有进行魂修,但本身底蕴深厚,超越寻常。

魂魄底蕴直接影响奴道的效果。

比如方源本体在南疆埋伏战之前,奴役不了多少的太古年兽,但如今却是轻轻松松二十多头。

箭蚁之争,泰琴失败。

她分心他顾所奴役到的箭蚁,全部方源消灭。

方源不仅是有箭蚁,而且还有工蚁、兵蚁与他的箭蚁配合。泰琴吃亏在兵力两分,这让她处境极为被动。

不过好在箭蚁攻坚,并不出众,尤其是在坑道很多的蚁巢中。

“我虽然积累了优势,但这优势并不大,尤其是这段时间,泰琴必然催动工蚁,加紧建设蚁巢。”

方源看似激进,实则是稳中求胜。

所有的战术,都是建立在知己知彼的基础上。

“接下来,只要出现甲蚁,我就能攻坚,摧毁蚁巢了。就算不是甲蚁,出现其他的种类,只要不是炮蚁,我也能凭借兵力优势,强攻胜利。”方源运筹帷幄。

他的胜面的确很大,场中的人族少年们尽管不愿意看到,但也不得不承认眼前的事实。

“龙人吴帅要成功了。”

“没想到这一场决赛,会如此落幕。”

“没想到吴帅竟隐藏了这么多的实力,他的确是厉害。”

“唉!可惜了,堂堂绿蚁居士的奴道传承,居然要传给一个龙人。”

泰琴脸色严峻至极,双眼死死盯着场地,一眨不眨。她也意识到自己正处在失败的边缘,但却不想放弃。

她死死咬紧牙关,心中萦绕着一个强烈的念头:“若是败给吴帅这样一个龙人,还不如让我上一场输给张双好了!我不能输,我不要输!”

就在这时,场地上空忽然升腾其丝丝云气,云气迅速凝聚成团,化为一团团的绿云。

几个呼吸之后,从绿云中飘飘洒洒出一蓬蓬的酸雨。

方源目光变冷,少年们哗然一片。

“这一次的变化,居然不是出现蚂蚁,而是战场变化。”

“若是出现地震之流,那泰琴不需要吴帅强攻,立刻玩完。若出现其他变化,也只是维持局面。但偏偏出现了酸雨……”

“是啊,泰琴的蚂蚁都在蚁巢之中,而吴帅的军力全都裸露在外,饱受酸雨侵蚀。”

“吴帅开始就地铸造蚁巢了。”

“为时已晚了,好一场酸雨,让泰琴绝地翻盘,吴帅冒险建立起来的优势荡然无存。”

泰琴得到酸雨相助,终于挽回绝对的劣势。

方源只得收缩兵力,铸造蚁巢,躲避酸雨。

酸雨过后,便出现了甲蚁,可惜方源就算奴役再多,也错过了最佳的时机。

不过几个回合之后,方源又依照着魂魄底蕴,不断压制泰琴,接连几场小胜之后,累积出来优势,又再次将泰琴逼入失败的悬崖边缘。

一场大会战,泰琴纠集残余兵力,负隅顽抗时,忽然从地面涌出一头王蚁。

要收服野生王蚁,不仅需要蛊师施展奴道手段,而且还要催动兵力将王蚁打得没有脾气。

王蚁出现的地方,却正是泰琴的兵力中央。

方源立即发动猛攻,但他的兵力再多,一时间也难以冲破泰琴仅剩不多的几道防线。

泰琴拼尽全力,不惜施展损害自己的手段,终于将王蚁收服。

王蚁虽然半残,但却有奇特威能,和它一方的蚂蚁速度更快,防御更厚,攻击更猛,综合实力暴涨一大截。

方源的蚂蚁没有得到增幅,数量虽多,却有沦为炮灰的嫌疑。

久攻不下,方源明智地选择撤退,留下一批部队断后,和泰琴重新对峙。

王蚁既然出现,绝不会仅有这一只。

果然很快,就有新的王蚁在场地中出现,接连被双方获取。

如此双方的争斗,又进入僵持阶段。

双方控制的蚂蚁越来越多,一边不断铺设蚁巢,占据场地,一边操纵兵力,相互攻伐。

泰琴脸色苍白,方源也好不到哪里去。

不过,他仍旧是凭借着丰富的经验,逐渐占据优势。

轰隆隆!

就在方源想要发动致胜一击时,战场竟发生了地震,方源损失惨重,蚁巢十之八九都因此毁了。

泰琴虽然也是如此,但她本身蚁巢就不多,损失比方源轻多了。

双方又回到同一水平线上。

少年们轰动了。

“又是这样!”

“每当大劣势的时候,就有变化产生,帮助泰琴了。”

“泰琴的运气未免太好了吧?”

“已经连续三次如此了。”

“不会是有人作弊吧?”

“怎可能?这场地可是绿蚁居士亲自布置的,泰琴能作弊的话,早就胜过吴帅了!”

方源目光更冷。

“有麻烦了。”他暗道。

按照他的气运,就算和本体失去了联络,也不会如此倒霉。

“最大的可能性,就是梦境显化,让我必须失败,战胜不了泰琴。”

“所以,每当我要得胜的时候,梦境就会出现演变,让我丧失致胜的可能。”

“那么我该故意留手,承受失败吗?”

若是寻常时候,这种试探是完全可以的。但如今龙人分身却仅有一次机会。

若是他猜错了,那么等若是自己放弃,那样一来,龙人分身陷落梦境,本体只能出手。

本体和四大龙将激斗,局面失控,很可能龙宫就要飞走。

方源之前的谋算,也就要落空了。

到底该怎么办呢?

ps:今天炼一更蛊的时候,运用到了新的炼蛊技巧,炼出了两个一更蛊。另外推荐一下好友的书——《日常公寓怪物录》悠闲疯著。给他留评语的时候,别忘了说蛊真人推荐的哟。

------------

\end{this_body}

