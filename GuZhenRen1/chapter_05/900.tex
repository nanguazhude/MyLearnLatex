\newsection{仙蛊应运}    %第九百零四节:仙蛊应运

\begin{this_body}

至尊仙窍。

小北原。

“都准备好了?”雪儿坐镇雪晶大阵,细心观察着数十团寒气。

“都准备好了,太上族老。”一位雪民五转女蛊师跪在地上,极其恭敬。

“那就去吧。记住,要热情招待。当然,若有委屈也不能平白受着。”雪儿想了想,关照道。

“是。”雪民女蛊师退下。

雪儿叹了一口气。

自从她被方源安排在这里,一直致力于雪民部族的发展。在方源的雪晶大阵的辅助下,雪儿励精图治,麾下雪民部族规模翻了数番,变成了一个庞大的异人部落。

“夫君终于引进了人族……若是有闲暇,我真想去亲自看看。可惜这里脱不开身。”雪儿望着寒气,心中忧愁。

她操纵大阵,细心梳理。自从方源吞并了气相洞天之后,气道道痕大涨,带给至尊仙窍的影响非常巨大,这种影响如今还未停止,仍旧在逐渐改造着至尊仙窍的方方面面。

就拿雪儿坐镇的雪民部族来说,这里寒气凝聚,越来越多,每隔数月,就得需要雪儿亲自出手,操纵大阵进行梳理。

若是一旦让寒气积累到了某个规模,冲破极限,寒气就会酿成寒潮。

寒潮对于体弱的雪民而言,也是一场灾害。

对于周遭的环境,更会造成巨创。甚至寒潮蔓延开来,还会祸及到整个小北原。

因为界壁的存在,其他小四域倒不会受到寒潮荼毒。

就在不久前,方源传信过来,告知雪儿会有一座城池,放置在靠近雪民部族附近的地方。届时,需要雪民一族出人出力,集结队伍,带去礼物,积极交流。

雪儿当然照办,只是一直以来,她身为异人都饱受人族的打压。如今这样和人族接触,触动了她人生中的阴影。

若是可能,雪儿当然愿意看到整个小北原,都是雪民一族的领地。

但这种事情,她也明白是一个妄想。

方源毕竟是人族,而且仙窍经营到一定的程度,都会引进和栽培人族。这是仙窍兴旺繁荣的标志。

雪民部族的人马刚刚出动,与此同时,西漠中的墨人队伍到达了目的地。

墨坦桑身为墨人城城主,此刻高飞云端,静静等待。

“来了。”他忽然眼中精芒一闪。

顿时,天地间闪过一道绚烂彩光。

围绕着这团彩色华光,大地向四面八方推行出去。彩色华光越来越大,逐渐显现出一座模糊的城池幻影。

这个幻影越加真实,最终华光消散,原来的地点新增了数千里方圆的广袤土地。

一座城池位于土地中央,正是兽灾洞天中的圣鹰城。

墨人蛊师队伍已经跪拜在地上,对这样的仙迹连声歌颂,直至墨坦桑传音命令,这群人这才动身。

圣鹰城的城门打开,里面城民也早已得到通知,组成了欢迎的队伍。

两方人马在城门口相遇,正式开始接触。

圣鹰城的人族好奇地打量着黑肤白发的墨人,而墨人们也渐渐放下心中的担忧。

他们也是长期生活在人族的重压下,没想到这批新加入的人族,居然如此好说话。交谈间都是平等的态度,这让墨人们发自内心的欢愉。

“阁下便是墨坦桑大人吗?”掌管圣鹰城的巨鹰勇士,飞到墨坦桑的面前,“今后还请您多多指教。”

“好说好说。”墨坦桑笑着,“你不用太过客气,我们都是自己人。”

小蓝天。

云土聚集成一块小型的悬空大陆。

星光绽放后,这块原本空无一物的大陆上,陡然增添了一座雄城。

这座城池伤痕累累,饱受摧残,正是之前抵抗毒烟浩劫的星罗城。

星罗城的中央,已经有了一座全新的星螺殿,这是方源特意打造的凡蛊屋。

至于原先充当殿体的星海螺,则已经被方源充入库藏中。这可是太古荒兽,螺壳乃是八转的仙材。若是交给战兽王老者,他可不会处理,只能任由仙材上的道痕自然逸散。这就太浪费了。

“这里就是你父亲的世界吗?真是广大空旷啊。”战兽王老者悄悄离开星罗城,远远地绕了一圈,感慨不已。

陪在他身边的正是战部渡:“这里是小蓝天。我爹的仙窍世界中,还有九处这样的地方。我相信我们一定可以在这里生活得很好。”

战兽王老者点点头,又问:“其他人我们该怎么联络他们?”

“用这个。这是信道蛊虫。”战部渡取出一只凡蛊来,交给战兽王老者。

战兽王老者修为高达八转,但见识很是狭隘,这样一只信道凡蛊让他也惊叹不已:“没想到蛊虫居然还能这么用。”

“这才是真正的用途。我继承了的兽灾真传,也是如此。”战部渡不好意思地笑了笑,“还请战兽王爷爷勿怪,是我爹关照我的,让我保守这个秘密。”

战兽王老者点点头:“可以理解。不知令尊需要老朽如何做?”

“短时间内,还请战兽王爷爷和其他战兽勇士,都学习正统的蛊修之法。我们并不安全,在外还有大敌。”战部渡有所隐瞒地道。

战兽王老者动容:“令尊愿意将真正的蛊修大法,传授给我等?”

“这是当然的。我们是自己人啊。”战部渡笑道。

方源吞并了兽灾洞天。

蔓延兽灾洞天的浩劫,的确耗费了他大量的仙元和精力,但终究阻止不了方源。

几乎以一己之力,解决了危害洞天的灾劫后,方源本体的威望达到了顶点。

随后,在兽灾洞天之主战部渡的一力撮合下,兽灾洞天被顺利地融入至尊仙窍。

和上一世不同,方源这一世提前就做好了充分的准备,推算详尽。

吞并了洞天之后,他就将这些洞天拆分,分别布置在至尊仙窍各处地方。

兽灾洞天中有几大聚居地,分别保管了一部分的兽灾真传。

方源分别将圣鹰城、青冥谷、灵泉森林、山崖城、星罗城,安置在了小西漠、小南疆、小中洲,小北原以及小蓝天中。

当然,除此之外,兽灾洞天中还有大量的资源点,都被方源妥善安置。

消化这样的洞天,需要时间。

生态方面的融合,重新达到平衡,更需要时间,还有蛊仙们的耐心调整。

方源再次得到万物大同变,同时麒麟天灵也留了在最大的一块原址上。

方源因此收获了二十多万的变化道痕!

上一世,兽灾洞天只提供给他十几万变化道痕。但这一世,方源提前布局,有了充分的时间,因此引动了第二次浩劫。

兽灾洞天的原主人勉强撑过第一次浩劫,随后陨落。方源发现兽灾洞天的时候,它已经逼近第二次浩劫。

渡过了第二场浩劫,兽灾洞天新增十万多的道痕。

因此,方源在变化道痕方面的收获,比上一世要翻了一倍。

道痕这种东西,当然是多多益善。

“哦,冰塞川苏醒了,仍旧要和我合作?”方源接到了这个消息,只是笑了笑。

长生天方面虽然无法直接联络方源,但旋即想到了办法。

他们知道南疆正道被方源勒索过,肯定有方源的联络之法。便和其中一家南疆的超级势力商讨之后,付出一定的代价,得到了这个方法。

冰塞川立即主动联络方源,再次表示希望和方源合作。

上一世,方源还有些惊疑。这一世,方源自然心知肚明。

“冰塞川的打算,无非是合纵连横,联络所有能够利用得上的力量,抢夺宿命蛊。”

“不过,我也需要这批人进攻天庭,为我试探一番,吸引火力。”

天庭中有许多让方源忌惮的地方。

比如塔中的元莲画道手段,又比如一缺抱憾亭中的双尊对弈。

方源答应和冰塞川联手。

这反倒是让冰塞川疑惑起来:“方源答应的怎会如此干脆?恐怕他是将信将疑,只是想看看接下来的动向吧。”

冰塞川眉头皱起,觉得应该找到一个机会,最好让他和方源并肩作战,体现出自己的诚意。

但算计寻找了一番,冰塞川却没有发现什么合适的目标。

上一世,他可没有这样的苦恼。不管是光阴长河中的石莲岛,还是龙宫之争,他都能方源联手抗敌。

但这一世,方源重生改变了许多。龙宫被他提前得到了,光阴长河天庭惨败,至今还恢复不了元气,不敢踏足这个方源一力打造的禁区。

天庭。

秦鼎菱看着凤九歌,微微带笑:“说吧,你主动找上门来,是有什么要事?”

对于这个优秀后辈,未来大梦仙尊的护道人,秦鼎菱还是非常欣赏的。

凤九歌犹豫了一下,终究还是开口:“晚辈想借取前辈手中的一只八转运道仙蛊——应运。”

秦鼎菱讶异了一下,旋即恍然:“好,我答应你了。”

说着,她就将应运仙蛊当场递给了凤九歌。

凤九歌不免诧异。

秦鼎菱道:“我知道你正在开创新的杀招,这只应运仙蛊一定对你有着极大的帮助,否则依照你的性情,绝不会开口向我借这只蛊虫的。”

凤九歌点头,接过应运仙蛊:“正是如此。”

他如今已经创出八成的命运歌,发现原本的八转命甲仙蛊,难以承载恢弘的命运歌,还得需要一只运道仙蛊,充当核心。

就凤九歌所了解的情况,应运仙蛊是最合适他的。

秦鼎菱望着这只仙蛊,叹了一口气:“这只应运仙蛊的仙蛊方,其实并非是我创作,乃是我当初偷看了巨阳仙尊的书稿。将来你运用此蛊对付北原蛊仙,也算是以其人之道还治其人之身。这只仙蛊就送给你好了。”

凤九歌谢过秦鼎菱,又道:“得了此蛊,晚辈还未竟全功。需要继续前往光阴长河,观古今内外,览众生轨迹。”

秦鼎菱皱起眉头:“据我所知,天庭虽然全力赶工,但宙道仙蛊屋只搭建出了一座。你想要现在去往光阴长河,无疑是送死。那就让我看看你的运势,能否发现其中蕴藏的机会。”

记住手机版网址:m.

------------

\end{this_body}

