\newsection{想输}    %第四百七十九节:想输

\begin{this_body}

梦境之中。

擂台上,方源操纵着少年盗天的身体,正和对手展开一场颇为艰难的对战。

这一次,他的对手是一位身材魁梧的少年,他双手呈掌,砍劈之间都有一道道的电光闪烁,飚射向方源。

这个少年乃是部族中夺取冠军的最大热门之一。

若换做寻常少年来战,恐怕早就被电光触及,电得个浑身颤抖,大小便失禁,当即丧失战斗力的下场。

但是此刻,却是方源掌握着占据。

依照少年盗天自身的手段,根本不足以闪避电光的打击。但是方源老辣的眼睛,却是紧紧注视着对手的一举一动。

当他的对手掌刀高高举起的时候,方源就能从他的肩膀、眼神、整个身体重心的变化当中,推算出他即将攻击的方向。

正因如此,方源总是能闪避掉电光的打击,屡屡做出叫人看了匪夷所思的闪避动作。

“你这个家伙,就只会躲吗?”他的对手不耐烦地大吼起来。

“嘿嘿,有种的你来追我呀!你来打我呀!”方源怪叫着,尽全力嘲弄对手。

果然他的对手还是太年轻,立即大怒,再度向方源扑来:“有种的你来和我打啊,你这个无赖!”

但方源就是躲。

周围的群众都看不下去,直接起哄,斥责方源胆小如鼠的举动。

更有人喊道:“实力差距如此之大,你还不赶快认输!卑鄙的家伙!”

然而方源却我行我素,充耳不闻。

他虽然躲闪,但实际上却是在操控着这场战斗的节奏。

他的对手战斗经验并不丰富,毕竟年龄摆在这里,落入了方源的算计当中,被方源带着走还毫不自知。

少年久攻不下,真元耗费大半,气喘吁吁,感到疲累。

“好,反攻的时候到了。”方源终于等到了机会,立即反扑过去。

“嗯?!”他的对手这才反应过来,但已经晚了,很快就被方源压着打。

不过少年还是有一些真元留着,此刻艰难抵挡。

方源心中暗道不妙。少年盗天的资质很差,只有丁等资质,真元储备就很少。

此刻交战良久,方源能够掌控的真元,在规模上还不如一直强攻猛进的对手!

“必须速战速决!”方源想到这里,立即大吼,“来,傻小子,吃小爷一口浓痰先!”

说着,便张开嘴巴,吐出一口吐沫,直飞对手脸面。

他的对手顿时双眼瞪圆。

这要是被浓痰吐在脸上,那多恶心啊!

少年连忙躲闪。

但这正是方源想要的结果。

他就是想要逼迫他的对手,往这个方向躲闪。

“你也下去吧。”方源忽的脚步一垫,如灵猫窜动,闪到对手的面前,然后矮身抬脚,狠狠一踢。

那少年已经将双臂竖起来,宛若盾牌一样,挡住方源的踢击。

但方源力道极强,直接把他踢飞,一直落到擂台下面去。

方源因此获胜。

“这小子居然胜了!”

“这还有没有天理了啊。”

“真是恶心,他居然吐痰,太不讲究了。”

“是啊,就算他胜了,我也要鄙视他。真的是一点风度都没有,简直不配当蛊师嘛!”

观战的群众爆发出一阵阵的热议。

针对方源的嘲讽、指责的声音,汇集成声浪,一*地传达到他的耳中。

方源却在此刻,丧失了对身躯的操控权,轮到真的少年盗天出场。

少年盗天亲身经历了这场战斗,的确是艰难,获胜很不容易。

毕竟少年盗天独自一人,而他的对手却是有着背景,大多是父母双亲基本上都是蛊师,或者至少其中一份,乃是蛊师。

这一次部族小比,能够参加的都是刚刚成为蛊师的少年。这些人因为底子太薄,只是一转,所以非常容易就能借助外力,将自身的战力拔升上去。

因此,那些富二代、官二代们,都或多或少受到背后长辈的爱戴和指点,战力得到许多提高。

这是方源战斗艰难的原因,同时还有第二个原因,那就是方源的战斗风格,已经广为人知,他的对手们都很警惕,每次战斗都有周密的防备,让方源难以下手。

“我、我怎么会吐浓痰?”少年盗天站在擂台中央,差点羞愧得要落泪了。

前世的他一直很注重仪态和风度,随地吐痰这种事情太不文明,他绝不会做。

但就在刚刚的那场战斗中,他不仅做了,而且还当着这么多人的面,堂而皇之,勇气十足地坐了。

这可让少年盗天情何以堪!

“我怎么会是这样的人?这真的是我内心最深处的写照吗?”少年盗天陷入到深刻的自我怀疑和否定当中。

周围的群众仍旧在嘲讽。

“这个小子心里高兴坏了吧!”

“居然让这种货色,闯入了小比八强。”

“哼,世风日下,这种人就应该直接被淘汰!”

“不过八强也就是极限了,我再也不想看到他再次获胜了。”

少年盗天:“……”

他很想分辨,告诉大家自己真的不是那种人。但他说不出话来,因为他现在自己都怀疑了,都动摇了。

前世养成的价值观、荣耀感,遭受一次又一次的轰炸,如今已经变得面目全非。

“可以了,我再也不想晋级了。这一次获胜,有了八强身份,就能够进入池塘那边。”少年盗天很累了,心累!

他不想再继续战斗下去了。

但是方源却很有这方面的想法。

“八强、四强、两强,直至桂冠,得到的族中奖励都不同,层次分明。”

“这里虽然是梦境,不是真实。但我若是帮助少年盗天,取得更好的名次,无疑就能获取更佳的奖励。”

“而这些奖励,恐怕也会顺延下去,让这片梦境中的少年盗天变得更强大。”

这个盗天梦境,非同寻常,乃是活的,没有固定死的标准。

因此少年盗天越强,他对梦境的影响也就越大,对于方源接下来探索梦境,更加方便。

这一次探索盗天梦境,解梦杀招效果不佳,等若是断去方源一臂!

所以方源要尽全力增长手中的筹码,提升自己探索梦境的成功可能。

“第二幕梦境,比第一幕要更加冗长,我应该尝试一下,争取第一。这也是对这等梦境的一个试探吧。”

方源思考了一阵后,便打定主意。

只是,他心中想法虽好,但掌控少年盗天身躯的机会和时间都是有限的。

方源也只能耐下心来的,等待时机。

梦境继续演绎下去。

少年盗天虽然获胜,但收获的却都是骂名、臭名,外在压力非常巨大。

别的少年正在积极准备小比,搜集对手的资料情报,少年盗天却在剖析自己,自我审查。

事实上,他都有些不敢面对自己,怀疑自己是一件非常痛苦的事情。

这导致少年盗天日渐消瘦,变得相当萎靡不振。

沙枭的声音传来:“你应当炼蛊了,孙子,没有新的蛊虫,接下来你很难应付得了对手,将会败多胜少。”

炼制凡蛊,和炼制仙蛊,那是两个完全不同的概念。

尤其是炼制一转蛊虫,成功的概率颇大。

“我不能直接提供你任何直接的用物,所以蛊虫我也不能给你,只有让你自己炼制出来,才会掩人耳目,不让人发现我。”沙枭又道。

说完这话,他就再度向少年盗天的脑海中,灌输了一大段的情报。

少年盗天看来,却是勃然变色:“不行,这种蛊虫太过阴损,居然要屠杀婴儿,夺其魂魄,当做蛊材。我是绝对不会做这种有伤天和的事情!”

ps:今天只有一更,太累了,状态十分不佳。(未完待续。)

\end{this_body}

