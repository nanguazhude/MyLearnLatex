\newsection{死后还背锅}    %第七百四十五节:死后还背锅

\begin{this_body}

%1
定仙游仙蛊终于到手了!

%2
如此一来,方源将来晋升八转,未来身杀招无效,也不用担心损失定仙游这类的手段。

%3
定仙游是用来移动的仙蛊,效果相当强大,会带给方源很多便利。

%4
上一世,方源推算出了杀招,以定仙游为核心,即便是八转修为,也能勉强带动他满世界乱窜。

%5
有了这只仙蛊,就意味着有了极其强大的机动性!

%6
不过在上一世,方源第一次就偷盗了定仙游仙蛊,但这一世却是费心劳力得很,使用大盗鬼手前后超越十次,耗费了大量的仙元。

%7
大盗鬼手杀招的缺点显露无疑。

%8
它虽然可以盗取仙蛊,但更多时候会偷来凡蛊。方源在凤九歌身上偷来的凡蛊数量多达数万,而仙蛊只有三四只。

%9
“接下来就是命甲仙蛊了。”方源冰冷的目光紧紧锁定凤九歌的身影。

%10
只要偷来命甲仙蛊,凤九歌也就完蛋了。

%11
凤九歌不断地逃窜。

%12
他从未想过,自己居然会有如此狼狈不堪的时刻!

%13
“方源对我非常了解!他一味地偷盗我的仙蛊,始终很少攻击我,毫无一丝重创我的企图,看来是明白命甲仙蛊的威能效用。”

%14
凤九歌一颗心沉入谷底,一股冰寒彻骨的冷意弥漫他的全身。

%15
随着他仙蛊越来越少,他战力直线下降。

%16
“方源!”凤九歌在心中咀嚼着这个名字。尽管他对方源的成长早有估料,但是绝没有想到自己此刻狼狈逃窜,毫无还手之力的情景。

%17
之前凤九歌追杀方源,把方源从南疆一路撵到西漠的经历,似乎就发生在昨天,凤九歌历历在目。

%18
没想到此刻,他们俩的境况就发生了彻底的颠倒!

%19
“何其速也!”凤九歌心中苦叹。

%20
不是凤九歌不强,而是方源实在是准备得太充分。

%21
重生的秘密他没有暴露,优势被他保持下来。

%22
琅琊福地的这座超级大阵,得到了方源的全力改良,比上一世同期还要强大数倍。

%23
原因之一是方源吞并了琅琊福地,能够调动一切的资源来建设仙阵,同时和三大异族都有交易。

%24
原因之二是方源本身也比上一世同期更加强大,不管是阵道、宙道等等方面,都更加优秀。

%25
这才组建出了这座超级大阵,围困住了凤九歌和陈衣。

%26
另一边。

%27
陈衣眼前光影闪烁,他在迷宫幻境中不断地穿梭。

%28
“真是该死!我竟然还未冲破出去!”陈衣陷在迷宫幻境之中,迅速转移,四处突破,都快要吐了。

%29
这座大阵屡次刷新了陈衣对它的心里评价。

%30
方源安排了几乎所有的异族蛊仙,操纵大阵,专门针对陈衣。

%31
陈衣的战力很强,仙道杀招有古木牢笼、巨木神像、因果神树等等,都是非常优秀的手段,方源处理起来都十分棘手。

%32
上一世陈衣在琅琊大战中,以一敌二,对战太古石龙、天婆梭罗,表现非常亮眼,风采夺人。

%33
但这一世,方源吞并了琅琊派,真正夺得了指挥权。他安排太古石龙、天婆梭罗对付陈衣,但两者只是辅助而已。真正对付陈衣的是超级大阵。

%34
陈衣在迷宫和幻境中挣扎,每当他有突破的希望时,太古石龙或者天婆梭罗就会出现,阻止他,拖延时间。

%35
然后异族的蛊仙就会乘机营造和弥补,将陈衣始终困在大阵之中。

%36
陈衣手中一大把的手段,可惜有力无处使。

%37
方源针对性的布置,一下子克制住了他。

%38
陈衣相当焦躁,时间拖延得越久,他越是不安。

%39
之前雷鬼真君尸躯真的只是幻境吗?

%40
陈衣不敢肯定!

%41
他当然愿意相信:这只是幻像。

%42
理由直接简单——依凭雷鬼真君的战力,方源怎么可能将她一下子斩杀?

%43
但有没有这种可能性呢?

%44
如果雷鬼真君还活着,为什么交战至今,时间过去这么久,还没有听到她的一丝动静?

%45
现在自己和凤九歌分割开来,已经很久了,凤九歌那边情形如何呢?

%46
“难道是梦境?”陈衣脑海中灵光一闪。

%47
他并不知道方源手中具备八转仙蛊偷生,但他却早就了解到:方源很擅长纯梦求真体的自爆,利用梦境本身的威能对付强敌。

%48
义天山梦境大战,魔尊幽魂被天庭俘虏。其后,方源被武庸、凤九歌接连追杀时,屡次用纯梦求真体自爆。

%49
没有纯梦求真体的帮助,方源很大可能就被追上杀死了。

%50
紫薇仙子身为智道大能,在行动之前,早就有过推算。方源手中的梦境、逆流河,都已经考虑进去。

%51
黑暗漩涡是星投的引子,本身也有感应的手段。周围若是有梦境或者逆流河,紫薇仙子也能隐有感应。

%52
事实上,方源向来缺乏对梦境、逆流河的精准控制。纯梦求真体一旦自爆,形成的梦境会自行流转,逆流河同样只能粗浅调动。

%53
不管是梦境,还是逆流河,只要沾染上黑暗漩涡,就能破坏它,从而促使星投杀招不稳,乃至失败。

%54
紫薇仙子只要撑大一点黑暗漩涡,就能试探出漩涡周围有无逆流河或者梦境了。

%55
“单纯用梦境或者逆流河做陷阱,是愚蠢的。但是若是方源开创出了新的梦道杀招呢?他可是有着春秋蝉,从前世重生回来的啊!”陈衣心道。

%56
但事实上,方源现在手中根本就没有梦境。

%57
他手中的纯梦求真体就已经消耗光了。(详情见本卷第五百六十七章。)

%58
绝大多数的纯梦求真体,都在方源躲避追杀的时候用了。

%59
这才有了上一世琅琊福地防卫战后,方源回到南疆再夺梦境的战斗。

%60
再后来方源和池曲由暗中交易,陆陆续续得到了许多梦境,手中的纯梦求真体这才囤积起来,形成规模。

%61
在此之后,方源埋伏夏槎等追兵,这些纯梦求真体立下奇功。可以说池曲由的功劳很大,正是因为他暗中和方源交易,方源才有了这么多的梦境可用。若是单靠强抢豪夺,根本没有这样的高效率。

%62
陈衣猜测失误,心情越来越沉重。

%63
“不行,我必须尽快将这座仙阵打破!”

%64
严峻的形势让陈衣下定了决心,他催出王牌杀招——因果神树。

%65
神树苍翠葱茏,华盖如亭,青烟袅袅,随风招摇。

%66
因果神树顶在头顶,周围的种种幻境、迷宫都被逼退开去一大段距离。

%67
陈衣眼前顿时一片空明。

%68
异人蛊仙们发出一阵惊呼声。

%69
“这就是方源长老所说的因果神树杀招?”

%70
“整座超级蛊阵的力量,都在被此招排斥,迷宫和幻境根本罩不住陈衣头上。”

%71
“不愧是元莲仙尊的杀招啊。”

%72
危急关头,太古石龙、天婆梭罗一左一右,杀向陈衣。其余异人蛊仙则加紧操纵大阵,企图找到重新迷惑陈衣的方法。

%73
陈衣冷哼一声,并不和石龙、巨人交手,而是四处疾飞,打量着大阵,尝试对大阵下手。

%74
太古石龙、天婆梭罗强攻陈衣,陈衣被动防守。

%75
因果神树杀招展现出极强的防御力,将所有的攻势都完美的遮挡下来。

%76
原本还稀疏的神树上,很快就结满了果实。

%77
这些果实一个个地飞射出去,打中太古石龙和天婆梭罗,令他们承受自己造成的伤害。

%78
“快快拦住陈衣,不要让他乱窜了。”方源意识到不妙,连忙下令。

%79
但异人蛊仙们忙得焦头烂额,也无法借助大阵的力量来迷惑陈衣。至于让他们亲自出战,以身赴险,也是不可能的。

%80
异人蛊仙们虽然是盟友,但都很惜命。他们躲在大阵中能不出来,就不出来,上一世直到大阵破败了,他们才不得已置身战场。

%81
轰!

%82
一声巨响,陈衣打中关键的点,将分为两半的大阵空间再次打通。

%83
陈衣虽然专修木道,但他到底是八转大能,触类旁通,再加上接触这座超级大阵时间久了,自然能看出一些奥秘之处。

%84
陈衣看到凤九歌立即惊呼一声,口中大叫:“坚持住,我来了!”

%85
原来凤九歌已是被方源逼入角落,情况万分危及。

%86
“小心!敌人有直接盗取蛊虫的手段。”凤九歌也在瞬间反馈给陈衣关键的情报。

%87
陈衣浑身一震,他猛地联想到了豆神宫之争。西漠蛊仙算不尽就直接偷道了太古传奇青仇的仙蛊。

%88
“他们也有偷道手段?!”陈衣心中咯噔一下,眉头紧皱。

%89
但和上一世不同。

%90
方源没有用出阎帝杀招,陈衣将两者联系起来,只是一种自然的反应。并没有任何的证据证明,方源就是算不尽。

%91
偷道流派在五域中也有不少的流传。

%92
西漠算不尽有偷道手段,琅琊福地具备偷道杀招也更有可能啊,毕竟盗天魔尊可是曾经和长毛老祖合作炼蛊的。

%93
陈衣杀来,方源冷笑一声,直接迎上去。

%94
他催动逆流护身印,同时又开始酝酿落魄印。

%95
方源身上气势陡然暴涨。

%96
“两大八转杀招?!”陈衣吓了一大跳,瞳孔猛缩,“一攻一防,相得益彰!该死!”

%97
方源和陈衣拼杀在一块,而太古石龙和天婆梭罗则找凤九歌的麻烦。

%98
凤九歌目前只有七转修为,本身仙蛊都被方源盗取了好几只,战力下降得相当厉害。面对两大八转战力夹攻,他很快负伤累累,被打得命甲仙蛊激发出来。

%99
天庭那边,紫薇仙子脸色剧变:“不妙!命甲仙蛊激发了,凤九歌危在旦夕,这可如何是好?”

%100
上一世,紫薇仙子和凤九歌只隔了一个阎罗战场,紫薇仙子迅速破解了战场杀招,把凤九歌救下。

%101
但这一世,紫薇仙子和凤九歌的距离,可是极其遥远的。

%102
而星投杀招才刚刚酝酿了一半而已。

%103
凤九歌绝不能死,他可是护道人!

%104
陈衣也知道凤九歌的重要性,但他爱莫能助,方源死死地缠住他。因果神树很强,攻防一体,但方源手持两大八转杀招,亦是凶威赫赫。

%105
时间迅速流逝,命甲仙蛊的薄薄光晕渐渐暗淡下去。

%106
一旦光晕散去,就是凤九歌丧命之时。

%107
“怎么办才好?!”陈衣焦急万分,就在此刻,他的耳畔响起了元莲仙尊的声音:“看好了,这是基于因果神树基础上的连招——来因去果!”

%108
杀招催动,仿佛风起,青烟随风摇晃,因果神树仿佛在风中飘舞。

%109
方源连忙收住杀招,同时心中松了一口气,总算将此招也逼了出来。

%110
有了来因去果,陈衣纵横战场,方源只得四下避退。

%111
陈衣迅速和凤九歌汇合:“回去吧,这是我的仙蛊,记得将这里的战况告诉紫薇仙子大人,将来给我报仇!”

%112
陈衣将用不上的仙蛊,都交给了凤九歌。

%113
下一刻,来因去果杀招发动,将凤九歌直接送走。就算是有大阵,还有琅琊福地的窍壁,也无法阻挡凤九歌的撤离。

%114
来因去果杀招毕竟是九转层次的手段!

%115
“凤九歌这一次让你逃了,下一次可就不一定了。”方源脸色如常。

%116
他知道来因去果,也清楚此招他无法阻挡。

%117
之前分割陈衣和凤九歌,就是为了防备这个杀招,可惜没有成功。

%118
陈衣、凤九歌都十分精明,战斗经验非常丰富。

%119
“走了凤九歌,你就纳命来吧,正好和雷鬼真君做个伴儿。”方源露出狞笑。

%120
陈衣淡笑一声:“雷鬼真君前辈果然是被你们杀了。来吧,就算是死,我也不会让尔等好过。”

%121
陈衣明知必死无疑,反而彻底放开。

%122
他虽然迷恋权势,但同样是天庭的一份子。

%123
只要是天庭的成员,都不怕牺牲,视死如归!

%124
斩杀陈衣的过程并不容易,非常困难,很长时间都是僵持的战局。

%125
但最终,陈衣授首,被方源夺走了性命。

%126
他没有留下任何仙蛊,但是残魂无法逃脱方源的魔爪。

%127
异人蛊仙们欢呼胜利,随后又失声痛哭起来。

%128
他们损失惨重。

%129
被陈衣斩杀了许多同伴。

%130
方源望着陈衣的尸体,深深地叹了一口气:“不愧是陈衣,不愧是天庭蛊仙啊!”

%131
随后,他又带着深切的悲痛安慰异人蛊仙们:“你们的牺牲是值得的,是荣耀的。就连八转蛊仙都死在你们的手上,这样的战绩说出来,会惊天动地的。”

%132
异人蛊仙们只剩下两三人,且各有重伤。

%133
“天哪,我们这一次的损失,真的太严重了。”

%134
“琅琊福地反而受损最少。他们都结阵去了,在天婆梭罗之中很安全。”

%135
“奇怪……为什么大阵会突然间崩解掉好几层呢。若不是这个变故,我们绝不会损失这么多人!”

%136
方源点点头:“这应当是陈衣的手段,真是厉害,我都没有发现丝毫的端倪!”

%137
异人蛊仙们闻言,都通红着双眼,咬着牙关:“是啊,就是这个老贼!”

%138
“杀了他真是便宜他了!”

%139
“他死不足惜!”

%140
可怜陈衣,不仅战死在这里,死后还为方源背锅。

\end{this_body}

