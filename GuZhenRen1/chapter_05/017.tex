\newsection{抢炼定仙游}    %第十七节:抢炼定仙游

\begin{this_body}

青光煌煌,赤芒似火。

两光相互交融,此盛彼熄,此弱彼强。

光辉渐渐消散,整个大阵的中央,又再度陷入一片黑暗之中。

黑楼兰盘坐在半空,缓缓地睁开双眼,眼底深处闪烁过一抹惊异钦佩之色。

“影无邪隶属影宗,是魔尊幽魂的下属。我原本以为,他会动用魂道或者暗道手段,来剔除我身上的信道道痕,帮助我解除盟约。没想到,居然是用的光道。可见幽魂魔尊、影宗的深不可测,小姨妈的信道造诣,已经是准大宗师级。影宗方面,布置出来的这个蛊阵,却可以用光道破解之……”

黑楼兰心中,不禁又将对幽魂魔尊,还有影宗的评价,再度提高一层上去。

眼前视野骤变,下一刻,黑楼兰又回到了她之前身处的地下陋室之中。

“你身上原有的道痕,已经被我尽数消除了。”影无邪站立一旁,语气平淡地道。

经过一番深思熟虑之后,影无邪最终决定保留下黑楼兰的性命。

毕竟,他现在最主要的任务,还并非是对付方源,而是尽快地收拾残局,重建影宗,拯救魔尊幽魂的本体。

方源只是次要目标。

影无邪虽然对方源深恨至极,但他不会因为仇恨而丧失了理智。

正因如此,他主次分明,所以犹豫了一番之后,决定将黑楼兰纳为己用。

这个人才,可不一般。

大力真武体,凭借影无邪的见识和眼界,稍加调教,就能成为麾下得力的干将打手。

黑楼兰检查全身。

他还未有观察出自身道痕的手段,此番检查,只是看看身上有无异处。当然重点更在于魂魄上。

谁都知道,幽魂魔尊可是开创魂道的祖师爷!

影无邪淡淡地笑了笑:“旧的盟约已经铲除,你的身上已经有了新的盟约。从今以后,你就是我的麾下。必须听从我的命令,不得违背,任何情况都不得隐瞒不报。”

黑楼兰冷哼一声,没有给影无邪什么好脸色看。

既然影无邪有能力驱除旧有盟约。证明他的手段,比黎山仙子还要厉害。

从今以后,黑楼兰的自由就丧失殆尽,一举一动都要听从影无邪的命令。

不过……

对于黑楼兰而言,也是不得不做出的无奈选择。

形势比人强。她若顽抗到底,必然死于影无邪之手。

至于背叛曾经的盟友方源,黑楼兰根本毫无一丝的愧疚留恋之情。 本来她和方源之间,就是相互算计,相互利用的人。

甚至,她对方源还有仇恨!

“下一步,你要我做什么?把方源哄骗引诱过来吗?我这里有星门蛊,可以借助黑天穿梭五域。”黑楼兰淡淡地道。

没有了盟约之后,她毫不留情地,就将方源给出卖了。

但影无邪却摇头。遗憾地道:“不怕告诉你,我影宗在琅琊福地还留有内应。我们也早知道星门蛊,此蛊虽然巧妙绝伦,但最多只能对那些六转垫底的蛊仙有效,连一只仙蛊都传输不了。更遑论现在的方源了。”

影无邪得到红莲真传中的魔尊意志指点之后,知道了很多关于至尊仙胎蛊的秘密。

他心中很清楚,方源的新肉身上,有着大量的道痕。炎道、水道、风道等等,堪称应有尽有。并且每一道道痕,都不下百条。

如此。这些道痕数量叠加起来,已经逾千。

身怀这么多的道痕,方源根本无法再运用星门蛊了。

“我下一步的计划,是要炼出定仙游。”影无邪道。

黑楼兰眼中精芒一闪。点点头。

仙蛊定仙游她当然知道,此蛊相当实用,影无邪决定首先炼制此蛊,无疑十分明智。

“定仙游才刚刚毁去不久,此时炼制自然把握很大。而要炼制定仙游,首先需要神游蛊。要得到神游蛊。就得有四种绝世美酒……”

黑楼兰还未说完,就被影无邪打断,道:“神游蛊正在琅琊地灵的手中。不过定仙游是仙蛊唯一,但仙蛊方却不是这样。一座城池,至少有东南西北四条大路通向,仙蛊也是如此。”

黑楼兰奇道:“这么说来,你有炼制定仙游的新蛊方?”

“这是当然。”影无邪自得一笑。

幽魂魔尊生前天下无敌,死后又建立影宗,积累和底蕴深厚无比。

红莲真传中的魔尊意志,根据定仙游这个仙蛊,为影无邪制定了一系列的计划安排。作为计划的核心,定仙游自然不能有问题。所以影无邪的脑海中,还记着足足三张仙蛊方。

都是能成功炼出定仙游的蛊方!

南疆。

方源在山林中漫布。

他刚刚以一人之力,屠尽了倪家山寨,此刻一边行走,一边在回顾刚刚战斗的得失。

前事不忘,后事之师。

唯有不断地整理经验,并且从经验中汲取出精华,补给自身,日后行走方能更加稳妥。

倪家全族的蛊师,不足为虑,方源只是动用了奴道的些许手段,就召来了万兽,将山寨踏平。

不管凡人蛊师有多少,至始至终,都未让方源出手。

而让方源亲自动手的,则是那头后来出现的荒兽泥怪。

就是这头泥怪,让方源颇费手脚,才收拾掉。

从这一战,方源察觉到自己许多的不足之处。

“我虽然有七转剑蛊,但对仙元的消耗实在太大!”

蛊虫有九转,蛊师也有九转,自古以来,蛊师修行都有一个共识,蛊虫和蛊师之间的转数应当一致。蛊虫转数低了,蛊师实力发挥不尽。蛊虫转数高了,蛊师的实力也发挥不出来。

皆因蛊虫转数高逾蛊师,蛊仙就宛若婴儿抡大锤,且不说抡锤伤人,本身提拿大锤的时候,对蛊师本身就是一种损伤。

就好像不久前。方源动用七转剑蛊,对付荒兽泥怪,就有一种力不从心之感。

每一次催动剑蛊,消耗的青提仙元。就超越一百。

方源就算是九五至尊仙窍,每天能产青提仙元十六颗,也经不住一次催动的。

幸好他从琅琊地灵那里,先行提取了不少仙元石,为自身补充了足够多的仙元。否则倪家山寨中。对付那头荒兽泥怪,还要出一大丑。

“我原先还有一些犹豫。不过现在看来,将这些七转剑道仙蛊逆炼,已是必行之事了。”

方源心中叹息。

蛊虫自然是转数越高,越是稀罕,价值越高了。

但方源为了自身实际情况,迫不得已,只能将这些七转仙蛊逆炼,将它们的转数从七转降至六转。

如此一来,方能更加适合方源。由他随意催动。

“我炼道造诣,也就马马虎虎。逆炼仙蛊的事情,还是拜托琅琊地灵出手最好。”

“他虽然已经变化个性,但仍旧是长毛老祖的执念,只要我愿意耗费门派贡献,不难请他亲自出手。”

方源心中不断思量。

逆炼仙蛊,虽然比正炼的成功率,要高一些。但失败的可能性,仍旧很高。

所以请琅琊地灵出手的话,比方源自己动手。更有把握一些。

“不知道我的这些七转剑道仙蛊,统统逆炼之后,还能剩下几只?”

“就算全部剩下,光有这些剑道仙蛊。也不行。”

方源又想起之前,他和荒兽泥怪的战斗。

虽然他催动了七转剑道仙蛊,但是对手却是泥怪。

泥怪这种存在,和其他猛兽不同。寻常的荒兽,有血有肉,有皮有骨。有着明显的致命弱点。

但泥怪不一样,它浑身上下都是烂泥。只是充斥土道道痕,才让它与众不同。

方源的飞剑仙蛊,擅长致命打击,催发出来,只是一道一尺长,几指宽的银剑。虽然锋锐无当,穿透泥怪轻而易举,但却无法造成巨大伤害。

所以那一战,方源虽然一直占据上风,但却憋屈得很。

最后即便胜了,消耗的青提仙元,也大大超出了方源的底线。

“飞剑仙蛊乃是薄青最常用的仙蛊,薄青以其为核心,创造出许多仙道杀招,专门应付各种情况。之前一战,我若是有剑道杀招,能将飞剑仙蛊的力量分化无数,形成漫天剑雨,荒兽泥怪恐怕不是我一招之敌!”

所以,方源接下来,不仅要逆炼剑道仙蛊,还要收集剑道杀招。

至于他自己创造,那就别想了。

智慧蛊他用不了,或者说还未找到应用的方法,剑道境界又是几乎空白一片。

若非有这些剑道仙蛊撑腰,方源早就走血道或者力道、宙道了。

思绪甫定,方源便停下脚步,在一处青石上随意坐下。

联系琅琊地灵。

现在的琅琊地灵,对炼蛊的兴趣缺缺,更大的**,是如何发扬光大毛民一族,将人族踩在脚下,让毛民成为五域霸主。

但是在方源愿意耗费门派贡献,请地灵出手的情况下,琅琊地灵慨然应允。

至于剑道杀招,琅琊福地中竟也有不少收录。

原来之前的琅琊地灵,整日整夜钻研蛊方,研究炼蛊。剑道蛊虫自然也在其研究之列,而那些剑道杀招,也是琅琊地灵可以借鉴的内容。

看了一眼,方源也有些拿捏不定,该换些什么杀招。

他话锋一转:“太上大长老,我愿意用最后一次机会,请你炼制一只仙蛊。”

琅琊地灵虽然变了,但盗天魔尊是和长毛老祖定下的约定,所以地灵仍旧要遵循。

方源已经消耗了前两次机缘,分别让琅琊地灵炼制出了星门蛊、第二空窍蛊。现在他终于下定决心,要将最后一次机会,也用掉。

“哦?是什么仙蛊?”地灵问。

方源的回答干脆利落

“定仙游!”

ps:最近力不从心,更新速度一直上不来。我深入思考了一番,发现是大纲不明的问题。三百多万字,书是越往后越难写,很多前面的东西我也忘了不少。所以设计新情节的时候,顾虑重重,生怕出现什么bug。所以这些天,我都在加紧时间整理大纲。大纲整理出来之后,速度才会加快。这里先向诸君说明一下原因。

\end{this_body}

