\newsection{宴请池伤}    %第三百一十一节:宴请池伤

\begin{this_body}

%1
酒宴开始了。

%2
这间小殿中,只有方源、武安、池伤三位蛊仙。武安当然是陪酒的酱油角色,重点当然是方源和池伤的互动。

%3
方源主动向池伤敬酒,言辞客气,面带微笑,尤其是话语间对池伤的阵道造诣非常推崇。

%4
伸手不打笑脸人。

%5
池伤稀里糊涂,懵懵懂懂,就这样喝下了不少的美酒。

%6
他打量方源,开始用一种不一样的视角来看待。

%7
“武遗海这家伙,其实挺和气的呀。之前怎么搞的?是故意激将我吗?还是他还要有求于我?”池伤心中嘀咕。

%8
“不瞒你说,池伤仙友,我平生最佩服的就是你这样有才华的人呐!”方源大笑,走过来敬酒。

%9
池伤呐呐,举起酒杯,方源还没喝,他就直接自己灌下一口。

%10
原来之前方源连续敬酒,已经成了池伤的条件反射。

%11
不过方源这话,真的听着舒服啊。尤其是方源情敌的身份,如此恭维赞美,让池伤有了一种巨大的成就感。

%12
“不。”

%13
“我要警惕!”

%14
“武遗海这个家伙也可能在演戏,他还是很奸诈的。”

%15
池伤没有放下心中的戒备,不过他怎么看,怎么觉得方源的满脸微笑,都是真挚和诚恳。

%16
呵呵,且不说方源的演技,光是态度蛊岂是说笑的?

%17
“池伤仙友请看。”方源伸出手,递给池伤一只信道凡蛊。

%18
“这是什么?”池伤心中嘀咕,心神探入一瞧,脸色骤变。

%19
原来,这只信道凡蛊是方源即将要交给乔丝柳的信。但在信中,方源却对池伤大加赞赏,赞美之情洋溢而出,更坦诚自己不足,先前用阵道难题来挑战池伤,纯粹是不自量力,班门弄斧,不晓得天高地厚,不值得南疆英雄。

%20
当然,这世界中是没有“班门弄斧”这个词语,总之是这个意思了。

%21
池伤看了这信,他自己都有点不好意思起来。

%22
方源的赞赏,言辞的确有点夸张。

%23
“我真的这么优秀吗?”

%24
“南疆英雄的称号,我从来没有这么想过呢。”

%25
“不晓得丝柳仙子看到这封信,会怎么想?”

%26
各种念头纷纷涌现,池伤心情无法不雀跃。

%27
当他再看向方源的时候,目光又发生了一些转变。

%28
方源直接在池伤的身边坐下来,这个亲昵的动作,居然让池伤没有反感。

%29
“池伤仙友,我纵观南疆蛊仙界,恐怕也只有你的才华,能配得上丝柳仙子的。这点我服了!”方源大笑。

%30
“哦?”池伤不知道说什么好,脸上有喜悦,但也残留着怀疑。

%31
方源又道:“但我不会放弃的,窈窕淑女君子好逑,我虽然才华不如你,但我会加倍的努力。这一次是你胜我一筹,咱们今后公平竞争!你看如何?”

%32
听了这话,池伤脸上的怀疑这才冰消瓦解,他第一次举起酒杯,敬向方源:“好,公平竞争!君子之间,就应该公平竞争。”

%33
言下之意,已经将方源当做君子。

%34
方源哈哈大笑,拍拍池伤的肩膀:“老实说,其他情敌我都很反感厌恶,但是池伤你不同,你才华横溢,光明堂皇,我很佩服你。如果将来你得到了丝柳仙子的青睐,我绝对会祝福你们两个,因为我知道,我若输给你,我不冤枉!至于其他人,哼,都是些跳梁小丑。”

%35
池伤被夸得很不好意思,摸了摸鼻子,讷讷地道:“其实其他人也有一些强者。”

%36
“咱们是不打不相识,来,喝一杯!”方源又举起酒杯。

%37
池伤这次动作更加干脆:“喝!”

%38
他一仰头,咕咚一下,喉结滚动,一大杯酒直接下了肚。

%39
然后他打了个酒嗝,脸上通红起来。

%40
“豪爽!”方源竖起大拇指,亲手又给池伤斟满酒。

%41
他笑容亲切,满怀热情,非常诚挚。

%42
但实际上,他的心中一片冷静,回想起关于池伤的情报。

%43
武家的情报渠道,值得信任。

%44
方源早已知道:这个池伤从少年之时,在一场意外中,被池曲由看中,认识出他在阵道上的天赋,于是就大加栽培。

%45
池伤在池家,有“阵痴”的名号。

%46
他从少年时期就被选拔,送入到万蛇山的后山洞中进修。这一进就是三十年。

%47
当他已经四十多岁的时候,这才出山,一身修为只有三转,但却在族中大比,布置蛊阵胜过四转、五转的蛊师,阵道造诣惊呆了池家族人。

%48
到后来,族人才了解,池伤之所以破关出山,竟然是因为他迷路,意外地走了出来。

%49
他在大比的表现获得了名声,财富和美色接踵而来,但他都一一推拒,唯一的请求竟然是再入山洞,继续闭关,研究阵道。

%50
此事惊动了太上家老。

%51
池曲由允许池伤的这个请求,甚至为他单独开辟了一间洞窟,每天都有专门的仆从服侍。

%52
池伤继续钻研,其刻苦和痴迷,让人骇然,仿佛他人生的唯一乐趣便是钻研阵道。

%53
他一头扑在上面,任由外界喧嚣,声色犬马,他只生活在自己的洞窟里,不见天日,终日里沉浸在阵道的修行上。

%54
等到他满头白发,大限将至的时候,池曲由再次见他,问他:这一辈子过去,后悔吗?临死有什么遗憾?

%55
池伤哀嚎大哭,他说:“人生苦短,匆匆百年,我的阵道研究还有许多许多呢,怎么可能甘心?这就是最大的遗憾。”

%56
池曲由感慨万分,竟然将一只寿蛊赐予池伤。

%57
池伤得之,恢复青春,又有了寿命。

%58
池曲由和他谈论阵道,指点了他三天三夜,由此终于确认,池伤乃是家族希望,未来阵道造诣可达大宗师!

%59
池曲由不仅仅只是说说,而是在此之后真的倾斜家族资源,对池伤大力栽培。

%60
池伤在家族的帮助下,修为突飞猛进,晋升成仙。并在短短数十年里,成为七转蛊仙。

%61
不过就算成仙,也没有让他脱离之前的生活习惯。

%62
他仍旧是扑在阵道上面,不断钻研,没日没夜,一闭关苦修就是数年,十数年。

%63
唯一让人惊讶的,就是他追求乔丝柳,并且态度十分坚决。

%64
这个消息当时传出来,让池家人大跌眼镜,这才知道除了阵道之外,池伤还喜欢女人!

%65
得到武家情报的时候,方源就觉得池伤是一个很有意思的人。

%66
正是这些认知,才促使方源采取了比较特殊的应对方式,来针对池伤。

%67
之前没有主动和池伤见面,实际上乃是方源的计谋。故意用阵道的难题挑战是方源的下一步计划,这一次宴请喝酒当然也在方源的计划当中。

%68
池伤的表现,都没有出乎方源的意料。

%69
一切都有条不紊地进行。

%70
前不久乔丝柳的来信,也在方源意料当中。作为当事人之一,乔丝柳不可能不表明自己的态度。

%71
她虽然没有明说,但是在信中却告知方源,她当初和池伤第一次见面的情景。

%72
按照乔丝柳所言,当初她带着乔家使命,造访池家,受到池家太上大长老之子蛊仙池谤的招待。

%73
哪知这池谤言语轻佻,故意借着美景,口述诗词,对乔丝柳暗含挑逗之意。

%74
若是平时,乔丝柳也未必恼怒。但当时,乔丝柳却是乔家使者,池谤行为是对乔家的不尊敬,这让乔丝柳心中怀恨。

%75
为了应对,乔丝柳便故意接近池伤。

%76
池谤乃是池曲由之子,池家内定的下一任太上大长老,可谓一人之下万人之上,别人都不敢招惹,除了池伤这个阵痴。

%77
池伤不明其中弯弯道道,起初对乔丝柳不假辞色,但乔丝柳却故意谈及阵道,述说乔家的一些阵道典籍。

%78
池伤被其吸引,大感兴趣。

%79
乔丝柳借池伤,来对付池谤,池谤无可奈何,暗地里只得和乔丝柳致歉。

%80
乔丝柳成功地完成了家族使命,满载而归,同时也收获了一位最热烈的追求者池伤。

%81
乔丝柳的这封信,带给方源相当重要的情报,让他对池家的内政立即有了一个相当清晰的了解。

%82
“池家太上大长老寿元有限,已经再用不了寿蛊。而他的儿子池谤却是受他荫庇,才晋升成仙,实则才干不足。”

%83
“池曲由对自己的儿子有多少能力心知肚明,偏偏又想他成为今后的池家太上大长老,所以意识到池伤这样的性情和才华之后,才大力地倾斜资源,栽培出这样的一个阵痴来。”

%84
“只要有阵痴的守护,池谤继任太上大长老之位,就大有把握。而阵痴对池曲由的推崇和敬佩,更使得他难逃池谤的掌控。”

%85
“亲情啊……啧啧。”

%86
乔丝柳的这封信,让方源看到池家内部的政治争斗。

%87
正是因为这内外原因,才塑造了池伤这样的人才。同时,也纾解了方源心中对于池伤的一些疑惑。

%88
“来!武遗海仙友,我敬你一杯。”这个时候,池伤首次对方源端起了酒杯,打断了方源的思绪。

%89
池伤继续说道:“你是一个很真的人,将来我和丝柳仙子在一起的话,你也不必难过。你其实很优秀的,世间的女子还有很多,但真正懂的阵道却少之又少!”

\end{this_body}

