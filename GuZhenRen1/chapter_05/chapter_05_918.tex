\newsection{棒子变!}    %第九百二十二节:棒子变!

\begin{this_body}



%1
东海蛊仙一行八人,已经来到了白天,并且破坏了不少星辰。

%2
他们的战略目标没有南疆、北原、西漠那般巨大,只是想单纯地劫掠一些中洲的资源,消耗中洲的底蕴。

%3
因为担心天庭主力会传送过来阻止他们,他们便将矛头对准了白天中的星辰。

%4
这个策略误打误撞,反而成为了当今整个战局的关键点!

%5
他们当中有张阴、容婆等四大龙将,吴帅并未将他们召回,而是仍旧放在这个队伍中,让他们来摧毁星辰。

%6
吴帅的这个决定非常明智。

%7
若是仅剩下沈从声、青岳安、华彩云、宋启元四人,恐怕也未有足够的胆量前来破坏星辰。

%8
东海八仙人手众多,摧毁星辰的效率不小。

%9
八仙又来到一个巨型星辰的面前。

%10
星辰如山般巨大,八仙的身形显得非常微小。

%11
天庭散布在白天中的星辰难以计数,并且相互勾连,铺散成网,自有布局。

%12
每一片区域都有一颗巨型星辰充当阵眼,在巨型星辰周围,是大量的大型星辰。

%13
而在大型星辰周围是无数的中小星辰,随意散落,不断飘飞,位置并不固定。

%14
宋启元先飞上巨型星辰,他双目微闭,静静感受着星辰之间的光芒牵引。

%15
十几个呼吸之后,他猛地睁开双眼,催动一记光道杀招,手指连连指向远方。

%16
每指点一下,就有一道锐利的奇光顺着他的手指尖儿电射而出,瞬间射出天边消失不见。

%17
在遥远的地方,一颗颗中小型的星辰随之爆炸,炸成粉末。

%18
这记光道杀招非常奇妙,能够顺着任何的光线追溯源头,实施打击。即便是目光也包含在内。

%19
上一世,宋启元就是利用这一招,击破了方源炼化的一颗星辰,令方源的星眸仙蛊失效。

%20
而这一世,宋启元再用这一招却是找上天庭的麻烦,效果同样立竿见影。

%21
直至宋启元摧毁了全部能够感应到的星辰之后,他这才收手,对青岳安点头示意。

%22
仙道杀招——万重山!

%23
青岳安立即低呼一声,催出的虚影,宛若群山峻岭一层层一叠叠,轰砸在巨型星辰之上。

%24
巨型星辰体积极端庞大,但本身只是仙材而已,在八转杀招的猛攻之下,逐渐崩溃,最终被彻底打散成无数星辰碎块。

%25
几乎在巨型星辰被摧毁的一瞬间,天庭传送大阵中的紫薇仙子顿时有所感应。

%26
她脸色微变。

%27
一颗巨型星辰被毁,本来是无伤大雅的。毕竟天庭散布在白天中的星辰有许许多多,损失一颗无所谓。

%28
但是在当下这种情形下,却又是一回事了。

%29
眼下,北原诸仙以及方源和其分身,不断地进攻天庭各处要地,紫薇仙子必须时刻催动大阵,将这些人挪移出去。

%30
大阵不断地催动着,就仿佛是狂奔的人,处于危险的平衡状态。一旦踩在石子上,哪怕这个石子非常微小,人因此失去平衡,就会跌倒负伤。奔跑的速度越快,跌倒后人所受的伤势就越重。

%31
传送大阵此刻不停地疯狂催动,一旦当中有一个环节出错,偏偏运转的过程中需要用到这个环节,那么传送大阵就会被自己强大的威能反噬,从而重创乃至崩溃。

%32
紫薇仙子原本以为东海诸仙威胁很小,但没想到时过境迁,在这种情况下,东海八仙反而成了她的致命威胁!

%33
她不得不谨慎小心,任何有嫌疑的星辰,她都不会冒险去用。

%34
方源敏锐地察觉到大阵的些微变化,他一边狂攻不停,不断催发万我杀招,一边传讯沈从声。

%35
沈从声接到方源的讯息,立即了解了天庭战场的战况,不禁十分振奋。

%36
他催出其余七仙:“眼下是最好的时机,方源已经出手,天庭开始自顾不暇了。我们摧毁越多的星辰,会对天庭造成越大的麻烦。天庭主力被困在毛脚山战场,而天庭更是遭受长生天、方源的联手攻击。我们此次若作得好,极可能将天庭彻底击溃!”

%37
沈从声的话不禁让七仙士气大振,行动效率再提一成。

%38
毛脚山战场。

%39
尖锐的凤鸣声刺破耳膜,贯穿战场。

%40
发出声音的是凤仙太子的杀招——凤鸣神炎!

%41
火焰炙热燃烧,烘烤天地,缠绕在帝藏生的身躯上,不断剧烈灼烧。

%42
而另一边,白沧水也催动杀招,无穷的水浪汹涌澎湃,一刻不停,冲刷在帝藏生的身躯上。

%43
帝藏生遭受的打击远不止这两处,它的体型太大,让天庭诸仙都有充分的攻击位置。

%44
帝藏生咆哮连连,遭受火烧水淹、电击毒伤,仍旧冲向九九连环不绝大阵,并未被天庭诸仙挑衅成功。

%45
这番表现自然不符合历史记载,曾经帝藏生作乱,疯狂攻击,很容易被敌人牵引了注意力,从未有过如此目标清晰,行动坚定的表现。

%46
“糟糕!这头太古传奇恐怕已经被人奴役了。”周雄信惊呼一声道。

%47
天庭主力当中并没有智道蛊仙,但周雄信专修信道,最擅长收集情报和线索。

%48
他更有信道杀招,能够模拟出智道效果来。

%49
交战了一阵子后,周雄信立即算出了帝藏生的真相。

%50
帝藏生被天庭诸仙狂轰滥炸,各处身躯都有了伤痕。要论防御,它并不如龙公,并且身上也没有野生仙蛊,只能凭借本能作战。

%51
它的爪子每每挥舞,都能形成狂风。天庭蛊仙纷纷避让,谁都不想尝试被帝藏生抓住的感觉。

%52
它的每一声吼叫,都能掀起惊天的声浪,猛烈冲击着每一个人的耳膜和心房。

%53
它的体格太过庞大,就算遭受无数打击,对于它整个身躯而言,并无关大碍。

%54
它硬生生盯着天庭蛊仙的狂轰滥炸,坚定不移地冲进九九连环不绝阵中,开始破坏。

%55
天庭诸仙均生出一股无力之感。面对帝藏生,他们即便是用尽全力,也是老鼠撼象。

%56
“怎么办?”

%57
“再这样下去,九九连环不绝阵可支撑不了多久。”

%58
“攻敌必救,我们也以攻对攻!”

%59
天庭诸仙经验丰富,立即想到了对策。

%60
旋空童子、君神光打先锋,随后朱雀儿、赵山河等人成为中军,纷纷杀向五界大限阵。

%61
就在这时,帝藏生龙口张开,飞出一座仙蛊屋来。

%62
只见这座仙蛊屋通体绽射橙金光辉,威武霸气,正是八转仙蛊屋——龙宫。

%63
一股烟气从龙宫飞出,如梦似幻。

%64
“梦道杀招?!”天庭诸仙辨认出来,惊骇交加,慌忙闪躲。

%65
梦里轻烟所到之处,天庭诸仙纷纷避退,主动让开一条道路。

%66
轻烟裹住五界大限阵,迅速收回。

%67
天庭诸仙连忙施展手段,纷纷轰击梦里轻烟,但是梦道杀招独步天下,天庭众仙的努力都沦为泡影。

%68
五界大限阵直接被收进龙宫之中,战部渡收起大阵,落到龙宫的地砖上,见到吴帅。

%69
吴帅第一时间催动龙宫中的手段,为他疗伤。

%70
天庭乃是强敌,即便是处于下风,也拥有反击之力。天庭主力采取了正确的战术,逼得战部渡不得不收起五界大限阵。

%71
此阵已经遭受无数次的轰击,处于危险的极限。

%72
不仅战部渡需要疗伤,这座五界大限阵也需要紧急修葺。

%73
没有了大阵供应,五色烟瘴虽然仍旧残留在战场之中,却是无源之水,开始迅速消耗。

%74
龙宫在吴帅的驾驭下,并没有和天庭主力纠缠,跟上帝藏生,冲进九九连环不绝阵中。

%75
“顶住!顶住!”主持大阵的蛊仙们纷纷呐喊,气氛相当凝重和紧张。

%76
为了保护毛脚山,护住不败福地,他们必须撑起九九连环不绝阵。

%77
然而不管是五色烟瘴,还是帝藏生都对这座大阵拥有致命的威胁。

%78
进入大阵,帝藏生实力猛地上涨一截。

%79
它在五色烟瘴中也遭受束缚,毕竟是中洲土生土长的太古传奇,并非九天生命。

%80
仙道杀招——无限风!

%81
趁着天庭主力回援大阵的功夫,武庸终于抓住时机,永久消耗自身风道道痕,施展出不灭飓风。

%82
飓风狂卷,一如前世在外对九九连环不绝阵施压,令天庭一方雪上加霜。

%83
帝君城战场。

%84
万像宫殿和豆神宫相互照应,牢牢稳住西漠阵脚。

%85
千变老祖更是只身一人,不断冲锋陷阵。种种变化道杀招威力卓绝,能把岳阳宫、寒螭庄击退。

%86
就像张飞熊、肉鞭仙一样,千变老祖同样具有肉身激战仙蛊屋的战力!

%87
不过总体而言,中洲一方仍旧是占据上风的。

%88
他们都受益于人中豪杰。这个人道杀招不愧是尊者手段,不能以常理衡量。

%89
帝君城中,大比也进行到了关键时刻。

%90
洪易紧紧盯着火焰,全神贯注。不管外界的仙战如何恢弘,都干扰不到他。

%91
叶凡同样如此,他一心想要和洪易决出胜负。但和洪易不同,他采取的是光道炼蛊法,霞光不断吞吐,熠熠生辉,映照在他的脸上。

%92
整个比试会场中,不断的有蛊师失败,失败道痕陆陆续续地传达到不败福地里。

%93
随后,不败福地中的成功道痕被天庭抽取出来,传输到袁琼都处。

%94
袁琼都操纵着炼道大阵,不断地修复着宿命蛊,进展颇多。

%95
这座炼道大阵中,还有车尾、从严两位蛊仙全力守护。

%96
一位位北原蛊仙冲到大阵前,就被星光射中,瞬间传送出去,根本不能抵挡。

%97
“不愧是紫薇大人呐。”

%98
“真是可靠。”

%99
车尾、从严心中感叹。

%100
“坚持,坚持住!”紫薇仙子此刻已经满身汗渍,脸色惨白,头脑却是热得冒烟。

%101
她不断地为自己打气:“局面还在我的掌控之中。坚持就是胜利!”

%102
然而就在下一刻,万年斗飞车上方源忽然大笑一声。

%103
仙道杀招——万物大同变。

%104
仙道杀招——落星棒子变!

%105
他忽然变成了一个大棒子,又黑又粗。

%106
随后,他又施展一记星道杀招——领袖群星!

%107
“不好!”紫薇仙子脸色剧变,顿时感觉自己仿佛坐在一个发癫狂奔的马车上。

%108
拉扯的骏马都疯狂混乱,掌控鞭绳的她根本无法再稳住这辆马车。

%109
轰!

%110
下一刻,整个传送大阵陡然自爆,星芒消散,再无一丝照下。

%111
紫薇仙子遭受反噬,重伤濒死,陷入昏迷当中。

\end{this_body}

