\newsection{镇运天宫}    %第二百二十二节:镇运天宫

\begin{this_body}



%1
茫茫云海的之上,一个人影驻足悬停,看着眼下云卷云舒,苍茫大地,默然无语。

%2
此人身着青金甲,胡须垂至胸口,身材魁梧,面容苍老,正是南荒仙人。

%3
“药皇参见南荒仙使大人!”一位身影显露出来,正是药家的太上大长老,八转蛊仙药皇。

%4
南荒仙人侧过身来,似笑非笑:“你我皆是八转,称呼仙友便可,无须如此客气。”

%5
药皇恭恭敬敬地行了一礼:“论修为,南荒大人高我数倍。论辈分,在下乃是南荒大人的重重重孙辈,怎能无礼?”

%6
南荒仙人点点头,叹息一声:“你很好,可惜,我等黄金血脉越发凋零。偌大的北原,长生天外,竟只有一位八转蛊仙,是我黄金家族的成员!”

%7
凤仙太子虽然也是宫家的人,但在南荒仙人眼中,不是血脉,仍旧是外人。

%8
“想我那一代,黄金家族中有四位八转,其中更有一位,人称虎神,有盖世神威,是当世被看好,能够冲刺九转境界的种子。可惜陷落在了黑天之中。岁月如梭,黄金血脉的确是一代不如一代了。”

%9
南荒仙人眼中,透露出一抹深沉的失望。

%10
药皇说不出话。

%11
这种批判整个黄金血脉的话语,药皇都没有资格,也就是南荒仙人才有资格评头论足。

%12
事实上,南荒仙人说的也是事实。

%13
药皇无从反驳。

%14
的确是这样,北原的黄金家族,都已经渐渐腐朽,人才凋零,反而是散修、魔道中新星涌现,层出不穷。诸如百足天君、楚度,还有那柳贯一等等。

%15
“不过,此等情势巨阳先祖亦早有所料。”南荒仙人忽然话锋一转,说出了令药皇感到吃惊的话。

%16
“哦?巨阳先祖崛起于中古时代,他竟能预知此时情景吗?”

%17
“先祖虽修运道。但智道造诣也绝对不弱。再者说所谓‘运’,本质便是变数,先祖开创运道,自然通宵世间变化至理。他在创建八十八角真阳楼后。便曾言语,此楼未来必倒,一旦倒塌,便是大时代的开始,天地变革的信号。”南荒仙人又道。

%18
药皇吃惊不已:“竟是如此?”

%19
什么是大时代?

%20
能够产生仙尊、魔尊的时代。便是公认的大时代。

%21
南荒仙人长长一叹:“巨阳先祖苦心孤诣,布置种种手段,就是想阴泽后人,存留实力。(’)让他的后辈子孙们,在这个大时代中迎风敌浪,成为弄潮之人。再不济,也要维护自身,保存血脉种子。”

%22
药皇迟疑道:“我们黄金血脉再是不堪,也总不会连维存自身都做不到吧?北原仍旧在我们的掌控之下,就算我们不济事。还有长生天在呢。”

%23
南荒仙人摇摇头:“长生天再强,也不过是先祖的仙窍所化。而那天庭,却先后有三位仙尊留下仙窍,并且历代天庭成员,都捐赠了自家洞天。长生天的底蕴,如何能和天庭相比?就像这一次,中洲主动来犯,直接出动了三位八转蛊仙,还有三座仙蛊屋。我们黄金血脉有这样的实力吗?”

%24
药皇咋舌:“什么?中洲来犯?!”

%25
南荒仙人却不再解释,只道:“跟紧我。”

%26
话还未说完。他便一飞冲天。

%27
药皇连忙紧随其后。

%28
两仙贯穿天罡气墙,进入黑天之中。

%29
在南荒仙人的带领下,药皇直朝东南方而去。

%30
飞行一段时间,药皇心中震动起来:“黑天之中。危机四伏,我虽然也常探险,搜寻八转仙材,但从未像今天这样,一路顺风,根本毫无阻碍。难道说。南荒大人手中掌握着黑天的线路图?”

%31
但事实上,线路图这种东西,在太古九天中并不适用。

%32
盖因九天中的云朵,都是在随时变化当中的。固定的地方也有,但是非常稀少。

%33
中洲一行,已经算是非常顺利,但仍旧遭遇了气罡飞天猪群、乌毒蛇群等等麻烦。南荒仙人、药皇两位,却是无灾无痛,悠然安全,仿佛是在逛自家的后花园。

%34
“好了,就在此处。”南荒仙人停下了身形。

%35
药皇也缓慢速度,飞到他的身边,站定。

%36
他感到有些奇怪,周围空荡一片,连一丝风都没有,怎么南荒仙人要专门停留在这里呢?

%37
下一刻,南荒仙人用行动做出了回答。

%38
他浑身爆发出冲霄的金光,宛若太阳碎片,逼得药皇都要闭上双眼。

%39
药皇心头一跳。

%40
黑天中晦暗幽深,这种放射强光的举动,无疑是自招猛兽袭击。

%41
但他警惕半天,也不见任何兽群来袭。

%42
“这还到底是不是黑天了?”药皇纳闷。

%43
片刻后,南荒仙人的仙道杀招终于催动完全,一道暗金宫殿,从虚空中浮现而出,显露出雄阔伟岸的殿身。

%44
药皇从未料想过,这里居然还藏着一座仙蛊屋。

%45
他很快就辨别出这座宫殿的来历,顿时神情激动万分,口中低呼道:“这岂不就是镇运天宫?”

%46
劫运坛、八十八角真阳楼、镇运天宫,便是巨阳仙尊生前拥有的三座仙蛊屋。

%47
八十八角真阳楼被巨阳仙尊布置在王庭福地当中,劫运坛留在长生天里,而这座镇运天宫,巨阳仙尊居然放在了黑天之中,隐形匿迹了无数年!

%48
“巨阳先祖将这座镇运天宫放置在这里,一定是有深意。只是他用意何在?”

%49
药皇心中正不断揣测,南荒仙人已回头看他一眼,道:“跟上来。”

%50
药皇便随着南荒仙人,一同进入到了镇运天宫之中。

%51
“你们来了。”大殿之中,居然有人。

%52
他盘坐在蒲团上,并非意志凝聚,此刻睁开双眼,目光如电,金芒四射。

%53
药皇见到此人,浑身一抖,心中惊骇得掀起滔天波澜:“巨、巨阳仙祖?!”

%54
……

%55
嗷呜呜!

%56
狼群嘶吼,追赶着三座仙蛊屋。

%57
这些狼体型修长。身姿矫健。它们有黑色的狼瞳和爪牙,都在散发着残忍的凶芒。最奇特的是,它们浑身无毛,而是皮甲。黑光锃亮,非常坚固。

%58
这是一只夜天狼群。规模很庞大,有上万头夜天狼。

%59
每一头夜天狼,都是荒兽。狼王是上古荒兽,更有一头狼皇。有八转战力!

%60
“逃不了了,已被包围。”

%61
“想不到狼皇身上,居然有隐藏行迹的八转野生仙蛊!”

%62
“这种规模的狼群,在黑天中也不常见,居然被我们撞到了。”

%63
三位八转蛊仙急速交流。

%64
灵威仰喝道:“向西北方向突围!”

%65
角连营打头阵,气势陡然上升,仙元疯狂消耗,凝聚成冲天巨角。

%66
巨角所到之处,洞穿狼群阵势,只一下就穿出个大洞来。数百头的荒兽夜天狼惨死。

%67
仙蛊屋角连营冲入狼群之中,身后的揽雀阁、风满楼紧紧相随。

%68
一场大战,惊心动魄。

%69
期间狼皇都亲自出手,巨大的狼爪,宛若小山一般撞来,将仙蛊屋直接拍飞。

%70
好在仙蛊屋中各有一位八转蛊仙坐镇,三座仙蛊屋也各展威能,杀得狼尸遍野。

%71
狼群的围猎战术,已经达到了巧若天工的程度,三座仙蛊屋左冲右突。任是冲不出重重包围。

%72
最后还是狼皇亲自呼啸,主动带着狼群撤退,才让这场厮杀结束。

%73
原来是狼群损失太多,狼皇不愿过度损失己方的力量。才主动退走。

%74
三座仙蛊屋都残破不堪,大量的蛊虫损失,其中还包括两三只仙蛊。三位八转蛊仙也都是一身汗,狼狈不堪。

%75
仙元的损耗,就更是一个庞大数字。

%76
“怎么越接近北原,就越是艰难?”

%77
“夜天狼群我们已经碰到了五支。不仅如此,还遇到了龙鳄、鬼灵狐……好像是无数兽群,都要猎杀我们似的。”

%78
“这些兽群也就罢了,关键是别再遇到黑白颠倒云。那一次,我这座风满楼就差点回不来了。”

%79
“先休整,主要将仙蛊屋修葺一下。”威灵仰道。

%80
三座仙蛊屋刚刚停下,周围数千里范围,亮起了无数灯火。

%81
就好像是万千鲜花盛开,淡红色的火光细腻非常,宛若片片花瓣层层叠叠,火焰之中,却是一抹深沉的黑点。

%82
“不好,这是黑天中的特有天象黑灯,里面孕育的都是瞎火,不要乱看!”碧晨天怪叫一声,连忙闭上双眼。

%83
但已经迟了,许多蛊仙只是看了一眼,因此失去了光明。

%84
三座仙蛊屋中,惊嚎声、惨叫声迭起,响成一片。

%85
赵怜云的运气很好,此刻的她正背对窗口,和余艺冶子对话。

%86
余艺冶子眼角撇到了一些黑灯瞎火,当即感到视力剧减,视野迅速模糊,连忙闭上双眼。

%87
他是知机得早,很多蛊仙就不那么幸运了。

%88
有的蛊仙正趴在窗口四处观望,一下子看了至少上万朵的瞎火,顿时视野漆黑一片,再也不能视物。

%89
三座仙蛊屋慌忙逃离险地。

%90
好在黑灯这种气象,并不像风雨,带来实质性的打击,仙蛊屋顺利地冲出来,很快将漫天的黑灯远远甩在了身后。

%91
三位八转蛊仙统计了一下,发现至少三人,彻底失去了光明,现有的手段如何都不能补救。有七位,视力下降得十分凄惨,已经严重影响到了战斗力,可以补救,但是代价不菲,一些治疗仙蛊还得从宝黄天中转运过来。

%92
“刚脱离狼群,想要休整,就正好碰到黑灯气象发生,还位于最中央的位置!这也太晦气了。”

%93
“等等,那是什么?”

%94
一大片蚁群,正迁徙而来。

%95
每一只蚂蚁,都是蛊虫。

%96
冥蚁蛊!

%97
“冥蚁开道,接下来必定就是魂兽大军,十万百万。快,快走!”

%98
三座仙蛊屋刚刚停下来,就不得不继续奔走。

\end{this_body}

