\newsection{万年斗飞车}    %第七百七十八节:万年斗飞车

\begin{this_body}

练习杀招,并不容易,稍有大意,杀招失败蛊仙还是遭受反噬,身上带伤。

威力越大的杀招,风险越高,但不练还不行,总不能在激战中胡乱催动吧?战斗中的环境更是糟糕,会引发对手的强烈阻击。

历史上,有过不少蛊仙的例子,他们因为催动杀招失败,遭受反噬,伤势过重而亡。

其中,练习杀招死亡的例子,也有许多。

偏偏方源掌握的五禁玄光气,是那种威力强大的杀招,所以他练习的时候都是极其专注,谨慎小心。

关于五禁玄光气和五界大限阵,这两个选择,方源还没有决定好。

两者是不能共用的,临战组阵太过困难,而若是五界大限阵组建起来,也不是那么容易拆解的。战斗发生时,拥有五界大限阵就意味着丧失了五禁玄光气。

从灵活的角度而言,五禁玄光气杀招自然要更强一筹。

但阵道亦有阵道的优势。

五界大限阵一搭建出来,就等若时时刻刻准备好的超级仙道杀招,往往能令方源占据极大的先手优势。

具体例子直接参考不久前的年流伏诛埋伏大战。

不管是前世还是今生,这场大战都是方源生平最得意的战绩了。

不仅是因为战果丰硕,更主要的在于整个战局都始终在方源的掌控中,方源一直把控,成功地让战局按照自己的轨迹走向终点。这一点是非常不容易的。

任何的计划,最完美无缺的时候,都是未落实的时候。真正实施起来,就算是天庭、长生天都会常常把控不住。

能够让方源埋伏成功,最重要的帮手就是那套宙道仙阵。

只要大阵搭建出来,阵道蛊仙往往就能做到以弱胜强!

但阵道的弊端,就是不灵活,有许多死板之处。

绝大多数的仙阵,一旦建设下来,都是不能移动的。一位阵道蛊仙很强,建设出了大阵,敌人发现后直接绕过去,这种尴尬的情形在史书上也多有记载。

所以,池曲由、紫薇仙子才感叹,方源的宙道大阵能如此隐蔽,让南疆群仙遭受了暗算。

还有一点,仙阵的铺设往往很不容易,耗时很长,搭建个五六天算是少的,两三个月的比比皆是,动辄一两年的都有。

比如上一世,天庭搭建九九不绝连环阵就提前了一年左右。不过此阵建成之后,在里面搭建子阵,速度就会变得很快,快到不可思议的地步。这就是星宿仙尊留下的手段,当然不容小觑。

饶是如此,布置子阵还是需要时间。陈衣等人在上一世的大战中,就不得不出阵抵挡武庸等人,为后方拖延时间方便布置子阵。

有许多超级仙阵,也不是随随便便的蛊仙就能出手布置的。

上一世大战,就是武庸等人打下手,池曲由、方源、玄极子主要布阵。

琅琊福地中的长毛炼道大阵,也不是长毛老祖搭建的,是他请动了当年的阵道大能九华仙后。

方源估计,若是他要选择五界大限阵,至少也要提前大半年的时间。

不过好在,他可以利用仙窍和外界的时间流速差比,大大减缓这个弊端。

在他的计划中,全新的五界大限阵,需要融入地脉、阵旗、阵灵。有了这三样,方源就能先将大阵布置在至尊洞天中,等到用的时候直接拿出来用,甚是方便。

然而这等五界大限阵的内容,方源还未开始推算。

这一段时间,方源的宙道分身是着重推算仙蛊屋。

仙蛊屋也是蛊阵,本身就是从阵道中发展出来的。起初的原因,就是许多阵道的蛊仙觉得布阵太不方便了,一点都不灵活,很多时候战斗突发,根本来不及布阵。

历史记载中,就有许多例子,有一些阵道的蛊仙甚至是大能,忽然遭遇战斗,一身本领发挥不出几成,战败已经是幸运的,关键还有不少战死的,心中定然是憋屈的。

仙蛊屋起初的理念,就是能够移动的仙级蛊阵。

宙道分身前段时间,在百忙之中抽空经营仙窍,随后就时刻笼罩在智慧光晕之下,他要推算改良的东西实在太多了。

而方源本体则苦练杀招,余下功夫多数用来修行魂魄,再抽点空余来经营仙窍。

杀招不断数量,魂魄底蕴每一次修行都是暴涨,仙窍的经营也一帆风顺。主要还是他的麾下一帮子蛊仙精英,帮助他打下手。

年华池相当关键,有了它,年兽就有专门的栖息之地。

它的优点太多了,节省空间、供年兽栖息、自产宙道蛊虫,甚至还能调节仙窍流速,就是一个微型的光阴长河!

方源上一世辛苦筹备,打造了一个年华池,结果吞并夏槎仙窍的时候发现,竟然她也有一个!

现在他直接拿了夏槎的年华池来用,这个池子里已经被夏槎建设得有模有样,但还够不上方源的眼光和大局。

这些天来,方源和麾下蛊仙都在重点微调年华池的构建,卓有成效。

方源手中的太古年兽都被关在里面,却十分惬意,不想出来。对于年兽们而言,这里面有吃的有喝的,泡在光阴河水中最是舒服了。关键还没有强敌,一族独大,不像真正的光阴长河。

年兽和鹰兽这两大兽群,一直是至尊仙窍的拖累。

如今年兽解决了,鹰兽方面也得到巨大进展。

为了搭建出鹰兽的食物链,方源先是大力建设,搭建出了更多鹰巢,然后他选择了草编鸟作为鹰兽的食物,这种鸟又称之为鸟织草。

这种生命比较奇特。

在春天的时候,它们只是一颗草籽,草籽发芽,冒出土地,逐渐长成一片草地。

当夏日照射,草地中的草,彼此纠缠,编织出鸟的形状,飞腾而起,群起而飞,这就是草编鸟。

在秋天,它们群飞迁徙。

到了冬天,它们落地,身形消散,重新化为一颗颗的草籽,埋入雪土之中,等待来年春天的发芽。

这种鸟好养活,繁衍得很快,数量很大。鹰兽虽然不喜欢吃,但方源现在这种情况,食物链才刚搭建,这方面是个空白,哪里能让它们挑肥拣瘦?

正好春蛊、夏蛊都已经被方源炼化,为他所用。

他就派遣一位蛊仙驻守在小九天之中,照看场地,然后方源催动杀招。

他改良出了两记杀招。

一招名为春芽,以八转仙蛊春为核心,木芽仙蛊为辅助,让鸟织草在云土中四处的冒,迅速染绿一片片广袤的云土。

另一招叫做夏日,以八转仙蛊夏为核心,一些光道仙蛊为辅助,形成炙热的阳光,让草编鸟一群群的飞啊飞,数量迅速扩张,很快就形成极其惊人的规模。

嗷嗷待哺的鹰群总算是有了果腹之物,方源也摆脱了从外界收购食料喂养鹰群的处境。

不过,因为栽种鸟织草导致云土消耗颇大,换做方源要收购云土。而催动杀招也要消耗八转仙元,算起来方源的消耗还比之前单纯喂养鹰兽,要大一些。

这也是搭建食物链的过程中,必要要有的阵痛。

方源上一世这个时期,就未对鹰兽下手。这一世他家大业大,不在乎这些阵痛,完全可以支撑,也就不在乎了。

此事导致又一项弊端。

因为动用两记宙道杀招,导致周围环境忽春忽夏,对于鹰群是个大刺激。导致许多鹰兽发春、产蛋。

又因为春夏轮转太过剧烈频繁,没有秋冬来调和休养,鹰兽越来越暴虐,相互之间因为争抢配偶发生大量的流血冲突。

这个时候,方源派遣过去的蛊仙就起到了保姆的作用。

这种简陋的食物链,当然只是起步,将来还会继续建设,直至形成完整、稳定的一条食物链。

这点也不是完全的弊端,至少方源的那些鹰巢中,就增添了不少的鹰蛋。

不像年兽有年华池,还有光阴长河这种现成的参照,鹰兽要麻烦得多。

食物链的种种,关系到方方面面,牵一发而动全身。有时候,一只小毛毛虫对于至尊仙窍,都有广泛的影响。

方源还需要考虑到未来。

现在,至尊仙窍还有海量的空白之地,有关这方面的推算,就已经比凭空推算一记六转仙道杀招,还要困难一些了。

将来,这种推算量就更恐怖了。

仙窍稳定经营,吞并进来的福地、洞天逐渐消化,有许多地方已然是繁花似锦,超过了世间大半的蛊仙了。

方源的宙道分身推算不休,也终于有了一个成果。

仙蛊屋万年斗飞车!

上一世,方源就一直想打造出一个仙蛊屋,方便探索光阴长河,同时能让麾下参加八转层次的战斗。

但可惜,他多次建设,却并不太成功。第一次的仙蛊屋雏形就被厉煌给毁了,之后多次建设,但过了勒索南疆正道的时间点,蛊虫杂七杂八,仙蛊屋虽然建成,但在中洲大战中牢牢处于下风,起到的作用并不大。

这一世,他吸取当中的教训,先暂且降低标准,

------------

\end{this_body}

