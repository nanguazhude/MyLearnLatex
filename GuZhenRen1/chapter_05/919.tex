\newsection{天庭危急}    %第九百二十三节:天庭危急

\begin{this_body}

传送大阵的自爆惊天动地,天庭战场亦为之一滞。

随后,北原诸仙爆发出欢呼和呐喊,声浪滔天。

“紫薇!”秦鼎菱大惊失色,身化一道虹光,电射而下,从传送大阵的废墟中救起紫薇仙子。

紫薇仙子昏死,状态极差,需要秦鼎菱立即施救。

秦鼎菱没有犹豫,立即退缩一隅,不打算再辅助龙公作战了。她很有自知之明,知道自己对龙公帮助不大,而紫薇仙子却是天庭智道首脑,位置相当关键,不可缺失。

方源自然想要乘人之危,直接杀死紫薇仙子,但却始终被龙公挡住前进的道路。

不得不说龙公的这招改良气墙,十分强悍,并且一扫元始气墙的呆板,灵活而又坚韧。从龙公主持此招之后,即便方源拥有万年斗飞车,也无法做出突破。

气墙中,龙公神情凝重,将方源当做生平最大的敌人对待,之前的轻蔑再无一丝存在。

直至方源摧毁了传送大阵之后,龙公这才明白过来:方源之前的万我杀招不过只是一个幌子,只是在吸引他们的注意力。

而方源真正的手段则在暗中酝酿,一经施展,立收奇效!

“但方源怎么这么巧刚好就有破阵的手段呢?”

龙公清楚,方源之前破解气墙,铁定是重生的原因。但这个传送大阵明显出乎方源的意料,他被传送大阵为难了好一阵子,若是上一世就有的,他定然会准备应付的手段。

“也就是说,方源是临时推算出了破解大阵的杀招?但这怎么可能?”龙公疑惑万分。

天庭做出这方面的布置,自然也防备方源,知道他的智道造诣。

单凭方源的底蕴,要破解传送大阵需要一段漫长的时间。

说起来,这一切还真的有些巧合。

方源探索龙鲸乐土,从沈从声那里得到了落星棒子树。

这株太古荒植的生长条件至今是一个谜团,并且从古至今都分外罕见,数量极其稀少。

它有独特的天赋,能不断地吸引星辰,落到树干中,化为一颗颗的星纹。所以称之为落星。

据传闻,元莲仙尊当年独游天下的一个动机,就是要收集这种树。

落星棒子树的价值,不只是它本身。

蛊仙将它栽种在仙窍中,可以利用它打造出一副星辰的运转体系。这是一个独特的生态,带来的收益每年递增,天长日久,年复一年,收益总量几乎无法估算。

若是没有这株树,方源就算知道落星棒子树的名头,也变化不了。但自从他得到之后,便悉心观察,耐心揣摩,这便有了落星棒子树的变化。

除了棒子树变化之外,方源还有东方长凡的万星飞萤杀招,耶律群星的领袖群星杀招。这些都是他的战利品。

尤其是领袖群星杀招十分精妙。它采取了星辰碎片为主体,消耗仙元很少,利用星辰碎片之间相互吸引,产生的诸多星光磁力,不断牵引星辰。星辰越多,此招威能越大。

当初北原血斗的时候,耶律群星凭借此招威势无两,几乎无人可制。

领袖群星杀招本身就针对星辰、利用星辰。在这样的基础上,方源还有星道、阵道双宗师境界,更有智慧光晕加持。临战改良了杀招,种种因素叠加起来,令他达成了这个惊人的战果。

他摧毁了传送大阵,令龙公都感到十分惊诧。

在龙公的眼中,方源已经开始变得深不可测起来。

万年斗飞车上,方源撤销了落星棒子树的变化,再度恢复人身。

仙道杀招——万我!

嘭嘭嘭……

海量的方源分身再次出现,充斥天地,宛若蝗虫一般向四面扩散。

天庭一方顿感棘手,此次交战他们算是吃够了人手不足的亏。

星宿天意虽然提前苏醒了诸多天庭八转,但眼下传送大阵被毁,这些主力还困在毛脚山战场呢。

“远水解不了近渴!”龙公叹息一声,无奈地施展出一记手段。

气道杀招——气盖山河!

刹那间,一股无以伦比的磅礴压力从天而降,盖压四面八方。

万物齐喑,天地镇静。

方源分身直接被镇灭,劫运坛也被压在地面上,艰难抵挡强压,咔咔作响。杀招范围之内的天庭仙近十座蛊屋更是不堪,直接被镇毁,沦为一片片废墟。

方源和万年斗飞车同样不能幸免,被气盖山河杀招笼罩,速度暴降,仿佛肩负群山峻岭,迅速从高空跌落。

方源长笑一声,不惊反喜。

龙公的气盖山河他早已熟知,上一世争夺龙宫时,龙公就凭借此招奠定胜局。

但此刻,龙公却是不得不施展此招,来化解方源的万我。他实在太被动了,连天庭自身的仙蛊屋都顾及不了。

仙道杀招——太古年猴变!

方源陡然再做变化,浑身道痕瞬间转化为一百五十万以上的宙道道痕。

万年斗飞车本是宙道仙蛊屋,此刻被方源狂催,威能疯狂暴涨,一扫颓势,宛若飞剑飚射而出,仿佛气盖山河根本不存在似的!

龙公怒视方源,此刻却追赶不及,眼睁睁地看着方源冲向仙墓。

方源并不打算向监天塔下手。一来监天塔周围还有大阵守护,突破需要时间,二来监天塔中可是留着元莲仙尊的画道手段,方源忌惮不已。

仙墓也是天庭重地,虽然也可能藏有尊者手段,但方源必须去冒这些风险。

仙墓乃是一个巨大的变数,提前捣毁绝对有利于方源等人!

一缺抱憾亭中。

见到方源直逼仙墓的举动,星宿仙尊虚影也不由微微变色,双指捏起一颗棋子,就要投射出去。

但这个时候,对面坐着的无极魔尊虚影轻伤一笑,在棋盘上落下棋子。

星宿仙尊虚影动作顿止,若是她将这枚棋子投向外去,那么无极魔尊必将形成绝对优势,这局棋星宿仙尊将再无机会挽回!

“无极,你真的很看好方源。”星宿仙尊虚影叹息一声,将手指间的棋子重新放回原处。

随后,她目光陡然锐利如刃,寒声道:“但你真想阻止我,可不是那么容易的事情。”

话音未落,星宿仙尊虚影连连出手,在棋盘上不断落子。

无极魔尊紧随而动,不甘落后。

双方你来我往,几个呼吸之间已经落下百十个棋子,棋盘上充斥着棋子密密麻麻,但奇妙的是,棋盘似乎无穷无尽,不管双尊虚影落下多少棋子,棋盘上都能容得下!

星宿仙尊虚影被无极魔尊牵制,使得方源长驱直入,再次来到仙墓上空。

“方源!你若出手,我必定让你死无全尸!!”龙公还在身后全力飞赶。

方源冷笑一声,施展出力道大手印。

轰隆一声,手印如山似缓实快,狠狠落下。

但大手印落下的过程忽然变得相当漫长,空间看似未变,实际上却是已经比之前多出万里之遥。

关键时刻,有人插手,但却并非是尊者手段。

“什么人?!”方源喝斥一声,双目绽射电芒,立即看向左手远处。

在方源的目击之下,一位女仙缓缓显露身形。

她模样奇异,浑身上下都是青木,她身穿着绿色长裙也是一根根藤蔓编织而成。她的脸虽是人脸,却有树皮样的皱褶,一双眼睛宁静如湖。

方源瞳眸微缩,单从气息上来看,这位女仙近似帝藏生般的层次!

“人仙,停手吧。不要再试图摧毁天庭,若你执迷不悟,那么我苍玄子会阻止你的。”女仙开口出声,声音轻柔却透露出强烈的坚定。

“苍玄子?”方源顿时了然。

天庭当中有三大太古传奇,它们分别是煞狴九十五、阮丹以及苍玄子。

苍玄子乃是一株苍天藤,在天庭中地位十分特殊,可以不听调不听宣,只需要每隔千年供奉一批苍天果子。

上一世,长生天进攻天庭,苍玄子从始至终都在观战,并未插手。没想到这一世方源入侵,却引得她下场参战!

“看来苍玄子和元莲仙尊另有隐秘协约,不为我等外人所知!”方源冷哼一声,再催万年斗飞车载着自己,亲自扑向仙墓。

苍玄子叹息一声,化作一道青色流光,在万年斗飞车前拦截。

砰的一声,万年斗飞车狠狠地撞在苍玄子的身上,冲势顿滞,苍玄子倒飞出去。

她吐出一小口青色的鲜血,又再次攻向万年斗飞车。

“方源,纳命来!”这个时候,龙公赶制,张口咆哮间满嘴的牙齿纷纷飞出,化作一记记白色巨刃,斩向方源。

一瞬间,方源陷入被前后夹击的困境。

他临危不乱,先手指一指龙公,催出春剪杀招,又一指苍玄子,脚底下的万年斗飞车迸射出无穷飞剑,赫然是破晓剑!

春剪飞出,连连剪断龙牙,势不可挡地杀向龙公。

龙公不得不避退。

破晓飞剑仿佛瀑布一般,冲刷苍玄子。苍玄子支撑了几息,便感到支撑不住,无奈后撤。

方源轻啸一声,万年斗飞车一飞冲天,又迅速下降,再次直逼仙墓。

但距离再次莫名延长,周围的空间发生了肉眼难以察觉的玄妙变化!

方源回首,恶狠狠地瞪视苍玄子,从牙缝中挤出一句话:“你找死。”

苍玄子满头大汗,她拼尽全力催动宇道杀招,只能暂时拖住方源。

幸好有龙公在,他再次杀向方源,然后再次被击退。

方源实力卓绝,力压龙公和苍玄子。

双方交手上百个回合,你来我往,方源越战气势越盛,而龙公、苍玄子身上伤势不断积累,狼狈不堪。

轰隆!

一声巨响,方源一记落魄印打得苍玄子尖叫,又一记大手印把龙公狠狠地拍在地面上。

“龙公,你已经老了!凭你这把老骨头,还能支撑多久?”方源冷笑。

“支撑到我死的那一刻。”龙公面色坚毅,战意始终高昂,他的一根龙角已经折断,龙鳞破碎,鲜血淋漓。

他死战不退!

而在炼道大阵之中,北原历代强者已经围死了从严、车尾二人。

“杀,杀死他俩!”

“他们已经不行了。”

“单凭区区两人,也想阻止我等?哼,不自量力!”

北原诸仙虎视眈眈,凛然的杀意弥漫整个空间。

从严、车尾都成了血人,只消站立几个呼吸的时间,他们的脚下就会积下一滩血液。

“看来我们的性命要交代在这里了。”

“呵呵呵,那就死吧。就算是死,也要拼尽全力,为袁琼都尽量争取时间。”

两人对视一眼,脸上绽放出视死如归的微笑。

“来吧,来最后一战!”身为弱势的一方,他们反而对北原诸仙们发出咆哮。

北原群仙面色变了。

“这两个均是可敬的对手。”

“我会用最强的杀招送你们最后一程,以示对你们的尊重!”

轰轰轰!

激战再次爆发。

帝君城。

无数道目光热切地停驻在最终大比的会场上。

绝大多数的参赛蛊师都失败了,一个个静悄悄离场。

整个场地中只剩下两人,一位是叶凡,一位则是洪易。最后的胜利者将从他们两人身上角逐出来!

“大比即将结束,我们守卫的任务也要成功了!”

“坚持住,诸位!!”

中洲守军提前欢呼起来。

“怎么办?”千变老祖久攻不下,也不由地流露出焦急之色。

房睇长默然不语,心底则在冷哼:“你们真是太天真了。就算我方被拦下,又有何妨?哈哈哈,终于来了!”

说时迟那时快,大地裂开,地动山摇,一个巨大的地沟迅速成形,恐怖的裂缝闪电般蔓延开来,竟直指帝君城!

“怎么会这样?”

“地沟!该死,帝君城就在地沟的前进路线上。”

“快,快救人!!”

中洲守军惊骇欲绝,想要施以援手。但房睇长早有准备,此刻操纵豆神宫越众而出,直逼中洲守军。

中洲蛊仙们顾此失彼,手毛脚乱地拦截豆神宫,却再也无力挽救帝君城中的无数凡人。

“我们完了!”

“谁还能拯救他们?”

“不,不能放弃希望!”

帝君城中乱成一片。

光阴长河。

河水滔滔,凤九歌立足在一座宙道仙蛊屋内,闭目沉思,静静地感悟着。

忽然,他睁开双眼,脸上全是惊愕之色。

一座石莲岛缓缓现身,出现在凤九歌的视野中。

“这是……”一瞬间,凤九歌想到了出发前秦鼎菱的话——

“我为你观运,你的运势预示着光阴长河。去吧,那里会有你的福缘,会成就你!你原本想去光阴长河感悟,补全你的命运歌,这个决定很明智,绝不会错的。”

“难道说,这就是我的福缘吗?”凤九歌强忍心头震动,踏上石莲岛。

\end{this_body}

