\newsection{继续腾飞}    %第五百九十七节:继续腾飞

\begin{this_body}

至尊仙窍。

仙道杀招搜魂!

澎湃的力量,狠狠地射在夏槎的魂魄之上,但旋即,夏槎魂魄上骤然亮起一道道的宙道道痕,将方源的搜魂杀招尽数遮挡。

但方源并不放弃,继续灌注仙元,一刻不停。

过了好一会儿,仿佛是用一双肉手艰难地掀起钢板的一角,夏槎魂魄上的宙道道痕出现了漏洞,令方源的搜魂杀招终于起效。

大量的信息在这一刻,被方源贪婪地汲取。但宙道道痕并不承认失败,防御漏洞周围的道痕,仍旧在坚定不移地围拢过来。

最终,随着时间的推移,这处好不容易被方源掀开的漏洞,又被宙道道痕填平了。并非在此处的防御程度,远比之前要更加森严!

方源的搜魂杀招,不得不因此终止。

“这夏槎不愧是八转蛊仙,底蕴深厚,魂魄上也有宙道防御的手段,能够抵御七转层次的搜魂杀招。”

“这种防御手段,还有自我复原、自我反击的效用,实在是妙不可言。”

方源俘虏了南疆群仙,在尽数偷取了他们的仙蛊出来后,便将他们的魂魄都摄取出来,和肉身完全隔离。

做到这种程度,关押这些南疆蛊仙的风险已经降到微乎其微的程度。接下来,便是针对这些没有肉身保护的魂魄,进行搜魂。

铁区中、商虎杖等人的魂魄,并没有带给方源多少困难,所有的情报都被方源尽数搜刮。

但遇到羊枯的时候,却是有些出现了一些麻烦。羊枯乃是专修魂道的七转强者,自然对这种搜魂手段有很强的抵御力量。

羊枯还不算麻烦,更麻烦的是刘浩。

刘浩的定空仙蛊等,受到天庭的布置,没有让方源获得。他的魂魄,天庭方面自然也没有漏过,照样有森严牢固的防御手段。

不过这两人最终,还是倒在了方源的手段下,被他尽数搜魂。

如果说荡魂山、落魄谷乃是魂修的两大圣地,方源掌握着幽魂真传,无疑是天底下第一魂道圣典!

方源因此得知刘浩的真实身份,竟然是天庭的内间。

不过所有人中最麻烦的,还是夏槎的魂魄。

夏槎魂魄上充斥宙道道痕,形成严密的防御,阻挡着任何一种外来攻势。方源的搜魂对夏槎不利,自然也是攻势的一种。

方源用尽全部手段,这才借助仙阵镇压,又改良搜魂杀招,使其威能达到眼下的极限,这才对夏槎魂魄可以搜魂。

不过这种搜魂的程度,十分艰难,进展很不顺利。

方源往往需要费尽九牛二虎之力,才能破开一个小小的漏洞。这个漏洞也不是一直存在,不过了多久,就会被宙道道痕重新覆盖住。

方源只有趁着漏洞形成的时间,进行搜魂。过了这个时间之后,漏洞表面会围拢比最初时更多的宙道道痕,防御更强一筹。

不过好在夏槎魂魄上的宙道道痕,并不会增多,漏洞此处的宙道道痕多了,就代表着其他地方的道痕比较稀疏。这就给方源下一次攻击,带来更好的机会。

其他人的魂魄,都已经搜刮完毕,就剩下夏槎魂魄。

方源一边攻略夏槎魂魄,另一边则利用宙道分身,借助定仙游,往来各个蛊仙的仙窍。

因为搜魂得到了第一手情报,方源完全知晓这些蛊仙仙窍中的景象,利用定仙游进去,完全是轻而易举之事。

并且,方源还清楚这些仙窍中到底藏着什么好东西。

蛊仙修行,自家的仙窍是重中之重。这些南疆蛊仙,各个都是强者,底蕴深厚,仙窍经营自然也是上佳的水准,里面的好东西自然是比比皆是。

这些蛊仙基本上,都是要被方源放回去的。

因为吞并他们的仙窍,收获道痕带来的收益,远远不如方源敲诈勒索那些超级势力。

道痕增长,带来的是底蕴,也是一种更光明的未来。

但未来再美好,过不了眼前这关,完全都是虚的。

眼前的难关自然就是宿命蛊,更准确地讲,是阻止天庭彻底修复宿命蛊。

但方源对此毫无头绪,他现在已经可以做到躲避天庭的追杀缉拿,但是宿命蛊保存在天庭深处,如何阻止天庭修复它,已经大大超出方源的能力范围。

所以,方源千方百计地想要进入光阴长河,继承红莲真传。

人族历史上,三大魔尊陆续攻上过天庭,唯有红莲魔尊破坏了宿命蛊,虽然没有彻底毁灭,但确确实实是损坏了宿命蛊。红莲魔尊死后,留下真传,极有可能会针对宿命蛊,留下后续的毁灭计划。

而要顺利地继承红莲真传,不被天庭消灭,就需要方源提升自己宙道方面的实力。其他流派的手段,在光阴长河的战场中,会受到极大的削弱,基本上是拿不出手的。

年华池,以及宙道仙蛊屋,还有更多,都需要方源和南疆正道进行交易,而方源交易的筹码自然就是这些蛊仙俘虏。

至尊仙窍时间,数天后。

嗷吼!

小南疆的上空,一条深绿色的“石龙”怒吼,庞大而又修长的身躯蜿蜒游动。

然后在方源的杀招主持之下,“石龙”一头扎进小南疆的地面,然后轰隆隆的声音不断响彻千里方圆,这头“石龙”硬生生地钻进地下深处,直至最终的尾巴都消失不见。

方源吐出一口浊气:“牵引矿脉的事情,总算是完成了。”

这条“石龙”,并非石人部族的那种石龙,而是一条矿脉凝聚成形。这条矿脉乃是由七转仙材乌青墨石组成,从一位专修土道的南疆俘虏的仙窍中,被方源挪移过来。

至于挪移如此大型矿脉的手段,幽魂真传中当然有记录,不过方源动的仙蛊还有杀招本身,都是来自于那位南疆俘虏。

他本身也是如此,将外界的乌青墨石矿脉,迁移到自己的仙窍中去的。方源如今做的,只不过是效仿他的举止而已。

“有了这条乌青墨石矿脉,我每年都能自行开采出一批来,放到宝黄天中贩卖了。”

但这笔收益,虽然不多,但胜在细水长流。

因为乌青墨石矿脉,会不断地汲取周围的地气,然后壮大自身。只要方源保持一定的开采程度,就能依凭这里,源源不断地开采出乌青墨石来。

乌青墨石虽然在所有的七转仙材中,价格不高,但到底还是七转仙材,价值并不容忽略。

“或许……我还可以将一批石人迁徙过来,依据乌青墨石矿脉繁衍生息。”方源忽然念头又一动。

这对石人而言,非常有利。

越是品级高等的矿脉,对于石人部族的繁衍、生息、壮大,都有正面助益。更秒的是,石人也会反哺这些矿脉,令它们的品质更加精纯。

这是互利互惠的双赢。

但是迁徙来石人的话,这个乌青墨石矿脉就不宜开采了,而是得留给石人作为新的家园。

方源思考了一下,便决定迁徙石人过来。

开采乌青墨石矿脉带来的钱财,他并不放在眼里,石人和矿脉所带来的未来收益,更令他在意。

方源从南疆俘虏手中,获得的当然远不止这一条乌青墨石矿脉了。

在小南疆,还多了数十座山。

大部分的山峦,都是普通凡山,不过本身凝聚的浓郁土道道痕,就是一笔宝藏。

还有一些山峦,达到了名山的程度!

比如内景雨山。

这座山的山头,始终笼罩着一层阴云。从阴云中不断地飘落雨丝。

这些雨丝可不是寻常雨水,而是六转仙材内景雨!

又比如铜印山。

这座山赤红作色,方方正正,仿佛是一方大印,矗立在小南疆上。

铜印山上没有一丝泥土,完全是各种各样的铜组成的。有混天铜、般若铜、雪花铜等等凡级铜材,更有仙材铜,比如苍玄铜、升龙铜。

当然凡材铜量最多,占据绝大多数,仙材铜只有很少的一部分,且多是六转仙材。最高的铜材,也能达到七转,种类唯一,那就是赫赫有名的霸铜!这种铜材一出,基本上就意味着不会有第二种七转层次的铜材出现了。

山峦、矿脉、树林……因为这些蛊仙俘虏都来自南疆,所以几乎一半的资源,都被方源迁移到小南疆中来。因为资源和环境,都非常贴切适宜。

另外一半的资源,都分布到了其他小四域以及小九天中。

比如小中洲,多了一片虚影花的花园,海量的仙农土铺设出了一片肥沃的平原。

小北原中,多了一具天柱风,这种风形如巨柱,极高极长。天柱风刮起来,从不移动,就在原地刮吹。从外面看过去,就像是撑天巨柱一样。

小东海中,有了一片全新的海域,名为浪花海域。海域中波涛滚滚,在浪涛中瞬间生灭出无数奇妙花朵,称之为断浪花。因为生长、凋零的时间极短,因此难以采摘。当方源从翼扬处获得这种资源的培养之法,也颇为惊喜。

小西漠中,栽种了十多株的扶桑树,这种树的每一根树枝都仙材!

“至尊仙窍太大了,若是换算成具体的数值,之前的至尊仙窍发展度不过只是百分之三四。如今因为这番吞并,已经有了百分之八九了。别忘了南疆蛊仙的这些仙窍,都还残留着许多资源,还有最大头的夏槎洞天!如此一来,绝对能突破百分之十啊。”方源兴叹。

这些资源有大有小,很多大型资源点,都是各个南疆蛊仙的经济支柱!

宝黄天中,天庭已经出手,大量低价售卖铜龙鱼、铁龙鱼以及银龙鱼,使得方源的龙鱼生意遭受巨大挫折。

但那又如何?

方源手中新添的这十多种经济支柱,将会支撑他继续腾飞!<!--80txt.com-ouoou-->

------------

\end{this_body}

