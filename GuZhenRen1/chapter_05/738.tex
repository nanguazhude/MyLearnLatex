\newsection{恩将仇报!}    %第七百四十五节:恩将仇报!

\begin{this_body}

“方源你竟然真的敢,你实在是太无知无畏了!竟然敢在琅琊福地算计我!我要让你明白你究竟犯下了何等严重的错误!!”

琅琊地灵大吼,满脸的愤怒,仿佛是即将喷发的火山,下一刻就要岩浆滚滚,将方源当做渣滓融化。

然而下一刻,琅琊地灵动作顿止,脸上的怒意陡然一僵,旋即变成了深深的惊愕。

“怎么会?我居然动弹不得,并且和琅琊福地的联系也变得若有若无!”一时间,琅琊地灵惊得呆住。

方源嘴角淡笑,一切都在他的掌控当中。

自从方源第一次见到琅琊地灵的时候,他的心中就埋下了一颗“对琅琊福地下手”的种子。

经过数年的光阴,这颗种子生根发芽,暗中茁壮成长,直至此刻开出了美妙的花来!

方源俯视琅琊地灵,语气淡然,洋溢着自信:“没有用的,你根本不清楚你我之间的差距。”

琅琊地灵极力挣扎,但他就像是一只蝼蚁,被死死镇压住,根本动弹不得,更别提去操纵整个琅琊福地了。

他尝试一番后,便骇然发现情况不仅没有丝毫起色,反而身上的束缚变得更加严重!

“这不可能!”他望向方源,满脸都是难以置信的神色。

方源微笑:“看来,你开始逐渐察觉到我们之间的距离了。”

方源的炼道境界,要远超琅琊地灵。

前者是汲取过长毛真意,炼道境界高达准无上,而后者不过是长毛老祖留下的执念,所形成的地灵,尽管炼道造诣非凡,但距离准无上这种程度还是颇远的。

方源的阵道境界,更是大大的超越了琅琊地灵。

琅琊地灵在阵道方面的造诣非常普通,而方源却是阵道的宗师。

更主要的是方源布阵的实践经验极其丰厚。洁身自好阵、四通八达上古战阵、龙鳞海食道仙阵、太古年兽钓来阵、光阴泄洪逐流阵、我意封天阵、智炼蛊阵……这些蛊阵很多都是方源自行设计,自我改良,自己搭建,从始至终,一手包办。

“好一个方源!你真不愧是世间公认的魔道巨擘了!”琅琊地灵双拳握紧,脸色狰狞,双眼好似喷火,死死地盯着方源。

他咬牙切齿,继续道:“原来这一切都是你的阴谋!万我仙蛊方本来就有问题,我却没有看出来。炼蛊过程中出现意外,是必然的事情,你借此机会成功布阵,然后诓骗我入阵。你的这个阵中之阵,应该是你仿造了血光镇灵杀招吧?”

“这样看来,万我仙蛊方中的八种主材,其中后四种血材都是用来陷害我的。真正用来炼制万我仙蛊的主材,只有前四种而已!”

琅琊地灵越说越镇定,他的心绪渐渐安定下来。

方源看着他,没有说话,神色平静。

琅琊地灵反而皱起了眉头,他在方源的眼眸深处看到了一抹笑意,好似是看出了他的打算!

琅琊地灵心中闪过一抹慌乱,但又很快被自己镇压下去。

他继续道:“要镇定!不能慌!长毛炼道大阵乃是我本体的巅峰杰作,岂是方源这么短时间里就能揣摩清楚的?他布置阵中之阵来陷害我,但这血道小阵是在我的炼道大阵之中。不久之后,单凭炼道大阵运转,就能挤破这个血道小阵。到那时,我抓住机会就能逃生出去。”

“只要我逃得出去,整个琅琊福地就都是我的力量。再将方源暗害我的事实宣布出去,异族大联盟也会站在我这一边。”

“到了那时,我定要叫方源好看!”

“哎呀!”

琅琊地灵忽然一拍脑袋,脸色铁青,懊恼至极,恨不得杀死自己:“该死的!我怎么把心里话都说出来了!”

方源便笑:“放心吧,地灵,你想到的这个破绽,我又岂能预料不到?”

“什么,你早就预料到了?”琅琊地灵先是一惊,旋即又镇定下来。

他哈哈大笑一声:“就算你预料到了又能如何?你看得透这座炼道超级大阵吗?”

方源摸了摸鼻子:“我对于这座大阵的认识程度,至少比你认为的要深得多。当然要论看透,我还差一些距离。”

琅琊地灵再度哈哈大笑:“这就是了!你等着吧,方源,等我出去,我要让知道背叛我琅琊派的可怕下场!”

然后,琅琊地灵的等待毫无结果。

片刻之后,血道小阵仍旧稳定如初。

琅琊地灵苦苦等候,长毛炼道大阵就是不见动静!并且,他的全身上下都已经被血光浸染。

“这不可能!你到底做了什么?!”忽然间,琅琊地灵对方源怒吼咆哮,他彻底陷入了惊慌之中。

方源微笑不语。

他接触长毛炼道大阵的时间,远比琅琊地灵想的要长。

因为上一世他就接触到了。

方源的阵道、炼道、智道三大流派的深厚造诣,令他充分地理解了长毛炼道大阵的精华和奥妙。

长毛炼道大阵乃是当初长毛老祖毕生心血凝造而成。

依照方源如今的底蕴,要让他自己建设出一个能媲美的炼道大阵,很难很难,需要海量的时间、精力、物力。

但是有了这座长毛炼道大阵摆在眼前,方源就并非是自己在白纸上作画,而是在欣赏一幅名家的画作。画作就在眼前,让他说出画作的笔法和妙处,他还是非常容易的。

当然,他到底是旁观者,无法彻底洞悉此阵的全部奥妙。

不过没有关系,作为间谍内应的毛六,此刻就掌控着这座大阵的某个重要方面。

借助毛六的策应,方源令长毛炼道大阵根本无法干扰到血道小阵。

琅琊地灵挣扎了一阵,终于意识到:在方源处心积虑的阴谋下,他已经无翻身之力了。

迫不得已之下,这位已经被血光彻底侵蚀的黑毛地灵自行退让,让另一面的白毛地灵登场,妄图翻盘。

遗憾的是,对琅琊地灵知根知底的方源,怎会漏算这一点呢?

白毛地灵登场后,便立即被血道小阵镇压,同时大量的血光也向他浸透过去。

“卑鄙无耻!”白毛地灵被镇压得死死,无法动弹,只有嘴巴能开口说话。

他大骂方源:“方源啊方源,你这个阴险狡诈的小人!”

“你之所谓能够走到今天这样的地步,我琅琊福地对你支助最大!”

“你还未成仙时,我就帮助过你,最终你才能顺利捣毁八十八角真阳楼。”

“你刚刚升仙的时候,毫无身家,一穷二白,是我和你交易,让你推算仙蛊方,我付出仙元石,这才让你成功起步。”

“你要炼蛊虫,我多少次亲自出手!福地中的毛民蛊仙哪一次没有帮你?”

“你渡劫风险太大。也是我们借给你仙元石,借给你仙蛊,让你渡过难关。”

“我交给你仙劫锻窍的妙法,交给你剑浪三叠等等杀招。”

“你搬迁星象福地,是我们借的蛊虫,提供的法门。”

“你的狐仙福地遭受攻击,是我们冒险出手,帮助你挽回了无数损失!”

“你的智慧蛊,还是我帮你喂养,耗费了我福地多少的寿蛊啊!”

“你的荡魂山被夺走,我按照约定赔偿你,哪怕是要付出炼道真意的代价。”

白毛地灵越说越是激愤,越讲越是委屈,滔滔不绝地说到这里时,眼眶都泛红了。

和黑毛地灵暴躁、强硬的性情不同,白毛地灵性情隐忍平和,得过且过,只对炼蛊方面的事情感兴趣。

黑毛地灵是激进的进攻派,白毛地灵就是文弱的保守派。

“方源啊,你的良心是被狗吃了嘛!”

“我们琅琊福地这么对你,你居然恩将仇报,对我下手!”

“你在外面混不下去,几乎五域的蛊仙都要为难你,是琅琊福地为你遮风挡雨啊!”

“你是一个人族,但不管是我还是黑毛,都努力地接纳你,任命你为太上客卿长老,给予重用,把你当做自己人!”

“但你现在是怎么对待我们的?”

“你现在摸摸你自己的心,你仔细看看,你现在的心中有没有一丝的愧疚?有没有感到对不住我们!”白毛地灵嘶声力竭地哭诉、呵斥着。

方源面无表情:“没有。”

琅琊地灵:“……”

顿了一顿,他又喊道:“你这样对付你的恩人,你的朋友,你的家人,你的良心不会遭受到谴责吗?!你的心情不沉重吗?!”

方源掏掏耳朵:“正相反,我很是兴奋呢。”

琅琊地灵怒气冲霄,恨不得将方源撕咬成骨渣碎肉:“方源,你这个魔头!你不要脸啊你!”

“你就是个畜生、杂种!你就是个卑鄙小人,阴险无耻,果然是人族货色!你笑吧,你笑不了多久的!”

“你这种人必定不得好死,我真是瞎了眼,居然相信你!老天爷不会放过你的!”

“如果上天再给我一次机会,我第一次见到你时,我就会杀了你,把你挫骨扬灰,把你碎尸万段!”

“哦。”方源云淡风轻,他微笑着摇头,“可惜你没有这样机会了啊。”

琅琊地灵无语了,他恨不得把眼前方源微笑的嘴角撕烂。

但他动弹不得。

他只有咒骂,发泄自己的怒火。

然而,面对方源这样不知羞耻,毫无道德感的人,他任何的咒骂都显得如此苍白无力。

\end{this_body}

