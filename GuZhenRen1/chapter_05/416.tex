\newsection{回来}    %第四百一十七节:回来

\begin{this_body}

石莲岛上。<strong>txt全集下载wWw.80txt.COM</strong>

幽魂意志的声音开始断断续续,并且整个形体也变得越加稀薄。

方源想要出手,帮助它充实自身,但却被它阻止。

“就让我消弭吧,我是幽魂意志,本体被天庭镇压,迟早会被天庭蛊仙算出来。”

“至于分魂……短时间内,却是无须担心的。”

“石莲岛不存,我的使命也完结了。”

“现在,趁着还有最后一丝底蕴,让我为你们加速时间,尽最后的一份力量罢。”

方源沉默,随后缓缓点头:“好。”

他答应下来,并且取出洁身自爱仙道蛊阵。

当即,白凝冰等人都进入仙阵当中,仙道蛊阵运转,发出震耳的轰鸣声响。

幽魂意志微笑着,石莲岛迸发出最后的孱弱余晖。

“黄史上人阵亡了!”

当这个消息传来,凤九歌以及天庭的两位八转蛊仙,心境剧烈地震动起来。

随后,更加详细的情报,从紫薇仙子处传达过来。

“方源失去了上极天鹰,居然还能掌握太古年猴这等八转战力?!”

“那红莲真传,竟是能操纵一段光阴河段,实在难以置信!”

“若非是红莲真传石莲岛,方源就算是有抗衡八转的战力,有太古年猴护卫,也绝不会能取走黄史上人的性命。唉……”

此时此刻,三仙已经绕过梦境防线,来到了光阴支流的面前。

之前方源精心布置的梦境,因为自身的流转,出现了缝隙,让三位蛊仙不费丝毫力气,轻松穿过。

梦境虽然能阻挠强敌一时,但却很不可靠,不能依赖长久。

尤其是这些梦境,已经被天意侵蚀。梦境的流转,很显然会倾向于天庭一方。

这也是为什么,之前方源四处撤离,也不愿意利用梦境结成防线,来负隅顽抗了。

得知黄史上人的死讯,三仙的脸色都很不好看。

依照原本的作战计划,是黄史上人在光阴长河中出手,捣毁红莲真传,并且将方源等人逼迫出来,再让三仙出手,头尾齐攻,围剿方源等影宗残余。

但现在,最强的一点,黄史上人反而率先折损,这让凤九歌等人的处境,就显得颇为尴尬。9;\&\#32;\&\#25552;\&\#20379;\&\#84;\&\#120;\&\#116;\&\#20813;\&\#36153;\&\#19979;\&\#36733;\&\#65289;

究竟是等还是不等?

继续等下去,方源等人没了逼迫,就会出来吗?

三仙迟疑,他们精心准备,千里迢迢而来,没想到这一战还未打,就已经士气下降。

“继续等。”这个时候,紫薇仙子传达过来新的指示。

“我已经推算清楚,方源能够掌握的情报中,光阴支流只剩下这一条。”

“黄史上人虽然牺牲,但却极有价值,探查清楚了方源等人的底细,提供了许多宝贵的情报。尤其是关于红莲真传,我天庭对红莲真传再也不是一无所知!”

“方源等人绝不可能终身都躲在红莲真传当中。他们必须得出来。”

紫薇仙子语气坚决,令凤九歌等人坚定了战意。

紫薇仙子乃是当今五域,屈指可数的智道大能,她的推算是绝不会错的。

“那就继续等!”

“该死的天外之魔,为祸苍生,此次我必定要取走你的性命,把你抽筋扒皮,挫骨扬灰!!”

两位天庭蛊仙愤愤不平,凤九歌面色沉静,望着潺潺流动的光阴支流,默默不语。

然而,一天过去了,两天过去了,连续五六天过去了。

光阴支流,不见动静。

十天过去,半个月过去,苦苦守候的三仙,连方源的一根毫毛都没见着!

不过,三仙却没有丝毫的动摇。

他们是现在或者将来的天庭蛊仙,如此人物,又岂会缺乏耐心?

尤其是凤九歌,和方源交手了数次,见识到他恐怖的成长速度后,早已经将方源列为世间大患。此次下定决心,定要尽早铲除方源,否则不会心安。

“方源究竟在何处?”

“他们还在石莲岛上吗?”

“为什么我总有惴惴不安之感呢?”

天庭中,紫薇仙子却是心生困惑。

自从黄史上人牺牲后不久,她就彻底失去了对方源等人的情报。原本种下的侦查杀招,方源等人借助洁身自好仙道蛊阵,以及红莲真传的威能,加速时间,彻底解决了。

虽然没有任何一个证据,证明方源等人已经逃脱光阴长河,但是紫薇仙子的感情,却是隐隐表明不妥。

这点引起紫薇仙子的重视。

智道蛊仙,并非单纯理智,智道修行有三大重点,分别是念、意、情。

其中情绪、情感,也是智道蛊仙的拿手手段之一。

因此,紫薇仙子心中的惴惴不安,绝非凭空而生,而是一种示警。

“假设最坏的情形,方源目前已经逃脱了光阴长河?”

“但凤九歌等人早已经将出口,围堵得严严实实,那么他究竟是如何成功?”

“是继承了红莲真传,拥有了什么极其出众的潜行手段?还是……借助了别的光阴支流?”

紫薇仙子忽然脑海中灵光一闪,方源等人掌握的两道光阴支流,目前已经全数暴露。

这个情报,十分确切,因为正是来源于幽魂本体!

幽魂本体被龙公镇压之后,天庭自然不会放过他,对他进行搜魂。然而幽魂乃是魂道巅峰人物,就算死后魂魄残余万一,又岂是寻常手段能够轻易搜得了魂的?

天庭方面竭尽所能,不断压榨,但幽魂本体仍旧在负隅顽抗,天庭搜刮而得的情报并不多。

“如果方源不是利用这条光阴支流,那么他选择哪里作为出口呢?”

光阴支流是一道道的出口,凤九歌等人堵住了最后一道。方源就算没有侦查,也绝对早有预料,这样的情况下,方源或许铤而走险,闯荡其他光阴支流。因为与其和准备充分的天庭伏兵对战,反倒不如突击那些毫不知情的势力。

“而在西漠,掌握光阴支流的势力,共有三个。”

“第一个是田家,此等家族主修木道、土道,经营仙窍最为著名,乃至五域都有美誉。是当今西漠蛊仙界中,底蕴最为雄厚的超级势力之一。”

“第二个是弓家。弓家蛊仙擅长遥击战法,尤其是诅咒杀招,无形无质,常常间隔万里,悄无声息地取敌性命。弓家最闻名的一项仙道杀招,名为杯弓影,威慑整个西漠蛊仙界。只需汲取到对象的一缕影子,收藏起来。然后施展杀招,不管间隔多么遥远的距离,都能杀伤目标。”

“第三个则是唐家。唐家破落,乃是西漠超级家族当中,实力中下。但唐家仍保留仙蛊屋,掌握的光阴支流,就在大本营的附近。因此,有着强大的防御力量,其他势力轻易抢夺不得。”

“如果方源要强行闯关,会是哪一家?”

“若是这个设想,现在三家都没有动静,恐怕方源还在光阴之河中了。”

紫薇仙子沉思。

她对西漠蛊仙界的情况,了若指掌,如数家珍。天庭的情报强大,由此可见一斑。

唐家。

仙道杀招的残余光晕,纷纷凝聚在方源等人的身上。

“此招下去,可保我等三个月时间。不仅抵御蛊仙推算,而且还能避免天意的关注。”方源缓缓睁开双眼,脸上的喜悦之色一闪即逝。

其余蛊仙,白凝冰、黑楼兰、影无邪、黑菟、妙音仙子,一个不缺。

紫薇仙子的猜测,只对了一半。

方源的确没有回去和天庭伏兵死磕的打算,他已经放弃了那条光阴支流。但是早在十天前,方源一行人就已然脱离了光阴长河,回到了西漠。

“走吧。是时候离开西漠了。”方源率众而出,身后群仙紧随其后。

这里是地底深处,众人一路直上,来到地面。

两位西漠蛊仙,已经在出口站着等待了。

“唐方明(唐烂柯),见过影宗宗主。”两位蛊仙都很年轻,见到方源之后,悠然施礼,风度翩翩。

方源哈哈一笑:“有劳二位,这是贵家族借出的仙蛊。”

说着,方源伸手一挥,几只仙蛊缓缓飞出,交到唐方明的手中。

唐方明仔细端详,查明仙蛊无损之后,这才谨慎地将它们收入自家仙窍里头。

他有些不好意思地道:“影宗诸位见谅,此次我以个人名义,向家族借蛊,干系颇大,不得不谨慎一些。”

方源笑了笑,唐方明如此说,自有深意。唐家毕竟是正道势力,和魔道影宗合作,若是暴露出去,必然引人诟病,惹出无数风波。

不过,阵营并不代表什么,利益才是世人永恒追逐之物。

唐家势弱,外部压力相当巨大。如此生活环境,让唐家蛊仙几乎都具备了拼搏、冒险、吃苦、隐忍的精神。

唐家的大略是发展梦道。

方源五百年前世,唐家因开发梦境走在前头,在五域乱战时期,成果无数,实力暴涨,最终一跃而上,成为西漠正道中的第一势力!

方源因此联系唐家,愿意用影宗在梦道上的研究成果,来换取对方开放光阴支流,为自己撤退打通一条最关键的退路。并且愿意在以后,和唐家一起联合,共同研究梦境,大力资助胆识蛊。

这等巨大的利益,让唐家上下都不得不怦然心动!

胆识蛊的支援,方源的解梦杀招,影无邪的引魂入梦,都是唐家蛊仙渴望至极的东西。

当然,后两者唐家要得到,绝不是那么容易的。(未完待续。)

\end{this_body}

