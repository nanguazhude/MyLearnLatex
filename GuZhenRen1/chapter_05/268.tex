\newsection{试演杀招}    %第二百六十八节:试演杀招

\begin{this_body}

有了境界,方源当场就设想出了一个仙道杀招。<strong>最新章节全文阅读qiushu.cc</strong>

因为是以万我杀招为根基,所以这招便被方源称为万我第二式。

但是这个杀招刚刚草创,距离运用,还有一段距离。

轰轰轰!

雪胡老祖、狗尾续命貂毛里球两位八转存在,不断夹攻逆流河,打得河水波涛起伏,逆反无数攻势,河面开始逐渐下降。

方源的所作所为,早已经触怒了这些人的底线。更何况他不愿低头,让雪胡老祖、毛里球等都甘愿耗费大力气,耗光逆流河,也要斩杀了方源。

方源视察了河面,心中估算:“照此下去,还有不少的时间。”

他的心中稍微安定了一些。

逆流河不愧是天地秘境之一,承受着两位八转存在的夹攻,也只是缓慢消耗。

当然,这其中也有一个原因,那就是雪胡和毛里球,都未全力动手。

他们之间相互忌惮,并且碧晨天、威灵仰等中洲蛊仙也站在一旁,伺机而动。这些人相互牵制,相互防备,怎可能将全部的精力投注在方源的身上?

正是因为这个复杂的场面,无形中替方源缓解了许多外在压力,让他能够从容不迫地试验杀招。

第一次催动。

方源小心翼翼。

首先是坚持仙蛊,它一直在损耗红枣仙元,让方源能够在逆流河中自由自在,不受天地秘境的限制。

随后是大量的水道凡蛊,一一升腾起来,在方源的仙窍中,围绕着体态庞巨的坚持仙蛊,不断上下飞舞。

不久,大量的水汽升腾而起,隐约的蓝芒在水汽中依稀绽放。

然而很快,这些蓝芒就消散掉,有的甚至由蓝转红,原本冰凉的水汽,也迅速升温,变得越来越热。

“失败了。”方源立即停下。

眼前的景象,超出了他的意料。

这不是他预想到的景象。

停下仙元的灌输,啪啪啪,一阵微响,大量的水道凡蛊在半空中就都陡然爆裂开来,当场毁灭。

方源沉下心来,不断思考。

按照他设想的,应当是蓝光渐盛,水汽越凉,为何会出现红光,并且温度反而越高呢?

方源不断在脑海中推算,很快,得益于他的智道修为,他查找到了根源所在。[看本书最新章节请到棉花糖小说网www.mianhuatang.cc]

那就是当中采用了两种比较对立的水道凡蛊。

方源立即将其中之一替换,在逆流河中,他能够开启仙窍,自然也能够沟通宝黄天,想要收购一些水道凡蛊,是轻而易举的小事。

修改了整个仙道杀招,方源再次试验行动,开始第二次催动杀招。

坚持仙蛊始终在催动着。

周围的水道凡蛊,陆续升腾而起,围绕着它不断旋转飞舞。

先是水汽蒸腾,又是蓝光渲染,很快就过了这一阶段,方源开始催动仙蛊挽澜。

六转力道仙蛊挽澜。

这是整个仙道杀招的第二核心。

第一核心,自然是当之无愧的七转坚持仙蛊了。

方源小心翼翼。

在催动了坚持仙蛊的基础上,再催动挽澜,是很具风险的一件事情。

一旦调和不好,两只仙蛊之间反而会相互干涉、损害,甚至有可能导致一只或两只仙蛊毁灭。

试验仙道杀招,不仅是对蛊仙有着风险,对于参与其中的仙凡蛊虫,亦是如此。甚至因为仙蛊本身,蕴含着大道碎片,若受反噬,损毁更甚。

方源的小心谨慎,很快得到了回报。

他遇到了麻烦。

尽管挽澜仙蛊渐渐催起,但散发出的玄妙力量,始终不能掺和到大框架里头。

“是哪里出现的问题?”

方源停下来。

又是一大批的水道凡蛊毁灭。

试验仙道杀招,必须付出这样的代价。

一个完整的行之有效的仙道杀招,从设想到试验,蛊仙都要付出巨大的精力和资源。

外界的攻势连绵不断,方源置若罔闻,他乃是至尊仙体,脑海中念头活泼灵动,此起彼伏。

再加上他智道宗师境界,智道道痕,智道手段,都促使他推算迅猛。

这一次比上一次还要更快!

仅仅几个呼吸的时间,方源就找到了根源。

坚持仙蛊、挽澜仙蛊的力量,不能融汇一体,显然是那些水道凡蛊数量和种类不到位。这些凡蛊就是在两只仙蛊中间,起着沟通和融洽的巨大作用。

方源思考片刻,将其中的一部分水道仙蛊数量减少,将另外一部分则稍稍增加,同时有添上其他一些种类的水道凡蛊。

第三次催动。

这一次,方源一路顺风顺水,将挽澜仙蛊成功地增添到了整体当中。

几乎在成功的一刹那间,方源脚下的逆流河水泛起了波澜。

终于开始见效了!

不过距离整个目标,还有一大段的距离。

方源没有停歇,眼前的情况,也不容许他停歇,他继续试验,不断催动仙道杀招,导致身上气息时强时弱,明晦不定。

“这柳贯一在干什么?”

“他身上时不时地浮现出各种蛊虫气息,看样子,居然在试验仙道杀招啊?”

除去雪胡老祖、毛里球,其余蛊仙都在旁观。

方源试验仙道杀招,每一次都会泄露出蛊虫气息,隐瞒不住在场的其他蛊仙。

要隐匿蛊虫气息,非得是特殊的仙道杀招。

这种仙道杀招,往往也比较稀有,比如方源的暗歧杀,就份属此类。

方源能设想出万我第二式,已经到达自身极限,不能在考虑加上隐匿蛊虫气息的效用了。

不过这也无所谓。

单凭蛊虫的气息,也很难判断蛊虫的种类。

更何况方源临时创造出的万我第二式,复杂多变,涵盖许多蛊虫,不怕被人当场瞧出什么破绽来。

第四次、第五次……

方源不断尝试催动仙道杀招,有时候进展迅速,有时候反因为一些微小修改,导致又回到前几步骤上去。

方源毫无气馁之情,面无表情,坚持不懈。

“果然是在试演仙道杀招啊。”

“呵呵呵,这个柳贯一莫非是痴傻?临时创造出仙道杀招来,他以为自己是仙尊魔尊吗?”

“就算他设想出了什么仙道杀招,我倒要看看他能否凭此,对付在场的所有蛊仙?”

观战的蛊仙们议论纷纷,对方源的天真,嘲笑鄙夷。

若仙道杀招真有这么容易创造,那还是仙道杀招吗?

但很快,有些蛊仙微微变了脸色。

他们发现,方源身边的逆流河正在发生一些玄妙的变化。

“第二十一次。”

方源伸出手掌,对准脚下的河面,微微用力虚抓。

“起来。”与此同时,他口中轻呼。

砰。

一声轻响,他脚边的河面上忽然鼓起了一个大包。

这大包自然全部由逆流河水构成,鼓起来后,旋即炸裂,河水四溅。

杀招催动失败!

方源的鼻腔中,缓缓流下两股鲜红的血迹。

“他的仙道杀招,对逆流河有效?”余艺冶子见到这一幕,双目顿时绽射一抹刺眼的精芒。

“不,更准确的应该这么讲,他要利用这条逆流河,形成仙道杀招!”不真子脱口而出道。

“哼,可笑。”玄极子冷笑连连,“想当初,我为了利用逆流河,设想出子母逆命祭炼大阵,足足耗费了我数年的光阴。柳贯一,你就算成为逆流河主,也未免太过托大,居然想在这么短的时间里,创造出相应的仙道杀招来?”

显然,几乎在场的所有蛊仙,都不看好方源。

但接下来,方源又开始施展。

呼啦!

这一次,他脚边的河面上涌起了一个小小的土丘。

土丘随即崩溃,河水将方源浇个通透。

方源面色骤然一白,身躯摇晃,这一次不仅是鼻腔,就连耳窍都流淌出鲜血来。

“又失败了,人如故!”下一刻,方源彻底恢复。

“找找失败的原因。似乎是最后的关头出了点小问题……”方源陷入沉思。(未完待续。)<!--80txt.com-ouoou-->

\end{this_body}

