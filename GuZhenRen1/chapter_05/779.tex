\newsection{感触命运}    %第七百八十二节:感触命运

\begin{this_body}

至尊仙窍,小南疆。

轰隆!

一声巨响。

大地颤抖,土道杀招的氤氲光斑,覆盖了附近方圆百里。

像是一只无形的大手捏动着软泥,周围的地貌开始内凹,外围的土壤则上凸,最终形成一片圆形山谷。

山峰一座座围成一圈,将里面的凹地包裹起来,形成内外隔绝的环境。

而在这凹地当中,并无植被,土壤也稀少,大量的钢材裸露在外,直面天地风云。

“这就是白钢巨碗山谷!”站在山谷外,石人蛊仙石狮诚仰望着这一幕,心中震撼。

方源的手段层出不穷,气势恢宏,而且随随便便就打造出了又一个中型资源点。

这些天来,石狮诚和其他蛊仙也有广泛交流,更知道了至尊仙窍是多么的广阔浩瀚,大量的资源点分布洞天各处。

这样的深厚底蕴叫人咋舌,更激增这些蛊仙对未来的信心。

然而更令石狮诚心中敬佩的,是方源的气魄!

别的蛊仙不敢,但方源却敢以弱对强,对南疆正道出手。

最关键的是:他不仅出手了,而且还大获全胜,俘虏了这些南疆蛊仙。

若没有这些俘虏,方源怎么可能敲诈得这些资源,轻易建设出白钢巨碗山谷呢?

和方源比较起来,石人一族的其他蛊仙族老就显得守成迂腐。

“方源大人才是值得我去追随的人呐!”石狮诚心中的倾向越来越偏向于方源,他完全被方源的魅力所打动,被他的气度折服。

这些天来,黑楼兰等等蛊仙都在四处忙碌,已经是将各个资源都安置妥当。

原本至尊仙窍贫瘠,而吞并来的福地洞天集中了大量的资源。而现在,这些资源被巧妙且融洽地分散出去,形成平衡和谐的生态。

没有混乱,也没有相互之间的干扰,而是互相促进。

这是方源动用智道,精心推算出来的结果。

接下来,就是补齐一些缺陷漏洞。

比如白钢巨碗山谷,就是铁区中想要建设,但还未着手的东西。有了它,铁区中就能自己喂养刃蛊,而不是依赖外界。

方源如今已是补齐了。

上一世他就做过这个工作,这一世更是驾轻就熟。

至于建设必需的资源,都是方源从南疆正道勒索过来的。

对于南疆正道的勒索,方源明显感觉到这一世比上一世,要容易得多。

南疆正道蛊仙抵抗的情绪更加低弱。

方源之前暴露天庭的情报,表明自己斩杀了陈衣和雷鬼真君,应当是带给南疆正道极大的触动!

“连天庭都栽在方源这魔头的手中,我们南疆正道栽了,有什么奇怪的?”

这种想法就算南疆蛊仙不承认,多少心里都有一些。

再加上方源勒索的经营丰富,技巧也是更加老道,这些南疆正道只有就范。

这一世勒索敲诈,就和上一世又有区别了。

上一世方源主要要建设年华池,勒索了许多珍稀仙材。现在他完全不用,改换其他东西,当然主要也是仙材。

这些仙材的一小部分,是用来填补至尊仙窍中的缺漏,比如建设白钢巨碗山谷。

另外的大部分则是为了将来升炼仙蛊做准备。

方源手中的仙蛊,比上一世还要多!

六转仙蛊占据绝大多数,对于八转修为的他而言,显得过于细弱了。

所以,大规模的升炼仙蛊是有必要的,对方源的综合实力会有极大的增强。

上一世他就这么干过,这一世吞并了琅琊福地,有这么的炼道蛊仙,他会更加从容。

他未雨绸缪,提前就先将这些仙材准备好。

吞窍的成果,已经快要消化完毕了。整个至尊仙窍开发程度,已经有百分之十六,这个数字已经靠近他上一世的极限。

但这一世,他手中还有刘浩、巴十八的仙窍没有吞并,还有兽灾洞天。

杀招继续联系,真传还在研究,人道仍旧在感悟。

方源抽空,再一次去往光阴长河。

他对纯梦分身还是念念不忘。

纯梦分身的计划虽然受挫了,但是方源又想到了石莲岛上的未来身杀招。

“或许,我可以依靠未来身杀招,来提前拥有蛊仙级数的纯梦分身!”

今古亭中四旬子时刻监察动静,但方源的踪迹他们根本无法发觉。

方源甚至比他们更加了解今古亭。

顺利地回到石莲岛,未来身杀招加持到了纯梦分身上。

然而,让方源失望的是,纯梦分身根本没有动静!

“是因为纯梦分身乃是由梦境所化,所以绝大多数的杀招对他无效么?”

“还是说,纯梦分身根本就没有成仙的未来?”

方源的纯梦分身已经是五转巅峰的修为,就差临门一脚,渡劫升仙。

然而,这一关有着致命的风险,方源明智地停下了脚步。

“如果未来身杀招无效,是因为纯梦分身没有成仙的未来。这就是说明:纯梦分身渡劫的话,我依靠如今的实力根本没有希望渡过?”

想到这里,方源忽然有了一些兴趣。

“尝试一下也是无妨。”

他当即就在石莲岛上,催出仙道杀招石洞天机。

此招以天机仙蛊为核心,能够洞察对象下一次灾劫的内容。

一片空白。

石洞天机对于纯梦分身,没有任何效果。

“是因为纯梦分身的特殊性,还是因为石洞天机目前无法针对梦道灾劫呢?”

方源苦恼。

能够对纯梦分身有效的蛊虫很少。

舍利蛊能对纯梦分身有效,那是因为它是人道蛊虫。纯梦求真体乃是人族之身,因此人道蛊虫有效。

但事实上,其他很多的蛊虫对于纯梦分身是无效的。

因为纯梦分身仍旧是梦境所化,残留着梦境的特征。

方源之前多次依靠梦道的优越性渡过难关,或者呼风唤雨,但现在他被这种优越卡住了。

就像监天塔主、龙公等人,无法抗衡引魂入梦,魔尊幽魂陷入梦境也要成眠,旧有的手段对付梦境效果微乎其微。

说实话,面对灾劫方源其实有不少的手段。

比如招灾仙蛊,又比如后患无穷杀招,但这些手段能够对付梦道灾劫吗?

方源心底摇头,这几乎是不可能的!

“唯有的一线可能,就是开发出相应的梦道杀招。”

然而,方源的梦道境界牢牢卡着他的脖子。他五百年前世可没有重点钻研梦道,而是在血道上打转。

从光阴长河回来不久,龙人分身那边却是有了令方源惊喜的进展。

夏家表示,南疆商家的宝界中就收藏有惊涛升龙火,目前正在着手交易。

“商家宝界……是商家的创始人商家老祖的仙窍吧。”

“商家老祖死后,将它留在了商量山中,无法进出蛊仙,就连商家的蛊仙都不知道它是洞天,还是福地。”

“看来这个宝界的底蕴,比我原先估量的还要深厚啊。”

方源感叹了一下商家的底蕴。

商家在南疆蛊仙界中,一直处于中高层,因为独到的地理位置,和其他的超级势力的关系保持着和睦和融洽。

“如果夏家能交易来惊涛升龙火,那么龙人分身的障碍就少了一半呢。”

方源一边耐心等待,一边安心发展。

对于催使五禁玄光气,他越发有心得体会,渐渐能够同时催出四种不同的光气。

魂道修为一次次的暴涨。为了龙人分身,方源又分出一团魂魄,即便如此,他的魂魄修为也逐步上升到了八千万人魂!

和池家的梦境交易,一直在暗中持续。

池曲由妥协了。

他一步走错,只能越陷越深。

当然,方源从未放下对他的警惕。这位八转蛊仙可不是好惹的人物,拿捏不稳,恐怕就要反噬过来。

好在方源早就选择扶持池家,布局将来。

在交易中池家得到梦道成果,这让池曲由痛并快乐着:既然这场交易已经铸成大错,为什么不继续下去呢?

池曲由不想冒险,但方源弄了这么一手,他只好咬牙,一条路走到黑。

余暇之时,方源对那些蛊仙魂魄搜魂。

虽然上一世他也这么干过,但这一世他还要这么干,因为他有另外的东西需要研究。

那就是墨水效应。

和上一世对比,这一世的种种细微不同,让方源感触颇深。

一种难以言说的玄妙感觉,逐渐萦绕他的心头。

“这就宿命吗?”方源仿佛接触了宿命蛊编织出的网的一丝一缕。

他渐渐有了一种强烈的预感:“在这方面继续钻研下去,我对宿命的了解将越发深刻。这是理解宿命最好的方向。”

“再重生几次,搜集更多的线索,然后加以推算,我恐怕能构思出宿命效果的凡道杀招了。”

“若再结合运道的奥义……命运么……”方源心头微震了一下,眼中精芒烁烁。

天庭。

凤九歌驻足在绣楼之下。

这座八转仙蛊屋乃是一座低矮的楼阁,飞檐琉瓦,红墙玉砖,精致华美,但却缺少一半基座,仙蛊屋上很明显可以看到破损残败的痕迹。

在绣楼的上空,有三张血皮被至直接缝在空中,仍旧泄露出一丝丝洪荒凶蛮之气。

当年,狂蛮魔尊进攻天庭,遭受绣楼阻碍。最终狂蛮魔尊把绣楼打得半毁,至今天庭都无法彻底修复,但他也留下了三张血皮。

从琅琊福地战败归来,凤九歌就一直沉浸在一种妙悟的状态中。

他并不修行变化道,专门来到此处却是从这三张血皮中,领悟到一丝“变化”的感觉。

这种变化,是一种意境,是命运的变化,是狂蛮魔尊的创伤。

狂蛮魔尊是强,绣楼是弱,强者被弱者留下了三张血皮,这是一种出乎意料的变化。

就像天庭之前进攻琅琊福地却遭受失败,南疆蛊仙追击方源却被俘虏,是一种同样的变化。

这连续的变化,带给凤九歌一阵阵灵魂深处的触动!

妙悟渐入佳境。

凤九歌的嘴角显露出一丝微笑:“看来我的下一首歌,便是——命运歌了。”

\end{this_body}

