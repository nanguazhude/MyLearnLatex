\newsection{人道准无上}    %第八百九十二节:人道准无上

\begin{this_body}



%1
时至如今,方源早已今非昔比。

%2
天庭成就了他震慑天下的名望和声威。即便他看起来只是七转修为,但是却已能令世间的八转蛊仙感到惧怕。

%3
“先祖。”沈从声递给沈伤一只信道凡蛊,里面有着他所知道的方源的一切情报。

%4
沈伤接过凡蛊,神念一扫,再看方源时完全是另外一番神色了。

%5
“方源?连天庭都奈何你不得,哈哈哈,有趣!”沈伤双目血芒爆闪,狂笑一声,“没想到天下竟生出你这等人物,真是好,如此一来……我就不会感到寂寞了。”

%6
言罢,一股气势从他身上狂涌而出,充斥天地,摄人心魄。

%7
方源的情报不仅没有令沈伤忌惮,反而令他涌起高昂的斗志。

%8
看到沈伤如此表现,沈从声蓦地从心底涌起一股信心来:“我恐怕不是方源的对手,但是有沈伤先祖在此,他可绝不是普通的八转蛊仙。有他在这里,战局还未可知!”

%9
正想着,一道银光猛烈绽射,充斥沈从声的眼眸。

%10
“该死!”沈从声心头警钟狂鸣,急忙飞退。

%11
下一刻,万年斗飞车呼啸而过,在空中划过一道亮丽的长虹。

%12
若晚上一息,沈从声还停留在原地的话,就会被万年斗飞车直接撞上。

%13
这个结局,沈从声都不敢想象。

%14
仙蛊屋就是这点好。

%15
催动仙道杀招会浪费时间,但直接调动仙蛊屋四处飞撞,却是瞬间就能收效的攻击手段。

%16
万年斗飞车再次飞射而来。

%17
方源调度万年斗飞车,放任沈从声不管不顾,只找沈伤的麻烦。

%18
沈伤当然也不敢用身体来试探万年斗飞车的威能,一时间,他被万年斗飞车撵得四处乱跑。

%19
沈从声以及一干沈家蛊仙想要出手相助,结果都尴尬地发现,自己仍旧不能对方源下手。

%20
“可恶!”沈从声咬牙切齿,如此一来,沈家人多的优势根本无法发挥,只能坐视方源和沈伤相斗。

%21
至于庙明神等人,自然也不敢掺和这样的大战,此刻已经不断后退,龟缩到了战场边缘的角落里。

%22
“沈怀,你一个沈家先祖,就只有四处乱窜的本领吗?”方源进入万年斗飞车内部,放声嘲讽。

%23
即便在万年斗飞车的内部空间,他也仍旧受到乐土仙尊的手段拘束。至尊仙窍中,方源也有大把的麾下,但都不能放出来。

%24
万年斗飞车只得由他一人操纵。

%25
要是能放出下属,他的人数优势可比沈家要大得多。

%26
“哼,小辈嚣张!”沈怀被方源言语所激,猛地在高空停住,施展出一记仙道杀招。

%27
沈怀飞行造诣极其雄厚,并且速度极快,能跟万年斗飞车直接周旋。

%28
他在逃窜的过程中,一直在酝酿杀招,此刻终于催发出来。

%29
仙道杀招——夜歌!

%30
凄凉的歌声迅速传播战场,蕴含的悲伤之意宛若潮水,不断充斥众人心防。

%31
歌曲节奏缓慢,以人声长吟为主,伴随鼓声和琴声,古朴且又悲凉。

%32
战场边缘的花蝶女仙、蜂将听了,已是潸然泪下。

%33
庙明神、任修平等人,也神色悲伤,情难自已。

%34
方源冷哼一声,万年斗飞车熠熠生辉,替他挡住夜歌威能。

%35
但沈伤已经达到了自己的目的。

%36
他催动夜歌杀招,并不是要直接对付战场的蛊仙,而是建设战场环境。

%37
歌声笼罩海天,一丝丝的暮霭从音波中散放,渲染全场,几个呼吸之后,整片悔哭海域就化为一片漆黑。

%38
蛊仙们困在这里面伸手不见五指,耳畔充斥着悲凉的夜歌,心中斗志剧烈下滑。再听下去,恐怕要任人宰割。

%39
方源虽然受到影响很小,但周围一片漆黑,他也随之失去了攻击目标。

%40
方源立即催动侦查手段,迅速感知到沈伤的位置。

%41
沈伤似有感应,哈哈大笑:“方源,你中计也。”

%42
说着,又催动了一记杀招——感伤。

%43
方源顿时闷哼一声,竟然直接中了招。

%44
他不禁心头惊疑,迅速辨认出来:“这是人道的杀招!难怪我的万年斗飞车防备不住。”

%45
仙蛊屋攻防一体,能够带给蛊仙强大的防御,十分安全。

%46
但这种防御,并非没有短板。

%47
方源在设计万年斗飞车的时候,考虑到了绝大多数的流派,但是人道他却是短板,当时他还没有人道宗师境界。

%48
等到后来他有了这等境界,万年斗飞车却早已搭建好了,大局已定,只能微调,难以大幅度改良。

%49
其中还有一个主要的原因是,方源缺少专门防御的人道仙蛊。

%50
“沈伤的手中,定然有高转的人道仙蛊!”方源中招之后,立即明白了敌人的一部分底细。

%51
在情报方面,方源还是比较吃亏的。

%52
他对沈伤并不了解,但沈伤却已经熟知了方源的情报。

%53
方源当然有底牌,譬如八转修为、五禁玄光气等等,但这些手段他还打算雪藏,并不想现在就暴露出来。

%54
在这龙鲸乐土当中,方源不能对沈从声、庙明神、任修平这些人出手。若是他们得知这些秘密,将来出了这片乐土,很容易就会流传到外面去。

%55
所以,方源还是尽量想用已经暴露出来的手段对敌迎战。

%56
仙道杀招——破晓剑!

%57
万年斗飞车气势暴涨,绽射出冲天的白光。

%58
银白光辉中汩汩涌现出光阴河水,从河水中又反射出一缕缕的银色光线。

%59
银色光线迅速凝聚成形,化为一柄柄利剑模样,暴射而出。

%60
飞剑群呼啸而过,直指沈伤而去。

%61
破晓剑乃是货真价实的八转杀招,威力绝伦,但沈伤见到飞剑群,竟然哈哈大笑,不闪不避。

%62
银白飞剑纷纷射在他的身上。

%63
沈伤身上涌动着一股玄妙气息,他没有被带走性命,而是遍体鳞伤。

%64
但很快,他浑身上下的伤势都消失不见,转变成一条条道痕。

%65
“好厉害!族中记载果然没有错,沈伤先祖最擅长的便是治疗手段哪。”沈从声见到这一幕,心中惊叹。

%66
沈伤年轻时,是一位治疗医师。在蛊师时期,他就开始游历东海,因为他治疗手段强大而且独到,很快就声名广播。又因为乐善好施,时常救助贫苦之人而不收丝毫诊费,为人称颂,走到哪里,都能得到热烈的欢迎。

%67
正是因为他从年轻时候,就建立起了这样的名声,导致后来他暗中堕落,种种魔道行径,东海蛊仙都没有将这些罪行联系到他的身上。

%68
万年斗飞车中,方源眯起双眼,感到不妙:“这治疗手法,又是一记人道手段!”

%69
正常情况下,蛊仙受伤,伤口上有浓郁的异种道痕,很难治疗。

%70
但沈伤身怀人道手段,治疗自己起来效率极高,收效极快。

%71
飞剑群连绵不绝,继续攒射,许多飞剑都洞穿了他的身体。

%72
但这些伤口几乎在瞬间,就被彻底治愈。

%73
仙道杀招——两败俱伤!

%74
沈伤见时机成熟,催出早已在酝酿的后续杀招。

%75
万年斗飞车中,方源立即闷哼一声,身上瞬间形成上百个伤口,伤口迸裂,血液四溅。

%76
伤口处充斥着人道道痕,然后这些道痕从伤口处迅速蔓延,四处扩散,连接一体。

%77
方源心头警钟轰鸣!

%78
这些人道道痕就像是一个个吊在脖子的绞绳,一旦规模再涨,就会危及到方源的生命。

%79
危机关头,方源首先催动的是智道杀招!

%80
一瞬间,他脑海中的念头像是烟花般喷涌绽放,各种各样的念头相互碰撞、汇聚、消融。

%81
“沈伤手段诡异,居然能够绕过万年斗飞车,直接伤害到我!”

%82
“如此恐怖的杀招……他的人道境界绝对高我不止一筹,恐怕是人道准无上大宗师!”

%83
“任何的杀招都要有所感应,感应到目标之后,方能生效。他是如何感应到我的?”

%84
“是了,之前的夜歌杀招,只是一个陷阱,引诱我侦查他。而他便顺着这份侦查的联系,反过来感应到我身上来。”

%85
万事万物都是相互影响的。

%86
一个拳头打在沙硕上,沙硕上出现一个拳坑,拳头上也必会沾上沙和泥。

%87
之前东海智道蛊仙华安推算方源,反被方源顺着这层联系,推算出华安的情况来。

%88
不久前,天庭蛊仙动用镇河锁莲大阵来封锁石莲岛,正是有这层镇压封印的关系,方源能顺着这层关系,反其道而行之,将镇河锁莲大阵摧毁。

%89
现在,沈伤逆反方源的侦查关系来感应到方源,也是一样的道理。

%90
仅仅半个呼吸的时间,方源已经琢磨通透。

%91
他立即动手,一边催动万年斗飞车飞退,一边挥散了破晓飞剑,同时他开始酝酿治疗杀招,准备往自己的身上丢。

%92
他的反应是如此迅捷,以至于沈伤都微微一愣。

%93
下一刻,这位被乐土仙尊亲手镇压的魔仙哈哈大笑,带着一丝疯狂:“方源!你以为你不攻击我,就没有事情了吗?哈哈哈,你看好了!”

%94
话音刚落,他竟再次催动杀招,五指张开,手掌按在自己的胸膛上。

%95
仙道杀招——轻伤!

%96
沈伤闷哼一声,中了自己的杀招,脸色变得苍白起来。

%97
万年斗飞车中,方源却是身躯猛颤,呕出一大口鲜血。

\end{this_body}

