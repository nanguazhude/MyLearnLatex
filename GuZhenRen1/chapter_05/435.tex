\newsection{胜利者}    %第四百三十六节:胜利者

\begin{this_body}

!--章节内容开始--荣欣心中那点残留的希望彻底泯灭了!

因为他知道,对方肯定不是如他一样的情况了,对方的货物一定很充足,因为这个价格绝对会引发一场年蛊收购的热潮。

“这一次,我败了。”荣欣苦笑,但没有办法,谁叫他货量不足。

不要提什么理由,失败都是有理由的,但这个世上人们看的不是理由,他们甚至不需要理由!

他们向来只看成败。

果然如荣欣估料的一样,方源降价之后,引爆了一场收购热潮。

很多买家纷纷出手,开始收购年蛊。

这么便宜的价格,可不多见。不逮住这个良机,怎么可以?

大批交易同时进行,方源日进斗金!

其他三方却按兵不动。

谢宝树、王明月在犹豫。

之前是荣欣犹豫,现在轮到他们俩了。

如果他们跟进,和方源一样降价这么多,无疑他们的年蛊也会有人大量蛊仙出手购买。但这样一来,利润真的是很低了。伤筋动骨,和他们之前的预料差距很大。

但如果不跟进,方源把他们的生意都抢走了,他们怎么办?

这个事情的关键,在于方源的存货到底有多少。

如果方源的存货多,那么谢宝树等人无疑是要跟进的。市场就这么大,蛋糕就这么多,方源货多就是胃口大,他吃了多,别人吃的就少。

但如果方源的存货不多,那么谢宝树等人就无须跟着降价,只需要维持价格不动。等到方源卖尽了货,他们仍旧再卖,而且是以更高的价格卖,自身的利益也就保住了。这场商战的胜利者,也会是他们。

所以,谢宝树等三方就开始猜测,方源究竟是个什么来头啊!

他到底是男还是女?

到底是什么势力?还是个人?

怎么之前一点风声都没有?突然间就好像从石头里蹦出来一样?!

三方是百思不得其解。

方源出现的太突然了。

其实,年蛊市场除了他们三大巨头,还是有些其他的小卖方。但这些人,三方都很快排除,因为都不可能是。

“现在这些小本经营的卖方,恐怕已经满脸惨白之色,看着这场激烈至极的商战了吧。”想到这里,荣欣的心情好了一点。自己这一次倒霉,很可能第一个出局,但是想想看,还有人更倒霉,他的心情也就稍微平衡了一点。

但这只是他安慰自己的一种方式。

对于现实根本无济于事。

三大巨头如何猜想,都猜不出方源的底细。

保险起见,谢宝树、王明月决定跟着方源的步伐,也降价,降低到和方源同等水准。

而荣欣却维持之前的价格不变。

“看来荣欣是觉得,对方没有什么存货啊。”谢宝树这么想着。

其实他不知道,荣欣是没有办法啊。

他的货不多,降价这么多的话,就会引发收购热潮,他的货就会销售光,提前出局。反而不如维持现在的价格,赌一赌。

若是方源货量没有这么多,他就赌赢了,他还有继续留在场中厮杀的机会。

赌输了也就输了,反正他肯定是要输的。

谢宝树不知道荣欣的底细,还以为他是觉得方源存活不多,想要冒一下险来赌一赌。

但谢宝树却是沉稳性情,他还是要跟进的,因为这样做最稳妥。

至于王明月,思量了一番后,也做出了和谢宝树同样的决定。

不过她身为女性,可能更多疑一点。

她心想:“荣欣怎么就不降价了?他是想赌一赌,还是真的知道对方的底细,或者有什么证据和消息?”

如果荣欣知道方源的底细,不告诉他们俩个竞争对手,也是很有可能的。

这叫坐山观虎斗,对于荣欣而言很具优势。

于是,王明月就主动联系荣欣,对他旁敲侧击。

荣欣接到王明月的消息,真是哭也不是,笑也不是,心底还生出一些对王明月的同情。总之情绪相当复杂。

于是,商战的局面就变成:方源、谢宝树、王明月三人同价,而荣欣价格更高。

这个时候,大量的蛊仙开始出手收购年蛊,或者是准备出手。

几乎所有人都心动了。

因为这个价格远远低于过往,价位真的很诱人。

要是再降价?

买家们都摇头,再降价也降低不到哪里去了,买家精明,卖方当然更不是傻子了。

于是,当方源再一次降价的时候,所有人都傻了。

“我有没有听错?那边又降价了?而且降低的幅度还是这么大?!”

“我的天呐,这是什么回事?”

“那个新人是不是傻?”

买家们都有些不敢置信。

“我才不管他是不是傻,还是有什么其他图谋?总之我知道他的年蛊是真的,他也的确再卖这样的价格。我此时如果我不买,我才是傻子!”

有蛊仙在心中呐喊。

这何尝不是其他人的内心想法?

一大波的蛊仙神念或者意志,团团包围了方源的摊位。

火爆的场面,前所未有!

买、买、买!

蛊仙们心中想的,几乎都是这个。

这种大便宜前所未有啊,再不买等到那傻子反应过来,就晚了。

买家们的心中都有一种莫名其妙的急迫感。

“这个人究竟怎么回事?!”得知这一情况,谢宝树再不能淡定。

这个价格,如果他继续跟下去,也同时降价的话,那么他的利润就真的只是那么一点了。

如果说,往年的时候,他贩卖年蛊,一只年蛊可以赚取一百,那么他若是跟着方源降下一样的价钱,那么他只能赚取个位数的利润。

他谢宝树辛辛苦苦栽培沧桑树是为了什么?不就是为了培育年蛊,赚取利润吗?

若是他降低价格,和方源一样,那么他的赚钱的初衷就几乎毁了!

这个价格大大突破了他的心理底线,若是这样卖,还不如不买。

“但是……”谢宝树皱起眉头,陷入深沉的思索当中。

与此同时,王明月也在忧虑。

她的情况和谢宝树一样,跟进方源的话,她同样不会有多少赚头。

但这样,就不卖了吗?

事情不是这么简单的!

如果他们不跟进,方源的货物规模庞大,那么宝黄天市场的蛋糕,就几乎要被方源一口吞下了。

还有一种可能,那就是方源故意降价,只是一种商战的策略。

他只是故意降低这么多,一方面是试探诸位卖方巨头,另一方面也可能是打响自己的名声。

“如果我不跟进,就是怯弱的表现,但可以保存实力。若是我跟进,就是和那新人一起疯,真正拼杀了。”

王明月想了半天,终于还是决定跟。因为三方当中,她的货量最多,有着底气可以继续下去。

但这个时候,谢宝树已经停住了脚步。

市场上年蛊的卖价,呈现出方源和王明月同时在低位领跑的情况,谢宝树第二高,荣欣最高。

很显然,后两者的摊位是无人问津的。绝大多数的买家,都集中在方源和王明月的摊位上。

谢宝树在观察。他需要知道,这是方源在逞强,还是他的实力就是如此强大。

王明月很紧张,因为这样的卖价,她是很亏的,但三大卖家之中,就只剩下她了。

荣欣反而成为最悠闲的一个人,看着他们厮杀。

“火候差不多了,是时候致命一击了。”方源一直很冷静。

商战至此,他已经大致试探出了三大卖方的底线,虽然还有王明月再跟进和他纠缠着,但按照以往的市场行情,三大卖家实力应当差不多,王明月纵然实力稍高一些,也高不到哪里去。

想到这里,方源再次降价。

但这一次降价,他却没有大幅度地降低,而只是降低了一小块。

不过,就是这样很小幅度的降价,却像是一把锐利的剑,直接刺在三大卖家的心头。

“这?!”荣欣脸色发白。

谢宝树眯起双眼,脸上阴云密布。

王明月双手捏紧,暗自咬牙,脸色惊疑不定:“何必如此?何必如此呢?”

方源的这一次降价,虽然幅度小,但是却很致命,因为都超过了他们三大卖家的成本价。

年蛊既然被贩卖,当然也有成本。

荣欣炼蛊需要蛊材,需要仙元消耗来催动种种炼蛊手段。谢宝树培育沧桑树,也需要耗费物力人力。王明月虽然是从光阴支流中捕捉野生年蛊,但捕捉的手法需要仙元,同时,维持和保护光阴支流,也需要资金的投入。

卖价低于成本,那就真的是亏本甩卖了。

这往往是那些积压的货,卖不出去,或者急需要资金周转,才会在百般无奈之下,才使出来的招数。

要是在商战中使用这一招,其他的竞争者兴许还不怎样,自己首先得吐血三升。

这一招太狠,对别人狠,对自己更狠。

王明月被吓住了,这一次她不敢再跟。

再跟她就是傻子!

年蛊啊,这种蛊虫居然会亏本甩卖,这是什么概念?

买家们也愣住了。

这种价格,大大超出了他们的想象极限,他们瞠目结舌,难以置信。

但旋即,蛊仙们反应过来。

宝黄天轰然!

无数的蛊仙把方源的摊位,团团地包围住,场面达到前所未有的火爆程度。

这已经不是买的概念。

而是赚钱的概念。

买这样价格的年蛊,按照以往的市场行情,自己本身就是赚的。

大量的交易在进行,就算是之前收购过年蛊的人,此时也按捺不住心动,再次向方源采购。

方源的生意难以想象的火爆,反观其他三大卖方,却是门庭罗雀,惨淡如云。

但到这个时候,三大卖家反而不急了。

“这个新人到底是什么来头?攻势如此猛烈,不过他这一次,恐怕是玩脱了。呵呵”

“观察!继续观察。”

“我就不信,他能将年蛊的价格始终保持这么低下的程度。”

三大卖家在等待,等待方源支持不住的时候。

但方源却没有支持不住这个概念。

因为他是用八转仙蛊似水流年,量产普通年蛊的,所需要付出的只是催动八转仙蛊的仙元而已。

对于方源而言,他是有成本的。但他的成本和其他三大卖家比较起来,后者的成本若算作十,那么他的成本只能算作三或者四。

年蛊人人都能卖,货色四方也差不多,商战降价的本质,其实就是比拼成本价。

方源用这个价格卖,他还有的赚,但是其他三大卖家却是亏本的。他们卖不了,方源卖得了,这场商战从一开始起,胜利者就属于方源,也只会属于他!(未 完待续 \~{}\^{}\~{})

\end{this_body}

