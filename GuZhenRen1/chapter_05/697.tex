\newsection{太古传奇第一战力}    %第七百节:太古传奇第一战力

\begin{this_body}

%1
藏龙窟战场。

%2
原本镇压帝藏生的超级仙阵,已是被东海八转们破开,大半崩溃,只剩下大阵核心中枢残存,勉强维系局面。

%3
悲风老人身受重伤,强撑局面。他的孙子风禅子则伴随左右,苦苦等候天庭援军,却始终不见援军踪影。风禅子的面色不由地越加绝望。

%4
中枢之外,东海八转们将大阵中枢尽数包围。

%5
华家太上大长老华彩云放出太古云兽,附着在大阵中枢上,不断地碾磨着大阵。

%6
“到底是天庭手笔,即便只剩下大阵的核心中枢,仍旧能顽强抵抗我云兽侵蚀。”华彩云感叹道。

%7
嗷吼!

%8
就在这时,从地沟深处传来又一声龙吼,惊天动地。

%9
声浪滚滚,将偌大的地沟都震了震,地沟峭壁上无数碎石洒落。

%10
东海八转们担忧地看了脚下方的幽深黑暗一眼,沈从声催促道:“大阵破损大半,已经快要镇压不足帝藏生了,为避免夜长梦多,我们快动手吧。”

%11
帝藏生乃是传奇太古荒兽中十分独特的一头。因为它是中洲地脉的力量,结合人的怨恨、恐惧等种种消极情绪,凝聚而成。

%12
所以,它天然地对人族抱有仇恨、杀戮之心,更关键的是,除非它的来源摧毁,否则它就是不死之身。它曾经给天庭造成巨大的麻烦,东海蛊仙们可不想平白无故地和帝藏生战斗。他们都希望夺走龙宫,一洗前耻之后,就放任帝藏生逃出生天,让它继续找天庭和中洲的麻烦。

%13
“已是可以了。”宋启元点点头。

%14
“咄。”他口中轻呼一声,

%15
仙道杀招破镜碎芒。

%16
咔嘣。

%17
一声脆响,宛若镜子掉在了地上,碎成无数碎片的声音。

%18
整个大阵中枢,也像是破镜一般,强撑的光晕破碎成万千碎片,露出里面的悲风老人和风禅子。

%19
“不!”风禅子吓得连连后退,惊恐大叫。

%20
宋启元的这一杀招,十分奇妙,刚开始时入侵目标,毫无异兆。等到酝酿足够,一朝迸发,常常能打人个猝不及防,出乎意料。

%21
这招破镜碎芒同样是宋启元的得意手段,最擅长的就是对付静止不动的对象,比如大阵。

%22
大阵一碎,引得帝藏生大吼连连,声浪滚滚,震撼整个地沟。

%23
“快,抓紧时间!”东海拥而上。

%24
“我要死了!”风禅子一下子向后跌倒在地上,吓得紧闭双眼,浑身发抖,面容扭曲,等待着死亡的降临。

%25
悲风老人深深地叹息一身,他已无再战之力,况且敌我差距太大,他似乎只有引颈受戮一途。

%26
“不要杀他,待我生擒活捉了。”沈从声忽然开口。

%27
东海蛊仙攻势为之一缓,心中纷纷诧异,难道沈从声竟有活捉生擒八转蛊仙的手段?

%28
“诸位且看好了。”沈从声自信微笑,“着。”

%29
东海诸仙顿时便见到,悲风老人身上忽然显露出无数淡光丝线,宛若一根根的蚕丝,将他五花大绑。

%30
“此是我最近方修行成功的杀招,名为绕梁之音!”沈从声解释道。

%31
悲风老人不由地面露悲愤之情,他原本有战死的打算,但没想到东海蛊仙居然还想生擒活捉他,这无疑是更大的耻辱。

%32
东海诸仙默然。

%33
看情形,沈从声真的能够活捉同级的蛊仙。尽管很大原因是悲风老人战力降至谷底,但沈从声有这样的手段,就足够让人警惕了。

%34
八转蛊仙的底蕴往往是很雄厚的,沈从声的出手可见一斑。

%35
悲风老人毫无反控之力,至于风禅子已经吓得浑身发抖,东海诸仙们将其只有六转修为,又这副怂包样子,连理都懒得理会。

%36
沈从声全力施展仙道杀招,想要活捉了悲风老人。一位活着的八转蛊仙,价值原本死去的大得多。别的不说,淡淡活捉一位同层次的大能,这份战绩就足够煊赫辉煌,能够让沈从声,乃至沈家在东海蛊仙界的声威,提升一大截!

%37
其余诸仙则开始寻找仙蛊屋龙宫。

%38
他们结伴而来的目的,就是为了这座八转仙蛊屋。

%39
很快,张阴就发现了一丝端倪。

%40
“在那里!”他手指着上方,神色振奋,随后打出一记侦查所用的仙道杀招。

%41
杀招过后,一座仙蛊屋显露身形,悬浮在诸仙头顶,距离悲风老人有数丈距离。

%42
东海诸仙大喜,此屋不是龙宫,还能是哪座?

%43
除去沈从声之外,其余东海诸仙连忙围拢上去,神色多少都有些兴奋。

%44
沈从声专心对付悲风老人,也不急迫。在来之前,东海诸仙都结下了盟约,盟约束缚力度还是相当强的。盟约规定:若是东海诸仙此行获得了龙宫,那么人人有份。至于份额多寡,则视此次东海诸仙出力多少。

%45
龙宫既显,宋启元等人自然出手,试图收服此屋。

%46
但是两三个杀招使用过后,龙宫只是剧烈颤抖,就是没有服软重新认主。

%47
“这仙蛊屋真的非同凡响!”龙宫的顽强抵抗,更让东海诸仙觉得兴奋,对龙宫更加志在必得。

%48
渐渐的,他们又在努力收服的过程中,洞察到了一个重大秘密。

%49
“原来如此!”

%50
“天庭将仙蛊屋龙宫放置在此处,是想用它来收服帝藏生!”

%51
“帝藏生虽是地脉和消极情绪的凝合,但却身具龙形,正被龙宫所克,因而有了可以收服的可能。”

%52
“龙公打的好算盘!当今五域地脉,正要合而为一。帝藏生的力量来源之一,便是地脉。一旦地脉合一,帝藏生的实力就会暴涨五倍!”

%53
“你们别忘了,地脉一旦合一,五域界壁消失,天下大乱,纷争四起,到那时定然会有越发浓郁的恐惧、憎恨、嫉妒等等情绪。这一点,又会令帝藏生实力节节攀升。”

%54
东海诸仙一交流,不少人倒吸一口冷气。

%55
如此一来,太古传奇荒兽当中,帝藏生恐怕就要成为当之无愧的第一。

%56
它的战力本来就是八转巅峰,经过这些增长,恐怕都有冲击亚仙尊战力的可能!

%57
“难怪龙公不惜代价,强夺龙宫,就是为了收服一位亚仙尊战力,来镇压局面啊。”

%58
“幸亏我们前来阻止,破坏了天庭的图谋,否则的话将来五域大局可就难了。”

%59
东海诸仙无不庆幸。

%60
“难怪大阵已破,我们只听得龙吼声声,却不见帝藏生挣脱而出。”

%61
“如今龙宫正在镇压帝藏生,看来我们需要再等等。”

%62
“没有错,等到龙宫将帝藏生镇压收服,我们东海蛊仙界不仅可以获得一座八转仙蛊屋,而且还能获得一位超强战力!”

%63
“天庭若是发现,恐怕会吐血吧。”

%64
“哈哈哈,实在是妙极,妙极。”

%65
帝君城战场。

%66
仙道杀招大盗鬼手!

%67
方源连连催动,所到之处,天庭的仙蛊屋无不辟易。

%68
大盗鬼手可以偷取蛊虫,方源掌握的这记杀招,连八转仙蛊都能盗取出来,即便是仙蛊藏身在蛊仙的仙窍洞天之中。

%69
仙蛊屋的本质,是许多蛊虫相互组并,形成的某种固定形态。可以看做是许多仙道杀招的相互融合,也可以看做是一座移动的仙阵。

%70
不管是仙道杀招、仙蛊屋、仙阵,都是蛊虫运用之法。外表有差异,本质是一致的。

%71
大盗鬼手连藏在仙窍中的蛊虫,都能盗取出来。更何况仙蛊屋这种,将蛊虫都裸・露在外从而搭建的移动仙阵?

%72
毫无疑问,大盗鬼手正克制仙蛊屋!

%73
方源依仗大盗鬼手,连连出手,盗取许多蛊虫。

%74
他之前也曾对仙蛊屋动过手,但那时是太宇寺,刚好有着应对大盗鬼手的手段。事实上,绝大多数的仙蛊屋都不具备,抗衡方源的大盗鬼手的能力。

%75
天庭精心布置的仙蛊屋防线,因为方源的关系,上下浮动,阵线四下胡扯,局面一片大乱。

%76
“方源贼子,纳命来!”危机关头,厉煌大喝一声,身披阳莽背火衣,扑向方源。

%77
天庭的仙蛊屋对付不了方源,但蛊仙却是可以应付。

%78
大盗鬼手抓取仙窍中的蛊虫,远比盗取仙蛊屋要难得多!

%79
尤其是厉煌有着阳莽背火衣,能防得住方源的大盗鬼手。

%80
方源见厉煌追杀自己而来,不退反进:“厉煌,你终于舍得出来,我早已等候你多时!”

%81
下一刻,方源猛地变成太古年猴,张口一吐。

%82
顿时,春剪杀招飞出,刺破空气,直射厉煌。

%83
之后,方源迅速地深呼吸一口气,年猴的两只巨手虚空一握,凭空显现出一把长柄大扇。

%84
正是仙道杀招夏扇。

%85
我剪!

%86
“什么?!”厉煌震惊的目光中,自己的背火衣被咔嚓咔嚓一通乱剪,防御大损。

%87
我扇!

%88
呼!

%89
飓风狂卷,天地变色。

%90
厉煌中招,一下子被扇飞出老远去。

%91
“他的实力,怎么会有如此恐怖的增长?”厉煌看着自己的阳莽背火衣只剩下一丁点的小火苗,神色十分难堪。

%92
要知道,他可是在遭受攻击的时候,见机不妙,又催动了一次阳莽背火衣。

%93
这记火道防御大杀招,还有一个难能可贵的优点,便是可以连续催动,相互叠加。当然相应的弊端就是消耗的仙元,将暴涨两倍!8)

\end{this_body}

