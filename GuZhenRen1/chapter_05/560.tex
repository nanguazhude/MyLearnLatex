\newsection{天庭招降}    %第五百六十二节:天庭招降

\begin{this_body}

雷鬼真君一回到战场,激战中的双方脸色都为之一变。

天庭诸仙面上皆有喜事浮现,方源等人却都面色一沉。

双方战得难解难分,这个时候,天庭一方增添一个八转战力,且又是一流强者,对于整个战局的影响,无疑是天翻地覆!

但雷鬼真君的汇报,旋即就让紫薇仙子目光阴沉下来:“什么,没有发现智慧蛊?”

智慧蛊虽是被星宿仙尊炼化过,炼化的过程也记载在了星宿真传当中,紫薇仙子有许多了解。但是紫薇仙子却不知晓,智慧蛊还有追逐信道环境的特性。

“智慧蛊乃是野生,他们难道将其驯服炼化了?还是毁掉?亦或者有什么抗衡智慧光晕,挪移智慧蛊的方法?”

紫薇仙子脑海中念头此起彼伏,一道道思绪如闪电般极速闪现。

她挺身而上,目视天婆梭罗银色巨人,转眼又看向方源,低喝道:“交出智慧蛊,我允许你们投降!”

天庭这一次进攻,主要目的,就是九转智慧蛊,甚至方源的问题还在其次。

仙蛊唯一,又难以炼制,九转仙蛊站立在金字塔的最顶端,当今五域,整个天下,存在着的九转蛊恐怕也不足十指之数。

九转智慧蛊,对于天庭的提升,实在太大了,仅次于宿命蛊!别的不说,单说紫薇仙子,依凭她现在的智道造诣,再以星宿棋盘辅助,乃是天底下前三的智道大能。

一旦她获得九转智慧蛊,并借助星宿棋盘成功炼化,那么她绝对是天下第一的智道大能,洞穿乾坤智道一切奥妙,稳稳站在最巅峰,把第二位也会远远甩开,拥有不可思议的推算能力。

别说方源拥有阎帝,就算是拥有八转程度的智道手段,也不能防住紫薇仙子的推算。唯有九转层次,方有可能。

但九转程度的智道手段,又有多少呢?至少方源手中的琅琊真传、幽魂真传,都是没有此类记载的!

“投降?”毛民蛊仙们俱都心头一震。

琅琊地灵怒极大吼:“投降个屁!我琅琊福地中,只有站着死的毛民,没有跪着生的奴才!”

方源倒是好奇:“我乃完整的天外之魔,你们天庭为了智慧蛊,也能宽恕我?”

紫薇仙子傲然一笑:“就算是完整的天外之魔,又能如何?天外之魔的存在,我们天庭先贤早已知晓,也招揽过天外之魔,只是没有明文记载在人族史书当中罢了。就看眼前,赵怜云就是最好的例子。”

“方源,天庭的气量远比你想象中,要大得多。就算你是完整的天外之魔,就算你拥有春秋蝉、幽魂真传等等,只要你浪子回头,褪魔归正,贡献上至尊仙窍,融入天庭,你就是天庭的一份子,按照天庭的规矩,绝对是一视同仁!”

这一番话说完,方源面不改色,带着丝丝冷笑,倒是陈衣面色复杂,目光中带着惊疑和嫉妒。

陈衣绝对相信紫薇仙子的话,因为天庭的规矩就是如此,不管你出身如何,只要归于正道,加入天庭,贡献仙窍,就是天庭一份子,绝不会区别对待。

“但是加入天庭何其难也!我陈衣辛辛苦苦大半辈子,勤勤恳恳,又是因为战力达标,又继承元莲真传,才得以加入天庭。他方源才不过修行了多少年岁?杀人放火,四处为祸,此刻放下屠刀,转魔为正,居然就能加入天庭!”

一瞬间,陈衣都有些心态失衡,觉得不公平,心中甚至还有一些委屈。

倒是雷鬼真君面色不变,当年天庭也曾经招揽过另外一个人,这个人屠戮天下,造成的危害比方源还要大得多。

他就是幽魂魔尊!

当幽魂魔尊还未成就九转的时候,天庭就屡次招揽过他。甚至成为九转之后,也招揽过一次。

“天庭是人族的天庭!方源你惊才艳艳,以七转纵横时间,单单躲避天庭追缴这么长的时间,就足以证明你加入天庭的资格。这一点我可以身家性命作保,你今日加入天庭,不仅能保住性命,还能继续修行,当你的逍遥蛊仙。”雷鬼真君开口道。

“方源……”一时间,毛民蛊仙们都担忧地望向方源。

就连蛊仙毛六也是如此。

方源哈哈大笑:“你们为了九转智慧蛊,不惜放过我,如此招揽我,可见九转智慧蛊的确对于你们而言,意义极其重大!那我就换个说法,你们若不今日放过我们,我就当着你们的面,捏爆智慧蛊,让你们图谋成空。”

“哼,冥顽不灵!堂堂正道,岂容威胁,巍巍天庭,怎会受挟?你既然想死,那我就成全你罢!”雷鬼真君大怒,向方源杀去。

轰轰轰!

方源勉强对战雷鬼真君,打得雷鸣滚滚,气浪翻天。

“哦?这么说来,智慧蛊就在你的身上?”紫薇仙子眼眸中精芒闪烁,跟着动手。

方源被雷鬼真君、紫薇仙子联手打压,立即被死死压入下风,只能靠逆流护身印硬抗,失去了全部还手之力。

他对琅琊地灵暗中传音:“太上大长老,事急!若有手段,再不用出,悔之晚矣!!”

琅琊地灵咆哮:“也罢,就让你们尝尝我琅琊福地真正的厉害!”

话音刚落,整个琅琊福地的海面都开始翻腾气泡,咕嘟咕嘟咕嘟,一瞬间仿佛是整片汪洋都被煮沸了。

无数的炼道道痕,散发出道痕光晕,不管是蛊仙还是凡人,通过肉眼都能观察得到。

仙道杀招的气息,洋溢澎湃,宛若是浩瀚飓风,一瞬间卷席整个琅琊福地,天庭蛊仙俱都不能幸免,直接中招。

“这是什么?!”陈衣等人俱都神色大变,浑身身上各种防护手段,都绽放出璀璨的华光,纷纷被激发出来。

神秘杀招威力浩瀚至极,就连方源都中招,逆流护身印上掀起无数涟漪。

“快,到我这里来。”琅琊地灵传音,驾驭银色巨人,主动飞到方源身边。

方源顺利融入天婆梭罗大阵之中,这当然不是他第一次参与这座上古战阵。

“这是什么招数?”方源脸上闪现一抹惊喜之色。

琅琊地灵施展的这一招,极其厉害,堂堂天庭四仙一时之间,竟都是全力防御,没有再出手攻击。

琅琊地灵苦笑:“这是我的本体身前,斩杀某位水道大能,获得战利品,从中感悟而出的最强炼道杀招――四海皆准!此招对海水的消耗极其剧烈,不到炼制八转仙蛊时,绝不动用。现在用来对敌,恐怕只能维持几十个呼吸罢了。”

方源仔细观察,果然见着琅琊福地的海平线,急剧下降,消耗的海水不计其数,都用来催动这记杀招四海皆准了。

方源顿时有些恍然:“原来琅琊福地的地貌,乃是精心设计。三大陆漂浮在汪洋大海之中,这海水并不简单,乃是琅琊福地的炼道底蕴,珍稀异常。”

方源又看向天庭蛊仙,叹息一声:“可惜这炼道杀招敌我不分,否则趁机出手,必定能占据上风!”

其实,四海皆准的杀招威能可以局限在某地,但这个地方事先就要铺设好相应的仙道大阵,用来接引抽取杀招威能,用来炼蛊。

而这座仙道大阵,不是别的,正是长毛老祖炼道大阵,之前用来给方源炼制万我的。

天庭蛊仙自然不会被诱骗到这么明显的陷阱中去,战况境界,琅琊地灵没有办法,只好将这招全数发挥,令威能遍及琅琊福地天上地下,四面八方。

不过这样一来,方源布置的超级仙阵也受到四海皆准的攻击。好在琅琊地灵早有准备,银色巨人奔赴过去,将仙阵尽量拆除,回收仙蛊,同时又将其他三族的异人蛊仙,也都收入银色巨人体内。

“接下来怎么办?”方源看向琅琊地灵,问道。

他总感觉,琅琊福地应当还藏有其他底牌。

而这一战的关键,就看这些底牌如何了!

\end{this_body}

