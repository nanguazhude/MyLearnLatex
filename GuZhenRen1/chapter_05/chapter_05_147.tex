\newsection{电炼马鸿运}    %第一百四十七节:电炼马鸿运

\begin{this_body}



%1
“你终于成为独当一方的强者了。纵观三个大陆,你如今的实力,绝对是前十之列。”方源假意,在方正的脑海中,故作欣慰地道。

%2
方正淡淡地道:“积累到了,这是水到渠成的事情。”

%3
这些年来,他杀人如麻。在生死存亡的压力之下,渐渐体会到了方源当时屠戮全族的心境。

%4
逼不得已。

%5
不是你死就是我亡!

%6
方正仍旧不能原谅方源,但他却理解了方源的这个做法。曾经充斥胸怀的仇恨,也随着时间流逝,随着方源假意屡次帮助他脱离险境,而渐渐地淡了。

%7
“逃啊!”

%8
“我们根本不可能会是方正的对手!!”

%9
很快,白毛、黄毛的蛊师联军溃败了。

%10
方正没有和黑毛蛊师们打招呼,稍微清理了一下战场,转身便走。

%11
留下一群黑毛蛊师,用复杂的目光看着他的背影,目送他渐渐离开。

%12
“他就是传说中的那位人族蛊师啊。”

%13
“真是厉害!眨眼间,就奠定了胜局。”

%14
“可惜他得罪了上层,又是异族,就算功劳再大,也只能达到现在的位置了……”

%15
“虽然很感激他,但为了自己的前途,还是和他保持距离为妙。”

%16
琅琊福地。

%17
方源在接受了一股假意之后,露出了然的神色:“哦,看来方正这边快要成了,反倒是血神子的蛊方还没有完善。我到底要不要分出一部分的精力,投注在这个方面呢?”

%18
本来,按照方源的计划,当然是在能够利用智慧光晕之后,再推算血神子蛊方,最为快捷省力。

%19
但是智慧光晕,遥遥无期。方源反不如直接推算血神子蛊方,要来得快些。

%20
仔细思考了一下,方源还是放弃了这个颇为诱人的想法。

%21
一方面。他有黑凡宙道真传,手段并不缺乏,血神子若是炼成,也只是锦上添花而已。

%22
另一方面。血道现在境况不好,一旦用了,就要遭受蛊仙们的追杀围剿。在方源的想法中,五域乱战时期才是血道可以冒头的时代。

%23
自从收取了成龙丘后,这些天来。方源一面等待楚度的动作,另一方面仍旧在辛苦修行。

%24
他的收益开始下降。

%25
因为黑凡洞天中,很大一部分的资源已经被他贩卖殆尽。

%26
黑凡洞天中,有许多资源,放在方源的至尊仙窍里面养不活。方源只好将它们贩卖出去,换取自己所需。

%27
如此一来,产生的利润,让他收益暴涨。

%28
可惜,这种贸易是无源之水,持续了几个月后。就萎靡将尽了。

%29
方源虽然赚取的仙元石很多,但耗用也大。

%30
每两个月就渡一次地灾,实在过于频繁。还有近些时候,方源在黑凡洞天、东海、成龙丘都屡次出手,动用七转仙蛊,耗费的青提仙元着实有点多。

%31
再加上他经营仙窍,屡次投资建设。仙元石来来去去,宛若股股洪水流过,虽然量大,但方源的结余却是少之又少。

%32
“目前能带给我巨大提升的。就是解决仙僵肉身上的魂道陷阱,再次利用智慧光晕。”

%33
“其次则是取得信道手段,可以拆解掉我身上的各种盟约束缚。让我从容利用各方势力,各种关系。达到借力打力的良效。”

%34
“再次便是升上七转,有了红枣仙元,将极大地缩减我的耗费,带给我质的飞跃。”

%35
方源对自己的情况洞若观火,可惜的是,知道并不代表能够做到。

%36
他陷入了修行当中的一种瓶颈状态。

%37
除非有机缘降临。否则的话,他只能苦苦捱过这些关口。

%38
但是机缘,怎么可能说有就有,说来就来?即便来了,方源也要怀疑这会不会是天意的布局。

%39
修行魂道的时间,方源保留在一个时辰内。

%40
其余时间,他不是在利用剑眉仙蛊,为自己增添剑道道痕,就是传送到外界去,催动星眸仙蛊,炼化黑天中的星辰。

%41
除此之外,他还要花费大量精力,偷偷锻炼宙道杀招,增长熟练程度,将来能用在实战之中。

%42
另外的主要工作,就是经营仙窍。

%43
说起来,至尊仙窍的基本建设,方源至今还未完成。

%44
原因就在于他的仙蛊实在太多了。当他能自给自足,不需要依赖外界,单靠至尊仙窍中的资源产出,就能喂养他的全部仙蛊。

%45
这个时候,基本建设才算完成。

%46
可惜,方源的八转仙蛊众多,态度蛊、慧剑蛊的食料,方源至今只是开了一个头,后续建设都未跟上。

%47
没办法,他已经做到了极致。

%48
每一天的时间几乎是掐着分、秒计算,他对时间的耗用极为吝啬,但时间和精力终究是有限的。方源只能仔细思考,妥善分配后,在抓住重点展开修行。

%49
如此努力,换来的是方源的迅速进步。

%50
几乎每一天,他都有可喜的收获。

%51
至尊仙窍本就逆天,方源的刻苦简直非人,他对旁人狠辣,对自己更是要求极其严苛。

%52
两相集合之下,仙窍里的发展堪称日新月异,方源实力在时刻拔升。

%53
如此过去小半个月,方源接到了楚度的来信。

%54
在信中,楚度请方源出手相助,并且事成之后,给与方源相应的丰厚报酬。

%55
楚度的语气十分客气,似乎忘记了方源本身就是他的盟友。

%56
“楚度能够隐忍这么长时间,算是难得了。也罢,我就走这么一趟。”方源早已心理准备,立即动手启程。

%57
一天一夜之后,他来到秘密集合的地点。

%58
当他到达这里,已经有不少蛊仙到来。

%59
“前辈,你来了。”迎接方源的正是那位,和方源一同进攻成龙丘的楚门中的蛊仙。

%60
“我来为前辈介绍一下,这几位都是师傅请来助拳的强者。这位是昊震前辈。”

%61
昊震中年模样,络腮胡子,身材魁梧,对方源点了点头。

%62
“这位是仇老五前辈。”楚门蛊仙接着介绍。

%63
仇老五一脸苦相。体型干瘦,背部微驼,对着方源展开笑容,露出一口参差不齐的黄牙。比常人哭还难看。

%64
“这位是李四春前辈。”楚门蛊仙说到这里,眼神有些躲闪。

%65
这位蛊仙就很有特色了。

%66
他国字脸,浓眉大眼,鼻梁笔挺,浓密的黑色胸毛裸露而出。明明是男性,却穿着花裙子。

%67
见到方源,他双眼放光,提起兰花指,娇声道:“哎哟喂,来了个帅弟弟啊。”

%68
楚门蛊仙浑身一抖,连忙让过此人,来到最后一位蛊仙身前。

%69
“这位是汪大仙前辈。”

%70
汪大仙身材矮小,三角小眼睛,酒糟红鼻子。瞅了瞅方源,说道:“我们都介绍过了,你最后来的,怎么不介绍一下自己?”

%71
方源淡淡地笑了笑:“我有什么好说的,不过是一届散修罢了。你们可以唤我为柳贯一。”

%72
“柳贯一。”楚门蛊仙连忙将这个名字记在心头。他还是第一次听到方源的名号,自从方源出手,轻而易举地赶跑了百足家的蛊仙,又将成龙丘夺走之后,他就对方源产生了强烈的好奇,还有崇拜。

%73
事实上。不只是他,其他楚门中的蛊仙也对方源十分神往。

%74
毕竟,他们都是主修力道。方源力道战力如此浑厚,在力道式微的大环境下。仿佛就是他们的启明之星。

%75
“柳贯一。”昊震口中喃喃,好似咀嚼这个名字。

%76
其余人也是默记在心。

%77
能够被楚度邀请过来,有胆量对付百足天君的人,自然没有一个庸手。

%78
昊震等四人,皆是七转修为。

%79
当然,方源此刻。动用见面曾相识,浑身上下也流露出七转气息。

%80
“柳郎,听说你把成龙丘都给拔了,直接带走。你可真有本事!”李四春娇媚地笑几声,向方源挨过来。

%81
众仙无不起了一阵鸡皮疙瘩。

%82
方源不动声色,只用冰冷的目光看向李四春,口中淡淡地道:“你离我远点。”

%83
众仙心中顿时一凛。

%84
方源流露出一丝杀机,冰寒彻骨,众仙皆是见多识广之辈,哪里还不明白眼前这位,必定是杀人如麻,恶贯满盈的凶徒?

%85
反倒是在场的楚门蛊仙们,历练较少,察觉不到方源的凶残本性。刚刚为方源介绍的那位蛊仙又道:“按照师傅的安排,这一次,我们楚门蛊仙战力低微,就不参与作战了。全仰赖诸位前辈出手,事成之后,家师必有重谢。”

%86
“还是说说如何作战罢。”汪大仙闷声道。

%87
楚门蛊仙一语惊人:“家师制定了周详的计划,这一次还请诸位前辈联手,齐攻铁鹰福地!”

%88
几乎与此同时,大雪山,第一主峰。

%89
炼道蛊阵的黄金光辉,近乎凝如实质,牢牢地覆盖在雪峰山巅。

%90
光辉内心之中,黎山仙子和雪胡老祖站在马鸿运的面前。

%91
黎山仙子满脸疲惫之色,小心翼翼地从怀中取出一枚雷球。

%92
雷球终有拳头大小,内里电芒激闪,闷雷声不断响起,外表时而凸出。一眼看去,就知道这雷球很不稳定。

%93
“你们,你们想要干什么?!不要,不要啊!”马鸿运大叫,满脸惶急惊恐,但奈何浑身被牢牢禁锢,动弹不得。

%94
“这是最后的关键一步了。”黎山仙子说着,将雷球缓缓地推到马鸿运的身前。

%95
雷球迅速地融入到马鸿运的胸膛里,转瞬即逝。

%96
“啊啊啊!”马鸿运遭受强烈的电击,浑身颤抖,仰起头颅,扯开喉咙,发出凄厉无比的惨嚎声。

%97
ps:第六更,爆更完毕。呼,有点累,但还是兴奋的情绪更多一些。在此还是强烈呼吁起点正版订阅,你们的每一次订阅都对本书意义重大!因为vip章节有些多,均订起来很不容易的。若是能全订,一下子就能提高一个均订了,蛊真人在此感激不尽!谢谢大家支持!明天双更!

\end{this_body}

