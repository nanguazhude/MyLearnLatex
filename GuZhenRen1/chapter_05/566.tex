\newsection{雪晶阵}    %第五百六十八节:雪晶阵

\begin{this_body}

两位纯梦求真体进入至尊仙窍,深入小绿天中,便暂且安置下来。

至尊仙窍同样分有五域九天,在所有的小九天里,小绿天是方源建设最少的地方。所以,这里就成了方源安置和囤积梦境的地方。

影宗成员影无邪、白兔姑娘、黑楼兰、妙音仙子,以及白凝冰则在小黄天中,各自潜修着。

琅琊大战,层次太高,参与者几乎都有八转战力。就算是异人蛊仙,也是依靠超级仙阵参加大战,仙阵一破,他们就没有了插手的能力。

关键时刻,方源还需要这些人组成上古战阵四通八达,所以这些影宗成员索性就都没有参战。

至于方源的六转宙道分身,此刻正在小北原的最北端。

“咄!”

一声轻响,随着宙道分身放置最后一只凡蛊,这座蛊阵便随之催动起来。

蛊阵覆盖的范围,十分广阔,囊括方圆数万亩。组建蛊阵的蛊虫,有数万的凡蛊,从一转到五转皆有,分门别类,但数量最多的还是冰雪道的蛊虫。而核心则是两只冰道仙蛊,都是从雪民蛊仙身上盗取过来,如今已经炼化,为方源所用。

而蛊阵的中心,赫然是三座八转仙材冰道晶精!

每一座冰道晶精,都大如山峰。

通体水蓝之色,宛若水晶质地,周身上布满了大大小小的孔窍。孔窍数量极多,从中不断地喷涌出冰霜寒雾。

让这些冰霜寒雾自由扩散的话,其中大量的冰雪道痕都会随之弥漫,印刻在周围,改变周围的环境,营造出适合雪民、雪怪生活的地方。

所以之前,方源就计划着,利用手中的那一座冰道晶精,来打造出一道雪怪的生产线,建成一项经济支柱。

他原先的冰道晶精,是和雪儿结亲的获利。不久前他屠戮劫掠北冰原下的雪民一族,又收获了两座冰道晶精。

“有了这三座冰道晶精,生产雪怪更不在话下了。”

“至于这座大阵就命名为雪晶阵吧。”

方源宙道分身视察一番后,颇为满意地点了点头。

境界是共通的,方源本体有着阵道宗师境界,分身便同样如此。

这一点和影宗成员不同,当年魔尊幽魂为了瞒天过海,所以利用分魂形成分身都有着境界方面的弊端,每个成员几乎只擅长某一个流派。

影宗成员就算和天庭蛊仙照面,都不可能被认出分身和魔尊本体的联系。这也是为什么影宗能够在天庭安插内奸的原因之一。

而方源的宙道分身,却极容易暴露,但方源也不需要宙道分身去外派、潜伏、杀敌。方源的目的和魔尊幽魂不同,他营造宙道分身,最主要的目的是好好利用智慧光晕。

而要在智慧光晕中推算各种内容,境界不可或缺。鱼和熊掌不可兼得,方源在这方面做了明智的取舍。

雪晶阵自然是宙道分身一手布置,它最主要的作用,就是将冰道晶精时刻吐息出来的冰霜寒气,均匀持续地传输、覆盖到周围每一寸地方去,不断地改变周围的环境。

当然了,此阵也有监察、调控之能。

“可惜我只是阵道宗师,而不是大宗师。宗师境界可以触类旁通,大宗师境界则能直接运用自然道痕。若我是大宗师,建设雪晶阵,根本不需要冰道仙蛊,单靠凡蛊和这三座冰道晶精即可。”

这样一来,方源就能省下这两只核心冰道仙蛊,挪作他用。

当然,方源是特例,他并不缺乏仙蛊,也就算了。换做正常情况,蛊仙掌握的仙蛊十分有限,若是用仙蛊布置蛊阵,往往这只仙蛊就要困在阵中,不能再用于其他地方。由此也可见,流派境界高上去后的巨大妙处能帮助蛊仙节省仙蛊!

布置好了雪晶阵,方源宙道分身便转身飞离。

他虽然不是地灵,没有传送之能,但身怀仙蛊,飞行速度绝对不慢。

片刻之后,宙道分身便跨越了数百里路,隐住身形,在高空俯瞰地面。

地面上,正有一群雪民正在跋涉。

这些雪民有凡人,亦有蛊师,都是方源从北部冰原掳来的异人奴隶,规模不小,有数千人。

上千人的雪民奴隶,在宝黄天中并不罕见。但这一批的质量,却是高出水平线太多,因为这些雪民中,蛊师的比例极高。这是绝不正常的现象,放在宝黄天中,这一批雪民能至少换取十倍多的雪民奴隶!

这些雪民骤然离开了生存已久的环境,来到至尊仙窍中,自然惊疑万分,神情惶恐。

此刻大队缓缓前行,各个雪民交头接耳,议论纷纷。

方源屠杀雪民、石人蛊仙的时候,是在仙道战场当中,保密得很。劫掠了这些雪民时,也是扮做雪民蛊仙的身份,所以这些雪民根本不知道什么是真相。

仙道杀招见面曾相识!

方源宙道分身忽然变作冰卓,缓缓从高空飘下。

又催动各种蛊虫,于是天空中便忽然刮起了寒风,飘下小雪。

“啊!是我族的仙人!”

“快跪下来,跪下来,拜见上仙!”

“恭迎冰卓大仙法驾。”

雪民们纷纷下跪,五体投地,朝拜方源宙道分身。

方源宙道分身面无表情,开口道:“混世战不休,方外有乐土。三山仙雾渺,雪族自无忧。”

他声音不大,但偏偏传到地上众人耳中,却是清晰无比。

呼。

一阵大风,卷起大雪飘飞,忽然风停雪住,雪民们察觉有异,纷纷仰头,只见天空空荡荡一片,哪里还有冰卓大仙的身影。

雪民们楞了一会儿,确信冰卓大仙真正消失后,他们纷纷起身,脸上浮现着惊喜,又皱眉思考方源的话。

雪民族长沉思一会儿,对身边的各个族老道:“之前我们来到此地,迷迷糊糊之间似乎也受到冰卓大仙的指点,要我们往这个方向走。现在冰卓大仙亲自现身,好像是告诉我们,前面有一处地方,有着三座大山,可以安家立业?”

又商议一会后,众人达成一致,再次启程,这一次他们甩去迟疑,虽是拖家带口,但速度比之前快了数倍。

单要生产雪怪,其实不需要雪晶阵,只需要将那三座冰道晶精,往那里一搁即可,简单粗暴却仍旧有效。

而现在宙道分身大费周章,布置出雪晶阵,自然是为了安置和照顾这些雪民。

“有了雪晶阵,就能让雪怪和雪民间隔下来,各自生活。当然,也不会完全隔离,还要时不时地让两者接触,彼此磨砺。”

“雪晶阵的维护,有着食道手段,并不是问题。但三座冰道晶精,却是无源之水,最终都会被消耗掉。”宙道分身一边飞行,一边沉思。

冰道晶精喷吐出来的冰霜雪雾,蕴含着晶精本体的冰雪道痕,这些道痕印刻在周围,维持一段时间后,便会消散,不会真正地为至尊仙窍增添冰雪道痕。但营造出来的环境,却是极为适合雪怪和雪民的。

不过这个问题,现在考虑还太早,要相当长一段时间,三座冰道晶精才会消耗光。

宙道分身安置了雪民之后,又前往小南疆,安置石人。

小五域中,小南疆中土道道痕最为浓郁,是最适合石人生存的环境。并且之前,方源就有一部分石人,生活在这里面了。

安置石人的过程也十分顺利,宙道分身采用的手法和之前大同小异。

就在宙道分身经营仙窍的同时,方源本体已经来到了南疆另一处地方。

这里是一处巨大的地沟,黑暗深邃,即便是白天下,朗朗晴天,也是暗影重重。

这些暗影,都是暗道流派的道兽影怪,而这处地沟,正是掠影地沟。

“当初我在这里,练成上古战阵四通八达,又借助不少纯梦求真体,阻止南疆正道追兵。哦……”

方源观察了一番,神色微动:“南疆正道在这里也建设了一座仙阵么,层次还不低呢。”

“什么人?这里是南疆正道重地,闲杂散仙速速避退,否则悔之晚矣。”这个时候,一位南疆蛊仙的声音忽然响彻云霄,他隐于仙阵当中,不见真身。

“哦?”方源嘴角微微翘起,眼中杀意腾腾,“我倒要看看,你们如何让我后悔。”

轰!

一声巨响,他悍然开战,强攻仙阵。

\end{this_body}

