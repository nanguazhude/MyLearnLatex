\newsection{龙人分身}    %第七百八十七节:龙人分身

\begin{this_body}

至尊仙窍,小赤天。

一座大阵正在嗡嗡地轰鸣,以白凝冰为首的异人蛊仙们严阵以待。

呼啦!

巨大的潮水声起,随后一头太古年兽冲出大阵绽放出的银白光辉。

“这是一头太古年猪!”白兔姑娘当即兴奋低呼。

“很好,年华池中正缺这种年兽呢。”一旁,妙音仙子神情振奋。

没有什么可说的,大战开启。

众仙驾驭着万年斗飞车,杀向太古年猪。

太古年猪皮糙肉厚,然而万年斗飞车也是坚厚无比,双方斗得个旗鼓相当。

激斗荡起风云,掀起余波。换做寻常仙窍,蛊仙肯定遭殃,会损失重大。不过方源的至尊洞天太过广博了。

他侵吞了大量的蛊仙的一生财富,而至尊仙窍也只不过开发了百分之十七八。

而小赤天也是被方源刻意放空,留作战场用的,没有经营什么内容。

目前位置,小五域已经是发展得欣欣向荣,小九天中有七天也有个别资源,小赤天空无一物,其次是小绿天存放梦境,小黄天中只有一小段的碎金河。

小五域中还有大把的空余土地,可以预见,小九天的贫瘠要维持很长一段时间。

一般而言,蛊仙晋升八转,拥有洞天之后,会积极发展天空,因为大地上已无余地。但方源是特殊情况,不能用常理度之。

八转仙蛊屋和太古荒兽的大战,持续了数天,终于以后者的战败、归降而告终。

方源本体在青鬼沙漠横扫四方,而在他的仙窍中,他的麾下则运用万年斗飞车来征服太古年兽。

一方面,是方源有意让这些七转战力的蛊仙,熟悉万年斗飞车,加深两者的磨合,以期将来某刻能够掺和八转层次的战斗。

另一方面,方源也在依靠太古年兽钓来阵,不断勾引太古年兽,企图凑成一套十二生肖战阵。

十二生肖战阵的主意,方源在上一世就开始打了。

这是一套上古战阵。

需要集齐十二个不同种类的太古年兽,以它们为核心,方能组成大阵。

十二生肖战阵擅长攻伐,组并之后,可以自由移动。平时,十二头太古年兽都封印起来,宛若石像,一动不动,不需要喂养。战斗的时候,就会解封,结成战阵,上阵杀敌。

此阵优点不少,其中一项,就是能节省大笔的喂养开支。

上一世方源正是看中这一点,打算用这个方法来缓解仙窍经营上的压力。

而这一世重生过来,方源富裕得很,节省喂养开支已经是次要的,方源看重的是此阵的强大威能。

十二生肖战阵只有太古年兽才有资格,充当它的核心。若用上古年兽、荒级年兽,方源是无法组成十二生肖战阵的。

从这点就可看出,十二生肖战阵必定是力压寻常八转蛊仙的手段。

不过,要收集到十二种不同的太古年兽,看似容易,其实相当困难。

不仅看实力,也很看运气。

方源上一世,一来自己没有这个精力和时间分配到此事上,二来运气也差,到了中洲炼蛊大战,也没有筹集全一套战阵。

光阴长河中的年兽是挺多的,但这些年兽的种类并不是平均分配。而是每隔一段时间,就有一种太古年兽数量激增。

方源也没有确定的把握,肯定这一世他能够筹集成功。

不过他在这方面投入的人力物力,时间精力,还有运道方面的增幅,远比上一世要多得多。

这是事实。

时间晃晃而过。

南疆正道被方源勒索着,暗地里没少想着对付方源的办法,但投鼠忌器。

西漠正道相互对峙,房家和万家辩驳的趋势似乎要无休无止,其余的正道势力按捺不发,背地里暗流汹涌。

北原方面,长生天一直在努力收拢整个北原蛊仙界,但进展不太大。上一世长生天千金买马骨,招揽接纳了琅琊派,这一手让北原蛊仙意识到长生天的诚意,抵触心理大大降低。但这一世,琅琊福地早已经被方源吃干抹净。

东海方面,一如既往的平静。偶尔有些小打小闹,也不成气候。最轰动的事情,还是登天野之争,不过此事也早已经尘埃落定了。

而中洲,天庭因为琅琊一战败北,正在舔舐伤口。紫薇仙子虽然极想找方源的麻烦,但奈何方源一心潜修,就算是有所行动,也做了伪装,让人难分真假。

五域蛊仙界都进入了一个短暂的平静时期。

方源没有浪费一丝一毫的时间,他充分利用这段难得的安稳日子,不断消化战果,积极壮大自身。

光阴飞刃杀招又向前完善了一步。

五禁玄光气经过这段时间的辛勤苦练,已然达到五色光气齐出的程度,距离彻底的纯熟只剩下一小截的路途。

而魂道修行的成果也非常喜人。

方源的魂魄已是超越了千万人魂级,达到了一亿人魂!

这种规模已经是人的极限。

幽魂魔尊出现之前,漫漫历史长河,就算再是惊才艳艳的人,魂魄的巅峰就是一亿人魂,再往上就没有了。

人是万物之灵,人族灵性最强,但比起力气,比起爪牙锋锐,比起魂魄雄厚,比不上许许多多的其他生命。

然而修行本身就是一种打破自身极限的过程。

蛊师运用蛊虫,就能够拥有压服猛兽的力气,比猛兽更锋利的爪牙,自然也能拥有底蕴更加雄厚的魂魄。

幽魂魔尊首先开创了这个突破极限的方法,方源继承了影宗真传,自然已是掌握。

超越人魂层次,再往上修行,就是荒魂!

亿人魂虽然凝实到了极点,能够干涉物质,但说到底还是比较脆弱的。

而达到荒魂程度,却有了翻天覆地的战力突破,单靠魂魄,方源就能匹敌绝多数的荒兽了。

在这个方面的巅峰极致,仍旧是魔尊幽魂。他曾在义天山大战中,单靠自身力抗天人两灾,赫赫魔威,神通盖世,给方源留下了极其深刻的印象,让他都不禁心生向往。

“不过在我突破荒魂之前,还有事情要办!”

至尊仙窍里,方源将分出来的魂魄,投入到长毛炼道大阵之中。

大阵运转起来,宛若烈火熊熊,火光映照得方圆百里都发红发亮。

之前已经失败过了几次,有了经验,方源动作熟稔。

“注意了,下面添加主要仙材。”方源一声令下,作为辅助的琅琊地灵立即点头,协助方源操纵大阵。

大阵中内藏空间,形成一个个专门存储仙材的隔间密室。

此刻一间密室打开,里面的仙材被瞬间投放进去。

这份仙材名为兴然晓春光,高达八转,虽然是一份光道仙材,但却凝聚如花。花朵盛开,纯白的花瓣上点缀着点点嫣红,更夹杂一块块的鹦鹉绿,栩栩如生,鲜嫩水灵,透射出勃勃生机。

兴然晓春光一进入火焰当中,便融化成一股纯白光流,浇在方源的分魂上。

光流将魂魄密密实实地包裹起来。

火焰不停地烧灼、烘烤。

一件件的辅材投放进去,同时大阵之外的个别仙蛊也在催动不休。

一位位毛民蛊仙也在全力辅助,他们或是处理个别仙材,或是掐动指诀影响火候,或是排查料渣。

这是群体炼蛊,比个人炼蛊要消耗更多,但炼蛊的难度也随之下跌一截。

上一世方源孤家寡人,自然没有这个待遇。这一世吞并了琅琊派,他掌握的炼道实力已经甚至要凌驾于天庭!

天庭才苏醒了多少位炼道蛊仙成员?就算是沉眠在仙墓中的炼道大能,拥有炼道准无上境界的,也是少之又少。

当然,天庭库藏极其丰富,这一点远超方源。哪怕方源吞并了这么些蛊仙的一生积累,也难望天庭项背。

炼了一阵,一切良好,方源精神暗振,又呼喝一声:“接下来是作假珍!”

作假珍不是珍珠,而是一种可以变化成任何珍宝的变化道仙材。

它同样是八转层次,和之前的兴然晓春光一样,是此次炼蛊的主材!

作假珍很快也融入到火焰中去。

方源又唤一声:“白凝冰!”

白凝冰点头,她早已站在一个阵眼中,此刻她把胳膊轻轻一划,伤口中一道血液飞射而出。

血液一进入熊熊烈火,呼的一下,就将赤红的烈焰转变成了妖冶的猩红之色。

并且,一股浓郁的血气向四周弥散。

毛六立即出手,将这些血气迅速收拢,不断排除在外。

若不这样做,这些血气将对接下来的炼蛊,造成严重的干扰。

血焰越烧越少,很快就缩成拳头大小,缩减的速度极快。

方源深呼吸一口气,其余的蛊仙也都面色严肃。

接下来又是关键一步。

“放出惊涛升龙火!”方源下令道。

惊涛升龙火富有灵性,一飞出来,当即就想逃脱大阵,但哪有这样的可能?

长毛炼道大阵将它死死挡住,并强行逼迫它和血焰对撞。

两团火焰起先泾渭分明,但不久后相互融汇到了一起,形成一大团的混火,色彩斑斓。

混火也同样越烧越少,并且浓烟滚滚,熏得人涕泪并流。

浓烟垂下,将焰火包裹,形成一个漆黑的烟球,从外看去,依稀还能看到烟球最中心的明亮火焰。

轰!

烟球持续了小半个时辰,陡然炸开,万千股浓烟仿佛群龙乱舞,一时间气流奔腾,仿佛龙啸之音,贯穿人耳。

一位龙人蛊师从浓烟中站起,迈开步伐,昂然现身。

他体格宽健,面容英俊,高挺的鼻梁、坚毅的唇角都将他塑造得英武绝伦。

他身披金色龙鳞,龙瞳琥珀,额头有一对金色珊瑚龙角。

见到他,琅琊地灵哈哈大笑,手舞足蹈:“成功了,我们成功了!”

没错,这位青年一转蛊师正是方源辛苦打造而出的龙人分身。

------------

\end{this_body}

