\newsection{脚踹八转大能}    %第二百五十五节:脚踹八转大能

\begin{this_body}

方源双手扑空,连忙顺势举臂,左右手齐抓。

左手抓了个空,右手却抓住了碧晨天的一条腿。

碧晨天撤退受阻,另一条腿宛若恶蛟出洞,又狠又急,向方源的太阳穴袭来。

这一击势大力沉,若是击中,必定让方源头昏眼花,碧晨天逃脱再无关隘。

但方源又在此刻,掏出爱意仙蛊,举在眼前。

碧晨天郁闷得想要骂娘,腿势顿时一顿。

方源争取到关键的反应时间,忽然把头往后一仰,整个身躯就在水中颠倒过来。

本来他和碧晨天一样,都是头上脚下。

现在他一手抓住碧晨天的腿,忽然变成了头下脚上的姿势。

碧晨天见他这番变化,顿时心中一沉,暗叫不妙。

这是因为,方源一手抓住他的腿,另一只手要拿捏爱意仙蛊,根本无法做出强力攻击。

但此时此刻,此情此景,方源却有两条腿可用。

反观碧晨天,一条腿被方源抓住,只有另外一条腿可以攻击。

碧晨天大惊之下,空出来的那条腿动向陡变,宛若穿花蝴蝶,灵动非凡,刹那间连续猛踹,在方源眼中直接形成了一片交织的腿影。

但方源怡然不惧,他双腿挥舞,同样是腿影翻飞。

两人在河水里交手,劲力互撞,打的周围河水掀起团团旋流。每一次腿脚交击,都能在河水中迸发出砰砰的闷响声。

碧晨天到底只有一条腿可用,哪里及得上方源双腿齐攻?

交手过程中,时常遭受到方源的踢踹。

碧晨天以守为主,心中憋闷无比:“我堂堂八转蛊仙,居然落到如此境地。臭小子!只要我出了河面,喘息过来,就要你好看!”

他双臂急速挥动,带动方源的身躯,向河面上窜游过去。

但下一刻,方源眼中精芒暴闪了一下,一腿架住碧晨天的踢击,另一只脚则照准某个关键部位,狠狠地踹了上去。

“哦!”

碧晨天被方源狠狠一踢,踢中了自己的裆部。

一瞬间,他整个脸都僵硬住了,双眼瞪得溜圆,口中所剩无几的气息,差点要全部喷吐出来。

碧晨天不愧是八转大能,忍耐之能远超寻常,他硬生生地认主,同时双臂飞舞,想要游上去,把头窜出河面。

砰!

方源又是一脚踹中。

碧晨天顿时浑身剧颤,这一次再也忍不住,口中连吐大串的气泡。

砰砰砰。

趁着对方露出巨大破绽,方源连续猛踹。

碧晨天满脸通红,双眼充斥血丝,再不想要游上河面,而是反身,想要和方源拼命!

“谁和你拼命?”方源松开手,又一脚,把碧晨天用力地踹出去。

与此同时,他也受到反向的力道,和碧晨天拉开了距离。

碧晨天脱困,微微一愣后,连忙舍弃了方源,朝着河面挣扎上去。

方源则掉头,趁机远离碧晨天,成功撤走。

碧晨天的脑袋破出水面,连忙张开大口,不断喘息。

胯下的剧痛不断地传来,一**的强烈痛感,持续袭击着他的神经,让他的脑袋都有些眩晕感觉。

“这个该死的家伙!一旦有机会,我要把你抽筋扒皮,挫骨扬灰!”碧晨天远远望着河面上,方源飞速游动,和自己的距离不断拉大,而他自己已经鞭长莫及。

方源成功地摆脱了碧晨天后,这才缓下速度,开始慢慢恢复体力。

他感到脚底发麻。

碧晨天的裆部,并不好踹。方源如此仙体巨力,虽然是他踹别人,自己的脚底板也感到麻木。

很显然,碧晨天对于裆部的防护,也照顾到了,木道道痕绝对不少。

最后关头,方源没有和碧晨天拼命。

他知道,依照碧晨天的手段,就算自己杀死了他,必定也会在他的疯狂反扑中,遭受重创,甚至是同归于尽。

这点,方源自然是不愿意的。

他没必要和一个八转蛊仙死磕。

要知道,他有至尊仙体,前途广大。而且他此行的目的,是影无邪。

若是和碧晨天同归于尽,岂不是要被影无邪笑掉大牙么?就算是碧晨天死,方源活,方源也必定遭受重创,直接失去了铲除影无邪的能力。

当然,还有一点比较关键。

那就是杀掉碧晨天,方源也不能获利!

在这里仙窍打不开,碧晨天的尸体完全是个累赘。方源总不能带着尸体四处游动吧,就算勉强带着,遇到影无邪怎么办?就算不遭遇影无邪,遇到哪些中洲蛊仙,又该如何是好?

七转蛊仙斩杀八转,的确是震古烁今的壮举和无上的荣耀。

但这种荣耀,在方源看来,屁都不如!

他继续向前方游动,寻找影无邪。

“影无邪,是你吗?”。

影无邪听到这个声音,立即回望过去,顿时脸上一喜。

“靠近,我拉你上来。”他连忙呼唤回应一声。

片刻后,太白云生被影无邪拉到了一张巨型的荷叶上。

“这是王莲的莲叶,能始终漂浮在水面之上。想不到你的运气这么好。”太白云生感慨道。

影无邪笑了笑,心想:“我有那方源有连运关系,怎可能运道不好?”

不过很快,他脸上的笑容没了,重现浮现出担忧之色。

他望着河流的前方:“现在的逆流河状况古怪,我们不能脱离。当务之急,是和紫山真君大人汇合,说不定还能够夺取马鸿运和赵怜云,以及反杀了追杀我们的那头上古剑蛟!”

提及紫山真君,太白云生的脸上顿时浮现出崇敬之色,点头道:“对,先和师傅汇合。”

“阿嚏!”

紫山真君狠狠地打了个喷嚏。

他愁眉苦脸,唉声抬气:“唉,想不到我老人家刚刚苏醒,就落魄到这种地步。唉,人老了,身子骨就不行了,稍微受点凉风凉水,就要有发风寒的迹象。”

而在他屁股下面,是雪胡老祖的一侧肩膀。

雪胡老祖正划动四肢,四处张望,不断地在逆流河中寻找着什么。

此时听到紫山真君的话,冷哼一声道:“你这家伙,明明坐享其成,还有怨言?你不是长着一对翅膀么?怎么不直接飞出逆流河?”

紫山真君此时已经变回原形。

就像方源从上古剑蛟,变回人身一样,紫山真君乃是小人八转蛊仙,之前变大,也只是仙道手段而已。此刻落入逆流河中,不得不变回小人。

正因如此,他才能坐在雪胡老祖的肩头。

“你以为我不想飞走?但这条逆流河,此刻受到了某种巨大力量的牵引,十分混乱,我根本无法脱离河水。嘿嘿,给你布置蛊阵的人是谁?”紫山真君问道。

雪胡老祖面沉如水:“孙名录。”

紫山真君哦了一声。

“应该是他了。这的确超出了我的预料。”雪胡老祖道,“他竟然是长生天的人!”

雪胡老祖也是精明果决,此时此刻,不难猜测到孙名录的阵营。

因为中洲蛊仙已经一同被陷害,而偌大北原,谁能够有这样的大手笔,谁能有胆量对付雪胡老祖?

除了长生天,就没有第二个了。

紫山真君嗯了一声:“长生天出手,必定是图谋马鸿运,赵怜云也必不会少。他们一定在前面。”

“我要先找我家娘子!”雪胡老祖道。

紫山真君拍拍他的肩膀:“那就更应该往前方游,万寿娘子当时就在马鸿运的身边,所以被逆流河卷走后,很有可能就在马鸿运的附近。”

逆命祭炼子阵,正绽射着绚烂的白金光辉。

光辉直冲云霄,宛若巨柱,根本掩盖不住。

此刻,玄极子在蛊阵中央,全力操纵蛊阵,接引逆流河。而洪极子则在蛊阵之外,防备任何被光柱吸引过来的不速之客。

\end{this_body}

