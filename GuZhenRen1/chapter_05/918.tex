\newsection{天庭三英抗方源}    %第九百二十一节:天庭三英抗方源

\begin{this_body}

“看来要摧毁天庭仙墓,必须得先破了这个传送大阵!”方源目光如电,紧紧盯住远处大阵。

他脚下的万年斗飞车心随意动,立即如巨型飞剑般飚射而出,冲向大阵。

大阵中央正是紫薇仙子坐镇。

见到方源袭来,她不禁低抽一口冷气,感受到强大的压力!

“方源,有我在岂容你恣意纵横”龙公呼啸一声,挺身而出,拦截住万年斗飞车。

看到这一幕,紫薇仙子顿时松了一口气:“幸好我方有龙公大人坐镇!”

轰!

方源和龙公狠狠对拼一记,龙公再次倒飞出去,而方源和脚下的万年斗飞车亦飞退上千步距离。

对拼的余波搅动万千风云,掀起滔天气浪,余波浩荡不绝,仿佛飓风。被余波席卷的八转蛊仙们,纷纷被掀飞出去。哪怕是劫运坛也是趔趄难行,屋内的冰塞川等人各个立足不稳,身躯剧烈摇晃。

下意识的,所有人都远离方源和龙公。

“方源被挡住了。”

“果然,只有怪物才能挡住怪物!”

“就让他们两个厮杀吧。”

唯有秦鼎菱咬着牙,硬着头皮,疾飞过来,仍旧打算辅助龙公。

龙公和方源再度交手,打得暴风卷席,雷霆炸响。

因为有龙御上宾杀招,龙公正不断地变强!而方源身上道痕众多,更有八转仙蛊屋万年斗飞车,因而微微占据上风。

龙公阵脚不断后退,两人的战团逐渐向传送大阵靠拢。

仙道杀招——自转游龙气墙!

迫不得已之下,龙公低呼一声,双臂一展,一道磅礴的无形气墙横亘在方源面前。

方源驾驭万年斗飞车往前飞冲,强行突入了数十步后,终究被气墙挡下,几乎寸步难行。

龙公早已修行了元始气道真传,自然知晓这记杀招。并且他的这个杀招似乎还经过改良,方源隐隐辨认出气墙中仍有无数气流攒动。这些气流一个个都呈旋转的龙形,龙首不断地围绕着龙尾飞速旋转,形成一个个的漩涡。

这些龙形气流十分玄妙,能够消解外来的冲击力,极大地增强气墙的坚韧程度。

龙公的自转游龙气墙虽然比不上元始气墙那般坚实,但却别有韧性,成为方源的阻碍。

方源虽然能够利用气海无量杀招,破解了元始气墙,但那是他建立在对气墙的了解基础上,再针对性地改良了杀招。

眼下,方源对龙公的这个气墙杀招却是了解有限,想要再复制之前的成功暂时是不可能的了。

方源眼中精芒一闪,既然他无法破解此招,那就直接强推好了。

他再度变化成匪猴,伸出双手,用力向前一推。

仙道杀招——力道大手印!

轰隆!

方圆数百丈的空气都被排挤开来,两个山般巨大的半透明手印,以碾压的姿态轰上龙公气墙。

气墙当中,龙公昂首挺胸,双臂平展,竭力维持着气墙杀招。

两个力道大手印突入了数百步后,速度越来越慢,几乎停滞不前。

方源和龙公二人都一脸严肃,各自发力,陷入僵持。

“好机会!”见到此情此景,秦鼎菱立即意识到自己出手的机会终于来了。

刚刚的对决,她根本找不到插手的时机,只能不断酝酿杀招。

此刻机会一到,她立即甩出一柄金色小刀。

金色小刀起初只有手指头大小,飞到空中,见风而涨。等降临方源头顶,已经变成了马车大小。

刷刷刷。

金色巨刃连斩,企图削除方源的运势。

方源头顶忽然现出煮运锅,后者飞升而上,尽数架住金色巨刃的斩击。

锵锵锵……

金色巨刃砍劈在煮运锅上,迸溅出一朵朵的金色火花。

几个呼吸的功夫,金色巨刃就劈砍了上百次,逐渐变淡,最终消散。

而煮运锅完全招架了这记杀招,表面浮现出丝丝裂纹,整体仍旧安然无恙。

金色巨刃明显是八转运道杀招,而煮运锅幸好被提升了转数,成为七转仙蛊屋后,堪堪能挡住八转攻势。

“煮运锅?”秦鼎菱惊异出声,立即辨认出来。

她毕竟曾是巨阳仙尊的女人,自然知晓煮运锅的存在。

“你的煮运锅不过七转,就算能挡我十次攻势,能挡我二十次、三十次吗?”秦鼎菱心中冷哼一声,再度对方源展开狂攻。

煮运锅到底只是七转,虽能对抗八转,但秦鼎菱不仅不是普通的八转蛊仙,而且还精通运道。煮运锅暂时抵挡是可以的,但绝不能持久。

方源也意识到了这一点,他立即抽回力道大手印,缓缓停息了这个仙道杀招。

龙公仍旧维持着气墙,一时间难以出手,让方源得以从容抽身。

“秦鼎菱你做得很好。”龙公暗中称赞一句。

刚刚的僵持,龙公落入下风,气墙中的龙形漩涡气流已经崩解了一小半。

幸而有秦鼎菱的干扰,使得方源主动收手。

方源脑海中念头如电闪雷鸣,他迅速思量。

他的手段非常丰富,但只要有传送大阵,大半的手段他都不能恣意运用。

一旦他在催动这些手段的时候,紫薇仙子忽然对他强行传送,就可能对方源造成严重的干扰,使得他施展杀招失败。

杀招失败,蛊仙必遭受不同程度的反噬。

方源若要遭受反噬,后果不堪设想!

因为他的道痕增幅杀招威力,至少是一千五百倍!反噬的伤害同样随之上涨。

更可怕的是,至尊仙体道痕不互斥,方源要完全承受反噬。如此一来,只要方源遭受一次杀招反噬,就要面临死亡。

只有类似力道大手印这等杀招,酝酿时间较短,催动后只需要很少的念头就能维持。就算杀招被迫,也只是力道大手印崩解,对于蛊仙本体反噬的程度很小。

“既然如此,那我就用另一招好了。”方源开始酝酿杀招。

他全流派兼修,手段实在太过丰富,这点哪怕是龙公、雷鬼真君这些双修的蛊仙,也万万比不上。

几个呼吸,方源的杀招就催使出来。

仙道杀招——万我!

嘭嘭嘭嘭嘭嘭……

一个个方源以万年斗飞车为中心,迅速出现,悬浮在高空,转眼间就形成了上万人的大军。

原本的万我杀招早已经被方源改良成仙蛊方,并且成功地炼成了仙蛊万我。

但方源此刻催动的万我杀招,却不是原版,而是以万我仙蛊为核心,改良而出的人道杀招!

下一刻,海量的方源冲向气墙。

他们的实力依照方源本体有一定的衰减。但方源本体的道痕实在太多了,导致这些万我分身一个个都有荒兽级的战力。

方源施展了这个杀招,等若一下子拥有了上万的荒兽群。并且这些分身都受到方源的操控,如臂指使,随心所欲。

万我分身大军有的冲向气墙,有的杀向秦鼎菱,但更多的则绕过气墙,开始攻击墙后的传送大阵。

龙公气墙虽是强大,但规模远远比不上元始气墙。龙公心头一跳,立即试图撤销此招,但方源本体驾驭万年斗飞车再次冲来。

龙公闷哼一声,只得打消之前的想法,继续努力维持气墙。

高手之间对决,催动杀招不容易,撤销杀招也同样不容易。

至于另一边,秦鼎菱也被方源的分身团团围住,不断厮杀。秦鼎菱战力强大,但这些分身虽然对她构不成强烈威胁,却也足够干扰到她,让她一时间疲于应对。

紫薇仙子深呼吸一口冷气。

她在瞬间领悟过来,自己面临着严峻的考验。若是她通不过考验,传送大阵被破,那么就会引发一连串的失败,从而导致整个战局崩溃,令宿命蛊修复失败!

“来不及调遣天庭主力了。只能依靠我自己!”紫薇仙子面容若铁,拼尽全力,向大阵灌输全部心神。

一心一意,全神贯注!

星宿棋盘嗡嗡狂转,棋盘表面上的棋子绽射璀璨星光,当中又以三颗为最。

这三颗正是星宿仙尊虚影投过来的三颗棋子!

紫薇仙子的脑海中,无数念头如亿万繁星,不断闪耀。

大阵转动,发出一道道琐碎的星芒。无数的星芒准确地射中方源分身,鲜有失手的情况,哪怕方源的分身都在极力自由灵活的躲闪。

只要被星芒射中,方源的分身就被传送到天庭之外,消失在天庭战场。

这些分身可没有天相杀招,也就意味着他们再无法参战。

“很好!”龙公、秦鼎菱顿时松了一口气。

紫薇仙子没有让他们失望,关键时刻展现出了智道大能的担当!

方源微微皱眉:“这样下去可不行。宿命蛊正在不断地修复,大势一直在天庭手中。紫薇仙子必不能持久,但她明显打的是拖延时间的主意。我这边却还有龙公、秦鼎菱纠缠……”

方源再看长生天一方。

这些人也被传送大阵困扰,劫运坛正冲向监天塔,屡屡被星光巨柱挪移回去。

“让这些人来支援自己,恐怕也会遭受相同待遇。况且他们到来,我也不能尽展身手,还是继续让他们四处进攻,以此来压迫、消耗紫薇仙子吧。”

龙公的气墙可以不断转移,挡不住万我分身,但一心挡住万年斗飞车不成问题。

方源一面和龙公交手,一面开始联络东海诸仙。

天庭的传送大阵有一项弊端,便是需要星辰中转。即便是在天庭中也是如此,星光巨柱都是从白天中散布的星辰射下的。

方源暂时破除不了传送大阵,只要破掉白天中的星辰也是可以的。

幸运的是,在那里正有一批东海八转!

------------

\end{this_body}

