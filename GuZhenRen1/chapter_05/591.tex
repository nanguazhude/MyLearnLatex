\newsection{紫薇的不安}    %第五百九十三节:紫薇的不安

\begin{this_body}

天庭。

宇殿千重,仍旧是那样的光明堂皇。

大殿中,紫薇仙子缓缓睁开双眼,她的美眸中紫光一闪即逝。

随后,她微微皱眉,语气空灵,呢喃自语道:“定空仙蛊自毁了。”

紫薇仙子执掌星宿棋盘后,乃是当今蛊仙界前三的智道大能,此次派遣刘浩、君神光对付方源,怎可能不防备方源的大盗鬼手呢?

吃一堑长一智,定仙游仙蛊就在紫薇仙子的眼皮子底下,让方源从凤九歌的身上偷走。这样的错误,紫薇仙子不会犯第二遍。

“袁琼都。”旋即,紫薇仙子联系上这位新晋天庭的八转蛊仙。

“紫薇大人。”袁琼都立即回应。

“去炼定空蛊吧。”紫薇仙子淡淡吩咐。

袁琼都顿时面色微变,此次在定空蛊等等上布置炼道道痕,就是由他出手,因此他知晓大概。

紫薇仙子这么一说,毫无意义,刘浩凶多吉少了。

刘浩和袁琼都乃是同一古派出身,袁琼都犹豫了一下,终究还是问道:“请问紫薇大人,刘浩他……”

“极可能是被方源俘虏着,但我更宁愿他被方源杀死。”紫薇仙子语气淡漠,和方源勒索南疆正道的冷酷,竟惊人的相似!

袁琼都神色惊愕了一下,旋即反应过来。

紫薇仙子此话,并非是和刘浩有私仇,盼着刘浩去死,而是刘浩被俘虏,意味着其他南疆群仙也被方源俘虏。

若是方源将刘浩杀死,其他南疆群仙恐怕也不例外。如此一来,方源和南疆正道之间就有着血海深仇,绝不会有任何妥协、合作的可能了。而南疆之前经历义天山大战、梦境大战,损失相当惨重,此次南疆群仙若又被方源斩杀一群后,真正可谓的元气打伤。整个蛊仙界的实力和底蕴,将落到五域垫底的位置。

等到五域乱战时期,界壁消失,中洲就可以首先对南疆下手。

南疆一被中洲吞并,中洲坐拥两域,本来就实力最高,立即就凌驾于剩下的三域,优势极大。格局也跟着改变,再非之前困守四战之地的窘境!

与此大局相比,牺牲一个区区七转蛊仙刘浩,又算得了什么?

明白这一点后,袁琼都脸上的惊愕消散,告退道:“我这便去抢炼定空蛊等等。”

袁琼都那边早就准备充足,他本身就是八转,又有着天庭炼道大阵相助,抢炼出定空仙蛊等等,应当是十拿九稳之事。

不过紫薇仙子的心中,始终萦绕着不安的情绪。

这种不安,对于智道蛊仙而言,就是预兆。

智道三元要素,分别是念、意、情。

情绪的升腾、改变、湮灭,对于其他流派的蛊仙而言,或许只是本身的喜怒哀乐。但对于智道蛊仙,却是意义不同。

“难道是我漏算了什么?究竟是什么,让我如此不安?”紫薇仙子沉下心来,手持星宿棋盘,深度推衍。

良久之后,她缓缓地睁开双眼,露出一丝疑惑之情。

推算的结果,和她之前的一模一样。

“方源此次能够设伏成功,真正的关键在于刚刚开始的第一步,将南疆群仙成功地诱入宙道大阵中去。”

“这一点并非是他的功劳,而是影宗很久之前,在光阴支流附近,结合自然道痕,布置出来的宙道大阵。”

方源有阵道宗师境界,最多只能依据仙材,布置仙阵。

只有到达阵道大宗师的程度,才能依据外界的天然道痕,布置出仙阵。

影宗当然不会缺乏这种人物,因此早早地在光阴支流附近,布置出宙道大阵,将这处光阴支流遮掩得严严实实。

这种依据自然道痕,布置出来的仙阵,虽然节省仙蛊,但也有弊端。

主要的弊端就在于,根据自然道痕的分部、排列等等具体情况不同,铺设出来的宙道大阵也随之不同。

蛊仙布置这种仙阵,不仅是要有自己的想法,更要因地制宜。换一个地方的宙道道痕,也就布置不出这种隐匿起来十分厉害的宙道大阵了。

因此,方源此次成功埋伏并俘虏南疆群仙的恐怖战果,并不能复制第二遍。

这一点,紫薇仙子可是耗费了巨大代价,结合君神光带回的关键线索,千辛万苦推算得来。

“缺乏仙阵禁锢,方源就算是动用纯梦求真体自爆,梦境扩张速度并不快,蛊仙就可从容逃跑。”

“若是用仙道战场,除非是梦道战场,否则的话,其他流派的战场会首先被梦境腐蚀,从而崩解破坏。”

“所以方源利用梦境的战术,并不具备多大的威胁。”

“那么究竟我为何不安?难道是魔尊幽魂?”

想到这里,紫薇仙子索性离开大殿,前往天庭某处。

魔尊幽魂就在这里,荡魂山自爆,令他借助这场良机,结合魂道道痕自保,负隅顽抗。

但天庭方面手段也从不缺乏,此刻紫薇仙子来到这里,一座恢弘的仙阵早已经搭建完毕,并且嗡嗡运转。

这座巨大的仙阵,就好像是一个巨大的石磨,不断地将里面的魂道道痕碾磨殆尽。

“幽魂你还想负隅顽抗?可惜,再过三天,你身边的魂道道痕就要被大阵彻底磨尽。你之前举止,不过是苟延残喘罢了。”紫薇仙子入得阵里,见到魔尊幽魂,冷笑连连。

魔尊幽魂默然如石,毫无回应。

紫薇仙子眼中忽然紫光爆闪,同时手中星宿棋盘散发出氤氲蓝光,在整个仙阵,以及魔尊幽魂身上不断扫视。三番五次之后,她再次确信魔尊幽魂身上,毫无蹊跷,也没有什么反抗挣扎的暗手。

“魔尊幽魂并无问题,看来我的不安并不应在此处。那到底是哪里?等等,难道是……”

紫薇仙子出了仙阵,忽然神色一动,想到了一个人。

于是她立即行动,没有丝毫犹豫,亲自出了天庭,来到中洲某个无名的山谷之中。

山谷郁郁葱葱,内有瀑布,龙公在瀑布下结庐而居,一心教导和守护凤金煌。

凤金煌!

如果能有什么令紫薇仙子不安的,恐怕就是这位大梦种子了。

瀑布轰鸣,如匹练落下,水汽浓郁。

一片梦境氤氲盘踞,少女凤金煌肤若白雪,端庄秀美,静静地盘坐在雪白巨石上,双眼微闭。

一位蛊仙,气息内敛至极,额前一对珊瑚龙角,身躯强壮,默默站立一旁,宛若擎天巨柱,目光关注着凤金煌。

“龙公前辈。”紫薇仙子悄悄上前,低声行礼道。

龙公点头,微微抬手压了压,示意紫薇仙子保持安静。

下一刻,凤金煌陡然睁开双眼,口中轻啸,宛若雏凤轻鸣。然后她催动杀招,对准眼前梦境,伸出食指轻轻一指。

哧。

一声轻响,白光绽射,正中梦境。

梦境忽然凝固不动,然后旋即哗啦啦,像是镜子被打碎一样,全部破碎,四散开来。

“碎梦杀招我练成啦!”凤金煌高举双臂,欢呼起来。

龙公点头不已,脸上全是欣慰之色。

紫薇仙子眼中不由暗含震惊之色,凤金煌在梦道上面的进展之神速,大大出乎她的意料之外。

旧有流派的任何杀招,哪怕高达九转,哪怕是九转仙蛊屋,打到梦境上,都如石牛入海,不能奈何梦境丝毫。

没想到,凤金煌的一记凡道杀招碎梦,竟能对梦境有效。

“煌儿,你也练了大半天,且去休息吧。”龙公打发凤金煌,神情和蔼可亲,哪里看得出一丝他纵横沙场,生擒魔尊幽魂的恐怖神威。

“是,师父。”凤金煌知道紫薇仙子和龙公有要事商谈,乖巧避退。

凤金煌走远后,只剩下龙公、紫山真君二人,龙公便问:“你寻到此处,有何要事?”

“别无他事,只是心中始终不安,却寻不到到根由。”紫薇仙子实话实说。

龙公顿时皱起眉头:“依凭你的智道造诣,心中不安可是大事,偏偏又寻不到根由?”

紫薇仙子点头:“我先后推算方源,又查看魔尊幽魂那处,仍旧不安,所以便想来这里看看。”

龙公微微一笑:“那你也看到了,此处如何?”

紫薇仙子也笑道:“是我多虑了,有龙公前辈在此教导,凤金煌进展迅速,远超料想。”

龙公摇头:“我在梦道上毫无建树,煌儿能够有此成果,全是她天资聪颖,独立修行。我不过是全程看护她,防止她在修行中出现意外罢了。”

紫薇仙子眼眸发亮,由衷地赞道:“此子实乃我天庭中兴,一统五域的希望!”

龙公点头,眸光一动:“我知你不安来源何处了。”

“哦?还请龙公大人解惑。”

“我在智道上抵不上你万分之一的造诣,但左右不过那些人而已。既然不是方源,也不是魔尊幽魂,更不是凤金煌,那就只剩下红莲了。”

红莲魔尊!

紫薇仙子双眼骤亮:“龙公大人是说红莲真传?”

三大魔尊进攻天庭,但唯有红莲魔尊差点毁灭了宿命仙蛊。正常人都能想到,红莲魔尊留下的真传,恐怕是心有不甘,想要后人继续努力彻底摧毁宿命仙蛊!

“寻找红莲真传一事,进展如何?”龙公又问。

紫薇仙子道:“今古亭一直在镇守光阴长河,只要方源进来,必会察觉。恒舟已差不多完工,数日后即可下水入河,一同搜寻红莲真传。我还计划组建出三秋黄鹤台、鲨流撬。有这四大宙道仙蛊屋,量那方源如何成长,也要在这铜墙铁壁上,碰得头破血流!”<!--80txt.com-ouoou-->

\end{this_body}

