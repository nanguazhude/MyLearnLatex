\newsection{感化方源}    %第五百二十三节:感化方源

\begin{this_body}

%1
方源暂时被困在胎土迷宫杀招之中,战局一下子稳定下来。

%2
铁面神眼中精芒一闪即逝,直接飞身下来,来到陆畏因身前行礼,主动问道:“这位前辈,接下来我们该怎么斩除方源这个魔头”

%3
陆畏因微笑,声调缓和地道:“上天有好生之德,即便这方源是天外之魔,那也是生灵,岂可轻言坏他性命呢”

%4
铁面神一愣,翼家的两位蛊仙也飞过来,其中翼南门急道:“前辈,怎可以姑息养奸,此魔罪孽滔天,扰乱世间秩序,此时是除去他的最好时机,怎可以一味仁慈呢”

%5
翼渔也劝说道:“前辈,今日不除去此魔头,将来他祸害世间,屠戮万千生灵,岂不是更糟糕除去他,是天下苍生之幸事啊。”

%6
但陆畏因摇头:“诸位言语都有道理,却也偏颇。世间绝无生来的恶人,也没有先天的杀人罪犯。若非如此,我们为什么要去学习种种杀人、作战的手段呢人心有善有恶,没有纯粹的恶,也没有纯粹的善。这位方源虽然是魔道蛊仙,手染鲜血,行事作恶多端,但并非生来就是恶人呐。”

%7
“可恨之人必有可怜之处,世间之事亦是有因有果。造成今天这般的方源,是昨日所酿造的因缘。方源既然能一步步变成今天的凶恶之辈,为什么不会在将来,转变成良善之人呢”

%8
“他实力高强,手段无数,若是用在正道,造福世间,那岂不是不救下千千万万的凡人,都更对世间有益吗”

%9
陆畏因一席话,听得铁面神、翼南门、翼渔三人都呆了

%10
这位神秘的八转蛊仙,居然打着要感化方源的主意

%11
“看来前辈是心有成算,那么晚辈就只好静候前辈佳音了。”铁面神叹息一声道。

%12
翼家两位蛊仙相互对视一眼,也纷纷从彼此的眼中看出极度无奈的神情。

%13
单凭他们三人,无论如何也不是方源的对手,所以要除去方源只能依靠这位神秘的八转蛊仙陆畏因。

%14
陆畏因手段高超绝妙,尽管出现得相当突然,又提前布置了仙道战场杀招,早有图谋,但种种迹象表明,他似乎真的能够除去方源。

%15
没想到一番交谈,陆畏因却有这样匪夷所思的打算,想要感化方源,让这位惊世的魔头,转变成造福人间的善仙。

%16
“恐怕这位八转蛊仙,有着某种手段,可以改变方源的心意”翼渔心中猜测。

%17
翼南门则直接开口道:“感谢前辈搭救,还请前辈开放了这片战场,让我等归去家族。”

%18
翼家蛊仙来到这里,本来是想对铁家、商家、侯家不利,结果事情发展得太快太突然,方源的出现,就直接让整个事情超出了这两位蛊仙的掌控。

%19
这两位翼家蛊仙也是倒霉,碰到了方源,又碰到了陆畏因,本身的图谋还未施展就彻底失败。

%20
现在这种情况下,他们当然是不愿意在这里久待,想要离开也是人之常情。

%21
铁面神望了望两位翼家蛊仙,目光闪烁了几下,却没有开口。

%22
陆畏因却缓缓摇头:“诸位仙友且稍安勿躁,我绝不会对诸位不利,但此时却非开放战场的良机。诸位若是脱离了这片战场,恐怕第一时间是想家族汇报,邀请帮手来对付方源。这却有违我的计划,所以还未诸位稍待片刻才好。”

%23
“这”三位蛊仙你望我我望你,只得答应下来。

%24
在这片仙道战场当中,他们无法沟通外界,本身又都不过方源和陆畏因,虽然心中都有不甘,但只能听从后者安排。

%25
“师父,那方源真的能够转为正道吗”叶凡在身后,疑惑地问道。

%26
陆畏因转过身,摸了摸叶凡的头:“世间之事,一切皆有可能不是吗我愿意给他一个机会,让他迷途知返。他若能浪子回头,我相信必定是人间的大幸事,更是今后史书中的一段佳话。”

%27
“真的可以吗”商心慈也问,神情犹豫,但目光深处又带着一丝期盼。

%28
她是正道蛊师,从小到大皆是如此,当然不愿意心爱的方源,走上魔道这条不归路。

%29
并且此时此刻,在商心慈看来,方源已经被困住,这位陆畏因手段强大无比,方源有着性命之忧。这个时候,若是让方源改邪归正,不仅是一件利民大事,而且对方源本人而言,也能保住性命。

%30
“当然是可以的。人是会变的,既然能从好变坏,自然也能又坏变好。商心慈啊,你觉得方源的本质会是一个坏人吗”陆畏因笑着问道。

%31
商心慈旋即摇头,态度坚决:“不瞒前辈,我始终觉得,方源绝非是天生的恶人,我感觉到是他的冷漠,而在他的冷漠之下,还有无尽无穷的悲伤。”

%32
陆畏因点头:“你是比较了解他的。难得,难得。我已知晓,他曾经帮助过你。那你知不知道,现在的他,更需要你的帮助”

%33
商心慈一愣,旋即追问:“请问前辈,我该如何帮助他”

%34
陆畏因笑道:“很简单,我刚刚已经说了,只需要你有劝说他,期盼他改邪归正的心意就可以了。我的这个杀招,号称胎土迷宫,可让人经历三生三世,让人阅尽红尘,体悟生死,领略人道沧桑,渐渐忘却心中恶念,从而改邪归正。”

%35
“我们每一个人都有先天的性情,但后天的遭遇也极其重要。商心慈啊,你的一份心意投射到胎土迷宫当中去,必定能让方源感受得到了,让他知晓人世间的美好,重新沟动出他心中的善良和光明。”

%36
“我明白了,谢谢前辈指点”商心慈神情恍然,目光坚定起来。从她的身上,升腾出一股意志,和其他人一样,也投射到胎土迷宫之中。

%37
见到这一幕,陆畏因顿时心底微微松了一口气,暗道:“有着商心慈的气运倒戈,我方在运道方面,也不会完全遭受方源的压制了。接下来,就看另一边的行动。”

%38
太古白天,陈衣身形如电,在云霄中疾驰。

%39
他身上还残留着伤势,这些伤势来源于房家太上大长老,还有传奇太古魂兽青仇,伤口中充斥力道、魂道道痕,极其浓郁。

%40
陈衣纵然治疗手段出众,但这些伤势也不是一时半刻就能康复的。

%41
和他身上的伤势比较起来,他心中的抑郁要更加沉重。

%42
“这一次,我居然失败了”

%43
“不仅豆神宫没有夺回手中,而且还惨败亏输,连青仇都没有除去。唉我是贪心过度,被他人有机可乘,是我的错,都是我的错”

%44
陈衣心中非常自责。

%45
一部分是回去难以向紫薇仙子交代,另外更多的是他身为天莲派的太上大长老,居然没有将仙祖之物收回,心中羞愧至极。

%46
“这一次回去,我会主动辞去天莲派太上大长老职务,潜行闭关修行修成之后,再去西漠夺回豆神宫”

%47
陈衣下定了决心,他需要雪耻,否则内心难安。

%48
他飞行多日,此刻中洲已经遥遥在望,不过就在这时,三位七转蛊仙渐渐从天边飞来。

%49
“陈衣大人且慢,我们已经等候您多时了。”这三位七转蛊仙联袂齐来。

%50
陈衣定睛一瞧,都是熟人,乃是中洲十大古派的其中几位太上长老。

%51
“你们这是”陈衣疑惑不解。

%52
三位七转中的一位,交给陈衣一只信道凡蛊。

%53
陈衣探进心神,一瞧,这是紫薇仙子留给他的信。

%54
信中紫薇仙子要求陈衣将功补过,前往北原上空的太古白天,进行接应。

%55
“紫薇仙子大人,不愧是智道大能呐”陈衣看着,眼中精芒爆闪,连连点头,感慨不已。

%56
“还请陈衣大人明示我等。”三位七转蛊仙询问。他们接到的,只是将这只信道凡蛊,交给陈衣的任务,对于接下来的安排和计划,他们一概不知。

%57
陈衣望着他们三人:“天庭已经对方源出手了。你们跟着我来便是”未完待续。

\end{this_body}

