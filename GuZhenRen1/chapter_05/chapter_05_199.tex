\newsection{登场南疆}    %第一百九十九节:登场南疆

\begin{this_body}



%1
方源继续说他的:“这个太上家老最好有些地位,最关键的是态度要强硬!因为在这个期间,武庸很可能会故意偏袒我,认定我就是他的弟弟,来否决这个提议。所以我们的人选,要能扛得住一位八转蛊仙的强势!”

%2
乔家太上大长老摊开双手:“这个着实太难了。要知道,现在的武庸不仅是八转蛊仙,而且还是武家的太上大长老,更继承了武独秀的仙蛊啊。”

%3
方源连头都没有转过去,仍旧望着窗外。

%4
他的话在瞬间变得冰冷和强硬:“别告诉我你乔家没有这样的人。若是没有,还对武家有什么图谋呢?趁早洗洗睡,做做梦更靠谱些。而我武遗海也绝不会找这样的家族合作!”

%5
乔家太上大长老微微一愣,他不禁再一次打量方源。

%6
站在他的这个角度,他只能看到方源的侧脸。

%7
窗外,夕阳的红光映照在方源的脸上,像是一层血。

%8
方源的脸色坚毅如铁,湛蓝的双目熠熠生辉。

%9
“这个男人……呵呵,有意思。不愧是武独秀之子啊。”乔家太上大长老心中感叹一声,口中答道,“那就依你的意思。”

%10
……

%11
武独秀虽然死了,但是在她的葬礼上,却上演了一幕有趣的情景。

%12
武独秀在东海的私生子,武遗海冒了出来,想要祭拜他的亲生母亲。

%13
各方蛊仙,都静悄悄地看着,没有开口插言。

%14
毕竟,这是武家内部的事情。

%15
当中,自然也有不少蛊仙,看出了一些潜在的东西。

%16
但武家的太上三长老武樵出声时,有数股目光都集中在他的脸上。

%17
武樵似乎毫无察觉,一副我心为公的大义凛然。他可是武家的太上三长老!

%18
乔家居然能渗透到这种程度,就连方源都有些吃惊。

%19
当然,他是参与其中的人。知道更多,判断更加清晰。不知道的人,并不了解种种因由,也无从判定武樵的立场究竟是什么。

%20
他究竟是为公为家。还是倒向乔家?

%21
武庸虽然意识到了这种可能,但也不能判断明确。

%22
武庸开口:“也好,就在母亲大人的葬礼上,为我弟认祖归宗。想必母亲在天之灵,看到这一幕。也十分欣慰吧。”

%23
他居然没有反驳,而是直接应承下来。

%24
方源不禁深深地看了一眼武庸,这个家伙果然不简单!

%25
乔家太上大长老也对武庸的反应,心中微微讶异了一下,立即将对武庸的评价暗暗提高一层。无弹窗,最喜欢这种网站了,一定要好评]

%26
很显然,武庸已经意识到,自己是无法阻止方源当众认祖归宗的了。与其一味反驳,丧失自己的威信和形象,还不如以退为进,允许此事。

%27
“诸位仙友在此。皆可见证,请仙蛊!”太上三长老武樵主动站了出来,这种事情他需要亲自主持,防止万一的情况下武庸等人动手脚。

%28
很快,武家取来仙蛊。

%29
武樵站在方源的面前,手持着一只仙蛊,开口道:“这是七转血道仙蛊血脉,乃是我武家斩杀了一位血道魔仙,收缴过来的战利品,正可验证二公子你的身份。”

%30
此话一出。当场的许多蛊仙,都翻了翻白眼。

%31
血道不容于世间,危害太大。

%32
但事实上,各个正道势力都在秘密研究。因为血道战力的确容易速成。

%33
很显然。这只血脉仙蛊一定是武家的研究成果。故意说成是什么战利品。

%34
“请二公子滴出一滴鲜血来!”武樵继续开口。

%35
方源的心不禁一紧。

%36
居然是七转血道仙蛊!

%37
这有点出乎他的意料之外。

%38
而且滴出鲜血,和验证他整个身体不同,这很难动手脚。

%39
众目睽睽之下,方源依言行动,滴出一滴鲜血,交给武樵。

%40
武樵催动仙蛊。对准自己手心中的鲜血,照出一道暗红色的光辉。

%41
没有发现问题。

%42
方源顿时松了一大口气。

%43
“幸好我的血本仙蛊修复好了,掺和进了见面曾相识之中,可以伪装血脉。上一次骗过黑凡真传中的布置,这一次又晃点了七转血道仙蛊血脉。”

%44
幸亏它只是七转层次,若是八转的话,说不定就隐瞒不过去了。

%45
这点,方源是有运气的。

%46
当然,武家通常也不会闲的蛋疼,将一只血道仙蛊提炼到八转上去。

%47
炼制八转仙蛊需要的仙材太过骇人,往往能将一个超级势力的家当底蕴抽空,也未必能炼成。

%48
但这还不算完。

%49
紧接着,武樵又问了方源许多关键性的问题。

%50
有些方源知道,有些方源并不知道。他都实事求是地答了。

%51
这些问题当中,有一些陷阱,明明是没有发生过的事情,当做发生了来问。

%52
但方源搜刮了武遗海的魂魄,对这些问题应付自若。

%53
真正的危险,就是面对七转仙蛊血脉的探查,他已经渡过去了,这样的问题并非难关。

%54
数百个问题相继问完,武樵仍旧是面无表情。

%55
但是其余武家的太上长老们,有的微微点头,有的交头接耳,有的看着方源,面露微笑,就已经说明了方源顺利通过。

%56
“其实本来还要验证魂魄和骨相,可惜武独秀大人已然去了,便作罢。”武樵的一句话,让方源庆幸不已。

%57
幸亏之前,他没有贪图什么八转仙蛊。

%58
若是当面对质,他的魂道、骨道手段可不怎么样。十有八九会露出破绽来的。

%59
“听闻武独秀死时,身化光萤,无一丝遗骸留下。就算是之前遗留的发丝等物,也随之一同化去。真是死得干净,死得好啊!”方源大感自己走运。

%60
接下来,武樵汇报武庸他自己的探查结果。

%61
不用说的,大家都亲眼作证了。

%62
武庸点头,其他的武家太上长老们也纷纷同意。

%63
于是,武樵又为方源当场炼出了魂灯蛊、命牌蛊。这些蛊虫今后都要置放在武家的宗祠之中,算是方源正式认祖归宗的标志。

%64
这一轮下来,方源才真正的以武遗海的身份,踏足到南疆蛊仙界中。

%65
之后祭拜武独秀,方源大哭一场,秀了一场精彩的演技。

%66
武独秀葬礼期间,武家并不适宜大摆筵席。但葬礼大典结束之后,方源仍旧是和许多武家的蛊仙接触,虽然只是喝喝茶,闲聊一会儿,也让方源收获颇丰。

%67
当然,也有蛊仙提出和方源较量切磋,都被方源婉拒。

%68
他现在的身份可是高得很!

%69
本身是七转修为,武家的太上大长老,八转蛊仙武庸,就是他的哥哥。

%70
虽然这兄弟两人都各怀心思,但至少从表面上看,的确如此。这份血脉关系,也是事实。

%71
七天之后,葬礼彻底结束,方源站到武家蛊仙的身旁,俨然一副主人模样,和各位来宾贵客送别。

%72
这些南疆蛊仙也都记住了方源。

%73
很多人都将他列为不可招惹的对象之一,不是因为方源的修为,而是方源是武庸之弟这层亲缘关系。

%74
武遗海的这个名号,彻底传入了南疆。虽然说不上无人不知,无人不晓,但也差不多这样了。

%75
三天后。

%76
清晨,旭日从天边冉冉升起。

%77
方源站在有熊山的山峰上,静静地看着日出的美好景象。

%78
自从他认祖归宗之后,他就被武家安排到了这座有熊山上居住。

%79
这里不是武家的大本营福地,方源也宣称自己想要住在外面。有熊山也算得上一座名山大川,山上最多的野兽,便是各种各样的熊。

%80
武家将这座山,直接划给了方源,从此之后,这里便是他武遗海的地盘了。

%81
但是,武家却一直没有安排方源相关的职位。

%82
武家蛊仙都为太上长老,武庸是太上大长老,其下有二长老、三长老等等。名次不同,权柄也轻重不一,话语权更是不一样。

%83
这一点,让乔家太上大长老有些着急,十分关注。

%84
因为乔家之所以扶持和帮助方源,就是为了将他插入武家的权利高层,在武家中为乔家争夺更多的利益。

%85
为此,乔家必须保证方源能够掌握武家更多的权柄,如此一来,才好为乔家说上话,出大力气。

%86
乔家的帮助自然不是无偿的,方源已经在暗地里,和乔家太上大长老达成了种种盟约。

%87
不过方源却丝毫不着急。

%88
他表面上是望着日出,实际上心中却是惦记着超级梦境。

%89
正式身份已经得手了,该怎样才能大摇大摆地,进入那座超级蛊阵当中,利用超级梦境,提升他方源的境界呢?

%90
方源正思考着这个问题,一只信道仙蛊飞了过来,带来武庸的传唤。

%91
如今,方源已经完美地认祖归宗,整个过程得到南疆无数蛊仙的见证,就算是武庸也不能害其性命了。

%92
木已成舟,生米已经煮成了熟饭。

%93
尤其是在武家的势力范围内,武庸更要反过来保护他这个“弟弟”。

%94
方源依照信道仙蛊上的领引,在一座大殿中,见到了武庸。

%95
武庸开口道:“你是我弟,刚刚认祖归宗。按照家族的规矩,每一位新升的六转蛊仙,都可领取一只六转仙蛊。升上七转之后,便能领取一只七转仙蛊。你拿着这只信道仙蛊,且去族中的库藏,挑选仙蛊罢。”

\end{this_body}

