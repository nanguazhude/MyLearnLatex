\newsection{异人豢养大计}    %第四百七十节:异人豢养大计

\begin{this_body}

这一口井,井的外围是方的,四四方方。井口的内围,则是圆的,圆圆整整。

外方内圆。

市井!

方源考虑良久,将市井摆放在了小中洲中。

继荡魂山、落魄谷、逆流河之后,这是方源得到的第四个天地秘境。

纵观古今,搜集到四座天地秘境,能达到这样的成就的蛊仙,已经屈指可数了。

市井记载于《人祖传》中,曾经乃是小人的生活领地。

如今这座市井中,原本的那座微型市镇,已经被方源摧毁。原本包含的方寸山,也早就和市井分离,最终折损在了北原。

不过,方源得手的市井,井底并非空无一物,而是存在着大量的仙窍福地。

这些仙窍的主人,都是被方源利用市井杀害,死不瞑目。

仙窍形成福地的过程中,要从外界吸取大量的天地二气,因此必须打开仙窍门户方可和外界交流。

方源早早趁此良机,已经去往这些仙窍福地中探索了一遍。

没有地灵,能够搜刮的仙窍福地,他都搜刮了。剩下的仙窍福地,仍旧保留了大量的修行资源,方源当时并没有动强。

“现在我已经有了血光镇灵杀招,可以镇压这些地灵,却少了上极天鹰,无法进入其中了。”

眼下这些仙窍都紧闭门户,方源一时间也无法进去。

“如果将这些仙窍都吞并,并且资源都统统收入囊中,我的道痕将会暴涨一大截,仙窍的开发程度也会上涨许多。”

这是一大块肥肉,方源目前还吃不到嘴里去,不过已经摆在了碗里,被抢走的可能性太小,方源也很放心。

“除了这些仙窍福地之外,市井本身的巨大价值,也绝不容忽视。”

有了市井,方源就可以直接豢养异人,甚至是纯正的人类。

这是很好的一件事情。

蛊仙界中,将仙窍的经营分成了七个层次。

第一层,建设凡级资源,产出凡蛊、凡级蛊材。

第二层,建设仙级资源,产出仙材,能够喂养仙蛊。

第三层,豢养仙兽、仙植,形成循环的生态。

第四层,利用产出的仙材,对外贸易,互通有无,赚取利润。

第五层,仙窍中能自然产生仙蛊。

第六层,豢养异人、人族,产出蛊仙。

第七层,产出寿蛊。

这七个层次,依次递进,乃是蛊仙界历经沧桑,古往今来的经验总结,流传至今,已然是蛊仙修行的金科玉律之一。

并非说蛊仙刚开始经营仙窍,就不能直接豢养异人或者人族,而是如果这样强行这样做,成果会相当微渺,甚至是反而耗费蛊仙自己的底蕴。

这七个层次,层层递进,乃是蛊仙世界千万年来,一代代精英,总结出来的成果。按照这样的步骤走下去,仙窍会非常的稳定和健康。

方源目前在第三层、第四层的阶段,距离第五层次还有相当遥远的距离。

但是他现在有了市井,就能够直接跨越第五层,进行第六次的经营。

市井能够帮助蛊仙,跨越层级,非常方便的豢养异人,甚至是人族。

豢养异人、人族有什么好处呢?

好处太多了。

每一个异人种族,都有它独特的资源。

比如雪民一族的泪冰、雪莲花精,石人一族的石龙,羽民一族的自由蛊等等。

这些资源若是形成产业,在当下的市场中,会有极大的效益。

除了这些独特资源外,还有一项最普遍的资源。

那就是——人气!

天气、地气、人气。

三气平衡,乃是蛊师升仙的关键。

升仙过程中,人气越多,说明蛊师底蕴越雄厚,成就的仙窍品质也会越好。

成就蛊仙之后,人气的作用就下降了许多。蛊仙更需要的是天地二气,每隔一段时间补充进仙窍,帮助稳定自家这片仙窍小天地。

但人气并不是毫无作用了。

甚至恰恰相反,人气对于蛊仙的作用相当重要!

但是运用人气的法门,早已经流失。气道这个流派,出现的时期比力道还早很多,曾经盛极一时,但到如今早就凋零。

现在,人气作为一种蛊材,也有不少的利用。尤其是一些气道蛊仙,更掌握着比较稀罕的运用法门。只是如今这些气道蛊仙,更是比较少见了。

不管是豢养异人,还是人族,都能积蓄出人气来。

异人、人族人数越多,精英越多,人气也就越多。

当然,同等情况下,一位异人的人气,是要少于一位纯正的人族。

除了人气之外,还有许多“高大上”的仙材,会随着异人、人族而产出。

比如,扇门风。

这是七转仙材,生长的地方,不在深山老林,更非大泽苍穹,而是在凡人家的门板上。每当有这样的仙材产生,凡人的门板就闭合不了,只能不断地开合。

这种奇特的仙材,在毫无人烟的大自然中绝对找不到。

“并且在影宗传承中,还有一个说法。”

“人气积蓄越多,越能帮助蛊仙领悟人道的奥妙!”

想到这里,方源眼中精芒一闪即逝。

人道!

这是人祖创造的流派。

纵观历史,一些八转大能,乃至九转尊者,在其一生中都会参悟《人祖传》,多多少少领悟出人道的奥妙来。

比如巨阳仙尊,就是参悟《人祖传》有了心得,开创了众生运、天地运。

又比如南疆柴家的老祖柴夫,就是在晚年从《人祖传》中领悟出了人道杀招,一举奠定了柴家超级势力的基业。

这种例子还有很多,就不一一例举了。

“像我这种情况,若是直接豢养人族、异人,只会破坏仙窍环境,劳心劳力,消耗自身底蕴。”

“但是有了市井,就可以在这里单独豢养,隔绝起来,一点都不会干扰我经营仙窍。而市井中豢养异人、人族,环境特殊,更能事半功倍。”

“人族就算了,灵性太高,消耗资源也多。当下,我还是选择异人中的一种,进行豢养即可。”

蛊世界中,有十大异人种族。

分别是毛民、石人、雪民、羽民、鲛人、墨人、蛋人、兽人、小人、菇人。

每一种都有着特别的优势。

“该选择哪一种呢?”方源还未想好。

市井只有一座,当然是只选择一种异人豢养最佳。因为每一个异人种族之间,生活习性还是区别很大的。同时兼顾,只会是顾此失彼。

“兽人首先排除,已经是灭绝了。”

“毛民吗?我的确需要毛民来帮助我炼蛊。”

“石人吗?我之前就豢养过一批石人,并且我有胆识蛊,对于石人的豢养帮助极大。”

“雪民也很好,我现在是雪民一族的女婿,有雪民一族的支持。”

“小人的话,本身就很适合市井这个环境,我经营仙窍,也需要小人来辅助我栽培仙植啊。”

“还有羽民、鲛人、墨人、蛋人、菇人……我选择其一,都会有不小的利处。”

方源一路上深深思考,直至他赶到目的地,都没有思考出一个结果来。

这个问题并不着急解决。

因为就算方源思考好了,他手头上也没有资金来运作这项大计划。

他现在手头上,仙元石的储备相当的低。

不过可喜的是,红枣仙元有很多,哪怕是他不久前刚刚和尤婵、秦百合一战。

这是因为至尊仙窍的光阴流速,在恢复了原状之后,还被方源利用宙道手段提升了一截。原本和外界有一比六十的流速比,现在更高!

因此,方源积攒红枣仙元的效率,也就拔升了好大一块。

“这一次东海之行,收取了市井,更斩杀了尤婵。龙鱼生意上,再无敌手。接下来日进斗金,几乎已成定局。只要我不断地出售龙鱼,资金方面的难题,会很快得到解决。”

“眼下的重点,是盗天梦境啊!”

方源一边想着,一边降下云头,落到沙漠上。

灼热的沙硕上,早已经站立着一位蛊仙。

他笑着,对方源拱手一礼,道:“唐方明见过方源大人。”(未完待续。)

\end{this_body}

