\newsection{冰道晶精}    %第四百四十九节:冰道晶精

\begin{this_body}

南疆。

细雨飘摇,寒风中,青葱山峦在昏黄的傍晚,平静祥和。

无名的山峰上,一棵巨大的青松,松盖如亭,从发芽长成,到如今已有六年光景。

“快了,快了,再过半个时辰,我就能炼出七转的松针仙蛊来。六年的苦功……”七转隐仙郑青心中期待着。

他乃是木道蛊仙,此番炼蛊,乃是运用木炼之法。

虽然耗时颇久,但却是采用了元莲仙尊的妙法,因此成功率相当的高。

郑青原本只是一个凡人,在山上砍柴,跌下悬崖,结果被山谷中一朵高耸如厦的巨莲托住。

机缘巧合之下,郑青因此继承了元莲仙尊留下的一份木道传承。

经过苦修,他成为蛊仙。又慢慢一心修行,鲜少外出,数百年光阴,修成七转境地。

他心性好静,不得不说,元莲真传真的很适合他。此番炼蛊,他化身青松,一呆就是六年,风雨不辍,默默潜修,任凭外界风云变幻。

轰隆隆……

就在这时,忽然从地底深处,传来深沉的轰鸣声,仿佛是万千巨怪在咆哮。

“怎么回事?”郑青自然听得响动,顿时警惕起来。

“没道理啊,我这炼蛊法门,乃是来源于元莲仙尊,不管是炼蛊的过程,还是炼蛊尾声,都是自然衍化,玄妙悄然,不可能有丝毫泄露。因此不大可能引人来抢夺,等等……”

郑青先是疑惑,旋即心中一动。

他想到了一个答案。

在数月之前,他也曾经经历过这么一场异动。

那一次,大地震动,仿佛一条旷古无前的巨蟒在地底深处翻了个身。虽然没有多大的影响,但是着实吓了郑青一跳。

事后,郑青沟通了宝黄天,这才得知,原来一场巨大的地震,规模空前,覆盖了整个南疆。

“难道这一次也是同样如此?”

郑青的猜测很快就得到了验证。

果然如他所想的一样,这一次和之前一样,仍旧是地震。不过规模程度,却出乎郑青预料。

轰隆隆!

恐怖的地震似乎要吞噬一切,天地剧震,仿佛末日来临。巨大的沟壑旋即产生,绵延千万里。

郑青非常不幸,他身处的无名小山正被沟壑吞噬。

小山崩溃,郑青还想坚持,但奈何到达极限,只好还原人形。

饶是他心境上佳,此刻也要气得吐血,六年的苦功眼看就要开花结果,却在此刻化为乌有。

“这地震究竟是怎么一回事!我怎么会如此倒霉……呃!”

忽然间,他充血通红的双眼猛地瞪大。

只见那壮阔无比的沟壑当中,显现出一座山峰般的巨石。这个石头通体橙黄,仿佛水晶般质地,石头表面有着无数孔窍,孔窍有的大如房屋,有的则小如拳头,大量的黄色烟雾,从这些孔窍中喷吐出来,源源不断。

这些烟雾越发浓郁,很快,就在周围转变成土壤。原本的橙黄水晶巨石,在短短十几个呼吸内,就被蒙上了一层厚实的土壤。

郑青呼吸急促起来,他有元莲真传,见识不小,识得如此宝物。

“这竟然是‘乎’地中的土道精晶!这是八转仙材啊……等一等!”郑青侦查杀招越扩越远。

他双目越瞪越大,口中惊呼:“我没有看错吧?不只是土道精晶,还有玉身鳞、地洞游炎、春雷凝露珠……”

这些仙材,价值很高,而且数量规模也非常可观。

炼蛊失败的情绪,已经被郑青抛之脑后,他连忙扑下,开始不断地收取这些天材地宝。

北原,琅琊福地。

方源修行着剑道杀招。

仙道杀招——剑浪三叠!

哗哗哗!

灿烂的银色水浪,滔滔汹涌,澎湃不绝。三层浪潮,气势浩荡,浪花万千,俱都锋锐无当。

仙道杀招——无形飞剑!

飞剑仙蛊猛地飞射而出,骤然消失,速度其快。

方源当然能够精准地感应到飞剑仙蛊的位置,但是旁人就难了。这是薄青拿手的杀招之一。

仙道杀招——云霄飞剑!

方源双掌一拍,顿时云雾喷涌,迅速扩散。剑势无边无际,无可琢磨,无形无质,无可抵御。

仙道杀招——万里飞剑!

方源伸手一指,剑光宛若一道白虹,贯穿天地,眨眼间就飞出数千里去,且精准无比。

最后的重头戏,仙道杀招——五指拳心剑!

方源双手一拍,在胸口出合十,随后右手五指蜷缩起来,像是将无尽无穷的剑光捏在手心当中。然后他的右拳缓缓举起,悬停在头顶。

首先是拇指一松,哧,一声轻响,犀利无比的剑光猛地射出去,将云盖大陆直接洞穿。

然后是食指、中指,相继伸直,又各有两道剑光飞射,威力十分强劲。

但无名指、小拇指却是伸无可伸,射出三指剑光后,方源就将这一杀招停歇下来。

一番练习完成,方源站在原地,体悟和总结。

“这些天来,我已经将剑道杀招整理完毕,总结出五大招数出来,除了剑浪三叠来源于琅琊派之外,其余四招均是薄青的手段,不过都被我改良了。”

“剑道这边的情况,就和智道相差一筹了。剑道仙蛊并没有智道多,境界上也远远不如,所以改良有限,杀招的威能和薄青原版也相差不少。尤其是最后一招五指拳心剑,原本享誉古今,乃是威震天下的杀招,但此刻我只能催使三指剑光,已达极限。同时,攻伐威能也只是七转巅峰一级。”

“若是我能够将八转慧剑仙蛊调用起来,剑道杀招的威能就能晋升到八转层级了。”

可惜的是,慧剑仙蛊催动困难,非得是八转仙元。

方源自身没有八转,产出不了八转仙元,只能作罢。

方源手中有三大八转仙蛊,分别是态度蛊、似水流年、慧剑仙蛊。其中态度蛊乃是传奇仙蛊,不愧是《人祖传》中记载的,催用的要求非常低廉。似水流年差了一些,根据蛊仙仙元的质量和数量,效果高低不一。慧剑仙蛊乃是正常的八转仙蛊,虽然是方源之物,但非得有八转仙元才可以催动。

所以,方源在剑道方面,还有着巨大的成长空间,只是目前为止,经过这一轮的开发,已经暂时达到了极限。

杀招练习完毕,方源不愿浪费点滴空闲,回到云城后,又开始仙窍方面的建设。

至尊仙窍,小北原北端。

大雪飘飞,寒气四溢,积雪深厚,气度方面已经和之前大不一样。

吼吼!

偶尔有几头雪怪,蹒跚在飞扬的大雪中,发出几声怪吼。

而在最北端,有一块山峰般的水晶石头。

这块石头,通体水蓝之色,宛若水晶质地,周身上布满了大大小小的孔窍。孔窍数量极多,从中不断地喷涌出冰霜寒雾。正是这些雾气,从根本上大大地提升了周围的环境。

冰道晶精!

此乃八转仙材,相当罕见。来源于天地秘境“乎”,市场上流通极少。

第一次见到这座冰道晶精,方源就感叹不已:“想不到雪民一族的库存中,居然还有这等好货色!”

不过旋即,方源也就理解了。

别看现在,雪民一族生活得相当不容易,处境尴尬,但是仔细想一想,经过时代的残酷冲刷,能够幸存下来的雪民族群,怎么可能是简单货色?

至少在所有的雪民族群中,这一支当时佼佼者。只是困守一隅,如今处境不好,昔日却有光辉的。

冰道晶精正是雪民一族的第三份彩礼中的重点内容之一。

至于雪民一族为什么表达出如此巨大的诚意,方源不用猜都知道,一定是他之前战胜太古荒兽万首盘龙,战斗情景被雪儿记录下来,传达给了雪民一族。

不过,这也是方源故意放纵,乐见其成的事情。

“有了这座冰道晶精,就可以营造出一道雪怪的生产线来,假以时日,这里将会成为我全新的经济支柱!”

正预想着未来,毛六那边暗中传来消息:“宗主,天机仙蛊的炼制已经到达关键步骤,还请您亲自过来炼蛊!”

方源顿时精神一振。

若是能炼成天机仙蛊,接下来的天地灾劫,将变得可以预测。

只要是蛊修,就当知道这里面的意义是多么的重大!

\end{this_body}

