\newsection{欺负八转}    %第八百八十七节:欺负八转

\begin{this_body}

撞击海流宛若饥饿的猛虎,但是在方源的布局之下,它却成了拱卫五岛岛链的家猫。

当方源做到这一点时,大局便已定了。

接下来就是一些扫尾工作。

这些工作并不困难,但相当繁杂,消耗时间和精力。

方源是看不上的。

于是,方源和庙明神协商,将蜂将、花蝶仙子留在这里,剩下的人则跟他回去。

白光一闪,方源、庙明神、鬼七爷便又回到了接引岛。

任修平正在琢磨功德碑面上的任务。

他是奴道蛊仙,手中奴役着许多荒兽,还有少量上古荒兽。这些存在各有千秋,让他的手段也非常丰富。

然而,即便如此任修平仍感觉到中型任务的困难。

方源等人出现在他的身边,令他心头一跳,暗想:“这三人怎么会忽然同时出现?是巧合吗?”

方源对任修平点头,微微一笑。

任修平愣住,方源这种意料之外的态度,让他一时间不知道如何回应。

方源走到他的身边,碑面上的文字一阵变化,显露出全部的大型任务来。

方源只扫视了一眼,便迅速接取了其中一项。

然后,他就将这个任务分享给庙明神、鬼七爷。

下一刻白光再闪,他们三人骤然消失在原地。

任修平亲眼目睹了整个过程,脸色顿时阴沉下来:“楚瀛接取的任务,完全和我的不同,又上了一个档次。看来他之前功德暴降,定是领取了新的名号,可以接取更高一级的任务了。这个名号很可能就是——大好人!”

好人名号可以接取中型任务,大好人名号可以接取大型任务。

这个简单的关系,蛊仙们很容易就从名字上能联想出来。

但问题在于,就算想到了,又能怎样呢?

“中型任务就已经如此艰难,大型任务必然难上加难呐!”任修平心情阴郁,“还有,为什么庙明神、鬼七爷这两人能和楚瀛一起行动?他们难道可以接取同一个任务吗?”

这个发现让任修平怦然心动。

中型任务有的非常困难,但若是有人帮衬,这种难度就是可以调解的。

“虽然多人一起完成任务,功德会分润出去。但只要操纵得当,完成任务的效率提升了,收获反而会提高很多。楚瀛不就是干着这样的事情吗?”

任修平十分精明,思考片刻,就知道了方源的用意。

方源并没有打算隐瞒,事实上,这也是他故意泄露出去的。

庙明神这伙人可以争取,任修平为甚不行呢?

之前的仇恨?

呵呵。

仇恨能和眼前的利益相比吗?

任修平不是疯子,正因为他的精明,更有了合作的可能。

至于任修平和沈家之间,必定是有盟约的。

但是没有关系。

功德碑上就有相关的奖励,可以消除蛊仙身上的盟约束缚。

这点不用方源提醒,相信任修平早已看到了。

第二个大型任务,难度自然要比中型任务高得多,但仍旧难不倒方源。

费了一番精力之后,方源打开了局面,留下庙明神、鬼七爷扫尾。

当他再回到功德碑后,他稍稍等待了一会儿,花蝶女仙和蜂将传送了回来。

他们一回来,第一项大型任务就宣告完成。

方源获得了五百多功德,而庙明神一伙的功德参差不齐,皆在八十左右。

方源看着功德榜上的数值变化,算了一下,这项大型任务总体的功德奖励有九百七十三。

方源和庙明神等五人将这九百多的功德瓜分,起主导作用的方源自然占据大头,五百多的功德和他之前独立完成大型任务已相差不多。

然而,方源耗费的时间和精力,却是大大缩减了。

花蝶女仙和蜂将对视一眼,均看到对方的惊喜之色。

在这个任务中,他们先是帮衬方源,抵御海浪,然后负责各种杂务。真正付出的只是一些仙元、时间、精力,却能够收获到八十多的功德。

蜂将想想自己独立完成的蜂巢岛任务,他付出了一批蜂群,包含荒兽黄陀蜂,最终的功德奖励也只是八十多而已。

两相对比起来,当然是和方源配合完成大型任务,更加轻松,收益更大。

“你们先去和庙明神他们汇合,一起完成第二项任务。我建议咱们先同时领取一个团队名号。有了统一的团队名号,我们就可以相互联络了。”方源建议道。

团队名号并不便宜,需要一百功德。

但蜂将、花蝶女仙并没有犹豫,都听从方源的建议,领走了这个名号。

方源又将第二项任务分享给了他们,蜂将、花蝶女仙传送出去,和庙明神、鬼七爷汇合去了。

方源随后接取第三个大型任务,先行一步。

庙明神等人完成了第二项任务,回到接引岛,功德榜上四仙的功德纷纷上涨一小截。

庙明神和鬼七爷也花费功德,领取了团队名号,至此四仙可以和方源随时联络。

正好方源在那边打开了局面,又将任务分享给庙明神四人。

庙明神等人手贴碑面,一一接取了方源分享给他们的任务。

庙明神等人传送到方源身边,方源指点他们一番后,便又回到功德碑前看下一个新任务。

方源全流派兼修的优势,在这里充分发挥了作用。

许多任务就算是沈从声,也难以完成,但在方源眼里,绝大多数的任务都有解决之道。而他要做的,就是筛选这些任务,将收益高的任务优先完成。

有了庙明神一伙人的配合,方源节省了大量的时间和精力。而庙明神等人也抱上方源的大粗腿,功德节节攀升。

两方合作,完全是双赢,但对于沈从声、任修平等人却是极其糟糕的情况了。

这些人眼睁睁地看着功德榜上,庙明神等人的功德一步步提升,超越自己,形成第一梯队,真的是又气又恨,偏偏又无可奈何。

再看看方源的功德,已经积累了数千!可谓一骑绝尘,把众仙都甩得远远。

如此巨大的察觉,让众仙都兴不起一丝追赶的欲望。

“仙蛊兑换的功德要上万,但对楚瀛而言,已经不远了。”许多人看到这一点,心情相当复杂,失落中带着艳羡。

“真是奇怪,楚瀛为什么能这么顺利地完成这么多不同的任务呢?”

“他必定是知晓这里许多的奥秘,可恨我们不能随意出手!”

“庙明神这伙人真的走了狗屎运,怎么就攀上了楚瀛?”

众仙嫉恨交加。

沈从声叹息一声,他在最近十多天里叹息的次数,比之前数年之和的还多。

功德榜上的排位十分刺眼,沈从声的排位还在花蝶女仙、蜂将之下。

他堂堂一个八转蛊仙,居然会比不上两位六转蛊仙,这让他颜面何存?情何以堪?

“若是能接取大型任务,我的八转实力才能充分发挥,到那时功德获取必会上涨一截。但如何能让其他人也和我接取一样的任务呢?”

大好人名号是可以想到的,但分享任务的名号是什么?

这点就需要尝试了。

而尝试就要考验沈从声等人运气,就要耗费他们的功德。

这对他们当下已经落后的糟糕情况,无疑是雪上加霜。

“我让沈家蛊仙贡献功德,那是可以的。但对任修平、童画而言呢?我这样做,是不是会把他们俩个推向楚瀛那边去?”

沈从声心存顾忌。

他知道,自己的影响力正在不断地下滑。

在得知蛊仙身上的盟约,也可以被功德碑中的奖励消除之后,沈从声对任修平等人的关系处理,无疑更加慎重。

方源和庙明神等人联合完成任务的模式,明显是先进一筹的。

沈从声很想模仿复制,所以他想团结住任修平和童画,这两人的流派是他们沈家所缺乏的。

沈从声已经有些后悔,之前不应该强逼庙明神等人,让他们消耗功德来兑换奖励的。

此举让群仙对他离心离德,当时是明智之举,现在看来却是失策得很。

这也不能多责怪沈从声,毕竟他对功德碑种种一无所知。

他当时做出的决策,在当时的情景看来,无疑是正确的。

功德碑上的任务种类繁多,中型任务开始就变得困难。若是有对应的流派,任务难度会下降很多。

所以,掌握越多的流派,对于完成任务就越是有利。

方源已经将庙明神等人团结在身边,沈从声吃了一亏,立即学聪明,对任修平、童画大加笼络。

他八转的威望还在,任修平身上的盟约也没有消除,更关键的是任修平还和庙明神、楚瀛结仇过。

种种因素导致任修平、童画仍旧围拢在沈家身边。

当方源再次来到功德碑前,一个新的大型任务忽然出现——悔哭海中囚禁的魔仙神智不清,请配合大阵,将其剿杀!

“出现悔哭海了?”方源微愕,旋即毫不犹豫地接取了这个任务。

这个时候,有关鲛人城的任务已经出现过四次,但鲛人圣女的选拔还未开始呢。

方源重生之后的影响,已经大大改变了功德碑上的任务。

凭借这个任务,方源完全可以提前收取悔蛊了。

这是一件大好事!

悔蛊一到手,他就能大幅升炼仙蛊,比上一世提前很多进度。

\end{this_body}

