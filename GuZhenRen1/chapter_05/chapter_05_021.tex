\newsection{天意为难便斩难!}    %第二十一节:天意为难便斩难!

\begin{this_body}



%1
戚灾脑海中,思绪起伏不定,连带着脸色也变得凝肃起来。

%2
方源也是眉头紧皱。

%3
他之前动用飞剑蛊,绞碎的阴气,转眼间又都恢复如初。

%4
碧沉沉的阴气长索,再次向方源绞来。大有不纠缠上方源,誓不罢休之势。

%5
“飞剑蛊虽强,但对付这种聚散如意的敌人,却不拿手。”方源心中闪过一丝明悟。

%6
之前,他对付荒兽泥怪的时候,就有同感。

%7
如今面对气道蛊仙,这种感觉再度浮上他的心头。

%8
当然,这也是他缺乏相关杀招的缘故。

%9
若是有一种剑道杀招,能够辅助飞剑蛊,令其攻击扩散,而不是集中一点。这种麻烦也就迎刃而解了。

%10
其实,方源早已预感到了这种情况。

%11
所以他才动用剑遁仙蛊,明智撤离。

%12
但没想到,对方居然有移动杀招,可以追得上来。

%13
方源陷入下风。

%14
之前,试探的时候,他凭借层不出穷的凡道杀招,占据过上风。

%15
但现在,动用了仙蛊、仙道杀招,开始玩真格的,他不可避免地陷入劣势。

%16
对方毕竟是七转蛊仙,修为上要比方源高出许多。

%17
方源打不过,很正常。

%18
能够挣到这种局面,也属不易了。

%19
“我只不过有两只仙蛊,而对方七转,修为比我高,手段比我多。若是有仙级战场杀招,我就算有飞剑仙蛊,也突破艰难。此地不可久留!”

%20
剑遁!

%21
方源再次催动这只七转仙蛊,一飞冲天而去。

%22
“嘿,倒是机警。”戚灾冷哼一声。

%23
他的争伐手段,最擅长的就是打阵地消耗。

%24
阴气长索被绞碎。气息弥散开来,随着时间推移,整个战场就会越发偏向于气道。

%25
到那时,戚灾不仅使用仙蛊时,威能更快更大,而且消耗仙元也相应缩减。有利无弊,十分了得。

%26
但方源只是交锋一下,就立即飞撤,迅速转移。

%27
打一枪换一个地方。

%28
这就让戚灾完全没有优势积累下来了。

%29
蛊师移动速度有限,常常局限于一地。但蛊仙却能出入青冥,纵横千里,大大有别于蛊师。所以,击败一位蛊仙容易,但要杀死蛊仙。却很难,除非有仙级战场杀招围困。

%30
不过,这种战场杀招,铺设起来的时候,也需要耗费时间。蛊仙对战时,当着别人的面,只有暗中铺设,手段必须隐秘。才能铺设成功。

%31
或者就像方源对付黑城的那会儿,提前布置战场杀招。再将敌人引诱进去。

%32
方源虽然遁去,但戚灾却又仙道杀招,可以迅速转移。

%33
方源屠戮了倪家一族,这地方就是戚灾负责的,若就这样放过方源,戚灾回去不好交代。

%34
更主要的一个原因。是戚灾担忧方源是其他四家的余孽。

%35
事关千年赌约,戚家已经胜利在望,这个关键时刻,可不能出幺蛾子。

%36
所以,戚灾必须要将这个事情。调查得水落石出。

%37
他不可能善罢甘休。

%38
方源遁去片刻,就发现戚灾追赶上来。

%39
“到底多大仇?要如此追杀我?”方源郁闷不已。

%40
前世记忆中,这个倪家根本就没有什么蛊仙长辈,所以方源才行事恣意。没想到却真的因此,引来了蛊仙追杀,还是一位六转一位七转,两位蛊仙。

%41
方源猜测戚灾的动机,毫无成效。因为对于五相赌约,他一无所知。

%42
他只好且战且退。

%43
每次交手,就动用飞剑仙蛊,打个两三回合,然后就动用剑遁仙蛊撤退。

%44
交手次数多了,方源也渐渐看出端倪。

%45
“他身边那位六转蛊仙,神态有异,似乎是晋升蛊仙不久之人。她口中称呼七转蛊仙为七爷爷,两人血脉相连,应当是亲族。”

%46
“这位六转蛊仙,应当不足为虑。若是她能出手,早就应该出手,干扰围困我。可整个战斗,她都一直坐在后方,无动于衷。”

%47
方源得到不少情报,戚灾也同样对方源了解加深许多。

%48
“这个古怪的家伙,似乎真的是位六转蛊仙。呵呵,居然用两只七转仙蛊……”

%49
戚灾对这情况,都不知道怎么评价才好。

%50
一般而言,大多数的六转蛊仙,连一只六转仙蛊都没有,十分渴求。

%51
眼前这位,也是没有六转仙蛊,但却跨越了一大步,直接拥有了两只七转级数的。

%52
“可是这一步,你跨得未免太大。我倒要看看你究竟有多少青提仙元,可以催动七转剑蛊的!”

%53
戚灾心中冷笑。

%54
他本就擅长打消耗战,现在修为比方源高一转,方源还越阶动用七转仙蛊,所以他更不着急了。

%55
方源边打边退,他就随战随进,阴险狡诈,难缠至极。

%56
方源暗道不妙:“交战以来,我已耗去数万仙元石。这都是从琅琊地灵处输送过来,耗费的都是我的门派贡献。”

%57
“此人打得好算盘,是想消磨我的仙元,偏偏我知道他的阴谋,却无可奈何。”

%58
若是用六转仙蛊,方源的持续作战能力,就要强很多,不至于落到如此尴尬的境地。

%59
以他六转修为,运使七转仙蛊,实在有些勉强了。

%60
戚灾尾大不掉,方源甩不掉他,只能僵持下来。

%61
双方又交战了十多次,方源为了支撑场面,不得不饮鸩止渴,不断地向琅琊地灵索要仙元石。

%62
而他的琅琊派贡献,已然见底。

%63
戚灾暗暗惊奇,方源能支撑到现在,大大超乎他的意料。

%64
“这个六转蛊仙,着实古怪得很。此人身上必有大秘密,若是我能得到,说不定是我修仙以来。最大的机缘!”

%65
斗到如此程度,戚灾心里越发火热,看着方源的阴寒目光,仿佛就是打量一个会移动的人形宝库。

%66
但下一刻,戚灾目光一凝,脸现惊诧之色。

%67
他暗叫不好:“前面那道山脉。不就是五界山脉?糟糕,他原来打的是这个主意。绝不能让他逃进山脉中去。”

%68
这五界山脉,来头不小。

%69
乃是源于南疆的一位八转蛊仙,其名陶铸,号称禁师。

%70
此人专修禁道,禁道乃是律道的分支,好比从智道流出的情道一脉。

%71
人称他为禁师,意喻此人乃是当代的禁道蛊仙之师,可见他在这方面的造诣有多惊人。

%72
他对五域界壁深有研究。企图找出方法,能供蛊仙轻松跨越界壁。

%73
这五界山脉就是他为了研究,故意改造而成。

%74
可惜,自他陨落,也未得到什么有效成果。但这五界山脉,却是遗留下来,成为南疆一处特殊所在。

%75
方源在被戚灾纠缠住后,就已急寻脱困之法。

%76
他对南疆地形十分熟悉。五界山脉便成了他的希望。

%77
戚灾一心想消磨方源的仙元,这正中方源下怀。让他利用剑遁仙蛊,费尽一番周折后,飞越数十万里,终于赶到了这里。

%78
戚灾久不出户,又被方源牵扯了注意力,没想到此节。

%79
他连忙出手。想要阻拦方源,但方源手上两只七转仙蛊在手,且有早已戒备,哪里能让他得逞?

%80
戚灾面色阴沉地看着方源,一头扎进了五界山脉之中。

%81
他犹豫了一下。便下定决心,对身旁的戚荷道:“你就坐在这里,我让气宗狮护着你,不会有事。待我杀了此人,再来寻你。”

%82
“七爷爷小心!”戚荷忙道。

%83
戚灾点点头,身涌气浪,紧追着方源的背影,也扎进五界山脉。

%84
一入山脉,戚灾顿感到一股强大的排斥之力,想要把他排挤出去。

%85
原来这五界山脉,乃是当初禁师陶铸仿造五域界壁,凝造而成。蛊仙进入此处,宛若穿梭五域界壁。

%86
方源选择的这处山脉,外散金白之光,乃是仿造的中洲圣贤界壁。

%87
戚灾乃是南疆蛊仙,进入其中,自然要被排挤。

%88
他心知方源打算:“这狡猾的小子,来了此处,他六转修为,反而成了优势。我七转蛊仙在这里遭受的排斥,要比他大得多。”

%89
“不成!此人身上疑点重重,且又关乎我戚家数千年赌约得胜之事。我必须擒杀了他!”

%90
戚灾意志坚定,没有因为眼前的难关,而有所动摇。

%91
双方都不再飞行,开始在五界山脉上艰难跋涉。

%92
就在这场追逐战再次展开的时候,中洲地渊深处,一场战斗已经终止。

%93
影无邪、太白云生、黑楼兰三人,围绕着上古魂兽的尸躯,正在抓紧时间,搜寻魂兽体内的魂核。

%94
魂兽的体内,孕育一颗魂核,只有鸡蛋大小,但却凝结了大量的魂道道痕,是上佳的蛊材。除此之外,魂兽的尸体会随着时间,缓缓消散,并无利用之处。

%95
“究竟在哪里?”太白云生一心一意地搜寻着魂核。

%96
黑楼兰则开口,有意无意地赞道:“布置在这里的蛊道大阵,居然如此厉害。短短片刻,这头上古魂兽即便有仙蛊魂啸,也只能饮恨身陨。”

%97
影无邪没有开口,眉头紧锁。

%98
“这是天意害我,将这头上古魂兽,引到这里来。这一次,我虽然剿杀了这头魂兽,但也动用了蛊阵。”

%99
“蛊阵动用的次数越多,这里暴露的危险就越大。毕竟中洲天庭,一定会千方百计地搜寻我的。”

%100
这时,石奴暗中向他传来讯息。

%101
影无邪暗自分析:“嗯?刚刚不久前,琅琊地灵利用宝黄天,再次向方源传输了暗渡仙蛊?”

%102
“这只仙蛊本是我影宗之物,可惜随着姜钰身亡而毁。后来,琅琊地灵创建琅琊派,为了今后门下蛊仙行走方便,就炼制出了暗渡仙蛊。”

%103
“此蛊能针对天意,遮掩气息。呵呵,也是。我这边被天意针对,方源那边肯定更加糟糕,说不定是被蛊仙追杀呢。”

%104
影无邪随意一猜,却真的被他猜中。

%105
不过,戚灾追杀方源,却没有成果。

%106
双方在五界山脉中交手,方源重伤,但他仗着自己修为低,行动更加自由,终究还是逃脱了戚灾的追杀。

%107
戚灾见方源逃脱无踪,自己又状态不佳,无可奈何之下,只得往返。

%108
快出五界山脉之时,他却见到了戚荷。

%109
“七爷爷,你回来了,孙女好生担心。”戚荷在原地踱步,看到戚灾后连忙迎接上去。

%110
戚灾心头一暖,但脸色却沉下来,口中训斥道:“糊涂!我叫你和气宗狮待在一起,怎么不听话,到这里来?这可是险地!”

%111
戚荷脚步一滞,慢慢走到戚灾面前,低垂下头,歉然道:“孙女,孙女错了……”

%112
戚灾嗯了一声:“刚刚那人已经被七爷爷重伤,算他运气好,让他逃走了。不过将来若是让我见到,绝不会像今天这般……呃!”

%113
戚灾陡然惊呼。

%114
他瞪大双眼,死死地瞪着戚荷。

%115
一道晦暗的飞剑,正中他的眉心,锋锐的剑尖从他的脑后突出。

%116
戚荷还原成方源相貌,对着戚灾淡淡而笑。

%117
震惊、懊悔、恐惧无数情绪,充斥戚灾的心头。

%118
但一切都已经晚了。

%119
砰。

%120
他倒在地上,发出轻微的闷响。

%121
他,死了。

\end{this_body}

