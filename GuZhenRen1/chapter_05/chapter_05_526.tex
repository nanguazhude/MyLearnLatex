\newsection{中凤有大风}    %第五百二十八节:中凤有大风

\begin{this_body}

%1
北原。

%2
一望无垠的草原,辽阔至极。

%3
晴空万里无云,看似空无一物,实际上却有三位蛊仙隐藏身形,漂浮于空中。

%4
“按照紫薇仙子大人推算的结果,这里便是最接近琅琊福地的地点了。”为首的一位蛊仙男子心中暗暗思量。

%5
若是让北原蛊仙看到这位男性蛊仙的面貌,一定会震惊低呼一声:“凤仙太子!”

%6
没错,此人就是北原明面上的五大八转蛊仙之一凤仙太子。

%7
时局变化,八转第一人的雪胡老祖,因为大雪山福地摧毁,已经携妻销声匿迹。

%8
药家的太上大长老药皇,被长生天选中,卸下家族重任,成为长生天中新一代的南荒仙人。

%9
百足天君曾是散仙,但现在已经占据黑家地盘,建立了百足家这个超级势力,跻身正道。

%10
五行大法师则仍旧受困于长生天的仙蛊屋劫运坛中。

%11
因为天庭蛊仙的入侵,长生天现世,百足天君由散转正,雪胡老祖消失,整个北原几乎成为正道天下,一片光明堂皇,魔道蛊仙都缩着脖子过活。

%12
五大八转蛊仙当中,唯有凤仙太子一人,仍旧是安稳如常,身上没有剧变。

%13
此人身份其实非常特殊,表面上他是外姓八转,宫家的太上长老,宫婉婷的丈夫,但内里他还有一重身份,便是灵缘斋很早之前,铺设在北原蛊仙界的棋子,乃是中洲在北原最高层次的间谍!

%14
紫薇仙子筹谋良久,这一次派遣凤九歌等人,奇袭琅琊福地,除了陈衣这股明面上的接应之外,还在暗处启动了凤仙太子这个重大棋子。

%15
“大人,距离这里最近的,便是北原十大禁地之一的松尾岭,琅琊福地极有可能潜藏于此。”凤仙太子身边,一位六转女仙开口道。

%16
凤仙太子身边,有两大婢女,因为凤仙太子的威名,也跟着名动北原。

%17
她们分别是乐瑶和幽兰剑师,这两人亦都是灵缘斋的蛊仙,此刻开口的便是乐瑶,他身着黄衣,娇美动人。

%18
而幽兰剑师此次留守大本营,并没有跟随凤仙太子。

%19
凤仙太子微微点头,他也是如此猜测。

%20
他目光微微一转,停留在身边的蛊仙身上。此时站在凤仙太子另外一侧的,竟然是一位六转毛民蛊仙!

%21
这位毛民蛊仙目光痴呆,一动不动,仿佛雕塑,显然是被凤仙太子以无上的手段牢牢控制。

%22
若是琅琊地灵或者其他琅琊派的毛民蛊仙在场,见到此人,一定会脱口而出,说出他的身份。

%23
皆因这位毛民蛊仙,就是琅琊福地中人。

%24
当初,义天山一战后,琅琊地灵帮助方源,转移走了狐仙福地中的资产。

%25
方源掌握了至尊仙体,要回归北原,却受到追杀。

%26
琅琊地灵为了接应方源,主动派遣了两位毛民蛊仙,但是这两位蛊仙一个死亡,另外一个神秘失踪。

%27
原来,其中失踪的毛民蛊仙另一一番际遇,可是最近被凤仙太子擒拿活捉,从这位毛民蛊仙口中拷问出来的种种消息,也就成了紫薇仙子推算琅琊福地的关键线索。

%28
正是因为他知晓琅琊福地中的地貌景象,中洲十大古派才能动用仙道杀招,将凤九歌等人直接送入琅琊福地。

%29
乐瑶眉头微蹙,有些担忧地道:“按照时间推算,凤九歌一行已经入侵许久了才是,怎么到现在都没有消息传来?是不是遇到了麻烦?”

%30
凤仙太子沉吟道:“麻烦当然会有,就算我们掌握了琅琊福地的诸多情报,知己知彼占据主动,但此乃天下第一福地,源自长毛老祖,底蕴深不可测,总会有一些出人意料的底牌吧。”

%31
乐瑶眉头皱得更紧:“此次中洲方面,只派遣数位七转蛊仙进攻琅琊福地,会不会有些……托大?”

%32
凤仙太子哈哈一笑:“那是你不了解凤九歌此人。”

%33
“哦?”乐瑶诧异。

%34
凤仙太子目光远眺,悠然一叹:“我凤仙太子也算是天资纵横之辈,十三岁开始修行,十六岁成就四转,十八岁五转巅峰。接受门派重任后,前往北原,独自发展。耗费三年,成为六转蛊仙,之后渡劫,无一败例。行走北原,加入宫家,百年之后,成就八转,虽然如履薄冰,但北原蛊仙界又有何人识破了我的身份?”

%35
“我曾以此自得,心隐骄傲,皆因我主要依靠的不是门派的暗中支持,而是自身努力,运筹帷幄,左右逢源,四方借力,终修成八转!直至有一天,我遇到凤九歌,方知我不如他。他比较起我,才是更加名副其实的天之骄子,才情之高,能惊艳这个时代!”

%36
乐瑶惊讶无比。

%37
在灵缘斋的核心高层中,早年有“北凤中凤,双凤齐飞,灵缘煊赫”的说法,说的就是八转蛊仙凤仙太子,以及七转蛊仙凤九歌这两人。

%38
灵缘斋的几位知晓凤仙太子身份的太上长老,都一度认为过,灵缘斋今后的成就主要会依靠凤仙太子、凤九歌。

%39
他们将凤仙太子、凤九歌并列,但凤仙太子乃是八转,名列前茅,凤九歌被人看中更多的原因是他的潜力。

%40
让一个七转蛊仙和自家八转修为的主人相提并论,乐瑶、幽兰剑师一度相当不忿,就连凤仙太子也心中不喜,曾写信回去,质问灵缘斋那几位核心太上长老。

%41
不过随着时间推移,乐瑶、幽兰剑师对此事,逐渐又有了新的看法。

%42
至于凤仙太子的心理转变,则是因为他和凤九歌的那一次交手。

%43
早在多年前,八十八角真阳楼刚刚倒塌不久,凤九歌率领中洲蛊仙数位,前来北原调查真相。作为灵缘斋安排在北原蛊仙界的最大棋子,凤仙太子暗中接见了凤九歌。

%44
正是这次接见,双方动手,切磋了一下。结果令人震惊,凤仙太子虽然未出全力,但也只能和凤九歌打个平手!

%45
想起曾经交手的场景,凤仙太子悠然长叹:“凤九歌拥有八转战力,并不让我吃惊,真正让我震动的是他使出的那记仙道杀招!这是前所未有的开创,历史上从未有人有过他如此创举。”

%46
“什么样的仙道杀招?”乐瑶便问。

%47
凤仙太子面露凝重之色,缓缓地道:“凤九歌将其称之为大风歌。”

%48
北原,琅琊福地。

%49
凤九歌深呼吸一口气,离歌也不起作用,当下,他只剩下最后的一个手段。

%50
这个手段,他向来不轻易使用。

%51
因为一旦催使,不但风险极大,过程凶险至极,而且造成的危害也太过巨大,大损战后收益,一直都让凤九歌使用起来十分谨慎。

%52
但现在,他不得不承认,他已经被逼入险境。

%53
他从门派、从紫薇仙子那里,得知了许多关于琅琊福地的情报。但是真正交手起来,琅琊福地展现出来的实力,远超他的估算!

%54
不只是这座超级仙阵,还有那些隐藏在仙阵当中的那些不知道从哪里钻出来的蛊仙,还有万籁俱寂杀招。

%55
凤九歌知道琅琊派手中,掌握了一招克制音道的仙道杀招,名为天地寂静。

%56
但他不知道,此招后来被方源特意改良,形成威力更加强大的万籁俱寂!

%57
凤九歌对杀招天地寂静有防备应对的手段,但是面对骤然拔高数倍的万籁俱寂,他却是有些无可奈何了。

%58
仙道杀招大风歌!

%59
凤九歌左右闪躲,争取到了一个机会,催动出他一生修行中,掌握的最强大的攻伐手段。

%60
呼呼呼!

%61
狂风骤起,天地变色!

%62
此风不是金风,不是落魄风,不是柳风,不是仙风,不是幽沉风,不是星汉风,不是扇门风。

%63
此风一现,琅琊地灵脸色狂变,毛民蛊仙震骇不已,其余异人蛊仙瞠目结舌!

%64
此风是大同风!

%65
是天下第一风,能同化一切,鲜有手段能够克制。

%66
仙窍毁灭时刮起来的,就是这样的风。

%67
“怎么会是大同风?这不可能!”有的毛民蛊仙失去了镇定,大声吼叫起来。

%68
“难以置信,居然有人能够操纵大同风?!”冰卓震骇至极,眼珠子都要瞪掉出来,哪里还有一丝平日里的冷酷风范。

%69
人族历史上,没有一个人能够将大同风化为己用的。但现在,这个记录被打破了,凤九歌成为有史以来的第一人!

%70
他竟然能够打出大同风!!

%71
凤九歌掌握七首歌,其中得宝歌、向天歌乃是辅助修行之用,碧玉歌、天地歌、俯首歌、离歌都是底牌,屡次使用。最后一首便是这大风歌,却是轻易不动用的。

%72
此招若是施展失败,凤九歌首先就会被大同风摧毁,凶险至极。

%73
许多歌能够通过一曲之士杀招,转变成分身一样的存在,但这首大风歌却不能够,它是最特殊的!

%74
这风太厉害了,一瞬间战局就彻底颠倒。

%75
异人蛊仙们束手无策,惊慌失措。

%76
皆因寻常的手段杀招,打在大同风中,都会被它同化,化为资粮,成就越加庞巨的大同风。

%77
“琅琊福地纵然是天底下第一福地,但面对此风,恐怕是要毁灭了!”一位石人蛊仙绝望地道。

%78
“绝不会!!”琅琊地灵大吼一声,他知道自己必须稳定军心。

%79
他连声吼叫:“居然真的是大同风,大风歌!方源告诫我时,我还不信,但事实就在眼前,我不得不信了!”

%80
“诸位放心,我们早有手段,对付此风!”

%81
“什么?方源居然连这样的秘密,都能打探到?”其余蛊仙听到这话,纷纷燃起希望。

%82
这一刻,同样拥有八转战力的方源,虽然并未出场,但却起到了关键作用,帮助琅琊地灵成功稳定了本方士气。

%83
方源是重生之人,知晓此招极其凶恶,在五域乱世中凤九歌凭此,屠戮四方,无人不惧,所以早就提醒了琅琊地灵。他曾经和凤九歌交手时,也是极其顾忌,谨慎异常。

%84
此招威能极端恐怖,但是也有重大弊端,比如在五域中释放,方源完全可以远远躲开。

%85
但是现下这种情景,大同风发生在琅琊福地当中,琅琊地灵必须硬着头皮趁早解决此风。

%86
他究竟有什么手段,能克制大同风?

\end{this_body}

