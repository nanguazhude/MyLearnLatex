\newsection{大会开启}    %第九百零五节:大会开启

\begin{this_body}

光阴长河奔流不息。

天庭。

议事大厅。

天庭蛊仙济济一堂,鲜少有未到达的蛊仙,这是天庭上千年来人数最多的一次集会。

龙公坐在最高的一层,环视全场:“商议既定,那就去做吧。能否恢复天庭的荣光,就全赖在场诸位的奋斗了。”

“我等必定竭尽所能!”全场众仙齐声回应,众志成城。

事关五域格局,天下大势,万众期待的中洲炼蛊大会,终于开始!

消息传出,天下震荡。

得益于方源的广泛宣传,蛊仙们都知道此届炼蛊大会的重大意义。而凡人蛊师们却多有不知,只认为这次的中洲炼蛊大会和往届没有什么不同。

蛊师们纷纷起身,向中洲进发。

每一届的中洲炼蛊大会,都会吸引到其他四域的蛊师前来参加,规模蔓延五域,极其庞大。

但这一次,从一开始就显现不同。

西漠。

一大群的蛊师,站到了界壁前。

一位少年发出惊叹声:“父亲,不是亲眼见到,真的难以想象世间居然会有这样雄壮的景象。”

“这就是界壁。穿越这里,我们就去往中洲了。”少年身边的中年蛊师关照道,“进入界壁,视野很受局限,到时候你要紧跟在爹身边,不能走散了。记住了吗?”

“明白了爹,孩儿记住了。”少年乖巧点头。

“嗯。”张全看着自己的儿子张平,心中欣慰。

他的儿子拥有很强的炼道天赋,为了他的成长和发展,张全便带他长途跋涉,前往中洲参加炼蛊大会,增长见识并且锻炼能力。

“什么声音?”下一刻,张全疑惑地仰头。

然后,他瞳孔猛缩,惊骇欲绝。

只见一颗巨大的流星,小山一般,向他们飞坠下来。

周围的蛊师们纷纷惊吼起来。

“怎么会这样?我们完了!”一瞬间,张全非常后悔。如果他独自前往中洲,他的儿子张平就不会死了。

轰!

流星坠落下来,和地面亲密碰撞,发生惊天动地的大爆炸。

张全等人尸骨无存。

高空,一位墨人蛊仙冷漠地扫视一眼,迅速离开。

相同的一幕幕,发生在五域各地。

这些蛊仙都是方源的麾下,接到他的命令,专门来阻击参加中洲炼蛊大会的人。

随后,方源就在宝黄天中公开发出声明:“谁若是参加此届炼蛊大会,谁便是我方源的仇敌,说不定哪一天,我就会找你们来算账!”

五域蛊仙不禁惊叹:中洲炼蛊大会才刚刚开始,方源就出手了。真不愧是小魔尊,如此嚣张残暴。

天庭震怒!

因为方源下属的阻击,导致参加大会的蛊师少于往年,这对修复宿命蛊自然很不利。

和上一世不同,方源命令下属只在北原、南疆、西漠、东海四域活动。如此一来,中洲和天庭的蛊仙也不好主动前往其他四域去保驾护航。

在这个节骨眼下,中洲、天庭蛊仙若是前往四域,那就是在**四域超级势力的敏感神经。

天庭只好斥责方源的罪恶,同时联系四域正道,希望他们发扬正道的精神,制裁这种魔头行径。

四域正道纷纷响应,但落实到行动上的却很稀少,就算有所行动,也只是雷声大雨点小。

上一世,方源是在中洲本土动手,因此不久后就遭受了阻截。

这一世,方源提前出手,立即收获丰厚战果。

蛊师们蜂拥中洲的浪潮得到了有效遏制。

谁也不肯为了参加一届中洲炼蛊大会,就丧失性命。

再加上,四域各大超级势力早已制止自家蛊师前往参加大会,导致这一届的中洲炼蛊大会变得萧条了许多。

中洲,归一派。

作为炼蛊大会的报名地点之一,这里人声鼎沸,喧闹无比。

“炼蛊大会终于开始了!真是奇怪,怎么不见其他四域的面孔?”罗生叹息着。

他在上一届的炼蛊大会中,就是因为意外而遗憾落败给了一位北原蛊师,这一次卷土而来,非常想见到北原蛊师成为自己的对手。

罗生心中急切,想要证明自己。

“只是中洲太大,五域太广,总有人才辈出。”

他目光扫视,发现在场中人就有三位是自己的强劲对手。

一位老者,成名多年,同样是炼道大师。另外的则是一位青年男子,还有一位明眸少女,前者在上一届的炼蛊大会中大放异彩,而后者则是近十年来崛起的闪亮天才。

“这一次炼蛊大会,竞争仍旧会那么激烈啊!”

“我一定要好好把握住这次良机,再不能错失了……”

“我要证明我自己,让妻子和儿子都抬起头来做人!”

轰!

下一刻,一头巨大的上古年兽从天而降,将报名的整个大殿都压成渣滓。

一切宏图野望都烟消云散,都化为乌有。

那些大好前途的年轻蛊师,受人瞩目的天才少女,亦都惨死当场,被压成肉泥血骨,混在一起,连尸体都辨认不清。

高空中,白凝冰的身影一闪即逝。

她负责腾挪转移,到达地点后,便直接释放上古年兽。至于结果,方源已经说了,不需要她管。

之前,方源派遣异人蛊仙在四域提前阻杀参赛蛊师队伍。现在,他则命令影宗诸仙潜入中洲,利用上古年兽专门破坏比试地点,酿成一幕幕的惨案,牺牲的蛊师精英数不胜数。

得知这个情况,天庭立即大肆宣传内幕真相。

“这世上原来真的有仙啊……”

“一切的罪魁祸首就是方源!”

“方源,你杀了我的父亲,此番血海深仇我一定会报!”

“唉……纵然五转,不成蛊仙,终为蝼蚁。”

“仙人高高在上,我何必为了区区一届炼蛊大会,丢了性命?去休,去休。”

“我一定要去。我们若是不参加此届大会,不正是方源希望看到的吗?”

有人恐惧,有人惊悚,有人退缩了,也有人反而坚定了信心。

“这是在引发民愤,累积人意啊。”上一世方源并不清楚天庭此举的动机,但这一世他是心中雪亮。

方源下令,尽数收拢麾下。

天庭和中洲的蛊仙们积蓄着愤怒和仇恨,正要大举出手,结果愕然发现方源的诸多下属突然消失无踪。

“方源不是要破坏中洲炼蛊大会吗?怎么就收手了?”看戏的四大域超级势力十分纳闷。

而在南疆的一场秘密的会面中,武庸分析出了方源的一部分想法:“方源占了便宜,再继续下去,恐怕就要被天庭方面推算到各个人的方位,他的下属会损失惨重。”

“那方源还会出手吗?”有人问。

“当然会。”武庸不假思索,“但在这之前,我们也该秘密动身了。”

上一世,方源亲自动手,四处破坏,结果行动久了,受到千夫所指杀招的标记,只得龟缩起来,让白凝冰等人出动。

而这一世,方源一直潜伏着,命令麾下四处破坏和杀戮。

但其他人可不像方源,拥有智道的手段能够防备紫薇仙子的推算。所以,方源只是令白凝冰等人突袭一遍后,便立即命令他们潜伏。

这场炼蛊大会才开了个头而已。

旷世大战还未激发,这些下属方源还有大用。

------------

\end{this_body}

