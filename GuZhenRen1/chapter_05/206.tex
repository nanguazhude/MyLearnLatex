\newsection{龙人}    %第二百零六节:龙人

\begin{this_body}

“我们战仙宗,乃是中洲十大古派之一。小子,你既然是我师兄推荐过来的,自然已经明白我们战仙宗的选拔规矩,就是以战果论高低。”

方源的视野中,一位蛊师居高临下地看着自己,然后继续道:“你要是想正式成为我战仙宗的弟子门人,那就好好打上一场,一百人中最后剩下来的十名,就拥有加入门派的资格。”

“战仙宗……”

“怎么会是这样的梦。”

“幽魂魔尊生前似乎没有加入过战仙宗吧?”

“看来是他吸收了生死门内的魂魄,得到了大量的修行经验。因此衍化出来的?”

方源心中猜测着。

这个时候,蛊师已经取出了三只蛊虫。

第一只蛊虫,像是毛毛虫,淡蓝色,有成年人拇指大小。虫头上端,还有一个尖尖的小角。

第二种蛊虫,是一只蝴蝶,赤红斑斓,散发着温热的气息。

第三只蛊虫,则是一只蛤蟆,石头一样的质地,灰不溜秋。

梦境中,蛊师接着开口:“这三只蛊虫,分别是水箭蛊、火衣蛊以及灵涎蛊。你已经是一转蛊师,挑选一只罢。”

“只能挑选一只吗?”方源询问道。

“所有人都只能挑选一只。”负责的蛊师面无表情地回答。

方源点了点头,他细心思量:“水箭蛊,无疑是攻伐所用,但在攻击蛊虫当中并不优秀。火衣蛊则是用来防护,结成一片火焰衣裳,倒也有一些灼伤敌人的微弱效用。而灵涎蛊,能给与蛊师治疗伤势的能力。”

“若是全部挑选,也就罢了。只能挑选一只的话,就说明这项考核,必须要和他人合作。面前的蛊师刚刚也说了,一共有一百人,最终只有十个名额。我要探索掉这片梦境。恐怕就是要顺着梦境走,夺得这十个之一的名额。”

“我选择火衣蛊。”方源思量了一下,立即回答道。

他必须争分夺秒。

梦境中的探索时间,可是很珍贵的。魂魄会时刻受到梦境的消磨。一旦魂魄削弱下来,虚弱到某个质变的点后,方源的思维能力,还有认知能力都会大大降低。这就会导致方源无法分辨梦境和现实,不知道自己是在做梦。以至于最终魂魄陷落在梦境之中。彻底消弭,方源只剩下一具肉身存留在世上,也就像之前的那些南疆蛊仙们落得个凄凉的下场。

探索梦境是很危险的。

尤其是方源独自一个人探索的时候,身边没有第二人照顾看护。一旦出了一些异常状况,方源无法自救,那就糟糕了。

但能够让方源信任的,又有谁呢?

为了获得境界上的飞升,方源不得不付出一些代价,冒一些风险。

“当然,我的话。自然有着底牌,不像其他蛊仙无脑乱冲,只搏运气。”方源想到这里,立即催动了梦道仙级杀招解梦!

大量的梦道凡蛊被消耗,玄妙的力量立即展现在方源的眼前。

一位蛊师蓦地出现了。

之前那位蛊师一副惊讶模样,对后面出现的蛊师问道:“师兄,你怎么来了?”

师兄便对师弟道:“我是来看看,师弟,把灵涎蛊也给这孩子罢。”

“可是师兄,如此作弊。若是被发现的话……”师弟为难道。

师兄笑着:“放心吧,灵涎蛊暗中运用,一点外在的表现都没有。况且这孩子还选了火衣蛊,到时候火焰衣裳遮蔽全身。谁能看得出里面的疗伤情况?”

师弟这才点头:“既然是师兄这么说,那师弟我也没什么话好讲了。接着吧。”

说完,他就向方源抛过来两只蛊虫,一只火衣蛊,一只灵涎蛊。

方源刚刚接到手中,周围的梦境便开始缓缓消散。转变成第二幕。

解梦杀招的效果,就是顺应梦境的规则和变化趋势,而产生一种利于探索者的梦境变化。

因此,解梦杀招的具体成效,会因为不同的梦境,而不同。

“但我这一次运用解梦杀招,居然只在梦中增添另外一只凡蛊。解梦乃是仙道杀招,此等效用未免有些低了。”

方源回想了一下,就算在星宿仙尊的梦境之中,他动用解梦,都是立竿见影的效果。比如直接解开风结草,对梦境的探索有最直接的拓展。不像在这里,施展解梦,只换得区区一只凡蛊。

要知道梦境中,方源的修为还是二转的层次,真元有限得很,催动蛊虫的次数也受到极大的限制。虽然获得了第二只蛊虫,但真元要精打细算,真正发挥的作用并没有表面上那么大。

“难道是因为这处梦境的规模太过于庞大,以至于梦境中的规矩也跟着强大起来,外力影响的效果会降低不少?五百年前世似乎有过一批蛊仙,相当认同这个理论……”

不管方源怎么想,梦境中的第二幕已经展开。

方源发现自己身处在一片迷蒙的灰雾之中,脚下的地形似乎是一片坦途。

“这是一个蛊阵形成的内部空间。既能提供百人乱斗的场地,也能让掌控蛊阵的蛊师们时刻洞察我们的情况么……”

方源正分析着,这时一位少年在雾中靠近自己。

方源念头一动,大喊道:“别动手,朋友,这次参加的人那么多,我们先联盟吧。”

“联盟?你得罪了龙公子,居然还想和其他人联盟?”少年冷笑一声,抬手一记水箭飞射过来。

方源身形一缩,灵巧躲闪,然后飞扑上去。

对面的少年见方源逼近,有些慌了神,连忙又是第二记水箭射来,企图逼退方源。

但方源经验老道,见到少年抬手的动作,就知道这只水箭射向哪里。

方源脚步微微一变,速度更快地冲锋过去,第二只水箭和他擦肩而过。

少年连忙后退,一边退,一边射箭。

方源观察在心,虽然有机会能近身,但方源却故意选择闪避。不一会儿,少年哑火了。

方源笑了笑,慢慢地走上去:“现在的我,完全可以让你失去资格,怎么样?投靠我,我们联手,对付其他人。我可是很有诚意的!”

“哼!”少年冷哼一声,“就我们两个对付其余的九十八人,你也太异想天开了。你得罪了龙公子,居然想要抢他的女人,你还想这一次顺利加入战仙宗,简直是做梦!我宁愿弃权,也不和傻到得罪龙公子!”

方源心中顿时一沉。

如此一来,他岂不是要独自面对其余参加考核的全部少年么?

这个什么龙公子,还真是坑啊。

“难怪前一段时间,南疆的正道蛊仙死伤惨重。这里的梦境,难度极大。比星宿仙尊的梦境,都要难上一筹!”方源缓缓后退,因为他发现自己被包围,大约三十来个少年,组成团,从四面八法包围了上来。

白相洞天。

“我成功了。”白凝冰淡淡地道。

他的目光仍旧冰冷,神色平淡,但他的外貌却是相当的狼狈,整个人都是乌漆墨黑,像是羊肉串放在烤碳上过度地烤制了无数次。

他张口说话的时候,还看到一点点的火星,从咽喉中往外冒。

白相天灵看着他,点了点头:“你居然以体藏火,将惊涛升龙火收走。你知不知道,你这个方法凶险非常,只有极低的可能成功,一旦失败,整个人都会烧成一团灰。”

“不管怎么说,我是成功了。”白凝冰死死支撑着自己,他说话时,吐出每一个字时都非常艰难,全靠着本身坚强的意志力。

白相天灵嘿嘿一笑:“既然成功,那你便是此处洞天的主人了。老朽拜见少主。”

白凝冰点点头,无力再说什么,他觉得自己再说一个字,就要昏倒过去。

然而一旦昏倒的话,按照影无邪交代的话,他就无法控制住体内的惊涛升龙火,会在顷刻间被烧成烟灰。

这时,白相天灵又道:“少主人,你现在的状况非常危险,稍有不慎,就会飞灰湮灭,连魂魄都不会留下来。如果你是炎煌雷泽体还好,但偏偏你是北冥冰魄体。老朽现有一道修行之法,却是相当符合少主人你现在的状态。这种炼蛊之法,是将蛊仙自己当做一份蛊材,结合惊涛升龙火,一起炼制,可以将少主你脱离人族,转变成异人中的一种龙人。”

“龙人?”白凝冰瞪了瞪眼睛,心中大感讶异,暗想,“《人祖传》中没有提及什么龙人啊。异人中有这样一支?”

白凝冰虽然没有说话,但白相天灵似乎知晓他心中的疑惑。

天灵继续道:“人祖传中的确没有记载,因为龙人本就是人族大能龙公制造出来的异人,原先的根本目的是用来延寿的。”

“龙公乃是天庭蛊仙,战力极强,就仿佛那个时代的剑仙薄青。他提出人人如龙的理念,将炼道、变化道集合起来,酿造出龙人这一全新的异人种族。”

白凝冰心中的惊讶又加重几分,他想:“这个龙公到底何许人也?我虽然升仙的时日较短,但见识方面却不是短板,反而比寻常蛊仙还要宽广。龙公既然是薄青似的人物,怎么蛊仙历史上却查无此人,毫无记载呢?”

“该不会是这个白相天灵乱说吧?”(未完待续。)

\end{this_body}

