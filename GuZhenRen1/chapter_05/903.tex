\newsection{南联破阵}    %第九百零七节:南联破阵

\begin{this_body}

万年斗飞车悬停长空,方源望着眼前的大阵,面露微笑。

他再次亲至义天山,这一次定能将这里的梦境一网打尽。

这个时机真的是恰到好处。

一方面,武庸等人率领南联诸多强者,在中洲动手,南疆陷入有史以来最弱的状态。二来,中洲方面的旷世大战也才刚刚开始,对于方源来讲,这恐怕是最后的一段闲暇功夫了。

“和上一世不同,我这一世屡屡占据先机,击败天庭。我重生的秘密广为人知,天庭必定有所针对。他们的手段和布置定然和上一世有着不同,我还是让四域蛊仙打头阵,为我探清虚实才好。”

方源对局面始终把握得当。

到了此时此刻,方源的重生优势已经所剩无几了。前世的记忆对于天庭这样的存在而言,并不能做太多参照。

天庭绝不会蠢笨得没有丝毫变化。

若是方源一头冲在最前方,恐怕就要陷入战争的泥沼中,无法脱身,无法自拔。甚至还可能因此阵亡。

毕竟天庭底蕴深不可测,方源也不敢肯定,他记忆中的仙尊手段就是天庭的全部底牌了。

仙道杀招——炼阵雨。

天空中忽然下起了小雨,淅淅沥沥,迅速渗透大阵。

大阵中自然有镇守的蛊仙。

蛊仙们一片慌乱。

大阵竟抵挡不住,外部边缘已经被方源炼为己有了。

这个大阵虽然是池曲由布置,但方源暗中和池曲由进行梦境交易,多次进入这里,获悉了许多情报。

方源的阵道境界、智道造诣,再加上炼阵雨杀招,正好克制这座大阵。

“方源这个魔头,没有攻打中洲,居然跑回了南疆!”

“怎么办?”

“求援吗?”

“我们的八转蛊仙几乎都被武庸大人带去中洲了啊。”

“那也得将这个消息告诉诸位大人!”

南疆蛊仙们交谈一番,谁也不敢出阵迎战方源,也求援不了,只好催动仙阵,进行一番番变化。

仙阵不断变动,顿时让方源炼化大阵的效率下降数筹。

不过方源反而暗中欢喜。

大阵变动,更能令他看出阵法中蕴含的奥妙。有智慧光晕辅助,参破这些奥妙简直是手到擒来的事情。

中洲,南联蛊仙已经进入了九九连环不绝阵中。

武庸等人接到了这个情报。

“什么?梦境遭遇袭击?”

“是方源这个贼子!”

“他到底有没有大局观?这个时候还对我们下手?!”

南联蛊仙们愤恨不已。

武庸沉默片刻,开口道:“他当然有大局观。”

“武庸大人此言何意?”巴德脸色铁青。方源此举已经严重损害了南疆所有正道的利益!

任谁都清楚,这些梦境的价值极其巨大。并且,它还会随着大时代的来临,越来越重要。

武庸反而笑了笑:“方源若没有大局观,早就对梦境下手了。不会等到我们在中洲出手。或许他还有利用我等,让我们成为炮灰的意思。”

“那我们岂不是为他做了牺牲?”有人开口,语气十分不甘。

“不。”武庸摇头,“宿命蛊一旦修复成功,对于我们的威胁要比方源这个天外之魔大得多了。他重生回来,恐怕上一世我们也像现在这般主动出击了。但没有办法,我们必须动手。这是真正的五域大局。”

“诸位,我们此行之前,我就已经告诫过诸位,或许后方家园会遭受袭击。不管是你的亲人、朋友,还是你的基业面临毁灭和杀戮,在现在都是小事!天庭若有了完整的宿命蛊,我们就将任人宰割,毫无未来!”

武庸说到这里,目光如电,扫视群仙,掷地有声地道:“我欲下令,让镇守大阵的蛊仙们全部撤退,不要做无谓牺牲。我知道这个命令很是冒险,但就算是我武家的大本营被方源摧毁,我也绝不会在这里后撤一步!”

武庸的决意让众仙心中凛然。

众志成城,南联蛊仙再次凝聚成一个拳头,杀进九九连环不绝阵的深处。

武庸的命令传达到南疆这边,镇守大阵的蛊仙们心中都松了一口气,将能带走的仙蛊带走,开始迅速转移。

片刻之后,方源炼化了已成空壳的大阵,不见一位南疆蛊仙的身影。

待他全部拆解了大阵之后,他还得到一只信道蛊虫,蛊虫里是武庸专门写给方源的信。

方源神念一扫,信中内容一览无余。

信中语句寥寥,只阐述一个观点——天庭和宿命的威胁,中洲的强大,四域的渺小。

丝毫不提什么让方源停手的建议。

“这个武庸……”方源不由感叹一声。

之前是他为难自己,施展政治手腕,抵制方源的勒索,让方源损失颇大。而现在同样是武庸,主动下令叫南疆蛊仙放弃防守,并主动撤退,省下了方源很多力气。

若是这些南疆蛊仙严防死守,必定会拖延方源很长一段时间。

方源收走梦境之后,大可以用节省下来的时间,去南疆各处找正道的麻烦。所以武庸这个命令非常大胆和冒险。

但武庸认定方源是有大局观的,绝不会继续在南疆为祸。

这就像方源知道,自己在这边动手,武庸绝不会放弃中洲的战斗赶回家抵御自己。

事实上,方源原本也没有打算去找南疆正道的麻烦。

耗费一番功夫,方源将这里的梦境全部打包收走。

义天山再次重见天日,连番蛊仙大战的痕迹刻印在这里,漫山遍野,毫无生命迹象,一切都早已不是当年的景色了。

方源开始往中洲赶。

他一边赶路,一边将这些梦境转化成纯梦求真体。

接下来的大战中,这些纯梦求真体将是他的得力手段。上一世他的纯梦求真体自爆,是被凤金煌抵挡住了。这一世估计凤金煌也同样掌握了这个能力。

但凤金煌始终只有一个,而且还只是凡人蛊师。方源却占据主动,几大战场任凭选择。

天庭,中央大殿。

龙公、紫薇仙子、秦鼎菱坐镇中枢。

正元老人也被请了进来。

“最新情报,南疆蛊仙破除第一阵,前往第二阵了。方源在南疆义天山出现,攻破大阵,尽取梦境。”紫薇仙子忽道。

龙公沉声道:“方源绝不会继续在南疆作乱了。他必定会立即赶赴中洲参战。看来,我们得要面对梦境了。”

紫薇仙子点头:“真是庆幸,凤金煌开创出了完美的纯梦求真体,能压制方源的这一梦道手段。”

龙公呵呵一笑:“你以为这是侥幸?不,这是宿命在发挥作用!这场战斗,天庭绝不会败,也绝不能败。”

就在这时,紫薇仙子忽然轻咦一声,面露异色:“南疆诸仙已经攻破了第二阵,前往第三阵了。”

“速度这么快?”秦鼎菱也非常惊异。

九九连环不绝阵内,诸仙士气大振。

“池曲由,干得好。”

“真不愧是我南疆唯一的阵道大宗师!”

“有池曲由在,何愁大阵不破?哈哈哈。”

就在刚刚,众仙进入第二子阵中,池曲由便道:“慢。”

随后,他观察片刻,指挥众仙,巧妙且又迅速地破解了大阵。

被众仙夸赞的池曲由却是微微摇头:“实不相瞒,这个大阵不是我看破的,而是我得到了方源的讯息。”

南疆众仙神情一滞。

池曲由继续道:“若凭借我的手段,当然也能迅速看破。但恐怕得用禁术,损害自己的寿元。如此看来,方源绝对是重生回来的。他在上一世恐怕就闯过这个大阵,因此了若指掌。”

众人纷纷点头,赞同这个说法。

却不知道,这个大阵的奥妙其实还是池曲由自己告诉方源的。

上一世,方源从帝君城战场赶来毛脚山,众仙为了破解九九连环不绝阵,想到利用五界大限阵。但没有完整的阵图,并且在敌方大阵中设阵,难上加难。

方源便运用自己的智道、阵道造诣,和池曲由、玄极子联手,共同推衍,迅速推算出了完整的阵图。

在这个过程中,池曲由为了让方源、玄极子更多的了解九九连环不绝阵,便将他们之前破阵的经验如数告知。

既然武庸主动让南疆众仙撤退,展现出了诚意和决心,方源这一次指点他们,也算是投桃报李。

知道了这个情况,武庸的身躯有些复杂,叹息一声道:“看来方源已经赶来中洲了。”

方源了解武庸,武庸也理解方源。

两人之间有着难以言喻的默契。

虽然平时,他们两个恨不得彼此立即去死!

事实上,上一世方源为了破解九九连环不绝阵,向南联索要梦境。

武庸没有犹豫,直接下令放行。

方源传送回了南疆,守护大阵直接开放,任凭他将里面的梦境取走。

只是后来,方源利用这些梦境转化成纯梦求真体,却被凤金煌挡住,并未破解得了九九连环不绝阵。

利用方源的指点,南联蛊仙又破开两阵,效率之高简直骇人听闻。

天庭方面反应过来,紫薇仙子紧皱眉头,急速思索对策的时候,天庭上空忽然泛起一片彩霞。

霞光漫天,八转仙蛊屋劫运坛从中暴射而出!

“天庭,我们来了。”冰塞川长啸,斗志昂扬。

“等你们多时了。”秦鼎菱缓缓站起了身。

\end{this_body}

