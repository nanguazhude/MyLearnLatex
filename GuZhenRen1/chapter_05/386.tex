\newsection{自爱炼成}    %第三百八十六节:自爱炼成

\begin{this_body}

“上极天鹰居然逃脱了,怎么会出现这样的情况?”天庭中,紫薇仙子依靠星宿棋盘,很快就发现了白天中的状况。

她皱起好看的双眉,眼中迅速闪过一丝犀利的光。

“是运道么。”

一个猜测浮现在她的心头。

她想到了八十八角真阳楼。此楼倒塌的真凶就是方源,方源继承巨阳真传的可能性有多少呢?

“可惜我的侦查杀招,只能标注方源等人的位置,不能视察出他们此时的情境。”

紫薇仙子叹息一声,又催动星宿棋盘。

下一刻,星宿棋盘上又变幻出光阴长河当中的一幕。

凤九歌和黄史上人,已经联袂飞至到了一个光阴支流的附近了。

“这处的光阴支流,规模不小,足够你出入了。”黄史上人笑了笑,手指着前方,“待会我便催动仙道杀招,送你出去,切勿抵抗。”

凤九歌点点头,优雅一礼:“劳烦了。”

黄史上人正要开口,忽然神色一变。

“那头虎形太古年兽又追来了吗?”凤九歌也微微色变。

“不是!”黄史上人目不转睛地望向远处,在那里的光阴河面上,忽然飘起了一层浓重的雾气。

“古怪。光阴长河中,何时会起雾了?”凤九歌皱起眉头。

黄史上人却是神情明显地激动起来,眼神中保有巨大的期待:“这般大的雾气,难道是说?”

就在这时,两人看到在这浓雾当中,忽然出现了一抹阴影。

阴影若隐若现,被浓雾遮盖,很不分明。

黄史上人神情更加激动,而凤九歌却是仍旧一头雾水。

很快,雾气的一阵波动,掀开浓雾中的巨大阴影的一角。这让凤九歌看清楚了阴影的面貌。

“好像是一座石头岛?”

“没错,那就是石莲岛!传说中,红莲真传的所在!”黄史上人语气都有些颤抖起来。

尽管红莲魔尊和天庭势不两立,但他是宙道的绝代巅峰,无人能及。作为专修宙道的黄史上人,红莲真传对他的吸引力无以伦比的巨大!

“你留在这里,我去去就来。”黄史上人丢下这句话后,立即电射而去,速度快的不可思议。

然而,整个浓雾像是拥有着灵性,黄史上人越要接近,浓雾就越是后退。

努力了好半天,黄史上人不仅没有缩短他和浓雾之间的距离,反而更加疏远了。

片刻之后,整个浓雾消散,神秘浓雾中的石头岛也不见了踪影。

黄史上人失魂落魄地归来,神情很不好看。

甚至就连天庭中,关注到这一幕的紫薇仙子,也满脸的失望之色。

但很快,她又皱起了眉头。

她陷入了沉思当中:“石莲岛为什么会忽然现身?是因为方源引爆了一条光阴支流吗?但是黄史上人和凤九歌的位置,早已经脱离了那里。还有更关键的问题——怎样才能进入石莲岛?难道必须得拥有春秋蝉这个钥匙吗?”

南疆,武家大本营。

一座超级蛊阵已经建设起来。

武庸站在蛊阵面前,视察良久后,他缓缓点头,露出满意的神色。

他对身旁的一位武家太上长老道:“武碑长老辛苦了。”

“为家族效命,本来就是在下的本份,谈不上辛苦。”武碑长老态度恭敬而且谦虚。

在武家的蛊仙当中,他专修阵道。

不久之前,他还代替方源,镇守超级蛊阵。

逆流河一战后,方源从北原归来,用计又将武碑排挤出去,自己再次混入超级蛊阵。

武碑太上长老反倒因此,躲过了一劫,没有参与到梦境战役中去。

这一战,南疆正道蛊仙损失不小,甚至一度连掏出来建设超级蛊阵的仙蛊,都被天庭卷走了。

不过,在不久之前,武庸追杀围剿方源失败,但却和天庭方面达成了秘密的协议。

武庸放过了凤九歌,而天庭方面,将卷走的仙蛊,都直接交给了武庸。

武庸回到南疆正道的面前,将这些仙蛊都一个不留,全都物归原主。

这一举动,直接将他的声望,推至南疆正道的巅峰!

南疆的各大家族纷纷感怀武庸的仁义,不仅能和天庭交涉,而且交涉成功之后,仙蛊一点都不贪图,全都物归原主,让他们赞佩不已。

再加上,武庸暴露出玉清滴风小竹楼这座八转仙蛊屋后,又在紫血先河阵一战中,展现出了自己强大的战力,使得南疆超级势力都非常忌惮。

所以,尽管没有围杀了影宗余孽,但武家的情况却发生了翻天覆地的改变。

之前侵占武家资源的各大家族,都主动撤离了自己侵占的地盘,甚至还转弯抹角地赔偿了一些东西。

乔家作为武家的附庸势力,他们被巴家、夏家侵吞的地盘,也趁着武家的声威,也被主动归还。

武庸当然不满足这样的战果,归来的这些天,他都在主持大局,和各大超级势力交涉,争取让他们吐出更多的利益。

目前,双方还在扯皮阶段,武家表现得相当强势,但是其他各个家族也不是泥捏的,也都要捍卫自己的利益。

武家这次的损失,说起来真的是很大。

前前后后,武家的蛊仙陨落的不少,其后还被方源拐骗了整整六只仙蛊,以及十万块仙元石。

偏偏这种事情,武家只能烂在肚子里,不能明目张胆地找方源算账。

武庸势要在其他的超级势力身上,弥补了自己这方面的损失,同时他也绝不会忘记方源这个魔道贼子!

眼前的这座超级蛊阵,便是武庸亲自吩咐下去,不仅让武碑太上长老全权负责,甚至还请了池家的蛊仙来帮忙铺设。

“可以了,动手吧。”武庸眼中厉芒一闪即逝,用平淡的语气,对武碑道。

武碑却是心头一颤,他感受到了武庸平静的外表下的怒意和仇恨。

“万万没有想到,武遗海居然是他人假扮的。哼,真是胆大包天,敢来蒙骗武家!现在,就让你知道厉害,在这个世界上,有很多人是你不能得罪的!”武碑心中冷哼,开始催动起整个蛊阵来。

这座仙阵是以地脉为基,方源的命牌蛊、魂灯蛊为线索,动用了近十只仙蛊,难以计数的凡蛊,搭建起来的。

其威能效用,便是和方源为难。每当蛊阵催动起来,就能爆发出无形无质的力量,直接跨越南疆,作用在方源的身上。

这仙阵的威能,自然非常厉害,远超武庸之前,对付方源的那个杀招。

同时,它还很持久。

这是蛊阵的一个优点。

一旦蛊阵建立起来,催动和停歇都很方便了。不像杀招,每一次催动,都要有一个复杂的过程,每一个步骤都要正确。若是不正确,就很可能施展失败,自己遭受反噬。

蛊阵不一样,铺设起来之后,除非被破坏,否则就不担忧反噬的可能了。

“先用一半的威能。”武庸又吩咐了一句。

武碑点点头。

武庸这样做,自然不是宅心仁厚,而是为了方便和方源谈判。

别忘了,武家的仙蛊还在方源的手中。

武庸当然想要拿回来。

就算是方源跑到了西漠,武庸还没有放弃这方面的打算。

“方源那逆流护身印虽然强大,可以逆反一切的攻势,但是可惜!这招太过牵扯精神,并且耗费仙元相当剧烈,方源怎可能时时刻刻一直催动着呢?”

“有了这座蛊阵,我们想要攻击便攻击,想要停止便停下,完全占据了主动。”

“先让他尝一尝苦头,然后再与他谈判。”

武庸心中早已经想好了谋略。

蛊阵徐徐催动起来,眼看着就要发动威能,但就在这时,忽然整个大地猛烈地颤抖起来。

轰隆隆!

整座武仪山都剧烈的震荡着。

“怎么回事?”

“怎么会有地震?是哪个蛊仙在渡劫?”

“不是渡劫,是地脉抖动,波及整个南疆,比渡劫要严重亿倍啊!”

“糟糕,武碑家老!!”

围观的武家蛊仙们纷纷惊呼,想到了不妙之处。

轰——!

仙阵猛地自爆,武碑大吐鲜血,凄厉地惨叫一声后,仰天而倒,当场昏死过去。

北原,琅琊福地。

毛六浑身大汗,汗水已经彻底将他的毛发打湿,凝结成一块,好不狼狈。

“最后一个步骤了!”毛六双目通红,充斥着血丝。

琅琊地灵闻言,立即将手中早已经处理好的仙材,递给毛六。

他虽然贵为太上大长老,但是方源既然已经要求了,门派的规矩又是他琅琊地灵自己订的,他当然要遵守。

所以,给毛六打下手,他并不介意,顶多是对方源颇有腹诽。

吟——!

一声悠长的清脆的响声,从冰炎中传达出来。

就好像是风吹银元的声响。

很快,声音消散,熊熊燃烧的冰焰熄灭下去,露出最中央的那块浑圆一体的球形玄冰。

毛六激动地喘了两口粗气,口中低声呢喃:“出来吧,七转仙蛊自爱啊。”

咔嚓嚓。

玄冰的表面随之出现了阵阵裂痕,然后玄冰崩溃,飞出一只仙蛊来。

七转仙蛊自爱——终于是炼成了!

\end{this_body}

