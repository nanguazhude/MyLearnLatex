\newsection{沈伤的破解}    %第九百一十六节:沈伤的破解

\begin{this_body}

%1
“这里便是中洲帝君城?”叶凡一边跟随着大部队行走在帝君城中,一边回顾四周,细细打量。

%2
他是南疆蛊师,也早已听闻帝君城之名。

%3
这是整个中洲,不,更准确的讲,是整个五域最大的凡人城池。在这里,三转蛊师很是常见,而搁在南疆等地,三转的蛊师通常都是位高权重的家老、族老。

%4
叶凡尚且是第一次来到帝君城。

%5
帝君城果然不愧是帝君城,布局严谨,构架恢弘,没有让叶凡失望。

%6
“只不过,和仙蛊屋一对比起来,帝君城也就相形见绌了。”想到这里,叶凡仰头望天。

%7
天空中,静静地悬停着数座仙蛊屋。

%8
有一座亭阁,小巧精致,悬梁上挂着无数鸟笼,各种鸟类在里面叽叽喳喳,正是天莲派的揽雀阁。

%9
有一座庄园,结构精妙,白玉地砖,灰青亮瓦,寒气森森,似有龙魂呼啸,乃是古魂门的寒螭庄。

%10
还有七层高楼,楼间幔帐飘飞,这是风云府的风满楼。

%11
又有营寨雄固,大旗飘飞,是战仙宗的角连营。

%12
除此之外,幻景园、岳阳宫、日月观……

%13
叶凡拜师陆畏因,从师父那里得知了许多蛊仙的讯息,他能够辨认出来绝大多数的仙蛊屋。

%14
这些仙蛊屋没有帝君城那般巨大,但无疑更加闪耀,时刻引发着全城人的议论和关注。

%15
这个时候,走在他身边的一位年轻人感叹道:“看来传闻是没错了。天地真的要大变,仙人之间也要开战。一个动乱的大时代就要来临了,仙凡的距离再不那么遥远!这就是我等的计划啊,你说是吗?叶兄。”

%16
叶凡看了看说话的这个年轻人,点点头:“你说的不错,洪弟。”

%17
和上一世一样,叶凡和洪易再次在中洲炼蛊大会中相识相知,并且结拜为异姓兄弟。

%18
洪易笑了笑:“此届炼蛊大会可谓盛况空前,前所未有。天上的那些蛊仙就让他们打去,那不是咱们能掺和的事情。”

%19
“哈哈,此言有理,这一次就让我们彻底来较量一下!”叶凡拍拍洪易的肩膀。

%20
两兄弟均是斗志昂扬。

%21
洪易道:“我有一种预感,此次炼蛊大会的最终胜利者将在我俩之间产生。”

%22
沈伤双耳微微一动,叶凡和洪易的对话一字不漏地传入了他的耳中。

%23
这两人虽是凡人,但沈伤却始终将一丝注意力,停留在他们的身上。

%24
“未来五域大乱战,说不定会给这两个年轻人成仙的机会。”沈伤乃是人道准无上,对于人族有着一种高于直觉的感知。

%25
天庭方面,万众一心等等人道杀招已经施展,将自在书生、郑青等人排除出去。

%26
但沈伤却仍旧伪装着混入最终大会的场地,,始终未受到丁点的怀疑。

%27
他毕竟是八转蛊仙,更是人道准无上大宗师!

%28
沈伤时刻侦查周遭,心中悄然兴奋起来。

%29
“我之前估计的并没有错,这座帝君城果然是中洲主要的人脉之一!”

%30
山有山脉,地有地脉,人也有人脉。

%31
“帝君城乃是当年元始仙尊所设,经过这么多年的变迁,位置不断迁移,但至始至终都是地表上中洲人脉的最大的集结之处。”

%32
“这里盛产人杰俊才,正是因为是中洲人族主脉。这里还有天庭历代仙尊布置的人道手段,嗯……我能模糊地察觉到它们的存在。”

%33
方源用自己重生当做理由,力劝沈伤出手,尽全力破解尊者的人道手段。

%34
这个深具挑战性的任务,本身就激发起了沈伤浓郁的雄心斗志。

%35
人道杀招相关的种种蛛丝马迹,对于方源等人而言,是发现不了的。但落到沈伤的眼中,却是有迹可循。

%36
他因此一路伪装,潜入到帝君城中,终于发现了破击尊者杀招的契机。

%37
“天庭收集到的人意,是储藏在天庭当中。”

%38
“那里的防御实在太森严,无法进入天庭,就无法毁灭人意。人意无法摧毁,那么就无法釜底抽薪。”

%39
“不过……在这里动手,也会有上佳的效果。”

%40
“天庭收集人意,帝君城在这个过程中起着主要的作用,相当于最大的中转输送地。”

%41
“破坏了这里,天庭积蓄人意的效率必定大降。”

%42
但沈伤并不打算破坏和毁灭。

%43
他有更大的企图。

%44
他想要顺藤摸瓜,以帝君城为点,反制尊者的人道杀招,甚至从中逆推出尊者人道手段的奥妙,从而偷师!

%45
沈伤艺高人胆大。

%46
他并不满足于破解尊者杀招带来的名誉和声望,若是能从中习得尊者的人道手段,对他而言将是巨大的提升。

%47
“我的人道手段再提升上去,说不定,就能够解除我时常发疯,身冒黑火的隐患了。”

%48
毛脚山战场。

%49
五色烟瘴滚滚不休,南疆蛊仙和天庭主力已经鏖战许久。

%50
海角阁、飞沙阁、无定府已经残破不堪,退居二线紧急休整。

%51
前线上是南疆的四位八转,还有玉清滴风小竹楼、太宇寺这两大八转仙蛊屋。

%52
天庭的主力一位未损,士气始终高涨。

%53
赵山河再次开始酝酿杀招,气势翻腾汹涌。

%54
战部渡连忙大叫:“快挡下他!他又要施展那一招了!”

%55
武庸、翼浩方咬牙,想要冲突进去,但被野樵子、旋空童子拦下。

%56
周雄信、肉鞭仙合力挡住玉清滴风小竹楼。

%57
而朱雀儿、万紫红则一左一右,企图绕过战场,杀向五界大限阵。

%58
武庸等人无奈,只得回防。

%59
他们到底是人数太少,但论防守就已经很艰难了,不及天庭一方战术灵活多变。

%60
赵山河成功地催动杀招,只见他猛地张开大嘴,吞吸周围的五色烟瘴。

%61
大股大股的烟瘴被他吸入肚内,他的肚子以肉眼可见的速度迅速涨大。同时他的脸上,浑身的肌肤都变成了五颜六色。

%62
赵山河竟将五色烟瘴吞吸入腹,造成一个短暂的空白之地!

%63
没有了五色烟瘴的束缚,天庭蛊仙们彻底放开手脚,施展出种种杀招,均都威能恐怖。

%64
武庸等人艰难抵挡,但防线到底是稀疏的,而杀招无孔不入,被防线粗略地筛了筛后,就都轰击在了五界大限阵上。

%65
五界大限阵一阵剧烈颤抖,大量蛊虫死亡,被虚弱的阵灵加紧修补。

%66
战部渡脸色难看。

%67
正是依靠赵山河的食道手段,天庭打开了局面,每一次这样的进攻,都会让五界大限阵损伤惨重。

%68
尽管全力修补,但根本来不及彻底恢复,就会迎来下一波的打击。

%69
如此损伤积累下来,五界大限阵此刻已经岌岌可危。

%70
怎么办?

%71
帝君城。

%72
最终大比已经争分夺秒地展开了。

%73
沈伤窃喜。

%74
他开始炼蛊,趁着炼蛊作为表面的遮掩,他已经暗中酝酿好了杀招。

%75
“终于赶来了!”房睇长驾驭着豆神宫,冲在最前方。

%76
在他身后,是西漠主力大军。

%77
“迎战!”

%78
“把他们打回去!”

%79
“誓死保护帝君城。”

%80
中洲一方的守军呼啸着,迎上西漠仙蛊屋,大战在此刻爆发。

%81
“妙哉!”不忘关注外界动静的沈伤暗乐,中洲守军被牵制,他更加无所顾忌。

%82
就是此刻!

%83
轰。

%84
他身上陡然爆发出仙蛊气势,大喝一声,玄白光辉在瞬间透射大殿,映照到整个帝君城中。

%85
“尊者已死,诸位的人道手段都给破了罢!”沈伤双眼绽射刺眼的精芒。

%86
噗!

%87
下一刻,他大吐一口鲜血,浑身龟裂,肺腑自爆,内脏的碎片和细小的骨渣从浑身上下的伤口里喷射而出。

%88
失败了!

%89
“怎么会?!”沈伤难以置信,他死死瞪着双眼,“我明明……”

%90
他不仅没有破解了尊者的手段,反而提前触发了它们。

%91
这便是人道杀招——众望所归!

%92
漫天白光飞舞,仿佛大雪倾天,汇集到中洲蛊仙的身上,形成洁白光晕。

%93
人道杀招——人中豪杰!

%94
附着在中洲蛊仙身上的白色光晕,开始迅速削减,而蛊仙本身的气势则随之节节暴涨。

%95
嗷!

%96
中洲大军怒吼,士气狂飙,杀向西漠一方。

%97
“糟糕!”房睇长也不禁变色。

\end{this_body}

