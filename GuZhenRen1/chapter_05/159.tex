\newsection{天人道法师论势}    %第一百五十九节:天人道法师论势

\begin{this_body}

%1
数天之前。

%2
长生天。

%3
这是巨阳仙尊的九转洞天!金色的光辉,笼罩四荒,沁人的青晕,照射八极。

%4
四荒、八极构成整个长生天的格局。就如同九天五地对于至尊仙窍一样。

%5
而在这四荒之心,八极中央,一座仙蛊屋绽射着玄白苍茫的光辉,玄光直朝九霄。而整个长生天都与之相合,发出嗡嗡的轰鸣声响。

%6
八转仙蛊屋劫运坛。

%7
三层圆坛,望柱重重,白玉栏杆,霞光映照。

%8
长生天中的八位蛊仙,分别站立在劫运坛的九个角落,竭尽全力,聚精会神地操纵着劫运坛的运转。

%9
劫运坛的中央,更是时而雷光闪烁,电蛟飞舞,时而气运蒸腾,云升雾罩。

%10
一个蛊仙的身影,盘坐在地上,抗衡着劫运坛的打击力量。

%11
在他的身边,环绕着一个圆环,圆环飞舞,罩着他的身躯,呈现五彩之色。

%12
“五行大法师,你就算是有八转修为,落到劫运坛中,就没有翻盘的希望!我劝你还是乖乖投降,或许等到四荒仙人苏醒,你还有一线存活的希望。”八位蛊仙中的首领天极子冷喝道。

%13
原来被困在劫运坛中心的蛊仙,正是北原五大八转的存在之一五行大法师!

%14
五行大法师哈哈大笑:“八转仙蛊屋劫运坛,也不过如此!你们八极子合力操纵,镇压我这么长的时间,甚至勾连了长生天的力量,来企图剿灭我。但是连我的第一重五行环都没有破坏得掉。天极子,你居然还好意思开口劝降老夫?”

%15
天极子冷笑:“那是你不知道劫运坛的妙用!它的真正厉害之处,是劫运互转,劫转运,运转劫。劫运相联,生生不息。这就是天地运的奥妙!你区区散修,如何懂得?”

%16
听到天地运三个字。五行大法师脸上的得意神情,顿时收敛了起来。

%17
巨阳仙尊开创运道,一生修行总结起来,就是己运、众生运、天地运。

%18
其中天地运乃是巨阳仙尊晚年所创。恢弘雄阔,洞悉天地奥秘。五行大法师陷于瓶颈,无法自我突破,便潜入到长生天中,企图盗取天地运的真传。

%19
没有想到。长生天中的四荒仙人虽然皆是沉眠,但防御力量仍旧超越了五行大法师之前的估算。

%20
一番缠斗之后,五行大法师深陷劫运坛中,被八极子联手镇压住,不得脱困。

%21
“巨阳仙尊生前有三座八转仙蛊屋,劫运坛我早已耳闻。但劫运互转,究竟什么是劫运互转?”五行大法师沉吟片刻,忽然问道。

%22
八极子面面相觑,纷纷讶然。

%23
但很快,他们就反应过来。

%24
天极子心中不禁感慨:“早就听闻这位五行大法师。求知欲极其旺盛,凡人时期就多次伪装身份,四处偷师。成为蛊仙之后,为了获取一些修行的经验和指导,不惜为其他蛊仙做牛做马。这一次,他居然在阵前向敌人请教。也真是个痴人!不过,无知者无畏,告诉他真相,也能狠狠打击他的士气和斗志!”

%25
想到这里,天极子便开口道:“如今已成定局。告诉你也无妨。这劫运坛被巨阳先祖刻意留在长生天中,就是为了护持长生天的安全。劫运互转,便是能将灾劫转为佳运。这一次你来犯我长生天,便是我长生天的人劫。我催动劫运坛。便能将此次的劫难,转化为佳运,转祸为福。让你发光发热,为我长生天永存不灭做出自己的贡献。”

%26
“妙哉,妙哉!”五行大法师听得双眼精芒烁烁,他又问道。“人劫如此,那若是天地灾劫,又当如何?”

%27
“自然是照旧转化,让天灾地劫毫无杀伤,变成滋养我长生天的甘霖雨露。”天极子傲然答道。

%28
五相**师瞪大双眼:“长生天乃是九转洞天,又从未吸纳过太古九天碎片。靠着劫运坛,就能让它渡过一次次灾劫吗?”

%29
“哈哈哈!你已经是八转蛊仙,居然连灾劫的缘由都不清楚?”天极子不屑地笑出声来。

%30
五行大法师一点也不气恼,反而更加虚心请教:“还望天极子能告知一二。”

%31
天极子察觉五行大法师真心请教,心中的不屑反而化为乌有,转为一丝丝的钦佩。

%32
他叹息一声,开口答道:“人为什么会有灾劫?事实上,不只是人才有灾劫,荒兽荒植亦有。就比如千蛇阴嬛树,便有劫电关照。”

%33
“灾劫的起源,便是天道。天道损有余而补不足,最讲究万物平衡,循环往复。任何一个破坏平衡和循环的存在,都会遭受灾劫照顾。”

%34
“只是人类乃是万物之灵,最具智慧。因此比其他猛兽植株,更具威胁。蛊仙更是营造自己的小天地,另辟炉灶,惹来天道的忌讳和打击。”

%35
“什么是劫?什么是运?”天极子忽然问道。

%36
五行大法师福至心灵,脱口而出,回答道:“我明白了。天道损有余而补不足,所谓的劫,便是损有余。所谓的运,便是补不足。”

%37
天极子微微扬眉:“你答对了,答案正是如此!而我巨阳先祖,才情绝代,开创运道。正是针对天道的运转。己运真传,是单纯的补益自己。众生运真传,是损他命而补自己。天地运真传,是损天地而补自己。”

%38
“原来如此。”五行大法师感慨万千,“难怪巨阳仙尊一生,都没有入主天庭。”

%39
天庭是崇尚天道,替天行道。

%40
而巨阳仙尊的运道,本质上却是篡夺天地的权柄。天道运转,认定你这个人太强,破坏平衡,所以降下灾劫打压你。巨阳仙尊的运道,便是避免灾劫,到了天地运的阶段,直接转变灾劫为福缘。天地打压我,要损我有余,补他不足。我就自己出手,来损你天地来补自己!

%41
正是道不同,不相为谋!

%42
“天道无法利用灾劫,来摧毁长生天,便用人劫来破坏。”天极子继续道,“而你,五行大法师,你便是受到天意的潜移默化,终究转变成了前来破坏我长生天的人劫!可惜你堂堂八转,被天意影响,还不自知。”

%43
“哈哈哈,受教了。”五行大法师发出豪迈的笑声,“就算天意影响,那又如何?因为这也是我的本意!而我始终相信,人定胜天。多么浩大的天意,也不及人的意志。纵观历史长河,人族从渺小孤零,被四处欺辱和排挤,到如今成为天地霸主,将一切异人、万物踩在脚下!整个人族才是最破坏平衡,可是天道能打压得了吗?人族中出了十位九转尊者,人族的蛊仙层出不穷,纵横天地,摘星拿月,移山倒海,天道能奈何吗?”

%44
天极子大笑,其余七子也或多或少浮现喜色。

%45
这种喜悦,是碰到知音,遇到了自己认可的见解而浮现出的情怀。

%46
天极子笑声收敛下去:“看来五行大法师你,也时常阅览《人祖传》。”

%47
“不错。《人祖传》中记载着人祖的人道真传!每一个修行到深厚境界的蛊仙,都会尝试从中受益。可惜,我才智浅薄,悟性不高,只能读出人道胜过天道的道理来。我曾听闻,巨阳仙尊自创己运之后,便原本陷入瓶颈。也是阅览了《人祖传》之后,得到启发,才开创了众生运。随后一发不可收拾,登临仙尊之位,再创天地运。”五行大法师继续道。

%48
天极子点头:“的确如此。”

%49
五行大法师又道:“我乃北原中人,土生土长。虽然不是黄金血脉,却对巨阳仙尊仰慕至极。人道可胜天道,巨阳仙尊的运道深得人道的精髓!人道的大势,是无法更改的,是无法违逆的。就算是天道,也不是对手!这些见解,诸位赞同吗?”

%50
“自然是如此。”天极子傲气地答道。

%51
五行大法师轻笑一声:“人道大势,诸位清楚。那么北原的大势,诸位看清楚了吗?”

%52
天极子面色微变:“你想说什么?”

%53
五行大法师朗声,侃侃而谈道:“当今北原的人道中大势,便是散魔崛起,正道退避。八十八角真阳楼倒塌,让许多散仙和魔道都掌握了力量。百足天君创建了百足一家,更是让正道中有了不同的血脉。”

%54
“巨阳仙尊的家天下,的确是宏图伟业,但未免有些狭隘。北原终究不是他的,而是所有北原人的。人死如灯灭,巨阳仙尊的影响笼罩了北原这么久,如此多年!也是时候消散了!”

%55
“一派胡言!!!”天极子大怒,其余七极子亦都纷纷喝骂起来。

%56
五行大法师反而很平和:“你们怒了,生气,正是因为你们心中忌惮。巨阳仙尊、黄金家族,笼罩了北原,称王作霸,正所谓盛极而衰。就像中洲十大古派,看似强盛,其实里面已经暗生了许多蛀虫和纨绔。而散修和魔道呢?你看看北原明面上的八转蛊仙中,有多少是黄金血脉?”

%57
“雪胡老祖是魔道,我乃散修,百足天君由散修转为正道,凤仙太子虽然是宫家蛊仙,但亦并非黄金血脉。唯有药皇一人,撑着黄金各族的颜面。”

\end{this_body}

