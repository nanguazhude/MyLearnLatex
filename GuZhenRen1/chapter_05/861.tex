\newsection{天庭议事}    %第八百六十五节:天庭议事

\begin{this_body}

天庭。

银白苍穹,仙殿林立。

议事大厅中,十多位蛊仙已经济济一堂。

最高层坐着龙公,其次是紫薇仙子、秦鼎菱。之后,白沧水、从严、车尾、袁琼都、周雄信等人亲自到场。

除此之外,便是顾六如、万紫红、凤仙太子、星野望、君神光等人的意志。这些人因为任务在身,不能亲自来参与此次天庭的议事。

每隔一段时间,天庭诸多成员之间,都会举办一场议事,相互讨论,各抒己见,达成某些共识。

而在平常时期,天庭一般是由智道蛊仙总揽政务。如今的天庭,龙公是一面精神旗帜,执掌权利的乃是紫薇仙子。而在她之前的上一任,则是监天塔主。

“袁琼都,第二空窍蛊的研究进展如何?”龙公询问。

袁琼都答道:“有紫薇大人协助,我已经勘破了一些秘密,我们可以炼制出一至五转的第二空窍凡蛊了。”

较真的话,天庭的炼道底蕴比琅琊派还要高深。

历史上的炼道三大无上大宗师,长毛老祖只是其中之一,其余的两位真传都几乎被天庭收录当中。

而袁琼都乃是当今天庭炼道的主持者,富有卓绝的天赋才情,上一世就是依靠他令宿命蛊修复成功的。

再加上紫薇仙子的推算能力,这三方面叠加起来,破解方源的第二空窍凡蛊并不困难。

听到这个消息,在场诸仙中许多人都面露喜色,交谈起来。

“这第二空窍蛊我已听闻,是个好东西啊。”

“不错,有了第二个空窍,对于蛊师而言,战力成倍增长。”

“若是能够研发出仙蛊方来,对于蛊仙而言,也是意义重大!”

“从这点上,我们都该感谢方源才是。哈哈哈。”

紫薇仙子声音平淡,插言道:“方源岂是无智之人?他向南疆正道抛出第二空窍蛊的时候,定然是早已预料到第二空窍蛊方的会被破解。第二空窍的确意义非凡,然而蛊师若有第二空窍,修行的时间、精力也要陡增一倍。对于修行资源的需求,也要相应膨胀一倍。”

“原来如此。”在场的蛊仙都是八转精英,紫薇仙子只是轻轻提点了一下,他们就立即明白过来。

“方源是故意如此的。”

“他抛出第二空窍蛊,一来是方便自身敲诈南疆正道,二来是为将来的五域乱战推波助澜。”

“没错。界壁消失,五域合一,蛊师、蛊仙的数量暂时都没有变化,但是第二空窍蛊必定会迅速风靡。到那时,对于资源的需求要暴涨一倍,这个缺口怎么填补?唯有向四周征讨,用战争来攫取资源呐。”

“这魔头的用心真是毒辣险恶,偏偏这个香饵,我们还得主动的吞下去。”

不吞还不行。

天庭是从南疆正道那边得到的第二空窍蛊。南疆正道本身也会研究,其余三域也绝不会坐视不管。

第二空窍蛊这种东西,一旦流传出来,必定会风靡天下。

就像当年血海老祖开创血道一样,哪怕整个五域正道都在禁绝血道的修行,仍旧有血道的蛊师、蛊仙层出不穷。

“目前无法研炼出第二空窍仙蛊吗?”一直沉默的龙公,蓦地开口。

他话语稀少,但往往一针见血,这一次也不例外。

袁琼都苦笑摇头:“距离破解出仙蛊方还有一段距离。若是能有几位和我相同炼道境界的仙友同道帮助,或许我能够尽快地破解出来。但是目前,我实在是精力有限,分身乏术。”

袁琼都非常辛苦。

早些时候,方源侵吞了琅琊派,他要改置宿命蛊的修复大阵。原先的那座大阵,因为参考了炼炉,为了保险起见是不能用的。

之后,方源俘虏南疆正道,袁琼都又负责抢炼定空蛊。

随后,方源在光阴长河大胜天庭,天庭的宙道仙蛊屋破碎。袁琼都又要疯狂炼蛊,抢炼出一些宙道仙蛊来。

最近,秦鼎菱苏醒,交给紫薇仙子一沓运道仙蛊方,还是袁琼都负责炼蛊。

炼蛊本身就是一件凶险艰难的事情,方源还屡屡抢占先机,袁琼都大多数炼蛊都遭受惨败。

这些天来,他因炼蛊而受伤的次数已经多得数不过来。若非天庭有着极佳的疗伤手段,他早就阵亡了。

“接下来,袁琼都仙友还要准备宿命蛊的修复,恐怕在宿命蛊彻底修复之前,第二空窍仙蛊方是无法破解出来的。”紫薇仙子遗憾地宣布道。

龙公点点头,没有再说话。

秦鼎菱叹息一声:“没有第二空窍的仙蛊方,就无法彻底令南疆正道摆脱方源的威胁和敲诈。”

“哼,这群南疆蛊仙实在是给正道丢脸,被方源勒索到这种地步!”大厅中,星野望语气不悦。

“事实上,南疆正道也有不少起色。他们大肆布置的烽火台,是一座非常值得我们警惕的仙蛊屋。”

“南疆正道正在摆脱方源,目前来看,只有巴家、夏家等几家因为渴求第二空窍仙蛊,仍旧为方源所制。而武家、柴家等等家族却是在积极反抗。他们和我天庭合作,宝黄天中已经大规模地钳制方源的各种生意。”

“我要提醒诸位,南疆蛊仙界中有一个值得注意的潜在势力,这就是道德乐土。这是乐土仙尊当年布置的真传,有不少的蛊仙,包含菇人蛊仙,当中最强之人便是陆畏因。”

“乐土仙尊虽然未入驻天庭,但和我们之间一直关系友善。然而道德乐土这块的态度究竟如何?我们还需要试验一二。”

“这点我已考虑到。”紫薇仙子开口,“再过一段时间,我便会亲自动手,前往道德乐土,求借仙蛊。”

天庭的确需要这只仙蛊,同时这也是对道德乐土的一种试探。

若是陆畏因方面态度良好,天庭将会进一步拉拢道德乐土,为自己所用。

南疆的事情商议好,天庭诸仙便开始商量西漠。

因为方源的谋算,西漠政局大变。房家成功掌握豆神宫,“击败”天庭人马,令天庭一直以来制衡西漠的战略遭受挫败。

“房家本身的确是有崛起的资本的。但这一次崛起,起到关键的作用,还是这个男人——房家的太上二长老房睇长!”

“这个人野心勃勃,是个枭雄人物。他已经联合了一批西漠正道势力,若非此次我们暗中挑拨,阻击成功,房家崛起之势将不可遏制了。”

“眼下陈衣已死,我们难以拿出一位上得了台面的元莲传人,豆神宫如何回收?”

“我建议,最好将房睇长暗杀掉!”

“难,这个人狡诈得很,又非常谨慎。有了豆神宫之后,几乎缩在里面不出来。”

“房家的太上大长老是房功,他对房睇长的感观如何呢?”

“房功和房睇长还是关系紧密,两人之间没有什么矛盾。我们暂时还无法下手。”

……

天庭蛊仙们商议了一阵,一致决定增兵西漠!

天庭虽然折损了一位卫风,但是大部队仍在,主要力量并未折损。

当初天庭之所以铺设军力,是方源逃窜的时候,利用了这里的天然光阴支流。天庭为了堵截,在光阴支流的位置铺设了仙阵。而后又发现千变老祖的虚弱,所以才逐渐增兵。

天庭对付方源、千变老祖的计划,虽然严重受挫,但并没有就此放弃。

他们商讨之后,一致决定,在关键时刻派遣光阴长河中的一部分蛊仙以及仙蛊屋,通过天然光阴支流,迅速支援。

而对付的主要目标,这一次也不是千变老祖了,而是房家!

千变老祖乃是散仙,而房家却是正道,正在逐渐地影响整个西漠正道。若是任其发展,西漠正道对于天庭的威胁会越来越大。

房家夺走了豆神宫,这点就决定了房家和天庭之间是无法调和的。

所以,天庭准备着重对付房家,也是合理的举措。

西漠之后,便是东海。

“东海政局也有大变的迹象,气海老祖举办酒宴,广邀东海蛊仙,究竟意欲何为?”

“不久前,气海老祖和龙公达成约定,不再相助方源。此人似乎是中立,但不可放松警惕。”

“这场东海盛宴,已经有我们的耳目。届时气海老祖有什么动作,我们都会迅速得知。”

龙公这时候再次开口:“气海老祖实力非凡,并不下于我。我个人意见,还是要积极拉拢此人。至少在宿命蛊修复之前,这样做能为天庭减少威胁。”

龙公的话,让天庭蛊仙们都有些沉闷起来。

东海出了一位气海老祖,居然战力匹敌龙公,这点让天庭蛊仙们都有些接受不了。

东海的事情商量完毕,轮到北原。

北原的长生天的任何动向,都是天庭时刻关注的事情。

四大域中,天庭最顾忌的便是北原。

北原政局一直在被天庭影响着,尤其是雪胡老祖和天庭的合作,是一个巨大的成果。

长生天想要整合北原蛊仙界,雪胡老祖就是最大的钉子。

北原事务讨论完毕,龙公提到了光阴长河。

“而今,我们有五座宙道仙蛊屋,却始终探索不出红莲石岛的丝毫踪迹,是否方法可以改进呢?”

然而,天庭众仙几乎都没有什么好办法,毕竟这些石莲岛都是红莲魔尊的亲手布置。

秦鼎菱也摇头:“单就运道方面,我们已经有凤九歌这样的护道人坐镇。他的运势强盛,连他都不行的话,我也无能为力。”

龙公沉默,微感失望。

宿命蛊的修复,是天庭最大的计划。

在龙公心中,什么武庸、长生天、方源、龙宫,都不及红莲魔尊的威胁。

但没有办法,天庭已经竭尽全力,镇河锁莲大阵就是一个很大的成果。可惜距离搜寻到红莲石岛,还有着一段距离。

议事大厅中一片沉默。

就在这时,紫薇仙子双眼猛地一亮,随后龙公、秦鼎菱的脸上也涌起惊异之色。

最后,剩下的天庭蛊仙们都纷纷激动起来。

“这个消息来得很突然!”

“光阴长河中显现出了一座红莲石岛。”

“顾六如已经证实,红莲石岛货真价实,绝对做不了假。”

“一道红莲真传终于被我们发现了!”

“只要我们得到其中一道,剩下来的红莲真传恐怕也能顺藤摸瓜出来呢。”

龙公点头:“这是好事,务必要夺得真传。另外……小心方源。”

天庭蛊仙们脸色微肃。

“不错,这番动静恐怕隐瞒不了他多久的。”

“到时候,他必定会出现。”

“尽快布置镇河锁莲大阵,我们更要增援。方源要来,就让他丢下性命!”

\end{this_body}

