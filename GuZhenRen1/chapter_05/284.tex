\newsection{喜获仙蛊}    %第二百八十四节:喜获仙蛊

\begin{this_body}

武家的七转蛊仙武雨伯,败于炎荒仙人之手。想看的书几乎都有啊,比一般的小说网站要稳定很多更新还快,全文字的没有广告。]

这件事情,在夏家有心的宣传之下,很快就被整个南疆蛊仙界得知,一时间人心浮动!

“武雨伯乃是七转强者,武家著名的战力,居然会败给了炎荒仙人?”

“炎荒仙人本来也战力不低,这一次更是得到了夏家资助,战胜武雨伯并非不可思议之事啊。”

“武雨伯心高气傲,这一次栽在手下败将手中,令人诧异。”

“关键是炎荒仙人居然能破掉了武雨伯的招牌杀招,这就厉害了。恐怕是夏家的夏流佩出手推算。”

夏家亦有一位智道蛊仙,在南疆蛊仙界中堪称数一数二的推演能手。

正是夏流佩。

“这一次,武家太丢脸了。武雨伯不仅败给了炎荒仙人,更差点丢了性命。若非武遗海出手破开战场杀招,武雨伯恐怕还要丢掉性命啊。”

“不会吧?”

“货真价实。已经有各种蛊虫流传开来,里面的影像重现了当初一幕。”

“武家居然因此失去了拜月碗,呵呵,真是个大笑话。”

“追根溯源的话,拜月碗原本就是武家巧取豪夺过来。这一次,也是武家罪有应得!”

“哈哈,武独秀一去,武家果然难有第一威名了。”

输了拜月碗,对于家大业大,底蕴雄厚的武家而言,完全称不上什么伤筋动骨。

但是,方源主动出手,撞破战场杀招,救出武雨伯的事情,却是让武家陷入被动局面。

这就是正道和魔道的差异所在。

方源出手,救出武雨伯,绝对是违背了比试约定的事情,让这场比试有失公允,丢了武家的脸面。

关键是,这件事情证据确凿,不容抵赖。

为炎荒仙人掠阵的,乃是夏家智道蛊仙夏擎苍。他做事自然滴水不漏,早就用信道仙蛊记录了当时的全部影像。既然是针对武家而来,夏家当然做了充分准备。

武家不得不承认此事,方源带着武雨伯归来之后,武庸便下令惩处方源,因其违背约定插手比斗,罚他面壁以及仙元石若干。

方源救下武雨伯,本是有功的。对于武家而言,失去了面子,但是救下了一位七转蛊仙,孰轻孰重,当然是后者更加重要!

方源有功劳,却反而被武庸惩处。mianhuatang.cc [棉花糖小说网]

不过他心中却无怨气,反而捏着鼻子认罚,回到武家大本营之后,便缩在他的山峰住处里,足不出户。

拜月碗的事件,还在发酵,影响不断扩大。

武雨伯生性高傲,这一次败给了炎荒仙人,羞愤难当,回归不久之后,当即宣布闭关苦修,以期将来再斗,寻回场子。

武家虽然输了拜月碗山谷,武庸也惩处了坏了规矩的方源,但武家的名望因为此事,不可避免的下降许多。

更关键的是,夏家此举无疑是给其他超级势力,带来了启发。

正道和魔道不同。

正道有正道的游戏规则。

夏家充分利用了这个游戏规则,让武家纵然失去了拜月碗,也无话可说。

夏家在南疆,甚至是在整个五域,都比较特殊。因为这个超级势力以光道、智道流派为主。智道蛊仙的数量虽然稀少,但是夏家却总会有四五位的智道蛊仙,并且其中不乏强者。

整个南疆的智道蛊仙中,夏姓始终是第一大姓,占据了很大的份额。

武家被夏家阴了一记,也不奇怪。

其余势力看到夏家的成功,自然蠢蠢欲动,想要效仿。

武独秀在世时,武家如日中天,侵吞他人资源,扩张疆域半途,拜月碗这种事情很多。炎黄仙人这样的仇敌,也为数不少。

这正是其他势力可以利用,来对付武家的地方。

种种情报传来,武庸立即感到不妙,一股风雨向着武家扑来。

果然,拜月碗事件之后,就有罗家向武家发难,运用的理由非常正当,和夏家极为相似。

随后,又有池家、柴家,企图染指武家的灵感蚕洞、春阳山脉。

武庸坐镇中枢,调兵遣将,奈何武家疆域太广,人手不足,暂时维持一个不败局面。

武家陷入麻烦当中,武庸为此焦头烂额,身为武庸之弟的“武遗海”却是过着舒坦的小日子。

至尊仙窍之中。

“放过我,求求你放过我。啊!”马鸿运的魂魄发出凄厉的惨嚎。

但是方源不管不顾,仍旧不断搜魂。

马鸿运之魂难以承受这样的折磨,很快变得虚弱不堪。

胆识蛊。

方源一只蛊虫下去,立即将马鸿运的魂魄治愈好,然后继续搜魂。

“你这个恶魔!!”

“杀了我,杀了我……”

直接作用在魂魄上的剧烈痛楚,让马鸿运难以承受,他不断诅咒、痛骂以及哭喊求饶,但方源始终充耳未闻。

片刻之后,他终于停手。

马鸿运的魂魄已经变得极其淡薄,仿佛是一团水雾,介于消失的边缘。

但下一刻,方源捏碎一只胆识蛊,又将马鸿运的魂魄治好。

“终于将马鸿运所铭记的运道真传,都搜刮出来了。”

“不过赵怜云未死,这个马鸿运的魂魄,还有继续俘虏,关押下去的价值。呵呵呵。”

方源在逆流河中,夹死了马鸿运之后,得到了他的魂魄。

这些天来,他都在搜魂马鸿运。

确信自己已经彻底搜刮,马鸿运对自己而言,再无丝毫秘密可言。

方源因此获得了一部分的众生运真传。

己运、众生运、天地运,是巨阳仙尊的毕生三大真传。己运真传在琅琊福地,众生运真传保存在八十八角真阳楼里,天地运真传则在长生天。

随着王庭福地摧毁,八十八角真阳楼倒塌,众生运真传也随之毁灭。

不过在当初,马鸿运和赵怜云曾经得到众生运真传的庇佑,因此二人各自获悉了一部分的真传内容。

一路辗转,方源搜魂成功之后,这一部分残缺的众生运真传,也就流落到了方源的手中。

“幸好雪胡老祖要炼制鸿运齐天仙蛊,所以要求健健康康的马鸿运,所以没有在他的魂魄上动什么手脚。”

“马鸿运脱离王庭福地时,是被影宗的秦百胜俘虏。他所得知的众生运真传内容,影宗也应当是知晓的。”

方源分析了一下,这部分残缺的真传内容中,最具有价值的就是鸿运齐天仙蛊方!

马鸿运正好记得这一部分内容。

这可是众生运真传中,最为精髓的部分。

不过想想也不奇怪。

正是因为马鸿运得知这个仙蛊方,导致了雪胡老祖在买下他之后,也获知了这记仙蛊方,从而导致了雪胡老祖产生炼制鸿运齐天仙蛊的念头。

“鸿运齐天仙蛊……”方源苦笑。

按照这个仙蛊方的内容来看,让他耗费全部身家,也筹集不到仙蛊方的十分之一。

除非是有马鸿运这样的活生生的蛊材,能够免除许多珍贵至极的炼蛊材料,能让方源勉强看到一线希望。

方源哪敢炼什么鸿运齐天仙蛊。

这可是八转仙蛊!

冒然炼制八转仙蛊,会怎样?雪胡老祖的下场,就在眼前。

这位北原战力第一的八转蛊仙,如今的日子一定很不好过。不仅竹篮打水一场空,炼制鸿运齐天仙蛊失败,而且还失去了逆流河、大雪山福地,就连爱妻万寿娘子都受伤了。

“我现在虽然家大业大,至尊仙窍底蕴雄厚,可以比拟一般的八转蛊仙。但是手头上的仙元石、红枣仙元都储备稀少。更关键的是,富余资源几乎很少。”

这一次,方源追杀影无邪等人,从南疆到东海,再到北原,仙元消耗非常巨大。更别提逆流河中,参悟仙道杀招。逆流护身印乃是所有仙道杀招当中,最耗费仙元的那个。

“至尊仙窍原本产出丰厚,但是由于大大延缓了时间,拖延灾劫降临,导致资源产出的效率极大降低。”

“虽然仍旧在赚取利润,但是本身消耗也不小。尤其是喂养仙蛊方面,慧剑仙蛊、态度仙蛊的喂养问题,仍旧没有解决,仍旧需要继续投入资金。”

方源要解决喂养难题,又要储备仙元,以防不测,这就导致他目前处境比较尴尬。

前不久,他因为违背规矩,抢救下武雨伯,更被罚了一笔仙元石,颇有些雪上加霜的味道。

“延缓了时间流速,导致仙窍中宙道资源剧减。如此一来的话,只好扩大生产,以量取胜。”

“不过,选择何种资源扩大规模,还需谨慎思考。有一些资源,市场已经饱和,扩大生产规模,只会搬起石头砸自己的脚。”

方源正思考着这些问题时,一股武雨伯的意志,夹裹着一只仙蛊,飞临到他的山峰住处。

“救命之恩,不敢忘怀。这只六转金刚念仙蛊,便送给武遗海大人,算是酬谢罢。”武雨伯的这股意志,竟是主动送给方源一只仙蛊来的。

金刚念仙蛊!

很显然,这是一只智道仙蛊。

拜月碗事件,方源也成了受益者他竟获得了一只仙蛊!

虽然只是六转,但仙蛊唯一,价值连城。

备注:呼……最近这些天终于把细纲完善了,明天开始双更!敬请大家多多订阅正版,你的每一份正版支持,都是我写下去的强劲动力!真人在此拜谢了!(\~{}\^{}\~{})<!--80txt.com-ouoou-->

------------

\end{this_body}

