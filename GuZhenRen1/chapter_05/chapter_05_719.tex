\newsection{元始天庭!}    %第七百二十二节:元始天庭!

\begin{this_body}

%1
山洞中的气氛越来越压抑,一位年轻的人族蛊仙勉强笑道:“仙尊大人呐,您现在才不过数百岁而已,还有数千年,甚至上万年的岁月呢。现在考虑这些,是不是太早了点啊?”

%2
元始仙尊的面色顿时肃然,他用平静地目光盯着说话的年轻蛊仙,后者支撑不了几个呼吸,低下头去。

%3
元始仙尊用低沉的声调道:“人无远虑必有近忧。若是不事先考虑,做出预防,待我死后,我们的种种牺牲将变得毫无意义,我们千辛万苦奋斗努力得到的成果,也都会分崩离析,。”

%4
“别的不说,一旦我死去,异人反攻,你们当中谁能一肩挑起重担?”元始仙尊责问。

%5
人族蛊仙面面相觑,无人应承,更无人出声。

%6
人族虽然崛起,但依靠的是元始仙尊一个人,其余的人,不管是蛊仙等级,还是蛊仙数量都远远不及异人种族。

%7
元始仙尊深深叹息:“中洲太大了,我们人族的蛊仙有多少,异人的蛊仙有多少?不提蛊仙,说说凡人。整个人族的人口又有多少?,他们的人口有多少?百万倍的差距总该有的吧。”

%8
“就算在我们有生之年,制霸中洲,统治中洲,还有其他的四域呢?”

%9
“现如今,我的实力曝光,异人深知不敌,已采取躲避的措施。我们根本没有有效的手段,来找到他们的确切位置。尤其是那些异人蛊仙的洞天福地,一旦寄托外界,就算我等走到面前,也万万发觉不了啊。”

%10
元始仙尊的意思,诸仙彻底明白了。

%11
就算人族已经崛起,但依靠的只是元始仙尊一人。而元始仙尊寿命有限,终有一天会死去。

%12
而在他生还的时段,人族想要统一中洲,也是非常艰难的。异人难以寻找,他们躲避起来,仍旧能繁衍生息。

%13
元始仙尊无敌世间,却没有有效的手段来挖掘出他们的位置。

%14
人族蛊仙中某位老者,语气沉重地道:“所以,依仙尊大人您的意思,对于我们人族而言,长久之计便是舍弃家族,组建师门吗?”

%15
元始仙尊神情严肃地点头:“是的。”

%16
“我们如何能够战胜异人,给整个人族带来光明的前景?单靠我一个人是不行的。”

%17
“异人打不过大不了都躲起来,因此就算我无敌天下,也只是带领中洲的人族独立起来,在异族的重重包围下,站稳脚跟罢了。”

%18
“真正要令人族崛起,必须得依靠我们人族自己。我们必须扩大人口,繁衍生息,同时栽培出尽量多的蛊仙。只要我们人族拥有比异族更多,更强的蛊仙,我们种族的前景才会光明起来。”

%19
“若是我们采用家族的制度,和异人又有什么区别呢?我们现在已经是非常的落后,再采取家族制度,是万万不能超越他们的。”

%20
“唯有创建门派,积极地挖掘一切人才,唯才是举,不惜成本,抛弃成见地去栽培他们。我们才能有赶超异人家族的希望啊!”

%21
元始仙尊的一番话,再次让人族蛊仙们陷入沉默之中。

%22
许多人的神色已经平和下来,他们知道元始仙尊的话是有道理的。但仍旧有一些蛊仙,轻轻皱着眉头,难以接受。

%23
一位蛊仙问道:“难道门派制度,就能够让我们始终精诚合作吗?一定就能唯才是举,没有内部的派系,没有打压排挤吗?”

%24
“当然不是。”元始仙尊摇头,“只要是人搭建起来的组织,就一定会有私心、私利。家族以嫡庶血缘为标准,而师门却是以能力和天赋评判,当然是后者更公开透明,会将有限的修行资源分配更适合的人,对大局会更有利!”

%25
那人张了张口,发现无法批判门派的优越,只好转变角度道:“难道家族就不能挖掘人才吗?家族也有吸收外来人才的体制,比如联姻,或者义子义女。”

%26
“的确是这样。”元始仙尊点头,“但是联姻的能有几个呢?义子义女就算再多,在家族当中,和那些有血缘关系的真正后辈比起来,他们恐怕都要遭受排挤和打压的吧?”

%27
人族蛊仙们终于哑然无语。

%28
元始仙尊用诚恳的语调,继续道:“我知道乍然创建门派,在感情上的确有些无法接受。蛊仙之道乃是我们苦苦索求,耗费了不知多少代价得来的成果,传给自己的后代也就罢了,却要传授给其他姓名的外人。”

%29
“但事实上,创建门派并不是意味着就要将仙道的奥妙,随意地传授给别人。而是选择人才,这种人才不仅是天赋好,而且还要有品德,最重要的是忠于自身的门派。”

%30
“再者,就算不创建家族,难道诸位的后人就不能得到你们的照顾吗?当然不是这样。这些人加入门派,自然会受到你们的关照,但是我希望诸位务必暗箱操纵,不要违背了明面上的门派规矩。”

%31
“只有我们从自身做起,守护门规,我们的后辈、徒弟、子孙才能积极地遵守门规。只要我们创建门派,并且积极地团结在一切,我相信我们人族必定会真正的彻底的崛起!”

%32
“其他四域我还管不了那么远,但是在中洲,我希望诸位都创建门派,舍弃家族。为了整个人族大局,我必须强调这一点:任何一位中洲蛊仙都需要创建门派,不可组建家族。任何一位将来晋升的蛊仙,都要依约如此。一旦有人违规,就要遭受我本人,以及其他仙友的联合制裁!”

%33
人族诸仙闷声不吭,当元始仙尊强势起来的时候,他们谁都无法违逆。

%34
不过令他们感到欣慰的是,元始仙尊并不是一味地禁止家族制,门派制度只规定了蛊仙,也就是说凡人和蛊师都能依照家族的形式,继续生存下去。

%35
并且就算蛊仙创建了门派,也可以照顾自己的后辈和朋友。

%36
这算是元始仙尊对于现实的妥协,但从这一点,却也更可以看出他的高明之处。

%37
元始仙尊继续道:“作为你们的领袖,我将以身作则!我在此立誓,我将创建门派,不立家族。我将广收门徒,将自己所能毫不保留地传授给我的徒弟们。”

%38
“并且,将来当我陨落死亡,我的仙窍也将贡献出来,充当门派根据之地,而不是留给我的血脉后代。”

%39
“元始仙尊大人!”

%40
“您……”

%41
一时间,在场的人族蛊仙们都震动了。

%42
这一刻,他们纷纷看向元始仙尊,从他平静的面色中感受到对方甘愿牺牲的伟大,维护大局的决意。

%43
既然元始仙尊都这样做了,那么他们又有什么理由不这么做呢?

%44
“元始仙尊大人,您说的对,我愿意响应您的号召!”

%45
“从今以后,创建门派,不立家族。”

%46
“有仙尊大人,实在是我人族之福!”

%47
“跟随着您,果然是我平生最正确的选择。”

%48
“仙尊大人,我对门派的创建还有一些不明白的地方。”

%49
“嗯,请说吧。”元始仙尊道。

%50
“请问您要创建什么样门派,门派的驻地又要选择何处?门派有什么规矩,创建门派又有什么要领呢?”

%51
元始仙尊深呼吸一口气:“关于门派,我早已构思良久。这话一说起来,就要涉及各个方面。就以我为例吧。首先,我的门派名为天庭……”

%52
良久。

%53
元始仙尊为人族诸仙解惑。

%54
众仙再无迷惑之处。

%55
元始仙尊扫视诸仙:“接下来就看诸位的了。我相信,只要我们广立门派,正确发展,就算是我死后,人族也必会越加昌盛,异族若无改变,必将没落!”

%56
“请诸位饱含信心!”

%57
“别看我们的门派,现在只是一个单纯的想法,从未实现。”

%58
“我相信,只要遵循我们的计划发展下去,门派将会在历史的长河中逐渐绽放出夺目的光辉。我的天庭,我们的门派,将会成长为天下闻名,雄踞一方,敌人为之变色的存在!”

%59
三百八十七万九千六百八十七年后。

%60
天庭战场。

%61
激战继续着,长生天一方得到巨阳一击的资助,占据上风。

%62
然而,在仙墓中,一个又一个的天庭成员不断苏醒。

%63
“是谁?胆敢犯我天庭!”

%64
“这是历代先人的成果,我决不允许你们破坏。”

%65
“多少人抛头颅、洒热血,区区家族制度的长生天,也想击败我天庭?哼,妄想!”

%66
“长生天,你是可鄙的。为了一己私利,就要引发人族的内战吗?”

%67
“我们天庭始终是人族的圣地,人道的天堂,人族的正统!”

%68
“我们代表着人族的精神,数百万年始终不变!”

%69
“我将追随元始仙尊,追随星宿仙尊,追随元莲仙尊,以死捍卫我们的——天庭!!”

%70
一声声的呐喊,或老或少,有男有女,形形色色的天庭成员从仙墓中钻出,参加战场。

%71
就像三百万年前元始仙尊说的那样,他的天庭成长为了庞然大物,盖压世间。

%72
他的精神,一直传承下来,从未变过!

%73
无数位人才,被门派挖掘栽培出来,追随早已死去的他,捍卫他的精神理念!

%74
北原蛊仙的悍勇无以伦比,但是天庭众仙却显出疯狂的气势,哪怕他们仙蛊不足,即便是同归于尽,他们都的脸上从未有迷惘或者恐惧。

%75
反而。

%76
他们的脸上像是闪着光。

%77
殉道者的光辉啊……

%78
是如此的灿烂耀眼!

%79
“为什么?明明我们的实力更强,为什么竟是被天庭的气势压制住了!”冰塞川狠狠咬牙,面色铁青。

\end{this_body}

