\newsection{原是天意谋划}    %第三百七十二节:原是天意谋划

\begin{this_body}

武庸主动撤退,并未让方源、凤九歌高兴,反而心头俱是一沉。

武庸撤退,自然也可以前进。

方源、凤九歌都希望,武庸能够和他们对拼,一直拼出个你死我活的结果来。

但武庸并没有这样做,而是非常理智和冷静地选择了撤退。

他是八转正道蛊仙,面对两位七转联手,居然主动撤退。

这事情一旦传出去,武庸的脸面也就丢了一小半。

之所以没有全丢,则是因为他面对的蛊仙是大名鼎鼎的凤九歌,还有柳贯一(方源)。

武庸这样做,需要不少勇气。

当然,更多的则是智慧!

仙道战场杀招四清四边风被分解,武庸遭受杀招反噬,浑身重伤。

但他有玉清滴风小竹楼,还有一战之力。

不过,等到他发现凤九歌有移动腾挪之能,又对方源无可奈何之后,他就很明智地选择了撤退。

这一撤,恰到好处。

甚至可以说,比之前的猛烈进攻更为犀利,一下子点中了方源、凤九歌两人的七寸上。

首先,玉清滴风小竹楼只分解了一些边角,速度极快。

方源、凤九歌要追击武庸,是不大可能的。尤其是方源的上极天鹰,也脱离了他的掌控。

其次,武庸争取到了宝贵的时间和空间,可以让他从容地疗伤。

毕竟他身上的伤势不轻,严重影响到了他本身的战斗能力。

最后,武庸还可以修补玉清滴风小竹楼,甚至向这座仙蛊屋中添加一些蛊虫。下一次面对凤九歌的仙道杀招,就不会那么被动了。

武庸的音道境界虽然普通,但是他的风道境界,早已经达到触类旁通的程度。完全可以在风道上做出应对,尽量削减凤九歌的杀招威能。

也就是说,等到武庸再次追杀过来,方源、凤九歌面对的,很可能就是痊愈的八转蛊仙武庸,以及一座巅峰状态的,专门针对凤九歌做出修缮的仙蛊屋――玉清滴风小竹楼!

武庸进退自如,完全用不着和方源、凤九歌硬拼,先撤下来,再等待更好的时机。

反观方源、凤九歌二人,却是无可奈何,完全陷入被动状态,不能拿武庸怎么样。

方源只有逆流护身印,比较能拿得出手,而凤九歌虽然有仙道战场杀招,但就算有时间可以布置,也禁锢不住武庸,皆因后者掌握八转仙蛊屋玉清滴风小竹楼。

实力差距!

尽管方源、凤九歌联手,但是面对武庸,还是有着巨大的实力差距。

若只是武庸而已,他们两人还有取胜的些微希望。

但是武庸手中掌握着仙蛊屋玉清滴风小竹楼,综合实力暴涨一大截,这就让两人无可奈何了。

“方源,你走吧。”就在这个时候,凤九歌对方源开口道。

他的语气很平淡。

方源的瞳眸不禁微微一缩。

凤九歌深深地望了方源一眼,面无表情,旋即又缓缓转头,望向天际远处的小黑点儿(玉清滴风小竹楼)。

“昔日你救我一命,今日我救你一次,从此你我两不相欠。”凤九歌继续道。

方源微微挑眉:“这不对吧。这一次你若是没有我相助,恐怕早就被武庸杀死了。你帮我,我也帮了你,所以你还欠我一次人情的。”

哪知凤九歌语气仍旧淡淡:“我说不欠,便是心安,自然不欠。你自认为你的,你要记住,这一次我若生还,必定以你为目标,展开追辑,就算你逃到天涯海角,我都不会放过你。你可要小心了。”

“哈哈。”方源忍不住大笑一声。

他彻底明白了凤九歌的心思。

凤九歌救他,是为了还之前的救命恩情。但这样一来,有违他的正道身份,也会让妻子女儿处境尴尬。

所以当凤九歌救过方源一次后,他便要追杀方源,将方源杀死,证明自己,也为妻子女儿考虑。

“好,下次见面,你我之间就是生死仇敌了。”方源说着,开放仙窍门户,放出黑楼兰等人。

上古战阵――四通八达开始徐徐催动起来。

凤九歌再没有回应,甚至他连侧身转头再看一眼方源的动作都没有,他直接飞向武庸。

方源若用四通八达离开,恐怕会引起武庸追击。为了真正还清方源的救命之恩,凤九歌舍命冒险,主动去纠缠玉清滴风小竹楼。

最后深深地看着凤九歌的背影,上古战阵轰然发动,带着方源和其他三仙,瞬间消失在了原地。

一直在注视方源、凤九歌的武庸,见到方源忽然离去,只能苦笑一声。

现在他有些后悔自己之前的安排了。若是他并非单独一人的话,自然有能力同时留下方源、凤九歌两人。

但为了武家名望,他苦心孤诣,细心筹谋,分开了其他南疆正道势力,像是撒了一张巨网,把方源等人赶向他这最后一道防线。

这里距离西漠,已经不远了。

方源此刻逃脱,只要不出意外,就是板上钉钉。

“是我大意了。”

“没有料到天庭出手,居然是帮助方源这个魔头!”

武庸望着飞来的凤九歌,目光阴寒森冷。

就是这个家伙,坏了自己的全盘筹谋,该死!

不过……

对方的天资,还有地位,让武庸有些踌躇。

杀死凤九歌的价值,反而不如生擒活捉。

若是杀了凤九歌,武庸是和天庭、中洲反目成仇,要知道天庭可是掌握着武遗海就是方源这个武家的把柄,在手中呢。

但若生擒活捉,武庸却是可以依仗此人,和中洲谈判的。

事实上,武庸之前就有这样的想法,若是他一心想要杀死凤九歌,绝不会是之前的攻势节奏了。

武庸不是一个纯粹的杀人凶手,他是超级势力的头领,他要考虑的东西,绝非只是一场战斗本身,还有这场战斗背后牵扯到的无数层面。

仙道杀招――离歌!

凤九歌离着玉清滴风小竹楼,还有一段距离,再度催发之前的奇招。

武庸冷哼一声,仙蛊屋已经被他增添了不少蛊虫进去,分解虽然还有效果,但比之前可要削弱许多程度了。

激战再度展开。

武庸很快占据上风,尽管他身上的伤势还在。

仙蛊屋玉清滴风小竹楼横行霸道,凤九歌只能游击,利用离歌拆解。

武庸老谋深算,已经退了一次,第二次撤退完全毫无心理压力。

只要是仙蛊屋被分解到一定程度,武庸就退走,飞出一段距离后,进行修正。

凤九歌的离歌,本身并无任何杀伤威能,蛊虫被分解出来后,被武庸得到手,又重新搭建。

唯一让武庸感到有些头疼的是,仙蛊屋并不容易搭建。

可以说,这场战斗已经定局。

凤九歌绝不可能胜利。

或许硬冲对打,双方毫不后退,凤九歌还有一些胜利的希望。

但可惜的是,武庸并不莽撞和蠢笨,在交手几个回合之后,他也舍弃了面皮和八转尊严。

这些换来的是他进退自如,一直将占据操纵在自己的手中。

即便是凤九歌这类的天骄人物,也只能被他死死压入下风,没有一丝胜利的可能。

与此同时,在中洲天庭。紫薇仙子手中托着星宿棋盘,望着棋盘上演绎的战况,赫然便是凤九歌战武庸的景象。

“差不多了。”紫薇仙子口中呢喃。

原来,凤九歌遭遇方源,都是星宿天意的安排!借助天庭之手,让凤九歌追逐偷道仙蛊“偶遇”方源。

就连凤九歌本身,就被蒙在鼓里。

甚至,天庭方面也很懵懂,不知道星宿天意为什么会如此安排。

凤九歌的气息已经变得微弱。

他浑身是伤,只余一击之力。

反观武庸缩在玉清滴风小竹楼中,状态完好,把转仙蛊屋也只是被分解了一些外围。

战后便退,退后又进,再度开战。

人的灵性是万物之首,武庸的正确战术让凤九歌毫无翻盘的可能。

“凤九歌,你若束手就擒,我还能留你性命,甚至将来将你遣送会中洲,交给天庭发落。”武庸淡淡地道。

凤九歌微微一笑:“武庸啊。你若能接下我这一招,我主动认输,又有何妨?”

武庸面色微微一凝。

战至此刻,凤九歌居然还有手段未出!

“昔日,我之所以成名,得到八转蛊仙的认可,认为我拥有八转战力,就是因为此招。且看我的第七歌――”

“且慢啊,凤九歌。”就在这时,两道身影忽然出现在战场当中。

俱都是八转气息。

来自中洲十大古派的蛊仙!

“嗯?”武庸惊异,天庭方面居然还有暗手未出,这有点出乎他的意料之外。

在他想来,若是有这样的暗手,之前为什么不发动?

“难道说是刚刚赶到的不成?”武庸猜测。

一边想着,武庸一边冷笑:“中洲的人,你们来的正好,我正有事情要与你们分说!你们不仅抢走了我南疆正道的众多仙蛊,而且你们的人竟然还包庇魔头!”

武庸不惧。

他有仙蛊屋护身,更有其他南疆正道蛊仙正在赶来支援。

但哪知这两位中洲蛊仙也是有备而来。

一人道:“你我两方虽然份属两域,但都是正道中人,自然要同气连枝。之前取走仙蛊,只是因害怕这些蛊虫遭受魔手而损。这些仙蛊,我方早就准备,现在就可将这些都还于原主。”

“哦?”武庸讶然。

另外一人又道:“至于包庇魔头,这自有交代。凤九歌,你愿发誓接受天庭命令,终生追杀方源,一日不取他性命,一日都不归还吗?”

凤九歌沉默了一下,开口。

“我愿意。”(未完待续。)

\end{this_body}

