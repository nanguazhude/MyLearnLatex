\newsection{影宗方源和谈?}    %第二百七十节:影宗方源和谈?

\begin{this_body}

幽暗的石壁上,附着着点点的幽火。

这些火光,个头都不大,小的如萤虫,大的也只是灯笼一般。

它们散发着冰冷的绿焰,惨绿的火光,映照在紫山真君、影无邪、黑楼兰、白凝冰的脸上。

从逆流河大战中首先脱离,他们就在紫山真君的带领下,直接传送到了这里。

不过在那之后,紫山真君就再次陷入疯癫的状态当中。

望着啃自己脚趾头的紫山真君,白凝冰对影无邪冷笑:“这就是你的计划?让这个八转大能,斩杀方源?”

这一次大雪山福地之行,影无邪虽然救出了紫山真君,但是却也损兵折将,石奴、太白云生都战死沙场。

黑楼兰沉默。

这样的八转蛊仙,怎么看怎么不靠谱。

但碍于影宗盟约,黑楼兰缄默其口。

影无邪被白凝冰如此挖苦,却是面不改色。

他经历多番挫折,如今已然养成了泰山崩于前而不动的城府。

他缓缓开口:“是我小看了方源,没想到他竟炼成了坚持仙蛊,成为逆流河主。难怪天意选择他为棋子。这种人物只可能被毁灭,不可能被打败。”

影无邪竟对方源毫不吝啬赞赏之词。

尽管是死敌,但影无邪心胸开阔,承认方源的优势和强大。

白凝冰冷哼一声,心中不由浮现出方源征服逆流河的那一幕,她没有再开口刁难。

狭窄的石洞,只剩下紫山真君大喊大叫的声音,不断地回荡着。

片刻之后,紫山真君忽然停住叫喊,小小的身躯摇晃了一下,旋即又站稳。

他挺直了身板,双眼再次涌现出清明之色。

“大人,您清醒了!”影无邪大喜,连忙靠近。

紫山真君手捂额头,脑袋里迸发出一阵阵的剧痛,疼得让他脸色都有些扭曲。

“之前在逆流河时,我是强制清醒,所以时间较短。这一次不一样,我清醒的时间比较充裕。过了多久?大战结果如何?”

“已是大半个月过去了。方源顺着界壁,逃离了北原,我估计是进了东海。毕竟他在东海,还有些根基。(WWW.qiushu.CC 好看的小说至于大战,还未有结果。中洲蛊仙意欲撤离,结果遭受雪胡老祖的追击,危难之间,从天而降一座仙蛊屋。双方已经打上了白天。”影无邪答道。

紫山真君点点头:“这一次,中洲蛊仙势必凶多吉少,当初巨阳仙尊将北原当做老巢经营,怎可能没有布置?单说长生天中的八转蛊仙,就未现身过。”

“大人,接下来,我们该如何行事?”影无邪问道。

他曾经是影宗首脑,如今紫山真君苏醒过来,自然是这位八转大能领袖影宗,而他影无邪退居下来。

紫山真君正要开口,这时洞中忽然传来呼呼的风响。

与此同时,周围洞壁上附着的点点幽火,不断扑闪,剧烈的摇曳,仿佛真的处于大风之中。

紫山真君眼中精芒一闪:“这处洞壁,乃是我精心布置的退路之一。这些幽火被我精心栽培,形成一片天然蛊阵,不着斧凿痕迹,防备他人推算。我疯癫之时,这些幽火闪烁了多少次?”

“一百零七次。”白凝冰淡淡地答道。

“恐怕其中,还有方源推算。他的侦查杀招,能够遵循运道的奥妙,找到我们的位置。即便我们身处仙窍,也无法防范。”影无邪沉声道。

紫山真君点点头:“方源此人不愧是天意栽培出来,对付我们影宗的关键人物。此人胆魄极大,才情惊世,意志绝伦,乃是万年才出一位的绝代天骄。”

堂堂影宗的紫,居然对方源有这么高的评价。

但在场的三位蛊仙,都沉默不语,似乎默认了紫山真君的话。

紫山真君接着道:“接下来,我们的计划,便是与方源和谈。”

一语惊人。

“什么,和谈?”影无邪等人纷纷变色。

“有什么好奇怪的?”紫山真君望了望三人。

他对现在的情势,了解透彻。这是因为在他苏醒之初,影无邪就动用信道手段,让紫山真君知道了一切影无邪知道的事情。

“你们还是太年轻,知晓的东西还不是很多。现在我们的首要大敌,绝非方源,而是天庭。”紫山真君摇了摇头,看向影无邪,“你以为营救本体魂魄是那么容易的事情吗?除了南疆的那片超级蛊阵,还有偌大的梦境之外,我敢保证,当你动手的时候,天庭方面必定会出来搅局!就像义天山上,他们驾驭监天塔,企图破坏我宗大计。又像此次,他们前往大雪山福地,尝试拯救马鸿运。”

影无邪面色一沉:“大人,此点我考虑过,但本体实在是危在旦夕,一直被困梦境,受到梦境的消磨,所以我才如此着急……”

“正是因为如此,我们才更需要和方源合作。”紫山真君继续道。

“首先,方源的实力,已经厉害非凡,单凭那记仙道杀招,就可以对付八转蛊仙。”

“其次,方源乃是完整的天外之魔,名登诛魔榜。天庭是我们共同的大敌。”

“最后,我们合作,对双方都有利。我影宗掌握着大量的修行资源,而他手中则有解梦的手段,正方便我们克服超级梦境的阻碍,救出本体魂魄。”

影无邪紧紧地皱起眉头,他努力放下心中的私人情绪,考虑此事的可能性。

但很快,他就满脸担忧之色:“方源狡诈阴险至极,早已经视我们为心腹大患,千方百计地想要铲除我们。如今他已经成为逆流河主,连八转存在都奈何不了他。我宗迫切地需要解梦手段,但方源他对于修行资源的需求,却不是那么迫切的。要让他和我们合作,恐怕不太可能。”

紫山真君点头:“你此言很有道理,不过,你却不知道至尊仙体是有重大弊端的。”

“大人此言何意?您是指,至尊仙体上道痕不互斥吗?”。影无邪奇怪地问道。

“当然不是。道痕不相排斥,这正是至尊仙体的最大优势之处,利远大于弊。我指的是真正的重大隐患!”紫山真君摇摇头道。

“居然还有重大的隐患?”影无邪吃惊不已。

“我们原先的计划,是炼成十转至尊仙胎仙蛊。只有十转的至尊仙胎蛊,才能真正称得上完美,没有任何弊端。但很显然,最终炼成的只是九转,那个重大的隐患此时并未彰显,但绝对会令方源不得不和我们合作。”紫山真君道。

至尊仙体还有巨大的弊端隐患!

究竟是什么样的弊端隐患,让紫山真君如此笃定,能够钳制方源?

影无邪等人都非常好奇,但紫山真君却是打住,没有细说。

同时,紫山真君心中也有猜疑。

“奇怪。”

“从影无邪的情报中,我宗炼制至尊仙胎蛊,颇为仓促。并且和原本的计划不相符合。”

“究竟是什么原因,让本体立即动手,如此着急呢?”

“看来,联络方源的同时,我还得去光阴长河一趟!”

东海。

湛蓝的海面上,方源快速地穿梭在漫漫白云之中,他目标直指南疆。

“又来了!”疾飞当中,方源忽然面色一变。

他能感觉到,身上的暗渡威能,在迅速地消耗。按照这样的速度,很快就会耗光。

很显然,这是有大能在推算自己。

左右无人,方源连忙落下至尊仙窍。

进入仙窍当中,顿时隔绝内外,排斥了大能的推算,暂时解决了自身暴露的危险。

方源叹息一声。

自从逆流河大战之后,他就遭受到接二连三的推算。

柳贯一这个身份,是彻底的火了。

不只是北原、中洲,更有东海、南疆、西漠的蛊仙,对柳贯一产生浓厚的兴趣,想要推算出更多的情报。

柳贯一只是假身份,根源仍旧是方源。

如此一来,方源就承担了全部压力。

纵然有暗渡仙蛊,也让他时常停下脚步,躲入仙窍中休整。

一路行来,方源暗中计算,遭受推算的次数,已经多达四五百次。

这大大干扰了他的行程,使得他不得不走走停停,仍旧逗留在东海之中。

若换做以前的速度,此时早已经人在南疆里了。

“若非有暗渡仙蛊帮衬,只怕这种情况更加严重。”

“不过,能引发出如此程度的轰动和关注,也实属正常。已经有人将我和凤九歌并列。”

“之前推翻八十八角真阳楼和王庭福地,有影宗帮忙遮掩,并不觉得这种压力。现在却是彻底体会到了。看来,研究出一个仙道杀招,专门对付这种推算,也是相当有必要了。”

方源是智道宗师,推算出一个仙道杀招来,也不是不可能。只是没有逆流护身印,那般容易了。

毕竟推演逆流护身印的时候,并非只是因为方源的水道境界,还有其他种种重要的因素。

最关键的是仙蛊相当合适,起点根基很正,才能一路顺风顺水。

而现在适合方源的仙蛊,却是暗渡,偏偏此蛊并非智道,而是隶属暗道。

方源在暗道方面的境界,一点都不出众。

ps:身体状态不佳,今天一更。(未完待续。)

\end{this_body}

