\newsection{不过如此}    %第五百二十四节:不过如此

\begin{this_body}

琅琊福地,激战正在进行。(wwW.qiushu.cc 无弹窗广告)

中洲蛊仙们领命,四下分散,开始布置仙阵。只要仙阵一成,就能接引其他中洲蛊仙,源源不断地进入琅琊福地当中,支援凤九歌等人。

对于琅琊地灵而言,这是绝对不利的坏消息!所以必须要阻止他们。

“这凤九歌好生算计!我这边毛民蛊仙只有集结一起,形成天婆梭罗上古战阵,才能力敌八转。他就是想要我分兵,我岂能如他所愿?”琅琊地灵心中冷笑。

若换做上一任白毛地灵,或许此刻已经六神无主,十分慌乱。

但这位黑毛地灵却是擅长争斗,立即看出凤九歌的谋算。

“分兵是万万不可!既然如此,那我就先宰了你们一些中洲蛊仙,和你们拼一拼速度。究竟是你们铺设仙阵来得更快,还是我杀你们更快一些!”

琅琊地灵眼中闪烁着铁血之光。

吼!

银色巨人陡然咆哮,声浪激荡四面八方,将凤九歌排斥开去。

随后银色巨人抬起右脚,迈开步伐。

它的动作十分缓慢,好像是右脚上挂着一座无形的小山。

“不好!”凤九歌却立即变色,因为他从银色巨人的身上感受到了一股强烈的宇道气息。

这是宇道杀招!

凤九歌想要阻止,再度催发出三绝音攻,但银色巨人蛮横无理,直接硬生生承受这些攻势,也要将宇道杀招完成。

银色巨人终于将右脚迈出去。

当它的右脚踩在地上的那一刻,它的脚下地面陡然缩减无数,一脚迈出去就跨越了上百里!

一位中洲蛊仙正在驾着云朵疾飞,忽然间周围空气猛地激荡开来,他整个身子和前方,都被巨大的阴影笼罩住。

他回头一看,瞳孔猛地缩成针尖大小,只见银色巨人已经站在他的身后,一双大手分别从左右两边,向他包抄抓来。

中洲蛊仙心头警兆大起,猛地咬牙,催动毕生手段,想要逃脱。

但银色巨人的两只大手的手心,忽然各自迸射出一道玄妙的光辉来。两道玄光射中中洲蛊仙,让他刚要酝酿成形的杀招陡然崩溃!

“快来救我!”中洲蛊仙被两道奇光锁住,动弹不得,他大声惊呼,慌忙向同伴求救。

下一刻,他就被银色巨人的两只大手笼住。

银色巨手十指并拢,两只大手相互融汇,好似一个球形的银色密封囚笼。

无比的压力从各个方面,逼迫而来,压迫得中洲蛊仙身上骨骼咔嚓作响。

中洲蛊仙脸色剧变,他能感觉到,身上的防护手段正在迅速崩溃,难以抵抗银色囚笼中的内部碾压!

“救我!!”他再无一丝从容,嘶声力竭地大吼。

噗。

下一刻,他全身化为血水肉泥,被银色囚笼中的压力彻底挤爆!

这位中洲十大古派中的七转强者,就这样惨死在银色巨人的手中。

“好!中洲十大派的蛊仙,也不过如此嘛。”银色巨人体内,琅琊地灵哈哈大笑,“就这样干!那凤九歌有八转战力,是最难啃的骨头,且不过去管他。先把这些软柿子都捏爆掉,我看天底下谁还敢来冒犯我大毛民的栖息地!”

天婆梭罗还是上古第二战阵,久经考验,乃是上古蛊修的巅峰结晶。它本来就有八转战力,但是之前毛民蛊仙战斗素养太低,就连白毛地灵也是如此,所以才在影宗的进攻中,表现极差,被影宗打坏八转仙蛊屋炼炉,从容而去。

现在却不一样,在黑毛地灵的大力改革之下,琅琊派组建,又经过方源一段时间的亲自教导,毛民蛊仙们的战斗素养早已经今非昔比,拔高极多。如此一来,组合而成的上古战阵天婆梭罗,就真的充分发挥出了八转战力,黑毛地灵还时常阻止毛民蛊仙,演练这座上古战阵,所以导致现在中洲七转蛊仙也要饮恨于此!

南疆,胎土迷宫。

一幕幕红尘,继续演绎着。

“看来我真的是失忆了。”月色下,方源叹息。

他一身单衣,走在土坯墙围绕而成的低矮庭院之中。庭院中有一个水井,井壁砖头都散落了大半,还有一株果树,此刻一点树叶都没有,干枯削瘦。[看本书最新章节请到求书 小说网www.Qiushu.cC]

“这就是我的家了。”方源扫视周围,又叹息一声。

他望了望老妇人入睡的屋子,比自己的卧室更破旧,纸糊的窗户上有着许多破洞。见到这一幕,羞愧的情绪就浮现在方源的心头。

又想到绣娘,方源的心中便涌出一股爱意。

无疑,这是他生命中最重要的两个女人。

休养了数月,他的伤势已经几乎都好全了,能够下床走动。

此时他脑海中思绪起伏,月色如霜,映照着他紧锁的眉头,躺在床榻上的这些天,他不止一次地思考自己的处境。

“绣娘如此真心对我,我一定要娶她回家,不辜负她这番心意!”

“娘亲其实很中意绣娘,只是碍于局势,不得不忍痛割爱。”

“我要娶绣娘,有两大阻碍。第一个是舒家,舒家家大业大,人多势众。第二个是绣娘的双亲,他们都是蛊师,看不起我这个凡人。”

“归根结底,还是我资质不行,做不了蛊师。若是能成为蛊师,修行到三转,绣娘的双亲恐怕就能够接受我了。”

“唉!”

想到这里,方源深深地叹了一口气。

这是一个死局,根本无法可解。

虽然也传闻有改变资质的蛊,但是方源这样的穷小子,谁会无缘无故地来耗费巨大代价帮助他?

绣娘纵然也有蛊师资质,但却只是丙等,潜力低微。她若前途广大,修成三转蛊师,也能定夺自身的幸福,不像如今这般无可奈何。

“该如何是好呢?”

方源望月兴叹,无限苦恼。

不过就在这时,他听到隐约间有一股琴声传来。

“咦?这是哪里来的琴声?”方源仔细听,又听出这股乐声不只是琴音,还有钟磬,竹笛等等。

他侧耳倾听,寻找乐声的来源,渐渐地走到了水井旁边。

“古怪,这乐声居然从我家水井中传出来的。这是何因由?”方源趴在井边往里看,只见月色也映照在井中,十分分明。

方源便看到这地面下的井壁,有一处地方,数十块砖块组成门的形状,竟然闪烁着白玉般的光泽,非同寻常。

他好奇心大起,把水桶上的麻绳牢牢系在一旁的果树上,然后便顺着麻绳,攀爬下去。

到了中段,他双腿攀附在麻绳上,面对白玉似的砖们,伸出一只手来,试着一推。

他力道很小,但被玉砖被他轻轻一碰,就化为虚无,又凭空喷涌出一股吸摄之力,将猝不及防的方源,猛地吸摄进去……

一夜过去。

方源起床,双眼放光,精气神已经完全不同,和之前有天壤之别。

原来,这井壁当中藏有一道蛊仙遗留下来的传承,名为道德真传。因为历史太久,已经残破不堪。

方源继承了这份残破的真传,当晚就开辟了空窍,成为一转蛊师,又得到数只凡蛊认主,实力暴涨,整个人生境界都完全不一样了。

“天可怜见,让我获得如此绝世机缘!这下子,我迎娶绣娘,就大大有望了!”

方源心中喜悦,几乎满溢而出。

但他又强自忍耐,知晓当下不可随意暴露。这份机缘太过重大,轻易暴露出来,便是杀身之祸!

“我要努力苦修,偷偷积累,直到有足够的实力自保,才能暴露自己的身份。”

苦修这份道德真传!

方源下定了决心,又微微皱眉。

“只是这份真传,要求比较奇特。需要继承者时刻保持心中的正义,做一个有道德的人。并且道德越高尚,修行起来就越加事半功倍。”

“道德?”方源迟疑了一下。

他心中隐隐对这个词有些不屑,但现在得了这份真传,这份不屑也就随风飘散。

接下来的日子,方源就努力修行,不断偷偷积累实力。

老妇人劝说他,不要和绣娘走近,方源点头答应,心中自有主意。

绣娘来找他缓解相思之苦,方源便温言相劝,让她多加忍耐,自己会有办法。

让方源无比欣慰的是,绣娘毫无保留地相信了他的说辞。

修行途中,老妇人忽然一病不起。

方源心如刀割,但寻常的草药没有一点效果,方源急得心焚如火。

他找蛊师医治,但那位二转蛊师却嫌弃他的微薄诊费,不愿在大冬天出诊。

方源打听到这个蛊师喜欢吃鱼,便走了十几里路,来到一片山间湖泊中。

方源焚烧篝火,融化冰层,又钻入水中,捉来湖鱼,送到蛊师手上。

蛊师为方源的这份孝心大为感动,终于出诊,治好了老妇人。

老妇人痊愈,方源对此十分高兴。更有一份意外之喜,是他发现自己的空窍中,有了一只二转的积德蛊。

按照真传叙述,这种积德蛊,只要方源平时积德行善,就能不断地自行炼成。一次行善积德越大,炼出来的积德蛊的转数就越高。积德蛊有着许多妙用,乃是道德真传的基础核心之一,其中最大的作用,就是能改善方源的修行资质,使其最终转变成正道善德身。

按照真传中的模糊记载,正道善德身仅次于十绝体,还没有十绝弊端,非同小可,代表着无限光明的前途和潜力!

方源至此便行善积德,不放过身边任何的机会,不断炼成积德蛊。

很快,他的德行和美名,就开始流传开来。

方源的资质越来越好,修行的效率也随之越来越高。两年后,他悄悄地成为了二转蛊师。

在一次意外中,他为了搭救一个落水的婴孩,不得不暴露了蛊师的身份。

刚巧,一位舒家的家老看到了这一幕,方源的秘密暴露了。

舒家家老想到方源区区一届凡人,居然能改善资质,修成二转,定然是有巨大秘密,于是便向方源出手。

方源和他一番苦战,最终超常发挥,使出凡道杀招以德服人!

这一招,耗尽了他积攒的积德蛊。

舒家家老中了这一招后,毫发无损,不过旋即痛哭流涕,跪在地上,不断扇自己的脸,然后悲呼:“我不是人,我不是人,居然动这样的邪念,敢对如此伟大德操的您动手!我是畜生,我不是人呐。请您宽恕我,给我忏悔和将功补过的机会吧!”

方源当即愣住,没想到这一招会如此厉害。

他从此便收服了这位舒家家老,对舒家少爷有了最关键可靠的情报来源。

至此,方源更加努力积善行德,他的美名在凡人当中广为传播,许多蛊师都听说了他的名字。

又过了几年,纸终究包不住火,方源的蛊师身份彻底暴露。

不过他借助其他蛊虫,遮掩住了自己的大秘密。绣娘的双亲,对他的看法大为改观,但是仍旧不想把自家的女儿,嫁给方源这样的穷酸。

和形单影只的方源相比,舒家少爷优势太大太多。

但绣娘却钟情于方源,令舒家少爷妒火攻心,屡次找方源麻烦。

方源生活虽然波折不断,屡屡被刁难和陷害,但只要有绣娘在,他就满怀希望地迎接每一天。

终于矛盾积累到了极限,舒家少爷约战方源,要和他生死斗。

方源答应了他,这个事情闹得很大,即便是周围的势力,都有很多蛊师听闻。

就在决斗的前夕,深夜里,方源遭遇了舒家蛊师的联手偷袭。

方源重伤逃脱到了山林野外,心中愤恨至极。

因为他知道,自己伤势沉重,根本不会是舒家少爷的对手,若是赴约而战,就是送死。但若他不去,他就会被认作战败,绣娘到时定会身不由己,受到双亲逼迫,嫁给舒家少爷。

“我怎么可以让绣娘落入这卑鄙小人的手中?就算是死,我也要……什么声音?”方源惊愕,转头发现,那边的山头一股恐怖的兽潮正在奔袭而来。

“好!老天开眼了,就让这股兽潮把城池冲垮,舒家猝不及防,必定大败亏输,也算是为我报仇雪恨了!”

方源大喜,但笑容在下一刻又戛然而止。

他想到自己的娘亲,想到绣娘,想到许许多多的邻居朋友。这些人可亲可爱的容颜,陆续浮现在他的脑海中,萦绕不去。

他若是任凭兽潮冲击城池,不晓得这些人会有多少因此丧命。

就算他偷偷施救,凭借他的能力,又能救得了多少呢?

“就算我救走了绣娘、我娘,那么这些人呢?他们难道没有孩子吗?他们也有自己的父母!我知情不报,就是害了他们呀。”

“还有,就算是舒家少爷再混账,但舒家的人并不是每一个都如此,他们当中也有可敬可爱的人,我为什么要迁怒于他们呢?难道他们就活该去死吗?”

“另外,这座城池我在这里生活了近二十年,它对我也有养育之恩,没有它的保护,我在野外早就被野兽叼了去。难道我也要眼睁睁地看着这座城池被攻陷吗?不,这是我的家园啊!”

想到这里,方源目光一定,猛地转身,往城池跑去。

尽管他知道,他在回头的路上,极可能会碰到那些舒家的蛊师,会被他们杀死。

但他也要去!

因为自从继承了道德真传之后,在这过去的数年当中,他早已经将道德印刻在自己的骨髓里,渗透到自己的思想中,他觉得自己有义务,有责任去做一些事情!

方源跑回城中,告知了大家兽潮来袭的事情,立即引发了高度重视。

这一次的兽潮来袭是如此的突然和隐蔽,但得益于方源的通告,城池中的蛊师们有了充分的准备,稳住了阵脚。

城主得知了方源的境况,对方源冒死来通告的行为,大加赞赏。

有城主撑腰,方源再不惧舒家动用势力来找他的麻烦,并且他和舒家少爷的约战也推迟下去。

之后,在耗时一年之久的攻防战中,方源表现极佳,救助了无数的蛊师和凡人。他乐于助人,不图回报,一方面使得他实力大增,炼成了不计其数的积德蛊,另一方面,人们对他推崇备至,让他声望变得极高。

兽潮退去之后,方源却是有了翻天覆地的变化。

他终于堂堂正正地和舒家少爷对决。

这一场生死斗,方源以绝对的优势取得了胜利。

但是他却最终饶恕了舒家少爷的性命。

“沈三啊,你以德报怨,我服了,从此为你马首是瞻!请受我一拜!”舒家少爷当场跪拜下去。

城池中的人们,无不对方源敬服万分。

三年后,城主自感老迈,将位置让给了方源。

方源已经有了五转的修为,成为了新城主。在正式上任的那一天,也就是他和绣娘成亲的好日子。

大红的喜字,热闹的宾客,美酒佳肴,洞房花烛。

绣娘装扮美艳,端坐在床边,望着方源深情款款地道:“三郎,你我今日终于成为夫妻了。我就知道我是没有看错人的。”

方源哈哈一笑:“那是当然。只是……”

他按住心口。

绣娘紧张起来:“只是什么?你哪里不舒服?”

方源摇摇头:“只是我忽然觉得有些空虚啊。”

绣娘噗嗤一笑,如牡丹花开,美不胜收:“你空虚什么?三郎,不知有多少人羡慕如今的你呢。”

方源脸色渐渐僵硬起来:“他们羡慕我?无非是财富、权势、武力和美色罢了。”

他说着这话,语气渐冷,目光透射出一股锋锐的力量。

绣娘微微变色:“三郎,你怎么了?你的话让我感到不安。”

方源认真地望着她,望着这一副令他朝思暮想、辗转反侧、魂牵梦绕、钟爱至情的美人儿。

他的目光是如此的认真,让绣娘心中越发忐忑不安起来。

然后,绣娘便见方源摇头叹息:“这如此种种……也不过如此啊。”

于是下一刻,周遭一切轰然崩溃!

ps:今天两更,但不方便分成两章,所以就合在一处,形成这一大章了。(未完待续。)<!--80txt.com-ouoou-->

------------

\end{this_body}

