\newsection{快!}    %第六十五节:快!

\begin{this_body}

半柱香的时间,方源已经远离龙象原。~,

前方出现了一大片的白云。

方源的速度下意识地缓了缓。

他现在对白云也多加注意,上一次在南疆,就碰到了那群讨厌的云兽。

吃一堑长一智,方源对任何大面积的白云,都抱着谨慎的态度。

侦查一番后,方源微微放松下来,这片白云没有任何问题。

他这才飞入云中。

视野中,顿时被一片白茫茫的云气覆盖。

不过没有关系,方源还有其他的侦查手段。

不管是听觉,还是声波,或者神念覆盖感应,都能让他在白云中自由飞行,不会像无头苍蝇四处乱飞乱撞。

到了蛊仙这种层次,凡人的手段已经是层出不穷,几乎应有尽有。

方源速度陡降,停下一切挪移手段。

然后,他脑海中一颗颗的念头频频闪烁,调动仙窍中的各种蛊虫。

大量的血道凡蛊,风起云涌,像是在至尊仙窍中刮起了一阵血红旋风。

然后是核心仙蛊飞起。

六转血本仙蛊!

它如王者登临,气象不凡,其余凡蛊皆是辅助,顷刻间为方源酝酿出一记仙道杀招血漂流。

血光从方源的浑身上下喷涌而出,又迅速液化,形成一股澎湃的血水。

血水载着方源,划破云空,向前方飞速前行。

速度之快,和之前完全是两个概念!

方源之前要离开龙象原,那是正道超级势力的范围,所以他不去调动仙道杀招。

因为大多数的仙道杀招,都有强大的气息流露出来,会惹人注意。

方源离开龙象原的路程中。都是采用的各种凡道杀招。

现在远离了龙象原,他终于可以催动仙道杀招了。

速度激增,迅速穿行。

一道修长的猩红轨迹,被拉出来。但又被浓厚的云层覆盖遮掩,外界见不到。

这是在云层中飞行的好处。

若是没有云层,方源就只能隐匿行迹。堂而皇之地在空中飞行。太过招摇了。

短短片刻,这片范围很广的云层,就被方源直接穿透。

数千里的路程,已被他跨越。

出了云层,方源就主动收起血漂流。

这个时间段,五域中血道魔仙还是见不得光,得夹着尾巴做人。血道人人喊打,如过街老鼠,方源可不想因为血道惹来麻烦。

尤其是天意还在密切地注视着他!

“除非将血漂流加以改良。[txt全集下载wWw.80txt.com]让所有人都认不出来它的跟脚。”方源心中闪过一念。

其实,这个血漂流,他已经又改良了一次。

在琅琊福地中修行,方源也在不断地练习血道杀招、力道杀招。

有一天,练习中方源灵光乍现,改良了血漂流。

本来血漂流发动时,是数百只凡蛊出现在方源的身边,环绕飞舞。宛若硕大圆环。

改良之后,这些凡蛊都在仙窍中飞舞盘旋。

别看这小小的一点改良。却是弥补了一个巨大的破绽。

在蛊仙身边飞舞,容易被人打击、破坏,在仙窍中却十分安全。

单凭这点,进步就不算小了。

但血漂流还是声势煊赫,飞在空中,划出一道长长的血河。让人不注意都难。还有逸散出浓郁的血腥气味,令人闻之作呕。

这些缺陷若不改良,血漂流的应用面就不太广。

若是智慧光晕可以利用,这些改良,方源可以一蹴而就。但仙僵肉身。方源还不能肯定它的安全。

有云层遮掩还好,没有云层,方源就不打算动用血漂流。

他直接催起剑遁仙蛊。

七转剑遁仙蛊!

方源斜刺上空,一路飞扬。

比起刚刚血漂流,他此时的速度还要更快!

单单一只剑遁仙蛊,就超越了六转血道杀招。

虽然依照他现在的修为,青提仙元催动剑遁仙蛊,十分不耐用,但方源却执意于此,态度坚决。

他此行的一个重要精髓,就是快!快!快!

强调高速。

这都是因为天意。

“天意是天道的意志,损有余补不足,视我为眼中钉,肉中刺。但天道亦有它运转的规则,天意必须依照规则行事,所以就有了局限。”

“天意不能随意行动,只有在灾劫的时候,才能亲自对我出手。平常时间,天意只能布局杀我。”

天意如何布局?

那便是影响众生。

就好像是当初,方源在王庭福地,受到墨瑶假意的影响一样,天意浩荡,无处不在,自然可以影响其他生命。

这种影响,是潜移默化,因势利导。

大多数情况下,都是一念之差。

天意能同时影响无数生命存在,一个个的一念之差,一次次的影响,将这些生命相互汇聚,形成陷阱杀局。

但这样一来,影响他人,布局陷阱,就需要时间。

所以方源才强调速度。

他速度越快,天意就越没有时间,来影响其他生命,布置出对付他的局面。

在南疆赶往北原的那路上,方源不知道这点,又缺乏有效的挪移手段,尤其是前期,给了天意充分的时间布局。

现在方源知道了天意的短板,自然要善加利用。

这亦是影宗对付天意的有效手段。

影宗的宗旨主要就是“潜藏积蓄,一招发动,如火山爆发,迅雷不及掩耳,达成目标!”

方源将这点学了个十足。

“对付天意,影宗、魔尊幽魂就是我的最佳榜样。他们是我的死敌,但死敌的身上,有比我优秀的长处,我自然要抛开一切的成见,去主动效仿学习!”

“而且我比影宗还有一个巨大的优势。那就是我可是完整的天外之魔!”

“影宗成员行事,还得思考自身,用我意时刻冲刷自己,防备天意的暗中影响。”

“而我却无须这点,因为完整的天外之魔,天意根本影响不了!”

为了防备天意。方源催动剑遁,不惜仙元剧烈的损耗!

幸亏方源将他积累的修行资源,多数保留下来了。

他现在日进斗金,夸张点说,是财大气粗。

虽然要建设和经营自家仙窍,但大多数的计划都没有实施呢。因为宝黄天关闭,他在琅琊派库藏中得到的东西,还是有限得很。

方源划破长空,一路上爆发出锐利的轰鸣声。仿佛整个人都成了一柄飞剑!

纵横荒原,锋锐无当。

忽然,一声鸣叫,充斥痛楚和惊惶。

一头上古荒兽,正在拼命逃生,追着它的是一只七转野生仙蛊。

两者就这样撞入方源的视野之中。

前面一头是独角六翼天马,后面那只野生仙蛊则是龙蜈。

四翼天马是兽皇,六翼天马是荒兽。额头生长独角,背插六翼的天马。便是上古荒兽。

而这只龙蜈,身形之修长,长达七里!蜈身龙首,数不清的节足,在扁平的身躯两侧,不断翻动。赤铜般的坚硬甲壳。在璀璨的光阳下,闪烁着耀眼的金光。十分威武,雄壮瑰美!

野生仙蛊有的寄生在荒兽、荒植的体内,不断汲取它们的营养,当做食料。生存下去。

这只是大部分的情况。

龙蜈是凶残的野生仙蛊,它直接捕捉荒兽、荒植,把它们当做狩猎的猎物,赖以生存。

这头天下唯一的龙蜈仙蛊,已经是七转程度,捕猎上古荒兽、荒植,填饱自家肚皮。

方源正在笔直飞行,这对追逃组合,就出现在他的正前方,恰好拦在他的路上。

“果然出现了幺蛾子!”方源心中冷笑不止。

上一次是云兽群,这一次是独角六翼天马和野生龙蜈仙蛊。

野生仙蛊本身就灵智很低,虽然有自身意志,但更容易被天意影响。

独角六翼天马是上古荒兽,马这类的猛兽,比大多数的更有些灵性。天意比较难影响。

但它此时逃生,慌不择路,天意影响起来,只需要一个念头,就能改变独角六翼天马的逃生方向。

这两者,都成了方源的拦路虎!

天马逃生,挡在它面前的一切障碍,它都会尽全力粉碎。

而龙蜈狰狞的目光,也投射到方源的身上。

方源的蛊仙气息四下洋溢。

他是六转蛊仙。

气息方面比独角六翼天马,要弱很多。但他的剑道气息洋溢,这来自于七转仙蛊剑遁。

这就比较吸引龙蜈了。

因为龙蜈本身,就富含剑道道痕!

它那数不清的足肢,都蕴藏剑道道痕,使得它随身携带成千上万只利剑。一旦缠绕住猎物,这些利剑上下左右前后夹攻,能将上古荒兽浑身上下,都扎出一个个的血洞,十分凶残。

方源反应极其神速。

他在瞬间收起了剑遁仙蛊,一下子气息消失。

龙蜈疑惑,但注意力还在。毕竟在它漫长的捕猎生涯中,也遇到过不少的能主动隐匿气息的猎物。

方源身躯一震,都出无数人影。

奴力合流,仙道杀招万我!

之前交易,方源得到许多力道仙蛊,他之前的最强杀伐手段又回来了!

虽然没有净魂仙蛊,但方源用了其他凡蛊替代。万我的作战威能下降了不少,但此刻用来惑敌,还是很好用的。

无数个方源,四下飞散,宛若深海中大规模的细小鱼群,忽然遇到了鲸鲨,四下飞散奔逃的样子。

龙蜈更加疑惑,速度都微微一滞。

然后它长达七里的漫长身躯,鞭子一般抽打。

砰砰砰。

无数的方源力道人影,都随之溃散。

独角六翼天马也沿途撞碎了不少方源虚影,飞速逃离。

龙蜈见天马要逃走,又将主要的注意力重新集中在它身上。

两者追逃出去,很快飞出了方源的视野。

方源吐出一口浊气。

他的真身已经下降到了低空处。

心念一动,还有无数的方源虚影,都在同时之间自行崩溃。

走!

方源再次催动剑遁,飞入高空,扬长而去。(未完待续。)

\end{this_body}

