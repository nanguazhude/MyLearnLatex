\newsection{羊探太丘}    %第六十六节:羊探太丘

\begin{this_body}

%1
方源一路日夜兼程,不眠不休。无弹窗,最喜欢这种网站了,一定要好评]←,

%2
见到浓厚的云层,他就采用血漂流。大多数时间,是用剑遁仙蛊。

%3
途中穿越了几个危险地带,有散修蛊仙的地方,或者是超级势力。方源就不敢明目张胆地应用仙蛊了,而是动用凡道杀招。

%4
他的策略相当正确,独角六翼天马和野生龙蜈的事件发生之后,就再没有其他意外出现。

%5
天意布局需要时间。

%6
另一方面,天意影响其他生命,也是有局限的。

%7
就像当初影响独角六翼天马和野生龙蜈,天意也不能直接命令龙蜈或者天马攻击方源。

%8
天意的影响,主要是因势利导。

%9
方源的速度十分迅速,一直让天意来不及布局。

%10
但天意还能思考,能谋算。

%11
这点主要归功于星宿仙尊。

%12
根据影宗提供的情报,天意在星宿仙尊之前,要容易对付得多。但在她之后,天意就变得相当“狡诈”。

%13
所以方源的飞行路线,也不是纯粹的直线。

%14
而是弯弯绕绕,曲折向前。

%15
若是直线,那么前进的路程就很容易被天意推测出来,沿途提前布局,等着方源一头扎进去。

%16
甚至,天意还会算出方源的真正目的地太丘,在那里重重布局。

%17
那就太危险了!

%18
方源此行的门派任务完成不了不提,自家性命都要岌岌可危。

%19
不过,在方源这般的前行下。天意始终来不及阻碍。

%20
当然,方源绝不会故意返程。万一天意的布局。在身后形成,自己本来已经飞走了。现在主动再撞进去,简直是蠢透了!

%21
这样一来,原本的路程就很遥远,现在的总路程就又延长了数倍。

%22
不过,总体而言,方源的速度还是极快。

%23
所以再遇到野生龙蜈仙蛊的三天之后,地平线上,太丘慢慢闯入他的视野。

%24
太丘在望了!

%25
漫漫的普通荒草,形成巨大的黄色平原。

%26
太丘的出现。就好像是一座浓绿的森林,高耸的玉塔楼群。在这片黄色的平原上,极其显眼。

%27
随着方源不断接近,浓绿的太丘在地平线上不断蔓延,在方源的视野中不断扩散。像是一滩在宣纸上迅速渲染的黛颜。

%28
最终,太丘蔓延整个地平线,覆盖方源的全部视野。

%29
海量的巨人草,覆盖在这片凶地上,里面兽影丛丛。虎啸狼吟震荡长空,时时传出。

%30
雄阔万千,一片蛮荒气象!

%31
到了这一刻,方源再对天意掩饰。已无用处。

%32
他速度减慢,停用仙蛊,直接进入太丘之中。

%33
和巨人草相比。他身矮体小。和整个太丘相比,就像是一座宫殿中。飞进了一只蚊子。

%34
茂密的巨人草,遮天蔽日。

%35
整个太丘。不止隐藏了多少的猛兽异植。

%36
前方传来隐隐躁动,似乎要酿成恐怖的兽潮。

%37
方源冷笑一声,催起暗渡仙蛊。

%38
仙蛊暗渡!

%39
能够遮掩气息,让人忽略,防备推算,一定程度上屏蔽天意。

%40
变形仙蛊!

%41
方源又化作一头盘山羊,融入到太丘之中。

%42
前方的躁动一阵混乱,又渐渐消散。

%43
“若天意有情,此刻定然是惊愕、震怒了罢。”方源笑了笑,又摇摇头,叹息一声,“可惜天意无情……”

%44
暗渡仙蛊,帮了方源大忙。

%45
当初,黑楼兰还是凡人蛊师时,就靠着这只仙蛊,掩藏了自己的大力真武体的气息,存活下来。

%46
十绝体的灾劫,为什么超出寻常?

%47
也是因为天意损有余补不足,不想十绝体这种存在影响了世界平衡。

%48
方源的至尊仙窍,更远超十绝体。灾劫之大,还要大过十绝体许多倍数。就连暗渡仙蛊也效果有限。

%49
方源忽然想到:“姜钰仙子是影宗成员,影宗对抗天意,暗渡仙蛊这类的工具一定不在少数。影无邪这次若真能逃离绝境,借助影宗残余势力,恐怕会柳暗花明,海阔天空。”

%50
他此时变化做盘山羊,这种荒兽在太丘中很是常见,虽然虚有其表,但融入得很完美。

%51
不像之前蛊仙之躯那样突兀,和太丘格格不入。

%52
很快,他就碰到了荒兽。

%53
一头金砂乌骓。

%54
这是一头大马。

%55
体格比方源变化的盘山羊,还要巨大几分。通体肌肉贲发,骨骼强健,皮发仿佛暗金浇筑,六个马蹄,蹄色乌黑深沉。

%56
金砂乌骓正在吃草。

%57
察觉到方源变化的盘山羊,金砂乌骓好奇地抬起头,上下打量了一下这个“不速之客”。

%58
荒兽都各自有各自的地盘,方源变作的盘山羊,就闯入了金砂乌骓的地盘里来。

%59
不过金砂乌骓并不食肉,性情也沉稳,而盘山羊也隶属于食草荒兽,在金砂乌骓的生存意识中,盘山羊是没有威胁的。

%60
不过它还是很警惕。

%61
一直盯着方源,直到方源绕过它远远离开,它这才重新低下头去,继续吃草。

%62
方源选择变化成盘山羊,自然是经过一番深思熟虑的。

%63
其实北原中,狼形荒兽也有很多,但若变成荒兽狼,就要遭到金砂乌骓的强烈反应了。

%64
跨过金砂乌骓的地盘,方源继续前行。

%65
他已经将那幅太丘地形图,深深记住。就算记不住,仙窍中也有不少信道蛊虫备份着。

%66
“第一个地点,就在东南方向上。”方源遥望。

%67
但巨人草越战越高,边缘处的巨人草高似大树,而往内部深入,巨人草就类比塔楼。太丘中央的巨人草。更是堪比山岳。

%68
盘山羊的体型也不小了,但随着方源不断深入。周围的巨人草越发高大茂密,让盘山羊都显得娇小起来。

%69
一路上。方源碰到了不少荒兽。

%70
这些荒兽有的独自行走,有的三三两两。大部分是食草的,但也有食肉的。

%71
方源开动脑筋,他的智慧可是荒兽不好比拟的。

%72
所以,费一些周折后,都跨越过来。

%73
咩咩咩……

%74
连绵不绝的羊叫传入方源的耳膜。

%75
前方的草丛中,隐藏着一大群的盘山羊。

%76
方源心中惊叹:“看来我已经算是深入太丘了,这是我见到的第一支较大规模的荒兽群。”

%77
这批羊群,数目有近百头。成群结队,有的在吃草,有的躺在宽阔的地形上晒太阳,还有的比较年幼的,在相互嬉闹。

%78
方源步入它们的视野。

%79
一只只野生的盘山羊,纷纷注视着他。

%80
这个陌生的同类,它们还是第一次见。

%81
方源小心翼翼。

%82
他虽然有变形仙蛊,但也有隐患。若是这群羊中偏偏有羊的身上,寄生了什么侦查野生仙蛊。能勘破方源的伪装,那就倒霉了。

%83
为了保险起见,方源又暗中催动态度蛊。

%84
顿时,羊群感受到方源的“态度”。望向他的目光顿时缓和了许多。

%85
方源走了几步,甚至就有两三头年幼的荒兽盘山羊,主动跑到他的面前。瞪着好奇的眼瞳近距离地看着他,还有两只围绕着方源奔跑。蹦蹦跳跳。

%86
方源安然走出来,离开羊群的视野。

%87
年幼的盘山羊跟随他走了一段路。逐渐远离羊群时,被它们的父母一阵叫唤,又唤了回去。

%88
这让方源暗暗可惜。

%89
若是能拐走这两三只幼羊,将它们活捉到至尊仙窍中放养,该是多好。

%90
遗憾的是,他没有下手的机会。

%91
离开盘山羊群的地盘,方源距离地形图身上标注的第一个地点,已经十分接近了。

%92
他没有大意,查看了一下自身。

%93
“暗渡仙蛊的遮掩力量,正在减弱,不过还能支撑一段时间。”方源做了精准的估算。

%94
暗渡仙蛊催动一次,就要休息许久,方能继续催动。

%95
它的力量遗留在方源的身上,为他遮掩气息。但随着时间流逝,或者其他外力,不断削弱减少。

%96
并且施加对象不同,护持的效果也有差别。

%97
比如对黑楼兰施展,效果就很好。对暗道蛊仙施展,效果就下降一些。对其他流派的蛊仙施展,效果就更降一筹。幸好方源的一身道痕互不干扰,否则他的效果就差得可以了。

%98
总之,方源必须趁着暗渡仙蛊的遮掩力量仍在的时候,将手上的这个任务完成。

%99
否则的话,力量消散,天意关注,让太丘中无数的荒兽、上古荒兽群起围攻方源,那方源的下场可就极度堪忧了。

%100
片刻之后,方源接近了第一个标注点。

%101
还未到达,方源就知道这个位置,已经彻底改变了。

%102
因为地形图上记载着,这里曾经躺着一具太古荒兽级的气宗狮的尸骨。

%103
护卫尸骨的,是一大群的气宗狮。有荒兽,也有上古荒兽。

%104
但现在小山般的尸骨,早已经不翼而飞。气宗狮群也没有了,方源动用侦查杀招,听到许多隐约的狼嚎声。

%105
方源心中微微一沉。

%106
“荒兽黑血狼!而且这个规模……至少有三十多头。”

%107
一支小型的荒兽狼群。

%108
虽然数量较少,但却是食肉的猛兽。

%109
狼群的领地,更要广阔。

%110
方源不用近距离观看,只是动用凡道侦查手段,靠听觉,就知道了很多情报。

%111
这就是人的智慧。

%112
这群荒兽狼群中,说不定也寄生有仙蛊,可惜不能自主运用。和掌握仙蛊的蛊仙相比,完全是两种概念。

%113
方源沉思:“第一个标注点,已经消失了。不管是太古荒兽级的气宗狮尸躯,还是那群荒兽、上古荒兽的气宗狮群。也不知道是怎么消失的,何时消失的,总之三十多万年后,这里由一支小型的黑血狼群占据。”

%114
方源没有犹豫,直接放弃了这里。

%115
他绕过这片狼群的领地,开始向第二个目的地进发。

\end{this_body}

