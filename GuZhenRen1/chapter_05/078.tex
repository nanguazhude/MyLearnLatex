\newsection{变化道宗师!}    %第七十八节:变化道宗师!

\begin{this_body}

至尊仙窍之中。

小东海。

这里是一片河泽地貌。

只是没有外界东海那般浩大深邃,最高的水面,也不过五六丈而已。

绝大多数的地方,相对于荒兽而言,简直是小泥坑了。

这是因为至尊仙窍中的水道道痕,还是不多。

此时,在小东海的一个角落里,两头荒兽正在对战。

嗷!

荒兽鱼翅狼长啸一声,飞扑而来。它张开大嘴,露出尖锐的獠牙,这一口要是咬实在了,山石都得被咬得崩碎。

但它面对的对手,一头荒兽飞熊,却是双眼闪过狡诈的光,猛地后退。

鱼翅狼用力过猛,发出的攻击被飞熊轻易闪过。

然后,飞熊高高扬起厚实的熊掌,照准鱼翅狼的脑袋,猛地一拍。

砰的一声巨响,巨大的力量,让鱼翅狼狠狠地掼到地上,摔得七晕八素,一时间居然爬不起。

不过方源,却对这击不太满意。

“我方才动用飞熊之力仙蛊,力量的确大增。但是飞熊变化,却险些出了差错。我暂时借用的这具仙僵身上,全身力道道痕十分丰富。虽然有变形仙蛊,但只是强行变化。力道道痕始终抵抗变形仙蛊的影响。”

变化道蛊仙,身上的道痕,主要是变化道痕。

这种道痕,可以在蛊虫的影响下,一次次变化成其他种种道痕。

变形仙蛊是变化道的精髓蛊虫,有了它,就能让蛊仙的形态发生各种各样的变化。花鸟鱼虫。草木山水,日月星辰。皆可变化。

但此蛊只具其形,还得搭配其他蛊虫。才有力量、速度、防护等等方面的改变。

“现在的至尊仙窍在我肉身之中,我只能借助力道仙僵,锻炼变化道的各种手段。但力道仙僵身负力道道痕,施展变化仙蛊十分费劲。倒是飞熊之力,却有力道道痕增幅,效果更佳一些。”

方源暗中比较,越发觉得自家肉身的妙用。

道痕之间不受干扰、阻碍。

这个优势实在是巨大无比!

各个流派之间,边际还是相当清晰的,可谓泾渭分明。就算是兼修。也必须分清主次。但到了方源这里,却是流派之间的边际变得模糊一片。他可以全派通修。

“看来用力道仙僵,也只能马马虎虎熟练变化手段而已,并不能让我真正了解到自己的进步还有能力极限!”方源心中一边叹息,一边遥控飞熊,灵活躲闪。

原来,那头荒兽鱼翅狼又爬了起来,凶性大发,疯狂攻击。

但方源有人族的智慧。进退有据,战术分明,操控得飞熊似行云流水。鱼翅狼几乎擦不了飞熊的边。

虽然方源曾有奴兽仙蛊,驾驭过这头荒兽。但地灾一过,方源要将奴兽仙蛊归还。

所以,在归还之前。他就将对鱼翅狼、刺脊星龙鱼的控制,都给解除了。

这也有好处。

没有奴兽仙蛊的影响。荒兽鱼翅狼就能展现出真正的凶性,充当合格的陪练。

砰。

又一声巨响。鱼翅狼已经不知多少次,被飞熊的大熊掌拍进泥坑当中,飞溅出无数泥水。

这一次,鱼翅狼呜咽一声,踉跄着爬起来,看了一眼毫发无损的飞熊之后,居然夹着尾巴,转身逃了。

还在期待鱼翅狼攻击过来的方源,为之一愣,不禁心道:“难道是当初收服鱼翅狼时,弹它小**弹得太狠了些?让它的凶性都为之大减了吗?”。

鱼翅狼逃了,方源一时间也觉得索然无味。

“不过这些天的对练,也并非没有收获。至少证明了我的变化道境界,的确已经成为宗师。”

变化道宗师!

既方源的血道宗师境界、力道宗师境界、智道宗师境界、星道宗师境界之后,第五个流派提升到了宗师境界!

前世五百年的经历,并未带给方源什么帮助。变化道之所以能有如此境界,最大的功臣就是狂蛮真意。

天意要铲除方源,不断谋划,将灾劫威能提升到天道允许的极限。

方源渡过了两次地灾,但这地灾的威力,早就超越寻常,能和诸多天劫媲美。所以连带着狂蛮真意,都被勾出许多倍来。

福兮祸所伏,祸兮福所倚。

灾劫这玩意,讲究的一个规律,就是福祸相等。

灾劫越是沉重浩大,方源越是危险困难,但渡过之后,收获得好处就越多!

方源的变化道宗师,前后经历了三次狂蛮真意灌溉。第一次是黑楼兰十绝升仙,后两次是自家地灾,通过仙灾锻窍杀招牵引出来。所以能达到宗师境地,也是积累到了。

“成为变化道宗师,虽然可喜可贺,但目前局势,却是相当不妙。”方源忧心忡忡。

灾劫一次比一次厉害。

渡过第二次地灾,方源已经大概清楚接下来灾劫威能会增加多少。

第二次地灾,他是十分惊险,才算渡过去的。

中途,雪月衍生出来,若非方源明智决绝,立即采取进攻措施,后果不堪设想。最后阶段,靠着荡魂山都被削成土丘,仙元稀少到连一个仙道杀招都发出不来,靠着荒兽这个伏笔,才堪堪过关。

可以说,方源的底牌差不多用尽了。

“第一次地灾,狂蛮真意分割出了灾劫的小半力量。天意酿造的雪怪,并未都未占据主动。”

“第二次地灾,天意就计算周详,连狂蛮真意的因素都算计在内,甚至是利用它,影响它变化成雪月,反过来增添风花劫的威能。”

“天意能够思考,第三次地灾。它必定考虑到荡魂山,我现有的仙道杀招。还有荒兽、蛊仙周中这些下属。”

知己知彼,方能百战不殆。

方源一次次渡劫。对天意了解更深。

反过来,天意也是如此,对于方源的情报也试探得越加清晰。

目前为止,方源渡劫的手段,主要有七个。

第一个是在北部冰原渡劫,利用仙劫锻窍杀招,引导出狂蛮真意,分化天意对灾劫的掌控。

第二个是狗屎运,用运道来虚弱灾劫。

第三个是态度蛊、暗渡蛊、变形蛊。还有见面曾相识,变化万千,迷惑天意。

第四个是力道、剑道、血道种种仙蛊,以及仙道杀招,是正面抗争的手段。

第五个是荡魂山,巨大的地利。

第六个是完整的天外之魔身份,天意无法影响到方源的思维。

第七个则是一些外力。比如刺脊星龙鱼、鱼翅狼这些荒兽,羽民蛊仙周中,琅琊派的毛民蛊仙等等。

“这些手段、底牌。天意都已经一清二楚。最多第七点外力方面,有些变化。”

“但这些外力,并不可靠。”

“我奴道境界不高,奴役手段也不擅长。刺脊星龙鱼、鱼翅狼这些。既能被我奴役,也能被天意利用灾劫,而重归自由。所以渡劫时。我只将这些埋伏在外,就是怕被天意反过来利用。”

“还有羽民蛊仙周中。他是异人蛊仙,有充足的灵智。若是重获自由,比荒兽危害更大。不到万不得已,我是不会将至尊仙窍的秘密,暴露给他的。更关键的是,他曾经是通过天随人愿,以地灾的形式,降临到太白福地中的。若是他像是雪怪那样,被天意轻易影响、策反,那就糟糕了。”

“按照现在的节奏走下去,我第三次地灾就是九死一生的格局,第四次地灾直接就是死透了的绝境。”

方源很清楚自己的处境。

第二次地灾虽然渡过去了,但宝黄天不开,他的修行没有大量资源助推,不会有多大的起色。

蛊仙渡劫就是这种情况。

灾劫威力一次比一次强大,蛊仙修行必须加紧努力,使得自己实力上涨的程度,要超过灾劫增强的幅度。

资质、底蕴越差的蛊仙,灾劫威力就小得多,间隔时间也很长,修行下去并不十分困难。但方源的资质、底蕴是古往今来的第一,至尊仙胎,全派通修!

所以灾劫才如此恐怖,威力提升的幅度骇人听闻,相互间隔的时间更短到只有两个月!

换做其他蛊仙早就死翘翘了,也就是方源自身素质过硬,还千辛万苦保留下了一笔不菲的修行资源,才支撑到了现在。

想想方源拥有多少仙蛊,多少仙道杀招,多少流派的宗师境界,还有荡魂山、琅琊派等等可以借助的外在因素。

可即便如此,方源的前景也变得很不妙。

“宝黄天的关闭,彻底打破了我的计划和节奏。这样下去,我是没有活路的。”

困扰方源的就在于宝黄天。

宝黄天若是开启,他一切就都顺畅了!

但宝黄天究竟何时开启?

谁都说不准。

也许就在下一刻,也许在数月之后。

就方源的情况而言,若是数月后宝黄天开启,那他绝对是要死翘翘的。除非另有奇遇,使得实力暴涨。

一天一天的等待,宝黄天始终不见丝毫动静。

方源的伤势早已经养好,仙窍的经营却是毫无寸进。

琅琊派的发展,也一直受挫。琅琊地灵数次召唤方源,与他商量讨伐落星犬的事情。方源自然没有答应,双方关系一降再降。

琅琊地灵将方源指点毛民蛊仙战斗的任务奖励,更是降低到最低点,让方源几乎毫无收益。

在毛六的挑唆下,其余的毛民蛊仙,也开始对方源冷淡,甚至仇恨起来。

第二次地灾虽然过去,但方源的处境却每况愈下。

究竟是坚持等待宝黄天开启,还是穷则变,变则通?

方源现在面临的,就是这样的难题。

\end{this_body}

