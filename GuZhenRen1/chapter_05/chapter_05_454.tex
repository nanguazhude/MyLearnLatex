\newsection{财富蛊}    %第四百五十五节:财富蛊

\begin{this_body}

%1
西漠。

%2
这里是一处戈壁,地势平坦,一望无际。

%3
到处是龟裂的石质地面,寸草不生。

%4
不过在地下深处,却有别有洞天。大量的矿脉,以及搭建好的矿场。

%5
上十万的石人,在这里生存,充当矿工,为蛊仙开采各种矿物。

%6
这里就是西漠大名鼎鼎的万矿戈壁,由超级势力之一的石家掌控。

%7
这里的矿物,种类极多,又储量丰富。

%8
除去石人矿工开采的凡级矿物之外,还有许多的仙矿,需要蛊仙自己动手,才能开采得动。

%9
比如暗流金钢,这是六转仙材中最坚硬的。又比如七转仙材乌青墨石,就算是上古剑蛟的龙爪狠狠一抓,都不会留下什么痕迹。

%10
而在这万矿戈壁的底下深处,更有一条壮阔的碎金河流。

%11
碎金河乃是太古黄天中的金属河流,而这片万矿沙漠正是当初有一块巨大的太古黄天碎片,掉落于此。

%12
太古黄天碎片一度成为蛊师、蛊仙探索的宝地,但沧桑岁月,时代变迁,太古黄天碎片越来越小,最终彻底消散。

%13
在这个消散的过程中,它流散出去的道痕,将周围的戈壁逐步改造成了万矿戈壁。而这条碎金河流正是太古黄天碎片中的产物,一直流传至今,仍旧没有枯竭。

%14
方源的小黄天中也有一条碎金河流,但它的规模非常的小,和万矿戈壁地下的这条碎金河流相比,完全是小巫见大巫。

%15
就在这条壮阔的碎金河流中段,建设了一座仙道蛊阵。

%16
蛊阵始终在徐徐的运转着,监控着整个万矿戈壁,更统领无数石人矿工。

%17
砰!

%18
忽然一声巨响,这座仙级蛊阵猛地爆闪出了刺目的光影,随后蛊阵裂开一道缝隙,打开了门户。

%19
一阵浓烟,随即从门户中袅袅飞出。

%20
“咳咳咳。”从五彩斑斓的浓烟中,走出一个石人蛊仙。

%21
不过随着他几步走了出来,他浑身上下的石皮都一块块地脱落下来,最终露出真身。

%22
这是一位纯正的人族蛊仙。

%23
他中年模样,目光沧桑,容貌普通,给人老实巴交的印象。

%24
此时此刻,他灰头土脸,不断咳嗽。

%25
“第八十九次炼蛊,还是失败了啊!”中年蛊仙叹息,不断向外吐血。

%26
他吐出来的血块,非常奇特,落到地上后,居然凝结成各种宝石、珍珠、玛瑙、玉石乃至黄金、白银、青铜等物。

%27
即便是方源在此,以他的眼界,见到这样一幕,恐怕也会惊奇。

%28
究竟是什么原因,令这位中年蛊仙受到如此奇怪的伤势呢?

%29
作为当事人的中年蛊仙,见到脚边的各种珠宝,神情却很平淡,一点都没有意外。

%30
因为他知道自己在干什么——炼制财富蛊!

%31
财富蛊,这是《人祖传》中明确记载的传奇蛊虫。

%32
按照人祖传书中的字句记录,它是由蓝海海水、人祖双手、忧患蛊、辛苦蛊、悲伤蛊、智慧蛊、愚蠢蛊参与进来,才炼出的蛊虫。

%33
不过,蛊仙若真的要完全按照书上的方法炼蛊,基本上是没有可能了。

%34
这不仅是因为,人祖传本身是一个故事,很多内容都有着隐喻,并非纯粹的字面解释。而且这些蛊材,在当今的情况下,根本无法收集。

%35
方源倒是有九转智慧蛊,可惜还未炼化。

%36
但蓝海早已经失踪,蓝海海水从哪里寻求?

%37
就算蓝海海水能有,人祖双手呢?

%38
但这位中年蛊仙,为何又在尝试炼财富蛊呢?

%39
还是那句话,蛊是天地真精,人是万物之灵。人可以为同一只仙蛊,创造出无数的蛊方。

%40
《人祖传》中记载的炼出财富蛊的方法不能用,那么蛊仙完全可以自己改良啊!

%41
这位中年蛊仙专修土道,但早年撞见过一场机缘,得到了一份仙蛊残方。

%42
也不知道是历史上的那位先贤,研究财富蛊的炼法,从而留下了这份残方。

%43
中年蛊仙得之,如获至宝,因为他看出来这份蛊方价值极高,留下这份蛊方的先贤贡献极大,只要照此发展下去,真的有可能炼出财富蛊来。

%44
在这份残缺的蛊方上,这位先贤甚至根据蛊方,推算出了财富蛊的威能效用。

%45
财富蛊若是炼制出来,就可以消耗自身,转变出天下任何一件事物。这种事物可以是金银铜铁,可以是六转仙材,甚至可以是九转仙材。

%46
但财富蛊无法转变出任何一种生命。生命涵盖动物、植物、人、异人以及蛊。

%47
“我若是炼出财富蛊,就能转变出天底下任何一种仙材,大发利市。只需转变一些已经绝迹或是极其罕见的蛊材,就能很快富可敌国,甚至修行资本积累庞大,能够凌驾于整个石家!”中年蛊仙怦然心动,一直努力试图炼出财富蛊。

%48
他始终秘而不宣,防止有心人的觊觎。

%49
这位中年蛊仙,姓石名忠。

%50
石家乃是西漠中,正道的超级势力之一。

%51
石忠在石家里的地位十分尴尬。他虽然有七转的修为,但却遭受石家蛊仙的排挤。

%52
这个有历史渊源,石忠的父亲叛变家族,导致石家受损严重,不仅是资源,更有蛊仙牺牲。

%53
因此,石忠从幼年时期,就不受石家蛊仙的待见。

%54
为了取得信任,石忠亲自为自己改名,姓石名忠,以此表明自己忠于家族的心迹。但仍旧无济于事。

%55
他被石家太上大长老打发过来,镇守这片万矿戈壁。

%56
这里物产丰富,但背靠着其他超级势力,和石家交恶,石忠镇守这里的风险很大。并且油水也很少,因为这座仙阵时刻监管,极其严格,没有石忠下手摸鱼的机会。

%57
事实上,就算监管不严格,石忠也不敢动手。一旦被发现,他的处境会更加糟糕。

%58
正因为有这样的情况,让石忠对财富蛊更加期待。

%59
“蛊仙修行,最根本的还是经营自家的仙窍。没有坚实雄厚的资本,就算战力再高,也是一时的,宛如万丈大厦建立在沙土之上。”石忠有着这样的觉悟,一直都在努力尝试,过程自然艰难困苦,但他始终没有放弃。

%60
回顾他此前的蛊仙修行,除了渡劫等必须之外,他的其他资源都投入到炼制财富蛊这项大计划中。

%61
“第八十九次失败,我看看,人已经不足了。”石忠治疗好自己的伤势后,检查库存,发现作为蛊材的人,已经不够数量,需要再次收集。

%62
石忠乃是正道蛊仙,居然拿人来充当蛊材炼蛊,这绝对是魔头行径。

%63
不过只要秘而不宣,就没有人知道,石忠没有把柄落到别人手中,谁都无法向他发难。

%64
石忠在这一方面,非常的谨慎。他绝不会自己动手捕捉人族,留下什么蛛丝马迹。他手中的人族奴隶,都是直接在宝黄天中收购,手脚非常干净。

%65
“我用人来替代财富蛊古方中的蓝海海水,已经人祖的双手。这一点应是无错的。只是是否程度不够,需要用寿蛊掺和进来呢?”

%66
石忠在思考。

%67
他手中也有寿蛊,但不敢随意乱用。

%68
“《人祖传》中明文记载,虽然智慧蛊不像愚蠢蛊那样牺牲,但也参与了炼蛊的过程,我没有智慧蛊,用了其他智道蛊虫替代,这方面大有改进的空间。”

%69
忧患蛊、辛苦蛊、悲伤蛊,这些凡蛊,石忠都有,并不难收集。

%70
甚至,愚蠢蛊他都能自己炼制出来。

%71
“蛊方正一步步接近成功,如今已经相当完善。”石忠疗伤完毕,开始检查自己吐在地上的血。

%72
这些血已经彻底转变成了各类珠宝、矿石等等。

\end{this_body}

