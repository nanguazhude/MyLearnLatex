\newsection{新道痕、尊者路!}    %第八百五十节:新道痕、尊者路!

\begin{this_body}

疯魔窟,第六层。

狼狈不堪的疯魔二怪倒在地上,秘谋人跪在地上,暂且不顾自身伤势,给不是仙疗伤。

不是仙缓缓睁开双眼,苏醒过来,他先是懵懂,旋即反应过来,无比的庆幸和感激道:“我居然逃出来了,老大,你又救了我一命!”

随后,他疑惑地问道:“怎么不见方源?等等,难道他没有及时撤退吗?”

秘谋人苦笑:“没有。他现在还留在第七层。不过,分别时他说的也没有错,在他那个距离要逃脱出去,已经是来不及了。”

不是仙愕然无语,半晌之后,才叹息道:“时也命也,没想到堂堂方源,居然折损在了这里。”

即便方源如此声威赫赫,他也不认为,方源会有多少生还的希望。

因为不是仙深知疯魔窟的可怕威能!

秘谋人倒是完全冷静下来,神色带着一丝古怪:“说来不可思议,我倒是有一丝玄妙的感觉,总觉得方源能够存活下来。虽然理智告诉我,这没有任何的可能性,我也无法推算出任何他能够生还的手段。但是,心中就是有这种感觉。”

“老大,你可是智道蛊仙呢。”不是仙也呆了呆。

“即便是智道蛊仙,难道一切的感情都是对的?”秘谋人苦笑摇头,“或许这只是我对方源下场的一种感怀吧。他是如此之强,超越你我,震惊天下,连天庭都拿他没有办法。但是这样的人物,却要折损在疯魔窟内。唉,或许他的下场,就是你我将来的下场。我是有一些同病相怜了。”

不是仙哑然。

半晌,秘谋人开口,打破了沉默:“走吧!退回去,好好休养一番。等魔音彻底停息下来,我们就去第七层搜寻方源去。”

不是仙点点头,尽管他明白方源几乎是十死无生。

他勉强站立起来,身体猛地一个踉跄,就要栽倒下去。

幸亏一旁的秘谋人及时地扶住他。

秘谋人将他架住,两人一边抵御着魔音贯耳,一边缓慢前行。

前方的路,隐没在黑暗中。

两人的身影逐渐融入黑暗,皆是步履蹒跚。

他们根本不知道,在第七层中方源不仅没有遇害,而且还有了新的发现。

“这是新的道痕吗?”方源双眼放光,神色颇有些惊奇。

在这斑斓的道痕波涛之中,他开始陆续发现一些全新的道痕。

方源早已今非昔比,各个流派境界皆有巨大提升,又前后继承了数个尊者真传,眼界开阔得很。

但现在,他可以肯定,眼前的这股道痕他从未见过。

这是全新的道痕!

“原本这些道痕静止的时候,这些全新的道痕因为数量太过稀少,分散各处,并不起眼。但现在魔音呼啸,各种道痕不断转移。又因道痕互斥,相同流派的道痕逐渐统一在一起。这才使得这些全新道痕规模渐大,让人更容易发现。”

方源迅速总结出了原因。

魔音刚刚开始的时候,各种道痕四处交杂,随意编织,错综复杂,色彩斑斓点缀。

当魔音响彻了一段时间之后,各种流派的道痕相互凝聚在一起,形成统一的色彩,比之前清明许多,更显得波澜壮阔。

在方源的前面,一股金道道痕,宛若金色巨蟒,蜿蜒游走。在方源的左手边,一股魂道道痕更加庞大,宛若黑云,大若城墙,泱泱推进。而在方源的脚边,就是那一串全新道痕,仿佛是一弯溪流,潺潺流淌。

方源微微停下脚步,仔细观察它。

不久,他流露出惊异之色。

从这股全新的道痕中,他“看到”了刀剑,他“嗅到”了斧钺,他“摸到”了锤枪,他“听到”了镖箭……

方源神情微肃,暗道:“这股全新的道痕,代表着一个全新的流派。这个流派颇为大气,竟然包含刀剑斧钺种种。将来若有人开创出来,必定是将当下的剑道、刀道都容纳吞并进去。这道新流派的战力必定很强,同时又明显借鉴了人道的许多精妙。”

不过旋即,方源又微微摇头:“然而,要开创出这个流派,不知道要多少代人,又要等待多少的时间。”

“或许有人已经早有了开创这个流派的想法和底蕴,他(她)若是能目睹这份景象,必定能带给他(她)巨大的帮助。又或许世间还无人有这样的想法,这个流派会一直掩没,就像是一份宝藏,被深深埋在地底深处,从不会有人开启。”

方源没有什么遗憾惋惜的想法。

这种情况是蛊界的常态。

人们常说怀才不遇,人才需要伯乐,其实一个优异的流派也需要有才情的人才能发掘出来。

方源更不想去借此开创什么全新的流派。

他现在手头上的流派,已经让他学不过来。而且开创新流派,需要的时间、精力实在太多,太大。他也没有这个闲工夫。

方源继续前行。

一路上,他又看到了不少新的道痕。

有一股道痕,凝聚成团,饱满圆润,方源仿佛从中闻到了草木的清香,仔细再闻,又仿佛是食物的香味。

方源猜测,这是否是传闻中的丹道道痕呢?

还有一片道痕,平铺下来,宛若地毯。明明是同一种道痕,但上面却是光影交织,时而如山水,时而如鸟兽。

当其他的道痕洪流卷席过来的时候,这片道痕就像是风中的树叶,也会被卷席而起,轻飘飘地附着在其他的道痕洪流的表面。既不影响这些附着的异种道痕,又不会让微小的自身崩溃。实在是奇妙至极。

方源第一眼看到的时候,便心头微震:“不出意外,这便是画道道痕了!”

他有心将这道痕占为己有,但如今魔音轰鸣,他却是没有手段的。

“房家那边倒是有盗天魔尊的传承,有那座仙蛊屋贼巢。若是利用它,我能否将这里的道痕窃为己有呢?”方源心中一动,来了灵感。

疯魔窟这里的仙材俯拾即是,但统统没有利用的价值。

究其根本,乃是因为这些仙材中蕴藏的道痕,都是杂乱无章的各种道痕。

然而仙材无用,却非是道痕无用!

方源将计就计,对房睇长下手。如今的房睇长,已经是他的分身。

从房睇长的魂魄中,方源搜刮出了房家几乎所有的情报。其中当然包括贼巢。

这的确是盗天魔尊的仙蛊屋,只是残破得很,因为涉及偷道,房家并不擅长。房家当初得手时,也没有得到仙蛊屋的完整内容,所以房家一直都难以修复。

如今的贼巢,只有一种手段,就是以一份偷道仙材为引子,将仙材中的偷道道痕瞬间消耗,然后将外界的道痕偷取出来。

偷取出来的道痕,必须附着在这一份偷道仙材之上。最终,这份仙材充斥杂乱无章的各种道痕,再无利用价值。

这和疯魔窟中的仙材情况,很是类同。

贼巢有盗取外界道痕的能力,但却没有利用的手段。这正是这个主要原因,方源没有想方设法将贼巢带在身上。

“但是这种情况下,却不一样了。”

“当这些相同的道痕凝聚在一起的时候,我动用贼巢,就能制造出相应流派的各种仙材了。”

“这些仙材当中,都是相同的道痕,应该是可以利用的。”

方源念及于此,胸口也不由地微微发热。

若是此法能行得通,那么他就拥有了一个能源源不断,制造八转,甚至九转仙材的渠道!

这是一笔多么庞大的财富!

即便是方源也难以估量。

制造出的仙材,不仅是用于贸易,用于炼蛊,而且还能参悟出相应的流派。比如利用这些全新道痕,就能凝聚出画道、丹道等等的仙材。这些仙材对于开创流派,具有极大的参照作用。

“不过,同一种道痕凝聚起来,必须是在魔音肆虐的时候。”

“这个时候,催动仙招,风险极大。仙招也是道痕的排列,很有可能会道痕混乱,仙招催动失败,引发反噬。”

“更别说蚁巢这座仙蛊屋,恐怕是不能直接拿出来的。它可不是至尊体,拿出来就要被排斥、碾碎!”

方源冷静下来。

要完成他的这个设想,还有巨大的技术难关。

他现在的雄厚底蕴,傲视世间,但也没有任何的把握。

最终,方源叹了一口气,将这个想法暂时放下,继续前行。

他开始加快脚步。

不管前方是什么道痕的洪流,对他而言,都像是无害的光影,如拂面的轻风。

不久后,他又看到了梦道的道痕。

疯魔窟中可谓万象千罗,梦道道痕也被它囊括在内。

方源谨慎地停下脚步,试探了一下,发现梦道道痕对他而言,同样人畜无害。

这只是道痕,而不是梦境。对于至尊仙体来讲,更类似于一种流动的仙材。

若是梦境,方源就要绕路了。

方源笔直地往前走,一路畅通无阻。

魔音响彻耳畔,给他造成了一些困扰。好在他有私底下改良的优秀手段,这个杀招很有讲究,催动一下,就能有长期效果。虽然魔音干扰,以往的长期效果也变得很短暂。不过,一切都在方源的承受范围之内。

就算他被魔音侵蚀,发疯嗜杀,但这片绝地也没有其他生命能够威胁到他。

哪怕是他一头撞在山石上,至尊仙体本身的素质也能确保他无恙。

周围是汹涌的七彩光澜,毫无规律,时而如海啸遮天,时而如马群奔腾,时而如瀑布倾泻,时而如鸟雀纷飞,而方源在当中飞奔。

若是疯魔三怪看到这一幕,恐怕下巴都要惊得掉到地上来。

这不是颠覆他们的世界观,而是直接把他们的世界观粉碎!

方源深入,走到疯魔三怪难以想象的深处。

当他终于靠近第八层的入口时,他轻咦一声,脚步微顿。

他竟看到了三条路!

一条路金光灿烂,气运蒸腾。

一条路阴森恐怖,杀意弥漫。

一条路朴实无华,厚德载物。

方源眼冒精光,瞬间明白过来:“这三条路恐怕是尊者的痕迹,各自充斥运道、魂道、土道道痕,应该是各自对应巨阳仙尊、幽魂魔尊以及乐土仙尊!”

这三位尊者都来过这里!

魔音浩荡,道痕混乱,形成惊悚的乱象,威能恐怖,能把几乎所有的蛊仙都剿灭得干干净净,一丝残渣都不会留下。

但这三条路上的道痕,却是坚定无比。即便周围各种道痕的洪流,在时刻摩搓、卷席、冲撞,这三条道路上的道痕,都是岿然不动。只是偶尔最边缘的地方,有些道痕的剥离,仿佛是碎片残渣。

方源双眼微微眯起。

他知道自己拥有至尊仙体,道痕不斥,所以他能来到这里,这是取了巧。

但尊者不一样!

尊者身上的道痕都是互斥,他们能够来到这里,是硬生生地闯荡开辟的结果。

这是三位尊者的无上风采!

疯魔三怪这等人物,才情、境界已经是世间少有,但仍旧需要找路来走。

尊者完全不需要。

他们双脚踏出去,就是一条无上通路!

有尊者来过这里,方源并不奇怪,事实上他已经有所预估。

毕竟无极魔尊可是上古时代的人物,历史上的第三位尊者,仅次于元始、星宿。他布置下来的疯魔窟长存至今,虽然隐秘,但疯魔三怪能发现这个秘密,其他人为什么就不行呢?

作为蛊仙当中的至尊,这些尊者一个个惊天动地,发现这里也稀疏平常啊。

指望价值无双的传承、遗产,无人发现,只为留给自己,那是涉世不深的年轻人天真无知、自以为是的想法。

人都是平凡的,但每一个人也都是独一无二的。自己有这样的机遇,为什么别人就不会有?

“但是,为什么只有这三位尊者的痕迹?”

“这三位尊者留下的路,为什么只在第八层的入口附近显现呢?”

脑海中有一些疑虑,方源动作却不停,他直接跳进第八层的入口中。

眼前的黑暗一闪即逝,下一刻,方源就脚踏实地。

“怎么会?”方源愣住。

疯魔窟第八层的景象,大大出乎他的意料,根本不是第七层的景象。

------------

\end{this_body}

