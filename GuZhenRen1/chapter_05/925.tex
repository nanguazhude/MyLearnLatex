\newsection{剑毁仙墓}    %第九百二十九节:剑毁仙墓

\begin{this_body}

%1
嗷吼——!

%2
横霸天穹的太古巨龙长鸣一声,龙尾一甩,身躯游动,速度之快竟在原地留下了一道淡白残影。

%3
轰。

%4
一声巨响,诛魔榜遭遇一记狠狠的冲撞,宛若流星般坠落下去。

%5
龙公立即赶来支援。

%6
方源冷笑一声,龙躯陡然升空,再次在原地留下一道残影。

%7
在高空,他猛地从极动变为极静,身上气势猛地一爆。

%8
仙道杀招——剑浪三叠。

%9
至少增幅了一千五百倍的攻伐杀招,威能和之前完全是天翻地覆的差别。

%10
巨浪滚滚,倾覆天下。每一滴银白的浪花,都是剑道的力量,璀璨耀眼,锋利至极。

%11
剑浪没有冲向诛魔榜,也没有罩准龙公,而是扑向仙墓。

%12
至始至终,方源的攻击首要目标,仍旧是仙墓。

%13
龙公咬牙,不得不再次施展自转游龙气墙。

%14
气墙挡住剑浪,但支撑了八个呼吸便轰然瓦解。

%15
龙公闷哼一声,鼻孔中流淌出血液。

%16
他还想继续催动气墙杀招,但滔天的剑浪已然冲刷到了仙墓。

%17
之前天庭还能从容防守,完全是依赖于苍玄子的宇道手段,能够拉长距离。这就能让龙公有充沛的时间,不断催动防御气墙进行防御。

%18
但现在苍玄子气息全无,在地上躺尸,没有了宇道手段遮护,龙公完全来不及再催动气墙杀招。

%19
龙公并非九转,和方源同样的修为,都是八转。但方源的道痕远超过他,龙公所能依仗的便只有龙御上宾杀招。但遗憾的是,就目前为止,龙公虽然不断地变强,但和方源仍旧有一些差距。尤其是当方源真正放开手脚,展现出最强的攻击姿态时。

%20
“不——!”龙公发出悲凉的怒吼,眼睁睁地看着剑浪卷席仙墓,冲刷一切。

%21
但凡沉眠在仙墓中的蛊仙,都毫无反抗之力,被剑浪冲刷成渣。

%22
即便是躺在地上的苍天藤,巨大的身躯也在锋锐至极的剑浪下不断消融,迅速缩减。整个过程,苍玄子都没有惊醒,看来真的是凶多吉少!

%23
诛魔榜迅速升空,但方正已然是呆住了:“他真的做到了,他在摧毁仙墓!”

%24
“仙墓!!!”比方正反应更大的是紫薇仙子、秦鼎菱这些人,她们是老牌的八转,真正意义上的天庭成员。

%25
仙墓不只是天庭的重地,更是她们内心的精神圣地,如今却被方源玷污亵渎,无数沉眠在里面的天庭先贤,就这样在沉眠中像是鸡仔般,毫无反抗地被彻底杀死!

%26
“方源,我要你死啊!”秦鼎菱发出惨烈的嚎叫。

%27
“我等愧对前辈啊……”啪的一声,中央大殿中正元老人跪倒在地,泣不成声。

%28
紫薇仙子只感觉全身都空空荡荡,仙墓遭受重创,她身为天庭首脑有着不可推卸的重大责任!

%29
“是我,是我没有做好啊……”两行悔痛的泪,从紫薇仙子的眼中迅速滑落。

%30
一缺抱憾亭中,光影剧烈震颤。

%31
星宿仙尊的虚影想要出手,但却被无极魔尊的虚影死死缠住,只得眼睁睁地看着方源大肆屠戮着天庭成员,摧毁着仙墓。

%32
震惊和愤怒充斥在星宿虚影的脸上,她的目光寒冷如冰,死死盯着自己眼前的对手:“好,很好!”

%33
无极虚影微笑:“是挺好的。当年我进攻天庭,也想动手摧毁仙墓。不过呢,念及当时异人的规模仍旧十分巨大,为了人族着想,便没有动手。”

%34
“哼,就算你摧毁了仙墓,摧毁了整个天庭。只要你摧毁不了宿命蛊,天意垂青,我中洲仍旧可以建立第二座天庭,第三个、第四个仙墓!”星宿虚影反驳道。

%35
“正是如此啊。”无极魔尊的虚影叹息一声,“宿命蛊便是你天庭之根,摧毁不了它,即便你天庭牺牲再多,也是无妨的。甚至反而会越挫越勇,淘汰旧人、输送新血,越发贴近时代的浪潮。”

%36
时代是在不断地变迁,蛊仙修行也在不断地发展。

%37
气道、力道的昌盛和衰弱,正是明证之一。

%38
无极魔尊当年留下仙墓,也是出于这点考量。留下被时代淘汰的气道、力道的蛊仙,总比摧毁了天庭,让天庭触底反弹,大量律道、智道等等蛊仙应运而生,被天庭吸收更好罢。

%39
“现在,我很好奇,仙墓被方源摧毁,你那边还能支撑多久?”无极虚影微笑。

%40
星宿虚影再不能淡然:“你这是不顾大局,抛舍人族大业!你枉为人族之尊。”

%41
无极虚影冷笑:“当年我的本体会被这个理由束缚,但如今……人族的大业不是已经实现了吗?人族,应当追寻更伟大的东西了。你应该知道我的意思。”

%42
星宿虚影眯起双眼,却是罕见的陷入了沉默。

%43
“方源,我要将你抽筋扒皮、挫骨扬灰!!”龙公仰天怒吼,刹那间他心中的愤怒、仇恨就积蓄到了极点,奋不顾身地向方源杀来。

%44
方源直接躲闪,不和龙公较劲,剑浪三叠催动不休。

%45
哗!哗!哗!

%46
仙墓被完全覆盖,甚至剑浪还掘地三尺,不放过任何一位沉眠的仙人。

%47
天庭数百万年的积累,毁之一旦!

%48
“啊啊啊!”龙公披头散发,面容扭曲,双眼充斥血丝,简直是要喷火。

%49
方源针对仙墓的每一次打击,都冲刷他的心房,让他心血淋漓,痛不欲生。

%50
仙道杀招——龙子龙孙。

%51
危难时刻,龙公愤怒狂吼,全身上下紫色龙鳞纷纷飞旋,混合龙公无数的血滴,转化为一位位的龙人蛊仙。

%52
他们各有名字,亦各有各的容貌和手段,六转、七转,甚至八转的龙人都有。

%53
龙公虽不是奴道蛊仙,但也有类似的杀招。

%54
“这是变化道的杀招,蕴藏人道奥妙!”方源瞳孔微缩,冷嘲热讽,“好一个龙公,当年屠戮龙人一族,原来是将他们都炼成了杀招。哈哈哈,你这等行径和魔道蛊仙有什么区别?你罔顾人伦,践踏血脉和亲情,不仅杀你的子孙后代,还将他们炼成傀儡,为你作战!!”

%55
面对方源的指责,龙公并未反驳,他的眼中闪过一抹沉重的悲痛,再次向方源杀去。

%56
与此同时,他施展出来的大量龙人蛊仙,如天花绽放,迅速分散,积极抵御剑浪还有无数的年兽。

%57
达到龙公这种程度,果然是没有任何的短板。

%58
一场超大规模的混战厮杀,在天庭中展开。

%59
诛魔榜四处纵横,在年兽大军中冲出一条条血路。不过诛魔榜内却是氛围压抑,秦鼎菱、方正二仙都保持着沉默。

%60
龙公施展出来的龙子龙孙,的确是龙人一族的族人,方源的指责并没有错。也难怪龙公极少运用此招,现在他不得不用出来,是着实被方源逼的没有办法了。

%61
许多龙人蛊仙向太古剑龙围去。

%62
方源心中冷笑一声,剑浪三叠已是催动完毕,便催动另一记仙道杀招金丝剑。

%63
一瞬间,剑龙身躯朦胧起来,每一片鳞片都散发出淡淡的温润黄光。

%64
忽然,从龙鳞上激射出一道金丝。

%65
太古剑龙的身上有多少片龙鳞,就有多少道金丝猛然射出,迅捷无比。

%66
金丝穿透空气,刺破沿途的一切阻碍,包围上来的龙人蛊仙一个个都被无数金丝刺穿成马蜂窝。

%67
嘭嘭嘭。

%68
这些龙人蛊仙遭受如此重创,再无法支撑身形,一个个轻微自爆,还原成鳞片和血珠。

%69
方源扫清周遭一切敌人,刚刚催发的剑浪还在呼啸蔓延,他便身躯一展,乘浪而下!

%70
仙墓已经扫荡,最大的变数已经扼杀,现在方源兵锋直指龙公。

%71
苍玄子已经毫无生息,藤躯惨不忍睹,诛魔榜根本无法独自对抗方源,只要龙公一倒,谁还能挡住方源的脚步?

%72
仙道杀招——五指全心剑!

%73
龙爪猛地紧握,方源气势陡然暴降到了谷底,似乎不足为虑。

%74
龙公心中却是警钟狂鸣,身躯飚射而出,紫金龙形气劲不断激闪,身形忽隐忽现,忽左忽右,忽上忽下,疯狂规避。

%75
第一剑!

%76
方源龙瞳绽射奇芒,一道剑光从龙爪中飞射而出。

%77
快!快!快!

%78
剑光之迅猛,简直匪夷所思。

%79
龙公刚看到方源龙爪绽射出一抹亮光,便发现剑光已经来到了他的面前。

%80
几乎不可能的情况下,他硬是偏转了身躯,强行一扭。

%81
剑光与他擦肩而过。

%82
鲜血飚飞,一个巨大的创口从他的左肩一直延展到左胸,创口空空荡荡,不管是骨骼、龙鳞亦或者一部分的左肺,都全然消失。

%83
五指全心剑恐怖如斯!

%84
这要是射中龙公的头颅,恐怕就要被立即削飞了去。

%85
交战以来,龙公首次感到强烈的死亡威胁。这是他多少年都没有品尝到的感觉。

%86
龙公大吼一声,身形如电,疾风劲鼓,向方源蛮横撞来。

%87
方源也不禁心中叫好。

%88
五指全心剑每一剑的间隙,都是酝酿、蓄势的过程。龙公一眼就看破这个弱点,趁机进攻,勇猛无畏。

%89
方源龙躯一转,猛然后撤,企图和龙公拉开距离。

%90
剑道攻伐优势相当明显,若是寻常对手,方源大可直接撞过去。但龙公却是卓绝的强者,万万不能大意。

\end{this_body}

