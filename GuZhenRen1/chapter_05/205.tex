\newsection{正道的游戏规则}    %第二百零五节:正道的游戏规则

\begin{this_body}

%1
床榻之上,方源缓缓地睁开双眼。

%2
他吐出一口浊气,发现自己仍旧躺在床上,周围布置的一些手段,也完好无损。

%3
这里是南疆正道势力共同铺设的防御蛊阵,非常安全。但谨慎的性情融入了方源的骨髓之中,让他每时每刻都会行事三思。

%4
他自重生以来,都是刻苦修行,殚精竭虑,没有将丝毫的片刻时间,放在其他没有必要的方面。

%5
这一次睡眠,自然也是如此。

%6
方源是主动沉入自己的梦境之中,在梦境中取材,获取了不少梦道凡蛊的蛊材。

%7
“加以炼制的话,能收获五六只的梦道凡蛊吧。”方源暗自估算了一下。

%8
他在梦道方面的进步,也相当的明显。

%9
以前,一场梦境只能收获少量蛊材,最多炼制出一两只的梦道凡蛊。但现在却不一样,收获大大提升,使得他炼制梦道凡蛊的效率也暴涨了数倍。

%10
当然,还有一个极其重要的因素,那就是魂魄。

%11
探索梦境,会令魂魄虚弱。

%12
因此,越是强大的魂魄,探索梦境的时间就越长,能够承受梦境反噬伤害的能力就越大。

%13
值得说明的一点是:魂魄就算再强大,也只是提供了很基本的条件。强大的魂魄,并不能让蛊仙在梦境中自由穿梭。

%14
魂魄和梦境之间,若是打一个不太恰当的比喻。

%15
梦境就仿佛是一片沙漠,魂魄进入其中探索,仿佛是在沙漠中迷路的旅人。魂魄越是强大,代表这位迷路的旅人身体素质更好,更能忍受饥渴,能在沙漠中煎熬更长久一些。

%16
但这并不代表,迷路的旅人能够走出这片沙漠。他不能明确自己的方向,不能在沙漠中自己制造水和食物,不能将沙漠缩小,不能自己脱离沙漠……

%17
而明确方向。制造水和食物,改造沙漠等等手段,目前阶段,只有梦境才能够达到。

%18
为什么要强调“目前阶段”呢?

%19
这是因为。各大流派境界达到一定高深的程度之后,就能触类旁通,达到其他流派的效果。

%20
假以时日,炎道、土道等等流派,也能构建出别出心裁的仙道杀招。模拟出梦道的效果。

%21
但是!

%22
这里有一个大前提。

%23
那就是梦道必须完善,并且模拟梦道效果的蛊仙,也必须熟悉梦道,甚至拥有不俗的梦道造诣。

%24
而现在,梦道只是刚刚研究而已,各大势力,无数蛊仙尝尽了苦头,并没有多少能拿得出来的成果。

%25
梦道的蛊虫虽然已经有不少,其中甚至还有野生的梦道仙蛊。但梦道这个流派,距离正式创立。还有很长的路要走。

%26
连开创都没有,更别谈完善了。

%27
开创一个全新的流派,并不是那么容易的。

%28
水道的蛊虫出现了不少,太古时代、远古时代、上古时代,都有无数的野生水道蛊虫,甚至还有人炼制出全新的水道蛊虫。但一直到了中古时代,灵缘斋的创派祖师爷女仙水尼,才统合成功,兼顾各方各面,正式创建了水道。

%29
所以。整个五域的蛊仙,都对梦道无可奈何,都知道这是一块巨大的肥肉,但都无法下口。

%30
很多蛊仙都折在梦境之中。

%31
各域这种情况都有。南疆的蛊仙在前段时间,更是伤亡惨重。

%32
也正是如此,才让南疆的正道蛊仙们认识到了事实,不得不减缓探索梦境的脚步,稳扎稳打,一步步前行。小心翼翼。

%33
很多蛊仙对梦境十分忌惮,就打起了歪门心思,仙缘生意正是在这种情况下,应运而生的。

%34
若是正道蛊仙拥有探索梦境的能力,怎么可能将这么巨大的利益,出让给散魔两道的外人呢?

%35
联想到仙缘生意,方源从床上坐了起来。

%36
仙缘生意,对他而言,必须存在着,但同时他又不能参与过深。

%37
仙缘生意有不少的妙处。

%38
一方面,它能够带给蛊仙相当巨大的利益。另一方面,它能够大大地缓解正道、魔道、散修三者之间的矛盾。

%39
若是正道蛊仙死死护住超级梦境,这会让南疆的魔道和散仙们怎么想?

%40
“正道肯定是得到了巨大的好处,却不分给我们。长期以往,正道越强,我们越弱,我们还有什么活路吗?这可是关乎大梦仙尊的修行资源,谁不想成为大梦仙尊?凭什么正道就能占据拥有,没有我们的份儿?”

%41
无非是这类的思想。

%42
这种思想,当然是很危险的。

%43
南疆的正道势力虽然强大,但若是逼急了,散魔两道联合起来,搞不好又是一场义天山大战,或者是类似北原的血战武斗大会了。

%44
仙缘生意一做,这些散魔蛊仙们这才会明白:哦,原来梦境非常的危险,也不过如此。现阶段,正道蛊仙们都无能为力,要不然怎么会做这种生意?把梦境主动让给我们去尝试?

%45
设想出梦境生意的,正是武家在这里驻扎的七转蛊仙,的确很有思谋。可惜方源更强势,有血脉关系,把这位七转蛊仙排挤走了。

%46
方源需要稳定。

%47
越稳定越好!

%48
他需要一个安定的环境,来探索这里的梦境。

%49
所以,避免散魔两道攻击这里,仙缘生意就有开展下去的必要。

%50
但方源不能参与进去。

%51
因为这种生意,并不是所有的正道家族都参与进去了。

%52
这是相当关键的一点!

%53
简单而言,就是分赃不均!

%54
武家等六七家正道蛊仙们,得了好处,其他一半的家族却是没有参与进入来。

%55
这种情况也很合理。

%56
蛊仙们都为利益动心,当然参加的人数越少,那自己得到的不就越多么?

%57
因此,导致一个结果一旦这种仙缘生意被披露出来,但凡涉事的蛊仙一个都跑不了。肯定要被惩处,因为这就是正道的游戏规则!

%58
这是一个坑,方源当然不能陷进去。

%59
武安等人,想拉他入伙,但方源知道。一旦将来这种丑闻暴露了,他就算是两只仙蛊的主人,武庸的弟弟,想要继续留在这里。也是很悬的。

%60
方源当然不会冒这种风险。

%61
他千方百计地混进这里,为的是什么?不就是梦境嘛!

%62
尤其是,武庸此人并不是偏袒他的这个“弟弟”武遗海的。他的态度是什么,还要看具体情况。

%63
但只要方源自己不犯错,又是这里的仙蛊主人。武庸将他调走的可能性就相当渺小了。

%64
方源将这一切,都看得相当通透。

%65
这要感谢他五百年前世。

%66
尽管他的修为不是很高,但是正道、魔道、散修他都混过,他有丰厚到无以伦比的人生经验。

%67
这使得他在正道中,毫无武遗海本身散修身份的桎梏和局限。他对这里的政治局势洞若观火,知道若作为正道蛊仙,该怎样玩这场人生的游戏。

%68
所以,他招揽了白兔姑娘。

%69
关键时刻,舍弃她,推脱自己不知道此事。

%70
关于仙缘生意中。分给自己的利润,方源也都交给白兔姑娘。

%71
这个利益,他是绝对不能碰的。

%72
不像魔道、散修,见到一些天材地宝,都要喊打喊杀。正道不能这么玩,有些利益可以吃下,有些利益是不可以碰的。

%73
但若是拒绝呢?将白兔姑娘、武安拒之门外呢?

%74
那就更加不妥。

%75
方源若是这样做,虽然没有事情暴露之后的危机,但是却恶了那六七家的蛊仙。这些人都是既得利益者,方源不做。没有人牵头,他们也不敢做。这些人对方源的感观,自然就是极差了。

%76
方源刚到超级蛊阵这里,就将这些原本亲近武家的政治盟友。都得罪光了。

%77
而那些始终没有参加仙缘生意的其他家族,诸如巴家、铁家、商家等等,更对方源虎视眈眈。

%78
方源若是这样做,无疑是自损城墙,蠢到不能再蠢,傻到不能再傻。

%79
现在的结果就非常好了。

%80
仙缘生意继续做。政治盟友皆大欢喜。巴家等人在旁边看着,却斩碎奈何不了满是党羽的方源。

%81
将来,若是事情曝光,方源把白兔姑娘推出来,牺牲掉,只说自己宠幸此女,没想到此女胆大包天至极,居然背着自己干出此等事情来。他都被蒙蔽,全然不知。

%82
总而言之,方源是进退自如。

%83
至于那些利润,方源只有舍弃了。

%84
他现在缺的并非是资源,至尊仙窍中修行资源,如今多得能叫绝大多数的七转蛊仙自惭形愧!

%85
“接下来,就开始正式尝试,进入这里的梦境罢!”方源准备行动。

%86
与此同时,在北原。

%87
血战武斗大会,已经步入尾声。

%88
百足天君、药皇至始至终都没有出现,但他们的影响力都是贯穿始终。

%89
楚度的脸色相当难看。

%90
情势对他这一方,明显很不利。

%91
昊震溃败下场,右手捂住左肩。他的伤势非常严重,整个左臂齐根而断,伤口处是充斥异种道痕,想要治疗相当的困难。

%92
“北原正道果然是底蕴雄浑。”楚度叹息,他再次陷入到无将可派的尴尬境地。

%93
昊震满脸仇恨,不甘地道:“楚度大人,我有一友,战力比我还强,只是一直潜修,声名不显而已。”

%94
楚度又喜又疑:“哦?你何不早早荐之?”

%95
昊震一脸为难的表情,小心翼翼地传音道:“只有一点不好,此人修行的是血道。”

%96
楚度顿时一脸肃容,毫不犹豫地否决道:“绝对不可!任何血道蛊仙,都是魔道蛊仙。眼下的血战武斗大会,乃是我们正道的内斗。怎可以牵扯血道魔头进来?”

%97
一旦牵扯进来,对方就能轻而易举地污蔑楚度一方,乃是魔道份子。

%98
楚度之前所做的一切努力,就都化为乌有了。

%99
这就是正道的游戏规则!

\end{this_body}

