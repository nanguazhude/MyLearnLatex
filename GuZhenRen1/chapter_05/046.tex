\newsection{双方炼蛊}    %第四十六节:双方炼蛊

\begin{this_body}

数日之后,中洲地渊深处。

石奴从外面回转,向影无邪汇报道:“主人,老奴已经探查清楚。因为超级塌方,引起不小震动,古魂门派遣出了三位蛊仙,前来探索。按照您的吩咐,老奴已经向塌方处引去了数只上古荒兽,还有五支荒兽群。加上似有野生仙蛊产生,古魂门的这三位蛊仙,定会忙得脱不开身了。”

石奴乃是石人身份,这种异人本身就有土道道痕。石奴一身土道修为,在地渊这种地形中,简直是如鱼得水。

影无邪点点头:“这件事你做得很好,不过天意无处不在,万万不能大意。说不定在我炼蛊之时,天意影响这些蛊仙,促使他们来到这里,坏影宗的大事!”

石奴想了想,道:“主人说的是。不过,在这个方向上,我们已经提前布置了三支荒兽群,阻碍他们。就算他们冲破了兽群的阻碍,来到这里,我们也有充分的时间反应。”

影无邪微微摇头:“你没有见识到天意的真正厉害。它不只是影响蛊仙,还会影响荒兽、荒植。到了那关键时刻,这些荒兽群说不定也会冲击过来。”

石奴脸色一变:“若是荒兽群奔袭过来……该如何是好?”

言下之意,自己之前的准备,故意勾引来的兽群,岂不是搬起石头砸自己的脚?

“无妨。我的依仗,并不是这些荒兽群,或是上古荒兽。而是这座超级蛊阵!”影无邪说着,眼中闪过智慧的光。

在这里,可以说是中洲目前最为安全的地方。

这座超级蛊阵,涵盖各个方面,不仅能遮蔽天意,防备智道蛊仙推算,而且还有十分强悍的攻防威能,更兼备炼蛊奇效。

魔尊幽魂在生死门中,不断吞魂。将无数蛊仙的修行记忆、经验纳为己物。他生前乃是幽魂魔尊,才情天赋纵观古今,也是顶级。

其中很多阵道、炼道等等经验,都被魔尊幽魂吸收。给他省去大量的精力时间,让他将自己的阵道、炼道等等诸多流派的境界,都推上了大宗师的程度。

而这座超级蛊阵,就是魔尊幽魂在生死门中,亲自设计。可想而知它的威能是多么的强大了。

影无邪依凭这座超级蛊阵,进行炼蛊,无疑是明智之举。

就算荒兽群或者上古荒兽扑来,他便能依靠超级蛊阵,斩杀这些荒兽、上古荒兽,利用它们的魂魄反扑超级蛊阵,临时增添蛊阵威能。所以不会有什么搬石头砸自己脚的事情发生。

石奴的担心纯属多余,只因他虽然生活在这里很长时间,却并不知道这座蛊阵的妙用和战能。

“我没有毁掉春秋蝉,这蛊中有天意寄居。所以超级蛊阵的位置,已经被天意获悉。不过没有关系,就算是天庭的八转蛊仙来围攻,我也有信心坚守在这里一段时间。更何况现在监天塔已毁,虽然有宿命蛊,天庭方面目前发现这里,可能性太小!”

影无邪也是经过一番深思熟虑之后,才没有放弃春秋蝉。

方源能够凭借春秋蝉,毁了影宗十万年的大计。那他影无邪为什么就不能将春秋蝉,当做他最后的底牌呢?

天意浩荡。在乎平衡,并不记仇。谁威胁到天地,威胁其他生命的生存,天意就针对他。

现在天意主要针对的是方源。并不是影无邪。

所以,就算是包含天意的春秋蝉,也对影无邪有很大帮助。

而且在影无邪的计划中,他还打算将春秋蝉重炼,将里面的天意排除干净,真正的将这只传奇仙蛊纳为己用。

不过。这都是后面的计划。

现在影无邪的全部心思,都放在炼制定仙游上。

这只仙蛊是他破局的关键!

有了这只仙蛊,他就将会是如龙入海,似虎归山。

“事不宜迟,我们现在就开始炼制春秋蝉。石奴你不要抵抗,我来催动蛊阵,将你挪移到阵眼之中,之后你要协助我炼蛊。”影无邪下令道。

“是,主人,老奴必定竭尽全力,纵死不悔!”石奴连忙领命。

安置好石奴,影无邪也进入最中央的阵眼当中。至于黑楼兰、太白云生,早就被安置在另外的两个阵眼里。

虽然只是六转的定仙游,但这一次炼蛊,不仅关系到影宗未来,甚至还隐隐影响到五域格局,意义堪称重大。

炼蛊开始了!

半个月后,北原,琅琊福地。

变形仙蛊,也开始了第一次的炼制尝试。

方源站在云盖大陆上,全阵以待,满脸肃穆,仰望着琅琊地灵。

琅琊地灵再不是之前那么仙风道骨,它威武雄壮,浑身肌肉贲发,一身黑毛,昂首悬浮在高空之中,宛若魔神。

他张口一吼,吼声震动天地,响遏行云。

一道道光痕,像是被吼声点亮,纵横漫布在方圆数十里的范围以内。

“这些道痕,绝大多数都是炼道道痕,辅助炼蛊乃是绝佳!”琅琊地灵张开大嘴,得意地介绍道。

他又再次大吼一声。

吼声中,一颗颗的光团,从整个的天空中纷纷亮起,各自绽射出瑰丽绚烂的色彩,耀人眼目。

“天空中是怎么了?”

“快看呐,仙迹再现了!”

“前不久是七彩天柱,这一次是漫天彩星,好一番壮丽奇景!”

一时间,海面上的黑毛、黄毛、白毛三座大陆,人声鼎沸,群情激动,下至乞丐,上至王公,纷纷仰头,用灼热、向往的目光,看着漫空的星光。

“每一颗彩星,都喷涌出浓郁的仙蛊气息,看来这些应当都是仙蛊屋炼炉的组成部分吧?”方源开口,用猜测的语气问道。

琅琊地灵点点头,遗憾地道:“你猜得不错。可惜现在许多核心仙蛊都丢失了,炼炉若是完整,炼成变形仙蛊的可能,高达六成!”

“不过丢了就丢了罢。炼炉我不打算去修复它,完整的炼炉又能如何?真正的强大,不在于仙蛊。而在于蛊仙本身。”

“注意,我要开始了!”

此时,方圆百里的道痕都被一一点亮,漫天的繁星绽射出五颜六色的华光。

琅琊地灵高悬于空。无数蛊虫飞舞旋绕在他的身边,大量的凡级蛊材被他转瞬炼化,速度非常的快,让方源看了都有一种眼花缭乱之感。

不一会儿,琅琊地灵低吼一声:“该你了。”

一瞬间。方源视野骤变,他取代了琅琊地灵的位置,悬浮在高空。而琅琊地灵,则站在他刚刚立足的地方,仰望他。

原来此次炼蛊,琅琊地灵根据具体情况,采用了双仙炼蛊的方法。琅琊地灵负责毛民炼蛊法,而方源则着重处理过程中的几项重要仙材。双方并无明显的主次之分,作用都一样重要。

方源毫无意外,早已做好了充足的心理准备。位于高空之中后。他立即双手一招,取出一大盆的内景雨,开始施为。

琅琊地灵看了一阵,微微点头,心底满意。

方源临阵不乱,功底扎实,手法娴熟中似还有一丝难以言述的奥妙韵味。

“这小子不愧是重生之人,有前世经验,炼道境界已经快要达到宗师地步了。不过要迈出这一步,可不容易。”琅琊地灵在心中评价。

炼蛊一帆风顺地进行下去。各项步骤有条不紊。就算出现了些许意外,也都程度微小,毫不足虑。

方源和琅琊地灵的准备工作,实在充足。再加上方源、琅琊地灵这等人物。都谈不上炼道弱者,堪称强强联手。因而有如此进展,并不奇怪。

连续三天三夜之后,琅琊地灵和方源不眠不休,所有的努力都化为了一只仙蛊雏形。

琅琊地灵见此大喜:“看来这第一次炼蛊,很可能就成了!只要最后几步不出差错……”

他话还未说完。就听见天空中轰隆作响,旋即雷声如鼓,响彻一片。

仙蛊将成,灾劫降临!

方源面色一变,但双手仍旧稳固如初,操纵着奇特冰焰,烤炙着仙蛊雏形。

“放心,区区六转仙蛊的灾劫,不算……什么?!”琅琊地灵语调忽然高昂起来,像是公鸡被人捏住了嗓子。

他双目瞪得溜圆,震惊地看着天上的滚滚劫雷。

灾劫还未爆发,但威势和气息早已凌驾于琅琊地灵记忆中的六转标准。

甚至,就算是七转仙蛊的标准,也及不上这场劫雷!

“怎么会这样?!”琅琊地灵万分不解,怒吼出声。

“太上大长老,你来主持炼蛊,我来对付这场仙劫。”方源咬牙道。

“不!最后的关键,你不能离开,你的狗屎运道可护持仙蛊。让我来对付这场灾劫,儿郎们,快与我组成上古战阵!”琅琊地灵大吼一声。

吼声刚落,近十位毛民蛊仙就飞到他的身旁。银光乍起,一位高大桀骜的银光巨人再次登场!

上古天下第二战阵天婆梭罗!

中洲,地渊。

“呼呼呼……”影无邪喘着粗气,双目赤红,像是快要赌输了一切,急于翻本的可怜赌徒。

从半个月之前,他就开始尝试炼制定仙游,到目前为止,他已经连续失败了八次!

“该死的,只剩下最后两次机会了。”

“我明明有蛊阵辅助,还连续失败八次。人族隔绝流的炼蛊方法,到底是没有毛民天地流的成功率高啊。”

“不过,就算余木蠢再世,我也万万不会动用毛民天地流。天意也是想除掉我的,一旦给了天意机会,让它借助灾劫显威。万一来个莫名其妙的灾劫,恰恰克制瓦解超级蛊阵,那就麻烦了!”

“方源不知道天意,采用了毛民天地流炼蛊。呵呵,自求多福吧。”(未完待续。)

\end{this_body}

