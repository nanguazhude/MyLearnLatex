\newsection{水道宗师}    %第二百二十七节:水道宗师

\begin{this_body}

%1
“这里……是哪里?”赵怜云望着窗外,面带惊异之色。

%2
但这一次,就连博文广志的余艺冶子,都无法回答她。

%3
倒是负责保护赵怜云安全的不真子,用不太肯定的语气,开口道:“这种奇异的景象,难道就是无间道?!”

%4
来自中洲的三座仙蛊屋揽雀阁、角连营以及风满楼,正在一个巨大的甬道中前行。

%5
这个甬道,以蔚蓝色为主色调,一条条色彩斑斓的细长线条,顺着甬道绵延开去。这些彩色的线条,镶嵌在蔚蓝甬道的内壁上,丝丝缕缕,灿烂如花。

%6
“什么是无间道?”余艺冶子追问。

%7
不真子用复杂的目光,看了一眼赵怜云:“你应该很熟悉才对?”

%8
“我?”赵怜云诧异,她完全不明白不真子为什么会这么说。

%9
不真子叹了一口气,道:“因为无间道,乃是盗天魔尊独创的仙道杀招!”

%10
南疆,超级蛊阵。

%11
梦中。

%12
“徒儿拜见师父!”方源跪在地上,对眼前的女性蛊仙行叩拜大礼。

%13
女仙嗯了一声:“我乃静妙子,专修水道。你虽然是我后代,但能否修行我的水道,还要看你的造化和悟性。通过这场考验,你才能真正的成为我的关门弟子。”

%14
“还请师父示下。”方源恭谨地道。

%15
女仙大袖一挥,梦境顿时改变。方源站在一片碧绿的潭水面前。

%16
这片潭水,幽绿可人,波澜不兴,宛若一块巨大的玉石。

%17
“长河滚荡,是水。大海澎湃。是水。而你眼前的碧波不兴,同样也是水。水道博大精深,为师专修静水一道。你步入此潭,什么时候在能够不用蛊虫的情况下,自己站在水面之上,我便正式收你为徒。”

%18
方源瞪大双眼,头一次听到这种考验。不由叫道:“师父。我若不用蛊虫,和凡夫俗子无异。人的肉体凡躯,如何站在水面而不沉?”

%19
“那就是你的事情了。”女仙大袖一摆,消失在方源的眼前。

%20
方源皱起眉头,围绕着水潭踱步,几圈之后,他踏足绿潭。

%21
扑通。

%22
一声轻响。他整个人干脆利落地落入水中。

%23
他很快爬上了岸,完全是落汤鸡,好不狼狈。

%24
方源咬咬牙,犹豫了一下,使出仙道杀招解梦。

%25
但这一次,无往而不利的解梦杀招,毫无作用,一丝变化都没有。

%26
方源心中长叹:“果然是没有用的么……唉,这下麻烦了。”

%27
不久之前,方源的运道境界提升。让他轻松自然地将气运交感杀招,向上提升了一个档次。至此,影无邪等人行踪暴露,就在南疆。

%28
方源发现此点之后,便积极开始筹备,不断增强实力,更向武家借来仙蛊。

%29
但是。他要出走,却遇到了麻烦。

%30
方源一走,武家在超级蛊阵当中,就没有七转蛊仙镇压局势了。再加上武家挑头,做的仙缘生意,事关重大,一旦没有强者照看,别的家族要闹事的话,必定是轻松自如。

%31
武庸虽然很痛快就答应了,帮助方源借取武家仙蛊,但是在此事上,他却没有松口。

%32
武家的蛊仙虽然多,势力也很强,但是家大业大,摊子铺得大了,反而更需要人手。蛊仙的数量又是这样稀少,导致人手方面捉襟见肘。有些职务,更不是说调就调的。就像方源现在镇守超级蛊阵的职位,武家上下并非所有人都有能力担任。就算有能力,政治方面也要考虑。

%33
方源忽然提出这样的要求,要擅离职守,这让武庸感到异常头疼。

%34
但方源态度坚决,武庸只好着手,开始调配各个蛊仙的职务。

%35
但这个事情,牵一发而动全身,武庸坦言,此事需要时间,方源只好选择等待。

%36
不过在等待的过程中,方源又遇到了一片上佳的梦境。

%37
梦境千奇百怪,不断流动。有写实的梦境,也有诡异梦境。

%38
方源一般都会选择写实梦境,进行探索。诡异梦境难以捉摸,方源也不想冒无谓的风险。

%39
“按照历史上的记载,女仙静妙子乃是水尼之徒。这片梦境的主人,师父就是静妙子,又是她的关门弟子。那就只有一个人,历史上赫赫有名的水道大能定海上人!”

%40
“中洲历史上,有太古荒兽水南无作乱,它让落天河改道,海啸盖陆。中洲中部洪水泛滥,东部海啸翻天。当时已经是天庭蛊仙的定海上人,挺身而出,几记仙道杀招,就让落天河道归位,海面重复平静,手段极为了得。”

%41
“我的解梦杀招,几乎无往而不利,居然在这个关卡上,毫无建树。奇怪,奇怪。”

%42
这已经不是方源第一次催动杀招解梦了。

%43
但是几次杀招催动下来,这片梦境岿然不动。

%44
方源思前想后,猜测这恐怕是这片梦境,十分强大,杀招解梦的核心仙蛊毕竟只是六转层次,威能无法破解这片梦境!

%45
梦境的强弱区分,不在于和梦境有关的人物。

%46
比如说,方源曾经探索过星宿仙尊的梦境,在那里,他动用解梦杀招,收获颇丰。

%47
但在此时此刻,方源的解梦却无法令他通过定海上人的梦境。

%48
定海上人只是八转级数,当然不如星宿仙尊。

%49
但是梦境的强弱,却和两者生前的修为实力,关系不大。

%50
至于关系究竟如何,方源也太清楚。

%51
别忘了,方源在五百年前世,也不太乐衷于梦境方面的探险,他的精力主要放在血道的修行上面。

%52
方源对梦境研究不深,但他却知道一点常识,那就是梦境越强,探索之后,得到的好处就越大。

%53
牢记这点,方源对这片梦境的兴趣又加深了许多。

%54
“武家方面的调令,还没有颁布下来。我想借的仙蛊,武庸肯定会扣在手中不放,直到他完成调遣。”

%55
“既然如此,这段时间,我就好好地钻研这片梦境好了。”

%56
接下来的几天里,方源沉下心来,鼓起干劲,一次次地进入梦中进行尝试和探索。

%57
在不利用蛊虫的情况下,让自身站立在水面之上。

%58
表面上来看,这无解的题目。

%59
因为不动用蛊虫,蛊师基本上和凡人无异。当然,这不算类似于鸿运齐天、黒豕蛊这类往蛊师身上刻印道痕的蛊虫。

%60
“的确有水足蛊这类的水道蛊虫,可以在蛊师的双脚上刻印水道道痕。让蛊师能够在水面上自由地奔走。不过,这和通过这片梦境,完全无关。”

%61
别说是,这片梦境中,无法运用水足蛊什么的。就算能用,方源也不会去用。

%62
因为他知道,这个难题不在于他的内容本身,反而在于他自己。

%63
至于什么将水面冻结成冰,然后站在上面的解决方法,也都是小聪明大错误,根本没有洞察到这个难题的本质是什么。

%64
“正如静妙子之前所言,她考验的是我的悟性啊。这种写实梦境,既然定海上人能够通过,我没道理不会领悟出来。毕竟我的水道境界,已经有了准宗师的程度了。”

%65
方源不断尝试,他已经数不清落入水潭的次数了。

%66
普通凡人,若是落入这片梦境,恐怕掉入水潭仅仅一次,就足够凡人魂飞魄散。

%67
而蛊仙落入水潭,三四次之后,魂魄也会虚弱到低谷。

%68
方源的魂魄底蕴却是雄厚的,他每一次进入梦境,至少得落水十七八次,魂魄才会虚弱到极限程度。

%69
这个时候,方源就会动用胆识蛊,进行疗伤。

%70
胆识蛊虽然是凡蛊,但源自天地秘境荡魂山,效用绝佳。

%71
方源有此强助,不断努力,渐渐有了感觉。

%72
这种感觉,用言语说不出清楚,非常玄妙。

%73
方源再一次尝试。

%74
当他落足水面时,扑通一声,他毫无意外地再次掉入了碧绿潭水之中。

%75
当他爬上了岸,便看到师父静妙子,驻足在谭边,看着他。

%76
静妙子恼怒地道:“你这是什么悟性?如此简单的一个考验,居然到现在,都做不到。你根本不配做我的徒弟。”

%77
方源却是微笑:“不,师父,我做到了。正如你所说,让我立足潭水表面,不沉不坠,我已经做到了。”

%78
静妙子怒极反笑,手指着方源浑身上下的水:“那这是什么?”

%79
方源摇头:“这什么都不是。我心中的潭水,一片波澜不惊,我已立足其上,往下看,便可看见如镜面一般的潭水上,有着我的清晰倒影。”

%80
静妙子脸上的怒意和笑容,都缓缓消失。

%81
她深深地看了方源一眼,这一次缓缓点头,淡淡地道:“不错,从今天起,你便是我正式的关门弟子了。”

%82
下一刻,整个梦境骤然消失,方源回到现实。

%83
“这片梦境只有一幕?”方源目露惊异之色。

%84
很快,他眼中的惊异都转变成了喜色。

%85
“好家伙,我的水道境界正式达到宗师级了!”

%86
之前,方源的水道境界正式准宗师级,稍稍和宗师搭上了边而已,真正距离宗师还有一段长长的距离。

%87
但是现在,这段距离直接被跨越过来,一下子让方源成为了实打实的水道宗师!

%88
继变化道、力道、血道、智道、星道之后,方源终于有了第六个流派,成就宗师境界。

%89
ps:今天就一更。有点难写,人物实在太多,剧情也很复杂。我在想,是不是该删减一些内容。

\end{this_body}

