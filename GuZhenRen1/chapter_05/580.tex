\newsection{渡劫和神技}    %第五百八十一节:渡劫和神技

\begin{this_body}

南疆。

至尊仙窍已经落下,此刻门户大开,如巨鲸吞水,海量的天地二气灌输进来。

哗啦啦!

疯狂的天地二气,磅礴浩荡,他人若见此盛景,早已惊骇――从未有这样的福地,能够一次吞吸如此海量的天地二气。

方源至尊肉身就在门户附近悬停,看着海量天地二气灌输进来。天地二气因为太过浓郁,已经如云似雾,带着湿意。

“吸吧,吸吧,这一次吸个够。”方源微笑,他明白至尊仙窍的“饥渴”。

市井、逆流河、落魄谷,三大天地秘境虽好,世间独一无二,但是对于福地的负担就太大了,天地二气消耗得非常猛烈。因此时不时,方源就得从外界吞吸进来海量的天地二气,补充自己。

“这还是荡魂山没有被修复成的情况,若是它修复好,至尊福地对于天地二气的需求,还要更多!”

从这点上看,荡魂山的损毁,对于方源而言,还有一丝益处。

当然,最关键的是方源本身的至尊福地,它资源丰富,宛若一块巨大的基石,有同时承载四大天地秘境的能力。若是换做寻常的福地,恐怕很难同时承担两座。

落下仙窍,打开仙窍门户,对于方源而言,有不少的弊端。

在这一刻,因为内外沟通,吞吸外界天地二气,方源的位置就要暴露,天意会立即发觉。

若是此刻有其他蛊仙推算自己,方源防备推算的难度,也要大大提升。

因此,方源之前都是尽量采购仙材,化解成天地二气,补充自身。但最近这段时间,因为他重点都放在修复荡魂山上,资金越来越少,手头渐渐拮据,因此就冒险落窍,吞纳外界天地二气了。

至尊仙窍绝对是大肚汉,消化掉海量的天地二气,是寻常福地的数十倍,乃至上百倍。

至尊仙窍吃饱了,但天地二气却仍旧不断地汹涌灌入进来,源源不断。

这番异象,没有令方源吃惊,因为他知道,这一次不同寻常,因为正是他灾劫到来之期。

方源身为七转蛊仙,十年一场地灾,五十年一场天劫,百年一场浩劫。

方源自从动用宙道手段,将至尊仙窍中的光阴流速调整到最快程度后,他迎来的灾劫就远比之前频繁得多!

这是一场地灾。

不管是地灾还是天劫,对于方源而言,都没有任何的挑战性。尤其是他还掌握着仙道杀招石洞天机。此招以天机仙蛊为核心,其余蛊虫辅助,能令方源直接推算出一定时间以后的灾劫内容。

所以这场地灾还未真正成形,方源就已经对它知根知底。

果然,接下来的灾劫衍化,一路方源所推算的那样。

地气回旋不休,在方源的脚下方,形成黄褐色的巨大漩涡。而天气翻腾不止,轰隆作响,青色的雷霆不断闪烁。

呜呜呜――!

一连串刺耳的响声,似乎要钻破人的耳膜,一具具形象狰狞的巨像,从黄褐地气漩涡的中心,缓缓浮出。

这些巨像,一座座都高达三丈,体格粗壮,獠牙支出嘴唇。猛然间,它们睁开赤红的双眼,背后紫色的蝠翼猛地张开,飞腾起来,连方源杀来。

“黄泉鬼武像。”方源悬停在半空中,背负双手,面色淡然地俯瞰这些巨像袭来。

他好整以暇是,敌势凶猛,他却在这关键时刻抬头看向天空。

天空中,青色霹雳猛地劈开厚重的云层,一座座和黄泉鬼武像十分相似的巨像,悍然登场。

黄泉鬼武像张牙舞爪,这些巨像却是排列整齐,宛若军队,一个个神情肃穆,同样是雄阔的体格,背后一对鹰翼宽大雄厚。

“青陀神武像。”方源心中微微一笑,将第二种巨像直接辨认出来。

这时,黄泉鬼武像已经冲到方源的身边,巨爪挥舞,血盆大口张开,对准方源撕咬绞杀。

方源一动不动,身体表面浮现出一层淡淡的涟漪。

正是已经名动天下,方源独创的仙道杀招――逆流护身印!

不管是黄泉鬼武像,还是青陀神武像的攻势,俱都被逆反回去,伤其自身。反观方源安然无恙,稳稳利于不败之地。

“都散了罢。”方源挥手,仙道杀招万蛟爆发而出,银鳞如海,万蛟飞腾,无穷无尽。

黄泉鬼武像、青陀神武像数量虽多,也随着天地二气,不断产出,但哪里及得上方源万蛟的数量。

刚开始,它们还抵抗了一下。但很快,它们就被万蛟大军淹没,严重减员。直至各自凝聚成一团的天地二气,都被万蛟冲散。天意不甘,想要重新凝聚,下一刻就又被凶残的万蛟大军摧残成一片片。

片刻后,灾劫渡过,仙窍门户关闭,至尊仙窍中风平浪静,回归安静祥和。

方源暗暗点头,总结得失:“这场地灾,原本只是单纯的黄泉鬼武像。但是我之前的那场地灾并没有渡,而是故意用了仙道杀招后患无穷来延缓。”

仙道杀招后患无穷,乃是黑凡真传中的内容。它能够将一次灾劫,挪移到下一次去,两劫同时爆发。弊端就是此法取巧,会惹来天怒,使得灾劫威能比单纯的两灾之和要大得多。

天意居心叵测,为了压制方源修为进步,刻意将每一次的灾劫威能都降至最低,使得方源每次渡劫的道痕收获都很少。

在这种情况下,后患无穷杀招的弊端,反而是对方源有利的优势!

于是方源便利用这个杀招,两灾叠加,一起渡过,增长道痕,让天意无可奈何。

“黄泉鬼武像是地灾,青陀神武像却已经是天劫层次。这一次,让我道痕收获不小!”方源检查了一下,重点是魂道和变化道道痕增长了。

这都是对方源有用的。

当然,方源目前最需要的,乃是宙道。

至尊仙窍地域广阔,方源特意选择了空无一物的地带渡劫。所以就算天意想要摧毁什么资源,也无目标可言。

灾劫虽然渡过,方源肉身却还停留于此。他催动起天消意散杀招,确定这里的残存天意被剿除干净后,他才会停止。

方源本体渡劫的时候,他的宙道分身一直在智慧蛊的面前,不断地推算着仙道杀招。

之前,方源成功地探索了梦境,使得宙道境界突飞猛进,一路暴涨到了宙道宗师境界。

有了这样的境界,推算宙道杀招,更是顺利无比了。

其实方源本身,就掌握了无数宙道杀招。幽魂真传中记载的最多,其次就是黑凡真传,琅琊派的所有宙道仙蛊方,在理论上,也都能被炼道准无上的方源,推算成相应的仙道杀招。

但对方源而言,黑凡真传还是最主要的参考。因为他手中的宙道仙蛊,大多数都是来自黑凡真传。

比如八转的似水流年仙蛊,七转的年蛊、以后蛊。

剩下的一些宙道仙蛊,都是六转。其中绝大多数,是惊鸿乱斗台破碎后的遗留,还有方源意外收获的日蛊。

宙道分身最主要的工作,就是以黑凡真传为主,借鉴其他仙道杀招,以及宙道仙蛊方,推算改良出适合方源的宙道手段。

虽然时间没有多久,但宙道分身的工作成果斐然!

目前为止,方源已经有了一整套的宙道手段,涉及攻防、腾挪、治疗等等方面。只需要练习一段时间,就能全数掌握。

“但这还不够!”宙道分身缓缓睁开双眼,叹了一口气。

周围的智慧光晕徐徐消退,智慧蛊悠然飞走。

宙道分身来到一处空地,开始试验杀招。

仙道杀招――光阴飞刃!

一柄飞刀划破长空,转瞬间就消失不见,速度骇人至极。

宙道分身却口角溢血,缓缓摇头,大为失望。

“我推算出来的宙道手段,都只是七转层次,算是常规手段。而要进入光阴长河,探寻红莲真传,面对天庭阻击,必定是要有八转层次的宙道手段方可。”

黑凡真传中,就有八转层次的防御杀招,以似水流年仙蛊为核心。但却没有相应的攻伐杀招。

对于黑凡而言,他在晚年才掌握了八转仙蛊似水流年。在有限的时间里,开发出八转防御杀招,已属不易。他拥有八转道痕,七转层次的攻伐手段,在他用来,也威能浩瀚。当然更关键的是,似水流年就只有一只,用于防御,就不能同时用来进攻。往往防御要优先于进攻,尤其是对于正道蛊仙而言。

“而这光阴飞刃,却是非比寻常。七转层次的光阴飞刃,虽然不能斩杀八转,却有着伤害之能。可惜我改良之后,威能却大大不如原版,甚至还有着沉重的反噬。”

这种情况,方源并不陌生。

就像他之前设计出来的万我仙蛊方,因为炼道准无上的境界,几乎到了改无可改的程度。任何一个微小的改动,都会弊大于利,使得新版大大逊色于原版。

方源现在改良光阴飞刃,也是如此。

“看来这完整的光阴飞刃杀招,恐怕真的是九转杀招了!”

就像幽魂魔尊开创的仙道杀招井井有条,也是九转层次,改无可改。

方源从梦境中得到的,虽然是六转、七转层次的光阴飞刃,算是九转的残缺版本,但也是改无可改的。强行改动之后,弊端多多,远超之前。

另外,按照方源的推算,这光阴飞刃的遗忘弊端,也大不简单。不只是梦境中老者所说的那么肤浅,遗忘不只是使用者,而是相关的一切都要遗忘。

任何关于此招的记载,不管是人的记忆,还是记载在信道蛊虫中,刻印在石板上,只要此招不断使用,这些记载都会在光阴长河中被逐渐抹去。

“魔尊幽魂当年应当也掌握着光阴飞刃,只是运用得多了,有关他这条线的光阴飞刃的记载,都被相应抹掉,逐渐遗忘。所以幽魂真传中就没有记载。”

“当然,也有一种可能,就是幽魂真传并不完整,还被影宗保留了部分,并全部未传授给我。不过这种可能很小。”

“不管怎么说,有一点我可以断定,那就是光阴飞刃真的是一种神乎其神的手段。若我真的能掌握在手,前往光阴长河的把握,就能提高两成!”

一天之后。

“就是这里了。”陆畏因半跪在地上,手掌撑地,感受着此处地气的微妙不同。

随后,数道光影从天而降,落到陆畏因的身后。为首的老妪正是宙道八转大能,夏家太上大长老――夏槎!

“方源就在此处,落窍开门,吞吸天地二气,并且……他还再次渡了一场灾劫。”陆畏因以肯定的语气道。

夏槎眼底精芒一闪即逝,陆畏因不愧是乐土传人,土道手段竟精擅到了此种地步!

背后的蛊仙们开始议论纷纷。

“有陆畏因大人在,我们追捕到方源,指日可待了!”

“方源拥有数座天地秘境,天地二气负担极大,有这样的线索在,不怕找不到他。”

“但你们不觉得奇怪吗?他明明有定仙游在,为什么一直逗留在我们南疆呢?他为什么不去东海、西漠等地,那里不是更加安全吗?”

“哼,他已经是人人喊打,天地不容,哪里都不是安全的地方。”

“这魔头居心叵测,留在南疆,恐怕还是想劫掠我等资源,尤其是想对义天山那处梦境下手,不可不防啊!”

“杀,必须要把这个魔头杀掉,我们才能真正安心!”

ps:大纲还在修订当中,但任务量太大了,我发现一个星期都完成不了。好在前期有一部分完成了,可以令我开始正常一更。此书在起点中文网首发,请假条在起点都可以及时看到的。如果今后哪天更新不正常,大家可以来起点中文网看看请假条。

------------

\end{this_body}

