\newsection{有点弱}    %第四百四十七节:有点弱

\begin{this_body}

嗷吼!

一声炸响,宛若雷霆,在毛民蛊仙们的耳中陡然爆炸开来。[\&\#26825;\&\#33457;\&\#31958;\&\#23567;\&\#35828;\&\#32593;\&\#119;\&\#119;\&\#119;\&\#46;\&\#77;\&\#105;\&\#97;\&\#110;\&\#104;\&\#117;\&\#97;\&\#116;\&\#97;\&\#110;\&\#103;\&\#46;\&\#99;\&\#111;\&\#109;

与此同时,一个巨大的龙头,宛若小山一般,横压过来。

巨大的阴影,投盖在群仙身上,还未袭来,气浪横飞,汹汹狂猛的气势,让人两股战战,难有抵挡之心。

“完蛋了!”三位毛民蛊仙无不大汗淋漓,其中一人甚至直接瘫倒在地上。

眼看着他们三个就要陨落在此,这时候,忽然三位毛民的耳畔忽然听到一个清冷的声音――“废物。”

如此辱骂,这三位毛民蛊仙却是大喜过望,齐声大叫:“我们在这里,凝冰仙子,快来救救我们!”

危难之际,一道白色娇影电射而来,正是龙女白凝冰。

她冷哼一声,双手一抬。

夸嚓!

下一刻,寒冰巨墙陡然升起,挡在三位毛民蛊仙的面前。

轰。

只是一个呼吸时间,巨大的龙头就狠狠地撞击在冰墙上。

冰墙不堪重负,轰然破碎。

但白凝冰已经拉着三位毛民蛊仙,急速后退,拉开距离,同时双眼投射出冰冷的光,直接照在龙头上。

狰狞的龙头很快覆盖了一层薄薄的白霜,速度顿时降低了许多。

白凝冰心中松了一口气。

电光火石间的交手,她已经测出,眼前的龙头不过是七转货色,依凭她的战力,应付起来绰绰有余。

但是――

“太古荒兽万首盘龙,拥有上万的龙头。我能对付眼前的一个龙龙,接下来若是让其他龙头参战,总会力有不逮,双拳难敌四手。”

白凝冰仰望天空,眉头蹙起。

这并非是正常的太丘天地,而是一片阴暗的阵道空间。

万首盘龙乃是阵道太古荒兽,可以营造出阵道空间,就好比上极天鹰能够出入仙窍洞天,这是太古荒兽的天赋本能。

“必须尽快突破这里,回归外界。”

“希望影无邪他们能牵制得住!”

此时,太丘外界。

黑楼兰等人,各自盘旋高空,面对着地面上昂首直立的数百条长龙,正各展其能,将各种攻伐杀招接连不断地轰砸过去。

场面非常壮观。

太丘上青色的巨人草,宛若树木一般高大,它们繁密生长,宛若丛林。

巨人草丛之中,数百条长龙,或是身披金鳞,或是额头生角,或有三只龙瞳,或者利齿森森,形态各异。<strong>最新章节全文阅读www.Mianhuatang.cc</strong>身躯摇摆,昂首直立,高度远远超过巨人草丛。

这些都是万首盘龙的一部分龙头,它们的龙尾深深地扎入地下伸出,上万龙身本就源出一体。

而黑楼兰等盘旋空中,宛若苍鹰,不断地投下各种杀招,绚烂夺目,宛若烟火绽放。

长龙们嘶吼,并非被动挨打,而是不断反击。大多数都在喷吐龙息,还有一些或是吐出毒雾,或是龙瞳射光。

“不算对付白凝冰去的那个龙头,一共还有三个七转龙头!其余皆是六转。”

“目前没有发现有什么野生仙蛊,但要小心。”

“不要靠近,以防被拖入万首盘龙的阵道天地里去!”

黑楼兰、影无邪等人一边作战的同时,一边不断地交流着。

琅琊地灵向方源求援,方源自然不能不救。一来,这次救援的门派任务,奖励颇多,令其心动。二来,自己本就是琅琊派的太上长老,不能坐视不管,否则有违身上的盟约。

所以,方源一边赶来,一边急命影无邪等人先行搭救,拖住场面。

战斗越发剧烈,时间并没有持续多久,但是影无邪等人却感受到了越发强大的压力。

“不妙!越来越多的龙头苏醒了。”

“再这样下去,太古龙头惊醒过来,我们救人不成,还会为此搭上性命!”

“宗主何时到?”

“这里是太丘深处,距离传送仙阵颇远,来往总得需耗费一些时间的。小心!”

交流间,又有一只七转龙头苏醒,从地底深处破土而出,嘶吼一声,吐出一口银色龙息。

黑楼兰险而又险地闪避开来,影无邪沉思了一下,对白兔姑娘和雪儿道:“你们退后,遥击辅助即可。”

雪儿只不过是六转修为,早已心头震荡,作战勉强。

她虽是雪民一族的天之骄女,但是和黑楼兰等人自然无法相提并论,闻言也不逞强,直接拽着白兔姑娘远远撤离。

“我们这样撤下来,是不是不太好?”白兔姑娘很是担忧。

雪儿白了她一眼:“咱们听无邪仙子的吧,以我们这种战力,过去只能给她们拖后腿而已。”

群仙又支撑片刻,越来越多的龙头苏醒过来,参与战斗。

上千龙身摇摆,龙首咆哮,黑楼兰等人已完全处于下风。

所幸万首盘龙这种太古荒兽,非常奇特,一经出生,就不能移动,龙头虽然数量众多,但是只能据守此地,并不能追击。

饶是如此,上千龙头喷吐,攻势惊人,黑楼兰等人一退再退,难以抵挡。

“糟糕了,情势对我们越来越不利了。”

“白凝冰还没有出来,恐怕是陷入敌阵,难以脱身!”

黑楼兰等人猜错并没有错,白凝冰在阵道空间中,陷入苦战。为了保护三位毛民蛊仙的性命,更是顾忌颇多。

就在这时,一股难以言喻的气息,扩散全场。

一个远超同类的巨大龙头,缓缓升腾起来。

太古龙头登场!

它睡眼朦胧,眼皮子耸搭着,还未有完全的清醒过来,但却带给群仙极其庞大的心里压力。

“可恶……”

“该如何是好?”

群仙正犹豫之际,影无邪断然下令:“我们撤。”

她们只是六转、七转,怎可能抗衡八转蛊仙。只能撤退,舍弃白凝冰和那三位毛民蛊仙了。

说时迟,那时快,太古龙头忽然张开大口,猛地一吸。

刹那间,天地改易,影无邪等人一点反抗的能力都没有,全数被拖入了阵道空间之中。

“太古龙头!!”白凝冰神色震动,随后她又看向影无邪等人,皱眉问道,“方源人呢?”

影无邪等人只有苦笑。

好在太古龙头困住了影宗群仙之后,就再没有动手,而是龙影渐消,几个呼吸的功夫,消失在了阵道空间中。

它是八转战力,在它眼中,六转、七转的蛊仙微不足道,并不值得它出战。

它将这场战斗,交给其他龙头来处理。

于是,一千多龙头在影宗群仙的四面八方,浮现出来。

影无邪等人顿时陷入艰难的苦战之中。

眼下情景和之前不同,这些龙头在阵道空间中,进退自如,忽然出现陡然消失,一扫之前只能原地据守的尴尬。

群仙尝试一轮,都发现无法冲破这片阵道空间,只得团结在一起,互为助力,拼命死战。

艰险的战斗中,让群仙很快都人人负伤。三位毛民蛊仙、雪儿、白兔姑娘更是险象环生。

逐渐的,影宗群仙难以兼顾其他弱者。

一位毛民蛊仙首先陨落。

雪儿惊呼一声,撤退并不及时,被几个龙头团团围住。

“想不到我要死在这里!”雪儿无力可挡,只能闭目等死,就在这时,一道黑色身影一把拽她,冲出重围。

“你们怎么搞的?害得老娘又要出来救场!”危急存亡的压力,让白兔姑娘转变成了黑菟,正是她出手,救下了雪儿。

雪儿都懵了:“你,你是白兔姑娘?”

“老娘可不是那个和你一样软弱的家伙。”黑菟冷笑,不屑地瞥了一眼雪儿。

雪儿无语。

她总算是彻底明白了:方源的这些属下,影宗的成员,一个个都是怪胎。和她们相比较起来,自己是最弱小的。

有了黑菟出力,场面终于险险稳住。

但好景不长,龙头出现的越来越多,太古荒兽万首盘龙开始真正展现出了它的恐怖战力。

群仙只能支撑,毫无脱困的可能。

她们只能寄希望于方源。

但是方源何时才能到达?

按照战斗的时间,方源早就应该到了。但很快影宗蛊仙发现,这阵道空间中,包含了宙道的成分,所以和外界存在光阴的流速差距。这个发现,像是一块巨石,重重的压在蛊仙们的心头。

不过,渐渐的,影无邪等人却是发现,阵道空间中的龙头竟然开始减少。

他们压力渐小,又稳住了岌岌可危的阵脚。

“定是方源来了!”明悟这一点,群仙士气大振。

但雪儿又不禁心生担忧:“不晓得外界的战斗如何,方源虽有八转战力,但究竟能否抗衡得住这一头万首盘龙呢?”

正想着这个问题,忽然阵道空间破开,众人重归太丘天地。

“啊!”眼前的情景,让雪儿忍不住惊呼一声。

只见地面上血流成河,无数的龙躯无力地栽倒在地上,方圆百亩之地竟都成了尸山血海。

那个太古龙头不断地惨嚎,毫无一丝之前的风采。

而方源……到处都有。

“一念化万千。”影无邪口中轻叹,认出来方源此时运用的仙道杀招。

方源将真身隐匿于万千的幻影当中,那太古龙头却无丝毫妙法,来窥破方源本体所在。

燃念飞石!

无数火焰巨石绵绵不绝,砸在万首盘龙的身上,将大量的龙头直接砸倒下去。

它们想起来,但是念头刚刚生出,就会火焰烧毁。

一念花开!

方源心头一动,顿时倒在地上的龙头龙身上,都长满了绚丽多姿的各种鲜花。

只感觉力量被抽走,原本强健凶猛的龙头陷入虚弱之中,鲜花的盛开仿佛燃料,更助长脑海中火势。

这些攻势,将六转、七转层级的龙头的战力,全都打散。它们趴在地上,宛若一条条肥壮的蚯蚓。

唯有太古龙头气势汹汹,却找寻不到方源的本体,一身战力无从施展。

“这就是智道杀招的威能,由此可见,曾经的紫山真君是有多么威风。”方源心中暗道。

这时,太古龙头见到出来的影无邪等人,立即有了目标,扑向他们。

两位剩余的毛民蛊仙,忍不住尖叫惨嚎起来。

方源现出本体,阻挡在龙头前进的方向上。

“小心!”雪儿脱口而出,心惊胆战地看着太古龙头撞在方源的身上。

巨力汹涌,方源飞射而出,但旋即又飞转回来。

反倒是太古龙头被撞得头破血流,狼狈不堪。

逆流护身印让天庭蛊仙都无计可施,玄妙威能可不是说笑的。

趁此机会,影宗群仙顺利撤到安全距离。

没有了顾虑,方源和太古龙头展开大战。

“好,好强!”雪儿看得心驰神摇,她亲眼目睹了方源如何大占上风,最终将太古龙头斩杀的整个过程。

“这头太古荒兽,有点弱啊。”战斗结束,方源居高临下,俯瞰着斩断的太古龙头,还有那瘫倒一地的数千龙头,口中轻叹。(未完待续。)

<!--80txt.com\_630book-->

\end{this_body}

