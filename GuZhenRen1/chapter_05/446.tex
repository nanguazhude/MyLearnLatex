\newsection{实力井喷}    %第四百四十六节:实力井喷

\begin{this_body}

琅琊福地,云盖大陆。想看的书几乎都有啊,比一般的小说网站要稳定很多更新还快,全文字的没有广告。]

方源独自一人,身处在云土空阔地带。

深呼吸一口气,他平复心境,随后开始,缓缓地催动仙道杀招。

“燃念飞石。”方源心中暗道一声。

他的双眼顿时开始闪烁着琉璃的紫光,然后脑海中念头纷纷涌动起来,身体表面骤然浮现出一颗颗的小石子。

这些小石子,仿佛是被火烧得通红,但却没有一丝热度。

“去。”方源心念一动,他身体表面漂浮的小石子,顿时飞射出去。

石子飞在半空中,刺破空气,却是悄然无声。

与此同时,小石子体积迅猛膨胀,越变越大,起初只是手指头大小,眨眼功夫,就成了面盆一般。

随后,又扩张成马车大小。

小石子变成巨石,与此同时,石头的表面也附着燃烧着赤红火焰。

火焰巨石数以千百,一时间,将这半片天空,都染成红色。

随后,火焰巨石砸在云土表面,掀起滔天的赤焰,但却没有一丝热意,甚至连轰炸的声音都没有。

这并非真正的火焰巨石,它表面上如此,实际上是念头所化。

这是紫山真君的主打攻势之一――燃念飞石――能够让目标的念头烧毁。

它还有一道连招,名为一念花开。

火焰巨石轰炸了一圈后,方源便伸出右手食指,轻轻一指。

顿时,火妖熄灭,从各个被砸出来的陨石坑洞中,忽然百花齐放,无数鲜艳的花朵,节节攀升,争相斗艳。

鲜花烂漫,瞬间将方圆百亩地区,都化为一片五颜六色的广阔花场。

“可惜对象不是蛊仙,只是没有念头的云土,这些鲜花也只是好看,没有任何的效用威能。”方源心念一动,转瞬间,充斥他视野的繁花就都消失不见。

方源身躯轻轻一晃。

仙道杀招――一念化万千!

顿时,在他周围,幻影重生。

起初,只是数十个紫光幻影,但几个呼吸之后,这些紫光幻影的数量个,已经增长到了近十万的恐怖规模!

而方源的真身,则隐藏其中,若是没有强大的侦查杀招,万万不能发现。

这招其实和方源之前的一个战术很像。[\&\#26825;\&\#33457;\&\#31958;\&\#23567;\&\#35828;\&\#32593;\&\#77;\&\#105;\&\#97;\&\#110;\&\#104;\&\#117;\&\#97;\&\#116;\&\#97;\&\#110;\&\#103;\&\#46;\&\#99;\&\#99;更新快,网站页面清爽,广告少,

就是方源先施展万我,然后利用见面曾相识,变作其中一个力道虚影,隐藏真身,让人分辨不出来。

紫山真君这一招的优势在于,它只是单纯的一招,效果上就抵得上方源两招“万我”和“见面曾相识”。因此,消耗的仙元更少一些。

还有一个优势,紫光幻影的数量增长速度,远超力道虚影。

不过弱点在于,这些紫光幻影并无攻伐威能,只是单纯伪装,混淆视听。不如力道虚影,本身还有攻伐之能。

方源消去一念化万千,又双掌轻推,直接推出一大股的云雾。

这是仙道杀招乱方混向雾,能够让蛊仙丧失方向感。

接下来,方源又尝试催动了却势念卫身、一念万流消等招,均是熟练,成功地催动出来。

他把重点放在最后。

仙道杀招――紫念洞悉灵动星芒!

此招催动最为繁琐,但是效用却是相当奇妙和强大。

它能够通过每一击,收集到对方仙道杀招的奥妙,并将断断续续的信息,反馈到方源心头。

一旦情报收集到某种程度,方源就能够凭借智道手段,在战斗中当即研究出如何对付对方的杀招。

然后,仙道杀招意解纷呈就能催发出来,破解敌人的手段,从而奠定胜局。

方源尝试催动。

紫念洞悉灵动星芒的确是步骤繁琐,方源在这方面连续尝试了三次,这才终于成功。

在过程中,还受了一点伤。

有人如故仙蛊在,方源当然不惧这点小伤小痛。

“这些天不间断的练习,已经将紫山真君的手段,差不多都掌握了。”方源心中评估。

此时,距离雪儿主动找上门来,已经过去了半个多月。

这些天里,方源主要还是修行魂道,增强魂魄底蕴。如今的他,已经有了千万级人魂。

仙窍的经营,按部就班。

除此之外,就是练习杀招。

杀招是需要练习的。因为杀招每一次催动,都会有风险。若是催动失败,反而残害自身。

方源拥有无数传承,便有海量杀招。

不过,杀招的基石乃是仙蛊。方源首先敲定练习紫山真君的智道杀招。

紫山真君乃是第一代分魂中的紫,其手段能够和龙公抗衡,自然非常了得。

并且,他在临死之前,还将自家所剩的所有智道仙蛊,都交给了方源。

所以方源才有这样的基础,催用紫山真君的仙道杀招。

当然了,紫山真君的仙蛊并不全面,还是有不少遗缺。这对于方源而言,并不要紧,有着智道宗师境界,关键是有智慧光晕可以依靠,使得方源可以迅速改良杀招。

如今他所催动出来的智道杀招,大多都是经过一些程度的改良。

这一次练习完毕,方源便又来到智慧蛊的所在地。

他盘坐在智慧蛊的身边,沐浴在智慧光晕之下,开始思考和总结这一次练习的心得。

每一次练习之后,他都会总结,并且根据自己练习的体会,对于杀招进行再次改良。

很多杀招的创造,是包括了蛊仙自身的经验、习惯、性情等很多的成分。

这些杀招,只是别人的杀招,远不如方源自创的万我、力道大手印等来的亲切和顺手。所以方源要进行不断地改良。

片刻之后,方源思考完毕,并没有对紫山真君的仙道杀招再修改什么地方。

“这些天来,我不断练习,不断微调杀招,终于到了改无可改的地步了。唯有将来智道方面有巨大进步,方才能继续改良吧。”

明白了这一点,方源也不强求。

他今天推算的重点,不在这方面,而是要完成仙道杀招吃力的推算。

原本,方源有一只仙蛊吃力,能够消耗仙元,催动仙蛊,增长自身的力道道痕。可惜如此极品仙蛊,已经损毁。

不过不要紧,大道三千,殊途同归。方源手中有着这么多的仙蛊,又有智慧光晕辅助我,完全可以迅速地推算出仙道杀招吃力来。

有了这样的仙道杀招,完全就能取代吃力仙蛊的作用了。

“推算成功!”约莫一盏茶的功夫后,方源带着满意的成果,离开了智慧蛊。

推算仙道杀招吃力,是从六天前开始的,到今天终于彻底成功。

这记六转仙道杀招,以小吃仙蛊为核心,其他力道仙蛊为辅助,通过消耗纯粹的力道仙材,来帮助蛊仙增添自身的力道道痕。

一次修行的效果,比六转仙蛊吃力,要更好一些。但消耗的仙元也多,因为牵扯太多蛊虫,也远不如单纯催动一只吃力仙蛊方便。

总体而言,是有利有弊。

接下来的日子里,方源除了练习巩固紫山真君的智道杀招之外,就是屡屡催动吃力杀招,帮助自己增长力道道痕。

他现在身上最多的道痕,是变化道痕,有近五万。其次是冰雪道痕、运道、气道、音道,都有一万多。

力道真的很少,方源没有专门渡过力道方面的灾劫,他吞并的仙窍也没有专修力道流派的。

这一次,他重点补充这个方面。

之所以选择力道,也是方源深思熟虑过的。

他现在手中最强手段,便是逆流护身印和万蛟。前者主防,后者主攻。两者都和力道有关,所以增长力道道痕,对方源战力上限的提升会很显著。

至于修炼的那些紫山真君的智道杀招,主要是丰富方源的手段,真正决定方源战力上限的,还是逆流护身印。

方源勤修苦练,将自己的时间压榨到了极致,每一分每一秒都要得到充分的利用。

这种安全的环境,相当难得。

重生至今,方源终于有了这么一个良机,可以充分地发展自己的实力,积累自己的底蕴。

仙窍经营、杀招练习、魂道修行等等方面,几乎每一天,方源都有一种全新的面貌。

他进步速度骇人听闻,即便是琅琊地灵,也是暗中震动。

这正是天庭方面估料的情形:在得到了影宗传承之后,方源的底蕴深邃到不可思议的程度,如今猛地爆发出来,就会形成实力上的井喷!

雪儿自从接触到影宗成员之后,就表现得相当努力和乖巧。她虽然本身战力不行,但在侦查方面却有一手,和其他影宗成员合作,为方源探索太丘,猎得不少魂魄。

方源对联姻的冷处理,让雪民一族内部出现了矛盾。雪儿的奶奶冰媛,主张坚持,甚至是再送一次彩礼,表达自家的诚意。但以冰风为首的一些雪民蛊仙,却持反对意见。这一方认为,和方源联姻的代价太大,反不如继续忍耐石人一族的欺压。

因为影宗成员的出力,导致琅琊派探索太丘的进度变得很快,这让琅琊地灵开心不已,越看方源越顺眼。

这一天,方源正闭目盘坐,进行仙窍中的建设,忽然他接到求援消息,缓缓地睁开双眼,

与此同时,琅琊地灵也瞬移,出现在了方源的面前。

“方源长老,你快去帮忙啊。太丘那边,出现了太古荒兽,围困了我派的好几个蛊仙!”琅琊地灵神色紧张,语气急切。(未完待续。)

------------

\end{this_body}

