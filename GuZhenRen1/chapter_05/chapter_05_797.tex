\newsection{消化气海洞天}    %第八百节:消化气海洞天

\begin{this_body}



%1
至尊仙窍。

%2
小白天。

%3
龙人分身微笑着,伸出了手掌。

%4
在他的面前,悬空站立着一头幼鹿。

%5
幼鹿在方源的龙人分身前乖巧地低下头,伸出粉嫩的小舌头,轻轻地舔着龙人分身的手掌心。

%6
“主人。”幼鹿开口,却是人的语言。

%7
它当然不是寻常的鹿,而是气相当年的执念,结合了气海洞天的天地伟力,所形成的气海天灵。

%8
气海天灵形如幼鹿,浑身雪白,双眼大大的,泛着金黄色的光,璀璨夺目。

%9
然而在幼鹿的身体右后侧,有一道深可见骨的恐怖伤痕,伤痕一直延伸到腹部之下,残留着浓郁的道痕。

%10
龙人分身叹息一声:“你这层伤势极其严重,我的手段恐怕还不行。”

%11
这是气海洞天渡万劫时,幼鹿操纵气海无量杀招,受了的伤。

%12
上一世,白凝冰攻入气海洞天时,天灵就因为此伤陷入沉眠,随后惊醒过来。

%13
幼鹿乃是天灵,本身这样的存在就很奇特,乃是执念结合了天地之力。

%14
要治疗幼鹿,智道明显是最贴切的方法之一。

%15
然而,方源尝试一番,都没有什么成效。

%16
方源的炼道手段更加厉害,但也不敢随意施为。

%17
不是他拿一道伤口没有办法,而是这些手段强则强矣,却不精妙细微,对于天灵幼鹿十分有害。很有可能治疗下去,把气海天灵都给直接消灭掉。

%18
如果施展治疗的结果是谋杀了气海天灵,那方源何必大费周章,去施展救治呢?

%19
暂时,方源对幼鹿的伤势也束手无策。

%20
不过这头天灵却很看得开,它显得很高兴。因为气海洞天被吞并之后,它却仍旧存在着,没有被毁灭。

%21
任何的生命都有生存的本能,地灵也是如此,它们倾向于认主,保护福地并保护自己。

%22
当然,前提是满足认主的条件,才能让它们服从。或者是动用血光镇灵杀招这一类的手段。

%23
方源思考地灵、天灵存在的奥妙。

%24
蛊世界的奥秘是如此之多,方源接触得越多,就发现自己知道的越少。

%25
普通蛊仙吞窍,即便有地灵,也会在吞窍完毕之后,地灵消失毁灭。但是至尊仙窍打破了这个常规,这当中一定有什么缘由。

%26
可惜影宗真传中没有任何记载。

%27
或许是魔尊幽魂有所保留,影宗真传中的内容并非是他掌握的全部。

%28
又或者至尊仙窍的这番变化,就算是魔尊幽魂也始料不及。

%29
就像方源拥有如此强的炼道造诣,对于炼制出来的陌生仙蛊,也只是参照蛊方了解大致,并不是彻彻底底的明白。尤其是炼制出了一个前所未有的崭新仙蛊,在尝试使用这种仙蛊的时候,蛊仙会常常获得新的发现。

%30
眼下,方源的至尊仙窍中有许多的地灵、天灵。

%31
地灵最多,首当其冲的便是琅琊地灵,他很特别,由两个执念共同组成。

%32
还有灵蛇模样的,星核外表的,黑犬外形的,五花八门,伴随着吞并进来的仙窍,随意地散落在至尊仙窍的各个位置。

%33
它们都是方源治理至尊仙窍的好帮手,掌管着一方水土。

%34
这些天灵、地灵,方源都没有消灭,而是保留下来。

%35
现在的时机不对,他准备在未来,搞清楚这些天灵、地灵存在的玄机。

%36
这也是他想要救治气海天灵的原因。

%37
气海洞天吞并进来,方源将它安置在小白天中。

%38
这带给至尊仙窍一些很明显的变化。

%39
各种气息渐渐增长,并且相互流传。小白天中最为明显,有无数的白云渐渐生成,并且迅速蔓延。

%40
其余小八天也是如此,各有云彩。

%41
这些云朵还只是单纯的稀薄雾气,并未成长为各色的云土。

%42
但这才只是刚刚开始。

%43
根据方源的推算,在接下来的一段时间里,至尊仙窍各个角落都会有气流产生。

%44
这些气息各种各样,分门别类,大多都是凡级蛊材。

%45
不过再过一些时日,有许多凡级气流积蓄得过多,就会从中产生仙材气流。

%46
再过一段较长的时间。这些仙材气流中,有些稳定下来,和附近的环境形成紧密的联系,这就是最初的微型资源点了。

%47
若是发展良好,这些微型资源点就会上升成小型、中型,乃至大型。

%48
当然,若没有人为主动干涉,至尊仙窍中自然产生大型资源点的概率,实在很小。几乎可以忽略不计。

%49
这一切,都是上百万的气道道痕,所带来的影响。

%50
这些气道道痕都是气海洞天的积累,如今气海洞天被方源吞并,这些气道道痕就扩散开来,几乎平均分布在小五域和小九天当中。

%51
每一次吞并仙窍,都会增添道痕,令原本的仙窍环境发生改变。

%52
比如方源上一世,吞并兽灾洞天。增添的变化道痕,会令整个至尊仙窍中的物种变得更加丰富。举个例子,方源的小北原中有一些普通的狐狸,它们会在今后的生活繁衍中,更高概率异变成红狐、灰狐、风狐、云狐等等,甚至转变成异兽秋水狐、流光狐……

%53
而吞并夏槎洞天,宙道道痕暴涨,会令光阴支流变得更加宽阔。方源能够调整的时间流速上下限,都会拓宽许多倍。

%54
因此可见,蛊仙冒然吞窍,也是不好的。

%55
因为,暴涨的道痕会令自家仙窍的环境,发生巨大的改变。若没有智道的手段,蛊仙难以安排调节的话,会令精心经营的仙窍损失惨重,甚至一贫如洗。

%56
仙窍中的生态,牵一发而动全身。食物链中,一个小小的草叶凋零灭绝,就很可能扩散成无数生灵的存亡危机。更何况吞并仙窍后,引发的这种大环境的巨大变更!

%57
方源之所以一直受益,主要是因为至尊太大,太宽广了,分薄了这种影响。

%58
绝大多数的蛊仙,仙窍和至尊仙窍相比,都非常的小。

%59
就像是相同体积的水,灌输在一个瓶子,和一个酒缸。瓶子会被灌满,而酒缸只是底部薄薄的一层水。

%60
除此之外,方源深厚的智道造诣,对应的流派境界,都可以令他从容应对一切的变化。

%61
“之前吞并夏槎仙窍,也就算了。但这一次吞并气海洞天,气道道痕上百万,至尊仙窍里的变化就明显得多!”

%62
夏槎本人没有渡过一次万劫,所以宙道道痕只有七万左右。

%63
方源上一世吞并兽灾洞天,收获变化道道痕十几万。这是因为兽灾仙人撑过第一场万劫时就挂了,遗留下来的兽灾洞天还未迎来第二场万劫。

%64
气相当初,实力也不错。

%65
当时,五相并称,纵横两天,名震五域,风头一时无两。

%66
气相去世,气海洞天留下来,一直没有吞下太古九天碎片,饱受灾劫困苦。

%67
苦难造就财富!

%68
一千多年下来,气海洞天渡过了十多场的万劫。每一场万劫,至少要有十万左右的道痕斩获。

%69
再加上气象生前积累,如此种种,就有了过百万的气道道痕了。

%70
这是惊人的成就。

%71
很不容易!

%72
须知万劫之恐怖,八转蛊仙都惶恐不安,很难撑过去。

%73
但是气海洞天一步步走到了今天。

%74
主要依靠的就是气海无量杀招!地灵操纵此招,八转仙元又是足够,屡屡帮助气海洞天化险为夷。

%75
“当然,此招精妙,不仅消耗仙元,更耗损各种气流。气流的损耗一方面能缩减仙元的开支,另一方还能增强此招的威能!”

%76
“这一千年俩,还有戚家蛊仙,四处帮忙,收集这些气流。”

%77
相比较之下,兽灾洞天只是一个小年轻。

%78
琅琊福地虽然历史更悠久,但它是洞天故意跌落成福地,它不受万劫,最高的灾劫不过是浩劫,因此斩获的道痕是很少的。

%79
除此之外,琅琊福地还利用杀招,影响灾劫的内容,故意增添了不少的水道道痕。

%80
这些水道道痕,又和炼道道痕相互融汇,产生炼水。

%81
琅琊福地为了人造天地秘境,损失掉了难以计数的炼道、水道道痕。

%82
“若非我吞并气海洞天,依照幼鹿天灵的伤势情况,也几乎撑不下去了。接下来的几场万劫,就很可能摧毁了这片洞天。”

%83
所以,琅琊地灵的选择也不算错。若是琅琊福地始终保持在洞天级数,就算获得数十倍的炼道道痕,也绝对撑不到现在来。

%84
方源继续潜修。

%85
一方面,他积极调整,努力将气海洞天完全消化。另一方面,他开始练习气道杀招。

%86
原本巴十八的连击术,方源浅尝辄止,已然放弃。

%87
现在的重点,是种种气道杀招,尤其是气海无量!

%88
气海无量的威能,自然是厉害至极,能够帮助气海洞天屡屡抵挡浩劫,就可见一斑。

%89
上一世方源气道底蕴不足,时间不够,资源匮乏,所以拆分了气海无量杀招。

%90
但这一世,方源气道大宗师境界,吞并了气海洞天,气海无量杀招随身携带。

%91
至尊仙窍中的时间流速,远不上一世还要高得多!

%92
这就为方源争取到了更多的,练习杀招的机会。

%93
方源气道的底蕴一步登天,转化出战力来,必将能一日千里,日新月异!

%94
“气道战力若成,就将超越宙道,成为我的主攻手段。”

%95
“上百万的气道道痕,呵呵,可不是说笑的。”

%96
“到那时,我就可以去尝试收服龙宫,有资格在八转蛊仙的围攻下,将这座仙蛊屋强行收取了。”

%97
方源目光幽幽,早有谋算。

\end{this_body}

