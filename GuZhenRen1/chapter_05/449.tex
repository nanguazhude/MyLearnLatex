\newsection{太弱了}    %第四百五十节:太弱了

\begin{this_body}

炼蛊大厅中,充斥着白色光辉。求书网小说qiushu.cc

光芒万丈,让人不可逼视。

一股浓郁的气息,介于凡蛊和仙蛊之间,显然炼蛊已经到达最关键的地步。

方源沐浴在白光的最深处,将全部心神都投入其中。在他身边,则站立着炼道蛊仙毛六,满脸的凝重和紧张。

“就是现在!”忽然,毛六开口,竟是将整个炼道蛊阵都撤消了去!

原来,要炼制出天机仙蛊,最适宜的法子,便是光道炼法。除此之外,其他流派的炼法,都只能事倍功半,甚至某些还有反作用。

而在这最后一步,还需要将炼道蛊阵都撤去,纯粹利用蛊仙的灵机妙念,来达成最后一步。

这是天机仙蛊炼制最为特殊的地方,其他仙蛊几乎没有这个要求。

唯有天机仙蛊,受不得任何约束和规限,须得在自由妙动的情形下,才能炼制得出。

毫无疑问,这对炼蛊的蛊仙而言,是一个巨大的考验。

因为,蛊仙需要在这一刹那间,取代原本仙阵的作用,引导整个炼蛊的过程达成最后一步。

但是蛊仙的引导作用,又不能超出分寸。一旦超出,就有了限制和约束,天机仙蛊就凝聚不了。

这需要把握一个玄妙的平衡,其中微妙之处,难以用言语表达,只有炼蛊的蛊仙才有亲身体会。

炼蛊最后一步。

方源脑海中无数念头喷涌,动用智道手段,他完美地承担起原本仙阵的作用。

周围的白色光芒,刺眼至极,但在方源的引导之下,都开始凝聚回缩。

渐渐的,这些光辉在方源的周身,凝聚成一个圆环形状。

巨大的圆环,将方源纳入圆心处,起初只是光辉凝聚,但几个呼吸之后,就有了实质的触感。

十几个呼吸后,炼蛊大厅内的光照已经恢复到正常状况,而圆环则凝如实质,有房屋大小,通体宛若白金浇筑。

圆环静静地悬浮着。

方源站在圆环的最中心。

毛六屏住呼吸,心中隐隐激动起来:“炼道杀招天光轮转,终于成型了。但是要至少转七个圈,才能真正炼出七转的天机仙蛊啊。”

第一转。

方源心念轻轻一动,圆环缓缓启动,速度相当缓慢。

这一转,就转了一盏茶的功夫,这才完全转了一个圈。

“第一转成功了。”毛六吐出一口浊气。

第二转。

方源将一直紧闭的眼帘,缓缓睁开。

从他的眼中,绽射出一抹奇光,照射在圆环上。

原本静止不动的圆环,似乎在这道目光中得到了动力,再次转动起来。<strong>小说txt下载Http://wWw.80txt.com/</strong>

这一次转动,速度明显比之前快一些。

半盏茶的时间后,转圈完成。

“第二转也成功了。”毛六捏了捏拳头。

第三转。

方源鼻息粗重起来,不断地呼吸。

这呼吸当然不是普通呼吸,而是动用了炼道杀招催谷而出。

气息和圆环产生交流,圆环再次转动起来。

这一次转动的速度更快,片刻后,转圈成功。

“第三转也完成了。”毛六心里砰砰直跳。

第四转。

方源张开口,开始低吟。

在他声音的催动下,圆环转速越加迅猛。

很快,第四转也成功了。

“总共七转,已经完成超出一半!但是,接下来会越来越困难,因为圆环转数越来越快,非常不容易把握。”毛六额头已经垂下汗渍。

第四转结束,方源却没有急着继续,而是休息了好一会儿,这才开始催动炼道杀招,令圆环自转。

第五转。

方源的双臂原本自然垂下,但此刻他开始缓缓提起双臂。

像是双臂上挂着两座山峦,方源以一种极其沉重、缓慢的姿势,抬升双臂。

圆环自转开始,但与前几次不同,它在自转的同时,还在缓缓地上升。

距离地砖越来越高,当它从方源的腰际升到方源肩膀的时候,忽然猛地一顿!

“糟糕!”毛六脸色骤变,与此同时,圆环猛地自爆开来。

轰!

一声巨响,剧烈的爆炸还未殃及周围,就被整个炼蛊大厅给镇压住。

但在中央的方源,却是承受了爆炸的全部威能。

噗!

方源脸色骤白,张口吐出一大口鲜血。

重伤!

仙蛊人如故。

方源立即催动这只仙蛊,无往不利的手段,立即显现出良效。方源重伤直接转为轻伤,但伤势并未痊愈。

方源炼制七转天机受伤,动用六转人如故,虽然发挥了效果,但是人如故的宙道效果,并不能完全抹消掉伤势。

方源这次伤势的严重程度,可见一斑。

天机仙蛊的炼制失败了。

这已经不是第一次失败。

前几次,毛六都在中途失败,这一次走到最后,方源亲自出手,仍旧是失败收场。

一时间,方源和毛六的脸色,都不好看。

这一次方源也是充分练习,炼蛊之前更是动用了诸多运道手段,但仍然没有成功。

炼蛊就是这样,想要成功,非常不容易。投入进去很多,未必能收获,最终很可能竹篮打水一场空。

关键却还有一点,那就是方源的灾劫即将来临。这一次炼制天机没有成功,也就基本上意味着,方源来不及再炼制第二次了。

“可惜了。”毛六叹息不已。

“无妨,即便是没有天机,我也有信心渡过。”方源却是摆手,短短功夫,他的心情又恢复了平静。

数日后,方源通过传送仙阵外出,来到无人的荒地,开始渡劫。

落下仙窍,他打开仙窍门户。

呼呼呼!

海量的天地二气,顿时汹涌而入,宛若天河倒灌,填补到至尊仙窍当中去。

浩瀚至极的壮观景象,将影无邪等人都看懵了。

这一次方源渡劫,为了确保万无一失,也将影宗群仙都带来。白凝冰也在其中,但雪儿却留在琅琊福地里。

雪儿还不受方源的信任,至于其他人,都是进入过至尊仙窍里的。

方源渡劫时,他们可以帮衬一二。或者当有蛊仙被吸引过来,前来破坏时,他们也可以外出,为方源阻挡来敌。

这就是人劫了。

方源渡劫时,受到天意的关注,这点是没有办法的。

天意千方百计想要铲除方源,自然就会布局,布局的时间一长,就会酝酿出人劫来。

“非是亲眼所见,绝难想象得到,竟有人的仙窍单单吞吸天地二气,就有如此壮阔气象!”

“至尊仙窍……实在太强了。我相信,就算是九转尊者吞吸天地二气,都未必有如此景象。”

影宗群仙交口称赞。

白凝冰冷哼一声,眼中精芒烁烁不定。

她心中震惊,不过口中则道:“你们太过小瞧九转尊者了。我虽没有亲眼见过九转仙窍吞吐天地二气的情景,但继承白相真传,却知光是白相当年,落窍吞吸天地二气的规模,就比方源宏大得多。”

影无邪淡淡地瞥了白凝冰一眼:“白相乃是八转,拥有的是八转洞天,方源此时不过七转,只是七转福地而已,两者不能比较。况且,至尊仙窍广博空阔,许多地方都未建设,成长潜力难以想象。反观白相当年,却是已经建设全面,资源充斥洞天,这些资源消耗的天地二气当然巨大。”

白凝冰默然不语。

“无邪仙友所言甚是。”妙音仙子娇笑一声。

其余蛊仙亦都点头,表示赞同。

好一会儿功夫,至尊仙窍才吞吸天地二气完毕,灾劫开始在小赤天中成形。

方源如今肉身身处在小赤天中,这场灾劫针对方源,自然不会在其他地方显现。

至于为什么选中小赤天,那自然是因为这里空阔一片,曾经作为和太古年猴战斗的场所,没有铺建任何资源,在这里渡劫,极大地减少损失。

若是在小南疆、小北原等地渡劫,灾劫肆虐下来,损毁无数资源,这绝非方源想要看到的情况。

“至尊仙窍太大了,若是我们渡劫,怎可能避免灾劫殃及资源?但是在至尊仙窍里头,灾劫根本都不能囊括小赤天!”黑楼兰以极其羡慕的语气叹道。

“我影宗辛苦积累十万年,这才炼成的至尊仙胎蛊,自然不同凡响了。”影无邪冷哼一声,语气骄傲。

“可惜一切都为方源做了嫁衣。”白凝冰悠悠一句,立即惹得影无邪怒目而视。

“怎么?我说错了吗?”白凝冰毫无所惧,挑衅地看向影无邪,找回了刚刚的场子。

影无邪深呼吸一口气,扭过头来:“至尊仙胎蛊给了方源大人,又有什么不同?他现在就是我影宗宗主,我们都是同一阵营。”

“哈哈哈。”白凝冰发出嘲讽的笑声。

影无邪却充耳不闻,只盯着已近成形的灾劫。

只见方源周围,天地之间,充斥着厚重至极的毒雾。

毒雾滚滚荡荡,分有五色,泾渭分明。

“这是毒道的灾劫啊。”白兔姑娘弱弱地道。

“准确的说,是五毒劫!”影无邪开口道,他的眼界自然更加宽阔。

果然如他所讲的那样,五种色彩的毒雾,分别凝聚成五个巨大的怪物形态,分别是蛇、蝎、蜘蛛、蟾蜍、蜈蚣。

五大怪物围绕方源,虎视眈眈,却不着急进攻。它们在积累力量,随着时间流逝,它们的体型越发巨大。

“方源怎么还不出手?”白凝冰皱眉。

此时的确是最好的出手时机,否则时间过的越久,毒物就越强大。

不过,方源的战术策略,却是和通常的思维不同。

他等待着,一直等到五大毒物的力量积蓄到了极限。

五大毒物各个体型如山,撑天踏地,向方源扑来!

白凝冰等人不由地屏住呼吸,灾劫威势之强,让她们观战的人都心中震撼。

这不是寻常的七转灾劫,而是在天意的加持下,达到天地大道允许极限的最高威能!

“只是如此么?”方源冷眸一转,身影骤然消失在原地。

轰轰轰!

下一刻,雷霆般的爆炸声不绝于耳。

白凝冰等人瞳孔微缩,震骇地看着方源纵横战场,五大毒物在他的种种攻势下,溃不成军。

片刻之后,毒雾分崩瓦解,只余方源如魔神般的身影,悬停在战场之中。

方源俯视战场一圈,又望了望自己的双手,口中低喃:“七转的天劫……太弱了。”(未完待续。)<!--80txt.com-ouoou-->

\end{this_body}

