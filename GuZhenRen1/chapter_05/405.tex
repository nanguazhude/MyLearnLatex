\newsection{打得凤九歌吐血!}    %第四百零五节:打得凤九歌吐血!

\begin{this_body}

方源径直地杀向凤九歌。

当然,他的身上早已罩着一层仙衣,仙衣波光粼粼,又有长长弯曲的绶带,环绕方源的腰际和手臂。

随着方源冲袭而去,仙衣绶带也随之飞扬摆动。

仙道杀招——逆流河护身印!

“又是这一招。”凤九歌见此,心中一叹,他对这一招很是头疼,非常无奈。

所以,凤九歌直接选择后撤。

很轻松的,他就和方源拉开了距离。

“操控逆流护身印杀招,牵扯太多的心神,方源根本无法再分心他用,即便他有腾挪的仙道手段,也用不出来。如此一来,当他催动逆流护身印的时候,除去防御之外,其他方面就很稀松平常了。”

凤九歌暗暗思虑,嘴角挂着一层笑意。

敌进我退,方源追不上凤九歌,尝试了好几次,都是徒劳无功。

凤九歌眼底闪烁着思索的光,脑海中的分析一刻都未停止:“怎么还是方源一个人?”

这一次对战凤九歌,方源周围并没有白凝冰、黑楼兰等人相助。因此,凤九歌只有面对方源,之前一战时,他采用的铲除其他蛊仙的战术,就直接落空了。

“但是,如此一来,方源根本就没有获胜的希望。就算防御出众,他也只能是一个靶子而已。不管是攻伐亦或者腾挪,都不及我,战场的主动权拱手相让。”

“不,方源绝不会如此不智!他一定是有其他的阴谋诡计。”

凤九歌没有大意,反而越发地谨慎。

方源追击过来,他就远远后退,拉出安全的距离。

“凤九歌,你追杀我而来,眼下却只是逃跑吗?”方源冷笑喝问。

凤九歌笑意扩散:“何必逞口舌之利?方源,你还是先追上我再说吧。你催动逆流护身印不容易,时间久了,不管是心力憔悴,恐怕连仙元都要见底了吧?如此强大的手段,必然有着巨大的代价。”

凤九歌牢牢占据主动,不仅是速度方面,更在战术上压制了方源。

这一次开战,和上一次大不相同。

上一次一交手,就是水深火热。这一次,两人一追一逃,除了说了几句话外,反而悄无声息。

方源冷哼一声,凤九歌如此做,他也有应对之法。

五百年前世的经历,可不是说笑的,别的不说,单单战斗经验,那绝对是丰富老道得很。

于是下一刻,方源就俯冲而下,落到了地上去。

凤九歌没有追击,而是谨慎地悬浮在高空中,然后细心打量这片沙漠。

他上一次就是吃了地利上的亏,被方源利用绝音戈壁坑了一次。

这个教训凤九歌当然铭记在心,来之前的路上,早已经做足了功课,并未发现这片沙漠有什么特殊的。

噗嗤。

一声轻响,方源直接钻入沙漠当中,向地底深处进发。

凤九歌不断侦查,动用仙道手段,确认真的没有什么猫腻,便采用拳鼓、掌钟,对地下的沙漠进行轰砸。

咚咚咚!

铛铛铛!

一时间,鼓声和钟声交响,炸得沙漠上出现一个个的深坑大洞,大量的沙子飞溅出来,将方圆数里都蔓延出黄褐沙尘。

但方源不管不顾,仍旧是继续向地底深入。

凤九歌皱起眉头。

越是深入地底,土道道痕就越多,对于凤九歌而言,自然不利了。

同时,他心底还有怀疑:“方源这一次如此自信,难道说,他提前在地底有所布置?或者是知晓这片地下,有着什么特殊的环境,可以利用?”

凤九歌脑海中各种念头急速闪烁,随后,他电射而下。

整个人如一根利箭,直接插入沙漠,向方源追去。

他不得不追。

因为方源若是在地底,利用仙阵继续削除身上的侦查杀招,凤九歌必定还要追下来。

就算不是削除身上的侦查杀招,而是在地底布置蛊阵或者战场杀招,那么凤九歌若犹豫不决,没有追击,反而会留给方源大量的时间,让他充分准备妥当。这样的话,凤九歌就太过愚蠢了。

所以,凤九歌必须要追击,要给方源施加压力,逼迫他暴露更多的底牌,打破他的战斗节奏。

拳鼓、掌钟!

即便是在地下深处,凤九歌的音道杀招,仍旧威力绝伦。

爆炸不断地在方源的身边发生。

方源满脸沉着之色。

这一次没有了绝音戈壁的地利,方源充分感受到了凤九歌的威猛攻势。

尽管在这地底,有着越加丰富的土道道痕,凤九歌的杀招仍旧压迫力十足。

这就是音道的优势。

每一个流派,常常有与众不同的优缺点。

音道虽然是一个小流派,但它也有自己的优势。

这个优势就是,音道利于传播,对于道痕互斥的效果比较小。举个例子,寻常流派,诸如土道杀招,遇到水道道痕充裕的环境,可能威能会被削减五六成。而相同威能的音道杀招,只会被削弱两三成。

“如果没有逆流护身印的话,我中了这么多击,根本支撑不到现在。”

“凤九歌究竟是如何修行?在七转地步,就拥有了可以媲美八转的道痕积累?”

“不想这些了,开始反击吧!”

仙道杀招——上古剑蛟变!

一瞬间,刺眼的光辉在方源的身体上爆发出来。

但是在密实的地底,光芒并不能透露扩散出去。

“嗯?”方源变化的那一刻,凤九歌的侦查杀招,就感受到了丝丝不妥之处。

然后,几乎在一个呼吸的时间,凤九歌面前的泥沙陡然爆炸,从中露出一个巨硕的蛟头。

凤九歌的小身板和这蛟龙头部比较起来,宛若是大象面前的老鼠。

“上古剑蛟!方源!”凤九歌的眼眸在这一瞬间,缩成针尖大小。

方源的反击,来得是如此突然,如此狂猛!

蛟龙口部大张,从中喷涌出一道璀璨澎湃的银色光辉。

剑蛟龙息!

一时间,银白色的光辉如千万利箭,带着强烈的侵略性,映照得周围银光灿烂。

犀利无比的气息,还有一段距离,凤九歌就感觉到面皮一阵阵的刺痛,好像是有数百小针在扎他的脸。

轰!

剑蛟龙息喷吐而出,狠狠地击中凤九歌。

凤九歌身上,顿时叮叮咚咚地一阵乱响,防御手段被激发出来,冒出无数的细小音波,又旋即被接踵而至,犀利无双的剑蛟龙息硬生生摧毁。

普通的龙息,只有短短一段,但方源拥有着坚持仙蛊,导致龙息生生不息,持久至极!

凤九歌被狠狠击退,在剑蛟龙息的冲击之下,艰难地调动手段,电射到上空去,暂时摆脱了龙息的冲击。

但下一刻,方源猛地昂首。

粗壮的剑蛟龙息,就仿佛是一道白银光柱,旋即又杀上来,仿佛一柄绝世利剑,往上轻轻一撩。

“来了!”凤九歌此刻再无微笑,满脸都是映照着银光。

眼看着剑蛟龙息就要再次袭来,他不得不再次撤退。

仙道杀招——一曲阳关。

刷的一声,他消失在原处,然后在千步之外的高空中闪现出身影来。

方源收住龙息,长啸一声,继续反攻压上。

矫健修长的蛟尾一甩,上古剑蛟本身的速度立即爆发出来。与此同时,他还催动了剑遁仙蛊。

数万的剑道道痕,一同增幅,带给方源骇人的冲击速度!

轰!

一声巨响,空气被挤破,爆发出音爆的声音。

刹那间,方源就冲到了凤九歌的面前。

速度之快,方源自己都有点反应不过来,眼界一片模糊,原本视野中一个小黑点的凤九歌,下一刻已经站在自己的面前,距离自己十几步的距离。

蛟口再一次猛地张开,但这一次,凤九歌也已经充分反应过来。

他轻啸一声:“来得好。”

随后,面对喷涌而至的剑光龙息,他竟然不闪不避,反攻过来。

轰!

一阵震天动地的爆炸,凤九歌倒飞而出,宛若炮弹一般。

他的脸上首次显现出一抹惊异之色,心底充斥着一个疑惑:“怎么可能?方源的身上,怎么还有那一层逆流护身印?”

之前一战,方源变身上古剑蛟,是无法同时维持逆流护身印的。

刚刚那一记互拼,凤九歌就是看准这一点,硬打猛冲,结果被逆流护身印将自己的攻势完全逆反了回来。

噗。

凤九歌胸口剧痛,直接吐出一小口鲜血,艰难地维持身形,停顿在了高空中。

刚刚那一击,不仅是有方源本身的攻势,还有凤九歌自己被逆反回来的威能。两者叠加,直接打得凤九歌吐血!

方源占据上风,一声龙啸,继续追杀过来。

刚刚来的太快,这一次凤九歌看清楚了,在上古剑蛟银光闪闪的修长蛟龙身躯上,仍旧覆盖着一层薄薄的,像是流水一般透明的衣裳。

“真的是逆流护身印?”

“他竟然能同时维持两个仙道杀招!是怎么做到的?”

一曲阳关!

方源扑了一个空,凤九歌消失在远处。

方源变化的上古剑蛟,没有直接破空的能力,只是直线飞行的速度极快。

而凤九歌的一曲阳关,却是有类似宇道威能,直接破开长空,从一个地点,瞬移到另外一个地点。

从这点来看,方源纵然变身上古剑蛟,又有剑遁仙蛊,仍旧比不上使用一曲阳关的凤九歌。

除非方源将薄青传承中的剑道移动杀招,成功地催用出来。

但这又怎样?

一曲阳关是有时限的。

时限一到,凤九歌在腾挪方面,就会落在下风了。

“必须尽快破局,趁着一曲阳关还有时间。”望着再度冲杀过来的方源,凤九歌心头一沉。

备注:修养的一周里,身体渐渐恢复,但是还没有彻底康复。另外情节构思方面,颇费脑细胞。接下来一周,偶尔两更,若是有第二更,就在早上8点。保底一更,在晚上8点。

\end{this_body}

