\newsection{黑凡用蛊}    %第一百零九节:黑凡用蛊

\begin{this_body}

%1
琳琅满目的仙道杀招,此刻都展现在方源的眼前。

%2
方源按捺住心中的激动之情,细心查看。

%3
一个个的仙道杀招,涵盖全面,涉及攻防、挪移、治疗等等方面。

%4
这都是宙道仙级杀招,运用的核心,大多数是年蛊、以后蛊这两只仙蛊。少部分的仙道杀招,是以仙蛊似水流年为核心的。以八转仙蛊为核心的仙道杀招,自然是效果极佳。

%5
对这点,方源早有体悟。

%6
他的见面曾相识,就是以八转态度蛊为核心的传奇手段。

%7
当然,现在的方源只是六转修为,青提仙元可催动不起似水流年蛊。不像态度蛊,完全可凭借心力就可操纵。

%8
“态度蛊可是人祖传中所记载的传奇仙蛊,在运用方面,似水流年仙蛊还是有所不如的。不过说起来,态度蛊不也是黑凡给与黑风月的么”

%9
这些仙道杀招,让方源大开眼界。

%10
其中涉及到的种种原理,蛊虫之间的搭配,让他稍稍一瞥,就有一种大有收获之感。

%11
仙道杀招的数量过了三十,有些仙道杀招还分设了六转、七转、八转的层级,有些仙道杀招的备注中,还留下了黑凡当年创造时候的心得和体会。

%12
“丰年?”方源目光微微一顿。

%13
他在其中现了一个很有意思的仙道杀招。

%14
丰年。

%15
这就是仙道杀招的名称。

%16
运用的核心仙蛊便是仙级年蛊。

%17
辅助的凡蛊,有数万只,分门别类。催动的过程也十分繁复,黑凡在这里留下了他的运用经验。即便是他这种传奇人物,亲自催动这个仙道杀招,也得耗用两三天的时间,才能将这杀招催动起来!

%18
消耗的仙元极多,但丰年的效果,却是在杀招催成功的时间算起,之后的一年之内。大量增加仙窍中的各项资源产出!

%19
这是一个用来经营仙窍的仙道杀招。

%20
很罕见的杀招,但实用性非常的强。黑凡对这个仙道杀招也赞誉有加。

%21
根据介绍,这个仙道杀招不是他创作的,而是和一位叫做丹仙的人换来的。

%22
方源稍稍细看了一眼。丰年要求的凡蛊有许多,他要准备的话,还得从宝黄天着手。

%23
这杀招他肯定是要用的。

%24
只是不仅是要付出大量仙元,还对年蛊的消耗也很大。

%25
“我得手的这只年蛊,只是一百多年。要将其平炼。才可运用丰年。否则的话,降为凡蛊,就很亏了。”

%26
方源继续看下去。

%27
这些杀招,有的威力大的惊人,有的精妙无比,出想象。但对方源而言,都不是特别重要。

%28
方源至始至终都没有忘记自己的主要目的。

%29
那便是寻找到方法,延缓自己的仙窍灾劫。

%30
“度日如年!”他双眼一亮,看到了自己的目标。

%31
这个仙道杀招,仍旧是以仙级年蛊为核心。大量的日蛊作为辅助。一旦催动起来,就能够让仙窍的时间放缓。

%32
单纯从字面上理解的话,仙窍中的一日光阴,延缓到之前一年那般漫长。

%33
不过,之后黑凡标注的使用心得中提到:将仙窍一日延缓成一年那般,只是纯粹的理论。真正的实际效果,还参照具体的情况。

%34
比如一位力道蛊仙,以力道道痕为主,这个宙道杀招的效果就要大打折扣了。

%35
若是宙道蛊仙自己施为,就像黑凡对自己运用。那效果就受到本身蛊仙的宙道道痕增幅,效果反而好上许多。

%36
再比如一个下等福地,和上等福地,同样的杀招。两者效果也不一样。

%37
“度月如年。”

%38
方源接着看,下面还有一个仙道杀招,和度日如年很是相似。

%39
它同样是延缓仙窍时间,将仙窍中的一个月份,延缓成一年。核心蛊仍旧是仙级年蛊,但辅助的凡蛊则是月蛊。

%40
日蛊、月蛊、年蛊。

%41
这些都是宙道蛊虫。可以相互类比。

%42
除了度日如年,度月如年之外,还有两外两个仙道杀招,分别名为度年如月、度年如日。

%43
前两者是延缓仙窍时间流,后两者则是加快仙窍的时间进程。

%44
延长仙窍时间,自然就要将每次灾劫的间隔时间拉长。譬如方源的话,原本五域外界两个月的时间就要渡劫。用了度日如年之后,就要数十年甚至百年之后,再渡下一次的灾劫。

%45
看到这里,方源算是松了一口气。

%46
他的心中荡漾起喜悦之情。

%47
之前的三次地灾,几乎将他逼得走投无路,只能硬着头皮去闯。

%48
现在有了这个仙道杀招,却是立即不同了!

%49
从此之后,方源可进可退,有了转圜腾挪的余地,意义极其重大。

%50
“不过目前,还是不用延缓仙窍时间的。多渡几次地灾,就能增长道痕,提升修为。有楚度帮我,渡劫成功的可能极大。”方源暗忖。

%51
一旦延缓仙窍时间的话,也是有弊端的。

%52
会让资源增长的度陡降,方源的修为提升度也下降一大截。

%53
“接下来就是准备这些蛊虫,仙蛊是无须准备的,日蛊、月蛊都需要,而且数量很多。恐怕就算是从宝黄天收购的话,也要持续收购很长一段时间。”

%54
黑凡真传中并未留下什么日蛊、月蛊、年蛊这些凡蛊,导致方源还得自己收购。

%55
这个事情虽然有些难度,但只要方源下定决心,就能促成。

%56
“这四个仙道杀招,可以算作是一套杀招了,专门用来改变仙窍的时间流。创造出这个杀招的黑凡,可谓才情高绝。”

%57
在最后,黑凡还提到:若是后继者能够得到仙级的日蛊或者仙级月蛊,那么这四个杀招就无须耗费那么多的凡级日蛊、月蛊了。并且杀招的效果也会提升不少。

%58
“我能得到仙级年蛊,已经很庆幸了。至于仙级日蛊、月蛊,就算了吧。”方源没有得陇望蜀。

%59
此行的主要目的,他已经达到,心中已是满足。

%60
他继续往下细看。

%61
仙道杀招年富力强。

%62
很快,方源就看到了一个很有意思的杀招。

%63
这个杀招,同样是以年蛊为核心。但却能赋予蛊仙巨大的力量。年蛊消耗得越多,蛊仙获得的力量就越强。

%64
这也是黑凡独创,是用宙道蛊虫来体现出力道效果。

%65
仙道杀招百年好合!

%66
同样是以年蛊为主,大量信道凡蛊为辅助。酿造出的盟约效果,有效期是一百年。

%67
这是用宙道蛊虫,展现出信道威能。

%68
“倒是可以和楚度用用。毕竟我和他的盟约,是他一手操办的,不太稳当呢。”方源心中迅闪过这个念头。

%69
仙道杀招流年不利。

%70
这个却是以似水流年为核心的仙道杀招了。方源目前还运用不了。只能看着流口水。

%71
以宙道达成运道的玄妙。

%72
这是攻伐杀招,一旦敌人中招,在一年之内就会运气衰落极大程度,处处不顺。

%73
“不愧是黑凡,八转蛊仙中的佼佼者!”

%74
“如果他是中洲十大古派出身,肯定会被招进天庭,绝对是天庭蛊仙中的精英。”

%75
能够以自身的主修流派,旁通到其他流派,达成其他流派的玄机妙用。这是八转强者的一个标志。普通的八转蛊仙,还达不到这种程度。

%76
黑凡的强大。类似于天庭中的监天塔主。

%77
监天塔主也是八转蛊仙,主修智道。但在薄青仙僵苏醒,剑光纵横中洲的时候,他为了修复监天塔,竟可动用修复。修复的效果,还比寻常的炼道手段要好得多。

%78
这便是以智道,达成炼道的妙用。

%79
“咦,后患无穷?这是什么招数?”

%80
在最后,方源又看到了一个仙道杀招。

%81
它的核心是以后仙蛊,蛊仙一旦催动成功。便能将这一次的灾劫挪移到下一次去,和下一次的灾劫一同爆。

%82
方源心头一跳。

%83
在这个杀招的介绍中,同样有黑凡留下的告诫。

%84
这个杀招不能过多运用,虽然能够将灾劫挪移到下次去。但却触怒天意,会使得灾劫威力翻倍。

%85
这个仙道杀招虽然也是黑凡创造,但他也只用了一次,就再未用过。

%86
为了警醒后人,黑凡特意取了“后患无穷”这个名字。

%87
“只能救急,不能常用。但却绝对是一个上佳的手段。”能得到这个。方源很是有些惊喜了。

%88
触怒天意?

%89
他早就无所谓了。

%90
反正天意想方设法,要将他铲除。方源的存在本身,就是对天意的最大冒犯和触怒了。

%91
花了一些时间,方源终于将仙道杀招的内容,浏览完毕。

%92
非常全面,连经营仙窍这个方面,都涵盖进去了。更别说,之后还模拟其他流派的妙用。

%93
虽然,黑凡真传的仙蛊,只有这么几只。

%94
但搭配各种凡蛊,组合成无数的仙道杀招,足以让蛊仙应付任何情况。

%95
这才是正统修行的常态。

%96
黑凡虽然强大,但本身连十绝都算不上。所以他养蛊,数量少,都是宙道蛊虫,可以负担。

%97
蛊虫多寡不是关键,运用蛊虫的蛊仙才是。

%98
黑凡就是个最好的例子。

%99
明明只有两三只蛊,但他用蛊,却能挥出各种妙用。

%100
没有最强的仙蛊,只有最强的蛊仙。这句话的道理,也在于此。

%101
ps:最近很累,身体好像也变得虚弱了,很不舒服。今天这章尤为难写,之前有个幸亏及时现改掉了。其实最近有很多次想放弃,想暂时休息一下。但想到大家,还是咬牙坚持下来。也希望大家能理解我一二。感激不尽了。

\end{this_body}

