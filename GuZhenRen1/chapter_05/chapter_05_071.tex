\newsection{我意蛊}    %第七十一节:我意蛊

\begin{this_body}



%1
方源的魂魄在落魄谷中游荡。

%2
他并不游走太远,只在方源肉身周遭范围。

%3
在这里每隔一段时间,就有一阵迷魂雾自行飘来。

%4
方源魂魄沐浴在迷魂雾中,像是书本泡在水中,渐渐松散开来。

%5
“越是凝实的魂魄,就越不容易松散。在这里修行了这么多天,魂魄的凝实程度,是以前的十多倍,可谓进步神速!不过,和魔尊幽魂比较起来,还是天地差距啊。”方源暗自评估,难免又想到义天山大战的情景。

%6
幽魂魔尊是魂道创派祖师,虽然身陨,但是留下的魂魄,却有着天下第一的魂道底蕴。

%7
魔尊幽魂的凝实程度,极其恐怖。

%8
正常的魂魄,像是一片虚幻的光影,最多干扰心智,无法对纯粹的物质有所影响。

%9
但魔尊幽魂由于凝实到了极点,竟然能只手遮天,硬抗浩劫。擎天之姿,一直深深印刻在方源的心中。哪怕他因为纯梦求真体而失忆,这一段记忆去没有泯灭。

%10
“恐怕魔尊幽魂来到落魄谷,这些迷魂雾已经对他不起作用了。”方源思维发散,又旋即暗自一叹,“我何时才能拥有他这般的修为?”

%11
若是方源的魂魄有魔尊幽魂的魂道底蕴,那么就算仙僵肉身上有再强的陷阱,也不惧怕了。

%12
迷魂雾飘来又散去,旋即,又有落魄风吹来。

%13
方源魂魄沐浴在风中,一丝丝的风,像是钢丝,将他的魂魄分离切割。

%14
比凌迟还要剧烈的痛楚,让方源魂魄阵阵颤抖。

%15
但方源咬牙坚持,撑过这阵风。

%16
这阵风还不是很大。持续时间也不长,方源能够撑过。

%17
但若是碰到大风,方源也会明智地撤退,将魂魄遁入肉身之中。

%18
有肉身的防护,落魄风的威能,就要剧减很多倍。

%19
这阵风之后。方源魂魄缩减了三成有余。

%20
谨慎起见,他选择回归肉身。

%21
胆识蛊!

%22
一颗颗的胆识蛊,在方源的仙窍中破碎,化为一股股玄妙神奇的力量,作用在他的魂魄之上。

%23
很快,他的魂魄就再次强壮起来。

%24
刚刚修行所造成的虚弱和创伤,简直是不翼而飞!

%25
荡魂山、落魄谷,可是魂修圣地。

%26
有这两大利器在手,方源的魂道修为是突飞猛进。进步十分神速。

%27
就这样,一阵阵的迷魂雾、落魄风之后,方源魂魄像是铁石,经历一次次的煅烧拷打,变得精粹,更加凝实。

%28
只是不知道什么时候,才能有把握进入仙僵肉身中去。

%29
大约一个时辰之后,方源魂魄开始感到一阵阵烦躁不安。

%30
方源明智地暂停了此次修行。魂魄归于肉身,重新走出落魄谷。

%31
一个时辰的时间。是方源多次修行后,摸索出来的自身极限。

%32
魂道修行,有三大方面。

%33
世人皆知的壮魂、炼魂,只是其中两个方面而已。

%34
除此之外,还有第三方面,那就是安魂。

%35
壮魂。能叫魂魄增强壮大。

%36
炼魂,能锻炼魂魄,打磨精粹。

%37
安魂,则是安抚魂魄,沉淀成果。

%38
三者相辅相成。一味强调某个方面,只能顾此失彼,得不偿失。

%39
其中,壮魂首选荡魂山胆识蛊。炼魂第一便是落魄谷中的迷魂雾、落魄风。而最有效安魂的,是迷魂湖中的安魂汤。

%40
迷魂湖就在生死门中。传说中,太日阳莽死后,就一直醉卧在此湖的河岸边上。

%41
“生死门在影宗福地里面。而影宗福地,则在南疆义天山,如今被超级梦境重重包裹住。要不然,我就有可能得到这生死门,好好尝尝三者齐聚,修行魂魄的便利!”

%42
方源也只是想想。

%43
那片超级庞大的梦境,就是天堑般的阻碍。目前看来,他根本没有能力去逾越。

%44
而且,最重要的一点,是天意。

%45
方源的一举一动,都受到天意的关注。

%46
今后就算有所行动,天意也是方源首要防备的对象。

%47
一路飞行,方源没有直接回去他自己的云城。

%48
而是赶往第十二云城。

%49
很快,他就见到了毛民蛊仙毛十二。

%50
“方源长老,你来了。这是你要求我炼制的我意蛊,你来验收一下。”毛十二主动热情地招呼道。

%51
方源一摆手,将所有的我意蛊都收起来,并说道:“还要验收什么。十二长老,我信得过你。”

%52
毛十二大笑,眼中满是感动。

%53
临走前,他握着方源的双手,感激不尽地道:“还是要谢谢你,给了我这个机会,让我炼制我意蛊。否则的话,接下来若是宝黄天开启,我可没有什么资本去买什么荒兽了。”

%54
“我们俩投缘,也是共赢,希望我们合作能永远这么愉快。”方源笑着回道。

%55
“那是必然的!”毛十二的回答相当干脆。

%56
双方交接完毕,方源赶回自家云城。

%57
“我意蛊。”静室中,方源手中捏着一只五转凡蛊,低声沉吟。

%58
这只蛊虫,像是一只蝎子,十分袖珍。它造型相当特殊,像是白纸折叠的一样。托在方源手中,也是轻飘飘的。当它活动足肢的时候,发出窸窸窣窣的轻响,在方源的手掌中央爬上一圈,速度缓慢。

%59
我意蛊的蛊方来自于影宗。

%60
方源如今是完整的天外之魔,不受到天意的影响。那其他蛊师蛊仙,不是天外之魔的正常人,又该如何对抗天意,防备它悄无声息地影响自己呢?

%61
影宗方面给出的一个答案,就是我意蛊。

%62
这是智道蛊虫,可以产出一股特殊的意志我意。利用我意时刻冲刷自身,就能有效地防备天意的影响。

%63
这是影宗对抗天意的研究精髓,是长期摸索出来的结晶。

%64
方源在和毛六的交易中,得到我意蛊的蛊方后,就开始自己炼制这种蛊虫。

%65
但收效不高。

%66
他不是炼道高手,凭他一人之力,很难有多大的成果。尤其是他的精力和时间,都非常宝贵,不能太过耗费在炼蛊方面。

%67
之前炼制变形仙蛊,那是因为意义太过重要。

%68
而我意蛊,只是凡蛊,方源的需求量还十分庞大。

%69
因此,在犹豫了片刻之后,方源就觉得将我意蛊的蛊方流露出去,交给琅琊派的毛民蛊仙,让他们出手,帮助方源炼蛊。

%70
请动琅琊地灵的代价太高,也没有必要。

%71
琅琊派中的毛民蛊仙,除了毛六之外,都很乐意为方源服务。

%72
因为方源已经和他们熟稔,甚至交好,关键是琅琊建派,一切都向门派贡献看齐。方源完成太丘之行后,手中捏着大量的门派贡献,正是其他毛民蛊仙所需要的。

%73
毛十二当然唯一一个,方源还拜托了其他毛民蛊仙。

%74
他将炼蛊费用压得很低。

%75
做到这点,并没有让方源费什么事。

%76
毛民蛊仙们比较单纯,更关键的是,这些毛民蛊仙都十分擅长炼蛊,对战斗还是有心理抵触的,所以都争相要接取方源的这个任务。

%77
而方源到手的这批我意蛊,更不是第一批了。

%78
将手中的我意蛊收入仙窍,方源又检查其他蛊虫。

%79
他在毛十二那边,表现得很信任,很豪爽,但谨慎如他,都是每次收货之后,回来细细检查。

%80
毕竟琅琊派中,可是潜藏着毛六这样的内奸。

%81
方源并不知道,毛六已经是仅剩的内奸了。他小心翼翼,防备毛六,更防备其他可能的内奸。

%82
检查耗费了不少功夫,但和炼制出这些凡蛊的时间,是万万不能相比的。

%83
检查无误之后,方源不禁点点头,眼中流露出赞叹的意味。

%84
“到底是毛民蛊仙炼制出来的,这品质真是叫人满意!”

%85
方源将这些我意蛊,送入仙窍。

%86
准确的说,是送到小北原中。然后一一催用,将这些我意凡蛊尽数消耗。

%87
大量的我意,冲刷刚刚的几个战场,将剩余的天意都冲刷个干干净净。除了雪怪体内的天意,战场上再无一丝残留。

%88
“这些我意蛊虽然是对付天意的利器,但缺陷很多。首先转数太高,每一只都是五转,而且都是一次性的消耗蛊虫,用一次就没有了。”方源暗自可惜。

%89
他心中对此充满了怀疑:“这个我意蛊方,应当只是残次品。影宗虽然交易给我,但他们手中恐怕有更好更加优良的蛊方!只可惜那场交易,我要得到的东西太多,已经做到了极致,最终没有敲诈出更好的蛊方。”

%90
就这样,方源着手铲除仙窍中的雪怪,事后用我意冲刷天意。

%91
同时,他还利用荡魂山、落魄谷,修行魂道。魂道底蕴积累神速。

%92
除此之外,他还挤压出一切闲暇时光,指点其他毛民蛊仙作战。暗地里,自己熟悉仙蛊,勤练杀招。

%93
宝黄天的关闭,同样带给方源巨大的影响。

%94
仙窍的经营建设,方源空有资本和想法,也只能暂缓一步。

%95
日子一天天过去。

%96
方源期待宝黄天开启的同时,第二次地灾也渐渐逼近。

%97
琅琊派对太丘的攻略,从一开始就没有过停止。

%98
不时地,就有毛民蛊仙接过门派任务,通过传送门,前往太丘。

%99
这些毛民蛊仙传送过去的第一要务,就是改造环境,稳定传送蛊阵。

%100
叫方源感到微微奇怪的是,这些毛民蛊仙没有受到任何挫折,一切都很顺利,天意似乎没有对他们出手。

\end{this_body}

