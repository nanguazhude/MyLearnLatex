\newsection{年华池建成}    %第五百九十四节:年华池建成

\begin{this_body}

%1
一抹晨曦,明亮清新,又不断旋转,如梦似幻,悬停在方源宙道分身的手掌中央。

%2
在方源周围,影无邪等人已经准备就绪。

%3
“开始罢。”方源招呼一声,身上蛊虫气息蓦地升腾起来,一瞬间仙元剧烈消耗,手中的八转仙材漩涡晨曦猛地一爆,覆盖周围百里方圆。

%4
几乎与此同时,影无邪等人出手,将这漩涡晨曦定住,不任由它扩散下去。

%5
漩涡晨曦的延展性极佳,被扩展了百里,虽然光彩不再,仿佛透明的水波,但方源宙道分身占据的中心地带,仍旧能感觉到漩涡的微风。

%6
方源视察一番,见影宗群仙已然将局面稳定,便又继续掏出第二团的漩涡晨曦,如法炮制。

%7
一团团的漩涡晨曦爆炸开来,将方圆百里渲染成一片梦幻的光影,明灭不定。

%8
这个时候,方源取出七宝水。

%9
这种水灰白色泽,毫不起眼,水中自然凝聚着各种疙瘩,宛若金银珠宝翡翠玛瑙。但若见这些疙瘩块儿捞取上来,只要一脱离七宝水的水面,这些疙瘩就会旋即化为七宝水,洒落下去。

%10
宙道分身双手高举,将一团七宝水举至头顶。

%11
接下来这一步,分身只有辅助作用,还得靠方源的本体出手。皆因方源宙道分身的修为,只有六转,不足以承担这份责任。

%12
方源本体的神念,早已经笼罩这里,任何一丝不妥之处,都能在瞬间被他洞察。

%13
此刻,方源催动手段,仙元迅速消耗。

%14
一蓬白光落到七宝水中,咕咚咕咚,旋即七宝水就沸腾起来,水面上冒出无数泡泡。

%15
七宝水中的疙瘩块儿,好似雪遇烈日,纷纷以肉眼可见的速度,迅速消解。

%16
七宝水体积不断膨胀,很快,水中的疙瘩块儿都尽数消弭。

%17
纯化了七宝水之后,又有一道奇光,电射而出,瞬间射中七宝水。

%18
七宝水旋即化为一团冰块,寒气四溢,铺散开来,将方源的宙道分身迅速吞没,又卷席四周,将方圆百里精准罩住。

%19
影无邪等人见此,连忙将手中的八部玉,投入伸手不见五指的浓郁寒雾当中。

%20
一切顺利地进行下去……

%21
至尊仙窍,数天之后,响彻此方天地的轰鸣声中,一座方圆百里的巨大水池彻底成形。

%22
“这就是八转层次的年华池了!”

%23
因为经过充分的准备,还有运势的情势,方源第一次铺设这座年华池,就收获了成功。

%24
方源细细检查了三遍,确定毫漏之后,便开始动用宙道手段,接引光阴支流。

%25
每一个仙窍世界,只要时间还在流逝,那么就必然有一条光阴支流。

%26
光阴支流平素隐匿不现,唯有动用特殊的宙道手段,才能加以针对和牵引。

%27
呼啦啦……

%28
至尊仙窍中的光阴支流,远比寻常仙窍,要更加宽阔和湍急。

%29
支流顺利地灌注到年华池中,并且又从年华池的另一端,流淌而出。

%30
不过,流出来的光阴支流,却是微弱如溪,潺潺流淌,再无之前的奔腾气势。

%31
年华池中,水面则在徐徐上升。

%32
在接下来的一段时间里,整个至尊仙窍的光阴流速,会缓慢下来。等到年华池中水位蓄满,流淌出来的光阴支流重复旧状,至尊仙窍的光阴流速也会跟着复原。

%33
年华池建设成功,把方源的宙道分身,以及影宗诸仙都累得够呛。他们连续数天不眠不休,精力又要全神贯注,仙元消耗更加不少。

%34
方源本体神念仍旧在关注,其余人都离开,各自休憩。

%35
方源的宙道手段逐渐失效,年华池和光阴支流变得越来越淡,片刻后,它们彻底消失,深深隐匿在至尊仙窍之中。

%36
年华池比较特殊,它和光阴支流联通一体,并不显形。

%37
它的功效,也不只是一个资源点,也有水坝的作用。

%38
比如将年华池泄水,或者积水,都能导致至尊仙窍中的时间流速发生改变。

%39
方源虽然有仙道杀招度日如年等等,但是这些招数一来消耗仙元,二来每次动用一番之后,会有一定时间的间隔,不能再对光阴支流施加第二次影响。

%40
但有了年华池,这就容易多了。

%41
因此不少蛊仙,就算不是宙道,也在自家仙窍中搭建这种池子,来改变自家的仙窍时间。

%42
当然,他们搭建的大多是六转、七转,不像方源这样直接搭建出八转层次的年华池。

%43
八转层次的年华池,造价太高,投入太大,而且需要阵道手段。

%44
本质上它是一座仙道蛊阵。

%45
幸亏方源的阵道境界达到了宗师级,如此才能利用仙材中的道痕,布置出蛊仙大阵来。

%46
年兽们开始欢呼。

%47
年华池中的水位还很低,就有少量的荒级年兽感知到,主动钻了进去。

%48
它们出生在光阴长河中,成长在河中,也死在河中,最适应的环境也是光阴长河。而年华池就是模拟出光阴长河中的环境,无怪受到这些年兽的欢迎。

%49
随着光阴支流的不断灌注,年华池还在缓缓地扩张。

%50
不管是漩涡晨曦、七宝水、八部玉,都有良好的延展性,并且又有凝聚成一体,不至于自行崩散的特征。

%51
等到年华池蓄满了水,这座八转程度的年华池,将达到方圆万里的恐怖程度。

%52
更妙的是,它隐匿深处,仿佛虚幻,并不占据至尊仙窍的任何一片空间。

%53
年华池等若是一个独立的小世界,可以任由方源布置出一套食物链,和至尊仙窍毫无关联。

%54
最令方源感到满意的一点,则是年华池本身可以拘束年兽,令它们只能在池中活动,跑不到外面去。

%55
有了这一点,方源可以收容更多的年兽,而这些年兽完全可以是野生状态。比如说野生的太古年兽。

%56
方源现在只能统摄七头太古年兽,这些太古年兽若是战损牺牲,方源别无兵源补充。

%57
但若是年华池中有着其他野生太古年兽,方源在牺牲了一两头太古年兽猴,便可以临时奴役,及时补充兵力。

%58
这点对于他前往光阴长河的大计,极为有利!

%59
经过这段时间的休养,方源本体的伤势已然痊愈。

%60
“接下来,就是要将这些蛊仙的肉身和魂魄分离,然后将魂魄摄取出来,再一一搜魂。”

%61
与此同时,方源对南疆正道的勒索也没有停止。

%62
不过这一次,却出现了强大的阻力。

%63
南疆正道各个势力,都开始联合起来,对抗方源。

%64
年华池造价高昂,方源前番勒索,已经让各大势力肉疼。方源口口声声,要求南疆正道显现诚意,他本人却毫无诚意显露,这让南疆正道私底下琢磨之后,决定联合起来对方源施压,统一阵线,逼他就范。

%65
数天后。

%66
“方源如果你不展露你的诚意,我们南疆正道是绝不会让你再得到任何一份资源的!”商家的蛊仙被推举出来,代表整个南疆正道,和方源谈判,语气强硬。

%67
但他很快话锋一转:“之前,我们南疆各族已经展示了我们的诚意,现在只有方源你也展露诚意,我们才能谈得下去。否则就算是你将这些蛊仙杀了,我们也绝不会妥协!”

%68
方源不禁冷笑:“你们要看到我何种诚意?是要我先释放一部分俘虏吗?”

%69
商家蛊仙很快回应道:“我们也不奢求你将人质释放,我们可以一步步来,比方说仙蛊。很多仙蛊你都用不上,完全可以将这些仙蛊先还给我们,而作为代价,我们付给你所需要的修行资源。”

%70
南疆正道并不蠢笨。若是先让方源释放一部分人质,那么得到好处的正道家族,很有可能在接下来的交涉中,出工不出力。所以,他们宁愿方源先分批次,将仙蛊、仙窍中的资源、蛊仙肉身、蛊仙魂魄,这样一批批地赎回来。

%71
如此一来,南疆正道始终可以团结在一起,共同进退,一起承担。

\end{this_body}

