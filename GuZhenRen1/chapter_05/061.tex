\newsection{贵精不贵多?}    %第六十一节:贵精不贵多?

\begin{this_body}

方源一番话,有理有据,琅琊地灵听得连连点头。<strong>小说txt下载HtTp://Www.80txt.Com/</strong>

“有道理,有道理!”他交口称赞。

“不敢。这只是我个人的一丝浅见。”方源谦虚。

地灵单纯,不懂得方源的谦虚,摇头反驳:“不,这绝不是浅见,而是真知灼见。你这番话,对我启很大,对琅琊派的贡献也很大。”

“本来按照琅琊派的规矩,单凭这个功劳,你的地位就能提升一格。例如毛六就能升为毛五。但你现在还不是真正的毛民蛊仙,所以你只能获取两百点门派贡献了。”

说到最后,琅琊地灵望着方源,遗憾地叹息一声。

方源的脸上,也立即露出十分可惜的神色。

他现在只是客卿,不需要做任何的任务,每个月都有十点门派贡献。但是其他毛民蛊仙,根正苗红,这方面的每月所得都要多得多。而且毛民蛊仙的排位越高,每个月领取的门派贡献就更多。

“真想成为真正的成为毛民啊!”方源感慨道,满脸的情真意切,似乎是真的这么想。

琅琊地灵道:“可惜宝黄天现在关闭了,一旦它重新开放,我就对市面上的荒兽进行大规模的收购。”

“对于这个,在下也有一点愚见。”方源连忙道,“毛民蛊仙众多,要培养他们,我派再是家大业大,也经不住这么消耗。这更不是正统的经营之法。我建议自己开源。”

“哦?怎么自己开源?”琅琊地灵问。

方源答道:“很简单,就像其他级势力那样,占据五域外界的资源。”

琅琊地灵面露难色。

这个想法,他早就有过,也谨慎思考了其中利弊,认为并不可取。

原因很显然,毛民是异人,在当今的大环境下,见不得光。

虽然在北原,也有异人势力。最明显的一个例子就是墨人城。可是这个墨人城,明面上只有一位墨人蛊仙。而且在历史上,墨人城的先祖一言仙,也是付出惨重代价。才和巨阳仙尊达成了协议。两重因素,才使得无数强大人族势力的夹缝中,有一座墨人城艰难生存下来。一

如果毛民出现这样的势力,又有这么多的毛民蛊仙出现,这必将严重地挑战当今人族蛊仙们的神经。极大地触犯他们的底线。

不管是正道、魔道还是散仙势力,都不会容许外族有这么一个强大的势力存在。

琅琊地灵虽然单纯,但是并不愚蠢。他虽然一心想要让毛民称霸天下,但目前的大局他还是看的很清晰的。

但方源早已料到琅琊地灵的这番反应,他继续道:“太上大长老,我并不是建议琅琊派和其他人族势力,公开争夺外界资源。(WWW.mianhuatang.CC 好看的小说棉花糖而是我们专门占据他们没有占据的资源,仍旧暗中展。”

琅琊地灵恍然:“你是说那些凶险禁地?”

北原广袤,但各大级势力已经基本瓜分了那些修行资源。剩下来的,主要就是十大凶地。

“太上大长老英明!”方源立即竖起大拇指。满脸钦佩之色,“十大凶地虽然危险,但同时蕴含着丰富的机遇。依我看,地沟就很好。我们可以在里面铺设传送蛊阵,让毛民蛊仙们能够在福地和地沟之间,自由往返。地沟中的荒兽、上古荒兽不计其数,正适合毛民蛊仙们对战训练。而且打杀了这些荒兽,我们可以就地取材,炼制变化道的仙蛊。捕捉这些荒兽,我们还可以用驭兽仙蛊奴役。使它们为我们所用,加强福地的防御力量。”

这事情的关键之一,就是传送蛊阵。

具体而言,就是能够传送蛊仙的蛊阵。

琅琊派无疑是拥有这样的手段的。

因为方源之前从南疆一路赶来。就是借助了在风伯崖下的传送蛊阵,才摆脱了那些麻烦的云兽,到琅琊福地中的。

那个传送蛊阵,给方源留下了深刻的印象。

琅琊地灵陷入沉思:“这个事情很重大,让我再好好考虑考虑。”

“我只是这么一说而已,派中大事。当然还是由太上大长老你来决断的。如果我建议有错,还望太上大长老你多宽恕。”

方源深知过犹不及之理,接下来便转移话题,牵扯到此次要换的流光果上。

一会儿之后,他从琅琊地灵手中,带走了大量的流光果。

和地灵告别之后,他就到自己的云城中,第一件事就是处置这些流光果。

流光果,不是寻常果实。

它是纯粹由一团团的光组成的,若是寻常人用手来拿捏,只会让手透光而过。非得用专门的手段,来收取这些果实。

这些果实,也不是像通常的果实那样长在草木之上,而是纯粹由浓郁的极光凝结而成。

极光色彩缤纷,五颜六色,所以流光果同样如此,赤橙黄绿青蓝紫等等各种颜色都有。

流光果,就是态度蛊的饲料。

方源之所以谋取流光果,就是因为如此。

自从交易之后,方源手中的仙蛊数量就暴涨上去,达到惊人的地步。

统计一下。

九转仙蛊智慧蛊。

八转仙蛊态度蛊、慧剑蛊。

七转仙蛊换魂蛊、剑眉蛊、浪剑蛊、剑遁蛊、招灾蛊。

六转仙蛊解谜蛊、妇人心蛊、血本蛊、暗渡蛊、狗屎运蛊、变形蛊、力气蛊、我力蛊、飞熊之力蛊、拔山蛊、挽澜蛊、江山如故蛊、人如故蛊、星眸蛊。

其中妇人心蛊,因为义天山大战前,方源为了炼化薄青的剑道蛊虫,留在了狐仙福地里。他利用智慧光晕,搭建了一套蛊阵,解谜蛊、妇人心蛊就是其中的组成部分。但是解谜蛊只在搭建的初期运用一番后,就功成身退,被方源带在身上。妇人心蛊却需要一直保留在蛊阵中,直到薄青的剑道仙蛊被炼化成功,才能撤销。

正因为如此,所以义天山大战之后,方源借助琅琊派之力,将两个福地的资源都统统搬迁走。妇人心也就跟随着一同来到琅琊福地,最终仍旧到了方源手中。

而原本属于方源的连运蛊、铁冠鹰力蛊,因为被影无邪拿去卖到宝黄天里,所以交易的时候。就无法得手。交易完成之后,方源更不可能拿来了。

星道的蛊虫,只有星眸到方源的手中,其余仙蛊诸如星芽、星痕、星光、星念,都烂在影无邪的手中。不知道还在不在了。

在义天山大战中,惊鸿乱斗台解体,很多蛊虫都毁灭了,但还剩下一些蛊虫,保留在仙僵的仙窍之中。这些蛊虫方源都不太熟悉,样子都记得不太全了(失忆的原因),也不知道还有多少被影无邪扣下。

方源也不是没想过,将这些蛊虫都夺来。但是还未谈到这一步,影无邪那边就已经终止了交易。

一般而言,仙蛊都是贵精不贵多。

凡蛊就算了。再多的凡蛊,蛊仙基本上都能养的起来。

仙蛊是衡量一个蛊仙实力的重要因素。

很少有蛊仙,供养仙蛊的数目,有方源现在这么多的。

就算是当今的那些八转蛊仙,都没有方源多。

说这话毫不夸张。

大多数的蛊仙,基本上供养两到三只仙蛊,就可以了。当然这些仙蛊,要符合他们的转数。

因为一只仙蛊,就能和大量凡蛊搭配,运用不同的仙道杀招。就能形成不同的效果,涵盖攻防、挪移、疗伤等等方面。

但区区一只仙蛊,是不够的。

因为当蛊仙只有一只仙蛊时,他只能催动一个仙道杀招。防御杀招使出来。攻击手段就没有了,移动手段也丧失了,只能被动挨打。攻势有了,防御没有,蛊仙太很危险。

至少要有两只仙蛊,可以同时催动。兼顾周全。

出三只仙蛊以上,仙蛊就形成了负担,通常而言,是养不起,又用不起。

毕竟催动仙蛊,也是要耗费仙元的。

方源这样越级使用七转仙蛊的例子,还是很少的。他的经历只能算是特例,和绝大多数蛊仙不同。

蛊仙选择仙蛊,要精挑细选。

这里的“精”,不仅是指数量上,而且还对仙蛊的范围有限制。

一个炎道蛊仙,当然选择炎道仙蛊最好,选用水道、冰道,那就纯粹是自找麻烦。就算有一些其他流派的仙蛊,顶多也是作为仙道杀招的辅助蛊虫。

仙蛊贵精不贵多,这是修仙界的常识,只要是蛊仙,几乎人人皆知。

但对方源而言,这点却不适用。

因为他是九五至尊仙窍。

仙窍地方够大,不怕空间不够。时间够快,资源再生极其迅猛。

这些都是潜质。

方源已经看清,自己身负着豢养大量仙蛊的潜质!

既然是潜质,自然要尽量挖掘出来。

第二个原因,方源日进斗金,目前的经济状况十分良好。

他的至尊仙窍,每年要产生九十六颗仙元。这里的一年时间,是指仙窍时间。放到外界五域时间,就是每天自产十六颗仙元。

这些仙元已经极多,十绝体都比不上。但和方源每月的盈利比起来,差距就更大了,简直是杯水车薪。

这就意味着,方源有资本可以去供养仙蛊。

再加上琅琊派作为外力,在某种程度上可以依靠。

这三个原因加起来,才使得方源决心大规模供养仙蛊,提升自己的战斗力量。

没有办法,不管是面对天意,还是灾劫,方源都必须保证自己的力量。

蛊仙的战斗力,大部分来自仙蛊,还有仙道杀招。

现在方源的仙道杀招少,智慧蛊又用不了,推算需要太多时间,方源只好在仙蛊这个方面下功夫了。

方源当然也清楚影无邪的谋算,就是将这些仙蛊抛给自己,让方源负担极重,牵扯方源的精力,让方源无暇顾及影宗的行动。

但方源必须要有大量的仙蛊,保证自己的战斗力。

他也有潜质和资本,可以供养这些仙蛊。

这和他仙僵时期,完全不同。

九五至尊仙窍,是当中最关键的因素!(未完待续。)<!--80txt.com-ouoou-->

\end{this_body}

