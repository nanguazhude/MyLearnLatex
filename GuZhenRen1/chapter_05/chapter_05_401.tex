\newsection{万蛟!}    %第四百零一节:万蛟!

\begin{this_body}

%1
凤九歌背靠天庭,自然情报不缺,对着影宗群仙有着相当多的了解。

%2
除去方源之外,场中还有白凝冰、黑楼兰、影无邪、黑菟、妙音仙子这五人。

%3
白凝冰龙女美貌,惊为天人,攻势亦是最为凶猛,极可能是白相真传的继承者。

%4
黑楼兰剑眉星眸,虽是女身,却极有英气,攻伐间火焰缭绕,威势惊人,本身亦是十绝体的大力真武体。

%5
影无邪如今顶着翠波仙子的肉身,绿裙媚眼,手段隐而不发,这对于凤九歌而言,却是新面孔,让他存疑。

%6
黑菟目光狠辣,此刻伤势已经大好,她行动迅猛,专修暗道,宛若刺客,在外围流转,双目紧紧盯住凤九歌伺机报复。

%7
而妙音仙子,已经施展出了招牌杀招妙手玄音,身怀六臂,美妙多姿,眼眸中浮现一对勾月,和凤九歌一样,都是修行音道。

%8
凤九歌一边运用一曲阳关,躲避方源的扑击,另一方面,也在打量这些影宗群仙。

%9
方源的逆流护身印,太过无解,但有利有弊,方源要维持这个杀招的话,就不能够催动其他的仙道杀招了。

%10
方源攻弱防强,凤九歌要击败影宗,唯有找其他人下手。

%11
这是避实就虚的上策!

%12
但是凤九歌究竟会找谁动手?

%13
这个疑惑,萦绕在方源等人的心头。

%14
凤九歌不会愚蠢到和方源死磕,同样的道理,方源等人也早已料想到了凤九歌会选择何种战术。

%15
这是很显然的事情。

%16
逆流护身印太过强悍,短时间内天庭也无法拿出破解的手段。

%17
破解杀招是很难的。

%18
但克制、缓解的手段,则较为轻易达到。

%19
然而逆流河堪称九转仙蛊一级,乃是《人祖传》中记载的天地秘境,这等存在怎可能被轻易克制或者削减威能?

%20
所以,只有当天庭琢磨出彻底破解之法,才能对逆流护身印有所办法。在此之前,就算是龙公亲至,也是无可奈何。

%21
“所有人中,就属黑楼兰修为最低,是六转。凤九歌会对付她吗?”

%22
“方源是不可能,对付妙音仙子很有可能。毕竟双方同修音道,依照凤九歌的雄厚造诣,应对妙音仙子的手段,自然要更加得心应手。”

%23
“他之前袭击了白兔姑娘,现在若是趁胜追击,扑向黑菟也有可能。”

%24
“白凝冰呢?白凝冰乃是我们当中攻势最强之人,是最强的矛,若是摘掉她,这场战斗凤九歌的优势就会变得巨大。”

%25
“影无邪也有可能。她顶着翠波仙子的肉身,对于凤九歌而言,最不熟悉,至少会有试探性的攻势。”

%26
各种分析和猜测,也在方源等人的脑海当中。

%27
若是能猜中凤九歌的想法,无意能在接下来的战斗中,更加应对自如,甚至能占据先机。

%28
凤九歌究竟会选择何人下手?

%29
下一刻,答案揭晓。

%30
凤九歌猛地一跃,一曲阳关发动,让他腾挪到了战场中央。

%31
随后,这记仙道杀招一曲阳关并未散去,而是气息陡然一变。

%32
凤九歌长啸一声:“来听听我这首碧玉歌如何?”

%33
这是他的招牌手段,群仙闻言,无不后撤。唯有方源一人,哈哈大笑,勇悍无比地扑了上去:“凤九歌,你总算是拿出点手段来了。”

%34
叮叮咚咚……

%35
歌声飘扬,如大珠小珠落玉盘,清脆悦耳至极。

%36
方源不管不顾,接近凤九歌,想要用逆流护身印,返还一部分碧玉歌的威能,让凤九歌好好地尝一尝。

%37
但没想到的是,凤九歌催发的碧玉歌杀招,居然将威能都尽数收敛起来。

%38
方源一路畅通无阻,根本就没有逆反任何杀伤回去。

%39
群仙诧异之间,一个模糊的人影,从凤九歌的身体中悠悠飞出。

%40
这身影酷似凤九歌,但通体碧绿,好似玉石所制。

%41
“这是什么杀招?!”影宗群仙无不从这神秘的人影上,感受到了强烈的威胁感。

%42
“一曲之士!”方源心中一沉,想到了前世的某个情报。

%43
这是凤九歌的一个厉害手段,能够让他所创造出的九歌,都凝聚起来,化为人形,扑击敌人。

%44
方源五百年的前世,凤九歌原本的九大歌曲,一个个威力绝伦,但是亦有弊端,就是一次只能催动一招,不够灵活,有点持续时间较短。

%45
凤九歌多番战斗下来,明白了自己的这项弊端,便闭关了一小段时间,出关后便掌握了一曲之士这个杀招。

%46
如此一来,他就能将九歌,都化为临时分身,对战强敌,极其灵活多变,战力因此又上涨,五域乱战中,多少次让敌人苦不堪言。

%47
“没想到这大名鼎鼎的一曲之士,就是一曲阳关的变招。如同我的力道大手印,和万我之间的关系。”

%48
“前世大约五百年后,凤九歌才开创一曲之士,这一次他居然提前这么久?”

%49
“不对,五百年前世是因为影宗逆天成功,硬生生拖延了大时代,或许凤九歌也遭受了影响?”

%50
“也或者,凤九歌提前自创一曲之士,乃是天庭方面的提点?”

%51
方源脑海中,无数念头风起云涌,一瞬间想过无数的猜测。

%52
碧玉歌士没有迎上方源,而是扑向了影无邪。

%53
影无邪眼皮子直跳,连忙后撤,不敢撄其锋芒。

%54
凤九歌再次动用一曲阳关,腾挪远处,三番五次之后,再待方源扑来时,他已经催动出有一记招牌杀招。

%55
天地歌!

%56
天地歌士杀向白凝冰。

%57
白凝冰大叫一声:“来得好!”

%58
她和天地歌士对上,硬拼起来,打得冰霜暴溅。

%59
“糟糕,这样下去可不妙!”

%60
“白凝冰、影无邪已经自顾不暇,我们对凤九歌根本无法施压。”

%61
“原来凤九歌的目标,不是我们其中一个,而是我们中除开方源的所有人!”

%62
明白了凤九歌的战术,影宗群仙感受到了强烈的威胁。

%63
战局正向凤九歌一方倾斜,但方源等人却是无可奈何。

%64
总体而言,方源这一方,还是攻弱守强的形势。而凤九歌智谋出众,采取了最明智的战术。

%65
其实,一曲之士这个杀招,有利有弊。若是催动九歌杀招,歌声无形无质,覆盖范围极广,威能浩大,难以防御。

%66
变化成一曲之士后,却是凝聚一团,可以针对、避让,乃至用遥击之法来不断削减一曲之士的威能。

%67
但凤九歌此时对战,情形和之前两场完全不一样。

%68
一来,他不需要面对仙蛊屋玉清滴风小竹楼。这座仙蛊屋速度奇快,若是用一曲之士,肯定被避让过去。

%69
二来,凤九歌不需要强攻光阴支流。

%70
这一次方源追逃之间,准备明显不充分,是没有时间来布置仙道蛊阵的。

%71
动用一曲之士,就算无法各个击破,也能让影宗群仙自顾不暇,形成不了合围之势。

%72
对于什么样的战局,采用什么样的战术,这才是身经百战的蛊仙。

%73
凤九歌毫无意识,是身经百战之人,实战经验非常的丰富。

%74
再加上他骇然的天赋才情,这一场战斗,带给影宗群仙十足的苦头。

%75
碧玉歌、天地歌之后,又是俯首歌、离歌。

%76
四位一曲之士,时而捉对厮杀,时而联手齐攻,让场面陷入混战当中。

%77
“混蛋!”黑楼兰咬紧牙关,不断飞退,她运用愤怒的小鸟,进行遥击,但是此招一旦击中碧玉歌士,火焰小鸟立即化为玉石小鸟,随后坠地摔毁。

%78
“彻冰刀。”白凝冰口中低呼一声,抡起手中巨大的冰刀,看向俯首歌士,但关键时刻,离歌曲士迎接上来,硬抗这一杀招。

%79
巨大的冰刀旋即瓦解,离歌曲士能分离一切,就算是仙道杀招也是如此。

%80
而天地歌士则具备镇压威能,影无邪、妙音仙子围攻,暂时不分上下。

%81
方源心中一片凝重。

%82
“凤九歌自修行之初,就立志创造九首歌曲,唱尽自身对命运、生活、天地的感悟。时至如今,他已经创造出了离歌,按照前世的情报,也就是说,具备了七首歌曲了。”

%83
其中四首,便是碧玉、天地、俯首、离歌,用于攻伐。方源等人已经见识到了。

%84
还有两首,得宝、向天,辅助修行。这点不足为虑。

%85
最让方源担忧的,是最后一首歌。

%86
“那首歌不易操纵,一旦失败,反噬几乎致命。但若是让凤九歌有机会,成功地使出那首歌的话,我们必定伤亡惨重……”

%87
方源心头焦虑,但是没有办法,他攻弱守强,对凤九歌施压完全是一个空想。

%88
凤九歌战力极强,操纵四大分身娴熟至极,单这一手,就可看出他的奴道造诣,比方源还要深厚!

%89
毕竟是成就蛊仙多年,积累浑厚,方源加上前世五百年,也比之不上。

%90
“这还是在绝音戈壁的战场中,凤九歌本身受到压制。当然,我方的妙音仙子也是如此。”

%91
“必须冒险了。否则的话,错失这次,接下来恐怕连冒险的机会都没有。”

%92
方源心中闪过一抹决意。

%93
下一刻,他竟然主动撤销了逆流护身印!

%94
仙道杀招——上古剑蛟变。

%95
方源刹那间化身上古剑蛟,然后,他稳住心神,深吸一口气,悍然催动另外的一记招牌手段。

%96
奴力合流——万我!

%97
这不是简简单单的“在上古剑蛟变的基础上施展万我”,而是涉及到仙道杀招的著名技巧之一——并招。

%98
上古剑蛟变,并,万我,从而形成——万蛟!

%99
吼吼吼吼……

%100
霎时间,万蛟现身,齐声嘶吼,剑气四溢,银鳞如海。

%101
“这杀招……曾经在追杀我的时候,在东海一击就破除了我宗的仙道蛊阵!如今终于再现。”影无邪心头震动。

%102
“方源。”白凝冰口中低咛一声,一种可靠的感觉油然而生。

%103
“了不起。”就连凤九歌也不禁动容。

%104
这一时间,他整个视野中,都是剑蛟飞舞。蛟龙似海,直接淹没四大曲士,同时还包括他凤九歌!

%105
ps:谢谢大家的生日祝福,心里暖暖哒!!

\end{this_body}

