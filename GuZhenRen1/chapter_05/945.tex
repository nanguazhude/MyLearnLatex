\newsection{红莲之死}    %第九百四十九节:红莲之死

\begin{this_body}

“为什么?我怎么会变成这个可怕样子呢?”

曾经的洪亭,对于柳淑仙的死,哀痛至极,悔恨交加。会死死地抱着柳淑仙的尸体,半跪着,心中哭泣哀嚎。

但是现在呢?

面对柳淑仙的死,面对柳淑仙只想在生命的最后一刻,和爱郎最后温存的期待,红莲居然转身就走!

难道这个柳淑仙,不是真正的柳淑仙吗?

当然是!

红莲非常清楚这一点。

她还是她,还是那个柳淑仙,不管红莲重生多少次,她的决定和行为都不会更改。那就是为了守护自己的爱人,宁愿牺牲自己,用自己的生命来抵抗成尊灾劫!

然而……

柳淑仙仍旧是柳淑仙,但洪亭已经不再是洪亭了。

“究竟是什么时候开始的?是什么时候开始的变化?”红莲拷问自己的内心。

这个问题并不困难,他很快就找到了答案。

那是无数次的重生,无数次的尝试,无数次柳淑仙的死亡,无数次他收获失望!

他的心一次次受伤,一次次流血,还未结成血痂,又被洪亭自己主动撕裂。

他受伤的次数太多了,他疼痛的次数太多了,他悔恨的次数太多了。

太多太多,终于让他渐渐习惯,终于让他……渐渐麻木。

于是,他开始精于算计。

于是,他开始冷静分析。

他利用手头上每一份资源,竭尽所能地壮大自己,武装自己。他尝试不同的方向,找寻到最有可能的强大途径,使得自己在渡劫时护住柳淑仙的性命。

当他开始冷静,面对柳淑仙的死开始冷静,哪怕只是最初强迫自己冷静,他就已经变了。

然后,逐渐的,一步步的,他终于变成了一副陌生的模样。

这个模样,让他猛地发觉,惊悚。

然后,无能为力!

这一幕,给凤九歌带来极其深刻的印象——

红莲站在原地,他低头垂手,年轻背影显得佝偻,仿佛行将就木的老人。

他默默无声,两行泪迹从脸颊滑落。

无声哭泣。

在红莲身后,是弥留之际的柳淑仙,仍旧在呼唤着他,期望着他转过身来,让自己在生命的最后一刻,能够再看一眼自己爱郎的样子。

但红莲始终没有转身看她一眼。

红莲已经不再爱她了。

试问,如果一个人是他心中所爱,爱人临死,他会冷漠地转身而去吗?

当然不会!

就算他的目的是重生,是再去尝试拯救柳淑仙,但那只是一种强烈的惯性而已。

他的出发点,已经不再是爱了。

真的很可笑。

明明是想因爱而去拯救,但是走着走着,红莲失去了爱。

柳淑仙不管死了多少次,都没有背叛他,都选择为他牺牲。

但红莲自己变了。

他背叛了曾经的自己,背叛了柳淑仙。

他想要重生来改变事实,但没想到反而是重生这个事实,改变了他。

那么接下来,他该做什么呢?

既然他对柳淑仙不爱了,那他为什么要重生去不断地尝试呢?

当然,他还有遗憾,他对自己的父母一直都爱。

但是红莲他敢去尝试吗?

他几乎都已经知道结果。那就是一次次目睹父母死状后,他习惯,他麻木,然后他会坦然接受这个宿命的结果。

红莲他不敢!

那么,是接受这样的结果?重生之后,装作一切都不知道,按照宿命的轨迹成为尊者,然后成为所有人期待的样子,让师父满意的天庭仙尊?

红莲他也不想!

他的心中还有爱,对父母的爱,对师父的爱。因为有爱,所以有遗憾。

他的心中还有恨,是对宿命的恨,是因为不再爱柳淑仙,还有恨自己。心中的许多情绪,红莲其实自己也搞不清楚。

红莲深深地拷问自己,审视自己的内心。

他仍旧是想毁灭宿命蛊。

只是他的出发点,已然不同。

他开始钻研,越来越发现《人祖传》中隐藏的奥妙。当他体悟到爱情蛊这个关键,他便利用爱情蛊成功地拯救了柳淑仙的性命!

当然,代价是红莲虽然渡劫成功,但却没有成为尊者。

凤九歌身为旁观者,却是心知肚明:这绝对是一个重大的突破!

成尊灾劫是不可人为掌控的。

不管红莲如何拖延,他都必须渡劫。

利用爱情蛊,他终于改变了过去的事实!那就是柳淑仙的死,现在的柳淑仙终于存活了下来。

当然,后遗症是红莲没有成尊。这也是事实。

两个事实都改变了!

“淑仙、淑仙,你还活着!你终于活下来了!”成功了的红莲,兴奋异常,抱着柳淑仙。

柳淑仙的情绪却很低落:“这是怎么回事?洪亭,你竟未成尊?!”

红莲哈哈大笑:“那是因为我利用了爱情蛊啊。你知不知道,我为了救你重生了多少次?你本来是应当在灾劫中死去的。但是通过我的不断尝试,不断钻研《人祖传》开创的人道手段……”

红莲兴奋至极,不断说出自己经历过的苦难,从中花费的心血,每一次的猜想和验证都充斥挫折和磨难。

“但是终于成功了,我终于成功了!”红莲振臂欢呼,“虽然只是小小的改变,但这却是一个正确的方向。总有一天,我会彻底成功的!”

啪。

一声脆响,柳淑仙扇了红莲一个巴掌。

红莲瞬间顿住,欢呼声戛然而止,瞪视柳淑仙。

柳淑仙满眼含泪:“洪亭,你真的是你吗?你居然有如此大逆不道的想法?想要重生改变过去?!你违背了宿命,难怪灾劫之后,你未成尊!而你做这一切,只是为了让我活下来?”

“那些协助你渡劫,因你牺牲的其他天庭蛊仙呢?你说你是为了做这一切,那你有没有考虑过我的感受?!”

“我柳淑仙啊,之所以出生,拥有十绝体,和你相遇,都是宿命的安排。为你抵挡灾劫的那一刻,我忽然明白,我此生最大的意义就是保护你,替你挡灾,助你登临仙尊之位!”

“但是你却这样做!为了我舍弃你最珍贵的成尊之机!”

“你无法成尊,天庭怎么办?天下怎么办?人族怎么办?”

“你忘记了你师父的教诲,你辜负了你死去的父母的期盼!我真的痛心,万分痛心!”

“我宁愿牺牲我自己啊!”

“你我之爱,不过是小爱。而大爱是为整个人族,为苍生天地!”

“洪亭,你太令我失望了。”

柳淑仙痛哭流涕。

红莲呆呆地望着柳淑仙,好一会儿,他轻轻地吐出一口浊气。

“原来是这样么。”红莲的神色一片平静,目光也变得幽深起来。

柳淑仙抓住他的双臂:“快想想办法,你一定有办法的吧?让一切都重归宿命的轨迹,哪怕让我牺牲。我们之间的爱,怎么能和天地大爱,能和天庭的伟业相比较呢?”

“呃!”柳淑仙忽然身躯一顿,她缓缓低头,看向自己的心口。

红莲的手直插入心,这是致命一击!

柳淑仙难以置信地望着自己的爱郎。

“或许杀了你,能够令一切重回正轨。”红莲凝视柳淑仙的双眼,淡淡地道。

柳淑仙顿时流露出幸福的微笑:“洪亭,我没有看错你。请你一定要……成为尊者。”

下一刻,她再一次死了。

和红莲经历的,无数次的死亡并没有什么不同,柳淑仙又死了。

但和红莲经历的无数次的她的死亡,又有截然不同的一点。

柳淑仙真的死了!

她在红莲的心中,彻底的死去!

从此,再没有活过来。

红莲望着她倒下的尸躯,心中一片平静,甚至连一丝波澜都未生出。

他曾经深爱的女人,曾经愿意为她付出一切,一次次重生,不断努力,从未放弃过希望的女人。

现在,他把她杀死。

亲手杀死!

红莲心中没有一丝悔恨,反而有一丝明悟。

造成这一切的悲剧的根源,在于天庭,在于宿命蛊。

这一刻,他毁灭宿命蛊的决心,从未有过的坚定!

凤九歌眼前的画面一转。

天庭。

大战已经结束,周围一片废墟。

残破的监天塔顶,龙公和红莲相对而立。

龙公明显历经了一场大战,浑身浴血,伤痕累累。

他看着红莲,发生一声深深的叹息:“洪亭啊,一步错步步错,没想到你已经错到如此地步。你想要摧毁宿命蛊?那就请动手吧。”

说着,龙公竟主动让开了路!

红莲微微一愕,旋即走到宿命蛊的面前。绝大多数的蛊虫本身都是非常脆弱的,然而不论他如何用力,宿命蛊仿佛世间最坚固的东西,红莲就是捏碎不了。

“不管你用什么办法,你都无法摧毁宿命蛊的。洪亭啊,不仅是你,就算是无极魔尊、狂蛮魔尊,当年也来到这里,动用种种方法都对宿命蛊无可奈何。”

“这个世间,真正能摧毁宿命蛊的唯有完整的天外之魔!然而天外之魔一旦来到我们的世界,却都不是完整的。所以,这个世间并无人可以摧毁宿命蛊。”

龙公说到这里,流露出浓重的疲惫之色:“洪亭,回来吧,浪子回头尤未晚!你虽然造成如此多的罪孽和错误,但我愿意,天庭也愿意给你赎罪,给你重新开始的机会!”

到了这一步,龙公仍旧苦口婆心,不愿意放弃红莲。

红莲沉默良久,转身面向龙公,微微一笑:“师父啊,既然两大魔尊都摧毁不了宿命蛊,那为何你一开始,不把宿命蛊直接放到我面前,让我尝试一番,好让我死心呢?”

龙公哑然。

红莲再笑:“所以说,你也害怕不是吗?你害怕我找到一种新的方法,能够摧毁宿命蛊。”

龙公苦笑:“这是当然。时代在不断地变化,蛊修的流派层出不穷。旧时的结论,放到现在很多都会不合时宜。即便是我,也无法确切保证会不会有新的法子,能够摧毁宿命蛊。但到了此时,我既然不能阻止你,那就只有冒险一试了。结果你也看到了,不是吗?”

红莲沉思道:“宿命蛊给世间万物都安排好了生命的轨迹,你我皆在其中,哪怕一块山石,一朵浪花也囊括在内。但师父,你有没有想过一点呢?”

“人不是山石,也不是浪花,人是有思想的。一个人如果不满意宿命蛊对其安排的人生,那这个人该怎么办呢?”

龙公凝视红莲:“你也读过《人祖传》,人祖遭遇宿命蛊,要挣脱它的束缚,追逐自由蛊。他怎么样了?他没有把握住自由,他失去了自由,他疯了,胡思乱想!”

“这便是下场,是警示!一个人要去接受他的命,要去承担,要肩负起宿命交给他的责任,去做他应该去做的事情!!”

红莲哈哈大笑:“我知道了,多谢师父的教诲。”

龙公见他神情反常,顿感不妙。

下一刻,龙公的脸上就浮现出极度惊恐之色。

他骇然发现,红莲五指发力,手中的宿命蛊竟发出咔嚓的轻微声响,身体表面浮现出道道裂纹。

“红莲!!”龙公暴怒,猛地出手。

他拼命了!

红莲不闪不避,被龙公击中。

这是致命的伤势!

“你?!”龙公震惊,他忽然间明白,红莲是故意激怒他,让他出手杀死自己。

生命最后的的时刻,红莲仍旧对龙公微笑:“师父,我一直敬爱你,即便是重生无数次,我对你的这份崇敬也未有丝毫减退。”

“但是我需要你明白一点。你看看我吧。”

“宿命对我这样的人,是如此优待,给我安排了一个如此崇高的身份和成就。但是我却不感到开心,想要反抗。那么试问天下,比我更惨比我更糟糕的那些人,对于宿命会有什么想法呢?”

“人不是山石,也不是浪花,人是有思想的。往往胡思乱想,才最是可怕,最有力量!”

说到这里,红莲将手中的宿命蛊递给龙公。

“我只能做到这一步啦。”

“利用爱情蛊,可以损坏宿命。呵呵呵,师父,你是不是很惊讶?”

“可惜,我终究不是天外之魔。”

“但我拼尽全力去安排,做了我所能做到的一切。我把希望留给未来!除此之外,我余下的生命也没有什么值得期待的了,干脆就这样死了罢。”

“但是我相信,总有一天,在未来的某一天,宿命蛊会被摧毁的!就算不是我,也会是其他人。”

“我更相信,将来不再如我一个人单枪匹马,而会是一群人来试图摧毁宿命。”

“师父你阻止不了,天庭也阻止不了,因为你们无法阻止人们去胡思乱想!”

------------

\end{this_body}

