\newsection{方源被擒}    %第四百九十一节:方源被擒

\begin{this_body}

盗天梦境。(wwW.qiushu.cc 无弹窗广告)

“蛊虫选取好了。来让我看看罢。”方源深呼吸一口气,驾驭着少年盗天的身躯。

他首先催动偷袭蛊。

二转蛊虫一发,便见得一个蓝色小孩光影,猛地窜出,直扑向前。

眨眼间,篮婴便没入到洞壁当中,不见踪影。并且整个过程悄然无声,似乎什么都没有发生过一样。

方源心底感叹一声:“这偷袭蛊的确是极品,不仅威力在二转中相当出色,而且悄无声息。这处洞穴按照梦境推演,定然是有防护手段。但我用了偷袭蛊,根本就没有触动防护。等将来有时间,我不妨炼制出一些偷袭蛊来傍身。”

偷袭蛊是方源借助梦境,思谋而出。到了外界,方源也能经过一番改良,将具体的蛊方扩展成套。

二转的偷袭蛊方,方源能从一转扩展到五转,甚至借助智慧光晕,推算出六转、七转的偷袭仙蛊方,都不成问题!

因为方源此时此刻的偷道境界,已经高达准大宗师的地步了!

用了偷袭蛊,方源便又加紧步骤,催动从洞穴中挑选出来的数只蛊虫。

按照小比的奖励,头名可选择三只一转蛊,或者选择一只二转蛊。

小地洞中的二转蛊,方源都看了,虽然不乏精品,但都不合用。方源便从大洞穴中,挑选出了三只一转蛊,当即炼化,相互组合,形成一记杀招。

这记杀招迸发力量,顿时形成一只巨大的犀牛力道虚影。

犀牛力道虚影向着洞壁,迈动四肢,直接用力地冲撞过去。

这处洞壁,原本就被偷袭蛊的威能击中,内里已经崩溃。此时又被犀牛力道虚影猛撞,顿时轰的一声,洞壁崩塌。

随后,呼啦一声巨响,大量的池水从洞壁的缺口处,迅速地涌进来。

原来,方源虽然旁观,但在之前已经考察清楚,他发现这地下洞穴位置就在绿洲池塘的旁边,紧紧挨着。

这种布置,也有道理。<strong>棉花糖小说网Mianhuatang.cc</strong>

因为池塘中有着元泉,越是接近池塘,浓郁的元气就能渗透泥土,传递过来。对于部族在地下洞穴中储藏、豢养蛊虫,有着巨大的益处。

虽然也有一些弊端,但一来部族方面,已经在这里布置了防护手段,二来平时能够进入这里的,一定是自己人。

就算是少年盗天,也是在这部族中生活了十几年,从小培养到大,部族对其也颇有信任。

但此刻,少年盗天被方源操纵着,倒行逆施,丧心病狂,为了完成沙枭的任务,直接在洞壁上开了口子。

大水喷涌而来,洞**壁忽然亮起一层光膜,挡住了大部分的水流。

“怎么回事?”

“快快快,出事了!”

洞穴外传出惊呼声,守卫察觉不妥,立即狂奔进来。

“此时发现又有何用?”方源冷笑一声,这一次催动偷袭蛊,又催用其他两蛊,形成另一杀招。

二转杀招令他浑身浮现出一层蓝色光影,速度大增。

方源迈开步伐,往那光膜和水流中一投,直接穿透两者,进入池塘当中!

这光膜也在方源的考量当中,部族的防护手段,终究只是凡人一级,落到方源这个蛊仙眼中,可谓破绽重重。

至于迎面而来的喷涌激流,纵然冲力甚大,但是方源此刻的杀招,却能轻松辟易,毫无压力可言。

那两个守卫进来之后,只看见方源的背影将将投入到水流当中去。

“这是内贼啊!”其中一位怒吼起来,双眼通红,“部族对你有养育之恩,栽培你,你如何敢如此做?!你的良心被狗吃了吗?给我死!”

说着,就要动手。

但另一位连忙拦住他:“你疯了吗?你那招数大开大合,殃及洞穴中的蛊虫怎么办?当务之急,还是拼尽全力,修补防护,挽留损失。损失越小,我们受到的责罚就越低啊!”

那位被苦劝一番,却听不下去,一甩膀子,直接把同伴的手臂甩掉。

“你来修补防护,我来捉拿这贼子!”他低吼一声,抛下这句话后,立即钻透光膜,向方源追杀过去。

“有敌人?”方源心中一紧。

他非常警觉,追兵入水之后,就察觉到了。

但光是察觉,是没有用的。

少年盗天只不过一转修为,资质低劣,真元极其有限。方源之前连续催动蛊虫,已经差不多耗尽了。

因此,纵然方源乃是蛊仙身份,但此刻巧妇难为无米之炊,面对三转修为的追兵,根本毫无还手之力。

“小贼,你居然敢这么做!你是因为被流放驱逐过,所以怀恨在心,还是受人驱使?”追兵在水中也能说话。

方源憋住气,极力拼斗。

但三两招之后,就被追兵擒获。

追兵恨极了方源,见他还敢反抗,立即手掌一拍,正中方源胸口。

方源顿时七窍喷血,血液在池水中蔓延。

“你这脏水,怎可污染我族灵泉?”三转追兵露出嫉恶如仇的神色,蛊虫一催,周围的血水就如倦鸟归林,统统被他收拢起来。

“苦也。偏偏碰到一个精修水道的三转蛊师来。”方源流露出苦涩的笑容,此时他被擒获,动弹不得,硬生生挨了一招,魂魄受到史无前例的凶猛震荡。

方源强忍痛楚,等到震荡过后,细细查看,就流露出一抹震惊神色。

“好家伙!这一击,足足打掉了我五千万的魂魄底蕴!”

他思维六转,不禁又猜测起来:“西漠中水道蛊师稀少,这人是梦境推演出来,专门对付我的吗?这五千万魂魄底蕴,一下子就打掉了。若是换做其他人来,恐怕是一击致命!就算是我之前,有着九千万人魂底蕴,遇到这种情况,瞬间就崩灭大半。而这第三幕还只是开头,接下来探索下去,必定凶多吉少!”

想到这里,方源不禁庆幸自己稳妥,足足积累了两亿人魂,才进入盗天梦境。

他旋即又猜测:“若是我不冒然冲进来,按部就班,按照梦境发展,恐怕不会有这般打击。不过……我这么一做,恐怕是省却了当中许多过程,节省了许多步骤。这种梦境的变化,值得揣摩深思。”

方源探索梦境,不仅仅是为了偷道境界,而且还有经验的积累,方便他探索其余的梦境。

这一次举措,虽然冒险了一些,但方源收获很大。

他虽然被擒,但脑海中反而灵光又一闪,有了新想法:“等等!这五千万的魂魄底蕴,不是随随便便就有的。不是魂道蛊仙的话,基本上不会有这等魂魄底蕴的。如果让其他流派的蛊仙,来探索这处盗天梦境,会怎样?”

方源又想到唐方明,更准确地说,是回忆起五百年前世,唐方明开创出盗梦杀招。

“五百年前世,随着梦境不断发展,有了层出不穷的梦道蛊虫,更有梦道杀招出现,能够对不同的梦境探索,提供针对的帮助。”

“这盗天梦境,对魂魄底蕴消耗剧烈,绝非是乱世前期能够探索得了的。若是常人要探索,恐怕就得有着足够的梦道积累,最佳的应对是开创出梦魂之类的杀招,能够抵消梦境对魂魄的消融,或者极大地削减这种消融程度!”

这处盗天梦境,出自蛊仙尊者,又这般庞巨,探索的难度非常的高。五域乱世的前中期,简直就是蛊仙坟墓。

方源能够探索,完全是拿着魂魄底蕴硬拼。

这种梦境最恐怖的地方,就是削减魂魄极其剧烈,但方源硬生生靠着雄厚的魂魄底蕴,直接一路闯荡。

池塘的灵泉之水开始咕咕作响,像是煮沸了一般。

“怎么回事?”三转追兵一手提着方源,一边正要上浮,见到这种情况,十分讶异。

“果然,我这一下行动,大大缩短了当中的过程,如今已是到了第三幕的最关键之处了!”方源却是大喜,双目炯炯有神,看着池水究竟会有什么变化。

ps:这个月的剧情会相当精彩,接下来十点还有第二更,求一下保底的月票!(未完待续。)

\end{this_body}

