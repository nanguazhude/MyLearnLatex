\newsection{父子交心}    %第八百一十三节:父子交心

\begin{this_body}

再仔细一看,龙人分身恍然。

出现在他面前的,虽然是一位龙人蛊仙,但并非龙公,只是面貌酷似而已。

“即便如此,此人恐怕和龙公也是关系密切。只是不知道,是否是我这个身份的父亲呢?”

方源正想着,这位龙人蛊仙走到了他的面前:“吴帅!你好大的胆子,我告诉你多少次,不要惹是生非,不要招惹他们!你倒好,居然这一次又赢了他们。你可知道,那张双、陈副都是本门太上长老的嫡亲血脉!”

酷似龙公的龙人蛊仙,神情十分严肃,目光喷火一般。

刚刚还雀跃欢笑的龙人少年们噤若寒蝉,大气都不敢喘,纷纷低下头,看着自己的脚尖。可见他们的父亲在他们心中,威望甚高,平时管教极其严厉。

方源并不能确认他和这位龙人蛊仙的真正身份,因此只能学习其他龙人少年,一样低头沉默。

龙人蛊仙越加愤怒,伸出手掌,敲了一下方源的后脑勺,用力之大,几乎要把方源打个跟头。

方源踉跄了一下,猛地向前跨步,这才稳住身形。

“吴帅,跟我走,到我的书房来罚跪!”龙人蛊仙冷哼一声,转身即走。

方源回首看了一眼龙人少年们,他们一个个看向方源,流露出同情、愧疚、庆幸之色。

“看来只有跟着这位龙人蛊仙走了。”方源心中暗想,低头紧随上去。

走了仅仅几步,周围的天地就陡然发生了变化。

下一刻,方源置身在一间书房当中。

书房颇大,落地的花瓶精致华美,书桌长达一丈,桌面宽阔,旁边摆放着笔墨纸砚,而中央则是草创一半的横幅大字,只有三个字——“龙行天”。

“拿到这个龙人蛊仙,叫做龙行天?不对,看着篇幅和后面的空白,显然还有一个字。”方源心中一动。

就在这时,他的耳畔传来严厉的喝责:“孽子,还不给为父速速跪下!”

方源抬起头,便见那位龙人蛊仙站在书桌后,盯着他,神情十分严肃冷酷。

方源翻了个白眼,心道:“你早这么说,我就知道你和我的关系了。”

当即十分干脆,啪的一下,双膝跪倒在地。

龙人蛊仙楞了一下,旋即冷笑道:“你这次认错的态度,倒是很有长进啊。”

“不敢,父亲的教诲自然有理。孩儿年幼无知,不能尽懂父亲的苦心,还望父亲赎罪!”方源当即敷衍起来,不过表情甚是贴切,给人诚恳的感觉。

龙人蛊仙不由又楞了一下,笑声更加冷酷:“哼!表面恭顺,内里桀骜,居然敢敷衍为父!该打!”

说着,他就拿起书桌边上的镇纸,狠狠地敲打方源的后辈。

那镇纸乃是金铁质地,长条状,边角硬直,又厚又重。本来是压在横幅大纸的两边,令其平整的。

书桌中央的大纸,和正常的门匾差不多大小,可见镇纸的规格。

这镇纸打在方源的后背,顿时一阵剧痛,令方源的魂魄底蕴猛地下跌一大截。

偏偏方源还不能反抗,在这梦境中,他根本不是龙人蛊仙的对手。

不过方源也很狡猾,顺着被打的力道,扑通一声栽倒在地,额头碰到地板,发出咚的一声闷响。

他呻吟一声,又跪直上半身,龇牙咧嘴,剧痛难忍,偏偏又要拼命忍住的样子。

龙人蛊仙到底是他这个身份的父亲,见到这样,心里也不由地浮起一股情绪:“我刚刚那一下是不是打得有点重了?”

“唉!”他长长地叹息一声,坐到椅子上,隔着书桌,盯着跪在地上的儿子。

他深切地道:“吴帅,你是我最优秀的儿子,奴道的天赋百年难得一见。我悉心栽培你,教导你,为的就是让你成人成材。你倒好,把我的教诲都抛之脑后。我告诉过你多少次了,不要和他们那些人争执啊。”

“你是一个天才,你的眼光已经远超于同龄的人。如今的局面,我早就给你阐述过,门派高层越加不满我们龙人一族的壮大。为父虽然是太上长老,看似威风,但终日都受着排挤。”

“你虽然斗赢了陈副、张双,但必定惹恼了他们身后的蛊仙。这些人族蛊仙必然不会给我什么好脸色的,对于龙人一族必然会更加严苛。”

“你赢了一口气,但却损害了整个龙人一族的利益,更激化了门派中的矛盾。你真是太乱来了,做什么事情都不去想后果!实在是太令为父失望!”

“为父打你,也是为了你好,为了整个龙人一族,为了门派的和谐安定,你明白为父的苦心吗?”

方源刚想要顺口回答出“明白”二字,但一瞥龙人蛊仙的眼神,顿时心中一凛。

龙人蛊仙的目光看似柔和,但内里深处却藏着一丝冰冷。

也就方源这种老于世故的人,方能看得清楚。

一瞬间,方源心中提起十二分的警惕,急速思索,各种线索在他的脑海中上下翻腾。

他这一次探索梦境,虽然有本体在外保驾护航,但魂魄肉身都进入梦中,根本就没有失败的机会。

失败一次,就会丧失成为龙宫之主的资格。甚至,还会身死道消。

当然,本体是不会走势不管,但那样一来,本体出手,必然和前世龙公的处境一样,引起的动静太大。

为了最大的利益,方源愿意承担这样的风险,所以他更要小心谨慎。

“这个问题看似普通,搞不好就是这一幕梦境的关键。我若是回答错了,那就是失败的下场!”

“我的分析,应当没有错……就这么办!”

想到这里,方源眼中闪过一抹刀剑般的锐芒。

他抬起头看向龙人蛊仙,脸色自然变化,嗤笑一声道:“父亲为我着想,孩儿感恩。但父亲为了大局的苦心,甘心忍耐,孩儿却万万不能认同。”

“孽子!竟然还执迷不悟!”龙人蛊仙顿时大怒,直接站起身来,一手抓住书桌上的镇纸。

方源忙道:“父亲要打孩儿,就算把孩儿当场打死,孩儿也绝不会怨恨父亲一分一毫。但父亲啊,你这样委屈求助,却是对大局无异,害了整个龙人一族!”

龙人蛊仙双眼喷火,拿着手中的镇纸,指着方源,口中呼喝起来:“好胆!居然敢指责父亲的不是!好,我给你一个机会,你倒是说说看,为父哪里错了?”

方源心中越发笃定,徐徐而谈:“父亲!我们龙人难道就一定要屈居于人族之下吗?我们有天资,有才情,生来就有奴道的道痕,他们人族有什么?”

“我们不需要蛊虫,单凭自身的体魄,就有强大的体能、力量、恢复力。”

“我们的身躯坚硬,我们的爪牙锋锐,而人族却是脆弱至极。”

“在看看寿命,我们的寿命天生是人族的十倍、百倍!我们活着,而和我们同龄的人族,却是早已经老去、死亡。我们甚至不需要对付他们,他们自己就老死了!”

“我们龙人天生如此优秀,更难能可贵的是,我们相互团结在一起,彼此认可,绝不像人族那般勾心斗角,热衷内斗!”

“父亲你在门派中的贡献,有目共睹。我这一次赌斗胜了,难道我就应该故意输吗?”

“凭什么他们这些人族就这样看不起我们,这样苛刻地对待我们!我们龙人应该有更高的地位!”

“甚至,这些凡夫俗子才应该在我们龙人之下!!”

“放肆,你太放肆了!”龙人蛊仙狂怒,快走几步,绕过书桌,举起手中的镇纸,就要狠狠打向方源。

方源却猛地站起身来,虎目含泪,对龙人蛊仙低吼道:“父亲!我不甘,我不服!就算你打死我,我也不甘心,也不会服气!”

“吴帅!!!”龙人蛊仙走到方源面前,俯视着眼前的少年,通红的双眼狠狠地瞪着他,但手中的镇纸却始终没有打下来。

方源毫无畏惧,和龙人蛊仙对视,他的眼中也似在喷火,喷涌出内心深处的不甘和愤恨!

父子俩对视半晌,龙人蛊仙终究是确认了什么,忽然把手中的镇纸直接抛在地砖上。

他双臂把着方源的肩膀,很用力地摇晃方源,忽然哈哈大笑:“吴帅,你不愧是为父的好儿子!为父没有看错你!”

“父亲?你……”方源愣住了,神情中有疑惑,有惊讶,极其自然。

“实话告诉你罢,为父的心思和你一样。只是一直试探你,不敢坦诚相告,这一切都是有苦衷的啊。”龙人蛊仙深深一叹。

“什么意思?父亲你的想法难道是?”方源好似反应了过来,双眼迸发出惊喜交杂的亮光,仿佛真的才刚刚接受过来,又带着一丝不敢相信的感觉,神情惟妙惟肖,毫无破绽。

实则内心深处,他已在感叹:“果然是这样,我分析的没有错。”

“门派中,龙人和人族矛盾激化,从少年间的争斗就可见一斑。”

“之前那处校场,乃是仙阵重地,门派高层必然有着感应。龙人蛊仙若真的忍耐,又怎会不清楚?不前来制止呢?”

“再看这书房的布置,家具高大,布局恢弘,无不暗藏着龙人蛊仙的雄心壮志。”

“他在龙人少年们面前,公开场合下,称呼我为吴帅,到了书房中才口呼孽子,是给我留着脸面和威望。”

“他就是想要栽培我,把我培养成他的助手、接班人。”

想到这里,方源对这片梦境的脉络,还有龙宫的考验用意,已是把握通透!

------------

\end{this_body}

