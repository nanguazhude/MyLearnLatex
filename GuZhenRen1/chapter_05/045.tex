\newsection{各自筹集仙材}    %第四十五节:各自筹集仙材

\begin{this_body}

血神子只是方源的一个长远计划。

方正是这个计划中的第二环节。

这个计划的第一环节,就是先炼出血神子仙蛊。这只仙蛊是方源五百年前世时,就想炼出的蛊虫了。他为夺得仙蛊残方,还主动找凶雷恶人的麻烦。

只有炼出血神子仙蛊之后,才能杀死血脉亲人,炼出一个个的血神子。

但现在,方源手中的血神子仙蛊方虽然有,但只是七成的残方。仙僵身份的时候,没有资源和精力来完善这个仙蛊方。到了现在,方源重获新生之后,智慧蛊方面却不能再蹭用智慧光晕了。

“就算不蹭用智慧光晕,单凭我的血道、智道两大宗师境界,也足以能够将血神子仙蛊方推演完善。只是耗费时间和精力,远不如蹭用智慧光晕那么有效率。”

“就算炼出了血神子仙蛊方,要杀死方正,炼出血神子,还得调教好他,至少让他对我不再憎恨。因为憎恨炼出来的血神子,会反噬主人,相当于双刃剑,能割伤别人,但稍不留神也会伤害自己。”

其实,要影响一个人对另外一个人的感观或者情感,在蛊师的世界中有很多方法。智道中就有无数手段,情道流派更专擅此事。

这些手段,方源掌握的就不少。只要他愿意,只需动用其中一种,就能让方正对自己俯首帖耳,忠心耿耿。

但这样一来,方正只是被外在力量影响,蒙蔽了本心本意。而当他被杀。炼成血神子的时候,动用这些智道手段刻印在他身上的智道道痕都会消除清空。只剩下最纯粹的血道道痕。

如此一来,方正的本心本意就会恢复。

方源只有慢慢调教。彻底改变方正的本心本意,方能炼出符合心中标准的血神子。

显然,方正的调教,是个漫长的过程。血神子仙蛊方的完善,也遥遥无期,至少方源目前无暇顾及此次。更别说,血神子仙蛊方完善出来,还得准备仙材,炼出血神子来。不过最后一点倒是比其他流派容易许多。血道蛊虫有个特点,就是炼制成本相当低廉。

这几个因素,决定了血神子的炼制,暂时只能是一个长远的目标。

但不管再长远的面部小,只要一步步接近,不放弃,持之以恒,总会有完成的时候。

方源现在做的,就是在百忙之中。抽空调教方源,一小步一小步地去接近这个目标。

不积跬步无以至千里,不积水滴无法成江河。

明白了方正的处境之后,方源又再次催动蛊虫。产出一股假意,飞离身边。

这股假意还会飞到方正的脑海中,补充前段时间以来假意的消磨和损耗。

又过了几天。方源对内景雨的处理,达到了炉火纯青的地步。琅琊地灵便又交给他另一份仙材。让他熟悉,锻炼他的炼蛊手法。

方源的主要精力。都投放在这上面,为炼制仙蛊变形做准备。

闲暇时候,他则将目光投注到至尊仙窍之中,查看仙窍中的起色,经营仙窍。

蛊师修行阶段,要温养空窍,空窍中产出真元,用来催动蛊虫。但不管是蛊虫、蛊材还是元石等等,蛊师的这些修行资源,都依赖于外界。

而到了蛊仙修行阶段,仙窍中不仅能产出仙元,而且还能生产无数资源,并且还十分安全可靠。不像外界的资源,需要蛊仙时刻看管,防止被他人盗取抢夺。

仙窍对于蛊仙而言,是最安全的生产基地。若是经营出色,对蛊仙的帮助极其巨大,是蛊仙的大后方。因为道痕一致,它能产出最适合蛊仙修行的资源,帮助蛊仙喂养仙蛊。对外贸易,为蛊仙创收更多的仙元石。甚至还能在仙窍中产生全新的野生仙蛊!

方源查看自己的仙窍。

他的仙窍分为十层,内有五域九天的格局。

为什么有这样的格局,是魔尊幽魂特意凝造的,还是冥冥中自然形成的?方源就不得而知了。毕竟至尊仙胎蛊不是他经手打造,他只是最后夺取了成果的幸运儿罢了。

方源也懒得为各部分一一取名,直接仿造五域九天的称呼,只是在前面添个小字。例如小北原,小东海之类。

观察次数多了,方源又渐渐发现了至尊仙窍中的一些小秘密。

他发现:至尊仙窍中的五域九天,不仅格局方面和外界天地相似,而且环境方面也很贴近靠拢。

比如小北原中,就是一马平川的草原地貌,仙窍中的风道道痕在这一块儿聚集得更多一些。

又比如小西漠,大多数的炎道道痕凝聚在这一块。南疆则是土道道痕多点,还有一部分毒道道痕集中。

发现这点之后,方源就将先前积累的修行资源,都分别布置下去。

他将幽火龙蟒群,放在小西漠中。

将龙鱼群布置在小东海里。

长恨蛛群,布置在小南疆里。

而箭竹林、星屑草等等,都安置进小蓝天中。

至于小北原,则有大量的雪怪群落。这些都是第一次地灾时候,遗留下来的,其中包括大量的荒兽雪怪,甚至是上古荒兽级雪怪。

这些雪怪,方源暂时还抽不出精力来处理,索性先搁置一边。反正小北原中空空荡荡,没有什么其他珍贵的资源。且先任由这些雪怪们活动,反正小北原大得很,雪怪们也不像墟蝠那样可以洞穿窍壁。

而气泡鱼、鱼翅狼、散文鲤、荒兽巨角羊、天地沙鸥死蛋、少量油水、少量石人、一小片血芝林、一小片的镜柳等等,也都安排妥善。

方源锤炼炼蛊技艺的同时,也抽出一部分精力。照看这些资源。

箭竹林、星屑草等等,虽然挪进了小蓝天之中。但蛊虫的产量却下滑得十分厉害。而幽火龙蟒、龙鱼群也在小幅度下滑。唯有长恨蛛平均每天的产量,和之前的数据相持平。

没办法。至尊仙窍虽然前景广阔,潜力无限,但现阶段而言,它只渡过了一次地灾而已。

它蕴含的道痕,虽然囊括各个流派,但绝大多数分别只有一百左右的道痕。

长恨蛛群的培育,依仗智道道痕,龙鱼群则靠食道、水道道痕,幽火龙蟒看生存环境中炎道道痕的多寡。箭竹林、星屑草场中的星道凡蛊产量。就看星道道痕。

先前这些资源,都放在狐仙福地、星象福地之中。

前者度过了五六次地灾了,道痕很多,炎道、水道道痕多过现在的至尊仙窍。而星象福地本身就是以星道为主,至尊仙窍在这方面,根本没法相比。

不过至尊仙窍之中,也并非所有的流派,都只有一百左右的道痕。

气道道痕,就高出一筹。这是因为方源之前在南疆的五界山脉中。斩杀了气道蛊仙戚灾,得了他气道道痕的积累。

但气道道痕还不是最多的,最多的是冰雪道痕。

方源前不久惊险度过的第一次地灾,说是“地灾”很不准确。总之威力恐怖至极,把什么十大凶灾都甩掉几条街外去了。

灾劫向来讲究福祸相应,灾劫威能强大。渡过去后蛊仙获得的好处就越多。

这里的好处,就是指道痕。

方源从第一场地灾中。获得了大量的道痕。

其中以冰雪道痕最多,然后是力道、变化道道痕。后两派道痕。要逊色于气道道痕。毕竟气道道痕是七转蛊仙戚灾的一生积累,而力道、变化道道痕不过是方源第一次地灾所得。

这些道痕,已经开始影响方源的至尊仙窍。

其他四小域还不见端倪,但小北原中,已经形成了一大块的雪原,并且雪原的边缘还在缓慢地扩散。

这样的环境,足以让方源移栽许多耐寒的植株进去,足够它们生长了。

就像当初,经历几次地灾之后,狐仙福地环境改良,狐仙这才能移栽了蓝度草、马蹄草、六神草、奶茶花等等。

经营仙窍的一个主要方针,就是因地制宜。

为了炼制仙蛊变形,方源在积极地做准备,另一边中洲,影无邪等人也没有闲着。

太古之光已经到手了,按照仙蛊方,他们开始筹措其他仙材。

他们凭借自身库藏,解决了其中的一小部分,但还有不少,需要外界贸易,互通有无。

影无邪继承了中洲影宗残余势力,更是逆组织的大先生,所以他的首要途径,就是逆组织的内部资源交换。

一次次商谈,讨价还价,然后一笔笔进行交易。

蛊仙逐利,逆组织又结构松散,关于仙材的交易并不顺利。

为此,影无邪和黑楼兰二人几乎磨破了嘴皮子。石奴、太白云生可不适合讨价还价。

费尽心思和力气,影无邪终于筹措出了三份完整的炼蛊材料。

“余木蠢已经阵亡,我此番炼制定仙游,只能舍弃毛民自然流,采取人族隔绝流的炼制方法。这个方法炼蛊成功率很低,只有三份仙材可不够,极不保险。”

影无邪没有停止收购,并且为了炼蛊,他和其余三仙冒险返回地渊之中。

影宗在地渊里布置的超级蛊阵,对炼蛊也有不少帮助。

当影无邪通过逆组织,艰难筹措到六份完整仙材时,一个意外发生了。

逆组织的成员之一,隐藏在明堂谷中魔道蛊仙公孙良,被天庭蛊仙突击生擒活捉!公孙良向逆组织求救,逆组织立即陷入极大的动荡之中。

公孙良隐藏得很深,居然被捉住了。更关键的是,天庭方面亲自出手。这点震撼人心。

逆组织的大多数成员,都是被中洲十大古派欺压排挤,为了生存暗中联盟。而十大古派亦不过是天庭的下宗罢了。

面对天庭这个庞然大物,逆组织人人自危,害怕公孙良暴露线索,将他们都牵扯出来。

“该死的天庭!”对此,影无邪唯有咒骂的能力。

因为这件事情,逆组织的其他成员都潜藏起来,不再敢和他沟通,更遑论交易仙材了。

影无邪知道,他们不仅是害怕天庭,更是对自己这个大先生产生了犹疑。毕竟前段时间,公孙良还将仙蛊借给了影无邪,此次他遭灾遇害,仙蛊还留在影无邪的手中。是不是贪墨仙蛊,所以大先生故意不救?这个疑问,让大先生的威信名誉受到极大的创伤。

公孙良被抓,直接断掉了影无邪筹集仙材的最佳途径。

迫不得已之下,他们只好借助宝黄天。

这是五域最大的交易市场。

但手续费很高,又不保密,很容易就会被有心人追查。

总之风险很多,若不是实在没法了,影无邪他们是不会动用这条线的。

“公孙良被抓,天庭方面,迟早会查到这里。监天塔主死了,但紫薇仙子却苏醒过来。我必须尽快炼出定仙游!”

影无邪心中越发焦躁不安。

他的情形也越来越窘迫。

影宗为了炼制至尊仙胎蛊,几乎将所有的库藏耗费一空,甚至就连下宗僵盟都当做材料,投入十绝大阵中炼化了。

石奴虽是七转,但仙窍经营不善。

加入的黑楼兰、太白云生,更是升仙不久,积蓄有限。

宝黄天中,虽然仙材琳琅满目,但影无邪却苦于手头拮据,只能一块钱掰成两半用。这样一来,锱铢必较,谁愿意和他这样的人进行仙材交易?

最终,当影无邪靠着宝黄天,终于筹集到整整十份炼蛊材料时,他差点辛酸落泪,满目尽是凄凉之色。

虎落平阳被犬欺。

若是义天山大战之前,要筹措一只六转仙蛊的仙材,只需从库藏中提出一个角落,就是十份,二十份完整的仙材。就算库藏中没有,也只需要一声令下,五域中的僵盟就会出动大量仙僵,筹措仙材轻而易举。

何至于此!

为了筹集仙材,现在的影无邪等人,几乎已经库藏殆尽,两袖清风了。

“这一次炼制定仙游,一定要成功!必须成功!”影无邪暗中发狠。

他这边困顿难捱,方源那边却十分滋润。

方源已经和琅琊地灵商量妥当,就算第一次炼蛊不成,那就继续炼。

至于第二次、第三次之后的仙材,方源都统统一力负担。

他手头充裕,或者说十分充裕。

别看这些主要资源减产了,但方源因此的收入,却比之前要翻了数倍!

为什么?

因为方源的至尊仙窍中宙道资源十分充沛,光阴流速和外界相比高达六十倍。

这是什么概念?

外界一天,至尊仙窍中就是两个月!

时间一长,资源生产的就多,完全弥补了减产的小小弊端。

不过这些资源,只有一部分让方源贩卖出去。和琅琊地灵商量的时候,方源故意抵押了荡魂山。反正方源还得起。

但这点却让琅琊地灵动力十足。

琅琊地灵现在都有点担忧第一次炼蛊就直接成功,这样的话,他岂不是无法趁机获得荡魂山了吗?

ps:这章4000+,今晚蛊真人微.信公众号上讲人物志剑仙薄青,欢迎大家关注。(未完待续。)

\end{this_body}

