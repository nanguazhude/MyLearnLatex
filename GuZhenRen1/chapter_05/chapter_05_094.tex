\newsection{开启天晶鹰巢}    %第九十四节:开启天晶鹰巢

\begin{this_body}

%1
至尊仙窍,小紫天中。,

%2
微光泛紫,一片朦胧。

%3
偌大的天空中,悬浮着一座鹰巢。

%4
这座鹰巢宛若一栋三层阁楼,无门无窗,仿佛是一团巨大的水晶簇。

%5
一种半透明的水晶,每一根犹如成人两臂的长短和粗壮,相互簇拥,浑然一色,搭建出这个密实得毫无缝隙的鹰巢。

%6
这种水晶,来历非凡。

%7
方源已经认出来,这是天晶,高达八转的太古仙材!

%8
天晶来源于天空,本身很轻,可以悬浮在空中,不受外力则永不坠落。

%9
别说鹰巢内部有什么,单单这个鹰巢本身,就包含上千斤的天晶!光这笔天晶,就具有极其庞大的价值。

%10
很显然,这座鹰巢正是铁鹰福地中的第十三座鹰巢。

%11
本身价值,就远超铁鹰福地中的其余鹰巢,高居魁首之位。

%12
方源和楚度联手之后,楚度便前往铁鹰福地,在黑家大战的最后时刻,展现出强大战力,将这座鹰巢抢下。

%13
不过,在楚度脱离战场之前,黑家太上大长老嚎叫了一嗓子,使得众仙皆知,这第十三座鹰巢非同小可,里面包含着黑凡真传的线索!

%14
这句话给方源带来了不少麻烦。

%15
楚度知晓这点之后,和方源重新谈判。

%16
方源知道自己必须让步。

%17
飞剑仙蛊、东海信道真传,还有这第十三座鹰巢,以及黑凡真传,结合起来的价值。十分庞大!

%18
楚度尽管专修力道,至死不渝。但并不意味着他不会拿这些东西,去卖个好价钱。

%19
幸好楚度一心想求得方源的渡劫之法。

%20
方源狠狠卡住他这个最大的弱点。索性和楚度加深合作,耗费大量口舌和精力,双方这才勉强谈妥。

%21
楚度利用招灾,从方源那里分得狂蛮真意。还会在将来,方源从这两道真传中的收获中,获取三成收益。

%22
“楚度是寻常蛊仙,不想冒然兼修信道和宙道,所以他对这两份真传的兴趣并不浓厚。”

%23
“不像我可以全派通修,任何一道真传都可以拿来自己修行!”

%24
“即便如此。我也很佩服他的果断,能舍能弃。北原霸仙,果然非同凡响,盛名之下无虚士!”

%25
方源一边查看着鹰巢,一边思绪则有些泛滥。

%26
他和楚度之间,算是最标准的利益同盟。交情极浅,纯粹是双方利用。

%27
对楚度,方源心底还是颇有忌惮的。

%28
他知道楚度绝不甘心,仍旧想要得到仙劫锻窍法门。

%29
楚度有勇有谋。绝非“无智”的霸仙。

%30
他的勇,能以一人之力,令超级势力黄金家族刘家吃瘪。黑家大战,更让群雄束手。

%31
他的谋。亦是不容小觑。刘家一战之后,他就鲜少现身,一直默默地栽培徒弟。帮助他们成仙,汲取狂蛮真意。和方源的谈判。他的口才更是教方源领略到,此人的城府和才智。

%32
霸仙雄心万丈。怎么可能满足手中的招灾?

%33
更令方源有些担忧的是,楚度似乎有些猜到了他的真实身份。

%34
第三次地灾,也是他们首次合作之后,他吟诗一首,后两句便是“青云托我瞰江湖,天地方圆一览无。”

%35
方圆、方源。

%36
隐隐有指破方源真身的意思。

%37
当然,这点也许是方源多想了。

%38
不过很多事情,方源宁愿多想一层,也不想少虑一分。

%39
楚度的出现也很蹊跷,就是在第一次地灾时,他差点要了方源的命!

%40
方源很怀疑:他就是受到天意影响,布局出来的,对付自己的“人劫”。

%41
和这种人物联手,有个成语似乎相当恰当,便是“与虎谋皮”。

%42
所以第一次联手,渡过了第三次地灾之后,方源和楚度也没有多聊,很快就告别了。

%43
楚度放任方源离开。

%44
一来,他手中有招灾仙蛊。二来,他和方源已经订下盟约。用的信道仙蛊,是他从其他蛊仙那里借来的。三来,他追不上方源!

%45
依照楚度的实力和手腕,说他交游广阔,也是名如其分。

%46
有一点方源考虑到了。为了防备天意,方源交给楚度不少的我意蛊。至少方源不用太担心,天意会对楚度产生什么不好的影响。

%47
“和楚度的合作,必须小心翼翼。稍有差池的话,恐怕就要万劫不复!”方源在心中暗暗警惕自己一句,主要的注意力开始放到力道仙僵身上。

%48
他操纵着一头力道仙僵,缓缓接近天晶鹰巢。

%49
这些天来,他不断侦查、揣度,收集各种线索,不断挖掘黑城的记忆,然后加以推算。

%50
“如果我算得没错的话,这一次……”

%51
方源怀着一股信心,默默催动起仙蛊来。

%52
接近天晶鹰巢的力道仙僵,旋即发生了彻底转变。原本死气沉沉,顷刻之间,大变活人,化作黑城,真假难辨。

%53
“黑城”渐渐接近天晶鹰巢,将手缓缓地搭在鹰巢表面。

%54
方源的神念不断发挥,灌注到天晶鹰巢之中。

%55
与此同时,一只只蛊虫飞出来,环绕在“黑城”的身边,不断旋绕飞舞。

%56
炼道杀招一个个接连催动,打在天晶鹰巢之上。

%57
这些炼道杀招,皆是凡道,乃是黑凡当年改良而得,流传下来。黑凡临时之前,叮嘱后人,但凡今后有黑家族人成仙,都要学习他的这些炼道手法。

%58
经过黑家蛊仙几代人的钻研,他们渐渐明白:这些炼道杀招,应当便是开启天晶鹰巢的钥匙。

%59
很多黑家蛊仙,为了黑凡真传,都曾经在天晶鹰巢上动用这些炼道杀招。

%60
当然。这些尝试都已失败告终。

%61
因为他们手中,缺乏最关键的因素态度蛊!

%62
在这些炼道杀招中。有个关键手法,必须截取态度蛊的一丝气息。才能发出。

%63
这个关键,阻碍了历代黑家蛊仙,使得黑凡真传无人能够继承。

%64
黑家蛊仙们也不是蠢货,明白这点后,立即着手暗中搜寻态度蛊的下落。

%65
事在人为,黑家历代蛊仙努力之下,渐渐发现端倪,有了阶段性的成果。

%66
到了最近这一代,苏仙夜奔。成为黑城爱妻,最终以外姓太上家老的身份,加入黑家。

%67
她不姓黑,黑凡真传这种重大秘密,是不能得知的。

%68
但苏仙本身就目的不纯,在其大姐焚天魔女的暗中相助下,窃取到黑家历代积累的关于搜寻黑凡真传的部分成果。

%69
借助这些成果,她查探到一些可疑地点。

%70
正当她接近成功的时候,事情发生突变。黑城杀死苏仙儿。三妹之死让二姐黎山仙子因此和其大姐焚天魔女闹翻,苏仙儿留下关键线索,以及一股亲情和大批仙元,交给爱女黑楼兰。

%71
黑楼兰升仙之后。按照这些线索,从北部冰原中取回了黑家遗失多年的态度蛊!

%72
但又经过一番曲折辗转,最终这只仙蛊落到方源的手中。为他所用。

%73
这几天来,方源都在熟悉炼道杀招。还有伪装变化,确保做到万无一失。

%74
一记记的炼道杀招。时而化作火焰,时而变作雷电,击打在天晶鹰巢上。每一击都使得天晶鹰巢微微颤动,发出清脆激昂之音。

%75
渐渐的,天晶鹰巢的震动幅度越来越大,发出的声音也越加高昂,开始响彻天穹。

%76
“最后一记!”

%77
方源的心不免提了起来。

%78
在他纯熟无比的手法下,一丝态度蛊的气息被抽取出来,融汇出一股透明的水花,悠悠飞向天晶鹰巢。

%79
水花顺利至极地融入到天晶鹰巢中,旋即就消失不见。

%80
天晶鹰巢开始持续颤动,一股股的光辉从内部深处贲发出来。渐渐的,将整个天晶鹰巢染成九种颜色。

%81
赤橙黄绿青蓝紫白黑!

%82
九色光芒冲天斥地,刺眼逼人,“令黑城”不得不缓缓退开一段距离。

%83
铃铃铃……

%84
光辉逐渐消失,风铃一般的声音不断在空中激荡。

%85
在“黑城”面前,这座闭合了无数岁月的天晶鹰巢,终于在此刻缓缓展开!

%86
每一根天晶,都在一步步的移动,不断拆解,让方源联想到了积木或者魔方。九色天晶相互夹杂、镶嵌,天晶鹰巢慢慢地展开一个入口。整个过程美轮美奂,绚烂得叫人下意识地屏住呼吸。

%87
“我明白了!”方源心头一跳,陡然领悟出了黑凡的手法。

%88
“天晶原本半透明,看似来源一处,其实不然!”

%89
“这些天晶各有九色,应当分别来自太古九天之中。但是黑凡将这些天晶都融汇到一起,把这些八转级数的仙材,动用炼道手法进行处理。使得它们都褪去颜色,看起来似乎是一体的。”

%90
“黑凡死前,留下一些炼道手法。用这些炼道手法,就能逆处理这些天晶,使得它们恢复本来面貌,从而引动内部布置,开启整个鹰巢。”

%91
想到这里,方源心中对这些天晶的估价,又不禁上升一个台阶。

%92
天晶这种仙材,除了太古九天中产出,还可在底蕴深厚的洞天里自然产生。

%93
原本,方源以为,这是黑凡洞天中产出的天晶。但事实上,这些天晶来源于太古九天。

%94
那这个价值就更高了。

%95
因为太古九天已经只剩下两天,其他七天都已经毁灭。

%96
当年的黑凡,搜集到这些太古九天天晶,一定费了不少事。毕竟现在的蛊仙要取得这些天晶,越来越难,唯一的来源就是散落在五域各处的太古七天碎片小世界了。

%97
“黑凡真传,我来了。”方源按捺住激动的心情,操纵“黑城”,小心翼翼地钻入鹰巢之中。

\end{this_body}

