\newsection{悔}    %第六百三十二节:悔

\begin{this_body}

“乐土真传?”五相公共洞天之中,方源接到了庙明神的来信,邀请他参加此次苍蓝龙鲸的探索。

方源眼中闪过一抹思索的光。

庙明神居然和任修平发生了冲突,提前开始探索苍蓝龙鲸,这有点出乎方源的意料。

在他的五百年前世记忆当中,这已经提前了很多。

“或许是我重生归来,改变的种种影响不断扩散,令庙明神和任修平的冲突提前了。”

“也或许我的五百年前世记忆,本身就有问题。毕竟之前的春秋蝉中可是埋伏着天意,它要动手脚,实在是便捷得很。”

“还有一种可能,那就是天庭设计,暗中推算出了我和庙明神的这层关系,便想要趁此机会来再次算计我。”

想到这里,方源沉下心,缓缓闭上双眼,开始推算。

许久之后,他睁开双眼,得出结论:“按照目前的种种线索推断,天庭设局害我的可能并不高。看来我得去东海一趟了。”

如今五域当中,中州、北原对于方源而言,因为有着天庭、长生天,可谓之龙潭虎穴。剩下的三域东海、西漠、南疆,当然是南疆最为方源熟知。尤其是方源潜伏在武家一段时间,前段时间更是俘虏了许多南疆强者,因此对各大家族都知道许多底细。

还有一层原因就是,影宗的总部就设立在南疆,影宗对于南疆的经营力度是最大的,暗地里还要不少天然仙阵等地。这都是潜伏在暗处,没有暴露。

种种因素,让方源选择待在南疆这块地方。

即便是黑楼兰等人渡劫,也在南疆。

虽说五域之间,天地二气有着差异,强行在异地渡劫会损害仙窍。但方源本身也不想他们把自家仙窍经营得很好。方源承担他们的修行资源,也是对他们强有力的掌控。

这段时间下来,方源已经将气海洞天中的诸多仙蛊,统统炼化,归于己用。

气海洞天、五相公共洞天中的资源,也初步消化。

剩下九转仙道杀招天相,却颇为顽固,抵抗着方源的淬炼。不过进展还是有不少的,假以时日,必定能够为方源所用。

方源现在苦恼的是,自家在宝黄天中的诸多贸易,都受到了中洲天庭、南疆正道联盟的联合打压,精确阻击,形成了一道巨大的经济封锁阵线。

方源想过不少方法,和他们斗智斗力,但效果均不佳。

不管是中洲天庭还是南疆正道联盟,都是财大气粗,底盘浑厚,方源的资本抵不上这两个庞然大物。

解决这个困境最直接的方法,就是胆识蛊之类的垄断生意。但此法却需要海量的胆识蛊,偏偏方源缺少魂核,荡魂山虽然修复好了,但产量却上不去。

他之前图谋青鬼沙漠的计划,已经告吹。若是成功,此刻方源面对中洲、南疆的联手封锁,就没有什么难度了。

青仇行踪至今不明,而房家更是遭受西漠超级势力的默契刁难。房家想要提升自己的地位并不容易。

方源不是没有想过利用房家,但在这点上,他颇有顾忌。因为他之前算不尽的身份,也已经被天庭洞悉了。

当初走青鬼沙漠这一步,从现在看来,都是对的。遗憾的是,事与愿违,人生中很多事情都和自己的想法背道而驰。

如此一来,方源的收入就不断萎缩。更令他难过的是,中洲天庭还加大力度搜刮宝黄天中的宙道仙材。

即便上,只要是宙道的仙材,天庭方面都能付出卖家想要的东西。因此这段时间,宝黄天中的宙道仙材行情前所未有的好。

这就苦了方源了。

方源本来就收益锐减,想要炼制宙道仙蛊,增强这方面的实力,好再下光阴长河,结果被天庭霸道无比地推搡开去。

天庭从远古时代就建立,历经沧海桑田,无数年的积累,方源怎么能比得上?

方源在宙道方面进展微小,反观天庭收刮了大量的宙道仙材,兴许还能炼出第五座仙蛊屋来。

此消彼长之下,方源寻找红莲真传的希望越来越渺茫。

“我需要大量的宙道仙材,才能炼制宙道仙蛊。乐土仙尊却是土道成尊,苍蓝龙鲸中的乐土真传中,恐怕没有多少宙道仙材。但即便如此,我也要去!”

这种未确定的将来,谁也说不准能获得什么。

就算不是宙道仙材,乐土真传中的东西当然都会是好东西。

“南疆的八转蛊仙陆畏因,就是当代乐土传人,我若能获得此道乐土真传,兴许就有对付他的办法。”

“更重要的一点是,根据我五百年前世的记忆,苍蓝龙鲸中似乎藏有八转仙蛊悔!前世庙明神多次探入其中,虽然始终没有得到乐土真传,但每次都有不少的收获。因为参与的蛊仙多,进出的次数多了,就有关于仙蛊悔的传闻流传。”

乐土仙尊以仙蛊悔为核心,开创了名垂青史的仙道杀招――悔过自新。他运用这一招,令不少的魔道蛊仙改变阵营,幡然醒悟,弃恶从善。悔过自新这记杀招,可谓战绩赫赫。

仙蛊悔自然是智道仙蛊。

智道三元要素,分别是念、意、情。仙蛊悔便隶属于其中的情,后来有蛊仙从智道中划分出情道这个小流派,所以仙蛊悔同样属于情道。

在漫漫史实记载中,仙蛊悔多次出现,有时六转,有时七转,辗转在五域的蛊仙手中。到了红莲魔尊手中,仙蛊悔似乎是首次升炼到了八转。后来乐土仙尊也不知从哪里,获得了八转仙蛊悔。

方源对仙蛊悔的调查相当深入。

因为他想要重现悔池。

说起来,仙蛊屋悔池的核心除了光阴支流外,就是仙蛊悔。

可惜当初,在东海残留下来的悔池也只是半成品,一直缺少仙蛊悔。

“若是这道乐土真传中真的有八转仙蛊悔,我有此蛊在手,就能仿造出一部分的悔池功效,将来对我炼蛊大有裨益!”

这是最让方源心动的一点了。

事实上,悔池也在方源的计划当中。他通过勒索南疆正道,辛辛苦苦搭建出来的仙蛊屋雏形,就承担了他在这方面的期望。极其遗憾的是,仙蛊屋雏形在光阴长河中被彻底摧毁,方源根本来不及回收任何的蛊虫。

数天的时间,一晃而过。

东海。

庙明神扫视一圈,在他身边站着八位蛊仙。

当中,自然就有方源了。

“诸位。”庙明神沉声道,“关于此处行动,都已经向诸位详细说明了,并且给了一天时间让诸位多加考虑。此行我也是首次,因此风险难以评估,或许有身死道消的可能。诸位当中若有人选择放弃,请现在离去。我庙明神礼送,绝不阻拦,也发自真心的理解。”

海风呼啸,吹拂耳畔。

庙明神又扫视一圈,没有人动弹。

乐土真传品阶太高,哪怕东海中土道蛊仙相当稀少,在场的蛊仙当中也就一位专修土道,谁都不想放弃。

人为财死鸟为食亡。

并且,古往今来的蛊尊当中,就属乐土仙尊最为慈悲仁义,他生平从未害过一人性命,布置的仙道传承也向来以温和著称。

这样的机缘,谁不抓住,那就是傻子!

“好,诸位既然都想要参与,就请先缔结盟约吧。”庙明神精神振奋。

对于此行,他当然有了充分的准备。

“还请曾落子仙友出手。”庙明神拱手邀请道。

曾落子便迈步而出。

他一身青白袍子,面容干瘦,脸上颧骨高凸,眼窝深陷,双眼中精芒电闪。

他有七转修为,专修信道,庙明神之前早就和他沟通过,因此当即道:“这是我们即将要结的盟约,诸位请细细查看一番。”

曾落子话音刚落,在场的不少蛊仙均身形一震,面露一丝骇意。

原来,在他们的脑海中竟同时浮现出了盟约的内容。

曾落子动这一手,无疑是来一个下马威,好确认自己在队伍中的超然位置。

这场探索乐土真传的行动才刚刚开始,蛊仙们之间就出现了明争暗斗。

------------

\end{this_body}

