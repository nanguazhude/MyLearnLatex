\newsection{得元始真传}    %第八百三十一节:得元始真传

\begin{this_body}

%1
费尽好一番周折,龙公终于和方源谈妥。

%2
东海两位八转沈从声、宋启元目睹全部过程,无法阻止。

%3
龙公不仅赔偿了方源大量的气道修行资源,更重要的是竟是将一份元始仙尊的真传赠与气海老祖!

%4
这让东海二仙无奈之余,也不得不佩服龙公的器量!

%5
当然,现场只是交接了一小部分。

%6
方源伪装的气海老祖和天庭休战,开始和平共处,而天庭会在今后陆续将这些资源补足。至于元始真传,也是一部分一部分地交割。

%7
毕竟双方都没有签订什么盟约。

%8
双方算是谈妥了,方源不仅没有直接离开,反而主动在前方领路。

%9
“方源的确是在收取龙宫,或许龙公你可以在今天了解了他和你们天庭的恩怨。”方源微笑着道。

%10
从脸上的神情看去,他就好像是要去看热闹的人。

%11
不过方源不想继续拖延时间,而是继续扯皮下去,容易给龙公、东海二仙看出破绽来。

%12
既然如此,不妨顺势而为。

%13
左右没有签订什么盟约,若是到时候该要动手,方源当然会立即直接动手。

%14
至于之前谈成的那些,统统都是狗屁!

%15
龙公往前疾飞,目视前方,神色平静下来,淡淡地回应道:“若是如此的话,那也算是解决了一个麻烦。”

%16
东海二仙相互对视一眼,事关龙宫,也没有急着离开。

%17
一伙人来到龙宫藏身之地,深入海底,结果龙宫早已经消失。

%18
“这里真的有龙宫存在吗?”东海二仙不太相信,他们怀疑方源故意带错了路,但又不敢明说。

%19
龙公却是没有怀疑,此行虽然是方源在前方领路,但实际上这个位置和紫薇仙子推算出来的,没有一丝一毫的误差。

%20
“看来龙宫真的是飞走了。”龙公叹息一声。

%21
方源笑着道:“真是有点可惜了,若是龙宫被方源收走,那你们天庭可就又败给他一次了。”

%22
方源一副看热闹不嫌事大的神情,用话语刺激龙公。

%23
龙公反应淡淡,并不受激。

%24
因为他对龙宫本身就有很多了解,当初孙子小八建造它时,就已经规定了必须是龙人才能成为龙宫之主。

%25
如此一来,自己手中掌握着龙人寂灭杀招的龙公,等若是杀手锏在手。若是方源真的收服了龙宫,必定是转变成了龙人,或者有着龙人的下属。

%26
“方源生性谨慎,又从未来重生,自己转变成龙人的可能性很小。但即便如此,我催动龙人寂灭也能打掉他一个得力的属下。”

%27
“但杀不杀人只是旁枝末节,关键是要收取龙宫。”

%28
“此次虽然失败,但也不无收获,至少缓解了气海老祖和天庭的关系,甚至还开拓出了将来招揽气海老祖的可能。”

%29
“至于龙宫……它如今已算出世,必定会留下蛛丝马迹,将来必定能撞见。”

%30
“唉,方源重生归来,优势太大,提前布局,优先出手。龙宫这方面我又是晚了一步。”

%31
龙公思索了片刻,他并不急着去催动龙人寂灭。

%32
这一记杀手锏不妨留着,在未来的关键时刻,打方源一个措手不及。

%33
当然,龙公也想到方源可能知晓龙人寂灭这个杀招。但若他知晓,就会有防范,龙公催动出来也没有多大成效。

%34
所以不管是哪种情况,龙公暂时不动这个手段也是明智之举。

%35
“龙宫落到方源手中,大大不妥。收取龙宫和对付方源就合并成了一件事情,作为克制方源的方正,还是得先赎回来。”

%36
念及于此,龙公便转身和沈从声交涉。

%37
古月方正就扣押在沈从声的手中。

%38
沈从声早有心理准备,对于龙公这个要求,没有丝毫的意外。

%39
毕竟他沈家和龙公背后的天庭,都隶属于正道,行事风格相差不多。

%40
沈从声当然愿意和龙公交易,送走古月方正,换取大量的利益。

%41
他对古月方正没有兴趣,方正对他也没有什么大用。沈从声暗暗兴奋,决定大敲天庭一笔。

%42
天庭财大气粗的模样,他刚刚可是亲眼目睹。

%43
方源立在一旁,也是暗自心动。但此时若是和龙公公然竞价,反而不美。

%44
一来,刚刚他才和龙公谈妥,现在就要当着龙公的面拆台,有违气海老祖本身这个角色的行为。

%45
二来,方源也没有把握能和天庭竞价。

%46
若是打杀沈从声,然后抢夺方正,这种可能性也比较低。

%47
不过,方源还是没有放弃,他暗中向沈从声传音,故意询问。

%48
沈从声听气海老祖居然对方正感兴趣,连忙回应道:“不瞒老祖,我刚刚已经是将方正搜魂,知道了他最近的经历。看这个样子,他的确是针对方源,在冥冥之中对方源存在克制。”

%49
方源眼中精芒一闪,当即暗中对沈从声蛊惑道:“方源此人虽只有七转,和我也只是初见,却给我留下深刻的印象。既然方正克制他,不妨将方正留下,和方源谈判。方源的手中可是扣着不少的尊者真传。你若是和天庭交涉,天庭能给你尊者真传吗?”

%50
方源企图先劝说沈从声自己扣押下方正,让他找寻机会再下手。

%51
毕竟方正在沈从声手中,和在天庭手中,这是两个难度。

%52
沈从声积极回应,不敢有丝毫怠慢,笑着道:“老祖说的是,这的确是一个好建议!”

%53
他一边回着,一边又对龙公道:“天庭若要赎回古月方正,给予一项尊者真传即可。”

%54
“哈哈哈。”龙公被逗得大笑,“沈从声啊,你这是狮子大开口。你以为你和气海仙友一个层次吗?气海仙友能得尊者真传,是和我打出来的。你若靠着俘虏轻易得来尊者真传,岂不是让气海仙友成了一个笑话?”

%55
“沈从声,你可要注意你的态度。”

%56
“若是你还有不切实际的奢望,那我们之间就先不妨比斗一场。”

%57
龙公的态度转变得非常明显,语气十分霸道,和方源之前交流完全不一样。

%58
沈从声顿感棘手,龙公口才了得,稍不留意就让他暗捧了一下气海老祖,甚至反借气海老祖来压价。

%59
但沈从声通过搜魂,更加明白方正的价值,当即冷笑,毫无畏惧地道:“龙公,我承认我自己不是你的对手。但那又如何呢?你能击败我,但暂时还杀不死我。况且最关键的一点,方正就在我的手中,他的生死就是我的一个念头而已。”

%60
“你若不给尊者真传,那你能付出什么?好歹也让我看看天庭的诚意啊。”

%61
沈从声也不迷糊,说话软中带硬,但他终究还是退而求其次了,没有在强硬地要求什么尊者真传。

%62
毕竟龙公的拳头可比他大得多!

%63
双方一阵扯皮,没有谈妥。

%64
沈从声要价很高,而古月方正对于龙公和天庭,又的确很重要。

%65
方源旁观,见微知著。天庭越重视方正,就越说明方正对自己有着克制,方源心头的杀机也就越加浓郁。

%66
他暗中撺掇沈从声:“依老夫之见,你可以和方源联系,谈一谈。或许他出价会比天庭更加大方。”

%67
沈从声却心中有数,回应道:“气海老祖前辈,晚辈更宁愿将方正交易给天庭。一来,天庭乃是正道,讲究信誉,需要活的方正。而方源只需要死的方正,未必能和我平心静气地交谈。二来,要论底蕴和财力,天底下有谁能比得过天庭呢?”

%68
沈从声到底是正道出身,天生对于魔道的方源有着隔阂,更愿意和天庭交流,哪怕他对天庭充满了忌惮。

%69
方源劝阻不了,也不好过分使力,防止被看出破绽,只得寻思其他良谋。

%70
龙公心中越加不耐。

%71
他之前和方源艰难谈判,费尽口舌,已经是大违他的平日作风。现在又要和沈从声扯皮,这让龙公有一种出手的冲动。

%72
但方源就在一旁,龙公考虑到气海老祖、沈从声都是东海蛊仙,便按捺下了这份冲动。

%73
这时,紫薇仙子提供给龙公一个情报。

%74
龙公听了眼前一亮,这个情报可是重要筹码,当即就暗中传音给了沈从声。

%75
沈从声脸色顿变,失声道:“当真?”

%76
“自然是真的。”龙公颔首,“如何?之前的那些资源,再加上这份情报。”

%77
沈从声竟毫无犹豫,直接答应下来。

%78
这让方源、宋启元都心头暗动,寻思两人私底下到底达成了什么交易。

%79
古月方正当场就被沈从声,交还给了龙公。

%80
方正已经昏死过去,龙公一手提着,和方源告别。

%81
方源无法挽留,只得任其离开。

%82
留下来的东海二仙都异口同声,热情地邀请方源前去他们地盘做客,更直接允诺有一批资源赠送给方源,当做他们俩无意破坏仙道战场的补偿。

%83
方源点头,他也需要通过这两人来影响整个东海蛊仙界。

%84
和他们约定好了日期后,方源便转身离开。

%85
远远飞行了许久,他停留在一处云端。

%86
云团中一座仙蛊屋缓缓升起,龙人分身走了出来。

%87
方源和他的分身相视一笑。

%88
这次前来东海收取龙宫,虽然有不少的波折,但有惊无险,并且收获远超预期!

\end{this_body}

