\newsection{太丘行}    %第六十四节:太丘行

\begin{this_body}

琅琊地灵所指之地,正是太丘!

方源心中真正的目标,也是太丘.pbtt

但是他不好明说,只能旁敲侧击,加以引导,让琅琊地灵自己“领悟”出答案来。

太丘,北原十大凶地之一。

在这里,长满了一种巨人草。这种草,极其粗壮巨大,一棵草堪比寻常的百年大树。

太丘上,到处长满了这样的巨人草。

巨大的草丛,连绵成一大片的原始丛林。

丛林里生活着大量的荒兽、上古荒兽,甚至还隐约有太古荒兽的影子。

所以即便蛊仙,也不敢擅自闯入。就算是八转蛊仙也要小心,一不留神动静大了点,就会在这里酿成兽潮。

东方长凡为了夺舍重生,就将传承布置在这里。而后方源等人,也在太丘上激烈的交战过。

“这里……你应该熟悉吧?”琅琊地灵望着方源笑笑。

方源道:“的确有点熟,因为东方长凡的事情。”说着,他望了琅琊地灵一眼,显然琅琊地灵也知道这段历史。

“我打算将传送蛊阵,布置在太丘当中。你觉得如何呢?”琅琊地灵问道。

方源沉吟道:“实话实说,我不建议我派在太丘上布置蛊阵。正是因我熟悉一点,才更加觉得不能这样干。这里就像一个蚂蜂窝,稍微动作大点,就能引发兽潮。这兽潮可不是简单的凡兽潮,至少荒兽,甚至是上古荒兽。引动太古荒兽出来,也有可能。”

“我之前在那里,有过一场激战。但那是因为东方长凡根据一具太古墟蝠的尸体,搭建了战斗的舞台。太古荒兽的气息,有效地阻挡了其他荒兽、上古荒兽。正是因为如此,才没有酿成兽潮。不过等到天灾地劫之后,太古荒兽的气息被冲刷干净,那具太古墟蝠的尸体。就立即被无数的荒兽、上古荒兽争相瓜分、吞食了。由此可见,太丘的可怕。”

“要在太丘布置传送蛊阵,若是在太丘边缘,虽然可以往外退走。但我派蛊仙也容易被其他势力发觉。若是在太丘内部布置蛊阵,一旦作战不利,我派蛊仙就逃脱不了。就算选取了一个上佳的地理位置,可以兼顾两者。但战斗稍有扩散波及,就会酿成兽潮。很容易就被人发觉了。”

“正是由于这些方面的愿意,才导致那些超级势力,都不想碰太丘这块地方。我派需要顾忌的,就更多了。这个地方,可不是一个好的选择啊。”

方源鞭辟入里的分析利弊,一副苦口婆心,忠心耿耿的样子。

琅琊地灵哈哈大笑。想看的书几乎都有啊,比一般的小说网站要稳定很多更新还快,全文字的没有广告。]

一边笑着,一边拍着方源的肩膀。

他笑了好一阵,这才道:“方源,你说得很好。不过你放心。你看,这是什么?”

琅琊地灵献宝似的,将一只信道蛊虫交到方源手中。

方源查看之后,装作震惊又大喜的模样,失声叫道:“这,这,这!居然有这样一幅详尽的太丘地形图。简直是瞌睡送枕头啊,啊,不对,我派的底蕴竟如此浑厚。若非我亲眼所见。恐怕不会相信!”

“嘎嘎嘎,好小子,见识到了吧。”琅琊地灵笑逐颜开,在原地蹦跳。手舞足蹈起来。

“厉害,厉害!”方源竖起大拇指,一脸诚挚和赞叹之色。

琅琊地灵兴奋地解释道:“这是我本体当年,为了在太丘中谋取仙材,屡次出动之后,结合自我经验。和他人的推算,得到的太丘地形图!”

“原来如此。”方源恍然大悟的样子。

其实他心中却是早已清楚。

这个情报,正是他和影宗交易所得的一小部分。

正是因为这份太丘地形图,才使得方源设计,故意说地沟,实际上就是将琅琊地灵往这方面牵引。

琅琊地灵手指着方源手中的信道蛊虫,再道:“这份地形图上不仅记录着太丘的各处地貌,而且还有相应的兽群分布。只是有一个巨大的缺陷,那就是这份地形图创作出来的时间实在太久了。经过这么多年,太丘那边一定有很多的变化。”

“太上大长老说的是,我心中也有这样的顾虑。”方源连忙附和。

琅琊地灵继续道:“你不是对那里熟悉吗?你又是人族蛊仙,现在又有了仙蛊变形,我打算让你跑一趟,带着这份地形图去。”

琅琊地灵还未说完,方源就直接站起身来,义无反顾,极其踊跃地道:“太上大长老只管下令,为本派效力,又关乎门派发展如此大事,这是在下的荣幸啊!”

琅琊地灵哈哈大笑,拍拍方源的肩膀,示意他重新坐下:“方源啊,你很好,我没有看错你。你放心,这一次你去,本派绝不会亏待你。你这次去,见机行事。主要是考察太丘地形,重新完善这份太丘地形图。若有机会,那就在合适的位置,布置初期的传送蛊阵。”

“你看这图中,有三个位置。这三个位置上,都标注有太古荒兽的尸体。你重点查看一下这三个位置,如果能在这里建立传送蛊阵,那是相当不错的。”

“当然,你要保全自身,行踪要隐秘,就算被人发现,也万万不能暴露了我派的存在。”

琅琊地灵细心叮嘱。

方源连忙点头,拍着胸脯保证:“太上大长老,你就放心吧。我稍作一些准备,即刻就动身!”

“好,很好。”琅琊地灵大笑,他对方源的表现十分满意。

其实也没有多少好准备的,琅琊福地底蕴深厚,方源将落魄谷、荡魂山还有智慧蛊,都放在这里,比较放心。

就算是有影宗内奸,也不妨碍这等大局。毛六能掀起的波澜很小,再加上影宗方面已经自顾不暇,不可能再对琅琊福地突袭。

至于天庭,就算发现了琅琊福地,又能怎样?

五域乱战时期,天庭攻击琅琊福地,也是攻了几次。最后,凤九歌牺牲,才攻下琅琊福地。更别提现在,五域界壁仍在的情况。

检查了一番仙蛊,方源选择将这些仙蛊几乎都带在身上。至于仙僵肉身,也随身携带。他的至尊仙窍,自然能带得起。

说起来,仙僵肉身是个麻烦,里面的春秋蝉更是麻烦中的麻烦。

春秋蝉现在已被莫名手段封印,无法汲取光阴长河中的水,只能被动挨饿,渐渐迈向毁灭的深渊。

不仅如此,春秋蝉中还藏有天意。

这点,方源已经从上一场交易中得知。

他还知道了不少关于天意的珍贵情报。

收获极大!

“我有至尊仙窍,这是区别独立于外界的小天地。但此次身怀春秋蝉,还有大量的天意雪怪,外出太丘必然有所阻碍。类似于我从南疆赶往北原的一路上,先后碰到云兽还有戚灾的追杀等等。”

方源心中了然。

他对天意不再是一无所知,反倒是知之甚详。

天意并不是无敌的,天意也有自身的局限。

了解天意之后,天意的威胁就下降了许多层次。

毕竟,未知的敌人才最麻烦。知己知彼,就能百战不殆!

第一次地灾,威力暴涨,方源知道,那就是天意要铲除自己的表现。但自己误打误撞,在北部冰原渡劫,这帮了自己一个大忙。

因为地灾中,有很大一部分,被狂蛮真意影响,排挤了天意,形成铁冠鹰、墟蝠等等灾厄。

而雪怪才是天意正统。

它们身上,灌输着纯粹的天意,势必要铲除方源,不除掉方源誓不罢休。

狂蛮真意的影响,很大程度上,削弱了雪怪的危害。

现在这些雪怪,还在方源的仙窍中。一个个都有天意,类似于春秋蝉。方源若要出去,在天意的感知下,就宛若黑暗中的火炬那般明显。

“天意不能亲自动手,除非是趁着灾劫的机会。我这次出去,天意一定会布局,影响他人来杀我。兵来将挡水来土掩便是!”

光芒一闪即逝,龙象原的中心地带,凭空出现了一位少年蛊仙。

他白衣如雪,温润如玉,黑发如瀑,垂至腰间,一双黑眸,深不见底。

正是古月方源。

“出来了!”方源深呼吸一口气,第一时间催动防御蛊虫,调动侦查蛊。

大量的侦查手段,一齐发动。

四下扫视,万步之内,在他的感应内,一览无余。

暂时是安全的。

琅琊福地自从上一次从月牙湖搬迁出来,具体落到哪里,只有琅琊地灵知道。

这是琅琊福地最大的秘密,地灵谁都没有告诉。

但是他却也在外面,布置了好几个传送蛊阵,方便内外的交流,蛊仙的出入。

这点应对,就比上一任的琅琊地灵高明数倍了。

琅琊福地的传送蛊阵,在风伯崖布置了一个。现在方源得知了第二个传送蛊阵的位置,就是这里,北原的龙象原。

“走,此地不宜久留。”

方源心念一起,蛊虫在仙窍中飞舞。

他身形似箭,一下子穿射上去,进入高空。

视野顿时开阔起来,俯瞰脚下,龙象原进入眼帘。

龙象原地势平缓,气概非凡。水草丰茂,一群群龙象群,有近的,有远的,四处散落。

这是是北原威名遐迩的龙象栖息地,受到周边的一个超级势力的掌控。

“大好河山,雄奇壮美。”方源赞叹一声,隐去身形,飞速离去。(未完待续。)

------------

看这里,领2016春节红包!

羊角挂双灯一灯吉祥一灯富,猿吭歌万曲千曲事业千曲情。(WWW.qiushu.CC 好看的小说

呼,刚给六个月大的儿子洗了个澡,热得一身汗。

哈哈,此刻窗外炮竹连天,新的一年即将到来,辞岁迎新之际,蛊真人祝愿广大的书友们,新年大吉,阖家欢乐!

在过去的这一年里,我真的十分感谢小伙伴们的鼎力支持。

2016年,让我们继续一起走下去!

今晚,在蛊真人微信公众号上,咱们也举办一个活动。

口令红包!

11点整,12点整,这两个时间点,有微信的同学就可以发送“红包”两个字,到蛊真人微信公众号,领取口令,得到支付宝红包。

详情请看今晚的蛊真人微信推送信息。

红包图的是吉利,让我们今年热热闹闹地进入一段崭新的人生岁月!

再次感谢大家!(未完待续。)

\end{this_body}

