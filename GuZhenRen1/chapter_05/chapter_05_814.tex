\newsection{气海老祖}    %第八百一十八节:气海老祖

\begin{this_body}

%1
梦境中。

%2
“师兄,我昨天做的饭菜可口吗?”绿蚁居士授课结束,师妹泰琴赶上前面的方源,期期艾艾地问道。

%3
方源笑了笑:“真的很不错呢,没想到师妹你如此心灵手巧,将来一定是一个贤惠的好妻子,不知道将来谁有这么好的福气能娶得你呢?”

%4
“师兄,你又打趣我。”泰琴脸上顿时升起两朵红云,接着她又用细微的声音道,“师兄你若是喜欢,师妹再给你做一些好了。”

%5
“行啊,那师兄我可就有口福喽。”方源朗笑着。

%6
这时,迎面走来一位龙人蛊仙:“我儿。”

%7
“爹,你怎么来了?”方源有些诧异。

%8
“啊,见过吴伯父。”泰琴慌张行礼。

%9
龙人蛊仙对少女点点头,微笑道:“你便是泰琴吧,我儿的家信中时常提到你,这是伯父给你的小小见面礼,收下吧。”

%10
“啊,这怎么好……”泰琴想要拒绝,但方源却是替她接过来,然后主动塞到她的手中。

%11
“你就收下吧,我爹来定是有着要事,所以今日课后的切磋,就不好进行了。”方源致歉道。

%12
“啊,不要紧,不要紧。”泰琴连忙摆手,当即道谢后,便很有礼貌地告退了。

%13
“是一个纯真的姑娘,我儿心动否?”看着泰琴身影消失在转角,龙人蛊仙打趣道。

%14
方源笑了笑,心想:梦境果然又起变化,口中谨慎地道:“爹你又怎会不了解孩儿呢?”

%15
“哈哈哈。”龙人蛊仙大笑一声,拍拍方源的肩膀,“我此次找你来,的确是有要事。书道阁主之女书九灵,年方二八,公开招亲,任何年轻俊彦皆能参与。我父想让你去参加。”

%16
“哦?”方源微微一讶。

%17
书九灵的事情,他还真就知道。

%18
就在前面一幕梦境中,泰琴就在闲谈中说过这个事情。

%19
书九灵本身和蛊仙范极相恋,怀了范极的孩子,范极却不愿和她结婚,始终推三阻四。

%20
书九灵看清范极的面目,痛不欲生,结果被母亲书道阁主软禁看管。

%21
书道阁主修为高达八转,并拥有仙蛊屋书道阁,乃是中洲散修当中的魁首。身为母亲将女人如此凄惨,自然心头大怒,想要找那负心汉范极算账。

%22
但范极却是黑天寺太上大长老空遗恨的关门弟子。空遗恨同样是八转蛊仙,没有任何子嗣,徒弟众多,但最宠爱的还是最小的关门弟子范极,疼若亲子。

%23
书道阁主不好随意拿捏,只好找上空遗恨,要求给个说法。

%24
空遗恨便带着爱徒范极,前往书道阁,进行协商。

%25
也不知道怎么谈的,双方不仅没有谈妥,反而动起手来。空遗恨和书道阁主试探了几招,便都收手。范极则和书九灵彻底闹分,书九灵一怒之下,威胁范极:“不要后悔,你不要我,天下有的人要我。”

%26
范极却是冷笑:“我此次来是和你好好协商,你却来得寸进尺。我倒要看看,谁有这样的胆子,敢来趟这个浑水。我还要看看,谁有这样的心胸,连你肚中的孩子都不在乎。”

%27
书九灵当场就气晕过去。

%28
身为母亲的书道阁主,当然要为女儿书九灵撑腰。

%29
书九灵苏醒之后,便执意要求公开招亲,书道阁主拗不过女儿,又觉得换一个人或许能给女儿幸福。就算女儿没有看中,招亲这件事情也能转移女儿的注意力,不至于让她始终郁郁,总想着轻生。

%30
泰琴说的时候,是一副看热闹的神情,眉飞色舞,把它当做一个笑话。

%31
方源当时还笑了,但现在却是笑不出来。

%32
他心想:“龙人一族一心想要独立、崛起,野心很大。眼下的确是一个良机,首先书道阁主乃是八转散仙,此次又和黑天寺对立。若是我能够成为书道阁主的女婿,就能凭借这层关系,拉拢她登上龙人一族的战车。”

%33
至于书九灵相貌如何、品性如何,以及她肚中的孩子,那都不是问题。

%34
方源绝不在乎。

%35
和心中的野望相比,这点的牺牲算得了什么?

%36
往往只有忍人之所不能忍,才能创造出常人难以开创的伟绩!

%37
“倒是师妹泰琴,对我情根渐渐深重,我若是这样做,未免有些对不起她。小丫头一定会很伤心的。”方源心头涌现起一抹愧疚之情。

%38
但旋即,他心头一阵悚然,惊醒过来:“不对,我这是怎么了?这里明明是梦境,我是方源,不是吴帅,怎么会心生愧疚?”

%39
“好厉害的梦境,居然勾动出了我的心中情怀!”

%40
“最近几幕的梦境,看似平稳过度,但其实是潜移默化,我居然被勾动了情绪。”

%41
“好险!”

%42
若是有冷汗,方源的额头定然已经是密布一层冷汗了。

%43
可惜这里是梦境。

%44
以往的情况下,方源探索梦境,都是肉身在外,魂魄深入,时常出来,因此现实、梦境区分得非常明显。

%45
但现在,方源这具龙人分身的肉身、魂魄都投入梦境之中,并且夜以继日地探索,沉溺于此。虽然现在方源还分得清梦境和现实,但情绪确实不知不觉间被勾动起来了。

%46
这是一个非常危险的征召。

%47
探索梦境,就害怕沉溺在梦境中,不能自拔。情绪就是最好的助攻,一旦被情绪困扰,纠结于各种情怀,方源就会陷入流沙当中,越陷越深,最终被这个梦境吞噬。

%48
龙人蛊仙仔细观察方源的神情,半晌,他满意地点点头:“我儿果然是没有被儿女情长所困,为父这就去拜见绿蚁居士,和他说明此事,替你请个长假。”

%49
“一切都谨遵父亲安排。”方源连忙道。

%50
光阴长河。

%51
轰轰轰!

%52
雷鸣般的巨响声中,一道道水浪激涌喷射。

%53
刷!

%54
一艘飞舟,陡然从这滔天的水浪当中,电射而出,宛若一道银色虹光。

%55
正是万年斗飞车。

%56
而在万年斗飞车的身后,则有五座仙蛊屋紧追不舍。

%57
万年斗飞车中,白凝冰等人各司其职,身心紧绷地作战。

%58
“这天庭果然是财大气粗,进入将今古亭、恒舟又搭建起来了。”

%59
“不仅如此,还有三秋黄鹤台、鲨流撬,以及那座神秘的仙蛊屋!”

%60
“天庭不愧是第一势力,底蕴深得恐怖。”

%61
白凝冰等人并不认得天庭的刹那台,为了建造这座宙道仙蛊屋,紫薇仙子甚至不惜将天庭中的百万天王画廊的核心仙蛊腾挪出来使用。

%62
其他四座仙蛊屋也就算了,关键是这座刹那台,同样是八转仙蛊屋。

%63
刹那台中还有八转宙道蛊仙顾六如。

%64
万年斗飞车能和刹那台对拼,但剩下的四座七转仙蛊屋也不是吃干饭的,因此敌我势力对比发生了转变,白凝冰一方只得采取游斗。

%65
“没有关系,我方拥有运道的优势。可以借助光阴长河的复杂环境,坑害天庭一方。”

%66
“不错,把他们削弱下去,我们就能伺机反攻了。”

%67
几天后。

%68
天庭。

%69
龙公起行在即,一脸苍白的秦鼎菱,还有紫薇仙子相送。

%70
秦鼎菱为了给龙公转运,状态极差,不能和龙公同行,也不能前往光阴长河镇守,只有留在天庭休养生息。

%71
紫薇仙子则要坐镇天庭,以防北原长生天可能出现的突袭。

%72
“光阴长河中仙蛊屋大战,已经持续了数天,方源始终未曾露面。不过有凤九歌的护道人气运,完全可以抵消了气运上的弱势,因此我方正处于上风。”

%73
“但方源不可小视,还要紫薇你多多留意啊。”

%74
龙公语重心长地关照道。

%75
“是。”紫薇仙子一脸肃穆,“方正也已准备好了,目前他正驾驭着仙鹤门的松鹤亭,已经在龙公大人此行的途中了。”

%76
“很好。”龙公点头,随即化作一道紫色龙形气劲,闪电般消失。

%77
方源再度观察分身气运。

%78
龙人分身的紫色小龙,又壮大了数倍,昂扬奋发,精神抖擞,在黑云中纵横驰骋。

%79
黑云中已经显现出大半的青紫之色,这一切都代表着方源分身成为龙宫之主的可能性大增。

%80
但是!

%81
在这黑色气运的头顶上空,一片巨大的血光气运,已经盖压下来,分外接近。

%82
血光气运当中,原本的金凤虚影已然消失不见,只剩下紫色龙影,却是越发庞大、狰狞。

%83
和这片血光气运相比,黑云气运和紫色小龙都显得虚弱渺小。

%84
“看来只有本体出手,才能阻挡住这样的变数。”方源已有觉悟。

%85
“嗯?来了。”下一刻,他眼中电芒迸射。

%86
只见太古白天中,一座仙蛊屋疾飞而来。正是仙鹤门的松鹤亭!

%87
方源深呼吸一口气,悍然掀动战场杀招。

%88
这一记气道战场,他早已酝酿,专门等候于此,阻截变故。

%89
“什么?!”松鹤亭中,方正大惊失色,一下子天地骤变,他成了瓮中之鳖。

%90
“有敌人埋伏!但我有松鹤亭,还好,还好……呃!”方正面现惊恐。

%91
只见无边的气流,宛若枪箭,数以万亿,朝着松鹤亭攒射而来。

%92
恐怖的威势让方正心脏都漏跳了一拍。

%93
“八转杀招!”方正面色瞬间惨白,松鹤亭只是六转的仙蛊屋,怎能挡住如此杀招?

%94
关键时刻,一道人影忽然出现。

%95
不是龙公,又是何人?

%96
原来龙公并未和方正汇合,只是暗中跟随,并嘱咐方正只身驾驭松鹤亭前往东海。

%97
这也是秦鼎菱的建议,给方正一定的自由,方便引出对于方源不利的变化。

%98
但方正一路前行,都是风平浪静,直到此刻陡然遭遇了强敌。

%99
没有龙公的出手,方正必死无疑。

%100
龙公双掌一推,附近的气流枪箭分崩离析。

%101
“谁?出来!”龙公大喝。

%102
“老夫气海老祖。”方源显露身形,却是换了一个面貌。

\end{this_body}

