\newsection{隔空交锋}    %第一百三十二节:隔空交锋

\begin{this_body}

%1
眼前是一片深蓝色的大幕,几乎占据了影无邪等人的全部视野。

%2
苍水界壁!

%3
五域之间,各有界壁,好像是一层膜,护卫着一方。

%4
苍水界壁,就是东海的保护膜。

%5
乱流海域本来就离着苍水界壁较近。正因如此,当初方源从南疆赶往北原的路上,就很接近乱流海域,碰到了刘青玉等人。

%6
影无邪等四仙运用上古战阵四通八达,直接传送到了这里。

%7
“又要离开了吗?”黑楼兰叹息。

%8
“刚刚那位是什么人?”太白云生疑问。

%9
“是敌人,而且是强敌、死敌!别看他只有六转修为的样子,实际上却有八转战力。”影无邪看了一眼太白云生,终究还是有选择性地说了一些情报。

%10
听闻此话,太白云生倒抽一口冷气,难以置信地低呼道:“八转战力!?”

%11
黑楼兰也装作震惊的样子:“难怪我们要直接撤退。”

%12
石奴始终面无表情。

%13
太白云生忽然想到了什么,脸上涌起悲伤之色:“唉!对方既然有八转战力,那么刘青玉恐怕是凶多吉少了。真是可惜了。”

%14
“能够为我影宗效死,也是他的荣幸。”石奴冷哼一声,不屑地道。

%15
“我们还是快走吧。对方既然能够追来,我们还是拉大距离为妙。那上极天鹰可是遨游九天的太古荒兽,极可能便是九天中的生命,在五域界壁中能进出自如。”黑楼兰暗暗传音,提醒影无邪。

%16
影无邪深深地看了她一眼,传音回道:“放心,我早已经查阅了北原方面的历来情报记录。方源的这头上极天鹰。来自黑凡。而黑凡培养出来的那头上极天鹰,却是北原物种,土生土长。因此即便进入五域界壁。也要遭受界壁的限制。我们来到界壁跟前,就已经算是安全了。”

%17
“是这样?”黑楼兰一愣。随后脸上的担忧缓解了许多的样子。

%18
“这一次我们要前往何处?中洲是不行的,难道是前往南疆,开始展开营救吗?”黑楼兰开口又问。

%19
影无邪笑了笑,他没有急着钻进苍水界壁,而是调动仙元,浑身蛊虫气息洋溢而出。

%20
他整个人悬停在高空中,缓缓转过身,看着乱流海域的方向。

%21
脚下的湛蓝潮水一*。哗哗作响。

%22
咸湿的海风,吹拂在四位蛊仙的脸上。

%23
“等。”影无邪说了一个字,算是回答黑楼兰的问题。

%24
“等什么?”太白云生疑惑。

%25
“等刚刚的那个敌人动手。”影无邪回答道。

%26
三位蛊仙面面相觑,都不明白的样子。

%27
既然刚刚直接撤退,为什么现在却要等待强敌?那个人可是拥有八转战力的,等他做什么?等他来杀死自己吗?

%28
黑楼兰忽然面色微变:“我知道了!”

%29
“知道什么?”太白云生忙问,他都被搞糊涂了。

%30
石奴虽然面无表情,但目光中也透着疑惑。

%31
黑楼兰笑了笑:“你们不觉得奇怪吗?对方为什么会知道我们的行踪?追赶我们的时候,定位如此精准呢?”

%32
这次,就连石奴都微微变色。

%33
太白云生更是声调都低沉下来:“黑楼兰。你的意思是说,我们这里出了内奸?不可能啊,我们四个怎么可能……难道说。那刘青玉是内奸?等等!他可是签订了盟约,才加入进来的。难道盟约有空子可钻不成?”

%34
“内奸的可能性极小。”影无邪摇摇头,接过话来。

%35
他看了一眼太白云生。

%36
就算是被蒙在鼓里的太白云生,影无邪也让他订下了一分盟约,不会让他成为自己这方的弱点漏洞。

%37
“最大的可能,就是对方掌握了侦查手段,能够追踪到我们的位置。”影无邪缓缓地道。

%38
“我们这些天,可是一直窝在营地中,极少外出的啊。”太白云生皱起眉头。

%39
黑楼兰笑了声:“蛊仙的手段。浩如烟海,诡异精绝。我们又能知晓多少?”

%40
“这倒也是。”太白云生点点头。

%41
黑楼兰又道:“如果我们不找出针对的方法,不管我们去往何方。那位大敌总会找上门来。区别只在于时间的早晚而已。”

%42
“所以,大人才暂停脚步,没有进入界壁。大人等对方动手,是指等对方动用侦查杀招啊。原来如此!”石奴恍然大悟。

%43
其实很多情报,只有影无邪知道。

%44
此刻,影无邪的脑海中,也是念头不断泛起,此起彼伏。

%45
他虽然是力道仙僵,思维并不敏捷,但是也有各种手段,来改变这个弊端。

%46
影无邪在迅速思考!

%47
“方源到底是用了什么手段,侦查到了我的方位?”

%48
“我来到东海,秘密无比。情报如何泄露出去?难道对方能在北原,隔着两个界壁,侦查到我的方位不成?”

%49
“这怎么不可能!就算是天庭蛊仙,我也有蛊阵护卫,遮掩天机,算不出什么东西。”

%50
“所以,就算黑凡真传再如何了得,恐怕也做不到这点。”

%51
天意!

%52
影无邪的脑海中,忽然浮现出一个答案。

%53
“我既然能够被天意布局,巧妙提点中,获知了方源的情报。那么方源那边,会不会也是如此?”

%54
这可能性相当的大。

%55
但影无邪并不满意。

%56
“若此时此刻,处于这样境地的人,不是我,而是方源。他会怎么想?怎么处理?”

%57
“他一定会考虑最糟糕的情形吧。”

%58
“那么最糟糕的情形是什么?”

%59
影无邪反问自己,然后给出答案“对方拥有侦查杀招,能够隔着界壁,就能侦查到我的位置。”

%60
影无邪笑了笑,这怎么可能?

%61
但忽然间,他笑容一滞。

%62
他猛地想到了一个可能!

%63
运道!

%64
方源的手中。始终掌握着一个最关键的线索。那就是他们彼此之间的运气,是相互勾连的。

%65
连运仙蛊。

%66
当初,影无邪走投无路。将这只仙蛊都卖进了宝黄天。

%67
和方源交易之后,影无邪获取了自己所需的炼蛊仙材。等到宝黄天重新开启之后。他便又将连运仙蛊拿回来。

%68
方源虽然拥有全新的肉身,但当初连运的效果,不仅针对肉身,还有魂魄。所以他之前,和黑楼兰等人的连运关系,仍旧存在。

%69
影无邪为了对抗天意,他深知影宗在这方面的宝贵经验,其中就有一条命难违。运可借。

%70
借助运气,可以有效地应付天意。

%71
谁的运气比较好?

%72
毫无疑问,那就是拥有狗屎运仙蛊的方源了。

%73
方源拥有****运的这个情报,影无邪也知道得很清楚。琅琊派中毛六的存在,不容忽视。

%74
于是,影无邪便利用连运,让自己的运气和黑楼兰勾连在一起。

%75
黑楼兰虽然不愿,但无法反抗。

%76
如此一来,影无邪和方源之间,也是货真价实的连运关系。

%77
“如果对方通过这层关系。动用运道手段,就算身处两域,感知对方。也是可行的了。”

%78
“会是运道吗?”

%79
就在这个时候,影无邪陡然身躯一震。

%80
他看到了!

%81
动用察运仙蛊,在这一刻,他看到了自己头顶上,和黑楼兰头顶的气运,都在发生细微的颤动。

%82
“果然……是运道的手段么。真是厉害啊,方源。不仅算到我连你的运,还利用这点,杀上门来。可惜。我已经瞧破了此点。”

%83
影无邪眼中精芒闪烁不定。

%84
虽然察觉到了方源的手段,但是影无邪却没有办法解除连运。

%85
他可没有断运仙蛊。

%86
不过。他的魂道手段相当丰富。影响、干涉气运的方法,也不在少数。

%87
当即一一尝试。结果却差强人意。

%88
气运的震动始终存在。

%89
而且,就算遮掩了自己这一边,别忘了还有黑楼兰呢。

%90
不像刘青玉,黑楼兰可不能轻易丢弃。

%91
她的利用价值很大,一旦舍弃了她,上古战阵四通八达就运用不了。所以影无邪很快打消了这个想法。

%92
“那么,钻入仙窍中试试看?”

%93
影无邪钻入石奴的仙窍中,结果他惊喜地发现,再结合了自己的魂道手段之后,他头顶上的气运,果然不再颤抖了。

%94
他便命令黑楼兰,也钻入石奴的仙窍当中。

%95
随后,太白云生和石奴联袂而行,钻入了苍水界壁中,两人的身影很快就消失不见。

%96
方源刚从青玉福地中出来不久。

%97
运道侦查杀招气运交感!

%98
他催动仙道杀招,一时间,在他心中感应到了几个点。

%99
东海、西漠、南疆都有。

%100
其中东海最多,也最为清晰。

%101
气运交感的侦查范围极大,但效果并不出色。若非连运的关系,方源不可能跨域侦查。

%102
“居然到了苍水界壁边上,这上古战阵的效果,居然这么强?”

%103
“哼!就算此战胜负难料,我也要追杀你。即便打杀不成,也要尽量削弱。削弱得你无法及时营救出幽魂本体,我就能奠定胜局。”

%104
“奇怪,他们为什么停下来?”

%105
“嗯?!”

%106
忽然,方源脸色微变,流露出一抹诧异的神色。

%107
在他的感应中,忽然间,两个点消失了。

%108
“哦?这么快就察觉到了吗?”方源一颗心直沉下去。

%109
如此一来,他就无法展开追杀了。

%110
“但只要你连运一日,你始终受到我这杀招的威胁。我占据主动,可以时不时地催动一下,来查看你们的位置。而你们却需要时时刻刻地防范,代价总会比我大。”

%111
“如此就形成对耗。而如果你断运,呵呵,这岂不正随我愿?”

%112
“也罢。还是先将乱流海域这边的好东西,都收入囊中吧。”

%113
ps:四月份比较忙,最近我会去一趟广州。所以月票加更,只能延迟到15号了。从15号开始逐天加更,满足我们之间的约定。

\end{this_body}

