\newsection{三道无上真传}    %第三十节:三道无上真传

\begin{this_body}



%1
不管是荡魂山,还是落魄谷,亦或者智慧蛊,都是无价之宝,天底下独一份,绝世无双。

%2
方源说可以交易,但琅琊地灵拿什么出来交易呢?

%3
方源并非拒绝,而是饱含期待。

%4
琅琊福地底蕴深厚,绝非普通存在。经历两位尊者,从中古时代屹立至今。王庭福地毁灭之后,更可以说是五域第一的福地!

%5
地灵生前的身份,更是号称古往今来炼道第一,隐隐超越之前的天难老怪、空绝老仙。

%6
这等人物,怎么可能拿不出东西来交易?

%7
就像仙蛊交易,每一只仙蛊,都是唯一的,举世无双。

%8
绝世无双之物,自然也可用同样绝世无双的物品,进行交换。

%9
其中价值的衡量,只在于具体双方对于自身处境的估量!

%10
荡魂山、落魄谷、智慧蛊,这三者在方源看来,也没有什么不能交易的,只要利益相符。

%11
琅琊地灵开始踱步。

%12
他眉头紧皱,陷入思考当中。

%13
如果是之前那位地灵,这根本不用考虑,肯定不会和方源交换。但转变之后,这位琅琊地灵却有与上任不同样的心思。

%14
“白毛地灵,是关于遁空蛊的执念所化,所以他只对炼蛊感兴趣,将所有的毛民蛊仙都培养成了炼道人才。而这位黑毛地灵,却一心想自身种族称王做霸,这点正是我可以大大利用的地方。”方源心中思量,看着琅琊地灵在眼前不断踱步,耐心的等待着对方的决定。

%15
方源心中萦绕着一份自信。

%16
他相信,琅琊地灵必不会让他失望。

%17
果然,不出方源所料,琅琊地灵思想挣扎了片刻。下定决心,要和方源交易。

%18
“太上大长老,我不怀疑你有资本和我交易,但交易向来你情我愿。若是你手头上的东西,打动不了我,恐怕……”方源故意说道。

%19
琅琊地灵一瞪眼睛。没好气地道:“放心!我琅琊派的珍藏,绝不比你那三样差到哪里去!”

%20
说着,他递给方源一只信道蛊虫。

%21
方源伸手,将这只蛊虫接过来。

%22
这只蛊虫,形如蚕蛹,虽分头、胸、腹三个体段,但通体如纺棰一般,有些浑圆可爱。它通体洁白,仿佛是瓷器。表面宛若上佳的釉质。方源将其握在手中,感觉光润丝滑,手感十分舒适。

%23
“这是书虫?”方源微微吃了一惊。

%24
书虫他是见过的,乃是一转蛊中的珍稀蛊。但叫他惊异的是,现在他手中的这只书虫,却明明散发着三转气息。

%25
琅琊地灵哼了一声,道:“这是我上任搞出来的玩意。他突破了书虫一转的桎梏,发展出了二转、三转的蛊方。你说他搞这些。有什么用?敌人打上门来,这些书虫能抵御吗?”

%26
话虽然这么说。方源还是从语气中,感受到琅琊地灵隐藏的那股微微得意之情。

%27
方源摇摇头,不再说话,而是调配仙窍中的无穷真元中抽出一丝,用来催动书虫,同时将一份心神灌注进去。

%28
书虫并非他所有。但琅琊地灵相借,自然让他催用起来,没有任何阻碍。

%29
方源双眼精芒骤盛。

%30
书虫中记载的内容,让他心神振奋,浑身气血都情不自禁地翻涌起来。

%31
“不出我所料!”方源心中赞叹。脑海中思绪起伏不定。

%32
方源赶回琅琊福地,又耗费许多时日疗伤,此时此刻,他的仙窍灾劫已经尽在眼前。

%33
什么影无邪、黑楼兰,或者狐仙福地、定仙游等等,都抛之一边。最大的难关,就是如何渡过这场灾劫!

%34
蛊仙仙窍每隔一段时间,就有灾劫降临。仙窍底蕴越深厚,灾劫威能就越强。

%35
方源现在的仙窍,还来不及经营,一片空空荡荡,但本身空间广阔到恐怖,宙道资源更极其丰厚,且又不禁五域界壁。如此仙窍,已经超越十绝总和,可想而知,面临的灾劫将史无前例的危险!

%36
可以说,这是悬在方源脖颈上方的一面雪亮斩刃。

%37
在之前疗伤的时候,方源一面调教布局方正,一面就在思考目前处境。

%38
距离这次灾劫,已经没有几天。方源却殊无把握,心中没底。

%39
他大部分的仙蛊,都留在肉身中,自毁了不少,还有一些残留在影无邪手中,这都使得方源实力大降。

%40
太白云生、黑楼兰也不知所踪,黎山仙子、焚天魔女已然身死,唯一可以借力的地方,就是琅琊福地。

%41
而琅琊福地,屹立至今,必然渡过了不少灾劫,却是仙窍渡劫的成功楷模,必定有许多值得称道,能让方源学习求教的长处。

%42
“怎么样?”琅琊地灵笑道。

%43
他向方源竖起三只手指:“我琅琊派有三大底蕴。第一,是我本体留下的炼道真传,涵盖一生修行精髓。第二,是巨阳仙尊留下的一份运道真传。第三,是盗天魔尊的偷道真传。我本体在时,为两位尊者炼蛊,要求的报酬就是各自一份真传。”

%44
方源微微点头。

%45
书虫中记载的内容,正是关于这三份真传的。

%46
不论哪一份,都是博大精深,方源只是浏览到一些肤浅的表面信息,就被深深的吸引住,可以说俱都价值非凡。

%47
琅琊地灵继续道:“这三份真传当中,盗天真传稍次。那是因为,本体虽然炼出遁空蛊,却无法催动,主动退了一部分报酬。不过,盗天魔尊的真传,都是两两相对的。你若运气好,得到这份真传,而相对应的那份真传没有被人继承的话,那么你还能因此得到第二份盗天真传的线索呢。”

%48
“盗天魔尊一共留下八道真传,但必须是天外之魔才能获得,惟独我这一份例外。方源你恰恰是天外之魔,若是获得了另外一份盗天真传,就可以轻松继承。”

%49
方源微微点头。

%50
他亲眼见识过无相手的厉害。有了偷道手段,他甚至可以盗取他人的仙蛊,为每次战斗保留最大的战利品。

%51
而他赖以生存的见面不相识,见面似相识,见面曾相识,也是盗天魔尊本人的仙道杀招。

%52
偷盗的妙用。方源深有体会。得到这份真传,绝不会让他失望。

%53
顿了一顿,琅琊地灵继续道:“盗天魔尊有八道真传,巨阳仙尊却只留下三道。我本体生前,为他成功的炼出仙蛊屋八十八角真阳楼,所以这份真传的价值,要比盗天真传高出一些。”

%54
“巨阳仙尊的三道真传,囊括了他对运道的一切手段,分别为己运、众生运、天地运。其中己运这份真传。就在我的手上。众生运真传,藏在王庭福地中。天地运那份真传,则在长生洞天之中。”

%55
己运、众生运、天地运。

%56
连运仙蛊、断运仙蛊、排难仙蛊,就是众生运真传中的一部分。王庭福地被方源捣毁之后,这些仙蛊一部分流落在外。至于修行的具体内容,已成绝响。

%57
方源和马鸿运打过交道,知道运道的厉害。

%58
“别的不说,若我有排难仙蛊。那我便可直接学习王庭福地,将灾劫排到外界去了。”方源心知运道对自己渡劫会有奇效。

%59
就算这份己运真传中。没有类似排难仙蛊的存在。方源本身的运气越佳,他面临的灾劫威力也会随之减弱,更容易应对。

%60
“方源,我建议你选择己运真传。因为你的运气,似乎很坏。”琅琊地灵开口道。

%61
“这一次,你从南疆赶回来。一路上出现了多少意外和变故?就连我这边,都损失了两位毛民蛊仙。起因你恐怕都猜不到!”

%62
“雪胡老祖要炼制鸿运齐天蛊,勒令大雪山福地中的魔道蛊仙,为他收集蛊材。结果此举大失人心,事到临头的时候。大雪山中蛊仙反叛,将其中一份关键蛊材偷走。”

%63
“雪胡老祖大怒,派遣蛊仙捉拿一切可疑人员。我派去接应你的那位毛民蛊仙,就是遭了池鱼之殃啊。”

%64
“至于第二位神秘失踪的蛊仙,他究竟遭遇了什么,我到现在都还没调查出来!”

%65
“我相信,真相总有一天会水落石出。他们绝不会白白牺牲的!”方源劝慰道。

%66
琅琊地灵磨了磨牙,没有纠结这个方面,而是继续刚刚的话题:“其实照我看,最适合你的还是第一份真传。炼道分有两大流派,你知道吗?”

%67
“略有耳闻。”方源答。

%68
琅琊地灵缓缓地道:“两大流派,分别是毛民天地流,人族隔绝流。我们毛民炼蛊,和你们人族不同,能够利用天地间的道痕,营造出炼蛊的最佳环境,从而增大炼蛊的成功可能。”

%69
方源耐心地听着,琅琊地灵说这话,必有他的原因。

%70
“曾经,巨阳仙尊、盗天魔尊之所以求助我的本体,正是因为本体走的毛民天地流的炼蛊法,炼制仙蛊的成功性,要大大高于你们人族隔绝流派。”

%71
“本体不仅掌握毛民天地流的精髓,更参照了空绝老仙的一份真传,超脱原本的桎梏,更上一层楼,构思出一道仙级杀招,名为仙劫锻窍!”

%72
“仙劫锻窍?”方源眼中精芒一闪即逝,心中大起兴趣。

%73
琅琊地灵带着一脸骄傲和得意,继续介绍。

%74
原来这仙级锻窍杀招,是采用数只仙蛊,十多万只凡蛊辅助,酝酿而成的炼道杀招。

%75
它构思极其巧妙,竟是将福地洞天当做炼制的本体,通过杀招,和仙窍之外的五域天地勾连在一起,从而影响灾劫,并利用灾劫锻炼仙窍本身。

%76
“仙窍灾劫千奇百怪,令人难以应付。但运用这个杀招,往往就能限定住灾劫类属。你知道之前,琅琊福地为什么要寄托在月牙湖那里吗?这是因为,月牙湖附近,充斥着浓郁的水道道痕,以及炼道道痕。”

%77
“我的上任,每次都运用仙劫锻窍杀招,应付灾劫。通常,都会形成和水道、炼道有关的灾厄。渡过之后,仙窍中就会增添水道、炼道的道痕了。”

%78
“十八万年前,琅琊福地中可没有什么海洋。但现在你看,汪洋一片,原本的广袤大地,只剩下三块大陆了。这就是水道道痕充裕的象征!”

%79
方源闻言,不由想起魔尊幽魂利用万劫,炼制至尊仙窍蛊的那一幕。

%80
“这仙劫锻窍杀招,却是和魔尊幽魂的手法,有着异曲同工之妙。不,影宗方面曾在琅琊福地中布置过卧底,或许魔尊幽魂的手法,正是来源于杀招仙劫锻窍呢?”

%81
方源猜的很对。

%82
魔尊幽魂的手法,正是脱胎于仙劫锻窍杀招。但他的手法更加完善,借助智道推算出了灾劫具体是什么,再加以着手准备。

%83
不像这仙劫锻窍杀招,只能影响灾劫,有失败的可能。琅琊福地渡过的灾劫中,也有炎道、雷道等等灾劫,不全是水道、炼道之灾。

%84
“好了,你有落魄谷、荡魂山、智慧蛊,我也有三道真传,一样换一样,你想怎么交易?”琅琊地灵最后问道。

%85
这次,轮到方源陷入沉思之中。

%86
他究竟该如何抉择?

\end{this_body}

