\newsection{灵}    %第八百五十二节:灵

\begin{this_body}

%1
“元莲仙尊、乐土仙尊还有狂蛮魔尊?”方源心头一动。

%2
按照这些人形云雾土著的描述,方源很快就将这三位圣人和历史上的三位尊者对照起来。

%3
方源脑海中各种念头如闪电般闪烁:“这三位尊者都来过这里?那么之前的第七层,我还看到了幽魂魔尊、乐土仙尊、巨阳仙尊留下来的道痕之路……这么说来,至少有五位尊者来过这里了。”

%4
方源有些恍然,更有些迷惑。

%5
他在第七层时,看到三位尊者的路时,当时就有疑惑:真的只有这三位尊者来过这里吗?如果有其他的尊者来过,那么为什么没有留下他们的道路?

%6
“现在看来,其他的尊者虽然也来过这里,比如狂蛮,比如元莲,但即便当时留下了道路,也因为时间太过长远,而被摧残毁灭了吧。”

%7
无极魔尊是上古时代,一百万年前,继他之后,又出现了狂蛮、红莲。

%8
之后是中古时代,距今三十万年前。相继出现了元莲、盗天、巨阳。这个时代中,还出现了长毛老祖。他因为是毛民的缘故,活得很悠久,和盗天、巨阳两位尊者都合作过。

%9
中古时代之后,便是近古时代,距今十万年前,有幽魂、乐土两位尊者。

%10
然后就是现在,根据三尊说的预言,会出现一位大梦仙尊,她是前所未有的尊者,将会超越历史上的十大尊者先贤。

%11
巨阳、幽魂、乐土在第七层中,留下了他们的路,其实也就是他们闯荡过来的痕迹。

%12
他们是距今最近的三位尊者,所以道痕能存留下来。

%13
而之前的尊者,就算留下了道痕之路,也因为时间太长,被剿磨灭尽了。

%14
方源的这个疑惑算是解决了,但是又有更多的疑问产生。

%15
“这些尊者为什么接踵而至呢?”

%16
“疯魔窟太吸引这些人了。”

%17
“这些尊者到达了这里,究竟有没有去过第九层?”

%18
“第九层的大阵应当还在运转中,那么如果这些尊者去过第九层,为什么不将大阵毁了,拿走里面的九转衍化蛊呢?”

%19
“如果他们没有去成,就止步于第八层,为什么又要建立什么道场,创建什么世界呢?”

%20
方源思索了一番,当即决定要去这些尊者的世界看一看。

%21
他根本找不到第九层的路。

%22
在第七层时,他从未料到过,第八层居然是如此恢弘气象。

%23
或许在这些尊者的道场和世界中,他能够找到相应的线索呢?

%24
就算没有相应的线索,即便这些后来的尊者都没有进入过第九层,那么这些世界中也很有可能留下他们的传承啊!

%25
即便没有什么真传,遗落个什么东西,尊者的只鳞片爪也对方源大有利益啊。

%26
“这就有意思了。你们知道这些道场、世界在哪里吗?”方源询问道。

%27
这数位人形云雾尽皆摇头,纷纷表示不知道,只是听闻过。

%28
“圣人啊,我们在这个世界中生存,没有能力闯出去。外面太过危险致命,不是我们能涉足的。”

%29
方源皱眉:“那你们又是如何得知这些消息的呢?”

%30
人形云雾便答道:“每隔一段时间,就会有圣人的门徒、弟子,穿梭虚无,来往世界,进行交易或者劫掠。我们的这些消息都是从他们手中得来的。”

%31
方源眼眸微缩:“这些门徒弟子,都是出自各代尊者,战力如何?能够穿梭虚无,穿梭各个世界,实力应当是不俗的!”

%32
但旋即,方源又意识到不妥之处,再次问道:“既然这些门徒、弟子有很多,那你们为何第一眼见到我,却是称呼我为圣人呢?”

%33
人形云雾们便答:“那是因为所有的圣人,都是您这番模样啊,而他们的门徒、弟子却是各种姿态。”

%34
方源心中顿时了然:“看样子,似乎这里没有人族寄居,所谓的圣人门徒、弟子恐怕都是一些异类?就像是眼前的这几个人形云雾?”

%35
说起来,这些人形云雾他还是第一次见到,既不是野兽,也不是植株,更不是什么异人,但偏偏是人的模糊形态。

%36
“还请圣人救我!”

%37
“请圣人布道,拯救这片世界吧。”

%38
数位人形云雾又再次大礼叩拜。

%39
“好说,好受。”方源笑了笑,却是悍然出手。

%40
“圣人!”

%41
“你要做什么?”

%42
“啊——!”

%43
惊呼声、惨叫声一时间响彻方源耳畔。

%44
仅仅只是两个呼吸,方源就将这些人形云雾拘拿住,他看着这些人形云雾,眼中闪烁着慑人的冷光。

%45
单单依靠询问,这样的情报谈何可靠?

%46
人形云雾若是想要欺骗方源,这就太容易操作了。

%47
方源谈话之间,已经将周围和对方暗中侦查得七七八八,此刻出手立即将他们拿下。

%48
然后,他就催动搜魂的手段,他要自己查看。

%49
结果却是有些尴尬。

%50
这些人形云雾居然没有魂魄!

%51
没有魂魄还算是生灵吗?

%52
就算是植株也有自己的魂魄呀。

%53
方源楞了一下后又恍然:“看来这片小世界中没有魂道道痕,这就导致了这里的生灵也不存在什么魂魄。”

%54
这倒是一个挺新奇的发现。

%55
在五域两天中,因为有魂道道痕,所有但凡生命都有自己的魂魄。

%56
这些人形云雾却是没有的。

%57
要搜刮出他们的记忆,还有一些麻烦。

%58
不过这些麻烦,很快就被方源克服。

%59
方源得到了对方掌握的一切情报。

%60
从这些情报中来看,之前人形云雾们根本没有欺骗方源。他们的生存经历很是单调,自从有了灵智之后,就一直宅在这里,直到这个世界快要毁灭。就算是期间有过圣人门徒穿梭,进来交易,次数也只有一次,时间也是很短,因此这些人形云雾的性情也颇为单纯。

%61
方源还发现,这些人形云雾其实并非个体,仍旧是这个世界的一部分。它们根本就不能脱离这个世界,去往外界。

%62
“这些人形云雾其实和地灵、天灵很相像啊。”方源的脑海中忽然又有一道灵光闪过。

%63
地灵、天灵都是蛊仙执念结合天地伟力成形,也没有魂魄,但有灵智,可以交流。

%64
事实上,不只是地灵、天灵,方源还有阵灵、龙灵。

%65
阵灵是从阵灵蛊中催生出来的,而龙灵则是龙人一族的秘法。

%66
“这就是说,这几个人形云雾,就是地灵、天灵的雏形?我若能弄懂它们的奥妙,不就可以批量制造地灵、天灵了?”

%67
方源眼中放光,一下子就看到了最有价值的东西。

%68
然而,半天后,他却一脸遗憾地看着这片世界毁灭。

%69
他很想参悟出这些人形云雾的产生奥秘,但是他没有足够的世界。刚刚研究有一点点的起色,这个世界就被摧毁了。

%70
“没有关系,我还可以继续寻找。”

%71
“这样的世界虽然稀少,但终归还是有的。”

%72
“若是能让我碰见那些尊者的道场,那就更好不过了!”

%73
于是接下来,方源便在虚无中穿梭,四处搜寻有价值的目标。

%74
噗!

%75
胖山吐出一口鲜血,猛地倒退一步。

%76
他身形巨大,这后退的一步重重地踏在地上,立即发出咚的一声巨响。换做正常的土地,必定会深陷出一个坑洞,但在这里第七层中,每一寸土地都是准九转的仙材,坚固得很。

%77
“没有查探到方源的任何踪迹。”胖山擦了擦嘴边的血迹,道。

%78
不是仙沉默。

%79
秘谋人则叹息一声。

%80
疯魔三怪这个时候倒是集齐了,之前是为了防范方源,但现在当魔音结束之后,他们就立即结伴下来。

%81
一来,是觉得方源就算活下来,也是苟延残喘,绝不会能对三怪造成什么威胁了。

%82
二来,也是想尽自己最大的努力,来拯救方源。

%83
蛊仙受伤,往往比较麻烦,不是那么容易痊愈的。

%84
不是仙、秘谋人都身上带伤,只有胖山完好无恙,但现在他也受了伤。

%85
在这个第七层中催动侦查杀招,可不是什么好主意。

%86
因为这里的道痕太过密集,几乎相当于一个战场杀招,道痕互斥,在这里它对所有的流派都不友好。

%87
胖山的侦查范围十分有限,因为符合他流派的道痕,和整个道痕相比,永远只是其中的一小部分。

%88
“魔音起来,第七层什么样子我是清楚的。二位,依我看,就这样吧。方源生还的希望……”胖山缓缓摇头。

%89
“没想到方源就这样没了。”不是仙仰头长叹。

%90
秘谋人也是极其惋惜遗憾的。

%91
他们倒没有什么为民除害的心思和想法,他们的人生目标很一致,很简单,就是要探索到第九层,掌握衍化大阵,从而寻觅到永生的某种可能。

%92
方源本体在第八层中久久不出,他的分身则各有紧张。

%93
房家大本营。

%94
豆神宫悬浮空中,绽射出冲天的碧光。

%95
此时宫外,房家的太上大长老房功,太上三长老房化生,以及房棱、房云都在远远观看。

%96
碧光猛地一变,化作无数雷电光球,向周围轰炸。

%97
轰隆隆……

%98
一连串的爆响声中,土石翻飞,气流狂飙。

%99
轰炸结束后,坑洼不平的地面上猛地长满了绿草,同时一棵棵的小树苗,以肉眼可见的速度,钻破土壤,长成参天大树。

%100
这正是豆神宫的杀招——万生春雷!

%101
这个杀招催动自如,证明了房睇长基本上掌控了豆神宫。

%102
宫殿打开,方源分身房睇长微笑走出。

%103
“好!”房功大笑,“二长老,你可是立下大功了。”虽然还是对房家内部的派系平衡有些担忧,但对于整个房家无疑是重大的好消息。

%104
“有了此屋,我们房家就能突破困境,真正崛起了!”房云兴高采烈地欢呼起来。

%105
房睇长笑了笑:“有了此屋,我们已经是立于不败之地。但还不忙出手,豆神宫还有可以提升的地方。最重要的一招豆神兵卒,就需要提前准备。”

%106
“这无妨,需要什么木道仙材,尽管从我房家库藏中提取便是。”房功再度大笑,发自内心的欢喜。

%107
ps:不知不觉间已经过了这么久啊,看到书评才知道,今天是这本书的生日,整整5年了呀。真是忏愧,又有些欣慰。我写的慢,但还在写。还是有朋友们愿意看,愿意等。真的衷心感谢你们的支持!走到这一步,真的不容易,我们彼此都不容易,但我们还要继续走下去。不管《蛊真人》这本书什么时候完结,当我们回想起来的时候,那绝不仅仅是方源,或者只是一本书,那是我们各自人生中五六年的青春时光啊。岁月荏苒,我们各自模样都有变化,没有变化的是我们每个人心中的方源,是我们每个人心中的执着和坚持。

\end{this_body}

