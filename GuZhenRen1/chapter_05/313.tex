\newsection{掠影}    %第三百一十三节:掠影

\begin{this_body}

一场宾主尽欢的酒宴,并不是结局,而只是一场开始。\&\#26825;\&\#33457;\&\#31958;\&\#23567;\&\#35828;\&\#32593;\&\#119;\&\#119;\&\#119;\&\#46;\&\#109;\&\#105;\&\#97;\&\#110;\&\#104;\&\#117;\&\#97;\&\#116;\&\#97;\&\#110;\&\#103;\&\#46;\&\#99;\&\#99;]

方源和池伤成为了好友。

在接下来的一段时间里,方源频繁地邀请池伤,而池伤也经常来到武家驻地,造访方源。

想要接近一个人,和某个人搞好关系,说难挺难,说容易也很容易。

无非是屈尊降贵,顺应他心这几个字。

方源老谋深算,又有态度蛊在手,池伤怎可能是他的对手?

池伤号称“阵痴”,寻常蛊仙都不入他的眼界。但偏偏方源已经是阵道的准宗师,和他这位阵道宗师,大有共鸣之处。

双方就阵道的种种内容,进行攀谈和交流。

池伤很快就惊叹于方源的阵道造诣,他对此表示疑惑。因为方源明明是主修变化道。

方源告诉他:“我一直对阵道保持浓厚兴趣,凡人时期更是钻研阵道,因为一些机缘,主修阵道。可是后来,命运弄人,只得以变化道成就了蛊仙。不过我已经很幸运了,能够成就蛊仙的散修,能够有多少呢?”

这份无奈,让池伤颇为感慨,内心深处生出一股同情。

“你若是改修阵道,我可以为你提供帮助。池家的阵道仙蛊并不稀少,我可以为兄台你交涉,达成仙蛊交易。”池伤主动提出。

他能提出这一点,相当不容易。足以证明方源这些天来的努力成果。

要改修流派,不是不可以,但是本命蛊需要调换。

方源非常感动的样子:“如果我将来要改修,一定寻求池伤你的帮助。”

池伤点点头:“到时候,一定要告诉我!”

他也知道,让一位蛊仙改修流派,并不容易,不是说做就能做的,这是大事,牵扯相当广泛。往往还对仙窍有一时的恶劣影响。

和池伤的这番交流,带给方源相当大的收获。

虽然不是境界上的提升,但是对于方源的眼界扩张,阵道的一些其他基础补充,有着巨大的有利帮助。[求书网www.Qiushu.cc想看的书几乎都有啊,比一般的小说网站要稳定很多更新还快,全文字的没有广告。]

方源对阵道,有了更加全面和深刻的了解。

蛊阵和杀招之间,有什么区别?

池伤告诉方源:“蛊阵就是杀招的一种。杀招是多只蛊虫一同催用,蛊阵同样如此。蛊阵是属于阵道的杀招,每一个不同的蛊阵,就是一个不同的杀招。所以说,阵道是所有流派当中,杀招最多的流派。当然蛊阵和其他杀招之间,也有细微的分别。比如说蛊阵持续的时间,往往更长。蛊阵布置之后,牵扯蛊师的心神、精力,往往更少……”

“阵道的本质是什么?”方源问池伤。

池伤摇摇头:“我虽然是阵道宗师,但谈论阵道本质,还是境界不足。不过我族的池曲由大人曾经对我讲过,阵道的精义便是营造环境。”

方源听到这样的答案,顿时有一种振聋发聩的感觉。

盘丝洞窟的蛊阵,不就是为了更多培养出长恨蛛,营造出来的一种生存环境吗?

“据说阵道大宗师,能够依据天地间的自然道痕,运用很少的一些蛊虫,就能布置出蛊阵。甚至能在一些自然道痕浓郁的地方,只用凡蛊,就能形成仙道蛊阵。是这样的吗?”方源又请教池伤。

池伤点头,坦怀解惑:“的确是这样。但其实,我们这种程度,也能运用道痕。有一些蛊阵,布阵的基石不仅是蛊虫,还有蛊材。不过我们只能是运用仙材中的道痕,进行布阵。时间久了,这些仙材就会慢慢损毁。阵道大宗师更进一步,不仅是仙材凡材,还能因地制宜,利用自然天地中的道痕。”

……

这些对话和释意,带给方源相当大的启发。

这让他更加热情地招待池伤。

方源很快发现,举办酒宴拉近关系的效果并不大,还不如和池伤来一场阵道方面的交流。

有时候,一场激烈争辩的讨论,更能激发出池伤心中英雄相惜的情怀。

方源还有一个大杀器,就是给乔丝柳写信。

每一次写信,他都重点夸赞池伤,赞叹他阵道上的惊人造诣,还有另人惊叹的才情天赋。

池伤被夸都不好意思了,更对方源刮目相看,觉得他是一个正人君子,行事坦荡,胸襟宽广。

投桃报李,他也是在写给乔丝柳的信中,夸赞方源如何如何,对方源主修变化道,竟然兼修阵道,并且造诣如此深厚,表示吃惊和佩服。

乔丝柳:“……”

她望着一封封的来信,感受到自己的这两个追求者,居然夸赞对方,几乎把她这个正主儿都给忘了!

她万万没有料想到会是这种情况,心中大翻白眼的同时,还得在回信中表示赞赏,尤其是对两人的广博胸襟。

毫无疑问,方源和池伤转敌为友,这个变化让想要看热闹不嫌事大的人们,都跌碎了眼镜。

南疆,掠影地沟。

这是一处南疆有名的地沟,因为暗道猛兽层出不穷,尤其是影怪数量众多而闻名南疆。

同样的,这也是南疆名列前茅的凶险之地。

如今,在这掠影地沟的深处,某个不起眼的地洞当中。

“哦嗷嗷……”一个满头紫发的老疯子,破衣烂衫,光着脚丫,在洞中不断的嚎叫,同时撒丫子乱跑。

扑通。

忽然,他一下子扑倒在地,然后浑身扭曲,学习蚯蚓走路。

走了一段之后,他猛地站起来,满脸傻笑。

不过片刻之后,他的傻笑声渐渐收敛起来,他浑浊的眼珠子里又重新显现出清明的光彩来。

“紫大人,您终于清醒了。”影无邪出现在洞口处,他叹了一口气,面色复杂。

原来这个发疯的老人,赫然便是曾经的紫。

影宗一行四人,自从北原逆流河一战后,销声匿迹,也不知是什么时候回到了南疆。又为什么来到这处掠影地沟的深处。

紫也深深地叹了一口气,他用手拍了拍自己身上的泥土,与此同时身形缩小,变回背生双翼的小人模样。

“还是叫我紫山真君吧。这将有利于我们接下来的行动。”紫山真君道,“这次带来什么消息?”

影无邪便答:“依照我们的计划,南疆政局正在朝着我们的设计方向而慢慢发展。只是可惜,前不久池伤和武遗海之间,并没有造成剧烈的冲突,他们两个反而成为了好友。”

“哦?”紫山真君微微讶异了一下。

详细查看了情报之后,他点点头,“这个武遗海有点意思,主修变化道不说,阵道方面的造诣居然能够得到池伤的认可。”

“这无疑帮助了他,不过他还是斗不过武庸,被武庸第二次发配到超级蛊阵中去。”

影无邪点点头:“武庸毕竟是八转蛊仙,如今更是大权在握,武遗海认祖归宗不久,还和乔家没有达成共识,不能得到乔家的全力帮助。”

“不过这更证明:武遗海此人虽是散修,但野心很大。因为他主动接近乔丝柳,却迟迟没有结合一起,只能说明他和乔家并未谈妥利益分配。他的野心,咱们是不是可以利用呢?”

“嗯……”紫山真君沉吟了一下,“这一点或许可以利用,但是我们对此人的了解还是太少。目前为止,还是要设计武家,将主要精力集中在武庸身上。武家乃至南疆第一势力,是正道的顶梁柱,它若倒下去,南疆政局必定会掀起一番巨大的动荡。到那时,我们趁势而起,趁虚而入,才有打破超级蛊阵,救出本体的希望。”

原来,武家动荡,并不是那么单纯的,居然有影宗的势力在幕后推波助澜!

“哼!南疆政局就算再是震荡,又能如何?”白凝冰也出现在洞口处,她脸色冷漠,“就凭我们这四人,就算超级蛊阵只有同样四位蛊仙防守,也是一个大麻烦。更何况我们这边的主要战力,根本就不稳定。”

白凝冰说话,非常不客气。

即便是面对八转存在的紫山真君。

但她有恃无恐,因为她在被方源追杀之前,就已经和影无邪等人,重新定下了盟约,双方地位平等。

而且紫山真君时不时的疯癫,也让她心中的敬畏之情,大为削减。

紫山真君不以为杵,反而笑笑:“光是让政局动荡,当然不行,毕竟这些正道势力绝不痴傻。此举只是尽量地削减他们驻守在超级蛊阵中的力量。光靠我们四位,当然也不行,战力单薄。所以接下来,我们要招收人手。”

“招收人手?”影无邪诧异,“难道南疆这里,还有我不知道的暗手?”

因为他知道,南疆这边影宗势力被打击得非常严重。而要冲击超级蛊阵的话,非得有他们一级的实力,普通的六转蛊仙根本就不够看。

但如果有这样的人物,影无邪早在方源追杀之前,肯定就及早地招收进来了。

追溯更久之前,若是有这样的人物,魔尊幽魂逆天渡劫,炼制至尊仙胎蛊的时候,也早就将其安排,与中洲蛊仙放对。

紫山真君点点头,又摇摇头:“你们可知,我因何而疯癫呆傻?”

备注:9点半第二更。(未完待续。)<!--80txt.com-ouoou-->

\end{this_body}

