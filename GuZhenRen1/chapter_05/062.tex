\newsection{流光斑斓养仙蛊}    %第六十二节:流光斑斓养仙蛊

\begin{this_body}

方源的大部分心神,此刻沉入仙窍。∽↗,

大量的光道凡蛊,一只只飞舞盘旋,然后带动一颗颗的流光果,飞上高空。

穿过一层又一层的“天”,经过漫长的飞行,这些流光果终于来到了最上面的一层。

方源的至尊仙窍,分五域九天。

思考之后,方源将所有的流光果,都放入到最上面一层中去。

“仙窍中的宇道道痕虽然不少,但是我却不能像地灵那样,直接运用宇道道痕,进行瞬移。毕竟蛊仙和地灵之间,有很大的区别。地灵可是天地伟力,结合蛊仙执念形成的独特存在。”

“至尊仙窍地域宽广,看来我今后还得在里面布置传送蛊阵,方便迁移资源,调整结构,安排布置。”

方源思绪随意散漫。

现在,所有的流光果都进入了最上一层的天空。

方源便调动蛊虫,带动一颗颗的流光果,飞到指定的位置。

噗噗噗……

下一刻,轻微的炸响不绝于耳。

一颗颗的流光果,接连自爆。

每一颗的流光果,体积不大,但是自爆开来,却迅速花纹漫天的极光,五光十色,美不胜收。

这些流光果的本质,就是极光的浓缩和凝聚。

方源之所以收集这些流光果,就是为了喂养态度蛊。但是为什么还要将这些流光果,自爆了呢?

原来,这一批流光果的数量,虽然不小。但是喂养态度蛊还是远远不够的。

态度蛊可是八转仙蛊,虽然喂养的时间间隔很长。但是每一次喂养,需要耗费的流光果的数目却是十分巨大。

这是蛊虫喂养的常态。

一般而言。蛊虫转数越高,喂养的食料就越多,但同时喂养的时间间隔也就越长。

蛊虫一共有九转,八转仅次于九转,态度蛊的食料是一个相当庞大的数目。

所以,方源收集这些流光果,是为了以果蕴果。

接下来,他将这些果实中的八成,都自爆了。留下两成果实。任由它们在高空悬浮,在漫天的极光中温养。

至于带着它们一路飞来的光道凡蛊,方源统统都收起来,留待后用。

“本来我的门派贡献不多,但没想到一番建议,却让琅琊地灵主动为了增添了门派贡献。将这些门派贡献都用来换取流光果,一下子就到了满意的数目。”

“接下来就是引进大量的凡蛊炫光蛊,来令这片极光增长。”

炫光蛊能爆发赤红的极光,要维持它们的生存。方源又要在这里增添它们的食料,名为炎鸿石。

不过这两者,琅琊派的库藏中都没有,方源还得等到宝黄天再度开启。才能实施这个计划。

“而且炫光蛊,只是增长极光的一种最简单的方法,并不高效。”

“宝黄天中。也有直接贩卖流光果的蛊仙。我记得其中就有餐霞怪客,青辉子和彩霞仙女。或许我还可以向他们讨教。付出代价,收购他们的经营心得。当然此事成与不成。还要看对方的想法了。”

距离态度蛊的饥饿,还有很长一段时间。

方源此举也是未雨绸缪,提前布置。

毕竟态度蛊的食量太大,仓促准备的话,会对方源很不利。

现在的局面,方源还是比较从容的。

毕竟仙蛊一大把,但大多还未到喂养的时候。

而且第一次地灾刚刚过去不久,还有比较充裕的时间。

接下来,方源打算好好沉淀积累,以经营自家仙窍为主。

“态度蛊的喂养,已经开始筹备。但是这只态度蛊,还不是我的,里面是黑楼兰的意志。现在已经明确,黑楼兰背叛盟约,投靠了影无邪。那么我身上的一些信道盟约,也自解开来了。”

“如此一来,我就可以直接着手,将这只态度蛊炼化,成为独属于自己的八转仙蛊!”

说起来,态度蛊也是方源最近的一项巨大收获。

就连方源当初向黑楼兰借蛊的时候,也未想过,会有这么一天,态度蛊落到自己的手上。

但是要炼化态度蛊,可不是一件简单的事情。

因为态度蛊不仅不是野生仙蛊,而且转数还高达八级,反观方源不过区区六转蛊仙。

要让他来强炼别人的八转仙蛊,不仅困难无比,而且搞不好还会令态度蛊损毁。

“但我也有一项妙法,可以帮助我炼化他人的八转仙蛊。那就是曾经设想出来的那座蛊阵。暂时就取名为智炼阵罢。”

方源随意取了个名字。

他不太在乎这些细枝末节,到现在才为这个蛊阵取名字。

当初,他从仙僵薄青身上盗取了不少剑道仙蛊,为了炼化它们,就利用智慧光晕和星念进行推算,设想出智炼阵。

利用智慧光晕,以妇人心、解谜两只仙蛊为核心,大量凡蛊为辅助,便能搭建出智炼阵。

换魂仙蛊、剑眉仙蛊、飞剑仙蛊、剑遁仙蛊,浪剑仙蛊,以及八转的慧剑仙蛊,都是智炼阵帮助方源炼成的。

不过现在的问题就是,智慧蛊不承认方源了,智慧光晕不出,如何铺设智炼阵呢?

但是要勾动出智慧光晕,必须让方源肉身、魂魄归一,得到智慧蛊的承认。

所以这个问题,又归结到之前的那一点上,那就是方源的仙僵肉身。

炼化态度蛊,将它彻底变成自己拥有的仙蛊,这个计划只能暂时搁浅。

方源接下来的打算,是喂养慧剑仙蛊。

这只仙蛊,和态度蛊一样,同样是八转仙蛊。隶属于剑道,也可算是智道。

慧剑斩情丝!

当初剑仙薄青就是用此仙蛊,斩去了爱情蛊种在他身上的智道道痕,挣脱束缚,重获自由。甚至凭此策反了灵缘斋仙子墨瑶,化为己用。

这只慧剑仙蛊,可谓是助薄青反败为胜的妙手!

而现在慧剑仙蛊,已经被方源炼化,掌握在自己手中。

不过目前情形,方源要运用它,还是相当困难的。

方源用七转仙蛊,就已经十分勉强了。青提仙元若用来催动慧剑仙蛊,举个地球上的例子,就好像是给航空母舰安装一两节五号电池。

慧剑仙蛊是很正统的八转仙蛊,需要八转仙元催动。

从这点上就可看出,态度蛊的优秀方面。

虽然都是八转仙蛊,但态度蛊可以用心力催动,六转蛊仙就可以自由运用了。不愧是《人祖传》中早有记载的传奇仙蛊!

“慧剑仙蛊需要的食料,是斑斓霸王花。”

“这种花,极其巨大,每一朵成熟的斑斓霸王花,覆盖方圆上千里地。每朵花的成熟,需要大量的光照,浓郁的甜气,以及珍珠土。”

这个难度就很高了。

比之前的流光果,要超出好多。

流光果不过是六转仙材,但斑斓花却是七转。

方源要大规模种植斑斓霸王花,需要三个条件。一个是光照,第二个是甜气,第三个是珍珠土。

第一个光照,就很困难。

方源的仙窍中有光道道痕百十来道,所以充斥着微微光明,并不像是他的仙僵死窍那样是一片黑暗的。

但这种光,是绝对不够资格满足斑斓花的需求。

除非是把流光果的那些极光,挪过来用。

不过这样一来,流光果的计划就要被彻底破坏了。光也是会消耗的,尤其是那些极光现在还是无本之源。

最根本的解决之道,就是往仙窍中增添光道道痕。

积累雄厚的光道蛊仙的仙窍中,往往也是光明无限。

但方源要在短期内做到这点,除了期待下次地灾是光道灾劫之外,就是斩杀光道蛊仙,吞并其一身道痕。

前者寄希望于天意,一点都不靠谱,后者很危险,也不靠谱。兵凶战危的道理谁都明白。当初斩杀戚灾,也是方源被戚灾一路追杀,戚灾轻视方源,方源又借助五界山脉那样特殊的地形,实施了见面曾相识,才杀了他,留下全尸和魂魄的。

第一个条件光照,是最难解决的问题。其余两个甜气、珍珠土,倒是有各种方法途径。尤其是珍珠土,在宝黄天中大规模贩卖,价格也不贵。

当然,琅琊福地中是没有的。

以上只是喂养态度蛊、慧剑蛊的计划,就已经足够方源烦神劳心的了。

幸亏九转智慧蛊的喂养,已经被琅琊地灵主动接过去了。

智慧蛊的食料是寿蛊。

方源甚至都有点不敢去想。

连他自己都要和智慧蛊抢夺食物呢。

“八转的态度蛊、慧剑蛊的喂养计划,都已经在安排了。接下来就是七转的换魂、剑眉、浪剑、剑遁和招灾。”

招灾蛊方源早已经喂过,是用的六头大蛇的黑血。

六头大蛇是长有六个蛇头的荒兽。方源曾经在宝黄天中采购过蛇骨,但现在为了长期打算,他打算迁移几头荒兽大蛇,进入仙窍中好好繁衍。

至于其他七转,亦各有各的喂养计划,需要方源好好安排和规划。

当然,方源也没有忘记,楚度那里还有他的七转飞剑仙蛊。

这就交给楚度喂养了。

反正楚度记挂方源的渡劫法门,怎么可能饿着飞剑仙蛊。

从这点看,楚度也算是给方源出力,替他减轻负担了。

\end{this_body}

