\newsection{一个人情}    %第二百二十三节:一个人情

\begin{this_body}

镇运天宫。

巨阳仙尊居然活生生地出现在药皇的面前。

药皇极为震惊,他感到难以置信。

“南荒仙使,参见仙尊。”这时,南荒仙人已经恭敬行礼。

“不肖子孙药三秋,拜见仙祖!”药皇立即激动得浑身颤抖,当场跪拜下来,老泪纵横。

巨阳仙尊微微笑道:“起来吧。我的本体早已陨落,如今在你们眼前的,不过是一具尸首罢了。”

药皇惊讶地抬起头,他仔细分辨,这才看清楚,原来眼前的巨阳仙尊并非活人,而是一位仙僵。

自从红莲魔尊打坏了宿命蛊,天下的魂魄可以驻留在天地之间,于此同时,僵尸也从那个时候开始出现。

渐渐的,蛊仙们发现寿命不足,转变成仙僵,是上佳的选择之一。

尽管天庭方面非常反感和抵制,但难以抵挡人心和大势,五域中的仙僵越来越多。

寻常的蛊仙都能转成仙僵,没道理堂堂的巨阳仙尊不能转变吧?

药皇想到此点,心中的惊疑就都消失不见了。

巨阳仙僵继续解释道:“现在的我,除了肉身之外,就是体内的一些残留意志罢了。”

但药皇仍旧很激动:“毫无疑问,您就是我们黄金血脉的源头,我们共同的祖先。只要您露一次面,就能让各大黄金家族抛弃彼此间的间隙。紧密团结起来。让整个北原,再次成为我黄金一族的花园!”

但巨阳仙僵却缓缓摇头:“我的本体早已陨落,徒留一具尸体。我终究没有超脱九转境界,没有得到永生,还露什么面?贻笑大方罢了。”

“况且。我遗留的这具肉身,只有一击之力,稍有移动,就会灰飞烟灭。”

“仅有一己之力?”药皇惊讶无比,但他眼中的崇敬和狂热,仍旧没有丝毫的褪色,“即便仙祖大人只剩下一己之力。也必定教天地震荡。日月失色。”

巨阳仙僵哈哈一笑:“这是自然。不过你呀,也不要指望了。我布置镇运天宫在这里,是护持北原。而留下这具肉身,则是另有因果。我要还一个人人情。”

“人情?”药皇大为讶然。

巨阳仙尊留下肉身,转化仙僵,枯坐在镇运天宫中三十多万年,为的就是还一个人情。

究竟是谁。能让堂堂一位仙尊,欠下人情?

而这个人情,又是如此巨大,值得巨阳仙尊如此付出呢?

药皇感到分外疑惑,但巨阳仙僵却没有继续解释下去,而是道:“此次中洲蛊仙来犯,我是无法出手的。不要指望我,一切都要靠你们二位。”

药皇缓缓站起身,满脸肃穆:“中洲蛊仙阵容虽强,但我方有镇运天宫。亦有我和南荒大人。先祖放心,就算三秋拼死也要阻挡住这些人。哪怕舍弃这条性命,战死沙场,也在所不惜。”

巨阳仙僵看了药皇一眼,缓缓摇头,闭上双眼,重新入定。

这一刻。他仿佛变成了一座雕像,再无任何言语。

药皇不知其意,这时南荒仙人拍拍他的肩膀:“跟我来吧。”

两仙告退,离开了主殿,进入了后殿。

药皇的语气中还残留着激动:“还请南荒大人下令,在下赴汤蹈火,在所不辞!”

南荒仙人摇摇头:“镇运天宫乃是八转仙蛊屋,更是巨阳先祖根据天地运真传,而亲手打造出来。先祖高瞻远瞩,将这座镇运天宫布置在这里,已经镇压了这片天域整整三十多万年了。你可知道,什么是天地运?”

药皇微微一愣,口中呢喃:“天地运……”

南荒仙人不待他答,笑着道:“一个人有他(她)自身的运道,一头野兽也有它的运道,一棵草亦有运道。但凡生命,都有运道。除此之外,一块石头,一弯溪水,也都有它们的运。整个天地,都有天运、地运。”

“天地运真传,就是洞察天地运的奥妙,研究它,利用它,改变它。”

“别说是中洲蛊仙来了三位八转,携带三座仙蛊屋。就算是阵容再扩大一倍,来到这里,也是有来无回。”

“我召唤你来,不是让你来拼命的。你可是我长生天外,黄金血脉中的唯一八转。”

“啊?”药皇惊讶无比。

南荒仙人继续道:“我已寿命无多,苟延残喘至今,寿蛊早已无效。接下来你好好看着,看我如何运用这座仙蛊屋,覆灭了这些中洲蛊仙。待我死后,便由你接任南荒之位。”

药皇瞪大双眼,不禁失声:“南荒大人!”

……

风满楼中一片沉寂。

黑灯就像是风雨雷电,是黑天中的一种独特气象。

并非没有防备,而是黑灯气象出现的实在突然。更关键的是,现象发生的时候,揽雀阁、风满楼、角连营这三座仙蛊屋,正好处于无数黑灯的中央。

一般而言,八转蛊仙在探索黑天的时候,遭遇黑灯气象,也是远远发现,然后迅速避开。

这一次,中洲蛊仙们刚刚摆脱夜天狼群,心情刚放松下来,没想到黑灯出现,所以很多蛊仙,都因此遭殃,受到了重创。

风满楼的最顶层。

“父亲!你的眼睛!?”六转蛊仙施正义双目含泪,哽咽着。

他的父亲乃是七转蛊仙中的成名强者,没想到却是彻底失明的那三人之一。

黑灯气象出现的时候,他正在用侦查杀招,观察狼群迹象。他的侦查杀招效果极好,反而成了他失明的罪魁祸首。

基本上可以肯定,从此以后,施阁就告别了光明,成为一个无法看清外界的盲人。

不过施阁却很冷静,反而伸出大手,抚摸施正义的头发,宽慰他道:“失明了又如何?我手段众多,各种侦查杀招层出不穷,就算是失明,也能继续修行,更能继续作战。有什么关系?”

“可是……”施正义低头咬牙,要垂泪的样子。

“我儿,不要轻易掉泪。蛊仙修行,本来就是充满了危险。只要没丢掉性命,我们就应该继续朝前进发。不要像个弱者那样,哭哭啼啼了。不是爱听评书,爱看话本么?想想那些主人公,是怎么样的?”施阁温声道。

施正义重重地点点头:“父亲,我明白了!”

就连七转蛊仙强者施阁,都中了招,彻底失明。

赵怜云目睹这一幕,她心头震动,头一次感到这座仙蛊屋风满楼,也不是那么厚实安全的。

不过,很快,她又想起了马鸿运。

“鸿运,你等我。就算是有千险万难,我也要到你,将你救出来!”

赵怜云暗自为自己鼓劲打气,双目中透射出坚定的光辉。

“余艺冶子,你怎么样了?”赵怜云询问眼前的少年蛊仙。

这位炼道蛊仙,自从刚刚看了一眼窗外的黑灯之后,就一直盘坐正在地板上,紧紧闭上双眼,催动某种仙道杀招,进行自我疗伤。

听到赵怜云的询问,余艺冶子没有停下仙道杀招,也没有睁开双眼,而是开口道:“幸亏我醒悟得早,及时地闭上了双眼。若是稍微迟一点儿,那就挽回不了了。”

余艺冶子没有再多说什么。

他闭上了嘴巴。

心有余悸未消,额头上还残留着冷汗。

渐渐的,蛊仙们都沉默下来。三座仙蛊屋中,无一不被沉闷的氛围笼罩。

仙蛊屋外,传来一阵阵的魂啸之音,即便有仙蛊屋阻隔,也非常刺耳难听。

不真子看向窗外,满脸忧愁之色:“不好办,我们陷入了魂兽大军的包围之中了。”

赵怜云闻言,也望向窗外,只见无数的冥蚁蛊,像是一股股的黑色河流,围绕着三座仙蛊屋。(未完待续。)

\end{this_body}

