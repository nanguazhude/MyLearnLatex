\newsection{分身之用}    %第四百六十二节:分身之用

\begin{this_body}

%1
分身渡劫成功,成为了宙道六转蛊仙。而方源的主体仍旧是七转蛊仙,?8??尊仙体,全流派兼修。

%2
有一个分身,对方源而言,很有好处。

%3
许多事情,尤其是经营仙窍这一块,就可以交给分身来打理。

%4
这是方源绝对信任的,远比什么影宗成员要得力得多。

%5
尤其是双方之间,本为一体,思想、意志等等都是完全一致的,所以沟通交流上不存在什么矛盾,甚至连误差都很少。

%6
看一看影宗的强盛,就可知道分魂的厉害之处。

%7
影宗成员众多,但都团结在一起,不计较个人的牺牲,皆因他们都是一体,都只是一个人。

%8
让分身成就六转蛊仙,方源付出巨大。

%9
先是艰难推算方法,耗费自己大量的时间、精力,而后平炼春秋蝉,又耗费了一批仙材资源,成就分身之后,分身也需要修行,虽然不是重点,但这一块上也需要投入。

%10
因为分魂,导致方源的本体魂魄,从原本的千万人魂,直接降低到了九百万人魂,可谓大缩水。

%11
这个魂魄底蕴,方源又得继续开始积累。所幸拥有两大圣地的他,这个方面的进展尤其迅速,对方源的发展并未多少凝滞阻碍。

%12
而方源的这个分身,因为有了方源的分魂,算得上是半个天外之魔了。

%13
肉身方面来源于这个天地,因此还要遭受宿命的束缚,这一点方源也没有办法。

%14
除非他能够效仿影宗,再炼制出一个至尊仙胎蛊来,交给分身使用。

%15
影宗上下积蓄了十万年,才炼出一只至尊仙胎蛊,要想炼第二只,基本上不可能了。

%16
接下来的日子,方源主体一边照常休息,另一边他的分身则置身在洁身自好仙道蛊阵中,消除自身上的力道道痕。

%17
方源原本是第二空窍升仙,成为了力道蛊仙。而后化为仙僵,仙窍崩解,力道道痕就都留在了肉身上。

%18
方源决定,利用洁身自好杀招,帮助分身剔除不必要的力道道痕。

%19
分身仙躯并不特殊,道痕在其身上,相互掣肘干扰,方源深思熟虑之后,还是决定让分身专修宙道。

%20
之前方源考虑力道、宙道兼修,是因为没有至尊仙体。

%21
现在有了至尊仙体和仙窍,将来的修行重点还是在这上面,分身只是一种补充。

%22
所以,为分身炼制第二空窍的方案,也就取消了。

%23
多出一个空窍,就是将来升仙,化为仙窍。方源既然选择专修宙道,那么这个仙窍自然也要修行宙道,但这又有什么用呢?

%24
多出一个上等福地,或者是特等福地,对于方源而言,毫无用处。

%25
因为建设这个福地,还不如将全部的精力投入到至尊仙窍当中去。而方源的至尊仙窍,开发程度只有百分之四,地广物稀,太需要经营了。

%26
“或许多出一个仙窍,最大的作用就是有一个本命蛊的名额。”方源心中思量。

%27
本命蛊最大好处,就是将来升炼的时候,就算失败,本命蛊也不会损毁。

%28
本命、本命,蛊虫的性命是和蛊仙连成一体的。

%29
当然,若是因为炼蛊失败,蛊仙死了,本命蛊基本上也会毁掉。

%30
目前方源对本命蛊的需求,还是很小的。

%31
他仙蛊众多,并且最主要的春秋蝉,已经是本命蛊了。

%32
分身除了帮衬方源处理一些杂事之外,就是保证春秋蝉在今后的炼蛊过程中,不会损毁。

%33
“你醒了?”方源望着床榻上的毛六。

%34
毛六缓缓睁开双眼,仍旧感觉到虚弱,看到方源之后,他开口第一句话就问:“宗主,此次炼蛊成功了吗?”

%35
“成功了。”方源颔首。

%36
“太好了。”毛六笑了起来,欣慰地道,“这样一来,第二期的炼蛊计划,至此都成功了。”

%37
第一期炼蛊,方源炼成了净魂、自爱以及小吃。第二期炼蛊,方源则炼成了春秋蝉和天机蛊。

%38
为了炼蛊,方源付出极大。总体而言,运气还是很好的,有的蛊仙甚至八百十次,都没有成功一次。

%39
这其中,方源掌握着运道真传还有运道仙蛊,很大程度帮助了他。另一方面,琅琊派的作用举足轻重,若非琅琊地灵、毛六这些人帮衬,方源的炼蛊计划怎可能取得如此丰盛的成果?

%40
“你身上的赤蛇纹身,是怎么回事?”方源目光深幽。

%41
“这……没有什么。”毛六目光闪烁了一下。

%42
“在你昏迷期间,我已探测了一番,这等炼道杀招似乎是直接折损你的寿命,来暴涨你炼道方面的实力。若我所料不差,当这些赤蛇纹身覆盖你的头顶,你就会彻底丧命吧。”方源直接道。

%43
毛六沉默了一下:“宗主所料没有差错。我的命死不足惜,只希望宗主你能够攻上天庭,拯救本体。”

%44
“天庭……”方源仰头,望着天空,叹息一声。

%45
毛六连忙道:“宗主不必气馁,敌人势大不假,但如今宗主你已经有了春秋蝉,继承红莲真传指日可待。红莲魔尊的真传,极可能蕴藏着制胜天庭的法门。毕竟当年,红莲魔尊可是天庭选定的仙尊种子。”

%46
方源点点头,毛六知道的东西,他都知道。眼下,单靠他自己,很难在十年的期限内,阻止天庭修复宿命蛊。但若是依靠红莲真传,可能还有一线希望。

%47
既然春秋蝉已经完全为方源所用,那么寻找红莲真传,似乎就成了下一步的计划。

%48
但方源又摇摇头:“我并不打算急着寻找红莲真传,如今我的实力还可以继续迅猛提升。天庭方面,绝不会坐视我接收红莲真传,定然已经在光阴长河中有了布置。”

%49
天庭虽然找不到方源的所在,但是却知道他定然会深入光阴长河,去继承红莲真传。

%50
方源用脚后跟来想,都能料到天庭必然会在光阴长河中设伏。

%51
寻找红莲真传是必须的,但此事的前提,不仅是完全掌控春秋蝉,还有自身的战力能否令方源击破天庭的埋伏。

%52
在这个方面,方源还需要下苦功。

%53
“将你自创的这个炼道杀招内容,全部告诉我,我来为你推算,解除这个杀招的后遗症。”

%54
“不要轻易牺牲了,毛六啊,将来我能否掌控琅琊派,还要靠你出力呢。”

%55
毛六不禁怦然心动,他这样一想也是。自己对伤势无能为力,并不代表方源不行。

%56
尤其是还有智慧光晕可以利用。

%57
平炼春秋蝉以及分身渡劫,自然都是在荒郊野外。

%58
将分身收入至尊仙窍中后,方源和毛六一同,通过传送蛊阵回到了琅琊福地。

%59
进入琅琊福地之后,就是在琅琊地灵的时刻监控之下,毛六和方源不能深谈,以同是琅琊派太上长老的身份,相互告别。

%60
方源没有回到自己的云城中休整,而是直接来到智慧蛊的面前。

%61
他打开仙窍门户,飞出分身来。

%62
分身乃是鲜活肉体,智慧蛊并无反应。

%63
不过很快,随着分身催动力转生死杀招,再次转变成仙僵之后,智慧蛊疑惑地围绕他不断飞旋。

%64
“怎么?”方源心中咯噔一下。

%65
智慧蛊只是飞旋盘绕,没有散发出智慧光晕。

%66
方源分身开口:“智慧蛊啊,智慧蛊,我就是和你约定的人。你难道忘记了我们曾经的约定吗?”

%67
智慧蛊悬浮在方源分身的面前,似乎在打量他,也似乎在犹豫。

%68
过了好一会儿,它才重新趴在树叶上,散发出智慧光晕来。

%69
方源这才松了一口气。

%70
真的有些惊险,智慧蛊差点不承认方源的身份。

%71
“我成就了宙道蛊仙,修为上升,本身的力道道痕也驱除了一些,还有分魂,这些都可能导致智慧蛊犹豫不决。”

%72
“不管是什么原因,将来还要注意,哪怕分身修为停滞不前,也要努力维持原状,不能放弃智慧光晕!”

%73
方源有些郁闷,《人祖传》中智慧蛊通灵得很,甚至成功坑害过人祖,怎么到他手中,却是如此单纯蠢笨?一点都没有智慧蛊的风采了。

%74
这或许是《人祖传》的文学修辞手法,也或许是其他什么原因?

\end{this_body}

