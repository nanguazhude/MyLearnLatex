\newsection{北原争乱}    %第八十二节:北原争乱

\begin{this_body}

夕阳西下,天边一片晚霞,赤红如火,美不胜收。

微风和煦,吹得地上掀起一层层的黄金麦浪。

“这就是黑家精心培育的麦场么,气象倒也是宏大。”蛊仙陆青冥悬停于高空中,俯瞰脚下,口中淡淡地道。

黑家在这里种植了大量的麦子。

这种麦子,称之为黄金麦。两年一熟,产生的却不是供人食用的大麦,而是一种金属,名为金精。

这片地方,本是贫瘠的荒土。因为富含太多矿物,甚至形成裸矿,所以寸草不生。

黑家种植黄金麦,利用此法开采这里,不仅是动了巧妙的心思,而且还是付出了许多心血。所以,才形成了这上万里的广袤麦田。就算是陆青冥此刻,于高空眺望,这麦场也是一望无际。

陆青冥身边,还站立一人。

此人姓苏名光,乃是光道蛊仙。

他眼含热切,注视着脚下麦田:“黑家的这处麦场,早已闻名北原。每一次产出,都至少能收获百万斤的金精。这些金精,一直都是黑家的一大进项。虽然这些金精只是五转凡材,我们用得少,但收刮起来,贩卖出去,却可卖出一笔好价钱的。”

陆青冥笑了笑,点点头:“的确值得动手。”

“我劝二位切莫自误,此处已被我关家所得。”话音刚落,一位身影凭空浮现,站在陆青冥、苏光二人的面前。

他满脸肃穆。战意凛然,额头上竖着第三只眼眸,极其惹人注目。

陆青冥、苏光对视一眼,立即达成共识。

他们二人早就察觉到这里,有蛊仙气息盘踞。刚刚只说不动,就是为了引出其他蛊仙。

“原来是关家蛊仙关神照。”陆青冥客气地道。

“游地三英的大名,我也早已知晓。只是东方长凡一役,韩东蛊仙已陨落太丘。今日二位若要强行出手,恐怕游地三英的名号,就要彻底除名了。”关神照却很不客气。

他是关家蛊仙,关家可是超级势力。黄金家族。因此底气十足。

苏光哈哈一笑:“你是正道,我们兄弟是魔道。正魔对立,我们岂会轻易退缩?自然要做过一场,分出胜负!”

关神照冷哼一声,干脆直接出手。

一时间仙气横扫,光芒爆闪,双方在高空中开始火并。

……

两个身影。一大一小,正贴着地面飞行。

“爷爷,我们这是去哪里?”小的那个,只是位少年,面带好奇之色,开口问道。

“嘿嘿,乖孙儿,咱们要去的地方就是涌泉林。你知道不?”大的身影,是一位蛊仙老者,身穿黄袍。宽大的袖口随风狂摆。

少年眼珠子一转:“涌泉林是地下的喷泉,形成的水柱森林。据说那里,有无数水道蛊虫、土道蛊虫,是黑家的重地。”

“不错。爷爷平时没有白教你。乖孙儿,你不是修行水道吗?这次咱们去涌泉林,要什么水道凡蛊没有?运气好,还能多搜刮点泉眼。种在爷爷的仙窍中。等到哪天你也成仙了,爷爷就将这些资源转交给你。”老蛊仙道。

少年涌起感动之色,又担忧道:“爷爷,你不过是六转垫底的蛊仙。冒然进攻黑家重地,万一丢了老命怎么办?”

少年直言不讳,蛊仙老者却没有动怒,反而笑嘻嘻地提点自己的这位孙子:“十多天前,黑家蛊仙就已经都龟缩到大本营中去了。如今在外的黑家重地,都没有蛊仙把守。”

“黑家蛊仙不在,万一有其他蛊仙呢?咱们昨天不还听说了,皮水寒那样的老魔,都现身了吗?”少年还是很担心。

“这你就不懂了。”蛊仙老者却是相当自信,“涌泉林虽然是黑家重地,但和其他比较起来,其实是最次一等的。皮水寒那样的老魔,根本看不上这里。”

“距离黑家最近的超级势力,依次是刘家、关家、药家。距离这三家的位置,那些资源咱们不能抢。距离黑家大本营太近的位置,咱们也不去动手。那些重中之重的宝地,都有强者争夺,咱们也抢不到手。就算在外观战,恐怕也有丢掉性命的危险。所以涌泉林是最适合我们的选择之一了。”

“原来如此。”少年恍然大悟。

风声灌耳,祖孙二人距离涌泉林越来越近。

忽然,蛊仙老者面色一变,他听到激斗之声。

“不好。”蛊仙老者面色一沉,飞扬上空。

此时,烈日高照,阳光炙热。

涌泉林映入祖孙二人眼帘,却发现这里早已经是一片混战。

原本如世外桃源般的涌泉林,此刻已经一片狼藉,混乱不堪。

泉水四处乱溅,爆炸轰鸣声不绝于耳。

大略一看,竟有近十位蛊仙,掺和其中。

一时间,祖孙二人相视苦笑。

他们能够想到的,别人早就想到了。明智的底层蛊仙,不再少数。

……

天空中,乌云密布,阴雨飘飘。

两拨人马对峙在这片平常无奇的草原上空。

关家和刘家。

关家三位蛊仙,刘家两位。

气氛却不算凝重,反而两方领头都攀谈得甚为热切。

若不是天气不应景,恐怕都可称为叙旧的茶话会了。

“太上三长老大人,我方势大,不如趁机出手,霸占住此地。”关家一方,蛊仙关神照终于不耐烦,传音觐言道。

关家太上三长老一边和对面热切交谈,一边却传音否决了关神照:“你是糊涂!你忘了数天前的教训了?胡乱动手。结果引起更大骚乱,保不住黄金麦场,让其他魔道蛊仙哄抢一空。”

“这个地方是隐藏着一块太古赤天的碎片,可谓重中之重。暗中关注这里的眼睛,不知有多少。只是我关家、刘家算是半个地头蛇。才有些许强势,使得我两家站在明面上争夺。”

“此时此刻,切不能焦躁不安,乱了分寸,胡乱动手。一动手,场面就控制不住,容易让有心人夺了胜利的果实!”

关神照被这一顿训斥。不由面现愧色。连忙传音道:“大人说得是,是神照莽撞了。”

……

东海,一处毫不起眼的无名海岛。

影无邪、石奴、黑楼兰、太白云生四人,现出身形。

“就是此岛了。”影无邪环顾四周,吐出一口浊气道。

“方源,这里应当是乱流海域吧?”太白云生猜测道,他曾经在东海闯荡过一段时间。手中有辨别方位的东海凡蛊很是正常。

影无邪微微一笑:“不错。这是师傅前不久,交代给我的秘密营地,我们就在这里进行休整。”

目前,他还顶着方源的身份,欺瞒太白云生。

这四位蛊仙当中,也就太白云生不知真正内情。

影无邪默默调动心念,片刻之后,一挥手,道:“你们都随我来。”

一瞬间,隐藏在这里的某种布置发动起来。

黑楼兰等三人视野骤变。再定睛一看,已经是到了海岛地下。

这里已经被挖掘出来,一座五重的华美宫殿映入眼帘。

“这好像是一座仙蛊屋?!”黑楼兰心中惊叹不已,影宗之势大,在五域各处都有布置。虽然义天山事败,但残留的能量绝不容小觑!

但接下来,还有更让她吃惊的。

他们三仙跟随着影无邪的脚步。来到宫殿之内。

宫殿大门无人自开,显然是认可了影无邪的身份,欢迎他这位主人。

“这竟然是?”

“好多的仙蛊!”

入得其中一座宫殿,石奴、太白云生忍不住低呼出声。

影无邪笑着解释道:“乱流海域有一大好处,那就是汇集天下各种水流,其中就包含光阴长河中的支流。正因为如此,这里虽然不是什么洞天福地,却是五域间罕有的宝地。”

“我影宗早在许多年前,就查明这道光阴支流的所在,所以便在此岛上做出布置,挖空海岛,坐落仙蛊屋,并且时刻汲取这道光阴支流的力量,进行炼蛊。所以,每隔一段时间,就有仙蛊自行炼成。”

太白云生早有所猜测,此刻忍耐不住,开口问道:“难道说,这个蛊仙蛊屋就是传说中的悔池?”

影无邪点点头,又摇摇头:“只是悔池的大部分而言,虽然重现了悔池最厉害的炼蛊能力,但本身的攻防却差得很。并且一旦坐落下来,就不可再移动分毫。”

太白云生闻言,激动得浑身轻颤:“天下三池,天池、悔池、酒池,名冠五域,历史留名。想不到我今天,居然有眼福见到悔池!”

黑楼兰见识稍差,对宙道当然不及太白云生那么了解:“炼制仙蛊极其艰难,这悔池居然单靠自己一座仙蛊屋,就能炼成这么多的仙蛊?”

影无邪微笑着解释道:“悔池乃是宙道仙蛊屋,它的炼蛊之能,说大可大,说小可小。我影宗蛊仙在这里炼制仙蛊,若是成功,就能将成功的印记保存在这里。等到仙蛊毁掉,悔池就可发动起来,借助光阴支流的力量,和保留下来的成功印记,重新炼出仙蛊,成功率往往十之五六。”

“十之五六?!”黑楼兰对如此高的数据,感到十分震惊。

“也有不少弊端的。比如炼蛊成功之后,之前保留的印记也就随之消散了,需要蛊仙重新补充。还有,若非隐藏在此处的光阴支流,要发动这项能力,必得耗费巨大,无法多用。”影无邪又补充几句。

黑楼兰的脸上还是残有惊异之色,她望着眼前种种仙蛊,从心底发出感慨道:“即便如此,也是极强了。”

影无邪仰头一笑:“之前危局,累你们都失去了仙蛊。这里的仙蛊就算是补偿,任由你们选取!”(未完待续。)

\end{this_body}

