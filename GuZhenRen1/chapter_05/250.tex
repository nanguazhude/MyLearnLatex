\newsection{子母逆命祭炼大阵}    %第二百五十节:子母逆命祭炼大阵

\begin{this_body}

小人蛊仙沉吟一声,随即催动某种仙道杀招,口中低呼:“大,大,大。”

于是,他的身躯就见风而涨,一次次膨胀,很快,就和常人无异。

并且,他身后的羽翼也同时收敛起来,从外表来看,和寻常人族别无二致。

至此,在场的蛊仙们才见到了这位神秘八转蛊仙的真正面目。

他是一个老者,貌不惊人,甚至有些邋遢的感觉。唯一的特征,就是一头紫发,乱糟糟的蓬松一团,仿佛乞丐。

扑通。

太白云生泪流满面,直接双膝跪地,激动得浑身颤抖,口中大喊:“师傅!”

“哦?原来是你呀,我记得曾经交给了一份宙道真传。”八转蛊仙笑了笑,露出一口残缺的黄牙。

“我宗大计虽然成功,但却出现了许多意外。如今的情况是这样的……”为了争取时间,影无邪上前,直接向紫发老者传达了许多意念。

须臾功夫,紫发老者就明白了一切。

“可惜了,最终还是被天意小胜一筹。”

“不过一切还未完全失去希望。”

“影无邪你做的很好,辛苦你了。”

“还有太白云生,起来罢。”

“现在,趁我还清醒着,便和雪胡联手,将这些中洲蛊仙留下!”

情况紧急,紫发老者立即意识到,八转蛊仙的这场战斗才是关键。

他没有任何犹豫,身体化作一道冲天的紫光,下一刻,就到了高空,和雪胡老祖并肩而立。

“请问仙友大名!”雪胡老祖并没有意外,之前万寿娘子早已经将影宗一行人的出现等等情况,都动用蛊虫传达给了他。

有了神秘的帮手,雪胡老祖心情振奋。

“我有很多名字,不过……”紫发老者说到这里,脸上露出一丝意味深长的笑容,“雪胡仙友不妨称呼我为紫山真君便是了。”

当即雪胡老祖和紫山真君联手,对付中洲的两位八转,兼同角连营。

紫山真君实力极其惊人,虽然出手次数有限,但每一次都有画龙点睛之效。

换做一位寻常八转,就能帮助雪胡老祖扭转局面。更何况现在多了一位深不可测的紫山真君!

威灵仰和碧晨天尽管咬牙苦战,费尽努力,都未能阻止局面向北原一方倾斜。

十几个回合之后,他们俩个居然只能被迫缩回了角连营中,难以只身抵挡雪胡老祖和紫山真君的联手。

各大峰主见到这一幕,无不振臂欢呼。

没有中洲蛊仙攻击逆命祭炼大阵,那么这一战的结果已经非常明确了。

反观赵怜云等陷落在大雪山福地中的五位蛊仙,却是面色惨白。战局如此剧变,让他们的求生希望,变得非常渺茫。

这可该如何是好?

大雪山福地之外,一头上古剑蛟隐藏在云层之中。

“影宗居然和大雪山方面,有所关系么……”

“那为什么,之前秦百胜和凤九歌等人在落魄谷大战,却没有请大雪山方面出手呢?”

“或许,这层关系并没有我想象中的那么紧密。”

气运交感告诉方源,影无邪等人就在大雪山福地之中,并未远离。

方源追杀受阻,并不甘心,仍旧徘徊在外,看看有没有什么新的机会。

而在千里之外的某个地方,两位蛊仙从天而降。

不是旁人,正是来自长生天的玄极子、洪极子二人。

“便是此处了。”玄极子说着,催动蛊虫,顿时引起不同凡响。

两人眼前的这处小缓坡,开始绽放出绚烂的光彩。

洪极子恍然大悟:“原来此处,已经被你布置了蛊阵。接下来需要我主持蛊阵么?”

玄极子摇头:“先入阵再说。”

他率先迈步,踏入这个蛊阵,洪极子紧随其后。

一进入蛊阵内部,洪极子顿时脸色微变,惊叹道:“这处蛊阵气息恢弘,不愧是你玄极子亲手布置,可想而知,一旦催动起来,必定威能惊人。不过……”

说到最后,洪极子语气变得有些迟疑。

“不过什么?”玄极子饶有兴趣地看向他。

洪极子沉声答道:“不过,凭我感观,似乎此阵意犹未尽,并不完整。好像是一片湖泊,在此之前,应当还有一处大海与其勾连。”

玄极子闻言,扬起眉头,鼓掌而赞道:“不愧是水道准大宗师,已经半步跨入了触类旁通的境地了。不错!这座蛊阵只是一座子阵,还有一座母阵与其呼应。”

“那不知母阵在何处?”洪极子皱起眉头。

不待玄极子回答,他忽然脑海中灵光一闪,恍然大悟道:“我明白了!那座母阵便是大雪山福地中的逆命祭炼大阵!”

“然也。”玄极子哈哈大笑,“我之前受命,以孙名录的身份,接近万寿娘子。最终创建了逆命祭炼大阵,灌溉逆流河。雪胡老祖虽是北原当今第一人物,但却不通宵阵道,没有看出我的隐藏暗门。”

“如今,只要我全力催动这座子阵,便可让大雪山福地那端的母阵崩溃,将逆流河牵引过来。”

“一旦蛊阵崩溃,大雪山福地的方圆万里之内,一切的人和物都会被逆流河卷席,一同来到此处。”

玄极子侃侃而谈,目露兴奋之色。

洪极子讶然:“可是这么一来的话,岂不是将那四位八转大能,都一同送过来了吗?”

玄极子摇头:“逆流河乃是天地秘境,非同凡响,道痕数量庞大到难以想象的地步。只要落入河流当中,八转大能也不能动用任何蛊虫。《人祖传》中,人祖跋涉逆流河,从未能在河流中运用过什么蛊虫。”

“那他们岂不是任由我们宰割?”洪极子兴奋起来。

但玄极子又摇头:“我们身处逆流河外,但对其发动的任何攻势,都会被逆反过来。所谓逆流河,逆字当头,自然是要实至名归。”

洪极子疑惑:“如此一来,我们如何从中救出马鸿运,俘虏赵怜云?”

“哈哈哈。”玄极子仰头大笑,“你忘了一点,这逆流河终究还是受到我这子母逆命祭炼大阵的影响。”

“当母阵崩溃,逆流河成形,就会冲向我们这里。而在那时,大雪山福地中的人物,会受到我的安排,比如马鸿运和赵怜云,就会排列在逆流河的最前端。”

“万寿娘子等人,必定要排在后面。中洲蛊仙则在更后方,那些八转蛊仙则在最后面。”

“妙啊!”洪极子听得双眼放光,击掌道,“如此一来,马鸿运、赵怜云为了躲避万寿娘子等人的追捕,必定会拼命挣扎,一心朝前走,拉开和身后蛊仙的距离。只要他们始终保持在前端,我们就能在此处,首先接触他们,将他们擒拿活捉。”

“不过……”洪极子又皱起眉头,“若是马鸿运、赵怜云中途坚持不住,他们就会像人祖那样,被逆流河冲走。这样一来,他们就会落到其他蛊仙的身后去,岂不是坏了我们的算计?”

玄极子苦笑一声:“这的确是计划中的最大破绽。不过,幸好跋涉逆流河和修为、体力无关,马鸿运、赵怜云他们并非没有优势。”

“但这还是太不保险。”洪极子并不满意这样的回答。

玄极子摊开双手,道:“那你让我一个七转蛊仙怎么做?这已经是最好的方法。毕竟我们算计的可不是一位两位七转蛊仙,而是大雪山福地,还有中洲蛊仙强者,牵涉到八转大能的存在,并且现在来看,这样的存在还多达四位!”

洪极子默然无语,无奈地点点头,这的确是他们自身能力的尽头。

说实话,玄极子能做到这种程度,已经相当不易。

“放心吧。我们不过只是其中的一环,别忘了,南荒大人已经苏醒,他一定还有其他后手计划的。我们只管做好自己的,就可以了。”玄极子拍拍洪极子的肩膀,安慰道。

\end{this_body}

