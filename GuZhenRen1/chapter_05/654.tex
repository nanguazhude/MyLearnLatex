\newsection{他是古月方源}    %第六百五十六节:他是古月方源

\begin{this_body}

%1
方源五百年前世。

%2
海神祭。

%3
天真?

%4
面对寒潮族长的长篇大论,方源陷入沉默。

%5
但他沉默的时间很短,旋即他轻笑一声,传音回道:“你以为我不知晓你说的这些吗?你觉得我是涉世不深的青年?不,这些把戏我都了解,也都清楚。我知道这些事实,也接受这些事实。”

%6
方源从青茅山被迫出走,辗转南疆,又去西漠,再临东海。他在濒死的线上挣扎过,他为一两块元石愁苦过。他把腰弯下,在强者和颠沛的生活前卑躬屈膝,他也曾坐在主位上,喝着茶悠然地听下属汇报。

%7
他卑贱,他辉煌,他高大,他平凡。

%8
算是地球上的生活,再算上穿越过来的历险,他的视野先天凌驾于世人之上,他的经历也丰富精彩,可著成书。

%9
这样的人,岂会对世情不了解?

%10
寒潮族长都有些抓狂了,连忙回应:“你既然知道这是事实,那就该明白,你现在是在行险!这是我们鲛人内部的政治争斗,你一个人族蛊师,实力并不强,掺和进来干什么?你喜欢谢晗沫?我可以保证,事成之后,送给你更多更美的鲛女!不要怀疑我的诚意,我可以向海神起誓!”

%11
“一位鲛人向海神起誓,这样的诚意自然无法怀疑。不过……”方源话锋悠悠一转,“我虽然接受这个事实,但并不代表我喜欢这样的事实啊。你以为我喜欢谢晗沫?不不,我只想帮她。我为什么这么冒险帮她?因为我这个人做人有个原则,那就是有恩必还,有仇必报。”

%12
“我用过寿蛊,我活得时间比你想象中要长得多。我以前希望长生不老,但现在却渐渐厌烦了这个想法。生活变得越来越无聊,有时候路的终点并不重要。重要的是走路的过程,以及走路时候的心态。”

%13
寒潮族长听了这话,不禁双眼微瞪,他难以理解方源的这种生活方式:“你是说,圣女之位并不重要,重要的是你帮助圣女的过程?”

%14
“不错,但还有更多。你说的阴暗政治,许诺送我的鲛人美女,也都不重要,我虽然知道,但我从不放在心上。这么说,你或许可以理解一些,我活得够久了,已经厌烦带着面具生活了。死亡对我而言,一点都不可怕。我现在……只想用自己最真实的面目活着,想用自己最想用的方式达成目标。也只有如此活着,我才能感受到生命的激情,以及对生活的渴望!”

%15
寒潮族长听得目瞪口呆,他终于明白一些来,大叫道:“原来你是一个疯子!你说了这么多,无非就是自己活得够久了,不想活了,想作死了!你若是蛊仙也就罢了,你一个区区的三转蛊师,还想凭借自己的心意活着?你这是痴人说梦!”

%16
方源便笑:“你以为成为了蛊仙,就能凭借自己的心意活着?不带着面具活着?有人的地方就有江湖,就有斗争。生存和生活是两码事。想要怎么样活着,不必看你的实力和修为,其实只看你自己的心。”

%17
顿了一顿,方源又说道:“其实,实力低微也很有乐趣的。当你真正用真面目活着的时候,实力低微会让你面临更多的现实的为难和挑战,跨越这些困难,面对这些挑战,人生处处都是精彩呢。”

%18
寒潮族长呆呆地站在原地,目瞪口呆,他再也说不出话来!

%19
他的视线越过重重人群,看着方源,看到他微微带笑的嘴角。他心中忽然升起一股寒意:这个人如此怪异偏执的想法,大异常俗,带着自我毁灭的倾向,恐怕是入魔了吧!

%20
不按规矩出牌也就罢了,更可怕的是他不按照规矩去思考。他思考的方式和普罗大众是完全不一样的,他太离经叛道了!

%21
这就是一个魔头啊!

%22
“这是一个真正的魔头!”寒潮族长心中凛然。他觉得这就是方源的本质,哪怕他没有随意大量屠杀过人命,即便方源现在正在做着知恩图报的好事情!

%23
同时寒潮族长感到深深的无力。

%24
若是一个涉世不深的小年轻,他还可以藏匿自己的本意,伪装成前辈,来指点他,教导他,让他知道社会的复杂和某些黑暗的真相。

%25
但方源却是什么都知道,几乎一切都心知肚明。最令人无奈的是,方源的想法和别人完全不同!

%26
“他太有主见了,他太偏执了。他明明只有三转修为啊,怎么敢?不可理喻,不可理喻!他是个疯子,他是个狂人!他太狂妄了,他居然蔑视生死!!对啊……他连死都不怕,还有什么不敢的?世间的一切财富、美色、权利地位,恐怕都不比不上他自身心意上的一丁点的满足!我还能拿出什么样的东西,才能诱惑得住?”

%27
寒潮族长简直要疯了。

%28
他越想越明白,越明白就越方源这个人毫无畏惧,也不接受任何的诱惑。或许有一天,他能被诱惑,但这绝对是他自己想要被诱惑,这是他内心深处的一个真实的心意。

%29
人活在这个世界上,不容易!

%30
鲛人也同样如此。

%31
别看寒潮族长这么位高权重,他更不容易。

%32
他头上还有鲛人圣城的族老会压着他,他底下有那么多的下属,有的再勾心斗角,有的再觊觎他的位置。他子女成群,嗷嗷待哺,妻妾众多,矛盾重生。一切的一切,都需要他监控,都需要他处理,都需要他安排。

%33
他贪腐,有错吗?

%34
没错啊!

%35
什么是贪腐?

%36
贪腐不过就是获得更多的利益,而这些利益让另外的利益既得者感到不公平。

%37
一块蛋糕,原来分配的情况是这样的,你一块我一块,现在我偷偷又拿走第二块,你看着眼红,你说你违背了曾经的分配的约定,你凭什么拿这么多?

%38
这就是贪腐。

%39
你以为支持圣女的大族老一方,就不贪腐吗?

%40
多多少少都会有吧?就算大族老本人不贪,她的那么多的手下呢?她的子女呢?只是程度没有寒潮族长这么严重吧。

%41
或者,就算大族老一方整体上上下下都不贪腐了。那他们也是高层啊,也是吃蛋糕的人,也是剥削他人的人。

%42
从这点本质上,大家都是剥削者,有什么区别么?

%43
一路货色!

%44
所以,寒潮族长从未觉得过自己贪腐有错,他只是想获取更多的财富、美色、权利。

%45
他贪腐越来越多,逐渐超过分配约定。但他不想停下来,心中的贪欲也令他停不下来。

%46
“不,不能说贪欲。应该说是理想啊!”多少次,寒潮族长在心中对自己如此高喊。

%47
有一个不是笑话的笑话——

%48
父亲问儿子:你长大的理想是什么?

%49
儿子答:金钱和美女。

%50
父亲给了儿子一巴掌!

%51
儿子又答:事业和爱情。

%52
父亲微笑点了点头!

%53
所以,事业和爱情是理想,金钱和美女(男)也是理想。

%54
所以,寒潮族长理直气壮,自己追求财富、女色、权位、名利,有什么不对?

%55
你觉得庸俗?

%56
这都是理想!

%57
哪个人的人生不都充斥着这样的理想?!

%58
寒潮族长打骨子里就喜欢这样的理想,因为这样的理想能鞭策他自己,同时也能诱惑其他人,令他们为各自的理想付出和牺牲,然后成全他寒潮族长!

%59
他贪腐,有什么不对,这都是理想!

%60
理想是需要实现的,是需要努力的。

%61
寒潮族长在第一次贪腐的时候,就明白他会有这么一刻,遭受其他人的反对,承受反噬。

%62
但这又如何?

%63
这是应该的,这是必然的,这是在实现理想的路上一定要经受的困难和痛苦!

%64
只要跨越这些困难,克服这些痛苦,寒潮族长就能实现自己的理想。

%65
放在眼前,只要他通过一系列的政治手腕,暗箱操纵、旁敲侧击,陈兵威慑等等,他就能实现自己的理想。

%66
在这方面,久居高位的寒潮族长相当自信。他的确有自信的资本,事实上若是没有方源横空杀出,他已经排挤掉谢晗沫,将自己的人推上圣女的宝座了。

%67
一旦如此,他就击败了大族老,成功地保住了自己贪腐来的胜利成果。

%68
今后再借助圣女傀儡,发布几个政策,美名其曰为了广大的鲛人,为了整个圣城的前景。他将自己的黑钱洗白,将自己的贪腐合法化。

%69
到那时,谁还能说他贪腐?!

%70
但就在寒潮族长快要成功的时候,他失策,彻底挫败了。

%71
因为他碰到了方源。

%72
方源这个人没有“理想”!

%73
不,也不能这么说。寒潮族长坚信,他也很喜欢财富、美色、权位、名利,但他更喜欢的是依凭自己的心意活着!这才是他的理想。

%74
你要这么高大上的理想干什么?

%75
你这样的理想,岂不是显得我们这些绝大多数人很庸俗不堪,很平凡普通么?

%76
你这是在找死啊!

%77
寒潮族长对方源恨得牙痒痒,这种憎恨因为心底的某种秘不可查的恐惧,而更加强烈。

%78
寒潮族长恨不得把方源抽筋扒皮,恨不得他立即就去死!

%79
但他现在不能,现在是海神祭。

%80
最后一首歌曲。

%81
谢晗沫和方源联袂登台。

%82
方源伴奏,谢晗沫的歌声随即飘扬而起。

%83
……

%84
沧海笑滔滔两岸潮

%85
浮沉随浪记今朝

%86
苍天笑纷纷世上潮

%87
谁负谁胜出天知晓

%88
……

%89
人生起伏,就好像那浪潮,有高就有低。成败胜负何必总是记挂在心头呢?

%90
潇洒、浪漫的情怀,一下子就让听者沉醉。

%91
……

%92
江山笑烟雨遥

%93
涛浪淘尽红尘俗世知多少

%94
清风笑竟惹寂寥

%95
豪情还剩一襟晚照

%96
……

%97
豪迈气概、洒脱不羁,红尘中种种“理想”都会被浪花淘尽。就算是生命本身,也会陨落。又有什么大不了的?

%98
君子不役于外物,超然物外,忘怀得失。

%99
命运无常,何必秉持性情,抛弃面具,找寻真我呢。

%100
我自有豪情,我自有寂寥。哪怕是生命中的夕阳,我也有我的精彩。

%101
众人痴了。

%102
寒潮族长脸色惨白,浑身颤抖,他知道自己这一仗是输定了!

%103
……

%104
沧海笑滔滔两岸潮

%105
浮沉随浪记今朝

%106
苍天笑纷纷世上潮

%107
谁负谁胜出天知晓

%108
江山笑烟雨遥

%109
涛浪淘尽红尘俗世知多少

%110
苍生笑不再寂寥

%111
豪情仍在痴痴笑笑

%112
……

%113
我在世俗红尘中摸爬滚打,我出世我入世。我过着我自己的生活,我按照我的心意活着,哪怕浪潮颠簸得我起起伏伏、上上下下、生生死死,我也从不感到委屈哀怨惧怕担忧,我品味此中滋味,我仍旧会痴痴笑笑。

%114
我有真性情。

%115
我是真人!

%116
台上,方源闭上双眼,尽情地催动蛊虫,琴声悠扬。

%117
寒潮族长盯着他,一脸呆滞,口中不住地呢喃:“魔、魔头啊……”

%118
谢晗沫唱得也痴了。她望向方源,美眸中带着一种意蕴非凡的光亮,她在心中痴想:“这样的潇洒,这样的人生,不就是自己向往的么?方源这个人能创出如此曲目,真的是有仙性!”

%119
今生。

%120
龙鲸洞天,鲛人圣城。

%121
海神祭。

%122
夏琳登场,第三首歌。

%123
沧海一声笑,滔滔两岸潮。浮沉随浪,只记今朝。

%124
苍天笑,纷纷世上潮。谁胜谁负,天知晓……

%125
全场震撼,苏怡一脸灰白,结局已经不言而喻。

%126
夏琳已经唱痴了。

%127
楚大师的形象,在这一刻,在她心中无限拔高,带着风光霁月,带着云雾缥缈。

%128
熟悉的旋律在方源的耳畔再次回响,现实和记忆这一刻在他脑海中交织。

%129
他曾经站在台上伴奏,闭目微笑。

%130
他现在站在台下观看,眼蕴幽光。

%131
数百年沧桑,时光的伟力,改变了他,又似乎没有改变他。

%132
他一直是古月方源。

\end{this_body}

