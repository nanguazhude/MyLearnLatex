\newsection{暂时的休整}    %第一百二十七节:暂时的休整

\begin{this_body}



%1
方源的神念在宝黄天中穿梭。

%2
“咦?”他的神念微微一顿。

%3
一个蛊仙的意志,驻留在一边,在其身后是小山的货物。

%4
“这些都是滑土?”方源主动上前询问道。

%5
韩玉的蛊仙意志立即从沉眠中醒过来,见有顾客,连忙回答道:“都是滑土!”

%6
报了价格之后,他又道:“这位蛊仙大人,你是要买多少?”

%7
滑土是一种五转蛊材,土质相当肥沃。

%8
方源心中点头,对方的出价很是公道,甚至比往常市价还要低一些。正巧自己也需要这些土壤,便道:“这些我都要了。”

%9
韩玉意志顿时震动了一下:“你说全买了?我这滑土可是很多的。”

%10
方源确定道:“当然是全买了。两百零二块仙元石,如果我没估算错的话,是这个价吧?”

%11
韩玉意志连忙点头:“是的,是的。”

%12
宝皇天中,任何的货物,都有宝光逸散出来。

%13
方源正是看到了宝光,还有货物本身,立即算出的卖价。

%14
他可是智道宗师。

%15
“交接吧。”方源行事相当干脆。

%16
“好,好。”韩玉意志忙不迭点头答应。

%17
情况的发展,有些让他出乎意料。

%18
他只是一位六转蛊仙,渡过了一次天劫,苦心经营仙窍。这一次将福地中的滑土卖出去,考虑到以往的经验。他还故意将价格压低了一点,没想到竟一下子就被人买光了!

%19
对于他而言,这是个巨大的惊喜。

%20
韩玉意志托着两百零二块仙元石,整个意志身躯都在微微颤抖着,显然十分激动。

%21
“前世的我。不就是这种境况么。”方源将这一幕看在眼底,心中有些感慨。

%22
韩玉无疑是六转蛊仙的普遍代表。方源前世甚至还有些不如他,毕竟修行的是血道。

%23
但今生不一样了!

%24
单单胆识蛊的买卖,就是垄断贸易,源源不断的利益来源。

%25
吞下了黑凡洞天的资源之后,方源的身家更是暴涨了一个骇人听闻的程度。别说是两百块仙元石,就算是拿出两千块来。也是眼睛都不眨一下的。

%26
总之。方源的经济状况是相当的好。

%27
买下这批滑土,方源又驾驭神念,逡巡几圈,分别又买下其他一些土壤。

%28
达到目的之后,方源便抽回神念,将这些土壤带回了至尊仙窍。

%29
他操纵数个力道仙僵,将这些土壤全都洒到至尊仙窍中的小南疆里去。

%30
小南疆中的情景。已经发生了翻天覆地的变化。

%31
一片片的森林,让浓郁的绿意覆盖了这片大地。参天的古木,高大的松柏,笔直的杨树,低矮的灌木丛,还有数不尽的青草和野花,一下子让小南疆变得郁郁葱葱,生机勃勃。

%32
这些还只是很平凡的植株,在这里面还夹杂着大量的上等资源。

%33
比如数目最多的曲丽木!

%34
这种树木根枝缭绕,别有风情。树枝上。常常停留着各种鸟类,最多的一种是气死鸟,引吭高歌,叽喳争鸣。

%35
原本的小南疆,一片寂静,现在却热闹无比。

%36
除了鸟类,森林中还有大量的鹿、松鼠、野兔、猴群、蛇、虎等等。交织成一个相互依附的循环生态。

%37
原先的小南疆,只是有些小土丘罢了。

%38
现在,直接增添了十几座山峦。

%39
这些山峦都是方源从黑凡洞天中,直接拔过来的。

%40
寻常的蛊仙没有这种移山的手段,不过他的拔山仙蛊,却正好能用在这个上面。

%41
目前山峦中,最高大的无疑是继仙山。

%42
山上还有石亭,方源都没有拆卸,只是石亭中的传承都被他收走了。

%43
忙活了好半天,方源操纵的仙僵才将所有的土壤,都洒到计划中的位置上。

%44
方源又视察了一会儿,这才满意收手。

%45
“有了这些土壤,应当能够让这些植株,好好生长一段时日了。”

%46
方源的至尊仙窍中,土道道痕还是稀少,内部空间又大,有五域九天,分到小南疆这块的土道道痕就更少了。

%47
土道道痕稀少,方源的至尊仙窍中,没有山峦丘陵等等,土壤就不厚实,也不肥沃。

%48
小南疆的这些草木森林,几乎全部都是来源于黑凡洞天。

%49
在黑凡洞天中,这些树木都生长得很好,密密麻麻,绿意深深。

%50
但是移栽到了方源的至尊仙窍中,因为土道道痕稀少的缘故,导致这些草木都干瘪下来,一副营养不足的样子。

%51
方源只好收购大量的肥沃土壤,比如滑土,洒在小南疆中,暂时应付局面。

%52
“但此法治标不治本!”

%53
“这些土壤过段时间,就会沦为普通土壤,里面的土道道痕都要消散。”

%54
这些土壤充其量只是凡级蛊材,内蕴极其残破的土道道痕。被草木汲取了营养之后,这些零碎的土道道痕,绝大多数都会彻底消散掉。

%55
“唯有通过渡劫,或者吸纳土道蛊仙的一身道痕,增长仙窍中的土道道痕,才算是根治之法。”方源心知肚明。

%56
第五次地灾,方源不可能再到北部冰原渡劫了。这个时候,就可以选择去南疆,或者其他土道道痕浓郁的地方渡劫,希望能够遇到土道地灾。

%57
一旦碰到土道地灾,那么渡过之后,方源的这个难题就会立即得到缓解。

%58
至于杀死土道蛊仙,吸纳道痕,并不容易。

%59
修为低的土道蛊仙,身上的道痕并不多。修为高的土道蛊仙。战力往往强大,方源要对付他们,风险很大。

%60
至今为止,方源已经渡过四次地灾。

%61
第一次是雪怪成灾,还有铁冠鹰、墟蝠等变化。第二次是风花劫、雪月劫,第三次是春晓翠鹂、巨祸焚木劫,第四次是玄白飞盐劫。

%62
因为至尊仙窍超越了十绝合体,为天意所忌,所以几乎每一次地灾,都提升到了遵循天道限制下的最大威力。

%63
这种地灾的难度,要大大超越寻常蛊仙数倍。甚至十几倍。

%64
所以。方源得到的道痕,也是超越寻常收获,达到十几倍的程度!

%65
因此,方源身上的冰雪道痕最多,之后是变化道、力道、气道、律道等,运道、血道也有一些。

%66
整个至尊仙窍,因为夺了黑凡洞天的资源。实现了大跨步。

%67
但因为空间太广,目前只是小南疆比较富裕,其余地方虽然也有一些资源,但没有小南疆这般健全的生态。

%68
荒兽有不少,巨角羊、鱼翅狼、龙鱼、鹰犬等。上古荒兽有天残犬、落星犬(幼体)。太古荒兽都有,那就是上极天鹰了。不过此时,它已经成了一个鸟蛋。

%69
其余的荒植也有一些,上古荒植有一株,那就是走肉树了。

%70
综合来讲,方源的资源积累。已经和七转巅峰的蛊仙差不多了。楚度这种人物,估计也就方源这种程度。至于凤九歌这种,被超级势力重点培养的,方源还略差一筹。

%71
撒完土壤,方源抽回心神,继续对蛊仙俘虏的魂魄进行搜魂。

%72
方源仙窍中的蛊仙俘虏还真不少。

%73
蛊仙魂魄已经有了一大堆。

%74
这一次搜魂的重点,是那黑凡洞天中的九位蛊仙的魂魄。

%75
冯军的运道真传。还有陈乐的隐身手段,都比较出色,方源很感兴趣。

%76
“你不是黑城!你究竟是谁?”

%77
“你这个阴险小人,有种就让我魂飞魄散!”

%78
“饶命,饶我一命,只要你让我重生过来,我愿成为你最忠心的奴仆!”

%79
这些蛊仙魂魄,有的质问,有的怒骂求死,有的则想要求存。

%80
方源在他们面前,没有再伪装成黑城,毕竟见面曾相识也要耗费仙元,态度蛊也会耗损他的心力。

%81
对这些蛊仙的喊叫,方源不闻不顾,只是一味的搜魂。

%82
每天,他都会搜魂一段时间。

%83
“再有三天,这些魂魄就会被我完全搜刮记忆,让我得知他们的一切了。”

%84
搜魂完毕,方源整理了一下此次收获,便彻底抽回心神。

%85
他离开云城,动身前往落魄谷。

%86
在那里,他遁出魂魄,进行艰辛刻苦的魂道修行。

%87
方源利用落魄谷、荡魂山已经有一段时日了,他的魂魄底蕴已经增长到了之前的数十倍!

%88
但是,将自己的魂魄落到原来的仙僵之躯中,方源还是没有底气。

%89
“这个难题,是最麻烦的。”

%90
“魂魄底蕴增加,其实也难以避免影宗的可能手段。”

%91
“说不定,影无邪布置的手段,是根据魂魄的底蕴而论威能。魂魄底蕴越强,影宗的陷阱就越加厉害。”

%92
“影无邪知道我有落魄谷、荡魂山,布置陷阱的时候,不可能不考虑这些因素。”

%93
每天的魂魄修行完毕后,方源魂魄重回肉身,微微摇晃地站起身来。

%94
他脸上神情有些忧郁。

%95
只要他能够再用智慧蛊,那么他的整体实力,就会有爆炸性的增长!

%96
很多发展中面临的问题,都和智慧蛊有关联。

%97
只要智慧蛊能够运用,很多难题和关隘,都能迎刃而解。

%98
“智慧蛊有,力道仙僵之躯也在,就卡在最后的一点上!”方源在心中无奈叹息。

%99
就这样,方源在琅琊福地待了七天。

%100
他勤修苦炼,没有一分一秒的放松。

%101
魂魄底蕴又增长了一截,九位蛊仙俘虏的魂魄都彻底被搜空,方源还用了大半的琅琊门派贡献,换取了一些琅琊天晶之外,他还自己发布了一些炼蛊任务。

%102
我意蛊、年蛊、月蛊、日蛊,以及其他种种凡蛊,他需要的数量很大。

%103
这些凡蛊他宁愿耗费门派贡献,让毛民蛊师、蛊仙为自己炼制。毕竟这能节省他大量的时间和精力。

%104
七天之后,他迅速动身,离开琅琊福地,直接前往东海!

%105
ps:本来是打算两更的,因为上个月的月票突破800,还欠加更一章。但今天实在不在状态,只能一更了。争取明天双更吧!

\end{this_body}

