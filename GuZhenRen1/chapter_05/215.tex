\newsection{修行本身就是享受}    %第二百一十六节:修行本身就是享受

\begin{this_body}

药皇的态度非常坚决。[八零电子书wWw.80txt.com]

百足天君笑了笑:“这一点,仙友放心。楚度早有觉悟,已经和我沟通过,在这一点上,他可以做出让步。”

“这就好。”药皇点点头。

双方又交流了一阵,棋局下了一小半,时间打发得差不多,两仙便一前一后,飞下白天。

随后两人当中宣布,刚刚的对决中,药皇优胜一手,此次血战武斗大会,以黄金家族的联合势力获胜告终。

楚门、百足家的蛊仙们,一个个脸色铁青,惟独楚度面色平静,早有估算。

反观黄金家族的蛊仙,则一片欢庆氛围。

之后的胜负处决不做详述,震撼北原蛊仙界,甚至影响波及其他四域的血战武斗大会,至此落下帷幕。

南疆,超级梦境。

一片深幽的海底。

黑暗,伸手不见五指。

方源沉浸其中,身上伤势极重。

海底中压力重重是,在他周围的海水中,有无数道修长的蛟龙身影,在他周围逡巡游曳。

这些都是海中生活着的蛟龙。

每一头海蛟,都有荒兽级的战斗力。

密密麻麻,层层叠叠的海蛟,数量成百上千。

方源此时此刻扮演的人物,已经成为六转蛊仙,但面对如此之多的海蛟兽群,也是无法匹敌。

“更要命的是,要通过这第九幕梦境,必须得杀死所有的海蛟。”

方源在心中苦笑。

若是放在现实当中,他连杀透重围,突击求生的可能性,都是相当的渺茫。

而在这个梦境中。自有它的规则。方源若是要突围求生,就是探索梦境失败。

这根本不给方源活路!

“想要除掉这些海蛟,单凭梦境中的底蕴和手段,是根本不可能的事情!”

所以,方源若是要打通此幕,摆在他面前的选项就只有一个。一那便是仙道杀招解梦。

“但是如此数量的海蛟,我的解梦杀招只能催发十次不到,根本不能解决眼下如此的大麻烦。”

“看来这个场景,恐怕是梦境主人的最后的修行生涯了。”

方源心中产生了一股明悟。[八零电子书wWw.80txt.com]

梦境分门别类,大多是都是写实梦境,是还原了某个生命的生涯当中的幕幕种种。少部分则是稀奇古怪,诡异怪诞的梦境。

方源探索的这个梦境,显然是写实梦境,可能在原有的事实程度上。有所夸张。但却也彰显了曾经发生的一部分事实。

“这位蛊仙曾经是战仙宗的蛊师。”

“天资不错,但升上六转之后,和周围的蛊仙相比,就相形见绌了。”

“中洲十大古派,又是按门派贡献,规划修行资源。这位蛊仙天赋资质差强人意,战力不强,手段并不高明。很难有什么门派贡献,就更难获取修行资源。所以。才想到去往东海探索。”

“结果,先是在通过五域窍壁的时候,折损了仙窍根基底蕴。后来听信什么仙蛊屋龙宫的谣言,前来这片海域探索,结果遭受到了海蛟兽群的围剿,丧命于此。”

想到这里。方源终于做出了决定,选择了放弃。

他主动离开了这片梦境。

一般而言,这种梦境探索的越是深入,突破一幕的收获,将远超之前。

但方源明智决断。选择了知难而退。

他保留了手中的解梦杀招,并没有放到这里作无谓的浪费。

片刻后,方源到了现实。

利用胆识蛊疗伤,又检查了一番自身状况。

水道准宗师。

这个境界,还是没有变动。

距离真正的宗师级,还有一定的距离。

不过,已经足够吞并市井中的许多仙窍福地了。因为大部分,都是六转水道福地。

“按照现在的情形,最好是平均提升,先将所有的流派境界,至少提升到大师。”

“如此一来,我就能吞并任何的六转福地了。”

“虽然六转福地对我的修为提升,帮助已经越来越小。但吞并的数量提升上去,效果也会比较可观。”

方源琢磨着。

接下来,他主要是投入自身的梦境之中,炼一些梦道的凡蛊。

当然,也会抽出一些时间,进行推算。

这些天来,他有了一个灵感,打算将万我杀招和剑蛟变化结合起来。

推算的进展,一直比较顺利。方源的力道境界足够,变化道也是宗师,唯一美中不足的智道推算的手段稍微欠缺了一点。

不过这无伤大雅。

白兔姑娘前来汇报这段时间,仙缘生意的状况,以及她搜集到的一些情报。

方源盘坐在蒲团上,闭上双眼对她道:“你说你的,我听着。”

方源一边推算自己的东西,一边倾听白兔姑娘的话。

他一心二用。

智道宗师的他,做到这一点并不困难。

白兔姑娘一边叙述着,一边望着方源的面庞,目光中尽是深情。

但是让她失望的是,一直到她汇报结束,方源都没有睁开双眼再看她一眼。

“好了,你可以退下了。”方源开口。

白兔姑娘只好满怀失望,带着留恋的神光,慢慢退出这里。

殿门外,武安早已经恭候多时。

“姑娘,这边请。”武安对白兔姑娘的态度,比之前更要恭敬很多。

白兔姑娘只要得到方源的接见,就证明方源对她恩宠未断,这层蛊仙,就让武安对白兔姑娘极为重视。因为白兔姑娘可是关乎仙缘生意的大计!

白兔姑娘有些担忧,问武安道:“武安,武遗海大人一直都是如此修行吗?”

武安楞了一下:“此话何意?”

白兔姑娘幽幽地叹气:“武遗海大人修行起来,真的是太刻苦了,几乎一分一秒都不想浪费。就连听取我的汇报,他都未停止修行。他是一直如此吗?”

武安眨了眨眼:“姑娘,在下实话实说。我虽然没有亲眼见证过武遗海大人如何修行的,但管中窥豹,也可见一斑。武遗海大人恐怕是我见过的,最为刻苦的修行者。别的不提,单单他将自己关在大殿中,足不出户这么长的时间,我就万万做不到。”

白兔姑娘脸上的担忧之色更重一分:“正因如此,我才担心武遗海大人。他修行起来,就好像是在拼命!修行讲究张弛有度,长期以往,我担心武遗海大人会支撑不住,坏了心境。武安,你若是有机会,一定劝劝武遗海大人,好么?”

武安心说:“你见面的次数和时间,都比我还多。我怎么劝?”

不过,他还是点点头,应承下来:“如果我有机会,一定会努力劝说武遗海大人的。”

方源分析了一下白兔姑娘的情报。

这让他对整个南疆的局势,一直保持在比较清晰的把握之中。

总体而言,南疆的局势还是安稳的。

虽然武独秀逝世,武家当中只剩下了一位八转蛊仙武庸。但此人接手了武家之后,陆续展现出了强大的战力,以及巧妙的政治手段。

正因如此,即便是巴家、铁家等虎视眈眈,也不得不暂时按捺住阴暗心思。

八转的对峙格局如此,自上而下,反映到南疆的各个方面。在超级蛊阵当中,也是局势平缓。仙缘生意在坐着,有的蛊仙受益,闷声发财,乐不可支,有的蛊仙冷眼旁观,暂时不做理睬。

“北原的血战武斗大会,已经落幕了,结果是楚门整改成楚家,百足家保存下来,黄金家族的联合大胜,获取了大量的修行资源。”

“这样,北原的大局也趋向稳定了。”

“东海、西漠以及中洲,我得到的情报泛泛,基本上都是来自于宝黄天,只有一个模糊的概念。不过也都是比较安稳平和的。”

事实上,安稳平和才是五域大局的常态。

谁好端端的安稳日子不过,要去打生打死的?

北原的动乱,其实源头还出自方源身上。是他捣毁了八十八角真阳楼,其中夹杂影宗图谋,最终引起了一系列的恐怖影响。

一直到五域乱战的时期,整个五域方才以战乱为主,一切秩序都趋向于崩溃,争斗成为了蛊仙修行的主旋律。

外部条件是非常宽松安定的。

对于方源而言,这是最好不过的事情了。

他沉入自己的梦境之中,开始炼制梦道凡蛊。

青山葱茏,商队在山脚下进行暂时的休整。

咻。

一道月光,在空中灵活地转了一个弯,绕过了正面的石头,准确地击中石头后的树干上。

顿时,木屑纷飞,略显纤细的树干,差点被月光斩断。

“练习了这么多天,终于成功了!”方源满身大汗,心中却是无限的欢喜。

“小子,挺不错的嘛。”胡子蛊师走到方源的面前来。

“胡子大叔,谢谢你教我这一手啊!”方源大笑一声,露出满口洁白的牙齿,阳光开朗。

胡子蛊师似乎都被这种阳光,晃了一下眼睛。他有些疑惑:“一般人,保持这种训练的激情,七八天已经不错了。你小子却起早贪黑,只要一有时间,就进行训练,足足持续了一个多月,并且热情一点都未衰减。这样枯燥的修行,真的这么有趣吗?”

方源握紧双拳,双眼闪亮,宛若晨星:“当然了!大叔,你不觉得很厉害很神奇吗?一个人挥挥手,就能发出绚烂的攻势。修行本身就是很享受的事情啊。”(未完待续。)

想友一下手机访问.<!--80txt.com-ouoou-->

\end{this_body}

