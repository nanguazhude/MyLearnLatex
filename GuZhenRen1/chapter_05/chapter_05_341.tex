\newsection{紫山真君战龙公}    %第三百四十一节:紫山真君战龙公

\begin{this_body}



%1
咻、咻!

%2
杀机凛冽,两道身影,在梦境的夹道中,飞速地穿梭。

%3
紫山真君战龙公!

%4
两者速度极快,宛若飞箭暴射,偏偏又灵活至极,宛若游蛇蜿蜒。

%5
轰轰轰!

%6
双方交手,爆发出连绵不绝的雷霆声响。

%7
一波波的气浪,澎湃汹涌,但是落入附近的梦境中后,就迅速消弭。

%8
两位八转蛊仙开始交手,一位是魔尊幽魂当年分出的第一代分魂,修行了近十万年。另一位则是历史上隐姓埋名,却有着准九转实力的天庭传奇。

%9
毫无疑问,两人的交手,是整个超级梦境战场中,最为核心,最为重要,也是最为激烈的战斗。

%10
监天塔和左夜灰的沉闷对撞,和这两人的激战比较起来,就显得次要了。

%11
紫山真君双目闪烁着琉璃紫光,忽然双臂平直伸展,身体表面骤然漂浮起无数火炭一般的小石子。

%12
石子飞出去,飞在半空中,越变越大,同时石块表面,燃烧起赤红的火焰。

%13
数以千百的火石,朝龙公激射而去,如火雨倾盆。

%14
龙公猛打猛冲,竟不避让。

%15
火石撞击到他的身上,被他全身环绕着的一层云光遮挡。

%16
云光岿然不动,但很快,龙公发现自己的脑海中,燃烧起了奇妙之火。

%17
他的念头升腾起来,这些火焰就将这些念头烧毁。

%18
云光需要龙公时刻维持,念想不断。此刻相关念头被毁,立即让他的防御丧失。

%19
火石接踵而至,狠狠地撞击在龙公的肉身之上。

%20
一时间砰砰乱响,火石宛若鸡蛋,龙公矫健高大的身躯宛若金钢,火石四分五裂,自我爆碎,无数火星和石块四下飞溅。

%21
龙公继续横冲直撞,他的身躯本身有恐怖的道痕积累,同时还有类似于碧晨天的木甲、方源的鬼不觉之类的防护手段。

%22
他气势惊人,就像是人形的蛮龙,顶着紫山真君的狂轰滥炸,勇往直前。

%23
一对龙瞳仍旧冷漠,龙公的脸上毫无动容,冰一样的神情。

%24
他的眉头都不皱一下,紫山真君的仙道杀招,仿佛是轻风淡云,不值一哂。

%25
龙公表现得非常强势,紫山真君乃是智道蛊仙,则讲究策略,边打边退。

%26
火石奈何不得龙公,紫山真君却嘴角微翘。

%27
他忽然伸出左手食指,对准追杀而来的龙公,遥遥一指。

%28
忽然间,百花齐放。

%29
在龙公的身体表面,莫名其妙地生长出无数鲜花。

%30
鲜花烂漫,数以千百,长满龙公全身。

%31
龙公身强体壮,防御高得吓人,但百花盛开,他的脸色终于微微一变。

%32
“连招么。”

%33
首次,龙公停下了追击的步伐,猛地悬浮在半空中。

%34
他从高速的飞行当中,瞬间变成静止,显示出令人叹为观止的移动手段,已经强劲的身体素养。

%35
鲜花在龙公的身上,接连盛开,越来越多。

%36
花瓣娇嫩,好像脆弱不堪,但却带给龙公巨大的麻烦。

%37
龙公伸出右手,刚开始虚握成拳,酝酿了一息时间,随后,忽然张开五指。

%38
蓬。

%39
一声轻微的炸响,从他猛地张开的右手掌中,飞出无数道微弱的淡白气流。

%40
这些气流,好像是一道道的弧线,飞在半空中,迅速地围绕着龙公的全身,不断旋转。

%41
一道道的气流,仿佛是锋锐的刀刃。不断地扫除他全身表面的朵朵鲜花。

%42
一时间,花瓣飘飞,花丛凋零。

%43
气流看似羸弱,其实锋锐至极,几个呼吸的时间,就将龙公全身上下的花朵铲除干净。

%44
紫山真君见到这一幕,瞳孔微微一缩。

%45
龙公的这一杀伐招数,非常锋利,并且气流多重,操纵起来需要极高的造诣和技巧。

%46
原本只是对敌的攻伐招数,但龙公操纵起来,却是对自己施展。稍有不慎,就会导致自己受伤。

%47
但龙公艺高人胆大,这一招熟练到了骨子里,仿佛如臂使指,许多气流就直接擦着他的肌肤、衣服的表面而过,偏偏对后者毫发无损,分寸掌握得极其到位,简直是妙到了毫巅。

%48
不过,趁此机会,紫山真君也已经酝酿成功。

%49
仙道杀招有催动的难度。如果赵怜云置身紫山真君的这样的处境,很可能催动杀招失败。

%50
但紫山真君老道无比,也稳定非凡,他抓住了这次良机,没有浪费。

%51
一记仙道杀招,催发出来。

%52
幻影重生,围绕着紫山真君本体,顷刻间形成了数十个紫光幻影。

%53
这些幻影还在分化,一化四,四化十,瞬间形成了一股大军。

%54
紫山真君身形隐匿,紫光幻影大军扑向龙公。

%55
龙公冷哼一声,忽然右手变作龙爪,猛地一挥。

%56
空气中顿时浮现出一道道的爪痕,这些爪痕长达数丈,所到之处,紫光幻影无不分崩瓦解。

%57
随后,龙公的左手也化为龙爪,左右开弓,爪痕蔓延,将空间都要抓出裂痕来。

%58
紫光幻影被很快清空,龙公瞳眸骤亮,绽射奇光,罩住一片隐藏空间。

%59
紫山真君立即知道自身行藏已被识破,浮现而出。

%60
双方于空中站定,背景是无数层层叠叠的错乱梦境,彼此间距离约有千步。

%61
“看来你这个老古董沉睡了这么多年,却也能与时俱进。”紫山真君淡淡地道,他是智道蛊仙,此刻长袍鼓动,紫发摇曳,目中蕴神,风姿上佳。

%62
“最近这段时间,练习了一下。时代在不断发展,这一百多万年真的是涌现了太多的人才,发展出了太多的精妙手段了。”龙公语气感慨。

%63
他叹息一声,继续道:“好了,这些小手段就不要使了,试探结束了,接下来,拿出真功夫吧。”

%64
紫山真君微笑了一下,点点头:“理应如此。”

%65
下一刻,两人身形如电,猛地冲撞在一起。

%66
轰轰轰……

%67
璀璨绚烂的光影色彩,像是泛滥的烟花,迸发出来,映照得周围梦境,更加迷离缤纷。

%68
“这两个家伙,还真是……”依靠着超级蛊阵,方源对大半战场都有直观的了解。

%69
观察了一阵子紫山真君和龙公的交手,方源有些无语。

%70
“两人都是八转蛊仙,而且不是寻常八转,真的是太强大了!”

%71
“通常而言,蛊仙初次交手,都是用凡蛊、凡道杀招。真正互拼,才用仙蛊、仙道杀招。这两人却直接用了仙道杀招,相互试探。”

%72
“这两位至少都是大宗师境界,不管是火石、鲜花,还是气刃、云光等等,都是触类旁通的表现。”

%73
“我若不用逆流护身印,一招之内就是惨败的下场。”

%74
“好激烈!”

%75
“这两人的战斗频率和火力,竟然双双暴涨了一大截!”

%76
情报对于蛊仙而言,非常重要。

%77
方源借助超级蛊阵,观察了一会人,便收获良多。

%78
不管是紫山真君,还是龙公,方源都不是对手。就算他有逆流护身印,也只能被动挨打而已。

%79
事实上,北原逆流河一役,方源能够逃出生天,还多亏当时混乱的局面。

%80
最终,方源借助界壁摆脱了追兵——狗尾续命貂毛里球。

%81
“照自己的实力,要在这样层次的混战中脱身,非常困难,希望渺茫得很!”方源眉头拧成了一个疙瘩。

%82
就在这时,他忽然神情微变。

%83
原来是仙窍中,一只信道蛊虫中,传达过来武庸的消息。

%84
“武庸并没有死,乔志材、铁面神也安然无恙,他们只是被困住了一段时间而已。现在正利用一座仙蛊屋,赶来这里!”

%85
武庸传信,让方源支持住,尽量保全自身性命,哪怕临阵脱逃也无所谓。

%86
他还告知方源,不仅是他这一边,池家太上大长老池曲由也驾驭仙蛊屋,赶来支援,只是后者的距离,比武庸这一路还要更遥远些。

%87
不只是这两路,其余的各大超级势力,也都纷纷派遣了仙蛊屋。

%88
南疆正道超级家族们,虽然之前相互对掐,但听闻了武庸的通告和超级梦境这边的战况后,立即偃息了相互之间的战斗,众志成城,联合起来,向超级梦境这里进发。

%89
若是鸟瞰整个南疆地图的话,义天山遗址位于南疆中心偏左的位置。而漫布在南疆各处的正道超级势力,纷纷派遣援兵,从四面八方赶来支援。

%90
方源脸上神情有些复杂。

%91
若是他真的是武遗海,此刻听到这个消息,必定欢欣雀跃,士气大涨。

%92
但他不是。

%93
他是方源。

%94
一旦南疆援兵到来,天庭绝对会暴露他的身份,引发混乱,让方源孤立无援。

%95
“至于武庸……”方源心底叹气。

%96
这个哥哥,他曾经最大的政治背景,搞不好就会成为今后最强大的追杀者。

%97
方源亲手杀掉了真正的武遗海。就算武庸和武遗海之间,兄弟感情极为薄弱,武庸不想报仇,但碍于正道的身份,他必须得复仇,斩杀方源。

%98
可以说,方源和武庸之间,矛盾无可调和。

%99
“武家居然还有一座仙蛊屋!武庸真是能隐忍。他来的很快,不过就算最近的一路援兵,最快也得至少一天时间。”

%100
方源心中迅速盘算。

%101
援军到来,天庭暴露了方源的身份,引发混乱之后,他能否借此脱身?

\end{this_body}

