\newsection{艰难渡劫}    %第七十六节:艰难渡劫

\begin{this_body}



%1
风花劫,雪月劫,单个而言,都只是一种地灾。

%2
风花、雪月相互叠加,就成为了一种天劫。威力暴涨一大截,渡劫难度剧增。风花速度极快,难以躲避,本身有锋锐至极,实难抵挡。而雪月散发冷光,冻缓蛊仙,两者之间几乎是完美搭配。

%3
方源心中沉重。

%4
其实以他如今的实力,尤其是和影宗交易之后,力道、剑道成为了他的主要依仗。不管是力道大手印,还是剑浪三叠,都能让他爆发出匹敌七转蛊仙的战力。渡过天劫,也是有可能的。

%5
让方源心情沉重的,不是眼前的天劫,而是今后的发展。

%6
蛊仙渡劫,一次比一次艰难。方源的第二次地灾,理论上,要比第一次威力更强。

%7
但第二次地灾,就提升到天劫的档次上,这增强的幅度,未免也太过巨大了些。

%8
以此推论下去,第三次地灾、第四次地灾,又该增强到何种程度呢?

%9
一位六转蛊仙,十年一地灾,百年一天劫。历经三百年,三次天劫之后,会晋升成为七转。

%10
统计一下的话,就是十八次地灾,三次天劫。

%11
就方源而言,不算第二次地灾的话,他还有十六次地灾,三次天劫需要渡过。

%12
若是第二次地灾,他的灾劫就暴涨到了天劫。那还能谈什么未来?推算下去的话,地灾的威力都能上涨到浩劫的层次了!

%13
雪月的冷光,照得方源一脸的苍白。

%14
就连脑海中的念头,运转、碰撞的效率都随之下降了。

%15
雪月散发出来的冷光,不仅是延缓蛊仙行动那么简单,它还能影响蛊仙的思考。

%16
方源连忙调动智道手段,护住脑海。

%17
他是智道宗师,这点程度很容易就能做到。

%18
“不应该啊。影宗的交易中,毛六明确地告诉过我地灾虽然一次比一次强,但增幅的程度有限。只要是地灾,仍旧会局限在地灾该有的程度。难道说。他是在欺骗我?卖给我假情报?”方源眉头紧皱。

%19
若是欺骗,自然不能排除这种可能性。

%20
但方源换位思考一下,这种欺骗对影宗一方毫无益处。

%21
因为这是轻易就可得证的东西。关键是影宗一方还想要擒捉方源,逆炼至尊仙胎蛊。坑方源这一下,让方源死在灾劫下,岂不是糟糕了?

%22
“或者说,还有另外一种可能……”方源眼中精芒爆闪一阵后,又沉凝下来。

%23
外界的风。呼啸滚荡。

%24
方源仰起头,远望天空,眼眸漆黑如夜。

%25
站在山石上,他的白袍和黑发,在风中摇曳甩摆,他好像是一头孤狼。

%26
剑浪三叠!

%27
力道大手印!

%28
忽然间,他展现出强大的攻势。

%29
璀璨的剑浪凭空而生,汹涌澎湃,逆天倾泻。而力道大手印,轰破空气。势沉力大,排开漫天狂风。

%30
两者都向半空中的雪月扑去。

%31
叮铃铃……叮铃铃……

%32
无数的青色风花,像是闻到了花香的蜂蝶,纷纷照准剑浪和大手印,围拢而来。

%33
锋锐至极的风花,和剑浪、大手印绞杀在一起。

%34
很快,就将这两股攻势清除一空。

%35
奇怪的现象。

%36
之前的风花,只管对付荡魂山。想要将这座名闻天下的天地秘境,给消磨毁灭。而对方源发出的剑浪和大手印,都不管不顾。甚至主动避退。

%37
但现在,这些风花却主动上前绞杀。

%38
皆因方源发出攻势,针对的目标,就是半空中的雪月。

%39
好像是这些风花。要保护雪月似的。

%40
雪月的存在,的确是带给方源巨大的压力。风花保护雪月,似乎并无不妥。

%41
但方源的脸色却闪过一抹振奋之色。

%42
万我!

%43
万我!

%44
万我!

%45
人群呼啸,一大群一大片的方源虚影,像是决堤的洪水,迅速铺散出来。

%46
方源一连催发了三次。刹那间,空荡荡的荡魂山上,变得人山人海。

%47
无数个方源,又铺天盖地似的,飞入风中,跳下山崖。

%48
但和上一次不一样,这一次方源的本体真身,竟然也变作其中一道力道虚影,冲出了荡魂山。

%49
大量的风花绞杀过来。

%50
但方源的力道虚影数量实在太多了,风花的数量反而显得少了。

%51
但方源的力道虚影几乎无法和风花对抗,大量的力道虚影,迅速牺牲。

%52
靠着周围力道虚影的巧妙掩护,还有见面曾相识、暗渡等等辅助,方源顺利地冲上半空。

%53
“是时候了。”

%54
剑遁!

%55
方源瞅准时机,忽然使出剑遁仙蛊。他就像是一头江中的白蛟,矫矫不群,忽然一飞冲天,冲破江面,腾云驾雾,惊艳世人。

%56
离得近了。

%57
方源再度施展剑浪三叠和力道大手印。

%58
天意猝不及防,风花还在剿除剩余的方源虚影。

%59
方源顺利地将半空中的雪月击爆。

%60
下一刻,大量的狂蛮真意灌溉下来。

%61
方源像是久旱逢甘霖的枯木,将所有的真意都全盘接受。

%62
恍惚间,他像是化身成月,悬挂于空,静静地散发出月光,俯瞰人世间种种。

%63
变化道的境界,迅猛暴涨!

%64
变化道,不仅是变化动物、植物,还包括山水风月等等天地万物。

%65
这一股狂蛮真意让方源获益不浅。

%66
他哈哈大笑起来,笑声给人酣畅淋漓之感。

%67
原来,所谓的“风花雪月劫”只是两两拼凑,并非是完整的天劫。

%68
第二次地灾,真的还局限在地灾程度。就算是天意如何愤恨,如何想要铲除方源,也是无用。它是天意,必须遵守天道的平衡。

%69
狂风怒吼,天意震怒,带着被识破的恼羞成怒。

%70
无数的风花,向方源袭杀而来。

%71
方源再度施展万我,大量的方源虚影,分散了天意的火力。

%72
风花绞杀方源虚影,这是一场货真价实的大屠杀。

%73
但收效甚微,方源真身隐藏在重重虚影之中,最终重新落到荡魂山上。

%74
有了荡魂山,就是一处可以固守的地方,能让方源有一丝喘息之机。

%75
撤销见面曾相识,方源身躯微微摇晃了一下。

%76
他后背有一道深深的伤口,鲜血淋漓,深可见骨。

%77
但是因为变形仙蛊的关系,使得他的伤口也伪装成功,欺瞒了天意。

%78
“防御还是薄弱了。”方源叹息一声,立即对自己展开疗伤。

%79
他没有动用人如故。

%80
人如故是仙蛊,滥用耗费仙元。而且人如故只能回复一瞬之前的状态。

%81
方源是在半空中被一片风花花瓣扫到,受了伤。

%82
那个时候,也不能催动人如故仙蛊。用了仙蛊气息爆发出来,就是暴露自己身份。

%83
“幸好我的身体道痕之间互不干扰,即便是凡道手段治疗,也能很有效果!”方源越发感到自己在这方面的巨大优势。

%84
通常而言,蛊仙疗伤很是困难。因为他们身负道痕,道痕阻碍,会使得凡道杀招的效果大大减弱。仙道手段疗伤,也得考虑到道痕之间,是不是相互排斥。

%85
方源就没有这方面的阻碍,并且动用凡道手段,还有一个好处,就是节约仙元。

%86
这一次渡劫,方源的仙元储备并没有上一次多。

%87
最大的原因,就是宝黄天关闭。

%88
冷意袭来,一片片寒霜从方源的眉眼间,蔓延开去。很快,他浑身上下都覆盖了一层寒霜。

%89
不仅是他,还有整个荡魂山也是同样遭遇。

%90
方源抬头一看,瞳眸微缩,暗叫不好。

%91
原来这天空中又生出雪月,并且数量多达三个!

%92
三倍的冷光,相互叠加在一起,冷意更加沉重。方源原先护持脑海的智道手段,也不顶用了,只好再用其他手段补救。

%93
另外肉身也要护持,数种凡道防御杀招撑起来,身上连续闪烁起数道光辉,终于维持住体温。

%94
雪月的数量,和衍生的速度,都大大出乎方源的估计。

%95
这使得他回到荡魂山喘息的打算,完全落空。

%96
方源心跳加速,当他看到三个雪月的时候,他就意识到:这是生死存亡的关头!

%97
拖得越久,雪月越多,成功渡劫的希望就越加渺茫。

%98
他必须立即反击,不能有任何懈怠之意。否则时机稍纵即逝,稍有差池拖延,就如温水煮青蛙,眼前的困境就会化为绝境,再无任何生机可言!

%99
换做其他蛊仙,兴许还会犹豫。但方源战斗经验丰富,意识到自己的处境之后,他毅然再次施展万我!

%100
故技重施,大量的方源虚影四下奔走,吸引走了大量的风花。

%101
方源真身夹杂在其中,义无反顾地冲上天去。

%102
一飞上空,冷光的威能迅速拔升,再不受荡魂山上的魂道道痕压制。

%103
这是一场激烈的交战!

%104
方源要击爆雪月,三倍冷光带给他的防守压力,比风花更大。更要命的是,第四轮雪月正在半空中缓慢成形。

%105
而天意则千方百计阻止方源,并且趁着他主动出击的绝世良机,大量的风花飞舞,企图将方源当场斩杀。

%106
一轮,两轮,三轮弯月,被方源接连击爆,过程艰难。

%107
而更多的雪月,也在同时形成。

%108
冷光照得方源心中一片冰冷,残酷的现实让他四肢发寒。

%109
生机变得越发渺茫,方源脸色坚硬如铁,唯有挺起胸膛,迎难而上。

%110
他的仙元在迅速减少,每一次催发仙道杀招,也不得不开始精打细算,小心翼翼。

\end{this_body}

