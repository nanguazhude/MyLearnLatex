\newsection{方源啊你死定了}    %第八百四十九节:方源啊你死定了

\begin{this_body}

疯魔窟第七层颇为宽广,在不是仙、秘谋人的感觉中,更是浩大。

皆因他们每一步行进,都要深思熟虑,都要时刻承受着海量道痕互斥的考验。

越是行走越是艰难,这种艰难更会在人的心中,拉大路程的遥远感。

方源按照正常蛊仙的走法坚持前行了一段时间,也对不是仙、秘谋人的处境有一定程度上的感同身受。

“没有至尊仙体,我也要挑选道路。但这道路越走越狭窄,刚开始时,会有不少的选择,但是越是深入,能供我选择的就越是稀少了。”

方源走了一段,停下了脚步。

他的路断了。

前方的路,充斥着其他流派的道痕。

不过好在挨过前面这一段,之后的路程中还有着方源挑选出来的路。

方源不着痕迹地望了望不是仙、秘谋人。

这两人已经被他抛下,留在后面,走的很艰难。

同样的,他们俩也在时刻关注着方源。

方源咬了咬牙,迈开步伐,继续前行。

他装作艰难的样子,一步一顿地往前走。片刻后,终于克服困难,又走到适合自己的道路上来,脸上又露出了一抹轻松之色。

“这道痕互斥对我毫无意义,反倒是伪装表演,耗费了我更多心力。”方源心中苦笑。

不是仙、秘谋人又远远的对视一眼。

方源的实力还在他们估量之上!

刚刚那一小截路程,方源走的虽然是艰难,但速度仍旧很快,换做不是仙处于相同的处境,恐怕就要吐血了!

接下来,这两人越走越慢,心中也越发沉重。

周围的道痕时刻压迫着他们,让他们举步维艰。不是仙已经开始吐血,秘谋人则面泛金紫之色。

不久之后,两人不得不停下脚步。

他们到达了极限。

方源仍旧在前面行走,看他的神态,比之前要疲累不少。但显然余力还是很足的。

“这个方源真是个怪物!”

“唉,对方可是继承了尊者传承的人,能够屡屡战胜天庭,有这样的造诣,也不足为奇。”

不是仙、秘谋人暗中交流,叹息感慨,充满了一种无奈。

“那我们就在这里看着他罢。我倒要看看,他究竟能走多远!”秘谋人冷哼一声。

不是仙犹豫了一下,问道:“你说,他会不会走出第七层,进入第八层?”

“这怎么可能?”秘谋人第一时间就否定,想都没有想。不过随后,他沉吟道:“这种可能性并不高,要走出第七层,至少是数十万步。我们现在走出来的路程,还不足一成!方源的确是实力强大,乃是百年甚至千年难得一出的七转妖孽。但是他毕竟不是八转,我甚至怀疑只有尊者才能走出第七层,进入第八层。”

“尊者……”不是仙想了想,“我们是不是该将那几位尊者来过这里的线索,告诉方源?”

“嗯。”秘谋人点点头,“等到方源达到极限,我们回转后,就将这些情报告诉他。”

秘谋人见识到了方源的实力,承认方源比他要强。告诉方源这些情报,或许有可能将方源的斗志和兴趣都激发起来,让他留在疯魔窟中,陪伴疯魔三怪一起攻略这里。

方源又走了一段路,微微停下脚步。

“这就是我上一次见到秘谋人的地方。”方源打量周围,却不见痕迹。

当初这里的道痕,方源记得是金道、土道为主,如今却已经完全改变。

方源毫无意外。

皆因第七层这里的道痕,也会变化。

每当魔音从第九层开始响起,就会传递开去,从第八层、第七层一直蔓延,直至第一层。

到了第一层后,魔音被一种无形的力量封锁起来,并不扩散到外界去祸害无辜。

方源继续行走。

秘谋人、不是仙停留在原地,望着他的背影,遥遥注视。

当方源不知不觉间,超过了秘谋人的最远记录后,这位智道蛊仙也不由地眼角一抖。

尽管这种情况他已经提前预见到了,但当他真正见到这一幕发生的时候,他的心中还是不由地升腾起巨大的失落感。

“我被超越了。”秘谋人心中深深叹息,“我乃疯魔三怪之首,一直是探索这里最远的人。在这里隐居数百年,一直潜心钻研,没想到今日被方源轻松超越。”

一种嫉妒甚至仇恨,还有一种对自己的愤怒、辛酸、悲凉等等感觉,一起涌入他的心头,让他五味陈杂。

不是仙的感觉,要比秘谋人好得多,毕竟他是疯魔三怪中成绩最差的那个。

见到秘谋人的神态,不是仙心中叹息一声,正想要安慰他,但下一刻面色骤变。

魔音忽起!

“怎么回事?”不是仙大惊失色。

秘谋人也是惊悚:“这个时刻,怎么会有魔音出现?”

不是仙叫道:“你之前的推算是正确的,魔音的规律已经彻底混乱了,我们再也不能拿捏准确。快,我们快撤!这里太危险!!”

秘谋人瞬间已经是满头冷汗,他高声对前方的方源大吼,提醒道:“魔音来了,快退!退出第七层才能安全。”

方源回头看着他们俩个。

这两人这一刻再无疯魔三怪的隐修风范,惶急如热锅上的蚂蚁,正在迅速往外退。

“看这样子,不像是故意来陷害我的。”方源眉头轻皱。

不是仙、秘谋人为了争取时间,根本来不及挑选路径,只能勉强看准方位,往外爆退。

魔音刚起,这里的道痕已经蠢蠢欲动。更可怕的是,当魔音真正轰鸣起来,第七层将掀起道痕的狂澜海啸!仙道杀招催动出来,立即混乱崩溃。届时,哪怕是龙公也危如累卵,弱小如同婴孩,身死道消的结局无可更改。

第七层已经成了吞噬一切的凶险之地,恐怕只有尊者才能抗衡。

疯魔二怪动用全部手段,拼尽全力,一边撤退,一边吐血。

“快,我们就要走出来了!”秘谋人看着前方,眼中透射出狂喜之色。

“我不行了!”不是仙却是一头栽倒在地上。

秘谋人咬牙:“坚持住!”

两人靠得很近,秘谋人不顾危险,将昏死过去的不是仙一把拽起。

危难时刻,显露出了他的真情实意。

最后,秘谋人连连迈步,一举冲出。

他七窍流血,面皮发青发白,双眼一阵阵发黑,直感觉全身如灌了铅沉重无比。

他却不敢就此停歇,魔音一起,第七层毫无安全之地,只有退回到第六层,才有生还的希望!

魔音开始浩荡,秘谋人心中的警钟也越来越响,他咬牙切齿,拼尽最后一丝力量,将自己和昏死的不是仙送上第六层。

“等等,方源呢?”临走之前,他冒险回头一看。

“嗯?!”一瞬间,他的双眼鼓瞪起来,好像是有人在他眼后打拳击。

秘谋人惊呆了!

方源不仅没有后退,反而比之前还要更加深入了一些。

“你、你、你!”秘谋人激动得说不出话来。

方源听到声音,回首看了看他,淡淡微笑:“我这个距离要离开,已经晚了。不过我有保命的秘法,还有一丝生还的可能。探索这里,自然会有性命之危。我早有了心理准备,不妨事。秘谋人仙友,希望我们有机会再见!”

秘谋人哑然,想要说方源啊你死定了,但终究没有说得出口。

最终,他留下一句话:“方源仙友,我们必会再见!”

这句话说得挺有深意。

一方面是鼓励方源,希望他坚持不懈,不要放弃努力,或许有生还的可能。另一方面,他也是在预示自己,像他这样探索下去,早晚有一天他也会陷入到方源这样的困境中,死后再见。

方源琢磨了一下,笑了笑:“这个秘谋人的确是有情有义。”

五百年前世,秘谋人就是这样,他本不想外出疯魔窟,但不是仙、胖山被牵扯到五域乱战中去。

这使得秘谋人为了维护同伴,走出了疯魔窟。

虽然他只有七转修为,但智道境界高深,给中洲和天庭造成了很大的麻烦和损失。

疯魔三怪虽为隐修,本身没有什么血脉联系,但交情却是非常的深厚,甚至超过了许多正道蛊仙。

人非草木孰能无情,这三人隐居在这里数百年,又拥有共同的目标,还相互扶助,早已经成为彼此间的生命羁绊。

方源看着眼前的道痕晃动,神态轻松至极,可谓云淡风轻!

就算这道痕再如何汹涌,掀动斑斓的海啸,对于他而言,也是毫无作用可言。

至尊仙体,道痕不斥。

方源反而比之前更轻松一些,因为不是仙、秘谋人已经走了。

他淡淡而笑,从容迈步。

看着眼前道痕的波澜,壮阔璀璨,他啧啧称赞,真是世间第一奇景!

五域两天中,多有奇景,有的壮美,有的灵秀,有的浩荡,有的精美。前世今生,方源几乎走遍天下,眼界开阔,经历丰富,见识过许许多多的奇景。

但他现在看来,记忆中的那些奇景统统不如眼前的道痕狂澜。

“若是将来,我或许留下传承,必定要将此地此景记载下来,流传后世。”方源心道。

------------

\end{this_body}

