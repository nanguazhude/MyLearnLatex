\newsection{柴家有人烟}    %第三百二十七节:柴家有人烟

\begin{this_body}

%1
开始了。

%2
第一步,方源化作卜卦龟。

%3
第二步,他开始催动大量凡蛊,随后是金刚念仙蛊,然后继续添加凡蛊。

%4
第三步,也是最关键的一步,催动防备仙蛊。

%5
噗!

%6
方源变化的卜卦龟,陡然喷吐出大量的鲜血。仙窍中,大量的凡蛊毁于一旦,好在防备仙蛊结实,没有丝毫受损。

%7
方源头昏眼花,仙道杀招尝试失败,让他内脏受创,内伤相当严重。

%8
幸好他变化成了卜卦龟,上古荒兽,皮糙肉厚,恢复力惊人至极。若是以人躯尝试,受到这样的伤害,五脏六腑说不得就在一瞬间彻底崩为血肉粉末。

%9
“怎么会这样?”方源有些懵。

%10
这和他设想的结果,完全是两种结果。

%11
事实上,防备仙蛊还未真正催发出威能,整个仙道杀招就崩溃了。

%12
有点让方源始料未及。

%13
这个改良出来的仙道杀招,方源利用爱意仙蛊等智道仙蛊,推算了许多遍,应该不存在这样严重的差错才是。

%14
这简直是一记迎头重击。

%15
方源疗伤完毕,开始静下心来,省视杀招。

%16
连续几次推演,方源越加纳闷:“没有差错啊?”

%17
忽然间,他脑海中灵光一现:“我明白了!”

%18
方源瞬间明白自己错在哪里。

%19
他错就错在防备仙蛊身上。

%20
防备仙蛊催动起来,不断消耗红枣仙元,但是等到它迸发威能,必须是十息之后。并非是第十息,而是第十一息才开始有效。正是这个微小时间的差错,让方源的整个仙道杀招,发生了崩溃。

%21
“居然会在这里出现差错,真是灯下黑!”

%22
“看来,需要添加一些宙道蛊虫,来弥补时间上的参差了。”方源的宙道境界很是普通,一直都没有多大提升。因为他从未遇到过有关宙道的梦境。

%23
不过,他身怀黑凡真传,又有智道推演手段,多少能弥补一些这方面的短板。

%24
只是这样一来,就不是短短片刻功夫,或者一盏茶、一炷香能够完成的推算。

%25
一天过去,方源不眠不休。

%26
第二天过去,方源推演不辍。

%27
第三天又过去,武家前来接应的仙蛊屋,已经行了一大半的路程了。

%28
方源的推算,终于首次有了一些眉目。

%29
发生了质变之后,接下来的推演,就开始顺风顺水了。

%30
南疆,人烟山脉。

%31
这片地方,生活着南疆领域中最多的凡人。一座一座的山寨,每座山峦上都存在着。有的山峰上,甚至会有好几个山寨,比邻而居。

%32
这里堪称盛世景象。

%33
人口繁衍到了极致,拥有海量的凡人。当然,论战斗力的话,一位蛊仙就能彻底摆平这里,并且强势到让所有凡人都毫无反抗的余地。

%34
而在这个山脉的最中心,也是是所有山寨的政治中心人烟山。

%35
超级家族之一,柴家的大本营!

%36
创建柴家的先祖,姓柴名夫,在人族历史中都留下过一笔记录。

%37
在他的那个时代,他是南疆蛊仙界中的第一人,专修炎道,兼修土道,因为阅读人祖传收获颇多,又拥有人道杀招。在晚年的时候,他选择在这里,创建了柴家,改造山脉后,这才有了人烟山。

%38
事实上,蛊仙中的强者、大能,都会去阅读人祖传。

%39
人祖传中,潜藏着人族的人道真传,这部真传堪称是蛊师世界的第一真传。

%40
它的传承方式,也非常特别。

%41
通常意义上的传承,是黑凡真传、白相真传这种。

%42
但人祖的人道真传,只是人祖传这本书,一份信息。人道的奥义,因此在这些信息内里,那位蛊仙能够获得,就得看各自的缘分和天赋才情。

%43
每个人的收获都是不一样的,仁者见仁智者见智。但每个仙尊魔尊都有收获,并且收获都不小。比如巨阳仙尊,正是因为阅读了人祖传后,他才能成功开创众生运、天地运。

%44
柴夫阅读人祖传有成,他死后,却是留下了他领悟出来的人道手段,遗留给了柴家血脉,他的子孙后代。

%45
似乎正因为如此,柴家的经营策略,才和其他超级家族不同。柴家大力发展人口,增长凡人基数。

%46
而其他超级势力,基本上只重点经营一部分族人,将他们安放在五域中,任其发展,多加栽培。这部分族人的血脉源头,乃是超级势力中最当权的蛊仙,也就是主家血脉。

%47
还有一部分人口,则隐居在洞天福地当中,并不出世。这些人,往往是分支血脉。只有当意外发生,主家血脉被全数灭绝,或者当权者死亡,并无后继之人,分支血脉才会顶上去。

%48
柴家大力发展的人口,并非是柴家血脉。

%49
若换做其他家族来做此事,就是自找苦吃。花费代价,培养其他姓氏的凡人,人口基数一大,就有越来越多的蛊师涌现。这些蛊师没有本家血脉,忠诚度大打折扣,也就没有培养成蛊仙的价值。若是自我修行成了蛊仙,更会对本家形成冲击,造成矛盾。

%50
但柴家的情况不同。

%51
他们培养越来越多的凡人,但这些凡人中,竟然资质都不高,涌现出的蛊师非常稀少,更别谈什么蛊仙了。

%52
反而柴家本身的族人后辈中,却连续不断地涌现出天资超绝的人才,能够栽培成蛊仙的种子屡见不鲜。

%53
这点和其他超级势力,形成了鲜明的对比。

%54
按照其他超级势力的推测,柴家这种不正常的情况,应该就是柴夫当年遗留下来的人道手段了。

%55
然而,尽管柴家不管哪一代,都能涌现出大量的天才后辈,但是因为资源不足,这些蛊仙种子中大多数,最终只能停留在五转、四转的层次。

%56
在这里,就要说一下南疆的地理格局了。

%57
南疆最为鲜明的,除了群山之外,就是三条大江。

%58
碧龙江、赤龙江、黄龙江,每一条都几乎贯穿南疆,将整个南疆分隔成几个部分。

%59
赤龙江,从最西北的位置发端,一路向右下倾斜,直至南疆的东南端。

%60
碧龙江在上,从整个南疆地图俯瞰,整条江河宛若一道缓坡。

%61
黄龙江则呈现几字形状,在中段忽然凸起,远比碧龙江陡峭。

%62
碧龙、黄龙两条江河并不交汇,但赤龙江,分别和碧龙江、黄龙江交汇。

%63
赤龙江和黄龙江的交汇处,便在武家领地中,形成江心大漩涡,在其东北岸边就是尸皇芋顶天。

%64
而赤龙江和碧龙江的交汇处,江心大漩涡的规模,是赤黄江心漩涡的数十倍。灵气磅礴浩瀚,江心漩涡中心,生有大量的元泉,数目成千上万。这些元泉并不稳定,每一息,就有十几个元泉生灭。

%65
因此造成磅礴的灵气,灌输在江畔。

%66
乐土仙尊游历到这里的时候,曾经评价,说纵观南疆,此处钟灵毓秀,当是第一。

%67
在这里,空气中充斥元气,大量的资源养生出来,媲美优秀的洞天福地。

%68
也就是在这里,聚集着三个超级势力的大本营!

%69
打个比方,赤龙江、碧龙江两江交汇,形成仿佛的形态,交汇处是江心漩涡。而柴家的人烟山,就位于上端,巴家的大宅山位于下方,夏家的延寿山位于右方。

%70
在过去的三天里,柴家的议事大厅中一直充斥着争吵。

%71
此时此刻,仍旧不例外。

%72
柴家太上大长老端坐在诸位上,目光闪烁,面色犹豫。

%73
在不久之前,他接到了方源的一封来信。在信中,方源以武遗海的身份,代表武家,表示愿意和柴家联合,攻击夏家、巴家。

%74
此事重大至极,柴家太上大长老不得不召集家族蛊仙,共同商议。

%75
议事大厅中的争吵,就是围绕此事。

%76
“进攻巴家、夏家,我以为万万不可!武家用意,昭然若揭,无非是让我们出手,替他们缓解外在压力罢了。一旦出手,我柴家就要同时面对两大超级势力。若这两家放过武家,一齐掉过头来,联手攻我,我家必定处于下风,无非翻身还手。”一位柴家蛊仙扯着大嗓门,站在议事大堂的中央,喊着。

%77
他刚刚喊完,从角落里站起一位蛊仙,提议道:“我们完全可选择其中的一家开战。武家虽然提出同时开战两家的要求,但这些条件都是谈判出来的。这并非是我们行动的准则。”

%78
“唉兹事体大。我们柴家繁衍生息这么多,豢养出了海量凡人。一旦开战,对方只需要一位蛊仙,就可生灵涂地,将这些凡人斩杀,让我族损失惨重。”一位年老的柴家蛊仙摇头叹息,他很是投鼠忌器。

%79
因为人道手段的局限,这些凡人根本不能挪出人烟山脉。如此一来,就需要仙级蛊阵或者仙蛊屋、蛊仙进行防御。

%80
“正是因为这样,我们才要开战,争夺更多的资源。”

%81
“没有错!我们柴家的优势就在于,我们拥有人道手段,家族后代都能涌现天才人物。只是一直苦于没有足够的资源,进行培养。一旦我们驱逐了其中一家,甚至两家,我们就能拥有大量资源,成为南疆第一势力也是指日可待。”

%82
这是家族主战派的言论。

%83
柴家太上大长老始终沉吟不语,听了一会儿之后,他看向大厅中的某个蛊仙,道:“有言啊,你说说你的看法。”

%84
ps要问为什么这么早更?因为今天要爆更!哈哈哈。未完待续。

\end{this_body}

