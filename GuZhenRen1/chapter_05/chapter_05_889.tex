\newsection{黑火}    %第八百九十三节:黑火

\begin{this_body}



%1
道痕不互斥!

%2
至尊仙体的优点也是最大的弊端,在这一刻显露无疑。

%3
一直以来,方源都刻意避免这个缺陷,但此刻沈伤的杀招着实诡异,实在叫人防不胜防。

%4
谁能想到夜歌杀招只是一个陷阱?

%5
沈伤被万年斗飞车追杀,催出这样的杀招来隐藏自己,是非常正常且自然的事情。

%6
若方源不动用侦查手段,就只能任凭沈伤躲藏在黑暗中,攻势陷入僵滞。

%7
方源伤口上的人道道痕在迅速增长!

%8
落在沈伤身上的轻伤,因为两败俱伤杀招的缘故,同样影响到了方源。

%9
而在方源身上,因为道痕不互斥的缘故,轻伤就变成了重伤。

%10
不过好在世间的种种治疗杀招,放到方源身上,效果也能完全发挥出来。

%11
当即,一记强大的治疗手段用在自己的身上,方源转瞬恢复。

%12
仙道杀招——重伤!

%13
沈伤狞笑,竟又开始自残,并且力度比之前还要大得多。

%14
方源身躯狠狠一震,眼前发黑,七窍都开始流血。

%15
“不行!要论治疗手段,我还是比不上沈伤,须得另辟蹊径。”方源明悟过来,脸上涌出一抹坚毅之色。

%16
仙道杀招——重伤。

%17
仙道杀招——重伤。

%18
仙道杀招——重伤。

%19
沈伤不断自残,逐渐疯狂,他哈哈大笑,张开的大口中血液飞溅。

%20
他重伤自己,又迅速恢复,然后又继续。

%21
这种罕见的战斗方式,让观战的蛊仙们都目瞪口呆。

%22
庙明神等人脸色难看,十分担忧。

%23
沈家蛊仙们则振奋不已。

%24
虽然他们没有看到方源的状况,但是飞剑群自行崩解,万年斗飞车不断后退是眼前的事实。

%25
“沈伤先祖竟能逼退方源!了不起。”沈从声兴奋不已,他仿佛已经看到沈家崛起的一幕。

%26
庙明神双眼精芒四射,暗道:“不妙。方源虽有万年斗飞车,但沈伤的攻击手段似乎能够直接绕过这座仙蛊屋,正好克制了方源。方源到底只是七转,而沈伤却是货真价实的八转啊。”

%27
鬼七爷、土头驮等人非常疑惑:“这个沈伤当年被乐土仙尊镇压,一直活到现在,这本来就极不寻常。”

%28
“更叫我疑惑的是,他似乎不缺仙蛊!当年乐土仙尊镇压他时,没有把他的蛊虫收走或者摧毁吗?”

%29
“呃!”就在这时,沈伤的狂笑忽然顿住。

%30
他的脸上出现了一抹震惊和疑惑。

%31
他对方源的感应正在迅速减弱。

%32
方源究竟是怎么做到的?

%33
怎么会这么快,他就能破解了自己的手段?!

%34
仙道杀招——洁身自好。

%35
洁身自好杀招能够将身上的不利道痕逐渐消除,方源凭借这个仙蛊和杀招,正在迅速消除身上的人道道痕。

%36
沈伤眼中血光迅速闪烁,他更加疯狂地自残,好使得方源身上的人道道痕迅速增加。

%37
只要有一丝人道道痕在方源的身上,他就能感应到方源,绕过万年斗飞车,继续攻击方源。

%38
但这时方源已经稳定了局面。

%39
万年斗飞车缓缓飞近,从中传出方源的声音:“该轮到我了,沈伤。”

%40
十二生肖上古战阵!

%41
轰。

%42
一声巨响,体型庞大的灰石巨人再次登场。

%43
巨人速度极快,两手张开,向沈伤抓去。

%44
沈伤眯起双眼,冷哼一声,飞速后退,企图和巨人拉开距离。

%45
灰石巨人在方源的指令下,只是捕捉沈伤,并不猛打硬功。

%46
灰石巨人的攻伐威能极其厉害,虽然在这里不如光阴长河,但也是八转之巅。

%47
关键是他的这个打法,以擒拿为主,并不是要杀伤沈伤,立即让沈伤退避三舍。

%48
战局开始僵持。

%49
沈伤一边躲避十二生肖战阵,一边催动杀招自残。

%50
而方源一边操纵万年斗飞车,一边治疗自己,一边用洁身自好消除身上的人道道痕。

%51
万年斗飞车和灰石巨人不断配合,对沈伤围追堵截。

%52
沈伤原本躲避万年斗飞车,就吃力勉强,此刻多了一个灰石巨人,立即落入下风。

%53
每当要被捉拿的时候,沈伤就发出一声声的咆哮,音波爆炸开来,为他炸开一条通路。

%54
双方你来我往,在战场纵横,余波殃及庙明神人等,还有沈家群仙。

%55
花蝶女仙和蜂将率先支撑不住,传送回去,脱离了战场。

%56
随后是童画、土头驮、鬼七爷等人。

%57
再之后,便是沈家群仙和庙明神、任修平。

%58
最终,战场上只剩下沈从声还留在这里,八转之争,不是谁都有资格观战的。

%59
沈从声脸色越来越难看。

%60
战局在向方源迅速倾斜,沈伤的处境越来越不妙。反观方源却是越加从容。

%61
“此子有雄厚的智道造诣,脑海中念头此起彼此,层出不穷,一心多用。这方面,我落后太多了。”沈伤死死咬牙,战至此刻,他感到自己就像是落入了沼泽里,全力挣扎仍旧越陷越深。

%62
战局已经脱离了他的掌控!

%63
忽然,方源走出船舱,出现在船首。

%64
他对着沈伤一指,一记宙道杀招发出。

%65
沈伤猝不及防,难以躲避,中了这招后,立即动作骤缓,自身时间流速被放缓了一倍。

%66
灰石巨人趁机一把抓住沈伤,手掌缓缓收缩。

%67
沈伤哈哈大笑,乱发飞舞,双目通红,毫无畏惧:“尽管捏死我吧!”

%68
方源冷哼一声,命令灰石巨人,后者双手并拢,形成一个十指囚笼,将沈伤牢牢束缚,令后者暂时动弹不得。

%69
明白了沈伤的攻伐手段,方源就能对症下药。只是拘束并不杀伤,那对他自己就是无害的。

%70
沈伤不断挣扎,疯狂嘶吼,却是挣脱不得。

%71
沈从声见势不妙,想要出手施救,结果再一次遭受反噬,他始终受到乐土仙尊的手段约束。

%72
“终于得手了。”方源吐出一口浊气。

%73
沈伤无愧于他的身份,能够引出乐土仙尊亲自出手镇压,的确是有几把刷子的。

%74
他的手段奇怪诡异,令人很难防范,方源也一度落入下风。

%75
当然,方源掌控着一部分战局,他的种种底牌仍旧扣在他的手中。

%76
就算再不济,方源还能传送回接引岛。有这一点保障,方源便始终处于不败之地。

%77
“啊啊啊……”沈伤嚎叫,面色扭曲。

%78
方源眉头开始皱起。

%79
沈伤此时的状态变得有点奇怪,他大笑不止,疯狂地挣扎,却不动用任何的仙道杀招。

%80
下一刻,他通红的血眸尽数转成一片漆黑,黑色迅速扩散,直至充斥他整个眼眶。

%81
蓬的一声轻响,一团黑火从他身上忽然燃烧起来。

%82
火焰中,沈伤身躯不断猛颤,仿佛癫痫一般,口水四流,张牙舞爪,口中发出野兽般的嚎叫尖啸。

%83
沈伤身上的黑炎并不起眼,但却蔓延到灰石巨人的双手时,灰石巨人遭受火焰灼烧,立即发出痛苦的咆哮。

%84
他的双手沾染了黑火后,开始迅速融化,竟是对这个古怪的黑火毫无抵御之力。

%85
方源目光越加冰冷,黑火强大诡异,远超他的想象。

%86
见机不妙,方源立即撤销了上古战阵。

%87
灰石巨人消失了,十二个太古年兽出现在半空中。

%88
其中两头太古年兽,却是沾染了黑火,它们失去了理智,在半空中打滚,发出惨痛的哀嚎,拼命挣扎。

%89
幸存的十头太古年兽,仿佛见到了天敌,甫一得到了自由,便惊惧不已地四散而逃,离黑火越远越好。

%90
方源、沈从声眉头紧锁,神情凝重。

%91
这团黑火太过古怪,他们两人竟都无法侦查出这黑火是何流派!

%92
与此同时,两人又同时感到一股强烈的威胁感,这种威胁感甚至令他们头皮都微微发麻。

%93
黑色的火焰仿佛是一切生命的天敌!

%94
两头太古年兽已经被火焰覆盖,它们一边嘶叫,一边从空中坠落下去。

%95
方源有一种强烈的预感,一旦这两头太古年兽落入海水当中,黑火必定会酿成更大的惨剧。

%96
但他没有出手,而是驾驭万年斗飞车不断后撤。

%97
没有了灰石巨人束缚的沈伤,悬浮在空中,仍旧狂笑,疯子一般。

%98
沈从声呼唤了几声老祖,但沈伤充耳不闻。见到这般诡异的情形,沈从声也开始明智地后退。

%99
就在那两头太古年兽要接触到海面的时候,忽然海底大阵喷涌出一道白金光柱,将这两头太古年兽托住,不断向上升起。

%100
仅剩四成的海底大阵,发出史无前例的轰鸣声,正全力运转!

%101
白金光柱猛烈膨胀,又将沈伤囊括进去。

%102
黑色的火焰在沈伤、两头太古年兽的身上,不断燃烧,但白金光柱将它们严密地禁锢住。

%103
如此异变,让沈从声愕然。

%104
方源忽然出手,破晓剑群再度飞射,纷纷穿透白金光柱,刺中沈伤。

%105
“住手!”眼睁睁看着自家先祖被方源怒射,沈从声急躁大吼。

%106
方源冷笑一声,飞剑催发得越加凶狠。

%107
这古怪的黑火明显危害极大,这点从海底大阵主动出击,似乎生怕海水碰上黑火,就可看出来。

%108
沈从声只能干瞪眼。

%109
方源心头一动,飞剑群飞出两股,又对自家的太古年兽下手。

%110
黑火在破晓飞剑的攻击下,仍旧燃烧,势头衰减的幅度很是微小。

%111
破晓剑可是八转杀招,居然近乎无效!

%112
方源坚持喷射,黑火陡然发生变化,化为一团团的银白水气。

\end{this_body}

