\newsection{男人一定要帅!}    %第九百四十一节:男人一定要帅!

\begin{this_body}

方源一直作战在最前线。

一方面他不断出手,和龙公等人激烈对抗,另一方面他始终保持着冰雪般的冷静,筹谋着全局。

但自从狂蛮魔尊的手段发动之后,他的耳畔就多了一道声音。

“看好了,后辈!你眼前的这三头巨怪,乃是我三记仙招所化。这三记仙招是我阅读《人祖传》中‘追寻自由’的篇章有所心得和灵感,皆为追寻自由而开创。”

“第一招——奔雷黄鸟变,以态度蛊为核心。你有没有羡慕鸟儿,想在天空自由飞翔?然而我们人族生来就受限制,根本就没有双翼,谈何飞翔?更残酷的情况往往如同无翼鸟,它们已经体会过飞翔的美妙滋味,却失去了双翼。不管你是谁,在哪里,你都会受到大大小小的限制。假使我们无法改变,无法挣脱这一切,就像是无翼的鸟,想要追求自由,却追求不到,怎么办?”

“这就是此招的核心思想,记住,后辈。即便我们受到最严苛的限制,无法追寻到自由。但至少我们还有一项自由。那就是我们可以自由地选择用什么样的态度来面对残酷的现实。记住,你的态度很关键!”

“第二招——囫囵蓝豹变,以变异蛊为核心。自由并不是一件坏东西,但同时它也不是一件好东西。它可能让你变得更好,也可能让你变得更糟,如同变异蛊的效用一样。猛兽缺少了牙齿,不能活下去吗?如果没有撕咬的自由,囫囵吞食又有何妨?说不定还能锻炼你的消化能力呢。”

“第三招——内息绿鱼变,以变通蛊为核心。世界和其相比显得多么丑陋与污浊,我们会被残酷的事实打击,会被痛苦和无望折磨。我们走在各自的人生路上,有时候遍体鳞伤,甚至无所遁形,无法逃避。人是在变化的,世界也是在变化万千的,跟不上时代的脚步,就要被整个世界淘汰。所以,变通吧,没有鱼鳃,那就用嘴呼吸,如果连嘴都不行,那就自我循环内呼吸。就算走得再慢,甚至变得面目前非。”

声音厚重有力至极,详细解释了三位巨怪的底细和由来。这是狂蛮魔尊留下的三记变化道杀招所化。

自从方源触发了狂蛮魔尊的手段之后,这道声音就一直奇妙地伴随着方源,其他人根本毫无所觉。

“难道这便是狂蛮魔尊的声音?”方源正猜测着,忽然声音一变。

从原本稳重厚实的声音,变得活泼轻佻起来。

“喂,你说了这么多,该我了,该我了!”

“哈哈,小子,不管你是谁,你一定是天外之魔。因为只有天外之魔,才能有希望打坏宿命蛊。所以当初我留下这份真传,只有天外之魔才能继承。”

“但除此之外,你还得至少具备三只蛊虫中的一种。态度蛊、变异蛊、变通蛊,这三只蛊虫分别对应三记杀招,能分别引动我的杀招。当你来到绣楼上空,就能勾动其中的杀招变化成巨怪,成为你的强大臂助。”

“随便你怎么用,杀烧抢掠也好,造福人间也罢。当然,最好能搞掉天庭的宿命蛊,因为我看它真的很不爽啊。哈哈哈!”

“当然,如果你能将三招全部引动,那么这三招还能合并为一招!这可是我精心构思出来的超超超级大杀招啊!”

“一旦你掌握了此招,你就能以人族形态,同时拥有三招的全部优势,并且没有任何其他方面的短板。也就是说,你将如同无翼鸟,动若闪电,喷吐雷霆。也会像缺腮鱼一样,拥有极其强大的恢复自愈的能力。更能和无牙兽一般,吞食一切,储藏肚腹之中。”

“是不是惊呆了?是不是超级厉害?哈哈哈,膜拜我吧!”

轻佻的男声说到这里,似乎又被赶下去,换成之前的沉重男声。

“最后的并招名为自由残缺变,它是以态度、变异、变通为三大核心,非常的繁琐和复杂。虽然你可以利用我留下的三怪,将它们合而为一,这比你独立催动要容易很多,但还请你不要轻易尝试。因为此招是增添作用在你的肉身之上,一旦失败,轻者重伤,重则死亡。切记,切记!”

沉重男声语重心长,但很快,之前的轻佻男声又冒了出来。

“拜托,你说的都不是重点啦!小子,让我来告诉你什么是重点!一旦你成功催动了此招,你就能力压一切太古荒植、太古荒兽,将三怪手段化为自己的天赋本能。你将成为人形暴龙,人形暴龙,战天斗地,纵横捭阖!你将立即拥有亚仙尊的战力!”

“当然,遇到像我老人家这样的尊者,你还得跪。”

“不过这些都不是最关键的重点。最关键的一点,你要记住。”

方源听到这里,不由地心神一凝。

只听那轻佻活泼的男音继续道:“最关键最重要的是,此招一旦催动,三怪合一就能化为一道血战披风,披在你的身上。这面披风真的是……超级帅啊!后辈小子,你一定得尝试一下,真的是帅呆了,你绝对绝对不会后悔的!”

方源无语。

伴随着声音,种种讯息流转在他的心头,使得他对三道杀招的种种内容了然于胸。

和男声所描述的不同,这三道杀招还有最后的并招,其实都不复杂,步骤方面也并不繁琐。

狂蛮魔尊的时代,毕竟是上古时代,蛊修风格不是大气磅礴,就是粗糙狂野,远不如现在的精致繁杂。

掌握这些杀招并不困难,尤其是对于拥有深厚智道造诣的方源而言。

“那么,是这样子组并的么……”方源酝酿成熟,蓦地抽身后退,催动杀招。

一瞬间,在最前方厮杀的三头巨怪忽然化作三股光辉,一黄一篮一绿,投入到方源的身上。

如此惊变,让交战的双方都猝不及防。

三色光辉笼罩方源,散发出磅礴气势,不断向四周喷涌,排斥一切人和物。

三光迅速融合,转为血红之色。

血光陡然凝聚,彻底凝实一体,覆盖到方源的后背上,化为一道鲜红的巨大披风。

披风随风飘舞,宛若战旗,猎猎作响的声音响彻在每个人的耳畔。

“这又是什么变化?”

“好像是方源继承了狂蛮真传?!”

敌我双方一惊再惊。

看到这熟悉的披风,一缺抱憾亭中,星宿虚影不禁回忆浮现。

一百多万年前,一个男人闯入天庭。

他一步步走来,平平凡凡的每一步在他行走,却是震荡天庭,惊天动地。

他的身躯极其雄壮魁梧,身着兽皮,浑身肌肉贲发。一片片的纹身围绕在他全身的皮肉上,描绘出无数奇形怪状的野兽植株,有的凶残野蛮,有的灵慧精秀。

他散发出来的磅礴气息,如巍峨群山,如浩瀚海洋,天庭众多蛊仙只是看他一眼,就感到一种窒息的压力。

在他的背后披着一面披风。这面兽皮披风极其宽大,是用人皮缝制的。也就是狂蛮魔尊这样的高大雄健的身材,若换做寻常人物,披风便要拖在地上一大截。

每当狂魔魔尊屠杀了一位八转敌人,便将对方的人皮剥下来,缝制到兽皮披风上去。这本是兽人一族的风俗,但在狂蛮尚未成尊的年轻时代,就被他学了去,进行了效仿。

以牙还牙,以血还血!

侵入天庭的半途中,狂蛮魔尊顿足在绣楼跟前。

半空中,星宿意志显现而出:“狂蛮啊,你是打坏不了宿命蛊的。你的强大本就来自于你的宿命。就好像不凭借其他手段,你自己就不可能举起你自己来。”

“这点我当然知道。”狂蛮魔尊呵呵一笑。

“那你来做什么?”

“来干架的呀!”狂蛮魔尊昂首,用轻佻的语气理所当然地道。

星宿意志:“……”

狂蛮魔尊继续笑道:“当我成就尊者后,打遍天下无敌手,我就感到了一种强烈的寂寞!没人能够和我打的了,那些异人八转啊都被我打杀得差不多了,还活着的老弱病残现在都躲了起来。我也没兴趣找他们。看遍五域,只有天庭这里才有点意思。你们为了保存宿命蛊,一定留下了许多手段罢。当年无极就是被你们挡下来的。来吧,把所有手段都用出来,让咱们好好较量一下。”

星宿意志沉默半晌,这才沉入绣楼当中:“你眼前的这座仙蛊屋便是当年,我的本体遗留下来的手段之一。”

“吼吼!”狂蛮魔尊顿时双眼发亮,“那我就上了!”

战斗结束的很快。

狂蛮魔尊捏着双拳,舔了舔嘴唇,不满地嘟囔道:“什么嘛,才两三下,你的绣楼就不行了。刚开始倒是挺不错的。不过你放心好了,我不会拆了你这座楼的。咱们是自己人嘛。”

“另外,我的这三层血皮就留在这里好了。这可是我的传承!”

“不管是谁,哪怕你们天庭的人,只要能够继承我的传承的话,就留给你们好了。”

“就当是我来到天庭留下的纪念吧。可不要乱拆哦,嘿嘿嘿。”

“真是期待呢。不知道会是哪个后辈得到我的传承,会不会被我伟大的力量而震慑,被我奇妙的思想所折服呢?”

“哈哈,当然最好的就是,这个后辈能够摧毁宿命蛊。做不到这个事情,真的令我不爽快!”

星宿意志再度沉默良久,这才开口:“所以要摧毁宿命蛊,这才是你的图谋。”

狂蛮魔尊耸耸肩:“那是当然。血皮我就留在这里了,有本事你们就毁掉它!但我要好心的提醒你们。这三层血皮已经和天庭的道痕链接一体了,随意乱动,对于天庭可是最直接的伤害。除非是有后人拥有我这样的变化道造诣,才能安全解除吧。”

星宿意志又道:“你想要摧毁宿命蛊,其实还有一个更好的办法。那就是你担任仙尊,成为天庭之主,和宿命蛊朝夕相处。我想凭借你的智慧,定会想到摧毁宿命蛊的方法的。”

狂蛮魔尊想都没想,直接摇头:“我不要!”

星宿意志皱眉:“为何?”

“因为魔尊这个称号听起来,就比仙尊帅气啊!”

星宿意志:“……”

很诡异,这个理由被狂蛮魔尊说出来,怎么就有一种理所当然的感觉。

狂蛮魔尊转身,看看自己,挥挥手臂:“看看!”

“看看我的大腿,多么结实。我的手臂,多么粗壮。”

“这就是美,这就是帅啊!”

“还有这道披风,血糊糊的,又是人皮所制,多么威风霸气!”

星宿意志一阵凌乱,好半晌才开口:“所以,你当初制造这件披风,不是为了血债血偿,向异人们复仇?”

“当然有这一个次要的理由,但主要是帅!”

星宿意志:“……”

“哦哦,对了。我留下的这三层血皮还能组成一招,招式成形之后,就是一道血皮披风。和我身上这件十分相似啦。将来若有后辈穿上,一定十分非常超级的帅气啊!”

“男人!”说到最后,狂蛮魔尊高举手臂,弯曲手肘,亮出自己上臂高高鼓起的肌肉,真情地自赞道,“就是该这样的帅啊。”

------------

\end{this_body}

