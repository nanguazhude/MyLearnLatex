\newsection{关键时刻关键人物}    %第九百四十节:关键时刻关键人物

\begin{this_body}

天庭战场,不论敌我双方的目光,都集中在了一个地方!

方源环视左右,他被三头忽然出现的巨怪包围在中央。但三头巨怪并未对他怎样,反而是齐齐护卫着他,敌视周遭,态度之鲜明一望可知。

“狂蛮魔尊……”方源一脸震惊之色。

“就算有狂蛮魔尊的手段助你,你也必败无疑!!”龙公大吼,声音震荡百里。他从空中俯冲而下,宛若一颗赤色流星,即便是有狂蛮魔尊的手段出现,他的斗志也始终坚定不移要摧毁一切魔道,要镇压所有叛乱!

仙道杀招气呼山。

龙公双掌遥遥对准方源,狠狠一拍。

轰隆!

一声巨响,一座巨大的气流大山,将方源和地下的三怪都覆盖住,重重镇压过来。

龙公催动过三气归来杀招,身上有众多的气道道痕,此刻催动气呼山这些气道杀招,威力暴涨无数。

这一座气流大山不仅比之前更加庞大,而且形体还十分凝实。不管是质和量都有了竟然的突破!

看到这样一幕,不管是武庸还是冰塞川等人都流露出凝重之色。

单单这道气呼山就已是难以抵挡,最好的办法就是躲闪。然而气呼山后还有龙公这个强悍得惊天动地的超凡人物,必定会阻止一切的逃难!

方源咬牙,正要动手。

就在这时,血皮所化的三怪之一,那头蓝色的巨豹忽然张开大口,重重一吸。

咻。

下一刻,整个气呼山猛地缩小,竟然被一股无法形容的力量直接吸入蓝豹的肚中去!

群仙心头狂震,主修食道的赵山河更是瞳孔猛缩,他从蓝豹的身上看到了食道的迹象。

就连龙公都为之一愣。

这样强大的气呼山竟然被这样直接破解了。

蓝豹吞了气呼山,干瘪的肚腹略微鼓起了一成。它双眼熠熠生辉,盯着龙公,目光令人悚然,仿佛是盯着一份鲜美的大餐!

龙公有着大无畏的勇悍,三怪即便深不可测,但他仍旧冲向方源。甚至他速度不减反增,直接刺破大气,发出轰然的音爆声响!

方源严阵以待,三怪之一的绿鱼忽然动了。

它只是鱼尾微微一摆,就陡然出现在了龙公的面前。它的速度如此之快,仿佛瞬移闪现,但偏偏又给人毫无突兀,一切如行云流水般自然的感受!

龙公前一刻还是视野宽阔,下一刻整个眼前就充斥着鱼怪的如山身躯。

鱼怪猛地甩头,狠狠一顶。

砰!

鱼怪的头直接将龙公磕飞,龙公以比俯冲时还要更快的速度,被直接顶飞出去。

他身负变化道、气道道痕,强悍到史无前例的程度。再用了数次三气归来杀招后,他防备寻常八转攻势,甚至已经开始不需要九龙纹护身!但他撞在鱼怪的头上时,却感觉自己仿佛是用一个肉拳硬生生地砸礁石!

一瞬间,龙公被撞得全身骨骼粉碎,内脏都差点被巨力挤爆,内外都大量出血。

“三气归来!”龙公大喝一声,暴跌的气势顿时又狂飙上去。

龙公紧接着又催动气道治疗手段,几个呼吸的时间,身上的伤势立即恢复,痊愈的速度极其骇人。

但他因为选择治疗自己,任凭自己被一路撞飞,竟直接撞到了天庭的苍穹极限之处。

天庭乃是公共洞天,极限处便是窍壁。龙公撞到窍壁上,身上仍旧残留着巨大的力量,尽数倾泻在苍穹的窍壁上。

众仙于是又听见一声撞击的巨响,然后就看到龙公身后的天空,几道裂痕骤然出现,然后迅速向四周蔓延,宛若蛛网。

群仙不由暗吸冷气。

鱼怪这一记头顶的力道简直是难以想象!就连龙公似乎都难以抗争。

而鱼怪不过是狂蛮魔尊留下的手段,存在了一百多万年,仍旧有着这样的威能。

那施展这个手段的蛊尊狂蛮,又有着多么强大的力量呢?

难以想象!

众仙不由地都想到了关于狂蛮魔尊的评价,世人公认他是尊者中的力量第一。

今日一见,果然如此!

“这就是狂蛮尊者的手段!”

“这一仗我们还未输,还有得打。”

“龙公大人又回来了,他的伤势再度痊愈,同时气势更加强大!”

“怕什么?我们可是有着完整的宿命蛊啊,天命在我!”

三域蛊仙们对于意外的三大强援,又惊又喜,士气回升。而天庭众仙同样士气昂扬,他们对于宿命蛊,对于龙公都有充沛的自信。

龙公再度俯冲而下,和三怪大战。但这一次他学乖了,以气道遥攻为主,可以规避近身战。

这三头巨怪力量实在骇人,即便是龙公也不想再被狼狈拍飞。

他选择的战术相当明智,三头巨怪的力量一个个都惊世骇俗,但它们的手段却乏善可陈。

交手十几个回合下来,龙公就摸清楚了。

六足黄鸟速度极快,口中能喷吐黄色闪电,但无法飞行,鸟喙极其坚硬猛锐,即便是仙蛊屋都要被一次洞穿。

绿色怪鱼可以悬浮飘行,效果媲美宇道的闪现,偏偏又十分自然,令人防不胜防。任何的攻势打在这头鱼怪身上,鱼怪都几乎在瞬间恢复过来。它最强大的地方就在这里惊人的恢复力,比龙公催动气道治疗手段还要强大。但是它也有缺陷,它的攻势虽猛,但节奏太慢,频率很低下,出手一次后,就得沉寂片刻,酝酿许久,方才能再一次发起进攻。

至于无牙的蓝豹,它也能踏空飞行,最强的手段是张开大口,吞食一切攻势。不仅如此,它甚至能够连蛊仙身上的人中豪杰杀招,都能吞吸入腹。对此,赵山河看得目眩神迷。他的食道手段还远远做不到吞食增益状态,这当中蕴含着他所不具备的食道奥义,如果能参透得出,必定能令他百尺竿头更进一步!

同时,无牙蓝豹的最大弱点也是这个几乎吞食的天赋。它每吞食一次,肚腹就鼓胀起来。被龙公等人故意喂招之后,无牙蓝豹肚腹滚圆,行动越发不便。

然而,即便天庭一方很快摸清楚了三怪的底细,但仍旧难以抵挡,逐渐落入下风。

三怪虽然有种种缺陷,强弱之处对比非常明显,但它们横冲直撞,根本无人敢挡在它们的前方。

再加上方源、武庸、冰塞川、沈从声、宋启元等人从旁策应,克制、化解天庭种种手段,令天庭一方分外被动。

三怪都将矛头对准监天塔,它们对于宿命蛊似乎有着天然的巨大仇恨。

监天塔不敢对抗三怪的力量,只能不断后撤。为了保护监天塔,龙公顶住巨大压力,一次次身负重伤,一次次转身再战。他紫发狂舞,双眼血红,浑身浴血,染遍龙鳞!他的战意一直都是不断激增,从未下降跌落过。

双方围绕着三怪和监天塔,天庭和三域蛊仙战作一团。你来我往,激烈异常。战团不断转移,所到之处,种种杀招如惊涛骇浪,余波阵阵摧毁一切,酿造无数废墟。大好的天庭,被蛊仙们精心经营,此战却是毁了个七七八八。

眼看着三域蛊仙牢牢占据上风,监天塔忽然顿住。

下一刻,仙道杀招命败再次爆发。

“小心!”武庸大吼。三域蛊仙早有防备,纷纷迅速应对。

白光消散之后,天庭战场再次爆发浩然炸响,激战再次爆发。

但这一次轮到天庭占据上风,三域蛊仙不得不一边抵挡,一边修复仙蛊屋还有自身的伤势。

三域蛊仙逐渐稳住阵脚。和第一次遭受命败打击相比,他们的情况要好得多。皆因多了三头巨怪可以依靠。

巨怪仍旧横冲直撞,在它们的发挥下,天庭蛊仙又再度被渐渐拖入僵局,而后又被三域蛊仙压入下风。

这个时候,监天塔再度催动命败杀招。

如此三番五次的酣战之后,天庭被打得到处都是废墟。

三怪的气势明显跌落下去,两方蛊仙的攻势也不复之前那般迅猛。他们不得不开始考虑仙元的消耗问题。

“糟糕,情况越来越不妙了。”武庸脸色凝重,三怪的战力不复之前,它们屡屡遭受命败杀招的摧残。

“当年狂蛮魔尊进攻天庭失败,这三头巨怪不过是他留下的手段罢了,自然也难挡命败之威!”冰塞川心中渐渐冰凉。

如果不在三怪发挥的期间,三域蛊仙得出突破性的战果,那么接下来他们必将遭受惨败和屠戮。

这是一个很明显的事情!

但天庭一方却采用了最明智的战术,他们不断转移,避敌锋芒,硬生生地消耗敌方。尤其是一次次的命败杀招,总是让三域蛊仙好不容易争取到的优势又化为乌有,一切又得从头再来。

又因为有着诛魔榜、监天塔的存在,导致战线上随时都有接应的大本营,天庭蛊仙一旦出现险情,便躲入其中,因此更无无一人减员。当然,这也有双方开始珍惜仙元,攻势减缓的缘故。

“主人,形势不妥。即便是狂蛮魔尊的手段,也难改大局。天庭毕竟有了完整的宿命蛊了。”远处,紫薇仙子冷漠地分析道。

魔尊幽魂此刻已经恢复成人形,身躯半透明,他背负双手,遥望战场,又看了一眼极远处的一缺抱憾亭。

一缺抱憾亭中,双尊虚影仍旧在对弈。

棋局上已经布满棋子,但玄妙的是,每当尊者虚影再落一子时,又总有地方可下。

只不过相比之前,双尊落子的频率也下降了许多。

星宿虚影手中捏着一颗棋子,观看棋局,又看了一眼魔尊幽魂的方向。

“你这样好吗?我们对弈的成果,都被幽魂窃取了一份去。”星宿虚影开口道。

无极虚影笑道:“他既然能够脱困,又能够将我们的成果窃取,都算是他的本事。这些对他都有帮助,我倒是盼着有一天,他能够达到我们都未达到的境地。”

星宿虚影沉默一会,这才叹息:“你果然是个纯粹的求道人。”

“可惜天地是一个牢笼,阻我求道。”无极虚影答道,“所以我才要追寻自由。”

星宿虚影立即反驳:“然而你的自由,是要直接打破这片牢笼。而你我都知道牢笼之外的是何等的景象!”

无极虚影微微摇头:“立场不同,多说无益。”

他此刻的身影越发虚幻,惟独双眼的位置却是精芒四射。无极虚影手中捏着一颗棋子,久久望着棋盘,迟迟不下。

这一局已经到了最关键的时刻了。

棋局上,无极虚影的情势十分恶劣,处于明显的下风。就好像是被星宿虚影的棋子四面八方包围起来,星宿虚影有着很明显的优势。

然而即便是因为天庭积累,也足以让无极虚影看出对手星宿的强大。

星宿意志一方面要对抗天意,一方面要布局五域,思虑无数方面,谋略囊括今古宇宙。且不说历代三尊攻打天庭未果的谋划了,单单就说眼前,魔尊幽魂重获自由,却没有得到生死门。

这恐怕便是星宿的手笔,毕竟连紫薇仙子都隐瞒,单靠龙公思谋的可能性不大。

星宿仙尊不愧是历代尊者中的唯一智道!

只不过,当下无极虚影也未必输了。他的主要棋子紧密地团结在一处,对手难以吃下,一旦强吃,必定损失极大,等若是将主动权送给无极虚影。

但若让无极虚影自己反攻突围,也尚且欠缺一份实力。整个包围圈凝结着星宿虚影的思谋,无极虚影若是盲目进攻,只会落入星宿虚影的算计当中。

所以,无极虚影只能选择固守待援。

这场从上古时代就开始,一直持续了上百万年的双尊对弈局,终于迎来了最关键的时刻!

没错,无极虚影是有援军的。

这个援军他等待了上百万年,双方都心知肚明。

“现在,就要看的成色了。”星宿虚影目光扫视战场,最终定格在方源的身上,面露复杂神色。

无极虚影同样回首眺望,口中呢喃:“方源啊,从开战以来你便如此。你……还要藏拙到什么时候?”

------------

\end{this_body}

