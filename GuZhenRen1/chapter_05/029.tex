\newsection{伤势痊愈}    %第二十九节:伤势痊愈

\begin{this_body}

d

这一日,云盖大陆上,忽然散射出一股冲天华光。<strong>最新章节全文阅读WWW.qiushu.cc</strong>△,

华光五彩缤纷,流光溢彩,宛若一根天柱,直冲九霄。

它是如此的耀眼,黑毛、黄毛、白毛三大陆上,毛民们只需要仰头,就可目睹。

“这是仙迹啊!”毛民们纷纷感慨。

若是以前,三大陆上,关于蛊仙的消息,都只是捕风捉影的缥缈传说。

但当琅琊地灵改变个性之后,这个情况就发生了天翻地覆的转变。

琅琊地灵亲自颁布仙旨,效仿王庭之争,挑动三大陆之间的战争,并从中挑选出精英种子,带入云盖大陆修行。

三大陆的战争,影响了所有的毛民,而现在云盖大陆上的“仙迹”,再次激发了毛民们的热情。

“成仙得道……”黑毛国主凝望长空,双眼绽射出精芒,令人不可逼视。

精芒收敛下去,黑毛国主忽然下了一个决心:“来人呐,传我旨意。仙迹当前,此乃大大的吉兆。我国上下感念,特举办比武大会。比武大会中的优胜者,我将开放国库,任其挑选珍藏!”

“是。”侍卫立即领命,躬身退下。

黑毛国主的这个念头,并不是突发奇想,而是已经深思熟虑过好多时日。

三大陆战争,让黑毛国主意识到,往日炼蛊第一的狭隘思想,需要抛弃了。为了今后的战争,他急切地需要强大的精锐,擅长战斗的蛊师。

黑毛国主一声令下。一场巨大的风波就从王城,迅速向四面八法辐射。影响越来越大。

晃晃乎,半个月后。

铁丝城中。迎来了一位贵客。

这位贵客,得到了铁丝城主的亲自招待,因为他的身份同样也是一位城主。

“话锋城主,你这次来,有什么要事吗?”迎接的宴会中,酒过三巡,铁丝城主问道。

“铁丝城主,你是明白人,我的来意你早已知晓。”话锋城主笑了笑。指着身后站着的几位毛民蛊师道,“国主大人要召开比武大会,但每个地区只选送三人。我们这块地方,只有我们两座城池。现在我身后的这些蛊师,就是我要送上去才加大比的人选。不过国主给下的人数有限,我们就先来比试几场,决定究竟是谁去参加比武大会!”

“哈哈哈。很好,我也正有此意。”铁丝城主大笑几声,拍拍手。将幕后的几位毛民蛊仙,都召唤上殿。

两方各派遣一人上场,开始了交锋。

几轮下来,话锋城一方屡战屡胜。铁丝城这边却是无一胜场。

话锋城主含笑,悠然喝酒,铁丝城主面沉如水。心头沉重:“没想到话锋城主这次有备而来,挑选的蛊师竟都是如此精锐。糟糕。我方若是毫无得胜,此战结果必定会被大加宣扬。今后数年,他话锋城就要压住我铁丝城一头了。”

“只剩下最后一场了。铁丝城主,还请调兵遣将吧。”话锋城主发难道。

铁丝城主冷哼一声,将目光集中到最后一位毛民身上。

但这位毛民蛊师,满头冷汗,现在这个局面,他也明白肩头的重担。但正是因为如此,他才感到越发信心不足。

不只是他信心不足,酒宴上铁丝城的上下高层,都感到信心不足。[看本书最新章节请到棉花糖小说网www.mianhuatang.cc]

这时,一位城中元老向铁丝城主暗中传音道:“城主,我有一个人选,可为我方挽回颜面。”

铁丝城主闻言大喜,连忙相问。

这位元老便答道:“城主大人,您莫非忘了,您最近收了一个人奴,他可是五转修为呢。”

铁丝城主一愣,面泛苦涩,传音回去:“这不妥!我这边和话锋城主交锋,是名正言顺,堂堂正正。若是要让我这位人奴出场,败了不说,就算是胜了,他就要被我举送上去,参加比武大会。这要是让国主看到,我铁丝城什么毛民不送,却送一位人族上来,还不责难我吗?”

元老一笑。

他心知,这都是铁丝城主的托辞。

其实是铁丝城主心中不舍罢了。

那个叫做“方源”的奴隶,自从被城主买下,几乎夜夜都要被铁丝城主临幸。这个情况,早已引起了许多高层的不满。

云老接着道:“城主大人,你尽可宽心。当今国主,宏大雅量,唯才是举,身边辅佐的国老中,就不乏雪人、羽民。城主若派遣方源出战,若败了,就可让他担负责任。若胜了,就保留了我城的颜面,将他选送上去,国主大人必定对城主大人您刮目相看,甚至引为知己,也未可知啊。”

铁丝城主大有深意地看了这位元老一眼,心中嘀咕:“这个老不死的,好算计!不就是因为我最近,冷落了你送上来的侄儿吗?居然献出如此毒计。胜了,我要把方源送上去。败了,我也要杀掉方源,将责任推卸给他。不管胜败,我都要失去方源。”

“唉,罢了!”铁丝城主心中大叹一声,“我岂会是因美色而弃大业于不顾的蠢货呢?”

铁丝城主虽然夜夜临幸方正,为他着迷,但她本质是却是个铁石心肠,权力**旺盛的毛民。

在她心中,区区美色不算什么。

于是,铁丝城主便临时撤换下那位毛民,将方正唤来,对他嘱咐一番。

方正得知缘由之后,走下场中。

“竟是五转蛊师!”话锋城主心中戈登一下,察觉到方正的五转气息,顿时警惕起来。

他笑道:“按照道理,这一场该由我方来决定比试的内容。前几场,咱们都是比试了武斗,但咱们毛民的精髓和传统,怎么能够丢弃?这一场。就以炼蛊论输赢罢。”

此言一出,铁丝城主差点破口大骂。

铁丝城一方的元老们。也都纷纷向话锋城主投去鄙视的目光。

但话锋城主担任一城之主多年,早已经练出深厚的脸皮。对这些目光视若无睹。

铁丝城主无法反驳,毕竟是之前设定下来的规矩,只好对方正挥手道:“你下场比试吧。务必把你的全力发挥出来。你要记住一点,此战有胜无败。若是败了,你就用你的性命来抵罪吧。”

方正听了,仍旧面无表情。

他这些天受尽了屈辱。铁丝城主**强烈,喜好各种姿势,一夜十七八次,毫无节制的索求。让方正生不如死。

方正真想一死了之,但身为阶下囚,受制于人,求生不得求死不能。

铁丝城主此时用死来威胁他,他反而有一种解脱之意。

脑海中,顿时一个念头泛起:“我还会留恋这个世界吗?不如直接认输,索性求死!”

但就在这时,他的脑海中响起一道声音:“愚蠢!大丈夫活在世上,要能屈能伸。点点侮辱算得了什么?有种的,就在将来复仇,将给与你一切屈辱的仇敌,都斩成碎肉骨渣。这才是英雄所为!”

“什么人?!”方正吓了一跳。惊呼出声。

那边,话锋城的最后一位毛民蛊师,刚刚下场。

还以为方正是问的他。于是胸膛一挺,大声地道:“在下茅十八!”

方正毫无反应。

他此时已经得了自由。还借得许多蛊虫,那个脑海中的声音陡然出现又消失。任凭他此时如何内视审查,都寻找不到任何痕迹。

“可恶的家伙,安敢辱我?!”茅十八大怒。

他报了姓名,但方正却无动于衷,似乎是不屑于将姓名报给他听,顿时让茅十八火冒三丈。

“这家伙怒什么?”方正被茅十八的吼声惊醒,心中暗暗奇怪。

因为这个误会,这场比试还未开始,就已经火药味十足。

在众人注目之下,比试开始。

茅十八是话锋城最强的蛊师,堪称文武双全,不仅擅长争斗之法,也很擅长炼蛊。

反观方正,虽然在仙鹤门中进修过一段时日,然而却是主修的奴道,炼道方面只是浅尝辄止。

比试刚刚开始,茅十八就一路领先。

比试到了中期,茅十八已经将方正甩得老远,领先优势十分巨大。

到了最后关头,观战的毛民们,不管是铁丝城一方,还是话锋城,都不认为方正有获胜的可能。

就连方正自己,也是这么认为。

“就要输了吗,也罢,死了也算是个解脱。”他心中叹道。

“放屁!”这时,刚刚出现的声音,又忽然再次响起。

方正一惊,双手一抖,炼出的半成品顿时功亏一篑。

“哈哈哈。”话锋城主大笑。

铁丝城一方,则皆脸色难看。

方正的脑海中,那股声音却是继续道:“男子汉大丈夫,小小的挫折,算得了什么?”

“你,你是……方源?!”这一次,方正终于听出了声音。

“呵呵,我的本体,早已经死了。只剩下这一股意志,残留在你的脑海中。你身为我的弟弟,活在世上,却不思进取,不图报仇,实在让我看不下去!”方源意志喝斥道。

方正冷哼一声,暗道:“已经死了,你怎么还出来作怪?”

他以前有天鹤上人的魂魄,附身指导修行,现在对话方源意志,轻松自如,甚至带给他一丝淡淡的熟悉之感。

而外表上,毛民们也看不出什么蹊跷。只当这面无表情的方正,是不是被这个结果吓傻了。

方源意志感受到方正的敌意,却不以为意地笑了笑:“我既然已经死了,那你为什么还要冒名顶替我?”

方正沉默。

方源意志接着道:“你受到屈辱折磨,不愿以真名告知他人,是说明你心中还有羞耻、愤怒和仇恨。那么你为什么现在,不发愤图强,争取这个上佳的机会呢?你也知道事情缘由,只要你胜了这场,你就能名正言顺地参加比武大会,脱离奴籍,成为自由之身!”

“我也想得胜,谁不想报仇,谁不想自由?可是此战败局已定了!”方正愤然地道。

“哈哈哈。”方源意志大笑,“你只需要按照我的指点,就可一举超越茅十八,反败为胜。”

方正一愣,旋即道:“你又有什么阴谋诡计?”

“哼。”方源意志冷哼一声,“我虽已亡故,但不甘心!我要报仇!杀死我的毛民蛊仙,一个都跑不掉。而你就是我复仇的希望!你不是假冒我的名字吗?那就替我报仇!”

“我为什么要替你这个魔头,丧心病狂到屠戮全族的人报仇?!”方正怒道。

方源意志却直接道:“你再啰嗦,就没有时间了。听我的吩咐,起文火,放三金。记住,玄金、冰金、泪金,要依次放入火中烤制。”

方正狠狠咬牙,虽然他心中极不愿听方源的吩咐,但方源意志的一番话,却让他看到了胜利的点微希望,事关又关乎他的性命。

犹豫了一下,他终于伸出了手。

重新炼蛊!

围观的毛民们见此,发出一阵的嗤笑声。

但半盏茶之后,他们都笑不出来,而是震惊地望着场中的方正。

他高高扬起右手,手中捏着的蛊虫,正是他在众目睽睽之下,炼制出来的。

“我胜了!”他深呼吸一口气,大声地宣布道。

殿中,仍旧无人开口。

就连他的对手茅十八,也似乎接受不下这样的事实,一脸呆滞。

气氛微妙。

在场的毛民们都不愿承认,一位人族在炼蛊的比试中,战胜了毛民蛊师中的强者!

半晌,忽有人道:“你们快看,云盖大陆上的五彩华光,正缓缓消散!”

众人不由地循声望去。

数十道目光,穿过殿中大门,果然看到云盖大陆上的五彩华光,逐渐消散。

消散的五彩华光中,走出一个人影。

“看来你已经痊愈了。”琅琊地灵望着来人,笑道。

“是的。”来者点头,“这还得多谢太上大长老你啊。”

说话间,五彩华光散尽,露出来者的真容。

若是方正见到这个人,必定会大吃一惊。

来者正是方源!

琅琊地灵哈哈大笑:“你要谢我,还不简单?将荡魂山、落魄谷或者智慧蛊,直接送我吧。”

方源亦淡淡笑道:“送是不可能的!但我们之间可以交易。不知道太上大长老你,能出什么价呢?”

“呃。”琅琊地灵的笑声戛然而止,“这……”

他口中迟疑,满脸尴尬之色。(未完待续。)<!--80txt.com-ouoou-->

------------

\end{this_body}

