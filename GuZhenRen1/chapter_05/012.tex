\newsection{如今迈步从头越}    %第十二节:如今迈步从头越

\begin{this_body}

南疆,一处无名的山峰顶端。

方源左手一只剑遁仙蛊,右手一只飞剑仙蛊。

他左右观察。

剑遁仙蛊形如金针蜜蜂,飞剑仙蛊则好像银翅蜻蜓。

两者各有千秋。

前者乃是移动仙蛊,移速之快,不逊色于七转的气遁仙蛊。后者则在历史赫赫有名,是剑仙薄青生前最常用的攻防蛊虫。

七转的气息,四下洋溢。

方源嘴角含笑,将两只仙蛊收起来。

经过义天山的这段时间,剑道仙蛊都已经在智慧光晕下,成功炼化,都成了方源之物。

这两只仙蛊自然也在其中,此时此刻能被方源随时催动运用。

有了这两只仙蛊,方源心中都因而安定了许多。

之前他有态度仙蛊、换魂仙蛊以及解谜仙蛊,严重缺乏攻伐、转移的手段。现在这两只仙蛊在手,立即填补了方源的短板,令他战力暴涨!

若再次面对火崆峒,方源都可以直接催动仙蛊,进行攻杀了。

通天蛊已经停用。

因为琅琊地灵送来了剑遁、飞剑两只七转仙蛊,宝黄天中热闹无比,越来越多的蛊仙加入其中,都在疯狂的议论此事。

蛊仙界中,还是六转蛊仙占据绝大多数的。

六转蛊仙当中,大部分人都没有一只六转仙蛊。

因此,这笔关乎到两只七转仙蛊的交易,才引发如此的轰动。

宝黄天只是一个市场,十分开放,无法保密。方源通过宝黄天输送仙蛊,也是迫不得已之事。

手中有了两只剑道仙蛊后,他就立即停用了通天蛊,至于宝黄天,就让它自己热闹去罢。

通过推杯换盏蛊,他继续和琅琊地灵联络。

“这一次,你一共耗费了三十的门派贡献。”琅琊地灵来信道。

利用宝黄天输送蛊虫。不是无偿的。

每一笔交易,宝黄天都要收取酬劳。手续费用按照宝光裁定,宝光越高,手续费用越高。

这些手续费用。都是琅琊地灵负担了去。转而琅琊派内,就是扣除方源的门派贡献。

于是,方源的门派贡献,就很快缩水,剩下二百七十。

当然。耗费的三十贡献中,有一部分是方源采购蛊虫的代价。

“我想和你详细谈谈,有关胆识蛊的贸易。”方源回信过去。

这一场交谈,耗费时间颇长。

直至半夜三更,双方才敲定了全部细节,彻底谈妥。

胆识蛊贸易!

方源双眼放光,脸上还残留着兴奋之色。

夜风呼啸,带来寒意,但此刻却扑灭不了他心头的火热。

一直以来,方源的胆识蛊买卖。主要都受限于毛民奴隶的数量。

因为胆识蛊需要气囊蛊,才能装载进去,离开荡魂山,对外售卖。但气囊蛊,一是需要黑楼兰的力气仙蛊催发力气,充当主要的炼蛊材料;二则需要毛民奴隶来不断炼制。

现在,黑楼兰失踪,力气仙蛊也不在方源身边。

不过这没关系。很早之前,方源就利用智慧光晕,研究出了新的蛊方。只是一直按捺不发。是想维持和黑楼兰的利益联系。

毛民奴隶的数量,才是制约此项贸易的主要因素。

然而要增长毛民奴隶,却是非常困难的。

一来,毛民奴隶在宝黄天中。是所有异人奴隶里售价最高的。尤其是炼蛊好手,更加高昂。

二者,炼蛊的过程中充满了危险,毛民奴隶会在炼蛊的过程中丧命。所以即便方源陆续补充了不少毛民奴隶,但在狐仙福地中,他们的规模还是没有增长。

方源一度想要自己豢养毛民。但因缺乏毛民的豢养法门。这个想法无疾而终。

但现在!

方源投靠了琅琊派。

琅琊福地的毛民,有多少?

数量多到难以想象的地步。

这块福地,或许就是当今五域中,最大的一块异人毛民的聚居地点了。

将荡魂山、落魄谷借给琅琊派,方源自然是有他的图谋。

利用琅琊福地中海量的毛民,为他炼制气囊蛊。胆识蛊的贸易数量,将远远超过方源之前的规模,暴涨到一种无法估量的地步。

“只要有宝黄天,就算中洲打压,又算得了什么?胆识蛊的贸易,不会受到任何的阻碍。”

“真是期待啊。按照刚刚商议的结果,就算琅琊派分走一大半的收益,我每个月单纯因为胆识蛊贸易,就能收获到六千仙元石!”

方源之前,加上其他的经营项目,每个月也不过毕竟两千仙元石的收益。

现在投靠了琅琊派,单纯胆识蛊贸易一项,他就能分到六千仙元石。

这是很多六转蛊仙,都难以想象的一个数字!

方源此次重生,决定加入琅琊派,是经过深思熟虑的慎重考虑。

第一,当方源有朝一日面对全天下的追杀时,还有什么能比琅琊派更可靠呢?别人觊觎方源手中的巨阳真传,但也同样觊觎琅琊福地。琅琊地灵改变之后,一心想要光大毛民一族,是方源的天然盟友。

第二,琅琊福地本身实力极强,五百年前世抵御了整整七波攻势。

第三,琅琊地灵无法说谎,一旦有异心,方源就能轻松察觉。

综上三点,方源才决心加入琅琊派。

果然,在重获新生的第一晚,就尝到了巨大的甜头。

方源不仅利用琅琊地灵,帮助自己平定了后方危机,而且还借助琅琊福地的底蕴,极大地增长了自己的财源。

“重获新生的虚弱期,已经基本渡过。接下来,就是好好检测一下我的这个九五至尊仙窍了。”

今夜夜空无星,寒风呼啸。

方源钻入屋蛊之中,饱餐一顿后,开始着手深入详实地查探仙窍。

有了一干凡蛊的帮助,自然效率更佳。

一夜无眠。

凌晨时分,方源钻出三星洞蛊,正好看到日出一幕。

南疆本是多山多雾,旭日冉冉升起,将重峦叠嶂的山的那边,燃烧起漫天的火云。

一轮大日,将天边拉开一角,起先只是现出一道金红的弧边。

随后,大日喷薄而出,携带赤红的热量,宛若从铁炉中流淌出来的金铁。

大日冉冉上升,山间的云涛波澜壮阔,被阳光映照得色彩斑斓,朝霞万顷。

方源心中,不禁涌起壮志豪情。

“雄山漫道真如铁,如今迈步从头越。险就一身乾坤精,我心依旧望苍天。”

低声吟罢,方源张开双臂,宛若一只大鹏鸟,跳下山峰,扶摇直上。好似踩踏着漫天云雾,朝着东南方向而去。

这一夜过去,他对至尊仙窍知道的更多。

仙窍空间广阔,超越五亿亩。不仅如此,宙道资源也十分丰富,外界一天,至尊仙窍中就过了两个月。

黑楼兰的大力真武仙窍,和外界的时间流速比,也不过一比三十八。

方源的至尊仙窍,是一比六十,大大超越了黑楼兰。

不过和仙窍空间的差距,比较起来,宙道方面并不出奇。

但这反而让方源松了一口气。

仙窍当中的时间流速过快,会使得灾劫来得极其频繁,会让方源疲于招架。

或许魔尊幽魂在当初炼蛊的时候,就已经考虑到了这个问题。一比六十这种程度,超越了特等福地,但也并不夸张,让方源能勉强接受。

除了宙道资源之外,方源还从至尊仙窍中,发现了十六颗青提仙元。

仙窍鲜活,自然不比死窍,可以自产仙元。

一天就能产出十六颗仙元,一个月就是四百八十颗仙元,远超凡俗,更把方源前世甩到几条街开外去了。

方源还发现,自己的这副肉身上,蕴含大量道痕。

这些道痕,什么流派都有。炎道、水道、木道、光道、暗道……甚至毒道、运道等等,都囊括其中。

更叫方源感到惊奇的是,这些道痕种类繁多不说,规模也很庞大。

他只能模糊地估算,平均下来,每一流派的道痕约有百条。

一百条道痕,就能增幅相应的蛊虫一成的效果。

难怪方源之前,运用炎道的火焰披风蛊、飞烟蛊等,效果出类拔萃了。

但这个发现,并未让方源有多开心,反而心头有些沉重。

蛊仙道痕向来追求单一。

因为道痕可以增幅蛊虫的效果,同样的,也会抑制和削弱。

比如精修水道的蛊仙,满身水道道痕,使用炎道凡蛊,反而不如凡人蛊师。

蛊仙就算兼修,也向来是一道为主。

这是蛊仙界的修行常识。

“但奇怪的是,我运用各种流派的凡蛊,并未受到其他流派道痕的抑制和削弱。反而呈现单方面的增幅作用,这显然违背了修行常识。究竟是怎么回事?”

“还有一点。历来蛊师升仙,都有本命蛊。但我这至尊仙窍,却是没有。这又是怎么回事?”

对至尊仙窍的奥妙,挖掘得越深,方源释疑的同时,也泛起更多的问题。

“昨晚的查探,已经用尽了我的手段。看来想要尽数了解至尊仙窍的奥妙,还得从魔尊幽魂、影宗、影无邪方面着手。”

“但义天山方面,变化太大,梦境笼罩,将来定会被南疆蛊仙们层层封锁,成为禁地。而影无邪又不知去向。”

“罢了,当务之急,还是先找到南疆这一代的泥相!”

ps:第二更。(\~{}\^{}\~{})

\end{this_body}

