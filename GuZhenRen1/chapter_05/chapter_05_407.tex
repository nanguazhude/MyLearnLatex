\newsection{谋取红莲真意}    %第四百零七节:谋取红莲真意

\begin{this_body}

%1
橘黄色的光辉,在凤九歌的身上,猛地一闪即散。

%2
这个迹象表明,凤九歌的仙道杀招一曲阳关,已然到了时限。他想要再度运用,非得等过七个时辰。

%3
但这都不是重点。

%4
凤九歌已经彻底脱离的战场,一曲阳关到没到时限,都没有关系。

%5
这位中洲千年难得一出的人杰,此刻双足缓缓落到地上,仰头望着苍穹。风采依旧的脸庞上,却是眉头微皱,良久,凤九歌发出一声感叹。

%6
“真乃人杰也!”

%7
这番话,自然不是他评价自己,而是说的方源。

%8
曾几何时,方源不过是凤九歌脚边的蝼蚁,甚至前不久,方源就算是集结群仙之力,也抗不过凤九歌一人,不得不主动撤退。

%9
但刚刚一战,却是凤九歌不敌,轮到他逃离战场。

%10
这样的转变,未免太过于迅猛,以至于凤九歌到现在,都有些难以置信。

%11
“方源原本就底蕴雄厚,从未来重生,又拥有春秋蝉、巨阳真传,如今成为影宗之主,更有了魔尊幽魂的积累。他就像是一个火山,表面上静默,但只要给予他一点时间或者机会,这座火山就能喷吐出冲天的岩浆!”

%12
“这一次他的爆发,并非偶然,而是必然。”

%13
“这样的人物,应当竭尽全力,直接打杀,以绝后患。天庭却要吊着,缓缓图之,谋夺红莲真传。这未免有些不智……”

%14
头一次,凤九歌对天庭的追剿计划,产生了怀疑,甚至还有一丝否定。

%15
天庭,紫薇仙子亦在沉吟。

%16
刚刚一战,她都通过凤九歌,尽收眼底。

%17
方源的战力猛地暴涨,不仅是让凤九歌措手不及,也让她这位堂堂智道大能感到惊异。

%18
类似凤九歌的心绪,其实也在紫薇仙子的脑海中酝酿。

%19
方源的底蕴太浑厚了,前后和三位尊者有着牵连。又有着重生的优势,所以短短数年时间,就已经从一个凡人,成为蛊仙。这还不算,经过刚刚一战,方源彻底和凤九歌分庭抗礼,成为八转蛊仙都无法忽视的存在了。

%20
他的进步太过迅猛。

%21
若只是机缘太多,类似马鸿运也就罢了,这并不值得紫薇仙子忌惮。

%22
可怕的是方源的性情!

%23
这样野心勃勃的魔道贼子,几乎毫无人性上的弱点,他努力到了极致,坚持自己从未妥协或者气馁,他能吃苦,更能隐忍,为达到目的,不计牺牲,不择手段。

%24
难以想象,究竟是什么样的经历,能铸就他如此品性。

%25
“方源重生以来,大部分时间,都在天意的掌控当中。义天山大战后,他脱离天意桎梏,但又在梦境中暴露了身份。他的修行之路,一直坎坷无数,波折不断。若真的给他一段时间潜心精修,他的实力增长将难以揣度估测!”

%26
忌惮。

%27
紫薇仙子也不由地从内心深处,生出一丝对方源的忌惮。

%28
方源的存在,让紫薇仙子感受到了威胁。

%29
真想把他铲除掉!

%30
越早铲除越好,越快铲除越妙!

%31
这般想着,紫薇仙子心中的杀意,越来越多。

%32
“紫薇啊,不必紧张。”就在这时,一声低沉的声音,传入紫薇仙子的耳中。

%33
这个声音,紫薇仙子印象极其深刻。

%34
她连忙转身,果然见到了龙公。

%35
“龙公大人,您出关了?!”紫薇仙子顿时惊喜交加。

%36
龙公淡淡一笑:“是的。”

%37
“这么说来,魔尊幽魂他也被您……”

%38
“不错,他已经被我彻底镇压。然而要搜刮他的魂魄,却是有些麻烦,需要一段时间,才能见效。”龙公轻笑出声。

%39
“啊!”紫薇仙子满脸的欢喜,一双美眸熠熠生辉。

%40
这个大好消息,让她精神陡振,整个人都在刹那间变得不一样起来。

%41
龙公继续道:“和红莲真传相比较起来,方源算不得什么。”

%42
“可是,他毕竟是完整的天外之魔……”紫薇仙子迟疑道。

%43
“那是因为你不了解红莲魔尊啊。”龙公叹息,语气中饱含深沉的感慨。

%44
“红莲设下的真传,似乎有七道,但我们天庭竭尽全力,都无法寻觅。幽魂魔尊获得了其中一道,只要我们得到这个方面的线索,就能顺藤摸瓜,将其余的红莲真传都统统连根拔起!”

%45
“我明白了,龙公大人。”紫薇仙子目光凛然,“光阴长河中已经部署了一位宙道八转,再加上凤九歌,以及其他两位八转,这个局您看如何?”

%46
龙公笑了起来:“你是智道大宗师,我却不是。此事由你来安排,我放心得很。接下来,我将引领新一代的大梦仙尊。哦,对了,将魔尊幽魂被关押俘虏的事情,也通告天下吧。”

%47
“天庭沉寂太久,是时候让天下都知道,天庭永远是天下第一的超级势力!”

%48
“紫薇遵命。”紫薇仙子深呼吸一口气,目送龙公的身影缓缓消失在原地。

%49
西漠。

%50
洁身自爱仙道蛊阵当中,方源正向至尊仙窍中投入心神。

%51
仙窍中,也有一座仙道蛊阵正在轰然催动着。

%52
和洁身自爱相比起来,这座仙阵就小了很多,不过核心仙蛊仍旧是有两只。

%53
一只是得自紫山真君遗藏的智道仙蛊,另外一只则是来源武家的血脉仙蛊。

%54
再辅助数千凡蛊,从而形成了仙道蛊阵——血红仙元阵。

%55
大量的仙元石被投入这座仙阵当中,被其碾磨转化之后,就形成了一颗颗的红枣仙元。

%56
方源正在补充仙元储备。

%57
刚刚一战,他虽是大跑了凤九歌,但是仙元的消耗却是巨大的。最大头的消耗,来自于逆流护身印,第二是万蛟杀招,全新杀招万我也比之前消耗更多,毕竟是用了自爱、爱意两大仙蛊产出海量的我意出来。

%58
这点并不是什么弊端。威力越强的仙道杀招,往往耗费越多的仙元,这是修行的常理。

%59
但这点是方源的短处。

%60
方源毕竟成为七转时日太短,没有太多积累,原本仙窍中的宙道资源,也因为顾忌灾劫的缘故,一削再削,直至谷底。

%61
血红仙元阵效率极高,很快,方源身上从武家得来的那些仙元石,就都消耗光了。

%62
方源又将从琅琊派处,兑换得来的仙元石,投入到血红仙元阵中去。

%63
红枣仙元的储备,便节节攀升上来。

%64
“我的战力提升了,足以媲美凤九歌。”

%65
“美中不足的是,万我需要提前准备。若是猝不及防的战斗,那就悬了。”

%66
面对凤九歌,方源或许还可以冒险,在战斗中催使新万我。但要面对八转蛊仙的话,方源就得立即动用逆流护身印保命。

%67
如此一来,在催动逆流护身印的基础上,没有万我辅助,方源就不能催动其他仙道杀招。最终只能顶着逆流护身印被动挨打。

%68
“等到身上的侦查杀招都被削除,我就该前往光阴支流,汲取红莲真传了。那道真传虽然是被幽魂魔尊获取,但里面的红莲真意生生不息,我若得之,便可将宙道境界暴涨至宗师级数!”

%69
在影宗的情报里,西漠这里一共有两处光阴支流,是受到掌控的。

%70
但是之前,方源已经耗用了其中一条,布置陷阱,坑害天庭追兵。如今只剩下一道。

%71
这一条光阴支流,就是方源接下来的目的地。

%72
眼下,中洲是禁地,方源万万进入不得。南疆呢,他会受到武家等正道的追杀。进入北原,长生天以及黄金家族都会群起而攻。

%73
只有西漠和东海,是方源的活动地带,压力较其他三域要小得多。

%74
当然,方源也可以去往东海。但是要去东海,必须得越过中洲、北原、西漠的其中之一。

%75
东海当然也有光阴支流,但方源与其去往东海,还不如直接在西漠中,就近谋取机缘呢。

%76
备注:八月份的明信片活动,今天寄出了大部分。获得明信片的同学,希望最近几天多注意自家邮箱。寄快递太贵了,一个外省快递12元。本来是想寄挂号信,但是需要真名,签收需要身份证,所以只好寄平常邮件。

%77
平邮已经大部分寄出去,但是也有一些,因为地址不够详细,寄出去肯定是收不到的。

%78
希望下列名单中的同学,能够联系我的qq,将详细地址给我。

%79
周玉、酷氏章鱼、郭新(无忧)、墨雨墨潇、王先生(我就最爱你)、葬冥邪、古月阴荒、罗京(蛋蛋)、魏佳锐(神游)、血月、膜拜楚轩、幽魂魔帝、菁华、由秀明(只是些许风霜)、颜胜东、定仙、夜弑苍雨、语宝(流の心语)、何恒(死神假面傀儡舞)、枫无涯。

\end{this_body}

