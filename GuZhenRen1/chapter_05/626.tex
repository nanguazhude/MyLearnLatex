\newsection{方正升仙}    %第六百二十九节:方正升仙

\begin{this_body}

%1
中洲。

%2
天空阴沉沉一片,无风。

%3
古月方正站在一处山巅,仰头望着天,想到即将要渡升仙之劫,心中升腾起一丝紧张的情绪。

%4
察觉到这种情绪,方正便失笑一声,心中暗道:“我紧张什么?真正该紧张的人,应当是仙鹤门的人,或者说,是更幕后的天庭吧。”

%5
“表面上,此次我渡劫只是有一位六转蛊仙护持着,但暗地里应当有许多关注的眼睛。有这些人在,我还担心什么呢?”

%6
尽管方正不知道为什么,中洲这些人要扶持自己,但他已确认自己是有利用价值的。

%7
很久以前,他绝不会甘心被利用。青茅山上方源曾经隐晦指出他被舅父舅母利用,成为争夺遗产的工具,这让当时的方正非常愤怒。

%8
但现在,方正不会愤怒了。

%9
在过去的十多年里,方正经历了琅琊福地三大陆的延绵战火。为了栽培出战力出色的毛民蛊仙,黑毛地灵不惜代价,令三大陆上的毛民相互内斗,方正也被卷入其中。

%10
在阴谋诡计迭起,血腥拼杀不绝的战争中,方正利用过别人,也被别人利用着。

%11
他渐渐明白了:有时候成为别人的工具,被别人利用,并非不是一件好事。至少,它证明了自己的价值。若是一个人连成为工具的价值都没有,那他就危险了,往往会被舍弃,沦为弃子。

%12
当然,即便是方正五转蛊师的实力,也有被抛弃、背叛的经历。毕竟他是一个纯正的人族,却生活在满是毛民的琅琊小世界中,遭受排挤和种族歧视,是相当正常的。

%13
“可以开始了。”这时,方正的耳畔忽然传来樊西流的声音。

%14
樊西流修为六转,乃是仙鹤门此次委派过来,帮助方正渡劫的人选。

%15
方正点点头,沉下心来,立即内视。

%16
他的空窍很快展现出来,里面真元充沛,表明他甲等的资质,而内壁上却是有着许多的裂纹,让人看了有一种心惊肉跳的感觉。要知道空窍对于蛊师而言,乃是修行的根本。空窍中出现这么多的裂纹,绝对是一件大事!

%17
方正心中变得相当平静,之前的紧张感也不翼而飞。

%18
事实上,在琅琊福地中的一场战斗中,他在绝境中迫不得已动用了禁忌的手段,虽然最终险死还生,但空窍受到严重的损伤,充斥裂纹,甲等资质也跌落到了乙等。

%19
方正被凤九歌救走,回到中洲之后,他空窍中的伤势被治好,资质再次回复甲等。只是内壁上的裂纹还在,这并非天庭方面没有治疗的手段,而是留下这些裂纹更有助于方正接下来渡升仙劫。

%20
方正注视着自己的空窍,他曾经为自己的甲等资质无比的骄傲自豪,但有了这样的一段人生经历后,他对资质变得并不那么在意。现在他的目光,反而留恋于空窍内壁上的这些裂纹。

%21
对于他而言,这是一种功勋,是从血火刀枪中拼杀出来的荣耀。

%22
“别了,我的空窍。”方正心中呢喃。

%23
随着他念头调动,空窍中的真元掀起惊涛骇浪,不断地拍击冲撞四周的内壁。

%24
内壁本身就有裂纹,并不牢固,很快就顺利地碎裂开来,形成漏洞。

%25
原本无缺的空窍,终于打通了和外界的联系,一股玄妙至极的力量顿时形成。

%26
这股力量引动外界的天地二气,一时间,天空中乌云翻腾汇聚,大地则阵阵颤抖,掀起尘烟。

%27
与此同时,无形的天地伟力托着方正,使得他双足离开山石,缓缓升上了高空。

%28
“天意!”潜伏在一侧的樊西流脸上微微变色,他感受到了浓重的天意。

%29
方正渡劫,吸引了远超寻常的天意关注!

%30
灾劫开始成形,大量的血气从天地二气中转化而出。天空中的乌云很快被浸染成了一片血红之色,规模庞巨,方圆万里皆可目睹。

%31
“血道灾劫……”樊西流眼中喜色一闪即逝,出现血道灾劫正是上面的企图。

%32
方正的身上也开始外溢气息。

%33
这是他的人气。

%34
每一个人都有各自的人气,人气的多寡取决于蛊师的底蕴、才情、天赋等等。

%35
樊西流密切地关注着方正,很快他的脸上有了一层异色。

%36
方正冒出来的人气十分浓郁,显示出他丰富的人生经历和底蕴。

%37
天气垂下,地气上涌,在半空中相汇,和人气纠缠。

%38
方正面色凝重,尝试着掌控三气,好让它们平和融汇。

%39
三气刚刚开始接触,正是熟悉和掌控的最佳时期。方正早已经得到樊西流的指点,对这方面十分熟悉。

%40
但这个时候,灾劫已经酝酿成功,大量的细雨飘洒而下。

%41
这些雨水竟都是血液,血腥气非常浓郁。血雨洒下,空气中迅速弥漫起一阵阵的白雾。白雾很快被血雨浸染,化为弥漫整个山峦的不祥血雾。

%42
方正开始渡劫。他撑起防护,又催动蛊虫,尽力抵消周围的血雨。

%43
这个时期,他动用任何一种蛊虫,都会遭受反噬之苦。很快,他手中的蛊虫就因为反噬而毁灭。

%44
这也是蛊师渡劫成功后,往往蛊虫毁灭得一干二净的原因了。

%45
不过方正得到天庭的支助,准备得相当充分,至少手中的蛊虫是不缺的。

%46
他冷静防御,始终沉稳,没有慌张过。

%47
在琅琊福地的战争中,他经历过远比此时更加惊险的情况,心境早已得到了充分的历练。

%48
樊西流远比方正紧张得多。

%49
他盯着方正,目光一眨不眨。

%50
对于五转蛊师而言,渡升仙劫往往十分困难。因为他们不仅要面对灾劫,而且还要分心操纵三气,时刻维持三气的平衡。

%51
“这血雨的灾劫,我可以替你遮挡一番。但这三气平衡,可都得靠你自己了,方正啊!渡劫之前,我早已给了你大量的此类训练,关键时刻你可不能失误。”樊西流暗道,同时他开始出手,直接干扰血云,从源头上大大的影响血雨的规模。

%52
这场血雨灾劫虽然规模很大,但始终不温不火,似乎天意故意为之。

%53
方正渡劫,不管内在还是外在一直都很平静,无惊无险。

%54
三气融合得十分顺利,过往的一幕幕在他脑海中闪电般回溯,同时他的体质也得到洗练。

%55
随后,天地交感,他进入到升仙劫最黄金的时刻。在这个时间段,他可以师法自然,和天地进行交流。

%56
这种交流对于任何一个蛊师而言,都有天大的好处。

%57
方正在这一刻忽然发现,原来这血道真的挺适合自己。

%58
时间流逝,三气逐渐浓缩,形成混元的气团。

%59
方正忽然张开双眼,深呼吸一口气,将一只五转血道蛊虫猛地置入混元气团当中。

%60
轰!

%61
耳畔一声惊天的轰鸣,气团爆炸,顿时炸出一个仙窍来。

%62
一瞬间,方正心神放空,陷入到毫无防范的最脆弱的时刻。

%63
片刻之后,他回过神来,开始往仙窍中投放核心蛊、关键蛊等等。他首先投放的,当然是血仇仙蛊,其次是大量的血道凡蛊。

%64
仙窍福地中,三气逐渐得到了梳理,阴阳调和,天地逐渐稳定下来。

%65
他得到的当然是上等福地!

%66
并且,还有残余的天地二气存在。

%67
方正心中欢喜,有了这些天地二气,正可将他的本命血道凡蛊提升成仙蛊!

%68
虽然此举颇为危险,并且过程中还会伴有灾劫。但方正却有足够的勇气选择此路。

%69
“方正,快让我进去,护卫你渡劫!”樊西流这时再次传音。

%70
“嗯?”方正眉头一皱,面色不愉。这仙窍乃是蛊仙最隐私的地方,樊西流想要进来,顿时令方正感到了一种被冒犯的恼怒。

%71
但旋即,方正便舒展眉头,打开了仙窍门户,语气平平淡淡:“也好,樊西流仙友,请进来吧。”

%72
樊西流心中讶异了一下,方正升仙之后,还未彻底陈宫,就立即改变了对他的称呼,态度还这么自然。

%73
“不过,天庭要对付方源,似乎是要大力栽培方正了。在这一点上,我还是真的很羡慕他啊。”

%74
等到数个时辰后,方正渡劫完毕,樊西流的羡慕之情又暴涨了一倍。

%75
因为方正成功地使得本命蛊升炼,成为第二只血道仙蛊。

%76
冷血仙蛊!

%77
数日之后,仙鹤门向蛊仙界公开宣布,方正升仙成功,他曾经遭受过魔头方源的残忍迫害,被侥幸救出,今后将成为打击魔头方源的骨干成员。

%78
经过天庭方面的刻意推动,这个消息渐渐传遍整个五域,打击方源声望,突出他孤家寡人、众叛亲离的窘迫处境,同时又显出自身的宽阔胸怀和气量。

%79
又过数日。

%80
“灵缘仙子,恕不远送了。”云端上,樊西流一脸微笑,对赵怜云道。

%81
赵怜云笑了笑:“劳烦仙友招待了。”

%82
她这一次来,是听闻了方正渡劫成功的消息后,专程来拜访方正。可惜的是,方正并不见客,被仙鹤门雪藏起来。

%83
樊西流不敢怠慢赵怜云,因为她可是灵缘斋的当代仙子,即便只有六转修为,但却掌握着九转的爱情蛊。

%84
樊西流保证道:“在下一定将仙子的贺礼,亲自交到方正仙友的手中。”

%85
赵怜云点点头,感谢一声,驾云而去。

\end{this_body}

