\newsection{又得律道八转仙蛊}    %第七百七十五节:又得律道八转仙蛊

\begin{this_body}

一只仙蛊悬浮在方源的面前,散发出八转气息。

这只仙蛊如同一个蚕蛹,但并非是白色,而是浑身漆黑如铁。把握在手,也有冷硬沉重的金属质感。

这是八转律道仙蛊——加!

它以珠线水为食。

珠线水是一种八转的水道仙材,表面上看去是一团清澈无垢的水,毫不出奇。但若是伸手探入水中,轻轻一捏,就能把握住一股水线,将水线提出水面。

水线绵绵不断,水线上凝聚着一颗颗的水珠,紧密并列,相互挨着。整个水珠的线,仿佛是一串项链,美轮美奂。

若是一直将水线提拉上升,珠线水团随之缩减,直至所有的水团都化为一条相当长的水珠线。

仙蛊加是一个椭圆体的漆黑蚕茧,虽无口舌、缝隙,但只需要将它浸泡在珠线水中八天八夜,便能吃饱,之后维持八十年的时间。

“距离下一次喂养,还剩下三十六年。巴十八的仙窍中,已经有了大半团珠线水,剩下的可以和东海蛊仙交易。”

方源对这一切都心知肚明,了若指掌。

因为他早已经将巴十八的仙蛊,还有他的魂魄都搜刮出来。

巴十八在自家的仙窍中也经营着一处湖泊,湖泊面积颇大,里面豢养了大量的上古水蜘蛛,以及两头太古水蜘蛛,一雌一雄。

珠线水就是太古水蜘蛛的涎水,在久居的水里逐渐凝聚成的精华。

一滴一滴的积累,颇为缓慢。若是收集不及时,这些珠线水还会被上古水蜘蛛吞食,助长自身实力。

巴十八身为律道蛊仙,仙窍也大多是律道环境,他为了凝造出这么一块湖泊,也是煞费苦心。

喂养仙蛊有时候会比较麻烦。

就比如巴十八,虽然仙蛊加是律道蛊,但食材是珠线水。

巴十八要自行喂养,就得在仙窍中经营出一份自产珠线水的水道资源点,这对于一个律道蛊仙很不友好。

对于方源而言,这就不是问题了。

他的小东海太宽阔了,水蜘蛛进去后,简直是从小村庄进入了大城市。

方源甚至可以专门划出一片海域,来供这群水蜘蛛生活。有了宽阔的空间,还有更优秀的水道环境,这些水蜘蛛会繁衍得很好。

拥有至尊仙窍的方源,能完美兼修多种流派,这在养蛊方面也具备极大的优势。

巴十八的仙蛊,已经搜刮完了。

仙蛊加是他仅有的一只八转仙蛊。

除此之外,就是七转、六转仙蛊,其中有两只七转数蛊,分别是一和三。

数蛊是律道蛊虫中极其庞大的系列,一二三四五……可以说,毫无极限可言。

别看方源拥有这么多的八转仙蛊,事实上,正常的八转蛊仙能够拥有一只八转仙蛊,就已经相当不错了。

巴十八拥有一只,夏槎拥有两只,主要是因为这两位乃是正道超级势力的首脑,有着底蕴和继承。

寻常的散修八转,都是苦求一只八转仙蛊而不可得。举个例子——北原的雪胡老祖,他战力强大,乃是当今北原第一魔道蛊仙,但手中却是没有八转仙蛊的。

就算是天庭,也是八转蛊仙多,八转仙蛊少,比例相差极大。上一世,长生天进攻,天庭方面就出现了一个很尴尬的情景,许多八转蛊仙苏醒了,但却没有足够的仙蛊催用。

巴十八最有价值的杀招,便是连击战法。

若是能连击到十八次,巴十八能轻易斩杀同级的蛊仙。上一世,他和清夜大战,这位天庭成员就非常重视,极力破坏,避免巴十八连击太多次。

说心底话,方源是有些小小的失望。

他原本以为,巴十八既然擅长连击战法,他手中的八转律道仙蛊很可能就是连蛊。没想到是加蛊。

连蛊比加蛊要有名气得多,似乎也更加实用。

有了八转连蛊,方源就能借助它的威能,将种种不同的杀招相连,形成连招的效应。

连招是杀招运用的技巧之一,两种或以上的杀招接连使用,能迸发出更强的威能。往往只有蛊仙中的精英才能掌握。

当初的焚天魔女就掌握了这种技巧。

她将愤怒的小鸟以及火囚杀招,组成连招,威能非凡。

能够设想出连招的蛊仙,至少要有大宗师级的境界。

可以看出,焚天魔女的这份连招搭配,都是炎道流派的杀招。而运用连蛊,能够将不同流派的杀招组合成连招!这一点的区别,意义十分重大。

比方说方源的万剑鬼蛟和大盗鬼手,亦或者翠流珠和力道大手印……组合的种类很多,能将战力提升到意想不到的层次。

巴十八的仙蛊、杀招、魂魄都被方源收取,只留下他的肉身和仙窍。

尽管方源可以吞窍,境界足够。

但他还是暂且将吞窍的时间往后推。

因为一点——吞窍之后,下一次灾劫降临的时间将清零,然后重新计算。

这个细节虽小,但相当重要。

吞窍暂告段落,但白凝冰等人,却陷入了忙碌当中。

这段时间,方源巧取豪夺,吞并了相当数量的仙窍,说出来要吓死人。

新增添的福地和洞天,实在有点多。虽然至尊仙窍仍旧稳定如初,但仙窍中的生态相互影响后,有了不小的混乱。

白凝冰等人就在处理这些混乱,之前那是粗略镇压混乱,而今是在精细地调整。什么地方的山峦应该搬迁,什么河流应当保留,什么样的兽群需要迁徙,一些植株需要重点栽培……

这些繁杂无比的事项,都被方源推算得巨细无遗,有一个十分完美的大局布置。

这就是智道手段深厚的好处。

新的布局不仅能够让这些资源相互和谐共处,而且能相互促进,最后还考虑到了未来的巴十八仙窍、兽劫洞天,提前留下许多空余之地。

换做其他流派的蛊仙,想得头昏脑涨、心力憔悴,都不可能安排得这么好,会内耗掉大量的资源。造成的损失,往往让蛊仙都为之心疼不已。

方源更忙!

他的宙道分身连同本体,一直都在苦修,争分夺秒却仍旧感觉时间不足、精力不足。

方源恨不得将自己掰开几份,一同努力。

方源的律道境界提升了。

境界是蛊修的根本,境界提升带动了方方面面,使得其他种种都因此改变,焕然一新。

尤其律道最是泛用,方源新得了这么多的律道仙蛊,若不构思出杀招,或者将他们融入到原来的杀招中去,方源都对不起自己的努力和筹谋!

仙窍经营,虽然很大部分可以让下属帮忙,但有一些地方还得方源亲自动手。

杀招的锻炼,也是一个重点。

巴十八的连击之术还可暂时放在一边,夏槎的种种宙道杀招前世已经熟稔,但陶铸真传的两大手段,方源是非炼不可的,而且还是重中之重。

巨阳真传、盗天真传、长毛真传都是博大精深,方源每每参悟一段时间后,都会有所收获。

比方说巨阳真传。

南疆追辑队伍中新增了巴十八,方源是推算不出来的。因为南疆正道也不缺乏智道蛊仙,他们联手将这个秘密遮掩。

引起方源警惕的,是他发觉自身的运势被莫名压迫。

而后在埋伏战中,他细心观察,发现了巴十八这大威胁。

方源的运道实力增长了许多,这点换做之前,他绝对察觉不到什么。

除去北原那波蛊仙,方源拿运道的手段来对付其他人,都比较可靠,甚至能获得意想不到的良效。

因为运道真传只有三份,被巨阳仙尊布置得很好,一直都没有广泛流传出去。

即便是天庭,也苦苦搜寻,却没有什么收获。

南疆正道更是如此了。

再比如说盗天真传。

没有最近的参悟收获,方源就不可能改良出新的大盗鬼手。

还有长毛真传。虽然方源有炼道准无上境界,但他参悟长毛真传也仍有收获。别的不说,蛊如故的设想得到了很多进展,正是因为方源连续几次参悟了长毛真传。

说心底话,方源这一次有一种吃撑的错觉,他感到自己很是“消化不良”。

他的至尊仙窍很强,可以容纳仙窍,仍旧很稳定。

但是他需要参悟的东西太多,需要推算改良的杀招太多,手中的真传也太多了。

这些真传,常人得到一个,就足以改变未来,一生受益,乃至影响整个天下的势力格局。

现在方源的手中,这样的真传一巴掌都数不过来!什么红莲真传、幽魂真传、巨阳真传、元莲真传,盗天真传还是两份,长毛真传绝不输给这些尊者真传,陶铸真传也很强大……

方源的仙蛊一大把,八转仙蛊多得让世间几乎所有的八转蛊仙都要忏愧。

九转的仙蛊都有!

还有他的下属。

和上一世不同,这一世他的下属非常多,并且很多都是人杰,比起一个正常的超级势力,还要出色许多。

到了这一步,方源的积累,他的底蕴,已经可说是惊世骇俗!

只需要一段悠长的日子,来消化完毕,方源的实力将傲居八转之巅。

但他偏偏缺少时间!

方源心底苦笑。

上一世他就感觉时间不够用,这一世情况变本加厉了,这份感觉更厉害了。

方源估算着,即便他有智慧光晕在手,依照他现在的情况,至少也得要百十来年的时间,才能真正消化彻底,将这些真传融汇贯通,并且自己钻研下去,推陈出新,自我领悟,延伸出更高的成果来。

“一方面,我可以调整仙窍的流速,尽量争取到更多的时间。”

“另一方面,我要积极探索梦境,提升境界。我参悟长毛真传,就远比领悟盗天、巨阳、元莲真传要容易得多。”

“最后,我需要帮手,更多的分身!”

在方源的殷切期盼下,数日后,荡魂山修复完成。

这比上一世同期,可要提前太多了!

------------

\end{this_body}

