\newsection{血漂流}    %第四十节:血漂流

\begin{this_body}

%1
北原。

%2
至尊仙窍内,第一重天。

%3
方源缓缓地睁开双眼,精芒一闪即逝。

%4
“耗费数日推算完毕,还是采用原先的名目,仍旧叫做血漂流罢。”这般想着,方源身边的蛊虫,就跟随他的心念调动起来,四下飞散。

%5
数百只凡蛊,在他身边环绕飞舞,宛若硕大圆环。

%6
血本仙蛊就寄生在他的胸口,赤红的印记开始绽射光芒,同时发挥澎湃的热量。

%7
忽然间,一股耀眼夺目的血光从方源胸口绽放。

%8
血光冲天,眨眼间就化作一股澎湃的血泉。

%9
血泉载着方源,冲突而起。

%10
激射到空中后,哗啦一声,在空中拉开一道长长的猩红轨迹。

%11
一道血光长河,在一重天中喷射飞行。

%12
所到之处,逸散浓郁的血腥之气,令人闻之作呕。

%13
血光长河中,方源全身都被包裹在里面,不见踪影。这点很好,有伪装的作用,敌人就算攻击,也不知道该具体攻击血光长河哪一部位。

%14
飞行一阵,方源便停息下来。

%15
血光长河骤然消失,把方源孤零零的身影显现在空中。

%16
“这血光长河的速度,还可以再提高一些,只要增加更多的凡蛊。”方源仔细体悟,总结得失。

%17
“但就算凡蛊再多,直线速度还是不如剑遁仙蛊的。”

%18
剑遁仙蛊是七转仙蛊,这份血漂流仙道杀招,采用的唯一仙蛊,也就是六转的血本仙蛊,速度比不上剑遁,很正常。

%19
除非方源将血本仙蛊的级数。提升到七转,这个血漂流方能在直线速度上,超越剑遁。

%20
不过这样一来。方源只是六转蛊仙,运用七转仙蛊。又会遇到之前的尴尬情景,得不偿失。

%21
“直线速度上不如剑遁,早在我预料之中,没有关系!我推算这份仙道杀招,完全是想在转折弯曲方面,弥补剑遁的短板。我可以将血漂流和剑遁仙蛊,交替运用。如此一来,也够勉强应付楚度了。”方源早已考虑清楚。

%22
血漂流是血道杀招。剑遁仙蛊却隶属剑道,这两者之间一点联系都没有,换做其他蛊仙兼顾两者,必定两头都不讨好。

%23
但方源却是个特例!

%24
他的这具肉身,极为特殊,乃是至尊仙胎蛊所化。一身道痕,五花八门,包含诸多流派,更妙的是互不干扰。

%25
方源完全可以将血漂流、剑遁仙蛊交替运用,一点都不损失这两者原有威能分毫。

%26
剑遁仙蛊令人直线飞行。虽然速度极快,但只能直线飞行,实在太傻了。极其容易被针对。

%27
若是停下来调整方向,这个瞬间的破绽对于霸仙而言,完全可以把握住。

%28
但剑遁、血漂流交替运用,互补短板,差不多能从容面对霸仙了。

%29
楚度是力道蛊仙,力道这个流派,擅长攻击和恢复,作战持久力最强,但速度方面通常欠佳。

%30
当然。这一切都只是方源的估算。

%31
是否有效,还得实践来检验。

%32
毕竟。霸仙楚度上一次的大肆出手,还要追溯到百年前。

%33
自从他和刘家太上大长老一战后。就鲜有行踪,一心潜修。

%34
方源这些天不仅推算血道杀招,还在千方百计地收集楚度的情报。

%35
但大多数的情报,都是一百年前左右的,早就过时了。

%36
这时候,方源就怀念起黎山仙子来了。黑楼兰的这位小姨妈,掌握山盟仙蛊,人脉极其广泛,与魔道、正道、散修都有极多往来。若是她还健在,必定能采集到更多更新的有关楚度的情报。

%37
“琅琊派虽然势大,但也隐居多年,最近才有积极入世的思想。情报方面,完全比不上黎山仙子。信道……这是我的一个短板,我今后得弥补起来。”

%38
方源思绪稍微发散了一下。

%39
他想起义天山大战。

%40
啊,那就是个坑!

%41
方源现在回想起来,还有点心惊肉跳。

%42
他能存活下来,完全是两个巨人打架,最后让他这个小蚂蚁捡了大便宜。险死还生,有不少运气的成分。

%43
说起来,还是信道造诣不深,情报不足!若是方源早知道有这样的内幕,他必定会改变计划的。

%44
“不过就算是我有了黎山仙子的信道手段,恐怕也打探不出关键情报来吧。义天山的水,实在太深了。五百年前世,完全没有搞这么大!”

%45
方源将逸散开来的思绪,都收敛起来。

%46
不得不说,他现在重获新生,思考问题时灵光闪现,游刃有余。完全不是之前的仙僵之躯,能够比拟的。

%47
再加上他智道宗师境界,许多智道手段辅助,比五百年前世时还要思维缜密,灵活聪颖。

%48
这个方面,他已经完全超越了自己的历史高度!

%49
血漂流虽然推算出来了,但还要演练,并且在演练的过程中,不断修改。

%50
仙道杀招开创出来,并不意味着就放着不管了。还需要蛊仙不断地调整、精修,甚至大改。

%51
想当年,东方长凡设想出的万星飞萤,就精修了无数次,大改了三四次。就算这样,他还私底下觉得不满意。

%52
有一些仙道杀招,不仅是设计者终其一生都在修改完善,甚至他(她)的历代传承者,都参与休整改进。

%53
例如无双偃月斩,这个仙道杀招就被天妒楼的历代蛊仙,改进过许多次。也正因为如此,这个仙道杀招,才成为天妒楼的招牌,在中洲赫赫有名,震慑四面八方。

%54
方源的仙级血道杀招,都还是草创,距离无双偃月斩这种层次,还差的很远。

%55
接下来的几天里,方源就专门练习血漂流和剑遁仙蛊。

%56
不断的实践,加深他的经验,令他对血漂流和剑遁的运用,有更多更深的体会。

%57
演练中,他也不断的进行调整,最终,他并没有再继续为血漂流增添凡蛊,反而在原来的基础上削减了一部分。

%58
“我需要血漂流配合剑遁,是需要它为我蜿蜒曲折。剑遁停息下来后,一瞬间,我就要催动血漂流成功。这个时间有点短,所以凡蛊数量多就显得麻烦。不仅浪费宝贵的时间,增添了催发难度,而且增添了那一点速度,也收效甚微。”

%59
“差不多可以了,现在应该尝试一下收起仙窍,逃脱这里。看看能不能摆脱楚度。”

%60
虽然方源还可以继续推算出更多的血道杀招,但已经没有这个必要。

%61
自从他得到血本仙蛊之后,他就以这只仙蛊为核心,凭借五百年前世的凡道杀招积累,顺利推算出了舍命血印、血愈湖、血漂流,这三个血道杀招。

%62
做到这个程度,已经足够了。

%63
毕竟血本仙蛊只有一个,用了这个血道杀招,那个血道杀招就不能同时运用。

%64
其他方面,方源完全可以用其他仙蛊来替代。

%65
自从义天山大战之后,他丢失了几乎全部的蛊虫积累,现在的这些蛊虫涉及许多流派,杂乱不堪,不成体系。但目前没有办法,只能将就着先用了。

%66
“但如果再给我一次机会,我也仍旧会,也必然会选择至尊仙胎蛊!”对于这点,方源毫无犹豫。

%67
虽然现在,方源镇守至尊仙窍,易守难攻,但也要防备楚度请他人援手,或者在冰原上布置仙级战场杀招什么的。

%68
总之,停留在这里越久,越是不利。

%69
主意已定,方源就没有再犹疑不决。

%70
心念调动,整个仙窍世界都开始了嗡嗡震动。

%71
震动的幅度越来越大,持续了片刻之后,仙窍世界骤然收缩。

%72
它收缩的速度极其快速,简直是电光火石!一刹那间,方源再睁开双眼时,已经来到外界,置身在北部大冰原中。

%73
无风无雪。

%74
苍白的阳光,映照着这片平缓广袤的冰雪大陆。

%75
仍旧很寒冷。

%76
方源提起十二分精神,目蕴神光,轻轻一扫,便见到楚度。

%77
霸仙楚度!

%78
和方源料想的一样,他一直留着这里,并未离开。

%79
但有点出乎方源意料的,是楚度的态度。

%80
“这位仙友,且慢,且慢!”楚度并未出手,而是面带微笑,看着方源,声音也很柔和,一点厮杀之意都没有。

%81
方源冷哼一声:“阁下这是先兵后礼?”

%82
“惭愧,惭愧。”楚度居然坦言承认自己的过错,以道歉的口吻道,“之前的事,是在下莽撞了。只因见到仙友渡劫景象,实在对在下今后的修行影响太大!所以一时间,心驰神摇,鲁莽出手。事后细想起来,后悔不迭。”

%83
方源一边紧紧盯着楚度,严加防范。另一边则一直在用诸多手段,侦查周遭环境。

%84
见方源沉默,楚度主动提出:“仙友,我无意与你为敌,只在意你引动狂蛮真意的窍门。我希望和你进行交易。”

%85
“交易?”方源眼中精芒一闪,以斟酌的语气道,“也不是不可以,只要你出的价码能打动我心。”

%86
楚度脸上喜色一闪。

%87
但紧接着方源又道:“但在交易之前,你故意出手,干扰我渡劫的事情先得好好算一算!”

%88
“我的赔偿,必定会让仙友满意!”楚度立即接道,态度非常诚恳。

%89
就连方源私下都有点小惊奇这还是霸仙吗?

%90
不过仔细一想,方源对楚度更加警惕。

%91
此人能屈能伸,是大丈夫!谁若因此轻视他,那就是十足的蠢货。

%92
“你想要赔偿的话,先将我的飞剑仙蛊还来,以示诚意罢。”方源旋即道。

\end{this_body}

