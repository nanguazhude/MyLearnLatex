\newsection{参悟盗天真传}    %第八百四十二节:参悟盗天真传

\begin{this_body}

砰!

房家太上大长老房功手掌猛地一拍桌面,腾的一下从座椅上站起身来,眼中闪烁着愤怒的焰光:“这个算不尽,真是贪婪!居然敢索要我族的偷道真传,他何德何能?不过是一介散修,七转的修为而已,居然勒索到我房家的头上!该死!该杀!”

听到房睇长的汇报,房功愤怒不已。

房睇长却已是冷静下来:“算不尽的确是有两把刷子。我房家招揽算不尽,也是应时局而做下的决定。算不尽身为智道蛊仙,自然精明至极,此刻索要偷道真传,我族还真不太好拒绝。”

“怎么?你真想将偷道真传给他?那可是盗天尊者的真传!”房功双眼瞪圆。

房睇长苦笑:“不给他,他就不干活,我们能拿他如何呢?是,他现在在我族的大本营中,或许我们联手就能干掉他。但干掉他之后呢?没有他的辅助,我要炼化豆神宫,十分困难,时间太长。房家的局势却是压力重重,前不久已经是有八转蛊仙动手了!”

这是方源派遣了两大龙将作的案,但手脚干净,房睇长又因豆神宫牵扯了绝大多数的精力,因此没有勘破。

房家上下都认为:这不是西漠的那些超级家族,就是个别的居心叵测的西漠魔道蛊仙,亦或者是天庭出手。

房功沉吟不语,到底是太上大长老,也渐渐冷静下来。

房家如今的现状,的确是需要算不尽出力。一旦将豆神宫掌握,房家就能翻身,改变整个危局。若是将来,再将落英馆、问津坞等三大仙蛊屋修复好,那房家的综合实力恐怕要凌驾于绝大多数的西漠正道超级势力了。

算不尽是房家改变危难,登上顶峰的钥匙!

“不愧是智道的蛊仙,卡在这个关口,呵呵呵,这个算不尽有胆量,也有好本事呢。”房功冷笑连连。

房睇长了解房功,听他这番话音,便知道房功答应了。

当然,等到房家渡过这场危机,回过头来房功必定会出手整治算不尽,好好出了这口恶气!

房睇长笑道:“其实算不尽还是有分寸的。他没有索要我的智道传承,也没有贪婪到觊觎我族仙蛊屋的秘密。若是他提出这两个要求,我们就真的为难了。”

房家的立身之本,并非是偷道。

从来都不是。

偷道只是房家的一次意外收获,从一位天外之魔的身上谋算得来的。

这么多年下来,房家对这道盗天真传的利用,已经达到了极致。

之前向董陆沉反击,出动了偷道的仙蛊屋,却是残破不堪。单单这一点,就已经说明了许多问题。

房睇长笑道:“这算不尽在智道造诣上,的确是深厚。不说这些天来,我在他的帮助下,进展颇多。单单看他这个要求,火候把握得极其精准、老道,既没有过了分寸,也在最大程度上获取利益。这一点要做到,可并不容易。”

房功也笑起来:“呵呵呵,你倒是很看好此人,居然还对他赞许有加。”

房睇长双眼微微发亮:“人才就是人才,我岂会因为自身缘故,而去刻意贬低?这种人才若真的为我族所用,那就是一件极好的事情了。”

房功眯起双眼,沉吟道:“要让算不尽这胆大包天的东西归心,真的忠心我族,可不是一件简单的事情。”

经过一番讨论,房家同意将盗天真传交给方源浏览。

方源很快就得到了真传的内容。

果然是盗天真传,最精华的便是偷道杀招。

当初盗天魔尊能够开创偷道,便是从这个杀招上发源的。

显而易见,这个杀招意义重大,乃是偷道流派的源头!

方源看了,简直是叹为观止。

在这道真传中,偷道尽显磅礴大气。皆因它偷的不再是单纯的人或物,而是整个天地自然。但凡天下的道痕,都能偷取到手!

除此之外,还有一项内容,让方源看了又看,流连忘返。

这就是盗天真传中的那座偷道仙蛊屋——贼巢。

“方源的偷道境界可是大宗师,房家蛊仙不会有这样的造诣。所以房家的优势在于搭建仙蛊屋,但始终那这座贼巢没有办法,任由它破败残缺。”

方源只是看了几眼,就从贼巢的相关内容中,看出了房家历代的努力——都曾经尝试过修补这座仙蛊屋。

“虽然也有不少奇思妙想,但是和偷道不相符合,顶多是个画蛇添足的程度。”

“但若让我来修补,那就不同了!”

方源心头暗动。

但他索要盗天真传的内容,就已经是极限了。方源知道:房家差点就要翻脸,硬生生忍耐了下来。就算房家给真传给的相当干脆,但这笔账他们定是记在心中的。

若是自己再进一步索取贼巢,那就太过了,大大超出了房家的底细,他们绝不会同意!

“再者,我身为智道蛊仙,对偷道太感兴趣,也是一个巨大的疑点。”

“房睇长之所以没有怀疑,还是因为我铺垫得好。之前盗取大盗鬼手,而后又动用智取杀招,斩杀了万良翰,这些都给房家提前建立了心里的印象,因此才自然而然。”

“只是这份真传,是盗天真传的全部内容吗?”方源无法分辨,至少他从这份内容中没有看到什么意犹未尽的东西。但这说不准,因为房家可是拥有智道大能房睇长的。

“若是此刻我出手,直接抢夺会怎样?”方源又想。

单凭自身战力,要对付房家,并不困难。房功虽强,乃是当今相当罕见的力道八转蛊仙,但仍旧比不上龙公。他和方源之间的战力差距,还是有一截的。

然而方源一锤定音的可能并不大,房功曾经力压陈衣,实力乃是八转一流。同时这里又是房家的大本营,布置多多,方源还不了解。

最大的可能是爆发大战,令世人知晓。到那个时候,不仅是方源的身份曝光,西漠正道超级势力也会找到最佳的理由——帮助房家围剿魔头,蜂拥而至,对付房家。并且天庭绝不会闲着,有这样的良机,绝对会派遣蛊仙伺机出手。

方源思考良久,还是决定帮助房睇长,将眼下的局势维持下去。

这样才是最符合他利益的方案。

方源阅览了盗天真传,再没有喊苦喊累,而是“通力合作”,配合房睇长强炼豆神宫。至少房家看起来,他是尽了全力的。

进展仍旧艰难,但一次次积累下来,终于量变引发质变——豆神宫宫门打开,蛊仙可以自由出入了。

这可是巨大的进展!

房家上下无不激动兴奋,房睇长也展颜欢笑。他们看到了房家的崛起!

然而接下来的事实,却是像一盆冷水浇在房家蛊仙的心头。

不管房睇长如何努力强炼豆神宫,都毫无一丝进展。这和之前形成了鲜明的对比。

“看来炼到这种程度,豆神宫内部的结构并不是我之前推算的那样,而是发生了翻天覆地的变化!”

“我们只有改变整座智道大阵,才能有新的进步。”房睇长对方源道。

方源点头:“但是如何改变这座智道大阵?在此之前,我们还得刺探豆神宫,了解更多。唉,若是有一位木道大能来帮助我们,那就容易多了。现在靠我们两个智道,只能是连蒙带猜!”

房睇长苦笑:“眼下除了连蒙带猜,还有什么好办法吗?”

强炼豆神宫一事,由此陷入了瓶颈。

豆神宫不愧是元莲仙尊亲创之物,房睇长、方源两人越是研究,越感到它的雄伟和精妙。

房睇长殚精竭虑,搜索枯肠,也没有找到好办法。

方源倒是有不少的想法,但是统统不提。

这段时间以来,他都没有机会施展因果神树杀招。若是施展出了这招,那必定能打开局面!

“呆坐在这里苦想,不符合我的修行之道。我还是先出去逛逛,散一散心,说不定就有许多的灵感了。”方源以退为进,主动告辞。

房睇长、房功苦留不住,又惦念着将来还需要算不尽出力,只好任由方源出走。

方源离开了房家,表面上是进入了青鬼沙漠,其实暗地里则开始探索梦境。

这段时间,他和池曲由的交易一直都持续着,又利用第二空窍凡蛊,继续敲诈勒索了南疆正道,专门筹集木道的梦境。

没有木道大能帮助?

没有关系!

我自己成为木道大能,不就好了吗?

方源破局的方式堪称简单粗暴。

这些木道梦境难不住他,论探索梦境的经验,方源早已经是世间第一人。龙公之流,连给他提鞋的资格都没有!

方源的木道境界一直都是普普通通,五百年前世也没有深入的接触过。

但不久之后,他就成为了木道大师。

外界又过了数天,他成为了木道宗师!

手中的梦境再次消耗一空,方源信心百倍,回头再来钻研因果神树和豆神宫。

顿时,就有无数的方案,浮现心头,并且大多都十分可行!

“似乎我应该要改善因果神树杀招。”

“但要改善这个杀招,就要炼制合适的木道仙蛊!现在的这些木道仙蛊并不适合。”

方源有些犹豫。

如果要炼制木道仙蛊,当然是可行的。毕竟他手中掌握着琅琊派!

但如此一来,运道仙蛊就不能兼顾了。

东海二仙搜魂了方正,又将情报告知了气海老祖,因此方源知道天庭谋算,也知道了秦鼎菱。

从毛民蛊仙炼制运道仙蛊的情况来看,他也算出天庭也在抢炼运道仙蛊。虽然绝大多数都没有抢过方源,但也有少部分被天庭得手。

如今掉转枪头,大规模地炼制木道仙蛊,就意味着方源放弃运道仙蛊的抢炼。

但若坚持抢炼运道仙蛊,方源只依靠当下的因果神树杀招,想要收服豆神宫,难度相当的大。

方源心中隐隐有一种预感,眼下的时机难能可贵,若是浪费了这次良机,再想要谋算豆神宫,希望就极为渺茫了。

------------

\end{this_body}

