\newsection{各方进展}    %第八百五十三节:各方进展

\begin{this_body}



%1
眼前是一片贫瘠的灰白土地。

%2
土地干涸龟裂,形成道道沟壑,毫无肥力可言,寸草不生。

%3
房睇长望着这一片广袤的土壤,挥手洒下一蓬黄玉豆籽。

%4
豆神宫中封存了不少豆籽,主要有黄玉豆籽、碧玉豆籽、红玉豆籽、黑玉豆籽。除此之外,还有比较特别的蓝彩豌豆籽,白霜蚕豆籽。

%5
这些豆籽种植下去,悉心照料,等待一段时间之后,就能长成各种类别的豆神兵了。

%6
所以,豆神宫的招牌手段豆神兵卒,并非是一蹴而就的杀招,而是需要经过较长时间的积蓄,方能成就规模,有所威效。

%7
也正是因为如此,当初陈衣镇压青仇,利用因果神树杀招来影响豆神宫和房家对决时,就无法运用出这一招来。陈衣当时只能用魂兽来暂且替代。

%8
房睇长现在栽种的,就是黄玉豆籽。

%9
这是所有豆籽中最基础的品种,栽种出来后,能长成黄豆兵卒。

%10
此刻,黄玉豆籽已经洒下。

%11
房睇长便催动豆神宫,顿时就有大风吹拂起来,无形的风在空中凝聚成一只只的淡绿锄头。

%12
锄头对准地面,开始锄地。

%13
片刻后,房睇长又是一挥手,天空中顿时下起了雨。

%14
雨量上有所讲究,既不能太大,也不能太小,主要湿润这片干旱的土壤。

%15
土地湿润了,黄玉豆籽开始发生某种玄妙的改变。

%16
但这种改变的速度相当缓慢,方源分身便又催动另外的杀招。

%17
这次的杀招下去,就看到这片土地闪烁着翠绿的波光,一阵又一阵,像是潮汐,如同海浪,不断地冲刷着土地中的黄玉豆籽。

%18
黄玉豆籽汲取到这些翠绿的波光,生长速度立即暴涨,以肉眼可见的速度,发芽破土,然后茁壮成长。

%19
“按照这种速度,一天一夜之后,就有黄豆兵卒诞生了。”方源分身欣慰地点点头。

%20
他心念一动,忽然抽离而出,又回到了豆神宫的大殿中。

%21
他转头回望一眼,便见豆神宫的这片墙壁上的壁画,发生了改变。

%22
原本壁画中是一片干涸的土地,但此刻当中却有一小片土地上,诞生了密密麻麻的豆苗。

%23
“这就是画道的手段了。豆神宫中就包含了画道仙蛊,里面的土地并非虚构,这手段真是奇妙。”

%24
“在这壁画中栽种豆苗,长出豆神兵卒,不断栽培。长成之后,这些豆神兵卒就会脱离壁画,来到大殿。”

%25
“土地极其贫瘠,但并不要紧。种豆需要的地力并不多,甚至将来种豆子次数多了,还能将壁画中的土地肥力提升上去,甚至转变成仙材。”

%26
“当中消耗的,一是水道仙材,二是木道仙材。”

%27
前者是用来湿润土壤,后者是加速黄玉豆籽的生长速度。

%28
仙材都存放在豆神宫中,方源之前催动的那两记杀招,就是将提前存放好的仙材消耗掉,形成杀招。

%29
“这效果的确是好,但代价也是不菲。好在无须我来操心,自有房家来掏腰包。”房睇长笑了笑。

%30
豆神宫已成了方源之物,房家一直被蒙在鼓里,替方源出资,帮助他栽培豆神兵卒。

%31
东海深处。

%32
古族大本营中。

%33
“可以了。”方源分身吴帅缓缓地停住杀招,他已是满头的汗渍。

%34
在他面前,又有一批新龙人诞生了。

%35
不过,这批新龙人之前的身份,就不全是兽人了,还有一批鲛人。

%36
“多谢吴帅大人出手相助,这是我族的谢礼,还请前辈收下。”一位鲛人蛊仙摆动鱼尾,缓缓漂到吴帅的面前,双手奉上一份仙材。

%37
这份仙材乃是一汪清水,散发着刺鼻的气味,被鲛人蛊仙的杀招拘束成一团。虽然鲛人双手晃动着,但这汪清水却是一动不动,仿佛结冰。

%38
“原来是不动腐水,恰好让我用来栽培军团蚁,我收下了。你们有心了。”吴帅毫不客气,当即收下。

%39
他要栽培箭蚁,正需要不动腐水。

%40
说起来,他的军团蚁自从栽培以来,就以一种骇人的速度不断发展壮大。

%41
当中除了方源自身的资源大笔大笔地投入进来,还有古族、鲛人的大力资助。

%42
当然,方源分身也并不是从他们两族身上敲诈勒索,而是交易所得。

%43
吴帅动用杀招,将兽人、鲛人转变成新龙人的酬劳。

%44
“每一天,我的实力都在迅猛增长,简直是日新月异。”

%45
“和古族的交涉效果极好,现在就连鲛人一族也搭上了关系。”

%46
“再配合气海老祖那边的影响力,我对东海蛊仙界的掌控是越发深厚了。”

%47
东海的鲛人王庭乃是超级势力,实力雄厚。

%48
这是全天下独一份的异人超级势力,当初是得到了乐土仙尊的帮助,才发展起来,并且屹立至今。

%49
即便如此,鲛人们也饱受着四面八方的巨大压力。之前,他们在东海人族势力的包围下,已经是战战兢兢。如今五域乱世将至,鲛人一族中的有识之士开始四处寻找族群的出路。

%50
五域一旦发生乱战,往日脆弱的平衡必定会被打破。

%51
鲛人一族前景相当堪忧,作为异人,必定成为人族势力首要打击的对象!

%52
一部分鲛人就看中龙人的发展前景,不惜转变族群身份。

%53
这是因为鲛人王庭中的高层,一直暗中资助古族,帮助古族遮掩形迹,也从古族身上得知了当年龙人的隐秘。

%54
在知道宿命蛊中,有着龙人当兴的启示后,一部分的鲛人高层就打起了这个主意。

%55
这才有了鲛人也转变龙人的事实。

%56
“鲛人王庭为了族群存续,已经开始分散投资了。”

%57
“人族势力也都预见到了将来的乱世,他们尤其是对中洲天庭的担忧和忌惮,正是我可以利用的地方。”

%58
“或许我能说动鲛人出兵,在天庭修复宿命蛊的时候,让一些鲛人蛊仙参战。”

%59
吴帅转动着诡秘心思。

%60
兽灾洞天。

%61
情势非常危急,已经到了刻不容缓的境地。

%62
“小渡,你真的要这么做吗?”山崖城主担忧地望着方源分身战部渡。

%63
战部渡仰着头,双眼炯炯发亮,语气坚定毫无一丝动摇:“是的,师父!你已经重伤,现在这个情形,我们都打不过那头怪兽。但若任由它四处搞破坏,山崖城就会彻底毁了!不知道有多少的人要死掉!我只有冒险突破,只要我成为战兽勇士,就能充分地发挥出箭尾雕的全部力量,从而干掉这头可恶的怪兽。”

%64
山崖城主注视战部渡良久,知道自己的这个爱徒已经下定了决心,自己再劝说什么也不会有任何效果,便挥手道:“唉,你去吧。”

%65
战部渡没有犹豫,立即转身离开这里,开始渡劫升仙。

%66
山崖城主等人紧紧地注视着他,无数平民跪在地上默默祈祷,为战部渡加油鼓劲。更有许多战兽使,拼死纠缠着怪兽。

%67
令人惊喜的是,那头巨大的怪兽许是累了,攻势放缓下来。

%68
这其实是战部渡暗自的操纵,眼下的局势也是之前就订好的计划。

%69
“我有本体的一部分记忆和底蕴,渡劫升仙十拿九稳。不过我还要做出凶险的样子,不仅是为了瞒过山崖城的所有人,还有这片洞天的天灵啊!”

%70
整个过程自然是波折连连,把山崖城主等人看得心惊肉跳。但结果有惊无险,战部渡终于晋升成仙,成为战兽勇士!

%71
接下来,战部渡大发神威,击倒了怪兽。

%72
“耶!小渡威武!”

%73
“小渡,你拯救了我们全城人的性命,我们真是不知道说什么好。”

%74
“不愧是小救星小渡啊。”

%75
“小渡,你恐怕是历史上最年轻的战兽勇士了。”

%76
人们将战部渡包围起来,又将他高高抛起,齐齐接住。

%77
战部渡成了全城人心中的大英雄,风头无两。

%78
人群外,伤重虚弱的山崖城主欣慰地看着这一切,满脸微笑。

\end{this_body}

