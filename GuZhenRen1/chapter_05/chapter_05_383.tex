\newsection{自爱仙蛊}    %第三百八十三节:自爱仙蛊

\begin{this_body}

%1
北原,琅琊福地。

%2
甲字号炼蛊大厅之中,持续了数月之久的炼蛊,已经到了最关键的时刻。

%3
琅琊地灵亲自站在一座炼道超级蛊阵的边缘,双目紧紧地注视着蛊阵中燃烧着的熊熊烈火。

%4
这火焰呈现冰蓝之色,一点也不炙热,反而非常寒冷。

%5
即便是超级蛊阵,也难以将这剧烈燃烧的冰焰之威,彻底隔绝起来。

%6
琅琊地灵站在蛊阵边上,整个眉宇、毛发都染上了一层薄薄的幽蓝冰霜。

%7
但他不管不顾,全神贯注地注视着火焰。

%8
更准确的说,是火焰中央的那团玄冰。

%9
玄冰在不断地熔解,又在不断地生成。最初的玄冰,棱角分明,如今它在琅琊地灵亲手炼制之下,已经被渐渐地磨平了棱角,有了一丝浑圆之感。

%10
“扇门风五缕。”琅琊地灵忽然开口道。

%11
为他打下手的,正是毛民蛊仙毛六。他闻言连忙动手,从仙材库存中,运用特殊的手段,提取出五缕扇门风。

%12
这扇门风,乃是七转仙材。

%13
非常奇特。

%14
生长的地方,不在深山老林,更非大泽苍穹,而是在凡人家的门板上。

%15
每当有这样的仙材产生,凡人的门板就闭合不了,只能不断地开合。

%16
这种奇怪的现象,在很久之前,就引起了蛊师们的注意,继而引发了蛊仙出手收取。

%17
只是刚开始时,扇门风这等仙材,蛊仙们都利用不了,不了解它。如今却是早已有了纯熟的手法,以及充分利用的方案。

%18
五缕扇门风被琅琊地灵小心翼翼地,送去冰焰中心。

%19
很快,五缕清风就围绕着玄冰,不断旋转,像是雕琢玉石一样的五只巧手,加剧了玄冰的变化。

%20
“不好。”好景不长,琅琊地灵忽然变了脸色,惊呼出声。

%21
毛六心中顿时一突,连忙望去。

%22
只见那冰焰陡然熄灭,那块玄冰直接破碎,连同那五缕扇门风,一切都化为乌有。

%23
“啊啊啊!又失败了呀!!!”琅琊地灵跺脚,大吼,十分气急败坏的样子。

%24
毛六深深叹息。

%25
太可惜了。

%26
真的已经到了最后的几个步骤。

%27
但失败了。

%28
之前数月的努力心血,都直接打了水漂,做了无用功。

%29
“若是利用毛民天地流,绝不会如此结果!”琅琊地灵冷哼。

%30
毛六连忙提醒道:“可是太上大长老,按照门规,这既然是方源长老耗费大量贡献发布的任务,我们也得按照他的要求来啊。”

%31
“唉!这方源脑子糊掉了,怎么一心想着用人族的炼蛊法门。他难道是存心要让我难堪吗?!”炼蛊失败,让琅琊地灵的心情很是糟糕。

%32
毛六连忙为方源说好话:“方源长老一直都神神秘秘,我虽然和他不对付,平常也看不惯他。不过呢,这一次他似乎动用了全部资本,不惜耗费如此巨量的门派贡献,要来炼制仙蛊。他要炼制的仙蛊,多达三只。如果是存心想让太上大长老您难堪,不至于耗费这么大的代价吧?”

%33
“嗨!我也只是随口一说罢了。”琅琊地灵垂头丧气,摆摆手,“先休息,休息一下。这已经是第五次失败了。而且,方源的门派贡献也不太够,你联系一下他吧,告诉他炼蛊的结果,若是他还想再炼,就必须有足够的门派贡献!”

%34
“是。”毛六应声。

%35
方源如一道利箭,飞翔在高空中。

%36
沙漠的广阔壮美,他无心欣赏,此时此刻,他的心思还在琢磨着刚刚和凤九歌的交手。

%37
那处沙流漩涡中的光阴支流,是影宗暗地里的修行资源,记录在影宗地图之上。

%38
“紫山真君临终之前,特意在遗嘱中交代我,一定要尽快地去往光阴长河当中,和鬼脸红莲中的幽魂意志接洽。”

%39
虽然方源继承了紫山真君的遗藏,掌握了大量的杀招、仙蛊方、秘闻等等。

%40
但是影宗的整个财富,他还只是继承了大半而已。

%41
而光阴长河中的鬼脸红莲当中,存在着的幽魂意志,则掌握着影宗的几乎全部修行内容。就连幽魂魔尊的真传,它都拥有着一部分。

%42
影宗存在十万年,幽魂魔尊生前更是纵横天下,屠戮苍生,令万物齐喑。整个影宗虽然不再,但是掌握的修行内容,却是难以想象的丰富和浩瀚。

%43
就比如仙蛊屋吧。

%44
方源目前,只掌握了十二座仙蛊屋的完整全面的建设内容。但实际上,魔尊幽魂和影宗掌握着的,绝不止十二这个数目。

%45
这些其余的修行内容,都得需要方源前往鬼脸红莲当中,亲自接收。

%46
“而且红莲真传,也在光阴长河当中。我有着春秋蝉,就是掌握着真传的钥匙。”

%47
方源肯定是要前往光阴长河当中的。

%48
五域外界的光阴支流,就是他进出的门户。

%49
虽然为了铲除天庭的追兵,方源损失了其中一道,但是没有关系,在影宗的记录当中,光阴支流可是还有五六道呢。

%50
“不过,目前西漠当中,影宗掌握的光阴支流,就只剩下了最后一道。”

%51
“保险起见,我还是先前往那两个资源点。在没有十足的把握之前,不要轻易前往那里了。”

%52
方源暗中做出了决定。

%53
现在他人在西漠,西漠和中洲、北原、南疆分别接壤。

%54
但是这三域,方源都去不成了。

%55
中洲有天庭、十大古派,是方源的禁地。

%56
北原有长生天,到处通缉方源。

%57
南疆的正道蛊仙们,不久前还追杀围剿过影宗呢。

%58
西漠是最佳的潜藏地点。

%59
至于东海,影宗方面只掌握了一道光阴支流的位置。但是很不巧,这道光阴支流已经被庙明神搜走了。

%60
说起来,这事情还怨方源。因为就是方源发现之后,告知庙明神一伙儿的。

%61
曾经,紫山真君为了解情况,也借助光阴支流,去了一次鬼脸红莲。他在东海时,找不到那条光阴支流,只好到了南疆,才如愿以偿。

%62
“我现在最大的麻烦,就是身上有着侦查杀招,位置始终暴露在天庭蛊仙的眼中。”

%63
“同时没有暗渡仙蛊傍身,天意也在时刻布局,企图剿杀了我。”

%64
“关键是缺乏仙蛊,我手中的杀招海量,就是独缺核心仙蛊。有了核心仙蛊,就能解除侦查杀招,也能再度蒙蔽天意!”

%65
正思考着,方源接到了毛六的来信。

%66
方源将心神探入到这只信蛊当中,里面的内容让他皱起眉头。

%67
炼蛊再一次失败了。

%68
真是该死!

%69
虽然他陷害了凤九歌,但天庭方面绝不会只派遣他来追杀自己。

%70
天庭一定还有后手,只是时间早晚而已。

%71
“这只智道仙蛊自爱仙蛊,一下子要炼出七转层次来,的确颇有难度。最近几次炼蛊,我都全程参与,琅琊地灵尽心尽责,绝没有敷衍了事的成分。”毛六如此说道。

%72
方源没有怀疑他的话。

%73
琅琊地灵的单纯和诚信,方源是相信的。

%74
炼制不出来,也很正常。

%75
毕竟是七转仙蛊,成功的可能很低。

%76
但这只自爱仙蛊,方源必须炼出来。而且要速度快。

%77
有了它,方源才能运用紫山真君遗藏中的一道杀招,解除自己身上的侦查杀招。

%78
这是完全可以的。

%79
方源对它保有充分的信心。

%80
“炼,必须接着炼,哪怕是倾家荡产,也在所不惜。”方源咬牙,暗自发狠。

%81
他现在获取琅琊门派贡献,再轻松不过。因为紫山真君的遗藏,丰富无比。

%82
“不过……”

%83
“我也不能无动于衷了。”

%84
“这一次陷害凤九歌成功,是出其不意,动用了八转仙蛊似水流年。这是天庭暂时得不到的情报。”

%85
“如今这张底牌已经暴露,我必须抓紧时间!”

%86
“看来,是时候动用那个方法了。”

%87
方源眼眸中闪过一抹决意。

%88
他回信道:这一次炼蛊,请毛六主持大局,让琅琊地灵打下手。

%89
毛六接到方源回来的信蛊,看到这部分内容,不禁感到相当的诧异。

%90
他自问自己的炼蛊水准,并不如琅琊地灵,琅琊地灵也没有徇私舞弊,但为何方源偏偏要让他来主持炼蛊大局呢?

%91
光阴长河之中。

%92
隆隆的水声,不绝于耳。

%93
凤九歌被滔滔巨浪,卷进长河之中。

%94
“这里就是光阴的长河吗?”他竭力稳住身形,立即感到无比的宙道威能,不断地侵蚀自己的仙躯。

%95
强。

%96
强大无比!

%97
凤九歌的仙躯防护,也是极为不俗的。

%98
但在这短短一瞬间,他就感到自己正被周围的环境极力排斥。

%99
凤九歌十分明白,这是因为他身上满是音道的道痕,和这里的宙道道痕格格不入。

%100
“不愧是光阴的长河,必须尽快出去!”凤九歌想要退走,但是回望一眼,之前的光阴支流已经毁灭,留下的只是一层薄薄的光阴斑斓。

%101
这层斑斓非常的巨大,但是对于凤九歌而言,却再不是什么出路。

%102
“这下麻烦了。我不是宙道蛊仙,单靠光阴斑斓是出不去的。该如何是好?”

%103
正彷徨着,凤九歌的耳畔,陡然一炸。

%104
虎吼声起,一头巨大的虎形年兽,从河底钻出来。

%105
砰的一声巨响,河水四溅。

%106
凤九歌瞳孔缩成针尖大小,他的身躯和这头虎形年兽比较起来,宛若老牛身旁的蚊子。

%107
“是太古年兽!”

%108
死亡的阴影,笼罩凤九歌。

\end{this_body}

