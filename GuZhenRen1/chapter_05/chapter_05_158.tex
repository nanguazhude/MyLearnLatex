\newsection{黑凡洞天之争(下)}    %第一百五十八节:黑凡洞天之争(下)

\begin{this_body}

%1
“呼呼呼……”

%2
方源悬浮在半空中,气喘吁吁,目光扫视整片战场。

%3
惨烈的厮杀终于结束了。

%4
战场已经再找不见原先的模样,各种坑洞中躺着冰刺神猿支离破碎的尸首,一些炎道杀招造成的火焰,还附着在地面上静静燃烧。更多的是冰霜,混合着血水、四处碎裂的骨肉,还有荒植雪柳的残根败叶。

%5
蛊仙们难免也出现了死伤。

%6
伤势各个都有,死亡的那位正是修行偷盗的女蛊仙。

%7
她没有被天龙杀死,但是在随后赶来支援的战斗中,被百足天君的几大分身联手暗算,失去了生命。

%8
方源的身上也有伤。

%9
他的左肺被一道冰刺洞穿,现在宛如投枪般大小的冰锥,还插在他的胸口。

%10
他的整个下肢,都覆盖了一层厚实的冰霜,许多肌肉已经被冻得坏死。

%11
至尊仙体有着巨大的优势,那就是道痕之间没有掣肘。但这一点,也是一个巨大的劣势。一旦方源中招,攻破了他的防护手段,打在身上的话,他受到的伤害就是完完全全,不会受到身上道痕的自然削弱。

%12
这一次厮杀,让方源再次深刻地感知到,至尊仙体的脆弱特性。

%13
“防御杀招还是薄弱了。”

%14
“血本仙蛊还在修复,血染征袍是用不了。宙道的防护手段,也并非优异。”

%15
方源虽然得了黑凡真传,但真正的核心仙蛊似水流年,一直被封印着。这只仙蛊有巨大弊端,那就是一旦气息泄露出去,就会有可能惹来太古年兽。

%16
一旦八转战力突然进入了方源的至尊仙窍,那方源必输无疑。只能魂魄离体,放弃整个至尊仙体。

%17
黑凡真传中的防御手段,其实也很优秀。但需要似水流年为核心。

%18
方源现在运用的防护杀招。是以年蛊为核,效果较弱。是黑凡早年时期。采用的手段。等到他七转之后,就再没用过。

%19
“百足天君不愧是八转奴道,还真是厉害!”

%20
“尽管他的真身还在洞天之外,仅靠分身指挥作战,就能达到这等战果。我是万万不及。”

%21
“还有这种扭曲空间,向洞天内输送荒兽、上古荒兽的手段,也是实用非凡!”

%22
这场厮杀,百足天君一方有上百头的冰刺神猿。还有数量更多的荒植雪柳。并且在交战过程中,还出现了扭曲光影。尽管蛊仙们拼力阻击,仍旧从中钻进来两头冰瀑神猿。

%23
冰刺神猿。

%24
冰瀑神猿。

%25
两者一脉相承,关系紧密。

%26
前者是荒兽,眼白碧绿,瞳仁霜青,浑身皮毛如一道道冰锥,布满体表。后者是上古荒兽,浑身如雪,双目赤红如血。根根猴毛如道道冰刺,直刺向天。

%27
方源对冰刺神猿也比较熟悉了。早年时候,琅琊福地中就有冰刺神猿。

%28
之后。方源还借过来,防护狐仙福地,抵御住了仙鹤门的一波进攻。

%29
后来,冰刺神猿就阵亡在琅琊保卫战中。

%30
冰瀑神猿,方源也和它交手过。

%31
那次是帮助黑楼兰升仙渡劫。北部冰原上的灾劫,因为狂蛮真意的影响,变化成的神猿形象。

%32
“现在看来,灾劫变化形成的冰瀑神猿,其战斗力远远不及真正神猿。”

%33
“其实。若只是这些荒兽、荒植,虽然麻烦。但并不难对付。关键是百足天君的分身策应和指挥,这就完全是两码事了。”

%34
“尤其是荒兽、荒植。竟然形成军阵,统一行动,施展出奴道杀招时爆发出的强大战斗力,叫人心惊。”

%35
方源回顾整个战斗,充分体会到了八转蛊仙的强大。

%36
即便是本体没有亲临战场,隔着洞天进行指挥,方源也完全不是对手。

%37
能够战而胜之,全是结合了众仙之力。

%38
“这样的惨烈战斗,若是再进行几场的话,我的青提仙元怕是要干涸殆尽。”

%39
“年蛊的年份也跌落了许多,需要加紧平炼上去才行。”

%40
“治疗身上的伤势,也要耗费不少时间和仙元。”

%41
这种消耗,方源也感到要承受不住。

%42
百足天君的损失,比方源还大。

%43
打到这种地步,楚度一方依靠黑凡洞天,将战斗拖入僵持阶段,双方完全就是拼消耗。

%44
蛊仙们耗费仙元,百足天君则损失洞天的底蕴。双方都骑虎难下,就看哪一方先支持不住。

%45
接下来的战斗,变得更加激烈。

%46
百足天君的尝试,虽然失败了,但也看到了效果。

%47
他屡次在黑凡洞天各处,企图打造出临时营地,稳住阵脚之后,不断送进上古荒兽,让楚度这些防卫的蛊仙苦不堪言。

%48
纵观整个黑凡洞天的攻防战。

%49
百足天君最初,是指挥分身直接进攻各个资源点,打击楚度的底蕴。他成功了。

%50
而后是大量投放荒兽,让整个战局向他渐渐倾斜。

%51
现在,他改变战术,建设临时营地,反攻为守,逼迫蛊仙们主动进攻。

%52
虽然楚度、方源人多势众,百足天君只有独自一人。但他牢牢掌控局面,将楚度、方源等人压在下风。

%53
尽管方源等人极力抵抗,也难以阻止局势向百足天君那边倾倒。

%54
双方都在坚持。

%55
但方源首先支撑不住。

%56
他表面上是七转蛊仙,实际上一直用青提仙元撑着场面。

%57
楚度的脸色也很不好看。

%58
尤其是和方源独处的时候,他卸下伪装,面罩忧愁。

%59
他对方源道:“能够招揽的人手,我都尽全力招揽过来了。”

%60
“但这场战斗,最终的胜利者会是百足天君。尽管我们没有和他真身交锋,但八转蛊仙的底蕴,也是我们不能相提并论的。”

%61
“真正胜负的关键,不在于这里的战场。而在于外界的百足家啊。”

%62
楚度对整个局势洞若观火。

%63
方源和他可谓英雄所见略同。

%64
百足天君本身是没有弱点的。他的唯一弱点就在于刚刚正式创建的百足家族。

%65
楚度所做的安排,正在北原蛊仙界发酵。

%66
流言纷起。

%67
“百足天君正在攻打黑凡洞天,若是让他得到这处地方。他的实力必将更上一层楼。”

%68
“百足家虽然新建不久,但有百足天君撑腰。唯一的弱点可能就是底蕴不足。一旦黑凡洞天成为百足家的地盘,那么数十年后,百足家将凌驾于各大黄金家族之上。”

%69
“百足天君乃是散修,并非黄金血脉。其实百足家的壮大,也有利于散仙和魔仙的生存。”

%70
……

%71
种种言论,喧嚣尘上,主体就是百足天君威胁论,千方百计地强调他的血统不是巨阳仙尊的后人。

%72
不仅如此。还有一些蛊仙乔装打扮,以百足家的名义,四处寻衅滋事,挑逗各大超级势力的神经。

%73
百足家的首领百足卫已经战死沙场,百足家的新任首领才具不足,压力极大,屡次上书百足天君,告知百足家的危局,堪称是四面楚歌。

%74
但百足天君岿然不动。

%75
他在回信中写道:“这正是楚度之计。楚度并非正道,也不是黄金血脉。我们百足家对付他,正合了那些黄金血脉的心意。”

%76
“楚度放出的这些谣言,还有乔装打扮。去刻意挑衅其他家族,这都是他的计谋,正说明了他虚弱的本质,外强中干。”

%77
“其余家族绝非笨蛋,早就看出楚度的计谋,只是慑于我八转之威,不敢动手。”

%78
百足天君的智谋也绝对出色,洞悉全局,势必要拿下黑凡洞天。

%79
他选的时机实在太好。

%80
药皇、雪胡老祖都在忙着炼蛊。五行大法师和长生天纠缠。凤仙太子虽然表面上无所事事,但实际上最近都在忙着回收仙窍福地。

%81
对方源而言。时间变得越来越难熬。

%82
为了提高青提仙元,他不得不开始向宝黄天中贩卖荒兽。

%83
刺脊星龙鱼本身还有两头。都被他果断卖了,得到大量的仙元石挽救危情。

%84
这已经开始危害到方源的仙窍底蕴。

%85
之前,他都是靠着良好的经济运作,进行良性循环。如今却是落入到恶性循环之中去。

%86
毫无疑问,这是个危险的信号。

%87
“单论福地底蕴,我比楚度都要高。但青提仙元是我最大的短板,所有的蛊仙中我首先支撑不住。看来此战之后,我必须尽快升上七转了!”

%88
战况越发激烈,已经陆续有临时营地建立起来。楚度不得不亲自出手,参与摧毁行动。

%89
百足天君看准时机,好几次惊险万分,真的是差一点就让他真身降临。

%90
方源暗中向琅琊派寻求支援。

%91
他向琅琊地灵言说:这可是招降楚度的最佳时机。

%92
但琅琊地灵面对八转蛊仙,果断怂了。他明确表示,琅琊派绝不会出手。方源若是出事,必须对琅琊派的事情万分保密,若有丝毫透露,身上的盟约就会直接索命。

%93
“尽管还可以在支撑下去,但我想了想,既然战局已定,无故的消耗还是算了吧。”楚度萌生退意。这个心思,他对方源没有遮掩,而是直接找方源商量。

%94
方源早就想退了,点头答应。

%95
不过就在楚度和方源商议撤退的计划时,北原蛊仙界忽然风起云涌,起了变化。

%96
各大黄金家族,正道的超级势力,忽然联起手来,要对百足家施以制裁。

%97
没有想到,竟然在最后关头,转机出现了!

%98
ps:今天手有点生,状态还是不好。修补一个bug:百足天君的流派是奴道,不是宇道。今天就一更,我打算好好休养身体,把状态调整好。这一更和昨天欠下的两更,我打算25号一起还给大家。25号我打算再爆一次更新,算是对大家的补偿吧!谢谢大家的正版订阅支持,既然大家如此支持,我也会尽全力的!

\end{this_body}

