\newsection{炼身}    %第二百一十二节:炼身

\begin{this_body}

关注「起点读书」,获得515红包第一手消息,过年之后没抢过红包的同学们,这回可以一展身手了。

“不过,用于解梦的凡蛊,却是消耗极大。如今,解梦的施展次数,只剩下十次不到。”

想到这里,方源不禁轻轻地皱起了眉头。

这是个问题。

解梦杀招虽然非常犀利,但是对于梦道凡蛊的消耗很大。

再加上这些梦道凡蛊,都是方源独自一人,炼制出来,产量方面一直无法提升。

将这些梦道蛊方,透露给琅琊派,让毛民蛊仙帮助炼蛊?

方源根本没有考虑过这一点。

在这个时间段,这些梦道凡蛊蛊方的价值,比仙蛊还要巨大!

当然,方源相信琅琊地灵能付得起这个价格,但是天下没有不透风的墙,流传出去的话,造成的影响实在是太大太大。

方源在梦境探索方面,可谓五域顶尖。这些梦道凡蛊方一旦为人所知,那么他的优势就荡然无存了。

这是方源绝不想看到的情况。

因此,他宁愿自己炼制,宁愿效率低下,也要严格把控源头。只要自己不透露,谁还能清楚?

“找他人帮助不行,但却可以借助春梦果树之类的外物,提高我炼制梦道凡蛊的效率啊。”方源再次将主意,打到春梦果树身上。

春梦果树是一种非常奇特的植株。

它存在于梦中,并没有实体。但它需要一株现实中的树,作为依据的基石,才能长存。

方源知道一株春梦果树,位于中洲的某个无名火山脚下。

这座火山每隔数百年喷发一次,形成肥沃土壤,一段时间后,就会有人聚居山脚。

方源曾经采摘过这株春梦果树上的果实,这些果实最终帮助了他很多。

可惜的是,方源并没有移栽春梦果树的手段。

火山脚下的春梦果树。依附着一株数千年的火枣树身上。方源若是挪移火枣树,反而会让春梦果树无所依托而毁灭。

但现在中洲对于方源而言,又是龙潭虎穴,根本不能去。

见面曾相识可以提供伪装。但是却不能避免智道蛊仙的推算。暗渡仙蛊能够在这个方面,帮助方源,但暗渡仙蛊却只是六转层次,说实话,并不够看。

这个问题。方源暂时无解,超出了他的能力范围。

让他再寻找第二株春梦果树,他也没有这个手段,只能作罢。

休息了片刻之后,方源并没有急着钻入梦境中去。

对于第八幕梦境,方源还需要推算。

他现在可是智道宗师,事前推算一些东西,能够有效地帮助他探索梦境。

这是他的优势,他当然得运用起来。

然而在智道方面,他缺乏的不是境界。而是仙蛊了。解谜仙蛊是他仅有的智道仙蛊。单单一只,且没有配套的智道杀招,所以解谜仙蛊的作用并不是很大。

目前,方源算是勉强够用的状况。

智道方面,并非他目前发展的重点,所以方源选择维持现状。

推算的东西,当然并不仅仅是关于梦境,还有方源本身身处的环境。

“如我之前预料的那样,巴家是我的最大敌人。前段时间屡次找我麻烦,都被我化解。现状看来似乎想要偃旗息鼓了。”

巴家不活蹦乱跳,反而让方源有些忌惮。

毕竟对方不出手,方源就不知道对方想干什么,在酝酿什么。有什么图谋。出手了的话,反而可以看清楚对方的盘算,再从容应对,兵来将挡水来土掩嘛。

“总体而言,外在环境还是不错的。顶着一个武遗海的身份,还真是好用得很。”

“原本仙缘生意是弱点。不过经过我之前的处理,已经没有关碍了,反而成了我摆在面前的一个局。说不定能套中脑筋不好的蛊仙呢?”

毕竟,智道蛊仙的数量,还是始终稀少的。

就算是智道蛊仙,也未必是正道,不知晓正道的游戏规则,也是不行的。

至于武安、武辽都各行其职,没有让方源操心什么。

武辽前段时间,还主动来汇报一些事情,但方源决心当甩手掌柜,把他训了一通后,告诫他没有什么大事,不要来麻烦自己。

武辽大为失望,觉得方源和前任七转,完全是一路货色,对方源的态度也隐隐冷淡下来。

方源敏锐地察觉到这一点,不以为意。

这片超级梦境,时时刻刻都在缓慢地扩张。

方源偷偷探索梦境,会导致梦境缩小,但对于整个梦境而言,却是很小的一部分,最多只是让梦境扩张的速度变得更加缓慢而已。

这点完全可以遮掩。

那些通过仙缘生意,进入梦境探索的魔道和散仙们,都是方源最好的掩饰。

这些人在梦境中,虽然几乎都是在碰壁,但偶尔也有那么一两位,虽然探索不成,但也能从梦境中带出一些梦道的蛊材出来。

这些梦道的蛊材,自然是能够炼出梦道蛊虫的。

这些蛊仙如获至宝,虽然只有绝少数人有这样的收获,但仍旧激发着其余蛊仙的巨大热情。

而正道家族们,也在暗地里,向这些蛊仙收购梦道的蛊材。

武家也不例外。

不过,拥有梦道蛊材的蛊仙们中,选择外卖梦道蛊材的人很少,基本上都会珍藏。

武家也只收购过来十多件蛊材,这些梦道蛊材方源可不敢乱动,武家上下都盯得紧紧的呢。

休整了之后,方源沉入梦境。

只是并非超级梦境,而是他自己的梦。

想要获取梦道蛊材,当然是自家的梦境最为优良,并且安全了。

可惜,那些蛊仙都不清楚这一点。

南疆,白相洞天。

一座浩大的蛊阵中央,白凝冰勉强盘坐下来。

白相天灵居高临下,操纵着整个蛊阵。

蛊阵悄无声息地运转着,时不时的有莫名色彩的流光。时而汇聚成股,宛若道道溪流,时而分散若花,窸窣漫舞。

白凝冰脸色惨白。但是气息却教之前,均匀平稳了许多。

这是因为蛊阵运转期间,替代白凝冰分散了许多压力。他用身体储藏了惊涛升龙火,实在是有些乱来,简直是将性命放在悬崖边上走钢丝。

“接下来。就是最重要的步骤,对你体内的惊涛升龙火下手!”白相天灵满脸严肃,“少主人,你可要坚持住,不能有丝毫的懈怠,必须时时刻刻和蛊阵配合,压制惊涛升龙火。整个过程至少要有七七四十九天。期间,你得不到丝毫的休息,只要我没有喊停,就意味着没有结束。哪怕在最后关头懈怠一点点。都会功亏一篑,身死魂消。”

白凝冰眼冒金星,一阵阵眩晕,几乎让他坐不住。

这种状况,别说是四十九天,恐怕就是半天,他都有支撑不住的危险。

危在旦夕,他却嘴角微翘,反而在笑。

“呵呵呵,真是好玩。”

“依照我目前的状况。这是不可能完成的事情吧……”

“正好可以丈量我的器量。”

“即便是这样子死去,也挺精彩!”

他的身体非常疲惫,但是精神上却反而亢奋起来。

深呼吸一口气,白凝冰低声开口。对白相天灵道:“还磨蹭什么,不赶紧开始?”

白相天灵嘿然一笑,心想:自己苦苦守候了不知多少年,等来这么一位主人,似乎格局不差,视死如归。

这样想着。白相天灵便调动了整个蛊阵。

轰!

下一刻,璀璨的光辉,从蛊阵中爆发出来,并且迅速弥漫了整个天地。

“啊!”白凝冰仰头惨叫。

一大股火焰,从他的眼鼻口耳等处,汹涌激喷而出。

炙热的火焰,在瞬间弥漫笼罩他整个身躯,把白凝冰燃烧成一个人形火炬。

痛痛痛!

白凝冰本身乃是个重伤濒死,都不会皱下眉头的人物。但是此刻,他却毫无风度,仰头惨嚎。

实在是因为,这种痛楚太过于剧烈,让他不得不这样做,才能发泄一二。

白凝冰原本也料想到,火焰燃烧时的痛楚。但实际上,却并非如此。

惨嚎声中,他忽然明悟过来一点:这个人人如龙炼蛊法,并非只针对惊涛升龙火,而且还针对白凝冰这个人。不管是他的肉身,还是他的魂魄,亦或者他满身的道痕,都在人人如龙炼蛊法门之下,不断拆解融分。

这等若是将白凝冰当做一种仙材来处理,痛楚自然猛烈无比,言语难以表述万一。

“撑住,撑住!”白相天灵急得大叫。

白凝冰已经双眼翻白,惨烈的痛楚,在瞬间将他逼至绝境,让他即将陷入到昏迷的状态中。

一旦他昏迷,只要那么一刹那,他就会被整个升龙火燃烧成灰!

北原,大雪山第一山峰之巅。

马鸿运也在惨嚎。

“啊啊啊啊啊!”他叫得脖子上青筋暴起。

剧烈的电芒,不断在他身上跳动,死死纠缠,让马鸿运整个肉身都在剧烈的颤抖,仿佛是癫痫,又似乎是在狂舞。

片刻之后,电芒消散,马鸿运气喘吁吁,汗如雨下,浑身一丝力量都没有。

“还不成功?”万寿娘子皱起眉头,有了一些怒意。

马鸿运已经被电炼了许多次了,虽然每次狼狈不堪,但从未缺胳膊少腿,仍旧完整无缺。并且,他的修为已经在不知不觉间,提升到了五转巅峰。

这让万寿娘子都有些无语了。

ps.追更的童鞋们,免费的赞赏票和起点币还有没有啊\~{}515红包榜倒计时了,我来拉个票,求加码和赞赏票,最后冲一把!(未完待续。)

\end{this_body}

