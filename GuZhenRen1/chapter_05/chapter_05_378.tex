\newsection{千变老祖}    %第三百七十八节:千变老祖

\begin{this_body}



%1
炙热的阳光,好似道道利箭,万箭齐发,普照在这片壮观的沙漠上。

%2
在这片沙漠上,伫立着无数的沙雕。

%3
这些沙硕凝聚的雕像,有的是人,有的是兽,有的在拼杀,有的躺倒在地上,似乎弥留着最后一口气息。

%4
沙雕有大有小,但最小的都有数丈之高。

%5
一座座沙雕,数量极多,汇聚成一座沙雕森林,景象极其壮观。

%6
正是万像沙漠!

%7
这座沙漠覆盖面积极广,拥有着难以计数的沙雕,在西漠蛊仙界中,无人不知无人不晓。

%8
它来历悠久,上古年间形成,流传至今已经有一百多万年了。

%9
这片沙漠的成因,也在人族的历史上,有着浓墨重彩的一笔。

%10
那是狂蛮魔尊的时代,百部大战,在这里厮杀,持续百天百夜,杀得血流成河,几乎蔓延成海。

%11
那个时代,变化道是主流。

%12
百部大战,人族惨胜,不管人族还是其他异人,大量的蛊仙陨落于此。

%13
仙胎种下,形成福地,道痕逸散,改造环境。

%14
渐渐的,这里成为万像沙漠。沙硕自动凝聚,宛若树木生长,每年一尺,最终长成一座座的巨大沙硕雕像。

%15
曾经的大战之地,百部混杂在一起,杀得天昏地暗,无数生命凋零。福祸相依,如今这里反成了资源充裕的修行圣地。

%16
每一座沙雕,只要长高到六丈,就是仙材,充斥变化道痕。

%17
这里的仙窍福地已经没有了,一百多万年的时间,这些福地不是被灾劫摧毁,就是被蛊仙掠夺。

%18
不过这里,有海量的野生蛊虫,甚至孕育出野生仙蛊。

%19
没有人会轻易涉足这里。

%20
这里的环境,道痕浓郁,凡人蛊师难以涉足,对他们而言,环境本身就是危机四伏。

%21
而西漠的蛊仙,来到附近,也往往会绕道而行,避而远之。

%22
他们这么做的原因,和一个人有关。

%23
千变老祖!

%24
这位专修变化道的八转蛊仙,已经占据这里长达千年之久。

%25
八转之威,无人敢轻易冒犯。

%26
万像沙漠就是他千变老祖的地盘,任何蛊仙不经允许,踏足这里,就要做好承受八转怒火的心理准备。

%27
沙漠地底深处,万像宫殿。

%28
大殿深处,幽深的黑暗角落中,传出一声声惨烈的痛嚎。

%29
“啊吼!”

%30
千变老祖仰头咆哮,似乎十分痛楚,神情狰狞可怖。

%31
每一声嘶吼,都掀起澎湃的音浪,四下倾泻。

%32
但是每当这些音浪,撞击到宫殿四周,宫殿表面就会浮现出一层薄薄的光晕,将音浪迅速吸收消融。

%33
仅仅一墙之隔,就宛若两个天地。千变老祖的动静,根本传不出去。

%34
痛!

%35
剧烈的痛!

%36
难以忍受的痛楚,不断地袭击千变老祖的身心。

%37
他额头上青筋冒起,终于忍受不住,痛得在地上打滚。

%38
这样的情景,如果让其他蛊仙看到,必定会吃惊不已。究竟是什么原因,让千变老祖这位八转蛊仙遭受这样的折磨,一点八转的威风气度都没有了?

%39
哐啷、哐啷!

%40
十几条粗壮的长铁链,束缚着千变老祖的周身、四肢,铁链的另一头,则深深地插在四周的墙壁当中。

%41
看这种样子,千变老祖竟然被囚禁着?!

%42
“不行了,再这样下去,我就要彻底疯狂,被无边的剧烈痛楚吞噬一切的理智,成为疯子!八转的万劫,第一次就如此恐怖难缠,之后两次我又该如何面对?”

%43
可恨!

%44
千变老祖的眼眸深处,还有最后一丝清明挣扎着。

%45
借助这丝清明和理智,千变老祖终于做出了决定!

%46
仙道杀招割肉变!

%47
他猛地催动杀招,这个杀招他滚瓜烂熟,在最关键的时刻,哪怕是痛彻心扉,也没有催动失败。

%48
这是八转仙道杀招,气势汹涌。

%49
千变老祖的整个仙躯,迅速膨胀起来。

%50
就像是充了气似的,越变越大,从原本一个非常精干的中年男子,转变成了一个肥胖肿大的肉球。

%51
大胖子千变老祖甩动一身的肥肉,很快便响起一连串的轻微声响。

%52
啵啵啵……

%53
随着这股声响,从他的脖子的一侧上,忽然鼓起了一个肉包。

%54
肉包越加肿大,也像是充了气似的,不断膨胀。

%55
痛楚加剧!

%56
千变老祖疼的龇牙咧嘴。他的脸上也是浮肿一片,连眼缝都几乎没有了。

%57
噗。

%58
很快,他脖子上的肉球膨胀到了脸盆大小后,达到了极限,陡然炸裂开来。

%59
肉球自爆,黄脓四溅,一个年兽从中跳跃而出,蹦到了地砖上。

%60
汪汪汪!

%61
这头年兽落到地上后,开口吠叫,赫然是一头狗形年兽。

%62
并且它的气息并不弱小,乃是一头荒级年兽。

%63
狗形年兽落到地面上后,千变老祖顿时感到一丝的轻松。就好像是他原本背负着一座山峰,而山峰上的一块巨石,忽然被挪走了。

%64
啵啵啵……

%65
下一刻,千变老祖的浑身上下,竟都涌现出一个个的肉包。

%66
噗噗噗。

%67
在很短的时间里,这些肉包不断膨胀、自爆,溅出脓水,跳出一头头的年兽。

%68
鸡猴羊兔种种形态的年兽,不一而足。

%69
这些年兽不只是荒级年兽,更有少数,乃是上古荒兽。

%70
年兽们彷徨了一阵后,开始联手,向千变老祖攻去!

%71
千变老祖没有丝毫意外,面色上他反而越加轻松了许多。

%72
他的痛楚大为减少,虽然还是痛着,但是已经落到他可以忍受的程度。

%73
面对十多头年兽的夹攻,千变老祖冷哼一声,心念一动。

%74
哐啷啷!

%75
一连串的激响,一条条的粗壮铁链,从四周的墙壁上猛地电射而出。

%76
然后以迅雷不及掩耳之势,铁链缠上宫殿中的一头头年兽,不管年兽如何奋力挣扎,都牢牢地束缚住它们,将它们镇压。

%77
千变老祖吐出一口浊气:“总算是缓过来了。”

%78
他信手一抓,将离他最近的一头蛇形年兽,抓到自己的眼前,催动手段,细细观察。

%79
这头蛇形年兽,不断吐信,发出嘶鸣声,想要咬上千变老祖的脸面,但苦于被束缚住,无法得逞。

%80
片刻后,千变老祖面色不变,心中却震动起来。

%81
“我刚刚用尽了手段,这只蛇形年兽,居然仍旧对我敌意重大,恨不得将我挫骨扬灰。”

%82
“我这招割肉变,乃是狂蛮真传中的仙道杀招自杀变中的一式。每当我用它,从仙躯肉体上分化出的生命,不算是什么形态,都是如臂指使,让我运用自如。因为它本身就是我的肉身一部分。”

%83
“但现在这些年兽,分明都是从我肉身中分离出来,却都不受控制。任凭我用尽手段,也毫无起色。”

%84
“好厉害的万劫!”

%85
千变老祖的额头,不禁垂落下一滴冷汗。

%86
他回想起半个月前,自己渡劫的情景,仍旧是心有余悸。

%87
八转蛊仙统共要有三次万劫渡过。

%88
千变老祖渡的是第一次万劫。

%89
“如果不是这座狂蛮魔尊留下的仙蛊屋,我根本不能渡过第一次万劫。现在就算渡过,也是残留伤势,只能凭借割肉变,进行壮士断腕。”

%90
千变老祖原本想要硬抗,寻找其他的解决方法,但他支持了半个月,达到了自己的极限,再也扛不住。

%91
无奈之下,他只能运用割肉变,将自己的肉身仙躯舍弃一部分,将身上的伤势都转嫁到这些舍弃出去的肉身上。

%92
他这一次,遭遇到了宙道万劫,身上残留着的伤势,都是宙道道痕。

%93
千变老祖将这些宙道道痕,都通过割肉变,舍弃出去。

%94
如此一来,这割肉变化,就化为了一头头的年兽。

%95
并且因为充斥天意,这些年兽转化成型之后,根本不受千变老祖的控制,都成为了他的死敌。

%96
千变老祖非常肉痛。

%97
他这样做,损失太大了。

%98
且不说万劫对于他的仙窍,造成了几乎毁灭性的打击。

%99
就说这割肉变舍弃出去的道痕,不仅是宙道道痕,还有千变老祖本身的变化道痕。

%100
这些道痕损失的非常巨大。

%101
千变老祖默默估算一下,他渡了万劫之后,不仅没有什么道痕收益,反而在道痕方面,还有不小的亏损!

%102
“这么一来,我渡劫成功了,反而不如渡劫之前?”

%103
念及于此,千变老祖饱经沧桑的心里,也不由地升腾起了一股辛酸之意。

%104
他深深地叹息一声,充满了无奈。

%105
没办法。

%106
他之前修行,勇猛精进,面对万劫,拖延了好长一段时间。实在拖延不住了,才进行渡劫。

%107
万劫虽然渡过了,但是千变老祖的底蕴几乎没有丝毫的增加,甚至道痕方面比之前还要薄弱。

%108
“我错了。”

%109
“一千年前,我仰仗这处狂蛮真传,以为自己会无往而不利。”

%110
“地灾、天劫、浩劫,我都闯荡过去,勇猛精进。但万劫却完全是两个概念!”

%111
“我太低估万劫的威能了。这些年来,我的心性不知不觉间变得自大。这一次遭受重创,是祸也是福啊。”

%112
千变老祖能修到八转,自然有过人之处。

%113
他痛定思痛,深刻反省,剖析自己。

%114
就在这个时候,这座大殿之外,传来一位女仙的声音:“夫君大人,祠堂中有些意外发生了。翠波妹妹的命牌蛊忽然半碎开来,魂灯蛊也暗淡无光。她此前去了易位沙漠,如今我们姐妹都联络不上她。事关重大,妾身不得不向夫君大人您禀告!”

\end{this_body}

