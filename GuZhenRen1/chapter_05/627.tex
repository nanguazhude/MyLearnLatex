\newsection{联手压制}    %第六百三十节:联手压制

\begin{this_body}

方正福地。更新最快

宇道、宙道的资源都超出正常水准,这是上等福地。

福地中的地貌,以平原为主,但每隔一段距离,就会有数座山峦突兀地矗立着。

方正此刻的注意力都一处山谷中。

山谷怪石嶙峋,里面一片血红之色。深入进去就会发现,山谷中已经种植了大片的血花树。

这种树足有数万棵,和寻常树木不同,它是先结果后开花。结出的果实如婴孩般拳头大小,饱满圆润。开花只是一瞬间的事情,整个果壳会陡然炸裂开来,迸发出一朵红白相间的巨大花朵。花朵以白色为底,但以花心为中点,会有血液飞溅的颜色。每一朵血花的颜色、形状都不相同。

在中洲,曾经有一段时间,有大批的蛊师豢养这种血花树,每当花开时节,都是一场盛景。当时甚至还有大型的庆典,吸引远近的蛊师前来观赏。只是到了后来,血道开创,荼毒世间,血花树作为血道中的基础蛊材,才逐渐被禁止栽种。

方正手中的这些血花树,当然是仙鹤门提供的。更准确的说,是由樊西流的个人名义来提供。

这是正道的把戏,方正如今已见怪不怪,习以为常。

在当今五域,血道被各大正道明令禁止,但讽刺的是,真正暗中大肆钻研血道的,都是这些名门正派。

“这些血花树都是凡级蛊材,但只要数量上超过一定的规模,就能堆叠使用,类比仙材。这是血道最独特的优势!”

“我掌握的这些血花树,也可以看做是一项仙材了。”

“只是如今,我已经拥有六转血道仙蛊冷血和血仇,需要的却是凡级血蛊辅助,从而形成仙道杀招。”

方正心中明白,他接下来的修行重点,就是炼制血道凡蛊,然后通过樊西流的传授,练习血道的仙级杀招。

还有就是他升仙不久,空窍和仙窍之间有着极大的差别,这点他需要时间来适应。

浏览一阵,方正便抽回心神,回到现实当中。

浏览自家仙窍时的淡淡成就和欣慰的感觉,顿时消散不见,涌上心头的是若隐若无的压抑。

“方源……”一回到现实,方正就不可避免地想起他的哥哥。

这些天来,他不只是升仙,还从樊西流那里得到了方源的情报。这些情报可比外界都要详尽得多,源头来自于天庭。

方正由此得知方源的近况,至于方源并没有死亡的事实,他早在刚刚被救回中洲后就得知了。很奇怪的是,当他得知这个事实的时候,心中并无一丝被蒙骗的愤怒。

反而,随着他近来知道的越来越多,他的心中却越发伤感起来。

“原来,我从未真正了解过你……方源。”

“天外之魔……春秋蝉……八转修为……”

了解的越深刻,方正就越发感受到自己的渺小,真正感受到方源的恐怖和强大!

想到他今后必须面对这样的敌人,他心中自然压抑,甚至还有一丝的绝望。

他正视内心深处的绝望。

他从不否认,也并未逃避。

人常言,初生牛犊不怕虎,年轻的时候天不怕地不怕,有着一种冲劲。但真正当一个人经足够后,就会发现自己的恐惧。原来老虎是能吃人的,老虎的爪子能轻易地抓破牛皮,原来自己的力气、自己的牛角并不像自己想象中的那么强。

想象都是美好的,只有和现实进行充分的对撞之后,才能感受到痛苦,并从痛苦中清晰地认识这个世界,以及自己。

“当我是一位五转蛊师的时候,我在琅琊福地,夹杂在毛民三国的纷争中,颠沛流离。”

“如今我成为蛊仙,则夹在方源和天庭之间,成为天庭的棋子。我的实力变得远超之前,可惜处境却更加身不由己,危机更大,稍有不妙恐怕就会身死道消了。”

念及于此,方正不禁苦笑连连。

到了现在,他也不知道自家有着什么样的价值,能够被天庭如此看重。

很快,他又联想到了赵怜云。

他升仙之后,获得了一些贺礼。这是五域都有的习俗。每当一位蛊师成功升仙,他(她)的亲朋好友,交好势力等等都会送上贺礼。

方正的贺礼,大多数来源于仙鹤门,所以赵怜云的这一份,就显得相当独特。

从仙鹤门的情报渠道中,方正对赵怜云的了解也很全面。

“你也是天外之魔,只不过并不完整。”

“你如此主动联络,是想和我联手,对付方源么……”

赵怜云的贺礼中,夹杂着一份信笺,信中释放了赵怜云的善意,但并未多说什么。方正却从中品味出了赵怜云的来意。

赵怜云的贺礼是一份血玲珑的栽培之法,极其适合方正。她在信中更阐述了自己对蛊仙修行的心得,其中有一段话令方正很是在意。

“蛊仙修行,战斗乃是霸道,经营仙窍则是王道,两者相辅相成,缺一不可。战力不足,没有霸道,仙窍经营得再好,也只会被人欺辱,辛苦所得沦为他人的战利品。反之战力强盛,仙窍底蕴不足,便仿佛是熊熊燃烧的篝火,只能强盛一时,等到木柴烧尽,就是一片烟火而已。”

方正微微点头,暗中对自己道:“除了这段话,还有关于灾劫的叙述,都是真知灼见。可见赵怜云此人的确是诚意十足。可惜啊,我本想与之会面一次,但仙鹤门却不允许,真是遗憾!你说是不是这样,方源?”

说到这里,方正目光微微一滞。

在早年修行的时候,他有师父天鹤上人作伴,琅琊福地的时候,他和方源假意为伍。直到被救回中洲,这才孤身一人。

不知不觉间,方正已经习惯于有人陪伴。

“缺少了你,的确有点不大习惯呢……”方正摇头苦笑,一股难以言喻的孤独,弥漫在心中。

天庭。

紫薇仙子欣慰一笑:“方正已经成为蛊仙,但还需要将修为提升上去,才能对方源起到更大的克制作用。”

赵怜云的心思,紫薇仙子也心知肚明。毕竟马鸿运虽死,但他的魂魄却仍旧被方源拘拿在手中。

紫薇仙子眉头微锁:“真正的麻烦还在于方源本身。”

自从方源在光阴长河中突破了重围,就销声匿迹,再没有公开现身过。

紫薇仙子倒宁愿他四处抢掠,搜刮资源。如此一来,方源留下的线索越多,她就越能够推算出有价值的东西,甚至还能突破方源本身的智道防御。

在紫薇仙子看来,方源的确是战力强大,但更令人忌惮的是他的经营手腕。每当他隐忍一段时间之后,展现出来的实力往往都会有突飞勐进,进步的程度更常常令人咋舌震惊。

这正是他擅长经营的效果。

也只有坚厚的基础,才能支撑着他强大恐怖的战力。

“如今,魔尊幽魂已经渐渐支撑不住,搜魂出来的情报越来越多,也越加核心。”

“光阴长河中,宙道仙蛊屋已然修复,再次达到四座。之前的宙道大阵,还需要改良。”

“与此同时,也不能令方源太过舒服,是应该联络一下武庸了。”

数天之后。

狂风骤起。

方源勐一推掌,强劲的杀招就顺着敞开的门户,灌入到楼兰福地中去。

黑楼兰面对的灾劫顿时分崩瓦解,陡然溃散。

黑楼兰眼皮子微微一颤,旋即面色平静下来。她站在自家仙窍中,对门外的方源拱手道谢。

“不用客气。”方源扫视了一眼楼兰福地,楼兰福地比较贫瘠,和至尊仙窍完全不能相比。黑楼兰的资质虽高,但一直以来都未有好好地经营过仙窍。

但这点正合方源心意。

修行资源他可以提供给黑楼兰,黑楼兰在这方面依赖他,也是对黑楼兰本身的一种掌控。

事实上,影无邪、白凝冰等人的情况亦大致相同。

“快离开这里吧,我的智道手段只能防备一时,而且每次渡劫都会大大影响我的手段。”方源催促道。

“嗯。”黑楼兰点头,不敢大意,连忙收起福地,钻入至尊仙窍中。

方源利用定仙游,立即撤离,让全力赶来的武庸等人扑了一个空。

“该死!又让他逃了。”

“这种残存的气息,又是有人渡劫啊……”

南疆蛊仙们脸色不佳,有的愤怒,有的沉默,有的则浮现出一丝恐惧。

“这已经是第四起了……很明显方源正在帮助他的走狗们进行渡劫,迅速增长修为。要让他这么下去的话,可就不妙了。”

武庸面色如铁,眼中闪过一抹寒芒,他缓缓地道:“方源拥有定仙游,逃得极快,我们只有大肆铺陈烽火台,同时在宝黄天中齐心协力,遏制他的生意,削弱他的收益。久而久之,才能积累出优势。”

气海洞天中翠光一闪,方源凭空出现。

“暂时安全了。”

“开始从宝黄天上全面遏制我的生意了么……”方源冷笑。

他的确感受到了压力,但他最近侵吞了五相公共洞天,还有气海洞天,得到大量资源,就算遭受了压制,也无伤大局。

不过这明显是天庭、南疆联盟一起合作的结果,放任下去,时间一长,方源也承受不住。

他从南疆蛊仙俘虏身上,得到了许许多多的修行资源,但南疆正道也清楚这点,所以大部分的贸易都遭到了相当精准的阻击。

“看来得想想办法了。”方源眼中闪过思虑的光。(未完待续……)r<!--80txt.com-ouoou-->

------------

\end{this_body}

