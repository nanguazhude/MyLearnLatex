\newsection{新龙人之祖}    %第八百三十四节:新龙人之祖

\begin{this_body}

至尊仙窍。

长毛炼道大阵轰鸣作响,大量的白荔仙元灌注进去,形成冲天的火焰。

在熊熊的烈焰当中,龙人分身盘坐着,紧闭双目,面无表情,仿佛沉睡了一般。

而他的身躯却是开始消融,宛若钢铁刚开始要融化成了铁水的样子。

这当然是非常痛苦的事情,但是龙人分身的魂魄却是提前遁出,此刻就站在琅琊地灵的身边。

方源本体亲自操纵这座炼道大阵,而琅琊地灵,毛六等人辅佐。

方源调动了手头上几乎所有的炼道蛊仙班底,可谓拼尽全力!

烈焰灼烧当中,龙人分身的皮肉逐渐膨胀、起泡,很快就面目全非。就像是胶体,或者是浑圆的油珠,还未彻底流淌出来的样子。

“小心一点,把持住。这方面就交给你了,琅琊地灵。”方源本体主持着局面,此刻吩咐道。

琅琊地灵点点头。

方源并不需要将龙人分身彻底灼烧融化,若是这样做,等若是杀死了龙人分身。

方源需要的就是眼下的这种状态,龙人分身达到自身承受的极限,但只是处于崩溃的边缘。

要维持这个状态,需要蛊仙时刻操纵着烈焰,同时无比关注龙人分身的状态。

方源将这个工作交给琅琊地灵,后者已是转变成了白毛地灵,老成持重,炼道造诣极深,方源也有所不及。这个方面最不能大意马虎,交给琅琊地灵,方源很是放心。

接着,方源开始催动炼道大阵,凝聚出一丝丝的电芒。

这些电芒皆是漆黑之色,粗细如筷,滋滋作响。

电芒探入烈焰当中,如入无人之地,丝毫不受烈焰的影响。

电芒在方源的操纵下,轻轻刺入龙人分身肥硕的身躯当中,穿透已经模糊一团的血肉,搭在同样松软的骨骼上。

漆黑电芒将骨骼锁住,随后缓缓抽出,带出龙人分身的一节臂骨。

臂骨并不耐烈焰烤制,立即就有了融化的迹象,但这个时候毛六早已经等候在一侧,催动炼道大阵,发出一团团的气泡。

方源讯速地将臂骨投入到气泡当中。

气泡包裹着臂骨,迅速上升,很快就飞离火焰顶端。

那里早有毛民蛊仙等候良久,准备好了一池雷电浆水。臂骨被投入雷电浆水当中不断洗炼,这是第一步。

随后第二步,是另外的一位毛民蛊仙接手,将臂骨防止在翠绿的气流中冲刷。

第三步,又有他人负责。臂骨被炙烤、洗炼、冲刷后,体积大减,直接埋入一团惨白的骨泥中。

随后,炼道大阵催发宙道威能,加速骨泥的时间。

骨泥不断被臂骨吸收,越来越少。臂骨只是吸收了大半后,停止下来,被取出来,之后还有三道工序等着它。

这只是一份臂骨,龙人分身身上共有三千多块骨骼,都要一一施为。

如此,只是将骨骼炼好,龙人分身的血肉、脑海、五官、指甲、毛发都需要重新一一炼制。

工作量非常的繁重,单靠方源一人效率太低,所以他将手头上的炼道班子全数搬上。

好在重炼龙人分身这事情虽然繁重多杂,但是难度不高。难度最大的就是时刻关注,维持龙人分身的状态,这个已经被琅琊地灵负责了去,方源很是放心。

如此炼制,至尊仙窍中过了三天三夜,长毛炼道大阵方才缓缓平息下来。

骨骼都被塞了回去,皮毛、血肉、指甲、五官、脑海等等都重新洗炼了一遍,彻底免疫了龙人寂灭杀招。

龙公当初开创龙人延寿法,也是变化道、气道、人道的精义,这三个方面方源现在也是不差。

况且方源本身有着炼道底蕴,这方面龙公是万万不及的。

再加上智道方面的实力,只要给予方源充分的时间,方源肯定能破解得了龙人寂灭杀招。

没有什么杀招是不能被破解的。

但这必然需要一个过程,一个相对较长的时间,需要一次次不断地尝试。

方源获得龙宫之后,获得的海量的龙人试验的记载,这大大帮助了方源。

其实当初的吴帅离成功很近了,虽然没有真正破解龙人寂灭杀招,但也构思出了许多妙法,只是缺少尝试的时间。在龙人一族灭绝的最后时期,龙人蛊仙积极配合吴帅,在中洲各地隐秘落窍,充当一个个的试验场,分别尝试这些手段,也是播撒希望的火种。

方正之前遭遇到的龙人福地,便是其中之一。当然,更准确的说,不是原先的第一批福地了。而是龙人福地孕育出来的龙人蛊仙的仙窍落地,一代代的不断交替。

到底绵延了多少代,方源不清楚,也没有兴趣。不过他从这个珍贵的情报中,却是判断出了当初吴帅设想的种种方法中那个成功的方法。

有关方正的种种情况,是沈从声搜魂之后交给方源的。沈从声当然不知道这份情报的珍贵,等若是给方源指出了明路!

方源有能力,又有关键线索,终是重新改造了龙人。从今往后,龙人分身将彻底免疫龙人寂灭杀招。

当然,这只是龙公的龙人寂灭。若是方源自己开创一个,龙人分身仍旧要受到制裁。

毕竟全新的龙人分身,就是方源一手打造的,知根知底,一切都了然于胸。

做到这一步,还没有竟全功。

方源龙人分身的魂魄,适应的是原先的龙人身躯,因此也受到龙公手中的龙人寂灭杀招的影响。

所以,接下来还要改造龙人分身的魂魄。

魂魄就不需要动用炼道大阵了,方源手头上的魂道手段同样是丰富如海。

要解决魂魄上的麻烦,可比肉身容易得多。

正巧龙人分身的魂魄底蕴需要进一步提升,所以暂时方源的龙人分身就在销魂万胆大阵中修行。

魂魄底蕴乃是奴道的基石。

当前,龙宫奴役住四大龙将,已经是极限。接下来再奴役什么,非得借助龙宫之主的魂魄进行承担。

从这点上,方源也琢磨出了上一世龙灵为何认主白凝冰。

“当时恐怕是情况紧急,需要白凝冰的魂魄承担,从而龙宫方能奴役住孽龙帝藏生。”

“龙灵本身能操纵龙宫,类比地灵、天灵,都倾向于认主,维持生存。”

“当然,前提是要龙人身份,并且矢志振兴龙人一族,方能得到龙灵的承认。”

念及于此,方源微微一笑。

振兴龙人,也无什么不可。反正方源之前就开始大规模地豢养异人,一来领悟人道,二来经营仙窍,三来谋夺异人传承。

多一个龙人,少一个龙人,有什么区别?

没有区别!

当然,方源要栽培豢养的是自己打造出的新龙人,绝非之前的龙人种族。

从这个方面来讲,方源绝对是新龙人一族的老祖宗了!

天庭纠结于种族之分,一直以为根本。但对于方源而言,不值一哂。

就算真的龙人当兴,又如何?方源根本不在乎这个。

此生之愿,唯有永生!

振兴龙人一族能够帮助他达成这个目标,那就振兴它好了。若是将来阻碍,那就铲除掉好了。

多么简单!

若是转变成龙人能够永生,方源铁定立马转变,抛弃自己的人族身份。哪怕无数的人族依次为豪。

龙人分身不是没有姓名吗?不妨就叫吴帅好了。

方源根本无所谓。

龙人分身开始魂修,改造魂魄。

宙道分身仍旧在利用智慧光晕推算种种。

梦道分身还压着箱底。方源原本是想借助龙宫来帮助梦道分身渡劫,但得手之后却研究发现,龙宫并无梦道防御的杀招。

这就让方源的图谋落空了大半。

若是召来梦道灾劫,方源如何防范?到那时,逆流护身印都不好使。

倒是还有一条路子,就是利用如梦令仙蛊为核心,推算出相应的梦道杀招。

但这杀招层次至少八转!方源智道造诣是有的,但是梦道境界却始终还未成就大师。

这就难了。

消耗的时间、精力必定极多,方源想了想,还是决定按捺不动为妙。

绝不给天意对付自己的机会!nt

\end{this_body}

