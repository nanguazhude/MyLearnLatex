\newsection{搜刮广寒峰}    %第二百九十二节:搜刮广寒峰

\begin{this_body}

%1
一场切磋,夏家蛊仙失败,直接撤走。

%2
方源便收回变化,重现人形,回到的广寒峰。

%3
广寒峰上的那座凡蛊屋,也被收走了,只剩下一座冰雪山峰,寒气四溢。

%4
方源轻轻的叹了一口气。

%5
他的心境仿若古井无波,这些胜利并不值得他牢记夸耀。所谓的荣誉感,已经早就离他远去。

%6
“说起来此番轻松得胜,也是我占了便宜啊。”

%7
知己知彼,百战不殆。

%8
夏家两位蛊仙不知道方源的实力,只是自己估算。相反的,方源却知道很多他们的跟脚。不管是杀招手段,还是性情等等,都借助武家了解的很透彻。

%9
由此可见,蛊仙争锋,情报相当重要!

%10
其实方源和夏家两位蛊仙切磋,本质上仿佛武雨伯败给炎黄仙人。

%11
都是情报上的优势。

%12
武雨伯本身实力很强,但是招牌杀招,被人识破,因此一着不慎满盘皆输。

%13
方源是了解更多,步步为营,层层设计,最终让夏家蛊仙知难而退。

%14
“仙道杀招刚念,效果不错。不过此番施展开来,今后想要收到这样的良效,却是不易了。”

%15
夏家两位蛊仙失败归去,一定会总结经验教训,想方设法来尝试克制,甚至是破解仙道杀招刚念。

%16
破解一般是很难的,但找到针对克制的方法,却往往较为容易。

%17
“武遗海的这个身份也开始出名了。”

%18
“面前只有卜卦龟变化,和仙道杀招刚念,能够拿得出手。但使用次数多了,就会被他人越加针对,乃至破解。”

%19
“其实逆流护身印也是如此,北原一战之后,肯定会被天庭、十大古派、长生天、雪胡老祖等等分析思考。下一次使用,应当注意一些。尤其是面对这些人的时候。”

%20
方源暗自警惕。

%21
武雨伯的下场,就在眼前。

%22
仙道杀招建设不易,但一记强大的仙道杀招,更不能经久不变,而是需要与时俱进,不断地提升、改良,如此一来,别人克制针对,甚至破解你的杀招,你才能有效面对。

%23
而改良、提升仙道杀招,自然也是件困难的事情。

%24
除了灵感乍现之外,真正考较的还是蛊仙的境界。

%25
“而提升流派境界的最好去处,普天之下,非是梦境莫属。”方源的心思,又不禁飘到了那片超级梦境中去了。

%26
正是因为梦境的广泛出现,方源前世五百年时五域乱战,才会有各种各样的人物,风起云涌,接连出现,共同缔结成一个波澜壮阔的战乱大时代。

%27
“可惜我现在身为武遗海,才刚刚加入武家不久,强硬地要求调到超级蛊阵那边去,会非常惹人怀疑。”

%28
“这一次事情虽然了结,但不知道我何时才能重获超级梦境。还是先着手眼前,一切顺其自然吧。”

%29
方源心中一边这样想着,一边便落到了广寒峰上。

%30
这处盛产冰道资源的山峰,他还没有好好地搜寻过。

%31
驱逐了夏家蛊仙之后,方源迎来了这次难得的机会,他当然不会放过。

%32
广寒峰上,寒气逼人,白雪皑皑。

%33
在这积雪深处,生活着大量的虫豸,包含许多野生蛊虫,还有许多种耐寒的动植物,它们相互之间构成过来一套平衡的生态景观。

%34
“寒玉。”很快,方源就有了收获。

%35
一大批的寒玉,出现在他的侦查感知当中。

%36
这些寒玉是在积雪的最底下,隐藏在山石的深处。

%37
“寒玉只是五转蛊材,不过里面有一些寒玉髓,可以称得上六转仙材了。”

%38
方源没有二话,直接开采!

%39
轰隆隆。

%40
积雪飞溅,露出裸露的山石表面。

%41
很快,这些山石也四下飞溅,露出山体中的寒玉。

%42
这些寒玉,以白色为主,玉中夹杂丝丝青意。它们散发出一股股强烈的寒气,不过自然对方源无碍。

%43
方源将这片地方的寒玉,都取出来。

%44
其中寒玉髓当然没有放过,只是没有抽取出来,仍旧留在这块巨大寒玉的中心。

%45
接下来,方源的足迹遍布整个广寒峰。

%46
一块块巨大的寒玉,被他挖出来。最小个头的寒玉块也仿佛大象体积,个头大的好似楼船。

%47
几乎每一块寒玉当中,都有寒玉髓。

%48
有的个头大点的寒玉髓,甚至是千年玉髓,这就是七转蛊材了。

%49
这些寒玉,并非没有被发现,而是一直被武家故意保留在这里,不愿削减了广寒峰的底蕴。正是这些寒玉,构成了广寒峰的根基,没有它们,就没有广寒峰如此独特而且优秀的资源点了。

%50
方源将大半的寒玉,都收入囊中。

%51
不仅如此,他还收刮了不少野生蛊虫。

%52
野生仙蛊是不用想了。

%53
一般而言,数百年甚至上千年,广寒峰中才有可能产出一只六转野生仙蛊。

%54
若是真有野生仙蛊,夏家蛊仙也不会如此轻易就承认失败。

%55
这些蛊虫绝大部分,都是冰道野蛊,也有一些土道、水道的蛊虫。冰道野蛊当中,数量最多的是一种寒潮蛊。

%56
这种蛊虫成群结队,往往一段时间,要在广寒峰周围游走。因此会带动一股强烈的寒潮,向广寒峰周边的峰峦蔓延过去。

%57
这也正是广寒峰的山名由因。

%58
搜刮结束了,方源飞到半空中,最后看了这座广寒峰一眼。

%59
原本白雪皑皑,风景如画的广寒峰,此刻却是已经成了一个漫山坑洞,高低不平,琐碎的山石漫山遍野的地方。

%60
不说大煞风景吧,整个广寒峰的资产直接锐减了一大半,并且因为寒玉的大量缺失,底蕴也大大损耗了许多。

%61
“如果我用拔山仙蛊,说不定能将这座山峰直接拔走,转移到自己的至尊福地中去。”

%62
“可惜此法不成。我明面上的身份,是武遗海,是武家中人,这次贪图便宜,克扣油水,已经达到了正道潜规则的最大极限,不能在继续下去。若真的拔山,暴露真实身份的可能不提,至少武家那边肯定说不过去的。”

%63
方源暗自叹息一声,再不留恋此地,身形如电,飞入高空。

%64
很快,他的身影就消失在云层之中。

%65
数天之后,方源回到武家大本营,再一次见到了武庸。

%66
“这一次,多亏了有我弟出手,替家族护住了广寒峰,更震慑了宵小之辈。”武庸对方源频频点头,面露赞赏之色。

%67
“兄长过誉,在下也是侥幸罢了。”方源谦虚道。

%68
“武罚长老,可清算好了没有?我弟的这一次功劳,可不是简简单单,应当从大局考虑。”武庸这时转头,对另外一位武家蛊仙道。

%69
这位武家蛊仙面貌普通,但权势却不普通。

%70
他是武庸的心腹,为家族各位蛊仙计算功过的赏罚。

%71
武罚微微皱眉道:“武遗海大人的勇武,我们已经见识到了。经此一战,大人的名号必将在南疆蛊仙界流传。不过若要论功行赏,一来,武遗海大人并未彻底解决此事,而是定下五年之约。二来,广寒峰却是遭受到了恶性开采,底蕴大损,价值大打折扣……”

%72
“哦?广寒峰居然出事了?”武庸诧异,目光又转向方源。

%73
方源唉声叹气:“我也没想到夏家两位蛊仙,居然是这样的人!”

%74
武庸微微扬起眉头:“兄弟你可是亲眼看到,夏家蛊仙开采了广寒峰吗?”

%75
“这倒没有,只是我的猜测罢了。兴许还有其他散修蛊仙?”方源回应得滴水不漏。

%76
武庸点点头,沉默了一下:“既然如此,那就请武罚长老定夺吧。”

%77
武罚长老便继续算了算,然后当堂交给方源一份清单。

%78
备注:10点还有第二更。

\end{this_body}

