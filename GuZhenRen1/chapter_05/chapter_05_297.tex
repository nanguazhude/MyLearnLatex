\newsection{斗茶}    %第二百九十七节:斗茶

\begin{this_body}

%1
乔丝柳只做了一份柳旋茶,给了方源。

%2
这点做法,顿时让这份柳旋茶的意义变得不同。

%3
方源面色微微一变,流露出一抹受宠若惊的喜色,心中却是一片清明,平静无波。

%4
“可叹。任你多美的红颜,不得永生,终究是枯骨一堆。”

%5
“红颜即白骨,世人却往往留念忘我。”

%6
“不过这个天露仙子,倒真是乔丝柳的好闺蜜,配合得十分出色。”

%7
唯一做的一份茶,被丝柳仙子当众送给了方源,罗木子、轮飞还想着品茗,结果却是这个答案。

%8
可想而知,他们此时此刻的脸色神情,是有多么僵硬。

%9
乔丝柳浅笑:“柳旋茶只是我独创之物,一点心意而已。这一次我为大家带来了奴娇茶,请。”

%10
她一挥丝缕长袖,顿时石桌上现出五杯茶水。

%11
这茶水又和之前的柳旋茶不同,不是杯盏,而是白瓷小盘。

%12
在这小盘的中央,有一颗拳头大小的露珠。

%13
这露珠清润如玉,一阵晚风吹来,表面微颤,吹弹可破。

%14
“奴娇茶,可是乔家名传南疆的茶啊,想不到今天我也能品尝一口。”罗木子自己给自己打圆场,实际上他的目光却是仍旧盯着方源面前的那杯茶上。

%15
轮飞放在石桌底下的双手,已经暗自握紧成拳,他咬了咬,端起白瓷小盘,请嘬一口,顿时奴娇茶便吸入腹中。

%16
“好茶。”他开口道,神情还是有些僵硬。

%17
实际上,奴娇茶的确是比柳旋茶高出了几个档次,后者不过是乔丝柳自己开创,前者却是乔家的招牌。一个超级势力的招牌茶,自然是比乔丝柳一位蛊仙的独创,要优秀得多。

%18
不过,落在罗木子、轮飞的心中,他们宁愿放弃百八十杯的奴娇茶,喝上一口柳旋茶。

%19
“奴娇茶还是这么清爽,令人回味无穷。丝柳,你摆出这茶,叫我的流彩茶如何拿得出手?”天露仙子笑着道。

%20
乔丝柳却和她这个闺蜜毫不客气:“那你就别拿出来了,说实在话,你的流彩茶我喝都喝腻了。我倒是很期待盛六仙友的花中醉。听闻这道茶水,乃是他当初与你一见钟情之时,灵感迸发,开创出来。不知道我们今天是否有缘,能够品尝一番?”

%21
“惭愧,惭愧。”盛六摸了摸鼻子,苦笑,“我的花中醉,只是涂鸦小作,难登大雅之堂。”

%22
天露仙子亦掩嘴轻笑:“丝柳,你这是想让我家六郎难堪呀,这可不行!花中醉只能我喝,外人是想也别想了。”

%23
说着这话,她眉宇间尽显得意和风情。

%24
她身旁的蛊仙盛六,深情地注视着天露仙子,两人在石桌下的手,已经握在一起。

%25
乔丝柳幽幽一叹:“唉,也不知道何年何月,我才能遇到一个人,为我独创一道茶水呢?”

%26
她轻轻呢喃,疑问中显现迷惘之情。

%27
此刻,月光如水,透入凉亭之中。

%28
月下美人,轻声低叹,当真是我见犹怜。

%29
罗木子、轮飞顿时心头一热,罗木子当即站起身来:“丝柳仙子,在下有一杯茶,名唤九回香。正是在下独创,值此良辰美景,献于仙子。”

%30
轮飞不甘落后,也道:“我亦有准备,此乃两仪茶,阴阳之间,泾渭分明,请仙子品味。”

%31
这两位蛊仙都是只献茶给乔丝柳,一如乔丝柳之前,献茶给方源。

%32
乔丝柳分别喝了一口,笑着道:“九回香名字取得贴切,茶水入喉,香味萦绕舌尖,仔细品味,竟有九次浓郁之时,十分奇特。”

%33
“哈哈,不敢当仙子夸赞。”罗木子大笑,一扫之前的颓势,笑得非常开心。

%34
乔丝柳又对轮飞道:“两仪茶,虽不是独创,但丝柳听闻,此茶功夫有三道层次。第一层,是混沌一片,阴阳混淆。第二层,便是阴阳分明。第三层,则是阴中有阳,阳中有阴,阴阳流转。轮飞仙友能将此茶做到第二层,可见功夫,整个南疆,恐怕不足十人,不愧是食道蛊仙。”

%35
这个世界上的茶酒佳肴等等,当然不普通。

%36
炼茶的方式,千奇百怪。做茶,更绝非将滚烫的开水,灌在装有茶叶的杯盏中那么简单肤浅。

%37
像今夜众仙谈论提及的茶水:柳旋茶、奴娇茶、花中醉、九回香、两仪茶,都可以看做是一道残缺的蛊方。

%38
而当这个蛊方补全之后,蛊仙再做茶时,就有可能炼成食道蛊虫。

%39
食道亦是当今的诸多流派之一,只是从未盛行过。虽然非常重要,也要大兴的根基,可惜因为历史缘由,留在世间的传承,一直都很稀少。

%40
“这轮飞竟然是主修食道的蛊仙?”方源盯着轮飞打量了一眼,心中有些诧异。

%41
“这轮飞什么来路?我身上仙蛊众多,若是能得他的食道传承,或许对我有很大帮助。”方源心中顿生歹意。

%42
人如伤虎意,虎有害人心。

%43
方源有些心动了。

%44
他目前的实力,已经达到了八转之下数一数二的层次。当他催动逆流护身印后,虽然攻伐方面还是弱了一筹,但已然可和凤九歌并驾齐驱。

%45
“先查查轮飞的跟脚,看看能不能下手。”

%46
“若真下手,武遗海的身份肯定不能受到牵连!”

%47
方源以武遗海的身份,肯定不能明目张胆地杀害轮飞。因为他是正道蛊仙,不能行魔道之事。

%48
一个超级势力虽然比一位散仙的实力,要雄厚得多。但往往这些势力,不轻言得罪任何一位蛊仙。

%49
武家同样如此。

%50
超级势力家大业大,一位蛊仙若是打杀不死,四处游击,破坏超级势力的各处资源,也会让超级势力分外头疼。

%51
武庸在螺母山一事上,采纳方源的意见,向驱山老怪做出了一些让步,这当中也有这等考量。

%52
“嗯……还是先打听一下,看看轮飞身上的传承价值。若是价值不高,那就算了。”

%53
“如果价值很高,真正动手,还得注意,绝不能让他魂魄自毁。”

%54
仙蛊是不要想了,绝对不可能得手的。

%55
但是方源可以暗杀了轮飞,俘虏他的魂魄,再通过搜魂的手段,获取食道传承。

%56
方源其实一直对食道传承,有着期望。

%57
可惜机缘不足,徒呼奈何。

%58
既然没有机缘,那就自己动手,自己抢别人的。

%59
方源可没有什么罪恶感!

%60
很多蛊仙,讲究人不犯我我不犯人。

%61
方源是人就算不犯我,我也要犯人。

%62
“这杯茶我连做了七天七夜,能得仙子称赞,一切都是值得的。”轮飞语气都有些颤抖,他脸上流露出激动之色。

%63
随后,他的目光瞟向方源,带着挑衅意味,道:“不知道武遗海大人,有什么茶展现一下呢?”

%64
他丝毫不知道,方源早就在肚子里,酝酿着谋杀他的念头。

%65
“东海富庶,五域第一。武遗海仙友曾经居于东海多年,拿出一两份茶水来,必定能冠盖其他,夺得第一。”罗木子重新坐下来,热情洋溢。

%66
方源笑了笑。

%67
这两人都是乔丝柳的追求者,意识到方源的“强大”,已经开始默契地联合起来,要给方源一个难堪。

%68
他们口中称赞,就是把方源抬得高高,让他难以应付。

%69
乔丝柳微微皱起眉头。

%70
罗木子、轮飞的心思,有些阴险。武遗海若拿不出来,自然尴尬。若是拿得出来,名声就不好听了。

%71
为什么?

%72
因为这场赏月会,是由乔丝柳主办。乔家的奴娇茶,是主。其他蛊仙的茶,是客。

%73
客不压主,这是赏月会的潜规则。客大欺主,乔家兴许不在意,但是武遗海的名声却是有损。

%74
混正道的,在乎的就是名声!

%75
方源却是没有犹豫。

%76
他很干脆地取出了五份茶水,伸手示意:“诸位请喝。”

%77
天露仙子眼前一亮,第一个端起杯盏:“我很好奇,武遗海大人的茶水究竟是如何?”

%78
但喝下茶水的第一人,却不是她,而是轮飞。

%79
他显得有些急切。

%80
这可是一次打击情敌的上佳机会,他身为食道蛊仙,又在他的擅长方面,怎可能不把握住?

%81
但他喝了一口之后,顿时眉头紧皱,连忙吐出来:“呸呸呸,这是什么破茶?”

%82
“难喝,太难喝了。”罗木子也喝了一口,直接将杯盏放下,“这是我这辈子喝过的最难喝的茶,恐怕是连凡茶都有不如。”

%83
他说话非常不客气,也不可能对一个情敌客气。

%84
两人紧紧抓住机会,大落方源的面子。

%85
天露仙子想帮衬一下,但喝下之后,面对这茶,无奈苦笑:“这茶咸涩,仿佛海水。”

%86
方源淡淡一笑,竟然承认道:“这就是海水。”

%87
“什么?”

%88
“海水你竟然拿出来?你置丝柳仙子于何地!”

%89
罗木子、轮飞纷纷发难。

%90
方源脸上笑容消失,变得肃穆起来:“我不喜欢喝茶,若是勉强算起来,这海水便是我的茶了。”

%91
“你们不太清楚。”

%92
“我在东海隐修,孤寂一人,孑然一身,灾劫压力时刻笼罩心头。”

%93
“我为了防备懈怠和懒惰,每天早起时分,便舀一碗咸涩的海水喝下,以此来警惕自己,要继续奋发坚持,不得有丝毫懈怠。”

%94
凉亭中顿时一片沉默。

%95
乔丝柳在沉默中,缓缓地端起杯盏,喝了一口后,再次缓缓放下。

%96
她笑起来,红唇白齿,一时竟把妙丽的月光都比了下去:“这是我此生喝过的最特别的茶,谢谢你,遗海。这茶,很不错,叫我心动不已。”

%97
罗木子、轮飞见心上人竟是这般态度,顿时无语至极。

%98
备注:可能还有一更,也可能没有。因为下一章非常的难写,不亚于写人祖传。大家今天就不要等了。

\end{this_body}

