\newsection{兄去弟继掌权柄}    %第三百二十二节:兄去弟继掌权柄

\begin{this_body}

宗族祠堂。

坐落在武仪山上的宗族祠堂,是一座凡蛊屋。运用了大量的五转凡蛊,虽然比不上仙蛊屋的威能,但对整个武家的重要意义,却是非同凡响。

基本上,不管是超级势力,还是普通家族,都会建设一座宗族祠堂。里面存放着命牌蛊或者魂灯蛊,亦或者血绳蛊。

前两者比较普遍,血绳蛊稀少一些,毕竟牵扯到血道。

就算是中洲的大小门派,也有类似于宗族祠堂的建筑设施,同样是存放命牌蛊、魂灯蛊等。

此时此刻,武八重站在宗族祠堂当中,目光直愣愣地盯着桌台上的蛊虫碎片。

命牌蛊破碎之后,宛若块块木片屑子,而魂灯蛊毁灭后,则是一块碎裂的瓢虫,再也发不出晦暗的淡蓝烛光。

武八重虽然年岁颇大,但体格健壮如熊,他双鬓花白,鼻翼宽粗,浓眉虎目,为人稳重,敢于担当。

他是武家的太上二长老,经历过大风大浪,平素稳如泰山,但此刻,他的额头尽是冷汗,呼吸紊乱,一直坚定的目光竟是散乱一团。

“武庸大人死了?!”武八重死死的盯着两只凡蛊碎片,他站在这里已经有一会功夫了,但此刻他的脑海中还是一片混乱。

“怎么会这样?堂堂的八转存在,我武家的顶梁柱,居然就这样陨落了?!”武八重不敢想象,难以置信。

“武庸大人,究竟有没有陨落???”他不断地问自己。

毕竟,命牌蛊、魂灯蛊只是凡蛊而已,动用一些手段,不难干涉。

但是武庸此刻处于失联状态,不管武家方面如何联系他,都是石沉大海。这也是不争的事实!

不管武庸有没有死,总之他的情况肯定不妙。

“能够让一位八转蛊仙,遭遇如此,究竟是谁下的狠手?”

“魔道、散修中的八转存在吗?”

“还是正道的超级家族,会是他们在暗中联手,一起对付我武家吗?”

武八重心中不断地猜测。

越是思考,他的目光就越加散乱。

武独秀逝去,武庸失踪,偌大的武家,南疆第一的超级家族,居然没有一位八转存在镇守。

这种情况非常的严重,武家已处于风雨飘摇的地步,更叫人暗感恐惧的是,幕后黑手还未显露!

震惊的情绪缓缓平复下来,迷茫和恐惧又盖压在武八重的心头。(www.MianHuaTang.cc 棉花糖小说)

平时头顶上有武独秀、武庸主持大局时,他还不觉得什么。

现在两位八转尽失,武八重这才发现,他自以为宽厚的肩膀,并不能挑起武家的大担。

他在武独秀的时代,就是太上二家老,武庸上位,他继续辅佐。论资格,他可以算是整个南疆正道蛊仙当中,年龄最大,资格最高的。

但是没有用。

他修为只是七转,战力并不出类拔萃,若和树翁巴德对战,肯定不是巴德的对手。甚至就连武家当中,也有不少七转蛊仙,能够和他交手不分胜负。

不过,他却是事发之后,第一个赶回来的武家高层。

这一点,让他占据了先机。

“怎么办呢?”

“武庸大人就算没有死,但他本人失联,命牌、魂道两蛊碎裂,这个消息一旦传出去,肯定会引发轩然大波,让人心剧烈变动。”

“我是不是该封锁消息?”

武八重很快摇摇头。

且不说,这个惊人的消息,已经被镇守这里的武家六转蛊仙,通告了大多数的武家蛊仙。

武庸若是灭亡,埋伏武家的幕后黑手,也绝对不会放过这个打击武家士气的绝佳机会。

因此就算武八重封锁消息,也封锁不住。

“那么当务之急,是着力寻找失踪的武庸大人,还是收缩防线,召集在外的太上家老们,防守大本营呢?”

武八重皱起眉头。

这两个选择,都有危险。

寻找武庸,危机重重,能够让武庸失踪,让他的命牌、魂道两蛊都破碎,这幕后黑手的力量一定极其强大。武家派遣出七转蛊仙去追踪寻找,如同羊入虎口。

而收缩防线,召回一干太上家老,也是家政巨大变动,会让大量的资源点失守。周围的超级势力若是趁机侵吞,那么武家的损失必定极为惨重。

该怎么办?

武八重心中犹豫,陷入两难境地。

好一会儿功夫之后,他才舔了舔干燥的嘴唇,决定这两种举措都暂缓执行,先以太上二家老的身份,召集武家蛊仙们商讨。商讨出结果来,再执行下去。

武庸失踪,无法联络,以武家为中心的一场巨大风暴,开始成形。

南疆,超级蛊阵。

方源盘坐在床榻上,一边闭目修行,一边分心两用,听取白兔姑娘的汇报。

白兔姑娘继承了白兔真传。

这道真传在南疆非常有名,继承者须得心性单纯。而在接下来的修行过程中,更要保持这种单纯心性。否则大部分的真传手段就无法动用,强行催动,还会遭受反噬。

正因如此,白兔姑娘得到天然信任,在散修、魔仙当中有着较好人缘。

武安当初正是看中此点,说服白兔姑娘,充当中间人,主持仙缘生意。

每隔一段时间,白兔姑娘都要进入超级蛊阵,为方源汇报这段时间以来的仙缘生意的情况。

方源尽管从未在仙缘生意中获取分毫利益,但是听取生意情况,可以帮助他洞悉超级梦境这边的环境变化。

白兔姑娘声音轻缓,徐徐吐音,声调仿佛柔水,她一边汇报,一边用温柔的眼神,瞧着方源的脸目。

方源此时是武遗海的容颜,面容英武,鼻梁挺拔,鼻翼宽大,男性雄武之气展露无疑。

白兔姑娘早已为之心折,故意拖延语速,分外珍惜这段她和方源独处的时光。

可惜方源每次接见她,话语都少得可怜,前后两句招呼,一句是招呼白兔姑娘坐下汇报,第二句则是让她离开。

不过就算如此,白兔姑娘也非常的满足。

她深知武遗海身份高贵,自觉自己出身低微,配不上方源。能够每隔一段时间,就这样和方源静静相处,自己就很开心,感到非常的幸福了。

就在这时,白兔姑娘忽然声音一顿。

因为她看到方源面色微变。

这种情况非常罕见,白兔姑娘立即揪心起来,暗想武遗海是怎么了?难道是修行遇到了麻烦?

方源缓缓睁开双眼,心念一动,打开此处宫殿大门。

“遗海,都这个时候了,你居然还有闲情逸致,和人**吗?”一位女仙风风火火,闯入殿中。

白兔姑娘转头一看,只见来者二八芳龄的少女模样,身着翠绿宫裳,裙袖飘飘。浑身肌肤细嫩似雪,身姿窈窕,行进间如扶风弱柳。

她的容颜精致美妙,尤其是一双妙目,最为惹人注目。在浓密的睫毛的遮盖下,好似泛着波纹的清澈湖水。

“她是乔家的丝柳仙子!”白兔姑娘认出来人,顿时心中一阵慌乱,目光立即低垂下去。

她知道这人和方源的关系。

事实上,前一阵子,方源和乔丝柳的绯闻,传得纷纷扬扬。

“丝柳仙子好美,果然不愧是当今南疆的三位仙子之一。”

“也只有这样的人物,才配得上武遗海大人吧。”

白兔姑娘心中暗道,一阵酸楚之情,在心头酝酿。

乔丝柳瞟了一眼低头垂眉的白兔姑娘,状似无意,实则心中高度重视。

乔家意欲施展美人计,让武遗海成为乔家女婿,乔丝柳怎可能不关注方源的情报。

而在方源的情报当中,白兔姑娘早就为乔丝柳熟知。

这一次,是乔丝柳首次和白兔姑娘碰面。

只见白兔姑娘好似粉雕玉砌,双眼如红宝石,一对兔耳朵毛茸茸,耸搭下来,眉目低垂,我见犹怜。虽然细看之下,容颜不如她乔丝柳,但男人都通常喜好这种可爱丰满,让人怜惜的尤物。

“不过现在却不是和她计较的时候。”乔丝柳心中冷哼一声,越过白兔,快步来到方源面前。

方源已睁开双眼,从床榻上起来,站在原地,嘴角含笑地望着乔丝柳:“仙子忽然造访,实在叫人惊喜。”

白兔姑娘远远站在一边,痴痴地望着方源,心想:“武遗海又笑了,我好久都没有见到他笑过。他笑起来的样子,就是是春天里的太阳。若是他能对我这么笑,该有多好!”

乔丝柳打量方源上下,随即打断方源的客套话:“看来你真不知道!唉接下来我说的话,可就只有惊,没有喜了。”

方源顿时面色一变:“发生了什么事?”

乔丝柳却没有直接回答,而是微微转身,对白兔姑娘道:“这位小姑娘,你先退下,我与你家大人有要事商议。”

“啊?”白兔姑娘微微一愣。

方源摆手:“白兔儿,你下去。”

白兔姑娘不敢违逆方源命令,连忙应声,推离这间大殿。

殿门合拢,乔丝柳道:“武庸失踪,他的命牌蛊、魂灯蛊都已破碎,这个消息外界暂时不知,但现在武家已然大乱。”

“什么?!”方源狠狠地震动了一下。

这个消息太惊人了。

命牌蛊、魂灯蛊破碎,意味着武庸可能已经陨落。

他是武家的大树,此刻若是真倒了,武家偌大家业,如何能守得住?

乔丝柳继续道:“遗海,我此次前来,就是助你一臂之力。你是武庸的弟弟,血脉关系最亲近的人,修为又足够。武庸一去,你当接任武家太上大长老的职位!从今以后,你将执掌武家的权柄。”未完待续。

\end{this_body}

