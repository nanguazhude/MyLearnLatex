\newsection{九九连环不绝阵}    %第七百一十三节:九九连环不绝阵

\begin{this_body}

因为药皇等人的提前告知,武庸对方源的到来,也早有心理准备。

他是枭雄本性,尽管和方源有着深仇大恨,但此刻天庭才是大敌。对付天庭,阻止对方彻底修复宿命蛊才符合武庸的最大利益,所以为了这利益,武庸可以暂且抛弃情仇,和方源联手。

“武庸仙友,果然是深明大义。”药皇点头赞道。

随即,池曲由等人也齐声附和:“不只是武庸盟主大人,我南疆各大超级势力,也愿意放弃前嫌,和方源共抗天庭。”

方源为了一己私利,把许多南疆蛊仙俘虏都撕了票,但此刻南联诸仙为了大局出发,做出了相同的选择。

这是他们的理智,但更多的是危险的局势,和天庭的强大!

冰塞川不仅将天庭的实际战况,告知了药皇、方源,也辗转到武庸等人的耳中。这令南联诸仙对于战局,有着相当清晰的认知。

天庭战场、帝君城战场都已经不抱希望,唯一的胜机就在于这里。

只有争分夺秒,尽快地摧毁大阵,阻止最后一波成功道痕输送到天庭中去,那么天庭势必就要再举办一次中洲炼蛊大会了。

“纯梦求真体,去爆了这大阵。”与此同时,在阵外,方源已经驱使数位纯梦求真体扑上前去。

当今时代,梦道不显,只是出现了许多苗头,还未真正发展起来。

对于梦境的侵蚀,几乎所有的仙阵,不管什么层次,都无法阻挡。除非仙级蛊阵中蕴藏梦道蛊虫,有相应的梦道手段才可抵御梦境的侵蚀。

但就在纯梦求真体接近大阵的时候,异变发生了。

仙道杀招碎梦!

仙道杀招纯梦求真变!

两记杀招迭出,方源的纯梦求真体第一时间碎裂开来,不成人形,当场四处弥漫。

随后,这些破碎蔓延的梦境,又汇拢凝聚,形成新的纯梦求真体,反扑向方源。

方源瞳眸一缩,暗暗吃惊。

他万万没有料到,天庭方面居然掌握着这样的手段。

尽管他早已预料过会有这么一刻,但没想到居然这么快就来临了。

但旋即,他心中失落消散无踪,再次恢复了冰一般的冷静。

脑海中思绪起伏,念头翻腾,急速思考着应对之法。

“居然已经可以反制我了么,并且也掌握了纯梦求真的变化?是因为魔尊幽魂被俘虏,记忆被搜刮出来的原因?”方源被数位纯梦求真体逼迫,只得后退疾飞。

他手中虽有不少强大杀招,但是对付纯梦求真体却是欠佳得很。

这是一步鲜,吃遍天,当初的监天塔,乃至龙公、紫山真君都对梦境避之不及。

不过好在,方源并非没有手段。

至尊仙窍中,宙道分身就处在智慧光晕之下,为他不断地推演。

方源虽没有碎梦杀招来对付纯梦求真体,但关于杀招纯梦求真变,他可是非常熟悉的。

即便是天庭,想要在这些年里参悟出纯梦求真变,可能性还是很小的。所以,从魔尊幽魂处搜刮出记忆的可能性极大。

正因为如此,方源完全可以从自己已经掌握的纯梦求真杀招,来逆反过去,借助智慧光晕的帮助,构思出克制的杀招来。

“宙道分身的推演,需要一段时间。”方源一边思考对策,一边转折方向飞上高空。

数位纯梦求真体对他紧追不舍,但也有一部分杀向无限风和其他蛊仙。

纯梦求真体自爆,无限风大受干扰,威力下降得很快。

凤仙太子等人也是只能暂避锋芒。

方源这时联络上了武庸,开头第一句话就让武庸一震:“北原一方的凤仙太子乃是天庭的内奸!”

武庸一边对大阵狂轰滥炸,一边回道:“证据呢?”

“我的话就是证据。这种情况下,我更不愿意这种情况发生!”方源当即回应,“我手中梦境不足,需要去往南疆搜刮梦境,制造纯梦求真体对战。你快快通知那边驻守的蛊仙接应我。”

方源和天庭势不两立,他指证凤仙太子,虽然没有任何证据,但的确如他所言,他的话就是证据。

毕竟,春秋蝉在他的手中的事情,天下皆知。

这个秘密暴露,让方源曾经大吃苦头,但在此刻却换来了武庸的信任。

关键时刻,武庸竟没有犹豫,既然已经和方源合作,那就不妨更深入一些,天庭才是真正的大敌!

“好,我来传信,你现在就可过去了。”这一刻,武庸的心情是很复杂的。他从未想过,自己会公然有一天下令,对方源放行!

就在这时,武庸的耳畔传来池曲由的声音:“好,我已参悟了此阵,快快攻击此处,这才是真正的阵眼!”

武庸大喜,将方源抛之脑后,连忙下达命令,攻击池曲由所指方位,顺利破坏了阵眼,顿时让这座大阵运转凝滞了许多。

玄极子孙名录投来惊讶和敬佩的目光。

进阵一来,池曲由屡屡超越他,许多次提前参悟出大阵奥妙,让他不得不佩服。

“就按照池曲由大人的安排来,我们也协助南疆盟友吧。”玄极子放下较量心思,对身边的药皇大人建议道。

药皇皱了皱眉头,点头应允。

方源轻易甩开追兵,这些纯梦求真体可远不如厉煌和清夜。

利用定仙游杀招,方源折返南疆和中洲,获得大批梦境。有了这些梦境,他再次凝造出纯梦求真体,直扑天庭大阵。

眼下,对付敌人的纯梦求真体还只是其次,最重要的还是破坏大阵,阻止天庭彻底修复宿命蛊。

方源从未忘记自己真正的目的。

但哪知,但凡接近大阵的纯梦求真体,都无故破碎,继而又被转化成地方的纯梦求真体,反扑方源。

并且,纯梦求真体之间的交流,方源这边也是败多胜少。

种种情况终于让方源意识到:“看来天庭对纯梦求真体已经有所改良!”

就在这时,武庸传信过来:“方源,继续让纯梦求真体扑击大阵,我方已乘机发现了多处大阵破绽。”

方源心中旋即恍然:“看来天庭方面,并未真正的将梦道手段和大阵融合起来。因此每次施展杀招来对付我的纯梦求真体,都会在同时极大地干扰大阵的运转,从而导致破绽。”

这些破绽,落到池曲由、玄极子的眼中,不要太过明显。武庸等人趁机进攻,一时间战果斐然。

但如此一来,方源的梦境会陆续为天庭所用,敌人的纯梦求真体会越来越多。

“看来得真正地破解对方的纯梦求真体了!”方源心头微沉,这是他无法逃避,必须迅速解决的难题。

不是他不想入阵去亲自破阵,而是阵内已经有池曲由、武庸等人,梦道手段不好用的情况下,阵外无疑更加需要他。

方源开始和这些纯梦求真体周旋。

与此同时,他传信给其他人:“眼下还有一个关键,那就是天庭的人道杀招,谁能处理掉这散漫四周的白色光点?”

无人应承。

最终,还是翼浩方气馁地答道:“人道的特长在于加持,因为人道道痕不会干扰到其他流派的道痕,更不会被干扰。除非是动用人道手段,至少得有八转层次,才有解决的可能。”

翼浩方专修变化道,也特别擅长加持别人。这是因为他有一记招牌杀招,名为人形变。他的这番大实话,让方源只好打消了这方面的期待。

人中豪杰无法破解,凤仙太子也是一大隐患。

但方源和武庸等人,不愿轻易揭破此人身份。要围杀一个八转,非常困难。

反倒不如现在装作不知道,全力破解大阵,阻止天庭大计才是最重要的。

时间推移,方源的推算终于有了进展。

他冒了一些险,成功地活着了一个敌方的纯梦求真体,加以研究。

“这竟是完整的纯梦求真体!已经被改良彻底,可以真正称得上是第十一绝体了。”方源大感收获巨大。

他迅速改良纯梦求真变,掌握了天庭的技术。

本方的纯梦求真体立即得到改良,伴随着武庸、药皇等人的奋发,一座座大阵被攻破。

“怎么还有?”药皇紧皱眉头。

一座大阵之后是另外一座大阵,仿佛天庭在此布置的大阵无穷无尽。

此番情景,让武庸等人也不由心生气馁之色。

“有古怪,天庭怎会在这里布置这么多的仙道大阵?难道是……”玄极子迟疑。

池曲由沉声道:“我明白了,这是九九连环不绝阵!”

“九九连环不绝阵,这个名字有点耳熟……”武庸心中大感不妙。

池曲由叹息一声:“传闻中,星宿仙尊算到今后天庭之难,布置了九九连环不绝阵。无极魔尊进攻天庭,进入此阵,连破百阵后,方才得脱。”

“此阵其实只有一座超级复合大阵,可由九座子阵任意组合而成。关键的难处在于,我们破开一阵,天庭就可重新在最中央重新布置一阵。只要我们破阵的速度,比不上他们搭建布置的速度,那么这座大阵就是无穷无尽的了。”

池曲由一番话,说得场中诸仙都像是被一桶冷水兜头浇过。

这可如何是好?<!--80txt.com-ouoou-->

\end{this_body}

