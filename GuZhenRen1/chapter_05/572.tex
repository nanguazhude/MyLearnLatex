\newsection{天下第九太宇寺}    %第五百七十四节:天下第九太宇寺

\begin{this_body}

南疆。

方源攻破云竹山脉,大肆收刮,但他亦明白此处资源对于池家的重要意义,所以早有准备,应付池家的反击。

仙道杀招——大盗鬼手!

隐藏于角落中的大盗鬼手,忽然飞出,抓住池家赶来的仙蛊屋。

仙蛊屋中池家群仙顿时一惊。

但旋即,仙蛊屋表面忽然升腾起绚烂的金芒,仙蛊屋的表面荡起涟漪,形成一个漩涡,竟将大盗鬼手吸摄进去。

下一刻,漩涡消散,涟漪不存,方源闷哼一声,瞬间失去了对大盗鬼手的感应。

他眼中精芒一闪即逝:“不愧是排位前十的仙蛊屋太宇寺!”

池家拥有两座仙蛊屋,均都不类凡俗,当中太宇寺更强一筹,能匹敌八转蛊仙!

从外看去,它仿佛大理石打造,纯白一片,拱形圆顶,绘有金边,仿佛在散发阳光,华丽绚烂,带给人温暖和希望的感觉。

方源五百年前世,五域乱战时期,各大超级势力相互交锋,天下纷乱,四处狼烟,仙蛊屋频繁出动,相互争斗。因此战绩丰富,使得世人对仙蛊屋之间的强弱,有了较为明确的认知,列出仙蛊屋的排行榜。

而池家的这座太宇寺,就是榜上有名,名列第九位,非同小可。

它有极其著名的一招,名为转宇寰空,能陷八转蛊仙。天庭攻打南疆,此屋就相继陷害了四位七转,两位八转,战绩彪悍,极端了得。

“恐怕我的大盗鬼手,就是陷入转宇寰空当中去,因此被断了联系。不过,转宇寰空催动起来,代价颇高,并且需要酝酿的时间。对方为何反应如此敏锐?似乎早有准备一样?”方源脑海中,念头此起彼伏,迅速察觉到不妥之处。

“竟然真的有这样的偷道招数!”

“方魔越发难以遏制了,幸亏我们之前早有所准备。”

“说到底,还是依靠天庭情报,这一次真的很及时。若是真叫方源得逞,盗取了仙蛊屋中的仙蛊,那还了得?”

池家的蛊仙们后怕又且庆幸。

“杀上去!”

“时刻准备好转宇寰空杀招,应付方魔的偷道手段。”

“但要小心,方魔可是有着逆流河组成的防御杀招。”

轰隆!

在池家群仙的操纵下,八转仙蛊屋太宇寺爆发出猛烈的攻势,迅速把方源压入下风。

方源一边退防,一边探入心神,进入宝黄天。

下一刻,他心头微微一震。正如他所料,宝黄天中天庭方面,已经挂上一番番光影,重演方源战斗情景。其中他用大盗鬼手对付凤九歌的关键景象,也在当中。

不仅如此,天庭方面还将许多推算的结果,广而告之,根本不需用钱财购买。

这些推算出来的内容,都是关于方源,揭示他的老底。

宝黄天方面,已经闹得沸沸扬扬,整个五域的蛊仙界都被惊动,一时间都在讨论着方源的种种蛊虫、杀招手段等,对方源拥有的各种资源,报以羡慕嫉妒恨种种情绪。

这一手真的太狠了!

关于蛊仙的情报,向来非常关键。很多时候,一旦仙蛊、杀招曝光,蛊仙就极容易被他人针对。

就好比现在,大盗鬼手不仅被池家蛊仙防住,而且还被摄取到转宇寰空之中。将来他们定然会研究这具大盗鬼手,从而回溯逆推,侦查出方源有关此招的更多弱点。

若是没有防备,方源得逞的话,说不定就能盗取出仙蛊来,让方源大发一笔,更能占据上风。

方源之前利用宝黄天,大肆宣传自己痛揍雷鬼真君的经过,极大地影响了天庭的权威。现在天庭方面,以其人之道还治其人之身,用同样的方法来对付方源。

“我有大盗鬼手,紫薇仙子恐怕是极为担心,我利用此招大肆盗取他人仙蛊,迅速壮大自己吧!”

“她既然能够分心,做出此等举动,恐怕琅琊福地那边的战事,已经结束了。”

方源心中暗凛。

他恨不得天庭和长生天死磕到底,最终琅琊福地毁灭才好。但双方却都是精明之辈,不会做出此等愚蠢的举动。

“恐怕是天庭方面主动撤退。毕竟之前,为了对抗琅琊福地的底蕴手段,不管是紫薇仙子还是陈衣等人,都受伤不轻。凤九歌和我一战后,基本上失去了战斗力。还有雷鬼真君战死……长生天却是有着琅琊地灵辅助,拥有绝对的主场优势。”

另外令方源注意到的一点是,凤仙太子的内奸身份仍旧是没有在此事中曝光。

“凤仙太子极有可能,就隐藏潜伏在松尾岭附近,等到良机。”

“看来天庭方面,准备得很充分。就算没有凤仙太子的接应,也安然逃脱了。”

对于凤仙太子的这层身份,方源没有傻到提前公布。

天庭知晓方源是天外之魔,重生归来,但并不知道凤仙太子的身份已经在方源这里暴露无遗了。

“现在我根本没有证据,来指证凤仙太子。其实就算是有铁一般的证据,凤仙太子身为八转蛊仙,长生天制裁起来,也颇费工夫,准备不充分,还会令其走脱。”

这样的把柄,方源当然要好好利用,说不定将来能借此谋取巨大利益。现在抛出去,不过是一时泄愤,成大事者当要海纳百川,城府深沉,怎可能因情绪波动,而做无谓之举呢?

“不过,也不能让你好过!”方源肚中冷笑,旋即动手,向宝黄天中抛入一段影像。

这段影像,正是天庭方面发动星投,一下子传送了数位八转蛊仙,深入琅琊福地。

就像是一块巨石,这影像投入到本就波涛翻滚的舆论浪潮之中,立即掀起轩然大波!

天庭居然有这样的手段?!

就像是方源的定仙游,战术上的突出优势,已经影响到了战略!

除去中洲,四大域的蛊仙,各方超级势力首脑,无不心中凛然,对天庭的戒备和警惕瞬间暴涨无数。

“再见。”方源笑着对太宇寺挥挥手。下一刻,他把握良机,催动定仙游,成功撤离。

七转定仙游真的太好用了!

太宇寺可没有仙道战场,就算有仙道战场杀招,也不容易让方源陷进去。

其实就算陷进去,很多仙道战场也拘拿不住定仙游。

除非是规格极高,亦或者特别针对定仙游,方源若陷进去,这才会比较麻烦。

下一刻,碧光闪烁,方源彻底脱离战场,来到池家第三处资源点附近。

凤焰山!

此山生长无数晚霞梧桐,枫叶如火,放眼通红一片。鸟类众多,走兽几乎看不见,少数火凤凰在梧桐林间气息。感受到方源的凶意,这些凤凰立即抬起头,毫不畏惧地注视方源,大有一股凛然不可侵犯的威仪!

这是巨型资源点,就算是超级势力也不可多得,同时也是池家最主要的经济支柱之一。

“方源老魔竟来我这里了!”镇守此处的,乃是池家一位七转,他第一时间全力催起仙阵防御,同时拼命汇报家族,请求支援。

“若是我攻下这处资源点,池家就真的要肉痛了。呵呵。”方源冷笑,猛地俯冲,宛若一颗流星,轰然砸下。

一声巨响,仙阵岿然不动。巨型资源点的防御,远比中型、大型资源点更加森严。

方源定住身形,一边眼光闪烁不定,脑海中飞速推算,另一边手中杀招不停,宛若廉价的烟火,尽数挥洒出去。

“方源贼子,竟是如此狡诈!”接到这个消息,池曲由愤怒不已,他再也坐不住,只得离开,启程赶往支援。

凤焰山这处资源点其实有些特别,它除了豢养凤凰之外,每年到了特定的时期,还会吸引到许多野生凤凰,前来这里,停留产蛋。

一旦凤焰山遭受剧变,野生凤凰极可能就另选他址,会令池家损失很大。

偏偏凤焰山,距离云竹山脉颇远,距离最近的便是池曲由。因此池家太上大长老尽管已经在掠影地沟处,布置好了阵中之阵,此刻也只得放弃守株待兔的打算,扑向方源,救援自家产业。

待他赶到这里时,方源已经将凤焰山的大阵拆除了大半,上古火凤也战死不少,方源还收取了三头。

“方源你这小贼,有胆的别走!”池曲由瞪眼咆哮。

方源却是微微一笑,风度翩翩地打招呼道:“池曲由前辈,我可终于见到你了。”

\end{this_body}

