\newsection{楚瀛身亡}    %第六百三十三节:楚瀛身亡

\begin{this_body}



%1
庙明神面色不动,实则目光正细细观察。

%2
曾落子这一手就在他的安排之中,好让他收集到更有价值的情报。

%3
曾落子的信道造诣,即便是庙明神也十分敬佩。此刻曾落子露出一手,庙明神察言观色,立即有了许多发现。

%4
“此行共有八人。除去我、鬼七爷、蜂将、花蝶女仙四人之外,还有曾落子、土头驮、童画以及楚瀛四人了。”

%5
除了曾落子、庙明神之外,还有童画、土头驮二人不动声色。

%6
其余人脸上皆有多多少少的动容。

%7
庙明神最信任的,就是鬼七爷、蜂将、花蝶女仙三人,但此次安排他却未告知他们,因此这三人都是本色出演。

%8
曾落子、土头驮、童画三仙,庙明神日常也多有接触,算得上知根知底,他们三位皆是东海七转蛊仙中少有的强者。

%9
在东海这块地方,散仙数量最多,同时层次也较其他四域更高一层,皆因东海资源最为丰富。

%10
庙明神心中隐有顾虑的便是楚瀛。这人是隐仙,接触过一两次,令庙明神印象颇为深刻。说实在,庙明神本不打算邀请楚瀛。但念及楚瀛和任修平很早之前就有矛盾,又对蜂将、花蝶女仙有着救命之恩,更关键的是他手中似乎还有许多寿蛊存货。这些因素叠加起来,才令庙明神最终敲定邀请楚瀛参加此次行动。

%11
苍蓝龙鲸的具体位置,只有他一人清楚。如此一来,他任何的邀请都是一份人情。

%12
庙明神想与方源更深入的合作。不提其他,单单方源手中的寿蛊就令庙明神心动不已。

%13
曾落子这一手,楚瀛同样面现异色。

%14
至于土头驮、童画二人,也不知是真的免疫,还是故作镇定。

%15
庙明神将众仙神情暗暗看在眼中,旋即哈哈一笑:“曾落子仙友好生手段。”

%16
土头驮冷哼一声,童画则冷冷地瞥了曾落子一眼。

%17
方源看向曾落子,目光惊异,实则心中冷笑。

%18
曾落子的目光扫过楚瀛,旋即和土头驮、童画二人的眼神碰撞在一起,随后他朗声一笑,又抛出信道凡蛊。

%19
信道凡蛊中同样记载着盟约的详细内容,群仙视察之后,有的点头表示接受,有的则建议修改一些内容。

%20
当然这些内容无伤大雅,庙明神显然以头领自居,领袖众仙,协商妥当之后,以曾落子出手终是缔结了盟约。

%21
盟约一成,群仙相视一笑,氛围顿时缓和许多。

%22
“事不宜迟,我们这就出发。”庙明神雷厉风行,群仙随后。

%23
庙明神在空中疾飞一阵,便落入海水当中去。

%24
“空中不易隐藏,极容易被发觉。任修平若是得到此次行动的消息,一定不会坐以待毙的。因此进入海水当中,是利用海底潜流么……”

%25
方源心中猜测。

%26
片刻之后,果然如他所料,庙明神领着群仙穿梭海底,见到一道宛如巨蟒,横亘海沟的巨大潜流。

%27
庙明神逐渐降低速度,最终缓缓悬停在海底潜流面前。

%28
庙明神看着曾落子:“进入海底潜流之前,为了保险起见,确保我们不被分散,还请曾落子仙友出手。我知道你有一记仙道杀招,可以标记我等众人,令我们能相互感知彼此位置。”

%29
这又是一层控制局面的手段。

%30
但庙明神并没有在缔结盟约的时候抛出,而是面对海底潜流时顺势提出,自然而然,叫人无法反驳。

%31
“庙明神仙友真的要这么做吗?”曾落子故作迟疑。

%32
庙明神朗笑一声:“我们已经缔结了盟约,未来还要精诚合作,探索出乐土真传,相互之间还不能信任吗?”

%33
他语意深沉,引得童画、土头驮等人目光一闪,心中加深认同,微微点头。

%34
曾落子扫视周围一圈:“那就请诸位仙友勿要紧张,切勿防御,我的这招毫无攻伐威能。”

%35
方源暗中翻了一个白眼,开口道:“还请施为吧。”

%36
“先让我来吧。”蜂将率先站了出来。

%37
曾落子便首先对蜂将施展仙道杀招,其他人随后轮流接替。

%38
不一会儿,众仙均中了信道杀招,能感知到彼此位置。

%39
“这就妥了。请跟紧我,海底潜流速度极快,若是稍慢一点出去,就是一段遥远距离。”庙明神关照一声后,直接钻入海底潜流当中去。

%40
方源等人随后紧随而上。

%41
东海中的海底潜流,方向不一,长短各异,新来旧去,乃是大自然的天地伟力。

%42
顺着海底潜流前行,群仙的速度比一些寻常的仙道杀招都要更快一些,关键是只要防御自身,非常省力。

%43
若有水道蛊仙有着手段,还能够利用海底潜流加速前行,乃至防御自身,悍击对敌。

%44
方源在队伍的中后段,他发现这个队伍的序列,也是有讲究的。比如,花蝶女仙就在他的身侧,不时地和他传音交流。

%45
方源一边敷衍,一边暗中查看自己身上的信道道痕。

%46
观察了一阵后,方源便放下心来。

%47
要清除这些信道道痕,并不困难,方源有着大量的方法。同时更有洁身自好这等仙道杀招,专门克制信道的束缚。

%48
但现在却不是良机。

%49
“诸位仙友,我们就要出去了,紧跟着我!”前方,庙明神忽然传音。

%50
几个呼吸之后,他钻出海底潜流,其余人紧随其后,一个不落。

%51
离开海底潜流之后,方源很快发现自己身处在一片普通的无名海域之中。

%52
接下来的行程中,群仙又利用了几条海底潜流赶路,没有海底潜流的话,也一直潜藏在海底悄然前行,非常低调。哪怕是遇到拦路的荒兽、上古荒兽,都以闪避为主。碰到什么天然资源,也舍弃不取。

%53
就这样赶路连续数天,随着不断交流,队伍氛围更加融洽。

%54
庙明神左右逢源,交际手腕一流,牢牢稳固住头领位置。

%55
正说笑着,忽然他声音一顿,道:“我们接近了!”

%56
群仙精神顿振。

%57
方源心中疑惑,他已经施展了数个侦查杀招,但并未察觉到什么。更看不清庙明神究竟用的什么方法,还如此肯定。

%58
苍蓝龙鲸的存在,世人早已知晓。但它的具体位置,一直缥缈神秘,庙明神能够得到地点,手段定然是极其独到的。

%59
方源本身修为八转,战力雄浑,要斩除庙明神,搜刮魂魄还是大有可能。但是此中也有风险,蛊仙的手段向来是神秘莫测,碰到怪异的,难免阴沟翻船。

%60
所以,方源按捺不动,一直潜藏,真正到了苍蓝龙鲸处再寻机出手也不迟。

%61
随着距离越来越近,群仙都发现异常之处。

%62
前面的海水变得越来越湍急,之前好像是清风细雨,如今却变成了狂风暴雨,前进起来阻力倍增。

%63
荒兽、上古荒兽数量激增,在狂暴的海流中身不由己。

%64
方源等人不得不一边前行,一边对抗,屠杀大量荒兽,杀出一条血路出来。

%65
这时候众人的部分实力,都开始显露出来。

%66
蜂将、花蝶女仙战力最弱,很快就沦为被保护者。庙明神仍旧率先前行,充当着前锋,顶着最大压力。

%67
他身后则是童画、曾落子。

%68
土头驮乃是土道,在这个环境中,战力受到了更多的压制。

%69
然后便是方源,此刻他变作一头袖珍的海马,只有巴掌大小,行动如电。他表现出来的实力,在队伍中只是中上,不温不火。

%70
艰难前行,前面路程越发恶劣,当重重的鱼影中出现太古荒兽的身影时,众仙皆知已经不能再这样冲下去了。

%71
“最强”的庙明神,也不过七转修为。

%72
跟随着庙明神,他们飞出海水,来到空中。

%73
轰隆隆!

%74
天空中飓风狂卷,闪电如林,雷声阵阵,阴云如城。

%75
海面上惊涛骇浪,足有百丈之高。

%76
情势险恶,根本不输于海底!

%77
大量的飞鸟、恶鹰在电闪雷鸣中尖啸盘旋,相互搏斗厮杀。

%78
漆黑的阴云中,魅蓝电影时时闪现,被太古荒兽雷凰追逐猎食。

%79
见到雷凰的尾翼在阴云中一闪即逝,庙明神等人无不倒抽一口冷气。

%80
“快,跟我来!”庙明神传音,浑身汗毛炸立,飞速冲向前方。

%81
蛊仙皆有潜藏自己的手段,但这种情况下,非常混乱,什么手段都不保险,唯有赶紧远离才是最明智的抉择。

%82
雷凰尖啸,发现了方源等人,俯冲下来。

%83
海水爆炸开来,一头太古章鱼庞巨如山,甩动万千触脚,也把庙明神等人当做猎食的对象。

%84
“该死,该死!”庙明神面容扭曲,几乎要将一口钢牙咬碎。

%85
还未见到苍蓝龙鲸,他们已经身陷绝境,死亡的气息扑面而来。

%86
“分开逃!”犹豫了一下,庙明神嘶吼出声。

%87
蛊仙们连忙分头逃窜。

%88
咔嚓!

%89
雷凰速度极快,眨眼间杀到方源的面前。

%90
“楚瀛仙友!”花蝶女仙被鬼七爷带着,见到这一幕,失声惊呼。

%91
“救我!”方源大喊一声,满脸绝望。

%92
下一刻,他就被雷凰扑中,浑身上下都被电成焦炭,当场阵亡。

%93
“楚瀛仙友……”见到这一幕,剩余七仙心头狂跳,逃窜得更加惶急。

\end{this_body}

