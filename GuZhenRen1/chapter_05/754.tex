\newsection{方源顿悟}    %第七百五十七节:方源顿悟

\begin{this_body}

阎罗战场中,双仙对峙着。

方源叹息:“我们的时间有限,我就直接说了:我想和池家做交易,不知道贵方对我手中的梦道成果感兴趣么?”

池曲由微微一愣,旋即森寒冰冷的目光发生了变化。

“梦道?我刚刚没有听错?”池曲由眼中,一道热烈的光一闪即逝。

方源有梦道成果,早已不是什么秘密了。

方源之前伪装成武遗海,就是为了接触梦境。方源有将梦境转化成人的手段,在梦境大战中就彰显过,令众多超级势力都羡慕得流口水。

方源拥有春秋蝉,重生归来。而未来是属于梦道的,他掌握一些梦道的奥妙,毫不奇怪。

可以说,方源的梦道成果,领先整个世界,立足于五域之巅!

“梦境之玄妙,梦道之深邃,我也只是稍稍前行了一小步而已。若是我再告诉池前辈,我的阵道境界就是从梦境中得来,池前辈不知有何想法?”方源笑着,言语轻飘飘,却仿佛是一块巨石砸在池曲由的心湖中,轰的一下,砸出波涛滚滚。

“哦?此话当真?!”池曲由动容,再也装不了淡定。

方源哈哈一笑,之前他精心设计,打压池曲由气焰,至此刻他已经完全掌握了对话的节奏。

这和上一世完全不同。

方源接着抛出一只信道蛊虫:“请前辈再看这个。”

池曲由接过一看,眼中精芒直闪,手上不自觉地用力,暗暗捏紧信道凡蛊,深怕它忽然飞掉。

信道凡蛊中的内容,涉及梦道种种奥妙,这是无以伦比的诱惑!池曲由不禁深陷其中,很快将内容看完。

池曲由舔了舔嘴唇,短短片刻,他竟有些口干舌燥。

无比的心动。

池曲由强迫自己冷静,他故作沉吟:“你想怎么交易?”

“我想要的就是梦境。你给我梦境,我给你未来的梦道成果。”

池曲由没有一口答应下来,而是道:“这就难办了,毕竟南疆当中几乎所有的梦境,都被正道联合看守着。”

方源微笑:“池前辈,我需要是义天山那里的梦境。那里的大阵虽然彻底改变,但仍旧是由你主持铺设的。我还知道,那里有多条暗道,贵方偷偷出手,盗取一些梦境出来是相当方便的。”

池曲由听了这话,脸色再变。

方源竟对自己的情况如此了解!实在是可怖!

那义天山大阵内的暗道,是他池曲由的得意之作。他在南疆正道诸多蛊仙眼皮底下布置,欺负他们阵道境界不足,直接将暗道布置下来。

“我以为瞒过了所有的人,没想到被方源看破。”

“嗯,或许不只是方源……唉,我小觑天下英豪了。”

池曲由又多想了。

方源的这份情报,同样是在上一世得来的,根源还是在于池曲由自己。

方源拿池曲由对付池曲由,效果真的很棒。要不然怎么说:自己往往是自己最大的敌人呢?

方源一步步打压池曲由,池曲由再无反抗之力。

每一场谈判,都是一场交锋,虽无实战中直来直往,但亦是实力的较量,谋略的切磋。

和上一世相比,方源拥有的牌面太多,知己知彼,实力大增。他知己知彼,又采用虚张声势的谋略,收效极佳。饶是池曲由这等人物,也着了他的道。

交易的细则很快就商量妥当,结果自然对于方源极为有利。

池曲由答应这场交易,对于他的身份而言,堪称“屈辱”。

但他无可奈何,他被方源吃定了。

首先,他拿方源没有多少办法,真要对付方源,风险太高,收益很不明显。

其次,他和方源合作,方源就会停止劫掠池家,还交易梦道成果。梦道成果对于池曲由太具有诱惑性了。

最后,池曲由还惦记着所谓的南疆正道的内鬼,他十分想要揪出正道内鬼。这点或许可从方源身上着手。

至尊仙窍,小南疆。

一座山峦光秃秃的,毫无植被,突兀地矗立在大地上。在它周围,地面并不平坦,但纵然起伏,程度也是有限,顶多算是一些小山丘。在这些小山丘的衬托下,这座光秃秃的山峦更显得高耸巨大。

此刻,在秃山之巅,站着一位六转石人蛊仙,正仰头望天。

“成就洞天后,果然这仙窍气象和福地大不同!”

“真是壮哉!这片至尊仙窍之大,超出常理,超出想象,不愧是魔尊幽魂炼制出来的奇迹!!”

石狮诚每一次观察这片天地,心中都赞叹不已。

这是伟大,这是奇迹,这是鬼斧神工!

他对魔尊幽魂佩服得五体投地,更对方源无一点反抗的念头。

在他的身后,还有一群石人蛊师,私下议论着。

“耗费了十多天的时间,我们终于将这座山丘造成了!”

“可累死我了,每天都是起早贪黑的干活啊。”

“造这么一座山丘,听闻是那位至高亲自下达的任务呢。”

“为了赶进度,我们不惜代价,长时间催动蛊虫,可是死了至少三十只土道蛊虫……”

“说起来,你们不觉得在这里炼蛊,成功的可能性很高吗?”

“唉,只要完成门派任务,就会有奖励。但愿那位至高无上的蛊仙大人的奖励会丰厚一些。”

“噤声!那位大人岂是你们能够议论的?”石人蛊仙再听不下去,转头训斥道。

这群石人蛊师连忙低下头,噤若寒蝉。

就在这时,青光一闪,一道身影落下,悬停于半空。

石人蛊师看到来者,正是方源六转宙道分身。众人顿时心头凛然,连忙大片的跪倒下去,行大拜之礼。

“属下拜见大人!”石狮诚也连忙拜见,态度十分恭谨。

方源分身神情淡漠,对他点了点头,问道:“这里的大阵布置妥当了吗?”

方源发布下的任务,不仅是人工造山,还有在山体内部布置仙道大阵。

石狮诚连忙汇报道:“大人,按照您给的阵图,属下已经将这座炎道大阵布置妥当,如今只需您催起即可。”

方源早就动用神念透射山体,入内视察。

他点点头,对视察的结果比较满意,正要挥退这些石人,却见石狮诚欲言又止。

石狮诚虽然只有六转修为,但曾经和方源一战,他的仙道战场杀招灰云石傀留给方源深刻的印象,是一个人才。

安置了石人之后,方源就将其余的石人蛊仙派发到石莲岛上去,熟悉未来身杀招。而石狮诚则留下来,和雪儿、墨坦桑一样,担任了石人一族的第一首脑。

“你有什么话要对我说?”方源分身停顿一下,问道。他对于这个人才,还是有些耐心的。

石狮诚眨了眨眼,神情局促不安,他似是下定了决心,开口道:“听闻大人您近日来屡屡出击,大获全胜,杀得南疆正道一片惶恐。那池家太上大长老池曲由更是对您无可奈何,任由您来去自如……”

方源眼中精芒一闪,他是多么精明的人物,一听之下就知道石狮诚的意思,直接打断道:“看来你们一族的蛊仙,都是想劝我住手,好好休养生息,没必要这样冒险是吗?”

“呃,属下惶恐!”石狮诚一躬到底,面带惶恐地道,“这只是属下斗胆,纯属一人之言,太上大家老等人皆在石莲岛上潜修,并不知情。”

方源分身心中冷笑。

石人一族和他并不齐心,这些异人族群想的是安居乐业,想的是稳定发展,不去冒险。他们巴不得方源躲在什么旮旯角落里,一躲百年、千年。

他们不想让方源冒险,因为后者一旦战死,至尊仙窍就要暴露,就要连累他们了。

但这怎么可能?

方源追逐永生,唯有摧毁宿命蛊,方有更进一步的可能。而且天庭若修复好了宿命蛊,也绝对不会放过他,迟早有一天会找上他,将他围杀。

躲避和拖延只会绝望,唯有趁着最后一届中洲炼蛊大会,中洲遭受四域齐攻,天庭最为虚弱的时刻出手,才最有成功的可能。

异人们的愿望,是不可能实现的。不仅如此,他们还要在将来为方源去战斗。他们已经被方源绑在了战车上,早已经下不来。

石狮诚的劝诫,很明显是许多异人蛊仙一致的意思,他们通过石狮诚提出这个建议,而不是自己出面,也是给自己留一个台阶或者后路。

这是很成熟的政治手腕。

方源没有拒绝,也没有答应,而是挥手道:“你带着族中精英,先行离开此山。”

“是。”

见石狮诚离得远了,方源分身便催动山体大阵。

瞬间,大阵像是一个火炉,内里升腾起恐怖的热量。这些热量迅速地将山体内部烧烤,形成一股股的岩浆。

片刻后,轰的一声,山巅炸开,烟尘冲霄而起,大股的岩浆顺着山体缓缓流淌,空气温度迅速攀升。

远处的石人蛊师们纷纷惊呼,这种改天换地,硬生生制造出一座活火山的手段令他们分外震撼。

石狮诚也动容不已,他阵道造诣很低,至此才明白自己布下的大阵究竟有什么作用。

但更令他动容的还是接下来的一幕。

只见头顶高空,忽然飞来一记大手,大手如山,五指忽然张开,向活火山洒下无数树木。

这些树木被一股股流光包裹,缓缓飘落,直接栽种在岩浆中。

可怕的岩浆迅速被树根吸收,大树吸收了充分的营养,很快树枝招展,树叶熠熠生辉,散发出晚霞般的美丽光辉。

这正是晚霞梧桐!

随后,又接连有大手飞来,这一次是直接向火山口中抛落火凤凰!

一头头火凤凰落到火山口中,旋即就被火山中的大阵囚禁、镇压。

全程都由方源分身亲自照看,不出一丝差错。

石狮诚长大嘴巴,神情呆滞地目睹这一幕。

他虽然知晓方源已经晋升八转,但绝没有这次亲眼目睹来得震撼人心。

这些火凤凰中,有不少上古荒兽,只需要一头,就能将石狮诚虐杀,毫无悬念。

方源分身最后又安置了一大片的炎道鸟禽,种类繁多,杂七杂八。

“从今日起,这座山便是火鸟山,也是你们石人一族的资源点。”方源分身飞到石狮诚面前。

这是一个巨大的惊喜!

石狮诚连忙跪地拜谢,激动无比。

方源分身微笑:“起来吧,我说过的,会大力扶持你们。这处中型资源点就当我奖励给你们,你们造山辛苦了。”

“属下惭愧。”石狮诚感激涕零,和这座火鸟山相比,他们付出一点点的的辛苦算得了什么?

这座火鸟山的大阵是方源提供的蛊虫,提供的阵图,所有的资源也都是方源之物,几乎等若是方源白送给他们的。

方源分身继续道:“以后我还会继续交代你们任务,火鸟山不会是最后一座。”

石狮诚楞了一下,感受到了方源此言的深意。

这些火凤凰、晚霞梧桐哪里来的?

是方源抢来的。

方源说火鸟山不会是最后一座,也就是说,今后他还会继续劫掠和冒险!

异人蛊仙们建议方源,方式很委婉,方源回绝的方式也很委婉,但细细品味,却会发现方源的霸道和威慑之意。

但石狮诚不敢反驳,接受的时候心中带着一丝欢喜。

“有了这座火鸟山,就是我石人一族的崛起之基啊!”

“方源大人是那种容易劝说,容易动摇的人吗?”

“我可不管这些了,方源大人安排我领袖族群,这些资源也是我的修行资源!只要我不过分,我完全可以迅速修行,将自己的修为提升上去啊。”石狮诚目光火热。

任何一个蛊仙,在灾劫的压迫下,对于任何增长实力的机会,都是非常珍惜,非常渴望的。

在方源的政治手腕下,石狮诚已是和其他的石人蛊仙分离开来。

石人挺适合管理和经营火鸟山,火山周围的地底温暖干燥,是石人上佳的沉眠栖息之地。

当然,最适合经营火山的还是蛋人。

不过目前而言,方源还不准备引进蛋人这种族群。发展眼下的这些异人族群,已经足够了。

石狮诚的觐言,带给方源不少感慨。

“和上一世不同,这一世我收容了许多异族,并且还要培养、壮大他们,需要分出一些精力去照看,分出一些物力去栽培。”

方源当然可以动用奴道手段,让这些异族服服帖帖,但这样一来,无疑就失去了方源栽培他们的本意。

“我要感悟人道,而人道和奴道不同,每个人都有自己的想法,都有自己的愿望。要维系一个组织,就要考虑到组织中成员的想法和情感。没有完全相同的人,再相似的人都有着各自的不同……嗯?”

方源忽然心中一动。

一道灵感仿佛闪电划破夜空一般,照亮他的脑海。

这是人道的灵感!

骤然间,方源悟通了某个关节,他神情微微复杂:“人道……原来如此,我其实早已有了人道的杀招……”

在这一刻,他的人道境界晋升为大师。

人道大师!

------------

\end{this_body}

