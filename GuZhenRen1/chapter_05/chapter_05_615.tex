\newsection{后勤}    %第六百一十八节:后勤

\begin{this_body}

%1
至尊仙窍。

%2
小西漠。

%3
方圆上千里地,均是一片雪白。方源的宙道分身涉足此地,弯下腰抓起一把沙子来,放在眼前端详。

%4
这些沙子颗颗饱满,晶莹如冰,透着一丝寒意。

%5
此时尚是晴朗的白日时分,小西漠的气温并不低,但在这片沙漠中却是叫人感到凉爽舒适。

%6
方源分身又拈起一小撮的雪白沙子,放到嘴里。沙子遇到口水后,立即融化,产生一股淡淡的咸味。

%7
这就是盐沙。

%8
盐沙在人族的历史上,有着一个流传很久的故事。

%9
在一百万年前,上古时代,人族虽然创建了天庭,但其余四域异人仍旧势力庞大,压制奴役人族。

%10
在西漠中,就有一位人族的城池,遭受周围异人的排挤和打压。

%11
异人们联合起来,伪装成盗匪,阻截抢掠任何去往人族城池的商队。久而久之,人族城池缺少生活物资,渐渐无法支撑。

%12
人族城主为了整个城池的安危,不得不向其中一支异人族群妥协,将其最心爱的小女儿外嫁出去,换取和平。

%13
城主的爱女识得大体,为了整座城池的族人,愿意牺牲自己。

%14
和亲的队伍很快就集结妥当,从城池出发,由人族城主亲自护送。

%15
在跋涉沙漠的时候,他们遇到了一位昏死过去的老者。

%16
老者伤势很重,处于濒死状态,浑身长满脓疮,恶臭逼人。城主见他是人族,便派遣手下将他救下,喂以清水。

%17
老人悠悠醒转,向城主道谢:“城主啊,你既然救醒了我,不妨再救我一下。给我你的衣袍,还有你的坐骑,让我能够自行离开。”

%18
城主手下便嘲笑起来,城主的衣袍多么珍贵,怎么能穿在你这样的老乞丐的身上呢?

%19
城主却摆手:“衣袍再珍贵,也比不上人的命。我有备用的衣袍,这就给你。不过我的坐骑,却非得实力高超的蛊师才能驾驭。给你的话,反而是害你的命啊。”

%20
说完,城主果然将衣袍,还有大量的清水、干粮,交到老者手中。

%21
而后,城主又命人牵来一头温顺的沙驼,送给老者。

%22
老者感慨不已:“城主,我素闻你的仁名,今日一见果然不虚。你既然要做好事,不如就将好事做到底。我这背上的脓疮,困扰我多年了。只要阴年阴月阴日出生的少女处子,亲自用嘴给我咬破,就可以痊愈。”

%23
老者刚刚说完,城主的麾下就痛骂起来,差点就要动手,杀掉老者。

%24
城主也有怒气,因为谁都知道,他的女儿就是阴年阴月阴日出生的少女处子。

%25
城主道:“老丈,你就不要消遣我了。我虽然是城主,但此刻我也是一位可怜的父亲。我爱我的女儿,但我不得不将她送给居心叵测的羽民一族,叫她充当别人的奴妾。”

%26
“父亲,如果我的牺牲,能够换来满城族人的存活,我愿意接受这个命运。”这个时候,城主女儿走了过来。原来是吵闹的声音,吸引了她的注意,从旁人的口中,她也了解到了事情的经过。

%27
她走到老人的面前,点头道:“老丈啊,请你将背后的脓疮给我看看。”

%28
“你愿意出手,治好我的伤?”老者疑惑。

%29
“是的。我虽然是城主的女儿,旁人都说我身份高贵,但这份高贵又能高贵到哪里去呢?”城主女儿苦笑,“只有我们整个人族高贵起来,我们的高贵身份才真的高贵。我已经没有希望,就要沦为异人的奴妾。既然如此,我为何不来救治你呢?我们人族的力量太薄弱了,多一名族人康复,也是好的。”

%30
城主和其他人听到这番话,都十分感动,没有再阻止。

%31
老者露出背后的脓疮,丑陋非常,脓疮横流,恶臭逼人,令人一看就有一种要呕吐的冲动。

%32
城主女儿楞了一下,但最终她还是强忍着不适,用牙齿咬破了老人背后的脓疮。

%33
疮口一破,留下的却是白银色泽的汁液。

%34
汁液流淌到了沙漠上,浓郁的香气旋即弥漫开来,令整个和亲的队伍都神清气爽。

%35
老者的身影骤然消失不见,只余下一股声音,传遍众人的耳中:“城主啊,你这样的人,才值得老夫出手救上一救。人族光是有蛊仙还不行,得有你们这样的蛊师才有希望。这片沙漠就交给你们,好生经营吧。”

%36
城主等人这才恍然,原来老人乃是一位人族的蛊仙,纷纷跪拜行礼。

%37
而这片沙漠,被白银汁液沾染蔓延之后,就变成了雪白的盐沙。城主掌握了这样一片巨大的沙漠,便将这些盐沙当做食盐贩卖,不愁销路,使得城池起死回生,更日益壮大起来。

%38
时至今日,盐沙已经不是西漠独有的产物,早已遍及其他四域。

%39
方源至尊仙窍中掌握的盐沙,乃是他吞并了南疆蛊仙仙窍后获得的。

%40
视察一圈,方源分身对这片沙漠有了最为深刻的了解,满意而归。

%41
“如今宝黄天中,天庭已经开始贩卖银龙鱼。看来他们定然是从魔尊幽魂那里,搜刮出了龙鱼的豢养之法了。加以时日,金龙鱼也会流向市场。”

%42
方源的龙鱼生意,遭受到十分强烈的冲击。

%43
短时间内,因为他本就抢占了市场,龙鱼这块还是有不少收益。但之后肯定会被天庭逐渐排挤。

%44
不过没有关系,方源吞并了南疆诸仙之后,至尊仙窍中已经多了十几块福地,更有了一片洞天。

%45
盐沙就算了,只能算是普通的蛊材。但仙材资源却是不少,其中有五六项资源点,都是可以媲美龙鱼的生意。比如夏槎洞天的年华池就是其中之一。

%46
因此,方源仍旧是日进斗金,甚至积累仙元石的速度比之前还要迅猛数倍。

%47
而之前他苦心经营,仙蛊喂养也几乎没有什么缺口。纵然是多了一些仙蛊战利品,但方源同时也得到了蛊仙的仙窍,完美地移栽了各项资源点。

%48
“仙蛊喂养还有一些小瑕疵,但并不为虑。”宙道分身最后扫视一眼盐沙之地,便开始催动定仙游。

%49
这片盐沙沙漠,是他视察的倒数第二项资源点。

%50
下一刻,方源分身来到市井之中。

%51
市井里头,已经聚集了一群规模可观的毛民。

%52
这些毛民大多是方源从南疆蛊仙仙窍中俘获的,南疆蛊仙或多或少都豢养了一批毛民,用来炼制凡蛊。

%53
毛民奴隶在宝黄天中,可是最吃香的一种。

%54
方源将这些来各个仙窍的毛民,都聚集起来,放到市井中圈养。

%55
目前而言,圈养毛民对方源的帮助是最大的。

%56
方源将梦道的一些蛊方,以及修行之法都传授给这些毛民。这些毛民的主要任务,就是进入自己的梦境,搜刮出梦道蛊材,然后炼制出梦道凡蛊,交给方源使用。

%57
方源的解梦杀招,对梦道凡蛊有着巨大的损耗。

%58
之前方源大肆探索梦境,已经是将手中积压的梦道凡蛊消耗一空,若非没有这些毛民及时炼出来的梦道凡蛊进行补充,方源还未必能够将手中的梦境都探索完全。

%59
方源分身来到市井之中的时候,天空逐渐晦暗下来,虽无日月,但好似傍晚。

%60
自从方源八转之后,至尊福地就晋升为洞天,拥有天象的变化。又因为至尊仙窍中宙道资源极其丰富,导致这份天象变化十分精细,几乎和五域外界大致相若。

%61
方源分身正要潜行下去观察毛民的情况,忽然身形微微一滞,眼中迅速闪过一抹精芒:“本体已经找到了五相封印的所在了!”

%62
方源在幽火小洞中稍稍休整之后,便离开小洞,启程进入白天。

%63
他仙窍经营情况上佳,后勤到位,这就导致他根本不需要静养,仙元储备一直在缓缓上升。

%64
他自然不想白白浪费时间,便前往白天。这个时间点正是要五相封印开启,再一次进行千年赌约!

\end{this_body}

