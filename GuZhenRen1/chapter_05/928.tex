\newsection{乐土插手}    %第九百三十二节:乐土插手

\begin{this_body}

“来一来,看一看啊,上等的丝绸!”

“糖葫芦,糖葫芦,三串送一串。”

“客官,您有什么需要?”一家贩卖蛊虫的店面中,店小二满脸热情地迎接洪易。

洪易先是定神瞧了瞧这个伙计,又仔细地扫视整个店面的格局和摆设,他暗暗皱起眉头:“这一切都是如此的栩栩如生,究竟是怎么回事?”

“听说你们这里有书生蛊卖?”洪易问道。

“呃,对。但是大人,这书生蛊可是五转蛊虫呐。”店中伙计有些迟疑。

洪易轻轻一笑,不再遮掩,释放出五转蛊师的气息,立即引得店中大乱。

店中伙计满脸惊容,店铺的店长也被吸引过来,拱手恭迎洪易,并亲自取出书生蛊,给他观览。

“果然是书生蛊。”洪易详细端详片刻,心中疑虑更深,“可否当面试验一二?”

“客人请便。”店长立即将书生蛊借给洪易使用。

洪易催动真元,灌输到书生蛊上。这只书生蛊立即挣脱洪易的手指,飞出去,落到地上化为一位白面小书生。

小书生对洪易作揖,神态恭敬:“学生拜见上师。”

洪易不由点头,大生兴趣。这种书生蛊十分罕见,乃是人道蛊虫,洪易也只是在传闻中听说。

洪易从自家空窍中取出一只智道凡蛊,一只信道凡蛊,都扔给书生:“用用看。”

书生接过这两只蛊虫,根本不需要熟悉的过程,上来就用,并且用的相当熟练。尤其是某些催动的小习惯,就好像是洪易本人在用一般。

“客官觉得如何?”店长笑眯眯地问。

洪易连连点头:“很好,这只蛊我要了。”

店长大喜:“承惠,六十万元石。”

洪易不禁扬起眉头。

这个价格不是贵了,而是实在是太过便宜。五转凡蛊的价格大约在十万和百万元石之间,但这种书生蛊属于相当稀罕的极品五转蛊,价格必须是在百万之上,并且有价无市,万金难求。

但在这里,却只卖六十万元石!

洪易奇遇连连,身家富足,当即付款,收得这只书生蛊。

走出店铺,洪易心中波澜四起:“这片地方一如帝君城,但帝君城此刻周围不是有仙人大战吗?这里却是安居乐业,平静祥和。到底是怎么回事?”

“还有,表面上看来这里和现实没有什么区别,但若细细深究,就会发现这里有大量的人道蛊虫。书生蛊只是其中之一而已。并且这些人道蛊虫如此集中,甚至泛滥的情况,在这些人看来,却是稀疏平常。而常规的炎道、水道等等流派的蛊虫,反而是珍稀之物。就好像这里人道乃是蛊师修行的主流。”

“真是奇也怪哉!也不知道叶兄那边,又打探到了什么?还是速速和他汇合罢。”

叶凡和洪易被元莲意志赶到壁画中后,相互商量后,就在这里面分头探索。

洪易发现的情况,叶凡也发现了。

但此刻,在街边的角巷子里,叶凡却是见到了一个意想不到的人。

“师父!”叶凡惊愕。

他也从店铺里购买了一只人道五转蛊虫,名为大侠蛊。出了店铺,他来到角落里偷偷催用起来,竟和店铺中使用的情况不一样。

大侠蛊竟然变作了他师父的模样!

“不必吃惊,我的徒儿啊,你我在这里相见,自然是缘分所致。”陆畏因微笑着道。

叶凡立即听出言为之意,不由大喜:“师父,你知道这里的蹊跷?这里究竟是什么地方?为什么我们会忽然陷落到了这里呢?”

陆畏因悠然答道:“这话就要从头说起了。三十万年前的中古时代,中洲诞生了一位仙尊名为元莲。他开创因果神树杀招,在游历西漠的时候,拒绝了一位孩童的复活亲人的请求,因此洞察到未来的某一幕景象,帝君城将因地沟而毁,一片生灵涂炭。”

“于是,他便当场埋下了豆神宫。后来回到中洲后,便又在帝君城中施展画道杀招安居乐业,形成壁画世界。你们二人因和方源连运,关系紧密至极,被元莲意志忌讳,因而打发到了这里,等若遭受封禁。”

叶凡听得目瞪口呆,一连串的丰富信息让他一时间反应不过来。

他感到十分的震惊:“这件事情竟然和元莲仙尊有关?还有师父,我什么时候和方源连过运?”

“那是方源暗中出手,想要借助你的运势来成事。上极天鹰也是他的太古荒兽。”陆畏因从容答道。

“是方源主动和我连运,要借助我的运势?”叶凡更加吃惊。

叶凡拜陆畏因为师,半只脚踏入蛊仙的世界,远比洪易知晓更多秘辛。他十分清楚方源究竟是怎么样的一个大人物!他是盖世的魔头,连天庭都无可奈何的绝世人物。自己居然和他扯上了关系,这,这简直像是天方夜谭!

陆畏因微微摇头:“徒儿,你切勿妄自菲薄。你乃是天道垂青之人,天生拥有强大的运势,虽起于微末,但将来必定能有巨大的成就。你这样的运势,整个五域也超不过十指之数。方源要借助你的运势,并不奇怪。”

“然而天道一饮一啄,他曾经借助过你的运势,得了你的好处,如今你也要借助他,获取天大的仙缘。上极天鹰只是其中的一部分,现在摆在你眼前的就是元莲的真传!”

“元莲真传?!”叶凡再次惊呼,他忽然明白过来,看向陆畏因的目光也变了,“师父,你对这一切了若指掌。这么说,这一切都是你设计的?”

“哈哈,徒儿你高看为师了。这一切都是乐土仙尊大人的布置,而为师不过是他的传人罢了。”陆畏因道。

叶凡又听闻乐土仙尊的大名,意识到自己也成为了乐土一系的徒子徒孙,不过他没有感到狂喜,他已经震惊到麻木了。

陆畏因向叶凡招招手:“时间有限,我们边走边说。”

叶凡连忙跟上,两人顺着街道的人流,向前走去。

陆畏因详细地解释道:“这道元莲真传价值非常之高,分有三部分。第一部分便是这安居乐业杀招,第二部分则是人道之子的运势,第三部分是豆神宫和帝君城组建而成的八转仙蛊屋——帝神宫!”

陆畏因顿了顿,继续道:“这三部分,我们先取第二部分,让你成为人道之子。”

叶凡追问:“师父,什么是人道之子?我究竟又该怎么做?”

陆畏因便答:“中洲和其他四域不同,采用师门制度,人道最是隆昌,积累雄浑。帝君城历来就是中洲地表人脉的最大集结点。这里为何出产涌现无数人才?便是人脉的缘故。只要你汲取人脉中的力量,便能极大助长你的运势,得到人道的钟意,成为人道之子。而要做到这一点,还得借助洪易的力量。”

“洪易?他是我的结拜兄弟,怎么又和他扯上关系?”叶凡吃惊不已。

陆畏因道:“方源同样和他连运,但他的强运和你不同,你是得到天道垂青之人,而他却是中洲人道之子中的一位,受人道钟意。你借助他的关系,方可顺势而为,在沈伤的运作下,悄然得到人脉认可。有了人道之子的身份后,我们再来借助方源之力,图谋真传的第一部分,甚至是第三部分。”

豆神宫壁画。

“方源、方源……”一道声音断断续续地传入房睇长的耳中。

房睇长已经停止施展因果神树,正催动智道杀招,不断思考如何脱困,这时听到异响。

“是沈伤的声音!他是从本体那里得知了我的身份?”房睇长犹豫着,心中有所猜测,但并未第一时间答话。

果然,下一刻沈伤的声音继续传来:“房睇长,你的真实身份我已经从你的本体那里得知了。没想到方源竟有如此一手,能将分身安插到西漠的超级势力里去。我现在就助你脱困。”

“哦?你如何助我?”房睇长疑惑。

沈伤笑了笑:“你应当知道,我一直在尝试破解尊者的人道手段,已经摸到了当中门径。”

“不错。”房睇长点头。

沈伤继续道:“豆神宫和帝君城融合一体,我也一直潜伏,并未让元莲意志发觉,所以现在我仍旧在破解人道杀招。”

房睇长直接问:“那又如何?”

沈伤耐心解释道:“这是一个绝好的契机!我已收集四域人意,不断灌输这里。原本对帝君城无法插手,就无法从内里干扰影响到尊者手段,我必须悄无声息的渗透。但现在豆神宫和帝君城组并一体,而你曾经炼化过豆神宫,权限只是不如元莲意志而已。如今你手中又掌握着因果神树杀招,正可从中发力,动摇整个仙蛊屋,配合我让尊者手段扩散开来,为所有人增幅。”

“这么说来,你是让我配合你破解尊者的人道手段?”房睇长一语中的。

沈伤笑了笑:“我们这是互帮互助!”

“好,我答应你。”房睇长没有犹豫,

\end{this_body}

