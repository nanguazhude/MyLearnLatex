\newsection{龙公转运}    %第八百一十七节:龙公转运

\begin{this_body}

%1
东海深处,龙宫梦境中又过了一幕。

%2
龙人分身凝神操纵着蚁群,大军向前开赴。

%3
黑压压的蚂蚁,足有数千头,军阵严谨。其中以甲蚁开道,顶在前方,而后则是箭蚁,局中的乃是王蚁。而在大军的最后,则是炮蚁,体型肥硕,移动缓慢。

%4
龙人分身眼前,泰琴同样操纵着一支蚁军。

%5
“师兄,我来了!”黄眉少女娇喝一声,大军展开冲锋,杀向方源的部队。

%6
方源连忙心念调动,稳住阵脚,以甲蚁固守,炮蚁轰炸。

%7
泰琴掌握的蚁军,多以枪蚁、奔蚁为主,进攻性十足。

%8
方源心中暗笑一声:“自从绿蚁居士的收徒大典之后,泰琴被我击败,仿佛受到了刺激,一改之前谨慎稳定的战斗作风,变得激进,崇尚进攻了。”

%9
这时,两支奔蚁的队伍一左一右,环绕过来。

%10
泰琴的战术相当明显,想利用奔蚁的速度,迅速击中方源的后方炮蚁。

%11
炮蚁远战强悍,射程极高,一炮下去杀得正前方的枪蚁伤亡惨重。但是它本身行动缓慢,防御薄弱,若是被这两支奔蚁近身,必定死无全尸。

%12
“很正确的战术,然而,我之所以如此布阵,又岂会没有算到这种阵型的弱点呢。师妹啊。”方源对眼前的少女笑道。

%13
“那就来吧,师兄!”黄眉少女高呼一声,双眼迸射精芒,已是用上了全力。

%14
奔蚁的突击军队环绕过来,方源的蚁军整体不断后撤,将后面的炮蚁重重包裹起来。整个过程,行云流水一般。

%15
泰琴见此,瞪大双眼,用一种可爱的表情惊叹道:“好厉害!师兄,你竟然能同时操纵全部的蚂蚁!!你已经掌握了师父传授我们的蚁念杀招了吗?”

%16
方源笑着点点头。

%17
泰琴咬牙:“纵然如此,我也绝不会轻言放弃!师兄你的前军,我就吃下了。”

%18
原来,为了整个队伍的后撤,方源将前面的甲蚁都留在了前线,抵挡住泰琴的枪蚁。

%19
方源的蚁群后撤成功,但也和前线的甲蚁军团脱节。

%20
甲蚁支撑没有多久,就被数量更多的枪蚁淹没。

%21
枪蚁宛若潮水一般,倾斜过来。

%22
方源早已准备好了箭蚁,排布在最外围,不断喷射蚁箭。同时上空,还有蚁炮高飞后,砸在枪蚁军团之中,炸死一片片的枪蚁。

%23
在方源如此狂射乱轰之下,泰琴支撑不住,蚁军被迅速消灭。

%24
“呼!我又输了,师兄,还是你厉害!!”黄眉少女吐出一口浊气,看向方源的眼眸中带着盈盈的一抹亮光,脸颊上也各有一团微微的红晕。

%25
方源看在眼里,心中了然。

%26
自从绿蚁居士的收徒大典,梦境已经是过了三幕。

%27
回想起来,方源还有些余悸。

%28
在收徒大典的那处梦境中,不管他如何优势,总会有各种意外发生,将他的优势抹平。

%29
但方源没有主动认输,而是在思考之后,选择藏拙僵持,和黄眉少女死拼。

%30
最终两人拼了一个平手,都到达了蛊师的极限,绿蚁居士现身叫停,将这两人都收为了徒弟。

%31
原来,绿蚁居士的择徒标准,表面上是操蚁对决,实际上还有另外一层用意,就是考察蛊师的心性,到底能够有多坚持。

%32
方源若是主动认输,必然不符合这个择徒标准,探索梦境也就失败了。

%33
所以,择徒大典想要通过,不是胜利也不是失败,而是平局。

%34
过了这道难关之后,接下来的几幕梦境,皆是绿蚁居士传业授道。方源蛊虫大量增加,主要的对手就是成为了师妹的泰琴。

%35
泰琴这个黄眉丫头自从和方源切磋,就从未赢过一次。

%36
刚开始时,还不服气,但渐渐的却是,却是被打服了,甚至还有一丝情愫暗生,或许她自己都不清楚。

%37
“好极了,看来分身在梦境中大有进展啊。”方源本体观测着龙人分身的气运。

%38
只见这紫色小龙气运,已经膨胀了数倍,周围的黑云则削减了三成。

%39
与此同时,黑气中转化而出的青紫之气也越发浓郁,比之前要明显得多。

%40
但就在这时,忽然又有一团血光,从西边飘来。

%41
这团血光十分庞大,悬停在紫色小龙的头顶上空,并不和之前的黑云融汇一起。

%42
血光中隐隐有金色凤鸟、神龙的光影闪现,透露出来的威势,叫人心惊。

%43
“怎么回事?又发生了什么变故?”方源本体心头一沉。

%44
他连忙推算。

%45
“这血光气运,比第一股要庞大十倍有余!又从西边而来,东海之西,就是中洲和西漠。”

%46
“血光气运悬而未决,显然还未到发动的时刻。又如此庞大,极有可能便是中洲天庭!”

%47
“天庭的人怎么能察觉到这里呢?”

%48
“上一世他们发现的时间,可不是现在。难道是我重生之后,引发出来的墨水效应?”

%49
方源缺少情报,但也推算出了多种可能。

%50
他并不惊惶,事实上,他对此早有安排。

%51
因为天庭一直是他最大的敌人!

%52
天庭。

%53
中央大殿。

%54
紫薇仙子徐徐收功,脸色浮现出一抹喜色:“果然是叫我推算出来了。”

%55
她掌握了仙蛊屋龙宫的具体方位!

%56
这是因为之前,天庭就一直派遣人手,秘密调查五域,搜寻龙宫所在。

%57
进展虽然缓慢,但这么多年积累下来,也是接近成功了。

%58
方源上一世,天庭就自己找到了龙宫。这一世,得到了关键性的线索后,这份积累的成果就提前爆发出来。

%59
紫薇仙子连忙将这个消息告知龙公、秦鼎菱。

%60
“龙宫必须要回收。这是老夫的因果,就让我亲自去一趟吧。”龙公主动揽下这个事情。

%61
紫薇仙子顿时放下心来:“有龙公大人出手,必定手到擒来。”

%62
就在这时,紫薇仙子脸色一变,她接到了战报:“不好,光阴长河中再次出现了万年斗飞车,方源又一次行动了!”

%63
“偏偏这个时候?”秦鼎菱皱起眉头,“这未免有些太巧了吧?且容许我窥视一番二位的气运。”

%64
紫薇仙子、龙公自无不可。

%65
秦鼎菱先看了龙公,只见龙公的气运呈现龙形,又好似火烧云。

%66
苍老的巨龙垂首,盘曲龙躯,似乎就要睡去这预示着龙公命不久矣。

%67
秦鼎菱闷哼一声,仔细分辨,又发现老龙的身上,鳞片呈现血红之色。龙头更有皱纹重生,呈现一个倒“山”字这就是败相了!

%68
秦鼎菱心惊不已:“这世间还有人能让龙公失败?”

%69
想到这里,她的视野陡然一片血红,察运杀招停滞下来,反噬的伤害让秦鼎菱大吐一口鲜血。

%70
“秦前辈!”紫薇仙子连忙扶住秦鼎菱。

%71
秦鼎菱站稳后,双眼仍旧在不断流血,无法止住。她暂时不能视物,苦笑道:“龙公前辈造诣惊人,道痕规模庞巨,我动用察运杀招,只能勉强一瞧。”

%72
秦鼎菱虽有察运的手段,高达八转层次,但具体实施起来,结果也因人而异。

%73
就像龙公,他本身的道痕就远超八转巅峰,身上又刻有龙纹,形成防御杀招,时刻起效。

%74
“受苦了。”龙公问,“结果如何?”

%75
秦鼎菱摇头,面色忧愁:“不太好,龙公前辈此行必有一番激战,甚至有可能失败。”

%76
“这怎么可能?”紫薇仙子惊诧。

%77
“没有什么不可能的。”龙公摇头,“天地仍旧广博浩瀚,光阴长河奔流不息,人不过是沧海一粟罢了。这下有趣了,我很好奇究竟是什么,能令我失败。”

%78
秦鼎菱又道:“气运乃是变数,这只是昭示龙公前辈此行,会有失败的可能。不过没有关系,我专修众生运,可用运道手段为龙公前辈增运、转运!不过……需要几天的时间。”

%79
“好,那我就不妨多等几天。”龙公思考了一下,做出了决定。

\end{this_body}

