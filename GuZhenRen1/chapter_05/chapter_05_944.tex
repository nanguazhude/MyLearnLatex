\newsection{凤九歌的红莲真传}    %第九百四十八节:凤九歌的红莲真传

\begin{this_body}



%1
命运歌声笼罩战场,但这一刻,凤九歌却是对天庭下手。

%2
不管敌我双方的震惊和疑惑,凤九歌沉浸在自己的歌声里,同时,心头的记忆也再度浮现。

%3
第二次进攻琅琊福地……

%4
方源在福地中布阵,凤九歌和陈衣被大阵隔离。

%5
凤九歌单独对战方源。

%6
“我的蛊虫的确是莫名其妙的消失了!怎么会这样?这种手段,应该是偷道的手段。只是不知道这是方源出手,还是这个超级大阵的威能?”凤九歌惊骇。

%7
因为方源的偷道手段,他陷入下风。

%8
伴随着时间推移,他一只只蛊虫不断被偷,其中就包含数只音道仙蛊。

%9
凤九歌战力大降,不得不决定利用定仙游仙蛊,先行撤离。

%10
镇宇仙蛊!

%11
方源忽然动用早就准备好的手段,令凤九歌撤离的愿望落空。

%12
仙道杀招——大盗鬼手!

%13
随后,方源终于将定仙游蛊偷了过去。

%14
凤九歌心中一片冰冷,心知自己此刻已陷入绝境!

%15
然而这个时候,他却接到了方源的暗中传音:“凤九歌啊,我不会杀你的,也不会盗取你的命甲仙蛊。你和我不是一路人,但你和天庭也绝非一路。正视你自己的内心吧,你真正想要的是什么?再问你一句,命运歌你开始领悟了吗?”

%16
凤九歌起先以为这是方源故布迷阵,扰乱他的战意和决心,但很快他就惊疑不定起来。

%17
因为方源真的放缓了攻势,没有再紧紧逼迫他。

%18
这让凤九歌不禁开始思考——

%19
“方源这段话是什么意思?”

%20
“他知晓我身怀命甲仙蛊?此次攻打琅琊福地,他应对如此出色,准备如此充分,难道是利用春秋蝉重生归来?”

%21
“若他真是重生,难道上一世发生了什么?他的话语里,似乎我会和天庭分道扬镳?”

%22
这个时候,方源再度传音:“凤九歌啊,我知道你心怀疑惑。不要紧,走下去,你会逐渐领会我的意思。接下来我会打开大阵,给陈衣造成是他打破的假象。我相信他会送你出去的。呵呵呵,后会有期了,凤九歌。”

%23
“方源,你给我说清楚了!”凤九歌传音。

%24
但下一刻,轰的一声巨响,两半的大阵空间被再次打通。

%25
陈衣看到凤九歌落入险境,立即惊呼一声,口中大叫:“坚持住,我来了!”

%26
“小心!敌人有直接盗取蛊虫的手段。”凤九歌犹豫了一下,也在瞬间反馈给陈衣关键的情报。

%27
交战中,方源辣手施威,重创凤九歌,激发了命甲仙蛊。

%28
陈衣为了救下凤九歌,催动了来因去果杀招。

%29
凤九歌震惊:“方源所言没错,陈衣真有送我出去的手段!”

%30
“凤九歌这一次让你逃了,下一次可就不一定了。”方源脸色如常。

%31
这话听在凤九歌的耳中,却是饱含深意。

%32
……

%33
琅琊福地一役,天庭战败。

%34
战后,凤九歌一直在思考方源的话。

%35
“他明明可以动手杀了我,但却没有取走我的命。为什么?”

%36
“方源乃是魔头,性情凶残狠辣,最重实利!他放我一马,只有一个可能,那就是我对他会有很大帮助。”

%37
“可我明明是中洲一员,天庭候补,我女儿凤金煌的护道人!难道说……天庭和我之间会有冲突?”

%38
“如果方源重生归来,难道在他的上一世,我和天庭作对,是他的天然盟友么?”

%39
心中的这些顾虑,让凤九歌没有将情况告知紫薇仙子。

%40
更令他触动的是方源提出的命运歌。

%41
“命运歌……”凤九歌隐隐感觉到,有一些灵感,但这些灵感太过缥缈和灵动,他根本捉不住。

%42
但他明白这会是一个方向。

%43
他不断摸索,很快就有不少碎片似的心得体悟。

%44
终于有一天,当他驻足在绣楼之下,看到绣楼和狂蛮魔尊的三张血皮。

%45
最近这段时间来的体悟,猛地发生了质变。

%46
凤九歌的灵魂深深触动,他不禁微笑起来:“看来我的下一首歌,便是——命运歌了!”

%47
……

%48
光阴长河之战。

%49
方源和凤九歌再次遭遇。

%50
“命运歌开创得如何了?”表面上激战,暗地里方源却是传音凤九歌。

%51
凤九歌:“方源,你究竟知道一些什么?”

%52
“呵呵呵。多进光阴长河看看,你也当发现在这里领悟,对你开创命运歌会有极大的帮助。不过最有帮助的还是石莲岛上的红莲真传。”方源道。

%53
“什么意思?”凤九歌询问。

%54
“根据我的猜测,那应当是红莲魔尊特意留给你的真传,对你会极有帮助。”方源道。

%55
凤九歌犹豫了一下,也回道:“我可不会凭白欠下你的人情。那我也告诉你好了,在天庭中绣楼的上空,有着狂蛮魔尊昔日留下的三份真传。每一份真传都有一个引子,分别是态度蛊、变异蛊、变通蛊。”

%56
“哦?!”方源眼中精芒一闪即逝。

%57
他没有觉得凤九歌在说谎,东西都在天庭里面,这没有什么好哄骗他的。当初凤九歌为了偿还大同风下的救命之恩,会为他挡住武庸的追杀。如今告知真传情报,也是凤九歌性情所能做出的事情。

%58
这一战,方源仍旧放水,而凤九歌生还。

%59
……

%60
中洲炼蛊大会召开,数处战场开辟,爆发蛊仙的大激战。

%61
在这样关键的时刻,凤九歌却是深入光阴长河。

%62
“那……真的是石莲岛?”凤九歌身心震动,迷雾中石莲岛主动出现在他的眼前,正如方源之前指点的那般。

%63
他上岛见到了红莲意志。

%64
红莲意志微笑:“你终于来了,凤九歌啊,我的这份真传是专门留给你的。”

%65
凤九歌凝眉:“即便你馈赠再多,恐怕也改变不了我的立场。”

%66
红莲意志摇头:“我从未想过要贿赂你,你该怎么选择,那是你自己的事情。我只是觉得这份真传应该给你。当然,你接不接那是你的事情。你就算摧毁它,我也不会反抗分毫。”

%67
凤九歌相信自己,思考片刻,决定接受这份真传,他沉浸到红莲生前的某段人生的记忆之中。

%68
红莲在渡劫,冲刺九转至尊境界。

%69
灾劫威力之恐怖,令凤九歌惊骇欲绝!

%70
灾劫终于渡过,帮助红莲渡劫的多位天庭蛊仙,只幸存了一半。最令红莲痛惜的是柳淑仙的死。

%71
这是他最爱的女人。

%72
“不要离开我,淑仙!”红莲紧紧地抱住柳淑仙,泪流满面。

%73
柳淑仙微笑:“没有用的,我身中灾劫,此刻能弥留一丝魂灵,看你最后一眼,已经是天大的幸运了。我又岂能奢求太多呢。”

%74
“是我没用,是我没用!我来渡劫,却连累了你!”洪亭垂首,泪水滚滚。

%75
“不,洪亭。那样的灾劫,只有我这样的特殊体质,才能去阻挡。你们就算付出全部身家性命,也只有失败的结果。我能出生,拥有十绝体,和你相遇,都是宿命的安排。在你九死一生的那一刻,我忽然明白,我此生最大的意义就是保护你,替你阻挡灾劫,助你登临仙尊之位!现在……我做到了。”

%76
“不,不!仙儿,我宁愿不要这仙尊之位,我只想你活着,我只要你活着啊!”洪亭无助地嘶吼着,泪流满面,浑身颤抖。

%77
“万事万物皆有定数,都有各自的宿命。洪亭,你不能这么想,你要好好活下去,你的宿命就是成为仙尊,领袖天庭,将正道的光辉照耀五域……你知道么,我一直期待着这样的情景,期待着站在你的身侧,陪伴你无敌天下,造福世间。可惜,我看不见了……”

%78
柳淑仙气息渐消,彻底死去。

%79
洪亭垂首,腰背深深弯弓,宛若老朽,浓重的阴影笼罩住他的面庞。

%80
这一刻,他像是失去了全部的生命的气息。

%81
痛不欲生,哀莫大于心死!

%82
悔蛊在红莲的仙窍中产生,他决定改变这一切的现实。为此他不惜和龙公,和天庭决裂。

%83
背道而驰!

%84
红莲利用春秋蝉重生,十多次、上百次。

%85
灾劫徐徐消散,这一次柳淑仙虽然身受重伤,但却还有一口气残存着。

%86
“仙儿,你还活着,还活着,真的是太好了!”红莲狂喜。

%87
噗。

%88
柳淑仙忽然口吐鲜血,散了最后一口气。

%89
“仙儿!!”红莲惊愕,呆呆地看着柳淑仙的尸躯,眼眶泛红。

%90
“我还可以的,我看到了希望。只要我继续努力下去,一定能够变得足够强大,能护住仙儿的命!”红莲目光凶狠,像是魔怔了一般嘀嘀咕咕。

%91
又重生,再重生,不断地重生。

%92
利用重生的优势,他越来越强,处理各种事物得心应手。他经验丰富,充分地利用每一份资源,最大限度地抬高自己的实力。

%93
然而就像是一个又一个的轮回,每一次他都得面对成尊的灾劫。

%94
灾劫的类型和威能,竟然随着红莲的改变而改变!这导致灾劫之后的结果,从未发生变化。

%95
该死的那些蛊仙,都会死亡。其中就包括柳淑仙。

%96
红莲不断地尝试,毫不气馁。十次百次、千次万次!

%97
他分析,他计算,将手头上的每一份修行资源都仔细规划,对于天庭的助力他同时竭力索求。

%98
……

%99
“仙儿!”红莲抱着柳淑仙。

%100
柳淑仙看了他最后一眼:“你没事真的太好了。”说完,她就没有了气息。

%101
……

%102
“仙儿!”红莲又抱着柳淑仙。

%103
柳淑仙没有力气说话,她奋起余力,想要抬起手抚摸红莲的面颊,但她终究没有成功。在半途中,她的手就无力垂下。

%104
……

%105
“仙儿!”红莲怒吼,双眼瞪大,眼睁睁地看到柳淑仙在雷霆中化为齑粉。

%106
……

%107
“淑仙。”红莲望着中了剧毒的柳淑仙,脚步不由放缓。

%108
柳淑仙全身黑紫,从七窍中不断外渗毒血,她惨笑道:“洪亭,别为我伤心。这一切的牺牲都是值得的。你可要成为尊者,带领天庭,领袖人族啊。”

%109
这是她最后的遗言。

%110
红莲顿足,远望着柳淑仙的尸体化为一滩毒水。他紧紧握拳,狠狠咬牙:“又死了!我还得继续努力!!”

%111
……

%112
“淑仙!”灾劫之后,红莲向柳淑仙飞奔而来。

%113
柳淑仙摇头,面色苍白:“我就要死了,看来我不能再陪伴你了,我的挚爱啊。”

%114
“你死不死,得看情况。我先查查!”红莲不甘心。

%115
“我的情况我知道,你听我说,在生命的最后,我想告诉你……”柳淑仙气息迅速衰落,说话断断续续。

%116
红莲充耳不闻,一门心思检查。

%117
检查的结果让他断绝了治疗的希望,当他反应过来,怀中的柳淑仙早已经没有了气息。

%118
“一定有什么方法,我还可以继续改进!”红莲提醒自己。

%119
……

%120
一次次尝试,一次次失败。

%121
洪亭就像是被困在了一个死胡同里,不管用什么方式,哪怕不去渡劫,柳淑仙都难逃一死。

%122
凤九歌看着红莲一次次的失败,又一次次尝试,一股强烈的悲怆和哀伤之感,不断地在胸中积蓄。

%123
他看着红莲咆哮,看着红莲激动,也看着红莲不甘心地咬牙,看着红莲含恨而走,然后一次次重生。

%124
每一次红莲都是怀着希望,收获失望。

%125
成尊的灾劫无法躲避,无法放弃,无法被人为掌控。

%126
柳淑仙的死,就像是一堵无法跨越的墙。红莲每一次重生,都要接受这样的残酷结果。他每每主动的尝试,到最后都是看到自己最心爱的女人惨死。他一次次的受伤,心中的伤更甚肉体的伤痛,而偏偏红莲每一次都主动重生,就好像是拿着自己的心口,主动往那柄利刃上撞去。

%127
凤九歌虽然对红莲充满了同情和敬意,但他时刻警示自己,不要太过同情,这或许是尊者的手段。

%128
身为旁观者,凤九歌一直保持着冷静,他逐渐发现:在起先重生的阶段,红莲会对这样的结果咒骂苍天,痛不欲生。

%129
而随后的阶段里,红莲不甘心,不断冷静分析,神情阴郁。

%130
到了最后阶段,红莲效率越来越高,每当发现柳淑仙救治不了,就立即选择重生,毫无犹豫。

%131
终于有一次。

%132
柳淑仙坠落在地上,摔得全身骨折,倒在血泊中,只剩下最后一口气。

%133
红莲飞速降落到地,迅速赶来。

%134
“红莲……我要去了,你一定要……”柳淑仙微笑着,看着红莲快步走到她的跟前。

%135
红莲一脸冷漠,淡淡地看了她一眼:“没有的救了。这一次又失败了。但是没有关系,我不会放弃,我会再重来一次。”

%136
柳淑仙惊愕:“洪亭,你在说什么?”

%137
红莲转身即走,身后传来柳淑仙的最后呼唤:“洪亭……”

%138
红莲起先脚步飞快,但听到柳淑仙的这声凄切呼唤,他的步伐越来越慢,然后他停顿住。

%139
他低下头,望着自己的双手。

%140
在死寂的沉默中,红莲瞪大双眼,身躯开始微颤。

%141
他似乎忽然间看到了自己,因此,他感到了极度的震惊,更有一种淋漓的恐惧。

\end{this_body}

