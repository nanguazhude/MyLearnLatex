\newsection{食道仙蛊小吃}    %第四百三十八节:食道仙蛊小吃

\begin{this_body}

%1
一只小巧的食道仙蛊,摆放在方源的手掌中央。

%2
它通体嫩绿,只有成年人的手指长短,但身躯分有六节,像是一头毛毛虫。

%3
只不过,在其额头上,长着一个细细的白银尖角。

%4
此刻,它趴在方源的手掌中,不断地蠕动,跟随方源心意,已经是被方源炼化。

%5
炼道蛊仙一旦炼成了仙蛊,绝大多数情况下,这只仙蛊就已经被炼道蛊仙炼化。也就是说,毛六是这只食道仙蛊的主人。

%6
不过,在毛六主动渡让之下,方源炼化这只食道仙蛊,远比炼化野生仙蛊要轻松太多。

%7
六转食道仙蛊——小吃。

%8
这是方源嘱托琅琊派替他炼制的指标之一。

%9
他在继承了紫山真君的遗藏之后,就拜托琅琊派上下,努力为自己炼制三只仙蛊。

%10
现如今,它们都已经全部完成,分别是七转仙蛊自爱,六转仙蛊净魂,以及六转仙蛊小吃。

%11
这三只仙蛊,都是方源急需之物。

%12
自爱仙蛊用处很大,乃是洁身自好杀招的核心,不可缺少。方源能够摆脱天庭的侦查,全靠了它。

%13
净魂仙蛊虽是六转,但炼制的难度和时间,是三只仙蛊中最高最长的。不过它炼成之后,帮助方源解决了力道八臂仙僵肉身上的魂道陷阱,使得方源再次能够利用智慧光晕。它的功劳也极其巨大。

%14
至于小吃仙蛊,它是食道仙蛊。方源为了解决自身仙蛊的喂养问题,所以对它需求很大。

%15
小吃仙蛊的仙蛊方,来源于影宗传承。其源头要追究到幽魂魔尊的身上,在他生前,他冒死从兽人蛊仙那里,继承了食道真传。

%16
而那位兽人蛊仙,正是开创食道的传奇人物。

%17
前人栽树后人乘凉,方源手中的这个食道真传,乃是天底下最为正宗的版本。

%18
试着灌输红枣仙元,小吃仙蛊毛毛虫般的身躯,顿时一颤,然后它像是喝醉了酒一样,后半身趴在方源的手掌上,然后前半身竖立起来,不断地摇头晃脑。

%19
在摇晃的过程中,它原本嫩绿色的身躯,不断地变红,最终变成岩浆一般的红色。

%20
同时,它的身躯原本细长宛若人的食指,也随之渐渐变粗变硬。

%21
噗嗤。

%22
最终,它从嘴巴里喷出一大股粘稠的橙黄色浆水,然后迅速地萎靡下来,身躯缩小,比原先正常的体型还要小一些。同时身躯的颜色,从岩浆红,转变成了苍白色,给人虚弱无力的感觉。

%23
有着食道真传,方源心中清楚:小吃仙蛊有催用的限制,催动一次之后,必须等到它缓过劲来,才能继续催动。若是连续催动的话,会让它支撑不住,吐出大量的橙黄浆水之后,迅速死亡。

%24
而只有等到它恢复过来,就是全身再次恢复到嫩绿的颜色之后,它便能继续催动了。

%25
方源将小吃仙蛊收入仙窍当中去,然后目光投向它喷出来的那股橙黄浆水。

%26
这股浆水非常奇妙,自行悬浮在半空中,并不坠落,并且已经缩成了一个小圆球。

%27
若是打个比喻,它就好像是一个健康的鸡蛋,把蛋清和蛋黄都搅拌在一起。

%28
这个橙黄色的浓稠浆水,才是方源想要得到的东西。

%29
有了它,就可以取代一部分的食材,用来喂养仙蛊了。

%30
对于其他的六转仙蛊而言,这一股橙黄浆水,可以取缔百分之六的食材(七转且以上,效果折扣)。百风之六看似很少,但是几天后,小吃仙蛊就可以再次催动,喷发出第二股橙黄浆水了。

%31
如此累积下来,对于喂养仙蛊的帮助极大。

%32
当然,橙黄浆水的效果也有限制。

%33
最多只能替代百分之四十的仙蛊食材,超过这个数目,喂养仙蛊再多的量,也没有效果了。

%34
方源随意取出一只仙蛊,它很快就吞下了这团橙黄浆水。

%35
试验完美。

%36
方源满意地点点头,心中也有一些感慨。

%37
他之前早就觊觎食道的手段了,因为重生以来,仙蛊的数量很多,喂养的重担积压在他的肩上。在此之前,很长一段时间里,方源都在为喂养仙蛊而奔波。

%38
“没想到,我最终会从大敌影宗的身上,获得食道真传。而且是天底下最为优秀的食道真传!”

%39
“而且炼制小吃仙蛊的过程,比净魂、自爱要容易的多。”

%40
炼制仙蛊这种事情,真的说不准。

%41
方源为了炼制净魂,失败了多少次,但是炼制小吃仙蛊,前后不过五次罢了。

%42
第五次就成功了,这是一个惊喜。

%43
有了这只小吃仙蛊,方源喂养仙蛊的压力,就大大减缓了许多。六转仙蛊能削减四成的食材,让方源用仙元替代喂养。七转、八转的效果虽然更小一些,但担子真的轻了不少。

%44
要知道,方源现在手中的仙蛊,可是很多的。

%45
小吃仙蛊对他而言,性价比非常的高。

%46
几天后,方源从宝黄天中,成功地购进了一批荒植。

%47
这批荒植乃是一层浅绿色的菌苔,附着在七彩斑斓的珊瑚群的表面上。

%48
正是光照菌。

%49
光照菌来自于东海的深海之中,那里一片黑暗,毫无光线可言。

%50
许是物极必反,光照菌就在这样的环境中诞生,散发出强烈的光辉,将原本漆黑如墨的海底深处,照耀得宛若白昼。

%51
方源曾经在东海的交易会上,就像收购一批过来,但当时拥有者女仙童画,只换不卖。

%52
她在事后,将光照菌卖到了宝黄天中去,结果经过一段时间的市场评估后,卖家越来越低。

%53
起初,蛊仙们对这种荒植都非常感兴趣。

%54
但是后来,买家们发现,这种植株最适合栽培的环境,只有是漆黑如墨的深海海底。那里不仅没有关系,而且水量无穷,压力巨大。

%55
脱离了这个环境之后,光照菌虽然不会死亡,但也萎靡不振,散发出来的光照,往往只能比拟烛光。

%56
在这样的情况下,方源和童画洽谈,最终以很低的价格,将她手中的光照菌存货几乎全部购买了过来。

%57
方源首先将其中的一批,置入仙窍小东海中养殖起来,作为备份。

%58
然后,他在另一批上开始试验。

%59
他需要改造这种荒植,令它脱离原来的生活习性,变得为自己所用。

%60
方源的木道境界是非常普通的,但是,他却很有自信。

%61
一方面,是他手段众多,影宗传承再加上琅琊派的真传,大量的改造手段,就囤积在方源的手里,其中涉及木道、魂道、炼道很多流派。

%62
另一方面,方源还有一个大杀器,那就是智慧光晕。哪怕是这些手段都不适用,方源也可以自己寻找出路。

\end{this_body}

