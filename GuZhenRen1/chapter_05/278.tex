\newsection{换蛊}    %第二百七十八节:换蛊

\begin{this_body}

交易会正在进行当中。\&\#26825;\&\#33457;\&\#31958;\&\#23567;\&\#35828;\&\#32593;\&\#119;\&\#119;\&\#119;\&\#46;\&\#109;\&\#105;\&\#97;\&\#110;\&\#104;\&\#117;\&\#97;\&\#116;\&\#97;\&\#110;\&\#103;\&\#46;\&\#99;\&\#99;]

庙明神作为本次交易会的牵引者,本来是高风亮节地居于最后一位登台。

但是,又因为引进了新人方源,所以方源才成为最后一位。

导致庙明神倒数第二位,登台交易。

这点在方源看来,却正可看出庙明神的交际手腕。

交易会当然是名次排在前面的蛊仙,最为有利。庙明神主动位于最后一位,示好所有蛊仙。

但他又不将方源,安排在他的前面。

这是因为规矩。

新人必须是最后一位。

若是违背了这个规矩,就会让人反感。这次交易是庙明神牵引主持,那么下一次呢?

所谓交际手腕,说起来,也没有什么,就是把握与人交往的分寸。

庙明神恰恰将这个分寸,拿捏得非常到位,让和他相处的人都能产生好感。

不过,现在方源的心思,并不在庙明神本人身上,而是将炯炯有神的目光,集中在了庙明神手掌上的那只仙蛊

七转,卜卦龟背蛊!

方源怦然心动。

为什么呢?

因为方源的很多手段,都不能轻易曝光。比如说上古剑蛟变化,又比如说逆流护身印。这些手段一旦使用出来,除非方源及时地杀人灭口,否则的话,就会暴露更多信息,甚至让蛊仙们追查出他更多的秘密,最终让他无路可走。

说起来,有点尴尬。

方源的实力不断增强,但是他惹的祸,也越来越大。

影宗、天庭、长生天……瞧瞧他的这些对手、敌人,哪一个不是巨无霸般的势力?

最近这段时间,他还惹上了雪胡老祖。那可是北原八转第一人!

所以,方源的压力也很大,他还没有成长到九转。

也只有修行到九转,方源才能堂而皇之地公布身份,天下在无人可挡。

在九转之前,哪怕是方源修上了八转,天庭、长生天等等,都有能力来找方源的麻烦。

目前为止,这是七转修为的方源,还需低调行事,伪装形象,迅速修行,继续积蓄实力。

“我伪装成武遗海,也有一个弱点。或者说,一个破绽……”

方源不能在伪装成武遗海的时候,当众出手作战。

因为他不是真正的武遗海!

通过搜魂武遗海,方源知道,武遗海的变化杀招有三个。一个是珊瑚形态,一个是海鸥变化,第三个则是乌龟模样。

在之前,界壁中交手的时候,武遗海就变化成乌龟形态,进行防御。

方源记得清清楚楚,那是角神龟。上古荒兽,甲壳坚硬,在东海的上古荒兽中名列前十的存在。

其实,这一点,从武遗海的仙窍福地中,也可见端倪。

武遗海的仙窍福地,是一片汪洋。里面最多的就是三类生命,第一是珊瑚,有上古荒植大片的静音珊瑚群。第二是海鸥,有荒兽白信蓝羽鸥。第三是乌龟,有上古荒兽角神龟。

身为变化道的蛊仙,往往会豢养一些自己精修变化的相关生命。

第一,这些生命可算是资源,可以切合变化道的修行。比如说武遗海豢养的角神龟,他就修行这种变化,今后需要什么仙蛊,完全可以宰杀了这些角神龟,从而炼制出来。

第二,变化道蛊仙要提升战力,就要对这些他变化成的生命,有着清晰而又深入的了解。豢养它们在自己的仙窍当中,即可时刻观察,进行模拟,最终变化之后,以假乱真,战力增强。

至于方源,这完全是一个个例。

方源的变化道境界,是通过梦境和渡劫,而迅速拔升上去的。

境界一提升,就影响修行的方方面面,其中就包括这一点。

所以,方源就省去了枯燥乏味,又耗时过长的观察和练习的环节了。

“若是能够有这样一只仙蛊,对我而言,当下即有用处了。”

方源越是琢磨,越是心动。

这卜卦龟乃是智道上古荒兽,有一项本事,天生能防备推算。很多蛊仙想要捕捉这种上古荒兽,靠智道推演算出它的位置,那是非常困难的事情。

若是方源变作卜卦龟的话,对自己出行,非常有帮助。

总结下来,这只仙蛊对于方源而言,就是两个字实用。

若再添几个字,那就是太实用了!

对于旁人而言,七转卜卦龟仙蛊,或许只是鸡肋。毕竟用它来护身,只护住后背。若是看中的附加优点,可以防备他人推算的话,也不见得有多效,反倒不如其他手段。

所以应用范围并不广。再说,寻常蛊仙,谁有事没事地来推算你?你当人人都是柳贯一啊!

方源不一样。

方源树大招风,恶名传扬五域,就连伪装的身份,都是如此。

还有一点,方源可以变化成完整的卜卦龟,借助变化道痕,这样一来,就是仙道杀招。防备他人推算的这项威能,自然比单个的卜卦龟仙蛊,要强悍得多了。

方源扫视周围群仙一眼,叫他较为安心的是,似乎这些人对于这只仙蛊的兴趣,都不太浓厚。

仙蛊的交易,历来只是以蛊换蛊。

这是因为仙蛊唯一,价值难定。仙蛊对于个人,完全是看具体情况。同一只仙蛊,有的人宁愿放弃,觉得是个拖累,毕竟喂养仙蛊,也要代价不小。有的人却是千方百计地想要追求,辗转反侧,誓不罢休,一旦得手,如获至宝。

“仙蛊啊。”

“没想到第一轮,就有仙蛊交换。”

蛊仙们感慨,小声议论,但都安稳地坐在座位上,没有动弹。

庙明神微笑着。

这种情况,他其实早有预料。

在场的蛊仙们,变化道流派的很少,似乎只有楚瀛一人。但就算是有,楚瀛也未必喜欢这只仙蛊。

变化道蛊仙常常是专修两三个变化,贪多嚼不烂。除非是获得变化道的精髓之一变形仙蛊,才能任意变化形态。

不过庙明神此举,自有他的用意。

这场交易会,毕竟是他牵头的。在第一轮就出现仙蛊交易,这个消息传扬出去,更会增添交易会的格调,将来吸引更多更有底蕴实力的蛊仙参与进来。

“我愿换这只仙蛊。”方源缓缓地从座位上站起身。

一时间,群仙都纷纷侧身,看着方源,流露出微讶的神色。

庙明神微微一愣,旋即反应过来,脸上笑容更盛。

“不知庙明神仙友想要换什么?”方源问道。

庙明神呵呵一笑:“楚瀛仙友是第一次参加交易会,又对我有恩。这样吧……其实我自己不大用得上这只仙蛊,换什么都行。”

这次轮到方源楞了一下,他没想到,庙明神给出的条件竟然如此宽松。

不过即便如此,仙蛊交易向来只能用仙蛊换仙蛊,用其他东西,是不成的。

方源沉思了一下:“我这里有飞熊之力仙蛊……”

“换了。”方源话还未说完,庙明神就毫不犹豫地道。

方源不由地再次打量一眼庙明神,心想此人难怪在东海蛊仙界,混得风生水起。明明是一个散修,但是身边却有好几位蛊仙,对他忠心耿耿!

不过有些话,方源还必须得说清楚。

他的脸上露出苦笑之意:“仙友之心,我感且佩。不过我的话,还未说完,我的这只飞熊之力仙蛊,只有六转罢了。”

庙明神神色一变。

座位上的群仙,脸色也纷纷沉下来,一些打量方源的目光,带着一些阴沉。

飞熊之力,乃是力道仙蛊。卜卦龟背仙蛊,却是属于变化道。

力道式微,变化道却是兴盛不衰。这两者之间,本来就是飞熊之力,要弱于卜卦龟背仙蛊一筹。

现在,飞熊之力居然只是六转。

六转仙蛊怎能和七转仙蛊,相提并论?

价值完全不等。(未完待续。)

\end{this_body}

