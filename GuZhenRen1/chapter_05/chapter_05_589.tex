\newsection{三种方案}    %第五百九十一节:三种方案

\begin{this_body}



%1
至尊仙窍。

%2
小赤天。

%3
方源宙道分身刚一出现,就立即引起无数年兽的察觉。

%4
啊哦哦!咯咯咯!吼吼!

%5
无数的年兽发出兽吼,纷纷向方源扑来。大量上古荒兽,将荒级年兽挤到一旁,目光中流露着渴望。

%6
但很快,一头太古年鸡雄赳赳、气昂昂地扇翅过来,一路上荒级年兽、上古年兽都慌忙避让,场面一片混乱。

%7
太古年鸡跑到方源宙道分身面前,顿时低下头,露出温驯之状。

%8
方源微微一笑,伸手抚摸了一下太古年鸡头顶,那赤红夸张的鸡冠,然后大袖一挥,洒下无数的凡级年蛊。

%9
太古年鸡咯咯一叫,睁开鸡嘴,爆发出一股强劲的吸力,将绝大多数的年蛊都吞吸入腹。只有很少的年蛊,洒落在周围,引发周围上古年兽的一阵哄抢。

%10
太古年鸡大吃一顿,却仍不满足,又盯着方源的宙道分身猛看,一双眼睛闪闪发亮。

%11
方源大袖连挥,海量的年蛊好似河滩上碎小的石子一样,喷洒而出,形成漫天的暴雨。

%12
这一下人人有份,年兽们顿时陷入了狂欢。

%13
良久之后,包括太古年鸡在内,所有的年兽都饱餐一顿,之前因为饥饿的微微躁动消散一空,又恢复了懒洋洋的状态,趴在一朵朵白云上,进入休憩的状态,不再动弹。

%14
方源的宙道分身扫视一周,微微点头,便疾飞而去。

%15
在这小赤天中,类似此处的年兽群落共有七个,这已经是方源喂养的第三个群落了。

%16
在至尊仙窍中,豢养的年兽本来就很多,不久前方源利用年流伏诛阵,狠狠地算计了南疆正道一把,俘虏了南疆群仙之外,最大的一批战利品就是在战斗中镇压收服的年兽了。

%17
这些年兽的品质都很高,至少是荒级年兽,大量的上古年兽,以及七头太古年兽。

%18
原先,方源的魂魄底蕴,只能奴役一两头太古年兽。不过现在他在之前魂道修行中突飞猛进,虽然最近一直在下滑,但打下的底蕴仍旧残存大半,奴役七头太古年兽不在话下。

%19
值得一提的是,这七头太古年兽当中,太古年鸡数量最多,足有三头。其他四头,分别是太古年虎、太古年龙、太古年兔、太古年马。

%20
光阴长河中,每隔一段时间,太古年兽中的一种,都会数量比较多。显然最近这段时间,是太古年鸡繁盛的时期。

%21
不过,方源掌控的这些太古年兽虽然多达七头,但是单独拉出一头来,深究战力的话,都不如方源之前掌控的那头太古年猴。可惜那头太古年猴已经战死沙场。

%22
当然,所有的七头太古年兽战力叠加起来,自然比一头太古年猴生猛得多。

%23
太古年兽之间,存在着战力的明显差异。这就好像是八转蛊仙之间,也存在着差异。

%24
方源一边喂养这些年兽,一边思量:“新添了这么多的年兽,喂养方面的代价,直接暴涨了数十倍!若非此次勒索了一批仙元石,恐怕都喂养不起来。”

%25
年兽的层次越高,食量就暴涨。而喂养年兽,采用的食料,那就是年蛊。

%26
方源通过八转仙蛊似水流年,炼出海量的凡级年蛊,堪堪支撑住这些年兽的肚皮。

%27
而催发似水流年仙蛊,消耗的就是方源的红枣仙元。

%28
之前,方源在喂养年兽方面,仙元就消耗不少。现在增添了这么多的年兽,太古年兽就有七头,因此在年兽喂养方面,已经成了方源支出的最大项目。

%29
“目前,荡魂山已经被修复到了八成,胆识蛊的产量已经不再是惨不忍睹,而变得差强人意,勉勉强强都能接受。”

%30
“等到荡魂山彻底修复,却是年兽方面成了新的拖累。”

%31
方源每隔一段时间,就要喂养年兽。通过似水流年炼出年蛊,已经是最节省,最经济的办法,但仍旧是巨大的负担。

%32
而方源又需要养住这批年兽,好应付将来的光阴长河之战,所以年兽的喂养问题就需要解决好了。

%33
其实这个问题,不只是年兽,还有鹰群。

%34
方源曾经为了奴役上极天鹰,购买了大量的鹰兽。这些鹰兽的喂养,也一直是个负担。

%35
要想让它们自给自足,真正地纳入到至尊仙窍这个世界中,那就需要方源耗费大量的精力和资金,专门搭建出一个完整的食物链。

%36
大鱼吃小鱼,小鱼吃虾米。

%37
完善出一个食物链,也是蛊仙对天道的一种深刻探究和理解。

%38
不过相比较鹰群,年兽规模大得多,最需要解决的就是年兽的喂养。

%39
而年兽因为本身的特殊,搭建出一个完整的食物链,更是相对容易得多。

%40
掌握着幽魂真传、琅琊真传等等,方源面对难题,总有解决方法。

%41
如今,他有三个方案供其选择。

%42
第一个方案,是要建造年华池。

%43
这是一个资源点,也可以将它看做是一个微型的光阴长河。它的作用,就是模拟出光阴长河中的环境,供年兽们生存、居住、繁衍。

%44
年华池的规格,有大有小,分为六转、七转、八转、九转四个层次。

%45
这个方案的优点在于,节省空间,同时年兽在里面生存,会自产一定量的年蛊。经营得时间久了,还会产生其他野生宙道蛊虫,甚至是繁衍出日兽、月兽等等,最终形成一个微型的光阴长河。

%46
而弊端则在于,造价太过高昂。单单六转档次的年华池,就是一笔巨大的数字。换算成仙元石,不是百万级,也不是千万级,而是破亿级的仙元石!

%47
而七转年华池,更是夸张,涉及到很多的稀缺仙材,大部分有价无市。六转档次的年华池,财大气粗的话,还是可以搭建的。七转层次就不是仙元石能够解决的问题了。

%48
至于方源,他需要的是八转年华池,如此才能装得下那七头太古年兽!

%49
第二个方案,是流金岁月台。

%50
这是一座七转的宙道仙蛊屋,比较特别。因为它是需要固定的,不像其他绝大多数的仙蛊屋可以自由移动。一旦布置下来,流金岁月台就会幻化成一颗明月,始终笼罩在福地的天空之中,并且不断升落沉降。

%51
流金岁月台中,可以装载大量年兽,并且能使得它们安稳沉眠。如此一来,喂养的需求就剧减。

%52
这个方案的优点在于,流进岁月台可以抽取年兽的力量,为其所用,而不需要方源另外消耗仙元。这就在和平时期,让方源充分利用了年兽,不至于白白喂养它们。

%53
这个方案的弊端也很明显。

%54
首先,构造这座七转仙蛊屋流金岁月台,需要大量的宙道仙蛊。而方源却需要这些宙道仙蛊护身。

%55
其次,流金岁月台要尽量固定,令它依附在仙窍中自行运转。所以它就算搭建出来,也不能帮助方源,在光阴长河中作战。它本身也并非擅长作战。

%56
再其次,这座仙蛊屋消耗的是年兽的力量,若恣意抽取,达到极限,年兽就会死亡,身上的宙道道痕也会毁灭,年兽尸体毫无价值。

%57
最后,而当年兽醒来,则需要一笔远超寻常喂养的庞大食物,来填补它的消耗。

%58
第三个方案,则是十二生肖战阵。

%59
集齐十二个不同种类的太古年兽,以它们为核心,就可以组成十二生肖战阵。

%60
这座大阵长于攻伐,和流金岁月台的侧重点完全不同。它虽然是大阵,反而可以移动。

%61
因为核心是十二头太古年兽,平时的时候,这些太古年兽都封印起来,宛若石像,一动不动,不需要喂养。但战斗的时候,就会解封,结成战阵,上阵杀敌。

%62
很明显,这种以太古年兽为核心的大阵,并非是现代蛊阵,而是上古战阵。类似于青城纵横、天婆梭罗,只不过后两者是以蛊仙为核心,而这座十二生肖战阵,则是以太古年兽为核心。

%63
这个方案的优点在于,它真的能够节省一笔喂养的开支。

%64
但缺点在于,它只针对太古年兽。若用上古年兽、荒级年兽,方源是无法组成十二生肖战阵的。

%65
不过要做到这一点,方源还得要再搜寻到更多的太古年兽,并且这些太古年兽还要分门别类,各个不同。

%66
这种收集的工作,不仅看实力,也是很看运气,难度很大,消耗时间极长。

%67
方源经过深刻的思考,决定采取第一种方案。

%68
虽然第一种方案,是投入最多的,但若建设成功,也是获益最大的。

%69
“依凭我自己的实力,要建造八转档次的年华池,恐怕三成都建设不出,就要经济崩溃,陷入困境。”

%70
方源的资产其实非常雄厚,但他很多资产是无法变卖的,比如逆流河。而且,他还需要面对天庭等外在压力,个方面都要平衡兼顾,不可能孤注一掷,什么都不管,去建造一座年华池。

%71
“但就现在,确实我建造年华池的最佳时机啊。”

%72
“夏家,你们想和我谈?先送上三份漩涡晨曦来。”方源向夏家摊位投去信道凡蛊。

%73
很快,夏家就有了回应:“方源仙友明鉴,我族虽铲除漩涡晨曦,但其实产地就在太上大长老的仙窍内啊。我族库藏中可是毫无存量的。”

%74
方源微微一愣,这个回答出乎意料,但他旋即冷笑道:“这点我岂会不知?送上三份,不,十份漩涡晨曦,我才看到你族的诚意,我们才有的谈。否则……”

%75
“可是这漩涡晨曦非常稀缺,就算是在宝黄天中也有价无市。三份还罢了,十份的话,我族从哪里去找啊?”

%76
“这我不管!这就是你们的事了。”方源冷酷地回答最后一句。

%77
ps:告诉大家一个好消息,那就是本书会有繁体出版。等出了书,届时会做活动,抽奖送书!一直以来感谢大家的支持,明天、后天都会有加更。感谢缘分,让我们相识相知不相离!

\end{this_body}

