\newsection{大时代!}    %第九百五十一节:大时代!

\begin{this_body}

“凤九歌!”龙公咆哮,直奔凤九歌杀来。

仙道杀招龙爪击。

仙道杀招回旋龙牙。

爪痕浮现而出,纵横近十丈,割裂大气,所向披靡。龙牙纷飞,宛若穿花利刃,盘旋飞舞,杀气凛然。

凤九歌一边徐徐飞退,一边仍旧高唱着命运歌。

他身边涌现出数位分身,皆是他的模样,却各有奇妙威能。

正是仙道杀招一曲之士!

碧玉歌士能玉化一切,天地歌士有镇压威能,俯首歌士能奴役生灵,离歌之士能分解杀招、克制仙蛊屋,得宝歌士能炼化野生仙蛊,向天歌士能轻松跨越两天,仙魔歌士能将种种杀招、仙蛊威能夺来己用。

唯有大风歌,无法演变成歌士。

一曲之士们替凤九歌出手,不仅挡住龙爪和龙牙,更和龙公缠斗,灵动至极。

数个回合下来,龙公这才将一曲之士击爆。

但随后,又有全新的一曲之士赶赴战场,与龙公厮杀。

龙公立即明白,单靠铲除这些分身无济于事,真正明智的战术还是要直捣黄龙,直接对付凤九歌本体。

仙道杀招龙门。

龙公猛地跨越一大段距离,向凤九歌逼近。

凤九歌从容不破,催动杀招一曲阳关,再次拉开距离。

龙公无法和凤九歌近战,只得动用气道杀招遥攻。

凤九歌凛然不惧,口中命运歌不断削弱龙公,同时手中变化,连续施展掌钟、拳鼓、指哨,组成三绝音。

凤九歌对战龙公,双方竟打得你来我往,不分上下!

八转的未来身着实带给凤九歌战力方面的质变。

三域蛊仙正犹豫着,是否出手帮助凤九歌时,却是惊讶发现:凤九歌根本需要他人相助!

一时间,他展现出来的战力,惊艳了所有人。

龙公拾掇不下凤九歌,不禁又惊又怒。他这才明白,之前凤九歌虽然一直在作战,一直出手,但却是藏拙了,根本没有拼尽全力。

眼下,龙公想要破坏红莲之谋。但方源那边是硬骨头,龙公知晓方源的强大,又有狂蛮手段护身。所以他选择第一时间来找凤九歌的麻烦。

这绝不是龙公对方源心生惧意,而是考虑如何迅速地破坏敌方谋算!

只要凤九歌被干扰,命运歌无法发挥作用,那么方源那边就等若没有仙蛊方,根本无法继续图谋了。

但凤九歌的实力大大超出龙公的想象,龙公犹豫了一下,最终决定加紧猛攻,期望尽快收拾了凤九歌。

龙公被凤九歌拖住,吴帅操纵着龙宫护卫方源本体,劫运坛则抵挡住诛魔榜。

天庭蛊仙掀起攻势狂潮,但皆被其他三域蛊仙,以及帝藏生挡下。

战局明显偏向了三域蛊仙,但方源此刻却出现了一些问题。

人意如海,汹涌澎湃。

将这些人意充当主要蛊材,来炼制命运蛊,着实是异想天开!

但偏偏此法可行。

在命运歌的作用下,这些人意和火焰相互交汇,炼成一种无形无质的奇妙蛊材。

而方源主持这场炼蛊,却是要承受人意的恐怖冲击。他要将人意都安抚住,就像是制服汹涌的洪水,让它们按照条条沟渠而流淌。

方源很快就脸色狰狞,额头青筋直爆,脑海中念头疯狂催生,却跟不上这种恐怖的消耗,念头储量迅速减少,简直要干涸见底了!

战场边缘,魔尊幽魂主仆三人一直保持着关注。

紫薇仙子见到方源的神情,沉声分析:“不好,方源的情况不太妙。他的智道底蕴我清楚,他最强的智道手段其实是防备他人推算,遮掩自己的真正位置。眼下炼蛊却是考量他的念头和思考能力。这些人意规模着实恐怖,即便是我,恐怕也坚持不了多久。”

但魔尊幽魂却是轻笑一声,神态从容:“放心吧,有一位尊者早就为此做了准备。方源应该知道自己该怎么办。”

魔尊幽魂不禁回忆生前。

当时他本体健在,进入八十八角真阳楼中探索。

在那里,他发现了智慧蛊,也见到了巨阳仙尊遗留下来的巨阳特意。

他当即就想要动手,将智慧蛊收为己用。

“幽魂仙友,还请你不要收走了智慧蛊。”巨阳特意不得不现身,阻止道。

幽魂魔尊冷冷一笑:“你阻止不了我。”

巨阳特意点头:“当然,你是当代魔尊,天下无敌,我根本无法阻止。不过你也和红莲魔尊达成交易了吧?”

幽魂魔尊脸色微变,在此之前,他的确继承了一份红莲真传,和红莲魔尊达成了某项约定。

“你的意思是?”幽魂魔尊面色迟疑。

“没错,这只智慧蛊已经被我本体动过手脚。当关键人物出现时,他会捣毁八十八角真阳楼,智慧蛊也会跟随此人而去。”巨阳特意坦言道。

“原来是这样。”幽魂魔尊深深地看了一眼智慧蛊,终究是没有动手。

眼下。

“拼尽全力了么……看来还是得调动智慧蛊啊。”方源叹息一声。

到了他这一步,他早就猜测出了许多尊者手笔。

智慧蛊就是巨阳仙尊的手段,只不过和元始仙尊、元莲仙尊的手段不同,它早就发动了。只是方源当初没有发觉而已。

方源刚刚打开至尊仙窍的一丝门户,智慧蛊便自行飞出来,悬停在方源的头顶,绽射神光金辉。

方源顿时如释重负,神态恢复如常,感到轻松自如。

“原来如此。”他仔细体悟,又有发现,“智慧蛊也是蛊材之一,难怪如此见效。只是智慧蛊新添进来,我念头消耗剧减,仙元消耗却是剧增。这种吞噬体量,根本不是我能负担得了的。”

尽管至尊仙窍中光阴流速极快,积累仙元十分迅猛,但方源历经大战,消耗良多。

此刻一旦跟不上消耗,炼蛊戛然而止,方源必遭滔天之伤,恐怖反噬!

但方源却是临危不乱,镇定自如。

皆因他清楚,事情到了如此地步,尊者们自然不会坐视。

只要他仍有一丝利用价值,不可替代,尊者们就不可能让他折损在这个关节上。

“接下来,应该就是你要出手了吧,巨阳……”方源心中呢喃。

在遥远的黑天某处,有一座暗金宫殿,静静悬浮矗立。

宫殿中,巨阳仙僵缓缓地睁开双眼。

他望向南方,双眼倒映出天庭中的一幕幕战况。

“终于到了此刻。”巨阳仙僵微笑,又一声轻轻的叹息,“红莲呐,终究是你筹谋得力,布局更大,因而胜我一筹。”

生前的一幕浮现眼前。

石莲岛上,还未成就尊者的蛊仙巨阳,得到了红莲的馈赠。

同时,也和红莲达成了交易。

“这就是命运蛊的仙蛊残方?”巨阳接过这份残方,顿时被吸引进去。他开创运道,无往而不利,但从这份残方中他却是瞥见了新的天地!

“红莲,是你损坏了宿命蛊,方才有我开创运道的空间。现在你又将这份命运蛊残方赠与我,我欠你一个天大的人情!”巨阳蛊仙沉声道。

“我会依照约定,布下局面。将智慧蛊留下,我还会开创出应运仙蛊,帮助那位开创命运歌的天之骄子。并且,我还会留下最后的一击,只要局势达到那种地步,我必定会出手,助推你的百万年的大局收宫!”

红莲意志微笑:“那就多谢了。但如果我的图谋失败了,还请你按照我们的约定,自行出手,努力炼制出命运蛊吧。”

“那是当然的!”巨阳蛊仙神色坚毅。

……

镇云天宫中,巨阳仙僵缓缓地抬起右手。

轰!

一道巨大的光柱,横亘苍穹,以极快的速度向着天庭的方向轰射而去。

这道光柱是如此的庞大,就好像是一条巨河,横霸天地。

所到之处,排开一切的空气,光柱前行,发出恢弘的声响,震耳欲聋。

如此浩瀚磅礴的杀招,让所有人都为之失神。

饶是龙公、凤九歌、方源之流,面对此招都感到自身的微渺。

光柱无可阻挡,所向披靡,从北原上方的黑天发端,一路冲刷,刺破界壁,直接轰进天庭之中。

天庭剧烈震荡,地动山摇,无数仙蛊屋龟裂破散。

一缺抱憾亭中,星宿仙尊虚影咬牙。

她死死地望着巨阳一击的到来,自身却是被无极虚影死死纠缠,根本抽不出手来应对。

巨阳一击汹涌澎湃,灌输到漫天的人意和火海之中。

三者交汇,瞬间形成一颗庞巨如山的金色光球。

光球中火焰飘洒飞舞,无数人意色彩斑斓,绚烂多姿。方源横亘不动,位于最中央,主持大局。

一根根的苍白光丝,从他紧握的手中逐渐产生,蔓延而出。

这些光丝是如此的眼熟,方源对此印象深刻。

他心头微震:“是天道道痕!”

数十根、上百根天道道痕,从他手指缝隙间探伸出来。

方源索性彻底张开手掌,砰的一声,成千上万的天道道痕漫天飞舞,纠缠缭绕。

一条条道痕缠绕在方源的身上。

方源闷哼一声,剧痛来袭,让他身魂皆震!

“我的身上增添了天道道痕?等等,不只是我……”方源眼中精光烁烁,他敏锐地发现许多天道道痕无故消失,追溯人意的源头而去。

“果然如此。”这一刻,方源终于彻底确定红莲魔尊的打算。

红莲魔尊的确是要炼制命运蛊,但也不是真正的想要炼制它出来。

方源抓紧一切机会,开始全力收集这些天道道痕。

每一根天道道痕纠缠、烙印到他的身上,就会令他的至尊仙窍积累更深一层。

这是千载难逢的巨大机缘!

每一根完整的天道道痕,都能令方源的仙窍世界产生一点质变。

经历过疯魔窟,方源更清楚这份机缘的滔天价值。

并且越多的天道道痕,便越能防备魔尊幽魂在至尊仙胎蛊上可能暗下的手脚!

只是天道道痕的融合,十分剧痛难忍,即便是方源这样心志坚定的人,撑过十根完整的天道道痕之后,就疼得头晕眼花,痛彻心扉。

但他此时正要主持炼蛊,偏偏不能妄动其他手段,否则干扰了炼蛊,造成失败,那就更加不妙了。

方源只要靠着自己的毅力,咬牙支撑。

他不是忍受剧痛的唯一蛊仙,而是亿万之一。

与此同时,在广阔的五域之中,无数民众开始痛呼惨叫。天道道痕追根溯源,也烙印在他们的身上。

每一个普通凡人,几乎难以承受一条天道道痕的万分之一,他们许多人都在第一时间痛得昏死过去。

而清醒的人是有福的,他们可以承受更多的天道道痕的部分。

痛!

撕裂的剧痛,痛彻心扉,剧痛难忍。

方源很快就咬破嘴唇,双眼下意识瞪大,结果眼角都被瞪破,流出血来!

他浑身大汗,身躯都在发颤,仍旧竭尽所能,融合一根根完整的天道道痕。

很快,散落五域的蛊仙们也开始承受。

天庭的厮杀戛然而止,一个个蛊仙面色狂变,咬紧牙关,面色涨红。天道道痕主动加身,让他们猝不及防,又难以忍受。

武庸身躯狂颤,战力暴降谷底,只能勉强作战。沈从声不断呻吟,冰塞川死死咬牙,只能维持着劫运坛定住不动。诛魔榜中,方正、秦鼎菱已然直接瘫倒在地,身躯抽搐,就差疼的满地打滚。

“不!”天道道痕加身,龙公却是悲呼,到了这一刻,他终于明白红莲的谋算。

龙公心中焦躁万分,见久战不下凤九歌,只得转身,抱着万分之一的希望,想要对方源出手,但剧痛同样严重影响着他。

凤九歌仍旧坚持着命运歌,见到龙公动向,立即加紧攻势,反过来纠缠拖延他。

天道道痕加身,他们俩仍旧在战斗!

纠缠中,轰然一声,金水迸溅,如山般的人意陡然轰散。

大功告成!

“天下的万民啊,我把宿命都交给你们了。你们每一个人都掌握着一部分的命,还掺杂着运!从此宿命蛊仍在,但天下亦再无宿命!”红莲真意的话,通过人意,准确地传达到了五域每一个人的心中。

五域人族一片哀嚎,痛不欲生。

因为剧痛难忍,当场自杀的并不在少数。

“红莲!”龙公怒吼,终于摆脱了凤九歌,杀奔到了方源面前。

但一切都已经晚了。

最后的一丝红莲意志,勉强凝聚身形,挡在方源的面前,面对来势汹汹的龙公。

“师父。”少年红莲微笑,“我真的做成了。宿命蛊已经被我拆分成无数份,分发给了天下所有的人。它仍旧存在,并不算毁灭。并且和运结合,再不能为天意所用了。”

龙公怒得发狂,双眼通红,怒发冲冠。

“孽徒!”他咆哮一声,冲势更猛。

砰的一下,他彻底将红莲意志冲垮,悍然杀到方源面前。

这一刻,龙公的气势再次冲破极限,达到前所未有的最顶峰,强悍得让帝藏生、凤九歌都感到了一种生命的窒息!

方源一动不动,竟毫无防御的架势。

龙公气势凶猛至极,但当他距离方源还有一步之遥时,忽然脸色一变。

他的气势凝结,冲锋戛然而止。

他的寿尽了!

比上一世还要早一些功夫。

方源没有丝毫意外。他捉住了之前逸散掉的灵光,发现了真相。

事实上,当初在东海,他扮演气海老祖时候对龙公动用偷生杀招,还是有效果的。

只是效果低微,不太明显,并且龙公伪装的也相当出色。

事后,龙公为了避免和气海老祖继续交手,甚至拿出了元始真传来瞒天过海。因为龙公并不知晓,方源掌握的偷生杀招次数极其有限。

这也导致了这一世,龙公的表现还要超过上一世同期。

一代雄才,天庭的支柱,红莲的师父,龙人的开创者龙公!

终于在此刻,停止了他生命的历程。

临死的前一刻,他仍旧在冲锋的途中。

不管他是可敬可畏,还是可鄙可悲,他终究为了天庭,为了大业,真正做到了鞠躬尽瘁死而后已!

激战和厮杀,显然让他忘记了寿命的限制。所以临死的那一刻,他的脸上除了愤怒和杀意,还有一丝的错愕。

“龙公大人……您一定很不甘心吧。”

“可恶,龙公大人您不能倒下啊。天庭还需要你。”

“龙公前辈!!!”

天庭蛊仙凄厉叫喊,士气陡然攀升,这是哀兵之志!

激战和搏杀再次展开,三域蛊仙竟被人数稀少的天庭一方死死压入下风!

方源却没有再参战。

他仍旧在忍受着残存的剧痛,他的目光似乎穿透天庭,俯瞰整个五域。

大地震颤,发出一阵阵的沉闷的隆隆之音。

这一刻,包裹着五域的界壁彻底消弭,地脉勾连,原本隔绝的五域彻底统合一体!

没有了界壁阻碍,五域将自由流通。同时天地二气的差异也会逐渐消弭,五域一统再无客观障碍。

方源背后的血战披风开始徐徐消散,他微微仰头,深呼吸一口气。

啊,多么自由的空气。

从此,他终于有了追寻永生的可能。

想到这里,方源的嘴角微翘,流露出一抹笑意。

“大时代,终究来临了!”

------------

单章:耐心写,耐心读

诸位读者朋友们,大家好。

这是罕见的单章,来说几件事情。

关于第五卷魔王雄霸,至此就结束了。关于这一卷写得的确很长,这也将是整本书中最长的一卷。但没办法,这一卷的方源实力跨度最大,真正的从一个棋子成长为一个棋手。相信大家看到这里,也就明白了。

说实话,第五卷的写作压力是很大的,比起前面几卷,甚至是即将到来的最后一卷,都是最重的。

需要描写出来的东西太多了。

这一场大战,涉及到整个五域,还有个别的洞天世界,可以说是把整个的地图都覆盖了。

涉及到的势力至少有六十个,涉及到的人物有武庸、池曲由、宋启元、沈从声、沈伤、张阴、容婆、五行大法师、冰塞川、毛里球、房睇长、龙公、秦鼎菱、紫薇仙子、袁琼都、方源、白凝冰、影无邪、帝藏生、魔尊幽魂、星宿、红莲、元莲、狂蛮、叶凡、洪易、凤金煌、凤九歌等等。

人物实在太多了!

这些人物都各有各的修为,有的只是凡人蛊师,比如凤金煌。有的实力卓绝,如方源、凤九歌、龙公。

他们有着各自的杀招。比如龙公的两套战斗体系,十几个杀招,都切合他的形象的设计。武庸的杀招,无限风和送友风,这两个一正一邪相得益彰。方源的杀招那就更多了,红莲的手段,宋启元的杀招,沈伤的手段。都需要有详略相参的描写。

这些人物都各有各的谋算。比如一心想要完成巨阳遗志的冰塞川,心怀野望的武庸,为了天庭付出一切的龙公、为了永生的方源,为了推翻宿命的红莲等等。

这些人物不仅有现在的,也有作古的。比如历代尊者,比如武庸的母亲武独秀。

这些人物都要尽量有血有肉。比如红莲魔尊,比如龙公,又比如方源。这三者都是着重描写,相信能够读者带来深刻的印象。我更想要做到的是一些小人物,也得有他们的标志性。比如袁琼都,这个负责修复宿命蛊的关键人物,他在关键时刻选择自我牺牲,我相信应该会留给读者们许多印象。这个人物的上一世和这一世的两次大战的回忆,可以组合起来,形成他自己的故事。彰显的是一种责任的理念。这方面又和《人祖传》中自由蛊带来责任重担,相呼应。另外的小人物,比如叶凡、洪易,也是有着他们的奇遇。关于秦松,这个人物是天庭仙墓中的成员,上一世戏份很少,但这一世却展现出了他的强大。天庭的传送大阵也要靠他来完成。

这些人物他们都有各自的人生的剧变。比如凤九歌从正道转变成了魔道,公然背叛天庭。关于他的这份转变,其实伏笔早有很多。比如他本身就是隐修,和十大古派作对过。又比如龙公曾经询问凤金煌是否信命,凤金煌说不信,龙公就曾反思过原因,觉得问题有一部分是出在凤九歌的身上的。这是家教问题。这些伏笔,大家再看一遍,就清楚了。又比如凤金煌,她是大梦仙尊的种子,但是宿命蛊摧毁了,天命没有了,她的父亲又背叛了,她该何去何从?又比如方源的分身房睇长,他被困在帝神宫中又会有哪些遭遇?

这么多的人物,围绕宿命蛊这一点,而展开史无前例的大乱战,非常考虑布局。

我相信,这不仅是本书最史无前例的一场大战,纵观网文界,也是罕见的一场超大规模的大战。

因为它涉及到的人物之多,地图之广,已经非常罕见了。更难的在于,这场大战要打两次,方源重生前失败的一次,方源重生后成功的一次。

这两次大战需要前后对比,需要层层递进,不能有太多的bug,还需要伏笔。

943-945的三节中,大家应该可以看到我之前精心铺设的伏笔。这些伏笔从大半年前就开始铺设,一直到现在才启用出来。这些伏笔表面上,只是一些不相干的故事,它们必须伪装得很好。然后当它们一起出现时,才能显现出作者真正的用心。值得高兴的是,还未有读者看破这一点。

为了完成我的写作目标,我需要细化人物,大半年前就开始布局之外,我还需要理念的碰撞。

上一世大战,我主要写了天庭的理念,很多人为此感动。这一世大战,我主要写方源、红莲一方的理念,很多人也为此激情澎湃。

这正是我想要达到的目的。

唯有在理念层次上的碰撞和战胜,才能彰显出各自的伟大,更有助于理解这些角色的动力和行为。

而我写这些理念之争,是想更深化本书所探讨的正魔之争。

什么是正道光明,什么是魔道黑暗?

成熟的人都清楚:这个世界并不是非黑即白。正道和魔道,黑暗和光明的界限,其实是很模糊的。

本书的前半部分,着重描写的是正魔之间的区别。后半部分则开始描写正魔之间的共性。

有时候正道就是魔道,换个角度,魔道就是正道。

这并不是最浅层次的“成王败寇”的理念。

举个例子:天庭的伟大,牺牲的正义,的确是冠名堂皇。但是柳淑仙宁愿自己牺牲,也要帮助红莲成尊的这种强烈意愿,其实是违背人性的。这难道不是一种魔性吗?红莲起初爱她而重生,最终却亲手杀掉柳淑仙,更多的是一种人性的选择。

再举个例子:方源这个大魔头成为摧毁宿命蛊的最大功臣,我们都知道他魔性深重,出发点也只是为了追逐永生,非常自私自利。但是单从他的行为和结果而言,他解放了全人族,让他们摆脱宿命的束缚,得到自由。这难道不是一种正义吗?

写完这一大卷,我着实松了一口气。

预期的目标:1人物塑造,2布局伏笔,3理念碰撞,4正魔理念的深化。

我大体上是完成了。

为了完成这个目标,我改了十几次大纲,构思伏笔的时候,把绝大多数看起来十分明显的伏笔,都给舍弃了。常常构思的时候,会整天整天地枯坐在桌子前,写一张张的草稿,然后扔一张张的草稿。人物涉及太多,我就将这些人物的描写还有招式等等,都答应出来,总共有上百张的纸。写作的时候,铺了我满满的书桌。

我说明这些实情,不是想夸赞一下我自己,而是想说明:这本书我是用心写的!

我承认,我写得比较慢,但没有办法,写快了我实力不够,并且容易出错。

我知道,我这本书写了大概有五六年了,时间很长了。但是我必须写的有耐心,越是到后期越得有耐心,越得沉得住气!

所以,我想在此呼吁各位亲爱的读者朋友们:

我耐心的写,恳请诸位耐心地读!

这本书和绝大多数的网文,是有区别的。写法、布局、题材、剧情构建的区别,需要大家在阅读的时候,报以更多的宽容、理解和期待!

我很有自信地说:只要诸位看到最后,这本书一定能够让大多数的读者满意。

第一点,我恳请大家耐心阅读。

第二点,我希望大家多多支持!

请支持正版阅读。

一个小节三千字左右,只是几毛钱。但是我要码出来,得要两个小时左右。

我的章节,还和大多数的网文不同,我需要在写之前,构思更多,花更多的心血布局。

别人写三千字耗费的精力和心血,和我写三千字耗费的精力和心血是不同的。但我们的定价一样,拿到的是一样的钱。

但我愿意花更多的心思和精力,去努力呈现给大家一个活生生的蛊世界,一个不一样的阅读体验。

事实上,因为这本书黑暗文的标签,写法上的诧异,一直让我饱受诟骂和误解。让这部作品趋于小众。

眼下这本书即将来到最后一大卷。

每个人的命运,都会迎来终结。比如方源,比如方正,比如商心慈,比如白凝冰。天庭的战败带来不一样的五域乱战。红莲、幽魂等各大尊者的阴影,也将越加浓重,成为方源前进路途上的重重阻碍。苍蓝龙鲸、疯魔窟、神秘黑火等等隐秘,也将一一解开。《人祖传》也将更新完整。关于本书的理念,也将真正升华到我想要写的那个层次。本书的主题也将会真正彰显。

请大家多多支持正版阅读,请耐心阅读。

敬请期待:《蛊真人》最后一大卷——魔尊永生!

\end{this_body}

