\newsection{方源爱情}    %第五百二十三节:方源爱情

\begin{this_body}

“犯我琅琊,留下命来!”琅琊地灵大吼一声,协同身边一干毛民蛊仙,催动天婆梭罗直接攻上去。[www.qiushu.cc 超多好看小说]

退!

见银色巨人来势汹汹,中洲的蛊仙们都身形如电,纷纷从四面八方暴射撤退。

就算是凤九歌也在退。

不过,他后撤的速度最慢,这绝非他速度比不上其他蛊仙,而是故意殿后。

毫无疑问,凤九歌是中洲蛊仙当中战力最强,也是此次领头大将,此行成败有大半关系落在他的肩头,方方面面都需要他镇守大局。

“这是中洲灵缘斋的凤九歌!”琅琊地灵主持天婆梭罗,见到凤九歌也是瞳眸微缩,满脸凝重之色。

来者不善,琅琊地灵感受到非常庞大的压力。

因为凤九歌的出现,意味着中洲十大古派,更意味着天庭成了琅琊福地的对手!

天庭,人族历史长河之中,古往今来的第一势力,同时也是全天下的第一洞天!

这样的对手,琅琊地灵怎可能不紧张,不感到重重压力?

“小五!”琅琊地灵口中呼唤。

蛊仙毛五连忙点头:“看我的!”

说着,他便催动仙道杀招。

杀招立即起效,反映在天婆梭罗银色巨人的身上,银色巨人忽然张开大口,喷射出一道恢弘光柱。

光柱来势汹汹,眨眼间就来到凤九歌的面前。

“好快的速度!”凤九歌也微微一愕,白色光辉映照在他的脸上,不过旋即,他嘴角微翘起来。

嗖。

一瞬间,他消失无踪,恢弘光柱穿透他的位置,又笔直射向远方,沿途激荡出一片片的空气涟漪。

凤九歌再次出现,却是不退反进,来到了银色巨人的额前。

“他就在我们的头上。”银色巨人当中,负责侦查的毛民蛊仙立即喊道。

毛民蛊仙们顿时有些慌乱。

“来而不往非礼这招吧。”凤九歌朗笑一声,大袖飘飞,一股磅礴气势从他的身上猛地爆散开来。

仙道杀招三绝音!

凤金煌一拳遥击,发出战鼓的轰鸣。又一掌拍下,发出铜钟的悠扬。再一指指出,产生尖锐的哨响。

这是三音真传中的绝技,分别是鼓拳、钟掌和哨指。[八零电子书wWw.80txt.COM]

先前凤九歌只选择修行了前两者,填补自己的手段,但自从方源逃脱了他的追捕之后,凤九歌回到中洲,反思自己的不足,苦心修行,又将最后一项哨指修行起来。

如此一来,三大手段齐齐施展,就能形成连招三绝音。

拳、掌、指,再拳、掌、指,按照如此的顺序,凤九歌不断进攻,每一击的仙元消耗都大大减少,并且攻伐威能还有稍许增加。

虽然每一式增加的幅度不大,但是凤九歌攻击频率极快,拳掌指车轮般不断循环,短时间内已经打出上百击!这样一来,随着攻击次数暴涨,三绝音的威能绝不容小觑!

道道音波打在银色巨人的身上,渐渐打得银色巨人步伐踉跄,毛民蛊仙们头昏脑涨。

银色巨人挥动双臂,搅起风云,想要将半空中的凤九歌拍撞下来。

但凤九歌飞行手段极其过硬,在半空中腾挪转折,灵动非凡,巨人手臂虽然粗壮得骇人,动作也凶恶迅猛,但就是拿凤九歌没有办法。

“看来还得用仙道手段才能克制此人!”琅琊地灵目光越加凝重,忽又低喝一声,“毛三毛四!”

两位毛民蛊仙齐声响应,纷纷催动仙道杀招。

毛三催动的杀招,让银色巨人的身上,渐渐浮现出一层墨绿色的木质藤甲。

凤九歌的三绝音打在上面,道道音波居然被这木质藤甲吸收。

毛四催动的杀招,则让银色巨人的头顶,喷射出无数金色螺旋气流。这些气流密密麻麻,向着凤九歌围剿过去。

凤九歌处境顿时急转直下。

他攻势再无良效,并且任凭他如何飞行,身边的金色螺旋气流越来越多,难以甩开。

不远处,中洲蛊仙方云华见此,心中一动:“根据情报,拿下这头银色巨人,此战就奠定了大局。凤九歌陷入麻烦,我可助他一助。”

想到这里,方云华正要催动杀招,忽然听到有同伴的警告声:“方兄快躲!”

“什么?!”方云华下一刻,就见到银色巨人再次张开嘴巴,喷射出一道恢弘光柱。

这光柱直接照准方云华射来,方云华连忙狂催防护手段。

但天婆梭罗乃是八转战力,方云华纵然是七转强者,但绝非凤九歌、方源之流,身上的防护手段只替他遮掩了三个呼吸的时间。

但三个呼吸已经足够,方云华趁机化作一片白色云光,四下分散,又在另一处高空汇聚,重新还原成他的本来身躯。

方云华嘴角溢血,脸色苍白,眼眸底部更有一抹骇然之意。

“我刚刚分心,想要酝酿杀招,露出破绽,对方就能够立即察觉。这是什么侦查手段?”

琅琊福地底蕴深厚,但都以炼道为主,前任琅琊地灵除了炼蛊,对其他事情都没有兴趣。

所以这个侦查手段,是来自影宗真传。

方源当初和琅琊地灵交换真传,一方面是为了交易,另外一方面也是帮助琅琊派,增长他们的战斗力!

琅琊派有仙蛊,就是缺乏用来争斗的优质杀招。

现在,方源的举措终于收到了良效。

凤九歌也是心中一沉。

这个侦查杀招,似乎能够瞬间体察到敌人的破绽,极大地方便蛊仙实施精确打击。

有这样的杀招在,对于中洲蛊仙围攻银色巨人,有着极其巨大的干扰和妨碍。

因为这些中洲蛊仙们来自各门各派,平时很少配合作战,一旦围攻银色巨人,必定会发出配合方面的失误。这样的失误一出现,就会立即被银色巨人利用起来。

“人数越多,似乎越麻烦。那么就让我一个人暂时拖住这天婆梭罗好了。”凤九歌一边作战,一边心思频动。

他迅速改变战术,指挥其余蛊仙:“你们分散各处去,布置仙道蛊阵,方便我等后续大军到来。”

“好胆!”琅琊地灵听了这话,顿时气得大骂。

凤九歌微笑起来。

他就是故意这样说出来,这是赤・裸・裸的阳谋,不怕琅琊地灵不就范。

银色巨人虽有八转战力,但是只有一位,而中洲蛊仙却总共有七位,银色巨人分身乏术,但若分化出毛民蛊仙出来对战,那就更着了凤九歌的算计。

这些毛民蛊仙虽然战斗素养提升很多,但是一对一,几乎都不是这些中洲七转强者的对手。

所以琅琊地灵面临着两难处境,分兵不行,不分兵也不行!

琅琊福地大战激烈的进行着,远在南疆的胎土迷宫之中,却是一片宁静。

“我这是在哪里?”方源睁开双眼,发现自己正躺在床榻之上。

他想要起身,却发现自己重伤。

他的眼中一片迷惘,喃喃自语:“我是谁?我似乎……忘记了一些很重要的东西。”

“儿啊,你终于是醒了!”这个时候一位老妇人听到动静,从屋外进来,见到苏醒的方源高兴地泪流满面。

“你是?”方源疑惑。

老妇人神情一震,旋即大哭起来:“儿啊,你是被打坏了脑袋,痴傻了吗?我是你的娘啊,你是沈三,前日里那舒少爷见不得你和绣娘好,便着他家中的蛊师将你殴打了一番呐。儿啊,听娘的劝,虽然你和绣娘两小无猜,但咱们家小业小,虽然祖上有过蛊师,但现在却破落败坏。你和舒家少爷争绣娘,是争不过他的呀。不妨,不妨就放手了罢!”

“绣娘……”方源口中呢喃,“难道这一切都是真的,我竟将这些种种都忘记了吗?”

数日时间,他渐渐了解了自己的身世。

原本他和绣娘是早就定下的娃娃亲,但童年时父亲战死,家中唯一的蛊师没了,家道立即败落下去。本来还和绣娘家门当户对,但现在却是攀比不上。绣娘的家里双亲更属意那舒少爷。舒家家业更大,乃是城中的豪族,族中舒家的蛊师就有数十人,再加上豢养的门客,蛊师数量能上百!

又过几日,绣娘来看方源。

“三郎,你伤势如何?我日夜都想着你,可叹双亲禁足,今日才谎称去请教蛊师修业,才能来你这里。我的三郎,我可怜的三郎……”绣娘见到方源躺在病床上憔悴的样子,顿时让她婉转低泣起来,目光中充满了真挚的爱意。

方源看着绣娘,觉得自己第一次见到此人,很陌生的打量她。绣娘豆蔻年华,雪肌贝齿,青丝如瀑,一身月白衣裳,身姿窈窕,又朴素清纯。此时美眼含泪,楚楚动人,乃是绝世的美貌。

“难怪那舒家公子会钟情绣娘了。”方源心中一叹,默不作声。

绣娘纳闷,这时老妇人叹息,说出缘由。

绣娘悲呼,又紧张,拉住方源的手,絮絮叨叨地述说往事,企图让他回忆过来,记得自己。

那一幕幕的往事,或许平凡细微,但都蕴含着两人的情意。

方源心中的爱意也渐渐被唤醒似的,他望着绣娘的双眼,伸出手来,抚摸她细嫩至极的脸颊,轻声唤道:“绣娘……”

“哎。”绣娘连忙答应,一把抓住方源的手,让他的手心紧紧地贴在自己的脸颊上。

然后她深情地注视着方源的双眼,又唤道:“我的三郎,我的好三郎。”

胎土迷宫之外,仙道战场之中,陆畏因微微而笑,自语道:“我果然没错,人非草木孰能无情?这方源并非绝情绝意之人,他的心中还隐藏着爱和情。”

陆畏因又扫视周围,不知何时起,不管是铁面神等等蛊仙,还是商心慈一众蛊师,都悬浮在空中,闭上了双眼。

其中商心慈紧闭眼帘,眼珠滚动,脸颊微红,似乎在一个不能自拔的美妙梦境当中。

陆畏因幽幽一叹:“红尘滚滚,演绎众生。为渡化此魔,尔等就在我的胎土迷宫之中,忘记自己,变换另外一种身份,演绎人生吧。这亦是机缘,只要你们悟性足够,便能有绝妙的好处。”(未完待续。)

\end{this_body}

