\newsection{宝黄天开启!}    %第八十三节:宝黄天开启!

\begin{this_body}



%1
影宗的残余势力,真的不容小觑!

%2
呈现在黑楼兰、太白云生眼前的仙蛊,足有十几只。

%3
不过看着这些仙蛊,两人脸上刚刚浮现而出的喜色,却在渐渐收敛。

%4
“我是大力真武体,但这些仙蛊中却没有一只力道仙蛊。”黑楼兰摇头叹息,语气饱含遗憾。

%5
“唉!方源你是知道的,我主修宙道,若用这些仙蛊的话……”太白云生的情况也是如此,这些仙蛊中偏偏没有宙道存在。

%6
大多数的仙蛊,都隶属于魂道。

%7
石奴沉默不语,他是土道蛊仙,这里倒有两只仙蛊是土道流派,不过他的仙蛊本就并未减少。

%8
影无邪点点头,脸上的微笑没有变化。

%9
他知道,这些仙蛊都是在义天山大战中,损毁的那些仙蛊。这些时日,仙蛊屋悔池自发运转,借助曾经的成功印记,和这里的光阴支流,将这些仙蛊重炼出来。

%10
所以魂道居多。宙道本会有,但却是炼制失败了。

%11
至于力道仙蛊,却是没有的。

%12
那位大力真武体,仙僵大荔是历史上都罕见的智、力双修,可惜他炼制力道仙蛊时,影宗还未重建悔池,也就谈不上留下什么炼蛊的成功印记了。

%13
“不要紧。你们无非是担心道痕之间相互内耗,我这里却有妙法,可以令你们兼修另外一门流派,并且这门流派的道痕和你们主修的道痕并行不悖,毫不冲突。”

%14
“这世间竟然有此等妙法?”太白云生惊得眼珠子都鼓瞪起来。

%15
石奴原本沉闷的脸上,也动容了。这种法门,简直是为所未闻!

%16
黑楼兰双眼眯起,她怀疑这是影无邪趁机对他们增强控制力度,她感觉这是一场阴谋。

%17
她淡淡开口:“这世间哪有如此便宜之事?有得必有失,如此妙法,恐怕弊端不小吧?”

%18
影无邪失笑:“我保证此法没有丝毫后遗症。”

%19
他深深地看了黑楼兰一眼,眼眸却阴翳下来。

%20
他自然而然地就联想到了方源,不免生出许多郁闷。

%21
和至尊仙胎蛊比较起来。两派道痕互不干涉,有什么值得惊奇的?影宗研究出来的至尊仙胎蛊,可是所有道痕都可同修,相互之间毫不干涉牵扯。

%22
魔尊幽魂在生死门中。吞并无数蛊仙魂魄,汲取海量修行经验。

%23
经过十万年的酝酿,他将无数流派的境界,都推上大宗师的程度。

%24
双派同修的法门,只是影宗很早以前。就研究出来的初级版本。最高的成就,便是至尊仙胎蛊!

%25
为了炼制出至尊仙胎蛊,影宗上下耗费近十万年光阴,耗尽了整个影宗、僵盟的底蕴,甚至搭上了魔尊幽魂本体,才堪堪功成。

%26
可是人算不如天算,至尊仙胎蛊虽然炼成,但在最后关头,被方源摘了桃子。

%27
至尊仙胎蛊是一次性的消耗蛊,方源用了重生。天地间再无至尊仙胎蛊,理论上是可以再炼出来的。

%28
不过要再炼第二只,付出的代价和风险,实在太大太大。就连影无邪都没有这个心气劲了。所以,至尊仙胎蛊基本上已经属于绝响。

%29
“除非是将方源捉来,当做最主要的仙材,进行逆炼!”

%30
“不过,现在还不是对付他的时候。”

%31
“还是先增强自身力量,收拢东海的残余势力,尽快拯救出本体!”

%32
影无邪心中叹息一声。影宗和方源的这笔账迟早要算,但事有轻重缓急。现在绝不是和方源纠缠的时候。

%33
时间匆匆流逝。

%34
北原,琅琊福地。

%35
距离方源渡第二次地灾,已经一个多月过去了。

%36
宝黄天迟迟不开。没有丝毫动静。

%37
方源经营仙窍的大计严重受阻,无奈之下,他只好一方面尽量修行,多多推演,另一方方面则将目光关注到北原争乱上面,并将黑凡真传提到近期计划之中。

%38
地灾相距只有两个月。如今一个多月过去。只剩下一个月都不到的时间,方源就会迎来第三次地灾。

%39
这世间说长不长,说短不短。

%40
方源似乎都能闻到灾劫的气味了。

%41
让方源揪心的是:第三次地灾,比第二次还要强大。而到目前为止,他的实力,却和第二次地灾时,基本上没有什么两样。

%42
“如果得到黑凡真传,倒是极有可能借助仙蛊,延缓仙窍时间。如此一来,不只是第三次地灾,就算对今后的种种灾劫,都是有利得很。”

%43
方源从黑城那边得知许多情报。

%44
黑凡真传十分了得,源自八转宙道大能黑凡。若是能继承这道真传,就能进入黑凡洞天,不仅成为洞天之主,并且还能尽得那里的许多宙道仙蛊!

%45
很多的蛊仙真传,只有仙蛊方、仙道杀招等等,缺乏仙蛊。这是因为,仙蛊的喂养存在难题,很难长存仙蛊,顺利地留给继承者。

%46
但是黑凡真传不一样!

%47
因为黑凡留下的仙蛊,都保存在他的仙窍中。

%48
蛊仙的仙窍经营完善,第一层标准就是满足自家仙蛊的喂养需求。

%49
黑凡这种层次的蛊仙,相当于当今北原的八转蛊仙药皇,经营仙窍绝对是完善的。

%50
方源若得到黑凡真传,整个实力都会上涨、翻倍。一旦延缓了仙窍时间流速,危险的局面将大为缓解,彻底改善。绝不会再像是现在这般,如同在悬崖边上行走。稍有不慎,就是人死道消的下场。

%51
这些天来,借助琅琊派,方源对北原争乱的情报也知道不少。

%52
这倒不是说,琅琊派搞情报很有一手。恰恰相反,毛民蛊仙碍于身份,很难打探情报,远远比不上曾经的黎山仙子。

%53
关键的原因,还是北原争乱的动静太大,卷席了几乎整个北原蛊仙界。他们之间的争斗,可能瞒得过凡人,但对于蛊仙而言,却是毫无遮掩。也遮掩不住!

%54
情报其实不需要打探,都摆在明面上。

%55
方源虽然身在琅琊福地,但早就关注整个事态的发展。

%56
黑家这次背锅,背得太大了。已经到了大祸临头,家族灭亡的关头。

%57
来自正道、魔道、散修这三大阵营的蛊仙,就像是饥饿的狼群,蜂拥而来。

%58
看北原版图,黑家位于北原东南地带。疆域辽阔,掌控大量资源。

%59
这些资源,吸引了无数的豺狼虎豹。

%60
最初时,这些强盗还有些克制,但很快重利当前,魔道蛊仙开始大打出手。

%61
正魔两道的摩擦,不断加剧。

%62
标志性的事件,就是关神照和游地双英之战。关神照双拳难敌四手,捷足先登的黄金麦场被两位魔道蛊仙抢了。

%63
随后,大量蛊仙粉末登场。以魔道蛊仙游地三英、鹿老、半月蛮师、卓战等代表人物。对黑家各处资源进行哄抢。

%64
场面混乱,一度打了正道措手不及。

%65
涌泉林就是在这种情况下,遭到毁灭性的劫掠。可惜了黑家数百年的精心经营。

%66
正道蛊仙迅速警惕起来,开始有了默契。

%67
就算是争夺太古赤天碎片世界,这样的重要资源,都克制忍耐,并未大肆出手。

%68
蛊仙界,历来都是正道为主。

%69
尤其是北原的正道,基本上都是黄金血脉,根源同一。

%70
魔道蛊仙的势头为之一沮。但很快,随着半数资源被瓜分干净,自在书生、皮水寒等魔道七转强者陆续现身,又带动起又一轮的纷乱。

%71
史无前例的。魔道的气势竟然压过了正道一头。

%72
黑家蛊仙最初时,还分兵驻守,但很快尝到苦头之后,就收拢全部蛊仙战力,龟缩在大本营中不再出现。

%73
这无疑助长了魔道蛊仙的嚣张气焰。

%74
当然,除此之外。还有其他原因。

%75
比如许久之前,雪胡老祖击败药皇、百足天君联手,以魔道身份,成为北原蛊仙界的第一人。

%76
最主要的还是八十八角真阳楼倒塌,仙蛊乱飞,让大量魔道、散修蛊仙收益。随后秦百胜举办的拍卖大会,让许多蛊仙都得到合适自己的仙蛊,战力陡然拔升许多。

%77
这才使得整个魔道、散修的实力,提升许多,可以和正道分庭抗礼。

%78
“我捣毁八十八角真阳楼和王庭福地,经过短短几年的时间,这个影响已经波及整个北原。现在的北原局面,已经和五百年前大相径庭了!”

%79
方源心中感慨不已。

%80
整个北原的大势,都因为他,而变得面目全非。

%81
其他几个大域,还好些。但在北原,方源已经失去了大量的重生优势。

%82
当然,自从他得知天意之后,他对重生优势也保持着相当程度上的怀疑。

%83
方源一直在耐心等待。

%84
黑家的许多资源,都让他心动。但他始终没有动身。整个黑家最重要的资源,就是黑凡真传。这点方源看得很清楚。

%85
又是数天过去。

%86
终于,方源得到了北原蛊仙,一齐围攻黑家大本营的消息。

%87
“黑凡真传就藏在黑家大本营中,是时候动身了!”

%88
方源立即出走,行动相当干脆,通过传送大阵离开了琅琊福地。

%89
琅琊地灵倒也支持他的行动。

%90
毕竟这是近距离观察北原蛊仙界,探清各方势力虚实的好机会,同时运气好的话,也还可捞取不少好处!

%91
琅琊地灵还不敢派遣琅琊派中的毛民蛊仙到哪里去,但方源是纯正人族,又狡猾至极,是最合适的人选。

%92
方源故技重施,一现身北原,就动用暗渡、见面曾相识等,遮掩了自家身份。

%93
同时,他用剑遁仙蛊、血漂流手段等交替赶路。

%94
他前进的速度很快,也没有遇到什么麻烦。

%95
北原蛊仙们几乎都集中到了黑家大本营那边,这让方源一路畅通无阻。

%96
不久之后,黑家的大本营已遥遥在望。

%97
“黑凡真传,我来了……”方源心中念叨,但没想就在这个时刻,宝黄天陡然重启!

%98
热闹了。

%99
宝黄天的重开,让苦苦忍耐的各方蛊仙,爆发出巨大的交易热情。

%100
方源刹住脚步,陷入进退的犹豫之中。

%101
一道信息通过宝黄天传来。

%102
方源展开一看,是楚度的!

%103
ps:有点疲累了。今天状态不好,卡文,晚了一些,十分抱歉。

\end{this_body}

