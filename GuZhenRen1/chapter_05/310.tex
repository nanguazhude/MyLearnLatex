\newsection{原来无解}    %第三百一十节:原来无解

\begin{this_body}

智汗阵。9;\&\#32;\&\#25552;\&\#20379;\&\#84;\&\#120;\&\#116;\&\#20813;\&\#36153;\&\#19979;\&\#36733;\&\#65289;≯

池伤再一次布下这座蛊阵,竭尽全力,推演阵法。

但是这一次,他遭遇到了前所未有的挑战!

难,非常的难,就算前两次的难度叠加起来,再乘以十,也不及这一次的难度。

池伤感觉自己就像是一个婴儿,企图攀登一座高山。

“这种难度……难怪那武遗海非常自信!”

“它和之前的两个蛊阵内容有相当的联系。”

“不行,我一定要把握住这个机会,狠狠地羞辱武遗海。我要让他在丝柳仙子的面前,承认自己不如我!”

一天一夜下来,池伤双眼充斥血丝,拿出了拼命的劲头,来攻克这道难关。

与此同时,方源也在琢磨着。

“这个难度,恐怕已经过宗师境界的能力了吧?”

不久前,方源再次获得了池伤的“帮助”,使得他盘丝洞窟的蛊阵改良计划,又大大地向前迈了一步,有了突破性的进展。

不过,就在要彻底成功的最后关头,方源遭遇到了史无前例的阻碍。

如果说第一次的难题,仿佛土坑,第二次宛若荒丘,那么这一次的阻碍,就是一座山峰。

只有攀登到峰顶,才能突破阻碍,完成这个蛊阵的改良。

这个阻碍的出现,大大地出乎了方源的意料。

“嗯……我之前估计过,若是按照设想,蛊阵改良出来,长恨蛛群的产量能翻三番。”

翻三番的意思,可不是增长三倍。而是增长了七倍,即是原来的产量乘以八。

这种增长,显而易见,是非常恐怖的。等若是蛊阵搭建好了,凭空另外增添了七个如今的盘丝洞窟。

方源的设想非常美好,但是在实践当中,他遭遇到了相当大的困难。

前两次他磕磕碰碰地解决了,但是这最后一次,却是让他一点头绪都没有。

距离成功,只差最后一步,方源怎可能甘心放弃?于是他又把主意打到池伤的身上。

又是数天过去,智汗阵停歇下来。

池伤脸色惨白,毫无血色。

他双眼失神,一点亮光都没有。

“可恶!!”他头蓬乱,宛若稻草堆,脸色狰狞,咬牙切齿。<strong>最新章节全文阅读www.QiuSHU.cc</strong>

皆因他拼尽全力,也没有成功。

这样一来,就胜不过方源,更糟糕的是,乔丝柳也已经知道了这第三次的挑战,池伤已经在信中夸下海口,扬言要攻破这道难关。

如果他失败了,武遗海那边且不去说,今后怎么面对乔丝柳?

池伤感到前所未有的庞大压力。

他胡吃海塞了一顿,稍稍休整之后,便又投入到紧张的推演工作中去。

池家。

位于南疆的西部,更具体来讲,在当今的级家族当中,它是位于南疆最西的方位。

与池家接壤的是羊家。

这个家族和武家又接壤,最近更是刁难武家,因为一只野生仙蛊而生了冲突。

池家、羊家、武家。

远交近攻是基本的外交策略,所以池家、武家之间一直关系都不错。

一份最新的情报,送到了池家太上大长老池曲由的桌前。

池曲由展开一看,眉头微微皱起。

这份情报,说的不是其他,正是级蛊阵中方源和池伤挑战一事。

位于级蛊阵中,池家的领池规,唯恐方源和池伤的矛盾升级,早已明智地将这件事情捅了上去,汇报给了池曲由。

如此一来,就算将来情况恶化,池规的责任也就能少一些。毕竟他并未知情不报。

池曲由也会在处理繁杂的政务时,抽出一部分注意力,关注这场矛盾的进展。

毕竟当事人双方的身份都不寻常,这种事情搞不好,就会升级成外交矛盾。

“池伤被难住了,这个蛊阵难题,嗯……已经越了宗师的能力了。”

情报中竟然详细到,方源的第三次的阵道难题。

池曲由乃是大宗师,只是扫了一眼,就看清了这道难题的底细。

“结合前两道难题,武遗海应当是想要铺设一个蛊阵,用于经营资源。只是设想出这个蛊阵的人,未免太过于贪心,偏偏境界又不足,看不清。”

“如果我不点醒他的话……依凭池伤的这种执拗性情,恐怕他会一直钻研下去,因为思虑太重而受伤,甚至死亡。”

“难道是武家的设计和阴谋吗?”

“看透了池伤的底细,故意想要除掉我池家的希望。”

“嗯……凭武家此时的境况,似乎不太可能。不过也不能大意啊。”

池曲由想到这里,眼中闪烁出一丝锐利的精芒,自语出声:“看来我必须要稍稍出手一次了。”

片刻之后,池规造访池伤,带来池曲由的一只信道仙蛊。

“这是太上大长老嘱托,一定要让我亲手将这只仙蛊交给池伤你啊。”池规如此说道。

池伤懵懂地接过这只仙蛊,探入心神之后,他看到里面的内容,顿时脸上变色,惊喜地欢呼一声:“这是当年隐修图元的仙道传承啊。他的地水风火四元阵,独树一帜,别无分家。只是声名不显,太过低调,所以很少有人知道他。太上大长老,居然将这份仙道传承交给了我!我之前求了几次,都求不来呢。”

池伤激动之下,立即沉浸在这份仙道真传当中,对外界不闻不顾。

当他回过神,抬起头时,一天早已经过去,已经进入了深夜。

“时间过得真快啊,咦?池规太上家老是何时走的,算了,不管了。”池伤双眼精芒爆闪,几乎令人不可逼视。

池家太上大长老的这份真传,来的太是时候了,正好对症下药,带给池伤最为正确的提点。

池伤对池曲由充满了敬爱之情。

“原来太上大长老一直关注我的事。”

“他看到我遇到难关,立即出手相助我。”

“这算不算作弊啊?”

“应该不算吧。太上大长老可没有直接提点我,只是给了我一份真传而已。解决的灵感,都是我想出来的。”

“嗯,就是这样!时间已经耗费得够多,我要抓紧时间,把这个难关攻克,给武遗海一个好看!”

池伤满怀激动和干劲,再次祭起智汗阵。

又是两天过去,蛊阵破开,池伤却是一脸恼怒地走了出来。

“原来这个蛊阵难题,根本就是无解的!”

“好你个武遗海,居然如此坑我!”

“太阴险,太狡诈了!!”

“幸好我得到了太上大家老的提点,才勘破了这个真相。”

想到这里,池伤将牙齿咬得嘎嘣作响,这些天他吃了很多的苦,几乎不眠不休钻研。

结果却现,这道难题根本无解。

池伤感觉自己被武遗海硬生生耍了一回,他感到相当的恼怒。

不过很快,他又笑起来:“对啊,这道难题无解,我只要道破这个事实,也算我胜了。甚至我还能向丝柳仙子揭,让她好好看看武遗海这厮的阴险嘴脸。”

池伤立即就做,两只信道凡蛊送了出去。

方源接连接到了两只信道凡蛊。

第一只,是来自乔丝柳的。

乔丝柳送来信道凡蛊,并不出乎方源的意料,他透入心神浏览一番后,有些纳闷。

然后他想到了什么,抛下看了一半的丝柳的信,去看池伤的那份。

方源顿时恍然。

“这个池伤又有进步,居然学会恶人先告状,先向乔丝柳说明了这次挑战的结果,然后再将信蛊送到我这里来。这一前一后,就是防备我反咬一口吗?真是进步不小。”

方源的笑容渐渐收敛起来。

第三次的结果,出乎他的意料。

他设想出来的蛊阵,居然是无解的!

这种情况,其实很常见。

在推算成功之前,任何创新出来的蛊方、杀招、蛊阵,都有可能是无解的,不可能成功。只是流于幻象中的假设,没有现实成功的可能。

有些难题无解,是一目了然的。但有一些,却不是这样。

尤其是阵道,阵道涉及的蛊虫和流派太多,非常复杂。一些设想,不推演到最后,不会现这是一个无解的题目。

“产量翻三番,这个设想出来的结果,还是太美好了,过现实了。”

“既然无解,那我就只有抛弃原来的设想,进行另外方式的改良。”

“不过现在就算了。”

“这一次挑战我当然是输了,宴请池伤吧。这个人有点意思。”

对于这一次的宴请,池伤想都没想,就直接答应下来。

他的心态和之前完全不一样了。

之前,他觉得方源拿大,他偏偏不去。现在他是一个胜利者,在他看来,这场酒宴就是赔罪宴,他当然要带着胜利者的姿态,去欣赏失败者的嘴脸。

所以,当他见到方源的第一句话便是:“武遗海,你可服了吗?”

方源摸了摸鼻子,大笑:“服了,服了,仙友高才,在下怎能不服?”

池伤楞了一下,他怎么也没有想到,方源会如此态度,居然直接服软了。

这让他有些手足无措。因为在此之前,他就设想了很多方案,如何应对方源的耍赖或者强词夺理。

但是现在方源直接坦诚失败,说自己服气了。

这让池伤有一拳打到空处的感觉。(未完待续。)8<!--80txt.com-ouoou-->

------------

\end{this_body}

