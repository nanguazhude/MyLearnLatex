\newsection{糊眼灰}    %第四百七十七节:糊眼灰

\begin{this_body}



%1
休整完毕,准备充分,方源再次进入盗天梦境。无弹窗,最喜欢这种网站了,一定要好评]

%2
周围人声嘈杂,围绕着池塘,正举办一场集市。

%3
西漠资源其实并不少,至少比北原多。沙漠下潜藏着大量的资源,有蛊虫、野兽,唯有植被非常稀少,是最大的弱项。

%4
因此集市中,主要买卖的,便是奶、酒等食物,或者是兽皮、兽骨等等蛊材,只有三家有着蛊虫贩卖。

%5
少年盗天在人流中行走,他一边观览周围的买卖,一边朝着部族中心的池塘靠近。

%6
“停步!这里闲人勿入。”

%7
“小子,快滚。”

%8
少年盗天刚刚靠近池塘一些,镇守在这里的两名护卫,就立即反应过来,一人一句话,将他阻止。

%9
少年盗天无法,只好装作一些惊惶的样子,连忙掉头离开。

%10
他在心中叹息:“池塘是绿洲中唯一的水源,看守森严,就算是开了集市,守卫也是外松内紧,根本混不进去。”

%11
与此同时,他的脑海中响起沙枭阴测测的声音:“哼,该怎么混进去,你要想尽办法。别忘了,离我们约定的时限,只剩下一个多月了。一旦到了时间,你没有任何进展的话,那就是你的死期。”

%12
少年盗天瞳孔一缩,在心中回道:“你不是实力高超么,为什么不直接攻进来,或者偷偷潜入进去。偏偏要我这个实力低微的人,为你刺探情报?”

%13
“哼,你需要激将老夫。老夫怎会中你这计?况且你什么都不懂,嘿,老夫也不会耐心给你解释。孙子!快照爷爷说的做,少问多做,你才能活得更久一点。”沙枭道。

%14
少年盗天受制于人,敌强我弱,不再言语,只是眼中精芒闪烁不定。

%15
之前他在枯井中,为了活命,留下回家的希望,被迫向沙枭屈服。但他心中却是从未真正认命过。

%16
“大丈夫能屈能伸,总有一天,我会找寻到机会,不仅脱离沙枭这老魔头的桎梏,而且还要将我所受的苦难也回报给他去。”

%17
“目前,我且暂先忍耐,虚以委蛇,等待时机。”

%18
接下来的几天,少年盗天想尽办法,竭尽努力,想要混进绿洲的池塘附近,探听情报。

%19
但各种尝试,都以失败告终。

%20
绿洲的池塘乃是部族的重中之重,自然防御非常森严。

%21
少年盗天的实力,又只是刚刚晋升一转,手段匮乏得很。

%22
就在他寻找不到门路的时候,一个消息在他的同龄人之间,广为传播。

%23
“什么,半个月后族中举办小比,得胜者便能够在池塘中选择蛊虫,作为奖励?”少年盗天也打听到了这个消息。

%24
他立即明白,机会来了。

%25
并且,这是他最有可能完成沙枭任务的机会!

%26
“我要参加小比,并且还要取得优胜!”少年盗天在心中,对沙枭道。

%27
沙枭嘿嘿冷笑:“臭小子,别以为老夫不知道你心中打的什么鬼主意。你是想借此,提升自己的实力,不断积蓄,想要将来推翻我的掌控吧?”

%28
少年盗天冷笑,竟直接坦言:“是又如何?自由谁不渴望。”

%29
“好啊!”沙枭却不以为杵,反而赞叹道,“你倒也率直,哈哈,你若是叫我一声爷爷,我便帮助你,助你夺得此次优胜。”

%30
“你!”少年盗天神情一滞,双眼中流露出愤怒之情。

%31
“怎么你不愿意?你可要想好了,除了我帮助你之外,凭借你的实力和资质,你怎么可能得到此次小比的优胜?”沙枭冷笑连连。

%32
少年盗天眉宇阴沉。

%33
“我出身贵族,一身荣耀,居然要卑贱讨饶!这是数典忘祖,败坏我孙家声誉。”

%34
“但是……”

%35
“我若不这么做,就真的没有任何希望。”

%36
“我的实力太弱了,回家的希望渺茫至极。若是连眼前这一关都不过了,就别提什么将来了。”

%37
“可恶!”

%38
少年盗天双拳捏紧,陷入极度的犹豫当中。

%39
之前,他虽然喊过沙枭“爷爷”,但那是生死存亡之间。此时此刻,他和沙枭分别,外界的压力并没有那么巨大,这就让少年盗天心中摇摆不定,抉择间心中非常的沉重和痛苦。

%40
“爷爷……”最终,他选择退让,从牙缝间艰难地挤出这两个字。

%41
“哈哈哈。”沙枭的狂笑声,立即充斥少年盗天的心头,“乖孙子,既然你喊我一声爷爷,那爷爷我就不可能让你落败。这是爷爷我交给你的致胜法门,你好好接收,化为自身实力,必定能在小比中优胜。”

%42
沙枭话音刚落,一股庞大的信息就猛地袭进少年盗天的脑海之中。

%43
少年盗天顿时低呼一声,面色扭曲起来,双手下意识地抱紧脑袋,巨大的信息不断冲击他的脑海,让他感到分外的痛楚,整个脑袋就像是要被撑爆了一样。

%44
信息袭击的时间,只有短短的九息时间,但是在少年盗天的感觉中,却像是一年那般漫长。

%45
他艰难地熬过来,浑身都是冷汗,脸色发白,惨白如纸,全身更是在微微颤抖,极其狼狈。

%46
沙枭手段爆烈,不过传达过来的消息,都是真材实料,价值颇大。

%47
其中内容,大部分是拳脚功夫,颇为精湛、简练。还有一部分,则是蛊方,蛊虫运用之法,甚至还有一记杀招。

%48
凡道杀招糊眼灰。

%49
这招是以沙坑蛊、炊烟蛊、清水蛊组成,步骤有八道,一旦催动成功,就能喷吐出一股白色的烟灰。

%50
这烟灰落到人的眼中去,就能让人双眼失明,蒙上白灰。

%51
中招之人,却万万不能用水来清理,一旦清理,白灰就被迸发热量,将蛊师的双眼灼烧,最终成为瞎子。

%52
“这招好生阴损,不是我的风格。”少年盗天细细查看,顿时心中一沉,面色不喜。

%53
他在穿越之前,出生在贵族家庭,从小受到良好教育,为人刚直正派,嫉恶如仇。

%54
正因为如此,他穿越而来,成为贫困孤儿,却恪守良规,保持美德品行,足足十几年,就算有成人的智慧,也混得不如意。

%55
“糊眼灰杀招需要的三只蛊虫,偏偏就是我身上的所有蛊。”

%56
“由此看来,这个沙枭蓄谋已久,其实早就想要增强我的实力。哼!”

%57
少年盗天暂且不想练习糊眼灰杀招,而是将全部的精力都投放到拳脚功夫上。

%58
他前世也是一个战士,虽然不是擅长近身肉体格斗,但多多少少也有些功底。他真正的战斗造诣,也足够深厚的。

%59
因此练习这些拳脚功夫,并无阻碍,一帆风顺得很。

%60
就这般练习了几日,少年盗天便感到自己战力飙升!

%61
“这个世界中的拳脚功夫,并不单纯,而是搭配蛊虫形成的完整招数。”

%62
“有什么样的蛊虫,就有什么样的招数。”

%63
“我此前苦于无人指点,这才弱小无为。现在得了这份秘籍,勤修苦练,族中小比便有了六成把握了。”

%64
“就算不用糊眼灰,我也能摆平大多数的敌手。”

%65
少年盗天对这个卑鄙的招数,相当反感,不想用它。

%66
他怀着信心,很快便迎来了族中小比。

%67
第一次登上擂台,他并不被周围人看好。

%68
甚至还有取笑他的声音传入耳中。

%69
而他的对手,也是一脸的傲慢,指着他的鼻子道:“原来是你这个废物,你乖乖认输,我就不打断你的腿。”

%70
少年盗天愣住,站在原地,一动不动。

%71
周围人哈哈大笑。

%72
“这就怂了啊,吓得动弹不得了。”

%73
“就他这个样子,之前怎么通过流浪的考验?”

%74
“大概是运气吧。我倒是听说,他是在千钧一发之际,被族中的蛊师从狼口中救下来的。”

%75
在这样的氛围下,少年盗天的对手,看向他的目光中,便有增添了一份鄙夷。

%76
但下一刻,少年盗天握了握自己的手,感受十指间的力道。

%77
“总算是轮到我操纵了。”方源心中一喜。

%78
原来,少年盗天愣神,是因为方源得到了掌控权,一时间没有反应过来而已。

%79
现在反应过来,他立即动手。

%80
凡道杀招糊眼灰!

%81
呼的一声,一大蓬的白灰喷在他的对手的脸上。

%82
对方猝不及防,连忙后退三步,然后惨叫起来:“啊!我什么都看不见了,这是什么东西!”

%83
“给我下去罢你!”方源几个大步,蹦到对手的面前,然后伸脚一踹。

%84
他的对手就被直接踹下了擂台。

%85
得胜!

\end{this_body}

