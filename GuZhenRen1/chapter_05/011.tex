\newsection{宝黄天输送蛊虫}    %第十一节:宝黄天输送蛊虫

\begin{this_body}

“怎么样?有没有考虑好?你可是琅琊派的客卿太上长老,完全可以用你的门派贡献,来换取其中的法门。(www.MianHuaTang.cc 棉花糖小说)”琅琊地灵笑道。

中洲方面,黑楼兰已被影无邪强逼投靠,方源并不知情,还在和琅琊地灵联络。

双方谈论到克制春秋蝉的方法,琅琊地灵不愧是源自长毛老祖,琅琊福地家大业大,底蕴深厚。

通过刚刚的交谈,方源惊讶的发现:琅琊地灵手中掌握的方法,不下十种!都是可以克制春秋蝉的有效手段。

这些手段多种多样,选择多了,一时间,方源都感到有些迷茫。

但他很快定下心神,想到一点,问道:“不知我现在的门派贡献有多少呢?”

“三百!”琅琊地灵犹豫了一下,这才答道。

“三百……”方源皱了皱眉头。他知道自己加入琅琊派之后,就没有为门派做什么事情。这笔门派贡献,完全是自己将荡魂山、落魄谷、智慧蛊,借给琅琊地灵,才得来的。

然而不管是荡魂山、落魄谷,还是智慧蛊,都是绝世无双之物,居然只值三百的门派贡献?

当即,方源心中生疑,但他却没有直接质问,而是婉转地问道:“太上大长老,我不太清楚你这三百贡献,是如何计算的,还请你为我解惑啊。”

琅琊地灵见方源称呼自己为“太上大长老”。心中顿时高兴起来。

他坦言道:“这很简单。荡魂山算你一百贡献,落魄谷一百贡献,智慧蛊同样是一百贡献。”

方源语气微微一沉:“智慧蛊乃是九转仙蛊,怎么和荡魂山、落魄谷一样的贡献?”

“你不必疑惑。”琅琊地灵迅速答道,“这三者都算是你借给门派的。我琅琊派还要承担地气的消耗,还要替你喂养智慧蛊。给你三百贡献,都是门规中清清楚楚记载的规定。我深知门派要做大,就必须要公正公平,你放心,该是你的本派不会亏待你,不是你的本派也不会补贴你。”

顿了一顿。琅琊地灵又继续道:“当然。你若是将这三者献给门派。那贡献就多了!甚至就让你来当这个门派的太上大长老,也无不可啊。”

方源听闻这话,不禁怦然心动。

担任琅琊派的太上大长老,岂不是获得了最大的权柄?就算不是琅琊福地的主人,也能调动福地中的毛民蛊仙等等力量吧?

但旋即,琅琊地灵又道:“不过你要成为太上大长老,必须脱离人族身份。改造成毛民。只有优秀的毛民,才能领导伟大的毛民一族!”

琅琊地灵的语气中,四处洋溢着大毛民主义的浓烈气息。

方源顿时打消了刚刚的想法。

他一点都不怀疑,琅琊地灵手中有将人族改造成毛民的手段。[求书网www.Qiushu.cc想看的书几乎都有啊,比一般的小说网站要稳定很多更新还快,全文字的没有广告。]但若方源转成毛民,代价未免太过高昂。

且不说,当今天下人族大势无可撼动。毛民等异人,就算成仙,也要靠边站,遭受排挤。

首当其冲的一点,就是毛民等等异人。先天就有缺陷,无法成为九转尊者。

方源矢志永生,修为只是一个辅助。但很显然,修为越高,越接近永生这个目标。

这个道理显而易见。试问蚂蚁和大树,哪一个更接近永生?

方源只有三百的门派贡献,琅琊地灵的这笔账算得很清楚。没有多给方源一分。也没有少给一毫。

方源旋即敏锐地意识到,这三百的门派贡献相当珍贵,价值很高,若能充分利用,一定能对现在的自己,有着巨大帮助!

于是他思考了一下,并没有急着耗费这些贡献,去兑换任何一种克制春秋蝉的法门。

那边,琅琊地灵又道:“门派贡献,也是拿来花的嘛。你不必太过珍惜了,本派有许多任务,可以执行。只要完成任一一项,都能增添你的门派贡献。”

“哦?都有哪些任务?”方源顺口问了这么一句。

琅琊地灵想利用方源的力量,自然知无不言言无不尽。

方源很快摇摇头,满脸失望之色。

这些门派任务,基本分为两种。一种是炼蛊材料的采集,另外一种则是陪练,和毛民蛊仙对战,培养他们的战斗素养。

原来,经过秦百胜等人强势攻击琅琊福地之后,琅琊地灵由白毛地灵转变黑毛地灵,管理琅琊福地的理念也发生了天翻地覆的变化。

黑毛地灵鼓励战斗,想要毛民当家做主,成为蛊师世界的霸主,将人族等等其他种族都统统踩在较小。

于是他积极锻炼麾下的毛民蛊仙,想要将他们的战斗力栽培出来。

在此之前,这些毛民蛊仙都只会炼蛊,平时的基础锻炼都没有,战斗素养已经不能用“渣滓”一词来形容了。

现下,方源可没有闲情逸致,去帮助黑毛地灵寻找什么炼蛊材料,更没有什么耐心,去当毛民蛊仙的保姆。

“这样吧,太上大长老,我这边缺少蛊虫。你先通过宝黄天收购一些南疆蛊虫,检查一下有没有问题。没有问题的话,就传送过来。”方源岔开话题,提出要求。

琅琊地灵自然没有拒绝。

一批批的蛊虫,很快通过宝黄天,穿梭通天蛊,到达方源的至尊仙窍之内。

首先第一批,是大量的凡蛊,其中包括酒囊花蛊、饭袋草蛊、甘泉蛊、餐风饮露蛊等等。

现在的情况不同了。

之前,方源是仙僵,无需进食。

但现在他已经重获新生,肉体鲜活,需要定时补充食物和水。

就算是蛊仙。不吃东西,也会饿死的。除非是有相关程度的食道道痕。

除了这些食道凡蛊之外,还有一些衣蛊。

方源对火焰披风蛊、飞烟蛊并不满意,火焰缠身的效果太显眼,太花哨了。

这些衣蛊一到手。他就立即将火焰披风蛊换掉。

还有屋蛊。

比如木道蛊虫三星洞。

用了之后,能化为一座树屋,树干中空。

第二批是推杯换盏蛊、星门蛊。

利用宝黄天,相互沟通,并不安全。

宝黄天只是个开放的市场,而不是隐秘的沟通桥梁。

而推杯换盏蛊,利用空门往来传输。避免了五域界壁。实是极品凡蛊!

不管是推杯换盏蛊的蛊方,还是星门蛊蛊方,琅琊地灵早年时分,就从方源手中获得了。

依照前一代琅琊地灵的个性,自然要多加炼制。

因此这两种蛊虫,都是从琅琊福地中直接提取出来的。刚刚进入宝黄天的时候,两只蛊虫都爆发出强烈的宝光。很是吸引了许多蛊仙的注意。

明明只是凡蛊,但是宝光强烈得匪夷所思。

这种异状引发了许多蛊仙的关注,他们纷纷上前询问,但都被琅琊地灵回绝。

第三批是一些凡蛊,来自琅琊派的库存,方源打算用它们来测验自己的至尊仙窍。

毕竟之前的检验,只是流于表层的观察。

要真正弄清楚至尊仙窍的奥妙,还得用蛊虫作为工具。

第四批是重头戏。

来自剑仙薄青的剑道仙蛊!

方源从仙僵薄青身上盗来的好些仙蛊,然后他利用智慧光晕和智道手段,将这些仙蛊哄骗住。再徐徐炼化。

他参加义天山大战之前,只有换魂仙蛊成功炼化。而后在义天山大战的关键时刻,被方源当做绝地反击的手段。

如今,换魂仙蛊仍旧在方源的手中。

至于其他剑道仙蛊,里面充斥着薄青意志,比换魂仙蛊更加难以炼化,因此都留在了狐仙福地里。并没有被方源带到义天山去。

毕竟,这些仙蛊没有炼化,带过去也用不了。

事实证明,幸亏方源没有随身携带。

要不然,按照事情的发展,这些仙蛊都留在肉身之中,反而会便宜影无邪。

这些剑道仙蛊分别是八转慧剑蛊,七转剑遁蛊,七转浪剑蛊,七转剑眉蛊、七转飞剑蛊。

琅琊地灵只见其中的剑遁仙蛊、飞剑仙蛊,给方源传了过来。

双方迅速交易。

第一时间,方源就用一块随手拾取的石子,买下了剑遁仙蛊和飞剑仙蛊。

“宝光高达数十丈,刚刚发生了什么?”

“好像是仙蛊交易!!”

“怎么可能?”

“我刚刚亲眼看到了。一方迅速买下了两只七转仙蛊,速度之快,迅雷不及掩耳啊。”

“买下两只七转?我的天!我可是连一只六转仙蛊都没有啊,究竟是谁这么财大气粗?”

之前的几批蛊虫转移,已经吸引了不少有心人的注意。这一场转移,涉及到了仙蛊,立即引发了宝黄天中议论的热潮!

仙蛊交易,通常都是以仙蛊相互换取,即便在宝黄天中也十分罕见。

因此,方源和琅琊地灵的交易,就好像是一块巨石砸在平静如镜的湖水之中,引起轩然大波。

“两只仙蛊……每一只的宝光,都有七八十丈。但是另一方竟似是用了一块石子付账的。”有幸亲眼目睹的某位蛊仙,传入神念道。

这个消息,再次引发了蛊仙们议论的狂潮。

“这不可能!”

“一个石子?你在开玩笑吧?”

“看来这不是单纯的交易,而是利用宝黄天进行的输送。”

“就算是输送仙蛊,也难得一见啊。宝黄天的手续费用,可是按照宝光计算的。”

一时间,宝黄天中各种神念相互飚射,无数遗留在里面的蛊仙意志,都在激荡,相互交流。

乱糟糟的一片,好像是凡间的菜市口。

因为方源和琅琊地灵的联络,整个宝黄天都乱了套。

ps:我知道了,上个月欠大家5更。这是今天第一更,9点左右会有第二更,是昨天月票突破100的加更。(未完待续。)

\end{this_body}

