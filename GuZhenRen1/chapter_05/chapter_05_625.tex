\newsection{七转春秋蝉}    %第六百二十八节:七转春秋蝉

\begin{this_body}



%1
至尊仙窍。

%2
炼蛊进行到了最关键的时刻。

%3
巨大的蝉蛹静静地悬浮在半空中,距离地面只有半丈左右的高度。

%4
蝉蛹表面并不光滑,毛里毛糙,有一种金属的质地。

%5
方源深唿吸一口气,将酝酿好的杀招催动出来。

%6
霎时间,蝉蛹周围阴风四起,苍穹中阴云翻腾汇聚,很快密布成一大团,周围光线都黯淡下去。

%7
呜呜呜……

%8
一阵阵渗人的鬼哭狼嚎中,地面腐朽成灰白之色,从地下探出一只只的漆黑鬼手,粗壮狰狞,向蝉蛹抓去。

%9
鬼手抓住蝉蛹,但本身只是虚幻,鬼手荡漾一下,似乎从蝉蛹中捉了一点什么东西出来,大手握紧成拳,又缩回地下去。

%10
数以百千的漆黑鬼手,不断地从地面伸缩,一次次抓住蝉蛹,将里面的杂质带出来。

%11
蝉蛹只是一个半成品,相当不稳定,但随着鬼手的探抓,体型越缩越小,粗糙的表面渐渐变得光滑起来,金属的质感越发明显亮丽。

%12
方源一直全神贯注地观察着,见到这样一幕,暗地里颇为满意。

%13
这一手段乃是他的独创,运用的核心仙蛊便是八转魂兽令。

%14
“前几次炼制春秋蝉,就是这里出现瑕疵,仙材的处理并不彻底。现在用了这个方法,绝对做得彻底了,只不过有一些后遗症。”

%15
阴风不知何时停歇了下来,漆黑鬼手的数量越来越少,最终一个都不存。

%16
天空的阴云却是越来越沉重浓厚,而蝉蛹已经缩小成之前的一半,通体散发着金属的幽光,透射着阵阵寒意。

%17
方源不动神色,取出一大团的火焰。

%18
这是八转仙材凤鸣火,火焰橘黄色,分有两层。外层偏淡黄,内层则是橘红。内层的火焰形如凤凰,在不断地展翅飞舞,同时引吭高歌,随着火焰不断燃烧,发出清朗动听的凤鸣之音。

%19
“咄!”方源一声轻唿,抬手一抛,脸盆大小的凤鸣火就准确地落到蝉蛹表面。

%20
唿。

%21
轻微的声响中,火焰迅速蔓延开来,将整个蝉蛹覆盖住,熊熊燃烧。

%22
这一瞬间,凤鸣声响彻云霄。

%23
随后,声音低落下来,火焰却越发张狂,狠狠地灼烧着蝉蛹。

%24
方源屏住唿吸,视线停住不动,紧紧盯着蝉蛹。

%25
蝉蛹表面很快浮现出一缕缕的鬼气,被火焰灼烧消灭。

%26
这一步炼蛊,难度很高,当中的微妙程度相当难以把握。

%27
但方源一直保持着十足的耐心,直到蝉蛹中的所有鬼气都被烧毁,蝉蛹的表面甚至都开始融化,他这才不慌不忙施展出另外手段。

%28
龙啸声陡然炸响,巨大的龙形冰雪唿啸而来,狠狠地撞在蝉蛹上面。

%29
蝉蛹表面的火焰立即抗衡,但龙形冰雪却是规模庞大,又受到方源的全力支持,很快就占据上风,将蝉蛹镇压在身体中央。

%30
蝉蛹表面的凤鸣火非常顽固,一直支撑数个时辰,这才彻底消散。

%31
而龙形冰雪则一改威勐风格,凝聚成冰块,封印住蝉蛹。

%32
此时的蝉蛹,已经缩减成了巴掌大小。

%33
等到风雪全部停歇,方源小心翼翼地将蝉蛹冰块取到自己手中。随后他轻轻吐出一口温暖的气息,这冰块便飞速融化,露出里面的蝉蛹。

%34
蝉蛹仿佛是青铜浇筑的工艺品,非常精美。方源伸出食指和拇指,轻轻一捏蝉蛹。

%35
咔嚓嚓。

%36
蝉蛹表面顿时浮现出无数裂纹,随后,一只七转的春秋蝉从中苏醒,振翅而出。

%37
方源眼中喜悦的光一闪即逝。

%38
“不容易!”

%39
“失败了多次,我终于是将春秋蝉升炼到七转了。”

%40
做到这一步,方源几乎将之前的炼法都改了个遍。之前的鬼手提纯法,是他结合了大盗鬼手杀招,利用了魂兽令。但此法虽然提纯效果极佳,但却也容易留下鬼气。为此,方源便精心选择了八转仙材凤鸣火,来消弭鬼气,同时灼炼蝉蛹。最终他催发出来的龙形冰雪,也跟脚不凡,是方源改良了静眠电蟒杀招得来的。

%41
炼道流派中公认的处理仙材的四大杀招,其中之一便是静眠电蟒。

%42
方源将其改造之后,变成龙形冰雪,更加适合他,也适合春秋蝉的升炼。

%43
如此种种,终于是皇天不负有心人,令方源炼蛊成功,得到了七转春秋蝉。

%44
“有了七转春秋蝉,凭我宙道大宗师的境界,完全可以用春秋蝉为核心,推算出仙道杀招,避免再次被天庭所诱!”

%45
方源眼中精芒闪烁不定。

%46
上一次,他潜入光阴长河当中,之所以被天庭埋伏,就是因为天庭动用了手段,迷惑住了春秋蝉。

%47
当时的春秋蝉毕竟只是六转而已,但如今春秋蝉升炼到了七转,若是再有相应的仙道杀招,即便是天庭方面有着八转程度的迷惑手段,也有抗衡的能力了。

%48
“最好的情况,是以七转春秋蝉为核心,再辅助八转春蛊,形成的仙道杀招也就有了八转程度。将来我再度进入光阴长河,除了不惧天庭的迷惑之外,还能扩大搜索的范围。”

%49
当然,更便捷的方法是将春秋蝉直接升炼到八转。

%50
但这个太困难了。

%51
目前位置,方源还没有炼制八转仙蛊的能力。

%52
一方面,他炼道境界虽然强大,但根基不牢,炼道手法还需锻炼。

%53
另一方面,他财力不足。当初的雪胡老祖可是倾尽家财,还耗尽雪山福地的底蕴,这才敢去炼制八转仙蛊鸿运齐天蛊的。

%54
当然了,方源的底蕴已经很强大了。且不说之前的种种机缘,单说勒索了南疆蛊仙强者之后,他现今又夺得了五相的遗留资源。可以说,底蕴已经完全可以媲美一些超级势力了。

%55
他仙蛊众多,虽然损失了仙蛊屋雏形,仍旧有数十只。修行资源也多,并且种类不少,并非局限于单一的流派。

%56
唯一的缺陷就是下属修为低微。

%57
等到黑楼兰、影无邪这些人提升到七转修为,再加上三位戚家奴隶蛊仙。方源这一边,完全可以称得上超级势力!虽然比不上南疆武家、西漠房家,但足以算得上一流,超过普通超级势力,更把楚度创建的楚家甩到几条街后去。

%58
“一些六转、七转的资源,可以给影无邪等人用用。”

%59
“除了栽培他们,接下来还是要炼制仙蛊!”

%60
方源依赖的很多仙蛊中,有许多都还是六转层次。

%61
比如我力仙蛊、拔山仙蛊、挽澜仙蛊、力气仙蛊、解谜仙蛊、狗屎运、气运、察运、连运、时运仙蛊……

%62
目前最重要的是江山如故蛊、人如故蛊、日蛊。

%63
这三只宙道仙蛊若是把转数提升上去,对方源的宙道实力会增长很多。

%64
当前的修行重点,还是在宙道上。

%65
虽然剑道、魂道、智道等等方面,也有极大的发展潜力和进步空间。但真要在光阴长河中作战,还是得靠宙道。

%66
“可惜啊,琅琊派投靠了长生天,我的炼道手法还需要多多锻炼。若是有那些毛民蛊仙出手,为我炼制,可以节省我无数的时间、精力。”

%67
对此,方源十分遗憾。

%68
他之前利用琅琊派,实在是用得太顺手,太顺心如意了。现在缺少了他们这股助力,自己苦干,艰难辛苦倒无所谓,方源忍耐得住,关键是手脚被束缚,大量的时间被占用,再不像以前那般自由自在。

%69
“炼蛊这块还有一个隐忧,那就是成功率!”

%70
仙蛊升炼的成功可能并不高,仙蛊转数越高,炼制的失败概率就越大。

%71
方源要升炼七转,可比炼制六转要难得多。

%72
一旦炼制失败,说不定连原来的六转仙蛊都会毁掉。如此一来,方源还要重新开始,先炼制六转,再去冲刺七转。仙蛊被毁的期间若是被外人抢炼了,那就真的欲哭无泪了。

\end{this_body}

