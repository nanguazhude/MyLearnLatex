\newsection{紫薇后怕}    %第六百零九节:紫薇后怕

\begin{this_body}

“怎会如此?怎会如此……”南疆战报已经传达过来,紫薇仙子口中呢喃,双眼中竟是迷惘。

以往雄阔空明的大殿,此刻也充斥着疑惑。

方源不仅没有渡劫,而且本身已经达到八转!甚至,他还因此在五界山脉布置了陷阱,俘虏了天庭蛊仙君神光!

南疆蛊仙被方源俘虏的事情,紫薇仙子早已知晓,没想到这么快,同样的遭遇就降临到了她天庭一方。

天庭的蛊仙竟被人活活俘虏!

这是多久没有发生的事情了?

在得知这个消息的一瞬间,紫薇仙子下意识地联想到了历史上数位魔尊的赫赫战绩。

她的眉宇间更显哀愁,并且比平常还多了一份慌张。

尽管她一直高估方源,但未想到今日,方源的成长仍旧大大超出了她的估料。

“自方源拥有逆流护身印后,便是一个巨大麻烦。如今成就八转,掌握夏槎的种种手段,尽管是借助了五界山脉的地利,很大程度上削弱了对手,但他的确已可对天庭蛊仙产生威胁!”

紫薇仙子长长叹息一声。

她也早已料到,会有这样的一天出现。所以,她千方百计想要将这种危险提前扼杀,但万万没想到方源的进步是如此的迅猛和神速!

“方源进步的速度越来越快了,似乎是量变引发了质变。只是他究竟是如何不声不响地成就了八转?”

紫薇仙子分外不解。

原本,她一直以为,方源有着如此神速的修为进展,应当是运用了宙道的手段,加快了自家仙窍的光阴流速。

但现在看来,紫薇仙子忽然惊觉,方源手中定然还有一套绝妙的方法,可以违反常规,提升自身修为!

“那么……这个方法是什么?”

紫薇仙子没有继续推算,而是离开大殿,来到魔尊幽魂的囚禁之处。

大阵轰鸣起来,良久之后,紫薇仙子带着一脸的震撼之情,回到了大殿。

“至尊仙胎蛊!至尊仙窍……原来如此!”紫薇仙子口中喃喃,脸上神情竟有些失魂落魄。

魔尊幽魂的野心,震惊到了她。

更令她忌惮的是,至尊仙胎蛊已然炼成,被方源得到!

她的心中旋即涌起浓郁的苦涩之情。

魔尊幽魂隐藏得好深!

若非这次方源暴露了线索,令紫薇仙子起了疑心,这才针对性地搜魂。否则,她还会被蒙在鼓里。

“难怪方源的修为突飞猛进!这一次,他能成就八转,应当就是吞并了诸多南疆蛊仙的仙窍,跨越了灾劫。”

“他能够催动夏扇杀招,正面对抗其他八转的攻势,证明他也吞下了夏槎的仙窍。如此一来,才有足够多的宙道道痕,增幅夏扇杀招。”

“仙窍只能以大吞小,但至尊仙窍却是特殊的。但吞并夏槎仙窍八转洞天,需要宙道大宗师的境界,方源怎么会有?”

“等等!”

忽然间,紫薇仙子的脸色变得极差。

她脑海中种种线索,都融汇贯通起来,之前的迷雾统统消散。

“方源有着手段,可以从梦境中提升境界!”

“难怪他进入南疆,一直在搜刮梦境。他的宙道大宗师境界,一定是从梦境中得益。”

“嘶……”

想到这里,饶是位高权重、见多识广的紫薇仙子,也不由地微微倒抽一口冷气。

方源拥有至尊仙窍,可以随时吞窍,跨越灾劫。

而吞窍的最大难关是流派的境界。

境界的提升,非常艰难,往往需要数百年光阴。但方源却可以利用梦境,迅速地提升境界。

如此一来,就进入了良性循环,爆发出来的修行速度,足以让整个人族修行历史的长河,都掀起惊涛骇浪!

“不,更准确地说,可怕的是魔尊幽魂呐。方源不过是受到天意操纵,重生过来,破坏魔尊幽魂逆天大计的棋子,只不过最终令他脱离了掌控。”

紫薇仙子全身发冷。

尽管天庭方面已经将魔尊幽魂囚禁,甚至紫薇仙子还每隔一段时间,就对魔尊幽魂严加拷打,进行搜魂,搜刮出关键情报。

紫薇仙子一阵后怕。

如果真让魔尊幽魂成功,那么他本身就有全流派大宗师的境界,杀性十足,一路吞窍,配合影宗、僵盟的势力,完全可以修为一路飙升到八转程度!

到了这种程度,魔尊幽魂本体完全就是亚仙尊战力。

更关键的是,等到五域乱战,梦境迭生,魔尊幽魂还有更近一步的可能。引魂入梦这招就可看出,他的野心何其庞大!他早已经在此方面,做了无数的准备。真的让他做到这一步的话,优势就太大了。哪怕是凤金煌这样的大梦种子,恐怕都没机会成长起来。

“幸亏他在最重要的关卡上,没有顺利突破。如今换做方源,顶替了他原本计划中的位置。从这点来看,他没有成功,但也没有完全失败。”

想到方源,紫薇仙子又深深皱起眉头。

“方源舍弃那么多的南疆蛊仙不去对付,优先活捉君神光。他的行为和我对魔尊幽魂搜魂一样,是想从君神光的魂魄中搜刮出天庭的布置。”

“不好……光阴长河!”

长河滔滔,从来处来,到去处去。

雄阔至极,浩浩荡荡,一直突破到视野的尽头。

乘坐在仙蛊屋中,方源静静地看着河水滔滔,起伏不断,激起无数灿烂多姿的浪花。

光阴长河!

没有错,进入仙窍福地,争取到了关键时间,方源就利用了定仙游组成的仙道杀招,成功脱离。

随后不久,他就进入了光阴长河。

此刻,君神光的魂魄,在他的至尊仙窍中,已经在承受着不断的搜刮。

正所谓,知己知彼百战不殆。

方源迫切地需要知道,天庭在光阴长河中究竟有什么样的布置!

“我这一次在五界山脉设伏,大战一场,虽然俘虏了君神光,但是也暴露出了许多情报。”

为了不给天庭方面应对的时机,方源便想打一个措手不及,在这个微妙的时刻,直接进入光阴长河。

“这个陆畏因很不简单。他出手太及时,在最关键的时刻,救下了那些南疆七转。恐怕他极有可能,早就躲藏一旁了。”

“乐土仙尊生性仁慈宽厚,拨乱返正,弥补五域幽魂创伤。尤其是南疆,要论五域总体的声望,十大尊者中当属他最高,最得人心。”

“陆畏因身为他的当代传人,自然也继承了这种声望。这个人虽然对付过我,但他真正的目的,似乎并不单纯,身上始终笼罩着一层迷雾。”

“可惜我的记忆中,五百年前世并未有这一号人物啊!”

思考到这里,方源越发觉得,自己有关前世五百年的记忆并不可靠。

当然,也有一种可能,是因为自己重生之后,间接的影响越来越大,也波及到了陆畏因这种八转层次。

方源微微摇头,将陆畏因暂且放之脑后,又想起了陶铸传承。

他心中颇有遗憾。

陶铸真传,必定是和五域界壁有关。

虽然五域界壁,会在未来相继消散无踪,五域联合成一体。陶铸对于如何穿越界壁的研究成果,看起来就像是一个笑话。

但仔细想想呢?

陶铸能够建造出五界山脉,这种模仿五域界壁的手段,不就可以替代界壁的位置,进行防御了吗?

尤其是对于方源而言,界壁中战斗对他极为有利的。

“可惜了,可惜了,当时大战情境,绝不容许我收取这份传承,让我与陶铸传承失之交臂。”

方源智道境界不弱,心中有一种强烈的感觉,这一次没有得到陶铸传承,对他个人而言,是一个颇大的损失。<!--80txt.com-ouoou-->

\end{this_body}

