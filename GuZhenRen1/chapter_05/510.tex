\newsection{炼化魂兽令}    %第五百一十节:炼化魂兽令

\begin{this_body}

至尊仙窍。求书网小说qiushu.cc

一只蒲扇大小的黑幽鬼手,正在半空中不断地胡乱飞舞。

它五指紧握,但里面的八转仙蛊魂兽令,却是极力挣扎,想要飞出来。

黑幽鬼手被这只魂兽令不断地冲撞,鬼手上不断凹凸,五个手指头都在乱颤,有把握不住的趋势。

不过,就算是魂兽令逃出鬼手的束缚,在这方源的仙窍中,也已经是无路可逃的。只能算是从一个小牢笼里,逃到了一座大的牢笼。

方源暗暗思量:“我这构思出来的仙道杀招大盗鬼手,乃是以九转杀招鬼不觉为核心,七转大盗仙蛊等等为辅助。因此就算是八转仙蛊魂兽令,都能强行盗取出来。只是……此招却不能持久,没有针对仙蛊的封印威能。这只魂兽令还得我努力一番,才能将其化为己用!”

思量着的时候,大盗鬼手越加不堪,里面的魂兽令仙蛊就要挣脱束缚,飞逃出来。

不过此时,方源驾驭大盗鬼手,已经到达了目的地。

封天山!

这本来是方源封印自家的仙僵肉身所用,针对天意。

但现在方源的仙僵肉身,已经被分魂占据,成为六转宙道分身,再用不上封天山。

“去。”在方源的操纵之下,大盗鬼手直接进入封天山内部洞穴之中,直接窜入蛊阵当中。

方源开启蛊阵,大盗鬼手终于不再颤抖,里面的魂兽令仙蛊被暂时镇压住,虽然还在挣扎,但是幅度已经较刚刚之前,缩减了许多倍数。

“一直这样勉强镇压住魂兽令,并非长久之计。当务之急,还是要寻觅一处地方,将这只八转仙蛊尽快炼化才是!”

方源这样想着,速度不降,出了豆神宫之后,就顺着桃花迷林杀招的漏洞,直接撤走,远离这片战场。

“他就这样走了?!”方源的行动,让诸多蛊仙都分外诧异。

陈衣更是恨得牙痒痒,却没有办法,他不仅被青仇针对,同时又被房功纠缠。

“好一个决断。”房睇长悠然赞叹一声,望着方源离去的背影,感慨不已。

方源离开战场后,立即迎头碰上无数的魂兽。

魂兽规模相当巨大,简直是铺天盖地而来,令方源怦然心动。但很快他念头一转,就忍耐住,没有动手屠戮,大规模猎取魂核。

他最需要做的,还是尽快地炼化这只八转仙蛊魂兽令。txt全集下载www.80txt.com至于那座八转仙蛊屋豆神宫,就让他们三方八转存在去争夺吧。

方源对自己的认识十分清晰。

就算是撑起逆流护身印,方源也不过是能战八转,拥有八转战力而已。他现在出手,掺和在三大八转存在中,和他们一同争夺豆神宫,希望很小,变数很大。

一旦他暴露出了逆流护身印,也就暴露出了身份,说不定还会引发房家的反戈一击。

这对于方源而言,完全得不偿失,因为他还要和房家搞好关系,借助青鬼沙漠,帮助自己魂道修行。

他身上和房家的盟约,也是一个因素。要解决这种盟约,可不容易,需要充分的时间进行准备。

魂兽令虽然被方源抢夺,但青鬼沙漠的魂道道痕,仍旧在侵蚀房家的桃花迷林战场杀招。

围绕着战场,仍旧有海量的魂兽聚集在这里。

方源非常轻松地就撤离了这里。

因为他拥有阎罗杀招,将鬼不觉杀招能够充分地利用起来,青仇都发现不了他,更何况其他魂兽?

他一路疾飞,毫不留念,直接飞出了青鬼沙漠之后,再确认安全,左右无人的情况下,这才停歇下来,落到一处无人的平凡沙漠当中。

方源深入地底,布置了一座临时蛊阵,谨慎地消去了气息和痕迹。

做完这一切后,他开始专注精神,全心全意地来对付八转仙蛊魂兽令。

魂兽令被青仇炼化,里面充斥着青仇意志,方源若要炼化它,十分困难。

这是炼蛊中最基础的部分――单炼。

就是单独炼化一只已经存在的蛊虫。这种单炼的本质,是用自我的意志来取代蛊虫本身蕴藏的原先意志。

蛊师们运用真元炼化凡蛊,是因为真元属于个人独有之物,拥有个人意志。

蛊仙运用仙元炼化仙蛊,也是同样的道理。

野生蛊虫中蕴藏野生意志,桀骜不驯,难以炼化。但人是万物之灵,若现在抢夺他人仙蛊,炼蛊的难度要更上一层楼。

方源在问津坞中,就筹谋良久,此时并非毫无准备。

七转仙蛊爱意,七转仙蛊自爱!

方源首先将这两只仙蛊取出来,他先催动自爱仙蛊,对准魂兽令施展。

魂兽令中的青仇意志顿时受到影响,顿时自怜自爱起来,接下来方源就算动作再大,这股青仇意志也不会铤而走险,自爆了魂兽令。

随后,方源又催动爱意仙蛊,将一股股的爱意,渗透进去。

两股不同的意志相互接触,顿时厮杀在一起。

爱意并非善于攻伐的意志,很快就被青仇意志杀得七零八落,溃不成军。

但是仍旧有零星至极的点点爱意,散落在蛊虫魂兽令当中,没有被完全剿灭。

这就是爱意的特点,也是自爱仙蛊的影响尚在。但随着时的流逝,这些残余的爱意都会被陆续剿灭。

方源继续催动爱意仙蛊,不断地将一股股的爱意灌输到魂兽令之中。

不管进来多少爱意,都被青仇意志剿灭,并且它自身减损很少。

青仇意志在魂兽令仙蛊当中,战力非常强悍,也非常顽固!

仙元剧烈消耗,方源却能够支撑,他有充足的仙元储备,之前的仙窍经营建设,让他此时底气十足。

过了半盏茶的功夫,方源终于停止往魂兽令仙蛊当中灌输爱意。

爱意在里面已经残留很多了,虽然绝大多数一进去,就被青仇意志剿灭掉,但剩下的爱意却也相当可观,达到了极限,再想要增长下去,几乎没有可能。

方源这才酝酿出我意,灌进魂兽令中去。

这我意一进来,顿时激起了青仇意志的愤怒,立即蜂拥而上,宛若暴怒的野兽。比较下来,之前青仇意志对付爱意,完全是和风细雨了。

方源我意和青仇意志展开厮杀,形势完全一面倒,方源我意溃不成军,根本站不住阵脚。

打个比方,魂兽令宛若营地,早已经是青仇意志的地盘。方源我意宛若抢滩登陆,每次只能灌输那么些去,军力根本铺成不开,来多少就被青仇意志消灭多少。

不过方源仿佛视若无睹,不停地灌输我意进去。

他的优势就在于,自身仙元充足,我意源源不断。而那青仇意志,却是脱离了青仇,乃是无本之源。

一炷香、两柱香、三炷香……随着时间推移,方源的局面渐渐打开,一步步站稳脚跟,最终和青仇意志二分天下,各占据魂兽令的一半地盘。

青仇意志剧烈翻腾,知晓自己不是方源的对手,居然想要引爆这个魂兽令仙蛊。

幸好这个时候,方源残留下来的爱意发动,酥化青仇意志的决心。同时方源又灌输假意,干扰青仇意志的判断。

最后,方源再大量灌进我意,一鼓作气,军势如潮,将剩下的青仇意志彻底淹没、剿灭。

“终于炼化了魂兽令仙蛊了。”方源吐出一口浊气,满头都是汗渍。

这一场下来,简直是比指挥一场千军万马的旷世大战,都要疲累。

算一算时间,外界五域居然过了三天三夜!

“幸亏我有智道手段,又有合适的仙蛊,否则还真不能够顺利炼化此蛊了。”

“唉,我虽然有大盗鬼手,能够利用鬼不觉,重现当初盗天魔尊的一丝神威。但到底是半路出家,恰逢其会而已。传闻中,盗天魔尊还有一招偷心,直接针对偷来的仙蛊里面的意志,能在瞬间炼化仙蛊,非常了得!”

方源心底叹息一声,收起魂兽令,又赶紧起身,重新向青鬼沙漠飞去。

“我现在有了魂兽令八转仙蛊,再辅助百八十奴杀招,极可能将那青仇奴役住!”

“有了青仇,我争夺豆神宫大有希望,毕竟它影响豆神宫程度最大。”

“就算是遭受房家盟约反噬,我也有能力保证自己不死,若是能夺得豆神宫,那绝对是有赚无赔!”

方源心中不断谋划。

他得陇望蜀,得到了魂兽令之后,又起心思,想要谋夺那座八转仙蛊屋豆神宫了。

但当他飞到青鬼沙漠的边缘时,就被一位蛊仙拦下。

“这位仙友请留步。”拦下他的是一位容貌普通的女仙,巧笑倩兮,浑身洋溢七转气息。

“你有何事?”方源心念一动,放缓速度。

女仙笑道:“我乃何辜,见仙友飞行极速,定然是心有目标。眼下这青鬼沙漠剧变,听闻是有巨宝真传现世所致,不知仙友掌握了什么情报?我愿意用重金换取。”

方源面色不变,心中却感到有些不妙。

原来房家抢夺豆神宫,造成浩荡声势,整个青鬼沙漠的魂兽都暴动起来,这点根本遮掩不住,被西漠无数蛊仙察觉,因此蜂拥而来。

方源炼化魂兽令,耗去三天三夜时间,已经引来许多西漠的蛊仙。

“恐怕大战已经结束了!”

“若是三方仍旧激斗,八转之争,任何一方都无法遮掩,威势浩荡恐怖,这些蛊仙又怎能查探不出来呢?”

“就是不知道最后大战结局如何?豆神宫究竟花落谁家?”

方源心中不住地思量。

ps:本书已经停更一百年,今天的更新又是一场意外。(未完待续。)

------------

\end{this_body}

