\newsection{掌握因果神树}    %第七百六十五节:掌握因果神树

\begin{this_body}

这几天,西漠正道的氛围,变得非常微妙。

虽然表面上,还是针对房家的诋毁谣言层出不穷。

但实际上针对房家的打压,已经是雷声大雨点小。

毕竟杀招偷生就放在那里,八转蛊仙也不能阻止房家的强势反击。

“我们前番突袭宝月绿洲的计划,已经有显著的成果了。”房家蛊仙为此欣喜。

房睇长的脸色却一直不佳:“可惜啊,若是没有燎萤戈壁的事情,那就比较完美了。究竟凶手是谁,到现在我们还查探不出来。可恨!”

虽然董家的嫌疑最大,但房睇长和房功都很明智,不想轻易就动手。

除了没有证据,师出无名之外,还有担心中计的缘故。

房家没有动作,董家动静却不小,这些天来董陆沉一直在调兵遣将,加强了对房家方向的防御。

董陆沉过得并不好,他在提心吊胆。若是下一刻传来房家进攻的坏消息,他毫不意外!

“快点调查吧,查出真凶来,还我董家清白!”暗地里,董陆沉是最期待房家的调查结果的。但表面上,他表现得很强硬,对宝月绿洲一事要求房家索赔。

不去报复进攻,而只是会索赔,这态度已经非常明显。

其余的西漠正道势力,也在猜测真凶会是哪一家。

他们把绝大多数的注意力,都投放在彼此身上。因为很显然,若是挑拨之计成功,无疑是其他的正道势力最得利。

这当中,嫌疑最大的就是挨着董家、房家的几个超级势力。

自然而然的,这几个超级势力的心理压力跟着也大了起来,期待着房家调查出真凶。

房家是在调查真凶。

但实际上,房家的力度并不大,甚至有些敷衍!

房家的真正危机,并不是燎萤戈壁的真凶,而是整个西漠正道的排挤和针对。

房家的处境很尴尬。

我若是真的调查出来真凶身份,指向某个正道势力,我是复仇呢还是不复仇?

不复仇,说不过去!

对房家的名誉是摧毁性的打击,对房家的那些六转、七转蛊仙也都交代不过去。

蛊仙乃是超级势力的高层,高层牺牲了,组织都不维护,都不复仇,那加入这种组织有什么用?为这个势力卖命真的值得吗?

若是复仇呢?

难道真的和某个超级势力死磕吗?

房家有杀招偷生,其他的超级势力当然也可能有类似的底牌!

关键是两个庞然大物死磕下去,得益的肯定不是彼此,而是其他的庞然大物。

这些天,房睇长已经冷静下来,他私底下和房功说了自己的观点:在调查真相的事上马虎一点。眼下最重要的是拖延住时间,将三座仙蛊屋修复好,将豆神宫炼化!

只要缓过这口气,房家实力暴涨,再调查出真凶。若是哪一方超级势力,房家占据大义,再去刁难,不仅师出有名,底气十足,而且能获取更大的利益。

房功听了这顿分析,深以为然,当即又夸赞房睇长一句,说他老成谋国。

不久后,房功就对外宣称:房家会积极调查真相,如今查探真凶已有许多进展。房家绝不会放过一个坏人,但也不会冤枉一个好人。

宣称的当天,房家蛊仙便带给方源大量的魂核,当中竟有一枚八转的太古魂核!

上一世,方源敲诈勒索房家的时候,都只是六转、七转的魂核,不见一颗太古魂核。

“显然,这是房家示好,来安定我心的手段了。”方源心中淡笑。

之前房睇长委婉警告,让方源吐出了二十滴天露。若是不出燎萤戈壁一事,这事情也就过去了。

但是燎萤戈壁凶杀案一出,房家之前的震慑效果大减,西漠正道势力之间暗流涌动,波云诡谲。人们在猜测哪一方暗中出手的时候,内心也被调得蠢蠢欲动。

所以,房家还要稳住方源这个强大的七转战力。

但归还那二十颗天露,显然不可能,房睇长不可能打自己的脸。

于是就用一颗太古魂核,进行补偿和安慰。

而方源冒险屠戮燎萤戈壁,所要营造的也是这样的结果。

方源不愿意看到房家的安稳。

若是房家一安稳,对于他的需求就少了,借助的程度就低了,开拓青鬼沙漠就会一拖再拖,并且力度也要大打折扣。

现在看看,房家主动向方源送上太古魂核来。

另一方面,方源需要这些西漠正道,来牵扯房睇长等人的注意力。若是房家安稳,房睇长就要更多的关注方源,说不定什么时候就察觉到不妥之处了。

说起来,房家也真是倒霉。

没有方源参与的上一世,房家渡过了危局,震慑了西漠正道,最终相互妥协,达成了某项约定。

如今吸纳了方源,有这样居心叵测的内鬼在,恐怕将来会一直局势不稳了。

接下来的日子,方源便坐镇天露绿洲,潜心修行。

青鬼沙漠方面,影无邪凭借魂兽令等仙蛊,发展势头极猛。

房家方面,则还在调动库存里的魂核交付给方源,履行彼此间的交易。

至尊仙窍中,各个地灵以及异人都在积极发展。

智慧光晕下,方源的宙道分身几乎时刻不歇地推算着。

仙道杀招——因果神树!

方源的头顶上空,升腾起一股袅娜青烟,青烟形成一株干瘦小树,树枝稀疏,结着两三颗小果子,好像营养不良的样子。

若是让天庭蛊仙看到这一幕,一定惊怒交加,捶胸顿足。

因果神树杀招乃是元莲仙尊的招牌手段,不想竟沦落到了方源这个魔头手中!

方源铲除了陈衣之后,虚窍中的蛊虫伴随着虚窍而毁,只捞到了陈衣的魂魄和尸躯。

通过搜魂,方源获得了两道元莲真传,分别是因果神树、来因去果。

有了原版杀招,方源最近凭借智慧光晕,以七转成竹仙蛊为核心,其他六转木道仙蛊为辅,勉强改良出了六转层次的因果神树杀招。

因为缺乏核心律道仙蛊“因”,还有核心木道仙蛊“果”,方源只得用大量的普通蛊虫替代。

如此一来,步骤变得繁琐,牵扯的心神更多,导致因果神树杀招酝酿时间过长,还不能用于实战。

“可惜我的木道境界普普通通,连大师级数都没有。”

“还有成竹仙蛊虽然属于木道,但作为因果神树的核心蛊,似乎不太妥当。”

“按照地球上的生物理论,竹子只是乔本,而不是木本啊。”

因此核心七转的仙道杀招,催发出来后,反而威能下降到了六转层次。

方源回想陈衣运用因果神树时的风采,端的神姿卓绝。因果神树杀招只要祭出来,堪称利于不败之地!

它虽然偏于防御,和逆流护身印、阳莽背火衣处于同一层次。但它更全面,潜力更大,还能用于其他方面,比如攻伐、腾挪、治疗。

来因去果杀招,便是因果神树基础上,发展出来的腾挪移动之法。只不过,它所针对的对象不是蛊仙本身,而是其他的对象,包括敌方蛊仙、植兽以及杀招。

毕竟是元莲仙尊的创作,层次极高。

方源要还原出这个杀招的神髓,任重而道远。

因果神树杀招还在起步,但方源附带掌握了一记木道仙招,却十分实用。

此招名为成竹在胸,乃是七转层第木道杀招,以成竹仙蛊为核心,但却是智道效果。

这杀招催动起来,便能在方源的胸膛处向外透射一团碧芒。

方源持续消耗仙元,推算某个事情。一旦真正推算成功,想彻底了,方源的胸口处就会形成一丛竹林的墨绿纹身。

成竹在胸杀招的优劣,方源已然洞悉。

此招最擅长的,是对一件事情进行通盘考虑,从头到尾的谋划。这是它的优点。

推算成功的标志,就是胸口的竹林纹身。

但若是谋划并未结束,就因为仙元缺乏或者外力干扰等因素停止了杀招,成竹在胸杀招就算是催动失败。这就是缺点了。

因为杀招催动失败,蛊仙就会遭受反噬。

方源也故意以身试法过。

胸膛处透射出的碧芒,以及画出一半的竹林纹身都骤然消失,方源的魂魄、肉身都在刹那间遭受创伤。

脑海倒是无恙。

毕竟这不是纯正的智道杀招,而是木道。

所以,用成竹在胸杀招推算某个对象,这个对象的范围不能太大太广,否则杀招催动失败,受累的是蛊仙自己。

除了成竹在胸杀招之外,方源还掌握了一记杀招,和它差不多情况。

那便是运道杀招——运筹帷幄。

这个杀招以六转仙蛊运筹为核心,属于运道流派,但却是智道效果。

运筹仙蛊的威能,是针对蛊仙自身资源如何有效利用的推算。

运筹帷幄杀招则是专门策划作战的计划。

蛊仙通常是专修一道,很少兼修第二流派。所以就有不少此类的杀招,能模仿其他流派的效用。

不过这些杀招,都可谓半路出家,虽然威力也不俗了,但都各有侧重和针对,弊端也较为明显。

和真正的智道相比,实用性就差了不少。

记住手机版网址:m.

\end{this_body}

