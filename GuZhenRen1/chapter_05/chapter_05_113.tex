\newsection{对战落星犬}    %第一百一十三节:对战落星犬

\begin{this_body}



%1
交接完成之后,黑凡洞天正式易主。

%2
尽管楚度极力挽留,但方源还是执意告辞。

%3
停留在黑凡洞天的时间,着实有些久了,方源该为第四次地灾做准备了。

%4
当然,在临走之前,他动用了仙道杀招百年好合,与楚度再定了一次盟约。

%5
方源匆匆离开,留下楚度,他望着荒凉至极的黑凡洞天,头疼不已。

%6
一路返回,直至琅琊福地。

%7
“哦,你们看,是方源回来了。”一位毛民蛊仙在琅琊福地中,看到天空中方源疾飞的身影。

%8
“他回不回来,有什么关系?”

%9
“是啊,咱们琅琊派有他没他,根本毫无两样。”

%10
“此人到底是人族蛊仙,不是我们毛民自己人啊。”

%11
“别管他了。咱们还是再计划计划,该如何将那头落星犬围剿!”

%12
很快,这几位毛民蛊仙,就都低下头,继续探讨战术。

%13
方源俯瞰下方,目光在这些毛民蛊仙的身上一扫即过。尽管相距甚远,但方源的侦查手段自然是将这些毛民蛊仙的谈话,一字不漏地听入耳中。

%14
一时间,方源的目光深幽起来。

%15
他在寻觅黑凡真传的行动中,耗费了不少时间,第四次地灾已近了。

%16
这一次地灾,方源打算自己独自来渡。

%17
虽然得到楚度帮助,更为安稳一些,但狂蛮真意势必就要被他分走。

%18
有好处,方源当然想独吞!一个人吞不下去的时候,他才会想其他办法。

%19
而且这一次,他得到黑凡真传,实力大涨,独自对付第四次地灾,他很有信心!

%20
“但要渡劫,我还得需要琅琊派的仙劫锻窍杀招。这就需要向琅琊地灵,借用仙蛊、凡蛊。”

%21
“以我和琅琊地灵现在的冷淡关系,恐怕是借不了的。除非是出让转卖荡魂山这样的巨大利益。”

%22
“如此看来。趁着还有一些时间,先完成那个剿灭落星犬的门派任务。一来,能缓和关系。二来,能获得不菲的门派贡献。可以支付借蛊的费用。”

%23
方源没有回到自己的云城,而是一路直飞,到达第一云城。

%24
在那里,他见到了琅琊地灵。

%25
“你还舍得回来?”琅琊地灵对方源毫不客气,态度冷淡得很。

%26
方源叹息一声。一脸苦涩地道:“太上大长老见谅,我亦有不得已的苦衷啊。之前并非我无动于衷,而是没有精力和时间。这一次外出,终于是暂时缓解了自身的危机。刚回到这里,我听到派中的太上长老们和议如何围剿落星犬。我便立即赶来,就是想从太上大长老您手中,领走这个任务。”

%27
“哦?”琅琊地灵闻言,展开眉头,“这么说,你打算出手了?”

%28
方源再度苦笑起来:“其实在下一直打算出手。毕竟我乃是琅琊派中的一员。可是情形所逼,无可奈何。如今终有余力,便想为门派做一些事情。”

%29
他的演技再配合态度蛊,效果立竿见影,顿时就让琅琊地灵觉得,方源似乎说的都是真话,而且很有诚意和歉意。

%30
琅琊地灵的神情又缓和了几分,刚想答应,但话到嘴边又咽了下去,转而道:“你现在想接这个任务。恐怕有些麻烦。之前毛十二等人,已经合伙,商定要一起出手。上一次他们虽然围剿失败,但也收获很大。几乎是两败俱伤。这一次恐怕就要成功了。方源你现在要领这个任务,不说你是否能成功,首先就和他们产生冲突了。”

%31
方源只是个外人,而毛十二他们却是根正苗红的毛民蛊仙,是自己人。

%32
琅琊地灵话里话外,还是偏袒毛十二他们的。

%33
方源点点头:“原来如此。不如这样。让他们先接了这任务,若是他们此次完成不了,我再出手如何?”

%34
琅琊地灵思考了一下,点头道:“可以。”

%35
方源又道:“还请太上大长老将此事广而告之。”

%36
琅琊地灵自无不可。

%37
此事一经公布,顿时在琅琊派内刮起了一阵旋风。

%38
“那方源想要出手了?”

%39
“哼,一定是他看到我们就要成功了,所以才想出手的。”

%40
“卑鄙小人!”

%41
“卑鄙与否,暂且不说。我只知道,围剿落星犬的赏格已经破千!谁能完成,获益极大。”

%42
“十二啊,你们这次一定要成功啊,不要给方源可乘之机!”

%43
毛民蛊仙们颇有同仇敌忾之情。

%44
毛十二点点头,下意识地捏紧双拳,心想:“这一次定要成功,准备充分,不辜负大家的期待!”

%45
上一次的战绩,给了这些毛民巨大的信心。

%46
毛十二原本准备两天后,就再度出手。但经过这一事之后,他决定更稳妥一点,准备得在充分一点。

%47
七天之后,方源接到情报,毛十二他们要动手了。

%48
他不请自来,和毛十二等人汇合一处。

%49
“方源长老。”毛六首当其冲,阴阳怪气地招呼道。

%50
他真实身份乃是影宗内奸,但就算方源当场揭穿他,旁人也不会相信。况且他还和方源做过交易,手头上有这个把柄,导致他有恃无恐。

%51
上一次他受命,代表影无邪,和方源交易。

%52
方源占据上风,让他吃了不少的瘪。

%53
方源没有及时出手,积极剿除落星犬时,毛六上蹿下跳,蛊弄人心,操纵舆论,让方源和琅琊派的关系一落千丈。

%54
“方源长老,我知道你的来意,但这一次,你恐怕是要失望了。”毛十二跨前一步,昂首挺胸,自信满满地开口道。

%55
他的语气就很正常,一片光明,不像毛六暗藏阴私。

%56
方源看向毛十二,立即发现这位毛民蛊仙的身上,有了明显的改变。变得比以前更自信,更稳重了。

%57
方源心底暗自评估:“毛十二主修炼道,辅修奴道。这一次对付落星犬,一直都是主力。曾经还受过重创。卧床休息了一段时间。这些磨砺都使得他进步巨大,精神气质都发生了巨大提升。”

%58
心中这样想着,方源表面上则微笑着,回应过去:“那我就在一旁静候。亲眼见证十二兄你的成功了。”

%59
“哼,假仁假义。”

%60
“表面上云淡放弃,其实心里着急得不得了吧?呵呵。”

%61
其余毛民蛊仙看不惯方源和气的样子,嘴里嘀嘀咕咕。

%62
方源听在耳中,却仿若充耳未闻。

%63
一群人站到母阵中央。

%64
毛民们抱成团。方源一个人独自站在外侧。

%65
“注意,我要催动蛊阵了。”毛十二淡淡地提醒一句。他瞄了方源一眼,心想:多了这么一人,恐怕传送时,仙元的消耗要多一些了。

%66
毛民蛊仙们神情自若,这个传送蛊阵他们已经用过多次,驾轻就熟。

%67
蛊阵渐渐催动起来,光影逸散,绚烂多姿。

%68
方源静静旁观。

%69
太丘那处的传送子阵,就是他亲手布置的。他回到琅琊福地时。就是利用的传送蛊阵。

%70
他独自用的时候,相当轻松。现在看这些毛民蛊仙们催动,却比他艰难得多。

%71
毛十二持续耗费仙元,过了好一会儿,蛊阵终于催动起来。

%72
耀眼的彩霞,几乎要遮蔽所有人的双眼。一股巨大的玄妙之力,猛地爆发!

%73
恍惚一下,群仙再看时,已是到了太丘。

%74
方源转身,果然背后就是那座庞大如山的巨象尸骸。

%75
“仙元的消耗。似乎没有多出多少啊。看来这个方源,道痕底蕴也不是很深厚。恐怕也只是战斗经验比我们多一些吧。”毛十二再次偷瞄方源一眼,心中多了许多不屑。

%76
“这一次只可胜,不可败!出发!”毛十二一挥手。大声呼喝。

%77
毛民蛊仙们纷纷低呼应和,士气很旺。

%78
他们开始启程,方源尾随在后,为了不惹矛盾,方源故意落在后面,并不靠近这些毛民蛊仙。

%79
没有前进多远。方源就发现了目标。

%80
“一头落星犬,似乎还未成年?”方源微微诧异了一下,随后就看到毛民蛊仙们一拥而上。

%81
“大狗,这一次我们一定会把你打趴下!”毛十二昂首高呼,胸中战意沸腾。

%82
落星犬发出一声低吼,猛地低伏下来,小山般的身躯忽然爆发出令人惊讶的速度,展开冲锋。

%83
毛民蛊仙们顿时被冲的七零八落。

%84
毛十二狼狈不堪,气极大吼:“好畜生,居然敢偷袭!来,让我和你大战三百回合。”

%85
说着,毛十二双掌一拍,一股湛蓝的光辉从他双手间绽放,旋即在半空中描绘出一片蓝光湖泊。

%86
蓝湖悬空,湖水喷涌,从湖底一下子探出三头荒兽。

%87
一头牛兽,一头虎兽,还有一头熊兽。

%88
牛还是那头板栗牦牛,虎是金白虎,熊是钻熊。

%89
三头荒兽迎上落星犬,在巨人草丛中展开了激斗。

%90
这里是太丘的深处,巨人草极其高耸,即便是落星犬、钻熊之流,也不及巨人草的高度。

%91
毛十二控制三头荒兽,正面牵制住落星犬。其余的毛民蛊仙,则躲避到巨人草丛中,见缝插针地遥攻落星犬。

%92
方源在更远处旁观。

%93
他可以明显看出,这处战场中有着多次激斗的痕迹。

%94
毛民蛊仙们的表现,有点让他刮目相看。战斗有板有眼,比之前有了长足进步。虽然仍旧不够看,但进步是不能抹杀的。

%95
落星犬的确没有成年,虽然和这群毛民蛊仙打得不分上下。

%96
方源笑了笑,仰头望天。

%97
他在心中呢喃:“我已来了,天意,这次你准备如何对付我?”

%98
ps:今晚第二更!

\end{this_body}

