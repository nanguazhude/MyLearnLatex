\newsection{算计青仇}    %第五百一十二节:算计青仇

\begin{this_body}

%1
“原来是这样。”方源当即就和房家蛊仙进行联系,得到了情报。

%2
这场围绕着豆神宫的大战,确实已经结束了。夺得豆神宫的,正是房家!

%3
天莲派的太上大长老陈衣败走前,留下话,要房家好好保管豆神宫,将来他必定亲自再来取之。

%4
至于青仇也逃出生天,它底蕴深厚,并非只有一只八转魂兽令,还有其他八转仙蛊。

%5
房家虽然夺取了豆神宫,但是三座仙蛊屋中,鸡笼犬舍、问津坞都有严重的损毁,太上大长老房功也受了不小的伤。而夺取到手的豆神宫,更是近乎半毁。

%6
不过,在方源联系之后,房家仍旧许诺,要给予方源报酬。并且严格地按照盟约,报酬很高。只是有一点,房家蛊仙想请方源缓一缓,因为这次闹得动静太大,瞒不了多久,房家获得豆神宫的消息,就会被推算出来。到时候,西漠蛊仙界必定动荡,许多超级势力不会坐视房家获得此屋,进而一跃成为西漠超级势力之首!

%7
到那个时候,房家必定面临着四面八方的压力,所以房家要趁着这段平静的时间,尽可能地提升自己,进行休整。尤其是仙蛊屋,必须得尽全力修复。

%8
所以,房家手头会很紧,要求方源方便的话,不妨宽限几天。

%9
方源想了想,答应下来。

%10
首先,盟约中对这报酬的支付时间,要求并不严格。其次,方源还是想和房家处理好关系。

%11
眼下的局面不同了。

%12
房家招惹了中洲天莲派,也等若招惹了天庭,方源伪装的算不尽在其中出力甚多,逃不出干系,在房家看来,方源和自己这一方有着共同的强敌,是天然的盟友。

%13
另一方面,房家面临压力,又忌惮方源的展露出来的实力。原本算不尽这个身份,就是智道蛊仙,又有奴役魂兽大军的手段,方源在豆神宫一战中,又暴露出了盗取八转仙蛊魂兽令的能力,简直是惊世骇俗!

%14
所以房家对方源的态度相当谨慎,不愿轻易招惹这等存在,尤其是在当下这个关键时刻,整个房家面临考验,房家更不愿意恶了方源这一头,甚至还打算在某些时刻,依赖方源,想将他充作外援。

%15
方源念头转了转,便明白了房家的打算。

%16
“先是唐家,看来这个房家也能成为抵抗天庭的棋子。”方源布局天下,眼光早已经灼照将来。

%17
“西漠的局势必将动荡起来,关键还是房家的实力究竟还剩下多少,夺得的豆神宫能发挥出多少威能来?”

%18
若是房家能抵挡住接下来一段时间里,来自各个方面的打压,那么方源不介意锦上添花,出手支援一下,巩固双方的关系。

%19
若是房家不能抵挡,甚至陷入灭族的危机,方源当然不会雪中送炭,直接落井下石,抢夺豆神宫还有巨大希望。

%20
方源坐山观虎斗,心态悠然。

%21
这次豆神宫大战,他收获极大。一方面是七转大盗仙蛊,有了它,再配合方源的偷道大宗师境界,就能利用九转杀招鬼不觉的威能,这是一个质变。另一方面是八转仙蛊魂兽令,八转这种层级,就算是有滔天资本,也不一定能够获得。第三方面是和房家的关系,得到突飞猛进的发展,并且房家还欠方源许多酬劳。

%22
就算方源得到豆神宫,他还面临着修补的难题,需要投入海量资本。同时又令仙蛊喂养的负担暴涨。

%23
现在这种结果,对于方源而言,也未必是坏事。

%24
“只是目前为止,青鬼沙漠中涌来大量西漠蛊仙,我倒不好在此着手布局了。”

%25
本来,方源打算将影无邪等人在青鬼沙漠落子,不断猎取魂核,帮助方源魂修。但现在看来是不行了。

%26
青鬼沙漠中人多眼杂,不利于行动。

%27
并且还有一个最大的威胁,那就是逃出生天的青仇。

%28
这头传奇太古魂兽,拥有着八转战力,方源没有和它正面对战过,也知道它战力雄浑,即便是自己催使逆流护身印等等,也顶多打个平手。

%29
现在,青仇行踪不明,很有可能仍旧藏身在青鬼沙漠当中,这样一来,不管是方源派遣影无邪等人驻扎,还是自己留下,都有危险。一旦双方打起来,单单暴露身份的问题,就会坏了方源很多大事。

%30
“还是走吧。”方源智道手段多种多用,短短片刻,他就做出最明智的决定,要离开青鬼沙漠,甚至是直接离开西漠这块地方。

%31
“仙友?”此时,何辜女仙还站在方源的面前,见他始终不回话,又问一句。

%32
方源嘿然一笑,目光转到她的身上,暗想此女也是精明人物,居然单从我的疾飞中看出许多端倪来。

%33
方源目光闪烁了一下,便道:“我的确有一些情报,十分珍贵,你若想要求购,也得付得起这份价钱。”

%34
何辜笑了笑:“仙友勿虑,我虽是散修,但也有些资本。只要仙友保证这份情报属实,我的报酬必定能够让仙友满意。”

%35
方源便和何辜缔结了一个简单的盟约,然后选择性地将青仇的存在说出来,也的确都是真话。

%36
何辜顿时眼冒精芒:“哦?一头魂兽身上,居然和之前的青家有关系?疑似青家秘藏也在它身上?”

%37
“此中的详情,我就不太清楚了。”方源摇头,不愿意再多说什么。

%38
何辜却已经相当满意的样子,当场支付了方源许多西漠仙材。

%39
依照方源的眼界,都有些惊异。这些仙材当中,有一种古神木,是相当罕见的六转仙材。虽然只是六转,但就算是宝黄天中也不多见,因此有时候能卖出七转仙材的价格来。

%40
这笔收获算是一个小小的惊喜。方源既赚了一笔,又把青仇暴露出来。

%41
这样做,一来为房家打掩护,短时间内将西漠蛊仙的视线,都集中在了青仇身上。二来,对方源自己也有好处。青仇不死,行踪不明,对方源的青鬼沙漠计划是个严重的威胁。方源正好利用这些西漠蛊仙,帮助自己探路,查明青仇所在。至于他们碰到青仇,发现这竟然是一头传奇太古魂兽的事情,那就只能自求多福了。

%42
“我的这份情报,都是真的,必定会有智道蛊仙推算,验证是真。到时候,事关青家遗藏,就算是八转存在都会很感兴趣。引动一两位八转蛊仙出动,那就是青仇苦恼的时候了。”

%43
方源身为智道宗师,借助这次小小的机会,狠狠地算计了青仇一把。

%44
那何辜女仙却还不满足,提出请求,想要和方源一道探索青鬼沙漠。

%45
散仙结伴,也是常态。

%46
不过,当然被方源拒绝了。

%47
方源辞别何辜,直接远去。

%48
那何辜还想追踪,远远地跟在方源后面飞,但被方源轻松一绕,又借住见面曾相识伪装身份,很快就把此女甩掉。

%49
方源没有深入青鬼沙漠,而是擦着青鬼沙漠的边,直接远去。

%50
一路疾飞,几天几夜不休,方源便来到西漠界壁。

%51
直接洞穿两层界壁后,方源再次来到了南疆!

%52
灼热刺目的阳光,被天空的阴云所取代。

%53
西漠干燥的风,在南疆这里,却是浸透着湿润之气。

%54
一眼望去,群山葱茏,延绵起伏,蔓延都是黛色。

%55
“南疆……”方源眼中目光微微一闪,又旋即清明。五域当中,他对南疆的感受无疑最为深刻。

%56
仙道杀招——气运交感!

%57
旋即,他催动杀招,目光中微微泛出冷意。

%58
他之前就和上极天鹰建立了连运的关系,现在动用气运交感杀招,立即就察觉到了它模糊的位置。

%59
方源这一次回到南疆,主要目的就是重新收服上极天鹰。

%60
这头宇道太古荒兽,方源还有大用。有了它,方源就能将市井中的所有福地,都统统吞并,令自家仙窍底蕴大涨,同时身上道痕暴增。

%61
“是在东南方啊……”方源暂停气运交感,身形如电,直插云霄而去。

\end{this_body}

