\newsection{天庭的底蕴}    %第七百二十一节:天庭的底蕴

\begin{this_body}

%1
刘流溜其实早就登场参战,但和他同期的北原强者早已经被击毁,只剩下他一人。

%2
这样的战斗太过惨烈,八转蛊仙的牺牲变得稀疏平常。

%3
刘流溜的手中战果辉煌,交战以来至少有八位天庭成员遭了他的毒手。他早就关注着紫薇仙子,他丰富的战斗经验令他迅速判断出紫薇仙子的重要性!

%4
此刻,他忽然出手偷袭,其实早已经酝酿已久。

%5
刹那间,紫薇仙子汗毛顿竖,致命的危机感觉充盈她的心胸,几乎让她不能呼吸!

%6
紫薇仙子心中雪亮,单凭现在自己的防御手段,根本阻止不了对方。但要现在催动其他的防御措施,根本来不及!

%7
眼看紫薇仙子就要陨落,关键时刻,从旁边杀出一位天庭蛊仙。

%8
正是之前支援龙公的吴双!

%9
噗嗤!

%10
一声闷响,刘流溜的杀招仿佛一柄灰色的利刃,刺穿吴双之后,又刺中紫薇仙子的肩膀。

%11
“该死,被吴用挡下了大半威能……”刘流溜面色骤变,身影迅速消散。

%12
紫薇仙子身受重伤,吴用却已然阵亡。

%13
刘流溜消失无踪,下一刻他再出现,说不定又会带走一条性命。

%14
轰!

%15
气浪翻飞。

%16
龙公向后飞退数百步距离,艰难地稳住身形。

%17
他终究是挡下了七极荒都的一记强大杀招,但代价是整个双臂,还有胸口前的龙鳞。

%18
他的双臂无力垂下,龙鳞尽数破碎,胸膛处血肉模糊。

%19
龙公立即催动治疗手段,但双臂上刻印下来的暗道道痕十分浓郁,并且排列组合巧妙,大大妨碍了他的治疗效果。

%20
“哈哈哈,龙公,你连续中了我们的蛇里引、暗缠命两记杀招,还想在短时间内恢复自己的伤势,做梦!”七极荒都发出嘲笑的声音。

%21
随后,劫运坛一改之前的战斗风格,忽然钻破地砖,从下而上偷袭龙公。

%22
一记仙道杀招爆发,击中龙公。

%23
一道巨型的橙金光球,将龙公关押在内,牢牢困住。

%24
光球不断收缩,挤压龙公的生存空间,将他的骨骼压得咯吱作响,龙鳞不断掉落,龙角上显现出明显的裂痕。

%25
见龙公被困,冰塞川大松了一口气。

%26
龙公速度惊人,而劫运坛的这记杀招却是速度较为缓慢,很难命中。

%27
幸好北原蛊仙战术极其明智,攻敌必救,让龙公心有牵挂,只能被动防守。七极荒都的配合越来越有默契,这才创造出了战机,射中龙公!

%28
咔嚓。

%29
一声轻响,一大截的龙角从龙公的头顶坠落下去。

%30
大量的紫色断发,飘零洒落。

%31
九条龙影浮现在龙公的全身,正是九纹龙护身杀招,艰难地抵抗着周围的重压。

%32
“龙公,你大势已去,宿命蛊必是我长生天之物!而你也将收获死亡!”冰塞川的声音传遍战场,短时间内他只能困住龙公,并不能取走他的性命,但没有关系,他可以凭此事来打击天庭一方的士气。

%33
果然,不少天庭成员见到龙公被困,皆露出惊色。

%34
龙公忽然放声大笑:“就算是老夫战死在这里,又如何?你们北原蛊仙悍不畏死,难道我天庭就有怕死的人吗?”

%35
一句话,说得天庭诸仙纷纷叫好,士气大振。

%36
“我们沉眠,只是为了责任,绝非怕死。”

%37
“开玩笑,怕死?我们可是天庭的一员!”

%38
“我们维护人族的荣光,从远古时代至今,我们从不畏惧牺牲!!”

%39
天庭的蛊仙开始反冲锋,不惜用血肉之躯抵挡北原蛊仙的杀招。一旦有机会,天庭蛊仙就不顾自身,发动猛攻,争取和北原强者同归于尽。

%40
北原蛊仙勇悍,天庭蛊仙却开始疯狂!

%41
“不好。”冰塞川暗叫不妙,他原本想打击天庭的士气,没想到龙公一句话竟巧妙瓦解。

%42
冰塞川冷笑,声音再次传遍战场:“可笑!你们的牺牲根本没有意义,就算我方的蛊仙牺牲再多,他们也不过是红莲魔尊的杀招所化,早已经死了。用已经死的人,来换取你们的牺牲,这可是百赚不赔的生意啊!你们天庭能有多少的底蕴,经得起如此的折腾?”

%43
“底蕴?”龙公也冷笑,“我们天庭有的是!”

%44
他开始反抗。

%45
他的双臂无法动弹,但是周身气息升腾起来,像是一睹高墙,将他包裹在内。

%46
龙公看向劫运坛,目光中流露出嘲讽的神色:“你们长生天的建立,只是在中古时代,三十万年多一点而已。而我们天庭呢?距今已有三百八十七万九千六百八十七年了!你算算看,是你们的多少倍!别想揣度我天庭的底蕴,哼,你们永远都是鼠目寸光!”

%47
“冰塞川,你不妨回头看看,看看我方的仙墓,再看看你那边可笑的光阴长河的虚影。”

%48
冰塞川的脸色沉下,哑然无语。

%49
他不需要去看,他一直都对这两个地方保持着关注。

%50
光阴长河中,走出来的北原蛊仙已经开始稀稀疏疏。但是天庭的仙墓中,却仍旧有着一批批的蛊仙,接连不断地苏醒。

%51
天庭的底蕴,深不可测!

%52
龙公继续道:“一个家族满打满算,有多少人?家族选拔出来的人才,会有多少?冰塞川,你再看看我们中洲。我们从中洲所有的凡人中,选拔出各种各样的人才!”

%53
“一个家族就算发现了人才,能真正栽培好吗?家族的血缘,各自的亲情关系往往成了阻碍。但我们门派中任人唯亲的现象,远比家族稀少得多。并且门派的竞争,更加公平,更加透明。”

%54
“就算一个家族的高层后代中出现了天才,家族高层的蛊仙又能花费多少心力,去培养这个天才后辈呢?蛊仙也有许多事情,也非常的忙碌。而在门派当中,一位天才后辈会得到许多人的指点。从他开始修行,到他成就蛊仙,都会有相应的指导。门派的许多教导任务,能够让指导后辈的人,获得相应的收益!试问一个家族这样做的,能有多少?”

%55
说到这里,龙公声调高昂激越,充满了骄傲和自豪:“北原诸仙啊,你们敢来入侵天庭,勇气可嘉!可惜你们的见识,早就被你们的家族束缚了,你们太低估我天庭的底蕴了!”

%56
冰塞川默然不语。

%57
他想反驳,但说不出反驳的理由。

%58
因为,事实就在眼前!

%59
天庭的仙墓中,各个成员苏醒,层出不穷。

%60
三百万年前,远古时代。

%61
元始仙尊看着周围的人族蛊仙们,微微点头:“现在,中洲人族的蛊仙都已经到齐了,并且还有几位来自其他几域的仙友。”

%62
“元始仙尊大人,听到您的召唤,我们就都赶来了。您有什么指示或者教诲,我们都将聆听于心。”

%63
元始仙尊微笑着,说出自己的建议:“从今日起,我们中洲蛊仙都要组建门派,广泛收徒,将各自的真正本领传授下去。”

%64
“什么?”

%65
“取缔家族制度吗?”

%66
“虽然早就听闻了这个风声,但元始仙尊大人,您真的要这么做吗?”

%67
“这未必也太……”

%68
人族蛊仙们议论纷纷,脸上流露出难色。

%69
有人小心翼翼地劝道:“元始仙尊大人,您可是当今世间唯一的九转蛊仙,您要是创建自己的家族,必定也是天下第一的超级势力啊!您的子孙将继承您的荣耀和本领,我等的家族必将以您的家族为领袖。”

%70
元始仙尊微微摇头,他看向劝导他的蛊仙,目光平静:“家族?试问天下,从古至今,可有长久不衰的家族吗?一旦家族坐大,血缘这份纽带,就会显得稀疏。在我未成仙尊之时,我们和其他的异人各大家族对抗,凭借的除了我等的团结,就是依靠他们内部的罅隙。我想要创造的门派,将是一个不依赖血缘关系的组织!”

%71
人族蛊仙们沉默一阵,有人勉强笑道:“仙尊大人呐,你说的没错,家族一旦坐大,就容易松散。但您是不一样的,您可是古往今来第一位修行到九转的蛊仙呐!说不定以后也未必会再出现九转仙尊了。”

%72
“只要有您在的家族,哪一个家族的成员会敢游离向外呢?有您在,就有天底下最团结的家族!”

%73
元始仙尊点头:“你说的不错。但是当我不在了呢?我又能活多少岁?”

%74
人族蛊仙被问得一愣。

%75
有人道:“仙尊大人,您现在可是春秋正盛呢!”

%76
又有人道:“我们会不断地搜集寿蛊,您必定是历史上活得最久的蛊仙!”

%77
元始仙尊微微一笑,以平静的语气道:“可是我仍旧要死的。世间谁能不死?”

%78
他摇摇头,叹息一声,自问自答:“没有人。”

%79
蛊仙沉默。

%80
元始仙尊继续道:“其他的延寿之法都有弊端,除了寿蛊。但就算是收集寿蛊,我们也都明白,寿蛊越来越少了,不是吗?”

%81
人族蛊仙一片沉寂。

%82
“这是天意。”元始仙尊仰头,目光似乎穿透洞壁,看到苍穹和云霄,“天道损有余补不足,从未有永生的至强!木秀于林风必摧之,堆高于岸流必端之。异人们的强者,不断地新旧交替,我们人族同样如此。”

%83
“我虽然无敌天下,但也只是暂时,总有一天我会死去。这是天道、宿命、规矩,《人祖传》中早有叙述。”

\end{this_body}

