\newsection{血战落幕}    %第二百一十五节:血战落幕

\begin{this_body}

龙公叹息一声:“是否有十转蛊仙,我亦回答不了你。不过十转仙蛊,我想应该是有的。因为这曾是乐土仙尊的亲口断言。”

乐土仙尊乃是继幽魂魔尊之后的九转尊者。他生性仁厚,与人不争,爱好和平,但和巨阳仙尊一样,也没有真正入主天庭。

真正入主天庭,并且将自身的仙窍洞天融入天庭的,历史上只有三位仙尊。

从古至今,分别是历史上第一位九转尊者元始仙尊,开创智道的星宿仙尊,以及开创木道,拥有天元宝皇莲仙蛊的元莲仙尊。

北原。

血战武斗大会终于迎来的最后一战。

楚度走下战场:“还请指教。”

在他的面前,站着一位女仙。她身着锦绣宫装,裙摆拖地,一头青丝高高挽起,玉吊金簪琳琅满目,肌肤雪白,眉毛修长,目光精锐,胸脯高耸,体材丰满,雍容华贵,给人不可冒犯之感。

正是宫婉婷。

凤仙太子的正妻。

宫婉婷满脸都是凝重之色,轻叹道:“霸仙果然是名不虚传,连败刘转身、药元婴两位仙友。此战不论结果如何,霸仙的威名,必将广传北原。”

被提及姓名的刘转身、药元婴两人,都是面色不虞。但没办法,楚度的实力就是比他们更强。

总体而言,血战武斗大会乃是黄金部族们,牢牢地压过楚门和百足家的联合。

但是临到尾声,楚度的登场,给了正道联合势力们一个响亮的耳光。

霸仙之威,让黄金血脉的蛊仙们心生忌惮。

百场大战,已经是最后一战,迫不得已。宫婉婷走下场。

这是双方主帅的对决!

“请。”楚度风度翩翩,身上虽然有着伤势,却仍旧从容不迫。

宫婉婷妙目中闪过一抹异色。她笑了一声:“霸仙果然豪气干云。”

话音刚落,一道紫光如刀似剑。向楚度电射而来。

楚度断喝一声,不退反进,勇悍绝伦地冲向宫婉婷。

一场好战,杀得惊天动地。

双方你来我往,斗了数百回合,不分胜负。

从天上打到地上,血战平原几乎全毁。

观战诸仙不得不后退三万里,才能免受楚度和宫婉婷激斗的殃及。

楚度威不可挡。攻多守少,宫婉婷则是攻少守多,姿态从容。

如此激战,又持续了上百回合,楚度浑身是伤,鲜血飙飞,宫婉婷亦是不再从容,凝重的目光不断扫射,谨防楚度突击,不敢有丝毫大意。

观战蛊仙早已动容。纷纷评价,这场龙争虎斗,已是七转巅峰对决。八转不出的前提下。北原蛊仙界恐怕数百年,都无后人超越。

日落月升,从白天达到夜晚,双方仍旧不分胜负。

这种不间断,全程都在交手状态下的激斗,烈度非常巨大。

到了第二天黎明时分,双方的状态都大幅度下滑,战斗步入尾声,生死胜负即可能就在下一刻揭晓。

观战的蛊仙们。无不屏气凝神,注视着最后的结局。

但就在这时。一道恢弘的光柱,忽然罩下。分开激战中的楚度和宫婉婷。

“差不多了,此战就以战平论断罢。”八转蛊仙药皇,在高空中显现出身形。

他望向不远处:“不知天君,是否赞同老夫此言?”

百足天君从云层中献身,点点头,道:“请上天一战。”

随后,两位八转蛊仙在众目睽睽之下,双双飞升,没入到白天之中。

很快,血战平原上方的众仙就听到白天中,传出来轰鸣和爆响,宛若是雷霆,又仿佛是兽吼。

霸仙楚度吐出一口浊气,眼下的局势正合他心意。

宫婉婷身份特殊,乃是凤仙太子的正妻,楚度念着此点,根本无法下杀手。一旦杀了宫婉婷,绝对会惹来凤仙太子。

楚度孤家寡人的时候,未必害怕。但现在建立了楚门,却是不同的概念了。

而至于楚度本人,此次血战武斗大会,表现卓绝,百足天君为百足家的后续发展考量,也不会让楚度于这场武斗中陨落。

就像药皇和百足天君在武斗大会之前,就商量好的那样,双方的战斗损伤,被成功地局限在了一定的范围内。

“嘶……好厉害,连白天都被打出了裂痕了。”

“听这样的声音,就知道战况激烈无比。”

“可惜我们无力洞穿天罡气墙,前往白天观战啊。”

双方众仙都未离去,纷纷交谈。

百足天君和药皇的战斗结果,决定了一切。

整个血战武斗大会的过程中,不管牺牲多少,付出多少,有多少显赫的战绩,都比不上这一战的结果。

不管是哪一方,都是损伤惨重。

不过,真正战死的蛊仙,深究起来并不多。六转蛊仙死了一些,七转蛊仙阵亡的更少,最惨重的损失,便是方源斩杀的耶律群星。

血战越到后期,双方反而形成了一种默契。

当然,蛊仙受伤也是一件大事。

因为道痕的缘故,难以治愈。往往疗伤的费用,会让一位蛊仙付出不菲代价。

白天之中。

百足天君和药皇相对而坐,两人之间摆放着一张棋盘。

“请品尝一下,我最新创造出来的金叶茶。”药皇笑呵呵,推荐他的茶水。

百足天君喝了一口,连连点头,然后他取出一大袋子的零嘴。

“这是我最拿手的炒蜈蚣。”

两位八转大能,一边喝茶,一边吃着炒蜈蚣,一边对弈下棋,好不惬意。

药皇感慨:“这炒蜈蚣味道鲜美,百尝不厌,实乃人间绝品。”

百足天君也赞道:“药皇兄的金叶茶,却是更胜银叶茶一筹。看来距离药皇兄炼成起死回生蛊的日子,也不远了。”

每一份茶、酒,都可以看做是一道未完成的食道蛊方。

有时候,很多蛊仙切磋各自的炼道造诣,为了避免劳师动众,或者照顾各自的颜面,都会上茶或者献酒。

蛊仙相互之间品茗,或者畅饮对方的美酒,即可明确对方的炼蛊造诣。

当然,此法过于含蓄,很多时候只是一种粗浅试探,并不能说明彼此间的真正实力。

药皇新创的金叶茶,百足天君从中品味出了不一样的味道。而且百足天君曾经尝过药皇的银叶茶,两相对比,就可以品味出药皇炼道造诣上的进步。

这种进步,很明显是药皇企图炼制起死回生仙蛊,而在历练中得到的提升。

至于百足天君,他制造出来的零嘴,不是那么简单的,也可以看做是本身炼道造诣的一种展示。

但是百足天君最近这段时间,忙于建设百足家族,而后又攻伐黑凡洞天,炼道方面却是没有增长,所以炒蜈蚣仍旧没有什么变化。

但药皇的目光,却更多投射在两人的下方。

在他们的下方不远处,有两道身影正在交战。

一道身影是百足天君的模样,另外一道则是药皇形象。

白天之下,血战平原上空的那些蛊仙,听到的响动,也是这两道身影造成的。

药皇赞叹道:“天君的分身手段真是越发高明了,居然连八转层次都能模拟出两三成的威能来。”

百足天君笑着摇头:“惭愧。这是日前,我攻打黑凡洞天不下,受挫而悟出来的仙道杀招。我曾经听闻,长生天中收藏了盗天魔尊的仙道杀招成双入对,可令蛊仙分化出一个完全和自己战力相同的分身出来。我的这些手段,还是和盗天魔尊不能相提并论的。”

药皇笑了笑。

他闻弦歌而知雅意,百足天君提及长生天,并不是表面上那么单纯。

药皇解释道:“天君勿忧。只要长生天中八转蛊仙不出,单凭一道长生令,却是调动不了我等的。”

“况且任何一域的局势,还不都是看八转蛊仙之间的对峙情形么?”

“如今北原,八转蛊仙之中,雪胡老祖当是战力第一。若是让他炼成了鸿运齐天仙蛊,更是无法可制。但我日前观望,发现大雪山福地守卫森严,即便我等,想要破之,也算棘手。”

“我等之争,不过是正道内斗。若是雪胡老祖炼成鸿运齐天蛊,北原局势便会形成正消魔涨的情状。”

药皇侃侃而谈,百足天君微微点头,认可前者对局势的分析。

本来八十八角真阳楼倒塌,就导致了北原格局大变,魔焰高涨。若是雪胡老祖真的炼成了鸿运齐天仙蛊,那就更不得了了。

比起正道势力的崛起,药皇更不愿意看到的是魔道嚣张。

“好在这一次血道武斗大会,我方邀约了不少散仙、魔道,其中一部分阵亡在大会之中,另一部分战后,便会加入楚门或我百足家,也算是放了血,助长了我正道实力。”百足天君笑着道。

八转蛊仙思谋深远。

一场血战武斗大会,一方面是帮助药家,削弱了其他黄金家族的势力,另一方面则是借助百足天君、楚度的身份,削弱了散修和魔仙方面的大量实力,稳定北原蛊仙界的大局!

不过提到楚门,药皇皱起了眉头。

“这个楚门,需要整改。”

“我们北原,绝不容许门派产生。”

“楚度若是要楚门长存,必须将楚门改为楚家,否则的话,我正道绝不容许!”(未完待续。)

\end{this_body}

