\newsection{围杀方源}    %第六百零四节:围杀方源

\begin{this_body}

“五界山脉?”君神光顿时心中一动。

方源此次大肆破坏烽火台,搞出来的动静太大,天庭就算在南疆的势力十分薄弱,也查探得出来。

南疆这边集结了众多智道蛊仙,推算出了三个地点,其中包含五界山脉。

而天庭却是由紫薇仙子独自推算,提前就算出了五界山脉。

君神光立即动身,向五界山脉赶去。

与此同时,他的脑海中浮现出有关五界山脉的种种情报。

这五界山脉,并非是自然成型的名山,而是人为打造。制造出这等山脉的人物,当然不简单,乃是八转禁道蛊仙陶铸。

此人号称禁师,流派境界卓绝,他对五域界壁深有研究,企图找出方法,能供蛊仙轻松跨越界壁,故而造出五界山脉。

可惜,至他陨落,也未得到什么有效成果。但这五界山脉,却是遗留下来,成为南疆一处特殊所在。

“雷鬼真君前辈追捕方源,曾经在界壁中被方源击败,此战泄露出方源在界壁中战力几无影响的秘密。可以说,五域界壁对他而言是天然的主场。”

“遗憾的是,五域界壁中天地二气万难抽取,他要渡劫,在界壁中万万不成。不过五界山脉却是不同,此地乃是人为改造,位于南疆之中,地气充裕,天气也不薄弱,同时也能为方源提供一些主场的优势!”

君神光速度极快,脑海中迅速思量的时候,他已经是跨越了数万里。地平线上,五界山脉已经遥遥在望。

之所以能这么快就赶到事发的现场,那是因为君神光本来就在这附近逡巡。

五界山脉这块奇特之地,早就为人所知,并非秘密。

数月之前,紫薇仙子就在谋算方源,推算出一些适合他渡劫的地点。五界山脉这块地方,当然是名列榜单前列的。

所以,五界山脉本身就是天庭重点的监控地点之一。

这一次,天庭为了对付方源,紫薇仙子不惜抽调了多位八转蛊仙,分别派遣到其他四域。南疆中有着两位,君神光是其中之一,还有一位八转蛊仙卫风,距离较远,但也移速奇快,正在迅速赶来。

君神光接近五界山脉,立即催动杀招,隐形匿迹,小心翼翼地靠近。

哗啦啦!

海量的天地二气,宛若大江浪潮,源源不断地被抽取出来,汇聚到五界山脉的中央一点。

那里被光影遮挡,完全看不清楚,就好像是一个无底的洞,将庞大规模的天地二气不断吞吸。

“这样的规模,着实恐怖……”君神光看得心中凛然,旋即就动用独到的侦查手段。

方源设阵埋伏,成功俘虏了一干南疆蛊仙之后,君神光也是亲临现场的人。

这个时候,君神光心中满怀戒备,纵然是身为八转蛊仙,修为比方源高出一筹来,也绝不敢冒然前进。万一进入五界山脉之后,发现这里却是方源的一个埋伏圈,那就惨了。

君神光侦查之后,确信正是有人在此渡劫。

但究竟是不是方源,还是两说。

君神光耐心十足,又施展其他手段,片刻后得到结果,这正是一位天外之魔在渡劫。

如此一来,君神光几乎可以百分百确信,这里就是方源本体所在。

“他竟真的在这里渡劫!”君神光口中呢喃,目光中有些难以置信。

但很快他反应过来,自嘲地笑了笑:“实则虚之,虚则实之,方源本体明明在此,我居然还有点不敢相信了。”

北原的黑家,都拥有可以侦测出天外之魔的独到手段。这种类似的杀招,天庭当然也有。

紫薇仙子曾经在琅琊福地中,对方源劝降,就透露过:天庭是人族的天庭,就算是天外之魔,也能接纳。天庭也的确招揽过天外之魔,成为天庭的成员,只是历史上没有明确的记载,不为世人所知罢了。

天庭拥有这样的手段,并不奇怪。

此刻,君神光就是运用的这种手段。

“若说方源提前考虑到这种手段,进行防备和欺骗,那可能性就太小了。毕竟就算他重生归来,我这个手段也在最近,被紫薇大人特意改良了一番。”

“据说袁琼都加入天庭之后,除了主持修复宿命蛊的工作之外,也在为方源专门炼制仙蛊。”

历史上,出现过八转魔道蛊仙为祸世间的例子。

天庭一时间难以遏制的时候,就会为此魔头炼制仙蛊。这种仙蛊专门克制这样的魔头,效果非凡。

方源只是七转修为,就已经得到天庭这样的重视和待遇,在历史上也是前所未有的。

“或许还有一个原因,那就是他的浩劫之期,已经无法再拖延下去,必须渡劫了!”君神光眼中精芒闪烁不止。

有人在此处渡劫,又确信是天外之魔,方源的身份已经昭然若揭。

换做旁人,谁会脑抽到来到此处渡劫呢?

再加上天地二气如此规模,又是天外之魔,普天之下,除了方源,还会有谁?

“好,好,好。方源,终于让我天庭逮到你了!”君神光一脸说了三个好字,脸上浮现出越来越浓郁的兴奋之色。

方源已经开始渡劫,这个时候,根本不能随意转移,已然成为了一个活靶子。

就算是他掌握着定仙游,又能怎么用?

可以说,这是方源最虚弱的时刻。

君神光的脸上神情渐渐平复下来,变得满脸肃穆,一股肃杀之气萦绕全身。

“若是我此行斩掉方源,这可是泼天的大功啊!”君神光心头火热,这种良机他能够遇到,真的是运气!

“难怪紫薇大人曾经对我讲过,我身上运势浓厚,虽然不及方源,但只要不先出手,就不会被方源的运势压制。若做后面的黄雀,等到方源运势薄弱了,我便有建立战果的可能。”

想到这里,君神光眼中已经是杀气腾腾。

“五界山脉又如何?它只是五域界壁的仿制品而已,就算是方源在五域界壁中渡劫,我君神光也要强行突入,不惜任何代价将你斩杀!”

轰隆!

音爆声陡然炸响,一座仙蛊屋快若青电,直扑五界山脉而来。

君神光顿时心生凛然之情:“玉清滴风小竹楼?这南疆群仙来得好快!果然,烽火台对他们帮助太大。集结起来,竟如此迅速,就算是我天庭恐怕也没有这样的手段。将来若是进攻南疆,必定要先拔除了这些烽火台。”

君神光见到这些南疆正道,心中还有些欢喜。他正需要这些人物去替他打头阵,消磨掉方源的战力和运势。

玉清滴风小竹楼直接升上高空,武庸为首的南疆正道群仙,位于竹楼之中,俯视整个五界山脉。

见到如此规模的天地二气,许多蛊仙都面露惊疑之色。

一些智道蛊仙联手推算,很快当中一人兴奋地道:“十有八九,就是方源在这里渡劫!”

“好得很!”

“妙哉,妙哉。”

“方源魔头罪大恶极,穷凶极恶,今朝定要将他授首,为报我南疆正道的血仇啊。”

“杀得此魔,乃是为天地正本清源,为万物苍生造福!”

“哼,这方源真以为我南疆正道是软柿子容易捏?竟然真的南疆渡劫。”

“呵呵呵,他恐怕是觉得,俘虏了我等族人,会令我们投鼠忌器。殊不知,我南疆正道的卫道之心,是何等的坚定!”

南疆群仙议论纷纷,战意腾腾。

夏家太上二长老夏兆、三长老夏沉渊相互对视,均十分无奈。他们虽然极想拯救自家太上大长老夏槎,但在这样的氛围下,却不得不随大流。

南疆正道之前被方源勒索,早就憋了一肚子的火气。只是之前找不到方源,利益算计之下,为保人质,这才接受勒索。

现在方源就在眼前,只要杀了他,兴许就能得到多名尊者的传承,还有无数修行资源,天地秘境就有好几个!

滔天的仇恨,再加上巨大的利益,让这些南疆蛊仙都通红了双眼。

“我南疆正道联盟新立,正需要一个有分量的魔头祭旗,方源正合适。诸位,这该死的魔头正在渡劫,时间有限,我们一起出手,从四面八方围剿过去,让明年的今日成为他的忌日!”武庸一声令下,南疆正道群仙蜂拥而出。

有池曲由、商无界等八转蛊仙,又有夏兆等位高权重之辈,还有姚独、侯腾、巴德等七转强者,阵容之强大,就算是躲在一旁的君神光也暗自咂舌。这几乎可以说是南疆正道倾巢而出,留守在各自大本营的恐怕就只有一些六转的小仙了。

此刻,至尊仙窍当中。

天气滚滚,无边云光中,凝聚成数十根巨大光柱,狠狠地向方源镇压过来。

地气凝结,牵扯方源的脚步。

方源面露凝重之色,他的注意力不在灾劫之上,而是在仙窍之外。

“好家伙,来的人不少。南疆正道是恨极了我,当然我身上的财富,更令他们趋之若鹜。除此之外,必然还有天庭的蛊仙潜伏在侧,只是我暂时无法发现罢了。”

他虽然俘虏了那么多的南疆蛊仙,一一搜魂,对南疆正道了解甚深。但此刻也不敢有丝毫的大意和轻视。

因为南疆正道同样底蕴深厚,有着压箱底的东西。尤其是一些深沉的底蕴,哪怕是各族的太上大长老,都未必知晓,更何况方源手中的俘虏,不过就夏槎一位太上大长老呢。

“烽火台的确是麻烦。”方源又一声叹息。

“我虽然破坏了那么多,但南疆正道建设起来,也十分容易。五百年前世,南疆节节败退之后,才开始大肆建设烽火台。这一生似乎是因为我的关系,使得烽火台的建设大为提前。果然居高位者都绝不简单,不可小觑啊。”

\end{this_body}

