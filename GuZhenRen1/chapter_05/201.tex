\newsection{当代仙子赵怜云?}    %第二百零一节:当代仙子赵怜云?

\begin{this_body}

ps. 奉上今天的更新,顺便给『起点』515粉丝节拉一下票,每个人都有8张票,投票还送起点币,跪求大家支持赞赏!

九转的爱情.蛊。

赵怜云并不认识这只蛊虫,但这并不妨碍她认出眼前蛊虫的不凡。

因为在她的视野中,这种蛊虫仿佛成了天地、宇宙的中心,它浑身上下散发出漫天的霞光,整个蛊虫仿佛世间最绚烂的火焰燃烧。一切的事实,世间的万物都要为其倾倒,都要围绕着它而运转。

“这是……奇迹吗?”赵怜云泪流满面,脑海中浮现出最后一个念头。

随后,她就瘫倒下去,当场昏迷过去。

灵缘斋震动!

爱情.蛊,乃是灵缘斋的镇派仙蛊。

它主动飞出自己的栖息地,来到赵怜云的身旁。更关键的是,即便是赵怜云昏迷过去,它也没有飞离远去,而是停留在了赵怜云的肩膀上。

当夜,灵缘斋召回全派蛊仙,紧急商讨此等撼派大事。

灵缘斋太上大长老、二长老、三长老、凤九歌、白晴仙子、徐浩、李君影等等,英杰汇聚,仙气飞扬,乃是灵缘斋数百年来,蛊仙到场最全的一次。

“诸位,我们都知道门派的规矩。如今爱情.蛊已经认可了赵怜云,那她就是本派的当代仙子!”徐浩开口,他面无表情,但语气中却透露出一丝狂喜之意。

当初,他指点赵怜云,也不过是算计凤金煌,打击她的名望,抬升赵怜云的小小伎俩。

没有想到的是,赵怜云死脑筋,仍旧长跪不起,执著于此。

徐浩对此也有许多无奈。

他培植赵怜云,将她当做棋子,来对付凤金煌。真正的目的在于打击凤金煌的父亲凤九歌。

人在江湖。总有内斗外斗。

组织大了,内部自然就会分化各个派系。

目前,以凤九歌为首的凤派,在灵缘斋中势力最大。

徐浩、李君影等人。被凤派压在下面,几乎喘不过气来。好在灵缘斋的两位八转蛊仙太上大长老、太上二长老,都公正无私,不偏不倚,以灵缘斋大局为出发点。这才使得徐浩、李君影没有被凤派,排斥出灵缘斋的权力核心。

赵怜云跪求凤金煌,绝大多数外人只能看到表面,只有少数人明白,这是灵缘斋中的政治斗争的一个斗争表象。

赵怜云虽然只是区区一个凡人,但自从继承了盗天真传之后,的确有资格有资历和凤金煌竞争灵缘斋的仙子之位。

徐浩将赵怜云当做他手中的斗争武器,一枚棋子。

本来,徐浩还琢磨着,不能让赵怜云再跪下去了。在他看来。这是一场政治秀,已经可以结束了。赵怜云如此跪下去,简直是痴傻无比,她不会再获得什么。

但是,在前几个时辰,当徐浩听到爱情.蛊居然主动出现,并停留在赵怜云的身上时,他惊呆了。

极度的震惊之后,就是狂喜!

他敏锐地意识到,这简直是一份从天而降的礼物。这份礼物太过厚重。以至于他都差点被幸福砸晕过去。

这是一个千载难逢的机会。

必须要趁这个机会,将赵怜云扶持为灵缘斋的仙子!

再在这个基础上,反击“凤派”,这将是他徐浩在门派中史无前例的机遇!只要他把握住这场机遇。那么他今后的门派生涯,将得到巨大的改观。

徐浩毫不怀疑这一点。

所以,他首先就直接站出来,强烈支持赵怜云登位,成为灵缘斋的当代仙子!

他继续开口:“我派门规中讲述的清清楚楚,历代仙子选拔。一听民意,二看资质,三重上意。”

意思是:先看看整个门派的人心所向,然后看候选人的资质和才情如何,最后着重是高层蛊仙们更属意哪一位。

徐浩继续侃侃而谈:“如今论民意,赵怜云虽是天外之魔,但有情有义,为了拯救情郎,不惜跪求对手。门派上下都知道她的秉性。再论资质才情,其实资质并不重要,只要愿意,完全可以提升一个凡人的修行资质。而赵怜云的才情,相信诸位也知晓她在北原的种种经历,才情方面也是人中精英。最后是诸位的心意。我非常认可赵怜云,我在此表面,本人十分支持赵怜云成为本派的仙子。”

“我附议。”徐浩话音刚落,李君影就开口赞同。

她是徐浩之妻,向来共同进退。

这并不令人奇怪。

但接下来的一幕,却是让其他有些震撼。

“我同意徐浩的观点。”

“我亦认可徐浩此言。”

“既然连爱情.蛊都认可赵怜云,那我还有什么可说的呢?”

……

一时间,居然有五六位蛊仙都附和徐浩的话。

白晴仙子有些坐不住了。

她感觉:似乎在场所有的人都认同了徐浩的话,觉得灵缘斋仙子的位置,应当属于赵怜云的。

白晴仙子是灵缘斋的蛊仙,更是凤金煌的亲娘。

灵缘斋的仙子之位,是很特殊的。

她是灵缘斋举派上下,从凡人蛊师中选举而出的修行种子。然后重点投资,着重培养,助她升仙。每一代的仙子,成长起来,几乎就是灵缘斋的门面。

灵缘斋是一个组织,组织中不仅要有老人,更需要新的蛊仙力量,需要新的血液,来填充,来运转。

白晴仙子本身,也是灵缘斋的仙子。她亲身感受过,整个门派资助她修行是什么滋味。

虽然凤九歌、白晴都是蛊仙,单凭他们俩也完全可以独立地资助女儿凤金煌升仙,但这种资助力度,完全没有整个门派资助那么强大。

白晴仙子并不愿意看到,女儿就这样落选。

但她亦深深的明白:现在的局面对她非常的不利!

因为门派的规定中,选举仙子除了“一听民意,二看资质,三重上意”之外,还有最重要的一条,那就是

一旦爱情.蛊认可了哪一位。不管这位多么不得人心,多么资质糟糕,多么不被蛊仙高层看好。那么这一代的仙子,非其莫属。且不论男女!

正是因为有这一条规定,所以徐浩的话一说完,就得到了许多蛊仙的认同,纷纷表示支持。

事实上,单凭徐浩的政治势力。并不强大。在平常时候,是绝不会得到如此之多的支持的。

“虽然门派规定如此,但是我并非没有希望。因为在门派历史上,就有按照这个规矩扶持出来的仙子,而她却……”

白晴仙子意动,她张口欲言。

但就在这个时候,她身旁的凤九歌轻轻按住她的手背。

“稍安勿躁。”

“我知道,你是想提及墨瑶吧?当初,她只是一位墨人奴隶,却得到了爱情.蛊的认可。被举派资助,成为蛊仙,感化薄青。但最后,却是她背叛门派,违逆天庭。”

凤九歌传音。

白晴仙子点点头,暗中交流:“不错,这是煌儿唯一的机会。我们只要照准此点,商讨下去,利用凤派的力量,说不定就能让门派改掉这个规定。毕竟。门规也是人制定出来的,不是吗?这个门规显然一点都不合理!”

但凤九歌摇摇头。

他否决了白晴仙子的主意:“徐浩乃是智道蛊仙,为什么他之前的话,并没有谈及此点门规呢?他只论前三项条件。似乎完全将这个对他最有利的门规忘记脑后了。”

白晴仙子微微一愣。

凤九歌继续道:“这是他的计谋,一个陷阱。你若是谈及此事,就是被他算计成功了。虽然出了墨瑶、薄青的事情,但门规始终没有改变。义天山大战之后,我派承受压力极大,但对此条门规也从未有改换的意向。这里面的水。深得很。今天你要是鼓动凤派的力量,不自量力地挑战,想要改掉这个门规,应当会惨败收场。我方重创,而徐浩却毫发未损,这正是他想要看到的。”

白晴仙子浑身冒出一层冷汗。

“是我鲁莽了。”她很快冷静下来,然后沉下心来,反思自己为什么会犯这样的错误。

“事关我煌儿,我因爱而乱。不,还有一点,我升上七转,又有七转仙蛊,比较之前,心境上浮。却罔顾正道之中,不是光看修为和拳脚。今天若不是夫君提醒,我险些就犯下大错。”

想到这里,白晴仙子不由地看向凤九歌。

她的心中涌起一股强烈的安全感。虽然凤九歌是魔道出生,但这些年来,在门派中他却不是靠着拳头说话,政治手腕也凌驾于自己,若非如此,怎可能有如今凤派的辉煌?

“夫君既然能识破徐浩的陷阱,那我就将此事交给你啦。”白晴仙子暗中传音,语调中透露出甜蜜的情意。

凤九歌却苦笑:“这个事情,我并不打算插手。因为插手也无济于事。”

“什么意思?”白晴仙子一愣。

凤九歌没有说话,而是抬头望了望天。

“居然没有上当……”徐浩原本还期待着有人跳出来,但却无人反对他的话。就连凤九歌都默不作声。这让徐浩感觉有些可惜。

但他很快收拾情绪,心中笃定:“也罢!既然无人反对,此事就是铁板钉钉了。”

哪知太上大长老接下来开口:“诸位的意思,我已经明白了。不过赵怜云毕竟身份特殊,乃是天外之魔。我会将你们的意思汇报上去,请上面亲自定夺。”

“什么?”徐浩顿时有些傻眼。

(未完待续。)

\end{this_body}

