\newsection{方源的修行契机}    %第一百六十六节:方源的修行契机

\begin{this_body}



%1
仙道杀招度日如年!

%2
无数的蛊虫盘旋飞舞,青提仙元消耗极剧,恢弘的光芒充斥整个至尊仙窍。

%3
哗哗哗。

%4
至尊仙窍中的那条光阴支流,显露出形态,肉眼可见。

%5
在度日如年的威能下,这条光阴支流越缩越小,河面也越来越窄。

%6
“呼!”

%7
吐出一口浊气,方源慢慢地放松下来。

%8
“成功了,仙窍的时间流速已经被我大大延缓。灾劫的每一次时间间隔,都要拉长很多。”确认杀招催动成功后,方源便抽回全部心神。

%9
现在,他身处于琅琊福地中的云城。

%10
自从吞并了韩东福地之后,方源就立即回程。

%11
太丘过于危险,方源身上的暗渡效果也有时限。天意可是时刻要谋害方源的。

%12
这一次,仙窍时间放缓了许多倍,一下子将灾劫都甩远了。

%13
方源顿时感到浑身轻松,压力骤减。

%14
做出这个决定,他也是经过深思熟虑的。既然改变了修行方法,那么自身渡劫就无须那么频繁。反正吞并福地,跨越灾劫,这个效果和方源本身福地的光阴流速没有丝毫关系。

%15
现在的关键,是仙窍福地。

%16
哪里有仙窍福地?

%17
方源首先想到的是中洲的狐仙福地、星象福地。

%18
但旋即,他就这个不切实际的想法,甩出了脑海。

%19
根本不可能!

%20
中洲就是龙潭虎穴,十大派,还有比长生天底蕴更深的天庭,都恨不得把方源大卸八块。

%21
就像影无邪等人逃离中洲一样,中洲对于方源而言,也是禁区。

%22
前段时间,星象福地迎来灾劫,地灵来信哭泣求救,方源都没有搭理,只是输送了一些凡蛊和仙元。交给地灵,让他自救。

%23
星象福地如此,暴露出来的狐仙福地,更是碰都不能碰。

%24
方源又想到了落魄谷。

%25
准确的说。是落魄谷原来在的地方。

%26
落魄谷曾经发生过大战,以凤九歌为首的中洲蛊仙,和以秦百胜为首的影宗北原势力,发生了激烈的厮杀,最终中洲蛊仙胜出。

%27
在落魄谷的原址空间中。寄托着不少蛊仙的仙窍福地。

%28
“这些福地,就在北原,我是不是可以前往收刮呢?”

%29
这个颇为诱人的想法,让方源稍稍犹豫了一下,但很快,他就抛弃了这个选项。

%30
因为这个选择,也蕴藏着极大的危险。

%31
尽管这场大战,发生在暗处,北原蛊仙们几乎都不清楚。但中洲十大派、天庭知晓,影宗方面也知晓。

%32
蛊仙陨落之后。形成仙窍福地或者洞天。

%33
这些福地洞天,都蕴藏着蛊仙生前的资源积累。即便是再家大业大的势力,也会千方百计地展开回收工作。

%34
“中洲方面既然能派遣调查队前来北原,查探八十八角真阳楼倒塌的真相,那么也可以再派遣一队,前来回收仙窍福地。”

%35
“还有影宗方面,兴许也在试图回收。”

%36
“我之前没有过去,现在就更不可能过去了。”

%37
方源的这个决定,无疑是正确的。

%38
因为此时暗地里为中洲方面,主持仙窍回收工作的。不是别人,正是八转蛊仙凤仙太子!

%39
如果方源盲目乐观,主动跑到那里去,凤仙太子碰到这样的好事。居然能守株待兔成功,估计将来会睡着笑醒。

%40
方源最后想到东海。

%41
东海的乱流海域,在乱流海域的中心地带,有着气泡海岛。

%42
海岛中藏着一个天地秘境市井,市井里面存有大量的蛊仙仙窍福地。

%43
这些蛊仙都是方源利用市井的特性,通过力道大手印。活生生拍死的。他还趁着福地落下后的关键时机,进入里面探索过,多少搜刮了一些资源回来。

%44
更加诱人的是,这里面还有一个丁齐福地。

%45
丁齐是七转血道蛊仙,他临死自爆,没有给方源留下什么资源、仙蛊和魂魄,但是仙窍却保留了下来。

%46
方源虽然能够吞并他的仙窍,但在当时却没有强行动手。

%47
一来,丁齐福地中的地灵,认主条件比较艰难,需要方源将一个血道仙蛊方“血仇”完善出来。方源若是有智慧光晕,必能速成此事。但当下智慧蛊不能动用,而血仇仙蛊方只有一成不到的完善程度,方源若要完善这个仙蛊方,即便他是血道宗师境界,也需要耗费大量的精力和时间,效率太低。

%48
二来,方源当时的修行方法,是注重未来潜能,按部就班的一次次渡劫。吞并丁齐福地,却是违背了他当时的计划。

%49
但现在不同了。

%50
方源先是没有达成东海之行的既定目标,虽然削弱了影无邪等人,但没有斩杀掉他们,后患无穷。方源对影无邪认知很深,认为他们发展速度将高于自己,下一次见面恐怕就要攻守易位。

%51
然而异族大联盟,经历了黑凡洞天之争,又加入楚门,不久后又要进行疯魔之约,身份越加复杂。方源就像身陷了一个个的激流漩涡,很多时候,不是他想怎样就怎样。盟约制约着他,让他身不由己。

%52
此一时彼一时,外在的情形不断发生变化,让方源意识到自己需要改变。

%53
前世五百年的修行经验,赋予了他对未来发展,和潜在危机的一种敏锐的嗅觉。

%54
他明智地预感到,若是他自己不主动改变,恐怕将来的结果会很不妙!

%55
发现了至尊仙窍吞并其他福地的特性优势,只是一个契机。

%56
“我是血道宗师境界,吞并七转福地,境界不是问题。但仙窍只能以大吞下,这里的大,究竟是指什么?”

%57
若是单指修为,方源六转仙窍,自然是不能吞并七转丁齐福地的。

%58
但若是指的是仙窍本身的底蕴,那至尊仙窍吞并七转丁齐福地妥妥的!

%59
这个问题,方源需要搞懂。

%60
也不难搞懂,试验一下,就知道了。

%61
“但除此之外,还有一个问题。那就是地灵认主。”

%62
“丁齐的地灵认主条件比较困难,难以让他沉浮。但若强行认主,又会导致地灵毁灭,福地跟着灭亡,刮起大同风。”

%63
“这个大同风在市井中刮起来,会不会波及其他仙窍?会不会直接损毁了市井?”

%64
这个顾虑,方源当时在东海就想到了。

%65
所以,他才放着那些仙窍福地,没有强行动手。

%66
如果让地灵毁灭,而福地不毁,那就给仙窍吞并太古九天碎片。

%67
可是太古九天碎片,唯有洞天才可以吞纳融入。

%68
就像是黑凡洞天,虽然方源毁了天灵,但是黑凡洞天仍旧存在。正是因为这个洞天世界,融入了一份太古九天的碎片。

%69
但地灵是不行的。

%70
而且乱流海域远在东海,方源一来一回,可要耗费不少时间。天意说不定也在布局,防止方源获取市井。

%71
方源细细思量下来,他发现还有一个弊端,限制他吞并仙窍。

%72
那就是境界。

%73
方源不是魔尊幽魂,后者各个流派的境界都相当的高,可以随意吸纳仙窍福地,融入至尊仙窍。

%74
但方源不可以。

%75
“或者说,影宗炼制至尊仙胎蛊,就是为了吞并仙窍去的?”

%76
方源发现,若不是自己夺取了至尊仙胎蛊,让魔尊幽魂重生过来。

%77
那他发展的速度,绝对是无以伦比的惊世骇俗!

%78
他杀性本来就是尊重中最重的,杀招众多,斩杀蛊仙后,要么汲取蛊仙的主流道痕,放弃仙窍,提升自身道痕底蕴。要么吞并仙窍,跨越灾劫,让灾劫来临的时间重新归零。

%79
两种手段,左右互搏,简直是潇洒自如。

%80
“难怪前世魔尊幽魂能够成功!”

%81
“难怪天意要千方百计,将我送到过去来,阻止这个家伙。”

%82
想到这里,方源不禁下意识地将视线移注到南边方向。

%83
那是南疆的方向。

%84
方源知道,那里有一片超级梦境,梦境中有魔尊幽魂的一缕残魂。那是影无邪等人,千方百计拯救的对象。

%85
“境界……梦境……南疆!”方源忽然身躯一震,想到了什么。

%86
他脑海中的思路,一下子变得清晰无比!

%87
“我方源的修行契机,就在南疆啊。就在那个地方……”

%88
大雪山,第一峰巅。

%89
炼道蛊阵持续运转着。

%90
万寿娘子精神奕奕,再次出现在马鸿运的面前。

%91
马鸿运看到她,浑身毛发炸立,高声叫喊道:“你又来电我,你又要来电我!”

%92
“知道就好。”万寿娘子笑了笑,缓缓地将手中的雷霆电球,推送过来。

%93
“啊啊啊!”马鸿运被电得剧烈抽搐,大翻白眼,但却没有口吐白沫。

%94
同时,他的叫声也很嘹亮,仔细听的话,还很有节奏感。

%95
总之,他支持了好一会儿,这才昏死过去。

%96
“修为又提升了?还有,抗击电炼的时间也延迟了三成。”万寿娘子明察秋毫,皱起眉头。

%97
从她的眼角、耳窍、鼻孔中,缓缓流淌出几股鲜红的血迹。

%98
这是炼蛊反噬所致。

%99
但万寿娘子经历了数次,也适应了这种反噬的伤害。之后的疗伤更是轻车熟路了。

%100
“有点麻烦。马鸿运被电炼的次数越多,他的抵抗能力就越强,这一关键步骤就越可能失败。该怎么办?改变炼蛊方法吗?不!现在改变方法,就要改变整个仙蛊方。之前搜索的仙材,一大半都不能用。大雪山已经承担不了第二次的波折了。”

%101
万寿娘子暗叹一声,忧心忡忡。

\end{this_body}

