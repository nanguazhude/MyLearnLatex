\newsection{怀疑重生}    %第七百七十三节:怀疑重生

\begin{this_body}

%1
天庭。

%2
光明无限,仙殿重重。

%3
中央大殿之后,一座高塔矗立着,浑身笼罩着一层白金光晕。

%4
正是八转仙蛊屋监天塔。

%5
围绕着监天塔,数位天庭蛊仙正相互配合,紧张忙碌。

%6
在他们的劳作下,一座炼道大阵正在逐渐成型。

%7
一位高大雄健的蛊仙,站在阵外打量。虽然只是简简单单的站着,却给人巨柱擎天之感。

%8
正是龙公!

%9
“拜见龙公大人。”紫薇仙子这时赶来,站到龙公身后。

%10
“如今琅琊福地落入方源之手,这就意味着他也掌握了炼炉的奥妙。原先天庭的炼道大阵,就是模仿炼炉所制。我便命人拆解了原来大阵,改而铺设另外的大阵。”龙公说到这里,缓缓转身,看向紫薇仙子。

%11
紫薇仙子的脸上闪现一抹忏愧之色:“是我领导无方……”

%12
话还未说完,就被龙公抬手打断:“自责的话,说多了也无用。我相信你已经反省良多,过去已经铸成事实,那么我们更应该认真地面对现在,筹谋将来。”

%13
龙公看着紫薇仙子,叹息一声:“如今我天庭仍旧大势在握,然而局面似乎正在脱离我们的掌控。方源这个人……是红莲魔尊安排的棋,前番他击退我方攻势,令陈衣、雷鬼真君阵亡,如今更是俘虏了南疆群仙。不管他只是七转,也不管他利用了什么手段,结果就是结果,事实就是事实。紫薇啊,你如何应对?”

%14
上一世,方源没有让龙公重视,认为是跳梁小丑。

%15
如今,方源接连大胜,战绩骇人,龙公也警惕起来。

%16
他不再像上一世对紫薇仙子那般信任有加,而是开始不放心,主动过问相关的事。

%17
紫薇仙子呼吸一口气,如何针对方源,她早已谋算妥当,此刻从容不迫地答道:“要对付方源,第一点要找到他的踪迹。这一点单靠智道手段做不到,如今我已秘密在南疆安排了三位八转蛊仙,暗中积极搜寻方源线索。”

%18
“第二点,是要对付他的定仙游为核心的杀招。此招比四通八达上古战阵更具威胁,令方源掌握主动,进退自如。”

%19
“我已下令,时刻准备抢炼定空蛊。”

%20
定空蛊在蛊仙刘浩手中,如今刘浩又被方源俘虏。但紫薇仙子自信:凭借天庭的手段,还是能够令刘浩手中的仙蛊自毁。

%21
有了关键的定空蛊,就有对付定仙游杀招的手段。

%22
“而第三点。”紫薇仙子说到这里,顿了顿,语气微沉,“就是要防备方源的春秋蝉。”

%23
“我现在越发怀疑,方源似乎已经用过春秋蝉,进行了一次重生。”

%24
“琅琊福地一战,我方败北,主要是因为星投杀招被识破。”

%25
“这一点,大有蹊跷。凤九歌留下的道痕那么多,分部各地,为何方源就笃定那一处?提前铺设了大阵,对我方围追堵截?”

%26
“然而我推算多次,却劳而无功,实在是线索太少。”

%27
紫薇仙子深深叹息。

%28
“春秋蝉的确是一个麻烦。”龙公点头,“你有什么应对之法吗?”

%29
紫薇仙子点头:“龙公大人,你我之前不是已经商议过,采用一视同仁杀招来对付春秋蝉。然而仁蛊却是在南疆仙界,当代乐土传人陆畏因的把控之下。”

%30
“不错。”龙公沉声道,“方源再用春秋蝉,未必是单纯用蛊,极可能运用杀招。而用一视同仁杀招来对付任何可能的春秋蝉的杀招,最是稳妥。”

%31
“正是如此。”紫薇仙子眼中闪烁着紫芒,“所以,我打算最近就联络陆畏因,和他商量租借仁蛊一事。”

%32
上一世这个时期,天庭从容不迫,游刃有余,远未到联络陆畏因的时候。

%33
但这一世因为墨水效应,导致天庭开始提前联络陆畏因。

%34
龙公哦了一声,有所感悟:“你是想借机刺探,看看方源是否重生?”

%35
“是的。假设陆畏因拒绝我们,那方源重生的可能性就更大了。因为我方没有一视同仁杀招,来克制方源。”

%36
“但若是陆畏因答应了我们,证明我们能掌握一视同仁的手段,方源重生的可能就要变小。”

%37
“与此同时,我们也能刺探陆畏因这位当代乐土传人的态度。”紫薇仙子徐徐地道。

%38
提到乐土之名,龙公的脸上也涌现出一抹复杂之色:“就这么办吧。”

%39
“不过,就算陆畏因不借,我们缺少应付春秋蝉的手段,也不用担心忧虑太多。”

%40
“只要我们大势在握,步步为营,谅那方源小贼如何蹦跶,也无法翻盘。”

%41
“当年红莲那样的实力和底蕴,最终也是无可奈何,落得失败下场。更何况他?”

%42
龙公说道这里,笑了一声,大气从容。

%43
紫薇仙子却是犹豫了一下,问道:“龙公大人,我有一事不太明白。我天庭底蕴如此深厚,为何对付春秋蝉,竟然是要商借外人的仙蛊。难道我偌大的天庭就真的拿不出克制春秋蝉的手段吗?”

%44
龙公脸上笑意收敛起来:“这里的主因恐怕在于红莲。当年,我和他大战一场,曾听他亲口所说。他说他在光阴长河中,布置了数座石莲岛,岛岛之间互不关联,但却又奇妙影响。所有的石莲岛组成一记仙道杀招,乃是他最为得意的手段。这个手段能够影响光阴长河的流转,从而间接地影响天下万物的轨迹。”

%45
若无光阴长河,整个天地都会处于静止之中。

%46
有了光阴长河的奔流运转,天地万物才会活动。

%47
借助这层联系,红莲魔尊施展了某个宏大的手段,影响一直延伸到现在。

%48
最为显著的成果,就是但凡能克制春秋蝉的蛊虫、手段,都和天庭无缘。

%49
出了红莲这事后,天庭一边掩盖事实真相,维护自身权威,另一边则积极搜寻、筹谋,想要找到对付春秋蝉的手段。

%50
渐渐的,天庭发现了古怪之处。

%51
任何能克制春秋蝉的仙蛊,都在刻意地“避开”天庭。天庭中招收进来的蛊仙,专修宙道的数量也很是稀少!

%52
“不过随着时间流逝,这种影响逐渐减弱。天庭库藏中也有一些手段,能够影响、干扰春秋蝉。这些你也知道,并不是太上得了台面。”龙公继续道。

%53
“原来如此。”紫薇仙子面色平静,心中震撼。红莲魔尊的这个手段,真的太厉害了,这简直是超脱了宙道的极限,有了一丝宿命的色彩!

%54
他不仅损害了宿命蛊,而且还硬生生地用宙道手段,模拟出宿命的威能,反过来对付宿命。

%55
紫薇仙子眨了眨眼,迅速平复情绪:“接下来,我会主动放出消息,宣扬方源企图渡劫,晋升八转的消息。”

%56
“同时,和武庸交涉,和南疆正道的其他势力接触,争取更多人抵制方源的勒索。”

%57
“最后一点,我已下令,大力栽培古月方正。此人是天道造就,对付方源的一大关键。如今他已在升仙了吧?”

%58
中洲。

%59
血雨渐息,山峰之上,古月方正悬浮于空,天地人三气逐渐浓缩成一点。

%60
他的经历也颇为曲折坎坷,这些磨难转变成财富,让他能顺利地把握三气平衡。

%61
方正忽然张开双眼,深呼吸一口气,将一只五转血道蛊虫猛地置入混元气团当中。

%62
轰!

%63
耳畔一声惊天的轰鸣,气团爆炸,顿时炸出一个仙窍来。

%64
上等福地!

%65
方正升仙成功,并且还有残余的天地二气存在。

%66
方正心中欢喜,有了这些天地二气,正可将他的本命血道凡蛊提升成仙蛊!

%67
“方正,快让我进去,护卫你渡劫!”樊西流传音,方正整个渡劫的过程都有他在一旁护持。

%68
“嗯?”方正眉头一皱,面色不愉。这仙窍乃是蛊仙最隐私的地方,樊西流想要进来,顿时令方正感到了一种被冒犯的恼怒。

%69
但旋即,他便舒展眉头,打开了仙窍门户,语气平平淡淡:“也好,樊西流仙友,请进来吧。”

%70
方正明白,不只是眼前的樊西流,自己升仙过程中,还有无数双眼睛在暗暗盯着。

%71
他已经渐渐明白了自己的价值,他是一个工具,之所以能得到栽培,是为了对付方源。

%72
片刻后,灾劫彻底结束。

%73
樊西流身处在他的仙窍中,神情有些复杂:“方正啊,你这可是上等福地,想当年我千辛万苦积累,冒险渡劫,也不过是中等福地罢了。”

%74
“你这片仙窍福地,至少有八百五十万亩,最高纪录不过九百万亩。”

%75
“你这里绝对是光阴大脉支流,和外界光阴流速相比,是二十八比一。上等的极限是三十比一,相差很小。”

%76
宙道、宇道资源,就甩出樊西流当年成绩一大截。

%77
拥有上等福地,就是六转蛊仙中的精英,本身数量就很稀少。

%78
“此次渡劫,你还将本命蛊升炼,拥有了第二只血道仙蛊。”

%79
“门派有令,还让我给你带来了众多资源,让你栽种,直接迈过成仙最初的艰难起步阶段。”

%80
樊西流说着,从自己的仙窍中掏出大量的血道资源。

%81
一份份资源,都是价值巨大。很多都是仙材,让樊西流这位积年蛊仙,都难掩羡慕嫉妒之色。

%82
门派对方正的栽培力度何其巨大!

%83
这些资源的价值总和,堪比六转散修的百年积累。即便是樊西流背靠着仙鹤门,也花费了五十年左右的光阴,才达到此等地步。

%84
可以说,方正什么都没有做,直接成为了六转蛊仙中的富翁。

%85
“门派的一片苦心,方正,你可不能辜负。”最后,樊西流说着,声音都有些沙哑。

%86
光是看樊西流满脸复杂之情,方正便心中有数。

%87
他也为自己被如此重视和栽培,而暗暗吃惊。

%88
“不管是什么目的……我终于成仙了。方源,你我再不是仙凡之别。”

%89
方正神念四扫,俯瞰着自家仙窍,心潮澎湃:“这就是我的仙窍,我的福地,我的基业!”

%90
这一刻,他感到前途一片光明,似乎有着无穷无尽的可能。

%91
“方源你虽然强大,但还不是被中洲、天庭撵得四处跑,不敢涉足中洲半步。”

%92
“我有他们的帮助,来对付你,未必不能成事。”

%93
“虽然我理解你,但不会原谅你。命运弄人,你我终究还是要兵戎相见的。”

%94
和上一世不同,方正得到了更多的栽培,并且升仙的时间也大大提前了。

%95
上一世,世人皆知南疆蛊仙的噩耗,方源魔威赫赫。

%96
而现在,天庭保留了琅琊战败的某些实情,南疆惊变的消息还未来得及扩散出去。

%97
所以,古月方正还不知道陈衣、雷鬼真君战死的秘密,更不晓得南疆诸仙被方源俘虏的消息。

%98
这让刚刚升仙的他产生了一种错觉:似乎这样走下去,还是能够走到方源的面前,抬起头看着他的。

\end{this_body}

