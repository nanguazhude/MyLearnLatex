\newsection{一百七十三万!}    %第四百三十一节:一百七十三万!

\begin{this_body}

“哦,楚度借了我十五万的仙元石?”

方源很快就得到了楚度的回信,并且楚度借出的仙元石数目,有些超过了方源的预料。(WWW.mianhuatang.CC 好看的小说棉花糖

而在楚度的回信当中,他言辞委婉,先是回顾了往昔愉快的合作经历,又述说命运弄人,彼此间的身份差异,处境尴尬。并且借出十五万仙元石,已经是能力极限。而归还时限,则让方源来定。

“看来,楚度仍旧是想与我继续合作下去的。毕竟狂蛮真意,还是相当诱人的。”一封回信,顿时让方源明白了楚度的态度和想法。

楚度原本出身散修,后来和刘家放对,被批成魔道阵营中去。现如今他组建了楚家,却是投到了正道。

但他骨子里,还是以利益为先的。

所以尽管和方源合作,风险很大,但是当中的利益却是更多,让楚度不惜铤而走险,不仅借出十五万仙元石,而且还亲自回信。

这只信蛊,虽然只是凡蛊,但是意义非凡。一旦方源抛出去,就是楚度勾结魔道的强大证据。

但这也是楚度故意为之。他特意用这种方式,授人把柄,来表达出自己的合作诚意。

而方源如今,也是充分感受到了这股诚意。

“楚度还是可以继续合作下去的。”

“当然了,我之前借给他的招灾仙蛊,也是我们之间合作的强硬基石。”

方源比较满意。

十五万的仙元石,已经不少了。

一般而言,六转蛊仙掌握的仙元石储备,最多以千计量。

七转蛊仙的话,大概以万计量。<strong>求书网WWW.Qiushu.cc</strong>大多数七转蛊仙,通常会有数千的仙元石储备。

而到了八转层级,数万仙元石储备则是垫底情况,往往会有数十万储备,资深前辈会达到百万级储备。

至于九转,仙元石储备就会上升到千万级,乃至亿级!

以上是根据蛊仙个人情况而言。

论势力的话,一般超级势力本身,都会有一百万上下的仙元石储备。

武家有这样的储备,所以在梦境大战中,武庸主动借给方源十万块仙元石。这是武家的能力范围之内的事情。当然,武家可以借出更多,但武庸是为了让方源稳住仙阵,而这十万块仙元石足够用了。

而楚家是楚度草创不久的超级势力,虽然蛊仙数量也有不少,但绝大多数都是力道,身家浅薄。按照方源的估量,楚家的仙元石储备能有六十万就不错了。

在这六十万中,楚度借给方源十五万,并且还不规定归还时限。这其中的诚意,自然是分量十足的。

毫无疑问地讲,仙元石储备也是衡量蛊仙、势力的一个因素。

不过,就像存钱,不是存放的钱财越多,就越好。钱财堆砌起来,不流通等若死物。

仙元石储备也是同样的道理。

有了楚度的支援,方源原本干涸见底的仙元石储备,一下子暴涨到了十五万余!

不过,方源并不满足。

他继续耐心等候回信。

他不仅仅是向楚度去信,同时他还向疯魔窟的三怪,东海的庙明神商量借取仙元石。

借仙元石只是目的之一,方源还有一个主要目的,就是获悉他人态度。

尤其是在当下,他的柳贯一身份曝光的敏感时期。

疯魔三怪的信,几乎和庙明神的回信,同时到达。

两方都答应了方源的请求,各自借出一笔仙元石。前者有二十万,后者有十万。

疯魔三怪对方源的态度,一如之前,甚至还更加热情了些,并没有因为方源的真正身份,而产生芥蒂。对于他们三人而言,没有什么正道、魔道之争,只有无极魔尊的真传!谁能帮助他们达成这个心愿,他们就认可谁。

而方源的实力越强,他们认可的程度就越高。

至于庙明神,之前和方源的另外一个身份“楚瀛”,相处甚欢。换蛊交易会上,方源还主动卖给庙明神寿蛊,这让方源在庙明神心中的地位急剧攀升。

所以庙明神在来信中特意讲到:他最近手头有点紧张,十万块是目前极限,但若是方源还不够,他还可向其他仙友借来一批,支援方源。

如此一来,方源有了这三批仙元石,总量便达到了四十五万。

不管前世还是今生,他还从未有过如此庞大规模的仙元石。

然而,方源并没有停止赊借的行为,接下来他又和琅琊地灵商量,借来一批仙元石,又用门派贡献换取一批,直至达到琅琊地灵的底线。

“不能再换给你了。虽然你的门派贡献很多,但是我派也需要一定的仙元石储备,来以防万一的突发情况。”琅琊地灵拒绝的态度很是坚决。

最近几年,因他更改福地发展大略,使得原本深厚的底蕴不断损耗。再加上之前,方源动用大批门派贡献,让琅琊派又出力又出仙材,不断炼制仙蛊,也狠狠消耗了一把琅琊派的底蕴。

直至琅琊地灵拒绝方源,琅琊派的仙元石储备已经不足三十万。

方源又向白凝冰下手。

白凝冰可是继承了白相洞天,那里仙元石储备可有不少。

在众多的影宗蛊仙中,也就白凝冰最为富裕。

黑楼兰最穷,影无邪也是如此,虽然顶着翠波仙子的肉身,但是她的仙窍中的一切,都是白凝冰的战利品。

白兔姑娘、妙音仙子听闻方源的举动后,也各自主动借来一批仙元石,只是数量颇少。

在所有人当中,白兔姑娘的条件最为宽松,她通红着脸,对方源说哪怕不还也没有关系。其次便是妙音仙子、楚度,没有划定归还时限。

疯魔三怪虽然想要利用方源,但毕竟交情浅薄,一切都按照市场规矩办事。琅琊地灵同样如此。这当中,庙明神的归还时限,最为宽松。

最苛刻的,要数白凝冰。她提出一个很有意思的约定,她可以借给方源大量的仙元石,但是方源借来越多,利息就越高,归还的时限就越短。

于是,方源将她手中绝大多数的仙元石,都借到自己手中来。

如此折腾了好一番后,方源的仙元石储备达到惊人的一百七十三万余!

一笔巨款。

这种程度的储备,已经甩开了楚家、琅琊派,在大多数的超级势力中,都可拿得出手。在蛊仙个人当中,几乎没有六转、七转蛊仙能有这样的资产,八转蛊仙中也是少有,只有那些资深强者,才有上百万的仙元石储备。

当然,这只是单纯比较数量。

真正“健康”的资产,是保持进出平衡。而方源虽然手握巨款,但都不是他的,只是他借来的。

一时间,方源负债累累。

不过,他没有丝毫的紧张,他有十足的自信,能够利用这笔仙元石,赚取更多!(未完待续。)

\end{this_body}

