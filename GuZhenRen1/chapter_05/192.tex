\newsection{仙缘生意}    %第一百九十二节:仙缘生意

\begin{this_body}

“什么?”听到树下两位蛊仙的交谈话语,方源的耳朵顿时支楞起来。

他直接从树丛中钻出来,出现在两位蛊仙的头顶上方,蹲在一棵大树杈上往下看。

对于方源现在变化的这种猴子而言,好奇心极强,这是非常正确的反应。

两位蛊仙连头都没有抬。

但方源清楚得很,这两位蛊仙早已经催动着侦查手段,也已经发现了方源。只是方源的见面曾相识,乃是八转仙道杀招,完全凌驾于两仙之上。

两仙不疑有他,继续交谈。

其中一人道:“罗兄尽管放心,我也是听闻白兔姑娘所言,才起了兴趣,想接触一下传闻中的梦境,试一试自己的机缘。我焦雷子虽然只是一届散修,但是凡事皆讲诚信。既然答应了你,又定下了信道盟约,自然不会做出一些出格的事情来。”

“正是因为焦雷子你有这样的信誉,我们兄弟也才会和你接触。否则的话,你连我的面都不会见到。跟我走吧。待会进入大阵入口,你就自称是罗家中人,为我兄带仙蛊而来。”罗姓蛊仙转身就走,一边走,一边丰富。

另一旁,自称焦雷子的散修紧随其后,悉心聆听。

方源坐在树杈上,看着他们远离,还跟上去,名目张胆地跳了几棵大树。

直到两位蛊仙消失不见,他这才停下身躯,左顾右盼,发出疑惑的叫声。

伪装得极其逼真,尽展变化道宗师的风采。

至始至终,这两位蛊仙都没有怀疑。

“有点意思了。”方源缩回树杈丛中,双目精芒暴射,心头上涌现出兴奋之情。

他意外地撞见了,两位蛊仙的秘密交易。

这两位蛊仙好巧不巧,选择在这座山峰上碰面,还偏偏落到了方源的眼皮子底下。

这运气还真是不赖!

听他们的话音。很显然是防御蛊阵中的一位正道蛊仙,和外面的散修蛊仙焦雷子做了一场见不得光的交易。

正道蛊仙为焦雷子领路,让后者进入防御蛊阵之中,尝试去触机缘。

对于焦雷子的想法。方源是嗤之以鼻的。

“你以为这梦境是这么好碰的么……机缘?没有特定的梦道手段,能有什么机缘?”

“而那位正道蛊仙自称为罗,恐怕就是超级势力罗家之人了。”

南疆正道超级势力,共有十三家,分布于南疆各处。雄踞一方。其中罗家就是其中之一。

“我是不是,也可以靠着这个渠道,去穿透防御蛊阵,接触到梦境?”不可避免的,方源脑海中一道念头闪过。

他最后凝望了一眼眼前的大阵幻景,转身便走。

方源变化的猴子,在树丛中迅速跳跃,攀壁下山,不断跋涉。花费了大半天,他终于离开了原来的山峰。

就这样跋涉。攀过几座山峰后,方源这才撤销了变化,还原本来面目。

他生性稳妥,超级蛊阵附近,若是冒然变化,获知催动仙道杀招,都可能引来正道蛊仙的警觉和注意。

“这个距离,应该比较保险了。”方源心中暗暗估量。

他并不肯定,因为他对这座超级蛊阵,没有任何的认知和概念。

不过。就算被正道蛊仙发现自己的踪迹,也没有多大关系。

“我就不信,这座正道超级蛊阵建立之后,那么魔道、散仙们不会前来试探!”

方源所料丝毫不差。

义天山大战。像是一颗雷球,将南疆蛊仙界原本平静的,波澜不兴的湖泊,炸得面无全非,炸出惊涛骇浪。

义天山大战,原本就涉及到仙蛊屋惊鸿乱斗台。吸引了南疆各大势力,无数蛊仙的注意。

而后正魔蛊仙交锋,影响了整个南疆蛊仙界。

但最终,义天山上忽然爆发惊世大战,一切都毁灭,参与大战的蛊仙们全都阵亡。而义天山遗址上,却莫名其妙地笼罩了一片巨大的外显梦境。

这让南疆蛊仙们,想要去探查义天山的遗址都做不到。涉事的己方,不管是影宗、天庭或者方源,都默契地保持秘密,南疆的这些蛊仙还都是一头雾水。

自从正道优先出手,在此围绕着超级梦境,布置出了恢弘蛊阵之后,魔道和散仙们都很不甘心,很多人都会到蛊阵方面刺探情报。

正道蛊仙们屡次发现这些人的踪迹,但只要他们不闯进蛊阵中,他们也就睁一只眼闭一只眼了。

方源选择了一处普通的山洞,在这里耐心等候了数天。

他终于等到了焦雷子,离开防御蛊阵。

焦雷子的行径,并没有隐藏。离开的方向,也是他之前来时的路。

“焦雷子既然伪装成罗家的蛊仙,出走时,自然要明目张胆,否则的话,就显得心虚了。”方源目光紧紧盯住焦雷子,见他飞走,方源也立即远离,跟了上去。

两人前后飞行了一阵子,焦雷子忽然速度减慢下来,停驻在一座无名的小山之巅。

他回首看着方源:“阁下跟着我飞行了这么长时间,不知有什么事情?”

焦雷子并没有剑拔弩张的态势。

因为方源跟踪的时候,主动显示着形迹,让焦雷子发现。

又用态度蛊,表明自己的善意,让焦雷子感知得到。

焦雷子因此未有对方源恶声恶气,不过仍旧有着戒备。

“打扰了仙友,实在抱歉。”方源行了一礼,十分客气地道明了来意。

焦雷子听了方源的一番话,脸上神情缓和了下来。

原来,方源自称自己是山野散修,对梦境相当好奇。无意中听了传闻,说焦雷子有门道,可以贿赂正道蛊仙,让自己混进防御蛊阵中去,亲自接触梦境,尝试能不能触发什么机缘。

方源便在外守候良久,终于等到焦雷子出来,这才跟踪上去。

“原来这生意已经传播出去了。就连你这样的外人,都能轻松探听得到。”焦雷子并没有怀疑,自失一笑。

“跟我来吧,你想要进入梦境中撞撞仙缘。这是人之常情。不过,我也不是关键的中间人,我带你去见白兔姑娘。”焦雷子很是热情。

他虽然不认识方源,但是方源一身南疆蛊仙的气息,做不得假。

并且。南疆乃是五域当中,隐修最多的一域。这是南疆的风格,很多蛊仙从升仙到终老,都不为世人所知。

不到一天的路程,方源就在焦雷子的带领下,见到了白兔姑娘。

这位女蛊仙二八芳龄的外表模样,双眼如红宝石,可爱的圆脸,还有亮闪闪的大眼睛。嘴唇有些微微凸起,说话时语气极快。突突突不停地往外冒词。

和白兔姑娘见面的情景,让方源感到十分惊异。

因为这位白兔姑娘,直接将一座凡蛊屋,放在山峰中,不少魔道、散仙,都从这个蛊屋里进进出出。

“我这里专做梦境的生意,你是来对了!”

“你要想进入梦境,我有最可靠的门路,就看你能付出多大的代价。”

“付出的代价越大,自然你就能越快地进入梦境。说不定你就能早点遇到自己的仙缘!”白兔姑娘说到这里,大拇指翘起来,面对方源,反手指向身后的墙壁。

那里已经挂满了各种小牌子。牌子上写着各个人名。

“排号的又多了五位啊。”焦雷子看了,感慨不已。

“你要知道仙缘是有多重要!”白兔姑娘继续盯着方源说,列举了好几个历史上的实例,然后又补充道,“三尊说的预言,你肯定知晓的吧。梦境。这可是关乎到大梦仙尊的机缘!想想看,你如果成为未来的大梦仙尊,会怎样?就算只碰到一丝机缘,也绝对能改变你的现况啊。如果你不去做,不去尝试一下,怎么知道自己不可以?”

“是这样子的。就算知道自己不可以,将来也不会后悔了。”焦雷子在一旁附和着,“能够让正道蛊仙们这么宝贝的梦境,我们能够有这样的机会,是多么的难得!”

“是啊,现在是刚刚开始,所以正道方面管理不是特别严格。”白兔姑娘继续道,故意激将方源,“等过一段时间的话,就不一样了。这个生意做不长,机会就看你能不能把握得住了!”

“那要付出多大的代价?”方源装作犹豫的样子。

白兔姑娘报出了一个数字。

方源面现犹豫之色:“我再考虑考虑吧。”

“那行,你可以好好考虑一下!”白兔姑娘笑了笑,态度仍旧很是客气。

这时,一位女蛊仙走近柜台。白兔姑娘的笑容顿时热切了几分,舍弃了方源这边,迎接上去:“妙音姐姐,您是想来再撞一次机缘吗?”

那位女蛊仙点点头,沉声道:“我觉得,这一次换一个其他的地点,进入梦境,一定会有不一样的收获。”

凡蛊屋中蛊仙真的不少,来来往往。也并非白兔姑娘一人招待,时不时的就有牌子被挂在墙上去。

方源走出凡蛊屋,和焦雷子告别。

“你可以再考虑一下,不过依我个人之见,还是尽早决定的好。不能拖啊。这种情况,你也看到了,你越往决定,就越会有更多的蛊仙排在你的前面。时间不能拖的。”焦雷子临走前叮嘱着。

方源口中感谢不已,离开了此地。

“怎么,那个人还没有下决心?”白兔姑娘招呼完她口中的妙音姐姐,见到焦雷子问道。

“他说要考虑一下,唉,我还以为自己能赚一笔举荐费,好弥补一些损失呢。”焦雷子失落地苦笑道。

“焦雷子你身家丰厚,何必在我这里哭穷?”白兔姑娘笑了笑,“不过呢,我看你带来的这人,恐怕是山野中的隐修,本身穷酸,并无多少资粮。否则的话,怎可能立即走人?真有些身家的,第一次来,都要在这里逡巡流连一番呢。”

焦雷子苦笑:“我身家怎可能丰厚?承蒙白兔姑娘谬赞了。说实话,这要价真的不菲,我也感到相当肉痛的。”

白兔姑娘笑起来,露出白白的牙齿:“这是生意,讲究你情我愿。要价高一点,自然是因为它就值这个价嘛。事关大梦仙尊的机缘,你说呢?”

焦雷子点点头,忽然下定了决心,沉声道:“我再出一份资源,我要再试一次。”

\end{this_body}

