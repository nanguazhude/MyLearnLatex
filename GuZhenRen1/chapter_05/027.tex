\newsection{东海追逃战(下)}    %第二十七节:东海追逃战(下)

\begin{this_body}

周礼身边原本承载汤诵、刘青玉的波涛虚影,径直向前奔腾,朝着方源淹没过去。

方源心中顿生压抑之感。

波涛虽只是虚影,但他却感觉自己仿佛面对百丈高的海啸,自己茫然无助,宛若蝼蚁,要被这海浪一口侵吞!

“好厉害的仙道杀招!”

“七转蛊仙,果然不好糊弄。”

危在旦夕,方源的脑海中迅速闪过这两个念头。

没有波涛承载,三仙停下脚步。

“此事已成。”周礼笑吟吟地道,双眼中闪烁着自信的光辉。

汤诵、刘青玉对视一眼,已是猜到周礼的筹谋,不禁对此人刮目相看。

原来,周礼不动声色,暗中收集方源情报,渐渐摸清楚方源的底细,知道彼此长短,要用何种手段才能克制方源的长处。

表面上他是主动帮助二人飞行,实际上却是借助二人牵扯方源的机会,自己不断地酝酿脚下的波涛虚影。一路追赶下来,这波涛虚影规模越加壮大,渐渐酿成大势。

到此刻,周礼出手。

他不出手则已,一出手便是惊涛海啸,天翻地覆。夹裹大势,更是速度快过剑遁仙蛊,难以躲闪和抵挡!

方源有七转飞剑仙蛊,但对付波涛虚影,和对付泥怪、云兽一样,效果不佳。

他还有剑遁仙蛊,但波涛虚影漫天而来。已经将方源包了饺子。方源的躲闪空间,只会越来越小,剑遁仙蛊亦来不及脱离波涛虚影的围剿,难有良效。

“想不到这周礼表面上懦弱可欺,实际上却是胸有筹谋。我之前还对他大呼小叫。他居然一声不吭,此人城府比我还深!”刘青玉对周礼心生忌惮。

汤诵的面色也不好看,他辛苦酝酿仙道杀招,本想挽回颜面,但至始至终都没有机会,最后反倒是让周礼大大露脸一回。

六转蛊仙也就罢了,能成就七转的。哪一位能简单?

周礼一出手。方源便如瓮中之鳖。

“战局已定!”

三位七转蛊仙,均是相同的念头。

却冷不防方源发出大笑:“哈哈哈,三位仙友,何其不智?居然赶来送命。以我看来,那些六转蛊仙却比你们聪明得多。”

三仙望去,只见方源不闪不避,背负双手。悬停于空,好整以暇。

三仙浑身一震,心中同时叫道:“不好!他有恃无恐,难道这里就是埋伏之处不成?”

方源又道:“此招何名?居然是连变之招,不过……能否能否突破我族的仙道战场杀招呢?呵呵呵,我认为希望不大。”

明明这些惊涛虚影,攻势卓绝,能取其性命,但方源竟然置若罔闻,仿佛面对的不是致命杀机。而是和煦的春风。

他甚至将身上的防御杀招狮毛甲,当着三仙的面卸除了。

“要糟!我这招几乎耗用了我全部的仙蛊,攻强守弱,一旦对方发动攻击……”周礼瞳孔猛地一缩,来不及多想,下意识地就调动波涛虚影回来,维护自身。

惊涛虚影。本来叠影重重,已成必杀之势,但周礼调动大部分回转,只余下一小部分对付方源,顿时就形成破绽,让方源抓住。

“三位,不送!”方源狂催剑遁仙蛊,一飞冲天,穿过间隙,电闪雷鸣之间,就扎进苍水界壁之中。

三仙齐齐一愣,旋即恼羞成怒。

“此子好生狡诈,居然虚张声势!”

“他刚刚已经穷途末路,却掩饰而如此逼真。咱们快追!”

汤诵、刘青玉怒不可遏,再次向方源追去。

周礼脸上忽青忽白,楞了一愣,忽然出手打了自己一个巴掌,这才启程重新追赶。

但刚刚那么好的局面,都亲手葬送,接下来还能有什么好机会?

方源险死还生,进入界壁之中后,拼命地往界壁深处钻营。

三仙叫苦不迭。

修为的优势,在这里,却变成了阻碍。

这一刻,三仙都恨不得自己成为六转蛊仙。

越是界壁深处,他们受到的吸摄之力就越是强大,令他们速度暴降。

被逼无奈之下,三仙只好寄希望于身后陆续赶来的六转蛊仙们。

但这些六转蛊仙,也受到很大影响,和方源毫无阻塞,出入自由的情形相比,差距甚大。

很快,身后的蛊仙追兵,就离方源渐行渐远。

“这人到底什么修为?在界壁中,居然受到的影响如此之小?”东海蛊仙们纳闷。

殊不知,这还是方源故意收敛的结果。

方源不想将这个秘密,彻底暴露出来,所以故意降低速度,表现出一副艰难跋涉的形态,让人只是心生怀疑。

追赶不上方源,东海蛊仙们却不愿就这样放弃。

“我就不信,他就真的藏身于此,不出来了!”

“他是六转蛊仙,在这里飞遁,仙元耗费更多。我倒要看看,一位六转蛊仙的底蕴,能必得上我堂堂七转?”

尽管东海蛊仙们十分不甘,但事实却很残酷。

方源将他们甩得越来越远,甚至身后的上古云兽群都后来居上。

再造成一番混乱之后,东海蛊仙们只能满嘴的苦涩,看着上古云兽群超越他们,追向方源。

这些上古云兽,来自白天,五域界壁对其毫无影响。

渐渐的,方源已经脱离了东海蛊仙们的追踪范围。他们只能追着上古云兽。

上古云兽和方源逐渐拉近距离。

见时机成熟了,方源就再次催动剑遁仙蛊。速度激增,扬长而去。

东海蛊仙们追赶片刻,终于有人忍受不住,选择放弃,挣脱界壁出来。

“真是倒霉!竟然碰到这种人物!”

“此人是谁。我等都毫无所知,他身后究竟有没有超级势力撑腰,恐怕还得打一个问号。”

汤诵和周礼交流,他们也感到希望渺茫。

唯有刘青玉一声不吭,始终奋力直追。

“刘兄,不要再追了。”

“他已经不见踪影,此事凭添许多麻烦。咱们还是先出去。一起商量一下吧。”

汤诵、周礼劝道。

刘青玉却道:“我有手段,擅长追踪。不追到他,我绝不善罢甘休!二位稍待,我再尝试一次,去去就来。”

说着,他身化一道青绿长虹,速度竟然再度暴涨一大截。向着上古云兽群追赶过去。

汤诵望着这道遁光,诧异道:“看来刘兄是急眼了,居然连这招都用出来。此招可是要耗费他身上的道痕,才能施展。代价高昂,是他压箱底的手段,很少使用。曾经他以此,摆脱过不少强敌。”

“你说什么?”周礼的脸色顿时一变。

汤诵楞了一下,脸色也跟着一沉。

二仙对视一眼,均看到彼此眼中的怒意。

这刘青玉恐怕不是追杀方源去了,而是自己暗中得了传承印记。借此摆脱他们!

毕竟,方源击溃血道魔仙送出来的传承印记,大家都是亲眼目睹。

而斩杀血道魔仙的人,就是刘青玉。

“刘兄慢走。”

“贼子狡诈,恐有帮手,我们二人来助你一臂之力,刘兄!”

汤诵、周礼连忙追赶。

刘青玉听到这话。跑得更急了,连回头的动作都没有,仿佛没有听到一样。

汤诵、周礼更加坚定自身的推算,脸色更沉一分,心中暗暗发誓,不追上刘青玉誓不罢休。

大半个月后。

一道身影,穿透青绿如玉的甘草界壁,正式进入北原。

“终于来到北原了。”来者正是方源,他满脸疲惫之色,浑身伤痕累累。

能自己治疗的伤,他早就治愈了。

但大部分的伤势,是由上古云兽、仙蛊甚至是仙道杀招造成的,这些伤势都有道痕附着,至少得用治疗仙蛊,才有治愈的可能。

方源手中只有三只仙蛊,只能拖着。

“琅琊地灵那边,应当有治疗仙蛊,只要回归琅琊福地,就能得到休养的机会了。”

方源暗暗为自己鼓劲。

之前他和东海蛊仙们一战,这些天来,为了摆脱上古云兽,又屡次动用剑遁仙蛊。现在他已经欠下琅琊地灵一屁股的债。

关键还是东海蛊仙,因为他们的追赶,导致方源选择在界壁中穿行。

如此一来,就绕了很长一段路,消耗仙元极多。

但这些都不重要。

重要的是接下来的仙窍灾劫!

时间越来越短,方源的不妙预感,也越发浓郁。

转头回望,上古云兽群的身影,已经在界壁中若隐若现。方源叹息一声,再次催动仙蛊剑遁,向着东北方向飞去。

具体的位置,方源也不知晓。

之前,琅琊福地是在北原的月牙湖,但因为影宗突袭战,琅琊福地便从月牙湖搬迁了。

和琅琊地灵联络之后,琅琊地灵没有告知方源,琅琊福地的具体位置,只给他一个方向,并说到时候会有蛊仙前来接引。

三天之后,方源来到接应的位置,却并未等到人来。

上古云兽追上,他只好再次奔逃。

和琅琊地灵联络之后,他这才得知,原来琅琊地灵派遣了毛民蛊仙,但居然在途中遭遇人族蛊仙,被当场击杀了。

琅琊地灵只好再派人接应,结果又出现意外。

这位毛民蛊仙居然莫名其妙的失踪了,怎么也联系不到他。

为了接应方源,琅琊派居然折损了两位毛民蛊仙,并且还未真正将方源接应回来。

琅琊地灵心中滴血,难以承受这样的损失,索性将关键地点,直接告知了方源。

七天后,方源赶到风伯崖。

在风伯崖底,他找到了琅琊地灵的布置。

这是一处仙道蛊阵,作用类似于僵盟的绿晶华英道。方源借此,这才真正摆脱了上古云兽群的追杀,回到了琅琊福地。

几经周折,他终于安全了!

ps:今日3更,能力有限,时间有限,一天天的补。经历越多,越发现身不由己。很多事情,心意是有的,但往往受困于现实的种种方面,有心而无力。见笑,心生感慨,就当是牢骚了,大家勿怪。(未完待续。)

\end{this_body}

