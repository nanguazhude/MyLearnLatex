\newsection{赤心行走榜新主}    %第八百七十一节:赤心行走榜新主

\begin{this_body}



%1
张阴一出手,八转的份量放在这里,立即让许多竞争者偃旗息鼓。

%2
别说争不过八转蛊仙,就算争得过,也得罪了这位八转散仙。

%3
有时候,散仙比正道八转还要更加可怕,因为他们无拘无束,自由自在,没有正道势力那么多的顾忌。

%4
至于张阴为什么对心血仙蛊感兴趣,那是人家自己的事情。在场的蛊仙也不好多问。

%5
君神光作为天庭的代表,看到这里,更加忧心忡忡。

%6
气海老祖主持的这场大宴,公然交易买卖仙蛊,交换各种仙材,互通有无,满足了许许多多东海蛊仙的修行需求。

%7
这场大宴之后,东海蛊仙们有了充沛的修行资源,必定实力倍增。

%8
这对东海蛊仙界是一件大好事,但对天庭而言,却不那么美妙了。

%9
尤其是刚刚气海老祖还宣布,这样的气海大宴每一年都要举办一场!这个频率实在有点高啊,令君神光都腹诽不已:你说你气海老祖是不是闲的没有事情做,每年都要举办一场大宴,你累不累啊。和宝黄天、各个拍卖会不同,你又完全不收手续费用,你到底图个什么?

%10
“看来气海老祖并非毫无野心,他这是在养望啊。借着气海大宴,每年提升巩固自己的声望。将来不管是加入天庭,还是留在东海,都是好的。”

%11
“听闻气海老祖似乎还有不少后辈,将来这些人行走东海,顶着他气海老祖的名头,定然是受到各地欢迎的。”

%12
君神光成为天庭八转,自然是有韬略的,逐渐品出了一些味道。

%13
就在这时,从天庭那边传来一道讯息。

%14
紫薇仙子告知君神光,让他配合其他蛊仙,尽量不着痕迹地帮助别人将心血仙蛊收购下来。

%15
君神光皱眉。

%16
他倒不是疑惑天庭怎么需要这只血道仙蛊,事实上,天庭也一直在研究血道,并且卓有成效得很。

%17
君神光只是感到为难,暗中回讯:“现在情况又出现了变化,就在刚刚,张阴主动出手,开出高价来竞争心血仙蛊。”

%18
天庭中的紫薇仙子迟疑了一下,安排道:“那你就作壁上观罢,我来安排其他的人。”

%19
君神光身为天庭代表,若是公然和张阴竞争一只血道仙蛊,对于天庭名誉会有很大的影响。

%20
紫薇仙子原来的计划中,也只是安排他助攻。

%21
天庭十分重视这场气海大宴,参与进来的人员中,既有君神光这样明面上的人物,八转修为,也有暗子,修为六转或者七转。

%22
“哈哈,真是巧了,这只心血仙蛊,我也感兴趣呢。”阳骏忽然出声。

%23
就在张阴以为大局已定的时候,这位东海的八转散仙也加入的竞争当中。

%24
这个变故让场中群仙微楞,就连方源的双眸中也不免闪现一抹异色。

%25
阳骏在方源上一世,就参与过争夺龙宫。

%26
他有一座仙蛊屋名为宝马香车,在追赶龙宫的途中,勘破了方源的踪影。让方源黄雀之梦落空。

%27
后来,阳骏忽然相助龙公,令龙公成功镇压住了八转仙蛊屋的反抗,将其收取。

%28
做成这个事情后,阳骏立即隐匿,消失得无影无踪。

%29
等到宿命蛊修复大战,阳骏也没有出现,再帮助天庭。

%30
大战期间,天庭一度困窘,绝无留手的可能。但阳骏始终没有现身,这就证明他并非真正的天庭中人,很可能只是和天庭达成了某种约定,必须出手相帮一次。

%31
“那么这一次,是否是他在为天庭出手呢?”方源心中念头一闪而过。

%32
这个就不好推断了。

%33
饶是方源智道造诣深厚,也因缺乏关键线索,难以判断。

%34
不管有没有这种情况,方源暗中命令张阴,让他强势一些,尽力将这只仙蛊争取下来。

%35
于是,张阴笑道:“阳骏仙友,实不相瞒,这只仙蛊我非常中意。阁下若要坚持,不妨我们稍稍切磋一次,如何?”

%36
阳骏微微皱眉,暗想:“张阴不惜动手,也要拿下这只仙蛊么?气海老祖能一招败他,是因为气海老祖实在太强!我和他若是交手,恐怕不分胜负。”

%37
“唉,我是坚持到底,还是帮助天庭?”

%38
阳骏不禁陷入犹豫。

%39
他当年落魄的时候,得到天庭的帮衬,欠下天庭一个出手相助的约定。

%40
这个人情债实在太大太重,渐渐成了阳骏一个心病。

%41
阳骏想了想,还是决定还了天庭的人情债,履行当初的盟约。

%42
“这是一个好机会。若是这里不出手,将来五域乱战,天庭让我对付东海蛊仙,我也不能反悔,只能照办。”

%43
“幸好我对溺水翁的情况,有些了解。”

%44
想到这里,阳骏笑了一笑:“张阴仙友,还请稍安勿躁。你我切磋并非关键,这只心血仙蛊乃是溺水翁之物,我相信我拿出来的这只水道仙蛊,必能被他看中。”

%45
说着,阳骏便取出一只仙蛊来。

%46
溺水翁见了果然大喜过望,低呼道:“换了,换了,我就要这只仙蛊。”

%47
张阴微楞。

%48
这种情况如何是好?众目睽睽之下,他根本无法抢夺,必定会引来气海老祖出手。

%49
虽然他和本体是沆瀣一气,但现在是人前做戏,各有立场。

%50
做戏自然要做到位。

%51
张阴不好出手,本体气海老祖也不能。

%52
就在这时,阳骏主动离席,一手抓过心血仙蛊,又将那只水道仙蛊抛给对方。

%53
众仙皆露异色。

%54
阳骏动作极快,赫然是动用了仙道杀招,这也未免太心急了点吧?

%55
阳骏却如释重负地吐出一口浊气,重新坐下来,当即将心血仙蛊塞到宝黄天中去了。

%56
买卖仙蛊在宝黄天,自然费用高昂,但天庭家大业大,根本不会计较这些。

%57
紫薇仙子得了心血仙蛊,立即又转给了另外一人。

%58
这人常年在诛魔榜中修行,得了这只心血仙蛊,欢喜长啸:“我道成也!”

%59
他跃出诛魔榜,顿时方圆百里血光四溢,芳香扑鼻,云霄响起仙音,一朵朵八瓣玲珑血花漫天飞舞。

%60
他身姿挺拔,面容英俊,嘴角含笑,头上缠着一道头箍,脑后闪烁着一团血色光晕。

%61
空气中像是有一道无形的阶梯,让他一步步安然踏下。

%62
古月方正正在诛魔榜前修行,见到这番景象,连忙拜倒在地:“方正拜见师尊!”

%63
“起来吧,我徒。为师得到一只关键仙蛊,生平构思出来血道之法,就有了核心和基石。我在这个诛魔榜中闭关了数百载,是我出关的时候了。”

%64
这位八转蛊仙语气温和,微微伸手,凭空一股力量生出,就将古月方正扶起来。

%65
方正抱拳再拜:“徒儿恭喜师父。”

%66
他自从东海一行,被沈从声俘虏后,又被天庭赎回。

%67
秦鼎菱观测他的运气,向紫薇仙子建议,让他成为下一代的诛魔榜榜主。

%68
紫薇仙子和龙公商讨之后,同意下来。

%69
方正便来到诛魔榜修行,拜当代诛魔榜榜主为师。

%70
诛魔榜乃是八转仙蛊屋,历史相当悠久。血海老祖祸乱五域时,天庭感到危险,就有天庭先贤按照宿命蛊的启示,寻找到了野生的血缘仙蛊。

%71
血缘仙蛊高达八转,被天庭俘获后,增添到诛魔榜中。这导致天底下任何一位血道修行者,只要不达到九转,都会被诛魔榜感应。

%72
天庭依靠诛魔榜上的准确信息,施展严酷手段,将中洲的血道消灭在萌芽之中。并且又将诛魔榜上的消息有选择地告知其他四域,令其他四域正道镇压血道效率大涨。

%73
血道一直都没有真正昌盛起来,被死死打压,诛魔榜可谓居功至伟。

%74
血道被血海老祖开创之后,天庭一直都在钻研,比全天下任何一个超级势力的成果,都要更高。

%75
当代诛魔榜榜主赤心行者,便是专修血道,一直在诛魔榜中闭关,苦心钻研,研究的成果极其丰硕,令天庭走在五域血道的最前列。

%76
灵光一闪,九灵仙姑忽然出现,来到赤心行者面前:“赤心道友,我领命前来和你汇合,接下来我们要前往西漠,寻找房家的麻烦。”

%77
赤心行者微微点头,临走前叮嘱古月方正:“诛魔榜不可一日无主。方正我徒,我走后你便要接任诛魔榜榜主。我的传承就留在榜中,不仅如此,榜里历代的诛魔榜榜主都有多多少少的修行心得。今后你就在诛魔榜中修行,对你会大有裨益,盼你坚持不辍,万勿懈怠,辜负了这番机缘。”

%78
“是,师父,徒儿谨遵师父教诲,一刻都不敢忘怀。”古月方正连忙回答。

%79
“嗯。”赤心行者点点头,转身离开。

%80
他和古月方正的师徒之情很浅薄,若非紫薇仙子下令,他是不会收古月方正为徒的。

%81
这层师徒关系,也带给他巨大的帮助。

%82
“运道果然玄奇。之前,秦鼎菱前辈找到我时,劝我说:若我将方正收为徒弟,成就师徒之实,我便能借助他的气运,来增益自己的修行。”

%83
“没想到竟这么快,就有了如此重大的成果。临走之前,我想拜见一下秦鼎菱前辈。”赤心行者满怀感慨道。

%84
九灵仙姑点点头。

%85
这点时间还是不缺的。

%86
秦鼎菱回归天庭,有专门一处仙蛊宫殿作为居所。

%87
两人拜见了秦鼎菱,后者为他俩观运,便见九灵仙姑的气运,呈现九道灵光,清澈动人,仿佛暴雨后的青叶,祸消福来,一派欣欣向荣的气息。

%88
又见赤心行者的气运,当空一枚太阳,又仿佛血丹,饱满圆润,毫无缺憾,正从云中越出,磅礴浩荡,气象雄阔。这正映照着赤心行者闭关多年,一朝出关,行走天下,将绽放光芒,名传五域。

%89
秦鼎菱微笑着,告诉两人:此行当会顺利,只要按部就班,不会有什么波折。

%90
二仙拜别秦鼎菱,空中疾飞,去往中天门,途中经过绣楼,看到下方一位蛊仙痴痴呆呆地驻足楼前,一动不动。

%91
九灵仙姑将赤心行者目光投下去,叹息道:“这位便是凤九歌,光阴长河战败后,他便这番颓废模样。”

%92
赤心行者摇头,面色凝重:“他正在悟道,我在榜中闭关时,偶尔也陷入他这样的状态,不可打扰他。”

%93
九灵仙姑面露惊异之色:“哦?难怪紫薇仙子放任他不管。忏愧,我继承九灵变的传承,一直都是按部就班修行,从未有过改良,因此不知真相。”

%94
赤心行者笑着安慰:“仙姑的九灵变,乃是我天庭最顶尖的变化道传承,经过历代先贤修整,恐怕早就近乎改无可改的地步了。”

%95
他虽然修行血道,但心性极佳,待人温和,如春风拂面。

%96
九灵仙姑笑道:“当年师父传授我真传时,就说过我资质不成。对于你们这等天才,我也就只有羡慕的份了。”

%97
二仙这般谈笑着,出了中天门,进入白天,径直飞往西漠去了。

\end{this_body}

