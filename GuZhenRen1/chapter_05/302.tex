\newsection{正道的束缚}    %第三百零二节:正道的束缚

\begin{this_body}

南疆,义天山遗址,超级蛊阵中。

“武安(武辽),参见武遗海太上长老。”超级蛊阵中,武家蛊仙武辽、武安,一齐向方源恭敬行礼。

方源点点头:“你们都坐,说说吧,到底是怎么回事?武碑长老居然会受此重伤。”

武辽、武安相互对望一眼,武辽闭口,由擅长言辞的武安禀告道:“启禀大人,是这么一回事儿……”

原来,巴家早就对武家蠢蠢欲动,武家受到多方刁难之后,巴家在这里的蛊仙首领树翁巴德,觉得时机已到,向武家发难。

巴德早已经筹谋良久,不动则已,一动惊人,让武家处于相当被动的局面。

武家自然不能坐以待毙,武碑出马,却被巴德设计,不得不和巴德展开一场切磋较量。

较量的结果,当然是巴家获胜,武家失败。

“只是我们也没有想到,武碑大人的伤势会这么重。切磋的时候,分出了胜负,当时武碑大人虽然战败,却仍旧风度慑人。”

“是啊,虽然败给了巴德,但武碑大人当晚还和乔家等其他家族的蛊仙,进行了一场秘密会谈。我们都以为他的伤势并不要紧。”

武安说完之后,武辽补充了几句。

方源心中暗笑,表面上则关照道:“情况我已经了解了,武碑既去,由我替代,一切依照我之前,敌不动我不动。”

“是,大人。”武辽和武安立即领命应和。

对于这个命令,他们俩都不意外。

在外人看来,武遗海的战力怎可能和巴德相比?

虽然武遗海最近是出了一些风头,但是他战胜夏飞快,只是一场切磋,并且规矩是由他定的。

他虽然平息了螺母山的纷争,但整个过程并未动武,而是最后和驱山老怪谈妥。

最后和乔丝柳的绯闻,更是带着桃色的气息,不会让人敬重。

“没有事的话,你们就退下吧。我一路急赶过来,有些乏了。”方源挥手。

武辽当即告退,但武安却犹豫,小声地道:“大人,有点事情小的需要禀告……”

“那你就说说。”方源心中已经知道武安想要禀告什么。

果然,武安接下来的话,不出方源意料,正是关于武家领袖其他家族,利用梦境做散仙的生意的事情。

“现在风声很紧,其他家族方面也惴惴不安,我们是不是暂时将这个买卖停下来呢?”武安担忧地问道。

方源沉思了一下:“不需要担心,我方才已经说了,一切照旧。没有其他事情,你就退下吧。”

“是,大人。”武安脸上闪过一丝喜色,躬身而退。

只要这个梦境的买卖依旧进行,武安就能从中谋取私利。他岂会不欢喜?

就算事情变坏,生意被巴家揭露,但武安身上的责任依旧变得很少了。因为他的顶头上司方源曾经开过口,让生意照旧。

方源对于武安的小心思,自然了如指掌。

他心中冷笑:“这个武安鼠目寸光。”

诚然此时,武家处境不太妙,但毕竟仍旧是南疆的第一家族。

巴德的确有脑子,他虽然动手了,但是没有在这个梦境生意上下手。这点很有分寸,因为这块利益不是武家独享,还有其他家族。

若是在这个方面下手,反而会激发其他家族同仇敌忾,和武家联合在一起。

他没有抓住这个把柄发难,正是他对整个南疆局势的清晰洞察。这个把柄虽然重大,但此时打出去,反而效果不佳。若是将来某一天,武家的名望真的大大下跌,那时候再发难,必定是墙倒众人推了。

巴德看准这点,方源亦同样如此。

世事如棋,上佳的棋手,向来谋定后动,知道什么时候运用什么棋路、什么棋子,才是最有效的。

所以,梦境生意只管继续做,真正决定巴德动手与否的,不在于这个生意本身,而在于武家!

世间的事情往往就是这样奇妙。

很多事情,决定的因素,不是事情本身,而是事情之外的东西。

念及于此,方源下意识地看向武家的方向。

武庸,纵然是八转蛊仙,但仍旧是着了方源的算计。

因为这不是修为、战力的比拼,而是经验和手段。

方源之所以主动接近乔丝柳,目的当然不是乔丝柳本身,而是她之外的东西,那就是回到超级梦境!

武庸可以容忍武遗海的贪婪,但他却忌惮武遗海和自己争权。

这是当然的事情。

种子早就在武独秀辞世之前,就已经埋下。武独秀临死之前,曾经留下遗言,要将身上的仙蛊交给武遗海。

武遗海血脉出身,已经得到承认,一旦把握住了部分权力,绝对是武庸施政方面的巨大威胁。

正如武独秀当权的时候,忌惮乔家,防备乔家一样,在武家内部,武庸要防备的,也就是两个。一个是乔家,一个是武遗海。

现在这两个凑得很近,似乎要走到一块去,能不让武庸警惕吗?

之前,武遗海加入武家,乔家启了武家的内应,竟然是武家的太上三长老。不少明白人心底早已经被吓出一层冷汗,武庸能不忌惮么?

现在武遗海和乔家似乎要联合起来,武庸能不提前下手吗?

武庸下手,也非常厉害。

他没有直接询问方源,问他还想不想回到超级梦境那边去?

他给方源的选择有三个,一个是玄冥山,一个是赤龙江,一个是翼家寿宴。

这三个选择的背后,自然藏有深意。

第一项玄冥山必然是争斗最为激烈的一项任务,事关野生仙蛊,定然要发生摩擦。方源若选择这个,是心向武家,不惧危险,武庸将来大可用之。

第三项翼家寿宴,则是一个试探。翼家和东海关系紧密,方源若是选择这项,说明他内心深处还是对东海有所留念的。

第二项预防赤龙江洪灾,就是一个陷阱。

武家、乔家都有地盘和赤龙江接壤,武家要预防洪灾,乔家当然也要如此做。

这样一来,就有了和乔家蛊仙合作的机会。

方源选择这项任务,顿时让武庸冷了心,哪怕他当时谈笑宴宴,其实心中早已经下定了决心必须尽快把武遗海调走!不能再给他和乔家接触的时间了。

乔丝柳乃是南疆三大仙子之一,美貌惊人,勾人心魄。武遗海就算一时没有动心,时间一长呢?

英雄难过美人关,更何况是乔丝柳这样的仙子!

更何况除了美色之外,还有泼天的利益。

一旦武遗海和乔丝柳结合,对于他本人亦或者乔家,都是大为有利的事情。

武遗海能够借助乔家这等外戚,极大地增长在武家内部的话语权。而乔家也终于突破历史,打破曾经的极限,有了武遗海这样的女婿,乔家的这条藤蔓,就真正的插进武家这棵大树的内心深处了。

武庸怎可能犯这样的错误?

而要隔绝两方的联系,直接强行否决拆散,是不可行的,搞不好还会适得其反。

并且正道行事,都得按照规矩来。可不能像魔道那般随心所欲。

武庸思考一番后,就发现很难处理武遗海。尽管武遗海有贪污的把柄在他手中,但武遗海到底他的亲弟弟。

思前虑后一番,武遗海想到超级梦境。

他可以借助这片梦境,将武遗海尽可能地长时间“流放”,一如之前武遗海待在蛊阵中,不都是安安静静的么?

之前的成功,让武庸终于下定了这个决心。

当他下定决心之后,便立即着手实施。虽然他给了方源三个选择,但实际上,从头到尾,他都没有让方源有过选择的权利,充其量也只是试探而已。

“正道就是如此啊。”

“不成为最终的上位者,都不会有自由。”

“但就算成为了上位者,整个组织反而成了束缚的枷锁。”

方源心中感叹,很快,他便收拾情怀,开始着眼于当前的梦境。

又回到了熟悉的地方,在这里,还有属于他的两只仙蛊。

梦境却始终在膨胀,在流转不休。

“嗯?这片梦境不错,是真实记忆和经历形成的梦境,而不是荒诞怪异的想象。”

很快,方源就考察好,投入到其中一片闪烁着明亮蓝光的梦境之中去。

视野大变。

一个缓缓的山丘上,一位中年蛊仙背对着方源:“我儿,你知道我们图家寨,以什么称霸这片山脉的吗?”

方源检查了一下自己,他发现自己已经变成了一个孩童。

“父亲,我不知道。”他想了想,应付道。

“阵道!”中年蛊仙忽的声调高扬,带着强烈的自豪的情绪。

“阵道?”方源口中呢喃。

“没有错!就是阵道。”中年蛊仙转过身来,露出他沧桑的面庞,坚定的双眸,一身强烈的上位者的气质流露无疑。

“纵观诸多流派,阵道可能就是最复杂的流派了。我儿,从今天起,为父将传授你阵道的精意,指导你修行。”中年蛊仙目光热切。

“是,父亲,我一定好好学。”方源应付道。

中年蛊仙却摇摇头:“不只是好好学,你一定要学成,努力成为最好,将所有的同龄人,甚至那些青年蛊师都比下去。因为你是我图事成的儿子,因为你将是未来的山寨之主!”

\end{this_body}

