\newsection{烽火四起}    %第九百零六节:烽火四起

\begin{this_body}

%1
中洲炼蛊大会举办得如火如荼,方源一伙儿却已经踪影全无。

%2
方源下令撤退是如此的彻底,以至于让五域蛊仙都有一种错觉,仿佛之前方源大举破坏屠戮,只是一个假象。

%3
“方源究竟在做什么?”

%4
“中洲炼蛊大会还在举办呢,你倒是出手啊!”

%5
“你可是小魔尊,别弱了你的名头。”

%6
四域超级势力腹诽不已,都很着急,盼着方源出手,好为他们打头阵。

%7
但一直和天庭势如水火的方源,却出人意料地龟缩起来,没有一丝动静。

%8
四域势力都快产生了怨念,眼睁睁地看着日子一天天过去,中洲炼蛊大会进行层层选拔,各路炼道精英一记开始更高一层的竞技。

%9
中洲的各大比试场地的数量,也随之大大缩减。在每一个场地,天庭都安置了数位中洲蛊仙严防死守。

%10
就在这样的情况下,方源命令再次下达,白凝冰、冰媛以及兽灾洞天中的各个蛊仙接连出动。

%11
中洲,玄武山脉。

%12
“白凝冰你为虎作伥,终会得到报应的!”驻守在这里的是一位灵缘斋的女仙,临走前她抛下狠话。

%13
白凝冰冷哼一声:“逃得倒是挺快。”

%14
玄武山脉乃是灵缘斋的一处超级资源点,覆盖范围十分广大,是众所周知的仙材宝库。

%15
“未免夜长梦多,我得赶快行动了。”白凝冰开始搜刮。

%16
中洲,囚谷。

%17
“就是这里了。”钢奔勇士左右观望。

%18
和他同行的是霜蝶勇士:“开!”

%19
杀招催出,顿时一道大阵在他们的面前显现而出。

%20
变身!

%21
两位来自兽灾洞天的七转蛊仙,没有任何犹豫,直接使出最擅长的手段。一个变作巨人,牛头人身,一位变作蝶衣小人,周围寒气四溢。

%22
一记记仙道杀招催出,大阵支撑了片刻后,轰然崩溃。

%23
“南华荆,整个山谷都是仙材啊。”钢奔勇士惊喜不已。

%24
霜蝶勇士则更加干脆,直接出手,将南华荆冻住,然后疯狂采摘。

%25
中洲,红河滩地。

%26
赶赴而来的中洲蛊仙气急败坏。

%27
他们来得晚了,只看到一片废墟。

%28
滩地中囤积的大量赤火石一颗都没有剩下。

%29
实在太干净了!

%30
方源不出手则已,一出手再次撼动天下。

%31
他的麾下蛊仙,几乎尽数派遣出去,就连毛民蛊仙都不例外。

%32
这些人得到方源的命令,并没有强攻防守严密的炼蛊大会的比试场地,而是向那些防御薄弱的资源点出手。

%33
每攻破一处资源点,这伙人就进行惨无人道的劫掠。劫掠了之后,还动用杀招,进行毁灭性质的破坏!

%34
一时间,整个中洲都燃起烽火,各大资源点遭到抢劫和毁灭。

%35
方源每成功一处,就将那一处被攻占的影响,都公布在宝黄天中。

%36
五域蛊仙尽皆哗然。

%37
“方源的手下怎么这么多!”

%38
“这么多的蛊仙,已经远超一个正常的超级势力了。”

%39
“他究竟藏有多少的实力!?”

%40
蛊仙们震惊之后,又涌起嫉妒和羡慕。

%41
方源的这些下属战斗力参差不齐,但哪怕最弱的蛊仙也能得手。中洲为了炼蛊大会,这些资源点的防御太过薄弱了。

%42
许多实力强劲的蛊仙,不由地想到:“这些异人蛊仙都能抢夺成功,为什么我就不能呢?”

%43
更多实力低微的蛊仙也浮想联翩:“骨头和肉都让那些强者吃去了,我总可以喝点汤吧?”

%44
于是乎,和上一世的情况相同,四域的蛊仙再也忍耐不住,尤其是那些魔仙和散修。

%45
这些人率先出手,向各个中洲资源点下手。

%46
中洲局面彻底崩坏。

%47
天庭。

%48
紫薇仙子脸色铁青。

%49
她就算涵养再深,看着各地资源点被攻击,被洗劫一空的情报,像是雪花般飘来,也得激愤。

%50
“方源……”她磨着牙,不由散发出浓郁的杀机。

%51
正是这个罪魁祸首,才引发了人心变幻,使得中洲沦为其他四域的猎物,被恣意劫掠,损失十分惨重。

%52
但是不管紫薇仙子、龙公、秦鼎菱都没有办法。

%53
中洲太大了,并且处于最中央的地方,被其他四域包围。

%54
而蛊仙太少,根本无法一一分配到数量庞大的资源点上去,缺口太大了。

%55
这就给了劫匪和宵小可趁之机。

%56
“人心不古啊。”秦鼎菱叹息。

%57
龙公一脸淡漠:“为了大业,牺牲是不可避免的。一切都以炼蛊大会为主,区区仙材舍了也罢。只要此蛊修复,将来他们都会百千倍地吐回来。”

%58
中洲处于被动挨打的状态。

%59
方源的时机掐得太好了,不远不近就是这个点。天庭方面不得不对炼蛊的场地加倍防御,这些都是精英,损失多了,就会令中洲的战争潜力打上折扣。放在眼前,也会对宿命蛊的修复造成不小的影响。

%60
方源选取的目标也很巧妙。

%61
比如玄武山脉这种地形太过宽广,天生就防御薄弱。还有囚谷、红河滩地这些地方,在方源的上一世就被劫掠过,方源深知这些地方的虚实。

%62
再加上精心的计划安排,导致方源的下属成功率很高,劫掠如此轻易,收获又如此丰厚,这极大地刺激到了四域蛊仙。

%63
四域的蛊仙们纷纷出动,对各处资源点下手。

%64
中洲方面并非完全放弃了资源点的防守,事实上很多资源重地都有强者守护。

%65
一时间,大大小小的战斗爆发,蛊仙们迅速交手,一旦发现这个资源点防御强大,往往进攻的一方就明智地放弃,选择攻击另一处。

%66
反正中洲的资源点这么多,何必在一处死磕呢?

%67
西漠土道蛊仙石敢当动手,捣毁地气石林,夺取大批圆坤石。

%68
忘道人成功盗取灵浒泉泉水,哪怕这个资源点有古魂门的强者杨峰亲自镇守。

%69
萧虎痴、萧十让、皮水寒联手劫掠成功,但分赃不均,相互拼斗起来。

%70
中洲一片烽火狼烟,动乱不堪。

%71
紫薇仙子守在天庭中,简直是度日如年。

%72
终于,正元老人开口道:“人意已经初步积蓄成功了,可以施展第一个人道杀招。”

%73
紫薇仙子大喜,连忙请老人出手。

%74
正元老人点点头,颤颤巍巍地站起身来,随后一股磅礴浩荡的气势,从他身上陡然爆发!

%75
仙道杀招——万众一心!

%76
积蓄的人意迅速消耗,很快就所剩无几。

%77
与此同时,整个中洲的蛊仙、蛊师、凡人身上,都泛起一层光晕来。

%78
光晕一闪即逝,仿佛是一个幻象。

%79
但中洲中人很快都发现了奇妙之处。

%80
他们的心意彼此沟通,相互之间能探查到最真实的倾向和想法。

%81
被袭击的恐慌,亲朋好友死亡所带来的痛苦和仇恨,对身家性命的担忧,对未来前途的期盼……

%82
从未有这么一刻,中洲人的心紧紧地联系在一起。

%83
潜伏在炼蛊场地里的自在书生被发现,但他仍旧成功地报了仇,斩杀了镇守的中洲蛊仙。

%84
南疆隐仙郑青也暴露了身份,但他对中洲和天庭毫无敌意,被礼送出去。

%85
“好一个人道杀招万众一心,果然是奇妙。”沈伤利用宝黄天,和方源交流。他的具体位置暂且不明,方源只知道他已经离开了东海,或许此刻就在中洲某地。

%86
“你参悟出了什么?我们能否克制这些人道杀招呢?”方源询问,他早就将记忆中的其他人道杀招的情报,也告诉了沈伤。

%87
沈伤叹息:“要解决这些杀招,最捷径的法子就是摧毁人意。没有了人意作为基础,人道杀招也就无从施展了。但我推算出这些人意,应当都保存在天庭之中。”

%88
“那么除了这个法子呢?”方源又问。

%89
沈伤再叹:“那就只有正面破解杀招,这个难度很高,我需要大量的时间。放心,我既然答应了你,必定会竭尽全力。我也绝不想看到天庭修复好宿命蛊啊。”

%90
方源继续耐心等候。

%91
武庸率领着南联诸仙,忽然出现在毛脚山前。

%92
“跟紧我,让我们捣毁不败福地,夺走所有的成功道痕,让天庭功败垂成!”武庸一马当先。

%93
“我等必紧随盟主大人左右!!”南联诸仙齐声应和,士气冲霄,紧随其后。

%94
天庭方面立即收到情报。

%95
紫薇仙子面沉如水,眼蕴寒芒:“好个武庸。”

%96
“武庸动手了。长生天估计也快了,那么我也开始亲自动手吧。”接到这个情报后,方源笑了笑,在南疆义天山附近现身。

\end{this_body}

