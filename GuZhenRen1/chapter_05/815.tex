\newsection{方源战龙公}    %第八百一十九节:方源战龙公

\begin{this_body}

%1
“竟然是龙公大人!”方正见到龙公,又惊又喜。

%2
他是知道龙公的身份,事实上,他和龙公也曾照过面。

%3
“若非龙公大人出现,我此刻已经是死了。有龙公大人在,我应当是安全的。”方正再不心惊胆战,感到一阵阵的安心。

%4
他知道龙公是何等人物,这位可是红莲魔尊的师父!

%5
方正再定睛去瞧,打量那位险些要了他性命的人——气海老祖。

%6
只见这位老者一身白袍,大袖飘飘,一大把白胡子垂至脚面,他面容肃穆,气度浩荡,身边云气飘飘,又给他增添几许仙意。

%7
方正当然辨认不出这是方源假扮,不禁好奇这位老者的身份:“这老头也是八转修为,但却和东海蛊仙中的八转大能对不上号。他究竟是谁?”

%8
不只是他好奇,龙公也很好奇。不过东海广博,资源又是五域第一,潜藏着一位八转人物也很正常。

%9
当即,龙公笑道:“气海老祖……你身为八转,称宗作祖,倒是可以。但在我面前,称作气海,口气着实不小呢。”

%10
方正闻言,心中一动:“对了,龙公大人是先修气道,再修变化道,乃是罕见的双绝!这位东海八转老头儿,也是修行气道,却犯在龙公大人手上。啧啧。”

%11
方正不仅为气海老祖感到默哀。

%12
这时,他又听到龙公声音一沉,喝问道:“说吧,你专门守候在此,埋伏我等,意图不轨,是有什么目的?你和那方源又有什么关系?”

%13
方源心中微讶,暗暗寻思:“他怎么会直接将我这个身份,联系到真正身份上来?”

%14
表面上,方源则是冷笑:“老夫乃东海隐修,人不犯我我不犯人,奈何你天庭野心极大,意图吞并五域,一统天下。未来更是要找老夫的麻烦,欲除我而后快。”

%15
龙公顿时皱起眉头,反驳道:“气海老祖,你定然是受方源那魔头挑唆。我们之间无冤无仇,你我甚至是第一次见面,怎可能为难你?”

%16
“方源魔头生性狡诈,挑拨离间,你不可不察。”

%17
方源冷哼一声:“老夫起初也是不信,但方源却是道出老夫的许多秘密,又给出确凿证据。他从未来重生而来,我不信他,还信你不成?”

%18
龙公眉头皱得更紧。

%19
方源的诬告有点赖皮了。

%20
若是诬赖天庭过去、现在的所作所为,龙公还能拿得出证据来。但诬陷天庭将来的某种行为,龙公根本无从反驳,甚至他自己也不确定!

%21
甚至,就连龙公自己的脑海中都有一些想法:“八转蛊仙岂是那么容易蒙骗的?方源既然说动这位气海老祖来对付天庭,恐怕未来的某个时刻,天庭真的对气海老祖出手了。既然如此,那言语劝说都是无用的了。”

%22
龙公心中叹息,又想到自己此行是来寻回龙公,结果在这里受阻。气海老祖明显是来阻截他的,这么一推测,方源很可能在对龙宫下手。

%23
“看来我此战得速战速决!”龙公眼中闪过一抹决意!

%24
他终究是被方源骗过去了。

%25
这不仅是因为方源改良了见面曾相识杀招,和气道手段相互搭配。更主要的原因在于,气海老祖此刻展现出来的是确确实实的八转修为。

%26
龙公一直认为方源是七转蛊仙。

%27
不只是龙公,智道大能紫薇仙子,乃至整个天庭,甚至全天下都是如此。

%28
不同于上一世,这一世至尊仙窍的秘密一直没有暴露。

%29
方源在这一点做的非常到位。

%30
仙道杀招——龙爪击!

%31
龙公心中决意一定,几乎不假思索,再无任何犹豫,悍然出手。

%32
这是变化道的杀招,立即就有一记爪痕,印在半空中。

%33
松鹤亭中的方正惊呼一声,未料到龙公居然率先出手,并且还是偷袭!

%34
但气海老祖的身体,虽然被击中,却是化作一阵云雾,当即散去。

%35
方源早就防备着龙公,再加上这里又是他的仙道战场之中,因此躲闪开来,分外从容。

%36
龙公面色不变,此招不过是他的试探。

%37
现在他试探出来,这里果然是气道战场,变化道的威能被压制得相当严重。

%38
既然如此,那就转换手段好了。

%39
龙公没有转身,直接伸手对身后一指。

%40
气流喷涌,迅速在松鹤亭上罩起一层气泡,将松鹤亭裹在当中。

%41
“多谢龙公大人!”方正不敢怠慢,连忙出声感谢。

%42
龙公并不理会他,而是又催动了另外的杀招。他酝酿片刻,双掌一推,推出数条气龙。

%43
半透明的气龙,迅速游荡,一旦搜寻到方源的踪迹,就会主动进攻。

%44
“龙公大人果然不是一般的八转蛊仙!”方正见这些气龙,一条条都长达十多丈,威武雄壮,且又灵动神俊,顿时心中暗赞。

%45
但就在这时,龙公猛地抬头,眼中厉芒乍现。

%46
“什么声音?”方正也听到巨大的轰鸣之声,连忙抬头望去。

%47
下一刻,他嘴巴大张,瞳孔猛缩。

%48
只见一道磅礴的刀气,长达百丈,又宽又厚,好像是神灵的战刀,劈砍下来,要把这方天地劈成两半!

%49
刀气磅礴浩荡,卷起无数风雷,方正悚然,在这庞大的刀气下,他渺小若蚁,之前的那些威武气龙,也在瞬间沦为了蚯蚓、小蛇一般。

%50
危难关头,龙公咆哮一声,悍然催动杀招。

%51
气墙!

%52
气墙乃是气道最为常见的防御杀招,但龙公此时催出的手法却不简单。

%53
他动用连招,连续催动气墙三次。

%54
气墙先是一层,随即蹭蹭暴涨两次,更厚更坚韧,等若是三场气墙叠加融合。

%55
轰!

%56
刀气轰击到气墙上,发出震耳欲聋的炸响。

%57
哪怕是在松鹤亭中,方正也眼前一黑,头脑眩晕,双耳瞬间迸溅出鲜血。

%58
气墙崩溃,刀气也消散了绝大多数,只余下一些残留余波,皆是强弩之末。

%59
松鹤亭宛若小舢板,在这余波中跌宕不休,剧烈颠簸。

%60
方正好不容易将仙蛊屋稳住,满脸纸白,气喘吁吁,狼狈不堪。

%61
“这位东海蛊仙,怎么会这样的强?!”方正仍旧惊骇不定,之前的安全感已经是荡然无存。

%62
龙公倒是毫发无损,但脸色却是凝重,再无之前的轻松之色。

%63
他仰头望向方源,方源身处高空,双方距离遥远。

%64
刚刚那招刀气,气息自然而然就泄露了方源的位置。

%65
龙公缓缓升空,对方源道:“好一招刀气,虽不精巧,却是磅礴。不过这样的招数,你又能催动几次呢?”

%66
方源微笑,从容淡定。

%67
之前那记杀招,因为威能太过浩荡,令龙公误认为是方源蓄势已久,方才催动出来。

%68
这种认知当然不错,但方源却是特例中的特例。

%69
“哦?你说的是这个,不过是我随意而为罢了。”说着,方源伸出食指指头,对着松鹤亭轻轻点了一下。

%70
轰!

%71
磅礴的刀气,一如之前,轰然汇聚后,重重劈下。

%72
方源食指又点一下。

%73
轰!!

%74
又一道刀气,浩荡恢弘,直似要开天辟地般,紧随之前那道。

%75
方源食指再点一下。

%76
轰!!!

%77
第三道刀气,份量丝毫不减,甚至还隐隐大了一轮。

%78
方正浑身僵住,当场呆住!

%79
“你这是……”龙公也是心头猛震,对方催发刀气杀招非常容易,简直是喝水般简单。

%80
但这怎么可能?!

%81
没有什么不可能的。

%82
方源一身的气道道痕,规模庞大,突破百万大关!再加上这里的气道战场增幅,做到这一步,也是情理之中。

%83
龙公当然也有气道道痕,但数量只是方源的三成而已。

%84
所以,方源一招刀气,他得连用三招气墙,才能勉强挡得下来。

%85
“对方战力浩瀚,东海竟藏着这样的人物!呵呵,有趣。”龙公眼中厉芒电射,被激发出无限的战意。

%86
他仰头望去,那三道刀气,简直是铺天盖地,已然逼近。

%87
“已经很久,没有如此挑战性的战斗了。”龙公深呼吸一口气,下一刻滔天的气势陡然爆发。

%88
龙吼声起,他直接扑向刀气!

%89
方源和龙公展开激战,而在海底梦境中,却是一片温情祥和。

%90
“师妹,小心,我这招可是蓄势已久呢。”龙人分身笑着道。

%91
黄眉少女泰琴娇笑一声:“师哥,你且尽管来,我早就准备着呢。”

%92
“那就好。”龙人分身轻轻一挥手,顿时蚁群飞舞起来,交汇成一条金色的大河,冲向泰琴。

%93
泰琴同样一挥手,从地底翻出一群飞蚁,同样汇聚如河,反冲过去。

%94
两只蚁群在半空中碰撞,却不火并,反而相互汇聚融合,一点损伤都没有。

%95
蚁群在半空中交汇,彻底融合一体,不分彼此。

%96
金光熠熠之下,方源的龙人分身目光火热地盯着泰琴。

%97
泰琴勇敢地对视,脸颊上升起两朵红云,眼中同样是藏不住的绵绵情意。

%98
“师妹。”

%99
“师兄。”

%100
两人走向彼此,深情相拥,蚁群在他们身边欢快飞舞。

%101
一阵温存后,又是告别的时刻。

%102
“师妹,我必须走了。唉,我多想永远地和你在一起,然而……父命难为,我已是有家室的人。这些年真是苦了你。”龙人分身深情吻别泰琴。

%103
泰琴摇头:“我不苦,师兄,你是身不由己,而这一切也都是我心甘情愿的。我一直都在这里,你随时来,我都在。”

%104
当龙人分身赶回书道阁时,已经是半夜时分。

%105
月色正明。

%106
他远望着山峰峰巅的那座楼阁,心中升腾起怅然之情。

%107
就在这时,一处阴影里,一位龙人蛊仙偷偷地飞上来,和龙人分身碰头。

%108
“兄长,您不在的时候,嫂夫人又和那范极秘密幽会了。”龙人蛊仙汇报着,脸色不愉。

%109
“哼,这个贱人。”龙人分身面色陡然阴沉下来,与此同时,一股怒恨之情从他心底深处涌现。

\end{this_body}

