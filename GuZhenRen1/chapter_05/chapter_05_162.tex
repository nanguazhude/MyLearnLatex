\newsection{进入陌生福地}    %第一百六十二节:进入陌生福地

\begin{this_body}

%1
方源十分顺利地回到琅琊福地。

%2
此刻,他的神念萦绕在至尊仙窍之中。

%3
“前后检查了十八遍,应该是没有问题了。不过即便出现了问题,刚刚孵化出来的上极天鹰,也不过是荒兽级。就算反叛,也不要紧。”

%4
方源心中思量着。

%5
这里便是至尊仙窍中的小紫天里。

%6
一座普通的鹰巢,漂浮在空中。

%7
鹰巢中,除了一颗上极天鹰鸟蛋之外,空无一物。

%8
在方圆百里范围内,方源部署了许多力道仙僵,还有荒兽。就连走肉树,也被他暂时移栽过来,确保万无一失。

%9
再一次孵化上极天鹰!

%10
方源开始动手。

%11
一大股的鲜血,被力道仙僵们取出,然后蛊虫一一催动起来,让这些鲜血完美地融入到上极天鹰的鸟蛋里面。

%12
这个事情,方源曾经做过一次。

%13
不过那一次,方源是抽取的黑城的鲜血。而这一次,方源充分准备,不仅篡改了上极天鹰的记忆,而且还改变了黑凡在上极天鹰身上的一些手脚。

%14
如此一来,方源就能够直接利用自己的鲜血,孵化出上极天鹰。而不必像之前那般,需要利用见面曾相识进行变化伪装,才能亲近上极天鹰,获得这头太古荒兽的认可了。

%15
一切按部就班,顺顺当当,没有出现什么意外。

%16
仙窍时间八天八夜之后,上极天鹰的鸟蛋再次破开,幼年的上极天鹰孵化而出。

%17
“啾啾啾啾。”

%18
上极天鹰看到“方源”,立即亲热地扑上过来。

%19
这位“方源”,自然不是方源的真身,而是一位力道仙僵,利用了见面曾相识,变成的模样。

%20
它虽然刚刚出身,但体型已然不小,昂首挺胸。有正常少年的高度了。

%21
小鹰将方源当做自己的亲生父母,围绕着他,扑扇着一对肉翅,不断打转、鸣叫。十分开心。

%22
“好了,好了。”方源温和地笑着,伸出手来,抚摸上极天鹰的脑壳儿。

%23
上极天鹰立即乖巧的一动不动,一直尖锐的鸣叫声也变得柔和婉转得多。

%24
“来。吃吧。”方源又取出一根又粗又长的天晶。

%25
天晶色泽白中带金,半透明状,放到小鹰的面前,立即吸引了后者全部的注意力。

%26
一对漆黑如墨的鹰眸,骤然间暴射出精芒。鹰嘴啄来,坚硬的天晶在鹰喙之下,脆弱得如同豆腐渣,被小鹰轻轻一啄,就吃下一口,吞入肚中。

%27
小鹰对方源极其信任。对于方源手中的天晶,它毫不犹豫就选择进食。

%28
一连吃了六根天晶,它这才满足地停下。

%29
小肚子都鼓胀起来,小鹰变成了一个小胖墩,直接坐在地上,哼哼唧唧,一对小肉翅不断轻轻地拍打自己的肚子,模样十分憨态可掬。

%30
“吃撑了吗?”方源温柔地笑着,靠着小鹰坐下。

%31
小鹰感受到了强烈的安全感,放下所有的戒备。双眼渐渐闭合,脑袋一歪,就靠在方源的大腿上,呼呼入睡。

%32
方源伸出手来。抚摸着小鹰的羽毛。

%33
它刚刚出生的羽毛,还都是嫩黄色的,十分柔软酥松,再加上温热的体温,手感极好。

%34
接下来的日子里,方源便操纵着这具力道仙僵。时刻陪伴着上极天鹰。

%35
除了培养和加深双方的情感之外,还有时刻监视,防止意外产生的用意。

%36
“天晶还是足够的。”方源的神念从这座鹰巢中抽回。

%37
放眼望去,数百里的外围,还有许多鹰巢。

%38
这些鹰巢数量接近一百,乃是方源从铁鹰福地中打劫过来的。

%39
方源选取了一座最大的鹰巢,用来暂时当做上极天鹰的巢穴。

%40
原本那座天晶鹰巢,早已经被上一世的上极天鹰吃个精光,渣滓都没有剩下一点。

%41
不过现在,鹰巢中的天晶虽然不能构成上极天鹰,但仍旧存储了一堆,比成年人还稍高一点。

%42
这些天晶当中,很小的一部分来源于琅琊地灵。这些都是方源利用门派贡献,和琅琊地灵换取的。

%43
另外的绝大多数,则是在楚门和百足天君联手之后,方源通过楚度,和百足天君交易,买卖来的。

%44
百足天君乃是八转蛊仙,拥有仙窍洞天,他的手中确实有不少天晶存货。

%45
方源为了这场交易能够达成,在向楚度透露出北部冰原的异人秘密的时候,就顺势将上极天鹰的存在,告知了楚度。

%46
其实这早就不算什么秘密。

%47
楚度或许知道,百足天君肯定知道,因为他招揽了黑家的几位太上长老,收为麾下。

%48
百足天君虽然对上极天鹰十分觊觎,但目前而言,他更需要楚门这样的盟友。而楚度也想和方源加深合作,所以种种因素之下,使得方源很容易的就促成了这笔买卖。

%49
方源付出了大量的资源。

%50
这些资源,原本都是黑凡洞天之中的。但为了收购这些天晶,方源都毫不吝惜地舍弃出去。

%51
不管是楚门,还是百足家,也都需要这份宝贵的资源。

%52
双方由于大战的缘故,底蕴都消耗很多,需要补充。并且更妙的是,这些资源本来就是黑凡洞天的产物,和环境十分契合,根本不需要适应的过程。

%53
成功孵化了上极天鹰之后,方源又定下心来,仔细观察了几天。

%54
等到他确信,上极天鹰毫无问题之后,方源便立即启程,利用琅琊派的传送蛊阵,直接来到了太丘。

%55
第六次地灾,已经将近,时间所剩无几。这个时候,方源前往太丘做什么?

%56
他是万万不能在太丘中渡劫的。

%57
因为太丘太过于危险。

%58
上古荒兽、上古荒植都灵性缺乏,智慧低下,很容易被天意影响,酿成汹涌的超级兽潮。

%59
方源按图索骥,一路前行。

%60
他动用见面曾相识,不断变化外形,伪装成上古荒兽,或者荒兽模样,在太丘中行走。

%61
太丘的最中心,方源是不敢过去的,里面危机重重,野生仙蛊有不少,甚至说不定还有什么太古荒兽!

%62
好在他此行的目的地,也不在太丘中央地带,而是位于外围。

%63
三天之后,他顺利地到达了目的地。

%64
眼前的地面坑坑洼洼,虽然巨人草十分茂密,但难掩这里的激战痕迹。

%65
一些残破的骨骼碎片,洒落在这里。

%66
方源目光扫视周围,没有发现什么危险,他吐出一口浊气,稍稍放下心来。

%67
“上一次地灾,令我增添了不少暗道道痕。这一次利用暗渡,遮掩天意感应,效果是之前的数倍!所以我暂时还是安全的。”

%68
“那么接下来就是……”方源开始调动一只只的蛊虫,开始探索周围。

%69
他小心翼翼,搜索得十分仔细。

%70
他的仙窍中,有一只七转仙蛊,这是方源特意向百足天君借来的。租借仙蛊,也是方源和百足天君交易的一部分内容。除此之外,还有一记仙道杀招。

%71
以这只七转仙蛊为核心,方源催动起仙道杀招。

%72
不久后,他有了发现。

%73
“好,我的记忆果然没有错!就是这里了!”

%74
仙道杀招让方源发现了一座仙窍福地,它正寄托虚空,隐藏在这里面。

%75
“好鹰儿,接下来就看你的。”随后,方源打开仙窍,放出上极天鹰。

%76
上极天鹰经过这些天的成长,已经体型壮大了数倍,但还是幼体,羽毛光亮,鹰目犀利,显得矫健不群。

%77
它飞上空中,用背载着方源,低空盘旋了几圈之后,在方源的指点下,很快就察觉到了什么,便猛地掉转方向,撞向地面。

%78
在距离地面仅一丈的时候,上极天鹰忽然撞开了一个口子,载着方源骤然消失在太丘之中。

%79
下一刻,方源就来到了一片完全陌生的福地之中!

%80
ps:明天再一次爆更新,呼吁大家正版订阅,目前朝着精品的距离越来越近了。大家可以猜猜,这是哪座福地。

\end{this_body}

