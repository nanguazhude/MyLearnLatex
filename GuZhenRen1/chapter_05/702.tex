\newsection{龙御上宾}    %第七百零五节:龙御上宾

\begin{this_body}

一瞬间,龙公明白了,这是一道元莲真传!

他明显可以感觉到自己的木道境界,暴涨到了极高的层次,同时有关于这记木道杀招穿枝拂叶的种种,都深深地印刻在他的脑海之中。

那片竹林,与其说是幻境,倒不如说是一种心境。

而穿枝拂叶杀招的威能,便是解脱蛊仙,消去蛊仙身上的有害束缚。

“真是狼狈啊。”龙公自嘲苦笑。

他再一次感受到尊者的强大。

事实上,他对一缺抱憾亭中的双尊棋局也多有了解。这棋局威能无穷,和天庭一体,源源不断地汲取天庭的力量。这是当年,星宿意志和无极魔尊的神秘约定的产物。

因为有着双尊的伟力,所以单靠一尊之力,很难对抗。

不过,束缚龙公的的银白锁链,只是这棋局中的某个棋子所化。元莲仙尊留下的穿枝拂叶杀招,正能为龙公解决这等束缚。

“元莲仙尊用心良苦,这记木道真传给了我,实在可惜了。”

龙公叹息一声,收拾情绪。

龙瞳转动,他微微抬头,盯上了不远处的劫运坛。

劫运坛庞大的身躯,横霸龙公大半个视野。

龙公被束缚的时候,就倒在劫运坛附近,如今获取自由,站立起来,自然离劫运坛不远。

深呼吸。

周围的激战声、爆炸声、呐喊声,都在心中越来越小。

被束缚的时候,龙公的悲愤、自责、仇恨充盈满胸,此刻他恢复了自由,这些澎湃的情绪就像是一个积蓄了数千年的火山,即将爆发。

不知不觉间,参战的所有蛊仙,都感受到了一股暴风雨之前的宁静。

呼……

吸……

龙公主修气道,兼修变化道,每一次呼吸都抽取海量的空气,造成巨大的声势。甚至这种声势,甚至隐隐约约地影响到了远处气墙的稳定。

龙公身后,龙尾微微甩摆,他的龙爪、龙角,都荡漾着一层紫金光晕,贵不可言,不容冒犯。周围的空气中荡起一缕缕的微尘,似乎都泛起一层相同的紫金华光来。

一股恐怖绝伦的气势,从他的身上缓缓地升腾而起,像是一头曾经开天辟地的巨兽,沉睡了无穷岁月,此刻正逐渐地苏醒。

无数道目光,迅速集中在了龙公的身上!

“该死。”冰塞川暗暗咬牙。

花子昏迷,龙公脱困,长生天的此次行动遭受到了有史以来最大的挫败。

炼道大阵中,牛魔、五行**师也预感到了不妙,拼尽全力冲杀。但是车尾、从严死死守住,不给对方一丝突破的机会。

在炼道大阵的更深处,袁琼都一边吐血,一边艰难地维持着炼道大阵的运转,他的心中已浮现死志。

冰塞川苦思良谋。

情势急转直下,令他额头垂下冷汗。

他手中的底牌已经见底了,无法再一次请无极魔尊转身。事实上之前,无极魔尊根本就没有转身,这种态度已经表明了一切。

龙公的气势攀升到了顶峰,变得滔天浩荡!

他虽未有一丝动作,但已经吸引了战场中所有人的高度重视。

“他便是龙公大人吗?”有天庭蛊仙投来仰慕崇拜的目光。龙公的辈分还是很高的。

但也有人辈分比龙公还高,一位蛊仙老者并不认识龙公:“他是我方的人?”

“龙公大人终于脱困了。”紫薇仙子激动不已。她纵览全局,心中非常清楚,在此刻天庭的阵容中,龙公是当之无愧的第一强者。

他挣脱枷锁,恢复自由,再次站起来的太及时了!

不,更准确地说,是元莲仙尊的布置太精准了。

想到元莲仙尊,紫薇仙子的心中不由地涌起崇敬之情。

“难以置信的手段!在元莲仙尊之后,监天塔其实已经修葺改善过多次,许多蛊虫被替换。尤其是最近十多年几番大战,屡屡遭受重创,大修了数次。元莲仙尊的手段,居然还完整无缺地保存着,隐藏得如此完美,没有人能够发现。”

“他的这记仙道杀招,可是隐藏在监天塔中的。监天塔可以说是侦查第一的仙蛊屋,居然至始至终都没有发现。并且更妙的是,元莲仙尊的这记仙道杀招,从来没有干扰过监天塔的运转。”

想到这里,紫薇仙子的脑海中忽然灵光一闪:“难道说,那个传闻是真的?元莲仙尊真的开创了画道?传闻中的画道流派,最大的优势和特征,就是可以完美地规避其他仙仙蛊、杀招的运转……这和有至尊仙体的道痕不互斥,有着异曲同工之妙!”

紫薇仙子也只能猜测。

十大尊者是蛊修历史上的最巅峰,是最明亮闪耀的星,光辉璀璨至极。元莲仙尊尽管不是智道,但已然通达天地之理,领悟变化之妙,埋下的伏笔影响后世。

更关键的是,这伏笔的出现太及时了。对于时机的把握,真的是妙到毫巅。

冰塞川紧张不已,口中下令:“防住龙公!”

无须他的提醒,所有的北原蛊仙都看得出龙公的威胁。

龙公的气势达到巅峰之后,竟开始缓缓收敛起来,就好像是有人将高举的拳头收起,一旦爆发,必定石破天惊。

此举带给北原一方极其沉重的心里压力。

“杀。”一位北原壮汉口绽雷霆,向龙公冲杀过去。

第二位北原蛊仙默然不语,紧随着在壮汉身后,一边冲锋,一边为他掩护。

“不,让我来会会你!”第三位北原强者忽然抢出,他的速度极快,率先杀到龙公的面前。

龙公一脸平静,静静地站在原地,一动不动,宛若雕塑。

嗷吼!

忽然间,龙吼声起,一道华丽的紫金龙形气劲,在天空中一闪即逝。

蛊仙们想要看清楚时,瞳孔中只留下了紫金龙气的些微残影。

抢杀到龙公面前的北原强者,静止不动。

他缓缓低头,带着难以置信的目光,看到自己胸前的巨大空洞。

顺着这个空洞,他可以看到自己身后的天空。

噗。

下一刻,他大吐一口鲜血,然后浑身上下再次冒出一股股小型龙形气劲。

砰。

一声闷响,他陡然自爆,化为一团血沫,战死当场。

他当然是北原知名强者,但竟难挡住龙公一击之威!

“怎么可能?”看到这样一幕,落后一步想要攻杀龙公的两位北原蛊仙,纷纷大惊失色。

“你们再看什么?”龙公淡淡的声音,忽然在身后传来。

“不好!”两位北原强者连忙催动最强防御,根本没有多想。

这种根植在骨髓深处的敏锐意识,救了他们一命。

下一刻,龙公两只狰狞的龙爪狠狠落下。

两位北原强者的防御在瞬间支离破碎,他们就像是两颗流星,狠狠地撞击在地面上,共同砸出一个巨大的深坑。

轰隆一声巨响个,地砖的碎片翻飞,烟尘滚滚蔓延。

两位天庭强者昏死在坑底。

龙公出手,果然是石破天惊!一瞬间,就造成一死两伤的战果,长生天一方立即折损了三位八转战力!

“恐怖!”

一时间,龙公爆发出来的战斗力,让天庭一方都楞了一愣。

那位不认识龙公的老者更是诧异得很:“我死后,竟然有如此惊才艳艳的后辈?”

“他是红莲魔尊的师父,也是护道人。”身边有天庭蛊仙适时解惑。

天庭老者这才恍然:“哦,难怪。”

历来尊者的护道人,只要成长起来,都几乎是位于尊者之下的第一战力。

当然,也不是没有例外。

比如狂蛮魔尊,又比如盗天魔尊。

但不管怎么说,这些护道人都在尊者的成长过程中,起到了举足轻重的护道作用。

“来了!”冰塞川深吸一口气,全神以待。

龙公击溃三人之后,长驱直入,直冲劫运坛,气势极其凶猛,令人心惊胆战。

冰塞川在这一刻,忽然感觉到雄厚的劫运坛,也透出一股脆弱。

“护住我!”冰塞川大叫。

长生天的蛊仙纷纷回援。

龙公此举,就像是捅了马蜂窝,大量的长生天蛊仙内外夹攻,形成一道巨大的包围网。

无数仙道杀招打向龙公,仿佛是璀璨的烟花,缤纷的华美中蕴含着致命的威胁!

紫金龙形气劲再闪!

龙公消失在远处,令许多杀招都扑了空。

剩下的杀招在蛊仙的引导下,纷纷掉转方向。

龙公再度现身,面对袭来的杀招脸上涌现出不屑之情。

“给我开!”他大喝一声,这一次施展出了气道杀招。

澎湃的气浪四面卷席,将一切杀招都排斥开来,使得周围瞬间清空。

几位长生天的蛊仙,只得扑向龙公。

“来而不往非礼也,尝尝我的杀招罢。”龙公狞笑。

长生天的蛊仙顿时又丧生两人。

龙公继续冲向劫运坛。

劫运坛在冰塞川的操纵下,不得不飞速爆退。

龙公一路追杀,不管是面对哪一位北原强者,都是摧枯拉朽地撕破防线。

当然,太过猛烈的杀招,他也会躲闪规避。

冰塞川的眼中闪过一抹惊惶:“不可能!龙公怎么可能,在这么短的时间里,忽然强大到如此地步?难道这也是元莲仙尊的手段?”

这一点,他猜错了。

龙公之强,在于自身。

仙道杀招龙御上宾!

当他的寿元越少,龙御上宾带给他的增幅就越加恐怖。

这种增幅覆盖全方面,使得龙公空前强大,每时每刻战力都在暴涨。

他的生命气息在迅速流逝,但是浑身气势却不断上升。这是一种回光返照,龙公要在他生命的最后阶段,爆发出史无前例的璀璨光彩!

\end{this_body}

