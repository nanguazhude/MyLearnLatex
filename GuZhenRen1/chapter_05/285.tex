\newsection{金刚念}    %第二百八十五节:金刚念

\begin{this_body}

这只金刚念仙蛊,有拳头大小,外表宛若琉璃,半透明,闪烁着金灿灿的光,卖相极佳。

方源握在手中,感觉沉甸甸的,仿佛握着一团实打实的黄金。

双方交接很顺利,过程很短。

当方源成功炼化了这只金刚念仙蛊之后,武雨伯的意志就辞别了方源,直接飞走。

就这样获得了一只六转仙蛊,却没有让方源感到意外。

他对武家有功!

方源救出了武雨伯,对于武家而言,功劳很大。尽管他违背了规矩,丢了武家的脸面,但却为武家保存了一位珍贵的七转战力。很显然,功远远大于过!

武庸虽然惩处方源,但方源理解得很。这就是正道,非得要场面上过得去才行。

武家肯定要奖励方源,但武庸是不能出面的。

武庸一出面,政治意义就完全不同,一经发现,会被他人诟病和攻击。

所以才是武雨伯的意志出场。

这只金刚念仙蛊,是武雨伯对方源的私人感谢,实际上却是武家家族支出。但必须要有这层辗转,这就是正道的行为方式。

“又得了一只仙蛊啊。”方源苦笑。

或许寻常蛊仙会喜不自禁,但是对方源而言,意义不太一样。

仙蛊太多了!

负担有点重。

“不过,金刚念仙蛊是智道仙蛊,倒是来得正好,可以弥补我当前的一些短板。”

方源智道境界不缺,缺的是智道仙蛊。

以前只有态度蛊、解谜蛊,现在多了爱意蛊、金刚念蛊。

四只智道仙蛊,这数量也差不多了,凭借方源宗师级的智道境界,将大有作为。

方源灌输红枣仙元,试着催动这只金刚念仙蛊。

顿时,他的脑海中,就浮现出了一颗颗金色念头。这些念头小巧精致,闪烁着金属光泽,给人坚硬厚实的感觉。

念头碰撞之下,似乎锵锵作响。

方源试着用这些刚刚产生的金刚念头,进行思维推算。

他很快发现,这些金刚念头其实并不擅长于思考。念头之间很难融合,经常相互之间直接撞碎。

方源暗道一声“果然”,心底并无意外。

金刚念仙蛊本身,就不是用来思考的,它的长处在于对敌作战。

方源将这些金刚念从自家脑海中调动出来,这些念头原本小巧精致,一旦脱离了脑海,来到了外界之后,却扩大膨胀,形成拳头大小的颗颗金球,表面闪烁着琉璃般的光泽。

方源盘坐在蒲团上,身边悬浮着数百颗的金刚念,一时间金光辉映,华丽非凡。

他思绪一起,其中一颗金刚念顿时飞出去。

刷!

速度飞快。

念头的速度本来就很迅猛,尤其是在脑海中,念头的速度比光还要快!

金刚念到了外界,速度自然比不上在脑海中,但亦是速度极快。

“这速度几乎有剑遁仙蛊的五成。”方源也是眼角微微一跳,他本人对这样的速度非常满意。

剑遁仙蛊可是七转仙蛊中,速度上佳的蛊虫。金刚念仙蛊只是六转,发出的金刚念头却有剑遁仙蛊的一半速度。

当然,金刚念这种类型的仙蛊,就算提升到七转、八转,产生的金刚念头也不质变,而是量变。

理论上来讲,一只凡蛊金刚念,能产生的金刚念头,和仙蛊产生的并没有什么区别。区别主要在于数量。五转凡蛊金刚念,能同时产生上百颗金刚念头。但是仙蛊金刚念,却能在瞬间产生上千、上万的念头。

那么一颗金刚念头的威力如何?

方源对准某块地砖,射出金刚念。

砰的一声轻响,金刚念在半空中划出一道绚烂的金色亮芒,眨眼间就射破了地砖,深入地底。

方源点点头。

威力还不错。

他身居的这处大殿,乃是一座凡蛊屋,防御力在蛊仙眼中不够看,但也并非完全是纸糊的。

金刚念的威力,让方源联想到了地球上的子弹。

不过它的块头,可比子弹要大得多。

“催动一次金刚念仙蛊,可以产生数千颗金刚念头。这么多的念头一齐射出去的话,也是蔚为壮观的。”

“等一下……”

方源的脑海中,忽然闪现了一下灵光。

他想到了两个蛊仙,一个是东方长凡,另外一个则是耶律群星。前者有一个仙道杀招,名为万星飞萤。后者也有一个仙道杀招,以星屑为主体,越打越多,越战越强。

“我现在有了金刚念仙蛊,何不仿造万星飞萤,构思出一记仙道杀招?”

方源心中念头一动。

智道宗师境界赋予的直觉告诉他,这件事情应该不难。

于是,方源便停止催动金刚念仙蛊,沉下心来,开始琢磨这记仙道杀招。

主要是模仿万星飞萤杀招,以金刚念仙蛊为核心。

果然很顺利,难度很小,一如方源之前的直觉。

小半个时辰之后,这记仙道杀招就有了雏形。

一天一夜的时间,方源不眠不休,变化成卜卦龟连续推演,得到了完整版的仙道杀招。

方源没有为这个全新的仙道杀招命名,因为他又有了新的灵感。

“单纯这记仙道杀招,只是以六转的金刚念仙蛊为核心,威能并不可观。或许我能够再结合卜卦龟变化,再改良这记仙道杀招,增长它的威力!”

这一次改良,方源却感觉并不简单。

虽然他有宗师级的智道境界和变化道境界,但是两相合流,却非一蹴而就之事。

改良的速度,缓慢无比。

但方源也不着急。

现在的外部环境,还算安逸。

虽然影宗未除,天庭、长生天都在通缉他。但只要借助梦境,方源相信自己的底蕴将迅猛暴涨。

不得不说,有了逆流护身印,他的底气增加了不少。

三天后,杀招改良进展不大,只改良了三成不到的程度。武庸却唤人来,他要召见武遗海。

在大殿中,兄弟俩寒暄几句后,武庸告诉方源:若是他对金刚念仙蛊不满意的话,可以在族中内库中挑选任意一只六转仙蛊,相互调换。

原来武庸见方源接受了金刚念仙蛊后,却迟迟不见动静,私底下便以为武遗海身为散修,不懂得正道中的道道,或许还真以为这是武雨伯的谢礼。

这可不行。

武庸便唤来方源,提点他,之所以能获得金刚念仙蛊,多亏了自己这位兄长。并且同时以换蛊为由,对武遗海表达善意。

他哪里知道,方源早就心知肚明,只是故作糊涂罢了。

“看来武家这段时间,风雨袭来,蛊仙力量捉襟见肘,武庸是想利用我这股力量了。”方源顿时对武庸的用意有所察觉。

武家的事情,方源当然不想管。

但近些日子里来,不少超级势力都默契地联合起来,一起对武家施难。

四面风雨,武庸颇有些疲于应付的架势。

超级梦境那边也是如此,巴家领袖一些蛊仙,对武家多有刁难。

本来,武庸是想将方源调换到那边去的,但巴家发难,超级蛊阵那边也是压力巨大。武庸稳妥起见,不敢随意临阵换将,导致方源调入超级蛊阵的事情,就搁浅在这里了。

“劳烦兄长记挂。小弟也想去内库中增长一番见识。”方源虽然对金刚念仙蛊,设计了仙道杀招,不太想换,但是先看看内库中的仙蛊也是好的。

上一次他在族中库藏中选取仙蛊,只是外库而已。内库中的仙蛊,自然更加精良,价值更高。

武家作为南疆第一势力,内库中有什么收藏,让方源也颇为期待。

内库中的仙蛊寥寥无几,但各个都是精品、极品。不过方源的仙蛊实在太多了,主要还是探探武家的底蕴,增长一些眼界。

\end{this_body}

