\newsection{李小白要上位}    %第八百六十二节:李小白要上位

\begin{this_body}

%1
方源端详着手中的变异仙蛊。

%2
这只仙蛊形象奇特,有天牛样的身躯,蜘蛛样的八爪,头部翘起独角仙似的角,背部是蝉翼,身前还有两只螳螂的刀臂。尾部发光,好像萤火虫,但发的是彩光。

%3
变异仙蛊乃是千变老祖之物,但这难不倒方源。

%4
方源脱离战场不久后,就在半路上将这只八转仙蛊炼化。

%5
不提那边千变老祖感应到,再次恨死了方源,方源尝试几次催动,大体上摸清楚了变异仙蛊的作用。

%6
这只仙蛊用于攻伐,威力颇大,能够让敌方的某一部分身躯反叛自身,碾压七转层次的存在,能带给太古荒兽等等巨大的麻烦。

%7
也可以用于自身,在变化上增添变化。比如,方源变作一头猛虎,用了变异仙蛊后,说不定就能从虎躯两侧,生长出一对翅膀。又或者能在虎头上,变出第三只眼睛来,又或者将虎尾末端长出蝎子的毒钩。

%8
当然,这只是好的变化。

%9
也有可能会产生比较坑爹的变化。

%10
比如老虎肚子上,长出象腿来,象腿超过老虎四肢的长度。又比如虎背上长出了痔疮,这种明显的弱点,就是给敌人竖起的靶子。

%11
单纯用变异仙蛊,效果好坏参半,说不准。

%12
因为这只仙蛊高达八转,增添的变化无疑都是八转层次,对于本体而言,影响非常巨大。

%13
所以,稳妥起见,还是得用其他蛊虫辅助变异仙蛊,形成杀招。

%14
用这杀招来增益自身,是最为保险的。

%15
变异仙蛊除了攻敌、增变之外,还能用于仙窍的经营。

%16
以它为核心,施展杀招或者铺设仙阵,能使仙窍中的生命产生更多的异变。普通的野兽,能逐渐生长为异兽。

%17
当然,这些异兽中也有残次品,但在优胜劣汰的自然循环中,会被逐渐清除。

%18
而那些优秀的异兽,却是会生存下来,繁衍壮大,形成新的族群。

%19
往往这些变异的生命,发展壮大之后,便是某个仙窍的特产,只此一家,别无分店。若能在市场上凡响尚佳,那就是蛊仙的一项长久的生意了,会给蛊仙带来一笔稳定可观的收益。

%20
“变异仙蛊作用如此全面,应当是狂蛮真传的核心仙蛊之一,难怪丢失了之后,千变老祖如此紧张了。”方源心道。

%21
除了变异仙蛊之外,方源还得到了其他数只仙蛊。

%22
大多数都是变化道仙蛊,还有一只七转律道仙蛊——关。

%23
此蛊好似蟅(zhè)虫,人们通常又称之为土鳖,它有婴孩的拳头大小,身体十分扁平,散发着钢铁般的光泽,给人极为坚固的感觉。

%24
当初,夜煞女、青岚仙妃第一次铺设仙阵时,它就是核心。

%25
凤九歌隐藏在远处,看到了这只仙蛊,还想将它夺来。

%26
这只仙蛊对他而言,具有很高的价值。

%27
没想到最后,这只仙蛊落到了方源手中。

%28
“单从这些仙蛊,还有千变老祖的手段,就可见这道狂蛮真传的厉害了。可惜,不太好下手啊。”

%29
千变老祖渡过了一次万劫,道痕底蕴是比不上方源的。但他手段丰富,性情隐忍,演技出众,卫风就是被他坑死的。除非方源能够将他诱骗到战场杀招里去。

%30
千变老祖还有万像宫殿傍身,这座八转仙蛊屋威能惊人,关键时刻又能变成千变老祖混淆视听,也是一个巨大阻碍。

%31
要图谋他,真的不容易。

%32
看看天庭,筹谋多时,施展了翻天覆地杀招,调集了四大八转,最终也是铩羽而归。

%33
更关键的是,方源本身也不能拿出真正的实力来。

%34
小不忍则乱大谋,一旦他成就八转的秘密暴露,危害更大。

%35
所以,他处理疯魔三怪的时候,都很小心。

%36
方源图谋的是大局。

%37
“距离中洲举办炼蛊大会,已经不远了。”

%38
“北原蛊仙界有长生天领袖,南疆有武庸这个权谋家。东海有我扮演的气海老祖提携,西漠则有房家、唐家可以利用。”

%39
“四域围攻中洲的大局,已见雏形。”

%40
这是方源重生以来,一直图谋的事情。

%41
终于在此刻,有了阶段性的成果。

%42
方源早已不再是一个单纯的八转蛊仙,他的麾下出众,人才济济。

%43
但他也并不是一个单纯的超级势力,他布局天下,以万物为棋子,要剿杀天庭这个障碍。

%44
天意利用他,魔尊幽魂利用他,红莲魔尊利用他,楚度利用过他,琅琊地灵利用过他……而他现在,已经成为了一名棋手。

%45
华文洞天。

%46
“我的眼睛花了吗?为何我的学生的名字,会出现在十大才子的名单中?”姜先生看着最新的情报,震惊不已,当场呆住。

%47
“老师在屋内吗?学生李小白求见。”这个时候,从门外传来李小白的声音。

%48
姜先生连忙打开屋门:“进来吧,快说说这是怎么回事?”

%49
李小白经过深思熟虑,已经考虑妥当,便将当夜发生的事情一五一十地告诉了姜先生。

%50
他忧愁不已:“老师,您可得为学生做主啊。”

%51
姜先生听了,呆呆地看着李小白,神情逐渐古怪起来,半晌,他似笑非笑地摇头:“你呀你呀,为师也不知道说什么好了。”

%52
在姜先生看来,李小白本身是没有什么希望竞争十大才子之位的。

%53
没想到,运气来了,挡都挡不住啊。

%54
这番艳遇,就算是姜先生,也暗感一丝羡慕嫉妒。

%55
“事后,苏琪涵小姐考较了学生一番,随后就告知我,让我名列十大才子之位。学生得老师您的教导,不敢狂妄,连忙拒绝,但奈何苏琪涵小姐铁了心这样安排,学生我无奈至极,胳膊扭不过大腿啊。”李小白连声诉苦。

%56
“你这番际遇,人人艳羡,你居然推托不要?”姜先生笑道。

%57
“老师,您就别打趣您这可怜的学生了。这一次来,就是想求先生救命。学生才学浅薄,如今成为十大才子之一,这是把学生往火上烤啊。”李小白拱手告饶。

%58
姜先生收敛笑容,点点头:“嗯,你能不骄不躁,认清现实,也不枉费平时为师的教导了。不过,你也不要想的太过严重。没有成为十大才子之前,你若痴心妄想,一力竞争,那是危机四伏。但如今你已经名列十大才子之一,朝廷就会力保你。谁若是害你性命,就是削朝廷的脸面,这点你懂吗?”

%59
李小白挠头:“可是学生仍旧如坐针毡啊。和其他九人相比,学生名不见经传,最会引人执意。而且发生了这种事情后,苏琪涵小姐那边对学生的态度,也是严苛得很。”

%60
姜先生再度点头:“为师虽不在庙堂,但也听闻苏琪涵这个后辈才情惊艳,心性高傲。她既然要对你负责……必然是想将你娶回去,哦,是让你入赘苏家的。”

%61
“她堪称是名动天下的几大才女之一,她的夫君必然是要门当户对。但她也不是盲目办事,先是考验了你的才学,见你有几分材料,这才把你提名进来。将来更是会给你铺路,一步步提你上去。”

%62
说道这里,姜先生看着李小白,神情就有些复杂了。

%63
想他也曾经是十大才子之一,自负才情,心比天高,但现实是残酷的,晋升之路只有那一条,却有无数人向上挤。

%64
他竞争不过,时光蹉跎,虽然声名远播,但本质上是游离于朝廷之外的。

%65
没想到自己这个学生,才学不如自己,背景不如自己,人脉不如自己,但运气也忒好!

%66
莫名其妙地就成了苏琪涵力捧的心上人了!

%67
苏琪涵本人且不说,她背后可是苏家,她的父亲就是当朝的苏丞相,李小白一下子上头就有人了。

%68
“不瞒老师,学生起初也很欢喜,但最近几日越想越不对,胆战心惊啊。苏琪涵小姐的爱慕者那么多,此事……又决不可大肆宣扬,我得苏小姐青睐,不知道要被多少青年才俊恨死呢。尤其是十大才子选定之后,我还要入朝面圣。这一路上,还有到了京师,我该怎么办呢?学生六神无主,只好来请教老师您了。”李小白唉声叹气。

%69
姜先生有些不悦,轻哼一声:“小白,你也不要妄自菲薄,你的才学是有的,为师的教导岂是白费?你是为师的学生,旁人来对付你,也要顾忌你老师我的颜面。这样……我为你写几分推荐信,你沿途拜访几户人家,他们都会为你提供一些方便。”

%70
说到这里,顿了顿,姜先生似下定决心又道:“这些天,你就不用回去了,就住在老师的家里,老师要将一些手段,传授给你。”

%71
李小白大喜过望,连忙深深鞠躬:“学生拜谢恩师!”

%72
苏琪涵要他严格保密,但李小白根本不放在心上。

%73
时机成熟后,他就立即将这个秘密告诉姜先生,表面上求助,实际上是展现自己的价值。

%74
姜先生被朝廷排斥,一直都走不进核心,见到自家学生有很大希望,一旦学生飞黄腾达,能忘了他这个授业恩师?

%75
绝不可能!

%76
且不说李小白的心性,姜先生也自认为十分熟悉,绝不是什么白眼狼。

%77
就算将来李小白忘恩负义,公众舆论之下,他也不敢不回报自己这个恩师。李小白的官越大,他就会越爱惜羽毛,越会向自己报恩。

%78
想到这里,姜先生便立即决定,要尽自己一切的努力,启动自己一切的资源,帮助自己的这个学生成功上位。

%79
学生的成功,就会成为他的成功!

\end{this_body}

