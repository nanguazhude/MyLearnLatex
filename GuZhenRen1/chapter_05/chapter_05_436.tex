\newsection{双喜临门}    %第四百三十七节:双喜临门

\begin{this_body}

%1
“终于结束了,这一场商战也是别开生面!”赵怜云心中惊叹。

%2
围绕年蛊的角逐,以方源获胜而告终。这个结果,让绝大多数的蛊仙都为之惊诧。

%3
很少有这样的情况发生,新来的人居然击败了老字辈的三大巨头。

%4
原本众人还以为:新来的是一头狼,而那三大巨头如同狮虎。结果没想到,新来的不是狼而是一头龙。三大卖家都不是对手,惨白亏输,他们手中的年蛊至今都没有卖出去过。

%5
可想而知,今后的一段时间内,这场年蛊之争的影响将会不断持续,作为茶余饭后的谈资,也会被蛊仙们津津乐道。

%6
赵怜云开始反思这场商战。

%7
她原本就想从这场商战中,汲取到知识,来帮助将来的自己。

%8
“那么,这场商战中,为什么新来的人可以获胜,而旧有的三大卖家都输了呢?”赵怜云问自己。

%9
旋即,她就自己回答了自己:“因为新来者的实力极其强悍,三大卖家就算联合起来,也不会是他(她)的对手。”

%10
“以绝对的实力,来横扫市场么?”赵怜云微微失神了一下,旋即苦笑起来。

%11
她最想借鉴和学习的,就是作为新人如何挤进市场。

%12
方源此法,颇有大巧不工,大智若愚的感觉,虽然威能绝伦,真的做到了横扫一切,打得三大卖家连头都抬不起来,但是赵怜云要学习却是很困难的。

%13
因为这是赤裸裸的实力!

%14
“既然新来者有这样强劲的实力,为什么不在第一时间,将价格压到最低呢?”赵怜云又问自己,她有点觉得方源这样做,多此一举。

%15
不过很快,当她充分思考之后,她的眼中闪过一抹了然的光。

%16
“我明白了。”

%17
“起先高价,之后再徐徐降低,一来是比降到底的方法,能赚取更多利润,二来应该是要探测三大卖家的底线吧。三来,就是示威吧。”

%18
赵怜云的脑海中念头攒动不休。

%19
“这一次商战结束了,新来者获得了胜利。有着这样的实力,胜利并不意外,但他(她)的胃口绝不会满足于这一场商战中。”

%20
“新来者是独霸宝黄天的市场,不只是将三大卖家击败,而且是让将他们排挤出这个市场。”

%21
“这个人性情好霸道……”

%22
“唉,我什么时候也能拥有这样的实力呢?”

%23
赵怜云想到这里,不禁发出一声悠长的叹息。

%24
她才刚刚经营仙窍不久,距离方源的程度,还差十万八千里。

%25
不管如何,她已经牢记住方源的胖子意志。赵怜云将其当座一座山,刚刚步入蛊仙阶层的她,只能憧憬和仰望。

%26
荣欣、王明月、谢宝树正在积极地商讨着。

%27
“唉,到底从哪里蹦出这么一号人物来?他(她)实力极其雄厚,以一人之力满足整个宝黄天市场,现在我们手中的年蛊,根本卖不出!”王明月叹息。

%28
“情况不是这么简单的,这个人是想把我们赶尽杀绝啊!”谢宝树面色发白,声音都变得有些尖锐。

%29
“谢兄一点说的没错。此人开如此低价,一下子贩卖这么多的年蛊,接下来两三年内,宝黄天的年蛊市场将萎靡到极点。不管他(她)是谁,这么做实在太过分了。不仅吃肉,连点汤水都不留给别人。”荣欣脸色阴沉如水。

%30
三大卖家手中都有年蛊,但他们都卖不掉了。

%31
年蛊是大家都需要的东西,方源一下供货这么多,引发哄抢,导致蛊仙们都在自己手中囤积了一大批年蛊。

%32
反正年蛊非常容易喂养,只需要吞食光阴河水就可以了。

%33
囤积年蛊一点都不困难。

%34
如此一来,不仅现在,就算是接下来的两三年里,三大卖家手中的年蛊都卖不出去。

%35
对于买家而言,我自己手中有年蛊了,我为什么还买你的?除非你的年蛊售价更低,我兴许还会再囤积一批。

%36
但三大卖家显然不可能,以方源这样的低价来售卖自己的年蛊。

%37
三大卖家暂时都将年蛊囤积在自己的手中。

%38
但这样做,又有什么用呢?

%39
等到两三年过去,市场逐渐消化了方源这一次提供的年蛊后,又会对年蛊有需求。

%40
年蛊的价格也会提升上去。

%41
那个时候三大卖家可以出手,但方源也可以啊。

%42
只要他如法炮制,三大卖家手中的年蛊,仍旧会烂在自己的手中。

%43
这就是实力!

%44
实力差距在这里,任何的阴谋诡计都是没有用的。

%45
方源也是多智之人,但这一次他直来直去,王道碾压,也最是有效。

%46
三大卖家讨论一阵后,就都纷纷沉默下来。

%47
氛围相当的凝重。

%48
因为他们心中都清楚,只要方源的实力摆放在这里,他们就都不会有好日子过了。年蛊贸易虽然不会断掉,但是这道财路将大大缩水,至少宝黄天的市场他们只能放弃。

%49
“现在,恐怕我们唯一的希望,就是那个人展现出来的实力,是虚假的。只是虚言恫吓而已。”王明月苦笑,打破了沉默。

%50
其余两仙却是微微摇头。

%51
这种可能,非常渺小。

%52
“我们可以合作。”荣欣很快说道。

%53
这是一个办法。

%54
当初他们三人,就是相互争执,不分上下,这才定下约定,进行合作,一起获利了许多年。

%55
“但我们拿什么和这人合作?实力差距就在眼前啊。”王明月叹息。

%56
谢宝树冷静地道:“那就要看我们个人的本事了。年蛊打动不了他(她),还有其他东西呢。”

%57
“那如果他不答应呢?”王明月又问。

%58
谢宝树沉默。

%59
荣欣苦叹:“如果他不答应,一心要把我们斩尽杀绝,那我们就只要舍弃了年蛊这块的生意了。”

%60
没有办法。

%61
竞争是相当残酷的。

%62
方源这样的实力,三大卖家根本没有活路可走,只能舍弃年蛊一途。

%63
但是说舍弃,就能轻易舍弃吗?

%64
年蛊贸易带来的利润,支撑着这三人,乃至他们背后的超级势力,忽然间垮了,谁能受得了?

%65
“不!我绝不能抛弃年蛊贸易!!”谢宝树暗自咬牙,他是散修,靠着年蛊的生意来支撑自己的修行,维护自己的人脉。

%66
一下子没有了,这对他而言,是绝对致命的打击。

%67
王明月的情况,要比谢宝树好一点,毕竟她曾经是散仙,现在已经转为正道,加入超级势力。

%68
她这边没有收益,也不是她的责任。另一个方面,她可以依靠她的丈夫,还有背后的家族。

%69
荣欣则压力最小。

%70
因为他也是超级势力的一员,而且是中洲的超级势力,实力雄厚。

%71
并且更关键的一点,他得到年蛊的方式,是通过炼蛊得来的。他不需要年蛊的话,只要停止炼蛊就行了。可谓船小好调头。

%72
“想要和我合作?”方源很快就得到了三大卖家传达过来的消息,不禁冷笑一声。

%73
为什么要合作?

%74
明明可以自己独自吞下的东西,干嘛要分给别人呢?

%75
方源毫不犹豫地回绝了他们。

%76
一千两百万块仙元石!

%77
这是方源此次商战的收获。

%78
虽然价格很低,但是销售的量实在太大了,薄利多销。

%79
“等到将来把这三家排挤出去,我的价格就可以稍微提高一点了。”

%80
“但这最近几年,三家恐怕还会坚持一段时间。”

%81
方源估计着。

%82
一方面是三大卖家不甘心,另一方面也是要继续试探方源的底细,防止他伪装夸大了自身实力。

%83
第三方面,则是宝黄天市场虽然被方源霸占住,但并非所有人都在宝黄天中购买东西。

%84
比如王明月,背靠超级势力,她所产的年蛊很大一部分,会自我消化掉。除此之外,还会有其他势力,关系过硬,仍旧会大规模地采购他们的年蛊,哪怕价格高一些。

%85
当然这种情况,只要方源不断地霸占宝黄天市场,用这样的低价排挤对手,过个五六年,三大卖家的日子会越来越难过。毕竟价格摆在这里,再硬的关系户,也会转投到方源这边。

%86
在整个蛊仙市场当中,宝黄天市场当然是占据大部分的。

%87
今后,方源只需要霸占住宝黄天市场,然后影响其他市场,最终他会成为整个蛊仙市场的年蛊生意的龙头老大。至于其他人,虽然也能做一些年蛊生意,但绝对不能和方源媲美了。

%88
“将之前的欠债归还,我还剩下一千多万的仙元石。不过两三年内,是不能依靠年蛊这道财路了。”

%89
正当方源谋算着,如何利用这笔仙元石来经营自家仙窍的时候,毛六主动找上门来。

%90
他带给方源一个大大的好消息——

%91
“方源长老,你要我炼制的食道仙蛊,终于炼成了!”

\end{this_body}

