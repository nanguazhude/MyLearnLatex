\newsection{第五次地灾}    %第一百四十三节:第五次地灾

\begin{this_body}

%1
“百足天君拥有入侵洞天的手段,但本体似乎不能,只能差遣分身。”

%2
“黑凡洞天空间广阔,黄钟地灵又被柳贯一毁了,我无法查看全局,不知道此时此刻,百足天君的分身,究竟来了多少!”

%3
“每当我要斩杀对方时,就有新的战力出现。这只能说明一点,那就是对方的分身,绝不只眼前这三个,必定有其他分身,早就成形,只是不现身参战。只有等到我摸清楚敌方手段,占据上风,就要痛下杀手时,他们才出现救场。如此举动,必定是要拖延时间!”

%4
楚度脑海中电光激闪,不妙的感觉越发浓重。

%5
百足天君故意拖延时间,必定有所图谋!

%6
楚度虽然还洞察不了他的真实意图,但毫无疑问的一点便是不能再按照对方的节奏去战斗。

%7
念及于此,楚度陡然爆发,身躯猛地膨胀,化为三丈小巨人。

%8
仙道杀招战力蒸腾!

%9
他浑身泛红,周身五万八千个毛孔张开,无数汗水升腾成蒸汽,笼罩他的全身。

%10
然后,他高举右手掌天,左手低垂握地,同时他口中低吟出声:“尝尝我这一招天地合力。”

%11
百足天君的三个分身,顿时动弹不得,感觉到天下地上传出无以伦比的恐怖力量,将他们牢牢禁锢住,然后缓缓碾压过来。

%12
楚度也不好受。

%13
他浑身上下,肌肉贲发,使出了全部劲力。汗水蒸汽中。他的脸部振奋得通红,一对眼球充斥血丝。尤其是他的一对手臂。上面青筋暴起,宛若一条条蚯蚓拱起。狰狞可怖。

%14
他的右手向下,左手向上,仿佛带着两座小山,缓缓地、艰难地向中间靠拢。

%15
仙窍中,红枣仙元剧烈损耗。

%16
伴随着楚度的这个动作,百足天君的三个分身更加不好受。他们感觉到周身的力道,急剧增大,碾压得他们几乎要粉碎开来。

%17
眼看这三个分身就要被碾压爆炸,但就在关键时刻。第四个、第五个、第六个天君分身一齐赶来救场。

%18
砰砰。

%19
两声闷响,第一位、第二位天君分身被楚度当场压爆,浑身化为齑粉,但仙蛊、仙元却消失无踪,没有被摧毁。

%20
“好厉害的杀招,攻防一体,还有强大的控制手段,几乎就是战场杀招的雏形!”第三位天君分身被救出来,仍旧是心有余悸。

%21
“再来多少的分身。无弹窗,最喜欢这种网站了,一定要好评]都是无用!接下来这招,是我专门为八转存在准备的。百足天君,你的本体若再不现身,这些分身就都救不了了。”楚度长啸一声。摆开架势。

%22
他整个人成大字型,双臂双腿都伸展开来,双手手心朝天。身上各种仙蛊、凡蛊的气息层出不穷,变化不休。单单气势就比刚刚的天地合力杀招,更加磅礴。震慑人心。

%23
百足天君的几大分身,无不稍稍后撤,催动防御杀招,全神戒备。

%24
砰砰砰。

%25
下一刻,天君分身陡然自爆开来,毫无预兆。

%26
“这是什么招数?”黑凡洞天之外,百足天君的真身本体皱起眉头。

%27
他的分身虽然尽数灭亡,但战斗记忆却是传达出来,让百足天君洞悉了一点楚度杀招的奥妙。

%28
“这个杀招,无形无色,覆盖范围。陷入其中,就会由内而外,产生一股巨大的内力,引爆自身。嗯?覆盖范围竟然扩张得如此厉害!”

%29
百足天君眉头拧成一个疙瘩。

%30
黑凡洞天中,楚度将他的这一神秘的力道杀招,直接覆盖了整个黑凡洞天。

%31
导致所有的百足天君分身,都自爆而亡。

%32
百足天君笑了笑:“有趣。看来不出动真身,是对付不了楚度的了。”

%33
话音刚落,他便催起一记仙道杀招。

%34
一道玄白蜈蚣似的光影,从他身上飞出来,钻破空间,在黑凡洞天中瞬间形成大片的扭曲光影。

%35
随后,百足天君迈开大步,踏入光影之中。

%36
“想进来?”楚度大笑一声,抬脚猛踹。

%37
一个力道大脚丫子飞了出去,打中扭曲的光影,将百足天君打退出去。

%38
百足天君愣住,旋即脸上露出一抹恼怒兼有无奈的神色:“这黑凡洞天易守难攻,我钻入进去,至少需要二十个呼吸。在这过程中,却足够楚度断我施法。送去天君分身,又会瞬间自爆,无济于事……”

%39
楚度全力防守,龟缩在黑凡洞天里头,百足天君对此就像面对缩壳乌龟,很是有种无从下手之感。

%40
“百足天君进攻黑凡洞天?”方源手中捏着楚度的求助信,吐出一大口浊气。

%41
他的担心,顿时放下。

%42
楚度虽然是他的盟友,但如今为了维护黑凡洞天,依靠洞天顽强据守,和百足天君分庭抗礼,这点对方源而言,却是有好处的。

%43
琅琊地灵要策反楚度,此事难度极大。成功了,楚度加入琅琊派,将极大地影响方源如今的地位。失败了,方源和异人联手的情报,就会曝光,后果无法预测。

%44
方源只好能拖就拖,能隐瞒一天就是一天。

%45
因为双方的盟约关系,方源不能哄骗楚度,尤其是在渡劫方面。方源又不能继续在北部冰原渡劫,如此一来,楚度自然要猜疑,这就会暴露出北部冰原之下的雪人、石人的联合部族。一个处理不好,惹来楚度攻击这个部族,方源夹在中间,处境将非常尴尬和危险。

%46
所以,结盟不能随意,很多蛊仙结盟都是慎之又慎的。

%47
方源原本打算,利用宙道杀招,延长仙窍的时间,然后告知楚度这个情况。

%48
这就不算哄骗了,毕竟是事实。

%49
然后,方源再缩短仙窍时间,在渡劫开始前一刻,告知楚度情况有变。如此一来,楚度来不及赶到这里,虽然会破坏和楚度的关系和信任,但也只能如此处理。

%50
楚度是很谋略的,和他的盟约没有什么漏洞可钻。方源能想到的,就只有偷奸耍滑。

%51
这样做,自然非常麻烦,耗费方源许多仙元不说,还会让他和楚度之间产生罅隙。

%52
总之,十分纠葛。

%53
但现在,楚度被迫防守黑凡洞天,和百足天君纠缠在一起,无法脱身。

%54
方源不禁在心中赞叹:这百足天君来得正是时候,相当于为自己暂时解决了一个大麻烦啊!

%55
“此番际遇,倒也是运气不错了。”

%56
方源按住一只信道凡蛊,开始回信。

%57
在信中,他对楚度的遭遇表示强烈的关心和同情,对百足天君的入侵表达最激烈的愤慨和仇恨。并且表示,一定会配合楚度,绝不会背弃盟约。但是!

%58
自己渡劫时期将近,得先渡过这个关口,才能帮助楚度,解决掉百足天君的这个大麻烦。

%59
希望楚度能够理解一二。

%60
楚度很快回信。他表示相当理解,并且告知方源:解决百足天君的入侵并不困难,他已经号召了其余帮手,希望方源解决了这个麻烦之后,迅速出手。

%61
方源立即回应:尽自己最大所能,尽自己最大的努力,请楚度坚持!

%62
几天后。

%63
方源离开琅琊福地,利用超级蛊阵,传送出去。

%64
随后,他马不停蹄地赶到地沟中,选择了一处不深不浅的地方,落下了仙窍。

%65
北原的地沟,主要有十七条。是由北原的僵盟掌管。但义天山大战之后,僵盟被影宗算计,被魔尊幽魂主动牺牲,用来炼制至尊仙胎蛊,导致五域各地僵盟都一夜尽亡。

%66
不过,影宗方面也有所布置。

%67
北部僵盟这块,大本营阴流城就沉没于地沟之中,除了影宗余孽之外,没有任何蛊仙能寻找到它。

%68
方源来到的这处地沟,自然不是僵盟地沟。那里有大量的散仙、魔仙身影,都是寻找阴流城的投机之辈。况且还有阴流城隐藏,说不定影无邪等人,就偷偷跑到这里去了。方源除非是傻,才会选择那里作为渡劫地点。

%69
大部分的地沟,都被各大超级势力瓜分,或者被蛊仙强者占据。

%70
方源选择的这条,已经被开发殆尽,舅舅不疼姥姥不爱,又经历过数十次激战,荒凉贫瘠,中部还有断层,总之人迹稀少。

%71
方源选中此地,纯粹是因为这里有丰富的土道道痕,暗道道痕的量也很丰富。

%72
仙劫锻窍!

%73
方源成功催动了这个仙道杀招。

%74
随后,他彻底敞开仙窍门户,天地二气汹涌而入。

%75
不久,第五次地灾成形。

%76
天空晦暗下来,一只巨大的骷髅头黑影,占据了大半天空。

%77
骷髅头桀桀大笑,张口一吐,吐出一股黑蟒般的暗流。

%78
暗流穿梭,速度飞快,所到之处,沙硕土壤无不被侵蚀得面目全非,顷刻之后,尽数消融,化为一滩滩的黑色腐水,散发阵阵恶臭。

%79
地灾腐蚀暗流!

%80
方源认出地灾跟脚,连忙出手应对。

%81
腐蚀暗流十分难缠,方源其实也没有什么针对克制的方法,只能是用攻伐杀招强硬对抗。

%82
骷髅头张口连吐,一股股腐蚀暗流,仿佛是群蟒来袭,让方源疲于奔命。

%83
天意将灾劫威能提升到了极限,方源狼狈不堪。

%84
这一次地灾,没有狂蛮真意的参与,导致全部受到天意的掌控,唯纯唯一,威能比前几次都要浩大恐怖。

%85
不过,方源实力暴涨,东海信道传承就算了,黑凡真传他全部得到,如今各种宙道仙级杀招催发出来也是有模有样。因此他虽然狼狈,疲于奔波,但一直留有余力,以防其他不测。

\end{this_body}

