\newsection{你心里还没数?}    %第七百四十节:你心里还没数?

\begin{this_body}

%1
方源算了算时间。

%2
还蛮充裕的!

%3
尽管他已经将蒙屠、睡姑都杀了,还跑了一趟光阴长河,获取了红莲真传,更远去南疆,将那只计划中的野生仙蛊拿了来。

%4
做了这么多的事情,此刻距离上一世张继贩卖浮生火,竟还有几天的空余。

%5
“主要还是因为上一世,我为了推算阎帝、阎罗子这些杀招,耗费了几个月的时间。”

%6
“而这一世重生,我直接就出发了,节省了许多时间、精力。”

%7
接下来的几天,方源当然没有闲着。

%8
借助未来身,他不断地从琅琊福地传送出去,足迹遍及五域,其中进入南疆的次数最多。

%9
方源专门去往的这些地方,都是地沟成形之处。

%10
前世五域动荡,地脉合一,各个地方都陆续开始出现地沟,并且这种现象越发猛烈。直到最终一战,方源战死在龙公手中,五域界壁彻底消弭,地脉一统,地沟成形的现象这才戛然而止。

%11
几天下来,方源收获十分丰富!

%12
仙材堆成了小山不说,还收取了三只野生的六转仙蛊。

%13
方源运气比起上一世,还是差了许多,但是获取这些资源,并不是难事。

%14
因为他熟知情报!

%15
地沟的成形,乃是天意,任何蛊仙都无法彻底揣度。每一条地沟成形,都纵横数万里,数十万里,并且地沟深邃至极,想要探索清楚,很不容易。

%16
就算是超级势力,想要做到短时间内探索地沟,搜刮到最有价值的资源,非常困难!

%17
方源却是有的放矢,早就知道哪里有收获,直奔那里而去。

%18
这都有赖于前世,他特意收集这方面的情报。

%19
地沟的成形,根本没有规律,对于绝大多数的蛊仙,这些情报毫无价值。但是对于方源,掌握着春秋蝉的他而言,这就完全不一样了。

%20
“上一世,我没有涉足地沟,因为单靠我个人能力,想要和超级势力比拼,非常困难。”

%21
“最关键的是天庭带给我极大的压迫,逼得我不得不面对,绞尽脑汁去图谋红莲真传!根本没有这个时间和精力去顾及地沟。”

%22
“但现在看下来,地沟的馈赠真的很丰富,叫人心动。”

%23
“这还是现在,接下来的几年,地沟成形、地脉动荡的现象会越来越猛烈,呈现出来的资源也会越多。”

%24
“而资源上的井喷,必定会带动蛊仙实力上的暴涨,实力上涨更能激发出蛊仙的野心!上一世,四域蛊仙齐攻中洲天庭,这些地沟中的资源在幕后起到了重要作用。五百年前世,之所以发生五域乱战,五域动荡不定,地沟资源和梦境带来的境界暴涨,都是主要缘由。”

%25
余暇时刻,方源都会分析局势,分析自己的处境,不断优化自己的重生大计。

%26
他这一次重生,和之前有许多的不同。

%27
其中主要一点,便是他如今的实力和影响力,举足轻重,远远超越之前的任何一次重生。

%28
这样的实力和影响力,导致他重生归来,墨水效应更大。

%29
方源一面出手,严格执行着自己的大计,一面也在全身心地戒备可能出现的问题。

%30
不出方源所料,问题真的出现了!

%31
并且还直中方源的要害。

%32
几天后,方源观察着宝黄天,却没有发现张继的影子。

%33
按照上一世的时间,张继已经开始贩卖浮生火了!

%34
方源连忙调查,幸运的是,有关张继的情报并不难以获得。

%35
“黑天寺的蛊仙张继在比试中,败给了仙鹤门的残阳老君了啊。因此在外出探险时,二仙意外遭遇到的浮生火,就都落到了残阳老君的手中了。”

%36
仙鹤门、黑天寺都是中洲十大古派。

%37
论门派势力,仙鹤门要弱于黑天寺,但是要论蛊仙个人战力,当然是残阳老君要强于张继。

%38
黑天寺的张继,虽有七转修为,但普通平凡。而残阳老君可早就声名远播,他的招牌杀招追命火,令无数蛊仙头疼不已。几年前,他还和凤九歌一道,进入北原,调查八十八角真阳楼倒塌真相。

%39
张继败给残阳老君的消息,很快传播出来。

%40
方源不用猜,单用脚趾头都能想到:这明显是仙鹤门在背后推动。

%41
仙鹤门弱于黑天寺,自然要珍惜每一次机会,如今便借助这场胜利来提升门派声威。

%42
残阳老君的胜利,方源没有意外,令他感兴趣的是,残阳老君为何会出现在这里呢?

%43
上一世,根本就没有残阳老君的影子,张继是平平安安,顺顺利利地收获了浮生火。

%44
“有点不太妙啊。”

%45
“和张继不同,残阳老君可是炎道蛊仙,浮生火对他是有用的。”

%46
“就算无用,残阳老君也可上缴给仙鹤门。他和张继不同,实力比张继强,在门派中的地位更是张继不能比的。因此上缴了浮生火,门派必有反馈,残阳老君的私人收益也不会低。”

%47
方源有些头疼。

%48
他不得不开始设计其他方案,看看有什么其他的途径来获取浮生火。

%49
不过就在这时,浮生火忽然被残阳老君挂在宝黄天中公开售卖了。

%50
方源微微惊愕了一下,便旋即明白了残阳老君的用意。

%51
或者更准确的说,是仙鹤门的用意。

%52
仙鹤门看来十分重视这次良机,想要借机大肆宣传,提升门派声威。残阳老君贩卖浮生火,和方源挂卖雷鬼真君的三根胸骨,出发点都是一致的。

%53
明白了这一点后,方源便加入了追逐浮生火的买家之中。

%54
浮生火是稀缺的仙材,售价很高,但并不离谱。

%55
上一世张继不识货,让方源讨了一个便宜。但这一次却是专修炎道的残阳老君,对浮生火的价值心知肚明,售价自然不一样了。

%56
“看来只好付出点代价,将浮生火拿下来了。”

%57
方源心里有所准备,但接下来的竞价,则再一次出乎了他的意料。

%58
购买浮生火的买家,似乎相当的狂热,自发的相互竞价,当场将浮生火的售价炒上了天。

%59
浮生火的价格虚高,但竞价的蛊仙却乐此不疲,吵得面红耳赤,言语上交锋的小小矛盾很快就上升到了面子和尊严的高度。

%60
方源恍然:这恐怕都是仙鹤门的托儿,特意炒出高价,自家卖自家买,付出一点对宝黄天的手续费,却把握住了更好的宣传良机。

%61
试想一下,这浮生火的售价这么高,自然会引发轰动和后续的广泛议论。

%62
谈论浮生火的蛊仙们,能不谈论一下残阳老祖吗?于是,自然而然地涉及到仙鹤门,更多的蛊仙就会发出这样的感慨:仙鹤门到底是十大古派之一,底蕴深厚啊。别看最近几天似乎不行的样子,恐怕是人家刻意低调也说不定呢。

%63
想通了这些,方源不禁苦笑。

%64
浮生火对他而言,十分重要,就算价格如此虚高,他也只有捏着鼻子买下。

%65
毕竟一来,他的其他方案都不太可靠。二来,眼下的时间才最关键,耽误了功夫,接下来会影响到方源整个重生的计划。

%66
得了浮生火后,方源就将收集到的八大主材,都交给琅琊地灵。

%67
琅琊地灵辨识之后,确认仙材无误,没有什么猫腻之后,便和方源以及其他的毛民蛊仙一起运用炼道杀招,不断地揣摩和熟悉这些仙材。

%68
如此准备了几天,琅琊地灵见时机成熟,便带着方源和一干毛民蛊仙,来到长毛炼道大阵前。

%69
“好一座炼道大阵啊!”方源发出赞叹。

%70
琅琊地灵笑了笑,满脸得意的神情无法遮掩,他的双眼中闪现自信的光:“阵好,仙蛊方也好。方源长老,你提供的这张万我仙蛊方,端的厉害。我揣摩良久,方才明白当中的大半奥妙。根据我的估计,按照这张万我仙蛊方炼制万我仙蛊,成功率能高达四成!”

%71
方源哈哈大笑:“正是如此,我才想先炼出此蛊来,增强我的战力,好对付天庭。好了,闲话不多说,我们入阵炼蛊吧。”

%72
谈到这里,琅琊地灵不禁眉头微皱:“方源长老,你炼道缺乏经验,真的要亲自炼蛊?”

%73
方源点点头:“这一次炼蛊,非同小可,对我而言,意义更是重大。我想亲自参与,即便是炼蛊失败,也会心平气和。太上大长老勿忧,我只是辅助,绝不胡乱插手。这点自知之明,我还是有的。”

%74
“也罢,既然你如此执意坚持的话。”琅琊地灵叹息一声,当先入阵,进入第一主位。

%75
随后,毛三、毛四、毛六、毛七等等蛊仙也鱼贯而入,最后才轮到方源。

%76
方源占据最末尾的阵眼当中。

%77
大阵徐徐催动起来。

%78
光芒四射,各种炼道手段接连催动,炼蛊过程顺风顺水。

%79
浮生火、刃蛊这两大主材,先参与炼制,随后是战场血沙、心头血等主材。

%80
期间方源出手了几次,大多数都是处理一些琐细小事,一如他之前和琅琊地灵担保的那样。

%81
几天后,琅琊地灵和毛民蛊仙都纷纷流露出喜色。

%82
仙蛊已有雏形。

%83
距离大功告成,只剩下最后几步了。

%84
忽然,光晕爆散开来,烟尘滚滚,恶臭漫天,炼蛊出现了变故。

%85
“怎么会这样?!”琅琊地灵等人大惊失色,连忙出手补救,却是越忙越乱。

%86
眼看就要失败,琅琊地灵等人束手无策之时,方源提议临时搭建一个蛊阵,来辅助炼道大阵,将炼蛊残渣排出去。

%87
别无他法,琅琊地灵只能死马当做活马医,采纳了方源的建议。

%88
阵中之阵一建成,方源便道:“还请太上大长老入阵!”

%89
琅琊地灵不疑有他,因此此情此景的确非他出手不可。

%90
但他一进入阵中之阵,竟被立即镇压!

%91
“方源,你想干什么?!”琅琊地灵惊怒交加。

%92
“琅琊地灵,到了此刻,你心里还没数么?”方源却是淡淡一笑。

\end{this_body}

