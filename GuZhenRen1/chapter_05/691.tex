\newsection{左路无极}    %第六百九十四节:左路无极

\begin{this_body}

天庭。

原本的彻白天空中,却是闪烁着七彩霞光。

一座巨大的仙蛊屋横霸高悬。

好一座仙蛊屋!

它是三层圆坛,恢弘大气。坛上望柱重重,白玉栏杆间总有霞光映照。

正是八转运道仙蛊屋劫运坛。

当年,巨阳仙尊亲创,掌管的三大仙蛊屋之一。

原本劫运坛是镇压长生天的底蕴,但这一次长生天方面进攻天庭,竟是将它都出动参战了。

如此举措,显示出长生天志在必得的信心,这令天庭蛊仙都多少有些心事重重。

紫薇仙子眉头紧皱。

就连龙公此刻,双眼中也有凝重之色。

不过,他的目光更多的不是停留在劫运坛上,而是关注着天空中绽放的七彩霞光。

龙公沉声道:“单靠劫运坛,长生天蛊仙的实力,并不能直接突破进来。这层七彩霞光确确实实乃是九转级数,应当是巨阳仙尊的手笔!”

“巨阳仙尊?”紫薇仙子惊愕,旋即脸上浮起一抹恍然之色,“根据天庭记载,巨阳仙尊曾经受邀,来到天庭一游。难道这就是他当时暗中留下的手段?”

“应当便是了。”龙公道。

紫薇仙子顿时脸色阴沉:“好一个巨阳仙尊,真是奸猾狡诈。早就计划图谋我天庭,算计竟如此深远。”

“历代的尊者都是拥有着大智慧,绝不容小觑。巨阳仙尊掌握运道手段,暗中留下伏笔,直到发动之前,我天庭都不知道。”龙公摇头叹息,心情有些复杂。

当初巨阳成尊,却是运道成就,对宿命蛊大有干扰。当时的天庭领袖金箍大仙为天下计,特意邀请巨阳来天庭一游,表面上向天下宣扬,要请巨阳入主天庭。

巨阳考虑之后,答应受邀,因此进入天庭一游。

金箍大仙亲自引导巨阳仙尊,一路走过,带着他先后参观天庭要地,尤其是当初无极、狂蛮、红莲三大魔尊战斗和止步之地。

巨阳仙尊提议,要去看宿命蛊。

金箍大仙没有拒绝,便带着巨阳一同上了监天塔,见识了宿命蛊。

巨阳下塔后,一路沉默。

最后离开天庭,说对入主之事,要多加考虑。

这番经历,还有巨阳和金箍二人的对话,都明明白白,一字不落地记录在天庭的史记上。

“九转之威,深不可测,巨阳仙尊当年做了手脚,到现在我们才发现,其实并不奇怪。毕竟他修行的可是运道。我们天庭对运道了解的真的太少了。不过,幽魂魔尊更是在天庭内部安插了奸细,如今他已经身陷牢笼。巨阳仙尊早已经陨落,现在只是劫运坛和一群北原蛊仙前来攻打,我大天庭怎会被轻易颠覆?”大殿中回荡着正元老人的声音。

紫薇仙子点点头,但眉头仍旧皱着:“劫运坛不是寻常八转仙蛊屋,可是曾经巨阳仙尊之物。可惜了,这一次凤仙太子没有加入此行,否则我们必定能提前得知长生天的突袭计划。”

“更关键的是,劫运坛中究竟藏有多少兵力?”

“嗯……等等,我估算不会太多。因为对于北原长生天的突袭计划,我们根本就没有得到任何风声。长生天能瞒得这么紧,显然是保守,消息一直局限在少数人之间。”

紫薇仙子迅速分析着。

“你说的不错,极可能便是如此。”龙公微微点头,他虽然从座位上站起身来,但此刻却是背负双手,并未动身出战,而是等待着什么。

几乎在下一刻,整个天庭忽然狠狠一震。

呼的一声,宛若大风咆哮,在所有人的内心深处刮过。

龙公、紫薇仙子、正元老人毫无意外之色,脸上均是浮现出一抹微笑。

“元始仙尊遗留下来的杀招发动了!”

劫运坛中,长生天的数位蛊仙也没有感到丝毫意外。

从巨阳仙尊留下的情报中,早已经叙述了这一点。甚至就算没有巨阳仙尊的叙述,长生天也有数道魔尊真传。不管是狂蛮、无极都是地地道道的北原人,他们攻打天庭的情报,长生天也掌握着不少。

“这就是气墙杀招?”

“气墙杀招本来普通,但元始仙尊年轻时就非常擅长,也非常喜爱。”

“等到元始仙尊无敌天下,这道气墙杀招也提升到了九转层次!”

长生天的蛊仙一边议论,一边操纵劫运坛,不断轰炸周围。

周围空气泛起阵阵涟漪,好像是无形的堡垒坚强,只有被轰炸得狠了,这才猛地爆散开来,还原成原来的空气。

但是很快,它就又凝聚起来,形成无形的巨大墙壁。

整个天庭中,全部的空气都凝聚如强,抵挡着长生天的入侵。而对天庭成员而言,当他们行动的时候,气墙就都还原成空气,毫无阻碍。

长生天的蛊仙们尝试了一阵,眉宇间都暗藏着一抹震撼。

“不愧是元始仙尊的手笔!”

“寻常普通的气墙杀招,在元始仙尊手上,已有九转威能,比之天罡气墙还要坚固。”

“那我们怎么办?”五行**师面色不佳,十分担忧。他刚刚加入长生天不久,而获悉自己要进攻天庭,也才在数天前。

五行**师的担忧很有道理,因为若是破解不了此招的话,他们一路前行就都要轰炸气墙。

如此一来,消耗就太大了。估计都闯不到龙公面前,就都仙元干涸,疲惫欲死了。

但若要破解此招,实在是困难。

首先这是九转杀招,非得有同级别的手段,才会大有希望。八转级数除非是能特意克制,方能有效果。

其次,天地变异,气道早已经式微很久很久。长生天一行数人,根本就没有人修行气道。

这该如何是好?

“法师勿忧,我们已有对策!”

“此招虽然是元始仙尊的手笔,但这么多年过去,历经沧海桑田,又承受过三大魔尊的进攻,早有弱点了。”

“没错。当年无极、狂蛮、红莲三大魔尊进攻天庭,也不能破解这气墙杀招。不过他们都凭借一身强悍实力,直接突破。”

“他们一共闯出三条路径,这三条路线就是气墙最为薄弱的地方,无法修复。我们只要轰炸一小段,直接进入这三条路径之一,就可沿着魔尊之道,一路厮杀过去了!”

五行**师顿时松了一口气:“原来如此。”

旋即,他又问道:“那我们该选择哪条路径?”

与此同时。

中央大殿之中,龙公和紫薇仙子交谈:“气墙杀招虽强,但早已经暴露。长生天方面既然突袭我处,必然知晓气墙的弱点,就是那三位魔尊曾经闯荡的路线。”

紫薇仙子点头,旋即如数家珍地道:“左边一条路径,乃是无极魔尊突破出来,图经中天门,名牌宫、太阳宫、五神殿、中央大殿、监天塔,忘道湖、最后终止在星驰山最巅峰的一缺抱憾亭。”

“中央一条路线,是狂蛮魔尊厮杀出来。经过仙帝苑、蕴空阁、须弥泽、恒沙洞、百万天王画廊、绣楼、中央大殿,止步于监天塔。”

“右边那条路线,是红莲魔尊的入侵路线。他先是经过荷仙池、明月宝光殿、然后是十二道灵关、离火亭、星磁岭、水木清华苑、中央大殿,止步于监天塔。”

“只要他们选择出来,我们就能知晓他们接下来的路径,进行布防了。”

紫薇仙子眼眸中,闪烁着智慧的灵光。

“正是如此。”龙公没有异议。

轰轰轰!

劫运坛载着长生天数位蛊仙,一路狂轰滥炸,终于闯进气墙的薄弱之处。

一进入这里,长生天一行人顿时感到压力暴降,气墙几乎名存实亡!

天庭蛊仙顿时看出端倪。

龙公和紫薇仙子对视一眼:“他们选择了左路。”

ps:大纲调整实在太耗费时间了,不过终于在今天赶出了一章。8)<!--80txt.com-ouoou-->

\end{this_body}

