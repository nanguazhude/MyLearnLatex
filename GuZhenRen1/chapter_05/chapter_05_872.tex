\newsection{除了我还能有谁?}    %第八百七十六节:除了我还能有谁?

\begin{this_body}

%1
接引岛并不大,群仙探无可探,毫无收获,很快就集中到了小岛中央。

%2
在小岛中央,他们有了震惊的大发现!

%3
一座高耸的方尖碑,通体宛若黄金制作,散发着柔和晶莹的微光。

%4
八转仙蛊屋——功德碑!

%5
功德碑的表面,有文字不断浮现,介绍了这座接引岛,也介绍了乐土真传的详情。

%6
要获得乐土真传,就必须接取方尖碑上的各种任务,完成之后获得功德。功德越多,能够兑换的东西便越有价值。

%7
若是不接受这个条约,蛊仙们就只能枯坐在这座小岛上,不能外出,去不得洞天世界的其他地方。

%8
当然,乐土仙尊慈悲为怀,不会让这些人空来一趟。小岛上的种种仙材,就是特意为他们准备的,给他们瓜分。

%9
三百天后,所有人都会被传送出去。

%10
群仙哗然。

%11
“小岛上真的是有仙材的。那些登云梯叶心、清风草茎、雄鸡铁等等,都不是假的。”

%12
“它们都被人取走了!这天杀的,搜刮的真够彻底,根本不给我们留一点啊。”

%13
“恐怕就是那个楚瀛了。”

%14
“楚瀛在哪里?怎么还不见他人影?”

%15
“我发现楚瀛了!”花蝶女仙忽然叫道。

%16
群仙连忙走到她的身边,来到功德碑的另一面,顺着花蝶女仙手指的方向,他们看到碑面上逐渐浮现的文字。

%17
功德排行榜——楚瀛,八十二功德。

%18
除了楚瀛之外,就再无他人。

%19
“这小子!”任修平咬牙切齿。

%20
就在这时,楚瀛的功德值又猛地跳动,从八十二变成了九十三。

%21
随后,功德碑猛地一闪。

%22
群仙连忙后撤几步。

%23
光影迅速消散,现出方源来。

%24
“楚瀛!”蜂将低喝一声。

%25
“你终于舍得出现了。”土头驮双眼通红,他想到了那片巨大的雄鸡铁的坑洞。

%26
“哦,你们醒了啊。”方源对他们点点头,神情平淡。

%27
“喂,你这是什么态度!”沈潭气愤不平。

%28
方源无所谓的态度,让群仙怒气腾腾。

%29
沈从声踏前一步,神情冷酷,气势凌人:“说吧,岛上的仙材是不是都被你取走了?”

%30
“除了我还能有谁?”方源反问一句,对沈从声的逼人气势无动于衷。他旋即就走到功德碑的另一面,自顾自地浏览碑上的种种任务。

%31
群仙怒瞪双眼。

%32
这小子,吃独食,还这么嚣张!

%33
任修平阴测测地道:“碑上的文字我们也看了,这岛上的种种仙材乃是乐土仙尊的布置,专门留给大家的。楚瀛,你一个人就拿走了全部,太过分了吧?你该给我们一个交代!”

%34
方源看都没看他一眼,将手掌往碑面一贴,又接取了一个任务。

%35
刷。

%36
白光一闪,他消失不见了。

%37
群仙愕然。

%38
沈泣气急大怒:“这小子!”

%39
“这小子从哪里冒出来的?根本不把我们,更不把沈从声大人放在眼里啊。”任修平开口,脸色铁青,声音冰寒。

%40
“出去后,好好查查他的身份。”沈从声到底是八转,沈家的一把手,此刻冷静下来,也学着方源那边,走近功德碑。

%41
“楚瀛这么有恃无恐,无非是依仗着在这片乐土中,自己不能对他下手。”沈从声暗中思虑,“楚瀛不过区区七转,一介散仙隐修,早晚能对付得了他。当下最重要的,还是这处功德碑。”

%42
方源刚刚的行为,无疑给场中众仙做了一个上好的榜样。

%43
上一世时,庙明神一行人来到碑前,还有一些犹疑,所以掉头先将岛上的种种仙材资源瓜分掉,这才回转过来接取任务。

%44
但这一世,方源提前苏醒,早就下手,把小岛搜刮个底朝天。

%45
落到众仙眼前的,也就这座功德碑了。

%46
功德碑上的任务有不少。

%47
帮助南滨海城中的鲛人,采集银湖中的金莲。

%48
鱼圆岛干旱,需要建设水井。

%49
三纹部族缺乏药材,瘟疫弥漫,需求大量药材。

%50
牧笛部族遭遇迷雾,指引他们走出。

%51
徐霞山上的元泉出现问题,急需治理。

%52
……

%53
蛊仙看着,一时间都有些愣神。

%54
和上一世一样,他们同样遭遇到了相同的困境。

%55
这些任务只是一些名目,根本没有详细的介绍,蛊仙们无法获知到实质的内容。

%56
就好比这条任务——三纹部缺乏药材,瘟疫弥漫,需求大量药材。

%57
三纹部族是什么样的一个势力?有没有蛊仙?是普通势力,还是超级势力?

%58
瘟疫弥漫,什么样的瘟疫?要知道许多瘟疫,连八转蛊仙都不敢碰触的。

%59
需求大量药材,什么样的药材?是仙材吗?又该怎么获取呢?

%60
没有人知道这些。

%61
功德碑上就这寥寥一句话而已。

%62
群仙可谓两眼一抹黑。

%63
“那个楚瀛是不是知道些什么?”有人嘀咕道。

%64
群仙默然,对于方源的怨恨和猜疑,又不由地加深了一层。

%65
虽然乐土仙尊是以仁厚著称,但这里到底是陌生环境,连八转的沈从声都被禁止杀伐,由不得蛊仙们不谨慎。

%66
再看看功德碑上显现的那些奖励。

%67
大量的仙材、仙蛊,都可以用功德去兑换!

%68
当然其中也有一些稀奇古怪的玩意。比如说一些空泛的名号,又比如说一些凡蛊。

%69
沈从声就看到其中的一只五转凡蛊,名为行善蛊,他还是第一次见识到有这样的蛊虫。

%70
要兑换一只行善蛊,需要五十点功德,是最贵的凡蛊。

%71
但具体蛊虫的功用是什么,功德碑毫无叙述,沈从声也一无所知。

%72
不过,能够被乐土仙尊放置在这里面,定然是有稀奇独到之处的。

%73
群仙浅浅商议了一会儿,看来只有实践才能出真知。首先由沈从声开始,随后众人纷纷走上碑前,接取任务。

%74
前世方源和庙明神一伙八人,来到功德碑前,一次性出现的只有十个任务。但这一世,似乎是因为来的人多了些,导致任务的数量也相应增加了许多。

%75
任务不少,一个个都接了。

%76
因为什么都不懂,也没有出现争抢的情况。

%77
沈从声接取的任务内容是:鱼圆岛干旱,需要建设水井。

%78
光芒一闪即逝,下一刻他就被传送到了一片沙漠之中。

%79
沈从声眼眸微缩。

%80
他是八转蛊仙,道痕浓郁,不是寻常手段可以传送得动的。但他被传送过来时,根本无法阻挡,一股恢弘且又巧妙的力量,就将他瞬间带到了这里。

%81
“不愧是乐土仙尊啊。”沈从声心中感慨,目光四扫,侦查杀招覆盖周围,又迅速向四周蔓延。

%82
沈从声旋即就察觉到了不一样的地方:“哦,在这里我的侦查杀招能有正常的威能了,不像刚刚的接引小岛。”

%83
在距离他数十里外,有一座沙漠中的小村庄。

%84
沈从声站在原地,双耳微动,就听见村庄中人们的交谈。

%85
片刻后,他确定下来,他身处的地方就是鱼圆岛。这座海岛面积庞大,有不少的村落,但植被在数十年前就开始锐减,现在是越发稀少,大片的沙漠已经完全包裹了仅有的几座村庄。

%86
凡人和蛊师们在村庄中生活,日子越来越难。

%87
“这么说,我是要在这些村庄附近,建设水井了?”打探到了具体情况,沈从声吐出一口浊气,放松下来。

%88
这个任务远比他料想中的,要容易得多。

%89
因为都是凡人层次,不涉及到蛊仙。

%90
就算是涉及到仙级,他可是高达八转的音道大能,大部分情况也应当能应付。

%91
沈从声又想到方源,口中自言自语:“那个楚瀛,不过七转修为,却迅速积累了九十三点功德。以此推想,他完成的那些任务也应当是容易的。”

%92
大致明白了情况,沈从声却没有急着出手。

%93
他需要试验的东西还有很多。

%94
比如自己暂时不管这个任务,能否飞出这片地方?

%95
能否沟通外界呢?

%96
自己真正动用的手段,又剩下多少?

%97
沈从声并没有去接触那些凡人,而是利用音道手段,在这些村庄的周围悄悄挖井。

%98
这里沙漠化非常严重,沈从声也是挖了许久,才从地底深处探到一些浅薄的地下水。

%99
他围绕着这些地下水,凝造出一口口深井。

%100
想了一想,他又动用手段,对井壁进行了加固。最后还在一个个的井口,浇盖上一座座茅草凉亭。

%101
“没想到有一天,我竟要做这些粗活。”沈从声笑了笑。

%102
他一共建造了三十多口井,任务对他这位八转蛊仙而言,并无多少难度。

%103
在完成的过程中,沈从声也尝试了不少东西,试探出了这片乐土的许多底细。

%104
比如自己根本出不了任务范围,只能在这片荒芜的鱼圆岛上兜转。

%105
又比如通讯的限制,别说是沟通宝黄天了,就连传讯给沈潭、沈泣这些同伴都不成。

%106
再比如他无法对普通的村民下杀手,所有的攻伐手段都会失败,令自己承受反噬。

%107
但自家仙窍是可以打开的,沈从声可以从仙窍中取出东西,但却不能随意地将鱼圆岛上的东西夺到自家仙窍中去。

%108
沈从声也从这些凡人的身上收集到了情报,他们有人出入鱼圆岛,外出航海过。在过去,也有几个村子发生械斗,出了人命的。

%109
总体而言,龙鲸乐土中的土著们自由度很高,但进来接取任务的蛊仙们,却一个个都受到严格的限制。

%110
至于是如今限制自己的,沈从声竟察觉不出端倪,就更别提破解或反制了。这让他对乐土仙尊更感钦佩。

\end{this_body}

