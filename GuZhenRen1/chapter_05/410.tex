\newsection{入河}    %第四百一十节:入河

\begin{this_body}

至尊仙窍,小绿天。

一大片朦胧的七彩梦境,静静地悬浮着,随着时间缓缓流转。

一股我意驻足于此,面目和方源一致,栩栩如生。

在我意当中,还有一小堆红枣仙元,以及蛊虫若干。

“这是目前最后一次尝试。”方源我意口中呢喃,紧接着他便开始动手。

红枣仙元迅速消耗,与此同时,自爱仙蛊、爱意仙蛊都被催动起来。随之而后的便是数只辅助凡蛊。

很快,大量的我意暴涌而出,让刚刚只是正常大小的方源我意,猛然膨胀成数丈的意志巨人。

停止催动蛊虫,方源我意稍稍休整一番。

整个我意巨人很快缩小了一成,变得更加凝实,神态灵动。

紧接着,方源我意又再次催动另外一记仙道杀招。

此招暂时无名,乃是方源近些天草创而出,专门针对梦境。

仙招迸发出璀璨的光辉,如道道利剑,刺入眼前庞大的梦境当中。

一盏茶的功夫之后,方源停止催动杀招,整个我意巨人缩水无数,只剩下婴孩大小。那一堆的红枣仙元,更是消耗到将近干涸的地步。

“终究还是失败了。”方源我意深深地叹息一声。

梦境大战,让方源收获了许多纯梦求真体。但在方源从南疆前往西漠的路途中,他牺牲了其中大部分的纯梦求真体,阻拦追兵。

而后,他手中所剩无多的纯梦求真体,因为都是残次品,都是到达时限,各自崩解还原成了梦境。

这些梦境,方源就都留在了小绿天当中。

方源的至尊福地,有着五域九天的格局。空间之辽阔,说出去可以吓坏世间蛊仙。

不过大部分的地域,都被方源经营,生机勃勃,资源重重。惟独两处地方最为空阔,一个是小赤天,一个便是小绿天。

小赤天在不久前,就被方源选中,成为他和太古年猴交战的战场。

而小绿天中,则是盛放梦境。

这些梦境,可都是被天意侵蚀,也就是说梦境并不单纯,危机四伏,充斥天意。

当天意存在于方源的至尊仙窍当中时,天意之前就能互通,哪怕至尊仙窍隔绝内外,即便方源拥有暗渡仙蛊,也会被天意察觉。

这是方源在许久之前,就已经自行参悟出来的真相。

不过,方源不久前一直身中天庭紫薇仙子施展的侦查杀招,暗渡仙蛊也没了,本来就是暴露的,所以把天意存放在至尊仙窍当中,也没有什么大不了。

方源对如何运用天意,从中榨取利益,一直都在动脑筋。

探索梦境,汲取当中的真意,提升自己的境界,这对方源而言,是有利的事情。尤其是这些梦境都是他精挑细选出来,特别适合他。

但是处于追杀当中,梦境又被天意侵蚀,种种情况,让方源根本没有从容探索的条件和环境。

“但若是自己,将梦境中的天意统统拔除出去呢?”

这个念头在方源的脑海中升腾起来,从此一发不可收。

理论上而言,这是可行的。

因为真意就蕴藏在梦境当中,使得常人探索梦境成功之后,便能汲取真意,直接晋升流派境界。

又因为天意侵蚀梦境,既然天意能够侵蚀,那么自然也能拔除。

更进一步,天意能够侵蚀梦境,我意难道就不能吗?

方源在五百年前世,也听闻过些许传闻,说某个地域有智道大能,研究出智道手段,可以利用意志或者情感,来入侵梦境,从而进行探索。

种种事例,都证明方源的这份想法,有着相当的可行性。

然而,方源在这条路上的进展却是不佳。

他运用了多少方式,乃至于设计出仙道杀招,结果都很不理想。

不仅是连天意的边角都没有摸到,而且我意侵蚀梦境的尝试,尽数失败,折戟沉沙。

“看来我对梦境方面,只是占据了重生的优势而已。真正论较起来,前生今世其实都没有对梦道有过深究。或许将来,我的智道境界突飞猛进之后,这方面才会有进展吧。”

方源冷静地认识到自己这点不足之后,便再不迟疑。

他对准梦境,开始催动另外的仙道杀招。

很快,这些梦境就都转变成了纯梦求真体!

方源继承了紫山真君的遗藏,自然也掌握着将梦境转化为纯梦求真体的法门了。当然,炼制成的纯梦求真体,也都是残次品。甚至因为缺少了某些仙蛊,导致这些纯梦求真体还比不上,在梦境大战中,紫山真君制造出来的那一批。

不过即便如此,方源也很满意了。

因为他想要的,只是将这些梦境从他的至尊仙窍中挪移出去罢了。

片刻之后,方源的至尊仙窍当中的梦境,就都转变成了纯梦求真体,被他一一取出。

然后,在方源的安排调度下,这些纯梦求真体皆被安放在早已经考虑周翔的位置上。

随着,方源令这些纯梦求真体自爆,再次化为梦境。

这些梦境构造成一道严密的防线,有阻敌良效。即便是天庭的八转蛊仙来了,面对梦境构成的防线,也要无可奈何。

布置妥当之后,方源开始布置仙道蛊阵。

此时的他,已经身处在一片无名沙漠的地下深处。

这里有一个巨大的天然溶洞,沙硕在这里凝聚成沙金,形成无数的金砂岩石,光芒一照,熠熠生辉。

不过最吸引方源注意力的,还是这溶洞的中央,那条潺潺流淌的光阴支流。

这条光阴支流的规模,并不如先前那道,不过优势在于稳定。辅助方源即将布置的仙道蛊阵,可以让方源通过这条光阴支流,前往光阴长河当中去。

没有错,这里的光阴支流,正是影宗暗地里在西漠掌握的最后一道了。

方源徐徐布置仙道蛊阵。

他阵道境界本就是宗师一级,在被天庭追杀的这段时间里,方源接触最多的也是阵道,不管是光阴放逐阵、洁身自爱仙道蛊阵还是太古年兽钓来阵,都是仙道蛊阵。这些经验大大充实了他的布阵能力。

因此,他现在布置这道仙级蛊阵,从容淡定,自信十足。

不出所料,这座仙道蛊阵在一盏茶的功夫后,就被方源成功地建设起来。

分散在周围,四处巡逻戒备的白凝冰等人,旋即被方源召回。

再次将他们塞入自己的至尊仙窍中后,方源又对影无邪下达命令:“你可以动手了。”

仙道杀招――燃魂爆运!

影无邪再次催动这记仙道杀招,因为连运之故,方源等人在接下来的一段时间里,将气运暴涨,好运连连。

方源也想自己再用一次。

不过,燃魂爆运会剧烈消耗魂魄底蕴,方源之前用过一次,如今已经没有资格再用。

影无邪乃是幽魂分魂,才有着雄厚底蕴,可以催动燃魂爆运杀招数次。不过方源也不能让他的魂魄底蕴下降得太过厉害。

因为引魂入梦杀招,也要求影无邪更强的魂魄底蕴。若是魂魄底蕴不强,会越加容易引发杀招反噬。届时一个不好,影无邪想要对八转存在引魂入梦,结果出现反噬,反会将他自己拖入梦境当中,不能自拔。

强大总是会有其代价。

世间的东西,向来都是有付出,才有得到。

仙道蛊阵在轰鸣,在它的影响之下,原本潺潺流淌的光阴支流,忽然掀起了浪涛,并且越加迅猛。

方源的至尊仙窍当中,早已经囤积了庞大规模的万我。

再加上之前的燃魂爆运,以及太古年猴。

可以说,方源已经尽全力做了准备。

他当然知道,此去光阴长河,乃是一场冒险。但是他必须得去。

仙道杀招――逆流护身印。

仙道杀招――上古年兽变!

下一刻,他变作上古荒兽,仙阵爆发出猛烈的光辉,冲入光阴支流,撕裂出一个巨大的豁口。

方源直接投身进去,一下子穿进豁口。

哗哗哗!

潮水澎湃,掀起万千波涛。

身后的豁口只持续了几个呼吸,立即消失不见。

方源的目光全被眼前的景象吸引。

光阴长河!(未完待续。)

\end{this_body}

