\newsection{宿命是什么?}    %第六百五十九节:宿命是什么?

\begin{this_body}

%1
湛蓝晴空,万里无云。

%2
枫叶城中人流穿梭,热闹非凡。

%3
自城主之子洪亭降生,已经是过去六年了。

%4
这六年来,枫叶城无灾无劫,风调雨顺,规模不断膨胀扩张,早已超越之前十多倍,乃是整个平原里的第一人族雄城。

%5
啾啾啾啾……

%6
不知从哪里飞来的一群灵雀,浑身金光烁烁,大群飞舞而来。

%7
它们灵性非凡,直奔城主府上空时,忽然砰砰爆散,化作点点金芒,洒落而下。

%8
城主府中,洪亭正挥舞着手中的小宝剑,在假山群中一边疾步穿梭,一边舞剑。

%9
一时间,不见洪亭的身影,只见剑光成团,亮银般四处闪烁。

%10
“好!”

%11
“厉害啊!”

%12
“公子爷真是天资绝世,这剑术只教了三个月,就已经大成,气象森严,真是难以想象啊。”

%13
周围的侍卫交口称赞,无不发自真心实意。

%14
城主洪铸摸了摸胡须,微笑点头。他对自己的儿子充满了骄傲,至于震惊的情绪,已经消失了。

%15
这些年来,洪亭的资质和表现一再刷新他的认知极限,让他明白自己过去对于“天才”这一个词的理解是有多么的肤浅。

%16
因为震惊的次数太多,洪铸早在几年前就麻木了,随后几年就变成了理所当然。

%17
锵!

%18
就在这时,一声脆响。

%19
银光剑影骤然消失,洪亭的身影再次出现,他手执小宝剑,将一块山石的一角直接劈下。

%20
顿时,叫好声、称赞声不绝于耳。

%21
“父亲,我总觉得这剑术意犹未尽,大有提升的空间呢。”洪亭唇红齿白,双眼如星,他走出假山,眉宇带笑。

%22
父亲洪铸大笑:“你这剑术乃是著名的剑修大师赵三思的秘籍,为父可是用上好奇珍向他换取来的。你只练了三个月,就看不上了?”

%23
洪亭眨眨眼,眼中闪过一丝可爱的狡黠:“父亲,我也是有感而发。说实话,这剑术也就这样,我练了三个月,已尽得精髓。”

%24
洪铸嗯了一声,再次感到头疼。儿子的天资太高,学什么都快,并且一学到底,尽得神髓。这点是好,但也有不好,早在几年前洪铸的底子就已经被洪亭掏空,他千方百计从外搜寻他人秘籍,来教导洪亭,但仍旧止不住洪亭黑洞般的学习能力。

%25
长久下去,这该如何是好?

%26
就在这时,他们的头顶上飞来一群灵雀,猛地自爆成金色光点。

%27
光点洒落下来,迅速融入到众人的头脑当中,其中绝大部分都被洪亭一个人吸收。

%28
一时间,众人脸上都显现出了大喜之色,他们纷纷得到了传授,掌握了千奇百怪的秘籍。

%29
“这又是仙人传法了啊!”

%30
“不知是哪位仙人出手?”

%31
“我们还要感谢小公子,都是受他的恩泽啊。”

%32
众侍卫狂热地看向洪亭,就连洪铸的目光都带着一丝复杂。

%33
洪亭则紧闭双眼,尽情地沉浸在陡然新添的丰富知识当中。

%34
洪铸等人已经见怪不怪,就连城主府外的满城子民,也只是稍稍惊异了一下,就都回复了正常的生活之中,各忙各的事情。

%35
异象、仙迹令人神往,但这种事情多了,也就变得寻常了。

%36
洪亭身上发生的仙迹,已经不是多这个简单的概念,而是多到了泛滥,多到了令人麻木的程度。

%37
从他出生之后,被龙公收为徒弟,他就时常会出现仙迹。

%38
龙公乃是天庭领袖,不提天庭中的八转蛊仙,单单天庭下宗中洲十大古派,就各个都是庞然大物,一方霸主,底蕴深厚。

%39
龙公收徒大张旗鼓,并无遮掩,所有人都明白他的意思,所以不管是中洲正道、魔道还是散仙,都十分关注洪亭的成长。

%40
洪亭是气运种子,会被天庭栽培成未来仙尊,这个早已经不是什么秘密。所以每隔一段时间,都会有蛊仙或者蛊仙的后裔出手,暗中帮助洪亭,结个善缘,混个脸熟。

%41
所以,金雀传法已经不稀奇了,还有灵鹤传书,祥云送果,圣风洗髓等等各种仙迹、异象,一直层出不穷。

%42
良久,洪亭彻底吸纳了脑海中的知识,他却皱起眉头:“这里面虽然有十八般兵器的演练之术,各个精妙高绝,但仍旧没有提前开窍的法门。父亲啊,我什么时候才能开辟空窍,操纵蛊虫呢?这些技击之法,再高超也不过是凡人之术。只有操纵蛊虫,才是堂皇正道啊。”

%43
“我儿,你不要着急。你又忘了,你师傅传信过来,是在信中怎么嘱咐的?”洪铸和颜悦色地道。

%44
洪亭不耐烦地扬手:“父亲,我知道了,我知道了。师父是说,我虽然能提前开窍,但必须得在十二岁那年才能。时候不到,机缘不至,不得提前。我很奇怪,师傅如此神通广大,为什么就不能令我现在开窍呢?”

%45
洪铸脸色微沉:“儿子,你师父乃是上仙,他想什么你不能理解,但也要听从。你要明白,他绝对不会害你的,就向我们父母对你的爱护之心一样。”

%46
“是,父亲。儿子说错话了,父亲不要动怒好吗?”洪亭连忙拱手,他对父母双亲孝顺。

%47
时光流逝,数年光阴一晃而过。

%48
龙公再次出现,将洪亭正式收为徒弟,一边行走五湖四海,一边亲自教导。

%49
洪亭十二岁那年,龙公为他开窍。他正式踏上蛊修之路,因为绝世的天资以及浓重的基础,他的修为一日千里,进展极其神速。

%50
不只是他的修为,他的阅历也在这一路的游历中,不断丰富。他见识许许多多的人,从凡人到仙人,从善人到恶人,他对人生,对天地的认知不断地加深再加深。

%51
他行侠仗义,他嫉恶如仇,他英俊潇洒,他足智多谋。

%52
“师父,我打探到了那屠村的恶人薛屠刀,就在附近的山头!”这一天,洪亭打猎而归,回到山洞,兴冲冲地对龙公道。

%53
龙公点点头,微笑道:“坐下来,我熬的汤快要好了。这可是上古荒兽的骨头汤,对你成长大有裨益的。”

%54
洪亭将手中的猎物抛到地上,咬了咬牙道:“师父,一年多前,我要向薛屠刀动手,惩奸除恶,你说我只有三转修为,而他却有五转,我不能敌,不可去。半年前,我已经有四转修为,杀招八种,取走薛屠刀性命把握极大,你又说不到十成把握,不要出手。三个月前,我已是五转修为,对付薛屠刀不过两三招的事情,你却说他命不该绝,还不到铲除他的时候。师父,这等恶人再要放纵他,该不知又坏了多少无辜人的性命。现在该我出手了吧!”

%55
龙公放下手中的长勺,不再搅拌锅中的骨头汤,叹息一声:“徒儿,良机未至啊。”。

%56
“我不管什么良机,我只知道,这一次若要再不动手,就是错失良机了!”洪亭态度十分坚决。

%57
龙公摇头:“这个时候,薛屠刀命不该绝。你若强行去杀,徒劳无果不说,还会陡生波折,造成惨剧。”

%58
“我不信!我只需一招,就能要了他的命!!”洪亭竖起一根手指,双眼紧紧地盯着龙公,目光灼灼,眼眸中好似燃烧着火焰。

%59
龙公沉默了片刻:“那你便去吧,年轻人,试试看罢。”

%60
“谢师父恩准!”洪亭大喜过望。

%61
“为师只是希望你到时候不要失望才是啊。”

%62
“我又怎么会失望呢!师父,请稍等片刻,徒儿一炷香的时间就会提着薛屠刀的人头回来的。”

%63
洪亭转身即走。

%64
但一炷香的时间过去,两炷香的时间过去,三炷香的时间再次过去,洪亭仍旧不见踪影。

%65
一切都在龙公的眼中,他知道时机成熟了,离开山洞,跨越一片林海,飞临到洪亭的身边。

%66
洪亭跪在地上,满脸都是震怒和悔恨之情。

%67
他呆呆地望着山脚,那里曾经有一处小山村,里面的村民生活祥和安宁。

%68
他双眼通红,充斥着红血丝,见到龙公,他抬头仰望,脸上还有清晰泪痕。

%69
“师父,我万万没有料到,这里居然有一处蛊仙传承。那薛屠刀就是想要继承这份传承,才秘密来此。我撞破他的计划,他就利用传承的力量来阻止我,我奋力战斗,不想战斗余波造成山崩地裂,竟、竟将小山村给彻底掩埋了。师父!是我,是我害死了山脚下的这些村民啊!”洪亭哭诉道。

%70
龙公并没有安慰,而是沉默片刻,忽道:“徒儿,你想报仇吗?想除掉薛屠刀吗?如今时候已至了。”

%71
“是吗?他在哪里?”洪亭浑身一颤,立即问道。

%72
“就在那边的山头。”龙公手指了指。

%73
洪亭立即赶过去,轻而易举地斩杀了薛屠刀。

%74
“我好不甘心啊,我明明已经继承了真传,却还未消化成果。刚刚险死脱生,虚弱无力,被你撞见!若再给我时间,我绝不怕我,我还能成就蛊仙!”这是薛屠刀临时的遗言。

%75
洪亭站在薛屠刀的尸体前沉默。

%76
龙公的身影再次浮现在洪亭的身后,默默无语。

%77
半晌,洪亭将呆滞的目光从薛屠刀的身上抽回,声调十分沙哑地问道:“师父,你说我若是听了你的安排,我就能轻轻松松地将他杀死。那个小山村的人也就不用受到牵连了是吗?”

%78
龙公没有说话,只是伸出手掌轻轻地拍了拍洪亭的肩膀。

%79
洪亭身躯一颤,又沉默片刻,方才道:“师父,我想请教您,宿命……是什么?”

\end{this_body}

