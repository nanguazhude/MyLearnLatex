\newsection{影宗新主?!}    %第三百五十节:影宗新主?!

\begin{this_body}

武庸的犹豫没有持续多久,他很快就决定,按照方源的意思来。[求书小说网www.qiushu.cc想看的书几乎都有啊,比一般的小说网站要稳定很多更新还快,全文字的没有广告。]

这是一个机会!

对于武家的良机。

武庸前段时间,一直坐镇武家,抵抗各家刁难。这种日子相当憋闷,他当然不想再过活去。

而且他也想明白了,只要自己这位八转蛊仙不陨落,武家在南疆蛊仙界中,就有立足根基,再坏也坏不到哪里去!

借!

方源要借名缰仙蛊,那就借给他。

血脉仙蛊?

也借过去。

但这种支援,绝非麻木。

武庸和整个武家蛊仙,都要详细地听取方源借蛊的理由。任何一条不合理的理由,都会遭到驳斥和拒绝。

这种认真谨慎,早就是各大超级势力的传统。

盖因仙蛊唯一,任何一只仙蛊,哪怕只是六转级数,对于任何的超级势力而言,都是底蕴上的直接增长。

仙蛊从不轻易乱借!

方源在这么短时间内,要编造出合适的理由来,其中相当的不容易。

不过最终,他还是借来了六只仙蛊!

“六只仙蛊差不多了,再多的话,你有那么多的仙元可以用吗?”武庸委婉拒绝了方源的再次请求。

六只仙蛊都积压在一个人的身上,这样的风险太大。

倒不是武家怀疑方源的身份,而是一但武遗海有个三长两短,那么这些仙蛊就都损失了,对武家的底蕴会造成严重的损害。

不要把鸡蛋都放在一个竹篮子里面,这个道理大家都懂。

方源深深地叹了一口气:“唉!兄长提醒的是,我的红枣仙元真的不多了。兄长大人,快快再借我一些仙元石,我要迅速补充仙元!”

他又顺势再捞一笔。

这次武庸没有犹豫,直接批准了。皇帝不差饿兵,没有仙元的话,仙蛊再多也没有用啊。借出了这么的仙蛊,总得要让这些仙蛊发挥一下作用的吧?

于是方源直接收获了十万枚仙元石。

武家不愧是超级势力,家大业大,可见一斑。

“节省着用。”

“这些仙元石和仙蛊,可都是借你的,你要万万谨慎小心!”

武庸细心地关照方源。[看本书最新章节请到棉花糖小说网www.mianhuatang.cc]

方源连忙应是。

仙蛊和仙元石通过宝黄天,很快,便直接输送到了方源的手中。

宝皇天轰动!

方源一到手,就开始积极汲取仙元石,转化成自家仙元,同时将那六只仙蛊都投入逆流河中去。

统统镇压!

这场战斗一结束,武遗海的身份肯定要保不住。

只需要天庭的一句话,就会引发足够多的怀疑。方源是支撑不住连续调查的,只需是搜魂手段,或者查看仙窍,方源就得露馅。

所以在落跑之前,要趁机狠狠地捞上一笔!

“可惜最大的遗憾,就是没有借来仙蛊屋啊。”方源咂咂嘴,狠狠地骗了一把武家上下,还觉得意犹未尽。

对于一个超级势力而言,仙蛊屋的意义,比仙蛊还要重要太多。

不可能直接借给方源的!

就像是地球上的原子弹等,国之重器,向来不轻假于人。

除非方源成为武庸这样的角色,掌握着武家最高权柄,身怀一座仙蛊屋,没有人会说什么闲话。

就像黑城一个人借得了仙蛊屋黑牢,那是因为黑家四大太上长老因为修炼青城纵横,有了后遗症,只能聚在一起,不能单独离开外出。黑城在这种背景下,执掌黑家最高权柄。

方源要借仙蛊屋,武庸首先想到的就是:“我这弟弟要借仙蛊屋,一旦此战之后,有借不还,他借助仙蛊屋的威能还有乔家的帮助,和我抗衡,怎么办?”

再者,仙蛊屋想通过宝黄天输送出去,并不能完整。

仙蛊屋的本质是蛊阵,没有人能够在宝黄天中布置蛊阵或者酝酿杀招。

所以,仙蛊屋要输送,就得拆分开来。拆分容易,组建就难了。

仙蛊屋是阵道最高结晶,要组建一座仙蛊屋,即便知道方法和正确的步骤,也需要阵道蛊仙的全神贯注和积极努力。

稍有不慎,就会组建失败,这和布置蛊阵失败一个道理,蛊仙不仅会承受反噬伤害,蛊阵、仙蛊屋本身,也会如此。

风险太大,能借给方源这么多的仙蛊,还有仙元石,已经是武家上下的极限了。

这还是念在方源是武庸的亲弟弟,同时在前段时间里,为武家贡献了不少。若是没有广寒峰、拜月碗还有螺母山等事情,方源也借不来这么多。

方源企图渔翁得利,让影宗和天庭死磕,自己最后收获好处。

他尽管很努力,但事与愿违,天庭龙公的强大,超出他的想象。

影宗的落败,天庭获胜,已经基本上确定了。

连幽魂本体都陷落,影宗若是有什么后手,早就用上了。

说起来,影宗的底蕴还是相当雄厚的,可惜在义天山大战中,为了炼成至尊仙胎蛊,实在是损失太过惨重!

打击太重,几乎就一蹶不振。

残余下来的几口气,还不断被方源追杀,难以顺利的发展。

“不过我俘虏了纯梦求真体,又收获了大量的天晶,上极天鹰不断地催熟。这一下,又骗来了六只仙蛊,十万仙元石。这样的收获,其实已经相当庞大了!”

方源善于捞取好处。

交战至今,他其实没有打生打死,没有出过多少力气,但是收获却是相当惊人的。

“可惜的是,暗渡仙蛊陷落进去了,此时此刻,不能仓促拔取。一旦蛊阵崩溃,我就面对紫薇仙子。”

“还有一个隐患,就是武家盟约。我加入武家,签订了盟约,还有命牌蛊、魂灯蛊,放置在武家的宗族祠堂当中呢!”

尤其是后一个问题,让方源大感头疼。

所以,他哄骗过来六只仙蛊,也是为了这个问题谋划。将来若是武家利用这层信道关系,来害方源,方源至少手中还握有武家的仙蛊。从这点上,他就有资本和武家谈判。

优秀的棋手,当然要走一步棋,算未来三步。

尽管方源现在还身陷战场,但已经开始为未来考虑。

“魔道中人,果然是阴险狡诈、贪婪无度。就算临死之前,也不忘坑害他人。方源啊,你的罪孽是如此的深重,你今日必死无疑!放弃任何逃生的想法,因为面对我,你根本没有任何逃生的希望。”这个时候,星宿仙尊的声音,再一次从超级蛊阵中传来。

紫薇仙子一定是知道了,武家借助宝黄天输送仙蛊和仙元石的事情。

没有办法,宝黄天就是这样,是个公开的市场,任何的交易都有暴露的可能。

天庭方面对于宝黄天,怎可能不关注,怎可能没有侦察?

方源冷笑:“这就是你们天庭,没有及时揭穿我的下场。看来你们是没有关键性的证据了。其实就算你们有这样的证据,又能如何?武家会相信吗?南疆蛊仙界会冒然相信吗?他们只会相信自己亲自证明的东西。这些正道势力,向来如此,每一个事情,他们都会往阴谋和政治上联想呢。”

紫薇仙子沉默,没有再说话。

她是明白了,方源已经彻底堕落魔道,且心志坚定无比,根本无法用言语来动摇丝毫。

无谓的事情,紫薇仙子选择不做,反而节省心力。

当即,她侵蚀超级蛊阵的速度,更快了一筹。

方源面色微变,暗道不好。

照这样的速度下去,上极天鹰就算催熟,也有点赶不上。超级蛊阵会提前崩溃。

他现在有了信道仙蛊名缰。

他的计划就是利用这只仙蛊,来掌控上极天鹰。

但这个方法,方源心里其实没有多少底气。

因为上极天鹰乃是太古荒兽,八转蛊仙战力,而名缰仙蛊只是七转。唯一能让方源有点希望的是,他自身的名气的确够大,希望能够借此,让名缰仙蛊超常发挥。

实在不行,他就只能放出失去控制的上极天鹰,制造混乱,自己逃跑。

此时的战场,分成四块。

第一块,也是最关键的一块,就是紫山真君和龙公的对决。他们在影宗福地交手,目前战况不明,但龙公占据上风,并且最终获胜的可能性很高。

第二块,是监天塔和左夜灰一战,谁也奈何不了谁。目前两者皆是负伤,左夜灰纵然有极强的恢复能力,也是伤痕累累。而监天塔也是边角崩溃,不复之前辉煌形象。

第三块,是南疆正道蛊仙和魂兽大战,乏善可陈。

第四块,则是方源和紫薇仙子相互之间争夺超级蛊阵的控制权。这个战场是最为隐秘的,即便是南疆正道蛊仙,也被蒙在鼓里。

在方源看来,大局已经定了,天庭已是胜券在握。这种情况下,他只有逃之夭夭,趁混乱摸大鱼的计划,已然失败。

但方源这个老魔头,就算是失败了,也收获非常丰厚。

“接下来就看自己能不能逃出生天了!”

正当方源为这件事情暗暗奋发努力的时候,接到了紫山真君的信道消息:“方源,我们的合作还在继续吗?”

方源心中一动:“当然可以继续。”

“呵呵呵,很好。我果然没有看错你!”

“我要将所有的一切,都赠给你。我的仙蛊、我的杀招、我的见识……而你不需要负担任何的义务和责任。”

“影无邪、黑楼兰、白兔、妙音、毛六,都将听命于你!”

“从今以后,你就是影宗之主!”

备注:求一下保底月票,昨天忘了说了,大家有月票的,都投一下哈!(未完待续。)

\end{this_body}

