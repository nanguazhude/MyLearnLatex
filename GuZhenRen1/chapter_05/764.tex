\newsection{第二次埋伏战}    %第七百六十七节:第二次埋伏战

\begin{this_body}

最新的情报,传达到池曲由这里时,他正站在窗前,眺望着远处的青山。

“哦?这么快就堵住了方源?看来他这一次,恐怕是在劫难逃了。”池曲由目光微动了一下。

“这方源居然在南疆渡劫,留下了线索。唉,就算是一心想要冲击八转,也不该如此莽撞啊。”

“也是……他前番击退了天庭进攻,而后又暗中与我交易,太过顺利了啊,难免有些心态膨胀,看轻我南疆正道。”

池曲由认为,方源之所以一直逍遥,是因为他太过阴险狡诈,一直都打游击,没有和超级势力真正的死磕过。见机不妙,往往就会逃之夭夭。

这一次南疆正道密谋,全程都是极其隐秘。南疆各大家族表面上看去一片风平浪静,不会无故惹来怀疑,更有智道蛊仙遮护,防备推算。

“还有陆畏因,若非有此人临时加盟,我们南疆正道恐怕也不会这么快,就追上方源。此人不愧是乐土当代传人,竟能从灾劫引发的地气,寻找到方源的位置,真是厉害。”

池曲由心中感叹着。

陆畏因插手此事,导致当今南疆正道的局势,又更加复杂了。

这一次南疆正道追辑方源,连陆畏因算起来,一共有三位八转,成功的可能性极大。

若是事成,陆畏因的威望名声也会迅速拔高,绝对会是一个需要更加重视的存在。

其实陆畏因在很早之前,就和南疆的各大正道势力多多少少的接触过。

他的南疆乐土,堪称是南疆的异人收拢所,吸纳了许许多多遭受人族破坏的异人。

这些异人在乐土中生活相当美好,根据可靠情况,有许多异人已经被栽培成蛊仙。

这令南疆人族都大为反感和担忧,但陆畏因毕竟是乐土传人,秉持当初乐土仙尊的志向,导致南疆正道也不好随意向陆畏因发难。

长久以来,陆畏因一直被排斥在南疆正道的圈子外,这是所有人的默契。

但最近这段时间以来,陆畏因出动频繁,这一次更是高调加入,主动参与追缴方源。若是事成,他的功劳不可抹灭,在南疆正道中将有一定的权威。

池曲由心底,一点都不看好方源。

尽管方源抵挡住了天庭入侵琅琊福地,但世人普遍认为,这是天庭低估了琅琊福地的底蕴。

最关键的一点是,天庭从未有暴露出雷鬼真君和陈衣阵亡的消息。

这让南疆正道蛊仙的心中,普遍弥漫着一股自信。

他们的信心也并非是空穴来风。

毕竟他们曾经就追杀过方源,把方源从南疆仿佛撵狗一般直接撵了出去。若不是中途有凤九歌插手,方源早就被打成死狗了。

池曲由更忌惮陆畏因一些,他在心中已经开始揣摩,如何在将来面对这位更具权威的乐土当代传人。

至于方源?

池曲由有些遗憾。

不管他生还是死,只要落到南疆正道手中,那么方源和池曲由的交易就彻底完结了。

这有损池家的利益!

所以,池曲由遗憾。

虽然方源曾经攻打池家资源点,损害过池家利益,但池曲由一点都不恨。

在他看来,他交易而来的梦道成果,已经完全弥补了这些损失。

池曲由更不会害怕方源被活捉后攀咬自己。

就像之前分析的那样,就算方源拿出了铁一般的证据,不容否决,那也只是池曲由和方源的交易而已。

依照池家的体量和底蕴,的确会有一些损失,但还动摇不了池家的根本。

不过若是池曲由这个时候,向方源告密,通知他南疆正道的追缴行动,那性质就又不一样了。

这就是通敌卖友!

若是被揭发出来,就算池曲由有着八转修为,池家也会遭受南疆正道的联合围攻,搞不好就会灭族。

池曲由身为一方正道领袖,早就对此中分寸掌握得极其精准。

他是不会犯这个错误的。

若他主动向方源告密,不管方源事后有没有沦为阶下囚,池曲由就等若将通敌的证据送到方源这个魔头手上。

更何况,他和方源之间向来只是利用加防备的关系,还没有好到为了方源,而冒如此风险,不惜搭上自己和池家来通敌告密的。

池曲由只是有一些担忧和惋惜。

“若是方源被擒,但愿他不要轻易地吐露出梦道的成果,好让我池家有充分发展的时间。”

“其实说起来,方源也是一个人杰。落到如此下场,也有些可惜了……若是他能崛起,将来的话,整个五域都会更加精彩吧。”

身处高位,必有过人之胸襟。

池曲由对方源即将落网的悲惨下场,的确真心感到一些遗憾。

与此同时,就在南疆的另一处。

方源被南疆蛊仙团团包围,士气如虹,气势汹汹。

为首的老妪,正是八转宙道蛊仙夏槎,她望着方源冷笑:“方源小贼,你终于还是落到我的手中了。”

方源望着她,心想:“还是老一套的台词啊。按照上一世的记忆,接下来就该你大笑了。”

方源望向夏槎的身边。

果然,下一刻,站在夏槎身旁的商家蛊仙商虎杖大笑起来:“陆畏因大人出手,果然非同凡响,真正这次就找到了这魔头!”

(方源:一样的台词啊,就是笑容比之前要浮夸了一些。当众拍陆畏因的马屁,是因为商家看中了乐土中异人特有的产出了?)

“杀,杀了这魔头,为我南疆正道报仇!”铁区中舌绽春雷,杀气腾腾。

(方源:终于碰到改台词的了,不过还是大同小异啊。)

“终于捉到方源了。”刘浩心想。他是天庭内应,此番掌握着定空仙蛊,蓄势待发,一旦方源有催动定仙游的迹象,他就得催动定空仙蛊为核心的仙道杀招!

刘浩比上一世要紧张得多。

因为上一世,他只需要催动定空蛊就可以了,但这一世因为方源提前掌握了杀招翠流珠,导致他单单运用定空蛊,不能克制得了方源,必须得用仙道杀招。

天庭当然不缺这类的杀招,就算是缺,也有紫薇仙子可以推算改良。

刘浩紧张的原因,在于这仙道杀招他还不太熟练,要在战斗中运用出来,很可能催动失败。

若是催动失败,辜负了天庭期望,那就很尴尬了!

“别紧张,有我在呢。”就在这时,另一旁的七转蛊仙传音道。

刘浩望了他一眼。

“别看我!”顿时,那位七转蛊仙传音冷喝道。

刘浩心里大翻白眼,皆因此人就是伪装了修为和面容的南疆八转蛊仙巴十八!

“有这个人在,我方可是有三位八转。此次又并非进攻琅琊福地,方源没有大阵相助……一旦开战,我应当有很多的机会,可以从容催动杀招的吧。”刘浩心中分析起来,担忧的情绪真的缓解了许多。

然而,下一刻。

轰!

大阵开启,将南疆蛊仙汇同方源,都包裹了进去。

“我擦咧!”刘浩双眼都要瞪出来,几乎要咆哮出声,“怎么又有个大阵?!”

大阵内,天下地上浑圆一体,形成一片湛蓝长空。

南疆群仙面色皆变。

“这是陷阱!”有人大叫着。

陆畏因沉默,夏槎目光越加阴冷,死死盯着方源。

“镇定!我们人多势众,区区一个方源,算得了什么?”

“不错。我们有夏槎大人在,还有陆畏因大人,不怕他任何的仙道战场!”

刘浩一愣:“对哦!仙道大阵哪有这么容易布置的?这么快就能触发,更可能是仙道战场杀招啊。不过方源有点蠢,用了仙道战场,他自己根本跑不了了。”

南疆群仙都是精英之辈,很快稳住了心神,重振士气。

这个时候,陆畏因缓缓开口:“这不是仙道战场,而是一座超级大阵。”

“宙道大阵。”夏槎附和一句。

(刘浩嘴角抽搐:说了半天,还是仙道大阵啊!)

“二位法眼如炬。”方源微笑,坦然承认,然后他大袖一挥。

门已关,现在放狗,哦不,放年兽!

嗷呜嗷呜!吼吼!呱呱!嘎嘎嘎!

一道巨大的漩涡凭空产生,呼啦啦,无数的年兽像是汹涌的潮水,直接灌注进来。

这些年兽有猴有蛇有龙有虎,形态各异,至少都是荒兽,夹杂着不少的上古荒兽。

而方源已经在原地消失。

野生的年兽智慧并不高,很快就将凶狠的目光,对准这些南疆蛊仙,露出獠牙,展开冲锋。

“杀了它们。”夏槎冷漠下令。

陆畏因叹息一声。

南疆群仙纷纷开火,和这些年兽展开厮杀!

刘浩大袖一挥,无数飞刃四下乱射,所到之处,年兽无不被削得支离破碎。

(方源身处阵眼,静静打量:你这个天庭内应果然也在,不枉费我一直注意墨水效应。定空蛊应该就在你手上吧?)

刘浩一边战斗,一边心中思量:“看来这里恐怕就藏着一道光阴支流。方源这是故技重施,此座大阵和当初他对战凤九歌的,何其相似!奇怪,我怎么一直心里毛毛的?”

\end{this_body}

