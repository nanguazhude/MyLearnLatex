\newsection{水文蛊}    %第五百七十七节:水文蛊

\begin{this_body}

%1
窗外月色正明,而书房内,烛光微凉。

%2
武庸在书房内踱步。

%3
他心中有着忧思烦愁,放置不下。

%4
今时今日,武庸修为八转,掌握武家权柄,携玉清滴风小竹楼以一己之力,帮助武家渡过难关。又向天庭索回南疆各大正道的仙蛊,声威无两。这段时间,他又励精图治,积极培养蛊仙种子,使得武家蒸蒸日上,有什么能值得他烦忧的呢?

%5
方源!

%6
伪装成武遗海,连武庸都欺瞒过去,哄骗了武家许多资源,甚至是仙蛊。不仅是把武家上下耍弄得团团转,而后武庸追杀,都没有任何成果。

%7
方源就像是武庸心头的刺,每当武庸想起来的时候,心中都会被刺痛一下,感受到耻辱、愤怒、仇恨。

%8
此次,方源忽然活跃在南疆,捣毁掠影地沟中的大阵,尽夺梦境,武庸看上去云淡风轻,实际上内心已经被这根刺,刺得鲜血淋漓!

%9
“商家太上大长老的法子,的确是能针对方源的定仙游。但是……”

%10
武庸皱起眉头,一边踱步,一边摇头。

%11
他总感觉不牢靠,单靠这样的方法,真的能够解决掉方源吗?

%12
想到自己曾经联合南疆正道势力,一起追杀方源,毫无成果。又想到天庭方面,屡次缉拿,方源仍旧逍遥在外。如今南疆正道联合起来行动一次,就能对付得了方源?

%13
武庸摇了摇头。

%14
他承认,一方面他是有些忌惮方源了。方源这个魔头,每出现一次,实力上都有飞跃式的进步,实在不可用常理揣度。

%15
而另一方面,武庸是对整个南疆正道不自信。

%16
武家乃是南疆正道之首,武庸在武独秀的栽培下成长,怎么可能看不出南疆正道的弊端!

%17
那就是各有心思,山头林立,矛盾重重,不到万不得已,根本不会有一个统一的领袖。

%18
哪怕是武家,哪怕是武独秀健在,和武庸联手,也不能承担起领袖的位置。

%19
武庸完全可以想象,真的等到行动的那一天,南疆正道蛊仙之间必定相互戒备,各自推诿,有风险退让,有好处争抢。这样的队伍,怎能对付得了狡诈如厮的方源?

%20
尤其是现在,只有池家遭受了方源的劫掠,其余各家没有切肤之痛,又值地脉翻动,争相夺取地脉精华的时候,各家怎可能真正花费力量,来用在方源身上?

%21
“或者,方源也看出此点。所以此次只抢夺池家的资源!”

%22
武庸蓦地心中一沉。

%23
从这一点上来看,他倒是巴不得方源大抢四方,惹得群雄皆怒。但可惜方源仍旧是这么狡诈,只得罪了池家一方。

%24
若是让武庸知晓,方源不光抢了池家,还暗中和池家做了买卖,为将来五域乱战布局南疆,针对天庭,不知又作何感想。

%25
左右踱步半晌,武庸终于下定了决心:“还是要和天庭联络,借助天庭之手,务必尽早铲除方源!”

%26
做出这个决定,是相当艰难的。

%27
尽管这个决定是明智的,敌人的敌人就是朋友,但是却犯了忌讳。五大域除了东海之外,都很排外,南疆尤其如此。

%28
若是让南疆各大势力以及蛊仙们知晓,武庸居然要联络外域势力,借助他人这里,来插手南疆事宜,那必定会被口诛笔伐,汹涌的舆情将淹没武庸,哪怕他是武家的太上大长老,八转修为,也不例外。

%29
可以说,武庸要做这事,是冒着很大的风险。一旦暴露,有可能武家当前的大好局势,就会沦丧大半。

%30
“哦?武庸主动联络我?恐怕是为了方源这魔头罢……”紫薇仙子正在休养,不过见武庸主动联络,她微微一笑,立即回应过去。

%31
武庸身份特殊,乃是武家太上大长老,对他施加影响,哪怕没有任何成效,单单合作本身,就是对他的一种钳制。

%32
双方联系几句,武庸直奔主题,谈及到方源身上。

%33
紫薇仙子心思:“果然不出所料。方源拥有定仙游,又经过我在宝黄天中,暴露了他种种底牌,武庸更加忌惮,甘冒风险,也要铲除方源。”

%34
天庭的近况,并不太好。

%35
不久之前,他们第二次突袭琅琊福地,预期的目标没有达成任何一个。智慧蛊被方源带走,方源逃出生天,琅琊福地被长生天保下,反而天庭中的雷鬼真君战陨。陈衣、紫薇仙子本人以及凤九歌,都身受重伤,目前都在休养。

%36
但就算如此,紫薇仙子在得知武庸的意向后,没有丝毫的犹豫,立即答应下来。

%37
方源是必须要铲除的!

%38
就算他有七转定仙游,此蛊也是有手段克制的。

%39
退一万步,哪怕是没有除掉方源,但不断逼迫他,令他没有时间发展自己,对于天庭而言,也就成功了!

%40
“方源的成长速度,太过可怕了!这还是天意不断施加影响,布置坎坷之下的结果。每隔一段时间,当他再次露面,他总会有惊人的进步。就算杀不死他,也要令他无暇修行。若他拥有什么新手段,及时地将这些底牌都逼迫出来。”

%41
紫薇仙子心中对方源有着更深的忌惮。

%42
为了追杀方源,紫薇仙子下了多少的苦功,现如今琅琊福地差点都被她攻下,但方源呢?

%43
仍旧逍遥法外!

%44
虽然天庭方面,仍旧占据绝对的优势。不管是宿命蛊的修复,还是大梦仙尊的栽培,亦或者是魔尊幽魂的囚禁,都是方源难以企及的优势。

%45
但是,紫薇仙子的内心最深处,也开始惶恐不安起来。

%46
这种惶恐,近乎莫名其妙,但就是惶恐。

%47
因为方源而引发的惶恐不安!

%48
“此次南疆正道联手,对付方源,这是个机会。哪怕不顾伤势,我也要拼尽全力,斩杀方源,除掉这个祸端。”紫薇仙子眼中闪过一抹坚定之色。

%49
南疆正道合谋方源,因为武庸的主动联络,天庭方面在行动还未开始时,就已经早早插手,隐于幕后。可想而知,一旦天庭方面出手,必定是石破天惊的打击。

%50
但可惜的是,方源也早已料到这一点。

%51
和池曲由交易之后,他就龟缩起来,一门心思地探索梦境。

%52
这是一片宙道梦境。

%53
在这梦境当中,方源化身成一位老者,正是梦境主角,修为有四转。

%54
此时,却是被五六位神色不善的蛊师围住。

%55
“柏廉大人,这是预付的医治酬金,还请您跟我们走一趟吧。”为首的蛊师也有四转修为,望着方源,语带威胁。

%56
方源先是视察了自家空窍中的蛊虫,都是宙道蛊虫,但长于辅助,并无医疗之能。想了想,方源便试探着回绝道:“老朽只是一个普普通通的宙道蛊师罢了,非是医师,诸位这样请老朽,岂不是强人所难?”

%57
对方果然回应道:“哼,让你去你就去。我中洲蒙家的请求,你敢不答应?你的底子我们早就查的清清楚楚,放心吧,既然是请你去,就是看中你的本事,知道你有手段能治好我家蒙逼大人的病。走吧,还要我们再请你一次?”

%58
“那就去吧。劳烦带路。”方源假意苦笑一声道。

%59
跟随这些蛊师,方源很快来到郑家,见到了病人蒙逼。

%60
只见此人面黄肌瘦,骨瘦如柴,躺在床上,虚弱无力地直哼哼。

%61
“怎会如此?是如何得病?”方源试着问。

%62
旁边一位小仆便哭诉道:“想我家老爷平日健硕如虎,奔走似狼,没成想竟落到这步田地。他是受了那该死的奸商陷害,买下了水文蛊。那奸商说,有了此蛊,就能行云流水般地写下美妙文章。老爷他平素蛮勇,其他老爷就常挤兑他空有蛮力,并无文采。老爷有这层心结,这才不惜高价,买下水文蛊,当晚就想要写出美妙文章。”

%63
“难道是这水文蛊有问题?”方源接着问。

%64
小仆摇摇头,又点点头:“起初时,老爷使那水文蛊,似乎没有问题。一篇篇文章,接踵而出,老爷喜不自胜,连夜写了五六十篇。到了清晨,却出现问题。老爷还未用那早膳,就忽然腹泻不止。拉肚子像是拉水,更可怕的是,从早到晚,一连二十多趟。可怜老爷原本虎狼体魄,竟一天之间,就落得如此凄凉了。”

%65
方源这才恍然,同时心中也有底了。

%66
水文蛊是没问题的,但是用多了,就有问题,谁也承受不住。

%67
过犹不及,乃是世间常理。用蛊是有风险的,落到这样的蒙逼下场,也是咎由自取。

%68
“蒙家强请我来,果是没错。我这些宙道蛊虫,虽然没有医师的手段,但是用在此处,却是恰到好处,可以治好蒙逼。”方源心道。

\end{this_body}

