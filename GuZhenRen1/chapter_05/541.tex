\newsection{巅峰盛景}    %第五百四十三节:巅峰盛景

\begin{this_body}

这是炼道流派的最高追求之一――将种种杀招,统合成一只蛊虫!

举个例子,凡道杀招我力。(www.MianHuaTang.cc 棉花糖小说)

此招乃是力道杀招,北原蛊仙苏仙儿为自家女儿黑楼兰开创出来。它的核心是力道,但也涉及魂道、气道,一旦催动出来,就能打出人力虚影,虚影形象逼真,是蛊师潜意识中的描绘。

并且,这种力道虚影能够能自动追敌,富有灵性。

组成这个杀招的蛊虫,一共有四五十只。

按照方源此时领悟的炼道至理:任何杀招,不管是凡道还是仙道,都可以算是炼蛊的残方。

也就说,我力杀招的内容上,完全可以不断地删减、增添种种蛊材,不断完善,最终形成一道我力蛊的蛊方。

当蛊方完善之后,我力蛊炼制出来,单单催动这只蛊虫,就能发出力道虚影!

这样做的好处,是显而易见的。

杀招浓缩成了蛊虫,蛊虫数量暴降到一只,蛊师催动的时候,不仅极为方便,速度变得很快,而且消耗的真元,也远比之前杀招来得要少得多。

再举个例子,仙道杀招万我。

此招乃是方源今生独创,奴力合流,在龙马精神、三头六臂、鼠疫、雷暴、豹突、马踏,三心合魂、兽影围杀、六臂天尸王等等杀招融汇提升。

核心仙蛊是我力、拔山、挽澜、净魂。辅助蛊虫有全力以赴蛊、苦力蛊、借力蛊、自力更生蛊、炼精化神蛊、功倍蛊、地力蛊、火水风电力蛊等、天元宝王莲蛊、潜魂兽衣蛊、敛息蛊、群力蛊,以及种种魂道凡蛊。

显而易见,这个杀招内容很多,催动起来,非常麻烦。

也是方源锻炼纯熟,深入骨髓,已经练成了一种身体本能,这才在实战中运用自如。

但实际上,它包含万千,非常复杂。

如果将这个杀招,改良升华,炼成一只蛊呢?

万我仙蛊!

很显然,这就非常方便了!太便利了!

方源只要灌输仙元,只催动这种蛊虫,就能爆发出成千上万的力道虚影,又方便又快捷。

以前,方源是完全不够资格,但是现在当他成为炼道准无上,他就完全有能力可以达到这样的程度!

将种种杀招提炼成蛊虫!

方源对此怦然心动,但也没有丧失冷静,他依旧非常冷静,不断深刻地思考。[txt全集下载wWw.80txt.coM]

“我现在已经有能力,将种种杀招升华成蛊虫。”

“不过,有的杀招可以浓缩成一只蛊虫,比如万我杀招,但有的杀招并不能够,例如紫血先河阵、燃念飞石等,这些杀招若要浓缩升华到极致,会有数只仙蛊存在。今后我催动这几只仙蛊,就能相互搭配,成功催出杀招来。”

“但此法有利有弊。”

“好处是方便快捷得多,对心智、精神的消耗急剧减少,但坏处也有。首先一点,比如万我杀招,浓缩成一只万我仙蛊后,可能要将其他仙蛊当做蛊材,在炼蛊的过程中消耗掉。并且单独催动万我仙蛊,威能未必比杀招万我强。”

杀招万我要催动那么多的仙蛊和辅助凡蛊,万我蛊虫却只有一只,若层级只是六转、七转,显然道痕有限,威能不足也很正常。

“然后还有一点,蛊虫单一之后,就容易被针对。杀招虽然繁复,但旁人推算破解的难度也就大得多。”

杀招还可以不断地调整,提升。比如万我杀招,就有变招力道大手印、逆流护身印。

但万我蛊虫只有一个的话,这种提升就没有了。

“当然,总体而言,浓缩升华成一只蛊虫,好处更多。因为完全可以将它当做核心仙蛊,再组建出杀招来。如此一来,威力可以提升,也不会被轻易破解针对。”

“就算炼制的过程中,有的仙蛊被当做蛊材消耗掉,我完全可以重新炼制出来啊!”

这个能力,对方源的帮助会非常的大。

别的先不说,单提万我杀招吧,方源的逆流护身印就是此招的变式。一直以来,方源催动逆流护身印都非常麻烦,需要一段时间来酝酿杀招,步骤繁琐,稍微失误,杀招催动失败,就会殃及自身,令自己身受重伤。

但如果有了万我仙蛊,方源完全可以将逆流护身印改良出来,必定能够大大简化步骤,极快地催发出逆流护身印。

这样一来,战力方面的提升就直接暴涨一截上去!

方源算了算,自己掌握的杀招非常多,但真正用得上还是那么些。

万我杀招是可以浓缩升华的,紫念洞悉灵动星芒可以缩减成几只仙蛊,上古剑蛟变杀招也完全可以浓缩成一只蛊虫。届时万我仙蛊加上古剑蛟仙蛊,单单两只仙蛊,就可以爆发出万蛟杀招出来!

卜卦龟变化、洁身自好杀招、百八十奴、斗战胜伏奴等等,都是可以浓缩升华的。

随着这方面的深入探究和思考,方源忽然又有了感悟。

“等一等,不管是杀招,还是炼蛊,本质上都是在运用道。”

“我的逆流护身印,不就是运用了逆流河吗?幽魂魔尊的杀招井井有条,不就是利用了市井吗?”

“市井、逆流河这些都不是蛊虫,为什么可以运用?不正是因为,它们都是天地秘境,蕴藏道痕极丰,被我们运用上了吗?”

方源再回过头来想,蛊虫养、用、炼,其实本质上都是道。

蛊修的流派有太多太多,炎道、水道、金道、木道等等,力道、智道、魂道种种,律道、毒道、偷道……

“这些流派,都在讲蛊虫的养、用、炼,实际上就是阐述天地间的一种道啊!”

忽然间,一句蛊修界最经典的话,再次闪现在方源的脑海中。

――人乃万物之灵,蛊是天地真精!

从未有这么一刻,方源对这句话的理解如此深入。

“我要将杀招浓缩升华成蛊虫,就要删减蛊虫,增添蛊材,将其融汇成一体。”

“仙蛊是大道碎片,蛊材则是道痕累积,只要流派一致,它们就能相互取代。”

“再看逆流护身印、井井有条,之所以能催成杀招,就是逆流河、市井的道痕太过丰富,取代了蛊虫的作用。从这点来看,这两记杀招其实已经是从杀招,向蛊方转换了!”

许许多多的疑惑,都在此刻,随着方源不断思考,而豁然开解。

炼道其实比较特殊,它虽然单独一道,但也涉及各种流派。方源成就炼道准无上所带来的感触,比他成就任何其他流派,都要来得更多更广更加深刻。

“再想想看,我如今若用江山如故仙蛊,修复荡魂山,表面上这是蛊虫在起作用,但也可以看做是杀招江山如故在运作啊。任何的杀招,都可以看做一份炼蛊残方,因此转变角度,这也就是炼蛊啊。”

“只是我炼成的不是一只蛊虫,而是一块九转仙材――荡魂山!”

方源只是七转修为,居然可以炼出九转仙材来?

别忘了,这里面更关键的是井井有条杀招,是幽魂魔尊的手笔。而当年,幽魂魔尊运用在荡魂山上的井井有条杀招,正是九转层数。

明白这一点,方源对修复荡魂山顿时有了许多全新的想法。

“荡魂山既然可以修复,也可以当做核心,仿造逆流河,形成仙道杀招!”

“哦!我明白了!幽魂真传中魂魄修行的最高巅峰――三尊六合千臂魔魂,不就是一座人造的天地秘境吗?”

“还有盗天魔尊的九转杀招鬼不觉,无数道痕数不胜数,相互配合,不需要仙元灌输,就能运转不休,这也是一座人造的天地秘境啊!”

一直以来,方源都不明白,为什么鬼不觉杀招居然不需要仙元,就能时刻催动,一刻不停。

但现在他恍然大悟。

荡魂山需要仙元吗?不需要。

落魄谷需要仙元吗?同样也不需要。

魔尊幽魂力抗灾劫,虽然是动用了仙元,催动杀招,但本质上他不需要仙元,也有绝强的战斗力。

但有一点,并不是说将种种道痕随意叠加在一起,只要量大,就能成为天地秘境,就能成为鬼不觉这类的杀招。

道痕之间还需要搭配。

这种搭配,不就是炼蛊吗?需要的蛊材需要精挑细选。

不就是养蛊吗?需要的食材也要特定标准。

不正是用蛊吗?蛊虫之间相互配合,就是道痕的配合。

就这样不断地思考,良久之后,方源缓缓地睁开双眼。

三天三夜已经过去。

他消化炼道真意,不过数个时辰,但自身感悟却足足用了三天三夜!

方源的黑眸,熠熠生辉,又深不可测。

“你……”见到方源这样的眼神,琅琊地灵惊疑不定起来。

“我怎么了?”方源问。

琅琊地灵叹道:“你刚刚那一刻的气质,让我仿佛再见到了曾经的巨阳仙尊、盗天魔尊呐!”

巨阳仙尊、盗天魔尊……任何一位蛊尊,都是世间公认的无上大宗师。

方源炼道准无上的境界,已经搭上了他们这一层次的边缘了。

不再是仰望高耸入云的山峦,而是看得见这个层次里的风景!

蛊修的巅峰盛景!(未完待续。)

\end{this_body}

