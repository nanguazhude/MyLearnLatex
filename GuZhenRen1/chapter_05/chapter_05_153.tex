\newsection{疯魔三怪}    %第一百五十三节:疯魔三怪

\begin{this_body}

%1
当然,楚度既然邀请方源出手,亲自去疯魔窟请人,在信中也给了方源一些防护魔音的方法,还有魔音强弱的规律。

%2
“不管如何,我都是要去一趟的。”

%3
楚度和百足天君相互争夺黑凡洞天,这事情的源头就出自方源。并且他和楚度利益相关,突袭铁鹰福地也有他一份。总之牵扯很深。

%4
方源雷厉风行,立即动身。

%5
通过超级蛊阵,他传送出琅琊福地,便一路往疯魔窟赶去。

%6
中途并无意外发生。

%7
十多天后,方源停在高空中,俯瞰下方。

%8
在平坦的黄土地上,有一个巨大的洞窟。洞口近乎圆形,半径就有十万丈。洞中生长着茂密的雨林,黄绿交映。时不时传出荒兽的嘶吼声,一大群的无羽怪鸟飞出来,赤红色的身躯让方源联想到地球上的史前翼龙。

%9
一片蛮荒景象。

%10
方源降下身形,洞窟在视野中越来越大,最终充斥整个视野。

%11
高度不断降低,最终方源的身影被庞巨的洞窟雨林吞没。

%12
进入雨林当中,闷热潮湿的气息就笼罩方源周身上下。

%13
树木枝叶茂盛,但阳光却显得十分热烈。树藤一根根垂挂在树干上,或者彼此缠绕,有时候让方源都难有下脚的余地。

%14
吱吱吱……

%15
远处,一头头的怪猴,通体灰毛,相互在树梢上追逐。

%16
忽然间,一根静止不动的“树藤”猛地窜起,张开血盆大口,飓风彪起,一下子就将数十只灰猴吸入腹中。

%17
原来是一头猎食成功的荒兽树蟒。

%18
灰猴群顿时躁动不安,狂叫不已。叫声吸引来猴群的王,也是一头荒兽。

%19
荒兽树蟒不断后退,它已经吃饱了,达到了自己的目的,没必要和灰猴王交战。

%20
灰猴也不是复仇性强大的物种。猴王见树蟒主动徐徐退去,嘶吼咆哮几声,也没有追赶。

%21
但就在这时,忽然天地间回荡出一阵刺耳的声音。

%22
“不好。是魔音。”方源连忙催动蛊虫,按照楚度交代之法,防御魔音灌耳。

%23
寻常的方法,是无法屏蔽魔音的。

%24
但楚度之法,却显然十分针对。效果立竿见影。方源顿时感到魔音消除,仿佛并不存在。

%25
猴王的吼叫和树蟒的嘶吼,却在这时传来。

%26
受到魔音的干扰,原本要相互避退的两头荒兽,此刻却发了疯似的相互进攻。

%27
这已经不是激战,而是死战。

%28
两头荒兽死战,相互纠缠在一起,四处翻滚,立即造成周围成片的密林倒折,无数生灵惨死。

%29
它们周围的那些猴子。也都发了疯,不仅疯狂地攻击周围的同类,有的甚至开始自残,主动撞击坚硬的岩石,或者用手抠出自己的眼球。

%30
原本安宁静谧的雨林,忽然间整个暴乱,到处都是兽吼,无数生灵陷入疯狂之中,相互剿杀,烟尘滚滚。浓重至极的血腥气弥漫而出。

%31
方源保持着冷静的神智。

%32
但周围无数的鸟兽,都向他进攻,甚至连脚边的蚂蚁,耳边的蜂蝶。都对他展开自杀性的攻击。

%33
“这就是疯魔窟啊……”方源心中叹息,催动蛊虫防护自身。

%34
他神智清醒,又有大量厉害手段,生存在这个疯狂的环境中,并无多少压力。

%35
毕竟,周围的这些疯狂的生灵。并不是只针对他一个人,而是随意进攻,胡乱妄为。

%36
魔音持续了片刻,骤然消失无踪。

%37
嘈杂喧闹的雨林,安静下来。

%38
不久前还活蹦乱跳的野兽,如今大量躺尸,残破的尸体和断肢,洒落在雨林各处。血液在有些地方,汇集成临时的小河,浇灌大地,让无数折倒的树木获取丰厚的养分,随后重新快速地生长。

%39
远处那两头死战的荒兽,已经双双折损。

%40
树蟒粗壮而又修长的身躯,死死地缠绕在猴王身上,缠了数圈,最后一道缠在猴王的脖颈上,将它的颈骨碾碎。

%41
不过猴王在临死的挣扎前,也将树蟒开膛破肚。刚刚吞下的猴子,虽然重见天日,但已悉数惨死。

%42
树蟒奄奄一息,魔音消退之后,它的眼神重复清明之状,但很快就彻底黯淡无光。

%43
它彻底失去了生命。

%44
方源心中一动,赶往战场,将这两头战死的荒兽收入自家仙窍之中。

%45
“我若是你,便不会做这种无用功。在疯魔窟生存的生命,不管是飞禽走兽,还是蛊虫花草,都没有任何利用价值。这种荒兽尸体,在宝黄天也卖不出去,因为它们身上的道痕都错乱不堪。”一个声音忽然传来。

%46
“谁?”方源顿时一惊。

%47
他虽然一直催动着仙道杀招见面曾相识,还有宙道的三息后现,但后者是令蛊仙能够看到三个呼吸之内和自己相关的情景。

%48
“看到”并不代表“听到”。

%49
三息后现也是有弱点的。

%50
比如现在,光是声音,方源就没有提前预知得到。

%51
但很快,方源就提前“看到”了。一位身着白袍的中年蛊仙,凭空现出身形,出现在他的右侧后方,距离他十步远。

%52
方源全神戒备,却仍旧东张西望。

%53
三个呼吸之后,从他右侧后方,仍旧是那个位置,果然出现了一位白袍蛊仙。

%54
这位蛊仙面容白皙,黑色胡须飘飘,宽松大袖,双目锐亮似星,风度不凡。

%55
方源赶忙回首,看着这位陌生人,满脸紧张神色。

%56
白袍蛊仙却是摆手笑道:“不要怕,小友,在下不是仙。”

%57
方源此刻流露六转气息,白袍蛊仙却是七转修为,没有故意遮掩。

%58
“什么?”方源一愣。

%59
“我号称‘不是仙’,常年居于疯魔窟。你身上的蛊虫正是我们疯魔三怪,交给楚度的信物。”白袍不是仙笑着道。

%60
方源心中一动,听不是仙这话,疯魔窟中竟还有另外两人。他立即行礼:“晚辈见过前辈,这一次冒昧前来,正是想请前辈出手,相助霸仙大人。”

%61
“呵呵呵,我们疯魔三怪隐居数百年之久,不理世事。楚度若是加入我们,我们欢迎至极。但若是让我们出山,离开疯魔窟,却是绝不可能的。”不是仙笑道。

%62
“哦?”方源顿时面现焦急之色,不过他心中却无一丝紧张。

%63
且不说不是仙是否故意这般说,以抬高身价。单凭霸仙楚度,就绝非胡乱出招的人。

%64
楚度肯定熟悉疯魔三仙,既然让方源前来这里,必定早有定计。

%65
“楚度既然将蛊虫都交给你,恐怕连使用的种种手法,还有此中路线都传给你了吧?”不是仙问道。

%66
“是的。”方源点头。

%67
不是仙不禁上下打量方源,能够让楚度如此看重,这位蛊仙后辈到底有何能耐?

%68
“既然如此。你就随我来吧。嘿嘿,也算你走运,正巧碰到我上来透气。疯魔窟近年来越加危险,路线也发生了变动,你跟着我走,若是行将踏错,后果自负吧。”不是仙说完,率先没入前方雨林。

%69
方源心中犹豫了一下,便也动身,紧随其后。

%70
两人深入雨林一会儿,便钻入一处洞口,进入第二层。

%71
疯魔窟共有九层,第二层中是一片炙热无比的火岩石滩。海量的火岩,相互堆叠,有大有小。大的宛若巨象,小的就是一块块的小石子儿。

%72
片刻后,两仙又下到第三层。

%73
第三层一片白雾朦胧,大量的雾虫在这里生存,行进的路程中,方源也看到一头头的猛兽在云雾中影影绰绰。

%74
最多的,是云竹。

%75
正是这种特殊的竹子,树林规模和第一层的雨林相差不多。因此才酝酿出无穷无尽的云雾。

%76
然后,方源竟在云雾中看到了一片城镇的影子。

%77
高达的城池,还有市集,各种摊点上游人如梭,只是没有丝毫声音传出,显得十分诡异。

%78
“那是雾都,死在这里的冤魂相伴,凝结而成,留恋无益。”前方传来不是仙的提醒。

%79
方源心中微惊,雾都他也知晓,只是第一次见。能够形成雾都的魂魄,至少是蛊仙,而且数量上少不了。凭方源如今的实力,若陷入雾都,必是十死无生,根本无法逃脱。

%80
疯魔窟乃是北原十大凶地之一,自然危险重重。

%81
不是仙在内的疯魔三怪竟然在这里常年居住,可见本领不凡得很。

%82
在不是仙的带领下,方源深入下去,一路上有惊无险,顺利到达第六层。

%83
“下面的三层,我们也不敢随意进去,太过危险。来,我为你引见蛊仙胖山,他和楚度接触最多,更是他引荐了楚度。楚度知晓我们三怪的规矩,但既然派你过来请援,必有用意。胖山应当知晓。”

%84
在不是仙的带领下,方源在第六层的一处小山谷中,见到了蛊仙胖山。

%85
好大一个人!

%86
胖山身高十数丈,躺卧在小山谷中,竟是将整个山谷当做椅背,半靠半躺着。

%87
他让方源联想到了地球上的那座乐山大佛。

%88
胖山睁开一丝眼缝,从沉睡中醒来。

%89
“哦?楚度请援……嗯,我们俩之前的确有过一个约定。”胖山开口,声音低沉而又缓慢。

%90
“果然。什么约定?”不是仙抚须问道。

%91
“借他仙蛊和杀招,帮助他解决身上的盟约。”胖山徐徐答道。

\end{this_body}

