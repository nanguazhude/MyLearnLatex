\newsection{武庸的恨与怒}    %第三百六十二节:武庸的恨与怒

\begin{this_body}

梦境大战,影宗大败亏输,南疆正道其实同样损失惨重。[\&\#26825;\&\#33457;\&\#31958;\&\#23567;\&\#35828;\&\#32593;\&\#77;\&\#105;\&\#97;\&\#110;\&\#104;\&\#117;\&\#97;\&\#116;\&\#97;\&\#110;\&\#103;\&\#46;\&\#99;\&\#99;更新快,网站页面清爽,广告少,

他们布置出的超级蛊阵,每一家都贡献了仙蛊,还有海量凡蛊。

投入了这么多,结果收益极小,更且在梦境的探索上,遭受迎头棒击,损失了数位蛊仙。

真正受益者,除了成为影宗新主的方源之外,就是天庭了。

天庭这一次不仅是俘虏了幽魂本体,而且还在临走之前,将剩余的第二层、第一层超级蛊阵,都卷席而走。

这里面的仙蛊,除了一只方源的暗渡仙蛊外,其余的蛊虫就都是南疆正道之物。

对这样的行为,南疆正道上下无不愤慨,想要找中洲天庭讨要说法。

毕竟大家都是正道,天庭这样搞太不地道!

在追杀方源等人初始,武庸就开始寻找天庭蛊仙。虽然南疆和中洲,是两大不同地域,但是超级势力之前,经常有着贸易往来。

这点功劳,得算在宝黄天的身上。正是因为有了宝黄天,才使得蛊仙间的交易,变得更加轻易和频繁。

商货的不断流通,也就意味着交流的频繁。

所以,武庸是完全有途径,找到中洲十大古派身上。

但是天庭蛊仙,就有点悬。

然而若不找到天庭诘问,光是纠缠于十大古派,南疆正道的这个问题根本难以解决。

“紫薇仙子……”此时此刻,武庸手持着一只信道凡蛊,口中呢喃。

他没有想到,天庭蛊仙竟主动找上了他。

并且这位素未蒙面的天庭智道蛊仙,一上来,就只用一只区区的信道凡蛊,将武庸逼上了艰难处境。

这只信道凡蛊中的内容,都是关于方源等人。

天庭手中掌握着的,影无邪、紫山真君、白凝冰、方源等等的情报。

包括他们拥有上古战阵四通八达。

以及最让武庸感到震惊的一点――真正的武遗海早已死亡,混入武家的“武遗海”,真正的身份,正是天外魔头方源!

“竟有这样的事情!方源就是武遗海,武遗海就是方源?”武庸还想拯救他的亲弟弟,结果发现,原来亲弟弟早就挂了,一直都是仇敌扮演。

武庸对这个消息,首先感到震惊,毕竟这着实超出了他的料想。

然后,武庸感到了愤怒!

对方源的愤怒。[txt全集下载wWw.80txt.coM]

这个该死的凶手,杀害了武遗海的真凶,居然如此胆大包天,犯下凶杀案之后,不仅不逃窜,还主动加入了武家。

这简直不把武家上下,包括武庸自己,放在眼里啊。

更关键的是,方源他偏偏还做到了这一点。

这是耻辱,是羞辱!

方源不仅看不起偌大的武家,而且还跑到武家面前,呼呼的扇了武家两三个大巴掌。

武家是什么势力?

南疆正道公认的第一势力!

武庸是什么人?

武庸可是武家当权第一人,太上大长老,八转修为,拥有仙蛊屋,还有数只八转仙蛊的当世强者!

方源这样的举动,简直是把武家和武庸耍的玩。

不过,除了对方源的愤怒之外,武庸还有对天庭的怒火。

天庭之前,将包括武家在内的布阵仙蛊,都卷席带走,这是其一。

其二,便是天庭将这个消息,送到武庸的手中。

这是什么意思?

言外之意,就是告诉武庸――你们武家有一个把柄在我的手中,你接下来行事,要充分地好好地考虑这一点。

对于武庸而言,这就是威胁。

为什么方源假冒武遗海的事情,反而成了武庸的一个把柄了呢?

这就是正道和魔道的不同之处了。

正道有正道的游戏规则。

正道讲究名声,讲究面子,哪怕是巧取豪夺,都要有一个正当的理由。都要站在大义的名分上。

就像之前,各家刁难武家,抢夺武家分布在各处的资源点。

侯家怎么做的?

先派遣一位家族的蛊仙,假扮魔道贼子,让另一位侯家蛊仙假意追赶,然后借着这个由头,来霸占资源点。

羊家怎么做的?

先是蓄谋已久,在武家的附近安插凡人的营寨,然后通过凡人之间的摩擦和产生的矛盾,获取大义和名分。

落到武家,武庸明明战力高超,又拥有仙蛊屋玉清滴风小竹楼,为什么之前却一直处于被动防御的态势呢?

嘿嘿。

这就是武庸的谋算。

他知道,武家占据的地盘过于庞大。若是他自己直接亮出实力来,别的家族只会忌惮,或许暂时按捺不发,但始终会蠢蠢欲动。

毕竟武家到底是只有一位八转蛊仙了。

但如果,武家先被动防守,任由其他家族欺压,获取大义和充足的借口之后,武庸再行动,就能报复回去,在各家的身上咬下一块肉来。

只有这样子做,才能让各家势力记住疼痛,数十年内再不敢轻易冒犯武家威严。

这是武庸的计划,就连他最亲近的心腹武罚,都不清楚。

其实在此之前,一切都按照武庸的计划,逐步发展。只是武庸没有想到,会出现影宗这样的搅局者。

一旦武家当中,混入方源这个魔头,甚至武庸自己,都将方源当做亲弟弟来看待。

别忘了,武庸在此之前,有好多次,都不忘拿武遗海来刷自己的声望,表达出对自己亲弟弟的照顾。

武家是南疆正道第一势力,居然出现这样的事情,绝对令自身声望大跌,令各家异心萌动。

原本武庸想要借助这些借口,来反攻各家。但此事一出,南疆各大家族只要死死咬紧这个话题,武家就虚了。

你武家身为南疆正道的魁首,结果被一个魔头混进来,又扬长而去。

你武家有什么脸面,来统领正道?

有什么资格来占据这些资源?难道是占据了这些资源之后,供给家族内部的魔道贼子吗?

这就好像是地球上,明星被发现吸毒,一直都宣扬自己热爱家庭的成功人士被发现了婚外情,警察队伍中发现了****卧底。

声威虽然看不清摸不着,但它的确是一种实力,能影响方方面面。

在武庸看来,武家失去的那些资源点,不算什么。

既然可以失去,那他也可以重新夺回来。

事实上,他反而可以趁机,刷一刷自己的个人声威,表现一下自己的超强战力,让南疆历史上留下自己的影子。

但是,方源这个事情完全不一样。

一旦被发现,就是给武家一个狠狠的重击!

武家上下,无数代人辛苦奋斗和维持的声望和名誉,都要遭受巨大的重创。

这件事情一旦宣扬出去,武庸还那什么脸面,去夺回武家失去的资源点?很长一段时间,武家上下都要抬不起头来。

“这个该死的方源!”

武庸咬牙切齿。

他从未这么恨过一个人。

说起来,也有点讽刺。也怪他时不时的拿武遗海,来刷刷声望,表达出自己的仁义和亲和。

结果这么一来,反倒是成了武庸昏聩无能,被魔头方源欺骗得团团转。

这就是他武庸一生的污点啊。

“这个该死的紫薇仙子!”

武庸也恨极了天庭智道女仙人。

对方的用意昭然若揭,告诉武庸这些,一来是警告武庸,二来是把武庸当枪使,督促他尽全力铲除掉方源等人!

武庸必须要把方源铲除啊。

只要把方源暗自杀掉,武遗海的死亡就可推托到影宗手中,顺理成章,理所应然。

到那时,就算天庭把这个事情抖露出来,武庸也不会惧怕。

这件事情,也不会成为武庸一生的污点了。

“紫薇仙子……”武庸又在口中咀嚼了这个名字。

他明明知道,紫薇仙子是要让他奋尽全力,来杀掉方源等人,完全是利用他。但武庸偏偏不得不这么去做。

关键是紫薇仙子的这只信蛊中,只有方源等人的情报,毫无一丝和武庸谈判或者威胁的话,这让武庸抓不住紫薇仙子的丝毫把柄。

“武庸大人,发生了什么事情?”乔家太上大长老乔志材,见着武庸神色不对,关切询问。

武庸瞟了他一眼,顿时连他也恨上了。

“就是这个老东西。”

“统领乔家,想要攀附我武家。”

“没有他,方源怎么可能如此轻易地,混进我武家当中来?”

不过表面上,武庸却是微微一笑,温言宽慰乔志材:“不妨事,只是挂念我弟,不知道他落到影宗,会是什么遭遇?”

乔志材心中奇怪,这里没有外人啊,怎么武庸还在关切武遗海?这么表现是为了什么?

武庸之前挂念方源,只是虚情假意,但这一次却是真的。

只是他挂念方源,是恨不得把方源立即处死!

“还没有联系到武遗海吗?”武庸暗中沟通远在大本营的武家蛊仙。

群仙皆道没有。

当中还有人说:“最近一次和武遗海大人沟通,就是他向家族借蛊。”

又有人接着道:“武遗海大人身上,可是有武家借出去的六只仙蛊,他可不能有事啊!”

武庸听了这样的话,心中的怒火,顿时更炙热了几分。

这六只仙蛊当中,还有他亲自批准,借给方源的内库中的仙蛊。更还有十万仙元石,竟是武庸自己主动提出来,借给方源的。

想到这里,武庸恨不得打自己一个嘴巴。

同时更深深仇恨方源。

就是这个家伙!

他也太能演了。

即将暴露之前,他还不忘大捞一笔。着实阴险,太过狡诈!

我堂堂武庸,居然也被这家伙坑了!(未完待续。)

\end{this_body}

