\newsection{墨水效应}    %第七百六十一节:墨水效应

\begin{this_body}

%1
至尊洞天。

%2
小南疆,一座残缺的山峦,静静地耸立在大地上。

%3
它虽然残破,但是和周围的土丘相比,绝对是鹤立鸡群。

%4
和寻常的山体完全不同,它仿佛水晶铸就,通体粉红,时刻散发着梦幻般的迷离色彩。

%5
它就是《人祖传》中记载的天地秘境之一——荡魂山!

%6
方源的神念萦绕周围,徐徐不散:“荡魂山已是修复到了八成,远比上一世同期进步太多了。”

%7
比较起来,方源的上一世就显得苦逼多了。

%8
他没有吞并琅琊福地,琅琊守卫战也耗费了他大量的仙元储备。而后,他屡次进攻南疆各地,搜集梦境,这才和池曲由交易,获得大量的仙元石,这才补充了仙元储备。

%9
上一世方源修复荡魂山,动用江山如故仙蛊,更维持消耗海量的仙元!

%10
江山如故蛊只有六转,方源的仙元只有七转层次,又一直饱受外在的压迫,不管是南疆还是天庭,都要想方设法地找方源麻烦。

%11
有很长一段时间,方源的仙元储备一直都在警戒线周围徘徊,根本积蓄不起来,消耗很多。

%12
但这一世,自从方源吞并了琅琊福地,处境就完全不一样了。

%13
他成为了八转蛊仙,相同时间下至尊仙窍凝聚出的一颗八转白荔仙元,相当于一百颗的红枣仙元。也就是说,方源在仙元上的收成是之前的一百倍!

%14
同时,方源还有七转的天元宝皇莲,还有琅琊福地本身就很多的仙元石储备。

%15
这样巨大的底蕴,支撑着方源,令他在仙元储备方面完全不是问题。

%16
仙元几乎不计消耗地砸下去,荡魂山的修复速度自然比上一世,要快得多得多!

%17
上一世这个时期,荡魂山只有三成多的恢复程度。但这一世,已经达到了七成。再过不久,就能将荡魂山修复。

%18
若是江山如故仙蛊的转数,能再高一些,方源的速度还会更快。

%19
目前,江山如故仙蛊还只是六转而已,用它来修复荡魂山,简直就是那小刀砍大树。

%20
方源手中虽然有着八转仙蛊升炼,还有长毛炼道大阵,但是稳妥如他,还是不愿意冒险将江山如故仙蛊升炼。

%21
除非将来他获得悔蛊,构建出悔池来。

%22
炼蛊的成功概率太低,万一升炼失败,仙蛊毁了,那就尴尬了。

%23
方源当然想过本命蛊。

%24
本命蛊的升炼,即便失败无数次,本命蛊都会保存下来,并不会毁灭。

%25
但方源的宙道分身本命蛊是春秋蝉,至尊本体因为是至尊仙胎蛊所化,本身就没有本命蛊这个概念。

%26
“我若能夺取一位蛊仙的肉身,再分魂过去,酿成分身。在这个基础上,改换分身的本命蛊,再升炼江山如故蛊,这就稳妥了。”

%27
可惜,方源想到的这个法子,目前也无法实施。

%28
不是他捉不住蛊仙俘虏,他现在手中就有一批呢。

%29
他现在的问题是,魂魄消耗很大,底蕴稀薄,难以分魂。

%30
每当智道蛊仙推算方源,阎帝杀招就会发动,消耗方源的魂魄底蕴。

%31
方源每一次制造纯梦求真体,也会消耗方源的魂魄底蕴。

%32
万我杀招同样如此。

%33
探索梦境更对魂魄造成伤害。

%34
……

%35
如此种种,几乎让方源的魂魄不堪重负。

%36
尤其是最近,紫薇仙子战败后总是三番五次地来推算方源,导致阎帝杀招不断催动,方源不胜其烦,手中的胆识蛊库存已近干涸。

%37
方源重生以来,墨水效应带来许多好处,但也有坏处。

%38
上一世,他在魂魄方面情况还能支撑。这一世,却是十分糟糕,再这样下去,胆识蛊用光更要糟糕,方源就得动用其他魂道手段来救场了。

%39
这些魂道手段各有各的后遗症,当然是没有胆识蛊优秀的。

%40
“还有一个对我不利的情况,那就是天庭出手,竟染指龙鱼生意了。”

%41
“天庭方面现在贩卖大量的银龙鱼、铁龙鱼,似乎也能产出金龙鱼,这对我的龙鱼生意将造成巨大的冲击!”

%42
“这个情况虽然在上一世也发生过,但还要在以后呢。没想到这一世,居然提前这么多发生了。”

%43
对此,方源也没有办法。

%44
魔尊幽魂被天庭俘虏,紫薇仙子很可能是受到了琅琊福地战败的刺激,加大了对魔尊幽魂的搜魂力度,这才导致龙鱼生意出现了不利变化。

%45
“看来魔尊幽魂真的是支持不住了。”

%46
“就算上一世长生天进攻天庭,他都没有出来,一直被囚禁着。”

%47
方源叹息一声,将感慨按捺在心底。

%48
他没有时间能去浪费。

%49
一分一秒的时间都要争取,说不定将来大战中,这多出来的一丝一毫的积累能够帮助到他。

%50
方源的重生大计有条不紊地进行着,且都十分顺利。

%51
琅琊福地被他吞并了,这令他一跃成为八转蛊仙,他趁势吞并四族大联盟,底蕴暴涨。

%52
人道境界也突破到了大师级。

%53
池曲由和他再次交易,并且交易内容比起上一世更有利于他。

%54
还有房家也准备吸纳他,开拓青鬼沙漠的大计前景一片光明。

%55
种种变化和成就积累出来的前途,是上一世完全不能比的!

%56
方源仍旧没有丝毫的松懈。

%57
他知道接下来,就是南疆各大首脑商谈,最终确立讨伐他的追辑队伍。

%58
一方面,方源主动进攻,对于南疆正道而言是巨大的挑衅,南疆正道必须做出积极的回应。

%59
另一方面,武家、铁家、巴家等等势力,也想趁机捞取利益,增加名望,提高自身的话语权。

%60
“应该还有一段时间。”

%61
“因为南疆各大正道家族要商讨出一个结果来,还比较困难的。”

%62
“他们要相互扯皮,毕竟地沟接连涌现,各大家族为了争夺自然资源,激增了许多新的矛盾和龌龊。”

%63
此刻,铁家的烽火台铺设计划,还在搁置当中,虽然铁家一力推行,但始终受到各族的阻挠。

%64
更别提什么南疆大联盟了。

%65
武庸虽然有这样的野心,但若此刻提出来,就是一个笑话!

%66
方源要做的事情有很多。

%67
首先,他要改良杀招翠流珠。这招以七转定仙游为核心,动用了方源手头中几乎全部的宇道仙蛊,不少的炼道仙蛊,还有大量的辅助凡蛊,组合而成。

%68
这杀招他用过不少次了,算是彻底暴露了。

%69
池曲由虽然和他暗中交易,但绝对会搜集相关情报,寻找智道帮手来推算此招,找出克制之法。

%70
为了防止被克制,方源就必须改良,至少保证手中还要有一个改良的版本,以防万一。

%71
因为现在,白凝冰、影无邪等人都还在石莲岛中修行。

%72
四通八达杀招,暂时是用不起来。虽然雪儿、墨坦桑、石狮诚成为了替代品,但是方源本体和他们演练的次数也比较少,距离真正掌握还有不少的差距。

%73
方源在腾挪方面,现在是严重依赖四通八达杀招,还有定仙游杀招挑着大梁。若是万一发生意外,这两项手段不好用了,方源又被重重包围,那就危险了。

%74
不怕一万就怕万一,有关定仙游的杀招,在万一的情况下,就是方源的退路。

%75
其次,方源要修复春秋蝉。

%76
目前,他已在运用仙道杀招,帮助虚弱的六转春秋蝉更快复原。

%77
再次,方源正在积极调整经营项目。龙鱼生意已经受到不小的影响,天庭家大业大,方源是绝对竞争不过它的。琅琊福地中的原来生意,很多方源需要满足自身的需求,剩下来的要改变经营策略,方能适合如今的方源。

%78
另外,方源还在宝黄天中伪装身份,对外收购太古年兽。

%79
有关于仙蛊屋雏形的设想,以及蛊如故杀招的推算,都在进行当中。

%80
方源要打造出一个属于自己的仙蛊屋。

%81
上一世他的仙蛊屋雏形被屡屡击毁或者打残。这样的糟糕经历,在现在看来,却是他的宝贵经验,让他受益极多。

%82
重生以来的种种丰盛的收获,被方源迅速消化。他抓紧每一分每一秒的时间,每时每刻都在快速成长!

%83
数日后。

%84
西漠,房家大本营。

%85
“算不尽仙友此次加入我族,实乃我族百年来的盛事!从今日起,仙友的名号将传遍整个西漠,甚至名传其他四域。”房家太上大长老房功热情地握住方源的双手。

%86
方源笑了笑,不动声色的将手抽了回来。

%87
眼前这老家伙,看似豪迈,其实很是阴险。方源就曾亲眼目睹,这老头子偷袭暗算陈衣,在豆神宫一战把陈衣打得鼻青脸肿。

%88
“我加入房家,房功就亲自迎接,还带着房睇长,又要迫不及待地将这个消息宣扬出去。看来房家的处境,真的很不好啊。”方源心中思索,目光投向房睇长,对他笑了笑。

%89
房睇长心头莫名一紧,他暗中诧异:“怎么回事?为什么会有一种十分不妙的感觉?”

%90
下一刻,房睇长、房功的面色都微微一变。

%91
房功怒道:“好大的胆子!竟有魔道蛊仙侵犯我族重地天露绿洲,这是觉得我房家现在孱弱可欺吗?”

%92
房睇长脸色微沉,盯着方源看:“事情有些复杂了,未必是魔道蛊仙出手,搞不好是那些势力的轨迹。尤其是在当算不尽长老刚刚加入我们的微妙时刻,我们刚刚将消息公布于宝黄天中,就发生这种事情……”

%93
方源被激得冷哼一声,面色阴沉:“此事就让我来处理吧。”

%94
房功和房睇长迅速对视一眼,房功一拍掌:“好,那我们就期待客卿长老你的捷报了。”

\end{this_body}

