\newsection{劫掠和建设}    %第五百七十三节:劫掠和建设

\begin{this_body}

%1
轰!

%2
方源衣袖一拂,黑色的雾气卷席八方,正是魂爆杀招。

%3
这是七转巅峰的手段,立即将池家建立在云竹山脉上的仙道大阵最核心之处,摧枯拉朽地炸毁。

%4
轰隆隆。

%5
又是一阵轰鸣,大阵核心瓦解,其他地方的部分仙阵也紧随其后,接连崩散。

%6
至此,整个云竹山脉大阵彻底宣告瓦解。

%7
“果然是走了。”望着空无一人的核心之地,方源目光幽深,却无一丝意外之色。

%8
他前期攻打大阵,对方防御激烈,并且应对明智。但到了后期的战斗,方源就明显感受到,这座仙道大阵应对失据,很多反应都不及时,变得呆板。

%9
不论是仙道大阵,还是仙蛊屋,都是需要人镇守的。就算是留下来的意志,或者拥有阵灵,也不能替代蛊仙的作用。

%10
“嗯?这里似乎有一只仙蛊被毁了?”方源很快察觉到异常的气息。很快,他就确定,应当是有这么一个未成型的木道仙蛊,被池家蛊仙撤离时痛下狠心,直接摧毁了。

%11
方源暗暗可惜,但也赞赏这种决断。换做他,也绝对会这么做。

%12
核心处并没有什么有价值的东西留下来,方源便飞升上空,开始对整个云竹山脉下手。

%13
云竹山脉中,生活着大量的云道生命。

%14
比如云狸。

%15
方源催动奴道手段,瞬间从身上爆发出橙黄光晕,笼罩方圆百亩。

%16
光晕当中,云狸以及其他生灵,忽然剧烈地挣扎起来,神情痛苦,不断嘶吼。但很快,这种抵抗就迅速衰减,最终它们一动不动,待在原地,都成了被方源奴隶的野兽。

%17
方源打开仙窍门户,这些生灵便主动汇集起来,像是一条河流灌输到方源的至尊仙窍当中去。

%18
方源不断搜刮,海量的生灵统统投入到仙窍里面。

%19
忽然,一声嘶吼传出,两头荒兽云狸冒出头来,惊慌失措地向山外奔去。

%20
“都给我回来!”方源大笑一声,疾飞赶上,奴道杀招一发,就令这两头荒兽云狸趴在地上,认他为主。

%21
云竹山脉,地域广阔,又是大型资源点,充斥浓郁道痕,孕育出荒兽荒植非常正常,就算是上古荒兽也有可能。仙蛊就更罕见一些,可惜这里虽有一个,但还未成型,就被迫毁掉了。

%22
当方源大致搜刮了一遍之后,他总共收获了七头荒兽云狸,三种荒植共十一株,分别是云气根、白毛参以及枪尖竹。最让方源感到惊喜的,是有一块七转仙材云泥,体量十分庞大。

%23
不过,却始终未见到上古荒兽,或者上古荒植。

%24
“恐怕是镇守在这里的蛊仙,将它们带走了。亦或者压根就没有。”

%25
按照自然规律,这里的确有孕育上古荒植荒兽的可能。但这里却是被池家主宰无数年,池家精通阵道,但是在奴道方面却是薄弱的。他们自然不会允许,一头可以匹敌七转蛊仙的上古荒兽,不受控制地留在自己的产业当中。

%26
方源也不以为意,他继续搜刮,这一次他将魔爪伸向漫山遍野的枪尖竹上。

%27
这些枪尖竹都是普通蛊材,荒植枪尖竹已经在第一波,被方源重点收入囊中。

%28
到这一步,却是有些麻烦。

%29
方源的奴道手段很丰富,但是木道却有些缺乏。毕竟手中的木道仙蛊太过稀少了。

%30
“移植这种枪尖竹,却是需要恰到好处的木道手段啊。”方源感慨。

%31
方源想要移栽这些枪尖竹,就需要精巧的木道手法。其他流派的杀招,会在枪尖竹上带来其他流派的道痕,哪怕是临时印刻下的道痕,也会损坏枪尖竹的道痕,这就会让它的品质下降。

%32
其实幽魂真传中,的确是有许多木道手段,但方源在木道仙蛊这块,却是个短板。

%33
所以,一时间方源只能动用凡道手段,来收取这些枪尖竹。

%34
这样一来,效率就大大降低,远不如仙道杀招。而云竹山脉却是宽阔至极,方源收取的枪尖竹不过九牛一毛。

%35
“可惜我的拔山仙蛊,虽然可以拔山,但本身只是六转层次,拔不动整座云竹山脉。”

%36
以前,方源拔取荡魂山、落魄谷,是因为这两个天地秘境,曾经被幽魂魔尊动用了井井有条杀招。

%37
而这座云竹山脉,却是没有这样的待遇。

%38
当然,方源也可以将云竹山脉打碎,不过打碎的结果,却不受方源掌控。

%39
树有树根,山有山根。除非他有相应的土道手段,可以分出山根。

%40
“什么?方贼现在攻破了云竹山脉!”池曲由眉头拧成疙瘩,面色阴沉如水。

%41
他是池家太上大长老,出了这样的事情,相应的战报当然是第一时间送达到他的手中。

%42
之前的摄心河滩,只是中型资源,损失许多,但问题不大。现在云竹山脉也损失了,这可是大型资源!

%43
“当初我族为了经营出云竹山脉,不惜重金邀请土道大能,来专门点脉起山。又细心经营上百年,这才形成气候。没想到今天,竟然是云竹山脉遭劫了。”

%44
池家身为超级势力,掌握的大型资源点不在少数。但是云竹山脉却是有些特殊的。

%45
因为它是池家起步阶段,还在弱小的时候,由池家先祖们省吃俭用,咬紧牙关,拼力建设起来的。

%46
在池家的历史中,这座云竹山脉也在之后的过程中,为池家提供稳定的经济来源,没有辜负当初建设者们的期望。

%47
池家在一代代人的努力下,逐渐昌盛起来,势力范围也在不断地扩张。虽然是有了越来越多的大型资源,云竹山脉渐渐处于池家地盘的腹地中,但从未被忽略,一直都有两位蛊仙镇守。

%48
云竹山脉不仅是资源点,而且还有着对池家上下的激励,蕴含着一种池家精神。

%49
“别的大型资源点也就罢了,但云竹山脉却是不容有失的。去将仙蛊屋发动,讨伐方源,收回云竹山脉!”池曲由思考了一下,立即下达命令,态度坚决。

%50
“那么太上大长老您呢?”池家蛊仙问道。

%51
池曲由冷哼一声:“这方贼似乎是察觉到了我在这里,他真正的目的,还是要盗取这里的梦境!我就镇守这里,他一定会回来的。当然,你们却要务必做出我亲自讨伐的假象来。”

%52
于是,池家开动仙蛊屋,急匆匆地赶回云竹山脉。

%53
方源却还未离开,他还在收取枪尖竹。这种竹子根系相当发达,扎入土中,能深达五六丈,要将它们一个个无损地挪移到自家仙窍,需要巧妙的手段。

%54
池家蛊仙们见到方源,迟疑了一下,这里可是云竹山脉,若是和方源在这里开战,没有仙阵护卫,云竹山脉恐怕要毁掉。

%55
仙道杀招大盗鬼手!

%56
方源却早已经有了准备,大盗鬼手潜伏多时,忽然发动,照准池家仙蛊屋直接下手。

%57
至尊仙窍中。

%58
小北原最北端,大雪飘飞,寒气四溢。

%59
雪民们风尘仆仆,一路兼程,终于来到了此地。

%60
寒冷的天气,却让雪民们的脸上浮现出欢喜的笑容。

%61
“这里正是让我们安居的地方啊。”

%62
“而且有着这么广阔的土地,再也不像之前那样拥挤了。”

%63
“感谢冰卓大仙的指引。呜呜呜……”

%64
一些雪民老人,已经跪在地上,泣不成声。

%65
“你们快看!”有人忽然大叫。

%66
雪民们连忙抬头,便见到令他们终身难忘的一幕。只见漫天的寒雾忽然退散,露出雾中的三座晶莹巨峰。蛊仙“冰卓”悬浮在空中,对他们点点头,随后寒雾又蜂拥而上,再次将三座晶莹山峰尽数遮掩。

%67
“是冰卓大仙啊!”

%68
“这就是大仙指点我们的家园三圣山了。”

%69
“看来我族仙人,应当是生活在三圣山上!”

%70
雪民们激动万分,跪倒一地,纷纷对着三圣山的方向朝拜。

%71
他们却不知道,雾气笼罩下的“冰卓大仙”,模样忽然变化,成为一个陌生人,正是方源宙道分身。

%72
“食物、环境、植株等,都布置妥当,环境虽然简陋,但足够这群雪民生存下去了。”

%73
“到这一步,雪民的安置就彻底完成。”

%74
宙道分身微微点头,又操纵雪晶阵,看向另外一处。

%75
在那里,一小群的雪怪,从雪堆中冒出头来,相互之间好奇地打量,然后开始四处的徘徊。

%76
“不错。在雪晶阵的作用下,雪怪也生成了。这一处的雪怪生产,也开始徐徐启动。假以时日,就能为我带来丰厚收益。”

%77
离开小北原,宙道分身来到小中洲。

%78
前不久,他在这里新设了一处河滩,名字直接照抄,也叫摄心河滩。

%79
河滩中,有着无数石块,大大小小,浑圆洁白。

%80
还有一汪河水,潺潺流动,一些摄心蛊在这里生活。

%81
方源收获的大多数摄心蛊,都被他直接挂到宝黄天中售卖。留下的是全部的河水、石块,还有少量摄心蛊,留作种子。将来若有可能,也可在这个方面,发展出一道经济支柱。

%82
宙道分身一路疾飞,在摄心小河滩的东南方向,飞了许久,这才停下。

%83
眼前是一大片的空地,荒无人烟。

%84
宙道分身便开始在这里栽种枪尖竹,释放云狸等等野兽。

%85
山脉是没有的,方源打算初步建造一个云竹园林。

\end{this_body}

