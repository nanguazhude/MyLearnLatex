\newsection{修为暴涨}    %第一百六十八节:修为暴涨

\begin{this_body}

%1
方源心头一跳,终于明白自己表现得有些过火。

%2
不过这只是一点点的小失误,不是什么大错。

%3
“我们继续前行,先定下盟约,再好好交流!”不是仙笑起来,态度比之前更加热情。

%4
方源刚刚的表现,让不是仙对方源的评价,直接提升上了好几层。

%5
三人在道痕中艰难跋涉。

%6
越往前,道痕越是密集,阻力越来越大。

%7
当然,这完全是对不是仙、楚度而言。不过方源也很艰难,他必须表演出艰难的样子,让楚度、不是仙不会过分的怀疑。

%8
三仙歇歇走走,走走停停,短短的路程,耗费了他们足足一炷香的功夫。

%9
最终,他们来到一块石头面前。

%10
“又有新人了?”石头忽然开口,一阵烟气从石头山升腾而起,旋即消散个精光。

%11
石头已经化为一个干瘦矮小的蛊仙。

%12
他有山羊胡子,身高只及方源的膝盖,一双小眼睛精芒烁烁。

%13
“这位是疯魔三怪之一,智道蛊仙秘谋人。”楚度适时作出介绍。

%14
方源心中啊了一声:“此人竟然就是秘谋人!”

%15
表面上却伪装得很好,像是初次听闻这个名号,当即行礼道:“你好,在下变化道柳贯一。”

%16
一天后。

%17
“柳兄、楚兄,你们几乎来了就走,真的不多留几天吗?”在疯魔窟的最顶层,不是仙最后挽留道。

%18
“不了。”楚度苦笑拒绝,“你也知道北原现在的局势,我新立了楚门,和百足家联合。如今正道势力对我们这些散修外派,看不顺眼,正在纠集人手,要对付我们呢。”

%19
不是仙长叹一声:“红尘浊世,有什么好留恋的。权利名声,死后又能留在手中吗?《人祖传》中。太日阳莽死后,魂归生死门,栖息在安魂湖畔,名声蛊是离他远去的。唯有永生。才是我辈蛊仙值得追求的目标啊。”

%20
楚度面露坚毅之色:“我誓要将力道发扬光大,这是我答应了某个人的话。如果做不到的话,恐怕我会一生不安!”

%21
方源忍不住看了楚度一眼。

%22
他还是第一次听到,楚度说出自己的志向。

%23
楚度此刻的诚挚神情,似乎也不是伪装。

%24
“看来楚度修行力道。也有故事。”方源正想着,就听到不是仙对他发问。

%25
“柳兄,你大有潜力可挖,为什么不干脆留在这里?”

%26
方源笑了笑:“我此次前来,主要是为了缔结疯魔之约。如今盟约已成,我还有其他要事去处理呢。”

%27
不是仙实则更看重方源。

%28
主要是方源第一次踏足的表现,让他有了深刻的印象。

%29
方源在这方面,比不是仙更有自信。

%30
他觉得自己,甚至能在倒数第三层道痕之地中,撒了欢的飞奔。

%31
不过那样。就太惊世骇俗了。肯定会把疯魔三怪惊得下巴都掉下来!

%32
“倒数第一层、倒数第二层,我没见过,不好估量。但倒数第三层的确对我构成不了什么障碍。衍化蛊就在倒数第一层中,可惜疯魔三怪在此把守,我根本不能独自行动。”方源心中有些遗憾。

%33
总有一天,他会回到疯魔窟,利用自己的优势对这里展开探索。

%34
但现在,明显不是时候。

%35
尽快提升修为,才是最要紧的事情。

%36
至于衍化蛊,就算方源拿到手。又有什么用呢?高达九转的仙蛊,让方源联想到了智慧蛊,方源用不上,也不会用。得到了反而是沉重的负担!

%37
“除了永生之外。其余的事情皆是俗物。柳兄,你处理完要耗费多久时间?”不是仙追问道。显然他不想就这样放弃方源这个好手。

%38
方源目光一闪,看向楚度:“其实我也是楚门中人,这一次黄金势力围攻楚门,我也当尽一份力。”

%39
楚度明知方源是拿自己托辞,但听闻这话。还是感觉心中一暖。

%40
“唉,你们呐……罢了。”不是仙见劝说不动方源,只得放弃。

%41
方源和楚度和不是仙告别,迅速飞入天际。

%42
在回程的路上,方源询问楚度,有关北原的大势,以及楚门面临的困难。

%43
方源也定下疯魔之约后,和楚度的关系又深厚一层。

%44
楚度没有隐瞒他,苦笑道:“这一次,我也感到压力颇大。尽管我楚门和百足家联手,但对方的实力,是我们的数十倍,甚至上百倍!不过好在对方人心不齐,各有地盘,能抽调出来的蛊仙力量比较有限。又是正道,讲究规矩和颜面……我们还是有机会的。”

%45
“对方何时发难?”方源问出一个关键问题。

%46
楚度嘿嘿一笑:“早着呢。别看他们四处宣扬,如何叫嚣,距离真正行动起来,还有一段时日。这些黄金家族之间,也各有矛盾和龌龊。他们考虑得东西,要更多,远远没有我们魔道简单。”

%47
方源点点头。

%48
楚度的一番话,让他对北原的当今局势,有了更加清晰的认知和判断。

%49
楚度看了方源一眼:“这句话问的有些冒昧,柳兄,你的变化道境界已经是大宗师了吧?”

%50
“啊?还没有。”方源答道。

%51
“即便不是大宗师,恐怕也是准大宗师了。”楚度一副我认定了的样子。

%52
方源笑了笑,反问道:“你是如何看出来的?”

%53
楚度嘿然道:“我是从你行走道痕之地的表现,推测出来的。境界越高,对于疯魔窟的探索就有越多的帮助。不是仙对我的进步,大感惊讶。其实是因为我最近在力道境界上,得到了一个质变的飞跃。”

%54
楚度盯着方源,目光颇有深意。

%55
方源顿时知道他肚子里的话,无非是想问及仙劫锻窍杀招的事情。

%56
毕竟,楚度只是利用招灾仙蛊,蹭了方源一些便宜罢了。

%57
真正引动狂蛮真意的核心手段,方源一直保密,没有向他透露分毫。

%58
方源知道楚度真正想问的是什么,但他故作不知,只道:“原来如此。”

%59
四个字说完,就再无下文,开始闷声赶路。

%60
楚度眼中闪过一抹深切的失望。

%61
他对仙劫锻窍一直觊觎在心,但此时局势,这个仙劫锻窍却不是重点。

%62
所以楚度问得很含蓄,他知道方源这样的聪明人,一定听出了他的潜在意思。

%63
但方源的回应,却只有四个字原来如此。

%64
楚度立即明了,这是方源在回绝他。

%65
正如楚度提问的很含蓄委婉,方源的回答也是如此。

%66
“原来如此。”代表着方源引动真意的手段,是非卖品,拒绝交易。

%67
“只能暂时放过了。”楚度心中遗憾,但更有理智。眼下的局势,让他要加深和方源的合作,而不是恶了这段交情。

%68
方源没有和楚度一同回到黑凡洞天中去。

%69
在半途中,他就和楚度分别。

%70
方源一路往西南方向疾飞,数天后,他落到一处平淡无奇的荒丘之上。

%71
“雪松子的记忆,就是标注的此处。”方源环顾四周,确认自己来到了正确的地点。

%72
他开始催动仙蛊,片刻后仙道杀招催动成功,让他发现了一处仙窍福地的准确位置。

%73
这仙蛊和仙道杀招,都是方源向百足天君商借而来。为此,他付出的代价不菲。

%74
不过,对于方源而言,也是物超所值了。

%75
“发现了!”方源双眼蓦地精芒一闪,当即唤出上极天鹰。

%76
上极天鹰承载着方源,盘旋几圈之后,带着他洞穿空间,降临到一片陌生的福地之中来。

%77
这片福地的主人,生前主修土道,所以这片福地中到处都是黑色的烂泥。

%78
许多小小的泥怪,在这里打滚。

%79
时而一头荒兽级的巨大泥怪,从泥浆深处翻腾而出,像是卷起黑色的巨浪。

%80
一个泥巴小人,头上长着一颗青色小草,悬浮飘起,来到方源的面前。

%81
“叭叭叭。”泥巴小黑人开口,对方源道。

%82
方源听到耳中,却是明白了对方的意思。

%83
他不禁将目光投向下方的泥浆当中,口中轻声呢喃:“要成为这片福地之主,就要打败五头荒兽泥怪的联手么?呵,真是小意思。”

%84
土道福地的认主标准,对于大多数六转而言,是比较困难的。

%85
但对于方源而言,却简单得很。

%86
方源对土道并不陌生,事实上,由于前世的一段历练,他的土道境界有大师级。

%87
吞并了这个土道福地之后,他的西漠之中,便凭空多出了好大一片的沼泽地。

%88
接下来的一个多月里,方源四处奔波,不断搜寻福地,吞并福地。

%89
上极天鹰展现出极大的利用价值,让方源能够轻易地进入福地中进行探索。

%90
这些福地的情报,基本上来源于方源俘虏的那些蛊仙魂魄。

%91
东方长凡、郑灵、雪松子、黑凡洞天中的九位蛊仙、乱流海域中的多位蛊仙魂魄,还有一位大活人黑城。

%92
细细数一下,方源的蛊仙俘虏还真的不少!

%93
这些俘虏早就被方源搜魂了多遍,榨干了每一丝的情报。

%94
方源的修为突飞猛进,提前跨越了一次又一次的灾劫。至尊仙窍的空间,还有资源也随着他每一次的成功,一次次上涨一大截。

%95
方源的第一次天劫,早就被跨越过去。

%96
当方源再次回到琅琊福地时,短短时间里,他已经是渡过二次天劫,距离最后的第三次天劫,也仅有三次地灾的六转蛊仙了!(。)

%97
------------

%98
求一些正版订阅支持!

%99
今天爆更,爆得我双眼迷糊,双手都有些抽筋了。

%100
希望大家喜欢。

%101
非常感谢大家的支持,正因为如此,我才有如此强大的动力来码字!

%102
这些天订阅在涨,不过目前,距离精品,还有一半的路程。

%103
我知道有一些读者,甚至开双号,来正版订阅本书,支持力度非常的大。

%104
但蛊真人还需要更多人的力量!

%105
蛊真人的贴吧有5万人,百度指数有两万三千多!只要其中一小撮的读者,贡献出自己的一份力量,我们就能达到目标,让《蛊真人》这本书成为精品。

%106
我不勉强大家。

%107
只要大家每个人贡献出自己的一些力量,哪怕是一点点的订阅都可以。

%108
我写《蛊真人》已经四年了。我觉得《蛊真人》完全对得起“精品”这两个字。

%109
蛊真人需要大家的支持!非常需要!

%110
谢谢大家了!!

\end{this_body}

