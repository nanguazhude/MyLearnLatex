\newsection{天庭来攻}    %第七百四十二节:天庭来攻

\begin{this_body}

%1
琅琊福地,长毛炼道大阵。

%2
血光逐渐消散,血道小阵缓缓打开,一前一后的走出方源和琅琊地灵。

%3
方源的阴谋得逞了!

%4
在上一世,琅琊地灵曾经令紫薇仙子都猝不及防。后者打杀了黑毛地灵,却轮到白毛地灵登场。

%5
黑毛地灵富有野心,尽管知道琅琊福地和长生天的求援约定,但也不想动用。

%6
白毛地灵登场后却是干脆利落,直接向长生天求救。

%7
最终,白毛地灵的这一举动,导致方源逃生,天庭败北,长生天收编了琅琊派,成了渔翁,得了最大的好处。

%8
“如今,我成功镇压地灵,再也没有长生天什么事了。”方源看了一眼身旁的琅琊地灵。

%9
此刻的琅琊地灵,已经在方源的命令下,轮换成了黑毛地灵。

%10
为了防止其他毛民蛊仙怀疑,黑毛地灵装模作样,表面上和方源仍旧是之前的关系。

%11
方源故意吐出一口浊气:“这一次多亏了太上大长老您亲自出手,这才能挽回局面啊。”

%12
黑毛地灵笑道:“主……主要是有炼道大阵在。我们还是快快入阵,继续炼蛊罢。”

%13
黑毛地灵的笑,有些低声下气,但除了毛六之外,其余的毛民蛊仙没有一丝一毫的怀疑。

%14
谁能想得到,惊天的剧变已经在他们的眼皮子底下发生了!

%15
方源进入长毛炼道大阵,继续参与炼制万我仙蛊。

%16
数个时辰之后,炼道大阵中猛地闪过一阵刺眼的玄光。

%17
玄光消散,露出一只仙蛊。

%18
此蛊体型修长,长达数丈,它形如蜈蚣,宛若赤铜浇筑,浑身上下散发金属光泽。头部口器狰狞,额前一对长须甩摆如烟,左边有五千足,右边也对称有五千足,便共有了万足。

%19
七转仙蛊万我!

%20
方源虽然主要是来设计陷害琅琊地灵,但他并没有因此放弃炼制万我仙蛊。

%21
上一世,万我仙蛊方的成功率高达五成。今生为了暗害琅琊地灵,方源尽全力改良改良万我仙蛊方,使得五成的成功率降低到了四成。

%22
他从未想过要这一次放弃炼制万我!

%23
万我仙蛊实在无法放弃。

%24
它看似对方源的提升不明显,实际上却是相当重要的。

%25
万我杀招乃是奴力合流,兼收并蓄,集合了两大流派之长。但是本身非常繁复,酝酿须久,耗费蛊仙大量念头。

%26
现在凝聚成一只蛊虫,方源只需要简简单单的几个步骤,就能迅速地催发出原本万我杀招的效果威能。

%27
除了万我杀招本身之外,方源的逆流护身印、万蛟、万剑鬼蛟等等杀招,也都因此多多少少受益。

%28
尤其是万剑鬼蛟杀招,此招乃是方源上一世,临近中洲炼蛊大会的时候才开创出来的。若没有万我仙蛊的话,那需要的蛊虫就太多了,酝酿杀招的时间太长,根本就没有实战的价值!

%29
万我仙蛊虽然不能增长方源的实力上限,但它却能为方源夯实基础,提升许多方面的潜能。

%30
炼制仙蛊暂且告一段落,接下来方源便巡视整个琅琊福地,清点此次所获。

%31
琅琊福地来头极大,原主人乃是历史上的炼道三极之一的长毛老祖。

%32
长毛老祖极具传奇色彩,历史上炼制仙蛊无数,就连盗天魔尊、巨阳仙尊都分别找他合作炼蛊。

%33
长毛老祖身份也很特殊,他不是人族,而是毛民。尽管毛民寿命悠长,但仍旧会有寿尽的一天。

%34
长毛老祖留下八转琅琊洞天,但是经过无数年的发展,琅琊洞天在白毛地灵的主持下,有意识地降低品阶,一步步落回到福地级数。

%35
福地的灾劫,可比洞天要小得多,又因为白毛地灵励精图治,琅琊福地发展得非常不错。

%36
它地域广袤,远超正常的福地大小,更在空间上超过了许多洞天。

%37
在宙道方面,它和外界的流速一般在一比三十六左右。值得一提的是,琅琊福地的光阴支流上有仙道杀招,可以调节仙窍的光阴流速。这是长毛老祖生前,亲自邀请的一位宙道大能的手笔。

%38
道痕方面,琅琊福地最主要的道痕,便是炼道道痕,数量之多,超过三十万大关!其余流派的道痕,以水道为主、土道次之。

%39
琅琊福地的格局,是四大陆。分别是海上的黑毛、黄毛、白毛大陆,以及空中的云盖大陆。福地中物产丰富,人口众多,蛊仙数量也不在少数,乃是货真价实的超级势力!

%40
海量的人口,保证了此处就是寿蛊的产地。

%41
毫无疑问,这是天下第一福地!

%42
琅琊地灵已被方源镇压,这里的所有东西就都属于方源。

%43
除了福地的资源和人口之外,修行资源中还有海量的蛊方、杀招以及蛊虫。

%44
琅琊福地的仙蛊有很多存货。比如跌落到七转层次的天元宝皇莲,方源曾经借过来施展菌光普照杀招的核心之一——木芽仙蛊。还有奴兽仙蛊、阵盘仙蛊等等(许多仙蛊已经交托给毛民蛊仙使用)。

%45
琅琊福地中有三大蛊阵。第一座就是长毛炼道大阵,第二座则是九子母传送蛊阵,第三座是上古战阵天婆梭罗阵。其中,九子母传送蛊阵的母阵在云盖大陆之上,而九座子阵当中有一些分布在北原外界,比如太丘、龙象原、风伯崖。

%46
此外,琅琊福地中有一座残缺很多的仙蛊屋——八转仙蛊屋炼炉。

%47
还有一片耗费巨大,但还未建设成功的人造天地秘境——炼海。方源上一世,天庭蛊仙进攻琅琊福地,琅琊地灵就依靠炼海中的炼水,施展出杀招四海皆准,暂时统治过战局。

%48
琅琊福地中库藏丰富,仙材库存极多,仙元石积累则不多。

%49
最后,琅琊福地中有三大真传,分别是盗天真传、巨阳己运真传、长毛炼道真传。

%50
八转偷生仙蛊,就是在盗天真传之中。

%51
当年,马鸿运在北原之外得到这个盗天真传的线索,来到琅琊福地,请琅琊地灵炼了三次蛊。

%52
这也是盗天真传的一部分内容。

%53
方源对这三份真传,都有所了解。多年前,他就在和琅琊地灵的交易中,接触过这些。

%54
当初,他为了对付强敌,不得已贩卖自家的天地秘境。如今整个琅琊福地都成了他的囊中之物了。

%55
“还剩下小半个月的时间。”方源估算了一下日子。

%56
他上一世就是在这段时间里,极力推算落魄印,想要弥补自身在攻势上的不足之处。至于防守,他已经有了逆流护身印,完全不用担心。

%57
结果落魄印最终是在数年后,方才推算出来。

%58
这一世,他已经有了落魄印,推算的主要杀招是将宙道奥妙结合到阎罗战场中,争取能有一记复合型的战场杀招。

%59
虽然镇压了琅琊地灵,炼道的境界也绝对足够,但方源并不急着吞并掉琅琊福地。

%60
他早已决定,采取五界山脉的战术思想,将琅琊福地也布置成一个陷阱,等待天庭等人上钩。

%61
一来,方源需要引诱凤九歌,从他的身上抢夺到定仙游蛊。毕竟等到他修为八转,未来身就不好用了。

%62
二来,引导天庭暴露出星投杀招。这个手段太过厉害,能够直接改变五域大战的局面,对于其他任何一个超级势力,都有强烈的威胁!星投杀招暴露出来,必定能极大的激起其他超级势力对于天庭的讨伐之心。

%63
三来,提前铲除一些天庭蛊仙,方源绝不会轻易放过任何一个削弱天庭的机会。

%64
四来,方源也可以借助天庭之手,击杀掉四族大联盟中的雪人、石人、墨人蛊仙。然后他再效仿上一世,乘机将这三家残余势力吞并。

%65
啧啧,敌我通吃,想一想就很美妙,不是吗?

%66
十几天后。

%67
中洲,天庭。

%68
星宿棋盘!

%69
这座八转仙蛊屋仿佛正在喷射的烟花,万千星芒不断地喷射而出,照耀整个天庭。

%70
星光映照在天庭蛊仙的脸上。

%71
他们又近十人,早已经整装待发!

%72
为首的一位,正是紫薇仙子,她仰头凝望天空,眼底的精芒比天空中的星光还要璀璨。

%73
“没想到智慧蛊竟然就在琅琊福地!只要得了它,星宿棋盘便能成为九转仙蛊屋!届时,我便是天下第一智道蛊仙。将来天庭修复宿命蛊,几乎是板上钉钉的事情。”

%74
“方源,我已经迫不及待了!我要亲手把你轰杀成渣。”雷鬼真君暗自咬牙。

%75
凤九歌沉默,一脸的平静。

%76
陈衣的心中十分期待。

%77
他此前在青鬼沙漠中表现不佳,没有将豆神宫带回来。

%78
“这一次我可不能再像之前那般了。”

%79
“我才刚刚加入天庭,定要好好表现才是。”

%80
“这一次战斗,就是一个良机!方源再强,也只是七转而已。琅琊福地名头很大,但内部空虚,凤九歌都能来去自如,便可见一斑。此次我方出动这么多八转,必定能一战而下。唉,紫薇大人太谨慎了。”

%81
陈衣有点埋怨紫薇仙子。

%82
紫薇仙子如此大张旗鼓,摆出狮子搏兔亦用全力的架势,导致陈衣不那么好立功了。

%83
在他看来,出动两三位八转蛊仙,还摆平不了琅琊福地吗?

%84
这可不是寻常八转,而是天庭的成员!

%85
轰。

%86
在众多八转蛊仙期盼的目光中,漫天的星芒陡然凝聚起来,形成一道巨大的湛蓝星光巨柱。

%87
仙道杀招——星投!

%88
“出发!拿下琅琊福地,俘虏魔头方源。”紫薇仙子一声令下,雷鬼真君最先不耐,直接冲入星光巨柱当中。

%89
“方源,我来了!你给我好好等着。”雷鬼真君狞笑着。

\end{this_body}

