\newsection{交易}    %第五百七十五节:交易

\begin{this_body}



%1
“你这么迫不及待地想要见到我,看来是真想找死了。”池曲由冷笑。

%2
“是太上大长老,这下有救了!”镇守凤焰山的池家蛊仙,心中大定,口中欢呼。

%3
“我是专门等你,不知道你对我手中的梦道成果,有兴趣么?”方源忽然暗中传音。

%4
池曲由微微一愣。

%5
下一刻,方源伸手一指,仙道杀招阎罗战场忽然成型,将池曲由囊括进去。

%6
池曲由心中微微一惊,但旋即镇定下来,甚至还好整以暇地打量这片战场:“这就是困住凤九歌的战场?”

%7
方源哈哈一笑:“池前辈是想说,这是被紫薇仙子破解的仙道战场吧?”

%8
天庭方面,不仅在宝黄天中曝光了方源的大盗鬼手等等情报,甚至就连紫薇仙子对阎罗战场的理解,乃至破解之法,也公之于众。

%9
因此,有了这样的参照,池曲由并不担心自家安危。他甚至一点都不着急,他是阵道大宗师,完全可以在阎罗战场中固守待援。方源却是时间有限,一旦被池曲由拖住,令其他南疆正道蛊仙有机会组成围剿之势,那就危险了。

%10
天下没有无敌之蛊,只有无敌的蛊仙,定仙游也有许多方面可以克制。

%11
相比较而言,池曲由更头疼的是,方源脱离此处,继续游击,不断袭扰池家的各大资源点。这就麻烦多了。

%12
“池前辈不妨再看看这个。”方源又甩过一只信道凡蛊。

%13
池曲由侦查一番,见真是一只信道凡蛊,便探入心神,旋即目光微微一闪。

%14
这只信道凡蛊中,记载的内容,却是关于池家在凤焰山上布置的大阵。方源身为阵道大师,又有智慧光晕可以利用,推算出这些来,并不困难。

%15
池曲由瞬间明白了方源的意思。

%16
按照信道凡蛊中的内容,方源完全可以破解和摧毁了凤焰山上的保护仙阵,但他却故意留手,吸引池曲由过来。

%17
池曲由微微点头:“你果是天才,阵道造诣竟如此雄厚,难怪池伤那孩子颇为认可你。”

%18
阵痴池伤,乃是池家蛊仙新星,被池曲由看重,认定他将来也会成为阵道大宗师。之前方源伪装成武遗海,就和池伤接触过,和池伤建立良好关系。至于后来,方源身份暴露,被四处追杀,池伤也公开宣称,与方源划清界限,成为死敌。

%19
方源哈哈一笑,目光灼灼地看向池曲由:“若是我告诉池前辈,我的阵道境界就是从梦境中得来,池前辈不知有何想法?”

%20
“哦?此话当真?!”池曲由动容,他再一次上下打量远处的方源,用略带一丝疑惑的语气,问道,“你真是想和我做交易?”

%21
方源反问:“我为什么不想和你做交易?我需要梦境,但南疆的梦境都有守护大阵,并且都是出自池家之手。我愿意用梦道的成果,来换取池家的配合。为了展现我的诚意,还请你再看看这个。”

%22
方源再甩出一只信道凡蛊,池曲由接过一看,眼中精芒直闪,手中不自觉地暗暗捏紧信道凡蛊,深怕它忽然飞掉。

%23
有关梦道的内容,对池曲由有着无以伦比的诱惑,他不禁深入其中,很快将内容看完。

%24
待他抬起头,再次看向方源时,脸上的神情再一次有了变化。

%25
“方源这贼子,不愧是继承了影宗,又利用春秋蝉重生归来。随意抛出的梦道内容,就已经远远超越我们南疆正道辛苦探索的总和!”

%26
“并且,这里面的内容,很明显只是一小部分,后面的东西意犹未尽,甚至有些东西,根本就没有说清,只是含混而过。”

%27
念及于此,池曲由已经怦然心动。

%28
由不得他不心动!

%29
实在是方源抛出来的这个诱惑,太过巨大了!!

%30
目前,蛊仙界公认,梦道必将是未来的发展趋势。梦道不仅会创立,而且会走向昌盛,甚至还会产生大梦仙尊。所以,不论是超级势力,还是八转蛊仙,都明白梦道的重要意义,也都在着手,探索梦境,争取在这个流派中走的比别人更远。

%31
谁要是放弃梦道方面的追求,谁就被接下来这个波澜壮阔的时代所淘汰!

%32
就好像是大盗魔尊开创偷道,狂蛮魔尊创建变化道,元始仙尊建立气道……人族的漫漫历史,无数史实,已经证明了这一点。

%33
然而,万事开头难,尤其是现在梦道只是一个构想,不管是谁探索梦境,都付出了相当巨大的代价,并且收效甚微。

%34
正是因为了解这一点,池曲由更能明白,他此时手中握着的这只信道凡蛊的重量。若抛出去,必定能令天下各大超级势力哄抢,哪怕是争的头破血流,也在所不惜!

%35
池曲由摩挲着手中的仙道凡蛊,用犀利的目光盯住方源:“你的诚意的确是有,但南疆的正道势力这么多,你为何偏偏要和我做交易呢?”

%36
方源哈哈一笑:“我与池伤有过交流,又曾经拥有影宗、武家提供的情报,因此对池家有所了解。池谤是你的儿子,可惜他要坐住太上大长老的位置,能力上还有欠缺,因此需要池伤这个未来的阵道大宗师辅佐。池前辈你的寿元有限,将来若去了,又值动乱,池谤能护得住池家吗?”

%37
“你!”池曲由脸色顿沉,一抹杀意不受掩饰地泄露出来,方源却仿若未觉。

%38
但就算是池曲由也不得不承认,方源说的没有错。

%39
南疆地脉动荡,其他四域也有这样的情况发生,像是池曲由这样的人物,已经对未来的乱世有了许多察觉。

%40
池曲由动用过许多延寿手段,寿蛊已经对他难有效用,他未雨绸缪,很早之前,就开始为自己的儿子登位做准备。

%41
尽管他明白,池谤才干不足,但是他身为人父,自然就有私情。但一味关注私情,也并不可去。万一池家在池谤手中败亡,池曲由就算死了,也不得安息。而且池家若毁,池谤焉能无恙?

%42
正因如此,他池家才更需要方源的梦道成果,有了这个东西,池家在未来,不仅能够自保,甚至还能够顺应时代的浪潮,崛起成更强的超级势力!

%43
“哈哈哈。”池曲由忽然放声大笑,好像之前的脸色和杀意,从未有过。他看向方源,摇摇头,“好你个方源,你果然不愧是当今五域第一魔枭。我今日才知你的成色,难怪天庭、武家都拿你无可奈何。这场交易,我池曲由接了!”

%44
方源微微点头,池曲由的回应,都在他的意料当中,并不出奇。

%45
池曲由乃是正道势力首领,方源却是人人喊打的魔头,双方竟能做交易?这有些匪夷所思。

%46
但其实,正道、魔道,每个人都有着不同的概念。真正的本质,不过是正道乃是既得利益者,魔道却非如此。

%47
看看池家,拥有这么多的资源点,当然要宣扬正道的好处,邪不胜正,将自己占据大义的高处,维护和平和稳定。

%48
而魔道蛊仙,想要应付灾劫,就有不断变强,变强就得要修行的资源。这些资源他们又没有,怎么办?唯有烧杀抢掠,偷盗骗取了。

%49
一切不过都是利益使然。

%50
正因这种本质,池曲由答应方源这场交易,又有什么稀奇?双方各取所需,都朝着利益而去罢了。

%51
交易谈妥,方源便撤销了仙道战场,装作一副被击败的样子,仓皇逃遁。

%52
池曲由虽然得胜,却让方源逃走,面色并不好看。在随后的几天里,他积极部署调防,一副小心翼翼,防止方源报复的架势。

%53
七天后,方源忽然降临掠影地沟,攻破大阵,成功夺取梦境。

\end{this_body}

