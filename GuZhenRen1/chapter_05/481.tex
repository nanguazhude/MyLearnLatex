\newsection{偷袭蛊}    %第四百八十一节:偷袭蛊

\begin{this_body}

“不,爷爷,你误会我了!这火势我真的控制不住啊。<strong>最新章节全文阅读www.Mianhuatang.cc</strong>”少年盗天还在演戏。

“呵呵呵。你胆敢欺骗老夫,简直是不把老夫放在眼里。也罢,老夫也不取你性命了。就让这火把你也吞了吧。你以为炼蛊是简简单单,安全无比的吗?你故意在炼蛊过程中失败,这是自找死路!死吧,死吧,这是你的坚持,这都是你自找的。”沙枭冷笑连连。

他话音还未落下,那火焰顿时暴涨开来,一下子就将少年盗天罩住。

“怎么会这样?”少年盗天想要冲突出去,但火焰中传来一股巨大的吸力,让他驻足原地,根本动弹不得。

眼看着他就要葬身火焰之中,这是方源发现,他又能操纵少年盗天的身躯了。

“怎么我感觉自己就像是专门给少年盗天擦屁股的?”方源心中也有点囧。

不过,他却不慌乱。

沙枭给予少年盗天的这份蛊方,方源早就接触到了。

这是一转蛊方,方源又是炼道宗师,所以一目了然,对于蛊方中的原理是清晰透彻,了然于胸。

现在少年盗天虽然出了差错,但落到方源眼中,也不是不能挽回。

沙坑蛊。

方源操纵着少年盗天的身躯,忽然照准火焰中的某个位置,直接抛出沙坑蛊。

这个蛊虫他没有催动,而是当做蛊材一份,直接抛出去。

蛊身躯孱弱,顿时吸引了火焰过去,很快就烧成了一堆粉末。

火焰吸力大减,方源趁机一跃,脱身而出。

他就地打滚了几圈,将浑身上下的余焰扑灭。

伤势很重。

少年盗天的脸面、脖颈、手臂等等各处,都被烧得很惨,几乎面目全非。

但方源丝毫不管身上伤势,而是将全部注意力都集中在那处火焰中。

少年盗天还未脱离危险,因为炼蛊过程还未真正终止。

若是任由这片火焰灼烧下去,势必又会再次烧到少年盗天的身上来。<strong>在线阅读天火大道Http://wWw.qiushu.cc/</strong>

所以方源催动炊烟蛊,发出炊烟,包裹着脸盆大小的这团火焰。

一重重的炊烟,将火焰包裹得相当严实。

危险逐步减小,方源也感觉自己和火焰的联系,也在讯速地减弱。

“就这样下去,用炊烟灭火,就能终止炼蛊了。虽然还会受一点反噬,但命是能保住的。”

做到这一步,方源也尽力了。

毕竟少年盗天太乱来,居然想要在炼蛊的过程中故意失败。

这一次炼蛊,反倒令他实力减退,因为丧失了宝贵的沙坑蛊。

就在这时,异变突生。

炊烟被火焰猛地吸收到了焰心中去。

“怎么回事?”方源也始料未及。

但旋即他看到了焰心当中的那个婴孩虚影,他顿时明白过来:“这个婴孩大有问题,恐怕体质不凡,有着相当古怪!”

婴孩作为蛊材,添进火中去。

方源一直旁观,没有仔细检查婴孩,所以不知道蛊材中有古怪。

这一个异变,超出方源的预料。

火焰再次暴涨,婴孩的虚影在炎心中活蹦乱跳,扑向少年盗天。

“糟了,这一次性命危矣。”方源心中一叹,少年盗天一死,他这一次探索梦境必定也会随之失败。

眼看着火焰又要扑上少年盗天的身子,方源忽然福至临心,脑海中灵感猛地一爆。

“有了!”

他先是抛出炊烟蛊。

炊烟蛊一出现,顿时吸引了焰心中小孩虚影的注意力,他咿呀一声,直接扑上去,把炊烟蛊一口吃了。

趁机方源猛退几步,拉开和火焰的距离。

但很快,焰心小孩咀嚼着炊烟蛊,又向方源扑来。

方源大笑一声:“来吧,还有这只清水蛊呢!”

清水蛊也被抛出来,焰心小孩又将它吞吃下去。

火焰骤然一熄,清水蛊在焰心中化为一团翠绿的水液,并未被火焰融化。

而赤红的火焰也转为橙黄色,笼罩着光晕,再无之前的灼烧焰息,反而很是内敛。

翠绿清水和橙黄火焰相互流转,悬浮在半空中。

方源驱动少年盗天的身躯,不退反进,来到这团奇异的水火面前。

“沙坑为粉,固本培元。炊烟包裹,内敛火信。清水一出,天翻地覆。再用鲜血,阴阳协调。妙妙妙!”

说着,方源咬破舌尖,吐出一口鲜血,喷洒在奇异水火上。

“咿呀。”奇异水火遭遇到鲜血之后,猛地化为一个蓝色的小孩虚影,欢叫一声,扑上方源。

方源不闪不避,任其扑中。

蓝色虚影小孩直接转入少年盗天的空窍当中去,化为一只二转蛊虫。

“怎么会这样?”少年盗天回过神来,当场愣住。

“我擦,又来!”方源无语至极,少年盗天一安全,他就被抛至一边,再次沦为旁观之人了。

“你居然阴差阳错之下,炼出了一只全新的蛊虫来?!”沙枭也充满意外地叫出声。

“不,不对。”旋即,他又领悟道,“臭小子,你似乎有本性掩盖,一旦遭遇生命危险,本性就会占据上风,让你顺应直觉做出最正确的应对。之前擂台战就是这样,现在炼蛊也是如此。”

“你本人天真顽固,愚不可及,但你这本性却是相当务实,狡猾之至,更有你本人难以企及的天赋才情啊!老夫很是欣赏,大大的欣赏!!”

沙枭说着,笑出声来。

“我自己炼成了一只……新蛊?”少年盗天还未完全反应过来,愣愣地道。

“不错,你且催使一下。这只新蛊是二转蛊,似乎还开创先河,乃是前所未有之物,老夫的眼界也看不出它的功效威能。”沙枭道。

本来催动一只陌生蛊虫,是相当危险的事情。

比如说春秋蝉,就是最好的例子。

但少年盗天刚刚修行,懵懂无知,而知晓其中风险的沙枭,又毫不顾惜少年盗天的生命安全。

所以少年盗天就傻呼呼地催动了这只蛊虫。

一催动起来,这只蛊虫就再次化为蓝色小孩的影子,咿呀一声,钻出少年盗天的空窍,扑向前方。

前方有一个石凳,蓝色婴影直接没入石凳当中去,然后又钻出来,旋即又回到少年盗天的空窍中,还原成蛊虫模样。

石凳毫发无损。

少年盗天走上前去,伸手摸了摸。

砰。

一声轻响,石凳化为了一堆碎石块。

少年盗天吓了一跳,忍不住倒退一步,颇为震惊地望着地上的一摊石粉。

“妙哉!”沙枭旋即交口称赞起来,“这蛊虫威能极大,乃是当世二转中的极品蛊虫。更关键是它似乎有穿脱防护的作用,你仔细看,石凳是从内部开始崩溃的。”

沙枭目光老辣,一下子看出许多奥妙。

少年盗天此时还有些懵懵懂懂。

沙枭又看了一下他的空窍,再次赞道:“大妙!你看你的空窍,这一次催动蛊虫,只不过耗费了一成真元,难以想象啊。这可是二转蛊虫,威力这么大,却只需要一转青铜真元一成而已!”

“可是我只有丁等资质,真元最高不过两成多。用这蛊虫,只能用两次而已。”少年盗天满脸忧色。

“你懂什么?有了这只蛊虫,你将无往不利,小比第一唾手可得。没有人会防得住你这只蛊的。”沙枭嗤之以鼻。

“有了这只蛊,我就能得第一?”少年盗天不由地瞪大双眼。

“这只蛊虫一旦催动,就会化为蓝色小孩虚影,扑击敌人。这速度你也看到了,快的惊人。三转蛊师或许还可勉强躲闪,但一转、二转蛊师,只要接近一段距离,几乎人人都会中招,躲闪不及。”

沙枭不断分析,沉吟起来:“嗯……综合这只蛊虫种种威能妙用,不妨叫做偷袭蛊吧。”

蛊虫本是少年盗天所创,但沙枭却擅自做主,主动给这只蛊虫起了名号。

“偷袭蛊……”少年盗天顿时皱眉,想要反对。

“就叫偷袭蛊,老夫起名还是很恰当的。哈哈哈!”沙枭强横霸道,少年盗天只能翻白眼,无可奈何。(未完待续。)

------------

\end{this_body}

