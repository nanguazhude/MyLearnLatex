\newsection{是机缘还是麻烦?}    %第二十四节:是机缘还是麻烦?

\begin{this_body}

一般而言,南疆、中洲、北原、东海、西漠这五域中的荒兽,难以跨越界壁,前往他域。

这些荒兽、上古荒兽、太古荒兽,可以媲美蛊仙,相当于一域之精华,很难脱离抚养它们的那一域天地。

五域界壁,对它们而言,就像是一个枷锁。

相比较而言,反倒是普通野兽,兽王,异兽,更加自由些,可以出入界壁。

但现在追杀方源的这群上古云兽,并非南疆抚育造就,而是来源于太古九天之一的白天。

五域界壁,对它们来讲,都没有亲疏远近之分,一视同仁,没有任何的束缚或者排斥。

或许也是因为不论哪一层太古九天,都是覆盖整个五域。

这个特性,早已广为人知。

之前,魔尊幽魂逆天炼蛊,天庭蛊仙之所以能这么快赶到南疆,就是利用监天塔,穿梭白天。

再之前,方源的星门蛊,能够利用太古黑天中的星光做引,跨越五域。

两者的本质,和现下这群上古云兽是相同的,都是利用的太古九天的这份特性。

方源在界壁中急速穿梭。

身后的这群上古云兽,方源也想管也管不过来,只能任由它们跟在身后。

这群云兽都是一根筋,盯上了方源,不达目的誓不罢休。

这个时候,方源不禁又怀念起定仙游的好处了。

若是有定仙游。他定能干脆利落地摆脱这些麻烦。

就算方源身上有伤,上古云兽追踪方源总得有个范围的极限。利用定仙游,超出这个极限即可。

但方源身上的移动仙蛊,只有剑遁仙蛊一只。

一味狂催剑遁仙蛊,虽然可以拉开距离。但付出的代价就太大了点。

最关键的一点,上古云兽的追踪距离的极限方源并不清楚。万一这个距离极限很高,方源就算耗费了整个身家,估计还不够催动剑遁仙蛊的损耗。

“琅琊地灵虽然单纯,不会撒谎,但他也一直在算计我呢。”

“之前对付戚灾,就被他算计了一把。输送暗渡仙蛊借给我。可是耗费了我不菲代价。”

“我若是急于摆脱上古云兽的追杀。恐怕还要被他趁火打劫!毕竟落魄谷、荡魂山、智慧蛊,实在是诱人!”

眼前的界壁,骤然转为深蓝之色。

原来方源已经过了南疆的瘴气界壁,正式进入到东海的苍水界壁。

没有什么可说的,方源继续闷头疾飞。

一路畅通无阻。

界壁中没有任何生灵定居生存,也无其他艰难险阻。(WWW.mianhuatang.CC 好看的小说

因为五域界壁本身,就是最大的阻碍了。

又疾飞一阵。方源扭头回望,那群上古云兽仍旧在他身后,不断追赶。

云兽没有固定的形态,腾飞之时,宛若流云滚动,洁白如霜雪,姿态曼妙。但落到方源眼中,却是实在讨厌。

到此时,方源的琅琊派贡献已经完全烧光,开始倒欠琅琊地灵仙元石。

“如果不是这些上古云兽。我何至于此?嗯?怎么回事?”

飞行途中,方源忽然神色变幻起来,眼中精芒瞬间暴涨。

他惊奇地发现,在这苍水界壁之中,自身的气息开始发生一种玄妙的改变。

属于南疆蛊仙的气息,正在不断地减弱缩小,而另一股属于东海蛊仙的气息。却随之不断地壮大。

若是方源此刻催动见面曾相识,他绝不会如此惊奇。

关键是,他此刻并没有催动这个杀招。

“难道说,我的这具全新的身躯,不仅可以自由地穿梭五域界壁,而且还可以不断转换气息,变成各域的蛊仙?”

方源心中暗自猜测。

当他正式飞出苍水界壁的那一刻,他浑身上下的南疆蛊仙气息彻底消散,真正转变成一位东海蛊仙。

方源当然又惊又喜。

“这至尊仙胎蛊真是玄妙,我现在根本不用见面曾相识,就能完美地融入到五域当中去,不会被本域蛊仙排斥。”

“不过现在,我还是催起见面曾相识为妙。”

方源回头望了望身后吊着的那群上古云兽,叹了一口气,老老实实地催起见面曾相识,变化成一副陌生相貌。

他原本想要静悄悄地穿过东海,回到北原的。

但现在身后吊着这么一群上古云兽,原先的想法自然落空了。

如此招摇,并非方源所愿,但他也实在是没有办法。

“不过,这里距离乱流海域,却是不远。或许我可以绕一个路,转折到乱流海域,借助地利,甩掉这群上古云兽?”

方源脑海中,忽然泛起了一个念头。

得益于他五百年前世的颠沛流离的人生经历,使得方源对五域的地形都了然于胸。

思考了一会儿,方源还是选择谨慎稳妥,放弃了这个想法。

乱流海域宛若迷宫,一旦不慎陷入进去之后,就会身不由己,把自己给坑了。

传闻中,乱流海域乃是蛊仙大能生死激战,形成的战场。一场血战,许多蛊仙大能陨落,留下不少传承和遗藏。

许多东海蛊仙,以及来自中洲、北原、南疆的蛊仙,常会来乱流海域探索,寻求机缘。

可惜,方源五百年前世,就从未听说过有人从中获得巨大机缘的。反而时常听到:某某蛊仙在乱流海域中失踪。或者某某蛊仙在消失数个月或者数年之后,脱困而出,重新现身。

乱流海域并不危险,但是因为特殊的地形,导致蛊仙常常被困。不得自由。

方源身形似箭,在高空中划出一道笔直的光线。

他没有选择乱流海域的方向,而是直朝最靠近北原的界壁飞去。

剑遁仙蛊催起,渐渐将身后的上古云兽甩远。

他早在许久之前,就咨询琅琊地灵。询问上古云兽的情报,但琅琊地灵所知不多。

方源又派出数十股意志,在宝黄天中搜寻,企图寻找到上古云兽的追踪距离的极限,但一直没有可喜的音讯。

没办法,方源只好先试着尝试,看看能不能甩开上古云兽。

狠下心来。方源一直催动剑遁仙蛊。一刻不停!

终于他将上古云兽甩出视野之外。

方源停下剑遁仙蛊,动用凡道杀招,继续飞行。

速度因此陡降。

“付出如此代价,若能甩开就好了。”方源心中暗暗担忧。

看不到上古云兽了,并不代表已经甩开了它们。

这种媲美七转蛊仙的存在,追踪的方式,往往并非简单的视觉目光。就看它们的追踪距离的极限究竟是多少了。

就在方源时不时担忧回望的时候。一道血芒,忽然从东南方向飞来。

血光划破天际,察觉到方源之后,顿时速度微微一滞,然后向着方源迅速接近。

“是一位六转血道蛊仙,似乎经历了一番激战!”方源察觉到来者的气息,顿时皱起眉头。

不管来者是何意图,方源已经不想再惹上什么麻烦。

他已经够麻烦的了。

身后的那群上古云兽,还不知道有没有甩掉。还有两个月不到的时间,灾劫就要来临。

于是。他立即转折方向,远离主动接近过来的血光。

血光中的蛊仙察觉到方源的意图,连忙大叫:“仙友慢走!我在乱流海域寻得无上机缘,只要你能助我脱离此劫,我愿动用血誓仙蛊,发血誓,将这份无上机缘和你分享!”

“乱流海域?无上机缘?还真有人从中获得过好处吗?”方源心中着实诧异了一下。

就在这时。又有数位蛊仙从血光的后方出现,急逼而来,气息强烈,气势汹汹。

他们似乎听见了,刚刚血道蛊仙对方源喊的话。

一位位皆连叱骂道

“血道魔头,危害苍生,人人得而诛之!”

“一律和魔头同流合污者,杀无赦!”

“前方仙友,拦下这位魔头,我汤家必有赏赐。”

“无须如此。这魔头中了我刘青玉的仙道杀招,肯定跑不了了。前方来者,若识好歹,就赶紧给我滚!”

一时间,方源被这些东海蛊仙又是劝说,又是警告,又是喝骂的。

方源冷哼一声。

他绝非怕事之人,但此刻他的当务之急,是要赶往北原琅琊福地,好应付仙窍灾劫。

方向一折,方源直接朝血道蛊仙,反向撤离。

血道蛊仙大急,他已经到达了自身极限,方源是他唯一的生存希望。

方源一退,他就连忙也改变方向,迅速朝方源靠拢过去。

“你若助我,我愿将这份机缘,都统统送于你!”他大喊道。

方源冷笑,义无反顾地后撤。

见此,东海蛊仙中有人大笑:“对!你这小子识时务得很,快给老子滚!”

也有人喊道:“前方仙友,不妨出手相助我等,事后必有酬谢。”

方源眉头皱起。

自己只想悄悄地回归琅琊福地,怎么一路行来,麻烦事一桩接着一桩。自己不想沾染,但这些麻烦却主动来黏上他。

“仙友,这就是传承的关键信息,你若得之,必将修为突飞猛进,成为仙上之仙啊!”血道蛊仙抛出一只信道蛊虫,飞给方源,企图诱惑他。

方源冷喝:“滚开!”

挥手一扫,发出一缕劲风,就将蛊虫击爆。

与此同时,他身形陡然拔高,一飞冲天。

血道魔修算计不到方源,见此顿时绝望无比,他忽然打出一道奇光,射向方源。

“我已将传承的关键,交付此人。你们要取机缘,杀我无用!”血道魔修大喊一声,身形猛折,向着反方向急退。

奇光速度极快,向方源逼来。

方源冷哼一声,极不愿沾惹麻烦,低喝出声道:“老子有要事在身,都别来惹老子!”

说着,他同时催起剑遁、剑光两只七转仙蛊。

剑遁让他速度激增,剑光仙蛊则直接劈中奇光,将其粉碎。

“啊,两只七转仙蛊!”

“此人是谁?!”

方源陡然爆发,众仙顿时大为惊诧,一时间脸色皆变。(未完待续。)

\end{this_body}

