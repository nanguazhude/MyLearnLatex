\newsection{宙道宗师}    %第五百七十九节:宙道宗师

\begin{this_body}

宙道梦境。

方源化身成一位青年相貌的六转蛊仙,此刻他紧闭双目,全身缭绕着点点白光。

随着周身白光越来越多,方源的气息也不断沉淀,越积越深。

忽然,他身躯轻轻一颤,整个气息旋即不稳,宛若高楼倒塌,深潭泄底。

噗。

方源张口,吐出一小口的鲜血。

“又失败了。”他睁开双眼,眼中精芒闪烁。

这已经是他练习缩时杀招第三次失败,这个专门用来辅助光阴飞刃的杀招,也是六转层次的仙招。

接连三场失利,方源每一次受伤,魂魄底蕴都受到削减。原本三千万人魂,此刻还有着两千九百六十万。

这到底不是盗天梦境,在盗天梦境中受伤,魂魄底蕴下降就十分恐怖了。

方源并无丝毫气馁之色,反倒是目光炙热,心中斗志越加旺盛。

别的不说,单单这个缩时杀招,就很有实用价值。它可不仅仅只是用来辅助光阴飞刃的,也可以用来辅助其他宙道杀招。

探索失败,方源魂魄退出梦境,回到至尊肉身中休养疗伤。

胆识蛊。

方源取出一只只的胆识蛊,用在自己的身上。

之前方源财大气粗,但现在这些胆识蛊,他都是精打细算。

胆识蛊的存量并不多了,方源打算将这些都用来治疗魂魄伤势。至于魂道的修行,只能暂时停止。

事实上,方源的魂魄底蕴不仅会因此停滞不前,甚至还会在不久的将来,因为屡屡探索梦境而导致下降一大截。这是预料之中的事情,没有办法。

魂道修行已经不是目前重点,重点是尽早地前往光阴长河,取出里面的红莲真传,这才是最明智的战略!

方源必须尽早探索。

一来,时间拖得越长,变数就越大。毕竟方源的对手可是天庭,天庭底蕴太过深厚,说不准什么时候,就又苏醒了某位大能。天庭在光阴长河中必有埋伏,随着时间推移,这些埋伏恐怕会逐渐增多。

二来,方源寄希望于借助红莲真传,抗衡天庭,阻止天庭修复宿命仙蛊。这是他目前最大的希望。但若作最坏的打算,红莲真传也帮不了他,那他就得另做打算。所以尽早取得红莲真传,也能让方源留出更多的时间,应付将来。

至于魂道修为停滞,甚至会下降,还有仙元储备不足等等问题,方源都置之不顾。当然,这不是他真的坐视不管,而是解决之道,早已经在他的手中。那就是等到荡魂山修复成功,一切就都会再次步入正轨!

除了这个方法,方源也没有什么良方,在短时间内解决这些难题。

目前,荡魂山已经修复了三四成,但魂道道痕仍旧没有达到质变,因此无法生产出任何的胆识蛊。

休养好魂魄,方源再次进入梦境。

这一次,他终于成功,将缩时杀招彻底催动出来。

前三次的失败,并不是劳而无功的,每一次失败,都将方源推近成功的终点。

方源掌握了宙道杀招缩时。

当然,梦境中获得的这招缩时,能不能在现实中施展出来,还是个问题。毕竟梦境和现实,是有区分的。

方源心中清楚的是,缩时杀招的主体内容还是正确的。就算是有差异,只要借助智慧光晕,消耗一些时间,也就能推算出正确的现实版本了。

达到这一步之后,这一层的梦境也宣告探索成功,整个修炼的场景在方源的视野中,徐徐消散。

宙道境界在此刻开始飞速上涨,但又很快停下。

“还是在宙道大师一级。不过……比较之前,是上涨了一大截,距离宗师似乎不远了。”

第二幕梦境,方源开始练习杀招光阴飞刃。

这一招更加困难,比当初的万我杀招,还要复杂得多。

方源沉下心来,不断练习,很快就感到头昏脑海,脑海中念头消耗巨大,宛若一潭死水。

方源极有耐心,一遍遍不断重复枯燥无味的内容,向着最终的胜利不断迈进。

练习的过程中,他再次受伤。

出了梦境,魂魄归体之后,方源惊愕地发现,这一次受伤,不仅是自己的魂魄底蕴下滑,带有伤势,就连自己的寿命都受到了一些削减!

“在这个梦境中受伤,居然会损失寿元!”方源心头一跳,这是一个全新的发现。但好在他手中有着大量寿蛊,损失一些寿命,无伤大雅。当然,换做其他蛊仙,未必就这么想了。

就这样,前前后后,方源又受伤四次,终于在梦境中将光阴飞刃杀招,成功地施展出来一次。

仅仅一次就够了,毕竟不是现实。梦境逐步化解,方源成功地进入接下来的第三幕。

老者再次出现:“耗费六年,终于掌握了光阴飞刃,还算不错。但真正成果如何,还是要在实战中检验。”

方源抱拳行礼:“还请您明示。”

老者哈哈一笑:“七转蛊仙厉薪乃是那人的左膀右臂,南部上古荒兽动乱,他正要去那里镇压,你杀了他!”

方源眉头微皱,他在梦境中的这个身份,只是六转修为,本身底蕴并不突出,如何能战七转?

唯一的优势,就在于他掌握的杀招光阴飞刃!

还未等方源继续深思,梦境陡然变化,周围忽然升腾起熊熊火焰。一个人在焰海中傲然直立,一对三角眼死死地盯着方源:“原来是薄家的余孽,哼,受死吧!”

话音刚落,呼啦一声,暗红色的火焰宛若巨浪,向方源盖压过来。

方源连忙避退,关键时刻,催动仙窍内的宙道仙蛊。

下一刻,火浪将他彻底淹没。

方源战死。

梦境探索失败,方源拖着重伤的魂魄,再次归入肉身。

这一次梦境,实在是有些坑。说打就打,根本让方源没有反应的机会。

“而且我除了光阴飞刃杀招之外,只能单纯地利用宙道仙蛊,其他的杀招都不会啊。”

方源苦恼不已。

这些宙道仙蛊,都是六转层次,单独使用,如何能面对七转蛊仙厉薪?而且他那暗红色的火焰,很明显就是七转杀招!

“看来,只有以攻对攻,直接用光阴飞刃对抗了。”方源沉思片刻,下定了决心。

若是现实中战斗,方源自然会以稳妥为主。但这梦境却不一样,和现实有着区别。

休养片刻后,方源再次进入梦境。

暗红色的火浪向他扑来,方源不闪不避,催动仙道杀招光阴飞刃。

一柄薄薄的飞刀,在瞬间凝聚成形。

飞刀飚射而出,光芒一闪即逝。

“呃!这是什么杀招……”厉薪满脸惊愕之色,然后旋即栽倒下去,暗红色的火焰到卷,将他顷刻焚烧成灰。

方源心中也是一震。

“这杀招厉害!能如此轻易地斩杀一位七转,并且这厉薪全身都有周密的防护。就算是在梦境中,这样的为嗯呢该……恐怕在现实里,也是不差!”

厉薪一死,梦境消散。

第三幕探索成果,方源正式成为宙道宗师!

------------

\end{this_body}

