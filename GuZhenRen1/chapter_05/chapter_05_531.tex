\newsection{四千年寿蛊!}    %第五百三十三节:四千年寿蛊!

\begin{this_body}

%1
荡魂山乃是天地秘境,五域唯一。

%2
荡魂山产出胆识蛊,乃是垄断贸易,只要魂魄供应得上,便是开采不竭的金矿。

%3
这样的价值非常巨大,再乘以十倍,就是琅琊地灵要赔偿方源的代价。

%4
若是寻常福地,乃至是洞天,这样的赔偿,直接把自己本身都搭进去还恐怕不够。但对于琅琊福地而言,却是完全有着偿还的能力!

%5
当初,方源和琅琊地灵定下这份赔偿的盟约时,方源怎可能不考虑这一点?

%6
毫无疑问,琅琊福地乃是五域第一福地,尤其是在王庭福地被摧毁之后。它底蕴极其深厚,完全有能力进行赔偿,这一点方源毫不怀疑。

%7
别的都放在一边,方源首先就向琅琊地灵提到了智慧蛊:“我需要将智慧蛊挪移到我的仙窍当中来。所以第一项赔偿就是,我派搬迁智慧蛊的方法。”

%8
这个方法,乃是琅琊派传承的核心,方源一直想要谋取,但是就算是之前双方互换真传,也没有得手。

%9
琅琊地灵将这个方法死死地捂住,他并不蠢笨,若是将这个方法交换给了方源,那岂不是代表着方源能够将智慧蛊自己带走?

%10
智慧蛊虽然是野生,但并不属于琅琊派,当初是琅琊福地的毛民蛊仙们,来到狐仙福地,将智慧蛊带来的。

%11
琅琊地灵的脸色顿时一变。

%12
方源直接提出这个要求,就像是一柄利剑,直接刺向琅琊地灵的心口,直接得让琅琊地灵不得不答。

%13
琅琊地灵一直的图谋是:利用寿蛊不断喂养智慧蛊,企图控制智慧蛊。所以他才一直心甘情愿,去付出寿蛊,来帮助方源豢养智慧蛊。

%14
可惜的是,直至现在,琅琊地灵都未能如愿。这个方法,始终不见效果。

%15
方源现在提出这个要求,等若是再次要将智慧蛊索回,如此一来,琅琊地灵之前的付出就都打了水漂,为方源做了苦功。

%16
他当然不甘心!

%17
但是要让他明确地反对,他又说不出口。

%18
毕竟,智慧蛊虽然是野生仙蛊,但终究是毛民蛊仙从方源的福地中取过来的。

%19
于是,琅琊地灵语气迟疑地道:“这样……好吗?其实,若我们能掌控得住智慧蛊,便能以它为核,建造出仙蛊屋隐士居。有了此屋在,不仅能护持自身,还能防备几乎天下所有人的算计啊!”

%20
方源笑道:“太上大长老,你的计划虽好,但是实现起来,却是非常困难,你有把握能尽快地实现它吗?”

%21
琅琊地灵噎住。

%22
方源又道:“现在局势危险,你遮护智慧蛊不利,被凤九歌发现了智慧蛊。凤九歌发现,代表着天庭也发现,他们绝不会善罢甘休,一定会再次入侵琅琊福地。到那时,若让他们夺走智慧蛊怎么办?”

%23
琅琊地灵深深叹气。

%24
方源说的是事实,隐士居只是空中楼阁罢了,摆在眼下的是天庭方面的进攻。

%25
其实,琅琊地灵也已经尽力而为,遮掩智慧蛊了,奈何凤九歌的侦查手段另辟蹊径,极其犀利,终究是被后者发现。

%26
智慧蛊曾经是星宿仙尊之物,现在被发现,可想而知天庭方面定然会无比重视这里,下一次入侵,必定是来势凶猛,威胁极大。

%27
方源说的句句在理,但是真让琅琊地灵放弃他的计划,真的是让他不甘心啊。

%28
这时,方源再道:“太上大长老,其实隐士居虽好,但也有不利之处。若真的将琅琊福地建设成世外桃源,无人问津的避难之处,那么设想毛民蛊仙们还会有积极修行的动力吗?我们要打造的是毛民的霸业,让这些毛民蛊仙生活在隐士居的遮护下,久而久之不就又回到从前的样子了吗?”

%29
“嗯?”琅琊地灵顿时一愣,方源说的大有道理啊。

%30
毛民蛊仙欠缺的就是一股危机感。

%31
前任琅琊地灵将毛民们保护得太好了,让这些蛊仙都几乎没有作战的能力。以至于影宗入侵,差点就被影宗征服。

%32
再不能回到从前的样子!

%33
琅琊地灵深呼吸一口气,沉重地点点头:“方源,你的这个要求,我答应你了。这个方法,我会原原本本的交给你。十座荡魂山的赔偿,这个方法能值半座。”

%34
“好。”方源没有怀疑琅琊地灵的报价。

%35
琅琊地灵不会说谎,和他交易谈判,对方源非常有利。

%36
“接下来,我需要一些寿蛊。”方源提出第二个赔偿的要求。

%37
他若是将智慧蛊接到至尊仙窍当中去,无疑将来会负担其喂养,方源手中虽也有寿蛊,但靠这些存货,一点都不保险。

%38
“行。”琅琊地灵这一次答应得很爽快。

%39
他是地灵,只有琅琊福地存在,他就能随之存在,所以对寿蛊的需求为零。

%40
当然,琅琊派上下的毛民蛊仙们,也需要寿蛊延寿。

%41
不过,毛民乃是异人,单论寿命,比人族要多不少。其次,喂养智慧蛊明显比给毛民蛊仙延寿,要更加重要。

%42
于是,方源便用了一座半的荡魂山赔偿,取得琅琊福地中的大量寿蛊存货。

%43
长毛老祖乃是三十多万年前的人物,琅琊福地历史极其悠久,经历了三十多万年,搜刮到大量的寿蛊。

%44
此次赔偿,虽然方源掏走了寿蛊的大半存量,但琅琊福地中还有剩余!

%45
方源算了一下,这样一来的话,自己手中的寿蛊若全部用在他自己的身上,就能够让他延寿到四千多岁!

%46
红莲魔尊才活了三千多年而已。

%47
这也就意味着,方源若是将这些寿蛊全用了,他能够活得比红莲魔尊更久!当然,前提还是这段时间里,他不会被外力屠戮了性命。

%48
而将这些寿蛊用来喂养智慧蛊,显然能支持很长一段时间。

%49
不过红莲魔尊是所有尊者中,寿命最短的。无极魔尊、狂蛮魔尊活了六千多年,盗天魔尊有七千多岁,而巨阳仙尊的年龄则超越了八千年。活得最久的三位,都入主天庭。元莲仙尊一万两千多岁,星宿仙尊一万九千岁,而元始仙尊则高达两万五千多岁,他是人族历史上第一位仙尊,同样也是所有尊者中活得最久的人。

%50
“仙魔尊者活得都挺久,不过历史上,也有不少八转、七转的蛊仙,也活得很久,有着数千岁的年龄。活过一些仙尊魔尊的,也不乏其人。”

%51
“只是现在五域乱战就要来临,大时代要开幕,这种时代,寿蛊会越来越少。尤其是当大梦仙尊成就了九转,寿蛊将极其稀少,天地不在出产,直至寿蛊存量全无。”

%52
方源暗暗思量,自己手上的这批寿蛊,价值极大,并且还会随着时代的前行,而不断升值。

%53
按照历史上的记载,许多大能老怪,为了延寿,多活几日,几乎什么代价都能付出。

%54
寿蛊只是凡蛊而已,但史书上有很多例子,记录着寿蛊交换仙蛊的笔笔交易。

%55
方源现在所要赔偿,琅琊地灵计算的价值,是按照寿蛊当下的价值来算的。这无疑让方源又占了许多便宜。

%56
当然了,他的荡魂山若要发展好了,前景也会非常可观。

%57
定下了寿蛊的赔偿后,方源又用一座荡魂山的赔偿,换得大量的琅琊派门派贡献,还有仙蛊奴兽。

%58
奴兽仙蛊方源曾经借用过数次,非常实用。

%59
至于天元宝皇莲,方源用脚趾头想,都知道琅琊地灵绝对不会将这只仙蛊赔偿出去的。

%60
如此一来,十座荡魂山的赔偿,就剩下了七座。

%61
方源这次准备来个大的!

%62
“我要用七座荡魂山的赔偿,来换取我派中的一样东西。”方源对琅琊地灵道。

%63
琅琊地灵顿时紧张起来,心中一个激灵,忙问:“哦?你要换什么?”

\end{this_body}

