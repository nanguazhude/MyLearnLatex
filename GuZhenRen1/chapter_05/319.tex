\newsection{栽赃嫁祸血潮坑}    %第三百一十九节:栽赃嫁祸血潮坑

\begin{this_body}

哗哗哗。

浪潮声滔滔不绝,还有充天斥地的血腥气息,尽数扑鼻而来。

武家两位蛊仙武原句、戎豪,此时缓缓飞行在高空中,各自俯瞰着脚下的一片血色汪洋,前者面色凝重,后者带着愁苦。

“血洪倾泻,汪洋一片,如今已蔓延了方圆十数万里,淹没高山十多座,再不镇压封印的话,这场灾劫所造成的损失,将会更加严重。”武原句沉声道。

血潮天坑原本有武艺髯镇守,但武艺髯丧失了性命之后,血潮天坑就被他人攻破,里面的蛊阵也被摧毁干净,导致漫漫不绝的血潮,喷涌而出,卷席四面八方,造成如今的灾难。

戎豪忧心忡忡:“可是要让我们两个人,重新布置蛊阵,将这些血潮重新封印进去,可不容易。若是再添一位六转,结合三位蛊仙之力,才能勉勉强强啊。”

这两位俱都是六转修为,要处理这场血潮灾难并不困难,但是武庸的命令中规定了时限。要想在时限之内完成,他们两人就有些力不从心了。

但没有办法,武庸实在是无法抽出更多的力量,来处理这边的险情了。

在武独秀掌权的时代,武家疆域不断扩张,没有人提出异议,因为单凭武独秀一人,就能镇压得住整个局面。

武独秀一死,武庸上位,武家只剩下一位八转,尽管武庸表现出了才干和手腕,但武家的疆域显得太过于宽广了。

在紫山真君的谋划之下,利用了其他超级家族蠢蠢欲动的不甘之心,武家遭受数个超级家族的为难。

武庸坐镇中枢,一直在勉力调兵遣将,维持局面。

但这一次,武艺髯出事,他一牺牲,岌岌可危的武家防线上立即形成了巨大漏洞。

能够派遣出武原句和戎豪出来,已经是武家的极限。

“全力出手吧,纵然超过时限,我们也算是尽力而为了。”武原句叹了一口气,开始动手。

戎豪连忙配合。

他并不姓武,乃是武家招揽的外姓太上家老。

南疆的家族都非常排外,能够让武家打破常规,招揽一名外姓蛊仙,加入武家,这点很不容易。

血潮在两位蛊仙的手段下,不断退去,血潮天坑就像是一张巨怪的嘴,开始吞吸外面的茫茫血潮。

灾情得到了控制。

情况在不断地好转。

武原句叹道:“这一次镇压了血潮之后,重建好蛊阵,戎豪你就得接替武艺髯,镇守这里了。希望你多加小心,这段时间风头不对。”

戎豪点头:“我明白。我听闻这处血潮天坑,并非自然成型,而是血海老祖刻意凝造的是吗?”

武原句嗯了一声:“血海老祖留下七道真传,这里便是他精心布置的七道真传之一的埋藏地点。原先毫不起眼,和寻常天坑毫无两样,后来商家的一位蛊师误入此中,侥幸获得了血海真传,得到血手印蛊。此人便是商家的上一代族长商燕飞,如今已经战死在义天山了。”

商燕飞取走了血海真传之后,天坑中便涌现无数血潮,滔滔不绝。

武独秀便派遣了蛊仙,将这处地方吞并,成为武家疆域。

血潮天坑经过武家的经营,很快就成为了一处血道资源点。武家发展血道,这处资源点的贡献颇多。

“你们倒是聊得很开心。”就在这时,戎豪的心底,忽然传来一道声音。

“什么人?!”戎豪一时间惊骇绝伦。

他刚要撤销手段,卫护自身,却发现自己已经动弹不得。

他连忙用急切的目光,瞧向身边的武原句,希望他能够帮助自己。但他旋即便绝望地发现,武原句同样动弹不得,和他处境一样。

一个小小的身影,不知何时,站在了戎豪的肩头。

他只有常人的手指头大小,背生双翼,宛若蝉翅,一头紫发颇为显眼。

不是别人,正是八转智道蛊仙紫山真君!

“小人蛊仙?”武原句惊异至极,立马沉下心来。紫山真君气息内敛,并未显现出八转气息。

戎豪也紧接着开口:“你身为异人,当知道此刻是我人族天下。不管你是被人指使撺掇,还是自身行为,我都劝你不要意气用事。因为我们俩不仅是人族蛊仙,而且更隶属武家。武家乃是南疆正道第一势力,你取走我俩的性命,后果难以估料。”

“其实咱们之间也无冤无仇。我们武家也从不仗势欺人,你有什么要求,说不定武家可以帮你实现,咱们并非不能交一个朋友。”武原句紧接着道。

他和戎豪配合,软硬皆施,也算是默契。

可惜他的话,还未说完,接下来就口齿不清,目光呆滞起来。

戎豪骇然,他脱口而出:“你是智道蛊仙?!”

说到这里,他也同样口齿不清了。

紫山真君的智道手段,严重干扰了武原句、戎豪两人思考,让他们俩连说话的思维都被打断,变得面目前非。

蛊仙操纵蛊虫,需要念头,一举一动,也是先想,然后身体施行。

紫山真君直接干扰他们思考,从最开始的出发点上制住他们两个,导致这两位六转蛊仙连反击的一丝机会都没有。

轻易地控制住局面,紫山真君望着下方的血潮天坑,笑了笑。

随后,八转气息喷涌而出,转瞬间,贯穿天地,横扫四方。

脚下的滔滔血潮,都似乎收敛了喧嚣。

酝酿片刻,紫山真君忽然张口,吐出一口幽幽气息。

幽气轻飘飘,缓缓地堕落到血潮天坑当中去。它毫不起眼,很快就被滔滔血潮吞没,没了踪影。

做成这一步,紫山真君微微侧身,望向远处,淡淡地道:“该你了。”

一团七彩的奇光,显现而出,里面的人带着一丝颤音:“您竟是八转大能!依您的实力,您完全可以在南疆横着走,何苦为难我这样的小人物呢?”

“我自有我的打算。得到的同时,往往意味着付出。你当初继承七幻真传时,就应该想到这么一天。”紫山真君道。

“可是在我之前的继承者,从未领过什么命令。为什么偏偏是我?”七彩玄光包裹着的神秘蛊仙,语气非常不甘。

紫山真君叹了一口气,望了望天庭的方向:“也许,这就是你的命吧。好了,用乔志材的手法,杀了这两个人。”

武原句、戎豪仍旧呆傻,动弹不得。

紫山真君生擒这两人,竟是要让这位神秘蛊仙出手。

神秘蛊仙不敢违逆紫山真君的命令,玄光迸发,不断幻化,随后轰然一声,打在武家两位蛊仙的身上,立即将他们打得魂魄消散,至于肉身则化为木雕。

紫山真君放弃控制,任凭这两具人形木雕掉落在血潮当中。

“随我走吧。”紫山真君淡淡地道。

神秘蛊仙岂敢违逆紫山真君的话?就算他不是八转大能,单凭真传中设计好的信道盟约,也足以制约神秘蛊仙了。

两人说走就走,撤退得非常干脆。

当然,紫山真君绝不会忘了扫清来往的痕迹。

只是片刻功夫后,高空中忽然卷起漫天狂风,无数云朵被悉数吹散,露出万里晴空。

八转的气息,磅礴浩荡,轰然一声,笼罩全场。

武庸面带怒色,双目微红,陡然降临。

是谁?

究竟是谁出的手?!

武艺髯死了,这次,武原句、戎豪竟也死了。有着命牌蛊、魂灯蛊,武家第一时间就知晓了死讯。

武庸勃然大怒,亲自出动。

武家连失三位蛊仙,一位七转两位六转,损失叫武庸心痛,更让他愤怒。

他倒要看看,究竟是谁,竟然敢突破底线,如此触怒武家!

然而下一刻,当他看到武原句、戎豪的尸首后,武庸也楞了一下。

“木像杀?”他眼眸猛地缩成针尖大小,“乔志材?”

片刻之后,乔家蛊仙乔志材拼命赶到了现场。

血潮天坑已经被重新镇压,在被血潮侵蚀得面目全非的地面上,并排放着两具武家蛊仙的尸首。

“乔家太上大长老,你怎么说?”武庸面色平淡,语气冷静。

但越是平淡和冷静,乔志材越是心惊胆战。

他正是乔家太上大长老,而招牌仙道杀招,便是木像杀,寻常蛊仙若是中了此招,便会魂飞魄散,丧失性命,肉身化作木雕,反而留有一些生机,可以栽种成树。

不顾额头上的冷汗,乔志材当即开口澄清道:“武庸大人容禀,这绝非是我出手,而是有人故意为之,栽赃陷害。乔武两家,联盟已久,可追溯到……”

乔志材话未说完,就被武庸抬手打断。

他双目闪烁精芒,盯着乔志材:“我当然更愿意相信你。乔武两家,世代联姻,联盟关系,怎可能被这样的一件事情就破坏?”

乔志材吐出一口浊气,躬身行礼:“武庸大人明见!”

武庸继续道:“只是若真有人栽赃陷害,这未免也过于明显了些。这些人是谁?有着什么目的?接下来还有什么计划?这是我更加关心的问题。等等吧,我已经邀请了铁家蛊仙铁面神!”(未完待续。)

\end{this_body}

