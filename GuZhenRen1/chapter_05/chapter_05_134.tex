\newsection{夺蚌}    %第一百三十四节:夺蚌

\begin{this_body}



%1
和花蝶女仙的交谈,只是一个小插曲。

%2
方源正式闯入乱流海域。

%3
海水中,无数股动乱的水流,好像是纠缠一起的麻花,又仿佛混乱至极的线团,充斥方源的视野。

%4
力道大手印!

%5
见左右无人,方源使出这个仙道杀招。

%6
大手拍碎了十多股乱流,使得海水一片混乱。

%7
方源循着某个方向,横冲直闯。

%8
这只是乱流海域的外围,所以可以强行突破。

%9
但片刻之后,一道巨大的淡黄色乱流,横亘在方源的面前。

%10
“这是黄泉水。”

%11
方源停下脚步。这股水流如此巨大,力道大手印都拍不碎。反过来,还可能殃及自身。

%12
黄泉水,来源于黄泉海域。

%13
那是僵盟总部的地盘。

%14
堪称五域中第一的养尸场,黄泉海域中充斥僵尸骸骨,随波飘荡。海底深处还有无数的腐烂珊瑚,珊瑚之间,生长着密密麻麻的绷带海草。

%15
黄泉水,本身蕴藏道痕,也是一种炼蛊良材。

%16
方源撑起防护手段,直接投身进去。

%17
这股黄泉水流,自有流动的方向。但方向并不和方源的目标相符,方源只得横穿这股水流,催动仙蛊带动自身,消耗仙元不在少数。

%18
小半盏茶的功夫,他才横穿了黄泉水流,来到另一道乱流之中。

%19
这股流水呈现苍白之色,方源进入其中,宛如在云海中穿梭。

%20
一幕幕的幻景,浮现在他的眼前。

%21
千幻水,来源于东海的千幻海域。

%22
方源动用智道手段,轻松抵御住眼前的幻觉干扰,花费不少功夫,进入第三股乱流。

%23
火浆乱流。

%24
这股乱流,却不是来源于东海,而是西漠最大的天坑炎煌天坑。

%25
天坑深不见底。通达地底极深处。每隔百年,都会从地脉中涌现出海啸般的岩浆。炙热的岩浆,不仅灌满整个天坑,还会漫溢出来。延祸方圆十数万里,改变地貌。

%26
火浆乱流,就是来源于岩浆深处,温度极高。

%27
方源进入其中,必须催动仙道杀招进行防护。寻找的六转蛊仙。没有仙级防护手段,只能躲避这股乱流。

%28
仙元消耗的速度,是之前两道乱流的数倍!

%29
但好在这股火浆乱流,并不庞大,方源耗费了几十个呼吸,就穿过来了。

%30
这一次,他没有遭遇乱流,而是进入了一个空无一物的空间里。

%31
除了方源之外,没有任何存在。

%32
回首望去,方源的身后。正是火浆乱流在滚滚流淌。

%33
而在左前方,是另外一股,漆黑如魔,浪花在内部翻溅,仿佛万千冤魂哀嚎。这是黑魂水,来自黑魂海域,内蕴魂道道痕、暗道道痕。

%34
另外在右前方,还有第三股乱流。一片黄金灿烂,仿佛无数碎金混杂一起,缓缓前行。璀璨夺目。

%35
方源见此,有些惊异:“若是我没看错,这是碎金流。传闻中,太古黄天中有许多条金属天河。太古黄天破碎。绝大多数的金属天河都随之泯灭。这道碎金河流,恐怕是某处太古黄天碎片世界中的。因为乱流海域的特性,汇集了天下各种河流于此处,所以我才能在此亲眼目睹。”

%36
火浆流、黑魂流、碎金流。

%37
这三股乱流相互纠缠,不分胜负,在三者之间。力量达成某种玄妙的平衡,就开辟出了一个空白的空间。

%38
这就是方源存身之地,仿佛是龙卷风当中的风眼一样,不能长存,但能够蛊仙提供宝贵的休憩机会。

%39
方源并没有休整,他实力大增,仙元储备众多,此时还远远未到达他的极限。

%40
他辨别方向之后,身形如电,钻入碎金乱流之中。

%41
前进的阻力很大,是之前在火浆乱流中的十几倍。但是仙元的消耗,却减少很多。

%42
毕竟火浆乱流过于炙热。

%43
方源横穿整个碎金乱流。

%44
在这个过程中,他随手收取了上千斤的碎金。

%45
这要搁在地球上,必定是一笔天大的财富。但在这个世界,也就如此,碎金只是一种比较普遍的蛊材而已。

%46
除了碎金之外,碎金乱流中还有大量的蛊虫。

%47
几乎都是金道。

%48
有金龙蛊,形似蚯蚓,但仔细看的话,有爪有头,像一头小龙。

%49
还有金霞蛊,这是可以令蛊师飞行的移动蛊虫。

%50
更有一种让凡人蛊师为之疯狂的罕见凡蛊浑金蛊,这种蛊是一次性的消耗蛊,但可以增强蛊师的资质!

%51
对于此时的方源,浑金蛊自然用处很小了。

%52
他很随意地收取了一些,只是随手而为。

%53
成功地横穿了碎金乱流,方源的至尊仙窍中,就增添了一条碎金小河。

%54
方源将这条小河,搁置在了小黄天之中,只能算是增加一些风景。靠这条细小的碎金小河,可收获不了什么东西。

%55
蛊师估计会如获至宝,但方源这等蛊仙,就完全不放在眼中里了。

%56
方源穿过一条条的乱流,渐渐从乱流海域的外围,向中央靠拢。

%57
片刻之后,一道恢弘的巨流,拦截在方源的前方。

%58
巨流通体深蓝,每一滴水上都电光微闪,使得整个巨流都熠熠生辉,夺人眼目。

%59
方源吐出一口浊气,知道这是雷水流。

%60
雷电一旦积蓄到某个极限,就会质变成水液。由狂暴的毁灭力量,转变成了可以激发潜能的生机之水。

%61
雷道的蛊仙,常常建立雷池,池子中就是盛满这种雷水。

%62
雷道蛊仙用这种雷池,豢养出大量的雷道凡蛊。

%63
雷池对于雷道蛊仙的意义,就相当于五光山对于光道蛊仙。

%64
方源到达这里,已经是乱流海域的中部地带。

%65
依凭他的实力,他再也不能横穿乱流了。乱流的力量,已经变得更加庞大。方源一旦被卷入其中,只能随波逐流。

%66
这个情形下,他要前往目的地,就要看运气了。

%67
运气好,碰到一股乱流。水流的方向和方源的目的地匹配,方源就能顺利前行。

%68
运气不好,数十股乱流之后,蛊仙只能原地转圈子。

%69
这就是乱流海域的奇特地貌。

%70
方源进入雷水乱流当中。

%71
这股乱流十分庞大。方源就好像是掉进了湍急大河中的小兔子。

%72
好在雷水的力量,十分温和。只有浪花尖端,才会爆发出雷电,拥有摧毁万物的猛烈威能。

%73
方源只管沉入雷水深处,仙元消耗很少。平静而又安稳。

%74
他顺着这个巨流向前穿行。

%75
约莫过了一盏茶的功夫,方源来到雷水乱流的尽头。

%76
尽头处,是另外一股乱流,规模和雷水流不相上下。

%77
方源又钻入其中,顺着水流不断前进。

%78
他的运气不错。

%79
虽然有些波折,但总体上,还是不断接近目标的。

%80
一天一夜之后,方源来到了一处乱流相互掣肘,形成的平衡空间。

%81
这已经是他一路上,遇到的第十二个空间了。

%82
“这处空间才刚刚形成不久。周围的乱流暂时不会改变,就在这稍微休整一下吧。”方源打算在这里休息片刻。

%83
因为即便是他,此刻也感到了疲累。青提仙元的消耗,也不算小。

%84
哪知刚刚休息了不久,从方源头顶上的乱流中传出了一些动静。

%85
随后,一只巨大的水蚌,从巨流中钻了出来。

%86
这头水蚌有三四头大象的大小,散发出上古荒兽的气息。表面白金之色,无一点杂斑。

%87
见到方源之后,这头水蚌的动作明显楞了一下。

%88
随后。一个声音从水蚌中传来:“咦?这里居然有人。你区区一位六转蛊仙,怎么可能深入到这里?”

%89
方源默不作声。

%90
水蚌缓缓展开两片贝壳,露出里面的淡红软肉。

%91
这层软肉就像是床榻靠椅,软肉上坐着三人。

%92
两位凡人女子。甚是美貌,衣着暴露,偎依在中央的蛊仙男子身上。

%93
这位蛊仙男子,青年模样,身材削瘦,目光轻佻。居高临下地看着方源,肆意打量,有些奇怪地道:“古怪,古怪。真的是六转修为,快快报上名来。”

%94
方源静默,目光从蛊仙男子身上,转移到了巨蚌。

%95
蛊仙男子有些生气:“喂,我问你话呢。你是聋子还是哑巴?”

%96
方源不答反问:“这就是传闻中的珍兽藏娇蚌么?”

%97
蛊仙男子顿时倍觉有面子,哈哈大笑:“你倒是有些眼光,这可是小爷我好不容易从我爷爷手中讨要过来耍的。我爷爷便是东海中大名鼎鼎的奴道蛊仙任修平!”

%98
方源点点头,淡淡地道:“任修平我知道,此人是七转巅峰强者,前不久传闻,他终于找寻到失散多年的亲孙子,并耗费不菲代价,助他成仙。就是你吧?”

%99
“哈哈哈。”蛊仙男子再笑,“不错,不错。我看你有些顺眼了,眼光不差。”

%100
“嗯。乖乖地献上藏娇蚌,我就饶你一命。”方源神情平淡。

%101
“什么?!”蛊仙男子差点以为自己出现了幻听。

%102
他神情一僵,旋即怒极反笑,手指着方源:“好,好!看来我是夸错了,你长着一双眼睛,却是睁眼瞎,得罪了不该得罪的人!真是不知死活的东西!就让小爷我啊!”

%103
蛊仙男子说了一半,就发出大大的一声惊呼。

%104
原来方源已经动手。

%105
力道大手印将藏娇蚌狠狠抓住。

%106
蛊仙男子满脸惶急惊恐之色,因为他发现以为屏障的藏娇蚌,正被力道大手肆意蹂躏,两片贝壳都开始发出咔嚓的微微声响。

%107
“误会,误会。”他连忙大叫,“我可是任修平的亲孙子。你是哪位蛊仙,说不定咱们还有交情呢!”

%108
“谁和你有交情?要我放过你也可以,把控制藏娇蚌的蛊虫交出来。”方源冷笑,“否则的话,你就死在这里吧。”

%109
“你是魔道蛊仙!你不能这样,我爷爷不会放过你的。你要好好考虑!”

%110
“任修平而已,算得了什么。”方源嗤笑一声,旋即身上气息升腾起来,从六转一路拔升,成为七转巅峰的气息。

%111
青年蛊仙顿时面如土色,身边的两位凡人美女浑身颤抖,早就花容失色,缩成一团,瑟瑟发抖。

\end{this_body}

