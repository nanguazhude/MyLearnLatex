\newsection{超级任务}    %第八百九十九节:超级任务

\begin{this_body}

对于超级世家而言,这世间从未有永恒的朋友,亦从未有永远的敌人。

这是超级势力的生存之道。

若不遵循这样的道理,超级势力往往长存不久。沈家自然不是这样的势力,因为它从乐土仙尊之前就已经存在,乐土仙尊时期发展壮大,长存至今。

作为曾经沈家的一员,沈伤对于沈家和方源联手,并不反对,甚至还感到欢喜。

因为沈家来的这批人中,都没有阵道蛊仙。沈伤要借助大阵封印自己,自然要借助方源的手段。

听到沈伤的这番话,沈从声神色不免有些古怪。他费尽心力救出了先祖,没想到现在先祖反而主要要求再被大阵封印。

不过,究其缘由沈从声完全理解,当下默不作声,目视方源。

方源面色微沉:“沈伤仙友,贵族虽和我联手,但海底大阵的仙蛊已经统属于我。现在要我贡献出这些仙蛊,搭建成仙阵封印你……这个要求令我感到为难。”

沈从声连忙道:“这一切都是可以谈的,方源。甚至我们可以交换仙蛊。”

到了这个地步,他重见沈伤恢复了理智,自然越加不肯放弃。

方源缓缓摇头,面容颜色:“沈从声,你忘了之前我们的任务内容是什么了。任务明确表示,要剿除大阵中的魔仙,也就是沈伤。这说明什么?”

方源顿了顿,继续道:“这说明功德碑也不看好沈伤,认为他身上的问题已经严重到无可挽回。我就算重修大阵,恐怕也于事无补啊。”

沈从声微微色变,正要开口相劝,却被沈伤伸手制止。

沈伤对方源微微一笑:“方源,你不必有这方面的顾虑。我并未有让你贡献仙蛊的意思,搭建仙阵的蛊虫都源于我。且看。”

言罢,他便从仙窍中取出一只仙蛊来。

仙蛊像是蚕的模样,通体淡黄色,蜷缩一团并不舒展,更显得娇嫩可爱。

“我的情况恐怕沈从声已经向你大致说明了。”沈伤对方源解释道,“我最擅长的便是治疗手段,这便是我当年开创的人道仙蛊——扶伤。我以此为核心,后来又开发出一记人道杀招名为救死扶伤,难度不高的话,就算是死人我也能当场复活。”

方源看着面前的这只蛊虫,不免心头震荡。

这显然是纯粹的人道仙蛊,更叫他惊异的是,这是一只货真价实的八转仙蛊。

八转人道仙蛊!

沈伤见群仙默然,微微一笑再道:“你们剿除黑火,是否发现对付黑火,人道手段其实最为有效?”

沈从声摇头:“这点尚未发现,但我们已知道就算是治疗杀招,打在黑火上,照样是有效果的。攻伐杀招、治疗杀招对付黑火,并无区别。”

沈伤点点头:“且看。”

说着,一股气势便从他的身上涌动而出。

见他酝酿杀招,周围的蛊仙连忙后退,神情警惕。

沈伤恢复了神智之后,并无害人之意,他缓缓伸开手掌,遥遥对准远处的黑火。

他的掌心忽然迸射出一股嫩黄色的光流。

光流涌动,螺旋前行,速度并不快,最终射入黑火当中。

黑火在黄光中支撑了片刻,迅速消散。

沈伤停止杀招,对众仙道:“你们刚刚看到了,这便是我的杀招救死扶伤。”

“果然人道杀招对黑火相当有效。”眼见为实,方源当即点头,认可了沈伤的说法。

“果然如此。”沈从声欢喜道,“之前的海底大战乃是土道仙阵,对于先祖身上的问题难以解除。但若动用先祖的人道仙蛊,便大有希望了。”

方源却未就此松口,而是凝神望着沈伤:“你可是人道准无上境界?”

沈伤眼中闪过一抹惊讶,他点点头:“正是如此。没想到方源你也精通人道。”

隔行如隔山,流派也是如此。

方源既然能够猜测到沈伤的人道境界,显然他在这方面也有不少造诣。

在场的众仙顿时又向方源投去异样的目光,纷纷在心中感慨:这就是名动天下的大魔头,深不可测!

方源又问:“可有阵图?”

沈伤笑了笑,流露出无奈之色:“的确有一份阵图,不过是残破的,好像是我自己开创,可惜我记不住全部。”

“什么意思?”沈从声紧张起来。

沈伤深深地叹息一声:“我失忆了。每当我陷入疯魔一次,掌握的记忆便会消失一部分。所以要尽快解决我身上的问题,迟了的话,我连我自己是谁,恐怕都不清楚了。”

方源笑了一声,道:“沈伤,你运气不错,碰到了我。”

沈伤颔首,赞同道:“这的确是我的运气。要布置出一座满足我要求的人道仙招十分艰难,条件很多。首先需要人道仙蛊,其次要推算改良阵图,最终布阵。布阵需要阵道造诣,推算改良阵图需要智道造诣,又因为是人道仙阵,还需要人道造诣。”

“换做任何一个专修阵道的蛊仙,哪怕是八转修为,都满足不了我的要求。但偏偏是你方源,你这三方面的条件统统满足。所以接下来,我真还得依赖于你。请你务必出手相助!”

“方源,你需要什么报酬,尽管提出来,沈某连同沈家必定会竭力满足。”沈从声郑重其事地许下承诺。

方源点点头,沈伤刚提出铺设仙阵的要求,他就意识到了这是一个好机会。

不仅是他布阵,可以从沈家获取丰厚的仙材回报。

更让他在意的是沈家、沈从声以及沈伤这些人。

通过和沈家合作,他可以更深一层影响东海蛊仙界,撬动整个天下大局。

方源的人道境界虽然已是宗师,但对付天庭的那些人道杀招,仍旧毫无头绪。若是让沈伤出手,那就完全不同了。

天庭实力太过雄厚,毕竟积累了无数年。方源一时间达不到对抗的标准,便需要合纵连横,巧妙地借助他人之力。

方源沉思片刻,说出自己的条件。

出乎沈伤和沈从声的预料,方源提出来的条件很是宽松,一点都不严苛。方源对仙材需求并不多,也没有索求仙蛊。

唯一的必须履行的条件,便是要求沈从声、沈伤必须动手,和天庭为敌,阻止天庭修复宿命蛊。

但事实上,不需要方源提出来,阻止宿命蛊修复成功已然是天下所有超级势力的共识!

许多有志之士对宿命蛊耿耿于怀,他们的担忧比方源还甚!毕竟他们都不是天外之魔,一旦让天庭修复成功,天庭的地位将牢不可破。

数日后,一座全新的大阵在海底建成。

沈伤入阵,主阵眼由沈从声主持,方源辅助。

大阵缓缓催起,一股股玄白流光形成道道锁链,将沈伤五花大绑。

沈伤沉默,盘坐着,很快他的全身上下都笼罩了一层淡白光雾。

片刻后,大阵徐徐停息下来。

方源、沈从声等人一无所获,他们根本查探不出沈伤的身体里存在着一滴一点的黑火。

沈伤叹息:“我之前已经查探了无数次,唯有当我再度发狂时,才会有黑火产生。”

方源眼中精芒烁烁不定:“你确定有关黑火的事情,全都记不起来了吗?”

沈伤只是摇头。

沈从声叹息一声:“既然如此,那就只好等待了。这个任务就先放置在这里,我会专门派遣一人留守大阵。”

这样的安排很合理。

蛊仙们在龙鲸乐土中活动范围是受限的,并不能随心所欲地出入各地。利用功德碑的传送能力,还有团队名号,他们就能即时沟通,一旦沈伤发疯,就会迅速汇集一起。

安排在大阵内的,必然是沈家嫡系,有时候甚至是沈从声亲自去坐镇。

沈家有此牵绊,在功德榜的争夺上更加不是方源的对手。

时间不断流失,一天天过去,方源功德从千数上升到万,又从万数不断增长。

在这个过程中,他也时常兑换名号。只是名号的秘密,他也不清楚。上一世的庙明神一伙人还未探索到这种程度,所以只能靠方源猜测和推断。

所幸就算他推断出错,兑换的名号也都是有一定用处的。

有了这么多的功德,方源也并没有兑换什么仙蛊。

他不缺仙蛊,缺少的是一些罕见的仙材。

有数次,方源消耗功德,从功德碑处兑换了仙材,支持毛民蛊仙们升炼仙蛊。

方源有大量的仙材,但在罕见的仙材上面还存在一些空缺。功德碑弥补了最后的短板。

当方源再次有所突破后,第一个超级任务出现了。

“解决天穹脊背上的黑火?”方源微微错愕,“怎么还有黑火没有被清缴吗?”

之前一段时间,有关黑火的任务接连出现。蛊仙们完成这些任务,为沈伤擦屁股。

“难道说,这不是沈伤留下来的?”方源脑海中忽然灵光一闪。

要验证这个猜想非常简单,借助功德碑,方源直接传送到了任务地点。

下一刻,他置身在云霄之巅,苍穹的最高处。

没有风,阳光和煦。

“这是?”方源瞳孔微缩,仰头看见无数的黑火在默默燃烧。

黑火的规模极其庞大,竟仿佛是暴雨天气中的乌云一般!

方源和众仙之前费尽全力剿灭的黑火之和,也不及眼前黑火的万分之一。

“这里的黑火之多,骇人听闻,绝不是沈伤造成的。如此说来,龙鲸乐土中本来就存在着黑火隐患!”

------------

\end{this_body}

