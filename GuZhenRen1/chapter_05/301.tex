\newsection{可借不可靠}    %第三百零一节:可借不可靠

\begin{this_body}

月亮节过去,方源却经常在外面出没。

他主动接近乔丝柳,后者虽是恼怒他在月亮节时的轻慢,但碍于家族任命,无法拒绝方源。

渐渐的,就有风声透露出来

武家的武遗海和乔家的乔丝柳,走的很近,时常游山玩水。

不管是身为武庸之弟的武遗海,还是南疆三大仙子之一的乔丝柳,毫无疑问的,都是南疆蛊仙界的知名人物。

他们的行为,很快就形成一些流言蜚语。

乔丝柳的追求者们,愤愤不平的同时,也不得不承认,武遗海是非常强大的竞争者。别的不说,光看武家和乔家的关系,就是武遗海极其巨大的优势。

这让方源有些无奈。

他其实不想出名,武遗海这个身份是越低调越好。

可惜的是,事与愿违,这一次借助乔丝柳,方源又狠狠地在南疆蛊仙界扬了一次名!

这种情况,很快就影响到了方源。

他被人推算的次数,刚刚有降低的迹象,现在又猛增上来,并且非常频繁。

光靠暗渡仙蛊,应对这种情形非常勉强。

方源值得时常落窍,躲进至尊福地当中去。

仙窍是自成天地,隔绝内外,只要方源没有关键线索泄露,被他人掌握,那么推算起来就异常困难。

除非是掌握“连运”这种程度的线索,否则的话,极难推算出具体方位或者其他情报。

好在方源本身,就要时不时地落窍,汲取天地二气。多了一条逆流河,对天地二气的消耗着实很大。

除了落窍之外,方源一边应付乔丝柳,一边则借助武家的渠道,了解罗木子、轮飞的情报。

现在还不方便动手。

知己知彼百战不殆。

蛊仙的手段层出不穷,神秘至极。有时候稍稍大意,就会阴沟里翻船。

就比如说方源通过逆流河,竟能抵抗八转蛊仙的攻势。大雪山战役前,谁能想得到这样的结果?

方源性情谨慎,就算是想要谋害了罗木子、轮飞的性命,也要在充分地了解他们的情报之后,才开始针对性地设置计划。

然后蓄势待发,一旦有了良机,便猛地出动,一击必杀!

如此才风险最小,最符合他的利益。

狮子搏兔亦用全力,更何况轮飞、罗木子能入乔丝柳的眼界,本身实力不差。

如此又过了半月时光,这一天,武庸召见方源。

“弟弟,我这里有几项任务,你可择其一项办理。”武庸开门见山道。

“兄长请讲,为家族贡献一份力气,也是我应尽之责。”方源从容答道。

武庸的袖口中飞出一只信道凡蛊,方源看了一眼,这蛊中记载了三项内容。

第一项,有关玄冥山。

近来有人发现,在这玄冥山的深处,似乎有野生的仙蛊气息。

这个消息,已经引动了不少散修蛊仙前往。

更重要的是,超级势力羊家也出动了人马。

若情况属实,能获得一只仙蛊,对于任何一个超级势力而言,都是能增长自身底蕴的喜事,武家当然不想错过这个良机。

现在武家的情况,不好不坏,在武庸的领导之下,已经稳住了阵脚。

更关键的是,玄冥山就靠着武家的范围,虽然不是武家的地盘,但和武家接壤。

若是武家这次不派遣蛊仙出马,恐怕会让其他超级势力觉得武家空虚。

所以在这项任务中,武庸还特意关照:就算夺不了野生仙蛊,也务必将其毁灭,绝不能留给羊家。

第二项事情,是近些天来赤龙江水暴涨,若是不加控制,极可能发生洪水灾害。

洪水一旦泛滥开来,会对沿途的万物生灵,造成巨大的威胁。造成资源减产,甚至会改变周围环境。

赤龙江沿途的超级势力都非常重视。

乔家、武家的地盘,也有和赤龙江接壤的部分。

所以要派遣蛊仙,前往监察,尝试控制,以防发生巨大水灾。

第三项,则是翼家的太上二长老一千两百岁大寿,举办寿宴。

武家需要派遣一位蛊仙,代表武家,前往参加寿宴,送上武家的礼物。

翼家的实力非常雄厚,这个家族的大本营乃是鳞翅山,这座山位于南疆的东北角上,和东海最为相近。

事实上,翼家和东海之间,也有千丝万缕的紧密联系。

至于武家,则位于南疆的西南区域之中,虽然不在西南角落,但在所有的超级势力当中,位于最南方。

翼家、武家相互之间如此的位置,决定了两家关系一直保持着良好的程度。

远交近攻,这是超级势力的政治外交,最为基本的原则之一。

即便是众多家族刁难武家,翼家也从未参与其中。为了维护这层关系,值此翼家太上家老的大寿,武家定然要派遣蛊仙,前往贺寿。

“怎么样,做好决定了吗?”过了片刻,武庸问道。

方源点点头:“我已考虑好了,就选择第二项吧。我战力不高,选择第一项去往玄冥山,恐怕要和羊家蛊仙交手,羊家擅长魂道,我对付这个流派并不擅长。”

“那为何不选第三项?”武庸笑了笑,“其实你的身份最适合做这项任务,因为你是我武庸的弟弟,去往翼家一定会受到热烈的招待。”

方源摇摇头,苦笑道:“兄长,你就饶了我吧。虽是贺寿,但寿宴上肯定有其他超级势力的代表。如今武家局势如此,我若参加寿宴,定然会受到其他势力蛊仙们的刁难。到那时,我双拳难敌四手,自己尴尬倒无所谓,关键是丢了家族的颜面,那罪责就大了。”

武庸哈哈一笑:“兄弟想到这一层,倒是兄长欠妥了。那就这么着吧。”

“兄长若是没有其他事情,那小弟便告辞了。”

“去吧,去吧。”武庸摆摆手。

方源便转身,当他要走出大殿之时,武庸又问:“哦,对了,你何时启程动身?”

方源便又转身回去,站在门槛附近:“赤龙江若泛滥成灾,必定生灵涂炭,这等大事,小弟认为不可轻慢,回去简单收拾一下,稍后便动身。”

“好,你有此认知,我就安心了。此事交给你,一定能办得妥帖。”武庸十分信任方源的样子。

“小弟一定尽心尽力!”方源郑重保证了一句,这才转身离开。

他却没有见到,在他离开之后,武庸脸上的笑容和信任,都缓缓消失了。

取而代之的是眉宇间的一抹阴沉。

“事情要成了。”方源心中荡漾着欢喜,表面上则不动声色,一如他对武庸所说,稍稍收拾之后,他就立即启程北上。

他一路疾飞,然而才刚刚飞跃黄龙江时,却被蛊仙武罚唤住:“武遗海大人,且慢,我有太上大长老的全新任命。”

方源心中哈哈一笑:“果然来了。”

表面上则是一脸诧异之色,他在高空中停住,问急忙飞来的武罚:“武罚长老,何事唤我?”

“唉!”武罚深深地叹了一口气,苍老的脸上全是愁苦之色,“真是屋漏偏逢连夜雨,超级蛊阵那边出了意外。巴家与我家为难,镇守那里的武碑长老已然负伤,难以继续镇压局面。大人你不是之前想要调回去吗?机会来了!”

“太上大长老正是念及此点,决定将大人你啊,和武碑长老对调一下,让武碑长老回来养伤。武遗海大人你坐镇超级蛊阵。”

方源深深地皱起眉头,有些猝不及防的样子:“怎么会这样?我还……”

他说着,似乎下意识地朝着西北方望了一眼,神情犹豫地道:“可是赤龙江这边,问题也很严重,急需要解决啊。”

“这个事情,太上大长老已经有了妥善的安排了,武遗海大人不必挂怀。大人你还是赶紧动身吧,那边急需要你去坐镇呢。这是武庸大人亲自给你的信蛊,大人直接前往义天山遗址即可,不必再返程回去武家了。”武罚说着,将一只信道凡蛊塞到方源的手中。

方源接过信蛊,当场灌注心神,查看一番。

这信蛊中的内容,自然是武庸的口吻,言说了超级梦境那边情况何等严峻,需要方源即刻动身,最好一点时间都不可浪费。

方源深深地叹了一口气:“如此的话,那我就只好前去了。”

“快去吧,武遗海大人。”武罚长老一脸着急的模样。

方源摇了摇头,又向西北方望了一眼,这才转变方向,向着东北方向飞去。

武罚长老一直目送着方源,直到他消失在天际。

武罚长老冷笑了一下,也望了望西北方向。虽都知道,西北方向上有一座山,大鹏山,正是乔家的大本营所在。

武罚长老回到武家武仪山禀告了武庸:“太上大长老,事情已经办妥了,武遗海大人,已经去往超级蛊阵了。”

武庸正在书房中,他站直身子,手中挥墨,在一张白纸上龙飞凤舞。

听到武罚的话,他微微点了点头,忽的手笔。

白纸上写着五个大字:可借,不可靠。

“很久之前,我还是一个少年的时候。母亲问我关于乔家的看法,我就回答了这五个字,我记得很清楚,母亲当场便嘴角溢笑。”说到这里,武庸叹息一声,“可惜这个道理,我弟还不太明白。”

备注:九点半第二更。

\end{this_body}

