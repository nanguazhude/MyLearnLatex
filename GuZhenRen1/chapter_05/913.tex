\newsection{天庭雄威}    %第九百一十七节:天庭雄威

\begin{this_body}

漫天的白光,宛若大雪飘飞,充斥中洲天地。

最先受益的是帝君城战场。

“强,好强大的力量!”中洲的蛊仙们感觉非常奇妙,全身上下涌动着一股全新的力量。这股力量是如此的浩荡和澎湃,宛若是汹涌的浪潮,不断地冲击心房,又旋即涌向四肢百骸。

一切的疲惫,全都一扫而空。

脑海清明至极,强烈的战意蹭蹭上涨。

不管是六转、七转,乃至八转,尽皆如此。

“这是我中洲天庭的底蕴啊!”

“我的实力至少暴涨了一倍,有这样的增幅,还怕什么敌人?”

“冲,杀过去,将所有的入侵者统统击败!让他们知道天庭之威,中洲之威!!”

中洲一方士气暴涨如虹,掀起凶猛的攻势狂潮。

西漠一方节节败退。

“退。”

“快快后撤!”

“如此增幅着实恐怖,这究竟是什么杀招?”

“让我们暂避锋芒,如此强大的杀招是不可能持久的。”

西漠一方蛊仙们慌忙交流。

“没有用的。这是星宿仙尊的手段,两记人道杀招分别是众望所归、人中豪杰,支撑整个大战是绝无问题的。”房睇长沉声说道。

西漠一方不断后撤,但房睇长操纵的豆神宫始终占据原地,宛若中流砥柱,毫无一丝退缩的意思。

房睇长一脸冷峻之色。

他没有料到,沈伤的破解居然会令这等杀招提前爆发。

在上一世,这两记人道手段是一个相当重要的筹码,改变了所有的战局。

这一世方源辛苦谋算,连沈伤联盟,就是要解决掉这两个手段。但尽管他做出了如此多的努力,结果仍旧不尽人意。就目前的结果来看,沈伤不仅没有克制和缓解这两个杀招,反而让局面更加麻烦。

人中豪杰杀招针对蛊仙本身,使得蛊仙各方面增幅一倍!

杀招威能极其恐怖,蛊仙得到增幅后,使用的杀招,包括仙蛊屋、仙阵等等,都会威能暴涨。从某种方面来讲,它比盗天魔尊的成双入对杀招还更要实用。

一波波的攻势,宛若惊涛骇浪,连绵不绝地轰击在豆神宫上。

豆神宫乃是元莲仙尊所创,也难以抵抗如此凶威,表面龟裂,创伤无数,大量蛊虫迅速死亡,惨遭毁灭。

不过豆神宫的强大之处,在于恢复,一如元莲仙尊的修为风格。

蛊虫虽然灭亡很多,但很快就能得到补充。受伤不死的蛊虫,甚至会迅速复原!

豆神宫宛若是一块青金礁石,稳稳地支撑在第一线上,默默承受中洲一方的狂轰滥炸。

“真不愧是豆神宫啊!”

“房家房睇长果然是有担当。”

“交手吧,既然已经走到了这一步,难道还要逃窜吗?”

西漠一方的军心终于稳住,西漠蛊仙们不再后撤,纷纷迎上中洲仙蛊屋。

一时间,仙蛊屋之间展开大战。

揽雀阁打开数个鸟笼,一群群飞鸟自由飞翔,飞出鸟笼之后,就迅猛涨大,形成漫天的鸟群,宛若一蓬蓬乌云袭来。

西漠一方鸡笼犬舍不甘落后,放出鸡犬荒兽群与之对战。

双方接触片刻,鸡犬荒兽们就明显的抵挡不住,被击杀很多。

“我来助你!”董家蛊仙们开赴赤河车前来支援。

赤河车酷似水车,不断地滚动,从车轮中飞出一道庞大红光,红光宛若丝带,绕着火浆鸟圈了一圈。

火浆鸟群立即被镇压大半,反被红光圈住,全部俘虏进了赤河车中。

“多谢你们的火浆鸟!”董家蛊仙放声大笑,赤河车威能因此上涨了两成!

西漠一方虽然讨了便宜,但仍旧是鸡笼犬舍和赤河车合力,对抗着揽雀阁。

这个缩影具有普遍性。

放眼战场,西漠一方的仙蛊屋往往只能以二敌一。

中洲的仙蛊屋一个个威力绝伦。岳阳宫暴射宏光,刺目冲霄;寒螭庄散发龙魂,寒气飙飞;风满楼卷席狂风,纵横战场。

从一接触,西漠一方被压着打,几乎抬不起头来。

“如此下去可不行。”房睇长纵观战场,不免忧心忡忡。

知己知彼百战不殆。

房睇长明白中洲实力,更深知己方底细。西漠的蛊仙们绝不会死战,此次纠集了这么一批人马,已经是房睇长拼尽全力运作的结果了。

如此激战,西漠一方一旦有什么伤亡出现,恐怕就会撤退了。

一旦有人撤退,立即就会有人跟从。

一言蔽之,这绝不是一个真正的军队,而是乌合之众,非常容易离心涣散。

房睇长率领这批人马,在这样的关键时刻,必须顶上,顶在最前线,保持己方队伍的完整。

“原本打算在最后关头使用……没有办法了。”房睇长叹息一声,将积蓄在豆神宫中的豆神兵卒全部派遣出去。

数量最多的是黄豆兵卒,也是最基础的兵卒,他们沦为炮灰,充分向前。

随后是绿豆兵卒可射出飞箭,红豆兵卒能够自爆,黑豆兵卒防御最强。

至于蓝豆兵卒、白豆兵卒数量最少,前者可将一定威能的杀招反射回去,后者则可以治疗本方兵卒。

兵卒大军一出,立即稳定了形势,形成场面上的僵持。但付出的代价,是这些兵卒数量的迅速消耗。

一旦折损超过一定的限度,那么西漠一方将会再次沦落下风。届时,房睇长几乎无牌可打,唯有等待支援。

毛脚山战场。

仙道杀招——三人成虎!

周雄信施展信道杀招,飞虎成群,呼啸而过,扑向武庸。

武庸连连摆手,催出无数风刃,挡住虎群飞扑。

在他身侧,朱雀儿、野樵子联袂杀来。

武庸叹息一声,明智地后撤了。

若是之前,他兴许还能周旋,但现在人道杀招增幅之后,武庸只得退避三舍。

说起来,他屡屡想要施展无限风杀招,却始终被天庭一方破坏,无法争取到施展杀招的那段宝贵时间。

“幸好我方有着仙蛊屋,可以充当堡垒,争取到喘息良机。不好!”武庸刚回到玉清滴风小竹楼中休整,忽然脸色大变。

他的离开,让池曲由的位置变得稍微突出了一点。

或许他域的蛊仙利用不了,但天庭的蛊仙却都是身经百战。

“池曲由,快退!”武庸连忙传音。

不需要他的警告,池曲由已经觉得不妙,开始后撤了。

“你想往哪里走?”万紫红牵制住池曲由。

武庸顾不得休整,飞射而出。

周雄信忽然出现在池曲由的头顶上空,手掌五指伸开,猛地一朝。

仙道战场杀招——流言笼!

这是天底下最为迅猛的战场杀招,搭建速度五域第一。

池曲由躲闪不及,被罩进战场。

周雄信、张飞熊、朱雀儿进入战场,而其余蛊仙则围在战场之外,挡下武庸、翼浩方、巴德的支援。

“池曲由危险了!”

“快,必须尽快攻破这个战场。”

武庸等人焦急万分。

然而流言笼战场已经浓缩一点,消失不见。找到它就要破费精力,更何况周围还有天庭蛊仙的全力阻挡。

仙道杀招——坐吃山空!

趁着南疆群仙全力营救的关口,赵山河和玉珠子再次合作,杀招重创南疆数座仙蛊屋。

南疆蛊仙顾此失彼。

而流言笼中,池曲由已然陷入生死边缘。

这是信道战场,自然有周雄信主攻,而张飞熊、朱雀儿亦都有变化道造诣,变成信道太古荒兽辅助猛攻。

忽然,吴帅也加入战场。

人道杀招——无双!

这个杀招曾经施加给龙公,帮助很大。这一次却是施加给了周雄信,周雄信本身就被人道杀招增幅,这一次又再次全方位提升,一招一式威能暴涨。

池曲由怒目圆瞪,终究抵挡不住,被信道虎群吞噬。

流言笼战场消解,周雄信等人再次加入战场。而武庸等南疆蛊仙却只能看到池曲由的一些残肢碎渣。

一代阵道大宗师,池家太上大家老阵亡!

南疆蛊仙身心剧震,如此惨痛的牺牲就发生在了众人眼前。

“池曲由大长老!”

“他竟战死了。”

“天庭蛊仙们杀了他……”

南疆一方士气迅速下滑。

“不妙!”武庸脸色难看,池曲由威望很高,这一去远比死了翼浩方、巴德更加严重,对南疆蛊仙们的士气打击很大。

然而武庸却不能做什么。

他虽然有无限风杀招,但天庭从不给机会。还有送友风这等奇异邪招,然而天庭的蛊仙们几乎都沉眠在仙墓中,武庸根本找不到机会和他们“交朋友”。

“己方处境本就艰难,池曲由阵亡,更加雪上加霜!”

“怎么办?”

武庸自身已经拼尽全力,但场面已经脱离了他的掌控。

“坚持!”他的眼中闪过坚毅的光。

“绝不能让天庭修复了宿命蛊,这是四域共识。眼下西漠已经进攻帝君城,东海、北原的蛊仙们定然会出动的。”

“还有方源……”

“我要等待援兵!”

“也只能如此了。”

帝君城战场、毛脚山战场被全面压制,处境凶险,天庭战场亦是如此。

龙公虽然不受人道杀招增幅,却仍旧强得可怕,单凭一己之力,压着长生天一方所有人马。

更糟糕的是,他有了援军。

“秦鼎菱!”劫运坛中,七次郎望着秦鼎菱咬牙切齿。

秦鼎菱受到人道杀招的帮助,已经恢复过来,立即加入战场,为龙公辅助。

她有极其强大的运道造诣,非常针对劫运坛,令七次郎等人越来越被动。

轰隆一声巨响,龙公拳下再倒一位北原强者。

他望着光阴长河的虚影,哈哈大笑:“再多的人马又有何用?”

“长生天!”

“这一次你们必定有来无回,惨遭败亡!”

“哪怕你们有再多的援兵,也不会改变这样的结局。”

秦鼎菱在一旁高昂着头颅,冷笑附和:“正是如此,我天庭之威,岂容尔等冒犯!”

\end{this_body}

