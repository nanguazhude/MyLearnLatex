\newsection{房家太上客卿长老}    %第七百六十节:房家太上客卿长老

\begin{this_body}



%1
房化生的“为难”和“诉苦”,其实是对方源的刁难。

%2
方源面不改色,心湖平静无波。

%3
因为他也早就听闻了这种流言蜚语。

%4
他既然要和天庭作对,眼界就得高超。他一直尝试着凌驾于五域之上,以整个蛊世界为棋盘,来和天庭对弈。

%5
说“算不尽和方源是同一人”的流言,并非是天庭传出来的。

%6
紫薇仙子没有证据,胡乱猜测,只会堕落天庭威信!他们顶多是怀疑,暗中推波助澜。

%7
真正传出这等流言的势力,是西漠的其他正道。

%8
房家在豆神宫大战中,大获全胜。击败了陈衣、青仇,缴获豆神宫,并且还招揽收纳了两位七转魔道强者:败军老鬼、鹰姬。

%9
房家收获巨大,若是将战果消化,实力必定暴涨,一跃而上,坐上西漠正道的头把交椅。

%10
这就打破了旧有的平衡,局面动荡之下,必然危及到许多正道势力的利益,因此他们才一拥而上,企图对房家进行打压。

%11
为了师出有名,占据大义,他们便四处寻找理由。

%12
什么陈芝麻烂谷子的事情,都被掀起来说。

%13
将曾经的历史矛盾抖抖灰尘,都拿出来用。

%14
实在找不到什么理由,怎么办?

%15
那就制造理由。

%16
于是各种小道消息、流言蜚语,一时间喧嚣尘上,说得煞有介事,但稍稍推敲一下,就都站不住脚。

%17
算不尽就是方源的谣言,就是其中之一。根本没有任何强有力的证据,但谣言需要证据吗?

%18
这些都不过是正道的游戏规则,是他们行动起来,一起打压房家的遮羞布。

%19
我管你什么证据不证据的,我有这层遮羞布就行了!

%20
方源甚至怀疑,有关自己的这个流言,恐怕是某位蛊仙忽然灵机一动想出来的。

%21
而制造谣言的蛊仙,恐怕也万万没有想到:他(她)真的是误打误撞说出了真相!

%22
但方源估计:房家不会相信这种谣言的。

%23
算不尽这个身份,虽然出现得比较突兀,但实际上却是有跟脚可寻。

%24
这点,方源早就做了充分准备,他把算不尽的身份和西漠散仙郑惊神挂上了勾。

%25
房家那边还是比较相信方源给出的依据。

%26
因为郑惊神是数万年前的人物,并且他和房家的渊源,是一个秘密,其他势力并没有多少了解。

%27
房家绝对不会承认,算不尽就是方源。除非给出直接证据,让房家无法反驳。

%28
上一世就是这样。

%29
房家因为方源的身份暴露,的确陷入被动,被西漠正道势力惊喜过望地找到了这面名义大旗!

%30
方源清楚:房化生之所以这么说,都是在为房家争取利益。

%31
方源心头冷笑,淡淡地瞥了房化生一眼:“既然如此,那就算了。我自寻其他家族合作罢。”

%32
说完,他便转身就走,仍旧是来去匆匆。

%33
房化生见方源瞥他的眼神,当即心头就咯噔一下,现在见方源掉头就走,心中大叫糟糕,连忙追上去:“算不尽仙友,且慢且慢!”

%34
“怎么了,还有什么事情?”方源放缓速度,悬浮于空中。

%35
他用戏谑的目光看向房化生,语气淡淡:“房家不是处境困难,不想合作吗?”

%36
房化生满脸苦涩的笑。

%37
他已是明白,方源是看穿了他和房家,故意这样做来拿捏自己。

%38
但房化生无可奈何,因为的确是被拿捏住了。

%39
关于这点,房化生也有心理准备。因为来之前,房家的智道大宗师房睇长就关照过他。

%40
算不尽的身份,毕竟是智道蛊仙!

%41
房化生向方源抱拳一礼:“算不尽仙友,还请勿怪。房家虽处境堪忧,但和仙友合作的决心从未动摇过!房家和仙友之间的渊源,要追溯到数万年前。这份渊源真的是很奇妙,值得我们彼此珍惜。”

%42
“上一次豆神宫大战,我族和仙友更是合作愉悦。若非仙友相助,我族想拿下豆神宫还要费一番周折呢。我房家还欠着仙友此战的报酬,这一点虽然此前争取到了仙友的谅解,可以拖延交付,但我们房家上下都从未忘记。来之前,太上大长老还嘱托在下,关于这一点要和算不尽仙友好好道歉呢。”

%43
房化生说了一大通的漂亮话,惠而不费。

%44
方源心中不屑,但脸上却浮现出微笑,一方面是配合房化生,另一面也是给自己找个台阶。

%45
方源道:“房家的处境,我也了解一二。你也知道我是智道的蛊仙,开拓青鬼沙漠的心情也非常迫切。我已是拟定了这场合作的细则,你看看。”

%46
房化生接过信道凡蛊,神念探入一览,面色就微微发白。

%47
方源提出的细则,明显是偏向于他的,对房家的要求并不过分但也不轻松,正好卡在房家的底线上。

%48
“仙友真的不愧是智道的蛊仙!”房化生看完,心头发凉,出声感慨。

%49
房家虽然得到豆神宫,但炼化此屋十分困难,进展很是微小。

%50
豆神宫争夺战后,房家的三座仙蛊屋损毁程度很高,房家的实力下降得十分厉害,陷入低谷。

%51
偏偏这个时候,西漠各大正道势力一齐出手,积极打压房家。

%52
房家处境堪忧,迫切需要外在的帮手。

%53
算不尽就是一个相当理想的援助!

%54
因为在房家诸仙看来:算不尽不只是智道蛊仙,他有奴役魂兽大军的手段,在豆神宫一战中,他还暴露出了盗取八转仙蛊魂兽令的能力,简直是惊世骇俗!

%55
房家对算不尽有许多忌惮,但好在算不尽和房家颇有渊源,并且还曾经亲密合作过。在这个节骨眼上,房家若有这样一位七转强者加入,必定能振奋士气,对于眼前局面大有帮助。

%56
而若是拒绝算不尽的合作要求,说不定就会将此人推向对立的一边。

%57
若是算不尽和其他正道势力合作,那么房家的处境必然会是雪上加霜。

%58
房家忌惮算不尽,同样也需要算不尽。

%59
算不尽虽只是七转蛊仙,但房家此刻处境下要拒绝他的合作要求,是拒绝不起来的。

%60
方源正是看准了这一点,对付房化生才有恃无恐。

%61
“我之前和池曲由交涉,可能会失败。但此次和房家交涉,我早已利于不败之地。”方源心中底气十足。

%62
“当然,算不尽的身份若是真的曝光,那么房家打死也都不会和我合作。所以上一世,我干脆就直接威胁,索性放弃了合作的打算。”

%63
这就是正道的游戏规则。

%64
至于池曲由为何冒大不韪,和方源交易?

%65
那纯粹是因为利益太大!

%66
另外池曲由也有自信:就算此事曝光,自己矢口否认下,池家有大阵守护,固若金汤,顶多是吐出一些资源利益而已。

%67
池家的处境和地位,又和房家不同。

%68
两族的具体情况不同,便有不同的策略。

%69
房化生看清楚了内容条款,便将方源的信道凡蛊递还给他。

%70
不过与此同时,他又递给方源另外一只信道凡蛊:“这是我此行前来,太上二长老大人亲自交托给在下的。若按照仙友你的方案,我们房家当然能合作。但这只信道凡蛊中的内容,是房家诚心实意的另外一种方案,还请仙友你好好考虑一下。”

%71
方源心中笑了一下,不用看他都知道这里面的内容大致是些什么。

%72
接过一看后,方源嘴角微不可察的撇了撇。

%73
果不其然,房家开出重磅条件,邀请方源担任房家的太上客卿家老!

%74
方源心头暗笑一声,脸上却露出犹豫之色,他对房化生郑重地道:“此事……我还需要考虑几天。”

%75
房化生大喜,他特别担心的是方源直接当场拒绝,现在方源说要好好考虑,那就证明方源心动了。

%76
不过心动也不奇怪,当初房化生第一眼看到房家招揽方源的条件时,也感到非常的震撼,万万料不到家族方面能付出如此巨大代价!

%77
房化生连忙道:“算不尽仙友还请好好考虑,房家的基业和诚意,你都是知道的。目前处境虽然困难,但前途却是一片光明的。实不相瞒,我族对于炼化豆神宫已经有了不少进展呢。”

%78
方源哦了一声,脸上的意动之色更加明显。

%79
但暗中却是大翻白眼。

%80
房家对于豆神宫,肯定进展微渺。

%81
为什么?

%82
因为在最后的中洲大战中,房家都没有拿出豆神宫来作战。

%83
房化生这是纯粹睁眼说瞎话!

%84
但这一点,不是方源关心的重点。

%85
他已经决定答应房家的要求,担任房家的太上客卿长老。只要给他充分的时间,依凭他自身的信道造诣,盟约什么的,根本束缚不住他的。

%86
有了这一层身份,青鬼沙漠的开拓大计必然会更加顺利。同时还有一大批的修行资源,会交付给方源,作为他加入房家的报酬。

%87
这些资源令方源也颇为心动。

%88
只是方源不能当场就答应,当场答应下来,就显得太过急躁了。

%89
这和算不尽的身份不符。

%90
房家的房睇长可是智道大宗师,风姿卓绝,手段高超,方源是亲眼见识过的,绝不容小觑。

%91
话尽于此,方源辞别了房化生。

%92
“仙友慢走。”房化生停留在远处,眼巴巴地望着方源离开,目送着他的神情亲切真挚。

%93
方源感到有些好笑:“过几天我就答应他们。只是将来若有一天,我的身份曝光了,不知道房家上下会是什么样的表情?”

%94
ps:多谢大家的祝福,好高兴哦!

\end{this_body}

