\newsection{圣女考验}    %第六百四十七节:圣女考验

\begin{this_body}

海神祭,圣城中人潮如海。

周围的观众议论纷纷,声音嘈杂一片。

方源在人群中,仔细观察着这些候补圣女,目光幽幽。

在他的计划中,他将选择其中一位,将她扶持上去,成为此代的圣女。

……

“扶持圣女上位?”

“不错,这一次寒潮部族族长耗费大力气,要扶持秋霜姑娘。”

谢晗沫的蓝鳞、赤鳞两位侍卫,听到确凿的情报后,相互对视一眼,均看出彼此的忧心忡忡。

厅堂上,方源坐在一旁,暗暗咬牙,面色也并不好看。

之前谢晗沫力抗监察使,保下方源,令流言发酵得越加厉害,最终导致了鲛人圣庭重新召开海神祭,选拔圣女。

“这是一个计谋。对方就是看准圣女大人心慈仁厚,不会放弃我,从而营造出了这个局面。”方源仰天长叹,感觉自己拖累了谢晗沫。

谢晗沫轻扫了他一眼,微笑道:“方源,你不必自责。若是我们当初放弃你,那就会有说我冷酷无情的流言蜚语了,照样是打击我的威望。”

“圣女大人,到了这个时候,你就不必宽慰我了。”方源苦笑,“这两种性质的流言,严重程度根本不同,我这诸位相处时间也不短了,岂会不清楚此中门道?”

谢晗沫笑了笑:“好了,咱们不说这些,说这些对局面无益。”

方源、蓝鳞、赤鳞侍卫纷纷脸色一变,积极调整心态。

谢晗沫继续道:“圣女这个位置真的不好坐,我刚开始查贪腐,就有人想要将我从位置上扯下来。这正说明了圣庭中族老贪腐情势之严重!诸位不必太过悲观,虽然是重新召开海神祭,但我也并未失去资格,不是吗?想要扶持出一位圣女,可不是那么容易的事情呢。”

有人的地方,就有江湖。

这句话对于鲛人而言,也特别适用。

鲛人之间也存在勾心斗角,也存在利益的纠纷。

圣女位高权重,鲛人内部势力若非能扶持出一位圣女,在接下来一段时间内,必定会受到政策上的倾斜和照顾,获得更大的利益。

一般而言,能够竞争圣女的鲛人少女,往往都是有着深厚的背景,或者在其背后站着一两个庞大的势力。

……

海底圣城,悠扬的歌声渐渐停歇,激烈的鼓点一阵阵敲响。

广场上,只剩下最后的数位鲛人少女,她们不知疲倦地舞蹈,各自显露出了竞争圣女的雄心。

方源站在广场之外,放眼望去,这七位鲛人少女各个都是美人,有的妩媚,有的青春,有的可人,有的端庄。

广场中央是一口巨大的元泉,号称海神泉,泉水喷涌,托起一位年迈的鲛人老嬷嬷,她便是这只鲛人族群中的大族老。

大族老扫视七位鲛人少女,微微点头,沉声道:“圣女第一关,黑油捞金针。”

话音刚落,立即就有十几个力道蛊师,抬着巨大的水缸,步入广场。

咚咚咚……

一连串的沉重闷响声中,七座水缸摆在了七位鲛人少女的面前。巨大的水缸比她们还要高出数倍,鲛人少女们唯有漂浮上去,才能看见水缸里面。

水缸中是盛满了的浓郁深邃的黑油,第一关的考验内容,就是要在规定的时间内,从黑油中捞出一枚牛毛般大小的金针。

“我要求我的跟随者上场。”白鳞鲛人苏怡望着眼前的巨大水缸,神态从容地道。

“请求批准。”大族老深深地望了一眼苏怡。

“不要紧张,不要紧张。”夏琳在心底不断地为自己打气,局促不安地走上广场。

“怎么回事?堂堂苏怡大小姐,居然请了一位二转蛊师,当做跟随者?”

“稍安勿躁,苏怡能这样选择,可见这位小鲛女定然有过人之处啊。”

周围人不断议论。

“是她?”方源看到夏琳,不禁眉头微微一扬,没想到此女成为了最热门的圣女候补的跟随者。

采油蛊!

夏琳漂浮到水缸前,催动这只五转蛊虫。

瞬间,水缸内的黑油就有了异动,被夏琳抽取出来,汇入到她高举的手掌掌心之中,不见踪影。

人群轰动。

“这是五转蛊的气息!”

“明明只是二转蛊师,居然能催动五转的蛊虫?我没有看错吧?”

“我明白了,这应当就是最近一直在疯传的五转采油蛊了。”

“原来这位小鲛女掌握这样的极品蛊虫,难怪她会被苏怡招揽,成为她的跟随者呢!”

采油蛊效果绝不是盖的,片刻之后,水缸一空,缸底留着一枚金灿灿的牛毛细针。

全场轰动。

无数道炙热的目光,集中在夏琳的身上。

夏琳更加紧张,满脸红晕,一副手足无措的可爱样子。

苏怡望着她,嘴角含笑,心中暗道:“将她招揽过来,果然是没错。”

第一场考验,苏怡依靠夏琳的惊艳表现而风头无两,

方源眼眸如深潭,随着散场的人流渐渐离开广场:“首场考验,七位候补晋升了足足六位,看她们的表现都是有备而来。显然这场考验是提前泄了题的。”

……

惨绿色的火焰,在谢晗沫的眼前熊熊燃烧,形成一道火路。

同样是白鳞鲛女的秋霜,已经从容地走过去,站在火路的彼端,戏谑地看着谢晗沫:“前任圣女大人,看您的了。”

“可恶!这幽火专门灼烧魂魄,需要用特定的蛊虫才能抗衡。我们准备的已经非常充分,但没想到这第一场考验居然这么偏门!”

“更可恶的是,秋霜居然真的备有蛊虫,能抗衡幽火。这是运气?哼!恐怕是早就提前知晓了。这是寒潮部族的阴谋,这是赤・裸・裸的徇私舞弊!”

蓝鳞、赤鳞两位侍卫义愤填膺。

“让我来。”方源走到皱着眉头的谢晗沫身前。

“你?”谢晗沫清澈如水的目光,打量方源。

“相信我一次,我有底气。”方源眼中精芒闪烁,直接望着谢晗沫。

两人对视一会,终于谢晗沫转移了目光。

“万万小心,这幽火可不简单……若是支撑不住,就在中途退下了罢。”谢晗沫道。

方源哈哈一笑,猛地转身,迈开大步,走入幽火当中。

痛!

极致的惨痛,从魂魄的深处迸发,一瞬间就袭遍方源全身。

方源浑身剧颤,一步步艰难跋涉。

他将牙齿咬紧,咬出血。

他瞠目怒视,眼眶都瞪裂开来。

他的魂魄在火焰中被炙烤,消融,好在他有着两世积累,又是天外之魔的身份,幽火的效果对他而言,要稍弱于常人。

他绝不会中途放弃,因为他知道,但他替代谢晗沫出手,就意味着一旦他失败,谢晗沫就失败。

当他终于迈出幽火的路,全场轰动,无数道震惊地目光集中在他的身上。

无数鲛人动容。

方源拼尽全力想要努力地笑一下,但下一刻,他就彻底昏死过去。

但就在他要栽倒到地上的时候,谢晗沫及时赶来,一把将他抱在怀中。

“你放心,我一定不会辜负你的这番付出。”谢晗沫深深地望着方源,又抬头看向秋霜。她温柔如水的双眸中,头一次出现了冷冽的光。

得益于方源的拼死奋战,谢晗沫跨越了一道精心设计的陷阱难关。

第二场考验、第三场考验、第四场考验……

她一路高歌猛进,很快就使得其他竞争者黯然失色,唯有鲛女秋霜才有一搏之力。

“如此看来,谢晗沫大人保住圣女的位置,应该是大有希望的。”场外,方源脸色仍旧苍白,虚弱地坐着,面色喜悦。

“这还是多亏了你啊,方源小子,没有你在第一场考验中的表现,我们不会走到现在。”

“哈哈哈,你小子成功走过火路,所有人都几乎看傻了。几天后,你又苏醒过来,你不知道你仍旧活着的消息,惊呆了圣城中多少的鲛人!”

蓝鳞、赤鳞两位侍卫大笑。

方源却收敛了喜色:“要小心,如今的局面对我们十分有利,但对方绝不会善罢甘休的。”

方源料到寒潮族长会出手,但没想到会以这种阴险的方式。

屋中,赤鳞侍卫跪在地上,满脸通红:“圣女大人,请你批准我出战,洗刷他们对我的诬蔑和冤屈。我怎么可能会欺负一位寡妇?!”

谢晗沫叹息:“你起来,我当然知晓你的为人,但此时情况明显是对方的诡计。你若这么冲动地冲出去,必然会令设计的人阴谋得逞。”

蓝鳞侍卫满脸愁容:“这位寡妇可不是寻常的寡妇,乃是圣庭前任三族老的遗孀步素莲。对方此计太过狠辣,恐怕监察使不久就会到来,扣押赤鳞加以审讯。如此一来,赤鳞无法参加接下来的考验,我们实力将大为受损!”

方源接着道:“这位前任三族老的遗孀步素莲,居然愿意舍弃名誉,来坑害赤鳞侍卫。她和寒潮族长这一系人必然是牵扯极深,贪腐内情必然是十分严重的,否则绝不会如此赤膊上阵。对方既然能够设下此计,定然是准备充分,我们一味要澄清事实真相,就会落入对方的节奏当中。唯有将错就错,方有一线转机。”

“如何将错就错?”

“很简单,就让我来顶罪。”方源淡淡地道。

蓝鳞、赤鳞瞪大双眼,呆呆地望着方源。

“不可。”谢晗沫断然否决。

------------

\end{this_body}

