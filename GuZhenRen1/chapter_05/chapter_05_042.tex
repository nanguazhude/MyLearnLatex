\newsection{变化道!}    %第四十二节:变化道!

\begin{this_body}

%1
望着方源的身影,已远在天边,缩成一个小点,楚度却仍旧目送着他,停留在原地。

%2
“就这样放走了他,我此举是对还是错?”楚度眼神闪烁。

%3
但很快,他苦笑一声,暗道:“我就算不想放走他,又能如何?”

%4
“我的手段,根本抓不住他,也追不上他。”

%5
“若是来硬的,此人跟脚神秘,并不好惹不说,真落到我手中,自爆了魂魄,我又从哪里得到那个法门?”

%6
“不过,看他态度,应当是真心和我交易。否则,他早就可以扬长而去,为何还要停留下来,和我商谈呢?”

%7
楚度仔细回忆一下他和方源刚刚的谈话。

%8
他品味良久,觉得方源还是诚意很足的,虽然方源没讲到什么重点,说出来的话似乎都是敷衍之词。

%9
但偏偏就是这样,楚度才觉得:这远比口中花花,肆意担保什么,要来得真诚得多。

%10
当然,给他最大信心的,不是方源的态度,而是他手中扣着的飞剑仙蛊。

%11
“这可是七转仙蛊!”

%12
就算是楚度如此传奇的人物,他手中的七转仙蛊也不是很多。

%13
方源一路飞行,并没有遇到什么意外。

%14
琅琊地灵也早已安排好毛民蛊仙策应,布置好了仙道战场杀招,还有用于挪移传送的仙道蛊阵。

%15
不过,摆脱楚度十分顺利。有点出乎方源的意料。所以,用于拦截楚度的仙道战场杀招,已经没有必要了。

%16
将战场杀招收拾起来,方源通过仙道蛊阵,直接回到了琅琊福地。

%17
“你明明有狗屎运仙蛊护身。居然还碰到了楚度,这不符合常理。会不会遭了别人的算计?”见到方源,琅琊地灵第一句话就很直接了当。

%18
方源皱着眉头,其实这个问题,他也早在思索了。

%19
“狗屎运仙蛊庇护,自身运道应当不差。难道说是因为我地灾太过猛烈,增添的狗屎运都用于削减地灾了?”方源心中萦绕着这个猜测。但他不好对琅琊地灵明说。

%20
他毕竟是人族蛊仙身份。

%21
虽然加入了琅琊派。但两个种族之间,还存在着鸿沟。

%22
就像之前,方源从南疆一路赶回北原,琅琊地灵就趁机在他身上捞取好处。

%23
“如果是这样,我面对的地灾其实已经被运气削减了许多。那么,真正的地灾,该有怎样的威能?”

%24
想到这里。方源心中顿时就升腾起一股寒意。

%25
他这次渡劫,简直是险死还生,生死悬于一线,惊险无比。

%26
可以说,这已经完全超脱了地灾应该有的程度。就算是十大凶灾,也远远及不上方源的这一次灾劫。

%27
“别忘了,我连运多人。狗屎运或许平均分配了许多出去,才效果不佳。”方源心中暗忖,嘴上则道,“太上大长老言之有理。这不是运气差的问题了,说不定真的有人算计我!”

%28
“方源”本身的身份,已经在义天山大战中,彻底暴露了。

%29
不管是天外之魔,还是春秋蝉,或是捣毁八十八角真阳楼的秘密,都已经为人所知。

%30
不管是天庭。还是影宗,或者北原的各大黄金家族,都极有可能算计方源。

%31
琅琊地灵叹息一声,眉宇间笼罩一层忧色,语气也沉重下来:“或许,也未必是算计你。别忘了,你已经是我琅琊派的一员。而我琅琊福地已经暴露,遭受突袭不止一次。前段时间,连折派中两位蛊仙。这里面的价值,足以让无数强者、势力动心了。”

%32
方源心中微喜,琅琊地灵能这样想,自然更符合他的利益。

%33
琅琊地灵目光炯炯,很快话锋一转道:“但如果我们能以智慧蛊为核心,建造出仙蛊屋隐士居,不仅能护持自身,而且还能防备几乎天下所有人的算计!”

%34
方源哑然一笑。

%35
原来,琅琊地灵是想说这个事情。看来他对智慧蛊仍旧恋恋不舍。可惜之前的交易,方源并没有将智慧蛊卖给他。

%36
“这一次渡劫,多亏有太上大长老你的蛊虫。现在都还给你罢,这些蛊虫可帮了我大忙。”方源说着,一大股蛊群便随着他的心念,从仙窍中如数飞出来。

%37
“那肯定啊!有它们帮助,区区地灾,想必是手到擒来。”琅琊地灵带着骄傲的神色道。

%38
方源没有吐露实情,但也不好撒谎,只好选择性地道:“尤其是仙灾锻窍杀招,竟然真的有效果。不过最后楚度出现,有些惊悚。好在我的移遁之能,刚巧胜他一筹。”

%39
谈话间,琅琊地灵一一将蛊虫收起,检查无误后,又略带遗憾,不甘地问道:“你的荡魂山、智慧蛊,真不打算卖给我吗?”

%40
原来当初交易,琅琊地灵要拿三个真传,换方源的荡魂山、落魄谷和智慧蛊。

%41
但方源深思熟虑之后,觉得炼道、运道、偷道虽好,而自身在这三方面的境界,却差强人意,并不和自身今后的道路相符。若是全部换了,并不划算。

%42
然而,炼道当中的仙灾锻窍杀招,运道真传里的狗屎运仙蛊等等,亦都能够方源渡劫提供巨大的帮助。

%43
所以,方源左思右想,斟酌好说词后,便劝说琅琊地灵:“你这三道真传,真的要全部换给我吗?真传之中,各有不少仙蛊,我若得了,将来琅琊派就不需要用了吗?”

%44
琅琊地灵其实也有不舍。

%45
就比如狗屎运仙蛊,早在很多年前,巨阳仙尊陨落之后,长毛老祖就第一时间将其炼制出来。

%46
而后,狗屎运仙蛊一直坐镇琅琊福地,给上一任琅琊地灵炼蛊带来巨大帮助。同时,也护持琅琊福地,多次趋吉避凶,有惊无险。

%47
但是,琅琊地灵若不拿出这些仙蛊,那么这些真传的分量就太轻了,如何能换得方源的三样东西呢?

%48
于是,方源又劝道:“太上大长老,既然你我都有些不舍,不如我们各取所需。咱们交易一部分,我再借用一部分。”

%49
“哦?具体说一说。”琅琊地灵。

%50
方源便说出自己的意图,结果遭到琅琊地灵的拒绝:“不行,不行。你这换了三道真传中的精髓,却只付出一座落魄谷。我亏大了,这可不行!绝对不行!”

%51
方源笑笑,继续和琅琊地灵谈判。

%52
琅琊地灵并不是谈判的料,心中有个底线。方源三下五除二,就轻易试探出来。

%53
这种情境下的地灵,单纯固执,远比蛊仙容易对付得多。

%54
最后,方源和琅琊地灵达成交易。

%55
方源将落魄谷正式卖给琅琊地灵,而得到狗屎运仙蛊、血本仙蛊、暗渡仙蛊这三只六转仙蛊,对琅琊派的所有欠款清零,同时借用大量凡蛊、仙蛊,弥补自身实力短板之外,重点就是用来组成仙灾锻窍杀招,和剑浪三叠。

%56
事实证明,仙灾锻窍和剑浪三叠,的确在渡劫的过程中,带给方源巨大的帮助。

%57
这笔交易虽然早已达成,但现在琅琊地灵对荡魂山、智慧蛊,仍旧不想放弃。

%58
这一次,地灵对方源又流露出了收购的意图。

%59
但方源哪会同意?

%60
不过他也不能太无情地拒绝琅琊地灵,当即心思一转,婉转地道:“这两样,我暂时还不打算出手。不过若接下来修行上有什么困难,情势使然,我自会用这两样,和太上大长老你进行交易的。”

%61
琅琊地灵张口欲言。

%62
方源摆手,先一步道:“这两样事物,我都寄存在琅琊福地之中,并不取走,相当于借给门派使用,一如之前的约定。太上大长老敬请宽心。”

%63
将智慧蛊借给琅琊派,琅琊地灵就要负责喂养,等于解决了方源的一个大难题。

%64
而荡魂山,方源还是不放在自家仙窍中的好。

%65
历来名山大川,对于天地之气的负担就极重。而这等天地秘境,更是如此。

%66
更关键的是,影宗方面对荡魂山、落魄谷知之甚详,万一用定仙游传送过来,到了方源的仙窍之中,那就糟糕透顶了。

%67
琅琊地灵不甘,还想纠缠这个话题,但方源何等人物,言语交锋的能力淬炼了五百多年。单纯的琅琊地灵,如何是他的对手?

%68
一番交流完毕,方源又从琅琊地灵手中,取回他的诸多修行资源。

%69
譬如幽火龙蟒、长恨蛛、龙鱼、箭竹林等等,皆是来源于狐仙、星象两大福地,仅次于胆识蛊的经济支柱。

%70
至始至终,方源都为对琅琊地灵提及,任何出售仙灾锻窍杀招的意向。

%71
楚度?

%72
不过是方源留的一条后路而已。

%73
虽然双方交换了信道蛊虫,可以相互联络。但方源都已经脱身,又没有什么信道约束,今后交易与否,还不是看方源的意愿?

%74
至于……七转的飞剑仙蛊。

%75
呵呵,区区一只七转仙蛊而已,取不回来又如何?

%76
寻常蛊仙对七转仙蛊珍惜若宝,但方源连智慧蛊这等九转仙蛊都拥有着,自然也就看得淡了。

%77
尤其是他前段时间,几乎一身的仙蛊都丢了,落到了影无邪手中。

%78
一只七转飞剑仙蛊,和那堆仙蛊相比,真是小巫见大巫。

%79
楚度不知道方源的跟脚,着实低估了方源的器量。

%80
总体而言,方源和楚度交易的可能性很小。

%81
狂蛮真意对于方源而言,远比一只七转仙蛊,要更重要得多!

%82
“因为我接下来,要走的道路,就是变化道啊。”

%83
方源这些天已经深思熟虑,想得十分清楚透彻。

%84
变化道!

%85
这是目前最适合他的道路。

\end{this_body}

