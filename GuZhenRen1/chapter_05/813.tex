\newsection{天庭察觉}    %第八百一十六节:天庭察觉

\begin{this_body}

天庭。

秦鼎菱悬浮长空,面容严肃凝重至极。

这一记杀招,她已然是酝酿了三天三夜。

“着!”下一刻,她终于爆发,催出一道恢弘的金芒。金芒如五尾孔雀,轻啼一声,展翅高飞。

待这金芒飞到某处天空,忽然炸裂开来,散成无数的点点碎芒。

碎芒当中,一个巨大的通道隐隐显现。

秦鼎菱身旁的紫薇仙子顿时失色:“竟然真有这样的通道!”

秦鼎菱七窍缓缓流血,身躯摇摇晃晃,脸色苍白,退下休整。

“辛苦你了。”龙公也在现场,此刻也是动容。

秦鼎菱缓过几口气,感叹道:“好险!若非我前段时间,搜魂赵怜云,得知一些巨阳众生运真传中的只鳞片爪,单凭我本身造诣,还不能发现巨阳仙尊的此处布置。”

龙公冷哼一声:“这巨阳仙尊图谋不轨,当年假意来天庭浏览,实则布置了这个暗手。这运道真传的确玄妙,偌大的天庭居然从未发现。不过天意仍旧眷属我天庭,在关键时刻,有秦仙子苏醒,识破了巨阳仙尊的诡计。”

紫薇仙子深呼吸一口气:“按照这样的布置,巨阳仙尊便将天庭打通。一旦发动起来,很可能就有长生天的蛊仙直接杀奔天庭了。”

“是的,这的确是北原蛮子的作风。”龙公道。

紫薇仙子眼眸中精芒烁烁:“不过,既然发现了这一点,我们就可以针对布置。甚至将计就计,暗算埋伏北原一手。这对于几年后的中洲炼蛊大会,有着巨大的帮助。”

秦鼎菱立下大功,面色却仍旧黯然,她叹息道:“巨阳仙尊有己运、众生运、天地运三道真传。我为了对付他,专修其中的众生运,又苦修这么多年,原本以为和他相差仿佛。结果这一次才让我明白,原来我和他的差距仍旧是那么大。”

紫薇仙子便劝慰道:“历代的尊者,都不可以常理揣度。前辈你有这样的运道造诣,已经是天庭之幸,弥补了天庭的历史缺憾。从此之后,天庭也有拿得出手的运道真传了。”

秦鼎菱道:“为了以防不测,我已经将我的运道真传,送入仙库之中。不过最好还是得到巨阳仙尊的真传。可惜的是赵怜云当初在王庭福地,虽然是亲眼目睹了众生运真传,但眼界有限,只得只鳞片爪。而那马鸿运却是鸿运附身,定然知晓更多,可惜却是被方源擒获了去。”

提到方源,紫薇仙子的眉头顿时皱起:“说起来,方源已经有一段时间没有露面了。我方已经在光阴长河中做出了严密部署,但方源却始终没有现身。他明明有着万年斗飞车,却似乎对红莲真传没有太多热情。这到底是我们侦察不出他的踪迹,还是他……”

提到红莲真传,龙公的眉头也微皱,沉声道:“那就做最坏的打算,认作方源已经得到红莲真传罢。”

“要制服方源,依我之见,重点仍旧是在古月方正身上。”秦鼎菱笑了笑道。

和上一世不同,古月方正得到了最大的重视,早已入了龙公、紫薇仙子的眼里。

得到天庭的安排,大量的资源倾斜,古月方正的实力也在迅速进步。

仙道杀招——血渐冷!

三头荒兽的动作越来越缓,最终都冻僵,瘫倒在地上,哼哼唧唧。

胜利者古月方正悬浮于空,俯瞰着脚下躺着的三只荒兽,一身灰袍,毫发不损。

“血道杀招果然厉害。”赵怜云就站在他的身旁,与他并肩而立。

方正拱手道:“还是多亏了灵缘仙子相助,否则我哪有这么多的实战机会呢。”

赵怜云便笑:“你我有共同的目标,自当同心协力,不是吗?”

和上一世一样,这一世赵怜云仍旧主动过来,结交方正。

赵怜云爱着马鸿运,然而马鸿运却是被方源夹死。后者虽然死了,但魂魄还在,仍旧有复活的可能。

“天庭崇尚天意,其实并不想鸿运复活。他若死而复生,不就是违背了宿命么!”

“我虽然成为了灵缘斋的当代仙子,但天庭仍旧只当我是一个棋子,不断利用。”

“之前,那女仙秦鼎菱肆意对我搜魂,就是最好的证明。在这点上,我和方正的处境是相同的。同样是棋子,只不过天庭更看重方正一些罢了。”

“所以,我要复活鸿运,只有依靠自己,寻找机会!”

赵怜云心中自有算盘。

赵怜云知道自己的处境,方正也不再单纯,经历很多,同样明白自己的情况,更理解赵怜云为什么如此结交自己。

方正并不反感赵怜云,双方处境相同,他心中有不少同病相怜的情怀。

两人谈话间,古月方正已是将这三头荒兽收入仙窍当中。

三头荒兽并未死亡,血渐冷杀招掌握分寸后,就成为了一招活捉敌对的上佳手段。

方正在这个杀招上的确下了不少的苦功。

原本,紫薇仙子栽培古月方正的计划,是让他不断闭关潜修。但秦鼎菱苏醒之后,却是劝说紫薇仙子改变了方针。

这一次,古月方正就是领了门派任务,前来镇压三头为乱的荒兽。

眼下任务已成,方正打算整理战场之后,便回归飞鹤山。

不曾想在荒兽的巢穴中,却是发现了一个隐秘通道。

为了防止有荒兽幼崽残留在通道深处,方正和赵怜云合计了一下后,便齐齐飞入通道当中。

结果让两人非常意外,这通道深邃得远超想象,两人不断往下疾飞,通道越加宽阔,竟可盛山装海。

“这里应当是一处地沟,地面上的那三头荒兽,应当和这地沟关系不大。”赵怜云分析道。

方正点头,但仍有疑虑:“我也如此猜测。只是这地沟未免有些古怪,按照常理,应当是生机勃勃,兽植出没,但我们往下飞了这么久,却是丝毫不见任何生灵。这处地沟必有蹊跷之处,不妨探探。”

赵怜云原本想要回返,但听方正执意如此,便跟从他继续下沉。

“咦,这里竟然有一处福地?!”不多久,方正忽然发现了异象。

随即,赵怜云也发现了,惊叹道:“这处福地正在渡劫,导致福地破损,形成漏洞,这才外显出来。我们来的时机太巧了,若是过了这个时机,福地关闭门户,寄托虚空,就算在我们眼皮子底下,我们都发现不得。”

“进去看看。”方正略有兴奋地道。

既有福地,恐怕就有蛊仙传承。毕竟福地乃是蛊仙仙窍所化。

两人轻轻松松进入福地,福地当中正处于一片浩劫,水深火热之间。

“奇怪,这里的人族似乎奇形怪状,难道是灭绝了的兽人吗?”

“就目前形势,应当是没有蛊仙人物的。”

方正和赵怜云商量了一下,便一齐出手相助。

他们俩尽管修为浅薄了一些,但都不是寻常蛊仙,手段非常精妙,立即平息了灾祸。

福地中幸存下来的人们,对两仙当场跪拜,感恩戴德。

方正询问之下,发现这些福地中的居民居然都不是人族,而是龙人。

“龙人,是哪一个异人种族吗?我怎么没有听闻过。”方正诧异。

赵怜云身为灵缘斋当代仙子,到底有一段时日了,她顿时皱起眉头:“龙人的确是异人之一,我们这下有麻烦了。”

她深知自己和方正,皆是天庭棋子,看似自由活动,但身上必定有着天庭的监控手段。

果然,片刻后,就有蛊仙降临。

却是两位女仙。

方正、赵怜云都认识,一位是紫薇仙子,另一位则是秦鼎菱。

方正还感到有些奇怪:“这片福地平淡无奇,并不出众,怎么会劳动天庭这两位大驾?”

秦鼎菱和紫薇仙子却是心情很好,这点从她们的脸色就能轻松分辨得出。

“方正,你又立了一功,做得很好。现在没有你们俩的事情了,都出去吧。”秦鼎菱点头微笑,一句话就打发了方正和赵怜云。

方正、赵怜云不得不听从,乖乖飞走,回归各自山门。

片刻后,紫薇仙子已收集到了情报:“这片福地,源自于龙公嫡孙吴帅之手,难怪能掩藏这么久,并且连龙公大人都能骗得过去。”

“在这片福地中的龙人们,都是吴帅的子孙后裔,有纯正的龙人,也有龙人和人的混血儿。”

“这些龙人并不重要,但福地深处却埋藏着一道线索,直指仙蛊屋龙宫!秦前辈你说的没错,方正果然是我们的福将!”

紫薇仙子语气欢喜。

皆因龙宫的主创人之一,就是吴帅。

天庭一直都在寻找龙宫,但进展一直都很缓慢。眼下得到了这道关键线索,只要破解开来,再结合天庭这些年的努力,应当就能推算出龙宫的具体位置了。

“事不宜迟!”秦鼎菱脸色严肃,“方正乃是天意布置,专门压制方源的存在。他发现了这片福地,引出龙宫,这就意味着,方源很有可能正暗中谋夺龙宫呢。”

紫薇仙子闻言,更加不敢怠慢,当即回转天庭,全力推算。

------------

\end{this_body}

