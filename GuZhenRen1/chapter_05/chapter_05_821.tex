\newsection{龙庭}    %第八百二十五节:龙庭

\begin{this_body}



%1
吴帅英明果断,富有远见卓识,更性情强硬,没有绝大多数龙人蛊仙身上的软弱。龙人一族在他的发展下,实力节节攀升。

%2
在暗中,他一直在秘密筹措资源,对于建设一座八转仙蛊屋始终念念不忘。

%3
然而,好景不长,龙人一族的壮大受到人族越来越严重的非议,暗中作的手脚越来越多。很多人眼红不已,事实上,就连吴帅自己也承认,龙人一族变得强盛的原因很大一部分是侵占了人族蛊仙的利益。

%4
只是碍于龙公,很多事情人族蛊仙不会公然发难。

%5
龙公大寿,普天同庆。

%6
厅堂上大红的寿字,极其醒目。

%7
龙公端坐主位,身边龙人环绕。

%8
一位位龙人蛊仙陆续走上来,送上各自精心准备的寿礼。

%9
“父亲大人,恭祝您福如山,寿似海。”一位龙人蛊仙送上一捆白蛟草。

%10
此草乃是太古荒植,蕴含龙性,形如白蛟,鳞片、龙角均非常精致。

%11
“父亲,祝您日月昌明、松鹤长春。”又有龙人蛊仙奉上一株软玉樱。

%12
“父亲大人,儿子在这里祝贺您……”吴帅之父也登场。

%13
龙公面容微肃,只是点头,在他身旁自然有侍者将这些贺礼一一收纳。

%14
儿子辈的轮流转换之后,轮到孙子辈。

%15
这个时候,龙公的脸色变得柔和,明显露出了笑意。

%16
“爷爷,孙儿给您拜寿了!祝您福如东海,寿比南山。这是孙儿亲手给您猎取的荒兽,希望爷爷您能喜欢。”一位龙人少年半跪在地上,双手举着一个盘子,盘子上有一头精致玲珑的小海马。

%17
“这是荒兽流欢海马,速度极快,很难捕杀的。”

%18
“七少爷只是六转修为,单纯捕杀就已经很困难了,没想到竟还活捉了它!”

%19
“听闻七少爷为了活捉这头流欢海马,潜伏在海底一个多月没有动弹丝毫啊,这份孝心真的是感人呐。”

%20
周围人议论纷纷,当中也有人族蛊仙参与这场盛大的寿宴。

%21
小七是龙公最疼爱的孙儿,龙公脸上满是慈爱的笑,他亲手接过装载小海马的盘子,道:“好好好,小七啊,你的贺礼啊爷爷收下了。”

%22
然而轮到吴帅贺寿,龙公却是收敛了笑容,只是端详吴帅:“小八,目前来看,孙子辈中就属你成就最大,但你要记住过犹不及的道理。”

%23
“是,爷爷。”吴帅恭谨而退。

%24
寿宴的当晚深夜,龙公又秘密遣人,召唤吴帅前往书房。

%25
“那座南华岛,你给人家还回去吧,小八。”龙公直接说道。

%26
吴帅心头一惊,勉强笑道:“爷爷,你有所不知,这南华岛可是我正儿八经地赢过来的,孙儿可计划着要将此岛当做家园,安置我的族人们。”

%27
龙公脸色微肃,目中显露些许厉芒,刺得吴帅面皮微疼:“吴帅,不要以为你的那些小聪明,谁都看不破!南华岛本是风云府的领地,却被你作局谋算夺来。你想干什么?好好的带领着你的那些族人,待在门派里不好吗?为什么要迁徙族群,千里迢迢地迁往南华岛?你告诉我,你究竟想要做什么?”

%28
吴帅屏住呼吸,沉默了半晌,这才道:“爷爷,孙儿没有什么野心,无非就是想带给族人们更好的生活。”

%29
“更好的生活?是更大的野心吧!”龙公声音底层下来。

%30
吴帅浑身一颤,索性坦陈道:“爷爷,就算是这是野心,又怎么样呢?难道我们就不应该有野心吗?难道我们就不应该争取更美好的生活吗?”

%31
“为什么一定要生活在这里?爷爷你难道不清楚这些人族蛊仙,是如何排挤我们,压榨我们的吗?”

%32
“什么排挤?谈何压榨?这些年来,爷爷我却是屡次听闻你的事迹。知道我为什么偏疼爱小七吗?因为他认为世间的美好,打心底里是善良的,从未有过人、龙之分。而你呢?却屡屡依仗着我的名头,在外恣意妄为,倾吞他人领地,贪污门派中的收益,速速停止你的这些举动!”龙公厉声道。

%33
“爷爷,如果不是这些人族蛊仙这样对待我们,我们又岂会这样做呢?”

%34
“什么人族、龙人,龙人本就是人族,何来族群之分?!”龙公驳斥道。

%35
“没有族群之分,那我额头的龙角是什么,我身后的龙尾是什么?”吴帅激愤起来,接连直呼,“爷爷,你是人族身份转换成的龙人,在你的生涯中大半都是人族,龙人的生活只是你生命历程中的一小部分。”

%36
“但是我们呢?”

%37
“爷爷,你有没有想过!我们生来就是龙人,我们从一出生,就头上长着角儿,背后长着尾巴。”

%38
“我从小就受到父母的教育,他们告诉我,我们源自人族。但是从小开始,周围的同龄人就取笑我的长相,排挤我,拉扯我的龙尾,挑拨我的龙鳞。这些都在告诉我,我和人族是不一样的!”

%39
“长大之后,我屡屡发现,不只是同龄人,就连那些人族师长也是带着异样的看法看待我,只是表面山不显露而已。”

%40
“我若是失败,他们会说这是龙人,败给人族也是正常。我若成功,他们会讲一个龙人居然能够做到这样的地步,是不是耍弄了什么手段?”

%41
“那些人族总以为他们是最高等的,将我们龙人看做下等种族。凭什么?”

%42
“人是万物之灵,但我们龙人也不差。我们龙人一出生就有奴道道痕,单纯的身体素质就远超人族,若非人族蛊师运用蛊虫,他们的拳脚甚至连我们身上的龙鳞都无法打破。更难得的是我们天生寿命悠久,人族却一个个都是短寿鬼。”

%43
“许许多多的蛊仙,为了延寿,都转变成龙人。明明自己抛弃了人族的身份,却仍旧还贬低龙人的存在,这一切不过都是过往的腐朽的观念作祟而已。”

%44
“够了!”龙公猛地一拍桌子,打断吴帅的话,他的脸色铁青。

%45
“还不够。”吴帅声音却是变得平稳,他勇敢地反视龙公,顶着那对饱含压力的龙瞳,“爷爷,是您开创了龙人延寿法门,没有您就没有龙人,您是我们龙人的老祖宗,但为什么您就一直偏袒那些人族呢?我们才是您的家人啊!南华岛我就算是死,也不会归还的。如果爷爷执意要我这么做,除非是让我死了。”

%46
说完这话,吴帅再不看龙公的脸色,直接转身,连迈几步,开门而出。

%47
龙公没有起身阻止,他仍旧坐在椅子上,他的心情非常复杂。

%48
“唉……子孙一个个都长大了,都有自己的想法了。”

%49
“但是他们却从未真正将人族的大局,摆放在心中。”

%50
“是我平时疏于教导?还是龙人族群已经成了一股洪水,而我只是开闸放水的那支手而已呢?”

%51
风云府被吴帅设计,醒悟过来后通过龙公来调解此事,企图收回南华岛,但吴帅却是劝说不动,执意如此。

%52
风云府当然没有放弃,动用各种手段,和吴帅为难。

%53
吴帅兵来将挡水来土掩,见招拆招,艰难维系着局面。龙公并未出面帮助自己的这个孙子,但其余的龙人蛊仙大多都有出手,只是或明或暗。

%54
依靠着这些人的帮助,再加上风云府顾忌龙公,手段并不过激,终究让吴帅成功迁徙了族群,并在南华岛上繁衍生息。

%55
但是南华岛的事情,并未完结。

%56
上百年过去,风云府仍旧念念不忘,时不时出手,尝试让南华岛回归。

%57
吴帅日理万机,手段滴水不漏,让风云府始终无法得逞。

%58
“百年大计,今朝终是得到了正果!”这一天,吴帅望着眼前的仙蛊屋,流下了激动的泪水。

%59
当初他之所以要夺取南华岛,正是需要一处隐秘之地,搭建仙蛊屋。

%60
吴帅之父也在一旁,脸上带着不可思议的神情:“你真的做到了!我的儿子,你比为父更有能力。”

%61
“虽然这些年,各大门派都对我南华岛施行了封锁和打压,但是我却是奇遇连连,屡获资源。同时更有诸多龙人蛊仙帮衬和贡献,终于耗时百年,做出这座七转仙蛊屋。”吴帅微笑着,“再给我几百年时间,说不得就有一座八转仙蛊屋了。”

%62
“此屋还未取名吧?”吴帅之父问道。

%63
“族中的各位蛊仙都有自己的建议,但我作为主创人力排众议,将其命名为——龙庭!”说道这里,吴帅声音拔高,眼中闪烁着亮光。

%64
“龙庭?”吴帅之父闻言色变,“这是否……有点不太妥当?”

%65
“哈哈哈,有何不妥?正是要和天庭呼应!”吴帅大笑道。

%66
“儿啊,你不妨再深思熟虑一些。这在那些人族蛊仙眼中,龙庭这个名字可不是和天庭呼应,而是分庭抗礼啊。”

%67
“呵呵。”吴帅冷笑两声,“他们怎么想,那是他们的事情。”

%68
“话可不能这么说,毕竟人族乃是五域的霸主。”

%69
“父亲。”吴帅抬起手,“你不必多说了,我意已决,就叫——龙庭!”

\end{this_body}

