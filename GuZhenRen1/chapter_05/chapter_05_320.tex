\newsection{死鬼快来铁面神}    %第三百二十节:死鬼快来铁面神

\begin{this_body}

%1
南疆,铁家,万程山。

%2
轰隆隆……

%3
一连串的沉闷声响,笼罩这片不起眼的小山谷。

%4
在山谷的底部,站着一位少女。

%5
她身姿高挑,一对长腿几乎占据全身的一半,颇有矫健的英姿。此刻,她笔挺如剑的双眉下,一双眼睛绽射星芒,仰望着头顶上空。

%6
一颗颗巨大的,宛若房屋般的石头,正顺着山壁,向她砸下。

%7
少女深呼吸一口气,竟然在生死存亡之时,缓缓闭上了双眼。

%8
数百颗的巨石,眼看就要砸在她的身上,就在这一刻,她猛地睁开双眼。

%9
刹那间,金光绽放,璀璨耀眼至极,整个山谷中仿佛埋着了一颗小太阳!

%10
光芒持续了刹那,便骤然消失。

%11
隆隆之音也旋即消失,无数的灰白粉末,洋洋洒洒,飘零而下。

%12
少女沐浴在飞扬的石粉中,坚毅的面庞闪现出一抹喜悦之色。

%13
“很好,铁若男,你练成我教给你的杀招,达成了最后的考验。从今日起,你就随我修行,我将传授你成仙之法。”一个恢弘的声音,笼罩整个小山谷。

%14
少女立即跪倒拜服在地:“谢太上家老大人指点!”

%15
“你嫉恶如仇,天赋又高,正适合我的这一脉的铁面真传。本来你的父亲是我预选的种子之一,没想到他的女儿的天赋才情比他更出众百倍……嗯?”恢弘的声音忽然一顿。

%16
不过很快,铁若男便又听到:“有一个突发之事,你暂且便在这里修行罢。我传授你的杀招,虽然炼成,不过并不熟练。这段时间,你多多练习,等我归来,再做打算。”

%17
“是,若男遵命。”少女立即恭敬地叩首,脸上难掩激动兴奋的神情。

%18
“终于,终于我做到了,父亲大人,我超越了你。您在天之灵就好好看着吧,我将修行成仙,拥有蛊仙的力量。”

%19
“这个世界太过于黑暗和混乱,唯有更强大的实力,才能让为恶者得到严惩,为善者得到奖赏,正义得到匡扶,天下一片清平!”

%20
……

%21
“章丽,你这个谋杀亲夫的贱妇,可怜我兄长全心全意地待你,结果你却如此狠心,竟暗害了他。今日,我定要把你大卸八块,死无全尸!”一位青年蛊仙咆哮着,怒气冲霄,浑身七转气息澎湃汹涌。

%22
方源:“……”

%23
自从他上一次撑过了有关左夜灰的梦境之后,暗道境界暴涨到了宗师级数。但是接下来,却是荒诞梦境。

%24
方源谨慎起见,不想在这种诡异古怪的梦境中探寻。

%25
耐心等待了一段时间,他好不容易等到了一片写实的梦境。

%26
结果在这片梦境中,他成为了一位女蛊仙,而且还非常的……浪。

%27
因为这个女仙,修行的是魅情道。

%28
这个流派是一个小派,是从智道中拓展,衍生出来的小分支。讲究的是以情对战,以魅惑人。

%29
此刻的方源,化身成一位女仙,一身粉红长裙,薄如蝉翼,领口处波涛汹涌,裙子开叉到了大腿根部。一颦一笑,更是勾人心魄。

%30
对此,方源也很无奈。

%31
毕竟这种他人梦境,他想要探索,就得成为梦境的源头身份。

%32
当然变成女性,并不会干扰到方源冷漠的心境。

%33
他一边戒备眼前的敌人,一边将注意力投放到仙窍中去。

%34
让方源欣慰的一点是,这个女仙身上的仙蛊并不少。

%35
“先试试这只蛊吧。”方源灌注仙元进去,立即发出了一声婴咛之声。

%36
这声音完全不受方源的掌控,充满了女性的娇柔和诱惑。

%37
那个青年蛊仙原本气势汹汹,听了这个声音,忽然全身发软,杀意陡降。

%38
他冲势一滞,狠狠咬牙,眼中重现清明之色:“妖妇,居然在你小爷身上耍弄手段!吃我杀招!!”

%39
说着,他双手一推,狂风大卷,一时间天地晦暗,无数风刃向着方源狂卷而来。

%40
方源连忙后撤,一边闪躲,一边试着催动第二只仙蛊。

%41
顿时,方源身上的粉裙,忽然化作了一团粉雾,缭绕在他周身,带着方源飞驰。

%42
飞驰的速度叫方源都暗自赞叹,竟然不比剑遁仙蛊逊色,唯一叫方源有些遗憾的是,随着飞驰的距离越来越远,他身上的粉色云雾不断减少,原本能包裹全身,此时已经被迫露出雪白粉嫩的大腿和胳膊了。

%43
方源倒不怕什么裸奔,这种羞耻心早已经离他远去,更何况还是在梦中。

%44
他遗憾的是,这种仙蛊效用明显不能持久,否则的话,单凭这等速度,对付眼前的敌人,就几乎立于不败之地。

%45
“贱人,你跑得挺快。但是你今日必定死在我的手中,为了杀你,给我兄长报仇,我愿意付出任何代价!”

%46
青年蛊仙双目赤红,张口咆哮。

%47
与此同时,他催动起仙道杀招。

%48
翠绿的风,在他的掌心中,凝聚成一只小小的匕首。

%49
匕首闪烁着幽幽碧光,方源单是稍看一眼,就觉得心惊胆战。

%50
“不好!”方源心头一沉。

%51
“单凭一两只仙蛊,根本无法抵挡这记仙道杀招。可是这个章丽有什么仙道杀招,我也不会用啊。”

%52
苦恼之际,方源连忙催动仙道杀招解梦。

%53
顿时,一股玄妙的力量,流转在方源此刻的仙窍当中。一只仙蛊,数十只凡蛊都在这个玄妙力量的包裹下,被催动起来。

%54
“这是?”方源察觉到这样一幕,顿时心中大喜。

%55
他连忙继续催动杀招解梦,在这记仙道杀招的影响下,玄妙力量持续不断,越加充足,将方源仙窍中的许多蛊虫都统合起来。

%56
然后,形成了一记方源完全陌生的仙道杀招。

%57
方源也不知道这记仙道杀招,究竟有什么作用。

%58
时间不等人,此刻,对面的青年蛊仙已经用嘴唇对准手掌心微微一吹。那碧风匕首,宛若利箭一般,暴射而出,直朝方源射来,又狠又疾!

%59
方源心中暗叹一声,此时此刻,他只能听天由命了。

%60
究竟自己借助解梦,使出来的仙道杀招,有什么效用,能不能当下这碧风匕首,方源心中也完全没底。

%61
不过就在这个杀招的效用下,方源不由自主地掐动了兰花指,对着青年蛊仙遥遥一指,媚眼如丝,柔声道:“死鬼,快来呀!”

%62
这声音正是妩媚诱惑到了极点,就连方源自己本身听了,都狠狠的打了个冷颤。

%63
随即,真的来了一头死鬼。

%64
这是蛊仙魂魄,有七转战力,悍不畏死地飞出来,直朝碧风匕首撞去。

%65
那青年蛊仙见到这个死鬼,顿时面色大变,口中惊呼道:“兄长!!”

%66
原来这位女仙章丽,竟然将丈夫暗害,并且还将他的魂魄炼成了一记仙道杀招。

%67
方源见到这样一幕,顿时灵机一动,口中高呼起来:“只要我撤销了杀招,你兄长的魂魄还能救下。你有种就将它也毁了吧!”

%68
青年蛊仙勃然大怒,但却唯恐伤害了自家兄长的性命,战斗起来束手束脚。

%69
方源心下狂喜不已,连忙出手,很快就扳回了局面。

%70
片刻之后,他借助对手的一次破绽,将其斩杀。

%71
梦境通过!

%72
“呼。”方源缓缓吐出一口浊气,他回到了现实当中。

%73
“这个梦境只有一幕啊。”方源稍稍有些意外。

%74
这是一个很小的梦境。

%75
方源一次性就通过了。细细品味,他发现这次增长的境界,不仅有智道,还有魂道。

%76
“魅情道从未真正脱离过智道,这世间也从未有什么魅情道痕,只有智道道痕。所以仍旧是智道境界的增长。”

%77
“至于魂道……恐怕这位章丽,也兼修魂道。尤其是她那仙道杀招死鬼快来,有着很浓重的魂道成分。”

%78
方源分析了一下,又整理了一下自己的境界。

%79
如今他的宗师境界多达八个流派。分别是血道、力道、变化道、星道、智道、水道、暗道、阵道。

%80
炼道是准宗师,运道大师,虚道仍旧是空白。剑道境界普通,魂道境界原本是普通,但是过了章丽梦境后,提升到了准大师的地步。

%81
“那个仙道杀招死鬼快来,有点意思。我虽然魂道境界不足,但是智道可是宗师。梦境写实,有很大可能,可以还原出一部分威能。虽然不能完全重现,但是也可仿造出一个相似的仙道杀招。”

%82
“不过就算这个仙道杀招仿造出来,也不如我改良刚背提升得多啊。”

%83
寻常蛊仙稍微有了一些可能,往往会很高兴的着手,开始推演改良仙道杀招。

%84
但方源不是这样。

%85
最近这段时间,他的境界提升了不少,灵感和可能有无数,他需要从中挑选出性价比最高的仙道杀招,改良出来。

%86
不得不说,这是一种幸福的烦恼。

%87
暗道境界提升为宗师,让他有了一个以暗渡仙蛊、坚持仙蛊为核心的杀招设想。这记仙道杀招,可以更好的隐藏他自己。

%88
阵道境界的提升,让他对仙道杀招刚背,有了一个全新的想法。

%89
这个想法,让他颇为心动。原本的刚背杀招,是以金刚念、坚持这两只仙蛊为核心,但在他的设想中,若是能增添防备仙蛊,以及一些阵道凡蛊的话,威能将会数倍暴涨。

%90
正巧章丽梦境之后,流转到方源面前的梦境,都不适合探索。

%91
方源便开始转移重心,改良起仙道杀招刚背来。

%92
就在方源迅速提升自家实力的时候,乔志材、武庸二人等来了铁面神。

%93
这位七转蛊仙,穿着武者劲装,胸背和腿脚,挂着轻甲,他的脸上覆盖着一层厚实的铁。

%94
这并非是他带着面具,而是铁面真传的修行特征。

%95
历代修行这份真传的蛊仙,必须得是心怀正义之人,同时他们往往也是南疆蛊仙界中最擅长查明真相的人。

%96
仔细查看了一下现场和那两具尸首,铁面神开口断言道:“不是乔志材下的手。”

%97
乔志材顿时吐出一口浊气。

%98
“那是谁?”武庸问。

%99
“有人模仿了他的仙道杀招木像杀。这种情况,其实早有历史记载。”铁面神答道,“不知武庸大人,有没有听说过七幻魔仙?”

%100
武庸皱眉:“你是说七幻真传的继承者?据说继承了真传的蛊仙,拥有一种可以模仿全天下,任何蛊仙手段的能力。铁面神,你能确定吗?”

%101
铁面神点头:“我族镇魔塔中,就曾关押过了一位七幻魔仙。七幻真传的内容,我族虽然没有打探得出来,但是一些秘密我族知晓更多。比如这七幻真传,分有七层。继承者可以自行选择,可以不继承任何一层真传,也可继承完整的七层真传。只是每继承一层真传,就要履行一份约定,在将来不确定的时间,为某个人做一件事情。七层真传,便是七件事。”

%102
“为了分辨七幻传人,我族早已研发出相对于的手段,可以分辨真相。刚刚我已用了,必定是七幻魔仙不假。”

%103
“有点意思。”武庸微微点头,他的眉头皱得更深了。

%104
虽然知晓了真相,但那又如何?

%105
七幻魔仙究竟是谁?他(她)背后又有什么人或势力?

%106
这些问题,武庸仍旧一概不知!

%107
但下一刻,铁面神的一句话,让武庸眉头舒展。

\end{this_body}

