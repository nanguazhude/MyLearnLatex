\newsection{麻烦的最后考验}    %第一百零一节:麻烦的最后考验

\begin{this_body}

石亭中的碑文上写着:不管是本家来人被天灵接引到这里,还是罪民中的凡人一路闯关,来到继仙山巅,想要继承黑凡真传的话,都需要通过最后一重考验。

但这最后一重考验的内容,石碑上却没有说,只是让后来者向天灵讨教。

天灵虽然愚昧懵懂,但却铭记着黑凡的旨意。

于是,方源主动询问。

很快,众仙便听到,黄钟天灵微微一荡,出一声悠扬的钟鸣。

钟声萦绕悬梁之际,石碑上的碑文又有了新变化。在原先内容的最后,渐渐浮现出了一行全新的文字。

众仙视之,无不惊呼。

“这上面说了,黑凡洞天本身就是真传的内容之一。谁若是继承真传,谁就是黑凡洞天的主人!”

“不过,要继承黑凡真传,最后一重考验,居然是这个?”

“难怪黑凡老祖当年定下规矩,一旦出现了真传的继承人,就要我们一起恭迎拜见!”

“老祖宽宏,设想周到。我们虽是罪民,但老祖并未忘记我们,心中还是念想着我们这些子孙后辈的。”

群仙议论纷纷,对黑凡老祖既感有佩,当中有的甚至眼眶泛红,险些淌下泪来。

方源注视着这行新字,眉头却是紧蹙,脸色阴沉下来。

这行内容,明确地告诉他,最后一重考验究竟是什么。

“只要通过了最后一重考验,我就能得到黑凡真传。但这算什么?最后一重考验,居然是要我获得黑凡洞天中,越一半人数的蛊仙的认可?!并且还规定了,必须在洞天时间的三年之内。”

方源心里直摇头。

黑凡老祖的这个最后考验,出乎他的意料之外。

很明显,是偏袒黑凡洞天中的这些蛊仙嘛。

作为一个外来人,方源要获取这些人的支持,是很艰难的。

不过方源很快又反应过来,他明悟了黑凡老祖的用意。

黑凡老祖设下这个规矩。是鼓励后来者,收编了这些黑凡洞天的蛊仙。毕竟蛊仙难以培养,收编了他们,对壮大黑家大有益处。

所以。这不仅考验后来者的个人能力,也考较继承者是否交际手腕,是否有团结他人的领导才华。

黑凡依靠自己的真传,选取的是黑家的领导者,并非是单纯的蛊仙强者。

“你这老家伙。明明是死了,还考虑这么多,真是多事!”方源腹诽,表面上则是仰头长叹,朗声动情地道,“黑凡先祖,一心为黑家着想,用心之良苦,让我这后人感佩至极啊!”

“是啊,是啊!”群仙对方源的话。很有共鸣。

本来这些蛊仙,心里七上八下的,毕竟方源继承了黑凡真传,他们这些罪民后代还不知是个什么结局。

但现在,黑凡老祖这么一安排,却是为他们着想。

方源缓缓收敛起感动之色,他转过身,站在石碑前,面对洞天群仙,直接询问道:“那么不知在下如何才能获得在场诸位的支持呢?”

不出方源所料。他的问话迎来的是一片沉默。

此一时彼一时啊。

以前群仙颇为忌惮方源,但现在方源必须要获得洞天中过一半蛊仙的认可,这样一来,之前的形势就颠倒过来。完全不一样了。

群仙你望我,我望你,一时间都没有说话。

只是他们看向方源的目光,都起了变化。

以前是小心翼翼,尽量散善意,掩饰恶意。现在的目光却带着疏远,端着架子,藏着考量。

方源也不着急,站在原地,悠悠等待他们的答。

又沉默一阵,作为资格最老的蛊仙陈尺,终于耐不住,咳嗽一声道:“今日之事,实在是生太快,让人猝不及防。一唉,许是年纪大了,思考这些难题叫老朽也是头昏脑涨,一时间接受不过来。想来上仙长途跋涉,来到这里,也是疲惫不堪吧?不如先休息休息,整顿精神,再来谋略商讨,也不迟啊。”

“这老狐狸。”方源心中冷笑。

但陈尺老仙的话,却得到了其余蛊仙的强烈支持,一声声的附和,接连不绝。

原本泾渭分明的两大团体,在这一刻,几乎都融合一体,齐齐来对付方源这个外人。

“不过,就先答应你又如何?”方源其实也早料到对方会这般反应,他脸色不虞,勉强点头道,“陈尺仙友此言有些道理。”

陈尺露出胜利者的微笑,但很快收敛下去:“在下所居虽是陋室,但也稍备了茶水。不如上仙且移贵足,你若是大驾光临,也是老朽的荣幸啊。”

陈尺热情地邀请道,但对方源之前的话,根本没有明确答复。

方源勉强含笑,点头:“那就叨唠了。”

说是“陋室”,当然是陈尺老仙的自谦之词。

他的居所,不仅不粗陋,而且还很华贵。

一处宫殿群落,坐落在山巅之上。

这山似乎是人为拔升的,顶部平坦一片,坐落着重重宫殿,金砖绿瓦,雕栏画栋。

陈尺和其麾下蛊仙,都生活在这里。

不仅如此,还有一大批的蛊师,甚至还有凡人。

“这些都是老朽的子孙后辈,呵呵呵,让上仙见笑了。这人一老啊,就喜欢含饴弄孙,享享天伦之乐。”陈尺为方源介绍道。

方源点点头:“这正说明陈尺仙友是念旧情之人呐。”

陈尺饱含深意地望了方源一眼:“有情有义之人,谁不欢迎呢?呵呵呵。”

“哈哈哈。”方源亦笑。

看着陈尺和方源交谈甚欢的样子,其余的三位蛊仙也都心情轻松起来。

就这样,方源便在这里暂时安居下来。

奇怪的是,自第一天之后,陈尺就没有再没有露面,也不和方源见面。方源却也不着急,安之若素。

四天之后。

群殿之中。

方源和陈乐在长廊中散步。

陈乐便是女仙之一,双髻,活泼俏丽的那位。从血缘关系上,乃是陈尺老仙的曾孙女。

“黑城公子,你看那朵荷花,开的颜色正是乐儿最喜欢的呢!”陈乐手指着荷塘,笑着道。

这长廊别具一格,穿行整个荷塘,横跨东西。

荷塘中长满莲花,各种颜色,暗香飘浮,美不胜收。

这几天来,虽然陈尺老仙再不见方源,但都由陈乐陪伴方源,观赏这片宫殿的各个美景。

“这朵荷花,嫩黄可爱,不妖不艳,正适合乐儿你。”方源笑着道。

陈乐低下头,满脸娇羞,低声柔柔地说:“公子,你说的哪里话?乐儿乐儿只是看着这荷花,心中欢喜罢了。”

“我看着乐儿你,心中也欢喜啊。”方源一边笑着,一边主动伸出双手,握住乐儿的手。

陈乐娇躯一颤,下意识想要挣脱,但方源双手攥得很紧。

陈乐满脸通红,饶是蛊仙修为,此刻脑海中亦尽是一片混乱,她挣扎几下,口中低呼:“公子,公子你”

方源跨前一大步,身躯几乎贴上陈乐。

陈乐连忙后退,仓促间身形不稳,向后跌去。

方源顺势,一把将她揽在怀中。

“小心,别跌着。”温柔的话语传进陈乐的耳中,陈乐反应过来时,现自己竟已经躺在方源的怀中了。

陈乐抬起眼,正看到方源嘴角含笑,目光中带着一丝戏谑。

陈乐羞恼至极,一记粉拳打在方源的胸口:“公子,你太坏了,欺负乐儿!”

说着,挣脱了方源的怀抱。

方源啊了一声,后退一步,脸上有猝不及防的痛楚。

陈乐连忙顿住脚步,满脸关切地走来:“公子,你怎么了?”

方源抽了一口冷气:“实不相瞒,在来之前,我可是经历了一场艰难的角逐。要继承黑凡真传,可不是件容易的事。至少家族中的蛊仙,有很多人都不愿意看到呢。”

“所以你身上有伤?你怎么不早说!”陈乐跺脚,一时间刚刚的羞恼都抛之脑后,望着方源的胸口,“还疼么?”

“小伤,没什么大碍。只是修为高了,道痕加深,受了伤比较麻烦而已。”方源笑了笑,忽然话锋一转,“不过,你家老祖炼蛊受伤,恐怕也是和我情况相当吧?否则,怎么这些天来,都不见他出面呢?”

陈乐眼中闪过一抹慌乱之色,嗯嗯啊啊,胡乱应付方源几声。

陈尺老仙自然不可能无故不见方源,想出来的借口,便是炼蛊失败,受到了反噬,伤势较重,不宜见客。

当然,在这个节骨眼上,他怎么会忽然炼蛊,莫名受伤?

双方皆知理由,心照不宣而已。

当天晚上,陈家四位蛊仙展开密谈。

陈乐汇报道:“老祖宗,黑城公子今天忽然问及你的病情呢。”

“哦?终于等不及了么”陈尺笑了笑。

“幸亏我当时遮掩过去了,他没有起疑心。不过日子一长,恐怕”陈乐担忧地道。

其余三位蛊仙对视一眼,纷纷笑了笑。

陈乐受到长辈们的爱护关照,还很单纯,不明白其实方源早就心知肚明。而他忽然问及陈尺老仙的病情,则是一种含蓄的提醒。

男蛊仙陈立志思索了一下,沉声道:是要和这黑城好好谈一谈了。”

ps:ps:今天两更,待会还有第二更哦,朋友们!

地一下云.来.阁即可获得观.】

\end{this_body}

