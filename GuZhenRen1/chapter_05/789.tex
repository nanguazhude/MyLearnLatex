\newsection{伐谋、求道}    %第七百九十二节:伐谋、求道

\begin{this_body}

天庭,仙墓。

一丝莫名的变化发生了。

一团火光,起先还很微渺,然而片刻之后,火光就暴涨,弥漫开来,将仙墓的上空尽数染红。

火光映照天地,立即吸引了紫薇仙子等天庭成员的注意。

紫薇仙子当即赶往仙墓的入口,又发现火光充盈之间,又浮现出点点的银白光辉,仿佛是无数钻石,点缀在赤红色的幕布上。

火光消散,一位豪迈的蛊仙老者,一身赤红大袍,昂扬大步,从仙墓深处迈出。

光钻汇聚,一道温和的光辉中也有一位老人,坐在木制的轮椅上,悠悠前来。

紫薇仙子眼中精芒闪烁,她惊喜地发现,这两位蛊仙老者的身份,她都清楚。

那赤红大袍的老者,乃是厉煌,炎道大能。而木椅老者则是顾六如,是宙道大能。

“晚辈紫薇见过二位前辈。”紫薇仙子施礼道。

顾六如点点头,没有说话。他苍白病态,形销骨立,一生经历坎坷,下半身瘫痪,无法治愈,因而性情冷漠。

厉煌则开口:“我两从沉眠中苏醒,不是毫无缘由的,定然是天庭需要我们的力量。不知眼下时局如何?”

紫薇仙子道一声惭愧,当即将局势如何如何,明明白白地告知眼前两位。

“方源?没想到天下竟出现这等的魔头!”

“哼,千古未有的大时代降至,总会有一些乱象。但这魔道横行,未免太过猖狂。归根结底,还是红莲魔尊种下的种子。”

厉煌冷哼,脸现怒意,十分不满。

顾六如面色如冰,语气冷静:“好在大势仍在我手,只要几年后,宿命蛊彻底修复,天庭仍将利于不败之地,即便是那方源再强,也翻不了太大的浪花。”

随后,他又对紫薇仙子道:“你是智道蛊仙,理当领袖天庭。接下来要我们这两个老头子做什么,都说出来,不必有什么顾忌。”

紫薇仙子脸上浮现喜色:“二位前辈刚刚苏醒,当然要先休整一番,配备好了仙蛊,恢复战力。”

“眼下光阴长河失去掌控,我方正准备不计代价,搭建宙道仙蛊屋。还要在光阴长河中铺设镇河锁莲大阵。二位前辈,若能坐镇光阴长河最好不过。”

顾六如、厉煌对视一眼,前者微微点头,后者哈哈一笑:“那就这样吧。让我好好会会这个小魔尊!”

单论战力,厉煌当然要强于顾六如,但紫薇仙子更加欢喜顾六如的出现。

这位宙道大能及时苏醒,填补了天庭在宙道方面的空缺,解了紫薇仙子的燃眉之急!

“只是,方源重生归来,这两人苏醒他知不知道呢?”

紫薇仙子想到方源,又皱起眉头。

方源能够重生,和这样的敌人作对,非常麻烦,总是让人怀疑自己的决断。

紫薇仙子不知道:上一世同期,只有厉煌苏醒。而顾六如是在最后时期,长生天进攻天庭的时候苏醒的。那个时候,顾六如和黑凡大战,难解难分,最终被刘流溜偷袭杀死。

方源提前出击,击溃了恒舟、今古亭,又杀死了清夜,导致墨水效应,引得顾六如提前苏醒。

这位八转的宙道大能,又在光阴长河这样的环境,必定能够对方源造成巨大的阻碍。

至尊仙窍,小中洲。

方源神念笼罩着这一片沼泽地。

这里有极其肥沃的优质淤泥,淤泥上面是一层清澈见底的河水,有成人膝盖那么深。

这是方源刚刚构造出来的资源点——花雾太泽。

太泽广阔,一望无垠。

每隔数里,就有一朵荒植太泽花的花骨朵竖立着,大若房屋。

当太泽花开,它的花瓣接连一体,形如喇叭,通常是白色为主,少数带着一丝丝的粉色。

而从太泽花的花心处,会渐渐弥漫出丝丝缕缕的白色雾气。

这些太泽雾气便是悔蛊的食料。

悔蛊喂养比较麻烦的一点,就是食材难以囤积。太泽花雾需要新鲜,所以最好的方法就是直接在仙窍中营造出一个相应的资源点。

上一世,方源就营造出这一片花雾太泽,不过是在得到了悔蛊之后。

而这一世,因为手头宽裕得太多,方源就提前铺设起来。虽然还没有弄到悔蛊,但是未雨绸缪,将来悔蛊一到手,就根本没有喂养的后顾之忧了。

连还未到手的悔蛊,方源就已经提前解决了喂养的难题,他手中的仙蛊喂养,已经全部自给自足了。

就连不久前,从巴十八手中缴获的八转仙蛊加,方源也筹措完成,在小东海中划分出了专门的海域。并且,又引进了一整支的水蜘蛛兽群,让八转仙蛊加也有充足的食料。

自然,这些水蜘蛛是方源敲诈南疆正道得来的。

尽管方源已经将南疆俘虏的仙窍,吞并了绝大多数,魂魄也搜刮了底朝天,但南疆正道却是不知道这一点,乖乖承受着方源的勒索。

就像天庭不知道红莲魔尊留给方源的那一份真传,其实没有直接破坏宿命蛊的方法,但天庭唯恐不妙,拼尽全力要挽回光阴长河方面的劣势。

地球上有一句兵法:上兵伐谋,其次伐交,其次伐兵,其下攻城。

至理名言也!

人年轻的时候,总喜欢喊打喊杀。其实,真正成熟之后,才会逐渐明白:争斗只是达到目的,获取利益的第三等的手段。

拼杀战斗,风险太大,收益也不稳定,往往是伤敌一千自损八百。只有到了不得不进行的时候,才会去祭起这个手段。

方源行事,更倾向于谋略。他不是一个莽夫,只有需要做拼杀的时候,他才会举起屠刀,并且毫不怯弱畏惧。

当然,蛊师世界和地球终究是有本质上的不同的——通过修行,个人的力量能够凌驾于集体和组织之上!

因此,就有幽魂魔尊这样的人物。

幽魂魔尊生前所有的手段,都只有一个——杀!

杀杀杀杀杀杀杀!

屠戮一切,毁天灭地。

人头滚滚,血流漂杵。

他既不伐谋,也不伐交,但偏偏取得了至高的成就。

为什么呢?

因为他修的是道!

是履行自身对天地的理解,对自我的理解。

他每杀一人,就是对自我的进一步坚持,对天地的更深层理解,这些都助长他的魂道境界。

而这些境界上的提升,极大地助益了他的实力提升。

若是在地球上,你若是这么干,那就是找死了。

方源前世五百年,忽然体悟到这个道理时,已经有一百多岁。

从那以后,他就在想:什么是自己的道?或者说,自己能开创出什么样的流派呢?

这个问题,他至今也没有得到一个完美的答案。

四百多年过去,他不断重生,仍旧在摸索。

他只有一个隐约又模糊的方向。

他在黑暗中跌滚。

而在这黑暗中,有着太多的艰难险阻,有着蠢蠢欲动的恶兽,要把他一口吞噬。

方源清楚:在这些恶兽当中,天庭就是眼下最大最强的一只。

他唯有奋尽全力,才有抗衡这头恶兽,脱离它猎杀,继续走自己的路的希望。

方源明白:他和紫薇仙子屡次互爆情报,都希望借助其他蛊仙的力量,来对付彼此。这样的对拼,两方中没有一个胜利者。就像是狮虎决斗,血腥气会引来无数的鬣狗。

他看似声威赫赫,但幕后暗流是多么的汹涌,多少人会在背地里想方设法地对付他,对付春秋蝉。

没有人希望有这样的威胁!

春秋蝉不是无敌,它只是一只七转蛊,有太多的办法来针对它。

就连琅琊福地,都有不少克制春秋蝉的法门。

事实上,天庭一直无法拿出有效的针对手段,一直令方源颇感诧异。

所谓小魔尊的称号,看似风光无限,其实是杀机暗藏。这是在提醒世人,方源若不再扼杀,他就要成长为魔尊,将来谁都不好过!

方源虽然在光阴长河中获胜,占据了主动。但他心中雪亮,这不过是小战略上的胜利,敌不过天庭的大战略。

宛若围棋对弈,他不过是在小角得胜,小胜一场,虽然化被动为主动,值得欢欣鼓舞。但在整个棋盘上,还是天庭占据大势。而这是天庭无数年的积累,元始仙尊提前三百万年前就已经落子。

“天庭的底蕴啊,或许墨水效应再次发生,从仙墓中又苏醒了什么强大的人物了。”

“唯有一刻不休,拼尽全力提升自己,才是最重要的。”

方源没有因为胜利而麻木自大,他欣喜的时间都很短,他又投入到紧张的修行当中。

说他争分夺秒,毫不夸张!

魂魄修行十分顺利。方源的魂魄从一荒魂,稳步提升,三荒魂、五荒魂、八荒魂……然后突破十荒魂。

五禁玄光气已经彻底演练纯熟。方源甚至加以改良,这种改良程度相当微小,但令它十分贴切自身的催用习惯,让方源每次催动都非常舒服。

方源开始改练其他的杀招,比如巴十八的连击杀招。

在这期间,他也频繁外出,动用翠流珠杀招,提前出手,往往在刚刚形成的地沟中收获良多。

其中,七转地脉仙蛊就是一项相当关键的收获。

------------

\end{this_body}

