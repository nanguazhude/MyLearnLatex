\newsection{天作之合!}    %第七百七十节:天作之合!

\begin{this_body}

%1
轰隆隆。

%2
方源悬浮高空,俯瞰脚下。

%3
巨大的山脉在崩溃坍塌,五色的光气仿佛沸腾起来,剧烈宣泄。

%4
但偏偏各色光气之间并不交融,也不相互干扰,各行其是。

%5
“要正确的摧毁五色山脉,并不是那么简单。上一世的武庸,也是得到了乔家钻研此处传承多年的积累,方能功成。”方源心道。

%6
陶铸真传要引出来,需要破坏整个五色山脉。

%7
这样的动静非常巨大,之前方源一直都按捺不动,到了此刻才忽然出手。

%8
一来,他刚刚大胜,俘虏了南疆群仙,实力大涨。二来,趁着南疆群仙六神无主,天庭的注意力也被吸引之际,迅速出手!

%9
喧腾的五色光气开始消融、崩解,以一种匪夷所思的迅速,不断缩减。

%10
方源满意的点点头。

%11
胡乱地改变此处地脉,除非运气极佳,瞎猫碰死耗子,否则正常情况下,五界光气仍旧会大量存在。

%12
惟独真正理解此处布置,利用正确的解法,方能够达到改变地脉,导致五色光气完全消解的结果。

%13
不一会儿,五色光气彻底消散。

%14
一股强大的气息,猛地升腾而起,巨大的五彩光柱直冲九霄,同时一个宏大的声音响彻整个五界山脉:“后来的小辈,做的不错,居然能通过老夫的最后一层考验,将整个五界山脉摧毁。现在,老夫留下的五界传承就是你的了!你可要好生修行,来日纵横天下,不要堕了老夫的名头啊。”

%15
五彩的光柱当中,一股意志如烟云般汇聚起来,形成一个徐徐如生的蛊仙模样。

%16
这是陶铸留下来的意志。

%17
方源冷冷一笑,对着陶铸意志遥遥伸手。

%18
陶铸意志顿时一僵,感到一股无形的巨力将他牢牢束缚。

%19
随后,他就像是泥地里的萝卜,被硬生生地拔出彩色光柱,嗖的一下,飞到方源身前,被他一手捏住喉咙。

%20
陶铸意志顿时又羞又恼,瞪着方源:“你这七转的小辈,休要羞辱老夫。老夫生前可是八转……呃!”

%21
方源故意泄露出一丝八转气息,陶铸意志感知到后,立即面色变了。

%22
他呵呵一笑,态度大为缓和,以平等的语气道:“原来是仙友啊。”

%23
暗地里则在思量:“这小年轻一表人才,风度翩翩,没想到行事如此恶劣霸道。明明是八转,却又伪装成七转,断然不是正道的行径。嗯,他不是什么好人!”

%24
方源微微一笑:“我乃当世魔头,陶铸老儿,你的真传我收下了。”

%25
说着,他一挥长袖,早已酝酿好的杀招射出去。

%26
大地震动,从五色光柱中迅速飞出大量蛊虫,夹杂仙蛊,纷纷投入方源的仙窍。

%27
随后,五色光柱迅速消散,再不惹人注目。

%28
陶铸意志目瞪口呆。

%29
方源的智道手段,能洞悉他的暗中思想,这并不出奇。

%30
他感到震惊的是,没有得到他的认可,方源居然直接将全部的真传内容都给拿走了。

%31
当初本体布置下来的手段,简直是形同虚设!

%32
“阁下到底是何方神圣?”陶铸意志态度再变,看着方源的目光带着探究和忌惮。

%33
单单方源露的这一手,已经超越了陶铸生前的造诣,已经他不仅完全看破了陶铸的布置,而且还完美的针对,成功破解。

%34
方源冷眸一瞥,陶铸意志顿时感觉身体被掏空,哦不对,是全身上下都被看透,毫无秘密可言。

%35
方源微微一笑:“陶铸老儿。”

%36
“我,我在。”陶铸意志直觉方源的微笑下,有着一种大恐怖,他再不敢拿捏身份。

%37
“你这真传很有意思,将来我方源驰骋天下,进犯天庭时会有大用。为了奖励你,我允许你在我的洞天中再择传人。我也不毁灭你这股意志,还会适时为你补充消耗,让你永久长存。”方源道。

%38
陶铸意志被方源拿捏,浑身动弹不得,宛若木偶,他心中充满了苦涩。

%39
本体当初钻研五域界壁的奥妙,在生命的最后时期终于有所突破,有了惊世的成果!

%40
但本体大限将至,没有时间来扬名立万,只好立下传承,留待后人。

%41
这是一种巨大的遗憾。

%42
所以,陶铸本体当年便依据五界山脉,布置下动静极大的真传,精心设计了种种关卡,进行考察和筛选,企图寻找到优秀的弟子,来继承自己的衣钵。将来弟子扬名立万,自己也能名动五域,算是弥补了当年的遗憾。

%43
结果呢?

%44
碰到方源这个硬茬子。

%45
不仅直接捣毁五界山脉,前面的所有考验都通通跳过,直接来到最后一关,而且还拿捏自家的意志,强行夺走真传内容,根本没有什么好脸色,居高临下。

%46
陶铸意志憋屈啊,这和他本体设计的剧本,完全不符合。

%47
但他又有什么办法呢?

%48
他只是一段意志而已。

%49
“甚至就算本体健在,恐怕也不是眼前这人的对手吧?”

%50
“听他刚刚那话,竟然要进犯天庭,没有实打实的本领,绝不会这样讲的。”

%51
想到这里,陶铸意志叹息一声,对方源道:“罢了罢了,时也命也。”

%52
“我也不管你是好人还是坏人,真传落到你的手中,说不定会引发腥风血雨、生灵涂炭,但这必然也会名动五域吧?这也是本体想要的结果。只盼你将来不要抹杀了本体的成就,如此真传给了你又有何妨?”

%53
“哈哈哈。”方源仰头大笑三声。

%54
他笑声激越,黑发在风中飞舞,眼眸中神芒激射,周围风云鼓荡。

%55
他笑毕,认真地看着陶铸意志:“放心吧,我岂会贪图这点死人的名誉?如果这点器量都没有,又如何能胜得过天庭,超越历来的尊者,追逐那缥缈至高的永生?”

%56
陶铸意志双眼再度瞪大,心想:“嗯?这人口气大得要上天!居然要战胜天庭,超越尊者?这是疯子还是傻子?糟糕,我的想法他都能洞悉!”

%57
方源再大笑三声:“若是没有这点野望,做人还有什么意思?失败了也无妨,大不了重来几次好了。就算最终也达不到,又如何呢?”

%58
陶铸意志猛地神色凝固,呆呆地望着方源。

%59
他看到方源呼吸一口气,眺望远方,他的双眸黑幽一片,然而陶铸意志却仿佛在这世间催深沉的黑色中看到了灿烂的烟火!

%60
一瞬间,许多的回忆浮现他的心头。

%61
无数人的质疑和轻视……

%62
“钻研五域界壁?这有什么好钻研的?”

%63
“真是狂妄啊,这等难题,万古长存。他陶铸何德何能,能解此题?”

%64
“一个六转的小仙,还不过是个散修,普普通通。”

%65
南疆正道的打压……

%66
“陶铸!你不要再研究了。”警告的人修为高深,面容冷酷如冰。

%67
“为什么?我又没招你惹你!”陶铸气愤反驳。

%68
“你还不明白吗?”南疆正道蛊仙深深地看着他,语气冷酷如冰,“五域界壁的真正意义是什么?”

%69
“如今中洲最强,四域皆弱。若是你真的研究有成,必将惹来天下动荡,五域相争,一片浩劫。”

%70
“为天下亿万生灵着想,你的这些研究成果我们就都收走了。”

%71
“不——!”

%72
轰。

%73
“不自量力的家伙,若不是看在那位的面子上,岂有你命在?”南疆正道蛊仙冷瞥一眼,扬长而去。

%74
爱人的别离……

%75
“陶铸,我们不合适,一切到此为止吧。”

%76
陶铸痛苦不堪:“是,我只是一届散仙,而你却是姚家太上大长老的掌上明珠!”

%77
“不,不是这一点!你还不明白吗?陶铸!是因为你啊!你日日夜夜钻研你的五界奥义,陪我的时间有多少?你扪心自问一下,你真正将我放在心上吗?你爱你的那些研究,胜过爱我!”眼前的女仙哭泣起来。

%78
陶铸哑口无言。

%79
女仙一抹眼泪,深呼吸一口气,用婆娑的泪眼望着陶铸:“最后一次,陶郎,我最后一次问你,你是要我,还是要你的那些研究?”

%80
陶铸低头,面色犹豫且又迷惘。

%81
女仙逼近一步:“和我在一起,入赘姚家,我们双宿双飞,生个一儿半女。不要再想什么五界奥义了,修行的资源不会短缺了你,有我父亲在,你也不必害怕什么灾劫。陶郎……”

%82
女仙深切呼唤,让陶铸心弦为之颤抖。

%83
陶铸看向女仙。

%84
女仙期盼的眸光,像是有无形的巨力,令他不由自主地倒退一步。

%85
“我、我……”他双拳捏紧,喉结滚动,想要说什么,但什么也说不出来。

%86
他不想欺骗女仙,更不想欺骗自己!

%87
女仙望着他,惊世的容颜却变得越加苍白。

%88
最终,她的明眸彻底暗淡下去。

%89
她淡笑一声。

%90
转身。

%91
晶莹的泪珠落下。

%92
她架云而去。

%93
不久后,传出姚家和另外一个正道势力联姻的消息,女仙正是新娘……

%94
“他就是陶铸?”

%95
“是个傻子吧?为了研究什么五界之秘,竟舍弃了这样的姻缘!”

%96
“是个疯子才对。我有一段时间经常看到他,专门往界壁里钻,弄得自己头破血流,狼狈如狗。”

%97
女仙新婚大典的那一夜,陶铸缩在阴暗的山洞里,望着眼前的凡道小阵,神情呆滞。

%98
这个小阵,只有区区两转程度,乃是他研究大半辈子得到的成果。

%99
他看着这座小阵,脑海中又浮出女仙绝美的身姿,那一颦一笑都惊艳春月的美景。

%100
他哈哈一笑,干涩的声音在山洞中回荡。

%101
他继续钻研起眼前的小阵,脸上泪水横流。

%102
一切的努力并非白费,偏执的种子扎在浸染心血的泥土中,也开出暗色的花……

%103
姚家的太上大长老闻讯而来。

%104
身为八转的陶铸,昂着头,看向他。

%105
姚家太上大长老平静地看着陶铸,目光中却带着一丝怜悯:“我听闻你最近钻研有成?”

%106
“侥幸。”陶铸冷笑,“你此次来,是想抢夺成果?”

%107
对面的老人缓缓摇头:“你才刚刚八转,我并不想以大欺小,以强凌弱。”

%108
陶铸的笑更冷三分:“你这话骗骗入世未深的人还行,对我说?”

%109
陶铸摇了摇头,脸上浮现出不屑的神色。

%110
姚家太上大长老却是笑了笑,不以为意。

%111
“我承认。”他叹息一声,点点头,“我也看走眼了,没想到你能晋升八转。若提前知晓这一点,我绝不会暗中设计阻拦,而是会栽培你,千方百计地撮合我的女儿和你在一起。就算你愚痴偏执,就算你不能让我的女儿幸福,一切也都要以我姚家利益为重。”

%112
陶铸脸色微变,一提到他深爱的女仙,他的目光顿时晦暗下去。

%113
“然而。”姚家太上大长老面容一肃,语气加重,“我是绝对不会让你研究什么五域界壁的奥秘!”

%114
陶铸眼中冷芒一闪,顿时恢复之前的冷酷神色:“哼,又是害怕引动天下动乱,五域混战,生灵涂炭的那套说辞?然而你可想过?五域界壁的奥秘一解开,还能带来繁荣和希望!界壁是阻碍五域交流的重大障碍,它若消失不见,五域蛊仙间的交往将更加自由和频繁,我们的交易会带动更多的修行功果。”

%115
“然而,战争的可能要远远大于和平,不是吗?”姚家老人打断陶铸的话。

%116
陶铸沉默,没有反驳。

%117
五域间的不同点实在太多了,不仅是地貌不同,社会风情也不同,更重要的是资源不平等,人口密度也不一样,野心家至古以来都绝不缺乏。

%118
不像现在的五域,相互间隔,交流不多。各有各的经济、政治、军事上的平衡,一旦界壁打开,平衡就会轰然瓦解!

%119
“所以,你还是来劝我收手?”半晌,陶铸打破沉默,他不屑地冷笑一声,“你知道的,这绝不可能。”

%120
姚家老人点头,郑重地道:“我必须得承认,要解决一个八转的存在,风险很大,我们南疆正道要做出这个决定,很难。”

%121
“非到万不得已,我们绝不会选择生死决战。所以,我此次来是想告诉你一个秘密。这个秘密我希望你听了之后,能够为我姚家保守下去。”

%122
“我为什么要非听不可呢?”陶铸笑了。

%123
姚家老人也笑:“因为这个秘密,就是关乎五域界壁。你想不想听?”

%124
陶铸动容,他的心中涌现一抹迫切之情,但旋即又被他强自按捺下去。

%125
姚家老人眼中怜悯之色越发浓郁:“这个秘密对你而言,实在有些残酷,但事已至此,我只得说给你听。在未来,五域地脉将合而为一,五域的界壁也会自行消失。”

%126
“什么?!”陶铸不禁震惊地叫出声来。

%127
姚家太上大长老几乎带给陶铸最致命的一击!

%128
他的这个秘密,直接否决了陶铸一生的追求。

%129
五域界壁会消失,那陶铸的研究岂不是没有了意义?他为此付出的努力、心血,冒的种种生命危险,舍弃最挚爱的仙子……种种代价,是否也成了对他的讽刺,成了一个笑话?

%130
“这绝不可能,你说谎!”陶铸的声音歇斯底里,却又透露出一抹恐慌。

%131
“这是我姚家继承了一道乐土真传,证据已经给了你。就算你不相信我,难道乐土仙尊的话,你也不信吗?”

%132
姚家太上大长老说道这里,微微一笑:“况且,你自己研究了几乎一生,五域界壁和地脉的关联你是十分清楚的,你自己也应当有所察觉吧?你的种种成果,定然是有许多部分可以验证这个未来的吧。”

%133
陶铸满头大汗,瞪大双眼,双眼却是空洞无比,他倒退几步,摇摇欲坠,低头看着地面,神情恍惚而且慌张。

%134
姚家太上大长老微笑。

%135
他望着陶铸,老好人的神态,却分外透露出冷漠残酷的意味!

%136
话说到这个份上,已经足够了。

%137
姚家老人转身离开,临走前,他又抛下一句话:“不要研究什么五域界壁了,它们迟早一天会自行消失,你的研究毫无意义可言。”

%138
“当然,你若是在五域界壁存在的时刻,研究出了成果。我想你的成功之时,就是我南疆正道的诸多八转,联手剿灭你的时刻了。”

%139
“言尽于此,陶铸小友啊,你好好想想吧。”

%140
陶铸呆呆地站在原地,一动不动,满脸灰败之色。

%141
……

%142
往昔如梦。

%143
回到现在。

%144
陶铸早已身死,他留下的一股意志被方源掐着脖子,攥在手里。

%145
真传已经被强人夺走了!

%146
然而,方源的话让陶铸的意志为之触动——

%147
“若是没有这点野望,做人还有什么意思?失败了也无妨,大不了重来几次好了。就算最终也达不到,又如何呢?”

%148
“哈哈哈,哈哈哈!”陶铸意志疯狂大笑。

%149
数年后,本体从姚家太上大长老的打击后恢复过来,也同样有着类似的觉悟!

%150
是啊。

%151
就算最终还是失败,研究不出来,一生都失败!又如何呢?

%152
就算五域界壁会在未来自行消失,又如何呢?

%153
就算我的研究毫无意义,又如何呢?

%154
我就是想这么做啊!

%155
你说我疯狂,你骂我白痴,你咒我偏执,你斥我愚蠢……

%156
好吧。

%157
我就是疯子,也是傻子,我不仅偏执而且愚痴。

%158
但是我——就是想这么做啊!

%159
就是想这么活着啊!

%160
这么活着,对我而言才算是有趣!

%161
然而,还有一些遗憾,所以我留下了传承。

%162
现在,这个传承让方源得到了。

%163
一瞬间,陶铸意志兴奋起来:“方源是么!我把这个传承交给你,并且在另外一处地点潜藏着的真意,也统统给你。”

%164
“我的本体并不擅长战斗,所以有毕生的遗憾。但是我感觉,传承交给你了,这些遗憾将会统统弥补!这是一场……天作之合!!”

%165
方源嘴角微翘,露出昂然自信且又带着一抹张狂的笑意。

%166
“那你就看好吧,陶铸老头。”

%167
“我会让你的真传——”

%168
“摄神惊仙,名震天下!”

\end{this_body}

