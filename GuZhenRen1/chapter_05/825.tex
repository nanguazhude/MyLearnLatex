\newsection{弑父梦启}    %第八百二十九节:弑父梦启

\begin{this_body}

%1
吴帅眉头紧皱,满脸苦涩之意:“我现在怀疑的是,若是这只是一个局,针对我和我父,专门陷害我族的阴谋,我该如何是好?龙人当兴,或许只是一个谎言,真凶专门设计陷害我族的虚假措辞。”

%2
古凉沉思片刻,否决道:“这种可能性相当的小。因为按照你父的述说,他能够接触到宿命蛊,真的是许多巧合叠加起来。并不能受到人为的控制,当中很多关键步骤,都是你父的自行选择而已。”

%3
“正是因为有太多的巧合,所以我才怀疑。”

%4
“仙友啊,当初人族是如何获悉宿命蛊中的天命昭示呢?你父亲的状况,不就是类似于此吗?”古凉一句话让吴帅哑口无言。

%5
就在这时,天空中传来一声洪亮的声音,是有天庭蛊仙前来,说是探望吴帅父亲的伤势。

%6
“来得好快!”古凉面色大变,“吴帅仙友,速速决断,若是稍迟一步,不仅是你大半辈子的努力化为乌有,甚至你们龙人一族也要被灭绝啊!”

%7
吴帅顿时神情惨然,脸色苍白如纸,身躯摇晃,直欲栽倒。

%8
形势所逼之下,吴帅不得不做出决定。

%9
他知道这个决定,他将为之痛苦一生,但他不得不这样做!

%10
哪怕他不敢肯定,真的是否有龙人当兴的天命昭示,或许真的是别人陷害,但他不敢赌。

%11
他若赌输了,输的不仅是他父子,还有整个龙人一族。

%12
当天,吴帅之父因搭建东天门失败,遭受反噬而形神俱灭。

%13
天庭蛊仙毫无线索,虽有怀疑,但对八转蛊仙的吴帅难以下手,只得离去。

%14
之后连续十几个夜晚,吴帅无法入睡。

%15
泰琴安慰他,吴帅抱着自己心爱的人痛哭流涕。

%16
泰琴温柔地道:“人死不能复生,还请师兄你节哀顺变。伯父牺牲,也是因搭建中天门而亡,天庭必有表彰。”

%17
具体的原因,吴帅不敢告知泰琴,只能将苦果咽在肚子里。

%18
他只得这样说道:“你知道吗,师妹。当年我斩杀松涛子后,便时常做着一个相似的梦。”

%19
“在这梦中,我成为龙人一族的王,带领龙人一族崛起,过着自由和平的生活。”

%20
“我深居龙庭,无为而治。龙人们安居乐业,平等安稳。”

%21
“黄维仍旧是我最得力的助手,帮助我处理俗务。”

%22
“我的父亲为我骄傲,会对我说:‘我儿,你真的做到了。你是爹的骄傲!’”

%23
“而我和你幸福地生活着,膝下儿女成群,在恣意欢笑打闹。”

%24
说到这里,吴帅仰头哀嚎:“可是如今,我只剩下你了啊!我的师妹。”

%25
泰琴也神情哀伤,这么多年来,她一直都没有怀上吴帅的孩子。

%26
龙人之间结合,繁衍多多,比人族还能生。但异人和人族之间的混血儿,概率非常的小。不只是龙人,其余的异人也是同样如此。

%27
这个事情给吴帅带来极其沉重的打击,之后大半年的时间,他都沉浸在悲伤当中。

%28
古凉前来劝说道:“吴帅仙友,你的悲痛我万分理解,但是还请你注意天庭,切勿过于悲伤自责,让天庭方面看出破绽。若是功亏一篑,你父岂不是白白牺牲了吗?”

%29
吴帅如遭电击,猛地惊醒过来,他擦擦头上的冷汗,向古凉拱手拜谢:“仙友,多亏你的指点,否则我就要误了大事了。”

%30
古凉点头,微笑:“你这些时日,实在过于悲伤,都没有继续东天门的建造。现在该继续下去了。”

%31
“可是宿命蛊中的昭示,那句龙人当兴是否是真的呢?”吴帅心有牵挂,他万万放不下。

%32
古凉笑道:“要证明这点,最好的办法就是让你再次接触宿命蛊。但这里的风险太大了。事实上,你大可不必焦急,只要安心等待,便能慢慢验证。”

%33
“你是指?”吴帅似有所悟。

%34
古凉点头:“没错。若真的是龙人当兴,那么天意必定会对你们一族大加关照。你的父亲就如同历史上的余祭。余祭还是蛊师的时候,只是一位奴隶,负责打扫石人圣殿,结果靠近宿命蛊,获悉了人族当兴的天命昭示。并且一路上按照宿命蛊的指点,将其盗走,献给了最关键的人物——还未成就仙尊的元始。”

%35
“余祭这人我知道,乃是元始仙尊的护道人。他得到天命昭示,却未身死,反而盗出了宿命蛊。为何我父却……”吴帅神情痛苦。

%36
古凉道:“这个原因,我也有猜测。还记得我和你说过,星宿仙尊曾经以身合道吗?从那以后,她的意志便干扰天意。恐怕就是因此,你父才遭此厄运,不过他也成功带出了龙人当兴的大秘密,也算是死得其所了。”

%37
古凉又继续劝慰道:“你父将这秘密抖露给了你,吴帅啊,搞不好你便是天命中的关键人物,如同元始一样。所以你万万不能气馁,不能放弃,更应该奋发图强才是!”

%38
吴帅摇头:“我岂能和尊者相提并论?况且我们异人,是不能成就尊者的。”

%39
古凉沉默片刻,神情莫名地道:“异人不能成尊,的确是一个现象,但究其原因,一直都没有定论,众说纷纭。”

%40
“说实话,我族一直在研究异人为何不能成尊。”

%41
吴帅奇怪:“不是说,人族乃是万物之灵吗?”

%42
古凉冷笑:“呵呵,这恐怕只是一个幌子。成尊有着大秘,或许和灵性多少有些关联,但绝对不只是这一层。人族的历史上,也有一些例子。一些蛊仙实力超绝,渡过八转的全部灾劫,却仍旧不能成尊,修为始终不能晋升九转。”

%43
“你们的老祖宗龙公,应当清楚更多。他就是这个例子,当年实力超强,却始终无法成就九转,这才转变成龙人。”

%44
“如果仙友能够探查清楚此中内幕,我族愿意付出任何代价,获悉这个秘密!”古凉郑重地承诺道。

%45
“仙友严重了,若是我能探知到这个情报,必定无偿与仙友分享!”吴帅指天发誓。

%46
临别之前,古凉又关照道:“吴帅仙友,还请注意一些。虽然宿命蛊不在你的手中,但若真的天命所归,必定有所启示。你绝对是当中的关键人物,这些启示恐怕会落在你的身上,或者是你身边亲近的人的身上,还请你多加留意。”

%47
“我省得了。”吴帅点头,“就让时间来证明一切吧。”

%48
然而,让吴帅失望的是,他并没有获得任何的启迪。

%49
他开始继续搭建东天门,以安天庭之心。

%50
龙人一族,乃至许多龙人蛊仙都对他颇有微词,暗中大骂吴帅属狗的。自己父亲因为搭建东天门而死,现在作为儿子的还仍旧巴巴地去搭建东天门,真是奴性深重,无可救药。死了老爹,也是活该!

%51
因此,南华岛上外迁的龙人,也越来越多,逐渐形成一种风向和浪潮。

%52
这一天睡梦里,吴帅再一次梦见了自己的父亲。

%53
他的父亲告诉他:“继续努力,为父已经深切地理解你。你的牺牲必定成就你的伟大!龙人一族需要你的领导,我的孩子。”

%54
他又梦到自己和泰琴终于有了子嗣。原来在东海的某个海底地沟深处,长着一种无名的杂草。

%55
这种杂草乃是上古荒植,蔓延大片,有着太古荒兽群落守护。

%56
只要他和泰琴吃了这种海草,便能够诞生子孙,跨越混血的窘境。

%57
梦醒之后,吴帅忽然一个激灵,他猛地意识到:这是否就是天命的启迪呢?

%58
他仔细回忆。

%59
黄维死前,他就在做梦,梦见黄维向他道别。

%60
父亲死前,他也做梦,梦见父亲谈及他早已经逝去的母亲,言语间无比怀念,直言想要去聚聚。

%61
要验证这一个猜测,其实非常简单。

%62
吴帅当即制定计划,借助古凉之力,让他回去东海,取得地沟中的无名海草。

%63
按照梦中的方法,吴帅和泰琴双双服用海草。旬月后,吴帅得到特大的喜讯——泰琴竟真的怀上了他的孩子!

%64
这种惊喜真的是太大了。

%65
不仅是因为他终于得到了朝思暮想的子嗣,并且他还验证了猜测,他终于明白:天命的启迪早就开始了,只是他一直都没有意识到而已。

%66
梦境开始变化,继续给吴帅带来启迪。

%67
吴帅按照种种启迪去行事,皆是事半功倍,马到功成。

%68
他不仅将东天门搭建好,并且还成功地布置了暗门,就连天庭验收时都无法发觉。

%69
为了取信天庭,他不在发展龙人势力,而是专注自身修为。

%70
依靠梦中启迪,他的修为不断上涨,再加上古凉的帮助,屡屡渡过灾劫。

%71
梦境又发生变化,出现了打造八转仙蛊屋龙宫的启示!

%72
“八转龙宫的核心仙蛊,就是八转的——如梦令?这是梦道之蛊,真正地切合大时代!”吴帅振奋无比,立即游历五域两天,搜寻梦道仙材。

%73
这些梦道仙材非常稀少罕见,通常都在犄角旮旯里,毫不起眼。

%74
但吴帅却是凭借着启示,按图索骥,一一将其收拢,顺利炼化。

\end{this_body}

