\newsection{天庭来信}    %第三百六十一节:天庭来信

\begin{this_body}

布置在掠影地沟中的超级蛊阵,虽然被破坏,但也给方源等人带来了警示。(www.QiuShu.cc 求书小说网)

此时,南疆正道蛊仙还在追来的途中,距离这里,还有一段路程。

方源利用上极天鹰,立即逃跑,还是可以的。

不过,他仔细思考之后,还是决定演练上古战阵四通八达,将这个手段提前配上,未雨绸缪。

因为现在方源的手段,并不能解开身上的侦查杀招,也就是说,他的位置始终处于暴露的状态。

在被追杀的途中,极有可能会遇到和敌人遭遇的情景。

就比如刚刚,方源一行人在撤退的途中,遭遇到了翼家的仙蛊屋海角阁。

在这样的情况下,方源等人乘着上极天鹰跑路,其实是冒着很大的风险。

因为上极天鹰不是仙蛊屋,蛊仙和它不是一个整体,没有仙蛊屋的攻防一体。这就导致,鹰背上的蛊仙们,很容易被外界针对。

但如果掌握了上古战阵四通八达,就可以瞬间挪移,跨越极长的路程,离开险地。

影无邪等人就是有了这样的手段,连天庭的追缉都能逃得掉。方源屡次追杀未果,这个上古战阵居功至伟。

现在,方源成了影宗之主,敌人的优势立即转变成了自己的优势。

方源将所有的蛊仙,都召唤到自己的面前,将上古战阵的内容,都传授给每一个人。

他们当中,白凝冰、黑楼兰、影无邪是早已掌握了这个阵道杀招。

但是方源、妙音、黑菟却是头一次接触。

不管怎么算,最多都只有三位掌握了四通八达,其中影无邪状态极差,完全不能指望和依靠。所以必须要大家召集一块儿,一起演练。

这种阵道手段,需要各个蛊仙配合,单独练习的意义不大。

就在方源等人紧急演练上古战阵的时候,以武庸为首的南疆正道蛊仙们,浩浩荡荡地开赴远地。

“武庸仙友,根据你掌握的情报,对方目前很可能就在掠影地沟。如今我们已经距离地沟不远,还请仙友下令,统筹全局。”池曲由道。

他也是八转蛊仙,修为和武庸一样,甚至论资排辈,乃是武独秀一代的人物,比武庸还要大一些。

奈何武家声威,仍旧鼎盛,池家却是始终偏安一隅,声望远不如武家。

再加上铁面神向外透露出的武庸的情报,真正让南疆蛊仙开始对这位武家的太上大长老,产生忌惮之情。

甚至还有一些恐惧。

毕竟之前,许多正道势力都刁难过武家,有几家甚至公开侵吞武家在外的资源点。

若是知道武庸的战力,还有武家的第四座玉清滴风小竹楼,他们就绝对会收敛,不会如此肆无忌惮。

“武家失去了许多地盘,武庸如此实力,绝对会秋后算账。”

“剿除影宗残余之后,恐怕就是武庸动手的时候了。”

“唉!我们都看错了。这个武庸心思深沉,完全和武独秀不同,他竟然藏的这么深。现在回想一下,恐怕这就是他的计谋。先示弱,然后占据大义,反击回去,获得更多的收益。”

行进的途中,南疆蛊仙们相互之间,不断地进行交流。

一种关于武庸的阴谋论调,很快就喧嚣尘上。

至于方源等影宗余孽,紫山真君已经灭亡,就连幽魂本体都被天庭俘虏,能够入南疆正道一干蛊仙眼界的,也就是方源的那头上极天鹰了。

南疆正道此次动用的仙蛊屋,多达六座。还怕什么上极天鹰?

掠影地沟。

“嗯!”白凝冰忽然轻嗯一声,脸上的痛楚神色一闪即逝。

演练四通八达,再一次失败了。

白凝冰受到了反噬,最为严重。不过她很快,脸色就平复下来。

人如故仙蛊!

方源连续催动这只宙道仙蛊,为自己和他人治疗。

再次享受人如故带来的疗效,黑楼兰的眼底深处,闪过一抹复杂神色。

人如故仙蛊仍旧健在,甚至她自己也再次和方源处于同一阵营,可惜的是太白云生已经陨落了。

还是被方源亲手杀死的。

演练暂停下来。

方源对自己用了人如故,伤势已经彻底痊愈,他开始思考和总结刚刚失败的原因。

至于其他人,还是在疗伤。

人如故虽然对她们也都用了,但是效果却远远不如方源。

至尊仙体的道痕不互斥,这带给方源防御上的一些薄弱,但同时也赋予了他疗伤方面的良效。

大家的伤势并不严重,再加上各自的真传都是上佳,很快就都痊愈。

“好了,我们再来一次。”方源呼吁。

上古战阵――四通八达。

从每个人的身上,都散发出光辉。光辉很快链接成一体,但却开始晃动起来,失去了原先的稳定状态。

砰。

一声轻响,光芒散去,化为无数点点萤火,漫天飞舞。

妙音仙子嘤咛一声,满脸潮红之色,这一次轮到她伤势最重。

方源反而没有什么伤,于是催动人如故仙蛊,首先治疗妙音仙子。

“我们的动作要一致。再来。”片刻之后,方源拍拍手。

上古战阵――四通八达!

再一次失败。

黑楼兰闷哼一声,眉头轻皱,美貌的容颜上闪过一丝痛楚之色。

休整之后,方源沉吟道:“我们不妨调整一下,你们都随着我的动作,感受我的力道,然后跟随运动。明白了吗?”

三女都微微点头,默不作声。

片刻之后,虽然方源再一次尝到了失败的苦果,但是进展却较之前要大得多。

“很好。”方源眼中精芒一闪。

上古战阵四通八达要成功催动,关键是四方配合要紧密团结,步调一致。

经过前期的煎熬和努力,他们终于渐渐找到了感觉。

不过就在这时,大阵陡然发出了轰鸣之声。

“不妙,我们的位置已经暴露,地沟上方,有好几座仙蛊屋!”负责操纵蛊阵的一位纯梦求真体,赶过来汇报道。

一时间,众女仙的目光都集中在方源的身上。

方源是影宗之主,她们都在等候方源的决断。

“速度倒挺快,不过也无妨。”方源微微而笑,“这个情形,正在我的意料当中。按照我安排下去的计划行事罢。”

“是。”纯梦求真体当即退下。

“我们继续。”方源对众女道,“仙蛊屋已经封锁了掠影地沟,时间有限,我们要加紧了。”

玉清滴风小竹楼中,武庸透过窗棂,居高临下地俯视掠影地沟。

他观察了片刻,眼中精芒一闪:“这座超级蛊阵倒是有些玄妙。布阵之人,利用了地沟中的天然道痕,仅仅付出大量凡蛊,就搭建出了仙级蛊阵。”

“还有影怪。好家伙,这规模庞大,全都被这座超级蛊阵聚集在了一块。”乔志材站在武庸的身旁,惊叹一声道。

这个时候,从池家的仙蛊屋中传来池曲由的声音,他见猎心喜:“好阵,好阵,不如让老夫来算出此阵的弱点,大家依此破阵。”

“不必了。”武庸淡淡地回绝,“此次剿魔,要速战速决。区区蛊阵,算得了什么?纵然影怪再多数倍,也难挡我等仙蛊屋的锋芒。”

说到这里,武庸顿了顿,然后道:“诸位,随我齐攻。”

话音刚落,玉清滴风小竹楼便勇往直前,夹裹狂风,俯冲下去。

一路冲杀,玉清滴风小竹楼所向披靡,影怪死伤惨重,一片惊惶。

南疆蛊仙目睹战况,俱都惊容满面,玉清滴风小竹楼的威能大大超出他们的意料。

“这就是武家之威啊!”许多蛊仙心生感慨,仿佛看到了曾经武独秀大杀四方的情景。

武庸到底是武家血脉,勇悍的性情早已经深入血脉和骨髓深处。

一时间,南疆正道蛊仙们的心头,都涌上了一股热意。

“杀!”

“一起冲。”

“杀光这些魔道贼子!!”

数座仙蛊屋如下山猛虎,横冲直撞,威势不可抵挡,重重影怪对比之下,竟孱弱得仿佛懦弱羊群。

喊杀声早已经传入方源耳膜,声音越来越大。

妙音仙子神情微异。

黑楼兰扫视方源几眼,却见方源毫无所动。

白凝冰仍旧是一脸冷漠。

超级蛊阵难挡数座仙蛊屋的猛烈攻势,很快就宣告瓦解。

影怪最高修为,不过七转,早已经被杀得四处溃散而逃。

就在南疆正道蛊仙要一举高歌猛进,直捣黄龙的时候,纯梦求真体们站了出来。

砰砰砰……

一连串的自爆,梦境瞬间形成。

“又是这招,快退!”翼家蛊仙吃过这个亏,海角阁嗖的一下,就退出老远。

其余仙蛊屋也不得不停下冲势,暂时驻足观望。

梦境形成一个全方位的防护,将方源等人层层包裹当中。

武庸等人神情复杂,这样的防线,他们如何能突破?

就是欺负你们都是梦道的白痴!

来势汹汹的正道蛊仙们,都有些傻眼。

武庸忽然轻笑起来:“影宗若是如此防护,已经是计穷了。他们挡住我们,我们也困住他们。我倒要看看他们能困守到什么时候?”

双方僵持住。

就在这时,武庸忽然神色微变,他接到了一只信蛊。

天庭来信!

备注:大纲发现还有些问题存在,今天只能一更了。唉,今晚打算通宵,紧急修改一下。(未完待续。)

\end{this_body}

