\newsection{仙尊悔而我不悔!}    %第六百六十一节:仙尊悔而我不悔!

\begin{this_body}

“你说什么?”龙公没有听清。??? ? 火然?文 ?? ???.?r?a?n??e?n`

“我说,我后悔了。”洪亭再次说道。

龙公眉头顿时紧紧皱起,用严厉的眼神盯住眼前这位他此生最得意的徒弟:“后悔?洪亭啊,不要再说什么胡话。你心里清楚,为了栽培你,将你扶持到现在的修为,在你身边有多少人奉献,有多少人牺牲吗?你说这样的话,对得起从小到大给你资助的那些人吗?对得起你的父母吗?对得起刚刚逝去的柳淑仙吗?对得起帮助你渡劫,陨落的诸仙吗?”

龙公对洪亭严厉喝斥。

“好了,龙公大人,淑仙仙子陨落这是多么大的悲痛,我们完全能够理解红莲仙尊的感受。他只是一时情绪激动罢了。”有其他的天庭蛊仙劝解道。

“红莲仙尊……呵呵。”洪亭不屑地笑了笑,他用通红的眼眸注视龙公,“如果要这样子牺牲,将来或许还有更多的牺牲,才能换来仙尊之位的话,那我宁愿不要这个仙尊之位!”

“够了!”龙公勃然大怒,“胡言乱语也要适可而止,洪亭!你以为你的仙尊之位,是想要就要,想舍就舍的吗?这一切都是宿命的安排,这是你天生的使命,你必须得履行!”

洪亭仰头大笑,他批头散发,狼狈不堪,随后他低头看向龙公:“堂堂仙尊,无敌天下,居然也不能随心所欲,就连不想当这个仙尊之位,都不可以?”

龙公冷哼一声:“你告诉我,洪亭,这个世间有谁能随心所欲?你的想法太幼稚了,你以为成为仙尊,你以为领袖正道,就没有牺牲了吗?这世间哪一件事情你不需要付出?你以为正道这两个字是那么浅薄的么?错!维护天庭正道,是最需要付出,最需要牺牲。如果你连这点牺牲的精神都不具备,那么我告诉你,你连加入天庭的资格都没有!”

洪亭被说得浑身一震。

“龙公大人、仙尊大人,你们都消消气,现在还是养伤,不宜争吵啊。”其余蛊仙连忙相劝。

洪亭缓缓垂首,但双拳却捏紧起来,以一种无比坚定的语气道:“我要复活他们。”

龙公眉头扬起,脸色宛若寒冰:“你要复活谁?”

“一切因为我的仙尊之位而牺牲的人。我的父母、柳淑仙,还有许许多多的人。”

“放肆!”龙公陡然咆哮一声,手指着洪亭,“你怎么会有如此大逆不道的想法?!你明明知道,我也不止一次地告诫过你,生死有命,人的生死是大道规律,任何一个生命的生死,都是宿命的巧妙安排。你想复活死者?你是想令整个宇宙混乱吗?你没有接受到教训吗?你越是这样胡来,你身上就会发生更加惨重的悲剧!”

“即便是再惨重的悲剧,我都选择承受!师父,徒儿一直有一个问题,为什么?为什么我们一定要接受宿命的安排?如果没有宿命,这个世界就真的会混乱吗?难道就没有变得更好的可能?”洪亭语气急促,连番发问。

龙公气得浑身直抖,这一次就连劝解的诸多天庭蛊仙,也都缓缓后退,用陌生的眼神看向洪亭。

“仙尊大人,你的这个想法太危险了。”

“是啊,没有宿命的话,就没有我们人族的昌盛啊。”

“宿命蛊乃是天庭至宝,仙尊大人居然想要毁灭它?这这这……”

“你们……”洪亭看着周围蛊仙,在这一瞬间,他感受到了难以言说的深邃孤独。

时间流逝,也不知过了多久。

光阴长河中,出现一个孤独落寞的身影。

他披头散发,满脸沧桑之色,仿佛流浪天涯,无处为家的浪子。

他的脸上依稀有着曾经的洪亭的影子,但神态早已发生了彻底的转变。

红莲魔尊望着滔滔不绝的光阴长河,幽幽叹息一声:“是时候留下传承了。”

他缓缓下降,脚尖就要触及到光阴长河的河面时,这才缓缓悬停住:“这第一座石莲岛,也是最重要的石莲岛,必以悔蛊为核心,留给天外之魔,在未来最有希望摧毁宿命的人!”

于是,一座石莲岛逐渐诞生而出。

红莲魔尊在这里留下了八转悔蛊,还留下大量的宙道仙材、仙道杀招,一股真意,包含了自己的一段最真实的记忆,还有一记威能奇绝的仙道杀招。

完成之后,红莲魔尊拖着疲惫至极的身躯,缓缓飞走。

时间匆匆,又过了许久许久。

这座石莲岛上闯进一个外人。

他粗布麻衣,身材单薄,头顶无发,面容普通。他赤着脚,脚底和裤腿上还沾着一些泥泞,仿佛是刚刚从田地里劳作归来的农民。

不过他的面庞十分年轻,一双眼睛熠熠生辉,充满了悲悯和仁慈。

红莲真意浮现而出,凝聚成红莲魔尊的虚幻样子,对泥脚少年微笑道:“贵客终于临门了。”

“在下乐土,费劲千辛万苦,终于寻到此地。”

此人正是已成就仙尊的乐土!

红莲真意直接问:“仙尊前来何意?”

乐土微微一笑:“红莲前辈何必明知故问呢?”

红莲真意也笑:“你说的没错,宿命只是损坏没有被彻底毁灭,所以一切都还有迹可循,从光阴长河中就可观测而出。”

顿了一顿,红莲真意再道:“悔蛊可以借你一用,至于你想抢夺我的继承人,那就得看你自己的本事了。”

乐土仙尊的脸上涌现出郑重之色:“他虽是天外之魔,但并非无情之人。放下屠刀,回头是岸,谁不想得到救赎呢?我愿意一试!”

红莲真意哈哈大笑:“也好。我本体建设此岛时,还有一些担心。毕竟天庭存在,没有十足把握。不过此刻既然有你出手,那便妥当了。这是悔蛊,你可接好。”

悔蛊缓缓飞出,乐土仙尊伸出光晕笼罩的双手,小心翼翼地接过来。

“不愧是悔蛊,能时刻传播、勾动出无穷无尽的后悔之情。唯有心中没有一丁点后悔的人,才能免疫这项最大弊端。除此之外,任何蛊仙靠近它,都会感受到无以伦比的悔痛!”乐土仙尊苦笑,“若非我此时已有九转修为,恐怕还真镇压不住。”

“看来你心中也有悔痛。”红莲真意叹息。

乐土仙尊轻声一笑:“谁的一生没有一件令其后悔的事?”

将悔蛊收入仙窍,乐土仙尊对红莲真意郑重一礼:“告辞了,我必定会将他带来此处。”

……

方源睁开双眼,恍惚之色从他的眼眸中徐徐散尽。

他阅尽了红莲魔尊留在此处的记忆,明白了一切前因后果。

“没想到居然直接到了红莲岛。红莲魔尊、乐土仙尊虽已陨落,但两位九转大能的布置,仍旧起了作用。”

方源叹息一声,看向红莲真意,还有漂浮在自己面前的八转悔蛊。

红莲真意笑道:“方源啊,我终于等到了你。一切都无须我赘述了,我也相信你会拼尽全力去摧毁宿命蛊的。所以,接受我这份真传吧,它已经在这里等待了你一百多万年!”

方源点点头,伸出手来,就要把握悔蛊。

红莲真意连忙道:“小心,悔蛊不可直接接触,否则哪怕你炼化了此蛊,也会被无穷无尽的后悔之情淹没!我已为你准备了手段,你只需……”

“没有这个必要。”方源面色淡然,一把抓住悔蛊,顷刻炼化。

红莲真意满脸惊愕之色:“你……”

ps:明天暂缓更新,需要整理一下大纲。这本书不同其他,线索太多,人物太多,坑太多,偏偏我还想写得更精彩一些,所以必须稳扎稳打,敬请诸位朋友理解和包涵。

\end{this_body}

