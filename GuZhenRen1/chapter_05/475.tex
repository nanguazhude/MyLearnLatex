\newsection{休整}    %第四百七十六节:休整

\begin{this_body}

仙道蛊阵当中,方源闭目盘坐在半空中,正在潜修。[看本书最新章节请到棉花糖小说网www.mianhuatang.cc]

黑楼兰等人连同唐方明,掌控着这座仙阵。

方源在心底总结得失。

“第一次探索盗天梦境,收获巨大。不仅试验了天消意散杀招,获得了圆满成果,而且还令自身的偷道境界一跃成为宗师级数!”

“不过,我为此也付出了巨大代价。魂魄原本有千万人魂级数,但现在却跌落到了百万人魂,损失很大。”

“原本以为,解梦杀招会用的相当频繁,没想到在这片梦境当中,解梦杀招并没有带给我多大的帮助。”

从这一次梦境探索,方源明白了一点。

那就是,他虽然掌握着解梦杀招,但这个手段并不是无往而不利的。

“虽然解梦杀招运用很少,仙元消耗变少,但是胆识蛊的消耗却极大的增长了!”

想到这里,方源缓缓睁开双眼,取出一只只的胆识蛊,当即用了。

“这么多的胆识蛊!”唐方明顿时双眼一瞪,流露出羡慕渴望的神情。

他探索梦境也有很多次,深知胆识蛊的优异。他苦苦寻购胆识蛊,总是供不应求,但现在却看到方源用这胆识蛊,像是吃蚕豆那般随意。

t

“到底他是胆识蛊的卖家啊。”唐方明心中感慨不已。

同时,他又将目光集中在了盗天梦境上。

盗天梦境巨大如山,仍旧散发着幽蓝的光晕,不过此刻山脚下的一大块,却是突兀的消失了。

这便是方源之前,探索盗天梦境,成功闯过了第一幕所致。

唐方明叹息一声。

他探索、研究这块盗天梦境,早已经数百次,乃至上千次,但他积累的成果,还比不上方源一次探索的零头。

唐方明乃是唐家公认的探索先锋,探索梦境颇有成果,自创出许多梦道凡蛊来。但现在和方源这功绩一比,顿时小巫见大巫。<strong>最新章节全文阅读Qiushu.cc</strong>

这让唐方明自惭形愧之余,更增添对方源的钦佩之情。

片刻之后,方源此时睁开双眼。

一抹精芒在他漆黑的眸底,一闪即逝。

胆识蛊不计损耗的动用,方源的魂魄底蕴再次上涨到千万人魂级数。

本来,他就是九百万人魂,这一次经过梦境消磨,魂魄重新上涨,隐隐有一种更加凝实的感觉。

“看来梦境消融魂魄,对于魂魄修行,也有些许的辅助效用。不过真正细究起来,效果不大,远不如落魄谷。”

荡魂山、落魄谷乃是幽魂魔尊都推举的魂修两大圣地,非是一般存在能够替代的。

但没有关系,方源这一次来到西漠,落魄谷就带在身上!

他此次出行,先东海,再西漠,考虑的非常周详。依照他老谋深算的性情,怎可能不做完全的准备?

“至于荡魂山,反正有宝黄天输送胆识蛊,带不带在身边,都一样。”

其实也不是方源不想带。

而是实在对仙窍中的天地二气,消耗太过剧烈。

如今,他的至尊仙窍中有逆流河、落魄谷、市井,平均每隔三天,就要吞噬大量的外来天地二气,弥补自家损耗。

若是不及时弥补,那么至尊仙窍就会底蕴削弱,造成这种资源减产,包括宇宙两道。严重的情况甚至会导致至尊仙窍破裂破碎。

“盗天梦境探索难度极大,第一幕已经如此艰难,第二幕必定难上加难。”有这样的觉悟,方源决定先将自身的魂魄底蕴提升更高一层,再去探索梦境,这样做比较稳妥。

于是,接下来的这段时间里,他就专心致志地潜修魂道。

因为胆识蛊不间断的供应,方源的魂魄底蕴再次飙升起来。

就在方源潜修魂道,为第二次探索做准备的时候。

东海,青岳至诚已经找到青岳家的太上大长老,寻求帮助。

“至诚,看你的样子,是决心要为那秦百合出头了。”太上大长老看着眼前这位,自己颇为喜爱的血脉后辈,用叹息的声音道。

“是的,小子心意已定,还请太上大长老出手相助!”青岳至诚一脸坚毅之色。

但太上大长老却缓缓摇头:“这件事情,还远不到我出手的时候。并且真相未明,我就算出手带给你的帮助也并不大。”

青岳至诚就很奇怪,因为太上大长老乃是堂堂八转蛊仙,怎可能对此事帮助不大?

但随即太上大长老告诉他的一番话,让他恍然大悟。

“小子恭谢太上大长老的指点,这就去寻求帮助了。”青岳至诚大喜。

“去吧,去吧。”太上大长老呵呵笑着,摆摆手。

青岳至诚雷厉风行,立即出了家族大本营,直往南宫一族的领海飞去。

原来青岳家的太上大长老指点他,说现在秦百合和尤婵是生是死都不清楚,她们遭遇了什么也不明了,最应该做的就是寻找一位智道蛊仙,来帮忙探查出真相。只有知晓了她们身上究竟发生了什么,才能最准确地查找到凶手,以做复仇。

而最适合的智道蛊仙,不是旁人,正是当今智道三杰之一,藏身在南宫家的智道蛊仙华安。此人和尤婵关系紧密,尤婵出事,他绝对会出手相助的。

青岳至诚听了太上大长老的分析,深以为然,立即前往寻求华安帮助。

但就在半路上,他便遭遇了南宫家的一位六转蛊仙。

这位蛊仙带来华安的一番话:“至诚公子,我家华安大人早已算出你要来找他,所以托我前来迎接你。他就在前面不远处。”

青岳至诚听了这话,顿时喜出望外,同时叹服华安之能。

片刻之后,两人照面。

华安道:“实不相瞒,尤婵仙子不久之前,就寻求过我的帮助,可惜当时我爱莫能助。”

华安和青岳至诚相互交流了一番情报,很自然的,他们就将首要怀疑的目标,定在方源的身上。

因为尤婵一死,方源贩卖龙鱼,最是得益。

当然,他们没有天庭的指点,只知道宝黄天中贩卖龙鱼的神秘蛊仙,最有嫌疑和犯罪的动机,但却不知道这个人就是方源。

“我前几日不断推算,尤婵和秦百合两位仙子,恐怕凶多吉少。”华安一脸沉重,“这一次我要前去寻求另外两杰出手相助,结合我们三人之力,再做尝试,推算出幕后凶手。”

华安的话,却让青岳至诚连连摇头。

“不,我相信两位仙子都还活着,她们并没有死!华安大人,你仔细想啊,两位仙子都是七转好手,谁能轻易地杀死她们呢?”

华安理解青岳至诚的感受和心情,安慰性的点点头:“你说的也有道理。”

然后话锋一转,他又道:“这样,我有一计,我们不妨大张旗鼓,将两位仙子的死讯传播开去。若是她们未死,必定知晓我们再寻找她们,想要帮助她们。若是她们正遭受凶手的追杀,这个谣言散发出来,也能混淆凶手视听,将那凶手的注意力转移到我们这边来。”

“此计大妙!”青岳至诚不禁双眼冒光,抚掌赞叹。

时间一晃,就是半月过去。

方源这里魂魄底蕴,暴涨到了九千万人魂级,距离亿人魂只差半步之遥。

这段时间,他没有锻炼什么仙道杀招,而是将主要的精力和时间,都用来魂道修行。如此一来,自然在成果上,比此前更加巨大。

虽然胆识蛊不断消耗,不仅没有给方源带来巨大盈利,反而在耗损方源的资本。但是方源的经济状况,却是在慢慢好转了。

造成这个原因的,就是龙鱼生意。

方源宰了尤婵、秦百合,天庭方面就算想要阻止,一时间也是有心无力。

宝黄天市场中,单就龙鱼生意这块,方源已然是第一霸主,无人可敌。

龙鱼生意因此越加红火,尤其是银龙鱼引起疯购,单价抬高得很是惊人。

“值得注意的是,东海那边,尤婵、秦百合的死讯已经传遍了,看来是有心人在幕后推波助澜。”

“我且不管它,当下重点还是盗天梦境。”

有着九千万人魂,方源也有了相当大的底气,再休整了一天之后,他再次魂入梦境,二探盗天迷梦。

备注:今天就一更,状态着实不佳,唉,太累了,心力憔悴。(未完待续。)<!--80txt.com-ouoou-->

\end{this_body}

