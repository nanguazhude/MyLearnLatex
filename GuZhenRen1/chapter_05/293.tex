\newsection{龙鳞海域}    %第二百九十三节:龙鳞海域

\begin{this_body}

%1
方源接过一看,清单上林林种种有不少蛊材,当然绝大多数都是凡道蛊材。仙材虽有,但很少。

%2
方源心知肚明,若是他交还出寒玉和玉髓等等,这份奖励就绝不只这么多。

%3
武庸和武罚其实在演一场戏,倒不是算计方源,而是想给方源一个台阶下来。

%4
方源挖掘广寒峰,虽然没有超过底线,但武家还是宁愿有一个原来的广寒峰。

%5
但方源却失踪没有交出来,他沉默了一会儿,这才道:“不知可否将这些奖励,都换成一种凡道蛊材龙鳞土呢?”

%6
武庸微微一愣,看向武罚。

%7
后者思考了一下,点点头。

%8
回到自己的住处,方源继续潜修。

%9
这一次收获了很多寒玉,还有玉髓,方源几乎可以在至尊仙窍的小南疆里,再造一个小型广寒峰了。

%10
不过,这不是重点。

%11
重点在于龙鳞土。

%12
这种奇妙的土壤,只是普通蛊材,颗颗结粒,土质坚硬。若是一层龙鳞土扑在地上,远望的话,宛若龙鳞层层叠叠。

%13
龙鳞土乃是龙类荒兽、上古荒龙,甚至太古荒龙常年居所地方的泥土,因为沾染了龙气从而渐渐演变形成。

%14
方源早年的时候,就收购过这种土壤。原因是有了这种土壤之后,龙鱼会有更多的繁衍冲动。

%15
现在他不要其他奖励,专门换取这份蛊材,也是打的一样的主意。

%16
“至尊仙窍时间延缓了许多倍,这也导致仙窍中的资源产出,减少了很多。”

%17
“如今仙元储备不足,仙元石不足,只要扩大资源的生产规模,就能缓解这方面的压力。”

%18
经过方源的考量,他把第一个扩大规模的资源点,定为龙鱼。

%19
荒兽龙鱼,还有普通龙鱼,一直以来,都是宝黄天中的大宗交易。

%20
龙鱼本身似乎是食道的创造,对于很多的蛊虫,都可以取代一部分,进行喂养。

%21
并非每一位蛊仙,都拥有仙蛊。

%22
但是每一个超级势力,每隔一段时间都会收购大量的龙鱼。

%23
这就是宝黄天市场上,龙鱼贩卖最大的流向。

%24
接下来的时间,方源就主要建设海域。

%25
如今,在至尊仙窍中的小东海里,已经不再是当初的浅滩或者水泊了。因为方源吞并了大量的水道仙窍,小东海中已经算得上波涛起伏,虽然不是很深邃,但是从表面上,也是汪洋一片。

%26
海岛几乎是没有的,海域本身也毫无区别。

%27
方源的计划是,打算建设出一个龙鳞海域。

%28
人造海域!

%29
这并不是方源的首创,也不是他的独创。事实上,在东海蛊仙界中,早就有许多蛊仙,进行人造海域。

%30
如今东海蛊仙界中的排行榜上,很多著名的海域,都是人造海域,一步步经营建设出来的。

%31
当然,自然形成的海域,仍旧占据榜单的大部分位置。

%32
人为的建造一片海域,自然并不轻松。

%33
方源花费了大量的精力和时间,铺设龙鳞土,然后又在当中建设凡道蛊阵,利用蛊阵的力量,将这片海域隔绝成一个特殊海域。

%34
然后,他才将原先的龙鱼群,都引进到海域当中去。

%35
龙鱼群进入这片海域,果然比之前更加惬意,交配的行为也增长许多。

%36
不过,这还不够。

%37
接下来,方源开始酝酿仙道杀招丰年!

%38
这个仙道杀招,来源于黑凡真传。但事实上,并非黑凡所创,而是来自一位号称丹仙的蛊仙大能。

%39
是他创造了这个仙道杀招,然后黑凡用了其他资源,和他换取过来。

%40
这是一个相当罕见的,专门用来经营仙窍的仙道杀招,没有任何的攻伐威能。

%41
不过实用性非常的强,就连黑凡也在其真传内容中,留下批语,他对这个仙道杀招赞誉有加。

%42
一只只蛊虫陆续升空,玄妙的气息不断膨胀,逐渐浓郁起来。

%43
方源现在催动的丰年,已经得到了他的一些改良。

%44
原版的丰年,核心仙蛊是年蛊,辅助蛊虫有数万只。但是现在,方源新得了一只七转仙蛊坚持。方源巧妙地将坚持仙蛊添加进去,作为第一辅助蛊虫。

%45
十多天后,丰年杀招酝酿完毕,彻底催动。

%46
玄妙的威能,覆盖了整个龙鳞海域,但是收效如何,短短时间里看不出来。

%47
“原来的丰年杀招,就算黑凡出手,也需要催动两三天的时间。这已经够长了。我改良之后,增添了坚持仙蛊进去,时间变得更长,几乎用了半个月的时间。幸好催动成功,没有失败受到反噬。”

%48
方源不断体悟、回味。

%49
“我的宙道境界还很普通,改良只是很勉强地将坚持仙蛊,添加进去。这便是时间极大延长的主因。”

%50
仙道杀招当然是催动的时间越短,中间的过程越简略,杀招成功催动的可能就越大,对于蛊仙而言,也就越安全。

%51
不过方源此举,是利大于弊。

%52
有了坚持仙蛊辅助,丰年持续的时间,将整整延长一倍!

%53
原版的丰年,只能影响仙窍时间中的一年。但现在方源只催动一次,却能影响两年。这也是变相的替方源节省了一部分红枣仙元。

%54
“丰年杀招还有大量的可用改良的空间。将坚持仙蛊真正融入进去,或者是将仙级日蛊也融合进来,这都能提升丰年杀招的效用。但是就我目前的境界而言,却是达到了宙道方面的能力极限了。也罢!”

%55
事情已成,这一次的仙窍建设,是方源有史以来,第一个大手笔。

%56
如今,在小东海中,第一个特殊的海域龙鳞海域,彻底成形。

%57
大量的龙鱼在这里栖息生存,龙鱼群中一两只荒兽龙鱼,分外显眼。

%58
丰年杀招的威能,不仅笼罩了整个龙鳞海,还囊括了这片海域周围的浩大范围。

%59
按照黑凡真传中所说:丰年杀招一经催动,虽然消耗的仙元极多,但能在之后一年内,大量增加整个仙窍中的各项资源产出。

%60
但事实上,并非如此。

%61
方源催动出来的丰年杀招,效果只覆盖了小东海中的一部分。

%62
一方面,这是因为方源新添了坚持仙蛊,减少了杀招的影响范围。另一方面,也是至尊仙窍太过广大,丰年杀招想要囊括整个至尊仙窍,无异于天方夜谭。

%63
从龙鳞湖,到龙鳞海,这是一个巨大的跨越,意义深远。

%64
但经此一事,方源手中的仙元石和仙元储备,就几乎要干涸见底。

%65
为了保险,方源将手头上不是必要的资源,都开始往外贩卖。

%66
就比如说,他在广寒峰中搜刮而得的寒玉和玉髓,以及大量的冰道凡蛊。

%67
有得必有失,他现在重点的经营项目,放在龙鱼身上。

%68
贩卖了手头上的一些资源,方源又拼尽全力,咬牙搜出一部分所得,又在宝黄天中收购了不少龙鱼。

%69
他将这些龙鱼,都放进龙鳞海域中去。

%70
可以说,他把重注,都压在了这个项目上。

%71
毫无疑问,这有些冒险。但若不温不火的话,反而效率更低。相比较而言,目前较为安稳的环境下,方源还是选择了激进一点,步子迈大一点。

%72
但这样一来,因为仙元储备太过稀少,方源显得束手束脚,不能再随意修行,参悟或者演练什么仙道杀招等等。

%73
日子显得有些难捱。

%74
半个月之后,第一批的龙鱼贩卖了出去,情况开始好转了。

%75
一个多月后,第二批龙鱼,方源直接卖给了武家!

%76
“没想到兄弟你经营龙鱼,是如此擅长啊!”就连武庸都有点惊讶。

%77
武家本来就要买卖龙鱼的,不过之前他们是从外面采购,如今算得上内销了。

%78
武家之外,还有乔家等一些,和武家关系紧密的超级家族。

%79
方源的第三批龙鱼,就都贩卖了过去。

%80
正是借助武遗海的身份,又以偏低的价格,方源的龙鱼顺利地挤进了南疆蛊仙界的市场当中去。

%81
第三批龙鱼贩卖出去后,方源情况大为好转,仙元储备蹭蹭上涨。方源脱离了这方面的困境,并且以一种全新的姿态,开始继续修行。

\end{this_body}

