\newsection{中洲蛊仙集结}    %第二百一十三节:中洲蛊仙集结

\begin{this_body}

%1
明天就是515,起点周年庆,福利最多的一天。除了礼包书包,这次的『515红包狂翻』肯定要看,红包哪有不抢的道理,定好闹钟昂\~{}

%2
中洲。

%3
某处山村。

%4
轰隆隆!

%5
巨大的轰鸣声,回荡在山间。

%6
无数的泥石流,宛若雪崩一般,从山上往下滚落,浩浩荡荡。

%7
“快逃命啊!”

%8
“我要死了,我要死了……”

%9
“爹,你在哪里,不要丢下孩儿啊!”

%10
眼见天灾降临,原本静谧祥和的小山村,已经彻底炸开了锅,乱了套。

%11
鸡飞狗跳,无数人四处奔逃,也有人瘫痪在地上,小儿啼哭,被母亲死死地抱在怀中,不少人已经放弃了抵抗。

%12
在无人注意的情况下,一个身影浮现在高空当中。

%13
这是一位七转蛊仙。

%14
他身着丝绸蓝袍,长发披肩,身材并不魁梧,反而有些柔弱的样子。

%15
此时,他皱起眉头,看着眼前的泥石流,口中呢喃:“这场泥石流,来得分外古怪。”

%16
一般而言,泥石流之前,都会有罕见的暴雨,但最近这段时间,一直都是风和日丽。

%17
事实上,这里方圆数万里的环境,都已经被这位七转蛊仙暗中调控,可谓风调雨顺,年年丰收。

%18
砰。

%19
山石忽然四处飞溅,从山中冒出了一个巨大的黄铜甲壳。

%20
七转蓝袍蛊仙悬浮在空中,看得分外明显。

%21
他微微皱起的眉头,此时舒展开来:“我道如何?原来是一头泥沼蟹。”

%22
泥沼蟹乃是荒兽,它有山一般的雄阔身躯。它没有眼睛,或者说眼睛已经完全退化,它浑身都被甲壳包裹,防御上毫无漏洞。

%23
它是荒兽中,泥沼里的君王。

%24
十对螯足,刚硬超凡,尤其是第一对螯足。分外粗壮,轻轻一夹,就能断山石,剪蛟龙!其余的十八只螯足。即便比较瘦长纤细。但实际上,也都比百年古木还要粗壮。

%25
七转蓝袍蛊仙见到这头泥沼蟹,双眼微微一亮,心中欢喜。

%26
他是水道蛊仙,而这头泥沼蟹的身上。却是拥有土道、水道两种道痕。斩杀掉这头泥沼蟹,对于七转蓝袍蛊仙而言,就是一大堆的可用仙材。

%27
“不过泥沼蟹只是荒兽,若是再加以精心的培养,或许能生长成为上古荒兽泥羹沼蟹。泥羹沼蟹之上,还有泥衣沼蟹。泥衣沼蟹乃是太古级荒兽,我倒是不用奢望了。我的仙窍福地,可培养不出这等太古荒兽。不过泥羹沼蟹,却是可以尝试一下。”

%28
想到这里,七转蓝袍蛊仙便开始动手。

%29
他从宽大的袖口中。伸出手臂。

%30
他的皮肤很白,十根手指也是纤细修长。

%31
他的十根手指晃动起来,仿佛莲花绽放,搅起一层层的斑斓光影。

%32
这是他特有的操纵蛊虫的方法!

%33
很快,他的身上就升腾起无数蛊虫的气息,有凡蛊气息,有仙蛊气息,一股股的气息相互叠加起来,形成四处散溢的复杂气场。

%34
山中的泥沼蟹,虽然没有双眼。但凭借野兽的直觉,敏锐地感觉到了来自半空中威胁。

%35
泥沼蟹开始往山体中缩去。

%36
巨大的黄铜甲壳,很快就消失了一半。

%37
但这个时候,七转蓝袍蛊仙已经酝酿完毕。仙道杀招催发而出!

%38
哗啦啦……

%39
澎湃汹涌的蓝色浪潮,凭空而生,浩浩荡荡地向泥沼蟹冲刷过去。

%40
泥沼蟹身躯颇大,躲闪不及,被浪潮卷席。

%41
但潮起潮落,泥沼蟹却岿然不动。它的身躯着实太过沉重。

%42
七转蓝袍蛊仙的嘴角处,却绽放出了胜券在握的微笑,他的十根手指晃动更疾,一根根手指的残影,停留在空气中,让人眼花缭乱。

%43
他的这记仙道杀招,并不平凡。

%44
随着红枣仙元的不断消耗,原本浅蓝色的浪潮,变成了蔚蓝之色,浪潮澎湃浩瀚,比之前的规模扩大了三倍有余。

%45
潮水在七转蓝袍蛊仙的操纵下,如臂使指,形成一片巨大的漩涡。

%46
泥沼蟹深陷蓝潮的漩涡中心,终于抵挡不住,被漩涡卷席。

%47
哗哗哗!

%48
泥沼蟹庞大沉重的身躯,在浪潮漩涡当中,越转越疾,好像是浮萍一般,完全身不由己。

%49
七转蓝袍蛊仙忽然双手一握,残留在空气中的无数手指光影,骤然消散,浪潮忽然高高涌起,形成滔天的海啸。

%50
轰!

%51
一声巨响,数万吨的水浪形成的海啸,重重地拍击在泥沼蟹的背上。

%52
泥沼蟹的背壳非常坚硬,但被这股海啸一拍,原本平整的背壳表面立即凹陷下去了一大块。

%53
泥沼蟹一动不动,被拍得当场昏死过去。

%54
七转蓝袍蛊仙朗笑一声,右手轻轻展开,食指由下至上,轻轻一提。

%55
顿时平复下来的浪潮中,涌起一股,宛若是喷泉一样,将泥沼蟹庞大的身躯,缓缓冲上空,升到七转蓝袍蛊仙的面前来。

%56
蛊仙打开自家的仙窍门户,将这头泥沼蟹收入囊中。

%57
“仙人!是仙人啊!!”

%58
“谢谢仙人,仙人救了我们全村的性命。”

%59
“这位仙人还将山中的蟹妖给击败了!”

%60
这样浩大的战斗情景,早已将山村中的凡人们看得震惊万分。

%61
直到战斗结束,他们这才反应过来,一个个狂喜呐喊,或者倒头便拜。

%62
七转蓝袍蛊仙关闭仙窍门户,用淡淡的目光扫视了下方人群一眼,他轻轻一笑。

%63
原来在他和泥沼蟹交手的过程中,他分成了一部分心神,操纵浪潮,将滚落下来的泥石流统统卷走了。

%64
“不愧是中洲有名的水道蛊仙沐凌澜。”这时,一个声音从更高空的云层中传来。

%65
七转蓝袍蛊仙沐凌澜十指气动,将漫溢山间的大水,悉数收起。

%66
然后他飞入云层高空,见到了另一位蛊仙。

%67
只见此仙身着白袍,国字脸,眉毛粗重。鼻梁高耸,一身正气凛然,不可侵犯的气质。

%68
沐凌澜笑着一礼:“原来是施阁前辈到了。”

%69
施阁还了一礼:“也是刚到,没想便有眼福。见到沐凌澜你轻取泥沼蟹的手段。”

%70
沐凌澜摆摆手,谦虚道:“我的手段,在前辈面前,只是小道,不足挂齿。听闻施阁前辈。也已接到天庭命令,参加此次北原之战。”

%71
施阁点点头:“没有错。沐凌澜你也在名单之中,我们不妨同行?”

%72
沐凌澜满脸欣然之色:“能与前辈同行,是晚辈的荣幸。”

%73
于是,双仙结伴而走。

%74
留下一地的凡人,还久久地仰望天空,兀自喟叹。

%75
沐凌澜、施阁一路交谈,倒不显得寂寞。

%76
两仙虽然都是七转修为,但施阁成名已久,已经度过二次浩劫。而沐凌澜却是一次都未渡过。所以,主要是沐凌澜请教,施阁指点。

%77
行了一段路程,施阁忽然降下云头,缓缓悬停在高空。

%78
沐凌澜不解,此处并非集合地点。

%79
施阁笑道:“惭愧,我有一子取名正义,最近刚刚渡劫,成就蛊仙。然而缺乏历练,还有许多小孩气。这一次北原之行。我打算将其带在身边,加以磨砺。”

%80
“原来如此。”沐凌澜恍然,当即顺着施阁的目光,注视下去。

%81
只见云层下方。有一小城。

%82
城中房屋不少,其中一座酒楼当中,一位说书人正讲行侠仗义的民间故事。

%83
“好!杀的好!”听客之中,一位少年郎,浓眉大眼,满目纯真。作农夫打扮。此时听到故事中的主角杀富济贫,不禁拍案叫好。

%84
他的喊声突兀,音量又大,一下子震得窗棂都微微颤动。

%85
说书人被吓了一跳,停顿下来。

%86
周围的听客都不满地嚷嚷道:“叫什么啊?”

%87
“忽然一大声,要吓死人哩。”

%88
“吵什么吵,安静听书,不然赶你走啊,小泥腿子。”

%89
少年郎满脸涨红,挠挠头,不好意思地四处作揖:“对不住,对不住了,各位。”

%90
听客们见他立即认错,态度谦恭,嚷嚷声顿时减弱了许多,不再过多计较。

%91
少年郎讪讪坐下,忽然神情一变,又腾的一下子站起来,把方桌和板凳都推到了一边,造成更大的响动。

%92
“又干嘛啊,你小子!”

%93
“臭小子,要找打!!”

%94
听客们群起而攻,忽然间,少年郎浑身冒光,身形如箭,射破窗棂,飞上过来高空。

%95
酒楼顿时沸腾,无数人惊呼大喊,一片嘈杂骚乱。

%96
少年郎来到施阁、沐凌澜两仙面前,抱拳施礼,恭恭敬敬。

%97
原来他正是施阁的儿子施正义。

%98
三仙继续启程,几日后,来到一处山脉。

%99
一座仙蛊屋已经悬停于此。

%100
它是一座亭阁,小巧精致,悬梁之上挂着无数鸟笼,各种鸟类在里面叽叽喳喳。

%101
正是天莲派的仙蛊屋揽雀阁。

%102
施阁见此,微微点头:“早已听闻风声,此次攻打北原,将出动仙蛊屋。没想到是天莲派的揽雀阁,此阁擅收飞禽,移速极快,的确是上佳之选。”

%103
沐凌澜附和道:“天莲派乃是元莲仙尊开创,拥有仙蛊屋最多,数量多达四座。天莲派能派遣出一座来,也是合乎常理之举。”

%104
施正义疑惑地问道:“天莲派拥有揽雀阁、岳阳宫、天池,怎么还有第四座仙蛊屋?”

%105
沐凌澜笑了笑:“小义有所不知,近些日子,天莲派已经开创出了第四座仙蛊屋,只是暂且秘而不宣罢了。”

%106
施正义哦了一声,心想:“沐凌澜前辈出自灵蝶谷,此派最是擅长信道手段,他知晓一些秘闻,也不奇怪。只是天莲派一派就拥有四座仙蛊屋,着实有些吓人呐。”

%107
ps:第二更会晚一些。

%108
ps. 5.15「起点」下红包雨了!中午12点开始每个小时抢一轮,一大波515红包就看运气了。你们都去抢,抢来的起点币继续来订阅我的章节啊!

\end{this_body}
\newsectionindepend{明天加更!}
\begin{this_body} \par
%109
明天加更!

%110
app上,有一个“作家荣登封面”的活动,蛊真人目前排行第17。

%111
这个位置,有点可惜,再进一个名次,就是质变的结果了。而且和前一名差距很小。

%112
在这里,拉一下票。

%113
大家能投一些赞赏票给我,就十分感谢啦。

%114
这个活动,很多读者朋友都一直在鼎力支持我。

%115
非常感动!

%116
谢谢大家了。

%117
这个活动快要截止了,让我们在最后关头拼一把。

%118
不管结果如何,我明天都要加更的。

%119
最近的这些章节,因为涉及到很多人物和情节,所以写起来非常艰难。

%120
一直在酝酿高潮。

%121
明天将是大雪山福地和中洲蛊仙之间的精彩对决,希望大家能够喜欢!

\end{this_body}

