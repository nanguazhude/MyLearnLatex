\newsection{光照菌生意}    %第四百三十九节:光照菌生意

\begin{this_body}

%1
茫茫东海,波涛翻滚。

%2
风浪冲击着远处一座无名的小岛,还未至傍晚,天色却已然黑了。

%3
女仙童画一边疾飞,一边抬头望去,只见头顶上正是乌云密布,一场暴风雨即将来临的征兆。

%4
她心中烦闷,但脸上平静无波,很快就飞临到无名小岛的上空,便见到一人,正站在岛上的最高峰处,眺望着海和天。

%5
“任仙友,有劳你久等了。”童画落到那蛊仙的身后,微微一礼道。

%6
神秘蛊仙缓缓转身,现出容貌。

%7
他是一位老者,高鼻细眼,此刻脸上透露出一股笑意,从容而谈:“无妨。还要恭喜童画仙子,听闻你在宝黄天中积压的光照菌存货,竟都贩卖了出去。”

%8
这位蛊仙老者,正是东海七转,奴道蛊仙任修平。

%9
童画仙子点点头,交给任修平一大批仙元石。

%10
任修平清点之后,故意迟疑道:“仙子,若是老夫清点的没错,这数目上还是稍差一些的。”

%11
童画仙子面色有些阴沉,她点了点头:“先归还你大半,剩下的一小部分,只好将来再还你。”

%12
任修平嗯了一声:“仙子此举,的确明智。按照你我定下的借约,欠债越少,利息便越少。看来仙子能还清欠债,指日可待了。”

%13
童画仙子冷哼一声:“那是当然。”

%14
她早些时间,渡过一场灾劫,仙窍重建,花费很多,不得不四处借款。任修平就是其中之一,且借给她仙元石最多。

%15
不过,她和任修平之间的借约,也是最为严苛,利息极高。

%16
原本,童画仙子处境已经岌岌可危,但没想到方源将她手中的光照菌存货都购买了去,导致她手中资金回笼。

%17
这一次将借款还清大半,原本的危机也就大大缓解了。

%18
剩下来的那部分仙元石欠款,对于童画仙子而言,偿还的压力并不大。

%19
双方交接完毕,童画仙子不愿多留,正要告辞,这时任修平却道:“仙子且慢。”

%20
“任仙友还有何事?若是还要再借我仙元石,恕我坦言,我是不会再借了。”童画仙子眉头轻蹙,语气冷淡。

%21
任修平笑道:“仙子放心,不是此事。而是我手中有一份残缺的真传,乃是来源于昔日一位木道大能。仙子不是要改造光照菌么?我相信这份真传的内容,会给仙子带来极大的帮助。”

%22
童画仙子顿时惊疑起来:“你如何知晓我正在尝试改良光照菌?”

%23
任修平呵呵一笑:“此事仙子也并未刻意隐瞒,老夫稍稍打探,便可得知了。”

%24
童画仙子满脸阴沉,宛若此刻的天色。

%25
哗哗哗。

%26
大风吹来,卷起无数浪涛,拍打在无名小岛沿岸,激起滔天的浪花。

%27
任修平说的没错,童画正是因为尝试改良光照菌,在这个方面投入太大,导致自家经济崩溃,四处借款,才能勉强维持。

%28
若非如此,依凭她原本的底蕴,还是可以支撑仙窍的再建。

%29
沉默了一会儿工夫,童画仙子再次开口:“你所说的,究竟是哪位木道大能?”

%30
任修平笑道:“木道蛊仙魁梧子,仙子听闻过吗?”

%31
童画仙子微微一惊:“他的确是木道大能,有八转的修为。可是历史上,他以战力而出名,自身仙窍的经营功夫,却并不佳呀。”

%32
任修平微微摇头,然后以非常肯定的语气道:“仙子此言差矣。你之所以有这样的印象,乃是因为魁梧子的战力实在过于耀眼,遮盖了他经营仙窍的功夫。实际上,他并不差。若无仙窍底蕴支撑,再强的战力,也只是朝露片刻罢了。再者,就算他经营功夫欠佳,到底也是八转级数,又是木道蛊仙,对于仙子而言,帮助会很大。这一点,毋庸置疑。”

%33
童画怦然心动,不禁暗想:“我此前改良光照菌,一直受阻,原因不在于我的光道境界,而是木道方面,我造诣不深。若是得此残缺真传,极可能为我提供帮助,说不定光照菌的改良就能成功了!”

%34
想到这里,童画迟疑了一下,问任修平道:“这魁梧子真传,是什么价码?”

%35
任修平心道一声“鱼儿上钩了”,口中则道:“不高,不高。”

%36
双方一阵讨价还价,最终童画仙子从任修平手中,购得魁梧子的真传,离开了无名小岛。

%37
她刚刚离开,天空中便下起了暴雨。

%38
海天一色,尽皆晦暗。

%39
咔嚓。

%40
一声雷霆巨响,巨蛇一般的电光,猛地闪烁在天地之间。

%41
在这一瞬间,电光照亮了任修平的容颜。

%42
他的脸上,布满笑意,正是那种阴谋得逞的笑。

%43
“此计成功了。”

%44
“童画已经被魁梧子的名声哄住,其实他的真传,并不能带给她想象中的帮助。尤其是在我故意删减了许多关键内容之后。”

%45
“光照菌乃是荒植,要改良一种荒植,岂是那么容易的事情?尤其是童画并非木道蛊仙,而是专修光道,没有个数年时间的研究,怎么可能成功?”

%46
北原,琅琊福地。

%47
方源的神念投入到至尊仙窍当中。

%48
仙道杀招——菌光普照!

%49
方源催起仙道杀招,顿时营造出一股璀璨的橙黄华光。

%50
橙光照射在面前的嫩绿色的光照菌上,光照菌开始变化。以肉眼可见的速度,原本嫩绿的光照菌,像是被颜料浸染,开始渐渐地转变颜色,从嫩绿色变成纯粹的白色。

%51
片刻之后,仙道杀招催动完成,方源眼前只剩下一片纯白的光照菌。

%52
这些光照菌和之前大不相同,一扫之前萎靡不振的状态,散发出强烈的光辉。

%53
方源动用侦查手段,细细地检查了一遍后,满意地点点头。

%54
“很好,如此一来,光照菌改良相当成功,可以大面积地种植了。”

%55
方源的木道造诣,比童画仙子还要差一些。但是他却有着智慧光晕,可以辅助思考。而影宗的传承、琅琊派传承中,海量的资料给与他最坚实的理论基础。并且还有大量的仙蛊握在手中,或者可以借得来。

%56
种种原因,令他一蹴而就,直接设想出了仙道杀招菌光普照。

%57
此招以坚持仙蛊、自爱仙蛊、木芽仙蛊为核心,小吃仙蛊为辅助,能够令嫩绿的光照菌,彻底转变改良,成为适合天空环境的纯白光照菌。

%58
这其中,木芽仙蛊是方源向琅琊派借的。坚持仙蛊、自爱仙蛊,方源本来就有。至于小吃仙蛊,刚刚炼成,能够立即用在这里,也算是一个小小的惊喜。

%59
随后,方源将这些光照菌,分别安放在小橙天、小白天当中。

%60
小橙天中,有着一片极光,烂漫多姿。极光之中,孕育着果实,正是流光果。

%61
而小白天中,则有一片小型的霸王花花场。

%62
光照菌放置上去后,顿时给两个地方,带来了强烈的关照。

%63
受此影响,小橙天中的极光,立即开始慢慢地膨胀起来。虽然程度有些轻微,但日积月累之下,增长的量不可小觑。

%64
而小白天中,原本耸搭着花瓣的霸王花,似乎也精神了起来。

%65
不管是极光,还是霸王花,都需要强烈的光照。改良后的纯白光照菌,能自行悬浮在空中不动,对这两处极有帮助。

%66
方源观察了一阵,心中相当满意。

%67
“只要这两处地方建设好了,按照这种进度发展下去,那就能大概满足态度蛊、慧剑蛊的食料问题了。”

%68
方源一共有三只八转仙蛊,其中似水流年蛊只是吞食光阴河水,喂养不是问题。态度蛊、慧剑蛊就难了。

%69
一直以来,方源都为此焦虑担忧,现在终于看到了真正解决的希望。

%70
“并且,光照菌还可以自我繁衍,将来我的光照菌规模将会越来越大。搞不好还能做成一项出口的生意。”

%71
这是很有可能的。

%72
方源改良过的纯白光照菌,相当有市场,比原版的嫩绿光照菌要优秀得多。一旦投放市场,必定能引来好评和收购的热潮。

%73
方源已经计划,这两种光照菌都要培养,嫩绿色的光照菌他投放一些到小东海中去,纯白光照菌则是重点培养。先满足自己的需求,然后再对外销售。

%74
“计算一下。”

%75
“光照菌要形成规模,满足自需,需要投入两百多万仙元石。”

%76
“要对外销售,做成大生意,还需要再投入一百八十多万块的仙元石。”

%77
主要是喂养八转仙蛊,负担太重。

%78
前前后后,就要将近四百万块的仙元石了。不过如果建设成功的话,那就是一笔垄断性质的贸易,非常赚钱。

%79
年蛊这两三年里,方源是完全指望不上的。甚至是四五年后,宝黄天的年蛊市场都会非常萧条。因为蛊仙手中囤积了大量的年蛊,需求很少。

%80
这一次方源手笔很大,年蛊倾销,还导致了日蛊、月蛊销量被大大地影响。

%81
方源要开源拓本,光照菌是一个很好的项目。

%82
当然,它的前景还是比不上胆识蛊。因为改良后的纯白光照菌卖得久了,就会有人破解其中的奥妙,令自己也能转变改良。而胆识蛊只有拥有荡魂山,才能够获取。

%83
“接下来,就是彻底放开束缚,令至尊仙窍中的光阴支流恢复原状了!”

\end{this_body}

