\newsection{穿枝拂叶}    %第七百零四节:穿枝拂叶

\begin{this_body}

紫薇仙子的分析,并没有错。

因为有着宿命蛊,古代的蛊仙不能穿梭到未来,未来的蛊仙也不会能到现在。

但伴随着宿命蛊的损伤,导致宙道方面的束缚松动了许多。春秋蝉就是最好的例子,虽然不能令蛊仙完整地回到过去,但是却能够将一股意志传输回去。

未来身、前有古人、后有来者这些杀招也是如此,是借用了过去或者将来,某个人物的状态,投射到现在来。

当宿命蛊彻底修复,也就是这些杀招失效之时。

紫薇仙子知晓了原因,冰塞川当然早就心知肚明。

但他没有办法。

炼道大阵他不能破坏,因为一旦破坏,天庭此次修复宿命蛊的成果就会毁于一旦。

过于残破的宿命蛊,就算是长生天一方得到,也没有什么用,还得效仿天庭,继续修复。

但冰塞川心知肚明,长生天方面可没有天庭这般雄厚的底蕴,没有后者的人道手段,更没有不败传承,所以根本就是无力修复宿命蛊的!

不能破坏炼道大阵,长生天一方就只能闯过大阵,进入监天塔,强行夺取宿命蛊了。

关于这个,冰塞川已经早早派遣出了五行**师以及牛魔。

但这已经是极限了。

不是他不想增兵,光阴长河的幻景中陆续不断地有北原强者走出来。

而是这座炼道大阵本身,是有极限的。

这座大阵的作用,是用来修复宿命蛊,并非用来战斗。里面形成的大阵空间,顶多只能承载五位八转而已。

再多的话,就要撑爆这座炼道大阵了。

所以,冰塞川不能再派遣他人增援,中洲方面也只能坐视不管,眼睁睁地看着。

围绕着劫运坛展开的攻伐大战,牵扯到无数的历史强者,打得极其惨烈。但则是炼道大阵中的战斗!

冰塞川咬牙,暗自思忖整个局面:“我这边虽然强势,但却要以防守为主,不过整体而言,我方还占据着上风。实在不行,就只有摧毁了这座炼道大阵,夺不走宿命蛊,也要令天庭此次修复彻底失败!”

双方都不愿意将炼道大阵摧毁,但逼不得已的话,长生天一方还是要毁掉这座大阵,两败俱伤。

当然,冰塞川此次领袖长生天突袭天庭,最主要的目的就是抢夺宿命蛊。若是毁掉大阵,他们也无疑是失败的。

“现在就看我们的了!”对于外界的战况,牛魔一直都有察觉。

远处,五行**师微微点头,满脸凝重之色。

忽然,五行**师身躯一震,澎湃的气息猛地汹涌而出,一记强大的仙道杀招带出五色霞光,罩向他的对手车尾。

车尾大喝一声,双掌在胸前一拍,也催动杀招。

顿时无数巨大的方形盾牌,在他面前显现,咔咔咔一连串的轻响声中,这些盾牌组成一堵墙,将五色霞光挡住。

“可恶!”五行**师加大力度,五色霞光侵蚀之下,一面面的盾牌破碎凋零。

但车尾的防御密不透风,旧的盾牌破碎,新的盾牌早已经形成,迅速弥补漏洞。

车尾的防御手段十分出众,令五行**师感到一阵阵的无力。

身处炼道大阵,不管是车尾还是五行**师,都要收敛力道,担心一个不注意打坏了这座炼道大阵。

另一边,牛魔和从严的战斗,也是大致的情形。

无疑,这对于防守一方而言,十分有利。

“怎么办?”五行**师传音询问牛魔。

牛魔只回答一句:“进攻,尽全力吸引他们的注意力!”

说着,牛魔再一次向从严扑去,双方缠斗成一团。

五行**师心头一跳,牛魔话里有话,他没有多想,本着对长生天的信任,也全力和车尾继续纠缠。

另一边的袁琼都,则始终在全神贯注地操纵炼道大阵,全力修复宿命蛊。

在所有人都没有注意的情况下,一朵娇嫩的鲜花,好似随风飘零,就这样慢慢地飘到了监天塔下。

“嘻嘻嘻。”鲜花落入监天塔中,忽然变作一位女童。

不是他人,竟然是之前陨落的女仙花子!

“糟糕,有人入塔了!”察觉到这一点,袁琼都大惊失色,连忙大喊示警。这一声喊出,他付出极大的代价,七窍喷血,脑海遭受重创,直接裂成了两半,无法相互交融。

车尾、从严慌忙撤退,想要赶回监天塔驰援,阻止花子接近宿命蛊。

“迟了!”牛魔哈哈大笑,和五行**师一起发力,将这两人死死缠住。

车尾、从严越战越慌,大骂牛魔卑鄙无耻,竟然如此阴险暗算。

这是牛魔、花子的大秘密。他们二人共享寿元,一人即便死亡,只要另外一人还存活,便能在一段时间内再次复活。

花子复活之时,已经飞入监天塔内,炼道大阵中仍旧是只有五位蛊仙,因此并未被撑爆,还在良好地运转之中。

“怎么办?!”袁琼都眼前一阵阵发黑,头疼欲裂,陷入此生最艰难的抉择当中。

他掌握炼道大阵,此刻完全可以对宿命蛊动手脚,令此次天庭修复的一切的努力都打水漂。

若不这样做,他就得看着快要修复好的宿命蛊,被敌人抢走。此举能保存修复成果,以后宿命蛊也可能会被夺回来。但若夺不回来,那就糟糕了,等若是天庭的所有努力都成就了长生天!

这个抉择是如此的艰难,再加上袁琼都脑海受到了重创,以至于思考速度暴降。

当花子冲上了监天塔的最高层,见到了宿命蛊后,袁琼都都还未下定决心,做出抉择。

“得手了!”花子双眼放光,伸出嫩白小手,就要抓住宿命蛊。

若是让她夺得,她势必将成为此次长生天行动中最大的功臣!

但就在这时,监天塔顶部四周的墙壁上,忽然显现出了一副竹林图画。

“什么东西?”花子的脸上先是惊疑,旋即就满是震撼。

三十万年前,中古时代。

元莲仙尊走下星驰山,心中苦笑:“这是星宿仙尊和无极魔尊联手所布的棋局,有着两大蛊仙的伟力,并且从天庭中源源不断地汲取力量时刻补充。我若要强行出手,等若是同时对抗两大尊者的联手啊。”

“当然,这只是两大尊者的手段,不是活人,比较死板。探查清楚后,就能针对。但就算解除了,也会对天庭造成伤害。”

“既然当年,星宿仙尊同意了无极魔尊的提议,要一起布局,那么自然有着考虑,对于天庭是利大于弊的。我还是不用做这个恶人了。”

但是尽管如此,元莲仙尊并不彻底安心。

他一边沉思,一边漫布,抬头时发现自己已经来到了监天塔的顶层。

他忽然没有了烦恼,对着空白的墙壁笑道:“有了,我可以在这里留下一幅画啊!”

在无人所知的情况下,他在这里作了一幅画。

当他创作完毕,这幅画便消失无踪,彻底隐匿,即便天庭中都没有丝毫的记载。除了元莲仙尊之外,所有人都被蒙在鼓里。

三十万年之后,这幅无人所知的竹林图,显露在花子的面前。

花子的手,距离宿命蛊只有不到一寸的距离,但这点距离宛若天堑,无法逾越!

花子的身体,悬浮在半空中,无法动弹一下。

她的目光也非常迷惘,只是对这竹林图稍稍一瞥,她就堕落到了重重幻境中去了。

竹林深深,只有她渺小的一人,迷了路,无法走出去。

花子心中焦躁万分,但却毫无办法!

一阵风吹来,完全竹叶晃动,碧绿斑斓的光影,笼罩着花子。

竹林中的花子只觉得天旋地转,扑通一下,栽倒昏迷过去。

但竹林图并未静止或者消失,墙壁上一根根碧竹继续晃动,好像是风越来越大。

呼呼呼。

一瞬间,激战中的诸仙都听到了耳畔,似乎有风声传来。

墙壁上,竹林飘洒下一片片的宛若玉做的落叶,景色极美。

这阵不存在的风,不仅穿透炼道大阵,传达群仙耳中,更吹入到龙公的心头。

一瞬间,龙公也堕落到幻境之中。

他趴在松软的地上,周围是一根根粗大的碧竹。

风在回旋,竹叶如玉片在优雅地飘落。

龙公的待遇和花子天差地别,他没有丝毫的不妥之感,反而思维无比地清晰起来。

在这刹那间,他感受到了木道的纯粹的力量和奥妙!

他感受到竹林的荫蔽,感受到大树的覆压,他感受到一根根小草钻破土壤而出的,那种青春和生长的力量。

然后,他的四肢像是草木生长一样,升腾起了新的力量。

在这种力量的带动下,龙公支撑着自己,缓缓地,在竹林丛中站了起来。

恍惚间,他似乎看到了元莲仙尊。

他青年模样,温文尔雅,身着青衫,头系白带,一头黑发披散在间。

元莲仙尊微笑着,一步步悠然地从龙公身边走过。

清风就像是他的话语:“爱在左,同情在右,走在生命两旁,随时撒种,随时开花,将这一径长途点缀得花香弥漫,使穿枝拂叶的行人,踏着荆棘不觉得痛苦,有泪可落,却不悲凉。”

说到这里,元莲仙尊已经走进竹林深处。

“你是元莲仙尊?”龙公转身,只能看到元莲仙尊的身影缓缓地隐没在竹林里。

只剩下他的一句话传来:招名为穿枝拂叶。”

下一瞬间,龙公清醒。

他愕然发现,自己已经站立起来,而曾死死束缚他的银白锁链,已是化为一片片的竹叶,飘零洒落。

这些竹叶,就像是玉做的一样。

ps:元莲仙尊所说的这句话,来源于冰心大家的《爱在左情在右》,我觉得非常合适,故用在此处。在此特意说明,向冰心前辈致敬!

\end{this_body}

