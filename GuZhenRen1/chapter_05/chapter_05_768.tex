\newsection{律道大宗师}    %第七百七十一节:律道大宗师

\begin{this_body}

%1
将陶铸真传收刮一空,方源便立即离撤离五界山脉。

%2
和上一世一样,这座山脉同样没有逃脱得了被毁的结局。但是对于方源而言,结果是不一样的。

%3
上一世,他埋伏设计,一力挑起五界山脉大战,意外地惹出陶铸真传,但已经没有机会,只得撤离战场。

%4
这一世,方源顾忌墨水效应,一直隐忍,直到埋伏了南疆追辑的队伍,大获全胜之后,这才觉得时机成熟,提前来收取了这份真传。

%5
再过一段时间,就是南疆正道同意,支持铁家四处建设烽火台。

%6
到了那个时候,方源若再动手,就麻烦多了。

%7
分布四处的烽火台,让南疆蛊仙支援速度和力度都随之暴涨。

%8
上一世最终大战,方源从武庸、池曲由那里,得到了陶铸真传的重要内容,并且还和他们一起组建了超级仙阵。因此对陶铸真传知晓许多,再加上他种种推算,做足了准备,顺顺利利接收了真传。

%9
“陶铸真传!”方源心中感叹不已。

%10
这个传承的手段,对他有极大的帮助,非同小可!

%11
别看陶铸只有八转,但这份真传的重要性,甚至还要超过尊者真传。

%12
上一世就是最好的证明。

%13
方源运用陶铸的手段,来对付九九连环不绝阵,让陈衣等人憋闷不已,进退失据,效果卓绝!

%14
用陶铸的手段,同样还能对付人中豪杰杀招。

%15
受到人中豪杰杀招的增幅,天庭、中洲的那些蛊仙各个实力暴涨。

%16
但只要在五色烟瘴之中,这些蛊仙越强,遭受的反噬就越大。

%17
这是一个奇招,效果好得不得了!

%18
方源也要竖起两个大拇指,怎么夸赞都不为过。

%19
九九连环不绝阵,还有人中豪杰都是尊者的手段,陶铸真传居然能克制住,当时就让无数蛊仙大跌眼镜,心头震撼不已。

%20
“这一世,就算我没有找到破解人中豪杰的方法,也能凭借五色烟瘴来克制,会让中洲的那群人大为头疼!”

%21
“只是这陶铸真传中的手段,并不是特别灵活。当初运用的时候,是搭建出了仙阵。我虽然已知晓这个阵法,但是一个固定的大阵可不是我想要的,还需要此此基础上多加改良。”

%22
方源一边思量,一边初览陶铸真传。

%23
果然不出他的所料,上一世他和南疆、北原蛊仙联手,搭建出来的大阵,就是陶铸真传最为精华的内容了。

%24
在这真传中,陶铸将那座大阵命名为——五界大限阵。

%25
让他有些惊喜的是,除了这个大阵之外,还有一记仙道杀招,其价值和五界大限阵仿若。

%26
此招名为五禁玄光气,一经催发,便能发射出一片五色组成的光气。

%27
光气仿佛界壁的效用,在这光气中,五域蛊仙都会遭受反噬,动静越大,反噬就越严重。

%28
当然,方源因为本身的特殊,他在光气内是不受限制的。

%29
五界大限阵、五禁玄光气,就是陶铸真传中的精华所在。

%30
除此之外,就是保留下来的两只仙蛊。

%31
两只仙蛊都是律道蛊虫,一只八转,名为“界”,一只七转,名为“限”。

%32
八转的界蛊,显然是陶铸的本命蛊,跟随他一路晋升成八。

%33
陶铸的仙窍没有保存下来。

%34
按照真传中的内容记载,陶铸很穷,绝大多数的资源都用于研究,仙窍也没有什么资源。

%35
当然更主要的,是他在寿尽之前故意抛弃了仙窍,留给了南疆正道,借以隐藏自己的真传所在。

%36
方源浏览真传,感慨渐渐增多。

%37
陶铸显然是个怪才,为了研究倾尽一生的时间和精力,投入了不知多少的人力物力。

%38
在真传的内容中,记载了他无数次的实践记录,许多次都是冒着死亡的危险去作死。

%39
即便他升为八转,也时常钻入界壁中感受玄妙。

%40
这绝对不是正常人能做得出来的。

%41
所以,他的仙窍很贫瘠,勉强维持着仙蛊喂养,始终发展不起来,八转蛊仙在界壁中穿梭遭受的反噬实在太强烈了。

%42
“接下来,就是要经营仙窍,拥有喂养两大仙蛊的能力。”

%43
“然后……”

%44
方源面临一个抉择。

%45
是选择五界大限阵,还是五禁玄光气?

%46
两者都是以界、限二蛊为核心,海量凡蛊辅助。搭建了大阵,就不能用五禁玄光气。同样的道理,若选择五禁玄光气,那么在战斗中就不太可能有充裕的时间,去搭建五界大限阵。

%47
搭建大阵,弊端很大,因为是固定的大阵,需要抽取地脉的力量。

%48
陶铸真传中也有明确阐述:五界大限阵不仅是需要铺设在大地上,更要根植于地脉。

%49
所以前世,池曲由就增添了地脉仙蛊进来。

%50
现在这个时候,地脉仙蛊还未出世呢。

%51
一旦搭建了大阵,就不太实用了。因为关键的时刻,方源拿不出来。除非他用阵旗蛊、阵灵来转移这个大阵,但这前提是要再一次大幅改良,效果未知。

%52
而选择五禁玄光气,则非常方便,威能更大,但远远不如大阵那么持久。

%53
更重要的是,它也有弊端。

%54
每用一次这个杀招,就要从蛊仙的仙窍中消耗掉海量的地气!

%55
这种消耗量非常庞大,即便是方源也为之咋舌。

%56
“不过,五禁玄光气的弊端也有解决之法……嗯?到了。”

%57
方源停止思索,来到一处平淡无奇的山洞里。

%58
“隐藏得很好,即便是我也察觉不到什么古怪的地方。”方源暗暗点头。

%59
接着,他便放出陶铸意志。

%60
陶铸意志向前走了几步,山洞顿时发生了变化,一股真意从洞壁内弥漫而出。

%61
陶铸意志将这股真意献给方源。

%62
方源先是检查了一遍,没有发现任何的问题和蹊跷,便毫不客气,当即吸收。

%63
片刻后,他的律道境界一跃成为大宗师!

%64
这份陶铸真意,能让一个律道境界一穷二白的人,直接晋升到大宗师。

%65
当然,方源本身就有积累,律道境界有着宗师级数。此刻方源虽然仍旧是大宗师,但他原本的底蕴也没有白费,量上仍旧有积累,但没有质变到无上大宗师。

%66
“这么看来,陶铸生前就是律道大宗师境界了。”方源微微点头。

%67
一个流派境界,攀升到了大宗师,都会迎来质变的提升。

%68
八转蛊仙这个境界,已经是此辈中的精英。

%69
池家的池曲由就是阵道的大宗师,他能够借助自然道痕而布阵。

%70
焚天魔女是炎道大宗师,当年将方源坑害,在仙材上动手脚,方源怎么检查都检查不出来。

%71
魔尊幽魂则是全流派大宗师,由此可见幽魂魔尊之强。许多八转蛊仙的一生成就,也只是魔尊幽魂的某一方面的素质而已。亏得方源翻盘成功,否则就是魔尊渗透天庭,影宗掌控天下的节奏。

%72
至于陶铸专修禁道,留下的真意却助长律道的原因?

%73
禁道本身就是从律道中,划分出来的小流派,并未完全脱离。

%74
就好像是从智道中,划分出来的魅情道,这个流派尤其擅长增长个人魅力,用情绪蛊惑他人。

%75
事实上,律道这个流派衍生出来的小流派,极其繁多,乃是当之无愧的第一流派。

%76
比较著名的,就是禁道,还有虚道。

%77
虚道也是从律道中划分,讲究虚虚实实,但也没有完全脱离律道。

%78
方源手中的律道仙蛊,有防备、大、斗等。

%79
从中便可看出律道最显著的特征。

%80
大小、高低、胖瘦、远近、好坏、方圆、曲直、真假、强弱、一二三四等等都是律道蛊虫。

%81
正是因为律道的概念太多,导致它能衍伸出最多的小流派。

%82
“所有的流派当中,律道蛊虫的应用范围最广,或者说最为实用。”

%83
“比如说律道蛊虫‘强’,它就可以搭配任何流派,让几乎所有的杀招都变得更强。”

%84
律道还有一个优势,蛊仙常常能凝聚出自身的真意。

%85
这同时也是智道的优势。

%86
真意,涉及律道的真,以及智道的意。

%87
除了这两个流派之外,其他流派的蛊仙能留下真意的,就很少很少了。

%88
十大尊者不算在内。

%89
这些怪物从各个方面,都是特例,不能用常理衡量。

%90
陶铸真意,对于方源而言,是一个不小的惊喜!

%91
因为他知道,前世武庸、池曲由他们可没有这份收获。

%92
这不难理解。

%93
方源早已推算出了不少真相。

%94
陶铸生前被南疆正道联合打压,他临死之前,故意抛弃仙窍,吸引南疆正道的注意力,然后布置了真传,又在另外一边暗藏了真意。

%95
南疆正道得到陶铸的仙窍,还以为他没有研究出什么,又看破不透五界山脉,因此没有发觉陶铸的真传。

%96
虽然上一世,陶铸真传仍旧流落到南疆正道手中,但陶铸真意却是隐瞒了下来。

%97
这份陶铸真意,对于方源而言,是瞌睡了送枕头,来得恰到好处。

%98
因为他刚刚俘虏了巴十八。

%99
这位律道八转蛊仙的洞天,方源之前的律道境界,是吞并不了的。

%100
“我原本还打算和池曲由继续交易,搜集一些律道梦境来提升自己的境界。”

%101
“没想到出现这个意外,根本不需要再筹谋了,节省了我不少功夫。”

\end{this_body}

