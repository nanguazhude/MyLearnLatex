\newsection{我意蛊}    %第七十一节:我意蛊

\begin{this_body}

方源的魂魄在落魄谷中游荡。[www.mianhuatang.cc 超多好看小说]『≤,

他并不游走太远,只在方源肉身周遭范围。

在这里每隔一段时间,就有一阵**雾自行飘来。

方源魂魄沐浴在**雾中,像是书本泡在水中,渐渐松散开来。

“越是凝实的魂魄,就越不容易松散。在这里修行了这么多天,魂魄的凝实程度,是以前的十多倍,可谓进步神速!不过,和魔尊幽魂比较起来,还是天地差距啊。”方源暗自评估,难免又想到义天山大战的情景。

幽魂魔尊是魂道创派祖师,虽然身陨,但是留下的魂魄,却有着天下第一的魂道底蕴。

魔尊幽魂的凝实程度,极其恐怖。

正常的魂魄,像是一片虚幻的光影,最多干扰心智,无法对纯粹的物质有所影响。

但魔尊幽魂由于凝实到了极点,竟然能只手遮天,硬抗浩劫。擎天之姿,一直深深印刻在方源的心中。哪怕他因为纯梦求真体而失忆,这一段记忆去没有泯灭。

“恐怕魔尊幽魂来到落魄谷,这些**雾已经对他不起作用了。”方源思维发散,又旋即暗自一叹,“我何时才能拥有他这般的修为?”

若是方源的魂魄有魔尊幽魂的魂道底蕴,那么就算仙僵肉身上有再强的陷阱,也不惧怕了。

**雾飘来又散去,旋即,又有落魄风吹来。

方源魂魄沐浴在风中,一丝丝的风,像是钢丝,将他的魂魄分离切割。

比凌迟还要剧烈的痛楚,让方源魂魄阵阵颤抖。

但方源咬牙坚持,撑过这阵风。

这阵风还不是很大。持续时间也不长,方源能够撑过。

但若是碰到大风,方源也会明智地撤退,将魂魄遁入肉身之中。

有肉身的防护,落魄风的威能,就要剧减很多倍。

这阵风之后。方源魂魄缩减了三成有余。

谨慎起见,他选择回归肉身。

胆识蛊!

一颗颗的胆识蛊,在方源的仙窍中破碎,化为一股股玄妙神奇的力量,作用在他的魂魄之上。

很快,他的魂魄就再次强壮起来。

刚刚修行所造成的虚弱和创伤,简直是不翼而飞!

荡魂山、落魄谷,可是魂修圣地。

有这两大利器在手,方源的魂道修为是突飞猛进。[八零电子书wWw.80txt.COM]进步十分神速。

就这样,一阵阵的**雾、落魄风之后,方源魂魄像是铁石,经历一次次的煅烧拷打,变得精粹,更加凝实。

只是不知道什么时候,才能有把握进入仙僵肉身中去。

大约一个时辰之后,方源魂魄开始感到一阵阵烦躁不安。

方源明智地暂停了此次修行。魂魄归于肉身,重新走出落魄谷。

一个时辰的时间。是方源多次修行后,摸索出来的自身极限。

魂道修行,有三大方面。

世人皆知的壮魂、炼魂,只是其中两个方面而已。

除此之外,还有第三方面,那就是安魂。

壮魂。能叫魂魄增强壮大。

炼魂,能锻炼魂魄,打磨精粹。

安魂,则是安抚魂魄,沉淀成果。

三者相辅相成。一味强调某个方面,只能顾此失彼,得不偿失。

其中,壮魂首选荡魂山胆识蛊。炼魂第一便是落魄谷中的**雾、落魄风。而最有效安魂的,是**湖中的安魂汤。

**湖就在生死门中。传说中,太日阳莽死后,就一直醉卧在此湖的河岸边上。

“生死门在影宗福地里面。而影宗福地,则在南疆义天山,如今被超级梦境重重包裹住。要不然,我就有可能得到这生死门,好好尝尝三者齐聚,修行魂魄的便利!”

方源也只是想想。

那片超级庞大的梦境,就是天堑般的阻碍。目前看来,他根本没有能力去逾越。

而且,最重要的一点,是天意。

方源的一举一动,都受到天意的关注。

今后就算有所行动,天意也是方源首要防备的对象。

一路飞行,方源没有直接回去他自己的云城。

而是赶往第十二云城。

很快,他就见到了毛民蛊仙毛十二。

“方源长老,你来了。这是你要求我炼制的我意蛊,你来验收一下。”毛十二主动热情地招呼道。

方源一摆手,将所有的我意蛊都收起来,并说道:“还要验收什么。十二长老,我信得过你。”

毛十二大笑,眼中满是感动。

临走前,他握着方源的双手,感激不尽地道:“还是要谢谢你,给了我这个机会,让我炼制我意蛊。否则的话,接下来若是宝黄天开启,我可没有什么资本去买什么荒兽了。”

“我们俩投缘,也是共赢,希望我们合作能永远这么愉快。”方源笑着回道。

“那是必然的!”毛十二的回答相当干脆。

双方交接完毕,方源赶回自家云城。

“我意蛊。”静室中,方源手中捏着一只五转凡蛊,低声沉吟。

这只蛊虫,像是一只蝎子,十分袖珍。它造型相当特殊,像是白纸折叠的一样。托在方源手中,也是轻飘飘的。当它活动足肢的时候,发出窸窸窣窣的轻响,在方源的手掌中央爬上一圈,速度缓慢。

我意蛊的蛊方来自于影宗。

方源如今是完整的天外之魔,不受到天意的影响。那其他蛊师蛊仙,不是天外之魔的正常人,又该如何对抗天意,防备它悄无声息地影响自己呢?

影宗方面给出的一个答案,就是我意蛊。

这是智道蛊虫,可以产出一股特殊的意志我意。利用我意时刻冲刷自身,就能有效地防备天意的影响。

这是影宗对抗天意的研究精髓,是长期摸索出来的结晶。

方源在和毛六的交易中,得到我意蛊的蛊方后,就开始自己炼制这种蛊虫。

但收效不高。

他不是炼道高手,凭他一人之力,很难有多大的成果。尤其是他的精力和时间,都非常宝贵,不能太过耗费在炼蛊方面。

之前炼制变形仙蛊,那是因为意义太过重要。

而我意蛊,只是凡蛊,方源的需求量还十分庞大。

因此,在犹豫了片刻之后,方源就觉得将我意蛊的蛊方流露出去,交给琅琊派的毛民蛊仙,让他们出手,帮助方源炼蛊。

请动琅琊地灵的代价太高,也没有必要。

琅琊派中的毛民蛊仙,除了毛六之外,都很乐意为方源服务。

因为方源已经和他们熟稔,甚至交好,关键是琅琊建派,一切都向门派贡献看齐。方源完成太丘之行后,手中捏着大量的门派贡献,正是其他毛民蛊仙所需要的。

毛十二当然唯一一个,方源还拜托了其他毛民蛊仙。

他将炼蛊费用压得很低。

做到这点,并没有让方源费什么事。

毛民蛊仙们比较单纯,更关键的是,这些毛民蛊仙都十分擅长炼蛊,对战斗还是有心理抵触的,所以都争相要接取方源的这个任务。

而方源到手的这批我意蛊,更不是第一批了。

将手中的我意蛊收入仙窍,方源又检查其他蛊虫。

他在毛十二那边,表现得很信任,很豪爽,但谨慎如他,都是每次收货之后,回来细细检查。

毕竟琅琊派中,可是潜藏着毛六这样的内奸。

方源并不知道,毛六已经是仅剩的内奸了。他小心翼翼,防备毛六,更防备其他可能的内奸。

检查耗费了不少功夫,但和炼制出这些凡蛊的时间,是万万不能相比的。

检查无误之后,方源不禁点点头,眼中流露出赞叹的意味。

“到底是毛民蛊仙炼制出来的,这品质真是叫人满意!”

方源将这些我意蛊,送入仙窍。

准确的说,是送到小北原中。然后一一催用,将这些我意凡蛊尽数消耗。

大量的我意,冲刷刚刚的几个战场,将剩余的天意都冲刷个干干净净。除了雪怪体内的天意,战场上再无一丝残留。

“这些我意蛊虽然是对付天意的利器,但缺陷很多。首先转数太高,每一只都是五转,而且都是一次性的消耗蛊虫,用一次就没有了。”方源暗自可惜。

他心中对此充满了怀疑:“这个我意蛊方,应当只是残次品。影宗虽然交易给我,但他们手中恐怕有更好更加优良的蛊方!只可惜那场交易,我要得到的东西太多,已经做到了极致,最终没有敲诈出更好的蛊方。”

就这样,方源着手铲除仙窍中的雪怪,事后用我意冲刷天意。

同时,他还利用荡魂山、落魄谷,修行魂道。魂道底蕴积累神速。

除此之外,他还挤压出一切闲暇时光,指点其他毛民蛊仙作战。暗地里,自己熟悉仙蛊,勤练杀招。

宝黄天的关闭,同样带给方源巨大的影响。

仙窍的经营建设,方源空有资本和想法,也只能暂缓一步。

日子一天天过去。

方源期待宝黄天开启的同时,第二次地灾也渐渐逼近。

琅琊派对太丘的攻略,从一开始就没有过停止。

不时地,就有毛民蛊仙接过门派任务,通过传送门,前往太丘。

这些毛民蛊仙传送过去的第一要务,就是改造环境,稳定传送蛊阵。

叫方源感到微微奇怪的是,这些毛民蛊仙没有受到任何挫折,一切都很顺利,天意似乎没有对他们出手。(未完待续。)<!--80txt.com-ouoou-->

------------

\end{this_body}

