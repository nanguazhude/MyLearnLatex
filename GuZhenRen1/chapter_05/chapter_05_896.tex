\newsection{沈伤传承}    %第九百节:沈伤传承

\begin{this_body}

%1
黑火并没有悬空燃烧,它附着在一根根的巨骨上。

%2
巨骨乳白色,有圣洁之感。

%3
巨骨有一根主干,从东往西,贯穿天地,宛若九天长龙,一直延伸到方源的视野之外。而在主干的两侧,是无数根排列整齐,向外延展的分支巨骨,仿佛巨大的象牙。

%4
整副骨架就仿佛是巨人的脊椎和肋骨,撑住苍穹!

%5
“天穹脊背……”方源口中呢喃,眼前的景象气魄万千,他身处在巨骨之下,仿佛是蚊虫般渺小。

%6
而这些巨骨宛若参天巨木,黑火汇聚如云,将巨骨覆盖了大半。

%7
方源打量一番后,琢磨出了几许奥妙:“这处天穹脊背竟是一具九转骨道仙材!”

%8
单独这一点,就很不可思议。

%9
天穹脊背应当不是人骨,哪一个蛊仙有如此庞大的身躯?

%10
既然不是人骨,那就是兽骨。

%11
但哪里会有这样的猛兽,能够达到九转的层次呢?

%12
这是不可能的!

%13
九转是蛊尊层次,历来只有人族才会出现。纵观历史也不过九位而已,其余的种族顶多是八转修为,亦或者是太古荒兽、太古荒植。

%14
难道这天地间还存有九转荒兽荒植?

%15
只有这样的存在,死后才会留下如此品级的仙材。

%16
“除此之外,难道是乐土仙尊认为打造的?”方源心中充满了疑惑。

%17
这个可能性也很小。

%18
因为苍穹脊背的作用,方源已经瞧破——它只有唯一的作用,就是为了加固这片龙鲸乐土。

%19
若是乐土仙尊想要加固这片乐土,完全可以用土道手段,乃至音道手段。这才是乐土仙尊所擅长的,何必要寻求骨道呢?

%20
“还有,乐土仙尊为何要这样布置,专门为了加固龙鲸乐土?好像龙鲸乐土很不安全,随时会崩塌一般。”

%21
“但事实上,苍蓝龙鲸可是太古传奇荒兽,实力乃是天下巅峰。能够威胁到它的存在,简直少之又少。它若是生命垂危,加固龙鲸乐土就有用吗?”

%22
“最大的疑点就是,为什么天穹脊背上会有如此庞大的黑火呢?”

%23
这显然不是沈伤造成的。

%24
若是沈伤能够外泄如此规模的黑火,他怎么可能还活蹦乱跳地回来,主动要求封印自己呢?

%25
“这里的黑火和沈伤身上的黑火,究竟有着什么样的联系呢?”方源不断推算,在这个过程中,他看到一根分支肋骨在黑火中彻底消融。

%26
漫天的黑火在猛地腐蚀,不断地摧毁着天穹脊背!

%27
一旦天穹脊背被彻底毁灭,会发生什么?

%28
方源不知道,但肯定不是什么好事情。

%29
他试着出手,操纵万年斗飞车,催发破晓剑群,对黑火展开攻势。

%30
正如他所料的一样,黑火如海,他的攻击宛若向海里投去石块,根本掀动不了一丝的异样。

%31
方源尝试了好一会儿,无可奈何地罢手。饶是他这样的人物,也感到了力不从心。

%32
这就是超级任务的难度。

%33
方源思考了一下,便离开此地,回到接引岛,又辗转去往海底大阵。

%34
沈伤还清醒着,方源便将天穹脊背的情况告知了他。

%35
“对于这个地方,你有什么印象吗?”方源询问。

%36
沈伤苦恼地摇头:“毫无印象。不过这肯定和我的黑火有着一种联系。若是弄清楚黑火之谜,说不定就能洞察到乐土仙尊当年的谋算,以及龙鲸乐土里最深的秘密!”

%37
说到这里,沈伤顿了顿:“我想要亲自去看一看,可以吗?”

%38
“当然可以。”方源一口应下。他既然把这情况告诉沈伤,就不打算瞒着。

%39
要试图完成这样的超级任务,单靠他一人,显得力量很薄弱,他得集合所有蛊仙的力量。

%40
很快,群仙集结一块,来到天穹脊背。

%41
看到满天的黑炎云海,众仙纷纷变色,叹为观止。

%42
群仙尝试着剿除这些黑火,进展极其微小。似乎黑火的规模越大,剿除它的难度就越高。

%43
群仙当中,唯有沈伤的人道手段效果出色,其余人等皆被他比了下去。

%44
清缴黑火的行动持续了数天后,花蝶女仙首先支撑不住。

%45
不是她身心状态不堪,而是仙元消耗太大。

%46
“这个任务实在太难了,我建议放弃。”庙明神偷偷找到方源,私下里跟他商量。

%47
不用他说,方源也已经打起了退堂鼓。

%48
时至如今,众仙拼尽了全力,才瓦解了一亩的黑火。而这样的规模,连漫天黑火的千分之一都不到。

%49
“我若是真正发挥八转战力,定然效率大增,但效率太低了。反不如陆续完成大型任务,几次下来,就能积攒出大量的功德了。”

%50
“这样的超级任务恐怕只有尊者,才会游刃有余吧。”

%51
方源决定放弃。

%52
沈伤、沈从声表示理解。

%53
但他们却不愿放弃,只是让其余的沈家蛊仙也跟着出去。

%54
沈伤对方源道:“我还想再探索一番,顺着巨骨主干走一走,看看能否发现一些关键和蹊跷之处。”

%55
方源深深地望了他一眼,点点头,直接离开。

%56
很快,天穹脊背这里就只剩下沈伤和沈从声两人。

%57
“跟我走走吧。”沈伤叹息一声。

%58
沈从声紧随其后,主动问道:“先祖,您主动支开所有人,难道是回忆起了什么吗?”

%59
沈伤摇头:“我是打算将我的人道传承交给你。”

%60
沈从声愕然,他从沈伤的语气中听出了不祥的意味,忙道:“先祖,你不必如此匆忙。我相信总会有办法能够解决你身上的问题。”

%61
沈伤微微一笑:“我并没有放弃希望,但也要为家族考虑。留下我的人道传承,家族受益,也算是我偿还曾经亏欠家族的债!”

%62
沈从声默然。

%63
沈伤接着道:“在海底大阵,我不好将传承给你,毕竟那是方源亲设的大阵。但在这里,应当是安全的。”

%64
“我不会将人道仙蛊直接传给你,因为我还需要用。我传授你的是我修行的经验,还有我开创的种种杀招、仙蛊方。”

%65
沈从声肃容道:“我定不会辜负先祖一番美意。”

%66
沈伤苦笑:“我的记忆残破不堪,记得的也是有限。就从最初的时候说起来吧,我曾经主修音道,而后在一次意外中,炼出了扶伤仙蛊,从中看到了人道的希望。”

%67
“为了转修人道,我暗中做了许多事,这些事都有违正道规矩。家族包庇我,帮助我隐瞒遮掩,但终究瞒不过乐土仙尊大人。”

%68
“乐土大人找到我,我不是他的对手,连他一击都接不住。但乐土大人没有杀我,也没有囚禁我,而是念在我曾经多年救死扶伤的份上,给我一次将功赎罪的机会。并且他许诺我,若是我能抓住这场机缘,兴许还能达到自己的目的,成功转修人道。”

%69
“于是我便来到了这里,在乐土大人布置的海底大阵中苦修。”

%70
“苦修的记忆已经彻底模糊,我只记得在这个过程中,我得到海底大阵的帮助,屡屡突破,稳步进展。”

%71
“我从扶伤仙蛊出发,逐渐扩展,陆续开创了救死扶伤、轻伤、重伤、致命伤、感伤、五劳七伤等等杀招。”

%72
“终于有一天,我又创造出了伤痕累累杀招。从这个杀招,我推出伤痕仙蛊的仙蛊方。”

%73
“我将其炼成,又替换成自己的本命蛊。有了这只仙蛊,任何攻势对我造成的伤害,都会转化成一道道的伤痕。”

%74
“伤痕积累起来,更能助长两败俱伤杀招的威力。方源之前一战,之所以被我一度逼退,就是我运用两败俱伤杀招,让他的身上也伤痕缠身。”

%75
“请你记住,伤痕仙蛊可以说是我的人道传承的核心,而两败俱伤则是我主要的攻击手段。”

%76
沈从声听得心驰神摇,沈伤开创的人道真传如此精妙动人,让他身心都受到震撼!

%77
“先祖,您开创的这个真传大有前途,或许能够从人道中开辟出一个全新的小流派啊。”沈从声衷心称赞道。

%78
“哈哈哈。”沈伤傲然一笑,拍拍沈从声的肩膀,“我恐怕是不成了。传承交给你,将来你若有意,可以转修人道。若是无意于此,就选择一个合适的传人吧。”

%79
南疆。

%80
道德乐土。

%81
一位女仙身着紫色丝绸宫裳,华美神秘,一步步走下山来。

%82
她肤若白雪,青丝垂至腰间,眼若幽潭,眉宇间笼罩一阵哀愁之意。赫然是天庭领袖,执掌星宿棋盘,恐怕是当今天下第一的智道大能——紫薇仙子!

%83
此刻的紫薇仙子却是一层薄汗,神色憔悴,但双眼发亮,精神振奋。

%84
山脚下,陆畏因早已等候多时:“紫薇大人,此行看来是得偿所愿了。”

%85
紫薇仙子微笑:“此次登山,受益良多。仁蛊已经借到,陆仙友,多谢你的成全。”

%86
陆畏因也笑道:“紫薇大人,我并非是为了成全你,而是为了成全天下,维护五域的安宁。”

%87
“唉,如今天下,魔仙纷起。不说方源,近来在西漠竟出现一位血道八转大能,专找超级世家的麻烦,屡屡侵犯各处资源点,行径极其嚣张,手段又分外恐怖。乱世要来了,我只希望这处道德乐土,能成为一处世外的桃源。”

%88
紫薇仙子叹息:“仙友的意思,紫薇明白了。将来五域变动,定有厚报。告辞。”

%89
“不送。”

\end{this_body}

