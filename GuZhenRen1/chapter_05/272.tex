\newsection{态度蛊的麻烦}    %第二百七十二节:态度蛊的麻烦

\begin{this_body}

%1
方源目光闪烁不定。

%2
影宗的这个消息,有点出乎他的意料。

%3
但是仔细想想的话,却也很合情合理。

%4
逆流河一战,方源虽然在修行资源上的收获并不瞩目,但战力却是得到了大幅度的成长,达到了八转之下第一人的地步,和凤九歌并驾齐驱。

%5
战力的提升,带来地位的提升。

%6
其实在这份消息之前,早就有其他的来信。

%7
比如来自楚度的信笺。楚度在信中,言语客气,首先恭喜方源得胜,恨不得当时在场。然后,他又主动叙述自己这边的情况。楚门已经改成了楚家,原先的太上长老们,都充当楚家的外姓太上家老。楚度教出来的弟子,都收为了义子。最后,楚度阐述他的难处,因为建立了家族,成为正道,所以当时不得不将方源逐出楚门。这并非是他自己的意愿,而是形势所逼。

%8
还有疯魔三怪的来信。信中,除了恭喜方源之外,三怪还非常热情地邀请方源,前来疯魔窟探索。并说:即便北原不容方源,疯魔窟也永远是方源的家。这话有些肉麻,展现了疯魔三怪前所未有的态度。

%9
长生天方面,则重点通缉方源。毕竟方源就是斩杀了马鸿运的凶手。

%10
然而,一直通缉方源,想要追捕他的刘家、耶律家,却反而偃旗息鼓。

%11
方源展现出来的战斗力,彻底让他们震惊了。他们将搜索方源的人员,都收拢了回去。

%12
反而担心方源会找他们报复。毕竟现在的方源,已经在北原声威大振,就连楚度都比之不过。

%13
“现在,影宗也来信,找我谈合作。呵呵。”

%14
“影宗已经有了八转蛊仙,但他似有巨大弱点,时而陷入疯癫。”方源思考着,他和紫山真君在逆流河中动过手,对紫山真君的状况有所了解。

%15
其实若是方源没有炼成坚持蛊,没有参悟出仙道杀招逆流护身印的话,逆流河一战,必定是凶多吉少,落入影无邪的算计当中。

%16
一直以来,影宗残余势力,都是方源的心腹大患。反过来,天意布局,以方源为棋子,夺取了至尊仙胎蛊,方源和影宗之间,是仇恨极深的死敌。

%17
影宗要回收成本,重新炼制至尊仙胎蛊的胡啊,就极可能效仿雪胡老祖炼制鸿运齐天仙蛊。雪胡老祖将马鸿运,当做炼蛊的主材。

%18
方源若是落到影宗手里,自然也难逃这样的下场,自己本身被当做炼制仙蛊的主要仙材,生不如死。

%19
但现在,方源战力增长,已经到了八转蛊仙都奈何不了的程度,所以影宗方面反而要主动和方源寻求合作。

%20
这种前后态度的转变,让方源也感慨万千。

%21
有了逆流护身印,等若是弱化版的仙蛊屋,八转都奈何不得,这让方源平添了太多的底气。

%22
“但是和影宗合作,无异于与虎谋皮!”

%23
虽然毛六信中所言不假,列举出来的理由各个充分,合作的确两利,对影宗、方源都有好处。不谈别的,方源的仙僵肉身上的魂道陷阱,以及至尊仙胎的秘密,都能在这样的合作中解决。

%24
但是,方源信不过影宗!

%25
还是那个老问题。

%26
方源和影宗,双方无法相互信任。

%27
本来双方是不共戴天的死敌,一下子转敌为友,怎么可能?

%28
纵然双方都是理智之辈,以利益为先,然而蛊仙之间联盟合作的基础,并非是个人的信誉,而是信道手段。

%29
偏偏方源信道不行,这是他的一处短板。而在影宗方面,信道手段就太厉害了。

%30
幽魂本体乃是全流派大宗师,别的不说,僵盟、还有中洲搞的那个逆组织,都是影宗的强大信道手段的见证。因为任何的组织,都是以信道盟约为基础建立起来的。

%31
方源根本信不过影宗,也不敢信。如果在合作中,遭受了信道方面的算计,那真是蠢呆了。

%32
不过,方源却没有一口回绝。

%33
他回了一份信,试探影宗诚意。在信中,他提出了一个要求如果想要合作,就想将态度蛊直接送给我。

%34
说起来,八转态度仙蛊虽然一直在方源的手中,但却不是方源之物,而是黑楼兰借的。

%35
当初商量租借仙蛊的时候,双方有盟约,也约定了归还的时限。

%36
时限一过,方源就无法使用态度仙蛊。因为态度仙蛊中的意志,并非方源意志,而是黑楼兰意志。

%37
如今,距离最后的期限,时间已经所剩无几。

%38
态度蛊乃是传奇蛊虫,耗用心力即可运用,使用条件低,效用出色。自从方源获得之后,不知多少次凭此化险为夷,带给他巨大的帮助。虽然态度仙蛊没有直接的攻伐效果,但是辅助起来,却是神奇非凡,妙用无限。

%39
一旦没有态度蛊,对于方源的影响是非常巨大的。

%40
其他不说,仙道杀招见面曾相识就运用不了。虽然方源可以利用变化道的宗师境界,重新设计这个仙道杀招,但是缺少了八转核心,整个仙道杀招将下降了许多档次。伪装成武遗海或者其他人时,被发现的概率将大大提升,非常危险。

%41
“如果能够借此机会,解决了态度仙蛊的问题,那自然最好不过。”方源生出一丝期待之情。

%42
时间差不多了,方源催动暗渡仙蛊,再次为自己增添暗渡效力。

%43
随后,他收起仙窍,重新出发。

%44
一路直向西南方向。

%45
中洲,天庭。

%46
大殿中,紫薇仙子蹙眉盘坐,不断催动着仙道杀招。

%47
她浑身上下,从周身的毛孔中,逸散出屡屡的淡紫轻烟。

%48
她正在运用智道手段,全力推算。

%49
然而,片刻后,紫薇仙子带着满头的汗渍,缓缓地睁开双眼。

%50
“我已经轮换了七种杀招,竟都没有算得出我方蛊仙的位置。”紫薇仙子叹息不已。

%51
中洲蛊仙进攻北原,结果逆流河大战后,方源撤离现场,而中洲蛊仙们则和雪胡老祖,以及长生天方面,展开激战。

%52
中洲方面自然想要救援,派遣援兵。

%53
但是第一波的中洲蛊仙,竟然在黑天中下落不明。紫薇仙子推算不出这些人的具体方位,如何实施救援?派遣援兵,都不知道该派遣到哪里去!

%54
一旁的龙公,缓缓睁开双眼。

%55
他仍旧非常虚弱,骨瘦如柴,老朽不堪。

%56
“看来,巨阳仙尊在北原布置很多。柳贯一的位置,算出来没有?”龙公问道。

%57
紫薇仙子缓缓摇头:“他似乎也有护身之力,可以对抗智道蛊仙的推算,不过并不强大。不过我们关于他的线索,终究还是掌握得太少了。若是有一两件关键线索,一定能让他暴露在我们的视野当中。”

%58
龙公闻言,点点头,沉默了一会儿,他这才再次开口:“也罢了。继续推算我方蛊仙,以及柳贯一的下落。另外,送我去藏龙窟。”

%59
东海。

%60
方源无奈地叹了一口气,不得不再次落下仙窍。

%61
柳贯一可谓名扬五域,导致方源身上的暗渡效力,不断消耗,速度极快。

%62
为了保密,方源只能走走停停。

%63
仙窍门户大开,大量的天地二气,被吸纳进来。

%64
至尊仙体,到了哪一域,都能吞吸天地二气,补益自身。

%65
不像寻常蛊仙,只能在自家地域,吞吸天地二气才是最佳。若是落入其他地域的话,虽然也可以强行吞吸天地二气,但往往会因为天地二气的分别,造成仙窍动荡,大量资源衰竭枯死。

%66
如今,至尊仙窍中,承载的资源太多,有大量的荒兽、上古荒兽。所以每一次吞吸天地二气,规模都非常巨大,宛若巨鲸吞吐。

%67
尤其是又新添了一道逆流河,这种天地秘境对于仙窍而言,负担十分沉重。

\end{this_body}

