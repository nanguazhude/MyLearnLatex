\newsection{布局东海}    %第八百三十五节:布局东海

\begin{this_body}

%1
东海。

%2
沈家大本营,紫檀万雀岛。

%3
沈家太上大长老沈从声热情洋溢:“气海前辈,此乃我沈家的六声茶,敬请品尝一二。”

%4
方源点点头,看向桌案前的杯盏。

%5
六声茶乃是沈家招牌茶水,非常了得,名传五域,比之灵缘斋的青浦茶,万龙坞的蛟影茶,南疆乔家的奴娇茶,房家的七里香茶都还要超出一筹来。

%6
六声茶并非是一杯茶,而是一套茶水,共有六小杯。

%7
这六小杯茶水的炼制工序、茶水温度、茶杯种类、喝茶顺序、具体喝法都有详细的讲究。

%8
方源先端起第一杯茶,这茶清明透亮,仿佛秋日晴朗的高空,一尘不染。

%9
“秋桐音。”方源心中默念一句,小口品茗。

%10
茶水入喉,只觉得一片清爽甘冽,脑海中一片空明澄澈,不带一丝一毫的污染。

%11
整个人就好像是无忧无语,化身成飞鸟,亦或者轻风,在秋天的高空中飘飞。

%12
回味片刻,方源又喝第二杯云流声。

%13
此茶口感奇特,软绵绵的仿佛不是茶水,而是棉絮。偏偏又入口即化,化为一股股的略带冰凉的流水,从咽喉流淌入肚,一路倾泻,毫无阻拦。

%14
茶水全部滚落腹中之后,之前流淌过的舌头、咽喉、肠道皆又升腾起云翳茶雾,这些浓郁的茶雾充斥心胸,又四周渗透,直至浸透五脏六腑。

%15
“好茶。”方源回味良久,接着喝第三口茶。

%16
这茶名叫江娥泣。

%17
茶水承载在一个长条形的杯盏中,茶水水面并不平静,掀起微型的波澜,仿佛是一条浓缩的江河。

%18
方源将茶水倒灌口中,一股玄妙至极的情感便升腾而起,令他流连忘返,仿佛化身一条江河,历经转折,饱受坎坷,最终坦坦荡荡,直入东海。

%19
秋桐音。

%20
云流声。

%21
江娥泣。

%22
玉空篌。

%23
凤凰鸣。

%24
芙蓉笑。

%25
这六杯茶单独一杯,就是极品茶水,更妙的是,它们层层递进、相互辉映,方源连喝六杯,恍恍惚惚,妙不可言的感觉笼罩五感,牵扯八观。

%26
待他醒觉之时,已是过了一个多时辰。

%27
“好茶。”方源品味良久,交口称赞,“即便老夫隐居,也听闻到六声茶的名头。今日一品,果然是名不虚传。”

%28
见方源满意,沈家太上大长老沈从声开怀极了,朗笑道:“能得气海前辈如此赞誉,也不枉费我族蛊仙连炼这茶三天三夜。”

%29
六声茶乃是仙材,动用仙材方能制造,并且还得炼蛊手法高超的蛊仙。为了炼出一套完整的六声茶来,沈从声甚至亲自动手。

%30
这茶在用料上自然有不同的规格,沈从声既然亲自出手炼制,用的就是最高规格,消耗大量的八转仙材。

%31
八转仙材都是价值不菲,用于炼蛊方为正途,但此刻去用来制茶。方源喝一遍后,就没有了,可谓奢侈,但也足见沈从声招待方源的热情!

%32
喝完了茶,沈从声又请方源浏览这座紫檀万雀岛。

%33
这座海岛本身就是一座超级资源点,海岛并非土石,而是一座太古荒植镇海紫檀木。

%34
紫檀木扎根海底,一直绵延生长,树冠高出海面,树枝相互纠结,形成了岛面。

%35
海岛上生活着无数种类的生灵,其中以鸟雀为主。

%36
每当这些鸟雀鸣叫的时候,紫檀木附近的树枝就会颤抖起来,发出美妙的和音。这些和音向四处传播,将周围的海域都安抚住。

%37
这里的海面平静无波,宛若镜面,而天空中更是一丝风都不会有。

%38
这正是“镇海”紫檀木的名号由来。

%39
沈从声热情招待了方源整整两天,全程都是他亲自在陪同,有时候也会招几位沈家的女仙伺候方源。

%40
方源和他交谈甚欢,临别之前,沈从声双手奉上一份礼物。

%41
这些礼物份量颇重,不仅是包含了大量的仙材,而且还有两株上古荒植级别的镇海紫檀木。

%42
沈家的这座镇海紫檀木,乃是太古荒植,须得经历八百年,才会产生一株有活性的分枝。

%43
移栽这种分枝,就能种活一株上古级别的镇海紫檀木。

%44
镇海紫檀木的枝丫虽然极多,但除此之外,移栽任何枝丫都不会有什么收获。

%45
这两株镇海紫檀木代表的,绝非自身这两株上古荒植,而是两处大型资源点。

%46
沈从声为了交好气海老祖,手笔很大,还要超过宋启元一筹。

%47
方源在赴沈家之约前,已是去过宋家了,同样是得到了宋启元的全程招待。

%48
“此次我辞别沈从声,相信我的名头已经传播东海。不,传遍五域了吧。”方源心中暗想。

%49
方源这个气海老祖的身份,注定是要名动天下的。

%50
这一点,从他和龙公交手不分上下之后,就已经决定了。

%51
“接下来,就是等待东海其他超级势力的反应了。他们必然也会邀请我去做客的。”方源算计着。

%52
看来最近的这段时间,他的本体就要四处赴宴了。

%53
这也是方源想要的!

%54
东海虽富,但却始终缺乏一个统一的声音,一个高瞻远瞩的领导者。

%55
方源五百年前世,东海就始终被其他几域压着打。方源上一世,就算东海数位八转联合行动,也只是出手抢夺龙宫,从未有过进攻天庭、帝君城等等想法。

%56
东海不像南疆,南疆有武家,有武庸这个枭雄。

%57
也不像北原,北原有长生天。

%58
更比不上中洲,中洲有天庭。

%59
东海和西漠这两域比较相似,但西漠蛊仙更擅长合作,并且资源上西漠比东海贫瘠得多。

%60
在方源的印象中,东海好像一直就没有什么十分出彩的表现。尽管东海的这些超级势力都是底蕴深厚,东海的散仙身家远超其他四域。

%61
东海的蛊仙实在太离散了。

%62
但若是有一个强大的领导者呢?有一个统一的声音呢?

%63
就像气海老祖这样的人物,他专修气道,在资源上和东海各大超级势力,几乎所有的蛊仙都不冲突。并且他还有龙公一级的战力,这是震慑他人的最大基石。

%64
气海老祖是否能够成为东海的领导者呢?

%65
当然,方源不指望东海的超级势力,都听他的摆布,这不切实际。

%66
但成为一个名义上的盟主,还是大大可行的!

%67
沈从声、宋启元不就是打的这个主意?他们想把气海老祖绑在东海这辆战车上,将来有了战争,他们的头上就有高个子顶着了。

%68
殊不知,这正是方源想要的。

%69
就算是名义上的盟主,干涉不了任何一家超级势力的内政,但只要能影响整个东海蛊仙界,那就达到了方源的目的。

%70
别忘了,东海的四大八转散仙张阴、容婆等,早就是龙宫的龙将了。

%71
有了他们暗中支持,再加上方源的盟主身份,就可影响很深。远的不说,至少中洲炼蛊大会的时候,天庭修复宿命蛊的关键时机,方源可以引导东海,将东海的这股力量用在更正确的地方!

%72
东海正道八转都要来巴结方源,热情招待不说,还要送上厚礼。

%73
就算是和平时代,气海老祖暴露出来的实力、立场,都只得他们拉拢。更遑论如今,地沟频发,五域合一就在眼前的局势了。

%74
龙宫深入海里,直直下沉,直至海底。

%75
这处的海底似乎平淡无奇,但龙宫悬浮片刻,便从中传出龙人分身的声音:“龙宫已至,古族何在?”

%76
声音滚滚,在海底处回荡。

%77
龙人分身耐心等待着,果然片刻之后,海底景象陡然变化,一处门户缝隙打开,显露出一座海底福地来。

%78
一位八转蛊仙出了门户,看着龙宫,目光中带着警惕、戒备又带着一丝一缕的期待,拱手道:“第三万九千七百六十七代古族族长在此,敢问龙宫之主何人也?”

%79
方源的龙人分身走了出来:“正是吴帅。”

%80
古族族长神情一震,又问道:“和方源有和关联?”

%81
龙人分身哈哈一笑:“相互利用,若非借助方源之手,我又岂能附体重生?”

%82
古族族长点点头,并不意外。

%83
龙公、方源,以及气海老祖伏击龙公的事情,如今早就传开了。

%84
龙人分身神情郑重地问:“如今天地大变在即,关键时机真的到了。古族族长,百万年前的约定,贵族还履行否?”

%85
古族族长笑了笑,话中却是不置可否:“此事还需详谈,吴帅前辈,里面请。”

%86
说着,他让开了道路。

\end{this_body}

