\newsection{冰魄仙蛊}    %第一百六十五节:冰魄仙蛊

\begin{this_body}

%1
虽然黑楼兰此次得到了不少的资源,极大地补充了她的最大弱项,但道痕的收益却少之又少。

%2
大约估量一下,还不到整个仙窍道痕的三成。

%3
灾劫直接向前跳了十几次。

%4
这对黑楼兰而言,是个无法估量的损失!

%5
因为按照她的十绝体,渡劫的时候,威力都会很强。福祸相依,如此一来,她每一次渡劫成功之后,收获的到道痕就会很多。

%6
十几次的灾劫,可不仅仅是地灾,还包括了天劫!

%7
前后一算,她的亏损很大。

%8
虽然渡劫危机重重,但对于黑楼兰这等枭雄人杰而言,不怕困难,就怕平庸。

%9
“但即便我不愿意,又能如何呢?我现在是深陷囚笼,身不由己啊……”黑楼兰心中叹息不已。

%10
影无邪察言观色,见到黑楼兰面色晦暗,心中一笑,开口道:“好了,我们继续前行。”

%11
一行人顺着冰霜的甬道,越发深入山体内部。

%12
终于,他们来到一片冰壁面前。

%13
透过白色的冰壁,群仙看到里面竟然有一颗颗的心脏。

%14
这些心脏呈现蓝色,密密麻麻,不下千颗。

%15
白凝冰懵懂。

%16
太白云生却惊呼出声:“天呐,这是冰心,全都是冰心啊。”

%17
黑楼兰也动容道:“不只是冰心,还有孪生冰心,后者可是七转仙材,而且数量这么多。”

%18
就连稳重寡言的石奴,也惊叹:“难以想象,玉壶山中居然有这么丰富的宝藏。如果被外界的南疆蛊仙知道,肯定会风起云涌,为争夺这些冰心。相互之间打得头破血流!”

%19
玉壶山并非渺无人烟,山上有一个寨子,以玉姓为主。

%20
玉壶山盛产玉石和玉蛊。玉家寨子靠山吃山,以此赖以生存。

%21
在凡俗势力中。拥有多位五转蛊师的玉家寨十分强大。但是对于蛊仙而言,却是不值一提的。

%22
谁能想到,一个凡俗势力控制的玉壶山中,居然有成堆的稀罕仙材!

%23
影无邪长叹一声:“玉壶山乃是内秀,很容易被人忽略。我影宗也是机缘巧合之下,才发现了山中的乾坤。不过这里的天然冰心,并未如此之多。而是我影宗蛊仙暗中出手,秘密布置蛊阵。将这里打造成了一个天然养蛊孕蛊之地。”

%24
“曾经这里,就产出过冰心蛊,不过后来已被取走。这里还有一只冰魄仙蛊,之前稍差火候,如今却是彻底孕育成功。白凝冰,上前来,你依照我传授你的方法,收取这只野生仙蛊吧。”

%25
“好!”白凝冰相当干脆。

%26
她直接迈步,走上前去。

%27
然后按照影无邪的叮嘱,双手伸开。直接放在冰寒无比的冰壁之上。

%28
没有动用任何的防护手段,白凝冰的双手直接被冻死,寒气扑上来。在她的双臂上迅速凝结成坚厚的冰层。

%29
黑楼兰目睹此景,微微扬眉。

%30
太白云生开口欲言,神情有些不忍。

%31
影无邪则微微带笑,石奴仍旧面无表情,默不作声。

%32
白凝冰苦苦忍受剧烈的痛楚,寒气通过双臂,强袭她的身躯,很快她就感到全身都麻痹了,思维似乎都要被冻僵。

%33
强烈的死亡危机。笼罩她整个身心。

%34
但她毫无动摇。

%35
她湛蓝的眸子里,透露出来的坚定神情。比眼前的冰壁还要冰冷和顽固。

%36
“忍耐!”

%37
“就算是死,也没有什么大不了。”

%38
“呵。这样被冻死,也有些精彩,不是么……”

%39
“而且!影宗遭受重创,正需要人手。影无邪既然千方百计地招揽我,怎么可能会让我这样死了?”

%40
果然,如白凝冰所料,异变在她眼前发生。

%41
冰壁中,蓝色的冰心忽然齐齐颤抖起来,颤抖的幅度起初很是微小,但旋即抖动越加剧烈。

%42
十几个呼吸之后,冰壁发出咔嚓咔嚓的脆响,一道道的裂痕,在群仙眼前浮现。很快,裂痕如蛛网般辐射覆盖,其中一道越来越深,直至破开到冰壁内部。

%43
仙蛊的气息,洋溢而出。

%44
太白云生差点要忍不住动手。

%45
这可是野生仙蛊,很难捕捉,再不动手,就晚了!

%46
但影无邪毫无动作,只是静静地看着。

%47
他不动,石奴和黑楼兰自然也不会乱动。

%48
白凝冰也动不了。

%49
因为她已经被彻底冻僵,寒气彻底入侵四肢百骸,整个人离死不远。

%50
然而就是这么一位濒死的人,却是分外吸引破冰而出的那只野生仙蛊。

%51
这只野生仙蛊,悠悠飞出来,主动向白凝冰投怀送抱。

%52
群仙视之,只见此蛊婴儿拳头大小,整个蛊虫形似蜘蛛,通体乳白色,八爪,背部却有蓝色花纹,描绘出一个冤魂凄厉哀嚎的模样。

%53
六转冰魄仙蛊!

%54
白凝冰强撑精神,打开虚窍门户。

%55
冰魄仙蛊直接飞入其中。

%56
待虚窍门户关闭,影无邪这才立即出手,为白凝冰治疗伤势,将她从死亡的边缘拉了回来。

%57
“好,冰魄仙蛊既已入窍,那它就成了瓮中之鳖,炼化它如翻书般简易。”影无邪击掌而笑。

%58
黑楼兰目光一闪,忽然开口:“我明白了。这冰魄仙蛊和白凝冰大有渊源。因为白凝冰就是北冥冰魄体!”

%59
“不错。北冥冰魄体的气息,对冰魄仙蛊的吸引和诱惑相当浓烈。所以,我才让白凝冰不动用任何防护手段,徒手捉拿。就是为了将其北冥冰魄体的气息没有丝毫遮掩地,完全的释放出来!”影无邪点头,解释起来。

%60
“但白凝冰若是人气太强,生机太旺,也会让野生的冰魄仙蛊迟疑。所以,唯有当白凝冰陷入濒死阶段,冰魄仙蛊才会感受不到任何的威胁。主动投向白凝冰。”

%61
“原来如此。”太白云生恍然大悟。

%62
“影无邪大人英明!”忠心耿耿的石奴也开口。

%63
“哈哈哈。”影无邪仰头大笑,“我影宗家大业大,跟着我。自有你们好处。就像眼前这片冰壁,就是我影宗用了数千年光阴。模拟自然环境,形成的炼蛊秘地。”

%64
“我影宗掌握的修行手段,也是你们绝不会想象得到的。资质低,那就提高资质。道痕少,那就增长道痕。一切皆有可能。”

%65
黑楼兰不禁心头一震。

%66
影无邪的最后一句话,很明显就是针对她的。

%67
“这个家伙,已经深得领导者的造诣,进步真是神速啊。”还不待黑楼兰暗中继续感慨。影无邪一挥手,“我们走!接下来,就是收复白相洞天!”

%68
北原。

%69
“灾劫跳过的还是太少了点。”方源叹息一声,对此他有些不太满意。

%70
几乎同一时间,黑楼兰嫌自己亏损太多,心情糟糕,恨不得灾劫跳过的越少越好。

%71
而方源却感慨,灾劫往前跳略的幅度,没有他预估中的多。

%72
如果可能,他甚至希望下一刻。他就能成为七转蛊仙!

%73
黑凡洞天中的连绵惨烈的战斗,让方源深刻感受到自己的一大弊端。那就是修为太低,青提仙元不顶耗用。

%74
随着方源的身份越来越复杂。他面临的危险也越来越多。

%75
以前的方源很好把控未来,因为他孤家寡人一个。现在他身负众多盟约,却是有些身不由己。

%76
再加上东海之行,没有剿灭影宗余孽,影无邪等人的成长,更让方源感受到压力重重。

%77
所以他宁愿牺牲发展的潜力,也要迅速提升战力。

%78
“没有现在,哪有未来?”

%79
“潜力虽好,但也需要发展的时间和空间。”

%80
“真正的修行态度。就是务实,一切从实际出发啊。”

%81
方源的亏损。比黑楼兰的还要巨大,但他却没有丝毫惋惜和遗憾。他的目光中仍旧果断、坚定。

%82
“只是,韩东已经度过了二次天劫,二十几次的地灾。我吞并了他的仙窍,却只是向前跨了四次地灾而已。”

%83
“难道说,是因为我的至尊仙窍太过巨大,导致吞并仙窍跳跃灾劫的效果,变得如此不明显了?”

%84
方源左思右想,觉得这是至尊仙窍的原因。

%85
六转蛊仙仙窍时间每过十年,就要渡劫一次。十年一地灾,百年一天劫。历经三百年,三次天劫之后,六转蛊仙就会成就七转的修为。

%86
整个过程,如果分成三个部分,那么方源已经差不多走过了三分之一。

%87
“这个速度还是不够,尽快提升到七转,会令我持续战力暴涨一大截!经济方面也会有一个飞速的提升。”

%88
“一个韩东福地不够,那就吞并更多的仙窍福地。”

%89
“虽然道痕方面的收益,远远比不上自己渡劫。但却节省了大量的时间,令我能更加从容地面对各个方面的挑战和危机。”

%90
“而且,吞并福地,道痕方面我是完全收获。就算目前有些亏损,将来打杀一些蛊仙,弥补回来就是了!”

%91
想到这里,方源眼中闪过一抹残酷狰狞的光。

%92
挖掘出至尊仙窍的另一项优势,方源发现了另外一种修行方法。

%93
以前的修行方法,是按部就班,一次次辛苦渡劫,获取道痕、真意等方面的高收益。但这种方法,速度太慢,战力提升缓慢,增长最大最快的是仙窍底蕴。不适合方源,会让方源的处境更加危险。

%94
现在方源采取的修行方法,是吞并仙窍,跨越灾劫,迅速提升修为。修为上来后,战力暴涨。方源再利用这种战斗力,斩杀他人,吞并仙窍,尽量避免自己渡劫,也能提升仙窍底蕴,增长道痕。

%95
这个方法,增长最快的是方源的战斗力,仙窍底蕴的提升没有那么大。但是却非常适合方源现在的情景,能够让他更有可能生存下去!

%96
两个修行方法,都能构成良性循环,各自的优劣不同罢了。

\end{this_body}

