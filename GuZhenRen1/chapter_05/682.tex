\newsection{杀八!}    %第六百八十五节:杀八!

\begin{this_body}

一群来自东海的八转蛊仙,在中洲的山脉中,低空疾飞。

龙公强夺走仙蛊屋龙宫后,却使得东海的正道、魔道乃至散仙八转,都联合了起来。

“应该就是这里了。”沈从声忽然停下身形,带着微微的疑惑之色,扫视脚下方的青山。

这座青山十分平凡,在周围的山峰环绕中,毫不起眼。

“这里?”华彩云皱起眉头,轻轻摇头,“我可没侦查出什么蹊跷来。”

“嗯……我却感觉此山有些不同。”青岳安道。

“就是这里。”容婆却以肯定的语气道。

周围的蛊仙顿时都看了她一眼。

容婆乃是毒道蛊仙,怎么在侦查手段上也有惊人造诣?

“这里有一处幻阵。”宋启元双眼闪烁琉璃之光,沉声道。

“可是我们这里却没有专修阵道流派的人呐。”华彩云道。

张阴冷笑一声:“还犹豫什么?直接破阵突进去。之前在东海,是因为我们各自戒备,相互掣肘,这才让龙公得逞。如今我们都结下盟约,共同进退,就算引出龙公,又能怎样?”

“哈哈哈,是这个理。我们直接杀进去!”宋启元大笑一声,立即出手。

轰轰轰!

八转蛊仙的手段,自然非同小可。

一连串的爆炸声中,幻阵仿佛是玻璃破碎,露出真正的景象。

只见青山的山脚下,有一座巨大的石窟,石窟中有着天然的龙纹,仿佛群龙乱舞,相互纠缠。

东海群仙只是略微一扫,就对这些石窟龙纹失去了兴趣。

他们飞入石窟当中,旋即就发现地下有一个巨大的洞口。

他们顺着洞口,一路飞下去。

越过一条颇为漫长的地底甬道之后,群仙眼前忽然开朗起来。

“这石窟下面,居然是一片天然地沟!”

“这不是新生的地沟,应当是存在成千上万的年头了。”

“难道龙宫就在这地沟的最深处不成?”

“龙公夺走了仙蛊屋,为什么要把它放在这里?”

东海群仙议论着。

地沟深不见底,一片黑幽。

和地沟相比,他们体型渺小,宛若蚊虫,毫不起眼。

吼!

忽然一阵强烈的飓风,猛地从地沟深处卷席而上。

同时整个地沟都剧烈震荡起来,地沟峭壁上石块滚滚坠落。

东海群仙变色,他们皆感知到一股强烈的威胁感。

震荡来得快,去得也快。

十几个呼吸后,整个地沟就又恢复了平静。

“这底下究竟有什么?”

“不管是什么,能让我们这些多人都感知到危险,必然绝不简单!”

“我倒要看看龙公究竟在搞什么鬼。”

东海群仙正要向着地沟深处进发,忽然地沟中闪烁无边的流光溢彩,一座超级仙阵猛地浮现而出,将所有蛊仙都涵盖进去。

“原来早有埋伏!”

“好厉害的仙阵!”

“什么人?”

一位中洲八转蛊仙老者,出现在东海众仙面前。

他一身灰袍,浑身上下都笼罩着一层深切的悲意。

东海八转都见多识广,顿时就有人认出这位中洲八转老者的身份。

“哦?你是风云府的悲风老人?”

悲风老人叹息一声:“正是老朽。诸位仙友联袂而来,袭击藏龙窟,却不是明智之举。有老朽坐镇这里,诸位不管什么企图,恐怕都不能如愿了。”

“藏龙窟!这里就是藏龙窟?!”许多东海蛊仙不由地心头一震。

藏龙窟中禁锢着传奇太古荒兽帝藏生,早已为人所知。只是具体位置一直没有暴露出来。

悲风老人本来可以成为天庭成员,但是因为爱孙风禅子犯错,悲风老人便代为受过,爷孙俩一起在藏龙窟中看守帝藏生。

因为某种缘由,龙公将夺来的八转仙蛊屋龙宫,安置在了这里。东海蛊仙一行人寻着线索,便摸索到了藏龙窟中。

“咄!”悲风老人身躯一震,催起超级仙阵。

这座超级仙阵能够禁锢帝藏生,自然威能超绝,东海蛊仙们都被仙阵力量卷席,纷纷眼前一花,再定睛一看时,群仙都被单独分化而走,独自陷入到某个大阵空间之中。

下一刻,东海群仙纷纷出手,对着周围的大战狂轰滥炸。

一场八转之间的攻防大战就此展开!

与此同时,中洲各地也频有仙踪显现。

山龟谷。

山谷巨大,山壁陡峭,几乎上下垂直。

魔道蛊仙九鬼指偷偷摸摸地来到这里。

他打量着山谷中的山龟,目光垂涎。

山龟数量极多,几乎是漫山遍野。其中夹杂着许多体型巨大的荒兽山龟。

“在东海罕见的山龟,在南疆、中洲却是常见。这座山龟谷就是整个中洲最大规模的山龟豢养之地了。嘿嘿嘿。”

九鬼指得意地笑了笑。

这里本来是十大古派之一的灵蝶谷管辖掌控的资源点。但是如今,镇守这里的蛊仙却是被抽调出去,保护某个炼蛊大会的场地去了。

“不过,人虽然走了,但是却留下了预警的手段。”九鬼指没有冒然深入,而是缩头弓背侦查了好一阵子,发现了端倪。

这是一个仙道杀招,属于土道流派。

“让我来算算如何破解此法!”九鬼指眼冒精光,十指飞快地掐动指诀进行推算。

指影翻飞,虚虚实实,原本是有十个手指虚影,但渐渐地却是缩减成九个手指头的虚影。

九鬼指的智道传承,便是如此独树一帜的景象。

独缺的一指,就是遁去之一,代表渺茫的天机。

九鬼指只是兼修智道,主修魂道。他一边手指虚影飞舞,一边口中嘀嘀咕咕,念念有词。

随着他推算深入下去,他眼中的精芒越来越亮。

而在相隔十万八千里外的月华丘,一场激战早已展开。

七转散仙谢宝树悬浮在空中,气势凌云,对着月华丘不断地狂轰滥炸。

他一身蓝袍,面容俊朗,温润如玉,一副风度翩翩,浊世佳公子的模样。

而他的对手,镇守在月华丘的女仙则很狼狈,她愤怒地道:“谢宝树,枉你还是东海蛊仙界中有名的散仙,居然行这魔道行径,来抢掠我的月华丘!有本事你去抢夺十大古派去啊,你我同为散修,居然相互为难!想当初,我可是购买过你的年蛊,照顾过你的生意!”

谢宝树叹息一声:“月华仙子,不到万不得已,我也不想出此下策。我的年蛊生意早已经名存实亡,唯一转换门路。你若是能分我一半的月华菇,我便掉头就走,绝无二话。”

中洲女仙气得要吐血:“这些月华菇乃是我辛辛苦苦,耗费精力心血,足足培育了三十年整!你一上来就要分走一半,你想得真美!就算今天我敌不过你,也绝不会将月华菇让给你。逃走之前,我一定会将这座月华丘都毁了的!”

“你毁得掉么?”忽然,一声轻语在中洲女仙的背后响起。

“有暗算!”女仙极其震骇,居然有人偷偷潜入到她的身后,她都没有发觉。

“是谁?!”女仙极力想要转身回头,但下一刻,她双眼一黑,昏死过去。

游仙子站在月华丘顶,看着倒地的中洲女仙一眼,又看向空中的谢宝树,咧嘴一笑。

游仙子专修偷道,是当今东海最擅长采集地沟黑油的蛊仙。他容貌不堪,气质猥琐,却和谢宝树相交莫逆。两人是至交好友。

谢宝树此次来中洲,图谋资源,游仙子当然要出力相助。

流言笼中。

方源大喝一声,施展出仙道杀招万蛟。

群蛟飞腾,和万千的银白猛虎绞杀在一起。

一时间,龙虎相斗,满眼都是银光灿烂,杀意腾腾。

几个呼吸之后,蛟群明显地就缩减下去,不是虎群的对手。

方源面沉如水,连忙继续催发万蛟杀招,支撑场面。

“周雄信大人的手段果然是厉害至极!”

“最妙的是这三招都相互结合,流言笼、人言可畏、三人成虎,越战越强。”

“时间拖得越久,对方源越是不利。但可惜他身陷绝境,已经逃脱不得了。”

三位中洲八转一直袖手旁观,在远处角落里掠阵。

单靠周雄信一人,就将方源压入下风。

任何一位天庭蛊仙,都是天资卓绝,才情惊人的强者。

“平心而论,我的战力只有八转高阶,逊色于厉煌。但是凭借信道的优势,在元始仙尊的杀招基础上,借力中洲亿万民意,战力会越来越强。”

“只要局势如此下去,方源绝对会被我耗死!”

周雄信虽然占据上风,但从未有一刻的大意。

他虽然刚刚苏醒不久,但丰富的情报已经让他对方源认识得十分深刻。方源的强悍,方源的狡诈,方源的威胁已经深深地刻在周雄信的心中。

“信道,还有人道……天庭果然是底蕴深厚,稍微一克制下来,就令我非常难受了。”

方源望着周雄信,不由地心生感慨。

眼前的大敌没有丝毫的破绽,对方源十万分戒备,极其擅长侦查,任何一丝风吹草动都十分敏感,这让方源许多战术谋算都屡屡落空。

方源晋升八转虽然时间很短,但是依赖于逆流护身印、落魄印等等高妙手段,战力稳定在八转高阶。

之前不是在五域界壁中,就是在五行山脉里,或者是光阴长河,方源巧妙利用地利因素,或者借助他人之力,大大缩短敌我战力的差距。

如今,他深入中洲,和八转强者真正交手,便被对方克制。

在信道方面,他造诣太浅。人道上更是不用提了。

“算了,试探就到此为止吧。该利用地利了。”方源的嘴角忽然闪过一抹诡异的笑意。

下一刻,他发动早已埋伏好的手段。

轰隆隆!

一座大阵在中洲外界,猛地升腾起来。

流言笼仿佛是一个鸡蛋,被恐怖的大阵之力狠狠捏碎!

“怎么会有一座大阵?!”中洲四仙无不震怒。

“你们猜?”方源哈哈大笑,猛地化身成太古年猴。他纵身一跃,扑到周雄信面前,拳头如流星般砸去。

流言笼破碎,反噬之力让周雄信大吐鲜血。

宙道手段发动,延缓周围时间,周雄信躲闪不及,被方源一拳击中,轰的一声,砸到地面上去。

“快,快救周雄信大人!”

“一起出手!!”

其他三位蛊仙连忙向方源扑来。

方源狞笑一声,大阵再转,将三位八转拖住一息时间。

趁此良机,方源巨大的身体轰然坠落到,宛若布偶玩具般的周雄信的面前。

仙道杀招――春剪!

一把巨大的剪刀,被巨猿方源双手持着。长达数丈的刀刃,先是深深地插在两边的土地中。

然后,巨剪猛地汇合,在锋锐至极的巨剪前,厚实的地面像是豆腐一般脆弱。

咔嚓!

一声轻响,巨剪终于汇聚,两半刀刃死死地合并在一起。

鲜血飞溅。

周雄信的人头在空中划出一个弧度,然后砰的一声,落到尘埃之中。

周雄信,阵亡!

\end{this_body}

