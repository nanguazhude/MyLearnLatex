\newsection{陈衣}    %第五百五十六节:陈衣

\begin{this_body}

银色巨人高大威能,伸开双臂,向中洲四位蛊仙横扫,瞬间掀起一股狂风气浪。

陈衣眼中精芒暴射,冷哼道:“让我来!”

和银色巨人比起来,他身躯单薄渺小,但气势狂涌而出,一瞬间就和银色巨人分庭抗礼。

仙道杀招捆仙藤!

陈衣伸手一指,无数碧绿光线,蜿蜒流转,迅速攀附在银色巨人身上。

银色巨人拳势威猛,受着一道道的碧绿光线阻挠,速度陡降,方向窜改,轰的一声砸在云泥上,立即砸出一个大坑,泥土飞溅。

反观陈衣冷眼旁观,始终一动不动,悬停半空,一根手指一直遥遥指着银色巨人。

银色巨人身上的碧绿光线,越来越多,将它层层包裹束缚,随后光线浓郁,达到极致,显化出实体藤蔓,根根粗壮,死死地纠缠住银色巨人。

天婆梭罗战阵之中,琅琊地灵愤怒大吼,中了这一记捆仙藤,他仿佛落入泥沼当中,尽管一身气力,但却是有力无处使,十分憋屈。

“毛六!”琅琊地灵吼叫一声。

毛六立即出手,施展仙道杀招,天婆梭罗浑身上下顿时冒出金鳞之光,一寸寸切割捆仙藤,开始迅速恢复。

陈衣冷笑,又施展出新的杀招。

无数断藤蔓被切割下来,落入云泥当中,在陈衣的加持之下,迅速疯长,几个呼吸间,就形成参天巨树。

参天巨树分列银色巨人左右,围成一个大圈,将银色巨人困在中央。

琅琊地灵察觉到不妥,但银色巨人仍旧被捆仙藤锁住,短时间内难以行动。

这时,巨树之间,繁茂的枝叶开始相互纠缠,云泥之下树根也笼具成球,蔓延出土。以树干为主,枝叶、根须相互纠结,形成一个巨大的牢笼。

仙道杀招古木囚笼!

银色巨人终于撕开束缚它的捆仙藤,却又被困在古木囚笼之中。琅琊地灵愤怒大吼,连连下令,毛民蛊仙们纷纷出手,打出一道道的攻伐杀招。

但这古木囚笼居然硬生生承受住,甚至能以仙道杀招为养分,不断壮大自己。

琅琊地灵发现不妥,连忙停手,又试探一番,发现只有徒手撕扯,才是最佳的突破手段。

毛民蛊仙们无奈,只好驱动天婆梭罗,挥动拳脚,对古木囚笼硬砸猛攻。

“该死!”琅琊地灵狠狠咒骂,满头都是冷汗。他受困于此,就难以支援方源,眼睁睁地看着凤九歌、紫薇仙子对超级仙阵出手。

无奈之下,琅琊地灵只好沟通石人蛊仙:“快,把那头老龙放出来!”

吼!

一声龙鸣,巨大的身影腾空而出,横亘在众仙眼前。

太古石龙!

它全身都由石块组成,充斥着浓郁的土道道痕,龙身厚重,龙牙宛若石林,古拙大气,威猛震撼。

石龙速度似缓实快,迅速腾空,跃到高空,在地面上投下巨大的阴影。

天庭四仙也被笼罩在阴影中,纷纷仰头而望,尽皆流露出惊愕之色。

“想不到这琅琊福地中,居然还藏有一头太古石龙!”凤九歌感叹一声。

紫薇仙子微笑道:“这并非琅琊福地所有,恐怕是那些异人蛊仙之物了。”

她身为智道大能,在这一瞬间,就推算出了真相。

“让我把这头石龙给拆了!”雷鬼真君舔舔嘴唇,斗志昂扬起来。

“不必!”陈衣低喝一声,伸手一指,故技重施。

仙道杀招捆仙藤。

仙藤蔓延,捆住太古石龙,石龙行动受阻,但仍旧冲向中洲四仙。

四仙纷纷闪避开来,太古石龙砸在地上,陈衣再次施展出古木囚笼杀招,又将它控住!

“怎么会这样!?”异人蛊仙们见到这一幕,心中的震撼和失望,溢于言表。

就连琅琊地灵也倒吸一口气,浑身冰凉。

他们寄予厚望的太古石龙,居然这样被控制住。反观天庭一方,陈衣大发神威,紫薇仙子、凤九歌一直压制着超级仙阵,雷鬼真君还未出手!

方源的脸色也不好看,不过他却没有多少惊讶的情绪。

“太古石龙之强,在于防御坚厚,勇悍无畏。一旦受伤,得到石人奉献自身,就能迅速地将伤势复原,这个疗伤手段极其变态!”

蛊仙受伤,都很麻烦,要驱除伤势,难度很高。越是修为越高,一旦受伤,麻烦就越大。

换做八转蛊仙,承受了伤势,定要专门腾出时间,集中精力进行疗伤。但太古石龙的这种治疗手段,在战斗中,须臾片刻就能搞定。因此,常常叫那些对手感到头疼无比。

太古石龙也因此闯出赫赫威名。

但是,今时不同往日,它面对的却是天庭八转蛊仙陈衣!

首先,天庭方面就是镇压屠戮异人蛊仙,在异人的尸体上,建立和崛起的,对于石人一族的石龙,那是知根知底。

其次,太古石龙并无八转野生仙蛊可用,而陈衣却是招招八转杀招,又采用最正确的战术,围而不攻,制而不创,让太古石龙最优势的一点,没有发挥的余地。

最后,陈衣施展的杀招着实精妙,尤其是古木囚笼,居然不惧仙道杀招,要摧毁它分外麻烦。

太古石龙轻易地败在陈衣手中,并不奇怪。

虽然石龙奋力抗衡,拼命挣扎,但却被越来越多的捆仙藤,束缚得更加严重,不靠外力帮助,它根本就没有突破捆仙藤的希望,更别谈在外一层的古木囚笼了。

“二郎们,还得靠我们自己啊!”琅琊地灵大声吼叫,银色巨人在此刻终于将古木囚笼,砸出一道口子,突破了重围。

“天婆梭罗不愧是上古第二战阵。”陈衣淡淡地评价一句,忽然微微一笑,呢喃轻语道,“但你以为,我的古木囚笼,是那么好突破的么?”

银色巨人动作顿止。

琅琊地灵微微一愣,回头一望,只见一个木质大手,紧紧地抓住银色巨人的右小腿。

“什么东西?”银色巨人返身,举起双手,两只手臂迅速化作刀斧,狠狠劈下。

木质大手被刀斧劈中,却只砍进一半,并未彻底断掉。

银色巨人意欲再砍,这时从泥土中,猛地钻出一个木质巨人来。

木质巨人放开大手,低头躬身,用肩膀狠狠地撞在银色巨人的胸膛,一下子将银色巨人撞倒在地上。

琅琊派的蛊仙们猝不及防,正要驱动银色巨人起身,但木质巨人攻势连绵不绝,忽然一跃腾空,然后轰的一声巨响,重重地将银色巨人压在身下。

巨大的压力之下,一位毛民蛊仙出现失误,发出一记仙道杀招,打在木质巨人的身上。

木质巨人顿时得到滋养,手臂上的伤势顿时复原,同时身躯碰撞,之前还矮银色巨人一头,现在却是和后者并驾齐驱。

陈衣得意地冷哼一声,这正是他的仙道杀招巨木神像,继承了古木囚笼的特性,仍旧不惧任何的杀招攻伐,只有拳脚肉搏才能对它有效。

“这就是元莲真传的威能么。”雷鬼真君眼眸中精芒烁烁。

一时间,陈衣大发神威,以一人之力将银色巨人、太古石龙统统压制!

这才是他的真正实力,身为元莲派的太上大长老,继承了元莲真传,战力雄浑,非同一般。

之前在青鬼沙漠中,他战得相当憋屈。绝大多数时间,是在豆神宫中动手。害怕破坏豆神宫,他一直都是束手束脚,反观最主要的对手房家的太上大长老,却是占据很大便宜,力道内敛,进行肉搏,把陈衣打得火冒三丈,偏偏又无奈得很。

但现在,陈衣终于有了能够充分发挥他手段的舞台,因此立即展现出强劲威猛的实力,压箱底的因果神树杀招都未用出,他就已经将银色巨人、太古石龙牢牢压在下风。<!--80txt.com-ouoou-->

\end{this_body}

