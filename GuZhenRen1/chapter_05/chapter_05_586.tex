\newsection{春夏秋冬}    %第五百八十八节:春夏秋冬

\begin{this_body}



%1
至尊仙窍时间,数日之后。

%2
方源的宙道分身,再次来到小绿天中的仙阵前。

%3
“小心一点,我要开始了。”宙道分身满脸肃穆之色。

%4
“嗯。”阵灵显形,连忙点头。

%5
一旁还站着一位蛊仙,正是影无邪,此时也是满脸的紧张之色。

%6
一位纯梦求真体,进入梦境,不一会儿,他就抓住夏槎的脚,将她拖出来。

%7
整个过程,都是小心翼翼。

%8
夏槎仍旧陷入沉眠之中,没有丝毫的感觉和异动。

%9
饶是如此,方源操纵着纯梦求真体分身,也只敢将这位八转蛊仙拖出一半的身躯。

%10
夏槎的上半身仍旧在梦境中,而下半身则裸露在梦境之外。

%11
最关键的仙窍,就在腹部这块儿,所以做到这种程度就可以了。

%12
接下来轮到方源的本体出手,酝酿一会儿,催出仙道杀招大盗鬼手。

%13
大盗鬼手幽沉幽沉,直接飞向夏槎的仙窍。然而还未接触到,夏槎表面就浮现出了一缕缕的道痕之光。

%14
这是宙道的被动防御手段,就算夏槎陷入沉眠之中,也能被激发出来。

%15
本来在梦境中,这种防御手段是无法起效的。但是方源的大盗鬼手,在梦境中也没有效用。

%16
只有拖出来,在梦境之外施为,方源才有盗取夏槎仙蛊的可能。

%17
见到大盗鬼手被抵挡,方源一丝意外都没有。

%18
数日以来,这已经不是方源第一次针对夏槎下手了。

%19
这种宙道防御的招数,的确很麻烦,但现在方源已经依靠宙道境界,利用智慧光晕,推算出了解决之法。

%20
于是,另一记杀招被方源催动起来。

%21
杀招的效果十分明显,夏槎身上的道痕之光旋即暗淡下来,大盗鬼手成功地没入仙窍,迅速消失不见。

%22
“幸好夏槎的这防御手段,并不如天庭那些人出色。恐怕也是临时铺设,专门针对大盗鬼手的。所以,我才能在短时间内,破解了此招。”

%23
这一招宙道防御,在层次上还达不到八转,只有七转。方源凭借手头上的宙道仙蛊,搭配不少凡蛊,终于是组成杀招,将它化解。

%24
“这一次铺设宙道大阵,埋伏南疆追兵,虽然重要的宙道仙蛊没有损毁,但绝大多数的凡蛊,还有数只六转宙道仙蛊都折损了。这对我接下来前往光阴长河的大计,颇为不利。接下来,就要看看在夏槎身上能收获一些什么了。”

%25
方源对此满怀期待。

%26
不一会儿,他忽然神色一动,旋即大盗鬼手摇摇晃晃地飞出来。

%27
它本来五指张开,现在却是握紧拳头,里面似乎有什么东西,不断地挣扎着,想要冲破而出。

%28
看到这样的景象,方源不惊反喜。

%29
一旁的影无邪,多多少少了解一些大盗鬼手,此时脱口而出,喜道:“好家伙!第一次就盗取出一只八转仙蛊来!”

%30
这时阵灵却道:“主人,不妙!夏槎魂魄似有波动。”

%31
方源宙道分身顿时神色一肃,连忙操纵纯梦求真体,将夏槎整个人都再次塞进梦境之中。

%32
纯梦求真体就停留在夏槎身边,观察了片刻,见夏槎稳定下来,再次深深沉眠,这才回转而出。

%33
普通的蛊仙,半个身子陷入梦境当中,也会不稳定,有脱离梦境的迹象,但时间要长得多。

%34
夏槎似乎是修行宙道的原因,这段时间大大缩减,少得可怜。这一点,方源在第一次挪移她,把她塞进至尊仙窍的时候,就察觉到了。当时把影宗群仙,都吓出一身冷汗。

%35
解决了夏槎的隐患之后,方源等人这才有功夫,去查看被大盗鬼手盗取出来的仙蛊,究竟是什么样子的。

%36
这只仙蛊强烈挣扎,但大盗鬼手可是借助了九转杀招鬼不觉的威能,当初盗取过魂兽令,所以盗取八转仙蛊,完全是能力范围之内的事情。

%37
再加上方源的丰富手段,炼化它是板上钉钉的事情,不会存在任何意外。

%38
耗费一番功夫,方源将此蛊成功炼化,化为己用。

%39
这只蛊虫,是一头乳白色的蚕儿,有成年人的食指般大小。托在手中,只有鸡蛋的重量,并且蚕体粉嫩,像是摸着婴孩的脸。在蚕的头部,还长着一片翠绿的小叶子,叶子看上去像是桑叶。有趣的一点是,当蚕蜷缩身体的时候,桑叶也随之将它的身体包裹起来。

%40
蛊虫浓郁的气息,时时刻刻都在昭示它不俗的身份八转宙道仙蛊!

%41
但具体究竟是什么仙蛊,方源还不能肯定。

%42
他只能猜测:“这恐怕是八转宙道仙蛊春。”

%43
春蛊若是单独使用,会令一片地域,进入到春天时分。冰雪消融,暖风和煦,新芽生发。

%44
蛊师养、用、炼,用蛊方面亦是博大精深。得到一只陌生仙蛊,需要不断试探,推算出它的威能效用,这方面往往也有不可测度的危险。

%45
方源虽然炼化了这只仙蛊之后,但并不急着去验证心中的猜想,因为他有更可靠的途径,来获取准确的答案。

%46
这个答案,就落到夏槎的身上。

%47
只要搜刮她的魂魄,方源不仅能获知这只仙蛊的身份、用法,甚至连夏槎催动过的春剪杀招,方源都能一清二楚。

%48
方源也没有急着搜魂,夏槎的魂魄受到她肉身的保护,八转修为搁在这里,搜魂并不容易。方源计划严谨,他打算先将夏槎的魂魄摄取出来,令她肉身和魂魄分离,然后在针对魂魄下手!

%49
但要把夏槎的魂魄,摄取出来,也相当麻烦,需要方源准备一段时间。

%50
并且在此之前,方源第一步要做的,是先将夏槎仙窍中的蛊虫,都尽量全部盗取出来。

%51
按照之前的法子,方源接着对夏槎下手。

%52
第二次大盗鬼手却只盗出一只五转凡蛊,和第一次的巨大成果形成鲜明的对比。

%53
“看来流年不利的效果已经消失,我的运势和夏槎本身的运势,落差缩减了许多。”方源旋即明白过来。

%54
八转蛊仙能够有这样的成就,本身就代表有着不俗的天然气运,或者也有后天增加的手段。

%55
方源借助残缺的运道真传,能够拥有高出寻常八转的气运。但对于凤九歌这种人而言,这种程度的气运,就构成不了太大的威胁了。对于凤金煌,那就更加谈不上。

%56
方源清楚原因之后,便继续对夏槎使用流年不利杀招,频繁动用大盗鬼手,盗取她的仙蛊。

%57
至尊仙窍时间,十多日一晃而过。

%58
方源收获了大量凡蛊,还有夏槎的四只仙蛊。

%59
这四只仙蛊形成一整套,分别是春蛊、夏蛊、秋蛊、冬蛊。

%60
春蛊如白蚕,头部长着翠绿的桑叶。

%61
夏蛊则像是一只黑色蚊子,巴掌大小,但背后的薄翼却是多达三对,飞舞中闪烁着七彩斑斓的光。

%62
秋蛊如同一只贵重古董木材雕刻出来的暗红蟋蟀,叫声极其响亮。

%63
冬蛊则是一只灰白色的虫蛹,散发着阵阵寒气。

%64
其中,春蛊、夏蛊有着八转层次,而秋蛊、冬蛊稍弱一些,只有七转而已。

%65
方源把这四大仙蛊盗取出来后,屡屡动用大盗鬼手,就只能盗取出凡蛊来。因此推测,恐怕夏槎手中,就只有这四只仙蛊!

%66
这和黑凡的情形很相似。黑凡手中的仙蛊也很少,当初方源收获的不过是八转似水流年,七转年蛊、以后蛊。

%67
两者比较起来,应当是不相伯仲。

%68
虽然夏槎的八转仙蛊多达两只,但黑凡的似水流年蛊带来的是巨大收益。拥有它,就是天底下年蛊生意的第一巨头!

%69
一下子多出四只仙蛊,并且都是宙道,两只八转两只七转,层次都很高,大大填补了方源在这方面的巨大需求缺口。

%70
不过如何运用它们,还得针对夏槎搜魂。

%71
方源得到这些仙蛊的同时,心中的压力也消散了许多。没有了仙蛊傍身,就算是夏槎意外苏醒,也等若猛虎没有了爪牙,实力下降极多。

%72
把这些蛊仙俘虏的仙蛊,都弄出来,这是首要任务。

%73
其次才是摄魂出来,让他们魂魄和肉身分离。毕竟肉身的操控,除了魂魄之外,还有意志可以解决。随便在仙窍中留下几股意志,就能暂时替代魂魄,操纵肉身。

%74
最后,方源才准备对他们的仙窍下手。

%75
不过接下来,方源盗取仙蛊的工作,还不得不暂缓一下。

%76
仙元不足了。

%77
之前就已经显出枯竭的趋势,最近又铺设仙阵,一场大激战,回到至尊仙窍,又铺设仙阵,屡屡催动纯梦求真变、大盗鬼手等等杀招。

%78
方源的仙元储备,已经薄弱到危险的程度。

%79
宝黄天中,年蛊生意十分惨淡,就算是方源有着龙鱼生意,此刻也支撑不住方源如此巨量的消耗。更别说,荡魂山也一直在不断地修复着。

%80
不过这一切,都未脱离方源的掌控。

%81
他冷冷一笑:“呵呵,是时候让那些南疆势力出出血了。”

%82
接下来的过程,一定会愉悦身心。

%83
不仅是勒索仙元石,方源还会勒索大量的资源,来帮助他喂养新得来的种种仙蛊。

%84
没有修行资源,如何崛起?

%85
崛起个屁!

%86
有一句话说的好杀人放火金腰带。烧杀劫掠,敲诈勒索,才是魔道中人的风采!

\end{this_body}

