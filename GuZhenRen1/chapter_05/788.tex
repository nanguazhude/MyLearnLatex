\newsection{紫薇迷茫}    %第七百九十节:紫薇迷茫

\begin{this_body}

“它冲过来了!”

“快挡住它!!”

“把今古亭的威能催至最强!”

四旬子心头乱颤,将今古亭催得嗡嗡作响,亭中的屏风冒出浓浓青烟,亭子暴射出的光柱简直要凝如实质。

万年斗飞车逼近今古亭。

“直接撞过去。”方源冷笑,傲立船首。

但一瞬间,今古亭的拼死反抗,令万年斗飞车的冲势变得十分缓慢。

“杀招——趋缓?给我破解了。”方源熟知此招,一声令下,万年斗飞车周身银光光晕陡然一凝。

哧!

轻响声中,今古亭杀招被破,万年斗飞车速度恢复如初,随后轰的一声,撞上今古亭。

今古亭不擅长防御,当初被方源亲自动手打坏,这一次碰上更硬更快的万年斗飞车,当即崩溃。

亭盖掀飞,亭柱截断,屏风破碎,亭基崩溃。

四旬子大吐鲜血,扑通扑通,当即跌落到光阴长河中去。

没有了今古亭的防护,四旬子在方源面前完全丧失了还手之力。不一会儿,就被方源全部杀死。

他们还是七转修为,和刘浩一样,没有加入天庭,保留了自身仙窍。

仙蛊却是已被他们毁掉。

方源刚刚将他们的尸首,还有散落在外的仙蛊打捞到手,恒舟到了。

恒舟上站着八转蛊仙清夜,看到万年斗飞车的甲板上四旬子的尸体,他勃然变色:“方源你这个魔头,我要令你血债血偿!”

方源回首,淡淡一笑:“又一个送死的来了。”

万年斗飞车二话不说,银光笼罩,船身下激流涌动,杀向恒舟。

它凶猛的冲锋姿态,顿时让恒舟中的蛊仙们一阵骚乱。

清夜额头青筋直冒,踌躇了一下,还是下达了明智的命令:“我们暂避锋芒,不要和这艘八转仙蛊屋硬拼。”

四旬子虽然阵亡,但战报已经及时传达出去,紫薇仙子还有清夜都知道了万年斗飞车的许多情报。

然而,恒舟想要逃避,又岂是那么容易的事情?

万年斗飞车速度飙起来,竟突破之前的极限,越来越快。

这座仙蛊屋,虽然号称飞车,但其实是船只形状,此刻仿佛化作一颗银色的流星,紧贴着光阴长河飞行!

恒舟是楼船模样,比万年斗飞车要雄阔庞大,想要避让,但速度又不快,哪里能避让得开?

轰!

一声巨响,恒舟剧颤了两下,前侧船体直接破开了一个大洞,左右可见。

方源的万年斗飞车竟直接将恒舟撞穿!

“怎么可能?”

“方才恒舟已经催动了最强大的防御手段,竟然也挡不住方源一击!”

事实是如此骇人,清夜看得双眼都鼓瞪起来。

“大人,方源那厮又冲撞过来了。”恒舟内的蛊仙惊呼一片。

话音刚落,轰的一声,万年斗飞车又撞穿了恒舟。

恒舟剧烈摇晃,清夜一个趔趄,立足不稳。他心中的斗志更是摇摇欲坠,这还怎么打?

恒舟中的蛊虫损失巨大,再撞几下,就会被直接撞散。

许多蛊仙都脸色苍白,满头冷汗,好在这个时候,他们听到清夜的命令:“撤!”

恒舟掉转船头,毅然撤退。

“走得掉么?”方源冷笑。

轰轰轰!

万年斗飞车一次次洞穿恒舟,恒舟运用杀招反抗,但方源早有针对。

知己知彼百战不殆。

只要天庭没有什么巨大变化,方源必然是最后赢家!

这场战斗从一开始,就已经注定了结局。

所以,当恒舟被撞毁,清夜摔入光阴长河中去,方源一方无人意外。

“杀了他!”不消方源下令,车内的其他蛊仙就有多人同时叫喊出声。

杀机鼎沸,万年斗飞车潜入河中,仿佛一头恶鲨,撞向清夜。

清夜拼尽全力,施展防御手段。

但在这光阴长河之中,他身为暗道蛊仙,被压制到了极点。

轰!

万年斗飞车直接撞碎他的黑色盾牌,将他狠狠地击飞。

清夜大吐鲜血,强忍剧痛,极力想要冲出河面。

眼看河面在望,他惊骇地发现,万年斗飞车早已等待多时。

轰!

又一声巨响,万年斗飞车用船底直接砸在清夜的脑门上。

清夜的额头被撞得凹下一大块去,整个人头晕眼花,巨力顺势传导他的全身,他的脊椎、胸骨等等都断裂开来。

咕嘟咕嘟!

大量的光阴河水,顺着他的嘴巴倒灌进去。

巨大的撞击力量,将他整个人像是拍皮球一般,拍到河水深处去。

“这还不死?”冰媛为清夜强盛的生命力咋舌。

“那就再来一下!”石人太上大长老双眼通红,已经杀得兴起。

趁他病要他命,万年斗飞车宛若一个就是杀手,展开了致命的追杀。

轰!

第三下,清夜再遭重创。

从他的右肩下去,大半片的胸膛都被撞成了一滩烂肉,五脏六腑统统被挤爆了。

清夜瞪大双眼,死死地盯着万年斗飞车,他还想挣扎,但已无余力。

最终,他成为了一具浮尸,丢了性命。

“他死了,他真的死了!”

“这可是八转蛊仙啊。”

“我们真的杀了他?!”

异人蛊仙们欢欣鼓舞,就算是白凝冰、黑楼兰等人亦都动容,妙音仙子、白兔姑娘也十分兴奋。

斩杀了一位八转蛊仙,这样的战绩可太稀罕了。尽管是借助了许多外在的力量,但单凭这个战果本身,就已经足够他们吹嘘一辈子。

方源将清夜的尸体打捞上来,可惜他的魂魄却是消散了。

“这清夜也是一个狠人,最后关头居然直接泯灭了自己的魂魄!”

“不过这具肉身,也是上佳仙材,可供我研究虚道。”

天庭掌握着虚道的至高奥妙。

正因如此,才能令天庭成员在贡献了自家仙窍之后,还能获得虚窍,脱离天庭外出作战。

天庭蛊仙一旦身亡,虚窍自行消散,里面的蛊虫无一不存,甚至绝大多数的道痕也涣散一空。

总之,和天庭蛊仙作战,就算是胜了也得益极少。

但是琅琊福地被吞并之后,琅琊地灵告知方源,天庭蛊仙的尸躯别有奥妙,残留着虚道的至高奥义。若是方源能够在这些尸躯上,领悟出一些虚道的奥妙,说不定讲究能破解得了天庭的虚道,甚至抓住天庭的把柄。

就算不能对天庭怎么样,掌握了这个虚道的手段,方源完全可以用于自身。

有了虚窍手段,方源就可以更严酷地控制影无邪等人,这是保证忠诚的极大利器。

万年斗飞车继续打扫战场。

方源一方的蛊仙们普遍都很兴奋,长期以来的训练得到了最好的回报。这一次出击,他们连续击溃两大仙蛊屋,更有蛊仙十数名。

今古亭中有四旬子,恒舟中蛊仙数量更多一些。

仙蛊屋从仙阵中发源,一般都需要多位蛊仙联合操纵,才能更加灵活、全面。有些仙蛊屋对里面的蛊仙数量,要求非常严格,不能多一个,也不能少一个人。

很少有仙蛊屋只需要一个人操纵的。

这些蛊仙没有一个人逃得性命,都被收拾,尸体摆放一边,魂魄纷纷镇压。

战斗结束了,但方源没有急着走,而是在河面上逡巡,等待天庭可能还有的支援。

白兔姑娘、冰媛、石宗等蛊仙议论纷纷,他们仍旧很兴奋,讨论最多的就是八转蛊仙清夜。

说起来,清夜也是挺倒霉的。

恒舟一毁,在这光阴长河当中,他遭受了极大的压制,十成本命能发挥出两三成就不错了。

这里充斥着宙道道痕,浓郁程度绝对是天下第一。

正因如此,宙道手段极其好用,威能倍增。而其他手段,诸如大盗鬼手、逆流护身印等等,都受到极大的压制。

天庭。

紫薇仙子脸色苍白,目光有些呆愣。

她得到了最新的战报。

“方源驾驭八转仙蛊屋,连破今古亭、恒舟,就连清夜都阵亡了?”

紫薇仙子感到难以置信。

她刚刚还在筹算谋划,想要遥控指挥,就算暂时杀不得方源,也要将他赶出去。

然后,她引以为重的两大战力,竟都分崩瓦解。

“这战败的速度未免也太快了!”

“完全不符合常理。”

今古亭、恒舟虽然只是七转级数,但都很优秀,结构非常精妙,拥有着丰富全面的手段。

若非如此,紫薇仙子也不会想到调动这两座仙蛊屋,来围剿方源。

到底出了什么问题呢?

“纵览战报,方源的神秘宙道仙蛊屋,虽然是八转,但手段有限。除了引导光阴河水来加速自身之外,最强大的地方就在于它的坚硬。”

万年斗飞车表露出来争斗手段,就只有一个字,那就是撞。

若换做两个字——硬撞。

再换三个字——加速撞。

很显然,这样的战术是十分单调浅薄的。但为什么就偏偏得逞了呢?

那是因为在交手的过程中,今古亭、恒舟的种种手段,尤其是防御措施,都被方源在瞬间直接破解!

紫薇仙子腾的一下,从座位上站起来。

她坐不住了。

她的额头有了一层细密的冷汗:“看来,方源真的是重生了!他对今古亭、恒舟的了解非常深刻,并且做出了最有效的应对,钻研出了破解的杀招。因此,今古亭、恒舟表现得脆弱不堪,毫无还手之力!”

正如方源所料,紫薇仙子本来就有严重的怀疑,此战之后,她万分肯定方源已是重生。

“怎么办?”

“方源如今有了八转仙蛊屋,纵横光阴长河,我方的三座仙蛊屋还在组建之中。”

“该叫停吗?他重生归来,对这三座仙蛊屋恐怕会很熟悉。”

“如果叫停,改换其他的仙蛊屋,会不会来不及?”

“方源之所以叫进入光阴长河,定然是为了搜寻红莲真传。若是让他得逞,那就糟糕至极,后果远比琅琊福地战败还要严重得多!”

这一瞬间,向来运筹帷幄的紫薇仙子都感到了一丝迷茫。

------------

\end{this_body}

