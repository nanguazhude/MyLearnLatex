\newsection{正道合谋方源}    %第五百七十六节:正道合谋方源

\begin{this_body}

“必须要铲除这个方源,竟公然袭击我正道大阵,实在是胆大包天!”罗家太上大长老在咆哮。

“此次我族蛊仙壮烈牺牲,方源魔头手中已经两次沾染了我族鲜血,必定要让他付出代价!”侯家太上大长老语气深沉,充满仇恨之意。

姚家太上大长老叹息一声:“方源必定是要对付的,但是此人可不同寻常的魔道蛊仙。他的身份非常复杂,不仅是天外之魔,更有春秋蝉傍身,从未来重生,捣毁八十八角真阳楼,拥有巨阳真传不说,还继承幽魂真传,成为影宗当代宗主,更获得琅琊福地大量资助。就算是中洲天庭屡次通缉、追杀,他仍旧逍遥法外,为祸世间,并且屡战屡强,实力进展迅猛无比,大大超出常理。说句心底的话,真是有一丝魔尊年轻时候的气象了。”

“哼!未来只有大梦仙尊,不会有大梦魔尊。姚家的你是否怕了?”夏家太上大长老立即讽刺道。

姚家太上大长老冷笑:“我只是说明事实。要想除掉方源,就得正视这个事实。就像之前你夏家败给我姚家一样。”

“你!”夏家太上大长老顿时气极。

最近这一段时间,地脉频动,将地底深处的大量修行资源翻到地上来,吸引南疆正道、魔道以及散修,竞相争夺。争夺之际,自然就发生许多摩擦和矛盾。夏家就在不久之前,中了姚家一个埋伏,吃了大亏,但姚家也付出了不菲代价。因此两方交谈,火气浓郁。

“好了,我们此次商谈,不是为了相互斗气,而是商讨出如何对付那方源。若是方源继续这样的劫掠,我们应当怎么办?”这个时候,商家太上大长老开口,语气缓缓,不急不慢。

一阵沉默。

商家一直固守中立,又最擅长贸易,和各家蛊仙都有着。所以商家太上大长老开口,其余人都要卖点颜面。

良久,巴家太上大长老打破沉默:“方源贼子一定是掌握了探索梦境的妙法,并且能获取巨大好处。否则当初,他为何千方百计地冒充成武遗海,混进我正道里来呢?此次他攻破掠影地沟的那处大阵,把所有的梦境都卷席一空,更说明这一点。不知道武家太上大长老如何看待此事?”

武家太上大长老自然不是别人,正是武遗海的哥哥――武庸。

武家一直霸占着正道第一的宝座,巴家对此暗中觊觎,可惜武庸爆发,又展露出玉清滴风小竹楼,使得巴家图谋破灭,武家守住宝座。但巴家却不会放过任何一个这样微小的机会。

武庸沉默了一会儿,这才开口:“方源贼子,人人得而诛之。昨日他四处劫掠池家资源,那么明日,他就可能忽然出现在任何一个地方,我们的资源甚至蛊仙族人,都有可能会遭到他的毒手。”

武庸说的不咸不淡,但也指出,对付方源乃是我们南疆正道所有人的责任。巴家或者还有其他人,想要我武家挑头,付出大力气,那是不可能的。

听了武庸的这番话,不少人都皱起眉头。

武庸和武独秀的风格,有很大不同。换做武独秀当家,必定是首当其冲,当仁不让。这正是武家的风骨。

但武庸却在这个时候缩了,好像之前方源留给武家的耻辱,并不存在一样。

巴家太上大长老却不想这样轻易放过武庸,又道:“处置方源这个魔头之所以麻烦,多半是因为他拥有着定仙游仙蛊。不知武家太上大长老有什么好的方法,来克制定仙游呢?”

武庸苦笑一声:“惭愧,武家这边却无什么妙法。我武庸自然无法和母亲相比,守成有余,进取不足。依我看来,之前池家太上大长老布置的蛊阵,的确是有着效果。虽然被方源攻破,但也阻止他许多时间。不如请池家太上大长老出手,为我等各处资源,布置出仙阵来,至少方源偷袭过来时,也能拖延一段时间。”

池家太上大长老池曲由连忙道:“惭愧!我布置出来的仙阵,屡屡被方源攻破,恐怕会辜负诸位的期待。”

武庸笑一声:“池家太上大长老不必过谦。你族的凤焰山,方源就没有攻破。建设在掠影地沟处的仙阵,也抵挡住方源一次强袭。只是方源贼子阵道境界不俗,曾经令贵族的阵痴都亲近交好,这个事实大家都心里清楚。”

武庸话中有话,池曲由心中冷哼一声,却不接口,只是沉默。

南疆各大超级势力手中,掌握的资源点有许许多多。池曲由要一一建设仙阵,要建设到猴年马月?

更关键的是,这种事情非常的敏感。

资源点都是各家的禁脔,要将这些地方的防御措施,交给外人来建设,各家心里都会有芥蒂。

现在各家当中,只有池家遭受方源劫掠过,没有切肤之痛,各家也没有这么强的意向,来请池曲由出手。

池曲由明白这一点,所以干脆不说话。

武庸见池曲由沉默,一时间也不好再开口。

这时,铁家太上大长老兴致勃勃地说道:“其实武家太上大长老的方法,乃是解决方源这个麻烦的正道,只不过只对了一半。池家太上大长老布置仙阵,多能抵挡方源一阵。这个时候,若是旁侧有一座我铁家的烽火台,我们就可及时挪移,堵住方源,形成围剿绝杀之势!两相集合,如此一来,方源不足为患也!”

这一下,蛊仙们更沉默了。

烽火台乃是铁家的仙蛊屋,最强之处,有点类似于西漠萧家的万里丝廊。每个烽火台之间,都能够相互传送蛊仙。若是多座烽火台在一起,甚至能传送八转蛊仙!

铁家太上大长老的方案,的确是没有错。事实上,铁家一直都在致力于,将烽火台分别布置在南疆各个地方。

但这种布置,对于其他超级势力,具有太大的威胁了。所以一直以来,各家势力都在抵制铁家铺设烽火台的这个计划。

现如今,铁家太上大长老利用这个良机,又推自家的烽火台,各家首脑没有答应,只是沉默,就表达了否定的意思。

“我还有一个法子。”良久,商家太上大长老这才开口。

持续了一炷香的功夫,这场南疆正道势力首脑之间的商谈,徐徐落下帷幕。

月夜下,池曲由停下杀招,驾着浓云在高空飞驰,目光幽幽。

南疆蛊仙界本来就山头林立,相互之间各有矛盾重重。最近这段时间,地脉翻动,带来利益的同时,也刺激更多的矛盾产生。这就使得各大超级势力之间,火气更大。

勾心斗角且不去说,单说池家这次被方源四处偷袭,至始至终就没有见到任何一个其他势力出手帮衬的。

又飞行一段,池曲由视野骤然大变,陷落到一处仙道战场之中。

他却一点都不紧张,因为这正是阎罗战场。

此次和方源会面,也是约定好的事情。

“这是给你的。”方源抛出几位纯梦求真体,还有一份信道凡蛊,“我的东西呢?”

“都带来了。”池曲由取出大量的仙元石,还有许多仙材,分门别类,还有信道凡蛊一只,里面记载的是阵道传承。

双方迅速验货,顺利交接,然后立即分离。只是池曲由微微皱眉,方源却是面带微笑。

“池家太上大长老,将来若有机会,我还会找你交易!”方源临走前,传音过去。

“你如何联络我?”

“你到时候便知。”

两人迅速撤离原地,警觉非凡,至始至终这场交易都无他人知晓。

“这蛊虫中的内容,仍旧意犹未尽得很,可见方源手中掌握的梦道成果很多。不过我此次得了这些,还有梦境,已经大大超越了南疆其他超级势力了!”池曲由望着自家仙窍内的一片梦境,心中感慨不已。

又想到方源,他目光便又阴沉下来:“方源这贼子,明显是还想和我合作。毕竟他还想知道义天山那处的大阵。我且不妨先与其合作,尽力谋取更多的梦道成果,再借助南疆正道,合力图算他,将他铲除!”

池曲由和方源交易,完全是利益。此时想要谋害方源,也是利益驱动。

毕竟,方源若是将这些梦道成果卖给其他超级势力,那池家的优势就不在了,等若是自身利益受到了极大的损失。

“这一次虽然交给池家几份梦境,不过宙道梦境已有许多,接下来就是安静潜修,增长宙道境界!”方源也在思虑和谋算。

池家有了梦道手段,当然需要一些私人梦境,供其探索。义天山处虽有巨大梦境,但是众目睽睽之下,偷偷探索,难保不出纰漏,风险太大。

方源一方面需要促成这场交易,另一方面也不希望池家暴露。

“我这一次成功强袭掠影地沟,南疆正道必定图谋算我。定仙游虽好,但也受克制。不说南疆正道的底蕴,就说武庸和天庭之间的紧密关系,恐怕天庭也会来插手。所以还是先避风头。”

其实,方源利用梦道成果,找任何一个超级势力,都能令他们化敌为友,进行交易。

而方源偏偏选择池家,自然是有原因的。

最主要的一个原因,就在于五百年前乱世,池家是抵抗天庭入侵的坚决势力!

前世武家虽然也是这样,但今生,武庸居然能从天庭那里,索回南疆正道的诸多仙蛊,这一点让方源一直暗记在心,不能免怀。

ps:今天状态不佳,挂水得持续三天。以前看武侠小说,武林高手被下泻药,跑了三五天厕所,就战力大减,对于这样的描写,一直不太理解。经过这次的急性胃肠炎,我是有了最切身的体会!真的太恐怖了,拉肚子拉水一样,昨天前后二十多趟,拉得整个人都虚脱了。好在昨晚挂水之后,高烧已经退了。对于最近更新不到位的地方,十分抱歉,同时万分感谢大家的关心!

------------

\end{this_body}

