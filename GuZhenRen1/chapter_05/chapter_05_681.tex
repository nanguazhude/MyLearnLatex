\newsection{三人成虎}    %第六百八十四节:三人成虎

\begin{this_body}



%1
方源悬浮在空中,像是在暴雪天气一动不动地站了一炷香的时间,身上积着一层白色的雪。

%2
远看的话,又好像是披着一层厚实的狐皮毛衣。

%3
这是改良过的宙道杀招冬裘!

%4
方源的防御杀招有不少,最强大的当然要数逆流护身印,但因为逆流河已经将近干涸,这招几乎不能再用了。

%5
除此之外,方源还掌握了阎帝、冬裘、血染征袍、紫念光护、发甲、狮毛甲、卜卦龟变化、鬼官衣等防御手段。

%6
发甲、狮毛甲只是凡道杀招,虽然方源曾经用过一段时间,比较实用,但早已淘汰。

%7
紫念光护是紫山真君的手段,曾经在大战中,防御住龙公的拳头,威能当然不容小觑。不过方源只是掌握了这个杀招,并没有修行,缺乏关键仙蛊。

%8
血染征袍,则是方源的手段,曾用来过渡。虽然核心仙蛊已是七转,但是七转层次的杀招还是上不了台面的。

%9
卜卦龟变化也能算得上是一种防御手段,但同样的原因,只是上古荒兽的卜卦龟掺和不了八转间的大战。

%10
鬼官衣来头很大,是幽魂魔尊设计出来,专门防备智道推算用的。

%11
结合了鬼官衣、魂兽令、鬼不觉的阎帝杀招,防御惊人,可以用来和八转对战。

%12
但是防御威能终究还是不如冬裘杀招。

%13
阎帝本身是一个综合的,全面的手段,攻防一体,搭配阎罗子,群战威能惊人。

%14
而冬裘杀招则专注于防御,并且方源吞噬夏槎仙窍,导致宙道道痕暴涨,乃是最多的道痕,极大地增幅了冬裘杀招的威能。

%15
更关键的是,经过前一段时间的改良,冬裘杀招的防御威能比之前还要出色许多!

%16
在不能运用逆流护身印的情况下,冬裘杀招成了方源的替代之物。

%17
轰轰轰……

%18
三位中洲八转轮番针对方源狂轰滥炸,方源不闪不避,依靠冬裘杀招,尽数将所有攻势都遮挡。

%19
“这是什么杀招?”

%20
“好像是夏槎的冬裘……”

%21
“怎么可能?夏槎的冬裘可不是这样。”

%22
三位八转神情各异,暗自交流,为冬裘的防护威能感到吃惊不已。

%23
随后,他们都将各自的目光转移到周雄信的身上。

%24
这三位八转,分别来自于中洲十大古派,而周雄信不同,他是天庭成员,最近刚刚苏醒。

%25
周雄信已经催发出了仙道战场流言笼,在这个战斗环境中,压制其他所有的流派,只增幅信道!

%26
方源的冬裘杀招被压制,三位八转的攻伐手段也同时被压制。

%27
三位八转看向周雄信的同时,方源则在打量着这处仙道战场流言笼。

%28
流言笼由无数的银白文字组成,这些文字有的大如车马,有的细小如蚁,它们交织在一起,组成一个巨大的圆球,将方源和其他四位八转罩在当中。

%29
因为字和字之间有着缝隙,所有整个圆球似乎和外界还有勾连,并不是完全闭合的样子。

%30
方源眼中有着一抹凝重。

%31
至尊仙窍内,他的宙道分身正沐浴在智慧光晕中,积极地推算着这座仙道战场的奥妙。

%32
但是关于信道,方源一点都不擅长。

%33
若是宙道、炼道的仙道战场,方源凭借宙道、炼道准无上境界,破解起来分分钟的事情。但是面对信道,他就有点茫然了,只有依靠智慧光晕硬生生地去推算了。

%34
不得不说,天庭派遣出信道蛊仙周雄信来对付方源,是一个明智的举动。

%35
换做其他的蛊仙强者,只要方源见过的,他都配备了一套应对的手段。但是眼前的周雄信,方源却是第一次见到。

%36
周雄信一身白袍,方形脸,一脸正气,体格强健。

%37
能够成为天庭一员的,都是八转中的精英、强者。别的不说,单单刚刚的那一记流言笼,妙不可言,速度奇快,乃是方源见识中第一迅猛的仙道战场杀招!

%38
“方源,你既然被我困住,就别想活着出去了。你造了如此多的杀孽,多少无辜的性命惨死在你的手中!现在,就让你好好品尝一下你自己造成的恶果吧!”周雄信义正言辞,周身气势勃发,连续催发出两记仙道杀招。

%39
仙道杀招——人言可畏!

%40
仙道杀招——三人成虎!

%41
顿时,整个流言笼的每一个银白文字,都开始绽射出璀璨的光辉。

%42
浓郁的光辉仿佛变成了流水,逸散漂流在空中,不断流转,然后汇聚。

%43
随后,从银色的流光中,蹦跳出一头头的银白猛虎。

%44
虎群袭来,方源身形猛地电射而出,反冲上去。

%45
力道大手印!

%46
方源冲到银虎面前,试探一击,结果大手印只是阻了一阻,就被银虎一头撞破。

%47
方源眉头微挑,银虎实力相当不俗。

%48
仙道杀招——春剪!

%49
碧绿色的剪刀风一边飞去,灵动非凡。所到之处,剪刀开合不断,银虎不是头颅掉落,就是被拦腰剪断,尸体化为一蓬蓬的银白琐碎的光芒,四下迸溅飞散。

%50
春剪杀招势不可挡!

%51
周雄信眼中闪过一抹异色,心中暗凛:“和情报上似有差别。自从红莲真传争夺战后,方源的春剪杀招似乎又改良过了,威能上涨了至少两分!”

%52
周雄信乃是信道蛊仙,对于一丝一毫的变化和端倪,都能注意并且迅速采集。

%53
“春剪杀招在夏槎手中,数百年都没有变化,到了方源手中却是一变再变。方源的智道造诣真的就这么强?或许他的宙道境界也不俗得很,毕竟这可是八转杀招,要改良它可不容易!”

%54
周雄信暗暗心惊。

%55
眼下还是在流言笼的压制下,春剪杀招仍旧是神挡杀神、魔挡杀魔,一时间不知多少银白猛虎被它剪断。

%56
不过,周雄信并不慌乱。

%57
“场面仍旧在我的掌控之中,慢慢的,你就会尝到厉害了,方源啊。”周雄信心中冷笑。

%58
片刻后,方源的眉头微不可察地皱了一皱。

%59
“银白猛虎源源不断,它明显是和这记流言笼战场一体,不管杀灭多少,都还会再生!”

%60
“而且,更关键的是,这些银虎在不断地变强!”

%61
“古怪……”

%62
一般而言,银白猛虎这类手段,是依靠数量的优势来对敌人展开消耗战。

%63
但银白猛虎不断增强的幅度和速度,都很惊人,超出了这类手段的正常界限。

%64
之前春剪杀招势不可挡,但现在却是显得有些疲累不堪的样子。银白猛虎不仅更加强硬,而且灵活敏捷得多,春剪之下时常逃得性命,令春剪剪空。

%65
这时,宙道分身终于推算出了一些成果。

%66
方源望向周雄信,开口道:“原来如此,这是信道手段和人道手段的结合!你是利用了整个中洲的民怨,来施展此招的吧?就像之前的千夫所指杀招一样。”

%67
周雄信不禁眉眼一跳,方源这么快就发现了其中的奥妙,令他心头猛震。

%68
“既然你看出来了,也就无所谓隐瞒了。不错,这杀招的根本在于元始仙尊的手段,整个中洲的民意都被调动过来。”

%69
“方源,你恣意屠杀了这么多的人,民意沸腾。一个凡人的力量虽小,但千千万万的力量叠加起来呢?这种力量大到你难以想象的程度。”

%70
“所以,尽管挣扎吧,你是绝敌不过万众一心的民意的。这是你一手早就的恶果,你是自食其果!”

%71
方源沉默。

%72
他开始对流言笼本身,展开狂轰滥炸。

%73
但是这记仙道战场,在没有探查出运转的奥妙之前,就算是他动用了落魄印,都收效甚微,反而白白浪费八转仙元。

%74
与此同时,在中洲某处。

%75
“方源被困,恐怕是被困在某个仙阵,或者是仙道战场之中。我对他的感应已经变得相当模糊,几乎要断绝了。”沈从声呐呐地道。

%76
他是东海八转蛊仙,超级势力沈家的太上大长老,修行音道。

%77
方源在东海觊觎龙宫时,不知不觉着了他的侦查手段。到现在,哪怕是被困在流言笼中,沈从声都能察觉到方源的一些动向。

%78
“从声兄台的侦查手段,令老夫不得不佩服。”一旁,宋启元满怀感叹之色,“想当年,我可是被从声兄盯着不放,遭了不少罪呢。”

%79
“哈哈。”沈从声笑起来,“那是年轻时候的事情,谁让你这小子贼溜,偷偷地取走了传承,让我们费尽辛苦,却都两手空空呢。”

%80
“方源出手,阻挠炼蛊大会,是预料中的事情。毕竟他和天庭一直都有深仇大恨呢!”

%81
“是啊,从数年之前,他就和天庭在宝黄天中各自揭短呢。”

%82
“天庭想要修复宿命,就算他是天外之魔,也绝不想让天庭功成。那样一来,天庭的优势就太大了。”

%83
周围的蛊仙开口附言,也有人一脸冷漠,无动于衷。

%84
放眼望去,这些蛊仙竟都是八转!

%85
宋启元、青岳安、沈从声、张阴、容婆、石淼等等,阵容之强大,前所未见!

%86
“好了,这样的情况不是很好吗?方源不是好东西,他和天庭对掐起来,各自牵扯精力,我们就少了许多变数了。嗯……龙宫就在附近,不会错了。”沈从声一直走在前方,领着路。

%87
原来,龙公在东海抢夺走了八转仙蛊屋,令东海诸位八转大失颜面。

%88
趁着这一次炼蛊大会,这些人居然暗中联合起来,要找回场子,再夺龙宫!

\end{this_body}

