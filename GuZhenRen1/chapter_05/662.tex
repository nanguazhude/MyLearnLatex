\newsection{进展}    %第六百六十四节:进展

\begin{this_body}

至尊仙窍。

小中洲。

方源宙道分身在半空中缓缓飞行,将手中最后的一朵花骨朵儿插种下去。

这里的地貌已经被他改造,形成浅滩,移来了大批优质淤泥,非常肥沃。淤泥上还有着膝盖高的清澈河水。

宙道分身拍拍手,最后查看了一遍花骨朵儿,便悠然飞上高空。

站在不一样的高度俯视,整个人造沼泽就落入他的视野之中。这片沼泽非常广阔,方源宙道分身此刻也是一望无垠,难以将所有的沼泽映入视野当中。

每隔数里,就有一朵花骨朵竖立着,大若房屋。这些花骨朵儿可不简单,乃是荒植太泽花的花骨朵。

有一小半的太泽花,都是方源从苍蓝龙鲸的仙窍中得来的,还有的则是他这些天从宝黄天中收购。

接着,方源宙道分身深呼吸一口气,催起夏蛊。

夏蛊来源于南疆八转蛊仙夏槎,本身也高达八转。方源的宙道分身只有六转修为,单靠自己是用不了夏蛊的。不过,方源本体却有白荔仙元,又将夏蛊借给宙道分身用。自己借给自己,令方源宙道分身几乎能够运用本体所有的仙蛊。

夏蛊催动起来后,便是其他的辅助蛊虫,片刻后,一记仙道杀招迸发而出。

夏槎开发出了一套宙道杀招,专门用来经营仙窍,便是春种、夏耘、秋收、冬藏。

方源宙道分身催动的杀招,却不是其中任何一种,而是从夏耘上改良而来,专门用于培育太泽花的。

太泽花在盛夏开花,在秋天结果。

方源的宙道境界已然是准无上,乃是世间峰巅。再配合智慧光晕,推算改良出宙道杀招来十分简单,毫不吃力。

随着杀招不断催动,方圆万里都进入了盛夏时期,光线变得越加明亮,空气中的温度大幅度提升,同时变得越来越干燥。

盛夏来到了,生命力极其旺盛的太泽花感知到这一切,纷纷汲取水土营养,以肉眼可见的速度长大,然后徐徐绽放。

每一朵太泽花都十分巨大,花瓣接连一体,形如喇叭,通常是白色为主,少数带着一丝丝的粉色。

原本深达成人小腿的河水,已经见底,泥泞的土壤也变得干燥起来,肥沃的地力消耗得极其猛烈。

方源宙道分身神情平静,这一切都在他的估料之中。

太泽花的栽培方法,不管是红莲真传,还是影宗真传中都有详细的记载,这些变化都在方源的安排之内。

很快,从太泽花的花心处,渐渐弥漫出丝丝缕缕的白色雾气。

这些雾气逐渐蔓延,渐渐连成了一片。方源宙道分身停止手头上的杀招,空气立即变得非常湿润。

雾气反哺土壤,渐渐改造这片人造的沼泽地。

“就将这片沼泽命名为花雾太泽吧,第一步已经完成了。”方源宙道分身欣慰地看着。

“有了这样规模的太泽花雾,喂养悔蛊不成问题。”

八转悔蛊食量很大,食物就是太泽花产生的雾气,麻烦的一点在于这些雾气必须要时刻保持新鲜。所以从外面采购,不太现实,也非常不稳定可靠。

“接下来,就是等到雾气浓郁到一定程度,在底部积蓄出一层水。然后,便是引进水草,以及小巧的生物,比如鸭子,乃至飞鱼。”

“然后就是继续扩大太泽花的规模,陆续增添数量。等到悔蛊饥饿的时候,仙窍中积蓄出来的太泽花雾,足够它饱餐一顿了。”

八转仙蛊的喂养,都是一项浩大的工程。

“算上这片花雾太泽,至尊仙窍的开发程度已有百分之十八了。目前仙窍的光阴流速已经减缓了许多,各项资源产出,白荔仙元的积累效率都因此降低了。”

方源心中暗自计算着。

这一次放缓了仙窍中的时间流速,并没有依靠宙道仙招,而是直接调控了年华池,非常的方便。既没有仙元消耗,同时也免除了宙道杀招的某些弊端,方源可以随时调整仙窍时间,极其灵活。

“虽然仙元需要积累,但是不能一味地提快时间,否则灾劫来临的就快了。”

目前为止,方源已经达到了八转,本身战力并未增强许多,之前光阴长河中了天庭埋伏,底牌几乎都暴露光了。

所以,现在渡劫是非常不明智的。一旦落下仙窍渡劫,方源就会成为一个固定的靶子。因为这个时候本身仙窍洞开,被天意察觉,更容易被天庭等势力发现位置而齐攻围剿。

“光阴流速延缓了,但是我还可以经营仙窍,令仙窍底蕴增强,仙窍本源壮大,从而产出更多的白荔仙元。”

仙窍经营的情况越好,产出的仙元就越多。

很久之前,方源经营仙窍,是为了喂养仙蛊。之后经营仙窍,是加大产出,赚取仙元石等修行资源。如今他更多的目的,是为了积累白荔仙元。

单纯用仙元石来凝聚白荔仙元,效率太低下了。一万块仙元石才能凝聚出一颗白荔仙元。

而方源本身的仙窍每年都能产出上百颗的白荔仙元,按照如今的仙窍光阴流速,换算成五域外界的时间,就是大半个月。

时间在方源的潜修中不断往前,五域到了夏季。

虽然没有方源的踪迹,但整个五域蛊仙界都时刻保持在一种沸腾的状态。

不管是正道、魔道,还是散仙,都闹哄哄的。

造成这个情况的原因,就是五域地脉的震荡。

全新的地沟不断地出现,最初是在南疆显现,但如今这种情况在其他四域也发生得十分频繁了。

轰隆隆……

烟尘四起,洪易力竭,一屁股坐在地上,望着眼前巨大的沟壑,脸色惨白,气喘吁吁,一动不动。

地脉震动的时候,他和其他的蛊师就开始了狂奔。

这是一场惊险至极的逃生之路!

其中有好几次,都是洪易运气好,才化险为夷地捡回一条命。

最终,他差一点就要陷进去,幸好关键时刻,他超长发挥,奋力一扑,这才扑上了悬崖,没有跌进深不可测的地沟中。

“什么时候我们这里也出现地震了!好险啊,我差点就丧命在这里了。”良久,洪易气喘吁吁地艰难站起来,他俯视地沟,只看到一片黑暗,呼呼的风回荡在地沟中,烟尘气息还十分浓郁,没有散去。

“这段时间,似乎中洲各地都在发生地震,产生这种地沟啊。”洪易身为凡人,也感受到了这种天地剧变。

“咦?那是什么?”从烟尘中竟缓缓飞出一只野生的仙蛊来。

一只受伤的野生仙蛊,艰难飞行,摇摇晃晃,最终力竭,落到了洪易的脚边。

洪易目瞪口呆!

天庭。

紫薇仙子娇喝一声,浑身上下暴射出夺目的亮紫光辉。

随后,整个天庭竟然也开始微微颤抖起来,一股庞大到无以伦比的无形力量,灌注到偌大的仙阵之中。

仙阵轰鸣,对着中央关押的魔尊幽魂开始了碾磨。

魔尊幽魂极力抵抗,但抵挡不住,最深处的记忆被搜刮出来。

“妙,纯梦求真体的奥妙我终于得到了,哈哈哈。”紫薇仙子满脸憔悴和苍白,神情却非常兴奋。

这段日子她非常辛苦,一方面要面对地脉翻动,中洲四面八方出现的意外状况,另一方面她要加大力度,对付魔尊幽魂。

为了尽快获得影宗的梦道成果,她更是费劲思量,引动了整个天庭洞天的威能,加持在仙阵上面。

功夫不负有心人,紫薇仙子终于得到了突破性的进展!

一转眼,到了盛夏。阳光炙热,也不抵蛊仙们心中的火热。

地脉震荡的情况,没有丝毫的停息和减缓,愈演愈烈。不时出现的仙材,野生仙蛊,各种各样的修行资源,引发了整个五域蛊仙界的欢畅盛宴。

五相洞天中,忽然回荡起方源的笑声。

“终于,天相杀招被我彻底掌控了!”方源的身边,白鹤形态的九转杀招乖巧无比,安安静静地站立着,一举一动都听凭方源的念头。

“没想到天相杀招的效用,居然是这个。”

“有了它,说不定我的修为还能够突飞猛进,并且灾劫的问题也能够得到解决呢!”

方源抚摸着白鹤的羽翅,双眼微微放光。

这个天相杀招来得相当及时!

\end{this_body}

