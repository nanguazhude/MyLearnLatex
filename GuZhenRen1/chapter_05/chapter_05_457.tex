\newsection{铜龙鱼}    %第四百五十八节:铜龙鱼

\begin{this_body}

%1
这只仙蛊形似蝉虫,它的头部、腹部是棕黄色的,表面上有树木年轮般的纹理。背部的双翼很宽大,半透明,就好像是两片树叶交叠着。

%2
它无时无刻不在散发着六转仙蛊的气息。

%3
它正是六转宙道仙蛊春秋蝉。

%4
此时的春秋蝉,处于仙僵肉身的空窍之中,遭受着方源的封印。

%5
它的身躯,散发着温润的油光。它的双翅,嫩绿嫩绿,好像是两片刚抽芽的鲜嫩树叶。这一切都表明,春秋蝉状态已经修复完好,可以再一次催用。

%6
若是刚刚重生,春秋蝉的躯干没有一点光华,显得粗糙黯淡,如同枯木。它的双翼也会充满枯黄之色,像是秋天即将凋零的枯萎叶子。并且翅尖都微微卷起,有残缺,就如同落叶的边角。

%7
春秋蝉来历极大,乃是人族历史上十大尊者之一的红莲魔尊,亲自开创出来的仙蛊。并且还是红莲魔尊的本命蛊,地位极其重要。

%8
它的效用,就是让蛊仙献祭所有,化为穿梭光阴长河的动力,载着蛊仙仅有的一团意志,逆流而上,重生到过去。

%9
有利有弊,春秋蝉威能可谓绝妙,但同时也有着巨大的弊端。

%10
那就是一旦成为春秋蝉的主人,自身的运气就会时时刻刻遭受削弱。简单来讲,就是春秋蝉会召来霉运。

%11
此时此刻,方源看着春秋蝉,感慨万千。

%12
这是前世今生的维系,没有春秋蝉就没有现在的方源。

%13
若是要问一个问题,哪知仙蛊对蛊仙的意义最为重大?这个问题显然因人而异,每一位蛊仙都有属于自己的答案。但对方源而言,他的答案只有一个,那就是春秋蝉!

%14
前世不提,单说今生。

%15
在青茅山、三晚山,王庭福地,春秋蝉屡次展露威能,不愧红莲魔尊本命蛊的声威,屡屡让方源在绝境中翻盘,脱离险境,逃出生天。

%16
不过,在义天山大战之后,春秋蝉一度丢失。虽然方源最终再次获得春秋蝉,但是却已然得知,春秋蝉中蕴藏天意,不能使用了。

%17
方源脱离天意的掌控,与天为敌,带着天意的春秋蝉,直接被列入了完全禁止使用的名单当中。

%18
要再次让春秋蝉的价值发挥出来,方源就必须驱除春秋蝉中的天意。不驱除,将来不得已重生,根本就没有希望,天意就会直接搞死方源。

%19
“现在就让我看看这一杀招如何吧。”方源口中呢喃一声,眼中暴射出阵阵精芒。

%20
仙道杀招天消意散!

%21
方源酝酿片刻,浑身上下都笼罩着一层灰白色的光晕。

%22
随后,他用覆盖了灰白色光晕的手,轻轻盖在仙僵肉身之上。

%23
灰白光晕仿佛是流水一般,从他的身上一阵流转,最终将力道仙僵肉身也完全覆盖。

%24
灰白光晕持续片刻,开始发生变化。

%25
一只只的眼睛,在灰白光晕中产生。

%26
这些眼睛,五彩斑斓,可谓异彩纷呈。

%27
它们先是在方源的手掌周围出现,随即越来越多,向四周蔓延开来。

%28
很快,这些五彩眼珠覆盖了仙僵肉身。它们不断眨动,每一次眨眼,都变化一种色彩。

%29
随后,五彩眼珠竟蔓延到了空窍之中。

%30
最终,春秋蝉的表面也覆盖了一层微小的彩色眼珠。

%31
彩色眼珠不断地眨动,每一次眨动,都是一次针对天意的绝妙攻击。

%32
春秋蝉开始震动。

%33
五彩眼珠则生生灭灭,方源持续不断地消耗着红枣仙元,让五彩眼珠绵绵不绝。

%34
春秋蝉震动的更加剧烈。

%35
如此震动持续了片刻后,方源便见到春秋蝉的表面,产生一道细微的裂纹。

%36
他顿时心中一沉,连忙停下杀招。

%37
噗。

%38
他身躯陡然一震,如遭电击。杀招停得太快,方源因此受伤。

%39
不过这种轻伤,完全不能和春秋蝉的安危相比。

%40
方源不顾伤势,连忙查看春秋蝉。

%41
春秋蝉上的裂痕其实并不明显,但是方源性情谨慎,心细如发,裂纹刚刚产生,就被他旋即发现。

%42
裂纹的出现,代表春秋蝉受了伤。

%43
几乎所有的蛊虫都是很脆弱的,不管是六转、七转乃至九转。春秋蝉即便是红莲魔尊的本命蛊,也不例外。

%44
仔细检查了一番之后,方源面沉如水。

%45
“天意……”方源咬着牙道。

%46
春秋蝉的伤势并不重,得益于他的及时发现,这点微小的伤,甚至称不上轻伤,所以很好解决,

%47
但是,要继续进行天消意散杀招就不行了。

%48
这样下去,就会让春秋蝉上的裂纹越来越大,越来越多,直至将它摧毁。

%49
天意对春秋蝉的影响,远比方源原先料想的,要严重得多。

%50
方源沉思片刻,想到了解决之法。

%51
现在摆在他面前的,有两条路。

%52
第一条路,就是继续改良仙道杀招天消意散,令它能够铲除天意,但又不危害春秋蝉。

%53
第二条路,则是影宗的法门,动用炼道手段,先将春秋蝉升炼,再逆炼。在炼蛊的过程中,将天意驱除。

%54
这两个方法,都有弊端。

%55
改良天消意散杀招,比较困难。事实上,方源能推算出天消意散,已经非常不容易,接近了他的能力极限。要在这个方面,继续做出突破,并非不可能,但花费的时间、精力、物力都会很巨大。

%56
而影宗的炼蛊之法,原本只是一个设想,真正要实现起来也很难。

%57
天意对春秋蝉如此的影响程度如此之深,再加上炼蛊本身就很有风险,在这个过程中,只要天意稍稍捣乱,就会频出意外。

%58
“看来只有将这两种方法结合起来,方是最稳妥的解决方法了。”方源思索良久,决定双管齐下,齐头并进。

%59
方源雷厉风行,很快干到智慧蛊的面前。

%60
“这一次又要靠你了。”

%61
智慧蛊围绕着方源盘旋一阵,然后默默散发出智慧光晕。

%62
沐浴在智慧光晕之中,方源驾轻就熟地催动起多种智道手段。他的脑海中,霎时间掀起惊涛巨浪,无数的念头此起彼伏,浩荡不绝,相互碰撞,激起无数的火花。

%63
红枣仙元迅速消耗,方源的额头很快就满布汗滴。

%64
推算的难度很高,单从方源渐渐锁紧的眉头就可见一斑。

%65
一炷香、两柱香……足足八炷香之后,方源这才停歇下来。

%66
他吐出一口浊气,满脸苍白。

%67
这一次智道推算,他竭尽全力,现在整个脑袋都隐隐作疼,魂魄黯淡,眼前一阵阵发黑。

%68
修养了一会儿工夫,方源恢复了一些状态,又继续利用智慧光晕进行推算。

%69
推算非常辛苦,智道宗师的境界也不能帮助方源什么,因为这已经超出了他的能力极限。

%70
方源是硬生生靠着智慧光晕,来做突破。

%71
不过方源性情坚忍,岿然不动,始终如一,根本没有任何退缩的念头产生。

%72
他甚至将魂道修行、杀招的练习,都摆放一旁,绝大多数的时间、精力,都投入到智道推算上。

%73
几乎不眠不休,但如此努力,进展仍旧缓慢。

%74
但方源仍旧坚持着。

%75
“只要有进展就可以,每一天的进展虽小,但是几天累积下来,十几天累积下来,几个月累积下来呢?”

%76
抱着这样的恒心和毅力,方源对这个难题全力攻坚。

%77
在此期间,他的龙鱼生意方面,却是渐渐有了突破。

%78
至尊仙窍,龙鳞海域。

%79
无数的龙鱼,成群结队,在海水中遨游。

%80
龙鱼形似红色鲤鱼,但它比鲤鱼要庞大得多。荒兽龙鱼体型则更加庞大,乃是水中巨物。

%81
不过龙鱼的战斗力,是公认的垫底货色,完全不值得期待。

%82
经过这一段时间的发酵,方源的龙鳞海域中,却是出现了一种奇特的龙鱼。

%83
这种龙鱼体型比正常龙鱼,要稍小一些,同时鱼鳞的色泽更黯淡一些,有着金属质感,宛若暗红的铜。

%84
“铜龙鱼终于繁衍出来了。”方源查看一番,相当满意。

%85
龙鱼本来就非天地自然物种,而是食道蛊仙所创。

%86
创造出龙鱼的,不是别人,正是开创食道流派的那位兽人祖师爷。

%87
方源从影宗这里,继承了这位兽人蛊仙的传承,自然对龙鱼了解很深。

%88
市面上的龙鱼,虽然有食道道痕,但是还掺杂了其他流派的异种道痕,杂七杂八。但是若采用了传承中的法门加以培养,就能一代代的优化,产生更优质的龙鱼品种。

%89
就像眼前的铜龙鱼,就是优于普通龙鱼的全新品种。

%90
造成这种情况发生的,就是方源辛辛苦苦铺设下去的食道仙级蛊阵!

%91
方源见铜龙鱼已经有了不少,他便将其中一部分,投放到宝黄天市场中去。

%92
“若是我才刚刚做龙鱼生意,这么一小部分铜龙鱼放在市场上,恐怕无人问津。”

%93
“不过前一段时间,我已经开始贩卖龙鱼,甚至还引起了尤婵的反击,其余蛊仙都知道了我也做龙鱼生意。”

%94
“呵呵。”方源想到这里,笑了笑。

%95
他不用猜都能预料到,这些铜龙鱼进入宝黄天市场后,会引发出什么样的反响。

%96
“尤婵啊,这才是我真正的依仗,你该如何应对呢?”方源不禁有些期待起来。

\end{this_body}

