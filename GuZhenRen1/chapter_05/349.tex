\newsection{兄长,快借我仙蛊}    %第三百四十九节:兄长,快借我仙蛊

\begin{this_body}

见到这样一幕,方源都不由自主地瞪圆了眼珠。

九转尊者,盖世声威,早已经宛若血液一般,流淌在了任何一位蛊师、蛊仙的心中。

但现在,方源却亲眼见证了魔尊幽魂的下场,竟然被一位八转蛊仙俘虏了。

“这龙公到底是何来历?竟然如此强大!”

“不过,仔细想想的话,幽魂魔尊早已陨落,留下的是魂魄而已。”

“他的魂魄若是巅峰状态,龙公肯定收拾不了他。可惜的是,魔尊幽魂硬抗灾劫,不计牺牲地强炼至尊仙胎蛊。虽然最终成功,击溃了监天塔,但是胜利的果实被我抢走,影宗剩下大小猫三两只,僵盟全灭,损失极其惨重。”

“而他魂魄本体,在义天山大战时,就只剩下一缕残魂。又陷入梦境当中,经过这么长时间的梦境侵蚀,说实在话,能够存活到现在,已经是一个匪夷所思的事情了。”

方源设想自己,若是自己的魂魄被丢入这么庞大的梦境中,遭受梦境的消磨和侵蚀。在不动用任何梦道杀招的情况下,方源恐怕只能支撑数天的时间吧。

魔尊幽魂虽然只剩下一缕残魂,但是质地惊人,硬生生捱过这么长时间的消磨。到了现在,仍旧有挣扎之力。

龙公能俘虏幽魂本体,并不奇怪。

双方实力之间,的确是龙公强盛,幽魂弱小。

幽魂的时代,终究是过去了!

他曾经屠戮天下,将魔威遍及五域两天,万马齐喑,世间混乱。

在他生前,他无敌于天下,是天地间最强的声音,最高的身姿。

可是,时光如流水,他终究还是死了。

死后,又有乐土仙尊出世,大大消弭了幽魂魔尊的世间影响,为整个天地带来和平和福祉。

幽魂即便死后,也不甘心失败。

他建立影宗、僵盟,企图再度逆天而行。

剑仙薄青是他的第一次尝试,可惜最终只博得一个亚仙尊的名号,尸沉落天河底。

至尊仙胎蛊是他的第二次尝试,然而被天地算计,十万年心血成果被方源抢夺。

义天山的失败,让魔尊幽魂惨白亏输,几乎耗干了底子。但就是凭借影宗的那些残余力量,仍旧掀起了许多风云。

直至现在,魔尊幽魂终于彻底失败,被天庭蛊仙龙公擒拿。

“大时代就要来临,你这等老古董,就乖乖地缩在一旁观看吧。幽魂啊,这已经不再是你的时代了。”

龙公感慨一声,伸手一抓,气流汹涌,夹裹关押着魔尊幽魂,塞入他的虚窍当中去。

“啊——!”紫山真君仰天嘶吼,双目赤红。

看着幽魂本体陷入囹圄,差点要丧失理智。

连本体都陷落了,一切都没了希望。

大败亏输!

但紫山真君终究是智道蛊仙,他硬生生地忍耐住寻找龙公拼命的冲动。

那是自暴自弃!

他深呼吸一口气,开始向影宗福地深处撤退。

“跑?你还想往哪里跑?”龙公伫立在高空中,龙瞳遥望着紫山真君的身影,神情冷漠。

影宗福地的仙窍门户直接关闭,魂兽大军也不再冲出去,而是留守回防。

龙公傲然一笑,转身凝望了超级蛊阵一眼,将紫薇仙子进展良好,他便对准影宗福地暴露出来的位置,催动一记仙道杀招——

龙门!

这道龙门,器宇轩昂,两边巨柱赤红,柱子上各有一条巨龙盘旋。龙爪扣在门柱之上,稳固如山。闪闪龙鳞,龙躯矫健。龙身围绕巨柱,一直超过巨柱的顶,然后两头巨龙的上半身,相互对应,龙睛互视,龙嘴张开,中间一颗龙族,宛若小太阳般,灼灼耀眼,形成双龙戏珠的格局。

龙门洞开,竟显露出影宗福地的内里景象。

这仙道杀招龙门,竟然能强行打开仙窍福地!

龙公大笑一声,抬脚跨入进去,龙门旋即轰然闭合,随后消失不见。

福地中的激战和厮杀,都传不出来,整个战场忽然陷入到诡异的静谧当中。

但可想而知,龙公进入影宗福地绝不平静,那么多的魂兽将会对他展开悍不畏死的冲锋!

“怎么办?”南疆正道蛊仙们面面相觑。

忽然间,他们发现自己已经没有了对手。

紫山真君和龙公,齐齐陷入到影宗福地中去。白凝冰、黑楼兰、影无邪等人都随着纯梦求真体,躲入了梦境当中。

唯一穿行在战场中的,就是魂兽。

许多魂兽,在战场中肆意奔腾。

他们中有一部分,陷入到了梦境当中去,不能自拔,很快就被梦境消弭。

大多数的魂兽则看出了梦境的危险,因为求生的本能,开始四散奔走。

魂兽依据本能,捕猎一切拥有魂魄的生命,通过噬魂,而借此壮大自身。

南疆正道蛊仙们,很快就成为了这些魂兽的目标,尽管他们并不想动手。

“武遗海,快将我们送入蛊阵中来。”南疆正道蛊仙们纷纷提出这样的要求。

他们没有想走的意思。

因为此时,影宗大败,天庭得胜的可能很大。而天庭一直都并未对南疆正道,显露过杀机。

南疆正道蛊仙们,在潜意识中认为,天庭是和他们一路的。

毕竟都是正道。一个是中洲正道,一个是南疆正道。

当然,还有一个更重要的心理因素,影响着他们的判断。

那就是梦境。

这片梦境虽然被搞的乱七八糟,但是在南疆蛊仙看来,都是他们的资源,是他们的囊中之物。

况且,他们亲眼目睹了龙公和紫山真君、幽魂本体的交锋,终于见识到了义天山大战的一抹真相。

事关九转尊者这样的大秘密,说不定还有大机缘,谁想走?

这里可是有着南疆的超级蛊阵,而且各大家族的援兵,正在迅速赶赴这里。

至于监天塔和左夜灰,它们一直在死磕着,谁也奈何不了谁。

方源却下令:“大家再努力一些,杀掉这些魂兽!家族的援军已经越来越近了。这些魂兽是敌人送出来的,杀掉它们,等若断敌人一臂。若是放任这些魂兽离开,就要引发祸乱,始终是我们正道的事情!”

这番话虽然有理,但很多南疆正道蛊仙并不买账。

“不是援军就要到了吗?就让他们动手吧。”

“是啊,我已经很累了,需要休养。”

“这一次即便没有受伤,但是仙元损耗极大,还是积蓄战力,不要随意耗用的好。”

方源冷哼一声:“让你们去就去,废话太多。”

南疆正道蛊仙纷纷破口大骂,但方源充耳不闻。

他正在干一件很激动人心的事情。

那就是向武家求援。

“快,我这边还是很危险,局面很复杂,快将我要的蛊虫,都通过宝黄天送过来!”

“不要担心,这里面的费用,都我一力承担。”

“我要血脉仙蛊做什么?我当然有急用,我需要掺和进蛊阵当中去,增添超级蛊阵的威能!”

“一下子借这么多仙蛊,让你为难?你为难什么?我以前又不是没有借过!”

“什么?借用武家内库的仙蛊,需要族长大人亲口批准?”

方源便立即联系武庸。

“兄长,快快助我一臂之力!”

武庸正在操纵着仙蛊屋,极速奔赴而来,听到自家弟弟联系他,立即回应:“发生了什么事情?你又需要什么帮助?”

“情况你想必也知道不少了,兄长,我现在掌控着超级蛊阵。这是一个最佳的机会!可以让我武家真正掌控这个资源点,打压其他家族,一扫之前困境的机会啊!”方源情真意切。

武庸沉吟了一下。

这片战场并没有封锁,南疆正道蛊仙始终能通过信道手段,沟通家族,不断求援。

武家这边,也只是方源在这里,还有武辽,还有乔家蛊仙。

所以,各方援军对于这片战场的新动态,都很了解,武庸也不例外。

方源在关键时刻救场,顶替池规,建成超级蛊阵的消息,武庸当然知道。

第一次听说时,他还有些惊奇,没有料到自家弟弟居然有如此强大的阵道造诣。但是武庸很快又释然。

为什么呢?

因为在此之前,方源就曾和池伤比试过阵道,因为其中夹杂着乔丝柳仙子,导致这样的绯闻,在南疆蛊仙界曾轰动一时。

方源暴露出来的阵道造诣,让人吃惊,但经过这个打底之后,南疆蛊仙接受起来,也不那么困难。

若是武遗海这个身份,是南疆土生土长的蛊仙,那么势必引发广大怀疑。妙就妙在,武遗海是东海升仙,很多情报南疆蛊仙都不清楚。

这就让方源有了充分的表演空间。

武庸一听方源说:可以利用名缰仙蛊,来加固超级蛊阵,为武家谋利。

他犹豫了。

但是他的这个犹豫,不掺杂任何的怀疑成分!

他相信方源。

始终没有怀疑过武遗海,会是他人假扮。

他犹豫的是,方源的这份方案靠谱不靠谱,成功的可能大不大?若是不大,施展失败了,那么武家将来的政治格局会怎样?

这些都是武庸要考虑的,身为上位者,必须得高瞻远瞩,不能只看眼前。(未完待续。)

\end{this_body}

