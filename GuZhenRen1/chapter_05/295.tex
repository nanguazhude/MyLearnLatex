\newsection{说服老怪}    %第二百九十五节:说服老怪

\begin{this_body}

南疆。\&\#26825;\&\#33457;\&\#31958;\&\#23567;\&\#35828;\&\#32593;\&\#119;\&\#119;\&\#119;\&\#46;\&\#109;\&\#105;\&\#97;\&\#110;\&\#104;\&\#117;\&\#97;\&\#116;\&\#97;\&\#110;\&\#103;\&\#46;\&\#99;\&\#99;]

细雨纷飞,青山连绵。

方源一身蓝衣,站在山石上,眺望着这大好河山,最终他将目光集中在了其中一座山上。

这座山很特别。

旁边的山峦,都是或高或矮,或险或缓。

而这座山峰,却是圆润有形。

这就是螺母山。

或者说是山螺母。

它本身就是太古荒兽!一头巨大的田螺,贝壳高耸宛若高塔,一层层螺旋上去,青苔满布。

当它静置不动的时候,它的软足肢体,都缩进贝壳当中,就仿佛是一座山峦,高高耸立。

现在它已经苏醒,正在缓慢的移动。

隆隆隆隆……

它移动的时候,方圆百里的大地都在微微的颤抖着,所到之处,鸟兽竞相奔走。

方源站在另一处山巅,看着这头太古荒兽山螺母迁徙,神情平淡。

它的速度很慢。

山螺母虽然是太古荒兽,但性情木讷温和,很难把它激怒,它可以说是无害的。

不仅无害,而且有益。

因为山螺母的身上,蕴含着丰富的奴道、土道道痕,导致它身边会形成特殊的环境,孕养出海量的奴道、土道资源。

山螺母的归属,一直是个悬而未决的话题。

很少有蛊仙,能控制山螺母,就算是八转奴道蛊仙也够呛。

因为山螺母本身,就有极其丰富的奴道道痕。

南疆这块地方,已知的山螺母有七头。很久很以前,南疆的超级势力便相互约定:这些山螺母只要不受他人控制,也就是野生的山螺母,它位于谁家地盘,就是谁家资产。

这头山螺母原先位于武家地盘当中,自然是归属于武家。

这点毫无疑问。

但是如今,到了山螺母迁徙的时候,它正在远离武家地盘,向外面挪动。9;\&\#32;\&\#25552;\&\#20379;\&\#84;\&\#120;\&\#116;\&\#20813;\&\#36153;\&\#19979;\&\#36733;\&\#65289;

山螺母本身的位置,就在武家的边界线上。

事实上,这还是武独秀时期,武家特意扩张地盘,将山螺母囊括进来。

现在山螺母迁徙,要离开武家的地盘,但它去往的地方,并没有其他正道超级势力。

这也就意味着,这头山螺母将成为野生无主之物。

武庸自然不愿看到这样的情况发生,无奈他手头人手不足,只好将这项任务交给了他的弟弟武遗海。

“可是武遗海么?老夫秦静升,有个外号驱山老怪。”一位蛊仙从另外一边的山巅上,飞了过来。

方源目视来者。

只见这位蛊仙老者,披着一身青灰石甲,头发有黑有白,乱糟糟如稻草一堆。他的衣裳边角,也是破烂不堪,不修边幅。

但是他本身七转气息洋溢,气势迫人,尤其是一双眼睛,闪烁着逼人的精芒,此时盯着方源四下打量,有一种长辈打量晚辈的意味在里面。

“正是在下,武遗海见过驱山前辈。”方源微微一礼,很有正道风范。

这个驱山老怪,名号里带个老字,自然是老资格、老字辈,年岁很大,有两千多年。

他有寿蛊,也有独特的延寿手段。

本身战斗力也是极强,是散仙当中的著名高手,几乎和树翁巴德、武家的武雨伯同一层次。

夏飞快这种七转蛊仙,却是比不上驱山老怪的。

所以,驱山老怪打量方源,态度并不客气,以老前辈自居,哪怕方源是来自武家。

“呵呵呵,算算时间,你们武家也该派人来了。咱们坐着聊吧。”驱山老怪笑了笑,落到山石上,直接一屁股坐了下去。

方源笑了一下,也坐下来。

驱山老怪见方源竟没有嫌弃,也不保持仙家风度,有样学样,不禁好感升起,道:“我倒是差点忘了,武遗海你出身散修,虽然加入了武家,但身上却是没有那些正道的虚伪气派。”

方源微微摇头:“正道自有正道的风度,至于我,大部分时间都是在东海隐修,这个根子是改不了了。”

“哈哈哈。”驱山老怪大笑,“我听说你虽然赶跑了夏家两位蛊仙,却还把广寒峰搜罗了一番。”

方源面露诧异之色:“前辈,这是空穴来风,流言碎语,我可从未承认过。”

驱山老怪再次大笑三声,笑完之后,又深深一叹:“正道蛊仙不理解你,老夫却很理解。咱们散修能够修行成仙多不容易?每一份修行的资源,都是自己双手挣来的。过了这个村,就没有这个店,所以当前的好处都要尽全力把握。谁知道今后会遭遇什么?”

“至于那些正道蛊仙,就算自己待在家里不动,也有大把的资源供给。他们又岂能知晓我散修的苦楚和难处?”

方源面色淡淡,没有赞同,也没有反驳。

他知道:驱山老怪这番话,绝不仅仅只是和自己套近乎,而是表明自己的决心他驱山老怪是散修,近在眼前的好处就是螺母山。他的决心不可动摇,曾经身为散修的武遗海应当清楚才是。

“我当然明白前辈的心意,不过也请前辈多多理解在下,在下此行前来,亦有自己的难处。”方源道。

驱山老怪嘿了一声:“人人都有难处。”

人老了,说什么话,都带着深意。

人人都有难处,不仅是附和方源刚刚的话,也是表明他自己也有难处,更关键的是武家目前正在困难的处境之中。

方源不再说话。

言语间的交锋,有时候沉默是一种手段。

果然,驱山老怪按捺不住,交给方源一只信道凡蛊:“你先看看。”

方源接过一看,这凡蛊当中的内容,价值不菲。方源若是得之,直接可以建设盘丝洞窟了!

“你我合作,我得山螺母,你得这些,如何?”驱山老怪笑着道。

他竟是打着贿赂方源的主意!

但这一点都不奇怪。

反而一直都是在方源的意料之中。

驱山老怪身为散修,站在他的这个角度,自然是不想得罪第一正道势力武家的。尽管武家目前处于一种难堪困苦的境地,但这样的庞然大物,也不是他驱山老怪一个人可以撼动的。

更关键的是,驱山老怪若是得到了这座螺母山,就不再是一个自由进退的人物了。

螺母山是不能收入仙窍当中的,除非是将它彻底控制。

驱山老怪乃是土道蛊仙,显然没有这样的造诣。如此一来,他要成为这座螺母山的“主人”,将来若有敌人攻击,他就要据山防御。

若是他得罪武家,武家蛊仙陆续来攻,单凭他一人,如何能抵挡得住?

就算能挡住一次,那么两次、三次呢?

就算都能挡得住,那作战总得有耗损吧?驱山老怪遭受屡次骚扰,还有时间修行吗?

所以,摆在驱山老怪眼前的,只有一条路。

那就是和武家和平谈判。

武家是不愿意放走这座螺母山的,但是偌大的武家,也是由一位位的家族成员组成的。只有是一个组织,都绝不是密不透风的堡垒,定然是有缝隙,可以专营的。

武遗海的出现,带给了驱山老怪一个更大的希望。

“老前辈难道就想以这些,换取一头太古荒兽吗?”方源拿捏着信道凡蛊,似笑非笑。

“这些已经价值非凡!这螺母山虽是一头太古荒兽,但谁能驾驭?我若成为此山之主,也不过是搜刮这里产出的资源罢了。当然,有什么条件,咱们可以再商量。”驱山老怪双眼放光,态度也变得热情多了。

方源既然没有当面拒绝,让驱山老怪心中大叫有门,希望更增一倍。

“这清单上的内容,至少得翻一倍。”这是方源的第一句话。

驱山老怪顿时皱起眉头。

“螺母山名义上,也不能是你的,仍旧份属于武家。”这是方源的第二句话。

驱山老怪瞪大双眼,脸色骤变,流露出怒意。

“但在实质上,前辈你是此山主人,收益大半都归你,但每年都仍旧需要上缴一部分,交给我们武家。”这是方源的第三句话。

驱山老怪开始冷笑,神情阴沉如水:“武遗海,你这是在消遣老夫么?”

“当然不是。”方源满脸严肃之色,他站起身来,毫无畏惧地和驱山老怪对视,“敢问老前辈三个问题?”

驱山老怪强自按捺心中怒气:“你说。”

“第一,前辈孤家寡人,并非任何超级势力,我武家怎可能将螺母山留在你手?超级势力也就罢了,让一位散修得益,可是关乎我武家声名。”

“第二,前辈就算暂时霸占了这座螺母山,又能守护多久?就算我武家不动你,其他超级势力呢?”

“第三,前辈以为单单贿赂了我,就能让整个武家坐视不管了么?”

驱山老怪沉默。

好半天,他才叹息一声,带着一丝沙哑,开口道:“那照你的法子,老夫岂不是成了你武家的看山人?”

方源淡淡一笑:“不过是点滴虚名罢了。凡事怎可能全美?我提的这些,其实并不过分,我也是看在大家同为散修的份上。要面子,还是要里子,老前辈尽管好好考虑一下。”

驱山老怪再度沉默。

这一次过了更长的时间,一直到天空中的细雨渐渐停下,他这才开口:“也罢了,就按照你说的办。”(未完待续。)

------------

半年总结,有些心里话想说!

呼……《蛊真人》写了四五年了。[求书网qiushu.cc更新快,网站页面清爽,广告少,无弹窗,最喜欢这种网站了,一定要好评]

此中的过程,坎坎坷坷,真心不容易。

期间有过断更,周更、月更的情况时常发生。但是我坚持下来了,哪怕是根本不赚钱,我都坚持写下来了。

我写这本书,图谋更多的是一种梦想,是想写一个与众不同的主角,一个比较特别的小说。

期间的谩骂、诅咒、举报,不胜枚举,我从之前的生气郁闷,到愤怒,再到哭笑不得,现在是平常心了。

哦,被举报了。

哦,又骂我了。

哦,又不满意了。

这也算是一种成长吧。

今年上半年,这本书成绩有所起色,我也终于获得了一些收入,当然和那些作者们是不能相提并论的。

一方面是我更新增多,另一方面是亲爱的小伙伴们一起努力投票,一起摇旗呐喊,一起加油的结果!

《蛊真人》这本书不仅仅是我的,同样是大家的。

感谢大家,真心谢谢了!

谢谢我的盟主,谢谢宗师、掌门、长老、舵主,谢谢一切支持正版阅读的朋友们。

谢谢给我微信公众号点广告的朋友们。

谢谢打赏的朋友们。

谢谢贴吧、qq群、微信群,给我意见,给我加油,关心我,关心这本书的朋友们。

谢谢给我投票的伙伴们,我知道很多人甚至开多个小号,给我投过票。

还要谢谢在这段时间里,从盗版转为正版的诸多朋友们,你们的这种改变,对我而言,有一种感动人心的力量!!

我在写书方面,还有许多不成熟。但大家一直都在包容着我,期待着我,我当然不想让大家伙失望。

我还会继续写下去。

用尽力气,拼尽心思。

有时候会改大纲,只是为了将心中的这个梦想,诠释得更加完美。

未来一定还会有坎坷,有起伏,有波折。

但是我会坚持下去,尽我最大所能,直至完成这部《蛊真人》!

然后,很多年过去。

我们再回头看。

《蛊真人》这本书就在那里,静静的。

不管伟大还是渺小,就在那里。

也许它那时已无人问津,如同青春逝去,蒙上灰尘。

但它独树一帜,与众不同。

曾经带给我们欢笑和泪水,当然也有牢骚和皱眉。

然后,无意中提及或者想起。

我们兴许会微微一笑。

如果能够得到大家的会心一笑,我的此生也就满足了!

谢谢大家!!

蛊真人

2016.7.1(未完待续。)

\end{this_body}

