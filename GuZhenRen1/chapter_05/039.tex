\newsection{罐河察运}    %第三十九节:罐河察运

\begin{this_body}

罐河,宛若一条修长的玉带,雕缀在中洲东部之中。

罐河的水,永远是这么舒缓的流淌,仿若一首优美长歌,曲调悠扬柔转。

影无邪站在罐河河堤,等待风禅子的到来。

而黑楼兰、石奴、太白云生等人,则隐藏在附近,为影无邪保驾护航。

河水潺潺流淌,河面上一两只蜻蜓,时而悬飞,时而停在罐草的尖端。

罐河中生长着许多的罐草。

这种草,根扎在河底泥泞之中,汲取营养。茎修长而又柔韧,支撑着草尖,探出河面。

罐草的草尖,结着一颗棕褐色的果实。

这种果实此时还只是拳头大小,没有成熟。果实是中空的,并不密合,顶部张开大大的口。

正巧这时,一只蜻蜓,循着果实内部的香味,莽撞地飞进果实之中。

它挣扎了几下,就被浓郁的香味迷昏,再也飞不出来。

影无邪静静地看着这一幕。

他知道:等到再过几个月,果实越长越大,会吸引更多的飞虫、水虫,甚至是小鱼小虾。随着生长的过程,果实的开口渐渐闭合。完全成熟之后,罐草草尖像是吊着一个褐色的瓦罐,彻底密封起来。

然后罐河就迎来每年一次的汛期。

汛期中,罐河一改温和的景象,水流汹涌湍急,将无数生长在河水中的罐草冲刷。一个个的褐色罐果,就被水流带动,脱离草茎。随波逐流。

罐河两岸,蛊师们会纷纷行动。打捞罐草的果实。

敲碎它们,就能得到鱼虫。还有罐草的种子。更有一定的概率,会在罐草中产生蛊虫来。

这些蛊虫,在罐草中陷入沉眠,能被人轻易炼化。

十大古派的超级势力风云府,掌管罐河,每到这个时候,都会组织一场罐河祭典。吸引无数蛊师,男女老少,不仅是风云府中的弟子。还有周围大大小小的势力,热闹非凡。

影无邪看着河面上的罐草,目光却是离散游荡的。

为了搞到太古之光,他在罐河附近已经停留了一段时间了。

表面上他云淡风轻,平静如水,实际上心中却积压着许多焦躁,并且这种焦躁不安随着时间的积累,越发浓重。

“我身上有春秋蝉,它已经被封印住。无法运用。在我重炼之前,天意能时刻关注着我。天意不仅会悄无声息,不动声色地影响我的思绪,还会布局。影响其他存在,来围剿我。我不能给天意从容布局的时间,停留在某个地方。绝不能太久。如果这一次,风禅子再爽约的话。我就只好放弃这条线了。”

影无邪心中暗下决心。

就在这时,一道青绿色的身影。紧贴着罐河宽阔的河面,急速低飞,向影无邪迅速接近。

“是那头六转草头神!”

“提高警惕,防止意外。”

影宗众仙暗中传音,纷纷打起精神。

草头神再见到影无邪,态度倨傲,昂首道:“太古之光就在我的手中,但主人说了,要换可以,但之前的价格不行,你们还得在这个基础上,提价三成!”

“什么?居然还要提价三成?”

“这是狮子大开口,他风禅子还真敢提啊。”

“本来价格就已经很高了,这要提高三成,实在太不划算!”

石奴、太白云生暗中讨论。

影无邪其实心中也颇为恼火,但没有办法,他实在太需要太古之光了。要炼制定仙游,必须需要太古之光。这是一个绕不开的坎儿。

沉默了一下,影无邪点头:“我答应你。”

草头神冷哼一声:“那就开始交易吧。”

影无邪早就做了充分准备,双方很快交易完毕。

草头神仔细清点之后,神情露出一丝满意。

影无邪损失惨重,付出代价颇高,但他千盼万盼,终于将太古之光弄到手中,心中还是欢喜更多一些。

他对草头神笑道:“这次交易大家都很愉快,将来有机会咱们可以再合作。”

草头神:“哼,再合作?那就得看我家主人心情了。其实本来这次,主人是不打算和你们交易的,但不久前他和浪子秋打赌输了,想要翻本,才派我前来。你们可要死死记住,这太古之光绝不是我家主人给你们的!”

“明白,明白。其实贵府家大业大,库藏丰富,可称得上浩如烟海。少一件小小的太古之光,算得了什么?根本不会有人察觉的。”影无邪恭维道。

“可以了,此事已成,你速速离开,不可在这里逗留太久。”草头神最后关照一声,转身离去。

其实不用它提醒,影无邪已经决定要立即离开!

隐形匿迹,一直飞出十多万里后,影宗一行人才落足在一处无名森林之中,进行短暂休整。

“这一次虽然换得太古之光,但风禅子要价太狠,我们付出太多。想要炼出定仙游,还得寻找其他辅助仙材。”石奴道。

“好在太古之光已经得手,接下来就好办得多了。毕竟我们可以借助逆组织。”太白云生提议。

“关键是这份太古之光有多少,能炼多少次?”黑楼兰的话一针见血。

蛊仙要炼制仙蛊,不是光只准备一份仙材就可以的。

不算毛民天地流,单以人族隔绝流的炼法来论,炼制六转仙蛊的成功概率不到百分之一。

从这个理论数据而言,影无邪他们至少要准备一百份仙材。

但实践经验看来,事实更加残酷。

因为成功概率,不是单靠仙材数量多,就能提高的。很多时候。就算超过一百份仙材,也会炼蛊失败。

“这份太古之光颇多。能尝试九次炼制。”影无邪说到这里,顿了一顿。

他暗中催动体内一只仙蛊。

六转仙蛊察运!

这只仙蛊。原本在秦百胜身上。机缘巧合之下,因为需要,所以被送到石奴手中,借给他运用。

义天山大战后,影宗失败,这只察运仙蛊便又辗转到了影无邪手中。

这是一只侦查蛊虫,专门用来观察其他存在的运气如何。

影无邪双目散光,抬眼看向自己头顶。

视野中,景象大不相同。

一个运道气柱。如梦似幻,落在影无邪的眼中。

影无邪稍稍意外了一下:“咦?我的运气怎么忽然之间,好转了这么多?”

他早在之前,就查探过自家运道,十分不理想。并且因为春秋蝉的弊端,运气还在不断地衰落下去。

现在这个情形,比他估算的,要好太多了!

“难道说……”影无邪脑海中灵光一闪。

他想到了方源。

又想到了琅琊地灵手中的己运真传。

“上一次,我影宗突袭琅琊福地。抢夺了部分炼炉,还殃及了狗屎运仙蛊,无意间毁掉了它。琅琊地灵就一直准备重炼狗屎运仙蛊。最近正好炼成了。恐怕他是将狗屎运借给了方源。”

影无邪一下子,就想到了这个最大的可能。

几日前。他就知道方源渡劫成功的情报。

因为影宗方面,还残存了一个毛民蛊仙内奸,留在琅琊福地之中。

“恐怕真是如此了!”

“原先方源和几人连运。运道均分。义天山大战之后,我的肉身是方源的。而他保留魂魄,两者仍旧连运。”

“他在那边动用狗屎运仙蛊。增强自身运气。但增幅的运气,好像水从高处流下,灌给叶凡、洪易、韩立、黑楼兰四人。这四人又和我这具肉身运气相连,所以让我讨了个便宜。”

“难怪我之前,等候那么久,都没有成果,希望更是渺茫。这一次,风禅子忽然就打赌输给了浪子秋,急需要资本,所以才和我交易,让我得到了太古之光!”

思考了一番,影无邪发觉,原来方源大有利用价值!

“他的仙窍两个月就要渡一次劫。渡劫前,他必定要用狗屎运,增幅运气,减弱地灾威能。这样一来,我就能因此受益!甚至都不需要去北原,找长生天借运了啊!”

“就算他知道此举资敌,也不得不这样做。因为至尊仙窍的灾劫,恐怖得远超常理,他必须要尽一切可能,争取渡过生死难关!不过……他选择在北部大冰原渡劫,真是英明举措。天意要对付他,千方百计地想要铲除他,必定借助地灾。但在北部大冰原渡劫,却是让狂蛮真意替代了一部分的天意,大大降低了其中的危险。哼,希望你能活下来,你的这具身体,我迟早要重新收回!”

在无名湖畔停留了一会儿,影无邪见无人追踪,再次率领三仙启程。

影无邪行色匆匆。

他要抓紧时间。

不仅是因为本体残魂陷落在梦境中,日夜煎熬,时刻消磨。而且,他要趁着运气好时,炼制定仙游,这会使得成功的可能大大提高!

北原,北部大冰原。

“师傅,我们还有在这里停留多久?”刚刚晋升成功的力道六转蛊仙,弱弱地问道。

霸仙楚度站在冰原上,望着被他拍出掌形深坑的地面,叹了一口气。

他没有回答徒弟的话,而是陷入懊悔之中。

“我之前太过鲁莽了,怎么会想都不想,就直接动手了呢?”

“他既然知道提取狂蛮真意的法门,此法不管是仙道杀招,还是仙蛊屋,我想要得到,绝不容易!”

“就算打杀了他,也要防备他魂魄自爆,否则就竹篮打水一场空,眼睁睁地失去这份天大的问道机缘!”

“我应该更稳妥一些的,至少我应当先礼后兵才是啊!”

“现在对方弥补了漏洞,居于仙窍之中,易守难攻,我却没有突破的手段。”

“亡羊补牢,为时未晚。或许我可以和他商量?对,我手中还有他的剑道仙蛊,这就是谈判的筹码!”(未完待续。)

\end{this_body}

