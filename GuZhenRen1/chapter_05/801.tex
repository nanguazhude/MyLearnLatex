\newsection{天地无正义}    %第八百零四节:天地无正义

\begin{this_body}

一“酒有毒!”陈族长身躯摇晃,愤怒地望着方源:“单副族长,是你献上来的酒,你想干什么?!”

这变化有点突然,方源呆了一呆。

他还未说什么,身边紧挨着他坐着的蛊师,却是站起身来,呵呵冷笑:“做什么?当然是杀了你,除掉陈家的人。这个山从古至今,都是我单家的,不管你们陈家还是聂家,最初时都不过是我单家的奴仆罢了!”

“你放屁!明明是我陈家的山,当年祖先好心好意,先后收容了两支家园被毁,被迫流浪的两个族群。”陈家的蛊师立即高声反驳道。

那位出声的单家蛊师继续冷笑:“恬不知耻的奴仆,和你们多说无用。现在一切都晚了,你们族长喝下了毒酒。这个毒酒可是用妇人心蛊制造出来的,正克住你们族长的正义蛊。没有了你们的族长,你们这些陈家的蛊师谁能斗得过我们单家?”

陈家蛊师们顿时脸色苍白,被说中了软肋。

方源再懵,轻声对身边跳反的单家蛊师道:“谁告诉你,妇人心蛊制造的毒酒,能够克制正义蛊的?”

那位单家蛊师明显是副族长的心腹,闻言也懵了,呆呆地反问道:“啊?大人,不是您说的吗?”

“唉?!”方源目瞪口呆。

轰!

陈家族长全身爆发出冲天的白光,他脸上的苍白和惊怒迅速消散,转而凝聚成满脸严肃。

他怒瞪虎目,死死地盯着方源,正气凛然地大喝道:“单星!你这个卑鄙无耻的小人!居然敢行此阴谋毒计。我真是看错了你,有眼无珠,枉我还以为你真的顾全大局,愿意牺牲自己,将族长大位相让于我。原来你是一直顾忌我的正义蛊,顾忌我的实力,所以才潜藏不发而已。”

方源大感头疼,心中急速思索:“陈家族长有五转修为,而我现在的实力不过四转而已,同时蛊虫也差一个档次,真要打起来,根本不是陈家族长的对手啊。”

他身边的心腹惊恐大叫:“怎、怎么回事?他明明喝了毒酒,但是正义蛊仍旧有效啊!单星大人,您不是说……”

方源顿时眼前一亮,他想到了解决之法。

没错,只要正义蛊失效,陈家族长就会中毒,毒性发作,他实力暴降,自己便完全可以打得过了。

“原来如此。”方源死在陈家族长的攻击下,却是微微带笑。

第二幕梦境再探。

大厅内,灯火辉煌。美酒飘香,佳肴满桌,数十位蛊师济济一堂,推杯换盏,好不热闹。

陈家族长喝了一大口酒,放下空空的酒杯,感叹道:“终于获胜了!贤弟啊,和你说句心里话啊,为兄到现在这才安下心来。自从听闻那聂家居然有着破坏山根的阴谋,我就难以入眠。他们争不过我们两族联手,就要实施如此丧心病狂的计划,不顾生灵涂炭。幸好苍天有眼,多行不义必自毙,这些人终究落到今天的下场,实在是他们自找的。”

方源冷笑一声,轻声道:“这件事情,其实内有隐情。聂家根本没有这样的阴谋毒计,而是我负责编排,故意欺瞒了你。你所杀的那些人,都是无辜的。”

“什么?!”陈家族长大惊失色,难以置信地看着方源,“单贤弟,你是酒喝多了?说的什么疯话!”

方源阴森森地看着陈家族长:“是你太天真,一直被我蒙在鼓里。你是一个刽子手,屠戮了多少无辜的人?你的双手沾满了血腥,你的正义虚有其表。你只是一个莽夫,一个蠢货。还有,是你酒喝多了。”

“什么?!你在酒里下毒?”陈家族长毒发,震惊地捂住腹部,当场瘫倒下去。

陈家蛊师哗然、惊悚,而单家的蛊师早有准备,纷纷痛下杀手。

陈家族长想要抗争,但身躯无力至极,只能眼睁睁地看着。

方源的心腹兴奋地叫喊道:“果然毒酒有效,陈家族长的正义蛊被克制了。”

方源差点忍不住去翻一个白眼。

要催动正义蛊,除了真元之外,还须得蛊师心中坚信自己就是正义的一方。

方源刚刚的一番话,让陈家族长心中一片混乱,怀疑自己,羞愧难当,正义蛊自然催动不了了。

战斗骤然爆发,迅速结束。

单家一方有备而来,陈家上至族长,下至蛊师,都遭了暗算。

等到大厅内彻底平静下来,陈家蛊师已经是死的死,伤的伤,但凡活着一口气的,都被俘虏。

第二幕顺利通过。

第三幕到来。

方源置身囚笼当中,眼前唯一一位的牢友就是那陈家族长。

他浑身伤痕累累,气息微弱至极,脸色惨白如纸,眼眶却是一片紫黑之色,毒性深重。

方源看他伤痕,显然是被严刑拷打,处于生死边缘。

饶是如此,陈家族长还被铁链锁住了四肢和脖颈,摊在地上,有进气没出气。

“这是什么情况?”方源再看自己,顿时有点无语。他也被铁链锁着。身体苍老无力,体瘦如柴,和陈家族长也差不多。

他虽然有着四转的空窍,但空窍已毁,并且什么蛊虫多没有。

“这要我搞什么?我不再是内奸副族长,又换了另外身份。嗯……此刻,牢内除了我,就是他。看来得和他搭上话,问问情况了。”方源干笑两声,正要开口说话,陈家族长已经先出声:“你不必来嘲笑我,爹。”

“爹?!”方源心头诧异,自己现在扮演的是陈家族长的爹?那他为什么被锁在这里?而且看情形,已经被关了很长时间,绝不是像陈家族长这样新进的牢房。

陈家族长又道:“爹,是我将你击败,关你进来。但我从来没有后悔过!因为你贪污腐败,证据确凿,毫无公道公正,你不配做陈家的族长。按照族规,你就当有如此惩罚。”

“原来是你坑了爹。”方源咧了咧嘴。

陈家族长继续道:“同样的,我也没有后悔过我现在的下场。我是罪有应得!我杀了那么多无辜的人,手上沾满了聂家全族的鲜血。我该死!”

“呵呵呵。”陈家族长惨笑起来,“单家千方百计想要逼我交出正义蛊,但是他们的逼迫算得了什么?他们的咆哮,他们鞭子抽打的声音,烙铁烙在我皮肉上的声音,我听在耳中,就是聂家全族人的哀嚎和控诉。”

“我宁愿他们抽打我更多一些,让我遭受的痛楚更多一些!可是这些,都无法偿还我的的罪!不能弥补我的错!我该死,就让我这么死了罢!这是我该得的。”

“呃!”

陈家族长呃了一声,头一歪,死了。

他的死立即惊动了看守者,很快,篡位成功的单星就过来了。

“真的死了!可恶,正义蛊是得不到了。”单星非常愤怒,咬牙切齿。

忽然,他转头看向方源,双眼喷火:“老东西,我们不是商量好的,让你好好劝劝你这个儿子!你做了什么?你什么用都没有,跟着你儿子一块下去吧!”

咔嚓。

单星说动手,就动手,将方源的脑袋直接砍了下来。

探索失败,方源荒魂再度被逐出梦境。

这一次,方源休整了许久。

梦境的第三幕,显然是最后的一幕,方源探索失败,荒魂受损颇为严重。

一边恢复状态,一边方源在心中细细思量:“第三幕,自己是陈家族长的老爹,却是残废的货色,光靠自己脱困是无能为力的。”

“真正的关键,就在于陈家族长本身。”

“那么,通过第三幕梦境的条件具体是什么呢?”

方源如今探索梦境经验十分丰富,堪称世间第一人。这个问题没有为难他多久,他很快就明白过来。

“我之前两幕梦境,都是不同的身份。但通过梦境的条件,其实都是一致的。那就是满足当事人在当时处境下的需求。”

“第一幕,我是陈家族长,需要带领族人获取胜利,并且这种胜利还要漂亮,自身势力不能损失太大。”

“第二幕中,我是副族长单星,需要的是谋反篡位成功,关键是要压制对方的正义蛊。”

“第三幕,我是陈家族长老爹,被儿子打倒,关押在牢房里已经很多年了。当然是要重获自由。所以,他才答应和单星合作,劝说儿子交出正义蛊,来换取自由。”

方源再入梦中。

方源冷笑两声。

陈家族长已经先出声:“你不必来嘲笑我,爹。”

“你是我儿子,我嘲笑你岂不是嘲笑我自己教子无方?我只是后悔,我当初牺牲自己成就你,想让你凭借正义,来带领陈家走向辉煌和鼎盛,结果你成为阶下囚,连自己都不相信。我实在看错了你,早知如此,我就不会伪造我贪污腐败的证据了。”方源道。

陈家族长懵了一下,被勾动了好奇心:“爹,你此言何意?”

方源冷笑两声,闭口不说。

陈家族长摇头:“不,爹你贪污腐败,证据确凿,怎么可能是伪造的证据?”

方源又冷笑两声:“你觉得真,便是真的?那我问你,聂家是怎么回事?”

陈家族长哑然。

方源继续道:“我表面上对你不够关系,其实背地里早就在精心地栽培你。要催动正义蛊,必须是蛊师坚信自己的正义。当初你太年轻,对于正义的理解太过肤浅,所以我不得不这么做,牺牲我自己,让你贯彻自己的正义。”

“爹,你说出清楚,当初的真相到底是什么?”

方源呵呵一笑:“真相是什么,重要吗?”

陈家族长毫不犹豫,立即答道:“当然重要!没有真相,如何贯彻正义?”

“所以说,你对正义的理解太过肤浅了。正义和真相真的关系紧密吗?”方源徐徐道,“我问你,你屠灭的聂家,有没有正义?就算他们真的有破坏山根的阴谋,在他们看来:你是一个屠戮他们全族的屠夫、刽子手,抵抗你,保卫家园,收取妻子儿女,这是不是正义的?”

“这个……”陈家族长沉吟。

“彼之仇寇,我之英雄,对于他们而言,所谓破坏山根,不过是一种威慑,害怕我们强势的陈、单联合。结果你把他们全族都灭了,他们至始至终也没有破坏山根。不是吗?”方源道。

陈家族长沉默。

方源笑道:“看来,你也认出他们也有正义了。你发动战争,屠戮他人,连累自家势力也有伤亡,你当时觉得你是正义无比,难道有错吗?你害怕他们破坏山根,那样一来,山峦崩塌,元泉消散,生灵涂炭,痛失家园,你为了家族大局着想,提前铲除威胁,也是没有错的。自然也是正义的。”

“你看,双方都有各自的正义,不是吗?”

陈家族长陷入沉思当中。

这种问题,他从未深入地考虑过。

方源察言观色,话锋一转:“我问你,羊吃草是不是正义?”

“这……如何谈得上正义?”陈家族长懵懂。

“在羊看来,没有草,就没有粮食,就会饿死,它吃草是必须的。而在草看来,它好不容易有了生命,钻出土壤,拼命生长,就是为了一点雨露和阳光。它如此努力,却惨遭羊口,不仅草叶被吃,就连草根都不被放过,都吞入羊的肚中消化,一丝生存的希望都没有。草是不是受害的一方?是不是太过可怜?”

陈家族长摇头:“羊吃草,不是天经地义的事情吗?我从不感觉草的可怜,因为这就是自然天地的运转啊。”

“正是如此。”方源点头,“天地间的真理,就是大鱼吃小鱼,小鱼吃虾米,生来如此,自然而然。何有罪恶一说?更无正义的标榜。羊吃草,人吃羊,都和正义无关,只是出于生存的考虑。”

“天地从来就没有什么正义,只有人间有正义。”

“从古至今,人要团结才有更多的力量,才能更有利于生存,在残酷的环境中挣命!”

“那么怎么样,才能让人们更加团结一致呢?”

“组织、律法还有道德。”

“我们组成家族或者门派,按照个人的能力不同,进行不同的分工。我们主要用律法约束众人,告诉人们什么事情是不能做的。然后用道德来提供方向,鼓励人们什么事情是可以做的。母慈子孝,邻里和睦,还有遵循正义,都是道德。不管是有意无意,古往今来,任何组织都不断提倡,因为组织也遵循着生存的本能。”

“或者说,往往拥有健全之律法、切实之道德,制度更加优越的组织,更容易生存下来。这些种种也就流传下来,并大行其道。”

陈家族长瞠目结舌,方源一番话让他为之震撼。

他从未想到过这一层。

他知道正义是对的,但他不知道正义为什么对。

他知道美德是好的,但他不知道美德为什么好。

现在方源告诉他答案——正义、美德,都是人的创造。这种创造可能是有意识的,也可能是无意识的。人们遵循这些标准,是为了自己个体拥有更好的生存环境,以及集体的长存不衰。

方源总结道:“当你明白正义的本质,你就当清楚——所谓的正义,就像是铁甲,是人的工具。你穿戴在身上,是要将它当做工具使用。但是你看看你现在,你身上的铁甲已经成了你的枷锁,囚禁住了你。”

陈家族长呆呆地望着方源,不发一声。

方源微微一笑:“真相和正义,关系紧要吗?聂家的真相,你爹我贪污的真相,究竟是什么?这和贯彻正义,有什么关系?你要明白,我的儿子,你的正义只是你的工具,你要拿来用,而不是被它束缚。”

陈家族长久久沉默,片刻后,他声音沙哑地道:“爹,我明白了。”

话毕,一道白色的光,从他的身上,缓缓地绽放而出,直至充天彻地。

而梦境也在这白光之中化为乌有。

方源荒魂回到现实。

梦境探索彻底成功。

“看来我所料不差。”

“陈家父子都是族长,都痛恨单家的篡位。我若是帮助单星,劝说陈家族长成功,就算是重获了自由,必然通不过最后一幕。顶多是魂魄无伤,被排斥而出。”

“唯有教导陈家族长,帮助他解开心结,令他重新掌握正义蛊,从而挣脱牢笼。这才是通过最后一幕的条件。”

“这份梦境,着实有些意思。”

“哦?晋升人道宗师了么。”方源查探后,淡淡一笑。

\end{this_body}

