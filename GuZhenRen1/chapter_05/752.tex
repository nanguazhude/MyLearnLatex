\newsection{成竹仙蛊}    %第七百五十五节:成竹仙蛊

\begin{this_body}

%1
云竹山脉。

%2
方源高悬于空,仙级奴道杀招四下频发。

%3
中中中!

%4
荒兽云狸在杀招的光晕中,剧烈的挣扎,神情痛苦,不断嘶吼。但很快,这种抵抗就迅速衰减,最终它们一动不动,待在原地,向方源低头。

%5
方源微笑,打开仙窍门户,云狸顺从地钻了进去。

%6
不只是云狸,还有其他海量的生灵,都逃不出方源的魔爪。

%7
云竹山脉,地域广阔,乃是大型资源点,充斥浓郁的云道道痕,孕育出荒兽荒植非常正常,就算是上古荒兽也有可能。

%8
方源迅速搜刮,有着前世的记忆,他非常清楚这些荒兽荒植的位置。

%9
很快,他就收获了七头荒兽云狸,三种荒植共十一株,分别是云气根、白毛参以及枪尖竹。还有一块七转仙材——散发七色的云泥,体量十分庞大。

%10
“时间很充裕啊!”方源算计着时间。

%11
上一世,他光是搜刮这些,就花费了不少时间。但这次因为是第二次抢劫,业务熟练,非常迅速,节省了大量的时间。

%12
“上一世,我只是大致搜刮,其实这片山脉中还有不少地方,没有下手呢。”

%13
方源对陌生的地盘探去,果然又发现了一头荒兽云狸。

%14
他正动用奴道杀招镇压,仙窍中琅琊地灵传讯过来:“启禀主人,我们新得的这只木道仙蛊,跟脚已经探查清楚了,它是木道六转仙蛊,名为成竹!”

%15
蛊是天地真精,毫无疑问,成竹仙蛊便是此时云竹山脉中价值最大的宝藏。

%16
上一世,方源被云竹山脉的大阵阻挡片刻,没有来得及。镇守在这里的两位池家蛊仙非常果断,在逃走之前,将即将要成形的野生仙蛊摧毁了。

%17
但今生,方源注意到这一点,刚刚出手就直击要害,打破大阵,将还未成形的野生仙蛊置于自己的掌控之下。

%18
池家两位蛊仙傻眼,大阵被方源攻破得太快!

%19
这座大阵在上一世还抵挡住方源片刻,表现得有板有眼,但既然方源上一世攻破了这座大阵,也就明白了此阵的运转之理,这一世出手轻轻松松将其攻破。

%20
方源得了一小会,野生仙蛊刚刚成形,便就被他当场炼化了去。

%21
成竹仙蛊,只有常人的小拇指头大小。它体型长而扁平,前胸背板平坦,盖住头部。头狭小。眼半圆球形。

%22
方源曾经放在手中把玩,它的身体与鞘翅都非常柔软,腹部有六块腹板,尾端放光,光辉碧绿深沉。

%23
总体而言,成竹仙蛊就是一个能发深绿光辉的萤火虫。

%24
“不错不错,这只仙蛊正合我用。”方源面露微微喜色。

%25
方源铲除陈衣,得到他的魂魄,一直都在搜魂。

%26
陈衣乃是元莲派太上大长老,年轻的时候就获得了一道元莲真传——因果神树。在琅琊福地的战斗中,又获得了一道元莲真传——来因去果。

%27
所以,方源手中有两道元莲真传!

%28
不过来因去果的根基,还在于因果神树。

%29
这两记杀招的核心仙蛊,除了律道仙蛊因、果之外,就是一些木本蛊了。

%30
陈衣阵亡,虚窍消散,里面的蛊虫早就被他自毁,方源要修行元莲真传,还得自己筹备蛊虫。

%31
凡蛊可以解决,但仙蛊就需要花心思筹备了。

%32
成竹仙蛊显然就是木本蛊,可以参与元莲真传的相关修行。

%33
当池曲由操纵着太宇寺,赶到云竹山脉的时候,方源早就跑得没影了!

%34
池家蛊仙们呆呆地看着云竹山脉。

%35
许多人双眼瞪大,流露出难以置信的神色:“光秃秃一片啊,这还是云竹山脉吗?我们的锦绣繁盛的云竹山脉呢?”

%36
一位老蛊仙差点把老血喷出来,他颤颤巍巍地道:“我曾经驻守过这里一段时间,经营山脉,耗尽心血,没想到却遭方源贼手!方源,你这个魔头,我和你势不两立!”

%37
有人则思考,语气沉重:“我们栽种在这里的普通云竹,规模极大,非得用巧妙的木道杀招才能迅速收取。方源这家伙显然是有备而来!”

%38
方源上一世的确是缺乏木道手段,最终将海量的平凡云竹都留在山上,只取走了荒植。

%39
但这一世,他带着重生的优势而来,当然是准备充分的。

%40
方源专门为了收取云竹,而设计出了一个木道仙招!

%41
他的木道境界虽然不高,但借助智慧光晕,推算出一个针对凡材的六转杀招,还是妥妥的。

%42
他吞并了琅琊福地,获得了数只木道仙蛊,正好在这里能利用得上。

%43
琅琊福地库藏丰富,白毛地灵主宰福地时间极长,经营了数十万年,每一个主流流派的仙蛊都有数只。它们不成体系,都是地灵炼制出来,或者是研究什么炼道的难题而准备的。

%44
方源就曾经借用过当中的木芽仙蛊,组成菌光普照杀招。

%45
池曲由满脸都是阴云密布,大袖中隐藏的双手,已经捏紧成拳。

%46
“方源……”他暗中咬牙,之前的幻想破灭了。

%47
方源就是针对池家的!

%48
但池曲由万万想不通,为什么方源要针对他,针对池家。

%49
当初追杀你,整个南疆正道都有份。和你仇恨最深的就是武庸,你去干武家啊。你若是嫌武家强大,你就去欺负弱小,比方说乔家什么的。

%50
我们池家不弱,在南疆正道中始终都是上流层次啊!

%51
我招谁惹谁了,你来对付我?

%52
你究竟有哪里看不惯我,我改还不行吗?

%53
你好好说,行不行?

%54
人和人之间,注重的就是交流啊!

%55
最烦的就是你这种人!动不动就搞突袭,就搞破坏,就搞劫掠!关键还一声不吭!

%56
你究竟是为什么这么做?你这个混蛋!

%57
不只是池曲由,整个池家的蛊仙都感觉委屈,感觉冤枉。

%58
“方源为什么针对我们,明显是刻意进行了充分的准备。他究竟在图谋什么?”

%59
“我们必须要先弄明白这个问题,这将是我们对付方源的首要条件。”池曲由沉思了片刻,说道。

%60
他放下了之前的侥幸,决定全力对付方源。

%61
不对付他,池曲由对家族不好交代。

%62
魔道蛊仙常常独来独往,但正道不行,维系一个组织,绝不能单凭个人喜好和个人自由。就拿池曲由而言,即便是八转修为,他还有血脉后代需要照顾。

%63
再者云竹山脉对于池家,并不是单纯的大型资源点。

%64
池家起步时,池家先祖们省吃俭用,咬紧牙关,拼力建设出云竹山脉。

%65
云竹山脉为池家的发展壮大,提供了稳定的经济来源。

%66
云竹山脉是池家的光辉历史,蕴含着池家奋发的精神。

%67
方源劫掠云竹山脉,本身就是践踏池家蛊仙的骄傲,蹂躏池家的历史情怀。

%68
就在这时,池曲由等人又接到了池家蛊仙的求援急报。

%69
“报——!方源忽然出现在凤焰山,目前正施展出万蛟杀招,攻打护山大阵。”

%70
“报——!方源出现在褴褛洞外,被洞外大阵暂时阻碍,神秘消失。”

%71
和之前不同,方源忽然出现在了两个地点。

%72
池家蛊仙们惊疑不定,难道方源还有两个不成?

%73
“一个是假扮的,一个才是真身?”

%74
“或者是方源的分身,也有可能啊。”

%75
“不,按照情报中的时间,方源是先出现在褴褛洞,随后又用定仙游,攻打凤焰山。”池曲由冷静地分析道。

%76
“如此一来,方源还是在凤焰山吗?我们要立即赶去支援!”

%77
“但他为何之前又要传送到褴褛洞呢?这是否多此一举?”

%78
“或许他是发觉褴褛洞的守护大阵十分强大,所以就选择了更易攻打的凤焰山?”

%79
“不,这两处都是巨型资源点,守护大阵都同等强大,并不存在明显的强弱之分。”池曲由琢磨着,忽然脑海中灵光一闪。

%80
他不动声色:“分兵救援吧。”

%81
“只能这样了!若是巨型资源点被方源攻破,那我们池家可就要伤筋动骨了。”

%82
“但这会不会是方源的谋算,他就是想诱导我们分兵呢?”

%83
“方源的实力今非昔比,再不是之前被我们追杀的时候了。凤九歌这样的人物,都栽得这样惨……”

%84
池家蛊仙们犹豫不决,对方源十分忌惮。

%85
“我们要不要求援?”忽然有一位蛊仙开口道。

%86
整个太宇寺中,顿时一片沉默。

%87
这其实是大多数蛊仙心中想要说的话,只是没有说出口而已。

%88
沉默中,群仙的目光纷纷投向池曲由。

%89
池曲由满脸寒霜,冷喝道:“对方只是一个七转蛊仙!而我族只是被攻破了两处资源点,这就让我们方寸大乱,去外求援?我们池家可丢不起这个人!还是谁战死了?池家只剩下老弱妇孺了?和方源一战都没有过,居然就想求援?你们的胆气呢?你们的勇武呢?你们身为池家人的荣耀呢?”

%90
群仙皆低头不语,被池曲由狠狠的一阵喝斥。

%91
池曲由训了他们一顿,旋即下令:“就按我说的办!我前往凤焰山,你们操纵太宇寺,支援褴褛洞。尽快支援,但也不要乱了方寸。巨型资源点的守护大阵,可不是那么脆弱的。时刻保持联络!”

%92
“是,大人!”

%93
没有人反对池曲由的这个命令。

%94
就目前的位置而言,他们距离凤焰山较近,褴褛洞较远。所以宇道仙蛊屋太宇寺,直扑褴褛洞。而池曲由速度不快,就近支援凤焰山。

\end{this_body}

