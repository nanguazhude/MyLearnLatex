\newsection{方源渡劫(终)}    %第三十五节:方源渡劫(终)

\begin{this_body}

%1
“不过只是第一次地灾,那么多的荒兽雪怪也就算了,居然还出现上古墟蝠……”方源仰望,满脸苦涩之意。

%2
上古墟蝠体型极大,投下的巨大阴影,将方源、还有无数雪怪都笼罩在内。

%3
一时间,风雪都减弱下来。

%4
雪怪们也停止了嚎叫和咆哮,纷纷仰望天空中这个威凌天下的猛兽。

%5
“在离开琅琊福地之前,我已经用了运道杀招,暴涨我现在的气运。渡劫以来,也一直催动狗屎运仙蛊,毫不停歇。如此气运,应当减少了地灾的不少威能。就算这样,这次地灾仍旧如此艰巨!”

%6
方源暗自咬牙,强振精神。

%7
上古墟蝠在空中划过一道优雅的弧线,庞巨无比的身躯像是山峦一般,朝着方源镇压而下。

%8
一时间,威势无两,就连空气都被压缩,凝重得重重压迫而来。

%9
剑道杀招剑浪三叠!

%10
方源无路可退,只有迎难而上,悍然发动第三剑招。

%11
哗啦!

%12
一声巨响。

%13
他身边的剑浪,顿时由静转动,汹涌澎湃,卷起百千浪花。

%14
这些浪花都呈现璀璨的白银之色,锋锐至极,杀伤力惊人。

%15
大浪汹涌,逆天而上,在空中分散开来,形成数里长的浪锋,重重地撞上上古墟蝠。

%16
墟蝠闷声不响,竟将这波浪锋直接撞碎。

%17
方源双掌一推,哗的一声,再推出一波巨浪。

%18
这波巨浪,比第一波要更加庞大,绵延十多里,风头浪尖上无数银白浪花。璀璨夺目,散发出强烈的危险感觉。

%19
剑浪所到之处,一片纯净。皑皑风雪被瞬间清空。甚至就连空气都被斩碎。

%20
显然,这是群攻杀招。

%21
以七转浪剑仙蛊为核心。还有几只水道仙蛊辅助,这些水道仙蛊自然是方源从琅琊地灵那里借出来的。

%22
第二波剑浪,撞上墟蝠。

%23
墟蝠下落之势,顿时隐隐一顿,同时,它发出一声嚎叫,似乎在宣泄痛楚。

%24
剑浪并不持久,迅速消退之后。方源便见到上古墟蝠的身上,布满一道道深邃的剑痕。

%25
若是正常情况下,墟蝠趋吉避凶,感到自身性命受到威胁,都会主动退去。可惜这头上古墟蝠,并非纯正的血肉之躯,而是地灾所化,受此创伤,连一丝停歇都没有,继续俯冲下来。不杀死方源誓不罢休。

%26
墟蝠本身乃是宇道怪兽,在它面前,方源的剑遁仙蛊都收效甚微。

%27
因为方源周围的空间。都已经被墟蝠的力量影响。看上去一两步的距离,真正行走的话,没有数百上千步,根本走不出去。

%28
一味躲避,乃是下下之策。虽然墟蝠飞行速度缓慢,但宇道力量也同样限制了方源的速度。

%29
唯有抗争,以攻对攻,灭了源头,方是上策!

%30
好在剑浪三叠。共能发出三波剑浪,威能一波比一波强大。

%31
第三波!

%32
剑浪掀起三丈高度。排山倒海一般冲向上古墟蝠,浪潮汹涌之声。响彻天地。浪花绽射犀利剑光,璀璨夺目,让方源这个始作俑者都感觉浑身一紧,心中寒意升腾。

%33
轰!

%34
剑浪和墟蝠狠狠相撞。

%35
上古墟蝠冲势顿止,仰头张嘴,发出充满了痛楚的嚎叫。

%36
剑浪覆盖它的腹部,从原先的伤口处不断渗透,剧烈消耗。最终,只剩下最后一丝水线,十分勉强地从上古墟蝠的背部透射出去。

%37
蓬。

%38
一声闷响,大片的血肉从上古墟蝠的身上,四下散落开来。

%39
被第三波剑浪切割,上古墟蝠瞬间“瘦身”,庞巨的身躯,一下子变得十分苗条。

%40
原本巍峨如山,镇压而来的恐怖气势,已经消散一空。

%41
方源感到浑身上下,自由轻松。原本影响他周围空间的无形宇道力量,彻底消失不见。

%42
重创的上古墟蝠,宛若断了线的风筝,飘飘荡荡一阵后,重重地砸在地面上。

%43
一声巨响,掀起漫天的雪花。

%44
“杀掉了!”方源心中一喜,忽然面色骤变。

%45
大量的感悟,宛若九天而下的瀑布,直接灌注到他的心田之中。

%46
这股感悟,远超之前所有的总和。

%47
一下子,竟然将方源的自我意识都击溃了。

%48
方源忘记了自己,仿佛真的成为了一只墟蝠,从出生,到成长,再到灭亡。飘扬在空中,洞穿寰宇,逍遥遨游在广袤的天地之中。

%49
一阵恍惚之后,方源惊醒过来,失神的瞳孔重新绽射出屡屡精芒。

%50
“看来是感悟太多,我如蛇吞鲸,一口吃成胖子,也有被撑死的危险!”明白这点后,方源的额头隐见汗渍。

%51
狂蛮真意是好的,是提升境界的无上捷径。

%52
但世间之事,都要讲究适可而止。过度之后,好的东西,往往会让人因此受害。

%53
狂蛮真意太多,会将蛊仙本来意识彻底消磨。幸亏方源本身就有智道的底蕴,身上的防护仙蛊之一,就是从琅琊地灵手中借来的智道仙蛊。

%54
否则,方源的意志被真意彻底冲毁,恐怕真的会认为自己就是一头墟蝠,失去清醒的神智,在随后的天灾地劫中陨落。

%55
“仙元所剩不多了,好在上古墟蝠也被干掉……”

%56
方源悬浮于空中,俯视地面。

%57
墟蝠尸体落到地面后,彻底化为雪堆,形态散尽。

%58
不过地面上的那些荒兽雪怪,竟然已经多达百头,小雪怪不计其数,七丈高的上古雪怪都开始出现了!

%59
“可恶,只是耽误了这么一点功夫而已。”看到这样一幕,方源感觉整个人都不好了,密密麻麻的雪怪群,让他心中的无力感越发强烈。

%60
“要清除掉这些雪怪,看来还得向琅琊地灵借贷仙元石。”方源并不太愿意。之前的交易已经将他的欠债还清了,但现在没有办法,他只能再去借。

%61
原本减弱的风雪。又渐渐强盛起来。

%62
巴掌大的雪花,漫天飞舞。呼呼呼的狂风。四处肆虐。

%63
两头上古墟蝠,在狂风大雪中渐渐成形。

%64
方源浑身一震,面色铁青,心中破口大骂:“我干脆死了算了,这哪是什么地灾?就算是天劫也没有这般恐怖吧!”

%65
这当然只是单纯的抱怨,小小的发泄了一下,方源咬紧牙关,身形似箭。开始主动出击!

%66
刚刚那一头上古墟蝠,就差点要了他的老命。这若要同时出现两头,围攻方源,这根本就打不下去!

%67
似乎地灾的力量,已经开始衰落,步入了尾声。

%68
这两头上古墟蝠,成形缓慢,远不如之前的那些迅捷快速。

%69
方源打算趁着这两头上古墟蝠还在成形,先下手为强。

%70
但就在这时,天空中忽然下起了雪弹风暴。

%71
好几个巨大的雪球。仿佛马车一样,高速撞来,幸亏方源有准宗师级的飞行造诣。躲闪及时,没有被砸中。

%72
方源定睛一瞧,原来始作俑者正是地面上的那些雪怪。

%73
经过风雪的灌溉培育,上古级的雪怪,数目激增,出现井喷一样的情况,似乎之前还是数头,如今已经多达二十八头了。

%74
这些上古雪怪,双手一握。就能凭空凝聚出雪球。咆哮一声后,双臂猛振。马车大小的雪球,就好似流星一般。逆空而上,直向方源轰击过来。

%75
方源一颗心沉入谷底,本来这两头上古墟蝠就不好对付,现在这些雪球来势汹汹,极大地阻碍了他的行动。耽误了他的战斗计划,真的让两头上古墟蝠成形,那恐怕方源的这次渡劫,就将以失败告终了!

%76
就在方源在仙窍内渡劫,情势危急的同时,在冰原上,也有人正在渡劫。

%77
这是一位蛊师升仙,引动天地二气,形成的灾劫。

%78
“师傅,救命!”蛊师惨叫一声,面对霜雷兜头打来,他毫无还手之力,只能眼睁睁地看着死亡降临。

%79
砰。

%80
一声轻响。

%81
关键时刻,一个身影挡在渡劫蛊师的面前。

%82
对于蛊师而言,无可阻挡的霜雷,在这个身影面前,却是脆弱如纸。

%83
霜雷出现的快,消失得也快。

%84
无功而返,甚至连那个身影的一丝衣角,都没有掀翻。

%85
“师傅的修为,是多么的强大啊!”蛊师心中充满了惊叹。从渡劫开始,他就不断的感慨,连续的震惊,自己都有些麻木了。

%86
“这不过只是飞霜跳雷劫,连十大凶灾都算不上。”蛊仙楚度面无表情,看着手中残留下的一丝霜白雷光,满是遗憾地道。

%87
“师傅,听你这话音,似乎还嫌弃这灾劫不够猛烈的样子啊?”蛊师叫道。

%88
楚度坦然而言:“蛊师升仙,会引动天地二气灌体,在这个过程中,和天意接触。蛊师询问任何修行上的难题,天意都会解答。这就是世人常说的师法自然。”

%89
“但是到了北部大冰原,力道蛊师升仙,会引发狂蛮真意。这个时候,蛊师就不是和天意接触,而是和狂蛮真意交流,令自身的力道、变化道境界突飞猛进。”

%90
“灾劫的威力越强,狂蛮真意就越多。为师培养你们这些徒弟,看中的就是你们升仙时的狂蛮真意。得到这些真意,能让为师的力道境界,飞速提升。”

%91
“原来如此。”蛊师恍然大悟,“这就是师傅当年收我为徒时,说的要借助徒儿的地方吗?”

%92
楚度正要开口,忽然一声惊天巨响,从千里外传来。

%93
“嗯?”楚度愿望,只见千里外空间破开,宛若一面碎镜,显现出仙窍内方源和上古墟蝠苦战的情景。

%94
“这!?”楚度雄躯一震,眼中顿时放射出无穷的精芒。

\end{this_body}

