\newsection{阳莽背火衣}    %第六百一十四节:阳莽背火衣

\begin{this_body}

%1
方源构建的仙蛊屋雏形,本身就不擅长防御,中招之后,分崩离析。

%2
仙蛊屋内方源等人连忙自保。

%3
大量的火焰,宛若毒蛇,吞噬烧毁沿途一切蛊虫,然后从四面八方,向方源等人围剿过来。

%4
方源大吼一声,催动仙道杀招冬裘,立即他的全身上下都像是披了一层洁白的寒霜。

%5
寒霜不仅是保护了方源他自己,更是在这关键时刻覆盖出去,维护周围宙道分身、黑楼兰、影无邪、白凝冰等人。

%6
在这一点上,冬裘杀招就比逆流护身印要优秀一些。

%7
因为逆流护身印只能保护自己,不能帮助他人。但冬裘杀招,却是可以推己及人。

%8
冬裘杀招挡住火蛇的攻击,但下一刻,就听到那石莲岛上的威猛老者猛地一喝:“煌煌之威,群蛇化龙!”

%9
呼!

%10
火焰猛地爆发,凶猛的热量惊人地喷发出来,火蛇纠缠汇聚,化为一头张牙舞爪,长达十多丈的恐怖炎龙。

%11
赤红色的炎龙,狠狠地撕破冬裘杀招,再次杀向方源。

%12
方源刚刚想要打开仙窍门户,将身边众仙收入至尊仙窍,没想到这火蛇杀招还有后招,威猛老者战斗经验极其丰富,没有给方源留下任何余地。

%13
炎龙栩栩如生,张开大口,从上而下,俯冲下来。

%14
方源等人不由仰望,满脸都被火光映得通红,惊人的热量扑面而来,仿佛将众人搁置在烤架上烧烤!

%15
白凝冰、黑楼兰等人,脸色皆变,目光中不免露出绝望之色。

%16
这条炎龙乃是绝不寻常的八转杀招,威能非凡,纵然是白凝冰的白相杀招,也难以逃出生天。

%17
他们唯一的希望,就在方源身上。

%18
方源鼻孔、双耳、嘴角都在溢血!

%19
冬裘杀招被破,方源承受着八转杀招的反噬。参与其中的大量凡蛊灰飞烟灭,万幸的是核心仙蛊还在,只是略微受损,暂时还能催用。

%20
方源面临困境。

%21
他身后就是自己的宙道分身、白凝冰、黑楼兰等人,这些人的修为太低了,单凭他们自己根本挡不住这头炎龙。若是有仙蛊屋雏形,他们还有参与战斗的最低资格,但威猛老者却是狠辣非凡,一上来就摧毁了仙蛊屋雏形。

%22
方源若是停留在原地,替他们遮挡,且不说冬裘杀招被破,没有什么好的手段,就算是有,也十分危险!

%23
因为对方就是想要逼迫方源与其对抗,一旦方源为了保护身后群仙,和他对拼,无疑就中了威猛老者的算计。

%24
方源心中几乎可以肯定,对方定然是还有杀招。一旦方源和炎龙死拼,就会落入到威猛老者的节奏中去,在接下来的战斗中,被他牵着鼻子走。

%25
一瞬间,方源做出了决定,身形如电,斜冲出去,躲开炎龙。

%26
威猛老者哈哈狂笑:“好一个魔头!那你们就都给我烧成灰吧!”

%27
炎龙竟不管方源,仍旧俯冲下去。

%28
方源冷哼一声,方向忽转,悍然向威猛老者杀去。

%29
威猛老者立即转移视线,盯着方源,眼睁睁地看着他冲过来,嘴角裂开,露出一丝狞笑。

%30
方源心中微微一沉。他攻杀过来,无疑是想围魏救赵,但没想威猛老者同样心性悍勇,不惧死拼。甚至极有可能,在这石莲岛上还有布置,令他十分自信从容。

%31
方源伸手一指,数月来几乎不眠不休的苦练成果斐然,春剪再次被催动出来,率先射向威猛老者。

%32
剪刀锋利,但剪在威猛老者身上的时候,威猛老者的头发忽然燃烧起来,化为一件巨大的火焰披风,熊熊燃烧,护卫周身,牢牢架住春剪。

%33
“哈哈哈,老夫此招名为阳莽背火衣,乃是从人祖传中太日阳莽的火焰长发中获得灵感,是我生平最得意的创作。实话告诉你,老夫虽然是炎道蛊仙,但论攻势,更强的还是防守!别说你春剪,就算是夏扇同袭,我也照样无惧。”威猛老者哈哈大笑。

%34
但下一刻,老者的狂笑戛然而止。

%35
原来,炎龙袭击过去,却是劳而无功,不管是方源的宙道分身,还是白凝冰、黑楼兰等人都保住了性命。

%36
老者眼皮子狠狠一抖,舔了舔干燥的嘴唇,恶狠狠地道:“这就是梦境?”

%37
原来,方源躲避炎龙,并不是放弃了自己的分身和其他蛊仙,而是在关键的时刻,令影无邪自爆开来。

%38
自五界山脉大战时,影无邪就换了肉身,再次成为纯梦求真体。也正是因为纯梦求真体山的梦道道痕辅助,他才在施展出引魂入梦之后,成功地令八转蛊仙君神光中招。

%39
现在纯梦求真体自爆开来,影无邪的魂魄自然陷入梦境之中。其他诸仙,也因为靠得很近,均被包裹进去。

%40
梦境初显,断绝一切其他流派的手段。威猛老者的炎龙攻势凶厉至极,但冲到梦境上,宛若一片虚光幻影,毫无作用。

%41
“哼,你这魔头果然生性狡诈,不过今日你必死无疑!”威猛老者狠狠跺脚,一飞冲天,脚下的石莲岛化为飞烟瞬间散去。

%42
老者披着一层火衣,宛若火焰流星,杀向方源。春剪如影随形,在他身上连连剪动,都被熊熊燃烧的火焰架住,伤不到老者一根毫毛。

%43
方源往后飞退,催动夏扇。

%44
飓风狂卷,把老者的火衣散得呼呼作响,却好像是火借风势,反而越来越旺。

%45
老者嘴巴大涨,露出锋利的牙齿,狂笑起来:“怎么样!老夫的阳莽背火衣,绝世无双,感受到绝望了吗?哈哈哈!”

%46
果然,如老者所言,春剪、夏扇齐发,都奈何不了阳莽背火衣。

%47
不仅如此,随着双方距离迅速拉近,一阵强烈的醉意向方源袭来。

%48
但很快,方源催动智道杀招,双眼重现清明。

%49
阳莽背火衣中蕴含的不仅是炎道的极致奥义,更模拟出了人道、食道等等的精妙之处。距离威猛老者越近,就会有越强烈的醉意涌上心头。

%50
方源一直极力后退,想和威猛老者拉开距离,但事与愿违,对方的速度要超过方源一截。

%51
“魔头,你身陷大阵,又能逃到哪里去?乖乖受死吧!”威猛老者咆哮。

%52
刚刚的石莲岛,不过是一场虚幻的光影,真正的本质是铺设在光阴长河中的超级宙道仙阵!

%53
这座大阵极其厉害,饶是方源也无法勘破。此刻发动起来,形成大阵空间,把方源拘束在内,制约方源四通八达、定仙游等等类似的手段。

%54
方源之前利用宙道大阵,埋伏算计了南疆正道。这一次,天庭也铺设大阵,算计了方源。

%55
不过,天庭的手笔要比方源大得多!

%56
这种建设在光阴长河中的宙道大阵,可不是方源能够铺设得了的。依照方源目前的能力,顶多是在五域外界的光阴支流附近搭建出宙道仙阵来。

%57
就算是历史上的宙道大能,可以在光阴长河上铺设仙阵的都少之又少。

%58
此中做到极致的,无疑是红莲魔尊。这位存在的手段,已经脱离了宙道仙阵的层次,直接布置出人造的天地秘境石莲岛,影响一方光阴河段,引为己用。

%59
天庭方面一直致力于搜寻石莲岛,虽然至今都没有搜寻到一座,但并非没有成果。此处的宙道仙阵,就是成果之一。

%60
大阵拘束方源,令他避无可避,躲无可躲。

%61
威猛老者逼近,方源冷哼一声,忽然不退反进,反扑向威猛老者。

%62
老者明显诧异了一下,但旋即就反应过来,一挥手,火焰熊熊,烧向方源。

%63
方源怡然不惧,挺身撞上去。

%64
熊熊热焰反而到卷而归,攻向威猛老者。

%65
正是仙道大招——逆流护身印!

%66
老者一阵手忙脚乱,招架住自己的杀招,但这个时候,方源已经来到他的面前。

%67
方源蛮横凶狠,直接用身体撞上威猛老者。

%68
逆流护身印对决阳莽背火衣!

%69
下一刻,方源和威猛老者双双弹飞出去。这一次比拼,竟是半斤对八两。

%70
威猛老者稍稍处于下风,但很快,他全身上下的火衣再度熊熊燃烧起来,恢复到巅峰的状态。

%71
逆流护身印对付攻伐杀招,效果惊人。但阳莽背火衣到底是以防御为主,因此被逆反回来的威能少之又少。

%72
“这就是逆流护身印?嘿,这么快就发现了……”老者目光阴沉。

%73
原来此处宙道仙阵,除了形成一片阵内空间,隔绝内外,还有一项核心的威能,就是营造出削弱宙道的环境。

%74
天庭的本意,是要用这座大阵来对付红莲真传石莲岛。这些岛屿都是人造的宙道天地秘境,落入大阵之后,就会被大阵削弱。

%75
正因如此,方源的春剪、夏扇两大杀招,才对阳莽背火衣如此无奈。

%76
阳莽背火衣的确很强,这招不单纯是炎道的巅峰之作,还从炎道扩展出去,牵扯到人道、食道的奥妙。

%77
单论精妙程度,逆流护身印都显得粗糙肤浅、直来直去。倒是万我杀招,兼并奴道、力道,精妙程度方面和阳莽背火衣难分高下。

%78
春剪、夏扇杀招更显普通,只是包含八转仙蛊,又被宙道道痕增幅,才显得威能脱俗。它们攻不破阳莽背火衣,但也能造成很大影响。不像现在战斗中,效果十分微弱。

%79
方源参悟出这一点后,立即大胆地取消冬裘杀招,反使出逆流护身印,果然稍稍扳回了一些场面。

\end{this_body}

