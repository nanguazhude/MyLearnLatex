\newsection{四清四变风}    %第三百七十节:四清四变风

\begin{this_body}

逆流护身印一出,立即受到奇效。txt下载80txt.com

指风龙反噬武庸而去,武庸断喝一声,全身陡然爆发出一股强猛气势。

他酝酿已久的仙道杀招,终于爆发开来。

阴风鬼镰爪!

他伸出右手,五指呈爪,抓向指风龙。

几乎是瞬间,一道巨大的鬼爪,猛地浮现在指风龙的身上。它出现的是如此突然,完全是毫无征兆。

完全由阴风组成的鬼爪,巨大无比,一把抓住指风龙,简直就是抓住蚯蚓一样。

鬼爪漆黑如魔,爪尖很长,锋利无比,简直是五把镰刀。此刻一把抓下,大有割神屠仙的凶厉!

指风龙哀嚎,在鬼镰爪下挣扎,剧烈咆哮。几个呼吸之后,它彻底分崩离析,化为一团团的风,随即消散。

这一幕,看得凤九歌不禁面色一凝。

指风龙毫无疑问,是八转杀招,但是武庸的这记阴风鬼镰爪,显然威力更要凌驾其上!

这一抓要是抓在凤九歌的身上,那后果不堪设想。

若非方源及时插手,此时此刻凤九歌还在被指风龙追击,又要面临阴风鬼镰爪的攻击,处境将极为危险!

武庸的面色同样也不好看。

他的阴风鬼镰爪,酝酿了半天,就是想要对付凤九歌,将他击败。

但方源插手之后,让他不得不转而对付指风龙,白白浪费了这记大好杀招。

事实上,阴风鬼镰爪并不只是眼前的威能,它还有后续的变化,名为阴风索,能够擒拿蛊仙。

凤九歌毕竟是中洲一方,武庸耗费巨大的精力以及大量仙元,去酝酿阴风鬼镰爪,就是为了活捉凤九歌。

一旦他目的达成,对他将大有获益。

首先,凤九歌这样的人物,武庸生擒活捉了,对于自身的威望,有着很大的提高,同时也对南疆正道有着巨大的影响。

其次,没有打杀凤九歌,而是活捉,不至于让中洲、天庭翻脸。

最后,擒获凤九歌后,武庸完全可以利用他,不只是从他身上搜刮油水,而且还能要挟灵缘斋,甚至是天庭。

若是能让天庭妥协,交出曾经从南疆正道手中卷席走的那些布阵仙蛊。哪怕只是一部分,那么武庸在整个南疆蛊仙界的声望,必将得到一个史无前例的暴涨!

甚至可以,直接稳定武家在南疆蛊仙界的正道第一的地位!

但是现在,因为方源的搅局,导致武庸的算盘直接落空。

“这两个家伙……”

“我若用寻常的七转杀招,凤九歌完全能够抵抗,甚至反攻。”

“若是用八转杀招,虽然能避退凤九歌,将其压入下风,但是碰到方源的逆流护身印。一旦反噬回来,吃亏的反而是我。”

武庸望着眼前,眼角不禁微微一颤。

棘手。

他明显感到有些麻烦了。

单独面对方源,武庸可以把他当做沙包打,完全处于主动。单独和凤九歌对战,武庸也可以强势压迫。

但是面对两人联手……

武庸都感到了棘手。

“短时内,要破解逆流护身印,希望并不大。毕竟我是第一次亲眼见着此招。本身也不是什么智道蛊仙。”

“或许天庭、长生天方面,有了经历,说不得已经对逆流护身印加以研究了。”

“现在唯一的方法,就是持续不断地轰击,让这两人疲于应对,从而出现破绽,再实施斩杀。”

没有战机,就自己制造战机。

想到这里,武庸忽然隐没了身形,消失在了方源、凤九歌的眼界当中。

“他消失了,在哪里?”方源看向凤九歌。

武庸以退为进,方源的侦查手段可不怎么样,根本无法察觉出武庸的真身来。

凤九歌也摇头:“似乎是借助了这方战场,隐没了形迹。”

他也只查探出一些模糊大概。

凤九歌的话音刚落,整个战场就陡然发生了异变!

呼呼呼……

狂风凭空而起,不断地呼啸,在整个战场中宛若无形的蟒群,在奔走游窜。

然后,风刃凝结而出。

一道道风刃,带着八转杀招的锋利威能,盘旋飞舞,向着方源、凤九歌二人杀来。

“来得好。”方源不惊反喜,身上仙衣飘飘,直冲而去。

风刃却是灵活地绕过他,向凤九歌集中射去。

显然,武庸是先除掉凤九歌,然后对付方源。

这个选择显然非常明智。

因为只有凤九歌,有着反攻的能力。只要他一死或者被擒拿,方源空有逆流护身印,又能如何?

凤九歌见此,不禁朗笑一声:“有趣,倒是把我当成了软柿子。”

语调并不愤怒,反而有些欢喜和好奇的情绪。

这是他从未有过的体验。

当即,凤九歌拳掌交击,破碎个个风刃。

但风刃毁掉之后,又会化为一股狂风,呼啸卷席一阵,就又转变凝聚成一道全新的风刃。

风刃不仅绵绵不绝,而且数量越来越多。

凤九歌不敢让这些风刃加身,尽量提前击溃它们,可是这样一来,就让他处于被动的境地。

方源赶去支援。

这一次,方源只在凤九歌的身边萦绕。

许多风刃打在方源的身上,就立即被反弹出去。

不过并没有追溯反噬武庸,而是直接打在这片战场之上。

原来,武庸虽然是发动者,但却是操纵仙道战场杀招,来施展攻击。这中间隔了一层。

武庸成了幕后黑手,反观方源的逆流护身印,只能反噬一层,所以风刃只是找这片战场的麻烦。

潜伏在角落里的武庸见此,心中自然大喜,更加奋力催动战场杀招。

方源被动防守,片刻之后,他眉头越走越深。

凤九歌有了方源的遮护,处境早已好转。他思考了一会儿,打破沉默,对方源传音道:“你来护我周全,我酝酿一记手段,拆掉这片战场。”

方源楞了一下,旋即答应。

之前,他早已经算得这片战场杀招,难以破解。凭借方源的实力和造诣,很难见效。反倒不如等到锁风消失,让白凝冰等人出来,再用四通八达,直接撤离这片战场。

如此一来,就要最大限度地拖延时间。

可是放任凤九歌和武庸对战,必定会让武庸迅速占据上风,压制凤九歌,直至解决,然后腾出手来对付方源。

凤九歌虽有八转战力,但到底还是七转的修为。

所以方源才来助阵。

凤九歌开始酝酿杀招,不做反抗,把护卫工作全部交给了方源。

这种信任,让方源也有点犯起了嘀咕。

武庸见此,催动战场杀招又急一分。

风刃奈何不得方凤二仙,整个战场便发出变化,开始发出阵阵雷声。

这雷声非常的特别。

寻常雷声,不是轰鸣就是炸响。

这雷声却是带着一丝空明和清脆,仿佛是用葫芦瓢,敲打百年以上的大竹筒。

几个呼吸之后,雷光闪烁,咚咚咚地砸向方凤二人。

“小心,这是正清碧雷。”凤九歌提醒道。

正清碧雷,只有拳头大小,但是速度极快,散发刺眼的青芒,让方源都不由地要闭上双眼。

他连忙催动变化手段,将人眼转变成龙瞳,这才抵挡住耀眼的雷光。

方源此时催动逆流护身印,已经牵扯了绝大多数的心神,再不能动用什么上古剑蛟变化。但是单独转变眼眸,这点手段还是可以的。

正清碧雷轰砸过来,方源直接挺身迎上。碧雷击打在他的身上,又逆反回去,不断破坏战场。

但也有稍许碧雷,砸向凤九歌。

不过这些正清碧雷,才刚刚接近凤九歌,就忽然消弭无踪。

凤九歌酝酿杀招,自然不会全盘相信方源,而是本身就提前作下了防护手段。

方源见此,也宽了心,防守起来更加积极从容,多次主动出击。

武庸见雷霆不行,就又让战场一变,雷霆消失,狂风转弱,一颗颗的水珠,青翠可人,飘飘摇摇,汇集成一股细雨,再度杀去。

“这是玉清滴风,更要小心。”凤九歌再度出声提醒。

这水珠似的风,飘洒在方源的身上,同样被逆反,方源安然无恙,稳固如山。

但凤九歌这边,却遭受麻烦。

玉清滴风在接近他的过程中,不断地消失。但仍旧有不少水滴,落到凤九歌的身上。

凤九歌的身上,同样有防护手段。

每落下一滴,就发出一声清脆的声响。

雨滴密密落下,凤九歌本身仿佛化成了一具人形乐器,奏出美妙动听的曲声。

武庸哈哈大笑,主动显露身形,他已经觑得胜机:“我这还有一招变化,名为玄清仙音。凤九歌你是专修音道,不妨来品评一番吧。”

话音刚落,整个战场上传出风的低吟。

风声婉转,又非常清澈,响彻在方源的心中。一时间方源身上的仙衣,不断震动,泛起波澜。

“不好,这玄清仙音好生厉害,是四种变化中最强攻势。我虽然有着逆流护身印,但单靠此印,只能遮护自己,如何能帮助凤九歌?”方源心中顿时一沉。

武庸的这记仙道战场杀招,名为四清四变风,乃是由一位武家八转先祖开创,一直很好地流传下来。

这招四清四变风,有四种变化,分别是少清风刃、正清碧雷、玉清滴风以及玄清仙音。

它是武家的招牌杀招,声威赫赫,帮助武家威慑南疆已无数岁月。(未完待续。)

\end{this_body}

