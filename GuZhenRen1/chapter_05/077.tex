\newsection{险成}    %第七十七节:险成

\begin{this_body}

叮铃铃!

一朵风花正中方源的后背。[txt全集下载wWw.80txt.coM]

方源身上的防御,像是纸糊的一样,立即被撕扯成碎片。

后背的衣袍被彻底绞烂,鲜血飞溅,碎肉飞离。

“运气不好了!”方源心头一沉。

他身边有数十个力道虚影,而飞来的风花就只有一个,随意攻击。偏偏真身就被砸中。

随着时间推移,狗屎运护持本体的效果越来越弱了,使得天意发现真身的概率大大增强。

更糟糕的,还有暗渡仙蛊。

暗渡仙蛊催发一次,护持方源,遮掩他气息的效果,也在剧减。

一旦这股护持力量彻底消失,比失去狗屎运还要严重。

时间流逝,方源渡秒如年,硬生生苦挨。

他已经不知道击爆了多少轮雪月,挨受了风花多少次攻击。

狗屎运、暗渡的护持效果锐减,但态度蛊、变形蛊形成的见面曾相识,则如中流砥柱,至始至终都支撑着方源。还有万我杀招,在此时此刻,它的攻伐之能已经起不了作用,只能当做炮灰战术来运用。

天意不是傻子,它能够思考。

方源的战术明了之后,每一次万我之后,它发现方源真身的时间越来越短。

方源受伤也越来越重,迫不得已,只能催发仙蛊人如故。

但这样一来,他的仙元也加剧损耗。原本就不多,现在已向干涸的境地发展。

天意高高在上,驾驭地灾,先天利于不败之地。

方源却宛若蝼蚁,在生死一线中挣扎。

雪月的数量越来越多,方源的脚步渐渐跟不上。

情势艰难无比,方源几乎看不到丝毫希望。

但他没有放弃的打算,仍旧咬牙坚持。

狂风像是嘲讽,嘲讽他的不自量力。

雪月高高悬挂,仿佛是道道冷漠的目光。看着方源这样的蝇蚊在徒劳无益地垂死挣扎。

方源浑身浴血,白袍已经彻底变成血袍,血和汗被冷光凝成冰渣。

他狼狈不堪,原本随风飘逸的黑发。也断裂无数,现在长短不齐,让方源看起来就像是个疯子和乞丐的结合体。[www.qiushu.cc 超多好看小说]

他面色冰冷,既没有狂笑,也没有呐喊。

他就像是一块坚冰。就要被天地碾碎,但他不发一言,身处绝境,一切只是默默坚持。

风声衰落下去。

风花劫的力量,在徐徐消退。

已经到达极限了。

地灾的力量也是有极限的。哪怕天意将它增强到最大程度,又亲自操纵。

方源没有被风花杀死,很快,风花数量剧减,天地间不再有狂风。

方源惨笑一声,狂催仙元。他的目标仍旧是雪月!

虽然风花劫消退。空中的雪月也停止衍生,但风花还有残余,雪月剩下的数量并不少。

艰难作战!

方源已经没有了咬牙的力气,浑身虚弱乏力至极。

万我。

力道大手印。

剑浪三叠。

见面曾相识。

他的状态很不佳,几种仙道杀招有时候还会催发失败。每一次失败,都让他遭受反噬,口吐鲜血。

地灾已经到了极限,方源本人也是如此。

此时逃跑毫无意义。不仅是风花阻截,雪月还可缓缓转移飞行,冷光的覆盖范围实在太大。

地灾还未完全结束。天地二气不平,回收仙窍也不可能。

不得不说,天意思谋良久,酝酿出最针对方源的灾劫。上一次地灾。狂蛮真意分化了地灾的一部分力量。这一次,天意虽然不能避免,却也影响了狂蛮真意,使得狂蛮真意影响的部分,化为雪月劫,反而增添了整个灾劫的难度!

现在的战局。就看双方虽能熬得过谁。

半个时辰之后,方源一头栽倒在荡魂山上。

他的仙元几乎彻底干涸。

见面曾相识也支撑不住,心力憔悴,早就散了。

方源费劲心思和力量,击爆大量雪月,但天空中却还残存着最后一轮!

靠着荡魂山的庇护,方源面前抵御着冷光的照射。

他浑身都是伤痕,很多伤口上,血色冰渣凝成一片。

方源已无力气,身上冷霜越积越厚,将他渐渐冻在冰块之中。

最后一丝风,萦绕在冰块的上空,承载着天意,要注视着方源走向灭亡。

仙元所剩无几,一个仙道杀招都催发出来,但方源的脸上却忽然展现出胜利者的微笑。

“此灾我渡过去了。”

下一刻,荒兽刺脊星龙鱼忽然显身,鱼尾用力一摆,狠狠一撞,将最后一轮雪月撞碎。

奴兽仙蛊!

半盏茶之后,天地二气平息下来,方源立即动手,离开北原。

在他走后,没有多久,两道身影从地下冰层中遁出来。

“就是这里了。”其中一位身影,小心探视,环顾四周一番后,说道。

另外一个身影,谨慎地抽动鼻翼,四处嗅嗅,然后以肯定的口吻道:“刚刚我们感应的没有错。这个地方刚刚有人渡劫,天地二气有残余丝丝波动,并未彻底平息呢。”

两个身影都是模糊一片,显然是故意用手段遮掩了身形。

不过,既然能知道天地灾劫,又能探查如此清楚,必定是蛊仙身份。

“唉!北部冰原乃是狂蛮魔尊当年,一手锻造。全是冰川,并无大地。这里天地二气较其他地方,要稀薄很多。若是任由蛊仙在这里渡劫,恐怕会更加损耗天地之气。达到一定程度,势必会引发冰川震裂,天地板荡。”两位神秘蛊仙之一,叹息着,语气充满了担忧。

“北部冰原是我雪民一族最后的世外桃源。我们在冰原地下生活,与世无争。没想到人族还不罢休,连我们这最后一块栖息之地,也要染指,也要觊觎!”另一位神秘身影,似乎年纪较轻,饱含愤慨地道。

原来,这两位蛊仙的身份并不寻常,都是异族雪民蛊仙。

年轻的雪民蛊仙继续道:“太上大长老太过迂腐了!依我的想法,当年就该铲除了那个楚度。现在你看看,不仅楚度时常过来渡劫,还有其他人也来了。久而久之,我们的栖息之地就会被越来越多的蛊仙光顾。”

年长的雪民蛊仙叹气:“唉!太上大长老的想法,我却能理解。他是担心和楚度一战,将我族的存在暴露于天下。如今的天下,可是人族制霸,地位稳固如山,不可动摇。我族暴露出来,必定会引发整个北原人族蛊仙的围剿。”

“难道就让情况这么发展下去?让越来越多的蛊仙,到我们头顶上渡劫不成?其实以我族的实力,又在这里冰原中作战,只要计划周详,隐秘地斩杀几位人族蛊仙,并不是什么难事。当然,我也承认,那楚度很强。我们稳妥起见,放弃他这个目标,但其他蛊仙完全可以下手啊。杀得几人,谁知道是我们做的?嘿!反正人族内部也是乱得很,正道、魔道、散修从未消停过。”年轻蛊仙侃侃而谈。

“唉,你说得也有道理。我们还是先将这个情况,汇报给族中,让其他几位太上族老都来合计合计罢。”

两位雪民蛊仙议论一番,模糊的身影悄然消失在原地。

方源虽然身负重伤,但仍旧马不停蹄,一路兼程,赶回到琅琊福地。

“天意要铲除我,当我渡劫时无疑是最好的时机。它不仅会将灾劫的威能,增添到极限,而且亲自操纵灾劫。更可怕的是影响其他存在,酿成人劫,来布局围杀我!”

方源脑海中牢牢记住这个情报。

这点认知,同样是他和毛六交易,得到的珍贵收获。

在半年前的义天山大战中,他本身也是关键的人劫,是天意对付魔尊幽魂的棋子。

正是因为亲身经历,方源对所谓的“人劫”非常警惕。

想那魔尊幽魂是何等存在,又有影宗帮衬,僵盟辅助,筹谋十万年,都还被天意策翻了最终成果。

和他们比起来,自己现在又算得了什么?

所以,方源几乎是不顾沉重伤势,就直接转移。速度很快,不留给天意酝酿布局的时间。

返程并未意外发生,方源顺利回归琅琊福地。

到了琅琊福地,他这才松了一口气,知道自己算是暂时安全了。

之后,方源龟缩在自家云城中加紧疗伤,同时归还借来的种种仙蛊。

几日后,方源伤势全消,状态复佳,又着手整治仙窍中的荡魂山。

说起来,这座大山给他渡劫带来了巨大的帮助。

荡魂山显得惨淡无比,之前是巍峨高耸,渡劫之后被无数风花削的只剩下一座平缓的小土丘。

江山如故!

在这只仙蛊的作用下,荡魂山再复旧状。

只有如此的荡魂山,才能最高效率地产出胆识蛊。

方源不敢大意,赶忙将荡魂山拿出来,交托到琅琊福地之中。

不过,虽然荡魂山可产大量的胆识蛊,但是贩卖的渠道宝黄天却关闭了。如今琅琊派的库藏中,有大量胆识蛊积压着。卖不出去,真叫方源有些着急。

“两个月之后,就是第三次地灾。威力比这一次,还要强大。我若不在这个阶段,做出一些突破,实力没有增长的话,恐怕是凶多吉少。”

“但实力增长,绝非凭空而生,需要种种修行资源。宝黄天……究竟何时才能再度开启?”(未完待续。)

手机用户请访问<!--80txt.com-ouoou-->

\end{this_body}

