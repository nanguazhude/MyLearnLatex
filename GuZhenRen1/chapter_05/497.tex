\newsection{因果神树}    %第四百九十七节:因果神树

\begin{this_body}

就在豆神宫追杀方源不休的时候,极远处的一座低矮沙丘上,一位青年蛊仙双腿盘坐着。<strong>在线阅读天火大道Http://wWw.qiushu.cc/</strong>

他一身青袍,浑身书卷气息,青年模样,但目光沧桑。

正是被天庭智道大能紫薇仙子派遣过来,当今中洲十大古派之一,元莲派的太上大长老――陈衣。

此时此刻,他双目紧闭,周身气息收敛至无,与周围天地似乎融合在了一起。

随着他不断催谷仙道杀招,从他的肩膀、头顶上,升腾起一缕青烟。

青烟袅娜,升腾到距离他头顶大约六尺的地方,就缓缓停住,不断流转,形成一棵大树形状。

这大树树干粗壮,枝干茂密,绿叶葱葱。树叶丛中,还结了数十颗果实。

但这些果实委实古怪,各种形状颜色,千奇百怪。有的好似漆黑核桃,有的好像粉桃,却大如脸盆,有的果壳上长满尖刺,有的果实镂空,从外可见里面的果肉、果核。

片刻之后,陈衣停住杀招,头顶上的青烟神树徐徐消散。

他双眼缓缓睁开,露出了然神色,心中道:“原是如此。”

前后种种,他都已经清楚了。

元莲仙尊乃是红莲魔尊之后,盗天魔尊之前,天庭之主。

三十万年前,中古时代,元莲西游。

当时,他已经成就仙尊之位,天下无敌,却隐姓埋名,白龙鱼服,扮做凡人蛊师,周游天下。

一天,他在沙漠的某个绿洲中,得到村中老幼的盛情款待。

见到当中一个孩童,孩童见他是蛊师,苦苦哀求他,希望他复活自己病逝的娘亲。

“人死不能复生。”元莲仙尊婉拒,正要温言安慰。

“你不出手相助就算了!”孩童却掉头飞奔,含恨而走。

元莲仙尊本不以为意,忽然头顶上青烟一颤,差点要喷涌而出。

他心头一惊,暗道:“我这九转仙道杀招因果神树,从无极魔尊的一处真传得到启发,自我开创。能够将命当做土壤,扎根其中,将运当做水流,滋润枝叶。能摆脱外物纷杂,万事迷乱,寻得事件始终。”

“我刚刚拒绝这孩童,只是小因,表面上看也是小事一桩。但因果神树萌动,却是彰显出,未来会因小成大,酿成巨大恶果。<strong>最新章节全文阅读qiushu.cc</strong>”

元莲仙尊虽已知晓,但却没有出手,去立即斩杀了那个孩童。

“万事万物,犹如种子,种在土壤中,生根发芽,逐渐成树。众生百态,就有树木千万万万,形成命运迷森。”

“让树木自行生长,不强加干涉,自然而然,无为而治,方能让世间欣欣向荣,一片美好。”

“若是强加干涉,以人欲干扰天规,得不偿失,何苦由来?”

元莲仙尊是天庭之主,依照宿命而行。

人死不能复生,这是宿命的规定,所以元莲仙尊即便有能力,也不会复生死人。

同样的,他和这孩童的相遇,在他看来,也是宿命的安排,无须抗拒。

“不过,一旦这树木生长成熟,凝结出恶果,当及时采摘,止恶杨善。我辈仙人,不过是这天地间的除虫之鸟罢了。”

念及于此,元莲仙尊有感而叹,留下了手中的一座仙蛊屋豆神宫。

这座豆神宫也未交给那孩童,而是直接被元莲仙尊抛入村中井里。

但随后事情不断发展,村子被风沙掩埋,豆神宫因意外落入凡人之手,如此辗转,在红尘中打滚,一步步跌宕起伏,最终落到青家手中。

青家得之,认出是元莲仙尊之物,狂喜,不可一世。结果惹恼了还未成尊的幽魂,幽魂出手,一场大战,屠戮了青家全族,震撼天下。

青家太上大长老在家族灭亡关头,将青家遗藏尽数封入豆神宫中,隐藏起来。并且布置了手段,能够引动生命种子重生,企图挣扎出一线生机。

青家全灭,但万物生灵的怨恨之气、无数蛊仙残魂碎片却都被牵引过去,在豆神宫中储藏。

时代变迁,青鬼沙漠逐渐形成。在这种环境的影响之下,这些残魂碎片本来无法重生,却逐渐凝聚融合,形成了一尊旷古未见的传奇太古魂兽。

魂兽有着人性,智慧十足,矢志要想幽魂复仇,仍旧以青为姓,以仇为名,是为――青仇!

可惜当青仇形成,幽魂不仅是成尊,而且已经失踪。

青仇想要寻找幽魂魔尊的亲友报仇雪恨,结果却被豆神宫囚禁。

当初,青家的太上大长老想要死中求生,借助豆神宫进行一场布置,并未彻底勘破豆神宫中的奥妙。

早在三十万年前,豆神宫的主人元莲仙尊,就因此埋下了一处伏笔,应对此处,要及时为天下铲除一桩恶果。

这场恶果之因,乃是源自元莲仙尊周游天下时,拒绝了一个孩童复生其母的要求。

经过时代变迁,命运跌宕,小小原因不断转变、壮大,最终酿出太古传奇魂兽青仇。

青仇借助豆神宫而生,因此也被豆神宫囚禁。

青仇自然不愿,在此后的无数光阴岁月当中,不断尝试,想要勘破豆神宫的奥妙,重而执掌这座八转仙蛊屋。

随着时间推移,青仇日夜苦攻,成果斐然。

他不仅能够强行操纵豆神宫一段时间,更能打开门户,放任一些外人进来。

那败军老鬼、鹰姬便是寻宝而来的倒霉鬼,深入豆神宫后,被青仇强行镇压,收为了奴仆。

“这青仇得天独厚,不仅本身是传奇太古魂兽,而且能够修行,身上还自然孕育出许多野生仙蛊。其中就有八转仙蛊魂兽令。”

“他将此蛊炼化,赋予败军老鬼。想要借助魂兽大军,来帮助他镇压豆神宫。”

“可惜,败军老鬼刚刚行动,就碰到房家二仙,又遇到外人算不尽插手,不仅失败,而且还让事情败露。”

“那房家也是想着青家遗藏,别有目的,青仇兴许有所察觉,所以设下了这场埋伏。”

陈衣目光闪烁不定,心中不断地盘算。

眼下这种情况,比他想象中的要复杂得多。

原本,他以为依照先辈传承,接收了豆神宫即可。没想到还牵扯出了一个超级势力,一个传奇太古魂兽。

“可惜了。”

“我的木道杀招因果神树,还只是八转层次。若是达到九转,就能在神树上根据我的愿望,主动结出因果。”

“到时候,我从这些善果中摘取一些,逆推成因。按照这些原因来行事,照本宣科,就能巧妙地解决这场麻烦事了。”

陈衣叹息一声,隐去身形,消失在原地。

眼下的情形,他只能先旁观,等待出手的良机。

战场中,一追一逃。

“对方究竟是如何,算得出我的真身?”方源心中疑问重重。

他并不知晓,这豆神宫乃是当初元莲仙尊,根据因果神树杀招,布置下来,看似随意抛弃,其实遵循着玄妙道理。

太古传奇魂兽青仇在豆神宫中凝聚,也因此得到因果神树的部分威能,化为执念,深深地根植在青仇的复仇欲望当中。

所以,青仇能够轻松地辨别出,任何和幽魂关系亲密的人物。

从本质而言,这的确不是智道手段,而是木道。

若是寻常的木道手段,也还罢了,关键是这木道杀招来源于元莲仙尊,更是他成尊之后才自创出来的绝妙手段。

另一方面,方源继承了幽魂真传,也成了一种把柄。

幽魂魔尊当年屠戮了青家全族,曾经种下的因,如今得到的恶果,就投到了方源身上。

不过方源的这番伪装,虽然没有骗过青仇,但却瞒住了战场之外的陈衣。

方源并未让逆流护身印显露威能,种种智道手段也改良过,添加了偷道凡蛊后更是大变模样,本身又有鬼官衣等等防备着,陈衣的注意力主要都在豆神宫上,没有发现方源并不奇怪。

现在的问题就是,那豆神宫还在追杀着方源。

“这敌人未免太执着了?埋伏包围已经被我破了,还想杀我?”方源疾飞。

刚刚豆神宫被双翅黑蟒太古魂兽一撞,速度大减,被方源趁机拉开了距离。

现在速度提升上去,但是想要追上方源,还需要一段时间。

“我虽然有逆流护身印,可以和这敌人死磕,但没有必要!敌人是冲着房家来的,我出来一战,差点身亡,已经展露出十足的诚意了。接下来就看房家的了。”方源眼底闪过一抹精芒,直接向房家的七转仙蛊屋落英馆飞去。

落英馆中,房安蕾深吸一口气,面对豆神宫,她感觉到庞大的压力。

但她一脸坚毅之色,对身旁两人吩咐道:“房棱房云,辅助我催动落英馆。我们先将算不尽救下来,再好好地和这豆神宫一战!”

情形其实有点搞笑。

青仇要杀方源,但方源不知道。

方源觉得,青仇是来对付房家的。

房家也觉得,青仇是来对付我们的,算不尽是帮忙的,已经很有诚意了。所以我们要当仁不让,和这豆神宫好好斗一斗!

就连旁观的天庭八转蛊仙陈衣,也觉得是这样。在他眼中,这个智道蛊仙算不尽还有点倒霉,摊上这种事情!

ps:第二更奉上,月初求一下保底的月票。这段时间生活上奔波劳累,无法集中精力。但还是要好生奋斗一下。再不努力挣扎一番,2016年就都过了!(未完待续。)

------------

\end{this_body}

