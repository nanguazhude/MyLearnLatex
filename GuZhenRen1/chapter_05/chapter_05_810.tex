\newsection{绿蚁收徒}    %第八百一十四节:绿蚁收徒

\begin{this_body}



%1
东海。

%2
潜藏在云层之中的方源本体,忽然神色一动:“嗯?分身的气运又有了变化!”

%3
方源放眼望去,只见煮运锅旁的有一大团的浓重黑云气运。

%4
黑云气运当中,各有四团异象,代表着四位八转蛊仙,镇守这股黑云气运。

%5
象征方源分身的紫色小龙气运,被黑云气运团团围住,在里面盘踞低吼。

%6
小龙精神比之前还要振奋,一丝一缕的青紫气运,不断地从黑云气运中转化,增添到紫色小龙的体内。

%7
紫色小龙受到这样的资助,体型好似膨胀了一些,爪牙也仿佛锋利了一些。

%8
最明显的变化是这头小龙,原本只是盘踞在原处,但现在却是舒展身躯,主动出击,开始绕着原地,向黑云气运中试探。

%9
“由静化动,士气昂扬,看来是分身在梦境中有了突破,掌握了脉络,所以变得如此主动。这是好事!”

%10
“可惜,我再不能动用运道手段,来帮助分身了。”

%11
龙人分身若是光魂魄进去,留着肉身在外,方源本体还可勉强向他灌输气运。

%12
但现在,龙人分身魂肉皆进入梦境,被梦境完全包裹,任何的运道手段都隔绝了。

%13
方源本体也爱莫能助,除非是将梦境取走一部分,将龙人分身暴露出来。

%14
或者等到遥远的将来,梦道的钻研有了成果,方源可以在运道中掺和梦道成果,如此一来也就可以针对梦境的遮蔽了。

%15
梦境继续着。

%16
龙人分身周详思考,谨慎分析,摸清脉络,一番对话表露自己的“心意”,得到了父亲的认可,也通过了第二幕梦境的考验。

%17
不过,情报还是要多刺探一些的。

%18
书房内,方源当即又道:“爹,既然咱们父子同心,那又给如何做,才能令我龙人一族昌盛呢?想必父亲已有定计,还请教教孩儿。”

%19
龙人蛊仙微笑道:“的确是早有大计,实话告诉你,不只是你和为父,还有诸多的龙人一族的有识之士。只不过这些人的身份,你现在还不宜知晓。这既是对他们的保护,也是对你的保护。”

%20
“行大事,首要便是隐秘。暗事好做,明事难成。这一点孩儿明白。”方源点头道,“但但凡大计,总得有个章程吧。”

%21
“这个章程,就是这个。”龙人蛊仙大笑一声,抬手取笔,在书桌上的大纸上写下最后一个字——“下”!

%22
和之前的三个字合起来,便是“龙行天下”。

%23
方源眼中精芒一闪。

%24
龙人蛊仙忽然吹了一口气,这气息锋锐如刀,直接割开了大纸。

%25
龙人蛊仙将其中一小部分纸张,拿起来,交给方源,上面只有一个字——“下”。

%26
而剩下的大半“龙行天”,龙人蛊仙一拂袖,纸张陡然自燃,迅速烧成灰烬。

%27
龙人蛊仙望着书桌上的这片灰烬,叹息一声:“总有一天,我们龙人一族都会高翔苍穹,诸天都在我们脚下。但是现在,人族势大至极,但是当之无愧的霸主。纵观大局,我们龙人势单力孤,在外根本没有一处可靠的异人势力结盟,在内我们龙人本身都是依附于人族。”

%28
“要高翔苍穹,必有一个过程从下而上。如今,我们就要屈居人下,默默积累,沉淀底蕴。我们龙人一族刚刚成形,底蕴太薄,幸好我们产生于人族,和人族关系最为密切。尤其是在中洲,我们在十大古派都是有同族之人。我们需要的是不断学习,不断积累,追上人族。”

%29
方源追问:“那孩儿又该如何去做?”

%30
龙人蛊仙笑道:“你早已经开始做了。在你孩童时代,为父就发现了你的天资,便刻意栽培你成为奴道蛊师。如今你的奴道造诣,已经登堂入室,接下来为父已经替你做了安排,参加绿蚁居士的收徒大典。”

%31
“绿蚁居士?”方源故作不解,心中却是恍然,联想起一个名号——居易仙侣。

%32
这两人乃是夫妻,中洲历史上有名的散仙大能。男子名号绿蚁居士,女子名为易酒仙姑,他们俩自小就青梅竹马,两小无猜。

%33
本来是凡人,但各有仙缘。因为一场灾害,男女少年时被迫分离,各自思念对方,均以为对方已亡。

%34
成就蛊仙之后,男子不娶,女子不嫁。天公作美,让他们两人碰巧相见,一时间难以置信,惊喜交加,如堕美梦之中。

%35
双方倾诉衷肠之后,女子笑称男子:“我还记得你孩童时候,最喜欢玩泥巴,斗蚂蚁。”

%36
男子也笑:“小时候,你家开了酒肆,家境比我要好。少年时代,你就挑着担子,走街串巷去卖酒。我就一直陪着你,真希望陪你这样一直到老。”

%37
于是,男子便自称绿蚁居士,女子也改了名号,称为易酒仙姑。

%38
两人俱都才情卓绝,品德上佳,结伴同修后,琴瑟和谐,举案齐眉。

%39
龙人蛊仙继续道:“绿蚁居士曾经和十大古派中的洪贞作赌而败,答应收得一徒,传授衣钵。如今按照约定,已经是到了收徒的时候。但十大古派中却没有商量定好人选,皆因当事人蛊仙洪贞早已渡劫而亡。”

%40
“绿蚁居士乃是人族大能,但却是散修。我儿若能成为他的唯一徒弟,不仅能够习得他的奴道手段,更且和居易仙侣搭上关系。将来龙人一族若有难处,这两位八转大能可为强助!”

%41
“原来如此,孩儿懂了。”方源顿了顿,继续道,“孩儿定会拼尽全力,成为绿蚁居士之徒。”

%42
龙人蛊仙点头:“好,收徒大典就在下月初,这阶段你就用功闭关,为父会请诸多教习,帮助你训练。”

%43
第二幕梦境消散,第三幕梦境来临。

%44
高山耸立,云雾缭绕。

%45
青竹林地中,一处茅屋前,小小的空地上,却是拥挤着十大古派的年轻一代。

%46
这些人俱都是少年才俊,各个天资卓绝,都是为了竞争绿蚁居士的徒弟名额而来。

%47
方源恍惚了一下,打量环境,发现自己正置身在少年当中。

%48
只不过,绝大多数的人族少年都下意识地疏远他,他身边仅有两位龙人同族。

%49
而空地中央,则是两位少年蛊师正在拼斗。

%50
方源顿时认出一人,暗道:“那不是张双么?”

%51
此时张双已经处于下风,他的对手乃是一位姑娘,眉发枯黄,容貌秀美,眼中双瞳,熠熠生辉。

%52
方源刚刚来到第三幕梦境,还不了解情况,立即刺探情报道:“你们看好谁?”

%53
他同族的两位龙人少年,听了这话,左边的那位龙鳞泛青,便道:“吴帅,你未免太看不起我们俩个。虽然我们两个都被淘汰,只剩下你晋级。但场中情况如此明显,再有十个回合,张双必败无疑。”

%54
青鳞的龙人少年比较高傲,口气有点冷。

%55
黄鳞的少年则明显性情温和许多,劝慰道:“青酸,你何必如此。胜就是胜,败就是败。你失败了,还计较什么?吴帅贤兄这么问,绝非是要找你茬的。”

%56
青麟少年青酸愤愤不平,咬牙道:“那都是人族狡诈,看我实力强盛,蓄意安排了那三个人族,故意来车轮战我,令我实力降低许多,然后再让泰琴轻巧胜我。黄维,难道你没有看出来吗?”

%57
黄鳞少年叹息一声:“我岂会不知?你我都被阻击,但事实上这也是因为我俩实力不足的缘故。若是像吴帅贤兄这样的实力,就算阻击,又有何妨?仍旧让贤兄闯到了决赛!”

%58
方源默默听着,心道:“青麟少年叫做青酸,黄鳞少年唤作黄维,而我却是已经闯入决赛了。”

%59
这时,黄维又道:“吴帅贤兄,我是佩服你的奴道造诣的。但你接下来的敌人,却也是非同小可。泰琴姑娘实力也非常高超,和贤兄相差仿佛。但贤兄你是激战不休,一步步血斗惨胜,晋升上来。而你的对手却是被一路保护,唯有到了此战,才遇上张双这个强劲对手。”

%60
方源这才明白,原来场中的黄眉双瞳的少女泰琴,便是自己接下来的对手。

%61
他心情有些凝重。

%62
因为观察片刻,他已有了许多评估:“这泰琴明显也是奴道大师境界,很多时候她调兵遣将,想都不想,依照心中的直觉,却往往效果极佳……是个劲敌啊。”

%63
方源的奴道境界,也只是大师级。

%64
这让他微微后悔,早知如此,他的本体就应该提前准备,让奴道境界提上一提。

%65
不过谁有能知晓得到,会出现这样的情况呢。

%66
“我输了。”片刻后,张双脸色苍白,身躯摇晃着拱手道。

%67
“承让了。”少女泰安回礼,神情淡漠,随即眼光又转向方源,紧紧盯住。

%68
终于到了最后一场。

%69
方源信步走上前去。

%70
他和张双擦肩而过。张双脸色复杂,想要为方源鼓劲,毕竟他俩分属同门,但又念及方源龙人身份,终究只是张了张口,没有说话。

%71
休息了片刻,双方正式开战。

%72
少年们紧紧围成一圈,对战事密切关注。

%73
一只只蚂蚁,破土而出。

\end{this_body}

