\newsection{什么东西?}    %第二百五十二节:什么东西?

\begin{this_body}

%1
义天山大战结束之后,方源就曾经在路上测验过这具身体的底细。

%2
他徒步奔跑,矫健如飞,快若奔马。

%3
轻轻一跃,就有三丈。

%4
从五十八丈跳下,用头部着地,只会感到头皮一阵酥麻,脑袋有点晕。两三个呼吸后,这些症状就会消散。

%5
并且力气很大。耐力十足,连续奔跑了一刻钟,都没有感到任何的疲惫之感。

%6
还有五感强大,动态视觉,简直优秀到了极点!视野极其广阔。遥看万步之内,一切景物,分毫毕现。

%7
剧烈运动之后,心跳很快就平复下来,每一次心脏的跳动,都沉缓有力至极。

%8
思考的速度,大大超越凡俗。就算到了智慧光晕之下,也能独自支撑良久。就算是蛊仙,单凭肉身,也极少达到这种程度的。

%9
别说是一根铁树,就是铁山撞过来,方源照样活蹦乱跳。

%10
这是至尊仙体本身的素质,和蛊虫无关,而是和自身身上的道痕有关。

%11
“在这逆流河中,我不能动用蛊虫。其他人也不可以!”

%12
“但我的至尊仙体,却是足够强大。一身道痕,更是非同寻常。”

%13
方源想得深入,心情也跟着渐渐激动起来。

%14
“影无邪!”

%15
他甩动四肢,就像一条箭鱼,朝着前方游去。

%16
整个逆流河十分湍急,但总体的大势,是受到子阵牵引,向玄极子、洪极子所在的方位奔流而去。

%17
吼!

%18
一头斑斓大虎,正用前肢爬在一块浮木上。

%19
方源游到大虎面前,大虎原本无动于衷。但方源气喘吁吁,也趴在了浮木上稍作休整。

%20
这块浮木哪里承受得了一人一虎的重量,立即开始下沉。

%21
大虎着急,立即张开血盆大口,就要撕咬方源。方源冷哼一声,双臂一伸,一手按住虎头,另外一只手捏起拳头,砰砰两圈,就将这头大虎直接打死了。

%22
“幸好这里没有什么荒兽,只有普通野兽而已。”

%23
若是面对荒兽,至尊仙体也是完全不够看的。

%24
大雪山福地中,当然不缺荒兽。但是为了炼制鸿运齐天仙蛊,开启大阵,防御来敌,大雪山福地中早已经整改过了。拥有荒兽的蛊仙,都将其收入仙窍当中去,不然凭白无故任由来敌斩杀,岂不是蠢笨的举动?

%25
方源在逆流河中游曳,一心想要找到影无邪。

%26
大雪山福地崩溃,逆流河汹涌而出,因此方源并不知道,紫山真君已经苏醒过来。

%27
他的心中,假想敌还是雪胡老祖。他初步估算,觉得影无邪和大雪山这一方有些联系,但关系并不紧密。若非如此,义天山大战时,怎么不见雪胡老祖出手相助呢?

%28
方源的分析和推理,自然没有什么错误。

%29
但他信道手段缺乏,情报太少。

%30
“现在这种情况,我即便遇到雪胡老祖也有自保之力。因为逆流河中,不能动用蛊虫,环境特殊!”

%31
“影无邪既然和雪胡老祖有关联,若是错过这一次机会,恐怕他们就要在雪胡老祖的庇护之下,不断壮大。”

%32
“我若错过这次机会,恐怕将再无追杀铲除影无邪一行人的可能!”

%33
抱着这样的想法,方源休息了一小会,便重新启程。

%34
然后,他就见到了碧晨天。

%35
中洲的八转蛊仙!

%36
方源都吓了一跳。

%37
“怎么回事?”

%38
“八转蛊仙,并非雪胡老祖,而且还是中洲蛊仙?!”

%39
碧晨天很狼狈,他趴在一只巨龟的背壳上,浑身是伤。

%40
这些伤当然不是他在逆流河中撞的,而是之前交战中所受的创伤。他是八转蛊仙,道痕互斥,一旦受伤,想要痊愈就比较麻烦。

%41
可惜交战中,他是处于劣势一方,怎可能有时间治疗自己?

%42
所以,母阵崩溃之后,他被逆流河卷席,身不由己,无法动用蛊虫,身上伤势就拖延了下来。

%43
不过幸好,他在湍急的河流中,遇到了一头浮游的野龟。借助这只野龟,碧晨天终于有了喘息之机。

%44
方源发现了碧晨天,碧晨天也很快发现了方源。

%45
但碧晨天并不认识方源。

%46
因为方源此时的相貌,是至尊仙胎体!至于原来的肉身相貌,倒是被天庭暴露,如今在五域广为流传。

%47
碧晨天虽然不认识方源,但他双眉皱起,相当的警惕。

%48
因为方源的气息,表明他就是北原蛊仙。

%49
这也是因为至尊仙体。

%50
穿梭过界壁之后,方源的气息可以在五域中任由转变,去了哪一块地方,就是哪一块的蛊仙,能和各地域完美融入。

%51
“此人是谁?大雪山福地的各大峰主,都没有他这样的人物。不过,也也有可能是雪胡老祖暗地里招揽的。”

%52
碧晨天从龟背上站起了身,目光紧紧地注视方源,盯着他的一举一动。

%53
在逆流河中,碧晨天虽然修为高达八转,但蛊虫催动不起,战力相当有限。

%54
方源也警惕地看着他,没有过多逼近。

%55
方源虽然有至尊仙体,但若是因此以为,自己是逆流河中的强者,那就是蠢透了!

%56
逆流河中蛊虫不能催动,然而身体却无碍,道痕可以发挥作用。

%57
方源的至尊仙体十分强力,是因为仙体身上的道痕。

%58
同样的,若是一位蛊师用过黑豕蛊,为自己增添了一猪之力。那么相应的力道道痕,就刻印在这位蛊师的身上。这位蛊师若是落入逆流河中,不能动用蛊虫,但仍旧可以凭借肉身,发出一猪猛力来。

%59
放在其他蛊仙身上,这些蛊仙在漫长的修行过程中,怎可能不运用一些类似黑豕蛊的蛊虫,来为自己的身躯刻印上各种道痕呢?

%60
这当然是一定的!

%61
谁也不是傻瓜。这种道痕的优点,一目了然。

%62
方源有肉身上的优势,其他蛊仙同样也会有,只是程度不一而已。

%63
对于八转蛊仙而言,底蕴雄厚,方源现在面对碧晨天,并不想与其开战。

%64
“你是中洲蛊仙?”

%65
“我虽然是北原蛊仙,但我和雪胡老祖有仇。”

%66
“我见他大炼八转仙蛊,并不愿意看到他成功,所以就潜伏在福地外,想看看有没有机会破坏。”

%67
“你是他请来炼蛊的助手吗?”

%68
方源一句句试探道。

%69
碧晨天的眉头稍微疏解了一点,他答道:“雪胡老祖同样是我的敌人。”

%70
“不久之前,我在大雪山福地中和他交手,没想到忽然意外发生了。”

%71
“我身上的伤势,就是雪胡老祖造成的。”

%72
双方交谈,互换了一些情报。

%73
方源这才明白,原来大雪山福地中,竟发生了这么多的事情!

%74
不过碧晨天的很多话,模棱两可,方源得到的情报并不多。

%75
他们俩彼此之间,还不信任对方。

%76
这是当然的。

%77
只是浅浅的交流一番而已,人心隔肚皮。

%78
尤其是在这种蛊虫无法催动的环境下,双方都保持着最高的警惕。

%79
方源知道自己不能在碧晨天身上,得到更多的情报后,他选择离开。

%80
他主动游离那头野生巨龟。

%81
即便是交谈过程中,他也没有主动靠近,而碧晨天也没有邀请方源上龟背来,稍作休息。

%82
碧晨天见方源主动退走,脸色更缓和了许多。

%83
但就在这个时候,一个小小的意外发生了。

%84
啪。

%85
一声轻响。

%86
一个拳头大小的东西,被河水冲击,一下子撞到了方源的脸上。

%87
这个小东西软软的,就像是水母一样,粘在方源的额头。

%88
“什么鬼东西?”方源楞了一下,正想要一手将其扯开。

%89
但这个时候,他的耳畔传来碧晨天的低呼声:“你轻一点,不要弄坏了它!”

%90
“嗯?”方源心中微讶,能够让八转蛊仙都紧张的,会是什么东西?

\end{this_body}

