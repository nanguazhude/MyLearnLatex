\newsection{万物大同变}    %第六百七十三节:万物大同变

\begin{this_body}

“仙道杀招万物大同变。 .更新最快”方源口中喃喃一声。

他端坐在祥云之上,太古白天中一片空荡,一望无边。

从兽灾洞天中主动撤离出来,他就积极地将此次行动的收获消化个干净。

兽灾仙人的那一超绝手段,便是仙道杀招万物大同变,使用此招之后,便能够增添海量的变化道痕,同时将对象身上的其他异种流派的道痕,都转变成变化道痕。

此招消耗大量的仙元,同时还有丰富的情感。这些情感必须是正面、光明的。从这一点上,方源终于明白兽灾洞天中,那些战兽勇士的性情了。

仙元方源不缺。这段时间以来,虽然升炼仙蛊规模巨大,但是方源的收益同样很多。尤其是炼蛊后期,方源进步极大,炼蛊的消耗不断下降。所以如今方源结余的八转仙元,已然不少,可以支撑两场八转蛊仙级数的常规战斗。

丰富的情感,方源也有。正面、光明的标准,方源也很容易通过智道的手段达成。他的智道造诣也相当不俗的。

方源缺的是仙蛊。

组成此招的仙蛊有四只,其中最核心的便是八转仙蛊变通!

毫无疑问,这是变化道的仙蛊,也是兽灾仙人的本命蛊。其余的三只仙蛊,则都是七转。

“这些仙蛊即便天灵不贡献出来,我也能抢夺得手。大盗鬼手已是经过我的改良,更加可靠。”

“现在只需要天相杀招不断地侦查,侦查到这些仙蛊的隐藏之地,我便能通过偷道手段,将其得手。”

“不过,要得到整个兽灾洞天,就有点麻烦了。”

对于方源而言,兽灾洞天本身也有巨大的价值。但是天灵却有摧毁洞天的本领,若是冒然用强,必然会得不偿失。

“甚至,从这个仙道杀招万物大同变来推算,兽灾洞天中的蛊仙恐怕还有一张底牌。”

荒兽、上古荒兽可以运用万物大同变,星海螺躯壳也能如此,这些蛊仙并没有掌握万物大同变,根源在于兽灾仙人将此招作用到了整个兽灾洞天当中。

也就是说,整个兽灾洞天在逼不得已的情况下,也能通过万物大同变,和蛊仙结合成一体。

这就能铸就一位临时的八转战力!

当然,方源不是没有把握战胜这种八转,但是最终亏损还是要落在方源头上。就算他战而胜之,经过万物大同变的兽灾仙窍定然是损失惨重。

“所以,我要进行推算,如何克制对方利用万物大同变牵连整个兽灾洞天。”

“同时,再运用见面曾相识杀招,进行第二次攻略。”

兽灾天灵说过,任何一人只有一次机会继承机会,但它的侦查能力如何?是否能辨别出方源的真实身份?按照方源的推算,这种可能性很小。

不过,就在方源暂时撤退,积极推算克制之法的时候,他意外地得知了一个情报。

这个情报相当重要!

“龙宫出世了?并且天庭龙公亲自出手,都没有收取得了?哦,是因为东海的四大八转蛊仙联袂出手阻挠的缘故?”

“有意思!现在八转仙蛊屋龙宫在东海的天空中疾飞,气息毫无掩饰,已经惹得天翻地覆,无数势力、蛊仙强者都在密切关注。”

方源眼中精芒烁烁。

对于龙宫的存在,他是知晓的。

影真传中,就有关于龙宫的线索。但是这个线索,只是表层的,并没有牵涉到关键。最有价值的内容便是:龙宫规定,它的继承者必须是龙人身份!

方源虽然觊觎过这种八转仙蛊屋,但一直以来,都没有什么良机和时间。虽然有白凝冰这个钥匙,但是要寻找龙宫显然十分困难,要耗费大量的时间和精力。

按照之前的情势,八转仙蛊屋虽强,但若方源冒险搜寻它,一旦得不到手,他就会被天庭接踵而至的攻势剿杀。

这里面的收益很大,但风险更大,所以方源一直都没有出手。

“原来天庭方面也一直在暗中搜寻龙宫,并且差点就要得手了。”

认识到这一点,方源不禁要对那四位东海八转称赞一声:干得好!

没有思考更久,方源旋即动身,离开太古白天,前往东海而去。

兽灾洞天就在这里,不会被轻易发现,可以缓一缓。龙宫这事则很紧急,绝不能被天庭夺走。

这可是八转仙蛊屋,又有梦道的神秘仙蛊!

数天后,东海。

轰!

剧烈的爆炸中,蛊仙石淼大口吐血,狼狈飞退。

龙公长啸一声,身化光虹,暴射而出,直扑前方的仙蛊屋龙宫。

眼看双方距离急速缩短,关键时刻,容婆及时出现,挡在了龙公面前。

她浑身紫气泛滥,身受重伤,身体虚弱,脸色苍白如纸。

龙公威势凶猛,见到容婆身边的紫色毒气,却是瞳孔微微一缩,身形猛地一折,绕了过去。

容婆乃是毒道蛊仙,这紫色毒气就是她的招牌手段,就连龙公都不愿意沾染。

容婆状态很不佳,这几天来的鏖战,令她损失极大,尤其是一只核心仙蛊毁坏,导致她速度暴降,陷入极大被动,靠着自身的顽强意志和其他三位蛊仙的帮衬,这才咬牙坚持到现在。

龙公绕过容婆,乃是最明智的战术。

容婆只能眼睁睁地望着龙公,满脸苦涩之意,根本没有追赶上的可能。

四位八转中另外两位,则还在数里之外。

轰!

龙公飞到龙宫上方,忽施辣手,重重一击将这座八转仙蛊屋直接打落下去。

砰。

仙蛊屋跌落到海水中,立即掀起巨大的浪花。

龙宫灵性十足,没有再飞上空中,而是顺势潜入海水更深处。

龙公正要追击,忽然身上一凝,猛地出手击向左边。

他的左边空无一物,但凛冽的攻势却忽然停滞下来,被一位蛊仙挡住。

蛊仙显露身形,对龙公微笑道:“不愧是龙公前辈,能看得破我的伪装。”

龙公皱起眉头,冷笑:“宋启元,你乃是宋家堂堂的太上大长老,八转的修为,居然行这偷袭之事?”

来者正是宋启元。

“看来东海的正道终于出手了。还有谁,都出来罢。”龙公望了一眼脚下的海水,龙宫正逐渐隐没,他却没有追击,而是悬停高空,战意凛然。

于是下一刻,又有两位蛊仙的身影出现。

一位青袍中年,乃是青岳家族的太上大长老青岳安。

一位明眸少女,袖珍的彩色云朵缭绕身边,乃是华家太上大长老华彩云。

“龙公大人……”华彩云刚开口。

“废话太多!”龙公轻啸一声,主动出击。

轰轰轰!

几记仙道杀招对拼之后,三位正道八转立落下风。

“龙公端得厉害,东海的正道八转,他以一敌三,都能轻松压制!”方源此时变化成一头海蛇,在海底里潜游,心中暗凛。

正道八转和散修八转,通常是前者更为强大。因为他们有更完整的仙道杀招,充沛的修行资源。

这三位正道八转联手,绝对胜过容婆、扬子河、张阴、石淼四人。

但仍旧不是龙公的对手。

龙公太凶猛了,直至现在,他都只是施展变化道的手段,最为拿手的气道杀招从未用过。

从这方面来看,紫山真君能逼得龙公施展气道绝招,也可见证前者的强大。

“不是紫山真君太弱,而是龙公更强!”方源远远跟着龙宫,渐渐脱离战场。

宋启元、青岳安等人既然出现,证明东海的正道之间已经商议妥当。在五域中,东海因为资源最为丰富,八转蛊仙的数量也是最多。

“如今只出现三位正道八转……一定还有人手。”方源决定按兵不动。

果然不出他所料,很快又有两位正道八转现身,拦住龙宫。

张阴、容婆等人也紧随而来,呼啸呐喊,浴血奋战,迎击两位正道八转。

海底的大战也旋即爆发。

方源连忙潜伏到海底的泥沙当中,等候良机。

\end{this_body}

