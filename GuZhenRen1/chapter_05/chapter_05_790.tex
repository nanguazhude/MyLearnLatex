\newsection{看到天庭}    %第七百九十三节:看到天庭

\begin{this_body}

%1
这地脉仙蛊形如蚯蚓,但握在手中,却很沉甸甸,一点都不滑腻冰冷,反而温润暖人。

%2
它通体半透明,就像是一个褐色的琥珀,十分精致。

%3
有了这只七转土道仙蛊,方源就可以组建五界大限阵了。

%4
关于五界大限阵的改良,方源的宙道分身已经顺利完成了。智慧光晕一如既往的给力,同时方源的阵道、律道底蕴也是有力的支撑。

%5
新的五界大限阵,方源不仅增添了地脉仙蛊,而且还有阵灵、阵旗。

%6
有了阵灵,方源就有一个极强的辅助者,帮助他操纵整座大阵,时刻洞察风吹草动。

%7
有了阵旗,方源就可以直接将大阵从仙窍中搬出去,或者又从五域外搬进仙窍中国来。

%8
太古年兽钓来阵一直在用着,进展稳定。

%9
年华池中已经有二十多头的太古年兽了,种类多达七种。

%10
已经收集了超过一半的种数,但未来能否筹集全部,方源也说不准,这点要看运气。

%11
为了增强自身运道,方源开始命令毛民蛊仙,着手炼制运道仙蛊。

%12
方源手中的仙蛊数量很庞大,早已经突破一百大关,但运道仙蛊并不多。

%13
尤其是在方源掌握了运道真传的情况下,运道仙蛊一旦炼制出来,就能上手运用。

%14
再加上天庭在运道方面,底蕴很薄。

%15
因此,培养运道方面的实力,收效是显而易见的高。

%16
至于运道仙材,虽然罕见,难以收集,但是方源有着那么多的南疆正道蛊仙可以敲诈。

%17
这些运道仙材筹备过来,方源就让那些毛民蛊仙炼制。

%18
因为不是升炼仙蛊,而是炼制六转运道仙蛊,所以即便失败,造成的损失方源也能够承受。

%19
他现在底蕴很足,手头宽裕,和上一世同期有着巨大的提升。

%20
关于仙道杀招的练习,五禁玄光气可以放置一边了,光阴飞刃杀招虽然没有练习,但方源却又有了不少心得体会。

%21
他对此招的本质又看清许多:这一招就像是一个橡皮擦,运用宙道的威能来抹除敌人,同时也抹除自己。

%22
有关此招记忆消散的弊端,是不能抹消的。这是光阴飞刃杀招本身的结构所决定的,一旦消除了这个弊端,就是将整个杀招的构架彻底推翻。

%23
如此一来,光阴飞刃的超强攻伐的特征也就全然消散了。

%24
方源只能无奈地接受这一缺点。

%25
本体进展不小,分身的进展就更大了。

%26
纯梦分身的魂魄逐渐改变,开始切合分身肉体。

%27
方源的龙人分身,则已经从一转蛊师,晋升成了六转的龙人蛊仙,拥有了上等福地。

%28
方源开始将更多的时间,用来探索梦境。

%29
这段时间,他和池曲由的交易不断进行了许多次,仙窍中已经积蓄了一定规模的梦境。

%30
方源遁出魂魄,钻入一片气道梦境之中。

%31
视野大变,他置身在一座山丘峰巅,放眼望去,雾霭层层,寒风彻骨。

%32
他再看自己,六转蛊仙修为,一身狼狈,伤势不少,有一些伤口还流淌着鲜血。

%33
“这算是哪一出?”方源心中正在琢磨,忽然身后狂风吹来,雾气消散,露出三位墨人蛊仙。

%34
“卫玉书你哪里走?!”

%35
“给我死!!!”

%36
话音还未落,就有三记杀招斩向方源。

%37
这异变来得太突然,方源还未熟悉仙窍中的蛊虫,就已然陷入生死绝境!

%38
砰!

%39
下一刻,他身躯被炸得四分五裂,惨死当场。

%40
方源只得抽回梦境,回到现实中来。

%41
“没想到一入梦境,就遭遇这样的难关,死上了一回。”

%42
他检查魂魄上的伤势。

%43
出乎意料的,魂魄上的伤势并不严重,伤势轻微,只需要上百只的胆识蛊就能治愈。

%44
“看来我拥有了荒魂之后,魂魄质变,能够承受更强的梦境反噬了。”

%45
方源休整一番,便又继续探入梦境。

%46
同样熟悉的场景,方源入梦之后,立即查看自家的蛊虫。

%47
他发现,手中的仙蛊只有一只,它形如蟑螂,灰白色,仿佛石头质地,但很轻。抚摸上去,冰冰凉凉,有一些粗糙的感觉。

%48
方源依凭记忆,迅速想到此蛊跟脚:“这是化气蛊啊。”

%49
一只六转的化气仙蛊。

%50
除此之外,则是大量的五转气道凡蛊。

%51
还有三股仙材,一股漆铁硬气,一股兰花生气,一股悠游龙气。

%52
方源有些懵:“化气蛊只是辅助性质,单凭这些凡蛊,如何抵挡得住身后追兵?”

%53
正想着,身后就传来大喝声音。

%54
“卫玉书你哪里走?!”

%55
“给我死!!!”

%56
砰。

%57
方源下一刻,再次四分五裂。

%58
来到现实,方源一边疗伤,一边沉思:“照眼下情形,很明显,是需要催动仙道杀招,才能抵御这三位追兵。”

%59
“气道的仙级杀招么……”

%60
方源气道境界很低,但他却有丰富的库藏。

%61
盗天真传、巨阳真传也就算了,影宗真传中包含广阔,就记录有许多的气道杀招。同时,琅琊派中也有收藏。

%62
就算这些都没有,单凭琅琊派中那些庞大的气道仙蛊方,方源依凭炼道境界,也能将其转化成仙道杀招来。

%63
有了这样丰富的理论打基础,方源又借助智慧光晕,很快就想到了解决梦境第一道难关的杀招。

%64
不过,气道杀招他还十分陌生。

%65
于是,他又在现实中多加练习。

%66
他当然没有化气仙蛊,但多少可以模仿此招,加以锻炼。

%67
之后,再回到梦境,方源终于有了底气。

%68
尝试了几次之后,方源便催动杀招成功。

%69
他浑身罩住一层漆黑的铠甲,追杀过来的三位蛊仙再也不能将他瞬杀。

%70
“卫玉书,你居然还敢反抗!”

%71
“区区一个奴隶,好大的胆量!”

%72
“你竟冒犯主人,死不足惜!杀了他,把他炸成肉泥骨渣。”

%73
三位墨人蛊仙咆哮连连,围绕着方源猛攻。

%74
方源勉强支撑了一阵,便迅速不支,最终又被他们三人杀死。

%75
出了梦境,方源总结经验教训:“看来不是要和他们对抗,而是以逃跑为主。”

%76
再一次尝试,他便不用之前的招数,而是改换成腾挪效用的仙道杀招。

%77
失败一次后方才成功。

%78
这一成功,方源就顺利地来到了第二幕梦境。

%79
噗。

%80
还未等他打量四周,他便当即吐出一口血来。

%81
“身受重伤,并且仙窍中的仙材气息,少了一大半。”方源心头微沉。

%82
他虽然只有化气蛊,但可以凭此仙蛊充当核心,用那三股仙材气息辅佐,施展出仙级的气道杀招来。

%83
第一股漆铁硬气,可以化作一身黑色铠甲,保护方源。

%84
第二股兰花生气,可以治疗自身。

%85
第三股悠游龙气,则是方源腾挪之用。

%86
方源在梦境第一幕,逃出三位蛊仙追杀,就是用的悠游龙气。

%87
现在,方源进入第二幕梦境,一上来就发现,仙窍中的悠游龙气消耗大半,硬气消耗一半,兰花生气倒是完整无缺。

%88
“眼下状态不妙,还是先治疗一下自身才好。”方源默默酝酿,催起一记仙道杀招——吐气如兰。

%89
仙元灌注,兰花生气猛地少了一小截。

%90
方源喉结滚动,腮帮瞬间鼓起,自然而然地张口一吐。

%91
他吐出的气息,清香四溢,生机灵动,像是一团浓烟雾气,带着淡淡的绿色,迅速将他整个人全身包裹。

%92
片刻后,雾气飞快消散,方源伤势恢复大半。

%93
“接下来该往哪里走?”方源又犯难了。

%94
他现在身处洞中,似乎是在山中,又好像在地下,总之地洞又长又多,四通八达。

%95
方源刚走几步,就看到身前闯出一位墨人蛊仙。

%96
这墨人蛊仙看到方源先是一愣,旋即大喜:“卫玉书!原来你在这里。哈哈哈,好得很,竟叫我撞见了你,让我得了这场功劳。真的是来得早不如来得巧啊!”

%97
方源也很吃惊,眼前的墨人蛊仙不是之前第一幕的三仙之一,而是陌生面孔。最关键的是,他不是六转修为,而是具备七转修为的异人蛊仙。

%98
七转墨人蛊仙扑向方源,方源哪里能敌,很快就被他打得四肢断裂,瘫倒在地上。

%99
墨人蛊仙冷笑,俯瞰着方源:“卫玉书,你平时不是很狂么?没想到会落到我手里吧。哈哈哈,说实话,我也想不到。你这小子本来锦衣玉食,深得王女宠爱,竟然蠢到逃跑,真是自罪孽不可活!!”

%100
下一刻,方源魂魄受创,再次被迫回到了现实当中。

%101
“这么说来,就算被捉住,没有被杀死,也算是我闯荡失败。”

%102
“看来这片梦境,是一片逃脱的梦境。”

%103
“在那种情况下,面对七转蛊仙,的确是没有办法抗衡的。”

%104
“再来!”

%105
方源冲击第二幕梦境。

%106
他尝试了许多次,屡屡失败。他发现捕捉他的墨人蛊仙竟然数量不少,六转蛊仙多达五人,七转蛊仙则有三位。

%107
都是墨人蛊仙。

%108
再加上之前那位七转蛊仙泄露出的情报,这让方源对于梦境中的时代,有了大致的猜想。

%109
不是远古时代,就是上古时代。

%110
异人的实力还是很强大的。

%111
追杀他的蛊仙那么多,手段也非常丰富。方源一败再败,毫不气馁,很快他发现了逃脱的契机。

%112
在这山里的溶洞通道中,存在着几处仙材气息。其中有一道秋鳞隐气,对他帮助最大。

%113
最终,方源凭借丰富的失败经验,还有就地取材后催发的杀招,顺利通过第二幕。

%114
到了第三幕,方源诧异地发现,梦境自行流转,他毫无插手之力。

%115
他瘫倒在地上。

%116
在他的面前,站着一位身材妙美的墨人女仙。

%117
她身上洋溢着八转气息,看着方源的目光中,带着愤怒、仇恨,也带着爱恋和怜悯。

%118
“卫玉书,我当初将你买下来,你才不过是个少年,连修行是什么都不知道!”

%119
“是我一步步栽培你,把你提拔成蛊仙。是我的宠爱,屡屡纵容你,包庇你,让你变得如此胆大包天吗?”

%120
“告诉我,为什么背叛我?”

%121
“我夜彤王女哪里亏待了你?!你是锦衣玉食,生活安然无忧,你只需服侍我,而我从未对你冷酷残忍过。”

%122
八转墨人女仙厉声责问。

%123
卫玉书惨笑一声:“然而,我生活得再好,我也只是你的奴隶!”

%124
墨人女仙更怒:“充当我的奴隶,哪里不好?有许多的墨人想要你的生活,都没有这样的机会!”

%125
卫玉书目光发愣,缓缓地道:“我以前也没有觉得不好,但当我听闻,这个世间还有一个天庭,还有人那样活着……”

%126
“所以,你就想逃脱,逃到天庭去?”墨人女仙冷笑,“你太天真了,这等邪魔外道的蛊惑之词,你也能信?!”

%127
卫玉书闭上双眼,声音嘶哑:“不是我信,而是我愿意去信。”

%128
听到这个回答,墨人女仙不禁目光微凝,怒气缓缓消散了下去。

%129
她回忆起往昔的美好时光,她是多么宠爱这个异族的奴隶,但这些日子一去不复返了。

%130
她遗憾地道:“玉书啊,我曾经多次赦免你的过错,但这一次不行了。如果再赦免你,损害的就是我这一族的威仪了。若不是我亲自出手,还真叫你逃走了。不愧是我亲自教导出来的努力。其实这里距离天庭的势力范围,已经十分接近了。”

%131
听了这话,卫玉书像是被注入了一股力量,他猛地睁开双眼,盯着墨人女仙。

%132
他哀求道:“我在死前,没有什么其他愿望,只有一个想法,请让我爬出这洞,望一眼天庭就好。”

%133
墨人女仙沉默了一会,随后叹息一声:“也罢。”

%134
她侧身让出道路,卫玉书伸出一双血手,拼尽全力,艰难地爬向前去。

%135
他气喘吁吁,汗水和血水混杂在一起,短短的距离,他费了好一会儿工夫,方才爬出洞口。

%136
这处洞口,位于半山腰,视野开口。

%137
他满怀希冀地朝前望去。

%138
然而,却只见到厚厚的云雾。

%139
“真是不巧,山间的雾气常有呢。”墨人女仙从身后踱步而出,“但其实就算没有这雾气,你也看不到天庭。我说了,眼前只是天庭的势力范围,最边缘的地带。其实只是一片荒地罢了。”

%140
但卫玉书却似乎得到了极大的满足。

%141
他无力地趴在地上,再没有任何力量来昂起头颅。

%142
但他的双眼却闪耀着星光,他长叹道:“不,我看到了,我看到天庭了。”

%143
“你……”墨人女仙动容。

%144
卫玉书说完这句话,强撑的一口气终于散尽,他命丧当场。

%145
墨人女仙陷入沉默当中,她在沉默中死死地盯着前方,那就是天庭的方向。

%146
她的神情,带着一种史无前例的肃穆和凝重。

%147
“天庭!”

\end{this_body}

