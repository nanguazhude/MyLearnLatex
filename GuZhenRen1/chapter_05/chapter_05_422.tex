\newsection{龙公之徒}    %第四百二十三节:龙公之徒

\begin{this_body}

%1
一个多月之后。

%2
中洲,灵缘斋大本营。

%3
高空,一座道观悬浮在云层之上。

%4
仙蛊气息四处洋溢,这座道观便是灵缘斋的宙道仙蛊屋日月观。

%5
牌楼、大殿、偏殿连成一体,这座道观乃是灵缘斋的底蕴之一,最擅长的便是给予蛊仙修行的方便。

%6
平时的时候,日月观都收藏在诸位灵缘斋的太上长老们的手中,由她们轮流掌管,辅助修行。

%7
但此时此刻,日月观却被摆放出来,置于高空。并且,还催发威能,整个道观的每个角落,每片砖瓦,都轮转散发出日月的淡淡光晕。

%8
华丽的光辉,映照着周围的云层,显得祥云蒸腾,华彩遍地,一片仙家盛世之壮景。

%9
道观的主殿中,已经有蛊仙济济一堂。

%10
灵缘斋的太上长老们几乎全都到场,凤九歌、白晴仙子、徐浩、李君影,乃至最新成就蛊仙的,当代灵缘斋仙子赵怜云。

%11
不过此时场面,主角却不是他们中的任何一人,而是凤金煌。

%12
凤金煌一身素服,干净利落,小脸满是肃穆之色。

%13
她跪在地上,对主座上的龙公叩首跪拜。

%14
没错,这便是凤金煌的拜师盛典!

%15
一般而言,已经加入门派的凤金煌,很难再拜门派之外的蛊仙为师。

%16
但龙公是何等人物?

%17
灵缘斋上下无不遵从,并且欢欣鼓舞。灵缘斋的几大太上长老,甚至商议,一切都按照最高的标准来办置,更且大张旗鼓,宣扬出去,抬高本派的声威。

%18
不过,这个建议被龙公否决了。

%19
龙公关照:一切从简,勿要张扬,但标准不能少。

%20
灵缘斋太上大长老亲自主持,殚精竭虑地去查阅古籍,参考龙公时代的拜师古制,花费大量心血。

%21
龙公面色上没有什么流露,但心里对这场拜师大典,还是较为满意。

%22
尤其是当他的目光,停留在凤金煌的身上时,眼神中更透露出一股欣慰。

%23
叩首磕头的礼节完成之后,凤金煌站起身来,取出一杯茶,双手奉上,递给龙公。

%24
悄无声息的大殿中,凤金煌脆声开口道:“龙公大人在上,请受金煌一杯金碧朝露茶。”

%25
龙公接过杯盏,首先打开杯盖。

%26
顿时,茶水绽射出绚烂而又温和的金色霞光,映照得众仙头顶上方的观顶,一片灿烂。

%27
龙公缓缓闭上双眼,将杯盏举到鼻翼之前,轻轻一闻。

%28
茶香似有若无,透露出清新至极的气息,好像是春天的清晨,青草上的露珠的气味,纯净无暇,又透露出一股光明和活泼的气息。

%29
“这茶不错。”龙公淡淡地笑起来,缓缓睁开双眼,然后品茗一口。

%30
茶水入喉,果然不同凡响。即便是龙公,纵观自己一生,也觉得这是茶中佳品,不可多得。

%31
“凤金煌,你乃是炼道大宗师,炼制的茶水自然非同凡响。好,从今日起,你便是为师的二徒弟了。”

%32
此言一出,大殿中顿时嗡嗡作响。

%33
各个蛊仙笑逐颜开,哪怕是徐浩、李君影亦不例外。

%34
“奏仙乐。”灵缘斋太上大长老轻呼一声,下一刻,嘹亮而又美妙的仙乐奏起,又不失庄严和恢弘。

%35
“徒儿拜见师父!”凤金煌再次拜倒在地上,这次才能称呼龙公为“师父”,并以“徒弟”自称。

%36
至此,时常一个多时辰的拜师大典,这才成功了结。

%37
大典的意义,非常重大。

%38
灵缘斋的太上长老们,都满脸红光,兴奋的神情难以掩盖。

%39
这可是天庭领袖龙公大人亲自收徒啊!

%40
整个灵缘斋都感到荣幸。

%41
白晴仙子作为凤金煌的母亲,更是当场激动落泪。

%42
凤九歌站在她的身边,轻轻握着爱妻的一只手,亦是欣慰至极。

%43
“从明日起,凤金煌将与我同往天庭,进行修行。”

%44
“凤九歌、白晴留下,其他人都退下吧。”

%45
龙公挥退群仙,留下凤金煌的父母双亲。

%46
“凤九歌(白晴)拜见龙公大人。”两人上前恭敬行礼。

%47
龙公微微点头,瞥了一眼白晴仙子后,便将目光停留在凤九歌的身上。

%48
“这一次擒杀方源的任务,失败了?”龙公开口,语气淡淡。

%49
凤九歌:“在下惭愧。”

%50
数月前,凤九歌等人就守候西漠的那处光阴支流,但哪里能等来方源?

%51
时间一长,紫薇仙子也察觉不妥,又因为凤金煌拜师这件大事,便将凤九歌召回中洲。不过其余两位天庭八转蛊仙,仍旧停留在西漠那处,并且开始布置仙道蛊阵。

%52
龙公继续道:“方源乃是天外之魔,并且还是世间唯一完整的天外之魔,他必定是天庭的大敌,也会是你凤九歌的此生必定要铲除的对象。”

%53
“我平生只收徒两位,这里的深意你们想必是清楚的。没错,凤金煌便是大梦仙尊的种子,而你凤九歌正是她的护道人。”

%54
“什么?我是煌儿的护道人?”凤九歌诧异。

%55
“此乃天机,本不可泄露。不过此事早已经被影宗洞察,也就无所谓了。大时代将至,这一次将会诞生大梦仙尊,并且天地之间将引来一场前所未有的巨大变革!方源将是我天庭的拦路石,也是你们父女两人生命中,不可逃避的死敌。你们要担负起命运交托于肩头的重任,领衔苍生,迈向全新的时代。”

%56
龙公说到这里,顿了一顿,留给场中三人反应和接受的时间。

%57
不一会儿,凤金煌一家三口都冷静下来。

%58
龙公又对凤金煌道:“煌儿我徒,今后你要跟随在为师的身边,接受为师的指引。你要和你的父母分别一段时间,你们一家就在这里好好叙旧吧。明日黎明,我便来接你去往天庭。”

%59
龙公嘱咐一番,身影淡淡地消失在座位上。

%60
“煌儿,没想到你就是被选中的人。到了天庭,你切要努力修行。你身上肩负着时代的重任,苍生的福祉就在你的一念之间啊。”白晴仙子双眼通红,对于即将的分离,很有些不舍。

%61
“娘,我还是有些接受不过来。这一切的发生,简直就像是做梦一般。”凤金煌投入白晴仙子的怀抱当中。

%62
“娘自从听闻龙公大人,要收你为徒,多少也料到你的事情。只是没有想到,你爹居然就是你的护道人。”白晴仙子感慨不已。

%63
“爹、娘,什么是护道人?”凤金煌好奇地问道。

%64
凤九歌解释道:“历来仙尊魔尊的成长历程当中,都会有护道人存在。每一个护道人都在尊者的成长历程中,起到了至关重要的作用。”

%65
“哦。”凤金煌又问,“师父说他一共收了两个徒弟。我是第二个,这么说我上面还有一位?他又是谁?”

%66
凤九歌、白晴仙子顿时微微变色。

%67
两人迅速地对视一眼。

%68
白晴仙子便认真地叮嘱凤金煌:“煌儿,这是一个忌讳,将来你到达天庭,千万不要随便提起。尤其是在你师父面前。”

%69
“为什么?”凤金煌更加好奇。

%70
这时,凤九歌给出答案:“因为龙公的第一个徒弟,便是历史上最为神秘的红莲魔尊,也就是你名义上的大师兄!”

\end{this_body}

