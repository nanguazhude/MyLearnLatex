\newsection{求人}    %第一百七十节:求人

\begin{this_body}

中洲,地渊深处。

美酒佳肴,摆上庄园。

“来,石兄,我早就听闻你喜好美酒。这些美酒,是我专门搜罗过来,请你品尝。”蛊仙杨峰端坐着,伸出手掌示意对面的贵宾。

这位贵宾有一头白色短发,一对金瞳闪烁着锐利的光芒,穿着一身劲装武服,腰间系着蓝银腰带,小腿小臂都有灿银铠甲。

他狼背蜂腰,浑身上下都散发出一股彪悍勇猛之气。只是他现在半倚半靠,姿态随意洒脱,稍稍冲淡了一些他的压迫气场。

蛊仙杨峰乃是七转变化道蛊仙,中洲十大派中古魂门的好手,但是在这位贵宾面前,却不可避免地被盖压风头。

因此这位贵宾,姓石名磊,乃是主修土道,兼修变化道,七转巅峰,人称仙猴王的中洲当世强者!

“那我就不客气啦。”石磊哈哈大笑,伸出手来,拿起桌案上的沉重酒坛。

拍开封泥,顿时一股冲天的酒气,仿佛是剧烈燃烧的火焰,汹涌澎湃地涌现出来。

一瞬间,偌大的庄园,就充斥着浓郁至极的酒乡。

酒气呈现赤红之色,在酒坛口附近,萦绕不散,仿佛是一堆火云彩霞。

石磊见之,眉飞色舞,交口称赞道:“好酒!原来是传闻中的赤帝酒。杨峰老弟,你居然能弄到这样的好酒,手段不小啊。”

蛊仙杨峰哈哈大笑:“不瞒石兄,我也是费了好大代价,才从古魂门中的库藏里讨要来的。说起来。也是沾了我门中先辈的光。三十万年前,我派中一位蛊仙先辈。在炼蛊大会中,和东海蛊仙酒中帝皇结识。两人炼蛊比试,约定哪一方得胜,就可收获另一方炼制的作品。双方连战九九八十一天,各有胜负,总体不分上下。这赤帝酒正是酒中帝皇,输给我派中先辈的东西。”

中洲十大古派中的古魂门,沾了一个古字,可谓名副其实,门派历史非常悠久。乃是十大古派之最。

三十万年前,那是中古时代。相继出现了元莲仙尊、盗天魔尊、巨阳仙尊。一位女仙蛊水尼,开创水道,创建灵缘斋,也在这个时代。

“哦?”石磊起了兴趣,“酒中帝皇,此人我知道,乃是八转食道强者,在人族历史上非常有名。他毕生精力。就是炼制出三皇五帝酒。根据传闻,将这八种酒合而为一,就能获得八转仙蛊酒虫。运用此蛊,能令八转白荔仙元提炼成九转黄杏仙元。此人死后。留下传承。当今东海的蛊仙醉仙翁,就是此传承的第一百三十七代传人。”

“石兄博文广志,在下佩服。请!”杨峰举起酒杯。

石磊却摆手道:“那酒杯喝太不过瘾。咱们就直接用酒坛吧。”

“好,石兄豪爽。在下奉陪到底。”杨峰大笑,一拍桌子。随即应和。

但石磊的手刚摸到酒坛上,忽然动作一滞,神色微动,转头看向门外空中。

一只信道仙蛊已经出现在庄园上空,盘旋飞舞,却下落不得。(www.MianHuaTang.cc 棉花糖小说)

石磊金瞳一闪:“此乃我门派仙蛊,还请杨老弟你解开防护。”

这个庄园,并非凡俗,结构精妙,布置恢弘,白玉地砖,寒气森森,隐有龙魂呼啸,乃是古魂门的仙蛊屋寒螭庄!

杨峰便操纵仙蛊屋,放出通路,让这只信道仙蛊飞到石磊手中。

石磊察看一番后,面色微微变幻。

别派之事,杨峰自然不好过问,举起酒杯,默默品尝赤帝酒。

没想到石磊主动开口:“北原蛊仙真是好战能斗,黑家刚刚灭亡,北原蛊仙又开始打起来了。这一次,动静更大。各个黄金家族联起手来,对付刚刚建立的百足家和楚门的联盟。”

“楚门?”杨峰疑惑不解。

“这是个最近刚建立的门派。北原楚度你知晓不?”

“略知一二,力道蛊仙,人称霸仙嘛。”

“不错,楚门便是他所创建。此人不可小视,居然凭借黑凡洞天,硬生生抵抗住百足天君的攻击。他还暗中栽培出了多位力道蛊仙,势力也不小。”

杨峰皱起眉头,消化了一下这个消息,他才道:“力道日暮西山,不足为惧。但这楚门创建,恐怕是北原第一例吧。这楚度也真是大胆!”

“不错。”石磊咧开嘴,“这一次可是要上演一场好戏。双方已经约定,进行武斗大会。嘿嘿,北原还真是热闹啊,整天都有架打,我怎么就没出生在北原呢。”

“咳、咳。”

石磊又一拍脑袋:“噢,差点忘了正事。天庭已经下令,命我们即刻攻打星象福地。”

“怎么,天庭已经确定,那方源不会回来了吗?”杨峰诧异问道。

原来,天庭方面自从影无邪逃脱出去,紫薇仙子从未放弃,一直在努力推算方源的踪迹。

方源的真身位置没有暴露,即便是凤仙太子那边也没有什么进展。但是紫薇仙子却是依靠一些些线索,结合自身的强劲实力,硬生生地算出星象福地来。

星象福地坐落在地渊之中,就在古魂门的领土范围内。

紫薇仙子特意指派蛊仙石磊来完成这件事情。

石磊却是战仙宗的太上长老,古魂门方面便派遣杨峰,配合石磊行动。

石磊接令行动,按照紫薇仙子的设计,暗中守候。若是方源哪一天回到星象福地,他就会出手捉拿。

可是等候了许久,方源的影子都没见到!

前不久,星象福地迎来灾劫,方源狠下心来,不管不顾。

紫薇仙子意识到了方源的决心,又推算出来,若是留着星象福地。会在将来对方源有不小的帮助。

所以她干脆直接动手,下令石磊去将星象福地攻陷。

灵缘斋。

湖心山。

凤金煌的住处。

“大师姐。你好厉害,连这种蛊你都能炼得出来!”秦娟双手捧着蛊虫。兴奋地叫喊起来。

凤金煌肌肤若雪,金眉修长入鬓,嘴角微翘,带着笑意道:“这也不算什么。”

她有仙蛊梦翼,炼道境界如今已有宗师级。炼制三转凡蛊,的确不算什么。

“大、大师姐,不好啦!”这时,又一位蛊仙急急忙忙地跑过来,婴儿肥的脸上有些慌乱。

秦娟立即不满地对来人道:“孙瑶。你好歹也是灵缘斋精英弟子,怎么这般没有城府。”

凤金煌微笑道:“不用慌,有我在呢。说吧,什么事情?”

孙瑶大大地喘了一口气:“那,那赵怜云过来了!”

“什么,是赵怜云?”秦娟变了脸色,她担忧地望着凤金煌,“赵怜云乃是天外之魔,又继承了盗天魔尊的真传。自从加入门派,就一直是大师姐您的对手。赵怜云一直想要和大师姐您争夺灵缘斋当代仙子的位置,她怎么忽然主动来大师姐您这里呢?”

凤金煌脸上的微笑,也渐渐消失:“此事我也不解。走,去见见她,亲自问问她。不就清楚了吗?”

三女走出房屋,便见一位女子。

此女身着一身白衣裙。如湖边静花,一头黑发挽着。好似丝绸一般。柳眉笼愁,肤如凝脂,一双眸子仿佛夜色融入其中,此刻流露出忧愁焦急之色。

如此美貌,和凤金煌不相上下,正是赵怜云!

“赵师妹,不知你来我这里,有什么事?”凤金煌主动开口问道。

赵怜云犹豫了一下,旋即眼眸中尽是坚定之色。

扑通。

下一刻,她直接跪在了地上。

秦娟、孙瑶未料到这般变化,都齐齐一声惊呼。

凤金煌也十分惊讶:“赵师妹,你这是做什么?”

“我实在是走投无路了。求求你,救救马鸿运!”赵怜云说着,泪水就顺着柔嫩光滑的脸颊滚落下来。

“马鸿运,他是你什么人?让你来主动求我?来,有什么事情你起来说,可别跪着了。”凤金煌快步走过去,伸出双手,要搀扶赵怜云起来。

她还是很有大师姐的担当和胸襟。

尽管赵怜云和她争夺仙子之位,争的很凶。但现在当赵怜云跪地上求她时,凤金煌并没有居高临下,而是主动伸出了双手。

但赵怜云没有起身,她挣脱凤金煌的双手,哭泣道:“现在能够帮我的,就只有你的,凤金煌师姐!”

……

南疆,古月山寨。

方源捏着双拳,咬紧牙关,对稳稳坐在主位上的两个人道:“舅父舅母,我父母的遗产你们二老已经得了。现在我就缺一些元石,你们二老就不能通融一下吗?”

舅母冷哼,尖声道:“这话你可说错了,你父母的遗产,我们可没拿,都是你弟得了。你弟弟啊是甲等资质的天才,可比你有前途多了。我相信就算你父母还在,恐怕也会这样做的。”

“方源啊。”舅舅随后开口,慢条斯理地道,“我知道你想干什么。不就是想要元石收购蛊材,用来炼蛊嘛。但你知道炼蛊多难吗?风险多大吗?你啊,你啊,实在是太年轻。不要抱这种不切实际的想法,好好行你的本份去。以你的丙等资质,不要再朝思暮想了。踏踏实实过日子吧。”

已是寒冬季节。

温暖的房屋里,方源心中却是一片冰冷。

他一直伫立在原地,沉默良久,才开口:“我明白了。”

他缓缓转身,往屋外走去。

身后,舅母尖酸地批评道:“这就走了?一点招呼都不打,还真是有礼貌!”

舅父则微微带笑,假意道:“方源啊,别急着走嘛,可以留下来吃顿晚饭。”

方源没有停留,他越走越快,步伐也越发坚定。

原来舅父舅母的住处,他走在人迹罕见的街头。

已是深夜。

虽然没有下雪,但冰冷的空气让方源感到一阵阵刺骨的寒意。

然后,他望向天空,暗暗握紧双拳。

天空中,群星璀璨,分外耀眼。

星光映照在方源的眸子里,他的双眸似乎也因此熠熠生辉。

“既然让我穿越到这个世界,一定是要有一番作为的。就算丙等资质又怎样,我一定可以的!”

方源没有看到,在他的身后一个身影,始终静静地尾随他,默默地看着一切的发生。

夜空的星光越发明亮,将这个身影的面目照亮。

赫然是另外一个方源。

只是这个方源面貌已经彻底改变,目光却更加沧桑。

在他的目光注视下,整个夜空,整个梦境都由浓转淡,仿佛是遮天盖地的一层浓郁烟雾,开始渐渐消散,直至消散得一干二净。

真实世界中,方源缓缓地睁开双眼,轻声自语道:“想不到……前世过去的一幕,我还能再重温。”(未完待续。)

\end{this_body}

