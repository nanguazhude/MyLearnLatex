\newsection{终现黑凡真传}    %第一百节:终现黑凡真传

\begin{this_body}

“这只上极天鹰,竟然是太古荒兽。<strong>在线阅读天火大道Http://wWw.qiushu.cc/</strong>○难怪给我感觉如此不俗!”

“若是要对付此人,恐怕要先对付这头鹰。”

“你没忘了他刚刚讲的吗?除了这头鹰之外,他还有几只上古荒兽呢。难道他是奴道蛊仙?”

“若是奴道蛊仙,就好对付多了。毕竟这个流派最怕斩首战术。就算他已经成仙,但我们也是蛊仙……”

“未必!稍安勿躁,继仙山上可见分晓。”

一直到众仙飞到继仙山的时候,都还有蛊仙在谈论着上极天鹰。

这不奇怪。

太古荒兽,成年之后就是八转战力。这带给众仙的震撼,着实有些大。

方源都将他们的话,听在耳中,表面仍旧谈笑风生,让人看不出破绽。

不知不觉间,众仙对他的猜测,已经距离真实歪曲了很多。

这点方源乐见其成,对他估算得越错误,他的优势就越大。

若真的将他当做奴道蛊仙来对付,呵呵,那就要让他们领教一下力道蛊仙、变化道蛊仙的厉害了!

相反,在交流当中,方源就不断地旁敲侧击,搜集到很多珍贵的情报。

“这些蛊仙似乎与世隔绝太久,不擅争斗。不仅是言语交锋,都没有外界那般奸猾,被我轻易试探出很多底细。而且城府甚浅,手段也局限于过去,老迈陈旧。”

“总而言之,这批蛊仙中,有两大团伙。一个是以陈尺老仙为首。四位蛊仙之间有血脉亲缘。另一个是结义金兰的三仙,分别是郑驮、冯军、周敏。居然奉一位血道为首领。剩下两个,如我之前所料。一直独来独往,离群索居。”

方源心底冷笑。

有人的地方,就有江湖。

黑凡洞天,自然也少不了纷争。

但很明显,这里的江湖很浅,纷争的层次很低,蛊仙们并不是特别擅长争斗。和北原比较起来,根本是小巫见大巫。

最关键的是,这群蛊仙的见识也很短浅。

方源已经试探出来。这些蛊仙都没有办法沟通宝黄天,甚至连宝黄天的存在都不知道。

不过……

想想他们的身份,乃是罪犯后代,黑凡老祖禁止他们接触宝黄天,也好理解了。(WWW.qiushu.CC 好看的小说

毕竟若是能沟通宝黄天,那就是完全不一样了。

不要说日新月异的手段,可以交流进来,就算是没有交易,长长见识。也是绝佳的成长。

继仙山高耸独立,并非名山大川。

它分为十层,上小下大,有很明显的人工雕琢的痕迹。山脚周围尽是茂密森林。蕴藏海量资源,倒是让方源为此刮目相看。

黑凡洞天虽然与世隔绝,但因为无灾无劫。又不和宝黄天沟通,里面的资源实在是积累得太丰富了!

众仙跟随着黄钟天灵。落在山巅处的唯一石亭之中。

石亭古朴,结构简单。虽然只是一个亭子。倒有些恢弘气象。支撑石亭的八根巨柱,粗壮无比。亭中只有一座石碑,最惹人注意。石碑巨大如象,几乎占据了整个石亭,表面刻有文字。

黄钟天灵进入石亭之中,就悬挂在石亭的梁上,那个正好有一个挂钩,然后就一动不动了。

陈尺老仙道:“上仙乍临此地,当有不少疑惑在心。且看这座石碑,应当能为上仙解惑。”

方源早就驻足,细心查看了。

半晌之后,他终于了然这一切。

原来当初,黑凡最心爱的孙女黑风月神秘失踪,态度蛊不翼而飞,

黑凡痛失爱孙女,但有心无力,寿元将近,只得落下仙窍,吞并太古青天碎片,从此失去自由。

几位黑家蛊仙,看管不利,被黑凡老祖关押到黑凡洞天之中,不仅是禁锢了他们的自由,而且还勒令他们交出仙窍,融合进黑凡洞天之中。

如此一来,使得黑凡洞天更加庞大。

黑凡吞并太古青天碎片之前,就已经留下天晶鹰巢,做了一概布置,留下他的真传。

他终究还是牵挂着最心爱的孙女,想要后人打探下落,所以画蛇添足地在最后手段上,增添了一个关键,那就是需要态度蛊的气息。

在他想来:就算他不再现身,以当时黑家的强势,必定能够尽早查出黑风月的下落,寻回态度蛊。

但实际上,却是他估算得太乐观了些。

或者说,亲情让他不愿意接受残酷的事实,从心底深处觉得黑风月健在的可能更大。

漏算了这么一着,使得黑凡真传一直都没有被继承。直到现在,反而被方源这个外人冒名顶替,开启了天晶鹰巢。

这不得不算是黑凡老祖的一个失误。

不过黑凡到底还是八转大能,临死前,也觉察到自己的这个错失。

他想到:若是本家无人继承自己的真传,那么又该如何是好呢?

若是出现这种情况,不是黑家被灭,就是衰落不堪,没有可堪造就的后辈。

黑凡当时设想:这样一来,反而不如另起炉灶,让洞天的这些人继承自己的真传,将其发扬光大。因为他们也是纯正的黑家血脉!

所以黑凡在布置了天晶鹰巢之后,又在临死前,为自己的真传设下新的规矩。

他规定了一个时间。

数百年之后,若是本家还没有人成功继承了自己的真传,那么就可以引导洞天中的血脉后辈闯荡继仙山,收获各自的仙缘。

碑文上是这样明确记载的:“(黑凡)老祖在继仙山山巅留下自己的本命真传,还设下其他种种小传承之后,寿元耗尽而亡。其余黑家罪仙痛哭悲鸣。自感有负期望,竟愿和老祖一同赴死。不过他们在临死之前。也同样效仿黑凡,在继仙山上留下各处石亭石碑。里面蕴藏着各自的真传。”

留在北原外界的黑家本家,一直着手寻找态度蛊,但始终不见起色。

时间流逝,早就过了黑凡老祖设定的时限。于是黑凡洞天这里,就开放了继仙山,让凡人们可以探索。

于是,黑凡洞天中便陆续出现了蛊仙。

不过,虽然这些凡人可以竞争黑凡真传,但是闯荡继仙山实在过于艰难。虽然有不少凡人。得到真传,一部分成为蛊仙。但始终没有出现过,攀登到山巅,取得第一真传的凡人。

按照石碑上所言:只要取得第一真传的凡人,修行成蛊仙,那么就成为黑凡洞天的主人,有权利赦免黑凡洞天中这些罪人的罪孽,让这些罪民罪仙重获自由,重归黑家。

在此之前。不论多少代人,都不能以黑为姓,只能取姓其他,以此时刻提醒他们罪民的身份。

看完所有的碑文。方源终于恍然,心中的一团尽数消除。

“难怪这些蛊仙对我都是神色复杂,情绪怪异。既有善意。又有恶意。”

方源设身处地,若换做自己。恐怕也是心里复杂至极。

按照黑凡的规定,如今已超时限。所以不管是黑家本家,还是这支罪名后裔,都可有资格竞争第一真传。

这些蛊仙一直卯足了劲,想要栽培出后代,继承了黑凡真传,打开洞天,重归自由之身。

至于这些蛊仙,已经继承了其他传承的人,是没有资格竞争黑凡真传的。

所以,黑凡洞天里的蛊仙们,往日里并无太多争斗,比较和平。

本来他们已经不指望本家的继承人了,毕竟这么多年过去,都没见任何身影。但现在方源忽然出现,打破了他们的旧有思维。按照碑文上的规矩,他们不得不迎接方源,但是心里自然是抵触的,拒绝的。

在他们看来,黑凡真传已经将是自己这一脉的囊中之物。

只要他们不断培养后辈,总会有一人能够登上山巅,将这黑凡真传继承的。

所以,他们并不愿看到方源的出现,但也不得不按照规矩行事。

天灵虽然愚昧懵懂,但黑凡死前已考虑此事,总之天灵虽然不太可靠,但却坚决贯彻执行黑凡临死前的遗志。

而且黑凡还有个制约的手段。

成就蛊仙,必有灾劫。

每每灾劫无法渡过的时候,这些蛊仙就会来到继仙山巅,向天灵救助。

天灵便调动第一石碑中的真传仙蛊,为他们延缓灾劫。虽然求得一时安稳,但却受制于天灵。

天灵只消动用真传中的某个手段,就能即刻引动蛊仙身上被延缓的灾劫,致他们于死地。

许多蛊仙,前期是自己勉强撑过灾劫,但他们不知道宝黄天,黑凡洞天的资源虽然丰厚,但种类总有限度,无法满足一位蛊仙修行的正常需求。

所以,一次次灾劫,威力更强,蛊仙们渐渐难以抗衡。

最终,他们都不得不求助天灵,一个个落入黑凡算计当中,受制于天灵。让黑凡老祖这个死去多年的大能,仍旧对这个黑凡洞天有着巨大的控制力。

“黑凡老祖,不愧是当年带领黑家走上霸主地位的枭雄!就算死了,也把这黑凡洞天统治得固若金汤,让这些蛊仙敬畏有加。”

“不过这碑文上的写法,很多都是春秋手法。嘿。我就不信当初这些个获罪的蛊仙们,都愿意和黑凡老祖一同赴死!显然,他们是不愿意也得愿意。黑凡老祖这手腕的确狠辣果断,值得一赞。若是放任他们活着,恐怕黑凡洞天早就从内部被攻破了。”

种种思绪在脑海中一闪而过,方源将注视碑文的目光收敛起来,微微仰头看向黄钟天灵。

他问道:“那么我又该如何,才能继承了这道真传呢?”

此言一出,其余蛊仙无不屏气凝神,纷纷投来极度关切的目光。(未完待续。)<!--80txt.com-ouoou-->

------------

\end{this_body}

