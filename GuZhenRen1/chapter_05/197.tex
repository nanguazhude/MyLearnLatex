\newsection{独秀、庸}    %第一百九十七节:独秀、庸

\begin{this_body}

方源审时度势,他知道,现在还远远不是“武遗海”这个身份出场的时候。

首先,他自己没有将武遗海的魂魄,彻底搜刮,他本身的准备还不充分。

其次,武独秀还没有死。

武独秀不死,方源这个武遗海的身份,就绝不能露面。

八转蛊仙!

谁知道武独秀有什么方法,来甄别“武遗海”是否是她亲生?

尽管这样一来,“武遗海”会损失很多。但方源真正所需的,只是“武遗海”这个身份而已。

这是他本来的目的。若是因为无谓的贪婪,使得他伪装失败,亲手将这个大好良机弄丢,那才是最不值得的事情,该自己打自己十个巴掌了。

最后,正道各家势力都是潜在的危险,就连武家也包含在内。

无风不起浪。

武庸的确有可能,也有动机,来对付武遗海。

方源若是提前出场,搞不好还会被武家的力量狙杀。

“说起来,这个武遗海的身份,还真是有点麻烦。不过好在,我是重生之人,知晓一些前世五百年的结果,可以辅助我进行局势的判断。”

就这样,方源继续潜修。

他一边针对武遗海展开持续不断的搜魂,另一边则关注着外在的情报。

武独秀的状态,一天不如一天,整个南疆都传得沸沸扬扬。

而在北原,血战武斗大会仍旧持续着。缺少了柳贯一,楚度一方起初被压得抬不起头来。不过很快,事态就出现了变化。由百足天君亲自邀请过来的三位七转散仙,连战连胜,一度帮助楚度一方,和正道联盟打成了均势。

刘家蛊仙刘长,四处放话,要柳贯一出来对战,否则就是缩头乌龟、孬种等等。

不只是刘长再找方源。还有其他的势力,诸如耶律家绝不会放过柳贯一。可惜柳贯一消失得无影无踪,仿佛北原蛊仙界没有这个人一样!

这些追查的势力,各自郁闷暂且不提。中洲方面许多蛊仙的下场更惨。

逆组织!

这个被影宗蛊仙创建起来的地下组织,已经被天庭蛊仙策划,十大古派出力,连根拔起,无一漏网之鱼。

除此之外。还有诛魔榜上有名的人物,也十有**被十大古派追拿归案,血道魔仙尽皆惨死,一些魔道或者散仙,考虑到罪行不重,被十大古派收纳,成为奴隶蛊仙。

大半个月之后的某天,武独秀再次在小屋中,召见了武庸还有武家的几位太上长老。

回光返照之下,武独秀口齿从未有过的清晰起来。她说道:“我平生有三大憾事。”

“第一件憾事,未能除尽凶峡七鬼,只捣毁了他们的巢穴。”

“第二件憾事,没有寻找得到足够的乾清一气,炼不成八面威风蛊。”

“第三件憾事……”

说到这里,武独秀顿了顿。

“没有见到我儿遗海。当年……我恨他的父亲,亲手杀了他,但不应该将怨念和愤怒,归结到一个纯真的孩童身上,他是无辜的。我想要弥补他。可是已经来不及了。”

小屋中,静悄悄,无人发出声响。

武独秀叹息一声,竟独自坐起身来。

她艰难地半躺着。倚靠在床头,对武庸虚弱无力地招了招手。

武庸知机,立即掀开麻布的垂帘,迈前几步,走到武独秀的床前,恭敬无比地双膝跪下。

武独秀看着自己的大儿子。良久,轻轻地笑出声来:“庸儿,你很不错,没有让娘失望。”

武庸浑身一颤,抬起头,已经是双眼通红,颗颗眼泪洒落下来。

这一刻他真情流露,眼前这位老人,毕竟是他的亲生母亲!

“我将武家交给你,很放心。”武独秀继续道。

“娘。”武庸张开口,声音沙哑,泪流满面。

武独秀继续道:“遗海那孩子既然赶不上最后一面,也是命,罢了。这些仙蛊,我都交给你。你也是修行风道,盼你能好好珍惜,发扬光大。”

“娘,儿子不要娘的仙蛊,儿子只想要娘你继续活着!”武庸哽咽抽泣。

“傻孩子,人谁能不死?就算强如仙尊魔尊者,也如巨星陨落。永生,不过是痴人的幻想罢了。”

说着话时,武独秀身上的一只只仙蛊,从她的仙窍中飞出来,落到武庸的身上。

垂帘另一边的武家太上长老们,默默地见证这一切。

双方交接得非常顺利。

仙蛊的种种气息,相继隐没在武庸的身上。

武独秀望着窗外。

阳光正明媚,透过小小的窗口,挥洒进来。

恍惚间,武独秀仿佛看到阳光中,出现了一个人影。

那个人影,不是她的父亲或者母亲,也不是她生命中,曾经热爱的某个男人,而是她自己。

那个年轻的自己,那个笑起来宛若阳光般明媚的姑娘。

武独秀。

武独秀。

“呵呵。”武独秀默念了两遍自己的名字,然后轻笑出声。

她的身上开始透出光亮,从腐朽的枯瘦如柴的身躯中,飞出无数的白色荧光。

荧光点点,是她最后肉与魂的消散。

片刻之后,床榻上空无一人,曾经威名赫赫的八转蛊仙武独秀,将自身的影响力笼罩在南疆数千年的传奇人物,终于在这一刻,彻底消失在世间。

没有留下遗体,魂魄也是一丝不存。

“大长老啊!”

垂帘外,武家的太上长老们发出了哭嚎之声。

哭声充斥在这座没有主人的小屋中。

万分的悲痛,在哭声中宣泄。

武庸默默地流泪,他跪在地上,低垂着头,像是一尊石像,一动都不动。

许久之后,他缓缓地抬起头,他已不再流泪,他的面色坚毅似铁。

他站起来身来。

然后,他透过垂帘。望见摊到在地上,痛苦流淌,毫无一丝威严的武家的太上长老们。

“诸位,还是暂忍悲痛吧。母亲已去。而我们还要将武家的光辉,继续笼罩在南疆的土地上!”武庸淡淡的语气中,却透露出一股寻常不见的威仪。

太上长老们渐渐止住哭泣。

武庸掀开了垂帘,缓步走出。

太上长老们已经站立起来,纷纷意识到了什么。

气氛陡然庄重肃穆起来。只见这些蛊仙长老们一起向武庸躬身一礼,齐声道:“我等拜见太上大长老!”

一代传奇武独秀逝世,作为她的亲生儿子,同样八转修为的武庸,成为武家的新一任太上大长老。

消息正式从武家传出后,以闪电般的速度,传遍了整个南疆蛊仙界!

南疆动荡起来。

武独秀终于去了,让不少蛊仙和势力,都松了一口气。

武家原本两位八转蛊仙,如今只剩下一位。它究竟还能不能继续待在正道魁首的位置上?

武家掌管的修行资源似乎显得有点多了。

武独秀一去,只留下平庸不堪,才具不足的武庸,他究竟能不能带领武家,保持辉煌?

不仅是武家之外的蛊仙们,心思泛起来,就像武家内部,也有不少成员,对新任的太上大长老有些怀疑。

还有关于武庸设伏,斩杀了他的同母异父的弟弟武遗海的流言。更在这一刻,被四处传播,猛烈至极,仿佛是烧开的滚水水面上。咕嘟嘟四处乱冒的气泡。

青阳山。

乔家的大本营就坐落于此。

乔家的两位蛊仙,伫立在山巅,望着武家的方向。

两仙的眼中云海重重,山风扑面,吹得衣袖翻飞。

乔家太上大长老喟叹出声道:“武独秀死了,武庸刚刚上位。就蒙上了弑弟夺宝的名声,呵呵,这可是我们乔家的机会啊。谁都认为,我们乔家就是武家的走狗,其实这不过是我们乔家的生存之道罢了。武家动荡,正是我们出手的良机。相比较其他家族,我们世代和武家联姻,十分名正言顺!”

“武庸看似平凡,但我总觉得他并不那么简单。就算他才具不足,也是一位八转蛊仙呐。”另外一位蛊仙开口道。

她二八芳龄的模样,一身翠绿衣裙,遮掩不住她窈窕的身姿。她的肌肤细嫩如雪,一双眼睛在浓密的睫毛的遮盖下,好似泛着清澈的水。

她是乔家的骄傲,崛起的天才,未来的希望,同时也是南疆的三大仙子之一乔丝柳。

乔家太上大长老笑了笑:“柳儿啊,这正是我找你来谈话的原因。我们要继续和武家联姻。”

乔丝柳娇躯一颤:“我明白了,我愿意为家族贡献自己的一份力量。”

“很好,家族没有白白地培养你。”乔家太上大长老大感欣慰。

“只是我若和武庸联姻,恐怕吃不住对方。他如今已是八转蛊仙,更执掌武家的权柄……”乔丝柳迟疑道。

“不是和他联姻。”乔家太上大长老笑道。

“那还能有谁?”乔丝柳疑惑不已。

“武遗海。”

“武遗海?”

“不错,前几日,他已经主动找上门来了。”

七天之后。

武家在风神山上,举办了一场浩大的葬礼。

各大正道势力,以及著名的散仙强者,甚至是极个别的魔道蛊仙,都被邀请,前来观礼。

在众目睽睽之下,武庸忽然发难:“左右,给我拿下这个家族叛徒!”

“什么?!”虚驼大惊失色,想要反抗,早已经来不及。

“主人,您这是干什么?!我可是你最忠心耿耿,忠心不二的奴仆啊!”虚驼叫喊得嘶声力竭。

“呵呵。”武庸冷笑,手指着虚驼,痛斥道,“好一个忠心不二的奴仆,虚驼!枉我这般信任你,你却陷害我于不义,偷偷放出我弟的路线情报,吸引无数恶贼伏杀。可惜我那弟弟,我还未见过一面,就被你们这帮狡诈的恶贼,阴险之徒害了性命啊!”

说到这里,武庸仰天长叹,泪流滚滚。

虚驼目瞪口呆地看着。

他心中的震惊,难以用言语来表述。

他还想狡辩,但旋即武庸就甩出了一大把的证据!

证据如铁般确凿。

虚驼辨无可辨,他这才明白,原来武庸早已经知道他的真实身份,只是按捺不发,一直等到了今天!

武庸利用他,铲除了弟弟武遗海,顺利地得到了全部遗产,他成为了武家的太上大长老,他还要在今天,将一切针对他的流言蜚语都统统破灭。

而虚驼就是他的牺牲出去的棋子。

“今天,我将赢得一切!”武庸心中大笑,表面上手指着虚驼,命令左右,“把他带下去!”

各方正道代表都骚动不安起来,就连那些魔道和散仙的脸上都浮现出异色。

武庸扫视一周,心胸大畅。

正要开口说话,乔家的太上大长老出列:“有一件事情要恭喜武庸大人。您的弟弟武遗海其实并没有死。他不仅没有死,还在我乔家的帮助下,来到了这里。”

“什么?”(未完待续。)

\end{this_body}

