\newsection{踏足南疆}    %第一百九十节:踏足南疆

\begin{this_body}



%1
耳畔传来呼呼的风声。

%2
下方的草原大地,随着方源的疾飞,往后迅速倒退。

%3
经过十多天的跋涉,巨大的青绿界壁,此刻已经出现在方源的前方。

%4
忽然,方源脸色一变。

%5
他感觉到身上暗渡仙蛊的威能正在迅速减弱。

%6
“又有人在推算我了!”方源心中了然。

%7
他连忙催起暗渡仙蛊,抗衡推算的力量。

%8
这种情况,方源之前就已经遇到了多次。最近这段时间,方源斩杀了耶律群星,又抢夺刘家两处资源,劫掠一空不说,还把刘家的一位六转奴道蛊仙刘勇也杀了。

%9
所以,方源在离开北原的过程中,碰到了多次针对他的推算。频率比以前高多了。

%10
好在方源有智道宗师的境界,同时又有暗渡仙蛊护身。

%11
但暗渡仙蛊有一个弊病,就是催动一次,便要休息许久,方能继续催动。

%12
方源必须考虑这一点,尽量避免想要用暗渡仙蛊的时候,暗渡却无法催动的不利情况。

%13
养、用、炼用蛊方面,博大精深,根据蛊虫不同,就有不同的注意事项。

%14
刘家。

%15
三位刘家蛊仙立于蛊阵之中,仙元持续消耗,维持着蛊阵的不断运转。

%16
片刻之后,蛊阵停止,三位蛊仙偃旗息鼓。

%17
“又失败了吗?”蛊阵外,七转光道强者刘长,拖拉着鞋拔子脸,语气很不好。

%18
“有违刘长大人所托,我等惭愧至极。”三位刘家蛊仙低头叹息,脸色也不好看。

%19
刘长冷哼一声,不满地扫视眼前三仙,但最终还是语气缓和下来:“也罢了。我们刘家没有智道蛊仙,只有这个蛊阵可以达到智道推算的效用。接下来,我就去找田下心帮忙。”

%20
“恕我直言,刘长大人。我等推算不出来,不是因为蛊阵不济,而是对方有强大的护身手段,十分克制他人的推算。”三位蛊仙中的首领。开口道。

%21
“竟然是这样?”刘长脸色更沉了一分。

%22
他不禁在口中喃喃:“对方乃是变化道蛊仙,擅长的是上古剑蛟变化,怎可能有这样的手段傍身?不过也说不准,变化道号称是一道映万道。但更有可能的是,柳贯一这魔头身边有他人帮助。霸仙楚度……他虽然放逐了柳贯一。但我却不信他真的就袖手旁观了。这些魔头散修,没有一个好东西!”

%23
说到这里,刘长就有些咬牙切齿,脸色也变得十分狰狞。

%24
柳贯一劫掠了刘家两处资源要地,还杀了刘家的蛊仙,这让刘家的面子往哪里搁?

%25
刘长受命追杀柳贯一,但现在这么多天过去了,他没有丝毫进展。

%26
之前,他为了自己的妹妹安危,没有及时追击方源。这件事情在家族中。引起了很不好的反响,现在刘长承受的压力非常的巨大,让他恨不得下一刻就能撞见方源。

%27
“柳贯一你这个缩头乌龟,到底在哪里?”刘长心中郁闷至极。

%28
他此时还不清楚,他在接下来的时间里,会越加郁闷。因为方源已经离开北原,去往他域。

%29
和刘长一样苦恼的,还有耶律家。

%30
耶律群星死在方源的手中,耶律家肯定是要报复回来的。现如今,方源已经被逐出楚门。耶律家便立即展开行动,对方源展开了调查和追杀。

%31
他们派遣出来的蛊仙,战力并不弱于刘长。但可惜的是,方源已动身离开北原了。

%32
方源顺利地穿越了甘草界壁。

%33
这个包围整个北原的神奇界壁里面。到处都是青绿的浓雾,浓雾中光影烂漫,仿佛一片片疯长的草叶,如蛇海,如发丝,不断缠绕。不断纠结。

%34
方源的至尊仙窍,却是穿梭无碍,一点阻力都没有。

%35
穿过了甘草界壁之后,方源就进入了东海的苍水界壁。

%36
等到他穿过了苍水界壁,正是来到东海的地域时,他身上的气息已经彻底发生了转变,成为了东海的一位蛊仙。

%37
这也是至尊仙体的奇妙特点之一。

%38
方源到哪里,都能和周围的环境完美融合,仿佛就是土生土长的本地人。

%39
不像是其他蛊仙,外域的气息始终很明显。渡劫时更加麻烦,必须回到自己的家乡去渡劫,否则吸取了外域的天地之气,会很糟糕。

%40
进入东海之后,方源没有着急赶路,而是来到了青玉福地,暂时休整一下。

%41
这片福地,正是蛊仙刘青玉所留。

%42
他被方源斩杀之后,鸭子地灵也被方源成功掌控。

%43
再一次见到方源,鸭子地灵十分兴奋,呱呱乱叫,上下蹦跶。

%44
方源温言安慰几句,让鸭子地灵感动得流下泪水:“呱呱呱!”

%45
“我就知道主人没有抛弃我!”

%46
遗憾的是,方源的境界不足,不能吞并这块七转仙窍形成的福地。

%47
方源在这里好生休整了一番,最主要的便是等待暗渡仙蛊恢复过来。因为这片福地自成世界,期间不管是谁来推算方源,都毫无效果,并且不会加剧暗渡效果的损耗。

%48
当暗渡仙蛊可以再度使用时,方源毫无犹豫,立即用在自己身上。

%49
补充之后,方源离开青玉福地,去往乱流海域一趟。

%50
转折往返,在乱流中耗费了他许多时日,最终他成功地进入市井当中。

%51
市井中有不少的仙窍。

%52
方源利用上极天鹰,进入其中,将一小部分的福地仙窍,都吞并了。

%53
方源又往前跨越了两次灾劫,效果并不理想。剩下的仙窍,大多是水道。可惜方源的水道境界,只是普通而已。

%54
离开这里之后,方源却没有着急启程。他仍旧缩在青玉福地之中,一段时间后,等到暗渡仙蛊快要再次恢复过来,他这才离开,继续赶路。

%55
离开北原,进入东海,来自北原的那些推算,就越加疲弱乏力。每一次推算。对于方源身上的暗渡效果削弱的程度,越来越低。

%56
方源如今也已经和之前大不相同。

%57
不仅对天意有很深刻的了解,知道如何防范,而且他还有暗渡仙蛊。许多智道手段傍身。当然最关键的,还是修为、战斗都上升了许多倍。

%58
比起义天山大战之后,方源从南疆赶回北原的那会儿,简直有云泥之别。

%59
现在,方源若是遭遇到一群上古云兽。他绝不会被追得四处乱跑,遭遇危险的不是方源,而是上古云兽了。

%60
回顾一下,方源也深深觉得:拥有至尊仙体之后,修为速度快得骇人。不愧是魔尊幽魂、影宗、僵盟等,耗费了近十万年的精力和心血,酝酿而出的超级成果!

%61
当然,至尊仙体越是优异,方源和影宗之间的仇恨就越是深厚。

%62
这种仇恨不共戴天,让方源对影无邪等人时常“牵肠挂肚”。

%63
可惜。影无邪等人滑溜得也很,方源始终没有找寻到他们的位置。否则的话,方源就算舍弃大把的仙窍福地不去吞并,也要优先斩除这些心腹大患!

%64
数十天后,方源终于连续穿过了东海的苍水界壁、南疆的瘴气界壁,正式踏足南疆。

%65
有趣的是,不管是方源,还是影无邪等人,都不知道他们已经共处一域。

%66
“早就听闻义天山大战的遗址上,已经被正道蛊仙们联合起来。建立了一个巨大的防御蛊阵。我还是先过去看看情况再说。”方源对如何进入超级梦境,也是毫无头绪。

%67
他决定还是先收集情报。

%68
一路向西南方向疾飞。

%69
飞行的路程,当然是曲折多变的。

%70
考虑到天意,方源的路线若是笔直一线。那就是直接给天意在路线上提前布局的机会。

%71
“嗯?又有人推算我了?”飞行途中,方源发觉身上暗渡仙蛊的效果,开始减弱,不禁有些好笑。

%72
北原、南疆,相隔两个地域,这让方源被推算的危险大为降低。

%73
果不其然。在北原的万豆田园之中,蛊仙田下心将一半的酬金退给了刘长。

%74
刘长的脸色相当难看,目光中更有些难以置信:“怎么?连你也不能推算出来?”

%75
智道蛊仙田下心苦笑摇头:“智道推算,也不是万能的。田某已经竭尽全力了,十分抱歉。”

%76
刘长离开万豆田园的速度很慢。

%77
他非常苦恼。

%78
更有些迷茫。

%79
若是连当今北原第一的智道蛊仙田下心,都无法算出那柳贯一的踪迹,谁还可以?

%80
“柳贯一,你最好祈祷,从今往后不要撞见我刘长!”刘长心中发狠,目光中冰寒至极,充斥杀意。

%81
不管他有再多的杀意,方源活得挺好。

%82
“唷,停止推算了。呵呵呵。等到这些人付出代价更大,苦头越尝越多,应该就能收敛一点了吧。”方源笑了笑。

%83
北原,大雪山。

%84
马鸿运望着步步逼近的万寿娘子,恶狠狠地瞪过去:“来吧。你这个毒妇!”

%85
万寿娘子冷笑一声,将雷球再次塞进马鸿运的胸膛之中。

%86
啪啪啪!

%87
马鸿运顿时浑身剧烈颤抖,被电得浑身冒烟。

%88
噗。

%89
万寿娘子口吐鲜血,脸色苍白:“怎么又是失败?”

%90
她双眼通红,失败了这么多次,之前的风姿仪态,已经渐渐消失。

%91
马鸿运强撑精神,有气无力地呻吟道:“我怎么知道?其实我也不想失败。成功吧,成功一次,让我死了吧。死了一了百了,再也不会受到这样的折磨了!”

%92
说完,马鸿运终于支撑不住,双眼往上一翻,当场昏死。

%93
ps:今天强化了一下细纲,又整理了一下bug。时间用的有些久了,第二更仍旧在10点。

\end{this_body}

