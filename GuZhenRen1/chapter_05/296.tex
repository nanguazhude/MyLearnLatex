\newsection{月亮节}    %第二百九十六节:月亮节

\begin{this_body}

几天后。(www.QiuShu.cc 求书小说网)

望着螺母山隆隆远去,而那驱山老怪就站在山巅,一副主人翁的态势,方源收回了目光,开始往回走。

山螺母一事,落下了帷幕。

武家以方源为代表,和驱山老怪达成了盟约。驱山老怪获得大部分的螺母山收益,但名义上,这头太古荒兽仍旧属于武家。

武庸下达给方源的任务,保住螺母山。

方源已经保住了,只是保住了一部分,但也算是完成了这一次的任务。

“驱山老怪能够答应我的条件,也是我依仗武家之势。”

“至于武庸,现在焦头烂额,忙着招架其他超级势力的刁难。我如此处理螺母山,他也只得勉强认可。”

“武家目前风雨如晦,我身为武庸之弟,还是不要在外面闲逛,尽早赶回去吧。”

对于这件事情的处理结果,方源心底比较满意。

他对武家有了一个交代,更关键的是,他从这次的任务中,也收获了不少好处。

盘丝洞窟已经开始建造。

因为收获了这笔资源,让方源立即有了余力,可以大规模建设第二个大型资源。

按照之前的发展速度,至少还要等待大半年。这期间还不能出现任何的意外,比如说战斗,激烈的战斗会损耗很多的红枣仙元。方源现在的至尊仙窍,时间延缓,本身产出的红枣仙元不如以前,方源就得耗费仙元石,转化成红枣仙元。

如今他已经不是青提仙元,一百块的仙元石,才能转化成一颗红枣仙元。

“建设盘丝洞窟,需要铺设蛊阵,还需要六转仙材阴柔丝,七转仙材恨水石。”

“有了这笔资金,这些都能完成,只是时间问题。”

“关键是我何时才能回归梦境?武庸的态度是关键啊!”

方源一边赶路,一边思考。

就在他往回走的路途中,武家大本营中,围绕着他武遗海的话题,正在两位蛊仙之间展开。

一位便是武家八转蛊仙,太上大长老,武庸。

另外一位则是武庸的心腹,武罚。

“没想到我弟这次,居然这么快,就解决了螺母山的事情。武罚长老,按照家族规矩,该怎样赏他?”武庸问道。

武罚沉吟了一下。

他知道,武庸问这话的意思,并不是单纯表面的言语内容,而是问他对武遗海的态度。

这就很考较武罚了。

因为武遗海的身份,比较特殊,他和武庸可是同母异父的兄弟!

武罚迅速思考,答道:“武遗海大人终究是散修出身。<strong>热门小说网WWW.QiuShu.Cc</strong>”

就一句话。

但武庸却哈哈大笑起来:“不错,你评价的不错。这个遗海啊,到底是散修性子,就算回归了本家,也没有改正过来,到处都想捞一笔,贪小便宜。”

武罚却道:“广寒峰、螺母山,这两次可不是什么小便宜啊。”

武庸笑声停息下来,眼中寒芒一闪,点点头道:“的确不好,若是人人如此,武家还怎么长存下去?不过他手法倒也妥当,没有落下什么把柄,让人攻讦。”

说到这里,武庸叹了一口气,评价道:“他能力还是有的。”

武罚听到这里,哪里还不明白武庸的心思。

武庸当然很看不惯方源这样子捞好处。

但是情势逼人,他手头上人员紧缺,有时候只得调动武遗海来处理一些事情。

方源也在这两次事件中,表露出了自己的能力。

这种能力得到了武庸的认可,所以接下来,武庸还会用方源。但如果时机一到,武家的局势缓解下来,方源就要被武庸放在一边,甚至若是方源落下了什么把柄,武庸还会抓住,敲打方源一把。

毕竟方源这样谋求私利,任何上位者都会心里不舒服的。

方源顺利地回到了武家。

他和武庸见面,简单述说了整个事情的经过,并且再一次得到了武家的奖励。

方源都换成了仙元石。

当然这笔仙元石,和他从驱山老怪手头中的收获,是不能比的,远远少得多。

接下来的日子里,方源潜心修行。

盘丝洞窟的建设,徐徐开展。这是目前的修行重点。

每隔一段时间,方源都要落下仙窍,吞吸天地二气,稳定至尊福地。

毕竟逆流河乃是天地秘境,仙窍承载的话,负担不轻。

这个方面有点麻烦。

方源不能直接吞吸天地二气,因为他吞吸天地二气,每一次的量都很大,相互间隔的时间也比较短。

这让方源不得不从宝黄天中,搜罗一些仙材,然后分解成天地二气,灌注自身。

这无疑消耗了方源不少精力和财力,但方源宁愿如此,也要把细节做好。只有这样下去,才会坚持够久,不会在短时间内露出破绽,让人识破他的身份。

日子一天天过去,方源开始有些不耐烦。

“超级梦境那边,波折已经渐渐平息。但是武庸却似乎没有一点,想要将我调回去的迹象。看来我担心的事情,还是发生了。唉!”

方源叹息。

人在江湖,身不由己。

武庸现在忙于应付四面八法的刁难,一力求稳。若是将方源调回去,结果超级蛊阵那边,出现了问题,不是自找麻烦吗?

而且现在,武庸发现方源还挺有能力,也想继续让他来处理一些麻烦。

至于方源在其中的贪污**,武庸此时此刻就睁一只眼闭只眼,不追究了。

他不追究,方源却想。

方源之所以伪装成武遗海,加入武家,不就是图的梦境吗?

“看来,是需要行动了。”

这一天,方源主动离开了自己的居所。

上有政策,下有对策。

方源老谋深算,怎可能没有应付的方法?

十几日后,在月华山上的一座凉亭中,几位蛊仙围绕着一面石桌团坐。

月光如水,温柔洒下。

青山葱茏,夜鸟鸣啼。

轻风拂面,美不胜收。

“今天是我南疆一年一度的月亮节,今夜能有幸与诸位共同赏月,可谓一场雅事。”乔丝柳微笑着道。

她的声音很美,如山泉潺潺,给人清澈纯粹之感。

她的人更美,今夜一身裙裳,洁白若雪,再结合她娇美的容颜,端的是人间仙子。

“能得到丝柳仙子的邀请,一同赏月,这是在下罗木子的荣幸。”一位青年模样的蛊仙,首先开口道。

他的笑容热情洋溢,尤其是看向乔丝柳的目光,透着一股炙热。

乔丝柳乃是南疆蛊仙界中,公认的三大美人之一,更瞩目的是她的家世背景。

乔家,虽然依附武家,但同样也是正道的超级势力。

在场的六位蛊仙,两女四男,除了其中一对道侣之外,罗木子和轮飞都是乔丝柳的追求者。

乔丝柳的追求者众多,但是能被她本人邀请过来赏月,这俩人自然不一般。

罗木子、轮飞自然心中欢喜,一得到乔丝柳的邀请,就连忙赶来。

但这时乔丝柳却对另外一位男性蛊仙,道:“遗海,你说想更快地融入南疆,今晚赏月,正是我南疆的风俗。”

“我南疆虽然多山,山寨林立,相互隔绝,但月亮节却是共同的节日。每年到这个时候,我们通常就会赏月。”

她说这话时,眼泛波光,柔声细语,红唇带笑,非同一般。

罗木子、轮飞顿时皱起眉头,死死地盯着方源,目光好似喷火。

乔丝柳不一样的态度,明眼人都能感受得出来,而且今天的座位安排,也早就让罗木子、轮飞异常不爽。

因为乔丝柳就坐着主位,右手边就是方源,左手边则是她的闺蜜好友天露仙子。

至于其他几位男性蛊仙,都被隔在远处。

“月亮节?”方源神情淡淡,端坐在主位上,明知故问道,“有点意思,但除了赏月,我们还做什么?”

“饮茶、吟诗还有解石。”乔丝柳笑着解释。

“解石?”方源作不解状。

乔丝柳没有回答,而是端起一杯香茶,递给方源:“先请遗海你尝一尝我亲手做的茶。”

看到这一幕,罗木子差点要站起身来!

轮飞眼角乱跳,恨不得自己取代了方源的位置。

方源端过来轻抿一口,评价道:“这茶好像不错。”

“岂止是不错,这可是丝柳仙子名传南疆的柳旋茶!”罗木子几乎叫起来。

“这其中有什么讲究?”方源看向乔丝柳。

乔丝柳先是和方源对视一眼,旋即眉目一转,笑声透着一股柔媚。

轮飞看得仙子娇笑,一颗心顿时砰砰乱跳。

这时,天露仙子非常适宜地插嘴道:“这种柳旋茶,只要轻轻晃动,茶水表面就会形成柳叶一般的漩涡。此时再喝,才算真正品茶,味道绝不一般。”

“哦?是这样?”方源按照指点,晃了晃手中的杯盏,顿时茶水表面就起了变化。

他再一喝,只觉得清香四溢,遗留满齿,叫人回味和享受。

“好茶。”方源脱口称赞。

“当然是好茶,丝柳仙子亲手酿的茶,可不是一般人能够喝到的!”罗木子酸溜溜地道。

“能得到遗海的称赞,也不枉费我家丝柳的一番努力了。你可知道,要做这份茶,得费三个时辰,且整个过程中不得有一丝一毫的疏忽。”天露仙子道。

轮飞感动万分:“今夜能品尝到仙子亲手做的柳旋茶,实乃在下三生之幸!”

天露仙子顿时扬起眉头:“谁说给你能喝到柳旋茶?丝柳可是百忙当中抽出空闲,做出来了一份。”

“呃。”(未完待续。)<!--80txt.com-ouoou-->

\end{this_body}

