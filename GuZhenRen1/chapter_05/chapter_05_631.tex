\newsection{功德方尖碑}    %第六百三十四节:功德方尖碑

\begin{this_body}

%1
轰!

%2
一记磨盘大的雷球,狠狠地在庙明神的后背轰炸开来。

%3
噗。

%4
庙明神浑身剧震,大吐一口鲜血,苍白的脸色更加病态。他连忙强振精神,催动仙道杀招,下一刻他消失在原地,直接出现在了前方数千步之外。

%5
但即便是在这里,也是雷电交加,猛兽飞舞,情势已经到了最坏的地步。

%6
“没想到我庙明神居然要死在这里了!”庙明神满脸绝望之色。

%7
乐土真传不是温和的吗?怎么会如此凶险?

%8
这些疑问此刻已经都不再重要,庙明神仙窍中的仙元已经干涸,他的仙蛊虽然大多都在,但他已是强弩之末,刚刚逃出来的那一记,是他最后的挣扎。

%9
兽吼声几乎要刺破他的耳膜,太古雷凰再次向他扑来。

%10
环顾四周,来的八位蛊仙只剩下庙明神一人了。他是坚持最久的人,其余人都陆续阵亡。

%11
“早知如此,我就万万不该来探索乐土真传。苍蓝龙鲸都没有见到一面,我就要死了。呵呵呵,任修平事后得知可真的要笑掉大牙。”庙明神心中十分后悔,但也充满了无奈。

%12
人生在世,前面的那条路是黑的,谁能算得尽?就算是智道仙尊星宿也能算得尽吗?

%13
“来吧。”庙明神叹息一身,挺起胸膛,坦然面对雷凰的利爪。

%14
一阵剧痛,旋即是最深邃的黑暗。

%15
“我已经死了么……”庙明神恍恍惚惚。

%16
这个时候,忽然有一股声音传来:“庙明神大人,醒来,快快醒来罢。”

%17
庙明神便睁开双眼。

%18
随后,他就看到花蝶女仙关切的神色,就站在他的面前,还有鬼七爷、蜂将、土头驮等人,就连最早阵亡的楚瀛都在。

%19
他们都嘴角含笑。

%20
“你们……这是怎么回事?”庙明神忽然眼中一亮,“等等,这是一种考验吗?”

%21
“庙仙友明见,我们讨论的结果也是如此。”土头驮哈哈一笑,回答道。

%22
“没想到还未见到苍蓝龙鲸,这第一重考验已经来了,但这考验未免太过凶险了。”庙明神摇头不止,心中仍有余悸,他一边说着,一边视察自身的伤势。

%23
下一刻,他便彻底愣住。

%24
原来他发现,他自己身上的伤势已经全都不翼而飞,蛊虫的损伤也全然消失,甚至是自家干涸的仙元储备都恢复如初。

%25
“这?!”庙明神猛地抬头,“难道刚刚的一切都是幻觉不成?”

%26
“正是如此啊。”

%27
“乐土仙尊的手笔真是厉害,我们身在局中一直都毫无所觉,直到最后阵亡才醒悟过来。”

%28
“仔细想想,乐土仙尊能够开创出轮回战场,设计出这一层考验对他而言,根本不是难事。”

%29
“历史记载中,乐土仙尊为了折服当时的八转强者战魔,便创造出这一类的仙道杀招。轮回战场便是此招的巅峰之作,这么多年来,轮回战场不知为中洲十大古派培养了多少蛊仙强者!”

%30
庙明神喘息几口气,头脑越来越清醒:“我想这也是乐土仙尊的劝导吧,毕竟一旦经历过死亡,什么事情都能看得开了。”

%31
“大人说得是,死亡那一刻的感觉令人刻骨铭心。”蜂将感慨道。

%32
“同时,这也能检验出一个人的战力层次。庙明神仙友,我不得不佩服你,你在幻境中的表现我们可都是看得一清二楚,你是我们当中坚持最久的人了。”童画笑着道。

%33
庙明神顿时又喜又忧。

%34
喜的是,他此番表现大大增强了他在队伍中的威信,忧的却是自家绝大多数的手段都已经彻底暴露了。毕竟生死存亡之际,谁还能有所保留?

%35
庙明神忽然想到方源,目光一转:“楚瀛仙友,亏得我还为了伤心悲愤,没想到你却是我们当中最省力的一个人呐。”

%36
他说的话自有深意。

%37
方源长叹一声,露出惭愧的苦笑:“说实在话,死亡来临的那一刻,我的感觉极不好受,心中痛骂我为什么这么倒霉,偏偏被太古雷凰盯上?结果醒来的时候,我也是懵了好久才缓过劲。”

%38
有着态度仙蛊,他的表演自然情真意切,庙明神深深盯着方源,见着他的真切神态,心中刚刚升腾起来的怀疑顿时大半消散。

%39
“那么这里又是何处?”庙明神扫视周围,只见他们都在一处无人的小岛上。小岛周围是浩渺的碧蓝海水,一望无垠。

%40
海水里可见一阵阵的鱼群,海面上浪花朵朵,海风徐徐,白色的海鸟欧欧鸣叫,有的高处盘旋,有的则紧贴着海面。

%41
“我们正要问你呢。醒来的时候,就在这里了,也不知接下来该前往哪个方向。”童画道。

%42
庙明神点点头:“诸位请稍等片刻。”

%43
说着,他便缓缓闭上双眼,催动起一记神秘的仙道杀招。

%44
这显然就是他定位苍蓝龙鲸的独到手段!

%45
但是光凭表面,根本难以体悟出这种手段的内涵。

%46
庙明神闭上双眼后,又很快睁开,他的双眸中流露出一抹震惊和喜悦。

%47
“诸位,若我所料不差,这里就应当是苍蓝龙鲸的仙窍世界了!”庙明神语出惊人。

%48
“什么?”众仙惊异不已。

%49
关于这道乐土真传的情况,他们多少都知道一些。

%50
乐土仙尊当然收服苍蓝龙鲸,并为它开启出仙窍洞天。在这个洞天当中,他留下了一道乐土真传。

%51
无数年来,许多蛊仙都试图寻找出苍蓝龙鲸,但最终什么都没有找到。

%52
此行蛊仙们还想打算一睹苍蓝龙鲸的风采,没想到竟稀里糊涂地来到了它的仙窍洞天当中。

%53
“乐土仙尊的布局,果然不是我们这些人能够猜得透的。”

%54
“这么说来,其实进入这里并没有什么难度?”

%55
“那在我们之前,为什么就没有人来到这里呢?”

%56
“是苍蓝龙鲸太过难寻了么……”

%57
“这么一来,我们接下来该怎么行动呢?”童画问道。

%58
群仙的目光又都集中在庙明神的身上。

%59
庙明神对这样的情况心中十分满意,蛊仙们来到了目的地,都没有分散开来,而是仍旧想要团结在他的身边,不枉费他之前掏空心思来稳固自身地位了。

%60
“说实话,我也不太清楚。不过我们既然是进来了这处洞天世界,几乎是成功了一小半。接下来就是好好找寻隐藏在这里的乐土真传。这里到底是陌生之地,也不知道无数年来,这里蕴藏生活着什么。所以稳妥起见,我们还是结伴而行吧,就算出了意外,也能够相互照应一些。”庙明神一番话令群仙都点头赞同。

%61
方源这时开口:“我有一个拙见,既然我们进来此地,乃是乐土仙尊的布置。那显然我们身处的这座小岛也大不简单,不妨先将这座小岛搜寻彻底一些。”

%62
“你们没有先去探索一番吗?”庙明神问道。

%63
群仙纷纷摇头,又有一些人看向方源。

%64
方源苦笑:“我虽是第一个进来,但明白了外界乃是幻觉之后,也唯恐这里的危险。所以一直守候在这里呢。”

%65
“好吧,那我们先就分散开来,搜寻这座小岛。”庙明神随后便逐一分派搜索的任务。

%66
这座小岛本身就不大,方源一方的人又多,很快就有人发现了线索,叫道:“快来小岛中央,这里竟然有一座八转层次的仙蛊屋!”

%67
片刻之后,八位蛊仙都站在了仙蛊屋的面前。

%68
这座仙蛊屋造型奇特,从外表看去,如同一座黄金制作的方尖碑。

%69
方尖碑的表面,不断地浮现住许多文字。

%70
蜂将刚看了几眼内容,便惊呼出声:“原来这里就是乐土真传!”

%71
另一旁的花蝶女仙则皱眉道:“但要继承这份真传,可真的麻烦多了。”

%72
按照方尖碑上的记载,这座仙蛊屋乃是八转层次的功德榜,照应整个仙窍洞天世界。外来的蛊仙要获得乐土真传,就必须接取方尖碑上的各种任务,完成之后获得功德。

%73
功德越多,能够兑换的东西便越有价值。

%74
若是不接受这个条约,蛊仙们就只能枯坐在这座小岛上,不能真正外出,去往洞天世界的其他地方。

%75
“按照碑文所说,我们停留在这里的时间只有三百天。”童画用遗憾的语气道。

%76
三百天的时间,自然是按照龙鲸洞天的光阴流速算。

%77
“三百天的时间,能完成什么样的任务?”曾落子眼中闪过一抹精芒,碑文上的任务有限,只有十个,而且每一条都不一样,这就意味着蛊仙之间会加深更多的竞争。

%78
其他人也不笨,很快也意识到了这个问题,一时间气氛渐渐沉重起来。

%79
庙明神敏锐地察觉到了这种氛围的变化,他开口道:“诸位,虽然碑文有着明确记载,但此地我们都是首次到来,不排除其他可能。依我之见,我们应当先试验一番,看看能否出得了这座小岛。同时也请诸位一齐研究这座仙蛊屋,若是能有破解之法,将其收入囊中,也是绝妙之事啊!”

%80
群仙怦然心动,这可是八转的仙蛊屋,若能破解,相互刮分,获得必然极为丰厚。

%81
庙明神又道:“我们搜寻的时候,也发现了这座小岛上也有许多修行资源,储量很大。我们其他的先不论,首先将这些分一下,诸位以为呢?”

%82
群仙有好处可拿,哪有什么不愿意的,当即齐齐响应。

\end{this_body}

