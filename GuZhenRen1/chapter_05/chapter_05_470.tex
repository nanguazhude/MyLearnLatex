\newsection{盗天梦境}    %第四百七十一节:盗天梦境

\begin{this_body}

%1
方源再一次见到了唐方明。

%2
这个人不简单,方源五百年前世,他在五域乱战中大放光彩,堪称是西漠的马鸿运。

%3
当然,现在方源继承了影宗传承后,知道了更多。

%4
他的五百年前世,幽魂魔尊成功重生,掌握至尊仙胎蛊,潜伏在天庭中,暗中支持五大域中的各方势力,布局天下。

%5
马鸿运的崛起,就有影宗在幕后操纵。

%6
唐方明乃至唐家的强盛,也是影宗在暗地里推波助澜。

%7
“别看现在,天庭人手稀缺,有些捉襟见肘的样子。一旦等到梦境四起,五域乱战时,各种各样的人物就会在天庭中苏醒,一次次惊动整个天下。”

%8
天庭的底蕴太雄厚了,就算是影宗,也难以撄其锋芒。

%9
方源越来越理解,影宗为何要这样应对。

%10
他现在也要继续影宗走的路,马鸿运被他杀了,但唐家还在,他要支持唐家。

%11
这个决定,方源早就定下来。当初他和唐家接触、合作,并非只是要借助一道光阴支流,脱离光阴长河,而是真的心怀合作诚意。

%12
天庭现在积极布局,为了应付接下来的时代剧变。

%13
尤婵、秦百合,就是紫薇仙子想要染指、操纵成棋子,所以方源肯定要铲除掉这两个人物。

%14
不仅是龙鱼生意的眼前利益,而且还兼顾未来的局势。

%15
要和天庭这样的庞然大物对抗,方源的格局首先就要上去,要俯瞰五域,布局天下!

%16
唐方明对于方源的到来,是欣喜若狂。

%17
方源上一次离开西漠,是逃离性质,唐方明未曾想到方源居然这么快,就又回来了。

%18
“带我去盗天梦境吧。”方源开门见山,直指目标。

%19
“方源大人雷厉风行,在下钦佩,请随我来。”唐方明当即领路。

%20
两人飞了一段路程,来到一处毫不出奇的沙漠上空。

%21
唐方明缓缓停下,伸手一指,顿时空中裂开一道口子,仙气四下洋溢。

%22
方源乃是阵道宗师,底蕴深厚,此刻见到这个仙阵也不禁双眼一亮,交口称赞道:“好一道仙级蛊阵!”

%23
这仙阵隐藏功夫相当了得,哪怕是方源时刻催动着侦查手段,也没有发现丝毫的端倪。

%24
“方源大人,这边请。”唐方明率先钻入裂口。

%25
方源毫不犹豫,也紧随其后。

%26
进入内里,方源发现这道仙阵营造出来的空间,仿佛是一片原野。

%27
视野非常开口,原野上青草漫布,而最为吸引方源目光的,则是原野的最中央那片盗天梦境!

%28
盗天梦境体积非常庞大,宛若一座山峦。

%29
它散发着蓝色的幽光,映照着整片阵道空间。

%30
五域乱战还未来临,这片盗天梦境却是提前显现,被唐家意外发现后,秘密地隐藏起来,一直在暗中研究。

%31
唐家虽然获得这片绝世珍宝,但是却苦于梦道一穷二白,开创艰难,一直进展很小。

%32
而这些进展,基本上还都是依靠唐方明的才情和天赋。

%33
即便如此,唐方明也险些折损在这片梦境当中,若非是他妹妹唐妙舍弃一道乐土真传线索,求得白海沙陀相助,唐方明早就死了。

%34
而白海沙陀的真实身份,便是幽魂魔尊的分魂之一。在义天山大战时,他带领西漠的影宗成员参战,最终都战死沙场。

%35
白海沙陀的这份因缘,方源并没有向唐家提及,唐方明、唐妙掌握的乐土真传线索,也辗转到了方源的手中。

%36
乐土真传不是方源眼下的重点,现在方源的目光聚焦在庞大如山的盗天梦境上。

%37
这片闪耀着瑰丽蓝光的梦境,非常危险!

%38
因为这里面蕴藏着天意。

%39
天意能够侵蚀梦境,这点方源在南疆梦境一役中,算是吃够了苦头。

%40
当初,他在繁星洞天碎片世界的梦境中,更是直接遭遇到了天意。天意以星宿仙尊的形象来提点他,使得他在不久之后的义天山大战中,险胜魔尊幽魂。

%41
俗话说,吃一堑长一智,方源这样的老魔,更是将这般血的教训,死死记在心头。

%42
“梦境一旦外显太久,就会被天意侵蚀。南疆梦境如此,西漠这片盗天梦境,外显的时间更长,肯定被天意完全侵蚀了。”方源双眼闪烁着阵阵幽芒。

%43
“若换做以前,我必定退避三舍,但是现在……呵呵。”方源嘴角微微翘起,然后他缓缓伸出手掌,对准眼前的这片盗天梦境。

%44
仙气盎然,方源浑身上下都荡漾出一层灰白色的光晕。

%45
他的一双手掌也不例外,被灰白光晕全面笼罩。

%46
方源向前迈步,让双手慢慢接近梦境,最终灰白光晕接触到了盗天梦境,而方源的双手却还是和梦境有着一层微小的距离,没有实质性的接触。

%47
方源是至尊仙体,不是纯梦求真体。一旦有了实质接触,魂魄就会陷入梦境当中去。所以他非常的小心谨慎。

%48
而那层灰白光晕,则好像是流水一般,悄无声息地渗透进幽蓝的盗天梦境中去。

%49
灰白光晕不断渗透,以方源的双手为中心,一股灰白之光开始向四面八方渲染开去。

%50
这种现象持续了片刻,就好像是幽蓝的梦境之山的山脚处,染上了一层灰白的光斑。

%51
光斑的面积还在不断地扩大,唐方明在一旁静静地看着。

%52
他当然看不出什么来,但却认得出这是一记仙道杀招。

%53
“方源大人果然厉害,居然已经开发出了针对梦境的仙道杀招!相比较他而言,我不过只是创出几只梦道凡蛊,真是天外有天,人外有人!”唐方明心生钦佩,同时更觉得方源深不可测。

%54
方源面无表情,心中其实有些惴惴不安。

%55
他此时催动的仙道杀招,正是天消意散!虽然不是第一次催动,但他是第一次对梦境催动。

%56
天消意散杀招,以天机仙蛊为核心,能够消除天意。之前方源对春秋蝉使用过,现在对梦境施展出来,就不知道成效会如何。

%57
时间缓缓流淌,一盏茶的功夫就这样过去了。

%58
灰白光斑已经扩张到相当巨大的程度,不仅是表面,更是渗透到盗天梦境的里面。

%59
这片盗天梦境大若山峦,但足足有十分之一的体积,已经被灰白光晕笼罩了。

%60
就在唐方明以为,灰白光斑还要逐步扩大的时候,光斑的边缘却停止了蔓延,与此同时,一只只色彩斑斓的眼睛,出现在灰白的光斑之中。

%61
这些眼珠子,非常奇特,一眨一眨,每一次眨眼,都变化一种色彩。

%62
斑斓眼球的数量,迅速暴涨,几个呼吸之后,就从数十个暴涨到了数百,很快,就有了数千颗。

%63
几十个呼吸之后,斑斓眼球数以百万,充斥了灰白光晕,不仅是在光斑的表面,还有光晕内部都有眼球在闪烁。

%64
“有效果!”方源心中大喜,面色上却不动,让唐方明丝毫察觉不到他的内心波动。

%65
斑斓眼珠不断闪烁,方源的红枣仙元则以一种恐怖的速度剧烈消耗,盗天梦境中的天意也随之被消灭摧毁。

%66
斑斓眼珠子往往眨动几下后,就怦然消失,但与此同时,又有新的眼珠产生。

%67
天意被迅速毁灭,片刻之后,灰白光晕中已没有一丝天意存在。

%68
这个时候,灰白光晕开始继续扩张,斑斓眼球迅速移动,在后面掩杀。

%69
方源这一次催动天消意散杀招,足足持续了三天三夜。

%70
最终这片盗天梦境里,再无任何的天意存在了。

%71
方源又继续催动杀招,让盗天梦境的表面始终笼罩着一层淡淡的灰白光晕。这样一来,只要天意继续侵蚀,灰白光晕中就会出现斑斓眼球,将天意消灭。

%72
至此,盗天梦境安全了!

%73
“接下来,就是探索这片梦境了。”方源吐出一口浊气,心中荡漾着喜悦之情。

\end{this_body}

