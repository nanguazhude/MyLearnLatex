\newsection{尊者才情}    %第六百六十六节:尊者才情

\begin{this_body}



%1
喳喳喳喳!

%2
一大群的眨眼鸟叽叽喳喳,围绕着一朵七色云彩,不断飞舞盘旋,向来犯的敌人警告。

%3
方源面色平静,但目光中却透露出一丝笑意。

%4
眨眼鸟的警告对于他而言,当然不算什么。

%5
这种鸟虽然乃是荒兽,但战力很低,就算是有这样一大群都不足为患。它们真正的厉害之处在于度。

%6
眨眼鸟几乎算是度最快的荒兽了,一旦它们的度爆出来,一眨眼之间就能脱离蛊仙的视野,所以取名就叫眨眼鸟。

%7
方源一身气息都收敛起来,宛若凡人一般。但这些眨眼鸟根本不攻击,看着人畜无害的方源,它们本能地感到一种心灵上的压迫。

%8
忽然,方源身上有一丝气息泄露出去。

%9
哗!

%10
下一刻,眨眼鸟的度猛地迸出来,刹那间消失无踪。原本密集的鸟群围绕着七彩云彩,如今却只剩下这片云彩在静静地漂浮了。

%11
方源微微叹气,他原本打算活捉这些眨眼鸟。但这些眨眼鸟的感应太过灵敏了,方源也并不太擅长隐匿杀招的气息。

%12
不过走了也就走了,眨眼鸟即便收服进来,也不好豢养。它的食物是太古白天中特有的白天飞尘,这种东西很不容易收集,蛊仙仙窍也万万不能产出。

%13
方源缓缓飞近七彩云朵。

%14
“没想到居然碰见了一朵七彩云泥。”方源感叹一声。

%15
这看似云朵,其实是一种泥土云泥。

%16
琅琊福地中,琅琊地灵就将大量的云泥凝聚一体,形成了云盖大6,始终漂浮在琅琊福地的高空之中。

%17
云泥最大的产地便是太古九天,当然五域中也有极少产出的资源点。

%18
单色云泥乃是最普通的蛊材,比如太古黄天中的黄色云泥,太古赤天中的赤色云泥。如今市面上最多的单色云泥是便是白色云泥,绝大多数都产出于太古白天。

%19
太古黑天中其实黑色云泥也不少,但更危险,毕竟黑天中有许多的魂兽。

%20
多色云泥一旦多达六彩,那么就是六转级别的仙材了。方源眼前的七彩云泥便是七转层次的仙材,这样一大片的七彩云泥,价值很高。

%21
将这一大片的七彩云泥检查一番后,方源又现了在这云泥当中,还有许多的鸟巢。鸟巢里头藏着许多鸟蛋。

%22
这都是眨眼鸟的鸟蛋!

%23
方源并无意外,从刚刚那些眨眼鸟的表现来看,就好像是守护着什么。其实,眨眼鸟的性情非常胆小,哪怕它是荒兽,一旦有什么风吹草动,就会立即爆出最大的度飞走。

%24
有了这些鸟蛋,方源就可以孵化出眨眼鸟了。孵化荒兽当然不容易,但对于方源而言,并无多少难度。

%25
不过方源却不太想孵化,而是想要将这些鸟蛋当做一种仙材,在炼蛊的时候用。

%26
眨眼鸟的食物太单一,不容易豢养。要是培养,就得时不时的放养到太古白天中,这就太麻烦了。

%27
方源财大气粗,已经看不上眨眼鸟这样的盈利了。

%28
自从天相杀招被炼化后,这一个多月来,方源就在太古白天中活动。

%29
他一边搜索太古白天中的仙材,一边在自家的至尊仙窍中炼蛊。

%30
外界的一个多月,至尊仙窍中的时间就更加漫长了。

%31
方源炼蛊得到了巨大的成果!

%32
人如故、江山如故、我力、拔山、挽澜、狗屎运、时运等等,都已经被他提升到了七转程度。

%33
他稍稍改变了一下原先的炼蛊计划,将其中的运道仙蛊提前升炼。这些天来,他在太古白天中同样收获巨大,得到许多的仙材。这其中就有这些七转运道仙蛊的功劳了。

%34
除了升炼之外,方源还改良了炼蛊手法。

%35
他知道太多的蛊方,也通晓海量的炼蛊方法,他的炼道境界更是准无上,但他的炼道经验还是比较匮乏短缺的。

%36
尤其是炼制仙蛊的经验,一点都不多。

%37
实践出真知。

%38
这段时间来,他接连不断地升炼仙蛊,从中现自己的许多不足,尽数弥补。同时还现更多的适合自己炼蛊的方法方式。

%39
如今,对于炼蛊这块,方源有了许多全新的想法和灵感。升炼仙蛊的成功率,也提升了一大把。

%40
尽管五域大地上,地脉仍旧震荡不休,蛊仙们收获仙材、仙蛊是欢天喜地,如火如荼。但方源深思熟虑之后,仍旧选择在太古白天中探索。

%41
他若是和五域蛊仙抢夺地沟中的资源,势必就要现身,极可能生战斗。

%42
每一只仙蛊升炼成功,都令方源的实力上涨一些。但上涨的全部程度,都难以抵消他上一次在光阴长河中底牌暴露的劣势。

%43
他需要时间,静静地消化这一时期的收获。

%44
一旦争斗起来,刚刚积累的八转白荔仙元,也会剧烈消耗。

%45
和蛊仙争斗,远比和猛兽争斗要凶险得多。

%46
虽然方源在太古白天中探索,也需要消耗仙元,也有不少的战斗。但他依赖天相杀招,侦查得非常到位。

%47
太过棘手或者强大的猛兽,他就不去招惹,提前避退。

%48
天相杀招在太古白天中的侦查范围,实在是太广了,更能轻而易举地现许多擅长隐匿的猛兽凶植。

%49
天相杀招强到连隐藏起来的仙窍洞天都能现,现这些猛兽凶植也是应有之理。

%50
更妙的是,天相杀招侦查并不需要方源一滴一丝的仙元。这记九转杀招的优异和强大,毋庸置疑!方源在使用的过程中,已经不只一次的称赞过了。

%51
不只是风险、消耗上,探索五域要高过探索太古白天,就连收益方面,前者也不及后者。

%52
蛊材、仙材在地沟中层出不穷,但这些东西,都需要蛊仙一一安排妥当,才能纳入仙窍去。

%53
经营仙窍可不是那么容易的,很多资源都有相互关联,相互干扰的关系。

%54
所以,大多数的五域蛊仙从地沟中得到好处后,都将这些仙材卖到宝黄天中去。

%55
他们的仙窍空间有限,万万不能和方源的至尊仙窍相比。很多时候,得到手的仙材也不适合自己仙窍的环境。

%56
地脉震荡不休的另一个现象,就是宝黄天的生意,或者更准确的说,蛊仙之间的交易次数、数额连续暴涨!宝黄天中的热闹程度,是以前的数十倍。

%57
方源若是想要什么仙材,完全可以从宝黄天中收购。

%58
而仙窍洞天就不同了,每一个洞天几乎都已经建立了完整和谐的体系。方源一旦吞并,非常省事,根本不需要思考什么。还能增加至尊仙窍的面积空间,修为、道痕都能暴涨一大截,灾劫更是可以直接跨域了。

%59
吞并一个仙窍洞天的好处,太多太多了。

%60
至于地沟中时不时冒出的野生仙蛊,方源当然也羡慕,只是这些野生仙蛊向来就都是蛊仙们争夺的重中之重。

%61
方源觉得自己,反倒不如在仙窍洞天中直接盗取他人的仙蛊,反正他有大盗鬼手这个手段。

%62
天庭。

%63
大阵徐徐停歇下来,紫薇仙子缓缓张开双眼,眼中精芒一闪即逝。

%64
她自语道:“魂穿杀招?果然是精妙至极!”

%65
离开之前,她又看了魔尊幽魂一眼,却没有说什么劝降的话。

%66
这些天来,紫薇仙子经常来搜刮魔尊幽魂的记忆,来强夺他一生的修行成果。

%67
紫薇仙子不得不佩服魔尊幽魂的惊艳才情,也不得不承认在魂道方面,哪怕天庭已经做出了巨大努力,但仍旧比不上魔尊幽魂。

%68
“可惜如今的魔尊幽魂,却沦为一个阶下囚。可悲可叹,在历代的尊者中他是最落魄的。其实若是他一死了之,还能在青史留名。如今却是求生不能求死不得……”

%69
想到魔尊幽魂仍旧顽抗,紫薇仙子又恨得牙痒痒。

%70
她确信魔尊幽魂身上,还有一个相当重要的梦道成果,那就是引魂入梦。

%71
虽然此招,已经被天庭破解了大半,但若得到手,对于天庭而言仍旧极有价值。

%72
可惜,魔尊幽魂一直抵抗,虽然扛不住整个天庭洞天的力量,但是魔尊幽魂非常狡猾,将一些次要的东西都抛出去,借此拖延时间。

%73
第一次紫薇仙子利用天庭的力量,的确打了魔尊幽魂一个猝不及防,使得纯梦求真体的成果被抢夺走了。

%74
不过缓过神来后,魔尊幽魂就采取弃车保帅之法,仍旧坚持抵抗。

%75
“但这些都只是徒劳无益的!”

%76
“时间的优势在于我们。”

%77
“炼道大会已经开始准备了。宿命仙蛊就要彻底修复成功,到那时魔尊幽魂也要重新纳入到宿命的轨迹中来,根本无从逃避。”

%78
“大时代的未天庭!”

%79
紫薇仙子想到了凤金煌。

%80
再得到纯梦求真体的梦道成果之后,龙公就将这个成果教给了凤金煌,总算是正正经经传道授业了一次。虽然这个成果还是别人的。

%81
因为要采用凤金煌的仙蛊,所以还得改良出合适的仙道杀招来,才能破解龙宫周围的梦境。

%82
紫薇仙子在全力推演杀招的同时,将凤金煌也带在身边,一起推演。

%83
最初,她和龙公的目的,只是想帮助凤金煌更快地成长一些。但没想到推演成功之后,凤金煌却提出了一个方法,能够改良纯梦求真体。

%84
全新的纯梦求真体,再无之前的弊端,可以长久存在!

%85
“尊者的才情,真的是太恐怖了。在他们面前,即便是我也感觉到自己的平凡啊……”紫薇仙子在心中长叹。

\end{this_body}

