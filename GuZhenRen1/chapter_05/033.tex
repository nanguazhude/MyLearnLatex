\newsection{方源渡劫(中)}    %第三十三节:方源渡劫(中)

\begin{this_body}

方源本就擅能吃一堑长一智,之前在赶往北原的路程中,碰到泥怪、云兽,他连番吃亏,今番渡劫,怎么可能不弥补这处短板?

说起剑痕索命这道仙级杀招,还有些曲折。

方源进入琅琊福地之后,一边为自己疗伤,一边求助琅琊地灵,要求他为自己逆炼飞剑蛊等等七转仙蛊。

他现在是六转蛊仙,运用七转仙蛊,实在过于吃力。

琅琊地灵单纯,觉得方源已经加入琅琊派,成为琅琊派的客卿太上长老,是自己人了,便给他提了一个中肯的建议。

“逆炼仙蛊,我当然可以为你做到。并且从七转逆炼回六转,远比六转提升七转更容易成功。但你不觉得可惜吗?更关键的是,逆炼绝非一蹴而就之事,距离灾劫,已经没有多少时日。你现在开始逆炼,根本来不及。不如选用这份杀招。”

琅琊地灵说完,便将剑痕索命这份杀招,交给了方源。

“你可真是走运啊,这份杀招,是我上一任在数百年前收购回来的,适用于很多的剑道仙蛊。”

方源当时就感觉奇怪,长毛老祖生前只是单纯的炼道蛊仙,琅琊地灵凭白无故,收购他用不到的剑道杀招做什么?

结果,他只是稍微浏览了一下,心中便释然了。

原来,剑痕索命这个杀招,是永久性地耗费剑道仙蛊中的道痕。刻印在目标的身上。剑道仙蛊飞走之后,剑道道痕就会自寻目标的弱点,加以攻击。

就好像刚刚,方源以飞剑仙蛊为核心,催动剑痕索命。洞穿了荒兽雪怪。这一击并不致命,仙蛊贯穿的伤口,很快就自己愈合了。但它刻印下来的剑道道痕,却游走在雪怪体内,将雪怪体内深藏不漏的雪核绞杀。雪核一破,雪怪遭受致命打击,立即分崩瓦解。

如此一来。飞剑仙蛊就能绞杀掉雪怪之类的对象。再不会事倍功半,收效甚微了。

但这个杀招,有个最大的弊端。

那就是永久性的消耗仙蛊上的道痕。

仙蛊可以看做一块大道碎片,无数道痕的集合。转数越高的仙蛊,大道碎片的规模就越大,道痕数目就越多。

剑痕索命这个杀招使用次数越多,剑道仙蛊本身的道痕就越少。无法回复。这样一来,剑道仙蛊会越来越弱。次数达到一定程度之后,就会引来质变,使得六转仙蛊跌落到五转凡级,或者七转仙蛊跌落到六转。

这个弊端相当严重,使得剑痕索命这个杀招价值不高,并不太受欢迎。

上一任琅琊地灵以价低的代价,收购回来,看中的乃是杀招本身。

事实上,每使用一次剑道杀招剑痕索命。就相当于一次逆炼仙蛊的过程。只是它逆炼的成效,十分微小。需要成百上千次,才会见到成效。

方源明白这点之后,也就理解了当初琅琊地灵为什么会收购这个杀招了。

毕竟,上一任的琅琊地灵,最大的兴趣爱好,就是炼蛊。

方源思索了一下。就当即接受了这份剑道杀招。

不得不说,这个杀招实在太适合现在的他了。

大风吹鼓,雪花翻飞。

一道剑光,钻入风雪之中。漫天的风雪,遮盖不住耀眼的剑光,被它轻易洞穿。

剑光在空中划过一道优美的弧线,正中荒兽雪怪的胸口,随后就从背后刺出,绕了一圈,又回到方源的身边。

伤势眨眼间痊愈,荒兽雪怪得意地嗷叫一声,正要跨步,忽然全身崩散开来,化为一堆冰雪。

一个呼吸之后,从这堆冰雪中,又钻出数十个小雪怪。

但这些小雪怪,身高都是一两丈,不足为虑。

又是一记剑痕索命!

这一招下去,效果极佳。荒兽级的雪怪,只需一击,就能击溃。

方源满意的点点头。

直到此刻,他终于见识到剑道流派的优势之处。

“蛊师修行,养用炼三大方面,皆是博大精深。仍旧是飞剑仙蛊,只是我换了一种运用方法,就收到翻天覆地的巨大成效。再来!”

方源再次施展剑痕索命。

风雪中,荒兽雪怪相继形成,又旋即被方源干脆利落的斩杀。

雪怪的实力和体型,有着相对应的联系。

一两丈高的雪怪,一转二转蛊师都可对付。六丈高的雪怪,便是荒兽,战力媲美六转蛊仙。七丈高的雪怪,可敌七转蛊仙。

这些六丈高的雪怪,被方源一一点杀,简直像是送菜。

方源的战力有了巨大的提升,想当初他在狐仙福地,面对荒兽泥沼蟹的狼狈,现在面对绵绵不绝的荒兽雪怪,却是云淡风情,牢牢控制着场面。

方源心里却没有一丝放松,时刻保持着警惕。

“我之所以如此轻松,一来是七转飞剑蛊,对付荒兽,二来是剑痕索命本就是耗费道痕的强劲杀招。而且当初泥沼蟹携带仙蛊和稀泥,这些雪怪却没有。嗯?”

方源忽然神色一凝。

他细心的观察到,那些从荒兽雪怪死后,成形的小雪怪,体型正慢慢膨胀壮大。

一丈的变成两丈,两丈的升上三丈……

这些雪怪的战力在迅速增长!

“若任由它们发展下去,会不会长成六丈的荒兽雪怪?”方源不敢大意,手中飞剑仙蛊暂停,大手一挥,无数暗漩喷射而出。

嘭嘭嘭……

这些黑球飞弹,密密麻麻,宛若暴雨般,倾泻而下。

小雪怪们立即遭受灭顶之灾。一时间死伤惨重。

但很快,从这些小雪怪的尸雪中,又站起无数同胞。

方源杀了数十个小雪怪,但站起来的却是上百个!

方源见此,眉头轻皱。感到微微棘手。

不过他细心的观察到,站起来的上百个小雪怪,虽然数目增多,但体型却比之前更加微小了。

方源有了猜测,心念再动,立时,成百上千道浅绿色的风刃。从他身边凭空而生。嗖嗖嗖飞射而出。

一波波的风刃,在空中交织成紧密的火力网,交替覆盖住地面上的众多雪怪。

荒兽雪怪屹立不倒,但小雪怪却被切割成无数飞雪。

一时间,地面上无数刀痕浅壑,白雪四下飞溅。

方源催动着侦查蛊虫,眼绽精芒。观察整个战场。

“果然,一丈的雪怪死亡之后,就无法再产生新的雪怪了。”验证了心中猜想,他的心稍稍放下。

旋即,他的脑海中又闪过一道灵光:“是因为这片风雪的缘故吗?”

分出几份心神,他在几个呼吸的时间,接连催起上百只凡蛊。

一片幽暗的领域,从他脚下蔓延开来,很快就铺散出一片夜幕。

正是得自蛊仙黑城的战场杀招夜幕。

仙窍中,通常不能铺设战场杀招。这是因为战场杀招是临时刻印道痕。改造环境,对仙窍本身的道痕,存在冲突和伤害。

但这里的战场杀招,是仙级战场杀招。方源使出来的夜幕,却是凡级,无伤大雅。

他还掌握了雪松子的凡级战场杀招雪境,使用出来。恐怕要增益雪怪,不适合现在的局面。

夜幕一出,范围之中的风雪顿时减弱了许多威势。

果然,笼罩在其中的小雪怪,成长速度缓慢了许多。

但好景不雪像被触怒,狂风越加咆哮,冰雪疯狂肆虐,夜幕的边缘不断被压缩,很快就缩减到方源的脚边,轰的一声,被彻底压溃。

组成夜幕的上百只蛊虫,存活不到两成,被方源收起。

“荒兽雪怪接连出现,尸雪中会新生出更多的小雪怪,小雪怪还会随着风雪,以匪夷所思的速度迅速成长。但若要从根源上隔绝风雪,反而适得其反。目前看来,只能不断斩除小雪怪,同时点杀荒兽雪怪么……”方源心中总结。

果然,他的感应没有出错。

这场灾劫,十分强大。

这只是至尊仙窍的第一次灾难,却已经远远超出狐仙福地的第五次(魅蓝电影),第六次(泥沼蟹),第七次(血毒棠)地灾了。

手中的仙元,在迅速的消耗。

注意力高度集中,心神也开始疲惫。

不过,方源催动剑痕索命杀招,也越发熟练了。

时间流逝。

仙窍内的天地二气仍旧在不断地动荡,形成越来越磅礴的风雪,起先只是覆盖方源周围数百里地,现在却已经扩张到上千里。

方源面色渐渐凝重。

他意识到,这才地灾不仅规模空前,恐怕时间方面也将持续很长,远远超出十大凶灾!

荒兽雪怪越来越多,方源动用剑痕索命,已经来不及点杀。

新生的小雪怪数量增多,方源还得动用凡道杀招进行剿除。

两头兼顾,令他难以周全。

“好在这些雪怪,无法飞行。我占据高空,虽然要留一份力抵御风雪,但主动权仍旧在我的手中。”

这个念头,刚在方源的脑海中闪过,一声尖锐的鹰啼,便刺入方源的耳膜。

方源抬头一看,顿时脸色一沉,暗道不好!

只见漫天的风雪中,一头冰雪组成的铁冠鹰,已经形成。

鹰翅扑扇,带动鹰身,夹裹着凶悍气势,直向方源扑杀而来。

方源连忙掉转枪头,动用剑痕索命,除去这鹰。

但短短功夫,又有三头铁冠鹰成形,盘旋飞舞,包围住他。

压力暴涨,危机来临!

ps:上个月炼制多更蛊失败,遭受了强烈的反噬,身受重伤。现在伤势渐好,我不甘心,还是想炼呐。丢失的节操虽然碎了一地,还是想捡起来啊。唉,说多了都是泪!

\end{this_body}

