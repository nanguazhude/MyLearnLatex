\newsection{蛟龙吐息}    %第一百七十七节:蛟龙吐息

\begin{this_body}

蛊乃天地真精,人是万物之灵。[\&\#26825;\&\#33457;\&\#31958;\&\#23567;\&\#35828;\&\#32593;\&\#77;\&\#105;\&\#97;\&\#110;\&\#104;\&\#117;\&\#97;\&\#116;\&\#97;\&\#110;\&\#103;\&\#46;\&\#99;\&\#99;更新快,网站页面清爽,广告少,

人的灵性,在万物之中为第一。但是,有得必有失,天道平衡,大公无私,人虽然灵性第一,但是本身却十分脆弱。

人类没有利爪,没有皮毛,没有翅膀,没有鱼鳃。

人类本身的力量,虽然不是最弱,但是也是垫底的。和寻常荒兽,根本无法较量。

人类的肌肤,在荒兽面前,跟纸糊的没有什么两样。

但方源变化成荒兽剑蛟之后,他却是有了剑蛟本身的素质。虽然和真正的剑蛟,还有所察觉,但已经远远超出人类种族的限制。

这便是变化道当初开创时的最初思想

学习万物,将万物所长,增添到自己的身上。

人类没有利爪,可以将人手变化成利爪。人类没有皮毛,可以变化皮毛。

没有翅膀,可以产生翅膀,学习鸟儿在天空飞翔。没有鱼鳃,可以化作鱼类,在水底遨游。

当然,到了现代,变化道的思想,已经不局限于此。

而是以一道映射万道,现在变化道的主要思潮,便是模拟其他流派的长处。

时代总是在进步的。

历史的车轮,滚滚向前。

自从狂蛮魔尊开创了变化道之后,这个流派经久不衰,在人族的历史上一直流传下来,并且推陈出新,至今仍旧焕发着勃勃生机。

方源蛟龙身躯轻轻一展,便飞到了七转仙材的面前。

这是一块巨大的乌青墨石,在七转仙材中,以坚硬厚实、难以处理为特征。

龙爪打在上面,根本不能留下什么痕迹。

方源早已估算到这一点,并未试用两只蛟爪,而是翻腾蛟躯,忽然一抽,甩尾!

啪。

一声脆响,龙尾狠狠地甩在乌青墨石之上。

龙尾生疼,让方源不禁倒吸一口冷气。

而乌青墨石上,着显现出了一丝儿裂痕。

单论龙尾的甩摆,比龙爪还具威胁性。

这是剑蛟的一个小特征。(www.MianHuaTang.cc 棉花糖小说)

很多蛟龙的特征,都在细微方面,有着差别。

有些蛟龙,咬合能力极强,利齿锋利,超过龙爪和龙尾。有些蛟龙,则擅长撕扯,龙爪是它最厉害的武器。

而剑蛟本身,体型修长,行动灵敏,龙尾的厉害还要超过两只小巧的龙爪。

而在龙尾之上,是剑蛟本身的速度。

方源之前已经演练过,剑蛟的长途奔袭能力并不强,但是由静及动时的速度爆发,十分可观。

虽然比不上剑遁仙蛊,但完全不下于方源的血道移动杀招血漂流。

并且,因为是剑蛟本身的移动本能,使得方源转折方向时,比血漂流还要灵动不少,十分容易就能操纵自如。

而剑蛟本身,最厉害的武器,是龙息。

蛟龙吐息。

这几乎是任何一个蛟龙,都有的天赋本能。

就像上极天鹰,有洞穿空间,进入福地洞天的能力一样。只是蛟龙吐息十分大众化,很常见。上极天鹰的能力,却是相当稀有。

上极天鹰本身就很罕见,八转蛊仙黑凡也是机缘巧合,才得到这么一只。除他之外,黑家历史上就再也没有第二头上极天鹰了。

呼!

方源试着喷吐龙息。

刹那间,一道白光闪过,正中方源面前的六转仙材暗流金钢。

暗流金钢轻轻一震,龙息白光已经消失,整个过程及其迅速,似乎什么都没有发生过。

但下一刻,暗流金钢缓缓地发生变化,一道裂缝清晰地展现在暗流金钢上,整条裂缝笔直无比,上下一贯,带着一丝倾斜的角度。

暗流金钢就被这丝缝隙,分成了两半。随后,因为暗流金钢本身的巨大重量,两半金钢相对滑落。

砰砰两声连续的闷响。

整个大石块一样的暗流金钢,被分成两半,倒在了地上。

切面相当的平滑,像是一柄天下最锋锐的绝世兵器,直接斩出来的效果。

方源的龙爪,只能在暗流金钢表面,留下寸许的痕迹。但是他吐出的龙息,却是锋锐无当,一下子就将这个巨大的暗流金钢分成了两半!

要知道,在六转仙材当中,暗流金钢可是堪称最坚固的仙材之一。

剑蛟的龙息威力可见一般。

方源尽管早有心理准备,此刻看到分成两半的暗流金钢,也不由地蛟头轻点,相当的满意。

接着,他开始继续试验。

他对着七转仙材乌青墨石开始吐息。

乌青墨石可比暗流金钢更加坚固,剑光一般的龙息吐上去,砍得乌青墨石一道道剑痕,深达数寸。

方源连续喷吐了十六次,到达了极限。

短时间内,他再也喷吐不出任何龙息,就像是一个沙漠中的徒步者,半个多月都没有喝水,嗓子干燥沙哑无比。再吐下去,就要把整个咽喉毁掉的感觉。

而连续十六次的龙息,整个乌青墨石已经被砍得不成样子。到处都是剑痕,纵横纠缠,有些触目惊心。

修行一段时间之后,方源感到自己有所恢复。

试着再吐一口,果然有了第十七口龙息!

方源将这个时间段,暗暗记在心中。

他不断估量:“这是我单凭本身力量,能连续吐出十六口龙息。休息片刻,大约半盏茶的功夫,能恢复过来,继续喷吐龙息。不过相当勉强,吐完第十七口,就陷入之前的窘境,仍旧要继续修养半盏茶的时间。”

“如果我在其中,增添一些龙息凡蛊,应当能够将龙息的数量,增添上去。”

方源的这个剑蛟变仙道杀招中,还没有组合进龙息凡蛊。

龙息蛊比较难以炼制。

这是因为,炼制这种蛊虫的蛊材,就是蛟龙的吐息。

方源得到的这头蛟龙,是死的,他从宝黄天中守候了许久,才发现有人卖,耗费了不少代价,买了过来。

唯有活着的蛟龙,才有吐息,才能利用这个蛊材,炼制出龙息凡蛊。

当然如果能改良这种蛊方,利用其它蛊材替代蛟龙吐息,也未尝不可。不过方源显然没有这种能耐。

而除了活着的剑蛟之外,采取其它活着的蛟龙,也可以炼成龙息凡蛊。

不过这种龙息凡蛊,虽然名称一样,但还是有些微差别。虽然也能用,但需要增添一些其他凡蛊,削免当中的一些差异。

变化道在这方面是有所讲究的。

因此,方源才特意从剑蛟尸躯上,炼出各种蛟鳞蛊、蛟爪蛊等等。

用这些蛊虫,他能更快更简单地,催动出仙道杀招剑蛟变。

试验杀招告一段落,效果没有出乎方源的意料。

选择剑蛟变,他是经过深思熟虑的,这些效果都被他在此之前,详细推算过。

“仅仅只有六转变形仙蛊为核心,剑蛟变虽然形成了一定的战斗能力,但是面对七转蛊仙还是不够看。”

一般而言,七转蛊仙的防御手段,其强度就相当于乌青墨石。

方源变化的剑蛟,运用最强的武器龙息,也不过只能将乌青墨石,砍得面目前非而已。

这样的剑蛟变,在血战武斗大会上,是拿不出手的。

这点方源非常清楚。

他自然还有后续手段。

收起变化,他就飞出自家云城,很快,便来到琅琊地灵的面前。

方源直接表明来意:“见过太上大长老,我这次来,主要是换取门派贡献。”

琅琊地灵眼中闪过一抹喜色,他知道方源得到了黑凡洞天,又是盗天传人的身份,弄塌了八十八角真阳楼,手中的好东西肯定不少。

而琅琊派要发展,自然需要更多更好更珍惜的修行资源。

只是,琅琊地灵不能强行逼迫其他蛊仙上缴,这会让人心背离,琅琊派会分崩离析。

“很好,你有这份就很好。”琅琊地灵被勾起了浓重的兴趣,“这次你要向门派贡献什么?”

“先看看这个。”方源取出一物。

“蛊仙魂魄?”琅琊地灵眉头微皱。蛊仙的魂魄本身,可以充当一种仙材,若是魂道蛊仙的话,就更有价值。而更有价值的,是在于蛊仙魂魄中承载的修行记忆。

蛊仙魂魄当然并不常见。

很多蛊仙有自爆魂魄的手段,最拿手的就是影宗。当年,凤九歌对战还未恢复记忆的秦百胜,就察觉到对方有强大的魂爆手段,心生忌惮,放过了秦百胜,不愿和他死磕。

蛊仙战斗中,也很容易收不住手。比如廿二平之杀死阴婆,直接将她魂魄都灭了。

琅琊地灵接过这个蛊仙魂魄,他却发现这个魂魄,已经被方源动过手脚,已经无法再搜魂,只能充当仙材运用。

“这样的话,你的这个蛊仙魂魄,就只能换取上百贡献了。”琅琊地灵有些失望地道。

方源却笑道:“不,我认为价格还能更高一些。我们可以将这个蛊仙魂魄投上荡魂山,化为胆识蛊。蛊仙魂魄非同小可,乃是精品,一定可以化为一笔庞大的胆识蛊的。最近这段时间,咱们的胆识蛊产量一直提不上去,有了这个蛊仙魂魄,必定可以满足石人一族的需求。”

琅琊地灵恍然,点点头:“的确如此。”

他估了一下价格,又道:“这只六转蛊仙魂魄,可以算你两百一十的门派贡献。”(未完待续 \~{}\^{}\~{}。)

\end{this_body}

