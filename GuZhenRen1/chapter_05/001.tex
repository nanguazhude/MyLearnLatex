\newsection{一切都是天意}    %第一节:一切都是天意

\begin{this_body}

%1
“一切都结束了。”望着脚下的影无邪,方源口中喟叹。

%2
明白自己被暗转之后,影无邪心中充斥着绝望的冰冷,彻底放弃了。

%3
他知道:自己是彻底败了,再无任何翻盘的希望。

%4
但就在这时,异变陡生!

%5
砰。

%6
一声轻微的爆响,影无邪魂魄还有方源的仙僵之身,在刹那间自爆。

%7
仙僵之躯,一身的道痕,还有影无邪的魂魄,都化为了养分,献祭给了春秋蝉。

%8
春秋蝉发动!

%9
一缕碧光,在方源的眼中一闪即逝。

%10
春秋蝉再入光阴长河。

%11
方源大惊失色!

%12
“怎么回事?这到底是怎么回事!春秋蝉中灌输特意,怎么会被影无邪催动呢?”

%13
“不对!看他当时的神态表情,已经毫无反抗的斗志,怎么会催动我的春秋蝉?!难道他也是在演戏?”

%14
“春秋蝉是我之物,我也特意多加防范,怎么会被影无邪催使出来?”

%15
方源心中充满了震惊,还有疑惑。

%16
种种疑问,他百思不得其解。

%17
站在原地,方源一动不动,宛若石像。

%18
惊变来得太过于突然了。

%19
很快,他的额头就满是冷汗。

%20
“这可如何是好?该怎么面对?!”

%21
方源曾经靠着春秋蝉,翻盘多次,挽狂澜于既倒,扶大厦于将倾。

%22
但现在,面对他人动用春秋蝉,他赫然发现,自己毫无应对的手段!

%23
他的宙道底蕴实在过于薄弱。

%24
没有春秋蝉,他在这方面,几乎毫无建树。

%25
“是我漏算了吗?到底还有什么,我没有考虑到的?”方源眯起双眼,冷汗从额头垂落,滴到他的眼角,又向下划过他的脸庞。

%26
没有思考出什么答案。方源摇头叹息。

%27
“没有办法,我已经做到极致了。”

%28
当时的情景,方源若不按照梦境中星宿仙尊的提点去换魂,他早就彻底失败。无法翻身。根本就不能走到现在这一步。

%29
“难道说,我是因为没有按照星宿仙尊的意思,去毁了至尊仙胎蛊,所以才有此异变吗?”

%30
“春秋蝉还未恢复,此时催动。只有死路一条啊!”

%31
“不,不对!”

%32
方源陡然想起,影宗方面还有一个后手。

%33
那就是光阴长河中的鬼脸红莲。

%34
霎时间,寒意袭遍全身,他汗毛乍立。

%35
“凡事都要做最坏的打算。如果他重生成功,我这边又会有如何的变化呢?”方源正想着,就感到视野中开始模糊起来。

%36
不,不是他的视野模糊。

%37
而是整个天地,都在宙道的力量下,变得模糊起来。

%38
因为影无邪的重生。一切都在改变!

%39
呼!

%40
犹如风声灌耳,影无邪一个激灵,反应过来,不由地惊呼道:“这,这里怎么是光阴长河?怎么回事!春秋蝉是怎么催动的?我的意识怎么会替代方源,存在春秋蝉之中?”

%41
他的疑惑,不比方源少多少。

%42
虽然是当事人,但就这样稀里糊涂地到了光阴长河之上。

%43
光阴长河,河水滔滔。

%44
一个大浪,朝着春秋蝉狠狠地拍击而来。

%45
影无邪的这股意志。顿时紊乱起来,大呼死矣!春秋蝉羸弱不堪,若被这浪头拍中,根本毫无幸存的可能性!

%46
“唉……原来如此。”

%47
一声叹息。就在这千钧一发之际响起。

%48
光阴长河之中,浮现出一个鬼脸。鬼脸神情苦涩,张口吐出一朵红莲。

%49
红莲绽开,挡住索命的巨浪。

%50
随后一股无形的吸摄之力,将载着影无邪意志的春秋蝉,吸入红莲的莲心之中。

%51
一瞬间。影无邪的意志进入到一片赤暗的空间里。

%52
而在这里,另一股意志凝成幽魂魔尊的形态,正盘坐在半空中,对他苦笑。

%53
“本体居然还在光阴长河中,留下你这个后手吗?”影无邪意志楞了一下后,顿时喜出望外。

%54
幽魂魔尊的意志点点头,又摇摇头:“你也曾是本体的分魂,理应记得我。可惜因为纯梦求真体,使得你记忆消散了。”

%55
“唉!这也是没有办法的事情,你当我愿意啊?”影无邪意志苦恼无比,“还有,这究竟是怎么回事?我还以为我落到方源手里,被他算计,已经死定了。为什么会出现在光阴长河之中。还有,我明明用不了春秋蝉,为什么春秋蝉会忽然发动?”

%56
幽魂魔尊的这股意志,再次长叹一声:“我寄生在这朵红莲真传之中,观察了整个过程,也是在刚刚才明悟过来。你能进入这里,都是天意在帮你。春秋蝉也是天意催动的。”

%57
“什么?!”这个答案让影无邪意志差点震惊得当场崩溃。

%58
这太出乎意料了。

%59
若非眼前是货真价实的幽魂意志,影无邪绝不会相信这个事实。

%60
“天意不是一直为难我们吗?怎么忽然间,又反过来帮助我们呢?”影无邪问道。

%61
“你的记忆已经全消,已经没有了原先的眼界。天意,从不是看我们不顺眼,来为难我们。也从不会看我们可怜,来帮助我们。天道无情,天之道,在于损有余而补不足。所以万物平衡,相互制约。所以猛兽几乎都有天敌,高空云雾会凝结成雨水落下,哪怕是九转尊者也会被寿命所限。”

%62
顿了一顿,幽魂魔尊的意志继续道:“而人之道,损不足以奉有余,与天之道恰恰相反。一个强者,会想要变得更强。一个富有的人,会想要变得更加富有。一个美人,同样会想要变得更美。寻得长生的人,想要更进一步,获得永生。人的欲壑难填,永不满足,和天意背道而驰。”

%63
“我有些明白了。天意倾向于我,是因为我失败了。方源成为完整的天外之魔,已经超脱限制,放任他的话。会大大滴破坏天地的平衡。所以天意转移了目标,想要借我之手,来对付他,不让他的阴谋得逞。但为什么天意能驱使春秋蝉?这只仙蛊。明明已经被方源炼化了啊。”影无邪还处在懵懂的状态。

%64
幽魂魔尊意志答道:“唉!这正是天意的手笔。我也是刚刚才领悟出来。方源当初得到春秋蝉,都是天意的安排。天意早就选中他为棋子,令其重生,破坏本体大计!”

%65
“依照白凝冰提供的情报,现在回想一下。方源表现出特别之处时,是在青茅山上。他虽然资质不良,但一路走来,却是顺风顺水,每逢紧急时刻,他都能做出最明智的选择,获取最大的利益。青茅山、三王福地、王庭福地、义天山……这一路,显然,他是用了春秋蝉不止一次的。”

%66
“春秋蝉不过六转,有失败的概率。他居然次次都成功。除了上一次。可能由我们助他之外,其余部分,他方源居然没有失败一次!要知道,春秋蝉可是会令他运气衰败的。这恐怕都是天意在暗中扶持相助,最终令他在短短数年之内,从一届凡人成为蛊仙,并落入本体眼中,被本体当做应付天意的棋子。”

%67
影无邪意志听的瞠目结舌:“等等,你是说……”

%68
魔尊幽魂的意志点点头:“不错!这一次本体失败,主要还是因为方源。本体利用方源。将他当做棋子。但实际上,方源早就是天意的棋子。”

%69
“不管方源重生几次,在他第一次重生之前,本体很有可能已经成功。不仅炼出至尊仙胎蛊。而且成长到了天地不容的地步!正因如此,天意才不惜借助方源,令其重生,前来破坏我本体大计。”

%70
“方源这个人很有意思。首先,他是天外之魔,所以才有资格重生。才有能力去改变过去。普通人就算再优秀,也脱离不了宿命的结果。然后,他却又不是完整的天外之魔。所以,他还受到天意的影响和摆布。他自以为春秋蝉是其所炼,为其所用。但事实上,从来都不是这样。春秋蝉这种大杀器,早就被天意侵蚀了。很显然,这场布局从方源获得春秋蝉时,就已经开始了。但他被天意蒙蔽,并未发现什么不妥之处。毕竟,借助春秋蝉重生,只能承载一部分的意志。而意志这种东西,连魂魄都不如,最容易被影响了。就好像现在,我很想将所有的记忆和所知,都灌输给你,可惜你这股意志能够承担的很有限。”

%71
一时间,影无邪张大嘴巴,完全说不出话来!

%72
光阴长河,波涛不绝,拍打在红莲之上。

%73
一朵朵的莲瓣,随之消散。

%74
很快,就只剩下了内层的花心。

%75
“你的时间不多了。”魔尊幽魂的意志叹息一声,继续道。

%76
“方源到底是天外之魔,天意尽管可以影响他,但最终他还是听从内心的声音,选择夺取至尊仙胎蛊,而不是毁掉它。”

%77
“如此一来,他虽然还是棋子,但却已经不受棋盘的拘束。天意当然不能容许这样的存在,所以启动春秋蝉,要借助你的手来对付方源。”

%78
“但天意又绝不会,让你重生之后,夺回至尊仙胎蛊。所以,你重生之后,应当是没有机会挽回大局的,你切不可心急气躁,坏了这最后的机会。”

%79
影无邪便问:“那我该怎么做?”

%80
“如今,僵盟已灭,影宗也消散了,甚至本体都陷落险境。索性影宗诸仙,都或多或少留下了后续手段。你要利用这些,重建影宗。”

%81
“命难违,运可借!北原长生天的势力,可以利用。中洲方面,也有隐藏势力,和十大古派并不对付……”

%82
“你意志有限,知道这些已经是极限了。你借助这些,重建影宗,拯救本体。最关键的,你要将这个重要的情报,告知本体。因为当你重生之后,这份红莲真传和我也会随之改变,不会记得这些事情。到那时,你才是唯一的知情人!”

%83
“记住,木秀于林风必催之,堆出于岸流必湍之。之前,‘蛰伏暗蓄对付天意’的策略是对的。我们既然成功过一次,那么就能成功第二次!好自珍重!”

\end{this_body}

