\newsection{鬼官衣}    %第四百二十九节:鬼官衣

\begin{this_body}

改良后的仙道杀招鬼官衣,以七转换魂仙蛊,六转净魂仙蛊为核心。方源经过几次失败之后,将其成功成功催发。

他从至尊仙躯中,直接遁出魂魄。

他的魂魄悬浮在空中,并不是很凝实。

魂魄底蕴因为燃魂爆运,消耗多次,如今只是百人魂级数而已。

不同的是,在他魂魄的表面,覆盖着一层宽大的衣裳。长衣袖,衣摆拖至脚背。

这就是鬼官衣杀招所形成的效果。

泛称为鬼官衣,若细分下来,六转级数的鬼官衣称之为鬼差衣,七转为鬼将衣,八转为鬼王衣,九转为鬼尊衣。

到了九转层次,就可防备九转蛊尊的推算。当然这只是理论,就算是幽魂魔尊当年,也没有将这个杀招,推至九转级数。

因为他不需要。

他生前堂堂魔尊,屠戮天下,根本不需要防备推算。

所以,他一生中鬼官衣最高只是八转。

方源魂魄查看了一眼,纳闷:“嗯?怎么只是六转层次的鬼差衣?”

他改良之后,杀招中核心乃是七转换魂、六转净魂,也就是说,是七转层次的杀招,怎么效果却非鬼将衣,而是稍差一层的鬼差衣呢?

“难道是我改良所致?”这个猜测刚刚在方源的脑海中升腾而起,就被他否定。

这一招的改良结果,是他借助了智慧光晕,怎会有误?

“我懂了。”方源念头再一转,明悟过来。

答案就在他的身上。

在鬼官衣的里面一层,仍旧有一层衣裳。

若是说鬼差衣仿佛麻布所制,比较粗制滥造的话,那么这层内里的衣裳却是宛若丝绸,非常紧密,并且给人精致丝滑的感觉。

这是道痕丝衣,乃是九转仙道杀招鬼不觉所化,单就层次上而言,鬼差衣是万万不及的。

“原本的鬼官衣,是直接依附在我的魂魄上面的。但现在却是隔了一层,所以导致从七转层次跌落到六转了么。”

这个弊端,方源没有料想的到。

一直以来,他都享受着至尊仙躯上,道痕不互斥的优势和便利。没想到今天,在这个小角落里栽了一下跟头。

“我的至尊仙躯上,是不会有道痕互斥的现象。”

“但是各种不同流派的杀招之间,却是仍旧存在互斥的。”

“鬼不觉虽是有鬼之名,却是偷道杀招,鬼官衣则是魂道杀招,两者相互之间是互斥的。要解决这个难题,最根本的解决方法,就是将这两招合并为一招。”

方源想到这里,苦笑起来。

难度太大。

鬼不觉杀招,方源就根本不了解。并且偷盗、魂道境界,方源都很欠缺。就算是有智慧光晕可以推算,但进展将会极其缓慢,按照目前的条件计算,数百年上千年不眠不休,搞不好方源都无法推算出最终成果来。

毕竟是九转杀招,而九转对于方源而言,还相当遥远。尽管他目前可以力战八转,但实际上,他连八转蛊仙都不是。

“那就这样吧。”

方源又将魂魄注入回去。

他并不打算更改手段。

他继承了影宗的全部传承,刚刚不久,又换取了琅琊派的几乎全部的传承。

当然,只是“几乎”,其中一些最关键的地方,双方都没有交换。琅琊派如此,方源也如此。

即便如此,琅琊派的几乎全部的传承,总量上也是规模相当庞大,只是比影宗库藏,稍逊一些罢了。

毕竟长毛老祖可是中古时代,三十万年前的人物了。而琅琊福地存在的时间,也跨越了三十多万年。

单论时间的话,琅琊派可比影宗长多了。

不过,琅琊派吸纳的传承,基本上都和炼道有关,就算是其他流派,也大多是和炼道擦肩的。并且琅琊派的收集能力,肯定是不如幽魂魔尊和影宗的。

方源将这两者都纳为己用,单就传承方面,他已经将中洲十大古派的任何一个,都甩远了。两三个古派联合起来,恐怕才能在这方面和他媲美。

至于天庭,那方源是不能比的。天庭有三大仙尊真正入主过,并且历史极其悠久,从远古时代就建立起来,距今是三百多万年前。

结合方源目前掌握的传承内容,鬼官衣是最适合他,也是最具发展潜力,将来最强大的仙道杀招,可以防备他人推算,以及蒙蔽天意。

可以说,鬼官衣这一套杀招乃是幽魂真传中的精华所在。

它非常特别。

一经催发之后,就能长存,不消耗仙元,而是消耗蛊仙本身的魂魄底蕴,来维持自身。

从理论上而言,只要蛊仙魂魄尚存,鬼官衣不遭受直接打击的情况下,它就会一直存在下去,直至永久。

正因为它和蛊仙的魂魄底蕴挂钩,所以它还有一个最大的优点。

那就是,它的威能效力,会跟随着蛊仙的魂魄底蕴,而出现变化。

蛊仙魂魄底蕴增长,鬼官衣的威能就会随之上涨。若是魂魄底蕴下降,鬼官衣也会变弱。

“以前我用暗渡仙蛊,每隔一段时间,就要补充仙蛊效能,十分不便,尤其是在追逃的途中。鬼官衣一经运用,就能长存,威能还会随着我的魂魄底蕴而上涨。”

这是相当实用的仙道杀招。

事实上,和它配套的,还有其他魂道杀招。

比如说,摇魂幡杀招,锁魂链杀招。

这两招就是和鬼差衣配套,都催发出来的话,方源遁出魂魄后,不仅是身罩一层鬼衣了,还会有武器,右手的摇魂幡,左手的锁魂链。

如此一来,方源单论魂魄,就有攻击之能。

这也是幽魂真传中的一个主要特色——让人的魂魄拥有强大的战斗能力。

不过,方源却只是挑中了鬼官衣杀招,没有选择摇魂幡、锁魂链这两记仙道杀招。

验证了一番鬼官衣的效用之后,方源继续沐浴在智慧光晕当中。

借助智慧光晕,他思考自身处境。

“有了鬼官衣,再加上琅琊福地,目前暂时是安全了。即便是天庭攻来,我依靠自身实力,也能逃脱。”

“但我需要做的,是彻底毁灭天庭的宿命仙蛊。”

“十年时间!十年之后,天庭必能修复宿命,到那时我恐怕有败无胜。”

这是最令方源头疼的问题。

天庭实力极其恐怖,方源别说是攻上天庭,就算是进入中洲,都非常危险。天庭方面只需出一两位八转蛊仙,就能让方源进退艰难。

即便是依靠智慧光晕,方源也想不出如何解决这个最大的难题。

“好在还有红莲真传。红莲魔尊当年可以破坏宿命蛊,又留下真传。我拥有春秋蝉这个关键钥匙,继承真传的可能性很大。”

方源借助红莲真传,消灭了八转天庭宙道蛊仙黄史上人。那么或许借助其他的红莲真传,未必不能见到毁灭宿命仙蛊的希望。

依靠宿命蛊,天庭屹立不倒,长存至今。

依靠宿命蛊,以及星宿筹谋,天庭抵御三大魔尊。

依靠宿命蛊,天庭始终占据天下第一蛊仙势力的宝座,岿然不动,牢不可破。

毫无疑问,毁灭宿命仙蛊,是击败天庭的关键。

若不趁机毁灭宿命,十年之后,就是天庭取走方源性命的时候。

方源继承了影宗传承,对这一点,有着非常深刻的认知。

“但现在却非获取红莲真传的良机。”

“天庭失手一次,获得大量珍贵情报,下一次必会出动更强的战力,更多的八转蛊仙。”

“我唯有先行发展,提升战力,再去继承红莲真传,方有摧毁宿命蛊的些微希望。”

这样的大略,是毫无疑问的。

方源现在仙元缺乏,仙元石储备见底,逆流护身印已经开始被针对,之前战斗中,凤九歌的音道杀招已经开始透过逆流护身印,波及到方源本体。光阴长河一战,斩杀黄史上人,是借助石莲岛,相当勉强。

而方源手中掌握着幽魂真传为首的无数传承,掌握着荡魂山、落魄谷,还有智慧光晕。

这样的情况下,不发展自身,还到处乱跑,那就是作死!(未完待续。)

\end{this_body}

