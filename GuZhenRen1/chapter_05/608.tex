\newsection{今古亭四旬子}    %第六百一十节:今古亭四旬子

\begin{this_body}

“陶铸真传……”

若换做数月前,方源或许还可利用南疆蛊仙,来交换这份传承。

但是现在,他手头上的南疆俘虏,几乎都已经被他吞窍,直接撕票了。

这些南疆蛊仙,无一不是七转强者,至于夏槎更是八转修为。至尊仙窍吞并了这些,还有市井中的诸多遗留仙窍福地,使得方源修为暴涨,一举从七转上升到八转,跨越了七转的最后浩劫。

本来方源打算,在五界山脉继续俘虏南疆蛊仙,然后勒索。但陆畏因的出手,破坏了方源的这个小计划。

不仅是单纯的修为提升,因为至尊仙窍吞窍毫无减损,方源身上的道痕也狠狠暴涨!

通过道可道仙蛊的仔细测量,如今方源身上的道痕积累十分雄厚。

宙道道痕原本极为稀少,几乎垫底,但是如今一跃而上,成为第一,共有七万三千余道痕。几乎所有的,都是夏槎贡献,可谓是“强取豪夺、一夜暴富”的典型。

有这么多的宙道道痕,方源的每一记宙道杀招,都能增幅七十三倍多。再加上春、夏二蛊,本身就是八转层次,因此爆发出来的战斗力十分可观。在五界山脉中,抵挡住诸多蛊仙的联合攻潮。

当然,这也有五域界壁的地利,南疆蛊仙之间配合不当,相互减损的种种因素在内。

第二位的是水道道痕,超过了六万大关。

水道本身之前就有许多积累,方源先后斩杀了散仙周礼、武遗海、尤婵等等,而曾经葬身市井的东海散修,也多是水道流派。

第三位的是变化道痕,拥有五万。这是原来的第一,一直都有积累,现在已经落到了第三位。

第四位是魂道,规模超过三万。仙窍本身的道痕,以及之前渡劫黄泉鬼武像地灾,青陀神武像天劫的收获等等积累,多达二万九千多。剩下的数百道痕,则是方源魂修,在魂魄上增添的。按照幽魂真传记载,百万人魂、千万人魂、亿人魂、荒魂……魂修程度达到荒魂,魂魄上的道痕才会有上千的规模。

再接下来,便是冰雪道、律道、金道、木道、炎道、风道等常见流派,都有两万左右的规模。

再次一等的是,暗道、光道、力道、气道、运道、音道、血道等等,徘徊在一万关口。

最后垫底的,则是食道、虚道等流派,都是几千、数百的规模,甚至为零的都有。

再看梦境。

方源手中所有的梦境,在这数月期间,都被他全部探索成功,各个流派的梦境都有不同程度的涨幅。

最高的当然还是炼道,准无上大宗师境界!

第二是偷道、宙道并列,大宗师境界。

第三梯队是血道、力道、星道、智道等等,都是宗师一级。律道、魂道也加入其中。

在下面诸多流派,不便详述。

不像道痕数量,境界方面是越高越难提升。提升最显著的还是宙道、魂道、律道。

除了道痕、梦境之外,方源在至尊仙窍的经营建设上,也因为吞并了大量的仙窍,开发程度飙升,达到百分之十五。

值得一提的是,在夏槎仙窍中,方源意外地发现她的仙窍中,居然也建设了一座年华池!

如此一来,方源的至尊仙窍里,就有两座年华池了。

除了这些之外,方源最大的收获,还有两个。

一个便是落魄印。

这记仙道杀招,方源一直在参悟,直到最近,这才推算完全。

第二个则是一座仙蛊屋雏形。

方源早就想搭建仙蛊屋,方便白凝冰、黑楼兰等人为自己出力作战。

为了搭建这座仙蛊屋,方源普遍参考了隐士居、云城、阴流巨城、玄冰屋、不坏铁堡、血河车、噩耗阴宅、绿波亭、织茧阁、金霄坛、悔池、羽圣城(空中城)、悔池、百合宫等大量仙蛊屋的构造,重点参详了惊鸿乱斗台。

又在之前的数月内,不断勒索南疆正道,以蛊换蛊为主,换取了许多关键仙蛊,这才搭建出了一个框架。

这座仙蛊屋乃是方源独创,可谓前无古人。又因为要前往光阴长河,所以以宙道仙蛊为主,诸类威能方面,又最擅长隐匿。

虽然只是一个框架,但在五界山脉中实战效果不俗。影无邪等人正是依靠此屋,和方源配合默契,成功俘虏了君神光。

光阴长河,潮起潮落,浪花飞溅,激起七彩华光。

方源就乘着仙蛊屋雏形,在河水中缓慢穿梭。

六转宙道分身,已经从至尊仙窍中出来,站在他的身旁,一直闭目,默默感应着。

红莲魔尊当初留下了多道真传,分布在光阴长河之中。而春秋蝉就是感应它们的关键钥匙。

不过光阴长河极为宽广,要在其中找寻到几座石莲岛,除了运气之外,更需要大量的时间和精力来搜索。

“此次探索光阴长河,若是能寻着红莲真传,那是最好不过。”

“但最重要的,还是试探出天庭在这里的布置,打它个措手不及,甚至若有可能的话,直接击溃这层布置!”

方源一边思考,一边在至尊仙窍中搜刮君神光的魂魄。

君神光的魂魄,原本有着光道道痕的防护,但中了方源一记落魄印之后,这层防护已经彻底被摧毁,同时还伤及君神光魂魄本体,使得他奄奄一息。

正因如此,影无邪催出引魂入梦之后,君神光才干脆中招,昏睡过去。

引魂入梦杀招的前提,是感应魂魄,然后再比较魂魄之间的差距,敌人差距自己越大,施展就越容易成功。

能对一位八转蛊仙造成这种战果,可见落魄印的厉害!

不过,比较逆流护身印,再想想方源是利用落魄谷来充当此招核心,有这样的威能也不足为奇。

如今,君神光魂魄如此孱弱,当然架不住方源的手段,源源不断的情报被方源接受过来。

“天庭方面,究竟有什么布置呢?君神光定然知晓一二。”

须臾,方源眉头微微一展,有了发现:“哦?宙道仙蛊屋今古亭……”

他刚刚在脑海中泛起这个念头,眼前的河水忽然高高涌起,一座凉亭宛若飞舟,乘浪而来。

这凉亭四面透风,结构简朴。亭盖似乎是黄草编织,亭柱是灰扑扑的白石,并未磨平。亭中有一屏风,算是最为华丽的装饰。而亭内站着四位蛊仙,皆是七转修为,望着方源的仙蛊屋雏形,眼中闪烁着危险的精芒。

“这今古亭乃是纯粹的宙道仙蛊屋,能察古观今,最擅长侦查。天庭布置此屋,我一入内,就被对方察觉了。厉害!厉害!”方源心头微微一沉,嘴上却交口称赞起来,为身边的影无邪等人解释。

影无邪、白凝冰等人顿时眉头轻皱,或者眼盖一层阴霾。

他们多少了解一些,方源的仙蛊屋雏形,最强的方面就是隐匿行迹。没想到直接被今古亭识破。这等若是在最擅长的地方,被对方狠狠击败。还未开战,就很被动。

今古亭冲到方源等人面前后,就戛然而止,距离仙蛊屋雏形千步左右,不动如山。

亭中一人语气冰冷:“中洲四旬子在此奉命驻守,魔头,你就不要想着前行了。”

身后一人紧接着喝道:“没错。想要过去,除非是踏着我们四人的尸体!”

“四旬子……”方源直接搜魂君神光,很快就得到他们四人的情报。

原来这四人,皆是宙道蛊仙,乃是四胞胎,由中洲十大古派灵蝶谷从小发觉,一直特意栽培。他们四人三兄一妹,分别是上旬子、中旬子、下旬子、旬果子。

\end{this_body}

