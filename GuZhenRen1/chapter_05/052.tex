\newsection{智慧难用}    %第五十二节:智慧难用

\begin{this_body}

%1
宝黄天,来源于多宝道人。

%2
此人乃是中古时代的蛊仙,修为高达八转。他一位获得一块黄天碎片,将其融入自身仙窍洞天之后,就形成了宝黄天。

%3
蛊仙的仙窍一旦吞并了太古九天的碎片,就会坐落在外界五域,再不能移动。

%4
宝黄天亦是如此,自第一次落下之后,三十多万年仍旧在原处。

%5
只是宝黄天的具体位置,究竟在哪里?这是一个巨大的谜团,没有人知晓。智道蛊仙们也曾前仆后继的推算,都没有得到准确的答案。

%6
宝黄天的原主人,便是多宝道人。其后也就再也没有出现过继任者。

%7
就连历代仙尊、魔尊,都未找到过宝黄天的具体方位。

%8
“宝黄天屹立至今,已经成为五域蛊仙界的第一交易市场。怎么会忽然间关闭了呢?”商家蛊仙们俱都惊疑不定。

%9
宝黄天自多宝道人创办之后,三十多万年,长年开放,已经不知不觉地成为蛊仙修行的主要助力,占据蛊仙心目中一个相当重要的位置。

%10
长久以来,蛊仙们已经习惯了宝黄天的存在。现在宝黄天忽然间关闭,实在是打了众多蛊仙一个措手不及。

%11
“你们说,宝黄天怎么会忽然关闭?”

%12
“糟糕了,我们要图谋义天山大战的赌资。这一下关闭,那笔庞大的赌资还在宝黄天中呢。”

%13
“难道说,宝黄天的天灵也想要贪墨了这笔巨款不成?”

%14
商家太上大长老微微摇头,否决道:“宝黄天的天灵是不会在乎这笔赌资的,要贪墨的话,宝黄天也不会成就今天的名声和地位。历史上都未有宝黄天天灵贪墨的记载。不过关于宝黄天的关闭,倒是有过几次记录的。”

%15
“是吗?我等愿闻其详。”商青青忙问道。

%16
商家太上大长老便徐徐开口。说出了一道秘辛。

%17
原来历史中记载的几次宝黄天关闭,都和同一件事物有关。

%18
这边是太古黄天的碎片。

%19
太古九天,赤橙黄绿青蓝紫白黑。如今只剩下黑天、白天。黄天早已碎去。

%20
当年。多宝道人就是心知渡劫不成,主动用自家仙窍。吞并了太古黄天的一块碎片小世界,成就了宝黄天。虽然终身被困在宝黄天中,失去了自由,但从此后无灾无劫。

%21
宝黄天吞并了太古黄天碎片,至此之后,也就只能吞并黄天碎片,才能扩张自身。

%22
历史记载中,有过数次交易。

%23
各有人物向宝黄天天灵。贩卖了太古黄天的碎片世界。

%24
天灵得到之后,便主动关闭宝黄天,全心全意地吞并黄天碎片,扩张自己。

%25
“这一次,如不出意外,应当也是这个情况。”最后,商家太上大长老猜测道。

%26
他猜的一点都没有错。

%27
能够打动宝黄天天灵的,世间恐怕也只有黄天碎片了。

%28
太古九天破碎已久,黄天碎片早已绝迹。除非大气运、大机缘者,才有希望获得。

%29
不过天庭历史悠久。库藏丰富到难以想象的程度。太古黄天碎片也拿得出,并且规模还不小。

%30
如果规模小,宝黄天天灵不需要关闭宝黄天。分出一份力量就能吞并了。

%31
如今,直接引发宝黄天关闭,定然是黄天碎片庞大,使得天灵不得不全心全力去吞并容纳。

%32
宝黄天始终保持中立,与世无争,又位置极端神秘,天庭都奈何不得。

%33
若是天庭强硬要求天灵,公开排挤影无邪。

%34
天灵根本不会搭理,而且其他四域蛊仙。也不会有多少顾虑。五域界壁尚在,天庭的影响和控制范围。只局限在中洲这一块领域。

%35
但天庭财大气粗,竟直接抛售了一块庞大的黄天碎片。如此一来。影无邪的自救计划就被破坏得一干二净!

%36
“可恶!天庭居然这样干!”影无邪满脸狰狞之色,双拳紧握,爆出青筋,一口牙齿咬得嘎吱作响。

%37
他心中愤怒无比,但天庭以势压人,无可抵御。他只能任由施为,无可奈何。

%38
噗。

%39
忽然,影无邪气急攻心,忍不住吐出一大口鲜血,直接仰头倒下。

%40
仇恨和愤怒的情绪充斥在他的心头,除此之外,还有冰冷彻骨的寒意。

%41
天庭直接抛售黄天碎片,付出这么大的代价,也要铲除影无邪,可见对方不达目的誓不罢休的决心。

%42
绝境!

%43
天庭此举,等若是堵住了影无邪的最后一条退路。

%44
实在太狠辣了。

%45
没有宝黄天,影无邪纵使想卖仙材,也无法子,只能烂在手中。

%46
更让他感到十分恶心的是,他的那只仙蛊铁冠鹰力都陷落在里面了。

%47
交易不来仙材,他如何继续炼蛊?

%48
没有仙蛊,巧妇难为无米之炊,他如何能打破眼前的僵局?

%49
宝黄天主动关闭,天灵全力吞并容纳黄天碎片,这绝非一朝一夕之事。历史记载中,最短的时间,都要两个多月。

%50
走投无路!

%51
似乎只能任由天庭推算,影无邪等人只能被动防守,被拖延时间,最终让难以抵御的强敌们找上门来。

%52
琅琊福地。

%53
方源此刻面貌已经大变,他再次成为了仙僵。

%54
他曾经改良了涅槃火,研究出了力道仙级杀招,能从生死两个身份相互转化。时而变成活人蛊仙,时而便是死人仙僵。

%55
但此刻,他用的却不是这个方法。

%56
他的力道仙蛊尽数丢失,根本无法施展那个力道杀招。

%57
方源的办法很简单,直接魂魄飞离出来,离开原来的肉身,钻进一个力道仙僵的尸体之中。

%58
这些力道仙僵的躯壳,来源于焚天魔女。方源重生之后,还是保险起见,为生死仙窍做足准备,就从焚天魔女处谋求来许多力道仙僵尸躯。

%59
这些尸躯,他都放在星象福地之中,义天山之后,从琅琊地灵那边又辗转回到了自己手中。

%60
除了这些仙僵尸躯,还有大量的珍稀仙材。

%61
许多准九转的仙材,都是方源从地沟的那处蛊阵中央得来。还有大部分的七转、八转仙材,是从落天河中捡来的。是沾了仙僵薄青苏醒,剑纵中洲的便宜。

%62
方源手中的仙元、仙元石不多,但他并不缺乏炼蛊仙材。只是这些仙材规格太高,用在当下,还有些不合时宜。

%63
仙材不是重点,重点是智慧蛊。

%64
方源一双眼睛紧紧盯着智慧蛊良久,却不见它有丝毫动弹,更遑论什么智慧光晕了。

%65
“我用其他仙僵尸躯替代自己,结果无效。还是不能引出智慧光晕啊。”

%66
方源对这个结果,倒没有意外。

%67
其实早在之前,他就已经在狐仙福地中做过类似的试验。

%68
智慧蛊一直都未被方源炼化,只是当初大同风下,为了生存才和方源达成协议。

%69
求生是万物的本能。

%70
但智慧蛊认定的,似乎是方源整个的人。肉身、魂魄缺一不可,才能引出它智慧的光晕。单单缺少某一个,不管是单凭肉身,还是魂魄,都是不行。

%71
“这么说,最直接的办法,就是把我的肉身重新夺回来吗?”方源心中思量。

%72
他原来的肉身,还是仙僵。

%73
不需要再改变,用来沐浴智慧光晕,再合适不过。

%74
不过此法,虽然直接,但不合实际。

%75
“影无邪……已经失踪,手中可能拥有春秋蝉、定仙游等等仙蛊。除此之外,还有黑楼兰、太白云生,说不定也在帮他。”

%76
之前方源是因为灾劫,无暇顾及。现在每次想起影无邪这个麻烦,都会令方源感到头疼不已。

%77
“所以,更要将方正炼成血神子啊。”

%78
方源如今的肉身,是从至尊仙胎蛊中变化出来的,无父无母,天地万物和他之间,没有任何的血脉联系。

%79
但方源仍旧要将方正炼成血神子。

%80
用方正炼出的血神子,可以克制方源的肉身。

%81
就算将来夺不回原来仙僵之躯,方源也可以在这方正血神子的基础上,加以改良。或许有可能,在结合自身魂魄的情况下,重新得到智慧蛊的认可。

%82
当然,这只是方源纯粹直觉。

%83
不过直觉向来不可小觑,尤其是方源的血道境界,已经高达宗师级数。

%84
方源试探智慧蛊的时候,琅琊地灵也在旁边观摩。看到结果后,他幽幽叹息,十分遗憾。

%85
方源向琅琊地灵投去关注的目光:“不知道太上大长老,有什么法子,可以利用智慧蛊?”

%86
琅琊地灵瞪了方源一眼,没好气地道:“有个最简单的法子。那就是你将自身修为提升到九转,直接将这只智慧蛊炼化了就行!”

%87
方源苦笑:“这个方法我自然知晓,我想问的是,还有什么其他方法?”

%88
“唉,还有个不是方法的方法。那就是我现在不断地用寿蛊喂养它,喂养的次数多了,说不定就能得到智慧蛊的认可了。如果这个方法成功,智慧蛊就归附于我了,到那时,你不给我也得给我。”琅琊地灵想到妙处,自己都嘿嘿的笑起来。

%89
方源眼皮耸搭下来:“你讲的这么直接,真的好吗?”

%90
“你以为我想啊?!”琅琊地灵气得直跺脚,“地灵只能说真话啊,该死,我怎么又把心里的图谋给说出来了呢?你这个可恶的家伙!”

%91
------------

%92
第三波净网活动开始了。

%93
如题。

%94
第三波净网,展开了。

%95
如果万一,此书被禁的话,那我就只能在微信上发布。

%96
总会有些人偷偷摸摸的举报,我也是醉了。碰到过很多次,已经麻木了。

%97
这本书写到现在,饱受“迫害”。我写书以来,还未感到过的艰难。

%98
但我还会写下去。

%99
一定会完本。

%100
尽我自己的全力,去写好它!

%101
感谢支持我的小伙伴们。

%102
可以的话,蛊真人微.信公众号,希望大家关注一下。

%103
未谋胜,先虑败。

%104
算是一份保险吧。

\end{this_body}

