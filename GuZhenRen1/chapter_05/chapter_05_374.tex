\newsection{收取西漠僵盟遗藏}    %第三百七十四节:收取西漠僵盟遗藏

\begin{this_body}

%1
方源一句话,说得三羽童人大惊失色,连忙呼救。?

%2
一道身影迅显现,来到三羽童人的面前,安抚他道:“弟弟别怕,有姐姐在呢。”

%3
这位女蛊仙,身姿玲珑较小,仿佛幼女,一身雕花的绿衣,双耳吊坠着两颗玉珠子,珠子内部仿佛含水,摇曳间,荡漾出绿水澜光。

%4
“小女子仙号翠波,阁下是谁?虚言恫吓我弟,很好玩么?”

%5
方源打量这位翠波仙子。

%6
她身上洋溢着七转蛊仙的气息,自认为和七转修为的方源,能够平等对话。

%7
她肤白若雪,只是被耳坠上的绿水波光衬得有些渗人。面容也算姣好,但是鼻翼稍微尖锐,破坏了形象。

%8
方源摸了摸自己的肚皮,叹息一声:“我实话实说,谈何虚言?嘿嘿,你们既然送上门来,那就都吃了罢。”

%9
至尊仙体的弊端,就是要吞窍。

%10
这点,方源从紫山真君的遗藏中,已然明确得知。

%11
其实,按照影宗的计划,是要炼制出十转仙蛊至尊仙胎蛊。

%12
至尊仙胎蛊,既然号称“至尊”,自然是要脱九转层次。这是魔尊幽魂的野心筹谋。

%13
可惜时运不到,最终他只来得及炼出九转层次的至尊仙胎蛊,还被方源给抢了。

%14
按照魔尊幽魂的设想,十转至尊仙胎蛊,并无弊端,几乎完美无缺。但是落到九转层次,就有了一个不是缺陷的缺陷。

%15
至尊仙胎蛊本身是仙蛊,仙蛊就需要喂养。

%16
至尊仙胎蛊的食物,便是仙胎。

%17
什么是仙胎?

%18
仙胎其实是远古年间的一种叫法。

%19
就好像是空窍,又有人称之为华池,或者称之为紫府。绝大多数人称之为空窍。

%20
仙人和凡人之间最大的区别是什么?

%21
仙窍,也即是仙胎。只是对仙窍的叫法,更为广泛。

%22
要用他人的仙窍,喂养至尊仙胎蛊。仙窍的流派不论,但是修为上却有讲究。

%23
比如现在,方源修行到了七转,就需要吞并七转仙窍,才能喂饱至尊仙胎蛊。等到他修为提升到了八转,就必须吞并八转仙窍。

%24
完美的是十转至尊仙胎蛊,它不需要任何食物,是真正能够转变成完美仙体的蛊。不像方源此时的情况,虽然是变作了至尊仙体,但是却可以通过一些法门,重新化为至尊仙胎蛊。

%25
现在方源虽然利用至尊仙胎蛊,转变成了至尊仙体,却仍旧有着弊端留下。

%26
幸亏之前,方源改变了修行计划,延长仙窍时间,放弃宙道资源,避让天灾地劫,以吞并他人仙窍来暴涨修为。

%27
如此一来,阴差阳错,反而喂养了至尊仙体。

%28
方源对三羽童人所说的话,一点都不虚假,在寻找到这里的途中,方源遭遇到了三位六转西漠的蛊仙,都被他杀死,俘虏了魂魄,吞并了其中两个仙窍,另外一个,方源境界不足,不能吞并,只能任由它落下去形成福地。

%29
翠波仙子和三羽童人,也算是一个强势的组合。

%30
他们俩耍弄心机,不仅将之前的妖蚊子骗入了蛊阵中,替他们试阵,而且翠波仙子还潜藏在一侧,进行埋伏。

%31
碰到寻常的七转蛊仙,翠波仙子自然可以打交道,平等对话,可惜的是,他们这一次碰到的是方源。

%32
“正巧来了一位七转蛊仙,还是修行水道。我吞下这人,短时间内至尊仙胎的喂养,就不成问题了。”

%33
方源心中早已经杀意沸腾,碰到一个能够吞并的仙窍,机会少有。

%34
尤其是在西漠这块地方,炎道、风道两个流派最为盛行,水道很稀少。和东海蛊仙界,形成鲜明对比。

%35
方源是水道的宗师,但是炎道、风道的流派境界,却是比较普通的。

%36
轰隆!

%37
方源悍然出手,催出仙道杀招万我第一式力道大手印。

%38
顿时,大手印排开空气,以势不可挡的气势,向翠波仙子、三羽童人二位扑来。

%39
三羽童人惊叫一声,连忙躲闪。

%40
翠波仙子也流露出一抹惊骇之意,双手一挥,挥出一股翠绿的酸水。

%41
酸水仿佛一条灵蛇,扑中大手印,顿时让大手印腐蚀削减了两成。

%42
不过,大手印还剩下八成威能,在方源的操纵下,舍弃三羽童人,一味地追击翠波仙子而去。

%43
翠波仙子一边飞退,口中一边喊道:“仙友且慢动手,我乃是千变老祖的第三房妾侍。仙友何不看在千变老祖的份上,与我们化敌为友,一同分享这僵盟分部的宝藏?”

%44
翠波仙子虽然是七转修为,但实力不济,或者说很是普通。

%45
但她身份却有些非同凡响。

%46
竟是能和千变老祖扯上关系。

%47
即便方源对西漠蛊仙界的情报,所知不多,但也早已听闻千变老祖的名头。

%48
这位老祖,专门修行变化道,年轻的时候,得到过狂蛮魔尊留下的一部分真传,统共有近千种变化。因此自称千变老祖。

%49
他虽然继承魔尊真传,但并非魔道蛊仙,而是西漠最著名的散修。

%50
他的个人趣味,则贴近巨阳仙尊,喜好美色华服,娶纳了成百上千的妻妾。其中就有蛊仙近十位!

%51
翠波仙子便是其中一位了。

%52
“只是要取走西漠僵盟分部的遗藏,就要招来千变老祖的仇恨么?”方源皱起了眉头。

%53
翠波仙子一边躲闪,一边察言观色,立即又道:“其实我此次前来,也是奉老祖之命。阁下战力雄厚,但要闯过守护蛊阵,可不是蛮干就能成的。宝藏还未到手,我们何必先得分出个你死我活呢?说不定耽搁了时间,就有另外的变故出现。”

%54
翠波仙子的口才不错,说得方源微微点头:“你此言有理。”

%55
翠波仙子和三羽童人闻言大喜,但下一刻就见方源打开仙窍门户,放出白凝冰、黑楼兰、妙音仙子三人。

%56
“你们去打杀了这两人,我来亲自收取遗藏。”方源开口。

%57
白凝冰冷哼一声,虽然不满方源号施令,但还是冲向了翠波仙子。

%58
黑楼兰对付三羽童人。

%59
妙音仙子则浅笑倩兮,留守旁观,作为策应。

%60
翠波仙子、三羽童人彻底惊惶起来。

%61
他们万万没有料到,方源仙窍当中,还藏着三位蛊仙,并且都是异域的蛊仙。

%62
这种情况很罕见。

%63
蛊仙不会没事,就把其他蛊仙藏在自己的仙窍里头的。

%64
很快,翠波仙子、三羽童人陷入下风,被黑、白二仙打得毫无还手之力。

%65
“这是两位十绝体成仙!手段精妙,前所未有!”

%66
“我想起来了,他们是通缉犯!魔道的贼子!这个人就是八十八角真阳楼倒塌的凶犯之一啊!”

%67
三羽童人、翠波仙子现了方源等人的一些身份后,顿时更加惊惶,斗志沦丧,转身就逃。

%68
黑白二仙紧追不舍,很快就追出了这片地下洞穴。

%69
方源站在沙暴面前,酝酿许久的手段对准眼前,催出去。

%70
几个呼吸之后,沙暴就明显地减弱下来,并且从中划分两边,露出一道小径。

%71
方源毫无犹豫,只身进入其中。

%72
妙音仙子仍旧留守在外,防止他人出现。

%73
沙暴随着方源的进入,又开始强盛起来,小径消失,恢复了旧状。

%74
一座城池,很快展现在方源的视野当中。

%75
尸地城。

%76
乍看之下,这是一座由黄土堆砌的城池,有些不起眼的感觉。

%77
但实际上,这座城池的本质是一座庞大的凡蛊屋。就像是北原僵盟分部的阴流巨城。

%78
不过,西漠僵盟分部,还是和北原有所区别。

%79
西漠僵盟分部,真正的大本营,是一片土道福地。

%80
尸地城位于福地之中。

%81
曾经,影宗动万年大计,逆天而行,炼制至尊仙胎蛊,结果大败亏输。

%82
西漠僵盟分部的大本营福地,就主动舍弃,将里面的资源精华,都藏入尸地城,将整座凡蛊屋搬迁到了易位沙漠底下,并布置了防护蛊阵。

%83
这个做法保全了西漠僵盟分部的大部分底蕴,义天山大战不久之后,原来的西漠僵盟的土道福地就被西漠蛊仙联手攻破。

%84
但是劫掠之后,这些西漠蛊仙并不满足,甚至有些恼羞成怒。

%85
因为他们没有得到,计划中的战利品。

%86
经过他们的不断推算和侦查,其中当然也有天意推波助澜,终于找到了易位沙漠,并确认西漠僵盟的遗藏,就藏在此处。

%87
但要找到正确的地气洞穴,并不容易。且不说这些洞穴,数量众多,繁若星辰,更关键的是这些洞穴还在不断地移动着。

%88
每时每刻,都有旧的洞穴消失,又有新的洞穴产生。

%89
这让西漠蛊仙难以探寻。

%90
方源当然不一样,他成为影宗之主,找到这里,并进入尸地城,对他而言,根本是轻而易举的事情。

%91
“不过,在此处遇到翠波仙子,恐怕也是天意布局了。让我初入西漠,就要和八转大能作对。”

%92
“我和白凝冰等人,身上都被种下侦查杀招。”

%93
“必须赶紧将这些遗藏取走,尽快转移!”

%94
方源打开仙窍门户,伸手一招,将整个尸地城,都纳入仙窍当中去。

%95
尸地城中并无仙蛊储备,不过守护蛊阵,却有仙蛊为核心。

%96
一共两只仙蛊,其中一只,正是方源来此的主要目的。

\end{this_body}

