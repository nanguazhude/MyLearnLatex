\newsection{不灭星标}    %第三百八十八节:不灭星标

\begin{this_body}

这一天,是令整个南疆仙凡,都震动惊慌的一天。txt下载80txt.com

地脉的全面震动,影响了整个南疆,重点区域更是天翻地覆,高山崩塌,流水阻断,天坑掩埋。天地伟力在这一刻,展现出了无以伦比的恐怖力量。

不幸的凡人们,惨死在这场浩大而又突然的地劫之中,各个超级势力都在统计损失。

很多凡人势力,都在顷刻泯灭。就连超级家族,都有不少的资源点彻底损毁,损失颇重。

突如其来的地脉震动,让原本紧张的南疆政局,为之一静。

各大超级势力,都暂时将之前的矛盾放置一边,不约而同地开始积极调查地脉乱动的缘由。

如此的震动,别说是上溯万年,十万年,数十万年,甚至就是到上古时代、远古时代,都没有过这样的情况发生。

古往今来,这是头一遭。

南疆蛊仙界,不管是正道、魔道,还是散仙,都很紧张,并且好奇,究竟是什么原因,导致地脉如此巨大规模的动乱。

“大时代的征兆,出现了。”远在西漠的方源,通过宝黄天收集到南疆地脉动乱的情报,他幽幽地叹了一口气。

此时此刻,他已经再次转移。

又到了另外一处地点。

夜幕渐去,远处的天边开始显现鱼肚白。

黎明来了。

很快,冉冉升起的朝阳,照亮了天边,恢弘的光,散射在广袤的沙漠上。

这是新的一天。

方源一身白袍,满脸风霜,目光沉凝地望着旭日,他的心头是沉重的。

旭日东升,他的前途并不如眼前的景象这般明朗。

以前他被蒙在鼓里,现在得到了紫山真君的遗藏之后,他这才明白前世今生当中,大时代和幽魂、影宗之间的关系。

大时代本就应该是这个时候来临的。

只是方源前世五百年,因为魔尊幽魂逆天改命成功,压制住了天庭、天意,运用无上手段硬生生将大时代往后拖延了整整五百年!

天意被幽魂逼入绝境,不得不破釜沉舟,选中方源为棋子,利用春秋蝉,将其送到过去,孤注一掷,险之又险破坏了影宗大计。

因此,整个世界的运转,又回到了正轨。

没有了幽魂和影宗的拖延,大时代已经缓缓地拉开了序幕。

地脉的震动,便是这浩大的序幕的一角。

“大时代来临,五域界壁都要消散。不过五域界壁消散的前提,是整个五域南疆、北原、中洲、西漠、东海的地脉,统统接触,并且融合成一个统一的整体。”

“接下来的这段时间里,地脉的震动,将会频繁发生。并且不只是南疆,其余的四域中洲、东海、西漠、北原都会陆续发生。”

“之所以是南疆首先爆发地脉的震动,是因为整个五域当中,就数南疆的土道道痕最为浓郁。因此南疆的地脉,最为浑厚雄阔,是五域第一。正因为如此,大时代来临,天地变易,才会最先影响到它。”

方源的目光中,透露出思索之色。

大时代就这样来临,无疑极大地干扰了他曾经制定下来的修行计划。

本来他的计划,时间是十分充裕和从容的。

方源不断潜修,利用梦境提升境界,积蓄财力,提升修为。<strong>在线阅读天火大道Http://wWw.qiushu.cc/</strong>等到五域乱战的大时代降临,他必定已经是八转修为,财力雄厚,即便达不到全流派宗师的境地,大概也差不多。

如此一来,他才能在五域乱战中,获取最大的收益。

但是现在,显然不行了。

大时代来临的太快,方源的计划彻底被打乱了。

不仅是这样,他如今还受到天庭的追杀,天意的围剿,处境狼狈不堪,每时每刻都在逃亡,都要戒备,因为说不定下一刻就有强大的敌人忽然出现,对自己实施袭杀。

“这是一场千古巨变!”

“目前为止,恐怕只有我、天庭、影宗,以及其他少数势力,知晓地脉震动的背后意义。”

“不过其他蛊仙都不会蠢笨,一番调查之后,总会多多少少的意识到这方面。”

“等再过一段时间,这股意识就会形成共识,全天下的蛊仙界都会明白:一场前所未有的天地变革,就要来临。”

“唉,我重生以来的优势,就又要减少一项。”

“若是没有梦境大战,我还有武遗海的身份。南疆地脉震动,会出现大量的地沟、天坑,并且无数的矿脉,地下的珍奇走兽、奇植诡草,都会被翻出来。”

“这是一场饕餮盛宴,会引发各大势力,全部蛊仙的争抢。五域都会发生这样的事情,但以南疆为最,因为南疆的地脉最为雄厚,地下的珍宝最多最有价值。最次的是东海,因为汪洋一片,在这场地脉的震动中,东海的机缘反而是最少的。”

方源回忆着。

东海一直以来,都是五域中资源最丰富的地域。这一点,引得其他四域的蛊仙,都要冒险前往,探索搜寻,辅助自己的修行。

但这场地脉震动,收获最小的也是东海。

方源前世五百年,就有蛊仙评论:这场机缘,好像就是天地赐予蛊仙海量的奇珍异宝,故意平衡五域的实力一样。

方源若是还有武遗海的身份,此时此刻,将迎来一场巨大的机遇。凭借他的记忆,他可是记得好多地点,会涌现大量的天地珍宝,乃至是野生的仙蛊!

可惜这一次,他的身份暴露了,这场浩大的机缘,恐怕要与自己失之交臂。

方源心中惋惜,却不知道,这场地脉震动反而是帮了他一个忙。

正因为地脉震动,导致武家仙阵破损,才让武庸计谋失败,令方源安然度过了一场劫难,没有被武家暗算到。

地脉震动是肯定要发生的,但早不早,晚不晚偏偏就在那个时候发生,显然是方源燃魂爆运的原因了。

“好在我这一次,掌握了七转仙蛊自爱。之前成功催动仙道杀招洁身自好,将身上的一些道痕都顺利清除掉,其中就包括了武家的盟约。”

这样一来,方源就彻底地脱离了武家,和武家没有关系了。

“想必此时,我的命牌蛊、魂灯蛊都已经破碎了吧。武庸也再不能利用这个把柄,来对付我了。”

这方面的顾虑解除了,但是也让方源看到了洁身自好的不足之处。

洁身自好杀招,虽然奇妙非凡,只要是对蛊仙不利的道痕,都能清除,但它到底不够灵敏如意。

就比如说此时的方源,身上和武家的盟约都被清除干净,但是和其他势力,和楚度、琅琊地灵以及其他异族,还有疯魔窟等等的盟约道痕,都没有影响。

这是因为,这些盟约道痕,对方源并非“不利”,所以洁身自好就不会清除掉它们。

然而一旦方源叛变,这些盟约道痕对方源不利时,往往盟约会首先发动,严厉地制裁方源,方源再用洁身自好来拯救自己,无疑就晚了许多。

最主要的,还是紫薇仙子种在方源身上的侦查杀招。

这些道痕,对方源非常不利,洁身自好杀招自然要清除它们。然而让方源吃惊的是,尽管杀招有效,将这些道痕清除了一些,但很快,道痕就开始回复,重新生成。

按照这种速度,再过三个时辰的时间,代表天庭侦查杀招的道痕,就又会恢复旧观了。

“洁身自好乃是紫山真君遗赠当中,最为强大的手段了。”

“它虽然对我有效,有帮助,但是这次面对的对手却过于强大。”

“我身上所中的侦查杀招,应当就是紫薇仙子依靠星宿棋盘种下来的。”

方源虽然是猜测,但心中基本上已经可以肯定。

因为这种道痕自行生长的现象,他曾经接触过。

那就是他原本的身体,那具六转力道――八臂天魔仙僵肉身了。

如今这具肉身中,第二仙窍已经消散至无,第一五转空窍当中还封印着春秋蝉。

春秋蝉当然不能催用,因为里面潜藏着天意。

而肉身当中,更是被影无邪动用魂道手段,布置下了险恶的陷阱。

一旦方源寄托魂魄进去,陷阱就会发动起来,剿杀魂魄,然后滋养肉身,生出更多的魂道道痕,让陷阱更加凶残顽固。

这就是方源第一次接触到,道痕自行生长的现象。

虽然是影无邪发动的杀招,但来源于盗天魔尊。

得到紫山真君的遗赠之后,方源也才知道,只有九转蛊尊级数的人物,才能创造出这种道痕生长的杀招来。

所以,方源才猜测,紫薇仙子是利用了星宿棋盘,才施展出来这一侦查杀招。

因为星宿棋盘就是星宿仙尊之物。

得益于方源现在的身份,他对天庭方面的认知,也暴增了许多。

方源现在回想一下,其实,其他的仙尊魔尊,也都有这样的类似手段。

道痕自生,不耗仙元。

比如,巨阳仙尊他开创出来的八转仙蛊鸿运齐天。马鸿运用了,鸿运效果一直都存在着,直至他肉身死亡。

鸿运效果究其本质,就是运道道痕。

马鸿运的肉身上刻印下运道道痕,并且吞吐与他息息相关的周围人的气运,让自己的运道道痕越来越多。并且在整个过程中,他都不需要耗费任何的仙元、真元。

盗天魔尊也掌握着这种程度的杀招。

那就是神不知、鬼不觉。

这两大究极防御杀招,都是在魂魄上编织出一层道痕法衣。前者让对象脱离天意等等的关注和推算,后者让一切魂魄感知不到。

这样的杀招效果,一直都存在,并且不耗费任何的仙元。同时,一旦有所破损,道痕法衣还会自行生长愈合。

“世间公认,每一位九转蛊尊,所主修的流派境界,都是无上大宗师。”

“按照紫山真君传给我的修行经验中规划:宗师境界是触类旁通,大宗师境界是运用自然道痕,无上大宗师境界便是道痕自行生长,甚至自行运转,脱离仙元的桎梏吗?”

如此就一目了然了。

紫山真君的遗藏,让方源对整个蛊仙境界,都有了一个清晰、全面的认知。

现在,用亲身经历的事实印证下来,更让方源体悟深刻。

八臂天魔仙僵肉身上的魂道陷阱,方源已有办法可以解除了。但是这具至尊仙体上的侦查道痕,方源却是应付得很吃力。

“我用洁身自好,虽然可以消弭一定数量的侦查道痕,但是这些道痕减少得越多,生长的速度就越快!”

“我单纯用洁身自好杀招,施展一次,能消弭掉其中一成的道痕。但是过个大半天的时间,消失的道痕就会重新生长出来。”

“如此一来,我若要彻底消除身上的侦查杀招,就要时时刻刻催动洁身自好杀招,至少半个月的时间,持续不多,中间不得有一丝的停息时间。”

方源按照现有的情报,大略估算了一下。

很麻烦,方源所中的侦查杀招,比他料想中还要麻烦得多。

而且,最关键的是,这不切实际。

洁身自好杀招,并非持续性的杀招,它是按照次数来的。催动一次,就是一次,每一次持续的时间是固定的。

虽然拥有了自爱仙蛊,也催动成功了杀招洁身自爱,但是方源并没有如愿以偿,将身上不利的道痕清除干净。

他遇到了另外的难题。

该怎么办?

“若是给你充足的时间,让你准备,或许还有解除的希望。可惜现在,你是不会有的。”天庭中,紫薇仙子笑了笑。

她对种在方源等人身上的侦查杀招,非常的有信心。

因为这是星宿棋盘中的手段,也就是星宿仙尊亲自创造的杀招――不灭星标。

在所有的侦查杀招当中,不灭星标只能标明对象的位置,单这个威能效用,远不如紫薇仙子种在凤九歌等人身上的侦查杀招星影演绎。

后者能让星宿棋盘,直接演绎出对象所经历的事情和场景,一目了然,如同身临其境。

但是不灭星标最强大的地方,就在于――它源源不断,生生不灭,寻常手段根本无法消除掉它。

可以说,只要这杀招存在的一天,方源就是她紫薇仙子棋盘中的一颗棋子,逃不脱紫薇仙子的掌控。除掉抹杀,只是时间早晚问题。

和方源等人的对决,紫薇仙子完全占据主动,大势在手,岿然不动。

并且只要不给方源自由、宽裕的发展机会,时间拖得越久,她紫薇仙子以及天庭的优势就会越来越大。

可以说,紫薇仙子已经利于不败之地!

通知:因为工作量太大,8月份签名明信片的回馈书友活动,统计名单地址的时间延长一周。当中出现了一些问题,目前正在积极调整,广大书友请多多谅解。另外,明天31号,在8月份的最后一天,我会首次更新一出《蛊真人》的番外剧情,这是关于某位魔尊的历史往事。届时更新会出现在31号晚上的微信公众号“蛊真人”上。我一直致力于,写好每一个人物,完善整个蛊世界。但是有些人物的精彩故事,的确不能放在正文里面,因为会影响整个剧情的连贯性,只好写番外了。大家有兴趣的话,可以关注一下蛊真人的微信公众号。(未完待续。)

\end{this_body}

