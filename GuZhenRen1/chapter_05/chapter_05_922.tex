\newsection{安居乐业}    %第九百二十六节:安居乐业

\begin{this_body}



%1
帝君城战场。

%2
轰隆隆,烟尘滚滚。

%3
大地开裂,巨大的深沟向前蔓延,吞噬路上的一切事物,直逼帝君城。

%4
帝君城的毁灭就在眼前!

%5
此时此刻,帝君城中纵然还有着数位中洲蛊仙守护,但要在这么短的时间内,救下全城这么多人的性命,这些蛊仙根本是无能为力。

%6
地沟是天然形成的,谁也无法测算出地沟会在哪里出现。惟独方源料到了这一点。

%7
因为这道地沟,在他上一世是同样发生,摧毁了帝君城。

%8
但中洲和西漠两方都不知情,地沟出现着实打了双方一个措手不及!

%9
只不过西漠一方是惊喜交加,中洲一方则是惊恐了。

%10
就在帝君城中生灵涂炭的关键时刻,一座仙蛊屋庭挺身而出,化作一道碧芒,闪电般飞到上空。

%11
是豆神宫!

%12
“阻止他!”中洲蛊仙们下意识地齐声狂吼。

%13
“房睇长要做什么?”千变老祖等人也感到奇怪,豆神宫的行为,并不在作战计划之中。

%14
房睇长面露震惊之色,此时此刻的他不仅联络不了外界,而且全身都动弹不得。

%15
豆神宫嗡嗡颤抖,一股无形的强大力量死死束缚住他。

%16
在环绕大殿的壁画中,一股意志缓缓钻出。

%17
意志迅速凝聚成形,变成青年模样,身着青衫,头系白带,一头黑发披散在肩,温文尔雅的样子。

%18
见到这股意志,房睇长脸色铁青,失声道:“元莲仙尊!”

%19
元莲仙尊……在历代尊者中仙元最是充沛,他是天庭第三代仙王,他创建天莲派,他扫清魔氛,荡天彻地,重归秩序。

%20
没错,这股意志正是元莲意志!

%21
“可恶,是潜藏在壁画之中,所以才瞒过了我吗?”房睇长一瞬间明白过来,他对画道始终都不了解。

%22
“但为什么这股元莲意志一出场,我就失去了对豆神宫的掌控,被死死的束缚住,失去了自由?”一瞬间,房睇长联想到了青仇。

%23
他心中一片冰凉,曾经一度以为自己完全炼化了豆神宫,但眼前残酷的事实表明,豆神宫仍旧还有秘密,还有最终的权限掌控在元莲意志的手中。

%24
这层手段不管是真正的房睇长还是方源的这具分身,都看不透,一直被蒙在鼓里。

%25
尊者的仙蛊屋,岂是那么容易夺取的呢?

%26
轰然一声,豆神宫稳稳落地,位于帝君城最中央,开始坐镇大局。

%27
无穷无尽的碧芒从豆神宫蔓延开去,如水一般温润,光线毫不刺眼,映照半个天边。

%28
帝君城各处不管是城砖墙角,都被碧光浸透。光芒中,一幅幅壁画在帝君城的每个角落逐渐显现。

%29
这是一幅幅的众生图,描绘的是帝君城中的某一时间某一幕的景象。

%30
有的是集市,人流攒动,商贩争相吆喝,各种商品琳琅满目。

%31
有的是几户房屋院落的剪影,院落的高树上,几个鸟雀的窝,窝里有雏鸟待飞。而在树下有一群孩童相互嬉闹、追逐,充满对未来的期许。

%32
有一支接亲娶妻的队伍,徐徐的从南边拐过来,新郎骑乘着大马,而在身后是一座凡蛊屋花轿。这种凡蛊屋一般是中洲嫁娶的时候才拿出来用,轿子表面繁花盛开,五颜六色。而在花轿的后面,则是一群脚夫挑着新娘的丰厚嫁妆。

%33
有一座茶馆,门口的街道上行人如织,茶馆楼上坐满了宾客,他们一边吃着早点,一边透过窗户看着街道。茶馆用于招揽顾客的旗幡在微微的晨风中飘动。

%34
还有一幅画描述的是黄昏。在接近城门的街道角落里,有一个摊子,一位或许有智道蛊虫的瞎眼老蛊师,正充当算命的先生,为一位女子预算前程。

%35
……

%36
不管是清晨亦或夜晚,是城门边还是闹市区,帝君城中人们安居乐业的一幕幕,都形成一幅幅的画面。

%37
毫无疑问,这是画道的手段,充斥人道的奥妙。

%38
仙道杀招——安居乐业!

%39
“帝君城竟然藏有元莲仙尊的手段?”

%40
“元莲仙尊曾经以凡人身份在帝君城居住了一段时间……难道说,这个传闻非是故事,而是事实吗?”

%41
“画道果然是玄妙无比!”

%42
中洲蛊仙们自然狂喜。

%43
西漠一方则感觉大大不妙。

%44
豆神宫和帝君城链接成一个整体,形成一座巨大的仙蛊屋。中枢当然是豆神宫,而外围则是曾经的帝君城。

%45
地沟冲刷过来,宛若巨兽张开巨口,瞬间将帝君城的地基摧毁抽空。

%46
但帝君城却已经截然不同,它悬空而立,岿然不动。

%47
城池内的万千民众们惊呼连连。

%48
他们欢喜相拥,他们竞相跳跃,必死局面居然逃出生天!

%49
一大部分人喜极而泣,跪倒在地,对一幅幅图画膜拜。

%50
忽然间,许多地方有产生了新的图画,图画上的内容正是这些民众跪地膜拜的景象。

%51
这记杀招安居乐业,似乎能够不断地从凡人中汲取力量,壮大自身。

%52
图画的力量不断积蓄,连绵不休,又回涌到豆神宫中。

%53
元莲意志执掌这股力量,更加牢牢掌控住豆神宫,他看向房睇长,脸上神情似笑非笑:“方源,我还要多谢你带得豆神宫过来,免除了一场人间浩劫。”

%54
这是元莲本体当初布置的手段!上一世豆神宫没有被房家带来,因此帝君城遭受摧毁。而这一世却是通过方源分身为桥梁,豆神宫及时出现,拯救了万千民众。

%55
心中的不妙感觉越发浓郁,房睇长极力挣扎。

%56
但这时又一股玄力加身,牵扯他,将他直接送入到豆神宫的壁画里去。

%57
房睇长视野骤变,再定睛瞧看,发现自己已经进入了画里。

%58
眼前的景象他非常熟悉,一片灰白的土地,地面上还有深坑,这是他当初种下仙豆的地方。

%59
如今,绝大多数的豆籽都孵化成了豆神兵卒用于大战,这片土地中仅剩下数颗豆籽,还在酝酿,默默生长。

%60
房睇长又发现自己已经可以行动自如,但没有用,不管他尝试什么办法,都不能冲破这层壁画。

%61
“我被封印镇压在画里了!”房睇长在一瞬间联想到了上一世的花子,她就是被镇压在了监天塔内地壁画中。

%62
房睇长试图联络本体,结果发现仍旧无法和外界沟通。他连宝黄天都感应不到,一切相关的沟通手段都失效。

%63
“冷静、冷静,我还有最后的希望!”房睇长沉默片刻,催起因果神树杀招。

%64
幸亏这个杀招经过了一番改良,一经发动,立即让画外殿内的元莲意志微微变色。

%65
这个手段的确是抗衡他的关键,即便他是元莲意志也不能完全阻止。

%66
此时此刻,帝君城内,中洲炼蛊大会的最终大比会场。

%67
这场惊心动魄的大比,终于迎来了最后时刻,胜负即见分晓。

%68
“我炼成了”

%69
“成功了!”

%70
叶凡、洪易异口同声,他们几乎同时炼成了蛊虫。

%71
真要决断二人的胜负,必须要请宙道蛊师深究时间上的微妙差距,

%72
但就在这时,碧芒忽然涌现,将这两人牢牢夹裹,带着他们疾飞出去。

%73
这股碧光速度极快,夹带着洪易、叶凡二人,很快就冲到了豆神宫前。

%74
豆神宫内元莲意志不禁咬牙,看这番模样,这碧光中的两人和方源牵连最深,恐怕不是他的棋子,就是他的得力干将!

%75
元莲意志连忙催动豆神宫,夹裹叶凡、洪易的碧光冲撞到豆神宫上,忽然流光一折,又投向远处,最终落进帝君城的某个壁画之中,叶凡、洪易二人也都被封印起来。

%76
“这是怎么回事?”

%77
“我们究竟谁胜谁负?”

%78
洪易和叶凡对视,均是一脸懵逼,他们置身在一副安居乐业杀招的壁画之中,眼前是一片祥和平静的生活景象。

%79
这不禁让他们深深疑惑起来。

%80
“我们究竟在哪里?”

%81
“帝君城外不是有蛊仙在激战吗?怎么连声响都听不到了?”

%82
“难道我是在做梦?”

%83
洪易狠狠掐了自己一下,顿时龇牙咧嘴,痛痛痛!

\end{this_body}

