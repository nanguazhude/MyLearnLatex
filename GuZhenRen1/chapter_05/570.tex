\newsection{云竹山脉}    %第五百七十二节:云竹山脉

\begin{this_body}

“收。”方源悬浮在半空中,打开仙窍门户。

大量的野生摄心蛊,被无形而又温和的力量牵引起来,汇集到空中,仿佛是一股亮丽的云雾,纷纷投入到方源的仙窍当中去。

这个过程,足足持续了半盏茶的功夫。

方源略微估算了一下:“五转摄心蛊有三千多只,四转则有五万,三转以下的摄心蛊规模多达百万。”

摄心蛊形如雨花石,不同转数之间,颜色和形态都不一样。它是智道蛊虫,用了之后,能摄取目标的心神,令目标恍惚,暂时失去洞察和思考的能力。若是搭配其他蛊虫,就能更进一步,形成杀招。摄心蛊在战斗中具有很高的实用价值,用途最多的是搭配奴道,增添奴役成效。当然,智道蛊师也用这种蛊虫,只是智道这个流派一直都稀少和神秘。

蛊师养蛊艰难,通常只有个位数的凡蛊在身上。但到了蛊仙这个层次,凡蛊都是数以万计,只有仙蛊才是蛊仙难求之物。

方源这种情况,是例外中的例外。

“摄心河滩乃是中型资源点,这一次猛攻,击破蛊阵,镇守在这里的蛊仙直接逃命去了,因此没有对这些摄心蛊下手,这才让我得了这样一笔丰厚的财富。”

若是镇守此地的蛊仙,撤离之前,将摄心蛊收取部分,或者直接摧毁它们,方源此次的战利品无疑就大打折扣了。

想来那蛊仙为了抵御方源,操纵仙阵拼尽全力,已无能力再分心收取摄心蛊。撤离之际,也没有手段,或者没有这个决断和狠辣,来摧毁这里。

方源收取了摄心蛊后,目光又停留在这处河滩上。

六转力道仙蛊挽澜。

他暗中催动这只仙蛊,使得河水像是被一片丝绸,被无形的巨手拈了起来,然后投入到方源的仙窍之中去。

这处摄心河滩,环境特殊,充斥智道道痕,当然也有土道、水道、木道等其他道痕,充当基石,只是规模总量远不如智道道痕。

方源收取的这片河水,也有着浓郁的智道道痕在。

“可惜,大部分的智道道痕还在这片大地上啊。”方源暗暗遗憾。

这就没有办法了。

他虽然有拔山仙蛊,但针对的是山峦,不是一块土地。

“这些石块也带走吧。”摄心河滩上的大小石块,也蕴含着智道道痕,都是上佳的蛊材。方源也统统打包带走。

反正能带走的,他都带走了。

片刻后,他催动七转定仙游,离开此处,留下一个被洗劫一空的凹地。

方源现在受到通缉,绝不能在一个地方停留时间过长,否则就是留给其他敌人包围、算计他的机会。

所以他刚刚强攻掠影地沟,被大阵阻挡一段时间后,就明智地暂且撤离了,并非是发现了池曲由在暗中的阴谋暗算。

方源心神一定,利用定仙游,他直接跨越了十多万里,来到一处山脉的上空。

只见这片山脉,东西纵横,巍峨壮阔,连绵起伏。山上没有树,而是长满竹子。这种竹子非常粗壮,每一根都有参天古木的体型规格。竹子的叶片,多呈三角状,根部厚,叶尖薄,质地又锋锐又坚硬,很多凡人采摘这种竹叶,绑在木棍上,形成粗制滥造的木枪。所以,这种竹子被称之为枪尖竹。

枪尖竹林广袤无比,覆盖整座山脉。竹林中生活着许多生灵,生机勃勃。它们大多数都有云道道痕,其中数量最多的就是云狸。这种雪白皮毛的狸猫,能够在空中短暂地漂浮。此时方源俯视脚下,就见到一群云狸在碧绿的竹林中,上下漂浮,相互嬉戏,一派生机勃勃,安静祥和的画面。

方源眼中精芒一闪:“这就是云竹山脉了,池家掌控的大型资源点。”

超级势力在五域中掌握的资源点,基本上分微型、小型、中型、大型、巨型五种。

方源至尊仙窍中的龙鱼海域,就是巨型资源点,可以产出金银铜铁等海量龙鱼。乃是方源仙子最主要的经济支柱,没有之一。

“糟、糟糕,方源这魔头来我们这了!”

“掠影地沟被他强攻,摄心河滩已被他攻破,没想到他竟来到云竹山脉。偏偏又是这么关键的时刻,唉。”

镇守在这里的两位池家六转蛊仙,见到方源,十分惊慌失措。

方源动用定仙游传送过来,动静本身就不小,并且方源也没有隐形匿迹的打算,身影没有遮掩,立即就让池家两位蛊仙察觉。

这两位蛊仙此刻正身处山脉的深处,在他们面前,有一株高达七丈的巨大枪尖竹,竹身粗壮无比,十个成年男子手拉手合抱,都未必能抱得住这株巨竹。

而在这巨竹的体内,一个仙蛊的雏形正在酝酿当中。

没有错,这里有一只野生仙蛊要成型!

池家两位蛊仙看守云竹山脉,重点也放在这里,仙蛊唯一,这只快要成型的木道仙蛊,就是眼前云竹山脉中,最有价值的宝藏。

轰轰轰!

方源大袖一拂,洒下无数烟花和雷霆。两位池家蛊仙齐声一叹,催起仙阵,挡住方源攻势。

池家擅长的就是阵道,池曲由又是阵道大宗师,可以利用天地自然中的道痕,再借助凡蛊,布置出仙阵来。

所以,池家把守的资源点,几乎都有仙阵。毕竟这些资源点,都有一个共同特点,那就是都有着浓郁的道痕!

随着方源一阵轰炸,云竹山脉上浮现出一层碧绿的光斑,挡住方源攻势。并且无数枪尖竹开始摇曳,竹叶暴射而出,狂风骤起,夹裹无数叶片,宛若千军万马齐射,锋锐的竹叶射向方源。

“这处仙阵不仅规模比摄心河滩要大,而且还有反击的手段。”方源笑了笑,忽然运起逆流护身印。

竹叶射在他的身上,旋即就毫无疑问地被逆反回去,杀向仙阵。

“这是逆流护身印!快快快,停了这个反击手段。”两位六转蛊仙一阵忙乱,又相互打气,“仙蛊就要成型了,还差半柱香的时间,我们要坚守住,等到家族的援军。”

仙阵在他俩的操纵下,再无攻势,而是一味死守。

方源无法借力,不以为意地笑了笑,停下逆流护身印。

他再次催出万蛟杀招,对着仙阵狂轰滥炸。

仙阵剧烈震荡,两位六转蛊仙满头大汗,咬牙死守,吐血支撑,场面十分艰难。

攻了一会儿,方源忽然眼中精满一闪,看出了此阵的破绽。

方源的阵道境界,可是有着宗师的水准!

他当即照准这个破绽,开始狂轰滥炸。

炸了一阵子后,云竹山脉的仙阵发出轰的一声巨响,覆盖整个山脉的翠绿光斑,忽然消散了一小片。

“不好,他打坏了一片仙阵!”

“坚持住,援军就要来了。”

池家两位六转不免大惊失色。

方源宛若流星,轰的一声,砸在不再设防的云竹山上。

阎罗子魂爆!

他发现魂爆更加适合强拆仙阵,便毫不犹豫地用出来。

万事开头难,这座云竹仙阵被打开一个缺口后,接下来破解的难度就下降数倍,这点破绽被方源充分利用。

两位池家蛊仙的如意算盘没有打响。

方源拆迁的速度,越来越快,仙阵被他拆得越来越多,最终波及到池家蛊仙身边。

“来不及了,我们快撤!”

“那这只仙蛊怎么办?就差最后的一小会儿时间,它就要真正成形了啊。”

“快走吧,再不走命就没有了。”

另外一位蛊仙顿时一惊,清醒过来,抹抹闷头冷汗:“你说的是,我们这就走!但这仙蛊怎么办?”

蛊仙同伴的脸色犹豫挣扎了一番,然后咬牙切齿:“毁了!”<!--80txt.com-ouoou-->

\end{this_body}

