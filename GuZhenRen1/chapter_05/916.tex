\newsection{方源战龙公}    %第九百一十九节:方源战龙公

\begin{this_body}

天庭战场。

“这就是关键之人——方源?”冰塞川口中喃喃,凝神仰望高空的方源。

牛魔、花子皆目瞪口呆:“不是说方源只是七转修为吗?他何时成为了八转?”

“这小子!”毛里球龇牙咧嘴,内心感慨无比。

它上一次见到方源,还是逆流河一役,方源依靠逆流河翻盘。除此之外,实力远逊于毛里球。

没想到这一次见面,他竟然能毁掉元始气墙!

没有了元始气墙,劫运坛宛若龙归大海,虎入深山。

一直困扰他们的束缚没有了,放眼望去皆是可以进攻的路线!

“要遭!”紫薇仙子面色惨白,一时间也想不到对策。

“进攻——!”下一刻,从劫运坛中传呼冰塞川的呼啸。

被前有古人杀招召唤而来的北原诸仙,顿时双眼绽**芒,嗷嗷直叫,分散开来,四处出击。

这群北原蛮子他们被压抑得着实太久!

在场的龙公、紫薇仙子、秦鼎菱,不过三位而已,根本无暇分身来阻挡这么多的蛊仙。

关键时刻,一声虎啸,跑出一头煞狴。又有鹤呖轻响,飞出一头青玉鹤。

这是天庭当中的两头太古传奇,一个是煞狴九十五,一个是阮丹。

两位太古传奇参战,拼命拦截北原诸仙。

紫薇仙子紧急回归中央大殿,那里还有正元老人需要保护。

而秦鼎菱则直接冲向方源。

方源立足在万年斗飞车中扫视战场,见到北原诸仙四处分散,他满意地点点头。

他破坏气墙的这一手真的是暗中准备了很久。

无量气海杀招原本就是当年气相名垂天下的手段。它气象雄阔,浩瀚磅礴,偏又繁复多变,可攻可守,堪称多面手。

它本身具有成长性,能够汲取天下万气,增长自身威能。气相开创出这一招后,就不断汲取各类气息,储藏在仙窍中,包含有生气、死气、剑气、刀气、暮气、朝气……

存储的气息规模、种类越多,无量气海杀招的威能便越大。

催动这招不仅会消耗仙元,同时也会消耗存储的气。即便不用,时间久了,这些气息也会在不断地自然损耗。

当初的气相洞天天灵就是凭借此招,帮助气相洞天渡过了许多灾劫,当中就包含多次的万劫!方源面对的时候,无量气海杀招已经被削弱到都谷底,就连天灵都因伤沉睡。

所以,无量气海杀招的威能,绝不是之前表现的那样。

从上一世重生,方源就一直琢磨着如何破解元始气墙。

得到气海无量杀招之后,他就敏锐地觉察到,能够利用此招克制气墙。

不过元始气墙到底是仙尊手段,哪怕方源拥有气海无量杀招,拥有气道大宗师境界,想要破解了气墙,也需要至少十多年的时间全力推算。

关键的契机,还在龙公身上。

龙公为了招揽气海老祖,不惜将元始仙尊的气道杀招当做诱饵。方源虽然只得了一半,但因为自身强烈要求,当中的内容就包括了元始气墙。

有了它,方源再利用智慧蛊的光晕,总算是赶在大战之前,勉强将气海无量杀招改良出来。

至于方源如何突袭进入天庭,依靠的便是天相杀招。他将宙道杀招缩时组并到天相杀招上,从而令他能够迅速入侵。毕竟他的偷道、宙道底蕴都十分深厚。尤其是房睇长分身那边,参悟了偷道仙蛊屋,带给了他不少灵感和启发。

气墙这个因素太关键了!

有了气墙,天庭的敌人只有三个路径可选,就是当年三位魔尊进攻的路线。

但气墙一旦消失,偌大的天庭完全可以自由闯荡,单靠现在的这些天庭蛊仙根本防御不过来。

天庭战场的攻守之势,顿时因此彻底颠倒。

方源一出手,就令长生天和己方化被动为主动,场面大优。

“方源,就让我秦鼎菱来试试你的成色。”秦鼎菱化作一道黄金光虹,划破长空,扑向方源。

方源瞥了她一眼,这还是双方第一次照面。

只见秦鼎菱身躯高于常人,双肩宽阔,身材高挑健美,尤其是双腿修长,气质出众,仿佛天生就高贵。她鼻梁高,嘴唇薄,双眼细长,此刻目光中绽射出凌厉的寒意。她不是传统意义上的美人,她的美令人心折,散发出凛然之威,让人不敢正视,不敢冒犯,印象极其深刻。

“难怪当年巨阳仙尊会将你选为妃子。”方源淡淡回了一句,立即让秦鼎菱的攻势又凌厉三分。

方源冷冷一笑,却是驾驭着万年斗飞车,电射而出,避开秦鼎菱。

秦鼎菱心中顿时咯噔一下,暗中叫糟。

她虽然实力强盛,从金道转修运道,当下天庭战力她仅次于龙公,没有任何短板。但速度方面却是比不过万年斗飞车。

此刻,万年斗飞车下激流澎湃,助推着这座八转宙道仙蛊屋贯通战场,直扑仙墓方向。

方源不出手则以,一出手就直指天庭的七寸!最致命的地方!

这一下,就连龙公都看不下去了。

仙墓乃是天庭的重中之重。

这里面沉眠了究竟多少蛊仙,就连龙公都难以窥得全貌。

这是天庭最大的财富,是天庭组建开始,长存至今,年年月月不断积累下来的雄厚底蕴。

若是上一世,长生天一方虽然都杀到监天塔内,但始终对仙墓无法威胁。但现在气墙一破,有无数条进攻路径可以选择,直达仙墓。

龙公舍弃劫运坛飞升而上,速度极快,挡在万年斗飞车前进的路上。

而在万年斗飞车之后,秦鼎菱一直紧追不舍。

一瞬间,方源陷入到龙公和秦鼎菱的夹攻的困境里。

“休要慌张,我来助你!”

“保住此人,他是巨阳仙祖所说的关键人物啊!”

“不错,就算是死,也要护住他的性命。”

周围的北原诸仙迅速向方源靠拢。

冰塞川没有丝毫犹豫,立即出动劫运坛,前去接应方源。眼下他们有着共同的强敌——天庭,必须要精诚合作。

方源既然支援他,帮助他脱困,那么冰塞川也要帮助方源,抵挡住天庭两大强者的夹攻。

这对于冰塞川而言,也没有什么好选择的。

他若是任凭方源被龙公消灭,那么不仅损失了一位强大战力,还会让万年斗飞车这座八转仙蛊屋无主。

面对眼前遭受夹攻的处境,方源却是一脸的平静。不过,他正要施展手段御敌时,忽然间一股火气充斥他的心田,让他怒火中烧。

然后,又有一股丧气缠绕他的全身,让他周身防御雪崩般节节下滑。

又有一团老气,纠结他的身心,令他寿元不断削减。

还有毒气由内而外,迅速扩散开来,带来致命威胁……

晦气盖顶,令自身运势迅速衰落……

傻气充斥脑海,念头碰撞思考出来的一些结果简直是蠢透了……

寒气灌体,令自己动作迟缓……

种种气息缠绕全身,方源实力迅速滑落谷底,短短几个呼吸,他就只能媲美七转蛊仙。再过一段时间,他连寻常六转蛊仙都要不如。

龙公见方源气势迅速崩塌,浑身上下缠绕各种气流,心头大喜:“这是元始仙尊的手段!方源破解了气墙,因此惹上了这一招。”

他毕竟修行过元始气道真传,立即辨认出跟脚来。

但下一刻,龙公就见一道微光从他身后射出,好像是一颗棋子似的东西,直接没入到方源的体内。

一瞬间,方源身上气息消散,仿佛之前的一幕只是一场幻觉而已。

龙公身心陡震,闪电般回头,看清一缺抱憾亭中无极魔尊的虚影竟史无前例地离开座位,站在亭边远眺这里,龙公的瞳孔顿时微微一缩。

关键时刻,无极魔尊出手,将方源身上的种种气流驱散干净。

北原众仙看到这一幕,顿时心头一松。

冰塞川倒是有些存疑:“原来方源也有令无极魔尊转身的手段。但他究竟是怎么做到的?我根本没有看清他究竟有什么动作啊。”

仙道杀招——龙啸波。

龙公猛地张口,发出一声恢弘的龙鸣。

龙鸣声浪滚滚向前,激荡四方。

仙道杀招——随身闪。

下一刻,龙公顺着龙鸣声浪一个瞬移,直接来到方源的面前!

方源此时还站在船首,两人照面,相距不过一臂的距离。

一方是红莲师尊,龙人之祖——龙公,一方面是红莲传人,新龙人之祖——方源!

“糟糕!”

“方源托大了。”

“快回到仙蛊屋里去啊!”

北原诸仙如坠地狱,狂呼出声。龙公神威,已经深刻在他们的心田,从未有人能够正面抵挡得住龙公。即便是劫运坛,巨阳仙尊亲自搭建的传奇仙蛊屋,也被龙公揍得找不着北。长生天的一伙强者迫于龙公威势,只能缩在劫运坛中,不敢出来应战。

仙道杀招——龙爪击!

龙公冷哼一声,右手忽然变作龙爪,对准方源的头颅狠狠一挥。

啪。

下一刻,一声轻响,方源抬起毛茸茸的左臂,架住了龙公的龙爪。

“挡、挡住了?!”

不管是北原众仙,还是秦鼎菱、紫薇仙子等人,都看得瞪大双眼。

从未有人能够正面挡住龙公,更何况此次抵挡,竟还是如此轻描淡写!

方源面目全非,已经变作了一头匪猴。

匪猴隶属力道,浑体皮毛如金,上面布满黑色的老虎斑纹。腰部然生长出皮毛,遮盖住裆部和尾部,宛若皮裙。

匪猴是南疆才有的野兽,崇尚力量,掰手腕是猴群中最主要的社交活动。小猴子一生下来,就能掰手腕。掰手腕,在匪猴群中不仅是游戏,更是化解纠纷的常用手段。

有蛊仙辨认出了匪猴,感到非常奇怪:“什么时候匪猴也能这么强大?”

正常的太古匪猴也架不住龙公的一击,若学方源这样,太古匪猴的左臂早就咔嚓一声,直接断掉了。

方源的变化当然不是这么简单的。

他在之前,先是动用了万物大同变杀招。

这一招是兽灾仙人的超绝手段,变化道的至高奥义,能够让其他流派的种种道痕都变作变化道道痕。

方源身上有多少道痕?

一百万以上的气道道痕,三十多万的炼道道痕,二十多万的变化道痕,这不是全部,只是最多的三部分。

单单只算是这部分道痕,通过万物大同变转化之后,就是一百五十万以上的变化道道痕。

方源变作匪猴,这些变化道痕绝大多数都转变成力道道痕。

一百五十万左右的力道道痕!

单凭道痕本身的加持,方源就能抵挡住龙公的杀招。

龙公的龙爪击打在方源的左臂上,就好像是击打在一座山峦。

龙公的心中刚刚升腾起不妙的念头,方源的右拳已经破风直捣。

轰!

一声巨响,龙公感觉自己就好像是被惊涛骇浪掀起的纸质小船,又仿佛是被山峦直接撞过来。一股难以想象的巨力,让他以肉眼几乎不可察觉的速度,暴射而去。

龙公整个人如同一颗流星,闪电般坠落到地上,轰隆一声,大地狠狠震颤了一下,漫天烟尘卷席而起。

全场死寂!

这一次,惊骇的神情同时涌现在北原、天庭两方的蛊仙脸上。

所有人的目光都集中在方源的身上。

方源变作的匪猴傲立船首,目光淡淡俯视着下方的深坑。

那是龙公刚刚撞出来的巨大深坑,深达三丈!

------------

\end{this_body}

