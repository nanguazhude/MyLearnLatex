\newsection{星宿对无极}    %第九百二十节:星宿对无极

\begin{this_body}

%1
自开战以来,龙公被打飞出去的情景,在场众仙还是头一次见到。

%2
方源不仅挡下了龙公的重击,甚至还反攻一手,直接把他打飞了!

%3
这一幕远远超出所有人的意料极限。

%4
“他居然这么强了!这是何等惊人的成长速度!!”毛里球长着大嘴,下意识地站起身来,不再趴着。

%5
“巨阳仙尊大人所等待的关键人物……的确,他完全当得起巨阳大人的期待!”冰塞川叹息一声,神情复杂。

%6
不知不觉间,秦鼎菱追击方源的速度下降了。

%7
开玩笑!

%8
方源一拳把龙公都打飞出去了,这是什么概念?

%9
秦鼎菱虽然强大,但更有自知之明,深深明白自己和龙公的差距。当初她为了观测到龙公的运势,付出重伤的代价,休养了许多时日才逐渐康复。现在这样一幕,直接让她打消了观测方源运势的任何念头。

%10
秦鼎菱的处境有点尴尬。

%11
她方才一心想要追杀方源,和他交手,现在一看,我的天!这人是何等凶猛!

%12
秦鼎菱干脆停在半途中,根本不敢和方源近战。

%13
方源展现出来的变化道修为,让近战这个战斗方式不但不是弱点,反而是他的强项。

%14
秦鼎菱瞬间打定主意,接下来专心一意地辅佐龙公,再不和方源直接交手。

%15
“这又是一个怪物!”

%16
“没想到我死后的未来,居然出现这样的魔仙,真有意思。”

%17
“走吧,他根本不需要我等的支援。”

%18
北原诸仙们纷纷掉转方向,再次向四面八方扩散开去。

%19
紫薇仙子紧皱眉头,心情十分沉重。北原诸仙四面开花,几乎每一个人都有一条进攻路线,一路奔袭一路破坏,这让天庭如何阻挡?

%20
就算是现在仙墓中苏醒了大量的天庭蛊仙,也需要时间发放仙蛊,根本来不及。

%21
“只能在数处要地进行重点的防范了!”紫薇仙子狠狠咬牙,痛下决心,决定弃车保帅。

%22
烟尘逐渐消散,深坑中,龙公缓缓站起身来。

%23
“你真的是一再让我惊讶,方源。”龙公仰头望着方源,神情平静,语气森然。

%24
他被方源拳头击中的胸膛,只是稍微凹进去一个弧度。但趁着说话的功夫,这个弧度也迅速消失,龙公的伤势彻底痊愈。

%25
他不愧是龙人之祖,一身恢复力惊世骇俗。

%26
方源同样俯视着龙公,目睹了这一幕,神情平静得很。

%27
一切都没有出乎他的意料。

%28
龙公根本没有遭受重创,刚刚那一击声势煊赫,但实际战果却非常微小,顶多是打了龙公一个猝不及防而已。

%29
毕竟当时,方源并没有动用任何的仙道杀招,他抡起来的拳头,也只是单纯的力道加持而已。

%30
一百多万的力道道痕加持,方源变作的匪猴力气大得惊天动地,龙公是万万不及。但龙公本身有着种种仙道杀招防御,整个人浑若一块坚硬的钢铁,钢铁被击飞出去,掉落在地,本身是无恙的。

%31
万年斗飞车下,光阴河水滔滔作响,这座宙道仙蛊屋忽然启动,仿佛是一柄巨大飞剑,划破长空,留下一道灿烂的银虹,继续向仙墓飞刺。

%32
龙公冷哼一声,眼中厉芒狂闪,浓郁的杀意不经任何掩饰,宛若恶浪汹涌弥漫。

%33
他陷入了被动,方源这一进攻,他就被牵着鼻子走,这种感觉真是很难受。

%34
元始气墙一破,令方源完全占据主动,目标仍旧锁定仙墓。

%35
龙公刚刚被方源一拳打飞出去,颜面大损,这次他不再做无用功。事实摆在眼前——选择和方源近身战,对他实施斩首战术完全是个笑话。

%36
仙道杀招——气呼山!

%37
轰隆!

%38
一道恢弘的气流迅速凝聚成一座半透明的大山。

%39
大山罩住方圆数百里,重重盖压下来。

%40
方源此刻就在山底,仰头望了一眼,大山镇压向他,仿佛天地都倾倒,一时间避无可避。

%41
方源却也没有躲闪的念头。

%42
他忽然撤销了匪猴变化,然后变作一头雄壮的气罡飞天猪!

%43
气罡飞天猪生活在天空之中,隶属气道,能凌空飞翔。灵活的仿佛飞鱼一般,带着和体型毫不搭配的敏捷,穿梭天罡气墙轻而易举。一个猛子,就能扎进了天罡气墙之中,然后宛若游鱼在海水中一般在气墙中畅游。

%44
方源四蹄踏足在万年斗飞车上,也不抬头,只是低声一吼。

%45
仙道杀招——暴气吼!

%46
这一招同样是气道杀招,却是来源于琅琊派的库藏。

%47
当初琅琊地灵操纵天婆梭罗,对战歧牙猪,就是用过这一招,将这头上古荒兽一招吼爆,惨死当场。

%48
现在方源使用出来,经过全身一百多万的气道道痕增幅,威能直接暴涨一千五百倍以上!

%49
轰!!

%50
气呼山被方源一声直接吼爆,无穷的气浪喧腾开来,一瞬间就形成飓风,狂扫战场。

%51
杀招被破,龙公遭受反噬。

%52
但这位红莲尊者的护道人即便受挫,斗志仍旧昂扬,又使出另一招。

%53
仙道杀招——气流剪!

%54
他伸出双拳,忽然张开十指。

%55
无数淡白气流从他的双手手掌中蜂拥而出,仿佛一道道弧线,斩向方源。

%56
几个呼吸的时间,淡白弧线气流已经成千上万,仿佛暴雨倾盆,又若白蛾狂袭,铺天盖地而来。

%57
每一道气流都锋锐如刀,即便是太古荒兽面对此招,也要皮开肉绽,直至被削尽血肉,成为一尊枯骨。

%58
面对这一招,方源变作的气罡飞天猪也再不实用,他就又换了另外形态。

%59
这一次,他变成了一头鹤。

%60
同样是气道仙兽——一气鹤。

%61
飞鹤如马般大小,神俊至今,鹤颈优雅高傲,鹤翅极其宽大,仿佛两件巨大的披风,铺散在万年斗飞车的甲板上。

%62
方源猛地振翅,鼓动双翼。一时间,双翼上千根羽尖闪闪放光。

%63
下一刻,仙道杀招弹指神通已经酝酿完毕,被方源催使出来。

%64
咻咻咻……

%65
上万道尖锐的声响连绵不绝,无数的气流从羽尖飞射而出,纷纷撞在龙公的气流剪上。

%66
两大群攻之术悍然对撞,绝大多数都相互抵消、崩解,只剩下少部分落到龙公、万年斗飞车之上。

%67
但不管是任何一方,面对这样的零碎攻击,眼皮子都不眨一下。

%68
仙道杀招——流气环!

%69
方源再使出一招,鹤嘴张开,尖啸一声,吐出一口白气。

%70
白气迅速扩散,缠绕在万年斗飞车的车尾,助推后者速度倍增,暴射向前。

%71
龙公神色骤变,暗叫糟糕,连忙紧追不舍。

%72
按照常理而言,流气环乃是气道杀招,来助推宙道仙蛊屋效用不大。但方源动用的此招,本身经过他的改良,已经高达八转。同时又得到海量气道道痕增幅,威能暴涨上千倍,蛮不讲理地推着万年斗飞车往前冲!

%73
方源迅速越过前方的数位北原诸仙,后发先至,抵挡仙墓上空。

%74
俯视仙墓,方源只见仙墓范围很广,青草绿地中立着无数墓碑,墓碑形制各异,有的高耸的金碑,有的是低矮的石刻,有的干脆连墓碑都没有,还有的泥土翻开,露出地下半开的棺材。

%75
一股悠久、沧桑、凝重、肃穆、神圣的浓郁氛围,始终笼罩着仙墓。

%76
方源却是冷冷一笑,又再度变作匪猴,正要对仙墓痛下毒手。

%77
就在这一刻,一缺抱憾亭中星宿仙尊的虚影终于叹息一声,抓起棋盘中的三颗棋子,飞洒出去。

%78
棋子却没有直接攻击方源,而是纷纷落入一座大阵之中。

%79
紫薇仙子正在大阵中枢,执掌星宿棋盘,总揽大局。

%80
这些棋子纷纷涌入星宿棋盘后,紫薇仙子立即感觉到了此阵威能暴涨,突破极限,达到了一种匪夷所思的程度!

%81
紫薇仙子狂喜,立即催动大阵,遥遥对准方源。

%82
一道巨大的星光巨柱陡然罩下,方源和万年斗飞车在星光中消失。

%83
下一刻,又一道星光巨柱在龙公的身后照落,随后方源和万年斗飞车出现在这里。

%84
得到星宿仙尊虚影的帮助之后,天庭的大阵顿时就有了挪移方源、龙公这等级别人物的威能了。

%85
“这是星宿尊者的手段?”方源看清原委,吐出一口浊气,这种异变同样和上一世不同。

%86
“我身上有这么多的道痕,星光巨柱仍旧能强行地挪移我。的确厉害!不过,这个距离恐怕也是极限了吧?”方源眼中寒芒烁烁。

%87
随后,他就看到经过仙墓的其他几位北原蛊仙,也被强行传送走。

%88
只是他们彻底消失在天庭战场,恐怕是被直接送出天庭了。

%89
方源再看其他地方。

%90
一些北原蛊仙已经得手,或是摧毁资源要地,或是击溃仙蛊屋。

%91
他心中顿时又有明悟:“看来大阵虽得星宿仙尊的助力,但威能也是有限的。只能让天庭方面重点防守几处要地。”

%92
即便如此,暂时也够了。

%93
天庭总算维持住了局面,不至于损失惨重。

%94
一缺抱憾亭中,无极魔尊的虚影回首,看向星宿仙尊:“你倒是舍得,一下子就用出了三颗棋子。”

%95
“只是这样一来,这个棋局你就要落入下风了。”说到这里,无极魔尊的虚影微微含笑,再次走回座位,悠然坐下。

%96
“当年我以摧毁天庭进行威胁,逼得你接受这片赌局。一百多万年过去,你我的力量在这棋盘中纵横交错,一直都处于僵持状态。现在我不过失去一颗棋子,你却失去三颗,我现在来攻,你可就麻烦了。”无极魔尊虚影微微一笑,手指落在棋盘一点,再回收时,棋盘上就又出现了一颗全新的棋子。

%97
这块棋子蕴含着无极魔尊虚影的力量,占据的位置正是刚刚星宿仙尊棋子失位的三点之一。

%98
星宿仙尊的虚影淡淡一笑,不以为意地道:“当初,你的本体是想要获得我天庭的天道底蕴,因而和我作赌。这场棋局暗通着你的晚年道场疯魔窟,你赢面越多,获取的天道底蕴就越多,对于疯魔窟那边的演化就越是有利。”

%99
“不错,正是如此。”无极魔尊的虚影承认得很是贪婪,这是阳谋,无须遮掩。

%100
星宿仙尊的虚影再笑:“眼下局势由不得我,就算让你获取一些天道底蕴,又有何妨呢?今时不同往日,你的本体早已寿尽,已经不在了。”

%101
无极魔尊的虚影微微一愣,眼中精芒一闪,恍然:“原来天庭在我的疯魔窟也有了布置。”

%102
星宿仙尊的虚影不吝夸赞地道:“疯魔窟得天独厚,你之后历代的尊者几乎都先后造访,不忍毁去,其中几位甚至都留下道场。在那里必有一场我等之间的较量,不过那是将来的事情了。”

\end{this_body}

