\newsection{仙蛊道可道}    %第一百三十五节:仙蛊道可道

\begin{this_body}

片刻之后,方源控制住藏娇蚌,成为它的新主人。∽↗,

啪啪啪。

他拍拍青年蛊仙的脸颊:“小子,算你识相。”

青年蛊仙神情僵硬,不敢动弹,眼眸中都是羞愤和仇恨。

方源却视若不见。

随后,他跳上藏娇蚌,坐在青年蛊仙曾经的位置上,扬长而去。

临走之前,还留下一句话:“本大人姓名楚瀛,就是我抢了你的藏娇蚌,记得回去告诉你爷爷。”

望着藏娇蚌贝壳合上,投入巨流之中,消失不见,青年蛊仙这才从牙缝中挤出一句话来:“好,楚瀛是么?你放心,你的模样我早就通过蛊虫,告知我爷爷了!”

“少爷,我们接下来怎么办?”两位美人泫然欲泣,紧紧攥着青年蛊仙的衣袖。

青年蛊仙神情一僵,望着周围的空间越发狭小,几股巨流已经开始要转向变化,他的脸色再度阴沉几分。

依凭他的实力,不能闯到这里,全靠刚刚的藏娇蚌。

“放心,虽然藏娇蚌失去了,但我有爷爷给我的游子蛊,可以回到慈母蛊的身边,也就是爷爷那里。”青年蛊仙咬着牙道。

“那真是太好了,我们有救了!”两位女子大喜。

砰砰!

两声闷响猝然发生,两位美貌女子瞪着惊骇欲绝的目光,一位看着胸口的血洞,另一位着望着青年蛊仙:“公子,你……”

青年蛊仙脸色阴沉如水:“谁让你们目睹了整个过程,哼!”

旋即。他的脸上又涌现出痛惜的神色。

他伸出手来,抚摸两位美人的细嫩脸庞:“可惜了这对脸蛋儿。是挺漂亮。要怪就怪那个楚瀛罢。”

说完,他便将这两具尸体随手推入一旁的巨流当中。

没有蛊仙的防护手段。凡人尸躯在刹那间,就被巨流冲刷成渣。

眼看着这片空间已经面临崩溃,青年蛊仙只要咬着牙关,催动了游子仙蛊。

这片乱流海域,各种道痕杂乱不堪,宇道也是如此,蛊仙不可以随意动用宇道仙蛊,最安全的方法就是徒步跋涉。

但青年蛊仙此刻运用的仙蛊,却不是宇道。而是情道。

情道这个流派,脱胎于智道。智道三元,就是念、意、情。

游子蛊、慈母蛊,乃是情道中赫赫有名的一套仙蛊。

乱流海域形成之时,情道流派还未正式开创来,所以此刻,青年蛊仙可以安全地运用游子蛊,并无任何危险。

青年蛊仙任凭游子蛊的力量,拖拽自己。

他像是进入了一个漫长的河流之中。整个人混混沌沌,思维迷糊。大约过了半柱香的功夫,他清醒过来,发现自己已经回到了任修平的福地之中。

“爷爷。我……”青年蛊仙面露惶恐不安之色。

“哼,趁着爷爷我坐落仙窍,吸收天地二气。稳固福地,你居然敢偷偷跑出去玩耍。早知如此。我就不把藏娇蚌送给你护身了。”任修平冷哼道。

“爷爷,藏娇蚌被人夺了。”青年蛊仙委屈地道。“我遇到一个魔头,他抢了我的藏娇蚌,还杀了我那两位侍女!”

“哦?”任修平扬起眉头,“你详细说说。”

青年蛊仙便添油加醋,说了一通。

“楚瀛……”任修平口中咀嚼着这个名字,皱着眉头。

这个名号,他还是第一次听闻。区区两个凡人女子算不了什么。七转蛊仙主动为难自己,抢夺了藏娇蚌,这的确是魔道行径。但为什么,他要特意留下姓名呢?

难道他并非楚瀛,而是其他人,故意栽赃陷害?

还是有其他什么图谋?

总之,任修平将楚瀛这个名号,暗暗记在心中。

他板着脸,继续教训孙子:“现在你知道蛊仙界的残酷了吧。一直以来,你跟在我的身边,见到的都是与我交好的仙友,他们对你态度宽容,甚至因为我而巴结你。这次,你要牢牢吸收教训,罚你闭关修行十年。十年之内,都要待在我的仙窍中,不得外出一步。”

“什么?”青年蛊仙大惊。

“滚。”任修平一挥袖,青年蛊仙视野骤然大变,已经陷入一座山洞之中。

“玉不琢不成器啊,孙子。不能再让你胡闹下去了,否则,就算有我的庇护,这个东海也有大把不买你爷爷账的人呐。至于这个楚瀛,爷爷早晚会收拾他!”念及于此,任修平的眼中闪过一抹狠意。

……

“既然要交好庙明神,不妨就得罪一下任修平吧。”

“话说回来,这藏娇蚌还挺好用的,可以随身携带。只要不是特别强大的乱流,完全可以在里面歇息,以逸待劳。”

方源藏身在藏娇蚌中,让后者顺着乱流前行。

就这样,距离目的地越来越近。

但是在乱流海域中,距离并不代表行程。

夺走藏娇蚌之后,方源开始运气低迷,连续碰到好几道乱流,方向错乱。让他不得不走了许多冤枉路。

辗转了一大圈之后,他在一天之后,才到达了乱流海域的中心地带。

这里的乱流,规模更加巨大,方源进入其中,宛若蝼蚁堕落江河。

寻常的乱流,可能会每隔一段时间,改变流向,或者位置。

但乱流海域中央地带的乱流,因为规模太过庞大,反而挪移不易,虽然每时每刻都在改变,但程度很小,所以相对稳定。

在这些乱流中,夹杂着许多气泡。

气泡有大有小,小的如人头,大的堪比山岳。

这些气泡也是大战中形成,不知道是哪位大能的仙道杀招。

气泡当中,往往有不少残骸或者遗址。更有许多气泡里面,包含着海水和小型海岛。

“找到了,就是这个!”方源寻觅到目标,艰难地接近,然后一下子挤了进去。

气泡薄膜被他挤破,但旋即就愈合起来,只是漏进去了一些流水,微不足道。

方源视野大变。

一片平静的绿色海面,波澜不惊。

天空是蛋白之色,纯洁一片。

简直是另外一个世界!

海面中央,有一个小岛,小岛上长着野草,草丛中耸立着许多高大的石柱。

有人说,气泡里的世界,其实就是乱流海域曾经的一角。只是大战之后,空间被撕裂成一块块。

也有人说,这是某位大能的仙窍碎片世界,因为某个玄妙无比的仙道杀招,最终产生了这些气泡。

气泡的成因,方源没兴趣关注。

他来到石柱之间,取出飞剑仙蛊在手上。

似乎是感知到了飞剑仙蛊上的信道印记,这些石柱开始散发出赤红的光辉,随后不久,它们开始自行缓缓地移动起来。

当它们缓缓停下的时候,一片水缸大小的空间陡然敞开,掉落下一具骸骨。

方源小心翼翼,走近查看。

这具骸骨上面,印刻着丰富的骨道道痕,可算是一副准八转的仙材。

骸骨的右手中,拿捏着一个墨绿气团。

方源取出这个墨绿气团,参看一番后,轻轻捏破气团,露出里面沉眠着的几只蛊虫。

一共两只仙蛊,五只凡蛊。

但可惜的是,绝大多数都已经死了,只是一具躯壳。风一吹,化为飞灰。

不知道过去了多少年,保管蛊虫的手法也不是特别好,难敌时间的冲刷。

最终,只剩下一只六转仙蛊。

方源连忙挽救,勉强吊住了一丝生机。

“这是什么蛊虫?”方源纳闷,他不认识。

又翻看骸骨,再无任何类似的气团了。

“看来这就是所谓的信道传承了。”方源叹了一口气。

换做寻常蛊仙,当是欢喜一场。但是刚刚取得了黑凡真传的方源,却感觉不佳。

再仔细检查骸骨,方源发现,这具骨骼的头颅中,还封印着许多念头。

念头已经不全了,幸好方源乃是智道宗师,十分勉强地从中挖掘出了许多信息。

留下这个传承的蛊仙,名讳不可考证。方源只知道他是在激战之后,勉强吊住一口气,进入这个气泡中,布置下的传承。

关于传承,的确是信道传承,但绝大部分的内容都丧失了。

不过方源并非毫无收获。

他知道了自己救下的这只仙蛊的名号道可道。

道可道仙蛊,信道的侦查仙蛊。作用似乎很是鸡肋,就是帮助蛊仙数清楚目标身上的道痕数目和种类。

至于这具骸骨的来历,却并非传承的布置者,而是被信道蛊仙斩杀的仇敌。

念头中,这位留下传承的信道蛊仙,告诉继承者,可以将这具骸骨当做仙材直接用了。因为这位仇敌身上的道痕,都集中在这具骸骨上。

这点倒让方源感到一丝惊奇。

因为蛊仙死后,道痕往往集中在仙窍之中。除非是仙僵,仙窍没了,道痕才留在体内。

“这位信道蛊仙,显然有着不一样的手段,能够将蛊仙的道痕都集中在骨骼上,保存下来。可惜传承内容缺失了太多,根本无法还原了。”

方源感到十分遗憾。

他本来还寄托希望,可以借助这份信道传承,让自己拥有解除盟约的本领。

可惜,世间之事不如意者十之**。

他最终得到的,只是一只道可道仙蛊而已。

将仙蛊收入仙窍,方源就准备离开。

“凶手,哪里走!”就在这时,一道血红的身影,冲进了这个巨大的气泡当中。

赤红的双眼,恶狠狠地盯着方源!

\end{this_body}

