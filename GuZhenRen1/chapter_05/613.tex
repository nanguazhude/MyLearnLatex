\newsection{飞刃摄敌}    %第六百一十六节:飞刃摄敌

\begin{this_body}

%1
一番苦战,斗智斗力,方源终于突破天庭的埋伏,逃进光阴长河之中。

%2
一出大阵,方源就感觉到维持逆流护身印艰难了数倍,他连忙停用此招,换成宙道杀招冬裘。

%3
仙道杀招见面曾相识!

%4
下一刻,方源变作了太古年兽。

%5
因为是在光阴长河中变化,变化道也受到大环境的压制。虽然成功催动了此招,但是仙元的消耗是正常情况下的数十倍!

%6
幸好见面曾相识的核心,乃是态度蛊,只需消耗心力。因此整体上,见面曾相识对于仙元的消耗并不多,方源还勉强能够承受。

%7
扑通。

%8
方源变作一条太古年蛇,钻入光阴长河中,迅速游走。

%9
“终于是突出重围,撞破了宙道大阵。”方源心中一阵庆幸。

%10
刚刚的情况,非常危险,时间拖得越久,天庭的防守就越加严密,方源生还的可能性就越低。

%11
这一次能够突围,有两大因素最为重要。

%12
首先是方源借助宙道分身利用智慧光晕,勘破了宙道大阵。利用智慧光晕,方源的智道水准,远超天庭一方的想象,这才打了对方一个猝不及防。

%13
其次是五指拳心剑。方源修为达到八转,有了自己的白荔仙元之后,就能够运用慧剑仙蛊。这只仙蛊本来就是五指拳心剑的核心之一,因此方源能够将五指拳心剑推上第四剑。这是目前的能力极限,要发出第五剑,方源还需要其他的核心剑道仙蛊。

%14
“八转仙元消耗很多了。”方源视察仙窍,仙元储备已经降低到了警戒线附近。

%15
尽管方源的至尊仙窍,时间流速快,积攒出白荔仙元的效率很高。但是此番接连大战,尤其是刚刚一战,和威猛老者厉煌苦战,突破天庭的宙道大阵,消耗极多。

%16
“先撤为妙!”这种情况下,方源脑海中只有这个念头。

%17
这是最明智的抉择,死磕下去,绝对是愚蠢的。

%18
“方源,哪里走!”背后忽然传出吼声。

%19
方源转头一看,顿时心头一跳。

%20
只见三座仙蛊屋,联袂追来,气势汹汹。

%21
左边那座,便是恒舟,船楼高耸,乘风破浪。甲板上,八转蛊仙清夜,七转蛊仙四旬子等人众志成城,正气凛然。

%22
中间这座,乃是三秋黄鹤台。清秀飘逸,橙黄的屋檐飞角,如鹤展翅。仙蛊屋表明又时刻笼罩着三层秋日的光晕,玄妙非凡。

%23
右边那座,则是鲨流撬。巨撬雪白如玉,站着一位八转蛊仙,身着湛蓝星甲,容貌粗犷。撬前有着七头巨鲨,森白锯齿,拖拽着巨撬,奔袭如飞。

%24
“三座七转仙蛊屋,至少三位八转蛊仙,七转强者数量不明……”方源一颗心不断地往下沉。这一刻,他彻底感受到了天庭要铲除他的巨大决心!

%25
方源尽管变作太古年蛇,但速度不成。这一次仙蛊屋雏形也毁了,令他难以隐藏。

%26
双方距离不断缩短,三座仙蛊屋各有攻势,遥遥击来。

%27
方源催动秋毫杀招,背后犹如长着眼睛,灵活躲闪。

%28
实在躲闪不开,就用冬裘杀招硬抗。

%29
冬裘杀招在光阴长河的增幅下,还要计算上方源一身宙道道痕,表现出惊人的防御威能,虽不及逆流护身印的巧妙,但也能和其媲美。

%30
“这样下去,可不妙!”局势不断恶化,方源苦思对策。

%31
落魄印酝酿的时间有些长,放在光阴长河中,会更甚一筹,并且威能还要受到削减。再加上天庭一方定然是满心戒备,指望落魄印改变局势,希望不大。

%32
万我、万蛟、力道大手印、阎帝等等,亦是如此。

%33
方源可以依仗的,唯有春剪、夏扇等宙道手段了。

%34
但是追杀方源的这三座仙蛊屋,一个个坚如堡垒,春剪、夏扇一时间都奈何不得。

%35
纠缠中,方源仙元储备迅速下降,情势越来越危险。

%36
天庭一方一路追杀方源,士气如虹,神情振奋。

%37
忽然,前方的河面上,发生异变,出现了一道道的巨大喷泉。

%38
“糟糕,这是突泉河段!”

%39
“这魔头精通运道,运势惊人,极难斩杀。果不其然,在这里出现了变故!”

%40
“都跟上,绝不能让他逃了。”

%41
天庭一方纷纷大吼,脸上均流露出紧张神色。

%42
方源一头扎进突泉河段之中。

%43
突泉非常危险,有着宙道威能。规模若大一些,完全能够危急八转蛊仙的性命。

%44
寻常时候,方源定是避之不及,但这个时候,突泉河段却成了他逃生的希望。

%45
太古年蛇在河中蜿蜒游走,突泉在方源的身边一次次爆发,带给他的实际干扰并不多。

%46
方源拥有察运仙蛊,至尊仙体道痕又不互斥,他可以明显地感受到,自己的运势在不断地剧烈下滑。

%47
运道平时不显威能,这一刻终于展现出了玄妙。

%48
天庭一方的三座仙蛊屋,追击的极其辛苦。

%49
刚进入突泉河段,恒舟和三秋黄鹤台就几乎同时中招,被突泉顶飞上去。

%50
好不容易俯冲下来,鲨流撬也跟着中招。

%51
方源遇到的往往都是六转层次、七转层次的中小型突泉,天庭一方却是连连遭遇八转层次的大突泉。

%52
方源虽然磕磕碰碰,多少被突泉干扰,但天庭一方却是像虚弱的病人来攀登崇山峻岭。两者之间的遭遇,形成明显的对比,让天庭蛊仙几乎脸色如锅底般黑,心口烦闷,几乎气得要吐血。

%53
“看来这一次,凤九歌并不在追杀的队伍里。”方源心中渐渐了然。

%54
琅琊攻防战,方源见识到了凤九歌的运势,十分强劲,自己都压制不了他。

%55
若是他在这里,天庭一方的遭遇就绝不会如此狼狈。

%56
一路有惊无险,方源冲出突泉河段。

%57
在他身后,鲨流撬、三秋黄鹤台已经被他甩得远远,唯有恒舟勉强跟得上,不久也冲出了突泉河段。

%58
没有了阻碍,恒舟速度暴涨,迅速拉近和方源的距离。

%59
“这样下去根本走不脱!”方源眼中闪过一抹电芒,决心一定,忽然停下转身对战。

%60
见方源不逃,恒舟上一干蛊仙均是大喜过望,但下一刻他们又露出满脸的警惕戒备之色。

%61
旬果子更是叫道:“大家小心,魔头不退而战,定然是有什么阴谋手段!”

%62
她话语刚落,方源的杀招已经催动出来。

%63
仙道杀招光阴飞刃!

%64
这一招还只是七转层次,但受到方源一身宙道道痕的增幅,威能爆照上百倍,同时又在光阴长河之中,立刻飙升到近乎八转的层次。

%65
一瞬间,为首的八转蛊仙清夜感受到致命的威胁,浑身上下汗毛炸立,心脏狠狠收缩,连忙躲闪!

%66
但光阴飞刃的速度之快,比五指拳心剑还要更胜一筹,方源刚刚发出,飞刃已经来到清夜的面前。

%67
恒舟上当然有着防御,但光阴飞刃如透薄纸也似,轻易洞穿!

%68
“这一招不是早已经失传了吗?他怎么会这一招!难道我要死在这里?!”刹那间,清夜满脸惊骇之色,陷入到深深的绝望当中。

%69
“让我来!”危难关头,旬果子挺身而出,站到清夜面前。

%70
呃!

%71
旬果子浑身剧烈一颤,被光阴飞刃射中,双眼瞪大无关,扑通一下栽倒在甲板上。

%72
“呼呼呼!”清夜大喘粗气,满身冷汗,侥幸拾得一命。

%73
“旬果子仙友……”清夜脸色苍白,流露出深切的感动之色。

%74
其余三旬子相互对视一眼,神情从容淡定。

%75
他们觉得:旬果子以身挡刃,替清夜去死,换得这位八转蛊仙好大的人情,绝对有赚无赔。

%76
“清夜大人,无须介怀。我等拥有独到手段,可以将四妹再次复活。”上旬子微笑道,故意示以真诚。

%77
哪知清夜却缓缓摇头,面色沉重至极地道:“你们的手段,我多少能揣测一些奥妙。恐怕这次你们的杀招要无效了。”

%78
三旬子惊愕,急问:“大人何意?还请详说。”

%79
“这是薄家的杀招,当年令整个中洲胆寒,名为光阴飞刃。”清夜说着,神情复杂,难掩余悸。

%80
“什么?”

%81
“竟是此招!”

%82
三旬子纷纷脸色剧变,一个个浑身颤抖起来,惊慌失措,再无之前的云淡风轻。

%83
“不,不可能,这一招不是明明已经失传了?”上旬子摇头不止,不相信清夜的话。

%84
“但事实就是如此,停止追击。”清夜深呼吸一口气,面色惨然。

%85
“大人!”其余七转蛊仙顿时大叫,明明方源就在眼前,这次多好的机会,杀了这魔头,可是泼天的大功劳!反之,若是放走了他,众仙身上的罪责也绝对不小。

%86
但清夜却是决心已定。

%87
“三座仙蛊屋中,恒舟速度最快,但防御较为薄弱。你们也看到了,抵御不住方源的光阴飞刃!”

%88
“我们追下去,只会让方源一一点杀了我们,反倒是将恒舟送给他去。还有一点,我的仙窍已是空窍,但你们却有着一身积累,万万不能滋养了魔头。”

%89
若是此刻,除了恒舟之外,还有另外的仙蛊屋,清夜或许还会继续追杀。

%90
但现在这种情况,追杀上去,根本没有一丝胜利的希望。

%91
ps:今天的微・信・公・众・号上,会有一篇番外,名为剑下败将厉煌传。

\end{this_body}

