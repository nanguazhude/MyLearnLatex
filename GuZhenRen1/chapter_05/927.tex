\newsection{紫薇幽变}    %第九百三十一节:紫薇幽变

\begin{this_body}

方源的战斗体系有几个呢?

就龙公目前观察来看,方源真正成熟的战斗体系只有两个。

一个是宙道体系,一个是剑道体系。

宙道体系是最全面的体系,拥有三息后现、年兽召来、万年斗飞车、春剪夏扇秋毫冬裘,以及春耕夏耘秋收冬藏等等杀招,覆盖了战斗、经营两大领域。王牌手段有两个,一个是缩时杀招搭配的光阴飞刃,另一个则是春秋蝉为核心的春秋必成!

而剑道体系则有上古剑龙变、剑浪三叠、金丝剑、暗歧杀、万念剑瀑、剑客等手段,底牌手段则是五指拳心剑。这套体系强于攻伐,在攻击一端,还要胜过宙道体系,尤其是方源改良了五指拳心剑。但在其他方面却是乏善可陈,移动、治疗等方面更被龙公的体系凌驾。这就形成了短板。在龙公这等战斗经验极端丰富的强敌面前,这些短板就成了弱点,拖累了整个剑道体系的攻伐威效。

除了这两个体系之外,方源的气道也有不少手段,不过在龙公看来,还都不全面。气海无量杀招明显是王牌,强大到可以破解元始气墙,不过使用了一次后就再没有用过,目前看来再用的可能性较小。

魂道方面,龙公知道方源定有丰富的手段,王牌杀招毫无疑问就是落魄印。但天庭和龙公在这方面,早已有了充分准备。方源开战至今都没有动用任何魂道手段,也算是明智之举。

除了这些之外,逆流护身印乃是律道,万我是人道,力道大手印属于力道,见面曾相识是变化道,引魂入梦、纯梦求真变、梦中换魂是梦道杀招,这些手段虽然厉害,但都是独立的,单方面的,如同一颗颗的大珍珠。而成熟的战斗体系则是珍珠项链,一个个手段相辅相成,如同小珍珠链接起来,从总体上的价值而言,反而是后者更优秀一些。

对付世间绝大多数的蛊仙,方源可以凭借大珍珠改变战局,奠定胜利。但是在龙公的面前,方源整体上的劣势却会显露而出。

“方源,你欠缺的仍旧是时间啊。相信你已经早有觉察了吧,可惜,一套成熟的体系至少需要数百年的积累!”龙公俯视方源,目光凌厉如刀,“就算你动用种种前人遗藏催熟,也仍旧不能够纯粹完美。”

他催动回旋龙牙杀招,遥攻方源,皆被逆流护身印所挡。

但龙公却不气馁,逆流护身印虽然十分厉害,但缺乏一个成熟的体系来支撑它,弥补它的弊端。逆流护身印最大的弊端就是要消耗逆流河水,尽管方源重生之后全力节省,但又怎能耐得住如此剧烈的损耗呢?

逆流护身印乃是利用逆流河为主体的仙道杀招,防御力极其惊人,并且能够逆反几乎所有的外来攻势。此招乃是方源开创,可见他的滔天才情。然而可惜的是,方源欠缺以这张王牌的稳定战斗体系。比如提前打击,减少外来攻势的辅助防御杀招,又比如增加逆反回去的威能的杀招,再比如治疗逆流河本身的手段。

打个不恰当的比喻,逆流护身印就仿佛是一道味道鲜美绝世的鱼。虽然生吃的味道也非常鲜美,但若是搭配盐,它可以变成咸鱼干,可以长久存放。若是搭配糖、姜、葱、蒜或者其他辅助食材,它能够做出各式各样的美味佳肴,更加耐人寻味。

与天庭相比,与龙公相论,方源的积累还是太不足够了。

他不可避免的陷入了困境。

龙公一次次攻势,顶着逆流护身印的反噬,强行削弱逆流河储备。他的治疗手段亦非常了得。

而凤九歌的命运歌,更是始终笼罩着方源。命运歌忽强忽弱,激发着逆流护身印的反攻。歌声之玄妙,赫然还在逆流护身印之上,即便有印加身,方源仍旧受到命运歌的削弱,只是削弱的程度要小许多。

更糟糕的是,命运歌不仅是削弱方源,还在帮助龙公。龙公战力飙升,越战越强,空中呼啸连连,完全处于上风,攻势宛若狂澜,极其威猛!

方源撤销太古剑龙变化,再度恢复人身。剑龙躯壳庞大,逆流护身印覆盖的话,遭受攻势更多,消耗更快,反不如人身安稳。

利用纯梦求真体,方源不断反击。

但很快,这些纯梦求真体也被莫名的针对,一个个碎裂自爆,化为梦境。随后梦境又转变成了新的纯梦求真体,为天庭作战。

是凤金煌在出手!

紫薇仙子通过中天门,不仅接回了凤九歌,自然还有凤金煌。有她在,方源的梦道手段就会受到极大的克制。

果不其然,方源此刻的梦道手段都被针对。

“凤金煌!”方源暗中皱眉,“上一世就是她凭借仙阵之助,施展出了仙级梦道杀招,坏我好事。如今在天庭,又是凭借仙阵还是仙蛊屋来针对我?”

方源很想找到准确的位置,对凤金煌下手,但是现在他却是身不由己。

他连监天塔都无法接近,并且距离越来越远。

皆因对付他的是龙公和凤九歌!

新老护道人在这一刻联手!

至于诛魔榜中的方正、秦鼎菱,已经插不上手,开始对付漫天作乱的年兽和剑客。

一位位天庭主力也从毛脚山战场,通过中天门,回到了天庭,参与清缴的工作。

年兽和剑客不是他们的对手,纷纷败北。

劫运坛是被天庭主力围攻的重点,更有人飞身赶回监天塔内,七次郎行动不便,处境十分危险。

“不妙!天庭主力重新撤回,岂不是说不败福地已经失去了作用?天庭拥有了足够的成功道痕,完全修复宿命只在顷刻之间。”方源面沉如水,坏消息一个接一个,情势变得极其恶劣和凶险。

“紫薇仙子,速速催动一视同仁杀招,他的逆流护身印快碎了!”龙公传音过来,催促道。

“是!”此刻,紫薇仙子已然稳定了伤势,能够再战。她催动仙蛊屋,飞升上空,视野中囊括龙公、方源还有凤九歌等人。

仙道杀招——一视同仁。

“一视同仁已经成功,方源已经在劫难逃!”紫薇仙子传讯回去。

“做得好,紫薇。”龙公发出一声大笑,攻势又狂暴了一成。

方源只得节节败退。

“没有用了,方源,再如何坚持,你也将一败涂地。”紫薇仙子笑了笑。

但下一刻,她却听到一个女声:“是吗?”

“谁在说话?”紫薇仙子震惊,这声音是如此的熟悉和陌生,让她大感意外,她从未侦查到身边有什么其他蛊仙的存在。

“你在找什么?我就是你啊。”紫薇仙子惊悚万分,她发现竟然是自己在自问自答。

怎么回事?

我竟然不受控制了!

魂魄,是魂魄出现了问题!

陡然间,她的双瞳中幽光大盛,迅速蔓延,竟将她的全部眼白覆盖,转化为一片幽深的黑。

尽管紫薇仙子极力抵抗和挣扎,但却无能为力。在最后的一丝清明中,她陡然间明白过来:原来她是被魔尊幽魂掌控,每一次搜魂其实都是对她的魂魄进行了一场改造。魔尊幽魂的手段太过高明,根本无从发觉。她早就被魔尊幽魂的力量侵蚀、渗透,中洲炼蛊大会之前所做的种种决断,亦有魔尊幽魂的暗中参与。

只是初期,魔尊幽魂暗中干扰的很少。发展到中期,这种暗中影响越来越多。到了现在,他甚至能给紫薇仙子造成某种幻觉。

明明刚刚紫薇仙子并未施展出一视同仁,却认为自己已经施展成功,还通报了龙公!

“或许上一世,我也是这样被魔尊幽魂侵蚀。虽然有一视同仁,但却没有真正的催动出来。”这是紫薇仙子最后的一个清醒的念头。

下一刻,她缓缓抬头,睁着满是黑光的双眸,静静地看着战场,嘴角微翘,勾勒出一抹极其阴森的笑容。

她一边维持着中天门,继续开放,一边遮掩身形,悄然来到镇魂大殿。

对于这里的布置,她最是清楚不过了。

她迅速出手,直接清除种种束缚,将魔尊幽魂重归自由。

“紫薇,恭迎主上!”她双膝跪地,拜倒下来,再无一丝天庭成员的骄傲,满脸都是对魔尊幽魂的尊崇。

魔尊幽魂虚弱至极,化为一颗半透明的魂球,缓缓漂浮到紫薇仙子的面前:“走吧,让我们出去,先看一场好戏。”

紫薇仙子双手捧着魂球,走出镇魂殿。

此刻,武庸、吴帅等人的身影,也出现在了天庭战场中。

“怎么回事?紫薇仙子呢!”龙公大怒,中天门还在催发,竟然将这些敌人也传送过来。

龙公联系不上紫薇仙子,心中一沉,就想要直接击溃中天门。但这个时候,方源却是再次扑向监天塔。

龙公气得怒发冲冠,只得拦截方源,放弃原先的打算。

中天门中一位位蛊仙走进来。

“追随武庸大人!”

“没想到天庭竟是这般模样。”

“杀杀杀!”

“监天塔就在那里,跟我来!”

记住手机版网址:m.

\end{this_body}

