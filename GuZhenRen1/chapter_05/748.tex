\newsection{退一步海阔天空}    %第七百五十节:退一步海阔天空

\begin{this_body}

碧空如洗,白云若雪。

秋日的草原,风声呼啸。阳光下,草地起伏,掀起一阵阵金色的浪潮。

在这浪潮的簇拥之下,坐落着一座庞大的漆黑城池。

若从高空鸟瞰,这座城池布局宏大,结构严谨。

外围的城墙,好似用一个个巨大的石块筑磊,漆黑如墨,石头表面饱经雨打风吹,刻印着时光的沧桑痕迹。

整个城墙是正六边形,雄踞在草原上。城墙周长超过两百里,高近十丈,厚亦有数丈,端的雄伟高大。常人站在草原上,很远就能看到一头漆黑的巨兽,匍匐在茂盛的草原中。站到近处,须得仰望高耸的城墙和巨大的城门。

城墙的每一个边角,都设有塔楼,用于射箭的孔洞密密麻麻。

每隔数里,都有宽阔的登城马道。

城墙中还有十分隐秘的藏兵洞。

对于凡人而言,单是城墙就固若金汤。在北原的王庭之争中,这雄伟的城墙也十数次抵挡住了蛊师大军,从未倒塌。

越过城墙,便是城内的繁华世界。

在这座城市,有超过八十万的人口,各色蛊师。

城内街道纵横,店铺林立,宫殿座座,房屋栋栋。车水马龙,川流不息。人声鼎沸,繁华至极。

一个特殊的情况是——城市人口中,墨人牢牢占据绝大多数。

走在大街小巷中,人族反而很稀少,随处可见的是墨人。

这是一座属于墨人的城池。

在北原,乃至五域都独树一帜!

这就是墨人城!

轰隆隆……

沉闷的巨响声陡然传来,随后剧烈的地震接踵而至,一下子便打破了城中的安宁和秩序。

“怎么回事?”

“是地裂了吗?从近几年开始,就有传闻,说大地会陡然裂开,形成巨大的地沟,可深达地心!”

“救命啊!”

“妈妈……”

一时间,城内大乱,无数凡人四处奔走,惶急不已,鸡飞狗跳。

烟尘翻腾中,整座墨人城的多数房屋、城墙都散发出一股光晕。

随后,墨人城缓缓升空,竟然向高空攀升而去。

“我的妈呀!墨人城飞、飞起来了?!”

“难道传闻是真的?这座墨人城竟然真的是一座凡蛊屋?”

“我的天!这么大的凡蛊屋,简直是前所未闻啊!”

凡人们万分震惊,骇然不已。

就在这时,一记仙道杀招飞出,散成漫天的绿光,将整座墨人城都笼罩在内。

“睡吧,先睡上一会儿吧。”方源淡淡微笑,悬浮在云端。

在他身边站着一位墨人男仙。

他便是墨人城主,北原墨人之王,六转蛊仙墨坦桑。

他也是墨人城仅有的两位蛊仙之一了。

在方源的智道仙招下,墨人城的凡人们都陷入了深深的沉眠当中,整个城池一下子就安静下来。

墨人城越缩越小,片刻后,顺利地钻入到方源的至尊仙窍当中。

“这座墨人城有人口八十多万,由当年的墨人传奇蛊仙一言仙创建,一直发展到今天。”

“中洲有人族的第一都市帝君城,我这座墨人城就相当于小半个帝君城了。”

方源招揽了墨人蛊仙,当然不会放过这片墨人的最大基业。

“从今以后,你仍旧是墨人城城主。尽力发展,栽培出更多的墨人蛊仙吧。我的至尊仙窍空间足够宽广,能够让你施展才华。”方源笑着对墨坦桑道。

墨坦桑一脸激动之色,当即向方源施跪拜的大礼:“墨坦桑愿全力协助方源大人,尽心竭力地治理好墨人城。”

“嗯,你也进去吧。”方源淡笑。

墨坦桑乖乖地钻入方源的至尊仙窍当中。

他跟随着墨人城,直接落到小西漠里去。

方源早就选好了地址。

墨人城落下,烟尘逐渐消散,沉睡中的人群开始后苏醒。墨坦桑有条不紊地发布命令,主持大局,维系着墨人城的稳定。

墨坦桑的心情其实很复杂。而在面对方源时,他不敢有丝毫的放肆。

墨坦桑早就和方源接触,那个时候的方源才只是六转修为。墨坦桑还卖好太白云生,给后者九云环凡道杀招,刻意结交。

如今太白云生早已不再人世,而方源已有八转修为,恐怖的战力,还有赫赫的战绩都让他兴不起一丝的反抗之心。

方源前来接收墨人城,墨坦桑提前得到通知,整个过程都是积极配合。

方源在心中笑了笑。

“墨坦桑是个人才。”

“我记得五百年前世,这人领袖墨人城,栽培出多位墨人蛊仙,和马鸿运平等合作。趁着五域乱战,不仅拔升自己的修为,而且还将墨人分城扩张了数百座,分布北原!”

“有他执掌墨人城,小西漠中又没有什么敌对势力,发展前景可谓一片光明。”

墨坦桑是一位十足的雄主。不仅眼光独到,果断敢行,而且能屈能伸,不可小视。

只要稍稍给他一点机会,他就能翻身,乃至创造出奇迹。

可以这么讲,他就是一个小一号的武庸。只是碍于异人身份,没有良好的修行环境,一直以来处境都很困顿。

只要给他一个平台,他的发展绝不会像现在这样。

方源并不担心他的成长和壮大,而是乐见其次。

这是强者的器量,也是方源的自信。

就算墨坦桑再如何成长,方源也相信自己能够驾驭他,控制他,让他为自己所用。

其实迁徙墨人城这事,还是有许多风险的。

说不定天庭方面,就会有什么埋伏。或者说紫薇仙子推算出了参战墨人蛊仙的跟脚,就是来自于墨人城,那么必定会想到方源会对墨人城下手。

这样一来,哪怕天庭方面不出手,只要紫薇仙子将这个消息告知长生天,长生天出手方源也会有大麻烦。

但幸运的是,方源获得了墨人城,一切都很顺便,并没有什么幺蛾子出现。

方源本体迅速撤离,而他的宙道分身则还在琅琊地灵身边。

长毛炼道大阵正在运转当中。

琅琊地灵亲自操纵大阵,对方源分身汇报道:“再过一个月的时间,琅琊本源的封印就能全面解除了。”

方源分身微微点头,视察着大阵。

每一个仙窍不管是福地还是冬天,都会有一个仙窍本源。

仙窍本源乃是一个仙窍的根基、根本。

发展仙窍,吞并其他仙窍,都会壮大仙窍本源。

仙窍本源受损,或者被抽调出来,好不容易经营起来的仙窍,就会发展倒退,空间萎缩,元气稀薄,万物凋零。

仙窍本源就像是一口元泉,蛊仙的仙元就是从仙窍本源中凝聚而出的。

琅琊仙窍原本是八转的洞天,仙窍本源乃是乳白色的一团。

长毛老祖死后,琅琊天灵难以抗衡八转灾劫,便耗费了多年光阴,设想出了解决之法。

天灵动用了一记炼道杀招,将琅琊本源封印起来,降落一个层级,变成朱红色的七转本源。

琅琊天灵因此也跌落成了琅琊地灵。

不过正因如此,琅琊仙窍此后面临的灾劫,便都是七转的程度了。

方源如今的八转,只是伪八转。道痕规模是上去了,但是仙窍本源还只是朱红本源。

只有当琅琊本源彻底解封,还原成乳白八转本源,再让方源吞并融合后,才能成为真正的八转蛊仙,仙窍晋升洞天,产出八转白荔仙元。

这种主动封印本源,降低仙窍等级的手段,非常厉害,意义更是极其重大!

和平时期,蛊仙大多数其实都死于灾劫,比如狐仙就是如此,死在魅蓝电影灾劫之下,留下狐仙福地。

若是用这种方法,降低一级,蛊仙就能从容地面对灾劫,还能趁机充分地积累道痕,不怕自己跟不上灾劫的节奏。

方源得到这个手段,非常欢喜,很是夸赞了琅琊地灵一番。

但琅琊地灵却告诉他:“我只是模仿罢了,本体是从盗天魔尊手中得到的这个手段,而盗天魔尊是则从无极魔尊的传承中获取来的。这个杀招的原版乃是律道杀招,名为‘退一步’。后来盗天魔尊大幅度改良,形成宇道杀招,取名为‘海阔天空’。我只不过是将其扭转成了炼道,完全称不上什么改良或者优化,只是粗劣的模仿。所以也无颜面改名字,索性将前两者的名称合起来,称之为——退一步海阔天空。”

方源闻言,感慨不已,没想到这个手段来头很大,源远流长。

仙道杀招——退一步海阔天空!

很明显这招只是辅助,用来经营仙窍,并无丝毫攻伐威能。

但它却是极其实用,价值高得要突破天际!若是流出,必定会造福整个蛊仙界,蛊仙的数量绝对会暴涨数倍,数十倍!

从这个侧面,方源也看出了无极魔尊、盗天魔尊的才情和强大。

有了这个手段,无极魔尊、盗天魔尊完全可以从容地渡劫,不断地积累道痕。

这个手段已经打破了蛊仙修行的常态,一扫蛊仙迫于灾劫压力,疲于奔命的状况,而变得游刃有余,悠然从容。

“那么无极魔尊、盗天魔尊完全可以在成尊之前,就积累出九转道痕了。”方源当时说了这么一句。

但琅琊地灵却告诉他:“主人,你太小瞧历代尊者了。根据盗天魔尊亲口述说,无极魔尊在蛊师的时候,就拥有了蛊仙级数的道痕。等他成就六转,晋升成仙,身上的道痕已经有八转巅峰的层次了。”

------------

\end{this_body}

