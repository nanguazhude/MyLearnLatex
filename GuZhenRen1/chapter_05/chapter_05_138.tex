\newsection{至尊仙窍的发展}    %第一百三十八节:至尊仙窍的发展

\begin{this_body}

%1
数天后,方源在这里等到了花蝶女仙。

%2
“这的确是光阴长河的痕迹,还很新!”花蝶女仙查看之后,非常欢喜。有了这层线索,她完全可以追踪到这股光阴支流。

%3
“楚瀛仙友,实在是麻烦你了。没想到你真的遇到了,为了这条光阴支流,我们数十年搜寻,都没有找到任何有价值的线索!”花蝶女仙语气十分激动,她终于可以向庙明神有所交代。

%4
“仙友还是尽快去追踪的好。报酬我不要了,将来说不定还要麻烦庙明神大人呢。在下告辞。”

%5
方源的善解人意,更让花蝶女仙大生好感。

%6
临走前,她再次关照方源:“要小心。血道魔仙丁齐出现了,还和周礼、汤诵等人展开激战。楚瀛仙友若是遇上,还是暂避风头,没必要趟这趟浑水。”

%7
“仙子的话,在下铭记在心,十分感谢。”方源脸上流露出十分感谢的神色。

%8
花蝶女仙还不知道:就是眼前这人,已经将丁齐、周礼、汤诵都杀光了。只是乱流海域地形复杂,众仙的失踪属于寻常,没人感觉奇怪。

%9
经常有蛊仙,在这里失踪了数个月或者数年,然后脱困而出,忽然出现。

%10
大半个月之后。

%11
方源顺利地回到了琅琊福地。

%12
此次去往东海,虽然没有达到原先的目的,但收获颇丰。

%13
云城密室中,方源静静盘坐着,神念不断催发,在全面地视察自己的至尊仙窍。

%14
至尊仙窍的改变有很多。以前是十分的荒芜,堪称空无一物,现在则有了不少资源。乍看上去,不会那么单调和空洞了。

%15
其实,方源收获的这些资源,摆放在寻常的福地中,说不定还放不下。

%16
但至尊仙窍的空间实在太过广大。分布下去,还是显得地广物稀。

%17
小北原中,已经有大半地域,覆盖了冰霜。另一半地域。则是长着稀疏的绿草。在这草原上,方源放养了一些荒兽,比如很早之前就得到的巨角羊,还有鱼翅狼。

%18
小西漠中,幽火龙蟒的地坑已经扩张到十一个。得益于六十倍的时间流速。再加上东方一族的绝佳豢养方法,使得幽火龙蟒的繁衍速度极高。

%19
在小东海,原本方源人为开辟了几处大湖,里面分别放养了龙鱼群、气泡鱼群、散文鲤等等。现在更多了许多常规资源,都是此次方源在东海剿杀蛊仙,收获不少。这些资源虽然并不是特别珍贵,但胜在数量不少,完全可以在此基础上继续发展。

%20
小中洲中,镜柳已经绵延大片,形成了覆盖方圆百里的大型森林。说起来。方源本是往狐仙福地中移栽的镜柳。辗转之后,这些镜柳移种到了小中洲里。还是因为至尊仙窍中的光阴流逝的速度飞快,方源几乎没有在这方面努力过,单凭镜柳本身的恣意扩张,就达到了如今的规模。

%21
除了镜柳之外,小中洲的地下,还有不少的血芝林,规模比镜柳也只是稍差一筹罢了。

%22
小中洲空间太广,镜柳、血芝林都没有竞争对手,完全是全力发挥。

%23
五域中最繁荣。最是生机勃勃的当属小南疆。

%24
从黑凡洞天中得来的资源,方源大部分都移动到此地。

%25
可是,仍旧没有铺满小南疆。

%26
大量的曲丽木,茶溪随处可见。还有气死鸟群,带给小南疆无限生机。

%27
超过个位数的山峦,如今也出现在小南疆中。

%28
排位第一的便是五光山,华丽多姿。其次是继仙山,石亭分布在山体各处,景致不错。

%29
小南疆中最主要的资源。就是长恨蛛群。如今规模也扩大了数倍。

%30
五域之上,便是九天。

%31
小橙天中,极光曼妙,极光中流光果悬浮于空,正是方源之前以果蕴果的举措。

%32
小黄天中,增添了一条碎金河,规模不大,聊胜于无。

%33
小青天里,有片角落里碧雾缭绕,这是方源买回来的丹青香。还有天晶的蓄养池,暂时搁置在旁边,没什么用。

%34
小蓝天里,大块的云土漂浮,上面种植有星屑草、箭竹林等等,为方源产出星道凡蛊。还有一只上古荒兽落星犬的幼体,算是被方源拯救了,经常在小蓝天中撒欢飞奔。

%35
小紫天中,原本的天晶鹰巢已经被上极天鹰吃光了,只剩下一颗上极天鹰的鸟蛋,还有些许零碎的天晶。这些天晶,自然是方源和琅琊地灵换取来的。

%36
小黑天里,有一株走肉树。

%37
小白天中,则有三头天残犬,还有一群鹰犬,许多斑斓霸王花生长在这里,光照环境已经得到了不少改善。前段时间,方源杀了几位光道蛊仙,所以仙窍中的光道道痕增添了不少。

%38
“可惜市井,没有装到至尊仙窍中带回来。”

%39
方源对乱流海域里的市井,还是念念不忘。

%40
市井虽然被摧毁得不成样子,但方源完全可以动用江山如故,来帮助这座市井复原。

%41
摧毁掉的那些房屋是不能复原的。不过方寸山,绝对可以重现。

%42
原本这座传奇山峰,因为黎山仙子死亡,而折损在了义天山大战之中。但方源现在却看到了重现它的希望。

%43
按照《人祖传》中的记载,方寸山本身就是市井的一部分。

%44
“我若有了市井,完全可以在井底中培养大批的异人族群。可惜了……”

%45
经营仙窍,大抵分三个阶段。

%46
初级阶段,是蛊仙栽培资源,满足仙蛊喂养的需求。

%47
中级阶段,则是喂养仙蛊的基础上,发展其他修行资源,满足蛊仙修行,同时还能贩卖出去,换取仙元石等等。

%48
高级阶段,就是仙窍中豢养异人,甚至人族。这对蛊仙而言,意义重大,好处很多。

%49
举个例子,就像之前。方源收购毛民奴隶,在狐仙福地中为自己炼蛊。这就省去了他大量的时间和精力。

%50
其余异人,小人可以让花草繁盛,石人可以开发地下空间。雪人能产泪冰,在冰天雪地中生存……

%51
若是纯正人族,灵性第一,不断发展繁衍,通过智慧不断创造。大量的蛊方或者杀招,都会随之产生,无需蛊仙亲自推演。

%52
发展更好一些,仙窍中会产生蛊仙,或者人族蛊仙。土生土长,忠心可见。这对仙窍之主而言,无疑是巨大助力。

%53
现阶段,方源的至尊仙窍只是初级阶段。

%54
因为他新添的一些仙蛊,还未完成喂养的需求。态度蛊、慧剑蛊这些八转仙蛊,喂养是个大难题。方源只是制定了计划。初步执行了一部分,还未真正完全解决这些问题呢。

%55
不过,他在中级阶段方面,却有卓越的成就。

%56
因为他从星象福地、东方一族、黑凡洞天等等中,搜刮过来无数资源。很多常规资源,比如星道凡蛊、长恨蛛、曲丽木等等,都是外售出去,赚取利润。

%57
更别谈还有一座落魄谷,胆识蛊一直以来都是方源最大的盈利项目,堪称日进斗金。

%58
马无夜草不肥。人无横财不富。

%59
造成方源至尊仙窍畸形繁荣的原因,正是方源发的一笔笔横财。

%60
也正是这些资源,成了方源修行的最大底气。

%61
没有这些资源,方源就算战力再强。他也会心中发虚。

%62
就目前而言,方源是没办法大规模豢养异人的。

%63
诚然,他可以像在狐仙福地那会儿一样,收购大量异人,放到仙窍中为自己干活。但这绝不是正统的经营之道。

%64
这方面的王道,就是让这些异人自己繁衍生息。和仙窍中的生态完美和谐地融为一体,共生共存。

%65
唯有如此,才是健康发展,长期下来,利益十分可观。

%66
若是收购异人奴隶,单纯让他们干活,方源只会亏本。这点,他在狐仙福地中时,就已经受到教训。方源石巢中,几乎每天都有毛民在炼蛊的过程中死亡。毛民们毫无繁衍生息的时间和机会,数量只会越来越少,唯有方源倒贴钱财,不断收购新的毛民奴隶进行补充。

%67
但是!

%68
若有市井的话,就不同了。

%69
市井是一个与外界隔绝的特殊环境,更有利于培养异人或者人族。并且异人的繁衍和死亡,在井底世界中的种种活动,还会对市井本身有着温养和提升的作用。

%70
此中妙处,不提也罢。

%71
总之,有了市井,方源就能在初级阶段,展开高级阶段的工作,收获更高一级的利益。

%72
这无疑会让他的修行更加顺利,修为提升的速度更快几倍。

%73
针对自己的至尊仙窍,方源考察结束。

%74
有满意的地方,也有令他不满意之处。

%75
让方源满意的是,至尊仙窍中的道痕有了全面性的提升。就比如土道道痕,虽然还是稀缺一些,但比之前的凄惨,无疑要好上许多。

%76
不满意的地方在于,除了小南疆之外,其余各地还是显得空荡荡,资源稀少。至尊仙窍,终究还是太大了。

%77
空间太大,是一种幸福的烦恼。

%78
这就会导致将来发生物种内耗的情况。

%79
比如镜柳,在小中洲发展得很大,扩展很快。但是若有另外一片树种,也发展很快。终究双方接触,必会产生僵持和争斗。

%80
树木之间的纷争,还是比较温和,那么兽群呢?两只凶猛兽群,领地终于衔接到了一处,能不争得头破血流吗?

%81
福地若小,蛊仙就可以及时调控。但是方源福地太大,等到双方内耗的时候,无疑损失就更大了。

%82
要减缓这个情况的发生,只有依靠方源的高瞻远瞩的智慧布局了。

%83
至尊仙窍不好经营。

%84
方源不仅要考虑眼前收益,更要考虑将来的格局。

\end{this_body}

