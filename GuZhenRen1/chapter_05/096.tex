\newsection{上极天鹰}    %第九十六节:上极天鹰

\begin{this_body}

上极天鹰。

这种太古荒兽,其实方源早已久闻大名,只是不太认识天鹰的鸟蛋是什么具体模样。

上极天鹰是宇道太古荒兽,成年的上极天鹰可以自由往来太古九天或者洞天之间。

这项本能极其厉害。

太古九天之外,皆有浓厚至极的天罡气墙相阻。洞天寄托于穹霄之中,若不主动现身,从外界极难察觉。

但上极天鹰只要去过某个地方,就能遵循本能,穿透虚空,重新往返。

它本身就是太古荒兽,八转战力,飞行速度极快,黑凡能够搜寻这样的一颗上极天鹰的鸟蛋,若无奇遇,必定曾耗费了巨大代价。

上极天鹰唯一的缺陷,恐怕就是寿元。

它的寿命,只有八十年。

正常凡人的寿命,都有一百年。

堂堂太古荒兽,上极天鹰的寿命居然比不上凡人,实在是有些掉价。其他的太古荒兽,动辄十几万年,数十万年的生命,上极天鹰根本无法比较。

不过,它也有延续生命的手段。

当它自然老死之前,会产下一颗鸟蛋。

这颗鸟蛋中,蕴藏着它过往的一切记忆,甚至是一身积累的宇道道痕。

当幼鸟破壳而出时,上极天鹰就又重回一世,重新长大,但却有前世记忆和道痕积累。若不出意外,上极天鹰只会越变越强。

当然,鸟蛋和幼鸟期间,无疑是上极天鹰最脆弱的时候。

明白这点后,方源大胆猜测:“这颗上极天鹰的鸟蛋,一定是黑凡故意留下来的!”

为什么要留下这颗蛋?

很显然,上极天鹰恐怕是去过黑凡洞天。它便是通往黑凡洞天的桥梁。

黑凡死后所留下的洞天,黑家蛊仙们一直在探查。就黑城的记忆中,这个探查并无成果。

“黑凡的真传,恐怕是留在他的自家洞天里了。这并不奇怪,因为他可没有什么食道手段。所以将仙蛊留在仙窍中,自行喂养,最为稳妥。”

“而黑凡洞天的灾劫问题,也不存在。根据黑城的记忆。黑凡生前获得过某个太古九天的碎片。如果我料得没错,黑凡一定在死前,将这块太古九天的碎片融入到自家的仙窍里面了。”

“可惜,他虽然至此无灾无劫,但却没有寿蛊。采用的延寿之法也到了极限。所以只能是陨落了。无弹窗,最喜欢这种网站了,一定要好评]”

“但为什么鹰巢中的天鹰鸟蛋,却是死的?”

方源又面临一个难题。

答案可能有两个

第一个自然是这个天鹰鸟蛋,已经自然死亡了。黑凡留下鸟蛋,为了不让鸟蛋自己孵化,他就动用这座天晶鹰巢,进行某个布置。但过了很长一段时间,都没有人来继承,开启天晶鹰巢。如此一来,鸟蛋久而久之,就真的彻底死亡了。

第二个可能。这个上极天鹰的鸟蛋,之所以生机全无,完全是黑凡自己做的。这个鸟蛋并不是真正的死亡,需要特定的手法才能加以解封。

如果是第一个可能,那方源自然捞不到更多的好处,只能是收获一座天晶鹰巢,还有一颗太古荒禽的死蛋。不过,这两者的价值也很高就是了。

站在方源的角度,他当然希望是第二种可能。

但若是第二个可能,那么可以解封鸟蛋的特定手法又在哪里呢?

“假设就是第二种情况。那么我该如何才能解封了这颗死蛋,让它重新散发生机,甚至孵出来呢?”方源陷入沉思。

他思考了半天,毫无头绪。

也难怪。

单纯来看。这颗蛋的确是彻底死亡的。若非如此,方源也不可能一开始,就确定这个事实。

“如果我是黑凡,要留下死蛋,让后人继承我的传承,必定会留下解决此事的方法啊。”方源忽然灵光一闪。

他发现自己。下意识地忽略了一个方法。

那便是黑凡曾经改良,并且特意流传下来的那些炼道手法。

“我正是用了这些方法,才开启了天晶鹰巢。若是我再用这些方法,对这天鹰死蛋施为呢?”方源心中浮现出一个大胆的想法。

但他没有立即付出行动。

因为这只是一个很纯粹的猜测和想法,没有事实作为依据。

方源很谨慎,他开始动用智道手段,深入分析这些个炼道手法,并且重点推算。

他很快发现奇怪之处。

什么奇怪的地方呢?

有一些的炼道杀招,黑凡改良得莫名其妙。原版炼道杀招,比他改良的还要好一些,优良一些。

黑凡如此改良,简直是刻意不讨好。

他究竟想干什么?

方源忽然想到黑城记忆中的一段,有关黑凡的生平记录。

黑凡作为八转蛊仙,在人生晚期,喜欢提携后辈,他就时常教导他们:学习的时候要带着自己的思考,不要盲目崇拜,也不要全部照搬前人经验和成果。只有经过自己思考,学以致用,方能在属于自己的蛊仙之路上走得更远。

想到这里,方源再看这些炼道杀招,又有新的感觉。

他试着将黑凡“改良失败”的那些杀招,从中摘取出来,结果他又有了一个惊人的发现。

“原来这些炼道杀招,并不是改良失败,而是将它们结合在一切,便是一个全新的杀招!”

“黑凡在这些炼道杀招中,故意添加的部分,若单个来看,都是败笔。但结合在一切,这些添加的部分,却能组合起来,形成一个统一的整体!”

更叫方源感到有些意外的是,这个全新杀招,居然属于血道,是一种血道之法。

堂堂八转宙道的黑凡,居然也研究血道?

这其实并不奇怪。

自血海老祖横空出世之后,血道之威,为世人共知。各大超级势力一边大力禁止血道发展,另一边则在背地里偷偷研究。

这个现象连中州十大古派都存在,更何况在北原这种四战之地,崇尚战力的北原蛊仙呢?

对于血道。方源就比炼道自信多了。

他可是血道宗师境界!血道是他的老本行了。

稍加研究,方源就发现:这个黑凡刻意隐藏,说留下来的血炼杀招,果然是针对上极天鹰的死蛋!

但这个血炼杀招。并不全面。

依照方源的见识,和血道造诣,他看得出来,这个血道杀招很多地方意犹未尽,只是一个下部分的残篇。

看出这点。方源并不失望,反而双眼熠熠生辉。

“黑凡应该掌握完整的血炼杀招,他先用上半部分,对这个上极天鹰的死蛋进行处理。然后再留下下半部分,交给后人。”

“但他并没有直接留给后代子孙,而是将这个血道杀招,以一种巧妙的方式,隐藏起来。”

“这样做的目的,第一是为了考验后辈,希望有一个能自我思考的优秀蛊仙来继承他的传承。第二也是因为血道见不得光。堂堂黑凡留下真传。居然是血道手段,说出去恐怕会惹来不少风波。这是黑凡不想看到的。”

方源老谋深算,推己及人,一时间就将黑凡当年的心思,揣摩得八九不离十。

搞清楚这点之后,方源更加确信,这个上极天鹰的死蛋非死,而是被黑凡首先用了特殊手法,进行了处理。

接下来就好办了!

方源已经不是仙僵之躯,取用身上鲜血。轻而易举,不在话下。

不过,就在他要血炼的时候,他忽然停下动作。

“好险!我差点没想到黑凡还有这一层用意……”方源额头隐见冷汗。

黑凡为什么专门用血道方法。留给后人呢?

说起来,不觉得有些奇怪吗?

众所周知,黑凡是宙道大能,到他这种程度,触类旁通。宙道手法肯定也能留下传承。

但他独取血道,而舍弃了自身最擅长的宙道。是为了什么?

这位问题恐怕只有唯一的答案,那就是血道之法,能够提供宙道手段不能的东西。

还能有什么?

身为血道宗师的方源,首先想到的就是血脉认证!

黑凡留下传承,预计的目标,便是黑家后辈。超级家族中的蛊仙留下传承,基本上都是肥水不流外人田,首要的条件就是本家后辈。很少有念头通达的家族蛊仙遗留传承,可以让其他人来继承的。

态度蛊、世代相传的炼道手法,只是预防手段之一二。

血炼之法,进行血脉人证,便是预防手段之三。

“如果不是黑家后辈,身上流动的不是黑家族人的鲜血,那么冒然血炼这块鸟蛋,说不定会招来反噬。更有甚者,是孵化出了上极天鹰,却被这头太古荒兽视若死敌!”

方源想到这一层,额头上的冷汗又多了一些。

尤其是后一种情况,若真的惹出了太古荒兽,八转战力这样的死敌,可不是闹着玩的!

上极天鹰速度极快,方源根本甩不掉。他本身防护又弱,几乎是必死无疑。

“看来只有先从血脉着手,才可再炼鸟蛋。”方源明智收手。

数天之后。

至尊仙窍的某个角落。

黑城披头散发,神情憔悴不堪。

方源操纵一头力道仙僵,缓缓飞来。

黑城像是触电一般,浑身剧烈一抖,满脸都是惊恐之色,叫道:“你,你又想对我做什么?!”

以前,方源对他搜魂,这也就罢了。但最近这些天,方源对他却是动用各种手段,进行“折磨”,让他痛不欲生,苦不堪言。

黑城被方源俘虏,落到这步田地,已经是彻底认命。什么蛊仙的气节,都散尽了。

但他就算咬舌自尽,也是不能的。

方源觉得:黑城身上还有可以利用的价值,就算黑家已经垮了,但黑楼兰可是还在呢。

所以,黑城现在是求生不得求死不能。

“放心,这一次恐怕是最后一次了,我有信心。”方源操纵的力道仙僵,微微而笑道。

黑城几乎都要哭出来,他剧烈挣扎:“你每次都这么说,都说是最后一次!不,不要,不要!啊!”

\end{this_body}

