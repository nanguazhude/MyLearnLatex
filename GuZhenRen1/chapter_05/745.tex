\newsection{评估战果}    %第七百四十八节:评估战果

\begin{this_body}

凤仙太子不愿意和太古寒蛟缠,还是以追杀和纠缠方源为主。

凤仙太子极力向寒潭深处潜游。

但随后,寒潭深处又升起第二条太古寒蛟,将凤仙太子挡下。

“这里的寒潭中,竟然藏有两头太古寒蛟?”凤仙太子和方源都吃惊不已。

凤仙太子狂怒,又无奈,他暂时被困住了。

方源大乐,因此钻入寒潭底部,逃脱了凤仙太子的追杀。

他很快就惊讶的发现,这寒潭底部竟有隐秘暗道,能联通他处。

方源进入暗道,潜游一阵后,又来到一处地下寒潭。

这片寒潭风平浪静,十分广阔,里面生活着大量的寒蛟。

“原来有如此规模的族群,难怪有两头太古寒蛟。哦不,是三头!”方源眼前一亮。

他发现这里还有一头太古寒蛟。

这头太古寒蛟在寒潭的岸边盘踞着,听到上方的激战声响,越发显得焦躁不安。

它晃晃脑袋,低吼一声,终于忍耐不住,一头扎进寒潭中,迅速支援同族的太古寒蛟去了。

“三头太古寒蛟,啧啧。”方源在心中同情了一下凤仙太子。

凤仙太子有麻烦了。

当然,太古寒蛟智力不能和人媲美,杀不死凤仙太子,甚至自己可能会死上一两条。

凤仙太子战力十分强悍,因为他从北原成长起来,一步步成为八转蛊仙,战绩彪赫。

“然而等到凤仙太子甩掉三头太古寒蛟,我早就逃之夭夭了。”

方源上了岸,正要离开,忽然身形微微一滞。

他发现岸上有好多的寒蛟蛋。

每一只寒蛟蛋都有磨盘大小,密密麻麻的挨着,数量成百上千。

“原来刚刚的太古寒蛟是护着蛋,所以不轻易离开。等等,这又是什么?”

方源在蛋的中央,发现了一个低矮的石柱,灰白作色,散发寒意。

“这是八转仙材广寒石?好大的一块,价值非凡啊。”

饶是方源的眼界很高,此刻心中也升腾起一股淡淡的喜悦之情。

至此,他这才真正明白为什么这片寒蛟族群中,有着三头太古寒蛟了。

他立即出手,将广寒石和数千只寒蛟蛋,都统统带走。

“我的运势压制了凤仙太子和这支寒蛟族群,因此被人追杀,反而因祸得福啊!”方源心中感慨。

凤仙太子可不是凤九歌。

没有凤九歌这个护道人的运势抵制,方源的运势立即收到奇效。

凤仙太子纠缠方源不成,反而成全了方源。没有他纠缠住三头太古寒蛟,方源绝没有这等机会来盗取这么大的一笔财富!

这便是运道的厉害之处!!

“真是要感谢凤仙太子的配合呢。”方源暗乐,他钻破地底,然后再催动定仙游,逃之夭夭。

凤仙太子还在和三头太古寒蛟激战。

忽然,三头太古寒蛟得到了其他寒蛟的传讯,知晓蛟蛋被盗,顿时发狂,双目赤红,口中寒息不要本钱似的疯狂喷吐。

强如凤仙太子,一时间也差点被打蒙,不断抵抗,十分狼狈。

“这些蠢蛇怎么回事?忽然间变得如此凶猛?”凤仙太子感到古怪。

“唉!我被这三头畜生拖延了这么久,方源定然是逃走了。我的任务失败了!”凤仙太子心中叹息一声。

没办法,方源根本就不和他交手,太滑溜了。

并且方源又有变化道造诣,在松尾岭这种地方,凤仙太子极其吃亏。

方源利用了此次巨大的地利!

蛊仙交手,只看着彼此的杀招、仙蛊等等,都是蠢材之辈。蛊仙中的精英向来擅长因地制宜,考虑天地环境,揣摩彼此的想法和情绪。

“那就回来吧。方源狡诈,你出现的时机太巧了,恐怕已经算出你的真正身份。”紫薇仙子传讯过来。

凤仙太子沉默,他其实很不甘心,但紫薇仙子说的很对,绝不能小瞧了方源。

当凤仙太子对方源动手时,就意味着自己身份的暴露。

“撤!”凤仙太子咬牙,飞出寒潭。

但三头太古寒蛟却是穷追不舍!

“怎么搞的,这三头蠢蛇疯了?居然还追着我不放!”凤仙太子大感晦气,碰到了三头畜生,根本毫不讲理,一味地纠缠他。

“真以为我对付不了你们?”凤仙太子被激起了心中的火气。

正当他要出手的时候,周围的山岭中接连响起狂暴的兽吼声,一头头太古荒兽显露身形,不怀好意地看着凤仙太子。

它们承认三头太古寒蛟,毕竟做邻居已经很多年了。

但这个人族蛊仙是哪里来的?

居然敢进犯松尾岭?!

凤仙太子脸色顿白,松尾岭可是北原十大凶地之一,八转蛊仙入内探索也得十分谨慎。

“搞不好会引发兽潮!先撤为妙!”凤仙太子心头一沉,连忙撤离。

“哎呀!”他冷不防,被太古寒蛟的寒息喷中,浑身冰霜覆盖,打了个大冷颤。

轰。

一声闷响,他又被太古寒蛟凝造出来的冰球,砸中了后背,打得他向前趔趄,好生狼狈。

三头太古寒蛟在凤仙太子身后穷追猛打,但凤仙太子铁青着脸,再不敢反攻,硬挨了许都攻击,终是吐着血逃出了松尾岭。

三天后。

炼道大阵轰鸣,最后的一丝不利道痕也消除了。

“终于成功了!”琅琊地灵叹息。

方源也吐出一口浊气,终于是安全了。

这三天来,紫薇仙子不断推算,带给方源很大压力。阎帝杀招虽然尽忠职守,但消耗的可是方源的魂魄底蕴。

“比起上一世,我的魂魄底蕴下降得十分厉害。”

“为了尽快炼化这些不利道痕,我也损失了大量的八转仙材。”

“不过和收获相比,这些损失完全不值一提。”

方源评估着这场大战。

直到不利道痕彻底消失,紫薇仙子再无凭仗,这场琅琊福地包围战才真正落下帷幕。

毫无疑问,这场战役方源大胜,收获也极大。

他本人还是非常满意的。

基本上整场战役的过程,都在他的掌控中,没有偏离他的算计。

这就是重生的优势!

春秋蝉的厉害,彰显无疑。

当然,没有方源本身能力的支撑,凡人哪怕重生多少次,也改变不了蛊仙的大局。

本质上,春秋蝉是一只辅助性的蛊,重生回来能改变多少,还得看蛊仙个人的谋算和实践能力。

“不过这当中,也出现了一些意外。”

“比如凤九歌利用来因去果杀招,轻松撤离了战场。”

“还有不利道痕的另一种威能,紫薇仙子放弃黑暗漩涡,而是直接推算琅琊福地位置,令我猝不及防。”

这是方源上一世里,并没有显露出来的情报。

重生一次,并不能让方源看尽敌人手中所有的底牌。

尤其是面对天庭,这种敌人的底蕴太过深厚了。

重生是有优势,但不是绝对的。

敌人不是傻子,你变化了,和上一世不同了,敌人也会迅速调整,做出不同的反应。

紫薇仙子强行推算琅琊福地,以及凤仙太子配合追杀,就是摆在眼前的明证。

这些事情,都是方源上一世没有经历过的。

大战之前,方源也考虑到了这一点。所以,他选择抵挡天庭,没有谋算长生天那边。

在他上一世琅琊地灵不敌,请求长生天方面支援,使得天庭失败,方源逃脱。

方源镇压了琅琊地灵后,便有一个很诱人的选择摆在眼前。

那就是利用曾经的约定,故意让琅琊地灵向长生天支援。然后以琅琊福地为战场,让长生天、天庭两方蛊仙死战,最终方源看准时机出手渔翁得利。

但方源想了想,就否决了这个方案。

太冒险了!

虽然一旦谋算成功,利益极其惊人。但其中的风险,太过巨大,方源不能承受。

别忘了,春秋蝉还在回复当中,此时还不能再用。

永远不能小瞧你的对手!

不管是长生天,还是天庭,这些八转强者一个个都是人精,都有深厚的底蕴。

稍不留意,他们就会做出和上一世不一样的选择,他们的手段太丰富了。变化影响变化,最后变化越大,变得面目全非。

方源的实力不行,不能在两方争斗中罩住场子。

所以,方源最终选择了依靠自身,抵挡天庭。

这样一来,变数大大减少,自己也能镇压场面。虽然有一些意外发生,但无伤大雅,最终方源大获全胜。

琅琊派的一切都属于方源。

现如今,三座毛民大陆,方源都安置在小东海中。

而炼水损耗不小,被方源单独收集,形成一下片的炼海雏形,挪移到小东海的最西边,靠近小中洲。

琅琊福地中的云盖大陆,方源则将其挪移到了小白天里。

琅琊派的残破仙蛊屋炼炉,被方源收走,拆分下来,打算改良大盗鬼手。

炼道大阵则完全保留,留在云盖大陆上。

运道己运真传、盗天偷生真传、长毛炼道真传,方源今后都会陆续研究。

“三天的时间已过,不知道那几个异人蛊仙,考虑的怎么样了。”琅琊地灵一边操纵着长毛炼道大阵令其缓缓平息,一边说道。

方源轻哼一声:“我现在可是八转的蛊仙!识时务者为俊杰,若他们真的不识抬举,不想并入琅琊派中,那么就将这些人铲除掉好了。”

\end{this_body}

