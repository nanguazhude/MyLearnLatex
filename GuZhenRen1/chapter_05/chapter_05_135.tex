\newsection{仙蛊道可道}    %第一百三十五节:仙蛊道可道

\begin{this_body}



%1
片刻之后,方源控制住藏娇蚌,成为它的新主人。

%2
啪啪啪。

%3
他拍拍青年蛊仙的脸颊:“小子,算你识相。”

%4
青年蛊仙神情僵硬,不敢动弹,眼眸中都是羞愤和仇恨。

%5
方源却视若不见。

%6
随后,他跳上藏娇蚌,坐在青年蛊仙曾经的位置上,扬长而去。

%7
临走之前,还留下一句话:“本大人姓名楚瀛,就是我抢了你的藏娇蚌,记得回去告诉你爷爷。”

%8
望着藏娇蚌贝壳合上,投入巨流之中,消失不见,青年蛊仙这才从牙缝中挤出一句话来:“好,楚瀛是么?你放心,你的模样我早就通过蛊虫,告知我爷爷了!”

%9
“少爷,我们接下来怎么办?”两位美人泫然欲泣,紧紧攥着青年蛊仙的衣袖。

%10
青年蛊仙神情一僵,望着周围的空间越发狭小,几股巨流已经开始要转向变化,他的脸色再度阴沉几分。

%11
依凭他的实力,不能闯到这里,全靠刚刚的藏娇蚌。

%12
“放心,虽然藏娇蚌失去了,但我有爷爷给我的游子蛊,可以回到慈母蛊的身边,也就是爷爷那里。”青年蛊仙咬着牙道。

%13
“那真是太好了,我们有救了!”两位女子大喜。

%14
砰砰!

%15
两声闷响猝然发生,两位美貌女子瞪着惊骇欲绝的目光,一位看着胸口的血洞,另一位着望着青年蛊仙:“公子,你……”

%16
青年蛊仙脸色阴沉如水:“谁让你们目睹了整个过程,哼!”

%17
旋即。他的脸上又涌现出痛惜的神色。

%18
他伸出手来,抚摸两位美人的细嫩脸庞:“可惜了这对脸蛋儿。是挺漂亮。要怪就怪那个楚瀛罢。”

%19
说完,他便将这两具尸体随手推入一旁的巨流当中。

%20
没有蛊仙的防护手段。凡人尸躯在刹那间,就被巨流冲刷成渣。

%21
眼看着这片空间已经面临崩溃,青年蛊仙只要咬着牙关,催动了游子仙蛊。

%22
这片乱流海域,各种道痕杂乱不堪,宇道也是如此,蛊仙不可以随意动用宇道仙蛊,最安全的方法就是徒步跋涉。

%23
但青年蛊仙此刻运用的仙蛊,却不是宇道。而是情道。

%24
情道这个流派,脱胎于智道。智道三元,就是念、意、情。

%25
游子蛊、慈母蛊,乃是情道中赫赫有名的一套仙蛊。

%26
乱流海域形成之时,情道流派还未正式开创来,所以此刻,青年蛊仙可以安全地运用游子蛊,并无任何危险。

%27
青年蛊仙任凭游子蛊的力量,拖拽自己。

%28
他像是进入了一个漫长的河流之中。整个人混混沌沌,思维迷糊。大约过了半柱香的功夫,他清醒过来,发现自己已经回到了任修平的福地之中。

%29
“爷爷。我……”青年蛊仙面露惶恐不安之色。

%30
“哼,趁着爷爷我坐落仙窍,吸收天地二气。稳固福地,你居然敢偷偷跑出去玩耍。早知如此。我就不把藏娇蚌送给你护身了。”任修平冷哼道。

%31
“爷爷,藏娇蚌被人夺了。”青年蛊仙委屈地道。“我遇到一个魔头,他抢了我的藏娇蚌,还杀了我那两位侍女!”

%32
“哦?”任修平扬起眉头,“你详细说说。”

%33
青年蛊仙便添油加醋,说了一通。

%34
“楚瀛……”任修平口中咀嚼着这个名字,皱着眉头。

%35
这个名号,他还是第一次听闻。区区两个凡人女子算不了什么。七转蛊仙主动为难自己,抢夺了藏娇蚌,这的确是魔道行径。但为什么,他要特意留下姓名呢?

%36
难道他并非楚瀛,而是其他人,故意栽赃陷害?

%37
还是有其他什么图谋?

%38
总之,任修平将楚瀛这个名号,暗暗记在心中。

%39
他板着脸,继续教训孙子:“现在你知道蛊仙界的残酷了吧。一直以来,你跟在我的身边,见到的都是与我交好的仙友,他们对你态度宽容,甚至因为我而巴结你。这次,你要牢牢吸收教训,罚你闭关修行十年。十年之内,都要待在我的仙窍中,不得外出一步。”

%40
“什么?”青年蛊仙大惊。

%41
“滚。”任修平一挥袖,青年蛊仙视野骤然大变,已经陷入一座山洞之中。

%42
“玉不琢不成器啊,孙子。不能再让你胡闹下去了,否则,就算有我的庇护,这个东海也有大把不买你爷爷账的人呐。至于这个楚瀛,爷爷早晚会收拾他!”念及于此,任修平的眼中闪过一抹狠意。

%43
……

%44
“既然要交好庙明神,不妨就得罪一下任修平吧。”

%45
“话说回来,这藏娇蚌还挺好用的,可以随身携带。只要不是特别强大的乱流,完全可以在里面歇息,以逸待劳。”

%46
方源藏身在藏娇蚌中,让后者顺着乱流前行。

%47
就这样,距离目的地越来越近。

%48
但是在乱流海域中,距离并不代表行程。

%49
夺走藏娇蚌之后,方源开始运气低迷,连续碰到好几道乱流,方向错乱。让他不得不走了许多冤枉路。

%50
辗转了一大圈之后,他在一天之后,才到达了乱流海域的中心地带。

%51
这里的乱流,规模更加巨大,方源进入其中,宛若蝼蚁堕落江河。

%52
寻常的乱流,可能会每隔一段时间,改变流向,或者位置。

%53
但乱流海域中央地带的乱流,因为规模太过庞大,反而挪移不易,虽然每时每刻都在改变,但程度很小,所以相对稳定。

%54
在这些乱流中,夹杂着许多气泡。

%55
气泡有大有小,小的如人头,大的堪比山岳。

%56
这些气泡也是大战中形成,不知道是哪位大能的仙道杀招。

%57
气泡当中,往往有不少残骸或者遗址。更有许多气泡里面,包含着海水和小型海岛。

%58
“找到了,就是这个!”方源寻觅到目标,艰难地接近,然后一下子挤了进去。

%59
气泡薄膜被他挤破,但旋即就愈合起来,只是漏进去了一些流水,微不足道。

%60
方源视野大变。

%61
一片平静的绿色海面,波澜不惊。

%62
天空是蛋白之色,纯洁一片。

%63
简直是另外一个世界!

%64
海面中央,有一个小岛,小岛上长着野草,草丛中耸立着许多高大的石柱。

%65
有人说,气泡里的世界,其实就是乱流海域曾经的一角。只是大战之后,空间被撕裂成一块块。

%66
也有人说,这是某位大能的仙窍碎片世界,因为某个玄妙无比的仙道杀招,最终产生了这些气泡。

%67
气泡的成因,方源没兴趣关注。

%68
他来到石柱之间,取出飞剑仙蛊在手上。

%69
似乎是感知到了飞剑仙蛊上的信道印记,这些石柱开始散发出赤红的光辉,随后不久,它们开始自行缓缓地移动起来。

%70
当它们缓缓停下的时候,一片水缸大小的空间陡然敞开,掉落下一具骸骨。

%71
方源小心翼翼,走近查看。

%72
这具骸骨上面,印刻着丰富的骨道道痕,可算是一副准八转的仙材。

%73
骸骨的右手中,拿捏着一个墨绿气团。

%74
方源取出这个墨绿气团,参看一番后,轻轻捏破气团,露出里面沉眠着的几只蛊虫。

%75
一共两只仙蛊,五只凡蛊。

%76
但可惜的是,绝大多数都已经死了,只是一具躯壳。风一吹,化为飞灰。

%77
不知道过去了多少年,保管蛊虫的手法也不是特别好,难敌时间的冲刷。

%78
最终,只剩下一只六转仙蛊。

%79
方源连忙挽救,勉强吊住了一丝生机。

%80
“这是什么蛊虫?”方源纳闷,他不认识。

%81
又翻看骸骨,再无任何类似的气团了。

%82
“看来这就是所谓的信道传承了。”方源叹了一口气。

%83
换做寻常蛊仙,当是欢喜一场。但是刚刚取得了黑凡真传的方源,却感觉不佳。

%84
再仔细检查骸骨,方源发现,这具骨骼的头颅中,还封印着许多念头。

%85
念头已经不全了,幸好方源乃是智道宗师,十分勉强地从中挖掘出了许多信息。

%86
留下这个传承的蛊仙,名讳不可考证。方源只知道他是在激战之后,勉强吊住一口气,进入这个气泡中,布置下的传承。

%87
关于传承,的确是信道传承,但绝大部分的内容都丧失了。

%88
不过方源并非毫无收获。

%89
他知道了自己救下的这只仙蛊的名号道可道。

%90
道可道仙蛊,信道的侦查仙蛊。作用似乎很是鸡肋,就是帮助蛊仙数清楚目标身上的道痕数目和种类。

%91
至于这具骸骨的来历,却并非传承的布置者,而是被信道蛊仙斩杀的仇敌。

%92
念头中,这位留下传承的信道蛊仙,告诉继承者,可以将这具骸骨当做仙材直接用了。因为这位仇敌身上的道痕,都集中在这具骸骨上。

%93
这点倒让方源感到一丝惊奇。

%94
因为蛊仙死后,道痕往往集中在仙窍之中。除非是仙僵,仙窍没了,道痕才留在体内。

%95
“这位信道蛊仙,显然有着不一样的手段,能够将蛊仙的道痕都集中在骨骼上,保存下来。可惜传承内容缺失了太多,根本无法还原了。”

%96
方源感到十分遗憾。

%97
他本来还寄托希望,可以借助这份信道传承,让自己拥有解除盟约的本领。

%98
可惜,世间之事不如意者十之八九。

%99
他最终得到的,只是一只道可道仙蛊而已。

%100
将仙蛊收入仙窍,方源就准备离开。

%101
“凶手,哪里走!”就在这时,一道血红的身影,冲进了这个巨大的气泡当中。

%102
赤红的双眼,恶狠狠地盯着方源!

\end{this_body}

