\newsection{巡天五相,戚家双仙}    %第十四节:巡天五相,戚家双仙

\begin{this_body}

%1
苍天一线,浩荡辽阔。

%2
白云滚滚,在脚下激荡。

%3
一头巨大的雄狮发出阵阵低吟,粗壮的四肢踩踏虚空,在云端奔腾而行。

%4
这头雄狮,浑身似雪,一点杂色都没有。

%5
浓密的狮鬃在风云中飘扬,巨大的气浪在它脚下喷涌。

%6
雄狮的背上,竟安置着一张巨大的座椅,两位蛊仙一男一女并肩而坐。

%7
女子老态龙钟,六七十岁的模样,但却双眼绽射精芒,满脸兴奋,坐立不安,四处观望。

%8
而男子则是青年模样,鹰钩鼻子,双目细长,浑身上下散发着阴冷沧桑之气。

%9
“七爷爷,您的坐骑真是了得,居然腾云驾雾,才半天,我们就已经飞越数万里路程。”女蛊仙交口称赞道。

%10
“这是荒兽气宗狮,并非是腾云驾雾,而是它天生能驾驭气流,因此能腾飞空中。”青年男子慢条斯理地说道。

%11
“能驾驭气流,飞在空中的狮子,孙女还从未见过。荒兽……是什么意思?”女蛊仙旋即问道。

%12
青年男蛊仙无声地笑了笑,道:“戚荷,你刚刚晋升蛊仙,蛊仙界的常识,你不了解也很正常。这气宗狮,身蕴气道道痕,一旦成年,就可遨游苍穹。不过这种狮子,已经数量很稀少了,五域中几乎绝迹,只有在白天、黑天中才可见踪影。”

%13
“至于荒兽,是一种对兽怪的品级。就好像蛊师中有一转、二转、三转,除了荒兽,还有上古荒兽、太古荒兽。荒兽对已六转蛊仙。因为一头荒兽往往可以和一位六转蛊仙匹敌。而上古荒兽对应七转,太古荒兽对应八转。”

%14
女蛊仙戚荷闻言。双眼放光,再一次打量脚下的气宗狮。心中暗道:“难怪我在这头狮子身上,感到强烈的威胁之感。幸亏遇到七爷爷,我才侥幸渡过劫难,成为蛊仙。真要让我和这头狮子一战,恐怕我要被它撕成碎片。”

%15
青年蛊仙戚灾年纪已经数百,七转修为,只是用了寿蛊,才显得如此年轻。

%16
他年老成精,瞥见戚荷脸色。便将她心中所思所想,猜的八九不离十。

%17
伸手拍拍这位后辈的肩膀,戚灾宽慰道:“小荷勿忧,刚刚成仙,都是这样。等到我手头上的事情办完,咱们就回归气海洞天,那里还有你的几个长辈。若是知晓咱们戚族又多了一个蛊仙成员,他们一定会很高兴的。你会在洞天里生活一段时间,期间要多向长辈们讨教。记住多看。多问,多学。”

%18
“孙女谢七爷爷指点。”戚荷脸色一肃,连忙站起来,郑重其事地向戚灾行了个拜礼。

%19
“嗯……咱们爷孙俩还是坐下说吧。”戚灾点点头。眼中闪过一丝欣慰之色。

%20
戚荷坐下,脸上浮现出疑惑之色,迟疑了一下。她开口问道:“咱们戚家既然有多位蛊仙长辈,为什么不效仿武家、商家这些南疆超等势力呢?”

%21
戚灾眼底流露出一抹笑意:“你是想问。为什么我们这些长辈成为了蛊仙,却不像武家、商家那样。照拂家族后辈。而是任由家中后辈偏居一隅,饱受欺凌,是吗?”

%22
“孙女不敢。”戚荷慌忙道。

%23
“当年我独自修行成仙的时候,也有和你相同的疑问。”戚灾叹了一口气,眼中尽是回忆之色,“这原因说来,话就长了。”

%24
“追根朔源的话,还要说到我们戚家的老祖,他乃是数千年前的巡天五相之一。”

%25
“巡天五相?”戚荷疑惑。

%26
“巡天五相,乃是当时南疆蛊仙界公认的五位蛊仙强者。这五位高人皆有八转修为,站在蛊仙界的巅峰。他们时常联手,一起共探白天、黑天,关系紧密。因此被统称为巡天五相。咱们戚家先祖,便是五相中之一的气相。而咱们戚家,也是以气道为主修。”

%27
“八转蛊仙!”戚荷咋舌不已。

%28
戚灾感叹道:“九转不出,八转称雄!八转蛊仙几乎每一位都是传奇,一举一动都有极大的影响力。而当时南疆的巡天五相,更是威名赫赫,在其他四域都有响亮名声。皆因五位高人志趣相投,关系紧密,常常联手抗敌。而通常情况之下,八转蛊仙都是独来独往。因此在白天、黑天中,其余四域的八转蛊仙,见到巡天五相,通常都会主动退避三舍,避其锋芒。”

%29
“那个时候,南疆的蛊仙界因为五相,压过其他四域,五相更是风头无两。然而数十年后,巡天五相却分崩离析。”

%30
说到这里,戚灾的语气不免低沉下来。

%31
“这是为何?”戚荷忙问。

%32
戚灾叹息一声,继续道:“具体的原因,我其实也不太清楚。据说当年巡天五相,在探索白天的过程中,发现了一个重大的秘密。这个秘密代表着难以想象的成就,又只能一人染指。五相当中谁都想掌握,皆不肯退步。但五相实力又相差不大,且念及曾经合作的旧情,于是五相先祖便订下一个赌约。”

%33
“赌约?”

%34
“不错。这个赌约,以一千年为期限。赌约内容规定:五相均得袖手旁观,任由各自后人血脉自行发展,自生自灭。千年之后,谁家新晋的蛊仙数量最多,谁便是获胜之人。”

%35
戚荷算了算时间,一千年早已经过去,忙问:“那究竟哪一家获胜了?”

%36
戚灾苦笑:“一千年之后,竟有两家新成的蛊仙,人数相同,不分胜负。其余三家不愿放弃,以三对二,又强行再次加入下一轮的赌斗。于是,一千年、两千年、三千年……直至现在,五相先祖早已逝世,但是五家的赌约却仍旧持续着。”

%37
“竟然是这样!”戚荷首次听闻这个千年隐秘,一时不免瞠目结舌。

%38
好一会儿,她才接受下来,感慨道:“这么说来,若是没有这个赌约,咱们南疆岂不是要多出五个超级势力来吗?”

%39
哪知戚灾却缓缓摇头:“五相之中,有散仙,也有魔道中人,而发展家族,大多是正道蛊仙才有耐心做的事情。”

%40
“咱们戚家乃是气相,那么其余四相又是哪些?”戚荷又问道。

%41
戚灾知无不言:“气相、泥相、白相、血相、吃相,合称为巡天五相。我接下来要去的地方,正是泥相后代的一处所在。”

%42
“七爷爷去泥相那里,是要做什么?”

%43
戚灾犹豫了一下,失笑道:“算了,告诉你也无妨。近日,南疆蛊仙界发生了一件泼天大事,大量的蛊仙在义天山离奇灭亡。我们戚家也有一位先辈,折损在那里。所以这一次,我们寻找当代泥相,就是要令他(她)助我们推算真相。”

%44
“啊!义天山……”戚荷低呼一声。

%45
不管方源前世如何,今生的义天山大战,囊括仙凡两界。所以,戚荷也早有耳闻。

%46
只是她知道的是,南疆各大势力都派遣凡间强者,攻打义天山,铲除魔道,捍卫正道。在南疆,凡间魔道还从未如此大张旗鼓,去组建势力。所以遭受正道的猛烈扑杀。

%47
戚灾便将义天山的旷世赌约,告诉了戚荷。

%48
戚荷这才明白,原来真相如此。

%49
她口中低喃:“仙凡之差,云泥之别。那些四转、五转的高手,只不过都是仙人的棋子罢了。”

%50
这个真相对于刚刚成仙的戚荷,别有一股冲击力量。

%51
戚灾听在耳中,心里笑了笑。

%52
蛊师成仙,不只是实力、修行方面的改变,还有心理的变化。

%53
此时,戚灾正好借助义天山一事,促进戚荷更快的转变,更加适应新的身份。

%54
当下无言。

%55
只有气宗狮不断奔腾,戚灾、戚荷之间静默无声。

%56
周边云涛翻滚,天地浩荡。

%57
因为座椅之间,早就布置了蛊阵,所以一丝风都没有,更无嘈杂之音。

%58
这番景象,岂是凡人能够目睹?

%59
戚荷看在眼里,只觉天地广阔,凡人如蚁,满腹难言的滋味被渐渐冲淡,心想:“是了,我已经是蛊仙,再不是那曾经的凡人。仙凡有别,我当牢记才是。”

%60
劝慰自己一声,戚荷的心情渐渐平复下来,整个心灵都发生了一种微妙的转变。

%61
戚灾暗暗点头,主动开口道:“其实,数千年下来,巡天五相的后代之中,咱们戚家最是鼎盛。血相后人因修血道,已经彻底灭绝。吃相后代缩在仙窍之中,宛若圈中猪狗,不思进取,好吃懒做,不足挂齿。白相的最后一支也在近些年里,全然丧命。而泥相族人,虽然在南疆分布了几支,但早已经没有倪家蛊仙。”

%62
戚荷大有感受,点头道:“成仙何其艰难,若不是七爷爷出手相助,孙女恐怕早就死在灾劫之中了。”

%63
戚灾笑了笑,缓缓摇头,饱含深意地望着戚荷道:“其实倪家是最有希望,产出蛊仙的一族。只是正因为它最有希望,所以才遭受其余各家的阻挠,便成如今这个惨状。”

%64
“自赌约成立的三千年后,倪家出了位才华横溢的炼道蛊仙倪仁,也是倪家历史上最后一位蛊仙。他在临死之前,为家族未来筹谋。不惜违反赌约的规定,出手相助倪家,将精挑细选出来的一些仙蛊,都炼到后代血脉之中。”

%65
ps:今晚两更,累大家久候了。状态恢复了一些,欠下的只能慢慢补,还是要保证质量的。写到这里,三百多万字,不容易。朋友们跟到这里,更不容易。实在是不能草率!

\end{this_body}

