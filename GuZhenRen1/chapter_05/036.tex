\newsection{方源渡劫(结)}    %第三十六节:方源渡劫(结)

\begin{this_body}

楚度。[棉花糖小说网www.Mianhuatang.cc想看的书几乎都有啊,比一般的小说网站要稳定很多更新还快,全文字的没有广告。]

世人称之霸仙,乃是北原蛊仙界的传奇!

他出身贫寒,并非黄金血脉,起点甚低,开窍都是侥幸。没有师傅和亲族教导,他独自修行。他主修力道,却并未被力道的窘境所限制。而是凭着他横溢的才情,硬生生以一己之力,创造出斤力蛊、十斤之力蛊、一钧之力蛊、十钧之力蛊等等。

凭借这些蛊虫,他一步步成长,在世人瞠目惊愕的注视下,最终蜕去凡躯,成就蛊仙。

但这并非他传奇的结束,站在蛊仙的层次上,他回首整理,创出人力钧力流。

这个小流派,给予日暮西山的力道一记强烈的振奋力量,仿佛是枯木中灌注进生机,令力道绽放出全新的气象。

继兽力虚影流、气象天地流之后,力道流派中增添进人力钧力流,被世人认同。

他以一人之力,改变了力道流派的格局。

没有任何一个人,可以否定他的伟大成就。

至今,北原的蛊师们,十有八九都兼修力道,走的路数就是人力钧力流。

命运是很奇妙的,常人无法揣度。

不管是楚度,还是方源,都未想过,会在这种情形下与对方相遇。

方源渡劫,因为上古墟蝠的力量,自身仙窍福地的窍壁,被撞毁一块,打通了外界。

而楚度是护卫徒弟,对付升仙灾劫,吸收狂蛮真意。

北部大冰原那么大,偏偏两人选择渡劫的位置,距离如此相近。

相隔千里,两人的视线在第一时间,对撞在一起!

楚度刚刚击溃徒弟的灾劫,一手背在身后,一手拿捏着手中的霜雷,云淡风轻,渊渟岳峙。

方源白袍翻飞。剑眉星目,浑身带伤,战意滔天,仿佛整个人都化身成一柄绝世神剑。锋锐的气息让楚度都轻轻皱了皱眉头。

“竟有人在这里渡仙窍灾劫。这是怎么回事?里面的上古墟蝠明明就是狂蛮真意,借助天地灾劫显化。此人手中竟掌握着绝世妙法,能复制蛊师升仙,师法狂蛮的成果,增益自己!”楚度心中充满了震惊。还有狂喜!

“霸仙楚度!”方源心中亦是震惊不已,“这个家伙护卫他人渡劫,定然是打狂蛮真意的主意。糟糕,我此番情景落到他的眼中,他定会主动找我的麻烦!”

方源从琅琊地灵处得到的仙灾锻窍杀招,对于霸仙楚度而言,诱惑力是绝对巨大的。<strong>最新章节全文阅读qiushu.cc</strong>

“只要我得到他的法门,我何须劳心劳力,培养徒弟升仙渡劫呢?我自己就可以在渡劫中,引动狂蛮真意。并且灾劫更强,真意更多,效率绝佳!必须,必须得到这个法门!!”楚度心中大吼,身形如电,直朝方源射来。

他号称霸仙,当年晋升六转,有资深散仙来找麻烦,被他打得半死。散仙又搬来救兵,结果三位散仙被刚刚升仙的楚度。一路追杀数十万里,终将三仙陆续杀死。

百年之后,楚度和刘家蛊仙产生矛盾,对战十几个回合。刘家蛊仙不敌,当众求饶,楚度毫无犹豫,直接杀之。

之后,刘家出动数位蛊仙,楚度双拳不敌四手。隐形匿迹了数十年。再出现时,已经是七转修为,他四处骚扰,三日内捣毁刘家十八个资源点。刘家三仙追杀,他不退反进,不仅杀退刘家三仙,还差点打进刘家的大本营。

楚度围在刘家福地,停留不走。刘家太上大长老夹怒归来,楚度与其大战三百回合,刘家太上大长老,乃是资深七转,竟然奈何不得楚度。最终只能任由他退走。

此战,令楚度名声大噪,更令蛊仙界领略到了他霸道的行事风格!

再没有任何一位蛊仙,胆敢轻视他。想要找他的麻烦,往往都要深思熟虑,三思而行。

不过此战之后,楚度却收敛行迹,以隐修为主,鲜少现身。最近也只是参加了北原拍卖大会,才显现在北原诸仙面前。

原来,当年楚度之所以能够实力暴涨,战胜刘家,就是因为得益于狂蛮真意。

他培养了数位蛊师,护其升仙,令自身的力道境界达到了宗师级数,创造出属于自己的仙道杀招。

时机成熟之后,他抓住刘家虚弱的大好时机,打上门去,不仅报了当年之仇,还正式确立了自己的赫赫威名。

这些年来,他隐形匿迹,仍旧是在北部大冰原里,孜孜不倦地挖掘狂蛮真意这份巨大的宝藏。

楚度见到方源,不仅是觊觎方源手中勾动狂蛮真意的妙法,而且还有一种自家宝物被人发现、偷取的恼怒。

所以,他没有多想,直接进攻,想要将方源拿下,再严刑拷打出勾动狂蛮真意的奥秘。

千里之距,对于蛊仙而言,算不得什么。

双方的距离,迅速拉近。

“嗯?”楚度正缓缓伸出右掌,遥遥对准仙窍中的方源,却不想一道剑光,已经先他射来。

他反应很快,但方源的反应比他更快!

几乎在第一时间,方源就催动剑痕索命,射出了仙蛊飞剑。

飞剑射破长空,璀璨的亮银剑光,竟让楚度也有不可逼视之感。

楚度嘿然一笑,伸出右掌,照准飞剑缓重一抓。

轰!

一声爆响,空气炸裂开来。

一只巨大的力道虚影,显然就是楚度的右手掌模样,挤压空气,宛若小山横撞,似缓实快,一把抓住飞剑仙蛊。

方源眼中闪过一抹惊异之色。

楚度的力道虚影巨手,和他的大手印竟十分相像。但很明显,他的力道巨手不仅比方源的大手印更强大,还要更灵活。

楚度的力道巨手,和他的右手掌同步动作。

抓住飞剑仙蛊之后,楚度重重一握,竟然直接将飞剑仙蛊封印住。

方源顿时感到,自己和飞剑仙蛊之间的联系,变得微乎其微。

楚度眉头忽然微不可察地皱了一下,他开口握紧的右手,发现手掌中央。出现了一道血痕。

正是杀招剑痕索命的伤害。

“有点意思。”他淡淡地评价一声。

话音刚落,他手心的血痕就已经痊愈。

“交出妙法,饶你一命。嗯?”楚度发出通牒,抬头一看。神情再变。

关键时刻,方源竟直接斩断自己的下半身,动用了某种血道杀招,将下半身当场炼化成一股浓稠的血浆。

血浆扑上仙窍漏洞,竟然眨眼间。就将宇道漏洞填补起来,迅速修复成功。

楚度连忙催动力道巨手,狠狠扑上。

但差之毫厘谬以千里,他失去了良机,巨手扑了个空,如流星陨落般,顺势砸在地面上,将地上直接砸出了个深达十丈的巨坑。

楚度飞到方源落下仙窍的地点,脸色铁青。他可没有直接破进仙窍福地的手段。

“该死!怎么会偏偏碰到霸仙?”仙窍内,方源只剩下上半身。悬空飞行。伤势惨重,他的腹部以下都已经被血炼掉了。

剧痛被身上蛊虫吸纳,方源不受影响,面对两头上古墟蝠,保持着冰雪般的冷静。

“必须尽快杀死这两头墟蝠!”方源狠狠咬牙,神色无比凶残。

剑浪三叠!

这个杀招被他连续催动,澎湃的剑浪汹涌而至,硬碰硬地和上古墟蝠对撞。

惨烈!

方源七窍喷血,头晕眼花,处在晕死的边缘。

他头疼欲裂。浪剑三叠这个杀招,涵盖了许多蛊虫,催动过程一点都不简单,为了调动那些众多的蛊虫。方源的念头耗得一干二净,整个脑海处于极度干涸的状态。

而且为了保护窍壁,不让上古墟蝠的力量再次破坏,方源只能以螳臂当车之势,拼死挡下上古墟蝠的每一次冲撞。

读秒如年。

每一次挡下上古墟蝠的进攻,方源都眼前一黑。对下次能否再抵挡得住,毫无把握。

根本没有能力想其他的东西,所有的念头刚刚生出,就被方源立即消耗,用来调动浑身上下的无数蛊虫。

双方就像是狭路相逢的死士,赤身刺刀拼杀,一刀刀刺在对方的身上,就看谁先支持不住,失败倒下。

也不知过了多久,一阵风吹来,方源恍惚的双眼,现出一丝清明。

“成功了吗?”他望着高空中,正在坠落的上古墟蝠,口中无意识地呢喃。

很快,他脑海中的念头丰富起来,让他能够更多思考。

一只上古墟蝠,终于被他杀死。但还剩下一头。

方源舔了舔干燥至极的嘴唇,惨白如纸的脸上,再现一丝扭曲的神色,他继续扑向上古墟蝠。

接下来的情景,却轻松多了。

独自对战一头上古墟蝠,方源压力大减,脑海中的念头越来越多,让他能想得更多,更冷静明智地选择应对墟蝠攻击的方式。

但他不敢有丝毫的放松。

万一在凝聚出一头上古墟蝠,那就糟糕了。

更糟糕的情况,还是外界的霸仙楚度,若让他闯进来,方源的下场极度堪忧,比直接死在灾劫中还要可怕。

不过最终,天地二气缓缓平息下来,方源成功斩杀掉第三头上古墟蝠,也不见更多的墟蝠产生,楚度更是不见踪影。

“灾劫停息了。这一场灾劫,总算是渡过去了!”方源无力地俯瞰下方,这一方天地已经成为冰雪之地,无数的荒兽雪怪,仰头咆哮,仍旧对他面露凶光。

“好在我的仙窍够大,也没有什么资源,这些雪怪就先放这里罢。单凭他们是打不通窍壁的。”方源直接飞升上了一重天。

他伤势极重,急需休养。

疗伤的同时,还要积极备战。

万一楚度忽然打进来,可不是闹着玩的!(未完待续。)<!--80txt.com-ouoou-->

------------

\end{this_body}

