\newsection{楚瀛再现}    %第二百七十四节:楚瀛再现

\begin{this_body}

“终于穿透界壁了。<strong>求书网WWW.Qiushu.cc</strong>”影无邪吐出一口浊气,望着汪洋海面,神情极为疲惫。

他身边的黑楼兰、白凝冰,亦好不到哪里去。

从北原出走,影宗一行人,前往东海。他们不是至尊仙体,每穿越一次界壁,都是一场艰难的考验。

好在他们个个实力不俗,乃是蛊仙中的强者精英。纵然仙窍受损不少,但是没有发生减员的情况。

他们并没有着急离开。

进入东海之后,他们随意选择了一处海岛。

在这座无名的小海岛上,影无邪指挥白凝冰、黑楼兰,开始布置一座炼道蛊阵。

蛊阵布置成功,影无邪便走向蛊阵中心,取出怀中的紫金石块,开始处理这块仙材。

至于白凝冰、黑楼兰二人,则位于蛊阵之外,分别驻守,防备意外来敌。

不久后,紫金石块就再度融化,小人蛊仙紫山真君,从中苏醒。

他乃是八转蛊仙,一般而言,跨越五域,要前往黑天白天。但是北原上方的黑天、白天,都已经被镇运天宫影响,长生天方面掌控。

影宗一行人,自然不可去自投罗网。中洲蛊仙的下场,就在他们眼前。

影宗底蕴雄厚,紫山真君便又陷入沉眠,周身表面结成了一层薄薄的紫金石皮。

他气息全无,生机收敛到了极致,和死物没有什么两样。

正是凭借这等奇妙至极的手段,紫山真君被影无邪带领着,顺利穿透了界壁,来到了东海。

紫金石皮薄薄一层,让掌握了专门手法的影无邪轻松炼化,比当初万寿娘子处理紫金石块时要轻松多了。

“东海……”紫山真君苏醒,看着岛外的广袤海面,心生感慨。

在他的带领下,影宗一行人首先将蛊阵收起来,将一切的痕迹都抹除掉。

随后四位蛊仙,正好催动得了上古战阵四通八达。

片刻后,他们赶到了乱流海域。

紫山真君为首,所到之处,一切海流都无法构成阻碍。(www.QiuShu.cc 求书小说网)

很快,影宗一行人便来到了乱流海域的中央地带。

“嗯?这一片的光阴支流呢?”搜寻未果,影无邪感到相当的奇怪。

“难道说,那道光阴支流,已经去往他处,不在中央,而在逆流河中段,或者外围?”黑楼兰猜测。

继续侦查,很快,紫山真君发现了一些关键线索。

“那条光阴支流,已经被他人收走了。”他语气一沉。

“怎么会这样?是谁干的?难道是方源?”影无邪第一想到的就是方源。

但紫山真君摇头:“这种手法,应当不是他。而是另有其人,也罢。既然这处光阴支流无法借用,我们就到下一个地方,那里也有一道光阴支流的存在。”

影宗一行四人,很快又离开了乱流海域。

而在东海的另一端。

碧涛万里,晴空一片。

在海面上,一场生死追杀正在上演。

“你快走吧,我中了他的雷魄针,时刻被他感应,已经走不了。我不能连累了你。”花蝶女仙躺在一位男性蛊仙的怀抱中,虚弱无比地道。

她身着宫裳,衣袂飘飘,粉嫩若腻。一双美眸,此刻透着虚弱的目光。嘴角不断地溢血,伤势很重。

尤其是她肌肤表面,时不时地还亮起一些细小尖锐的电光,这正是仙道杀招雷魄针的中招迹象。

“不行。”男性蛊仙想都没想,毫不犹豫地拒绝。

他狼背蜂腰,一身战甲,虽然也是伤痕累累,但仍旧掩盖不了精悍之色。

“怎么可以丢下你不管?我若临阵逃脱,事后如何向庙明神大人交代?”他口中虽然这么说,但看向花蝶女仙的目光深处,却是隐藏着屡屡情意。

他正是蜂将。

蜂将、花蝶女仙、鬼七爷,这三位蛊仙正是东海宇道强者庙明神麾下。

庙明神希望找寻到乱流海域的光阴支流,因此这三位蛊仙每隔一段时间,都会轮流看守乱流海域。遇到每一位前往乱流海域探索的蛊仙,都会上前交谈,希望他们出手帮忙寻找。

“那你们一起做对亡命鸳鸯吧!”身后的追兵只有一人,但却是七转修为。

他姓葛名温,蓝袍加身,浑身皮肤干瘪粗硬,宛若树皮。眼中冒着赤光,牙齿尖锐,神情狰狞。

这人资辈不小,在东海也有薄名。

他乃是雷道宗师,但一次施展某个仙道杀招的时候,失败反噬,导致他不得不转变成仙僵。

即便如此,他战力出众,拥有两只仙蛊,将六转修为的花蝶女仙、蜂将,打得四处飞逃。

“将年轮散交出来,你们两个胆子不小,居然敢抢我的仙材。你们若乖乖交出,或许还可以给你们两个留一个全尸!”葛温桀桀大笑,凶威滔滔,越追越近。

花蝶女仙气急:“这年轮散,本来就是我们先发现的。怎么是你的东西?”

“我在这里守候了大半个月,结果一不留神,被你们捷足先登!”葛温怒吼。

花蝶女仙怒气交加:“这年轮散,庙明神大人在三年前就发现,在周围布置了蛊阵。每隔几个月,我就来到蛊阵中,对其栽培看顾。你再信口雌黄,也改变不了这个事实。”

花蝶女仙道破事实,葛温不再反驳,脸色又阴沉了一分:“哼,臭丫头牙尖嘴利,追上你后,我第一个杀你!”

“魔道贼子,休得猖狂!”

就在这时,忽然一道响亮的声音,响彻在三仙的耳畔。

“什么人?”葛温面色微变,立即断喝。

他话音刚落,海面忽然凸起,像是鼓出一个小山丘。

随即,“山丘”被里面的猛兽一头撞破,呼呼风响,水花四溅。

一头巨大的蓝鳞海龙,出现在三仙眼中,海龙口吐人言:“那魔头,纳命来!”

三仙见此,哪里还不明白,这位蓝鳞海龙便是一位蛊仙变化而成。

花蝶女仙、蜂将又惊又喜。

葛温怒上加怒:“叫你多管闲事!吃我一记杀招。”

说着,他深呼吸一口气,然后照准右手掌心一吐。

吐出的自然不是口水,而是一团鸡蛋大小的雷光。

雷光附着在葛温的右手掌心,轰的一声,一道雷光巨柱,从他的右手掌上暴射喷涌而出。

蓝鳞海龙不闪不避,见着雷光巨柱射来,它猛地张口大口。

龙息喷吐!

葛温见到这一幕,不禁肚中暗笑:“我的这招可持续一盏茶的功夫,龙息却是一口口喷吐,如何抵挡得住?”

但蓝鳞海龙的龙息,竟然持续不断!

龙息宛若水柱,和雷光巨柱直接对撞,随后庞大强悍的龙息,将雷光巨柱一路消抵摧毁。

几个呼吸的功夫,就到了葛温的眼前。

“怎么可能?这是什么龙息?!”葛温的脸色涌现出难以置信的神色。

他连忙躲闪,避让开去。

蓝鳞海龙微微掉转龙头,龙息如柱,再次横扫过来。

葛温眉头紧皱,怒哼一声:“算你们走运,这次就先绕过你们!”

随后,他竟直接撤离,化作一道雷电,速度极快。

蓝鳞海龙也不追击,看着葛温远离。

“他是要回去蛊阵,那里还有大量的年轮散残留着。”花蝶女仙一句话,道破了葛温的心思。

蜂将将花蝶女仙放下,对蓝鳞海龙拱手道:“仙友高义,出手击退强敌,救命之恩,不敢忘怀。请教仙友尊姓大名!”

蓝鳞海龙忽然变化,现身一位容貌普通的蛊仙,嘴角含笑:“在下楚瀛。花蝶仙子,别来无恙乎?”

花蝶女仙顿时惊喜交加:“没想到竟是楚瀛仙友!”

见蜂将掉头看向自己,花蝶女仙便解释道:“正是托了这位楚瀛仙友的福,我们才得以发现乱流海域中的那道光阴支流。庙明神大人才能将光阴支流收入囊中啊。”

ps:今明后三天都是一更,我要整理大纲。(未完待续。)<!--80txt.com-ouoou-->

------------

\end{this_body}

