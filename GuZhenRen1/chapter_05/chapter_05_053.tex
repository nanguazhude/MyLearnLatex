\newsection{内奸暴露}    %第五十三节:内奸暴露

\begin{this_body}

%1
方源摇摇头,叹息一声。,

%2
智慧蛊的喂养,代价太大了。也就是琅琊地灵,愿意付出。

%3
方源利用至尊仙胎蛊获得新生,现在的年龄是十六岁的样子。这个世界的人族,平均寿命是一百岁。一百减去十六,方源还有大约八十四年的寿命,算是比较长久的。

%4
但就算这样,方源自忖,若是自己得到寿蛊,基本上都会用在自己身上。现在就算不用,也要囤积起来,以备不时之需。

%5
寿蛊只是凡蛊。

%6
然而寿蛊是蛊仙交易中,最强劲的硬通货,几乎没有任何一个能与之媲美。

%7
蛊仙修行有两个最大的关卡,一个是灾劫,另一个就是寿命。

%8
强如仙尊、魔尊,都要受着这两个关卡的制约。

%9
事实上,当年巨阳仙尊布置八十八角真阳楼,其中最重要的每一个目的,就是在整个北原搜刮寿蛊,助他延寿,至于喂养智慧蛊的事情,放在其次的位置上。巨阳仙尊晚年,很长一段时间,智慧蛊都陷入饥饿状态,不得不深度沉眠,不可运用。

%10
后来幽魂魔尊探索八十八角真阳楼,也发现了其中的智慧蛊。可惜他用不了,也养不起,寿蛊难寻,他自己都不够用呢。不过,以他魂道境界,早已触类旁通,对智慧蛊的需求并不多。

%11
等到幽魂魔尊陨落,属于他的一个大时代徐徐落幕。盛极而衰,天地间又重新开始陆续出现寿蛊。而后,衰极而盛。出现了另外一位惊天动地的大人物,那便是乐土仙尊。

%12
他为了延寿。也在晚年搜寻寿蛊,寿蛊数量入不敷出。又渐渐稀少,最终绝迹。

%13
乐土死后,天地循环,休养生息,寿蛊渐出。八十八角真阳楼每隔十年,开启一次王庭之争,搜刮北原寿蛊,将智慧蛊又渐渐喂养,恢复了状态。直至方源捣毁王庭福地。

%14
然后,在求生的本能促使下,智慧蛊和方源辗转来到狐仙福地里去。经过一番事情,又落到如今的琅琊福地里来。

%15
“三王福地的时候,我一连用了两只寿蛊,企图炼制第二空窍蛊。但那是没办法的事情。我上一世,从焚天魔女得过一只寿蛊。重生之后,却没有了这项机缘,手中一只寿蛊都没有。焚天魔女既能给我一只寿蛊。她的积累中应当不止一只寿蛊了。她最终死在十绝大阵之中,真是可惜,手中的寿蛊也因此毁了……嗯?不对。”

%16
方源思绪发散,忽然念头一转。

%17
“寿蛊难寻。即便是仙尊魔尊都要觊觎,都有需求。焚天魔女是受魔尊幽魂算计而亡,她一身的积累。应当是尽数被魔尊幽魂提前取走了!魔尊幽魂的大计虽然失败,但影宗、僵盟的残余下来的势力。还有财富,仍旧十分惊人!”

%18
至于琅琊地灵手中。也有一笔寿蛊的积累,数量还相当可观。

%19
这些寿蛊,几乎全部都来源于琅琊福地。

%20
琅琊福地的历史相当悠久,有三十多万年,前后经历两代尊者。

%21
琅琊福地原本是琅琊洞天,内部空间宽阔无比。后来故意跌落成福地,却用了炼炉融入仙窍世界,导致琅琊福地的空间不减反增,比许多洞天还要广阔。

%22
这里面生活着海量的毛民,形成了三个国度,难以计数的生灵在里面繁衍生息,一代又一代。

%23
寿蛊的产生,就源于生命。

%24
但凡是有生命的地方,都可能天然孕育出野生的寿蛊。

%25
空间广阔,人口无数,历史悠久,再加上琅琊地灵对福地的彻底掌控,一旦有什么寿蛊产生,便能立即知晓,迅速采摘。这几个因素结合在一起,就导致了琅琊地灵手中的寿蛊积累,越来越多。

%26
方源虽然不知道具体的数目,但用脚后跟都能想得出来,这个数目绝对不少!

%27
毕竟寻常的蛊仙,还会寿蛊。琅琊地灵却不需要。

%28
“驱使智慧蛊,需要耗费蛊仙本身的寿命。而豢养智慧蛊,则要消耗寿蛊。暂时,还是交给琅琊地灵喂养的好。”

%29
琅琊地灵的图谋,方源不用试探,都隐约猜到了。

%30
这是没办法的事情,方源自己没有寿蛊,喂养不了。

%31
而且他连自身都有些难保,蛊仙渡劫一次比一次艰难,灾劫一次比一次更强。方源正在为下一场地灾发愁,根本无暇顾及智慧蛊。

%32
自从智慧蛊被带出来后,方源就没有喂过一次。

%33
话说回来,琅琊地灵也算是帮了方源这个忙。

%34
“退一万步讲,就算是智慧蛊投靠了琅琊地灵,也是一桩好事。我也可以利用现在的身份,借助琅琊地灵的手段运用智慧蛊,顶多付出一些门派贡献吧。”

%35
所有的蛊虫,都是工具。

%36
一切的努力,都是朝着那个虚无缥缈的目标去前进。

%37
为了这个目标,方源舍弃了太多太多,也没有什么不能舍弃的。

%38
对智慧蛊的这场试验,以失败告终。

%39
不过方源倒也没什么失望的情绪,临走之前,他向琅琊地灵索要了一样东西。

%40
一只信道凡蛊,里面记录着的就是他炼制变形仙蛊时的情景。

%41
“这一次炼蛊,极为古怪,灾劫的威力忽然变得那么强!能够成功,实在是侥幸。方源你要研究出什么来,一定要告诉我啊。”琅琊地灵在分别时,叮嘱方源。

%42
“太上大长老你都没有研究出什么,我更不可能了。总之我尽力而为罢。”方源如此回答。他隐隐觉得,灾劫威力诡异暴涨,恐怕是和至尊仙胎蛊有关。不过这个猜想,打死他也不告诉琅琊地灵。

%43
除了研究灾劫之外,方源还有一个目的,就是调查琅琊派内部,查清是否还有内奸存在。

%44
一直以来,方源心中都有这个疙瘩。

%45
影宗在琅琊福地中,埋设了两个内奸,竟都是毛民蛊仙。影宗突袭琅琊福地,这两个毛民蛊仙内奸发挥了举足轻重的作用。

%46
“但除了这两人之外,究竟琅琊派中还有没有其他内奸呢?”

%47
这个问题的答案,对方源而言,十分要紧。毕竟他现在,已经加入了琅琊派。未来很长一段时间,他都要依靠琅琊地灵,帮助自己渡劫、修行。

%48
这个调查,他才刚刚开始。之前一直都没有精力和时间。

%49
飞行了一段时间,方源的住处已经遥遥在望。

%50
这是一座高耸雄壮的城池。

%51
洁白的城墙,威武伫立。高大的塔楼尖端,各有大旗飞扬。

%52
这是云城。

%53
云盖大陆上,共有十二座云城。每一座城池,都是凡蛊屋。同时催动,就可形成十二波云迷澜大阵,可以覆盖整个云盖大陆!

%54
这十二座云城,就是从曾经的十二云阁提升出来的。

%55
这一任的琅琊地灵,喜好排场,和上一任完全不一样。

%56
方源视野中的这座云城,就是琅琊地灵调拨下来,供他镇守居住的地方。其余的毛民蛊仙,包括琅琊地灵自己,都各自占据一座云城。

%57
城中的居民数量极少,不过以后,会越来越多。因为琅琊地灵的计划,就是发动三大陆的战争,从中挑选出优秀的毛民蛊师,将他们搬迁到云盖大陆上悉心培养。

%58
方源飞到云城上空,徐徐落下。

%59
几位毛民蛊师立即出来迎接。

%60
他们都是毛民蛊师中的精英,得到琅琊地灵的看中,被安排在方源身边。一方面是照料方源生活中的诸多杂事,另一方面也有让方源指点他们修行的潜在意思。

%61
方源对这些毛民,自然没有放在心上。不过领头的毛民蛊师却恭声禀告方源,第六城主前来拜访方源,已经在城中贵客厅中坐了好长一段时间了。

%62
第六城主,便是蛊仙毛六,镇守第六云城。

%63
“他怎么会来?”

%64
方源心中带着疑惑,一路来到贵客厅里,见到了这位毛六蛊仙。

%65
“在下毛六,见过第二城主。今次拜访,主要是想向城主请教战斗之道。”毛六蛊仙一边施礼,一边说明来意。

%66
方源面色不变,眼神却陡然变得冰寒无比。

%67
原来,毛六口中这么说,私底下对方源暗中传音,却是另一番说辞。

%68
“方源,我这一次是代表影无邪大人,前来和你做一场交易的!”

%69
影无邪!

%70
影宗!

%71
毛六就是内奸!!

%72
一瞬间,方源脑海中迅速闪过许多念头,好像是闪电般。

%73
“好说,好说。都是一派中人,请教说不上,切磋即可。”方源哈哈一笑,脸色热情洋溢。

%74
他先屏退左右,请毛六重新落座,自己再走到主位,施施然坐下。

%75
这几步路,并不远。但方源的脑海中,却是思绪翻腾。

%76
“毛六是内奸,那他是琅琊福地中唯一的内奸吗?”

%77
“他代表影无邪和我交易,是想和我交易什么?”

%78
“影无邪那边,是什么情况?忽然和我交易?”

%79
“他是诚心的吗?还是算计我的陷阱?或者是琅琊地灵的试探?毕竟我加入琅琊派,可是身负信道盟约的。”

%80
想到这里,方源刚刚坐下。

%81
他脸色忽的一板,双眼锐利如刀,逼视毛六,眼中露出滚滚杀意,寒声道:“毛六,你好大的胆子!你刚刚说的话,可是当真?若是当真,我即刻就将你交给太上大长老审办!”

\end{this_body}

