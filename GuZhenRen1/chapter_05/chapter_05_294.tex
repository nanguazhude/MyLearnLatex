\newsection{收购神鹿果}    %第二百九十四节:收购神鹿果

\begin{this_body}



%1
南疆,武家。

%2
修行的密室中,方源悬浮在半空中,满脸肃穆之色。

%3
他的长发飘飞,浑身上下蛊虫的各种气息不断升腾,相互纠缠混杂在一起。

%4
“关键一步……”方源心中呢喃,目光变得犀利起来。

%5
他小心翼翼地驱使出一只凡蛊。

%6
这是金道凡蛊,只有一转级数而已。

%7
单独催动此蛊,自然毫无风险。但是此刻情况却是不同。

%8
砰。

%9
一声闷响,从方源的身上传出。

%10
方源顿时如遭电击,雄躯狠狠一颤,面色发白,嘴边上溢出一丝儿鲜红血迹。

%11
“又失败了。”他心中大叹一口气。

%12
从半空中落到地面上来,手扶着胸口,检查伤势。

%13
“还好只是用了一只一转凡蛊,此次受伤并无大碍,无须动用人如故仙蛊疗伤。单凭至尊仙体本身的强大恢复能力,三两天就能恢复了。”

%14
方源这是在试图继续改良刚背杀招。

%15
仙道杀招刚背,以六转仙蛊金刚念为核心。方源依仗了这个杀招,并用智谋,成功击败了夏家蛊仙夏飞快。

%16
事后,方源总感觉这个杀招意犹未尽,还有可以提升改良的空间。

%17
这种感觉,得到方源的重视。

%18
他知道,自己智道、变化道双宗师境界,这种感受并非空穴来风,而是很有操作意义。

%19
“可惜的是,这些天推算改良,虽有一些进展,但都收效甚微。看来我境界方面,还是不足,暂且作罢。”

%20
方源叹了一口气。

%21
其实,他这种进展速度,落到其他蛊仙身上,已经是非常可观。

%22
当初,凶雷恶人为了推算一记仙道杀招雷神子,整整耗费了数年光阴。他的进展速度,很多时候,连方源此时的一成都没有。

%23
不过方源放弃,也是明智之举。

%24
因为他有可以拔升境界的手段,只要将境界提升上去,改良杀招刚背,完全可以做到一蹴而就。

%25
随着方源的龙鱼陆续贩卖出去,他的经济状况得到了相当程度上的好转。

%26
这才有余力,去改良仙道杀招。

%27
推算仙道杀招,其实很耗费资源。比如演练失败后,蛊仙会受到反噬,承受伤害,总得疗伤。蛊虫也跟着损毁,需要不断的补充。若是仙蛊受损,更加麻烦。

%28
但如果境界足够的话,却是能一下子改良出来,节省大量的精力、时间和资源。

%29
方源现在改良,境界方面比较勉强,虽然也能改良出来,但付出的代价是一笔比较庞大的数字。

%30
他思索之后,果断放弃了这个修行当中的小计划。

%31
休整片刻之后,方源催动蛊虫,将心神灌注到宝黄天中去。

%32
在宝黄天中,他的心神轻车熟路,拐到一位蛊仙意志的面前,进行交流。

%33
“你的这些神鹿果,我出一千三百块仙元石,都卖给我吧。”方源道。

%34
这位蛊仙意志摇摇头,有些无奈:“我说了,这个价格低了。仙友你到我这里,已经来了三回。你若真要想买的话,就再出两百块仙元石,这些货你直接拿走。”

%35
这种神鹿果,乃是六转仙材。

%36
并不是稀有蛊材,但卖的人却比较少。

%37
原因无它,市场需求少。

%38
不过方源却是需要它来喂养金刚念仙蛊的。

%39
他得到了金刚念仙蛊,总得要解决喂养仙蛊的问题,不能让这只仙蛊饿死。

%40
方源没有说话,而是不断打量。

%41
这个摊子上的神鹿果,品相很好,方源逛过宝黄天已经好几遍,发现这里的神鹿果质量是一流的。

%42
价格方面,也比较合理。

%43
所以,他才往返这里好几次。

%44
神鹿果并非树上、花草上结出来的果实,它们一个个形如婴儿拳头捏起来,生长在一种荒兽麋鹿的鹿角上。

%45
挂果麋鹿。

%46
方源早已经打探出,这种荒兽的具体名号。宝黄天中也有贩卖。

%47
但是,就算他买下来一批,能够产出多少的神鹿果,却是个相当值得探究的疑问。

%48
任何的资源产出,都是有着栽培的门道。

%49
方源能产出龙鱼,也不是随意胡闹尝试的结果,而是有着东方一族的心得经验。

%50
这可是来自一个超级家族的底蕴。所以,方源的龙鱼品相上佳,产出顺利,成本微低,收益较多。

%51
“现在,还是先筹集一部分的神鹿果实,应付金刚念仙蛊的第一次喂养。”

%52
“等到将来,手头上更加充裕之后,再考虑如何豢养挂果麋鹿,达到自给自足,独立喂养金刚念仙蛊的程度。”

%53
“至于以后,能不能将这个神鹿果量产,成为另外一个盈利的点,那就看将来的机缘了。”

%54
方源心中一边思量,一边和那蛊仙意志交流:“那我就再看看吧。”

%55
但这个时候,方源眼前的蛊仙意志忽然一顿,随后改了口:“我本体的心神投注进来了,他说可以再商量一下价格。还是你们自己谈罢。”

%56
说着,蛊仙意志便退居二线。

%57
方源大喜。

%58
他等的就是这种情况。

%59
这位贩卖神鹿果的蛊仙,将神鹿果留在宝黄天中,又留下一股意志照看。

%60
但这股意志虽然能够思考、交流,但是却恪守蛊仙本体交代的价格底线,这使得方源根本无法洽谈。

%61
现在蛊仙神念穿梭进来,方源和他本人进行交流,却是大有希望。

%62
有很多内容,就可以拿出来谈了。

%63
“我需要神鹿果,是为了喂养仙蛊。”

%64
“所以,我们若是能够达成交易,绝不会只是这一次。”

%65
“怎么样?我出一千三百块仙元石。”

%66
方源劝说道。

%67
那位蛊仙犹豫了一下:“你我第一次接触,空口无凭,我该相信你,还是不相信?呵呵呵,算了,一千四百块仙元石,咱们就可以成交。”

%68
“做成三单这样的买卖之后,价格就按照一千三百块来。你看呢?”

%69
显然,这位蛊仙不是笨蛋,也很精明。

%70
方源却是拒绝:“我若是手头充裕,不会与你这样讨价还价。一千两百五十块仙元石吧。”

%71
那位蛊仙苦笑:“也罢。若非我急着周转,神鹿果又卖得慢,我是不会答应你这样的价格的。”

%72
一手交钱,一手交货。

%73
交易达成,方源松了一口气,金刚念仙蛊的喂养问题,是暂时解决。

%74
接下来,轮到卜卦龟背仙蛊了。

%75
这只仙蛊吞食海藻。

%76
当然不是普通的海藻,而是古墨海藻,六转仙材。

%77
这种海藻浑身漆黑如魔,一旦表皮破裂,就会将周围的海水渲染成一片乌黑的墨水。

%78
方源现在手头紧,还是先打算收购,囤积一部分古墨海藻,用来喂养卜卦龟背仙蛊。

%79
等到将来,再进一步打算。

%80
从这点上比较,仙级日蛊就讨喜得多。因为它同样吞食光阴河水,不需要蛊仙考虑它的喂养问题。

%81
宝黄天中,贩卖古墨海藻的蛊仙,却是比神鹿果要多些。

%82
但方源先后接触下来,却是没有达成交易。

%83
在价格方面没有谈拢。

%84
不过他也不着急,距离卜卦龟背仙蛊的下一次喂养,还有不少时间。

%85
“幸好龙鳞海域建设成了,只要照此发展下去,将来手头会越来越宽裕的。”

%86
“接下来,就是建设盘丝洞窟,大规模豢养长恨蛛群了。”

%87
方源考量之后,长恨蛛的市场也很大,尤其是西漠那块。虽然没有龙鱼的需求那么广泛,但需求量也很多。

%88
不过现在,方源手头紧缺,还需要时间积累。

%89
积累出一笔资金,才能开始建设。

%90
“照此情况,预期一个多月后,我就能结余出第一笔资金,可以建设盘丝洞窟了。”

%91
“希望那个时候,武家平静下来,能让我调回到超级梦境中去吧。”

%92
武遗海的身份,限制了方源的行动自由。尤其是武家这种情况下,他也不好表态。

%93
这一个多月,武庸统领武家,对抗各大超级势力的为难。其中有得有失,大体上是堕了武家的威名,但是实质上的利益损失却是很少的。

%94
因此,这种情况还要持续下去。

%95
武庸想找回场面,挽回声名,其他超级势力各大家族,则不甘心,想要夺得一些实质性的收益。

%96
正道之间的交锋,和魔道不同。

%97
魔道蛊仙之间往往猛烈突然,正道之间却是细雨柔绵,常常旷日持久,恰到好处。尤其是超级势力之间,都是家大业大,很难让两个势力,不管不顾其他,相互死掐,斗个你死我活。

%98
方源自顾自的修行生活,并没有持续多久。

%99
几天后,他接到了武庸的另外一项任务。

%100
“山螺母?驱山老怪?”

%101
方源眉头皱起。

%102
他不是对完成这个任务没有信心,而是这种趋势,有点不太妙。

%103
上一次,方源解决了广寒峰的事情,其实保留了实力和手段,得到的结果差强人意。只是五年之约,而且还自己贪污了广寒峰中的资源。

%104
虽然说,这是正道的潜规则,但是他也将将碰触了底线。

%105
这当然是方源有意为之,给武庸留下不好的印象。上位者当然不喜欢用这样的人,所以早点打发方源去那超级蛊阵,正合方源的愿望。

%106
但武家现在,的确是人手稀少,地盘太大。山螺母的事情,属于突发,让武庸不得不启用方源,去解决这个难题。

%107
备注:惭愧,今晚只一更。六月份出了不少事情,生活方面一些破事,牵扯了我许多精力。但我不得不去做。唉……关键是问题还在,还待解决。

\end{this_body}

