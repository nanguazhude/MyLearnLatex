\newsection{钓年兽}    %第四百零八节:钓年兽

\begin{this_body}



%1
这片天空极其辽阔,并且充斥着鲜红的光,仿佛朝霞漫空。

%2
原本这里空无一物,但此时此刻,却漂浮着一座巨大的仙道蛊阵。

%3
这座仙道蛊阵,逸散出磅礴的宙道气息,很明显是一座宙道仙级蛊阵。

%4
在这座大阵里面,方源、白凝冰、黑楼兰、白兔姑娘、妙音仙子、影无邪都已经严阵以待。

%5
这座宙道仙级蛊阵,已经持续催动的了很长一段时间。

%6
它是以七转年蛊为核心,八转宙道仙蛊似水流年为辅助。并非所有的仙阵,都是以品级最高的仙蛊为核心,就像眼下这座仙道蛊阵,八转似水流年仙蛊反而是辅助地位。

%7
这座仙道蛊阵正是来源于黑凡真传。

%8
主要的效用,就是勾连仙窍当中的光阴支流,向光阴支流中散发出七转年蛊,以及八转似水流年仙蛊的气息,用来引诱太古年兽的注意。

%9
这座仙阵被黑凡命名为:太古年兽钓来阵。

%10
名字有些意思,别人是钓鱼,黑凡是钓年兽,并且钓的还是太古年兽!

%11
黑凡自从开创了八转仙蛊似水流年之后,便深知这只仙蛊的弊端,那就是蛊虫气息对于太古年兽,具有极其强烈的诱惑。很有可能,让太古年兽顺着光阴支流闯入到自家的仙窍中来。

%12
为此,黑凡便殚精竭虑,想了种种处理方法。

%13
他首先想到的,便是方源之前坑害凤九歌,令其落入光阴长河中的那座宙道仙级蛊阵。此阵可以自爆光阴支流,让仙窍中时间完全静止,封堵住太古年兽的路。

%14
这样一来,太古年兽就无法闯进仙窍中来了。

%15
不过,此法有巨大的弊端。

%16
仙窍中时间一旦静止,各种资源就都无法生长。而蛊虫在仙窍里面,也无法催动。毕竟催动任何蛊虫爆发威能,哪怕只需要一瞬间,都是时间。而任何的过程,都需要时间。

%17
所以,一旦采用此法,蛊仙就得将蛊虫从仙窍中挪移出来,依附在自家肉身上。

%18
这就相当危险。

%19
尤其是在战斗当中,很容易就会被强敌破坏。

%20
黑凡才情卓绝,开创这个方法,只是留作最后底牌。他当然不会满足这样的方法,于是继续探索。

%21
他很快发现,只要自家仙窍中有了一头太古年兽生活着,那么在其他太古年兽的感知中,就会认为,他的仙窍成了其他太古年兽的领地。即便是似水流年仙蛊气息诱人,别的太古年兽通常都不会虎口夺食。

%22
于是,黑凡便首先设想法门,来伪装出太古年年兽的气息。

%23
可惜,他的变化道境界非常普通,一直都没有成功。

%24
然后,他考虑自家培养出太古年兽来。

%25
但是他的洞天,可不算宽广,各种资源充斥,很难开辟出一个适合太古年兽生活的环境。并且,他要培育出太古年兽,至少得从上古年兽培养。宝黄天中可没有太古年兽贩卖。

%26
然而这个过程,耗时太久。那个时候的黑凡,却已经时日无多。

%27
两种路途不通,黑凡便设想另外法门,他打算利用似水流年仙蛊召唤出一头太古年兽,为自己驱策。

%28
于是他设计仙道杀招。

%29
他的奴道境界有宗师级,不像变化道境界那般不堪,于是便有了年兽召来。

%30
不过这个杀招,只能召唤上古年兽,太古年兽却不好招揽。

%31
在他生命的最后时期,他创造出了太古年兽钓来阵。

%32
这座仙道蛊阵,可以将太古年兽引诱进来,但却无法驾驭操纵。所以只是“钓来”,并非“召来”。

%33
钓来的太古年兽是不受控制的。

%34
黑凡真正的设想是,召唤出一头太古年兽,并成功地操纵它驾驭它。

%35
直至去世,他都没有达到这个目标。

%36
这不得不说是一种遗憾。

%37
或许给他更多一点时间,他就能成功。

%38
红颜敌不过时间,事实上,才俊英豪也是同样如此。

%39
方源若有时间,他也会循着黑凡的设想,继续朝这条路线进发,假以时日,定然会研究出直接召唤太古年兽的仙道杀招。

%40
这其实才是真传的继承。

%41
后继者继承了前人的真传之后,并非只是一味地照搬,而是在修行的过程中,不断完善真传中的部分内容,或者添加进自己的东西,然后再度流传下去。

%42
真传的继承,不仅是索取,更有付出。

%43
不过,现在方源并没有时间去完善真传。

%44
他要做的只是借助这座太古年兽钓来阵,钓上来一头太古年兽,然后运用百八十奴杀招,强行收服。

%45
“黑凡若是有百八十奴杀招,必然已经收取了太古年兽。”

%46
“我此去光阴长河,危机四伏。不仅是光阴长河本身的凶险,天庭方面极可能有着布置。”

%47
之前方源陷害凤九歌,将其放逐到了光阴长河中。虽然最终,凤九歌逃了出来,但方源并非做了无用功。

%48
“凤九歌能逃出生天,或许是他个人能力,也或许是天庭在光阴长河有所布置。”

%49
“不管是哪一种,我进入光阴长河,最坏的可能便是要遭遇到他们的阻击。”

%50
“而我在光阴长河中,毫无地利,并不熟悉在长河中作战。非得有太古年兽护航,才有生命的保障。”

%51
方源谋划详细周到,不可能只身涉险,当然得有所准备才是。

%52
渣渣渣渣……

%53
就在这个时候,一股刺耳的声音,从光阴支流中传达过来。

%54
“看来,终于是有太古年兽被诱惑了!”方源双眼骤亮,连忙催动仙道蛊阵。

%55
蛊阵当中,光阴支流显现出实体。

%56
一个巨大的猴子,正透过这股光阴支流,观察蛊阵这边。

%57
它巨大的眼珠子,都有成年象的大小。透过光阴支流的潺潺流水,就像是透过一个小洞,观察门后的黑暗景象。

%58
在这一片黑暗当中,这头太古年猴根本发现不了方源等人,它的目光本能地被似水流年仙蛊吸引过去。

%59
一发现这只八转仙蛊,太古年猴就食欲大增,口水横流。

%60
年兽本来就是以年蛊为食物,而似水流年仙蛊却是可以产出年蛊。太古年猴见到似水流年仙蛊,就像是饿了百年的大肚馋鬼,见到了满汉全席!

%61
渣渣渣渣!

%62
它在光阴长河中躁动起来,兴奋地抓耳挠腮。

%63
然后它深呼吸一口气,开始试图钻入方源的至尊仙窍里来。

%64
年兽本身就有着从光阴长河中钻到世间任何角落的能力。方源之前催动年兽召来杀招,每一次都是从光阴长河中召来年兽。这些年兽都是直接钻出来,及时赶赴战场。

%65
只要有光阴长河的流淌,或者是光阴支流的所在,年兽都能进入其中。

%66
眼下,方源的至尊仙窍也不例外!

%67
只是要进入这里,可比进入五域两天中要困难一些。

%68
空间破开,就在太古年兽钓来阵的不远处,猛地伸出两只巨大的猴手。

%69
然后,这对毛茸茸的大手,用力向两边扒拉,撕扯开一个巨大的空洞。体型庞大的太古年猴缩头缩脑地钻了进来。

%70
“我的至尊仙窍虽然还只是福地,但完全可以承载太古年兽了。只是对于太古年兽而言,这个笼子未免有些小巧了,进出并不方便。”

%71
见到这样一幕,方源心中泛起喜悦的情绪。

%72
进出不方便很好,待会打起来,太古年猴要是想要逃跑,也就不容易了。

%73
太古年猴彻底钻了进来,身后的空间破洞旋即收拢至无。

%74
渣渣渣!

%75
太古年猴兴奋地吼叫着,鼻子不断抽动,感受到似水流年仙蛊的强烈气息,直接向仙阵扑杀而来。

%76
轰隆!

%77
仙阵有着强大的防御威能,被太古年猴狠狠一撞,却是岿然不动。

%78
当初黑凡设计此阵,自然会考虑到此时的情境。

\end{this_body}

