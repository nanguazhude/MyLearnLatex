\newsection{万我仙蛊!}    %第五百五十二节:万我仙蛊!

\begin{this_body}



%1
琅琊福地,云盖大陆。

%2
琅琊地灵深呼吸一口气,然后低吟出声:“大阵起!”

%3
像是万千蜜蜂疯狂起舞,嘈杂的嗡鸣声中,炼道超级大阵缓缓催起。很快,就有一大片的白金光泽,扩散而出,充盈这方天地。

%4
“入阵!”琅琊地灵又一声下令,与此同时,他的身影首先进入仙阵之中。

%5
随后,毛六、毛三、毛四、毛七等诸多毛民蛊仙,也接连入阵。

%6
方源则站在仙阵边缘,作为旁观者。

%7
这一次炼蛊,并不是由他亲自操刀,而是交给了琅琊派的地灵以及毛民蛊仙们。

%8
方源虽然有准无上的炼道境界,但炼道的基本功,实际操纵的水准,和琅琊派的蛊仙们是不能相比的。

%9
“若换做平时,或许我还能参与炼蛊。但此次炼蛊,琅琊地灵却是不惜动用长毛老祖当年亲手布置的炼道仙阵。这道仙阵,却是非毛民蛊仙不可的。我一个人族蛊仙,进去之后,就要被仙阵炼化绞杀。”

%10
琅琊福地底蕴非常深厚,这一次,又冒出长毛炼道大阵出来。

%11
当然,依凭方源的手段,完全可以利用变化道的造诣,自己变化成毛民蛊仙。但恐怕还是琅琊派想要坚守这份秘密,不愿意向方源公布开放。

%12
“我之前利用幽魂真传,和琅琊地灵互换了琅琊真传。但显然,这座长毛炼道大阵,就是琅琊真传中压箱底的精髓,是不拿出来交换的。就像幽魂真传中的引魂入梦等杀招一样,我也是敝帚自珍。”

%13
方源站在一旁观察,眼中精芒烁烁不定。

%14
即便琅琊派将方源排斥在外,但对此,方源心底其实非常满意。

%15
面对天庭强大压力,琅琊派上下都急了,不惜开启长毛炼道大阵,尽快地为方源炼制出万我仙蛊来。方源加入琅琊派,可算是自己人,现在增加方源的战力,对他们搬迁琅琊福地大有益处!

%16
长毛炼道大阵徐徐催动,光辉不断变化,起先是白金光泽,随后转变成橙金,再之后又转变成墨金。

%17
墨金的光辉中,又开始闪烁出点点星芒,璀璨耀眼。

%18
仙阵的嗡鸣声却渐渐小了,几个呼吸之后,变得静谧至极,没有一丝声音。

%19
入阵的数位毛民蛊仙还有琅琊地灵,各自站定,围成斩成一个大圈。

%20
琅琊地灵首先出手,催动仙阵,开启仓门,放出一团早就处理好的仙材。

%21
这份仙材好像是霜打过,焉了的向日葵花,刚刚出现在仙阵中央,就化为屡屡的烟雾,汇入道墨金的光辉中。

%22
方源目光沉凝,这是第一个关键步骤,须得烟雾汇同,完美融合在墨金光辉中方可进行下一步。

%23
方源并没有太多的心理压力。

%24
若是这一步失败,他损失并不大,完全可以重新再来。

%25
不过,半个时辰之后,仙材彻底化为烟雾,和墨金光辉之间融合得相当完美顺利。

%26
仙阵中,墨金光辉不断凝聚浓缩,从之前渲染半边天的规模,完全凝聚成房屋大小,被拘束在长毛炼道大阵的中央。

%27
待得墨金光辉中,那些点点星芒完全转变成冰星,琅琊地灵便催动仙阵,手法一变,开始第二步骤。

%28
时间不断地流逝,一份份仙材被琅琊派的毛民蛊仙们,有条不紊地投入仙阵中央,不断地熔炼于一体。

%29
三天三夜之后,毛三手掌一摊,开始向仙阵中央投入蛊虫。

%30
他首先投入进去的,是自力更生蛊。

%31
这种蛊虫,只是三转程度,形如蟑螂,体型扁平,通体黑褐色。长丝触角,分有前后双翅。

%32
自力更生蛊乃是力道蛊虫,拥有治疗效用。蛊师的力量越强,治疗效果就越佳。

%33
这是珍稀蛊虫,凡人蛊师常常苦寻一只都得不到,但现在毛三投入的却是数百只。

%34
自力更生蛊的数量,当然有严格的规定。

%35
因为这牵扯到道痕的量,炼制仙蛊,就要在这方面锱铢必较!

%36
当所有的自力更生蛊都投放进去后,琅琊地灵猛地大喝一声,狂催长毛炼道大阵。

%37
一时间,大阵中,升腾起熊熊火柱,成百上千的火柱,高达数丈,在墨金光辉中,灼灼燃烧。

%38
澎湃的气浪,不断向四周扩散,但都被炼道大阵完美地阻截下来。站在仙阵圆边上的毛民蛊仙们,都感觉不到气温的一丝提升。

%39
长毛炼道仙阵非常优异!

%40
橘红色的火柱,一直燃烧了二十四个时辰,这才缓缓停歇下来。墨金光辉已经被灼烧彻底,连同那些自力更生蛊,化为一团马车大小的古怪石头。

%41
又等了一会人,见到石头彻底凝固,方源松了一口气。这炼制万我仙蛊的第二步关键,也成功地渡过了。

%42
随后,琅琊地灵等人开始休息,回复状态,毛四、毛六则不停歇,继续催动仙阵,对这石头进行雕琢。

%43
过了两三炷香的功夫,石头被彻底碾磨,化为一滩碎石块,散落在地上。

%44
毛三停下休整,取出饮刃酒,对这些碎石进行浇灌。

%45
这饮刃酒正是方源不久之前,斩杀蒙屠所得的战利品。此时用了,对症下药,立即将所有的碎石,都消融成一团团的灰色烟气。

%46
这些烟气各个漂浮着,并不相互融汇,而是凝缩成一个个的烟球,像是小灯笼般,静静地悬停在半空中。

%47
毛四、毛六开始休息,毛三取出全力以赴蛊。

%48
这种蛊虫,宛如独角仙,当然都是凡蛊。一只只都有成年人的手掌大小,沉甸甸的。它们呈长椭圆形,披着装甲般的背壳,脊面隆拱。头部是一只雄壮有力的独角,显得厚重而又威武。它有数对长足,均强壮有力,通体深棕褐色,散发着油亮的金属光泽。令人一见便知此蛊不是寻常之物。

%49
方源目前还未炼成全力以赴蛊的仙蛊,这些都是凡蛊,从一转到五转,各层转数都涵盖进去。

%50
蛊虫冲入烟气当中,并不交融变化。

%51
这时琅琊地灵开始出售,他打开仓门,取出处理好的仙材浮生火。

%52
在这种透明的火焰中,大量的全力以赴蛊和灰色烟团,才开始缓缓融汇。

%53
这一步耗时很长。

%54
足足半个月后,融汇才完成,不管是浮生火、全力以赴蛊还是灰色烟团都消失不见,取而代之的是蒙蒙细雨。

%55
这阵细雨,在炼道仙阵的约束下,被禁锢在大阵中央,不断淅淅沥沥地下着。

%56
到达这一步,第三步关键也就成了。

%57
时间变得紧急,细雨并不稳定,本身正在迅速地削减。

%58
所有的毛民蛊仙还有琅琊地灵都各自出手,每个人手法不同,负责的炼蛊内容也大致不一。

%59
正反狙神针等等仙材,接连不断地抛洒进去。除此之外,还有大量凡蛊。

%60
方源这一次推算出来的万我仙蛊方,因为不消耗任何的仙蛊,所以采用大量的仙材和凡蛊作为替代。

%61
这样做,仙蛊是没有任何损耗的,但是仙材和凡蛊的消耗就暴涨了数十倍,同时炼蛊的工作量也疯狂上涨。这也是琅琊派不惜开启长毛炼道大阵的原因之一。

%62
随着时间流逝,一步步的关键步骤,都无惊无险地迈了过去。

%63
琅琊福地时间,足足持续了两个多月,炼制万我仙蛊的过程终于迎来最后一个关卡。

%64
方源自然有宙道手段,但这时间却是没办法浓缩。宙道缩减时间,本质上是刻印宙道道痕,加以影响。而在炼蛊过程中,宙道道痕冒然增添进来,会极大地影响炼蛊的成果。

%65
能够浓缩到两个多月时间,已经是方源目前的能力极限。他早已将宙道手段应用了进去,并且将增添的宙道道痕,也作为炼蛊的材料进行完美清理。

%66
仙阵轰鸣剧震,冲霄的彩光漫天遍野。

%67
琅琊地灵长舒了一口气,其余毛民蛊仙各个脸色疲惫,却都透露出喜色。

%68
最后一步也成功了。

%69
彩霞陡然收敛,凝聚成一只七转仙蛊。

%70
这只蛊虫,体型修长,长达数丈,它形如蜈蚣,宛若赤铜浇筑,浑身上下散发金属光泽。头部口器狰狞,额前一对长须甩摆如烟,左边有五千足,右边也对称有五千足,便共有了万足。

%71
七转仙蛊万我!

%72
“没想到第一次炼蛊,就成功了!”

%73
“哈哈,我们终于炼成了。”

%74
毛民蛊仙们欢叫起来,脸上都是喜色。

%75
万我仙蛊被拘束在炼道大阵中央,因为不是方源主持炼制,此时还是无主之物。

%76
不过很快,方源在炼道仙阵的帮助下,极其轻松地将其炼化。

%77
“七转仙蛊,一次性就炼成,哈哈。”方源也是满怀喜悦之情,这种事情在以前根本就不敢想象!

%78
五成的成功率,实在是太高了。

%79
方源试着灌输仙蛊,万我仙蛊得到仙元之后,立即浑身冒光,从光辉中暴射喷涌出无数方源的力道虚影,正如之前万我杀招一样。

%80
“威力只有稍微一些减弱,毕竟万我杀招所需的蛊虫太多了。”方源感悟其中变化,“仙元的损耗却是极大地降低了,同时催动起来,极其方便,只需要一个念头。而之前杀招,不管多么熟练,也是需要时间酝酿,心神的维持和消耗也极多!”

%81
方源试演一番,对这万我仙蛊满意至极!

\end{this_body}

