\newsection{挑选任务}    %第六百三十五节:挑选任务

\begin{this_body}

岛屿虽小,但是蕴藏着的修行资源却不少。

这种密度、储量显然是人为栽培。

方源在内的八位蛊仙一番搜刮后,又进行了分配。方源扮演的乃是楚瀛,变化道的蛊仙,因此收获了不少变化道的仙材。

但因为变化道以一道映射万道,这类仙材种类繁多,包含广阔。比如魂兽的魂核,本身是魂道的仙材,但若是有变化道的蛊仙专门变作魂兽的话,那么魂核也能算是变化道修行的资源了。

方源以变化道为遮掩,从中重点收取了许多宙道仙材。

曾落子状似随意地道:“楚瀛仙友难道有宙道的变化吗?怎么在外界幻境的时候,不见你用?”

方源便笑:“我这是讨巧了。诸位想必都知道方源那魔头吧。这些宙道仙材可以在宝黄天中,换取不少好东西呢。毕竟我若是和诸位竞争,未免也伤了和气。”

“哈哈哈。楚瀛仙友好想法,我怎么就没想到呢。”土头驮大笑起来。

曾落子眼中精芒一闪,不再言语。

经过外界那一场幻境的惊吓之后,蛊仙们的心情振奋好转起来,毕竟这些收获可是不菲的。

尤其是庙明神并未偏袒自己,平均分配,使得队伍中的氛围相当融洽。

半天之后,群仙再次齐聚在八转仙蛊屋功德方尖碑前。

“按照功德碑上的记载,这些资源应当都是乐土仙尊当年布置下来,为了就是让不愿意接受任务的蛊仙也不白来一趟。乐土仙尊真是好心肠啊!”童画感慨不已。

鬼七爷沉吟道:“现在我们所有人都出过手,一一试探过,这座小岛的确不能强行突破而出。”

“还有这座功德碑,也不是我们能够破解得了的。”蜂将接着说。

群仙一阵沉默。

乐土仙尊的布置,岂是那么容易破解的?尤其是这些蛊仙,除去方源之外,都只有七转修为。尽管有一些人乃是七转中的精英强者,但是面对仙尊的手段,还是太不够看了。

“不知庙明神仙友有什么真知灼见呢?”童画问道。

庙明神微微摇头:“我对乐土真传也是一无所知,和诸位别无差别。在我看来,既然我们不能强行破解了乐土仙尊的布置,无疑就只有两条路。第一条就是不接受方尖碑的任务,在这座小岛上呆足三百天,时限一到,应当就会传送出去吧。第二条则是按照碑文上的指示去做。”

群仙面面相觑。

毫无疑问,小岛上的资源都已经被瓜分完毕了,谁想在这里傻傻地待上三百天?

众仙因此也都尝到了甜头,对于乐土真传兴趣更大了。

在他们想来,小岛上的修行资源只不过是大餐前的开胃小菜,由此更可见乐土真传的价值。

“可是要接受任务,必须要得到功德碑的承认。这个过程有些类似于超级家族的命牌蛊、魂灯蛊,但更具强大的约束性。若这是一个陷阱的话……”

“世间之事,向来都是利益和风险具备的。我们此行来探索乐土真传,不就是来冒险的么?我已决定接受任务。”

“既然这是乐土仙尊布置下来,应当是没有问题的。真传就在眼前,若是在此驻足不前的话,将来恐怕会后悔。”

……

一番交流下来,在场的八位蛊仙没有任何一人退缩。

于是按照碑文上所述,八仙轮流走上前去,将手掌紧贴在功德方尖碑上。

一阵华光闪过,八仙的姓名就都由全新的碑文显现而出,在原先的空白地方,形成了一个排名的表单。

方源的名字赫然在内,不过显示的乃是“楚瀛”二字。

这让方源暗中松了一口气。

这座功德碑他探查过,发现是以土道、音道为主,不是他能够破解得了的。当然,若是强行摧毁,依照方源的战力也有可能。但这样做风险太大,不如继续伪装下去,和这些人一同探索下去。

蛊仙世界有着无穷的手段,这座功德碑可是八转仙蛊屋,又是乐土仙尊的手笔,显露出蛊仙的真名大有可能。幸好没有这一层威能,否则方源身份暴露,恐怕要对周围人痛下杀手了。

毕竟他的身份太过敏感。难保这些人没有什么诡异的手段,就算没有威胁,引来天庭的力量也是极不好的。

“不摸清楚状况,还是不要轻易对这些人下手。毕竟这是乐土真传,他讲究的是得饶人处且饶人,是仁慈和宽容。我若是轻易杀戮,恐有不妙的情况发生。”

八仙都已名列功德碑上,便开始研究这些任务。

碑文记载的任务,只有十个。

每一个任务,都是言简意赅,需要蛊仙进行揣摩。

比如排在第一个的任务剿除海雕城周围的水怪。

水怪和魂兽类似,都是道兽,形态千奇百怪,死后绝大多数的身躯都会消散,只留下水核、魂核。这是仙材,蕴含着丰富的水道、魂道道痕。

什么层次的水怪,是荒兽还是上古荒兽,并无详述。

海雕城?哪里的海雕城,在何处?也无指示。

剿除水怪还可揣摩一些,但其余的任务就有些千奇百怪。

比如为鱼圆岛建设水井,又有为三纹部采集药材,还有治理元泉、保护商队、修补大阵等等。

八仙仰望功德碑,琢磨良久后,曾落子叹息一声:“我的信道手段对于这座八转仙蛊屋根本无用,看来我们选择什么任务,全凭运气了。我先来吧,我就选第一个任务剿除水怪了。”

其他人一阵沉默。

曾落子虽是这么说,但其实各个任务也有区分。

按照常理推断,任务难度越高,获得的功德应当就越大了。水怪至少是荒兽,剿除水怪显然是难度颇高的一个任务,对应的功德奖励应当不俗。

至于治理元泉、保护商队等等这些任务,明显是凡人层次,难度高不过剿除水怪这类的任务。

曾落子如此选择,无疑是先抢走了一块肥肉。队伍中的氛围顿时有些改变。

曾落子沉默不语,其实也有一些紧张。

这时庙明神点头,算是答应的意思,然后他道:“你们先选吧,我最后一个。”

此话一出,蛊仙们看向庙明神的目光,又变了一些。

庙明神摆手,又道:“诸位不必高看我,我只是觉得乐土仙尊既然做出如此布置,怎可能没有想到公平二字?这些看似简单的任务,说不定并不是那么简单的。”

对于庙明神的这番推论,群仙中有人微微点头。

童画开口:“那我选择第二个任务。”

土头驮也随即做出了选择。

他们的选择都和曾落子一致,明显是看上去就比较困难的任务。

土头驮选择完,还剩下一个任务,比较困难的样子。其余的任务则都显得平庸普通。

而剩下的人中,还有庙明神、方源、鬼七爷这三位七转。

“接下来……还请楚瀛仙友选择吧。”庙明神看了一眼,笑着道。

蜂将、花蝶女仙站在他的身边,始终没有二话。

“慢着。”鬼七爷这时却道,“诸位仙友,庙明神大人,实不相瞒,我第一眼时就看上这个任务了。”

群仙顿时神情各异。

庙明神转头看向鬼七爷,带着一种愕然、批判的目光。

但鬼七爷却不看到,而是目光幽幽,直视方源。

“这是柿子要捡软的捏么?”方源心中呵呵一笑,直接开口,“既然鬼七爷能有这样的眼缘,那就说明这任务和鬼兄你有缘啊,我可不夺人所好。嗯……就选择第七条吧。”

众人视之,只见这第七条任务是去采集海底深处的地沟黑油,这向来是凡人蛊师干的事情。

庙明神忙道:“楚瀛仙友……”

“庙兄不必再谦让了,你有这样的同伴实在是令人羡煞。这也是你应得的,若非你的缘故,我怎可能进得来这里呢?”方源哈哈一笑。

鬼七爷看向方源的目光,顿时缓和了许多。

没错,他强要最后一项任务,乃是为了庙明神,并非是为了他自己。

但也正如方源所说,诸仙能来到此地,关键就是庙明神。他这样的功劳,的确是应得的。

庙明神还是推辞,但方源一力坚持。

庙明神无奈,只有接受。

片刻之后,每个人都选择了一项任务。功德碑暴射出冲天的光柱,蛊仙们一一步入光柱,被迅速传送出去,当场消失。

方源只觉视野骤变,再一眨眼,就来到了一处陌生的海岛上。

“我乃是八转蛊仙,但那功德碑传送过来毫不费力。它似乎是这座仙窍洞天的核心,大有奥妙,果然不愧是乐土的手笔啊!”

对于这些任务,他和庙明神有着类似的推测。当然推让那个任务,并非方源谦让,其实是对庙明神的试探。

方源等人走后,只剩下庙明神和鬼七爷。

“鬼七,你啊……”庙明神叹息。

鬼七爷拜道:“还请大人恕罪。”

“你何罪之有?只是你未免看轻了楚瀛这人。他是第一个在幻境中阵亡的人,你就以为人家的本事不强了。”庙明神摇头。

“大人你难道觉得楚瀛深藏不露吗?”

庙明神又摇头:“我也没有看出什么。他隐藏实力的可能性并不大。只是在这个陌生的环境下,我们还是要先团结起来,尽量避免内斗。”

“是,在下明白了。”(未完待续。。)

\end{this_body}

