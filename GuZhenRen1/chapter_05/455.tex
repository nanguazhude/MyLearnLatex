\newsection{天机蛊成!}    %第四百五十六节:天机蛊成!

\begin{this_body}

石忠在这方面浸淫很久,从他的血块中,就能分析出自己有无进步。

“虽然这一次也有进展,但是距离真正成功,我感觉还差不少关键的步骤呢。需要努力啊!”石忠这方面经验非常丰富,隐隐产生了一种直觉。

他算了算时间,又到了仙蛊喂养的时候了。

石忠继承了他这一脉的遗产,掌握着数只仙蛊,这无疑让石家的其他蛊仙眼红。但可惜这是石家的传统,也是规矩,牢不可破。

石忠掌握这些仙蛊,也是他被石家蛊仙针对的原因。

他检查了一下库存,顿时有些头疼。

“有点麻烦,缺少了一些食材。我忘了之前炼财富蛊,有了短缺,就将囤积好的一部分仙蛊食材,拿出去卖了。”

“这个时间……要再筹集一批食材出来,比较困难了。不如收购一些龙鱼,进行替代,先喂养了仙蛊再说吧。”

想到这里,石忠也不犹豫,当即沉入心神,直接探进宝黄天中。

“龙鱼……”他目的明确,很快就发现不少蛊仙正在贩卖龙鱼。

作为买家,他自然要货比三家。

龙鱼生意不像是年蛊,终年都能贩卖。因为龙鱼繁衍不挑时候,所以终年都是它的上市期。

石宗优先去了尤婵的摊位。

众所周知,尤婵乃是宝黄天中,龙鱼生意的第一人。

不过她并不只是贩卖龙鱼,还有其他物产,但多以鱼类为主。

石宗以往若是要买龙鱼,基本上都是在尤婵这里买,因为尤婵的信誉有着相当的保障。

这一次若不出意外的话,他也会选择在这里收购一批龙鱼。

看完了尤婵手中龙鱼的货色,石宗没有急着购买,而是选择去其他地方逛一逛。

于是他很快发现了方源的摊位。

“咦?这里也有人贩卖龙鱼吗?货量不少啊。”石宗巡视一番,感到意外。

他仔细查看,发现这个摊位上的龙鱼,不仅货量巨大,而且成色也好,和尤婵所卖的龙鱼不相上下。

“而且这个摊位上,人气更旺一些。”石宗用神念扫视一番,发现这里始终围绕着数位蛊仙意志,并且还有一些蛊仙神念,正在不断地沟通。

“没想到我闭关炼蛊许久,宝黄天中的龙鱼市场,已经出现了一个强有力的竞争者了。”石宗正想着,这时一道蛊仙的神念主动凑上来,与其沟通。

“这位仙友,你也是想来收购龙鱼的吗?不如我们一起拼凑着购买,能够便宜不少呢。”这个神念如此说道。

“哦?这又是怎么回事?”石宗便问。

那股神念便告知道:“仙友不知晓吗?这里贩卖龙鱼,有着不一样的规矩。只要龙鱼数量多出标准,卖价上就能便宜。这里面具体有各种标准,但总的来说就是一句话,你买的越多,就越便宜。”

石宗闻言,顿时心中暗赞:“这个方法真的不错。”

然后,他旋即想到自己:“或许我将来贩卖,也可用这样的方法。”

石宗越是回味,越觉得此法很妙。

虽然本质上,是降价了,但是规模上去,薄利多销,便保证了自身的盈利。和之前年蛊角逐,那种直接降价的方法,显得巧妙得多,也柔和得多。

“采用这种方法,不仅是贩卖的龙鱼能够更多一些,而且还能打出名气。若真有人要买龙鱼,蛊仙为了便宜一些,必定会邀请友人或者直接和陌生人搭伙,如此一来,就是为这位卖方宣传龙鱼生意了。”

石宗越想,越觉得这个贩卖龙鱼的蛊仙,脑袋瓜很是灵光。

“难怪这里的蛊仙,盘踞得比那一边的龙鱼老字号,多得多了。”

“嘿嘿,这里面说不定就有卖家布置的托儿,并非是真的来收购龙鱼,只是炒热氛围。”

石宗顿时对这位卖家起了点兴趣。

“也罢,不如这一次就在这里买好了。有便宜不占么!”想到这里,石宗便对那股陌生声沟通,同意搭伙。

对方自然欢喜,双方商量之后,一起合购了十万头龙鱼。

买了之后,当即就在宝黄天里直接分掉,双方皆满意而归。

暂且不提石宗回去如何喂养仙蛊,单说那个和石宗合购的蛊仙神念,在得到了这批龙鱼之后,立即提货,将其拿出宝黄天,然后便送到了至尊仙窍里。

真叫石宗猜对了,和他一起合购的正是卖方的托儿,幕后之人却非方源,而是影宗的其他蛊仙。

但这层关系,就算有人猜得,在宝黄天中如何验证?

验证不了。

其实这个方法,并非方源独创。

石宗只是以炼蛊为主,对于宝黄天的市场关注不够多,事实上这种法子很早之前就有人用了。

方法新旧无关紧要,最重要的是有没有效果。

不管别人如何,总之方源采用这个法子,很有效果。

这些天来,随着他不断加量,更多的龙鱼投放到了宝黄天市场中,终于惹来了诸多蛊仙的注意。

庞大的龙鱼货量,展现出了方源的实力。买家当然更愿意选择实力雄厚的卖方。

另一方面,方源采用了正确的方法来贩卖,使得他的龙鱼生意的口碑,也开始广为传颂。

方源的龙鱼成色很好,又有食道仙阵辅助,接下来的成色只会越来越好。

起初来向他购买龙鱼的蛊仙,不仅人数少,而且每次买下来的龙鱼数量也不多。但是渐渐的,越来越多的蛊仙向他收购龙鱼,同时龙鱼买卖的数量,也在急速增多。乃至于,已经有了回头客。

而作为方源此次龙鱼生意最大的对手,东海的尤婵,此刻却仍旧不急不缓地贩卖着她的龙鱼,不为所动。

“尤婵不为所动是对的。因为短时间内,我采用这种方法,并不能对她构成威胁。反过来,她若是采用相同的策略,就会大大危害她原本的收益。毕竟向她收购龙鱼的蛊仙,要比我的多得多。”方源心中思量。

尤婵在这方面表现得非常沉稳,有着大将风度。

“不过再等一段时间,她就应当着急了。呵呵。”方源自信十足。

他布置下的食道仙阵,采用了多只仙蛊,耗费仙元,不断运转,自然非同小可。

只是这个栽培之法,需要龙鱼繁衍出几代,乃是十几代,才能见到成效。

尤婵不着急,方源更不着急,越到后期,他在龙鱼生意上的优势就越大。

将神念从宝黄天中抽取回来,方源来到炼蛊大厅,再次见到毛六。

毛六形容枯槁,为了方源炼蛊,他真的是竭尽所能,殚精竭虑。

现在,他再次将天机蛊炼至最后的阶段。

轮到方源出手了。

白色的光辉,充斥炼蛊大厅,万丈之光,照得人睁不开双眼。

方源闭上双眼,缓步走进白光之中,直至中心位置,这才停下脚步。

一股浓郁的气息,扑鼻而来,似凡非凡,似仙非仙,介于两者之间,显然是到了蜕变的最后一步。

“可以开始了。”方源调整了一番心境后,对毛六传音。

毛六倒数几声后,将大厅中的炼道仙阵猛地撤去。

方源旋即接替!

他的脑海中无数念头喷涌而出,智道手段接连不断催动出来,他成功地接替了仙阵的作用,在他的努力下,周围的白光不断凝聚回缩,最终形成一个巨大的圆环。

圆环与地砖平行,围绕着方源,将他至于圆环的中心点上。

炼道杀招——天光轮转!

第一转。

方源心念一动,圆环就开始转动起来,速度非常缓慢。

第一转成功,第二转。

方源睁开闭合的眼帘,用目光推动圆环,圆环速度更快了一些,开始自转。

半盏茶的功夫后,第二转功成,进入第三转。

方源用鼻息推动圆环,片刻后,圆满成功。

第四转、第五转、第六转,随着时间流逝,方源稳步进展,居然进行到了最后的第七转。

“看来宗主在那一次失败后,勤加苦练,下过许多苦功。”毛六一直在旁观看,提心吊胆,心中了然。

第七转。

方源咬牙,奋尽全力,转动圆环。

此时的圆环,已经缩成脸盆大小,悬浮在他的头顶处。

轰!

一声轻鸣,圆环猛地收缩,整个圆环直接缩成了一个点。

这个点就是一只蛊。

七转天机蛊,啪嗒一声,落到方源的头顶上。

“居然成功了。”即便是当事人方源,都感到意外。

他第一次炼制失败,没想到第二次就成功了。

意外之后,惊喜的情绪旋即笼罩方源心头。

“或许也是我动用了运道手段,运气上佳的缘故吧。”

方源将天机蛊虫头顶上拿下来细看。

乍一看这只仙蛊,和蜻蜓非常类似,但细看的话,又觉得像是一段柳枝。

整个天机仙蛊非常修长,有着成年半个胳膊的长短。

它的身体很是细长,像是柳枝,十分柔软。柳枝的两侧分别是七片羽翼,薄如轻纱,透着绿意,仿佛是叶片。

当它悠悠地飞在半空之中,或者用飘这个字形容它的悠缓,更为恰当一些。

方源观察了片刻,吐出一口浊气,感叹道:“天机仙蛊终于炼成了!”(未完待续。)

\end{this_body}

