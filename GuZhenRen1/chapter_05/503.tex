\newsection{要挟房睇长}    %第五百零四节:要挟房睇长

\begin{this_body}

房家的仙道杀招接连不断,轰击在魂兽大军的身上场景璀璨,宛若万千烟火爆闪。mianhuatang.cc [棉花糖小说网]

彩光映照在房家蛊仙的脸上,这些人的脸色却都不好看。

“这魂兽虽有折损,但根本无伤大雅。它们身上的青甲居然可以连接一体,分摊伤害,好生了得!”

房家六位蛊仙久攻不下,就在这时,又从身后的仙蛊屋鸡笼犬舍中,飞出一位紫衣女仙。

“让我来!”她娇喝一声,将早已经酝酿好的仙道杀招施展出来。

咻咻咻!

一时间,天空中浮现出浓烈的毒云。

庞大的毒云投下阴影,将豆神宫以及魂兽群都笼罩住。

从毒云中飞出无数的飞雀,这些飞雀鸟喙如箭,速度奇快,转折灵便至极,乃是仙道杀招所化。

数以万千的飞雀飞入魂兽大军当中,令大军一阵纷乱。

但是过了片刻,房家蛊仙们的脸色又都沉郁了几分下去。

杀招声势很大,但是效果却不佳。

“就连房芝的毒云箭雀都是无用,毒道也没有效果,看来这青甲菜园真是毫无破绽可言了。”问津坞中,房睇长的眉头紧锁。

他微微侧头看向方源:“不知算不尽仙友有什么见解?”

方源摇头:“惭愧,在下也没有发现这杀招的破绽。不过依在下浅见,这豆神宫的表现和之前完全不同,诸多应对似乎变得很有灵智。青甲菜园杀招是一个方面,这些魂兽抵御轰炸和突击时,进退有据,配合默契,将豆神宫守得水泼不进。两者之间的变化,真的太大了。”

房睇长点头。

他也知晓方源、房安蕾之前,和豆神宫的对决景象,豆神宫只能横冲直撞,没有施展出任何的杀招,宛若莽夫一名。

但现在,这座豆神宫却是不动如山,宛若沙场智将,将营盘防御得密不透风,让房家蛊仙难以下手,久攻不下。

这一前一后的表现,的确差别很大,其中的缘由更是叫房睇长感到担忧。

“我方布置了桃花迷林,豆神宫居然连突围的举止都没有,直接摆出严防死守的态势,难道是无法挪移?亦或者就是想要拖延时间?”

房睇长更担心后一种情况。(WWW.mianhuatang.CC 好看的小说

若是对方打的拖延主意,必定是别有用心的。

念及于此,房睇长下定了决心,当即召唤外出作战的房家蛊仙:“都回来罢,利用桃花迷林强攻!”

桃花迷林乃是仙道战场杀招,能困住八转蛊仙,自然不会肤浅简单。

房家蛊仙均回归仙蛊屋中,桃花迷林开始疯长。

原本光秃秃的枝丫上,开始冒出嫩绿之色,随即绿意扩散蔓延,眨眼间就成了一片郁郁葱葱的森林景象。

一阵狂风骤起。

桃林树叶沙沙作响,掀起一声声的翠光浪潮。

浪潮一**,向着豆神宫冲刷过去。

魂兽大军节节败退,外围的荒级魂兽、上古魂兽都陆续被翠光浪潮拍击崩溃。

桃花迷林的这番变化,攻势宏大雄阔,居然直接将整个魂兽军团都击退。

这些魂兽各个嘶吼咆哮,但翠光浪潮乃是大势,单靠个体根本无法抵挡。

豆神宫中陈衣皱起眉头:“这桃花迷林有些门道。”

此刻他还在镇压强逼青仇,抽不出手来,便调动豆神宫。

豆神宫猛地爆发出一阵玄光,将剩下的魂兽尽数笼罩,随后玄光将绝大多数的魂兽都吞没,尽数归拢在那三头太古魂兽的身上。

三头太古魂兽好似得到了巨大的资助,一个个身形膨胀起来,形成更加巨大的怪物。

三头巨怪宛若三座山峰,将翠光浪潮尽数抵挡在外。

桃花迷林再生变化。

翠光浪潮消退,绿意更加凝沉,从万绿从中绽射出一点红晕。旋即,无数红晕如夜空繁星般接连绽放,那是一朵朵的桃花。

桃花鲜艳欲滴,又一阵狂风骤起,无数花瓣随风飘舞。

一时间,落英缤纷,花瓣在半空中凝聚成人形。

“此乃杀招花间风流士,劳烦算不尽仙友操纵其中部分。”房睇长对方源道。

方源点头,得到房睇长的放权之后,他能够操纵花瓣凝聚的人形。

至于数量……

方源首先尝试一下,一下子就调用了十二位花间风流士。

“咦?”方源眼中顿时闪过一抹惊异之色,“这每一位花间风流士,似乎能灌注杀招进去,积蓄酝酿成后,自爆其身,将杀招施展出去!”

“仙友感觉敏锐,这正是花间风流士的用处。”房睇长微微一笑,答道。

方源试着施展了仙道杀招燃念飞石。

顿时,他操纵的其中一位花间风流士,猛地飞射到豆神宫的头顶上空,然后忽然自爆。

爆发出来的,是一蓬蓬的念火陨石,轰隆隆砸到下面去。

豆神宫岿然不动,但那三头太古魂兽却是身形摇曳,附着上了念头之火。

方源双眼顿时一亮,不禁脱口而赞道:“这招数爆发迅捷,几乎和真身无差,不仅弥补了仙蛊屋的不足,充分发挥本方人数优势,而且还不惧消耗,很多战术真身需要谨慎抉择,动用这招却是能够屡屡冒险施展。”

正说着,方源又催动一位花间风流士,忽然俯冲下去,来到一位太古魂兽的身后。

仙道杀招百八十奴!

这招下去,太古魂兽立即身形一滞,受到了无形的冲击。

与此同时,另外三位花间风流士猛地降临,陡然爆开,各是三计仙道杀招,轰轰轰,把它轰炸得七荤八素,面目全非。

但下一刻,豆神宫中青光一闪即逝,那受到方源创伤的太古魂兽,本来伤势就不重,此刻立即得到了痊愈。

难!

即便是有花间风流士这样巧妙的变招,但是对于豆神宫而言,防御实在是太坚厚,三头太古魂兽宛若三座大坝,任凭外界洪水如何滔天,都是稳守营盘,不动如山。

方源留了好几手,没有动用真正底牌,房家蛊仙们也无多少建树,场面再度僵持。

房睇长眉头几乎要拧成疙瘩,桃花迷林只有三式变化,如今已经第二变,结果居然局面还是没有什么突破性的进展。

豆神宫的防护是如此的森严!

就在房睇长要打算转变第三式的时候,忽然方源动手。

百八十奴!

百八十奴!

百八十奴!

三位花间风流士接连炸开,方源灌输在里面的,都是奴道杀招百八十奴。此刻三招叠加起来,一起轰在那头双翅黑蟒太古魂兽身上。

这头太古魂兽顿时僵如石像,随后不久,方源轻喝一声,它竟脱离了对方掌控,投靠到了方源麾下!

第二头太古魂兽入手。

一时间,敌我双方的首脑陈衣、房睇长都为之一愣。

豆神宫在后面支撑,这些太古魂兽身着青甲,防御坚厚,受到任何伤势,都能在瞬间痊愈,房家攻势极其艰难。

不过,方源却是绕过这些,直接将太古魂兽夺到自己手中。

“仙友好手段,原来奴道便是这青甲菜园的弱点所在。接下来还望仙友继续出手。”房睇长一顿,加重音调,“房家必有重谢!”

陈衣却是冷哼一声:“居然被敌方钻了这个空子。可惜了,这不是豆神宫自产的豆神兵卒,只能暂时用这些魂兽顶替。若是豆神兵卒,对方怎可能这样夺了去?”

问津坞中,方源却是苦笑:“侥幸得手而已。之前已经施展过一次,压服了一头太古魂兽,现在是第二头,已经到达我的能力极限。接下来要施展这招,必定要付出巨大代价!不知……太上二长老大人所说的重谢,是指什么?”

房睇长眼帘微垂,遮掩住眼底厉芒,心中怒火暗生:“这算不尽一得到机会,居然敢来要挟我了!要付出巨大代价?也就是说,此招他是可以继续施展的,就是要看我房家能不能弥补他的付出。他与我定下盟约,这方面的欺瞒是做不到的,可惜我房家并无奴道强者,居然受一外人要挟!”

ps:9点还有第二更。)

:/6/6110/

\end{this_body}

