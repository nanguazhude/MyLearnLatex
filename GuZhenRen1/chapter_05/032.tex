\newsection{方源渡劫(上)}    %第三十二节:方源渡劫(上)

\begin{this_body}

d

“不想又来到了这里。∮,”方源心中感慨,停止疾飞,从高空落下。

眼前是白茫茫的一片。

冰天雪地。

呼呼呼……

耳畔是不绝于耳的寒风呼啸。

凛冽的狂风中,夹裹着无数雪花,甚至是细小冰雹,四面卷席。

因为漫天的风雪,方源的视野大大受到了限制。

并且温度很低,方源不得不催动更多的蛊虫,来保持自身的体温,以及防御寒风。

这里仍旧是北原。

具体的位置,是北原最北端,世人皆知的北部大冰原。

此处原本也是陆地草原,但因为一场毁天灭地的强者之战,将这里原本的自然地貌完全打坏,化为一片虚无。

战后,胜利者狂蛮魔尊,变成太古荒兽冰凰,口吐极寒玄冰,将这里重新冻住。这才形成了如今的地貌。

当初,黑楼兰就是选址在这里,渡过灾劫,晋升成仙。

皆因,冰原中蕴含着狂蛮魔尊对力道、变化道的真意。力道或者变化道蛊师,若在这里升仙,就会形成道痕共鸣,力道、变化道的真意,便会借助地灾外显。

渡劫之人击溃地灾,便会得到真意灌体,等若是狂蛮魔尊的教导。

一幕幕的回忆,在方源脑海中闪过。

鲜活的画面,仿佛是昨天才刚刚发生。

但此刻,却已物是人非。

曾经修为最高的黎山仙子已亡,黑楼兰、太白云生恐怕已经被策反。独自留下方源。

此时的局势处境,也和之前有了巨大差异。

“力道、变化道蛊师。要得到狂蛮魔尊真意灌体,只有一次机会。那就是晋升成仙的灾劫。不过现在,我有仙灾锻窍杀招在手,或许能重复之前的成功。”方源暗中思量。

这种事情,他还没有做过,并不能十分肯定。

只是和琅琊地灵商量之后,觉得大有可能。

“黑楼兰是大力真武体,渡升仙劫时,灾劫浩大,损失了仙蛊。才堪堪渡过。我这仙窍之灾,恐怕要比她渡的劫,更加恐怖!”

方源心中压力很大。

若情非得已,他并不想这么快就渡劫。

虽然每隔一段时日,仙窍中就有灾劫产生。但蛊仙也有许多方法应对,最常用的就是宙道手段,延缓仙窍内的时间流速,拖延灾劫的到来。

不过这样一来,仙窍中的宙道资源。却因此锐减。

利弊参半,还看蛊仙的具体情况。

琅琊福地,就曾经动用过宙道手段,改造了仙窍内的光阴支流。使得仙窍中的时间,大为延缓。

不过这个手段,并不是琅琊地灵掌握的。而是当年。长毛老祖请了一位宙道大能出手相助。

“我这仙窍宙道资源太多了,外界两个月。就要渡劫一次。修为虽然增长得快,仙窍内资源再生也多。但若支撑不住,在灾劫中陨落,那都是一场空。看来,此灾之后,还得搜寻宙道手段,改造仙窍内的光阴支流。”

这都是后话,关键还是迫在眉睫的这场灾劫。

“呼!”

吐出一口浊气,方源身躯微微一震,体内仙窍顿时发生玄妙变化。

“仙窍,落!”方源咬牙,眼蕴神光。

耳畔似乎响起一声轰鸣,旋即他视野大变,定睛一瞧,已经置身在自家仙窍里面。

五域九天。

里面空空荡荡,广袤无比。

没有任何修行的资源,因为方源还未来得及经营。

只有大大小小的蛊虫,存放在里面。

此刻,这些蛊虫都环绕在方源的身边,数目众多,宛若一片黑云。

正常情况下,仙窍是寄托在蛊仙肉身上的。不过此刻,仙窍却是落入外界天地之中,反而将蛊仙的肉身包裹进里面来。

此时此刻,方源肉身就仿佛是地灵,在仙窍内如鱼得水,但是却出不去。而仙窍本身,却可以时刻沟通外在的天地二气,稳固自身。

蛊仙渡劫时,一般都会将仙窍落到五域天地之中。

还有一种情况,就是蛊仙仙窍内资源太多,仙窍本身蕴含的天地二气嫌少,所以得落下仙窍,汲取外界的天地二气进行补充。

在原先方源站立的位置,他已经彻底消失。仙窍落在冰面上,凝缩一点,寄托虚空,并不可见。

原先肆虐的冰风,仍旧在吹着。

没有任何大的变化,好似方源就没有来过。

一切照旧。

但是下一刻,当方源主动打开仙窍门扉时,一切就不同了。

霎时间,天地嗡鸣,冰川震荡,无数的天地之气风起云涌般,朝着仙窍门户奔涌而来。

很快,大量的天地二气,就淹没了这里。

寒风和冰雹,统统被转化为了天地之气。

仙窍门户,是沟通内外的通道。随着它的敞开,方源的仙窍,开始和外界沟通,不断向内接引天地二气。

就好像是潜伏在海底深处的巨鲸,冒出水面,张开大口,吸收氧气一样。

海量的天地之气,蜂拥而入,整个仙窍天地都发出微微的震荡,但震荡的程度极其微小,常人很难察觉。

与此同时,仙窍内的光阴支流也开始融汇到五域外界的光阴长河当中。仙窍内的时间流速,开始大大的延缓。

落在天地之间的仙窍,里面的时间流速要大大减缓,贴近五域外界。

琅琊福地本身受到宙道蛊仙的改造,又只能寄托在五域天地中,里面的时间流速和五域外界相差无几。

很多蛊仙为了自身寿元着想,也会选择落仙窍,将肉身龟缩在仙窍之内。

冥冥之中。有一股感应,告知方源灾劫即将到来。

不过随着仙窍落地。时间流速延缓,方源可以明显感觉到。灾劫来临的“速度”减慢下来。

这个情况很正常。

每一个蛊仙,灾劫将临时,都会产生这种感应。

就好像是地震来临时,家畜乱走长吼,都是提前感知到了末日危机。

天地二气奔涌进来,声势渐缓,开始时仿佛瀑布倾泻,很快就宛若大河流淌,一会儿之后。就仿佛潺潺溪流了。

仙窍的胃口,也是有极限的。

方源一脸平静。

他有丰富的经验,见到天地二气如此,便开始有条不紊地布置蛊虫。

仙窍内的蛊虫,其实大部分都已经提前布置好了,主要还是外界。

海量的蛊虫,宛若一**的蜂群,顺着门户,飞了出去。

方源虽然不能离开这里。但蛊虫却是可以的。

接着,他遥控蛊虫,一一布置。

约莫半盏茶的功夫,布置妥当。方源毫无犹豫,立即灌注仙元,催动蛊虫。

一只只仙蛊相继升起。悬浮于空,散射出各色华彩。

光辉交织在一起。随后又和外界的蛊虫相呼应。

最终,形成一片巨大的苍青色的光影。不仅弥漫了整个仙窍,而且还扩散到外界,覆盖冰原上千里范围。

仙道杀招仙劫锻窍!

大量的仙元疯狂消耗,转眼间就已经成千上万,但这还只是刚刚开始!

这杀招中的核心仙蛊,大多都是七转,方源的青提仙元消耗极快。

幸好他在来之前,就已经从琅琊地灵手中,得到一大笔的仙元石,还借用了仙蛊宝皇莲。

一刻钟后,杀招终于催动成功,方源手中的仙元竟因此耗费了六成有余。

“这等消耗,已经大大超出琅琊地灵所说的数字。不过他也说了,具体的消耗,会因为仙窍小世界的不同,而发生改变。”

还未渡劫,方源手中的仙元,就消耗了一大半。

不过,他早就和琅琊地灵商议妥当,要向他借贷仙元石,很快就能到手。

此番渡劫,琅琊福地可以说成了方源的大后方。

但此时索借,却是不妥,会令琅琊地灵心中生疑。推杯换盏蛊等等手段,早就准备充分,就算在渡劫时相借,也不困难。况且灾劫究竟是什么,还未可知。

感应中,灾劫越发逼近。

方源收回蛊虫,关闭门户,在仙窍中静静等候。

趁着这个功夫,他又仔细检查了手中蛊虫。

半个时辰之后,灾劫终于降临。

仙窍开始震荡,无数的天地之气,从四面八方,各个角落里翻涌而出。

这是仙窍中蕴含的天地之气,眨眼间,天地二气相互冲击,形成无边风雪。

大雪纷飞,狂风呼啸,一下子将整个仙窍变成冰天雪地。

方源见此,心头微喜:“难道仙劫锻窍杀招起效了不成?”

呼呼呼!

漫天风雪,在仙窍中肆虐,风雪中一只只雪怪接连成形。

“雪怪之灾吗?斩!”方源呢喃一声,悬浮于空,身边蛊虫飞舞,撑起一道防护光罩,同时射出飞剑仙蛊。

嗖。

飞剑蛊直接洞穿雪怪的脑袋,从脑后穿出来。

几乎是瞬间的事,就又飞回到方源的身边。

雪怪嗷嗷直叫,脑袋上的伤口旋即自愈,继续向方源冲杀而来。

雪怪类似于泥怪、云兽,不把体内的雪核破除,便很难将其杀死,是十分麻烦的强敌。

单纯的飞剑蛊,正被其所克。

“但我此时用的,却是剑道杀招啊。”方源嘴角溢出一丝冷笑。

雪怪跑了几步,忽然一声咆哮,体内雪核毁灭,宛若抽了骨骼,浑身一下子散落开来,化为一堆积雪。

剑道杀招剑痕索命!

ps:方源渡劫,我不久前也渡了一场劫。好厉害,外劫,内劫齐至,差点没缓过来!心生厌倦,怀疑人生。喘息到今天,才有点儿状态。也令我对人生有了新的感悟。呼……感谢朋友们一如既往的支持。此书预计于2016年下旬完本,欢迎大家订阅蛊真人微信公众号,今天微信公众号中讲春秋蝉哦。(未完待续。)

\end{this_body}

