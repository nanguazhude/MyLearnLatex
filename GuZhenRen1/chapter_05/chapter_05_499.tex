\newsection{算计了人还要发难}    %第五百节:算计了人还要发难

\begin{this_body}

%1
房安蕾中了紫色舌剑,顿时坚硬如石,一动不动,并且脸色迅速涌起一片诡异的紫色。

%2
那紫色舌剑穿透她的身躯,居然灵动至极,转折方向,弯曲身体,又向方源杀来。

%3
但这时房云、房棱已经反应过来,连忙催动仙蛊屋,顿时内生华光,将紫色舌剑压制住,行动迟缓下来。

%4
藤蔓伸长,围绕房安蕾周身,浓郁花香扑鼻,彩色烂漫的繁花冒出,花瓣紧贴肌肤,一股股流光顺着藤蔓,流进房安蕾的体内,为她紧急治疗。

%5
方源一脸凝重之色,眼底隐藏着冷漠,脑海里思绪不断起伏。

%6
房家援军来到,在场的就是三座仙蛊屋,一座落英馆,一座鸡笼犬舍,一座问津坞。这三座仙蛊屋能形成战场杀招桃花迷林,威能绝妙,极其厉害。

%7
方源五百年前世,此招就在五域乱战时期,大放光彩,威慑四方。影宗真传中记录的情报,也着重叙说了此点,让人警惕。

%8
交战良久,方源已经看出,这豆神宫有古怪。

%9
刚刚青仇撞开宫门,奋尽全力的一击,更让他了解到某些真相。

%10
驾驭这座八转仙蛊屋的,似乎是一头太古魂兽!并且只会横冲直撞,无法发挥出八转仙蛊屋的玄妙威能。

%11
仙蛊屋攻防一体,直接撞过来,无疑是最直接的,最省时间的方法。但仙蛊屋亦各有玄奇妙用,诸如落英馆就有镜花、昙花一现、空谷幽兰这些个杀招手段。而那黑家的黑牢仙蛊屋,则有一门手段,便是储藏奴役上古荒兽。

%12
北原黑家势弱破败,自然不能和西漠的顶尖势力之一房家相提并论。

%13
可见仙蛊屋之间,也有高低之分。八转仙蛊屋若是不能发挥自身手段,光靠横冲直撞,定然不是房家三座仙蛊屋的对手。

%14
但房家若是得胜,压制住这座八转仙蛊屋,甚至收服它,方源又能有什么好处?

%15
按照他们定下的盟约,方源所得不过是一只六转仙蛊,还有诸多仙材罢了。

%16
甚至更可能,被房家收服了八转仙蛊屋后,连同四座仙蛊屋强势压迫,迫使方源定下一些盟约!

%17
况且此时情景,仙蛊屋近在眼前,方源怎可能没有浑水摸鱼的想法?

%18
若要浑水摸鱼,定然要平衡双方实力,方源便心生一计,借助紫色舌剑来削弱房家力量,先对付了房安蕾。

%19
本来按照盟约,方源和房家之间,怎可以彼此暗算?

%20
但实际上,盟约内容关于此处,却是泛泛而谈,相当宽松。

%21
所以,方源此次陷害了房安蕾,屁事没有!

%22
房安蕾难道不知道这项缺陷吗?

%23
当然不是!

%24
说起来,还是房家这边有不轨企图。

%25
那就是房安蕾本来就存着利用方源的心思,诸如无常石的收集,就是利用方源,算计方源。

%26
所以,房安蕾早就算计在先,她若是在们盟约中禁止彼此谋算,那就是先对付她自己!

%27
因此房安蕾故意泛泛而谈,在这方面不做约束,再加上当时情况紧急无比,方源就算发现,提出异议,她也能借此说辞。

%28
方源当然发现了!

%29
不过狡诈如他,也动起了心思,按下不表,装作糊涂。

%30
随着激烈的战况迅猛发展,最终房安蕾没有算计方源,反而被方源算计。

%31
“好胆!”

%32
“孽畜该杀!!”

%33
房家的两座仙蛊屋中纷纷传出房家蛊仙的怒吼声,房安蕾受创,他们旋即得知。

%34
鸡笼犬舍、问津坞宛如流星赶月,直插战场。

%35
但还未等到他们动手,豆神宫就大放光芒。

%36
青仇拼尽全力,都没有杀掉方源,顿时变得虚弱到了极点,被豆神宫死死压制。

%37
豆神宫自行运转,青芒蔓延战场,吞吐之间,将昏死的败军老鬼,脸色惨白的鹰姬,以及其余魂兽,尽数敛入宫中,随后一飞冲天而去。

%38
竟然是直接撤了!

%39
这一幕,出乎几乎所有人意料。

%40
就连方源都有些吃惊,这有违对方的一贯表现。

%41
豆神宫飞行速度极快,房家蛊仙犹豫了一下,鸡笼犬舍仙蛊屋停留下来,问津坞则是划破长空,追击豆神宫而去。

%42
这鸡笼犬舍仙蛊屋,造型奇特,宛若两座楼阁并肩而建。左边那栋屋通体明黄,檐下有着圆盘式的匾,门匾上只有一个大写的“鸡”字,右边那栋则是朱红,圆盘门匾上有个“犬”字。

%43
楼阁门户洞开,飞出一位房家蛊仙,奔入落英馆内,神情惶急,口中疾呼:“安蕾,安蕾!”

%44
但房安蕾满脸紫气,昏迷不醒。

%45
房云、房棱满脸疲倦和愁容,后者行礼满脸愧色道:“是我们失职,没有来得及反应。”

%46
这位房家蛊仙狠狠地瞪了他们一眼:“你们是有罪!”

%47
随后,他脸色猛地一变,侦查到房安蕾气若游丝,就要赴死,又大叫起来:“安蕾,安蕾你坚持住啊!!”

%48
他动用治疗手段,但根本毫无效果。

%49
就在这时,一道橘黄玄光从鸡笼犬舍中发出,正中房安蕾眉宇之间。

%50
房安蕾脸上的紫气顿时得到缓解,性命险险保住,只是仍旧昏迷不醒,勉强吊住一丝性命。

%51
“这杀招诡谲得很,根由是魂道杀招,却有毒道的效果。她这是中了毒,老夫只是暂时稳住她的伤势,要想解决,力不从心!”这时,又走进来一位蛊仙老者,花白头发,一身黄袍,面容肃穆。

%52
“见过太上三长老。”房棱、房云纷纷行礼。

%53
“房家太上三长老房化生……”方源心底迅速闪过一系列有关他的情报。

%54
“是你!就是你躲进落英馆中,把攻势引来,我家安蕾才中了对方的手段,这一切都是你害的!!”那位抱着房安蕾的房家蛊仙,猛地对方源咆哮起来。

%55
方源冷冷地瞥着看他,态度倨傲,冷哼一声:“你是什么人?”

%56
“我乃房沉……”那蛊仙开口。

%57
但他还未说完,方源又一拂袖:“我管你是什么人,你若不服,大可划下道来,是斗智还是斗武,我都接着。你若是想调动房家势力来找我算不尽的麻烦,呵呵,我也一并接了。”

%58
说着,方源眼中冷芒四射,扫视四周,目光逼人,气势凛冽。

%59
“不好。”房棱、房云均是心头一沉,方源毫不退让,相当强势,他们两人顿时担忧起来,唯恐双方起了冲突。

%60
方源躲闪到落英馆中,这在他们俩看来,并没有什么问题。

%61
毕竟有着仙蛊屋这等利器,加以利用,乃是人之常情啊。更何况是心思灵动的智道蛊仙呢?

%62
方源乃是他们俩的救命恩人,房云更是觉得方源是他的贵人。最关键的是,方源在之前的战斗中,也出了大力气,若非他镇压收服了太古魂兽飞蜘蛛,那么落英馆早就散了。

%63
至于方源将攻势引来,定然是他未料到房安蕾的情况。

%64
事实上,就算是房棱、房云也为料到,房安蕾居然无法闪躲。正是因为如此,他们二人才反应不及。

%65
“房沉,休得多言。算不尽仙友乃是房家盟友,已经定下盟约,之前战斗更出力甚巨,岂容你诋毁?”房化生脸色一变,对房沉喝斥。

%66
随后,他又转移目光,神情缓和,诚恳地看向方源:“这房沉乃是房安蕾的丈夫,心忧妻子性命,刚刚冒犯,不是本意,还望算不尽仙友海涵海涵。”

%67
房沉乃是房安蕾的丈夫,居然也姓房?

%68
方源顿时明白,这房沉必然是房家赘婿,从外面招来的散仙一流,入赘房家后,连自己的名字都改为了房。

%69
房沉被房化生训斥了一下,顿时不敢向方源发作,低下头来,不断查看房安蕾的情况,哪怕他手中的治疗手段无效,也一直没有停歇下来。

%70
方源眼中精芒一闪而过,心道:“房棱、房云担忧,房化生老奸巨猾,房沉也不简单,身为房家赘婿,恐怕也并非感情冲昏了头脑,而是妻子受创昏死,必定要做出样子来。只可惜我有恃无恐啊。”

%71
方源有恃无恐,还在于那份盟约。

%72
房安蕾代表房家定下盟约,自身和方源之间也牵扯最深。

%73
房家若是违背盟约,恐怕房安蕾就要遭受信道反噬。她这种状况,哪里还能遭受什么伤害?

%74
哪怕是一丝反噬,就是死掉的下场!

%75
这一点,也正是方源想要看到的。

%76
当然,盟约也有违背的手段,不过这些手段,也总会有一些代价,并非全然安稳的。

%77
即便房家拥有这样的手段,房安蕾的现状,也会让他们投鼠忌器。

%78
可以说,昏死过去的房安蕾,正是方源的一块保护符。

%79
方源有恃无恐,当即冷笑:“房化生仙友,此事你们房家得给我一个交代。否则,我算不尽不会善罢甘休!”

%80
“嗯?”房棱、房云面面相觑,怎么算不尽前辈要向房家发难?

%81
就连低头的房沉,也不由地抬眼望向方源,目光中尽是“你这人怎么如此不知好歹,我还未继续刁难你,你反倒得寸进尺,来刁难我们了?”的意思。

%82
房化生脸上浮现起恰到好处的疑惑和冷硬:“仙友此言何意?难道我们房家亏欠了仙友什么吗?”

%83
ps:今天一更,对后面的情节不太满意,需要稍微调整一下细纲。

\end{this_body}

