\newsection{再掌天相}    %第八百零五节:再掌天相

\begin{this_body}

“天庭……”秦鼎菱站在监天塔上,遥看天庭远景,面露缅怀之色。

良久,她轻轻一叹:“变化虽大,但依稀有着我记忆中的轮廓。三十多万年了。”

在她身边站着的便是紫薇仙子。

紫薇仙子神情恭谨。

眼前这位女仙前辈,虽然辈分上不及龙公,但也大得吓人。

她原本是中洲十大古派中天妒楼的某代太上大长老,原本第二天就将成为天庭成员,结果因为巨阳仙尊的一句话,被迫成为巨阳仙尊的妃子。

巨阳仙尊广纳后宫,在北原、南疆、东海、西漠以及中洲,都创建行宫。

秦鼎菱便是当时中洲行宫的仙后!

巨阳仙尊逝世后,秦鼎菱神秘消失,没成想竟在前几天忽然出现。

“秦鼎菱前辈的身份应该无误,只是根据历史记载,她应当是专修金道,怎么会变成运道蛊仙呢?”紫薇仙子心中藏着疑惑。

这时,秦鼎菱缓缓转身,细长的眼眸中透射着洞察世情的光。

她并非是传统观念中的美人,她的鼻梁虽高,但鼻翼却有些尖,她的双唇很薄,双眼细长。然而,种种混合在一起,形成了一种独特的美感,让人一眼看去,就印象深刻于心。

她身躯高于常人,双肩很宽,站得笔挺,富有英气。一系宽大的披风,从肩头垂落至地。

披风材质特异,似乎是一片片的金甲组合、镶嵌起来,时不时的披风表面会流淌过一丝丝的金色流光。

宽大的披风,将她的身躯衬托得更加高挑、健美。尤其是她长长的双腿,也裹着紧紧的片甲,黑金相间的颜色。

她的气质极其出众,仿佛天生就是贵族,要凌驾于芸芸众生之上。当她看着别人的时候,因为身高的缘故,常常要用俯视的角度。她自然而然散发而出的凛然之威,让人不自觉地垂眉低头,不敢去正视她,仿佛目光只能盯着脚面,如此才不算是对她的冒犯。

秦鼎菱轻声开口道:“紫薇仙子,我知道你心中的疑惑。事实上,我的失踪乃是有意为之。当年,巨阳仙尊无敌天下,掌控整个时代,无人可用抵抗他。幸而他并非魔尊。我当初虽是被迫成为他的仙后之一,但也带着自身的用意,就是想从他的身上学习到运道的奥妙。”

“巨阳仙尊死后,我亦有许多心得,决意舍弃金道,转修运道。然而我当时已有八转修为,又是要转修运道,风险很大。我得天庭之助,得到沉眠延寿的机会。又将全身的金道道痕舍弃,化为金棺,将自身埋藏在地脉深处,闭死关,悟运道。”

“没想到这一睡就是三十多万年,当我在此醒来,行走天下,已经是物是人非。”

秦鼎菱语含无限的感慨。

三十多万年,沧海桑田。

曾经笼罩着她的巨大的人生阴影巨阳仙尊,早已经逝去。

并且,其后还出现了幽魂魔尊、乐土仙尊。

如今,整个世界又面临大时代的浪潮,根据种种迹象和预言,一位前所未有的至强尊者将要出现。

“原来如此。”紫薇仙子吐出一口浊气,脸上露出惊喜愉悦的笑,“前辈您的出现,实在是太及时了。晚辈有愧,才干不足,使得天庭屡遭败绩。其中一项主因,便是运道方面不足,束手束脚,令那魔头屡屡逞威。”

秦鼎菱微微点头:“情报我都看了,没想到竟出现方源这般的魔头。这在人族的历史上,也是罕见之辈。不过,紫薇仙子你也不必妄自菲薄,你的决策无误,只是对方狡诈阴险,同时又得到尊者遗藏。尤其是后者,令方源变得极难应付。”

“我此次苏醒,重临人间,恐怕就是专门为了对付方源的。”

紫薇仙子好奇:“前辈此言何意?”

秦鼎菱饱含深意地笑了笑:“紫薇仙子,我此次的苏醒自然有玄机。我且问你,我是谁唤醒的?”

“是古月方正。”

“不错。”秦鼎菱徐徐道,“天外之魔不在宿命之中,乃是天然的变数。天意驱使这样的棋子,来对付魔尊幽魂逆天而行,自然有着钳制天外之魔的手段。就像毒蛇栖息的地方,往往生长了解蛇毒的草药。古月方正便是天道钳制古月方源的一层保障。”

说到这里,秦鼎菱用赞赏的目光看着紫薇仙子:“你对运道知之甚少,但能推算出这一层,并且着手栽培方正升仙,已然不易。”

“前辈谬赞,晚辈只晓得拼尽全力,维护我天庭大局,守卫我人间正义。”

“哈哈哈,说的好,这便是我天庭的精神,就算过了三十多万年,也从未改变过!”

秦鼎菱笑了几声,状极愉悦,顿了顿,又继续道:“天外之魔方源乃是变数,古月方正便是钳制他的存在,会因方源的变化而变。魔尊幽魂逆天失败,但方源却脱离了天意的掌控。他越来越强,毫无破绽,成长极快,令人骇然,乃至连天庭都有些束手无策。于是,为了遏制这样的变化,便有了方正惊醒我,令我出世的事情发生了。”

“命乃定数,运则是变数。修为越高,实力越强的蛊仙,本身运势就很宏大。这是因为他们的一个心念,一次行动,就能引发周遭巨大的变化。方源在运道上颇为强势,但你栽培了方正,令他成仙,因此变数足够大,才有了我苏醒、出现去克制他这方面的优势。接下来,我估计,还会有类似的情况发生。方源这个魔头,不会蹦太久了。”

说到这里,秦鼎菱叮嘱紫薇仙子:“你还是要加大力度栽培方正,并且确保他的安全。只要我们着手这一点,顺应天意的布局,就能事半功倍。”

“晚辈明白了。”紫薇仙子双眼发亮。

一直以来,她对付方源,都建树不大,主要原因是在于方源狡诈奸猾,防备得滴水不漏,让紫薇仙子推算不出他的方位来。

紫薇仙子虽然明白古月方正的意义,但并不太重视,她更相信自己和天庭的力量,在方正身上布局,只是辅助作用。

但现在经过秦鼎菱的提醒,紫薇仙子明白:古月方正完全就是克制方源的最佳的着力点!

她先前栽培方正升仙,已经收到良效引发了秦鼎菱的出现。

只要她继续大力栽培,类似秦鼎菱的人物将会出现得更多。

“方正实力越强,自身的运势才会更加宏大,如此才能引发出更多的变化。而这些变化,正是天意布局,专门来钳制方源的。”

“既然如此,那就以方正为主,天庭为辅。”

紫薇仙子审时度势,理智英明,立即做出了最正确的战略调整。

“哦,还有一事。我这里有许多仙蛊残方,乃是我当年在巨阳仙尊身边,偷偷打探、揣摩出来。这些蛊方都是有关巨阳仙尊的运道仙蛊,还要拜托你推算完善,再加以炼制。”秦鼎菱取出一只信道凡蛊。

紫薇仙子连忙双手接过:“晚辈必定竭尽全力推算,请前辈放心!”

太古白天,五相洞天。

方源最近都在此潜修。

仙道杀招万念剑瀑!

轰隆!

一声巨响,白相杀招内部陡然降临一道恢弘的瀑布,银白色的瀑布气势宏大,狠狠地砸在天相杀招底部的那层僵死的意志上。

瀑布的每一滴水,都是一个念头,外表呈现出剑的模样。

瀑布狠狠冲击,瞬间轰击出一道巨大的坑洞,消灭了海量的意志。

持续了几个呼吸后,瀑布消失,在僵死的意志大陆上,出现了一个巨大的湖泊,湖泊中的水便是流窜的小剑念头。

仙道杀招剑客!

方源心念调动,再催一招。

此招发出后,湖水**滚动,仿佛煮沸了一般,随后一个又一个的战士,从湖水中跃出。他们一个个都貌似方源,手持长剑,对着僵死意志不断劈砍,每一击都能割裂出一个巨大的沟壑。

这两招都是来头不小,均由方源所创的复合杀招。

万念剑瀑以慧剑仙蛊为核心,容纳了剑道的奥义,对付僵死意志远比单纯运用慧剑仙蛊要好得多。

而杀招剑客则是兼收并蓄了剑道、人道的精妙道理,这也是方源最近晋升人道宗师后,才顺势开创出来。

一个多时辰后。

忽然一声鹤呖,响彻整个五相洞天。

沉眠中的白鹤,仿佛从睡梦中苏醒过来,舒展身姿,仰颈亮翅,优雅从容。

“终于再次将天相杀招掌握了!”

“哈哈,比上一世节省了一半还多的时间。”

“看来是时候动身,去找兽灾洞天、华文洞天了。”

方源笑了笑,眼中精芒阵阵闪烁。

好事成双,就在这时,他又听到琅琊地灵的汇报:又有一只运道的六转仙蛊被炼成了。

方源还未获得悔蛊,并不想大规模地升炼仙蛊。

但他敲诈勒索南疆正道,得了许许多多的运道仙材,最近这段时间主要就是来炼制运道仙蛊!

仙蛊唯一,他炼成了,别人就没有机会了。

至尊仙窍已经开发得不错,方源财力雄厚,区区六转仙蛊,完全喂养得起!

------------

\end{this_body}

