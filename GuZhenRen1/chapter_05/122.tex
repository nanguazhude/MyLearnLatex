\newsection{我……欢喜啊!}    %第一百二十二节:我……欢喜啊!

\begin{this_body}

北部冰原,地下深处。

洞窟大厅中,石凳、石桌围成一个圈,粗藤攀附周壁,青草生长在石缝间。

三堆火焰,在大厅中央熊熊燃烧。

据说,这是上古时代就流传下来的风俗。异人族群中来了贵客,就要设宴款待,并燃起三堆大火。

雪人、石人、毛民以及方源,诸多蛊仙,在这里齐聚一堂。

火光映照在每个人的脸上,都见笑容。

各种瓜果摆卖了个人面前的石桌上,很快,又有石人侍从为每人端上了一大碗的酒。

这是我石人一族的唇鱼酒,都喝喝看!”石宗招呼着,从座位上站起身来,端起酒碗。

石宗乃是这支石人族群中的太上大族老,并未参与剿灭方源的战斗。

他全身都五光十色,身上镶嵌雕缀了无数金银珠宝,很是华丽。

这是石人一族的风俗。

石人全身都是石头,在漫长的生命中,会生长出各种金银铜铁。石人以此为美为荣,地位越高的石人,身上的金银铜铁都会越多越奢华。

不只是石宗,在场的其他几位石人蛊仙,亦都如此。

石宗既然号召,众仙都随之响应,纷纷高举酒碗。

石宗等仙大口喝下,毛民们也是一样,唯有方源小口抿着。

石宗喝完之后,将酒碗朝下,一滴酒都没有剩下。

叫好声顿起。

其余蛊仙亦跟着效仿,将碗口翻下,皆是一口干掉,不剩分毫。

众仙大笑,原本平淡的气氛变得有些欢燥起来。

惟独方源一人,将酒碗放下,唇鱼酒只喝了一点点。

雪人、石人一方蛊仙,都看在眼里,视若无睹。

反倒是毛民蛊仙中,毛六开口指责方源道:“方源长老。你这是做什么?难道你不知道,你这行为太不礼貌了吗?主人都喝掉了碗中的酒水,你身为客人,亦要随之。碗底不能有任何残留,这才是对主人的尊重。啊,我忘了你是人族,不是我异人种族,不知晓这个从远古时代就传承下来的风俗啊。”

毛六乃是影宗安插在琅琊派的内奸。见缝插针,也要给方源找点麻烦。

方源冷笑一声,目光扫视一周,慢条斯理地道:“这个风俗我岂不知?只是这唇鱼酒,乃是地下石鱼流下的口水,我并不喜欢饮用。”

“你这家伙”毛六脸色阴沉,心中却欢喜。方源此言,无异于得罪对方,这正合他意。

方源去信琅琊派通知地灵,自然是没有说清事情。只说了大概。毛六等人并不知道,方源和雪人、石人激战死斗过,最后搬出太古上极天鹰,才令对方妥协。

事实上,方源加入琅琊派之后,琅琊地灵就布过一个任务,便是寻找在这里生活的异人盟友。

方源既然找到了他们,自然就完成了这个任务,必定会得到琅琊派的嘉奖。

作为影宗的一员,毛六不愿看到方源得利。展下去。此刻开口,就是千方百计地打压方源。

“哈哈,方源长老既然不喜此酒,那就换一种。来人。将我自酿珍藏的冰狼酒拿出来。”一位雪人蛊仙立即笑道。

雪人有雪白的皮肤,冰蓝的瞳眸,水蓝的头。

这位雪人蛊仙身躯矫健,袒露上身,胸膛上有深蓝纹身。水蓝头都扎起来,露出线条刚硬。精悍干练的脸庞。

此人姓名冰卓,正是让方源陷入苦战的双枪蛊仙。

很快,就有雪人侍从将酒端上来。

这是冰做的酒碗,里面的酒水也近乎成冰,冒着寒气。

“老实讲,我本人也不太喜欢喝唇鱼酒,所以自己酿的这个酒,贵客不妨尝一尝。”冰卓热情地道。

方源用眼光余角一扫,故意轻慢地道:“且让我先喝一口,看看喝不喝得惯。”

“请喝,请喝。”冰卓笑着。

方源喝了一口,起先这酒刚入口时,冰寒雪冷,但是入喉之后,却变得火一般热烈。冰火两重天的感受,让方源顿时眼****芒,脱口而出,叫了一声:“好酒。”

“哈哈哈。”冰卓仰头大笑,“既是好酒,那就多喝一些。”

话刚说完,群仙就见方源端起酒碗,仰头一口喝掉。

群仙微微一愣,顿时轰的一声,爆出热烈的叫好声。

方源不着痕迹地瞥了毛六一样,心中一笑。

毛六的心思,他早就心知肚明。

毛六是不会让方源遭到什么不测的。因为影宗救下幽魂本体之后,下一个目标必然就是他方源。正因为此点,毛六就不会害了方源性命。

所以,毛六最多是打压方源,拖延他展度。好让影宗在今后更方便下手。

这一点,毛六之前就已经做得很好,让方源和琅琊派的关系变得十分冷淡。

不过,他怎么也不会料到,方源手中有太古上级天鹰,并且还利用它成功地震慑了雪人、石人群仙。

太古战力在手,由不得这些异人蛊仙不敬重!

这就是实力!

在他们的心中,不管毛六怎么挑拨和诋毁,方源的分量都要比其余毛民蛊仙的总和加起来,还要重要得多。

事实上,异人种族对力量的崇拜情节,普遍都很浓重。这是太古、远古、上古时代,一直流传至今的异人特征之一。

因为不管在茹毛饮血的猎食时期,还是各族大战时期,亦或者被人族屠杀的年代,强大的力量能带给异人们更多的生存空间和机会。

反倒是人族,在成为五域的霸主之后,地位岿然不动,反而对力量崇拜的情结有所减退,远远不如这些在世界角落里苟延残喘的异人们。

方源故意拿捏,在这些异人蛊仙看来,这是相当正常的。

哪一位强者,没有点脾气?

若方源不作,他们反而要惴惴不安呢。

方源深谙这些异人蛊仙的心理,当然他的用意还有一层。那就是试探对方的心意。

现在,方源已经基本确信,他已经安全了。

“怎么会这样?”看着眼前异人蛊仙们笑逐颜开的样子,他心底大为吃惊。

接下来。雪人、石人蛊仙们几乎是围绕着方源转,反而把其余毛民蛊仙,都放置一旁。

“方源长老你一碗。”

“方源长老。果然是好酒量啊!”

方源来者不拒,一一饮了。

毛民蛊仙们看得面面相觑。

毛六心中纳闷至极:“这方源到底给他们灌了什么**汤,居然受到他们如此欢迎?”

要说方源变作毛民形象,兴许还符合常理。但偏偏他露出原形,乃是一位货真价实的纯正人族啊。

“方源长老,雪儿来为你斟酒。”席间,一位雪民中的女蛊仙,靓丽娇美,竟亲自走到方源面前,接过侍从的工作。为方源倒上满满一碗冰狼酒。

其余的毛民蛊仙们,眼睛都瞪得溜圆。

毛六几乎要脱口大喊:“你们搞什么鬼!你们到底还是不是异人啊?眼前这个方源,可是人族蛊仙!人族!把你们异人大肆屠杀,打垮排挤的人族!!”

“雪儿,你就是那个现我真身的雪人吗?”方源上下打量雪人女仙,神态肆意,直接问道。

雪人低下头,娇羞地笑道:“雪儿只是擅长一些侦查手段,方源长老你才是大英雄呢。”

毛十二正在饮酒,听了这话。忍不住把酒一口喷了出来。

毛六看得火冒三丈,差点要指着女蛊仙雪儿咒骂:“你这什么意思?喂喂!你脸红什么?低下头一股娇羞是何意?还偷偷打量方源,你这眼神以为我看不到吗?该死!你到底还是不是雪人啊!注意你的身份啊!”

“英雄之名,我愧不敢当。”方源哈哈一笑。敷衍道。就连他自己,心中都有些诧异。这些异人蛊仙的前后态度,似乎转变得有点大。

冰卓这时也走过来,端着酒碗:“方源长老若不是英雄,谁还能堪称英雄?”

雪儿适时地介绍道:“这位是冰卓大哥。他可是我族第一战力呢,平时他都冷冰冰的。十分傲气。也只有和方源长老你说话,他才这般热情。”

方源站起身来,打量眼前蛊仙,沉吟道:“若我猜得没错,你便是那位手持双枪,和我大战数十合的那位吧?”

冰卓哈哈大笑,直接竖起大拇指:“方源长老你好眼力!”

方源不再多说闲话,直接一口干了。

冰卓大喜过望,连忙仰头,一口饮下碗中之酒。

叫好声又响起,雪儿在旁边拍掌:“这就叫英雄惜英雄。”

毛民蛊仙们巴巴地看着,没有人搭理他们。另一边的蛊仙们,目光都集中在方源的身上。

毛六差点要拍桌子了。

冰卓的态度,让他更加不爽。

他在心中大叫:“你们究竟是搞什么啊?!那个叫雪儿的女蛊仙也就算了,方源这模样是挺俊俏。你一个男人来凑什么热闹啊!看你这笑嘻嘻,恨不得要和方源拜把子的样子,是想干什么?你还第一战力,还族中英雄?你和一位人族蛊仙这么亲近,想叛族吗?!”

冰卓叹息一声:“和方源长老相比,我哪里算得上什么英雄?方源长老的麾下,竟有太古上极天鹰。若非方源长老是自己人,恐怕我们已为我族惹下灭族的大祸了。”

“什、什么?太古上极天鹰!”一时间,毛民蛊仙们面面相觑。

“方源长老,我刚刚没听错吧?你竟然有一头太古上极天鹰?!”毛六终于忍不住了,他腾的一下站起身来,用尖锐的目光,死死地盯着方源。

方源摸了摸自己的鼻子,有些不好意思的样子,道:“前不久,我与楚度联手,攻取了黑凡洞天,尽夺黑凡真传。这头太古上极天鹰便是其中的收获之一。”

“什么?!”毛民蛊仙们顿时惊呼一片。

另一边的雪民、石人蛊仙,亦都纷纷交换眼色。

毛六倒退一步,难以置信的死死地望着方源,然后又看向其余蛊仙。

刹那间,他终于明白过来,也完全理解过来,为什么这些异人蛊仙会这么看重方源!

太古荒兽,八转战力啊!!

扑通。

毛六受不住这个巨大的打击,一下子又坐到石凳上去。

“这些我还未来得及告诉你们。来,咱们琅琊派自己喝一碗,这可算得上喜酒呢。”方源笑着,举起酒碗。

毛民蛊仙们反应过来,纷纷笑逐颜开。

“我们琅琊派竟也有八转战力了啊!”

“方源长老,你真是深不可测,太厉害了!”

“方源长老,之前的事情,是我错了,我是有眼不识天梯山啊。”

毛民蛊仙们都一脸“我服气”的样子。

这就是八转战力的声威!

毛六也举起酒碗,脸上在笑,暗中却在气得喷血。

方源打量他一眼,笑道:“毛六长老为何双手抖啊?”

毛六向他笑着,不愧是潜伏多年的内应,神情没有丝毫破绽,只是声线变得有些沙哑:“我这是激动万分,为方源长老你,为琅琊派上下欢喜啊。”(未完待续。)

◆地一下云来.阁即可获得观.◆

\end{this_body}

