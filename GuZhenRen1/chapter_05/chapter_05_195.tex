\newsection{界壁伏击}    %第一百九十五节:界壁伏击

\begin{this_body}

%1
这龙息锋锐至极,还未射中,武遗海便感觉脸上犹如针刺一般。

%2
他心中大骇,根本来不及躲避。

%3
锵!

%4
一声金铁交接的脆响,一枚小巧的金色圆盾在生死关头,浮现在武遗海的面前。

%5
剑光龙息射在金盾上,几乎眨眼间,就将这枚金盾破开,然后余势不减地仍旧朝武遗海的脸上射来。

%6
不过武遗海有金盾的这一下抵挡,总算争取到了活命的宝贵时间。

%7
他也不愧是东海拥有薄名的七转蛊仙,此时低喝一声,施展出变化道杀招。

%8
光辉一闪,武遗海消失在原地,却而代之的是一头巨龟。

%9
巨龟的四肢和头尾,都死死地缩在了壳中。

%10
剑光龙息打在巨龟的壳上,立即印刻出一个长长的痕迹。

%11
“好厉害的龙息!我这变化出来的角神龟,乃是上古荒兽,甲壳坚硬,在东海的上古荒兽中也是名列前十的。”武遗海心中震动万分,“看来此次来敌之强,远超前面的所有埋伏!现在就看张叔还有冷兄弟的了。”

%12
武遗海乃是变化道蛊仙,但他变化出来的角神龟,体型巨大,在海中速度也算不错。到了陆地上,就缓慢无数倍了。

%13
战斗力也相应下降了许多倍,武遗海利用的只有角神龟的坚厚防御。

%14
防高攻弱,武遗海虽然保住了性命,却完全丧失了战场的主动性。

%15
他当然也有其他手段,可以变化成其他形态。但是别忘了,变化道蛊仙有一个重大的弊端。

%16
那就是:不能随即变化。变化的不同形态之间,要有一定的时间间隔。蛊仙毕竟将身上的临时道痕印记铲除干净,才能继续变化。

%17
方源拥有至尊仙胎蛊,那是个特殊的例子。他还曾经被戚灾误以为是,得到了狂蛮魔尊的真传,可以随意变化形态。

%18
所以,当武遗海变化成角神龟后,他虽然保住了性命。但是短时间之内,无法做出什么像样的反攻了。

%19
不过武遗海并没有慌张。

%20
因为他知道,在他的身边,还有两位七转蛊仙强者。

%21
一位是张叔。另一位是冷兄弟,虽然武遗海还并不清楚这两人的姓名,但却知道这两位都是他母亲派遣过来,专门护送他前往南疆的干将!

%22
在之前的一路上,这两位蛊仙也展现出了强大的作战能力。非常可靠,帮助武遗海一路杀退了数波埋伏。

%23
武遗海寄希望于张叔和冷兄弟,但张叔却在张口吐血。

%24
他不得不吐血。

%25
因为他的金盾,被那道剑光龙息所破。

%26
这是他张叔的防御杀招,被破之后,张叔不可避免地受到了反噬,立即大吐鲜血。

%27
和武遗海一样,张叔的心中也非常震惊。

%28
“在这瘴气界壁当中,任何的仙道杀招都要受到瘴气的削减。这道剑光龙息穿透瘴气,不仅射破了我的金盾。还在角神龟壳上留下印记,那龙息本身是有多强!”

%29
张叔在吐血,冷兄弟则飞扑上去。

%30
两人的合作相当默契。

%31
这位姓冷的青年蛊仙,知道张叔的防御杀招别破,需要调息休整。

%32
他冲得很猛,气势汹汹。

%33
转眼间,他便从人形,变化成一头雷芒巨狼。

%34
雷芒巨狼乃是上古荒兽,浑身毛皮都是蓝色,狼毛尖端是转化成奇特的蓝色晶体。散发出微微的电芒光亮。

%35
雷芒巨狼一出现,就照亮了暗紫色的瘴气。

%36
它顺着龙息喷吐,形成的通道,向前方冲去。

%37
四爪扒地。威武雄壮的狼身肌肉贲发,带给冷姓蛊仙强劲的速度。

%38
冲刺的同时,雷芒巨狼的身上,还涌现出无数的电蛇。

%39
电蛇凝结成一股,跨越雷芒巨狼的头顶,向前方飞射而去。

%40
这是冷姓蛊仙催动了仙道杀招。

%41
见到这一幕。张叔心中顿时安定下来。

%42
几个呼吸之后,他强压下体内的伤势,浑身闪现出耀眼的光辉。

%43
眨眼间,他变化成一头金色的剑齿虎,体型比雷芒巨狼还要壮大两倍。

%44
原来,他竟然也是一位变化道蛊仙。

%45
这当然是武独秀的刻意安排。

%46
因为五域的界壁,环境相当特殊。要穿越界壁,各个流派都不分高低。但若是在其中作战,往往变化道蛊仙拥有优势。

%47
蛊仙在界壁中,每次动用仙道杀招,都会引发仙窍震荡,可谓未伤敌先伤己。

%48
但若是蛊仙变化成猛兽,单纯用身体肉搏,却避免了这个弊端,拥有了更加持续的战斗能力。

%49
蛊仙张叔回望身后一眼,他心知角神龟的防御能力,叮嘱武遗海一句后,便也循着雷芒巨狼的路线,冲了过去。

%50
很快,张叔便见到雷芒巨狼正撒丫子往回跑!

%51
在雷芒巨狼的身后,一道道的剑光龙息,喷射过来,并且在浓郁的暗紫色瘴气中,传出龙兽咆哮的声音。

%52
“怎么回事?”张叔十分吃惊,他知道冷姓青年蛊仙骁勇善战,绝不会无故后撤逃跑。如今他这么做,必然有其原因!

%53
“不要打了,快退。”冷姓蛊仙见到金色巨虎,立即口吐人言。

%54
张叔更加吃惊,为什么冷姓蛊仙要一力避战。

%55
他正要询问,这时候他终于见到了攻击他们的罪魁祸首

%56
一头剑蛟穿透暗紫色的瘴气,出现在他的视野当中。

%57
银白色的龙鳞,密密层层。龙瞳苍白,龙角尖刺向空,四只龙爪狰狞可怖,龙牙利齿闪烁阵阵寒芒。并且身上仙蛊气息洋溢!

%58
“难道是一头野生的上古剑蛟!?”张叔瞪大双眼,难以置信。

%59
但铁一般的事实,就摆在他的眼前,他不得不信。

%60
在刹那间,他终于明白,为什么平时上刀山下火海都不皱眉头的冷姓蛊仙,居然会反常后撤。

%61
因为他碰到了一头野生的上古剑蛟,并且这头剑蛟并非来源五域,而是生活在白天或者黑天之中。

%62
为什么张叔会这么判断?

%63
因为这头上古剑蛟在瘴气界壁中,进退自如,动作舒展流畅,根本没有任何的拉力或者斥力影响。

%64
对于蛊仙而言,总要有地域之分。因此进入界壁当中,必定要受到极大的影响。

%65
只有在上古白天或者黑天中土生土长的生命,才会在五域界壁中进退自如。

%66
张叔没有犹豫,也跟着调头,撒腿就跑。

%67
不受到界壁的影响,单单这个优势就极大了。蛊仙在界壁当中,能发挥出来的实力,肯定只有五六成的样子。

%68
就算是搁在外界,单独面对上古剑蛟,还是拥有野生仙蛊的上古剑蛟,张叔都没有把握。更何况如今,是在界壁这种特殊的环境之中呢?

%69
“我们怎么会如此倒霉?好巧不巧,居然碰到了一头野生的上古剑蛟,在界壁中游荡?”

%70
“我猜,很有可能这头上古剑蛟是被某些蛊仙特意吸引下来,阻击我们的!”

%71
奔逃中,张叔和冷姓蛊仙迅速交流。

%72
很快,他们就和变成角神龟的武遗海汇合。

%73
武遗海也提前知晓了情况,他连忙传音:“尽量不要和这头上古剑蛟纠缠了。说不定它是被人故意引过来,就是要让我们和它战斗,消耗我们的仙元和战力!”

%74
“那我们该如何是好?”

%75
“后撤,远离上古剑蛟。尽量不要激怒它,但也要展现出我们的强大实力。野兽智慧不足,但当它们知道这个猎物威胁性太大,它们也会放弃的。”

%76
三位蛊仙很快就商量好了对策。

%77
攻少防多,任由上古剑蛟吊在他们三人身后,不断地喷吐剑光龙息。

%78
三位蛊仙徐徐后退,被剑光龙息打得灰头土脸,基本上抬不起头来。

%79
“怎么办,这头上古剑蛟似乎盯上我们了!”

%80
“兴许是我之前,动用了雷道杀招,激怒了他。”

%81
“我有一个仙道杀招,可以消弭这等猛兽的怒气。只是需要改变杀招。”

%82
三仙迅速商量了一下,决定由冷姓蛊仙、张叔进行防守,而武遗海变化形态,来抚平上古剑蛟的怒火。

%83
武遗海小心翼翼地化为人形。

%84
他躲在其余二仙的身后,催动手段,迅速消弭身上的道痕印记,准备下一番变化。

%85
上古剑蛟到底这是野生猛兽,智慧低下,一味地朝着巨狼和金虎攻击。

%86
张叔和冷姓蛊仙因此承受了上古剑蛟的猛烈攻势,防守得相当艰难。身上已经多出负伤,但为了尽快将这头上古剑蛟赶走,他们并没有动用什么攻伐仙道杀招,防止情况恶化,落入他人算计。

%87
“差不多了,我这记仙道杀招可是……呃!”武遗海欢喜的语气,陡然转变成了惊呼。

%88
因为他看到,张叔所化的金色巨虎,被一记仙道杀招贯穿了头颅,惨死当场。

%89
剑道杀招暗歧杀!

%90
一切来得是如此突然,毫无一丝征兆。

%91
随后,那头上古剑蛟在刹那间,爆发出来惊世绝伦的速度,一下子就像是一道银色的闪电,在三仙的瞳眸中留下一道绚烂的光影,就消失在原地。

%92
冷姓蛊仙震惊万分!

%93
“中计了!糟糕!!”

%94
他转身回望,只看到武遗海被击碎的头颅碎片,还有白色的脑浆。而他的无头尸体,被上古剑蛟抓住,然后塞入仙窍当中。

\end{this_body}

