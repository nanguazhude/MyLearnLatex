\newsection{都要争取方源}    %第八百二十三节:都要争取方源

\begin{this_body}



%1
龙公开出好大的砝码!

%2
但场中无人觉得奇怪。

%3
方源的实力就摆在这里,他的态度很大程度上影响东海,影响中洲。

%4
这就是价值所在!蛊修世界,实力为尊!

%5
宋启元心中越加焦躁,冷笑道:“龙公大人,你真是好不要脸,明明刚刚还和气海前辈打生打死,现在忽然又想招揽,你这样的话谁能信?”

%6
龙公微笑从容:“信与不信,只看个人。你们不信,是害怕天庭将来侵吞你们的利益,你们各自都有家业,并且囊括诸多海域,是为了维护自身势力。但气海仙友却是独身一人,他和你们是不同的。”

%7
“可笑!气海前辈乃是东海蛊仙,加入你们中洲天庭?不被你们排挤才怪!”沈从声道。

%8
“排挤?”龙公不屑地扫视沈从声一眼,“你如此格局,难怪执掌沈家多年,沈家也没有多大起色。门派制度正是海纳百川,量才唯贤。再者什么中洲、东海?将来五域合一,天下蛊仙都是同一个地方的人。”

%9
龙公口才了得,一时间把宋启元、沈从声都驳得说不出话来。

%10
不过,就算龙公说的再有道理,方源怎么可能加入天庭?

%11
要是龙公知道,他力劝的气海老祖就是方源本人,不知道会是什么表情。

%12
沈从声张口欲言,不愿放弃。

%13
龙公冷哼,目光中透露出一丝狰狞:“你们小小沈家,也要妄图染指天下么?野心是和能力匹配的。天若不给报应,那是时候未到!你若等不及,我也代天行罚,现在就给得你沈家一些报应。”

%14
沈从声被龙公直接威胁,心中又惊又怒,不过暂时还是闭了嘴。他此番来劝说气海老祖,并不是真的想要将沈家搭进去,让沈家成为天庭的眼中钉。将来五域乱战,沈家沦为先锋炮灰,那更不是他愿意看到的事情。

%15
宋启元深深明白绝不能在此刻弱势,鼓掌道:“好气派,龙公,你一位中洲蛊仙,在东海威胁我等三位东海八转,是欺负我等东海蛊仙的脾气好么?”

%16
方源顿时也冷哼一声,面色难看地道:“这番口气老夫也熟悉得很。将来你们中洲天庭蛊仙围攻我,便是这等语气,总觉得你们天庭是天王老子,天下第一,谁都应该臣服于你们。”

%17
此言一出,沈从声、宋启元立即精神一振,腰杆子顿时挺得笔直起来。

%18
殊不知方源是见他们俩有被龙公气势所摄的趋势,故意出言帮衬一二,为他们俩撑腰。

%19
眼下局势有些诡异,之前明明是龙公和方源互怼,结果演变成东海二仙和龙公相互争取方源。

%20
方源乐意见到这样的情景,他站在中间,看着两头掐架,谁若是劣势一些,他就帮衬弱小的一方。

%21
“这种机会可是少见得很。说不得此番运作之下,你们沈家、宋家就都成了天庭的眼中钉了。”

%22
方源决定挑拨离间,好好利用这次良机,务必使得双方矛盾激化,从而影响到整个东海蛊仙界!

%23
龙公见气海老祖脸色难看,心中暗恼。

%24
他恼怒的对象,却不是气海老祖,而是方源。

%25
“这魔头居心叵测,最擅长耍弄阴谋诡计。劝说气海老祖出手,必定是添油加醋!”

%26
“气海仙友。”龙公想了想,开口道,“你我之间合则两利,分则两害。许多小人挑拨,是要将你当做枪使啊。”

%27
他语气软中带硬,并不是一味的利诱,隐隐展现出天庭的强势。

%28
另一旁,那两位东海八转见气海老祖维护自己,连忙感谢,接着又说天庭的坏话,胆子比之前还大。

%29
方源稳居中央,挑拨离间,看着双方斥责彼此,相互拆台。

%30
一会功夫,双方火气见涨,语气越加锋锐。

%31
方源察言观色,将龙公神色越加不耐烦,他知道过犹不及的道理,便意味深长地笑道:“那方源谋算,老祖我岂会不知?只是那小辈也端的有意思,允诺我诸多事项,竟然连尊者真传都能舍得。老祖我虽然专修气道,但也有子孙后代,在仙窍中繁衍。给子孙后辈多留一条路也是好的啊。”

%32
方源说话毫不要脸,这些真传本来就到了他的手中,说多少也不为过。

%33
龙公听了,方源竟然也招揽气海老祖?

%34
他迅速想想:也对!这种人物乃是绝世高手,仙尊不出,全天下都要看他的脸色。自己对付他,也是半斤对八两,相差仿佛。他的杀招虽然并不精妙,气道八转仙蛊似乎也不多,但是一身道痕,规模恐怖,浩荡磅礴,也不知道他是如何修炼成的!

%35
龙公不知道自己又被方源骗住了,他大笑一声:“方源,暴发户而已,他是尊者的棋子,尊者布局时用到了他,因此给了他一些甜头,让他奔命。单论底蕴,他怎么能和我天庭相比呢?”

%36
“气海仙友,你若加入天庭,仙窍可不必献出,自己保留即可。罢了,咱们先不谈你加入天庭之事,我只是表明我方的诚意。将来气海仙友若是有意,我龙公必定扫榻相迎。”

%37
“咱们先谈和解的事情。仙友既然专修气道,我天庭愿出一批气道八转资源,这都是外界绝迹之物,还有气道传承,都是精良优异,可供仙友借鉴。”

%38
龙公决定一步步来。

%39
方源仅听他说了几项气道资源,也不由地眼瞳放光。

%40
的确是世间孤品,也就是天庭这样的势力,从远古时代传承至今,才有这样的收藏。

%41
其他的势力,年头和天庭根本不能比,根本不会有这样的珍藏。

%42
沈从声、宋启元见方源明显意动,连忙出声:“气海前辈,你可能不知,龙公此次秘密潜入东海,是为了仙蛊屋龙宫而来。我俩方才俘虏了古月方正,从他那里搜魂,得到了许多情报。”

%43
“没错,龙公是想暂时稳住前辈,先收取了这座关键的仙蛊屋。一旦他做成这个事情,接下来他是什么态度,真的不好说啊。”宋启元也在添油加醋。

%44
方源心中暗乐:这仙蛊屋龙宫高达八转,所以沈从声、宋启元都分外心动。比我自己还更着急,千方百计地想要阻止龙公呢。

%45
表面上,方源则脸色肃然,沉声道:“龙宫之事,我也知晓一二。并且我也拿到了尊者真传,和方源达成了约定。龙公你就罢手吧。你既然想要表明诚意,这就是最好的方式。将来曾经发生的事情,我便既往不咎。更进一步,或许你我双方还有和平罢战的可能。否则的话,就算斗不过你,我也能去中洲看看。”

%46
龙公冷哼一声,肚中暗骂:这气海老祖别看外表风姿卓绝、仙风道骨似的,但其实不要面皮得很。

%47
方源扬言回去中洲看看,当然绝不只是“看看”这么简单了。龙公毫不怀疑这种事情,气海老祖真的能做得出来。因为之前,方源就是专盯着松鹤亭猛攻,让龙公很是被动,挨了不少打。

%48
就像东海二仙有着家业,担心中洲天庭的威胁,中洲天庭比沈家、宋家家业更大,更忌惮气海老祖这种人物来捣乱。

%49
气海老祖乃是散修,独自一人,没有什么领地的束缚,想来就来想走就走。关键是他实力强悍,天庭还难有对付的手段!

%50
其他的魔道、散仙,甚至是方源(天庭一直以为方源乃是七转修为),龙公都不是很防备。因为他们实力不强,把他们大多数集合起来,恐怕还不是龙公一人的对手。

%51
但气海老祖不一样,双方较量过后,龙公也为之忌惮。

%52
他是有杀手锏三气归来,但气海老祖就没有吗?

%53
肯定有!

%54
坦白来讲,身为一个散仙,要达到气海老祖这种层次,是极端罕见的。这种人物的存在,对各方势力都是巨大的影响,哪怕他一门心思潜修。更是对天庭大业的巨大障碍。

%55
按照龙公的想法,这种人物必须铲除!因为对宿命蛊的修复计划,威胁实在太大了。

%56
所以,方源哄骗他说将来自己被天庭围攻,龙公打骨子里就相信。

%57
但现在方源重生,坏事了,让气海老祖警惕起来了。

%58
然而仙蛊屋龙宫,龙公又不想放弃。

%59
这该怎么办呢?

%60
龙公头疼。

%61
他想:和气海老祖冲突起来,必定是方源喜闻乐见之事,甚至这就是方源的算计。但若不和气海老祖产生冲突,难道自己真的就要放弃龙宫吗?

%62
绝不可能!

%63
“气海仙友,之前的那些东西,只是天庭的补偿而已。”龙公沉吟道,“接下来咱们可以谈谈有关龙宫的交易。方源那魔头狡诈多端,出了重金,请你出手阻拦我等,打得好算盘!但他区区一位七转蛊仙,自以为能将我等八转玩弄于鼓掌之间,实在可笑。”

%64
说到这里,龙公语气郑重地道:“他出了多少资源,我天庭出的比他更多!并且我也不要求气海仙友你出手,只需要作壁上观即可。”

%65
“哦?此言当真?”方源微楞。

%66
“当真!”龙公十分干脆。

%67
“气海前辈……”东海二仙想要阻拦,但方源却伸手,制止他们的言语。

%68
方源对着龙公一笑,透露出些微戏谑之意:“龙公仙友,方源可是拿出了尊者真传给予了我。”

\end{this_body}

