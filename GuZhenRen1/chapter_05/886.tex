\newsection{领土蛊}    %第八百八十九节:领土蛊

\begin{this_body}

悔哭海域,群仙悬立。

庙明神等四人各占据一座辅阵,不断催动,大阵的力量持续渗透到海底大阵内部。

方源正在布置仙阵。

他气息勃发,仍旧是七转层次,浑身笼罩一层淡青光晕,双眼充斥蓝芒,如海如电。

一挥手,大量的凡蛊仿佛蚊群,泱泱一片飞到半空,盘旋飞绕,很快就形成一个辅阵雏形。

随后方源取出一份份的仙材,动用一记记杀招,充实辅阵雏形。

新加入进来的任修平等人,目光灼灼,紧紧盯着方源,看着他布阵,皆是沉默。

蛊仙们的脸上神情都很复杂,就连沈从声也是如此。

“能够利用仙材布阵,这就是阵道宗师啊。”

“楚瀛修行变化道,没想到还在阵道上有如此造诣,极为难得。他是如何兼修的?”

“恐怕他是找寻到了变化道和阵道兼修的妙方。就像天庭的雷鬼真君兼修魂道、雷道,九灵仙姑兼修变化道、奴道一样。”

“可不只是阵道宗师境界那么简单,楚瀛布阵的手法是如此熟稔,更可怕的是,他根本不缺阵道仙级杀招,布阵的效率真是当世罕见!”

众仙越看越是吃惊,方源显露出来的阵道造诣让他们心头震动。

“有了这样的手段,功德碑上的许多任务,他都能应付得来。”

“功德碑上的任务,大多数整理山河,调谐生态,这对楚瀛而言,真的是太有利了。”

“难怪他能完成这么多的任务!”

“寻常阵道蛊仙要布置仙阵,就需要使用仙蛊。但楚瀛乃是阵道宗师,只需要付出仙材即可。这个质变太可怕了。”

蛊仙们看了不免又嫉又妒,同时暗暗羡慕庙明神等人,他们傍上楚瀛真的是傍上了一条大粗腿。

任修平已经开始认真考虑:是否这一次任务完结之后,他就抛弃沈家这边,来投靠方源。

就目前来看,方源器量很大,之前任修平和他结的仇,完全被他暂时抛在了脑后。

任修平担忧的是,方源身边已经有了庙明神一伙人,就算是土头驮、曾落子和方源的关系,也比他好很多。

“对于楚瀛而言,他完全可以和看得顺眼的人合作。何必和我这个和他结仇的人联手呢?”

任修平心中一动:“虽然有了阵道宗师境界,楚瀛就能利用仙材布阵。所以,他最大的桎梏便是仙材。若是仙材短缺的话,我便能利用此点,主动贡献仙材,和他展开更加密切的合作。”

任修平决定舍弃一些自身利益,唯有如此,才能打动方源,让他和自己合作。

“此阵布下,还请任修平仙友先去坐镇。”方源这时开口道。

他已经布置好了一阵,这座辅阵和之前的几座一样,虽然是在半空中搭建,一旦搭建完成,就沉降下去,没入海水以下一截。

“好说,好说。”任修平呵呵一笑,飞落下去,他也不检查辅阵,直接入阵。

任修平没有阵道造诣,就算方源在他眼前布阵,他也看不到这座辅阵的奥秘,但他相信乐土仙尊的手段,不怕方源算计他。

乐土仙尊禁止蛊仙们相互攻击,极大地增强了蛊仙们之间彼此信任的程度。

方源继续布阵。

一座两座三座……

一位位蛊仙入阵,辅阵的力量越来越强。

任修平的脸色却不太好看了。

方源手中的仙材竟是层出不穷!

这段时间里,他看到了霸铜,看到了地极天罡,还看到了青玉风流、乌青墨石、天翔玉……

方源的身家之丰厚,让群仙们哑然。

任修平更是暗中倒吸了好几口冷气。

对方如此身家,自己怎能比?

更让任修平有些灰心丧气的是方源的态度。

他付出这些仙材布阵,居然眼睛眨都不眨一下,干脆利落,不把这些付出放在眼里!

方源虽然缺少仙材,但那是为了大规模升炼仙蛊。用来这里布阵的仙材,不过是方源库藏中的一小部分,就算再来上百座这样的辅阵,方源也完全能承担得起。

所以,方源的表现落到任修平等蛊仙的样子,就是完完全全的财大气粗了!

耗费了不少时间和精力,辅阵终于布置妥当。

沈从声等人都落入辅阵之中,这些辅阵分部在海底大阵的四周,若从高空鸟瞰,就像是孔雀开屏。

每一座辅阵就仿佛是孔雀尾巴的羽心,它们催吐出来的力量,汇集成流,流入到海底大阵这种。

辅阵看似相同,但实际上彼此都有异样。

这是方源根据蛊仙各自的特点,进行的调整,比如庙明神的辅阵中就混杂了不少宇道仙材,更有助于他的发挥。任修平的辅阵中还有几个空缺位置,可以让他的荒兽坐镇其中,充分发挥他奴道的特长。

除此之外,辅阵的位置也有考究。方源将花蝶女仙、蜂将这类修为薄弱的蛊仙,安排在了外围。而沈从声的辅阵则距离海底大阵最近。

方源也给自己布置了一个辅阵。

他进入辅阵。

这座辅阵乃是一座炼道蛊阵,唯一的作用就是增幅方源的炼道杀招威能。

在辅阵中,方源催动了杀招——炼阵雨!

一时间,海面上细雨飘飞,淅淅沥沥。

无数的雨滴进入海底大阵,炼道的力量开始运转,一只只蛊虫被方源炼化。

沈从声等人起先还不觉得这雨水有多厉害,等过了片刻,这些人纷纷瞪圆了双眼。

因为在这段悄无声息的时间里,方源已经不知不觉地炼化了海底大阵最外的一层边!

“这是什么杀招?连海底大阵都能攻克!”

“好像是炼道杀招?”有人不太确信。

“如此厉害的炼道杀招……难道楚瀛还兼修炼道?”有人顿感匪夷所思。

“这应该是他的那座辅阵的作用吧!”

蛊仙们思量良久,还是下意识地选择相信这一切是方源的那座辅阵的威能。

如果楚瀛在炼道上,还有如此恐怖的造诣,竟能将乐土仙尊布置的大阵炼化。那岂不是三系同修?

这叫任修平、庙明神这些主修一道的蛊仙精英们情何以堪!

即便是辅阵的作用,也让这些蛊仙心头震动不已。

沈从声瞳眸微缩,比起辅阵的威能,他更惊讶于方源的野心。

他已经看出来了。

自己这些人主持的辅阵,就像是一根根的铁棍,而海底大阵就宛若缝合密实的木板箱。

辅阵的力量渗透到海底大阵中,如同铁棍撬动了一块块的木板。

而方源操纵的那座辅阵才是最核心的力量,它能够顺着木板被撬动起来的缝隙,渗透进去,然后将这些木板纳为己有!

“这等阵道造诣,还有这等野心……他是想将整座大阵都收入囊中啊。”

“楚瀛……”沈从声神情严肃,他暗暗地将这个名字深深地印在自己的心头。

炼阵雨这个杀招,方源已经不是第一次用了。

此招以八转水炼仙蛊为核心,方源又采取了诸多蛊虫辅助,将八转气息遮掩,伪装成了七转。

这一世重生后,他吞并琅琊派,获得八转仙蛊水炼,开发此招。

而后就回到南疆,用这个杀招,炼化了池曲由布置的大阵,俘虏了阵内许多蛊仙。

这一次方源再用,同样是七转气息,又经过辅阵的掩护,沈从声也看不出当中八转的实质来。

八转杀招的威能自然非同凡响,海底大阵被方源逐渐炼化。

他炼化得越多,就对大阵掌控越深,再有阵外的多座辅阵协助,一时间顺风顺水,进展颇为迅速。

不过,当他炼化了三成大阵之后,炼阵雨杀招的效果就猛地下降一大截,进展开始骤缓。

“炼阵雨只是杀招而已,真正的效用还得看我对海底大阵的理解程度。”方源早有预料,并无意外,他有充分的耐心。

他决定在这里耗费一段时间,为了悔蛊,以及大阵中的其他仙蛊,这点投资是完全值得的。

本来他还有点担心其他蛊仙。

但现在,就连沈从声都被他邀请进来,方源根本就不怕其他蛊仙趁机超越自己了。

炼阵的速率越来越慢,终于,方源遭遇到了一个关口。

这是一只土道七转仙蛊,身躯扁平如圆盘,通体黄褐色,有头有尾,还有四只触脚宛若弯钩,均匀生长在圆盘周边。

“这是领土蛊。蛊仙将其种入泥土深处,就能散发一股玄妙威能,浸透泥土,不断向四周扩散。扩散到极限后停止,方圆千里之内的土地上的一切事物,都被蛊仙掌控。”方源认出此蛊。

领土蛊其实很常见。

尤其是在南疆这块地域。

很多的山寨中,哪怕不是擅长土道的,也会想方设法得到一只三转或者四转的领土蛊。山寨会将这只蛊种在地下,由位高权重的蛊师负责,极大地增强领地的掌控程度。

领土蛊是一只实用的蛊虫,但价位高,常常被势力所用。个人蛊师并不追逐这个蛊。

领土仙蛊原来收藏在龙鲸乐土当中,这只蛊对于方源而言,也挺实用。

方源拿这只蛊虫经营仙窍,非常方便。领土内的山河地貌,方源都能通过领土仙蛊随意调整。

就算不实用,方源也得将这只土道仙蛊炼化了。它现在就是一道关卡,卡住了方源继续前进的道路。

------------

\end{this_body}

