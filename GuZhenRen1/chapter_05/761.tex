\newsection{凶杀案}    %第七百六十四节:凶杀案

\begin{this_body}

房家的出击,震撼了整个西漠正道。

毫无疑问,偷道杀招是一个大杀器,一时间所有的西漠正道势力都被震慑住。

方源很快也得到了此战的战报,内容还很详实,有着整个战斗过程的影像。

这战报当然是房家主动给他的。

“房家应该掌握了一道盗天真传!”方源对这个发现也分外吃惊。

方源旋即想到了大盗仙蛊。

这只仙蛊便是他从房家手中索要来的。

根据某种传闻,房家收获这只大盗仙蛊也是一场意外。是从一位投靠而来的散仙手中得到的。

“或许当年,房家得到的不只是大盗仙蛊,还有盗天真传的线索罢。”

“超级势力的底蕴,真是……”

方源摇了摇头。

就像武庸抛出玉清滴风小竹楼,房家拿出了偷道仙蛊屋,还有杀招偷道来,也迅速改善了不利的局势。

“毫无疑问,这是一种战略性质的震慑。房家的局势将因此改变,看来上一世就是因为这个,导致正道势力间的妥协。”

杀招偷道能够盗取道痕,一旦用出来,就能直接摧毁自然资源点。

正道势力的主要基石,便是分布在各个地点的资源点。

和破坏这些资源点不同,杀招偷道效果立竿见影,并且是绝户计,釜底抽薪!

房家出击的这一战,董家太上大长老董陆沉无恙,但宝月绿洲从此除名。

真要逼急了房家,房家就如法炮制,西漠正道中哪个势力能挡住这般的突袭?

就连董陆沉这样的八转蛊仙在场,都做不到!

方源不禁暗赞房睇长的选择。

他专门挑了一个硬骨头,有董陆沉在的地点来进攻。

这对其他势力的震慑程度就到位了。

但同时,他也有分寸。

宝月绿洲只是一个中型资源点,连大型都算不上。它对董陆沉有不同的纪念意义,因为是埋葬他爱妻的所在。所以每年到了某个固定的时期,董陆沉就会来此地悼念亡妻。

所以,董家的损失并不大,董陆沉的愤怒也不多。因为只是道痕损失,他爱妻之墓却未受分毫损伤。

“这里的分寸,把握得恰到好处!并且还充分利用了此战战果,不仅是震慑西漠正道,还来震慑我。”

方源捏着手中的信道凡蛊,冷笑了几声。

在这信中,房睇长还特意提到天露,他说道:天露的战损应当很少,只在十滴左右的程度。让方源再找找看,说不定能发现另外的二十滴天露。

方源之前扣下三十滴天露,直接报告给房家,说是战损,乃是之前影无邪入侵所致,并且恐怕以后都会减产。

这只是一个借口,对于真相,他和房家心知肚明。

但房家没有和他心照不宣,而是在战后委婉地警告他:不要太过分了,扣下十滴就可以了,剩下的二十滴你得交上来!

天露毕竟是八转仙材,房家不想放弃这份收益。

当然,里面更多的是打压方源。你一个外人刚刚加入房家,就想上房揭瓦?给我收敛点!你看看这一战,我们房家还是很有实力的,你好好想想,仔细掂量!

这才是房睇长送信方源的真正意思。

正道的交流比较含蓄,不像魔道那般直接。

“只是这恰恰说明你们的虚弱啊。”方源心头一片雪亮。

此战,房家没有动用其他仙蛊屋,证明豆神宫一战,那几座仙蛊屋都有极大的损伤。

房家催动的偷道仙蛊屋,明显是残破的。催动出来的杀招偷道虽有八转程度,但方源已看出此招对仙蛊屋的负担极大,应该不能多用。

甚至并不稳定!

若是杀招稳定,又能频繁催动,房家早就拿出来用了,不必拖到这个时候。

杀招偷道是房家的底牌,房家并不想用,但没有办法。

很可能,房家掌握的盗天真传也并不完整,又或者他们没有完全消化吸收。

因为房家最多的是阵道蛊仙,擅长的也是阵道,和偷道风马牛不相及。

方源先是给房睇长回信。

信中说明二十滴天露找到了,是自己之前视察的时候粗心大意。

随后,他眼中阴芒一闪即逝:“该我亲自动手了。”

仙道杀招——翠流珠!

这段时间来,他已经是掌握了天露绿洲大阵,并且还布置了阵中小阵,不仅能够营造出自己镇守的假象外,还能遮掩方源的大动静。

而翠流珠杀招也改良成功,催动的时候更快,在目的地那边气息也减弱了许多,远没有之前那般的张扬。

方源来到燎萤戈壁,这里怪石嶙峋,植被稀少,充斥着一种虫群,叫做燎萤。

燎萤数以亿万,当中有着海量燎萤蛊,从三转到五转。

燎萤蛊是一种炎道蛊,比较高端,最低的转数就是三转,但至今没有六转出现过。

这里是房家的中型资源点。

房家在这里布置的仙阵,利用燎萤蛊营造出炎道环境,豢养其他的大量的炎道蛊虫。

镇守这里的,只有一位六转房家蛊仙。

方源动用见面曾相识做出伪装,随后直接扑杀过去。

他杀死这位房家蛊仙,丢下尸体,又摧毁了大阵,还屠戮了所有的蛊虫,并将戈壁破坏得面目全非,这才扬长而去。

值得一提的是,他动用的是土道手段。

琅琊福地的库存中,就有数只土道仙蛊,并且方源的土道杀招也是十分丰富。

做了这事之后,他便回到天露绿洲去。

房睇长这边,刚接到方源的回信,笑了一笑,随后便得知了自家燎萤戈壁被摧毁的噩耗!

刹那间,这位房家的智道大宗师,都微微一愣。

刚刚的好心情荡然无存。

房家死人了!

燎萤戈壁不重要,重要的是房家的蛊仙战死了!

一直以来,房家蛊仙都较为稀少,和其他西漠正道势力对比,处于劣势。只是最近吸纳了败军老鬼、鹰姬还有算不尽之后,这才有了起色。

但这三人都是外人,不姓房,终究隔了一层。

房家蛊仙死了,哪怕是六转,都对房家而言是一个惨痛的损失!

房功重视无比,立即赶来和房睇长商量。

交谈了几句后,房功留守房家大本营,房睇长当即动身,亲自前往燎原戈壁查探凶杀现场。

“凶手虐杀了我族蛊仙!”

“还把戈壁里的所有资源都摧毁了。”

“这明显不是谋财害命,而是充斥着报复的意味!”

房睇长动用智道手段,很快感知到凶杀现场残留着的某些残暴的念头,愤怒和仇恨交织的情绪。

“土道手段……难道是董家的复仇?”

房睇长旋即想到这一点,董家擅长的是土道,但他又很快微微摇头。

“若是董家做的,未免太过明显,更有可能是其他正道势力浑水摸鱼啊。”

“情况有点糟糕……”

房睇长脸色阴沉下来。

他是智道大宗师,对于祭出杀招偷生震慑西漠正道势力的后果,有过许多推算。

其中就有一种情况,便是西漠正道势力中有人想要挑拨,不敢公然刁难,隐于幕后,施展诡计。

方源的手脚相当干净,留下的线索都是他想要留下的。

还有一个要素,那就是房家缺乏信道蛊仙,房睇长是智道,不修信道。

若是专修信道的蛊仙来,那就有点悬,说不定能找到有关方源的蛛丝马迹。

但房家不成。

房家就算邀请信道外援,首先就不能是其他西漠正道,这就排除了一大半。要邀请散仙和魔道,至少得要有七转修为。

但身为正道的房家,平时和这些散仙、魔仙关系也不是很好。

纵然有关系不错的散仙,或者暗中交易的魔仙,但这些人未必就是信道流派啊。

房睇长没有找到有价值的线索,他回转动大本营的时候,脸色阴沉如水。

西漠正道那边很快得到了消息。

当即就有八转大佬问信董陆沉。

董陆沉听到这个消息,当即就有些懵!

他吸了口冷气,心中升腾起怒意,暗中诅咒:“究竟是哪个王八羔子,想要害我们董家!他房家有杀招偷生,没有对付我董家的超级资源点,是有分寸的。但是这王八羔子杀死了房家的蛊仙,那就破了底线了。真是该死!”

“这人用了土道手段,而我就是土道蛊仙,我董家主修土道,这明显是想要嫁祸啊!真阴险,真够毒,偏偏我还不能直接反驳!”

董陆沉心中积蓄着一股闷气。

他还不能四处大叫,宣扬自己不是凶手。

首先没有人公然怀疑他,就算有,他这般宣扬,岂不是示弱?

刚刚房家破坏了你董家的一处中型资源点,你这样大叫无辜,岂不是说明你害怕了房家?

这不仅有损董陆沉的个人名誉,也打击董家的声威。

所以,董陆沉尽管恨极了方源,也极想宣扬自己的无辜,但对于问信的八转大佬,也只能哼哼两句:“大快人心(我很生气)!只是此事是谁做的?我董陆沉要对此人大加奖赏(抽筋扒皮)!”

“呵呵。”那边的八转大佬回应得意味深长。

董陆沉顿时心情更不好了。

你呵呵的笑个屁啊!

我容易么我?

我董家的势力范围和房家挨着,房家被排挤打压,前番出手就是为了震慑我们。这次房家蛊仙被害,若房家强势反击,我董家的危险最大啊!

\end{this_body}

