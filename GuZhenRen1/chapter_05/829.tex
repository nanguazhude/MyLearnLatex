\newsection{清点收获}    %第八百三十三节:清点收获

\begin{this_body}

%1
至尊仙窍。

%2
一座雄阔的紫金宫殿,悬空而立。

%3
仙道杀招——梦里轻烟!

%4
龙人分身驾驭着这座仙蛊屋,立即就散发出大股大股的浓烟。

%5
浓烟所到之处,将一切吞没。方源用以试验,特意迁来的一只太古年兽,根本无法抵抗,只是惊嚎一声之后,就被梦道浓烟卷席了去,不见踪影。

%6
“收。”龙人分身心念一动,梦道浓烟便像是巨蟒回窝,迅速缩进龙宫中去。

%7
它之前所吞没的一切,自然也都卷进了龙宫之中。

%8
这一招就是上一世,方源亲眼目睹的梦道杀招。当时,龙宫催出浓烟,将监天塔都侵蚀。哪怕监天塔都已经虚化,也着了道。

%9
真的是一招鲜吃遍天,短时间内,五域蛊仙界哪怕是天庭,都严重缺乏应付梦道的手段。

%10
梦道的效果立竿见影。

%11
不过此刻方源亲自催动,倒是发现这一招梦里轻烟,并非是之前他料想的梦道攻伐杀招,它更多的是擒拿俘虏的作用。

%12
就像刚刚,龙宫只用这一招,就将太古年兽八转级别的战力俘虏擒获,然后镇压在宫殿之中,轻轻松松。

%13
龙宫乃是奴道的仙蛊屋,即便是有如梦令这只八转梦道仙蛊,也改变不了它的根本性质。

%14
梦里轻烟杀招就是为了配合龙宫所设——先将要奴役的对象,擒拿捕捉进来,然后加以奴役。

%15
除了梦里轻烟之外,龙宫中还有两记梦道杀招。

%16
一个名为梦启,智道效果,可以让蛊仙睡觉入梦,在梦境中启迪蛊仙。

%17
另一个名为梦中之梦,可以在梦境中构造出梦境中的梦境。就像龙人分身探索梦境时,成为吴帅,而吴帅又入梦屡屡得到梦启。这就是梦中之梦。

%18
至于之前笼罩龙宫的那片梦境,则是如梦令本身的威能。

%19
方源考量了一番后,发现就目前阶段,也就梦里轻烟比较实用。

%20
梦启杀招虽然高明,但是不只是消耗仙元,还需要消耗天意!也就是说,这个杀招是由天意参与的。

%21
对于吴帅而言,天命所归,龙人当兴。所以,天意扶持龙人,他用这一招完全没有关系,能得到天道垂青和指点。

%22
但是方源不行啊。

%23
方源乃是天外之魔,受到天意的打压和针对,一直以来他都是和天意为敌。他掌握的一个杀招,就名为天消意散,专门消除天意用的。

%24
要指望天意来帮助他,那怎么可能?

%25
至于梦中之梦,这一个杀招是针对梦境,不是直接针对活物。

%26
催动它之后,能够在梦中形成更深一层的梦境。

%27
这有什么用呢?

%28
就是改造梦境。

%29
吴帅当年主要用它改造了龙宫梦境,形成梦中之梦。方源的龙人分身探索的时候,屡屡遭遇梦中之梦。毫无疑问,这样的改造能够让梦境更加复杂,更能令探索者沉迷当中,忘却自我。

%30
方源主要能用它做什么?

%31
短时间内,方源想到的是还是探索梦境——可以改造这些梦境,在梦境当中构筑出梦中之梦。

%32
这些梦中之梦是方源构造的,也是属于他的。如此一来,他就算在梦境中打造出了前营阵地,可供休整的安全场所。

%33
重中之重,还是如梦令仙蛊!

%34
这只仙蛊和方源手中的魂兽令比较相似,前者是梦道蛊虫奴道威能,后者是魂道蛊虫,同样奴道效用。

%35
这只八转如梦令仙蛊的具体运用之法,方源已经从龙灵口中打探得出。

%36
当初,龙公施展龙人寂灭,吴帅也不能幸免,但终究有过努力,作了大量的龙人试验,有一些成果,能延缓死期。

%37
吴帅便利用最后的一段时光,来催动如梦令仙蛊。

%38
他消耗的不只是毕生的八转仙元,还有他的一切情感、记忆、意志。

%39
布置妥当后,他形销骨立,魂魄只剩下点滴,最终坐在龙椅上,因消耗过大,弥留了一会功夫,咽气而亡。

%40
到了最后,他连自己都牺牲了。

%41
他付出了一切。

%42
所以,就目前而言,催动如梦令仙蛊的代价是很大的!吴帅其实并没有被龙人寂灭杀死,而是因为催动了如梦令。

%43
方源暗暗估量:“他是八转巅峰的奴道蛊仙,牺牲自己催动了如梦令,营造出龙宫梦境。”

%44
“近代,原本龙宫藏身之所的福地毁了,龙宫遁出,藏身海底,陆续收服了容婆、张阴等四大龙将。”

%45
“虽然有了四位八转战力,但也因此导致梦境底蕴大减,龙人分身进入探索的时候,龙宫梦境显然有些兜不住的迹象。”

%46
一个很明显的细节,那就是风云府、黑天寺。

%47
这两大门派乃是中洲当今的十大古派之二,并非是百万年前红莲时代的门派。

%48
但是梦境衍化不了,只有依赖龙人分身的记忆。

%49
这就是露馅了,显示出了那个时候的梦境其实已是到达了极限。

%50
“也就是说,牺牲了吴帅这位八转巅峰的蛊仙,大约可以奴役四位八转,一位七转么。”

%51
若是完整状态下的龙宫梦境,龙人分身依靠解梦杀招,恐怕还是要被改造的。所以要感谢这四个东海八转蛊仙,大大消耗了梦境的力量。

%52
单纯用如梦令,谁用谁死,如梦令仙蛊会抽干蛊仙的一切底蕴,转变成梦境。

%53
梦境的规模和威能,应当是根据具体情况,看到底是何等蛊仙催动出来的。

%54
当然,七转、六转的蛊仙没有八转仙元,根本没有资格催动如梦令仙蛊。

%55
八转蛊仙也不能强迫,非得自愿主动。

%56
“也就是说,每一次催动如梦令仙蛊,就得消耗一位八转蛊仙么?”这个代价就大了,方源当然不想亲自尝试。

%57
“但在将来,等我梦道境界足够,是完全可以用如梦令仙蛊为核心,搭配出相应杀招来,限制如梦令,防止它抽干蛊仙,从而有节制地营造梦境。”

%58
“不过用处也不是很大。”

%59
方源细细琢磨了下,单纯的如梦令仙蛊,对他而言并不是很实用。

%60
就算将来有了杀招,可以有节制地营造出了梦境,而不会被如梦令仙蛊直接抽干。但是那也得消耗方源大量的底蕴,仙元、魂魄什么也就算了,关键是真意也会被转化和消耗。这就意味着,方源的境界会随之跌落。

%61
没有了真意,梦境就只是浅薄的梦幻。正是因为梦境中包含真意、各种情绪,所以梦境能够迷惑人。若是排除探索成功,剖除各种情绪干扰,蛊师获得真意,就能擢升流派的境界。

%62
“不过将来,梦境迭出,四处涌现。我倒可以损失一份境界,营造出梦境来作为陷阱,坑害关键人物。将这样的关键人物化为己用,当能打敌人一个措手不及。”

%63
如梦令、梦里轻烟、梦中之梦、梦启,这是梦道上的收获。

%64
对于方源而言,并不是各个都很实用,但的确令他的梦道势力暴涨了数倍。尤其是这些杀招都融入于仙蛊屋龙宫之中,使得方源根本无需多加练习,就能随意使用,非常方便。

%65
除了梦道,其他方面也有巨大的收获。

%66
最明显的,就是张阴、容婆等四大龙将,方源麾下喜添八转战力,而且一添还是四个!

%67
吴帅以及龙人一族的收藏,都在龙宫中保存往后,如今都便宜了方源。

%68
这些遗产,绝大多数的内容都是有关于奴道方面。但价值最高的却是乾坤晶壁、非议峰、军团蚁等。

%69
吴帅当年镇压奴役了书道阁主,掌握了她的仙蛊屋书道阁。吴帅发现:书道阁之所以强大,是因为里面收藏了大量的乾坤晶壁。

%70
乾坤晶壁乃是天地秘境之一,《人祖传》中早有记载,只是一直都处于拆分离散的状态。书道阁中的乾坤晶壁自然也只是当中的一部分而已。

%71
非议峰的价值,打个比方的话,就有点类似于炼海的雏形。琅琊白毛地灵一心想要营造出天地秘境炼海,付出无数代价,至今都没有成功。非议峰却是打造好了,但它充其量只能算是一个天地秘境的雏形,比炼海当然要差一个档次。

%72
军团蚁是绿蚁居士的真传内容,吴帅继承后发扬光大,是他当年主打的攻伐手段,只要豢养出一定的规模,威能十分恐怖,能在八转蛊仙中以一敌众。

%73
最后就是龙宫本身、龙灵,还有奴道境界。

%74
龙人分身闯过梦境之后,奴道境界直接晋升成了准无上大宗师。

%75
当年的吴帅在奴道境界上,恐怕就连无上大宗师只差临门一脚,虽有龙人本身的奴道天赋在,但这份才情也端的恐怖。

%76
古往今来,多少豪杰,你方唱罢我登场。历史长河中,都是繁星璀璨,星光熠熠,这些风流英杰即便是上头有着十大尊者的照耀,也没有丧失自身的独到色彩。

%77
收获是巨大的,方源等若是继承了龙人一族的传承!

%78
要消化这样的成果,非得是经年累月不可。

%79
方源有些无奈,他时间有限,只能选择一些补充到自己的身上。

%80
军团蚁暂时是要搁置的,这种手段只能锦上添花,方源如今已经是可以在八转中以一敌众了。并且豢养军团蚁,要从零开始,消耗的资源也非常庞大。

%81
乾坤晶壁、非议峰其实都是信道的底蕴,方源缺少这方面的手段。之前他还在华文洞天中布置了李小白这个分身,也是在为这方面做打算。

%82
让方源感到有些遗憾的是:龙宫缺少梦道的防御手段。若是有这样的手段,按照目前的情势,梦道领先世界,极可能比冬裘、逆流护身印都实用!

%83
不过再想想上一世,方源也就释然了。若是龙宫有这样的梦道防御手段,也不至于被龙公直接杀进门去,不是吗?

\end{this_body}

