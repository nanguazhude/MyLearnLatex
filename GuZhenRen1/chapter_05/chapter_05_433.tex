\newsection{价格战}    %第四百三十四节:价格战

\begin{this_body}



%1
王明月出货了。

%2
原本宝黄天的市场,只有方源一个人在大批量地供应年蛊。现在多了一个竞争对手,立即引起一阵波澜。

%3
“女人就是女人,这点忍耐都没有。”中洲荣欣得到这个消息之后,嗤之以鼻。

%4
“如此一来,只怕众多蛊仙都会选择观望吧。”他贩卖年蛊经历丰富,立即猜测到接下来市场的变化。

%5
果然,宝黄天年蛊市场的热潮,忽然降低了下来。

%6
王明月成为方源的竞争者,市场上的年蛊一下子供大于求,蛊仙们都按捺住,等待接下来的好戏。

%7
宝黄天中充满了一种风雨欲来之势。

%8
往年自家贩卖年蛊,向来都是生意火爆,今年提前出售,却是景象“惨淡”,几乎无人问津。

%9
对于这一点,王明月也很无奈。

%10
没办法,她手中的年蛊存量比较多,如果此时不早点出手,等到其他两大卖方出售,那么留给她的市场空间就会小很多了。

%11
“王明月居然提前出售了?”和荣欣岿然不动的情况不同,谢宝树那边却是陷入迟疑。

%12
谢宝树比较了解王明月,此人乃是老卖家了,经验丰富,怎可能看不出提前出货的弊端呢?

%13
“最大的可能,恐怕就是她手中的年蛊存货也是不少啊。”谢宝树表面上不显什么,但心中却越发担忧起来。

%14
一天之后,他终于决定也开始动手,在宝黄天中出手年蛊。

%15
“谢宝树也出手了。”

%16
“三大巨头当中,已经有两家提前售卖了。”

%17
“那是!今天不同以往啊,原本的草场中进来了第四头狼。”

%18
宝黄天中,蛊仙们议论纷纷。

%19
“可恶。这两个家伙……”中洲的荣欣得知这个消息之后,咬了咬牙,也只能跟上。

%20
局势已经和之前不一样了。

%21
之前,方源单独售卖,荣欣可以忍耐,不动如山。再加上王明月,荣欣也不过稍稍紧张一下。但若谢宝树也出货,荣欣的压力就大了。

%22
四大卖家,就只有他一个人不出手的话,那么最吃亏的无疑就是他了。

%23
只是荣欣也有难处。

%24
他手中的年蛊,是炼蛊得来。

%25
因为没有到往年的时限,他手中的货量还没有积累足够。所以四大卖家当中,他是手中存货最少的。

%26
逼不得已,荣欣只能咬牙顶上,在宝黄天中出售年蛊了。

%27
“我手中货量最少,可不能硬来。”荣欣已有所预感,他现在最需要的就是时间,好让更多的年蛊被炼制出来。

%28
想了想,他便主动联系谢宝树、王明月两方。

%29
这三位早已经熟知多年,在年蛊这块上,相互较量和交流,自然有着联系彼此的信道蛊虫。

%30
“虽是有新人进来,但我想我们三方还是应当依着旧约的,不是吗?”荣欣如此提醒二人。

%31
谢宝树、王明月二人自然是一口答应下来。

%32
本来,他们俩提前出售年蛊,已经就破坏了一些旧约。不过因为是特殊情况,可以理解。

%33
但若是再破坏其他约定,尤其是当中的年蛊价码,那就是坏了规矩了!

%34
卖方多了,自然会打价格战。

%35
这也是商战的强力手段。

%36
不过太强力了,往往伤人伤己。

%37
谢宝树、王明月、荣欣三人相互之间,已经斗了不知道多少回,约定售价不变,除了奈何不了对方之外,更主要的原因还在于维护自身利益。

%38
这里就有一个问题。

%39
若是方源一方,忽然降价,那该如何是好呢?

%40
他们三人可能会遵守什么约定,但方源却是新人,就算荣欣来劝告他,他恐怕是不会听从的。

%41
不过,荣欣、谢宝树、王明月等人,似乎并不担心这一点。

%42
四家同时售卖年蛊,方源那边顿时感到有一股销售不动的趋势。

%43
“我到底是新人。虽然货物都是精品的,但是在其他蛊仙的心目中,地位怎么比得上那三大巨头呢。”

%44
王明月、谢宝树、荣欣三人,做这样的买卖,已经许多年,早已建立了口碑和声誉。这一点,是方源万万比不上的。

%45
这是方源弱势的地方。

%46
不过,方源自有手段来解决。

%47
很快,荣欣、谢宝树、王明月三人就几乎同时,探知到了一个重大的消息。

%48
那个神秘的卖家,开始降低年蛊的售价了!

%49
“果然用了这一招么。”

%50
“哼,我还以为会有什么手段!”

%51
三大卖方并不着急,甚至有些不屑。这种自己降价的手段,他们早已经玩腻了,甚至因此吃过不少苦头。

%52
商战中主动降价,是一柄双刃剑,伤人伤己。

%53
但对于买家而言,方源主动降价,无疑是一个好消息。

%54
他们等待的,不就是这样的机会吗?

%55
他们就像是闻风而动的鲨鱼,立即蜂拥而至。

%56
方源的年蛊开始大量出售,和谢宝树、王明月、荣欣三人形成鲜明的对比。

%57
“对方似乎很是经验老道啊,这价格降低得很有水准。给人感觉很多,但其实并没有减少多少呢。”王明月暗中观察。

%58
“但这又如何呢?”

%59
“整个宝黄天的年蛊市场,非常巨大,你一个人怎可能供应全五域的买家?到头来,还是亏了自己。明明可以卖出更高的价格呢。”

%60
三大卖方一直保持着价格不变,但几天之后,他们坐不住了!

%61
“怎么回事?这人的年蛊存货这么多?似乎绵绵不尽啊。”

%62
“他怎么可能囤积到这么多的年蛊?情况有点不对劲。”

%63
三大买家的心情都变得有些凝重。

%64
但是要让他们跟着降价,却又让他们迟疑。

%65
说到底,是心态不同。

%66
方源是新人,赚多少都是新鲜。但对于这三大卖家而言,却是因为经验老道,每年都会预想估算出大概的利润。

%67
这样一来,若是他们跟着降价,心中的利益就会缩减。

%68
犹豫是很自然的。

%69
几天后,方源售卖仍旧火热,三大卖家见着势头不对,王明月首先商谈其他两位,要不然也跟着降价?

%70
谢宝树面色冷淡:“眼下这局面,那人的货量恐怕是很足的。不降价,我们这边根本卖不动。”

%71
“是啊。”王明月附和道,“此人手段不俗,降下的价格,恰到好处,让我们三家都处于下风。”

%72
荣欣冷哼一声:“失利与否,还要最后年蛊售卖的结果呢。若是降价太过,卖得多,总体亏损,又有何用?”

%73
他不太同意此法,还想再隐忍一番。

%74
没办法,他手中的货不多,若是一降价,买的人多了,货物不经售卖,手里缺货的话,会很折损形象和声誉。

%75
但没办法,其他两家既然想要降价,那荣欣也只好奉陪了。

\end{this_body}

