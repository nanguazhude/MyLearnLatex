\newsection{我拒绝}    %第七十二节:我拒绝

\begin{this_body}

%1
琅琊福地,云盖大6。?

%2
第一云城之中,琅琊地灵招待方源坐下。

%3
“来,方源,喝口茶。这可是有名的火虹茶!”琅琊地灵热情洋溢地笑着。

%4
方源接过杯盏,只见这杯茶水,红彤彤的散着光。仔细分辨,里面又有一道虹光,像是游鱼一般,忽左忽右,时上时下,欢快跳脱得很。

%5
方源有五百年前世经验,眼界开阔得很。

%6
寻常的好茶,要细细品味,浅尝辄止。但此茶却是个例外,要一口气喝掉。最关键的就是茶水中的那道虹光,喝进肚中,才能算是真正品味了火虹茶。

%7
方源端着杯盏,注视着茶水,却没有喝。

%8
全身的肉身就是优秀,一时间,他脑袋里念头纷起,思绪四下散开来。

%9
这一任的琅琊地灵,喜欢喝火虹茶。而上一任的琅琊地灵,却是爱喝云烟茶。

%10
两种茶水不同,也体现了两任琅琊地灵大相径庭的秉性。

%11
而云烟茶的主要配料之一,就是浮球茶草。

%12
这是太白云生在他的福地中,主要栽种的植株。这种浮球茶草漂浮在天中,很是不凡。

%13
浮球茶草算是比较高端,生长在天空之中,凡人很难材采摘得到。

%14
但在北原,还有一种相当亲民,接地气的茶花,名为奶茶花。

%15
这种花在北原的草地上,生长得很多。在一些水草丰盛的地方,几乎随地可见,数目极大。奶茶花的花形,呈现杯子形状。每隔几天,杯子里都会积蓄出满满欲溢的花汁。

%16
花汁香甜可口,吸引无数野蜂群蝶汲取,同时也借助这些蜂蝶,来为奶茶花传播花种。

%17
当然,常人也可以喝。

%18
生活在北原的蛊师,乃至凡人。几乎都喝过这种天然饮品。

%19
其实茶和酒,不管是在北原,还是在其他四域,都是很盛行的食物。

%20
在南疆。就有蓝海云茶砖贩卖。这种茶砖的产地,却不是南疆,而是东海。南疆的级势力之一的翼家,和东海有千丝万缕的联系。每年他们往南疆各地,贩卖出难以计数的蓝海云茶砖。赚取海量的财富。

%21
而在中洲,茶的种类更是名目繁多。灵缘斋的青浦茶,万龙坞的蛟影茶,都是名声广传的茶水。

%22
还有一些茶,是蛊仙特制,只有该蛊仙才会制作的茶水。

%23
比如黎山仙子的雪油茶,凤九歌的碧海潮生茶等等。

%24
琅琊地灵见方源久久不喝,咳嗽一声,不由催促道:“方源长老,快喝快喝!你可能不太清楚。喝这火虹茶的要诀。”

%25
方源摆手道:“火虹茶我早闻名已久,只是此时端起这杯茶水,忽的想到其他种种,一时失神。”

%26
琅琊地灵:“那方源长老你想到了什么?莫非是我琅琊派开太丘的近况吗?”

%27
“不是这个。”方源笑了笑,将他脑海中随意联想的东西,悠悠说了出来。

%28
琅琊地灵一脸失望之色,勉强附和道:“你是春秋蝉之主,重生之人,眼界宽广,经历众多。对于茶知道的不少。但你知道吗?不管是茶和酒,都是经过食道流派,才扬光大,广为流传的。”

%29
“哦?有这样的事情?”方源顿时涌起兴趣。

%30
准确的说。他对食道很有兴趣。

%31
他现在正经营仙窍,要的目标就是经营各种资源,满足手中仙蛊喂养的需求。

%32
但若掌握了食道,他喂养仙蛊势必更加容易。

%33
可惜的是,食道非常神秘,历史上盛行一时。但这个时间也非常短暂。越往后,流传的就越少,比智道传承还要稀少。搞不好,都快要彻底泯灭了。

%34
关于食道,方源在义天山大战时,通过魔尊幽魂的忆,知道一些鲜为人知的秘辛。

%35
原来食道的创始者,不是人族蛊仙,而是兽人蛊仙。这个原因,恐怕也是食道如今境况的要因素。

%36
但对方源而言,不管是谁创造出来这个流派,只要对他有用就行。绝大多数的人族蛊仙,都有种族自豪感,但方源没有。

%37
如果毛民更能帮助他,达到永生的目标,那他绝对会义无反顾地舍弃人族身份,转变成毛民。

%38
现琅琊地灵对食道也有许多了解,方源连忙继续询问。

%39
琅琊地灵便透露道:“其实任何一种茶水的制作,或者酒水的酿造,都是一份未完成的蛊方。”

%40
“如果彻底完成,便能炼制出特别的茶蛊或者酒蛊。这个世间,已经有很多的酒蛊、茶蛊。就比如很有名的酒虫、大烟茶蛊。”

%41
酒虫不用多说,方源重生之后,就受益于它,初期展很是迅猛。

%42
大烟茶蛊,则是一种能让蛊师上瘾,愈加颓废的蛊虫。受到当权者的打压,算是这个世界的毒品。

%43
“可惜食道流派并未流传开来,已有泯灭迹象。”

%44
“在当今,许多蛊仙要考较彼此的炼道造诣,也不用中洲炼蛊大会中的文斗武斗,而是直接面对面,做一杯茶,相互喝,往往就能分出高下了。”

%45
琅琊地灵的答,让方源有些失望。

%46
地灵知道的也不多,并且,知道的内容也是关于炼道方面的。

%47
“魔尊幽魂本体,一定知道很多,熟知大量食道手段。毕竟他生前,继承了食道真传。可惜,他现在被困在级梦境里面,日夜消磨。我要打他的主意,太难太难”这个念头在方源心中一闪而过。

%48
琅琊地灵又和方源闲谈几句,他见方源的话题始终不离茶水,不由急躁起来,直言道:“方源,和你明说了罢。我请你喝茶,是想有个任务交给你做!”

%49
“来了。”方源心道一声,目视地灵,“请说。”

%50
地灵单纯,他的用意早已经被方源猜到。

%51
果然,下一刻琅琊地灵开口:“我派开采太丘的大计,你是知道的。这个建议,更是你亲口提的。还有传送蛊阵,也是你亲自布置下来的。”

%52
“正因为有你打头。最近一段时间,我们的蛊仙才能顺利出入太丘,不断改造环境,扩张那边的营地。但现在却出现了一个麻烦。”

%53
方源点点头:“太上大长老是想说那头落星犬?”

%54
“不错。正是这头该死的上古荒兽。已经打伤了我派数位蛊仙。我早已布了剿灭它的任务,可惜派中虽有人响应,也都是惨败而归。我知道真正有希望战胜这头落星犬的人,就只有方源你了。可是你为什么不来接这个任务呢?”琅琊地灵直接质问道。

%55
方源苦笑一声:“我最近很忙。”

%56
他有苦衷。

%57
距离上一次地灾,到现在已经有一个多月过去了。

%58
第二次地灾。将要来临。

%59
方源现在都在准备应付这场地灾。

%60
但地灾的事情,方源不打算告诉琅琊地灵。事实上,他的至尊仙窍,这个秘密也是隐瞒了琅琊地灵的。

%61
只要不对琅琊福地造成损害,就不算违背琅琊派的盟约。

%62
这个秘密,是方源的底牌,方源打算保守在心底。当然,影宗那边是一清二楚的。

%63
但方源算定,影宗定然是想收至尊仙胎蛊的。如此一来,影宗必不会大张旗鼓地去宣扬至尊仙窍的存在那岂不是给自己平添对手。找不自在吗?

%64
在琅琊地灵看来,方源重获新生,已是极大幸运的好事。对于仙僵重生,方源也有一大堆的借口和解释。因为他真的知道很多种,仙僵重获新生的方法!

%65
方源有苦衷,但他不能明说。

%66
琅琊地灵却点点头:“你的确很忙。”

%67
他是地灵,琅琊福地生的事情,他一清二楚。

%68
琅琊地灵游说道:“你修行起来真的十分刻苦。但完成任务,也是帮助你修行。你可别忘了你的狗屎运仙蛊。你要喂养它,就得寻找到六头不同的荒犬。用它们的粪便来喂养这只运道蛊虫呢。”

%69
“在以往,我上一任喂养狗屎运,都是借助宝黄天,从那里采购。但现在宝黄天关闭。就算将来开启,我也建议你在仙窍中豢养犬兽。你也只是懂修行的,知道这样做的好处。”

%70
“大人的话,是真知灼见!”方源立即点头。

%71
狗屎运,对他而言很重要。

%72
对付天意,运道可借。

%73
命运。命运。

%74
前者宿命,命是定好的,不能更改。

%75
后者运气,运是变化的,可以借助。

%76
这个珍贵的情报是方源从和影宗的交易中,得到的成果之一。

%77
方源探索探求,得到天意的关注,引兽潮,想要铲除他。但是最后关头,却让方源恰巧现了一具太古尸骸,是合适布阵的新地点。

%78
方源事后想,觉得此事并非是巧合,恐怕狗屎运是其中的功臣。

%79
这只仙蛊虽然对攻防腾挪没有任何帮助,但隐形的助益,似乎很大。

%80
“巨阳仙尊己运、众生运、天地运狗屎运就是己运真传的精髓之一,掌握我手。而众生运埋藏在王庭福地,连运、招灾、鸿运齐天就是源自于此。真是可惜了!”

%81
最近方源还有个有些可怕的猜想。

%82
王庭福地毁灭,是因为自己。自己那时候是受到墨瑶假意影响,但他难道没有遭受天意的影响吗?会不会是天意,借助他这个棋子,将巨阳仙尊的众生运真传毁灭呢?

%83
对于天意而言,损有余补不足。

%84
宿命就是天意最好的手段。

%85
宿命蛊被红莲魔尊破坏后,运道的力量才能勃,诞生出巨阳仙尊。

%86
琅琊地灵直起上半身,低呼道:“方源长老,你若能担此重任,完成任务,不管是对落星犬杀还是擒,事后门派贡献都将高达八百点!”

%87
“八百点”方源双眼顿时微微一亮。

%88
然后,他沉思一番后,答道:“我拒绝。”

%89
ps:虽然很艰难,但还是更新了这一张!过年期间,我没有断更,我实现了自己的诺言!还有,感谢大家的支持!

%90
地一下云.来.阁即可获得观看】

\end{this_body}

