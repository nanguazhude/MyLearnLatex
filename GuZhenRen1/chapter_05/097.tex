\newsection{未来的八转战力}    %第九十七节:未来的八转战力

\begin{this_body}

%1
黑城的面皮极度扭曲,发出凄惨至极的嚎叫。

%2
好在他现在境界高深,血道、变化道都是宗师级。因此寻常蛊仙难以做到的事情,他连半个月都不到,就完成了。

%3
算一算时间,几乎一个月下去了。

%4
距离第四次地灾。还有一个月多一些的时间。

%5
万事俱备,方源抓紧时间,立即动手,对鸟蛋下手。

%6
果然如他所料,在血道杀招的作用下,原本生机全无的鸟蛋立即焕发出了一股生机。

%7
随着鲜血灌溉得越来越多,这股生机也越发旺盛,并且对方源越加亲切。

%8
足足八天八夜之后,蛋壳从内部破开,从中钻出一只小鸟来。

%9
这小鸟虽只是刚刚出生。但体型不小,驻足于地,如同正常少年的高度。

%10
小鹰望着变作方源模样的力道仙僵,满脸热情。犀利的鹰眸中竟是孺慕之情。就好像是婴孩见到了自己的亲生父母。

%11
“来,到我这里来。”方源操纵仙僵,试着开口。

%12
小鹰立即乖乖地蹦到仙僵面前,仰头望着他,鹰眸黑得极其纯粹,毫无防范意识。

%13
仙僵伸出手来。摸摸小鹰的脑袋。

%14
小鹰的脑袋上,只有一层嫩黄的小羽毛,很稀疏,方源可以摸到它的皮肉,温热温热的,手感很好。

%15
方源抚摸的过程中,小鹰乖乖地低下头,没有丝毫反抗。不仅如此,鹰眸还微微闭合起来,鸟喙开合,发出啾啾的可爱叫声,似乎是在像方源撒娇。

%16
方源已看出来不少东西,不禁心中暗赞:“黑凡手段真是了得,居然抹除了这头上极天鹰的记忆,只保留前世的宇道道痕积累。不过若是有前世记忆,这上极天鹰,堂堂太古荒兽,也的确不好控制。”

%17
“还有这个血道杀招,居然可以奴役太古荒兽。不,说是奴役,并不太准确。真正的效果,是让太古荒兽极其亲近自己,相当于最可靠的血脉至亲。”

%18
“如此一来,让它驮着我,找寻到黑凡洞天,应当不是什么难事了。”

%19
脑海中正萦绕着这个念头时,小鹰又发出几声鸣叫,叫声略显急促,同时身躯也在晃荡不定。

%20
“是饿了吗?”方源操纵仙僵,收回手掌,然后一手指着天晶鹰巢,“那么就去吃吧!”

%21
上极天鹰的食物,正是天晶!

%22
小鹰得到允许之后,欢快长鸣,然后蹦蹦跳跳地扑过去,轻轻一啄,就在天晶上啄出一小块来。

%23
小鹰用鹰嘴叼着小块天晶,却不下咽,而是转头看向方源样貌的力道仙僵。

%24
力道仙僵对小鹰笑了笑,温和地道:“快吃吧。”

%25
小鹰明白了,忽一仰头,将嘴里的小块天晶吞入腹中。

%26
然后,它继续啄食。

%27
坚不可摧的天晶,在它嫩嘟嘟的鹰嘴下,脆弱得仿佛是豆腐渣,叫方源站在一旁暗中观察,也不由的有些心惊。

%28
别看不起小鸟,这可是货真价实的太古荒兽上极天鹰!

%29
“也就是说,从今以后,我就有一头太古级的战力了吗?不,它还太年幼,战力不可能这么高。”

%30
“它似乎能听懂我的话,看来也是黑凡留下的手笔。既然能抹除上极天鹰的上一世记忆,添加一些人族语言,也无不可了。”

%31
“不过,还有一个巨大的弊端!”

%32
“我必须伪装血脉,才能让上极天鹰感到极度亲近。如果是真实的自我。那就完全不行了,甚至还会惹来上极天鹰的攻击。毕竟这种太古荒兽,可是性情高傲,很有侵略性。”

%33
方源思考着方方面面。

%34
接下来的几天。方源操纵的力道仙僵,就一直陪伴着小鹰,看着它进食。

%35
小鹰也不是一直在进食,吃一段时间,它就蹦到力道仙僵身边。偎依着他,沉眠休憩。

%36
它对力道仙僵十二万分的信任,毫无防范。

%37
连续几天下来,天晶鹰巢已被小鹰啃食了大半。

%38
这种进食的恐怖速度,让方源暗暗感到头疼。

%39
为了收获太古战力,付出一些太古仙材是难免的。但这喂养的代价,未免太过高昂了一些。

%40
小鹰长得很快。

%41
原本只是少年的高度,但几天后,已经和方源一般高矮了。

%42
它立足于地上,鹰首高昂。体格粗壮,羽毛丰满起来,鹰目更加锐利,鹰爪漆黑尖锐,一抓之下,天晶立即被撕烂,就像是一层薄纸。

%43
时机已经成熟,方源不愿再继续等待了。

%44
他撤回力道仙僵,收回仙僵身上的各种仙蛊凡蛊,并在真身上施展见面曾相识。

%45
如今的见面曾相识。已经超越了原版,新添了血本仙蛊,和其他种种凡蛊。

%46
如此一来,方源就连血脉都可伪装。

%47
方源离开琅琊福地。选择一处无人地界,打开仙窍门户,放出上极天鹰。

%48
小鹰见到方源真身,不疑有他,立即挨过来,用鹰喙轻轻地啄方源的手掌。还低下头,用额头的羽毛蹭蹭方源的肩膀,态度十分亲热。

%49
方源笑了一声,叮嘱小鹰:“快载着我去黑凡洞天,帮助我继承黑凡真传。”

%50
小鹰微微一愣,旋即记忆中的某个部分鲜活起来,然后它高高鸣叫一声,展翅飞上天空。

%51
方源见机,果断跳到小鹰的背上。

%52
小鹰再一次长鸣,载着方源一路飞升,直朝东南方向飞去。

%53
周围气流翻涌,耳畔狂风呼啸。

%54
方源催动蛊虫,坐在鹰背上岿然不动。

%55
他对上极天鹰的速度非常满意,已经不亚于剑遁仙蛊了。

%56
更关键的是,这头上极天鹰还是幼鹰,距离成年还很远,体格还未真正长成。真正成年的上极天鹰,体型十分庞大,比巨鲸还大数倍。

%57
方源也预估出来:它此时的战斗力,也没有达到八转级数。目前,顶多是六转程度。

%58
尽管它身上的宇道道痕不少,但没有野蛊依附。影响荒兽、荒植战力的重要因素,还有它们的体格。这点和人族不同,人族的体格不值得期待,荒兽本身的攻防、恢复能力,就极其强大。太古荒兽更不用说了。

%59
方源还是第一次接触上极天鹰,之前臆想一出生就是八转战力,显然是错误的。

%60
虽然有些失望,但未来能成长成八转战力,那是妥妥的!

%61
上极天鹰负责赶路,方源坐在鹰背上,每隔一段时间,就喂给它天晶吃。

%62
其他的时候,方源也没有闲着,他在分析黑凡这个人物。

%63
从黑城的记忆中,已经对黑凡极具推崇。

%64
这个历史上,将黑家带上鼎盛的宙道大能,的确是不简单!

%65
方源回顾自己破解黑凡真传的过程,简直就是一场他和黑凡的对决。

%66
这场对决不同寻常,可谓别开生面。

%67
黑凡留下的炼道杀招,天晶鹰巢,上极天鹰死蛋,隐藏着的血道杀招等等,都让方源深刻领教到了这位宙道大能谨慎的心思,深沉的谋略,高傲的秉性。

%68
能够修行到八转的蛊仙,都不是简单人物。

%69
黑凡更是八转中的佼佼者,黑凡真传不容易继承!

%70
黑家蛊仙努力了这么多代,都没有成功。当然,这其中也有态度蛊遗失的客观原因存在。

%71
但不可否认,若换做其他外人,基本上都会栽了。也就是方源这种老谋深算的东西,结合恰当的机缘巧合,方能步步勘探,层层递进。最终走上前往黑凡洞天的路。

%72
“到了黑凡洞天,是否就能继承真传?我总有种预感,似乎黑凡的考验,并未结束。”

%73
“不过考验越多,越证明真传不俗,并非是件坏事!”

%74
ps:这一章是月票2的加更。欢迎大家关注我的微・信公众号,直接搜蛊真人就可以了。

\end{this_body}

