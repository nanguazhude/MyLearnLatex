\newsection{打一炮换个地方}    %第七十节:打一炮换个地方

\begin{this_body}

方源查看着自家的仙窍。<strong>热门小说网WWW.QiuShu.Cc</strong>

神念在仙窍中广阔的空间里穿梭,到处都是大片大片的荒地。

“土道道痕还是少了。这些荒地,地力缺乏,大部分的植株都无法栽种。要从根本上提升,就得增添土道道痕。除此之外,其余方法都是治标不治本。”方源在心中念道。

至尊仙窍的空间实在太大,远超常理。

即便是方源的神念,彻底覆盖下去,也要耗费相当长的一段时间。

不过,方源查看仙窍时,有重点,有目标,不是无的放矢,这就大大提高了效率。

数股神念,向各个方面迅速传播开去。

渐渐的,一个个方源专心经营的资源点的景象,都展现在他的脑海当中。

西漠,土地多呈现沙化。

一个巨大的地坑,气温很高,坑地沙土焦黑。

在坑中,数目庞大的幽火龙蟒,在相互纠缠,繁衍生息。

这是最原始的地坑。

原本幽火龙蟒,是以家庭为单位,三三两两生活的。但是方源在这里,采用的却是北原东方一族特有的豢养方法,将许多幽火龙蟒放在一起,促使它们繁衍出更多的幼蛇。

这批幽火龙蟒,本身就是来源于东方一族的大本营福地。

而豢养之法,则是方源从东方长凡的魂魄中,搜刮出来的,百分百原版不说,连各种细节,甚至种种猜想都一清二楚。

两者结合一起,导致这些幽火龙蟒繁衍得十分旺盛。

距离这个地坑数里之外,还分布着三个地坑。

每个坑中的龙蟒,都有不下于原始第一坑中的数量!

“五域时间一个月不到,但是在这里时间加速六十倍。幽火龙蟒的数量翻了整整四倍。”方源暗自点头。

六十倍,这个时间的流速比,实在是恐怖!

幽火龙蟒放在狐仙福地中。有过一段相当长的时间,但起色不大。孕育成果和方源眼前的根本没法比。

这是因为狐仙福地坐落在五域外界,时间流速比大大降低,导致资源的生长成果不佳。

幽火龙蟒群如此,其余的龙鱼群、长恨蛛群、箭竹林、星屑草等等,亦是如此,规模扩大,各有多少。

不过论长势的话,箭竹林、星屑草等等。就不如之前在星象福地中那般生机勃勃了。

这是因为至尊仙窍中的星道道痕,远不如星象福地中那样多。

蛊仙经营仙窍,要时常查看仙窍的方方面面,以防疏漏。<strong>热门小说网WWW.QiuShu.Cc</strong>

尤其是仙窍空间较小,各种资源挨得很近,容易相互影响,出现差错。

方源的情况还是不错的。

因为至尊仙窍空间很广,各个资源都相当于各有各生存环境,远远的隔离开来。出现问题的概率很小。

“等到宝黄天开启,我将这些资源贩卖出去。收获将十分喜人。”方源查看一圈,十分满意。

宝黄天关闭,对他也影响很大。

他向对仙窍多加经营、建设。但缺少宝黄天,很多东西都买不到。琅琊派的库藏虽然丰富,但多是炼蛊的仙材。对于方源,帮助不是很大。

消散多余的神念,只剩下最主要的一股,奔赴到小北原。

此番,小北原才是方源的重点。

吼、吼、吼……

小北原上,大量的雪怪不断游荡。

尤其是越往北,天气越是寒冷。甚至还飘着零碎的小雪。地面上,也都是一层薄薄的冰雪。

方源仔细观察。冰雪覆盖的范围,又比他上一次侦查时。扩张了许多。

这些都是上一次地灾的得益。

第一次地灾,方源在冰原渡劫。灾劫极其凶猛,这也导致了方源收获很大,获得了大量的冰雪道的道痕。

正是因为这些道痕,才有了小北原中小雪飘飘的景象。

这样的环境,无疑更适合雪怪们生存了。

在如今的小北原,普通的雪怪不计其数,大量的荒兽雪怪,还有少数的上古雪怪称王做霸。

在和影宗交易之前,方源还打算留着这些雪怪,当做一种资源,在仙窍中经营下去。

但交易之后,方源明白了天意,便决定必将这些雪怪铲除殆尽!

皆因这些雪怪的体内,都充斥着满满的天意。

方源带着杀机而来,此刻神念调动,一头力道仙僵携带大量仙元飞起,身边一大群蛊虫环绕飞舞。

剑道仙招剑浪三叠!

蛊虫被一一催发,仙元灌溉,须臾之后,爆发出澎湃的浪潮。

哗啦啦……

银白色的剑浪,翻腾推涌过去,浪花朵朵,散发出比冰雪还要寒冷的杀机。

普通的雪怪们没有任何反抗,就被汹涌的剑浪直接吞没,然后绞成碎片。碎片再绞碎成渣,然后渣滓又被冲刷,干干净净,没有任何存在。

而那些荒兽雪怪,则有抵挡之力。

不过也只能抵挡几个呼吸,就被剑浪刷掉。

剑道,攻势极强。是当今蛊仙界公认的,五大强攻流派之一。

剑浪三叠更是其中的优秀杀招,发出的剑浪一波比一波还要强势。方源当初渡劫,就是依仗此招,杀了冰雪所化的墟蝠,险死还生。

三波剑浪过去,斩杀了数千头普通雪怪,六头荒兽雪怪,战绩赫赫。

数千头雪怪,几乎没有让剑浪有多少的损耗。关键是那六头荒兽雪怪,每一头都折损不少剑浪,最终让三波剑浪消散一空。

吼!

随着一声剧烈的吼叫,高达七丈的雪怪从远处狂奔而来,夹裹无边的怒火。

随着它一同的,还有数头荒兽雪怪,上万头普通雪怪。

一个庞大的雪怪兽群,浩浩荡荡。如大军席卷战场似的,朝方源这边杀来。

方源调动力道仙僵,转身就跑!

他现在真身在外。只能遥控这头力道仙僵,携带仙元和蛊虫作战。

这头力道仙僵。本是方源从焚天魔女那里讨教而得,仙窍早有消散。

仙元和各种蛊虫,都只能寄托在仙僵身上,作战时候,很不安全。

所以,方源铲除这些雪怪,选择遥遥打击的方式。

还有十分猥琐的游击战术。

打一炮,换一个地方!

再打一炮。再换一个地方。

雪怪群落声势浩大,但是追到凶案现场时,方源早已经逃之夭夭,不见踪影。

雪怪们的愤怒无法发泄,纷纷仰头咆哮,声震四野。上古雪怪怒气冲冲,乱扔雪球,硕大的雪球像是一个个的流星,一时间四下横飞。

人是万物之灵,智慧上的差距。导致方源牢牢占据主动,雪怪们只能被动挨打。

脱离到安全的位置,方源遥控的力道仙僵停下脚步。进行短暂的休整。

这具身躯虽然没有受伤,但方源并非没有付出代价。

代价就是青提仙元。

催动剑浪三叠,耗费仙元不少!

还有琅琊派门派贡献。

因为组成剑浪三叠的仙蛊,不止是核心仙蛊浪剑,还有其余辅助的水道仙蛊。而这些水道仙蛊,方源都是向琅琊地灵借的,自然要耗费不菲的门派贡献。

总的而言,催发一次剑浪三叠,成本还是不小的。

不过现在方源家大业大。手头宽裕,仙元耗费可以承受。门派贡献方面。因为之前的太丘之行,得到大量门派贡献。短期来看。也不成问题。

休整片刻后,雪怪群已经重新安定下来,方源继续展开杀戮行动。

“必须在第二次地灾来临之前,将这些雪怪全部杀掉,否则的话,让天劫利用起来,里应外合,我恐怕就要栽在第二次地灾上了。”方源心头雪亮。

蛊仙的灾劫,都是一次比一次强。

方源第二次地灾,自然要比第一次更强些。

若是第一次地灾遗留的雪怪还有,和第二次地灾相互汇聚,那方源渡劫的希望本就渺小,这样一来,几乎就是绝境!

这不是方源第一次行动了。

回到琅琊福地,已经十多天过去。几乎每天,方源都要对雪怪群进行剿杀。

雪怪在大量的折损,规模日趋缩小,但剩下的还有很多,方源还需要更努力一些。

但一天之内,袭击的次数多了,动静就会越闹越大。

雪怪们虽然智慧不高,但也有生存本能,兽吼除了发泄情绪之外,也是相互之间的通知和警告。

方源袭击了数次,这些雪怪们已经全都警觉起来,依附在上古雪怪的庇护下,抱成团据守。

方源不容易下手,便将力道仙僵撤离出来,暂停了今天的雪怪剿杀。

不再关注仙窍,方源站起身来。

飞出云城,很快,他就来到落魄谷。

他开始每天一次的魂魄修行。

“方源大人。”

“方源长老又来了?你修行起来真是刻苦啊,叫人赞佩。”

落魄谷中,还有其他蛊仙。这些毛民蛊仙已经和方源很熟了,基本上都是被方源指点过战斗技巧的。

方源入谷,他们都主动打招呼。

方源也热情地回礼过去。

得益于内奸毛六,方源现在和这些毛民蛊仙相处得相当融洽。

选择一地,检查之后,方源便飞出魂魄,展开修行。

深入魂魄的痛楚,让他的魂魄颤抖不已。

方源咬牙坚持,毫不动摇。

好一会儿,关注他的几位毛民蛊仙才收回各自的目光,暗自交流。

“真是一个修行狂人!”

“难以想象,他每天都要在落魄谷待上一个时辰!”

“这种痛楚……我连半刻钟都挨不下来啊……”(未完待续。)<!--80txt.com-ouoou-->

\end{this_body}

