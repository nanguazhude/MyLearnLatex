\newsection{风花与雪月}    %第七十五节:风花与雪月

\begin{this_body}

方源一身白袍,黑发翻腾,降落在至尊仙窍的小北原之中。<strong>最新章节全文阅读www.QiuSHU.cc</strong>

这一次渡劫,他仍旧打算在这里进行。

其他几个地方,都有一些资源分布。小北原这里只有冰雪,最关键的雪怪、天意都被铲除得一干二净了。

至尊仙窍空间宽广,但灾劫的唯一目标就是方源。

所以他在哪里,灾劫就发生在哪里。

仙窍已经落下,种在北原之中。仙窍中的光阴支流和光阴长河合流,时间流速大大降低。

首先是检查布置。

至尊仙窍中已经布置下了大量的蛊虫。

其中大部分,是仙灾锻窍杀招的相关蛊虫。

除此之外,还有一座荡魂山,坐落在方源的身边。

此时的荡魂山,已经面目全非。这是方源为了防止他人,利用定仙游,在他渡劫的时候,忽然传送进来。所以提前消除了这个隐患。

第一次渡劫,方源没有拔山仙蛊,也没有江山如故,所以不能将荡魂山带进来。

现在却不一样。

有了拔山,方源可以自己独自一人,就可将荡魂山塞进自家仙窍。否则若用其他毛民蛊仙出力,不仅要耗费一笔门派贡献,而且还会让毛民蛊仙进入至尊仙窍,暴露出方源的这个大秘密。

而有了江山如故,也不怕荡魂山在灾劫中损毁。荡魂山损毁的程度越严重。胆识蛊的产量就越低。

检查一番后,方源确信一切布置都没有差错。旋即,他心念一动,至尊仙窍门户顿开。

呼呼呼……

大量的天地之气,通过门户。涌进至尊仙窍当中。

一时间,声势壮观,仿佛是澎湃的潮水巨浪,倾泻进来。

“这一次,我带来荡魂山,给至尊仙窍增添了负担。所以天地之气,较上一次还要多得多。”

天地之气蜂拥进来。方源在此同时。开始在仙窍外布置各种蛊虫。

上一次地灾时就布置过了,这一次方源更加熟练。

方源渡劫,自然要用到这招仙劫锻窍。

一来,仙劫锻窍杀招可以变相地削弱地灾的威力。

二来,它能抽取出狂蛮真意,帮助方源修行变化道。

和第一次渡劫不同的是,这一次。组成杀招仙劫锻窍的许多凡蛊都是方源自己的。并且对于仙劫锻窍这个杀招本身,他还有一些微小的改动。

改良仙劫锻窍,也是方源为渡劫所做的准备之一!

虽然他炼道境界不高,但智道却有宗师程度。[看本书最新章节请到目前改良的部分,只是锦上添花,可有可无。但只要继续深究下去,方源相信,总有一天他会将这个杀招改造成最适合自己的版本。

毕竟,方源不可能一直依靠琅琊派。每一次渡劫,都要向琅琊地灵借来这么多的蛊虫。着实代价不菲。方源改良仙劫锻窍,也是想要摆脱琅琊地灵对他的这个控制。

片刻之后,方源布置妥当,开始调集仙元,一一催动蛊虫。

青光泛滥而起,弥漫仙窍,同时勾连外界。渐渐的。仙窍内外的青辉融汇成一片巨大的光影,覆盖方圆上千里的范围。

成千上万的青提仙元消耗,这才让方源成功催使出仙劫锻窍杀招。

第二次地灾来临了。

天地板荡,风起云涌。

起先只是微风拂面,很快,风势就蹭蹭蹭地往上狂涨!

眨眼之间,就形成了漫天狂风,大有卷席天下,吹飞万物生灵的凶猛气势。

“这是什么劫?”方源站在荡魂山巅,严阵以待。

虽然他还没有认出这次地灾的跟脚,但就前奏来看,这一次地灾和上一次一样,威力爆棚,绝对是超出常理的。

狂风大作,卷起地面上的冰雪,吹得一片天地茫茫。

至尊仙窍中本身光道道痕就较少,只有微光泛滥。此刻风雪弥漫,遮天蔽地,方源的视野一下子就局限起来。

叮铃铃……叮铃铃……

“什么声音?”方源早就催动各种手段,时刻侦查,此时忽然听到狂风声中竟隐约传出风铃般的清脆响声。

这番古怪的情景,让方源眉头轻皱。

“难道说,这是……”他脑海中忽然灵光一闪。

恰在此时,风铃声音大作,一朵巨大的青花从风中旋转飞出,直接撞在荡魂山上。

这朵青花的速度,是如此之快。

简直是迅雷不及掩耳之势,就撞在荡魂山,立即将荡魂山撞出一道长长的伤痕。沿途的山石都被绞碎成粉末,旋即就被狂风吹走。

叮铃铃!叮铃铃!!

风铃声一时大作,下一刻,十八朵青花飞旋而出,接连撞在荡魂山上,粉碎自己,撕裂出十八道崭新的深痕。

“这是风花!”方源双眼暴射精芒,心底确认了答案。

风花,是一种风中精粹。只有极其猛烈的狂风中,才会孕育而生的特殊植株。

它无根无须,在风中旋转飞舞。大如车马,色呈淡青,极其锋锐。

叮铃铃铃!

方源才刚刚认出风花,下一波的风花,就再次强袭而来。

这一次风花数量更多,有四五十朵。并且袭击的方向,也是来自四面八法,接二连三地撞毁在荡魂山上。

荡魂山巨石破碎,烟尘滚荡,碎小的石块又很快被周围包裹着的狂风,吸收吞没。

方源连忙撤退,从山顶缩下去。

他暗暗咬牙,感到棘手。

“上一次地灾,天意催生出大量雪怪,但我可以飞行,牢牢占据主动。”

“这一次,天意显然吸纳了教训,改变战术,用风花劫来对付我。”

这恰恰是方源的软肋!

方源的速度比不上风花,防护手段又只有凡级程度。

“幸好我这次搬来了荡魂山,否则的话,无险可守,恐怕要糟糕了!”方源顿时觉得十分庆幸。

江山如故!

他催动宙道仙蛊,荡魂山上的累累伤痕顿时消失,又转成原来模样。

但紧接着,几乎是下一秒,绵绵不绝的风花再次打来,将一层层的山岩撞碎,削成渣滓。

“这样下去不妙!”

“刚刚催动仙劫锻窍,仙元就消耗了大半。”

“现在虽然暂时安全,但耗费的却是大量的青提仙元。用这些仙元治疗荡魂山,苟延残喘而已。”

方源双眼一眯,脑中思绪急转。

他战斗经验极其丰富,很快就想到了应对之法。

力道杀招万我!

变化道杀招见面曾相识!

仙蛊暗渡!

他浑身一抖,手段爆发,顿时无数的方源,占据整个荡魂山。

其中一位仰天呼啸,所有的方源虚影,就都向山外的狂风内疾奔而去。

与此同时,方源本体催动见面曾相识,借助万我的遮掩,变化成一块荡魂山上的石头。

有了变形仙蛊后,见面曾相识回归原版,变得简单易行,瞬间变化。不像之前的改良版本,方源用大量的凡蛊取代变化仙蛊的作用,导致见面曾相识杀招发动十分缓慢,不能救急。

变化成一块不起眼的山石之后,方源还不停歇,再用暗渡仙蛊,遮掩住了自身气息。

叮铃铃……

大量的风花,转头攻击成千上万的方源虚影。

灾劫火力骤减,荡魂山承受的压力大为降低。

片刻功夫,方源的万我虚影就都被风花铲除殆尽。

方源便再次催动万我杀招,变出无数虚影,携带自身气息,再次狂奔而出。

风花散漫,在狂风中旋转飞舞,一个又一个的虚影被瞬间斩杀。

两三次之后,天意洞察到了方源的战术,所有的风花再不管什么方源虚影,直接集中火力,强攻荡魂山。

方源咬牙切齿,感受到天意的险恶用心。

“荡魂山是我最大的经济支柱。天意想要彻底毁掉它,坏了我的前程!”

方源幸好有江山如故在手。

只是利用江山如故治愈荡魂山,只是让方源维持僵局,不能改变他此刻极端被动的局面。

青提仙元在剧烈的消耗。

剑浪三叠!

力道大手印!

方源尝试反击,但风花的速度太快,两大杀招效果欠佳得很。尤其是力道大手印,虽然气势磅礴,但飞行的速度比较慢,飞入狂风之中,根本捞取不到什么风花。除非风花亲自来斩它。

毒气吐纳!

方源张开一喷,紫黑色的毒气立即顺着狂风,四处弥散。

大量的风花,只要沾染到毒气,都迅速萎靡。

方源脸上顿生惊喜之色。

毒气吐纳这个杀招,本是他从智道杀招包藏祸心上改良出来的。以妇人心为主核,弊端很多,比较鸡肋。

方源很少动用它。

没想到此刻动用,却是收到了奇效!

在毒气吐纳的作用下,风花的猛烈攻势顿时下降到了谷底。

狂风也减缓下来。

一蓬冷光照射整个荡魂山。

方源仰起头,竟看到天空中不知何时,悬挂着一轮弯月。

方源双眼骤然眯起,脱口低呼出来:“雪月!”

这轮弯月,洁白如雪,照射下的苍白光辉,让荡魂山温度骤降。

很快,一层层冷霜覆盖住整个荡魂山的表面。

方源的心像是陡然放入了一块千斤巨石,暗道:“原来这不是风花劫,而是风花雪月劫!前者是地灾,后者是天劫。这第二次地灾未免增强也得太离谱了!”(未完待续。)<!--80txt.com-ouoou-->

\end{this_body}

