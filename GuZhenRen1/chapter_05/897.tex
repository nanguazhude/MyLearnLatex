\newsection{终极任务}    %第九百零一节:终极任务

\begin{this_body}

%1
龙鲸乐土。

%2
功德碑下,方源和沈从声并肩而立,凝神望着碑面。

%3
有关清缴黑火的超级任务仍在,但明日便是三百天的最终时限了。

%4
方源的功德傲居榜首,这是一笔庞大的数字。他发现黑火任务难度太大后,便只选择大型任务以及中型任务,一笔笔收获功德。

%5
庙明神一伙人一直为他打下手,也是收获颇丰。

%6
沈从声等人在意识到超级任务的难度后,也明智地转向,学习方源,如今功德也积累到了可观的程度。

%7
“三百天匆匆而过,似乎昨天我们刚来到这里。”沈从声苦笑。

%8
这是货真价实的宝地,通过完成种种任务,蛊仙们就能直接获得仙蛊、仙蛊方、种种杀招。饶是沈从声这样的八转蛊仙,都有些流连忘返,不想离开。

%9
但没有办法,时限一到,他们就会被龙鲸乐土踢出去。

%10
“可惜这一次,我们还未探索到龙鲸乐土的全部秘密。”方源遗憾叹息。

%11
龙鲸乐土的蹊跷之处还有很多。

%12
黑火是其中关键一点。

%13
这个黑火从哪里来?为何有这样的种种变化?为何看不出它隶属何种流派?又为什么人道手段对它非常有效?

%14
除了黑火之外,乐土仙尊留下的天穹脊背,源自何处?乐土仙尊的真正用意是什么?

%15
乐土仙尊当年和沈伤究竟达成了什么约定?沈伤置身在海底大阵中修出人道,却又莫名其妙地沾染上了麻烦的黑火。

%16
还有,除了沈伤一人外,方源不见其他的蛊仙。龙鲸乐土中本就没有蛊仙吗?还有功德碑中的仙蛊,又来自哪里?若都是自然产出,那数量未免也太多了。

%17
沈从声沉默片刻后,道:“这些秘密只好留待以后,我们再来探寻了。好在我们已经都兑换到了名号,一旦将来苍蓝龙鲸出世,我们就可以感应到它,再来探索。”

%18
方源点点头,这的确是一个可喜的发现。

%19
正如他之前推断的那样,庙明神之所以能够发现苍蓝龙鲸的具体位置,也是因为这个名号的缘故。

%20
也不知道是谁将这个名号,悄悄地转到他的身上。

%21
在功德碑提供的所有名号中,有极少数的几种名号可以转让出去。

%22
还有一个发现,也是有关名号。

%23
兑换了这个名号之后,蛊仙离开龙鲸乐土时,可以传送到方圆十万里的任何位置。

%24
这是方源曾经哄骗庙明神一伙人的借口,但事实上它真的存在。

%25
就在这时,功德碑面上的文字忽然浮动变化起来。

%26
不管是小型、中型、大型乃至超级任务都消失不见,只剩下一项终极任务。

%27
终极任务——跟随苍蓝龙鲸出走,并成功往返。

%28
“这是?”沈从声惊愕。

%29
方源眉头微蹙:“难道说,超级任务之上还有终极任务?这个内容描述十分模糊,但恐怕难度远比清缴黑火,还要高得多!”

%30
这时,功德碑面的文字又有新的变化。

%31
沈从声不由瞪大双眼,口中复述:“任务时间不定,接受之后,将不受时限长留乐土。完成终极任务后,方可获得功德碑主的资格,并且保留个人功德。”

%32
方源眼中精芒爆闪,暗道:“原来乐土仙尊已经有所布置,为功德碑择主。只要按照他的布置来,我就能成为这座八转仙蛊屋的主人了。可惜、可惜。”

%33
方源心中遗憾。

%34
他已经挖掘到了一个收获功德碑的正确方法,但是他的实力有限。超级任务他都完成不了,更何况终极任务?

%35
终极任务显然是难度极高的,就算是八转蛊仙也希望不大。

%36
这点很容易推算出来。

%37
来到龙鲸乐土中的蛊仙,远不止方源这一批。前人们来到这里,或许有人选择接受了这项终极任务,但功德碑仍旧是无主状态。

%38
这点说明从未有人成功过!

%39
方源不打算接受任务的原因还有一个。

%40
那就是功德碑面上的文字所叙述的——任务时间不定。

%41
若是时间太长,等到他完成任务出去,却发现天庭已经彻底修复好了宿命蛊,那方源还玩什么?

%42
“唉……”沈从声长长地吐出一口浊气,他也恢复了理智,认清了当中的风险,眼神不再灼热。

%43
方源伸手指着功德碑面上的任务,分析道:“沈兄,这个终极任务还有言为之意。选择了这项任务后,蛊仙的个人功德将获得保留。这么说来,没有接受的人,恐怕只要一离开龙鲸乐土,所剩下的功德都会清除干净吧。”

%44
沈从声心头一紧,连连点头:“恐怕就是这样了。我们当初来到这里,功德榜上可是空无一人啊。”

%45
如此一来,蛊仙们就得尽量将积累下来的功德花出去了。

%46
但具体怎么消耗,还得需要蛊仙自己好生考量一番。

%47
次日。

%48
东海某处角落,三团白光一闪,现出方源、沈从声、沈伤三仙。

%49
所有的蛊仙都放弃了终极任务,选择离开。

%50
不过庙明神一伙人并没有和方源一同离开,他们对方源还有顾虑。

%51
任修平、童画则是和沈家其他的蛊仙们一道撤走。这两个东海散仙经此一事,算是彻底上了沈家这条战船了。

%52
方源转身回望,只见苍天空空,大海平平,身后空无一物。

%53
他并不想就此放弃,便和沈从声、沈伤一起,不断侦查周围,结果一无所获。

%54
“我们三人故意选择苍蓝龙鲸这个位置,但苍蓝龙鲸何在?”

%55
“别找了,这是乐土仙尊的手段。别看眼前景象栩栩如生,或许仍旧是一层幻境呢?”

%56
“说起来,我们至始至终都没见到苍蓝龙鲸的真面目啊。”

%57
三仙感叹不已。

%58
苍蓝龙鲸的秘密真的不少。

%59
苍蓝龙鲸是被乐土仙尊点化,拥有仙窍,可以修行的太古传奇,应有不输给常人的智力。但蛊仙们探索的整个过程中,从未见过苍蓝龙鲸主动和他们联络过。

%60
苍蓝龙鲸的行迹也非常神秘。它什么时候出现,又如何消失,在这个过程中它又去了哪里?世人并不清楚。

%61
现场的这三位蛊仙,哪怕是去往龙鲸乐土走过一遭,也没有搞不清楚。

%62
方源心中叹息,原本他还想将苍蓝龙鲸、功德碑收入囊中的。

%63
但事实证明,他真的想多了。

%64
这是乐土仙尊设置的大手笔,有着某种神秘的目的。就好像是北原的疯魔窟,是由无极魔尊亲自铺设,也是大手笔,方源只能止步于第八层,进不去最核心的第九层。

%65
“等等,疯魔窟……疯魔?”方源忽然心头一动。

%66
无极魔尊布置出的疯魔窟,魔音灌耳,会令生灵疯狂。乐土仙尊的龙鲸乐土中,黑火也会令生灵疯狂。

%67
这当中是否有什么联系呢?

%68
“回想起来,其实龙鲸乐土中发疯的生灵不在少数呢。比如我之前杀掉的虎王鲵……”方源思索着。

%69
这些生灵发疯,无药可救,方源也查探不出什么原因,只有铲除掉。

%70
除了不冒黑火之外,他们和沈伤的问题几乎是一样的。

%71
从这一点也可看出沈伤的强大!

%72
这个人物就算身中黑火,也只是发疯一段时间,凭借自身雄厚的人道底蕴和道痕,能硬撑下去,直至恢复正常的神智。

%73
方源都没有自信做到这一点。

%74
他自问,自己身上的道痕虽多,但却不是人道积累。

%75
“就目前来看,沈伤恐怕是当今五域的人道第一人!就算是天庭中的正元老人,也不及他。”

%76
想到这里,方源不免有些期待,宿命大战已是不远了。或许沈伤的出战,能够带给天庭一些麻烦。

%77
三仙探查不到苍蓝龙鲸,相互辞别。

%78
龙鲸乐土一行,他们都有重大收获。但收获最多的无疑是方源。

%79
方源独自一人,又变作气海老祖模样,隐形匿迹回到气海。

%80
气海深处的大阵还在运转,在阵中,他见到了镇守在这里的吴帅。

%81
吴帅向前去往南疆,是想和池家交易梦境。池曲由见机涨价,吴帅念及大局,不想和池曲由一般见识,打算秋后算账,想要继续交易。

%82
但不想梦境的损失,竟被武庸查探到。他大发雷霆,专门找池曲由独自商谈了一次。

%83
池曲由当然撇个干净,就算是事发,他也打死不认。

%84
但遭受警告之后,池曲由只好收敛,暂时断绝了梦境交易。

%85
吴帅无法,只好又回归东海。

%86
方源和吴帅见面,从他那里得知了当今局势。

%87
局势有些糟糕。

%88
南疆这边,武庸的影响力越来越高。因为梦境的神秘消失,池曲由不可避免地遭受怀疑,武庸借机大占上风,施展政治手腕更加得心应手。

%89
而在西漠,曾经被方源寄予厚望的房睇长分身,却是陷入了泥沼般的困境当中。

%90
在他联合数个超级家族击退天庭之后,西漠出现了一位血道大能,四处为乱,他似乎特别仇恨超级势力,房家便是受害者之一。

%91
除了这个之外,房家的冰火沙漠中还出现了一头太古荒兽作乱。

%92
这头太古荒兽分外狡猾,房家多次围剿都以失败告终。

%93
至于华文洞天方面,李小白倒是顺风顺水。

%94
兽灾洞天中,战部渡的麻烦更大,已有性命之危。

\end{this_body}

