\newsection{光阴长河第二战!}    %第七百九十六节:光阴长河第二战!

\begin{this_body}

%1
万年斗飞车中,黑楼兰等人看得心驰神摇。

%2
场面极其壮阔,年兽大军浩浩荡荡,从四面八方围杀而来。

%3
“好厉害的杀招!”

%4
“难怪这一招一直都没有在至尊仙窍中催动。”

%5
“吸引过来的年兽太多了,若是在至尊仙窍里,必定会引发浩劫,生灵涂炭。”

%6
天庭一方也是震惊不已。

%7
“万年斗飞车……”顾六如咀嚼着这个名字,面色凝重。

%8
厉煌一脸肃容,声音沉重:“这已经形成兽潮了,威力竟如此巨大!”

%9
即便是他这样的传奇人物,此刻见到这样的磅礴景象,也不由地感叹万分。

%10
方源微微而笑。

%11
万年围猎杀招,是他从太古年兽钓来阵中汲取的思路,并将其发挥到了极致。以似水流年仙蛊为核心,经过光阴长河的增幅,达到超绝的威能效力。

%12
万兽嘶吼,兽潮卷席而来。

%13
轰轰轰!

%14
天庭出手,杀招接连不断,砸在年兽的浪潮海啸之中。

%15
一大波一大波的年兽被炸翻,天庭的两座宙道仙蛊屋皆有七转层次,能抗衡八转战力,对付绝大多数的年兽都没有问题。

%16
但问题是——光是杀戮这些年兽,有什么用?只要万年斗飞车还在,那么年兽就源源不断。

%17
这是奴道的战术,就是要来消耗天庭一方的战力。

%18
天庭轰炸过去的杀招,都是要消耗仙元的。

%19
“这这些年兽纠缠,就是落入方源的算计中了。我们当擒贼先擒王!”三秋黄鹤台中,一位七转蛊仙建议道。

%20
话音刚落,他旁边的一位蛊仙就道:“但是,那艘万年斗飞车的坚硬程度,我们也已经知晓了。如何实施斩首战术?”

%21
七转蛊仙顿时傻眼:“这……”

%22
然后,下一刻,他就听到周围人在叫喊:“小心!方源又杀回来了!!”

%23
万年斗飞车速度飙上来,竟然不退反进,直冲鲨流撬而去。

%24
“好胆!”鲨流撬上厉煌大怒,方源居然还敢回来,明显是没有把他们看在眼里。

%25
当即,天庭两座仙蛊屋以及宙道八转大能顾六如,围绕着万年斗飞车,展开厮杀。

%26
万年斗飞车硬抗一阵阵的轰击,同时周围闪耀无数银色飞剑,嗖嗖嗖地飞射出去,宛若疾风暴雨。

%27
双方你来我往,打得水深火热。

%28
“不妥!方源这是要纠缠住我们,好让这些年兽大军的包围合拢!”顾六如看出了方源的打算。

%29
厉煌冷笑:“只要我们一心想要突围,这些年兽再多,又有何用?”

%30
顾六如叹息一声:“别忘了我们身后还有镇河锁莲大阵。”

%31
厉煌一愣,他急切之间没有想到这一茬,被这么一提醒,顿时心头一跳,肚中大骂:“方源这魔崽子好生阴险歹毒!”

%32
镇河锁莲大阵乃是天庭辛辛苦苦搭建而成,能够在光阴长河的河面上缓缓飘流,里面驻扎了大量的蛊仙,操纵着各自的阵眼。

%33
若是让这座大阵被攻破,天庭损失的不只是大量的仙蛊,还有这些七转蛊仙。

%34
并且在这短时间内,恐怕再不能搭建出第二座镇河锁莲大阵了!

%35
天庭的处境很是尴尬。

%36
他们原本想利用镇河锁莲大阵,埋伏方源,结果被方源看破,反而这座大阵成了天庭一方的破绽,被方源抓住不放。

%37
镇河锁莲大阵虽然能够漂流,但是速度很缓慢,根本别想脱离眼下的战局。

%38
“先将大阵中的蛊仙按序撤走。”顾六如思索了一下,道。

%39
厉煌却在同时开口:“杀了方源,斩除首恶,就能摆平这一切!”

%40
两位蛊仙立即对视一眼,在这关键时刻,他们出现了分歧。

%41
吼吼吼!

%42
因为方源的纠缠,年兽的先锋浪潮达到了这里。

%43
首当其冲的,便是三头太古年兽。

%44
这些太古年兽皮糙肉厚,顶住了天庭的轰击,一头撞进仙蛊屋的厮杀场中。

%45
“找死!”厉煌大怒,操纵鲨流撬前的巨大猛鲨,立即撕咬上去。

%46
与此同时,他又催动鲨流撬中的杀招,在十几个呼吸之后,将两头太古年兽都接连咬死。

%47
“跳梁小丑!”厉煌杀意稍泄,就听到顾六如的求助声——“快来助我!”

%48
原来,厉煌去对付太古年兽,顾六如和三秋黄鹤台则围攻万年斗飞车,暂时平分秋色。

%49
随后方源催动破晓剑杀招,专对准镇河锁莲大阵狂轰滥炸。

%50
这座大阵内强外弱,本来是对付石莲岛的,被方源这番辣手轰袭,立即不支,多处崩溃。

%51
不少阵眼中的蛊仙,都命丧当场。

%52
万年斗飞车又坚厚硬实至极,顾六如和三秋黄鹤台无法攻敌必救,只好来遮挡方源的攻势,掩护大阵中的蛊仙撤退。

%53
如此一来,方源就占据了主动,纵情出手,指哪打哪,把顾六如和三秋黄鹤台杀得狼狈不堪,左支右绌。

%54
厉煌见情况不妙,连忙绕过去,从背后夹攻万年斗飞车。

%55
方源怡然不惧!

%56
若是只有他一个人,或许还兼顾不全面,但车里的众多蛊仙在打下手,经验老道,辅助极佳。为了再创斩杀八转的战绩,他们个个通红了眼,兴奋异常。

%57
方源始终站在船首,任凭外面风浪险恶,忽然他捉到顾六如的一处破绽,催出一记杀招。

%58
正是春剪!

%59
春剪飞射而去,并不是对付顾六如,而是躲过了他的拦截,咔嚓一声,把一位中洲女仙剪短了脖颈。

%60
顾六如心头一惊:“这不是夏槎的招牌手段吗?方源已经掌握了,并且改良成了七转程度!”

%61
方源哈哈一笑,又催动夏扇杀招,同样是七转层次。

%62
厉煌兼顾不周,只能眼睁睁看着狂风呼啸,把镇河锁莲大阵吹出更多的破漏大洞,数只仙蛊瞬间毁灭。

%63
“哈哈哈,我想要杀人,你们拦得住吗?”方源朗笑一声,大肆嘲讽。

%64
厉煌睚眦欲裂,骂道:“小贼,我必定要把你挫骨扬灰!”

%65
砰!

%66
一头太古年兽扑到万年斗飞车上,把这座八转仙蛊屋撞得倾斜过去。

%67
天庭一方顿时愣住。

%68
几乎瞬间,两位八转大能纷纷反应过来。

%69
“原来,方源虽是能掀动年兽兽潮,但是却不能掌控它们!”

%70
“甚至,这些年兽的目标就是万年斗飞车!!”

%71
发现这个秘密,厉煌、顾六如气得差点要吐血。

%72
敢情战斗半天,之前他们对付这些年兽,是为了方源挡刀!

%73
难怪方源没有作壁上观,而是直接冲杀上来,纠缠他们。

%74
方源也没有办法。

%75
他已经做到了眼下的极限。

%76
利用似水流年蛊勾引来年兽,引发年兽狂潮,这是宙道的力量。若是能操纵这些年兽大军,那就是奴道的效用了。

%77
周围年兽越来越多,战场变得非常拥挤。

%78
之前,天庭一方还对这些年兽浪潮狂轰滥炸,稍稍阻止过它们的脚步。但因为方源的纠缠,不断轰击镇河锁莲大阵,又袭杀那些七转蛊仙,导致天庭一方无暇他顾。

%79
“方源,你给我死开!”

%80
“给我滚!!”

%81
厉煌大吼连连,但方源就是驱动着万年斗飞车,紧紧贴着他的鲨流撬,或者是三秋黄鹤台。

%82
年兽的目标是万年斗飞车,结果身边的鲨流撬和三秋黄鹤台都遭殃了。

%83
“你们不是要除我而后快吗?我现在就在你们眼前啊,你们倒是动手啊。”方源微笑道。

%84
厉煌气得双眼通红,就算是顾六如也呼吸急促,死死盯着方源的目光仿佛深渊的寒冰,杀意凛然。

%85
三座仙蛊屋周围逐渐连腾挪的空间都没有了!

%86
若从远方遥望,不断地有年兽冲杀过来,到了最中央的战团,这些年兽、上古年兽、太古年兽都密密麻麻,仿佛凝聚成了一个蚂蚁圆球。

%87
圆球越塞越满,越来越大。

%88
球心里的三座仙蛊屋,仿佛是绞肉场,哪怕是太古年兽进来,都支撑不了十几个呼吸,就被绞杀阵亡。

%89
厉煌、顾六如均打出真火。

%90
镇河锁莲大阵已经彻底玩完了。天庭这段时间的努力都打了水漂。

%91
能救下的中洲蛊仙,厉煌、顾六如都已经尽力了,但大部分仍旧惨死在方源的手中。

%92
“走吧,再没有什么幸存者了。”顾六如喊道。

%93
没有了大阵羁绊方源,方源操纵万年斗飞车纵横战场,来去自如,根本拘杀不得。

%94
至此,天庭此次的战术已经彻底失败。

%95
“方源,你给我等着,你嚣张不了多久。迟早有一天,你会死在我的手里!”厉煌咬牙切齿。

%96
他和方源第一次交手,方源的作为让他感到十分的愤怒、仇恨和恶心!

%97
这个魔头实在是阴险狡诈!

%98
“哼!厉煌,依我看你才是嚣张!”万年斗飞车忽然发难,底下是杀招激流勇进!周围是破晓剑!银色的飞剑纷纷扎入激流当中,带来超乎意料的爆炸般的速度!

%99
“这两招居然还能配合?!”厉煌大惊,无法及时防范。

%100
万年斗飞车直接撞上伤痕累累的鲨流撬!

%101
轰隆。

%102
巨响声中,鲨流撬直接被撞散,剩余不多的巨鲨横死当场。

%103
关键时刻,厉煌催起最引以为傲的杀招阳莽背火衣。

%104
万年斗飞车撞碎鲨流撬,余势不减,结结实实地撞在厉煌身上。

%105
厉煌的阳莽背火衣剧烈摇曳,前一刻是燎原巨火,被撞之后,瞬间减弱成了小火苗。

%106
“厉煌你的死期到了!”方源眼冒精芒,施展夏扇、春剪,齐袭厉煌。

%107
“八转层次?!”厉煌瞪大双眼,从夏扇、春剪杀招中感受到了致命的威胁。

%108
他一口气都来不及喘,拼死防范。

%109
结果真接了招,春剪、夏扇立即露出本来面貌——只有七转的层次!

%110
“方源你骗我!”厉煌大吼,精力被方源成功牵扯。

%111
随后,万年斗飞车紧随而至。

%112
厉煌瞪得老大的瞳眸中,瞬间充斥着万年斗飞车的银色巨影。

%113
轰!

%114
厉煌身上的背火衣被彻底撞灭,厉煌胸膛完全凹下去,大量的鲜血像是喷泉般爆涌而出,期间混杂着各个内脏的碎片。

%115
嗖嗖嗖。

%116
无数的银色飞剑随后袭来,宛若蜂群一般密密麻麻,深深地扎进厉煌的肉身中,又从前后左右不断穿透出去。

%117
厉煌被破晓剑带动,身躯不断地颤抖。

%118
他眼眸的神光迅速消散。

%119
临死之前,他的脑海中的最后念头仍旧是难以置信:“我竟死在这里了?!”

%120
“我们快撤——!”顾六如低吼一声,眼中含泪。

%121
他额头青筋暴起,满脸狰狞之色。

%122
厉煌虽强过他,但在这光阴长河中,没有了仙蛊屋保护,必定是凶多吉少。

%123
顾六如不是没有想过要去救他,但方源蓄谋已久,怎么会给他施救的机会?

%124
周围密密麻麻的年兽,也是巨大的阻碍。

%125
顾六如操纵三秋黄鹤台,迅速脱离战场。

%126
方源紧追不舍,杀机鼎沸,:“天庭的蛊仙,都给我留下命来罢。”

%127
顾六如闷声不吭,只顾逃窜。

%128
他把这份战败的耻辱深深地印在心头:“正消魔长……呜呼!眼下最明智的就是保存实力,留待将来再战。方源……你这个魔头尽管就现在猖狂吧。”

\end{this_body}

