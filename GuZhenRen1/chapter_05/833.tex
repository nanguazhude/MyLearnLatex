\newsection{炼蛊交锋}    %第八百三十七节:炼蛊交锋

\begin{this_body}

南疆。

巴家,大宅山。

巴十八小心翼翼地拿捏着一只一转凡蛊,双眼一眨不眨,神色极为专注:“这就是第二空窍蛊?”

他细细打量着,只见这只蛊虫形如甲虫,两头尖尖,中间肥大,仿佛青玉所制,只有手指头大小。摸在手中,感觉一片温润清凉。在它圆滚滚的背部,还长着一只金色的眼珠子。金色的瞳孔,闪电般游弋不定,灵性十足。

为了换取这只蛊虫,巴家几乎掏空了库藏中剩下来的所有的运道仙材。

方源是狮子大开口,但巴十八却只能承受,皆因这只小小的一转蛊,却是代表着他生活下去的全部勇气和希望!

第二空窍蛊使用非常方便,根本不需要什么真元、仙元,只要鲜血滴入蛊虫体内,一直等到蛊虫饱饮了鲜血后自爆开来,即可大功告成。

第二空窍蛊自爆之后,立即在巴十八的胸膛中间形成了一个小小的空窍。

这是一转空窍,最低等的修为。

但巴十八见此,却是眼眶泛红,激动得身躯都微微颤抖起来。

“即便方源那魔头要价不菲,但我也认了!”

“有了这个空窍,我就能重新修行。从一转到五转,自有舍利蛊助我一臂之力。到了六转层次后,还有族中蛊仙帮我渡劫。”

“只是要达到第一仙窍的底蕴和成就,那就难了。”

想到这里,巴十八咬牙切齿,心中充斥着对方源的仇恨。

方源将他俘虏,又彻底搜魂,还把他的第一仙窍都给取走了。就算他送来了第二空窍蛊,也不过是换了一种形式的勒索敲诈罢了!

对于八转修为的巴十八而言,乃是奇耻大辱。

“方源你给我等着,早晚有一天我要一雪前耻。你施加在我身上的种种痛苦,我会百倍地奉还到你的身上,让你也好好品尝一番滋味!”

巴十八脸色狰狞,这些日子他可不好过。

他乃是巴家的太上大长老,但被方源俘虏之后,不仅声望大跌,赎回后更沦为了废人。

比起夏槎,巴十八的情况还要更糟糕一些。

因为巴家在他被俘期间,又有了一位八转蛊仙。

那就是巴德!

巴德成为八荒蛊仙后,立即晋升为巴家太上二长老,统领巴家事务。

巴十八已经被夺权了。

“幸好有这么一只第二空窍蛊!”

“我有充沛的修行经验,大大小小渡劫数十次,再次升仙非常容易,几乎十拿九稳。”

“我得抓紧时间,尽快地重修成蛊仙。毕竟我的影响力会随着时间越来越小,这种影响力需要及时使用,必定能给我的修行带来极大帮助。”

巴十八早有谋算,舍利蛊他也准备好了。

当即一只只用掉,使得巴十八迅速成为一转巅峰的蛊师。

然而到了这一步后,不管巴十八如何冲击窍壁,窍壁皆是纹丝不动。

“怎么会这样?”

“我的修为被卡住了!这个第二空窍很有问题!”

巴十八探查清楚情况后,惊怒交加,感受上当受骗,花了巨大代价却买了一个假货。

他连忙呼唤曾经的亲近蛊仙,要求他们立即联系方源,责问他这到底是怎么回事。

方源正在钻研着元始真传,神色微动,笑了笑,眼中闪烁着阴芒:“看来他们已经发现了不妥之处了。”

不过,他对此早有准备,当即做出回应。

巴十八得到这个回应,当即气得将最喜爱的杯盏打碎:“魔头,欺人太甚!”

原来方源在之前的来信中,只是说第二空窍蛊能够开辟第二空窍,但并没有说明修为方面拥有的限制。为了误导南疆群仙,方源还阐述了自己曾经使用了第二空窍蛊的经验和感受。

巴十八便一厢情愿地认为,有了第二空窍蛊自己就能晋升成仙,重修回来。

倒是夏槎那边拥有智道蛊仙,辨认出此中猫腻。方源便告诉夏槎,这第二空窍蛊本是仙蛊,经由他改良之后,形成一个完整的系列,从一转到六转皆有。用了一转的第二空窍蛊,修为只能局限在一转,二转、三转乃至六转同理。

夏槎便想直接收购六转的第二空窍蛊,结果方源回了一句话,让她当即无语——我目前只卖一转的第二空窍蛊!

“看来巴十八的处境,比夏槎还要差了许多啊。”方源笑了笑,他见微知著,从种种细节中探知到了巴十八内心的焦躁。

“由此看来,那巴德恐怕是成就八转了,虽然巴家一直秘而不宣。”

“这是一件好事情啊。”

南疆群仙虽和方源有着不共戴天之仇,但双方却有一个共同的大敌——天庭。

天庭要修复宿命蛊,上一世的经历已经让方源看明白南疆正道的立场。

所以,方源将南疆群仙的肉身、魂魄都还了回去,仙窍虽然吞并了,但是又给他们交易了第二空窍蛊。

目的就是要让南疆群仙保持一定的战斗力,足够去找中洲天庭的麻烦。

经历了这么多,一路艰难坎坷,方源终于从天庭的重重压迫下挺了过来。他如今底蕴深厚,下属众多,四处插足五域政局,或明或暗动用各种手段,最终形成一个包围天庭的大局!

不知不觉间,他已经是从棋子成长为了一名棋手。

“方源我恨不得把吃你的肉,喝你的血啊!”巴十八望着新的回信,气得浑身都在颤抖。

他彻底明白了方源的企图。

方源炼制出的第二空窍蛊乃是一整套,巴十八若想重修回去,就得从一转、二转、三转……如此这般不断地用下去,直到用了六转级数的第二空窍蛊,方才拥有可以健康成长的仙窍,可以真正地重修回八转修为去。

也就是说,巴十八要想恢复从前的荣光,得被方源勒索敲诈整整六次!

巴十八大声咒骂,又摔碎了好些物什,终于缓缓平静下来。

望着一片杂乱的书房,他沉默半晌,忽然哈哈大笑:“巴十八啊巴十八,你怎么过会去了?此事本来就是你技不如人,沦落成了俘虏,能存活一命,就已是侥幸了。如今又何须患得患失呢?”

“就算重修不成,又能如何?只要拼尽全力奋斗过来,便是至死无悔。”

他自言自语,像是劝说自己。

说完这句话,他陡然全身轻松起来,只感觉心堂一亮,似乎是登高望远,见到了曾经八转时都不曾见到的风景。

这是人生的风景!

他彻底地平静下来,甚至嘴角泛起了微笑。

“方源你这个魔头,真是厉害,我又着了你的道了啊。”心中的仇恨并无一丝衰减,但巴十八却开始真正笑着面对,不再患得患失。

这一刻,他虽然毫无修为,但比之前八转时还要从容潇洒。

“如今没有什么办法,只好继续接受勒索。毕竟重新成仙,才是最重要的事情啊!”巴十八心中一片雪亮。

家族可以依靠,但只能依靠一时。往日的影响力和情分,会随着时间迅速衰减。

尤其是族中又有了新的八转蛊仙。

他必须抓紧时间!

“腐土血粉,地中藏花。玉骨成瓣,冰肌化茎,花心金舍利。星火烂漫,汇拢冰雪成原。其下有阳云升火如丹,其上有阴云落沙似金,中空增添兽影,直至电光霹雳,生兽力胎盘,便可集人窍……”

这是第二空窍蛊的蛊方。

方源从三王福地中得来。

前面的部分和原版相差不多,方源改变得很少,但到后面一部分却是发生了剧变,几乎是面目全非。

方源改良了这个蛊方,将原本只是六转的仙蛊方,扩展成了一个系列。

方源拥有炼道准无上境界,又是人道宗师,更有智慧光晕相助,改良出来不是什么难事。

“有了这一套蛊虫,就可以一路敲诈上去。没有机会敲诈勒索,我就创造机会敲诈。”

“事实上,这一套流程还可以用在东海、西漠、北原的蛊仙身上。至于中洲蛊仙,还是算了,该杀就杀。”

方源面厚心黑,乃是大大的奸商。

凡蛊级别的第二空窍蛊,成本极低,根本不需要什么寿蛊或者仙材。

只是需要人窍多一点。

人窍也就是蛊师的空窍。

方源虏点人族蛊师来充当蛊材,十分容易,简直是物美价廉。

但是卖给南疆群仙,却是价格极高,成百上千倍都不止的利润。

偏偏这些南疆群仙把第二空窍蛊当做救命的稻草,人生的唯一希望,方源不管怎么开价,他们都会捏着鼻子苦苦承受。

当然,方源也会计算他们的承受能力,还有整个围困中洲的大局。

“成功了,方源大人,我们又炼出了一只运道仙蛊!”毛民蛊仙又传喜讯。

“很好,再接再厉。”方源随口夸奖了一句。

这一次敲诈南疆群仙,因为有第二空窍蛊,效果比之前还要好很多。许多家族压箱底的运道仙材,都给了方源。

有了这样一批仙材,方源下令让毛民蛊仙们大肆炼制运道仙蛊,无须节制。

炼制仙蛊的代价是高昂的,运道仙材库存节节下降,但收获也有许多。

这些日子里,又有数只运道仙蛊炼成,落到了方源手中。

天庭。

“该死,又失败了。”袁琼都面色铁青,望着眼前的一堆残渣,整个人都很不好。

最近这段时间,他几乎要怀疑自己的人生!

先前抢炼定空蛊,失败。

最近又炼制运道蛊虫,失败,失败,还是失败!

“我的手法没有什么错,整个过程也没有什么失误,但是仙蛊唯一。”

“很多仙蛊已经存在,我根本不可能炼出来!”

“唉,如果能事先知晓那些仙蛊早已存在,那就好了。”

方源抢炼运道仙蛊的战术,让天庭方面吃尽苦头。

这是一场隐形的交锋,谁能占据先手,谁就能获得最大的收益。

天庭虽然底蕴深厚,库存深不见底,但方源借力南疆正道,牢牢把天庭压在下风。

除非方源主动收手,袁琼都将会继续失败下去,成果稀少。

------------

\end{this_body}

