\newsection{南仙众志}    %第六百九十节:南仙众志

\begin{this_body}



%1
“不好,以武庸为首的大批南疆蛊仙,忽然出现在了不败福地附近!”天庭大殿中,紫薇仙子陡然变色惊呼。

%2
“哦?终于出现了么。”龙公淡淡一句,仍旧坐定主位,毫无动容之情。

%3
紫薇仙子深呼吸一口气,眉头紧皱,感受到了不同寻常的压力。

%4
因为南疆蛊仙此次行动,非常团结,和之前的那些四域蛊仙四处散开作乱不同。最关键的是,他们冲击的目标非常重要,正是不败福地!

%5
若是不败福地被攻陷,那对天庭而言就太糟糕了。因为之前付出如此惨重代价,才强行举办的炼蛊大会,一切的成果都要被南疆诸仙截下。

%6
“稍安勿躁。不败福地的位置被泄露,也是我们预估的情况之一,为此我们早已经在附近布置了超级仙阵。如今大局的主动权仍旧掌握在我们的手中,一切都按计划进行。”龙公吩咐道。

%7
紫薇仙子恭声应命。她催动手段,大殿中立即浮现出不败福地的情景。

%8
随后,她犹豫了一下,暗中向武庸传达消息去:“武庸,你可别忘了,你曾经和我天庭秘密合作过。你若是想继续担任南疆盟主,就该知道怎么去做。”

%9
武庸接到消息,冷笑一声,心中闪过一丝不屑之情:“堂堂紫薇,纵然是智道大能,格局也不过如此。”

%10
他连回应紫薇仙子消息,都懒得去做。

%11
他几乎全部的心神,都集中在了自己的杀招上。

%12
仙道杀招——无限风!

%13
墨绿色的狂风骤起,卷席四面八方。

%14
很快,狂风壮大,形成天柱一般的龙卷飓风,带着雷鸣般的隆隆声响,气势骇人地向前横冲直撞。一路上,拔起大树,带飞山石,威势凛然,无人可挡!

%15
眼见大风袭来,恶意昭昭,镇守附近的天庭蛊仙不得不叹息一声,加大力度催动超级仙阵。

%16
咯吱咯吱。

%17
仙阵凝聚的光晕,被无限风不断碾磨,很快便发出不堪重负的声音,令人头皮发麻。

%18
中洲蛊仙连忙催动仙阵中的手段,但是任何削弱无限风的招数,都效果有限。

%19
“没有用的,我这一招从未轻易施展,皆因它会消耗掉我身上上万的风道道痕。不过有着如此代价,它会长久不息,除非是特殊手段才能克制。”武庸开口,表情淡然,目光中却闪烁着凌厉的寒芒。

%20
“这样的决心,这样的意志……”目睹这一幕,远在天庭的紫薇仙子只有在心底深深叹息。

%21
武庸付出如此代价,施展出无限风,展露出了他无可更改的决意和斗志。

%22
武庸中年模样,相貌普通,体格强健,眼眉细长,给他整个人增添了一分阴鸠之气。此刻他位居诸仙最前方,傲立长空,领袖群仙,风采慑人。

%23
紫薇仙子将他的模样深深地印刻在心底,她发现自己仍旧是小瞧了这个人。

%24
南疆武庸!

%25
不世枭雄!

%26
他有武家勇武的血性,却又兼备隐忍的品质。武独秀在位时,他声名不显,韬光养晦,就连武独秀都看低他,直到临死前才明白自己这个亲身儿子的手段。

%27
他初上位,南疆纷乱,齐齐对武家施压,大片领土沦丧。武庸依靠自身实力,还有玉清滴风小竹楼力挽狂澜。

%28
梦境大战之后,武庸又冒着风险,和天庭合作,取回南疆诸仙的仙蛊,又将这些仙蛊统统归还,此举给他带来极大的威望!

%29
五行山脉他想要铲除方源,稍稍受挫。但之后,陆畏因就算有着乐土传人的名义,最终也没有争得过他,最终还是让他担任了南联的盟主。

%30
此刻,天庭修复宿命蛊到了最关键的时刻,武庸率领南联诸多精英,悍然出手,不顾紫薇仙子的要挟。

%31
因为他深深地清楚什么才是大局!

%32
绝不能让天庭再度拥有完整的宿命仙蛊!!

%33
呼呼呼!

%34
无限风不断吹袭,碾磨大阵,镇守的天庭蛊仙尝试了多种方法,都收效甚微。此招端的厉害,是武庸压箱底的手段,却尚是第一次公然亮相。不说天庭一方,就算是南疆诸仙目睹此风,也接连动容,惊喜的脸色下隐藏着忌惮和畏惧。

%35
很快,大阵被破开一个缺口,武庸一招手:“诸君,用命就在此刻,随我杀进去!”

%36
南疆诸仙鱼贯进入,镇守的天庭蛊仙紧张起来:“他们进来了,快快,启动第二层大阵!”

%37
南疆诸仙只感觉眼前大变,碧蓝之光骤然升起,化为无数粉红飞花,飞速旋转,遮天盖地,数以亿万,向南疆诸仙笼罩过来。

%38
一位七转蛊仙站了出来,他须髯如戟,雄威骁猛,乃是铁家的中流砥柱,姓铁名干戈。

%39
他有七转巅峰战力,面对大阵攻势,毫无畏惧,大喝一声,身边金光闪烁,密密麻麻的金色战戟凝聚出来。

%40
“去。”无数战戟宛若暴雨倾盆,撞击在飞花的洪流之中。

%41
对拼下,飞花、金戈爆散成粉色、金色的光辉碎屑。

%42
大阵攻势有八转之威,但铁干戈居然能强撑住,一时间竟没有落到下风。

%43
阵道大能,八转蛊仙池曲由就在他旁边站着,不断地推算着此阵的破绽。此番进攻,池曲由必将发挥巨大的作用。

%44
几个呼吸之后,金戈被漫天花雨的磅礴大势渐渐压下,七转蛊仙商缺挺身而出。

%45
他伸出手掌,掌心忽然破开一个洞口,强劲的吸力爆发出来,吸进许多花雨。

%46
翼黑霆大吼一声,绽射无数蓝色电芒,帮衬铁干戈撑住场面。

%47
而七转智道蛊仙夏流佩,一直站在池曲由的身后,一只手贴在池曲由的后背上,催动着智道手段,极大地加快池曲由的推算速度。

%48
不一会儿功夫,池曲由忽然动手!

%49
术业有专攻,在他的手段下,大阵的运转立即迟缓下来,露出的破绽即便不是专修阵道的蛊仙,都能看得清楚。

%50
轰轰轰!

%51
南疆蛊仙自然没有留手的打算,立即出手,数道威能卓绝的杀招轰击下去,立即轰破第二大阵。

%52
南疆诸仙迈进第三阵中。

%53
天庭蛊仙震动!

%54
“有池仙友在此,谈何大阵阻碍?”武庸大笑三声。

%55
其余诸仙亦都士气高昂。

%56
池曲由苦笑一声:“我这推算的手段从不滥用,皆因要时刻消耗我的寿元。不过在此战,我绝不能留手!”

%57
南疆诸仙均心头一震。

%58
武庸不由地看向池曲由,两人均看到彼此的决意!

%59
不过很快,池曲由的眉头微皱起来:“这第三阵我不用推算,就一眼看破,此阵相当简单,阵眼便是那林中最高大的巨树。”

%60
大阵简单易懂,乃是天庭故意为之,破除起来难度更要大得多。

%61
皆因这片树林非同凡响,能湮灭一切杀招威能。

%62
每一株都是湮圣木,最中央充当阵眼的那一株更是太古荒植级数。

%63
南疆诸仙要攻破这些树木,只得动用肉身。然而树木当中,又牵扯无数藤蔓,这些藤蔓亦都大有来头,非同小可。有的是青龙藤,有的是电绞藤,有的是锯齿钻心藤。

%64
第三阵的根本,乃是利用这片树林的道痕,布置出来的大阵。要破除此阵,强攻其他地方,都是无用功。唯有摧毁这片湮圣树林方可。

%65
南疆诸仙思索了一下,纷纷放出仙窍内的上古荒兽、太古荒兽。

%66
兽群冲击树林,和藤蔓纠缠绞杀,血雨纷飞。片刻后,大半兽群牺牲,剩下的在树林里晕头转向,一小部分甚至自相残杀起来。

%67
南疆诸仙脸色皆沉。

%68
“原来这里面还种下了七星内斗草,足以让上古荒兽自相残杀。”

%69
“太古荒兽虽不受这种内斗草的影响,但是却有消兽烂漫气的阻拦。”

%70
这种消兽烂漫气,蕴藏浓郁的毒道道痕,对于仙植、异人、人族毫无影响,只影响野兽身躯,能令太古荒兽瘦骨嶙峋,皮消肉绽,内脏消融。

%71
南疆诸仙尝试失败,武庸正要祭出玉清滴风小竹楼,有一个站出。

%72
“让我来吧。”

%73
群仙看去,只见这位蛊仙头皮光滑,毫无毛发,虎背熊腰,体格雄健。正是巴家七转蛊仙巴熊!

%74
“我虽不是力道蛊仙,却修行变化道,让我冲进去,摧毁湮圣木。”

%75
“不过,我只有七转巅峰战力,还请翼浩方大人为我加持。”巴熊神情淡漠地道。

%76
群仙皆是微微动容。

%77
巴熊只是七转巅峰战力,并非八转蛊仙,此刻出击,极其冒险,稍有不慎,就有陨落危险。

%78
翼浩方看向巴家太上大长老巴十八,见后者微微点头,便对巴熊出手。

%79
他的加持手段不只是南疆,更闻名于五域天下,巴熊顿时感到一股无以伦比的力量,流转在身体之中。

%80
他暴射而出,施展杀招,变成巨人,一路披荆斩棘,杀向最中央。

%81
天庭蛊仙自然不会坐视,立即催动手段阻止。

%82
南疆蛊仙纷纷出手,为巴熊掩护。

%83
最终,巴熊浑身浴血,皮开肉绽,数十处伤口深可见骨,拼尽全力终是将最中央的湮圣木推倒。

%84
一炷香的功夫还未到,南疆蛊仙已经连续突破三阵!

%85
“接下来就只得靠诸位了。”巴熊伤势极重,已经失去了战斗力。

%86
“仙友甘冒奇险,拼力死战,不管此行结果如何,都是我南联大大的功臣,回去后必有重奖!”武庸斩钉截铁地道。

%87
巴熊轻笑一声:“我亦是看到武庸、池曲由二位大人的举动,方才兴起拼杀之心。天庭的手太长了,屡次来我南疆。若是让他们彻底修复了宿命蛊,那还了得?”

%88
“是这个理。”武庸缓缓点头。

%89
他望着身边诸仙,信心比之前更增十倍!

%90
中洲万众一心,而我南疆诸仙众志成城,有这样的雄师,什么大阵堡垒不能攻破?

\end{this_body}

