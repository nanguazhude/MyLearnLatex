\newsection{升炼仙蛊}    %第六百六十三节:升炼仙蛊

\begin{this_body}

至尊仙窍。

小中洲。

乌云压顶,电闪雷鸣。

方源宙道分身位于高空,目光凝重,神情肃穆,时刻监察着整个炼蛊的过程,防止出现意外。

在乌云之下,原本是一片宽阔的平原,但如今却是布满了深坑。

一道道的闪电,击中在地面,砸出半球形的坑洞。坑洞里土壤被烤制成陶瓷一样,烟气袅袅。

“火候差不多了,本体应该要出手了!”宙道分身刚起这个念头,无数道闪电猛地汇聚成庞大的一股!

轰――!

天柱一般的湛蓝闪电,狠狠地轰在平原最中央的一块白色巨石上。

一下子,白石被闪电炸得四分五裂,无数的碎石向四周狂暴飞溅。

闪电迅猛无比,骤然汇拢轰击,又骤然消失无踪。

大如小山的白色石块也被炸毁,但半空中却停悬着一只小小的蛊虫。

蛊虫形如蝴蝶,五光十色,正是人如故仙蛊。

“失败了六次,终于练成了,不错不错!”方源宙道分身见此,喜形于色。

人如故仙蛊源自太白云生,其来源还能追溯到紫山真君身上。

原本的人如故仙蛊,只有六转层次,治疗六转蛊仙效果明显,但是治疗七转就差强人意了,时常还会催动失败。治疗八转,那效果就更不堪了。方源能有效用,多亏了他至尊仙体的特性――异种道痕之间不相排斥。不过如今八转修为,六转人如故显然是低级了一些。

但如今,人如故仙蛊成功地被升炼,从六转上升到了七转!

方源宙道分身仔细端详,旋即发现七转人如故仙蛊和六转之间的区别。

它们虽然外形都是蝴蝶,但七转人如故显然更加光彩夺目,并且蝴蝶的躯干部分发生了剧变,好似一个赤・裸的白嫩小人儿。

虽然八转人如故仙蛊,更加适合方源,但是升炼到七转这种程度,方源暂时就已经满足了。

要炼出八转仙蛊,会消耗海量的修行资源,方源目前完全没有能力承担。

当初,雪胡老祖为了炼出八转程度的鸿运齐天蛊,几乎把底蕴都耗尽了,搞得大雪山福地上上下下都怨声载道的。雪胡老祖修行的岁月可比这一世的方源,要多得多了。

除非是方源砸锅卖铁,比如卖出荡魂山、落魄谷、逆流河、智慧蛊之类的,就能迅速换取海量资源来。但这完全得不偿失,人如故仙蛊虽然十分极品,但并非春秋蝉这类的关键蛊虫。

炼出八转仙蛊,方源是没有这个奢望的。他打算将手中大部分的六转仙蛊,都升炼到七转。

因为数量太多,炼蛊的难度大,失败概率又很高,所以方源目前手中的资源都投入进去,还是够呛。

方源从五相洞天、戚家洞天、蓝鲸洞天以及红莲石岛中,都陆续收获了一批批的修行资源。

但炼蛊这行为,实在是一个无底洞,投入多少都不嫌多。

这还多亏了方源得到了悔蛊,布置出了炼道仙阵,可以反悔炼蛊当中的小过程,重新来过。

当方源宙道分身收起七转人如故仙蛊,天空中的乌云已经消散。炼道大阵徐徐启动,这大阵结合了洁身自好杀招,正将残留在场中的道痕去除干净。

“下一个就轮到江山如故蛊了。”宙道分身心中暗道。

炼蛊在紧锣密鼓的进行着,人如故、江山如故这两只宙道仙蛊首当其冲,但下面还有很多。

比如日蛊、我力、拔山、力气、金刚念、星眸、星念……

解谜仙蛊也是一个重点,它升炼之后,将令解梦杀招威能大涨,非常利于方源的梦道修行。

还有狗屎运、气运、察运、连运、时运这些运道仙蛊,升炼之后对方源也会有巨大的帮助。

运道方面,方源一直缺乏攻伐手段,但它辅助的作用根本无法忽略。在此之前,方源就借助运道方面的优势,逃脱了天庭的多座宙道仙蛊屋的追赶,逃出生天。

升炼仙蛊几乎都是方源本体在操作,宙道分身修为太低,不是很够格。

接受了红莲真传之后,方源仍旧身在红莲石岛上。

除了他在接连不断地炼蛊,白凝冰、黑楼兰等人也在岛上活动。她们相互切磋,努力适应着未来身。

在未来身的作用下,她们的战力均暴涨到七转高阶,乃至巅峰的程度。

这已经是杀招效果的极限。

手底下的人实力一下子暴涨这么多,之前的许多尴尬就没有了。下一次方源激战,这些人至少有参战的资格了。

遗憾的是,妙音仙子、白兔姑娘、毛六都已经阵亡。虽然方源手下还有一些奴隶蛊仙,不如戚家蛊仙,但这些人远不如前者那么可靠。他们身上的奴隶约束,很可能会因为未来身的强大而消散。所以这些人方源并不打算提升他们,哪怕未来身杀招还有几个晋升的名额。

升炼仙蛊成了方源最主要的工作,几乎占用了他全部的时间,偶尔他会利用定仙游杀招传送出去,去往五相洞天,继续炼化天相杀招。

方源已经继承了真传,主宰了这座红莲石岛,因此可以利用定仙游传送往返。

在此之前,这座红莲石岛一直被红莲魔尊的力量保护着,免疫定仙游此类的手段。

这和苍蓝龙鲸的仙窍洞天情况一致。

方源早就试过,可惜利用定仙游根本回不了龙鲸洞天。

虽然春秋蝉已经七转,可以扩大侦查的范围,但方源并没有去搜寻其他的红莲石岛。

一来,他目前实力并未上涨许多,还需要消化。

二来,天庭方面也在光阴长河中搜寻,碰到他们这股力量,方源还得退避三舍。

三来,方源的八转仙元储备也一直在积累,升炼仙蛊的工作消耗仙元也很多的。

时光匆匆流逝,一晃数月过去。

南疆。

轰――!

地动山摇,土石迸溅,一条地沟翻腾而出,野生仙蛊的沛然气息四下洋溢。

“这只野生仙蛊终于出世了!”云霄上,三家蛊仙俯视下方,均露出微笑。

“好了,按照我们之前商议的方法来赌斗吧。”

“嗯。地脉翻动的情况越来越剧烈,时不时就有野生仙蛊出世,我们速战速决。”

三家蛊仙公平决斗,气氛和谐。

南疆正道超级联盟的影响,越来越大了。

虽然之前武庸率领南联,没有对付得了方源,反而被方源以渡劫的名义,在五行山脉中算计。但南联并非没有收获,比如陶铸真传。陆畏因从方源的手中救下了诸多南疆七转蛊仙后,他代表背后的乐土势力,也正式加入了南联。

应许也有方源的关系,南联中的正道蛊仙有了共同的大敌,各自收敛了许多。

他们心中都有一层顾虑――南疆地脉频动,修行资源迭出,难保方源不会忽然出现争抢。所以南联蛊仙们再没有争得头破血流,反而和和气气。

天庭。

紫薇仙子刚刚听到汇报,此刻眉头扬起,眼眸中露出明显的喜悦之情。

“龙宫找到了!这可是一个好消息,之前龙公大人就一直关照,如今仙蛊屋龙宫终于被我天庭发现了。”

龙宫这座仙蛊屋非常强大,即便是天庭方面都很重视。

“这座仙蛊屋和龙公大人牵扯极深,还是先将这个消息告诉他罢。”

龙公得到这个消息,毫不迟疑,当即动身,临走前将凤金煌带往天庭,亲自交给紫薇仙子好生看管。

数天后,龙公却铩羽而归。

紫薇仙子感到意外,龙公实力极强,当今什么样的手段能够阻止他?

“龙宫周围竟然有一层梦境遮护,我不能通往。”龙公叹息。

得知这个答案,紫薇仙子顿时恍然,她劝慰龙公道:“龙公大人勿忧,我方的大阵已经布置妥当,能够借助整个天庭洞天的威能,压迫魔尊幽魂。过不了多久,魔尊幽魂掌握的梦道成果就能被我们得到了。”

龙公听了这个消息,哈哈一笑,连说了三声好。

魔尊幽魂的梦道成果,当然就是纯梦求真体。掌握了这个手段,天庭方面也能将梦境收拢凝聚,变成纯梦求真体。如此一来,龙宫周围的梦境难关就迎刃而解了。

“方源那魔头最近都没有动静么?”龙公问道。

紫薇仙子摇头:“自从他从光阴长河中逃脱,就再无他的音讯。如今五域都有地脉频动,南疆尤最,翻出海量的修行资源。我倒是希望他能够出现,四处搜刮这些资源,好让我们有可趁之机,但很可惜他一直销声匿迹,我一直在动用智道手段分析一切情报,都未发现他的踪迹。”

龙公点头:“继续关注,不过这魔头虽是达到了八转,但却要面对更强的灾劫。这绝对能令他头疼一番。他虽然有至尊仙体可以吞窍,但终究还是需要八转的洞天,才能跨越浩劫。还有更加恐怖的万劫呢!”

一提到万劫,紫薇仙子也是心头一悸,点头微笑道:“龙公大人说的极是,方源虽然升上八转,但天底下哪有那么多的八转洞天可以供他吞噬?他的修为短时间内不会有多少变化了。”

“眼下,方源缺乏资源,尤其是宙道资源,难以抗衡我方在光阴长河中的布置,基本上也就和红莲真传无缘了。”

“上一次虽然令他逃脱,但他的仙元消耗极大,底牌几乎全部曝光。他要迅速积累八转仙元,靠仙元石效率太低了,唯有依赖仙窍自产。但仙窍自产需要时间,他若是加快仙窍光阴流速,就会加快灾劫的到来。”

“一旦他缺乏仙窍吞并,无法绕过灾劫,那么就只有直接渡劫。”

“一旦他真正渡劫,就是我天庭取他性命的时候了!”

紫薇仙子说了这一番话,双眼熠熠生辉。

龙公也微微点头,赞赏地看向紫薇仙子:“有你主持天庭,我很放心。”

\end{this_body}

