\newsection{一曲离歌退武庸!}    %第三百七十一节:一曲离歌退武庸!

\begin{this_body}

不论是少清风刃、正清碧雷,还是玉清滴风、玄清仙音,这些变化的本质,都是风道。

显然,创造出这记仙道战场杀招的武家先祖,已然修行到了触类旁通的境地,从风道延伸出去,涉及剑道、雷道、水道和音道。

武庸运用此招,果然是效果卓绝,打得方源无可奈何,凤九歌身陷险境。

玄清仙音越发密集嘹亮,而凤九歌身上的乐声则被压制得几乎微不可闻。

方源毫无办法。

他身上仙蛊很多,荒兽、上古荒兽等等不在少数,仙道杀招也很多,但是面对八转蛊仙武庸,这些东西就显得杂而不精,层次完全不够,登不上台面。

唯有逆流护身印,算得上方源维持不败的凭借。可惜这等底牌,也不能解决眼前的危局。

玄清仙音越加清澈响亮,方源身上逆流仙衣都在震荡不定,逆流河水不断地削减,速度之快,让方源为之侧目。

方源更担心的是凤九歌。

鲜红的血迹,从凤九歌的鼻腔、嘴角、眼角外溢出来。

他的身躯不断颤抖,并且幅度越来越大。

“糟糕!如此下去,凤九歌酝酿的杀招被打断,必定反噬,随后重伤,若算上武庸的攻势,甚至当场死亡也有可能。”

方源明白这一点,却是爱莫能助。

他当然不是牵挂凤九歌,而是凤九歌一死,他就要直面武庸。

将来就算锁风效果消失,方源可以再用四通八达,面对武庸,他也没有多少机会可言。

不过,就在局势渐渐向武庸倾斜的时候,凤九歌忽然吐出一口浊气,整个人放松了下来。

然后,他的身上迅速升腾起一股独特的气息。

凤九歌缓缓开口,从口腔中开始传出一道微弱的歌声。

这声调一点都不高,低弱微小,但偏偏却清晰地钻入方源、武庸的耳中。

然后歌声稍微大了一点,像是低吟,又仿佛是在安静的夜晚的丝丝梦呓。

一股油然而生的情感,从武庸的心头升腾起来。

武庸面色顿时变化!

他身上覆盖着仙级防护,可谓层层包裹,防御出众,但当他听到这个歌声的时候,这些防护竟然形同虚设,让歌声勾动出了他内心的感情。

方源的脸色也发生了变化。

他虽然没有产生什么情感,但是逆流仙衣的表面,却是泛出越加繁复庞大的涟漪微澜。

这种阵仗,竟是比之前受到玄清仙音的攻势,还要严重!

“好家伙,这音道杀招非同凡响!乃是一网打尽的威能,我也被牵扯其中,还是离凤九歌远一些的好。”方源连忙远撤。

这不仅是为了他自己,而是为了凤九歌着想。

武庸同样做出如此选择,他撤退得更彻底,再次隐去身形,消失不见。

能退能进,这就是占据战场地利的优势。

而整个仙道战场,则攻势更急,更加狂猛。玄清仙音、少清风刃、玉清滴风、正清碧雷,四种变化一齐出现,层层叠叠地覆盖凤九歌。

凤九歌不管不顾,用身躯硬抗种种攻势,他没有停止,歌声在持续。

歌声只是单纯的乐音,却蕴藏着音道的无穷奥妙。

歌声渐渐上扬,但并非一飞冲天,而是寰转柔婉,一圈一圈,回环往复,交替上升。

情感不断地缠绕在武庸的心头,让他更加惊疑。

“这到底是什么音道杀招?又是什么威能?”

歌声忽然又低垂下来,无以伦比的抑郁和伤感,袭击武庸心头,让他都要一种落泪的冲动!

不由自主地,武庸想到了一个词——离别。

离别的苦楚,离别的悲伤,离别的抑郁,离别的不舍。

和情人的分手,和朋友的再不见,和亲人的生死绝别。

别离,往事依旧。

别离,故人挥手。

别离,夕阳映映。

别离,落红亦悲愁。

歌声时而凄切,像是控诉命运的不公,时而哀怨,像是质问现实的残忍,时而悲愤,像在心中低吼,时而垂落,欲哭无泪,泪在心中。

噗!

武庸忽然雄躯一震,向外大吐一口鲜血。

他的脸上显露出难以置信的神色,因为在他的感知中,他惊骇地发现,自己的仙道战场杀招四清四变风,居然在瓦解,在分离!

组成这个仙道战场杀招的蛊虫,并没有因为歌声而损毁任何一只,就算是凡蛊,也安然无恙。

当然,方源用逆流仙衣逆反回去的攻势,自然是造成了蛊虫的损毁,不过武庸也在同时进行修补。这点无伤大雅。

让武庸感到吃惊的是,这些构成仙道战场杀招的蛊虫,并没有丝毫的损毁,但是却不受他的操控,开始相互分离,不愿意再进行配合和运转。

这还了得?!

一两个蛊虫的配合紊乱,只要不是杀招核心,不算个事儿。

一部分的蛊虫毁灭了,只要不是核心,仍旧可以维持局面。

但是现在,却是所有的蛊虫都相互分离,都要撤走,武庸根本操纵不住。

一切的罪魁祸首,就是这首歌曲,就是凤九歌。

很快,方源也看出端倪,他侦查到整个战场正在分离,正在崩解。

方源心中自然也惊奇得很。

武庸心知不妙,想要撤销战场杀招四清四变风,但为时已晚。

下一刻,整个战场崩溃,三仙重见晴朗天空,明日白云。

武庸狂喷一口鲜血,面色惨白,显露身形,踉跄飞退,遭受重创。

仙道战场杀招和蛊阵不一样,蛊阵被破坏,蛊仙反噬往往并不严重。但是仙道战场杀招被破,反噬通常都会很沉重。

杀招的威能越强,反噬的伤害就越大。

四清四变风!

这可是武家名镇南疆的超级杀招,威力绝伦,此刻被破解,带给武庸的反噬伤害极其剧烈。

“好杀招!此招何名?”武庸双眼死死瞪着凤九歌,问道。

凤九歌却不答他,而是仍旧歌唱不休。

武庸面色再变,此时此刻,再顾忌不得,拿出了仙蛊屋玉清滴风小竹楼。

这座八转仙蛊屋承载武庸,提供坚强防护,让武庸可以在里面从容疗伤。

仙元的消耗问题,已经不是主要问题了。

武庸不得不这么做。

但是很快,让他吃惊的事实再度发生。

整座玉清滴风小竹楼,竟然也在歌声的作用下,开始蠢蠢欲动,有了一丝分解离别的趋势。

这可是堂堂的八转仙蛊屋!

如此看来,凤九歌的这记杀招,绝对是八转层次。

武庸见此,双眼暴射出骇人的精芒,再顾不得疗伤,直接催动仙蛊屋向凤九歌撞去。

仙蛊屋没有短板,乃是阵道的巅峰结晶。仙蛊屋横冲直撞,向来在战场上都是纵横披靡的。

这不仅是过去的人族历史上的事实,也是方源五百年前世,饱经检验的常理。

当然,每一座仙蛊屋之间,也都各有所长。

譬如黑家的黑牢,可以驯化上古荒兽。

再例如白相洞天中的三十三天殿,就是一座仙蛊屋,防御很出色,最擅长的是储藏仙材。不过可惜的是,三十三天殿,只是一座残缺的仙蛊屋,曾经遭受重创,如今已经失去了移动之能。

玉清滴风小竹楼自然有杀招,但武庸此时直接撞向凤九歌,也是明智之举。

毕竟催动杀招,都需要时间。

横冲直撞却是耗时最小的,也是最直接的。

凤九歌使出仙道杀招,虽然威能效果惊人至极,但似乎是不能离开原地,好像只能被动挨打。

这点破绽,早已经被方源、武庸看在心底。

凤九歌身上的防御手段,也已经被削弱到谷底,此时若再被玉清滴风小竹楼撞上,绝对是撞成一滩肉泥的悲惨下场。

危难之间,方源自然不能坐视不管。武庸解决了凤九歌,下一个就轮到他。

现在锁风杀招,还未过时效,方源再次参战,只身挡在玉清滴风小竹楼的前行路线上。

方源无法催动力道大手印,他大部分心神要维系逆流护身印,只得面前将双臂和手化为龙爪,覆盖龙鳞,然后催动七转龙力仙蛊。

这些仙蛊,都是他从武家“借”得,却用来对付武庸。

武庸见了,胸口一闷,仙元狠狠地灌输到仙蛊屋中去。

砰!

双方撞在一块儿,没有任何的意外,方源被直接撞飞。

他的力量不如玉清滴风小竹楼,像是一颗炮弹,被弹飞老远。

不过他身上一点都没有什么伤势。因为逆流护身印实在太过优秀。

反倒是玉清滴风小竹楼,在对撞之后,损失了不少凡蛊。

有了这个缺口,凤九歌的歌声更有效果。

武庸抓紧时机,继续催动仙蛊屋,撞向凤九歌,与此同时,他开始分出大部分心神,要催起玉清滴风小竹楼的杀招。

眼看着仙蛊屋就要撞上凤九歌,方源回援不及。

不过就在这时,凤九歌忽然抽身,一飞冲天。

武庸扑了一个空,满脸惊怒。凤九歌明明可以离开和移动,原来之前的只是假象,是凤九歌故意示敌以弱!

歌声一直在继续,仙蛊屋玉清滴风小竹楼开始分解。

从外形看去,它本身仿佛是一座两层的吊脚竹楼。它虽然是八转仙蛊屋,但它动用的蛊虫规模很少。不像是那些宫殿,往往体型越庞大的仙蛊屋,蛊虫的数量就越多。

总共就这么多的蛊虫,凤九歌的歌声就显得更有效果。

先是一根根的竹子,开始离散,边缘的竹子则直接脱离仙蛊屋主体。

这些竹子当然不是真的竹子,很快,脱离主体的粗壮竹竿,就散发翠亮的光辉,分解成无数的小蛊虫。

武庸面色阴沉如水。

局面对他非常不利!

这时,方源再次赶到。

武庸冷哼一声,盯着方源、凤九歌深深地看了一眼,随后他调动玉清滴风小竹楼,忽然飞撤。

不一会儿,玉清滴风小竹楼就已经撤离战场,成为天边的一个小黑点。

他竟是主动撤了!

\end{this_body}

