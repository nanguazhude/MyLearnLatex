\newsection{局势骤缓}    %第三百二十八节:局势骤缓

\begin{this_body}

议事厅中沉静下来,众人的目光集中在了一位蛊仙身上。

这位蛊仙模样年轻,面容普通,并不起眼,但实际上他却是柴家蛊仙中的重要人物,柴家太上大长老的心腹、得力干将。

身为信道蛊仙的柴有言,酝酿了一下,而后徐徐道:“我思考的东西,可能和大家不同。这几天,我其实都在琢磨武遗海这个人。”

“武庸失踪,传出死讯,不管是否是事实,究竟是何人出手,武家目前是武遗海统领。他的性情如何,他的诚意到底有多少,都是我心中的问题。”

“武遗海的来历,大家都知道。我们对他的了解,实在太少了。毕竟他不是南疆土生土长的蛊仙。”

“嗯,说下去。”柴家太上大长老的眼中,泛出感兴趣的光来。

柴有言笑了笑,这才继续道:“其实武遗海之前的情报,大家都知道,我在这里,也无须赘述。从这些情报当中,大家或许多少能了解,这个人的性情和心态。我说一个大家可能都忽略的东西吧。”

“哦?我们都忽略的?”

柴有言摸了摸自己的鼻子:“就是那封信。在武遗海寄来的信中,他表示要和我族联盟的愿望,共击巴家、夏家。他并没有选择巴家,或者单独选择夏家,而是将两者一同列为对手。”

“武家的行事风格,向来霸道,这不就是他们一贯的作风吗?”有柴家蛊仙不解。

“不不不。武遗海是不一样的。”柴有言缓缓摇头,“这点从他处理广寒峰,还有和驱山老怪达成协议的事情中,都可以看出来,他的行事手法更接近于散修。毕竟,他才加入武家多长时间?”

“武遗海缺少武家的霸气,但却有散修隐修的精明,他善于妥协,而且在我看来,他很有混迹正道的天赋。”

“为什么他要直接将巴家、夏家都列为打击的目标呢?就像刚刚我们有人说过,为什么不选择其一,和另外一家妥协呢?因为当我们真的和武家联手之后,选择其中一家开战,另外一家怎可能作壁上观?必定会联合一家,共同对抗我族和武家的联盟!”

柴家蛊仙闻言,不由纷纷点头,赞同柴有言的说法。

柴家、巴家、夏家,这三大家族都集中在了赤碧江心处,关系非常复杂。一家强势,其余两家必定联手抗敌。

三者形成实力的平衡,相互制约,这也是柴家发展这么多年,一直都没有打出去的原因所在。

柴家若联手武家,势必会打破这样的局部平衡,巴家、夏家怎可能不联手呢?

所以,方源在信中,干脆就直接将两家都列为打击对象。txt下载80txt.com因为他知道,这是必然要发生的!

“这么说来,武遗海有相当成熟的眼光。他既然能够提出和我柴家联合,势必已经有了许多的准备。对于武家耍我们一道的担忧,似乎没有必要。”有人道。

柴有言点点头:“我想的讲的,是武遗海此人有很深的政治造诣。从这些种种表现和细节来看,他很具有合作的诚意。”

“事实上,我们柴家是肯定要出动的。”

“我族发展至今,压下巴家、夏家,并不容易,是数代的努力,这过程中还有运气的成分。”

“一旦让巴家、夏家侵吞更多资源,底蕴增长。我们柴家的优势就会沦丧。”

“当然,和武家的合作也需要谨慎。”

“如何开战,这场战争又局限在何种程度比较合适?还需要大长老您来定夺。”

“不过,至少我们应该分出一部分的蛊仙,进行佯攻。至少不能让巴家、夏家,太过轻松地扩张领地吧?”

这番言论一出,大厅中的柴家蛊仙们纷纷点头。

几乎与此同时。

南疆,万蛇山,池家大本营。

池家太上大长老手中拿捏着一只信蛊,沉吟不语。

池家位于南疆的最西部,右下方是羊家,右边则是乔家。

池家和这两家都有领土的接壤,不过和羊家距离更近,地盘接壤的部分也最多。

远交近攻,一直都是通常的外交选择。

所以,池家和羊家一直关系不好。羊家的鬼手山,位于黄龙江发源地,资源丰厚,地形险要,羊家蛊仙主修魂道,战力突出,一直都是池家的心腹之患。

在过往的相处过程中,羊家和池家也不断发生过许多矛盾,有着历史积累。

羊家的右方,顺着黄龙江一路往东,就是武家的地盘。

其实武家和池家,也算是挨着。虽然两者并不接壤,但是乔家和池家接壤啊。

乔家是武家某一代的太上大长老,大力扶持出来。乔家安插在南疆西部的中心地带,北抗巴家、夏家,西对池家,甚至还能牵制南疆最西南的羊家。

池家和乔家时有摩擦,谁都知道乔家就是武家豢养的看门狗,所以池家和武家关系一直是时好时坏,随着时局而变动。

若用一个词来形容池家和武家的关系的话,暧昧这个词或许比较恰当。

在不久前,池家出手刁难过武家。但在义天山遗址上的超级蛊阵中,池家却是和武家合作,搞起仙缘生意。

“武遗海,你可是给老夫出了一个难题啊。”书房中,池家的太上大长老池曲由长长的叹息一声。

方源在信中,要求联合池家,一同对付羊家。

池曲由自然心动,因为他知道这是一个很好的机会。这种机会,在他有生之年,都不多见。

任何势力都要发展壮大。吞并羊家的资源,就算吞并不了,削减羊家的实力,对于池家而言看,都是百利无一害的事情。

池家非常有余力。

不仅是因为,池家大本营地理位置偏僻,而且更关键的一点是,池家擅长蛊阵。一套套仙级蛊阵安置下来,池家大本营,还有各处的资源点的防御力都非常出色。

池家的后顾之忧是很少的。池家的仙蛊屋也有两座。别忘了,还有池曲由这位八转蛊仙的存在。

武庸始终,死讯传来,南疆半壁江山都乱套了。

在这样的乱局中,池家何去何从,需要池曲由的迅速决断!

数日后,超级蛊阵中,方源出关。

他的脸上带着一丝淡淡的喜色。

闭关修行很有成果,他已经将刚背杀招彻底改良完成,并且演练了好几番。

而就在这数天里,南疆的局势也发生了巨大的变化。

柴家派遣蛊仙,降临江心,做出一番威胁态势,巴家、夏家不得不收敛一些力量,进行防备。

池家蛊仙直接侵占羊家地盘,两家比邻而居多年,想要找一些借口,自然轻而易举。

双方已经在九幽山脉一带,直接开打了。

池家老爷子很有魄力,直接带着一座仙蛊屋,亲自动身。导致羊家大为紧张,不得不派遣出仙蛊屋和大量蛊仙面对。羊家没有八转蛊仙,这让他们陷入被动当中。

除此之外,还有罗家、铁家、翼家都有异常动向。

因为这几个家族,方源也都去了信。当然信中的内容,又和池家、柴家不同了。

毕竟这四家,地理位置太过遥远,都处于南疆的东部。

即便如此,这几家的异常动向,都把侯家、姚家的注意力,牵扯了几分回去。尽管他们也知道,这些家族动手的可能不大,但这种事情不怕一万就怕万一,不得不防。

至于商家,这个家族一直恪守中立,方源虽然去了信,但商家随即就回信,拒绝了武家的联合提议。

忽然一夜春风来,武家、乔家原本紧张严峻的形势,骤然缓和下来。

武家、乔家蛊仙对这样的结果,大为惊叹。

别看方源只去了几封信,但事实上通过一封信,能说服其他超级势力出手,是一件很不简单的事情。尤其是在信中,武家舍弃多少资源,表达多少诚意,都是值得斟酌的问题。

方源处理得相当棒,棒得超出很多人的想象极限。

很少有人知道他的那些信的内容,但很多人都明白,就是这些信,直接打动了许多超级势力的蛊仙。这让武家、乔家有了直接转守为攻的机会!

“看来我这方面的功底,并没有下滑多少。”方源前世五百年,担当过一方势力首脑,那时候五域乱战,政治格局更是混乱无比,五域内的合纵连横,五域外的联合、欺瞒哄骗、设计、将计就计、出卖背叛等等层出不穷。

现在的政局乱象,不过是围绕武家、乔家,仅仅在南疆西部的局部混乱。

在方源看来,并不复杂。

根本不需要抽丝剥茧,他就能直击要害。

“武家的仙蛊屋,终于要到了么?”方源又收到另一份情报。

本来仙蛊屋早早就出动了,但是在半途中发生了意外,耽搁了一段时间,不管是谁出的手,或许真是意外,总之方源还要等待一天,才能得到这座仙蛊屋的接应。

“看来必须要离开这里了。”这片梦境,带给方源太多的提升,期间也发生了很多的事情,对这样的宝地,方源有些依依不舍的情绪。

ps:第二更。未完待续。

\end{this_body}

