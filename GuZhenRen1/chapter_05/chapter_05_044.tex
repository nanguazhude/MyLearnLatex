\newsection{再炼仙蛊变形}    %第四十四节:再炼仙蛊变形

\begin{this_body}

%1
琅琊福地,几日之后。

%2
“这就是内景雨了。”琅琊地灵说着,将手中的一盆水,递给方源,让方源好好瞧瞧。

%3
方源伸出双手来接。

%4
“小心点。”琅琊地灵叮嘱道,“这份六转仙材十分特殊,有些‘沉重’。”

%5
琅琊地灵的话中,对沉重二字,明显加注了重音。

%6
方源点点头:“放心,这种仙材我亦有所耳闻。”

%7
说着,方源将这份仙材接到手中。

%8
顿时,他就感到心中一沉。

%9
仔细瞧去,盆中的水,清澈见底,又带着一丝丝清爽的绿意。方源明明端的很是平稳,但水面上却无故泛起涟漪,好像是细碎的春雨打在水面上的情形。

%10
这就是六转仙材内景雨!

%11
这种雨水,十分奇特,蕴含道痕,本身实际的重量并不重,但却会给接触的生命带来心灵上的负担。

%12
琅琊地灵强调的“沉重”,并非是重量上的沉重,而是心灵的沉重。

%13
“你真的决定,要耗费掉第三次,也是最后一次机会,让我炼蛊。但不是炼制定仙游,而是改炼变形仙蛊吗?”方源细心打量盆中仙材的时候,琅琊地灵又问道。

%14
“嗯,我已确定,不会再改!”方源点头,回答得干脆利落,毫无犹豫。

%15
方源经过深思熟虑,已经打定主意,今后主修变化道。

%16
定仙游虽好。但一来缺乏最关键的仙材太古之光,二来方源并不完全确定它已经毁了,三来远远比不上变形仙蛊对方源的重要意义。

%17
所以,方源毅然改变之前的主意,不惜耗费最后一份机缘。对炼制变形仙蛊发起勇敢的冲刺!

%18
仙蛊唯一,变形仙蛊乃是变化道的精髓!

%19
这只仙蛊,在历史上也是赫赫有名,曾经被狂蛮魔尊用过,提炼到了九转程度。

%20
但后来狂蛮魔尊陨落,高达九转的变形仙蛊也随之损毁。后人再炼出来时,就得从六转开始。

%21
论起底蕴。变化道远比剑道、血道要深厚得多。狂蛮魔尊之后。变化道中也陆续出了许多蛊仙强者。这些人宛如天空中一颗颗明星,在历史长河中熠熠生辉,惊才艳艳,传奇经历,叫后人缅怀、崇拜、向往。

%22
变形仙蛊作为其中最为重要的角色之一,也随着这些强者起起伏伏。历史上,变形仙蛊数十次损毁。又更多次被人重新炼制出来。大多数情况下,它是六转级数。七转级数的变形仙蛊,历史上出现的比较少。八转变形仙蛊就更少了,屈指可数。

%23
盖因仙蛊转数越高,越难炼制。且不论成功率越发渺小,为炼制准备的仙材也越加珍稀昂贵。想想当今的雪胡老祖,为了炼制八转仙蛊鸿运齐天,几乎倾家荡产地准备仙材,还勒令大雪山福地中的魔道蛊仙为他四处奔波,疯狂地搜刮仙材。从这一点就可以看出。炼制八转仙蛊的恐怖代价。

%24
所幸方源此次要炼的,只是六转程度的变形仙蛊。

%25
仙蛊方他有了,变形仙蛊被人炼出来的可能性也很小。

%26
仙材也准备得差不多,这点主要还是得益于琅琊福地本身丰富的仙材储备,而且这些仙材方源不用掏一分钱。

%27
因为盗天魔尊和长毛老祖的约定,琅琊地灵会免费地为方源炼蛊三次。凡蛊担保必成,仙蛊不保证成功。

%28
这一次是第三次。也是最后一次。

%29
“真是可惜。当初你有不败道痕,炼出变形仙蛊几乎是必然成功的。你说你重生回来,怎么恰巧就是那炼蛊时最关键的时候?”琅琊地灵现在回想起来,颇为惋惜。

%30
关于许多事情,方源都坦言告知琅琊地灵。毕竟他的这些秘密,已经彻底暴露了,琅琊地灵就算不知,也会从外界打探出来。

%31
“我那次重生,是被魔尊幽魂利用,但我现在回忆起来,总觉得还是疑点重重。”方源皱眉道。

%32
琅琊地灵摇摇头:“罢了,不说这些了。咱们虽然没有不败道痕,但我继承本体,拥有毛民特有的自然炼蛊法。此法能利用天地间的道痕,辅助炼蛊,虽有灾劫产生,但利大于弊。”

%33
“况且这些年,我琅琊福地利用仙灾锻窍杀招,渡过无数次灾劫,增添了海量的炼道道痕。再加上炼炉残屋的作用,第一次就成功炼出变形仙蛊,也是大有希望的。”

%34
“不过,为了更加保险,此次炼蛊方源你也要参与。你现在是狗屎运仙蛊之主,利用自身强悍的运气,也能增添炼蛊的成功可能。纵观炼制过程,内景雨的处理是第一个难关。这些天你就用你手中的这盆内景雨,多多练习一下。”

%35
运道辅助炼蛊,常有奇效。

%36
当初,巨阳仙尊和长毛老祖合作,硬生生炼出了一座八十八角真阳楼。这当中,就有运道的辅助。

%37
而盗天魔尊和长毛老祖携手,炼出一只遁空仙蛊,却还无法运用。管中窥豹,很明显偷道这个流派,远不如运道对于炼蛊的帮助大。

%38
接下来,琅琊地灵手把手教导方源一些毛民自然炼蛊法的奥妙,还有一些珍贵的炼蛊手法,甚至还包含许多炼蛊的凡道杀招。

%39
当然,这些东西都不是无偿的,是要从方源的门派贡献中扣除的。

%40
即便如此,方源也是欣然接受。

%41
这一任琅琊地灵,虽然炼蛊方面不如上一任,但到底是长毛老祖的执念,炼蛊造诣比方源不知高超到哪里去了。

%42
方源可谓得遇名师,炼道方面的许多疑惑和困扰,一一消除。那些珍贵的甚至已经失传已久的炼蛊手法、杀招。让方源受益匪浅。毛民的自然炼蛊法,更是让方源大开眼界,私底下觉得此法若是大成,远远比人族的隔绝炼蛊法更加优异!

%43
有些意思的是,方源在琢磨炼道的同时。他那位亲弟弟,也在云盖大陆底下的黑毛大陆中,星夜钻研炼道。

%44
方正盘坐在黑毛大陆王都的一间庭院里,手里摆弄着三色火焰。

%45
他脑海中的方源假意,强调道:“明天的比武,敌人十分强大!方正你虽然有五转修为,但是严重缺乏可用的蛊虫。你又是纯正的人族蛊师。在黑毛大陆上遭受排挤。只能自己炼蛊。今夜你必须炼成五转蛊吞江蟾。如此才能克制强敌的水道手段。”

%46
“我知道,你能不能闭嘴?!”方正在心中冷哼。

%47
方源假意不再说话,但很快,方正手中的三色火焰猛地崩散。

%48
方正大吐一口鲜血,炼蛊失败,让他身受内伤。

%49
“呵呵。”方源假意发出冷笑,“我愚蠢的弟弟啊。你果然还是一如既往的无能。这才刚刚开始炼蛊,你就在第三步失败了。”

%50
方正脸色十分难看,但毕竟事实摆在眼前,他无法辩驳,只好一声不吭,再次盘坐下来,默默地催动蛊虫为自己疗伤。

%51
方源假意再道:“无能其实也不要紧,谁都是从弱小和无能中,成长起来,强大起来。但关键是。不承认自己的弱小,意识不到自己的无能,那就没救了。你准备的蛊材,最多只能尝试三次炼蛊。这一次才刚刚开始,你就失败了。没有我的帮助,你如何能炼成吞江蟾?呵呵,明日对战。不论生死,你就算求饶对方也不会绕过你。你的确是没救了。”

%52
方正恼羞成怒,心中低呼:“我丧命那是我的事,你太聒噪了,快给我闭嘴!”

%53
方源假意阴森地道:“我说话你管得着吗?你以为我关心你,在意你的命?哼,弟弟,你还是和以前一样天真无知啊。我只是想报仇,你不过是我用来报仇的工具而已。”

%54
“我可以成为任何人的工具,但绝不会成为你的!”方正的回答斩钉截铁。

%55
方源假意哈哈大笑:“你早已身不由己,就算不愿意,又能如何?你参加比武大会以来,这一路晋升,杀了不少毛民蛊师,都是在为我复仇。今晚你更要听从我的指点,你才能炼蛊成功。你也必须接受我的指点,否则的话,你明天必死无疑!”

%56
方正脸色铁青,却没有反驳。

%57
他双拳捏紧,陷入死一般的沉默之中。

%58
足足一个时辰之后,他开始重新炼蛊。

%59
方源慢条斯理,出言指点教导。方正一声不吭,但却照着方源的指点行动,没有丝毫违背。

%60
终于,当远方的夜空渐亮,凌晨时分,方正成功地炼制出了吞江蟾!

%61
云盖大陆。

%62
方源吐出一口浊气。

%63
眼前的内景雨,终于被他处理妥当,仿佛结冰一样,化为一块几乎完全透明的青玉。

%64
但事实上,将仙材处理成这个样子,远远不是结冰这么简单的事情。

%65
这当中包含的步骤,多达五十多步,运用了超过二十种炼蛊手法,十三个凡级炼道杀招。并且,还要求炼蛊的蛊仙在半盏茶的时间内全部达成。

%66
方源炼废了三盆之后,终于在第四盆内景雨上,有所成果。

%67
这时,一股假意,飞到他的面前。

%68
方源双眼神光一扫,确认无害之后,便将假意收入脑海。

%69
很快,他的嘴角,泛出一丝微不可察的笑意。

%70
“方正为了求生,只有按照我的意志行事。他虽然对我极为厌恶憎恨,但是为了活下去,他只有这么做。”

%71
“每一次他依靠我的指点生存下来,他就会感到屈辱和愤怒。但这种事情,只要第一次发生,接下来就顺理成章。然后次数增多,他会渐渐麻木,最后甚至会习惯我的存在,依赖我的帮助。那个时候,就是炼制血神子的合适时机了。”

%72
“但他现在这个心态,还远远不合格,还需要更多的调教。”

%73
ps:今晚微信公众号上讲落魄谷,欢迎大家关注蛊真人微信公众号!

\end{this_body}

