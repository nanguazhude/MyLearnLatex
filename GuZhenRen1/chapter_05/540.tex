\newsection{炼道准无上!}    %第五百四十二节:炼道准无上!

\begin{this_body}

战斗也要分和谁打。<strong>八零电子书HtTp://Www.80txt.COM/</strong>

方源若是战胜那些六转、七转蛊仙,没有人会为此惊讶。

但对手是历史上有名的八转强者雷鬼真君,那就不是一种概念了。

其实这场战斗,方源狼狈得很,但偏偏就是这一战,真正确立了他天下七转第一人的名头,将凤九歌压在脚下。

可以说,这是方源名震天下的一战!

仅此一战,即便是那些八转的蛊仙,也要真正重视方源这位七转的怪胎了。

当然,这一切都是方源的算计。

虽然这场战斗是方源失败,主动撤离,但并非没有收益。

一场战斗的胜败,本质上其实并不重要,重要的是胜败之后代表的损失和收获。

方源此战虽败,但收获不小。三根胸骨的价值,并不在于它们本身,而是上面蕴含的意义。

方源将这个战利品运用到了极致,五域蛊仙界都为此轰动,且不说楚度的复杂情绪,武庸的铁青脸色,陆畏因的叹息。

单所方源回到琅琊福地后,就立即感受到了很大的不同。

这些异人蛊仙,包括琅琊地灵在内,面对他的态度发生了非常大的变化,真正的敬畏有加。

尤其是在天庭虎视眈眈的局面下,方源俨然成了这群人的主心骨。

毕竟太古石龙只是太古荒兽,上古战阵天婆梭罗虽有八转战力,但仍旧抵不上一个货真价实的八转蛊仙。

而现在,方源就将这样的八转蛊仙痛揍一番,尤其是雷鬼真君还是公认的八转中的强者!

这个时候,谁还管方源的七转修为?

货真价实的战绩就摆在眼前呢!

正因如此,当方源回来琅琊福地的时候,这些人就眼巴巴地过来迎接。

方源态度却稍显冷淡,只是和琅琊地灵、冰媛等人说了一些话,对雪儿点了点头。雪儿顿时娇羞地低下头去,雪人蛊仙们因此笑逐颜开。

异人蛊仙们又提议,要摆下酒宴,为方源接风洗尘,被方源毫不犹豫地推辞了。

“眼下局势不妙,风头正紧,我们还是要抓紧时间,渡过此次危机,才是当务之急。无弹窗,最喜欢这种网站了,一定要好评]”方源平平淡淡的一句话,又引起异人蛊仙们一片赞同附和之声。

方源的地位,真的不同了。

这些异人蛊仙,真正将他当做八转蛊仙对待。之前和方源平等相处,现在却有一些垂头耸肩,低声下气的态势。

一来是方源战力彪悍,能痛揍雷鬼真君,二来大敌当前,他们也是有求于方源。

“诸位都散了吧,我要和大长老私下详谈。”方源毫不客气,挥退群仙。

方源又悄悄对琅琊地灵传音:“我先是琅琊派的太上长老,其次才是联盟一员啊。”

这句话情真意切,立即把琅琊地灵听得心情愉悦起来,心想方源没有忘本,还是我们自己人呐。之前因为方源强行说服他,索要炼道真意的郁闷,也不由地削减了很多。

群仙自然不会不识趣,纷纷散去。

琅琊地灵一把抓住方源的胳膊,倏地消失在原地,下一刻他们二人就出现在第一云城之中。

“借助你的道可道仙蛊,我们发现福地中果真是种了他们的手段。要迁徙琅琊福地,首先要斩除这些不利的道痕呐。”琅琊地灵忧心忡忡。

方源顿时有些奇怪:“凭借太上大长老您的炼道造诣,炼化了这些道痕,乃是最擅长的本事吧?”

琅琊地灵苦恼:“天庭果然不愧是天庭,这次的手段极其特殊,我已几乎用尽法门,却是对这些道痕无能为力。”

方源眼中精芒一闪即逝。

天庭方面,果然是准备充分,底蕴深厚至极。既然他们能种下这个手段,就不担心被发现,更不担心被破解。或许紫薇仙子早已经将琅琊地灵的本事,也计算在内了。

“不知道我的洁身自好杀招,能否有效?”方源对这个手段颇有自信,毕竟曾经的不灭星标,就是依靠此招被方源顺利破解。

但此刻,方源却不想立即拿出来试验。

他对琅琊地灵道:“既是如此,那我还是干脆直说吧。还请大长老您先将炼道真意为我取出,我成就炼道准无上大宗师后,必定能解决这个麻烦。”

琅琊地灵神色顿时一变,这股炼道真意乃是他心目中的至宝,现在要取出来,还是要交给方源这个人族蛊仙,当然是心情别扭,并且极其不舍的。

不过,他既然已经是之前答应了方源,并且严峻的局势就摆在眼前,就算心里再是不舍,此刻也只得点头。

于是片刻之后,方源终于见到了来源于长毛老祖的炼道真意。

真意也是意志的一种,但它非常特别。

方源也是第一次亲眼目睹真意的存在,只见这股真意就是长毛老祖盘坐的模样,但它周身上下熠熠生辉,晶莹剔透,没有肉体凡胎,仿佛纯粹是由钻石铸就。

绝大多数的意志,都能自行流转,不断消耗自身,进行思考等活动。但真意完全是静止的,好像雕塑,一动不动。

不过也正是因为这个特性,这股真意才会长存至今。若是如同其他意志那样,可以自行流转,不断思考,那早就消耗光了。

“你……拿了去吧。”琅琊地灵神情复杂至极,深深凝望长毛真意好一会儿,这才苦叹一声,对方源道。

方源点点头,迈步上前,双眼一瞪,顿时凭空一股吸摄之力,将这股长毛真意吸摄到了自家脑海当中。

长毛真意进入方源脑海,顿时暴射出万丈的光辉。

钻石般的光辉照耀方源的脑海,形成一片堂皇光明。

方源当即盘坐下来,紧闭双眼,满脸肃穆之色。

他开始调动许多智道手段,顿时脑海中掀起惊涛骇浪,无数的念头,种种意志,都扑到长毛真意上去,不断地碰撞,不断地汲取、学习和领悟。

一切都进行得非常顺利,数个时辰之后,方源就将这股长毛真意全数消化,转化为自身积累。

这些都是长毛老祖对于炼道的理解,如今全数成了方源的修行资粮。

方源炼道境界,一直都比较普通,如今却是飞跃而上,一举成为准无上大宗师!

“这简直是一步登天啊!哪怕是梦境显现,五域乱战,我探索梦境,也没有这样的成效!”就连方源自己都感慨万千,这一次他的收获真的是极其巨大!

虽然是彻底消化了炼道真意,但方源还是盘坐在地,一动不动,全身心地感悟着自己的这番提升。

炼道!

这是一个特殊的流派。

修行这个流派的蛊仙,最擅长的就是炼蛊。

蛊道修行,从始至终,都有三元要素。那便是养蛊、用蛊、炼蛊。

如今方源在炼蛊方面,已经是天下翘楚,整个五域所有的蛊仙之中,单论对炼道的理解,几乎无人能抵得上如今的方源了。

“不过,光有境界还是不够的。我现在欠缺的是炼蛊的经验,炼道仙蛊还有炼道杀招。”方源对自己还有着非常清晰的自我认知。

“但只要我练习这些内容,进展必定是一日千里,迅猛至极。毕竟我已经是准无上大宗师!”

“没想到,偷道境界不久前才大宗师,如今炼道就后来居上,成为准无上大宗师!”

方源闭目沉思,仔细体悟着炼道境界带来的改变。

他随便翻阅任何一个蛊方,凡蛊方简直是一目了然,只需一眼,方源就洞彻其中的奥秘,再一转念,就能将这道蛊方提升到最佳层次。

再翻阅仙蛊方,仙蛊方当然比凡蛊方的内容要丰富许多,难度也暴涨。但这难不倒方源,他发现只要自己沉下心努力学习,一些难题只要稍稍想想,就能明白其中的道理。

其中一些更加复杂的八转仙蛊方,方源需要运用智道手段,不断思考。但因为有着准无上的智道境界打底,他绝不会思考错误。

有时候,他甚至不需要推算,单凭心中的这股直觉,就能炼制仙蛊,判断一个仙蛊方究竟是不是正确的。

面对一张他明白全部的仙蛊方,他能在瞬间想出大量的改良方案,越是深入思考,改良的方案就越多。

而对于残方,以往他推演起来,需要借助智慧光晕,但是现在却是难度暴降无数。当然,这方面他还做不到全知全能,许多残方的推算还需要该流派的境界。

“历史上的炼道,也不过区区三大无上大宗师。我现在有准无上的境界,别说是当今,就算是历史长河当中,单论炼道境界,也绝对是前十位了吧。”

方源的战力虽然没有变化,但是眼界已经截然不同,几乎是上升到了蛊修的最高层次。

站在这个高度,方源简直是俯瞰整个蛊修的过程。

他的心中涌动出无数的明悟。

“蛊道养用炼,其实都是一体的!”

“蛊虫的养法中,蕴藏着炼法,炼法中蕴藏着用法,用法中也都包含养法、炼法的奥妙。”

现在的方源,完全可以从一张完善正确的蛊方中,看出这只蛊虫该怎么豢养,有什么效用威能。

触类旁通之下,他也能从养法、用法中,逆推出蛊虫该怎么炼制。

“等一等,我明白了!其实任何一记杀招,不管是凡道杀招,还是仙道杀招,都算得上是一张炼蛊残方。”

方源忽然身躯一震,领悟到了炼道中的至理!(未完待续。)<!--80txt.com-ouoou-->

\end{this_body}

