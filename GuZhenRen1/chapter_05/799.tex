\newsection{天庭的正义}    %第八百零二节:天庭的正义

\begin{this_body}

%1
血气非常容易获得!斩杀一群凡兽,用血道手段进行抽取,就能收获一大批。在仙窍中豢养荒兽,就有源源不断的仙材血气。

%2
正因为血道资源很容易就能获取,血道才显得可怕,最终被五域正道联合禁止。

%3
换做其他仙材,方正或许会求助门派,但只是血气而已,而且方正本身需求量的也不多。

%4
对于方正而言,获得自己需要的血气,不过是过程麻烦一些,时间浪费一些罢了。

%5
想到这里,方正便向樊西流告知自己的计划,他身份特殊,不能随意走出飞鹤山。

%6
樊西流知道之后,便交给方正一个任务:“正好!西北方有一头金山牛作乱,你去解决这件事情。剿杀它或者俘虏它,若是能活捉,门派贡献将多出一倍。”

%7
方正心中一凛,金山牛可是一头荒兽,他还从未和荒兽作战过。

%8
反应过来后,他的心中便有些兴奋,当即答应下来。

%9
方正早已经不是毛头小伙,在琅琊福地经过不少磨砺,寻思一下,心中已有较量。

%10
他并没有急着去找金山牛麻烦,而是先去往仙鹤门中的信楼。

%11
白鹤信楼乃是一座仙蛊屋,高有九层,立足在飞鹤山的一处偏僻山腰。

%12
那里云深雾绕,草木葱茏。

%13
方正驾鹤而去。

%14
他虽然专修血道,但早些年在仙鹤门学艺过,拜师天鹤上人,因而对于驾驭飞鹤也十分熟稔。

%15
在琅琊福地的时候,他参与三陆大战,没有什么机会收服飞鹤,但心中始终有着一丝飞鹤的情怀。

%16
命运变化,他被凤九歌救走,又回到了仙鹤门,并且得到天庭看重,还提携成仙。

%17
有了蛊仙修为,更成为了仙鹤门的太上长老,方正轻轻松松就调遣来一头飞鹤,充作代步工具。

%18
白鹤信楼大门前,无人守卫,但附近却有两只荒兽飞鹤栖息生活。一旦有什么意外发生,这两头荒兽飞鹤就会及时出现,充当合格的守卫者。

%19
方正驾驭飞鹤,刚刚接近白鹤信楼,脚下的飞鹤便不断颤抖,轻啼声也化作了哀鸣。

%20
方正不免心中叹息:“我这头飞鹤乃是五转异兽,最初在仙鹤门时,觉得它甚是神俊强大,如今却也不过如此。”

%21
“仙凡之差,果然是云泥之别。”

%22
方正落到白鹤信楼大门前的地面上,他脚下的飞鹤差点腿软到跌倒。

%23
方正叹息一声,将它收入仙窍。

%24
他有着太上长老的身份,白鹤信楼大门无人自开,方正进入其中。

%25
信楼中的每一层,都摆放着大量的信道凡蛊。

%26
方正要从中获取消息,就要拿自家的门派贡献来换取。

%27
信楼本身就能存储信息,但仙鹤门不做这种事情。因为信楼一毁,这些情报也就毁了。信楼是单方面的仙蛊屋,就像天庭中的仙蛊屋一样,只有信道的威能。

%28
若是完整的仙蛊屋,兼顾防御,自然就不需要两头荒兽白鹤守护了。

%29
方正的信道底蕴一穷二白,自然看不出信楼的什么奥妙。但是他却完全可以料想到,信楼对于这些蛊虫都有保存、传送,乃至毁灭的布置。

%30
白鹤信楼有着采集情报,更正消息的作用,绝大多数的情报,都能够查询得到。

%31
当然,真正的秘辛,只是在高层曝光。

%32
方正虽然是太上长老,但想要成为仙鹤门真正的高层那是不可能的。他身份特殊,始终被排挤在外。

%33
换做以前,方正涉世不深,懵懵懂懂,但如今他历练许多,看得透彻。

%34
这份透彻,并没有让他沾沾自喜,反而更多苦笑。

%35
他因此更加明白了,在青茅山时方源曾经的处境。依凭方源当时的境况,不受高层待见,要想谋求进步,只有坑蒙拐骗,强取豪夺。

%36
“事实上,仙鹤门的创建,到现在的举动,不就是这样么?他们强占资源,许多的资源点都是曾经的散仙、魔道蛊仙掌控的。”

%37
“而天庭,号称正义,也不外如是。追溯历史,他们的地盘是从异人手中抢夺而来。”

%38
“天庭,只是人族的正义而已。”方正心中想道。

%39
换做其他中洲正道蛊仙,通常不会这么想,这种想法有些大逆不道。

%40
但方正他是在毛民群落中生活过的。他虽然被排挤,但却真切感受到毛民的种种风土人情,更加了解毛民。

%41
换做其他蛊仙,怎可能在异人环伺的环境中生存、成长?

%42
独特的经历,带给方正独特的视角。而曾经的艰难困苦困苦,如今都转变成了他的财富。他的眼界虽然不高远广阔,但也独到。

%43
“只是屁股决定脑袋,天庭这种正义我还是很喜欢的。”方正苦笑一番,开始查阅资料。

%44
“荒兽金山牛,嗯,就是这个。”他攀升几层后,找到了相应的信道凡蛊。

%45
神念探伸进去,顿时一波波的信息流淌进心头。

%46
金山牛体大如山,只有雌性,喜好吞食金精。往往吞食了数百斤金精之后,便钻入山洞,陷入沉眠。

%47
“吃了睡,睡了吃?倒是好生活。”方正笑了笑。

%48
他接着又看到:金山牛全身,当属牛角价值最高。牛角每百年长一丈,坚硬弯曲,不断蔓延伸张,最终盘绕整个头部、颈部,直至延伸到腹部。当金山牛角延伸到腹部之后,尖锐的牛角尖将抵住牛肚皮。金山牛只需要将头微微一昂,就能带动牛角,从后向前,将自己的整个肚皮都割开来。而金山牛的后代,一只只的黄金小牛,就会从这道巨大的伤口中出世,蹦跳而出。

%49
方正眼眸一动,口中呢喃:“这繁衍的方式,倒是和石人有些相似。”

%50
方正继续浏览。

%51
金山牛产子之后,因为伤口巨大,痛苦不堪,发出巨大的哀嚎声,同时血流不止。

%52
新生的黄金小牛,若是选择舔舐母牛的伤口,小牛舌头上的独有道痕,会令金山母牛的伤口迅速恢复,直至痊愈。

%53
若是黄金小牛不用舌头舔舐伤口,那么母牛就会死亡。小牛吞食母牛的血液,能迅速壮大自身,直接跨越幼年的削弱时期。

%54
“有意思!”方正感叹不已。

%55
金山牛的生活方式,是吃了睡,睡了吃。这就意味着,母牛产子,黄金小牛仔救下了母牛,母牛仍旧不会照顾它们,会大量进食后,陷入沉眠当中。

%56
对于小牛仔而言,救下母牛反倒不如不救,更有利于它们存活。

%57
所以,大多数的小牛仔都是看着母牛身亡,只有少部分舔舐伤口,救下母牛。

%58
“母牛产子,往往意味着一场人间惨剧么。虽然为了生存,无可厚非,我倒是更喜欢那些拯救母牛性命的小牛仔们多一些。”

%59
方正看完这些情报,心中便有了对策。

%60
他旋即便出了信楼,对于这场门派任务的把握,已是和进楼之前完全不同了。

%61
他决定按照习性,对金山牛下手。

%62
金山牛不是喜欢吞食金精吗?

%63
那他就做出一份金精来,旁人钓鱼用鱼饵,而这份金精就是吊牛用的牛饵!

%64
方正虽然实力不俗,这段时间窝在自家仙窍中,熟悉了不少的血道杀招,已经可以和金山牛硬对硬。

%65
但能用五分力,为什么要用十分力呢?

%66
最稳妥安全的选择,就是用智慧对敌,尽量地运用到身边所有的优势,留下几分余力来应付任何意外。

%67
这是方正在战火纷飞的岁月中,锻炼出来的习惯。

%68
数天之后,方正来到目的地。

%69
他悄然进入山谷,并在金山牛活动的范围边缘,投下了一份巨大的金精。

%70
当金山牛徘徊到这里的时候,它鼻子一阵耸动,双眼金芒一闪,迅速发现了地底下深深埋藏的金精。

%71
金山牛便用自己的牛蹄,还有巨大的牛角,进行挖掘。

%72
挖得巨石滚滚,尘土飞扬,金精很快就暴露出来,然后被欢喜的金山牛三下五除二,吞食入腹。

%73
“中计了!”方正亲眼目睹此幕,心中暗喜。

%74
他耐心等待,看着吃饱了的金山牛钻入之前的山洞,又再次陷入沉睡之中。

%75
足足过了三天三夜,从山洞深处忽然传出金山牛的哀嚎声音。

%76
熟睡中的方正被叫声惊醒,反应过来后,他大喜过望:“毒性终于是扩散全身,发挥作用了。”

%77
但是他仍旧没有进入山洞,而是在洞口处耐心等候。

%78
毒发带来巨痛,让金山牛发疯发狂,在山洞中胡咬乱撞。闹得整个山峦都微微颤抖起来,周围的山林喧嚣一片,大量的生灵四散奔逃。

%79
又过了好一会儿,山洞中的可怕动静逐渐停息了下来。

%80
方正这才进入山洞,和半残的金山牛展开了激战。

%81
战斗了一小会儿,方正见金山牛被自己彻底激怒,便主动退出山洞。

%82
在这山洞中和金山牛对打,太过危险。一旦坍塌,方正纵然是六转蛊仙,也有不好的麻烦。

%83
况且方正此次来,只是为了活捉金山牛,而不是为了摧山毁林,生灵涂炭的。

%84
方正出了洞,金山牛也紧随其后。

%85
两者在山谷间再次搏杀,最终,方正依靠者血渐冷杀招,将金山牛冻住,继而活捉。

%86
不过,正当他要将这头金山牛收入仙窍时,一股强大的气势,从金山牛的身上升腾而起。

%87
与此同时,一座黄铜金鼎从金山牛的身体内,直接撞出来。

%88
“这是?!”方正大惊。

%89
下一刻,黄铜金鼎中跃出一个八转蛊仙来。

%90
“一点灵性金中藏,三十万载蕴神光。”

%91
“今朝运来重见天,奉正殉道血玄黄。”

%92
这位八转女仙缓缓而歌,落到方正眼前,对方正微笑道:“小友,看来便是你助我苏醒?”

%93
方正目瞪口呆,未料到竟有如此变化!

\end{this_body}

