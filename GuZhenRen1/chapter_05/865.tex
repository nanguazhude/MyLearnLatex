\newsection{老祖神威!}    %第八百六十八节:老祖神威!

\begin{this_body}

刹那台、今古亭、三秋黄鹤台、恒舟、鲨流撬、日月观。

顾六如、凤九歌、星野望、九灵仙姑。

天庭在光阴长河中投入了庞大军力,对红莲真传势在必得!

反观方源一方,只有一座万年斗飞车,虽是八转,在天庭这样的强大阵容面前,也显得单薄。

天庭六座仙蛊屋、四位八转蛊仙将万年斗飞车重重包围。

万年斗飞车陷入困境,方源的声音却是不急不慢,还有心情谈笑:“呵呵呵,想要占领石莲岛?既然我得不到,那你们也别想得到。”

仙道杀招——万年围猎!

从万年斗飞车上忽然传出阵阵异香,异香迅速扩散,掀起周边无数道兽吼。

一波年兽兽潮正在迅速酝酿。

上一世,方源和冰塞川联手,和天庭抢夺红莲真传,结果被凤九歌破坏,直接毁灭了石莲岛。

这一次,方源也同样这样选择,利用万年围猎发动无边兽潮,对石莲岛下手。

这个手段可以说是万年斗飞车的最强手段,但弊端也非常明显,就是年兽兽潮敌我不分,事实上更针对万年斗飞车。

上一次,天庭蛊仙着了方源的算计,以为方源能够操纵兽潮,屠戮了大量年兽,为万年斗飞车挡了不少刀。吃了血亏后,天庭蛊仙最终发现,这兽潮方源也难以掌控。

万年围猎杀招的弊端,天庭蛊仙已然掌握。

但方源这次再用,却是打在天庭的七寸上。天庭正在铺设镇河锁莲大阵,躲闪不得,只能硬捱。

顾六如冷哼一声:“贼子毒辣,果然是又用这招。”

他面色一片冷静,立即调遣人手,恒舟、鲨流撬迅速脱离,向战场之外赶去。同时九灵仙姑也跳出仙蛊屋,化为一头太古荒兽,撤离战场。

天庭占据优势,方源的万年斗飞车本来就被围困着,因此这些人的撤离无法阻止。

远离了战场之后,恒舟、鲨流撬上忽然亮起璀璨的光晕,两股近似似水流年仙蛊的气息四处扩散。

刚刚形成的兽潮,顿时被这两座仙蛊屋吸引,分出八成冲向这两座仙蛊屋。

而剩下两成,则被九灵仙姑所化的太古荒兽截住,纳为己用。

“果然是她!”

“九灵仙姑么……中洲历史上,奴道、变化道双修的八转大能。”

万年斗飞车中已有蛊仙辨认出九灵仙姑的跟脚。

此刻,方源的本体仍旧在东海,车中只是他的一团意志。意志思考着,和上一世的记忆做出对比。

九灵仙姑上一世并没有苏醒,但这一世却醒来了,并且还是提前苏醒。

这是一个截然不同的变化。

当她变化成太古年兽的时候,光阴长河对她而言,再不是阻碍。

天庭虽然没有正面破解了万年围猎,但却想到了应付的方法。

他们分析出了万年围猎杀招的原理,从这方面着手,巧妙化解了还未彻底成形的年兽兽潮。

接下来,就算有年兽赶过来,也只是陆陆续续零星的几只,根本不成气候。

这就是天庭的底蕴。

也是方源一直忍耐,隐藏手段的缘由。

一旦他的手段暴露出来,天庭方面就能迅速反应,积极推算,从而在下一次战斗中,做出针对。

破晓剑如此,激流勇进如此,万年围猎同样如此。

能够改变局势的万年围猎杀招,还未生效,就被天庭破解。

剩余的四座仙蛊屋死死围住万年斗飞车,不断发起猛攻。

“方源,你还有什么手段尽管使出来罢。”顾六如冷笑。

万年斗飞车中,方源意志沉默,没有回应。

战斗在继续,万年斗飞车陷入下风,难以翻盘。尽管它屡屡尝试突围,但都没有成功。

天庭这边的四座仙蛊屋,由刹那台负责正面,同为八转级别的仙蛊屋,又有顾六如这位宙道大能亲自坐镇,刹那台完全不虚万年斗飞车。

同时,另外三座仙蛊屋积极辅佐,和刹那台的配合不仅是默契那么简单,而是十分精妙。

显然,这些仙蛊屋中的蛊仙这段时间,并未闲着,而是积极演练,战斗力因此上涨许多。

“方源已成困兽之斗,不足为患了。”

“还是要小心,和这个魔头战斗,千万不能大意。”

“我方的镇河锁莲大阵已经接近完成,方源还能有什么手段阻止我们?”

天庭一方,蛊仙士气飙升。

许多人仿佛已经看到方源授首的画面。

但就在这时,石莲岛上陡然升腾起一股磅礴气势。

金色的华光陡然绽放,一个灰石巨人缓缓站起。

这头巨人高大无比,浑身肌肉贲发,毛发浓密,双眼闪烁着凶厉的兽芒。

“这是什么怪物?全是宙道的气息!”

“我们还未真正探索石莲岛,就触发了石莲岛上的守卫了吗?”

“糟糕,镇河锁莲大阵还未铺设完成呐。”

天庭一方,蛊仙面色剧变。

巨人猛地咆哮,竟发出虎啸龙吟之声,声波震荡开来,辐射四面八方,激起岛边惊涛骇浪。

镇河锁莲大阵猛烈颤抖,铺设大阵的蛊仙吐血无数。

“快,稳住大阵!”

“纸笼阵被吼破了,赶紧修补。”

“四班,四班的蛊仙速速支援!”

布阵的蛊仙们一阵手忙脚乱。

灰石巨人吼完,双臂抖了一抖,上臂、前臂的肌肉像是吹鼓的气球,猛地膨胀三倍。

他的额头生出漆黑厚重的牛角,微微低头,双臂汇拢在胸前,双手张开,掌心相对,无数的气流呼啸席卷,在掌心处交汇,形成澎湃的气流漩涡。

顾六如见此,眼眸中闪过一抹骇然之色。

他辨认出来:这气流只是表象,实则是宙道道痕!

越来越多的宙道道痕,被巨人抽取,汇聚起来,形成一个巨大的银色光球。

光球越发巨大,巨人牛哞了一声,陡然用力,双臂一振,将银色巨球猛力投掷出去。

银色巨球宛若流星,暴射而出,沿途掀起无边的风浪。

中洲正道的蛊仙们无不瞳孔缩起,为这个银色巨球的威能感到震惊。

“快,让我们挡住它!”顾六如咬牙,硬着头皮,操纵刹那台挡在银色巨球前进的方向上。

轰!

下一刻,一声巨响,银色巨球狠狠地砸在刹那台上,恐怖的威能爆发,刹那台像是跌入了飓风当中,极力抵挡,仍旧不断飞退。

“挡住了。”刹那台终于停住,顾六如吐出一口浊气,刹那台的整个正面破碎不堪,充满了裂痕,大量的凡蛊阵亡,更有仙蛊损伤。

刹那台中的蛊仙已经忙乱开来,再无之前围困方源的惬意,全力修补这座仙蛊屋。

银色光球轰炸的余波,掀起剧烈的风浪,仍旧冲撞上镇河锁莲大阵,震死大量蛊虫,令布阵的蛊仙忙上加忙。

灰石巨人见一击无功,便又开始凝聚银色光球。

顾六如见此,连忙咬牙,催动刹那台直扑而下。

但刹那台刚刚接近石莲岛,石莲岛就绽射华光,一股无形巨力将刹那台挡住。

趁着这个功夫,灰石巨人的双掌中再次凝聚出了一记银色光球。

光球暴射而出,顾六如无奈之下,只得咬牙,再度催动刹那台拦截光球。

轰!

又是一声巨响,猛烈的爆炸中,刹那台的正面已是面目前非,大量的砖瓦飞溅出去,在空中还原成一只只蛊虫的碎尸。

灰石巨人将自己的攻击又被挡住,暴跳如雷,怒吼一声,又开始凝聚银色光球。

顾六如紧皱眉头:“这头巨人明显是宙道手段,似曾相识!没想到红莲魔尊留了这么一手。”

“然而就算知晓它的跟脚,短时间之内,我也只有拦截它的攻势,牺牲刹那台来护住镇河锁莲大阵!幸好我方有一座刹那台。”

顾六如心中有些庆幸。

刹那台的招牌手段,就是挪移方位,能在刹那间从一个位置,转移到另外一个位置。

这并非是宇道手段,而是宙道,本质上并非瞬移,而是将刹那台转移耗费的时间无限压缩,只有一个刹那。若是距离更远,就是几个刹那。

顾六如极力想要保住大阵,他的目的非常明确。只要大阵一起,就能隔绝内外,到时候石莲岛跑不了,岛上红莲魔尊的种种布置,也会随之削弱,遭受压制。

对内如此,对外更是一层强大保护。

到那时,就算方源想要进来,有这座大阵,还有数座仙蛊屋协防,就非常困难了。

方源显然也明白这个道理,万年斗飞车左冲右突,剩下的几座仙蛊屋只有七转层次,围困它颇为艰难。就像是一张简陋的渔网,要网住一条鲨鱼。

天庭每况愈下,见势不妙,顾六如冷哼一声,从刹那台中奔出,加入对万年斗飞车的包围。

他是宙道大能,在光阴长河中如鱼得水,有他出手,天庭再次将局势稳固下来。

“坚持,坚持住就是胜利!”

“只要撑过这段时间,等到大阵铺设完成,就是方源的死期。”

顾六如总领全局,一边激斗万年斗飞车,一边还不忘振奋士气。

终于,镇河锁莲大阵铺设成功,巨大的力量形成圆罩,不仅包裹住石莲岛,同样还将万年斗飞车囊括进来。

大阵发动,灰石巨人闷哼一声,受到强烈的压制。

万年围猎杀招也受到禁锢,异香根本传不出去,无法形成兽潮。

“先解决了方源,再探石莲岛!”顾六如低啸。

没有了万年围猎杀招的威胁,恒舟、鲨流撬以及九灵仙姑及时回转,加入对方源的围困。

“方源,这里就是你的葬身之地!”顾六如双眼精芒爆闪,面容冷峻。

“一代天魔就要陨落于此吗?”凤九歌眯起双眼,心中荡漾起一股有关命运歌的灵感。

“为免夜长梦多,杀了他!”星野望驾驭鲨流撬,杀意弥漫,奔袭而来。

九灵仙姑沉默,动作一点都不慢,绕到万年斗飞车身后,气势勃发,酝酿杀招。

但就在这时,万年斗飞车中传出方源的冷笑:“天庭诸位,尔等皆成瓮中之鳖,还不自知吗?”

话音未落,石莲岛陡然暴射强光,岛旁的河面上升腾起数十根庞大水柱。

强光、水柱强行沟通镇河锁莲大阵,阵眼内的中洲蛊仙无不神情剧变。有人骇人大叫:“怎么回事!我们失去了对大阵的控制!!”

如此惊变,非同小可。

天庭四大八转勃然变色,难以置信,但又不得不信:这一切竟都是方源铺设的陷阱!

“不枉费我牺牲一座石莲岛来构陷,天庭诸位,这里不是我的命丧之地,而是你们的。”方源话语刚落,石莲岛上那头灰石巨人背生鸡翅,头生华羽,口变鸡嘴,手变虎爪,飞腾而出,闯入包围圈,接应万年斗飞车。

顾六如脑中灵光乍现,终于认出灰石巨人的跟脚,惊怒交加:“这是十二生肖上古战阵!没想到竟是方源的手段。”

“这一切都是方源的算计!”

“他已经得到了红莲真传,能够操纵石莲岛!”

“这个十二生肖上古战阵恐怕就是真传的内容,他宙道底蕴暴涨,竟能反过来制住我们的大阵。”

“此地不可久留,速撤!”

天庭四位八转迅速商议,决定撤离。

然而尴尬的是,他们费尽全力铺设的镇河锁莲大阵,此时却反过来成了阻挡他们的障碍。

“现在想走?晚了!”万年斗飞车中方源笑声传出,破晓剑群飚射,罩住三秋黄鹤台、鲨流撬。

灰石巨人怒吼一声,扑向今古亭。

今古亭周围掀起滔天浪潮,灰石巨人不管不顾,继续冲锋,抡起拳头,狠狠砸去。

灰石巨人太过高大,一个拳头比今古亭还要大出几分。

拳头还未击中,就有虎啸龙吟之声响彻战场。

拳势之强,简直是要毁天灭地!

凤九歌眯起双眼,其余蛊仙骇然不已,面对这一拳,只感觉天地坍塌,任何的躲闪都是如此苍白无力。

躲不开!

轰!!

一声炸响,整个今古亭被灰石巨人一拳击溃。

里面的蛊仙死伤惨重,凤九歌狼狈逃窜而出。

刹那台!

关键时刻,刹那台仿佛瞬移而来,接住了凤九歌。

面对近在咫尺的刹那台,灰石巨人却是看也不看,扭头转身,奔向日月观。

日月观乃是灵缘斋的宙道仙蛊屋,本身层次就稍差一筹,灰石巨人逼近,日月观无法及时转移。

巨人的速度太快了,在光阴长河中,又受到石莲岛的力量加持,战力简直是骇人听闻!

轰!!

第二声巨响,日月观炸裂,蛊仙十之八九都身死道消。

“不!!”顾六如狠狠咬牙,气得双眼通红,恨得浑身发抖。

天庭蛊仙再无之前的气势,力求撤退。

灰石巨人的战力,让他们感到了绝望!

东海,气海上空。

蛊仙们皆是双眼放光,看着气海老祖,又看黑魂海域之主张阴。

张阴身为散修,却高达八转,在东海蛊仙界谁人不知。

众仙有些没有想到,他竟然会当中挑战气海老祖。但仔细一想,也不奇怪。毕竟当初龙龙和气海老祖之争,只有宋启元、沈从声两人亲眼目睹,而这两人都是正道蛊仙。

众目睽睽之下,气海老祖淡淡一笑:“蛊仙张阴颇有胆色,但能接得住我这一招否?”

张阴傲然一笑:“老祖未免太过瞧不起张某,别说一招……呃。”

他猛地顿住,其余的蛊仙无不面色剧变。

刹那间,他们感觉整个天地开始嗡鸣,无边的气流从四面八方响应。

万里方圆,皆压力暴涨。

群仙瞳眸猛缩,冷汗涔涔,直感觉压力无边无际,自己就像是琥珀中的虫子,卑微不已。

“这是什么杀招?”

“这就是气海老祖的手段吗?”

“我们只是被波及而已,处于中心的张阴,又该是何等压力?”

群仙望去,只见张阴面色涨红,在半空中一动不动,双眼鼓瞪,奋力挣扎,竟只能颤抖,根本无法挣脱。

群仙心头一阵狂震。

隆隆之音,从头顶响起。

东海众仙连忙仰望,一个个脸色皆变苍白。

只见一记大手,完全是气流组成,苍白透明,绵延千里,只手遮天!

巨手缓缓压迫下来,东海群仙无不感到窒息,就像是一座座山脉要砸在他们的头上。

“这是何等的威势!”

“不是亲眼所见,万难料想啊。”

“这恐怕已能匹敌九转了吧?”

“气海老祖!!”

席上,很多蛊仙都不自觉地站起身来,身躯微颤,惊惶不已。

张阴奋力嘶吼,全身阴芒绽射,又有微弱的雷光响应。但就是挣脱不得,被牢牢定在半空。

这一幕,让宋启元等八转蛊仙看了,无不心惊肉跳。

“看样子,张阴已经拼尽全力,竟然挣脱不了束缚?”

“难道张阴要死在这里?!”

“气海老祖的战力,竟然如此恐怖。难道他想杀鸡儆猴,利用张阴的性命立威东海?”

难以想象,一位八转蛊仙在气海老祖手下,居然毫无还手之力!

但事实就摆在眼前,宋启元等东海蛊仙不得不信。

气流巨手临近张阴,忽然崩散,穿过张阴后,贴在气海上。

没有任何惊天动地的炸响,气流巨手接触到气海后,迅速崩解,汇入气海当中。

张阴重获自由,劫后余生,衣服已被冷汗打湿,一脸余悸,双眼瞪大,只顾着喘粗气。

“你们看气海!”宋亦诗一声低呼。

众仙看去,只见原本浩荡险恶,漩涡亿万的气海,竟然被抹平了,仿佛成了一面镜子!

东海群仙心头剧震,许多人张开嘴,说不出话来。

“老祖神威,晚辈服了。”张阴吐出一口浊气,颓然行礼,“多谢老祖手下留情。”

气海老祖仍旧是端坐主位,眼皮似合非合,云淡风轻一拂袖:“入座吧。”

“谢老祖。”张阴在一片死寂中入了座。

东海群仙还有许多人站着,坐着的也好不到哪里去,仿佛雕塑。

气海老祖一招击败张阴,把他们彻底震慑。

“老祖神威!”半晌,宋启元反应过来,站起身来,恭敬行礼。

“老祖神威!”沈从声也站起身来行礼。

“老祖神威。老祖神威。”无数东海蛊仙行礼,无不拜服。

方源微笑,淡淡地道:“请入座,再开宴。”

ps:祝大家新年好!今天这一章有5000多字,码的时候忘记了时间,码出来后发现已经迟了,抱歉。希望大家喜欢这一章。

------------

\end{this_body}

