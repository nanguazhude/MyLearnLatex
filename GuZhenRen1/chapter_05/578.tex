\newsection{遗忘}    %第五百八十节:遗忘

\begin{this_body}

梦境终于到了第四幕。

“我已杀了厉薪。”方源道。

老者站在在方源的面前,微微点头:“看来光阴飞刃,你已经练的颇有火候。不过你炼成的只是六转层次的光阴飞刃罢了。接下来我要传授你七转杀招光阴飞刃,你听仔细了。”

方源立即沉下心来,仔细记下。

七转杀招,比之前的六转,还要繁复多变数倍!

方源光是听闻这些步骤,就已经脑袋隐隐作疼。这是他见过的,有史以来最复杂的杀招,没有之一。

老者传授完毕,临走之前,给方源留下一句话:“等到你练成这七转杀招,再来找我。”

梦境中便又只剩下方源一人。

“看来这第四幕梦境,仍旧是要我练出杀招来。”方源心中了然。

明白这点,他便埋头苦干,在梦境中勤练不辍。

杀招虽然复杂,难度极大,但方源耐心十足。

他的性情早已经被前世今生,颠沛流离的命运,打造成一块黝黑的钢铁,坚定顽固。

这些小小的挫折,绝不会让他产生任何的气馁情绪,更不会放弃。

就在方源探索梦境,在梦境中苦练杀招的时候,至尊仙窍当中,方源的宙道分身盘坐在智慧蛊面前,利用智慧光晕,也在不断地推算着。

须臾,他缓缓睁开双眼,口中呢喃:“大阵初步推算完毕,接下来,就是光阴飞刃杀招了。”

说起来,有点奇怪。

这光阴飞刃杀招,威能不俗,十分优异,但为何却在幽魂真传中,并无任何记载呢?

要知道,这片宙道梦境可是源自魔尊幽魂,按照常理推断,幽魂真传中应当记载着光阴飞刃杀招才是。

但事实上,方源继承的幽魂真传中,却是没有此项内容。

“还没有推算出方源的位置?”夏家太上大长老脸色阴沉如水。

在他的面前,一群南疆蛊仙陷入沉默,各自的神情也不好看。

因为之前方源捣毁掠影地沟,抢走了全部的梦境,南疆正道便联起手来,各自出力出人,组成了一批专门追捕方源的队伍。

这个队伍的首领,就是夏家太上大长老夏槎。

夏槎白发苍苍,脸上皱纹层层叠叠,虽是老妪,但气度雍容,不怒自威,此刻目光扫视,震慑全场。

南疆群仙纷纷低下头来。

这些人中谁不想找到方源的下落?奈何方源拥有阎帝杀招护身,就算是当今天庭智道大能紫薇仙子,也暂且无能为力。更何况这些草草组建起来的南疆蛊仙呢?

当然,南疆正道的底蕴十分雄厚,绝不容小觑。这些人并非没有能力,推算出方源的下落来。只需要更多一点线索,比如方源再次出现,袭击某个资源点。

只要这些线索一多,依照这些人的各种手段,相互合作,勘破阎帝杀招还是有许多可能。

阎帝杀招虽强,但总是老样子,被不断推算的话,也会被破解的。

夏槎也知道这一点:“但这该死的方源贼子,他抢劫了掠影地沟之后,就再也没有出现!”

夏槎心中也颇有压力。

方源一日不除,她担任此次首领,威望就在不断地下滑。严重的话,甚至连带整个夏家也要因此遭受影响。

“方源这魔头,龟缩不出,定然是在探索梦境,捞取好处!我们必须尽快地找到他,除掉他,否则时间一长,他的实力又要突飞猛进了。”夏槎幽幽地道,“诸位有什么底牌,尽管启用,我夏槎以个人名义,给予你们充分的补偿。”

南疆群仙你望我,我望你,都没有动弹。

夏槎心中越发失望,她目光扫视一圈,看向其中一人:“不知刘浩仙友有什么妙法么?”

刘浩微微一愣。

他身材欣长,一身青衫,山羊胡须,黄色面皮,乃是武家代表。

武家蛊仙因为之前的动乱,牺牲许多,因此在最近这段时间里大肆招纳外姓蛊仙,成为客卿太上家老。刘浩就是其中一员。

但实际上,他的真正身份却是中洲蛊仙,是紫薇仙子秘密派遣而来,在武庸的安排下,安插在南疆正道当中,充当内间的人。

夏槎对其他蛊仙都很熟悉,但惟独对刘浩不太了解。刘浩表面上的身份,乃是南疆隐修,声名不显。

刘浩反应过来,苦笑摇头:“惭愧!此事我无能为力。主要还是我们手中掌握的线索太少,尤其是和方源紧密相连的关键线索。依在下看来,其实也并非我等本事不济,实在是巧妇难为无米之炊啊。”

“是啊是啊。”

“刘浩仙友说的极是!”

刘浩的话,立即引发了周围人的一阵共鸣。

夏槎心中彻底失望,她倒不怀疑刘浩说谎,毕竟武家和方源的仇恨摆在那里呢。

“这下该如何是好?难道就这样等下去吗?若是方源贼子始终不现身,难道我们就什么都不做,一直在守候他出现?”

夏槎想到这里,头都开始隐隐发痛。

但就在这时,一个陌生的声音忽然传入众仙耳畔:“此事,我可以出一些力气。”

“什么人!?”群仙大惊失色。就连夏槎都站起身来,双目暴**芒,瞪视来人。

一位陌生的蛊仙,虎背熊腰,头戴斗笠,不知何时,竟混入到仙阵当中。若非他出声,南疆群仙还毫无察觉!

“你是何人?!”南疆群仙如临大敌,气氛顿时紧张无比。

“等等,你是……”铁家代表忽然神情一变,拦下群仙。

“不错,我正是陆畏因。”来人淡淡地笑了一声。

梦境。

老者缓缓点头,面色欣慰无比,语气中又带着一丝震惊:“七转层次的光阴飞刃,你竟也练成了!”

方源面色淡淡,拱手道:“幸不辱命。”

老者点点头:“看来你在此招上颇有天赋,很好很好,我族复仇大计终于是见到了一丝希望。不过,我还是有几点,要告诫于你。”

“请指教。”

“第一。”老人竖起右手食指,“六转光阴飞刃能杀七转,但七转光阴飞刃,却不能杀死八转蛊仙。七转和八转之间的差距太大了。所以你就算炼成光阴飞刃,也不要轻易地去寻仇敌。你现在不过六转,找上门去,就是送死。”

“第二。”老人竖起右手中指,“你修为还只是六转,勉强催用七转光阴飞刃,十分勉强。今后的重点,还是提升修为,达到七转,乃至八转!当你修为上升之后,你才能修行八转层次的光阴飞刃。有了八转杀招,你才具备复仇的能力。在此之前,你需要忍耐再忍耐!”

“第三。”老人竖起右手无名指,“光阴飞刃杀招极端,威能猛烈,有着非同一般的弊端。层次越高,弊端就越强。你现在应当也感觉到一些了吧?没错,光阴飞刃会消除你的相关记忆。你使用此招次数越多,你忘记的就越多。迟早有一天,你会连我,连仇人,甚至连这记杀招本身都会忘记。所以,在忘记之前,你定要克制自己,绝不能滥用此招啊。”

\end{this_body}

