\newsection{方源渡劫(下)}    %第三十四节:方源渡劫(下)

\begin{this_body}

%1
d

%2
三头铁冠鹰,齐齐向方源扑杀而来。

%3
剑痕索命!

%4
方源心中低呼,却不想关键时刻,却掉了链子。

%5
杀招催动失败,反噬之下,让他口角溢出一丝血迹。

%6
毕竟,接触这个剑道杀招的时间还是太短了点。紧急情况下,心神不集中,失误的概率就会节节攀高。

%7
一声鹰啼,刺破耳膜。

%8
左前方的铁冠鹰,首先扑来。

%9
方源连忙躲闪,后背闪烁虚影,轻虚蝠翼出现,勉强躲过雄鹰的扑击。

%10
但罩在他身上的流风气罩,却是破了,只是擦了一个边,就被铁冠鹰的鹰爪撕扯了开来。

%11
方源眯起双眼,决心再用剑痕索命,这次成了!

%12
剑光锋锐至极,轻而易举地洞穿铁冠鹰的后背,继续朝前方射出。

%13
忽然脑后寒风乍起,方源心脏漏跳一拍,根本不及多想,他直接催动剑遁仙蛊。

%14
嗖的一下,他身形似箭,带着刺破苍穹之势,猛地拔升而起,险而又险地让袭击他后背的铁冠鹰扑空。

%15
暂时脱离险境,方源停下剑遁。

%16
剑遁仙蛊直线飞行,速度极快,方源一下子就攀升到高空,距离脚下方的铁冠鹰十里有余。

%17
吐出一口浊气,方源的心跳这才平稳下来。

%18
稳住阵脚后,他立即用心念,召唤飞剑仙蛊。

%19
刚刚他来不及召回飞剑,此时这只仙蛊洞穿铁冠鹰之后,已经射出老远。要召回到方源手中,还需十几个呼吸的功夫。

%20
而那头被洞穿的铁冠鹰,发出一声哀鸣,直接往地面坠落下去。

%21
还未摔到地面上,它的整个身体就已经化为了一大蓬的冰雪,四下洒落。

%22
剩下的两头铁冠鹰猛力振翅,一左一右,夹裹风雪之势。再度向方源杀来。

%23
方源不愿硬碰硬,便利用剑遁,不断后撤。

%24
他现在的肉身虽然优异无比,但本身却不如仙僵那会儿刚硬。

%25
剑道更是以攻强守弱著称。

%26
剑遁仙蛊令方源身形似箭。划破长空,瞬间就和追杀的铁冠鹰拉开距离,效果立竿见影。

%27
就连飞剑仙蛊都没有方源这般快速!

%28
方源在高空绕了一个圈,终于迎上飞剑仙蛊,将其收回。

%29
方源把飞剑仙蛊握在手心之中。灌注仙元,同时默默调动身上的其他蛊虫。相继催起了上百只辅助蛊虫之后,一股玄光产生,投入到飞剑仙蛊的身上。

%30
正巧那两只铁冠鹰追近,一前一后。

%31
方源五指张开,只感觉手心随之一震,飞剑剑蛊暴射而出,化作惊鸿剑光,近似有洞穿宇宙的威能!

%32
飞剑仙蛊、铁冠鹰双方一来一去,相向而行。几乎眨眼之间。距离就缩减成零,飞剑仙蛊刺中前一只铁冠鹰后,余势不减,又洞穿后一只。

%33
一箭双雕!

%34
前头那只铁冠鹰,飞行一阵,忽然全身崩解。

%35
后面那一头,却安然无恙,继续飞扑过来,仿佛方源和它有不共戴天的仇恨。

%36
方源正要收回飞剑仙蛊,忽然心头一震!

%37
冥冥当中。一股真意灌输到他的心头,方源眉头一挑,难掩惊喜之色。

%38
“这是狂蛮真意灌体!看来我原先的设想并没有失败,而是成功了。”

%39
方源故意选址在北部大冰原中渡劫。就是为了狂蛮真意。

%40
因为历史原因,力道、变化道的蛊师若在冰原中升仙渡劫,就会得到狂蛮真意的灌体,从而得到境界拔升的奇遇。

%41
但这个待遇,只有蛊师升仙时的天灾地劫才有。蛊师成仙之后,经历的灾劫。是在仙窍中发生,并不会引发真意灌体的待遇。

%42
但方源获悉了仙劫锻窍杀招之后,就猜测利用这个杀招,是否可以得到真意灌体呢?

%43
先前他点杀荒兽雪怪,屠杀小雪怪的时候,并没有得到真意灌体,还以为失败了。刚刚杀死第一头铁冠鹰时,似乎是有真意灌体,但情势凶险,他并未感受切实。

%44
现在,真意灌体终于被他清清楚楚的感觉到了。

%45
一股明悟,在他心头泛起。

%46
他仿佛变成过来一头铁冠鹰,从小到大,从出生到第一次飞翔,再到搏击苍穹。他感受到双翼拍击大气,反馈而来的力量。他握紧双手,两只鹰爪宛若钢钳。再举目四望,千里之外,洞若观火。

%47
力道、变化道的境界,都大幅度上升。

%48
还有飞行的境界,也在略微上涨!

%49
得到真意灌体,就好像是得到狂蛮魔尊毫无保留的教导。

%50
毫无疑问,境界的直接提升,带来的增益是巨大的,更是全方位的。

%51
若按部就班,境界的提升,动辄数十年,上百年。除了梦境之外,恐怕也就只有这里的狂蛮真意灌体,是直接拔升境界的无上捷径了。

%52
“其实梦境本身,就是情感的宣泄,执念和渴望的结合,也是无数意志的残留。进入梦境探索成功,能提升境界,也就是排除梦境中情感的干扰,不迷失自我,从各种纷杂的意识提纯出真意,从而拔升境界。”

%53
“真正要说本质,都是真意灌体!”

%54
方源眼中精芒闪烁。

%55
铁冠鹰扑面而来,方源轻易避开。

%56
只剩下最后一只铁冠鹰,方源拉开距离后,再度用剑痕索命,将其毫无悬念地杀死。

%57
检查了一下仙元,还有许多剩余。

%58
虽然动用了数次剑道杀招和七转仙蛊,但方源准备充分,并不向之前赶路时那般捉襟见肘。

%59
还没有到向琅琊地灵借贷的地步。

%60
之前损失大半,完全是仙劫锻窍杀招耗费太多了!

%61
空中暂时没有敌人出现,方源便将目光投到地面上。

%62
雪怪们伫立在地上,仰着头,向他咆哮。

%63
“数目更多了。”方源皱了皱眉头。

%64
下一刻,无数青绿色的锐利风刃,还有黑沉沉的暗漩,凭空出现,暴射而下。仿佛长河奔腾,大江倾斜,火力狂猛至极,简直是铺天盖地,瞬间将地面上所有的雪怪淹没。

%65
轰轰轰……

%66
雷霆般的炸响声不绝于耳。

%67
持续了十几个呼吸,方源才停下攻击。

%68
密密麻麻的雪怪们,骤然减少无数,冰面上被清空了一大半,只剩下荒兽级雪怪屹立不倒,宛若数十根巨柱。

%69
刚刚不过是凡道手段,不足以对它们构成威胁。

%70
噗噗噗。

%71
一声声轻响,完全被风雪的呼啸声掩盖。

%72
更多的小雪怪,从雪堆中冒了出来。

%73
“才耽误这么一会儿工夫,这些雪怪竟有顽固不化的趋势!”方源大感头痛。

%74
只好继续动用剑痕索命,加紧点杀荒兽雪怪,同时,也用凡道手段,覆盖打杀似乎绵绵不绝的小雪怪。

%75
令他感到奇怪的是,他杀死雪怪,不论多少,是不是荒兽,都没有得到任何的狂蛮真意。

%76
“这场地灾很是古怪,历史上根本就没有此类记载。不过目前的强度,已经远超黑楼兰的升仙劫了。”

%77
这时,高空中风雪形成漩涡,再度异变。

%78
一只只飞鹤,从漩涡中振翅飞出。

%79
这些大鹤,羽色素朴纯洁,体态飘逸雅致,鸣声超凡不俗,赫然是荒兽九宫鹤。

%80
方源舔了舔嘴唇,带着嗜血之意,挺身而上。

%81
杀了一头,果然有真意灌体,不禁让方源更加兴奋,双眼几乎要放光。

%82
大雪纷飞,狂风呼啸,方源和这群九宫鹤展开激战。

%83
他身形忽退忽进,纵横左右,飞腾上下,灵活多变。

%84
起先,他孤身作战,以寡敌众,难免落入下风,场面被动。

%85
待等到他斩杀了几头九宫鹤后,便争取到了更加开阔的空间,进退更加自如。

%86
随着战果越来越大,他渐渐把握主动,开始游刃有余。

%87
这次的九宫鹤,直接出现了九头,数量是铁冠鹰的一倍还多,但仍旧被方源一一斩杀,只是耗费的时间延长了许多。

%88
成果是可喜的。

%89
方源感到各个方面的境界,又有了长足的进步。

%90
这种明显让人感受到自身变强的感觉,是如此的让人迷醉。

%91
不过,等到方源看见地面上密密麻麻的雪怪时,心情顿时又沉重起来。

%92
刚刚,他面对的荒兽雪怪,才不过十多只,现在却已经有五六十头!

%93
并且风雪中,更多的小雪怪在成长,荒兽雪怪也在向上古级数靠拢。

%94
这种愈演愈烈的势态,让方源下定决心要尽快阻止。

%95
“目前情况,就只能动用第三个剑招了!”

%96
仙级杀招剑浪三叠。

%97
方源深呼吸一口气,六成以上的心神都集中起来,同时催动五百多只凡蛊,并且更多的辅助蛊虫陆续被催动起来。

%98
这个杀招的难度,还有酝酿的时间,都要比剑痕索命要超出许多。

%99
因此,不能常用。

%100
最后,方源放出浪剑仙蛊。

%101
这只仙蛊,似刃似水,在半空中不断膨胀,很快化为一圈淡蓝水浪,将方源四面八方,前后上下都包裹起来。

%102
方源正要对地面上的雪怪们出手,这时,一片巨大的阴影忽然笼罩下来。

%103
方源抬头一看,微抽一口冷气。

%104
只见一头巨大的墟蝠,已经成形,横霸长空,覆盖方圆上百里!

%105
“糟糕!”方源心中戈登一下。

%106
这墟蝠的体型庞大,已经是上古荒兽,可战七转蛊仙。

%107
同时,墟蝠乃是宇道猛兽,在仙窍中作乱,破坏力极为惊人,搞不好它甚至会打穿仙窍的窍壁!

\end{this_body}

