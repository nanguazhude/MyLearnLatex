\newsection{谈谈赔偿}    %第五百三十二节:谈谈赔偿

\begin{this_body}

南疆。

万蛟!

龙吼声起,将整个仙道战场彻底冲垮,银鳞如海,方源电射而出。

一脱困,他便收到大量的求援消息。

“什么?琅琊福地被天庭突袭?还是凤九歌带队?”方源顿时一惊。

他遭遇到陆畏因,暂时被困,与此同时,琅琊福地就遭受天庭突袭,这两者之间若说没有联系,鬼都不信!

“哦?已经败退了吗?中洲蛊仙多数阵亡,唯有凤九歌一人生还?”方源看了看琅琊地灵传来的消息,又转头看毛六来的信。

毛六乃是影宗安插在琅琊派的间谍,是方源一方的人,他的消息才最为可信。

从毛六的信中,方源迅速了解到了整个经过。

“原来是凤九歌见局势不对,主动撤走了啊……”

方源眼中精芒烁烁,但这里并不是让他静心深思的地方,搞不好南疆正道势力中的八转蛊仙,正在赶来这里的途中。

方源迅速撤离,在途中,他直接催动预留下的手段,悍然将荡魂山引爆!

“呵呵,天庭方面突袭琅琊福地,显然是预谋已久。拔山的手段,并不常见,而凤九歌却是直接将荡魂山带走,这是处心积虑,要坏我根基。可惜……我的荡魂山岂是那么容易抢夺的?”

方源将荡魂山放在琅琊福地,一方面是为了表现合作的诚意,让琅琊派出力,经营胆识蛊生意,节约他的精力,另一方面也是方源留下后手,不怕琅琊派或者其他蛊仙私自吞没了荡魂山。

疾飞的途中,方源又将心神投入自家仙窍。

江山如故!

他催动六转仙蛊,对准早就保存下来的一大块山石。

这山石正是来源于荡魂山,是方源早就切割下来的部分。

此时荡魂山主体已经自爆,这块山石就是世间残存最大的一块荡魂山石了。

在方源动用江山如故仙蛊的情况下,这块山石立即膨胀,逐渐涨大,开始向完整的荡魂山的方向发展。

“天地秘境宛如仙蛊,都是唯一。荡魂山当初被幽魂魔尊改造,这块荡魂山石在江山如故蛊的照耀下,必定能重新还原出完整的荡魂山来。”

方源信心十足。

荡魂山虽然被夺走,但他并未失去,假以时日,荡魂山将重新回到他的手中。

不过,这个过程会有点漫长。

江山如故蛊毕竟只有六转,而荡魂山却是类比九转仙蛊的天地秘境,方源截留下来的荡魂山石块,体积上也远比整个荡魂山逊色。

天庭谋划许久,又是凤九歌亲自出手,方源前后损失还是很多的。

首先是上极天鹰没有夺回来,其次是荡魂山暂时毁灭,要重现荡魂山,需要一段时间,还需要大量的仙元投入,最后古月方正被俘虏带走,方源的血神子修行计划受挫。

庆幸的是,智慧蛊没有被带走,也没有被毁灭。

当时的情景,其实很危险。

凤九歌若是下定决心要毁掉智慧蛊,单凭智慧蛊本身,很难有生存下来的希望。

不过凤九歌最终没有下杀手。

这一点,让方源庆幸的同时,也明白天庭接下来的打击,必定会接连不断。凤九歌主动撤退,也说明了天庭定会有下一次的进攻!

但什么时候进攻?下一次进攻琅琊福地,天庭会派遣谁来?又通过什么样的方式?

这些都是问题。

方源落到一处普通的山谷中。

此刻他已经彻底远离战场,来的一路上他更是小心谨慎,暂时应该是安全的。

方源敛息匿形,随手布下蛊阵预警,随后闭目盘坐,进行推算。

片刻之后,他睁开双眼,心中已是将整个事件都了然于胸,仿佛亲身经历了琅琊派和凤九歌一行人的激战。

“可以确定定仙游仙蛊就在天庭的手中,而且似乎已经提升到了七转。还有配套的仙道杀招……”

琅琊福地的任何异变,都逃不出琅琊地灵的洞察。凤九歌一行人忽然出现的一幕,也早就被琅琊地灵记在心中,这些类似的线索,都送达到方源的手中。

有了充分的线索,再加上方源智道的底蕴,立即让方源明白了凤九歌究竟是如何突袭到琅琊福地内部的。

“为了防止定仙游仙蛊,我已经窜改了荡魂山的面貌,凤九歌此行却是出现在三大陆上。这方面情报的泄露,恐怕是和失踪的毛民蛊仙有关了。”

“这三大陆上要改变地貌地形,着实得要耗费一番功夫。就算这样做了,恐怕也阻止不了天庭继续入侵。”

“若我所料不差,天庭蛊仙虽然没有成功布置出仙道蛊阵,单看凤九歌如此轻易撤离,必定是在琅琊福地中留下了什么后手。”

种种思绪,在方源的脑海中迅速闪现。

天庭不出手则以,一出手就石破天惊!若非这段时间以来,方源也为这种情况提前做出了多番预防的手段,失去了炼炉的琅琊福地,此时恐怕已经被凤九歌征服了。

方源已经做得足够多,但天庭的底蕴太过深厚。就像这次,单凭宙道大能梁凉的遗留手段,就让凤九歌等人突入进来。

天庭这种对手,恰恰是最难抵挡的。因为不知道它还有多少的底牌!

方源叹息一声,这才开始联络琅琊地灵。

他将自身的经历,都告知琅琊地灵,并说出自己的猜测,陆畏因极可能是乐土仙尊的传人。

琅琊地灵虽然也有怨愤,在关键时刻,方源没有出现,保卫琅琊福地,但方源说的都是事实,琅琊地灵也不好追究什么。

“这一次天庭进攻,并未得手,又发现了智慧蛊,定然会再次前来。方源你身为琅琊派的太上长老,琅琊派需要你的战力,快回来镇守吧。”琅琊地灵道。

方源笑了笑。

这当然不可能!

琅琊福地的确帮助过方源很多,也算得上是方源的老巢,但是要让方源因此受到束缚,那是不可能的。

眼下天庭大势在手,只要修复宿命仙蛊,就能立于不败之地。

方源若是镇守在琅琊福地中,天庭一方必定都要笑坏了。这正是他们想要看到的。

方源镇守琅琊福地,完全没有出路,必须出去不断闯荡,搜寻修行的资粮,让自己飞速成长。

成长到一定程度,方源就要取出红莲真传,不管真传能帮助他多少,他都要在十年内,破坏掉天庭修复宿命蛊的大计。

只有这样做,他才不会落败身死,这也是唯一的出路。

否则的话,等到宿命蛊修复完成,全天下的蛊仙几乎都要受到宿命的安排,像方源这样的天外之魔,能有多少?天庭将重登巅峰,方源绝然不会是其对手。

“即便是琅琊福地被攻破,智慧蛊被摧毁,我也不会因为琅琊福地,而扰乱了自己的修行大计啊。”

方源眼中闪过一抹冷酷的光。

当然这些话,他绝不会对琅琊地灵去讲。

此时的方源,身上的盟约已经束缚不住他了。这段时间以来,洁身自好仙阵不是摆着看的。

“我们还会谈一谈赔偿的问题吧。”方源敷衍了几句后,对琅琊地灵道。

“呃!”琅琊地灵顿时脸色沉下,当初他和方源约定过,荡魂山、智慧蛊这些资产,若是丢失损毁,要按照十倍价值进行赔偿。

智慧蛊没有被毁,这让琅琊地灵松了一口气,但是荡魂山……

至少现在的确是丢了!

“你可知道,若是天庭方面得到荡魂山,拥有无穷无尽的胆识蛊供应,对于他们将是何等的助力啊。”方源见琅琊地灵沉默,便施加压力。

“你,你不要说这些了。我一概都赔偿你,你放心吧。”琅琊地灵回答道。

“我信你,太上大长老怎可能是无信的小人呢?”方源呵呵一笑。

“你想要什么样的赔偿?”琅琊地灵问道。

方源目光一转,他想要的可就多了!(\~{}\^{}\~{})

\end{this_body}

