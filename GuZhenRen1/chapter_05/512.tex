\newsection{叶凡vs铁若男}    %第五百一十四节:叶凡vs铁若男

\begin{this_body}

轰轰轰!

剧烈的爆炸声,响彻耳畔。(WWW.mianhuatang.CC 好看的小说凡道杀招,接连不绝。两道身影时分时散,激烈对拼。

一位是叶凡,气运浓郁,精修变化道、光道,得到商家蛊仙指点,又认神秘蛊仙陆畏因为师。最为难得的是,他不是光靠自身机遇,而是走出自己的蛊修道路。

而铁若男,虽然是出身于超级家族铁家,但幼年颇为坎坷,童年丧母,少年丧父,自立自强,成为铁家八大少主之一,励精图治,允文允武。声望渐隆,吸引许多追随者。尤其是得到铁家蛊仙铁面神的栽培,成为铁面真传的继承者。

这两人对战,可谓是将遇良才,棋逢对手,声势惊人,余波激荡,开拓出一大片空地。寻常交战的蛊师,只要进入这片空地,遇到两人交手的余波,擦着便伤,挨着便死,因此无不退避三舍,

三家蛊师们都被这两人交手的威势,震得目瞪口呆。

“小姐,叶公子居然实力这么强?我们以前怎么不知道?”商心慈身边的婢女小蝶开口道。

商心慈一双妙目闪烁了几下,沉吟道:“叶凡公子恐怕是另得了一些际遇,使得实力有了突飞猛进的提升。这是一件好事,只是他那对手也不简单,叶公子还望多加小心才好。”

小蝶捂嘴,嘻嘻一笑:“叶公子人的确挺好,一表人才,英俊潇洒,而且他对小姐的心意,明眼人都能看得出来。叶公子若能听到小姐你这么挂怀他,他一定会很高兴的。”

商心慈微微摇头,想要说什么,但终究没有开口。

小蝶一直察言观色,她是商心慈的心腹,嫡系中的嫡系,在张家时就存在的丫鬟,当初一路跟随商队,和商心慈一同来到商家城。

所以,别人称呼商心慈为族长或者城主,但惟独小蝶一直称呼商心慈为小姐,商心慈也没有让她改口,这是两人旧情深厚的体现。<strong>80电子书wWw.80txt.com</strong>

此刻小蝶却是心中生忧,暗道:“看来小姐还是忘不了那位魔道蛊师方源。唉,也不知是做的什么孽,那方源只是随同一路,护送我家小姐来到商家城,他就从此成为了小姐的心上人。也不知道他现在在哪里?恐怕早就已经死了罢。唉,若真是死了,见到尸体也好啊,能够让小姐彻底死心,从而走出来。”

小蝶资质乙等,终究只是一个凡人蛊师,还未接触到蛊仙这种层次上,因此根本就不知晓方源如今的境况,还以为方源仍旧是那个魔道的蛊师。

商心慈虽然从商家蛊仙商青青处,得知了方源如今的情况,但也没有必要和小蝶说这样的事情。

“真是江山代有才人出,我真的是老了。”侯家族长看着叶凡和铁若男的争斗,心中震撼之余,也越发苦涩起来。

铁家族长倒是吃惊的成分更多一些,心想:“我原以为派遣出若男,必定能够扫清一切障碍,奠定胜局,但是现在看来,结果难以预料啊。没想到商家那边,居然也藏有如此强者!”

与此同时,战场上空,九霄云外,凡人蛊师根本无法察觉的地方,三位南疆蛊仙也将目光,投注在铁若男和叶凡的争斗上面。

他们分别来自商家、侯家、铁家,此刻三仙盘坐在白云之上,相邻而坐,中间摆着一面圆桌,上面杯盏中茶香四溢。

蛊仙喝茶品茗,却是将各自家族中的凡人,当做赌注,进行一场赌局,决定这处烟罗暖玉田的最终归属。

这片烟罗暖玉田乃是上等资源,即便是超级势力,也要重视,若能拥有,就是一项基业,为家族盈利。

只是用凡人来作赌,蛊仙不出手,会不会有些儿戏?亦或者,最终输了,会有蛊仙不甘心耍赖呢?

仙家矛盾,用凡人做赌,其实并非表面上这么简单。

蛊仙中那些散修、魔道,独来独往,也就罢了。但正道势力,却要讲究绵延而长。因此更注重后代培养,比如商家就有少主选拔政策,让各大少主挣钱,还有演武场吸纳外姓蛊师。铁家也设立八大少主,其他超级势力各有政策,都是为了筛选出凡人后辈中的蛊仙种子!

让族人互拼,意义重大,非同一般,代表着将来的家族成就。尤其是其中的蛊仙种子,每损失一位,蛊仙也都会心疼不已。

但蛊仙们需要这样做,因为此事利大于弊。

一者,只要家族蛊仙不损,就是根基不动摇,绝不会伤筋动骨。蛊仙不出手,凡事都好商量。

二者,让他们这些凡人争斗,也是一场场的磨砺考验。让人才在生死之间脱颖而出,至于死去的天才那就不是天才。正道家族代代培养,就算是蛊仙种子,也能折损得起。

三者,争斗中,必定会建立恩怨情仇,这就能更有家族的凝聚心。蛊仙种子日后成长起来,其他超级势力中或许就有同辈的死敌。这个时候,他就更不能轻易脱离家族,反而要依赖家族更多一些。

任何一个政策都有考量,非是蛊仙以凡人做棋子的浅薄游戏。上位者都绝不简单。

此时,三仙目光都聚焦在叶凡和铁若男的身上,三人均知:这两人对局的结果,便是关键,能够决定烟罗暖玉田的最终归属。

坐在左侧的商家女仙商青青,微微一笑开口道:“恭喜铁面神大人,寻觅到铁面真传的优秀继承人。”

铁家拥有铁面真传,是南疆蛊仙众所周知的事情。并且也都知道,此道真传要求严苛,寻觅合适的继承者颇为艰难。

真传是一套成熟的,较为优秀的修行内容。所以有时候,真传也不是所有人都能继承的,也会有要求和条件。

坐在最中央的,便是铁家蛊仙铁面神。

他身穿武者劲装,胸背和腿脚,挂着轻甲,他的脸上覆盖着一层厚实的铁。

这并非是他带着面具,而是铁面真传的修行特征。历代修行这份真传的蛊仙,必须得是心怀正义之人,同时他们往往也是南疆蛊仙界中最擅长查明真相的人。

他之前选择神捕,也就是铁若男的父亲,充作铁面真传的继承者,但神捕死在青茅山上,他在最近发现铁若男的优秀潜质,选作继承人。

这一次,铁家派遣他来主持这里的局面,为铁家争夺这片烟罗暖玉田,因此铁面神便将铁若男也带来,借助这次机会,对她磨砺。

在场的三位蛊仙当中,侯家、商家的蛊仙都只是六转修为,惟独铁面神拥有七转修为,所以他坐在中央位置。

言语交谈之中,商青青和那位侯家蛊仙,也对铁面神颇为尊重。

铁面神微微点头:“商家仙子所料不差,这正是我选定的真传继承人。只是还年轻,需要多加磨练。”

一旁的侯家蛊仙侯耀,则始终沉吟不语。他关切地注视着下面战场,虽然目前侯家最弱,但对于他而言,也并非完全丧失了机会。只要商家、铁家拼得个两败俱伤,事情最终是什么结果,还真的是个未知数。

烟尘飞扬,叶凡猛地后退,然后深呼吸一口气,胸膛宛若气球鼓胀起来。

身后铁若男如雄鹰扑下,双手呈爪,狠狠抓来。

叶凡猛地回首,张口一吐,吐出一道赤红的光箭,射向铁若男的门面。

铁若男临危不乱,双目精芒爆闪,忽然把头一仰,整个身躯如燕子翻飞,躲过赤红的光箭。

但此时,叶凡已经抓住良机,双掌一推,反扑过来。(未完待续。)<!--80txt.com-ouoou-->

\end{this_body}

