\newsection{白兔投怀(下)}    %第二百零四节:白兔投怀(下)

\begin{this_body}

武辽并不理睬武安,他直接对着大殿的门深深一礼,高声嚷道:“武安此人居心叵测,武辽求见大人,举报此人!”

武安的心,立即提到了嗓子眼。[看本书最新章节请到棉花糖小说网www.mianhuatang.cc]

他害怕,他恐惧。

他生怕眼前的这道殿门在下一刻,轰然打开,然后白兔姑娘被赶出来,双眼垂泪的样子。

那他武安就完了。

但是没有。

殿门静悄悄的闭合着,似乎方源没有听到一般。

但这怎么可能?

这里不是普通的地方,而是超级蛊阵的内部空间,形成的一大片能够令人居住的宫殿群落。

方源又执掌其中的两只关键仙蛊,对武家的区域可谓是洞若观火,了若指掌。

武辽、武安的一举一动,他都能视察得清清楚楚。

武辽之前的喊话,方源怎么可能没有听到?

绝不可能!

见到方源没有理睬自己,原本信心十足的武辽楞了好一下。

他知道,武安在前一段时间里,千方百计地想要求见方源,目的是什么,武辽也当然明白。

武辽没有阻止。因为他知道,人心是贪婪自私的,很多人和自己不一样,就比如说上一任的武家七转蛊仙。

但后来,方源一次都没有接见武安。

发现这个事实之后,武辽非常高兴。

他以己度人,觉得方源既然拒绝了武安,必定是反感仙缘生意这种事情。

“也是!像武遗海大人这等人物,如何在乎这点蝇头小利呢?他初入家族,更在乎的是自己的名誉吧,在乎的是今后在武家高层中的发展。”武辽心中猜测。

怀着这样的想法,他前来求见方源,满心以为方源会接见自己。

但现在呢?

殿门紧紧关闭,像是一堵墙,又仿佛是化作了堵在武辽心中的一座高山。

武安哈哈大笑。

他擦了擦额头上的冷汗,一颗心从喉咙处重新落回了原点。

武辽的拜见。是一次最好的试探。

武安顿时觉得,他已经懂得了方源的心意。[\&\#26825;\&\#33457;\&\#31958;\&\#23567;\&\#35828;\&\#32593;\&\#119;\&\#119;\&\#119;\&\#46;\&\#77;\&\#105;\&\#97;\&\#110;\&\#104;\&\#117;\&\#97;\&\#116;\&\#97;\&\#110;\&\#103;\&\#46;\&\#99;\&\#111;\&\#109;

武辽不信邪,再次高声求见,殿门仍旧闭合。没有丝毫反应。

武安的笑声更大,心中也笃定起来。

他开口道:“武辽,你还是不要叫了。大人正在里面处理重要的事情,你这么打扰他,好吗?”

武辽的脸色铁青一片。非常难看。

武安冷笑一下,摇了摇头。

武辽很蔑视武安,武安也觉得武辽这个人太固执,一点都不通情达理,不是处世之道。

“现在看来,这一招虽然行了险,但是效果很好啊。”

“也是,白兔姑娘的底子本来就很好。当初我第一眼看到她的扮相,差点心肝儿都跳出了嗓子眼。恨不得将她纳入怀中,狠狠地揉捏摸搓!”

“见到白兔姑娘这样的美人儿。只要是男人都会就范!武遗海大人,也是血气方刚的男儿。而且他初得权位,美人主动献身,更能满足他的心理。并且,他之前是东海散修,白兔姑娘是南疆散修,散修和散修之间,当然相互了解,心心相印啦。更难得的是,白兔姑娘乃是货真价实的处子之身。这男女见面。犹如干材烈火……嘿嘿嘿!”

正这样想着,下一刻,让武安猝不及防的事情发生了。

只见大殿巨门轰然敞开,白兔姑娘就站在门口的另一边。出现在两位武家蛊仙的眼前。

“怎么回事?这就出来了?怎么时间这么短?”武安诧异万分。

“一定是大人听到了我的话,所以将这女仙送了出来。”武辽大喜,看向白兔姑娘的目光却是微微一愣。

即便他很厌恶反感,但此刻见到白兔姑娘的美貌容颜,心中不禁一荡,不得不低头承认白兔姑娘的美貌。

“时间这么短。莫非是没有成功!?”武安的心,顿时沉入谷底,目光发直,口干舌燥。

但很快,他又看到白兔姑娘换了一件衣服,不是之前的带有白色绒毛的简陋兽皮衣裙。

而且!

更紧紧吸引武安视线的是,白兔姑娘的双耳吊起来一对翡翠吊坠。吊坠呈现圆球形状,十分饱满。

这是南疆的风俗。

一旦姑娘嫁人,破了处子之身,就会佩戴上这种样式的吊坠,示意男女和合,得到了圆满。

武安大喜,狂喜!

武辽则脸色苍白,他的眼神也非常犀利,看到了白兔姑娘双耳上的玉球吊坠。

“成、成了吗?”武安走上前来,小心翼翼地问道。

白兔姑娘面泛古怪之色,却仍旧点点头,没有说话。

武安吐出一大口浊气,闭上双眼,巨大的压力离他远去,一下子让他有些不适应,差点失力当场倒在地上。

“但为什么这么快?”武安心中又泛起一个念头,“不应该啊,难道武遗海大人在那个方面,也是个……雏儿?这种情况并非没有,南疆就有不少这样********修行的人。嘿嘿,若是这样的话,今后我可以贡献许多奇妙的蛊虫,交给大人运用啊。不过,武遗海大人已经是七转蛊仙,一身道痕,寻常的凡蛊恐怕效果不太大。”

“这对玉珠,就是武遗海大人赐予我的。大人嘱咐我关照二位,他今后将以潜修为主,二位大人各行其职,不要打扰他修行。”白兔姑娘又道。

此言一出,武安立即用胜利者的目光,得意洋洋地看向武辽。

武辽雄躯晃荡了一下,后退一步,愤恨地看了武安一眼后,转身就走。

“武安大人,我们也走吧。”白兔姑娘道。

“姑娘还是不要称呼我为大人了,我武安哪是什么大人呐。今后直接称呼我的名字即可了。”武安的脸上浮现出十分亲切,又不让人觉得唐突的笑容。

现如今,白兔姑娘的身份已经不一样了。

武遗海既然不仅临幸了白兔姑娘,而且还赐予了玉珠吊坠,这是认可了她的妾侍身份。

白兔姑娘感受到武安的态度前后变化,不禁心中更加五味陈杂。

她开口道:“那么武安,武遗海大人有一些话,是特意带给你的。”

“请姑娘不吝赐教!”武安连忙脸色一肃。

白兔姑娘语出惊人:“其实……武遗海大人,并没有要了我。”

“什么?!”武安失色动容失色。

白兔姑娘脑海中,回忆起之前的那一幕。

当她主动投怀送抱,心绪纷乱无比的时候,一双有力的手就把住了她的肩头。

然后,从两只温热的大手上,传出一股强有力的,不容许反抗辩驳的力量,将白兔姑娘推离出方源的怀抱。

白兔姑娘脸色煞白,一瞬间,她觉得自己失败了。

但下一刻,她就听到方源的声音传入她的耳中:“我在东海的时候,一直是散修。散修的苦楚,我十分清楚。有的东西,虽然甜蜜,但要吞入腹中,是要付出代价的。想必此刻,你也有了这种觉悟。不过,我却不想做这种趁人之危的事情。”

话音刚落,白兔姑娘就看见一只手伸到她的胸前,手中拿捏着一只蛊虫。

“这是衣蛊你先用上,再与我说话。”方源的声音,再次传来。

白兔姑娘听命行事,衣蛊化为衣裳,遮蔽了她的娇躯。

当她再度抬起头来的时候,她看到方源正对她微微而笑:“我虽然加入了武家,但骨子里还是个散修,这恐怕一辈子都改变不了吧。这只蛊虫,就算是散修对散修,同道之间的一点小小的关照吧。”

白兔姑娘的心揪了起来,难以言喻的强烈感动,冲击她的身心。

一瞬间,她哽咽了,眼眶泛起了泪花,视线模糊起来。

但就是在这模糊的视线中,白兔姑娘却能感觉到方源的微笑,那多么温柔的笑容,像是太阳一般照耀在她的心田,给与她巨大的温情。

“不过,你若是就这样回去,恐怕也无法交代。你的来意,武安之意,我都清楚得很。所以,这件东西你也得接着。”方源说着,掏出一对玉坠。

凭他之能,要造出这等凡物,只在一念之间而已。

白兔姑娘满含着泪水,接过这对圆珠玉坠。

“从今天开始,你就是我的侍妾了。”方源笑着道。

白兔姑娘轻轻的嗯了一声,微微点头。

“仙缘生意,你们尽管去做。但是我不会接手,出了事情,我会帮一把,但不会正式出面。”

“这是侵犯了其他正道家族利益的事情。毕竟整个超级蛊阵,是结合了大家的力量,才构建出来的。”

“我初到南疆,立足不稳,需要的不是这些蝇头小利。”

“你如此聪敏,应该会明白的吧?”

“嗯,我明白。”白兔姑娘连忙应答,声调颤抖。

“回去吧,把我的意思转达给武安,还有他身边的人。我的那份收益,就送给你了。”方源逐客道。

白兔姑娘鼓起全身的勇气,想要再看方源一眼。

但她终于还是没有能够。

她就这样低着头,转身往回走。刚刚背对方源的时候,眼泪就一颗颗,宛若珍珠般滚落而下。

情况比她想象中,要好得太多太多。

武遗海这个名字,将深深地印刻在她的内心最深处。

“我明白了,我明白了!”听完了白兔姑娘的转述,武安满头都是汗渍,之前的洋洋得意,统统不翼而飞。

他下意识地回望方源的住处,目光中有忌惮,更有敬畏。(未完待续。)

\end{this_body}

