\newsection{方源的第三次地灾}    %第九十节:方源的第三次地灾

\begin{this_body}

南疆.

群山耸立,青黛连绵。山风不断呼啸。

十多位蛊仙,相距数千里,围成一个巨大的圆。

在这圆心的高空之上,一位蛊仙老者不断调动仙元,聚精会神,正在布阵。

一座大阵,已具雏形,覆盖了十数万里的超大范围,正将一团瑰丽梦幻的光团,罩在阵中。

这个巨大的光团,五光十色,绚烂多姿,正是超级梦境。

义天山自大战之后,已成一片废墟。魔尊幽魂被天意算计,被方源夺得胜利果实,而他的本体则受困在梦境中,不断被梦境损耗。

南疆蛊仙们虽然不知道义天山大战的内幕,但却明白这处超级梦境所代表的巨大利益。

场上这些蛊仙,都分别来自南疆的各大正道超级势力。

经过一场艰难的谈判,他们达成协议,共同掌控这处超级梦境。

这片梦境实在太过重要,没有哪一个超级势力能够单独吃下。又担心魔道、散仙过来占便宜,所以正道蛊仙们提前就联合起来,将这里霸占,并在这里布置出一座超级蛊阵来,以便在将来慢慢瓜分梦境的好处。

“起!”最高空的蛊仙老者,忽然大喝一声,全身绽射出无边的华光。

紧随着他,南疆正道蛊仙们也相继发力,大量的仙元消耗,无数凡蛊飞舞,一只只仙蛊分别从各个蛊仙手中飞出,落到合适的位置上。

不少蛊仙都满脸大汗,神色紧张。更有人双手十指都在微微颤抖。

他们在这里布置蛊阵,已经持续了九天九夜。

很显然,到了现在,蛊仙们布置这座超级蛊阵,已经到了紧要关头。

“他们快要成功了!”

“我们还不动手吗?再不动手的话……”

“唉!没有机会。在他们周围,还埋伏着蛊仙强者,甚至我怀疑还有仙蛊屋坐镇!”

“这样的话……”

一场交流悄无声息地展开。

魔道、散仙也都明白这片超级梦境的价值。就算没有这片梦境,这里发生过神秘大战。说不定留下什么蛊仙残骸或者传承,这都是珍宝!

但正道蛊仙们行事极其稳妥,始终没有给这些魔道和散仙机会。

无奈之下,这些蛊仙们明智地选择了撤退。

轰!

一声巨响一股巨大的光柱,直冲九霄云外。

光柱出现得快,消失得也极快。

一座恢弘的超级蛊阵,渐渐隐去实体,消失在众仙的视野当中。

“辛苦了多日。终于是成了!”主管布阵的蛊仙老者,缓缓降下来。

正道蛊仙们靠拢一起,齐聚一堂。

“这次布置蛊阵,我武家出的仙蛊,多达五只。所以这处梦境,应当有三成归我武家。”

“呵呵,此言有理。不过……我罗家出的仙蛊,也绝不比你武家少啊。”

“这都是什么话!没有我们大家一起出力,耗费仙元,这座超级蛊阵能布得起来吗?”

“依我看。要说贡献,我家太上大长老池曲由大人主导蛊阵布置,乃是当之无愧的第一!”

超级蛊阵虽然布置下来,但正道蛊仙们对于梦境的瓜分,还未有定数。

一时间,议论的声音越来越大,竟朝着争吵的程度发展下去。

单纯的争吵当然是没有结果的。

正道蛊仙们不欢而散,开始为自己的利益筹谋奔走。

几日之后。

商量山。

“心慈拜见青青大人。”商心慈躬身行礼。

商青青看着眼前的商心慈,满意地点点头:“我没有看错你。这些天你担当族长,做得不错。商家成员都有渐渐归心之象。”

“心慈能有今日。还是青青大人一手提携。”商心慈真诚地感谢道。

商青青步入主题:“我这次唤你来,是有些事情要询问你。你担当族长后,做的第一件事,就是为黑白双煞洗去通缉令。你可知道。你那黑煞哥哥,并不是方正,而是方源。”

“方源?”商心慈一脸疑惑。

她万万没有料到,商青青大人专门召见她,居然是跟她讲黑煞哥哥的事情。

商心慈担当商家族长,眼界开阔。知道许多秘辛,更知道眼前的美貌仙子,乃是高高在上,傲视凡俗的蛊仙!

这位蛊仙居然和她谈一位凡人蛊师,怎么不叫她疑惑?

商青青一脸严肃地道:“心慈,接下来我的话,你要好好地听,一字一句都很重要。”

“是,心慈谨遵教诲。”

“你结识的这位黑煞方正,实名古月方源。他来头极大,非同小可。神出鬼没,极其危险。他是魔道蛊仙,犯了天大的案子,现在不仅是我南疆,还有中洲、北原、东海、西漠的蛊仙,都在追捕他。”商青青语气低沉地道。

商心慈张大嘴巴,听得目瞪口呆。

一时间,她都感觉自己出现了幻听。

什么时候,黑煞哥哥竟然成仙了,而且还让商青青大人都如此忌惮?

“你不要怀疑自己的耳朵,我岂会拿这种事情和你开玩笑?你和魔头方源有过一段交往,你现在跟我说说整个事情的经过。”商青青问道。

“啊……啊,是,是。”商心慈好不容易回过神来,开始回忆,“我是在商队中见到黑煞哥哥第一面的。他其实是一个好人,助我脱离困难……”

商心慈回忆往事,渐渐深入,不免脸上流露出一丝温柔之色。

整个过程,商青青都静静地听着,不发一言。

商心慈说完,努力鼓起全身的勇气,轻声为方源辩解道:“青青大人,我觉得是不是搞错了?方正,嗯,方源哥哥明明只是凡人蛊师,怎么会是危害天下的大魔头呢?”

“呵呵。”商青青的脸上,流露出嘲讽的笑,“他不是大魔头,谁还能是?他的狡猾和阴险,不是你能想象的。他把整个北原蛊仙界闹得天翻地覆,中洲天庭抓他,都没抓住。没有人知道他在哪里,又在做什么坏事,策划什么阴谋诡计。你和他的交往,不是你认为的那么简单,我能听得出这里面有他的算计!”

“你知道么?因为你和他的这段交往,现在连我整个商家都饱受南疆蛊仙的压力。从今以后,你要和他一刀两断,再无瓜葛。好好做你的商家族长,下去吧。”

商心慈只得退下。

“你就这样放了她?”商心慈离开之后,一个身影浮现出来,站在商青青的身旁。

此仙身材瘦削,一脸阴翳之色,正是商家蛊仙商贪墨。

商青青一笑:“那还能如何?”

商贪墨正色道:“至少不能听信她的一面之词,要搜魂,知晓经过!”

商青青笑意更盛:“你看这是什么?”

说着,她伸出所在大袖中的手。

她的手,已无人色,仿若树干,指甲处生长出朵朵翠兰小花。

商贪墨见了,微微一愣:“我倒是忘了,你有这个仙道杀招,可比搜魂靠谱多了。”

“我已知道那方源和我商家的一切交集。他来我商家城,是想谋取修行的机缘,不愧是拥有春秋蝉的重生之人。这并不是什么大事,也没有什么算计阴谋被他埋伏下来。”商青青道。

商贪墨点点头:“那我放心了,不过,外面那帮人该怎么和他们解释?”

商青青冷笑一声:“他们?无非是想谋夺更多的梦境,所以拿方源这个借口,来攻击我们商家。如今我的这份调查结果,已经足够堵住他们的口了。”

商贪墨眼中精芒一闪,他抬头远望,目光好似穿透了商量山,见到了那片超级梦境:“或许……是我们商家蛊仙客气得太久了,别人都以为我们不善战斗了呢。”

局势在不断地变化着。

北原这边,黑家大战已经结束,尚有一些余韵缭绕。正魔两大阵营,还有散仙,都各有收获。黑家灭亡,百足家取而代之,改变蛊仙界的大局。

而南疆中,围绕着超级梦境,正道蛊仙们联手,技高一筹,首先将魔道和散修都排除在外。但在正道内部,一场瓜分超级梦境的争斗,即将展开!

不论何时何地,追逐利益的种种戏码都在上演。

就这样过去了十多天。

北原,北部大冰原。

方源落下仙窍,抬头望着高空,口中呢喃:“终于到了这第三次地灾!”

天地二气翻滚不休,无云无雾,至尊仙窍中呈现出一片诡异的风和日丽的景象。

一只只飞鸟,凭空而生。

它们浑身发着绿意微光,飞行极速,在方源的眼眸中,甩出一道道的修长光线。

鸟声清脆无比,蕴含无限生机。

“春晓翠鹂。”方源眼眸微微一缩,认出了这种飞鸟的跟脚。

别看它体型极小,类似麻雀,但却是货真价实的荒兽!

“此鸟在荒兽中十分特殊,身负大量的律道‘生’道痕。但它寿命极短,从蛋中孵出后,即可翱翔苍穹。所到之处,身上道痕逸散而出,令天地改易,焕发无限生机。十几个呼吸之后,就会寿终而死。”

方源心中念叨,便见这处小北原中,原本铺有薄薄冰雪的地面,已经冒出一片绿色的草苗。

很快,在春晓翠鹂的影响下,青草拔高,迅速生长,中间还夹杂稀少的野花。

“怎么回事?这是什么地灾?天意是想来帮助我建设仙窍么?”方源纳闷。

\end{this_body}

