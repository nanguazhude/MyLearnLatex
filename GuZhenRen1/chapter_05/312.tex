\newsection{宾主尽欢}    %第三百一十二节:宾主尽欢

\begin{this_body}

池伤喝的有点上头,满脸通红,话变多了。

似乎这一次战胜了方源之后,信心爆棚,认定自己可以抱得美人归。

蛊仙当然有各种各样的手段,可以醒酒。不过如此一来,就大大丧失喝酒的乐趣,同时也是对酒友的不敬。

方源连忙端起酒杯,喝下手中的这杯酒。他笑道:“如果真是那样,那我就只好自认倒霉,谁让我碰到池伤你这样的竞争对手呢!唉……”

方源一声感叹,既生瑜何生亮的意思,立即溢于言表。

池伤哈哈大笑,伸出手来,竟然拍拍方源的肩膀,表示安慰。

方源心中失笑。

池伤到底是单纯呐。

这个事情早已经有了定局。

只要方源松口,乔丝柳必定会自投怀抱,这几乎是板上钉钉的事情。

只是乔丝柳目前没有表现出明确的意向,只是这位仙子心中也有计谋和骄傲,想要欲情故纵,待价而沽罢了。

其实,方源早已经看得清楚透彻。

就拿这一次乔丝柳的来信来讲,其实这就是她表明自己的态度。

“乔丝柳见到我和池伤相争,自然是非常高兴的事情。在此之前,一直作壁上观,因为这正符合她的心意。”

“等到我最后失败,她这才站出来,主动告诉我她和池伤之间的事情。目的就是要安抚我,告诉我其实她和池伤之间,并没有什么动情一说。顶多也就是阵痴单方面的相思罢了。”

“但这个聪明的女人,从不直说自己的态度,而是阐述曾经的一个事实,来表明自己的清白,同时更证明她有巨大的追求价值。”

男人追求女人,是天性和本能。一个女人有没有男人去追求,被多少男人追求,被什么样的男人追求,这些方面都能或多或少地体现出她的价值来。

“乔丝柳想和我共结连理,这不是她的个人意愿,而是乔家的政治图谋。”

“只因为这一点,我就已经胜过她所有的追求者。”

“就算我失败再多次,再花心,再不堪,再在月亮节那样的场合胡闹,又如何?”

“我已经胜利了!”

“甚至就算我移情别恋,再不去追求乔丝柳。乔丝柳也会过来倒追我。”

“当然,这不是爱情,而是……政治!”

方源心中冷笑,他早已经看清这一切。

但是他绝对不能够接受,乔丝柳的这一份“爱情”。

只要他一接受,他就要彻底陷入汹涌险恶的政治漩涡当中,他不再是单纯的武遗海,而是乔家安插在武家内部,为整个家族争夺利益的最主要的棋子。

而武庸也绝不会像现在这样,给方源如此宽松的环境,而是会对方源严加监控,不断打压,时刻保持提防。

如果是这样,方源还能安心地汲取梦境的精华吗?

显然是不可能的,搞不好还会暴露真相。

那样的话,就更加麻烦了。

方源的身份一旦暴露,首先就是乔家的猛烈打击,因为身为正道势力,肯定要第一个和方源摘清关系,乔家、武家为了证明自己,维护名声,他们对于方源的讨伐定然会无比的疯狂,比任何其他的势力都要尽心尽力。

“不过我虽然不能接受,也不会接受,但是我却可以利用这份‘爱情’。”

借助乔丝柳的这份爱情,方源成功地让武庸引起警惕之心,间接地让武庸安排自己,重新回到了超级蛊阵,达到了自己的目的。

然而世间没有十全十美的东西,这样的一个办法,虽然巧妙,但也有后遗症。

武庸的警惕,罗木子、轮飞、池伤,还有其他乔丝柳的追求者,都是后遗症。

纯粹的爱情,并非没有,而是非常稀少。

人活在这个世界上,要考虑的东西很多。

乔丝柳的美貌、身家、修为等等,都是吸引和诱惑。

或许她的追求者中,池伤最为单纯,但仍旧夹杂着阵道的因素。其他人就更不用说。

至于方源,他算是最特殊的一个。

池伤的到来,并不出方源的意料。

事实上,他早有了心理准备。

就算没有池伤,也会有其他人过来找茬。就算不是当下,也会在将来。

因为方源动了“乔丝柳”这块可口的蛋糕,触怒了很多人。这些人有追求者,也有追求者背后的利益集团。

俗话说,红颜祸水。

一位美人,不是那么好追求的。

只有单纯的人,才想着只用爱情和感动,就能征服一个美人。

其实远远不止这些。

女人的美丽,就是天然的嫁妆。

这笔嫁妆丰厚至极,是一大笔的财富。

谁不想获得这样的财富?

但不是谁都有能力守护得住这样的财富。

如果有美人,愿意因为单纯的爱情,而将这笔财富交给另一半。那么就请幸运儿好好对待吧。

但事实上,这种情况太少太少。

如果有就请珍惜,如果没有,也不要奢望。

不要说“俗”。

因为这就是现实。

当方源回到超级蛊阵这里的时候,他就开始琢磨着,如何处理这件事情的后遗症。

真正高明的棋手,是要在走棋之前,不断算计前面的棋路。

事实上,当方源决定利用乔丝柳的这份“爱情”,来达到自己目的的时候,他就已经开始设想,如何处理这个方法所带来的后遗症。

如果他处理不好,那么池伤只是第一位来找茬的,接下来还有第二位、第三位,乃至绵绵不绝。

这当然是对方源探索梦境的巨大干扰。

同时,武遗海这个身份,也让方源束手束脚,并不能应对这些接踵而至的挑战,或许还会在应付挑战的过程中,露出马脚。

方源需要谨慎且又正确地解决这个事情。

怎么处理呢?

说难也难,说简单也简单。

那就是塑造出一个肉靶,比方源更能吸引那些追求者的火力,更能吸引麻烦。

方源心中,早已经期待着有人来找他的麻烦。

池伤主动送上来门来,而且这么及时,这么快,其实让方源暗中欣喜。

当他阅览了池伤的情报之后,他心中的欣喜扩大了数倍!

对手的愚蠢,就是自己的幸运。

当然,要说明的一点是:这种愚蠢往往和智商、情商无关。

那方源在地球上的一个鲜明的事件来举例子,就是岳飞。

岳飞抗金,接近成功之时,却被身后的皇帝用十二道金牌召唤回来,再用莫须有的罪名处死。

为什么?

是因为皇帝的愚蠢吗?

是愚蠢!

但皇帝和宰相是白痴吗?

当然不是!

岳飞抗金若要成功,岳家军就会声威更隆,皇权弱于帅权,威胁太大。更关键的是,岳飞若是成功,就会迎回先帝。那现在的皇帝还做什么做啊?

放在现在,池家太上大长老池曲由刻意栽培出池伤这样的蛊仙来,难道是因为他痴呆,不知道这样的池伤,一心钻研阵道,在政治倾轧中,会毫无反抗之力,最终只能沦为被利用的工具吗?

他当然不痴呆,正是因为知道这一点,他才会如此栽培。

因为他的儿子需要继任太上大长老之位,所以需要一个“工具”来辅佐他的儿子,那就是池伤。

很多人认为,池伤的到来,是方源的麻烦。

方源从来没有这么想过,他反而觉得这是他的一个机会。

只要他处理好这件事情,他就会解决这件事情的后遗症,而在将来的一段时间里,他才能安安稳稳地探索梦境,暗中提升自己的底蕴。

所以方源必须输。

不输,如何才能塑造出更强更具有威胁性的池伤出来?

没有这样的肉靶子,怎能将他身上的火力,吸引出去?

池伤无疑是容易算计的。

通过情报,方源知道,池伤最喜欢的就是钻研阵道,最擅长的就是解决难题,设计出符合要求的蛊阵来。

方源便拿这一部分挑战他。

池伤这种研究狂人,在自己最擅长的领域,有着独属于他的骄傲。

他在其他方面可以不在乎,甚至不修边幅,不贪恋财富,但是在这方面,他必须在乎。因为这是他整个人生的意义所在!

方源一个变化道蛊仙,拿这个东西来刁难他,本身这样的行为,就是扇池伤一大巴掌。

池伤这样的人,怎么可能忍不住?

他当然忍不住。

一忍不住,就直接落入方源早已铺设好的陷阱当中去,完全爬不出来。

不仅爬不出来,他还向方源敬酒,觉得方源这个人真诚!

“你虽然是竞争者,但、但怎么说呢?咱们真的是不打不相识,哈哈……”池伤大笑,和方源勾肩搭背。

“什么话都不说了,咱们再喝!”方源又灌了一口。

池伤瞪大双眼,盯着方源猛看,然后竖起大拇指:“兄台爽气!来,干了这杯。”

双方一扬脖子,都干了。

“好酒啊。”池伤感叹,“呃,说实在话,我还是第一次,第一次喝这么多的……”

方源哈哈大笑,眼底深处仍旧是一片冷然。

正道的酒,不是这样容易喝的。

魔道打打杀杀,直来直往,你退我进,你进我退。正道的一场酒宴,一个流言蜚语的背后,往往是阴谋算计,各种诡谲心思,看起来平静的水面下却是暗流汹涌,吃人不吐骨头。

方源挑战池伤输了,他很开心,他会大加宣扬此事。

输了又如何?

有时候,失败也是一种达到目的的手段。

池伤也很开心,这场酒宴他觉得是自己来到这里,最令人开怀的一场。

一时间,宾主尽欢。(未完待续。)

\end{this_body}

