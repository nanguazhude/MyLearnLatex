\newsection{意外惊喜}    %第一百六十四节:意外惊喜

\begin{this_body}

“怎么回事?”方源十分诧异。

在福地地灵粉红灵蛇的主动配合下,他十分顺利地将韩东福地,直接吞并,化为直接至尊仙窍的一部分。

于是,至尊仙窍中的小北原里,出现了一片石头浅滩。

浅滩中,有着碧绿的清澈水源,各种小鱼小虾,在清可见底的绿水中恣意畅游。石头缝隙中,青苔和水草随着水波荡漾,还有螺丝和贝壳等等。

最多的是粉红色的灵蛇,有的孤单一条,在石缝间好奇探险。也有成群结队,在绿水中游荡,统一行动,仿佛是一团红丝线。有些灵蛇群,有上百条,游荡起来的时候,情景壮观。

值得一提的是,这里还有一条荒兽级的粉红灵蛇。

它栖息在石头缝隙之中,似乎陷入酣睡之中。

整个福地搬迁的过程,也没有把它惊醒。

方源将整个韩东福地,都搬了进来。融合的程度堪称完美。

本来,若是方源单独提取这些资源,移到自己的至尊仙窍中。他还要挖掘河池,开辟出一个环境来。

如此一来,因为环境的改变,福地道痕的差异,这些资源还会衰减和死亡。就像是方源之前拿走了黑凡洞天中的全部资源,在此之后的很长一段时间里,他都不得不将黑凡洞天中的许多资源,往外贩卖。

因为这些资源,都无法适应至尊仙窍中的环境。

吃力不讨好。

但现在,方源直接吞并了仙窍,就省去了许多的麻烦。

韩东福地中的这些资源,不管鱼虾水草什么,都不需要再适应环境。因为方源将整个环境,都搬了进来!

如此一来。方源是完完全全,百分之百地将韩东的一生积累,都吞并。并且在瞬间消化。

“韩东福地中,盛产的只有一项。那就是灵蛇。他已经能培育出荒兽级灵蛇了,虽然这片福地是下等福地,但能经营到这种地步,算得上出色了。”

今后,方源完全可以继承韩东的资源,往外贩卖灵蛇。

灵蛇本身是一种异兽,但种族上限很高,可以一直培育到上古荒兽灵蛇。

方源稍稍估算了一下。这片石滩成了豢养灵蛇的基地,带给他的效益,还要稍稍超出小蓝天中星屑草场的利润。

这些资源,方源暂且放在一边。

让他诧异的是其他事情。

首先,是韩东福地的地灵,那头粉红灵蛇,此刻仍旧存在于石滩之中。

它,它居然没有灭亡!

“咻咻咻?!”粉红灵蛇扭曲小巧可爱的身体,不断摇头晃脑。就连这头地灵都显得分外诧异。

福地被吞并,地灵就会旋即灭亡。这是蛊仙界的修行常识。但到了方源的至尊仙窍,这个常识却被打破了。

这一点,方源之前也万万没有料到。他也做好了福地地灵牺牲的心理准备。

但没想到,事实却是这种样子。

方源的意志,降临石滩,凝聚成他人形模样。

粉红地灵蛇立即游窜过来,开心地向方源吐舌信子:“咻咻咻咻!”

“主人,主人!这是怎么回事,我居然没有死唉。我居然还存在着,主人主人,你好厉害!”

方源意志哈哈一笑:“那是自然。这是我的秘法,属我独创。蛊仙界中独一无二。你以为我会伤害你吗?不过你的确忠心,愿意牺牲自己。来奉献于我。”

粉红地灵蛇双目闪闪发光,对方源产生了浓郁的崇拜之情:“咻咻咻!”

“哇,我的主人好厉害。主人的才情,能比天高!能有您这样的主人,我实在是太幸福了。”

方源意志点点头:“好了。我虽然自创秘法,但吞并他人仙窍,还属于第一次。你现在对这片石滩,还能掌控多少?”

灵蛇便答道:“一如既往啊,主人。主人,你好厉害哦!”

“哦?”方源意志沉吟片刻,又道,“那么小灵蛇,你对其他地方,能够掌控么?”

灵蛇感应了一下,半晌后摇头道:“不行哦,主人,小蛇做不到。”

“仍旧只能掌控原来的地方么……”方源意志一边想着,一边开口,“那么你试试看,能不能走出这片石滩呢?”

灵蛇十分听话乖巧,当场做了实验。

可惜,正如方源隐隐预料的一样,它不能离开石滩半步。

“所以,小灵蛇今后只能管理这片石滩么……”方源心中有些遗憾。

他正缺少一个能替他打理至尊仙窍的能手。之前他还一个劲地希望,有狐仙小地灵这样的角色帮助自己,为自己打下手呢。

毕竟至尊仙窍实在太过宽广,方源神念覆盖都需要时间,打理起来更是麻烦。平日里耗费了方源不知多少精力和时间。

粉红地灵蛇的出现,是一个小小的意外和惊喜。

方源原本想让它,替自己打理整个至尊仙窍,但现在看来,别说是整个至尊仙窍,就连这小北原,地灵蛇都打理不来。

因为它的活动领域,仍旧局限在原来的范围内。

“也罢。这片石滩能够有管理者,被粉红地灵蛇时刻监控,稍有风吹草动,都能察觉。我又何必得陇望蜀呢?”

方源很快收拾心绪,继续冷静思考。

粉红地灵蛇没有毁灭,仍旧存在,只是让方源诧异的一个方面。

另外一个方面,就是道痕。

“我这一次吞并韩东福地,居然将整个福地的道痕,都吸纳进来了。这是怎么一回事?”方源疑惑万分。

按照蛊仙界的修行常识,蛊仙吞并仙窍,获得的道痕会较原来,损失很多。

但方源却直接包圆了!

在吞并福地之前,方源就利用道可道仙蛊,测查过韩东福地中的道痕。最多的是变化道痕,拥有七千余条。其余的水道、土道道痕比较多,有近千条。宇道、宙道、炎道等等道痕,杂七杂八,数量不等,有的一百条都不到,有的则有小几百的规模。

这些道痕,方源全部都吞并了。随着韩东福地的完全融合,也都归于至尊仙窍本身。

道可道仙蛊的探查结果,十分有力地证明了这一点。

“我有些明白了。”方源思考了许久,察觉到了一些端倪。

能够造成这个现象的,恐怕是因为至尊仙体上,异种道痕之间不掣肘内耗的特性!

寻常蛊仙,哪怕是十绝体,吞并其他仙窍,异种道痕之间,就会相互掣肘内耗。蛊仙身上的道痕,就相当于房屋里的主人。而吞并仙窍的道痕,宛若客人。

客人要进入房屋,常驻不走,成为房屋的新主人。原来的主人,就会驱赶他们。

所以最终,原主人接纳了一批他们看得顺眼的客人,这些客人合乎原主人的秉性,便是同流道痕了。至于其他的异种道痕,则纷纷被排除出去,很少有能进入房屋,成为新主人的。

但方源的至尊仙体不同。

这个房屋是敞开的,欢迎任何的客人进驻。异种道痕之间不互斥,使得房屋的原主人热烈欢迎任何的新客人的到来和加入!

“海纳百川,有容乃大。不仅是道痕不互斥,就连地灵都能收容,仿佛就是外界的广袤天地,没有什么不能承载和容忍的。”方源忽然感受到了幽魂魔尊的广阔胸襟和非凡气魄。

至尊仙体乃是至尊仙胎蛊所化,而至尊仙胎蛊更是魔尊幽魂一力推算,耗费无数光阴、精力和资源,险险练就出来。

这只仙蛊,透射出了魔尊幽魂的理念和精神。

方源不禁在心中赞叹:“幽魂魔尊以杀证道,创建魂道,无敌天下。但他死后,魂归生死门,反而更进一步,突破生前的瓶颈,达到了海纳百川、兼收并蓄的人生境地。实在是了不起!”

异种道痕之间不互斥,虽然让方源本体真身显得很脆弱,但这完全是利大于弊的。

小小的弊端,方源可以用更强的防护手段来遮掩。而这些巨大的便利,却是其他手段无法达到的!

吞并了韩东福地,方源的最后一个收获,也是他最主要的目的跳略灾劫。

他也做到了。

原本,他的第六次地灾已经迫在眉睫。

但当他吞并了韩东福地之后,不仅第六次地灾消弭,灾劫降临时间重新计算之外,而且还向前跳了四次。

“我原本渡过五次地灾,即将面临第六次地灾。但现在跳了四次,就是渡过了九次地灾,面临第一次的天劫了!”

蛊仙一次次渡劫,就好像是一次次的得到天地的承认。

前人在这方面的成果,也会随着仙窍被吞并,让后人能享受到一部分。

虽然方源也不清楚这是什么原理,但历史上无数蛊仙实践的结果,都表明:吞并他人福地,的确可以跳跃地灾次数。

这也算是一种渡劫的方法,但实施起来,条件非常苛刻,限制很多。

而且弊端也不少。

“唉,吞并了这个力道仙窍,虽然让我收获不少,但却是牺牲远景,来换取眼前。”南疆玉壶山中内部,黑楼兰也吞并了他人仙窍。

只是她脸色并不好看。(未完待续。)

\end{this_body}

