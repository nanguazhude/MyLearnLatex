\newsection{惊喜发现}    %第二十二节:惊喜发现

\begin{this_body}

%1
五界山脉,五股华光,常年弥漫冲天。

%2
整个山脉,被五色华光,划分为泾渭分明的五块地域。

%3
紫黑色的华光,是当初禁师陶铸仿造的南疆瘴气界壁,青绿色的山脉,是仿造的北原甘草界壁。赤红色的部分,来源于西漠的狂炎界壁。深蓝色,则是东海苍水界壁。

%4
至于白金色山脉,就是中洲圣贤界壁。

%5
七转气道蛊仙戚灾,如今就死在五界山脉中白金色的地域里。

%6
方源吐出一口浊气,抽出插在他眉心之间的飞剑仙蛊。

%7
飞剑仙蛊原本银光璀璨,但此刻却是晦暗无关,毫无气息可言。

%8
这正是仙道杀招暗歧杀!

%9
此招隐蔽非凡,催动起来,毫无七转仙道杀招的气息流露,根本不为外人察觉。

%10
一旦发动,剑势极快,迅雷不及掩耳。

%11
唯一的弊端,就是攻击范围狭小,必须在施展者的百步之内。

%12
这个距离实在太短。

%13
和寻常仙道杀招,动辄百里,千里的攻击范围,完全不能相比。

%14
所以方源不惜冒险,运用见面曾相识伪装身份,故意接近戚灾。距离越近,杀死戚灾的可能就越大。

%15
为此,方源甚至不惜故意和戚灾交手,并在他手中受创,以尽量打消戚灾的疑虑。

%16
戚灾死在此招之下,并不算冤枉。

%17
因为方源用来哄骗他的,乃是大名鼎鼎的见面曾相识。

%18
虽然不是原装货色,但仍旧以八转态度蛊为核心。此蛊催动起来。直接影响人心,让人感知到方源的“态度”。很难起疑。

%19
戚灾虽然谨慎,难以在他本人身上寻找到破绽。不过依照方源的老谋深算。却是在之前的交手中渐渐看出戚荷的浅薄。

%20
利用戚荷,方源冒险,以六转修为杀死了七转强敌。

%21
战绩可谓骄人,若搁在寻常六转蛊仙身上,恐怕恨不得广为人知。

%22
但方源此时,却只是松了一口气,并无丝毫骄傲得意之情。

%23
重生之后,他经历极多,见识过魔尊幽魂对抗浩劫之后。杀死一位七转蛊仙,在他眼里,已然是一件区区小事。

%24
“若非你穷追不舍,我也不会冒险刺杀你了。也罢,就让我看看,你为什么对我要这般追杀到底。”

%25
方源伸手虚抓,便将戚灾的魂魄从他的尸体上,抓了出来。

%26
戚灾修行气道,因此他的魂魄。只是寻常蛊仙魂魄,并不出奇。

%27
人生灯灭,他可不是什么影宗之流,魂魄虽然不甘愤怒。极力挣扎,却只能任由方源随意施为。

%28
方源首先查看一番,确定这魂魄没有任何问题之后。就将他直接塞入九五至尊仙窍。

%29
落到了仙窍之中,大量的魂道蛊虫。便朝着戚灾魂魄蜂拥而上。

%30
搜魂!

%31
方源搜魂的同时,也开始继续催动见面曾相识。渐渐变化成戚灾模样。

%32
他当然没有忘记,还有一位六转蛊仙戚荷,留在外面呢。

%33
杀她定然比戚灾简单,但为了易于成事,稳妥起见,方源还是决定故技重施,将戚荷也突袭刺杀了。

%34
他的这道见面曾相识,并非原版,缺少变化仙蛊,用数千只凡道辅蛊勉强替代。

%35
因此,改变伪装的模样,并非瞬间而成,而是花费了一番功夫。

%36
等到他成功便成戚灾模样之后,他将戚灾尸体收起,施施然飞出五界山脉。

%37
结果,等到他找寻到戚荷的时候,后者却已经死了。

%38
杀了戚荷的,不是旁人,正是她胯下的荒兽气宗狮。

%39
原来这气宗狮,本是受到戚灾的节制奴役。但戚灾一死,气宗狮也就回归自由之身。

%40
气宗狮被戚灾鞭策,一路奔驰,早已经饿了。

%41
戚荷若修其他流派,兴许还有活命的机会。但她偏偏是气道蛊仙,身上气道道痕,气息洋溢,对于气宗狮而言,乃是极致诱惑的美餐。

%42
结果,气宗狮饿了,脱离控制之后,直接反噬。

%43
戚荷不过刚刚晋升而已,哪里是气宗狮的对手,抵抗了几个回合之后,被气宗狮杀死,咀嚼吞下。

%44
可怜她一生苦修,侥幸成仙,刚刚看到一点蛊仙世界的风光,还未开始享受,就死在荒兽口中。

%45
方源追上气宗狮,直接利用飞剑仙蛊,三两下后,就将后者的脑袋洞穿。

%46
银光灿烂的飞剑仙蛊,在气宗狮的脑海中一阵乱绞。气宗狮痛得从高空衰落到地上,发出好一阵子惨绝人寰的哀嚎,最终渐渐挣扎不动,彻底死亡。

%47
如此战绩,终于让方源心底有了一丝安慰。

%48
这才是七转飞剑仙蛊真正的威能!

%49
对付蛊仙、气宗狮这类的对手,飞剑仙蛊一旦刺中致命要害,就能创造出惊人的杀伤。

%50
不像荒兽泥怪,唉呀,这对手毫无要害弱点可言,让方源之前打得好生憋闷。

%51
除了欣慰之外,方源心中更多的还有心疼。

%52
这七转仙蛊太耗仙元了,杀死气宗狮之后,方源的耗用,又增添了一笔。

%53
不动用仙蛊的话,利用凡道杀招,其实也可以磨。

%54
但这时间,耗费得就太长了点。

%55
为了避免夜长梦多,方源还是决定速战速决。

%56
杀死气宗狮之后,方源立即对它开膛破肚,将戚荷的碎尸,都寻找出来。

%57
他对此一一分辨,他想要找的是腹下的关键部位,那里正是寄托着蛊仙的仙窍所在。

%58
戚荷一身老迈皮囊,早就被气宗狮咀嚼成渣,又落入肚中,混淆一堆,被胃酸分解了大半。

%59
方源捣弄了好一阵子,忽然目光一振,将其中一块挑选而出。

%60
和戚灾尸体一样。同样被方源置入仙窍里面。

%61
前文早已提过,杀死一位蛊仙之后。有两种利用方法。第一种,是将仙窍落到五域天地中去。形成福地或者洞天,再攻略福地洞天,收获其中的修行资源。第二种,则是将仙窍直接置入自家仙窍里面,令仙窍自行毁灭,但仙窍中蕴藏的道痕,则遗留下来,补益胜利者。

%62
方源此时采用的是第二种方法。

%63
“经此一战,我身上的气道道痕数目。必然暴涨许多。今后,我若运用气道蛊虫或杀招,增幅效果,定然超越其他流派了。可惜我并不擅长气道,气道境界也相当普通。不过仙窍受益,环境会逐渐改变,将来经营气道方面的修行资源,却是助益颇多。”

%64
方源升上高空,一边隐匿行迹。在云中飞行,一边总结此战收获。

%65
仙蛊什么的,是不用想了。

%66
戚灾魂魄,是一笔隐形财富。目前还在挖掘。

%67
气道道痕究竟能增添多少,方源还估算不出来。但他知道,不管气道道痕增添多少。也比不上他的另一项收获!

%68
这项收获,只是一个发现。但却着实超乎方源的意料,让他惊喜交加之外。又生出许多疑惑和期待。

%69
原来,他深入五界山脉中时,有了一个惊人的新发现。

%70
他的这具全新身躯,居然在五界山脉中,不受任何影响!

%71
按照常理而言,方源是南疆蛊仙,一身南疆气息四下洋溢,不管进入五界山脉中的何种区域,都要受到影响才是。

%72
但偏偏方源进入之后,自由穿行,毫无受阻迹象。

%73
乍然发现此点,方源惊疑之后,便是狂喜,脑海中不可遏制地升起一个猜想:“这五界山脉本是禁师陶铸仿造五域界壁而成,我在这里畅通无阻,不受拘束,难道我还可以自由穿梭五域界壁吗?”。

%74
随后,他又想到身后追兵,连忙表现出一副受阻,吃力跋涉的样子。

%75
现在,方源杀了戚灾,又解决了气宗狮和戚荷,细下想来,越发觉得他的猜测大有可能。

%76
“我这副全新的身躯,并非寻常躯体,可谓无父无母,无根无源。它本是九转仙蛊至尊仙胎所化,若是将其看成仙蛊的话,它能自由穿梭界壁,也就很好理解了。”

%77
五域界壁,针对的是蛊仙,并不针对仙蛊。

%78
当初,凤九歌深入北原,寻找八十八角真阳楼真凶,传递家书的时候,就用了一只六转信道仙蛊,往来穿梭北原、中洲两大界壁,轻松自如。

%79
“如果这个猜想是真的话,我何须在去赶炼定仙游呢?”方源一路疾飞,向着最近的界壁迅速接近过去。

%80
他急着去验证心中的这个猜想。

%81
对于他而言,首先他并不能确认定仙游真的毁了。只是因为特意设定,第一个自毁的仙蛊是定仙游,所以它最有可能。

%82
方源询问泥相问题的时候,重点是确认影无邪的状态。

%83
其次,他想要炼制定仙游,无非就是要穿梭界壁,赶回琅琊福地中去。

%84
为什么呢?

%85
第一,琅琊福地安全,五百年前世足足支撑七波,害了凤九歌的性命之后,才被天庭真正攻破。第二,琅琊福地中有他大量的修行资源,可以让他安心发展。而用宝黄天输送,实在成本太大。

%86
最后,赶炼定仙游,需要关键仙材太古之光。方源尽管有些线索,但他内心深处也知道,希望渺茫。而他最心忧的一点,是自身的灾劫。距离第一次灾劫,只有两个月不到的时间了。

%87
福祸相依,他的福地底蕴这般深厚,资质又大大超越十绝,灾劫威力定然是超出寻常的。

%88
“若是猜想确实,那就舍弃定仙游!直接奔赴琅琊福地,在那里充分准备,应付灾劫!不到两个月的时间……”

%89
他不再是仙僵了,仙窍又好得超凡脱俗,灾劫就像是时刻悬在头顶的一把利刃。

%90
方源望着茫茫云海,紧迫感始终萦绕他的心头。

\end{this_body}

