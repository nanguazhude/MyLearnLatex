\newsection{凤九歌主动撤退}    %第四百零六节:凤九歌主动撤退

\begin{this_body}

吼吼吼!

方源张开蛟龙巨口,喷吐地喷吐出剑光龙息。

龙息仿佛是一道道的银色闪电,以迅雷不及掩耳之势,暴射喷涌,袭向凤九歌。

凤九歌催动一曲阳关不断闪避。

而当他催用出一曲阳关杀招的时候,往往就意味着变招一曲之士的出现。

双方从地下深处,一直杀到高空中去。

你来我往,彼退此进,龙啸声声,鼓钟阵阵。

碧玉歌士!

在追击的途中,凤九歌催出一记分身。

碧玉色的分身,向方源所变化的上古剑蛟冲去。

凤九歌依仗着一曲阳关杀招,在腾挪上占据优势,再加上他本身战斗经历丰富,冷静理智,所以催动杀招从未失误。

方源心中冷哼一声,不管不顾,仍旧杀向凤九歌。

凤九歌心知,碧玉歌士并非方源的对手,别的不说,单就逆流护身印就无法破开。这层防护不破,根本就没有击败方源的希望。

所以,凤九歌操纵碧玉歌士,以纠缠方源为主。

碧玉歌士宛若苍蝇一般,围绕着方源打转,不断干扰他的行进。

几个回合下来,方源不胜其烦,猛地掉转枪头,开始对付碧玉歌士。

凤九歌瞳孔猛地一缩,方源的转变太过于突然,不过凤九歌的反应也很快。

方源冲到了碧玉歌士的眼前!

上古剑蛟巨大的身躯,碧玉歌士就宛若一只苍蝇。

龙吼声起,狂风呼啸,锋利狰狞的龙爪猛地出击,狠狠地击中碧玉歌士。

碧玉歌士身体表面,浮现出密密麻麻的裂纹,发出叮叮咚咚的脆响,像是一颗炮弹,被方源击飞出去。

但是碧玉歌士并没有破碎到崩溃的程度,他形体仍存,还有一战之力。

“狡猾。”方源心底冷哼。

刚刚一击,碧玉歌士根本没有任何攻势,只是被动挨打。如此一来,方源的逆流护身印就无法反击对方的杀伤威能回去,只是依靠上古剑蛟本身的龙爪一击,并不能一击粉碎碧玉歌士。

毕竟,凤九歌身上的道痕可以媲美八转,碧玉歌士的防护能力,在众多曲士当中,可以说是第一,也不为过。

暂时打飞了碧玉歌士,方源旋即再次杀向凤九歌。

但这时,凤九歌已然催生出第二个曲士――天地歌士。

如此几十个回合之后,凤九歌依靠一曲阳关杀招,不断闪避,相继催出四大曲士,纠缠方源。

方源以一敌五,却仍旧是精神矍铄,战意昂扬,把凤九歌追得到处飞跑。

“怎么回事?”

“方源之前维持逆流护身印,已经是他的极限。”

“现在却同时维持逆流护身印、上古剑蛟变,战斗了这么长的时间,居然还游刃有余?”

凤九歌目光深沉,他故意催生一曲之士,就是凝造出更复杂多变的战斗环境,激发方源不断思考,不断耗用脑海中的念头,处理这些复杂多变的战局变化。

若是方源勉强同时催动两大杀招,此时他的脑海中的念头早就不够用了,不至于此时此刻,仍旧有如此强势的表现。

“这么说来,他在维持杀招方面,有了突破性的进展?”凤九歌心中不断猜测。

就在这时,方源的身形微微一顿。

凤九歌密切关注方源的动态,见此心中一喜:“难道说,他终于到达极限了吗?”

但下一刻,凤九歌就将这个猜测彻底抛之脑后。

因为方源催发出了仙道杀招――万蛟!

一时间,蛟龙齐飞,万鳞交汇成一片银色海洋,气势浩荡磅礴,让人心惊胆战。

和之前一战时的类似战况,再一次出现。

万蛟飞腾,将四大曲士重重包裹,不断围剿。

四大曲士无法力敌,相继灭亡。

凤九歌也不去管他,而是催动一曲阳关,暂时脱离万蛟肆虐的范围。

一曲阳关的时限还未到,这让凤九歌游刃有余。

“那么这一招呢?”他望着万蛟群,方源已经混迹其中,无法分辨。

然后,凤九歌缓缓地伸出右手,抚摸左胸心脏的位置。然后他的右手五指,不断抬放,每一次手指头点在自己的胸膛上,都发出一声或激越或沉闷的鼓声。

咚咚咚……

鼓声接连响起,很快蔓延整个战场。

凤九歌不需要分辨,究竟那么多蛟龙中谁是方源的本体。因为这记仙道杀招本就是群攻杀法,覆盖范围很广。

方源本体被笼罩其中,立即受到影响,感受到心脏开始微微乱跳,之前稳健的节奏被打乱。

方源吃了一惊。

要知道他可是身罩着逆流护身印,却没有防备得住这记仙道杀招。

“不,这记杀招的大部分威能仍旧被逆反回去,只是有一小部分,透过护身印,影响到了我。看来天庭方面,破解我的逆流护身印杀招,已经得到一些成果了。”

方源心中一叹。

没办法,他使用逆流护身印的次数已经不少,天庭方面更是有着天下可数的智道大能紫薇仙子,再加上星宿棋盘,获得这些成果并不奇怪。

砰砰砰。

一声声的闷响,方源身边的蛟龙力道虚影,在凤九歌的鼓声中相继自爆,化为乌有。

“这记仙道杀招乱心鼓点音,乃是由天庭传授,有着智道效果。一旦中了,便可干扰蛊仙念头运转,影响杀招的维持。方源,你还能支持到什么时候呢?”凤九歌心中期待。

“很厉害的杀招,而且是特意针对我的。”方源仔细品味着这招的厉害之处。

此时,在他的至尊仙窍当中,大量的人形万我分身,也在不断地自爆,损毁的数量规模,比万蛟还要庞大。

仔细看,这些万我分身,已经和之前有着明显区别。

他们的容貌,比之前要更加栩栩如生。不仅如此,神态也更加生动。

此时此刻,他们都盘坐在地上,似乎在沉思冥想。

乱心鼓点音杀招之下,这些万我分身,不时的就发生自爆,炸成一团白烟,然后随风消散。

“增添了我意之后,这些万我分身就有了智道的灵动,反而更被凤九歌的这记杀招克制么。”

方源分出一丝心神,投入至尊仙窍,察看损失。

几乎每一个呼吸,都有数十个万我分身消亡。

“不过,这些消亡的万我分身当中,并非都是被凤九歌的杀招所害。而是大部分因为维持我的逆流河护身印,以及上古剑蛟变化。刚刚催动万蛟杀招的时候,可是一次性地损失了上千个万我分身呢。”

没有错。

方源之所以能够在催动了逆流河护身印的基础上,还能运用其他的杀招,就是因为仙窍中的这些万我分身帮助着他。

一直以来,方源都困顿于此,逆流护身印虽然强大,但是牵扯他的心神太过巨大。

和凤九歌一战,他受到一曲之士的启发,得到了一个灵感。

“我也有万我分身。如果让这些分身,同时帮助我一同思考的话,无疑能大大提升我的战斗能力!”

要实现这一点,有两个最大的难点。

第一个难点,原本的万我分身是无法思考的。要思考,就需要智道的念头、意志或者情感来辅助。

第二个难点,就算是有大量的念头、意志或者情感,分散到了各处分身当中,也很难将这些力量统筹到自己身上。

方源突破第一个难点,是利用自爱仙蛊和爱意仙蛊,两者本身就很搭配,方源辅助了很少的一些凡蛊,就形成仙道杀招,催发出海量我意!

第二个难点,方源从紫山真君的遗藏中找到了解决方法――仙道杀招集思广益!

当然,这两个难点虽然突破,最终还得依靠方源的智道宗师境界,才能彻底改造万我,将其提升到一个全新的层次当中去。

“凤九歌,你的这记杀招还不够看啊。”方源在心中冷笑。

因为此战之前,他早已经做足了充分的准备。

他在至尊仙窍当中,囤积的万我分身多达百万规模!

死伤个成千上万,根本就是小小瘙痒罢了。

仙道杀招――三息后现!

仙道杀招――万蛟!

仙道杀招――剑痕索命!

一瞬间,方源的身上猛地升腾起了无数蛊虫的气息。

这些气息杂乱无比,有无数凡蛊,有大量的仙蛊,一起升腾出来,形成磅礴大势!

“怎么可能?!”凤九歌悚然动容,“方源竟然同时催动多个仙道杀招!!”

他彻底地吃惊了。

难以想象!

他一直以为,方源就算是同时催动了两三个杀招,也是勉强的,快要到达自身极限的。

但此时此刻这样的景象,让凤九歌彻底明白,原来方源在这方面的提升,是如此的巨大,大到远超他想象的地步。

这个曾经制约方源的弱点,已经不再是弱点,反而一跃成为方源的巨大优势!

第一记万蛟还有大部分残存,第二记万蛟再次催动,形成充斥天地的蛟龙银海。

三息后现让方源敏锐洞察,剑痕索命在数万剑道道痕的增幅之下,对凤九歌也造成不小的威胁!

凤九歌咬紧牙关,不退反进,冲向蛟龙群中。

轰轰轰……

一声声如雷霆般的炸响,响彻天地。

凤九歌在万蛟围杀中左冲右突,上下纵横。

他越战越吃惊。

“不是偶然的爆发,方源真的做到了这一点,并且还留有余力!此地不可久留,撤!”趁着一曲阳关还未到时限,凤九歌主动撤离战场。

他竟是被方源一人逼得暂逃!(未完待续。)

\end{this_body}

