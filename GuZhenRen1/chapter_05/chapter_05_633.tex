\newsection{前世至爱}    %第六百三十六节:前世至爱

\begin{this_body}

%1
依照方源的侦查手段,这座海岛在他眼下几乎毫无秘密。 .更新最快

%2
他发现被功德碑传送过来的这座小岛,十分普通、平凡。小岛上的物产并不多,驻扎着一个小渔村,有着一些人生活着。

%3
有渔村,自然就有一口元泉。其中当然是有蛊师,但按照这样贫瘠的环境和人口,能够有个四转蛊师那就顶天了。

%4
方源此次接受的是采集海底地沟黑油的任务,果然在他的侦查中,很快就发现海岛的边缘就有黑油外溢的一丝些微迹象。

%5
换做其他蛊仙,他们兴许立即就下海搜集地沟黑油去了。但方源并不打算这么干,见到功德碑的第一眼,他在心中就开始揣摩乐土仙尊的用意。

%6
隐形匿迹进了渔村,这座小渔村人口并不多,蛊师也只有五六位,最高修为的是一位三转蛊师,完全和青茅山古月山寨不能相比。当年古月山寨的周围,就附庸了这样的村庄近十座呢。

%7
方源直接找上村中唯一的三转蛊师。

%8
这是一个老人,白发苍苍,满脸皱纹丛生,从别人的称唿中泄露了他的村长身份。

%9
方源相当满意。

%10
位高权重,且又年岁颇高,知道的东西定然是不少的。

%11
搜魂!

%12
方源动用仙家手段,老村长的秘密顿时全部暴露。

%13
一瞬间,方源得到了大量的情报,极其详实。和他料想的一样,这个小村庄非常普通。而像这样的小村庄、小海岛在周围还有不少。

%14
“这座小渔村看来是没有什么猫腻蹊跷的。”方源明白过来,便又悄悄离开了村庄。

%15
老村长只感觉恍惚了一下,微微诧异之后,就又做自己的事情去了。

%16
方源的搜魂手段,面对凡人蛊师实在超越了太多。

%17
方源暗自离开这座小海岛,飞向周边的海岛去。

%18
但是之前在功德碑上海岛的一幕,再度在方源的身上上演。

%19
哪怕他已经看清海平面上,冒出了一个小海岛的尖儿,但任由他如何飞行,双方的距离就是无法缩短。

%20
尝试了全部的手段,方源依旧徒劳无功。

%21
“乐土仙尊早年修行音道,后来专修土道,这种咫尺天涯的手段也是他的拿手好戏啊。”

%22
人族史上有着不少的战例,乐土仙尊就站在原处,任由敌人出手。但敌人的拳脚、杀招,都飞不到他的身上去,哪怕距离看似很短。

%23
“除此之外,乐土仙尊还有一个最令人向往的手段,就是能营造出乐土。”

%24
乐土……

%25
这也是乐土仙尊的称号主要由来。没有人知道乐土该如何营造,但乐土仙尊为五域两天留下了大量的乐土。

%26
乐土有一个最显着的特征,那就是无灾无劫!

%27
“这片龙鲸洞天显然也被乐土仙尊点化,成为了一片乐土。”

%28
从之前老村长的魂魄中,方源搜刮到的情报,也在很大程度上证明了这一点。若是有灾劫,老村长即便自己感受不到,村子的正典史册中也应该有相关记载,但这一切相关的情报都并不存在。

%29
既然其他的小海岛方源进不去,他便改变方向,一头扎入水中。

%30
随着他不断深入,海水的压力也越加庞大。

%31
这种水压是所有的采油蛊师需要面对的难题,但是对于八转蛊仙方源而言,那就是微不足道的东西。

%32
对于蛊师而言,海水或许很深,但在蛊仙看来靠着小岛,海水算得上浅。

%33
方源很快就发现视野中,出现了一大片的黑沉暮霭。它像是一条巨型海带,从东向西,一直延伸出去。

%34
海底的这座地沟非常壮观,平凡小岛几乎就在地沟的边缘。而从老村长那得来的情报,周围的小岛附近,也都有海沟延伸。

%35
地沟中盛产黑油。

%36
不管是普通的地沟,还是海底的地沟。

%37
黑油是一种仙材,隶属于食道。品级很高,产量很多,且易于开采。

%38
这好像是食道流派的一个特征,产量高、物廉价美、亲民。

%39
黑油对仙僵极有帮助。将黑油洒在死窍福地中,可以延缓福地的崩解速度。放到外界,对于其他蛊仙而言,也是用途极为广泛的炼蛊材料,市场广大。

%40
僵盟最主要的经营项目之一,就是黑油。这个地位,就好比胆识蛊买卖对于方源。但黑油的贸易规模,可比胆识蛊买卖大得多了。

%41
“这里的黑油大概有六百万斤,我若全力出手,半天之内即可搜刮得一干二净。但是我不能离开这座海岛太远。黑油虽然也能流动,但比水可缓慢多了。我搜刮了这里,等到其他段的地沟的黑油补充流淌过来,然后再采油,效率太低。”

%42
黑油本身是很容易采集的,当然这是对于蛊仙而言。

%43
对于凡人蛊师,相当有难度。

%44
有一个专门的分类,就是采油蛊师。他们专门深入地沟,捞取地沟黑油,这些人讨生活的技巧,一般都是家传。采油蛊师有着自己的一套蛊虫,相互配合起来,能收取少量的黑油。

%45
虽然他们战力不强,但因为收取的乃是仙材,财力富足,地位颇高,很受欢迎。

%46
方源五百年前世,就曾经在东海当过一段采油蛊师。那是他的一段机缘,差一点就当了某个采油老蛊师的上门女婿,采油的手段也是这老蛊师选择性地传授给方源的。

%47
靠山吃山靠水吃水,小海岛上的蛊师就是主要依赖采集黑油修行、生存。方源搜魂的老村长,就是村子里最厉害的采油蛊师。即便年岁很大了,但他每年都还是会下水,采集

%48
此时见到这里的黑油,方源的脑海中不由自主地浮现出前世的一些记忆。

%49
曾经的他作为采油蛊师,深入海水中采集黑油,可谓是如履薄冰、战战兢兢。首先解决的是唿吸的难题。其次,他不能太过深入海底,海水的压力就能把他挤爆。然后海水中并不平静,有着大量的勐兽。最后,采集黑油也需要独门手段。

%50
凡人采集仙材,没有一些独门手段怎么可以?

%51
尤其是黑油若是采集不当,就会污染蛊师本身,令蛊师身着黑色油污,顽固不褪,时间一长就带有恶臭,臭不可闻不说,还会逐渐腐蚀蛊师本身的肉体凡胎。

%52
前世五百年,方源就受过黑油的污染。那是传授他手段的老蛊师故意保留,意图是逼他成为自己的上门女婿。

%53
当时是很愤怒,感觉被算计,但现在方源回想起来,却是苦笑一声。

%54
老蛊师膝下无子,后继无人,只能出此下策,不得已将家传的手段传授给方源这个外来蛊师。他是逼不得已的,隐藏手段、留下一手有什么错?人之常情!

%55
当时的方源,却颇有一些身陷绝境、走投无路之感。

%56
没有特殊的手段,无法清除黑油的污染。就算方源知晓这种方法,还得有一系列的蛊虫才行呢。这两者就形成天堑,逼他低头。

%57
想到这里,方源又有了一丝当初的心境。

%58
当初的他,自然是非常焦急和痛苦,又对捞取黑油的利润十分向往,心底暗暗发誓若有能力,一定改变现状!比如有些水道蛊虫,可以令他自由潜游,灵活非常。又比如设计出一些蛊虫,更能抵抗海水压力。

%59
现在的他能力完全够了,仙蛊一大把,轻易间就能炼出大量的优秀凡蛊来。

%60
但此刻的他,却也完全不需要这些蛊虫了。

%61
人生的很多无奈就是这样。面对当下,有些迫切需要的东西却没有。等到有了却不再需要。

%62
所以说,瞌睡了送枕头这种事情,也叫做一种机缘,遇到了就要珍惜。

%63
“当初我在采集黑油的时候,险些身亡。”

%64
“不过,福祸相依,竟在濒死前遇到了沫儿……从这种角度上,老蛊师也算是做了一件好事。”

%65
方源的嘴角流露出一丝复杂深沉的笑意。

%66
他的心底再次升腾出一位白色的丽影。

%67
她的面貌似模煳,又似清晰。

%68
她姓谢,名晗沫。

%69
谢晗沫。

%70
这个名字方源一直从未忘记,因为这是他前世的一生至爱。

\end{this_body}

