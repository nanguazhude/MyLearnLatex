\newsection{获取五光山}    %第一百三十一节:获取五光山

\begin{this_body}

%1
呱呱呱!

%2
刘青玉的福地中,一只呆头鸭子,正对着方源叫喊。

%3
这就是刘青玉死后,执念结合福地中的天地伟力,从而形成的地灵。

%4
地灵的外表,千奇百怪,一头鸭子地灵并不算稀奇。

%5
方源仔细聆听。

%6
虽然他耳中听到的是鸭子呱呱的声音,但反应到内心世界中,却玄妙地变换成了能够让方源自如理解的意思。

%7
事实上,黑凡天灵虽然外形是一座黄铜大钟,也可以和其他人交流。可惜,黑凡临死前,让自己的仙窍世界吞并了太古青天碎片,导致天意入主,使得天灵懵懂愚昧。

%8
“继承这个青玉福地,成为地灵之主的条件,竟然是这个么……”方源眼中精芒微微一闪,透露出些许惊喜的意味。

%9
刘青玉虽然是被方源杀死,但是他的执念,却不是找方源报仇。也不是找影无邪等丢弃他的人算账。

%10
而是信道传承的关键线索的下落。

%11
“信道传承的线索,就刻印在飞剑仙蛊之上。”方源取出飞剑仙蛊,当场向鸭子地灵展示,并且详细解释了一番。

%12
片刻之后,鸭子地灵围绕着方源团团转,不时地用鸭嘴碰一下方源的白色靴子,亲热无比。

%13
方源成了这片福地的主人!

%14
然而,这片福地却是荒芜一片。

%15
并非是刘青玉本身底蕴不足,而是在他加入了影宗之后,仙窍中的各种修行资源,都被影无邪勒令上缴了。

%16
影无邪没有收纳刘青玉的两只仙蛊,这是想保持他的战斗力,让他为自己卖命。

%17
但影无邪收缴了他的修行资源,这是釜底抽薪之计,今后刘青玉根本无法翻身,想要继续修行下去,自己缺乏资源。只有更加依靠影宗。

%18
以上原委,方源稍稍一琢磨,便想通了。

%19
至于刘青玉的蛊虫,早已经自毁了。

%20
两只仙蛊是这样。其余的凡蛊也是如此,没有给方源剩下什么。

%21
“这片福地中,究竟还剩下什么?”方源扫视周围,皱着眉头询问地灵。

%22
青玉福地被搜刮得底朝天,几乎和被方源搜刮过后的黑凡洞天相差无几。

%23
方源此时此刻。也多少体会到当初楚度的一些心境了。

%24
鸭子地灵:“呱呱,呱呱呱!”

%25
方源眉脚一挑:“带我过去。”

%26
鸭子地灵却无瞬移之能,带着方源来到目的地。

%27
一座色彩斑斓的山峦,不高也不矮,展现在方源的视野当中。

%28
山石绽放着绿、红、黄、蓝、白五种微光。山上无树,却长满了各种小草。这些小草的颜色,有十种颜色,相互辉映,比花海还要艳丽几分,缤纷夺目。令人眼花缭乱。

%29
“五光山、十色草。”方源口中喃喃。

%30
这是光道蛊仙,通常都会经营的一种资源产地。

%31
特殊的环境,会令这座山峦中产出大量的光道凡蛊。

%32
很显然,这座五光山被刘青玉灌注了许多心血,规模很大。

%33
“影无邪没有拔山仙蛊,没有手段拔取这座山峰,所以只好留在这里。”方源心中猜测出了真相。

%34
树有树根,山有山根。

%35
山不容易移动,动用蛮力移山,只会让山峦崩溃。

%36
而挪移山峦的手段。却是当世稀有。

%37
当初,方源拿走落魄谷时,就连八转蛊仙凤仙太子都诧异。

%38
影无邪虽然继承了一部分影宗的财产,但显然缺乏取走五光山的手段。

%39
他原本是有拔山仙蛊的。可是通过那场交易,他卖回给了方源。

%40
想必当初,他见到这座五光山无法下手,心里也有些不爽吧。

%41
总而言之,这座五光山就便宜方源了。

%42
没有犹豫,方源动用拔山仙蛊。将这座五光山顺利地挪进了自己的至尊仙窍之中。

%43
他将这座山峰,放进小南疆里。

%44
小南疆多山,本来就是土道道痕比较集中的地方。现在五光山增添进来,立即将黑凡洞天里的那些山峦都排挤下去,继仙山都只能屈居第二了。

%45
毕竟继仙山只是用于传承,而此时山上的传承都被取走了。五光山成为小南疆的第一山峦,实至名归。

%46
“总算是有些收获了。”方源吐出一口浊气。

%47
若是没有这座五光山,他就亏本得厉害。

%48
别看这场战斗,他虽然胜利了。对于方源而言,胜利本身并无多少意义,战利品才是重点。

%49
可惜就算如此,收获了五光山,方源总结下来,他和刘青玉的这场战斗,还是有些亏本。

%50
没办法。

%51
他现在都用青提仙元,而很多仙蛊都是七转,仙道杀招种类多,运用得也多。

%52
一场战斗下来,青提仙元的消耗往往都是十分剧烈的!

%53
当然,福地是可以相互吞并的。

%54
不过,刘青玉主修光道,这座福地便是光道福地。方源光道境界只是普通而已,这又是七转福地,自然无法吞并。

%55
接下来的时间里,方源询问鸭子地灵,再次得到不少情报。

%56
鸭子地灵还记得本体生前的不少事情,这使得方源对影无邪布置下的陷阱,了解得更加清晰。

%57
还有关于信道真传的一些情况,方源也知晓不少。

%58
挖空了情报,方源留下一些凡蛊,交给鸭子地灵,嘱咐它没有自己的允许,永远不能开启福地的门户。一旦有敌人入侵,就第一时间透过宝黄天,通知方源等等。

%59
刘青玉的七转红枣仙元,还遗留了一些,方源分毫未取,都留给了鸭子地灵。

%60
“呱呱呱。”鸭子地灵眼泪汪汪,挥动一只翅膀,对方源的背影依依惜别。

%61
方源头也不回,直接出了福地。

%62
福地门户,在他身后旋即关闭。

%63
“这座七转福地,只要谨守门户,寄托虚空,一般都不会有人发现。”

%64
“不过,天意害我,要针对这片福地,恐怕也有不少方法。”

%65
“这片七转福地能保住就保,保不住就算了。毕竟利用价值太低。”

%66
方源可没有这闲情逸致,去提升什么光道境界。而且他此时修为,距离七转还要老长一截的路要走呢。

%67
继续追杀影无邪吗?

%68
方源有些犹豫。

%69
从鸭子地灵身上得到了更多的情报,方源对影无邪等人的实力,有了更加清晰的了解。

%70
影无邪一方,共有四位蛊仙。一位七转石人蛊仙,名为石奴,擅长土道。黑楼兰如今是力、火兼修。太白云生的宙道仙蛊失去了,但是却掌握了云道仙蛊。他本身就兼修着云道。

%71
至于影无邪,肉身只是力道仙僵。他仍旧运用魂道仙蛊,修为最低,但战力搞不好不是第一,就是第二。

%72
方源若有太古上极天鹰,必定直接杀上去!

%73
但现在,上极天鹰只是一颗鸟蛋,方源尽管仙蛊众多,但估算了一下,凭借自己的实力和智谋,对付影无邪这些人,胜负只能五五分。

%74
“这才多久!影无邪这些人,居然实力增长得这么多,实在是超乎我的意料。”

%75
方源的成长程度,让影无邪等人震惊。影无邪这些人的实力增进,也让方源感到惊讶。

%76
他原本以为,上一次交易对影无邪等人构成了极其重大的打击。没想到仅仅这么一段时间过去,他们居然又复起了。

%77
影宗就变已经残破不堪,但遗泽绝对不容小觑。

%78
影无邪这些人,实乃方源的心腹大患!

%79
因为至尊仙胎蛊,影宗和方源的矛盾完全不可调和。

%80
影无邪现在不想对付方源,只是因为营救魔尊幽魂本体更加重要。他的策略是明智正确的。

%81
方源之前没有对付影无邪,是因为能力不足。他必须要应对地灾,努力适应并经营至尊仙窍。

%82
现在他掌握了黑凡真传之后,有了实力,就立即过来对付影无邪。

%83
影无邪一旦救出魔尊幽魂本体,方源心知肚明,自己根本没有胜利的机会。因为对方是十大尊者之一,哪怕已经故去,自己也绝非敌手。

%84
当然,还有一个很重要的原因。

%85
那就是智慧蛊。

%86
智慧蛊没有丢失,目前暂时存放在琅琊福地之中。方源又有了曾经的力道仙僵躯壳,就差关键一点他害怕这具仙僵肉身中,藏有影无邪针对他的魂魄,布置的陷阱。

%87
方源不敢冒然行事。

%88
他也想过,利用其他蛊仙的魂魄,将他们塞进仙僵肉身中,进行试验试探。

%89
但此法又有许多弊端。

%90
一来,这种试验要偷偷进行,他不能明目张胆地将自己的仙僵肉身拿出来。这会让琅琊地灵识破许多秘密。

%91
二来,就算利用蛊仙魂魄试验,也未必可靠。因为方源不清楚,影无邪可能布置出来的陷阱,究竟是针对弱魂,还是强魂,或者是单独针对天外之魔的魂魄呢?

%92
但利用智慧蛊的这个难题,偏偏又是目前方源发展的最大制约。

%93
所以,方源想出了这个东海的计划。

%94
主要目的,就是对付影无邪。

%95
若是成功,从影无邪处搞不好,就能直接查清楚这仙僵肉身中到底有没有什么陷阱?若有陷阱,该如何应对解决?

%96
影无邪想要压制方源的发展速度,方源也担忧影无邪发展得太快,营救出幽魂本体。

%97
所以他这一次,前来东海,信道传承都是次要的。

%98
他要直捣黄龙。

%99
影无邪才是主要目标!

\end{this_body}

