\newsection{凤仙追杀}    %第七百四十七节:凤仙追杀

\begin{this_body}

%1
天庭。

%2
中央大殿。

%3
紫霞漫延,澎湃汹涌。忽然,霞光一滞,像是奔流的大河遭遇了大坝的拦截。

%4
紫薇仙子轻咦一声,睁开双眼。

%5
她推算正有进展,徐徐推进,成功在望,但忽然间阻力大涨。

%6
这不同寻常的变化,就像是她之前在推一扇木门。

%7
木门已经摇动,推开大有希望,一旦推开,紫薇仙子就能获知琅琊福地的精确位置。

%8
但现在这扇木门,忽然间变成了一扇沉重的铁门。

%9
“怎么回事?”

%10
“之前方源明显是出手阻击我,但是智道手段层次不行,难以抗衡。”

%11
“现在忽然奏效,这种手段……等等,我明白了,他是将琅琊福地吞并了。”

%12
紫薇仙子眼中精芒爆闪,想到了答案。

%13
紫薇仙子之前推算方源的位置,十分艰难。当方源拥有阎帝杀招后,更是没有成功的希望。

%14
方源只要将琅琊福地吞并,那么琅琊福地就属于至尊仙窍的一部分。

%15
紫薇仙子接着推算琅琊福地的位置,不就是推算方源吗?

%16
所以,紫薇仙子的种种推算,都被方源的阎帝杀招阻挡下来。

%17
方源的阎帝杀招,可是借助了盗天魔尊的传承——九转杀招鬼不觉的力量,因此紫薇仙子感受到极大的阻碍。

%18
紫薇仙子思绪闪动,手上动作不断,接连交换了好几个智道手段。

%19
一番刺探下来,紫薇仙子立即确定:“这些反应都和之前推算方源时一致,看来他真的将琅琊福地一口吞了!”

%20
“要吞并七转福地,至少要宗师境界。方源的炼道,也有宗师境界?”

%21
紫薇仙子发现自己再一次低估了方源。

%22
她的神情阴沉如水。

%23
“仙窍只能以小吞大,方源的仙窍究竟是什么?居然可以直接吞并那么庞大的琅琊福地?”

%24
紫薇仙子现在还不知晓至尊仙窍的秘密,她暗暗吃惊,方源展露出来的底蕴已经有深不可测的架势了。

%25
“可恶,方源得到了琅琊福地,底蕴暴涨。这些异人蛊仙干什么吃的?居然眼睁睁地就让方源吞并了家底?这群蠢货!”

%26
紫薇仙子恨得牙痒,毛民等异人蛊仙太令她失望了。

%27
“等等,这恐怕也是方源的谋算!”

%28
“他一边利用这些异人蛊仙,来对付我天庭。同时,也利用天庭,帮助他削弱这些异人蛊仙。”

%29
“经过大战,这些异人蛊仙一定伤亡惨重,因此被方源强行吞并。”

%30
“说不定,他还利用了我这次强行推算的危机,来诱骗这些异人蛊仙同意!”

%31
紫薇仙子在这一瞬间,想了很多。

%32
方源留给她的卑鄙阴险的印象实在太深刻了,怎么把方源往坏里想都不过分。

%33
紫薇仙子大感郁闷!

%34
她感觉这一次进攻琅琊福地,很可能成全了方源!

%35
“方源,我不会让你得逞的。凤仙太子……”紫薇仙子眼中怒意连连,立即联络远在北原的凤仙太子。

%36
凤仙太子早已得到命令,潜伏在松尾岭中,原计划是有机会的话,就配合天庭的蛊仙,一齐攻略琅琊福地。或者当北原蛊仙界有所察觉时,他及时现身,阻挡他们,拖延时间。

%37
“凤仙太子,方源很可能现身。你速速查探,一发现目标尽力纠缠。”紫薇仙子下令。

%38
“明白。”凤仙太子忽然话锋一转,“等等,我已发现他了!”

%39
“你一个人要十分小心!”紫薇仙子连忙关照,“方源狡诈阴险,偷道仙蛊的手段可能来自于琅琊派,也可能就是他的。我已派遣蛊仙出发,潜入北原,前来接应你去。但短时间内,你是没有帮手的。”

%40
凤仙太子面色凝重:“明白。”

%41
他是八转蛊仙,对付一位七转蛊仙,居然要小心翼翼?

%42
但他心中没有一丝的不妥或者荒谬的感觉,而是认为理所应当。

%43
不只是凤仙太子,紫薇仙子和一干天庭蛊仙也是如此。

%44
之前的轻视,只能显露自身的愚蠢。

%45
雷鬼真君、陈衣战死,更证明方源这个魔头非同寻常,根本不能以常理度之。

%46
别人可以轻忽,但轻视方源那是和自己过不去!

%47
“果然来了。”方源在半空中疾飞,他掉头看了一眼,凤仙太子已经迅速和他拉近距离。

%48
吞并了琅琊福地,动静并不小,方源本体暴露在外。

%49
凤仙太子一直埋伏在附近,立即感应到,追杀过来。

%50
方源向前飞逃。

%51
他不想和凤仙太子作战。

%52
经过刚刚的大战,他的状态并不是巅峰,仙元消耗很多。

%53
同时,他还要防备紫薇仙子的推算。阎帝杀招紧贴在他的魂魄上,遮蔽他的踪迹,也消耗着他的魂魄底蕴。

%54
“方源你往哪里逃?”

%55
“你若是个男人,就回来和我大战三百回合!”

%56
“你这个孬种,只会抱头鼠窜吗?”

%57
……

%58
身后的凤仙太子态度嚣张,大骂不休。

%59
但实际上,凤仙太子十分戒备,看似张狂,其实心态摆的很正。

%60
方源冷笑,不管凤仙太子再怎么挑衅自己,他也无动于衷。

%61
方源就算不是巅峰状态,就算修为只是七转,依凭逆流护身印、落魄印,还有天婆梭罗、太古石龙的帮衬,也完全可以和凤仙太子交手,不会落到下风。

%62
但这有什么利处?

%63
和凤仙太子交手,一个不小心,就会发生意外。

%64
毕竟这里可是北原十大凶地之一的松尾岭!

%65
方源会暴露种种手段,泄露珍贵的情报,仙元消耗会更加严重。天庭的援兵说不定在战斗中出现,到那时局势就又要重新评估了。

%66
王不因怒而兴师!

%67
不管正道、魔道,管理不好自己的情绪,不管暂时的成就有多大,本质上仍旧是跳梁的小丑。

%68
“到了,就是这里。”方源疾飞,当他看到前方山峡中,出现了一片巨大的寒潭,他心中顿喜。

%69
凤仙太子早已出手,在方源身后,对他狂轰滥炸。

%70
两人一追一逃,迅速接近寒潭。

%71
寒潭中掀起一股惊人的波涛,一头太古寒蛟显露而出。它头有独角,有十多丈长的身躯,白色鳞片层层叠叠,一出现方圆百里空气温度急剧下降。

%72
太古寒蛟嘶吼,警告方源和凤仙太子,不要靠近它的领地。

%73
方源找的就是这这太古寒蛟!

%74
琅琊地灵当初迁徙福地,从月牙湖迁徙到松尾岭,具体位置一直都没有告诉方源,也是防了他一手。

%75
如今琅琊地灵被方源镇压,琅琊福地的具体位置,以及周围的势力分布,也因此被方源掌握。

%76
太古寒蛟警告无效,方源和凤仙太子仍旧冲着自己来。

%77
它勃然大怒,喉咙涌动,张开大口,就吐出无穷寒息。

%78
寒息如冰霜组成的潮汐,洁白无瑕,所到之处,闪烁晶莹的冰光。

%79
方源撑起逆流护身印,身形没入漫天的寒息之中。

%80
凤仙太子大叫不妙,也连忙追杀进去。

%81
寒息铺天盖地,大大干扰了凤仙太子的侦查杀招。

%82
方源有逆流护身印,可保自身安然无恙。

%83
寒息喷过,天空中无数马车大小的冰块,坠落下来。

%84
凤仙太子浑身燃火,只是眉眼间染上了一层霜意,高高地悬浮在空中。

%85
抵挡住了寒息,凤仙太子却没有一丝的得意,反而大为恼怒而焦急。

%86
因为他跟丢了方源。

%87
“方源有见面曾相识杀招,他恐怕是变作了冰块,坠落下去了!”

%88
凤仙太子知晓一些重要的情报,同时战斗经验极其丰富,立即想到了方源失踪的最大可能。但现在要让他去侦查方源的真身,那就不容易了。

%89
尤其是他刚准备行动,太古寒蛟就向他扑来。

%90
“蠢畜生!”凤仙太子气得大骂一声。

%91
太古寒蛟智力有限,方源消失了,它便下意识以为是方源被它消灭了,剩下的就是身上燃烧着火焰,极其醒目的凤仙太子。

%92
比较起方源七转的气息,凤仙太子八转气息洋溢,和太古寒蛟是同一层次。

%93
在太古寒蛟的眼中,方源不足道,真正有威胁的当然是凤仙太子!

%94
轰轰轰!

%95
凤仙太子迫不得已,和太古寒蛟展开激战。

%96
同时,他抓紧一切的机会,对地面上的冰块展开侦查。

%97
他很快就有了发现。

%98
一块白冰坠落到寒潭中去,并未浮上来,而是变作了一头上古寒蛟,向寒潭底部钻去。

%99
“方源,你逃不了的!”凤仙太子立即舍弃了太古寒蛟,杀向寒潭。

%100
方源进入寒潭,一味向下钻。

%101
很快,他就看到大量的寒蛟,有上古寒蛟,还有荒兽寒蛟。

%102
这是寒蛟的大家族!

%103
这些寒蛟只是好奇地看向方源,觉得它是同类,没有丝毫攻击的欲望。

%104
凤仙太子杀进寒潭,那待遇就完全不同了。

%105
许多上古寒蛟纷纷杀向他,结果被凤仙太子烧成肉干。

%106
上古寒蛟混乱一片,少部分激起凶性,纠缠凤仙太子,大多数则四散奔逃。

%107
方源心中大乐,直接混入进去。

%108
“方源,你有种的出来!”凤仙太子再一次失去了方源的踪迹,气得在潭水中大吼,恐怖的音浪激起无数寒潭狂流。

%109
太古寒蛟也悲吼一声,子子孙孙就在它眼前,遭受人族蛊仙的屠戮,血染寒潭,这让它怒到了极点。

%110
它凶狠无比地杀向凤仙太子!

\end{this_body}

