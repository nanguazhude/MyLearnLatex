\newsection{功德倒扣}    %第八百九十六节:功德倒扣

\begin{this_body}

看到任修平和童画都不能幸免,将自家仙材交给了方源,庙明神等人的心都提了起来。

若是方源勒索他们,他们该怎么办?

拒绝吗?

恐怕是不成的。

庙明神等人不可能在这片龙鲸乐土中待上一辈子的,他们还要在东海中修行。

尽管方源向他们索要仙材,十分不合道理,本质上就是勒索,但为了今后长远打算,庙明神等人还是会低头的。

这个世界向来是先讲力量,再讲道理。

就在庙明神等人已经有了充分的心理准备的时候,方源却没有看他们,而是转身面向功德碑。

沈从声也像是将庙明神这伙人忘记了。

唯有任修平、童画看着庙明神一伙,前者目光怨毒憎恨,后者却是情绪相当复杂。

方源看着功德碑面。

毫无疑问,之前他接取的剿除魔仙沈伤的任务,已经彻底失败了。

功德碑面上,这个任务已经消失不见。不仅如此,方源还发现自己的功德被倒扣了一大笔。

其他人也没有幸免,都有功德倒扣的情况。

“太上大长老,您的功德!”沈家蛊仙惊呼出声。

沈从声脸色铁青,面色凝重。

他的功德是被扣得最严重的,比方源还要多得多。

方源虽然从第一暴降到最后一名,但好歹功德还是个位数。沈从声的功德却直接呈现负数!

这个情况让方源也感到很是惊奇。毕竟上一世可没有出现这样的奇葩现象。

“沈兄,看来是因为你主动破坏海底大阵,释放出了沈伤,所以功德碑才会对你严惩。而我也是消极怠工,甚至收走了海底大阵,所以功德也被狠狠倒扣了。”方源分析道。

沈从声这时面色微微一变,有了一抹焦急之色:“我刚刚又得到了功德碑传来的讯息,我必须在十天之内,将功德回涨到零以上。否则的话,我就会被踢出龙鲸乐土,并且今后再没有资格进来。”

方源却是心头微动,他拥有智道造诣,立即分析出这番话后的情报。

“看来龙鲸乐土还能多次探索。今后我们还有机会进来?但该怎么进来?”

最大的难题是及时地找到龙鲸乐土的位置。

方源不着痕迹地看了看庙明神。

“庙明神能够感应到苍蓝龙鲸的位置,进入这里。是不是就是这样的资格和机会呢?”

早在之前,方源就有一些推算。

庙明神感应苍蓝龙鲸的手段,十分奇妙,方源也看不出什么端倪。

要知道方源眼界宽阔,各流派兼修,许多流派的境界都很高,而庙明神专修宇道,动用的感应手段却非如此。

这就很古怪了。

正因为太过于古怪,方源也投鼠忌器,没有急着对庙明神下手拷问他的这个感应手段。

如今种种线索表明,恐怕他的这个感应的手段,并非是他自己的本领,而是来源于龙鲸乐土本身,来源于功德碑。

沈从声看着自己的功德,眉头微锁,上面的数字太过刺眼,他第一个走向功德碑:“方源仙友,我时间有限,就不和你多叙了。”

方源点头:“沈兄请自便。若是需要帮助,就在碑下留一只信道蛊虫,我定会全力相助。”

身后群仙见两人如此和谐,各个神情古怪。

沈从声乃是正道领袖,方源却是五域公认的魔头,你们两人关系搞得这么好,这么正大光明,真的好吗?

沈从声听了方源这番话,对他笑了笑。

他相信方源的诚意,但更相信要请方源出手相助,代价绝对高昂。

所以,能自己解决这个危机,他还是倾向于自己解决。

沈从声迅速接取了一个中型任务,消失不见。

对于沈从声而言,这种规模的任务收获功德,最为效率。

方源第二个走近功德碑。

他还是选择大型任务。

“嗯?回收仙蛊的大型任务……”下一刻,他就发现功德碑面上一个全新的任务。

这个任务要求回收的仙蛊,正是他刚刚拆解了海底大阵收获的那一批!

方源细细打量着任务内容。

这个任务内容十分丰富,详细列举了每一只仙蛊的回收价码。

方源先数了数数量,正好和自己收获的一致。

然后他仔细阅览,发现这个任务简直就是功德碑专门为他而来的。

说实话,方源之前拆解大阵搜刮一通的时候,还未细看自己得到的这些仙蛊。现在看这个任务内容,远比他自己琢磨来得准确和轻松。

方源由此得知,原来自己此次收获的仙蛊中,就有三只高达八转,七转最多,六转反而其次。

每一只仙蛊的回收价都非常的高,就算是最低的六转,也高达上万的功德。

身后群仙中掀起一阵惊呼。

他们也看到了功德碑面上的这个任务。

一时间,群仙神情各异,有的人脸上羡慕、嫉妒的神情,简直是喷涌而好处,怎么都掩盖不住。

按照这个兑换,方源只需要上缴一只六转仙蛊,他就能收获到一万以上的功德。这份庞大的功德,远比方源之前的积累要多得多!

好嘛。

他们之前操纵炼道辅阵,他们出了力。战斗开始后,他们躲避余波,消耗大笔仙元,还担惊受怕。

这场剿除魔仙的任务失败了,他们毛都没捞到一根。

但看看方源!

人家的收益是一大把的仙蛊,真要兑换了功德,数值难以料想。

没有对比就没有伤害。

群仙顿感心里堵得慌,他们不甘心,甚至怨愤,但又不敢对方源这个大魔头放肆。

庙明神等人还好些,沈家蛊仙还有任修平、童画更感觉自己冤得慌!

你们损失些仙元罢了,我们呢?

我们连自己的大半仙材都交出去了。

这算是个什么事!

没有收益还倒贴,这种体验着实太糟心了些。

方源一只仙蛊都不想上缴。

仙蛊唯一,过了这个村可就没这个店了。

想要功德,再接着赚就是了,何必急功近利舍弃珍贵的仙蛊呢。

这些仙蛊大多都是土道,对于寻常蛊仙而言,不合自己流派,自己养着还要消耗资源,当然想要抛出去。

但方源不同。

他是全流派通修,土道也很合用。更何况他最近土道境界大涨,他琢磨着,是否就利用这些仙蛊来组成第三座仙蛊屋呢?

------------

\end{this_body}

