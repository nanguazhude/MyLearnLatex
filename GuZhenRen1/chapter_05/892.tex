\newsection{众人的仙材}    %第八百九十五节:众人的仙材

\begin{this_body}

方源的一番话让沈从声沉默。

海风吹拂着他的衣摆,他悬停于空,陷入深思。

方源停下了破晓剑,任由他静静思索。

五域乱战的未来,已经被各大势力预见。沈家贵为东海的超级势力,但在这样的乱局中,也如同惊涛骇浪中的舢板。

沈从声并不昏聩,和他一样,但凡有远见卓识的势力首领都在考虑着自家势力该何去何从。

这是一个大变局,稍不留意,大时代的浪潮会淹没一切不识时务的人!

这当然包括超级势力。

纵观历史,哪怕是在相对和平的年代,消逝的超级势力就不计其数。更何况有尊者出世的大时代,原有的秩序几乎都被颠覆,在此当中遭受毁灭的超级势力接连不断,五域的格局必是变得面目全非。

如何在这样的时代自处?

这是每一个拥有远见的蛊仙,都在琢磨的事情。

该不该和方源合作联手?

这是沈从声现在要思考的问题。

要论资格,方源完全有资格和沈家合作,事实上,沈伤还有点担忧沈家不太够资格。因为他自己并非是方源的对手。

但是如果救出沈伤,并且解决掉他身上的隐患,再加上沈家的仙蛊屋……再和方源合作,沈从声也就有了底气。

沈从声知道:沈家需要方源这样的外援。

和平时期或许可能性不大,但是未来五域乱战,沈家有更大的可能将面临危难,生死存亡的关头。

这个时候……方源当然是靠不住的!

他若不落井下石,就很不错了。

沈从声非常清楚方源的魔头性情。

但是当沈家需要对付一个同等的超级势力呢?比如谢家、蔡家、华家、汤家,甚至是宋家……

沈家处于东海腹地,周围都是庞然大物,就好像是陷入了包围圈。

不寻求一个方向突围,不成长起来,五域乱战时期恐怕会沦为他人口中的食物。

在这种情况下,若是沈家能够支付一大笔资源,就能换取来方源这样的强大助力,又有什么不可呢?

方源拥有八转仙蛊屋,多个尊者传承,能够屡屡将天庭都踩在脚下。这样的强大战力,是沈家所需,也是每一个超级势力所需。

此时不拉拢方源,将来周围的势力拉拢了去,怎么办?

眼下的局势,天庭要修复宿命蛊,方源要对付天庭,他需要大量的低转仙材,这就是拉拢他的大好良机!

就像方源所说的,这对于沈家和他自己,都是一场机缘。

双方各有所需。

当然顾虑还是有许多的,和方源这样的魔头合作实在是有太多的风险。

比如他将来耍诈赖账,他自由自在的,天庭都无可奈何,沈家又能拿他有什么办法呢?

而且在这里无法缔结信道盟约,双方都缺乏信任的基础。

但实际上,就算有盟约,难道不可以破解吗?

将来方源耍赖?

只要双方合作下去,各自有利可图,方源怎么会耍赖呢?

风险是绝对存在的,但这个世上做什么事情没有风险呢?

风险是次要的,利益才是主要的。

风险大,收益小的事情,沈从声不乐干。但风险大,收益也大的事情,沈从声从不抵制!

而最让沈从声欣赏的地方,不是方源的战力,而是他能够和庙明神,乃至和自己分享任务。

从这点上,沈从声看出来:方源和寻常的魔道蛊仙是不一样的!他有理智,有正道的思维和想法。

思考良久,沈从声缓缓点头,一脸郑重之色:“这是一个机会,对你我都是。呵呵呵,也好,为什么不赌一赌呢?将来的天下,不进则退!若是沈伤先祖重新恢复理智,我定会力劝他再不找你的麻烦。”

“这是我目前所能拿出来的所有低转仙材,请你收下。”沈从声再次展露诚意。

方源统统收入仙窍,再不催动破晓剑杀招,缓缓退去:“好!沈仙友,我这就下去收了海底大阵。”

收取海底大阵虽然很是耗费了方源一番手脚,但最终还是成功了。

方源刺射沈伤,虽然白金光柱没有阻止,但破晓剑仍旧对光柱造成伤害,进而损害海底大阵。

海底大阵本来就只剩下四成,又被沈伤突围,有了巨大破绽,方源下手时满眼都是裂口。

若是再打下去,沈伤还未死,海底大阵恐怕就要先崩溃了。

它一旦崩溃,很有可能连累里面的仙蛊。具体受损、毁灭的仙蛊究竟会是哪一只,方源也做不了主。若是悔蛊因此被摧毁,那方源的计划将会严重受挫。

这也是为什么方源和沈从声联手的缘由之一。

大阵被方源收走,横亘在天地间的白金光柱消散,沈伤重获自由。

“先祖。”沈从声一边高喊,一边却是后撤。

方源驾驭着万年斗飞车,谨慎后退。

沈伤仍旧嘶吼着,陷入疯魔的状态,并未有丝毫苏醒的征召。

“沈从声,需要出手相助吗?”方源问道。

“不必。”沈从声盯着沈伤,对方的状态让他也无从琢磨,无法下手,“我们还是等到先祖自行苏醒罢。”

“也好。”方源言罢,驾驭着万年斗飞车,离得更远了些。

就在这时,沈伤忽然一动,身形似箭,飞射而去。

“先祖!”沈从声呐喊,连忙追赶。

方源同样操纵万年斗飞车追击。

但沈伤速度极快,万年斗飞车也只是高出一线而已。

方源不敢动用万年斗飞车,去直接拦住沈伤。就在这个空档,沈伤彻底飞出了悔哭海域。

沈从声、方源想要继续追击,却是被一股无形的屏障拦下。

“该死!”沈从声脸色铁青,差点要破口大骂。

这是乐土仙尊的手段,他之前早已经领教过,就是防止外来的蛊仙四处乱窜,因此每次任务都有地域范围的限制。

这个拘束的手段,并没有对沈伤生效,但阻挠了方源和沈从声。

两人只好眼睁睁地看着沈伤疾飞而去,最终化为天边的一个小小的黑点。

“看样子他是不会回来了。”方源叹息道。

沈从声神情僵硬,眺望良久,终于恢复了往日风采。他转头对方源勉强一笑:“既然如此,那就先回去吧。”

方源劝慰道:“沈兄不必太过挂怀,令祖虽然逃走,但必定出不了龙鲸乐土。他若是清醒过来,必定会想方设法地联络你。若是仍旧疯魔,那就更好办了。功德碑绝不会坐视他危害生灵,定然会有更多的任务出现,就像这一次的一样,来剿除令祖沈伤。”

方源此言在理,沈从声听了连连点头,脸上神情也随之缓和了几分。

功德碑下,气氛紧张又凝重。

两伙人泾渭分明。

一伙人是庙明神等人,包括曾落子、土头驮在内。

另一伙人则是沈家集团,任修平、童画依附着。

“也不知道悔哭海的情况怎么样了……”

“是啊,对方可是方源呐。这个魔头三番五次地击败天庭,着实可怕。”

“我族有两位八转蛊仙,相互照应。尤其是有沈伤先祖在,方源并没有什么值得畏惧的。”

“别忘了,方源之所以能击败天庭,是在光阴长河之中。”

“可是我们的太上大长老也受到乐土仙尊的拘束,根本不能和方源交手啊。”

“唉……”

沈家蛊仙低声议论,而庙明神一伙人则是保持着沉默。

他们很想传送回去,看一看战局进展,但又唯恐自己刚传送过去,就被恐怖的战斗余波殃及,尸骨无存。

没有人愿意傻乎乎地冒险,因此都在这里焦急地等待战果。

两道白光一闪,现出方源和沈从声。

“回来了!”

“怎么不见沈伤先祖呢?”

“方源还活着,糟糕!”

群仙心头震动,纷纷变色。

“呵呵呵,方源仙友,多谢你的这些情报,能让我少走许多弯路啊。”沈从声感叹道。

方源微笑着摆手:“这便是我的诚意。我说过的,你我两方合作,必定是双赢啊。”

群仙再惊,纷纷傻眼。

这是什么鬼!

你们两个刚刚还对战来着,怎么忽然间就和好了?

究竟发生了什么?

沈伤先祖呢?

蛊仙们不禁惊愕诧异。

沈从声哈哈大笑:“我已与方源达成共识,暂且合作。你们身上的仙材都取出来吧。”

“啊?”沈家蛊仙们呆了呆。

沈从声点头,强调道:“这是我答应好的条约,将你们手中的六转、七转仙材都转交给他罢。”

“大长老,这……”众仙面面相觑。

沈从声面容一肃,沉声道:“这是命令!”

沈家蛊仙无语。

这都是怎么搞的?!

我的太上大长老,你究竟是怎么想的呀?

他们简直要怀疑眼前的太上大长老,是不是被方源给控制了!

沈家蛊仙们磨磨蹭蹭,终究是将各自的仙材提取出来,当场交给方源。

方源笑着一一收纳。

他知道这些沈家蛊仙定然不会将仙材全部交出,但没有关系,方源炼蛊的缺口已经填补大半了。

沈家蛊仙们转交完毕,沈从声又看向任修平和童画。

“我等也要?”任修平难以置信,脸色铁青。

沈从声目光冰冷,点点头道:“沈家必有回报。”

任修平差点要破口大骂。

你沈家答应和方源合作,关老子什么事?!

我依附你,你却这样搞我。我特么……

沈从声见任修平迟疑,微微一笑:“难道任仙友还不信沈某,就算不信沈某,我沈家的信誉也不值得你相信吗?”

“大人勿怪,我当然是信的。只是一时之间没有反应过来。”任修平哈哈一笑,忍着痛,将大半仙材转交给方源。

童画脸色难看至极,也不得不这样做。

------------

\end{this_body}

