\newsection{龙公}    %第二百零七节:龙公

\begin{this_body}

%1
白凝冰当然知道,但凡地灵、天灵,都是蛊仙死后执念留下,结合了仙窍之中的天地伟力,形成的一种奇特存在。

%2
正是因为执念的缘故,所以地灵、天灵,一般都会有认主条件,并且不能说谎。

%3
但地灵、天灵不能说谎,并不代表它们不能乱说。

%4
地灵、天灵若是接受了错误的认知,但它们却信以为真,说出来时,并不能算是说谎。

%5
白相天灵的话,有点匪夷所思。

%6
薄青之所以是薄青,乃是因为人族历史上极其罕见。

%7
薄青号称什么?

%8
剑劈五洲亚仙尊,为情所系幸苍生!

%9
他被公认为,古往今来的蛊仙当中,仅次于九转蛊尊的人物。

%10
这样的人物,放在滔滔不绝的历史长河当中,也如夜空中的月光般,夺人眼球。若是这所谓的龙公,能像薄青一样的大能,怎可能不被历史所记载?

%11
白凝冰心中充满怀疑。

%12
但这一次,白相天灵却没有解决他的疑惑,而是道:“少主人,你现在命垂一线。只有运用人人如龙炼蛊法,将自己转变成龙人,才能解决你现在的危险。”

%13
“不过,此法亦有相当多的危险。”

%14
白相天灵说到这里,深深地皱起眉头,叹一口才继续道:“你定当知晓,炼蛊向来对蛊材要求严苛。高明的炼道大能,甚至还要考究每一分蛊材上的道痕究竟有多少。”

%15
“而人人如龙炼蛊法。却是将蛊仙自己当做一种仙材来炼制。然而这种仙材,却是因人而异。每个蛊仙身上的道痕究竟有多少,是什么流派,都大不相同。因此,此法成功可能并不是很高。”

%16
“若是少主人你是炎煌雷泽体。或者本身修行变化道,或者炼道,都会更有把握。可是少主人却偏偏是另一个极端北冥冰魄体!”

%17
“唉……你若用此方法,成功率百不存一。并且在炼蛊的过程中,你身上的全部道痕都会影响最终的结果。少主人你一身的冰雪道痕都可能化为乌有了。”

%18
白凝冰却是双眼骤然放光!

%19
“冰雪道痕会化为乌有,那岂不是说我身上的信道盟约,还有变化道痕。也会随之消失?”

%20
他心中陡然激动起来。

%21
一直以来。困扰他的就是变化道痕。

%22
当年,他在南疆的青茅山,因为转身蛊,由男变女。到如今,一旦他运用虚窍,暂时成为蛊仙,他身上的变化道痕就会被压制。从而转变成男身。若是他不运用虚窍,变回凡人,那变化道痕又会令其变成女身。

%23
当然,关乎影宗盟约的信道道痕,更是桎梏他白凝冰自由的最大枷锁。

%24
他白凝冰岂是甘居人下之人?

%25
只是一直以来,没有好的门路和方法而已。

%26
“就、用、此、法!”白凝冰没有在犹豫,挣扎着,奋尽全力,吐出这四个字。

%27
中洲,天庭。

%28
一位女仙。正拿捏着上报过来的信道蛊虫,细细查看。

%29
她一身锦绣紫袍,肤若白雪,青丝垂至腰间,难掩窈窕身姿。眼若幽潭,眉宇间始终笼罩一层哀愁之意。

%30
正是智道大能,取代砚石老人。暂领天庭领袖之责的紫薇仙子!

%31
她将心神从信道蛊虫中抽出来,嘴角微微上翘,流露出一丝笑意。

%32
“爱情蛊居然会主动认可赵怜云?有些意思。”

%33
紫薇仙子并不诧异。

%34
她知道很多历史上的秘辛,爱情蛊在历史上的记载,向来如此不靠谱。想当初,它既然能够认可墨人墨瑶,如今认可了天外之魔赵怜云,又有什么奇怪?

%35
“灵缘斋的太上长老们,无法定夺此事,于是将这个难题抛到了我的手中?倒是变得聪明了。”

%36
紫薇仙子旋即又一声叹息。

%37
灵缘斋最近并不好过。

%38
义天山大战一出,薄青即便身死,化为仙僵,也在其中扮演了非常重要的角色。曾经,便是灵缘斋招揽进来的剑仙薄青。薄青剑光纵横中洲,带给中洲、十大古派,乃至天庭蛊仙都有巨大的伤害。义天山大战结束之后,灵缘斋因此承受了巨大的压力。

%39
是否将赵怜云认定为灵缘斋的当代仙子,灵缘斋的这些太上长老们也拿捏不定主意。认可和不认可,都是两难的决定。

%40
将这个难题上报给紫薇仙子,似乎是躲避自身责任的举动。

%41
但紫薇仙子却没有任何的反感。

%42
因为她本身,就是灵缘斋出身。

%43
紫薇仙子曾经便是灵缘斋的蛊仙,后来资质出众,升上八转之后,才能天庭接纳。

%44
十大古派虽然是天庭的下宗,但之间的关系,并不是单纯的上下级那么简单。

%45
很多成为天庭成员的蛊仙,从未忘记过照拂曾经出身的门派。而这些门派中的后辈,若是被天庭吸收,就会成为这些蛊仙的天然政治盟友。

%46
所以,紫薇仙子当然是偏向于灵缘斋的,也愿意为灵缘斋解决这个难题。

%47
事实上,她和白晴仙子还有些血脉渊源。

%48
而白晴仙子乃是凤金煌的母亲,凤金煌又和赵怜云争夺仙子之位,谁也不相退让。

%49
灵缘斋太上长老们,若是认可了赵怜云,无疑是否决了凤金煌。因此更要上报给紫薇仙子,让她来定夺。

%50
这样的难题,对于紫薇仙子而言,却不是个事儿。

%51
她正要在信道蛊虫中做出回应,忽然间,她感觉了一股震动。

%52
这个震动的程度很微小,以至于紫薇仙子都差点以为,这是错觉。

%53
但很快。震动又来,幅度比之前更大了一些。

%54
紫薇仙子这才吃惊地站起身来。

%55
她不得不吃惊。

%56
因为她此刻身处在天庭之中,天庭这样的重地,经营了无数年,防御森严至极。怎么会莫名其妙地发生震动?

%57
震动继续,这种异状吸引了越来越多的蛊仙注意。

%58
沙婆婆停止炼蛊,满脸惊异。

%59
威灵仰走出自己的大殿,不断观望。

%60
紫薇仙子终于坐不住,也走出殿外,然后她在震动中听见了隐约的龙啸。

%61
龙啸悠扬,越来越响。天庭震荡的幅度。也随之一路上涨。

%62
在紫薇仙子震惊的目光中。一道龙形光柱从天庭深处,冲霄而起。

%63
“那个地方,不就是仙墓深处吗?难道是,有哪一位先贤苏醒了吗?”紫薇仙子皱起眉头。

%64
这时候,沙婆婆、威灵仰亦都来到她的面前。

%65
“这种景象,发生在仙墓深处,恐怕是天庭的前辈苏醒。就是不知道究竟是哪一位?”沙婆婆说出自己的判断。

%66
威灵仰则道:“仙墓深不可测。神秘至极,就算是我们也不知道里面究竟沉眠了多少蛊仙。但天庭都如此震荡,此次苏醒之人,恐怕极为不凡!”

%67
沙婆婆点点头,她也是从沉眠中苏醒过来的人物,但她苏醒时,天庭静谧如常,根本毫无动静。和眼前此景,形成了鲜明对比。

%68
紫薇仙子深呼吸一口气:“去看看不就知道了吗?”

%69
三仙飞到仙墓深处,只见原本空无一物的草地上。出现了一个深坑。

%70
坑底处,躺着一个干枯如柴的老头子。

%71
老头子见到了三位蛊仙,艰难地睁开一丝眼缝,虚弱无比地道:“水、水……”

%72
紫薇仙子连忙上前去,动用治疗手段。

%73
沙婆婆、威灵仰面面相觑,之前动静不小,没想到见了面之后。此人居然如此虚弱。

%74
这不禁叫人十分意外。

%75
“我的治疗手段都不见效果。”这时,紫薇仙子皱起眉头,忽道。

%76
威灵仰连忙上前,施展浑身解数,老头子身上仍旧不见任何效果。

%77
威灵仰沉声道:“不妙,这位前辈身上大有古怪,我的方法也不见效果。前辈的生机已经渺茫得很,再这样下去,恐怕一盏茶的时间后,就会彻底消亡了。”

%78
紫薇仙子双目闪光,分析道:“刚刚前辈说水,肯定是某种水能救他。普通的水,肯定不行。那究竟会是什么水?”

%79
沙婆婆乃是炼道能手,她一边施展自己的治疗手段,一边道:“天下有三火,亦有三水。皆是九转仙材,不管哪一种,天庭都有库存,取来便是。”

%80
但三水取来,却仍不见效果。

%81
老头子已经呼吸近乎于无,危在旦夕。

%82
三仙急得要跳脚,这可如何是好?

%83
紫薇仙子忽然手指着老头子的额头:“前辈的额前似有两个凸起,宛若石子,这是什么?”

%84
沙婆婆得到这个提醒,忽然灵光一闪:“我知道是什么水了!”

%85
她连忙取出真龙天水,将这种八转仙材,一股脑儿倒入老头儿的口中。

%86
老头子得了此水,立即生机开始旺盛起来,浑身皮肤都开始鼓胀,很快他喉结滚动起来,竟然有力气开始主动喝水了。

%87
效果当真立竿见影!

%88
三仙大喜过望。

%89
老头儿喝了一斤的真龙天水后,他伸出手,将沙婆婆缓缓推开。

%90
沙婆婆知机了,不再倒灌真龙天水。

%91
老头子颤颤巍巍地站起身来。

%92
他整个人都不一样了。

%93
原本松弛的皮肤,都饱满了几分,整个脸盘也像个人样了,不再是之前简直骷髅一般。

%94
他睁开浑黄的老眼,看着面前三仙,叹息道:“这是我最后一次生命,真龙天水也不太管用了,一斤足以,再多就是浪费。我感知到了有人,在某处运用人人如龙炼蛊法,我嗅到了命运的味道,所以我不得不醒来。”

%95
“您,您是龙公?!”紫薇仙子瞪大双眼,结舌道。

%96
“惭愧,正是老朽。”

%97
ps:今天就一章,因为我要写一下《人祖传》。人祖传很难写,比诗词还要难写。有时候想个两三天,都没有用。必须和正文结合起来,又要蕴含哲理情思,还有点明主题思想。灵感非常重要,我必须抓住这次难得的灵感。望理解!

\end{this_body}

