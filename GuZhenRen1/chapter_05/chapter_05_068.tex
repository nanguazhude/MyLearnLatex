\newsection{阵道大宗师}    %第六十八节:阵道大宗师

\begin{this_body}

%1
腾跃!

%2
吞火猴的身影在空中化作灵巧的弧线,一跃而起,在烟尘的掩盖下,飞离兽潮,迅速接近太古荒兽的尸骸。

%3
从外形上看,这是一座大象尸骸。

%4
一场灰蓝色的象皮,破破烂烂,皮下的血肉已经消弭殆尽。森白的骨头,嶙峋成架子,和庞巨的象皮组成了一座巨大的“帐篷”。

%5
这是一座半塌的“帐篷”,残破中透出惨烈的气息,显然这座太古荒兽生前,遭受过一场十分激烈的战斗。

%6
方源的心中,不禁生出一股疑虑:“能够杀死这头太古荒兽的存在,又是什么?”

%7
此刻他的位置,已经很接近太丘的正中央。

%8
这座北原有名的十大凶地之一,的确是凶险无比,深不可测。太古荒兽隐藏其中,令超级势力都感到头疼。

%9
毕竟,太古荒兽的战力,往往能媲美八转蛊仙。

%10
而超级势力中,拥有八转蛊仙的,也只是少数。

%11
偌大的北原,明面上的八转蛊仙,只不过四五人而已。

%12
就算是太丘的边缘位置,稍有不慎,就会引发兽潮。太古荒兽、兽潮,这两个因素导致了太丘不受北原各大蛊仙势力的干扰。

%13
方源越加接近太古荒象的尸骸。

%14
和这座庞大无比的尸骸相比,方源就像是一只苍蝇,飞向一座破烂的灰蓝“帐篷”。

%15
没有一丝风。

%16
但方源却渐渐感到阻力增加。

%17
一股无形的力量,在一波波地抗拒着方源,让方源感觉到他就好像是逆着浪潮,往海的深处行走深入。

%18
更进一些之后,方源甚至出现了幻听。

%19
哗哗哗……

%20
他的耳畔充斥着潮起潮落的喧嚣。

%21
“水道道痕!”方源心中微凛。

%22
毫无疑问,这头太古荒象的身上,有极其丰富的水道道痕。

%23
太古荒象虽然死亡,但水道道痕却跟着骨头和象皮,残存下来,时时刻刻地默默影响这片领域里的天地。

%24
方源之前也接触过太古荒兽的尸骸。比如东方长凡渡劫那次。

%25
但那次的太古墟蝠的尸体,已经被东方长凡改造,所以方源没有像现在这样的感受。

%26
在地沟的超级蛊阵当中,方源也拾取了不少的太古仙材。有些仙材上道痕极其浓郁众多,甚至产生了肉眼可察的道痕光晕!

%27
但这些仙材,到底只是很小的部分。在道痕总数上,和方源眼下这具几乎完整的太古荒象的尸躯,完全不能相提并论。

%28
太古荒兽一身的道痕积累。是相当恐怖的。

%29
最关键的是这头太古荒兽,似乎死去没有多久。

%30
所以,方源才感觉这样难受,很难接近。

%31
“太古荒兽、荒植,每每停留在一个地方时间长一些,身上的海量道痕就会影响周围的环境,将周围天地慢慢改造。”

%32
方源心头闪过一念,脚下却运动不停,身形不断腾挪,尽全力靠近。

%33
正是因为这些水道道痕。所以没有风。整个尸骸附近,静悄悄的,没有任何生命活动的迹象。

%34
“等到这片天地慢慢改造完成,就会有终年不散的云雾笼罩,甚至会积累出一片湖泊,然后催生出许多不一样的植被和野兽。”

%35
方源分析出这片区域,还不是稳定的。

%36
太古荒象的一身道痕,正在潜移默化的改变周围的环境。

%37
这个过程,可能持续数十年,数百年。甚至上千年。在这个过程中,太古荒象身上的许多水道道痕会损失、逸散,周围天地的其他道痕也被排挤出去,以水道为主。

%38
最终。周围天地中的水道道痕,和太古荒象尸身上剩下的水道道痕,形成一个全新的平衡。荒象尸身残骸,不再破损,剩下的水道道痕也不再减少,反而像是被周围的环境温养。

%39
除了方源之外。没有一点植物、动物的痕迹。

%40
方源顿时显得扎眼无比。

%41
兽潮忽然混乱起来,原本整齐划一的步伐,现出不寻常的纷乱之象。

%42
方源心中咯噔一下,转头注视。

%43
但兽潮只是稍微混乱一下,很快就又恢复如常。它好像是一波惊涛骇浪,绕过这具太古尸骸,冲刷到别的地方去了。所过之处,草木倾倒,烟尘滚荡,一片狼藉景象。

%44
方源吐出一口浊气,心道:“看来影宗方面的情报无误。荒兽虽然容易被天意影响,但短时间影响的程度有限。兽潮大势已成,就算是天意,发现了我的蹊跷,也难以操纵兽潮转过来冲撞太古尸骸。在这个范围内,我就是安全的!”

%45
片刻之后,方源停下脚步。

%46
他扫视四周,微微颔首。

%47
这个距离恰到好处,距离太古尸骸中心,既不太近,也不太远。

%48
更关键的是,方源手中的蛊虫有了反应。

%49
临行前,琅琊地灵交托了他一套蛊虫,用来布置。现在起反应的蛊虫,便是专门的侦查蛊。只要在一定范围内,感知到布阵合适的环境,便会发出蛊仙才会感受到的声音和震动。

%50
布阵!

%51
方源停下前进的脚步,站定,一一灌注仙元,催动蛊虫。

%52
一只又一只的仙蛊,接二连三被调动而起。

%53
有一些飞出方源的仙窍,环绕在方源周围飞旋。有一些则停留在方源的仙窍中,不断腾舞。

%54
仙光萦绕,彩霞蒸腾。

%55
海量的心神被牵扯其中,方源不得不消去变化,现出真身。

%56
天意震怒,彻底发现方源,空中响起阵阵闷雷。

%57
但无济于事。

%58
天意真正能够亲自动手的时刻,在于蛊仙渡劫时。现在并非方源渡劫,也不是动用毛民流派的自然炼蛊法,去炼制仙蛊。

%59
瑞气条条飞舞,玄音绵绵不绝。

%60
以方源为中心,渐渐形成一个巨大的七彩漩涡,似气雾,若潮水,气象壮阔,美不胜收。

%61
漩涡不断扩大。有条不紊地往外扩张。

%62
一只只的蛊虫,在彩霞中布置下来,有的深埋在土中,有的就遗留在地上。还有更多的虚化,停留在半空中,更有一些印刻于空中,肉眼不可观察。

%63
海量的蛊虫,以闪电般的速度。被安排下去。

%64
方源主要担当仙元的提供者,真正布阵的,却是一只仙蛊。

%65
六转的阵盘蛊!

%66
它像是一个圆盘,凡间吃饭盛菜的陶瓷盘子。此刻静静地悬浮在方源的头顶上空,调度着各种蛊虫。

%67
整个布阵过程,持续了三个多时辰。

%68
直到太阳落下,在天边洒下暗红的余晖,方源才收起蛊虫。

%69
海量的凡蛊,都被布置下去,全部的仙蛊则回收。所有的蛊虫。形成一个隐秘而又繁杂的蛊阵,隐藏在这片天地当中。不发动的时候,就连方源都侦查不出。

%70
“真是令人惊叹的造诣!”方源感慨。

%71
此番布阵,也让他受益不浅。

%72
虽然他的阵道境界,极其普通,但是眼界却有。

%73
“阵道蛊仙若是有大宗师境界,就能巧妙地利用天地间的道痕,进行布阵。这座蛊阵竟也是如此。这倒让我想起一个历史人物来。”

%74
这个人,名唤九华仙后,乃是历史上赫赫有名的阵道大宗师。

%75
她布阵的风格。就是霞光万道,华丽非凡。

%76
更关键的是,她和长毛老祖是同一个时代的人物。哦,更准确的来说。长毛老祖活的年岁太久了。

%77
“或许这套传送蛊阵,就是九华仙后和长毛老祖的一场交易的结果。”方源心中暗自猜测。

%78
蛊阵已经建立。

%79
方源犹豫了一下,直接开始催动这座蛊阵。

%80
本来依照他谨慎的个性,自然要对蛊阵加以检查。

%81
但他对阵道真的没有什么研究,而这座蛊阵似乎是大宗师的手笔,太过高端复杂。方源想要检查是否含有纰漏。也没有能力做到。

%82
蛊阵徐徐开启,约莫半盏茶的功夫,才完成第一步。

%83
光影飞腾,形成一座覆盖方圆一里多的虚幻大阵。方源置身中央,感到耳畔的潮水声越加频繁。但是周遭却没有来时的无形阻力。

%84
“这座蛊阵调动了太古尸骸上的水道道痕,难怪琅琊地灵一定要选择在这种地点,才布置蛊阵。”

%85
“能够传送蛊仙的蛊阵,很是稀少。这座蛊阵虽然能够传送,不过启动的时间过于缓慢了些,不能用来迅速逃生。”

%86
方源心中迅速评估了一下,然后催动仙窍中的关键仙蛊。

%87
这种仙蛊的力量,从他的体内逸散出来,顿时引起迅猛的变化。

%88
周围的灿烂光影,向着方源呼啸拥来,由虚凝实,陡然间就簇拥着方源,浓缩一团。

%89
方源陡然感到压力剧震。

%90
但转瞬之间,巨大的压迫力量又消散无踪。

%91
砰!

%92
巨响声中,光团猛地一爆,化为漫天的点点彩光。

%93
而方源却已经消失无踪。

%94
半晌后,一切归于平静。

%95
就好像什么事情都从来没有发生过一样。

%96
“已经到了琅琊福地?”方源雄躯顿震,视野中一片五光十色,天旋地转。忽然间,他感到脚踏实地,再定睛一看,视野清明,已经是重回到琅琊福地里来了。

%97
他置身在一座巨大的蛊阵中央。

%98
这座蛊阵,他早已接触过。

%99
当初从风伯崖的蛊阵中,传送过来时,就是落到此处。

%100
方源心中顿时有了更多的明悟:“看来这是母阵,而分布在太丘、风伯崖等处的蛊阵,则是子阵。要催动超级蛊阵,就需要关键性的仙蛊,它就相当于钥匙。”

%101
琅琊地灵的身影,陡然出现。

%102
他哈哈大笑:“方源,你这小子果然没有让我失望!”

\end{this_body}

