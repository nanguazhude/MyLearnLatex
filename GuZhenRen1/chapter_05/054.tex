\newsection{话本平常蛇龙变}    %第五十四节:话本平常蛇龙变

\begin{this_body}

毛六看着方源,哈哈一笑,却是一点都不害怕。±,

若是要审办他,那么方源早就该在之前,毛六刚刚传音的那一刻,就直接动手的。

但是方源却没有,而是坐下之后,才虚言恐吓。

至于内中缘由,毛六也心知肚明,当即便道:“十二云城,本就是由我负责督办。你来之前,这里也已经被我布置了仙道蛊阵。不仅可以完美伪装,不让琅琊地灵发现,而且还有压制盟约的功效。你大可以试一试。”

方源已经加入了琅琊派,身负盟约,不可能随心而欲。

尤其是和毛六这个琅琊派内奸洽谈,更是违背盟约的事情。

不过听毛六这么一说,方源便开始稍稍尝试,果然一些触犯盟约的小事,都没有引来反噬。

方源目光深沉,半晌才道:“看来你真是琅琊派的内奸。你要与我做什么交易?”

毛六心头微微一松,他也害怕方源什么都不问,直接把他拿下审办。现在看来第一关算是过去了。

不过……也不能大意,方源卑鄙无耻,阴狠狡诈,毫无底线。

而自己冒险前来,不惜暴露身份,仍旧危机四伏。

但没办法,影无邪那边实在是山穷水尽了,只有出此下策。

和方源交易,简直是与虎谋皮,实在是太危险了。

唯有交易达成之后,才能坐实证据,随时都能反咬他一口。拖他下水。到那时,毛六才会彻底安全。

但如何才能确保达到己方的目的。又不丧失过多的利益。毛六感到很难办,因为他本就处于弱势的地位。

“影无邪和我有深仇大恨。现在忽然主动找上门来,连内奸都不惜暴露,有求于我。有意思……我倒要看看他的葫芦里卖的什么药。”方源脑海中也在流转着种种心思。

“实不相瞒,影无邪大人那边已经到了山穷水尽,危在顷刻的存亡关头。否则的话,我也不会主动现身,和你交易了。”毛六忽道。

“哦?”方源挑了挑眉头,不由再次打量了对方一眼。

毛六如此坦诚,让方源暗自惊异。

哪里有交易谈判之前。先暴露自家的短处?

方源眼里精芒一闪,旋即察觉到毛六故意反其道而行之的妙处毛六即便不说,方源也能猜到。说出来,看来是想破釜沉舟,置之死地而后生。

“看来和琅琊地灵相处的时间长了,心中放下了戒备。对方虽是毛民,但骨子里却是影宗内奸,怎可能像地灵那般单纯可欺呢?”方源暗自提高警惕。

想了想,方源开口。语气慢条斯理地道:“影无邪有我肉身,以及一身的蛊虫积累,还有黑楼兰、太白云生帮衬,怎可能落魄到如此地步?我不相信。”

毛六摇摇头:“方源你又何必再做试探呢?你种下特意。令蛊虫自毁,打了影无邪大人一个措手不及。他尽全力保存下来的蛊虫,不过只有三成而已。”

此话一出。方源这才彻底确信,眼前的毛六真的是影宗潜藏在琅琊派中的内奸。

方源心思电转。刚刚他也有部分诈语,结果毛六一番回答。让方源立即得到了很多宝贵情报。

“没想到影无邪真的控制了黑白二人。黑楼兰乃人中枭雄,本来不好控制,但大小姨妈尽皆陨落,控制难度降低了不止一筹。她可是大力真武体,若能利用得好了,绝对是个上佳的棋子。”

“而太白云生是老好人性格,虽然优柔寡断,但是两只宙道仙蛊,平时的时候能帮衬不少。居然都被影无邪控制,还真是可惜!”

方源心中有很多感慨。

命运正是奇妙诡异。

义天山大战之后,他和影无邪二人,等若对调了位置。自己夺得了至尊仙胎蛊,而影无邪则继承了自己大部分的积累。

自己可以利用黑楼兰,控制太白云生,影无邪乃是魔尊幽魂的分魂之一,看来手腕也很不俗,将这两人牢牢控制在身边。

“影无邪得到我的仙蛊,还得到黑白二人辅佐。这次主动求我交易,怎么也得让他付出惨重代价!”方源心头暗暗发狠。

然而接下来,毛六在自爆其短之后,忽然有话锋一转:“交易之前,我先来表明我方的诚意。就当我无偿奉送给你一个情报吧。这个情报极其宝贵,关乎着方源你未来的修行,更准确的说,是关乎你的生死存亡!”

方源冷哼一声,没有好脸色地道:“情报是否宝贵,不是看你吹嘘。而是让我获悉内容,自己估算。”

“方源你可知道什么是天意?”毛六问道。

“天意?”方源疑惑。

毛六察言观色,点点头,了然地道:“你果然忘记了很多呢,方源。”

“你这是什么意思?”方源眉头微微皱起。

毛六便说道:“你曾经依靠天意的指点,和影无邪大人换魂,伪装成纯梦求真体,最终抢夺走了至尊仙胎蛊。但你却不知道,这个第十一绝体,是砚石老人研发出来,本身有着重大缺陷。任何入驻纯梦求真体的魂魄,都会陷入沉眠,遗忘一切记忆。后来砚石老人虽然尽力弥补弊端,但也只是消除了沉眠的弊端,纯梦求真体内的魂魄仍旧会不断失忆。时间越长,忘的就越多。也正是因为如此,影无邪大人才会沦丧记忆,只能记得引魂入梦仙道杀招。”

说到这里,毛六取出一只信道凡蛊。

他用手指摩挲了蛊虫几下,心中则在叹气。

这蛊虫对于他自己的价值,没有多少。但是对于方源的价值,却极其巨大!若不是情势所逼,怎么可能把这蛊虫交到方源手上?

原本在影无邪的计划中,他还要靠着方源的这个破绽,来对付方源。

万万没想到,会有这么一刻,由他们主动来弥补了方源的这个破绽。

“这只信蛊中,记录着义天山大战的前后经过,你先看看吧。”毛六说着,便将手中蛊虫抛给方源。

方源接过来,查了查,没有问题,便催使蛊虫,义天山的一幕幕在他的脑海里重现。

他发现真的欠缺了许多关键记忆。

“难怪我总感觉有些不妥,但又发现不了什么。原来是失忆了。”

这是真正的失忆。

彻底丧失记忆,独自一人根本发觉不了。若是感觉自己忘记了什么,那不是彻底的忘记,还是有可能想起来的。

不过,信道凡蛊中的情景,带给方源一种似曾相识的感觉,但也不排除对方欺哄隐瞒的可能。

反复查看了两三遍之后,方源面容平静,目光深沉无比地道:“我似乎是失忆了,但怎么知道这里记载的是真的,是假的也大有可能。”

毛六咬了咬牙。

他知道,这点正是这场交易难办之处。

方源、影无邪,双方本身就有深仇大恨,是不共戴天的死敌,难以信任。如何表明诚意,让对方相信彼此,是最重要的难点。

如果这个难点攻克了,那么这场交易就很容易办成了。

毛六心思电转,脑海中念头此起彼伏。他在斟酌措辞,他不得不如此小心翼翼,方源可是杀人不眨眼!一旦说错话,自己陨落无妨,关键是影无邪那边可是担负着拯救本体的重任,不容有失的。

好半天,毛六这才开口道:“方源你拥有春秋蝉,经历重生,这点为人公知。但其实,你还有更多的秘密,自己反而不如我们这些外人清楚。”

“天意浩荡,无处不在,遍布五域九天,纵贯过去、现在、未来。天道,损有余以补不足。人道,损不足而奉有余。两者恰恰相反,互为矛盾。”

“所以,历代尊者,任何传奇,都是天意的打压对象。灾劫、寿蛊,就是天意的两大杀手锏。方源你前后渡劫两次,一次是地灾,一次是仙蛊成劫。两次灾劫威力暴涨,远超常理,正是你以至尊仙胎蛊重获新生,新的肉身和仙窍潜力无穷无尽,一旦成长起来必定超越历代尊者。因此天意深深忌惮,极力制约,想要提前扼杀了你。”

方源心头震动。

自从渡劫一来,他一直在思考这个问题。

现在毛六忽然提供给了他答案。

“天意……”方源口中喃喃,咀嚼着这个词。

毛六幽幽叹息一声,继续道:“我本体魔尊幽魂,创建影宗、僵盟,炼制至尊仙胎,筹谋了十万年,最终失败。这也是天意作祟,不仅有浩劫、万劫加身,更有人劫围剿。以监天塔主为首的天庭,就是其中代表。而你方源,亦是人劫中的潜藏杀手!”

“此言何意?”方源眯起双眼,精芒烁烁。他心中兴趣大增,敏锐地感觉到,毛六接下来的一番话,将带给他翻天覆地的改变!

毛六犹豫了一下。

他心知:接下来的话,一旦说出口,方源不知天意的这个巨大破绽,就会被彻底弥补。甚至,还会造成方源不可遏制的上升之势。

方源现在,就像是一只困在泥潭中的巨蟒。这头巨蟒双眼被一层黑布蒙蔽,不知道东南西北,只能在泥潭中打滚。

但如果将这些告知方源,就相当于解开巨蟒双眼上的黑布。巨蟒会眼前大亮,看到广阔的天地,看到困着自己的泥潭,原来很浅很小。

有了明确的方向,明白自己的处境,这头巨蟒将会昂首向天,呼风唤雨,更会成就蛟龙,纵横世间!

别看这只是轻飘飘的一席话,但是对于方源而言,却是质变的点,价值无可估量。(未完待续。)

\end{this_body}

