\newsection{南疆正道最大的叛徒}    %第三百四十三节:南疆正道最大的叛徒

\begin{this_body}

%1
“合作?”

%2
听到方源的这个提议,影无邪的脸上不可避免地流露出一抹复杂的神情。

%3
方源和影宗之间的关系,可谓是错综纠结。

%4
身为半个天外之魔的方源,曾经是天意安排好的棋子,孤注一掷,回到过去,破坏了影宗的大计。

%5
成功抢夺了至尊仙窍蛊的方源,从此和影宗之间,势不两立,不共戴天。

%6
义天山一战,影宗大败亏输,恨不得将方源碎尸万段。

%7
然而,方源充分利用时间,以及至尊仙体的优势,逐渐强势起来,竟开始追杀影无邪等人。

%8
若非影无邪等人审时度势,又有优秀的逃生手段傍身,早已经死在方源的手中。

%9
逆流河一役后,方源已经壮大到可战八转蛊仙。

%10
影宗方面由紫山真君领导后,开始改变策略,愿意和方源合作。这是双方关系转折的节点。

%11
正是因为有了这样的铺垫,此时此刻,当方源提出合作的时候,影无邪点了点头,回应过去:“那就合作吧。”

%12
对于和方源合作,影无邪并不反感。

%13
准确的说,他曾经反感、厌恶,紫山真君提出来的时候,很不理解。

%14
但是人总是会变的。

%15
若是在义天山一战中,影无邪刚刚出世,懵懂无知,根本不会有和死敌合作的想法。

%16
影宗强势的时候,方源弱小如蚁,影无邪也绝不会脑抽到和方源合作。

%17
义天山大败亏输之后,影无邪极度仇恨方源,双方合作基础为零。影无邪朝思暮想的,是如何恢复影宗的往昔荣光。

%18
但是当方源实力不断飞涨,接二连三地追杀影无邪等人后,狼狈不堪的影无邪,不可避免地对方源产生了忌惮。到了最近,方源一路将他们追杀到北原,影无邪除了对方源的忌惮之外,还增添了一股恐惧之情。

%19
逆流河大战之后,影无邪又对方源产生了一种全新的情绪。

%20
那就是——佩服。

%21
方源做到了影无邪做不到的事情,这样的人物,不愧是被天意选中,专门回到过去,来与影宗为难。

%22
“不是自己不强大,而是对手是方源啊!”影无邪抱着这样的想法后,忽然发现,自己一路的失败也不是那般难以接受了。

%23
“怎么合作?我可是快要顶不住了。”影无邪直接问道。

%24
方源微微扬了扬眉头,影无邪的态度还真是干脆,让方源有些意外。

%25
“当然是我先出手帮助你们,天庭势力太强,只有我们联手,才有更多的希望。”方源叹息一声,开始沟通和影无邪作战的那位夏家蛊仙。

%26
“撤退?我为什么要撤?我的对手已经被我探清了底子,我胜券在握了。”夏家蛊仙非常不解。

%27
“那边的姚家蛊仙岌岌可危,需要你立即协助。快去,我统领全局,你怎么会有我这样的大局观?”方源毫不客气。

%28
夏家蛊仙心中非常不爽,但是碍于武遗海的身份,以及他如今掌控超级蛊阵的事实,只能在心中冷哼一声,转身撤走。

%29
“那你怎么帮我?你到底是快出手啊!”影无邪催促着,忽然双眼瞪大,十分惊讶地看到他的对手,忽然后撤,用非常不甘心的眼神望着自己,然后迅速飞走。

%30
“吵什么,我已经在帮你了。”方源的声音,适时传来。

%31
影无邪心头狂震,叫道:“你还说不是你假扮的?!或者,他是你的人?”

%32
“别废话了,说说吧,你们影宗到底有什么计划?别告诉我,你们没有预备的后手。”方源立即问道。

%33
影无邪苦笑,回答道:“我们的后手自然有,那便是我。”

%34
轰隆隆!

%35
一声剧烈的爆炸,喷涌的气浪中,紫山真君从漫天的华光中钻出来,极速向下俯冲。

%36
“你逃什么?来战!”身后,龙公紧追不舍。

%37
他此刻长发狂舞,健壮至极的身躯,已经大变模样,不仅龙角生长得更长更锋利,两只手变作龙爪,身后龙尾甩摆,整个皮肤都覆盖了一层厚实密集的龙鳞。

%38
紫山真君有些狼狈,但他目光还是一片清明冷静。

%39
“龙公,历史记载,他主修气道,因为寿命短缺,又兼修了变化道。”紫山真君心中不断思考。

%40
“单单变化道的手段,就如此强悍,不愧是和薄青一样的人物。”

%41
“还是先让他尝尝这一招罢。”

%42
想到这里,紫山真君忽然身形一顿,灵活至极的转折方向,斜向上飞去。

%43
与此同时,他十指连弹,仿佛拨弦。

%44
咻咻咻……

%45
连绵不绝的紫色星芒,一个个从他的指尖迸发出来,划破长空,对住龙公激射而去。

%46
这些星芒璀璨耀眼,不仅速度极快,而且在空中飞行,能拖出一道长达数丈的光尾,非常亮丽。

%47
龙公嗯了一声,感觉到这小小的紫色星芒,并不简单。

%48
他连忙催动攻伐手段,迎接上去。

%49
但紫色星芒竟然非常灵活,直接躲闪开来,透过龙公攻势的缝隙,再度向龙公袭来。

%50
龙公见到这一幕,不禁在心底暗赞一声:“对方不愧是修行了智道,仙道杀招大多都是灵活多变的类型,尤其是这一招,堪称此中翘楚。在这方面,我是比不上他的。”

%51
智道蛊仙向来念头繁多,最擅长操纵仙道杀招,做出变化多端的繁杂攻势。

%52
“我何必与其纠缠?扬长避短才是明智举措。”龙公索性放任这些紫色星芒不管,依仗着自身的雄浑防御,在星芒攒射之中,横冲直撞,勇往之前。

%53
看到这一幕,紫山真君眼眸深处,闪过一抹神光。

%54
“终于是中了我的计了。”

%55
原来他这杀招,还有一层奥秘。就是接触到任何的仙道杀招之后,都能将一段琐碎的信息,传达到紫山真君的心头。

%56
无数琐碎残缺的信息,不断合并,就是对手仙道杀招的奥秘。

%57
只要收集到的信息足够,紫山真君就可对症下药,针对龙公的防护手段,加以还击。

%58
龙公在用反攻的方式,确认了紫色星芒的攻势程度之后,选择了这样做,无疑是正中紫山真君的下怀。

%59
紫山真君对龙公的一身防护,感到头疼。收集信息需要时间,紫山真君又使出另外一记仙道杀招。

%60
他双掌一推,大量云雾生成,阻挡龙公前行。

%61
龙公的龙瞳极速闪烁精光,打量了几眼,顿时看出这记仙道杀招的效能。

%62
他哈哈大笑一声,直接闯入淡紫色的雾气当中。

%63
一瞬间,他不辨东南西北,彻底丧失了方向感。

%64
龙公索性停住,悬浮在原地一动不动,然后猛地张口,发出恢弘响亮的龙鸣。

%65
龙啸声激荡四方,淡紫雾气不断稀释,直至消散。

%66
龙公再次见到紫山真君,将他仍旧在不断后撤,一直企图拉开他自己和龙公的距离。

%67
龙公笑了笑,忽然身形消失,转眼间,就出现在紫山真君的面前!

%68
“吃我一拳。”龙公大喝一声,右拳直捣,毫无花俏。

%69
这不是仙道杀招。

%70
这么短的时间里,突袭过来,龙公还需要心神和精力保持身上的众多防御杀招,所以没有再催动什么其他的仙道杀招。

%71
但是单凭他此时的肉身强度,这记普通的直拳,也非常的具有杀伤力道。

%72
砰。

%73
紫山真君躲闪不及,被龙公一拳击中。

%74
但他中拳的胸口,爆闪出一阵玄光,并未对他肉体造成多大的伤害。

%75
紫山真君身上当然也有优秀的防护手段。

%76
龙公一拳下去,紫山真君毫发无损,并且还顺着这一拳的力道,加速向后撤退。

%77
但下一刻,龙公再一次出现在紫山真君的身边,挥拳攻上。

%78
砰。

%79
相同的一幕再现,不过龙公锲而不舍,不管接下来紫山真君飞出多远,他似乎都能瞬移过去,对紫山真君饱以老拳。

%80
龙公拳势如潮,接连不断,拳影翻飞,笼罩紫山真君。

%81
紫山真君就像是一个皮球,被龙公不断地击飞,一时间竟只能被动挨打。

%82
紫山真君身上,不断爆散出阵阵奇光,抵御龙公的拳头。

%83
龙公的拳势,很快变得不再简单。

%84
乱龙拳!

%85
一记仙道杀招催发出来。

%86
但效果不佳。

%87
龙公心道:“我这乱龙拳,一旦击中对方,就能让敌人脑海中的念头混乱。拳头接连不断,对手往往会思维紊乱,只能被动挨打。但用在他身上,这种效果根本就打不出来。”

%88
到了龙公这样的层次,已然触类旁通。龙公用变化道的杀招,附带智道良效。

%89
但用在本是智道蛊仙紫山真君的身上,却未免有些班门弄斧了。

%90
龙公对付其他敌人,往往能够用乱龙拳将敌人硬生生的揍死。但对付紫山真君,只能抢占上风罢了。

%91
龙公在试探紫山真君的智道造诣的时候,紫山真君也在分析龙公。

%92
他虽然是在挨打,但是心中一片冰雪般冷静,毫无慌乱之情。

%93
智道有三元要点,念、意、情,紫山真君当然有着手段,可以在战斗中,摒除一些负面情绪的干扰。

%94
“龙公的身法,显然是一记仙道杀招,能够让他不断瞬移。”

%95
“他的龙瞳中,还时刻催动着另外一记仙道杀招,侦查能力非常的强。”

%96
“初步推算,他的身法显然不是靠目光来指引方向的。难道是……”

\end{this_body}

