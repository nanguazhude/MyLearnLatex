\newsection{龙公苏醒}    %第三百四十六节:龙公苏醒

\begin{this_body}

天庭有着顾忌,但是这层顾忌是什么,星宿天意并没有明确的告知。

星宿天意交代了之后,就消散在空气当中。

“紫薇啊,不要望着我。我虽然寿命悠长,但知晓的秘密,也不是你想象中的那么多。告诉你实话,这是我生平第二次,见到星宿天意的主动降临。”龙公发出深深的感叹。

顿了一顿,他接着道:“我们要相信星宿天意,就按照它所说的做。我们此行必将获胜,这是命中注定的事!”

龙公坚定不移的情绪,感染了紫薇仙子。

之后天庭的行动,都按照星宿天意的嘱托,龙公等人并未一丝的违背。

监天塔在白天中穿梭的过程中,遭遇到了紫山真君的几手阻碍。它载着龙公、紫薇仙子,还有来自十大古派的许多蛊仙,加入了这场关键之战。

方源分析的都没有错。

尽管方源想不通,天庭方面为什么不派遣出更多的蛊仙,但是这不妨碍他看破天庭的底细。

正是因为如此,方源才做出决定,利用超级蛊阵帮助影无邪,从而影响到了紫山真君和龙公这场大战。

利用宝黄天,方源成功地从影无邪那里,得到了一只信道蛊虫。

这只信道蛊虫,可以直接和紫山真君交流。

方源帮助影无邪,初步取得了影宗的信任,否则哪能得到这只蛊虫?

这只蛊虫和紫山真君联系紧密,有了它,就等于掌握了一道关键线索,从而可以更容易地推算紫山真君。

所以这种东西,向来都保管严格。

也是在这种特殊的情况下,紫山真君才会将信道蛊虫,交给方源。

换做平时,非得双方缔结信道盟约不可。

得到信道蛊虫的那一刻,紫山真君就主动沟通方源:“方源,我还得感谢你。紫山真君这个名字很不错,多谢你给予我的这个名号,我很喜欢。”

方源自己也没有料到,曾经一个哄骗太白云生的谎言,居然有一天成为了现实。

紫山真君的交际手腕非常老道,这个开场白一说,就消除了他和方源之间的隔阂,大大拉近了关系。

方源笑了笑:“那我就直接称呼你为紫山真君吧。我的诚意你们已经收到了,你我双方合作,可不是只有我单方面的付出。现在,该你们表达诚意了。”

紫山真君一边猛攻龙公,一边回应:“好,你想要得到什么?”

他知道方源掌握的力量,在现在的局势中至关重要,所以他没有丝毫的犹豫。

方源随即便道:“天晶!我需要大量的天晶。”

天晶这种仙材,实在太难获取了。宝黄天中虽然有的卖,但很罕见,量很小,完全满足不了方源的需求。

但影宗有啊。

方源毫不怀疑这一点。

影宗虽然残破不堪,损失惨重,但是资源方面并不缺乏。这点方源从屡次对影无邪等人的追杀中,可以明显的感知到。

果然,紫山真君立即答应下来:“可以,我手中的确有大量的天晶,这些都可以给你,表达我方的诚意。”

“你要天晶,是想催熟上极天鹰吧?”

“你的这个想法很稳妥,逆流护身印的确不能随意再用。但是上极天鹰即便催熟成真正的太古荒兽,恐怕也不会受你的掌控。你的奴道手段,我很了解。”

“不过你放心,我的手中还有让你掌控上极天鹰的方法。”

紫山真君的这席话,让方源听了,大喜的同时,暗呼紫山真君的厉害。

紫山真君非常大气,直接舍弃天晶,换取方源的支持,毫不犹豫。

他原本处于被动的地步,于是说出后面的一段话,用控制上极天鹰的方法,来诱惑方源。这是转守为攻,从被动占据主动。

不愧是八转智道蛊仙,言语间充斥智慧,让方源感受到劲敌的气味。

方源沉默了一下,然后道:“再增添一个掌控上极天鹰的方法,也无不可。不过当下,还有一件更要紧的事情要去做!”

这话说的,好像他真有什么可以掌控太古上极天鹰的法子似的。

如此一来,就让紫山真君觉得手中筹码并不大。

紫山真君笑了一声:“什么更要紧的事?”

“便是先让你我两方都罢战!”方源道。

这件事情,对影宗方面有利,对方源也有利,惟独对天庭没有利。

当即,紫山真君开始下达命令,召回影宗残余的蛊仙。

方源也同时下令,南疆蛊仙虽有不解,但念在武遗海的身份和掌控蛊阵的事实,都选择了听从。

双方开始撤离,相互分隔。

影宗方面折损较多,只剩下影无邪、白凝冰、黑楼兰、妙音仙子等人。

南疆正道这边,虽有折损,但无伤大雅,阵容还是不小。

不过,影宗的纯梦求真体仍旧在不断地破茧而出,它的势力在不断地增长,梦境不断的削减消失。

这种情况,惟独对天庭是最不利的。

“双方的实力差距开始缩小了,这个时候,天庭该如何应对呢?”方源时刻关注着天庭的动向。

龙公继续沉睡,紫薇仙子加快对超级蛊阵的侵蚀。

监天塔仍旧和左夜灰纠缠在一起。

轰!

左夜灰高高跃起,挥起左爪,将监天塔狠狠地拍到地上去。

监天塔闪烁着洁白的光辉,毫发无损。

左夜灰猛地张开大口,吐出一记灰夜杀招。

监天塔再度催动命败。

两记杀招对拼,夜灰在洁白的光中消散,左夜灰骨断筋折,远远抛飞出去。

但几乎是在下一刻,它催动了仙道杀招,浑身的杀伤就都消失。

重复巅峰状态,左夜灰再次咆哮一声,带着满满的仇恨和怒火,再次攻向监天塔。

监天塔中响起叹息声,随后拔空而起,飞上高空。

左夜灰也有飞翔的杀招,也飞上了天去,对监天塔紧追不舍。

位于塔中的中洲蛊仙们,神色都有些紧张和焦躁。

“该死!这头太古荒兽太过皮糙肉厚,关键能够使用杀招,短时间内监天塔都奈何不得。”

“怎么办?龙公大人被暗算,陷入了梦境,任由那个八转魔头攻击。”

“出去!我们必须派遣援军,支援龙公大人去。再这样下去,可不太妙。”

“可是龙公大人临走之前关照嘱托,让我们必须待在监天塔内,不要随意单独行动。”

“此一时彼一时啊。”

“我们也有八转蛊仙,且龙公大人陷入危局,已经丧失了对战局的把控。我们若再不支援求助,龙公大人危险!”

快速商量了一阵后,监天塔发动猛攻,将左夜灰暂时击退,创造出机会。

两位中洲蛊仙趁机飞出了监天塔,迅速向龙公赶去。

“哈哈哈,出来了两个小老鼠!”左夜灰大笑,一对兽目暴射出凶残狰狞的光。

它猛地张开血盆大口。

两位中洲蛊仙忽然浑身一僵,像是遭遇到了天敌一般,极其巨大的危机感笼罩全身,让他们难以自持。

咔嚓!

左夜灰大嘴猛地合上,它明明和两位中洲蛊仙都有一段遥远的距离,但是牙齿咬合的时候,偏偏就好像是咬到了这两位中洲蛊仙一样。

一股鲜血,从它的嘴角漫溢出来。

与此同时,两位中洲蛊仙浑身鲜血喷涌,恐怖的伤口洞穿全身,内脏被利器捅破,整个身躯都在刹那间面目全非。

“这是食道的杀招!让人防不胜防。”

“左夜灰拥有人的智慧,它完全懂得避实就虚的战术。”

“快,把两位仙友迎接回来。”

监天塔连忙施救,好在两位中洲八转亦都拥有不俗的自救手段,险而又险逃得一命,没有被左夜灰吃掉。

侥幸捡回性命,中洲蛊仙们真正意识到,左夜灰的恐怖!

“不愧是传奇太古荒兽,狂蛮留下的东西!”

“这样一来,我们如何救援龙公大人啊?”

“将监天塔开赴进去。”

“太危险了。万一被拍进梦境中,失了监天塔,这样的责任你我担当得起吗?”

就在中洲蛊仙一筹莫展之际,龙公陷入到有史以来最大的危机当中。

紫山真君的杀招终于起效,他勘破了龙公用作护身的种种手段。

“都给我解开!”紫山真君飞到龙公的身边,喷吐出一口紫色的氤氲之气。

这股奇妙的气息,笼罩在龙公身上,将他身上的层层防护以此分解开来。

仙道杀招――意解纷呈!

“死!”紫山真君双目暴射紫色精芒,施展杀招,对准龙公。

但就在这一瞬间,龙公猛地睁开了双眼。

轰!

爆炸声惊天动地。

紫山真君从滚滚烟尘中飞退出来,而龙公重新屹立在原地。

他竟清醒过来。

“怎么会这么快就脱困了?”影无邪大吃一惊,连忙再用引魂入梦。

紫山真君大感不妥,龙公苏醒的时机未免太过巧合,正要阻止,已是晚了。

龙公瞬间消失,影无邪莫名其妙地遭受杀招反噬。

他自己竟是陷入梦中去了!

“这一招,好像是叫做引魂入梦?的确是好招式。可惜若再给我一点时间,我便能彻底洞悉它的奥妙了。”龙公缓缓开口。

冷漠的龙瞳中映着紫山真君如临大敌的面孔,龙公继续道:“至于你,你的手段若只有这些的话,可接不下我接下来的这一招。”

“因为,这可是元始仙尊所创!”

龙公主修气道,兼修变化道。此时此刻,他终于开始动用真正拿手的手段!(未完待续。)

\end{this_body}

