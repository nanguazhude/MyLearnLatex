\newsection{很有意思}    %第六百二十四节:很有意思

\begin{this_body}

面对方源凝如实质的杀意,白凝冰却展颜一笑,双眼坦然地直视方源,反问一句:“我为何要逃?”

顿了一顿后,她又道:“戚家蛊仙虽然逃了,那是他们对你不了解。 .更新最快你我同是青茅山的人,依我对你的了解,他们能逃得掉?呵呵,既然逃不掉,我又何必逃呢?我不仅不会逃,而且恰恰相反,我还要加入影宗。”

方源皱起眉头:“我也曾想过,可惜你的白相杀招,实在是太精妙了,几乎能破解一切盟约。”

白相的绝妙之处,在于将蛊仙转化成一种极特殊的存在,任何一小块残片,都能够重生回来。

盟约的实质在于信道道痕,白凝冰利用白相自残,就能甩开相应的信道道痕,达到摆脱盟约的目的。

因此,白相杀招还能用于疗伤,将残留在身上的其他道痕排遣出去。

“的确如此。”面对方源表露无遗的顾忌,白凝冰哈哈一笑,丝毫没有担忧自己的安危,“但正因如此,不更精彩吗?”

方源冷哼一声,不悦地道:“不要把我当做你,白凝冰。”

白凝冰点头,收敛了笑意,龙瞳中闪烁着精芒:“不错,我一直都在追求自身的精彩。所以我羡慕你,方源。你过的生活,真是太刺激,太精彩了!你每一次算计,每一次成功,甚至每一次失败,都是万分的精彩。尤其是最近一次,你被天庭埋伏,逃得像狗一样狼狈,哈哈,最是精彩不过!”

“其他的影宗之人,是体会不到我的感受的。他们有的一心想救魔尊幽魂,有的心中惧怕恐慌,有的压力重重抑郁寡欢。而我的感受却是兴奋和欢喜。越是颠簸,越有风雨,自然就越加精彩,不是吗?”

“别人或许不懂我,但我知道你一定懂我,不是吗?”

白凝冰盯着方源的眼睛,目光发亮。

方源沉吟不语。

早在青茅山上,他就判断白凝冰此人乃是真魔。他的判断并没有错,白凝冰就是如此。为了追寻精彩的人生,她可以舍弃种种,甚至是自己的生命。大众的价值观念,繁杂的恩爱情仇,都和她没有关系。

她与大众截然不同,她和世界格格不入,她只照着自己的轨迹走。纵内心的情意,轻万物的生死。

说白了,白凝冰和方源很像,他们是同样的一类人,只不过他们追求的东西不一样而已。

方源追逐永生,能够牺牲任何东西,包括自己。哪怕最终不成功,死在半途之中,他也会开心、欣慰,因为这样的目标这样的活法,让他自己感到自己活着充满了一种盎然的趣味和生机。

白凝冰说方源懂她,这一点绝对没错,就像她也看得懂方源。

俗话说,英雄相惜,魔头之间当然也有这样的感觉。

方源眼中的杀意,渐渐消散。

这一层杀意,不过是他故意凝聚出来,说实话,他心中并无多少杀意。

为什么要杀白凝冰?

如今的方源实力远超白凝冰,后者对方源的威胁已经很小了。

有时候,看到白凝冰,方源就像看到自己的影子。像他们这样的真魔,太少太少,碰上了其实也是一种幸运,一种见证自己的欣慰。

因为彼此最像彼此,这是本质上的某种相同,或者说认可。

所以当年,白凝冰付出生命救下方源,说出一番话让方源替她活下去,见证生命的精彩。本质上就是白凝冰觉得方源,可以在某种程度上替代自己。

有时候,方源也在想,若是他自己和白凝冰之间相互转换身份,那么他是否也会做出和白凝冰一样的选择?

“哈哈哈。”白凝冰忽然仰头大笑,“白相洞天中的资源,我都可以给你。但是白相真传中的蛊虫,我都要留在自己的身上。我之所以愿意加入影宗,是因为和天庭对抗,的确精彩!这本是你的精彩,也将是我的精彩。”

“如此一来,事情的发展就有三种可能。”白凝冰眯起双眼,眼眸中透射出亢奋,甚至一丝疯狂的神光。

“第一种可能,你被天庭杀死。依凭你的实力,就算被杀死,也一定会死的很绝妙,在临死的挣扎中迸发出前所未有的光辉。我见证你的死亡,将是我一生都极其难得的美妙体验。”

“第二种可能,天庭被你击溃,甚至毁灭。哈哈哈,那样定然也是精彩万分的。天庭,人族的第一势力,这样的巨无霸倒下的过程,就像是流星坠落到地上,星火的璀璨将惊艳世间。”

“第三种可能,我最期待,因为这样的情况最为精彩!那就是天庭和你僵持不下,两败俱伤。这样一来,我守候在你的身旁,就有了绝世的良机,可以渔翁得利。说不定,我能够成为最终的胜利者,而你们两方都是我的垫脚石。”

方源静静地看着白凝冰,面色冷漠,平淡地道:“如果我现在把你杀了,你的这三种可能就都是不可能。”

“是,是的……”白凝冰皱起眉头,神情变得有些沮丧,目光闪烁不定,语调也低落下来,“是这样的。我现在若死在你的手中,虽然我的反扑也会竭尽全力,但我和你的差距太大,精彩程度简直乏善可陈。”

“不过,如果这是你的选择,我也会接受的。”她微笑起来,耸肩道,“因为我实力太弱,不得不接受。这种残酷,也挺精彩的,不是吗?呵呵呵。”

死亡对于白凝冰而言,一点都不可怕。

她是一个疯子,又有理智和冷静。这样的人,无疑很可怕。

而比她更可怕的人,就在站在她的面前。

那是当今天下,让天意无奈,天庭无策的魔道巨擘方源。

方源嘴角微翘,也跟着笑出声来:“很有意思。”

白凝冰这个人本身就很有意思。

如今是白凝冰孱弱,任由方源宰割的局面,白凝冰一番劝说,丝毫没有为自己开脱,甚至反其道而行之,还说出了自己将来对方源的可能威胁。

这是因为她懂方源。

方源这样的人物,心中雪亮,早就将杀与不杀的利弊都分析清楚。而方源一旦下定决心,不管白凝冰说什么,都是没有用的。

不用白凝冰说,方源心中清楚,此刻的他为了跨越天庭的障碍,寻到红莲真传,任何一份力量都需要珍惜。白凝冰显然是一份这样的珍贵力量。若是给她机会,她肯定会背叛方源,但那又如何?

一把刀锋口犀利,当然会有割伤自己的危险。但若因此而弃刀不用,徒手对敌,岂不是愚蠢吗?

如何用好这把刀,不给她反叛的机会,这就是要考较方源的能力。

“若是我能力不足,让白凝冰反叛成功,也只能是怪我自己,不是吗?”

想到这里,方源笑道:“白凝冰啊,你真是令我欣赏,我就暂时不杀你了。跟随我,见证我的成功或者失败吧。如果我最终死在你的手中,也的确是很有趣。不过现在,展现你的价值吧,把戚家蛊仙的人头带给我,还有他们的……气海洞天。”(未完待续。)<!--80txt.com-ouoou-->

------------

\end{this_body}

