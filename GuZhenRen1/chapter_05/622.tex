\newsection{血仇}    %第六百二十五节:血仇

\begin{this_body}

九转杀招天相内部。燃文小?说    ??.?r?anen`

方源的意志,宛若蛟龙腾飞,盘旋于空,随后狠狠地俯冲下去。

轰隆!

方源意志宛若流星陨落,重重地撞在底下的意志海洋当中,一时间大量的意志迸溅毁灭。

然而,这片意志的海洋实在是辽阔深邃,方源这一击所摧毁的,不过是总体的九牛一毛而已。

“我虽然是抢夺成功,霸占了杀招中枢,但这里面却充斥着五相及其后代的意志。只有当我将这些意志剿除干净,才能真正催动起这记九转杀招。”

这片意志的规模,着实庞巨。更麻烦的是,这些意志受到九转杀招的影响,变得宛若石头般坚固。方源要剿除它,很多巧妙的手段都没有效果。在九转杀招的影响下,方源能用得上的乏善可陈。

方源忽然念头一转,一股全新的意志蜂拥而入。

刷刷刷!

意志凝聚成剑,万千把意志长剑宛若暴雨倾泻而下,射在意志海洋上面,旋即打出无数的坑洞来。

不一会儿,这片意志海洋就被削弱了一层下去。

和之前的意志蛟龙的战果对比起来,这场意志剑雨简直是好得惊人。

方源见此,不禁暗自感叹一声:“不愧是八转仙蛊慧剑!”

他如今是八转修为,便能催动此蛊。此蛊虽是剑道仙蛊,但和智道关系极其紧密。

方源以之为核心,推算出一招相当简略的仙家手段,可以催发出大量的剑意。

运用这些剑意冲刷意志海洋的效果,是方源屡屡尝试以来最佳的。

只是如此一来,方源投入就上涨了许多倍数。毕竟慧剑仙蛊乃是八转,催动它需要消耗白荔仙元。而仙道杀招更要加剧仙元的耗费。

但这种耗费,方源经过推算,发现还是能够接受的。

毕竟他的后勤非常强大,仙窍经营可谓红红火火。

光阴长河突围战,方源虽然逃出生天,挫败了天庭的剿杀计划,但是他也损失很大。尤其是仙蛊屋雏形的毁灭,令他到现在都为之心痛。

这座仙蛊屋雏形,乃是方源俘虏了诸多南疆蛊仙之后,获得了他们的仙蛊,又和南疆正道之间进行了许多次的谈判、交易、勒索,这才好不容易换来仙蛊,组合而成。

这当中凝聚了方源相当巨大的心血,耗费了他大量的精力,但现在都化为了泡影。

“希望这记天相杀招,能够帮助到我罢。”

方源对天相杀招报以相当巨大的期待。

换做寻常的杀招,方源兴许能够侦查内里,推算出具体的威能。但这记杀招高达九转,层次太高,不是方源能够看得懂的。

这是盗天魔尊的手笔,他擅长宇道、偷道,即便此招不是用来攻伐,也一定效果绝伦。

虽然盗天魔尊始终没有明说,此招的威能究竟是什么,但方源也可以从其他方面推测。

比如影宗的态度。

影宗的砚石老人就特意帮助白凝冰,扶持她,主要目的就是图谋天相。

“砚石老人同样是智道大能,恐怕是他推测出了什么,这个天相杀招一定对影宗大有帮助。”

方源乃是当今的影宗之主,这个天相杀招自然对他也会有巨大的助益。

除了剿灭杀招内的意志海洋外,方源还同时着手,破解五相封印,将山巅的仙蛊屋,以及这片五相公共洞天置于自己的掌控当中。

这处洞天千年才开启一次,资源相当的丰富。戚家蛊仙临走前忙着逃命,所以无心搜刮,这些都将成为方源的修行资粮。

轰!

巨大的爆炸,把白凝冰炸得四分五裂。

但下一刻,她化身的白相又再度聚拢起来,冲向戚家蛊仙。

戚家蛊仙面色相当难看,充分体会到了白相杀招的恐怖!

白凝冰有此招傍身,几乎就是不死之身。当然她此刻一敌众,并非是她一人之力,影无邪、黑楼兰也在幕后帮她。

方源虽然是令白凝冰追杀戚家蛊仙,但亦知道她的能力有大小。别的不说,最关键的仙道战场杀招就没有。

方源早就布置了仙阵,埋伏起来。自从上次他成功俘虏南疆正道之后,就对这个手段十分喜爱。

戚家蛊仙进来之后,立即被仙阵拘束住自由,暂时无法外出,只得和白凝冰缠斗。

“啊!”一声惨叫,戚家蛊仙中最近成仙的那位,遭到白凝冰的致命打击,当场战死。

“该我了。”影无邪早已准备妥当,此刻立即出手,将戚家蛊仙的魂魄拘摄到阵中去。

剩余的戚家蛊仙怒极咆哮,激烈反击。

白凝冰被再次打成碎片,几个唿吸之后,她再度重生,呈现完整状态的白相。

戚家蛊仙虽然人数众多,战力也十分雄厚,但此刻都不免流露出绝望的神色。

因为他们都知道,白凝冰根本杀不了,又中了埋伏暂时出不去,这样打下去,迟早要被她拖累死。

“怎么办?”

“白凝冰实在难缠,更关键的是,方源太过阴险,居然将这大阵埋在我大本营气海洞天的门口附近!”

戚家蛊仙一路上都十分谨慎,结果刚要回到大本营,就全被逮住了。

他们并不知晓,方源早已经杀了戚灾。戚灾虽然没有进去过五相洞天,但气海洞天的情况他是相当清楚的。

这气海洞天乃是当初气相的仙窍,落到白天中后,就无法挪移了,一直停留在原处。

方源正是依靠这个情报,才将仙阵暗中铺设下来,最终算计到了戚家一伙人。

“这么下去,我们必败无疑啊!”

“拖!我就不信她的白相杀招如此威能,就没有一丁点的代价或者后遗症?”

“你们别忘了方源,他要是赶过来,这里又不是仙蛊屋大殿,再没有五相的封印可以制约他了。”

“这仙阵我们暂且无法突破,只有让戚平素出来,与我们里应外合,方能最快地攻破此阵。”

“不妥!戚平素乃是我们留守在大本营的唯一人手,他要是出来,必定要开启门户,这就给了影宗可趁之机。”

“说不得方源早已潜伏在旁,专等着洞天门户大开,好让他突破进去呢!”

方源其实压根就没有来。

但他的恐怖和狡诈,早已经深入人心,让戚家蛊仙此刻都饱受压力,慌乱无措起来。

与此同时。

中洲的某个角落里,一位青年男子静静地站着,看着眼前的仙蛊。

这位青年男子的面色颇为沧桑成熟,显露出他曾经坎坷的生命程。

不是别人,正是古月方正。

古月方正原本要被仙鹤门处死,但被方源留了一手,伪装死亡,令他在琅琊福地生活。

古月方正在充斥毛民的环境中,饱受排挤,又长期经战乱,朝不保夕,有时候还有生死一线的险境。

如此种种,已经令他今非昔比,真正成熟。

后来,天庭入侵琅琊福地,古月方正因为特殊的身份,就被凤九歌救走。

救走之后的古月方正,最终回到了中洲。

“如果我没有猜错,这是一只血道仙蛊吧?”古月方正望着眼前的蛊仙,嘴角流露出一丝讥讽的笑意,“你想我晋升血道,成为魔道蛊仙,来对付古月方源?这就是你们仙鹤门的正道做派吗?”

仙鹤门六转蛊仙樊西流微微蹙眉,笑了一声道:“任何的力量,用之正则为正,用之恶则为恶。这个道理,你还不明白吗?就好比我修行毒道,完全可以福泽苍生。你修行血道,同样也可以。”

古月方正摇头:“我修行血道,只不过是更有利于对付方源罢了。毕竟我和他乃是血亲!”

方源虽然拥有至尊仙体,但宙道分身仍旧和古月方正有着血脉联系。

樊西流点头道:“的确是有这样的原因。但你不想报仇吗?毕竟他屠杀了古月山寨的所有族人啊。”

古月方正长叹一口气,尚显年轻的面庞却面露沧桑的萧索之色。

“这么多年过去了,说实话,我虽然仍旧恨他,但我也理解了他。”

“我回想过,小时候的我很幼稚,懵懂无知,方源给过我许多的照顾。反倒是我的舅父舅母,却是拿我当做棋子利用。让我给这样的人报仇吗?我并不甘愿。”

“当然,族长、古月青书都相当的照顾我,栽培我,给他们报仇雪恨,我是愿意的。”

“但我终是累了……唉,你杀我我杀你,总是这样回环往复……我不想报仇了。”

樊西流面色不变:“看来这些年,你的确经了很多。”

古月方正抬眼看他,嘴角的嘲讽之意越发浓重:“讽刺的是,当我想要报仇的时候,我的门派不支持我,甚至想要除掉我。当我不想报仇的时候,他们却救回了我,支持我去复仇。”

说到这里,古月方正看向血道仙蛊,目光幽幽:“并且我知道,我别无选择,只有接受。”

樊西流再次打量一番古月方正,眼中异色一闪即逝,轻笑道:“你这样实话实说,真的好吗?”

“没有什么好不好。我知道的,依凭你的手段,我心中所思所想能瞒得过你吗?”说着,古月方正伸出手来,轻轻地握住那只血道仙蛊,又问,“此蛊何名?”

“血仇。”

\end{this_body}

