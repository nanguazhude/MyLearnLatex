\newsection{与方源联姻}    %第四百四十节:与方源联姻

\begin{this_body}

北原,北部冰原。

寒风呼啸,冰川上大雪飘飞。而在这冰川之下,亦有一场氛围凝固如冰的谈判。

生活在这里,隐姓埋名的两个异人种族――石人和雪民,在不久之前,因为一场领地的纠纷,而展开谈判。

如今,这场谈判已经持续了六天六夜。

“莫离地区,乃是我石人一族的领土,早在四百多年前以前,就是我石人一族不可分割的地盘。这是我石人一族的底线,还请雪民一族理解!”石宗的声音很大,声音在房间中回响。

参与谈判的石人蛊仙们,都是神情冷峻。

而雪民一族的蛊仙们,则是脸色铁青,更加难看。

石人一族的态度非常强硬,给雪民一族带来巨大的压力。

雪民蛊仙冰媛开口,道:“四百多年前,的确是这样。但是之后,我族的一位蛊仙前辈和石宗太上大长老你打赌赢了,莫离这片地区就是赌注。”

石宗连忙摆手:“那只是一个玩笑话,当不得真。”

冰媛冷笑:“当时,可都是立下了赌约的。”

石宗面容严肃,反问道:“赌约何在?”

冰媛顿时气得说不出话来。

这石宗不要面皮,动用手段消除了自身信道盟约不说,还抵赖赌约的事情,身为堂堂石人一族的太上大长老,这副嘴脸未免太过于无耻。

但是雪民一族的蛊仙们,虽然是气愤不已,但同时也能理解。

因为在这片冰原中,真正的土地很少,非常稀有,石人、雪民一族又经过这么长时间的繁衍生息,人口过多,对于土地早已经是超负荷。

石宗舍弃面皮,也要夺回莫离地区,也是为了石人一族的大局着想,为石人一族谋求利益。

但这样一来,雪民一族的利益,自然是受到了侵犯。

最终,这场艰难的谈判有了结果。

雪民告负,石人一族得胜,掌握了莫离地区。

当即,石人一族就下达命令,要求莫离地区中的雪民统统搬迁出去,留出空间,交给石人一族的接手者。

回到自家的领地,七转蛊仙冰卓怒气冲冲:“这石人一族太可恶了,依仗着自己有用八转战力太古石龙,就如此颠倒黑白吗?!”

冰媛劝慰道:“冷静,冰卓啊,石人一族的态度这千年来,我们还没有看透吗?”

北部冰原中,石人一族和雪民一族相互依存,但总体而言,石人一族占据上风。[看本书最新章节请到

原因只有一个,那就是他们拥有太古石龙,这是八转战力。

有着这样的强硬战力,使得石人一族在很多次的矛盾冲突中,都占据上风,维护住了自身利益。

提到太古石龙,冰卓脸上的愤怒迅速消退,转为一声无奈的叹息:“唉!八转战力,太古石龙……我族若是有八转战力,能够分庭抗礼,该有多好。”

冰媛沉默了一下,方才道:“要拥有八转战力,也不是不可能。我有一计,若是真能运作成功,那太古石龙算得了什么?”

冰卓听了,双眼骤亮:“什么计策,快快说来。”

“其实此计,早已经在我心中酝酿多时,今时今刻,也算是一场契机,我便告诉你知晓罢。”冰媛便缓缓道出。

冰卓闻言,惊愕出声:“你是说柳贯一,哦不,是方源?”

柳贯一的身份,就是方源,这个消息早已被天庭公布,传遍天下。雪民一族虽然偏安一隅,但是这等重大的消息,还是知晓的。

“方源乃是八转战力,天下公认。又有上极天鹰,不久之前更是和我们结下了善缘,所以若让雪儿嫁给他,让我们雪民一族和他联姻,就能将他捆绑到我族的战车上。我们有此依仗,再不会受石人一族的欺负了。而且,有一位八转坐镇,对我族的发展将有极大的好处。”冰媛道。

冰卓一边思索,一边点头:“的确,方源虽是人族,但举世皆敌,又早已经投靠琅琊派,阵营上没有问题。但是等等……”

冰卓疑惑地看向冰媛:“我族和他还有善缘?我怎么记得,我们只是和他一场激战,埋伏了他。虽然最后化干戈为玉帛,但是距离善缘这个层次,还远得很呢。”

冰媛笑道:“冰卓你前些时候正在闭关,琢磨杀招的改良,所有你才有所不知。在最近这段时间,方源动用琅琊派几乎举派之力,大炼仙蛊,过程中遇到困难,需要雪莲花精仙材。我便做主,大开方便之门,将众多的雪莲花精都卖给了琅琊派的蛊仙毛六。”

“原来如此。”冰卓恍然,但旋即又皱起眉头,“既是卖掉,我们也得了利,为什么当初不如直接将雪莲花精送掉呢?”

冰媛便苦笑:“兹事体大,我当时也只得如此。一来,和方源联姻的念头还不甚强烈。二来,若是直接送,无事献殷勤,方源绝非蠢笨之人,恐怕也不会接受。三来,此事早已被人所知,若是直接送,会引来石人一族的怀疑和猜测。”

冰卓点头,默然不语。

他心中明白,这说是善缘,其实也谈不上,只能说是关系改善了一些罢了。要和方源联姻,困难重重。

因为之前,方源拥有上极天鹰的时候,雪民一族就曾向方源表达过善意,但对方的反应很是冷淡。

现在方源自身成为八转战力,又拥有上极天鹰,地位要更上一层楼。雪民一族要和方源联姻,不是对方源青睐有加,而是攀附方源。

方源丢失上极天鹰的情况,雪民一族还并不知晓。

但冰媛却是比冰卓更加自信,她道:“这你放心。我家孙女容貌动人,又是蛊仙,乃是上佳的联姻人选。在这方面,石人一族怎有我雪民的优势?这就是扬长避短。”

“我族与方源联姻,虽有攀附成分,但方源那边有我族帮衬,对他也是有利。别的不好说,雪莲花精的存量还是有的。”

“再者说,此计若是不成,也无伤大雅,并不会恶了和方源的关系。不妨试一试。”

这个“试一试”的说辞,终于打动了冰卓。

后者点头:“那就试一试吧。”

至尊仙窍。

哗啦啦!

光阴河水的浪潮声,澎湃浩荡,激响在方源的耳中。

方源的神念四处横扫,观察片刻后,没有任何差错,便尽数收拢了神念回去。

动用了黑凡真传中的宙道手段,方源又再次将自家至尊仙窍中的光阴流速,还原到了本来程度。

至尊仙窍中的宇宙资源,是非常优越的,远远超过了特等福地。

宇道资源方面,整个至尊仙窍有五亿万亩,宽阔至极。

宙道方面,现在还原过来,和外界的流速比高达一比六十。十绝体特等福地,也只不过是一比四十。而上等福地,一比三十左右,比至尊仙窍整整缩减了一半。

同时,至尊仙窍的仙窍本源也很雄厚,仙窍中每年会有九十六颗的仙元产生。换算到外界的时间,那就是每天十六颗仙元。现在方源是七转蛊仙,所以得到的是十六颗红枣仙元。

十六颗红枣仙元,若是再换算成仙元石的话,那就是一千六百块仙元石!

也就是说,方源每天,哪怕自己什么都不动,光是仙窍方面自产的红枣仙元,就相当于一千六百块仙元石的价值了!

这就有点恐怖了。

但事实上,还远不止这样。当方源的宙道道痕再增加,仙窍的光阴流速也会增加。仙窍本源变得更加雄厚之后,产生的仙元还会更多。

到了八转修为的话,那就更恐怖!

若只按照最基础的水准,每天十六颗仙元的话,八转的黄荔仙元转化成仙元石,那就是十六万块仙元石!

方源在宝黄天中,倾销年蛊赚取了上千万的仙元石。等他到了八转修为,什么都不用做,只是外界百天时间,就是一千六百万的仙元石价值。

修为越高,资产就越雄厚。这种差距,不是简单的线性,而是几何级数地暴涨。

当然,这是方源的情况。

至尊仙窍,乃是特例,天底下独一份。影宗上下用了十万年积累,几乎全部用来炼制此蛊,有这样的恐怖优势,也并非难以理解。

之前,方源是被天意和灾劫逼得没有办法,才将光阴流速变缓。

但现在,方源战力飙升,有自信能够度过灾劫,同时又有砚石老人的仙道杀招石洞天机,可以提前测算出灾劫的内容。

再加上自身发展的需求,还原光阴流速乃是必然的了。

甚至,方源还打算动用宙道手段,将光阴流速再提升上去一些。

“不过目前还要等一等。黑凡真传中的宙道手段,每次催用之后,必须等候一段时间,才能催动其他杀招,来影响光阴支流。”

“有了这样的光阴流速,光照菌就能迅速繁衍了。态度蛊、慧剑蛊暂时先放在仙窍外面吧。”

方源算了一下,如此一来,他大概就能满足喂养态度蛊、慧剑蛊的最低标准了。

态度蛊、慧剑蛊也该到喂养的时候了。

但是八转仙蛊喂养难度太大,这个难题一直积压在方源的心中。并且随着时间的推移,压力越来越大。

现在,能解决这个问题,方源总算是松了一口气。(未完待续。)

\end{this_body}

