\newsection{蒙屠}    %第五百四十五节:蒙屠

\begin{this_body}

%1
在北部冰原的东南,墨人城以北,腐毒草原的西北方位,蒙家势力的南端,有一处地方,名为秋刀原。

%2
这里地貌平坦,寸草无生,乃是自然生长出的一块奇地。

%3
蒙屠一步步行走在秋刀原上。

%4
他身材魁梧,皮肤呈现红铜般的色泽,穿着马甲,露出肩膀,裤腿挽起,露出小腿。

%5
没有穿鞋,他正赤脚行走。

%6
一步又一步,平实的地面却锋利如刀,蒙屠踩踏在这层银白色的土壤上,就像是行走在刀山,每一步都留下鲜红的足迹。

%7
但他面色宛若石雕,毫不变化,整个人沉默如铁,紧闭双目,眉头微锁,心中存在疑惑。

%8
“这处秋刀原中,充斥刀道道痕,自然孕育出无数刃蛊。我徒步行走,徘徊往复,已数年光阴,究竟是缺乏什么,令我仍旧突破不了刀道的境界呢”

%9
刀道流派,是一个非常小的流派,和金道、炎道、土道等完全不能相比。

%10
提起刀道,就不能不提剑道。

%11
这两个流派一直恩怨极深,但又相互牵扯。因为这两大流派的源头,都是一个,那就是刃蛊。

%12
关于刃蛊,人祖传中就有明文记载,人祖行走在平凡深渊之中,在这里有些地方是泥泞沼泽,十分容易泥足深陷,而且恶臭熏人。有些地方是荆棘满布,尖刺密密麻麻,人祖被刺得伤痕遍体。还有的地底,埋藏着刃蛊。人祖踩在上面,脚底就被尖锐的刃边割伤,伤口宽大,血液横流,走起路来,痛彻心扉。

%13
刃蛊就是刀道、剑道的源头。

%14
自然界中也会产生许多野生刃蛊,秋刀原就是五域可数的刃蛊产地,由超级势力蒙家掌控。

%15
蒙屠修行刀道,修为七转,战力巅峰,年幼时期就有天才之名,成长迅猛,乃是北原蛊仙界著名的强者,蒙家的战力支柱之一。

%16
数年前,他就主动向家族蛊仙提出了要求,主动镇守这片秋刀原,一方面为家族看管这块重要的资源点,另一方面则是为了自己修行考虑。

%17
他的刀道境界,僵持在准大宗师的地步,蒙屠想要借助秋刀原,效仿当初人祖的行为,来令自己的刀道境界,真正跨入大宗师的行列

%18
但是这数年来,他心中虽有一些所得,但始终隔一层纱布,一直都未跨入大宗师的行列。

%19
蒙屠性情坚忍,虽然数年都未见结果,也丝毫没有放弃的念头产生。

%20
“唉,突破大宗师境界,何其难也族中太上长老们都赞誉我是千年不出的刀道奇才,但凭借我的天赋,数百年来也只是准大宗师。”

%21
“虽然始终感觉隔层纱布,但却是近在眼前,远在天边,难比登天啊”

%22
“不过,我还是要继续行走下去,这个方法还是有效的。总有一天,我必定能晋升为刀道大宗师”

%23
“嗯什么人”

%24
就在这时,覆盖在整个秋刀原上的蛊阵轰然崩溃,露出晴朗的蓝天。

%25
一个身影如魔似神,一言不发,魔威浩荡,向蒙屠直接扑杀过来

%26
“好胆”蒙屠顿时怒发冲冠,口绽雷霆之声,不闪不避,双手如刀,悍然反击过去。

%27
一声巨响,来者不动如山,蒙屠被反震巨力远远的击飞倒退开去。

%28
“我这仙道杀招别看气势全无,却是威能内敛到了极致,七转蛊仙的防护一般都要被我摧毁。没想到此人硬挨了我一击,居然毫发无损这是大敌”

%29
蒙屠心中吃惊不已,迅速稳住阵脚,瞪向来者,怒喝道:“来者通名竟胆敢犯我蒙家要地”

%30
“哈哈哈。”方源狂笑几声,声调已经全然改变,“蒙屠当然你不认识我,不过也没必要认识我,因为死人是不需要知道太多的。”

%31
万我仙蛊方中,需要海量的刃蛊,更需要秋刀原中地底深处的一种奇特的七转仙材饮刃酒。

%32
这酒并非天然生产,乃是由蒙屠数年来足下血迹,渗透到了地底深处后,汇同此处的刀道道痕,逐渐形成的奇物。

%33
方源五百年前世中,这蒙屠为了突破,进入刀道大宗师的境界,在这里夜以继日的不眠苦修,却始终没有等到契机。

%34
蒙屠突破不成,便想效仿人祖,要将刃蛊插在自己心头,更深入地体悟刀道。当他开创的杀招并不完善,有着隐晦的严重弊端,蒙屠便渐渐心态转变,时常陷入疯魔状态。

%35
蒙屠也知道不妥,但执念于此,仍旧修炼不辍,对其他人苦苦隐瞒。

%36
五域乱战开启,蒙屠参战,战场中却是弊端爆发,陷入疯魔之中,杀害了自家同族的蛊仙。

%37
蒙屠清醒过来后,毁尸灭迹,编造理由,隐瞒了家族。

%38
但终究事发突然,蒙屠处理程度有限,纸包不住火,渐渐被北原蛊仙察觉端倪。

%39
蒙屠压力极巨,这是天庭来人,秘密策反他。

%40
蒙屠起初心存侥幸,想要利用天庭,为自己建立功勋,将功补过,但天庭岂容他一个刀道蛊仙算计

%41
蒙屠反被天庭算计,一步错,步步错,最终走投无路,只好投靠天庭。

%42
天庭为了辅助这枚棋子,又推算出晋升大宗师的契机,便是这地底深处的饮刃酒。

%43
蒙屠依照天庭指点,来到秋刀原,深挖地底,开采出大量的饮刃酒。

%44
他小酌一口,五脏六腑宛若刀割,大喝一口,痛不欲生,再喝第三口,晋升成刀道大宗师。

%45
至此,他一门心思为天庭服务,成为继凤仙太子后,地位第二高的北原内奸。

%46
而后来他被马鸿运机缘巧合之下,揭露身份,那就是另外一番故事了。

%47
方源一番推算,算定此处已经有了饮刃酒,虽然量不多,但他需要的也少。至于这蒙屠,也是方源精心挑选,适合他试演阎帝杀招的对手。

%48
方源不可能找一个八转蛊仙,来试验阎帝杀招的成色,那是自找苦吃。普通的七转蛊仙,方源已经不放在眼里。七转巅峰战力的蒙屠,成名已久,又是刀道准大宗师,却是非常适合。

%49
所以,方源一番长途跋涉,来到秋刀原,直接找蒙屠的麻烦。

%50
“刚刚一击,乃是蒙屠的底牌杀招,威能不俗,但我凭借阎帝杀招,就抵御住了。这防御威能,还是不错的。”

%51
方源暗暗评估。

%52
此时的他,已经催动出了阎帝。

%53
他的整个人都覆盖了一层黄褐色的皇袍,变成高达一丈有余的巨人。

%54
巨人头戴冕冠,前后都垂下珠帘。面目一片深幽,似乎笼罩着一层雾气,又仿佛就是黑暗的漩涡。两袖宽大,玉带系腰,胸前背后有一层青铜之质地的甲胄。衣袍的其余表面,有着金线描绘的数千头鬼怪,排列整齐,纵横有序,不动如林,气度森严。

%55
刚刚蒙屠一击,也并非无效,阎帝胸前的青铜薄甲,就内凹了一个弧度。

%56
不过很快,方源的魂魄底蕴下降了一些后,这个弧度就立即重新鼓起,恢复原状。

%57
这便是阎帝杀招的特殊之处

%58
九转鬼不觉根本不需要消耗什么,自行运作,鬼官衣则依靠魂魄底蕴。

%59
两者参与组合的阎帝杀招,不需要仙元催动,需要耗费的就是方源的魂魄底蕴

%60
“区区蒙屠,不过如此。你在这里闭关数年,开创了什么全新杀招临死之前不用,那真是可惜了。”方源挑衅道。

%61
蒙屠气极哈哈狂笑,然后他用怜悯的目光,看向方源:“不管你是谁,来这里挑战我,真的是自找死路这里曾经有过仙阵守护,但你知道为什么我要做主,说服家族收回仙阵吗”

%62
“因为只要我在这里,这里就不需要仙阵护卫”

%63
说到这里,蒙屠右脚狠狠跺地。

%64
下一刻,地动山摇,亿万刀光升腾而起,无数刀道虚影汇集成大军,浮现而出,将方源团团围住。未完待续。

\end{this_body}

