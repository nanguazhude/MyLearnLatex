\newsection{暗歧杀}    %第十八节:暗歧杀

\begin{this_body}

“定仙游?”琅琊地灵诧异,语气迟疑。…≦,

方源知道他的顾虑是什么,无非是仙蛊唯一。

他淡定地回应地灵道:“放心,我有七成的把握,可以肯定定仙游已经不存于这个世上了。”

他找上泥相,问的问题,就是关于定仙游等等仙蛊。

方源问:定仙游、春秋蝉……仙蛊,是否之前都毁过一次。

当代泥相答:非。

于是,方源立即确定了两个消息。

第一个消息,定仙游等等仙蛊,有存留下来的,也有毁去的。

第二个消息,影无邪极可能已经识破了他在原来肉身上的布置。

当代泥相已经年过半百,催动一次说是道非仙蛊,就要耗费五十年寿命。因此回答了方源一个问题之后,倪家老族长便当场衰老至死。

但他的回答,已经帮助方源良多。

“若做最坏的打算,影无邪既然识破了我的布置,那么就必然隐藏得极好,不会让我找到他的。说不定,他甚至已经不在南疆。”

“黑楼兰、太白云生二人,我一直联络不上,这两人不是被策反,就是被害死了。”

“我原来身上的仙蛊,不知道具体毁掉的是哪些?但定仙游毁掉的可能性最大,因为我布置的特意,是让它第一个毁掉。”

方源当初之所以这样布置,一是顾忌定仙游的威能,二是以防夺回原本肉身之后,来不及一下子将特意收回。所以才令蛊虫陆续自毁。

“最大的可能是,定仙游毁掉之后。影无邪发现不妙,采取措施。将其他的仙蛊暂时封印住。”

“我和他之间,是魂魄互换。他比我的情况更糟糕,没有蛊虫可以自由操纵。居然能封印住我的其他仙蛊,这样一来,他要么是哄骗了黑楼兰、太白云生,要么是借助了影宗的残余势力。”

影宗既然是幽魂魔尊所建,又苦心经营了数万年,有残留势力理所应当。这一点,方源早已经估算到了。

“影宗遍布五域。连僵盟都是影宗的下宗。义天山大战在南疆发生,估计影宗在南疆的势力,损失最为惨重。但其余四域,却要相对好些。”

“若是定仙游已毁,影无邪要对付我,光凭黑楼兰、太白云生是远远不够的。必然要先行收拢影宗残存势力,积累优势,再来杀我。所以,他恐怕也想要炼制定仙游。”

方源的这具新身体。实在是出色。

思考起来,灵光频频乍现,如一道长河东流,毫无阻碍。

而对于他而言。定仙游同样是急需之物。

因为五域界壁的存在,还有星门蛊无法再对他这具新声有用,定仙游成了他穿越五域的最佳工具。

“只要有了定仙游。我就能直接回到琅琊福地,将所有的仙蛊。都收入囊中。”

“原先的修行资源,也可以都存放到九五至尊仙窍之中。”

“还可以进入地渊。将星象福地直接吞并了!”

不管哪一项,都能极大地增长方源的实力。

但若利用宝黄天输送的话,且不说宝黄天太开放,会为方源惹来无数的追杀,单论宝黄天的手续费用,就贵得惊人。

方源若强行动用宝黄天,只是输送八转慧剑蛊的费用,就要让他辛苦积累的修行资源付之一空,还远远不够。

“你确定要炼制定仙游?”琅琊地灵再次问道。

方源察觉到一丝不妥,眉头微扬:“怎么,难道这事情已超出你的能力之外?”。

琅琊地灵摇头:“当初和盗天魔尊的约定,我绝没有反悔抵赖之意。但是我的手中,虽然有神游蛊,还有其他辅助蛊材,却独独缺少最关键的那一份材料。你可猜得是什么?”

方源想了想,皱起眉头:“难不成是太古之光?”

“不错。”琅琊地灵回道,“不管你用何种蛊方,来试图炼出定仙游。这所有的蛊方中,都必须要有一份关键的蛊材,那就是太古之光。光芒这种东西,本身就难以捕捉和储藏。太古……更是超越上古、远古,是人祖存在的年代。那时候,整个世界中的人族,只有人祖和其十子。”

“不管用什么蛊方,都不成?”方源虽然知道琅琊地灵不会骗自己,但仍旧很不甘心。

“当然!”琅琊地灵回答得很肯定、干脆,“我手中的定仙游仙蛊方,也有两份。看过的定仙游残方,不下十种。不管哪一种蛊方,都绕不过太古之光。充其量,只是每一张蛊方中,太古之光的多少,有些差别而已。”

“定仙游乃是蛊中极品,用处极大。若非如此,早已经被人炼出,哪里还轮得到你呢?”

方源听到这里,这才明白,原来当初自己炼出定仙游,是多么的侥幸。

当初的太古之光,是借助萧芒手中的太光蛊。

但现在,萧芒早就死了。

因为萧山、萧芒,乃是萧家太上长老选择的两个棋子。

这两人是活脱脱的蛊仙种子,可惜身为凡人,就得沦为棋子。死到临头,恐怕都没有明白义天山的真相。

萧芒一死,太光蛊自然也就跟着毁灭了。

“萧芒已死,连魂魄都没有残留下来。我上哪里去弄太古之光来?太上大长老,你说,若是利用智慧光晕,是否能思考出不需要太古之光的定仙游仙蛊方?”方源又问。

涉及到九转仙蛊,琅琊地灵也不确定,毕竟他生前也只不过是八转蛊仙而已。

就算帮助两位尊者炼蛊,对于九转智慧仙蛊,也知之不详。

“罢了。就算能够运用智慧光晕,设想出不需要太古之光的仙蛊方。我现在已经不是仙僵,恐怕也运用不了。我还是先去萧家看看,萧芒的太光蛊是盗墓而得,兴许我能在萧家查看到什么线索。”

方源主动结束了和琅琊地灵的联系。

他改变方向,向萧家赶去。

不过在结束联络之前,他没有忘记,将屠杀倪家全族的魂魄收获,转给了琅琊地灵。

而后,又向琅琊地灵索要了一份清单。

清单的内容,自然是关于琅琊派中的剑道杀招。

方源一边向萧家赶去,一边查阅这些杀招。

剑道杀招虽有不少,但大多数都是凡道杀招,仙道杀招满打满算也只有三个。

其中一个,是仙道残招,如同残缺的仙蛊方,借鉴价值很大,但无法直接运用。

另外两个,一个是用于移动,一个是用于攻伐。

移动的仙道杀招,名为“剑闪雷音”,核心仙蛊有两只,分别是闪剑蛊、轰雷蛊。

攻伐的仙道杀招,为“暗歧杀”,核心仙蛊也有两只,分别是飞剑蛊、暗渡蛊。

又看到熟悉的暗渡仙蛊,方源心中叹息一声。

很久以前,他就曾向谋夺这只仙蛊,可惜没有成功。

姜钰仙子死后,暗渡仙蛊也随之毁灭。

“若是我有暗渡仙蛊,再加上手中正好拥有的飞剑仙蛊,这招暗歧杀,就巧好能用得上了。可惜……”

方源正盘算之时,忽然一道声音从侧方传来:“前方的仙友且慢。”

方源回头,只见一头荒兽气宗狮飞奔而来,狮子背上布置着一个大红色的长椅,上面坐着两位蛊仙。

中洲,地渊。

魂兽仰头长啸,发出无声的嘶吼。

“小心,魂啸仙蛊又发动了!”影无邪高声提醒。

无形的音波,掀起澎湃的气浪,向影无邪、黑楼兰、太白云生三人扫荡而去。

三人无奈之下,节节败退。

这只野生魂啸仙蛊,乃是七转层数,寄生在魂兽身上。影无邪、黑楼兰、太白云生三仙一旦将魂兽逼至险境,这只仙蛊就有所感应,爆发出来,给寄生的魂兽争取足够的时间和空间。

“怎么会这样?这头魂兽,不过是荒兽级数,身上怎么会有七转野蛊寄生?”太白云生七窍溢血,嘶声喊道。

其余两人也不好过,头晕目眩,魂魄受创不轻。

影无邪脸色相当难看。

真是人在家中坐,祸从天上来。

刚刚还好端端的,忽然就有一头魂兽,闯入影宗在这里布置的地盘中。

“魂啸仙蛊发动,不需要仙元,而是抽取魂魄底蕴。这头荒级魂兽,恐怕不久前还是上古级数。但经历激战后,败逃而来。魂啸仙蛊频频发动,使得它从太古魂兽,跌落到普通荒兽级数了。”

影无邪嘶哑着声音,回应太白云生道。

而同时,他的心中则泛起两个字天意!

“我影宗大计不成,我又违逆天意,重生之后,没有去和方源同归于尽。这就是天意来找我麻烦了。”

天之道,讲究万物平衡。

影宗之前隐匿不现,尽量不让天意发觉,暗中行事,所以阻碍甚少。

但现在影无邪已经彻底暴露,天意酝酿片刻,终于引出魂兽来攻。

“天意无处不在!”

“当初,方源炼定仙游,它就巧妙安排,将太古之光主动送到方源的眼前。现在又试图来铲除我。”

“难道我真的要把体内的春秋蝉毁掉不成?”

影无邪犹豫万分。

“这魂兽我们根本就无法近身!到了现在这个程度,你还藏着掖着干什么?还不赶紧启动这里的蛊阵?”黑楼兰则暗中传音,催促道。

\end{this_body}

