\newsection{再收红莲真传}    %第七百三十七节:再收红莲真传

\begin{this_body}

%1
北原。

%2
一片普通的贫瘠草原,荒无人烟。

%3
忽然,天空中闪过一道奇光,闪现出四个蛊仙来。

%4
正是方源、影无邪、妙音仙子、白兔姑娘四人。

%5
重生到这个时间段,妙音仙子、白兔姑娘还有潜伏在琅琊派中的间谍毛六,都未牺牲,仍旧活着。

%6
“好了,你们回去继续修行。”方源打开至尊仙窍门户,让三位蛊仙钻了进去。

%7
眼下影宗这些残余蛊仙,都龟缩在方源的至尊仙窍中修行。

%8
方源要利用他们组合上古战阵——四通八达,正是用了这个手段,方源才能迅速腾挪,长途跋涉。

%9
四通八达杀招对于方源而言,分外重要,弥补了他在移速上的短板。

%10
上一世,方源在失去了定仙游蛊后,这一招完美地替补上去,帮助方源渡过许多难关。

%11
“但真正比较起来,还是定仙游仙蛊更加方便啊。”方源心中感慨。

%12
他这一次从神针谷原址出发,跨越了几乎大半个北原,才来到这里。

%13
四通八达杀招用出来,气息无法遮掩,引人瞩目,方源为了隐秘在途中特意绕了几个圈,方才必过了数个超级势力。

%14
若是有了定仙游,自然不需要这么麻烦。

%15
方源在心中又补充一句:“当然,定仙游至少得有七转程度,才能适合我用。”

%16
“这个时间,七转的定仙游应该在凤九歌身上吧?”

%17
上一世,方源就是在琅琊保卫战中,动用大盗鬼手盗取了凤九歌身上的七转定仙游蛊。

%18
这只蛊虫带给方源不少帮助。

%19
即便方源后来修为提升到了八转,七转定仙游便充当核心,组建成了宇道仙级杀招。杀招威力提升无数,依旧帮助方源满天下四处腾挪。

%20
定仙游真的很方便,乃是宇道的极品仙蛊,有了它满天下都能乱窜。

%21
四通八达虽然也很不错,但比较定仙游蛊就稍逊一筹了。不管是跨越五域界壁或者天罡气墙,亦或者组成人数等等,都落在下风。

%22
“上一世是我运气好,运用大盗鬼手,盗取出了定仙游仙蛊。但这个法子,并不稳定可靠,靠的是运气啊。这一世,我该如何谋算才能得到定仙游仙蛊呢?”

%23
方源依靠春秋蝉重生了,带来了宝贵的情报、境界和杀招,但同时也让自身运道跌落了一大截下去。

%24
这些天来,方源都是依靠一些运道的小手段撑着。

%25
但在运道方面,说实在话,即便他闯荡过王庭福地,捣毁了八十八角真阳楼,但在运道上的收获,远远低于外界的估计。

%26
随着方源修为不断提升,这些运道的手段有些跟不上节奏了。

%27
“不过好在前番对战,不管是蒙屠还是睡姑,战力都远弱于我。即便我运势不好,也无伤大雅。”

%28
方源对之前的两次战斗,还是比较满意的。

%29
整个局势,都在他的掌控当中。

%30
这种尽在掌握的感觉,无疑非常美妙!

%31
前世他就缺少落魄印这样的强大的攻伐手段来一锤定音,所以只能依靠大盗鬼手出奇制胜,或者动用万蛟、阎罗子来消耗战,处境尴尬。

%32
这种尴尬接连导致了蒙屠自爆,琅琊守护战中方源败北逃生等等结果。

%33
上一世,方源为了落魄印杀招,不顾艰难险阻,闷头推算,耗费了海量心血精力,以及数年光阴。直到五界山脉大战时,他才推算成功,一经运用,就活捉了八转蛊仙君神光,把武庸都惊住了。

%34
这一世,方源直接用出落魄印,效果当然立竿见影。

%35
杀了蒙屠、睡姑,方源炼制万我仙蛊的主要仙材就搜集了三份,只剩下最后一份浮生火了。

%36
按照上一世的经验,浮生火并不需要方源亲自外出搜寻,到了一定的时间段,就会有中洲的蛊仙张继,在宝黄天中贩卖这个仙材。到时候,方源自己买上一些就可以了。

%37
方源来到这片看似平凡无奇的平原,是有另外的目的。

%38
他飘落到地上,犹豫了一下,还是催动了一记仙道杀招。

%39
燃魂爆运!

%40
“我现在魂魄底蕴并不多,燃魂爆运还是需要谨慎使用。但这一次是要进入光阴长河,为了稳妥,用一次吧。”方源在心中叹息。

%41
上一世,很多蛊仙都推算他,这种断断续续的推算一直持续到方源重生。

%42
每一次推算,虽然都没有成功,因为方源有着阎帝杀招。但阎帝杀招防备推算,消耗的也是方源的魂魄底蕴。

%43
眼下,方源虽是有一批胆识蛊的存货,但魂魄修行已经终止了。荡魂山刚在紫薇仙子的手上自爆,方源依靠手中的荡魂山碎块,要修复好荡魂山还需要很长一段时间。

%44
用了燃魂爆运,方源运势陡然上涨,但和上一世的同期运气还是无法媲美。

%45
春秋蝉的弊端很大,随着修为上涨,运气跌落得更加巨大。真不知道当初红莲魔尊是如何克服的。

%46
方源一边迈步一边催动蛊虫,在草原上绕了一个不大不小的圈后,他轻轻一跺脚,便开启了这里的天然仙阵。

%47
随后,他步入仙阵中去,身形消失在原地。

%48
和南疆相差不多,这里同样也有一处天然的光阴支流,影宗发现之后,动用阵道手段,结合这里的天然道痕,布置出仙级大阵,将这里隐藏起来,无人发觉。

%49
方源看着流淌着的光阴支流,不时的有宙道凡蛊从河水中飞出来,或者盘旋在河水上空。

%50
方源沉下心神,小心翼翼地催动了一记宙道仙级杀招。

%51
此招乃是他几天前推算出来,名字还未来得及起,主要的作用就是遮掩自身的存在,防备侦查。

%52
用了此招后,方源深呼吸一口气,噗通一声,跳入光阴支流。

%53
顺着支流,他很快进入到光阴长河的主流当中去。

%54
在极远的地方,一座仙蛊屋静静地停顿在光阴长河的河面上,任凭河水滔滔,掀动一阵阵的壮阔波涛。

%55
这座仙蛊屋造型乃是一座凉亭,四面透风,结构简朴。亭盖仿佛是黄草编织,亭柱是灰扑扑的白石,并未磨平。亭中有一屏风,算是最为华丽的装饰。

%56
此刻亭中或站或坐,有四位蛊仙,三男一女。

%57
正是中洲灵蝶谷的蛊仙四旬子,他们皆专修宙道,有着七转修为,并且是孪生兄妹,情深义重,关系密切。

%58
方源借助幽魂魔尊发现的那一份红莲真传,铲除了天庭蛊仙黄史上人之后,天庭在宙道方面的人才就有些短缺了。

%59
四旬子虽然不是天庭成员,但他们实力不俗,且又忠心耿耿,被紫薇仙子提拔上来,镇守今古亭,看守光阴长河。一旦方源进入光阴长河,他们就会立即察觉。

%60
“嗯?”此时此刻,上旬子忽然轻咦了一声。

%61
“怎么了?”四人当中最小的妹妹旬果子询问。

%62
作为兄长的上旬子,手指着今古亭中的屏风,有些迟疑地道:“我刚刚好像看到了屏风微微一颤的光。”

%63
“有吗?”中旬子和下旬子对望一眼。

%64
“那就查一查吧。”旬果子尽管性情活泼跳脱,但此刻身负重任,丝毫不敢大意。

%65
四旬子便催动今古亭,狠狠侦查了数十次,皆没有任何的发现。

%66
一切正常。

%67
“看来是我看错了,哈哈。”上旬子有些赫然。

%68
“大哥,你勿要太过紧张。那方源不过是七转巅峰而已,我们四旬子不依靠今古亭,都有能力胜他。”中旬子劝慰道。

%69
黄史上人虽然折了,但天庭已经查明,是红莲真传的力量,也分析出了方源此时的宙道实力。

%70
旬果子妙眸一转,嬉笑起来:“放心吧,放心吧,就算三位哥哥都战死了,只要有我旬果子在,你们一定能再活过来的。”

%71
“呃,居然咒兄长战死?”

%72
“你呀……”

%73
“真讨打!”

%74
“哥哥们饶命!”

%75
今古亭中一片欢声笑语。

%76
嬉笑打闹中的他们,却万万没有料到,方源真的已经进入了光阴长河!

%77
上一世方源被今古亭发现,即便是有了一座仙蛊屋雏形。

%78
但这一世不同了!

%79
首先,方源面对的只有一座今古亭,时间太早,天庭还未来得及组建另外三座宙道的仙蛊屋。

%80
其次,方源对今古亭非常了解。上一世的今古亭就在他手中毁了两次,而四旬子都是死在他的手中,其中三位的魂魄都在战后被方源俘虏,进行了搜魂,更加了解今古亭中的许多奥秘!

%81
最后,方源重生换取了好几只宙道仙蛊,其中一只宙道七转仙蛊——时隐,便是能隐去蛊仙身形的蛊虫。以它为核心组成的仙道杀招,专门针对今古亭的侦查,又在光阴长河这样的环境中得到巨大增幅……

%82
如此种种,七转的今古亭终究没有探查出方源的踪影。

%83
方源在光阴长河中潜游,一路跋涉。

%84
自然有一些艰难险阻。

%85
比如突泉,危险的突泉能陷杀八转蛊仙。又比如光阴斑斓,这种苍白光斑能削减寿命,任何蛊仙都要深深忌惮。

%86
还有常见的年兽。光阴长河中多的是荒级年兽、上古年兽。

%87
一路潜游过来,方源遭遇了至少三次太古年兽。

%88
幸亏他来之前用过了一次燃魂爆运,运气不错,许多麻烦都顺利渡过,大致上是有惊无险。

%89
对于这些年兽,方源都以躲避为主,尽量不出手。

%90
一方面是因为今古亭,方源一旦动手,暴露身份的可能性就会暴涨。

%91
另一方面是方源现在还没有年华池呢。没有年华池,就无法大规模屯兵。捉了太多的年兽,对于方源而言,反而是一种累赘。

%92
终于,春秋蝉开始颤动了。片刻后,方源找到了石莲岛!

\end{this_body}

