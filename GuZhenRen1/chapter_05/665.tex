\newsection{华文洞天}    %第六百六十七节:华文洞天

\begin{this_body}

太古白天。

一朵祥云承载着方源,向前飞射。

方源心神一半留在至尊仙窍之中,一半则警戒外界。

换做之前,心神平分两半来炼蛊,方源是做不到的。不过最近这段时间,他真的收获很大。

境界、蛊方他都不缺,唯一的短板就是实践。

亲自炼蛊令他飞速提升,升炼七转仙蛊的效率再次飙升一截,成功率也暴涨了一块,因此消耗的仙材也比之前少得多。

炼制仙蛊从来都是一个大坑,不过方源除了炼蛊之外,本身也在太古白天中收刮了许许多多的天然资源。

他将一部分贩卖到宝黄天中去,买来自己所需之物。除了用来炼蛊,还有许多剩余都用来经营仙窍。

宝黄天中生意火爆得不得了,地沟震荡,导致五域蛊仙都有迫切交易的需求,市面上的仙材交易极其红火。

当然,最红火的仍旧是宙道仙材。

天庭方面始终毫不动摇地压制方源,任何宙道仙材都能高价收购,从未有力竭气虚之像,展现出了天庭超绝恐怖的雄浑底蕴!

若是天庭知晓方源已经获得了一份红莲真传,拥有大量的宙道仙材,不知会怎么想?

“还有两只六转仙蛊,其余的都已经升炼上了七转。咦?”

方源忽然身躯一震。

祥云速度再增一倍,激射而去。片刻后,方源再次见到了天相杀招。

只见灵鹤外形的天相杀招,此刻正围绕着一片平平无奇的空气,不断地盘旋飞绕。

“发现一处仙窍洞天了。”方源心头大喜。

随着天相杀招不断侦查,这片仙窍洞天内里的情景,都不断地传入到方源的心里。

“原来这个洞天,叫做华文洞天……竟是一位八转信道蛊仙的仙窍……哦?这片洞天中人口极多,并且还拥有很独特的修行风俗!”

方源不禁面露新奇之色。

华文洞天中的人口,比琅琊福地的毛民还要多。初步估计,蛊仙的数量会有不少。

这片洞天以信道道痕为主,洞天环境自然对信道资源更加有利。所以蛊师、蛊仙基本上都是专修信道,操纵信道蛊虫。

华文洞天与世隔绝已经有许多岁月了,但因为吞并了太古九天的碎片,这里的天灵一直浑浑噩噩。

不过,洞天的原主人施展的仙道杀招仍旧有效。此招名为――济文才,似乎只要洞天一直存在,就会一直有效,而效用也非常独特。

任何一位纯正的人族,只要创造、吟诵出出色的文章,题材不限,就能够获得这片洞天的感知和奖赏。凡人吟诵好诗,就能吸引万里之外的希望蛊飞来,为其开窍。蛊师创作出妙词,就能得到蛊虫,甚至是修为直接拔升。

“虽然这片洞天我不能吞并,有些遗憾,不过我心中的惊世诗词可是不少呢。”

方源哈哈一笑,当即灌输仙元到天相杀招中去。

天相杀招徐徐放光,浸染到华文洞天中。方源缓缓飞入白光之中,就要降临华文洞天。

“好胆!何方妖孽胆敢犯我家园?!”一道正气浩荡的声音旋即传来。

随着,一位白袍书生形象的老者,携带着八转气息,出现在方源的面前。

方源此刻还在白光之中,不能随意出手,老者却无丝毫犹豫,伸手一推,顿时无数的文字显形,宛若暴雨一般,激射而来。

天相杀招并非攻伐之用,暴雨般的文字倾泻而下,白光轰然崩散,方源处境很尴尬,上不上下不下,最终他只能主动撤退出去。

“这里竟然有一位八转蛊仙镇守!”方源有些意外。

按照正常情况而言,仙窍洞天中出现八转蛊仙的例子很少,因为洞天本身很难有这样的资源来支撑起一位八转蛊仙。

这片华文洞天的天灵又是懵懵懂懂,不晓得打开仙窍,蛊仙们就很难出得去。

就算是这位八转蛊仙有手段,能偷渡出去到太古白天中收刮仙材,但此事风险也很大。太古白天中的太古荒兽、太古荒植数量不少,各种诡异凶险的地方对八转蛊仙也有威胁。这里不是八转蛊仙的后花园,他们要在这里收集资源,不仅是需要运气,更要冒风险。

“换一个地方再试试。”方源休整了一下,第二次将白荔仙元灌输到天相杀招中去。

天相杀招高达九转,几乎可以在任何的仙窍洞天中联通内外。第一次是在高空,而第二次方源选择在地底深处。

然而当他再一次踏入白光中,想要偷渡进华文洞天的时候,那位八转蛊仙竟然再一次的及时出现。

“竟还敢来此?有老夫在,你绝无一丝可能!”八转蛊仙说着,便立即出手。

方源身处白光之中,非常无奈,因为他不能催动任何其他的手段。

没有办法,他只得任由白光被对方打散,自己再次退到了外面。

方源脸色不免阴沉下来。

他在心中思量:“前后两次的地点,相差何止十万八千里?简直是天各一方,居然仍旧被察觉,然后被及时阻挠。这是什么防护的手段?”

被察觉出来,是有很大可能的。

毕竟这处华文洞天乃是信道洞天,信道又是最擅长侦查,收集情报、传播消息的流派。

但对方八转蛊仙居然也能迅速转移地点,及时出现在方源的面前,这是什么手段?

接下来,方源又试着出手了几次,不只是天相杀招,定仙游的杀招也用了,都被对方迅速发现,阻拦下来。

方源通过白光,需要的时间很短,但这段时间在八转蛊仙的阻挠之下,却显得极其漫长。

这一刻,方源充分体会到了当初,百足天君进攻黑凡洞天时的感受了。明明自己的实力要高于对手,但是进入仙窍洞天却是相当困难,被敌人卡住关口,难受极了。

“华文洞天,总有一天我会回来的。”方源选择了撤退。

反正华文洞天的位置,已经被他探查清楚了。

它是吞并了太古九天碎片的洞天,位置不可能再移动了。

在方源铩羽而归的时候,龙公却带着凤金煌,来到了东海的某处海底。

群蛟在海底里嘶吼,掀起暗涌和狂澜,却丝毫奈何不了龙公。

龙公周围方圆数百步内,一片风平浪静,凤金煌静静地站在他的身旁。

龙公带着凤金煌缓缓下沉,片刻后,凤金煌的眼前忽然涟漪一动,像是突破了一层无形的薄膜。随后,她双眼瞳孔微微一缩,发现了一座雄阔的宫殿,静静地盘踞在海底。

八转仙蛊屋――龙宫!

它通体散发橙金微光,雕梁画栋,亭台楼阁均华丽壮观。不过在龙宫周围,始终围绕着一大片的梦境。

这些梦境不断地流动,并且流动的速度还很快。龙宫被梦境时遮时掩,诱惑着来者。

梦境变幻无常,尽管有时候会出现一些缝隙,似乎能够穿梭,让外人直接进入龙宫深处,但龙公却非常谨慎,宁愿多等一段时间,不去冒不必要的风险。

现在,他已有手段来对付这片梦境了。

仙道杀招――纯梦求真变!

他把凤金煌的梦翼仙蛊借用过来,仙招本身则是紫薇仙子主导推算的成果。

在杀招的作用下,一小块的梦境迅速变成一具人体,双目紧闭,陷入沉眠之中。

“这就是纯梦求真体,并且还是毫无弊端的完美体。”龙公双眼绽**芒,细细打量,他也曾经被这个纯梦求真体坑过,因此此刻心中有些感怀。

“煌儿,你这番改良真是绝妙!从此之后,这个纯梦求真体就可算得上是第十一绝体了!”龙公又赞叹道。

凤金煌眨了眨眼:“师父,不瞒你说,徒儿也是机缘巧合。因为前段时间,徒儿自创出了碎梦杀招,而这番改良正是用了碎梦的成果。徒儿也是纯粹想要尝试一下,没想到竟然真的成功了!”

龙公笑道:“不必不安,我的好徒儿。你以为你自创出碎梦杀招,是巧合吗?这应当就是宿命的安排。是宿命让你……”

凤金煌顿时撇嘴:“师父,你能不能别说这种话了,徒儿可不想再听了。若是这一切都是宿命的安排,那要徒儿我有什么用?”

龙公被噎了一下,想要说什么,但最终只是长叹一口气:“罢了罢了,不说这些了。说这些对你来说还太早,时候到了,为师相信你会悟的。”

\end{this_body}

