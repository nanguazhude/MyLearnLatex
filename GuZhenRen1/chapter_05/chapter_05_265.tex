\newsection{方源的坚持!}    %第二百六十五节:方源的坚持!

\begin{this_body}

%1
“你若是认为,等到我们混战,寻机脱身,那你就趁早打消这个主意吧。我雪胡今日的目标,就只有你一个。你知不知道,我为了炼制鸿运齐天蛊,付出了多少的代价?你坏我修行大事,就是我生平最仇恨的人。我不把你杀死,难消我心头之恨!”雪胡老祖阴森森地望着方源,一番话像是一波冰凉彻骨的冷水,浇在方源的心头。

%2
但方源仍旧面无表情。

%3
他就像一个石头,一个铁做的聋哑人,任何的话语,不管是威胁还是引诱,都没有任何的效果。

%4
时间流逝,方源在不断前行。

%5
不知不觉间,狗尾续命貂毛里球的吼叫声,低弱了下来。

%6
这头巨大的紫貂,脸上的神情发生了变化。

%7
它用一种很古怪,充满难以置信的语气,望着方源呢喃开口道:“不会吧……”

%8
“不会什么?”玄极子感到奇怪。

%9
就在这一刻,雪胡老祖、碧晨天、威灵仰的脸上,也都发生了微微的变化。

%10
他们也感应到了什么。

%11
很快,在场其他的蛊仙也俱都动容,纷纷惊呼出声。

%12
“唉?”

%13
“这是蛊虫的气息?但又有些似是而非!”

%14
“什么蛊虫?不可能,难道他也有九转仙蛊不成?”

%15
“不对劲。这不是仙蛊的气息,或者说不完整,好古怪!”

%16
忽然,余艺冶子双眼一亮,他猜到了真相,脱口而出:“这个情形,他正在炼蛊!”

%17
立即有蛊仙嗤笑:“怎么可能?在逆流河中炼蛊?”

%18
是啊。

%19
逆流河中无法动用蛊虫,除非是九转仙蛊。

%20
方源虽有智慧蛊,但留在了琅琊福地,他根本无法动用任何蛊虫,更遑论炼蛊。

%21
方源也感受到了体内的玄妙变化。

%22
他面无表情,心中却是震动:“似乎……真的在炼蛊?这是怎么回事?随着我每一次前行,体内的气息便壮大一分。怎么回事?”

%23
虽然方源并不了解具体的原委,但是他却明白,这似乎是他的转机!

%24
情况已经糟糕透顶,不管怎样,方源愿意继续试下去。

%25
他继续朝前走。

%26
一步又一步。

%27
他的步伐很稳定,面无表情,却给人一种相当强大的感觉。

%28
“为什么他面无表情?”不知哪位蛊仙忽然开口问道。

%29
群仙这才反应过来。

%30
是啊。

%31
除去方源之外,其他跋涉之人,不是神情扭曲,就是痛苦万分,不是疯狂,就是哭泣。

%32
为什么方源,他始终面无表情?

%33
群仙沉默。

%34
他们都知道,问这个问题的人,其实内里更想问的是另外一个问题。

%35
那就是

%36
“为什么这个柳贯一,他能坚持这么久?并且,他似乎还能坚持更久,甚至能永远坚持下去的样子!”

%37
这怎么可能?

%38
大家都被淘汰,都被逆流河冲刷出来。

%39
他怎么还留在这里,还一副留有余力的样子?

%40
凭什么是他?

%41
为什么是他!

%42
没有人能回答这个问题,所以蛊仙们沉默。

%43
片刻之后,随着方源不断的前行,他体内蛊虫的气息不断壮大,已经到了一个极致。

%44
很快,方源的身上,从内而外,开始散发出一种洁白的光辉。

%45
“居然……真的发展成这个样子了!”毛里球看到这一幕,双眼瞪得溜圆,张口结舌,无法相信眼前的景象。

%46
“毛爷,这究竟是怎么回事?”玄极子忍耐不住,问道。

%47
“唉!事到如今,我即便说出来,也无所谓了。”毛里球发出深深的叹息,脸上有一股深深的挫败之色。

%48
“他的体内,正在酝酿出一只仙蛊。这只仙蛊的名字,就叫做坚持!”毛里球语出惊人。

%49
“坚持仙蛊?!”玄极子震惊无比。

%50
其他的蛊仙也好不到哪里去。

%51
“《人祖传》中有言,坚持仙蛊就是可以征服逆流河的关键仙蛊?它不是不存在吗?”余艺冶子发问。

%52
“不,它存在。”碧晨天道。他是天庭蛊仙,知晓无数秘辛。

%53
“并且曾经就有人掌管过它,那个人,便是天庭之主,元莲仙尊!”威灵仰补充道。

%54
“什么?!”众人再惊。

%55
“只是坚持仙蛊的炼法,我们从未料到,竟是这样子的。元莲仙尊也没有透露过。”碧晨天摇头叹息。

%56
“他当然不想透露。因为这可是他生平的一件丑事,嘿嘿嘿。”毛里球插言,任何打击天庭声誉的事情,它都颇感兴趣的样子。

%57
此言立即换来施正义的叫嚷:“胡言乱语,仙尊之名,岂容你诋毁?!”

%58
能够当面和传奇太古荒兽叫嚣,施正义勇气可嘉,中洲不少蛊仙都不禁为他捏了一把汗。

%59
毛里球却没有恼怒:“小娃娃,你懂什么?仙尊魔尊,修为虽高,但都是人。是人就有感情,就有破绽,就有弱点。”

%60
“当初元莲仙尊就是闯进逆流河,被困在河中,无法脱身。结果长期跋涉在最上游,导致体内炼出了坚持仙蛊。从此之后,他就征服了逆流河,成为了历史上,逆流河的第一代主人。”

%61
“逆流河的第一代主人?”就连雪胡老祖都流露出吃惊的神色。这条逆流河落到他手中,已经好一段时间了,没想到还有这等秘辛。

%62
“这么说的话……”一下子,群仙的目光都转移到了方源的身上。

%63
毛里球叹息一声,目光深沉地望着沐浴在白光中的方源,无奈地道:“没有错。按照这样的情形发展下去,他就要炼成坚持仙蛊,成为逆流河的第二代主人了!”

%64
众仙失语,就看着方源一步步朝前走去。

%65
他面无表情,不管走多少步,逆流河永远流淌在他的脚下,仿佛是绝大的命运的嘲笑。

%66
但是他仍旧走着。

%67
他从前世五百年走来,不知要走到什么时候。

%68
但他知道,自己要去往何方。

%69
似乎……没有人能阻止他。

%70
至少……如今的逆流中,已经无人可阻。

%71
前世五百年前。

%72
方源倚在竹楼上,看看山寨,又仰望背后的青茅山。

%73
双手握拳,稚嫩的小脸上,满是希冀。

%74
“是时候放弃过去了。”

%75
“穿越到这里来,这是我的福缘!因为在这里,可以实现长生。”

%76
“我要把握这样难得的机会!不然,怎么对得起自己,对得起这份机缘?”

%77
“当然,目前阶段,是提升我和弟弟的生活环境。嘿,那个小家伙……”

%78
开窍之后。

%79
演武场上,方源垂下头,一脸惊怒。

%80
“被暗算了!”

%81
“是谁暗算我?不愿意让我战胜对手?答案不言而喻!”

%82
“哥哥,放弃吧,你不会是我的对手。因为天资不同,我们注定不同。”眼前的古月方正带着快意道。

%83
方源转头就走,他的眼中全是坚定之色。

%84
“既然山寨不栽培我,舅父舅母甚至都故意排挤我,我留在这里有什么意思?”

%85
“难道在这里就可以变强,实现长生么?”

%86
“不如出去?”

%87
商队里。

%88
大胡子蛊师死了,方源站在他的墓前。

%89
少年满含泪水,哽咽地道:“胡子大叔,你安息吧。”

%90
“谢谢你临走前的礼物。”

%91
“你说:小时候,你想成为顶天立地的人物,就像是正道的那些传奇人物那样。少年时,觉得成为一族族长也不错。青年时,能够成为家老就感觉很棒了。中年后,被家族流放,发现其实能养得活自己,养得起身上的蛊虫,就能让自己满意。”

%92
“我不会这样,让梦想随着年龄而萎缩。”

%93
“这个世界太大,而我们都是小人物……但我会加油的!会一直努力!”

%94
……

%95
童年、少年、青年。

%96
青茅山、商队,一路行走。

%97
壮年、老年,终究获得寿蛊。

%98
南疆、西漠、东海、中洲。

%99
春秋蝉重生后,青茅山、三王山、狐仙福地、王庭福地、义天山、逆流河!

%100
一步步走来,一路风雨。

%101
碧晨天皱起眉头,他盯着方源的身影,心中呢喃道:“这是何等的意志!他究竟为什么坚持?是什么能让他如此坚持?”

%102
雪胡老祖冷哼,眼中闪过郑重之色,再无之前面对一般七转蛊仙的轻蔑:“这么说来,三十万年前有元莲,三十万后有柳贯一……逆流河主啊。”

%103
毛里球望着方源身上越盛的光辉,无可奈何,龇牙咧嘴,爪子下意识地在地面上挠,挠出道道深痕。

%104
白凝冰、黑楼兰俱都眼角狂跳,神情动容。

%105
赵怜云此刻悠悠醒转,她望着方源另一个胳膊下夹着的,马鸿运的尸体,她的眼泪夺眶而出。

%106
她在心中哭嚎:“鸿运,鸿运,你怎么可以离我而去。没有了你,我在这个世界上,就是孤家寡人。我活着还有什么意思?你知道吗?一个人的坚持是有多难!”

%107
一个人的坚持会有多难?

%108
在场的所有蛊仙,都能回答这个问题。

%109
因为他们当中,有的因为责任而坚持,有的因为仇恨而坚持,有的因为精彩而坚持,有的因为爱情而坚持……

%110
而方源的回答呢?

%111
他仍旧面无表情,毫无所动地向前进。

%112
我曾经呐喊过,渐渐的我不发出声音。

%113
我曾经哭泣过,渐渐的我不再流泪。

%114
我曾经悲伤过,渐渐的我能承受一切。

%115
我曾经喜悦过,渐渐的我看淡世间。

%116
而如今!

%117
我只剩下面无表情,我的目光如磐石般坚硬,我的心中剩下坚持。

%118
这就是我,一个小人物,方源的坚持!

%119
光芒骤放,不可逼视。

%120
坚持仙蛊,在这一刻,炼成!!!

\end{this_body}

\newsectionindepend{蛊真人的坚持!}

\begin{this_body} %begin a body

%121
这一章的高潮,我修改了很多次,不惜删减了许多存稿,最终几乎是重新写的。

%122
结果,总算是写出了我心中的一些感觉。

%123
毫无疑问,这个剧情的中心思想便是坚持。

%124
《人祖传》一如既往地起到了点题的作用。

%125
在前几天,本书限免,订阅正版的读者朋友们可能在逛起点书评区的时候,看到了一些其他新读者对本书的不佳评论。

%126
其实这些已经算好的了。

%127
诋毁、谩骂、牵涉家人的诅咒、一次次的举报。我从四五年前开始写这本书,经历的这种东西不胜枚举,数不胜数。

%128
其中有这样一种观点,能够写出这样主角的作者,一定也是个变态。

%129
这个观点太好笑,也有些可悲。

%130
照这种逻辑,写小白文的作者就是本性小白?写黑暗文的作者就本性黑暗?写厚黑传的李宗吾先生就厚黑?那我之前写过不少种马文,我是不是种马了?

%131
很多人不理解方源,就好像不理解我创作本书的初衷一样。

%132
方源一直都没有让自己的梦想萎缩。我也始终没有忘记创作本书的初衷,那就是写一个反派。

%133
我知道绝大多数人都看惯了正派,因为那样做很轻松,没有道德上的压抑。然而现实生活永远不那么轻松,而多是冰冷和残酷。所以我建议许多读者朋友们,再看本书之前,好好看看本书的序言。

%134
我是要写一个反派的,写反派,就无法不写残酷和冰冷。

%135
很多读者朋友们不喜欢马鸿运、赵怜云。

%136
我理解你们,真心的!

%137
但是对我而言,我爱主角方源,也爱那些配角,诸如白凝冰、马鸿运、黑楼兰、影无邪等等。

%138
因为在他们的身上,有着他们各自的意志,个人的精彩,每个人不一样的机缘,不一样的思想。

%139
他们是活生生的,影射着我们现实生活中的一部分人,一些方面。

%140
总有人需要自己努力,总有人运气特好,能不劳而获,总有人起点高能拼爹,总有人颜值高惹人爱……

%141
社会就这样,我写的小说,营造的是一个世界,它也理应这样,包含这些东西。

%142
而我本人倾向于个人努力和个人自我价值的实现。

%143
所以,主角是方源。

%144
我知道,大多数的读者朋友们,也倾向于个人努力和自我价值的实现,所以喜爱方源这个角色。

%145
马鸿运的死,看似有些搞笑,但我觉得是一种必然。

%146
今晚微信公众号做一起马鸿运的专题,对这个人物进行解读和分析。

%147
我觉得,咱们每个人活在这个世界上,应该有一颗包容的心。我们喜爱方源这个角色,不妨也看看其他角色的闪光点。就好像现实生活中,我们身边总会有一些我们比较讨厌的人。但如果我们换一个角度,说不定能发现这些人身上的闪光之处。

%148
有这样眼光的人,是有福的。

%149
因为他(她)的心胸会开阔,不会因为外在而容易惹得自己不开心。

%150
人能活着开心,就是挺美的一件事情不是吗?

%151
更重要的是,这些人目光开阔,不会因为心中的成见,阻碍自己的认知。

%152
这样看的话,也会发现一些生命中很有意思的事情。至于前几天的那些书评,也就能站在一种高度来俯瞰这些了。

%153
谢谢大家的支持。

%154
方源一直在坚持。

%155
作者我也一直在坚持,从未放弃!

%156
再次感谢大家一如既往的支持。请大家多多订阅正版,这样的支持永远不嫌多。谢谢大家了!

%157
请大家关注我的微信公众号“蛊真人”以及“作者蛊真人”,今天做马鸿运这个颇受争议的人物分析,也希望大家多能发表自己的看法。

\end{this_body}

