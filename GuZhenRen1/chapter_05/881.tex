\newsection{老好人}    %第八百八十五节:老好人

\begin{this_body}

蜂巢岛是一座造型奇特的海岛。

整个海岛就是一个体积庞大的暗黄蜂巢,蜂巢表面坑洞无数,内部更是结构精密,坑道无数。

一位六转蛊仙,狼背蜂腰,身着碧绿战甲,正身处于蜂巢之中。

此人正是庙明神的追随者之一——蜂将。

咻咻咻!

密密麻麻的赤线蚯,铺天盖地般向蜂将暴射而来。

赤线蚯一曲一弹,就能将自己射出去,速度极快,蜂将也只能看到一道道的红色光线。

赤线蚯的攻势非常犀利,成群结队的赤线蚯能够让荒兽避退。

蜂将且战且退。

他承受相当巨大的压力,不一会儿,已是满头汗渍。

他战力当然远超荒兽,有着不少杀招手段。但他接取的任务,却是平复蜂巢岛中的赤线蚯灾患,让整个蜂巢岛的生态重归平衡。

蜂将若是不择手段,发动猛烈攻势,当然能够轻轻松松地解决眼前的这些赤线蚯。

但是如此一来,蜂巢岛就会遭受惨重的打击,蜂将收获的功德将会少得可怜。

“可恶。”蜂将心存顾忌,只好退避三舍,飞出蜂巢,悬浮于高空。

赤线蚯群仍旧紧追不舍,在空中弹射,追杀蜂将。

从无数的洞口中,喷涌出大量的红线,蜂将叹息一声,拔高身躯,远离蜂巢,这才让这些赤线蚯无功而返。

赤线蚯从高空坠落下来,也没有摔死,它们身躯柔软又坚韧,在蜂巢表面盘踞一阵后,就都重新钻入蜂巢之中去。

它们不是顺着蜂巢原有的坑洞,而是直接钻破蜂巢的岩壁。

这对蜂巢岛而言,自然是极大的伤害。

“赤线蚯太多了,里面恐怕还有荒兽。毕竟我接取的可是中型任务啊。”

“还有一处难点,这蜂巢内部的通道越往内部延伸,就越是狭小,对我大大不利。”

蜂巢岛已经被赤线蚯蛀得千疮百孔,蜂将若是在内部开战,恐怕要把整个蜂巢岛都毁了。

蜂将沉默良久,终于下定了决心。

他打开仙窍门户,从中唤出一大群的野蜂。这些野蜂十分精锐,只只都有五转的战力。它们的首领更是气势勃发,乃是一只荒兽野蜂。

野蜂的腹部形态奇特,不是椭圆体,而是陀螺一般,腰部最为宽大,从腰部延伸出去直至尾部,越来越小。

陀螺般的腹部表面,还有黑色的螺旋纹路,相当醒目。

所有的野蜂浑身都用浓密的绒毛,翅膀又大又宽。

这是黄陀蜂。

蜂将的仙窍中,豢养了大量的野蜂,黄陀蜂便是其中一种,地位十分特殊。

寻常的野蜂,虽然成群结队,但都是同一种类。不同种类的野蜂并不杂居。

然而黄陀蜂不仅能和所有的野蜂一起生活,还能够在生息繁衍中,助长另外的蜂种,帮助它们壮大族群。

黄陀蜂可以说是蜂将仙窍中的核心蜂种。

蜂将乃是东海蛊仙,修行资源丰富,仙窍中豢养了多种蜂群,并且有十多头荒兽野蜂。黄陀蜂一直是受到他的重点栽培,但荒兽级的黄陀蜂只有三头。

这一次,蜂将取出这一群的黄陀蜂,对他而言,绝对是付出了巨大的代价。

在蜂将的指挥下,黄陀蜂群嗡嗡嗡地仿佛一团黄褐色的云,迅速钻进蜂巢之中。

蜂巢岛中还是存在着蜂群,只是比起赤线蚯的规模,完全处于下风。

有了黄陀蜂群的帮助,蜂群先是混乱了片刻,旋即就开始展开反扑,对蜂巢岛下部盘踞的赤线蚯群发动猛攻。

这是族群的生死存亡的大战,自然是非常惨烈。

蜂巢岛不断震动,表面迅速出现大量的裂缝,并且剧烈扩张。

蜂将早已准备就绪,此时立即出手,动用种种手段修补蜂巢。

他让蜂群主攻,自己则沦为了保姆。

也是他运气好,当赤线蚯群损失达到了某种程度之后,那头唯一的荒兽赤线蚯不战而退,率领着族群,主动离开了蜂巢。

蜂将看着海水中,数百万的赤线蚯交汇成一道巨大的红流,向着远方开赴,长叹一声,心头的巨石终于放下。

他犹豫了一下,对于这些赤线蚯,是继续追杀,还是放任不管呢?

犹豫的功夫,赤线蚯群已经潜入深海。

蜂将放弃了追击。

在这深海中交手,牵一发而动全身,这片海域中还盘踞着为数不少的海兽。

“多一事不如少一事,我还是将这蜂巢岛修补一番吧。”

这番修补又耗去了他一天一夜的功夫。

蜂巢岛修补完善不说,还被蜂将加固了许多关键地方。同时,他还从自家的仙窍中掏出了许多蜂蜜、完善生态的特殊花草。

做完这些,蜂巢岛已经是焕然一新。不仅生态重归平衡,而且比之前还更有发展潜力。

“这已经是一个中型资源点了。”蜂将感慨。

若是有可能,他当然要将这座蜂巢岛收入囊中,但是事关功德任务,又且受着乐土拘束,蜂将只能遗憾地放弃这个想法了。

传送回到功德碑,蜂将看了看自己的功德,发现这一次任务完成,他的功德顿时增添了八十多。

蜂将楞了楞,旋即就后悔起来:“没有达到一百功德,看来我做的并不完美。当初应该将赤线蚯群剿灭才是。放任它们离去,将来恐怕蜂巢岛又会被它们侵袭。”

“也不对。任务是要我平衡蜂巢岛的生态,我真的做到了吗?其实并没有。”

“所谓的生态,虽然核心是蜂巢岛,但海岛附近的海域也是蜂巢岛生态的一部分。我只是重点改造了蜂巢岛,却没有对周遭的海域下手,这是我的失误。”

想到这里,蜂将叹息一声。

数天前,方源回到功德碑下,见到榜单后,立即明白了情势,便主动将好人名号的秘密,告知了庙明神一伙人。

蜂将这才换取了好人名号,可以接取中型任务。

现在,他完成了第一个中型任务,但心情并不多么欢畅。

这不仅是因为他没有解读透彻整个任务的内容,更关键的是他在这个任务中,付出了许多。

为了获取功德,那一群黄陀蜂他是无法回收的。

当然,这样的交换是正确的。

毕竟功德关乎乐土真传,蜂将付出蜂群,得到功德,累积上万就可以换取仙蛊!

这个买卖绝对超值。

因为在外界,蜂将根本就没有这样的渠道去获取仙蛊。再多的荒兽野蜂,也换不来一只仙蛊。

“只是中型任务的难度,真的很高。我擅长豢养蜂群,这个任务还是对口的,也只是得到八十多的功德。”

“其他任务呢?”

蜂将看着碑面。

有许多的任务就放在那里,但他不能接取。

比如说修补仙阵的任务,蜂将根本就没有手段来修补仙阵啊。

又比如说治疗某个上古荒兽的伤势,蜂将的治疗手段根本拿不出手啊。

蜂将忽然间明白过来,为什么方源要保守好人名号的秘密。

只要率先领跑,方源就能领取到适合自己的任务,从而迅速地积累功德。

但蜂将想想,又感到了蹊跷:“不对。就算如此,楚瀛积累功德的效率也未免太高了。他每一次任务几乎都是完美完成下来。难道变化道真的这么实用?”

就在这时,蜂将楞了一下。

在他身边,白光闪过,方源出现在他的眼前。

“楚瀛大人。”蜂将主动行礼。

方源对他点点头,目光旋即转移到碑面上。

在碑面上,他的功德从垫底位置,猛地蹿升,跨越大多数的人,又登上了前三位。

“一下子上涨了五百多功德?!”蜂将见此,心头狠狠一颤。

“我又花了一千功德,兑换了名号大好人,因此可以接取更高一等的任务。”方源笑着道。

蜂将再楞,旋即深深一拜:“多谢大人指点!”

方源点点头,走到另一碑面,兑换了第三个名号“老好人”,顿时他的功德锐减五百,再次垫底。

蜂将张了张嘴,十分好奇方源的选择,但很识趣,方源没有说,他自然不会去问。

但蜂将可以肯定,方源一定是兑换了极有帮助的东西。

“算算时间,他们也该回来了。”方源走到蜂将身边,并没有再接任务。

蜂将点点头,知道方源口中的“他们”指的是庙明神三人,回应道:“应该是的。”

蜂将在庙明神一伙人中,实力并不出众,他已经完成任务回来,其他三人也快了。

方源不在乎等这么一小会儿。

果然,不久后庙明神等人陆续回来,见到了方源。

蜂将便将大好人名号的秘密,告诉了庙明神三人。

庙明神心头一动:“好人名号可以接取中型任务,大好人可以接取大型任务。这是一脉相承,知道好人名号,大好人名号的秘密其实已经不算秘密,很自然就能联想到。”

但即便如此,庙明神对方源还是非常感激的。

因为方源在此之前,主动提点他,告诉他好人名号的事情。而沈家、任修平那些人却始终瞒着他们。

功德榜的前后顺序非常关键。许多奖励都是唯一,先到先得。

“庙兄不必害怕沈从声。我有把握可以在传送出去的时候,远离沈从声等人,距离足够我等安然离去。”方源开口就是谎言,却叫庙明神等人心头一震。

鬼七爷皱起眉头:“是功德碑上的一种奖励吗?”

“是的。”方源点头,实则根本不知。

庙明神等人却不由地相信他。

为什么呢?

看看楚瀛之前对沈从声的态度,太作死了!如此有恃无恐,必定是有所依仗的。

方源又道:“我可以接取大型任务。这不是我一个人能够胜任的,需要诸位帮助。”

说着,他将碑面上的大型任务接过来,转身就分享给了庙明神一伙。

庙明神等人心神一震,同时接收到任务的讯息,纷纷脸色微变。

还有这等操作?!

事实上,这个秘密还是上一世庙明神自己挖掘出来的。

老好人名号,就是能将任务分享出去。

对于蛊仙而言,一个中型任务有时候就很困难,有帮手的话,就会比较容易。虽然功德分薄了一些,但是完成任务的效率提升了,长久来看,利远远大于弊。

庙明神可以选择接受,也可以不接受。方源能分享任务,但并不能强制他人接收。

庙明神只是思考了一下,便点头接受了这个任务。

中型任务他刚刚完成了一个,和蜂将的感受差不多。

方源是全流派兼修,才能做到手到擒来。而庙明神等人是正常的蛊仙,专修一道,很多时候只有一部分的任务适合他们。

既然完成中型任务并不容易,还会有不少的付出,庙明神更愿意和方源合作,做一番新的尝试。

\end{this_body}

