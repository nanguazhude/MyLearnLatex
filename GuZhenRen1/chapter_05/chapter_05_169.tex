\newsection{血原武斗大会}    %第一百六十九节:血原武斗大会

\begin{this_body}

%1
北原,凤仙洞天。

%2
八转蛊仙凤仙太子,静静地盘坐着,双目紧闭。

%3
他的周身火光缠绕,似霓裳飘扬,绚烂多姿,似龙蛇飞舞,卓而不群。

%4
蓦地,凤仙太子张开嘴巴,向外吐出一颗小小的火焰珠子。

%5
火焰珠子随风见涨,一个呼吸之后,就膨胀成了婴孩脑袋大小的橙红火球。

%6
火球围绕着凤仙太子不断盘旋飞舞,时上时下,忽左忽右。

%7
须臾,凤仙太子又张开嘴巴,再吐出一颗火焰珠子。

%8
片刻之后,凤仙太子的身边已经有了八颗火球,在呼呼飞旋,相互之间并不碰撞,各自以玄奥的轨迹,始终围绕着凤仙太子。

%9
就在这时,凤仙太子猛地睁开双眼。

%10
锐利无比的精光,从他的眼中透射而出,让人不可逼视。

%11
砰砰砰……

%12
八颗火球相继自爆开来,化为零散的火焰,旋即消散在空中。

%13
“又失败了么,没有推算出任何方源的迹象。”凤仙太子目光炯炯,紧紧盯着火球爆散开来的火焰。

%14
若是推算成功的话,这些火焰会在瞬间描绘出方源伸出的位置,以及他正在做什么的画面。

%15
但显然,凤仙太子再次失败。

%16
他一无所获。

%17
“我到底不是智道蛊仙啊,仅仅以炎道手段,模拟智道,看来是无法推算出方源的位置了。”凤仙太子叹息一声。

%18
他受命缉拿方源,但苦恼的是,方源潜藏得太深了。很多时候,他都深居福地中,不出来。福地、洞天,只要不是融合了九天碎片的话,就完全是另外一个世界,和五域外界完全隔绝,就算是紫薇仙子也推算不出来。

%19
凤仙太子主修炎道,更是不可能了。

%20
当然。方源也会外出。

%21
但他有见面曾相识,更有暗渡仙蛊。搭配他身上不少的暗道道痕,使得方源每次都能遮掩自身的行迹。

%22
“凤仙大人,宫家来人。要求见大人。”这时,一位身着黄衣的女蛊仙,来到大厅,开口汇报。

%23
她貌美如花,双眼水灵灵的。樱桃小嘴,正是凤仙太子的两位蛊仙侍女之一乐瑶。

%24
“宫家来人?哼。”凤仙太子眉头微扬,“不见。”

%25
“可是宫家的蛊仙,带来了长生令。”乐瑶迟疑道。

%26
凤仙太子冷笑一声:“长生令又如何?它管得了黄金血脉,我又不是。”

%27
乐瑶微微撇嘴:“可是大人,你忘了,我们现在都是安插在北原的卧底。你表面上的身份,还是宫家的女婿,外姓太上长老呢。公然不遵长生令,恐怕不妥吧?”

%28
“哈哈。”凤仙太子朗声一笑。“乐瑶你只说对了一半。长生天的确不容小觑,它是九转洞天,有着巨阳仙尊的底蕴,深不可测。在北原就如同中洲的天庭!但长生天这一次忽然下令,要结合黄金部族,各大正道超级势力围攻百足家,这是黄金血脉对外的统一争战。八十八角真阳楼的倒塌,百足家的成立,以及楚门的创建,连续挑逗长生天的底线。长生天这一次。绝不会下令来要求我动手。药皇才是他们唯一的选择!”

%29
凤仙太子一针见血的分析,让乐瑶眼前一亮,但她仍有不解:“那为什么宫家蛊仙,会专门手持着长生令而来?”

%30
凤仙太子眼中闪过一抹冷光:“宫家这些人。这是狐假虎威,想利用长生令糊弄住我,让我出面。宫家的心思,你又不是不知道,这些人还做着成为正道第一的美梦呢。”

%31
“原来如此,那咱们就不见了!”乐瑶笑起来。露出珍珠般的贝齿,笑容靓丽,宛若春景。

%32
一切果然如凤仙太子分析的那样,几天后,长生令传遍了整个正道的黄金血脉家族,最终落到了药皇手中。

%33
药皇用枯朽的老手,拿捏着长生令,悠悠地叹了一口气。

%34
“这长生令,终究还是要到我手中的。”

%35
听他这样自言自语,原来药皇也早有觉悟。

%36
事实上,情势其实很明显。

%37
百足天君乃是八转蛊仙,北原正道势力联合起来,要对付百足家,那么就势必至少要有一位八转蛊仙来对付百足天君。

%38
纵观北原明面上的八转蛊仙,百足天君首先排除,雪胡老祖是魔道,五行大法师是散修阵营,冒犯长生天,已经被关押在劫运坛之中。只剩下凤仙太子和药皇。

%39
凤仙太子出身是最大的硬伤。

%40
他不是黄金血脉,只是宫家的女婿!

%41
若是依靠凤仙太子,战胜了百足家,那么岂不正是应了五行大法师的话了吗?堂堂的黄金血脉,巨阳仙尊的子孙们,难道就能依靠自己解决强敌吗?

%42
所以长生天方面,最属意的始终只有一人,那就是八转炼道蛊仙药皇了!

%43
其实,药皇内心深处真的不想摊这潭浑水。

%44
这段时间,他还在忙着炼制起死回生仙蛊呢!

%45
被严重干扰,药皇的心情自然很不愉快。

%46
啪。

%47
一声轻响。

%48
药皇将手中的长生令,随手丢在了桌子上。

%49
“还是先找百足天君谈一谈罢。”药皇长叹一声,收起仙窍,来到北原外界,冲入白天之中。

%50
片刻后,接到药皇邀请的百足天君,犹豫了一下,还是决定先和药皇会面,交谈一番再说。

%51
两位八转蛊仙面对面,在白天中进行了密谈。

%52
其实他们之间,颇有交情。

%53
很早之前,百足天君就为创建百足家做打算了,所以在当时身为散修的他,就和药皇走的很近。

%54
百足天君想要在正道立足,但正道蛊仙只有两人,凤仙太子独居一方,出身不正,和宫家的关系有不好。

%55
百足天君能够亲近的目标,其实也只有药皇一人。

%56
后来,百足天君还和药皇一起,斗过雪胡老祖。虽然双双战败,成全了雪胡老祖的名声,但此事也让百足天君和药皇之间的关系更近一层。

%57
所以,双方见面,并无火气或者剑拔弩张的氛围。

%58
“老友,怎么说?”药皇首先笑道。

%59
“唉,事情发展到这一步,不瞒你说,我也是始料未及啊。”百足天君摊开双手,他肚子里的确有一肚子的苦水。

%60
他原本的目的很单纯,就是想要强攻下黑凡洞天。

%61
但没想到楚度的个人实力,还有战斗力,要超出百足天君之前的估计。导致事情一再变化,让百足天君无法抽身。

%62
其中,北原蛊仙界的那些谣言,关于百足天君威胁论的话,基本上都是楚度安排人手随意散播出来的。

%63
这点,百足天君心知肚明。

%64
明眼人都知道。

%65
但现在,事实是,百足天君和楚度这个放出谣言的罪魁祸首联手,对付他根本不想的正道势力。

%66
百足天君走到这一步,他很无奈。

%67
药皇见到百足天君的这番表情,深刻理解他的感受,药皇指着他,笑道:“老友啊,现在你明白了吧?身为一方正道势力的首脑,这日子并不好过啊,很多时候哪怕我们是八转蛊仙,也要身不由己。”

%68
“我以前羡慕你散修的身份,也劝说过你,不要建立什么正道势力。你现在感受到了吧?”

%69
“感受到了。但我努力了这么久,好不容易有如此成果,我不甘心放弃。”百足天君直接坦言道。

%70
药皇沉默了一下,这才开口:“我其实也不想掺和这件烂事。但身不由己,长生天的令牌已经落到我的手中。”

%71
百足天君长叹一声,心中既有欣慰,也有悲哀。

%72
欣慰的是,他和药皇长期培养出来的交情,足够深厚。悲哀的是,他和药皇明明是好友,却不得不开战。即便他们是八转蛊仙,也是身不由己。

%73
百足天君深吸一口气:“那我们就只好做过一场了。”

%74
药皇却摆手:“这是迟早的事情,但现在不忙。我虽然总领此事,但药家只是正道势力之一,我上头可还有长生天瞧着呢。他们摩拳擦掌,我们先打一场,解决不了这事情。依我看呐,就让小辈们先锻炼切磋好了,磨磨他们的火气。”

%75
百足天君沉吟片刻,点头道:“那就如此罢。咱们定下个日子,来一场武斗大会。”

%76
“好。”药皇想了一下,“血战平原位置不错,就择址此处,如何?”

%77
“甚好。”百足天君一口答应下来。

%78
两位蛊仙交谈了一小会儿,就定下了血原武斗的事情,然后双双下了白天,回归各自家族。

%79
他们分别将血原武斗大会的事情,安排下去。

%80
一时间,北原蛊仙界再次热闹起来。

%81
一场崭新的风暴,正在酝酿,隐隐有将整个北原蛊仙界席卷进来的趋势。

%82
黄金部族的蛊仙们,各个摩拳擦掌,他们自觉地己方势力浑厚,稳压百足家和楚门的联盟,因此士气旺盛。

%83
百足家和楚门的联盟,单论底蕴,的确是比不上正道势力的联合的。但是他们可以邀请帮手啊。

%84
百足天君乃是散修阵营,楚度则是魔道中远近闻名的强者!

%85
距离约定的时期,还有一段日子。血原武斗的消息,已经风传出去,就连中洲等外域,也有耳闻。

%86
双方开始积蓄势力,正道黄金家族开始精挑细选,选取精兵强将,而百足家和楚门则开始广邀好手,积极准备。

%87
ps:今天第七更!真心爆不动啦……几乎虚脱。累死了,我要睡觉去!

\end{this_body}

