\newsection{方源最强攻伐!}    %第九百二十八节:方源最强攻伐!

\begin{this_body}

%1
天庭战场。

%2
仙道杀招——龙啸波。

%3
仙道杀招——随身闪。

%4
激战中,龙公身形陡然出现在万年斗飞车的右舷一侧。

%5
距离太近了!

%6
几乎在他出现的同时,环绕万年斗飞车飞舞着的破晓剑群就蜂拥而上,瞬间淹没了龙公。

%7
龙公身上浮现出无数龙形纹身,抵挡住剑群的猛烈攻击。

%8
这个正是龙公的主要防御杀招——九纹龙护身。

%9
九只是虚指,意味着无数的纹身。这是罕见的被动防御手段,平时潜伏在龙公的皮肤上,一旦遭受攻击,就会自行催发,根本不需要龙公的任何念头进行维持,极其方便。

%10
龙公凭借此招硬生生顶住破晓剑群的攻势,随后他双手合拢并成一拳。

%11
仙道杀招——憾世龙锤!

%12
轰!

%13
龙锤重击之下,直接将万年斗飞车打飞。

%14
万年斗飞车被击退,忽然停滞下来,周围的空间陡然崩溃,像是无数破碎的玻璃片。这些玻璃片每一个都锋利无比,但凡碰撞到万年斗飞车上,都会切割出一道道的深痕,造成大量蛊虫的湮灭。而站在船首的方源,也因此脸颊上被划破一道血口。

%15
这是苍玄子的手笔,利用宇道杀招铺设在战场各处的无形陷阱,威力十分了得!

%16
但这一幕,只是在方源的脑海中闪现,还并未发生。

%17
这是宙道杀招——三息后现!

%18
三息后现是来自黑凡真传中的侦查杀招,能够让蛊仙看到三个呼吸之内的有关于自己的未来。

%19
经过方源周身道痕增幅,威能暴涨了一千五百倍,因而可以在和龙公级别的强者大战中发挥作用。不过,因为此招本质上只是七转,因此并非次次有效,有时候灵验,有时候失效。

%20
下一刻。

%21
仙道杀招——龙啸波。

%22
仙道杀招——随身闪。

%23
龙公身形陡然出现在万年斗飞车的右舷一侧,一如三息后现侦查杀招展现的那样。

%24
方源冷哼一声,操纵万年斗飞车立即远离,和无影无形的宇道陷阱擦肩而过。

%25
“可惜!”苍玄子看到这一幕,心中感到分外遗憾,只差一点方源就能出发宇道陷阱了。

%26
她悬停高空,没有任何明显的动作,暗地里却是有一个个的宇道陷阱,不断地形成,悄然铺设着。

%27
和方源激战许久,沉睡在她记忆深处的战斗经验又纷纷鲜活起来,苍玄子的手段开始越发老道。

%28
自从三十多万年前,苍玄子和元莲仙尊达成秘密协约后,她就搬迁到了天庭过上了安稳的日子,几乎没有出手过。但在此之前,苍玄子的名号还有太古传奇的地位,都是她亲手打出来!

%29
方源脚踩万年斗飞车,一边躲闪,一边遥遥注视苍玄子。

%30
苍玄子虽然战力在不断复苏,但战斗风格非常明显,她保留着身为太古荒植的习性,在作战中倾向于停驻一方,遥攻打击,利用宇道手段拉开自己和敌人的距离,再铺设宇道陷阱。

%31
从这个战斗风格中,亦可看出她的性情,喜欢稳坐钓完鱼,非常有耐心,但同时这也意味着她并不激进,不太会选择冒险。

%32
尽管苍玄子的手段还有许多神秘,但方源已经开始渐渐摸清苍玄子的底细。而与之相反,龙公却是仍旧琢磨不透、深不可测。

%33
虽然方源上一世就和龙公作战,龙公的手段也大多被方源了解,但龙公的作战风格却是千变万化,时而保守时而激进,忽而狂攻忽而闪避,让人摸不清头脑,找不到明显的脉络。

%34
这种战斗的境界,不单单是战斗经验,还有龙公的天赋异禀!

%35
对于方源而言,要击败苍玄子虽然困难,但大有希望。但要击败龙公,到目前为止,他仍旧看不到丝毫可能。

%36
“除非是直接破解了龙公的防御杀招!”方源心中不断分析。

%37
龙公最主要的防御杀招,乃是九纹龙护身,方源已经在战斗中全力搜集线索,不断地解析它。

%38
但收效甚微。

%39
要临阵直接破解强敌杀招,是一件非常困难的事情。

%40
比起九纹龙护身杀招,方源更想破解的是龙御上宾。但这个杀招更加无迹可寻,方源连搜索情报线索的途径都没有。

%41
本来按照预想,偷生杀招是对付龙公的最佳手段,但是经过亲自试验,偷生杀招又不管用……

%42
“等等,我好像忽略了什么?”忽然间,方源脑海中灵光一现。

%43
但当他想要深究下去的时候,又被袭来的攻击打断。

%44
方源不用瞧也知道,这是仙蛊屋诛魔榜在出手。

%45
这座八转仙蛊屋的确是个麻烦!

%46
它没有短板,速度极快,并不亚于万年斗飞车,并且防御也十分雄厚,方源强攻突破也是可以的,若单对单,诛魔榜绝非方源对手,当然的确要耗费好一番手脚。

%47
偷道手段是对付仙蛊屋的不二法门,但眼下要在龙公、苍玄子的面前使用,并不容易。

%48
方源目前变化为太古年猴,和万年斗飞车搭配,可谓相得益彰。一旦转变成偷道道痕,战斗力必定下滑一大截,给龙公、苍玄子带来机会。

%49
更让方源顾虑的是,就算他变身偷道太古荒兽,施展出了大盗鬼手,也未必能击中诛魔榜。就算打中,也未必能盗取出仙蛊来。方源的运道优势已经被秦鼎菱弥补了一大截,诛魔榜中又是古月方正,从运势上更克制方源。

%50
所以,偷道手段并不是改变战局的正确法门。

%51
“方源……”此时此刻,主持着诛魔榜的古月方正心绪相当的复杂。

%52
方源对他而言,是亲兄弟吗?

%53
不算是。

%54
方源的肉身换了,魂魄也是天外之魔。

%55
是陌路人吗?

%56
也不是。

%57
童年、少年的记忆不可磨灭,他们俩个的确曾经相依为伴过。

%58
愤怒吗?

%59
有的。

%60
方源玩弄自己,罔顾自己的性命,将自己安排在琅琊福地中,定是有阴险的图谋。

%61
仇恨吗?

%62
当然也有。

%63
方源屠杀亲族,是方正童年的阴影和梦魇。

%64
那么,除此之外呢?

%65
经历了这么多事情,方正开始逐渐地理解方源,有时候甚至不得不佩服方源。

%66
人在江湖,身不由己。方正逐渐明白自己的浅薄,拥有了自知之明。所谓的甲等资质,不过是一种自恋,方正已日趋成熟。

%67
除了理解、佩服之外,方正对方源甚至还抱着一丝感激。

%68
他感谢方源带给他的挫折,带给他的痛苦,带给他的磨难。这些都是财富,帮助方正成长至今,若没有琅琊福地的战乱中的颠沛流离,方正有怎么可能融合诛魔榜主的种种修行经验,自我开创出血亲心仇杀招呢?

%69
曾经,方正梦想着有这么一天,自己能够出手,和方源交手,亲自报仇。

%70
但现在当这一刻真正的来临,方正的心情却和自己料想的大相径庭,他没有复仇的快感,也有超越的欣慰。

%71
他的心绪很复杂,很复杂。

%72
“龙公、苍玄子、方正和秦鼎菱,除此之外,还有……紫薇仙子。”方源陷入围攻,面色越发严肃。

%73
伴随着龙公越来越强,天庭一方的蛊仙们配合越发默契,方源脑海中的念头几乎要全部用来维持战斗。

%74
他对整个局势的判断力正在不断地下滑。

%75
不管是催发杀招,还是操纵万年斗飞车,都极其消耗心神。方源还要总揽大局,照看各方战场,暗地里对各域蛊仙施加影响。

%76
事实上,龙公等人也是如此。随着战况越发激烈,他们催动杀招越发频繁,杀招威力越大,往往步骤越多,参与的蛊虫越多,需要牵扯的念头、心神也就越多。

%77
“若是让紫薇仙子休养好了,不管她是来参战,还是总揽大局,都对我相当不利。”方源眼中寒芒闪烁。

%78
“哈哈哈!”龙公蓦地大笑,毛脚山的最新战况已经被他得知,“方源,任凭你重生多少次,也是无用的。动用你的全部手段吧,再不用就晚了。”

%79
方源面沉如水,他也获悉了最新战况。

%80
帝君城战场,他痛失分身房睇长,豆神宫投敌。毛脚山战场,凤九歌现身,一首命运歌威力绝伦,即便是帝藏生都落入下风!

%81
天庭的底蕴真的是太深厚了。

%82
方源不只是和龙公等人作战,他是和天庭数百万年的积累在较量!

%83
外界不值得依靠,所有的压力都转到方源的肩膀上来。

%84
方源目光如电,陡然凌厉起来:“的确不能再留手了。”

%85
交手以来,他一直都没有找到龙公的破绽。但现在情势所逼,即便没有破绽,方源也必须彻底全力出手。

%86
万年斗飞车迅速升空,拉开距离,方源开始酝酿杀招。

%87
龙公并不追击,悬停于空,仰头凝神注视。他神色如临大敌,再不敢小瞧方源分毫。

%88
苍玄子深呼吸一口气,十万分戒备。

%89
秦鼎菱、方正以及战场边缘养伤的紫薇仙子,亦都聚精会神,目光紧紧盯住方源,一眨不眨。

%90
就连正在疯狂摧毁炼道大阵,轰击监天塔的冰塞川都抽出一部分心神,密切关注方源的一举一动。

%91
谁都知道,接下来的战斗胜负极可能将决定整个胜局。

%92
仙道杀招——十二生肖战阵。

%93
方源先是召唤了十二头太古年兽,施展出了这个已经暴露的手段。

%94
但和之前不同,十二生肖战阵并未凝聚汇变成一头巨大怪物,而是转变成了一个金光闪闪的船舵。

%95
船舵仿佛车轮,有六个轴,船舵表面是十二个把手,每个把手上各有十二个神态各异的年兽雕纹。

%96
十二生肖战舵飞入万年斗飞车之中,迅速融合一体。原本银光璀璨的万年斗飞车表面,忽然升腾起一大股黄金光影,宛若飞沙,变幻无穷。黄金光影时而变作金龙蜿蜒,时而化作巨虎雄踞,时而成为雄鸡高歌,时而又成为蛮牛飞冲……

%97
十二头太古年兽和万年斗飞车完美的结合在了一起!

%98
方源曾经的构想,在此刻展现在世人面前。

%99
“他竟是将上古战阵和仙蛊屋完美地融合在了一起!”秦鼎菱不由瞪大双眼。

%100
方正呆呆地看着。

%101
“才情了得!”即便是龙公,也不由地开口称赞。

%102
他从未见过这样的一幕,这是历史以来第一人!

%103
虽说上古战阵、仙蛊屋的本质,都是仙道杀招,但是将两个杀招融汇起来,在方源之前,还从未有人成功过!

%104
“小心,它攻过来了!”苍玄子大叫,流露出一丝惊惶之意,万年斗飞车此刻的气息恐怖得让她都感到震惊。

%105
仙道杀招——破晓激流!

%106
万年斗飞车俯冲而下,沿途击穿一层层的空气,发出凄厉的音爆炸响。

%107
它本就是相当优异的八转仙蛊屋,融合进来的上古十二生肖战阵更是拥有着和龙公作战的雄浑战力。两者相互交融之后,战力叠加,更令人忌惮!

%108
龙公低啸一声,施展出气墙。

%109
万年斗飞车冲势稍微一滞,就直接冲破了龙公气墙。

%110
龙公眉头一皱,再次催动气墙。

%111
万年斗飞车毫无疑问地再破一道气墙。

%112
龙公再建,万年斗飞车再破。如此连破五道,万年斗飞车终于冲势大减。

%113
方源冷笑一声,仙道杀招——万年围猎!

%114
下一刻,万年斗飞车下的光阴长河虚影陡然扩张,无数的年兽宛若潮水般汹涌而出,旋即就铺天盖地!

%115
漫天的年兽大军向仙墓展开冲锋,像是滔天的海啸要吞没一切。

%116
大军当中有数不胜数的年兽,有大量的上古年兽,太古年兽也绝不少见。

%117
更加恐怖的是,这些年兽大军一概听从方源调遣,简直是如臂指使。

%118
“恐怖!单凭这一招,方源就能以一己之力对付天底下任何一个超级势力啊!”

%119
“他是想要将光阴长河中的年兽都召唤过来吗?”

%120
“这几乎就是翻版的前有古人杀招了!只不过红莲的手段是召唤历代蛊仙强者,而方源是召唤光阴长河中的年兽!”

%121
“但是万年围猎杀招,不是只能勾引年兽,而不能指挥操纵的吗?”

%122
“这恐怖就是十二生肖战阵融汇之后的质变吧!”

%123
天庭一方震动不已。

%124
方源的底牌果然是强悍绝伦,只是翻出第一张,就展现出滔天威能。龙公、苍玄子等人顿感棘手。

%125
这支年兽大军的规模着实太大了,并且方源还在源源不断地召唤,究竟他的极限在哪里?

%126
年兽大军分散开来,四面开花,攻击扩散到天庭每个角落,更加让人手缺乏的天庭一方难以防守。

%127
苍玄子狠狠咬牙,不断催动宇道杀招,企图袭杀尽可能多的年兽,缓解当前压力。

%128
“小心!!”忽然,龙公惊呼。

%129
“嗯?”苍玄子似乎看到了一柄小刀的虚影,她瞳孔猛地缩成针尖大小,她这才刚刚反应过来,就已经中招了!

%130
“怎么可能……这么快?我明明已经铺设了……”

%131
苍玄子难以置信地望着方源,浑身僵滞,在她的额头上出现了一个血洞,前后贯通。伤口处充斥着浓郁的宙道道痕,即便是太古传奇的自愈能力也无法复原。

%132
仙道杀招——光阴飞刃!

%133
这是源自中洲薄家的超级手段,此刃一出,例无虚发,无不中的,中无不死,曾令整个中洲胆寒。

%134
杀招的内容极其繁杂,步骤十分多变,在方源掌握的手段中,复杂程度是当之无愧的第一。

%135
更变态的是,它施展条件苛刻,酝酿的时间不得超过三息。为此方源每一次催动,都需要辅助杀招缩时。

%136
它的后遗症也非常恐怖,每一次使用,关于它的记载都会消弭一部分。它如同橡皮擦,抹除强敌的同时,也在消耗自己。

%137
击中苍玄子之后,方源顿时心头一空,自己关于光阴飞刃的记忆模糊了大半。

%138
“按照这个程度,待我再射中龙公之后,恐怕就要彻底失去此招了。”方源心中如冰雪般冷静。

%139
他瞥向龙公。

%140
下意识地,龙公向后飞退了几步。

%141
轰!

%142
一声巨响,苍玄子坠落到地面上,在半空中她就还原本体,化为了一株巨大的青藤。

%143
这株青藤曾经撑天踏地,宛若天柱极其醒目,但现在它却是栽倒在仙墓上,一动不动,气息完无。

%144
仙墓遭受这一击的轰砸,损失远超之前总和,但龙公却顾不得这些了。他目光如电,死死地盯着方源一眨不眨。

%145
至少增幅一千五百倍的光阴飞刃杀招,他必须全神贯注,才有可能防御得住!

%146
诛魔榜中,方正张大嘴巴,看着地面上的苍天藤,满头都是冷汗。

%147
他身后的秦鼎菱更是身躯僵硬,宛若石雕。

%148
“如果刚刚这招射中诛魔榜,我们能挡得住吗?”两人心中都在闪过类似的疑问,他们根本没有信心说一声“防得住”!

%149
这一招的威能也太过惊悚了,一招下去,苍玄子生死不知,气息衰弱至无。

%150
“不要被他所摄,他刚刚施展此招,是借助了万年围猎杀招的气息遮掩。单纯运用此招,气势必定博大,不可隐藏。”关键时刻,紫薇仙子的话传遍战场。

%151
“不愧是智道大能。”方源冷哼一声,被说中了弱点。

%152
龙公的脸色这才缓和了一些。

%153
但下一刻,方源竟主动离开万年斗飞车,独自飞扑而下。

%154
“他要拼命了!”一瞬间,龙公心头大凛。

%155
仙道杀招——万物大同变。

%156
仙道杀招——太古剑龙!

%157
嗷吼——!

%158
方源瞬间变成一头巨龙。

%159
六转七转为蛟,八转称之为龙。

%160
太古剑龙!

%161
龙躯长达百丈,龙头上龙角如双枪,锋锐至极,龙瞳一片银白之光,龙鳞反而纯白,散发森森剑意。

%162
方源一身的道痕,都因此转变成了剑道道痕。

%163
而剑道,谁都清楚,这个流派的优势就在于攻伐!

%164
在此刻,方源终于展现出了最强的攻伐姿态。

\end{this_body}

