\newsection{成与败}    %第四十七节:成与败

\begin{this_body}

轰!轰!轰!

一声声的巨响,连绵不断,响彻云盖大陆的上空,震耳欲聋。[看本书最新章节请到求书 小说网www.Qiushu.cC],

“这是天鼓雷音劫!”上古战阵天婆梭罗凝成的银光巨人中,琅琊地灵低呼着。

他的脸上充满了震惊和疑惑。

“这天鼓雷音劫,每一次雷声轰鸣,就会产生无形罡雷,直接作用在蛊仙体内,炸毁渡劫蛊仙的五脏六腑,威力可炸山裂石,偏偏又是音道、雷道兼容,寻常蛊仙难以防备,遇到这种劫难九死一生。不过,方源你不用怕,我们这片天地都被炼道道痕笼罩,本身就是绝佳的防御。况且我也有克制手段,只要我这天婆梭罗还在,你就安然无恙!”

急切间,琅琊地灵向方源传音,他唯恐方源顾虑灾劫而分心,导致炼蛊失败。

虽然琅琊地灵想打荡魂山的主意,方源要是炼蛊失败,正趁了他的这个想法。但是琅琊地灵本身就是依约践诺,这一次炼蛊是当初盗天魔尊和长毛老祖的约定,所以他必须竭尽全力帮助方源,不会有丝毫的其他算计。

方源悬浮于空中,对外边的轰鸣雷声充耳不闻。

他面色平静,心中也是一片冰雪般的冷静。

琅琊地灵的担心是多余的,方源经历的阵仗无数,五域乱战、义天山大战这样的场面都经历过,对于这些干扰几乎可以免疫。

他毫不慌乱,十指缭绕,如繁花盛开。稳步推进炼蛊进程。

轰轰轰。

每一声雷响,银色巨人天婆梭罗的身躯就微微震动一下。

组成战阵的毛民蛊仙。无不感到心烦气躁,浑身长毛都炸立起来。感到好似末日来临。

琅琊地灵冷哼一声:“正好我有一招,正克音道。看我的仙道杀招天地寂静!”

银色巨人骤然绽射出无边的光辉。

银白色的光芒十分刺眼,仿佛太阳碎片。

方源及时的闭上双眼,身上数道防御杀招,早已催起。

银光骤然出现,又旋即消散无踪。

天地寂静!

再无一丝雷音传来。

天空中,阴云密布,云中无数电光不断的闪烁。

但每一次电光伴随的雷声,都彻底消失。

好半天。雷声才渐渐响起,由小渐大。

“就算你这天鼓雷音劫的规模,再扩大十多倍,也不能干扰此次炼蛊。”琅琊地灵刚开口,语气得意,天空中阴云就迅速发生了变化。

无数蓝色光团,从阴云中电射而出,每一个光团竟都是魅蓝电影!

这是雷电精灵,战力可媲美六转蛊仙。

当初狐仙渡劫。就死在一头魅蓝电影之下。现在方源面临的魅蓝电影,竟然多达二十头!

琅琊地灵的眼珠子瞪圆,大喊:“怎么回事?为什么还有第二道魅蓝电影劫?而且规模还直接暴涨了二十倍!”

饶是琅琊地灵经历了悠久岁月,见多识广。此时也不禁动容。

方源心头猛地一跳,眼前的情景和他不久前渡劫时,是何等的相似。

“我渡第一次地灾。灾劫就恐怖到超越天劫。这次炼制变形仙蛊,也是一样。魅蓝电影竟一下子出现二十头。若是我渡地灾时遭遇,根本就是十死无生!”

方源的第一次地灾。天地二气形成数头铁冠鹰,还有两三头的上古墟蝠,更碰到了不少上古雪怪、荒兽雪怪等等。

论规模,自然他第一次地灾更强大,方源勉强渡过去。

但若是他碰到的地灾,是眼前的二十头魅蓝电影,基本上就没有什么活路可言了。

他渡地灾的时候,上古墟蝠等等形成不久,方源果断,没有犹豫,立即爆发全力攻击,及时地把这些地灾消灭掉。方源攻强守弱,此举是扬长避短。

但若是对战二十头魅蓝电影,则完全不同了。

魅蓝电影速度奇快,战力媲美六转蛊仙,方源就算使出剑浪三叠,也恐怕打不中它们。

方源仙蛊不乏七转、八转,但都不是防御仙蛊。方源攻势极其强大,但防护手段却相差很大,十分被魅蓝电影克制。

“说到底,还是我发展太过迅猛,根基不稳,短板极多。”方源心中闪过这个念头。

其实,这位问题他早就发现了。

但奈何义天山大战以来,时间太紧,他根本毫无闲暇,去弥补这些东西。

二十头魅蓝电影,齐齐向方源电射而来。速度之快,堪比电光,只在方源的眼膜中留下数道蓝痕,就已经扑到了近前。

琅琊地灵急得大吼一声,使出某个仙道杀招。

银色巨人的全身光辉绽放,纯白的光芒中,银色巨人体型不断膨胀,眨眼间就涨大了十多倍。

本来银色巨人的体型就很庞大,这一番变化,直接化成了脚踩云盖大陆,头顶仙窍穹顶的天柱!

银光巨人山一般雄伟的身躯,猛地一震,再度催发一道仙级杀招。

魅蓝电影本已入侵到方源左右,忽然眼前出现磅礴光墙,狠狠撞上之后,远远的反弹出去。

成功解围之后,银光巨人又伸出双手,十指合拢,窝一个圆球,将方源包裹在掌心之中。

它的双手空间很大,竟涵盖了方源周围数里空间。

“情形有些不妙,我这杀招虽然能防住魅蓝电影,但对你炼蛊也有妨碍。接下来你要注意,很可能会有强烈的波动侵袭到你那边来,你若稳不住,此次炼蛊必遭惨败!”琅琊地灵再度传音。

方源点点头,没有说话,满脸肃穆之色。

他此时炼蛊,周围有无数炼道道痕相帮,高空中还有炼炉残屋辅助。但银光巨人和魅蓝电影交战。会干扰道痕,引发无数震荡。这对方源而言。极为不利!稍有差错,就会前功尽弃。

轰轰轰!

一声声的轰鸣声。传到方源的耳中。

魅蓝电影像是蓝色蜂群,萦绕在银光巨人周围,不断向巨人的双手撞击过去。

它们速度奇快,银光巨人却显得笨拙不堪。

“你们这群蠢货!”琅琊地灵大骂。

他要维护方源,时刻保持双手稳定,无法分心。所以对抗魅蓝电影的事情,就只能交给其他毛民蛊仙。

但这些毛民蛊仙战力十分低下,操纵巨人很不得力。虽然都尽力出手,但催发的仙道杀招很多居然直接失败。就算成功的杀招,反映在外面,也是声势浩大,战果微小。

打了半天,魅蓝电影只消散了一头。这还是瞎猫碰到死耗子的运气。

琅琊地灵连叫骂的心气劲都没了。

“这一次,恐怕真的不太妙了。”他满眼的担忧之色。

银光巨人乃是天婆梭罗所化,此时膨胀了十多倍,防御不如之前。又只能被动挨打,被魅蓝电影狂轰滥炸。已经越发淡薄,隐见透明。

而方源虽然稳住了局面,但因为外界干扰,速度大降。

琅琊地灵按照他这个速度估算。银光巨人根本支撑不到那一刻。

他现在已经分身乏术,而方源全神贯注在炼蛊方面,也无法抽身作战。

没有外力介入的话。可以说,败局已定!

“好在这是第一次尝试。就算失败也无妨,我还有资本。”方源心态倒是很好。态度也摆的十分端正。

炼制仙蛊,谁敢奢望第一次就能成功的?

方源可以承受失败,但另一边的中洲地渊中,影无邪却已经走到了穷途末路。

“最后一次,这是最后一份仙材,再不能失败了!”影无邪满头大汗,狠狠咬牙。

他就像是一头被逼到悬崖的孤狼。

一旦失败,他再想筹集仙材,就难上加难了。

“此时此景,唯有动用底牌……就算有着后遗症也管不了了,先渡过眼前这个难关再说其他!”

影无邪下定决心,催动某个仙道杀招。

无边的剧烈痛楚,袭进他的魂魄深处。

在仙道杀招的作用下,他的整个魂魄都开始化为燃料,迅速消耗。与此同时,他的气运开始节节暴涨!

“本体曾经探索了八十八角真阳楼,创出这一杀招……幸亏前不久,我得了七转魂啸蛊,稍微改变一些凡蛊,就能运用。”

影无邪眼中闪过一丝决绝神色。

他孤注一掷,要以庞大的气运,来护持自己炼成定仙游。

“这一次一定能成!!”他在心中呐喊。

琅琊福地中,已经是岌岌可危。

银光巨人已经虚弱不堪,濒临崩溃。

巨人体内的毛民蛊仙们,更是遭受创伤,有的心脉震断,喷洒鲜血,有的已经昏倒过去,有的剧烈颤抖,口吐白沫。

他们组成了上古战阵,上古战阵受创,自然要反映到他们的身上。

魅蓝电影还剩下十一头。

不管是方源,还是琅琊地灵都已经基本绝望。

“第二次炼蛊的话,我坚决不会采用毛民天地流的炼法了!”方源心中发誓,他现在觉得人族隔绝流的炼法,是多么的可亲。

“这次居然要失败,砸了我琅琊福地的招牌,都怪这灾劫规模莫名其妙地暴涨这么多!”琅琊地灵也很气愤。

“好了,差不多可以收手了。”蛊仙毛六心中暗道。

他就是影宗安插在琅琊福地中的内应,原先有三人,如今只剩下他一位了。

就是他和影无邪暗中联系,将福地、方源的情报透露给后者。

这一次对抗魅蓝电影,他也没少出力,当然都是破坏捣鬼。站在影宗的立场上,他自然不愿意看到方源炼成变形仙蛊的。

“最后一击,就用杀招壮士断臂!”毛六心思流转,差点笑出声来。

这一招使出来,虽然威力强大,但却要大大的削弱银光巨人本身。如今银光巨人濒临崩溃,壮士断臂一用,必定直接崩溃。

“我若用了这招,上古战阵崩溃,不仅让方源炼蛊直接失败,而且战阵反噬,还能坏掉不少毛民蛊仙的性命。我幸存下来,更能被委以重任。真是妙哉,妙哉!”

眼中闪过一丝冷酷的光,内奸毛六果断催动壮士断臂!

银光巨人的一个胳膊,顿时脱离身体,落下的同时,绽放出冲天光辉。

“不好!是哪个笨蛋……”琅琊地灵大惊失色,刚要怒骂。

忽然!

这断去的胳膊,又被另一个仙道杀招打中,爆发的力量竟被悉数夹裹,冲向远方。

轰!

一声巨响,惊天动地的爆炸,好巧不巧殃及到周围的魅蓝电影。

这些魅蓝电影四处乱飞,偏巧那一刻,绝大多数都聚集在一块,被这场大爆炸炸个正着,一下子竟都灭了。

惊人的战果!

方源看得一呆。

毛六直接傻眼:“这,这也行?!”

琅琊地灵也傻眼,好半晌忽然大叫:“打得好啊!!”

经此异变,银光巨人崩解,毛民蛊仙们尽皆受创。但魅蓝电影只剩下三头,战局骤然天翻地转。

但靠琅琊地灵本人,就足以抵挡这些魅蓝电影了。

毛六后悔不迭,还想出手,但被琅琊地灵下令,叫他照顾其他毛民蛊仙。

半个时辰之后,方源捧着炼成的,还有些热乎的变形仙蛊,自己都有点不敢相信。

“第一次炼制,这就成了?”

中洲地渊。

噗!

影无邪鲜血喷吐,仰头而倒。

他面色苍白如纸,一下子昏倒过去。

最后一次炼制定仙游,他再次失败!(未完待续。)

\end{this_body}

