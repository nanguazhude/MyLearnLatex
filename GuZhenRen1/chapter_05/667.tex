\newsection{兽劫洞天}    %第六百六十八节:兽劫洞天

\begin{this_body}

中洲,地沟。

嗷吼……

一头荒兽巨猴发出最后一声嘶吼,它还想要站起来,但面对眼前身体渺小的人族蛊仙,它感觉到浑身的冰冷。

这种冷洞彻心扉,令它沸腾的热血也冰冻起来。

仙道杀招――血渐冷!

出手的不是别人,正是方源的弟弟古月方正。

“恭喜方正仙友,经过这些天的战斗,你的冷血光杀招已经是如臂使指了。”赵怜云缓缓飞近,面带微笑。

古月方正望着冰冷下去的巨猴尸体,眼中赤红的光徐徐消散,他面色仍旧一片平静,不骄不躁地看了赵怜云一眼,流露出感激之情道:“谬赞了。这些天还多亏了怜云仙子你在一旁护持,否则我怎么会有这么多的良机来练习我的杀招呢。”

中洲地脉频动,意外频发,任何一个超级势力都陷入人手不够的窘境。仙鹤门为了镇压自家局面,不得不派遣方正出马。

当然,方正的举动时刻受到天庭方面的监控,只要不出意外,天庭也乐于见到方正实力提升上去。

方正被凤九歌救走之后,回到中洲,受到天庭方面的支持。不仅得到血仇仙蛊,还在升仙的过程中获得冷血仙蛊。

紫薇仙子早已经安排妥当,血渐冷等等仙道杀招都交到方正的手中。如今,他已经将血渐冷杀招掌握,并且能够熟练地用于战斗之中,进步是日新月异。

东海,某处海底。

龙公和凤金煌两人炼化梦境,已经有数天时间了。

围绕龙宫的梦境远比想象中要多得多,龙公这些天操纵纯梦求真变杀招,陆续转化了数十位纯梦求真体,但是仍旧有新的梦境从龙宫之中蔓延出来,补充外界的损失。

两相比较,纯梦求真变只是六转杀招,威能有限,因为它的核心仙蛊梦翼也只是六转而已。

其实凝练出了第一个纯梦求真体后,龙公就可以借助这具奇特肉身,自由地穿梭梦境,去往龙宫深处了。

但是龙公此行的目的,就是为了这座八转仙蛊屋。不把这些梦境排除,他也无法安然无恙地将这座仙蛊屋收取走。

龙公凝练纯梦求真体的时候,凤金煌一直在刻苦修行。

她主要研究的就是纯梦求真体,有时候遁出魂魄,投到纯梦求真体的肉身上去,有时候则魂归本体,针对纯梦求真体加以研究。

“影宗研究出来的纯梦求真体,因为太过追求稳定,反而不能长存。”

“梦境变幻无常,特点就是变动二字。所以我索性走上碎梦的道路,令它更不稳定。结果在紫薇大人的推算下,反而负负得正,得到了完美的纯梦求真体。”

凤金煌每次回想,都觉得此事侥幸。

“难道这就是宿命的作用吗?它是要借助我的手,来令完整的纯梦求真体现世?”

凤金煌旋即摇摇头,将这个念头从脑海中扫除出去。

她又回想起师父龙公对她关照的话――

“煌儿啊,你虽然研究出了完美的纯梦求真体,但是还不够。你要将这份成果研究透彻,才能用于你的修行。当你找到让你本体资质自然晋升成纯梦求真体的时候,就是你渡劫升仙之日了。”

蛊仙!

凤金煌的积累,早已经足够升仙,但一直都被龙公压制着。

现在龙公终于松口,凤金煌兴奋之余,也很为难,她该如何将纯梦求真体真正吃透?

这个问题,凤金煌自己在想,龙公也在默默期待。

“我是不是有点勉强她?”

“毕竟,要将本体变化成纯梦求真体,就要涉及到人道的奥秘!”

“《人祖传》中就明确记载,十绝体都是人祖所创,这些都是人道奥秘。我当初阅读《人祖传》的成果,兴许可以传授给她了。”

龙公又号称龙人之祖,当初全新异人种族――龙人的诞生,就是龙公一手促成。

而龙人这个成果,就是龙公参悟《人祖传》而得。

“虽然收了凤金煌为徒,但我还真没有教导她什么。她恐怕是不能完成我的要求的,不过这份龙人奥秘,交给她也好。当然,届时还得告诫她,万万不能制造龙人!”

“不,依照她的性情,恐怕会偷偷制造得吧。煌儿和洪亭大不相同,尤其是她这个性情,还有对宿命的想法……唉,人心不古。这种危险的思想,应该和白晴无关,恐怕更多是受到了凤九歌的影响。”

龙公当然也调查过凤九歌和白晴仙子二人。

阿嚏!

凤九歌摸摸鼻子,满身血污,不满地嘟囔着:“谁在念叨我呢?”

他狼狈不堪,一身衣袍已经碎裂大半,露出他矫健硬实的胸膛。

咳咳!

他忽然咳嗽,当即吐出了几口鲜血。

潜伏在一旁的两位蛊仙,此时现身,看到凤九歌重伤模样,却是各自都带着喜色,道:“恭喜凤九歌大人,成就八转蛊仙!”

“嗯。”凤九歌神情淡然地点点头,他心中对自己的八转修为,并无多少振奋之意。事实上,他早已可以晋升八转,但一直是没有合适的机会。

总有各自束缚和牵绊。毕竟,他是有家室的人,同时也是灵缘斋最重要的成员。

“八转修为并无什么大不了,虽然有了白荔仙元,但是积累起来却需要一段时间。”

“我可没有多少财力,去将自己的核心仙蛊升炼成八转。”

“就算有财力,也得要其他人出手帮助我炼蛊。”

“或许这一步,我可以依赖天庭?”

凤九歌想到自己护道人的身份,这里面大有利用的空间。

“不过最关键的,还是渡劫时联想到《人祖传》,从而得到的灵感。或许,我可以从中创出一首命运之歌呢。”

“天地间有宿命,原本一切都有安排。但运道诞生后,就有了变化的余地。世间万物万事,就变得精彩纷呈。”

“命运之歌就是描绘的这种精彩,讲述天地、生命的变与不变!”

时间流逝,又过了一段时日。

太古白天。

方源望着飞旋绕圈的天相灵鹤,脸上神情振奋。

“找到了,我终于发现了第二个仙窍洞天。”

“这是……兽劫洞天?”

“好洞天,居然没有吞并任何的九天碎片啊!”

------------

\end{this_body}

