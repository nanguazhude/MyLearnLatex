\newsection{四处袭击}    %第六百八十三节:四处袭击

\begin{this_body}

归一派,作为炼蛊大会的报名地点之一,这里同样是人山人海。

“炼蛊大会终于开始了!没想到我今生还能得到第二次机会,实在是感谢上苍。”中年蛊师叹息着道。

在他周围的一些蛊师认出了他,顿时议论纷纷。

“快看,竟然是炼道大师罗生!”

“他在上一届的炼蛊大会中,因为意外情况而遗憾落败,这一次他是要卷土重来啊。”

“本来炼蛊大会每百年举办一次,罗生应当是没有机会的。谁能想到时隔十年,十大古派竟然又举办了!”

中年蛊师将这些议论都听在耳中,面色一片平静。

他早已过了虚荣的年纪,现在的他一门心思想要证明自己。

“这一次的对手,同样不容小觑啊。”

“我虽然励精图治,苦心积累,但是中洲太大了,五域太广了,总有人才辈出。”

“单单现在这里,就有不少新人天才……”

中年蛊师目光扫视,将主要注意力集中在三位蛊师身上。一位老者,成名多年,同样是炼道大师。另外的则是一位青年男子,还有一位明眸少女,前者在上一届的炼蛊大会中大放异彩,而后者则是近十年来崛起的闪亮天才。

“这一次炼蛊大会,竞争仍旧会那么激烈啊!”

“我一定要好好把握住这次良机,再不能错失了……”

“我要证明我自己,让妻子和儿子都抬起头来做人!”

轰!

下一刻,一头巨大的上古年兽从天而降,将报名的整个大殿都压成渣滓。

中年蛊师的一切宏图野望都烟消云散,化为乌有。

那些大好前途的年轻蛊师,受人瞩目的天才少女,亦都惨死当场,被压成肉泥血骨,混在一起,连尸体都辨认不清。

“下一个地点。”高空中,方源的身影一闪即逝。

这已经是他袭击的第十五个地点了。

中洲蛊师损失惨重,这些人都只是凡人蛊师,哪里能对付得了上古年兽?

就算是六转蛊仙,也未必能匹敌这些上古年兽。

方源无情地屠戮这些蛊师的性命,造成一场场巨大的惨案,牺牲的蛊师精英数量越来越多,让天庭众仙都感到一丝心疼。

“方源这个魔头,真是穷凶极恶,他果然来捣乱了!”

“可恶,难道我们就这样坐视吗?”

“紫薇大人,不知您……”

紫薇仙子微微摇头:“方源小心谨慎,每一次出击,都只释放野生的上古年兽出来,祸乱世间,屠戮蛊师,极少自己亲自出手。”

紫薇仙子深深叹息:“他的智道造诣非常深厚,很明白自己出手越多,留下的线索就越多,就越容易被我们发现和追踪。我要确定他的位置,十分困难,需要更多的线索,时间会很长。”

“但他哪里来那么多的上古年兽?”有蛊仙皱眉。

“根据南疆那边的情报,夏槎的仙窍中建设了一座年华池。应该是依靠此池,方源才暗中收拢了这么的年兽。”紫薇仙子道。

她面色虽然平静,但心中却有着怒意。

方源身为蛊仙,居然如此肆无忌惮地向凡人出手,以大欺小,倚强凌弱,真的是太过无耻,没有一点点的底线!

绝大多数的魔道蛊仙,都不会这么干。

这是蛊仙之间的默契,但方源直接打破了。

“方源这个家伙是万物生灵的大害啊!”紫薇仙子心中愤慨,又为那些无辜丧命的生灵而悲悯,“他活在这个世上越久,就会有越多的生灵遭受他的毒手和迫害。”

龙公却是真的平静,心中宛若古井无波。

“好了,方源出现,不是我们已经预料之中的事情吗?按照计划行事吧。这一次就算付出再大的代价,也要修复宿命蛊!”龙公语气坚定无比。

天庭蛊仙凛然听命,他们均感到:世间任何事情,都动摇不了龙公此刻的决心。

方源的确是一个麻烦,天庭方面不可能不估算他,因此早有应对的计划。

数天后,遭受方源四处袭击而震恐中的人们,得到了确切的情报和证据。

一切都指向方源,他就是罪魁祸首!

群情激奋。

“方源竟然已经成为蛊仙了?!”

“这世上真的有仙啊……”

“方源,你杀了我的父亲,此番血海深仇我一定会报的!”

“纵然五转,不成蛊仙,终为蝼蚁啊。”

“方源明显是在破坏炼蛊大会,为了小命着想,我还是不去了吧。”

“去!为什么不去?我们若是不去,不就正落入方源的阴谋算计中了吗?”

“是啊,我虽然只是一个凡人蛊师,根本不是蛊仙方源的对手。但是为了报杀妻之仇,我豁出性命,也不能让方源这魔头如愿!”

有人恐惧,有人感慨,有人退缩了,也有人反而坚定了信心。

时间流逝,一天天过去,炼蛊大会在持续着。

方源只身一人,中洲却是密布炼蛊大会的报名地点。他虽然有定仙游这种手段,但是也不能尽数铲除这些蛊师。

因为有好多地方,都被中洲的蛊仙暗中布守。

尤其是在方源袭击之后,天庭方面立即做出举措,分派任务,令无数蛊师领取相关任务,四处走动,一一考核他人的参与资格,并发放令牌。

这样一来,中洲方面化整为零,再没有人集中去一个地点报名,让方源屠戮的效率大减。

方源虽然在努力,死在他手上的蛊师已经上万,但却无法阻止炼蛊大会的开展。

报名阶段过去,正式比试已经开始多日。

“嗯?”方源忽然出现在中洲某地,正要出手,忽然眉头一挑。

“方源哪里走?”

“给我纳命来!”

一位中洲八转蛊仙,向方源扑来。

方源看了对方一眼,没有丝毫开战交手的欲望,再次催动杀招成功撤离。

“最近几天,中洲蛊仙支援的速度越来越快了。是因为我的定仙游杀招使用太多次,被推算出来了吗?”

“但即便如此,他们似乎能够提前预料到我会出现在哪里。这是什么侦查手段?宙道?”

下一刻,方源周围景色骤变,距离刚刚的地方相差数十万里。

方源却是脸色陡沉,他敏锐地感知到三位八转蛊仙,正从三个方向,向他围剿过来。

“方源,你还想逃去哪里?”

“天庭为你准备了十几道侦查杀招,你已经逃不了了!”

“尤其是其中一道,乃是元始仙尊所创,名为千夫所指!哈哈哈,魔头你没有想到吧,你杀的人越多,就越有人恨你。这些凡人蛊师的恨意,将不断地标注你的位置。”

三位八转齐齐杀到方源的面前。

仙道战场杀招——流言笼!

忽然,第四位八转蛊仙出现在方源的头顶上方,一记仙道战场将方源包裹进去。

“捉住方源了!”

“周雄信大人的流言笼,不愧是催发速度最快的战场杀招,即便是方源都没有来得及反应啊。”

四位蛊仙大喜过望。

“龙公大人,方源终是被我们困住了。”天庭中,紫薇仙子神情振奋。

“嗯。”龙公淡淡地应和一声。

紫薇仙子双眼发亮:“龙公大人,机会难得,我建议您亲自动手,确保万无一失,将方源铲除!”

龙公却微微摇头:“方源不过是跳梁小丑,修复宿命蛊才是眼前大局。还不到我出手的时候。”

“龙公大人……”紫薇仙子正要再劝,忽然面色骤变。

在这一瞬间,中洲各地,尤其是炼道大比的场次,受到蛊仙级的猛烈袭击。

甚至这一次,不仅是炼道场所,中洲一些著名的资源点,都遭受了蛊仙的突袭和掠夺。

“就连不败福地,都遭受袭击了!”紫薇仙子面沉如水,她难以置信地道,“居然会有这么多的蛊仙对付我们?”

\end{this_body}

