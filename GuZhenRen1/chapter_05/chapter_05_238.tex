\newsection{方源一打五(下)}    %第二百三十九节:方源一打五(下)

\begin{this_body}



%1
红枣仙元一下子消耗了二十多颗。

%2
下一刻,血色火焰彻底消失,方源重回之前的状态。

%3
“唉!”

%4
“人如故只对人体有用,我变化成其他生命形态,就不能直接运用人如故了。这很不方便。”

%5
“而且人如故的局限也很大,幸好我有三息后现这一侦查杀招!”

%6
方源心中感慨,索性他又催动另外一个仙道杀招。

%7
奴力合流,万我之术。

%8
呼啦啦。

%9
大量的方源,成千上万,汇集成茫茫人流,向着影无邪等人冲去。

%10
“怎么可能?他居然这么快,就缓过来了?难道他已经消除了血肉盛炎?”黑楼兰惊诧无比。

%11
方源的反击来得太快,仍旧那般强势,让黑楼兰一颗心沉入谷底。

%12
黑楼兰知道方源手中,拥有着人如故仙蛊。

%13
但人如故仙蛊对于时间的要求,比较苛刻。即便是方源,也很难幸免于难。因为他在中招之后,首先要发现自己的状态,其次还要撤销变化形态,这期间或许还会有些犹豫不决,最后他才会催动起人如故。催动这只仙蛊本身,也是要耗费时间的。

%14
如此一折腾,黑楼兰设身处地去想,换做她自己,也很难消除得掉这个血肉盛炎。

%15
方源不清楚黑楼兰、白凝冰的底细,黑楼兰同样不明白,方源手中掌握着三息后现。

%16
这个侦查杀招,能让方源提前三个呼吸洞察到自身的不妥。正是因为如此,方源才有了更多的时间去反应和处理。

%17
总而言之。黑楼兰付出重伤的代价,却给方源带来很小的负面影响。

%18
红枣仙元的确消耗了不少。不过方源家大业大,经济良好。这点钱财却是不放在心上。

%19
仙道杀招魂啸。

%20
影无邪张口咆哮,身边无数道方源虚影,都砰砰爆碎。

%21
仙道杀招刺林地。

%22
石奴右脚跺地,地面上升腾起无数的地刺,将数以千百的方源虚影彻底刺穿。

%23
然后,石奴轻喝一声,地刺刷刷暴射升空,天上的大片万我,也被瞬间清空。

%24
这一招效果极佳。蕴藏第二式变化,让方源的万我大军几乎分崩瓦解。

%25
土道七转蛊仙石奴,此战中他堪称中流砥柱,没有他的防守贡献,影无邪等人怎可能支撑到这一刻?

%26
“下一个就是你了。”方源重现变化成剑蛟,心中狞笑。

%27
吼。

%28
这个时候,一旁的虎形年兽低吼一声,忽然撤退。

%29
这头上古年兽被方源召唤过来,虽然一直在战斗。不过被影无邪一方稳稳防备,攻效不佳,反而自己受伤不少。

%30
虎形年兽并不忠诚于方源,这点比不上影无邪的魂兽召来杀招。

%31
受伤严重到了一定的程度。它立即拔腿就走,洞穿空间,重新回到光阴长河中去了。

%32
但是没有关系。

%33
仙道杀招年兽召来!

%34
方源再度催动这个杀招。这一次唤出了一只羊形上古年兽。

%35
虽然年兽召来每一次,只能召唤一头年兽。但光阴长河当中年兽数量并不少,这只跑了。再召唤另一头就是了。

%36
羊形上古年兽冲向影无邪等人,石奴低吼一声,忽然身躯膨胀,化作一个巨人,体型不下于年兽,站稳脚跟,主动反扑,运用巧妙的格斗技巧,将年兽一把摔倒在地。

%37
轰隆一声。

%38
上古羊形年兽摔在地上,引得周围地表都狠狠颤抖了一下。

%39
羊形年兽很快就爬起来,带着愤怒,羊角上顶,刺向石奴。

%40
石奴闪躲的同时,顺势两手抓住弯曲的羊角。

%41
两个巨型生物,相互纠缠在一起,一时间难分胜负。

%42
方源这边,剑光龙息的频率和数量都下降了许多。

%43
他感到喉咙一阵阵的灼痛,嗓子都快要冒烟了!

%44
剑光龙息,乃是上古剑蛟的最强手段。真正的上古剑蛟,连续喷吐,也不过二三十次而已。

%45
但方源却是喷出了成百上千次,终于到达了肉身能够承受的极限。

%46
在喷吐下去的话,嗓子都会喷坏掉,剑光龙息的威力也将大打折扣。

%47
战斗到现在,上一个虎形年兽已经伤重撤退,方源则多少露出了一些疲态。

%48
他虽然已经成为七转蛊仙,有了红枣仙元,战力暴涨,达到了七转中的上等层次。但是他面对的敌手,也都非同小可。

%49
黑楼兰乃是女中枭雄,大力真武体,又在玉壶山吞并了一个力道福地,道痕增长许多。同时,她还继承了焚天魔女的真传,还有一身仙蛊,战力出类拔萃。

%50
白凝冰则是北冥冰魄体,又转化为了神秘的龙人,掌握白相洞天,继承了白相真传。

%51
影无邪更不用说了,本身就是魔尊幽魂的主要分魂之一。曾经是第十一绝体,影宗最后的希望,魂道造诣惊人至极,各个流派的境界具体不明,但应当不低。至少运道方面是这样的。

%52
石奴本身是异人种族中的石人,这个种族天生就有丰富的土道道痕,再搭配石奴主修的土道,可谓相得益彰。

%53
最不济的算是太白云生,即便如此,这位老家伙也是老而弥坚,主修宙道,完美兼修云道,手中掌握着仙蛊,云环杀招的防护威能也非常可观。

%54
这五个人虽然大部分都是六转修为,但和之前六转时期的方源差不多,战力超凡脱俗,都有七转层次。

%55
此刻联起手来,一个个都是精英,心性俱佳,相互配合,又有影宗残余的资源在不断支撑他们,更关键是手段层出不穷,绝对比单个的耶律群星还要强大,还要难缠。

%56
蛊仙之战,切不可急躁冒进。

%57
因为仙蛊种类繁多,仙道杀招更是层出不穷,有点奇妙无比,有的天马行空,有的霸道蛮横,有的阴险诡秘。

%58
方源以一敌五,牢牢占据上风,影无邪等人疲于应付,始终都没有翻过身。

%59
“可恶,到现在他都没有用出那一招!”影无邪想着,从鼻腔出缓缓流下惨绿色的仙僵之血。

%60
影无邪用手背擦拭掉,但很快,仙僵绿血又流淌下来。

%61
“该死!”

%62
“我屡次动用仙道杀招,在这界壁中承受反噬,就连仙僵之躯都有些承受不住了。”

%63
“我尚且如此,其余人可想而知,只有更糟。”

%64
影无邪望向周围同伴,发现他们无不脸色惨白,状态很是糟糕。

%65
方源在这里不受限制,但他们不行。他们不仅要时刻稳住身形,抵御界壁中的排斥和吸引之力,而且还要和方源对战,战斗中频频使用仙道杀招,仙窍的损失必然更加惨不忍睹!

%66
“我方有影宗残余的资源,战力提升很正常。”

%67
“方源的战力竟然也提升得这么快!”

%68
“这种提升的速度,简直是骇人听闻。我们五个人联手,都不是他的对手。”

%69
影无邪越想,心中对方源越是忌惮。这种忌惮的深处,甚至还产生了一丝畏惧!

%70
是的,就是畏惧。

%71
影无邪开始有些怕了。

%72
他不得不承认方源的强大,更心悸于方源的成长速度!

%73
“白相,开。”

%74
一直酝酿杀招的白凝冰,忽然低呼出声,身上陡然升腾起冲天的气势。

%75
影无邪顿时喜色满面,他望向白凝冰:“你终于成功了!”

%76
白凝冰此刻已经相貌大变。

%77
她彻底转变了模样,她在眨眼间,变成过来一个高大一丈五六的小巨人。

%78
她身着冰甲,三头六臂,赤着双脚,身边霜气四溢,宛若仙衣绶带环绕,脚下踩着两朵寒云,透着淡淡的蓝意。

%79
仙道杀招白相!

%80
“方源,来战!”白凝冰轻喝一声,速度猛地爆发出来,好像是一道白虹划破长空,她竟然一改之前守势,转为主动进攻。

%81
“新仇旧恨,就在今天了结。”方源龙睛微微一缩,不闪不避,直接冲上去。

%82
一道银光,一道白光,双方在半空中交错而过。

%83
速度的惯性太过强大,不管是白凝冰还是方源,都各自飞出去老远。

%84
“受了点轻伤。”方源龙头微微低俯,看见龙身上一道长长的剑痕,痕迹很浅,但所到之处的龙鳞都被切碎了,有些地方刺破了龙皮,渗出一些龙血。

%85
反观白凝冰则比方源凄惨得多。

%86
她的一个脑洞,三个手臂都被毁掉,几乎半边身子都没有了。

%87
不过她没有流淌出任何的血液,寒气肆溢,眨眼间就恢复如初。

%88
“白相。这个仙道杀招,传闻中,号称是不死不灭。哪怕只剩下一个指甲盖大小的残躯,也能转瞬重生,恢复如初。盖因形成白相之后,已经不是血肉之躯,而是某种奇妙的生命状态!”影无邪心中十分激动。

%89
果然,接下来的战斗中,白凝冰的表现如同影无邪所估料的那般。

%90
虽然,她每一次比拼,都不是方源的对手。但是凭借白相的特性,她转瞬之间就能恢复如初。

%91
白凝冰和方源交战,吸引了方源大半的火力。

%92
战斗持续着,方源都感到了疲累,毕竟是血肉之躯,但白凝冰却仍旧生龙活虎。

%93
就在方源打算放弃白凝冰,专门对付影无邪等人的时候……

%94
“时机到了!”影无邪忽然轻喝一声,身上爆发出无数蛊虫的气息。

%95
他看向方源,口中低呼

%96
引魂入梦!

\end{this_body}

