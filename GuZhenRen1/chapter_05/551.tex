\newsection{落魄印?}    %第五百五十三节:落魄印?

\begin{this_body}

琅琊福地,云盖大陆。

仙道杀招万我大手印!

方源漂浮在半空之中,浑身气息陡然爆发开来,轰隆一声,一道巨大的手掌,凭空产生,势大力沉地向地面拍去。

刹那间,气浪翻滚,巨掌如山峦倾倒,狠狠地拍打在云泥陆地上。

云盖大陆乃是琅琊地灵亲自主持设计,全都是云泥浇筑构造,能悬浮在高空永不坠落的神奇陆地。

这种云泥却是不够坚厚,方源的万我大手印却是有着七转层次的攻伐手段,这一掌下去能立即将方圆百里的云泥都震得碎落,掉入下面的波澜汪洋之中去。

方源此次试演杀招,当然不是为了拆除云盖大陆,所以在巨掌拍中云盖大陆的时候,笼罩整个琅琊福地的超级仙阵,猛然一动,在云盖大陆上覆盖了一层流水般的银芒。

万我大手印拍在云泥上,银芒骤起又旋即被拍碎,一声闷响之后,云泥四溅,方圆数里地貌剧变,大手印之下,轰出一个手印深坑,深达一丈左右。

方源心念一动,大手印瞬间崩解,消失不见。

对着效果,他满意地点点头。

“有了万我仙蛊,大手印杀招施展出来,简直方便太多了!”

以前,方源要施展出万我大手印,首先第一步,就要成功催动万我杀招,在仙窍中储藏成千上万的力道虚影。然后,在这些力道虚影的基础上,在使用群力蛊等等,才能用出力道大手印。

整个过程非常繁琐,如果不事先在仙窍内储备大量的力道虚影,基本上在一场生死激斗中,就很难用出力道大手印了。然而力道虚影这种东西,却是万万不能持久储藏的,有着时限。时限一到,就会自然崩散。

有准备的战斗,还好说,毕竟可以事先准备。但是没有准备的战斗,遭遇突袭等等,万我大手印就相当不方便了。

好在方源智谋出色,又谨慎小心,相当注意这个方面,倒是没有在此处遭遇尴尬。

“以前,单单用万我杀招,就要涉及我力仙蛊,苦力蛊、借力蛊、自力更生蛊、炼精化神蛊、功倍蛊、地力蛊、火水风电力蛊等无数凡蛊,现在却是只需催动万我仙蛊即可。”

“而现在的万我大手印,经过我的改良,只需要以万我仙蛊为核心,增添一些群力蛊等等凡蛊辅助,总体蛊数不超过三十,简化的程度,简直是一落千丈,从山顶到山脚。”

之前的万我大手印,显然是不能用的。

在万我仙蛊炼成之后,方源就又改良了万我大手印,形成全新的版本,以万我仙蛊为核心。之前的核心我力仙蛊,却是被方源抛弃。

如此一来,我力仙蛊就可以用在其他方面。

当然,也可以添加到万我大手印杀招中去,增添它的威能。

这就看方源如何进一步改进了。

“其实万我大手印,大有改良空间。比如说一直都速度不够,可以增添许多凡蛊,弥补此处弱点。又比如说,增添魂道仙蛊,使得大手印能够强行抓取魂魄。再比如说,大手印威力受限于我本身的力道道痕,其实威力可以不断上涨。现在万我大手印,只是七转层次战力,算不上巅峰,都是因为我至尊仙体的力道道痕不足以承受更强大的威能。”

对于万我大手印杀招,方源只是初步改良,替换了核心仙蛊,将来还大有潜力可挖。

以前,杀招内容太过厚重,步骤繁杂,催动时消耗的念头、时间都很多,再增添蛊虫的话,实用性就会下降很多。

但现在万我杀招浓缩成一只蛊虫,步骤简略至极,完全可以增添更多的凡蛊、仙蛊进来,形成更加强大、优秀的大手印。

方源这一次试演,并非只是力道大手印,还针对防卫琅琊的超级仙阵。

这仙阵在之前凤九歌入侵时,立过大功,让诸多异族蛊仙在生命有所保障的基础上,拼力出手,帮助琅琊派渡过危机。

凤九歌对仙阵动用多次离歌,但方源早已料到他的手段,仙阵分有多层,凤九歌拆开一层又有一层,差点要让他吐血。

这一次,方源归来,不仅是改良了万我大手印,也改善了这座仙阵,使得它的威能覆盖到琅琊派的所有角落。

刚刚力道大手印砸在云盖大陆上,就是因为这座超级仙阵,才使得破坏效果大为降低。

试演了力道大手印后,接下来才是重点。

仙道杀招逆流护身印!

这个杀招决定了方源的战力上限,也是方源力抗八转的唯一本钱。

现在,杀招的核心便是天地秘境逆流河、万我仙蛊、坚持仙蛊、挽澜仙蛊,其余辅助凡蛊上百只。虽然比万我大手印要繁杂许多,但比较之前的逆流护身印,就简略太多了!

杀招简略之后,催动的成功几率就飞跃提升,变得更加实用。

“之前我在青鬼一役,遭遇偷袭,逆流护身印都来不及催动出来。现在却是随时能够催动,酝酿的时间大为缩减,应付一般的突袭已经不在话下了。”

“只是……我现在却不能将逆流护身印再做伪装。之前欺骗天庭,布置陷阱,坑害雷鬼真君的事情,只能做这一次了。”

方源心中暗暗可惜。

逆流护身印杀招,虽然有万我仙蛊,实用性暴涨,但也已经达到方源目前改良极限。就算增添一些其他仙蛊,也是艰难。

像之前布置陷阱,然后猛地施展出全新的逆流护身印,短时间内已经没有可能了。

一番试演之后,方源就再次闭关,开始推算其他杀招。

别看方源从容淡定,其实心中也是压力重重。

面对天庭这等庞然大物的时刻威胁,谁能不有压力呢?

万我仙蛊炼制出来,意义绝非只是几个杀招的提升。方源真正的目的,是第三变招落魄印!

“我落到陆畏因的仙道杀招之中,经历三生三世的梦境。这梦境玄奇,尤其是第三世,带给我落魄印的灵感。”

“只是之前推算落魄印,却是遭遇许多阻碍,有鸿沟难以跨越。这一次炼成万我仙蛊,就像是陡然获得了一座大桥,说不得就能跨越这条鸿沟了!”

方源心中激动,暗含期盼。

虽然他借助雷鬼真君一战,将自身名望抬高一个台阶,压过凤九歌一头。但方源始终都有自知之明,知晓自己在许多方面,都不及凤九歌。

凤九歌能力战八转,是因为他本身音道道痕,浓郁至极,达到了八转程度。就算没有音道八转仙蛊傍身,使出来的仙道杀招,通过自身道痕增幅,也能够有准八转的威能。

而方源能力抗八转,主要是借助天地秘境逆流河中的海量道痕,他本身的一些流派道痕能达到七转层次,但距离八转还有很长一段距离。

就算是他将市井中的那些仙窍都吞并,也距离八转很远。

方源和凤九歌比较起来,真正差的并非才情。

凤九歌能开创出大风歌,成为史无前例的,能够催动大同风的蛊仙。方源这边则是设计出了万我杀招,奴力合流,开历史新篇章。这两者之间,层次是一样的。

现在方源凝聚杀招,炼出万我仙蛊,层次又稍稍高上一些。因为,在这个起点发展下去,说不定就能开创出一个全新的蛊修流派!比起原来单纯的杀招,要有更广阔的发展空间。

蛊修流派就是这样发展起来的,有的是通过两派融合,有的是通过单派分化……

方源和凤九歌相差的,其实是时间积累。

方源重生以来,才修行多少年?凤九歌呢,已经是数百年了。

方源连五十年都不到,形成的成就,能和凤九歌相比,已经非常不容易。和其他人相比,那就更加变态,修为之突飞猛进,骇人听闻,难以用常理评价。毕竟方源重生的机缘的确获取了无数。

其实,方源在自身道痕上的积累,也颇为重视。

只是当今,天意居心叵测,令他渡劫程度最轻,获取道痕稀少。吞并仙窍,是个好方法,但现在上极天鹰并未寻回,其他法门却需要特定的仙蛊,还需要炼制,比较麻烦。通过食道手段,利用小吃仙蛊等,形成吃力等等杀招,增长道痕,方案也不错,但效率较低。

远水解不了近渴,所以很自然地,方源就将主意打到其他的天地秘境上来。

方源拥有天地秘境不少!

荡魂山、落魄谷、逆流河、市井,四大天地秘境。

其中,逆流河已经利用,形成逆流护身印,使得方源能力抗八转存在。剩下的荡魂山,现今破碎,还在加紧修复当中。市井要用来豢养异人,并且律道方面,方源境界不堪。

最适合的,就只有落魄谷了。

方源在魂道方面,境界也普通,但胜在有幽魂真传,这是世间第一的魂道传承,无人能否认这一点。

有这样的基础,再加上之前方源在交感梦境中的三生三世磨练所得的灵感,如今又有了万我仙蛊……

“还要再算上智慧光晕,推算出落魄印,应当可以成功吧?”方源沉吟不语。

\end{this_body}

