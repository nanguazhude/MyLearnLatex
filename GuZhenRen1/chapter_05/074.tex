\newsection{充分准备}    %第七十四节:充分准备

\begin{this_body}

%1
半个时辰之后。

%2
荒兽雪怪被方源杀得,就只剩下一颗头颅。

%3
吼!

%4
就算只是一颗头颅,雪怪的凶威仍旧不减,对着方源嘶吼,喷吐出凌厉的冰风。

%5
雪怪和通常的荒兽不同,它是由冰雪组成,类似于云兽、血兽、泥怪这些。

%6
方源遥控的力道仙僵,冷哼一声,站到荒兽雪怪的头颅面前,手搭在雪怪头颅的上面。

%7
顿时,一片薄薄的冰霜,就覆盖到了力道仙僵的手背上。

%8
方源不管不顾,催起大量的我意蛊。

%9
我意汹涌澎湃,好似浪潮,照准雪怪头颅死命地灌溉进去。

%10
雪怪的头颅中,充斥着大量的天意。

%11
方源的我意进入其中,立即和天意绞杀在一起,宛若水火对撞,毫不相容。

%12
天意量少,我意却是源源不断。

%13
不一会儿,天意就节节败退,被我意冲刷殆尽。

%14
砰。

%15
最后关头,忽然雪怪的头颅发生自爆。

%16
一大蓬的冰雪,四下飞溅。

%17
方源的力道仙僵瞬间就被这冰雪埋没。

%18
不过,这临死之前的自爆,根本没有任何的杀伤力。很快,方源驾驭的力道仙僵,就从雪堆中钻了出来,毫发无损。

%19
不过脸色却是有些难看。

%20
“就算我用我意冲刷,将天意刷尽,最终的结果也只是让雪怪自爆。无法收服吗?”

%21
“不过就算能够收服,我意蛊只是冲刷了一个头颅,就耗费了上百只,成本着实不少。”

%22
我意蛊虽然只是凡蛊,但也高达五转。

%23
同时。这些我意蛊不是方源亲手炼制,而是用琅琊派贡献,请其他毛民蛊仙出手炼制的。经过这个步骤,方源的成功就又提高了一些。

%24
更关键的,还是时间成本。

%25
方源首先得将一头荒兽雪怪,折损到重伤地步,无法反抗。让它动弹不得。才能持续不断地灌输大量我意,冲刷天意。

%26
一只荒兽雪怪,时间耗费或许不太起眼。但方源的小北原中,还有那么多的雪怪呢。

%27
数量上去后,时间就多了。第二次地灾将临,不值得方源去这么做。

%28
“其实就算我奴隶了荒兽雪怪,将它们外卖出去。也是危险的。”

%29
“当初,影宗第一代的分魂之一,代号为绿的蛊仙,在北原地沟的超级蛊阵中,研究出了生死仙窍法。研究中发生大爆炸,许多仙材都被天意充斥,绿避之如蛇蝎。就这样,将这些仙材都直接抛弃了。如此分明的态度,已经说明了很多问题。”

%30
“我若是真将这些针对我的天意雪怪,卖出去。恐怕会受到更不好的连锁影响。”

%31
尝试失败之后,方源只好执行先前的计划安排,继续斩杀这些雪怪。

%32
务必要在第二次地灾来临之前,将这些雪怪斩除干净,一个都不剩。就连天意,都要彻底冲刷,一点都不留。

%33
普通雪怪。都是炮灰,方源轻而易举就能斩除,不足为虑。

%34
但是荒兽雪怪,就有些难缠,要费一番手脚。至于上古雪怪,更加麻烦。

%35
方源不禁开始怀念起飞剑仙蛊了。

%36
以飞剑仙蛊为核心,催发出来的仙道杀招剑痕索命,对付上古雪怪,效果奇佳。而剑浪三叠,则是以群攻为主,对付单个的上古雪怪,有点大材小用的意思。

%37
而且运用剑浪三叠,成本一点都不少。

%38
好在方源重新得到了力道仙蛊,使得万我杀招,能重新催发出来。

%39
力道流派向来是对真元、仙元的依赖较少,万我第一式力道大手印的催发成本,要比剑浪三叠少上许多。

%40
接下来的这段时间,方源就以力道大手印为主,重点铲除这些上古雪怪、荒兽雪怪。

%41
他首先用奴兽仙蛊,尝试奴役一头雪怪。

%42
奴役失败后,雪怪受到挑衅,陷入极度的愤怒状态,就会对方源穷追不舍。

%43
方源立即撤退,将这雪怪引到很远的地方,再动用力道大手印进行斩杀。

%44
如此一来,其他的雪怪被惊动的可能就下降了很多。

%45
以前方源三四次出手后,就会让雪怪群抱成团,现在却完全不一样,效率大大提高。

%46
奴道手段尝试失败后,就会令目标十分仇恨蛊师。这一点,被方源充分利用起来。

%47
之前方源也尝试过其他方法,但凡级的奴道手段,层次太低,一些智道的手段,在天意面前也不起效果。

%48
所以方源借来奴兽仙蛊,并非一无是处。

%49
日子一天天过去,雪怪已经不足原先的三成。距离第二次地灾,则不足十天。

%50
方源遥控力道仙僵,来到至尊仙窍中的小南疆。

%51
在某个角落,摆放着方源的仙僵肉身。

%52
“虽然有仙僵肉身,但却只能看着,不能运用。”方源叹息一声后,调动我意蛊,开始布阵。

%53
个把时辰之后,一个以数百只我意蛊为核心的蛊阵,成功建立。

%54
这个蛊阵主要起封印的作用。

%55
方源从影宗交易中获得的这个布阵之法。

%56
用大量的我意蛊封印住天意。

%57
影宗布置在五域各处的超级蛊阵,就采用的这个方法。

%58
检查三遍,确认无误之后,方源才松了一口气。

%59
方源的仙僵肉身躺在蛊阵中央,而蛊阵已经完全封印住了春秋蝉。

%60
春秋蝉原本是被天庭蛊仙威灵仰,动用无上手段隔空封印住了。使得影无邪一度无法催动春秋蝉,进行重生翻盘。因此在中洲地渊中陷入绝境,只得和方源交易。

%61
交易中,影无邪将方源的仙僵肉身,连带春秋蝉,也都还给了方源。

%62
方源得手之后,发现春秋蝉仍旧被封印着,连汲取光阴长河的河水都无法做到。一天天虚弱下去,陷入到饿死的边缘。

%63
方源曾经尝试过不少方法,但发现这份封印极其厉害,他连一丝跟脚来历都查探不出。无奈之下,方源只好暂时放弃这方面的尝试。

%64
结果不久前,八转禁道蛊仙威灵仰的手段,似乎是耗尽了。春秋蝉的封印忽然自行解除,方源知道春秋蝉中隐藏天意,非得用炼道手段才能根除。

%65
方源的这具仙僵肉身,他不敢放到外面去。保存在自家仙窍中最为安稳妥当。

%66
但方源又担心第二次地灾,春秋蝉中的天意,会和灾劫勾连在一起。所以今次,他布置下这层蛊阵,封印住了天意,提前将这个隐患消除了。

%67
又三天过去,方源的至尊仙窍中,雪怪被彻底铲除。

%68
接下来,是用我意冲刷掉一切的天意。

%69
为此,方源也付出了不菲的代价。大手印虽然成本低,但运用次数这么多,积累起来的消耗就很高了。

%70
现在方源手头上,仙元的数量并不多,仙元石也是如此。

%71
因为宝黄天关闭,使得方源空有那么多的修行资源,却卖不出去。琅琊门派贡献,也日益减少。

%72
虽然方源时常指点毛民蛊仙作战,有所补益。但琅琊地灵那边也学坏了,动了心思,将这方面的任务奖励,降下数成。同时还大大提高铲除落星犬任务的报酬。

%73
“宝黄天……宝黄天……你究竟何时才能开启?”

%74
方源一边苦苦期待宝黄天的开启,一边为第二次渡劫做更多准备。

%75
距离第二次地灾,还有五天时,方源将仙窍中的天意彻底冲刷掉,然后他来到荡魂山前。

%76
动用万我杀招等等手段,他将荡魂山搞得面目全非。

%77
然后再动用拔山仙蛊,他将荡魂山塞到自己的至尊仙窍中去。

%78
以前,方源没有江山如故,不敢这样随意搞,现在这方面的顾虑少了很多。

%79
这番动静自然吸引了琅琊地灵现身。

%80
他神色紧张地发问:“方源长老,你这是想要干什么?”

%81
他很担心,毕竟荡魂山是方源之物,不像落魄谷。方源想要取走,完全是可以的,他也无力阻止。

%82
方源巧舌如簧,说出早就思量妥善的借口,将要离开琅琊福地一段时间。

%83
琅琊地灵信以为真,他还想要方源为他铲除落星犬呢,结果方源要走不说,还带走了荡魂山。

%84
琅琊地灵十分担忧。

%85
方源趁此良机,将手中的门派贡献耗用一空,同时抵押了智慧蛊,向琅琊地灵借走了许多仙蛊,还有大量的仙元石。

%86
总算是补足了手中的青提仙元!

%87
遗憾的是,一直到方源离开琅琊福地前,宝黄天都没有开启。

%88
方源叹息一声,选择将仙窍中的资源,搬出一部分来,暂时交给琅琊地灵保管。

%89
“这么多的资源?看来方源你经营仙窍,很有心得。”琅琊地灵对资源的数量,有些诧异。

%90
但其实,这才只是方源拥有的一小部分。三成都不到的程度。

%91
方源自然不会将全部资源都搬迁出去,琅琊地灵不是傻子,会立即怀疑到方源的至尊仙窍。

%92
离地灾来临不到半天,方源通过传送蛊阵,再次来到北部冰原。

%93
北部冰原虽然是狂蛮魔尊一手打造出来,隐藏无数狂蛮真意。但天意仍旧存在。

%94
所以方源没有提前到这里来。

%95
冰原浩大广袤,方源随意选择了一个方向,找到合适的位置,就落下了仙窍。

%96
第二次地灾会是什么?

%97
不管是什么,第二次地灾的威能,必定比第一次要更加强大。

%98
方源严阵以待!

\end{this_body}

