\newsection{星宿的牺牲}    %第七百三十一节:星宿的牺牲

\begin{this_body}

即便是监天塔,拥有九转宿命蛊的监天塔,此刻也难以抵挡梦道杀招的侵蚀。

龙宫乃是八转仙蛊屋,拥有八转仙蛊如梦令。由此而来的八转梦道杀招,恐怕是当世第一了。

正所谓一招鲜,吃遍天。

监天塔中没有任何梦道防御手段,因此龙灵一出手,效果立竿见影!

“照此下去,监天塔将不复存在!”

“此招不是单纯梦境的蔓延,而是梦道的杀招,就算是天庭有着纯梦求真变,也无法可想了!”

一瞬间,药皇等人都看到了胜利的希望。

“没想到真的到了最后关头,竟是白凝冰力挽狂澜!”

“撑住!”

“一齐出手,护卫龙宫!”

南北诸仙积极配合,天庭蛊仙大惊失色,之前自信之色荡然无存,紧张焦躁地展开狂攻。

“不太对劲!”关键时刻,方源将注意力放在龙公身上,旋即就发现了不妥之处。

和帝藏生纠缠的龙公,没有丝毫的慌乱之色,神情平静,甚至目光中透露出一丝缅怀。

一百多万年前……

天庭深处。

“洪亭,你竟然还敢进入天庭,来到这里自投罗网!你重生的秘密已经暴露,你竟敢违背宿命,妄图逆反天意,复活死人,实在是令为师失望!”龙公望着眼前的红莲,神情如铁,目光中燃烧着怒火。

红莲魔尊微微一笑:“师父,我既然来到这里就没想着活着回去,更没想去破坏宿命蛊,还请您且慢动手。”

“哦?那你是想要做什么?”

红莲魔尊深深地看了龙公一眼:“我是想要劝服师父,帮助我摧毁宿命蛊!”

龙公一愣,旋即哈哈大笑一阵,然后蓦地收敛笑声,一对琥珀龙瞳深不可测:“那你就说说看,为师倒要好好听听。”

红莲微微一笑:“重生了数万次,我也是意外得知了线索,随后又用了十多次重生的机会,这才探查清楚。宿命蛊早已经不再护佑我们人族了,如今它的启示是四个字……”

他还未说完,龙公却已接道:“龙人当兴。”

红莲一愣,旋即大笑:“师父啊,你果然是知晓的!你这么维护宿命蛊,是要真的率领龙人一族崛起吗?毕竟数目最大的一股龙人,就是你的血脉子嗣啊。”

“当然不是!”龙公摇头,声音低沉,“洪亭,你只知其一未知其二。龙人当兴的秘密,其实是为师第一个发现,也是第一个上报。为师虽然转变成龙人,成为龙人之祖,但真正的心从始至终都是人族真心!不管你信还是不信,事实便是如此。”

“我信!”红莲毫无犹豫,“我相信师父的真心。师父啊,你是多么的了解我,同样的,我也了解师父你。你是绝对不会为了一己私利,背叛人族的。刚刚那番话,只是我怕你否认,故意的激将。”

“哼,臭小子。”龙公冷哼一声,心中涌过一抹欣慰之情,继续道,“为师开创龙人延寿之法,本意只是为了好好教导你,辅佐你执掌天庭,领袖人族,万万没有想到在宿命蛊中,会初选龙人当兴的启示。”

“不过,经过最初的震惊和惶恐之后,为师逐渐明白过来:宿命蛊乃是天道流派的蛊虫,天之道,损有余而补不足。人族之昌盛,统治五域两天,比当初的各大异族还盛,自然会惹来天道的折损,补其他的不足。就像曾经元始仙尊手握宿命蛊,得到人族当兴的情况一样。”

“然而为师仍旧要维护宿命蛊,你可知为何?”

红莲眼中精芒一闪即逝,问道:“为何?”

龙公深深地叹息一声:“其实宿命蛊的隐患,早在元始仙尊时代,就已经被他提出来。人族是因为宿命蛊而团结在一起。领悟了天命所归,绝大多数的人族方才有勇气反抗异人各族。然而,一旦人族当兴的天命消失了,人族又该何去何从呢?”

“毫无疑问,这是一个巨大的隐患,直到元始仙尊逝世,也没有解决这个难题。弥留之际,他将这份重担交托到了他的徒弟手中。这位徒弟便是后来的星宿仙尊,当她寿元将近的时候,她终于想到了解决隐患的方法。那就是——以身合道。”

红莲皱起眉头:“以身合道?”

“不错。”龙公继续解惑道,“宿命蛊唯有天道方能直接操纵,人族、异人族的蛊仙至多能观察一二,得到一些启示。或许未来随着流派争夺,会有运用宿命蛊的方法,但现在是绝无可能的。”

“星宿仙尊想到:既然无法从宿命蛊上着手,那么就不妨从源头下手!于是,她牺牲自己,以身合道,用自己的意志来参与天道的运转,干扰和影响天意。天意被影响,宿命蛊就能被我天庭一直掌控。”

“原来是这么一回事!”红莲魔尊心头大震,恍然大悟,“宿命蛊编织丝网,叫做万般网。世间的万事万物,都在这片网中,受到宿命的摆布和操控。星宿仙尊操纵天意,就能影响宿命蛊,使得人族始终当兴。”

“难怪宿命蛊对于天庭如此重要。难怪星宿仙尊设计,能够阻挡包括我在内的三位魔尊。难怪天庭始终屹立不倒,始终是人族圣地,天下第一的超级势力!”

龙公点头:“洪亭,你的悟性从未叫为师失望过。天下之势,分久必合合久必分。你见过哪一个超级势力,能维持长久的?从内部而论:人心易变,又充满欲望,而欲壑难填。再强大的势力,再亲厚的羁绊,也难以抵挡人心的不足。从外部来讲,惹上一位强大的蛊仙,往往就要让一个超级势力焦头烂额。若是这个蛊仙修行到八转,多数就能令一个超级势力衰落乃至灭亡。在这个世界,个人的实力很容易就凌驾在组织之上。”

“很少有一个超级势力,能够维持万年之久。但是我们天庭却是一个意外。为什么?就是因为有宿命蛊。”

“在宿命的安排下,我们始终是人族圣地,我们的内部会始终和谐,就算有矛盾,也会有种种机缘和巧合顺利瓦解。我们的敌人往往成长不起来,就算是有八转层次的强敌,也会渐渐地没落,甚至不需要我天庭亲自出手,强敌自己就会因为种种际遇死亡甚至投降。”

“能够达到如此的奇迹,是九转智道仙尊的牺牲换来的。没有九转智道的修为,根本无法抗衡和干扰天意的正常运转。然而,即便是星宿仙尊的意志对抗天意,也时常力有未逮,偶尔落入下风。”

红莲了然,插话道:“就像龙人当兴?”

“不错。起初之时,我也吓了一跳。但当我得知这等远古秘辛之后,我选择等待。果然过了不久,宿命蛊的启示就再次发生了变化,龙人当兴没有了,变回了人族当兴。”龙公欣慰地笑道,“后来我查阅天庭秘史,发现这样的例子其实出现过很多次。龙人当兴,不过是最近的一次罢了。龙人当兴的启示,是做不得数的。”

红莲的心不断地往下沉。

既然师父龙公早就知晓此事,那么他为什么又要听自己的问话呢?

他忽然意识到:不是他自己深入天庭,而是龙公早就在这里等他,等着他来询问,然后告诉他这份天庭的远古之秘。

红莲停止思索,目光一抬,盯着龙公。

龙公微笑道:“洪亭啊,为师等你来,说这样一番话,是要让你回头是岸啊。你能开创春秋蝉,重生回去,实乃魔头行径。看似是有改变过去的希望,但事实上,你有么?”

“没有。”龙公摇头接道,语气斩钉截铁。

“因为你的头上始终有天,有着天道,有着星宿意志,有着宿命蛊的启示。你如何能翻得了天?你怎么能改变得了过去?”

“就算你有魔尊九转战力,又能如何?无极、狂蛮就是最好的例证。一位尊者能有多久的寿命?活的最长的元始仙尊大人,不过是二万五千寿龄。而天庭呢,它有数百万年的历史和积累。它有历代仙尊的牺牲和守护。它有宿命蛊。这是大势,这种雄厚的积累你撼动得了吗?尤其是你重生归来,只是依靠未来身杀招,方才有的九转战力。本质上还并非是真正的九转尊者。”

红莲沉默。

即便是强大如他,面对天庭,也时常感到无奈。

他可以毁了天庭,但这又有何用呢?

只要有宿命蛊在,天庭就会轻易重建起来。因为星宿意志一直在影响着天意,宿命蛊中始终是人族当兴,始终会有天庭是人族第一圣地,第一势力的启示。

在宿命的操纵下,布置出种种机缘,无数的良才美玉会积极投靠天庭,组建出天庭。

天庭不是关键,星宿意志和宿命蛊才是关键。

对于星宿意志……

首先,红莲不是智道九转,对付以身合道的星宿意志,他手段不足。

其次,他也不想对付星宿意志。毕竟星宿意志干扰着天意,维护着人族的地位,始终让人族兴盛。

红莲只是想复活自己的亲朋爱人,他从未想过背叛和伤害自己的种族。

所以,留给红莲魔尊的仍旧只有一条唯一的路,那就是摧毁宿命蛊!

记住手机版网址:m.

------------

\end{this_body}

