\newsection{陆畏因}    %第三百八十九节:陆畏因

\begin{this_body}



%1
“和命相比,运是无法持久的。”

%2
“虽然出了一点意外,但又能如何?”

%3
紫薇仙子笑了笑,优雅从容。

%4
她手中的星宿棋盘中,正透露出那两位天庭蛊仙的身影。

%5
仙蛊屋在飞驰。

%6
两位天庭蛊仙面沉如水,此时正在相互交流。

%7
“上极天鹰通过黑白颠倒云,逃去了黑天。但按照现在的时辰来算,五域已经是白天了。我们现在身处在白天当中,黑天已经到了我们头顶。所以我们这一次,要飞升到高空,穿透天罡气墙,才能进入黑天。”

%8
“不错,我们已经在上极天鹰的身上,种下了侦查手段。只要到了黑天,我们就会有微弱的感应,不愁找不到它。”

%9
“再遇到上极天鹰,我们可不能重蹈覆辙了,尽一切努力,避免之前的情况再度发生。”

%10
两位天庭蛊仙斗志十足,自信经过一次打击之后,也彻底恢复。

%11
给予他们充分自信的,是他们掌握的实力。

%12
实力的差距就在这里,明明白白,就算上极天鹰好运爆棚第二次、第三次,又能如何?

%13
难道次次都能如此吗?

%14
只要有一次不成,那么天庭蛊仙就胜利了,而上极天鹰则会身陷囚笼,再也无法翻盘。

%15
星宿棋盘轻轻一震,画面变化,显现出凤九歌来。

%16
此时此刻,凤九歌已经脱离了光阴长河,重新来到了西漠。

%17
只是这一次,因为他通过不同的光阴支流,回到西漠,所以他并不能原路返回。

%18
紫薇仙子出声指点他,按照凤九歌现在的距离,离方源还是颇远的。

%19
“凤九歌乃是护道人。”

%20
“他的气运,稳如大山,岿然不动,也只有这样的浓烈气运,才能对抗得了方源等人。”

%21
“若是我派遣其他天庭蛊仙前去,人多手杂,气运被方源反制,反倒会帮助方源。”

%22
“方源的那只上极天鹰,就是明显的例子。两位天庭八转蛊仙联手,都能逃脱。更何况对付方源?”

%23
“就先让凤九歌不断压迫方源,迫使他寻找红莲真传,就算这个目的达不到,也能逼迫他不断泄露底牌。当他底牌全部展露,就是我亲自出马的时候了。”

%24
想到这里,紫薇仙子的眼中闪过一丝骇人的杀意。

%25
她不是不重视方源,而是非常重视方源。

%26
为了绞杀这个完整的天外之魔,紫薇仙子要亲自动手,才能安心。

%27
不过,现在不是时候。

%28
一来,方源的底牌似乎还未露尽,本身也未到绝境。

%29
二来,天庭这方面需要镇守,因为龙公还在天庭深处,努力镇压幽魂本体。整个中洲大局,需要她紫薇仙子日理万机,去统合操纵。

%30
三来,大时代临近,南疆已经开始出现地脉的震动,中洲的位置最为劣势,必须进行全面充分的准备。

%31
即便是方源掌握了自爱仙蛊,也没有超脱紫薇仙子的掌控。

%32
上极天鹰虽然逃生了一次,但仍旧在被追杀,未脱离险境。

%33
紫薇仙子更在意的,不是这两者,而是光阴长河中忽然出现的石莲岛。

%34
“红莲真传为什么会突然出现,又神秘消失?”紫薇仙子皱起眉头,百思不得其解。

%35
掌握着星宿棋盘的她,已然是天下前三之列的智道大能了。

%36
但是面对人族历史上,最为神秘的红莲魔尊,紫薇仙子仍旧感受到自己的无能和不足。

%37
“红莲啊……”紫薇仙子心中叹息。

%38
因为她知道,在所有的魔尊当中,就数这位红莲魔尊和天庭纠葛最深。

%39
甚至,他曾经一度,极可能成为红莲仙尊,入主天庭……

%40
黑天。

%41
黄绿色的菌草,宛若一张巨型的地毯,静静地生长在黑天的角落里。

%42
附天菌。

%43
这是一种六转仙材,只有太古九天中才能生长。

%44
它生长的条件,比较苛刻。

%45
必须单一生长,周围不能有任何的异种道痕、其他生命,甚至连陨石都不能存在。

%46
不过好在,太古九天虽然是上古级成群结队,太古级生命横行,但是空间还是广阔。总会有附天菌生长的环境。

%47
不过此时,在这片附天菌上,却逗留着两个人影。

%48
吱吱吱吱!

%49
一连串急促的鸟鸣声,突兀地响起。

%50
一位青年模样的蛊师,吃力地举起右手,只见他的前臂和右手,都笼罩着一层七彩霞光。

%51
这霞光非常活泼,好像要脱离青年蛊师的手臂,有了自己的灵性,但始终又差那么一点。

%52
青年蛊师咬牙切齿,努力支撑着,很快便是满头的汗渍。

%53
“去吧,霞鸟!”青年蛊师支撑到了极限,明白自己必须赌一把,他猛地低喝一声,然后艰难向前甩手。

%54
吱吱吱!

%55
下一刻,霞光鸣叫声变得更加清脆高昂。

%56
然后七彩的霞光,真的跃出了青年蛊师的手臂,一下子跳到了空中,然后旋即变化成一头飞鸟形态。

%57
鸟翼一振,七彩霞光速度惊人地向前直射。

%58
轰!

%59
一声巨响,七彩霞鸟正中不远处的附天菌,直接把附天菌轰炸得四处飞溅。

%60
霞光消散之后,附天菌上出现了一个井口般大小的浅坑。

%61
青年蛊师见此,顿时又惊又喜。

%62
喜的自然是,自己努力了这么久,辛苦锻炼,多少次险死还生,终于特训得到了成果,练成了这个杀招。

%63
惊的是这个杀招的威力,居然这么大,一如师父所预言的一样。

%64
“师父,我练成了!”青年蛊师来到另外一人的面前,行了一礼,然后兴奋地道。

%65
“嗯,不错。”他的师父淡淡地评价一声。

%66
青年蛊师见到师父的神情,心中的兴奋不由地疏解平复了许多。

%67
他的神情也旋即变得平静了许多,当即又是一礼,声调缓和地道:“是徒儿失状了。”

%68
“无妨。”青年蛊师的师父摆摆手,缓缓迈步,走向远处。

%69
几步后,他停了下来,仰头望向一处天边。

%70
他一身灰色的麻衣,非常朴素,此时随风摇摆,强健如熊般的体格,根本遮掩不住。

%71
他带着斗笠,又宽又大,斗笠的边沿在他脸上投下浓重的阴影,遮住他的面庞。

%72
青年蛊师虽然认他做了师父,但从未见过他的真面目。

%73
事实上,只有到现在,当他仰望天际的时候,青年蛊师才得以窥见他师父的面容。

%74
但也只能看见一个下巴。

%75
他的下巴又宽又厚,给人坚定不移,敦实厚重的感觉。

%76
“师父,谢谢你这些天的指点,否则以我个人苦修,怎能有如此惊人的进步。”青年蛊师道。

%77
他的师父沉默了一会儿,这才缓缓开口:“叶凡啊,你天资聪颖,悟性惊人。这杀招本就是你自己所创,将你修行的变化道和光道的领悟,都凝聚在一起,为师不过指点你了一些关窍而已。这杀招你虽然已经练成了一次,但今后还需要多多练习,只是要注意安全,因为为师不在你身边了。”

%78
“师父,我们就要分别了吗?”青年蛊师顿时一惊,脱口急问。

%79
此人正是南疆叶凡。

%80
他暗地里倾慕商心慈,商心慈初登族长之位,叶凡挺身而出,帮助商心慈料理商家世俗的难题。

%81
但是人外有人,天外有天,叶凡遇到了白凝冰,遭遇惨败不说,更是经历生死间的大恐怖。

%82
他捡回一条性命,一度失魂落魄。

%83
这次难忘的经历,让他沉淀下来,开始思考人生,思索自己。

%84
一次意外,让他撞见了现在的师父,并接下师徒的情缘。

%85
没想到,他的这位师父赫然是一位蛊仙。

%86
师父为了栽培这位极有悟性的弟子,亲自带领叶凡,来到黑天,特意锻炼他。

%87
相处的时间虽然很短,但是叶凡却极为敬佩自己这位神秘的师父。

%88
人和人之间交往,是需要投缘的。有的人相处一辈子,都可能很陌生。有的人相处时间很短,却能引为知己。

%89
虽然叶凡并未见过师父的真面目,但却能感受到师父对他的真诚之心。

%90
他亦是对师父百分百信任,毫不怀疑师父有什么其他居心。

%91
“缘起缘灭,你我师徒缘分并不多。今日耗用,明日就尽。反不如细水流长,以待将来。叶凡,你知道为什么我要带你来到这里吗?”师父道。

%92
叶凡眨了眨眼:“师父不是说,黑天的环境压制我的光道,在这里修行,虽然艰难困苦,但却因为压制,变得安全。”

%93
“此乃原因之一,这原因之二,便是为师算到,这里有你的一份因果。”他的师父低沉地说道。

%94
“因果?”叶凡面露疑惑之色。

%95
“世间之事,有因必有果,有果必有因。你看,你的因果来了。”说着,神秘蛊仙手指着远方天边。

%96
叶凡投去视线,惊呼一声。

%97
在那里,一头上极天鹰正向这片附天菌扑来。

%98
“好俊的一头神鹰!啊,它受了伤。”双方距离缩短,叶凡看见了上极天鹰身上极为沉重的伤势。

%99
叶凡虽然得遇名师,但眼界还是比不上蛊仙。事实上,寻常的六转蛊仙,也未必认得太古荒兽上极天鹰。

%100
上极天鹰此时已经神智模糊不清,它的伤口一直在流血,从未愈合过。

%101
天庭的追兵虽然暂时甩脱,但是黑天、白天的环境,都是危机四伏的,并不安全。

%102
当它看到这里有一片附天菌的时候,它便飞了过来,想要赢得宝贵的喘息的机会。

%103
但是它没有飞到附天菌上,距离还有数千步,它就力竭昏迷过去。

%104
不过,叶凡的师父及时出手,将这头上极天鹰一把捞过来。

%105
他张开手掌,也不知道他用的什么手法,在叶凡惊愕的目光中,这头巨型的上极天鹰就受到一股无形之力,越变越小,最终变成雏鸟一般的体型,倒在了叶凡师父的掌心中。

%106
叶凡这位神秘的师父,又伸出另外的一只手,轻轻地在上极天鹰的表面抚摸了一下。

%107
瞬间,上极天鹰的伤势消散全无,神态安详地陷入沉眠当中。

%108
“叶凡,你身上因果纠缠,这头上极天鹰或许就是你今后,解除因果的关键。”

%109
“你把它带在身边,等到它醒来,它会带你离开黑天,回到南疆。”

%110
“为师这便去了。”

%111
说着,这位斗笠灰衣的神秘蛊仙,便飘然而上,缓缓飞向远方。

%112
“师父,您慢走。”叶凡接过小鸟一样的上极天鹰,连忙追赶。

%113
他满含泪水,极为不舍。

%114
他一直追到附天菌的边缘,看着师父的身影越来越小,忽然喊道:“师父,可否告诉徒儿您的名讳?”

%115
“凡人畏果,我畏因。你可以称呼我为——陆畏因。”

%116
他师父的声音,飘飘渺渺,隐约传入叶凡的耳中。

%117
“师父……陆畏因……”叶凡口中呢喃,已是痴了。

\end{this_body}

