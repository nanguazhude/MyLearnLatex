\newsection{全知全能}    %第八百四十三节:全知全能

\begin{this_body}

方源相当重视这一次的预感。

上一世的时候,他在五界山脉中意外目睹了陶铸真传的开启,结果没有得手。事后就有了类似的感觉,冥冥中知道自己错过了一次重大的机缘。并且这一次错过,会令将来的自己十分遗憾。

果不其然!

在之后的中洲大战中,五界大限阵起到了非常关键的作用。若是方源掌握了这种手段,不仅能够对付星宿仙尊的久久不绝连环阵,并且连人中豪杰这样的杀招,也能够用五禁玄光气来应对。

而这一次,面对豆神宫,这种感觉又来了。

“但和上一世的预感不同,这一次的预感是在我抉择之前,而上一世是在事情发生之后,我已经错失了陶铸真传后,才感觉得到。”

“这是否也是一种进步呢?”

“如果说,这是一种进步,那么引发这种进步的原因又是什么?”

方源陷入沉思。

对于预感本身,方源兴趣浓厚。但他更感兴趣的是,是什么导致这种预感的产生。

推算之后,方源找出了答案——流派境界!

“我的智道境界应是主因。智道大师就能产生直觉,智道宗师能够触类旁通。智道蛊仙精修念、意、情,任何一种情绪的产生,都是一次预兆,意义重大,和其他流派的蛊仙不同。”

“其次是运道。运道在于变动,改变。我的运道有大师境界,也产生了直觉感应。”

“然后是宙道。上一世,预感陶铸真传错失的时候,宙道境界已经起到了作用。这一世宙道境界是准无上,所以更能感应到将来的某种情景。正因为这一点的改变,导致我如今提前预感,知道了此次抉择的重要性!”

宙道境界的暴涨,令预感和事情发生的前后关系发生了颠倒,这是一种质变的提升!

“最后是木道境界。因为事关豆神宫,我之前木道境界极其普通,所以预感并不灵敏。当我成为木道宗师之后,预感终于补足了最后一块,从而凝聚成形。”

智道境界、运道境界、宙道境界、木道境界……

这四大境界相辅相成,从而形成了奇迹般的预感。

寻常的蛊仙,万万难以达到这一点。方源这种兼修各种流派的个例,就目前为止,恐怕就只有他一人了。

更准确的说,方源五百年前世,幽魂魔尊炼成了至尊仙胎蛊,也是至尊仙体。

将来就说不好了。

在未来,梦境四处涌现,蛊仙通过梦境,也可以得到各种流派境界的飞速提升。

这些境界组合起来,说不定也能得到预感。

方源继续仔细分析:“预感的基础是智道境界。宙道境界的高低,决定了是事前预感,还是事后预感。运道境界能够让我对某种变化的趋势,更加敏锐。而木道等其他流派的境界则是预感的范围。”

“我这次预感,是木道境界提升了上来,才有了清晰的感应。将来我若是将其他的流派都补足了,就能预感到各个方面的事情了。这是一种怎样的境界?”

一个词刹那间在方源的脑海中闪现——全知全能!

“不,全能还根本算不上,但是全知却是可以达到的。”方源满脸郑重之色。

当所有的流派境界,都达到一定的程度,就可以做到囊括万物的事前预感。这种程度的预感,已是近乎全知了。

“若是在至尊仙体的基础上,发展出全流派的卓绝修为,那也近乎全能了吧?”

“全知全能……”方源口中呢喃,心头一个激灵,眼冒精芒,“如果我达到了这样的层次,追逐永生的道路,必然应当是能看得清楚了!”

方源暗自激动。

他虽然经历丰富,屡屡重生,但过往的经验早已经对他帮助不大了。

很多时候,很多事情都需要他摸索、实践。

就比方说九转尊者修为吧,方源身上的道痕已经突破了八转的局限,吞并气相洞天之后,灾劫往前跳,也应该跳出了八转的范围。

但他却没有晋升成九转。

如何成为九转,是他需要摸索的。

而今,他从这一次的预感中,摸索出了各大境界的奥妙,惊喜的得到了兼修的意外收获。

寻常蛊仙或许止步于此,但方源究其根本,竟又从中探索出了永生的希望!

历代尊者都无法永生,方源也只是有一个永生的目标而已。究竟该怎么实现这个目标,他没有任何的概念,或者可实行的想法。

终于到了今天,他从一个小小的预感中,琢磨出了一条可能到达永生的道路!

“虽然还不清楚如何永生,但全知全能是基石。接下来,先摧毁宿命蛊,让永生变成可能。然后就要努力做到全知全能!”

“要做到全知,就要努力探索梦境,将各大流派的境界提升上去。”

“要做到全能,则要依靠至尊仙体修行全部流派,将所有的手段都牢牢掌握。”

不管是全知还是全能,单做到一项,就已经超越了历代的所有先贤!

纵然是那些尊者,也不过是兼修两道、三道而已,达不到全知的境界。能力上虽然超凡脱俗,令方源也难以理解,但不是至尊仙体,也做不到全能。

方源若能成就全知全能,必定超越了所有的尊者。到那时,尊者无法达到永生,而他却未必不可以!

“等一等,三尊说的预言中,明确指出大梦仙尊会成为尊者中的最强者。难道是因为她会达到全知的境界吗?”

这很有可能。

大梦仙尊必定是梦境探索的大能,各大流派境界必定超凡脱俗,拥有很高的水准。从而做到全知,也不奇怪。

方源心头微跳,对于三尊说,对于未来的脉络,又有了一层全新的更加深入的认知!

他现在是八转修为,战力可敌龙公,已然是世间巅峰人物。然而,对于这个天地,对于未来,他仍旧是渺小的,他需要学习,需要了解的东西还有很多很多,他需要成长的方面还有无数。

全知全能这样的修行目标,太过高远,简直是能吓死人。大多数人看了,恐怕都会绝望到窒息吧,根本不会有一丝尝试的想法。

但方源却是被激起了无穷的斗志!

“如此一来,所有的纷争,一切的努力,我的生命才会色彩缤纷啊。”

数天后。

琅琊福地。

“快快快,火候加大!”琅琊白毛地灵一边主持长毛炼道大阵,一边催促道。

负责火候的毛民蛊仙,立即拼尽全力,鼓动火力。

但琅琊地灵尤嫌不足,仍旧高声嚷道:“火还要更大,更大!”

“但是大人,这已经是最大的火了。”毛民蛊仙叫喊起来。

琅琊地灵经验丰富,眼珠一转,便想到了方法,旋即下令:“毛六,你来催动大阵中的风阵,我们用风鼓火!毛八,你用炼道杀招,待会迅速拔取残风,净化仙蛊雏形!”

有了风,就有了风道道痕的污染。所以要净化之后,才能令仙蛊炼制顺利进行下去。

琅琊地灵准确调度,忙而不乱,须臾,炼道大阵猛地一震,一道虹光爆发出来,直冲九霄,而后迅速消散。

大阵徐徐停止了运转,一只全新的木道仙蛊被炼制了出来。

“这一次又成功了!”

“真不愧是琅琊地灵大人呐。”

毛民蛊仙们兴高采烈,赞叹不绝。

琅琊地灵却是充满了感慨:“还是多亏了主人的蛊方啊!”

方源有着炼道准无上的境界,又有木道宗师境界,改良出来的木道仙蛊方达到了极高的层次。

再加上琅琊地灵巅峰级的炼道造诣,自然是事半功倍,成功率很高。

其他的毛民蛊仙也有同感。

“地灵大人说的是!”

“一直以来,仙蛊炼制的成功率太低,都是局限于炼道造诣和蛊方。绝大多数的蛊仙,都不是专修炼道,炼道连兼修都算不上,炼道造诣可想而知。而蛊方改良,则不仅需要相应的流派境界,同时也需要炼道境界,还有智道推算的造诣。”

炼制仙蛊一直都是一个巨坑,但在方源这里,却渐渐成了寻常之事。

皆因方源拥有智慧光晕等智道手段,又有炼道准无上境界,其他流派的境界也非同小可。

所以,一直制约仙蛊炼制的几大障碍,对于方源而言,却已经荡然无存了。

至尊仙窍中光阴流速非常的快,这么多天下来,方源收获了数只木道仙蛊。

他终究是选择了木道,并且是己方来炼制。

其实南疆的木道蛊仙蛮多,木道仙蛊也有不少,但方源没有去敲诈。

南疆正道现在开始大肆铺设烽火台,同时也渐渐整合起来,和天庭展开合作,在宝黄天中遏制方源的各种生意。

这种情况在上一世也出现过,表明了武庸正在逐渐发力,统合整个南疆正道。

方源若是敲诈南疆正道木道仙蛊,必定会被天庭知晓。

紫薇仙子就会产生疑惑:“方源收集木道仙材,难道是想要在木道上有所发展?他想要炼制木道仙蛊?他要干什么?”

天庭因为豆神宫,对于房家又分外关注。

将这两者联系起来,很是容易。

------------

\end{this_body}

