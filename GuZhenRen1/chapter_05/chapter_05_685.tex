\newsection{中洲混乱}    %第六百八十八节:中洲混乱

\begin{this_body}



%1
中洲。

%2
高空中悬浮着一座逍遥岛,这是大乘阁的驻地。

%3
一场炼蛊大比已经步入尾声。

%4
镇守逍遥岛的三位蛊仙也放松下来。

%5
一位望着岛外云层,伸了个懒腰:“终于结束了,这一届的炼蛊大会可是我监管以来,最累的一次了。”

%6
“哈哈,是啊。也是没有办法,方源那魔头闹得太凶,许多外域的人也是居心叵测。”

%7
“好在这次任务已经完结,我请二位喝酒。”

%8
“哦?仙友的九流琼酿我早已闻名已久,没想到今日能有这个口福。”

%9
“哈哈哈,过誉了过誉了。”被夸奖的蛊仙连连摆手,忽然神情一滞。

%10
“呃!”他带着满脸的惊愕,浑身僵硬如石,一头仰倒下去,倒到地上的时候,就彻底死了。

%11
“戒备!”

%12
“敌袭!他的魂魄散了!”

%13
剩下的两位七转蛊仙大惊失色,连忙撑起防护手段。

%14
然后,他们就看到场中参赛的蛊师们,一个接着一个都栽倒下去,当场阵亡。

%15
两位蛊仙看得睚眦欲裂,极其愤怒,就在要成功的关口竟遭遇到了偷袭。

%16
“居然如此残忍,对凡人蛊师下狠手!”

%17
“快出来,究竟是哪个魔道宵小,就会暗箭伤人么?”

%18
两位蛊仙大吼。

%19
大比的赛场上已是一片混乱,不断有人倒下。

%20
“你们要我出来?”一个笑声传出,“那就如你们愿好了。”

%21
笑声中,影无邪显露出身形。

%22
“邪魔外道,受死!”两位蛊仙咬牙切齿,直接扑杀过去。

%23
影无邪站在原地并未动弹,只以戏谑的笑容看向他们俩个。

%24
两位蛊仙杀来的途中,忽然脸色剧变,速度猛降。

%25
“我、我们……居然也中招了?!”

%26
“明明刚刚……”

%27
“这到底是什么手段?”

%28
两位七转蛊仙倒在地上,仰望影无邪不断走近。

%29
影无邪笑了笑,将三具蛊仙尸体收起。

%30
“你们的这些仙窍,宗主大人可都是需要呢。”

%31
他回首最后看了一眼整个比试场地,里面再没有站着的人,都是一片倒地的尸体。

%32
片刻后,影无邪悠然飞出逍遥岛。整个逍遥岛一片死寂,无一个生灵存活。

%33
与此同时,磷光河上游。

%34
黑楼兰大吼:“区区仙阵也想阻我?”

%35
话音未落,她气势暴涨,充天彻地,辐射四面八方,卷席无边风云。

%36
轰!

%37
一记杀招打出,将达到极限的防御仙阵彻底轰破。

%38
噗噗噗。

%39
镇守此阵的数位蛊仙,接连大吐鲜血,遭受反噬。

%40
黑楼兰狞笑一声,身上力道虚影层层叠叠,杀入蛊仙之中。

%41
百节林地。

%42
白凝冰踩在雪地里,悠然行走。

%43
原本郁郁葱葱的竹林,已经被冰霜冻住,宛若琥珀中的虫子,一动不动。

%44
草地也被皑皑积雪覆盖,再无一丝生机。

%45
竹林中央的比试场地中,数百位凡人蛊师都冻成了冰雕,无人生还。

%46
白凝冰、黑楼兰、影无邪,这三人一直在石莲岛中修行,加持未来身杀招,借来未来的状态,战力皆是飙升,成为七转中的精锐强者,寻常人物绝非他们的对手。

%47
看到他们这样的战果,方源也颇为欣慰。有了未来身的加持,终于令这三人突飞猛进,有了参与大战的资格。

%48
“这三人都是人杰,才情卓越,一旦战力提升上去,立即使得同等层次的蛊仙黯然失色。现在就看天庭如何反应了。”

%49
天庭。

%50
新的战报传来,紫薇仙子脸色又沉下一分。

%51
她号令天庭,不惜高昂代价,收取宝黄天中的宙道仙材。结果不仅没有限制住方源的提升,现在就连方源的三位下属都有了如此实力。

%52
战报交到龙公手中,龙公见到其中的战斗影像,眼眸中厉芒一闪,不由动容。

%53
紫薇仙子没有辨认出来,龙公却一目了然。

%54
他沉默了一下,这才道:“好一个方源!是我小瞧了你。”

%55
紫薇仙子感到疑惑:“龙公大人,您何出此言?”

%56
“令这三人实力暴涨的原因,乃是杀招——未来身。”龙公沉声道。

%57
“未来身?”

%58
“这是我大徒弟开创出来。当初他利用春秋蝉重生,失去了仙尊修为,为了弥补此点,便钻研出了未来身杀招。”龙公答道。

%59
这次轮到紫薇仙子脸色大变。

%60
她眉头紧锁,难以置信地道:“魔尊幽魂的记忆我已查探,他收获的红莲真传绝无未来身这项。如此说来,方源已经获取了一份红莲真传?否则,未来身是如何做到的?光阴长河明明已经被我们封锁了才对。”

%61
龙公微微摇头:“现在说这些,已无关紧要了。看来我们要密切注视方源的动向,说不得他手中掌握着某些特殊手段……”

%62
龙公再不敢轻视方源。

%63
或者说,他绝不敢轻视红莲魔尊!

%64
一时间,紫薇仙子、正元老人的心情俱都沉重。

%65
红莲魔尊带来的阴影,再次笼罩他们心头。

%66
但是接下来,方源却是再不露面,只有白凝冰、黑楼兰、影无邪不断出手,胜多败少,大闹中洲。

%67
“方源究竟在干什么?”

%68
“他是不是在偷偷继续,想要弄一个大阵仗?”

%69
方源不出手,轮到天庭这边感到难受。这种威胁就像是弓拉开,箭未射的状态,令人忧心忡忡。

%70
方源在等。

%71
但是他始终没有等到龙公现身。

%72
“局势越来越麻烦了。”方源暗中咬牙,感到不妙。

%73
龙公仍旧不现身,中洲炼蛊大会虽然被屡屡破坏,不断有牺牲者,但仍旧在持续开展下去的。

%74
尤其是当大比进行下去,剩下的蛊师越来越少,大比的地点也跟着减少,中洲蛊仙的防守人数和力量,都得到了明显的提升。

%75
白凝冰、黑楼兰等人开始结伴同行,单靠他们一人,已经奈何不得大比之地。

%76
方源同样没有等到长生天的人。

%77
在中洲炼蛊大会举办前夕,冰塞川主动联络方源,要和他联手破坏这场炼蛊大会,不让天庭得逞。

%78
但现在,长生天方面始终没有消息。

%79
“单靠我和影宗,根本无法阻止大比进行下去。除非有更加强大的势力介入!”

%80
方源很清楚这一点。

%81
影宗的确屠戮了许多生命,阻挠了大会的进展,但龙公雷打不动,仍旧在一步步将大会坚定地推进下去。

%82
龙公不现身,其他四域的蛊仙就有顾忌,也不敢过多出手。

%83
这正是龙公的智慧之处。

%84
所以,方源不得不来做这个急先锋。可是现在看似方源战果辉煌,但实际上局势仍旧牢牢控制在龙公之手。

%85
对于方源而言,中洲的局面并未打开。

%86
该如何是好?

%87
苦思冥想之后,方源毅然决定改变策略,他不再命令白凝冰等人去强攻大比场地,而是转换目标,去劫掠各个资源点。

%88
中洲资源相当的丰富,储量也很多。

%89
白凝冰等人每劫掠了一个地方,方源就将相关的影像,挂到宝黄天中去,任由其他蛊仙观看。

%90
这个策略初期几天,并不显效,纵然白凝冰等人收获颇丰,也难以令方源的心情高兴一分。

%91
但数天之后,开始有更多的纷乱发生了。

%92
北原散仙睡姑忽然现身,攻破战仙宗布置的大阵,将一处资源点洗劫一空。

%93
疑似萧虎痴、萧十让的两位蛊仙,强攻琉璃山脉,竟将此地的一个超级势力剿除干净。势力所属的三位蛊仙,全都丧命。

%94
运输向仙鹤门大本营的太古仙材,在途中遭受了一批神秘蛊仙的联手突袭,太古仙材被全部劫走。

%95
……

%96
又过数天,纷乱愈演愈烈,竟是八面烽火,十方混乱。

%97
各大闻名的资源点正被劫掠,或正准备被劫掠,一个个蛊仙的身影似乎活跃在中洲的每一个角落。

%98
“哈哈哈,人心终于是乱了!”暗地里,方源大笑三声。

%99
他此计毒辣,正中他人的阴暗心思。若是鼓动其他势力或者蛊仙,去直接强攻,去直接破坏炼蛊大会,他们会犹豫会观望。

%100
但是让他们去进攻,去劫掠一些资源点,却是避免和天庭直面对决,这个心理难度就很低了。

%101
同时,抢夺到的资源会激发这些蛊仙,令他们的胆子和胃口越来越大。

%102
而那些观望的蛊仙,看到别人抢走了那么多的资源,能不眼馋么?能不气躁么?毕竟许多资源,被别人抢走了,自己就没有份了。

%103
方源改变战术,立收良效。

%104
整个中洲彻底陷入混乱局面,西漠、南疆、东海、北原的蛊仙身影接连不断,甚至中洲本土的一些散仙、魔仙也在浑水摸鱼。

%105
“方源……我定要让你不得好死!”紫薇仙子恨得咬牙切齿。

%106
龙公则长叹:“人心不古啊。”

%107
即便如此,龙公仍旧不改初衷,一门心思推进炼蛊大会进行下去。

%108
那些比试地点,都被大量的蛊仙重重守护。

%109
正是因为天庭聚集了这些力量,才导致外界的资源点防御极其薄弱,被他人偷袭屡屡得手。

%110
可以说,方源是将计就计,在龙公战略的基础上特意转变了自家战略。

%111
龙公的决心毫不动摇,哪怕中洲局势烂得不成样子,他都牢牢守护着炼蛊大会。

%112
他的这个战略,让方源咬牙切齿之余,根本无法下手。

%113
龙公以不变应万变,逼迫方源强攻己方阵地。方源绝不会冒然进攻,因为他知道自己这边绝非天庭对手,况且其他势力还未登场,过早出手只会变成别人的踏脚石。

%114
所以,方源只得默默等待。

\end{this_body}

