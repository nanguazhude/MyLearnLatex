\newsection{落英馆}    %第四百九十三节:落英馆

\begin{this_body}

%1
“偷道大宗师境界……”方源深吸一口气,脸上神情有些复杂。

%2
终于有一天,他的流派境界,达到了这种程度。只是没有想到的是,不是血道、力道,也不是智道、宙道,而是偷道。

%3
偷道境界,方源原本是一片白纸也似。但是就因为这处盗天梦境,一路飙升,后来居上,竟然成为诸多流派的第一。

%4
由此可见,梦境的非凡能够让蛊师、蛊仙,极大地缩短积累时间。

%5
这也是五域乱战的乱战主因。

%6
正是因为有太多的人因为探索梦境受益,境界飙升,从而对修行资源、个人地位等等加倍需求,自身的*也随之扩大暴涨。

%7
当然,梦境不是那么容易探索的。

%8
方源虽然取得如此成就,但是当中付出的代价也非常的巨大,足以让一些八转蛊仙都力不能及。

%9
大宗师境界!

%10
方源到达这种地步,就是顶点,也是五域世间的极限。

%11
拥有这种境界的方源,对世间任何的偷道奥妙,都了然于胸。哪怕受到一些外在的迷惑,但只要稍稍想想,就能悟通了解。

%12
若他还想再进一步,偷道大宗师之上,便是无上大宗师。

%13
这就难了,也就是意味着方源要超盗天魔尊的成就!

%14
想起盗天魔尊,方源也感慨不已。

%15
梦境虽然虚幻,但这种写实梦境,却是源自历史史实而生发。

%16
方源帮助盗天多次,让梦境不断推演下去。但盗天的性情却是如此,大大出乎方源意料。

%17
在真正的历史中,究竟是什么,促使少年盗天最终成长为魔尊呢?方源觉得,沙枭恐怕是此中关键。

%18
达到了偷道大宗师境界,剩下的盗天梦境对方源而言,就失去了价值。

%19
不过这种事情,方源没有必要说出口。

%20
“可惜了。”方源对唐方明告辞,语气中饱含遗憾,“有急事发生,我必须离开,前往处理。盗天梦境还是留待下一次探索罢!”

%21
唐方明热切地询问:“不知道能有什么,可以让我以及唐家相助的?”

%22
唐方明并非热心的好人,而是按照唐家和方源之间的盟约,任何一方付出多了,自然就有更多的回报。

%23
唐方明从方源的指点中,已经尝到巨大的甜头,现在是趋之若鹜。

%24
方源看着唐方明这般态度,心中暗暗点头。这正是自己想要达到的情况。

%25
唐家,只要方源不断地运作下去,扶持下去,将来五域乱战时,必定是天庭横扫四方的巨大障碍!

%26
“还是算了,此事我自己就能处理。”方源摇摇头,回绝的话让唐方明目光暗淡,十分失望。

%27
不过,很快,方源就取出一只信道凡蛊,交给唐方明。

%28
“这,这莫非是?”唐方明心有所感,顿时激动起来,连忙双手捧着,接过这只信道凡蛊。

%29
“这蛊虫当中,有我对梦境探索的一点浅见,希望能够带给你一些助益。”方源笑了笑,动身启程,身后是唐方明不迭的赞佩之声。

%30
预先取之,必先予之,这是颠扑不破的真理。

%31
布局天下,也唯有这种气魄格局,方能和天庭较量下去。

%32
“只是……我这样布局天下的大战略,也比不上天庭啊。”离得远去了,方源望了望中洲的方向,深深一叹。

%33
天庭之强,不仅是本身实力,而且还有战略和远见。

%34
拥有宿命蛊,天庭就具备了极巨的战略优势。方源此时的战略参考了魔尊幽魂,已经是非常优秀,但和天庭一比,完全处于下风。

%35
只要修复宿命蛊,天庭就能在五域乱战中立于不败之地。

%36
若是再让龙公真的栽培出大梦仙尊,那必然就是时代胜者,五域独尊了。

%37
这个战略稳中求胜,几乎没有破绽,唯一的一个点便是天外之魔,因为天外之魔是不受宿命束缚的。

%38
完整的天外之魔,便是方源。而半个天外之魔则是赵怜云,此时已经被天庭控制,反而沦为走狗,和方源为难。

%39
可以说方源乃是对抗天庭的最大希望,可是就算是方源自己,也不知道出路在哪里。

%40
“若我单干,哪怕纠集影宗、琅琊派、唐家等等,也绝非是天庭对手。唯一的可能,还在于红莲真传。可是要取这魔尊真传,还不是时候!”

%41
之前方源前往光阴长河,是因为提前知晓了影宗布置,所以有底气。现在若要去探索,完全是两眼一抹黑,不知道会发生什么。

%42
没有石莲岛的帮助,方源极可能就会陷入天庭埋伏中,丧失性命。

%43
这种情况下,方源只能谨慎保守,不能盲目冲动。

%44
“还是要千方百计地增长实力,尽快地独自闯荡光阴长河,看看能否得到红莲真传。”这是方源的当前战略。

%45
要增长实力,方源首先想到的便是市井中的那些仙窍福地。

%46
可是这些仙窍,却很难再次开启。

%47
方源虽有方法,但颇为繁琐,代价颇高,远远不如将上极天鹰重新找回来。

%48
方源的上极天鹰,在南疆丢失,但是不要紧。

%49
在此之前,方源便未雨绸缪,将自己与上极天鹰连运,双方有了连运的关系。

%50
方源如今,只需要动用运道交感杀招,就能感应到上极天鹰的隐约位置。如此顺藤摸瓜下去,必然会最终找寻到上极天鹰。

%51
“不过,现在并不着急就去南疆。仙窍福地就存在市井当中,短时间内又跑不了!”

%52
方源的计划,是先在青鬼沙漠修行一段时间。

%53
“之前我救下房棱、房云,和房家也算是接触了。依照房家这样的超级势力,必然会调查我,考察我。可惜这段时间,我隐藏行迹,偷偷回来,探索了盗天梦境,房家恐怕也查不到什么。这个时间,已经差不多,房家极可能要派遣使者,来接触我了。”

%54
方源对这些正道的做派非常了解。

%55
因为方源救助了房家的蛊仙,房家若不感谢,就是知恩不报。这在正道的舆论中,将遭受非议。

%56
再说句自私点的话,房家若知恩不报,今后若房家蛊仙再陷入危难当中,谁还会再施展援手呢?

%57
所以,就算房家实质性的奖励没有多少,也要做足表面功夫。

%58
这便是正道的游戏手段。

%59
除此之外,还有更重要的一点,那就是方源展现出来的实力,令人忌惮。他操纵这么多的魂兽,本身又是智道蛊仙,在青鬼沙漠中发展,就算是房家也不敢小觑。

%60
房家地盘紧挨着青鬼沙漠,方源就是房家的邻居。这种人物,房家怎么不可能关注了解?怎么可能不来接触?

%61
若是接触的感觉好,沟通也良好,房家还可能会花大力气拉拢方源。

%62
当然了,这一切都要建立在方源的真正身份,没有暴露的前提之下。若是房家知道这是方源,绝不会来接触方源,更别谈拉拢了。

%63
而方源的计划就是,伪装身份,和房家交好,方便他今后布局青鬼沙漠,将这片地方布置成魂核的猎取地。

%64
这对方源的魂道修行,将有着举足轻重的作用!

%65
数天之后,青鬼沙漠。

%66
魂兽大军浩荡奔腾,方源坐在最中央的上古魂兽的背上,悠然自得。

%67
他正在试验各种偷道蛊虫。

%68
比如五转偷袭蛊!

%69
方源心念一动,催发出来,顿时从他的周围蹦跳出数百个蓝色婴影。

%70
这些小孩光影,不仅更加迅速,而且还非常隐约,稍不留神,就捕捉不到它们的行踪。

%71
一时间,数百个蓝色婴影在方源的周围乱窜,活蹦乱跳,有的穿梭在魂兽的腿脚之间,灵活至极,有的跳跃在魂兽背部,相互嬉戏。

%72
它们的行动没有任何的声音,在这阴沉的沙漠中,描绘出一副诡异恐怖的画面。

%73
方源如今是偷道大宗师,推算出五转偷袭蛊,简直是一念之间的事情。甚至方源已经掌握了六转偷袭蛊方,根本不需要什么智慧光晕的帮助,轻松至极。

%74
除了偷袭蛊之外,方源还构思出了许多其他偷道蛊方。

%75
方源在偷道方面,已经是傲立世间巅峰极限,这种感觉非常美妙,不是什么灵感,而是好似本身就蕴藏着这股能力。方源每次构思偷道蛊方出来,就是挖掘这种能力,印证这种能力!

%76
数天来,方源构思出大量的偷道蛊方,他都交给琅琊派来炼制。

%77
琅琊地灵都对此表现出了强烈的兴趣。

%78
方源得到了大量的五转偷道凡蛊。

%79
对于蛊仙而言,凡蛊中也就是五转蛊,稍微能用一用了。

%80
除了得到了这些偷道凡蛊,方源还屠戮魂兽,猎取浑厚,将魂魄底蕴重新攀回一亿人魂层次。

%81
不仅如此,他还趁机将这些偷道凡蛊,加入到他之前掌握的那些杀招当中去,进展颇多。

%82
就在方源正要闭目,继续改良杀招的时候,一座仙蛊屋从阴云中忽然冲出,直接向着他飞了过来。

%83
方源抬头望去,只见这座仙蛊屋并不庞巨,而是精致玲珑,如同林间小屋。

%84
但这小屋可不寻常,散发出七转仙蛊的繁杂气息,是一座七转仙蛊屋。

%85
它外表好像是木头搭建起来,在房屋的表面,生长着无数鲜艳的花朵,色彩缤纷,颜色绚烂。

%86
“这是房家的仙蛊屋落英馆。”方源心头微微一笑,果然如他所料,房家真的来和他接触了。

%87
ps:今天赶火车,回来晚了。但是待会十点左右还有第二更!

\end{this_body}

