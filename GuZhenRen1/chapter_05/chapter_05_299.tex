\newsection{斗茶}    %第二百九十九节:斗茶

\begin{this_body}



%1
“好诗,好诗。”天露仙子娇笑起来,“尤其是最后一句,月下花梦佳人处,真美。我想一定是说的我了,嘻嘻。”

%2
她故意插科打诨,一下子让凉亭中的氛围,缓和了下来。

%3
轮飞无奈地苦笑,重新落座:“诸位见笑,一点拙作,难登大雅之堂。”

%4
“轮飞仙友谦虚了,在下亦有一诗。”罗木子道。

%5
“哦?我等洗耳恭听。”乔丝柳微笑着,用期待的目光看向罗木子。

%6
罗木子沉默了一下,继而开口。

%7
登山寻仙处,

%8
寸步间高险。

%9
浮尘似光流,

%10
暗蛊藏心沟。

%11
金玉如一梦,

%12
万年恨寂寥。

%13
五域九天功,

%14
尽在一气中。

%15
这首五言诗一出,其气象和意境,顿时让在座的蛊仙微微动容。

%16
细细品味。

%17
登山寻仙处,意指蛊仙修行。

%18
寸步间高响,蛊仙修行,要应付灾劫,要殚精竭虑地经营仙窍。好似登山,每一步往高处走,都会有艰难险阻。

%19
浮尘似光流,意思是光阴长河,滚滚流逝,自己置身俗世,身有浮尘,身似浮尘。

%20
暗蛊藏心沟,这句表面上是指蛊仙将仙蛊、凡蛊,储藏在自家仙窍之中。但在场的蛊仙皆有文学底蕴,已经品味出来,这句还有一层意思。

%21
暗蛊,是意喻,阴暗、挫折、失败、妥协、失望种种情绪。

%22
凡人都道神仙好,神仙有苦人不知。试看世间万物命,谁能真正得逍遥?

%23
蛊仙在修行之中,承受着偌大的压力,心中产生各种负面情绪,不可避免,就算是仙尊魔尊亦是如此。

%24
登山寻仙处,寸步间高险。浮尘似光流,暗蛊藏心沟。这两句言语简练,但意蕴深远,细细品味,引起蛊仙强烈共鸣。

%25
在此之后,“金玉一如梦,万年恨寂寥”是指黄金美玉,不过虚无,种种财富,宛若梦幻。时间一久,爱恨情仇都消解瓦散。显露出诗人,淡看风云,红尘不加身的脱俗出尘的心境。

%26
最后一句“五域九天功,尽在一气中”,非常大气,一扫前言的抑郁沉重,气象磅礴。另外还有几层意思,人活在世间,都是活着一口气。一口气断了,人就死了。人活着,也是为了争一口气,蛊仙修行,若能类比天地,成为五域九天中的翘楚风流,就在自己的努力奋发之中。

%27
整首诗先抑后扬,气象庞然,叫人心折。

%28
一时间,凉亭中诸仙沉默,皆默默品味此诗。

%29
乔丝柳心中暗暗称奇:“这却怪了,按照我对罗木子的了解,他的心性如何能做出这种诗词来?恐怕是他人佳作,被他挪用。嗯,他也没有说过这是他自己作的。”

%30
她打量罗木子。

%31
罗木子表面上稳定如松,笔直端坐,静静饮茶,实则嘴角的微笑却已经出卖了他的内心情绪。

%32
乔丝柳暗笑一声,却不揭穿。

%33
她妙目一转,又看向身旁的方源。

%34
方源的神情竟有些古怪!

%35
“这是五言气绝诗啊?怎么可能!气绝魔仙的洞天,不是要等到五域乱战,梦境四起的时候,才会出世的吗?”

%36
“怪!怪!怪!”

%37
气绝魔仙乃是上古时代的大能,八转蛊仙,战力极强,曾经和无极魔尊三战,一胜一平一负。

%38
当然,前两次战斗,无极魔尊并未成就九转。

%39
而最后一次,无极魔尊以九转威能,凌驾于气绝魔仙之上。但双方仍旧打了九天九夜,后者方才败北。无极魔尊却未动手杀死气绝魔仙,而是将他放走。

%40
无极魔尊说了这么一句话:“你是我平生大敌,但没有你,我也不会刻苦拼命。我能有如今的修为,你也有促成功劳。”

%41
能够让堂堂魔尊如此认可,气绝魔仙名垂青史。

%42
他死后留下洞天,一直长存到了今天。

%43
方源前世五百年,五域乱战,梦境纷起,界壁消融,五域归一。如此大变,造成了天地二气的汹涌震荡,使得许多隐藏在世间各处的福地、洞天,都被震得显形而出,被世人洞察。

%44
气绝魔仙的气绝洞天,就是在这样的情况下,展露在世人眼中的。

%45
毫无疑问,它这一出现,顿时引起了五域轰动。

%46
“不应该啊。”

%47
“按照道理来讲,这首五言气绝诗,乃是雕刻在气绝洞天中的气死碑上。这个罗木子,怎么会提前知晓?”

%48
“难道说,他已经进入气绝洞天过?!”

%49
方源一时间,心都有些乱。

%50
气绝洞天中,藏有魔仙的真传。这位魔仙是什么层次的人物,很显然,是类似剑仙薄青之流,就算是黑凡都弱他一筹!

%51
“如果我能得到这份真传的话……”

%52
“杀了这个罗木子,得气绝真传?”

%53
方源心中又泛起了浓郁的杀意。

%54
罗木子尚且不知道方源的想法,他还用挑衅的目光看向方源:“不知武遗海仙友,可有什么佳作,我等都很期待。”

%55
“那是必须的。”轮飞紧接着开口,“武遗海大人出身不凡,经历丰富,东海更是资源丰富,武遗海仙友肚中的墨水,我是比不来的。”

%56
这两人把方源捧得高高,话说得很漂亮,心思却充斥恶意。

%57
乔丝柳心知肚明,此刻并没有维护方源,而是一瞬不瞬地盯着方源,鼓舞道:“我也很想听听遗海你的诗词,我相信这一定与众不同。”

%58
“是啊,是啊。”天露仙子连忙跟上。

%59
一时间,方源被挤兑,他摸了摸自己的鼻子,苦笑道:“诸位仙友高看我了,我哪里有什么诗词,压根就不会作。”

%60
“武遗海仙友谦虚了!太谦虚了!”罗木子哈哈大笑。

%61
方源摊开手:“我是实话实说,事实上,我根本不知道,赏月还要吟诗啊。”

%62
“即便如此,武遗海大人不妨现场作一首出来,想必也是高妙之作,时间长一点也不要紧,我等都愿等候。”轮飞道,不给方源下台的一丝机会。

%63
方源深深的叹息一声。

%64
他肚子里的诗词,当然不在少数。

%65
地球上的诗词,很多都是千古流芳,那些名言绝句随意摘抄一些应景,也足以撑得住这个场面,化解诸位蛊仙的刁难。

%66
但是……

%67
那又如何呢?

%68
方源扫视周围一圈。

%69
罗木子、轮飞爱慕乔丝柳,对方源自然看不顺眼,双人默契,一起发力,对排挤打击。和这等俗人有什么好较力的?

%70
天露仙子是乔丝柳的闺蜜,端的好帮手,帮助乔丝柳可谓尽心尽力。至于她的那位道侣,话不多,很多时候沉默寡言,就在一旁静静喝茶,这正是他的明智之处。

%71
而至于乔丝柳……

%72
这位仙子集美貌和家世于一身,号称是南疆三大仙子之一,自然有着骄傲。

%73
乔家虽然有令,命她与方源多加接近,但她自有一套手段。

%74
今天主持这场赏月雅会,她用足了心思。不仅是在座位这等细节方面,考虑周翔,而且还带来闺蜜助阵,更妙的一手,是邀请了轮飞、罗木子两人前来参加。

%75
当两个男人相互竞争的时候,就算是一头母猪,都会觉得好。只有当其中一位竞争获胜,另一位彻底出局之后,胜利的男人看向这头母猪,这才会惊觉:哦,原来这是一头母猪啊!

%76
这番话有点夸张,但道理是共通的。

%77
当竞争者出现的时候,便会显得被追求的女子,更加宝贵,更加值得珍惜。

%78
乔丝柳深谙此中之道,如此布局,就是想勾动方源的心思,化被动为主动,只要方源主动追去,乔丝柳便能顺势应对,将方源勾起。

%79
若是换做真正的武遗海,兴许已经落入了这位美人仙子的算计当中。

%80
可惜的是,她面对的是方源。

%81
方源一直都不为所动,因为他知道,武家和乔家的关系,还有乔家高层的谋算。

%82
占据这一点,方源就是高屋建瓴,稳定如山,罗木子、轮飞不过是两个跳梁小丑罢了。

%83
方源打量亭中蛊仙的时候,蛊仙们也都将目光集中在他的身上。

%84
凉亭中沉默下来,这种沉默无疑是一种施压。

%85
“罗木子、轮飞想要对付我,让我难堪。乔丝柳也想逼着我迎战,嗯……或许还有一些羞恼的情绪。毕竟之前的海水茶,对这样的美人,是一种轻慢。至于天露仙子,完全站在乔丝柳那边,无可厚非也不足为虑……”

%86
方源心念一动,暗地里笑了笑,然后道:“那我就来一首吧,你们可别笑话我。”

%87
“洗耳恭听仙友的大作!”

%88
“我等翘首以盼啊!!”

%89
罗木子、轮飞的脸上都堆满了笑。

%90
然后下一刻,众仙就听到方源吟道

%91
“大海啊,你全是水。”

%92
“骏马啊,你四条腿。”

%93
“美人啊,你有大大的眼睛,还有一张嘴!”

%94
吟诗结束。

%95
全场寂静!

%96
其余蛊仙的面容,像是僵硬住了。

%97
就连乔丝柳、天露仙子两人,都不例外。

%98
“这、这、这……什么鬼啊!”

%99
“这是诗词吗?这简直是胡扯啊!!”

%100
“莽夫,这个武遗海完全就是个莽夫。”

%101
“什么玩意儿?赏月吟诗,何等高雅,氛围完全都被破坏了!”

%102
蛊仙们心声似乎共通起来,都在脑海中咆哮。

%103
方源笑眯眯,看向乔丝柳:“不知道乔家仙子可还满意?”

%104
“满意个屁啊!”

%105
“这种问题你还好意思问?你这脸皮可真够厚的啊!!”

%106
罗木子、轮飞心中怒吼,但是碍于风度,表面上没有表现出什么,凉亭中仍旧是一片沉寂。

%107
“呵、呵、呵。”乔丝柳笑起来,怎么听都有一种很勉强的感觉,“这首诗真的很特别,说实话,我……我从未听过这样的诗。不愧是遗海你作出来的,嗯……如今细细品味,别有诙谐趣味,让我回味不绝呢。”

%108
罗木子:“……”

%109
轮飞:“……”

%110
备注:今天码字前看了上一张,昨晚感觉挺好,今天觉得还有改良之处,的确是后面的内容有些累赘。因此我删减了一部分,增添了新的内容,将这段剧情变得更加流畅。前文我已经修改上传,大家先回到前一张看,再看今天这一章,就不会有突兀的感觉了。稍后9点半的样子,还有第二更。今天这章的诗词,我用了两个多小时构思,诚意十足,大家有票票的话,就请投过来吧,谢谢了!

\end{this_body}

