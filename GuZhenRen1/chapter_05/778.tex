\newsection{勒索九转仙材}    %第七百八十一节:勒索九转仙材

\begin{this_body}

哗哗哗。

海浪声传入方源的耳中,还伴有海鸟清越的鸣叫声。

眼前的东海海面,在温暖的阳光下,闪闪起伏。

海风拂面,令人心旷神怡。

方源的心中却带着微微的失落:“果然是时机未到么……”

他运用翠流珠杀招,从南疆来到东海,目的就是为了寻找苍蓝龙鲸。

现在对于方源而言,掌握悔蛊是美妙的事情。

因为他要升炼大量的六转仙蛊,提升自己的综合实力。悔蛊在手,能减损他无数的损失,将炼蛊的风险降至最低!

事实上,重生以来,他就多次来到这里。

这片海域,便是他上一世跟随庙明神等人,见到苍蓝龙鲸的地点。

可惜方源一直没有等到苍蓝龙鲸。

“或许我应该对庙明神下手,直接抢夺了他掌握那个手段?”

“有了那个手段,我就可以自己寻找苍蓝龙鲸的具体位置了。”

这种想法在方源的脑海中一闪而过,他迅速评估此中的风险。

要动手的话,其实庙明神本身并不难对付,方源甚至还可以利用楚瀛的身份,和智道的手段来设计陷害他们。

真正的风险是,方源并不知道这个手段具体是什么。

若是寻常的仙道杀招,也就罢了。就怕是其他一些稀奇古怪的东西,贸然出手,恐怕弄巧成拙。

按照方源推算,后者的可能性还不小。因为五百年前世,也没听说有人从庙明神手中抢夺到这个手段。

东海也有八转蛊仙,庙明神不过七转,他的手段关乎到乐土真传,不可能不让八转蛊仙动心。

但这些蛊仙似乎都没有出手的欲望,已经说明了这里面的水还是比较深的。

“按照上一世的轨迹,我安心等待,也能跟随庙明神进入其中。”

“但此法也有一个弊端,那就是墨水效应。上一世庙明神选择楚瀛,是在犹豫之后,这一世他会不会仍旧选择楚瀛?”

方源利用翠流珠杀招,再次回返南疆。

目前,五域中南疆是最安全的。就算南疆正道来对付他,他手中也有大量的俘虏可以讲条件,并且他对南疆的情况是了解的最为透彻。

方源没有急着向庙明神下手,他的重生计划进行得很顺利,考虑得很周详,就算方源想要对庙明神下狠手,这个时候就动手,有点操之过急了。

什么事情都要讲究火候,要等待时机的成熟,一件件的事情一步步去做,有条不紊。

人生有时候需要快,有时候需要慢。

太快,火候过大,好东西都要被烧焦。太慢,火候小,始终热不上来,浪费时机。

快慢的火候,方源早已经掌握得很纯熟了。

接下来他的主要工作,还是分身!

他需要凝聚一具龙人分身。

白凝冰掌握的龙人变化之法,已经被方源得到,他早就已经开始研究此法。

此法名为人人如龙炼蛊法,就是蛊仙把自己当做一份蛊材,结合惊涛升龙火,一起炼制,最终改变种族,彻底成为龙人。

此法成功率不足百分之一,并且在炼蛊的过程中,蛊仙的全部道痕都会影响最终的结果。

方源初看时,为此法的奇思妙想感到惊叹。

随后,他就发现此法的不足之处。

此法乃是龙公所创,龙公擅长气道、变化道,在炼道上造诣并不深厚,多靠的是气道、变化道两大境界的触类旁通。

而方源却是炼道准无上,此法在炼道方面完全可以做一次巨大的提升。改良之后,成功率也会提升许多,甚至有可能达到百分之十。

也就是十分之一。

这个概率已经是方源改良的极限,再不能提升更多。

因为人人如龙炼蛊法,实质上是人道的杀招,涉及到变化道的奥义。

方源的变化道境界,比不上龙公。人道境界更是如此。

因此,人人如龙炼蛊法也是一份上佳的材料,能让方源参悟出人道的点点精髓。

“真正妨碍我的,还是惊涛升龙火!”

惊涛升龙火可是九转仙材,天地人三焰中的人焰,极其罕见。

白相洞天中原本有着一团,但已经被白凝冰用掉了。宝黄天中可是没有的。

在宝黄天中公开求购?

不,方源还有更好的选择!

夏家大本营。

大殿中气氛凝重压抑。

方源的勒索信还是第一个发到了夏家手中。

“方源贼子,罪大恶极,该死!该杀!”夏飞快低吼出声,满脸狰狞之色。

其余的夏家蛊仙亦是个个面色铁青,难看得很。

但除了怒火和仇恨之外,这些正道蛊仙也有一种如梦似幻的不真实感。

堂堂夏家,超级势力,居然会有这么一天,被一位魔道蛊仙公然勒索!

而勒索的依仗竟然是夏家的最强者,夏家的根基,夏家的掌权人——太上大长老夏槎!!

如果在埋伏战之前,有人这样告诉夏家蛊仙,夏家蛊仙们定会打人几个巴掌,笑骂这人:给我清醒一点。

但在这个时候,夏家蛊仙们只有打自己几个巴掌的冲动:这事情怎么落到现在这种地步?

身为夏家太上二长老的夏兆,是压力最大的人,他哀叹一声:“那只信道凡蛊你们都看了,方源要求我们拿出惊涛升龙火。诸位有什么想法,都说说看吧。”

夏家群仙相互望望,都不说话。

夏兆只好点名,看向夏飞快:“你来说,你刚刚不是吼得挺大声吗?说出你的意思吧。”

夏飞快性情急躁,对方源愤恨至极,但是此刻他却沉吟良久,方才道:“方源这恶贼连天庭的蛊仙都能干掉,虽然是七转修为,但实在是太穷凶极恶了。他若真的想要我族夏槎大人的性命,一定不会有什么顾及。或许我们可以联合其他家族?方源勒索我们,肯定也勒索其他家族!”

此话一出,夏流佩立即反驳:“我们失去的可是太上大长老,那些外人怎么靠得住?现在他们巴不得我们失去夏槎大人!既然方源肯定会勒索其他势力,但你可见什么势力联络我们了?”

“你!”夏飞快瞪了瞪眼睛,却说不出反驳的话来。半晌才道,“就算我们要妥协,但惊涛升龙火这种东西,我们可是没有的啊。”

“我族没有并不代表其他人没有。”

“或许方源那魔头只是想先抬一个高价,然后等我们协商?”

有夏家蛊仙陆续发言道。

和上一世一样,夏家的最大人物栽在了方源的手中,夏家蛊仙只有认栽,被方源勒索。

方源的回应很直接:“别跟我谈其他东西,我就要惊涛升龙火!我不管你们怎么弄,弄过来我们才能继续谈下去,否则夏槎的命?呵呵。”

方源比上一世更加从容。

上一世他勒索敲诈南疆正道,还有许多试探的成分。

这一世他知己知彼,熟知每一家蛊仙的底线,夏家被他吃得死死。

虽然这一世夏家和巴家同有八转蛊仙被俘虏,但夏家的情况仍旧是最糟糕的,因为巴家还有一位强力的八转候补——巴德!

所以,方源首先勒索的第一个对象仍旧是夏家。

夏家没有惊涛升龙火,也在方源的意料当中。因为他早就搜刮过夏槎的魂魄,并没有在库藏中发现什么。

若是其他家族的七转蛊仙,或许不知道全部库藏,但夏槎身为太上大长老定然是知之甚详的。

事实上,方源还真不知道南疆正道中,有哪家势力收藏了惊涛升龙火。

有可能哪家就收藏着,但九转仙材压在库存底部,非是太上大长老而不得知。

也有可能整个南疆正道都没有此火。

方源不管这些,这种头疼的问题,他准备让南疆正道替他头疼去。

在方源的计划中,龙人分身也同样是一件大事。

因为这个分身,关乎到龙宫的获取!

方源对龙宫的情报,其实知晓得并不清楚,但他至少推算出了龙人身份是继承龙宫的关键。

上一世,白凝冰成为龙宫之主,就是最好的例证。

方源一边敲诈勒索南疆正道,一边继续潜修,努力消化战果。

不久后,光阴飞刃杀招改良成功,全新的杀招威能比上一世更强三分!

这是因为方源的宙道仙蛊远比上一世更多。

不仅是因为吞并了琅琊福地,而且年流伏诛大阵也不像上一世遭受南疆群仙的重创,许多宙道仙蛊都保存了下来。

还有一点,方源连南疆正道库藏中的某些宙道仙蛊,都算了进去。

先是夏家,然后巴家,之后是翼家。

翼家的蛊仙翼扬,乃是翼家太上大长老的嫡亲血脉,是太上大长老刻意栽培的接班人,十分重要。

第四位则是池家。

对于池家,方源的态度很和蔼可亲:“池曲由老哥,咱们俩谁跟谁!这一次能够俘虏南疆群仙,还多亏了你。你家的太上二长老我就直接放回来给你罢。”

池曲由面色漆黑,宛若锅底!

“不要叫得这么亲热!”

池曲由心中极为膈应。

他发现自己上了方源的贼船,并且下不来了!

这些天来,他的心情都十分沉重。

因为他知道,方源此次能够俘虏南疆群仙,最关键的依靠就是梦境。而这些梦境就是他偷偷交易给方源的。

如果让南疆正道知道这个大秘密,不仅是池曲由,就连池家都危在旦夕!

梦境的交易已经成了池曲由的心头病,他现在极想停止这桩危险至极的买卖,但可能吗?

想到自己要面对的是方源这样的人物,池曲由心中冰冷一片。

------------

\end{this_body}

