\newsection{天上掉蛋}    %第八百零八节:天上掉蛋

\begin{this_body}

%1
巨阳仙尊的己运真传,已经将运气的种种奥妙,阐述得相当全面。

%2
总体上而言,运气分为黑、灰、白、赤、黄、青、紫这七种颜色。也有其他杂色,但基本上是这七种颜色的混合。

%3
运气的形态堪称千奇百怪,和具体的人,以及人物身处在具体的环境、本身的状态有关系。

%4
就拿方源自己而言,他和他的分身各个的运气形态、颜色都大相径庭。

%5
方源掌握了完整的巨阳己运真传,还有部分巨阳众生运真传,只要看一眼这些气运的颜色和形态,就能知道对象的状态,推算出许许多多的情报。

%6
煮运锅中有着察运仙蛊、时运仙蛊、狗屎运仙蛊、气运仙蛊、连运仙蛊等等,这些运道蛊虫的威能,只能算是煮运锅的基本功用。

%7
这座仙蛊屋最主要的威能就是——煮运。

%8
自身运气的形态、颜色、规模,代表着人物当下的改变趋势。

%9
假设方源如今有着浓厚的黑棺气运,预示着他面临死亡的趋势。方源就可以利用煮运锅,将这黑棺气运煮化,从锅中升腾其全新的气运。

%10
比如财运、桃花运等等。

%11
巨阳己运真传,就是研究自身运气,并且如何改变自己的气运。而煮运锅便是这道真传中的巅峰杰作,有了它就能煮化任何种类的运气,然后转变成自己想要的。

%12
“不过,我现在拥有八转修为,煮运锅不过六转,还不能对我本体的气运有什么影响。”

%13
“但我的其他分身,修为最高的只有七转,大多数还都是凡人。”

%14
“就拿刚刚煮运,支援过去来说,效果应当是立竿见影的。”

%15
方源揣摩了一番后,便又继续潜修,练习自己的气道杀招。

%16
煮运锅当然是要提升转数的,但如今悔蛊仍旧在苍蓝龙鲸的乐土洞天当中。

%17
方源可以命令毛民蛊仙们大肆炼制六转运道仙蛊,但把它们提炼到七转层次,还是缺少有利的条件。

%18
兽灾洞天。

%19
方源分身战部渡缓步行走。

%20
周围吵吵嚷嚷,人流如梭。

%21
“这里就是城中最大的蛊师市场了。倒是挺繁华。”战部渡细细观察。

%22
他现在修为不过一转而已,还是太年轻。

%23
并且囊中羞涩,此次前来这里,不过是采集情报,加深对周围环境的了解。

%24
“蛊师修行需要资源,我却是一穷二白。当务之急,就是生财。”

%25
战部渡一脸稚嫩,内心却是自信满满。

%26
他可是方源的分魂,有着蛊仙的眼界,想要寻找到一些凡人蛊师的生财之道,还不简单?

%27
但细心观察一阵后,战部渡的眉宇间却是笼罩了一层淡淡的苦闷。

%28
“这里的蛊师,都称之为战兽使。虽然也运用蛊虫,但这些蛊虫都是为了培养战兽,和野兽、植物合体。根本就没有蛊师单独作战的例子。”

%29
方源转了一圈市场,发现蛊虫很少,绝大多数的店面都是在贩卖兽植。

%30
“也是。”

%31
“这片兽灾洞天,有着万物大同变杀招遮护,蛊师要和兽植合体,极其容易。”

%32
“若是换做外界,要做到相同程度,非得付出十多倍的代价才有可能。”

%33
“因为和兽植合体的路线,太过容易,太占优势,导致正统的蛊修早已经埋没。”

%34
“当然,这也和兽灾仙人有意为之有关。”

%35
兽灾仙人因为第一次万劫伤重而亡,这片洞天过去是他精心经营,故意营造成这种情况,应当是利于管理吧。

%36
毕竟,兽灾洞天中的人族规模真的很大。

%37
战部渡要攻略这里,自然不能启用蛊修正统,而应当入乡随俗,也成为战兽使。

%38
成为战兽使,说容易挺容易,说难也难——需要蛊师拥有一只野兽或者植株,可以成功合体。

%39
这是最基本的。

%40
然后再到战兽公会,花钱注册,成为公会一员。

%41
兽灾洞天中,战兽公会是最大的也是唯一的超级势力。

%42
战部渡若是加入,会成为最低级的战兽学徒。学徒之上是战兽公会的中坚力量——战兽使者。

%43
战兽使者之上,就是战兽勇士,必须是蛊仙修为,往往担任城主之位。

%44
战兽勇士的上面,就是唯一的战兽王。

%45
组织结构相当的简单粗糙。

%46
战部渡如今不仅缺少相应的蛊虫,更缺少战兽、战植。尤其是后者,售价不菲。

%47
战部渡收集到了足够的情报,心中已有定计。

%48
“按照我的计划,大约一个多月,我就有足够的财力,买下蛊虫。”

%49
“再有一个月,就能收购到最低级的战兽了。”

%50
“先买下门牙鼠,用一段时间。这种野兽被世人大大低估,实在是物廉价美。”

%51
“有了实力,我就能加入战兽公会,以它为平台,接取任务,迅速扩充实力。”

%52
“嗯?!”

%53
就在这时,战部渡听到一声高喊:“小心头顶!!”

%54
他连忙看去,只见天空中一个漆黑之物,正急速地坠落下来。

%55
而更高一点的空中,有一位老者操纵着胯下的巨鸟,飞快下落,想要抓取那个漆黑之物,但是看样子已经来不及了。

%56
“什么东西?”

%57
“快跑啊!”

%58
周围人顿时一哄而散。

%59
战部渡也赶紧迈开步伐,躲闪到一处店家的屋檐下。

%60
砰!

%61
几乎下一刻,漆黑之物就砸在了这家店的台阶上,破碎开来。

%62
破碎的碎片四处飞溅,将周围人打得哀嚎不已。

%63
战部渡离得最近,却意外的一根毫毛都没有伤。

%64
“这是……一个蛋?”战部渡望着眼前一人高的蛋,心中了然,“这样的气息,明显是一个荒兽蛋啊。”

%65
正想着,蛋壳咔咔破碎,从中钻出一头可爱的小雕来。

%66
小雕看着战部渡,啾啾出声,一下子就扑倒了战部渡,用嫩嫩的鸟喙碰战部渡的脸颊。

%67
“怎么会这样?!”那位骑着巨鸟的老者,随即降落到地上,看着这样的一幕,有些傻眼。

%68
“老爷爷你好,我叫做战部渡。”战部渡有着见识,明白眼前的老者乃是一位蛊仙,不敢怠慢,立即起身行礼。

%69
老者还未说话,那头被战部渡手掌拨开一边的小雕,又扑到战部渡的身边,用双翼轻轻地拍打战部渡的后背。

%70
老者的目光顿时复杂无比,他看看这头小雕,又打量战部渡:“唉,我的老伙计怀孕了三十多年,竟于今天忽然产蛋。这颗蛋好巧不巧,落到这里来,孵化出了小箭尾雕。小雕第一眼看到的就是少年你啊,它把你当做了至亲的人了。”

%71
战部渡傻眼,连忙摆手:“老爷爷,很抱歉,我、我不是有意的。”

%72
蛊仙老者呵呵一笑:“你叫做战部渡?那我就叫你小渡好了。小渡啊,你别紧张,我可没有怪罪你的意思。甚至,我还要对你赔礼道歉呢,毕竟刚刚差点就要砸到你。”

%73
“看来这一切都是命运的安排啊。如果我的孙子还活着,也像你这么大呢。小渡啊,你愿不愿意和我一起修行呢?你得到了小箭尾雕的认可,将来说不定能成为战兽勇士哦。”

%74
“战兽勇士?”战部渡顿时双眼瞪大,闪闪发亮,一脸纯真和兴奋。

%75
他双手握拳:“我最大的梦想,就是成为战兽勇士了!老爷爷,我真的能行吗?我真的可以和你一起修行吗?”

%76
老者哈哈大笑:“小渡啊,你能不能成为战兽勇士,还看你今后的努力。但现在,我们还是先走吧。”

%77
“好的,老爷爷。”战部渡便随着老者,一起骑在箭尾雕身上,在众目睽睽之下,飞走了。

%78
“我的天!”

%79
“我刚刚看到了什么?”

%80
“这个少年运气也太好了吧?”

%81
“刚刚那位老者,好像就是山崖城的城主呢。”

%82
“他就是山崖城城主?哦!我是听说,最近山崖城城主要来我们城,和我们的城主商量什么大事的。”

%83
周围群众像是炸了锅一样,议论声越来越大。

%84
“那个少年是谁啊?好像是叫战部渡?运气怎么这么好!”

%85
“唉,刚刚怎么就不是我呢?”

%86
“早知道这样,我就把他挤出去了。”

%87
“我咧个去!战部渡衣着不堪,明显是个穷小子,没想到竟然被一位战兽勇士大人看中了啊。这是咸鱼翻身,一飞冲天啊。”

%88
“他还得到了小箭尾雕的认可呢!这可是仙兽,仙兽啊!”

%89
“是啊,和仙兽一比,整个市场中的兽植都是渣,连对方的鸟屎都比不上!”

%90
有的人气得跳脚,懊悔不已,刚刚为什么没有抓住这个千载难逢的机会?

%91
有的人瞪大双眼,眼珠子都红了,有的人则大吼大叫,吐沫飞溅。

%92
华文洞天。

%93
一场诗会正在举行着。

%94
著名的大才子望着堂下端坐的学子们,笑了笑:“柳镇果然是文风鼎盛,我看堂中学子皆是文气洋溢。看来姜贤弟教授得颇有成果啊。”

%95
“申兄谬赞,小弟的这些学生皆是才思浅薄,不值一提。此次能让他们来观看你我二人的文斗,已经是他们的造化了。”私塾的姜先生谦虚道。

%96
申大才子按手道:“姜贤弟,你我文斗不分上下,索性就此止住。接下来,不妨接鼓传花,选取三位学生上来献诗,也好让我领略一番后辈的文采,如何?”

%97
“也好。”姜先生沉吟一番,点头答应下来。

%98
顿时,台下的众多学生的眼睛都亮了起来。

%99
这可是千载难逢的好时机啊,有着台上的两大才子坐镇,不管吟诗如何,只要登场亮相,就能名传千里。

%100
鼓声开始了。

%101
“选我,选我!”

%102
“传过来,传到我这里来。”

%103
“唉!鼓声停了。”

%104
“是谁拿了红花?还请上来献诗一首。”申大才子睁开双眼,微笑道。

%105
在众人的灼灼目光中,李小白摸了摸鼻子,走上了台。

\end{this_body}

