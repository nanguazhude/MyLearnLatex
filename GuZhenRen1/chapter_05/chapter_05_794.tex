\newsection{恐怖兽潮}    %第七百九十七节:恐怖兽潮

\begin{this_body}



%1
光阴长河上掀起疾风恶浪。

%2
万年斗飞车不断射出破晓飞剑,宛若暴雨倾盆,始终笼罩在三秋黄鹤台上。

%3
三秋黄鹤台中,诸多中洲蛊仙涌起死志,纷纷对顾六如道:“大人,还请您先撤!”

%4
顾六如神色犹豫。

%5
这些人又道:“中洲可没有我等,但不可没有您啊。眼下的战局太过危险,一起撤退都要被那魔头留下。再者,天庭方面也传达了确切的命令!”

%6
紫薇仙子已然下令,叫三秋黄鹤台奋力阻挡万年斗飞车,让顾六如先撤。

%7
顾六如长叹一声:“不想事情竟至于此,诸位,你们绝不会白白牺牲!”

%8
他跳出三秋黄鹤台,直接遁入光阴长河之中,猛地遁走。

%9
万年斗飞车上,方源迅速察觉,立即掉转车头,但三秋黄鹤台奋不顾身,拼死拦截。

%10
方源被三秋黄鹤台缠住,无法迅速脱身,只得看到顾六如顺着光阴支流,回到五域两天中去。

%11
“既然如此,那你们就都死罢!”方源冷哼一声,把火力集中在三秋黄鹤台上。

%12
战到此时,三秋黄鹤台早已经是强弩之末,在万年斗飞车凶猛的连续撞击下,轰的一声,彻底沦为碎片。

%13
十多位蛊仙坠入河中,混乱不堪,周围的年兽嗜血围拢,对这些蛊仙下手。

%14
“畜生都滚!”方源大喝,驾驭万年斗飞车,企图抢回这些战利品。

%15
但是年兽太多了,兽潮一波波,猛烈冲击着万年斗飞车。

%16
就这么耽搁了一小会,许多中洲蛊仙已经被发狂的年兽群吞没,渣都不剩一丝。

%17
轰!

%18
一头巨大的太古年牛,宛若流星坠地,狠狠地撞到万年斗飞车的一侧。

%19
万年斗飞车差点倾翻过去,奋力挣脱之后,一片的船身已经完全变形,还有两个大洞。

%20
是牛角直接撞出来的!

%21
方源的麾下慌乱一团。

%22
“快修补!”

%23
“好多蛊虫都被摧毁了。”

%24
“这头太古年兽非同一般,实力超绝!”

%25
方源侧目,仔细打量。

%26
周围都是年兽,天生飞的,河面跑的,水里游的,都把万年斗飞车当做进攻的目标。

%27
就这一时间,就有八头太古年兽,或远或近,包围着方源。

%28
刚刚的那头太古年牛,皮毛顺滑,色泽金黄,四蹄踩踏河面而不成,稳步如山,混乱的年兽大军也不能撼动它!

%29
方源眼中精芒一闪即逝。

%30
十二生肖战阵中,尚且缺少一头太古年牛呢。

%31
并且就算之前有太古年牛,方源也会想换了。

%32
上古战阵的威能,和这些组并大阵的太古年兽的实力完全挂钩,这头太古年牛无疑是太古年兽中的佼佼者!

%33
“但眼下,却非是镇压收服它的时候啊。”方源苦涩一笑,立即下令,“撤!”

%34
万年斗飞车也开始了逃亡之路。

%35
但不管它驰骋到哪里,都有年兽挡路。

%36
汪汪汪!

%37
一头太古年狗纠缠上来。

%38
它浑身矫健,肌肉贲发,狗眼放电,皮毛同样是黄金色,油顺神俊。

%39
万年斗飞车不断催发的破晓剑,射中它的身上,立即崩散,只削短它的狗毛。

%40
汪!

%41
太古年狗猛地一跃,竟速度暴涨,逼近船首。

%42
它张开大嘴,露出满嘴如刀似枪的银色亮压。

%43
咔嚓一声,万年斗飞车直接被它咬碎尖端船首。

%44
“快走,此狗身上有宙道野生仙蛊。”方源面色冰寒。

%45
太古年狗也是他缺少的太古年兽,但在这种混乱的情况下,他根本不想和一个不明底细的强敌作战。

%46
好在年兽之间也存在竞争和打压。

%47
太古年狗只有一头,太古年猴数量最多。

%48
方源巧妙转折,专门冲向这些太古年猴,然后万年斗飞车宛若一条银色的闪电,在年兽的缝隙间电射腾挪。

%49
太古年狗被这些太古年猴拦截下来。

%50
就这样一路飞逃,终于来到一处光阴支流的出口。

%51
万年斗飞车挣扎着,冲出支流,回到五域两天中去。

%52
大量的年兽竟然紧追不舍。

%53
吱吱吱!

%54
一头小巧的太古年鼠,忽然现身,像是一道金色的闪电,嗖的一下,撞进万年斗飞车中去。

%55
“糟糕!”

%56
“方源快快出手!只有靠您了!”

%57
群仙脸色皆变。

%58
万年斗飞车外强内弱,若是被太古年鼠肆意破坏,搞不好整座仙蛊屋都要毁了。

%59
危机关头,方源也不顾上遮掩,直接爆发出八转的修为,挡在太古年鼠面前。

%60
真正八转层次的春剪杀招爆发,在狭小的空间中,灵活飞转,刀刀剪在太古年鼠的身上。

%61
这头太古年兽体格极小,只有成人的拳头大,网络一个黄金疙瘩,硬的不得了。

%62
春剪杀招剪在他的身上,发出铿锵之音,仿佛是斩在铁石上,爆出一阵阵的金黄火星。

%63
方源已然全力出手,但这头太古年鼠却怡然不惧。

%64
它一双眼睛贼亮贼亮,看准一只宙道仙蛊,猛地出嘴,一口叼住,转身就跑。

%65
它速度奇快,仿佛是金色的闪电,一眨眼就消失在方源的视野里。

%66
“方源大人,不得了了!八转似水流年仙蛊,被这头太古年鼠叼走了!”

%67
方源又好气又好笑,又有一丝庆幸,从容答道:“这是我的伪装,虽然是一只宙道仙蛊,却不是真正的似水流年。”

%68
方源亲自推算出的万年围猎杀招,怎可能不知道此招的巨大弊端?

%69
所以,他早就未雨绸缪,在这万年斗飞车中布置了伪装的诱饵。

%70
只待万年斗飞车破损到一定的阶段,就将这个诱饵抛出去,引发太古年兽之间的哄抢,从而创造出逃脱的机会。

%71
这个法子或许骗不了天庭蛊仙,但太古年兽智力有限,不如天庭的八转大能。

%72
果然,太古年鼠叼走诱饵后,年兽大军立即感知到一股强烈的诱惑,从万年斗飞车上转移到了这头小巧不已的黄金年鼠身上,立即发起了围攻。

%73
万年斗飞车这才真正脱离了这波恐怖的年兽兽潮。

%74
安全之后,方源第一时间将这座八转仙蛊屋放入仙窍中维修。

%75
整个万年斗飞车表面坑坑洼洼,尖端船首被啃掉了,船侧破了好几个大洞,触目惊心。还有好几道爪痕,好像是巨大的砍刀劈在木头上。

%76
检查一番,万年斗飞车中海量的凡蛊阵亡,仙蛊也有不少损伤,尤其是防备仙蛊,损伤最为严重,毕竟是承担了万年斗飞车防御的核心。就连似水流年仙蛊,都有一丝损伤。

%77
看上去,万年斗飞车还有骨架,还挺完整,其实再纠缠片刻,就要面目全非,伤亡惨重了。

%78
光阴长河乃是天地秘境之一,栖息着规模恐怖到无法揣度的生命群体。这些宙道生命当中,不乏强横之辈,就连方源都在短时间内拿它们没有办法。

%79
留给方源印象最深的,就是那三头皮毛黄金色的太古年兽。

%80
“不过,这也在我的算计之内。”

%81
“只要是攻击过万年斗飞车的太古年兽,都被标记起来。短时间内,我就可寻迹找上它们,将它们一一压服。”

%82
“要集齐十二生肖战阵,还是要冒险啊。”

%83
方源心中感叹。

%84
十二生肖战阵要收集到十二种不同种类的太古年兽,分别是鼠、牛、虎、兔、龙、蛇、马、羊、猴、鸡、狗、猪。

%85
光阴长河中,每隔一段时间,就有一种太古年兽昌盛起来,族群规模暴涨到第一位,成为最强族群,欺压其他族群。

%86
然后,在接下来的十几年里,这种年兽的族群规模从最强最大沦落到最弱最小,在生存中遭受其他年兽的重重排挤和打击。

%87
但因此存活下来的年兽往往在同级中,最为强悍。不强悍不足以生存下来。

%88
就像方源刚刚遭遇的太古年牛、太古年狗和太古年鼠,都是最好的例证。

%89
大部分的太古年兽,因为数量还多,比较容易收集。但有几种类别,因为整个族群都在最低谷徘徊,不仅稀少罕见,而且个体实力非常强悍。

%90
万年斗飞车的修复,并不是什么难题。

%91
凡蛊虽然损失极多,但方源管够。仙蛊的损伤最为麻烦,但方源也有蛊如故大阵!

%92
方源的炼道底蕴太雄厚了,修理这些仙蛊妥妥的。

%93
“只是这些中洲蛊仙的仙窍中,一穷二白、干干净净,看来紫薇仙子也学乖了么。”方源笑了笑。

%94
这些中洲七转的尸体,残留的仙窍都很荒芜,无数资源都在战前被刻意搬出去。自从光阴长河第一战后,紫薇仙子就唯恐这些资源滋养了方源这个大敌!

%95
方源也不以为意。

%96
对于他而言,这些资源只是锦上添花而已,可有可无。

%97
“接下来,先修复好万年斗飞车,随后就前往光阴长河,将缺少的太古年兽捕捉了。”

%98
“然后探索梦境,勒索南疆正道得来的梦境已经有一大堆了。”

%99
方源算了算时间,可能不够用了。

%100
不久后,就是五相公共洞天开启的时刻,千年赌约完成的事件。

%101
“还是尽力而为吧!”方源叹息。

%102
与此同时,天庭中的紫薇仙子也在苦叹。

%103
她忧心忡忡,天庭在光阴长河中再度惨败,这可如何是好?

%104
万年斗飞车暴露出的手段和战力,令她也感到为难不已。

%105
她知道方源的战略,就是用光阴长河这块地方钓鱼,可事关红莲真传,天庭不得不上钩!

%106
“就算是血坑,也得咬牙往下跳!天庭从不畏惧牺牲。”紫薇仙子眼中厉芒闪烁,旋即又眉头紧锁,“只是接下来,组建的宙道仙蛊屋当加以改良,速度为重。今后交手,也以干扰方源为主,只要阻止他得到红莲真传,便是大功一件!”

%107
ps:迟到了,我以为定时了的,不好意思。

\end{this_body}

