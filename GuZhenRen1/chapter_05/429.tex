\newsection{赊借仙元石}    %第四百三十节:赊借仙元石

\begin{this_body}

眼前地貌平缓,一望无垠。

赵怜云的神念不断扩散,最终将她的整个福地纵览无余。

不久前,她成功升仙,成为灵缘斋的智道六转蛊仙,成就了眼前这片上等福地。

怜云福地。

总计八百六十万亩,大型光阴支流,与外界时间流速为一比三十三,仙窍福地中每年可自行产出三十多颗青提仙元。

“一半的草原,一半的平原么。”赵怜云口中呢喃。

福地地貌和蛊仙个人具体情况有着关联。

草原地貌,可能是赵怜云心系北原,穿越以来,大半时间都是在北原渡过。至于平原地貌,则是可能继承了中洲这边的风情。

没有种植和栽培,此时的草原上已经是一片绿绿青青。而平原上,也是水土肥沃,稍加栽培,就会有丰富的物产。

除了福地之外,还有仙蛊。

九转爱情仙蛊就不说了,赵怜云因为上等福地升仙,还得到了第二只仙蛊——追念。

这是一只智道仙蛊,形如飞蚁,本身飞行速度惊人,通体白银之色,静止的时候,还会散发出一轮淡淡的虚幻白色光晕。

“鸿运啊,你可知道,我现在已经不再是凡人,而是成为了蛊仙!”

赵怜云抽回心神,缓缓睁开双眼,注意力回到外界。

她轻轻地叹了一口气。

成仙之后,虽有喜悦,但更多的是空虚和落寞。

逆流河一役,马鸿运虽然被方源夹死,但是他的魂魄尚在。

这不是赵怜云的胡乱猜测,而是灵缘斋智道蛊仙徐浩的推算结果。

赵怜云选择相信,并且这也成了她修行努力下去的动力和希望。

想到马鸿运还有一丝复生的希望,赵怜云便收敛起情绪,起了身,离开这个闭关的密室。

今天是一个重大的日子。

赵怜云走过一段廊道,女仙李君影已经等候着她。

“怜云见过君影姐姐。”赵怜云当即行礼。

李君影微笑道:“怜云妹妹不必客气,走吧,我陪着你前往秘传峰。”

秘传峰乃是灵缘斋蛊仙,经常涉足的一个地方。

赵怜云还是第一次涉足,果然便见此峰不同凡响。它悬空而浮,周围虚空渺渺,似近实远,山峰上气息厚重,引而不发,显然是铺设了重大的仙级蛊阵。

掌管此峰的乃是灵缘斋的中立派太上长老,人称流芳仙姑。

赵怜云此行却未见着流芳仙姑本人,而是受她留下的一段意志招待。

赵怜云自己没有觉察不妥,但李君影却是微微皱起眉头。

按照常理而言,赵怜云身为灵缘斋的当代仙子,会被门派着重培养,第一次进入秘传峰,领取门派补助,是会受到秘传峰主的亲自招待的。

但流芳仙姑却是没有按照这个惯例。

“晚辈怜云拜见流芳前辈。”赵怜云向流芳意志行礼道。

“不必多礼,你可是我派当代仙子呢,还请细看此蛊。”流芳意志笑眯眯的样子,态度上十分客气,递给赵怜云一只信道凡蛊。

赵怜云接过一览,里面琳琅满目,列举颇多,让她看了,一时间不免有眼花缭乱之感。

“怜云眼界浅薄,还请君影姐姐为我把关。”赵怜云是个聪明人,连忙又将此蛊递给身旁的李君影。

李君影接过一看,微微点头。

这只信道凡蛊中的列举条目,相较以往,并无克扣和削减,乃是实打实的。

她当即传音赵怜云道:“怜云,你既成蛊仙,有了仙蛊爱情和追念,还有上等福地,起点已然极高。”

“一旦成为蛊仙,就要经营仙窍,积累底蕴。首当其冲的任务,便是喂养仙蛊。所幸你那爱情仙蛊,虽是高达九转,却没有喂养上的难题。”

“所以,如今所要考虑的,便是追念仙蛊。而要喂养此蛊,食材便是亮睛花。灵缘斋的福利中,正好有一项亮睛花的培育法门,你当择之。”

赵怜云连连点头,传音回道:“姐姐说的极是。那另外两项呢?”

成为灵缘斋的当代仙子,自然有着非同一般的福利。

按照门派的规矩,一旦当代仙子的修为有了突破性的进展,就可以从秘传峰中,选择三项资源,无偿获取。

而其他蛊仙,除了立下大功之外,只能是用自身的门派贡献,来兑换这些资源了。

李君影微微一笑,又道:“除此之外,我建议你选择胜算草。你既然是智道蛊仙,自然要从智道方面经营下去。胜算草不仅是智道资源,而且培育较为容易,宝黄天的市场上又有上佳的行情,选择这一项,你绝不会亏。”

“那就选择这一项,姐姐的选择绝不会错的。”赵怜云连忙道。

李君影对赵怜云的态度,大感满意,语气不由地又柔和了几分。

“那么接下来的最后一项,我该选择什么?”赵怜云再次问道。

“自然是第一项。”

“仙元石?”

“不错。”李君影见赵怜云神色疑惑,不由轻笑一声,“不管是亮睛花还是胜算草,门派提供的规模,绝不够你大面积种植。这种资源,若不大面积种植的话,就失去了意义。所以,你需要大量的仙元石,收购更多的种子,进行栽培。”

“仙元石乃是蛊仙必须储备的重要物资。因为它不仅是蛊仙交易中最普遍的货币,而且能够转化成仙元,供你耗用,最为保值不过了。”

“原来如此。那我就选择仙元石了。”赵怜云没有一丝迟疑,当即向流芳意志,换取了以上三种。

两人出了秘传峰,飞回住处。

临别前,李君影又取出一批仙元石,交给赵怜云:“怜云妹妹,这是我和你姐夫徐浩借于你的仙元石,共三千块。你尽管利用,因为它借期无限。”

“啊!”赵怜云轻呼一声,流露出感动的神色,“门派供给我的,也不过一千仙元石。姐姐你却直接借我三千,这如何使得?”

李君影拍拍赵怜云的肩膀:“傻妮子,门派供给你的已经很多了。你是门派中人,有着依靠。你可知道,那些散修魔仙,都是扣着一块块的仙元石过日子。”

“要论仙元储备,六转蛊仙刚刚升仙,几乎是一穷二白。有上百块仙元石储备的六转蛊仙,已经是中等层次。上千块仙元石储备,六转蛊仙中几乎没有,非常罕见。”

“你的情况不同,因为你不仅是中洲十大古派的蛊仙,而且更是灵缘斋当代的仙子,受到门派的重点支持,乃是特例。”

“要论正常情况,五域两天中,大部分的六转蛊仙手中仙元石不过百,少部分有数百,极少的个例,能有上前仙元石。”

“而七转蛊仙当中,至少会有数百块仙元石,大多数破千,极少数上万。”

“这些常识,你知道就可以了。”

“多谢姐姐指教!”赵怜云郑重一礼,满脸感激之色。

李君影的话,是告诉她,门派没有亏待过你,同时我们夫妇也在大力地支持你。

“我一定好好努力。”赵怜云正色道。

李君影点点头:“去修行罢,若有什么疑问,不妨来信问我,问你姐夫徐浩也可。”

“是。”赵怜云眼中闪过一抹喜色,有人指点和无人指点,当然是两种概念了。

望着赵怜云离去的背影,李君影脸上的笑容却是渐渐隐去。

自从凤金煌拜在了龙公门下,她的日子就开始不好过了。

她和夫君徐浩,可是著名的倒凤派系。但是现在,灵缘斋的太上长老们几乎都心里清楚,凤金煌就是仙尊种子,未来前景无可限量!

这一次,流芳仙姑的态度,就可明显地看出来,灵缘斋的政治风向。

李君影也料到此节,特意陪伴赵怜云前往,就是为了防止有人克扣赵怜云的门派福利。所幸这样的事情,并没有发生。

可以预料,对于他们夫妇二人,将会迎来政治上的寒冬。而在这股寒风之下,他们只能团结一切可以团结的人,比如赵怜云,然后抱团取暖。

北原,楚家。

著名的力道蛊仙强者楚度,放下手中的信道蛊虫。

“方源啊,和你相处,还真是不容易呢。”楚度苦笑一声。

这只信道蛊虫,正是方源主动来信,要想他借取仙元石!

方源曾经用柳贯一的假名,和楚度合作一段时间,双方渊源很是不浅。不久前,天庭方面出手,暗杀了蛊仙酝良,破坏了长生天的政治图谋后,同时又公布了方源和柳贯一就是同一人的秘密,来栽赃嫁祸给方源。

所以,楚度此时已然知晓,柳贯一便是方源。

不过,霸仙楚度并不有多吃惊,事实上,他曾经怀疑过这方面,甚至还唱诗一首,特意试探过方源。

不管是柳贯一,还是方源,楚度都想和对方合作。

但是现在楚度的身份,却是今非昔比。

他如今乃是楚家的太上大家老,门下蛊仙众多,已然是正道的一员。

作为北原正道蛊仙,他若是和方源继续合作下去,一旦曝光,他之前在势力方面的所有努力,都将化为泡影。

和方源合作,风险极其巨大。

原本楚度还打算置之不理,保持模糊态度,但方源却没有给他这个机会,直接将军,来了一封借取仙元石的信。

而偏偏在信中,方源只说借仙元石,却故意不说具体数目,意思是让楚度看着给。

这就更让楚度感到棘手!

“不愧是方源呐。”楚度感慨不已。

\end{this_body}

