\newsection{东海追逃战(上)}    %第二十五节:东海追逃战(上)

\begin{this_body}

%1
方源气息只是六转蛊仙,在他没有出手之前,并无非凡之处。

%2
再加上两番避让,不由地让东海蛊仙们心生轻视之情。

%3
但当他暴然出手之后,顿时让人心头一震。

%4
东海众仙惊诧,来势顿滞,无不重新打量方源。

%5
但见方源一身白衣,长发在狂风中飘扬。飘逸绝伦,英俊多姿,别有一番气度。此时方源满脸怒色,宛如雷霆森然,杀意好似冰霜凝聚,让人感到分外压抑。

%6
东海蛊仙人多势众,此刻却沉默下来。

%7
他们不约而同地掉转枪头,对准血道魔仙。

%8
这血道魔仙想要逃跑,但似乎身上的伤势发作,速度陡然下降,几个回合之后,就被围攻的蛊仙们杀死。

%9
以此看来,他的确是强弩之末,难怪他不惜让出机缘,想要从方源身上祈求那么一丝生机。

%10
杀了血道魔仙,东海蛊仙中的一位七转领头人刘青玉,却是脸色微变,沉声道:“印记不在他的身上。”

%11
另一位七转蛊仙顿时反应过来,转头对方源轻喝道:“这位仙友且慢!”

%12
方源身怀两只七转仙蛊,已经让众仙心中的印象改变,再没有人如刚刚那般叫嚣。

%13
他们彼此互望,随后一同飞行,向方源徐徐压来。

%14
方源停在半空,冷笑不已。

%15
“这位仙友……”说话的七转蛊仙停顿下来。他说了一半,忽然发现很难启齿。

%16
开启传承的印记,至关重要,现在找不到了。

%17
最大的嫌疑,就是方源。

%18
但若要方源让他们搜身。对方肯定不愿意。更别提察看他人仙窍,本身就是一件犯忌讳的事情。

%19
若是方源只是普通六转蛊仙,这话仗着人多势众,说了也无所谓。

%20
但现在方源却是坐拥两只七转仙蛊,并且这还只是对方表现出来的东西而已。

%21
“周礼,你怕什么?”第三位位七转蛊仙,长袖一拂。跨步而出。

%22
他面容冷酷。目光尖锐,直接盯着方源,道:“鄙人乃是东海汤家蛊仙汤诵,今天这个事情难以善了。不过我有一法,只需你照做,就可证明自身清白。”

%23
“哈哈哈!”方源蓦地大笑,“证明自身清白?我为什么要证明自身清白?”

%24
笑声中。战意四溢。

%25
“看来诸位是认为,我是一个惧战之人了。”方源声音低缓。

%26
东海蛊仙脸色皆微微一变。

%27
方源眯起双眼:“可笑你们被他人当面欺瞒,却还不知。刘青玉,你亲手杀了那血道魔仙,分明获得了传承印记,却来诬陷我,也算你有点计谋了。”

%28
“这……”众仙气势再次一滞。

%29
许多人纷纷看向七转蛊仙刘青玉。

%30
刘青玉望向方源,目蕴怒意,暗中却道一声:“好厉害!此人言辞锋利,明明六转蛊仙。却有两只七转仙蛊。究竟是何方神圣?东海蛊仙界何时出了这样人物?我却居然不知晓!”

%31
同时,他口中高呼:“诸位勿忘了我们定下的约定,我刘青玉是什么样的人?怎么可能欺瞒诸位?”

%32
方源呵呵一笑:“什么约定?我只知道不管是什么信道盟约,都可以用信道手段驱除。什么样的人?我只知道人都是会变的,重利当前,知人知面不知心啊。”

%33
方源句句诛心,令东海蛊仙不禁更加迟疑。

%34
刘青玉大怒。手指着方源:“我明明看到,他将传承印记抛给你了!”

%35
“那其他人也看到了,我亲手斩了印记。”方源迅速接道。

%36
“呵呵呵。”刘青玉阴测测地笑起来,“世间障眼法太多了,谁知道你是不是用了什么手段,暗中将印记接下。”

%37
方源仰头长叹一声,语气萧索:“我有太上大长老的要务在身,本不愿惹麻烦,但麻烦却主动撞来。既然如此,那我们就做过一场罢。”

%38
听到太上大长老几个字,许多蛊仙无不瞳孔微缩,暗想:原来此人并非孤家寡人,而是背靠超级势力。难怪他有两只七转仙蛊!

%39
一时间,心中忌惮之意更深一分。

%40
尤其是当中的散修,独来独往,身单势薄,自然是不愿意得罪一个庞大势力的。

%41
“慢来。”这时,汤诵再次站了出来,“我以汤家名誉保证,只要阁下配合我们检查一番,只要查明传承印记不在你的身上,我们必定放行,绝不阻拦。”

%42
汤家是东海的超级势力,地位相当于中洲的十大古派之一。

%43
“汤家?”方源冷笑一声,眉头微挑,银光璀璨的飞剑仙蛊,在他身边旋绕,“好了不起么?搬出汤家,难道我族会惧怕你?呵呵,既然如此,那我就试试你汤家的手段!”

%44
话音未落,方源催动剑遁,竟反向七转蛊仙汤诵攻去。

%45
“你!”汤诵未料到方源竟然不买账,一时间猝不及防,被方源一阵快攻,落入下风。

%46
其余的东海蛊仙则一哄而散,作壁上观。

%47
“此人毫不惧怕汤家,来头肯定不小。”

%48
“汤家虽然在东海堪称霸主,但和它不对付的,还有沈家、苏家。难道此人就是这两家中人?”

%49
“他既然挑上汤诵,这是好事。我抽身事外,不妨耐心看看这两位的手段。就算接下来动手,也有所准备。”

%50
这群东海蛊仙根本就人心不齐,之前众志成城,是因为要追捕血道魔仙。

%51
方源深谙人心,一番言辞和举措,就在无形当中将众仙分化。

%52
他挑选出来的对手,也很有讲究。

%53
若挑选六转蛊仙。只会让人觉得他是仗着两只七转仙蛊欺负人。唯有挑选七转蛊仙,才有威慑力。

%54
至于为什么挑选汤诵,正是因为方源察言观色,知晓他是三位七转蛊仙之中,唯一一位身后有庞大势力的蛊仙。

%55
散修蛊仙不愿招惹超级势力。超级势力通常也因为家大业大,顾忌散仙之流。

%56
所以,汤诵身处在组织当中,行事说话,就会平添许多顾忌。皆因他代笔的不是自己,还有他背后的超级势力汤家。

%57
双方你来我往,交手不断。

%58
十多个回合之后。汤诵仍旧处于下风。被方源杀得一身冷汗。

%59
原来,这飞剑仙蛊着实锋锐,汤诵乃是音道蛊仙,防御手段稍显不足,不能硬抗飞剑仙蛊,只能四下闪避。

%60
他又惊又怒。

%61
堂堂一位七转蛊仙,居然被一位六转。逼迫至此,脸皮都要丢尽了。

%62
汤诵面沉如水,暗想:“要挽回脸面,非得施用那个仙道杀招了。只要我生擒活捉了这个六转蛊仙,之前落入下风的表现,也会被人认作故意示弱的战术。”

%63
只是这仙道杀招,并不容易催动,统共包含七七四百九十只蛊虫。

%64
汤诵要应付方源的锋锐攻势,只有慢慢挤出一些心神,一只只调动仙窍内的蛊虫。然后在慢慢组成杀招。

%65
“这飞剑仙蛊,虽然对付泥怪、云兽不行,但对付蛊仙,却是异常好用。一旦刺中汤诵的头脑或者心脏,他就会立死当场。当初剑仙薄青横扫中洲,号称九转之下第一人,由这飞剑仙蛊就可见一斑。”

%66
方源心中暗暗感慨。

%67
他虽然占据上风。但一刻都未曾大意。

%68
汤诵乃是七转蛊仙,底蕴肯定比方源深厚。只是失了先手,被方源一阵抢攻而已。

%69
此时局面,非常危险。越往后拖延,方源就越加难以脱身。

%70
关键,他现在想要动用暗歧杀,也不行。

%71
因为他必须时刻保持对汤诵的压力,方源先下手为强,一直逼压他。若让他喘过气来,从容施展手段,恐怕就不是这般局面。

%72
所以,方源得催动飞剑仙蛊不停。飞剑仙蛊乃是暗歧杀的核心,需要和暗渡仙蛊搭配来用。

%73
酝酿一段时间之后,结合其他辅助凡蛊,才能催动暗歧杀。

%74
“仙蛊稀少,若是手头上再有几只仙蛊的话……”就在这时,方源心有所感,手上动作一缓,回望身后。

%75
那群上古云兽,已经出现在他的视野边际处。

%76
再次见到这群上古云兽,方源却没有厌恶之情了,反而充盈着喜悦。

%77
他苦等已久的变数,终于来了!

%78
“怎么回事?这些是什么鬼东西,气息竟然如此强盛。”

%79
“那是云兽,我的天,好多的云兽!”

%80
“这是上古云兽,怎么会忽然出现这么多?!”

%81
东海蛊仙们也相继发现了上古云兽,惊疑不定。

%82
汤诵的主要注意力,却是集中在方源的身上,见方源攻势松懈,他顿时大喜,仙窍中酝酿的仙道杀招,顿时进程加速,短短功夫,已经催动了三分之一。

%83
“就快了!待我用出这招,将你生擒活捉,看我如何整治你。”汤诵暗自咬牙切齿。

%84
方源高呼:“哼!这些上古云兽乃是我出入白天,亲自引下来了。就是为了配合家族的几位太上长老,将这些上古云兽引入陷阱之中,一举围猎。尔等阻我于此,坏我族大事,定会找你们一一清算。现在想来找死的,就跟过来罢!”

%85
说着,方源舍掉汤诵,一飞冲天而去。

%86
众仙讶异无比,很快又释然。

%87
方源身上的伤势,很明显就是上古云兽造成的。

%88
上古云兽又紧追着方源不放。

%89
这一切似乎都佐证了方源的话。

%90
毕竟,普通的云兽就已经十分稀罕,这一大群的上古云兽,恐怕真的来自白天。

%91
ps:最近和妻子闹矛盾,唉,一言难尽。心情很不开朗,昨日苦闷至极,枯守在电脑前几个小时,只憋出了几百字。其实她说的话,也有道理,我除了工作,就是写小说,一天里很少有时间陪她和小孩。关键是写小说,收入寥寥,让她不能理解。自己的梦,旁人又如何能真正的理解呢?

\end{this_body}

