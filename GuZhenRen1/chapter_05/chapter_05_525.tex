\newsection{方谋在前,九歌身危}    %第五百二十七节:方谋在前,九歌身危

\begin{this_body}

%1
轰!

%2
一声巨响,黑毛大陆上又一座城池崩塌,毛民瞬间伤亡无数。

%3
“凤九歌你该死!”琅琊地灵发出狂躁的怒吼,吼声中夹杂着一股痛彻心扉的悲意。

%4
凤九歌冷笑,再次飞射而去。

%5
银色巨人在琅琊地灵的驱使下,不断狂追。临走前洒下一片片的华光,那些城池废墟当中,受伤未死的普通毛民,都在华光中受益,伤势迅速转好,保存性命。

%6
但这样是对战局无用的。

%7
银色巨人有八转战力,但是凤九歌却同样如此,两者都阻挡不了对方攻杀。

%8
除非是双方之间的战力,有了倍差。

%9
“方源还没有回应吗?”琅琊地灵双目通红,喝问身边。

%10
负责联络的几位毛民蛊仙纷纷摇头。

%11
“可恶!”琅琊地灵把牙齿咬得嘎吱作响,双手捏拳,抓狂万分,“关键时刻,他就用不上了!难道他是想故意逃避?不,不对!他不是这种缺乏勇气的人,只要他和我联手,中洲的这些蛊仙都能留下性命。这么说来……恐怕方源那边也遭受了天庭的打击!”

%12
琅琊地灵虽然暴怒狂躁,但是理智仍旧是有的。

%13
“太上大长老,我们该怎么办?就这样追逐那凤九歌,可不行啊!”毛六开口问道。

%14
毛民蛊仙们可不是琅琊地灵,虽然他们出生、成长在这里,但是成仙之后,观念自然转变,有了仙凡之分。

%15
这些普通毛民死了也就死了,顶多之后在花费时间培养便好了嘛。但若是任由那些中洲蛊仙自由布阵,整个琅琊福地都会有大难。

%16
这些毛民蛊仙也想劝说琅琊地灵,但是胆气不足,唯有毛六身份特殊,乃是幽魂分魂,此时关键时刻,他知道自己必须首先站出来。

%17
出乎其他毛民蛊仙的意料,琅琊地灵听了毛六的建议,却没有更加暴躁或者愤怒。

%18
他反而狰狞地笑了:“你们放心!我怎么可能把那些中洲蛊仙忘了?没有方源,我们还有其他的援兵啊,哈哈哈!这凤九歌自以为得计,其实不过是垂死挣扎,这些人都得要死!都要死!!”

%19
“难道说?”毛六听了这话,顿时双眼骤亮。

%20
其他毛民蛊仙也反应过来。

%21
“对啊,我们不是孤家寡人啊。”

%22
“不错,不错!我们琅琊派参与了四族大联盟,中洲蛊仙有援手,我们也有啊!”

%23
当年,方源促成的四族大联盟,其实早就为今天这样的情景做了考虑。

%24
琅琊福地的重要性,方源心知肚明,促成的联盟便是为琅琊福地的安全提供一层有力的保障。

%25
琅琊地灵以及诸多毛民蛊仙,被凤九歌牵着鼻子走。琅琊派的力量都集中在了银色巨人身上。

%26
中洲蛊仙自以为安然无恙了,但实际上,琅琊一方还有许多力量可以出动。

%27
事实上,来自墨人、石人、雪民三族的援军,已经来到,只是没有现身罢了。

%28
“终于要到我们出手的时候了吗?”七转律道石人蛊仙石兇接到了琅琊地灵传来的讯息,口中呢喃。

%29
“动手吧!还等什么?”七转雪民蛊仙冰风有些按捺不住。

%30
“真是遗憾,这场战斗不能让我亲自出战。”另一位七转近战雪民蛊仙冰卓,语气遗憾。

%31
“有如此大阵,何须我们动手呢?”来自墨人城的墨人蛊仙哈哈一笑。

%32
除去他们四位七转,还有不少的六转蛊仙。

%33
琅琊派受敌,按照当初四族联盟的约定,其余三族必定要守望相助,团结一体,共抗来敌。若是违反,惩罚力度极其严重。

%34
方源更是定下盟约的时候,强调了支援的速度,没有放过任何一个细节。

%35
不过这三族来援,并非逼迫,也是心甘情愿。这些异人对人族当然都没有好感,唯一比较大的顾虑是自家蛊仙战死。

%36
但是琅琊福地隐藏着的仙阵,却是完美地解除了他们的这番顾虑。

%37
自从八转仙蛊屋炼炉被影宗攻破,核心仙蛊被夺走后,琅琊福地只能以残炉勉强维持场面。

%38
琅琊福地的防御力量大大减弱,方源智者千虑,早就注意到这一点。许久之前,拥有阵道境界的他,和琅琊地灵互换传承,里面自然有着大量的仙阵,可以替代炼炉,防卫福地。

%39
琅琊派并不缺仙蛊,方源便改造其中仙阵,暗中布置了一座超级蛊阵,潜藏在整个琅琊福地当中。平时的时候,隐藏不动,唯有等到强敌来犯,蛊仙们便能发动起来,引出浩瀚威能!

%40
这一点,当然也是方源故意为之。

%41
他深思熟虑过,考虑到毛民蛊仙独自的战斗力,又考虑到来援蛊仙的心理,布置仙阵是最佳的选择。

%42
而且还有一点,就是仙阵布置下去后,不能随意转移。琅琊福地的实力虽然强大了,但将来若有个万一,要对付方源,这个仙阵是万万无效的。甚至这座仙阵中,还被方源留下了一些暗门……

%43
这些异人蛊仙援兵来到这里后,就被琅琊地灵安排到了仙阵当中。熟悉仙阵,花费了他们一些时间,现在他们一齐出手,灌输仙元,令这座超级大阵开动起来。

%44
“怎么回事?!”

%45
“这、这里居然还潜藏着超级蛊阵?”

%46
“不可能啊,琅琊一方怎么还有人手?”

%47
仙阵一发,威势沛然,不可阻挡,中洲蛊仙们刚刚布置起来的仙阵,立即崩溃毁灭,令当中许多人都身受重伤。

%48
冰卓等人趁势痛下杀手,毫不留情,仙阵发出无形力量,浩瀚恢弘,宛若巨磨,碾磨之下,旋即连坏了两位中洲蛊仙的性命。

%49
凤九歌勃然变色。

%50
“好一个琅琊地灵,我竟中了你的计!”

%51
残酷的事实,像是一个巴掌,狠狠地打在凤九歌的脸上。

%52
这位当代的传奇人物,也在脸上浮现出了怒意。

%53
这样的情形之下,凤九歌就不能持续之前的战术了,他立即赶回去,企图将剩下的中洲蛊仙救下。

%54
但是银色巨人死死缠住他,整个福地中还有一座超级蛊阵,对他施加影响。

%55
凤九歌拼尽全力,也只能眼睁睁地看着最后一位中洲蛊仙,身死道消。

%56
“情报大大有误!琅琊福地中,竟然潜藏了如此巨大的实力,这简直难以置信!”凤九歌面沉如水,受到银色巨人还有超级蛊阵的夹攻。

%57
仙道杀招碧玉歌!

%58
这样的情况下,只剩下凤九歌一人,他终于豁出全力,开始施展出压箱底的手段。

%59
碧玉歌歌声叮当清脆,宛若大小珠子落在玉盘当中。

%60
歌声传播开来,银色巨人身上立即蒙上一层绿意,旋即表面就转化作一层厚实的玉。

%61
歌声作响,所到之处,白云化玉,黄土化玉,流水凝滞也化为玉。一切都转变成了玉石,方圆上百里,眨眼间成为了玉石的静止世界。

%62
就连银色巨人也被冻住,变成一个玉石巨像。

%63
“这碧玉歌果然不同凡俗。”从玉石巨像中,传出琅琊地灵的声音。

%64
旋即咔嚓咔嚓,玉石巨像表面裂纹骤起,满布全身后,轰然崩散,银色巨人挣脱束缚,再次杀向凤九歌。

%65
凤九歌也不惊异,这一招碧玉歌有八转的恢弘威能,但是对付银色巨人,却只能做到暂时阻碍的作用。

%66
不过他已经争取到了足够的时间,施展出第二记仙道杀招俯首歌!

%67
此招涵盖了智道、奴道、音道、魂道四大方面,影响人心。由凤九歌亲创,从五指拳心剑得到的灵感。它能暂时性的让荒兽、上古荒兽改变阵营,成为歌唱者的麾下。

%68
歌声迅速传播,传遍整个琅琊福地。

%69
但是没有用!

%70
琅琊地灵哈哈大笑:“凤九歌,你以为我们会不做准备吗?你的俯首歌,方源早就将情报告知于我。趁着刚刚交战,我们已经将福地中所有的荒兽、上古荒兽荒植,都收入自家仙窍当中了。”

%71
知己知彼,方能百战不殆。

%72
天庭能刺探琅琊福地的情报,琅琊福地自然也会收集情报。其中凤九歌,更是方源着重叮嘱琅琊地灵,让他严加防备的对象!

%73
凤九歌深呼吸一口气,施展仙道杀招天地歌!

%74
歌声嘹亮恢弘,勾动天地,引导天地威能,能够镇压心志,削弱敌方强烈攻势。

%75
但此招威力更弱。

%76
这片天地乃是琅琊福地,是敌人的大本营,对凤九歌有着深沉的敌意,怎么可能会为他所用?

%77
仙道杀招离歌!

%78
凤九歌左右闪躲之际,使出王牌。

%79
此歌非同凡响,乃是破除仙阵、仙蛊屋的上佳手段。当初武庸的八转仙蛊屋玉清滴风小竹楼,就曾经险遭此招拆分,令武庸大惊失色!

%80
银色巨人乃是上古战阵,冰卓等人操纵的则是当今仙阵,都是仙道蛊阵,自然要受此招克制。

%81
一旦被分解,这些异人蛊仙单打独斗,怎可能是凤九歌的对手?

%82
凤九歌的离歌,原本发动时,只能一动不动。但他在这段时间中,也是进步良多,此刻催发离歌,完全能够自由移动,不断闪躲。

%83
此歌一出,起先非常微弱,像是低吟,又仿佛是在安静的夜晚的丝丝梦呓。

%84
然后,一股股油然而生的情感,从在场所有的异人蛊仙的心头升腾起来。

%85
歌声渐渐上扬,但并非一飞冲天,而是寰转柔婉,一圈一圈,回环往复,交替上升。

%86
蛊仙们心中无不由戚戚、悲凉、哀痛之感。

%87
歌声忽然又低垂下来,无以伦比的抑郁和伤感,袭击异人蛊仙们的,让他们都有一种落泪的冲动!

%88
离别的苦楚,离别的悲伤,离别的抑郁,离别的不舍。

%89
和情人的分手,和朋友的再不见,和亲人的生死绝别。

%90
别离,往事依旧。

%91
别离,故人挥手。

%92
别离,夕阳映映。

%93
别离,落红亦悲愁。

%94
“我别离你老母啊!!!”忽然,琅琊地灵的咆哮声,如雷霆般陡然炸响,响彻云霄。

%95
此时此刻,他的仙道杀招也酝酿完毕。

%96
仙道杀招万籁俱寂!

%97
银色巨人骤然绽射出无数的光辉。

%98
银白色的光芒刺眼至极,仿佛太阳碎片。照射到哪里,哪里的歌声就消散。

%99
凤九歌的脸上,终于显现出一抹震惊之意。

%100
此招有八转威能,正将他克得死死!

%101
“凤九歌,明年的今日就是你的忌日了!”琅琊地灵得意大吼,银色巨人狠狠地扑向凤九歌。

%102
凤九歌爆退,一时间只能躲闪。

%103
他的目光极速闪烁,琅琊一方的战斗素养,让他三番两次都吃惊不已。

%104
这招万籁俱寂,乃是大杀器,琅琊地灵故意留在最后面,才使出来。这一招太关键了。一下子让凤九歌陷入绝对的下风。

%105
这种中洲当代的传奇人物,终于感受到了一股浓郁的死亡气息!

%106
ps:从今天起每天一更,要放慢一下速度,这一段要好好写,直到这段剧情结束。

\end{this_body}

