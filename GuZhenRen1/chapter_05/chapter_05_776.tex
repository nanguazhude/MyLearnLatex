\newsection{蛊如故大阵}    %第七百七十九节:蛊如故大阵

\begin{this_body}

%1
天庭。

%2
组建恒舟的工作,已经到达最关键的时刻。

%3
天庭八转律道蛊仙从严,取来一只仙蛊。

%4
这只仙蛊宛若蜈蚣,长达一前臂,通体橙金之色,甲背上天然生长着无数颗细碎的小钻,紧密挨连,外形极为绚烂华贵。

%5
正是律道仙蛊——恒。

%6
此蛊原本用于天庭的恒沙洞,现在为了组建恒舟,抽调而出。

%7
“最后一步。”从严深呼吸一口气,满脸凝重之色,小心翼翼地运用手法,将恒蛊送进恒舟仙蛊屋雏形中去。

%8
仙蛊屋雏形首先是吸纳,随后气息澎湃不定,起起伏伏,迅速滑落失控的深渊。

%9
从严脸色大变:“不好!”

%10
轰。

%11
下一刻,一声雷炸,仙蛊屋雏形崩溃,海量的凡蛊当即毁灭,辅助的仙蛊亦损伤不少。

%12
唯有恒蛊,被从严牢牢护住,哪怕口吐鲜血,伤势颇重。

%13
恒蛊是核心,损伤不得。不像其他辅助仙蛊,就算毁了,也有替代之物。

%14
与此同时。

%15
紫薇仙子却是出了天庭,下落中洲,来到龙公之处。

%16
“碎梦杀招我练成啦!”凤金煌高举双臂,欢呼起来。

%17
紫薇仙子没有想到此行,居然亲眼见证了凤金煌的一次重要成长。

%18
她望着碎裂不堪的梦境碎片,仿佛是镜子被打破的样子,心头暗自震惊。

%19
凤金煌在梦道上面的进展之神速,大大出乎她的意料。

%20
凤金煌雀跃欢笑,龙公将其打发,便询问紫薇仙子来意。

%21
紫薇仙子实话实说道:“这些日子以来,我一直心绪不宁不定,却始终寻找不到根由,因此前来向大人您讨教。”

%22
龙公微微点头:“推算方源可有进展?”

%23
紫薇仙子眉头微蹙,露出苦恼的神色:“不得不说,方源在智道方面的防备已做到滴水不漏,我方已经竭尽全力,却算不得他。”

%24
龙公安慰道:“你之前针对他的数策,已是很好,我也做不到你这般谋略,毕竟我可不是智道蛊仙。那定空蛊摧毁了吗?”

%25
“那些蛊虫身上的炼道杀招,还未有发动。但我已命袁琼都时刻准备着了。”

%26
龙公点头:“看来方源要搜刮蛊仙俘虏的蛊虫,并不是那么容易的。”

%27
想了想,又问:“可去魔尊幽魂处查探了么?”

%28
紫薇仙子回答得很干脆:“他到底只是一团残魂罢了,没有仙元没有仙蛊,只是阶下之囚。琅琊福地战败后,我就对他多加‘关照’,并无问题。”

%29
龙公叹息一声:“或许就是他了。”

%30
“谁?”

%31
“红莲魔尊。”

%32
紫薇仙子心头微震,便听龙公继续问道:“寻找红莲真传一事,进展得如何?”

%33
紫薇仙子脸色有些难看:“今古亭一直在镇守光阴长河,只要方源进来,必会察觉。但就在刚刚,恒舟搭建失败,还需重来。我还计划组建出三秋黄鹤台、鲨流撬、刹那台。有这五大宙道仙蛊屋,就算寻不得红莲真传,也不会让那方源如愿!”

%34
龙公听了这番话:“前四座也就罢了,那第五座刹那台的核心仙蛊,可是百万天王画廊中的仙蛊。”

%35
紫薇仙子道:“这是我推算所得,是当中最为稳妥之策。”

%36
在龙公心目中,虽然重视方源,但方源终究比不上红莲魔尊。他沉吟道:“红莲真传的确是最大的变数,五座宙道仙蛊屋封锁光阴长河,的确是稳妥。就依你计罢。”

%37
“我虽不修行阵道,但也知道仙蛊屋搭建困难,对于从严,就不必苛责了。”

%38
“是,龙公大人。”

%39
因为琅琊福地战败,这一世天庭更加激进,步调更加紧凑。

%40
古月方正提前升仙只是一方面,恒舟也在加紧筹建,并且还增添了第五座宙道仙蛊屋!

%41
这点墨水效应,是方源不清楚的事情。

%42
至尊仙窍。

%43
“到了最关键的地方,似水流年仙蛊,去吧。”方源心念调动,一只仙蛊飞出,直射向前。

%44
到了仙蛊屋雏形中,立即掀起色彩斑斓的光辉。

%45
光辉吞吐,明灭不定。

%46
方源全神关注,不敢有丝毫的马虎,掌控局面。

%47
一时间,他的脑海中无数念头此起彼伏,仿佛海上掀波,囤积的意志如积雪消融。

%48
一会儿工夫,他的本体上下就被汗液打湿。宙道分身也被抽调过来,在仙窍中协助,强自支撑,脸色惨白如纸,身躯摇摇欲坠。

%49
这番心血和苦功不是白费的。

%50
原本斑斓的光辉,逐渐融汇合并,形成清明的蓝色。

%51
光辉也不再像之前明灭吞吐,而是逐渐稳定下来。

%52
最终,当蓝色的光辉统统收敛,消失无踪后,场中显现的就只有一座仙蛊屋了。

%53
八转仙蛊屋——万年斗飞车。

%54
核心是似水流年仙蛊,使得此屋攀升到八转层次!

%55
方源不建则已,一建就是一座八转仙蛊屋。

%56
本来似水流年仙蛊,主要是用来产出年蛊。海量的年蛊作为食料,喂养着方源手中的年兽族群。

%57
但现在,因为有了年华池,有了完整稳定的食物链,方源可以让这些年兽族群在里面自给自足,似水流年仙蛊也就抽调出来。

%58
这只八转宙道仙蛊,乃是黑凡所创,擅长后勤,乃是仙窍经营的极品仙蛊,但一直缺乏攻伐杀招。

%59
方源此次算是弥补了这层缺憾,将它组建成八转仙蛊屋。

%60
除了似水流年仙蛊之外,万年斗飞车的主要辅助仙蛊,是防备、斗两只七转。

%61
前者原本方源用于卜卦龟变化,和金刚念仙蛊一同搭配,但现在这等变化杀招,已经不合方源所用,所以抽调出来,极大地增强了万年斗飞车的防护威能。

%62
后者是律道仙蛊斗,一直都没有开发多少用处。因为之前方源的律道造诣并不深厚,身上律道道痕稀少,而律道境界也是最近才攀升上来的。

%63
律道是应用面最广的流派,这两只仙蛊和似水流年搭配起来,恰到好处。

%64
整个万年斗飞车,体型狭长,宛若一艘扁舟,通体银光灿烂,闪耀人眼,又给人尖锐坚固的强烈感觉。

%65
天庭方面组建恒舟失败,但方源却是成了。

%66
原因就在于方源不仅是宙道准无上,更有律道大宗师,阵道宗师的境界。

%67
长久以来,方源投入精力,兼修多种流派,经过初期发展的阵痛后,妙处已经逐渐显露,并且这种好处将来还会越来越多。

%68
“这是第一座仙蛊屋,由我本人开创,全新之物,将来对战,必定能给敌人一份大大的惊喜。”

%69
方源心中欣慰、欢喜。

%70
上一世,他一直打算组建一座仙蛊屋,但没有真正成功,出来的只是过渡品,威能功效差强人意。

%71
这一世,他建设出了万年斗飞车,自己很满意。

%72
但这还不是终点,这只是第一座仙蛊屋。

%73
方源的计划很宏大。

%74
他预计至少要搭建出三座仙蛊屋来,在他的推算中,这个方案最能高效利用这一世可获取的所有资源。

%75
万年斗飞车先放置一边,宙道分身为了搭建此屋,差点晕厥过去,状态极差。方源只好暂时安排他休整,不再去推算什么东西。

%76
方源本体也难得休息了两日。

%77
外界两日,至尊仙窍中已过了数月。

%78
宙道分身早已在康复,并且推算又有了成果。

%79
仙道蛊阵——蛊如故大阵!

%80
费了一番精力后,此阵被方源成功建成。

%81
方源这才决定对刘浩下手。

%82
刘浩的仙窍还在肉身之中,魂魄却已是被方源摄取出去,囚禁在别的地方。

%83
方源从刘浩的仙窍中取出定空蛊。

%84
几乎在瞬间,定空蛊就要摧毁,然而蛊如故大阵早已发动,施加着剧烈影响。

%85
定空蛊上光芒闪烁不定,一道道的炼道道痕被剔除、销毁,片刻后,定空蛊恢复了宁静,只是带着微微的损伤。

%86
方源手握此蛊,吐出一口浊气。

%87
终于将此蛊拿下了!

%88
上一世,天庭的手段发动,不仅是定空蛊自毁,刘浩身上的其他蛊虫也统统自毁。

%89
不过方源也有所得。他从定空蛊的残骸碎片中,得到了仙蛊——蛊如故设想的希望。

%90
上一世,他对这些残骸深入研究,并且一直没有终止这方面的推算。

%91
有了这样的积累,到了这一世,他终于推算初步成功,有了一座蛊如故的大阵。

%92
这座大阵牵涉的蛊虫就多了,足有上千万的蛊虫,仙蛊涵盖了炼道、宙道、律道,多达二十多只。

%93
方源的设想是蛊如故仙蛊,所以说,这座蛊如故大阵只是初步的成果。

%94
等到他将这座大阵,改良简化成实用的杀招,就是中期成果了。

%95
将来再将杀招浓缩成一具仙蛊方,炼出蛊如故仙蛊,这才能达到最终的成就。

%96
“只是就目前而言,蛊如故大阵已经是我的极限。”

%97
“我要继续进步下去,还得增加底蕴。”

%98
“我吸收了长毛真意,炼道境界达到准无上,不过炼道的历史上还有两大丰碑,就是和长毛老祖齐名的天难老怪、空绝老仙。这两人的绝大多数传承,天庭已收录囊中。”

%99
“或许,天庭布置在定空蛊上的炼道手段,就是出自于这两大丰碑。”

%100
“只要我获取到这些内容,就能底蕴大涨,浓缩仙蛊方或许达不到,但绝对能够简化成杀招,得到中期成果了。”

%101
方源眼中仿佛燃烧着烈焰。

%102
上一世,他看不到希望,觉得天庭是天堑绝壁。

%103
但这一世,他却不这么想。

%104
“天庭,我要攻入天庭,掠夺到这些真传的内容!”12971

\end{this_body}

