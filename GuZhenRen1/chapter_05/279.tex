\newsection{再换}    %第二百七十九节:再换

\begin{this_body}

炼出一只六转仙蛊的难度,和炼出一只七转仙蛊的难度,也是完全不一样的。

方源想要用区区一只六转仙蛊飞熊之力,换取七转的卜卦龟背仙蛊,说得好听些,是痴心妄想,说得不好听一些,是贪婪无度,想要占便宜。

这种“占便宜”,还带着一点蔑视的意味。

这把庙明神看成什么了?

是觉得他好欺负吗?

还是觉得其他人都很傻?

所以,蛊仙们的脸色阴沉,也就可以理解了。

方源却是笑了笑。

群仙的态度和神情,他都看在眼里。这些人的心态变化,他也心知肚明。

不过他仍旧道:“不知庙明神仙友,换还是不换?”

庙明神笑了笑,笑容有些勉强,但他终究还是点头:“换!仙友于我有恩,两次助我,凭这些便完全可以。”

其他蛊仙闻言,无不对庙明神刮目相看,更对方源表示反感。

其中一位,甚至冷哼一声,对方源表达深切的不满。

方源忽的哈哈大笑。

他接着道:“世人皆传,庙明神大人如何如何,我心存疑,今日一试,却是冰消瓦解。我怎可能如此占仙友便宜?我用另外一些,来弥补当中差价,定能叫庙明神仙友满意!”

说着,方源就向庙明神秘密传音。

庙明神听闻之后,顿时脸色动容,神情激动:“楚瀛仙友,此话当真?”

“哈哈,是真是假,咱们马上做了这笔交易,仙友不就清楚了吗?”方源又笑道。

“抱歉了,这场交易我俩需独自进行。”庙明神打了个招呼后,方源便随着他,离开此处地下洞穴。

去到外面,完成了交易之后,两人回来,各个脸上都带着笑容。

方源轻松,庙明神的笑容却透露出十分满意之色。

两人把臂同游的架势,任是谁都明白,两人交情又突飞猛进一层。

坐下之前,庙明神对方源恭敬一礼,道:“不想又欠下仙友第三次人情。qiushu.cc [天火大道]”

“仙友风范,叫人心折。这只是一场交易,谈不上什么人情。”方源表现得十分客气。

这番对答,更叫其他蛊仙感到非常费解。

一时间,群星心底都在暗暗猜测:这楚瀛究竟出了什么筹码,竟能轻松弥补当中差价,还能让庙明神仙友如此表态?

庙明神入座,方源却没有。

这场交易完成之后,就轮到他方源,所以他直接走上了高台。

底下群仙看向他的目光,都悄然发生了变化,变得郑重起来,更多几分好奇和探究之意。

同时,他们对方源究竟想要交换什么,也起了不少兴趣。

这是因为,他们对于方源都不了解,这完全是一个陌生人。

根据方源的交易内容,蛊仙们就能推测一些他的情报。比如说一些数量很多的仙材,往往是蛊仙福地的自家产出,如此一来,就能推导一些内容,比如方源的仙窍发展到什么程度,又比如说仙窍中的哪一种道痕比较浓郁。

如果是一两头的上古荒兽的残破尸体的话,往往就是蛊仙自己的战利品了。毕竟若是仙窍中豢养的上古荒兽死亡,尸体往往不会那么残破。从这样的尸躯中,群仙们也能对方源的战斗力,进行一番推算。

蛊仙向来都是非常注重情报收集的。

这方面,信道蛊仙最有优势。

但方源拿出来的东西,却叫群仙吃了一惊。

“诸位请看,这是一只骨道六转仙蛊,名为骨刺。乃是我一次意外中得来,可惜,我却不是骨道流派。这次就拿出来交易,哪位蛊仙有意此蛊?至于换什么,我和庙明神大人一样,也没有太多的想法。”方源笑道。

“居然又是一只仙蛊。”

“这才第一轮,居然出现了两笔仙蛊交易。”

“这楚瀛身家也是丰厚,明明是变化道蛊仙,却有两只其他仙蛊。”

“前一笔已经完成了。不过这一笔,却是悬了。在场的哪里有什么骨道蛊仙?”

蛊仙们私底下议论,他们彼此之间都比较熟悉。

“没有吗?”方源等候片刻,不见有蛊仙开口。

不过他心中却没有多少失望。

单单他得手的七转仙蛊卜卦龟背,就已经值得参加此次交易会了。

就在方源想要下场的时候,一个声音传来:“且慢,我想……可以交易这只仙蛊。”

群仙连忙循声望去,出声的是巫马杨。

“奇怪,他是暗道蛊仙,要换取骨道仙蛊做什么?”这个问题顿时在群仙心底流转不休。

骨刺仙蛊,自然和暗道关联不大。

但若是巫马杨有什么仙道杀招,需要骨刺仙蛊做补充,那么他求购这只仙蛊,也并非不可能。

每个人都有各自的秘密,蛊仙们虽然疑惑,但都按捺住心思,静静旁观。

“竟还真有蛊仙想要骨刺仙蛊?”方源也感到好奇。

他问道:“不知巫仙友,想要拿什么交易呢?”

巫马杨皱起眉头,他有一些犹豫,沉吟了一下后,他咬牙道:“用这只仙蛊。”

他展示出自己的这只仙蛊。

这也是一只六转仙蛊,品级方面,和骨刺仙蛊一样。

群仙视之,立即当中有人低呼道:“这是仙蛊日啊。”

没错,这就是六转宙道日蛊!

方源呆了一呆。

“没想到,这场交易会我真的是来对了!”方源心中欢呼,表面上则是不动声色,好像是陷入思考当中。

其实哪里还要思考!

方源见到这只仙蛊的第一面,他就想要换取了。

为什么?

因为他继承了黑凡真传,有了七转年蛊,有了以后仙蛊,有了似水流年,有了一大把的宙道仙级杀招。

他记得真传中,黑凡一直对仙级的日蛊、月蛊念念不忘,叨叨唠唠。

他还记得度日如年等四个相似的仙道杀招。

“若是后继者能够得到仙级的日蛊或者仙级月蛊,那么这四个杀招就无须耗费那么多的凡级日蛊、月蛊了,并且杀招的效果也会提升不少。”

这是黑凡真传中的原话。

方源看过这番话,但当时没有多想。

他没有得陇望蜀,有了年蛊,就已经很满足了。

但命运奇妙的地方,往往就在这里。方源没有寄希望于探寻到仙级的日蛊、月蛊的时候,偏偏日蛊就这样主动出现在自己的眼前了。

“那就换了吧。”

“虽然我对宙道不是很了解。不过喂养骨刺仙蛊,的确是一个负担。”

方源这样说道。

“我这日蛊,会自行汲取光阴长河的河水,无须劳烦喂养的问题。不过,我对这只仙蛊运用不了,不如交易了这只骨刺仙蛊,还能掺和到我的其他杀招里面。”巫马杨如此道。

两人当场进行了交接。

两只仙蛊中的原本意志,一步步撤退而出,让对方的仙元顺利浸染。

仙元中蕴藏着蛊仙的意志,这才是炼蛊的关键要素。

意志对于蛊虫而言,打个地球上的比方,就仿佛是一段设定好的程序。

意志是可以思考的。野生蛊虫中,有着属于它们独自的意志。炼化野生仙蛊,就是用蛊仙的意志配合仙元,剿除掉野生意志。

野生意志以求生为主,并不懂得自毁。所以蛊仙炼化野生仙蛊,虽然困难,但风险不大。

他人的仙蛊就不一样了。

蛊仙的意志和野生意志大有差别。

直接蛮干,炼化他人仙蛊,只会导致仙蛊自毁。就像是方源当初,在自己的仙蛊中种下特意,呼唤一声,就能回归一样。基本上所有的蛊仙,都不会任由自己的仙蛊,被他人直接炼化。

一旦出现这种情况,仙蛊中的意志就会强行令蛊虫自毁。

除非是动用智道手段,镇压了仙蛊中的意志。不过往往,很难在一瞬间就全部镇压住。稍有不对劲的苗条,仙蛊就自毁了。

正是这个原因,方源虽然俘虏了黑城,却没有得到他的仙蛊暗箭。

也是因为这个原因,方源和黑楼兰的盟约虽然解除了,但是态度蛊却不能直接炼化,十分麻烦。(\~{}\^{}\~{})

\end{this_body}

