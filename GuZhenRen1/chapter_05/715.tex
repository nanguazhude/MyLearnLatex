\newsection{陈衣撤阵}    %第七百一十八节:陈衣撤阵

\begin{this_body}

不败福地战场。

九九连环不绝阵。

“对方的这个大阵……”

“我方的蛊仙在受到强烈的反噬伤害!”

“第一阵退下休整,接受治疗,第二阵接替,第三阵准备!”

陈衣、白沧水一边操纵着大阵,一边指挥中洲蛊仙,有条不紊。

“没想到对方竟然会出这样的奇招。”一旁,凤九歌轻轻皱眉,盯着阵中的五界大阵,“这下麻烦了,搞不好九九连环不绝阵会因此……”

凤九歌心中浮现不详的预感。

五界大阵在池曲由的主持下,不断向外挥散出五色烟气。

这些烟气不断扩散,越变越淡,渗透到九九连环不绝阵中。

“好,就是这样!”

“其余人对付大阵,让中洲蛊仙频繁操控,大阵运转越多,他们承受的反噬就越大。”

“哈哈哈,九九连环不绝阵越强,这些操纵的蛊仙遭受的伤害就越大!真是妙极了。”

南联和北原诸仙士气大振。

他们改变了战术。

原本已经看破了此阵,明明可以对准破绽,将大阵攻破,但现在却不急着去做了。

五界烟瘴不断渗透,他们可以凭此巧妙地针对阵后的中洲蛊仙。

甚至从某种程度上,九九连环不绝阵反而成为方源一方,攻击中洲和天庭的强大武器。

“我们该如何应对?是否派遣蛊仙攻破地方大阵?”白沧水看向陈衣。

这里的第一领袖,还是陈衣。

陈衣面色相当的难看,他沉吟道:“这大阵应当属于五界真传,南疆陶铸的手段了。”

五界山脉一战,天庭也插手过,此役君神光被方源擒拿活捉,因而天庭一方对于陶铸真传,对于五界山脉一战的前前后后,都有许多详尽的了解。

“陶铸能营造出五界山脉,那么这些渗透到大阵中的五色烟瘴,显然就是模拟出五域界壁,从而令我方蛊仙屡遭大阵反噬。”

陈衣语气颇为苦涩:“九九连环不绝大阵,乃是复合大阵,可以随时拆子阵、布子阵,故而能绵绵不绝。当初无极魔尊闯阵,也是连破百阵后脱困。”

“但这陶铸的手段,居然绕过了九九连环不绝大阵,专门针对我方的蛊仙!”

“这不仅是避重就轻,而且我方大阵越强,主持大阵的蛊仙所受的伤害就越深。”

“如果我们猛催大阵来对付对方的大阵,必定是对方想要看到的,落入对方算计,并不可取。”

“但若是我方暂歇大阵,派遣蛊仙出阵,却又是舍长取短。并且敌方阵中可是有武庸、方源等凶悍匪徒,派遣去破坏大阵的蛊仙十分危险。”

陈衣之前中了武庸的送友风杀招,因为凤九歌而侥幸生还,心中仍有余悸。

更关键的是,他们这边的八转蛊仙只有陈衣、白沧水、凤九歌三人,并且人人带伤,派遣谁出去都不合适。

分析到这里,陈衣深深叹息一声:“没想打九九连环不绝阵,居然在今日被敌方克制!”

陶铸不过是一位寻常的八转蛊仙,根本不能和武庸、方源等人相提并论,而九九连环不绝阵的创造者却是大名鼎鼎的星宿仙尊。

五界真传克制九九连环不绝阵,实在是陶铸的荣耀!

说起来,陶铸生前在八转蛊仙中并不起眼,名声也不大,只是营造出了五界山脉而已。

但是没想到,他真的研究有成。

经此一战,陶铸必将名动天下,尽管他已死去很久。

江山代有人才出,各领风骚数百年。

就像天庭战场中,盗天魔尊的成双入对杀招能被破解一样,星宿仙尊的九九连环不绝阵同样也能被后辈克制。

星宿仙尊、盗天魔尊作古太久,而时代却是在不断地发展,日新月异。他们的手段虽然强大,但也难逃过时的尴尬。

“把那子阵撤了。”陈衣毅然下令道。

身边的蛊仙吃了一惊。

陈衣居然下达撤阵的命令,他的脑袋烧糊了么?

不过很快,凤九歌就首先反应过来,他双眼精芒一闪,赞道:“好应对。”

“撤阵了,对方居然主动撤了阵,哈哈哈。”五界大阵中,一些北原蛊仙看到这一幕,开怀大笑。

武庸、方源却是面色微微一沉。

“不妙,这陈衣好生狡诈。”

“若是我们在对方阵中,五色烟瘴渗透将又快又深。现在子阵一撤,效率大减。”

“没错。天庭战场长生天已经快要支撑不住,天庭牢牢占据上风。陈衣恐怕就是考虑到这一点,所以采取了拖延的战术。”

陈衣一方不管是大力催动九九连环不绝阵,还是派人来攻阵,都是方源等人乐意看到的。

但是现在,陈衣主动撤销子阵,选择了被动防守。

五色烟瘴渗透效率大减,中洲蛊仙只是维持大阵,并不猛催,受到的反噬伤害也减少许多。

偏偏陈衣等人还在阵后,继续布置仙阵!

方源等人需要多得多的时间,才能破坏这座九九连环不绝阵。

最要命的是,方源等人必须争分夺秒,一刻都不能耽误。一旦天庭那边将长生天的兵力剿灭,那么他们就能腾出手来,杀下天庭,对付方源等人。

陈衣极其稳健,老谋深算。

他故意退让,是从大局出发,只是退了一小步,却反而比发起攻势还要可怕。

他把方源等人直接逼入了悬崖边缘。

“唉,早知如此,我就应当将阵灵、阵旗仙蛊都参进五界大阵。如此一来,五界大阵就可以整个腾挪转移,而不是静固在原地了。”

方源在心底叹息一声。

这倒并非是他的失误,也不是他不愿意贡献。

而是阵灵和阵旗仙蛊等等,组合到五界大阵中去,推算量急剧暴涨,势必要消耗方源更多的时间才能推算清楚。

其实,就算是五界大阵能够腾挪,可以被蛊仙整个搬走,方源等人也解决不了眼前的危机。

陈衣主动撤销子阵,向后退守。原本子阵中的五界大阵,便暴露在外。

九九连环不绝阵外,诸仙早已经混战一团。

四位八转蛊仙清夜、厉煌、凤仙太子、巴十八各自捉对厮杀,玉清滴风小竹楼、寒螭庄、风满楼等纵横战场,时而狠狠碰撞,时而爆发仙道杀招,一片片光影璀璨,一阵阵雷霆炸响。

“不好,无限风杀招正在脱离我的控制。”武庸面色一变。

之前他在大阵中,还未有清晰的感应。现在陈衣主动撤阵,使得武庸发现不妙。

没有迟疑,武庸飞出五界大阵,朝着风满楼扑去。

药皇、百足天君等人亦接连出动。

有了他们的增援,厉煌、清夜等人压力暴涨,连连后退,和中洲的仙蛊屋一起被压入下风。

方源悄悄出动,忽然出手,再一次施展大盗鬼手。

中洲仙蛊屋立即遭殃,被方源盗取一大把的蛊虫。

方源连连抓动,不一会儿就又收获了三只仙蛊。算上他之前在帝君城战场中的收获,他从中洲仙蛊屋上抢夺的仙蛊,已经有十多只。

对于这些仙蛊,方源有的认识,有的不认识。

他的宙道分身一直沐浴在智慧光晕中,不断构思推算这些蛊虫该如何巧妙运用。

见情况不对,厉煌舍弃凤仙太子,赶来纠缠方源。

之前在帝君城战场的一幕,再次重演。

方源被厉煌纠缠,再难以对仙蛊屋下手。

“哼,这个凤仙太子,装得惟妙惟肖,故意放走厉煌。”方源心中冷哼,按捺不发。

凤仙太子始终是一个麻烦。

更令方源感到麻烦的,乃是天庭催发的九转人道杀招人中豪杰。

受到人中豪杰的增幅,厉煌、清夜等人,以及中洲的仙蛊屋都战力暴涨数倍,抵挡着南疆和北原联军的狂猛攻势。

“不能再这样混战下去,时间有限!我们撤到大阵中去。”方源高声呼唤。

武庸等人收缩战线,纷纷收起仙蛊屋,钻入五界大阵。

五界大阵全力催动,大股大股的五色烟瘴,迅速蔓延,囊括整个战场。

在五色烟瘴当中,中洲蛊仙都遭受到强大的压制。

和陈衣之前的尴尬一样,这些人受到人中豪杰杀招助益,杀招威能暴涨,反噬伤害也跟着节节攀升。

“没想到陶铸的手段,竟有如此奥妙!”大阵内,巴十八感慨不已。

武庸沉默。

这个情形并不出乎他的意料。

正是因为他早已发现,所以才不惜泄露真传内容,也要找池曲由合作。

池曲由也知道这是好东西,但之前没法用出来。现在大阵改良,成功布置出来,立收良效。

“只可惜这五色烟瘴不分敌我,中洲受到压制,我们在里面也要受到反噬。”

“唯有方源一人,才能行动自如啊。”药皇明显鼓动道。

许多人瞧向方源,纷纷流露出羡慕的神色。

“可惜。”方源暗叹一声,若是他能得到陶铸真传,依凭智慧光晕以及不俗的阵道、智道境界,他的战力将再一次飙升暴涨。

单说一点,他对人中豪杰杀招始终没有办法,但若是有了五色烟瘴,厉煌等人将遭受更大压制。说不定帝君城战场中,方源就可一力平定了。

当初,他从五界山脉撤走,无法收取五界真传时,就感到一阵阵的遗憾。

现在看来,他当初的感觉并没有错。

,请用搜索引擎各种小说任你观看

\end{this_body}

