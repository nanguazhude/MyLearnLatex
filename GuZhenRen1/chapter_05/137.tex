\newsection{市井}    %第一百三十七节:市井

\begin{this_body}

话说。\&\#65288;\&\#26825;\&\#33457;\&\#31958;\&\#23567;\&\#35828;\&\#32593;\&\#32;\&\#87;\&\#119;\&\#119;\&\#46;\&\#77;\&\#105;\&\#97;\&\#110;\&\#72;\&\#117;\&\#97;\&\#84;\&\#97;\&\#110;\&\#103;\&\#46;\&\#67;\&\#9

人祖的四女儿森海轮回,从大树上掉落下来,没有跟上人祖的步伐,困在了平凡深渊之中。

一天,她在睡觉,遇到了一位小人。

小人嚎啕大哭,森海轮回便问为什么。

小人说:“我是我们部族中体型最大的了,我常常因此而勇敢、骄傲、得意。今天我打算攀爬一座山,没想到这座山居然是一个人。天底下居然有你这么大的人,我还是头一次看到,忍不住就哭了!”

从此之后,森海轮回便和小人成了好朋友,形影不离。

“小人啊小人,你的父亲在哪里?你有没有兄弟姐妹?为什么我从未见过你的家人?”有一天,森海轮回问小人。

小人默默地将森海轮回引到一口井前面:“你看这口井,井里就是我曾经的家。”

森海轮回便趴在井口,探出脑袋,往里面瞧。

然后她惊呼出声:“哇,有这么多的小人啊。”

她看到在井里面,有密密麻麻的小人,生活在一起。

他们建立了无数的房屋,相互挨着,有集市,有花草,其乐融融,一派和谐的样子。

井底的小人们也发出惊呼。

“怎么天忽然间暗下来了。这么快就到晚上了吗?”

“还想起了雷,却不下雨,也不见电光!”

小人们混乱了,议论纷纷,整个市镇都嗡嗡作响。

“这里有你的家人吗?为什么你要跑到井外来,不和他们一起生活呢?”森海轮回问道。

小人摇摇头:“是他们驱逐了我,他们认为我是一个怪物。”

“哦?这是为什么?”森海轮回很好奇。

小人悲伤地回答道:“我告诉他们,其实我们世世代代一直生活的世界,只是一口井而已。外面的世界,还很大很大。但他们不相信,叫我不要胡言乱语。”

“我又告诉他们,其实我们城镇边上的那座山,只是很小很小的土丘。但他们不相信。他们觉得那是最高的山峰,他们叫我不要散播谣言。”

“是哪座山呀?”森海轮回问。

“就是那座方寸山。”小人指点道。

森海轮回咯咯地笑起来:“这还算山啊,放在我手掌上都可以!”

忽然,森海轮回一拍手:“那么就由我来告诉他们真相吧。”

小人摇头不止:“没有用的。”

森海轮回不信。她趴在井口,朝下面大喊,井底的小人们更慌乱了。

他们觉得今天太古怪了。

不仅一下子天就黑了,闷雷不断,雷声还越来越大。越来越密。

“老天爷发怒了,我要忏悔,我要认罪!”

“不,是怪物,怪物把天给吃了,正在打饱嗝。”

“救救我吧,这个世界要毁灭了!”

有的小人跪在地上求饶,有的小人满脸绝望,有的小人疯狂乱走。

森海轮回的呼吸,传达到井底。变成了巨大的狂风。

森海轮回的吐沫,落到井底时,变成了瓢泼大雨。

森海轮回的话语,到达小人的耳朵里时,化为一声声的惊雷,差点要震破他们的耳膜。

森海轮回终于放弃了努力,她趴在井口,疲累了,有气无力地道:“这些小人怎么这么笨呐?他们就不会像你一样,爬出来看看外面的世界吗?”

小人摇头叹息:“他们已经觉得自己很大了。觉得天空就是圆的,山就那么高,生活就是那样子。”

“那你又是为什么爬出来呢?”森海轮回瞪大一双好奇的眼睛。

小人苦笑:“那是因为我认识到自己很渺小啊,所以我见到了世界的大。”

……

方源环绕市井走了一圈。双眼中的精芒一直闪烁不定。

这是他遇到的第三座天地秘境。

前两个,一个是荡魂山,另一个则是落魄谷。

影无邪为了陷害方源,铺设陷阱,就是利用的市井。可惜被方源打了个措手不及,从刘青玉那里得知了宝贵情报。结果反而让方源利用了这个市井。剿杀了一群东海蛊仙。

正如《人祖传》中记载的那样,森海轮回呼吸成风,吐沫成雨。方源站在井口处,发出的仙道杀招,落到井底时,威力会增幅数十倍、上百倍!

不过,方源并非毫无代价。

他为此付出的仙元耗损,也上涨到数十倍、上百倍。

方源虽然只前后催发了三记大手印,但是和他催发百十来次大手印的仙元消耗,没有什么两样。

不知情者,若陷入到市井井底,就会被困住。不过,一旦明白了自己的处境,洞察到这是市井,便能立即脱身而出。

就像《人祖传》中所言,只要小人意识到自己的渺小,想要见证井外世界的广大,就可爬出市井了。

可以说,脱离市井说难也难,说容易也容易。

正因如此,影无邪才没有借助市井,来对方源实施狠辣打击。

一来,他控制不住力道,说不定把方源轰成焦炭,或者拍成肉酱,他到哪里去找蛊材逆炼至尊仙胎蛊去?

二来,方源说不定立即就得到了真相,携带上极天鹰脱困而出。那时候,影无邪连跳的机会都没有。

关键是第二点原因。

正是因为太不确定,影无邪这才将市井作为一个陷阱消耗方源,而不是亲自埋伏来斩杀方源。

这座市井,方源非常想纳为己有。

因为这个天地秘境,对于方源今后的仙窍建设,有着巨大的辅助作用。

拔山仙蛊!

方源开始尝试。

可惜市井只是稍微颤抖了一下。

拔山仙蛊可以拔山,但市井却不是山。不过它坐落在地上,也要汲取地气。

力道大手印!

方源试着将市井用力道巨手直接挖出来。

但没想到,力道大手只消接近市井,就迅速萎靡,由大变小,宛若破了的气球,几个眨眼的功夫,就消失殆尽了。

“看来影宗虽然掌握了这处天地秘境,但是却难以搬迁。”

方源又尝试几次,都无效果。效果最明显的,还是他第一次尝试,动用拔山仙蛊,让市井的井壁颤抖了一下。

“如此的话,拔山仙蛊的确有效果。但是单纯运用拔山是不行的,非得是设想出一个仙道杀招来,专门收取这座市井才行。”

意识到这点后,方源只好暂时选择放弃。

他钻入井中,落入井底时,身形已然变得十分微小。

打扫战场。

唯有周礼、汤诵二人留有全尸,方源都塞入至尊仙窍。

其余蛊仙,已经都被拍成肉糜。

血道魔仙丁齐,则是直接自爆,不过血道仙窍当然是留下的。

除了他,还有其余的数位蛊仙,同样留下了仙窍。

很快,井底世界就开始动荡起来,天地二气涌现而出,一座座福地开始成形。

方源花费了大量时间,进入这些福地中,进行探索。

仙蛊是不用想了。

不过东海的常规修行资源,倒是存在不少。

这些蛊仙,可不像悲催的刘青玉那般,大概身家都保留下来。

惟独血道魔仙丁齐,却是连自家仙窍中的资源都全数销毁了。即便如此,方源乃是血道宗师,境界足够,能够吞并丁齐的福地,扩张至尊仙窍。

这些福地中,有些存在地灵,有些福地却没有。

无地灵的福地,方源将其资源都搜刮进来。有地灵的福地,就比较麻烦了,需要认主。不认主的话,方源是无法搬迁里面的修行资源的。

当然,强行劫掠也并非不行,只是图眼前利益,不顾长远。

方源考虑了一下,没有用强。

花费了大量时间在井底,出来之后,方源就着手整治这个气泡。

气泡中有海有岛,市井就是在小海岛的中央。因为影无邪的布置,外人挤进来,都会被直接投入井底去。

不过方源有了刘青玉提供的情报,自然不会陷落于此。反而利用了这层陷阱,全歼了追兵,省去了好大麻烦。

若是影无邪知道自己的布置,不仅没有给方源带来什么麻烦,反而帮助了方源,不知道会不会气得吐血?

方源保留了影无邪的一些布置,同时也增添一些自己的手段,让气泡能够隐藏得更加隐秘。

随后,他出了气泡,开始牵引气泡,转移位置。

不久之前,影无邪也这么干过。

这个藏有市井的气泡,原址并非在这里。是影无邪想陷害方源,才费尽辛苦,花了不少精力和时间,才挪移过来。

在乱流中,方源身不由己,会被水流冲走。

幸亏有了藏娇蚌可以当做歇脚之地。

方源来来回回好几十趟,才将这个气泡挪到另外一股巨流当中去。

费的仙元和时间,是影无邪的十多倍。

完成这个事情之后,方源又一路辗转,来到影无邪栖息的营地。

那里,又是一处气泡中的海岛。

方源深入地下,悔池早已经被影无邪拆除了,就连周围的蛊阵都没剩下。

原本被影宗拘拿在这里的光阴支流,已经消失不见。

这是乱流海域的特征,每一股水流都是不断变化和移动的。

“不过有这一点,也足够向庙明神示好了。”方源便利用信道凡蛊,通过宝黄天,传信出去。(未完待续。)

\end{this_body}

