\newsection{黑凡炼蛊}    %第一百一十节:黑凡炼蛊

\begin{this_body}

“相比较黑城,我虽然有众多的仙蛊,却是开发不足啊。≧,”方源自省,反思自己。

黑凡真传带给他许多触动,不过他并没有动摇:“我喂养这么多的仙蛊,当然也不算错。”

蛊仙对于灾劫,都是再重视也不为过。修行越往上,渡劫越是难。

黑凡仙蛊少,喂养的负担就小,但是手段多,就足以应付各种情况。资源大部分都累积下来,就算是卖掉换来仙元石,也能化作丰厚的仙元储备。

而且黑凡还有个家族,超级势力,黄金血脉。黑凡又喜欢提携后辈,其实他有许多仙蛊,不太重要的,在布置真传之前,就都陆续分给黑家正统蛊仙了,或者是充当黑家库藏,传承下去。

黑凡并非十绝资质,仙窍潜力有限,经营的资源也有上限。

综上所述,他喂养少数仙蛊,开发大量杀招,绝对是明智之举。

但方源不同。

具体情况,向来要具体分析。

第一个最大的不同,就是至尊仙窍。

至尊仙窍潜力极其巨大,养蛊是负担,喂养仙蛊更是负担巨大。但方源可以负担得起,甚至还能绰绰有余。

第二个不同,方源是孤家寡人。

他独自一人,一人吃饱全家不愁!不用担心什么家族,也不牵挂什么子孙后辈。

这是孤家寡人的优势。

家族的优势,在于资源共享,相互帮衬。但落到黑凡这种层次。就是自己出力多,家族受惠多了。

不过黑凡从未想过脱离家族。

他是在家族的这种氛围下成长起来。整个过程中受到家族方面的照顾很多,产生亲情、友情、爱情等等羁绊。他心系家族。还有他的那些子孙后辈。

一个人的成长环境,对于一个人的影响是极其巨大的。

而方源是自由自在,这方面负担几乎为零。他虽然加入了什么琅琊派,但也纯粹只是利用而已。

第三个不同,是方源有巨大的生存压力。

他必须要有大量的仙蛊。仙蛊唯一,数量多,分门别类,单独运用就能应对各种情况。比如要对敌,就运用飞剑仙蛊。要撤退。便动用剑遁仙蛊。

哪怕是没有杀招,方源单靠仙蛊,也能应付得来。

事实上,仙蛊多了,对于构造仙道杀招,也有巨大的好处。

仙道杀招,以仙蛊为核心,凡蛊为辅助。比如方源掌握的仙道杀招剑痕索命,它就是以飞剑仙蛊为主。(WWW.mianhuatang.CC 好看的小说棉花糖其他少量剑道等等流派的凡蛊为辅。

但若方源手头上没有飞剑仙蛊,只有剑遁仙蛊,他要用剑遁取代飞剑,再用剑痕索命。该如何?

剑遁取代飞剑,也是可行的。因为两者都是剑道仙蛊,有这样的基础。完全可以替换。

但是整个剑痕索命杀招,就要大改变。

飞剑是用来攻伐的仙蛊。用它来构建攻伐杀招,自然是方便简单。

剑遁是用来挪移的仙蛊。若用它来替代飞剑,那么剑痕索命所需求的凡蛊,不仅要改换,而且数量要剧烈增长。如此一来,剑痕杀招催动起来,就更加繁复艰难,时间更长,失败的可能更高,实用性大大降低。

综上所述,黑凡的养蛊策略,十分大众化,但并不适合方源。

方源的养蛊策略,对他而言,是十二正确的。不过其他蛊仙若学的话,恐怕就是自找死路了。

方源继续看下去。

黑凡真传的内容中,在仙道杀招的后面,接着介绍了凡道杀招。

这些杀招数量更多,有一大部分是宙道,但方源基本上已经用不上了。

仙凡层次不同。或许以后,方源可以在和其他蛊仙接触战时,将这些作为一种试探手段。不过应用也很小,因为方源早已经有许多优秀的凡道杀招了。

他本身就已经掌握了无数血道、力道方面的杀招。

这些凡道杀招,唯一的作用,恐怕就是当做推演的基础。

凡道杀招之后,就是一些没有创造成功的杀招,有仙道杀招,也有凡道杀招。

完善程度各不相同,有的甚至纯粹只是一种灵感,或者猜想。

这是在修行中,黑凡至死未完成的部分。

推演杀招,也是蛊仙修行中不可避免,不可缺少的一部分。

方源稍稍看了一眼,却没有多少共鸣。

他的宙道境界太低,智慧蛊又不能用。再说方源之前的手段已经很多了,足够用了。

如此,杀招部分告一段落。

“回去之后,我还要特意花费大量时间,去演练这些杀招了。”方源计划着。

杀招运用,也有失败的可能。

要是在激烈的战斗中,催动失败,蛊仙还会遭受到反噬。若是偏偏是战斗的关键时刻,那就更糟糕了。甚至很可能,因为一个杀招的成败,影响战斗结果,甚至是生死之别。

所以杀招,千万要演练纯熟!

这方面又需要一笔资金。

构造杀招的那些辅助凡蛊,要收购准备吧。演练的时候,催动仙道杀招,无疑要消耗仙元吧。演练的次数,至少不止一次吧。

这些都需要钱,钱,钱!

所以蛊仙经营仙窍,历来是修行的重中之重。

经济是军事的基础。

况且兵凶战危,又有灾劫威胁,除非涉及重大利益或者矛盾,蛊仙之间一般都是和气生财。

“要演练这么多的杀招,可是很大一笔开支呢。”方源皱起眉头,心中有喜有忧。

这纯粹是一种幸福的烦恼。

大量的仙道杀招,更多的凡道杀招,还有不少的未完成的杀招。方源看过这些杀招之后。接下来就看到蛊方。

仙蛊方!

首当其冲的,就是仙级年蛊的平炼蛊方。

关于平炼的蛊方。还不少。

一共有五个蛊方,都能达到平炼。提升年蛊的效果。

在后面,黑凡还记录了心得体会,叙述了这五种方法的优弊,平炼中的小窍门等等。光是这些炼蛊手法的记载,就已经让方源收获不浅。

五个平炼的仙蛊方中,黑凡用的最多的,是最后一个。

这个仙蛊方,是黑凡在晚年的时候,才开创出来。用料最少。成功率最高,消耗时间和精力也最低。具体而言,它是采用了大量凡级年蛊,添加到仙级年蛊上面,实际效果很好。

开发出来之后,黑凡就几乎放弃前面的四个蛊方,专门采用这个平炼蛊方。

不过为了黑凡真传的全面性,他还是将这四个都录入进来。

“八转仙蛊似水流年可以产生无数凡级年蛊,再用这些凡级年蛊。平炼仙级年蛊,还真是方便呢。”

方源感叹一声,继续看。

在之后,是炼制以后仙蛊的仙蛊方。

方源已经有了这只仙蛊。所以只是匆匆一瞥即过。

总的来讲,这些仙蛊方的规模也十分惊人。

竟然有近百条!

而且都是完整的,十成十的仙蛊方。

大部分是六转仙蛊方。少量七转,两三个八转。

涉及的流派。主要的是宙道,占据绝大多数。但还有其他流派。比如炎道、运道等等。

相比较仙道杀招,这些仙蛊方就比较杂乱了。有近代的仙蛊方,还有古代的,有些古方上的仙材如今都已经灭绝,自然是不能用的。

整个仙蛊方的内容中,黑凡重点叙述了似水流年。

似水流年这只八转仙蛊,竟然也是黑凡的独创!

他的灵感,是从年蛊的不断平炼中得来的。

黑凡在此中,还详细描述了炼制似水流年的整个过程。为了筹集足够多的仙材,他几乎耗费了黑家全部库藏,还花费好大人情和代价,才请的两位炼道大能出手。最后,仍旧欠了一屁股外债。

为了炼出似水流年,黑凡可谓苦心竭力。在炼制过程中,他遭受到一次次的失败,有好几次都想过放弃。

这些心理历程,黑凡都记录下来,不怕后人笑话他的软弱。可见黑凡布置这道真传,实是业界良心。

好在最后他运气不错,失败了九次之后,便成功了。

字里行间中,黑凡炼制八转仙蛊的艰辛和风险,几乎满溢而出。

即便成功之后,黑凡在此后多年也是余悸未消。

他郑重地告诫后继者,一定要万分珍稀前人得来不易的成果。将来若是修行到八转,若要炼制其他八转仙蛊,一定要慎之又慎。

方源看到这里,不由地就想到了雪胡老祖。

如今,这老东西也在炼制八转仙蛊。

而且这只仙蛊,很是珍贵,乃是鸿运齐天蛊!根据琅琊地灵的透露,巨阳仙尊的晚年运道精髓,便是这只仙蛊了。

炼制八转,是一场巨大的冒险。

雪胡老祖有了马鸿运,可以充当主材,抵掉了许多开销。但他为了筹措仙材,仍旧连脸皮都不顾,亲自盗取僵盟的仙僵尸躯,又到黑家库藏中抢夺偷盗一番。大雪山福地,更被他闹得鸡飞狗跳,人心惶惶。

药皇也在炼制起死回生仙蛊。根据北原蛊仙界的传闻,这只仙蛊似乎也是八转级数。

其实大家都知道炼制八转仙蛊的风险。

为什么还要不断冲刺尝试呢?

事实上,有八转仙蛊,和没有八转仙蛊,完全是两个概念。

八转蛊仙也要渡劫,寿命往往短缺,很多时候没有办法,只能冒险。

雪胡老祖是为了渡劫,药皇是为了延寿。灾劫没有困住黑凡,但是寿命局限,最后他仍旧是陨落了。

“灾劫和寿命。”方源心知自己追寻永生,无疑要跨越这两大难关。(未完待续。)

\end{this_body}

