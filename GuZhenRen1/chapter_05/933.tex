\newsection{责任和牺牲}    %第九百三十七节:责任和牺牲

\begin{this_body}

宗门后山,山青水绿,鸟鸣啾啾。

“小都,躺在地上干什么呢?”一个温婉的女声传入耳畔。

“是师姐啊。”少年时代的袁琼都睁开双眼,便见到一位少女俯身瞪着一双水灵灵的双眼看着他,眼眸中带着笑意。

少女的目光转移到小都的手中,一声惊叹发出:“哇,小都,你又炼成了一只三转蛊。你真的很厉害呢。”

“可是我一点都不开心,你知道的,我并不喜欢炼道的修行啊。”袁琼都嘟囔着,爬起来。

“别这么无精打采的样子,你这样的天赋别人可是羡慕不来呢。”少女师姐拍着他的肩膀安慰道。

但袁琼都还是精神不振,垂头耸肩。

“哈哈,这样吧,我将我的一个秘密分享给你。几天前,我在父亲的书房里寻到了一份蛊师传承的线索,它就藏在宗门的后山某个角落,想不想仪器搜寻它?”师姐妙目一转,对袁琼都道。

不出所料,少年心性的袁琼都耐不住寂寞,闻言顿时双眼一亮:“还有这种好玩的事情?妙啊,宗门后山的确是有历代前辈留下的传承。蛊仙传承极其罕见,但蛊师传承却是很多呢。快告诉我线索啊,师姐。”

“嗯,线索只有一句诗词,我这几天一直在琢磨,我念给你听。”师姐并不藏私,这只是一份蛊师传承罢了。而师姐的父亲也即是袁琼都的师父,可是蛊仙存在。

袁琼都听了一遍,并没有多想,便一拍巴掌:“我知道了,这个线索是这样解的,取首尾的字搭配起来,再取第二位以及倒数第二位的字再搭配起来。如此,就是一个方位了。”

师姐琢磨了一下,雀跃道:“原来是那里啊!小都,真是聪明,一下子就猜出来了。”

“那我们快去寻找吧。”袁琼都一马当先。

“等等我。”师姐跺了跺脚,连忙追去。

“放心吧师姐,这个传承我会留给你,我可不会跟你抢的,哈哈。”袁琼都的身影却是很快没入山林深处。

最终,两人从一处山洞中寻觅到了传承。

“这是一份炎道传承呢。”袁琼都和师姐二人仔细查探,“意火?是用意志作为燃料灼烧的火焰啊。这个手段真是别出心裁,令人惊叹。创造出这样的火焰的蛊师究竟怎么想到的?真是有意思。咦,这个落款,怎么是师父?”

袁琼都非常惊诧。

师姐也感到相当意外,不确定地道:“可能是同名同姓吧。”

袁琼都却琢磨起来,双眼蕴藏一抹光亮:“恐怕不是!师姐,我记得你最开始时说过,这份线索本就是从师父的书房里得来的。”

“是啊,我整理爹的书简,你知道我爹爱好收藏这些古物。然后偶然间发现,从一份陈年的书简中掉落了一块竹签,上面就是传承线索了。”师姐回忆道。

“这份传承的确是我年轻时候,布置下来的。”这个时候,袁琼都的师父缓步走进山洞,出现在袁琼都的面前。

和他一道的,还有袁琼都的师伯,矮胖身材,圆脸,和蔼可亲。

袁琼都、师姐当即行礼,齐声道:“拜见师父(父亲)、师伯。”

袁琼都撇嘴:“师父,你这是故意消遣徒儿吗?”

“当然不是。”他的师父摇头道,望着袁琼都手中的传承,目光中蕴含一丝缅怀和感叹。

“哈哈哈,你这个鬼灵精。”一旁的胖师伯指了指袁琼都,“我和你师父刚从灵应峰回来,一路闲谈时发现你们俩的踪迹。这处传承我可以作证,的确是你师父年轻时候留下的。他当年是真心喜欢炎道啊,辛辛苦苦攒了五年的元石,就为收购一只炎道三转蛊。”

“但师父明明修行的是水道啊。”袁琼都瞪大双眼,没想到师父年轻时和他一样喜欢炎道。

“没有办法。”胖师伯叹息道,“师门的水道传承需要继承人。个人的喜好怎能和师门大计相提并论呢?要知道每一份蛊仙传承都需要一代代的继承下去。每一代人都在前人的基础上不断改良,方能使得每一道传承不脱离时代的发展,不被时代淘汰。这是我等身上肩负的责任呐。这份传承是你师父诀别炎道的时候,亲手立下的。我当时就在现场,亲眼看到他哭得稀里哗啦。”

“咳咳。”袁琼都的师父打断道,“既然你们俩发现了这份传承,那就交给你们吧。我们先走了。”

两位长辈离去,山洞中只剩下袁琼都和少女师姐。

“师弟,我修行木道,不需要这个,就送给你吧。”师姐也告别袁琼都。

袁琼都望着手中的传承,心中荡漾着微微的波澜,口中轻喃:“师父……”

再睁开双眼。

袁琼都发现自己仍旧身处在监天塔内。

“该死!我是炼蛊的过程中遭受干扰,所以遭受了反噬,当初昏死过去!”袁琼都回想起来,不由地心头狂跳。

炼蛊最忌讳受到干扰,眼下自己伤势有多沉重并不重要,重要的是宿命蛊前往不能有事,不能遭受牵连啊。这可是关系到天庭数百万年的大计!

袁琼都几乎已经绝望了。

他自己作为唯一炼蛊的蛊仙,都已经昏死过去。那么宿命蛊作为炼蛊的对象,能会没有事吗?

宿命蛊虽然不会因此而毁,但极有可能的是,之前修复的成果会损耗大半,甚至很可能被打回最初时的受损状态。

现在,袁琼都只能希望宿命蛊的情况较好一些,不能让他和天庭无数年来的努力全部化为乌有。

但当袁琼都抬起双眼,看到宿命蛊时,他自己都愣住了。

“这?!”

一团火焰取代了他,正替他炼蛊,宿命蛊安安静静地位于焰心之中,距离修复只差半步之遥。

震惊之后,袁琼都的心瞬间温暖起来。

这是意火!

用意志作为燃料的火焰,当初他少年的时候,从师父的传承中获得。后来他虽然修行炼道,也不忘炎道,将意火不断改良,发扬光大,超脱凡俗,达到仙招层次。

“意火早已经成为我最熟悉,最拿手的炼蛊手段。在昏厥的刹那,恐怕是我下意识地催动了意火!”袁琼都暗中猜测。

这种情况并非第一次发生。事实上在他一生的岁月里,无数次炼蛊的经历中,有许多次通过意火来救场。每每在他自己快要支撑不住的时候,就催动意火来暂时代替自己,稳住局面。

无数次后,在关键时刻使用意火已经成了他,一种下意识的深入骨髓的习惯。

袁琼都吐出一口浊气。

他静静地看着意火燃烧,心中不由暗想:“与其说是一种习惯,更准确地来讲,是一种责任吧。”

他不由地想到了他的师父,于是他露出了微笑:“责任……师父,看来我没有令你失望呢。”

袁琼都知道自己的情况,他此时伤势太过沉重,已经无法继续炼蛊了。但是希望就摆在他的眼前,他将用最后的生命和意志来助长这团意火。

“燃烧吧,燃烧吧。”袁琼都喃喃自语,他用他的性命点亮了意火!

意火静静燃烧,火焰明亮起来,非常的稳定。

在焰心当中,宿命蛊终于走出最后半步,彻底修复!

而袁琼都却已毫无生息。

就像他的师父,就像他的师姐,就像无数天庭的先辈一样,他们将自己的光和热,投入到天庭的伟业之中,奋不顾身地用自己的性命为天庭筑基!

我之仇寇,彼之英雄。

天庭的基业,偌大的中洲,确确实实是一寸江山一寸血。

意火最终消散。

“呵呵呵。”寂静的监天塔塔顶,响起从严虚弱至极的笑声,“袁琼都仙友已经做到了,现在该轮到我们了。”

“可是你我的状态,根本连传音交流起来都很困难呐,更遑论爬起身来催动监天塔了。”车尾传音,语调低沉。

从严却传音回道,似乎很开心的样子:“还有一个法子。我的体内仅剩下的仙元,还能催动我的招牌手段,你说巧不巧?”

车尾和从严乃是挚友,车尾立即明白了从严的话:“不要这样做,你汲取了我身上的伤势,你会立即死去。”

“死有何惧?眼下的情形,诸位同道在外奋战,战线岌岌可危,恐怕是抽不出人手进来的。车尾兄,你我相交多年,你的才干比我强得多。还有你的道,并未走完呢。真是期待啊,你即将开创的流派……可惜我不能亲眼目睹了。”

从严淡淡地说着,身上开始散射一抹淡蓝的光辉,光辉覆盖到车尾的身上。

两位蛊仙的肉身俱都破烂不堪,像是两摊骨头、血肉组成的垃圾。但伴随着蓝光的作用,车尾的肉身开始恢复,而从严却越发严重。

“吾友……”车尾闭起的双眼,缓缓睁开,默默地流淌出两行热泪。

一旁的从严已经毫无生息,车尾的伤势仍旧非常沉重,但他拼尽全力,却是能够勉强爬着来到宿命蛊的跟前。

“催动吧,催动起来……求求你。”车尾眼前一阵阵发黑,他将所有的仙元都灌输过去,然后只能默默祈祷。

\end{this_body}

