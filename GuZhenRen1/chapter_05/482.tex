\newsection{魂魄缺乏}    %第四百八十三节:魂魄缺乏

\begin{this_body}

%1
西漠,唐家仙阵空间。

%2
方源盘坐着,正闭目静养。

%3
心神扫视自家魂魄,原本千万人魂级的魂魄底蕴,此时已经跌落到了百万人魂,并且只是一百多万。

%4
“之前闯荡梦境中的第一幕,魂魄底蕴能剩余九百多万人魂。现在第二幕过去,居然只有一百多万,差一点就要沦落到了十万人魂一级。”

%5
方源在心中,默默比较两者差距。

%6
第二幕梦境比第一幕要冗长一些,并且魂魄的消耗程度更加严重。

%7
若是换做寻常蛊仙,魂魄底蕴稍差一些,根本就没有办法支撑。别说是一幕梦境,就连半幕都无法渡过。

%8
这一切都表明,方源眼前的这片盗天梦境,探索难度极大。也难为唐家和唐方明,这么多年努力,就像是一个刚出生的婴孩,来啃这么一块坚硬如铁的巨石。

%9
“唐方明仍旧能开创那么多的梦道凡蛊来,可见此人的确有着才情和天赋,并且相当不俗。”

%10
方源想到前世,五百年前世唐方明开创出了梦道仙招——盗梦。

%11
以前,方源以为这是影宗在幕后推波助澜,现在连续探索了两幕盗天梦境之后,他则是觉得盗梦这招,更可能是真的被唐方明开创出来的。

%12
第一幕盗天梦境,让方源偷道境界飙升到宗师级。

%13
第二幕梦境闯过,方源的偷道境界则上涨到准大宗师。

%14
“距离真正的大宗师境界,还差一步之遥。”方源静心体悟,就能感受到那种仿佛近在咫尺的美妙前景。

%15
“再探一幕盗天梦境,若是成功,我定然能够达到偷道的大宗师程度!”方源念及于此,饶是他定力非凡,也不禁心头丝丝火热起来。

%16
大宗师和宗师,是有巨大区别的。

%17
宗师境界,是晓阴阳乾坤,知宇宙奥妙,超凡脱俗,成就仙中之仙,贤上之贤。

%18
达到宗师境界,就能触类旁通。用自己主修的流派手段,模拟出其他流派的特点优势。比如黑凡真传中,有一招百年好合,这本是宙道杀招,却能达到信道的良效。

%19
而大宗师的境界,是对一个流派的见解已然近乎于道,对此流派的认知已经达到天地的极限。堪称穷尽了此流派的奥秘!

%20
已经逝去的焚天魔女就是炎道大宗师境界,还有当今的八转蛊仙雪胡老祖,也是雪道大宗师。

%21
到达这一境界,焚天魔女能够对炎道道痕,有着极其细致的敏感。因此之前,她在每一个炎道仙材上,都添加了一丝炎道道痕,然后暗算方源。

%22
雪胡老祖,则可以将各种仙道杀招随意拆解开来,寰转自如。之前在逆流河战役中,他就是巧妙地将招牌战场杀招,直接拆解开来,对付天庭等强敌,占据上风。

%23
而阵道的大宗师,就能直接利用天地自然中存在的道痕,进行布阵。往往只需要耗费一些凡蛊,就能在特定的地点,布置出仙级蛊阵。

%24
至于大宗师之上,还有一层境界,那就是无上大宗师。

%25
大宗师是穷尽天地此道的奥妙,若是方源成为偷道大宗师,那么他对整个天地的偷道奥妙,就会了然于胸,关于偷道几乎没有他不理解的。

%26
所以说,大宗师就是将天地间的某个流派,都解读尽了。

%27
无上大宗师则是要推陈出新,朝前跨越那么一步去,得到天地都没有的流派奥妙,直接领先于整个天地自然!

%28
宗师境界,一般需要上百年、数百年的积累,方源前世五百年,积累出了血道宗师境界。

%29
大宗师境界,则往往需要上千年、数千年的积累,一些八转蛊仙拥有此等境界。

%30
至于无上大宗师,那就是不是简单的积累问题,而是才情和天赋。没有这种惊世骇俗的才情天赋,就算活个上万年,都不能成为无上大宗师。

%31
所以,古往今来,无上大宗师的蛊仙极其罕见。

%32
世人公认十大尊者,都是无上大宗师。除去之前,还有一些传奇人物,比如世人皆知的炼道三位无上大宗师——长毛老祖、天难老怪、空绝老仙。

%33
“宗师境界、大宗师境界,我可以依靠梦境,不断飙升。但是无上大宗师境界,依靠梦境就不可能了。”方源心中了然。

%34
这是很明显的事情。

%35
无上大宗师境界,是开创,从无到有,而梦境中的这些,都是先人遗泽。所以依靠梦境提升境界,最高只能提升到大宗师一级。

%36
“一旦我达到偷道大宗师境界,就算盗天梦境还有许多,但是我却不能再在其中,令偷道境界上升了。”

%37
这也是方源和唐家合作的基础。

%38
这片盗天梦境太大了,方源一个人的胃口是有限的,他吃饱了,唐家这边还有大量剩余。

%39
所以,在方源表明了这个原因之后,唐家才愿意和方源合作。

%40
若是盗天梦境稀少,方源一个人独吞没有剩下,唐家怎可能会和方源合作呢?

%41
接下来的日子,方源就为他第三次探索盗天梦境做准备。

%42
重点就是修行魂魄。

%43
但在空暇之余,方源也指点唐方明,教导他梦道方面的秘诀。

%44
唐方明获益匪浅。

%45
尽管方源只是传授了他一些粗浅的梦道知识,但是对唐方明而言,他欠缺的就是这么一盏指路的明灯。

%46
万事开头难,任何流派的开创,最开始时是最艰难的。

%47
一旦开头过了,接下来按部就班,也会有迅猛的发展。

%48
对于方源而言,指点唐方明的原因,除了履行盟约,方便他继续探索梦境之外,就是布局未来,仿造幽魂谋略,扶植四域,形成包围中洲,对抗天庭的旷世大局。

%49
时间匆匆,二十多天宛若弹指一挥,就这样过去。

%50
方源的魂魄底蕴,重新晋升到了千万人魂级数。

%51
“第三幕盗天梦境,势必更加困难。千万人魂,并不安全。”方源思量后,决定继续修行,冲刺到亿人魂。

%52
若是魂魄底蕴不足,魂魄在梦境中消融耗尽,也就等若是死了。即便是还有一副肉身,没有了魂魄,就等若植物人。

%53
“或者我能够推算出梦道杀招,能够减少魂魄在梦境中的消耗。”

%54
这个点子在方源的脑海中一闪即逝,旋即就被方源自己否决。

%55
方源的梦道境界,普普通通,连准大师一级都算不上。梦道仙招?还是算了吧。

%56
那么摆放在方源眼前的,只有一条路,就是尽量地增长魂道底蕴,晋升到亿人魂。

%57
不过,他的这个计划却是很快便遭遇了阻碍。

%58
胆识蛊短缺!

%59
方源将荡魂山放在琅琊福地,琅琊地灵亲自领导,利用这座天地秘境来生产胆识蛊。

%60
然而,胆识蛊并非凭空而来,是需要消耗魂魄的。

%61
魂魄的质量、数量越高,荡魂山能够产出的胆识蛊就越多。

%62
问题就出在这魂魄上。

%63
方源需要的胆识蛊数量极多,琅琊地灵手中却是没有足够的魂魄,来充当生产原料了。

%64
之前胆识蛊能够连续大规模产生,是因为方源命令影宗等人,前往太丘探索,很是斩杀了一些荒兽、上古荒兽。

%65
现在影宗这群人,都在方源的身边,之前留在琅琊派的魂魄库存,也在他们离开的这些天里,消耗殆尽了。

%66
方源思量了一下,现在让他赶回去,屠杀太丘中的荒兽取魂,还不如就近取材。

%67
西漠和北原一样,同样是浩瀚无比。

%68
西漠中就有一处地方,比较独特,正适合方源。

%69
“尤婵已死,龙鱼生意已经制霸整个宝黄天市场。近几年内,应当没有人能够与我匹敌了。”

%70
方源打算将这些获益,用来经营仙窍,而不是收购魂魄。

%71
用来收购魂魄相当吃亏,方源想起一个至理名言——自己动手丰衣足食。

%72
于是,方源就将大体情况告知了唐方明,随后便带着影宗成员,暂时离开了仙阵,赶赴那片特殊的沙漠。

%73
ps:还未吃饭,待会去吃饭,今天应该是只有一更,累成狗了!又饿又困。

\end{this_body}

