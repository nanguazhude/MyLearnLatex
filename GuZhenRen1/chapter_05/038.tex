\newsection{狗屎分运}    %第三十八节:狗屎分运

\begin{this_body}

西漠,某一段商路中。

“杀啊,杀了这群畜生,钱财都是我们的!”

“抢了他娘的!!”

“守住,都给我守住,这些货物丢了,主家绝不会绕过我们!”

喊杀声震耳欲聋,围绕着一座低矮的沙丘,一场盗贼和商队之间的血腥拼杀,正在展开。

这只盗贼团伙,常年流窜在沙漠中作案,成员皆是人族蛊师,各个凶残勇悍,久经沙场,身手不俗。

而商队中,人族反而稀少,占据主体的还是羽民。

这些羽民蛊师,有的站在地面上,护卫商货,有的则飞在空中,和盗贼激战。

一*的火弹纷射,热浪滚荡,时而夹杂着风刃,在空中划过阴损的线路,砸在对方的阵型中,掀起一阵血花。

西漠中盛行炎道、风道蛊虫,蛊师们也以这两种流派最为常见。

片刻之后,盗贼们牢牢占据上风,损伤不大。而商队中,却是死伤惨重。

韩立夹杂在货物之间,已经浑身是血,眉目焦黑。

他在刚刚的激战中,不幸被一枚隐形的风刃射中,在胸膛处划了一道又长又深的伤口,现在还血流不止。而眉目间的焦黑,则是一颗火焰流弹,砸在他身边,滚烫的沙子四下飞溅,大半罩在他的脸上所致。

“可恶!难道我今天,就要将这条命交代在这里了吗?”战场情景,让韩立心生绝望。

他好不容易踏上了修行之路,侥幸成为蛊师。但被驱逐出来,只能四处流浪。

为了糊口,他应聘成为这支商队的一员,没想到这么大型的商队也会遭受劫匪的抢掠。

“逃啊!”

“这支商队完蛋了。老子才不和他们陪葬。”

“聪明的就跟我走,这些羽民完蛋了,就算是活着回去。也要被主家处死的。”

败局已定,商队中开始出现了逃兵。

都是受聘担当护卫的人族蛊师。

“这些家伙!”

“别管他们了。节省真元,就算是死,也要战死!”

羽民蛊师们咬牙启齿,愤懑不已。

“怎么可以这样?”韩立呆住了,他还很年轻,初入江湖,十分稚嫩,看着这些逃跑的人族蛊师。他有些反应不过来。

“你怎么不逃?”一位羽民蛊师走了过来,看到了韩立。

韩立啊了一声,有些手足无措,他认识这位羽民蛊师,他可是这个商队的首领。

“虽然你的修为低微了一些,但某些方面可比你的那些同族,要强百倍啊!”羽民首领感叹一声,拍了拍韩立的肩膀。

他的手掌闪着光,拍了韩立的肩膀三下,顿时就让他的伤势痊愈了!

“好厉害!这就是四转蛊师的威能吗?”韩立震惊。张口想要感谢,但羽民首领已经迈过他,奔向前线。

前线已经十分危急。羽民首领不得不亲自参战。

四转蛊师一出手,顿时杀得盗贼们人仰马翻,一时间伤亡惨重。

盗贼团伙中,也有四转强者,但此刻却坐镇后方,冷漠地看着,嘴角带着丝丝冰冷的笑意。

羽民首领很快统治了这个战场,大杀四方,看得韩立热血沸腾。[看本书最新章节请到棉花糖小说网www.mianhuatang.cc]其余羽民更是纷纷叫好。

但羽民首领心中,却是阴郁沉重。

他知道。对方心思残忍,是用这些炮灰来消磨他的真元。等到盗贼中的强者登场作战。他真元不如对方多,必定会落入下风。

呼!

就在这时,从天边传来剧烈的风声。

风声浩大,宛若荒兽嘶吼。

众人循声望去,顿时有人惊呼:“糟!是金丝龙卷风!”

西漠中的龙卷风,分金丝、银丝、铁丝、铜丝等等,威力各有大小。金丝龙卷风威力强悍,就算是四转蛊师,只要被陷入进去,也要凶多吉少。

羽民们一阵慌乱,盗贼团伙也乱了一下,旋即就像是爆炸的火药,纷纷呐喊咆哮,面色狰狞地向商队冲杀而去。

他们要趁着金丝龙卷风席卷这里之前,将这只商队拿下,然后携带珍贵货物,扬长而去。

“支持住!”羽民首领大喊。

龙卷风虽然恐怖,羽民陷入进去,九死一生。但他们天生能飞,处境会比人族蛊师要好得多。

这场惨烈的激战,进入白热化阶段。

几乎每一刻,都有人丧失生命。

韩立缩在货物之间防守。

他修为低微,并不惹人注目,悍匪们觉得他不足为虑。

韩立的确不足为虑,他浅薄的真元已经耗尽,和普通人没什么两样。

战场上的火弹、风刃也很少波及韩立,因为韩立身边的货物,让双方都心照不宣地选择收敛手脚。

龙卷风速度极快,之前还在天边,但很快就卷席到了战场附近。

大风呼啸,无数黄沙漫天飞舞,击打在韩立的身上,源源不断,又麻又痛。

“撤!”尽管非常不甘心,但明智的匪徒首领,还是大叫一声,选择了撤退。

劫匪们像是一阵风,带着浑身的伤,和赤红的双眼,来得快,去得也快。

“快!将这些货物尽量都搬走。”羽民首领身上伤势恐怖,但关键时刻,还是把货物放在第一位。

羽民们忙着抢救货物,没有人关心韩立。

龙卷风袭来,韩立身不由己,第一时间被吸入风中。和他同样遭遇的,还有不少羽民,以及大量的货物。

韩立被卷入风中,天旋地转,根本分不清东西南北。他身如柳絮,随风狂舞,时上时下,危在旦夕。

砰的一声,他不知道撞在石头上,还是货物上。一下子晕死过去。

也不知过了多久,他悠悠醒来。

“少年,你终于醒了。”一位老者盘坐在他的身前。虚弱地道。

“你,你是谁?”韩立神智还不太清楚。他扫视四周,发现自己躺在沙子上,身边尽是石头、尸体,还有随处洒落的货物。

“龙卷风停了?我居然活下来了?!”韩立楞了一下,旋即大喜。

“没有我救你,你如何能活下来?”老者笑了笑。

“韩立多谢老人家的救命之恩!”韩立连忙施礼,真诚道谢。

老者欣慰地点点头:“想当年,我未成仙时。也如你这般处境。今朝你被卷入风中,也是因我而起。我和田劲的赌斗,终于还是我输了……临死之前,老夫就将这一生真传,都付于了你罢。”

中洲。

剑断山谷之中。

“剑气蛊飞到哪里去了?”

“快追!”

“这只剑气蛊是我的,谁都别想抢走!”

一群蛊师大呼小叫,形成一股奔腾的人流,从谷内一头撞出来。

“小子,让开路!”

“挡路者死!!”

洪易刚走到剑断山谷的谷口处,就见到一大群蛊师疯子般。飞扑过来。其中不乏三转、四转的强者。

洪易脸色刷的雪白,连忙逃窜,让开道路。

一群蛊师卷起漫天烟尘。轰隆隆地从洪易身边奔腾而去。

“这,这是怎么回事?”洪易口中喃喃,还是一阵心惊肉跳。

身旁路人的交谈,算是替他解惑。

“听说那是一只四转的剑气蛊呢。”

“难怪引起这么多人哄抢啊。”

“我们也快进去,说不定也能碰到四转的剑气蛊。”

“想得美呢。这个剑断山谷才形成多久?四转剑气蛊,说不定也只产生一只。别异想天开了。”

“剑气蛊……四转?!”洪易心中震动了一下,目光中透射出羡慕之意。

他连忙动手,跑进山谷。

这山谷其实他早就探索过,只是之前。并不是如此地貌。

原来,义天山大战之前。仙僵薄青出世,从落天河源头河底。暴射出无穷剑光。

有一道剑光,就落到这里,将这里的山脉斩成两段,形成了山谷地貌。

这就是剑断山谷的由来。

起先,人们并没有过多在意,但很快,中洲蛊师们意外发现,这里的山谷中渐渐出现了野生的剑道蛊虫。

原来,当初的剑光并非寻常,每一道剑光都蕴藏剑道道痕,剑光落到这里后,剑道道痕就刻印在山谷中,形成了特殊环境,孕养出越来越多的野生剑道蛊虫。

纸包不住火,越来越多的蛊师前来这里寻宝。

刚刚的一幕,就是四转剑气蛊被人发现,逃窜中引动众人哄抢。

“四转野生蛊虫,可遇而不可求,我就算遇到了,恐怕也捕捉不了,太过危险了。”洪易一边心中思量,一边小心翼翼地在谷内搜索。

他渐渐远离人流,在一个不起眼的角落里,发现了惊人的一幕。

一只普通的毛毛虫,正在蜕壳。

但奇妙的是,毛毛虫的硬壳上,正闪闪发光。

洪易楞了一下后,旋即认出了这一幕,他心中狂喜:“好运道!我居然碰到了虫子升华为蛊的那一刻。这股逸散出来的气息……好强!不知道是四转,还是五转蛊?”

洪易不待毛毛虫完全蜕壳,连忙上前,将其捉拿到手。

立即转移,到了更安全的地方,他就催动真元,开始炼化。

片刻之后,他炼化成功,得到一只很古怪的剑道五转蛊虫。黑不溜秋,巴掌大小,像是缩小的铁制剑鞘。

五转剑鞘蛊!

“这只蛊有什么用?我虽然得了五转蛊,但我真元跟不上,根本用不了啊。”洪易正暗暗惋惜,忽然一只野生的剑道蛊虫,从一旁的草丛中钻了出来,悠悠地飞到他的身边,然后飞进他手掌中的剑鞘蛊内。

洪易惊得呆住,一时间都说不出话来。

南疆,无名小山。

夜幕笼罩,暴雨倾盆。

“商心慈,今天你就把命交代在这里罢。”一个矮壮的男子,背负双手,昂然走向小山之巅。

在那里,有三位蛊师,正据险防守。

一男二女。

其中一位女蛊师,伤势很重,是商心慈的婢女小兰。

另一位女蛊师,一头柔顺如瀑的青丝,肤白若雪,清丽绝伦,正是商心慈。

望着来者,商心慈惨然一笑:“商赑屃!想不到真的是你,你我何必同室操戈?”

商赑屃哈哈大笑:“父亲死了,大哥死了,嘲风也死了,只要杀了你,我八哥就能登上商家族长之位。所以你今夜必须死!”

商心慈听闻这话,像是被无形的巨力狠狠撞击了一下,身躯一晃,差点栽倒在地上。

她语气凄凉哀婉至极:“就为了一个族长之位,商蒲牢就真的如此冷血无情吗?”

商赑屃冷笑一声:“是,你是救过我八哥一命,但那又怎样?指望他知恩图报,将族长之位拱手让于你?哼,商家族长之位,是万人之上的权柄!你妇人之仁,怎可能争得过我八哥!”

商心慈摇头:“我救他时,从未想过什么族长之位。”

“正是如此,你才落到现在这个下场。哈哈哈!”商赑屃紧接着道。

“心慈小姐,何必与这种小人多说!”叶凡眉间扭成疙瘩,语气中对商赑屃厌恶至极。

“叶公子,你走吧。他想要的只是我的命,我给他便是。你本身局外人,不必掺这趟浑水。你快走!”商心慈推叶凡的后背,催促他离开。

“我不走!”叶凡大叫,“心慈小姐,解我困顿,滴水之恩涌泉相报,我怎能一走了之?”

“哼,说得很动听啊。不过你就算想走,也走不了。今天你们三个,都必死无疑。商心慈,你还是这么天真!我不把你们三个全杀掉,难道日后让幸存者来诋毁八哥的名声吗?”商赑屃说着,步步逼近。

叶凡紧紧咬牙,哪怕他身受重伤,站都站不稳,也要挡在商心慈的面前。

“嗯,是个汉子。”商赑屃淡淡地评价了一句,旋即一挥手,就将叶凡挥倒在地。

“你若是巅峰状态,我还顾虑你三分。可是你现在真元早就耗尽,一点威胁都没有。哈哈哈。”商赑屃大笑,走近商心慈。

商心慈缓缓闭上双眼,放弃了无谓的反抗。

临死之前,她的心底最深处,忽然浮现出一个男子的身影。

一个念头,不由自主地泛起如果能在死之前,在见他一眼,该有多好。

然而等待了片刻,却久久不见商赑屃动手。

商心慈困惑地睁开双眼,只见商赑屃一动不动的站在她的面前,距离她只有几步之遥。

但他僵如石像,一脸惊恐的表情,都定格住了。

“哼,因为些许权利富贵,就罔顾亲情,简直是我商家的败类!”一个女仙显现出身形。

“您是?”商心慈满眼惊异。

女仙欣赏地看着商心慈,温声宽慰道:“别怕,我是你商家先祖,商青青。我已决定,从即刻起,由你继任商家族长一职。”

ps:这章4000字,因此来晚一点。欢迎大家关注蛊真人微信公众号!(未完待续。)

\end{this_body}

