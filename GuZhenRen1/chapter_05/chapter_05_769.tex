\newsection{正道头疼}    %第七百七十二节:正道头疼

\begin{this_body}

%1
武家大本营。

%2
书房中,光线晦暗。

%3
武庸脸色铁青,直直的坐着,宛若一尊石像。

%4
他手中的信道蛊虫,因此之前的催动,还残留着一些余温。

%5
刚刚又是一场南疆正道内部的高层会议,热闹得仿佛菜市场,夏家、巴家的蛊仙几乎在尖叫,在高喊,他们的声音明显带着恐慌和无措。

%6
绝大多数的人都在嘈杂的议论,近乎争吵,似乎在宣泄着震惊和惶恐的情绪。

%7
“这怎么可能?!”

%8
“方源埋伏了我们,他大获全胜,我们的蛊仙都被他俘虏了!”

%9
“这下该怎么办?这在我们南疆历史上,根本就没有这样的先例啊。”

%10
“这事情传出去的话,我们南疆正道会沦为五域的笑柄!”

%11
“名声已经不重要了,重要的是我们的人,如今被他扣押在手上啊!!”

%12
“之前方源留下的信道蛊虫,不是说要我们等待他的开价吗?”

%13
“是啊,但他根本没有送来任何新的消息。我们……就这样苦等吗?”

%14
……

%15
武庸的耳畔,似乎还回荡着这些人的交谈。

%16
“哼,一群无能之辈,现在光是讨论,又能济什么事?”

%17
武庸不悦地冷哼一声,他的双眼眯起来,原本就细长的眼睛显得更加细长。

%18
一阵阵的阴森的冷芒在他眼中闪烁不定。

%19
而他的双拳则不自禁地握起。

%20
“这下麻烦了。”

%21
“方源这厮占据了绝对的主动!”

%22
“就算我族拒绝他的勒索,夏家、巴家肯定不会,他们失去的可是太上大长老。”

%23
“天庭的布置也没有起到什么效果。哼,枉我还特意通知了紫薇。”

%24
“不过,这不是天庭孱弱,而是方源此战的表现,实在太过骇人!”

%25
“在如此恐怖的战果下,纵观我南疆正道,也就池曲由稍有一些成色,临危不乱。”

%26
武庸清楚地记得,在刚刚的议论中,是池曲由一脸镇静,首次提出主动出击,仍旧要针对方源实施追杀的计划。

%27
虽然应者寥寥,但这才是正道的风骨!

%28
武庸根本不知道,池曲由只是强做镇定,他内心中充满了深深的震惊和恐慌。

%29
他假意提出这个继续追杀方源的计划,就是为了隐藏他自己。

%30
“方源居然赢了?!”

%31
“方源不仅赢了,还俘虏了两位八转,数位七转强者!”

%32
这是一个笑话吗?

%33
若是此战之前,池曲由一定会认为这是一个笑话。

%34
但他现在一点都笑不出来。

%35
池曲由不敢去相信,但陆畏因拿出来的证据,是铁证如山,由不得他不信。

%36
此战方源大获全胜,不仅引起南疆正道混乱,更是拿捏住了池曲由的把柄!

%37
因为,方源制胜奇招就是梦境。

%38
而这些梦境,好死不死就是池曲由交易给他的呀!

%39
也就是说,池曲由就是方源此战最大的助攻者,也是南疆正道最大的叛徒!

%40
池曲由之前,当然也想过方源利用这些梦境来对付南疆正道。

%41
但他思维受限。

%42
一来,是因为和方源交易,池家受益太大了。这份重利的诱惑,笼罩住池曲由的视野。

%43
二来,方源之前也利用纯梦求真体自爆,帮助他成功逃生。但整个过程,方源狼狈不堪,因为自爆后的梦境,方源无法操纵自如。所以池曲由觉得,纯梦求真体自爆的威胁是比较低的。

%44
“但是万万没想到,方源居然架设了一座宙道的超级仙阵!”

%45
“若是没有这座仙阵阻碍蛊仙腾挪转移,梦境如何能建功?”

%46
“宙道仙阵其实还不可怕,可怕的是这座仙阵居然隐匿至深,不是方源发动,根本看不出来。”

%47
池曲由想想就脑仁子生疼。

%48
他是阵道大宗师,对阵道当然有所了解。

%49
但凡每个流派的仙阵,都有各自的优劣。宙道仙阵的优势在于能影响阵内的光阴流速,方源正是依靠这一点,在大阵中大战几天几夜,把追杀他的队伍一锅端了。而外界只是过了数个时辰而已。

%50
宙道仙阵的优势,不在于隐匿啊!

%51
擅长隐匿的大阵,是暗道、虚道、变化道这些。

%52
宙道仙阵一旦铺设,就是杵在那里,一目了然,明明白白,很不容易隐藏自己才是。

%53
怎么到了方源手中,它就隐匿了呢?

%54
池曲由被这个问题暂时困扰住,与此同时,天庭中的紫薇仙子却已找到了这个问题的答案。

%55
她刚刚结束了一场推演,中央大殿中还残留着些许紫色的光澜。

%56
紫薇仙子深深地叹了一口气。

%57
“应当是影宗的遗泽了。”

%58
“影宗曾经分布五域各地,以僵盟为幌子,暗地里寻找到天然的光阴支流,并利用周围的宙道道痕,加以布阵。”

%59
到了大宗师,阵道蛊仙就能直接利用自然存在的道痕,进行布阵。

%60
但是布置出来的大阵,会因为自然道痕的分布状况,而有不同的优劣。布阵的蛊仙往往只能微调,不如用蛊虫布阵那般自由。

%61
“偏偏南疆那边的宙道大阵,就极为擅长隐匿,被方源利用起来。”

%62
“他又在里面增添其他的宙道大阵,最终利用梦境,将追杀他的南疆队伍几乎一网打尽!”

%63
紫薇仙子尽是阴霾之色。

%64
她有点后悔了。

%65
早知道会是这样的结果,她应该将陈衣、雷鬼真君的死,告诉南疆蛊仙的。

%66
琅琊福地一役天庭战败,紫薇仙子只是选择其中的一部分,将其公之于众。

%67
只要南疆正道得知了陈衣、雷鬼真君之死,必定会更加重视方源,绝不会只派遣两位八转蛊仙。(陆畏因是主动加入,不受南疆正道掌控。)

%68
这当然不是紫薇仙子失误,而是她之前也有一番考量。

%69
起初的时候,紫薇仙子还是保留期望,觉得陈衣、雷鬼真君还有生还的可能。所以只向天下公布了方源的秘密,而对这一部分隐瞒。

%70
后来时间拖久,紫薇仙子也就明白,陈衣、雷鬼真君应当是阵亡了。

%71
她选择将这个消息继续隐瞒下去,一来是不想助长方源的声威,二来是担忧其他四域的蛊仙对方源妥协。

%72
方源有谋害八转蛊仙的力量,这是一种质变的影响!

%73
其余四域不像中洲,他们势力交错,山头林立,堪称一盘散沙,他们很容易被方源各个击破,更容易相互之间达成交易,暗中妥协。

%74
紫薇仙子料的一点都没错,西漠唐家、南疆池家就是如此情形。

%75
紫薇仙子隐瞒这个事实,促使其他四域积极对付方源。不管战果如何,必定相互消耗,不管是哪一方有所损失,对天庭来讲都是个好事!

%76
与此同时,双方暴露出来的情报和线索,也能让天庭只是静静旁观,就大有收获。

%77
等到恰当时机,天庭算定方源的具体位置,全力出手,就有铲除这个魔头的机会!

%78
紫薇仙子的打算是很好的,她着眼未来,已经开始为五域乱战做准备。

%79
但是事情还是大大出乎她的意料。

%80
方源居然大获全胜,尽数俘虏了南疆诸仙!

%81
“方源虽然只有七转修为,但上一次,他利用了琅琊福地的底蕴。这一次则是利用了影宗的遗泽。”

%82
“此战结果,完全是一边倒,并且我方的布置都没有来得及生效。”

%83
紫薇仙子恨得牙痒。

%84
方源太过狡诈奸猾!

%85
要对付他,真的是越来越难了。

%86
紫薇仙子开始有一种力不从心的感觉。

%87
明明方源只是七转修为,但就是拿他没有办法。

%88
其中一个原因,就是他兼修的流派太多。

%89
比方说智道。

%90
方源智道造诣很深。

%91
紫薇仙子现在就算全力出手,也推算不出他的位置来,只能竭力搜寻相关线索。

%92
一般而言,不是智道的蛊仙,很容易会被推算出位置。

%93
天下的魔道蛊仙数量不少,但为什么很少有人像方源这样胡作非为?

%94
一旦他们这样干,恣意为祸四方,正道很快就能找到智道蛊仙,进行推算,找出他们的位置,加以通缉围捕。

%95
又比方说炼道。

%96
琅琊福地不是那么好吞的,很多蛊仙想吞都吞不了。

%97
甚至某些八转蛊仙也不能够。

%98
因为吞并七转的琅琊福地,至少需要宗师级的炼道境界!

%99
再比如说梦道。

%100
方源从未来重生,有许多梦道手段。

%101
义天山梦境大战,他就是凭借这些手段化险为夷,逃出生天的。

%102
还比如这一次埋伏战。

%103
方源就展露出了阵道的造诣。

%104
能够如此程度的利用宙道大阵,他的阵道境界至少是大宗师!

%105
“当然,这也有可能是影宗本就布置了大阵。但方源为何之前不用?这可能性较小。”

%106
“还有一种可能,那就是方源吞并了琅琊福地,当中的异人蛊仙有大宗师的阵道境界。”

%107
“线索不足啊……”

%108
紫薇仙子伸手按了按自己的脑门,眼中厉芒一闪。

%109
她立即起身,离开中央大殿,再次来到魔尊幽魂囚禁之处。

%110
搜魂!

%111
不久后,紫薇仙子满意得微微点头。

%112
她准确地搜刮出了魔尊幽魂的记忆中,有关于那处宙道大阵的秘密。

%113
“果然是一座隐匿极强的大阵,并没有召唤年兽作战的威能,后者应当是方源后增添进来的。”紫薇仙子沉浸于思考当中。

%114
她眼眸中的幽芒,原先只是一抹,如今却已悄然扩散成片。

\end{this_body}

