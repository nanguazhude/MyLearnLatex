\newsection{好尴尬啊}    %第二百九十节:好尴尬啊

\begin{this_body}

夏飞快见方源既然动用了仙道杀招,那么他也只好动用仙道杀招来对付。[www.qiushu.cc 超多好看小说]

纯白色的光球,速度并不快,一点都没有光道流派的迅捷优势。

但它飞在空中时,排挤空气,发出嗡鸣之声。

砰!

酒钢大小的光球砸在卜卦龟的龟背上,竟然有山岳般的重量,将方源变化的整个卜卦龟砸进了山地里去。

庞大的卜卦龟整个身体下陷,陷入地下一尺有余。

夏飞快的这招,的确厉害。向来光道都以迅猛著称,但夏飞快的苍元一击,却是反其道而行之,宛若一记重拳,只要轰中,庞大的力道常常会让敌方猝不及防。

不过,方源的卜卦龟背,却是完好无损。

“嗯?”一击之后,夏飞快检查战果,看得眼珠子都瞪圆了。

别说是什么深坑了,整个卜卦龟背完好无损,甚至连一点凹痕都没有。

当然是这样了。

夏飞快的这招,是用光道模拟力道,算是一记奇招。

而方源的卜卦龟变化,却是七转仙道杀招,以变形仙蛊、卜卦龟背仙蛊为核心。尤其是后者,方源单纯催动它,就能拥有不弱于上古荒兽卜卦龟的龟壳。

现在是仙道杀招,拥有这么多的辅助蛊虫,龟背上的防御威能,比真正的上古荒兽卜卦龟只强不弱。

上古荒兽向来皮糙肉厚,生命力旺盛,而龟类又是所有物种当中,最为坚硬的生命之一。

夏飞快这招苍元一击,没有效果,也不奇怪。

但对于当事人而言,这就有点夸张了。

“苍元一击直接无效,好硬的龟壳!”夏琢磨心中惊异。

夏飞快直接停手了,他感到了一种久违的情绪,那就是头疼。

望着这么厚的龟壳,他感到头疼,难以下手。

其实他打方源的其他地方,也就罢了,偏偏选择最硬的龟背打。当然了,这也有他故意试探方源实力深浅的想法在。

“该怎么办?”夏飞快飞快地思考着。

他虽然性情急躁,但切磋中,却能重新冷静下来。

但这个时候,方源又催动了第二个仙道杀招。[求书网qiushu.cc更新快,网站页面清爽,广告少,无弹窗,最喜欢这种网站了,一定要好评]

各种蛊虫气息逸散出来,有仙蛊也有凡蛊,方源张开口,吐出无数的小龟壳。

小龟壳的模样,和卜卦龟的龟背很相像。如果是将卜卦龟的头尾、四肢都去掉,再缩小无数倍,几乎就是小龟壳了。

大量的小龟壳,在半空中漂浮,自己旋转不休,速度很快。

“这是什么东西?”夏飞快停下思虑,谨慎起见,他向后飞退,和方源拉开距离。

轰!

他挥拳直捣,依旧是苍元一击。

但这一次,酒缸大小的沉重光球,却没有碰到卜卦龟。

那些拳头大小的小龟壳,就好像是闻到了腥味的苍蝇,又仿佛是蜂群,呼啦一阵,对准苍元一击,直接围拢上去。

砰砰砰……

小龟壳似乎很坚硬的样子,在对撞中,直接消弭了数百颗。

但苍元一击也被消耗殆尽,最终化为一蓬光辉碎屑,无奈地消散在了半空之中。

仙道杀招刚背!

这就是方源从之前的仙道杀招,改良而出,结合了自身卜卦龟的变化,在南疆,甚至是在五域中首次亮相登场。

这记仙道杀招的核心,当然就是六转仙蛊金刚念了。

不过,又增添了七转坚持仙蛊,还有大量的凡蛊辅助。

威力自然上涨了许多,没有枉费方源耗费这么多的精力和心思。

“啊!这看似龟壳,其实本质上却是念头啊。”夏琢磨眼中绽射奇光,一口道破了秘密。

他是智道蛊仙,对各种念头都很敏感。这正是他擅长的地方。

至于夏飞快修行的是光道,不是智道,纵然七转修为,但也是得到了夏琢磨的提醒,才明白过来。

“夏琢磨仙友好见识,好配合。”卜卦龟口吐人言,语含讽刺。

其实方源也没有想故意遮掩,只是结合了卜卦龟变化之后,原本的那些金闪闪的念头,就都变成了黑不溜秋的小龟壳了。

“小磨噤声,此战我定要打破他的龟壳。”夏飞快开口。

夏琢磨嗯了一声,不再说话。被方源讽刺,让他面色有些不好看。

夏飞快闭上双眼。

他开始酝酿。

各种蛊虫气息不断升腾,接连而起,很快蔓延开来。

红枣仙元剧烈消耗。

与此同时,他双臂直直舒展,慢慢高举,最后在头顶合十。

身上的光辉宛若流水般涌动,皆汇入到他的手掌中央。

还在酝酿。

光辉内敛起来,强烈的气息,却是不断膨胀,越来越危险。

方源将龟.头低垂,四肢稳稳地踩踏在地上,仿佛一座沉默的黑山。但与此同时,他的身边各种小龟壳,不断飞出来,数目极速增长。

原本只是近万,随后就突破两万、三万、四万……

增长的速度极其迅猛。

六转金刚念仙蛊,单独催动起来,就能有成千上万的金刚念头。

如今是仙道杀招,效果更胜一筹。

最关键的是,方源此时身上的智道道痕。

其实方源的智道道痕并不多,但是他有变化道痕啊。他的变化道痕很多!

因为他斩杀了不少变化道蛊仙,尽得他们的道痕积累。最主要的一笔收获,是在界壁中斩杀了包括武遗海在内的三位蛊仙,单单战斗所获,就让方源得到近五万变化道痕。

一千道痕是一倍增幅。

一万道痕是十倍增幅。

五万道痕就是五十倍增幅。

这是什么概念?!

寻常的七转蛊仙,身上的主修道痕,也就在一万到三万之间。

夏飞快是七转强者,光道道痕超出了三万。

但怎么能和方源比?

于是,夏飞快、夏琢磨两位蛊仙就目瞪口呆地看到,大量的小龟壳迅速蔓延,很快就漫溢出整个山谷,扩散开来,黑压压的,密密麻麻,像是蝗虫,又仿佛蚂蚁。

夏飞快一脸沉凝。

这些小龟壳数量一下子变得这么多,带给旁人相当沉重的心理压力。

但夏飞快却巧妙地运用了切磋的环境,按照切磋定下的规矩,方源只能防守,不能主动出击,所以方源不能操纵小龟壳,主动攻击夏飞快。

夏飞快稳住情绪,将这记仙道杀招成功催发出来。

光道仙级杀招锯轮光!

高举在头顶的双掌,猛地下劈。

一道刺眼至极的光轮,从夏飞快的两手掌心中飞了出来。

光轮见风而涨,迅速地膨胀成船只大小。

光轮不断旋转,周边锯齿密布。

“好!”夏琢磨见此,不禁暗暗喝彩,“龟背虽然坚厚,但用这招却是能割锯,再厚的龟背也能被切割破开。”

然而下一刻。

咻咻咻……

无数的小龟壳飞了上去。

啪啪啪啪……

锯轮光在无数小龟壳的自杀性的围剿下,才飞行了一小段路程,就彻底崩碎了。

夏琢磨:“……”

夏飞快:“……”

方源赞叹道:“好厉害的杀招,这一招非同小可,居然废掉了我六千多颗的念头啊。”

这话就像是一个巴掌,扇在夏琢磨、夏飞快的脸上。

夏飞快酝酿这记仙道杀招,需要不少时间。

一般而言,酝酿仙道杀招的过程,越短越好,因为很危险,容易被攻击打断。一旦被打断,催动杀招失败,反噬之下,蛊仙往往就会重伤。

夏飞快充分利用了切磋的规矩,安然催动出了这记仙道杀招。

但是!

在他催动的时候,方源也在催动杀招。

夏飞快催出一记锯轮光。

方源同时间,能催出三四万的小龟壳念头。

然后一记锯轮光,就被六千多的小龟壳念头围剿,崩碎了。

这笔账,傻子都能算清楚。

夏飞快寄予厚望的手段,居然如此结果……

好尴尬啊!(未完待续。)<!--80txt.com-ouoou-->

\end{this_body}

