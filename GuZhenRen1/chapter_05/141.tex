\newsection{魂道陷阱}    %第一百四十一节:魂道陷阱

\begin{this_body}

试用了一番道可道仙蛊,时间又到了方源去修行魂魄的时候。<strong>棉花糖小说网www.MianHuaTang.cc</strong>

他飞出云城,去往落魄谷。

不过,对于道可道仙蛊的运用,还未终止。

方源分心两用。

至尊仙窍,小南疆。

稀疏的森林,再无当初的荒芜景象。一座座的普通山峦,屹立各处,平均分部在小南疆之中。

一位力道仙僵,被方源操纵,携带着一些仙元,还有道可道仙蛊,飞驰在半空中。

片刻之后,它速度缓缓降下来,然后悠悠地落到一座小山上。

这座山,方源取名为封天山。

名字十分大气,整个山体却比较矮小,貌不惊人。山峰被方源逐步改造,山上布置着一个巨大的蛊阵,山腹中空,封印着一具仙僵之躯。

正是方源原先的肉身。

力道仙僵进入山腹,来到封印蛊阵的中心。

方源这具仙僵肉身,身高两丈,背后一对小型的蝠翼,青面獠牙,双目紧闭,八只手臂各不相同,分别是修罗臂,天魔臂,血鬼臂,梦魇臂,病瘟臂,地魁臂。

此刻,八只手臂,宛若孔雀开屏,伸展出来,手指头紧紧扣住地面。

因为被封印,仙僵的心跳,变得极为缓慢,大约一盏茶的功夫,才跳一次。

一道道的流光,从蛊阵中发出,宛若轻风,吹拂在仙僵肉身的表面这证明封印蛊阵运转良好。

方源操纵着力道仙僵,催动道可道仙蛊,星光汇集成流水,覆盖在方源仙僵肉身之上。

片刻之后,方源得到结果,心中冷哼一声:“果然是动了手脚!”

之前的谨慎,是完全正确的。

影无邪在方源仙僵肉身上,竟然留下了上万魂道道痕,比方源曾经主修的力道道痕都要多出许多来。

收起道可道仙蛊,方源操纵着力道仙僵。稍稍将封印蛊阵打开一些缝隙,随后力道仙僵用力在自己身上一扯。

一道蛊仙魂魄,发出凄厉的哀嚎声,被力道仙僵撕扯出来。

“放过我。放过我!”

“我愿意奉你为主,饶我一命啊!!”

蛊仙魂魄极力挣扎,似乎预感到了自己的末日。\&\#65288;\&\#26825;\&\#33457;\&\#31958;\&\#23567;\&\#35828;\&\#32593;\&\#32;\&\#87;\&\#119;\&\#119;\&\#46;\&\#77;\&\#105;\&\#97;\&\#110;\&\#72;\&\#117;\&\#97;\&\#84;\&\#97;\&\#110;\&\#103;\&\#46;\&\#67;\&\#9

方源操纵的力道仙僵目无表情,狰狞的右爪牢牢控制着蛊仙魂魄,岿然不动。

漆黑的山腹空间里。封印蛊阵时而闪过一丝流光,却不能带来光明和温暖,反而更衬得环境幽深冰冷。

蛊仙魂魄的哀嚎,更是为此增添了几分毛骨悚然。

力道仙僵缓缓行动,将蛊仙魂魄慢慢地按入到方源仙僵肉身之中。

蛊仙魂魄并不愿意,但奈何方源早有充足准备,强行将其塞入仙僵肉身里面。

随后,方源立即调动蛊阵,将封印蛊阵完全开启。

方源仙僵肉身原本一动不动,平躺在地上。宛若雕塑。但蛊仙魂魄进入之后,仙僵肉身便开始发生颤抖。

很快,颤抖的幅度越来越大。

猛地,颤抖全然消失,方源仙僵肉身静止不动,下一刻,他猛地睁开双眼,露出赤红如血的双目!

吼!

他张口怒吼,巨响声在山腹中回荡,宛若雷霆交鸣。

方源却是冷笑。受他操纵的力道仙僵伫立在原地,一动不动,俯视着脚下的仙僵肉身。

果然,受到蛊阵的制约。还有之前早就在蛊仙魂魄中动过的手脚,方源的仙僵肉身挣扎欲起,却感到无力酸软,根本连坐都做不起来。

“你想干什么,你究竟想要干什么?”蛊仙魂魄发出大叫。

他不明白方源的用意,更感到一股淋漓尽致的恐怖。

很快。蛊仙魂魄的凄厉叫喊,陡然拔高:“你,你对我做了什么?我的魂魄正在消融,你好狠毒的心,我只剩下魂魄,你还不放过我!!”

方源操纵力道仙僵,聚精会神地看着,对蛊仙魂魄的惨嚎充耳不闻。

“魂魄进入此中,会被消融么……”方源动用各种手段,对其侦查。

“嘶,好厉害的魂道手段。这个蛊仙魂魄才多久,就已经不行了!”方源微微倒抽一口冷气,心念一动,立即操纵力道仙僵,伸出右手,企图将仙僵肉身中的蛊仙魂魄再次抽取出来。

但下一刻,方源肉身仙僵的胸膛上,陡然浮现出一个鬼脸,对着方源桀桀大笑!

一股澎湃的力量传来,方源操纵的力道仙僵,连连倒退三步。

忙定睛一看,鬼脸已经消失无踪。

方源肉身仙僵再度一动不动,宛若雕塑。

方源操纵着力道仙僵,小心翼翼地接近,方源肉身毫无动弹的迹象。

方源接着进行查探。

之前强硬塞入仙僵肉身中的蛊仙魂魄,已经被消融得一干二净,一丝魂魄残片都没有剩下来。

“消融的速度真是迅猛,而且魂魄进入其中,就像是落入深邃的陷阱。算动用外力强行扯拽,也不能令其脱离!”

方源在心中迅速分析。

影无邪布置的陷阱,狠毒非凡,而且手段高超。

方源十分庆幸自己小心谨慎,没有冒然去钻入原先的肉身,否则后果不堪设想。

检查几遍后,方源再次催起道可道仙蛊,对肉身仙僵进行侦查。

“怎么回事?魂道道痕居然没有损耗,而且还上涨了两千多!”方源心头震动。

到此刻,方源终于对影宗的魂道手段,叹为观止了。

消融蛊仙魂魄,这是个巨大麻烦,但方源还有办法应付。

最笨的法子就是不断用蛊仙魂魄,塞入仙僵肉身之中,陷阱发动几次之后,自然而然就要效果减弱,最终彻底消散。

但,影无邪布置的这个陷阱,却是非同寻常!

它不仅能够让进入方源仙僵肉身中的魂魄,无法离体,对蛊仙魂魄剧烈消解,而且消解了蛊仙魂魄之后。它竟然能增益自身,弥补损耗,越战越强!

“着实厉害!如此一来的话,我连用蛊仙魂魄进行试探的举动。都要受到限制。除非是真正能有手段,对这肉身上的魂道道痕进行针对。否则的话,只会让这个陷阱变得越来越强大恐怖!”

方源咬了咬牙。

其实,东海的信道传承,具备针对道痕的诸多手段。

这点。从飞剑仙蛊上的信道道痕印记,就可以看出来。飞剑仙蛊是剑道道痕碎片,但是却直接往上面增添了信道道痕,考虑到异种流派的道痕互斥。做成这个事情,难度极大,八转蛊仙都未必能够做成这事。

其次,方源得到的那具准八转的人骨,也是一个鲜明的证据。

蛊仙死亡之后,身上的主修道痕就会缩进仙窍,仙窍形成固定福地。形成不了福地的话。道痕就会消散,归于五域世界。

但这位信道蛊仙,居然能将对手的主修骨道道痕,直接留在他的骨骼之上。手段奇特巧妙,有些匪夷所思。

最后,道可道仙蛊更是证明了,这位信道蛊仙对于道痕,有不少针对性的手段。

“完整的信道真传,恐怕真的有对付道痕的手法。可惜,我得到这份真传的时间太晚了。绝大多数的蛊虫,都已死亡。种种信道手段,也都遗失。”

世间之事,怎可能完美?

方源能得到完整的黑凡真传。已经是十分幸运了。

东海信道传承,因为布置得太久,而且明显是信道蛊仙杀死强敌,激战之后,伤重濒死前,仓促布置了传承。所以。真传内容保存得不够久。

虽然收获了道可道仙蛊,但方源还是需要增添信道手段,让自己不受各种盟约的限制。

“寻求专门解决盟约的信道仙蛊,是一个途径。但此事要保密,寻求琅琊派帮助什么的,是不用想了。更别提让琅琊地灵出手,炼制这种类型的信道仙蛊。”

信道传承不是那么容易遇到的,方源的信道境界也十分普通,根本无法自己设计仙蛊方,推陈出新。

东海之行,虽然收获不少,但方源原先的目的,并未达到。

他对信道方面,仍旧渴望、需求。

收回投注在至尊仙窍中的一半注意力,方源开始全神贯注地展开魂道的修行。

一个半时辰之后,他离开落魄谷。

每天的魂道修行,几乎雷打不动。因为魂魄底蕴的急剧增长,导致方源能够在落魄谷中支持的时间,也足足增长了一半。

不过现在,方源彻底见识到了仙僵肉身上陷阱的厉害,对于之前的魂修计划,产生了动摇。

影无邪既然知道方源拥有荡魂山、落魄谷,怎可能不考虑这个因素?魂修计划是正确的,但让方源积累到足够的魂魄底蕴,能够抗衡陷阱,恐怕需要的时间相当漫长,付出的精力更会庞大无比!

而方源的时间和精力,不可能无穷无尽。他需要将有限的时间和精力,详细分配,周详计划,考虑具体情况,进行修行和发展。

“不过信道的事情,现在要放一放。第五次地灾,已经近了。”

这一次地灾,和以往前四次不一样。

前四次,方源都是利用仙劫锻窍杀招,在北部冰原渡劫。但是这一次,方源不能在前往北部冰原,他必须寻找一个新的渡劫地点。

究竟该选择哪里才好?

几天后,楚度主动来信,提及要主动帮助方源渡劫,分润狂蛮真意。

一时间,方源陷入被动境地!

ps:感谢可爱的书友们的指正,前文修复了两个bug,一个是五大宗师境界,少算了一个星道,现在已经增添上了。第二个是血道道痕,有一万二的样子,来源于血道蛊仙郑驮。至于东海的丁齐,他是自爆,无法收益。(未完待续。)<!--80txt.com-ouoou-->

------------

\end{this_body}

