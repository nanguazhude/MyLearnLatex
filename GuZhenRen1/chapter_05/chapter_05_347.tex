\newsection{三气归来}    %第三百四十七节:三气归来

\begin{this_body}

%1
龙公说完这话,浑身气势陡然间变了。

%2
他刚刚击退紫山真君,又令影无邪的引魂入梦无功而返,气势惊天,宛若冲霄之巨龙。

%3
但现在他的气势,却开始一点点的向下降。

%4
说是“降”,其实并不妥当。

%5
而是夯实、浓缩。

%6
原本磅礴惊人的滔天气势,越缩越实,几个呼吸之后,龙公身上惊人的气势,全然收敛起来。整个人宛若万丈深渊,深不见底,神秘莫测。

%7
紫山真君如临大敌。

%8
在他的感受中,这种状态下的龙公,远比气势惊天动地的龙公更要危险!

%9
吼!

%10
龙公张开口,陡然发出一声龙啸。

%11
然后,他整个人在瞬间,消失在了原地。

%12
“不好!”紫山真君顿时心中一沉。

%13
龙公这是故技重施。

%14
他曾经依靠这个手段,不断移动到紫山真君的面前,施展乱龙拳杀招,将紫山真君死死地压入下风。

%15
换做寻常蛊仙,即便识破了龙公的打算,也未必能够反应得过来。但紫山真君不同,他修行智道,任何思考都非常迅速。

%16
几乎在龙公咆哮的同时,紫山真君就思考完毕,并且做出了应对。

%17
在之前的交手中,紫山真君就看破了龙公瞬移的奥妙,此刻他也故技重施,将周围的龙啸声音尽数消弭净化。

%18
果然,下一刻,龙公瞬移结束,重新出现在紫山真君的面前。

%19
但这一次,他并未成功近身,他距离紫山真君还有数百步的距离。

%20
但是对于龙公而言,这个距离已经足够。

%21
“人气归来!”他口中轻喝一声,对着紫山真君一招手。

%22
招收的动作,平淡无奇,好像是一个友人在随意地打招呼。

%23
但紫山真君却在瞬间遭受了重创。

%24
一大股的人气,从他的身上喷涌而出,并且不受控制地冲向龙公。

%25
眨眼间,就投入龙公的手掌心中,迅速消失不见!

%26
这么一大股的人气损失,紫山真君顿时眼窝深陷下去,脸色陡然间变得苍白无比,原本亮紫色的头发变得枯燥黯哑,一股发自心底最深处的虚弱感觉,充斥紫山真君的全身。

%27
紫山真君眼眸深处,惊骇的神光一闪即逝。当他听到龙公口中的轻喝时,他就立即想起来一个大名鼎鼎的气道杀招。

%28
三气归来!

%29
这杀招乃是元始仙尊创造,有两个最广为人知的特征。

%30
第一个特征是,非常难练。

%31
这个杀招极其复杂,蛊虫规模庞大,并且相互之间的运转变化,催动的步骤,多得能让智道蛊仙都感到头晕目眩。

%32
第二个特征就是风险巨大。

%33
因为杀招本身非常复杂,蛊仙催动时稍有差池,就会导致失败,从而反噬回来,殃及自己。

%34
若是杀招催动成功,不能打中敌人,这个杀招也会折返回来,伤害自己。

%35
三气归来的杀伤力,可是非常恐怖的。

%36
这点要从升仙说起。

%37
凡人升仙,讲究的是三气——天气、地气、人气。

%38
蛊师不断修行,从一转一路修行到五转,整个过程中不断地积累人气。等到升仙之时,蛊师再汲取天地二气,三气融合一体,蛊师精心调控,然后再运用本命蛊一炸,炸出仙窍来。

%39
蛊师的人气越多,能够融合的天地二气就越多。炸出来的仙窍,品级就越高。

%40
一般而言,十绝体质,就意味着人气浓重,远超凡俗。或者蛊师自己积累雄厚,不论哪种流派,有宗师境界,人气都会很雄厚。

%41
形成仙窍之后,蛊仙不论修行,还是经营仙窍,都要讲究三气平衡。

%42
天气、地气、人气,哪一种不能缺少,让平衡丧失。

%43
像方源,拥有逆流河这等天地秘境之后,至尊仙窍内天地二气损耗很大。就得时常的落下仙窍,吞吸外界的天地二气,补充不足,始终维持三气平衡。

%44
关于三气,不论哪位蛊仙,即便是尊者,都是万万马虎不得。

%45
天下五大域,天地二气微有差别。不是土生土长的蛊仙,汲取外域的天地二气的话,往往会对自身的仙窍造成损失,虽然可以暂时补充不足,但只能救急。次数一多,就绝对不行。

%46
蛊仙死在异域,也能形成仙窍福地,比如落魄谷一战,中洲许多蛊仙战死沙场,他们的仙窍就都在北原形成了福地,但是因为汲取了外域的天地二气,福地的根基还有各种资源,都损伤会很多。

%47
说这么多,就是想说明:天地人三气,对仙窍、蛊仙尤其重要。稍有不同,就会对仙窍造成严重的伤害。

%48
而这仙道杀招三气归来,就是针对此点,创造而出。它直接攻击仙窍根本。

%49
乃是连招、变招两相合一。因此,催动这种杀招的蛊仙,需要非常高的技巧。

%50
三气归来,共分有三式。

%51
第一式,人气归来。对象中招之后,人气大损,影响方方面面,底蕴大减。当然,若是没有人气,比如荒兽等等,这招就没有什么威能。

%52
第二式,地气归来。此战若中,蛊仙仙窍中的地气就被疯狂抽取出来。地气严重损失,三气平衡彻底失去,仙窍危如累卵。

%53
第三式,天气归来。这是抽取天气,连中三招,仙窍崩溃成了瞬间之事。

%54
紫山真君爆退。

%55
他现在已经中了人气归来,若是再中地气归来,劣势就大到难以承受的地步了。

%56
对于这招,紫山真君虽然知晓来历跟脚,但毕竟是首次面对,仓促之间难以破解。

%57
盖因这招非常复杂,越是复杂的仙道杀招,虽然催动起来,难度越高,但是对于敌人而言,也越难破解。

%58
凡事不能一概而论,都有利有弊。

%59
紫山真君既然无法破解,只好连施手段,他的身影在顷刻间,不断地分化。

%60
一时间,化为数百个紫山真君,猛地扩散出去,四处乱飞。

%61
同时,他的本体隐形匿迹,消失不见。

%62
三气归来虽然强悍无比,但若是打不中的话,也就失去了效用。

%63
不过,龙公却毫不动容,紫山真君的这个应对,早就在他的估算当中。

%64
当即,他冷笑一声,声音不大,却传遍整个战场:“晚了,紫山真君,你中了我的人气归来,接下来的两式,你怎么躲也躲不过去,只有受着!”

%65
“来,地气归来。”龙公顿了顿,再度招收。

%66
他说的话,并非虚张声势。

%67
明明龙公招手的方向并不对,龙公也没有刻意地去侦查,他并不知道紫山真君的真身何在,但是紫山真君仍旧瞬间中招。

%68
紫山真君动容!

%69
海量的地气,从他的仙窍中奔腾而出,宛若滚滚长江,滔滔不绝,一齐涌向龙公的手掌中,然后被其全部收纳,消失不见。

%70
紫山真君的仙窍损伤极大,根基彻底折损,三气平衡早已失去,仙窍崩溃只在几个呼吸的时间,就会发生。

%71
这个时候,一个不得已的选择,横亘在紫山真君的面前。

%72
那就是落窍。

%73
在这里直接落窍。

%74
如此一来,吞吸天地二气,弥补三气平衡。

%75
但这个选择,紫山真君立即选择了割舍。

%76
这个应对糟透了!

%77
因为紫山真君一旦落窍,他的本体就会出现在自身仙窍当中,等若暂时抽离战场,彻底丧失了主动性。

%78
而且还有关键的一点,仙窍洞天只有打开门户,才能向外汲取天地二气。

%79
这是必要的沟通通道。

%80
但是在外面,还有龙公虎视眈眈。

%81
若是仙窍门户一开,龙公直接冲杀进来,随意之间,就能对仙窍造成极其恐怖的伤害。并且他能打能退,反而紫山真君只能被困在自家仙窍当中。

%82
“就算是落窍,弥补了地气,我的三气平衡也已经丧失了,因为人气太少。而要补充人气,我却没有相应的人道手段。唯有收购进来大量的人族或者异人,才可以。”

%83
紫山真君的脑海中各种想法,电光火石般闪烁而过。

%84
在很短的时间里,他有了决断!

%85
他猛地伸出手来,一把抓住自己的小腹,然后他的手掌泛起诡异的紫光,一寸寸地拔高,手掌心离开腹部。

%86
随着他的这个动作,他的仙窍竟然也被他取了出来。

%87
见到这一幕,方源的心头顿时涌起了一股熟悉的感觉。

%88
他很快就想到了焚天魔女。

%89
焚天魔女也有一招,那就是用炎道的方法,直接取走他人的仙窍。

%90
这招曾经给方源留下了深刻的印象。

%91
“紫山真君如何也会用这样的手段?”方源小小疑惑了一下,但旋即便又释然。

%92
焚天魔女本身就是僵盟成员,黑楼兰又被影宗牢牢控制,紫山真君掌握这样的手段,不足为奇。

%93
况且天下流派,修行到高深地步,都能做到触类旁通,殊途同归的效用。

%94
紫山真君的这一招,明显是智道手段,或许是他自己研发,也未必和焚天魔女有关。

%95
紫山真君毅然取出自家仙窍,然后将其放置到外界去。

%96
仙窍门户洞开,开始猛烈地汲取地气。

%97
但这个洞天自救的希望,极其渺茫,就算地气平衡了,人气实在太过缺乏。

%98
见到紫山真君有如此决断,龙公也不免讶异了一下,旋即赞叹之情在他的脸上一闪即逝。

\end{this_body}

