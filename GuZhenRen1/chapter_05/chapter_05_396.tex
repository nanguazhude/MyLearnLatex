\newsection{红云追来}    %第三百九十六节:红云追来

\begin{this_body}



%1
方源无奈。

%2
他无法逼迫琅琊地灵做什么。

%3
现在,卡在这里,即便他想要继续推算仙阵洁身自好,没有关键仙蛊,当然不行。

%4
“关键是,我还需要借助琅琊派的力量,帮助自己炼蛊。”

%5
方源得了紫山真君的遗藏,缺少很多仙蛊,未来他在炼蛊方面,有着一大堆的计划。

%6
如果琅琊地灵思考良久,仍旧拒绝方源,该怎么办?

%7
这个问题,方源早已经想过。

%8
他向来行事,都是未虑胜先虑败。

%9
“如果琅琊地灵不同意借出阵盘仙蛊的话,那我就只好借用其他人的仙窍了。”

%10
方源身上道痕众多,要进入其他影宗蛊仙的仙窍当中,分外困难。

%11
除非他们落窍,形成福地,根基稳固,方有这种可能。

%12
方源进入他人的仙窍当中后,便可采用宙道手段,将其仙窍的光阴支流扩张数倍,令其光阴流速暴涨。

%13
如此一来,外界一日,光阴支流中就很可能过了十天半个月。

%14
借助这样的时间,就能够争取到大量的宝贵时间。

%15
然后,在福地中布置出仙道蛊阵洁身自好,让方源置身其中,针对本身的不利道痕,进行删减。

%16
此法是方源走投无路之下,才会采用的计划。

%17
因为它的弊端同样很大。

%18
落窍之后,等若位置固定下来,这个时候凤九歌等人追杀过来,就很危险。凤九歌手中有没有突进仙窍福地的手段?

%19
很可能有。

%20
就算凤九歌自己没有,他背后的灵缘斋,乃至天庭,都不会让他出现这样的短板。

%21
还有一点更加重要,一旦宙道手段实施下去,仙窍中的光阴支流发生变化。那么这个变化,在短时间内,不能再次修改。

%22
举个例子,方源若让黑楼兰的仙窍光阴流速暴涨,那么至少在未来的数月内,不得再次修缮光阴支流。

%23
黑凡的宙道真传中,不管是度日如年,还是度年如月等等杀招,只要是延缓或者加快仙窍中光阴支流的流速,都有这样的弊端。

%24
因为这些杀招,都是一次性催发,长久有效。在某个时限之内,连续影响光阴支流,会导致宙道道痕紊乱,光阴支流发生暴流的灾害。

%25
方源已经知晓每个蛊仙的灾劫时间,他计算了一下,发现只要采用这个方法的话,就得有人必须承受仙窍灾劫的考验。

%26
这就难了。

%27
要知道,天意一直以来,都是想方设法地来剿杀方源的。

%28
方源虽然选择其他人的仙窍,但天意洞悉一切,不管是谁,都和方源关系密切。渡劫时,天意肯定会爆发出极限威能,来屠杀渡劫的蛊仙。

%29
方源等人可以联手,共度灾劫。

%30
但别忘了还有天庭的追兵。

%31
算清楚时间和速度,落窍布阵的方法,肯定不能一次性地帮助方源,除尽身上的侦查杀招。这样一来,方源必定会在期间,承受一次仙窍灾劫,并且极可能遭遇到天庭蛊仙的围杀。

%32
这就太冒险了。

%33
风险太高!

%34
若是让方源动用自己的至尊仙窍,那就更加危险。

%35
方源心中还有一层担忧:“上一次只是来了凤九歌,下一次指不定有多少天庭蛊仙呢!”

%36
所以方源这段时间,都是四处转移,不给天庭蛊仙合围的机会。

%37
西漠那么大,地域宽广,天庭又是异域作战,方源等人又拥有四通八达这样的手段,要找到围杀方源的机会,并不容易。

%38
到了约定的时间,远处的天空飞来几个身影。

%39
正是白凝冰、黑楼兰等人。

%40
一般方源在修行或者推演的时候,她们几位都会散布开来,为方源防守侦查。

%41
“方源,你那仙道蛊阵究竟何时功成?这样的逃亡,还要持续多久?”这一次,白凝冰飞下来,眉头皱着,不太耐烦地问道。

%42
方源不用扫视周围,都可以感觉到,其他蛊仙的目光也都因为这句话,向他投射过来。

%43
“再等一等。”方源语调平淡。

%44
白凝冰冷哼一声,面色不愉。

%45
她是所有人当中,最不耐烦的一个。这种逃亡的日子,一点都不精彩,让她感到非常的无聊。

%46
“你不是能够布置那道仙级蛊阵吗?再布置一座,将天庭的追兵吸引过来,围剿掉他们。”白凝冰建议道。

%47
方源摇头:“你以为这个仙级蛊阵是那么容易布置的?得需要一条光阴支流才行。”

%48
之前陷害凤九歌的仙道蛊阵,来自黑凡真传。

%49
黑凡设计此阵的原本用意,是为了防范太古年兽。

%50
因为使用似水流年仙蛊,可能将引来太古年兽,顺着光阴支流,进犯蛊仙仙窍。所以黑凡的这道仙级蛊阵,就是布置在仙窍的光阴支流上,一旦有太古年兽侵犯,蛊仙自己又分身乏术,就引爆这个仙级蛊阵,将整个光阴支流摧毁,从而使得太古年兽没有通道可以入侵仙窍。

%51
当然,这个方法,是万不得已的时候,最后的关头,才施展的底牌。

%52
光阴支流一毁,整个仙窍世界就都静止下来,想要再引入一道光阴支流就非常麻烦,需要一道外界的光阴支流才可。

%53
“我和你的心情一样,被追杀的感觉,并不好受。”

%54
“不过,我宁愿先将仇恨积蓄在心中,等到时机成熟,反攻回去,给那些追杀者深刻的教训。”

%55
“你要期待未来的精彩,不是么?”

%56
“哼。”白凝冰不再言语。

%57
“好了,我们转移吧。这个地方已经不安全了。”方源话音刚落,白凝冰等人的脸色骤然变化。

%58
高空处,一朵红云俯冲下来,上面载着红云舞娘。

%59
她的脸上,粉纱飘飘,一双妙目将沙丘上方源等人,都映入眼帘。

%60
“翠娘,这些人是谁?你怎么和他们厮混在一起?”红云舞娘当先开口,责问出声。

%61
她口中的翠娘,自然指的是翠波仙子。

%62
“红云?”影无邪诧异了一下,然后脸上浮现出恰到好处的欢喜之色,继而问道,“你怎么来了?”

%63
红云下降的速度变慢,红云舞娘开始审视眼前的“翠波仙子”。

%64
影无邪的情况,乃是魂魄夺舍了翠波仙子的肉身。往往会留下破绽,毕竟魂魄变了另外一个,总会和新的肉身不搭配。

%65
若是换做寻常蛊仙夺舍,红云舞娘势必能看出破绽来,但是现在她面对的却是影无邪。

%66
影无邪是谁?

%67
他是魔尊幽魂的分魂之一,魂魄底蕴比方源还要雄厚数倍。

%68
魔尊幽魂玩的就是魂道,他的很多分魂,都有不同的身份,混迹在五域各处,组建势力,长年以来都毫无破绽。影无邪自然也有遮掩夺舍破绽的手段了。

%69
果然,红云舞娘再使用一些侦查手段后,始终没有发现“翠波仙子”身上有什么不妥之处。

%70
她的脸色缓了缓:“你出了事,你的命牌蛊、魂灯蛊都有异状发生,我便赶来援助你。没想到你状况还挺不错,叫我和夫君大人都白白担心了。”

%71
然后,红云舞娘再次问道:“他们是什么人?咦,有点眼熟。”

%72
红云舞娘的目光,停留在了方源的脸上。

%73
然后她面色一变,流露出吃惊的神色:“柳贯一?”

%74
红云舞娘背靠千变老祖,自然情报方面远比寻常蛊仙要灵通得多。逆流河一战,柳贯一的面貌,广传天下,并且随着时间推移,将会被越来越多的蛊仙知晓。

%75
而柳贯一的样貌,便是方源这具至尊仙体的本来相貌。

%76
这一次,红云舞娘出现的非常突然,方源没有来得及变化其他面貌。

%77
“是我,不知仙子怎么称呼?”被发现了身份,方源一片平静,语调冷淡地道。

%78
红云舞娘不忙答话,而是首先驾驭座下的红云,往后挪了一段距离。

%79
面对柳贯一,她满脸凝重之色,心里生出不小的压力。

%80
这可是能力战八转的存在!

%81
前些时候,名动天下,和凤九歌一时瑜亮。

%82
“幸好自己身上,有着夫君的意志和八转仙蛊,否则的话……”红云舞娘心中庆幸,同时眼珠子微微一转,对翠波仙子道,“翠娘,枉费我和夫君都惦念着你,你却和异域蛊仙混迹在一起,更过分的是,连信都不回一个。”

%83
红云舞娘自然有她的心思,这样说,无疑是挑拨翠波仙子和千变老祖的关系。

%84
千变老祖平素里十分宠爱翠波仙子,红云舞娘心生嫉妒,当然不会放过这个上佳的机会,来给翠波仙子上点眼药。

%85
她这话,是对影无邪讲的,其实更多的是对身上的千变意志。

%86
方源目光微微一沉,白凝冰、黑楼兰不着痕迹地对视一眼。

%87
红云舞娘的这番话,明显有点不把堂堂的柳贯一放在眼里。她定然是有所依仗的,不是表面上的这些战力这么简单。

%88
方源、白凝冰、黑楼兰都体味过来,影无邪、白兔姑娘还有妙音仙子,则没有想到这一层。

%89
“好了,说说你的遭遇吧,翠娘,为夫可是非常的好奇。”这个时候,千变老祖的意志,从红云舞娘的身上,浮现而出。

%90
“果然有底牌。”影无邪心中一突。

%91
“关键的时刻到了,就看影无邪如何蒙混过关!”方源眼底闪过一抹精芒。

%92
白凝冰等人却是有些紧张。

%93
影无邪虽然成长很快,但到底年岁不长,他的演技可远远不如方源,要在千变老祖的意志下不出破绽,这是相当困难的!

%94
他能行吗?

\end{this_body}

