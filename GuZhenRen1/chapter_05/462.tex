\newsection{铁龙鱼}    %第四百六十三节:铁龙鱼

\begin{this_body}

毛六的问题,比方源想象中的要更有难度一些。

他开创的这个炼道杀招,名为炼蛇寿元花,不仅蕴含炼道奥妙,更牵扯了律道中的因果。

毛六能创出这一招,也是从琅琊派的库藏中,一个炼蛊残招上得到的启发。

方源推算了片刻,就发现:炼蛇寿元花并不那么容易破解,若是一个不小心,还会令毛六当场丧命。

所以,方源只好改变策略,徐徐图之。

“若是我的炼道境界能够再高超一些,推算出破解炼蛇寿元花的时间,就会减少许多。”

方源不会像之前推算天消意散炼道仙阵那样,集中全部的精力和时间来攻坚。所以,关于毛六的这个推算,无疑要很长一段时间,才能见到成效了。

不过,毛六这边也并不着急。

他只要不再运用这一杀招,就能维持下去,不会丧命。

方源在推算的同时,继续魂道的修行。

荡魂山、落魄谷不愧是两大魂修圣地,方源本体的魂魄底蕴,再次踏上飙升的道路。

不过分身却没有进行这方面的修行。

原因在于道痕互斥。

荡魂山、落魄谷两大圣地,充斥魂道道痕,不管是胆识蛊还是落魄风等等,在本质上都是给方源的魂魄增添魂道道痕。

分身里的魂魄,方源早已特意划分出极限的份量,再增加魂魄底蕴,会对宙道的修行产生干扰。

所以,方源就连鬼官衣都不给分身使用。

至于如何防备天意,方源手中当然有相对应的宙道法门,只是效果要差于鬼官衣。

但宙道法门,和分身主修的方向是一致的,因此利大于弊。

而方源本体因为道痕不互斥,所以各种流派都能兼修。[\&\#26825;\&\#33457;\&\#31958;\&\#23567;\&\#35828;\&\#32593;\&\#119;\&\#119;\&\#119;\&\#46;\&\#77;\&\#105;\&\#97;\&\#110;\&\#104;\&\#117;\&\#97;\&\#116;\&\#97;\&\#110;\&\#103;\&\#46;\&\#99;\&\#111;\&\#109;然而即便如此,不同流派的杀招之间,也要注意。比如鬼官衣的效果就被鬼不觉干扰了许多。

除了推算毛六问题,以及宙道防备天意的杀招之外,方源还要改良运道杀招。

智道、剑道、魂道的杀招,都已经改良到了他能力的极限,现在轮到运道了。

方源拥有许多优秀的运道传承,除开那些杂七杂八的之外,最主要的当属巨阳仙尊的我运真传、众生运真传。

其中,前者方源获取了完整的内容,后者则是残缺的。

而方源的运道仙蛊,则有****运、气运、察运、连运、时运,清一色的六转,这对方源而言不大合用,毕竟方源已经是七转层次了。

运道仙蛊方源暂时不会升炼,他现在炼蛊方面的重点只有一个,仍旧是春秋蝉!

六转的春秋蝉只能令六转蛊仙重生,方源如今修为高达七转,必须要有七转的春秋蝉。或者,以六转春秋蝉为核心,组建出七转层次的仙道杀招来。

升炼春秋蝉还处于起步阶段,毛六负责整理全部仙蛊方,并且开始采集必要的蛊材。

方源没有忘记杀招的练习。

魂道杀招,他早就改良完毕,如今更是演练纯熟。先前的智道杀招、剑道杀招等,方源也会隔三差五地拿出来,温习一遍。

方源非常明白,只有平时不嫌辛苦,真正下苦功,将来在战斗的时候,才能如臂使指。若是关键时刻催动杀招失败,掉了链子,那就尴尬了。

练习的所有杀招当中,智道杀招石洞天机是重中之重。

这个杀招以天机蛊为核心,能够洞察出蛊仙的下一次灾劫内容。当然,它也有能力上的极限,比如现在,方源就不能推算出他的下一场浩劫具体是什么。他需要再等一段时间,才能够推算出来。

除了石洞天机之外,方源开创的天消意散杀招,也需要练习。

这个杀招,可以说是方源在最近这段时间里,推算杀招的显著成就。方源是绝不会放弃的。

石洞天机,能够让方源知己知彼,揣摩出天意的想法。而天消意散,则是更直接的反制手段,只是它还需要改良,因为酝酿的时间较长,目前只能用于幕后,不能用在分秒必争的战斗当中。

红枣仙元勉强够用。

仙元石储备早已经干涸见底。

而影宗成员,各个都发展的不错。借助太丘,他们逐步积累,之前南疆逃亡战的损耗已经弥补回来,甚至还有精进的地方。

他们捕猎的荒兽、上古荒兽等等,方源只取走魂魄,留下尸躯。这些都是仙材,能卖出好价钱。

方源不动她们的利益,一方面是有点看不上这些蝇头小利,另一方面也是为了维护影宗的秩序。关键时刻,方源可以找她们商借资源,用来救急,这点方源也不会含糊。

不过现在就动,无疑是破坏了他们的发展节奏,对于大局而言,明显是弊大于利。

虽然继承了影宗传承,对天庭了解增多,但仍旧有很多方源不知道的地方。面对这样的庞大巨物,方源五百年前世影宗的策略,就是值得效仿的最好榜样了。

所以,方源现在除了自身修行之外,也开始布局。

他现在和五域都有多多少少联系,北原这边,虽然遭受长生天等正道的通缉,但是异族大联盟却是因为之前的订婚,而彻底融入进去,现在方源和他们的关系紧密得很。

现在的难点是,方源缺少资金。

一连串的炼蛊、修行,锻炼杀招等等,令他手中的资本剧烈消耗。尽管省吃俭用,又有红枣仙元补充,但方源前进速度十分快速,现在他的资本入不敷出,已经跟不上方源的发展需求了。

在这方面,方源将突破口放在龙鱼生意上。

但在不久之前,尤婵居然也贩卖出铜龙鱼,这让方源稍稍意外了一下。

方源很快就接受了这个事实。

龙鱼和胆识蛊不同,新品上市,是可以破解其中的奥秘,随后自己培养出来的。

这就相当于方源之前,改良光照菌一样。

他可以改良别人的物产,别人自然也能改良他的。

像胆识蛊这样的生意,除我之外,别无他家,这才是每一个蛊仙都十分渴求的。

所以,铜龙鱼在其他人手中出现,并不是一件奇怪的事情。方源只是稍微讶异了一下,尤婵破解的速度。

忽忽几日过去。

仙道杀招――度年如日!

至尊仙窍中的光阴支流,迅速扩张,河水流速立即飙升上去。

至尊仙窍中的宙道资源本就极高,如今再加上这记度年如日,让光阴流速差距拉得更大。

如此一来,各种灾劫就来来临得更加频繁。

不过方源怡然不惧。

什么地灾天劫,他都已经无视。至于浩劫,有天机蛊在手,又掌握了石洞天机,他怕什么?

宙道资源提升,带动的是各种其他资源的产量。

尤其是龙鱼。

龙鳞海域中,海量的龙鱼争相游动着。绝大多数都是普通的赤红龙鱼,大约三成是铜龙鱼,除此之外,还有一抹黑沉之色。

“这便是铁龙鱼了。”方源笑了笑。

有这座大阵,他的龙鱼根本无需自己动手研究,就能自行改良。

“这一次,我将铁龙鱼投放出去,不知道尤婵会如何反应?”(未完待续。)

\end{this_body}

