\newsection{天消意散}    %第四百五十七节:天消意散

\begin{this_body}



%1
东海海底。

%2
“真是惭愧,尤婵你千里迢迢而来,最终我却没有帮到你什么。”华安苦笑,对着尤婵道。

%3
两人此时分别在即。

%4
尤婵来华安出寻求帮助,但是由于方源完善的智道防备手段,使得华安不仅没有测算出什么来,还将尤婵自己的情况首先暴露了出去。

%5
听到华安颇显惭愧的话语,尤婵摇了摇头,她望着华安两鬓的白发,诚恳地道:“华安,你不要这么说。你为了我的事情,不惜损耗寿元,进行推算,此情此意,我尤婵铭记在心。之所以没有成功,只是对方强大而已。”

%6
华安点点头:“你这一次的对手,很不简单。不管如何,我这一次没有帮到你什么,将来若是有什么需要帮助的,尽管来找我。只要你不嫌弃我智道造诣低微。”

%7
“呵呵呵,你呀。”尤婵笑了笑,华安的保证让她眼前一亮,心中的阴霾也消散了不少。

%8
两人告别。

%9
尤婵在高空中疾飞。

%10
迎面而来的风,吹的她衣袖飘飘。

%11
借着这层冷风,她冷静如冰。

%12
“对手虽然很强,但我不会就这样放弃!”

%13
“虽然华安没有推算出什么来,但并不代表我就没有力量。是时候该用我自己的手段来反击了。”

%14
尤婵下定决心,望着海天一色,满脸坚毅之色。

%15
尤婵有了新的举动,方源很快就得到了情报。

%16
他也一直在关注这尤婵。

%17
尤婵贩卖龙鱼,新添了一个规矩。只要是老顾客在她这里购买龙鱼,就有折扣。为其十年,只要连续几年都在她处购买龙鱼,折扣就会越来越多。

%18
此法一出,她立即稳定了局面。之前很多被方源吸引走的买家,也都回归她处。

%19
尤婵此法,充分利用了自家优势,就是欺负方源在龙鱼生意这块,是个地地道道的新人!

%20
相同的方法,若是方源照搬,效果绝对没有尤婵的好。

%21
尤婵不仅这样做,同时还加大了龙鱼的储货,她向宝黄天中灌输了更多的龙鱼,几乎堆砌成山,各种荒兽龙鱼可谓成群结队。

%22
她这是在赤裸裸地展现自家雄厚的资本。

%23
这双管齐下,顿时收到极好的效果。

%24
宝黄天中的舆论氛围,几乎完全倒向尤婵。

%25
“尤婵到底是第一巨头,没想到她在龙鱼这块,实力是如此的雄厚。”

%26
“那个挑战者,估计现在也有点傻了吧。不管他是谁,和尤婵还是有着差距的。”

%27
“尤婵经营日久,她的第一宝座,不是那么好挑战的。”

%28
“还是再观望一阵吧,我总觉的这个神秘的挑战者不是那么简单的。”

%29
也有少数的蛊仙期待方源的反应。

%30
不过在接下来的一段时间里,方源都没有任何的动静,好像根本没有接到尤婵反击的消息一样。

%31
方源仍旧按照之前的规矩,贩卖龙鱼,虽然刚有起色的生意,遭受到了巨大的打击。

%32
“尤婵目前的实力,远超于我。我此刻实力单薄,若是和其死磕,非是智者所为。”就连方源自己,见识到尤婵的手笔之后,也不得不承认,自己在龙鱼生意的底蕴上要差了尤婵不止一筹。

%33
方源豢养龙鱼才多长时间,尤婵比他的时间要悠久得多。这是积累方面的差距,方源短时间之内,是追赶不上的。

%34
尤婵的反击和挑衅,方源都忍耐了下来。

%35
这让原本那些观望的蛊仙们,彻底放弃了对方源的期待。

%36
惟独尤婵本人去,却没有丝毫的侥幸心理。

%37
方源的不动声色,让她感到颇为意外,更感到一种担忧。

%38
未出鞘的剑往往最恐怖,因为若是方源有所动作和反应,尤婵就能根据方源的举止,来筹谋如何反击。但方源现下一动不动,尤婵就不知道是如何酝酿计谋,有担忧情绪是自然的事情。

%39
“那我该趁胜追击吗?逼迫他做出反应?”尤婵想了想,还是选择了稳妥的方法。

%40
“也罢了,如今的局势对我而言,非常有利,照此下去,不出数月,这个神秘对手之前经营的局面就会都分崩瓦解,丧失殆尽了。”

%41
“这样的情况,我着急什么?应该是对方着急才是。”

%42
尤婵修行水道,深知上善若水,过犹不及的道理。

%43
方源继续观察了几天,很快就发现尤婵也开始按兵不动了。

%44
他心中暗赞一声,更加明白:此次龙鱼生意,他遭遇到了一个相当难缠的对手。

%45
不过,方源并不着急。

%46
对于龙鱼生意,他早就有了完善的部署。

%47
“只要时间越往后,我的优势就越大。此时还是暂且忍耐吧。”

%48
仙窍经营这块,卡在了龙鱼生意上,但这并不妨碍方源的其他修行。

%49
魂魄底蕴已经磕磕碰碰,上升到了八千万人魂级数。本来应当早就登上亿人魂,但这段时间里,方源不断地练习仙道杀招分魂,导致魂魄时常受伤,耗损底蕴。

%50
不过这样一来,方源的收获就是分魂杀招,越发数量。在此期间,他甚至还利用智慧光晕,将分魂杀招的催动步骤,进行了一些小小的改良。

%51
除了分魂之外,其他的一些魂道杀招,方源也在练习。

%52
不过这些杀招,威能要大大弱于原版,因为仙蛊着实不足。

%53
方源此时的重心,已经摆放在了天机仙蛊上。

%54
和绝大多数的蛊仙不同,每当方源拥有一只仙蛊,他的实力就能迅速实现一次增长。这是因为他继承了海量的传承,绝不缺乏运用仙蛊的方法。

%55
天机仙蛊乃是砚石老人所创,砚石老人同样也是幽魂魔尊的分魂。

%56
方源取出天机仙蛊。

%57
天机仙蛊摇摆着七对羽翼,在方源的面前,悠然漂浮。

%58
直至现在,方源每次看到这只仙蛊时,都要感叹一下自己的运气。

%59
他万万没有料到,他参与炼制天机的次数只有两次,第二次就成功了!

%60
要知道这可是七转仙蛊!

%61
这次成功非常侥幸,方源回想起来都觉得像是一场梦。真要让他再来一次,他毫无信心,这种成功无法复制。

%62
炼制仙蛊就是这样,能让蛊仙心绪上下起伏。

%63
有的时候炼蛊,失败个百八十次,成功都遥遥无期。而有的时候炼蛊,一两次就成功了!

%64
方源经历过炼制许多次,都不成功的情况,也经历了这一次炼成天机仙蛊的事情,对于炼蛊,他真的说不准。

%65
这就是一个坑!

%66
和其他的仙蛊不同,围绕天机仙蛊构建的仙道杀招,非常稀少。

%67
这是因为,砚石老人创造出天机仙蛊来,实际上也没有多少年,又值影宗大计,需要砚石老人全面筹划,所以他只设计出了几个杀招罢了。

%68
其中最主要的,也是价值最高的,就是石洞天机了。

%69
这个杀招,居然能够提前预测到天灾地劫的内容,这里面蕴含的意义是多么的重大!

%70
只要预测成功,有了针对性的防范,蛊仙渡劫的成功率将疯狂飙升,这是对整个蛊仙界都有着巨大影响的一只仙蛊。

%71
“或许将来,我能够做这样的生意?”方源心中一动。

%72
这样的生意,一旦形成口碑,谈不上日进斗金,但绝对是三年不开张,开张吃三年!

%73
绝对大有市场。

%74
不过,当下的情形,方源思考了一下,还是决定算了。

%75
因为他现在还不打算暴露仙蛊。

%76
最大的敌人就是天庭。

%77
十年时间,他需要尽全力阻止天庭彻底修复宿命蛊。

%78
如今,虽然他发展迅猛,但是距离这个目标,仍旧有着十万八千里的遥远差距,看不到任何成功的希望。

%79
在这样的情况下,方源需要将天机仙蛊当做自己的一份底牌,若是做起生意来,天庭的智道蛊仙绝不是吃素的,必然能够探知到这样的情况。

%80
因此,方源还是打算将天机仙蛊隐藏起来,秘而不宣。

%81
不想浪费一分一秒,方源开始锻炼杀招。

%82
这个杀招,以天机仙蛊为核心,但却非是石洞天机,而是天消意散。

%83
短时间内,方源不用急着去演练石洞天机。

%84
因为地灾、天劫都不恐怖,对他而言,很容易就能渡过去。

%85
在七转阶段,唯一有风险的,就只有浩劫了。

%86
而他距离浩劫,还有比较长的一段时间。

%87
天消意散是一个智道杀招,以天机仙蛊为核心,专门针对天意,能够将天意驱除干净。

%88
此招乃是砚石老人所创,但他没有成功,所以落到方源的手中,天消意散只是一个残招。

%89
方源演练了片刻,残招自然是不会成功的,不过这个经历却让他更加熟悉天消意散的内种奥妙。

%90
他不禁对砚石老人的才情,暗赞一声。

%91
有了天机仙蛊的牵引,天意也可以被蛊仙直接攻击,甚至抹除了。

%92
不过这其中,还有许多的关隘,距离真正的成功,距离遥远。

%93
但这些对于方源而言,却没有什么关系。

%94
他智道境界不如砚石老人,但底子还是有的,他最大的优势在于他可以利用智慧光晕!

%95
智慧蛊在这里,再次体现出了九转仙蛊的绝妙威能。

%96
数天之后,方源将天消意散杀招补全。

%97
掌握了这个手段,他开始对春秋蝉下手!

\end{this_body}

