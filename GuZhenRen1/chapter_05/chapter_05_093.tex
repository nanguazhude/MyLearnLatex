\newsection{两道真传取其一}    %第九十三节:两道真传取其一

\begin{this_body}

%1
时间匆匆,一晃便是数日过去。

%2
琅琊福地,云城之中。

%3
方源盘坐在密室中静修。

%4
动用通天蛊,他的神念在宝黄天中和某位女蛊仙的意志,进行着交流。

%5
“你这丹青香的成色,比蛊仙西子所卖的,要明显低上一层。价格再便宜一些罢。”方源讨价还价。

%6
但那位女蛊仙,并没有像方源这样,亲自散神念,来主持生意。

%7
她放到宝黄天中的,除了要寄卖的货物之外,就是她的一股意志。

%8
很不巧,这股意志乃是刚意,最是顽固不化的那种。

%9
果然,女仙刚意直接摇头拒绝,一点犹豫的态度都没有:“卖价就是这个,你想买就买,不想买就作罢!”

%10
方源不甘心,还在做努力:“你再考虑考虑?我要的货量可是很大的。”

%11
刚意顽固,没有退让。

%12
方源有些气闷。

%13
这些天他经营至尊仙窍,免不了要向宝黄天中交易,贩卖自产资源不提,同时还在采购其他种种资源。

%14
但自从第三次地灾之后,他明显现自己办事相当不顺利。

%15
眼前的女仙刚意,就是其中一例。

%16
这种情况还算是好的。

%17
方源最近经常碰到:有蛊仙掺和进来,恶意抬价,或者碰到货物短缺,被八转大能扫清存活之类的事情。

%18
“我虽有狗屎运护持,但第三次地灾中,遇到巨祸焚木,坏了我不少佳运。”

%19
想起刚刚渡过不久的第三次地灾,方源叹了一口气。

%20
“若非第三次地灾中,我思量良久,决定和楚度联手。否则,只怕是凶多吉少。就算渡过去,也是伤势沉重,厄运缠身。现在这种情形。只是稍微不顺遂一点,对我修行其实并无大碍。”

%21
方源很快收敛思绪,神念继续和宝黄天中的女仙刚意洽谈。

%22
既然女仙不同意让价,方源也就捏着鼻子认了。

%23
“你这些丹青香。我都买了。”

%24
这女蛊仙的刚意都楞了一下。

%25
方源的确是大手笔。

%26
这笔买卖,他要付出近万块仙元石!

%27
但方源不动声色。

%28
经过前段时间的积累,他的财力十分雄厚。即便是渡过了第三次地灾,在渡劫中耗费了无数青提仙元。而近万块的仙元石,他仍旧拿得出手。

%29
购得丹青香之后。方源就抽神念,不再关注宝黄天。

%30
仔细检查了一番后,他确认丹青香没有问题,便将这些都挪移到至尊仙窍当中。

%31
他的心神,投入自家仙窍里。

%32
至尊仙窍中分出五域九天格局,方源在每一处地方,都布置了力道仙僵。

%33
很快,一头力道仙僵在他的操控下,飞过来,运用蛊虫接住这些丹青香。

%34
没办法。他的至尊仙窍内部空间实在太大了,只能用这些力道仙僵代劳。

%35
每当这个时候,方源就不禁怀念起狐仙地灵,还有星象地灵,有地灵的话,这些工作就由地灵代劳,无须他费事劳神了。

%36
地灵就相当于一方福地的大管家。

%37
有地灵还是很方便的。

%38
但可惜的是,方源还未听说过,蛊仙再没有陨落前,自家仙窍中就产生地灵的。

%39
这种丹青香。宛若碧色烟雾,不断缭绕,氤氲一片。

%40
方源将所有的丹青香,都放入小青天之中。让当中的一大片苍穹都变得暮霭沉沉。

%41
虽然力道仙僵是闻不到气味,但丹青香的香味,实际上非常美妙,宛若墨香,又似草气,雅淡悠远。

%42
方源安置了这些丹青香后。又布置了一些蛊阵。

%43
这些蛊阵,都是由凡蛊组成,没有别的作用,就是将这些丹青香拘在一定的范围内,不让它们随意飘散。

%44
“一下子囤积了这么多,足够喂养换魂仙蛊三次有余了。”方源心中较为满意。

%45
但这些丹青香,用一分,就少一分,不能自己再生。

%46
若要自产丹青香,这可就麻烦了。

%47
丹青香的产生环境,需要有真元丹顶鹤群、青松林、墨竹林,以及芸香蛊。这四种囊括兽、植、蛊三方面,不仅如此,每一种的数量都有严格的标准,任何一种多一些,都能产生出丹青香来。

%48
而这个数量标准,就是行业秘密。

%49
被贩卖这种丹青香的蛊仙们,严格把控,极难外流。

%50
方源目前为止,也只能暂时囤积这些丹青香先用着。

%51
丹青香是喂养换魂仙蛊的食料,好在这种资源,在宝黄天中卖的人很多。方源这一次又囤积了这么一大批,足够应付一段时间了。

%52
又检查了一番后,方源微调了一下这里蛊阵。他最后望了一眼,只见小青天中,出现了一小片浓郁的青斑,显得和以前不太一样。

%53
尽管方源大力经营、建设仙窍,但从整体而言,至尊仙窍还是很荒芜贫瘠的。

%54
关键是至尊仙窍的空间,实在太大!

%55
方源投下去的这些资源,若换做其他蛊仙的普通仙窍,早就已经是铺得琳琅满目。

%56
说起来,方源的至尊仙窍中包含的道痕也不少,但要影响这么大一块地方,道痕影响的效果就分薄了很多。

%57
方源抽留在小青天中的神念。

%58
丹青香布置妥当,意味着方源经营仙窍的工作,达到了一定阶段。

%59
在此之前,方源已经收购了几批其他资源,将剑眉、浪剑、飞剑等七转仙蛊的食料问题,都一一解决了。

%60
“八转仙蛊、七转仙蛊的喂养,都已经布置妥当。如此一来,接下来建设仙窍,就该解决那些六转仙蛊了。”方源计划着。

%61
蛊仙建设仙窍,第一个基本标准,就是能够喂养得了手中的仙蛊,做到自给自足。

%62
方源之前是力道仙僵,仙窍已死,所以没有按规矩来。

%63
这一次重获新生,他稳扎稳打,已经在大致上解决了八转、七转的仙蛊喂养难题。

%64
接下来就是那些六转仙蛊。

%65
不得不说。方源的仙蛊数量实在太多。

%66
说出来,要吓死人!

%67
但没办法,方源必须养这么多仙蛊。不仅是抗衡灾劫和天意,而且还要对付其他蛊仙。

%68
方源的处境很难过。

%69
以前有影宗方面。替他遮掩,防备推算。但义天山大战之后,这个保护伞没有了,天庭蛊仙将他的名字登上诛魔榜,杀他之心极其坚决。还有北原蛊仙对他极端仇恨。方源端了王庭福地,把八十八角真阳楼都弄没了,这比端了这些黄金血脉的祖坟还要招人恨。

%70
其余东海、南疆、西漠的蛊仙,都觊觎他身上的运道真传。尽管方源没有得到众生运的真传,但这个理是说不清,没有人会相信。

%71
总而言之,方源的身份是绝对见不得光的,一大波的蛊仙在虎视眈眈。

%72
前段时间,黑家大战,黑家蛊仙就和百足天君合谋。想要埋伏方源,结果方源因为谨慎,没有去铁鹰福地,逃过了一劫。当然,这个事情,方源目前还不知晓。

%73
所以,方源在外行走,取了个假名柳贯之。

%74
他要是表明真实身份,估计楚度的态度就要变了!

%75
目前,方源已经和楚度联手。

%76
在大半个月前。宝黄天开启之时,方源就接到楚度的来信。

%77
信中,楚度说明情况,让方源得知。原来飞剑仙蛊上,还有一道信道传承的线索!

%78
那一刻,方源心动了。

%79
在此之前,他还时常念叨着信道的好处。

%80
有了强大的信道手段,他可以和旁人定下契约,不怕对方反悔。又可以自己偷偷地解除身上的盟约。自由撕票,随意反悔,无拘无束。

%81
他还可以收集大量情报,帮助自己做出更加英明的决策。

%82
信道和智道是好搭档。

%83
信道搜集情报,智道利用这些情报进行推演,两者相得益彰。

%84
飞剑仙蛊上的这道信道传承的印记,十分了得,就连智道蛊仙田下心都忍不住心动,想向楚度讨要。

%85
这个情况,楚度也向方源做了说明。他要让方源清楚这份信道真传的价值,好让他更加顾忌。

%86
方源沉思片刻,就决定和楚度做交易。

%87
他答应楚度,让他得到狂蛮真意,而楚度也必须付出一些代价。

%88
黑家大战之后,楚度在方源的指点下,抢到了第十三座鹰巢。

%89
在之后的某一天,他和方源在北原暗中碰面,双方进行了交换。

%90
方源重得飞剑仙蛊,以及鹰巢,而借出招灾仙蛊。

%91
仙劫锻窍之法,方源是不可能给楚度的。虽然这是仙道杀招,仙蛊唯一,但楚度完全可以用其他仙蛊进行替代。

%92
蛊虫不同,但却能形成威能相似的仙道杀招。

%93
楚度不愧是七转强者,名动北原的当代传奇!

%94
有了他在一旁策应,方源十分顺利地就渡过了第三次地灾。

%95
前两次,他都伤重濒死,惊险绝伦。但第三次地灾,只是有些波折而已,大头都被楚度解决了。

%96
招灾仙蛊的威能,实在是厉害。

%97
难怪当年,墨瑶会想用这个仙蛊,帮助自己的爱郎薄青渡劫。

%98
不过最终,她最终还是失败了。

%99
招灾仙蛊只是七转,薄青要渡的劫,却是冲击九转境界的万劫!

%100
飞剑仙蛊上的信道真传,方源还暂时不想去取。

%101
乱流海域是个麻烦的地方,很容易迷失,而且那个地方经常有蛊仙出没。

%102
方源重点研究鹰巢。

%103
铁鹰福地中的第十三座鹰巢,平时都隐形匿迹,唯有铁鹰福地濒临崩溃之时,它才浮现而出。

%104
就在这里面,藏着黑凡真传的线索。

%105
方源已经摸索了一段时间,已有心得!

%106
比起信道真传,能够延缓仙窍时间的黑凡真传,无疑更实用,更重要。

%107
方源已经决定:趁着刚刚渡劫的这段空闲功夫,他要将黑凡真传拿到手!

\end{this_body}

