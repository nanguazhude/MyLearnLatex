\newsection{参悟尊者真传}    %第七百六十六节:参悟尊者真传

\begin{this_body}

%1
方源的智道手段十分丰富,并不需要成竹在胸、运筹帷幄来弥补,他练习这些杀招,更多的是增加对手上的元莲真传、巨阳真传的理解。

%2
这些天来,方源本体一直在钻研着元莲真传、巨阳真传以及盗天真传。

%3
长毛老祖的炼道真传,方源已经全部掌握,理解得十分容易,毕竟他是炼道准无上境界。

%4
对于元莲真传,方源组建出了六转因果神树杀招,算是打开了局面。

%5
但因为木道仙蛊、木道境界的缺乏,方源并没有在此投入过多精力。

%6
他把重点放在巨阳真传、盗天真传上。

%7
前者属于巨阳三大真传中的己运真传,能够让方源掌握自身运势。

%8
后者盗天真传,核心便是八转偷生仙蛊,讲述了如何令偷生仙蛊解封的做法。

%9
当然,两者都讲述了运道、偷道的基础修行,尤其还有两大尊者对各自开创的两个流派的个人理解,令方源受益匪浅。

%10
“什么是运?”在巨阳真传的开篇,巨阳仙尊就问了这个问题。

%11
随后,他自问自答:“要谈运,先谈命!”

%12
在红莲魔尊之前,是没有运道的。

%13
红莲魔尊损害了宿命蛊后,天地方才有了一些微妙的变化。

%14
人们最初察觉运道,是从人气中发现了某个成分——运气。

%15
因为天地人三气需求平衡,方能升仙得道。蛊师人气积累越多,升仙成功后得到的仙窍的品质就越是上佳。

%16
因为对人气的重视,人们研究运气。

%17
很快,他们就发现运气的某些特征。

%18
自从宿命蛊被红莲魔尊伤害之后,万物生命的轨迹再不是固定的,一成不变的,而是可以有限的改变。

%19
人们发现,所谓运气和这种偏离宿命的改变,有着非常玄妙而且直接的联系。

%20
但究竟是有什么联系?

%21
巨阳仙尊直接讲述了自己的观点:“命是定数,运是变数!”

%22
对于一个常人而言,什么是好运?

%23
好运是对自身有益的变数。

%24
那什么是坏运?

%25
坏运就是对自身有害的变数。

%26
一个人的人气中运气属于好运,那么这个人就很可能得利。

%27
一个人的人气中运气属于坏运,那么这个人就很可能不利。

%28
然而不管好运、坏运,它都不是命。

%29
命是定数,是固定的,必定要经历的。而运是变数,只是一种可能性,真正的结果还需要结合现实的具体情况而定。

%30
若是一个人有了坏运气,却用自身的实力和才智抗衡扭转,仍旧能够达到某种目的或成就。

%31
若是一个人有了好运气,自身却是愚蠢不堪或者实力不佳,把握不住,那么仍旧会失败。

%32
随后,在巨阳己运真传中便见到了运气的种种规模、颜色以及形态等等。

%33
有的人运气很多,有的人运气很少。

%34
一般而言,修为越高,底蕴越深的人,拥有的运气规模就多。

%35
运气的颜色主要有:黑、灰、白、赤、金、青、紫这七种,但也有粉色、血色这类不常见的颜色。

%36
运气的形态各种各样,简直是千奇百怪,无奇不有。普通人的形态往往单调平常,天才或者霸主之流的运气,往往大异常人,千姿百态。

%37
比如方源曾经有过的黑棺气运,运气漆黑如墨,形成一个庞大的棺椁形状,将其浑身都覆盖在里面,散发着浓郁的凶煞死意。

%38
又比如曾经在炼道大会上观察郑山川,此人的运气宛若七彩虹光,光泽动容,夭矫不群。

%39
至于洪易、叶凡、韩立这些人运气,也都各有奇妙形象。

%40
巨阳仙尊留下的己运真传,就是对自身的运气着手,改变它的规模、颜色、形态的手段大全!

%41
比如想要异性缘分方面的变数,就形成桃花运。

%42
想要获得修行的资源比如元石,就形成财运。

%43
若是自己身上有霉运,那就动用一系列的手段将其改变。

%44
……

%45
巨阳己运真传的内容极其丰富,盗天偷生真传却是言简意赅。

%46
“偷道,不是不劳而获,而是最讲究效率的流派!”

%47
“蛊师、蛊仙要获取蛊虫、仙蛊,不管是炼蛊、交易还是战利品,无一不是成本太高,风险太大。”

%48
“偷道直接针对蛊虫,将蛊虫盗取出来,节省成本,缩减风险,提高效率!”

%49
然而蛊虫的本质是什么?

%50
蛊是天地真精。

%51
偷取蛊虫,也就是偷取道痕的碎片。

%52
因此又衍生出杀招偷道。

%53
还有偷生仙蛊,寿命显化便是人体内的某些个道痕。偷生仙蛊将相应的道痕偷取出来,便能缩减人的寿命。

%54
“不要认为偷道无耻或者卑劣。”

%55
“很多事情,不妨换一个角度来想想。”

%56
“窃钩者诛,窃国者诸侯!任何的家族、门派,都会让成员为组织贡献,这是明目张胆的偷盗,但却被认为理所应当。”

%57
“譬如天庭,加入的成员都要贡献出自己的仙窍,等若是将一生修行的成果奉献上去。从偷道的角度来看,这是天庭窃取了蛊仙的修行成果。但人们往往认为这是荣耀,趋之若鹜。”

%58
“弱小的凡人也能够偷取蛊仙的修行成果,这就是蛊仙遗留下来的各种真传。但这个世界却是习以为常,这就是文化。”

%59
“所谓的偷,便是最有效率地收获。建立制度,凝聚荣耀亦或者利用情怀、文化,都是偷道的手段。”

%60
“偷道最厉害的手段是——包容。”

%61
“洞天福地毁灭,大同风刮起来,一切并非化为乌有。而是仙窍的道痕被这片天和地包容、消化。”

%62
“最厉害的大盗便是这片天地!”

%63
“它偷取历代英杰的修行成果,积累更多的道痕,或者全新的凹痕。它将我等从别的世界偷取过来,便是天外之魔。它偷取万物的寿命,让所有的生命都必须死亡。”

%64
“我开创出来的偷道,便是效仿天地的流派!用最有效率的手法收获利益,增益自身……”

%65
方源看了之后,只觉胸襟一阔,偷道在他眼中一扫之前的狭小印象。

%66
盗天魔尊的思维和常人不同,另辟蹊径,且又偏僻入里。他用偷道的思维,理解了整个世界!

%67
他的格局非常宏大,带着天外之魔的个人特色。

%68
蛊师修行,对蛊虫的养、用、炼,经营仙窍,以及各种各样的境界,无非就是了解自己,了解天地。

%69
每一个流派,便是一种不同的理解天地、了解自身的角度。

%70
“偷生仙蛊乃是我生平第一只仙蛊,并非由我炼制所得。”

%71
“使用次数多了,渐渐有了感悟,最后我在偷生仙蛊上布置了手段。”

%72
“顺用偷生仙蛊,可以盗取其他生命的寿元,配合我开创的杀招,能增添自身寿命。当然,还是有着缺陷,不如寿蛊完美无缺。”

%73
“逆用偷生仙蛊,便是将自身寿元给与其他对象。这看似愚蠢,却是别有奥妙之处。也算是这道真传的真正精髓所在。”

%74
“后来人若是有缘,达成这一步,便能解除我在偷生仙蛊上布置的手法,同时得到与之相对的另一道真传的信息。”

%75
这是盗天魔尊在最后的留言。

%76
方源看了久久无言。

%77
他的偷道境界虽然也有大宗师程度,但是仍旧很难理解盗天魔尊的用意。

%78
逆用偷生仙蛊?

%79
方源摇了摇头,他没有信心去领悟这其中的奥妙。

%80
根据他的推算,至少得有准无上的偷道境界,才能接触到这一层次。

%81
在此之前,他若逆用偷生仙蛊,就纯粹是给自己减寿。

%82
“况且,我还得利用这偷生仙蛊来对付天庭蛊仙呢。”

%83
“其他人就算了,龙公特别被偷生仙蛊针对!”

%84
就算方源用光了两次机会,偷生仙蛊毁灭,在盗天真传的记载中,也有着偷生仙蛊的八转仙蛊方。

%85
“用坏了偷生仙蛊,大不了以后有机会再炼出一个新的来。”

%86
“只是……”

%87
根据方源的推算,盗天魔尊留在偷生仙蛊上的布置,真的价值很大。若能顺应盗天魔尊的指点,在将来领悟到这记真传的真正精髓,方源的收获绝对要远远大于现在。

%88
就在方源积极潜修的同时,南疆的正道首脑们经过不断的扯皮和协商,终于开始达成了一致。

%89
“必须要铲除方源,他太过无法无天了,上一次他伪装成武遗海,这一次更是直接露面,侵犯我正道的利益。”

%90
“地沟之争,是我们的内部矛盾。而方源是外部矛盾,对付方源我们应当团结一致!”

%91
“但是要谨慎!方源虽然只有七转修为,但这人太狡诈,底蕴深厚,不能以常理揣度。”

%92
“没错。尤其是他竟然抵御住了天庭的攻伐,如今吞并了琅琊福地,恐怕得到了大量的异人蛊仙的投靠。再加上他之前吸纳了影宗残余……我提醒大家,方源他不是一个人,他是一个超级势力!”

%93
“一位八转蛊仙带队,恐怕不够。为了稳妥起见,还是两位吧。”

%94
“呃,这未免有些太小题大做了吧?”

%95
“我同意。”

%96
“同意。”

%97
“我巴家没有异议。”

%98
“表面上,应以夏槎大人为首。而另一位大人则潜藏身份,混迹在队伍中,在关键时刻出手。”

%99
“不错的安排。”

%100
“我附议。”

%101
“同意。”

%102
……

%103
和上一世不同,方源因为吞并了琅琊福地,击退了天庭进攻,令南疆正道感觉到了更大的威胁。

%104
他们不只是派遣了夏槎,更还有第二位八转蛊仙参战!

%105
而这一切,方源都还蒙在鼓里。

\end{this_body}

