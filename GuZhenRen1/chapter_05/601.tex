\newsection{方源渡劫}    %第六百零三节:方源渡劫

\begin{this_body}

数月后,南疆。

一处无名的普通山谷。

“嗨!”六转女仙巴桃断喝一声,身上气息澎湃。她双手高高举起,像是支撑着万斤重物,很快就满脸赤红之色,身体微颤,吃力不已。

“快一点。”巴桃催促道。

另一旁,同样有一位蛊仙,也在催动杀招。

此人六转修为,青年模样,来自铁家,姓铁名义,乃是铁区中之子。

一块块的赤色光球,盘旋上升,猛地轰然一声,化作冲天的赤色光柱,就要直上云霄。

但这个时候,第三位蛊仙老者显露身形,他就站在云端,一挥手将这赤色光柱镇压下去。

赤色光柱消散,显露出一座七层宝塔,宛若赤铜铸就,塔顶上静静地燃烧着一团火焰。

“好了。”铁义吐出一口浊气,面露欣慰之色,“有一座烽火台建成了。”

“累死我了。”巴桃如释重负,一屁股坐到了地上去,气喘吁吁,额头上满是汗渍。

最后第三位蛊仙老者,缓缓飘落而下,同时无数蛊虫飞舞汇集,纷纷钻入他的袖口之中。

这是一个建设烽火台的小队。

共有三人,铁家蛊仙铁义主要负责搭建,巴桃辅助,而蛊仙老者则是布置大阵,做出伪装,毕竟烽火台搭建成功时动静很大,容易被人知晓。

像这样的小队,还有十数支。自从南疆正道有意联合起来后,这些小队就四处奔波,在南疆各地建设烽火台。

“烽火台不愧是铁家招牌的仙蛊屋,真是不可思议,区区凡蛊建成之后,就有仙蛊屋的威能,可传送蛊仙!”蛊仙老者望着眼前的烽火台,感慨不已。

铁义则走到巴桃身边,一把将后者拉起,闻声心中一动,便道:“烽火台的真正厉害之处,是在于建设的越多,总体的威能就越大。要达到仙蛊屋的程度,至少要上百座的烽火台。如今我们在南疆,烽火台总数已经突破一千大关,完全可以输送七转蛊仙。但要到达八转,至少要有一万左右的烽火台。”

“多谢解惑。”蛊仙老者频频点头,感谢道。

铁义和他对视一眼,两人同时微微一笑。

这种关乎烽火台的秘闻,换做以前,是绝不可能说出口的。但是现在因为南疆联盟正在组建,蛊仙之间对待彼此的态度,交往的方式,也在逐渐改变。

“若是有一座烽火台,就建设在那片光阴支流附近,我们南疆正道一定就可以出动援兵,及时救出我的祖父了。”巴桃望着烽火台发愣,口中呢喃。

“巴桃仙友,我们还有希望,翼家的蛊仙不就已经被方源放出来了么。方源这个魔头无非是想要更多的修行资源而已。”蛊仙老者劝慰道。

铁义却不说话,保持着沉默,心中的思绪却在翻滚。

他和巴桃有着类似的处境,他的父亲铁区中也是方源的俘虏之一。

他对方源自然是有着愤怒和仇恨,但也不得不承认,若非方源这一次俘虏了这么多的南疆蛊仙,将南疆正道打得生疼,南疆正道又岂会这么快联合起来呢?

安逸和痛苦,当然是后者更能触动人心。

有时候想想,铁义私底下也觉得有点讽刺。烽火台的大计,乃是铁家上下多年的梦想,结果却因为一个魔道巨擘而实现。当然还有一个关键的因素,还是大时代的来临。但不管怎样,方源起到的作用也是举足轻重的。

嗤嗤。

烽火台忽然发出怪异的响声,还微微颤动了一下。

“怎么回事?”这个情景,立即引起蛊仙老者的重视,他皱起眉头,看向铁义。

巴桃也将一对妙目,凝住在铁义的身上。

她也参与建设了不少烽火台,从未见过有这样的情况出现。难道是这座烽火台在建设的过程中,留下了一些缺陷和瑕疵?

铁义的眉头皱得更深:“二位勿忧,如此情景并不是这座烽火台有问题,而是分布在南疆各处的烽火台中,有一座被彻底破坏了。”

“哦?是这样。”蛊仙老者眉头微微舒展了一下。

巴桃则冷哼一声。

这样的情形,也在南疆正道的估料之中。因为烽火台的建设,虽然得到了南疆正道的允许认可,但南疆蛊仙界中可还有魔道、散修两大势力。

烽火台掌握在铁家手中,代表着的是铁家以及其他南疆超级势力的利益,和这些魔道、散仙毫无关系。

建成的烽火台被发现,然后被这些蛊仙摧毁,也是极可能发生的事情。

“查出来是谁动的手,我们要杀一儆百!”蛊仙老者杀气腾腾,这也是南疆正道商议之后,应对此事的方法之一。

然而下一刻,烽火台的异状让三仙都惊得呆滞。

嗤嗤嗤嗤!

一连串的怪声,连响了一片。刚刚建成的烽火台不断颤抖,像是得了癫痫的病人。

“怎么回事?”

“这……难道是意味着大量的烽火台被接连破坏了吗?”

巴桃和蛊仙老者再次看向铁义。

这一次,就连铁义也犹疑起来,口中呢喃自语:“奇怪,难道真的是这座烽火台有问题?”

但下一刻,他的这份怀疑就被打破。

三仙面色顿变,因为他们几乎同时接收到了一个惊人的消息。

疯狂破坏烽火台的,只有一人,不是别人,正是那万恶不赦、罪大恶极的魔道巨擘方源!

“方源有智道卓绝手段,推算出这些烽火台的位置,并非难事。他又有定仙游,随意来去,极其自由。做到这一点,并不奇怪。”铁义面色阴沉如水。

“该死!天庭无能,居然让方源盗走了定仙游!”巴桃握紧双拳,愤恨不已。

蛊仙老者到底老成持重:“我们还是走吧,回归家族去。方源既然行动,外面已经不安全了。”

铁义、巴桃顿时心头一紧,想起方源的手段,再无逗留于此的丝毫想法。

与此同时,武庸、池曲由等各大家族的首脑,也在紧急秘议。

“方源恐怕是看出了烽火台的重大威胁,提前出手,疯狂破坏烽火台!可见我族的烽火台,对方源的威胁甚大,已经让他不安和恐慌。”铁家太上大长老道。

“哼,让他拆就是了。烽火台建设的难度颇高,需要蛊仙亲自出手。但是本身只是凡蛊铸就,成本低廉,就让他破坏就是了。”商家太上大长老冷笑一声,颇有家大业大的豪气。

武庸却是皱起眉头,他更加了解方源一些:“方源狡诈奸猾,绝不能小觑。我总觉得他此举另有深意,绝非表面看起来那么简单。”

武庸的话,立即引发其他蛊仙的重视。

其中一人沉吟道:“会不会有这种可能?方源要在我南疆渡劫,而这些烽火台中的一座,十分靠近他的渡劫地点,因此他这才悍然出手,摧毁了一大批。”

“有道理!”

“快,纠集智道蛊仙联手推算!”

“方源只要出手,就会留下种种线索,此时他大动干戈,推算的话,定然是比之前更有收获。”

一番推算过后,果然是成果斐然。

南疆正道得到了三个地点,若是方源要渡劫,这几个地方最有可能。

“速速查探。”

“重点关注这三个地方,一定要严防死守。”

相似的命令下达下去,南疆蛊仙界史无前例地统一了步伐,这一切都是为了剿杀他们共同的大敌。

须臾,一个惊人的情报就传到南疆各大势力的手中――五界山脉处出现异常的天地二气波动,并且范围极大,绝不正常!

\end{this_body}

