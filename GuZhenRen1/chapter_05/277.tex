\newsection{卜卦龟背仙蛊}    %第二百七十七节:卜卦龟背仙蛊

\begin{this_body}

方源需要光。[www.qiushu.cc 超多好看小说]

第一个是极光,他引进过大量的炫光蛊,增长仙窍中的极光。极光越加浓郁,就能凝聚浓缩成更多的流光果。而八转仙蛊态度,便是以此为食。

光照菌这种荒植,所能提供的光,自然不是五颜六色、色彩缤纷的极光。

不过,这并非就是盖棺定论。

人灵性最足,最懂得变通。这个世界上,亦有针对植株的改造和嫁接手段。

木道蛊仙最擅长这个。

每个流派都有各自的特长,而木道蛊仙往往在经营仙窍方面,非常擅长。

所以,方源若是引进了光照菌,说不定以后能通过木道手段,加以改造,形成新的极光照菌,也并非没有可能。

除了态度仙蛊之外,还有慧剑仙蛊。

喂养慧剑仙蛊的食料,是斑斓霸王花。

要大规模种植斑斓霸王花,需要三个条件。一个是光照,第二个是甜气,第三个是珍珠土。

后两者可以轻易达到,但是光照却是较难的一点。

方源的仙窍本身,光道道痕虽有,前一段时间也吞并过光道福地,但是距离方源的标准,还有一定的差距。

运用光照菌的话,斑斓霸王花的大规模种植难题,就可以得到解决。

不过这个方面,也有一点存疑。

那就是光照菌的生长环境,是在深海当中。若是换做另外环境,是否能保持生机?

“这个应该不是什么问题。毕竟光照菌,乃是荒植。”

“极光也就算了,之前我洗劫了刘家的璇光坑,有了很多补充。距离标准,虽然差了一点,但是也相差不多。”

“但斑斓霸王花这事情不一样,和标准还是有比较大的差距的。毕竟我的光道道痕,不是很多,而且也不能全部的光道道痕,都集中在斑斓霸王花这块儿。其他的生命,也需要光照。”

“哦,对了,我还有变化道痕,关键时刻,可以将其转化成光道道痕,以做补充。”

方源思考的时候,童画继续开口道:“我愿用这批荒植光照菌,换取三头重楼水象。txt小说下载80txt.com”

此言一出,当中便有一位蛊仙答道:“重楼水象……我手中就有,不知道童画仙友要换取多少重楼水象?”

这重楼水象,乃是东海特有的野兽,体型大小各自不已,牙白皮蓝。一重楼的水象,形如小船,背上有一层蓝玉骨骼,形成楼台模样。二重楼水象,体格更大一些,背上有两层蓝玉骨骼。

以此类推,六重楼水象,背部宽阔,犹如楼船,乃是荒兽一级。七重楼水象,上古荒兽。八重楼水象,即为太古荒兽,极其罕见。九重楼水象,并不存在。

不管是禽兽、植株,乃至异人,都未有出现过九转的存在。人族除外,当中陆续涌现了十位九转蛊尊,在万物生命之中,独树一帜。

童画展示了手中光照菌的规模,告知台下的蛊仙,愿意换取三头六重楼水象。

这价格并不昂贵,反而非常公道。

那蛊仙却皱眉道:“可是我手中,却只有一头六重楼水象而已。光照菌应当可以拆解,童画仙友可愿交换?”

童画犹豫了一下。

对方所言,未必是事实。

思考了一下,童画点点头:“那就交易吧。”

于是第一笔交易达成,双方当场交接,非常快速。

“我想要一些光照菌就,不过手中却没有重楼水象,不知仙友可否接受仙元石?”方源问道。

童画却摇头,但也没有完全拒绝方源:“重楼水象没有,那就算了。不过,我的这些光照菌,将来还会在宝黄天贩卖,这位仙友倒时候自购便可。”

事实上,东海多是水道蛊仙。

而光照菌,非常适合东海蛊仙。就像是武遗海,他的整个仙窍就是一片汪洋。

童画只是先拿这个新鲜玩意,来试试水,看看有没有机会,获得更多的利益。

至于贩卖得仙元石,暂时童画还不想去做。

因为她不太了解行情。

这种新鲜事物,第一次投入市场,童画并不知道,这究竟能价值多少。

不过,她对此抱有强烈的期待。

因此拒绝方源,毫不犹豫。万一在宝黄天中,这个光照菌大受欢迎,那么她现在卖出的价格,岂不是亏本了?

但若是价格开高,对方肯定不会交易。反不如直接宝黄天最为妥当。

童画既然不卖,方源也很无奈。

不过他可以理解。

一种全新的荒植,完全可以借助手段,将价格炒高。

这个世界虽然经济并不发达,但是蛊仙们也有精明之处,懂得哄抬物价。

尤其是这种光照菌,世间目前就童画手中独一份儿。

换做方源的话,他也会趁着刚开始的黄金时机,将价格开高,大赚一笔。

毕竟光照菌,只是六转荒植,流出去多一些,时间一长,就会有人寻找或者研究出来,打断童画的垄断。

毕竟,东海那么大,光照菌绝不会只生长在一个地方。

蛊仙们参照手中的光照菌,完全可以研究出一种专门探寻它的手段,这样一来,就会导致光照菌的发现,大大增多。

见没有人继续和自己交易,童画便主动走下去。

立即,就有第二位蛊仙接替了她的位置,走上高台。

这是交易会约定俗成的规矩。

每一轮中,蛊仙们都会按照某种顺序,依次上台。每一次交换的资源,都只能是同一种类。

这样一来,才显得较为公平。

至于方源,这位刚刚加入进来的新人,也有他参与交易,并且上台主动贸易的机会,不过他只能排在最后一位。

“这是吸髓石,七转仙材,我想要换取一段千年漂流木。”走上台的蛊仙,言简意赅,一句话就将自己的意图道个干净。

话音刚落,他便先当众展示了自己的货色。

这吸髓石,通体黝黑,表面上和普通的海边礁石没有什么两样。但实际上,只要是靠近它的拥有骨骼的生命,都会被吸收骨髓进入石内。

所以,吸髓石又称之为魔石。

魔石的这个叫法,来自于鲛人。

鲛人是异人中的一种,世间绝大多数的鲛人,都生活在东海。

鲛人在海底生活,能自由呼吸,碰到吸髓石,都尝过苦头,所以畏惧,称之为魔石。

在东海的历史中,有些鲛人部落还将魔石当做神明一样崇拜。

不过对于蛊仙而言,这吸髓石只是一块仙材。

这次交易同样成功。

千年漂流木虽然在市面上比较罕见,但此次参与交易会的蛊仙,都是身家深厚,要不然也不会聚集在一处。

物以类聚人以群分,能够和庙明神相处的蛊仙,自然不俗。

可以说,这场小型的交易会,已是东海蛊仙界中难得的强者聚会。

第二位蛊仙下台后,土头驮便走上高台。

“我有一头上古荒兽王母蟾,它可号令六转及其以下的巨蟾,若是培养得当,还能孕育出奴道仙蛊。各位可感兴趣?”

王母蟾一身的奴道道痕,非常浑厚。

但土头驮的交易却以失败告终。

第四位上台:“我要寻找关于气海的线索。谁知道该如何寻找到气海,并且进入当中?我愿意开高价收购相关有价值的情报。”

情报也属于交易的内容之一,可惜在场的蛊仙,都默不作声。

这场交易,也以失败告终。

气海是东海中,一片极其特殊,包含神秘色彩的海域。

也许在场的蛊仙们都不清楚,也许有人知道,但并不想拿来交易。

蛊仙们依次,轮流上场。

第一轮就要结束了。

方源压底,是最后一位。

而庙明神居于倒数第二位。

他走上台:“我这里有一只七转的卜卦龟背仙蛊。谁愿达成这场交易?”

方源闻言,双眼骤亮!

备注:今天状态极其不佳,仍旧一更,明天两更。(\~{}\^{}\~{})

\end{this_body}

