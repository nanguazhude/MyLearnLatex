\newsection{我就是变化道蛊仙}    %第四十三节:我就是变化道蛊仙

\begin{this_body}

为什么方源要选择变化道?

义天山大战之后,方源获得新生。[八零电子书wWw.80txt.COM]@,他思考最多的一个问题,就是自己今后的修行之路,该选择哪一条。

他前世选择血道,那是因为血道速成,威能巨大,能以战养战,对修行资源的需求量较少。而且前世他就这一个像样的机缘,对他而言,血道发展的前途最大。

重生之后,抛弃血道选择力道,是因为这不是五域乱战时期,秩序井然,血道没有出头之日。另一个原因是境遇造就。方源搜索记忆,找出适合他现状,能够得到手,并且能够运用的机缘。而这些机缘,都跟力道相关联。

其实,不管是血道、力道,都有弊端。

血道境况不好,人人喊打。不管是蛊师,还是蛊仙,选择修行血道,一旦被发现,那就等着全世界的排挤吧。

力道也式微,尽管霸仙楚度给力道增添了人力钧力流,但也难掩力道日暮西山的尴尬。

说实话,这两道流派都不是最佳选择。

方源自己也总结了一下:重生之后,自己获得的机缘着实不少。

有智道,有运道,有偷道……

这些机缘,有部分都极为优异。比如东方长凡的智道传承,又比如万象星君的星道传承。

星道、智道,这两个流派无疑比血道、力道的境况,要好得多。

而且有传承,和没有传承,完全是两码事。

前者修行。可以借助前人的智慧,步步前行。勇猛精进,前景可期。后者是盲人摸象。一片茫然,毫无头绪,在错误中挣扎摸索着前进。

如果方源五百年前世,也得到了这两道真传,那么只要情况允许,他肯定选择东方长凡的智道传承为主,万象星君的星道传承和前世血海老祖真传之一为辅。他会成为智道蛊仙。

因为论潜力和前途,自然是东方长凡的智道传承,最为优秀了。

智道蛊仙数量稀少。在蛊仙界中,也颇为金贵,受人欢迎的同时,也受人忌惮。里子、面子都有。

但真正要选择什么样的道路,单看前景可不行,还得论眼前。

东方长凡的智道传承,来源于天庭的八转蛊仙,又经历许多代继承人的完善,前景极为广阔。

但是这份传承修行出来。有个弱点,就是不擅争斗。

这份真传最显著的优点,是精于推算。

正是为了弥补这个弱点,东方长凡才创出万星飞萤仙道杀招来。总算是弥补了一些短板。

走东方长凡的这条线,并不适合方源。

因为眼前现状,可谓危机四伏。

影无邪。影宗的残余势力,天庭。中洲十大古派,北原各大黄金家族等等。方源树敌很多。

尤其是义天山大战中,方源的秘密都曝光了,不管是春秋蝉、天外之魔,还是王庭福地案犯真凶,他几乎彻底暴露。

所以,方源需要的不光是前景,还有眼下。

从某种程度而言,眼下比前景还要重要。

没有眼下,前景再广阔,也是空中楼阁,看着好看,没有用。

应付眼下,方源就需要战力极强的路线。

金道、炎道、雷道、剑道、血道,这五大流派,是公认的战力最强的五大流派。

前三道历久弥新,占据主流。

后两者,不管是剑道,还是血道,历史都很短,属于另辟蹊径。剑道中出了薄青,血道出了血海老祖,但除此二人之外,难有出彩人物。所以这两个流派的底蕴,还远远不及前三道。

对于方源而言,不管是金道、炎道,还是雷道,境界都是普通,相关的传承也没有一个。

反倒是血道境界达到宗师程度,剑道方面虽然也是普通境界,没有传承,不过偷了许多薄青的仙蛊。

各大流派的境界也是一个相当重要的考虑因素。

没有传承的情况下,若是境界高超,便可以白纸作画,推陈出新。只是比有传承,要辛苦很多,更需要灵感和才情。

方源的力道、血道、智道、星道境界,均是宗师级,血道是前世底蕴积累,力道是前世、今生的积累,智道、星道则是机缘巧合,通过梦境迅猛拔升,暴涨上去的。

炼道是准宗师级,在蛊仙当中,也算能拿得出手。

奴道是大师级,蛊仙中并不出色。

运道是准大师,其余流派多是普通(比如剑道、光道、暗道等),甚至是空白一片(比如虚道)。

若从境界这个角度出发,自然是走四大宗师境界的流派最佳了。

但血道不提也罢,力道日暮西山,智道、星道并不以战力出色著称,虽然各有传承,但遗憾的是,方源积累的仙蛊几乎都丢了。

若是这些传承中的仙蛊,还未炼制出来,方源或许还可在修行中,一只只炼成。

但这些仙蛊保留在影无邪手中,哪一只存在,哪一只毁灭,方源并不清楚。这个情况,就给他今后的修行造成了极其恶劣的阻碍。就算他愿意冒险,耗费巨大代价,炼制仙蛊,但因仙蛊唯一,未必能成呐!

综上所述,方源的处境可以说是相当尴尬的。

义天山大战之后,方源思考道路问题的时候,一时间也颇为茫然。

直到他渐渐发现,这具全新身躯的奥妙时,他心中笼罩的层层阴暗迷雾,这才开始消散,出现璀璨温暖的阳光。

现的是,自己身上道痕有上千条,各大流派都有。

这点并不出奇。

蛊仙一次次渡劫成功,仙窍中也会因为灾劫,而衍生出不同的道痕来。只是有很明显的主次之分罢了。

重要的是后面一点这些道痕之间互不干扰,互不掣肘!

这点简直是太妙了!

方源第一次发现的时候。首先就是不敢相信,世间居然有这等好事!

因为这一点。已经打破了旧有的修行常识。说出来,别人都会以为是天方夜谭、流言蜚语。

但想一想至尊仙胎蛊九转仙蛊,魔尊幽魂、影宗积蓄了十万年,最精华的存在。

这样一想,打破修行常识,也很合理了。

然后,方源狂喜。

因为这等好事,没有发生在别人的身上,而是自己身上!

单凭这一点。他就可以全流派通修!!

很久之前,方源曾经设想过,两个仙窍同修,一个主修宙道,一个主修力道。但至尊仙窍更为变态,它可以直接全系同修,将从古到今,所有的蛊仙流派都一网打尽!

这简直是梦幻般的美事。

纵观古今,就算是仙尊魔尊。也大多主修一道,兼修一道。就算有蛊仙兼修多道,最终的结局也往往是贪多嚼不烂,顾此失彼。下场惨淡。

因为他们身上的道痕,相互掣肘,兼修越多。内耗就越多,得不偿失。

但方源完全没有这个阻碍。他可以做到全道通修,内耗为零。前途之广大光明,比历史上所有的仙尊、魔尊都要胜出无数倍!

方源很快就冷静下来。

遐想总是美好的,全流派一网打尽,只是美好的设想。虽然可以实施,但那需要资源、时间、精力,最关键的还是一个良好安宁的修行环境。

而这些,方源统统没有。

他精力有限,没时间,资源有限,仇敌遍布五域,包含最擅战的北原蛊仙,历史上当之无愧的第一蛊仙组织天庭,掌握中洲的十大古派,无数想要从他身上得到运道真传的龙蛇草莽……

前景无限,现状却堪危。

开创流派的事情,想都不要想了。方源还没有这个底蕴。他只能从现有的流派中,选择一个流派走下去。最多最多,兼修一门。

再多,现状就不允许了,精力牵扯,时间分散,会让他自取灭亡。

现有的流派当中,又是哪一派,战力出色得能够让方源渡过现状,又能充分发挥自家特长,更且兼顾境界、传承等等因素呢?

多番思考之后,方源的脑海中,就只剩下一个流派。

那就是变化道!

变化道号称“以一道映射万道”,修行变化道的蛊仙,可以变化做金道、雷道、剑道、光道……其余所有流派的存在。

但变化道有个从古至今都存在的修行缺陷,那就是变化道的蛊仙,变化不同形态时,需要清理道痕,防止道痕之间相互干扰、掣肘。

这个缺陷,对方源而言,根本不存在。因为得益于至尊仙胎蛊,他身上的道痕互不干扰。以至于戚灾追杀他时,一度认为方源掌握了狂蛮魔尊随意变化的绝世奥秘。

可以说,变化道最能发挥方源的长处。

还可以说,方源是当今,哦不,是纵观古今,唯二的适合变化道修行的人物。

更妙的是,一旦修行变化道,那么对方源浑身道痕互不干扰的这个秘密,也是一项完美的遮掩。

方源可以变化之后,堂而皇之地运用各种流派的仙蛊,而不会引起他人疑心。

而且变化之后,再运用相应流派的仙蛊,无疑更增威能。比如变作剑道猛兽,身上增添无数剑道道痕,再运用剑道仙蛊,无疑威力增幅更多更强。

变化道的境界,方源虽然不高,但没有关系,狂蛮真意在呢!

虽然方源没有变化道的传承,但关隘也不大。为什么呢?因为变化道的杀招很容易就能得到,完整地变作一个形态,就是变化道的一个杀招了。

至于变化道的相应仙蛊,也是所有流派中,最平民,最容易炼制的蛊虫。对于方源而言,可以迅速积累,重新开始。

另外方源还有其他考量。

比如琅琊福地。

这一任的琅琊地灵,认可毛民身份,方源若是修行有成,完全可以变作毛民蛊仙取信他,得到他更多的支持!

将来遭受追杀,他可以变作其他事物或者人物,蒙混过关。若是现在的身份也暴露了,天地广袤他却举步难行,那就变化伪装,混淆视听,照样行走江湖。

“从今以后,我就是变化道蛊仙!”

ps:最近整理大纲,整的我头都大了。新的细纲方面,我又有了一些新的,刺激的想法,所以在修改当中。但牵一发动全身,写书越写到后面就越发艰难,尤其是这本书的字数已经远超我过往的极限。每多前进一步,对我而言,都是一个挑战!在新细纲没有彻底成型之前,剧情还是以缓慢为主的。大家无聊时,可以关注蛊真人公众号,蛊真人微.信里还有有很多好玩的东西,给大家解解闷。今天微.信上也有哦。(未完待续。)

\end{this_body}

