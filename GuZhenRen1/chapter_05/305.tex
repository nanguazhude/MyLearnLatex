\newsection{四元思潮}    %第三百零五节:四元思潮

\begin{this_body}

%1
这种蛊方,当然实用价值极低了。

%2
因为蛊方上面的蛊材,有好多都已经灭绝,不存于世。

%3
方源若是要炼制这只凡蛊,只有修正蛊方。

%4
他当然不会做这种无用功,有了这些情报,他已经对这只凡蛊深刻了解。

%5
接下来方源阅览了那些阵道的凡人传承。

%6
这充实了他的一些阵道基础。

%7
尽管他对阵道一直没有专门主动研究,但前世五百年的经历,或多或少还是旁敲侧击,或者机缘巧合,接触到一些阵道内容的。

%8
所以充实的程度,并不多,很有限。

%9
只是这一次,他对梦境中的考验,有了针对性的了解。

%10
他搜集的阵道情报,也是有目的性的,在某些情报中他得知:在近古时代,阵道方面的有一个流行的思潮。

%11
这股思潮,简称为地水风火,认为这四种基元,可以让蛊阵变得相当的稳定,构成的最基础的蛊阵,包容性最强。

%12
当然,这种思潮只是流行一时,几十年后,便被世人抛弃。和整个阵道的发展历史相比,几十年的时间太短,这股思潮就像是一朵小小的浪花,泯灭在历史汹涌的长河当中。

%13
“就像是人类起源。地球上一度认为,人由神造,不管是华夏的女娲造人,还是西方的上帝造人。这就是一股思潮。统治了人们的思想认知很长一段时间后,被达尔文的进化论所攻破。但达尔文的进化论,究竟是否直指真相,随着科学的进步,人们相继发现仍有瑕疵。”

%14
方源思绪随意发散了一下。

%15
这股地水风火的思潮,也是如此。

%16
因为随着阵道的发展,人们相继发现,地水风火四基元所构成的蛊阵,并非是包容性最强的蛊阵。

%17
人们不断的发现,不断地深入认知,不断地进步。即便是不久后被抛弃的思潮,也是当时的一种进步。

%18
这个情报相当有价值,直接给了方源一个通过第一幕梦境的正确答案。

%19
“原来,要真正组成一个蛊阵,就必须将全部的蛊虫都运用,掺和在一起啊。”方源心中顿悟。

%20
同时也有些感慨:“这个图事成,还真是有些阴险,明明是要四只蛊虫一个不落,全部利用,居然还打马虎眼,说什么至少要运用到两只他派凡蛊。”

%21
紧接着,方源又陷入沉思当中。

%22
因为即便是知晓了正确的答案,但是要如何才能将四只蛊虫,围绕着阵心蛊,最终搭建成一个有效的蛊阵出来,还是未知。

%23
如果这四只凡蛊,都可以炼制出来,方源完全可以在梦境之外,现实当中,可以尝试推演。

%24
这也是梦境前期,那些门外汉努力破解梦境的常见方式。

%25
但现在,那只土道蛊虫却是难以炼出,就算改造蛊方的话,炼制出来的土道凡蛊,真的就和梦境中的一样吗?

%26
要知道这可是梦境,梦境中有时候和现实也存在着偏差,并非是现实的翻版。

%27
方源思考了一下,还是决定到梦境中进行尝试。

%28
第二次尝试,一切犹如上一次重演。但是这一次,方源选择将这四只凡蛊都同时催动,和阵心蛊匹配。

%29
失败了。

%30
四只凡蛊同时自爆,方源受伤,被踢出梦境。

%31
第三次,方源选择土道蛊虫为先,水道蛊虫第二,结果再次失败。

%32
第四次尝试,第五次尝试……直到第七次,方源终于成功。

%33
围绕着阵心蛊,四只凡道蛊虫相互飞舞旋绕。水道、炎道两只蛊虫,飞成椭圆形状的轨迹,距离阵心蛊忽近忽远。而土道、风道蛊虫,前者悬浮在阵心蛊的下方,而后者则悬停在阵心蛊的上风,两者俱都一动不动。

%34
这四只蛊虫在阵心蛊的调动调和之下,形成一股玄妙的力量,不断地外面散发出七彩朦胧的光芒。

%35
置身在这些光芒当中,蛊师的真元以两倍的速度,开始快速恢复。

%36
“对于蛊师而言,非常实用的蛊阵!”方源心中评价。

%37
“哈哈哈,不错,非常好,不愧是我图事成的儿子,果然是有阵道的天赋!”图事成见到方源成功,哈哈大笑,非常开怀的样子。

%38
方源:“……”

%39
这片天地和山丘,就在方源的视野中,缓缓消散。

%40
梦境的第一幕通过去了,第二幕开始。

%41
方源吐出一口浊气,第一幕他始终坚持没有动用解梦杀招,最终消耗的只是十多颗胆识蛊而已,成本还是很低廉的。

%42
没有仔细体悟自身阵道境界,有了多少的提升,方源将自己的注意力都集中在眼前。

%43
他发现此时自己,居然置身在一个牢笼当中。

%44
巨大的牢笼中,还有一个小牢笼。

%45
这个小牢笼里,却是困着一头漆黑的山豹,它干瘪的肚皮,还有冒着饥火的眼眸,都显示出了这头山豹的危险。

%46
“怎么回事?”方源纳闷。

%47
这时,图事成站在大牢笼之外,对着方源道:“我给你一盏茶的时间,将这些蛊虫都布置成蛊阵。一盏茶的功夫后,这头山豹就会重获自由,到那时,我是绝不会出手的,就看你能不能运用自己的蛊阵,抵抗住这头山豹了。”

%48
“什么?!”方源顿时瞪眼。

%49
这种教育方法,喂,你真的是做父亲的吗?

%50
“你是我图事成的儿子,一定是可以的。如果你做不到,那你就不配成为我的儿子!”图事成接着道,面无表情。

%51
“我去!”方源一咧嘴,看向自己。

%52
他发现自己并未长大多少,和第一幕时差不多,顶多十四岁罢。

%53
“这个图事成,看样子应该是正道蛊仙,居然如此狠心?或许这只是一场恫吓,逼迫他的儿子充分发挥出自己的潜能?”

%54
方源在心中猜测。

%55
“谁当了这个图事成的儿子,也真是命途多舛。”

%56
时间有限,方源很快平复心境,开始检查自己手头上的蛊虫。

%57
山豹只是一只普通猛兽,但是单凭他现在梦中的少年身躯,肯定是不能抵抗的。

%58
只能依靠布置蛊阵,进行抵挡对抗。

%59
蛊虫只有五只,仍旧是五只一转凡蛊。其中一只阵心蛊,其余四只分别属于土道、风道、水道、火道。

%60
和之前的第一幕,相差不多。

%61
不同的是,这四只他派的蛊虫,虽然流派一样,但具体的蛊虫并不相同。

%62
这些蛊虫又该如何组合,才能形成蛊阵?

%63
方源开始尝试。

%64
有了第一幕的宝贵经历,方源手法熟练。

%65
第一次尝试很快失败了,在梦境中的方源大吐一口鲜血,反噬极重,庆幸的蛊虫没有损伤。

%66
他也没有被逐出梦境。

%67
“看来只要一盏茶的时限没到,我就能继续尝试么?”

%68
“虽然还有继续尝试的机会,但实际上,却希望渺茫了。”

%69
方源皱起眉头。

%70
受伤之后,蛊虫虽然没事,但是自己状态非常不佳,真元有限,又受了伤。

%71
方源很快发现,因为伤势的拖累,让他再继续尝试的时候,非常不稳,蛊虫虽然飞起来,但摇摇欲坠。

%72
第二次尝试,也宣告失败。

%73
虽然时间还有富余,但方源已经无法继续尝试,因为蛊虫已经在第二次失败的时候,已经死了一只。

%74
若是在那四只他派凡蛊,也就罢了,偏偏是方源的阵心蛊毁灭了。

%75
“我的蛊虫毁了,父亲,能够我第二只阵心蛊吗?”方源忙问道。

%76
但换来的答案,只是图事成无情的摇头:“设身处地,你若是处于战斗之中,谁会给你第二只替换的蛊虫?儿子,你太让我失望了,你已经没有机会了。”

%77
果然时限一到,小笼子打开来,山豹扑了上去,一下子就咬断了方源的喉咙。

%78
这个梦境非常的真实,方源可以清晰地感受到,自己的喉咙被咬破,呼吸变得极其困难,皮肉被山豹的利齿一口口撕扯下来,痛彻心扉。

%79
血液从他身上的任一伤口,肆意横流,被山豹满足地畅饮。

%80
图事成缓缓摇头,叹息一声:“你不配做我的儿子。”

%81
随后,转身离开。

%82
“还真的没有救啊,这图事成究竟混的正道,还是魔道?”被梦境逐出,方源魂魄归体,心中自然郁闷。

%83
运用胆识蛊治疗好伤势,他继续探索。

%84
第二次失败,第三次失败,第四次失败……

%85
每一次失败,方源都要被山豹扑杀,重新体验一遍自己被猛兽吞食的痛苦悲惨的经历。

%86
若是换做旁人,恐怕早已经心神崩溃。

%87
不过方源却无所畏惧,他所遭受的痛楚,早已经凌驾于此不知多少倍数。

%88
只当是清风拂面,心神仍旧专注在如何搭建出蛊阵。

%89
第十次失败,第十一次,第十二次……

%90
这第二幕梦境,比第一次梦境要难得多。方源被逐出梦境之后,魂魄的伤势也比第一幕时,更加严重。

%91
第一幕需要两只胆识蛊即可康复,但第二幕时,却需要至少三只胆识蛊,有时候甚至是四只。

%92
方源冷静地计算着成本。

%93
“这样下去,成本越来越大了,难道要用解梦杀招吗?”

%94
“等等。”

%95
苦恼之际,方源忽然灵光一闪。

%96
他进入梦境,继续尝试。

%97
有门!

\end{this_body}

