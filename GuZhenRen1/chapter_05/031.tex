\newsection{榜上有名,天庭追剿}    %第三十一节:榜上有名,天庭追剿

\begin{this_body}

中洲,狐仙福地。哦亲

紫薇仙子缓缓睁开双眼。

她一身锦绣紫袍,难掩窈窕身姿。眼若幽潭,眉宇间笼罩一阵哀愁之意。肤若白雪,青丝垂至腰间,浑身八转气息充盈洋溢。

吐出一口浊气,她收回智道杀招,从半空中降下。

“小煌参见紫薇祖师。”在地面上,凤金煌早已经等候多时,见到紫薇仙子,连忙恭谨的跪拜在地。

紫薇仙子嘴角微翘,露出一丝罕见的笑意,顿时眉间哀愁消淡,整个人仿佛绽放出光彩来,丽颜仙容,动人心魄。

她开口道:“煌儿起身,不必多礼。说起来,你母亲还和我有些血脉联系呢。”

看得出来,她对凤金煌这位后辈十分满意。

凤金煌站起身,神态恭敬,又带着仰慕之色。

眼前这位,可是八转蛊仙!

在凤金煌心目中,九转境界她不敢奢望,八转却是她的毕生目标。眼前这位紫薇仙子,出身于灵缘斋,门派中辈分极高。她至少已活了一千六百年,修行智道,加入天庭。人又端庄美丽,可以说,具备凤金煌崇拜的一切条件,简直就是后者理想目标的现实参照。

“紫薇祖师,在狐仙福地中推算了三个月,可有所获吗?”凤金煌好奇问道。

紫薇仙子微微点头。

此时,她已经降落到凤金煌的面前。只是脚未着地,悠悠悬浮。和地面还有一段距离。

她生性如此,有些洁癖。

本来这些事情,没必要和凤金煌多说,但紫薇仙子却仍旧开口,答道:“经过这些天的推算,我可以肯定。这片狐仙福地的原主人狐仙。便是逆组织的一员。”

“逆组织?!”凤金煌瞳孔微缩,神色讶异。

她虽是凡人,但双亲都是蛊仙名流,清楚很多世间隐秘。

这逆组织她也曾有所耳闻,乃是中洲的一些蛊仙,秘密联合起来,隐藏在暗处的一股强大力量。

中洲十大古派,占据绝大多数的资源,把持大权。控制或者压制着其余势力或者散修魔仙。

哪里有压迫,哪里就有反抗。

联合抵抗中洲十大派的组织,不再少数。

但这些组织,往往存留不久。就被十大古派或是攻溃,或是离间,分崩瓦解。

然而,这逆组织却是当中隐藏最深,存在时间最久的联合力量。

十大古派不是没有针对过,但收效却不大。关键就是,这个组织隐藏得真的太深了!

甚至组织中的各个成员。相互之间,都不知道对方的真实身份,只以数字代号相称。

现在,凤金煌冷不丁地听到,原来狐仙就是逆组织的一员,这着实有些出乎她的意料。

“狐仙表面上是一位散仙,但当年却机缘巧合,能获取荡魂山。现在看来,是有逆组织在背后帮衬的。”紫薇仙子继续道。

凤金煌目光一闪,想到当初自己惨败在方源脚下的情景,猜测道:“这么说来,方源也是逆组织中的一员了?难怪他当年,忽然出现,又那么巧合!”

这次,紫薇仙子却微微皱起眉头:“我推算出来的结果,却是似是而非,是又不是。但不管怎样,方源和逆组织必有很深的瓜葛。小煌,你一直留着狐仙地灵,没有成为这片福地的真正主人,是还想将来把这片福地还给方源吗?”

“紫薇祖师……”凤金煌顿时额头冒冷汗。

紫薇仙子淡笑一声,安抚道:“我们是自家人,你不用担心。我知道,方源无疑间救了凤九歌。你父亲这个人,才情惊艳,天赋惊人,历史上都属一流的人物,更难得的是,他恩怨分明,有情有义。他让你这么做,我也很理解。”

凤金煌连忙道:“紫薇祖师宽宏大量……”

紫薇仙子摆手,阻止凤金煌的话,语气微沉,继续道:“但你要记住,方源是我天庭通缉的要犯,诛魔榜上有名!你要报答他的救命之恩,也要适可而止。将来,你真的将狐仙福地归还给他,我不会阻止,甚至还会替你们担待下来。但这一点,已经是极限,是底线,希望你们父女二人,不要执迷不悟,犯了大错。”

凤金煌低头:“祖师之言,煌儿必定铭记心中。”

紫薇仙子微微颔首,随后身躯飘摇,悠然飞上高空。

她伸手一划,狐仙福地中顿时空开一个缺口,直接沟通了外界。

随后,她化作一道紫光流彩,宛若流星般飞射出去,瞬间消失在凤金煌的视野当中。

紫薇仙子一路飙飞,直至天庭。

天庭中,宫殿重重,正大光明。

紫薇仙子化身的流星,划破长空,落入一座大殿之中。

大殿里,紫薇仙子本体正盘坐在蒲团上,与另外两位天庭蛊仙商议着事情。

紫光闪耀,投入她的手中。

化作一股意识,还有数只仙蛊,几颗仙元。

原来,之前留在狐仙福地中的“紫薇仙子”,近乎真人,但真正的身份竟只是紫薇仙子的一份意识!

这份意识,带着推算成功的结果,一路向上,直至紫薇仙子的脑海之中。

紫薇仙子将仙蛊和仙元,都收入仙窍里面,缓缓闭上双眼,获悉推算成果。

几个呼吸之后,她便睁开幽深的黑眸,淡淡地道:“已经找到了方源的线索。”

她面前盘坐的蛊仙老者,号称万海龙流,顿时微笑出声:“紫薇仙子不愧是天庭中,与监天塔主齐名的智道好手,居然用一股意识。就推算出了关键线索。”

意识越是思考,就越禁不住损耗。就连巨阳仙尊在八十八角真阳楼遗留的特意。都要选择沉眠,避免因为思考,而损耗自身。

但紫薇仙子的手段,却玄妙高绝。

竟然直接动用一股意志,替她本体来进行推算。不仅成功了,而且这份意识的损耗也不到两成。

如此造诣。实在叫人叹为观止。

紫薇仙子本体眉笼轻愁。对万海龙流的恭维不置可否,继续道:“若非狐仙福地被方源执掌经营过一段时间,我也不会算出线索。方源和逆组织关系密切,我虽然没有得到方源的具体位置,但却知道只要顺藤摸瓜,将逆组织连根拔起,就能寻到方源。”

万海龙流连连点头:“方源身怀春秋蝉,又捣毁了王庭福地,继承了红莲魔尊、盗天魔尊、巨阳仙尊的一些传承。本身是天外之魔,竟又从南疆战场存活下来。这种危险分子,必须要将其铲除!不过要对付他,必须要首先对付他身上的春秋蝉。”

“这点还请放心。我苏醒以来,就立即着手准备。前几日,已经催动杀招成功,这春秋蝉已被我感应到,成功封印,三个月内都无法催动。”在场的第三位蛊仙开口道。

他声音沙哑,中年模样。眼袋深沉,无精打采,给人精力消耗过度,疲惫不堪的感觉。

但紫薇仙子、万海龙流对他似乎都很信任。

后者笑道:“既然是威灵仰你出手,那春秋蝉就不足为虑了。”

紫薇仙子接着道:“那么接下来,就是前往明堂谷,俘虏公孙良了。”

与此同时,中洲,罐河河畔。

影无邪站在河岸上,看着杨柳青青,清风和煦,河水舒缓,心中却是有些焦躁。

绕开了地渊塌方之后,他带领着太白云生、黑楼兰、石奴,又赶了一段路,借助蛊阵,传送到了中洲的东部。

这里是风云府的管辖地区,影无邪顶着方源的肉身,已经成为天庭、中洲十大古派通缉的要犯,却还冒险停留在这里,并且一停留就是三天时间。

他在等一个人。

是什么人,能值得他冒如此巨大的风险,等待会面?

一道青影,低空飞来。

影无邪顿时心头一震,迎接上去。

青影落到地面上,站定,却非真人,而是一个傀儡。

这个傀儡,状似人形,双手双脚,体格健壮。但浑身上下,似乎都是用青草粗藤编织起来,头上,双肩,乃至后背都长满了青草,一根根好似匕首,竖直向上。

青草傀儡看了一眼影无邪手中捏着的蛊虫,辨认出他就是目标,带着傲慢的口吻,嗡声道:“我家主人因被邀请,参加空空老祖的鉴宝会,不能来了。你若要见他,就再等三天罢。”

“再等三天?”影无邪面色微变。

“怎么?你若没有耐心,也可以不等,我这就带讯回去告知主人便是。”青草傀儡冷漠地道。

“等。”影无邪连忙笑道,“我已经等了三天,再等三天又何妨呢?只是三天过去之后,那就是整整六天了。如果再无缘见到你的主人,那我也只好放弃这场交易。”

青草傀儡冷哼一声,转身便走。

几个跨步之后,他双脚轻轻蹬地,身体蹦到半空,再次化作一团青影,疾飞而去。

他走后,影无邪的脸色顿时阴沉下来,望着青草傀儡的背影,目光闪烁不定。

直至青草傀儡飞出影无邪的视野,三道蛊仙身影,从周围显现而出,向影无邪围拢上来。

正是太白云生、黑楼兰、石奴三人。

“刚刚那具青草傀儡,难道就是传闻中的草头神?”太白云生好奇地问道。

“不错。”影无邪点头,“正是草头神,乃是六转草傀蛊凝成,有六转战力。风禅子开窍之时,他的爷爷悲风老人就将草傀仙蛊当做礼物,送给了他的这位爱孙。”

“风禅子高傲狂妄,我们等待他这么久,居然都见不到他一面?最后就派遣一位草头神,来打发我们?”黑楼兰语气不满地道。

石奴叹了一口气,无奈地道:“唉,要炼定仙游,就必须得有太古之光,任何蛊方都绕不过去这道坎儿。而太古之光这等稀罕蛊材,正是风云府的库藏之一。我们有求于人,没有办法。”

影无邪冷哼一声,没有说话。

他要拯救本体,就要穿透界壁,可他不像方源,定仙游是穿透界壁的捷径,近乎“必须”,极其重要。

眼下,他只有暂时忍耐。

他心中暗暗思量:“春秋蝉已经被封印,看来天庭已经开始搜查我了。中洲不能久待,有了定仙游,就赶往北原,上长生天借运!说不定,还能借力顺便铲除掉方源。嗯?不对,算算时间,明天就该是方源渡劫的日子。这至尊仙窍的灾劫,可不是那么容易渡的。天意虽然影响不了方源,但仍旧可以对他布局。所以,可怕的不仅仅是天地灾劫,还有啊。”(未完待续。)

\end{this_body}

