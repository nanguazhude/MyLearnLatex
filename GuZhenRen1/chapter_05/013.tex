\newsection{人祸}    %第十三节:人祸

\begin{this_body}

南疆,烂泥山。[求书网www.Qiushu.cc想看的书几乎都有啊,比一般的小说网站要稳定很多更新还快,全文字的没有广告。]

临近正午,旭日在高空照耀,晴空万里无云。

烂泥山的后山林某处,一位小小少年正和一只大熊对峙。

气氛紧张。

大熊高达九尺,膘肥体壮,浑身的皮毛呈现棕色,浓密而且油亮。此时熊口大张,露出尖锐的兽牙,赤红的双眼紧紧盯着眼前的少年,凶光毕露。

和这大熊对峙的少年,恐怕只有十五六岁的年纪。

他身高不过五尺,和大熊比起来,更显得娇小细嫩。

少年浓眉大眼,目光炯炯有神,勇敢地和大熊对视,毫无退缩之意。

吼!

棕熊猛地大吼一声,张开大口,向少年扑杀过去。

别看棕熊外形笨重,但老猎人都知道,熊兽的爆发力都是极强的。

棕熊由静即动,速度猛地爆发,快得惊人!

少年只感觉眼前腥风呼啸,眨眼间棕熊就已经扑到他的眼前。

少年却面不改色,在间不容发之间,催动蛊虫。

移动蛊虫让他倏地挪移一段距离。

棕熊扑了一个空,一头撞在少年背后的大树上。

砰的一声闷响,粗壮的树干旋即被棕熊撞断。

吱呀声接着响起,大树轰然倒下,砸在地上,又发出一声闷响。

周围大量的鸟雀,往高空慌乱飞逃。

少年暗自倒吸一口冷气,心想:幸亏自己躲避得及时。要不然被这熊兽直接撞到,就算是有防御蛊虫,也得骨折重伤。

不过,既然躲避了这一击,场面就朝少年这一方大大的倾斜过来。

少年眼中精芒绽放,低喝一声:“大笨熊,看本少主的剑气蛊!”

话音未落,他伸出右手,食指和中指并在一处。照准棕熊的背后遥遥一点。

下一刻。

哧。

一声轻响,从少年的手指间,射出一道淡白色,半透明的剑气。

剑气迅速划过半空。射中棕熊的后背。

然而棕熊的身上,忽然扬起一股野蛊的气息。棕熊的后背处,皮毛陡然间凝结起来,形成硬板似的模样。

剑气射中硬板,发出一声闷响。旋即剑气消耗殆尽。

棕熊抖擞全身肥膘,一点伤都没有。

它晃了晃脑袋,已经从刚刚的眩晕中清醒过来,再次掉头对准少年。

少年瞠目结舌。

“有没有搞错?这头大笨熊身上,居然有野生的防御蛊。这还让本少主的剑气蛊怎么打?爷爷啊,棕熊身上的野蛊,不会是你故意放上去的吧?”少年大叫。

“呵呵呵,好孙儿,这头棕熊可是爷爷我走了十多里山路,好不容易寻来的。对你来说。可是个好对手啊。”树干上,传下声音来。

原来,少年的爷爷一直就坐在高高的树干上,看着自家孙儿和棕熊的这场战斗。

少年最大的手段,就是剑气蛊。

但此刻,对付棕熊,却是成效极低。每一次剑气,顶多只能削下点熊毛。

没办法,少年只好四处躲避。

棕熊来势凶猛,但到底是野兽。智慧不足。

少年虽然奈何不了棕熊,但灵性十足,过往的战斗也不再少数。他利用经验,四处躲闪。棕熊屡扑不中,反而撞倒了不少树木。

看到少年狼狈不堪的模样,爷爷哈哈大笑:“臭小子,现在知道你剑气蛊的弱点了吧?它是穿刺性的攻击,一旦遭受克制,就会徒然耗真元。毫无效果。来,孙儿接蛊。”

说着,爷爷手一扬,将一只蛊虫抛给了少年。

少年为了借助这只蛊虫,差点被熊掌拍到,跌了个狗吃屎。

幸亏他机敏,在地上连连翻滚,总算又凶险万分地躲过了棕熊的一记啃咬。

拉开距离后,他挺身一跃,又站了起来。

“这是泥泞蛊!”

少年轻喝一声,认出手中的蛊虫。

这只蛊虫,虽然不是他的,但是被他爷爷主动借予,少年催动起来,毫无关隘。

真元灌注到泥地蛊中,蛊虫绽放出蒙蒙灰芒。

少年手掌轻轻一抖,蛊虫身上的那团灰色光芒,就脱离而出,落到棕熊的脚下。

咕噜噜……

大量的气泡,从棕熊脚下的泥土中冒出来。

眨眼间,这片土地化为一滩软泥。

棕熊两只脚,都陷入泥泞之中,被困住。

它疯狂挣扎,甩出无数烂泥。

烂泥打在少年的身上、脸上,少年紧张得顾不得这些,连忙再催泥泞蛊。

灰色光芒再中那滩烂泥。

棕熊本来已经陷到泥坑底部,猛烈的挣扎中,已经快要爬出来。

但再中灰光,泥坑旋即变深变大。

棕熊四肢都陷入其中,它越是挣扎,就越陷得更深。

棕熊人立起来,但泥泞之深,已经达到它的腰部。

少年第三次催动泥泞蛊,彻底奠定胜局。

棕熊再次陷下去,最终只露出一个脑袋,在不甘地大吼。

“总算胜啦。”少年浑身疲软,一屁股坐在地上,呼呼地喘着粗气。

他满脸苍白,一身的真元,已经消耗殆尽。

一声轻响,少年的爷爷从树干上蹦下来,轻飘飘的一跃就是数丈距离,站到少年的面前。

“臭小子,现在知道泥泞蛊的好处了吧?没有这只蛊,你如何能战胜这头棕熊?”爷爷开口教训道。

少年并不答话,继续喘息了几下后,才冷哼一声,昂首看着爷爷道:“爷爷你这是故意的。不就是想让我放弃剑道,改修咱们倪家的土道吗?”

爷爷曲起手指,一敲少年的脑袋,又爱又恨地道:“小子,你这聪明劲,要是能放到土道的修行上,该有多好啊。”

少年双手捂住脑袋,叫道:“可是我就是喜欢剑道啊。剑气一发,多么潇洒帅气啊。土道土不拉几,爷爷。你看看我身上,都是泥点啊。一场战斗下来,什么风度都没了。”

爷爷闻言,立马瞪起双眼。正要训话。

但就在这时,忽然听到前山传来铛铛铛铛的钟响。

爷孙两个顿时动容。

少年腾的一下,从地上站起来,眺望前山,口中急道:“啊!这是族中的警钟蛊。警钟这么急促。究竟发生了什么事?”

“走!”爷爷更加干脆,大手伸出,提住少年的衣领,双脚连踏,向山前飞奔而去。

少年只感觉耳边呼呼风响,眼前一根根的树影,迅速从他身边退去。

他心中震惊:“这就是五转蛊师的真正实力吗?好快的速度……”

十几个呼吸之后,少年眼前一定,被他爷爷放在地上。

他云里雾里,速度陡降。让他一阵难受,胃里剧烈翻涌,差点吐出来。

“族长大人。”

“见过族长大人。”

少年旋即听到家族一干家老的熟悉声音。

他勉强站稳身体,惊异地发现,他居然已经站在倪家山寨的寨墙上。

少年的爷爷正是倪家族长,五转蛊师,倪坤。

倪坤微皱眉头,肃容问道:“发生了何事?要连连敲响族中的警钟蛊?”

“族长,情况紧急,请看!”

一位家老催动家族蛊阵。这是侦查蛊阵,作用在倪坤的身上。

倪坤双眼闪烁着各种画面,在此刻,他已经看到距离山寨一百里外的景象。

他呼吸一顿。眉头紧皱起来,脸上又露出疑惑之色:“竟然是兽潮!奇怪,这兽潮一年多前,咱们已经渡过。山寨周围的兽群规模,还根本不足以构成兽潮啊。”

“是啊,我们也在奇怪。”

“事出反常必有妖!我建议派遣精锐蛊师。一定要侦查出原因来。”

“还是以防御为主吧。此次兽潮,规模空前,能不能守住山寨,还是个问题。”

倪坤脸色如铁。

危机如此突然,又如此严重。

他的孙子倪健更是有些恍惚。明明之前是风和日丽,宁静祥和,但转眼间就要面临家破人亡的危机。

“兽群来势汹汹,此等规模,更是数十年都未见的恐怖。倪家寨已经面临生死存亡的关头。将三重防御蛊阵,都催动起来!二长老,三长老,你二人速速下去,率领两堂精英,主持天火蛊阵!六长老,你统率药堂,以做后援。七长老,命你检查传送蛊阵。一旦事有不谐,就将山寨中的种子,都传送走……”倪坤连连下令。

众家老亦都明白局势危急,领命动身,尽显精悍之气。

兽群如惊涛骇浪,浩荡翻滚而来。

所到之处,烟尘四起,山林树木被撞断无数。

少年倪健站在寨墙上看到这一幕,心惊不已,脸色越发苍白。

他从未见过,有这样的凶猛兽潮。

一般而言,兽潮都是以单一兽类为主,比如狼灾、虎灾等等。但眼前的这股兽潮,包含无数野兽,狼虎豹牛、鹿狐雀蛇等等,都夹杂其中。

“古怪!这些野兽怎么不自相残杀,居然统一行径,来攻打我族山寨?!”爷爷倪坤沉吟道。

下一刻,倪坤身躯一震,身旁的倪家蛊师也都双眼鼓瞪起来。

庞大恐怖的兽潮,速度陡降,随后竟都停步不前。

难以计数的庞杂兽群,相距寨墙万步之遥,蠢蠢欲动,虎视眈眈。

倪家蛊师们相互对视,俱都惊疑不定。

一头丘虎越众而出。

丘虎乃是异兽,体型庞大,远超寻常,宛若一座小丘。

方源半躺在丘虎背上,半眯着双眼,看着倪家众人。

见到方源,倪家蛊师们纷纷惊呼。

倪健瞪大双眼,他这才惊觉过来此次兽潮竟不是天灾,而是人祸!(\~{}\^{}\~{})

\end{this_body}

