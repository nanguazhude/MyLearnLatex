\newsection{炼道真意}    %第五百三十四节:炼道真意

\begin{this_body}

“换什么?”方源淡淡一笑,继而道,“自然是炼道真意了。”

“啊!你居然知道我派中藏有炼道真意?!”琅琊地灵震惊地差点跳起来,这可是琅琊派最大的秘密,也是所有的库藏中最深最有价值的宝藏!

什么是真意?

真意其实就是一种意志,智道有三元要素,讲究念、意、情。真意便是三元中的“意”,当然不只是真意,还有假意、特意、故意、刻意、敌意、战意等等,不胜枚举。

意有无数,但真意却是当中最具价值的,也最引人入胜,能让蛊修尊者都要追逐不休!

北原的北部大冰原,乃是当初狂蛮魔尊一力打造,蕴藏着狂蛮真意。因此但蛊师在北部大冰原升仙力道,就会勾动出狂蛮真意。蛊仙得此真意,便能引得变化道或者力道的境界凭空上涨。

梦境中也蕴藏真小说+3.意,方源探索的几个梦境,之所以能够令各个流派的境界飙升,本质上是因为他汲取到了其中的真意。

所以,真意往往意味着的,就是境界飙升。

五域乱世,为什么是史无前例,最为波澜壮阔的大时代?归根结底的原因,就是天地各处涌现出来的梦境,让许多探索成功的人,获取了其中真意,使得境界飙升。

境界的积累,往往需要相当长的一段时间。但是因为真意,大大缩短了这个过程,这才让世界各地,人才精英不断喷涌而出,相互碰撞较量,追逐厮杀,营造出空前的大乱世!

真意的重要性不言而已,而琅琊派收藏在最深处的真意,来头更大。

皆因此真意就是来源于长毛老祖。

长毛老祖是什么人?

炼道八转大能,号称古往炼道第一仙!

人族漫漫历史,横跨一千多万年,在这段光阴长河中,炼道最杰出的巅峰共有三座。分别是天难老怪、空绝老仙、长毛老祖。

长毛老祖在炼道上的才干,就算是盗天魔尊、巨阳仙尊都甘拜下风,两大尊者都先后请过长毛老祖出手,帮助他们炼蛊。

有后人统计过,长毛老祖一生,至少炼成了三十八只仙蛊。这还只是从确凿的历史事件中总结出来的,并不算那些传说和故事。若算上野史,传闻等等的话,这个仙蛊的数量足以突破一百大关!

更恐怖的是,这些仙蛊大多数都不是六转级数,而是七转、八转,这样的高层次!

琅琊派的炼道真意,就是长毛老祖死后遗留下来的真意。保存得非常完整全面,方源若是得到它,尽数吸取,他的炼道境界就能立即飙升到准无上大宗师的地步!

这等若是一步登天,一只蝼蚁瞬间成长成史前巨兽。又好比是忍饥挨饿的乞丐,忽然间成了全国首富。

就是这样的概念!

之所以要添加一个准字,是因为无上大宗师境界,是需要在这一个流派上超越整个天地的底蕴,从无到有的完全创新。

这一点,方源当然不行。

他若是得到这些炼道真意,只是完全继承了长毛老祖的全部境界,还没有在这个基础上更进一步,超越整个天地的炼道底蕴。

方源并不奢望什么无上大宗师,在他看来,这个称号更多的是一种至高荣耀。准无上已经绝对足够方源,傲视整个五域两天的所有炼道蛊仙。只要得到这股真意,他将从一个小矮丘,一举成为当中的最高峰!

蛊是天地真精,人是万物之灵。世间没有最强的蛊虫,唯有最强的蛊仙。

而蛊仙何谓最强?

境界绝对是其中最主要的评判标准之一。

有了准无上的炼道境界,就等若有了超然的高度,最雄厚的根基,其余的东西都能依附而来。

用七座荡魂山的赔偿,只换取一股意志?

方源亏不亏?

当然不亏。

再多的荡魂山,也只不过代表着大量的修行资源。而众多的修行资源中,什么样的资源最珍贵?

境界这个答案,绝对是最多人的回答!

“你想要我本体死后存留下来的炼道真意?这不可能!”琅琊地灵一口回绝。

方源对琅琊地灵的态度,早已有了心理准备,他笑道:“太上大长老,我们琅琊派收藏了这股意志,长达三十多万年。这股真意一直存放着,没有丝毫动用,这和死物又有什么区别?难道说,你要将这些炼道真意,留给现在的其余毛民蛊仙吗?”。

琅琊地灵很快摇头:“自从有了琅琊福地,前任地灵就一直在考核,但是这些毛民蛊仙还不够资格。”

“那就给我呀。”方源立即道。

“你?你更不够资格!”琅琊地灵直接地道,态度相当坚决。

方源发出无声的微笑。

眼下,他打交道的这位琅琊地灵,是有大毛民主义,一直坚定地认为,毛民比人族更优秀,应当是由毛民称霸,人族依附毛民生存。

琅琊地灵连那些毛民蛊仙都看不上眼,就更加看不上身为人族的方源了。

“长毛真意乃是我毛民一族的无上瑰宝,就算再不济,哪怕便宜我族毛民蛊师,也不能让你一个人族享用了啊。”琅琊地灵的心里想法大概就是类似于此了。

方源深呼吸一口气,接下来的劝说,将相当关键。

他早已思量妥当,此时郑重地对琅琊地灵道:“太上大长老,眼下的局势你也不是不明白,非常危险!天庭进攻一次不成,还会进攻第二次,第三次……直至完全成功或者完全失败,天庭绝不会放弃的。”

“这点我明白。”琅琊地灵语气沉重。

“我若有了这股炼道真意,我的实力将会暴涨许多,帮助琅琊派抵御天庭更有把握。琅琊派帮助我这么多,这就是我的家,我绝不会坐视天庭摧毁了它!”

“我若成就炼道准无上,我便能炼制仙蛊,改良如今琅琊福地中的超级大阵。你也知道,我继承了影宗遗藏,这里面有许多关于天庭和中洲十大古派的情报,之前咱们只是互换真传,这一次我要将这些情报都拿来与你分享。我已推算出,凤九歌等人是利用定仙游的杀招,入侵福地的。咱们谈妥赔偿之后,我更会利用智慧蛊,想出防备定仙游的手段来。除此之外,别忘了我还有一些下属,这些蛊仙你也清楚,都是战斗精英啊。”

琅琊地灵沉默。

方源说的非常有道理,和平时期也就无所谓了,但眼下这个局势,天庭觊觎,虎视眈眈,完全不晓得下一次攻击会在什么时候到来。

琅琊地灵不是无知,他对天庭了解越多,心中的压力就越大。

琅琊派虽然有天婆梭罗这样的上古战阵,可以媲美八转蛊仙,但之前和凤九歌一战,已经暴露出这个手段的种种弊端。

琅琊派急需更多的助力。

雪民、墨人、石人那边,也能勉强出动一个八转战力。

最强的战力,还是方源这边。

琅琊地灵心中早已明白,拥有逆流护身印的方源,是他们当中最强的一股战力。琅琊地灵,琅琊福地在将来,都需要方源这股力量。

“方源啊,你若是需要炼蛊,我亲自出手就是了,别忘了还有其他太上长老,大家可以一齐帮你啊。”琅琊地灵沉默了片刻后,这才说道。

方源淡淡一笑:“诸位的力量,绝对不容小觑,这一点我早就在之前拜托诸位炼蛊时,就深刻地见识到了。但是诸位的炼道能力,能和准无上相提并论吗?”。

当然不能!

“而且借助他人之手炼蛊,和自己亲自炼蛊,这当中也有许多差别的,不是吗?”。

琅琊地灵再次沉默。

这些都是货真价实的道理,就摆在琅琊地灵的面前,就算他想否认,也不行。

不过单凭道理,是无法说服琅琊地灵的。

方源知道,他还要晓之以情。

“太上大长老,你我之间交流,一直都是交心。我如今的处境,你还不清楚吗?我如今的心意,你还不明白吗?我脱困之后,看到琅琊福地被攻打的消息,真的是焦急啊,心中像是烧了一团火,恨不得立即赶回来。”

“琅琊派就是我的家,偌大的天下,也只有这里才容得下我。我虽然是人族,但更愿意是毛民一族啊。可恨是苍天戏弄,让我生下来就是人啊,而不是高贵的毛民一族。”

“但是我愿意在将来,永远变化成一位毛民!我更愿意永远生活在琅琊福地中,将来和雪儿成亲,团结其他异人,辅佐您成就毛民一族的天下霸业啊!”

方源说得非常动情,自己身上鸡皮疙瘩都掉了一地。

琅琊地灵长叹一声:“方源啊,你的心意我懂的,我早就明白。只是……事关重大,我要好好考虑考虑,深思熟虑一番。我现在……做不了这个决定啊。”

方源眼中精芒一闪即逝,这个情况下,绝不能让琅琊地灵多想。他越想,方源的希望就越小。

于是方源道:“太上大长老,您多思考一下,绝对是对的。不管你思量多长时间,我都会等你的答复。”

“这个时期,我会盯住天庭方面,拼尽全力打探情报,他们一有可疑动向,都会向您禀报的。”

这番话顿时让琅琊地灵做出了决定。

他苦叹一声:“局势危险啊,若是还保守真意,恐怕连琅琊福地都保不住。方源你是自己人,我信得过。老实说,你若是一位毛民,我恐怕早就将整个琅琊派托付给你了。这股炼道真意……我便交给你了!”

“我必不辜负大人您的期盼!!”(未完待续。)

\end{this_body}

