\newsection{算不尽,请你安息}    %第八百四十四节:算不尽,请你安息

\begin{this_body}

紫薇仙子身为智道大能,掌握星宿棋盘,极可能就顺势推算出方源在房家这边的行动,从而进行干预和破坏。

之前也有类似的情况。

方源修行气道,想要收集气道仙材,却没有去勒索南疆正道。

但是到了东海,龙公一开口,就指出气海老祖和方源之间有联系,当时就让方源惊异了一下。

若是方源之前向南疆正道勒索气道仙材,东海争夺龙宫的时候,恐怕就是龙公一眼看穿方源真面目了。

能够瞒住龙公,其实分外惊险,只是一步之差而已。

人生很多时候,不能贪图方便,目光要放长远。若是方源之前贪图方便,直接向南疆正道勒索了气道仙材,搞不好天庭就辨认出了气海老祖的真正身份。

从而,方源也不会得到元始仙尊的一部分气道真传了。

方源生性谨慎。

他的一切冒险,除非迫不得已,绝大多数其实都是谋定而后动。

他挣扎了数百年,走了无数的路,经历生死危难,在尊者、天意之争的缝隙中残喘挪移,终于到达了今天这个地步。

困苦挫折是人生的财富,有这样的精力,方源更不会跋扈张狂,失去定性。

求稳,一切都在求稳!

东海龙宫之争,方源还获得了方正的情报,知道了他曾经两次干扰自己,更知晓了秦鼎菱的存在。

秦鼎菱乃是巨阳仙尊时代的人物,中洲仙后,本是修行金道,而后专修运道,沉眠了三十多万年。

按照方源的推算,秦鼎菱的运道境界至少是大宗师,准无上大宗师也不是没有可能。

对方有这样的运道境界,就算只炼成两三只运道仙蛊,也足以改良杀招,形成全面丰富的运道手段。

在这种情况下,抢炼运道仙蛊的战术意义其实已经不大了。

之前方源重点投资运道,是因为天庭在运道上底蕴薄弱,方源投资一分,能收获十分。

而现在他投资,收获将很堪忧。

就像之前,方源掌握了影宗真传,却没有选择重点投资魂道。魔尊幽魂都被天庭俘虏,这份魂道真传的价值就不大了。专门塑造魂道方面的实力,极有可能就会被天庭破解、克制!

综合了这几个方面,方源这才选择暂停运道炼蛊,而是转投木道。

八转仙蛊屋豆神宫,可是能极大增强方源实力的存在!

天庭一直是方源最大的阻碍。

要阻止天庭修复宿命蛊,并非只是一场战斗,而是涉及到情报、伪装、误导、炼蛊、资源等个全方面的竞争和博弈。

评估对手的牌面是什么,这才能做到有的放矢,从容应对。

所以说,情报是非常重要的东西。历史上,很多场著名的战争,有些胜利看似轻而易举、简简单单,其实暗中的较量,前期的准备,殚精竭虑的思考早已经进行了不知多少次!

无脑突突突,只会被人突死,愚蠢的人根本走不长远。

既然运道方面的投入、产出比率发生了巨大变化,那么现在方源及时改变,就是一种止损。

深思熟虑之后,方源再无犹豫,果断选择了这条路。

时间流逝,五域外界表面风平浪静,暗流涌动,至尊仙窍中因为时间流速更快得多,显得日新月异。

整个至尊仙窍进一步得到了开发。

之前吞并气海洞天之后,得到的上百万气道道痕,已经深刻影响到了至尊仙窍的方方面面。

现在的小九天中,都漂浮着大量的云气。这些云气各带色彩,小黄天中是黄色云气,小白天中是白色云气……

接下来,这些云气中的一部分,会凝聚成团,形成云朵。

方源原本还计划着引进云朵,但眼下单靠至尊仙窍自然发展,就已经发展得极其良好,省掉了方源大量的财力物力的投入。

“道痕才是仙窍发展的根本啊。”

“有了这上百万道痕,单靠仙窍的自行演变,就能产生各种良好变化。”

“当然,也有风险。在这个过程中,我需要时刻关注,并且做出及时的调整,防止这些变化危及到之前的生态。”

方源的至尊仙窍是因为太空旷,上百万的气道道痕陡然增添上去,影响分摊开来,至尊仙窍自己就能承担下来。

在小五域中,也在多地产生了气流。小南疆中有一些名山级别的峰峦,有了山间的雾气。小北原的天空中开始笼罩一层大范围的冰霜雪气。小西漠热气蒸腾,小中洲中产生了多处气流漩涡。

真正增长巨大的,还是最基本的二气天气、地气!

方源可以明显探查得到,在至尊仙窍广袤无比的地下深处,已经积累了相当可观的地气。

而在天空,海量的天气分薄开来,倒没有地气凝聚一团那般明显。

天气继续增长下去,小九天中将会更加云雾缭绕,同时小九天之间会产生最初的薄薄气膜。这层气膜再发展下去,最终就会成为天罡气墙。

而地气则有两个发展方向。

一个是方源不去管它,任凭地气浓缩一团。这团地气就会堆积在一起,天长地气之后,在地底深处,撑出一些地下溶洞。而这些溶洞充满了地气,任何其他外物进入当中,就会被地气侵染,化为石像。这种地气太过浓郁,还会形成地气沙暴,令环境更加险恶。不过在地气沙暴当中,会持续产生大量的土道仙材。

第二个方向,则是方源精心疏导这团巨大的地气。在它的周围,打造、挖通一个个的甬道,让这些地气顺着涌动,分散到四面八方去。

将地气分摊到仙窍各地,就不会有什么地气沙暴产生了。此举会令小五域各地的土地更加肥沃,种植、矿物等等更多产量。而那些疏通地气的细细长长的小通道,会被地气冲刷,逐渐扩大,布满土道道痕,最终形成地沟。

这两种发展方向,各有利弊。

方源没有思考,就直接选择了第二个方案。

土道的仙材他并不缺乏,挖掘通道,打造出未来的地沟脉络,将全面地提升整个至尊仙窍!极大地扩展出发展前景!

至尊仙窍的开发,方源都委派给了下属,这些蛊仙劳力绝不会闲着。

至于方源本体则努力修行着因果神树杀招。

原来的因果神树杀招,只是方源勉强拼凑出来的,以七转成竹仙蛊为核心,其他六转木道仙蛊为辅。缺乏核心律道仙蛊“因”,还有核心木道仙蛊“果”,方源只得用大量的普通蛊虫替代。当时因为木道仙蛊、木道境界的缺乏,方源并没有在此投入过多精力。

如此一来,步骤变得繁琐,牵扯的心神更多,导致因果神树杀招酝酿时间过长,并不能用于实战。不仅如此,核心仙蛊高达七转的因果神树杀招,实际威能反而下降到了六转层次。

经过方源此番努力提升,因果神树杀招的两大核心因、果仙蛊仍旧缺乏,当时成竹仙蛊已经被替换,换了更加正统的木道仙蛊。

同时,因为方源木道境界的提升,因果神树杀招本身得到了巨大的改良。现在方源催动此招,不仅酝酿时间大大缩短,可以用于实战,而且威能上也回升到七转层次了。

“这招用来对付八转蛊仙,显然是不足的。顶多是欺负欺负七转。最主要的还是用于收服豆神宫!”

“至于来因去果杀招,暂时还是不能还原出来。”

有了因果神树杀招,方源跃跃欲试。只要祭出此招,豆神宫方面必定大有突破和进展。

然而,豆神宫在房家大本营,方源之前一直辅助房睇长,日夜接触豆神宫,都没有任何的良机。

如何才能争取、创造出这样的机会呢?

方源还没有解决这个问题,房睇长又来邀请他回归房家。

事实上,在方源外出苦修木道的这段时间,房家已经三番五次地催促他回来了。

但这一次房睇长亲自来信,并不像之前几次,在信中房睇长明确地告诉方源:他已经找到了炼化豆神宫的方法了。只要照此方法开展,必定能炼化豆神宫,将这座仙蛊屋掌握在手!

这不仅让方源产生了浓郁的好奇心,房睇长那边也得到了突破?

“究竟是怎么回事?”方源再次伪装成算不尽,回到房家。

“这便是我的方案,算不尽老弟感觉如何?”房睇长将他的计划告知方源后问道。

方源沉吟起来。

房睇长不愧是智道大宗师,还真给他想到了一个切实可行的办法。那就是以房破房!

房睇长的具体方案,就是在豆神宫中临时搭建出一座仙蛊屋,这座仙蛊屋先是推算刺探,之后自毁,爆发出极强的威能,从而一举冲破豆神宫的内部防御。

房家不像方源,可以在短时间内在木道上得到突破,成为木道宗师。但房睇长充分发挥了房家的优势,来对抗豆神宫。

“只是这样强攻,得到的豆神宫恐怕是残缺的了。”方源道。

房睇长苦笑:“没有办法,房家局势一日不如一日,但只要我们炼成了豆神宫,保守住豆神宫残缺之秘,外人谁会清楚呢?”

方源皱眉:“第一步,是要在豆神宫中搭建出仙蛊屋,这步虽然困难,但依凭我们房家的底蕴,应当是能够做到。但是接下来呢?要刺探豆神宫的更深层的运转奥妙,必须要有蛊仙坐镇殿内的仙蛊屋,时刻抗衡豆神宫,依照豆神宫的运转而催动仙蛊屋。然后在这样的基础上,蛊仙一刻都不能停止催动仙蛊屋的前提下,自毁仙蛊屋,达到冲毁一部分豆神宫,让其再无防御之力的目的。”

“但这样一来,留在仙蛊屋中的蛊仙岂不是相当的危险?”说道这里,方源目光有神,紧紧盯住房睇长。

操纵仙蛊屋最佳的人选,无疑是智道蛊仙。也就是方源和房睇长两人。

究竟派遣谁去呢?

方源顿时理解了房家这一次坚决要求自己回归了。

他可不会去干这种危险的事情!

房睇长当然知道方源的想法,他笑起来:“放心吧老弟,我们不会只让你独自操纵仙蛊屋,整个过程都有我和你联手。事成之后,房家也绝不会亏待了功臣的。”

方源沉吟道:“我可以答应。但整个搭建仙蛊屋的过程,我需要全部参与进来。”

房睇长眼中精芒一闪,犹豫了一下,便点头道:“完全可以!来吧,让我们现在就开始动手。”

搭建殿内仙蛊屋并不容易,这比布置阵中之阵还要困难许多。有很多次,建立起来的仙蛊屋莫名其妙地崩塌。

房家损失不小,但好在这方面的底蕴实在深厚。

再加上方源偶尔真正出手,贡献了几个正确的思路,仙蛊屋终于是搭建成功了。

房睇长便和方源一起,进入圆顶帐篷一般的仙蛊屋内,开始刺探豆神宫的内部运转奥妙。

有了这座仙蛊屋,的确是进展很大。

不久后,房睇长觉得时机成熟,对方源道:“可以进行最后一步了。你准备好了吗?”

“准备好了。”方源点头。

“那就好。”房睇长说着,发动仙蛊屋,一道道锁链猛地从四面墙壁射出,在瞬间将方源五花大绑!

方源大惊失色:“二长老,你要做什么?”

房睇长深深地看着他,意味深长地对他道:“算不尽老弟,房家不会忘记你的贡献的。”

“原来如此!这座仙蛊屋是不是要自毁,你们要毁灭的是我!”方源猛地惊醒过来。

“仙蛊屋还是要毁的。”房睇长道,“但单纯自毁了仙蛊屋,爆发出来的威能只是乌合之众。唯有像你这般的智道蛊仙献祭,才能让我在顷刻间突破瓶颈,短时间内达到更高的智道层次。我指挥自毁而得的巨大威能,才能针对克制豆神宫,将它一举收服!”

“算不尽老弟啊,我也知道对不住你。”

“但是没有办法,房家的处境你也看到了,为了房家的明天,还请你去死。”说着,房睇长对方源深深一礼。

方源慌张起来,破口大骂。

房睇长面色转冷:“你也是房家中人,也算是死得其所了。安息吧。”

话音刚落,他就发动了最后的手段。

------------

\end{this_body}

