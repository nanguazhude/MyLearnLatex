\newsection{与天意同化}    %第三百一十四节:与天意同化

\begin{this_body}



%1
紫山真君左顾而言他,转移了话题。

%2
白凝冰默不作声,影无邪道:“不是因为天意干扰的原因吗?”

%3
紫山真君点头:“的确是天意,不过这却是我主动为之。”

%4
“此言何意?”白凝冰听了这话,不禁流露出诧异之色。

%5
听紫山真君这样的话音,似乎他疯癫呆傻,竟然是主动刻意的行为。

%6
紫山真君幽幽一叹,双眸中似乎浮现出了往昔的烟尘:“当年,还是第一代分魂的时候,我们发现了天意这个大敌,自然要琢磨方法,如何应付。”

%7
“我专修智道,是分魂当中的首领人物,找寻到如何应对天意的方法,自然当仁不让。”

%8
“通过打探到星宿仙尊的真传的一部分内容,再结合我本身的智道造诣,我才研究出了一个方法,便是与天意同化。”

%9
“与天意同化?”白凝冰和影无邪对视一眼,纷纷皱起眉头。

%10
“没有错。天意它浩荡无穷,无处不在,我们当时只是刚刚发现和接触,不像现在通晓了天意的许多秘密和弱点。在当时,我们处处受制,每况愈下,情势越加危机。越来越多的成员,被天意针对而灭亡,希望渺茫无比。”

%11
紫山真君继续道:“在不得已之间,我便设想出了这个不是办法的办法。既然敌人如此强大,我便假意投降,打入敌人内部,探知情报消息。正所谓知己知彼百战不殆。”

%12
白凝冰、影无邪俱都眼前一亮。

%13
放到现在,也能领略到此法的新奇古怪。

%14
天意无形,紫山真君居然想要打入天意的内部,真乃奇思妙想。

%15
紫山真君不只是想想,而是真的做到了!

%16
紫山真君继续解释道:“天意,你们应当都很清楚了。仙窍洞天若是吞并了九天碎片,就会被天意掺入,若有天灵,便会变得痴痴傻傻。天灵如此,蛊仙也同样如此。每当我与天意同化之时,便会变得疯癫呆傻,没有理智。”

%17
“不过,也正因为如此,我才能大大降低天意要剿除我的意图,存活至今。”

%18
“这样做还有一大好处。便是能和天意暂时地融为一体,更清楚地感知天意,获悉它的优劣长短。”

%19
“当我获知天意的想法之后,我便能站在天意的格局之上,为我自己,为我影宗布局。”

%20
“天道不仁以万物为刍狗,在天意眼中,世间万物,皆为棋子。而在命运的棋盘当中,总会有那些一些关键位置上的棋子,引发我的注意。”

%21
紫山真君徐徐描述,不管是白凝冰还是影无邪,都是听得越加入神。

%22
因为他们已经隐隐感觉到,接下来紫山真君的话,将极为重要。

%23
果然,紫山真君接着道:“我与天意同化,发现在天意的布局当中,有许多棋子非常的关键。于是每当我痴傻呆疯的时候,我都会借助天意,刻意接近这些棋子,然后等到我恢复清醒,便努力策反,动用自己的力量,争取让这些棋子为自己所用。”

%24
“在不久前的逆流河一战中,我相信你们已经看到了其中一枚棋子的效用了。”

%25
白凝冰身躯一震。

%26
影无邪恍然大悟,脱口而出道:“紫山真君大人是说太白云生?”

%27
紫山真君点头:“没错,太白云生正是命运棋盘中,一个位置比较关键的棋子,虽然被天意安排布局,但也被我成功影响,最终利用他的力量,帮助了你们脱险。”

%28
太白云生曾经在蛊师时期,碰到紫山真君,得到紫山真君的真传。

%29
在他的身上不仅有天意的布局,还有紫山真君的影响。

%30
所以,当方源承载着天意的临死一击,回到过去,推翻幽魂大计的时候,在这个过程中,方源得到了太白云生的帮助。

%31
太白云生能够归附方源。

%32
一部分原因,是方源本身的智谋。另一部分原因,则在天意作祟。

%33
但到了后来,太白云生和方源分开,反而帮助了影宗。

%34
这是因为,方源反叛天意,在最后关头,没有摧毁至尊仙胎蛊,反而自己用了。太白云生在紫山真君曾经的影响下,帮助影宗,出了大力气。

%35
太白云生战死沙场,表面上是因为方源。但换个角度的话,这是天意和影宗相互角力时的牺牲品。

%36
影无邪听到这里,终于明白紫山真君的意思。

%37
他低呼一声:“我懂了,这么多年来,紫山真君大人足迹遍及天下,影响了许多关键位置上的棋子,他们将是我们冲破超级蛊阵,救出本尊的新力量!”

%38
紫山真君微微一笑:“正是如此。”

%39
南疆,超级蛊阵。

%40
至尊仙窍,小中洲。

%41
一股强烈的翠绿光辉,带着芳香之气,直冲九霄。

%42
无数的蛊虫,在绿色的光芒中,不断飞舞缭绕,稍稍看一眼,就能让人眼花缭乱。

%43
方源时刻保持着高度的注意力,随着时间推移,一部分的蛊虫不再飞舞,而是落入地底深处去。

%44
越来越多的蛊虫,落入地底,它们和之前的蛊虫相互辉映,渐渐凝聚成一股玄妙的力量。

%45
翠绿的光柱越来越小,越来越弱,方源心念一动,一只仙蛊飞了过来。

%46
这只仙蛊赤红如血,浑圆如珠,有鹅蛋大小。珠子上无数氤氲光纹,不断萦绕变化,似万莲盛开,又似流云翻涌。

%47
正是六转仙蛊血本。

%48
方源深呼吸一口气,这次布置蛊阵,已经到了最关键的时刻。

%49
血本仙蛊轻飘飘地投入到了翠绿的光辉当中。

%50
几乎是瞬间,绿光被仙蛊血本的色彩浸染,变成了鲜艳的红色。

%51
并且红光越来越盛,几个呼吸之后,达到了和之前绿光一样最强烈最耀眼的程度。

%52
方源吐出一口浊气,最关键的一步渡过去了,接下来就是按部就班。

%53
鲜红的光辉越来越弱,最终彻底消散。但是芳香之气,却还是残留着缥缈的一股,在洞窟中萦绕不断。

%54
这洞窟宽阔至极,有三百亩大小。

%55
它深处在地下,距离地面有数里的距离。

%56
洞窟不是完全处于地下的封闭状态,而是有上百条通道,仿佛是通气口一样,一处链接洞窟,另外一边则沟通外界,在地面上形成无数的坑洞,黑黝黝,深不见底,却又往外散发出一股芳香。

%57
“盘丝蛊阵布置成功!盘丝洞窟的计划,已经成功了大半了。”

%58
“接下来的工作,就是将之前的那些长恨蛛,移动过来,进行培育。”

%59
“这个过程中,还需要调试蛊阵,大约有一两个月,就能彻底大功告成。”

%60
方源心中充盈着喜悦之情。

%61
池伤离开这里,回归池家之后,方源便又继续探索梦境。

%62
之前那个阵道梦境,还剩下两幕,皆在方源的解梦杀招下,以此瓦解。

%63
一如之前所料,方源的阵道境界,达到了宗师级!

%64
借助宗师级的阵道境界,还有和池伤交流后的丰厚收获,方源轻而易举地又推演出了一个全新的盘丝蛊阵。

%65
这个盘丝蛊阵,仍旧用到血本仙蛊,但增长效果只有一番。

%66
方源之前设想的蛊阵,能有翻三番的奇效,但本身是个无解难题,根本没有布置成功的可能。

%67
能够翻一番,方源也挺满足了。

%68
他手头上的积蓄还有很多,解梦杀招还能催动许多次,再加上池伤离开,替他吸引了火力,让他能够更安静地偷偷探索梦境。

%69
方源万分珍惜这种得之不易的机会,接下来的日子里,他以闭关为借口,专心致志地扑在梦境中。

%70
就在他孜孜不倦探索梦境的时候,一个惊天动地的消息传出,震撼了五域中几乎所有的蛊仙。

%71
中洲十大古派、天庭战败!

%72
他们派遣到北原的强大阵容,并没有成功地救出马鸿运,而是惨败亏输。

%73
三位八转蛊仙威灵仰、万海龙流、碧晨天,战死两位,失踪一位。

%74
杀死八转蛊仙的,便是巨阳仙尊昔日的坐骑,传奇太古荒兽狗尾续命貂。

%75
不过中洲方面并非没有幸存者。

%76
赵怜云、余艺冶子、施正义等人,得到威灵仰的拼死护送,在随后中洲援军的接应下,惊险万分的逃得一命。

%77
这个消息传出,天下大哗!

%78
中洲的实力和底蕴,在五域当中,是公认的第一。

%79
十大古派更是执掌中洲的超级势力,不管挑出哪一个,便能胜过五域中大部分的超级家族。

%80
而天庭作为十大古派的上宗,不仅统治着中洲,而且影响力早已扩散到其他四域。

%81
它在中洲人的心中,高高在上,不可动摇。在其他四域的印象中,一直都被认作天下第一势力。

%82
毕竟三位仙尊入主,世间还有哪一个这样的势力?

%83
但是偏偏这样强势的天庭,居然在北原折戟沉沙。

%84
不过世人在震惊之后,了解到了更多内幕,亦都纷纷表示理解。

%85
因为北原彪悍,本身战力第一,而且此次出手的,乃是巨阳仙尊的后辈子孙,是长生天!

%86
至于天庭战败,也是远征之故,本身实力并未真正发挥出来。

%87
事实上,天庭组建队伍前往北原,也不是为了和长生天对战,而是要找雪胡老祖的麻烦。

%88
不管是什么原因,成王败寇,长生天、巨阳仙尊等等威名剧盛,反观天庭、十大古派则是不可避免的声威大降。

\end{this_body}

