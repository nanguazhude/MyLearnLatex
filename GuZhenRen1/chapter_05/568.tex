\newsection{勾心斗角}    %第五百七十节:勾心斗角

\begin{this_body}

%1
“居然死了?”方源见到这样的战果,也是微微一愕。

%2
“看来,这个羊三目的运气真是不太好。居然连续两次,让我偷盗了杀招中的核心仙蛊。”方源神情平淡,摇了摇头。

%3
这就是运势上的压制。

%4
之前没有压得住凤九歌,凤九歌乃是护道人,运势和方源相差仿佛。但羊三目就普通得多,所以死在了大盗鬼手之下。

%5
方源忽然发现,其实运道、偷盗之间搭配起来,也相当厉害。虽然他的运道真传并不完整,偷道真传更是残缺,但是这两者来头太大,均是源自仙魔尊者。同时继承了两位尊者的真传,这在历史上,也是相当罕见的情况了。

%6
仙道大阵之中,南疆正道的蛊仙们见到羊三目被困在仙道战场中,都变得紧张起来。

%7
池囚催促夏繁道:“快快快,还没有推算出破解仙道战场的方法吗?”

%8
夏繁闭目思索,似乎没有听见。

%9
其他的蛊仙则在讨论,惊疑不定。

%10
“对方施展的是什么仙道战场,居然如此迅速,简直是前所未见的。”

%11
“难怪对方有胆量独自一人前来挑衅,原来是身负奇技。”

%12
“现在就看夏繁大人的了。”

%13
夏繁将这些交谈,都听在耳中,心底大翻白眼,腹诽不已:“这些智道的门外汉!真当仙道战场是随随便便就能破解的了?”

%14
夏繁感受到其中的难度,暗中祈祷:“羊三目啊,但愿你能支撑得久一点。”

%15
随着时间推移,他心中的压力越来越大。

%16
这里只有他一位智道蛊仙,他若不能推算出来,就算南疆蛊仙再多,也不能攻破仙道战场,进去支援。若是他失败,背负一个能力不足的名声还是轻的,更可怕的是有谣言蜚语,说他明明有能力,但是却见死不救。若是这样,他的名誉就要真的败坏了。

%17
但方源的阎罗战场,岂是那么容易破解的?

%18
紫薇仙子能够破解,是因为她是当今蛊仙界中,至少前三的智道大能。而夏繁显然不是!

%19
“推算不出来!这杀招根基很深,来头似乎大得可怕!”又默默推算一阵,夏繁已经是满头汗渍。

%20
南疆蛊仙们还在不断地讨论,夏繁越发感到心烦意乱。他睁开双眼,眼中浮现出一股怒意:“你们能安静一会儿吗?”

%21
众仙交谈声戛然而止。

%22
池囚咳嗽一声:“是啊,要安静下来,我们不要干扰到夏繁大人,要相信夏繁大人的能力,绝不会让我们失望的!”

%23
一番话,直接将夏繁架在高台上下不来!

%24
夏繁顿时阴测测地看了池囚一眼,后者则是一副支持他的样子,让夏繁恨得牙根痒痒。

%25
但他到底是智道蛊仙,念头一转,微微一笑道:“诸位稍安勿躁,你们大概忘记了,羊三目大人最擅长的可是侦查的手段。他既然能够出阵迎战,定然是胸有成竹,想来恐怕是侦查到了什么内幕和真相,只是我们不知罢了。”

%26
其实夏繁和羊三目之间,都相互看不对眼,彼此厌恶得很。

%27
但此时此刻,夏繁却是对羊三目推崇备至,将后者捧得高高。

%28
众仙微微一楞,旋即恍然。

%29
若真如夏繁推测的那样,羊三目这样做,可就不地道了!这是把其他人放在一边,自己出去捞功劳。

%30
现在,南疆各处地脉搏动,时不时就有仙材,甚至是仙蛊翻出面世,谁想困守在这里呢?谁不想出去寻觅仙缘?

%31
羊三目若是捞到功劳,自然就能脱离这里,调派到外面去。

%32
这不仅是羊三目的*,也是其他人想要干的事。

%33
于是很快,众仙交谈的内容为之一变。

%34
他们纷纷附和起夏繁的话来。

%35
“夏繁大人说得是啊,羊三目大人的侦查手段,名誉南疆,谁人不知呢?”

%36
“没错没错,我印象最深的,是他那次侦测到黑天异动。结果汇报了羊家,提前出动,竟在准确的地点,等候到了一场流星雨。据说羊家获取的星道仙材,堆成了三座小山呐。”

%37
“羊三目大人还曾经追杀过魔道蛊仙念忘怀。想那念忘怀狡诈多端,一直潜伏,踪迹不显,就连铁家几次追捕,都以失败告终,结果却栽在了羊三目大人的手里。”

%38
“依我看,羊三目大人此次出击,定然是有着绝对把握。我们要相信大人的判断,不要自乱阵脚。当另外我还要提醒大家,此战羊三目大人得胜之后,也要将他安置在副阵中,严加审核。毕竟,方贼的事情才发生不久啊。”

%39
方贼自然是方源了,方源利用见面曾相识,伪装成武遗海,混入正道时间很长,梦境大战中更是把南疆正道蛊仙耍的团团转。

%40
因此南疆蛊仙们提到他,顿时就义愤填膺起来。

%41
“哼,方源这个魔道贼子,居然混入正道,早晚有一天,他必定遭受制裁!”

%42
“邪不胜正,这是世间至理。或许有魔头嚣张一时,但早晚报应必至。就算是幽魂魔尊,死后魂魄不也被天庭擒拿了吗?”

%43
“我恨不得手刃了这个魔头,唉,只可惜那场大战我未有幸参加。”

%44
池囚没有开口,他用隐晦的目光,扫视夏繁一眼。

%45
夏繁又闭上双眼,沉浸在推算之中。

%46
池囚心中一叹:“不愧是智道蛊仙。”

%47
他刚刚那话,是要挤兑夏繁,结果夏繁却是洞悉人心,两三句话就扭转了战局。

%48
夏繁的话,暗指羊三目贪图功劳,不愿意分润给别人,要吃独食。但场中的其他蛊仙,各个人精,很快就反应过来,听出夏繁话中的另外一层意思。

%49
那就是:若不是这样情况,羊三目真的是张扬肤浅,轻敌冒进,又该如何?最糟糕的是,他若是战死阵亡,我们这群人也都脱不了干系,担负一层“心思阴险,坐视同道仙友奋力死战,自己却龟缩在阵中懦弱不前”的恶名吗?

%50
所以,群仙这才纷纷称赞羊三目。而夏繁者利用了这层心思,将池囚的刁难完美地化解掉。

%51
但夏繁此时的心情也十分糟糕:“这仙道战场,恐怕不是我单凭一人之力,就能破解的。该死!求援是必须的,但在此之前,我也得拿出一些功绩,不能显得如此无能啊。呃?!”

%52
就在这个时候,阎罗战场撤销,方源重新现身,而那羊三目已经彻底不见了踪影。

%53
“这这这!”

%54
“怎么回事?”

%55
南疆群仙无不倒抽一口冷气,夏繁、池囚两位七转更是脸色大变。

%56
情形一目了然,只剩下神秘来敌一位,羊三目恐怕已经凶多吉少。

%57
但这速度怎么会这么快?

%58
羊三目再不济,也有着手段,战力也说得过去,怎么就这么快就阵亡了呢?

%59
对方的仙道战场,很明显有着浓郁的魂道气息,宙道气息根本就没有。也就不存在仙道战场内外存在时差的因素了。

%60
于是,最大的可能就摆在南疆群仙的眼前,那就是这位魔道神秘七转蛊仙的战力,已经到了碾压羊三目的程度!但真的有这种可能吗?或者更多的,应当是他掌握了什么诡谲的偏门手段吧?

%61
南疆群仙震惊之余,还有难以置信。

%62
方源望着眼前大阵,眼中泛着冷光。

%63
本来,他还想伪装成羊三目,撤销阎罗战场,造成假象,尝试混入大阵之中。但当他俘虏了羊三目的魂魄,搜魂之后,却是发现了南疆正道这边的布置,遂打消了这个计策。

%64
南疆正道的底蕴也是非常雄厚,看低他们,就是自己的愚蠢。见面曾相识既然已经曝光,再来复制一次“武遗海”的成功,几乎是不可能了。

%65
哪怕这个杀招,源自盗天魔尊。但这蛊道修行,从未有无敌不败的杀招。况且时间过去了这么久,时代一直在进步。

%66
当然,南疆正道未必能随随便便,就拿出正面看破见面曾相识的仙蛊或者杀招,但他们却可用其他方法,来辨别蛊仙身份。

%67
尤其是在天庭的暗中提点之下,南疆正道之间的身份审核,种类繁多,过程极其严格,涉及多个方面。尤其是通过族中的魂灯蛊等隐秘联系,就算是方源利用见面曾相识,也通过不了。

%68
“魔道贼子,你到底把羊三目大人怎么了?”

%69
“还不快快把羊三目大人放了!”

%70
“说不定羊三目大人,已经用了什么手段,遁出了那仙道战场。”

%71
南疆蛊仙们纷纷开口呼喝,许多人还抱着侥幸心理。

%72
方源神情越发冰冷。

%73
“这仙阵乃是由池曲由亲自布置,若是按部就班地破解,时间太有限,根本来不及。看来只有强攻,我倒要看看此阵的成色!”

%74
念及于此,方源便撤掉阎帝,变作上古剑蛟,披上了逆流护身印。

%75
随即,他又施展出仙道杀招万蛟,一瞬间群蛟嘶吼,杀向大阵。

%76
这番杀招,在今时今日已经成了他方源的标志,见到这样的场面,南疆正道蛊仙们若还是认不出方源的身份,那就只有一个原因,就是他(她)故意装傻!

%77
“啊!此人是方源!”

%78
“天,是那魔头来了!”

%79
“他又来了,简直是胆大包天,恣意妄为,不把我们南疆正道放在眼里啊!”

%80
南疆蛊仙们连连咆哮,难掩脸上的惊恐之色。

%81
除了惊恐之外,还有着畏惧的情绪,被他们努力隐藏着。

\end{this_body}

