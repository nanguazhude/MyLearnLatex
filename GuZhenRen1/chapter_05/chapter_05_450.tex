\newsection{贸易之思}    %第四百五十一节:贸易之思

\begin{this_body}

%1
“尽管天意极其想要铲除我这个天外之魔,将七转的天劫威能提升到了极限,可惜仍旧无法奈何我。”

%2
这一次灾劫,方源渡过并不困难。

%3
为了这场灾劫,他做了充分的准备,但大多数的后手都没有使出来。

%4
影无邪等人算是观摩了一场方源的个人武力秀,组团打了一场酱油,得到的是满满的震撼,以及对至尊仙窍的羡慕。

%5
七转蛊仙总共要渡二十四场地灾,三次天劫,三次浩劫。

%6
方源耐心地等到五大毒物实力最强时,这才动手渡劫,从而刺探出天劫的最大威能。七转的天劫对他而言,已经没有任何的威胁了。地灾就更不可能。唯有浩劫,才是未知数。

%7
具体浩劫的威能如何,方源还不太清楚。因为他没有渡过。

%8
他的七转修为,绝大多数都是因为吞窍,强取豪夺过来的。这一次渡天劫,还是他七转修为时的第一次渡劫。

%9
事后,方源用道可道仙蛊测算,得出这一次天劫,让他收获了一千余道毒道道痕。

%10
今后,方源若用毒道仙蛊,就可有一倍的增幅。

%11
天劫平均是七百五十道痕收获,方源能有超出一千余道,显然是天意将天劫威能增幅到了极限所致。

%12
不过这些收获,并非方源目前所需。

%13
首先,他没有毒道资源需要培养,其次他的毒道境界很低,最后他的毒道仙蛊只有一只妇人心。

%14
显然,这是天意故意选择毒道,给方源下绊子的结果。

%15
“下一次我再渡天劫,恐怕天意会选择将天劫威能降至最低吧。”

%16
方源不用猜,都能料想到这一层。

%17
这一场天劫,让天意发现,本身无法奈何方源后,恐怕就不会再将天劫威能提升到极限了。毕竟这样做,只会给方源增加更多的道痕,除了资敌,一点麻烦都不会制造出来。

%18
“这也就意味着,今后我通过渡劫的道痕收益,将会很低。唯有吞窍,才是正途了。”方源心中了然。

%19
吞窍也是至尊仙窍的优势所在,将来肯定会大加利用,只是目前缺少了上极天鹰,方源也在潜修发展,没有机会罢了。

%20
光阴长河,永不停息。时间匆匆,恍惚间便过了一个月。

%21
期间,方源又渡了地灾,发现果然和他猜想的一样,道痕收获很少。

%22
不过在其他方面,方源进步神速,收获巨大。

%23
魂道方面,他的魂魄底蕴上升到了五千万人魂级数!魂魄凝练至极,已经有了一丝由虚返实的征兆。

%24
更重要的是那仙道杀招鬼官衣,越发宽厚,为方源提供强大的遮护力量。

%25
黑楼兰那边,已将星辰炼化完毕。方源尝试过催动星眸仙蛊,侦查效果非常好。有此仙蛊,极大地增强了他侦查方面的实力。

%26
仙窍经营方面,基础建设已经完成了。

%27
消耗了几乎全部的仙元石储备,还有雪民一族的前后三份彩礼,最后时期,方源还向琅琊派借贷了一批物资,终于是将基础部分补足。

%28
如此一来,方源虽然并未真正喂养一次,但基础已经放在这里,就好比虽然还未吃饭,但一盘盘的佳肴已经摆在桌上,饭菜也在锅中煮着。只要就此发展下去,不出意外,就能收获成功。

%29
仙窍的经营,有七个层次。

%30
第一层次,建设凡级资源。

%31
第二层次,建设仙材资源,能成功喂养仙蛊。

%32
第三层次,豢养和产出仙兽、仙植,形成自洽生态。

%33
第四层次,那仙窍的种种产出,进行贸易,互通有无,赚取利润。

%34
……

%35
方源以前是大跃进,畸形繁荣,一脚跨到第四层次上,还有一只脚却仍旧停留在第二层次。

%36
如今第二层次已经圆满,方源的后一只脚终于迈上来,踏足到了第三层次上。

%37
第三层次,豢养和产出仙兽、仙植,形成自洽生态。

%38
这一层成功的标准在于:这些仙兽、仙植,都被完美地纳入到仙窍的自然循环之中,和其他生命形成一整套自洽的生态体系。

%39
这个方面,方源以前已经达到,现在却有了欠缺。

%40
因为最大的两个兽群,一个是鹰群,一个是年兽群落,都没有真正融入至尊仙窍当中。

%41
鹰兽若干,需要捕猎大量的食物。这些目前,方源都是从琅琊派、宝黄天采购。

%42
年兽以年蛊为食,这些方源都依靠自产,八转似水流年仙蛊在其中发挥了巨大的作用。没有它,方源不可能豢养得住如此庞大规模的年兽族群。

%43
“第三层次的建设,却是要缓一缓。”方源也想继续经营下去,但他不得不暂停,他手中的资本已然不足,要想继续经营仙窍,还得先从第四层次上,先大力发展贸易。

%44
如今方源手中有九项经济支柱了。

%45
第一的经济支柱,当然是胆识蛊。它是垄断性质的贸易,潜力极其巨大,不过投入也很大,而且又要和琅琊派分摊利益。方源如今的胆识蛊收益,全被他用来修行魂道,这部分的利润就是零。

%46
其次就是年蛊。年蛊的市场,方源还未真正夺下,但经过之前的大手笔,就算他在这方面发展,两三年乃至四五年内,获益都会很糟糕。

%47
但不可否认,年蛊贸易方面,潜力很大。原因就在于方源掌握着八转仙蛊似水流年。

%48
这只仙蛊乃是北原宙道大能黑凡所创,讲真话,乃是八转中的极品仙蛊。不仅容易催动,而且拥有它,就意味着拥有一道成本低廉的批量制造年蛊的生产线。不过也有弊端,那就是会吸引太古年兽,前来自家仙窍捣乱。

%49
黑凡自创出似水流年仙蛊后,就一直着力于解决这个弊端。可惜的他,一来他自身局限,虽是宙道大能,但其他方面却不太可行。二来机缘局限,若他像方源这般,有着百八十奴杀招,也能解决弊端。三来寿命局限,事实上黑凡已经在逐渐处理这个弊端了,太古年兽钓来阵、光阴泄洪逐流阵都是成果,只可惜最终寿命到了。

%50
所以,似水流年仙蛊在它开创者的手中,并未彻底绽放过光辉。到了方源这个外姓人手中,这才展露威能妙用。

%51
怎么说呢,方源也时常感叹:真是时也命也。

%52
年蛊之后,就是龙鱼、长恨蛛这两个大型资源点,对应至尊仙窍里的龙鳞海域、盘丝洞窟。当初方源在南疆的时候,就想方设法,向武家借力,慢慢建设出来,很是耗费了一番苦功。

%53
但如今,龙鱼生意遭受了重创。因为方源伪装武遗海的秘密暴露,之前的龙鱼生意都是靠着武家的人脉,主要贩卖给南疆的众多正道势力,现在方源身份暴露,这方面的渠道自然就断了。所以,长恨蛛成了方源目前,最主要的支柱产业。这项贸易的最大买家,就是西漠萧家,这层贸易关系是越来越稳固。

%54
龙鱼、长恨蛛之后,就是幽火龙蟒、一系列的星道资源(星镖蛊、星屑草等)、灵蛇、雪怪、光照菌。

%55
幽火龙蟒贸易,起初是方源在王庭福地得到一些种子,后来黑家中得到独家的豢养妙法,一步步发展成今天这个规模。规模不大不小,不如龙鱼和长恨蛛。

%56
星道资源,是方源继承了万象星君的遗产,这部分资源一直都没有重点培养。原本很重要,但现在确实地位越来越低,因为很多资源规模都超过了它。而它毕竟是从一位六转蛊仙身上流传下来,起初规模就不大。

%57
而灵蛇和星道资源的情况差不多,它是韩东福地的产出,被方源吞并,也没有大力栽培,仍旧如常。

%58
最后的雪怪、光照菌,则是未来项目了。前者方源正利用冰道晶精,不断改造仙窍小北原的环境,后者则有重任在身,无法当做货物贩卖出去。

%59
总结一下,其实方源的贸易情况并不乐观。

%60
胆识蛊是没有收益的,龙鱼生意遭受重创,惨淡异常,长恨蛛虽然一如从前,但独木难支。剩下幽火龙蟒、星道资源、灵蛇,对于六转蛊仙而言,每一项都是重要的收入来源,但对方源而言,就规模显小得很了。这三者加上长恨蛛贸易,勉勉强强地为方源支撑着局面。

%61
最后的雪怪、光照菌生意,前者还需要方源不断投入呢,后者不能动,其实也要加大规模。

%62
之前年蛊倾销的巨利,已经消耗得见底,这种情况下,方源急需要重振对外贸易。

%63
他家大业大,虽然基础建设完成,仙窍的发展程度从原先的百风之三,达到了百分之四,但不管是建设仙窍、改良练习杀招、魂道修行,都是需要不断地投入资本。

%64
想要进步,想要提高,自然要先付出。投入的资本越大,进步的程度就越快越高。

%65
对于方源而言,他掌握着海量传承内容,自然也就拥有无数资源经营的妙法,很多甚至都是独家秘传,留给他的选择有太多太多。

%66
但在这其中,究竟哪一个才是最适合自己,最优秀的选择呢?

%67
方源陷入思索当中。

%68
通知:三件事情。

%69
1,明信片已经都寄出去了,因为是平信,许多人收到了,也有少数人没有收到。没办法,我现在已经了解,邮政的平信服务比快递的确差许多。收不到的话,我也无法可想,不过好在这样的活动,还会再次举办的。接下来会改寄挂号信,这样就可以查询了。

%70
2,龙套活动。很多同学想要龙套客串,那么我想再举办一次龙套征集活动吧。参与活动,请各位看官只需阅读起点书评区的相关置顶帖“2016.10.龙套征集活动贴”,然后按照格式跟帖,写上自己心目中的龙套。另注:这一次的龙套,很可能就是最后一次龙套征集活动了。

%71
3,更新情况。最近一段时间,每天一更,要向大家道一声惭愧。但这种更新状况,还要持续到11月中旬的样子。按照目前的进度,本书在今年是万万完结不了的,预计在明年的上半年完本吧。

%72
最后,谢谢大家的支持。我也会一直坚持下去的,虽然走的慢了一点,但一步一个脚印吧。小说向来是越到后期越难写,尤其是这本书乃是我之前的写作生涯中最多字数的一本,可以说,每写新的一章,都是对个人记录的开创,也是一场挑战。我会尽全力将其写好!

\end{this_body}

