\newsection{我不信!}    %第六百零一节:我不信!

\begin{this_body}

%1
灿烂的光影在石室中徐徐消散,凤金煌深呼吸一口气,丹凤眼目光炯炯,一眨不眨地望着眼前。

%2
此刻,在她的面前,有一个光团仿若蚕茧,只是有磨盘大小,静静地悬浮在半空中。

%3
“最后一步了。”凤金煌深呼吸一口气,蓦地掐动双手十指。

%4
刹那间,十指如莲花绽放,无穷的光影斑斓闪烁,映照着整个石室明灭生辉!

%5
凤金煌的手法是如此的娴熟,哪怕是炼道宗师见了都要面露凝重之色。

%6
时至今日,凤金煌早已经是炼道大宗师,有着这样的绝妙炼蛊手法,也并不奇怪。

%7
凤金煌采用的乃是金火双流炼道法门,她已经找寻到最适合她的炼道方式。

%8
轰!

%9
片刻后,一声爆响,蚕茧却未自爆,而是猛地收缩。

%10
原本洁白的蚕茧表面,逐渐化为深幽的蓝色,同时蚕茧逐渐变硬,带出一些金属的光泽。

%11
“失败了好几次,终于是炼成了!”凤金煌眼中闪过一抹喜色,她的心中始终保持着冷静,炼道大能的风范已经开始展露。

%12
五转的蛊虫是炼成了,但是后续却还要有适宜的手法进行处理。

%13
凤金煌张开小口,小心翼翼地吐出一缕缕的冰凉微风。

%14
在微风的吹拂之下,幽蓝蚕茧微微颤动起来,并且发出嗤嗤的声响,大量的热气蒸腾而出,很快将整个石室充斥湿热的水雾。

%15
动用金火双流炼道法门,就是会造成这样的后遗症,蛊虫内部的温度过高,若是不及时降温,虽不致死,但会令蛊虫受损严重,难以应用。

%16
炼蛊博大精深,门道极多。任何一种炼蛊的方法,都有其优劣之处。若是用冰炼之法,来炼制这只蛊虫,那么后续就不需要冷却,反而是要泡在温泉中温养一段时日。

%17
凤金煌收了种种手段,离开蒲团,站起身来。

%18
五转蛊静静地飘到她的手中,她凝神催动了一下,严肃的面庞如冰破花开,绽放出令人感到炫目的惊艳笑容。

%19
她旋即离开密室,来到外界。

%20
打开门的那一刻,瀑布巨大的轰鸣声,鸟鸣啾啾之音,树叶被风催动的沙沙之声,都闯入她的耳畔。

%21
静谧至极的闭关密室已经远去,世界再度生动起来。

%22
青山葱茏,鸟语花香,阳光明媚,一派祥和。

%23
凤金煌目光一扫,果然在水潭边上的巨石,看到龙公盘坐在那里,静默如石。

%24
山光悦鸟性,潭影空人心。

%25
“师父,师父,你看!我把这梦枕蛊炼成了。”凤金煌跑到龙公面前,手中举起刚刚炼出的五转蛊,笑着炫耀道。

%26
龙公缓缓地睁开双眼,一缕目光瞥过梦枕蛊,微微颔首,平缓地道:“不错,不错。”

%27
凤金煌微微撅嘴:“何止是不错?师父你不知道,我为了炼制这只蛊虫,失败了好几次,这一次才成功。有了这只梦枕蛊,凡人蛊师只要头靠着入睡,一定就能进入梦中去。对我灵缘斋,乃至整个中洲,都有绝大利处。”

%28
凤金煌自从拜了龙公为师,一直潜心修行。龙公并未在修行上指点她什么,而是向她灌输整个天下时局,指点江山,提高凤金煌的眼光和见识。

%29
凤金煌得到龙公的悉心教导,如今已是今非昔比,拥有战略目光,考虑问题能从大局出发。

%30
凤九歌炼出来的这种梦枕蛊看似平凡,但正是因为它是凡蛊,才能令广大蛊师受用。只要我们今后大量炼出此蛊,我们中洲的蛊师就能更加轻松地进入自身梦境,挖掘梦道蛊材,在大时代中占据先机。

%31
可以说,这是一种能够提升一州战略优势的蛊,意义重大,非同小可!

%32
龙公却没有丝毫的惊异:“你做成这事,有这样的成就,也是应当的。煌儿,你可是未来的大梦仙尊……”

%33
龙公还未说完,就被凤金煌打断,神情中显露出一丝不甘:“好啦,好啦,你又念叨这话了。难道我取得这些种种的成果,只是因为我是未来的大梦仙尊吗?”

%34
龙公淡笑起来,定睛凝实凤金煌,话锋一转,忽道:“煌儿,你信命吗?”

%35
凤金煌皱起眉头:“师父是说宿命蛊吗?”

%36
龙公点头:“不错,正是《人祖传》中记载的宿命蛊,也是我天庭即将要修复的宿命蛊。”

%37
《人祖传》中记载着,人祖耗费巨大精力,收集蛊材,甚至付出了自己的双手,终于炼成了财富蛊。

%38
他便带着儿子炎煌雷泽,女儿万金妙华,再次来到羽民居住的地方。

%39
但是很奇怪,这么的羽民统统消失,不见了踪影。

%40
“这是怎么回事呢?”人祖疑惑。

%41
“那是因为我来到了这里,那些羽民都害怕我,所以全都跑了。”一只黑白相间的蜘蛛悠然漫步,出现在人祖的面前。

%42
“你是谁呀?”人祖问。

%43
蜘蛛笑道:“人啊,你走过我在生死门中开辟出的路,还不知道我是谁?我就是宿命蛊。”

%44
万金妙华接着问:“宿命蛊啊,你还没有我的手掌大,那些羽民为什么害怕你呢?”

%45
宿命蛊笑道:“因为他们都要追求自由,而我宿命却要束缚他们,限制他们。”

%46
炎煌雷泽抱怨起来:“原来你打的主意,和我们一样。你真是失败,连一个羽民都没有捉到,还连累我们。”

%47
宿命蛊哈哈大笑:“谁说我失败了?这些羽民都在追求自由,可他们知道些什么?他们成功逃脱,都是肤浅的表象,其实我早就束缚住了他们。他们追求自由的路,都是我安排出来的,他们却自以为成功,什么都不知道。你们也是一样,看看你们自己罢。”

%48
人祖、炎煌雷泽、万金妙华便看自己的身体。

%49
他们发现不知何时,自己的手脚还有身体,都粘着苍白色的蛛丝。

%50
他们又发现,不仅是自己三人,连周围的花草树木、石头流水都有蛛丝牵连。

%51
这些蛛丝,一根根汇聚起来,形成一片蛛网,从人祖三人的视野蔓延出去。

%52
“这就是我编织的丝网,叫做万般网。世间的万事万物,都在这片网中,受到我宿命的摆布和操控。你们所遇到的人,发生的事,都是受到我的操纵。”宿命蛊道。

%53
人祖三人心生寒意,连忙挣扎。

%54
宿命蛊便笑:“没有用的,你们不可能挣脱得出。宿命是不可更改的。”

%55
人祖愤怒地瞪视宿命蛊:“宿命啊宿命,你为什么要摆布我们,要捉弄我们?照你这么说,我所遭遇到困苦和不幸,都是你的缘故。我失落了儿女,也是因为你的关系!”

%56
宿命蛊悠然地道:“人啊,我知道你想要救出你的大儿子太日阳莽,可是他已经死了。死亡是人必定的宿命,你根本救不活他的。还有你想依靠财富蛊,来救你的女儿森海轮回,那也是不可能的。”

%57
说着,一条蛛丝就拽住人祖的财富蛊,将它拉扯出来,拖拽到宿命蛊的面前去。

%58
“快放下,那是我们的蛊虫!”炎煌雷泽气得大叫。

%59
万金妙华红了眼眶,抽泣道:“这是我的父亲牺牲了自己的双手,十分辛苦才炼出的财富蛊。你凭什么拿走?”

%60
人祖极力挣扎,但蛛丝却是越来越紧,将他们三人牢牢地束缚在原地,不能动弹。

%61
宿命蛊哈哈大笑:“生死有命,富贵在天。人呐,按照你的宿命,你注定贫穷、卑贱,饱受折磨和屈辱,你会发疯,最终你也必定死亡。你虽然炼成了财富蛊,但你没有这个命来享有它。命若穷,掘着黄金化作铜;命若富,拾着白纸变成布。这种种一切都在我的操控之中。”

%62
人祖和炎煌雷泽、万金妙华都非常生气,痛骂宿命蛊。

%63
宿命蛊一点都不恼怒,悠然自得:“骂我的多了去了,但那又怎样呢?人呐,不管你怎么痛骂宿命,都不会改变什么。”

%64
宿命蛊说着,忽然发力,猛地拽动蛛丝,将炎煌雷泽和万金妙华都远远地抛飞出去,消失在人祖的视野尽头。

%65
“我的儿女啊!”人祖悲号。

%66
宿命蛊幽幽地道:“人呐,你不要怪我,这一切都是你的宿命。其实不止你,孤独是每一个人的宿命。即便是儿女,也不会相伴你一生,总会离你远去。一切的相逢都是暂时的,分别才是正常的。”

%67
人祖却一个劲地挣扎,但他越挣扎,身上的蛛丝就缠绕得越多,将他紧紧包裹住。

%68
人祖感到庞大的压力,并且这股压力越来越大,从四面八方挤压着他,他几乎要窒息。

%69
他张开大口,狠狠地喘息,因为无力,渐渐停止挣扎。

%70
然后人祖呜呜地哭泣起来,泪水滚落脸颊:“我的命,为什么这么苦啊!”

%71
宿命蛊沉默。

%72
但这个时候,一个声音从人祖的内心深处传出来。那是自己蛊发出的声音:“人啊,你与其哀叹自己的命,倒不如相信自己的力量!”

%73
人祖停止哭泣,他忽然意识到:“对,我虽然没有力量蛊,但自己蛊却是吞吃了力量蛊一口,拥有自己的力量。自己蛊,我只能靠你了。”

%74
自己蛊便迸发出耀眼的光,企图撑破蛛丝。

%75
蛛丝被撑破一些,但很快更多的蛛丝把人祖缠绕。

%76
“自己的力量不行吗?”人祖着急起来,“对了,自己蛊啊你不仅咬了力量蛊一口,还咬了爱情蛊一口。力量不行的话,我们就依靠自己的爱情吧。”

%77
于是自己蛊迸发出柔和的光,尝试拉断蛛丝,但同样失败了。

%78
宿命蛊道:“人啊,你怎么还不了解?爱情就是一种宿命,我安排它,令太日阳莽爱上古月阴荒,让石人也爱上古月阴荒。我还将成功和失败,都安排在他们的人生当中,所以最终他们都死了。”

%79
“不!不——!”人祖嘶吼、哭嚎。

%80
宿命蛊静静地聆听着。

%81
人祖渐渐没有力气哭嚎,他有气无力地呢喃自语:“我现在明白了,为什么羽民都要追求自由。”

%82
宿命蛊笑道:“人呐,你也想追求自由?”

%83
人祖点头:“不错,我若是自由,就再不受你宿命的束缚了。”

%84
宿命蛊:“但你看看那些羽民,他们也追求自由,还不是受到我的摆布吗?”

%85
人祖摇头:“我追求的自由,和他们不同。我追求的是绝对的自由。”

%86
宿命蛊哈哈大笑:“一个人的绝对自由,那就是疯狂。人呐,你看,你要追求自由,其实就是走向疯狂。我说过的,你会疯。那就是我给你安排的道路,你无法摆脱我的控制。”

%87
“不!我不信!我会用我自己的力量,还有智慧,来得到自由。我不信你的话,我会摆脱你的控制!”人祖反驳道。

%88
宿命蛊的笑声更大了:“人呐,你真的要疯了,你已经神志不清。你不记得了吗?你的自己蛊只是吃了力量蛊、爱情蛊一口,所以你只有自己的力量和爱情,并没有自己的智慧啊。人呐,当你自以为聪明,这就是你要疯了的征兆。”

%89
“哈哈哈,哈哈哈。”这次轮到人祖大笑起来,“我不信你,宿命蛊啊,我不信!我不信世间有命。”

%90
宿命蛊沉默了一下,这才道:“你就算不信,我还是存在的。”

%91
人祖却道:“不,不是这样。当我不信的时候,你就不存在了。我不信命,命就不存在!哈哈哈!”

%92
宿命蛊摇头,叹息:“真是可怜,人呐,你已经疯了。”

%93
人祖披头散发,鼻涕和眼泪糊了一脸,他挣扎,他跪地,他瘫倒在地上四处打滚。

%94
正如宿命所说,他成了疯子。

%95
……

%96
时光悠悠,过往的一幕记忆忽然在龙公的脑海中浮现。

%97
一百万年前。

%98
“你信命吗?”龙公站着,温柔的目光注视着眼前的徒儿。

%99
他的徒儿还只是一个少年,天庭饱满,面容俊朗,双眼闪闪发亮。他有一头黑亮的长发,垂至腰间,而他的眉间有一朵红莲的胎记,栩栩如生。

%100
龙公继续道:“红莲呐,你就是将来的仙尊,必定带领我们人族走向新的鼎盛和辉煌。你必将成功,开创自己的蛊虫和招数,你将成为你父母的骄傲,你会无敌天下,你将名垂青史。你会入主天庭,成为人道的领袖,福泽苍生,光耀宇和宙。”

%101
少年红莲眨了眨双眼,然后笑起来,露出白得有些耀眼的整齐牙齿:“这似乎没有什么不好。我信命!”

%102
龙公恍惚了一下,回到现实当中。

%103
他凝视凤金煌,郑重地道:“煌儿,你要明白,你将是未来的大梦仙尊,超越过往一切尊者!你将开创梦道,纵横世间,所向披靡。你必定光耀千古,成为我们人族不朽的丰碑。不要害怕,不要犹豫,你将一次次收获成功,一往无前,勇猛精进,直至你走上世间的最巅峰!”

%104
凤金煌听着,眼眸中的光越来越亮,她笑起来,美不胜收。

%105
龙公也微笑。

%106
凤金煌道:“如果这一切都是命中注定……那我就不信命!”

%107
“嗯?”龙公脸上的微笑僵住。

\end{this_body}

