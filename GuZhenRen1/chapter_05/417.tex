\newsection{这个男人与天作对}    %第四百一十八节:这个男人与天作对

\begin{this_body}

西漠。

炙热的阳光,灼射天地,普照万物。

碧蓝的天空万里无云,金黄的沙漠升腾着扭曲的热气。

“告辞了。”方源等人和唐家两位蛊仙告别。

“影宗诸君慢走。”唐烂柯笑着道。

她专修宙道,修为七转,乃是光阴支流的掌控者,这一次方源从唐家的光阴支流中出来,她便是主要提供帮助的人。

方源轻声而笑:“二位已然相送万里,不必再送了。”

唐烂柯表面微笑更洋溢一分,心中却是苦笑。他们和方源合作,想要的远远不只是胆识蛊的贸易,而是梦道方面的诸多成果啊。

但是方源自从回到西漠之后,对于这方面,却一直不提。临到分别之时,他仿佛将此事彻底忘了。

虽然双方结成了盟约,有着约束,但是没有真正的深入合作,唐家方面都有些心虚之感。

唐方明沉默了一路,此时开口,直言询问:“影宗宗主,诸位仙友贤达,不知接下来我们如何合作?”

方源望了唐方明一眼,眼中闪过一抹欣赏之色:“我对盗天梦境很感兴趣,这方面我们可以加深合作。不过此时,我们正遭受天庭方面数位八转蛊仙的追杀,并非合作的良机,唯有期待将来。你们放心,我们结下了盟约,不是吗?当然了,若是贵方意志坚决,我们也可直接留下。只是彼此身份不同,以及天庭追兵方面,压力颇大呀。”

唐方明和唐烂柯对视一眼,均看到彼此苦涩的神情。

方源的意思是,你们要想合作,获得梦道成果,那就帮助我们抵御天庭追兵。并且还要有着,当情况暴露,唐家和魔道勾结,正道名誉扫地的恶果!

这显然是唐家不能承受的。

偏偏唐家不能硬来。

因为方源的拳头,已经足够强硬,丝毫不惧怕拥有仙蛊屋的唐家。

“那在下只能祝福诸位仙友,一路平安了。”唐方明微微一礼。

“嗯,告辞了。”方源点点头,神情淡然,转身即走。身后,诸仙尾随其后。

沙丘上,唐方明、唐烂柯并未离开,一直目送方源等人的身影,缩减成天际的小黑点。

“小明,你这就把他们放走了?”唐烂柯皱着眉头。

唐方明神色平静,瞥了唐烂柯一眼:“不是我把他们放走,而是我们唐家上下,都强留不住对方。”

唐烂柯沉默,良久,吐出一口浊气,叹息道:“唉,我心中惴惴不安,也不知道此事对我族究竟是利是弊。”

唐方明:“我只知道,凡事皆有利弊。往往风险越大,利益就越大。再说此事也是由唐家几位太上长老商定了的,要和方源等人合作。”

唐烂柯立即肃容纠正道:“此事可是我唐家没有关系,只是你我二人私自和方源等人接触!一旦事发,也只有你我二人担待这个干系!”

唐方明呵呵一笑,心中有些不屑。

唐家为了维系正道身份,抛出他和唐烂柯出来,当做挡箭牌。事后若是暴露,也有遮羞的布。

这是正道的把戏,唐方明当然也能理解。

“方源是何等人物?早已经看破唐家的把戏!光阴支流就在唐家大本营附近,停留十日,却是连唐家太上大长老都不露面。如此合作,岂有诚意?既然合作,就应当做出取舍,首尾两端,犹犹豫豫,如何能成?”

唐方明心中嘀咕着,唐烂柯又问道:“你觉得方源此人如何?他说正遭受天庭数位八转蛊仙的追杀,此言是否言过其实?”

唐方明心知,这唐烂柯的问题别有深意,他最好的应对,便是沉默不言。

若他说方源不好的言辞,唐烂柯便可向唐家汇报,唐方明心中芥蒂,让影宗一方不愉,借以推托此次责任。

若他说方源的好,唐烂柯又可收集这等证据。万一将来事情暴露,唐家要那他们俩个当做弃子,掩护家族,唐烂柯便可呈现证据,说自己全然受到亲族某位蛊仙的蛊惑,以减轻自己身上的干系。

唐方明并非愚笨之人,或许在刚刚回归家族的那段时间里,不大适应。

如今,他却是混迹正道的好手。

然而此时此刻,唐方明望着天边渐渐消失的方源等人的身影,心中却涌起一股复杂的情绪。

他叹息一声,直言不讳地说出自己内心深处的想法:“虽是没有见到他战斗的英姿,但接触短短数日,已可断定盛名之下无虚士。方源此人乃是当代传奇,风度翩然,气概宏阔,教人心折。他……可是一位和天作对的男人。单凭这一点,我等就已经万万不及了。”

唐烂柯眼中迅速闪过一抹喜悦之色,对唐方明的话,始终默不作声。

方源疾飞,心中却也在想着唐方明这个人。

在五百年前世的记忆中,这个唐方明是西漠蛊仙界的传奇!

他的地位,几乎可以和当时北原的马鸿运相提并论。

唐方明在幼年时期,因为唐家蛊师高层夺权被殃及池鱼,遭受到唐家的无情抛弃。

他和亲妹妹唐妙在外流浪,吃尽苦头。不过在屡屡奇缘的眷顾之下,他修成了蛊仙,而妹妹唐妙也成为五转巅峰的蛊师。

为了拉拢一位蛊仙,唐家的几位太上长老们舍弃了唐家的蛊师高层,换回来唐方明的回归。

接下来的事实,证明了唐家的几位太上长老的决定,是如此的明智。

唐方明回归唐家之后,不仅提携了自己的妹妹唐妙,也成就了蛊仙,而且还积极探索盗天梦境。

他对梦境的探索,有着一种常人难以企及的天赋和才情。

自行开创梦道蛊虫,为整个唐家的崛起,贡献巨大。

五域乱战时期,梦境在天下四处显现,梦道隆昌,而唐家占据先机,利用梦道威能,一跃而出,成为西漠蛊仙界中最强的超级势力。

其威赫赫,其势昭昭,纵观五域,也是顶级的势力,西漠都笼罩在唐家的威名下,有越来越多的人,称西漠蛊仙界为——唐家盛世!

而在唐家之中,最为引人注意,风采慑人的当代传奇,便是转修了梦道的唐方明!

“现在,马鸿运已经被我杀死,但是唐方明却会因为接下来的合作,更加迅速地成长起来。两者对比,也是让人感慨。”

“不过,按照真正的宿命轨迹,马鸿运本来就是要死的。”

想到这里,方源的眼角闪过一抹阴沉。

从石莲岛上,方源获得的,不只是真传、幽魂的九转仙元,还有幽魂意志借助红莲真传,推算得到的一些珍贵情报。

方源的五百年前世,影宗逆天成功,魔尊幽魂重生,并且成功地潜伏在了天庭当中。

而后,五域乱战,大时代开幕。

影宗将触手遍及五域两天,暗中操纵一切。

马鸿运就是由影宗多次出力保下性命,最终成长为北原蛊仙界中,对抗天庭的标志性人物。

而唐家也是影宗暗中扶持的对象!

没有错。

在方源的五百年前世,唐家也和影宗合作。唐家之所以能够开创出“唐家盛世”,其幕后就是影宗扶持出来,要和中洲天庭作对。

方源选择和唐家合作的很大一部分原因,也在这里。

西漠中,拥有光阴支流的,可不仅仅只是唐家。但是和其他两大超级势力相比,唐家更有合作的基础和诚意。

随后的事实,也证明了方源的选择是成功的。

借助唐家的光阴支流,方源逃脱光阴长河,让另一边的凤九歌等人仍旧苦苦守候着。

“只是……”

“在五百年前世,即便是影宗大计成功,幽魂复生,也似乎未重登尊者之位。并且影宗实力保存,幽魂却仍旧选择暗中和天庭对抗,扶持其余势力。这点证明,天庭的实力即便是当初的幽魂也难以正面对抗!”

“今生,幽魂大计因我而败,自身本体被俘虏,影宗风吹雨打,只剩下我等。局面比之五百年前世,可谓糟糕透顶。”

“接下来,我恐怕也得延续幽魂的谋略,暗中潜伏不动,扶持唐家等诸多势力,找天庭的麻烦。”

“但宿命蛊怎么办?”

想到这个难题,方源不禁眉头紧皱。

五百年前世,宿命蛊始终没有修复成功。因为僵盟健在,而魔尊幽魂这个最大的该死之人,他不仅没有死,而且还复生了。不仅复生了,还潜伏天庭,暗中操纵,酝酿无数阴谋,始终阻挠着天庭修复宿命蛊。

但现在不一样。

摆在方源面前的局势,更加恶劣!

幽魂本体被俘虏,僵盟消失,大量的逆命存在已然毁灭,导致修复宿命蛊更加容易。

宿命仙蛊是悬在方源头顶上的利刃。

幽魂意志重点关照方源,要他千万小心宿命仙蛊,若真的修复成功,结果将不堪设想。

方源牢牢记住这个告诫,但是要如何摧毁宿命仙蛊,他毫无设想,想不出方法来。

打不过就投降?

对于影宗而言,根本就没有投降这个选项。

对于方源而言,更是如此。

“当我在前世五百年,不断辗转奔波,颠沛流离,受到天意的算计、折磨、打压时,就早已经没有了退路。”

“天意选择我为棋子,尤其是当我没有毁掉至尊仙胎蛊,而是为自己催用,这就注定了我将与天作对,别无其他选择。”

想到这里,方源仰望天空。

无垠的苍穹,浩荡的天意!

而其相比,自身是多么的渺小。

与天作对?

“有意思。”方源无声一笑。

\end{this_body}

