\newsection{上古战阵四通八达}    %第六十九节:上古战阵四通八达

\begin{this_body}



%1
进入太丘,布置传送蛊阵的事情,以方源功成身退而结束。,

%2
琅琊地灵也没有亏待他,事后给予了他如数的奖励,一共六百的门派贡献。

%3
用这些贡献,方源可以换取很多东西,其中最吸引他的,还是琅琊派库藏中的许多变化道的仙蛊方。

%4
琅琊派有三大真传,源自巨阳仙尊、盗天魔尊和长毛老祖这三位传奇大能。变化道的真传却是没有的,但琅琊派的仙蛊方数目十分庞大,里面就包含了很多变化道的仙蛊方。

%5
不过,方源没有急着耗用门派贡献,去换取它们。

%6
他选择将这些门派贡献,暂时都保留在自己的手上。

%7
现在还不是换取变化道仙蛊方,炼制变化道的其他仙蛊的时候。

%8
时机并不成熟。

%9
经历了和影宗的暗中交易,以及太丘之行后,方源对自身处境,更加明察秋毫!

%10
虽然成功地布置了传送蛊阵,让他也可以在琅琊福地和太丘之间,迅速往返。但方源却不想再亲入太丘。

%11
再到那里去,就太危险了。

%12
“暗渡仙蛊的力量,已经彻底消退。就算等到一段时间之后,能够再用,也不安全。天意也可思考,之前那次还是打了天意一个措手不及。加上布置蛊阵无法遮掩,天意也知晓这个情况,肯定会在太丘潜移默化,不断影响周围的荒兽群、上古荒兽,甚至太古荒兽!”

%13
既然天意千方百计地想要铲除方源,方源不用猜都知道。接下来天意一定会在太丘方面,围绕着太古荒象的尸骸。进行种种布置,形成重重陷阱。

%14
方源还是不去为妙。

%15
不过。他虽然不去,但琅琊派却会往那里行动。

%16
“既然琅琊地灵已经接受了我的建议,那么他就会动员许多毛民蛊仙,兼修变化道。”

%17
“修行变化道,就要炼制变化道的仙蛊。而炼制仙蛊,就要耗费仙材。”

%18
“琅琊派库藏虽然丰富,有不少仙材。但是上一任琅琊地灵已经执掌福地多年,最勤于炼蛊。使得库藏中仙材虽然是不少,但多是八转级数、准九转级数的仙材。而六转、七转仙材却是稀少。”

%19
“所以琅琊地灵。定会在不久后,颁布门派任务,鼓励毛民蛊仙前往太丘,斩杀荒兽。一来是搜刮仙材,方便炼蛊,二来是锻炼毛民蛊仙的实战能力。尤其是第二点,一直都是琅琊地灵的最大心病!”

%20
“就让这些毛民蛊仙,当做我的马前卒罢!天意能够思考,就算他们被针对。也好过我被天意下手。”

%21
方源眼底深处闪烁着冷光。

%22
琅琊派的整体实力虽然比他强大许多倍,但此时此刻,他却成为了棋手。而琅琊派上下,却沦为他的棋子。

%23
更妙的是。方源虽然算计了琅琊派,但却也是为琅琊派着想的正确发展方向。所以不会对他身负的盟约,有所违背。

%24
“这就是情报不对等。形成的层次差别。才有战略上的优势。”方源感慨,再次领悟到信道的优势。

%25
蛊师各大流派。皆有所长。

%26
信道在其中,似乎并不显眼。

%27
但若能发挥出来。信道带给蛊师的优势,反而比其他大多数的流派还要更加强大!

%28
当然,这一次情报不对等,完全是方源在和影宗交易中的好处。

%29
而影宗为了得到的天意、琅琊派的情报,一定是付出很多高昂代价的。

%30
“信道的手段,我一直很欠缺。今后一定要在这个方面,多加弥补!”方源暗下决心。

%31
信道蛊师,擅长的是情报搜集,盟约建立,知识传承。乍看之下,似乎没有什么,但真正落实到修行上,却是能建立出相当巨大的优势。

%32
中洲,地渊。

%33
轰鸣之声,在此中回荡。

%34
“有发现了!”

%35
“在那个方向,明明我已经搜索过不下十遍……真是隐藏至深呐。”

%36
古魂门的蛊仙,立时反应过来,神情振奋,冲向轰鸣之地。

%37
影无邪吐出一口鲜血,半跪在地上,不仅神色委顿,而且身受重创,几乎半边身子都没了。

%38
“区区仙僵……”与影无邪交手的古魂门蛊仙,悬停于半空中,俯瞰脚下的影无邪,面露不屑之色。

%39
“留下活口!”

%40
“纵是仙僵,但一手隐藏技法的确独到,一定要为门派拷问出来。”

%41
其余两位蛊仙高喊着,划破空气,飞速赶至。

%42
影无邪忽的泛起嘲弄的冷笑:“你们想要我的隐藏之法?可以,我现在就告诉你们!”

%43
不妙!

%44
三位古魂门的蛊仙,齐齐色变,意识到这是对方的计谋。

%45
影无邪故意落败,就是要将己方三仙都引诱过来。

%46
这是诱敌之计。

%47
既有诱敌,必有歼敌的后续。

%48
古魂门的三位蛊仙反应不慢,但影无邪做了充分准备,怎么会让他们轻易逃脱。

%49
一瞬间,本就蓄势待发的超级蛊阵,轰然开启。

%50
古魂门三仙顿时陷入绝境,只感到天地宇宙,四面八方都有无尽的力量,将他们牢牢禁锢住。

%51
一时间,他们身不由己,动弹不得!

%52
随后,在三仙惊骇欲绝的神情下,他们的身躯、魂魄都被超级蛊阵的力量,直接碾至粉碎。

%53
虽然三仙也各有抵抗,但最多不过撑了三个呼吸的时光。

%54
“哼,就先从你们的身上,收点利息。”影无邪冷笑,目光阴寒。

%55
他话音刚落,地渊陡然震荡起来,无边的炫光,刺目耀眼,将方圆数千里,都一齐遮盖。

%56
“好快!这才多久,天庭就已经推算到了我的具体位置,并加以传送!”影无邪心中一惊。

%57
但很快,他的心重新安定下来。

%58
他飞入超级蛊阵。

%59
超级蛊阵经过他这些天的频繁催用,且要对抗天庭的推算,刚刚又强行斩杀了三位古魂门的精锐蛊仙,此刻已经不堪重负,千疮百孔,摇摇欲坠。

%60
不过影无邪毫不在意。

%61
这座营地,他早已打算舍弃。

%62
本来也是保不住的。

%63
“主人。”石奴出现在影无邪的身旁,跟随的还有太白云生、黑楼兰。

%64
后二者仙蛊稀缺,战力下降到了最低点。最强的还是石奴,尤其是影无邪将他的仙蛊,又尽数归还给他。

%65
太白云生担忧地望来:“方源,你这伤势……”

%66
影无邪笑了笑,一摆手,制止了太白云生的话:“不妨事。”

%67
经过和方源的交易,他失去了方源肉身,面貌虽变,但对这三人仍旧有绝对的控制力量。

%68
哄骗太白云生还是很容易的,肉身因为炼蛊损毁的借口,更是恰到好处,不会惹来什么怀疑。

%69
“大敌当前,不便多说,我们先走。”影无邪心念一动,超级蛊阵顿时分崩瓦解,组成蛊阵的数只仙蛊,纷纷投入他的手中。

%70
其余三仙早有默契,默默地围拢上来,和影无邪各站了东、南、西、北四个方位。

%71
临行之前,影无邪望着空中浓郁的玄光,忽的大笑一声,语调昂扬地道:“这中洲……我们迟早会杀回来!”

%72
他经过和方源交易,从绝境中挣扎出了一份生机,整个人都似乎有一种脱胎换骨,石中磨玉的意思。

%73
此时虽是撤退,却让他这么一说,反而士气抖振起来,凭空给人增添无穷希望。

%74
走!

%75
剧烈的光辉,从影无邪等四仙身上爆发。

%76
他们早已演练纯熟,此时各自催动不久前炼制出来的仙蛊,爆发出来的力量,相互融汇在一起,形成玄机妙用。

%77
轰……

%78
巨响声中,影无邪等人消失不见。

%79
而天庭的数位八转蛊仙,则透出漫空的玄光,刚刚来到。

%80
“走的好快!”天庭蛊仙中为首的是一位女仙。

%81
她身姿窈窕,着锦绣紫袍,眼若幽潭,肤若白雪,眉宇间笼罩一阵哀愁之意。

%82
正是紫薇仙子。

%83
智道造诣不亚于监天塔主的棘手人物。

%84
几次出手,翻云覆雨,将影无邪等人逼到绝境。若非方源帮衬,影无邪此刻已必死无疑。

%85
她望着狼藉的地面,掐指一算,顿时皱起眉头:“一共四位蛊仙……奇怪,原本推算出方源在此,但此刻却变了。不管怎么说,都是影宗的余孽,必杀。”

%86
话语轻飘飘的,但若细细体味,别有一股凛然如冬的刺骨杀机。

%87
身后一位天庭蛊仙,悠悠开口:“谅他们这些小仙也逃不出去,这一次,我就亲自出手。”

%88
紫薇仙子淡淡而笑,不过笑意中却泛出一丝苦意:“我们还是不方便出手了。这四人已经逃到了界壁处。差之毫厘,谬以千里。此时此刻,只怕是已入界壁。”

%89
几位天庭蛊仙诧异。

%90
有人道:“这么快?”

%91
“这是上古战阵,名为四通八达。当初仙僵薄青苏醒,联合毛民蛊仙余木蠢、仙僵七星子、血龙宋紫星三位,从落天河源头,挪移到天莲派,就是用此手法。”

%92
“我们虽然掌控了定仙游,但上古战阵却是以人为阵眼,对仙蛊的需求标准更低一些。防备不住这个手段。”

%93
紫薇仙子再掐指一算,随后徐徐而道。

%94
顷刻之间,她竟然将影无邪的这个脱命手段,道个一干二净,明明白白。

\end{this_body}

