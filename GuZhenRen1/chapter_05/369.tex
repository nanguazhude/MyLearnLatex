\newsection{方凤战武庸}    %第三百六十九节:方凤战武庸

\begin{this_body}

“哈哈哈。”仙道战场杀招当中,武庸蓦地发出大笑声。

笑声中,充满了怒意。

很显然,他对凤九歌帮助方源,非常的生气。

“好。”

“很好。”

“你们两位虽然都只是七转蛊仙,却有着八转战力。”

“今日我就好好的领教一下,你们两人是何等功力!”

八转蛊仙的骄傲,还有自信,让武庸胸中的杀意越加充盈、浓郁。

他绝不信自己会输。

就算以一敌二,又如何?

他武庸是八转蛊仙,而对方两位不过区区七转。

七转和八转的鸿沟,岂会轻易就能跨越?

更何况,他还有一座仙蛊屋在手。

玉清滴风小竹楼——八转仙蛊屋!

八转蛊仙加上八转仙蛊屋,怎么会败?

武庸想不出自己会失败的理由。

开战再即,氛围顿时紧张起来。

凤九歌缓缓地飘向武庸,头也不回地对方源道:“你曾经救我一命,现在我还你一命。如今恩怨两情,你走吧,能不能突破这里,逃出生天,就看你自己的造化了。”

方源讶然。

看这样样子,凤九歌竟然不是想他联手,而是想要自己独斗武庸!

“好。”方源立即退缩到角落去。

他是什么样的人?

怎么可能有便宜不去占?

见到凤九歌独自一人,面对自己,武庸心底闪过一丝赞叹,开口道:“好,那就让我会一会闻名已久的中洲天骄。”

说着,他竟主动收起了仙蛊屋玉清滴风小竹楼。

这座八转仙蛊屋,是他的王牌。

由数只八转仙蛊构造建成,消耗的仙元自然庞巨。之前武庸为了布局,为了追赶方源,已经持续催动了许久。

这一次,武庸主动将仙蛊屋收入仙窍。

就好像是未出鞘的神剑,对手永远不知道武庸何时动用这个王牌,心中必定忌惮,压力丛生。

同时,也显示出武庸的自信和骄傲。

他愿意独自对战凤九歌。

战意盎然,跃跃欲试!

武庸身边忽然狂风大起,旋即又衰落下去。

凤九歌、方源俱都目光一凝。

这是武庸施展了仙道杀招。

好快的速度!

而且,之前气息收敛到了极致,不管是凤九歌还是方源,都没有丝毫察觉。

凤九歌连忙后退,谨慎地拉开自己和武庸的距离。

面对八转蛊仙,怎敢托大?

尤其是仙道杀招,不试探一番,就横冲直撞,那绝对是莽夫行径。

狂风衰落,变成微风,微风寰转不休,转眼间凝聚成一个个的巨大形象。

它们形似人,但高大威武,庞大若象,身上肌肉贲发,磊磊如石。浑身青黑之色,口中长有獠牙,一双怪臂粗壮如柱,垂至脚底。通体上下,长满绿毛,双眼全是墨黑之色。

仙道杀招——刚柔风魁!

武庸一招使出,造出六头风魁。

风魁有的大叫,发出呼呼的狂风之声,有的尖啸,几乎要刺破人的耳膜。

它们分成两路,一路三头风魁,不仅向凤九歌扑去,更同时没有放过方源。

凤九歌要独斗武庸,武庸自有骄傲,想要一力对战,将凤九歌、方源二人统统收拾掉。

凤九歌见得杀招面貌,眼中闪过一抹骇人的精芒。他再没有丝毫后退,举掌遥击。

当——当——当——!

他每拍一掌,就会发出一阵黄钟大吕的悠扬之音。

音波震荡开来,接连打在六头刚柔风魁的身上。

刚柔风魁原本气势汹汹,被凤九歌的这一记仙道杀招打得冲势顿止不说,还接连后退,有些难以抵挡,要败落的架势!

这样一幕,看得方源、武庸都瞳孔微缩。

他们无不感到震惊。

凤九歌只是七转修为,但仙道杀招的威能,却是不输给武庸。

不管是方源还是武庸,都看出来,这都是因为凤九歌身上有着极其浓厚的音道道痕。

他的音道道痕,居然积累如此雄浑,竟和武庸不相上下!

这究竟是如何修炼的?!

方源远远观战,心中惊诧不已。

他之所以能够和八转对战,是因为那记仙道杀招——逆流护身印。但凤九歌能和八转对抗,却是因为他本身拥有极其雄浑的音道底蕴。

这两者之间比较起来,自然是方源要大大弱过凤九歌一筹了。

“这凤九歌究竟是如何修行的?居然有着这样的底蕴,难怪他当年能够力抗中洲十大派,以及有自信,来对付武庸!”

“他究竟是真的拥有如此底蕴,还是某种应急手段,不能持久?”

方源旋即又想。

当然,他最主要的心思,还是放在了武庸的这记仙道战场杀招,以及杀招锁风上面,不断地分析、推演。

与此同时,他仙窍中的几位影宗蛊仙,也在为此努力。

啪啪啪!

武庸见刚柔风魁劳而无功,便再次催动又一个杀招。

他的手中,握着一条修长的风鞭,不断抽打。

风鞭似乎有无限长短,随时收缩,甩打出去,每一击都击破空气,打出一声声的清脆爆响。

凤九歌毫不示弱,左手拍掌,对付刚柔风魁,右手则握拳,对抗风鞭。

他左掌拍击空气,每一掌都会伴随着黄钟大吕的悠扬钟声。

他右拳捣击前方,每一拳都会爆发出雷霆战鼓般的轰鸣炸响。

当当当……

轰轰轰……

一时间,凤九歌竟和武庸对攻,打得有声有色,丝毫未处于下风。

把方源都差点看呆。

最主要的原因,便是凤九歌的道痕不输给八转蛊仙武庸。

他是如何修炼出来,这大大有违蛊仙界的修行常理!

要知道,八转蛊仙和七转之间,最大的区别,就在于彼此的道痕多寡。

一场地灾,平均能给蛊仙带来二百五十的道痕。

一场天劫,则是七百五十。

一场浩劫,有七千二百五十道痕。

一场万劫,则是八万六千七百五十道痕。

七转蛊仙和八转之间,道痕往往差距极大。九转蛊仙和八转之间,差距更是仿佛天和地。

这就是九转无敌天下,八转蛊仙,绝大多数会碾压七转的主因之一。

蛊仙修行越到后期,道痕的增长程度是爆炸式的,很难想象的。

原因虽然大家都知道,但是要从这方面着手,简直是太难了。

八转蛊仙本身就意味着,渡劫的数量和质量,都比七转要高。

在七转的时期,就拥有八转蛊仙的道痕底蕴,这几乎不可能实现。若非如此,蛊仙界中早就有七转对抗八转的例子,并且比比皆是了。

但凤九歌却硬生生地做到了。

“若是他临时爆发,使得道痕数量暴涨,这还容易接受些。若是始终就有这么多的道痕积累,那简直是太恐怖了!”方源一边观战,一边在心中评价道。

武庸见凤九歌竟能和自己打成平手,他却不恼羞成怒,反而神情越加平静。

“好。”

“不愧是凤九歌。”

“那么,接下来,再接我这招如何?”

说着,武庸伸手一指。

他的指尖遥遥对准凤九歌。

叮咚一声脆响。

一条碧墨小虫,从他指尖,飞速弹出。

小虫速度极快,直直冲向飞向凤九歌。

飞行途中,它猛地涨大,身躯急速膨胀,一尺、五尺、一丈、五丈、十五丈!

几个呼吸的时间,它化为一头二十二丈的凶恶风龙,张牙舞爪,大有一股要把凤九歌吞噬入腹的凶狂气势。

正是武庸拿手的仙道杀招——指风龙!

凤九歌见此,瞳孔微微一缩,随后身躯电射而去,向后爆退!

指风龙的威能磅礴,凤九歌难以撄其锋芒,只得选择后撤,拉开距离,进行游击。

没办法。

皆因武庸的指风龙杀招,是以八转仙蛊为核心的。

凤九歌的道痕积累,和武庸相差不多,差距很小。但是他却没有八转仙蛊在手。

现在武庸运用八转杀招指风龙,威力极大,凤九歌就难以用那些七转仙蛊为核心的杀招来对抗了。

或者说,只能用数量换取质量。

凤九歌正是打的这个主意。

他一边后退,一边不断出拳出掌,用一次次的七转杀招,叠加数量后,尽量削弱指风龙,直到它消耗殆尽。

但这种事情,无疑会让凤九歌的七转红枣仙元剧烈的消耗!

反观武庸,他却是拥有八转蛊仙特有的白荔仙元。

仙元这方面,凤九歌要大大弱于武庸。

八转蛊仙强于七转,不只是道痕方面,其他方面也是凌驾于七转,仙元只是其中一个例子而已。

武庸一用出指风龙,立即打破了之前的平手局面,将凤九歌压入下风。

“这个杀招很危险。我能感觉到它体内蕴藏的凶暴力量,绝不能让它近身!”

凤九歌心中灵觉非凡。

武庸的确在这一招上,还有一个后招,就是乱弹刃。

一旦用出这招,整条指风龙就会自爆,陡然化作无边的翡翠风刃,四处****。威力非凡,曾经在紫血先河阵中建功立业。

凤九歌虽然是第一次见识这招指风龙,但是凭借他丰富的战斗经验,还有智慧直觉,隐约觉察出了武庸的这记后手。

不过这个发现,虽然让凤九歌免于危险,但也越加令凤九歌被动。

他只能不断后撤,遥击削弱指风龙。

如此一来,就给了武庸充足的机会和时间,来酝酿全新的仙道杀招。

这就非常危险了。

凤九歌也知道这一点,可惜的是,指风龙速度极快,带给他的压力非常的大,再加上那六头刚柔风魁,凤九歌很难再兼顾到干扰武庸这个方面了。

“麻烦了!”凤九歌眉头开始皱起。

照此下去,局面对他会越加不利。

一旦武庸催发出下一个仙道杀招,必定会令凤九歌面临的局面更加不堪。

不过就在下一刻,忽然一个身影电射而来,猛地介入战场!

轰!

指风龙直接撞在方源的身上。

然后,就被逆流护身印反弹了回去。

困扰凤九歌一时的指风龙,呼啸着,向武庸反噬回去。

还在酝酿杀招的武庸,眉头大皱!

“好一招逆流护身印!”这次轮到凤九歌对方源刮目相看。(未完待续。)

\end{this_body}

