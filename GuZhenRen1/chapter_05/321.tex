\newsection{紫血先河陷武庸}    %第三百二十三节:紫血先河陷武庸

\begin{this_body}

“我能追踪得到七幻魔仙。[〉”铁面神的这一句话,让武庸眉头顿时舒展开来。

原来自从铁家俘虏了一代七幻魔仙之后,就开始针对七幻真传进行布置。

在众多的南疆级家族当中,铁家无疑是对魔道打击最为严厉的势力。尤其是铁家镇魔塔建立之后,铁家一举成为正道中追剿魔道蛊仙的顶梁柱。

当即,铁面神催动仙道杀招,对在场的痕迹进行侦查。

“这边走!”辨明方向之后,铁面神旋即向东方疾飞。

武庸、乔志材对视一眼,纷纷紧随其后。

线索时断时续,但在经验丰富的铁面神的追踪下,三仙进展出色。

半个时辰之后,他们竟再次回到的血潮天坑。

乔志材惊异:“怎么又回到了这里?”

铁面神冷笑一声:“最危险的地方,就是最安全的地方么……”

乔志材反应过来,也不禁暗赞一句:“好胆量。”

武庸眼眸横扫,冷光四溢,几息之后,他的目光定格在血潮天坑之中。

嗷吼!

原本平静的血潮表面,忽然汹涌澎湃,从中冲出无数血兽。

大量的荒级血兽,夹杂着少量的上古血兽,成百上千,凶神恶煞,向武庸等三仙扑来。

血兽乃是道兽的一种。

它们不是繁衍下来,而是天生地养,因为血道道痕而凝聚成型。

类似于血兽的,还有雪怪、泥怪、云兽、魂兽、年兽、石龙。

“血潮天坑竟有这等玄妙?是孵化血兽的场所?”铁面神沉声低呼,看向武庸。

武庸亦流露出微微诧异之色。

血潮天坑中藏有血海老祖的真传之一,被一位凡人蛊师商燕飞所得,之后才被武家接手。武家将这血潮天坑纳入自家版图之后,也曾经探索过这里,从未现什么另外蹊跷之处,在之后的日子里,也只是当做寻常的一座资源点。

“有如此多的血兽埋伏,难怪这七幻魔仙隐藏在这里。”乔志材恍然。

血兽结成大军,汹涌而来,它们模样千奇百怪,有的虎头马身,有的蛇躯兔头,有的龟背龙尾,有的宛若大树,挂着猴头般的果子……

腥臭的血气扑鼻而来,耳畔充斥着血兽们的嘶吼,狰狞恐怖。

这样的一股力量,突然爆出来,即便是级家族也要被打个措手不及。

不过,乔志材、铁面神都面不改色,毫无撤退的打算。

因为此时此刻,武庸已经站在他们二人的身前。

“一群杂碎。”武庸冷哼一声,目光中闪烁着怒意。

他很生气。

最近这段时间,他坐镇武家,调兵遣将,左遮右挡,疲于应付,早已憋闷无比。现在又出了这么一个七幻魔仙,用心险恶,图谋不轨,竟害死武家两位蛊仙武原句和戎豪!

一丝一缕的微风,在他身边酝酿。

翠绿的微风,不知不觉间,也在冲锋嚎叫的血兽中围绕。

血兽仍旧汹涌,冲锋的度并未有丝毫的下降。

眼看就要冲到三仙的面前,这个时候,忽然有血兽自爆开来!

这种自爆,并非雷霆炸响,而是非常温柔,仿佛是肥皂泡泡一个个破灭一样。

接二连三的血兽自爆,近在三仙眼前,但毫无威能,伤不了三仙一根毫毛。

之前的咆哮和冲锋,像是一个玩笑。

数百六转战力,数十头七转战力,在武庸的面前,分崩瓦解,脆弱得仿佛是纸做的玩具。

这就是八转之威!

“这是什么仙道杀招……气息收敛到了极致,根本察觉不出!”乔志材满脸的震惊。

铁面神目光深沉,望着武庸的背影。

武庸面沉如水,目光紧紧锁定下方的血潮。

血兽形成,并非没有代价,血潮干涸,原本天坑中几乎漫溢而出的血潮,此刻已经枯竭见底。

乔志材叹息一声,暗暗可惜。他感知到原本浓郁的血道道痕,竟然溃散至无。从此之外,血潮天坑再不存在,只留下一个最普通的天坑地形了。

武庸不管这些。

在常人眼中,深不见底的天坑,在他的侦查中,却是一览无余。

一个浑身笼罩七彩玄光的身影,出现在武庸的侦查范围之内。

“就是你胆大包天,胆敢挑战武家的威严么。”武庸口气淡淡,但任是谁都能听出他语气中的愤怒和杀意。

天坑底部驻留在此的神秘蛊仙,苦涩一笑:“武庸大人,我也是别逼无奈,迫不得已。接下来……多有得罪了!”

“嗯?”武庸面色微变。

下一刻,他周围的环境生骤变!

原本湛蓝的天空和山地,都消失不见,四面八方、上下左右,都化为一片紫红色彩。

三仙不知近,不知远,空间距离感在这里被严重的干扰。

哗啦啦……

浪潮声传入耳畔,刚刚击溃的血兽残骸,都化为了血水,凝聚成一道长江大河,在三仙下方川流不止。

血色浪尖,一朵朵血色浪花绽放,无数的紫色念头,像是暴雨一般,向三仙袭来。

“该死!原来那些血兽竟只是前奏和陷阱而已。好生阴险!”乔志材破口大骂。

铁面神则紧张地四处张望:“这究竟是仙道战场杀招,还是级蛊阵?”

凭他的侦查造诣,竟是看不透此处环境!

武庸面色不愉,冷哼一声,大袖随意一摆。

轰的一声,狂风骤起,一道巨型龙卷风球,罩住方圆一里空间,囊括武乔铁三仙。

紫色念头如暴雨般袭来,打在风球上,噼里啪啦绵绵脆响,突破不了风球防护。

武庸岿然不动,面色冷漠,缓缓举起右手,屈指一弹。

叮咚一声脆响。

一条碧墨小虫,从他指尖,飞了出来。

小虫钻出风球,度极快,飞向血河。

飞行途中,它猛地涨大,身躯急膨胀,一尺、五尺、一丈、五丈、十五丈。

几个呼吸的时间,它化为一头二十二丈的凶恶风龙,张牙舞爪,冲破紫色念头的雨幕,狠狠地撞进血河当中。

血河激涌,形成漩涡,好像是巨兽张口,想要将风龙吞没。

武庸眼中绿意一闪,风龙仰天咆哮一声,陡然化作无边的翡翠风刃,四处****。

血河漩涡直接被风刃绞碎,风刃蔓延飞射,将整条猩红血河都拦腰切断。

乔志材、铁面神皆是瞪大双眼,带着憧憬、震动和惊异的神色。

武庸的每一击,都是仙道杀招,威能凡脱俗,远七转层次。换做他们来做对手,根本撑不下一招半式。

如此强大的仙道杀招,武庸竟然信手施为,仿佛本能,轻易地如同呼吸。

“武庸大人的每记杀招,都是气息内敛到了极致,深藏不露啊。”乔志材心头震动。

“这条风龙杀招,竟还有第二式变招,化为无数风刃……武庸绝不是传闻中那般中庸无能,这般雄浑战力……他竟隐藏如此之深!”铁面神眼眸中精芒爆闪。

他望着武庸的背影,忽然庆幸,铁家和武家一直保持着盟友关系,并未翻过脸。

武独秀、武庸乃是母子关系,前者如山顶之风尖啸凛冽,后者如山谷之风盘旋有力。他们行事风格、性情俱都不同,但相同的是战力凶悍!

不过,在武庸凶悍的进攻下,周围的环境却并未生改变。

断裂的血河,化成两股,居然双双膨胀,形成两道血河,任何一道,都和之前规模相同。

乔志材微微变色,能够在八转蛊仙的攻击下,都能岿然不动,困住他们三人的,到底是何种手段?

他的疑惑并没有持续多久,一个身影在血河中浮现出来,淡淡开口,给出了答案。

“这是紫血先河阵。武庸啊,你若是不用你母亲留给你的那两只八转仙蛊,单凭你那只八转微风蛊,是无法突破此阵的。”紫山真君说道。

他此刻并非小人身躯,而是变大,和常人无异。

他开口的同时,身体缓缓上升,一座“小岛”在他脚下浮起。

乔志材、铁面神脸色剧变,而武庸也是动容。皆因紫山真君并未收敛自身气息,八转身份展露无疑。

“你是何人?有何企图?”武庸低声喝问。

紫山真君微笑,仰望武庸,紫眸琉璃:“自然是杀你。你一死,武家震荡,又受其他家族围攻,必定分崩离析。”

武庸一愣,旋即仰头,哈哈大笑。

这是怒极而笑。

笑了一阵,他猛地低头,双眼中暴射出凌厉的凶芒:“好大的胆子。既然如此,那我就杀了你,真正奠定我的八转声威罢。”

说着,他的身上升腾起一股股强烈的气息。

无数的蛊虫,在他仙窍中被一一催起。

乔志材、铁面神连忙撤退,他们均从武庸的身上感到一股无以伦比的威胁。

紫山真君脸上的微笑收敛起来。

武庸前几次仙道杀招,都是悄无声息,气息收敛到极致。但此时此刻,他酝酿的仙道杀招,气息澎湃,宛若惊涛骇浪。

从这个角度来看,武庸已开始全力战斗!

是什么仙道杀招,让武庸都无法收敛气息?

不管是什么杀招,一旦催出来,必定是石破天惊,威能浩荡非凡。

怎可能让你如愿?

紫山真君念头一动,无数的紫色念头,又好似暴雨,喷涌****而上……

武家。

宗族祠堂。

“这、这!这!!”镇守宗族祠堂的武家蛊仙,忽然面色剧变,冷汗从额头滚滚而下,脸色仿若死人般惨白。

他失声惊呼:“武庸大人的命牌蛊碎了、魂灯蛊灭了!武庸大人……难道死了?!”

备注:一更的情况将持续到7月24日,在25****会爆更,回馈一直支持我的朋友们。最近这段时间,内外交困,职业写手不好当,压力真的很大,有一段时间我都怀疑自己得了抑郁症。不过最近我稍稍调整过来了,好像是沙漠中快要渴死的人,喝到了一杯凉白开。现在,我开始平衡外界的物质所需,和内心精神追求的两端。

职业化是需要一个适应的过程的,我开始明确该用如何的态度对待这份职业,对待读者,对待自己,对待生活。

谢谢朋友们的默默支持和宽容理解。

有一句话。

当你想当上帝的时候,一定离成为魔鬼不远。

我不欢迎上帝和魔鬼,我欢迎朋友还有知己。

因为人人平等!

\end{this_body}

