\newsection{本体突破}    %第八百五十四节:本体突破

\begin{this_body}



%1
疯魔窟第八层远比方源料想中的,还要广阔无垠。

%2
自从得知这里有尊者道场之后,他便在这里探索了很久,但是结果并不美好。

%3
方源见到了许许多多的世界生灭,也进出过一些大的世界,却始终找寻不到他想要到达的尊者道场。

%4
不过,他也并非没有收获,这些天的探索让他对第八层有了一层更深刻的理解。

%5
“这片虚无中,毫无上下左右等等方向的概念,所以它既是有限的,也是无限的。”

%6
“本来第八层只是一片洞窟而已,但是经过无极魔尊的布置之后,就成了无垠之所。”

%7
“说不定这些天来,我也只是在绕小圈子而已。不参透这里的布置,我恐怕无法找寻到尊者道场。”

%8
方源猜测着。

%9
要在虚无中寻找到自己的目的地,无疑是让方源多多提升虚道的境界。

%10
虚道,这只是一个小流派,从律道中衍生出来。

%11
不过这个小流派却是相当的成功。

%12
虚道讲究虚虚实实,万法不沾,先天不败。

%13
虚化是虚道最招牌的手段,监天塔一旦虚化,除了梦道外几乎所有的手段都奈何不了它,极其安全。

%14
虚道也有至高奥义。就像魂道的至高奥义之一,是修成亿荒魂,比如魔尊幽魂的三尊六合千臂魔魂。变化道的两大至高奥义,一是千变万化,随心所以,二是万道归变,万物大同变。

%15
天庭的成员都会奉献自己的仙窍,每当他们离开天庭时,就会被临时赋予一个虚窍,用虚窍来装载仙元、仙蛊等等。

%16
这个虚窍,便是虚道的至高奥义。

%17
而当下,方源身处的这片虚无,无数世界在这里生灭。很显然,这片虚无也是虚道的至高奥义。

%18
虚道虽然只是律道的分支,要提升虚道境界,本质上仍旧是律道真意。

%19
但这个流派内涵深厚,单靠方源目前的律道大宗师境界,想要参透这片虚无,还是有一段距离的。

%20
既然正面暂时突破不了,方源思考了一番后,便决定另辟蹊径。

%21
他心念转动,开始催动煮运锅。从运道上下手,祝福自己,或许能有所突破。

%22
虽然煮运锅层次较低,难以彻底影响到本体的运势,但改变一部分还是可以的。

%23
至少煮运锅里的气运,方源是可以随心所以地改变的。

%24
有了煮运锅,改变自身运气非常的便捷,很快方源就做成了。

%25
他又顺便查看了一番分身的气运。

%26
他的宙道分身气运,还是老样子。一道小型的光阴长河般的气运,熠熠生辉,潺潺流淌。仔细分辨,河面上还有一缕缕的黑色坏运不断产生,缓缓积蓄。

%27
方源的梦道分身,气运规模最小,毕竟只是五转蛊师修为,空窍还被封印了起来。

%28
最成气象的,便是分身吴帅。吴帅已是七转巅峰的蛊仙,又获得了龙宫,目前精修军团蚁。所以,此时吴帅的气运便是一头蜿蜒游动的蛟龙,龙鳞、龙目闪闪,龙角、龙爪尖锐,很是英武神俊。在蛟龙气运周围,是一波波的淡蓝水晕,蛟龙遇水,这预示着他正得到外界的资助——古族和鲛人。而在蛟龙的口中,有一颗巨大的龙珠。龙珠中,有着一座宫殿,赫然是龙宫的缩影。

%29
又有一颗龙珠,体积小很多,被蛟龙的一只龙爪扣在手中。龙珠里面藏着密密麻麻的蚂蚁,代表着军团蚁。

%30
其次气运最大的,便是分身房睇长。

%31
他的气运比较特别,仿佛是一株大树,树干不细,但也不是极其粗壮的那一种。树冠苍翠,巧妙堆叠,形成豆神宫的形状。而在地面上,树根纠结,有许多只裸露在外——这暗含着房睇长和房家千丝万缕,牵扯不断的深刻联系。

%32
除了这些之外,便是战部渡、李小白这两具分身。

%33
战部渡的分身气运,要比李小白的大得多,形如雄鹰,展翅高飞。

%34
战部渡刚刚成就六转蛊仙不久,而李小白仍旧还是蛊师层次。

%35
李小白的运气,宛若花朵盛开,并且花蕊中积蓄着一层薄薄的花蜜琼浆。这层花蜜就是李小白这段时间来,努力成长,刻苦修行的成果。而在花朵气运之外,有一只蝴蝶,洁白如霜,扑扇大翅,掀起冰风,正向花朵徐徐飘来。

%36
“这个气运迹象……”方源看到这里,眉头微蹙。

%37
“蝴蝶往往预示女性,它要采摘花蜜,同时掀动寒风,对花朵无益。”

%38
“从规模上来看,蝴蝶体型之大,还要超过花朵一丝。掀起的寒风却是凛然浩荡,代表着某种大势,主因并非是蝴蝶,蝴蝶只是一个引子。”

%39
“分身李小白的处境,目前很好,但是接下来就不太妙了啊。万一有个意外,就是寒冰降临,花朵凋残。但蝴蝶和花亲近,这本身也预示着一层机遇。”

%40
分身李小白一直在攻略华文洞天,但这个洞天和兽灾洞天不同,因为有着八转蛊仙坐镇,方源并不能像对兽灾洞天那样施展自如。

%41
兽灾洞天里,方源是随心所欲地施展计谋,安排种种计划。但在华文洞天这里,他却是束手束脚,要想保存住最大的成果和利益,绝大多数的手段他都不能用。

%42
所以,气运上的支援就成了方源本体对李小白最主要的帮助了。

%43
当即,方源调动煮运锅,开始对李小白的气运进行改造。

%44
片刻后,李小白的花朵气运变得鲜艳红润,仿佛春日里的桃花——正是桃花运!

%45
改变气运并不是随意,而是要遵循一定的法则。

%46
就比如李小白的气运,转化成桃花运后,事半功倍,和外运蝴蝶形成巧妙互动。

%47
“接下来,就看分身那边能否把握得住了。”方源继续前行。

%48
气运只是一种变化的趋势而已,真正机遇来临,还得看本人的能力是否把握得住。

%49
“嗯?”方源眉毛微挑,他在改运不久后就有了可喜的发现。

%50
他看到一个漂浮着的大鱼,正在虚空中游动,晃动着巨硕的鱼尾,看似速度缓慢,实则惊人。

%51
这还是他第一次,看到能够遨游虚无,脱离世界存在的生命。

%52
“恐怕就是那些人形云雾曾经提到过的圣人门徒了。”方源追击上去。

%53
那头花白大鱼,酷似金鱼,见到方源之后,非常惊讶,连忙叩拜:“黄土圣人门徒小花,拜见圣人!”

%54
花白大鱼虽然体型堪比鲸鱼,但口吐人声,却是脆嫩的女童声音。

%55
“很好,很好。我找寻黄土道场已经很久,你带我去吧。好处少不了你的。”方源笑道。

%56
但花白大鱼却是摇头:“启禀圣人,非是我不愿意帮助您,而是我也无法找寻到黄土道场的位置呀。”

%57
“怎么?你是被放逐的?”方源皱眉。

%58
花白大鱼又摇头:“不是,不是。我是修行到了瓶颈,主动辞别同门,脱离了黄土大世界,来到外界行走,想要接受一些磨练,争取突破的。”

%59
说到这里,花白大鱼瞪着巨大的鱼眼,看着方源,眼中闪烁着期待之色。

%60
若是由这位圣人出手点化,它的瓶颈根本不算什么。

%61
但方源岂是那种人物,他不愿再废话,当即出手,将花白大鱼禁锢,随后恣意搜刮对方的记忆。

%62
“原来是这样。”

%63
“这些尊者道场虽然长存不灭,但却始终处于移动之中。”

%64
花白大鱼就诞生于黄土道场中,从它的记忆中,方源得以窥探出这片世界的丰茂。

%65
黄土道场广博浩瀚,比之至尊仙窍也丝毫不逊色。里面土道道痕浓郁,其余道痕健全,仙材库藏丰富,蛊虫众多,包含仙蛊。还有各色生灵,虽无人族,但确实是名副其实的宝地。

%66
即便方源此时的身家,也对这个黄土道场起了觊觎之心。

%67
“我说能得了黄土道场,完全吞下,总资产将翻三番!”

%68
方源大炼仙蛊,不断剧烈消耗,成果已经消化得七七八八,仙材库藏缩水得十分严重。

%69
若能得到黄土道场,必定是一个巨大的补充,相当于一口气又从瘦子吃成胖子。

%70
然而要到达黄土道场,却很是困难。

%71
黄土道场中的生灵各色各样,乐土仙尊当年布置下来,就是有教无类。

%72
生灵们都拥有魂魄,因为乐土仙尊留下了蛊师修行的方法,绝大多数的生灵都是拥有空窍,展开了蛊师的修行。其中的佼佼者们,更是修成了蛊仙,虽然数量不多。

%73
而在这蛊仙当中,也有精英。这些精英往往拥有脱离世界的能力,当它们的修行陷入瓶颈,再无寸进之时,它们就会主动脱离道场,前往各处磨砺自身。

%74
当它们身处虚空,远离道场后,它们也没有办法找寻到回家的路。

%75
不过不要紧,等到一定的时间,缘分到了,它们突破了瓶颈后,就会看到道场,从而顺利回归。

%76
这是方源从花白大鱼中所得的全部情报。

%77
别看这头花白大鱼被方源轻松擒获,但它却是拥有七转修为,乃是黄土道场中的蛊仙精英。

%78
方源眉头皱得更紧。

%79
他推算出了更多的情报。

%80
“缘分到了,就能自然而然地回归道场——这是黄土道场里流传的说法,并且花白大鱼的记忆中,也有不少前辈就是这样的回归经历。”

%81
“看来,黄土道场本身就拥有接应这些外出生灵的能力。每隔一段时间,黄土道场就将里面的蛊仙精英放出去锻炼,等到锻炼有了成果,道场就会主动接回来。”

%82
为什么要这么做呢?

%83
花白大鱼的记忆中,同样有着答案。

%84
这是因为每个道场之间,并不是和平共存的,都是相互征讨,四方乱战。

%85
道场栽培出的生灵会相互厮杀,大打出手,企图侵占彼此的道场。

%86
“当初尊者布置道场时,恐怕就是这个安排。”

%87
“但让它们相互争斗,有什么好处呢?尊者们是处于什么样的目的?”

%88
方源想到这里,微微摇头。

%89
他只有进入一个尊者道场中,恐怕才能找到这些问题的答案了。

%90
方源将目光投向花白大鱼。

%91
他微微一笑:“你不是想我指点你吗?碰到我是你的幸运。”

%92
花白大鱼浑身打冷颤,它结结巴巴说不出话,已是被方源吓倒。

%93
虽然和方源相处的时间并不长,但花白大鱼深知:眼前的圣人极其可怕,拥有残忍冷漠的魔性,绝非是乐土圣人那样的好人!

%94
自己遇到他,真的是不幸啊。

%95
“好了,就算你不说,我也知道你的疑惑。”方源搜刮了花白大鱼的所有记忆,对它的瓶颈是什么,一清二楚。

%96
他当即指点它,言简意赅,且又浅显易懂。

%97
花白大鱼渐渐不再颤抖,被方源的话吸引,深深沉浸,暂时忘掉了自己的危险处境。

%98
“怎么还未突破?”方源皱起眉头,半天功夫,他说得口干舌燥。

%99
花白大鱼仍旧懵懵懂懂。

%100
它虽有灵性,但完全和人族不能相比。

%101
方源耐下性子,继续调教。他帮助花白大鱼挣脱瓶颈,说不定就能引来黄土道场的接应。

\end{this_body}

