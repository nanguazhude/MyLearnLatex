\newsection{黄史追击}    %第四百一十三节:黄史追击

\begin{this_body}



%1
轰!

%2
一道巨大的突泉,猛地从原本平静的河面,喷涌而出。

%3
和这道突泉相比,飞行在半空中的黄史上人,仿佛是桌腿旁的一只苍蝇。

%4
突泉还未袭近,黄史上人就感到突泉的速度越来越快!

%5
这当然不是真相,事实上突泉爆发出来,直至达到顶端,速度会越来越慢。黄史上人之所以有这样的感觉,则是因为他本身的速度变慢了。

%6
这是光阴长河,突泉的喷发会导致周围的存在,时间变缓。如此一来,就会让涉险的对象,越加难以逃脱。

%7
“哼。”黄史上人临危不乱,关键时刻,他的光头上猛地闪过一瞬的亮光。

%8
下一刻,他全身恢复如常,猛地后退,让突泉冲袭落空。

%9
轰隆!

%10
又一声巨响,突泉没有击中黄史上人,落到了河面上,掀起巨大的浪涛。

%11
黄史上人擦了擦头上的冷汗。

%12
“这次的突泉,足有八转杀招的威能,即便是我也要十分小心。一旦被击中,必然受伤。”

%13
“而若是遇到九转级别的突泉,落个重伤是肯定的,死亡都有可能。”

%14
黄史上人心中很有压力。

%15
尽管他是八转蛊仙,在光阴长河中开战,对他而言,又具备地利。

%16
但是他清楚,方源掌握着运道手段,甚至可能有较为完整的众生运真传。

%17
在突泉河段这样的地方,其实依靠的除了实力之外,就是个人的运气。

%18
“我现在追杀方源,对他不利。按照运道的奥妙,定然会让我这一路追击不顺,处处坎坷。”

%19
“方源选择这样的路径来为难我,也是好谋算。”

%20
“不过……”黄史上人的眉头皱起来,心中疑惑,“方源就算运气再好,也绝好不过处处规避了突泉吧?就算是九转级别的突泉稀少偶有,他遭遇不上,那些七转级别的突泉,却是数量众多,八转也是如此。他是如何渡过去的?难道是一路依靠着逆流护身印?”

%21
若是依靠逆流护身印,这对黄史上人而言,反而是一个大好消息。

%22
因为这突泉河段路程很长,方源一路依靠这个杀招横渡,必然会造成仙元的剧烈消耗。这对于接下来,黄史上人和方源开战,自然是有利的。

%23
黄史上人没有猜到,方源会奴役了一头太古年猴,当做挡箭牌。

%24
尽管方源能够操控上极天鹰,但天庭把这方面的原因,归咎于黑凡真传。方源以七转修为,控制八转存在,这在奴道蛊仙而言,也是非常惊人的成就。

%25
但真相却是百八十奴仙道杀招,以及似水流年仙蛊的弊端。

%26
尤其是后者,黑凡怎么也不会自暴其短。这个秘密,天庭不知,也就难以推算出方源居然有着勾引太古年兽的手段。

%27
如此种种,决定了天庭方面并不知晓方源掌握着太古年兽这一情报。

%28
半盏茶的功夫,黄史上人彻底飞出了突泉河段。

%29
紫薇仙子的指引,又传达过来。

%30
黄史上人转折方向,速度猛地暴涨,再次向方源一行人追去。

%31
片刻后,他的身形猛地停滞下来,惊疑不定地看向前方。

%32
“阴织蛛!”黄史上人面色微变,低呼出声。

%33
在他面前,一头庞巨的蜘蛛,通体苍白,爬在一张巨大的网上,一动不动。

%34
蜘蛛的体型,比之突泉更要巨大,宛若一座山脉,肢体间似乎丘峦起伏。

%35
和蜘蛛相比,黄史上人越发显得渺小,宛若大象面前的蚊子。

%36
比阴织蛛更大的,是它趴着的那张巨网。

%37
这个巨网宛若水晶质地,有点纤细如发,有的粗壮如树。整个巨网并非一张平面,而是结成了一片森林似的牢笼。

%38
数以万千的蛛丝,宛若触脚或者树根,深深地插入到光阴河水当中。任凭河流如何冲刷,这些水晶蛛丝却是岿然不动。

%39
海量的蛛丝相互衍生、链接,形成一个庞大至极的水晶网巢。一个个的网眼,或大或小,交汇出无数繁复的路线。

%40
黄史上人心中陡然一沉。

%41
他有些不敢相信,难道方源等人就是穿越了这里,去往红莲真传的?

%42
他确认了一下,紫薇仙子指点他的方向,的确是这里没错。

%43
这让他越发肯定,这条路线是方源等人设计好的!

%44
“方源是想要借助这片天险,让我等追兵知难而退吗?”黄史上人面色冷峻,轻哼一声后,他就飞入这巨大的水晶网巢之中。

%45
阴织蛛每一只都是太古荒兽,横霸在光阴长河的一片领域之中,在它的地盘,即便是太古年兽群,都不敢横冲直撞。

%46
它是最顶端的猎食者之一。

%47
阴织蛛依仗的,除了它庞巨如山的体型,难以想象的雄浑怪力,以及锋锐至极的口器之外,就是这片水晶网巢。

%48
这网巢的每一根水晶蛛丝都价值非凡,蛊仙先贤给它取了一个相当浪漫的名称,并且一直沿用至今——

%49
岁月静好丝。

%50
但凡七转或者以下的存在,碰触到这个蛛丝,就会陷入到时间静止的状态,一动不动,任凭行动缓慢的阴织蛛前来,将其慢慢捕食。

%51
而例如黄史上人这等八转存在,岁月静好丝虽然一根难以奏效,但来个百八十根,也要让黄史上人陷入险境。

%52
而整片水晶网巢,何以百八十根的岁月静好丝?绝对是数以亿计!

%53
黄史上人速度很慢,很慢。

%54
他小心翼翼地穿越一个个水晶蛛丝,交织而成的网眼。

%55
他不是没有想过,绕过这片险地,继续追击方源。

%56
但这样一来,他多绕了一段远路,势必会让方源争取到更多的时间。

%57
并且,万一这红莲真传就在这水晶网巢之中,又当如何呢?

%58
所以,黄史上人决心踏足险地,直接追击。

%59
“只要我小心翼翼,不碰触任何一道岁月静好丝,安然度过这里,不是问题。”黄史上人正想着,忽然轰的一声,爆发出天崩地裂似的巨响。

%60
水晶网巢随着巨大的声浪,震荡起来。

%61
阴织蛛被惊动,一百八十个复眼,在一瞬间猛地盯住了黄史上人!

%62
“该死!”

%63
“定是方源等人布置的陷阱!”

%64
黄史上人怒气顿生,惊诧之感倒是少得很。若是他自己当做方源,也会布置陷阱,坑害追杀之人。

%65
“我既然能够涉足这里,自然有着把握,可以在惊动了阴织蛛的情况下,穿越这片险地!”

%66
一声冷哼,黄史上人开始发动仙道杀招,轰击水晶蛛丝。

%67
这些水晶蛛丝并不容易被破坏,尽管它本身质地很脆。但是任何的杀招,只要近身,都会被时间静止,给人悬停下来的错觉。

%68
当然,对于那些八转级数的杀招,岁月静好丝还是有着自身承受的能力极限。

%69
黄史上人放开手脚,催动八转杀招,不断轰破面前的水晶蛛丝,速度反而暴涨上去。

%70
他可是宙道大能,对于岁月静好丝更有针对的手段。

%71
而那头阴织蛛,却是速度缓慢,一时间竟然追不上黄史上人。

%72
黄史上人不愿和阴织蛛死磕,他还有重任在肩。

%73
不过他的撤退路上,绝非一帆风顺,时常有仙道杀招爆发出来,或是音浪,或是冰霜,或者焰火,阻挠他的前行。

%74
黄史上人因此速度受到很大的限制,最终让阴织蛛赶上,双方爆发激战。

%75
轰轰轰……

%76
激烈的战斗持续了好一会儿,黄史上人摆脱了阴织蛛,撤离出了水晶网巢的范围。

%77
不过这时,他可比之前要狼狈得多,灰头土脸不说,身上更是负了轻伤。

%78
“这该死的方源,究竟埋伏下了多少陷阱!而且这些陷阱每一次爆发出来,都恰到好处,触发之中大有玄机!”

%79
黄史上人吃了这么一个闷亏,澎湃的怒意几乎充斥心胸,他更加不会放过方源。

%80
他一边疾飞,一边治疗伤势。

%81
不久之后,他进入了一片古怪的河段。

%82
“怎么回事?这片河段中,我居然感受到了强烈的剑道、刀道的气息?”黄史上人脸色的惊诧之色,一闪即逝。

%83
他速度缓慢下来,细细观察,只见这里的光阴浪潮中,更是透射着漫漫的剑气,重重的刀光。

%84
“等一等,难道说,这里是……”

%85
黄史上人皱起眉头。

%86
“我此行逆流而上,去往过去,算算路程,转为时间,应当是在过去的十万年前,近古时代!”

%87
“近古时代的剑道、刀道……”

%88
“有了!是西漠的习渊,此人专修剑道,自创剑渊,屠戮过三位八转蛊仙。还有一位刀道魔仙,人称刀九郎,魔威滔滔,纵横西漠,闯破了逐客阵,轰动天下。这两人有过一场大战,打得天地晦暗,鬼神惧惊。”

%89
黄史上人再用手段查探,果然如他猜测的一样,这片光阴河段正是被这两位八转大能的激战所影响,导致这里形成了一片刀剑河段。

%90
要闯荡这里,比之突泉河段更加危险,因为很可能,刀九郎或者习渊的仙道杀招,就会忽然重现,斩向自己。

%91
突泉好歹爆发的时候,有个过程。这里的河段却是没有。

%92
“方源等人居然跑到这里去了?不想活了么!”黄史上人咬牙切齿着,也紧接着进入其中。

%93
方源等人既然能够进去,他堂堂八转大能,更不能退缩了。

%94
紫薇仙子不断指点他,黄史上人距离目标越来越近。

%95
吼!

%96
太古年猴在一片惊涛骇浪中,不断搏杀,已然是伤痕累累,鲜血横流,染红周围。

%97
太古年猴身上,是白凝冰、妙音仙子等人,据险而守。

%98
“终于发现你们了!”黄史上人双眼精芒绽射,想都未想,直接扑上!

\end{this_body}

