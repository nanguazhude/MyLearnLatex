\newsection{畸形繁荣}    %第四百四十二节:畸形繁荣

\begin{this_body}

至尊仙窍中,仍旧充斥着安宁和静谧。<strong>最新章节全文阅读www.Mianhuatang.cc</strong>

因为没有气象的显著变化,甚至没有白天黑夜,所以平静,是这里的主题。

方源是七转修为,至尊仙窍只能算是福地一级,而福地向来是没有天象变化的。唯有到达晋升洞天的边缘,或者成就了洞天,仙窍才会有天象的变化。

没有天象变化,但至尊仙窍中却并不死寂,而是有着灵动和生机。

小北原较为寒冷,尤其是北端仍是一片雪地,空无一物。吞并的韩东福地,如今成为豢养灵蛇的资源点,由着粉红色的灵蛇地灵来管理。除此之外,还有骨葬场,收容了大量的骨头仙材。而在南端,方源正打算铺建一片血红草原,以血镰草、赤斧花为基础,安置那些花粉兔、狐狸群、狼群等等。

小中洲里,有血芝林、镜柳林,都是方源无意积累下来,没有刻意栽培,任由发展,所以规模都不大。还有一个璇光小坑,承载一些光道的凡蛊。

小西漠中,则重点有幽火龙蟒地坑、沙鸥土滩、腐黑沼泽、烂泥沼泽、黑犬丘陵(曾经的刘勇福地,有地灵黑色卷毛犬)。另外最近新添了一座凡蛊屋,就是方源从西漠的易位沙漠中,得到的凡蛊屋尸地城。里面的仙材大多数都交易给了千变老祖意志,所剩很少。

小东海中,再不是之前的浅水,而是真的有海洋的一丝气象。尤其是两大海域――龙鳞海域、武遗海域,前者是方源在南疆时耗费巨大财力、精力,特意搭建起来的大型资源点,后者是吞并了武遗海的福地,有海龟地灵为方源管理,里面生活着上古荒兽角神龟三头,一片上古荒植静音珊瑚,还有六头荒兽白信蓝羽鸥。

除了两大海域之外,还有一片血湖,规模不大,是方源曾经修复血本仙蛊而设,一直保留下来,血湖中的湖水,掺杂了大量的荒兽、蛊仙的血液。

血湖之下,就是众多浅海,都是被方源精细安排,区分开来,豢养了气泡鱼、散文鲤、青鱼鲫鱼、血玉鲫鱼等等,更有一头上古荒兽藏娇蚌,是方源曾经从东海夺来的。

天地秘境之一的逆流河,也暂时被方源搁置在这里。

小南疆是所有当中,建设程度最高的地域了。

这里有五光山、继仙山、成龙丘、封天山以及无名山峦十多座,还有盘丝洞窟、石钟乳洞窟。\&\#65288;\&\#26825;\&\#33457;\&\#31958;\&\#23567;\&\#35828;\&\#32593;\&\#32;\&\#87;\&\#119;\&\#119;\&\#46;\&\#77;\&\#105;\&\#97;\&\#110;\&\#72;\&\#117;\&\#97;\&\#84;\&\#97;\&\#110;\&\#103;\&\#46;\&\#67;\&\#9山林和地底生活着少量的石人,大量的原始丛林中,有曲丽木、茶溪、气死鸟等等,形成严谨自洽的生态,生机勃勃。

而在小九天中,小赤天里安放了大量的年兽,还有那头太古年猴。

小橙天中,在最近被方源重点照顾,有着光照菌,使得极光规模不断扩张,极光中孕育的流光果也越来越多。

小黄天中非常辽阔,只有一小段的碎金河。

小绿天里直接是空无一物,不过曾经被方源盛放过梦境。

小青天里有丹青香,这是换魂仙蛊的食物,很久之前就被方源囤积进来一批。另外有一个天晶蓄养池,位于小青天的最中央地带,可惜无用。

小蓝天是九天当中,发展程度最高的。里面有大量的云土,上面栽种了许多星道资源,比如箭竹林、星屑草场、陨石群坑,还有一片悬浮着的星屑大陆,这是耶律群星的福地,被方源吞并过来,由星核地灵管理。

哦,还有一头落星犬幼体,在这里吃吃喝喝睡睡。它形单影只,很是孤独,不过要把它放到其他地方,难免会和其他荒兽、上古荒兽起冲突。方源将其收容进来,就很少关照它了。

小紫天里,鹰兽成群。这些鹰兽有荒兽,也有上古荒兽,方源在梦境一役中,不惜贩卖了仙蛊,强行收购。它们曾经帮助方源,奴役住上极天鹰,帮助方源渡过生死关卡,公开不可没。方源曾经一度想将它们贩卖,但卖出去肯定不会收到仙蛊,等若是亏本甩卖。正好从北原黑家大本营中,还抢夺来了八十多座鹰巢,虽然是没有天晶鹰巢那么高端,但却正好合适这些鹰兽栖息繁衍。

剩下的两天,小白天中有一只鹰犬群,三只上古天残犬,还有斑斓霸王花的大规模栽种的花场。小黑天里,因为毫无光线,环境特殊,如今只有那几株上古荒植走肉树了。

估算一下,这些资产的价值,已经能够让方源和一些八转蛊仙媲美了。

但是对于整个仙窍而言,这些资源还是显得很少的。

为什么?

因为整个至尊仙窍,实在是太辽阔了。超过五亿亩的辽阔空间,这些资源只占据了仙窍很少的一部分。

从整个至尊仙窍出发,如果将整个仙窍空间都利用起来,算是一百的话,那么方源现在,不过是三点左右。

百分之三的开发程度。

这当然不是一件坏事,而是大大的好事。

至尊仙窍有着无以伦比的开发前景,但从另外一方面,什么时候能有百分百的开发程度,方源心中也没有把握。这绝对是一个非常艰巨,并且耗时极多的超级工程!

古往今来,蛊仙们在修行中,已经总结出了一套评价仙窍发展的标准。

它详细地分为七个层次。

第一层,是建设凡级的资源。产出凡蛊、凡级蛊材,普通的兽植。

这些资源,往往是论规模的。凡人养凡蛊,通常都是五六只罢了。蛊仙养凡蛊,是上百万,上千万只的放养。至于方源前段时间贩卖的年蛊,数量上早已经破亿。

第二层,是建设仙级资源,产出仙材。这一层次成功的标准,就是能让蛊仙自给自足地喂养自家仙蛊,不让它饿死。

第三层,是搭建一套生态,豢养仙兽、仙植。

第四层是,利用多余的仙材、仙兽、仙植,对外贸易,互通有无,或者赚取利润。

……

从这个标准来看,方源处于第四层上,因为他有不少的生意,都做的很好,赚取大量利润。甚至前段时间的年蛊生意,方源凭借极强的实力,冲击了整个宝黄天的市场,那失败的三大巨头,目前还在角落里画圈圈呢。

“但从另一个角度而言,我还在第二层,因为我的仙蛊喂养,还未真正达到自给自足的程度。”

这点,方源一直都有自知之明。

没办法,方源手中的仙蛊实在太多了。

八转级别有三大蛊虫:态度、慧剑、似水流年。

七转有剑眉、浪剑、飞剑、剑遁、年蛊、以后、卜卦龟背、龙息、龙鳞、防备、大、斗、爱意、自爱、坚持、名缰、换魂、血脉、龙力、阵旗等。

六转有我力、拔山、挽澜、力气、春秋蝉、江山如故、人如故、日、解谜、金刚念、****运、气运、妇人心、血本、变形、星眸、道可道、黄沙、净魂、察运、连运、时运、星念等。

其中察运等后四者,曾经被方源掌握,义天山大战后,因为换魂的缘故,被影无邪得走。方源借助了特意,摧毁了一批,但影无邪及时出手,也保留下来一批。

这些当中,还有紫山真君遗赠的一些智道仙蛊,曾经组建惊鸿乱斗台的一些宙道仙蛊,不做详述。

算一下数目,已然超过五十!

骇人听闻的数字!!

要知道绝大多数的蛊仙,都在苦求仙蛊。

蛊仙界中,数量最多的自然是六转蛊仙,但他们绝大多数都没有一只仙蛊!

七转、八转蛊仙手中,通常会有两三只仙蛊,多的能有五六只的样子。

但有一个规律――六转蛊仙苦求六转仙蛊,七转蛊仙苦求七转仙蛊,八转蛊仙苦求八转仙蛊,九转尊者同样追求一只九转仙蛊。

方源呢?

过百的仙蛊!!!

六转、七转、八转的都有。

而且这还不算借给楚度的招灾,放在琅琊福地中的智慧蛊,被天庭夺走的暗渡仙蛊。还有阵灵,毕竟他也是仙蛊所化。

方源的情况,实在是太特殊了。

天外之魔,重生之人,红莲魔尊的期待,天意曾经的棋子,影宗如今的宗主,继承了无数真传。

虽然过程中,磕磕碰碰,艰险处处,时时绝境,艰难逢生,但不得不说,巨大的风险带来庞大的收益。

方源现在整个是畸形繁荣。

因为重重机缘,导致他有了第四层的经济支柱,但却还达不到第二层的成功标准。

喂养仙蛊,一直对他而言,是一个极其巨大的负担和难题。

“不过好在,我努力至今,终于见到了解决这个难题的希望了。”

“九转智慧蛊有琅琊地灵充当冤大头,最困难的三只八转仙蛊的喂养,已经基本上解决了。”

“接下来,就是全面基础建设,了解六转、七转仙蛊的喂养难题,做到自给自足的程度。”

这样一算,方源手中的仙元石还嫌不够!

“和雪民联姻么?”

方源沉思。

雪儿带来了礼物雪莲花精的同时,也带来了雪民一族联姻的意思。

这些异人倒是非常的干脆和直接。

“如果我自己努力,解决喂养难题,耗费时间会很久。但是借助雪民一族的话,进度就会大大提升上去了。”

“我现在最缺的,不就是时间么。”

“和雪民联姻,也不是不可以。”

Ps:为写出这一章,耗费了我大量的时间整理资料。好在总算是完成了。现在,读者朋友们应该能够对方源的资产,有一个全面而且清晰的认知了。只不过,一个人精力有限,还请大家帮忙看看有没有欠缺的,没有算上的。(未完待续。)

\end{this_body}

