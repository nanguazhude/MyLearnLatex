\newsection{方源承认失败}    %第三百零九节:方源承认失败

\begin{this_body}



%1
池家驻地,书房中。

%2
池溜得到召唤,听得池伤的意思之后,神情为难:“大人,这样好吗?我们池家和武家关系……嗯,还是不错的。至少武家如此情境,我们池家都一直恪守中立,没有出手。”

%3
“那有怎样?别的家族都能出手,为什么我们池家就不可以?”池伤有点恼怒起来。

%4
池溜暗自咬牙,心想:“这池伤大人还真是和传闻中一样呢。这下怎么办?”

%5
一边想着,池溜一边道:“大人的心情我很了解,但是咱们这边,呵呵,尤其是咱们有些生意,一直都是和武家亲密合作的。冒然翻脸的话……”

%6
池伤顿时恍然大悟。

%7
“我说呢。”

%8
“你是指的梦境生意吧?”

%9
池溜点头:“大人英明。”

%10
武家想要做梦境生意,套取灰色收入,怎可能迈过池家?

%11
这套超级蛊阵中的仙蛊,都是南疆超级家族分摊,但真个蛊阵布置却是由池家的太上大长老池曲由亲自动手。

%12
池家虽然贡献的仙蛊较少,但是对于这块超级蛊阵而言,却是有着巨大的影响力。

%13
因此,武家做这梦境生意,必定要把池家捎上。池家的蛊仙对于蛊阵了解更多,如果不打动他们,这项生意是做不下去的。

%14
池伤心中叹了一口气,难怪太上大长老曾说过,人心难测。唉,为了一己私利,池溜是我池家的蛊仙,居然不站在我这一边。

%15
他面色阴沉,有些下不来台。

%16
他想和方源公然翻脸,但是第一步就遇到了挫折。怎么办?

%17
一时间,池伤也很无奈。

%18
因为他不过是空降于此,在池家一直都是苦修为主,很少接触池家蛊仙,一心钻研阵道,在池家的影响力相当的小,可以说是七转蛊仙当中最小的一位。

%19
现在他在这片超级蛊阵中,更是初来乍到,现在这个弊端显现出来,他堂堂一位七转蛊仙,居然调动不了一位池溜。

%20
池溜的心情也很糟糕,忤逆一位七转蛊仙,尤其是家族中着重栽培的红人,他自然心情不好。

%21
不过下位者自然也有下位者的生存方式。

%22
池溜想了想,开口道:“大人,不是在下不愿意替您办这个事情。只是您身份贵重,一旦捅破这件事情,造成的影响会很大。这样吧,请先让在下将这件事情,禀告给池规大人。他是这里的首领,我们听他定夺,您看如何?”

%23
池溜语调谦卑而又客气,更是暗捧了池伤一把。

%24
池伤听在耳中,舒服极了,当即点头:“也罢,那就看看池规大人如何说的。”

%25
池规听得池溜的汇报,眉头也皱起来。

%26
这种争风吃醋的事情,闹到谁的头上来,谁都不好受。

%27
池规感到很为难。

%28
一方面是自己人池伤,不看僧面看佛面,池伤的背后可是池家的太上大长老。

%29
另一方面这件事情若真的捅破了,就不是个人的私事,而是关乎政治的大事,能影响到武家、池家两家的外交。

%30
这个世界,是一个强者为尊,个人战力凌驾于组织之上的世界。

%31
因此政治,也是个人政治。

%32
池伤、武遗海的身份,都很高层,他们之间发生了矛盾,就是武家和池家的矛盾。

%33
这点毫不夸张。

%34
“尤其是武家,正风雨如晦,受到各方刁难。如果此时我池家再出手,有损我池家的中立名誉,会摊上落井下石的恶名不说,还会恶了和武家的关系,损害我家族的利益。”

%35
“但是池伤这边,我又不能不安抚啊……”

%36
池规苦思冥想了一番,真给他想出了一个办法。

%37
于是他私底下给池伤出主意:“咱们这事儿不能闹大,闹大了不好,影响会很恶劣,就算咱们家赢了,也会叫人看笑话。这毕竟是私事,建议按照私事处理。你可以将这些情况,告知丝柳仙子。武遗海也对乔丝柳有意,你请丝柳仙子坐证,他武遗海绝不会在耍无赖了。”

%38
池伤听了这话,一拍大腿,高叫一声:“妙啊!”

%39
池规微笑,暗自也有些得意。

%40
对于池规而言,此事能完美处理,对于池伤而言,更有着巨大的吸引力谁不想在心爱的女子面前,将情敌狠狠地踩下去,大大地露一把脸?

%41
此事就这样处理,很快池伤就将这里的情况,通过信道蛊虫,发给了乔丝柳。

%42
不管方源那边如何,池伤却立即感觉到了一股庞大的压力。

%43
“现在丝柳仙子已经知道了。我说这是君子切磋,又是他武遗海主动挑事,应该不会让仙子反感吧?”

%44
“我得赶紧完成武遗海的这个挑战!”

%45
“这是我最擅长的地方,绝不能失败。”

%46
巨大的压力,赋予池伤巨大的动力。

%47
推算眼下的这个难题。

%48
“这是另一部分吗?”

%49
“和之前的那个蛊阵部分,有些微的关联呢。”

%50
“恐怕是整个蛊阵的两部分了。可恶,如果我知道整个蛊阵的话,那就容易多了。现在却只能按照这个局部来推演。”

%51
伤脑筋!

%52
池伤开始动用手段。

%53
他是阵道蛊仙,掌握了太上大长老亲自教导他的一座奇妙蛊阵。

%54
这座蛊阵以智道蛊虫为主,辅助的蛊虫却是一只仙蛊。

%55
一般而言,不管是仙道杀招还是仙道蛊阵,都是以仙蛊为核心,凡蛊为辅助。但是这座蛊阵却是反其道而行之。

%56
用来辅助的仙蛊,是一只水道仙蛊,名为汗水。

%57
蛊阵名为智汗阵,在池伤的仙元灌注下,徐徐催动起来。

%58
没有绚烂的光影,而是形成了一股介乎香臭之间的奇妙气味。

%59
气味越来越浓郁,将池伤重重包裹起来,渐渐的,气味凝聚成烟雾,将池伤的面目笼罩,看不分明。

%60
连续三天三夜,池伤不眠不休,不间断地推演。

%61
终于,烟雾轰然消散,蛊阵停息下来,一些凡蛊消耗死亡,只剩下一成左右,被池伤重新收入仙窍当中,其中便包括汗水仙蛊。

%62
池伤浑身大汗,像是淋了一场暴雨,面色都显现苍白。

%63
不过他的双眼,亮得吓人。

%64
蓦地,他仰头大笑:“哈哈哈。终于解决了,这一次,我要看你怎么说,武遗海!”

%65
片刻之后,方源便接到了回信。

%66
“哦,学乖了,也利用乔丝柳向我发难么?”

%67
“可惜,我志不在此,这点威胁有和用处。”

%68
方源很快又拍节叫好起来。

%69
“有意思,有意思,这样子解决,果然是个好方法!”

%70
“不过,为什么我就没有想到呢?”

%71
方源虽然得到了解答,但是却没有流连于此,而是进行反思。

%72
池伤的这个解答里面,有许多地方,是方源没有掌握的。阵道的修行内容,最为广博,方源得到的传承也只是凡人的东西,不知晓也很正常。

%73
至于池家,却绝不缺少阵道仙级传承。并且经过世世代代的发展,还能在原有的雄厚基础上推陈出新,乃是南疆阵道第一的势力。

%74
“不过,我即便没有仙级传承,若是境界达到宗师,也应当能够解决这个问题。”

%75
境界是对大道本质的理解。

%76
打个易于理解的比方的话,就好像是解决一道数学题。

%77
方源要改良蛊阵,达到自己的目标,就像是要得到一个数字,比如5吧。

%78
尽管他手中有蛊虫,但他不知道,如何才能得到5。这个时候,池伤的回答就是,2 3=5。

%79
方源恍然大悟,原来是可以如此做,达到自己的这个目的。

%80
如果境界足够,就算不知道,也能凭感觉,触类旁通,举一反三,得到2 3=5,或者说10/2得5这样的答案。

%81
当然,蛊阵绝对复杂多了,尤其是蛊虫掺和越多,就越难布置成功。

%82
蛊阵的内容相当繁杂。有单纯蛊阵、复合蛊阵之分,还有阵中阵,依照型状还分方阵、圆阵等,按照催动的类型,有即时阵,也有延迟阵。按照流派划分就更多,有炎阵、水阵等。有的蛊阵如同宝塔,一层叠加一层。有些蛊阵很容易改良,就像是一件衣服,改良的地方就仿佛打补丁。有些蛊阵则不行,宛若积木,变化其中一块,就会引起整个蛊阵的崩塌。细分下去,涉及蛊虫就更加复杂。比如有些蛊虫要相互统一,有些蛊虫则要分散开来,各自影响各自的对象。

%83
到了第二天的晚上,翘首以盼的池伤得到了方源的再次回应。

%84
“这一次,有丝柳仙子作证,我倒要看看你怎么抵赖?”池伤迫不及待地将心神透入手中的信道凡蛊当中。

%85
很快,他就眉开眼笑起来。

%86
因为方源在信中承认了他的实力,不愧当初丝柳仙子的称赞。不过,这样的程度,他武家也有人可以做得到,并不为奇。

%87
池伤看到这里,皱起了眉头,不过他也无法反驳什么,因为武家蛊仙的确有能解决难题的人在。

%88
这时,他看到最后末尾处。

%89
方源又出了一个难题。并且承诺,如果池伤能够解决这个,那么方源就承认池伤强过自己,哪怕在丝柳仙子面前,也要自认不如。

%90
池伤顿时鼻息粗重起来。

%91
光是想一想,他自己和武遗海同时出现在乔丝柳的面前,这个时候武遗海抱拳说:我不如你。

%92
一个情敌在仙子面前,承认我不如你。

%93
这个样子的情景,池伤越想就越带感。

%94
“武遗海身份高贵,是我最重要的对手。”

%95
“我如果能赢下此局,就能击败他。”

%96
“更关键的是,有丝柳仙子坐证,我根本不惧他武遗海反悔啊!”

%97
ps:今明两天都是一更。大家不喜欢这个情节可以跳过,也可以选择不看本书。只是每一个情节,都是我精心设计,为之后**铺垫。跳过的同学会有一点阅读上的小小干扰,不看的同学请走不送,当然也欢迎再来。

%98
说一句心底话,作者是人不是神,是人都会有状态不佳的低谷期,坚持不下去的同学请慢走。但我肯定会坚持下去,我从未太监过任何一个作品。还是那句话,哪怕就只有一个人看,我也会坚持到底。与成绩、金钱无关。这只是我写书的坚持而已。

\end{this_body}

