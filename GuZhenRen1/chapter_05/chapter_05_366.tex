\newsection{武庸杀至!}    %第三百六十六节:武庸杀至!

\begin{this_body}



%1
狂风扑面而来。

%2
脚底下方,莽莽群山,郁郁葱葱,宛若无边的象群,不断地朝后方奔腾而去。

%3
上极天鹰的鹰背上,方源等一干蛊仙,却没有丝毫心情来观赏这壮美的河山。

%4
他们此时的情势非常危急。

%5
整个南疆正道蛊仙,都在试图围剿他们。或许还有其他的散修、魔道蛊仙,正潜伏着,看看有没有便宜可捡。

%6
南疆正道已经是无法匹敌,任何一位八转蛊仙,或者仙蛊屋,都能给方源等人带来巨大的麻烦和凶险。

%7
然而除此之外,更让众仙担心的是——天庭。

%8
天庭卷走了超级蛊阵之后,就没有留下。

%9
虽然方源等人并不清楚,这里面有龙公要镇压幽魂本体的原因。但天庭虽然撤走,但影响力却仍旧存在,对方源等人的剿杀之心,始终都没有消退,并且更有愈演愈烈之势。

%10
方源的判断,深入人心,得到黑楼兰、白凝冰等人的一致认可。

%11
换位思考,若她们是天庭一方,也绝不会放过方源这群人。除了将情报泄露给南疆正道蛊仙,利用他们之外,为了确保万无一失,天庭必定还会留下暗手。

%12
只是不知道,这个暗手会是什么。

%13
未知更让人心中压力丛生。

%14
天庭的暗手,就仿佛是悬在众人头顶的刀刃,不知道什么时候会忽然落下。

%15
就算是方源等人逃到了西漠中去,这个天庭暗手恐怕也不会消失,极可能会尾随其后。

%16
“天意居然能侵蚀外显的梦境。”

%17
“我探索梦境,简直是自暴机密。枉我还不断利用暗渡和见面曾相识伪装自己,自以为一直蒙蔽住了天意。”

%18
“这是我梦境大战中,最失败的地方。”

%19
方源站在鹰背上,逆流护身印已经停下,这杀招维持的成本太高,现在方源催动着各种侦查手段,其中就有三息后现。

%20
他一边侦查周围,一边反省自己,寻找自身的差错和过失。

%21
人非圣贤,孰能无过?

%22
犯错是很正常的事情,九转尊者的一生都是错误不断。局限于眼界,受困于性情,被环境拖累等等原因,没有人能够在自己的一生中,做出永远正确的选择。

%23
犯错之后,认识错误,并加以改正,才是人杰所为!

%24
若是方源提前知晓这一点,他绝不会这么随意就在梦境中探索。

%25
方源现在回想起来,发现很多天意的布局。

%26
比如他在超级梦境中的获益,水道、阵道、暗道境界,分别达到了宗师级。

%27
但方源最需要的,其实是信道、宙道的境界提升。信道境界一提升,方源就算没有信道的仙蛊,也能自己逐步解决麻烦。宙道境界一直都是普通,若是提升上去,配合黑凡真传,必定能令方源战力暴涨。

%28
就算不是信道、宙道,剑道、炼道、运道、魂道这四大流派,若是能将境界提升到宗师级数,也是相当棒的。

%29
尤其是剑道。

%30
方源现在有好几只剑道仙蛊,更有薄青的真传。若是剑道宗师,战力必定疯狂飙升。

%31
可惜都没有。

%32
“甚至在探索梦境的后期,我守在超级蛊阵中,却只能碰见一些荒诞梦境。”

%33
“天意侵蚀梦境,梦境流转,恐怕这是天意不想让我再次提升。”

%34
“前世五百年的记忆,居然没有梦境会被天意侵蚀这一条?”

%35
或许这是方源前世五百年,还没被人发现的秘密,更可能的是,这是在方源利用春秋蝉重生的时候,自身意志被天意动了手脚,令他独独忘却了这一点。

%36
“我的暗渡仙蛊,也丢失了。”

%37
“现在我等直接暴露在天意之下。好在上极天鹰速度迅猛,留给天意布局的时间并不多。”

%38
南疆正道、天庭、天意,这三方围追堵截,都想要害掉方源的性命。

%39
方源临危不乱。

%40
他仔细筹谋,如何在这种情况下保住性命。

%41
事实上,自从八十八角真阳楼倒塌之后,方源就一直在琢磨此事。也有了充分的心理准备。

%42
此事暴露身份,远比之前暴露要好得多。因为方源手中,能够打出去的牌面,同样不少。

%43
就看他如何运用了。

%44
此刻,他动用智道手段,脑海中的念头此起彼伏。不断地设想某种情况,然后推演,如何解决危难,如何在消耗最小的情况下,保住自身。

%45
“东南方向,正有一座仙蛊屋疾驰而来!”忽然,白凝冰陡然开口道。

%46
她继承了白相真传,远距离的侦查手段,远比方源如今掌握的三息后现,要更加适用此时的情况。

%47
随后,妙音仙子也是脸色骤变:“我也发现,这是侯家的仙蛊屋飞沙阁!”

%48
她对南疆的情况,比较了解,一下子就辨认出来。

%49
然后妙音仙子又惊叹一声道:“来得好快!”

%50
飞沙阁乃是南疆蛊仙界中,仙蛊屋里移速最快的几座之一,比上极天鹰还要迅速。

%51
若是方源等人只利用上极天鹰赶路,迟早要被飞沙阁追上来。

%52
不过好在方源未雨绸缪,早已提前料到这般情形,提前与众仙演练了上古战阵四通八达。虽然被池曲由布置的蛊阵干扰,但也只是一点波折。此时众仙重新集结,自然可以再用这个手段。

%53
飞沙阁在群山中奔驰。

%54
这是它独特的手段,可以借助土道道痕,减少仙元消耗,更能增加速度。

%55
毫无疑问,在这群山林立,土道道痕浓郁的南疆当中,飞沙阁如鱼得水。

%56
在疾驰当中,飞沙阁仿佛泥沙浇筑,内里黄褐的泥沙不断蠕动,外表则是烟尘滚滚。里面的蛊仙,并不止侯家一族,还有一位八转蛊仙商无界!

%57
商家乃是南疆正道当中,名列前茅的超级势力,自然有着一位八转蛊仙。

%58
商家的太上大长老商无界,一直秉持着商家中立的传统,和各家的关系都不错。

%59
若是换做姚家、罗家,甚至铁家,侯家蛊仙绝不会给这个脸面,邀请他们进入飞沙阁。

%60
“嗯?”追击的过程当中,商无界忽然神色微动,在他的感知中,方源等人忽然消失无踪。

%61
“是用了那个上古战阵四通八达吗?”商无界心中一动,旋即脑海中浮现出了一副南疆地貌图。

%62
结合种种情报,商无界很快就推断出方源等人最有可能撤离的几个方向。

%63
而在这些方向上,都分别有南疆正道的蛊仙。

%64
“武庸的谋划不差。”商无界淡淡地评价了一句,随后他的眉头又微微皱起。

%65
他意识到,按照这种布局的话,方源等人最终都会被追逐到南疆的西北位置。

%66
而在那里,武庸早已经严阵以待。

%67
玉清滴风小竹楼的速度,堪称当世南疆仙蛊屋第一。

%68
不过,武庸能够到达那里,却是用的铁家的仙蛊屋烽火台。

%69
烽火台散布在南疆各处,正是利用这座威能奇特的仙蛊屋,南疆正道才能迅速地铺开一层大网。

%70
而方源等人就成了网罩下的一条大鱼。

%71
刷!

%72
晴朗的天空下,突兀地出现了四位蛊仙。

%73
正是利用上古战阵四通八达,挪移到此的方源等人。

%74
“这已经是第五次了。”

%75
“我身上的道痕可是消耗不少。”

%76
“有点不妙,对方知道我们的位置,围追堵截,像是一张大网,正在迅速收紧。我们是不是要转折方向,打他们一个措手不及?”

%77
白凝冰等人相继建议。

%78
方源摇摇头:“我们身上的侦查杀招才是祸根,不解除它,我们迟早要被追上。眼下之计,便是争分夺秒,闯过包围网,去往西漠。”

%79
只要到了西漠,天庭再想发动西漠的正道来追缴方源等人,就不是那么容易的了。

%80
“走。”方源放出上极天鹰,借助太古荒兽,继续飞驰。

%81
其实真正的上古战阵四通八达,跨越距离很长。可惜的是,影无邪等人掌握的,就不是完整版本。在之后的一连串的战斗中,四通八达的的核心仙蛊不幸地在战斗中损毁。导致方源现在运用的,其实是四通八达的残缺版。

%82
残缺的四通八达,不仅是跨越距离少很多,而且对蛊仙身上道痕的削减更为严重。

%83
“影宗余孽,你们还想往哪里走?”骑着上极天鹰飞了一段,方源等人的耳畔忽然传来武遗海的声音!

%84
“什么?!”妙音仙子顿时花容失色。

%85
“不用怕,这是风语仙蛊而已。”方源冷哼一声,宽慰道。

%86
风语仙蛊的施展范围极广,只要一直催动,不惜仙元损耗,理论上能够蔓延覆盖整个南疆。

%87
但很快,玉清滴风小竹楼出现在了天边。

%88
这座八转仙蛊屋的速度极其惊人!

%89
饶是上极天鹰已经是有数的太古荒兽,飞行速度堪称一绝,但是方源和武庸的距离却在迅速缩短。

%90
“用四通八达!”方源一见情势不妙,虽然还隔着老远,但他毅然选择催动上古战阵。

%91
面对八转蛊仙,再远的距离也不安全。

%92
收起上极天鹰,下一刻,四通八达催动成功,四位蛊仙骤然消失在原地。

%93
然而当方源等人再次出现时,他们的脸色都变得非常难看。

%94
“怎么回事?我们挪移的距离居然这么短?连之前的一成都没有?!”

%95
武庸的声音在群仙耳畔再次响起:“你们中了我的锁风,四通八达已经不好使了。”

%96
话音刚落,玉清滴风小竹楼再次出现在天边。

%97
众仙心中无不压着一块巨石。

%98
有史以来,最大的危机,已迫在眉睫!

%99
八转蛊仙武庸,外加八转仙蛊屋玉清滴风小竹楼!

\end{this_body}

