\newsection{我算不尽是那种人吗?}    %第八百四十一节:我算不尽是那种人吗?

\begin{this_body}

房家大本营。

一座仙蛊屋青砖金瓦,巍峨若山,肃穆中夹带活泼,庄重里酝酿生机,散发出浓郁的草木的清新香气

宫门敞开,可见内里宽阔巨大的殿堂。大堂中有巨柱数十根,根根粗壮,闪烁青铜光泽。

大门上端,有着一座牌匾,上书三个大字,正是——豆神宫。

而在豆神宫的周围,还建设了一座月牙形的智道蛊阵,将豆神宫半包围。

这座智道蛊阵闪烁着淡蓝色的光辉,映照在豆神宫中,蓝光若水,努力向豆神宫内部渗透,但效果不佳。

持续了片刻后,智道蛊阵停歇下来,从里面走出来两位蛊仙。

一位身着黄杉,容貌普通,有胡须,宛若凡人,此刻脸色苍白,额头密布汗渍,正是房家太上二长老房睇长。

另一位一身黑袍,中年模样,身形消瘦,长发披肩,双鬓斑白,显得孤高傲气,正是方源伪装而成的算不尽。

此刻,方源感叹道:“这豆神宫真是了得,哪怕没有人主持镇守,也宛若活物。它就像是一尊参天古木巨树,我们就像是两只小虫,就算是拼尽全力渗透,对于豆神宫而言,也不过是九牛一毛。”

“是啊。”房睇长点头,“更糟糕的是,就是我们折断了这株巨木上的些许枝叶,过不了多久,它就能重现生长起来。”

方源分析:“这的确是元莲仙尊的风格。根据历史记载,他可是历代尊者中仙元最为充沛,并且恢复能力第一,深得木道精髓。二长老,你我都是智道蛊仙,要炼化这座木道仙蛊屋,难难难!不妨去延庆一两位木道的大能,配合我们,必定能有良效。”

房睇长苦笑:“老弟,你以为我没有尝试吗?可恨,西漠本就木道稀少,并且我族夺得豆神宫的消息早已经暴露出去,其他的超级家族组成包围,早在这个方面发力,卡住了我们。”

“当然,主动提供木道蛊仙的势力也有,老弟你猜猜看,是哪一个?”房睇长问道。

方源念头一动,无须多想,当即道:“想来是天庭吧?”

“老弟一猜即中。没错,就是天庭!龙公亲自保证,派遣木道蛊仙取回豆神宫,过往矛盾一笔勾销,并且还允诺了海量资源。”房睇长冷笑。

经历过东海龙宫争夺后,方源对龙公更加了解,此刻听闻龙公妥协,毫不意外。

他道:“天庭打得如意算盘,是想借助西漠当今局势,逼我房家就范。然而资源再多,又怎能比得上八转仙蛊屋豆神宫?只要此屋在手,为我们所用,还怕夺不来资源吗?”

“哈哈哈。”一声大笑,一位蛊仙走向方源、房睇长。

“算不尽老弟,说得太对了。”这位蛊仙乃是一位老者,白发白眉,胡须宛若狮鬃,浑身肌肉贲发,线条强硬,气势浩荡,正是房家的太上大长老房功。

房功拍拍方源的肩膀,态度热情:“老弟,难为你第一天到来,热茶还未喝上一口,就钻进阵中艰苦推算。我已经摆下宴席,给老弟接风洗尘。”

方源笑道:“有劳太上大长老挂怀。攻略豆神宫此事,也激发了我的兴趣和斗志。并且我也是房家的客卿太上长老,帮助房家,就是帮助自己,不是吗?”

心中却是透亮:“这房家压力太大,终于要请我出手相助,尽快炼化豆神宫。房家对我态度热情,丝毫不奇怪。看来,我可以趁机要挟,坐地起价,先捞取一些好处再说。”

方源的心简直是黑透了!

不仅要谋取豆神宫,而且还要在此之前,对房家下手,压榨利益。

房家也是倒霉,把他吸纳进来。若是上一世,没有方源的干预,房家的境况很好。单凭房家自身努力,就解决了危局,即便在最后中洲炼蛊大会的时候,似乎都没有炼化了豆神宫。

一场酒宴,规格隆重,房家真的是要借助算不尽的力量。房家高层态度都很热情,身为八转的房功更是平易近人。

方源当场保证,会出全力相助。

于是,宾主尽欢。

房家若是知道眼前这个受到自己隆重招待的蛊仙,就是为难房家的最大的幕后黑手,搞不好会把方源当场抽筋扒皮!

接下来的日子,方源就跟随着房睇长,辅佐他,通过智道仙阵,炼化豆神宫。

豆神宫生生不息,能自我恢复,排斥渗透。今日进展若是十,明日开始后就会从十退减到三、四。

方源和房睇长每一次动手,都把自家脑海推算得发热、发烫,达到极致,才无奈罢休。

“虽然我没有拼尽全力,但是也展现出了七转智道蛊仙的一流水准。”

“即便如此,房睇长有我相助,也进展缓慢。更别说没有我帮助之前,他是何等局面了。”

方源了解到了情况,不由地对房睇长有了一丝同情,也有一丝庆幸。

正是因为房睇长每次都拼尽全力,强炼豆神宫,才使得自身没有余力和闲情,导致方源算计房家都成功了。

真要比较起来,房睇长乃是智道大宗师,境界上比方源还要高出一筹呢。

每天都是如此竭尽全力,方源虽然隐藏了一部分势力,也渐渐感到疲惫不堪。

房睇长更是如此,他是真的竭尽心力。

依靠着这种拼命的顽强精神,炼化豆神宫虽然一直都很艰难,但也累积出了不小的成果。

“房家有智道、阵道造诣,想要凭此强攻下豆神宫。”

“这是真正的强炼!”

“成功的希望,当然是有的。但是付出的代价却是极大的,别的不说,单单仙元就是一笔巨款。”

“若是增添木道、炼道,恐怕会轻松很多。当然,若是我真正出动全力,不仅是智道,在阵道上也不藏私,那就是另一番景象了。”

方源当然不会这样做。

感觉火候已到,他在某一日,主动停歇下来,没有及时出现。

房睇长连忙赶过去询问:“老弟,你是受伤了吗?要不要紧,我早已经准备妥当,可为老弟你医疗。”

方源咳嗽一声:“有劳二长老牵挂,的确是受伤了,但不要紧,这伤是小事,我自己可以处理。只是……仙元损失着实惨重。并且心也累了,工作量太大太重了。这要彻底炼化豆神宫,该是猴年马月?我实在有些支持不住。”

房睇长听闻这话,双眼微微一瞪。

算不尽要撂挑子!

这可如何是好?

眼下好不容易有了一些成果,若是每天不坚持下去,这些成果也会迅速消散的。

又想到房家的处境,房睇长心焦啊。

“老弟!”房睇长笑道,“我们房家不会亏了老弟的,尽管放心好了。”

方源认真地道:“二长老,您可切勿以为我是故意撂摊子啊!我可并非这样的人!我信任房家上下,加入房家就是最好的证明。同时,我和房家的合作早就开始,当初攻打豆神宫的时候,我们就缔结了盟约。这个盟约我从未忘记啊。”

方源说的语重心长,房睇长身为智道大宗师,立即知道了方源潜在的意思。

他再次笑道:“老弟啊,我明白了,哈哈哈。当初的盟约我们也从未忘记。的确,老弟出力帮助我房家夺取了豆神宫,按照约定,我们房家要交付给老弟仙蛊的,到了现在还拖欠着,是我房家不对!我给老弟道歉了。老弟需要我族的哪知仙蛊,我这就安排人送过来。”

“唉!使不得,使不得。二长老啊,我真的不是这个意思,你误解我了。”方源道。

“误解?我会误解你?!”房睇长心中嗤之以鼻。

他早就看透了算不尽,这是个无利不起早的枭雄!

当初攻略豆神宫,他就是这样趁机开价,把大盗仙蛊给索要走了。

现在又玩这一套!

房睇长很想甩方源几个大嘴巴子,但硬生生忍耐住了。

没办法,眼下需要算不尽出力!有算不尽帮助自己,的确效果良好。

于是,房睇长强烈要给与方源仙蛊,方源却是一直推脱,坚决不受。

房睇长预感有些不妙,算不尽如此表现,必定有更大的图谋。三番五次的推来让去后,房睇长直接道:“老弟,有什么想法,你就直说吧。”

方源正色道:“二长老休要看轻我,我也是房家一员,当下房家局势如此,我岂会做出落井下石、趁机要挟的不齿举动呢!我算不尽是那种人吗?我反而觉得,之前的仙蛊报酬太重,我愿主动退让一步,不需要仙蛊报酬了。”

房睇长更感不妙,然后他就听方源道:“若实在要我说,那我也只好坦诚布公,二长老你是自己人,呵呵,我也没有什么可以隐瞒你的。说实话,我对盗天魔尊的偷道真传啊,有一些兴趣。我也不需要什么偷道的仙蛊了,只是想看一看真传的内容而已。当然,如果二长老为难的话,那就算了吧。”

饶是房睇长智道大宗师,平日风度极佳,此刻也面色一变,双眼鼓瞪地看着方源,心中大骂:“原来你竟是想打我族盗天真传的主意!好大的狗胆子!”

\end{this_body}

