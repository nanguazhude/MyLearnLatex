\newsection{花蝶女仙}    %第一百三十三节:花蝶女仙

\begin{this_body}

%1
既然追查不到对方位置,方源就明智地放弃了追杀。

%2
重获新生之后,这是方源第二次和影宗交锋。第一次是影无邪被迫,和方源交易,结果被方源大占便宜。

%3
第二次交锋,方源却只是稍占优势。尽管方源的实力,因为黑凡真传的原因,暴涨了许多倍数。但最终,方源没有达到本来的目的,只是斩杀了影宗一位成员,对影无邪等人稍有削弱。

%4
方源的实力在增长,影无邪的实力也在增长!

%5
并且,方源实力的增长,不稳定,不长久。毕竟黑凡真传这样的好事,堪称千年机缘,绝不容易碰到。

%6
反过来,影无邪这边的实力增长却是稳定得很。

%7
因为他是去继承各域的影宗传承,本来就是属于他的东西,自然稳定而且可靠,可以预期。

%8
从这点而言,方源提前打杀的计划,无疑是明智的。

%9
因为再不趁着这个机会去行动,双方的差距就会越来越明显。一旦最终,魔尊幽魂被拯救出来,势必就是方源的末日。

%10
而影无邪主动避退的决策,虽然无耻了一些,但也是绝对的理智、果断。

%11
对影无邪而言,只要他不断接收影宗的残余势力,实力就能十分迅速地膨胀起来,并且最终膨胀得相当厉害!

%12
乱流海战,虽然双方没有亲自交手,但隔空对决了一次。

%13
方源得到了影无邪等人的不少情报,尤其是见识了上古战阵四通八达,还有影无邪、黑楼兰等人的最新情况。

%14
而影无邪却是窥破了方源的侦查杀招,虽然只是暂时解决,治标不治本,但他前进的脚步没有被阻止。他的战略仍旧在顺利执行。

%15
方源和影无邪之间,必有生死激战。

%16
但现在,却不是决战的时机。影无邪选择了主动避退。

%17
总有一天,方源和影无邪。必有一人死亡。双方的恩怨,还远远没有到了结的时候。

%18
方源转身回到了乱流海域。

%19
苍穹中白云苍茫,海水里激流涌动,一副混乱景象。

%20
方源正要进去,这时从西边来了一女仙,娇呼道:“仙友慢走。”

%21
方源转头看去,只见此女身着宫裳,衣袂飘飘。粉嫩若腻。一双美眸,外溢秋波,美不胜收。

%22
当下,方源他眉头稍扬,停下动作,静待女仙过来:“原来是花蝶仙子。不知仙友唤在下,所为何事?”

%23
花蝶女仙笑了一下,樱红的小嘴露出珍珠般的牙齿:“真是惭愧,仙友知晓我名,我却不知仙友贵姓?”

%24
“在下楚瀛。”方源随意报了个名字。他此时模样自然也是伪装。

%25
楚瀛,除影也。

%26
虽然此行未能达到方源的本来目的,但方源对付影宗的决意。反而更加坚决。

%27
“楚瀛仙友,我观仙友似乎想进入乱流海域。有一事相烦,若是有成,对仙友自有好处。”花蝶女仙道。

%28
“请说。”方源笑了笑,态度很温和。

%29
心中却在打着主意:不知道杀掉这花蝶女仙,成功的机会大不大?

%30
美人美矣,但在方源心中,自然不如转化成他的道痕积累。

%31
花蝶女仙的情报,方源也知道一些。此人乃是东海的变化道蛊仙。六转修为,已经渡过了两次天劫。

%32
战力方面。考虑以往战绩,应当在六转巅峰左右。

%33
对于方源而言。自然不足为虑。

%34
变化道!

%35
关键是这一点。

%36
目前为止,方源的变化道境界是宗师级,变化道痕不少,但积累更多一些总没有什么坏处。

%37
花蝶女仙也笑了笑。

%38
她感觉方源虽然貌不惊人,但待人接物态度温和,并非那种丧心病狂的魔道蛊仙,估计和自己一样,是东海的散修。

%39
若是让她知道方源现在的心理想法,不知道又作何感受。

%40
“我虽然无法铺设战场杀招,但战力明显强上一截。但花蝶女仙却是和庙明神关系密切。我若杀她,有不少把握。”方源考虑此中的可能。

%41
庙明神乃是东海的七转强者,地位类似于未恢复记忆的秦百胜。

%42
关键庙明神乃是宇道蛊仙,拥有可以扩张仙窍空间的手段。因此,很多东海蛊仙,尤其是散修,都要求到他的身上。

%43
毕竟,仙窍空间有限,很多蛊仙要栽培修行资源,都要精挑细选。庙明神能扩张仙窍空间,很多时候带给了其他蛊仙生相当巨大的便利。

%44
加之庙明神本人,又颇有雄心壮志,擅长屈己待人,可说是富有人格魅力。所以,在他的身边,已经结成了一个蛊仙的利益团伙。

%45
这个利益团伙一共有四人。以庙明神为中心,花蝶女仙、蜂将、鬼七爷为庙明神马首是瞻。

%46
其中,花蝶女仙、蜂将皆是六转修为,鬼七爷是七转蛊仙,庙明神的修为和战力,都是此中最高。

%47
方源若是杀了花蝶女仙,无疑是和其余三位结死仇。

%48
庙明神关系众多,交游广阔,很多东海蛊仙都想和他交好。就算是超级势力的成员,也不愿得罪他。说不定哪天,就求到他的手上呢。

%49
若是庙明神和方源结仇,他必然会牵动身边的关系网,号召大量的东海蛊仙排挤方源。

%50
不过,那又如何?

%51
方源有见面曾相识,可以自由变化身份,管你是不是庙明神!

%52
只要不是八转蛊仙就好。

%53
“只是,我杀了花蝶女仙。见面曾相识却不可遮蔽智道推算,不能保证我身份不暴露。”

%54
见面曾相识,不是万能的。

%55
当初,盗天魔尊依靠此法,设计算计了两位八转蛊仙,也是对方没有针对的侦查手段。

%56
这个战绩虽然显赫,但引发他人警惕。盗天魔尊在此之后,就再不能复制辉煌。

%57
不久前,方源和影无邪碰面。

%58
影无邪只用察运仙蛊。观察方源头顶气运,就洞察出他的身份。瞬间果断撤退。

%59
区区一个六转察运仙蛊,就能让见面曾相识无效。看似匪夷所思,实则不然。

%60
天道平衡,万物生来,都有食物,也都有天敌。仙蛊也是如此,仙道杀招本质上就是蛊虫的组合运用,无法脱离这层规律。

%61
所以说。没有最强的仙蛊,只有最强的蛊仙。

%62
“但我还是智道宗师。只要将来我的智道手段提升上来,将真实身份遮掩住。再用见面曾相识千变万化,谁又能知道我的身份?”

%63
尤其是他至尊仙体的特殊,穿越界壁轻轻松松,还能彻底转化气息,到哪里都是当地人。

%64
不像黑楼兰他们,顶着北原蛊仙的气息,在东海还好些。到了南疆、西漠、中洲这些地方,就要遭受当地蛊仙的强烈排挤。

%65
“只是……我好像记得。这庙明神有发现了那头太古荒兽苍蓝龙鲸的手段,曾经秘密招揽过十多位东海散仙,一起前往其中探索。我是不是该和他搞好关系。借助这层关系,也进入那苍蓝龙鲸中分一杯羹?”

%66
这样考量下来,苍蓝龙鲸显然比区区花蝶女仙要重要得多。方源便悄然打消了斩除花蝶女仙的打算。

%67
若能和庙明神交好,方源还有一个好处。

%68
东海的修行资源,在五域中为最。东海蛊仙时常举办聚会,相互换取修行资源。这些资源往往都十分珍稀,规格很高,动用寿蛊交易。相比较而言,宝黄天只是大众市场。不是这种私人的拍卖会。

%69
庙明神恰恰是这种私人拍卖会的常客。有他担保,方源也能混入其中交易了。

%70
花蝶女仙却不知道。自己已经在鬼门关转了一圈。就因为自己和庙明神的关系,才险险避开死劫。

%71
她微微带笑。道明来意:“楚瀛仙友,既知我名,想必也知道庙明神大人。庙明神大人,已经考察乱流海域多年,希望从中寻出无数乱流中的那一股光阴支流。若是仙友此番有所机缘,见到了那股光阴支流,还请仙友跟随追踪,并告知我等。我等届时,必有重谢。”

%72
至于庙明神一位宇道蛊仙,为何要追寻光阴支流,花蝶女仙显然不想解释。

%73
“原来如此。”方源欣然答应下来,“庙明神大人,我早已闻名已久。重谢不敢当,能和庙明神以及花蝶仙子你们结一场善缘就好。”

%74
方源这么识趣,花蝶女仙脸上的笑容不免更盛一分。

%75
方源并非她交谈的第一位,事实上,花蝶女仙、鬼七爷、蜂将每隔一段时间,都轮流看守乱流海域。遇到每一位前往乱流海域探索的蛊仙,都会上前交谈,希望他们出手帮忙。

%76
有人拒绝,但更多的人会选择同意。

%77
原因自然是庙明神那可以开拓仙窍空间的手段。

%78
所以,方源答应下来,花蝶仙子也没有什么意外。

%79
“那就祝楚瀛仙友佳运了。”花蝶女仙转身就要离开,她对楚瀛并不看好。

%80
因为她从未见过,也未听过东海有这号人物。

%81
当然,隐修蛊仙东海从来不缺。只是花蝶女仙自诩交友广泛,楚瀛这种连名声都听不到的人物,又只是六转修为,能强到哪里去?

%82
“庙明神大人寻找光阴支流的事情,不需要修为,需要的是运气。说不定这位楚瀛,就能寻到呢?唉,每一次我都有了这种念头。可惜,已经寻找了十几年,都不见那条光阴支流的踪迹。”

%83
花蝶女仙心中叹息,和方源告辞。

%84
不过,远离了一段距离,花蝶女仙又利用信道凡蛊,传过话来。

%85
“差点忘记告诉仙友了。乱流海域中来了一位血道魔仙丁齐,此人的弟弟丁言,曾因为掌握了一份信道传承的关键线索,遭受刘青玉、周礼、汤诵等人的追杀。丁言已死,丁齐矢志复仇。前不久听闻刘青玉来到了乱流海域,似乎是要开启信道传承,所以赶来。血道魔仙都是丧心病狂之徒,不像你我,他们的心理已经魔化病态。见到一位蛊仙,他们就考虑是不是杀了,来增添自己的战力。仙友若是遇到此人,还请小心为上。”

%86
方源缓缓收起信道凡蛊,摸了摸自己的鼻子。

%87
花蝶女仙的关照之言,显露出此女的善良品性。不过她却不知道,她叮嘱的这人,正是血道魔头,而且曾经是在中洲崛起,恣意屠戮千万生命,罪孽深重之人!

\end{this_body}

