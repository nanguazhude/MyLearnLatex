\newsection{战斗体系}    %第九百三十节:战斗体系

\begin{this_body}

%1
方源变化成太古剑龙,同样具备了速度的优势,龙公追击一阵,竟发现无法快速缩短两者之间的距离。

%2
仙道杀招——龙门!

%3
龙门陡然出现,两根赤红巨柱,各有巨龙缠绕,巨龙蜿蜒,上半身延伸,离开巨柱,在中间交汇,形成门头。巨龙嘴巴张开,一颗明珠在龙口之间熠熠生辉,形成双龙戏珠的格局。

%4
两道龙门同时出现,一个出现在龙公的面前,一个则出现在方源的身后。

%5
龙公踏入眼前的龙门消失无踪,下一刻就从方源身后的龙门中出现。

%6
方源心叫不妙,连忙躲闪,但龙公及时施展气盖山河,拖延了他的速度。

%7
仙道杀招——乱龙拳!

%8
下一刻,龙公双拳狠狠地砸在太古剑龙的龙头上,并拳拳乱打,将方源硬生生地从高空打到了仙墓上。

%9
轰。

%10
太古剑龙狠狠地砸在地上,造成烟尘滚滚,土石翻飞。

%11
太古剑龙艰难昂首,捏成拳头的龙爪并未散架,而是继续暴射出第二道剑光。

%12
五指拳心剑第二剑!

%13
龙公早就全身戒备着,方源刚有动作的时候,他就立即躲闪,翻身到方源的龙背上。

%14
第二剑顿时射空,擦着龙躯,直贯苍穹,直接把天射穿,形成一个窟窿。

%15
五指拳心剑的剑光速度极快,但却不能绕弯。龙公或许难以招架剑光,但他却能闪避。闪避剑光或许有些困难,但他却可以根据方源的动作,来判断方源射出剑光的时机,从而提前躲避。

%16
毫无疑问,五指拳心剑是非常强大的杀招,威力绝伦,并且和光阴飞刃比起来有一个巨大的优势——它不会自我遗忘。

%17
但是在龙公明确的战术下,五指拳心剑反而不能建功,更连累方源变化而成的太古剑龙遭受龙公的近身缠斗。

%18
短短十几个呼吸以内,方源已经遭受了不知多少下的龙拳重击。

%19
最强大的从来不是什么仙蛊,亦或者杀招,而是蛊仙!

%20
五指拳心剑最理想的施放环境,就是在刚刚释放第一剑的时候。方源和龙公相距甚远,可以遥攻,自身稳稳立于不败之地。龙公就算反应敏锐至极,也难以招架,每一剑都能够让他陷入死亡的重大威胁之中。

%21
但是当龙公近身缠斗的时候,方源的龙躯反而成为了剑光的阻碍,龙公的保护伞。

%22
回顾战斗,从某种程度上可以说,龙公动用了龙门杀招破解了方源的五指拳心剑。拉开不了距离,就没有施展五指拳心剑的空间。

%23
所以,第二剑之后,方源立即撤销了五指拳心剑,转而改用金丝剑杀招。

%24
龙公看到方源的龙鳞开始闪烁黄光,立即明智地飞退,再次动用龙门,远远撤离。

%25
无数道金丝覆盖了方源身边每一寸的地方,但龙公早已撤走。

%26
不过没有关系,方源已经再次拉开了他和龙公的距离,他又可以继续施展五指拳心剑了。

%27
龙公处境危急起来,他不得不再次逼向方源,企图近身缠斗。

%28
当他近身缠住方源,方源就得再度撤销了五指拳心剑杀招,改用其他手段来战龙公。

%29
“这魔头将五指拳心剑竟然改良到了这种程度!随意撤销杀招,都没有任何的反噬伤害。或许应该这么说,他把每一剑都分割开来,形成单独的个体了。”龙公心中迅速思量。

%30
五指拳心剑乃是薄青的王牌手段,天庭怎可能不知?

%31
龙公自然知晓许多,方源不改良必定会被他针对。

%32
龙公的道痕不及方源,所以难以用防御杀招直接抗衡方源的种种剑道手段,他最好的选择就是躲闪。

%33
幸好要论战斗经验,他远比方源丰厚得多。

%34
并且最关键的一点,只要他不死,伴随着寿命的缩减,他都在不断地变强!

%35
“五指拳心剑虽然经过我的改良,可以随意撤销。但牵扯到的念头却是大规模暴涨,导致我在和龙公激战的时候,不能同时催动另外的攻伐手段……还有一点,龙公越来越强了!”方源心头焦虑。

%36
之前他能够力压龙公、苍玄子联手,很大程度上是战术优势——方源攻击仙墓,天庭一方不得不竭力防守,被动挨打。

%37
但现在仙墓被方源摧毁,或许还残存个别的沉眠蛊仙,但龙公已然没有了包袱和牵绊。

%38
因此,龙公可进可退,立即展现出他原本的战力和风采。

%39
方源力战龙公,两人不可避免地陷入到了僵持局面。

%40
这个局面龙公乐于见到,但对方源却是十分不利的。

%41
所以,龙首一转,方源张开一吐,就是一道瀑布。

%42
仙道杀招——万念剑瀑。

%43
剑瀑轰鸣,气势宏大,每一滴水都是剑念。

%44
仙道杀招——剑客!

%45
剑念激荡,不断地碰撞,形成一个个人形战士,貌似方源,皆手持长剑。无穷剑客冲杀而出。

%46
这一招是方源晋升人道宗师后,顺势开创出来,兼收并蓄了剑道、人道的奥妙!

%47
龙公第一次接触这个杀招,连忙避退。

%48
方源趁机脱身,向监天塔撞去。

%49
监天塔外,劫运坛正在摧枯拉朽地摧毁炼道大阵。而塔内,七次郎正奋力挣扎,和身上残余的微微碧光死死较劲。

%50
龙公见到方源动向大惊失色,诛魔榜奋力阻挡。

%51
结果方源一记龙爪,直接将它打飞。

%52
就在这个关键的时刻,一道歌声忽然降临战场,缠绕在太古剑龙的身躯上。

%53
方源顿时感到状态急剧下滑,仿佛再次回到了人生的低谷,受制于人,才干和志向无法抒发,不得自由,前景黯淡无光。

%54
太古剑龙因而速度暴降。

%55
龙公迅速赶到,拦截住方源:“干得好,凤九歌!紫薇仙子!”

%56
关键时刻,正是凤九歌的命运歌成功地干扰到了方源。

%57
而及时指引他进来的,正是紫薇仙子。

%58
紫薇仙子开启了中天门。

%59
这座仙蛊屋本就是天庭门户,有甑辨和传送威能,只是远远不如传送大阵那般宏大、便利而已。

%60
命运歌持续不断,方源战力下滑得十分严重,而龙公却是得到增幅,战力不断上涨。

%61
轰轰轰!

%62
太古剑龙被龙公不断击退,再无法靠近监天塔。

%63
交战以来,方源还是首次落入这样的下风,迫不得已之下,他停下一切剑道杀招,开始酝酿逆流护身印。

%64
龙公见他不断后撤,气息勃发,立即知晓他是要酝酿另外杀招,哪里肯给他机会?

%65
龙公连连咆哮,身绕紫金龙形气劲,追着方源不断猛打狂攻,掀起暴风骤雨般的攻势狂澜。

%66
方源化身的上古剑龙被打得皮开肉绽,龙鳞纷飞,鲜血飚射,更可怕的是龙公的乱龙拳能够扰乱思维,导致他催动杀招失败。

%67
艰难时刻,方源打开仙窍门户一丝,放出囤积好的纯梦求真体。

%68
纯梦求真体一个个飞向龙公,接连自爆。

%69
龙公忌惮万分,连忙躲闪,攻势收敛起来,方源这才有了喘息之机,成功催起逆流护身印。

%70
龙公看到此印,不由叹息一声。刚刚的时机非常好,但他没有抓住,他和方源交手这么多的回合,早已经看透方源的战法。

%71
方源通过某个变化道的杀招,变成太古兽植的同时,还能将一身道痕转变为同一流派。在这样的基础上,施展任何杀招都是威能倍增。即便是七转杀招,也能威胁到八转。

%72
方源手段众多,很难想象他在一百年不到的时间里,就修行、积累到了如此程度,即便是有尊者传承,也足以让人咋舌。

%73
龙公相信:只要给予方源充足的时间,他一定能够达到一种前无古人的高度。幸亏宿命蛊的修复形成大势,让方源不得不提前进攻天庭。

%74
方源或许能够凭借一些精妙、恐怖的手段,战胜世间绝大多数的敌人。但是面对龙公这等存在,些许手段是远远不够的,必须还有一套足够强大、稳定,并且拥有杀手锏的战斗体系。

%75
就像龙公的变化道、气道手段,形成了两套战斗体系。

%76
九龙纹护身形成强大防御,澄龙澈瞳用于侦查,龙爪击、乱龙拳、回旋龙牙、憾世龙锤担负进攻之责。这四个手段覆盖近战、中程和远战。其中憾世龙锤更有重击一点、突破难关的威势。而龙啸波、随身闪是常规的腾挪手段,而长距离传送则有龙门这个优秀杀招。还有龙子龙孙弥补群攻手段的匮乏。而这套体系的最强手段就是龙御上宾。

%77
至于气道体系,有气墙、自转游龙气墙作为防御手法,有气呼山、潜龙气爆、气盖山河、一气大手爆、气流剪作为攻伐杀招,比起变化道的体系,这些攻伐手段无疑在远程威力更显著。紫金龙形气劲还是万金油,这个杀招是如此的优异,乃至于在各个方面都对龙公提供着帮助。当然最主要的还是负责腾挪转移。逆反虫龙杀招几乎是专门为了针对战场杀招所准备的。而这套气道体系的王牌杀招则是三气归来。

%78
龙公的这两套战斗体系都兼顾了攻伐、防御、腾挪等等诸多方面,并且没有短板和空白。更难能可贵的是,这两套战斗体系还相互弥补。比如变化道的攻伐手段,在近战、中程更加强势,而在远战方面薄弱。而气道的攻伐手段则恰恰相反。

%79
相比较而言,王牌手段根本不需要多。每一套一个、两个这种手段,都足够用了。

%80
真正战斗的时候,最实用的仍旧是那些常规杀招。就比如龙公,虽然拥有三气归来杀招,但和方源交手至今,都未使用出来。战斗的整个过程都非常激烈,方源又非蠢货,龙公根本没有这样的机会来用它。

%81
又例如方源使用五指拳心剑,这当然是王牌手段。但方源即便用了,效果也没有一锤定音。龙公仍旧能够维持住僵局,实质上是他的整个战斗体系更加全面,

\end{this_body}

