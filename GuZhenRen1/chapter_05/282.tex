\newsection{卜卦龟变化}    %第二百八十二节:卜卦龟变化

\begin{this_body}

南疆,武家。\&\#26825;\&\#33457;\&\#31958;\&\#23567;\&\#35828;\&\#32593;\&\#119;\&\#119;\&\#119;\&\#46;\&\#109;\&\#105;\&\#97;\&\#110;\&\#104;\&\#117;\&\#97;\&\#116;\&\#97;\&\#110;\&\#103;\&\#46;\&\#99;\&\#99;]

方源以“武遗海”的模样,踏足南疆之后,便一路兼程,回到武家大本营。

得知他的到来,武家当代太上大长老武庸立即召见。

“兄长,这是之前商借族中的仙蛊,如数奉还。”方源将仙蛊双手奉上。

武庸接过来,身居主位,面带微笑:“二弟此去东海,可还顺遂?”

每个蛊仙都有自己的秘密,尤其是对于武遗海这样特殊身份的存在,武庸也不好过多的探听方源。这番问话,以客套居多。

“收获不小。”方源脸上稍微露出一丝喜色,还有一些感慨,表现得非常真实。

“这就好。”武庸点点头。

他怎么也不会想到,方源收获的丰富。

这一次离开了南疆,方源一路追杀影宗残余势力,虽然最终的目的没有达到,但是收获却极其巨大。

方源不仅收获了坚持仙蛊,成为逆流河主,更战力飙升,成为八转蛊仙都奈何不住的人物。柳贯一之名,名动五域!

并且,在回归的路途中,还参与了一场东海的高端交易会。在会中换取了卜卦龟背仙蛊、六转日蛊。

“劳兄长记挂。小弟何时返回蛊阵中去继续隐修?还请兄长安排。”方源又道,语气和态度都非常谦逊。

武庸对方源的表现非常满意。

这些天来,他执掌武家上下,因为方源的主动退让,武家的内部阻碍大大降低,让他顺利地统筹一切,掌控了权柄。

此时此刻的武庸,已经和当初刚上位时,有了一丝微妙的改变。让他变得更加从容,更加威仪,给人大权在握,不可侵犯之感。

对于方源如此识时务,武庸自然欣慰。

家族体制,不同于门派,更注重于血脉渊源。

武遗海的身份,让他在武家蛊仙中分外特殊。单纯以血脉而论的话,方源可算是一人之下,万人之上。

武遗海既然一心想要隐修,武庸当然要成全他。

当即,武庸就答应下来:“弟弟且先回归住处,暂且休息。为兄这就安排,几日之内,必有安排。”

“那就谢过兄长了,小弟告退。”方源见目的已经达到,立即抽身退下。

他在武家也有住处。txt小说下载80txt.com

在刚刚加入武家之后,他就被分配了一个山峰供方源独居。

“这武庸已经初步掌控了武家。”

“武遗海这个身份很敏感,若是和乔家这样的野心家汇通,一起搞风搞雨起来,也会让武庸头疼不已的。”

“所以武庸稳妥起见,绝不会是放任武遗海四处游走,为家族建立功勋,培养威望。”

其他武家蛊仙可以,但武遗海却不行。

至少暂时是不行的。

毕竟武独秀才刚刚去世不久,武庸掌握了武家,但还不稳定。

方源对武家的局势,洞若观火。

他并不担心,自己回归不了超级蛊阵中去。

在山峰的日子里,他继续修行。

但是仙窍是不能落的。

更别谈打开仙窍门户,汲取天地二气了。

“我这至尊仙窍,底蕴太强,每一次吞吸外界的天地二气,都是巨大异变,掀起呼风,卷动啸云,动静过大。”

“不过逆流河的确很是损耗天地之气。看来我即便回到义天山遗址中去,也要时隔不久,就要汲取外界天地二气,陆续补充,才能维持至尊仙窍,不至于底蕴耗减。”

这是一种负担。

不过性质甜蜜。

虽然武庸关照他多休息,但方源并没有休息。

他继续修行,仍旧是争分夺秒。

期间不少家族蛊仙,来探望这位身份很特殊的武遗海,但都被方源回绝。

这个态度和作风,让武庸非常满意,认为是武遗海不想染指家族内务的表现。于是,多次以个人的名义,送给方源不少仙材。不仅是示好和嘉奖,更为他自己怒刷了一波“爱护亲弟”的名望。

山峰之巅,空阔的大殿中方源已经化为一头巨龟。

这头龟大如房屋,身体黑幽,龟背又厚又硬,壳上纹路纵横,数千上万道,让人看了眼花缭乱,偏偏又井然有序。

正是卜卦龟。

这等上古荒兽隶属智道,变化之后,方源一身的变化道痕,就都转变成了智道。

再加上他之前的智道道痕积累,顿时达到了一种巨大的量变。

有着这样的基础,方源再动用智道手段推算,效率顿时节节暴涨,是之前的数倍,甚至有时候能超过十倍!

一只很简单的卜卦龟背仙蛊。单纯要论蛊虫本身的话,它和态度蛊、慧剑蛊无法相比,就连在剑遁、飞剑、龙息等蛊前,也黯然失色。

然而,对于方源个人而言,它却是再实用不过了!

忽然,卜卦龟的眼眸深处,闪过一丝精芒。

“经过数十天的努力,终于将卜卦龟变化杀招,推演成功了。”

方源心念一动,重新恢复人形。

随后,他试着催起推算成功的完整卜卦龟变化杀招。

一次、两次、三次……

连续十几次之后,他不断在细微处调整,终于大功告成。

全新的卜卦龟变化!

新添了大量的变化道凡蛊,诸如龟首蛊、龟尾蛊、龟足蛊等等。还有一些智道蛊虫,将卜卦龟防备推算的天赋威能,增添到方源的能力极限。还有诸多防御蛊虫。

当然,核心只有两个。

一只是七转卜卦龟背仙蛊,另外一只则是变形仙蛊。

全新的卜卦龟,比之前的体型还要巨大几分,并且栩栩如生,仿佛真的是一头上古荒兽卜卦龟了。

尤其是龟背山,道痕密密麻麻,纵横交错,多达十多万道。

“很不错了,已经到达了我目前的能力极限。”方源试验了一番之后,非常满意。

这多亏了他有智道、变化道的宗师境界。

推算这些杀招,宛若吃饭喝水一样的本能。

类似于上古剑蛟变化,以变化道融入了剑道。那么这个卜卦龟变化,就是让变化道和智道相合。

至此,方源又增添了一记全新手段。之前不完整的草率版本,还拿不到台面上。这个却是完全可以。

最关键的是,在上古剑蛟、逆流护身印、万我等杀招不能随意使用的情况下,这记卜卦龟变化补充进来,非常及时,解了方源燃眉之急。

方源收起变化,休息了片刻,缓过精神。

他闭目沉思,将心神都投入到至尊仙窍中。

江山如故!

六转仙蛊催动起来,对准仙窍中的逆流河。

红枣仙元剧烈损耗,逆流河在江山如故仙蛊的威能下,缓慢增长了一丝丝河水。

“效率太低了。”方源叹息。

这已经不是他第一次试验,结果让他十分失望。

江山如故仙蛊,虽然能够对逆流河有着效用,但效果很不明显。

这和荡魂山、落魄谷形成了鲜明的对比。

“为什么我用江山如故,恢复荡魂山、落魄谷效果上佳,惊人般有效。但是对于逆流河,却是如此无力呢?”

方源心生疑惑。

是逆流河,比之前的两座天地秘境更加特殊吗?

还是其他的原因?

若是能够立即修复逆流河,那就太美妙了。逆流河水几乎运用不绝,使用不尽。

但事实比方源料想的,要不完美得多。

“等等!荡魂山、落魄谷曾经落于幽魂魔尊之手。而幽魂魔尊创造出至尊仙胎蛊。或许……这两座天地秘境已经遭受了影宗改造,就像之前,黑楼兰那些人,可以运用其他流派仙蛊,也不受道痕掣肘一样。”

方源心中忽然灵光一现。

他猜测得非常正确。

影宗不仅有对于蛊仙的秘方,能让蛊仙修行其他流派,而道痕之间不互斥。更有对于天地秘境的改造秘方,让天地秘境对于某些流派的仙蛊,受效奇佳,不受排斥削减之扰。

对于荡魂山、落魄谷,若是动用宙道仙蛊,效果是百分百,不受山谷中无数魂道道痕的恐怖削减。

这是因为江山如故、人如故,本来就是影宗之物。后来才被紫山真君,交给了太白云生。

至于逆流河,影宗并没有得手,事实上也对这道河流,没有多少兴趣。

因此,逆流河并未收到类似的改造,这就导致了当方源对河流运用江山如故仙蛊的时候,收到河水中海量道痕的削减,效果极低。这还是方源成就了逆流河主的基础上。否则的话,江山如故蛊的效用会被整条河流直接逆反出去。

“逆流河水削减了许多,被八转存在耗费了不少。不过幸好,它始终都在自我修复,只是恢复的速度很慢就是。”

“我能不能借助逆流护身印,来对抗太古年兽呢?”

方源很快就摇头。

借助逆流河,他顶多做到自保的程度。太古年兽若是要破坏仙窍,他就拿它没有丝毫办法。

之所以这么想,方源当然是念及那只被封印的八转仙蛊似水流年。

可惜即便战力突飞猛进,方源还未等到解封似水流年的时机。

就这样方源默默修行,静静等候武庸安排,让他重回超级梦境。

没想到的是,几天后,发生了一个小小的意外。(未完待续。)<!--80txt.com-ouoou-->

------------

\end{this_body}

