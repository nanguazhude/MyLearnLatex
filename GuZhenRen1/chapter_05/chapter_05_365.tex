\newsection{方源的判断}    %第三百六十五节:方源的判断

\begin{this_body}



%1
轰!

%2
剧烈的爆炸声中,火焰四下卷席。无弹窗,最喜欢这种网站了,一定要好评]

%3
炙热的气浪,形成厚实的气墙,猛地向四周排开一切。

%4
一时间,蛊阵中黄沙漫天,烟尘四起。

%5
尘土渐渐落下,显露出一个孤单的身影来。

%6
正是黑楼兰。

%7
此刻,她呼呼地大口喘着粗气,浑身披着一层火焰衣裳。原本绚丽多姿,火焰缭绕,不过此时,这层衣裳就只剩下了点点火苗。

%8
“就连这一招,都不能轰破这座蛊阵吗?”发觉自己仍旧身处蛊阵当中,黑楼兰的心已沉入谷底。

%9
激战已经持续了大半个时辰,黑楼兰什么方法都用尽了,怎么也轰破不了这座蛊阵。

%10
如此一来,她完全陷入被动和下风。

%11
事实上,整个交战的过程中,她连对手是什么模样,都不得而知。

%12
这位南疆正道蛊仙,一直位居幕后,操纵蛊阵,从未露面过。

%13
“我必须承认……黑楼兰你的实力很强。”

%14
“但可惜,你根本没有一丁点的阵道造诣。”

%15
“实话告诉你,就算你的战力再暴涨一倍,也不能强行轰破我的这座招牌蛊阵。”

%16
幕后的阵道蛊仙说到这里,发出得意的长笑。

%17
黑楼兰冷哼一声,心中则叹息:“仙元不足,手段用尽,没想到我黑楼兰最终,是折损在这里。唉!娘,我不能为你报仇了,虽然黑家已经破灭,但没有手刃黑城这个贼人,这是我此生最大的憾事!!”

%18
就在这时,阵道蛊仙的笑声,忽然止住。

%19
“该死!”然后他发出一声低吼,吼声中难掩惊骇之情。

%20
“有转机?!”黑楼兰顿时精神一振。

%21
然后下一刻,她就看到蛊阵的灰暗世界,忽然亮起一道长长的光。

%22
这是剑光!

%23
剑光一闪即逝,但在暗黑的天空中划出了一道长长的白色痕迹。

%24
白色痕迹迅速扩大,黑楼兰就听到嘹亮的蛟鸣之声,从小到大,旋即响彻耳畔。

%25
黑楼兰心头顿时狠狠一震。

%26
她对这个蛟龙的叫声,太过熟悉了。

%27
印象太深刻了。

%28
因为就在不久之前,黑楼兰就是被这个声音的主人追杀,狼狈不堪,依靠着影无邪等人的通力合作,这才险死还生。

%29
“是方源!”黑楼兰差点欢呼出来。

%30
从未有这么一刻,他觉得方源的蛟鸣,是如此的美妙和嘹亮。

%31
白色的剑痕在暗黑的天空中,分外显眼。

%32
剑痕迅速扩大,就像是撕开了一道口子,刺眼的光明从这个越来越大的口子中,倾泻进来。

%33
“不——!”那位操纵仙阵的南疆正道蛊仙,发出不甘的嘶吼。

%34
然后下一刻,让黑楼兰头疼无比的仙阵轰然崩溃。

%35
视野中一片光明。

%36
黑楼兰全神戒备,眯起的双眼慢慢睁大,适应了光线之后,她发现方源正站在她的身边。

%37
而那位南疆正道蛊仙,正被妙音仙子和黑菟姑娘两人联手夹攻。

%38
“安全了!”黑楼兰顿时吐出一口浊气,心神放松下来。

%39
绝处逢生的喜悦,很快淡去。

%40
枭雄心性的她,只是瞥了方源一眼,便开始直接盘坐在地上,积极为自己疗伤。

%41
对于黑楼兰而言,刚刚的情势非常惊险。

%42
她身上的沉重伤势,足以说明惊险的程度,完全是命悬一线。

%43
方源没有帮助她疗伤,除了人如故仙蛊之外,方源并没有其他拿得出手的治疗手段。

%44
若是方源自己受伤,除了人如故仙蛊,或许他还能变作上古荒兽,依靠上古荒兽的自我恢复能力,进行疗伤。

%45
但落到黑楼兰身上,这种变化道的法子肯定不行。

%46
战斗很快结束。

%47
妙音仙子,以及黑菟姑娘都是七转中的强者,两人合力围攻另外一位七转蛊仙,自然牢牢占据上风。

%48
而这位七转南疆正道蛊仙,本身是修行阵道。布置蛊阵是他的拿手好戏,但是若单枪匹马地独自作战,那是他的短处。

%49
更要命的是,他所布置的仙阵被方源破坏,蛊阵一破,反噬即临,顿时让他身受重伤。

%50
如此内忧外患的情况下,这位七转南疆正道蛊仙很快就命丧当场。

%51
这是七转阵道蛊仙,比较少见,方源将他尸首收入至尊仙窍,并未急着吞并。

%52
他唤出上极天鹰:“走!我们去支援白凝冰。”

%53
众仙皆知此时情况紧急,稍有延误,很可能便是南疆大部队的围剿。

%54
正因为如此,方源才连这具尸体上的仙窍都不忙并入至尊仙窍,而是赶紧乘着上极天鹰赶路。

%55
当他们见到白凝冰的时候,战斗刚好结束。

%56
方圆千里,尽是冰天雪地。

%57
白凝冰处于白相的形态,傲立雪峰山巅。而她的对手,已然两死一伤。

%58
白凝冰以一敌三,竟然大胜!

%59
“白相杀招,没错,这就是白相杀招!曾经带给整个南疆的白色恐怖,没想到竟然在白凝冰的身上重现了!”

%60
受伤的南疆蛊仙,心中充斥着震骇之情,拼尽全力,向后奔逃。

%61
当他见到天边出现上极天鹰的时候,他几乎都要绝望了。

%62
不过方源并没有追杀他,而是接过白凝冰,转而直接撤离。

%63
此刻逃亡,分秒必争,没必要为这个七转蛊仙,浪费极其珍贵的时间。

%64
这位七转蛊仙既然能够带伤逃亡,而其他两位战友已然身死,必定通宵逃生之能,短时间杀掉他,并不容易。

%65
上极天鹰掉转方向,直朝西漠飞去。

%66
白凝冰散去白相杀招,彻底瘫倒在鹰背上。

%67
她已经油尽灯枯,白相杀招虽然强大变态,但牵扯心神极多,仙元消耗也极其迅猛,对于白凝冰而言负担非常沉重。

%68
若是哪位受伤而逃的南疆蛊仙,再加把力,说不定就能勘破白凝冰的虚张声势。

%69
只可惜,他被白相杀招吓坏了。

%70
这仙道杀招端的厉害,只需些微渣滓残片,都能复生,几乎是打不死的白色怪物。

%71
除非有人破解这个杀招,或者骤然间将白凝冰杀得一点渣滓都不留。

%72
白凝冰一散去杀招,就彻底昏死过去。

%73
若非方源等人救助,她恐怕会心神衰竭而死。所幸方源通晓智道手段,正是治疗白凝冰此症的良医。

%74
当白凝冰悠悠醒转,上极天鹰已经飞跃了十多万里。

%75
“你居然会跑过来救我?”白凝冰望着方源,开口的第一句话,很不客气,表达出她心中的惊异。

%76
白凝冰再清楚方源的性情不过,从未准备得到方源的救援。

%77
黑楼兰怀疑方源的目标,曾经猜想方源会不会将她当做弃子。白凝冰心底想得更彻底,肯定方源会抛弃她们,独自逃生!

%78
方源目无表情地瞥了白凝冰一眼,淡淡地道:“情况比你想象的还要严重。我们每个人的身上,都中了侦查杀招。所以你们才会在撤退的时候,遭遇到南疆正道蛊仙的拦截阻击。”

%79
方源虽有道可道仙蛊可以侦查蛊仙身上的道痕,但他并不清楚其他蛊仙身上的道痕,原来的数量有多少。

%80
没有这个数量,就算方源侦查到黑楼兰、白凝冰等人身上的道痕种类数量,又有什么用呢?

%81
不过这一次,从敌人的表现就很明显地可以看出,她们的身上也中了类似的侦查杀招。

%82
“是南疆哪一位蛊仙出手?方源你已经成了影宗之主,继承了紫山真君的遗藏,也不能解开吗?”白凝冰问道。

%83
“我已经尝试过了。”方源叹了一口气。

%84
紫山真君的仙蛊并不全面,在战斗中损失了不少。虽然有着好几个手段,可以应付眼前情景,但方源却是巧妇难为无米之炊。

%85
黑楼兰听到这里,目光一闪,她想到了琅琊派!

%86
既然仙蛊不足,自然是可以炼蛊,就她所知,方源和琅琊派之间关系极为密切。

%87
方源完全可以借助那边的关系。

%88
事实上,方源也这么做了。

%89
在不久之前,他决定回援白凝冰等人的时候,便提前与毛六进行了沟通。

%90
结果毛六并不承认方源的身份,他怀疑方源的动机,要求方源救下影无邪。毕竟如今只剩下他们两位幽魂的分魂了。

%91
而影无邪、白兔姑娘,都在妙音仙子的仙窍中暂存。

%92
这是方源回援的一个理由。

%93
毛六不愿意配合,方源就主动找上琅琊地灵。让方源感到庆幸的一点是,琅琊建派,琅琊地灵再不像之前那位难以沟通,只需要方源的门派贡献,就能借助琅琊派上下的力量,进行炼蛊。

%94
方源只是将紫山真君遗藏中的某些内容,上缴上去,就令他的门派贡献蹭蹭上涨,换得琅琊地灵亲自出手,为他炼制仙蛊。

%95
在方源的一再要求下,琅琊地灵舍弃毛民天地流的炼蛊方式,采用人族隔绝流。

%96
没办法,天意瞩目之下,若是采用毛民天地流法炼蛊,有天意干扰,恐怕死活也炼不出什么成果来吧。

%97
琅琊派虽然开始为方源炼蛊,但炼蛊终究是需要一个过程的。并且仙蛊难以炼制,未必能够迅速成功。

%98
“情况更复杂,我怀疑天庭已经和南疆正道联手,齐力剿除我等。天庭拿南疆正道当枪使唤,但绝不会真正袖手旁观。我们要小心,除了南疆正道蛊仙之外,很可能还有天庭蛊仙潜伏着,伺机对我们下手!”方源沉声道。

%99
ps:可能是前段时间太疲劳了,状态一直都很低迷,最近这些天我要休息一下,20点会有一更,这是保底的。如果是两更,那么会在白天早上的8点放送出来。不想水文,诸君且容我稍稍养精蓄锐。

\end{this_body}

