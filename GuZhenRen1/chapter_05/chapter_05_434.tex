\newsection{宝黄天震动}    %第四百三十五节:宝黄天震动

\begin{this_body}

%1
降价了!

%2
随着方源的降价,谢宝树等三大卖家支撑了数天后,也只好跟着降价。

%3
不过,他们降低的价格,也不高,且步调都很一致——就是和方源一样的价格。

%4
“我在年蛊生意上只是一个新人,身份不明,来历神秘,相比较我而言,谢宝树、荣欣或者王明月,都是声誉上佳,经营多年,深得他人的信任。所以他们三人只要降下的价格和我的一样,那么那些有意购买年蛊的人,首先会选择的,仍旧是他们。”

%5
对于谢宝树等三方的谋算,方源是心知肚明。

%6
这就是谢宝树等人优于方源的地方,之前多年的经营和积累,早已经深入人心,不是开玩笑的。

%7
“不过……这样又能如何呢?”方源淡淡地笑了笑。

%8
就在谢宝树等人降价的当天,方源继续降低卖价。

%9
宝黄天中各位蛊仙,早已经关注着这场罕见的大商战,方源一降低卖价,消息流通得非常快,随后不久,几乎关注此事的蛊仙和势力,就都知晓了此事。

%10
“听闻谢宝树降下价格,我原本还打算在他那里买呢。毕竟往年都是如此。”

%11
“是啊,不过我更倾向于荣欣的货。”

%12
“可是那第四人居然又降价了。看来真的是实力雄厚,要和三大卖家叫板啊。”

%13
买家们喜笑颜开,都很放松,也很开心,毕竟打了价格战,讨便宜的还是他们。

%14
而且坐着不动,就看看一场龙争虎斗的好戏,也是一件难得的乐事。

%15
在同一价位上,方源自然和其他三方比不起来,没有什么竞争力。但是当他降价,价格方面低于三方之后,他的优势就大了。

%16
毕竟,方源的货本身就很优秀,价格又低,难道蛊仙们会那脑残到专买贵的么?

%17
所以,接下来的这一天,仍旧是方源生意火爆,其他三家无人问津。

%18
得到这个消息后,中洲蛊仙荣欣当即笑了,当然是冷笑。

%19
他和其他两方交流道:“很有意思啊,这人真的要和我们打价格战了。”

%20
王明月微笑:“这么多年来,还没有其他人能够撼动我们三方的地位。这一次,倒要看看此人成色。”

%21
谢宝树最后附和道:“那我们也跟着降价吧。”

%22
没有什么好说的,既然方源再一次降价,其他三大卖方也迅速降低价格,变得再次和方源的年蛊售价别无二致。

%23
之前的犹豫,都已经犹豫过了。既然方源主动“开战”,这三方蛊仙也自觉地实力雄厚,毫不畏惧地跟进。

%24
第二次降价了。

%25
宝黄天中掀起一阵小小的热议。

%26
其余的蛊仙们,变得渐渐兴奋起来,因为他们都知道,这降价才第二次,只能算是这场商战的开始,接下来肯定还有第三次,第四次降价。

%27
“只是,这就要看那神秘的第四方,选择什么时候降价了。”

%28
“任何的降价,对于卖方而言,都相当于在自己的身上割肉。之前那第四人算是挑起了战争,三大卖方毫不畏惧,你要玩就陪你玩,气概十足啊。”

%29
“不错。接下来很可能会有第三次降价,就是不知道什么时候。”

%30
观战的蛊仙们纷纷交流。

%31
一般而言,这种商战中,降价是很常见的,但一次次降价之间,还是有着时间间隔。

%32
因为卖方需要观望市场,需要时间充分思考和定夺。

%33
降价是一柄双刃剑,虽然能打击竞争对手,但同时也伤害自己的利益。

%34
如果说,自家贩卖的景象还不错,那么或许就不会再降价了。毕竟要保存自身利益,若是再降价,就算是多销,也未必能赚得多。毕竟竞争对手也会降价。

%35
如果卖方都不退让,你降价我就降价,你降到底我也降到底,一味死磕,那么很可能就同归于尽了。

%36
但对于蛊仙而言,这种卖方死磕到同归于尽的情况,几乎没有。

%37
卖方当然不会是傻子,卖东西就是要获取利润,若是相互死磕,自己都讨不了什么好处,还卖东西做什么?

%38
卖方精明,买方也很精明。

%39
三大卖方降价,和方源平价后,他们并不多买,年蛊买卖会变得非常“惨淡”。因为这场商战,最关键的点,就是方源。

%40
而当方源主动降价后,他们会买一点。

%41
为什么呢?

%42
首先,价格方面,方源更低,当然是更实惠。然后,也是鼓励方源,施压三大卖方,继续降价。最后,方源总会支撑不下去。因为主动降价的一方,自然会吃亏一些。但价格降到底后,买方无疑是最占便宜的。

%43
若是四方价格最终一样,他们大多数人也不会选择购买方源这边的年蛊。

%44
商战就是这样的一种心理博弈。

%45
第二次降价,四方再次平价。方源的举动再一次出人意料,他积极关注宝黄天,三大卖方降价的时候,他旋即就做出应对,再一次降价。

%46
“哦?!”许多人为此惊异,方源的反应,未免太迅速了。

%47
而在这种迅速中,许多有心人都感觉到了方源的一种强硬的态度。

%48
“对方来势汹汹啊。”荣欣感慨,他感到了一丝压力,因为在这所有的卖方当中,他恐怕就是存货最少的一人。

%49
这种商战打到后期,往往货物价格降低很多,会引爆购买的热潮。这个时候,如果货物量少,那就尴尬了。

%50
别人想买你的,但是你没货?

%51
提前出局不说,也会让买家觉得你实力薄弱,多年积累的声望就要大大受损了。

%52
荣欣当然不愿意这种情况发生。

%53
他一面催促他人,拼尽全力炼制年蛊,补充货源,另一面则再想,自己是不是要降价一次。

%54
更具体的说,降价幅度大一些,大到让其他人望而却步,让其他竞争对手不敢跟进,因为这样降价,就太过于吃亏了。

%55
但幅度也不会太大,至少让买家觉得,还可以再观望一下。

%56
如此一来,只要争取到充裕的时间,荣欣就能补充货源,在最后的商战时期争取到胜利。

%57
“不过,如此一来,我主动大幅度降价,其他两方恐怕会猜到我货量不多的底细啊。”荣欣很犹豫。

%58
因为他既然能想到此招,来帮助自己度过难关。其他两方,也都经验丰富,也就有很大的可能,来逆推到自己这边的处境。

%59
毕竟,推算这个东西,并非什么难事。

%60
只要是经验丰富之人,都会有这种敏感。

%61
如此一来,荣欣要这么做的话,那就是自曝其短,很可能导致自己提前出局。

%62
所以荣欣不仅是犹豫,而且是头疼。

%63
他感到很头疼。

%64
因为他不仅要考虑到买卖双方的心理,而且还要重点关注这些竞争对手。

%65
就在他犹豫不决的时候,谢宝树、王明月两方又降价了。

%66
降低价格的结果,和之前一样,仍旧是保持和方源售价齐平。

%67
他们两个人货物都很充足,尤其是王明月,所以他们不急不慌。方源既然想要开战,想要降价,那他们就奉陪到底。反正只要是同一价位,买家更愿意买他们的货,这个优势是很大的。

%68
这是第三次平价。

%69
得到这个消息,方源哈哈大笑一声,眼眸中充斥寒光:“既然如此,那我就稍稍加大一些步伐吧。”

%70
于是,他再一次降价。而这一次降价,则降低幅度很大,远超之前的规模。

%71
这在宝黄天中,引发了一场轰动!

%72
买家们非常兴奋,这都是他们克制的结果,也是他们胜利的成果。

%73
“一下子降价这么多?!”谢宝树面色凝重起来,他感到方源的咄咄逼人。如果说前几次只是几下轻拳,那么这一次,就是一脚重踹而来。

%74
但谢宝树不得不接招。

%75
他若不接招,那么就出局。、

%76
购买者毫无忠诚度可言,商战是很残酷的。

%77
谢宝树率先降价,继续跟进,随后王明月也降价,双方再次平价。

%78
荣欣头疼无比!

%79
他自己刚想要大幅度降价,犹豫了一回,没想到自己还没出手,对方倒率先出手了。

%80
这样一来,他之前的打算就落空了不说,就算还要再继续施行原本的计划,那就要让荣欣降低更多价格。

%81
可是这样一来,这价格就很危险。因为远远低于年蛊正常售价的话,一定会爆发买家的收购热潮的。

%82
“或许,那个神秘卖家的货也不多?他和我是一样的打算?”荣欣心中残留着一丝希望。

%83
他最后跟进,降下价格,和其他三方等同。

%84
“都降价了。哈哈哈。”买家们自然开心,但让他们更开心的,还在后头。

%85
当荣欣一降价,方源紧接着动手,再一次降低自己的年蛊价格。

%86
而这一次降价幅度,比前一次还要巨大。

%87
宝黄天震动!

%88
三大卖方惊诧!

%89
“这个家伙,不得了……”谢宝树脸上充斥着凝重之色,他原本云淡风轻,推崇喜怒不形于色的行事风格,但此时此刻他掩盖不住了。

%90
方源这一次降价,极大地缩减了年蛊贸易利润,对于三大卖方而言,若是他们跟进,无一不伤筋动骨!

%91
“这是要血拼的节奏啊。”王明月咋舌不已,她感到了一股久违的寒意。

%92
这是她许久以来,都未感受到的了。

\end{this_body}

