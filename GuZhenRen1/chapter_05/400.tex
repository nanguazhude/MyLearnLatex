\newsection{影宗群仙战九歌}    %第四百节:影宗群仙战九歌

\begin{this_body}

%1
凤九歌一身红白相间的长袍,玉树临风,此时望着影宗群仙,温和地笑着,更显得风度翩翩。

%2
他浑身毫无战意或者杀气,但影宗众人,却无不感到巨大压力。

%3
“凤九歌……”方源越众而出,口中呢喃。

%4
他风姿卓绝,白袍如雪,面冠如玉,双眼如点漆,深不可测。

%5
看到这样一幕,白兔姑娘的心头猛地一动,暗道:“这当今世上,八转之下最杰出的两位,终于要在今天决出个高下了!”

%6
说起来,方源和凤九歌虽然交流的时间不长,但其实彼此之间的缘分,却是绵延已久。

%7
早在几年前,方源抢夺了狐仙福地,夺取了凤金煌的机缘后,他就一举跃入了凤九歌的视野当中。

%8
不过那个时候,凤九歌是高高在上的七转蛊仙,成名已久,巅峰人物。而方源不过是个凡人,蝼蚁一样,就算是有点出彩,也只是如此而已。

%9
谁能想到在短短的几年后,方源不仅成就了蛊仙,而且还是春秋蝉之主,捣毁八十八角真阳楼,先后参加义天山大战、逆流河大战、梦境大战。如今风头无两,天庭通缉榜上有名,在名望上已然是和凤九歌齐名!

%10
方源救过凤九歌,凤九歌也救过方源,一切的恩怨都已经了结。

%11
现在两个人,将以最纯粹的阵营身份,进行一场真正实力的比拼。

%12
“凤九歌,你千里迢迢来追杀我等,岂不知同时也将你的性命,送到我的手中了?”方源冷笑一声,身形电射而出,直接扑向凤九歌。

%13
说话间,他身上浮现出仙衣绶带,黑发激扬,眼眸射电,悍然动手,拉开这场激战的大幕。

%14
仙道杀招——逆流护身印!

%15
凤九歌眼眸一凝。

%16
他早已经见识到,这个名震天下的杀招威能。

%17
他也多少猜测到,逆流护身印这招恐怕是比较复杂,催动困难,需要酝酿的时间也比较长。自己若是偷袭,不给方源催动出这一招的时机,战斗起来必定占据上风。

%18
可惜的是,方源一直都没有给凤九歌这样的机会。

%19
并且,影宗群仙的侦查杀招都非常优异,让凤九歌始终找不到偷袭的可能性。

%20
眼看着方源扑杀而来,凤九歌嘴角的笑意,却是扩散开来。

%21
他自顾自地道:“我能直接腾挪过来,乃是用了自创的仙道杀招,一曲阳关。更关键的是,我这一曲阳关还未奏完哦。”

%22
话音刚落,他整个人猛地散发出金光。

%23
金光没有任何的杀伤威能,但金光一闪即逝,旋即消散。

%24
而凤九歌整个人,也彻底消失在原地。

%25
“什么?”几乎是下一刻,方源就听到一阵古筝弹奏似的激越声响。

%26
声响忽然变大,又旋即减弱,像是弹奏的人,从他的耳畔飞离过去。

%27
方源心神一震,猛地想起来。这正是凤九歌的移动手段,曾经在五域乱战中大放光彩。

%28
它能让凤九歌化身为无形无质的声音,传播出去,然后再凝聚出身形来。音道流派其实并不主流,这一招更显得偏僻奇妙,寻常方法基本上都无法遏制凤九歌的转移。

%29
“小心!!”黑楼兰高声示警。

%30
她们当然也听到了古筝的弹奏声音,一个个都下意识爆退。

%31
但她们的速度比不上一曲阳关,并且不知晓凤九歌的具体方位,撤退的方向也只是随意选择。

%32
白兔姑娘就忽然听到,耳畔古筝声猛地一炸,旋即凤九歌的身影凝聚而出,就在她的左近。

%33
锵!

%34
凤九歌带着有些奇异的目光,对着白兔姑娘遥遥一指,发出好似兵器交击互撞的激响。

%35
白兔姑娘发出尖锐的惊叫身,在刹那间,她感受到了强烈的死亡威胁。

%36
她连忙后退。

%37
尖锐的音波,正中她的胸膛。

%38
下一刻,鲜血飙飞,白兔姑娘浑身涌起汹涌的黑雾。

%39
黑雾中,黑菟出现,脚踩着一头壁虎,猛地飞上了高空去。

%40
六转修为的白兔姑娘,不是凤九歌的对手。但关键时刻,她的另外一个身份黑菟救驾出现,挽回了自己的生命。

%41
尽管如此,黑菟也是重伤,连忙脱离,试图疗伤。

%42
凤九歌却也不追击。

%43
他知晓自己此次突袭失败,已经丧失了先机,让所有人都警觉非凡,再没有这第一击的效果了。

%44
关键是……

%45
凤九歌微微垂眉,看向脚下的广袤戈壁:“这里是?”

%46
“绝音戈壁,你的葬身之地!”说着这话,方源再次扑来。

%47
“原来如此。”凤九歌恍然大悟,下一刻,消散成音波,一曲阳关再次建功,让方源扑空。

%48
然而,再次在不远处闪现出来的凤九歌,脸色却不那么好看。

%49
嘴角的微笑已经削减不少,目光中更是带着一丝凝重。

%50
西漠的绝音戈壁,是和南疆的寂静岭齐名的地方。它们早已闻名天下,对于音道蛊仙而言,都是绝境禁地。

%51
因为在这两个地方,音道道痕被压制得很惨,任何的音道杀招威力都会被抵制很多。

%52
之前,凤九歌是想一举,腾挪到白兔姑娘的背后,施以斩杀。

%53
结果,却距离白兔姑娘足有百步,就已经显露了身形。

%54
这才让白兔姑娘逃脱了致命杀机。

%55
“绝音戈壁,当初这块奇地,可是被盗天魔尊亲自偷走的。没想到是落到了这里,方源是怎么知晓的?呵,倒是差点忘记了,他是影宗新主,知道一些秘闻情报,也是应当。”凤九歌在心中猜测。

%56
他其次猜测错了,这块地方影宗也不清楚。

%57
方源之所以知晓,是因为在他五百年前世时,五域乱战,这个地方就暴露了出来。现在被他利用。

%58
“呃?”凤九歌忽然面色一变,惊异地看向自己的手臂。

%59
不知何时起,他的手臂上,乃至后背,凝结了一层冰霜。

%60
“是那个龙女白凝冰?”凤九歌望向远处的白凝冰,眼中闪过一抹赞叹之色。

%61
仙道杀招——冷眼!

%62
这正是白凝冰施展出来的手段。

%63
凤九歌虽然腾挪很快,但是冷眼杀招发动的更快,只要在白凝冰的视野范围之内,几乎是同时可至。

%64
凤九歌朗笑一声,伸出手掌,在冰霜上轻轻一抚。

%65
顿时,一阵悠扬的钟声响起,冰霜旋即消融。

%66
白凝冰冷哼一声,一双龙瞳冷光暴涨。

%67
“有点麻烦。”凤九歌微微皱起眉头,再次运用一曲阳关,消失在原地。

%68
但下一刻,他出现时,不仅是要面对白凝冰的冷眼,而且还召来了妙音仙子的勾月。

%69
仙道杀招——勾月!

%70
妙音仙子的眼眸中,浮现出来两个弯弯的弦月。

%71
和冷眼一样,这同样是目击之术,发动得极其迅猛。

%72
凤九歌中招,身躯微微一震。

%73
妙音仙子脸上动容,她很自信勾月的威能,但打在凤九歌的身上,只是让他身躯一震罢了。

%74
凤九歌的防御,当然底蕴雄厚。

%75
当初就算是面对武庸的仙道战场杀招,他也能硬抗那么久。硬吃一记勾月杀招,不算什么。

%76
吱吱喳喳!

%77
忽然,传来鸟鸣的声音。

%78
一群火焰小鸟,围拢向凤九歌。

%79
轰轰轰轰!

%80
下一刻,火焰小鸟纷纷自爆,炸响声不绝于耳,但又很快消失。

%81
因为这是在绝音戈壁的战场上。

%82
爆炸将凤九歌覆盖,炙热的火焰随着澎湃的气浪,向四周喷涌。

%83
一时间,方源等人的脸上都被映上了一层火光。

%84
空气中的温度,旋即暴涨。

%85
黑楼兰站在远处,目光中绽射着强烈的战意。这正是她催发出来的仙道杀招——愤怒的小鸟。

%86
远处,凤九歌再次显露身形。

%87
他毫发无损,愤怒的小鸟自爆,并未殃及到他。

%88
方源旋即扑上,其余影宗群仙的攻势从四面八方打来。

%89
凤九歌无声一笑:“躲闪的够久了,来吃我一招吧。”

%90
仙道杀招——天地歌!

%91
一瞬间,天地高歌。

%92
歌声传入群仙的耳中,他们身躯猛震,感觉整个天地像是陡然间变得天高地阔,而他们自己却是越变越小,变成蝼蚁,变成微尘。

%93
渺小、卑微的感觉,油然而生。

%94
和整个天地相比,自己微不足道,又算得了什么呢?

%95
这正是天地歌的威能!

%96
不仅是影响蛊仙,就连他们的仙道杀招也是同样如此,在天地歌中被镇压住,威力越来越小,几个呼吸之后,消散于无形当中。

%97
这一招,正是凤九歌的招牌杀招之一,当初在落魄谷中激战,秦百胜一方都吃过此招苦头。

%98
现在是轮到方源等人品味了。

%99
黑楼兰等人自然不好受,但方源却是仍旧一往无前!

%100
逆流护身印岿然不动,只是表面上浮现点点涟漪波纹,不仅如此,它还将天地歌的部分威能,反击到凤九歌的身上。

%101
凤九歌首次品尝到自己杀招的滋味!

%102
望着横中直撞而来的方源,他眉头深深皱起,不得不运用一曲阳关,再度挪移躲避。

%103
逆流护身印让天庭蛊仙、长生天、紫山真君都没有办法,武庸也束手无策,更何况是他凤九歌?

%104
凤九歌一看到方源就头疼。

%105
他总算是深切地体会到了,武庸当初的感受了。

%106
“方源放一放,先将他的这些羽翼都给铲除了!”凤九歌不想和方源纠缠,将目标打到其他人的身上。

\end{this_body}

