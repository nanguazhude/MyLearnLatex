\newsection{正反狙神针}    %第五百四十九节:正反狙神针

\begin{this_body}

“线索到达这里,就断了!”蒙家的仙蛊屋问鼎院倏地悬停于空,驻足不前。

蒙家的两位蛊仙面面相觑。

蒙自在望着前方,便见一片山谷,郁郁葱葱,和周围草原形成鲜明对照。

按照自然规律,不会有这样的山谷成形,不过蒙家两位蛊仙却都不意外,蒙家太上二长老用低沉的声音道:“这是睡姑的神针谷啊。”

蒙自在皱起眉头:“难道是她?”

蒙家太上二长老沉吟不语。

睡姑乃是七转散修,专修音道,一身战力并不突出,但是她却掌控着一头太古荒兽刺神猬。这头刺神猬乃是剑道太古荒兽,有一项能力,浑身尖刺暴射而出,让人来不及躲闪,能屠神杀仙。

睡姑营造出神针谷,生活在蒙家、慕容家两大势力的中间地带,成为一方霸主。

她并非只是手段高明,能掌控太古荒兽,政治智慧也非常突出。她在两大超级势力之间生存,两方借力,让任何一方都不敢去真正对付她。因为这样做的话,随时都有可能令睡姑,倒戈向另一方超级势力。

蒙家两位蛊仙沉默了一下。

这事情有些难办。

他们是追踪线索而来,但没有真正得到什么关于方源的确切信息。

睡姑的本身战力,虽然比不上蒙屠,但她若是派遣了刺神猬的话,也是有着斩杀蒙屠的可能。

照此看来,她的确是有嫌疑的。但是……

此事处理起来会很麻烦。

长久以来,蒙家和慕容家都在示好这位睡姑,和她关系都颇为不错,笼络她,防止她倒向另一方。

蒙家和慕容家族的关系,显然不会好到哪里去。

任何一个超级势力,都有扩张的本能,蒙家、慕容家挨得这么近,关系怎么可能好?

若睡姑就是凶手,那么单她一人怎可能有这样的胆量?因此极可能她已经倒戈向了慕容家,而慕容家身为超级势力,正道的黄金部族,不可能凭白无故地对另一正道超级势力下杀手,把睡姑当做刀枪来使,也很合情合理啊。

若睡姑不是凶手,那蒙家的两位蛊仙又该怎样探察?一个不小心,恐怕就要让睡姑泛起其他心思,倒向慕容家,那蒙家的损失无疑就更大了。

两位蒙家蛊仙思考了半天,蒙家太上二长老终于开口:“走吧,我们一起去造访睡姑,将蒙屠之死直接告诉她。只有这样,坦诚布公,才能让睡姑明白我蒙家的动机和决意。而且毕竟线索就在她的神针谷戛然而止。”

蒙屠点点头。

问鼎院并未遮掩踪迹,蒙家两位蛊仙的到来,早已惊动了睡姑。

不过,睡姑不动声色,自有资历和底气,以不变应万变,耐心等候两位蒙家蛊仙主动登门。

蒙家二仙登门后,立即将事情原委述说一番,睡姑这才了解情况,顿时知道这事态非常的严重!

首先,她连忙表明自己的清白,当场和蒙家蛊仙证明。其次,对线索在神针谷这边消失,她自己也有焦虑。

“这神秘凶手能杀得了蒙屠,手段定然是高超,战力恐怖。还请二位仙友随我一同,搜索整座神针谷,帮我这个忙。”

睡姑主动要求道。

蒙家二仙大喜,这正是他们想要要求的事情,立即点头应承下来。

三仙联袂而行,针对神针谷搜寻得非常仔细。

睡姑也害怕呀。

问鼎院的威能,她多少有些了解,既然这线索在神针谷消失,极有可能凶手就隐藏在自家谷中。这神秘凶手既然能对蒙屠下手,更穷凶极恶地将其杀害,那么为什么就不能来对付她呢?

三仙搜寻了好一会儿,都没有什么成果收获。

蒙家二仙对视一眼,蒙家太上二长老故意说道:“神针谷几乎都搜遍了,难道凶手真的不在这里?”

蒙自在状似无意,口中接道:“不是还有一处地方没有搜吗?”

睡姑心知他俩在演戏,不过她并非凶手,真心是想要仔细搜寻一下,便点头道:“这神针谷乃是我专为刺神猬所设,如今就剩下谷底深处没有搜了,还请二位随我来。”

蒙家蛊仙便随同睡姑,来到谷底。

只见幽暗的谷底中,栖息着一头洁白如雪的巨大刺猬。它趴在地上,缩成一团,显露出沉眠之中,呼吸间,栖息悠长。周身尖刺,锋锐至极,充斥剑道道痕。

刺神猬大如小山,蒙家两位蛊仙缓缓地飞到它的身边,仿佛是蚂蚁面对车轮。

“好一头神骏的太古荒兽!”蒙家太上二长老轻声赞道。

睡姑面色柔和了几分,这正是她能周旋在两大超级势力之间,最重要的本钱。

蒙家太上二长老仔细观察片刻,又道:“看来,这一次的正反狙神针质地非常优质啊。睡姑你何时采摘呢?”

刺神猬浑身的尖刺,是可以拔下来的八转仙材。每一根尖刺,就是一根正反狙神针。

刺神猬每遇到危险时,就会暴射出身上的尖刺杀敌。这些射出去的尖刺,自然不会返回。刺神猬沉睡的时候,身上就会重新长出尖刺来。

睡姑手段独特,能够令刺神猬时刻沉眠入睡。再喂饱它之后,睡姑就有了一项经济支柱,就是专门贩卖正反狙神针。

她培养出来的正反狙神针,大部分都分给了蒙家、慕容家,剩下一小部分则在宝黄天抛售。

睡姑因此手头宽裕。北原这块地方,自然资源相对贫瘠,但睡姑却依靠这头刺神猬,活得远比大多数的北原蛊仙滋润。

“就在最近采摘吧,这一次会有四百三十六根。”睡姑正色道。

刺神猬身上的尖刺,超越一千之数,但睡姑还要留着那些低矮的尖刺,继续成长。采摘收获的都是长成到最长的尖刺。同时,她也不可能将所有尖刺都卖掉,因为失去了这些尖刺的保护,刺神猬的战力就会下滑得相当严重。

“我们开始吧。”蒙自在提议道。

“请二位尽管施为,我来令刺神猬继续沉眠。”睡姑道。

蒙家两位蛊仙要搜寻线索,催动问鼎院,动静颇大。但睡姑的音道杀招,始终令刺神猬沉眠。

刺神猬若是从睡眠中被惊醒,一定会暴怒,迁怒他人,胡乱攻击。睡姑并没有完全折服此兽掌控住它,毕竟她不是奴道蛊仙。

“发现了新的线索!”搜寻良久,蒙自在忽然开口,神情振奋。

睡姑悚然:“这人居然潜入到谷底,我竟是完全无法发觉?”

沿着痕迹,三仙继续追寻下去,发现痕迹一路向下,深入地底。

三仙顺藤摸瓜,不知不觉深入地下深处,线索也变得越来越多。

不久后,痕迹又转折方向,向上蜿蜒。

三仙顺着痕迹,小心翼翼行进了片刻,又来到地面上。

“这里已经是慕容家的势力范围,距离神针谷有数百里之遥。”蒙家太上二长老皱起眉头。

线索到达这里,彻底消失。

蒙自在目光闪烁,睡姑沉默,打死也不在这个微妙的关头发表声明意见。

三仙随后又梭巡良久,没有发现任何线索。

睡姑提议,到她谷中小聚休息,蒙家二仙苦无线索,想了想,应承下来。

哪知回到谷中,三仙有了一个震惊的发现。

原本神骏异常的刺神猬,仍旧在呼呼大睡,但却是浑身光溜溜一片,所有的正反狙神针,都消失不见了!

“怎么回事?我的狙神针!这可是我耗费上百年,守候在神针谷足不出户,耗费大量食材喂饱刺神猬,又不惜屡屡催动杀招,哄它入睡,才一步步培养出来的规模啊!”睡姑大惊失色,心痛到失声惊呼。

蒙家二仙面面相觑,冷汗从他们额头垂下,因为他们猛地发现,自己似乎被人利用了!

\end{this_body}

