\newsection{落英馆战豆神宫(上)}    %第四百九十八节:落英馆战豆神宫(上)

\begin{this_body}



%1
见到方源居然在自己的合围埋伏中脱身,还进入了落英馆,传奇太古魂兽青仇顿时气得怒吼连连,一双眼睛充斥血丝,杀意沸腾到了极点,几乎要择人而噬。

%2
它对方源的仇恨,已经不是后天,而是先天。

%3
原本它的形成,就是青家蛊仙的残魂碎片,以及对幽魂深入骨髓的滔天仇恨,复仇之意。

%4
因此即便它有着人的理智,但是遇到方源,又斩杀不掉的时候,心中的复仇之情完全占据了上风。

%5
豆神宫中,青芒闪烁,向青仇镇压过来。青仇不管不顾,仍旧狂催豆神宫,向落英馆撞去。

%6
“主人,我快不行了……”败军老鬼传来惨然的呼救之声。

%7
“给我杀,杀不死他,你们就都死吧!”青仇咆哮。

%8
败军老鬼、鹰姬纷纷脸色剧变,他们还从未见过,主人青仇如此动怒,如此仇恨一个人!

%9
这滔天的仇恨之意,几乎扑面而来,让他们不寒而栗,下意识就想逃避。

%10
但奈何他俩受制于青仇,已经无法反抗,当下只得咬牙,前者催使太古魂兽,后者则酝酿仙道杀招,齐齐杀向落英馆。

%11
方源一进入落英馆,青仇一方就只剩下这个目标,自然要攻击它。

%12
敌方来势汹汹,房安蕾驾驭着落英馆,飞速避退。

%13
这座七转仙蛊屋,比方源的飞行速度还要快上一些,但比不过上极天鹰。

%14
太古魂兽双翅黑蟒被渐渐甩远,飞蜘蛛太古魂兽却是速度极快,钻向落英馆的屋顶。

%15
与此同时,远处鹰姬轻喝一声,双手鹰爪疯狂挥舞,刷刷刷,无数钢铁般的爪痕,立即爆散而出,宛若疾风暴雨,向落英馆笼罩过去。

%16
落英馆此时已经无法躲闪,但房安蕾也不慌张,早已经严阵以待。

%17
她对房棱、房云轻喝:“你们操纵落英馆飞行,我来驱使杀招抗敌。”

%18
说着,落英馆便在她的催动之下,爆散出洁白如雪的光辉。

%19
落英馆周围表面,原本有无数繁杂的鲜花绽放,宛若花屋,此刻纷纷凋零。

%20
雪光收敛起来,逐渐在落英馆的屋顶,形成了一朵巨大的花朵。

%21
这花朵比屋顶还要大上三分,片片花瓣犹如镜片,折射着光泽。一声铃铛般的脆响声中,花朵彻底绽放,美轮美奂。

%22
仙道杀招——镜花!

%23
那袭击而来的狂暴爪痕,刚刚要轰击到落英馆上,房顶上的镜花陡然爆发出一阵强烈的吸摄之力。

%24
叮叮咚咚!

%25
一连串的脆响声中,这些漆黑爪痕竟都被镜花吸摄进去。

%26
吞吸了全部爪痕之后,镜花花瓣凋零破碎了许多片,但剩下的十几片则是缓缓收拢,又形成了花骨朵。

%27
就在这时,豆神宫逼近。

%28
镜花残破的花骨朵再次绽放,将全部的爪痕重新喷射出来。

%29
爪痕打在豆神宫上,顿时激起无数璀璨的青铜火花。

%30
豆神宫冲势受阻,落英馆借着喷射爪痕的这股反推力量,又速度激增,和豆神宫拉开距离。

%31
“不好,那只太古魂兽飞蜘蛛就要钻进来了!”这时,房云猛地大叫一声。

%32
落英馆虽然暂时击退了鹰姬、豆神宫的轰击,但那只飞蜘蛛,却是占据了体型微小的便宜,附着在了落英馆的一处墙角,使劲钻来。

%33
落英馆到底是仙蛊屋,攻防一体,飞蜘蛛现在连表层还未钻破,更距离核心还要很远距离。

%34
但事实上局势却是非常危险。

%35
因为一旦钻破表面,飞蜘蛛就能进入内里。

%36
仙蛊屋乃是由无数蛊虫组合起来,让一头太古魂兽进入其中,必定会大肆破坏。

%37
蛊虫脆弱不堪,到那时候结果不堪想象。

%38
落英馆虽然有快速回复的能力,但这种回复速度万万比不上一头太古魂兽的破坏速度。

%39
危机关头,房安蕾再次出手。

%40
仙道杀招——昙花一现!

%41
那屋顶的镜花已然凋零败落,却在原来的位置上,又生出一朵昙花。

%42
昙花不大,显得娇柔,忽然一现,就化为点点光影,随风飘散。

%43
但一股玄妙力量,已然是缠绕在了飞蜘蛛的身上。它钻透的速度,骤然放缓了许多倍。

%44
“这座木道仙蛊屋落英馆当真不俗!难怪前世五百年,能够在五域乱战中大放光彩。”

%45
“先前镜花乃是木道杀招,却推出律道良效,反射攻击。现在的昙花,也是木道杀招,则演绎出了宙道的精彩!”

%46
方源在屋内观战,目光连连闪烁。

%47
房安蕾先用镜花,暂退豆神宫、鹰姬,又用昙花,延缓飞蜘蛛的危机。此刻脸色发白,神情都有些恍惚。显然是连续催动这两记杀招,付出的绝非仅仅是仙元!

%48
这两记杀招,威力真的非常不俗,虽然只是七转层级,但在落英馆发出,却有了一丝八转风采。尤其是那飞蜘蛛,可是太古魂兽。

%49
只是威力越大,催动杀招的代价也就越大。方源目光如炬,轻轻一扫,就感觉到房安蕾似乎减损了自身寿元!

%50
“我这落英馆已经是七转仙蛊屋中的极品,但是同时对付两头太古魂兽、豆神宫还有两位七转魔道蛊仙,实在是太勉强了!眼下拼尽全力,也只是延缓了杀机而已。”

%51
房安蕾心头乱跳,她不由地看向一旁的方源,心想:“这人果然手段非凡,他之前面临的压力,比我还要巨大,并且千钧一发,生死存亡关头,居然能闪电般地思考出逃脱之法。眼下局面,还得要依赖此人,否则我连片刻时间都撑不过去。”

%52
念及于此,房安蕾便催动手段,传给方源一点消息。

%53
她此刻忙于周旋,根本没有时间说话,和方源好好交谈,只能如此沟通。

%54
方源接过这点消息,一看便知是房安蕾要和方源正式联盟,共同对抗外面的强敌!

%55
房安蕾在内容中,陈述利弊,眼前情景非得双方精诚合作,才能拖延杀机,等到房家的支援赶到。

%56
房安蕾觉得方源没有逃脱的能力,但实际上方源不仅有,而且只要催动逆流护身印,还有一战之威。

%57
不过此时,方源既然选择留下来,无非是想和房家接触,建立关系,方便他将来的战略安排。

%58
“我能逃脱,房安蕾却是不能。因此这盟约条件优越,这么一来,我还得要谢谢这埋伏的强敌了。若非是他们,我和房家绝不会如此轻易结盟,并且条件还如此宽松。”

%59
原本,房家抛出一个关于无常石的合作内容,现在盟约仍旧包含这一点。

%60
虽然已经发现了豆神宫,无常石的合作不过只是幌子,但这也正是房安蕾的高明之处,仍旧保留了这项。

%61
除了六转仙蛊的报酬之外,房安蕾还允诺方源不少仙材,当做酬劳。

%62
“六转仙蛊已对我不太合用,但我此时若提出七转仙蛊,便有要挟之嫌了。而且仙蛊唯一,房安蕾也未必有代表房家,答应我的资格。也罢!此事已经是良机,没必要继续贪心,过犹不及。”

%63
方源想到自家战略,当即心思一定,答应房安蕾道:“此盟我结了!”

%64
他思考问题,念头喷涌,非常迅猛,实际上时间只是一瞬。

%65
房安蕾大喜,她之前还有点担心方源的自负和疑心,见到方源几乎一瞬间就做了决定,心中不免暗赞:这个算不尽不愧是智道能人,当断则断,识时务,英明果决。

%66
房家这次接触方源,自然早就备好了仙蛊和手段,双方当即定下了盟约。

%67
但此时,落英馆已经被重重包围。

%68
轰!

%69
豆神宫狠狠撞上来,落英馆被其他强敌堵住去路,躲闪不及,如同流星般砸落下去。

%70
“杀!杀!杀!”豆神宫中,青仇咆哮,复仇的快感冲击全身。它周围的青芒已经凝如实质,如根根利箭插在它的身上,但它丝毫未觉。

\end{this_body}

