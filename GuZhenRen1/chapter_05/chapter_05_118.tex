\newsection{身陷绝境}    %第一百一十八节:身陷绝境

\begin{this_body}

%1
“时间还是太短,不足以将过往来动演练纯熟。”方源叹息一声。

%2
若单独只催动过往来动这招,成功的可能十之八九。

%3
但落实在实战中,就不一样了。

%4
譬如方源,要时刻催动血染征袍、见面曾相识、三息后现等等这些仙道杀招,这些杀招涵盖了大量的蛊虫,牵扯了庞大的心力。

%5
在这种基础上,催动“过往来动”这种还演练不纯熟的仙道杀招,能够在第三次才运用失败,已经说明方源的运气足够的好了。

%6
仙道杀招的效果虽然上佳,但是心神消耗太多。更何况生死激战之中,还要时刻观察对手,考虑战术,及时做出反应等等。

%7
所以,仙道杀招在战斗中,也不是催动越多越好的。

%8
万一因为心神被牵扯太多,对敌人的突袭应对不正确,或者反应不及时,导致落败身亡。那就成了一个笑话了。

%9
事实上,这种笑话在蛊仙历史上一直在层出不穷。

%10
一些极其强大的杀招,因为催动失败,直接导致蛊仙当场身亡的例子,也有不少。

%11
“要解决这个难题,就能将我的战斗力再提升一筹!”

%12
“除了多加练习,以及改良仙道杀招,让它们更加精简之外,那就是从智道方面解决。”

%13
说到底,仙道杀招是无数蛊虫的搭配。

%14
调动每一个蛊虫,至少要耗费蛊仙的一个念头。成功催动一个仙道杀招,蛊仙消耗的念头往往成千上万!

%15
智道正是专门解决这个难题的流派。

%16
智道蛊仙往往能较其他流派的蛊仙,运用出数量更多,或者更复杂的仙道杀招来。

%17
“我得到的智道传承,最主要的还是东方长凡的智星真传。它长于推算,并不擅长指挥蛊虫,催成杀招。不过我现在乃是智道宗师,完全可以凭此基础,往这个方向上发展。以期做出突破。”

%18
方源打跑了上古鹰犬之后,在回转的过程中,自我反省,敲定了智道方面的下一步发展方向。

%19
落到地上。方源一拍小腹,开启仙窍门扉。

%20
门户大开,露出至尊仙窍中的部分光景。

%21
毛六直勾勾地看着,但光从门外看,哪里能看出什么东西来?

%22
他只看到了一片冰天雪地。荒芜一片的景象。

%23
那是至尊福地中的小北原。

%24
毛六这才微微松了一口气。

%25
“看来他是着重提升战力,仙窍经营方面就落下了。也难怪,毕竟这至尊仙窍可是两个月就渡劫一次呢。”

%26
毛六根本不知道方源已经全面掠夺了黑凡洞天。他以为方源还是个穷小子,殊不知在至尊福地的其他地方,一片山清水秀,林木森森,资源丰富得吓人。

%27
“尽管如此,这可是货真价实的七转战力。”毛六望着鸡年兽,眼角抽搐了一下,“方源虽然只是六转。但已经有七转占领。再添加一个七转战力……此事我定要汇报给影无邪大人,让他知晓!”

%28
鸡年兽喔喔地叫着,十分愉快。

%29
方源喂给了它大量的年蛊,让它十分满意。

%30
最后,它双翼一扇,跳入光阴长河,身形徐徐消失不见。

%31
将鹰犬一个个塞入仙窍里(落星犬也包括在内),又顺利打发了鸡年兽后,方源转过身来,面对在场的毛民蛊仙。

%32
“这一次任务。应当是算我做成的吧?”方源淡淡笑着,以征询的口吻询问这些毛民。

%33
毛民蛊仙们点点头,脸上都或多或少有些不情愿,无奈还有尴尬。

%34
事实如此。

%35
没有方源出手。他们甚至连性命都有危险。

%36
毛十二张口,想说话,但嘴唇翕动了几下,始终没有说出什么话来。

%37
他想到自己之前的表现,简直是浅薄至极,不知天高地厚。心中不免生出惭愧之情。

%38
毛六察言观色,这时阴笑一声,开口道:“方源长老你真是厉害,居然一个人直接打跑了一群鹰犬!如果我们不是亲眼所见,根本不会相信的。要是你能早些出手,什么落星犬根本不在话下吧。”

%39
毛民蛊仙们听到此话,脸色微微变化,心情更加复杂。

%40
“我亦有苦衷,也不想和你解释。其中的原委,太上大长老知晓一二。”方源淡淡地笑了笑,环视众人一眼之后,看向毛十二,“好了。事情已经了解,我们回去吧。”

%41
通过传送蛊阵,众仙十分方便地,就回到了琅琊福地。

%42
听到众人的汇报之后,琅琊地灵的态度也缓和许多。

%43
他倒对方源的战力,没有多大惊讶。

%44
因为他知道,方源曾经拥有春秋蝉,是重生之人。六转修为拥有七转战力,这虽然少见得很,但搁到方源身上,并不难理解。

%45
有了这种缓和的态度,还有落星犬任务完成后奖励的上千贡献,方源顺利地借得蛊虫,能够再用仙劫锻窍。

%46
接下来的日子里,方源便安心修行。

%47
主要是利用荡魂山、落魄谷修魂,另一方面,私下里演练过往来动等全新的杀招,闲暇功夫,就稍微照看一下刚刚得来的落星犬幼体,还有那群鹰犬。

%48
对于这些荒兽,还有一头上古鹰犬而言,只是环境换了一下,稍稍适应之后,就安心在这里生存了。

%49
在哪里生活,不是个活?

%50
这就是没有智慧,显得有些没心没肺。

%51
至于那些从黑凡洞天中搬迁过来的无数资源,方源将其大部分,置入小南疆当中。

%52
因为小南疆的环境,比较贴近黑凡洞天。

%53
尽管如此,因为道痕比不上黑凡洞天,所以还是有无数资源减产,更有一些濒临灭亡。

%54
所以,方源还有一个烦心事,就是赶紧将那些养不活的资源,都往外卖。

%55
除了和琅琊地灵交易之外,重点就是在宝黄天。

%56
方源在很短的时间里,囤积了海量的仙元石。用了一部分之后,之前稍显不足的仙元储备。就都不再是隐患,反而成了方源的一项长处。

%57
“这么多的青提仙元!就算催动剑浪三叠这种仙道杀招,持续上百次,也不担心了。”

%58
方源还从未拥有过这么的青提仙元。

%59
这让他对第四次地灾。更具信心。

%60
时间匆匆,终于又到了方源再次渡劫的时候。

%61
他离开琅琊福地,来到北部冰原。

%62
选择一个地点之后,他就坐落仙窍,打开门户。汲取天地之气。

%63
这一次汲取天地之前,持续的时间是之前的数十倍!

%64
原因就在于方源仙窍中的资源猛增了数百上千倍,消耗了仙窍本身大量的天地之气。

%65
之后,方源再布置下仙劫锻窍。

%66
一切驾轻就熟。

%67
准备妥当之后,方源关上福地门户,开始迎接第四次地灾。

%68
仍旧是在小北原。

%69
方源迎来了漫天的飞霜。

%70
“这是……玄白飞盐劫?”

%71
方源楞了一下,辨认出来。

%72
他有些意外。

%73
这个劫并不强大,反而是比较容易渡过的那种。

%74
天空中,结成一长片的淡黄云朵。

%75
大量的盐霜,从高空中悠悠下落。整个过程寂静无声。盐如白雪,一丝风都没有。

%76
大量的飞盐落到地上,将地面上的冰雪融成液体。

%77
冰雪中还生长了一些耐寒的野草野花,被飞盐腌过后,很快生命垂危,不久就彻底死亡。

%78
“这是想破坏我的小北原的生态?”方源有些纳闷。暂时,他没有行动,只是作壁上观。

%79
虽然有些损失,但他大部分的资源,目前都集中在小南疆中。

%80
小北原一直是方源用来渡劫之地。这些损失方源以前就完全能够承受。更别谈现在,他收获了黑凡洞天的积累,亿万身家,财大气粗了。

%81
等了一会儿。一只只雪人在漫天的盐霜之中出现。

%82
这是狂蛮意志的影响。

%83
只要方源斩杀了这些雪人,他就能得到狂蛮真意的灌输,令自己的变化道、力道的境界突飞猛进。

%84
这个时候,可不像第三次时有楚度分羹,方源完全是资源独享。

%85
方源连忙动手。

%86
这些雪人刚刚出现,就被飞盐融化。天意的意图。似乎在此刻彰显出来,那便是尽最大可能削弱方源在地灾中的收益。

%87
方源与天意抢出手。

%88
万我再次催动起来,剿除这些雪人。当然还有其他一些手段。

%89
此时,只能是由方源动手。

%90
其他存在,不管蛊仙还是荒兽,谁杀了雪人,谁就得狂蛮真意。

%91
一股股狂蛮真意,灌输到方源心间,迅速提升着他的流派境界。

%92
至于玄白飞盐本身,乃是劫数,不是方源能直接铲除扼杀的。

%93
第四次地灾,持续的时间很长。就算将前三次的时间加起来,都不及第四次的一成。

%94
三天之后,玄白飞盐的规模才开始有些下降。

%95
“这是要打消耗战么。”方源皱起眉头,心中有些疑虑。

%96
又过数天,玄白飞盐劫终于停止。

%97
方源望着漫地的霜盐,心中的疑虑越加浓重。

%98
按照他之前的推算,第四次地灾应当更强,应对要更加困难。但事实却是,玄白飞盐劫总是不温不火,徐徐落下,方源一直都没有感受到什么压力。

%99
就连雪人,他都铲除了绝大多数,没有浪费多少狂蛮真意,收获颇丰。

%100
方源都有种错觉,似乎天意准备放过他了。

%101
“之前,我在太丘中运用一些宙道杀招,也是为了测验天意的推算能力,看看它如何对付我。结果却是玄白飞盐劫。”

%102
“不管如何,第四次地灾终究是渡过去了。”

%103
“这些盐地也可以保留下来,用我意冲刷之后,完全可以当做一个特殊的环境经营下去。”

%104
这一次地灾,出乎意料的轻松。方源甚至连一丝伤,都没有受着。

%105
方源收起仙窍,就要赶回琅琊福地。

%106
但就在这时!

%107
天地骤变,方源陷入一个战场杀招之中。

%108
近十位蛊仙浮现出身影,更有一头龙形猛兽,气息磅礴浩荡,对方源虎视眈眈。

%109
太古荒兽!

%110
方源的心顿时沉入谷底,他脑海中闪电般明悟过来:“人劫!原来天意是故意拖延我的时间,布置出了这个必杀的局!”

%111
刹那间,方源深陷绝境。

\end{this_body}

