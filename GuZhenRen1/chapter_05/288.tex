\newsection{无赖!可耻!}    %第二百八十八节:无赖!可耻!

\begin{this_body}

方源好似没有察觉一样,笑眯眯地道:“首先是第一句话,我就没有听清。”

夏飞快便复述一遍。

方源点点头:“然后是第二句话……”

夏飞快扬起眉头。

方源继续道:“也没有听清。”

夏飞快便再复述一遍,语气很不愉快:“仙友还有什么地方没有听清?”

“第三句话……”方源开口。

夏飞快顿时将眼珠子瞪了起来,他直接道:“恕我心直口快,阁下一连三句话都没有听清?这是在消遣我吗?”

“仙友实在是误会了啊。”方源连忙摆手,“第三句话……”

“我是听清了。”

夏琢磨脸色阴沉下来。

夏飞快眉头高高扬起,眼珠子瞪得溜圆,脸上现出了怒意:“那仙友还有什么好说的?”

方源便又道:“实在是,第四句话……”

夏飞快一摆手:“那我就再说一遍,你给我好好听真切了,听清楚了。”

他便真的复述了一遍,这一次语速放得很慢,一个字一个字,很清晰。

方源又起身,向夏飞快施礼。

夏飞快莫名其妙,隐有不妙预感,连忙还礼。

他语气缓和下来:“仙友这些应当清楚了吧?”

“清楚了,清楚了。劳烦仙友复述,在下再清楚不过了。哦,原来是这么一回事啊。”方源拍拍额头,复又坐下。

“所以,贵方要归还广寒峰,实乃合情合理之请求。我夏家也是秉持公道而来,这位张开醉便是张三峰的血脉后代,如假包换,仙友出手验证一下吧。”夏飞快再次提出要求。

但方源只是打量了张开醉一眼,笑眯眯地摆手道:“不忙,不忙。”

“仙友又有何事不解?”夏飞快面色一沉,语气不善地问道。

“劳烦仙友叙述了许多,但是真正情形,究竟是否这样?我还不能确定。”方源道。

夏飞快被气笑了:“我说的句句是真,仙友尽管求证!”

“哈哈。”方源也笑了,一拍手,“仙友快人快语,那在下就向族中先求证一番。”

说着,他闭起了双眼,似乎将心神沉入仙窍之中,催动信道手段去了。

其余三人等了半晌,也不见方源睁开双眼。

夏飞快忍不住催促:“武遗海仙友,你还没有求证好吗?”

方源睁开双眼,不好意思地笑了笑:“在下兄长日理万机,没有空闲啊,咱们等一等吧。”

关乎八转蛊仙,夏飞快也不好说什么,夏琢磨则插口道:“武庸大人最近的确很忙,也不得不忙。武遗海仙友,完全可以向其他武家蛊仙求证。”

方源顿时哈哈大笑,向夏琢磨竖起了大拇指:“仙友不愧是智道蛊仙,好主意啊!”

这份夸赞,顿时让夏琢磨神情一僵。

然后,方源又道:“可是在下对其他蛊仙,不是很熟。仙友有什么可以推荐的吗?”

“推荐你妹!”夏飞快在一旁听了这话,顿时气极,脱口而出。

“我妹?我只有兄长,却无妹妹啊。夏仙友想来是记错了吧?”方源笑眯眯的。

“……”夏飞快无语。

他和夏琢磨对视一眼,暗中交流,都看出来方源是在插科打诨。

“没想到武家竟派遣这样的人物过来!简直是可耻啊!!”夏飞快心中充满了愤怒。

他想好好谈事情,但方源就不。

偏偏碍于正道身份,夏飞快不能做出什么出格的举动来。

现在看方源笑眯眯的样子,他真想一拳就这样打上去,把方源揍得个满脸开花。

“不要着急,不要生气,一生气就着了他的道了。看来这人是个交涉老手,他正故意激怒我们,让我们露出破绽。”夏琢磨连忙劝道。

夏飞快只能呼哧呼哧,喘了几口粗气。

方源开口,一脸认真地看着夏飞快:“难道我真的有一位亲妹妹?还请夏仙友告知在下啊。”

夏飞快:“……”

他又喘了几口粗气,喉结滚动了一下,这才道:“刚刚是我口误,请仙友谅解一二。”

“哦?是这样。”方源点点头,关切地道,“我看仙友口干舌燥,胸膛起伏不断,似乎上火之征兆啊。还请喝茶,多喝茶能降火。说起来啊,这四季茶真是不错,仙友多喝几口啊。”

夏琢磨:“……”

夏飞快大吼起来:“喝什么喝啊,我们正在谈正事呢!”

方源一脸认真:“仙友这话就偏颇了,什么是正事?关乎身心之健康,难道不是正事吗?所谓广寒峰,不过是外物俗物,一笔资源罢了。仙友切不可因为这等外物,就生气上火,伤害了自己啊。”

他这番苦口婆心的劝慰,却好似火上浇油,夏飞快一巴掌狠狠地拍在桌案上。

“武遗海!”他怒吼道,“你不要以为你这样做,就能蒙混过关,将广寒峰留在你们武家。哼!”

方源连忙变色,极其诚挚地道:“夏飞快仙友,你这是哪里的话?真是冤枉我了呀。广寒峰的去留归属,咱们都得按照规矩来。这是张三峰散仙的东西,若这张开醉是其后人不假,自然要归还。我此次代表武家,必定会秉公办理,绝不会贪污**,为谋求私利坏了正道规矩!”

“仙友是明白人。”夏琢磨勉为其难地笑了笑,话语饱含深意。

夏飞快皱起眉头,不再大吼,而是道:“所以说,你赶紧验证一下,张开醉就是张三峰的血脉后裔,货真价实!”

“不忙,不忙。”方源笑眯眯地又开始喝茶了,“且让我先从家族中求证一下,不是我不信任两位仙友,实在是在下责任重大,不敢随意决断啊。”

“那你倒是快求证啊!”夏飞快喊道。

“武庸兄长,还没有回应啊。”方源为难地道。

“那就其他人啊!”夏飞快嚷起来。

方源一拍大腿:“说的是啊。可是我初来乍到,和家族的其他蛊仙并不熟悉。我也不知道向谁求证的好。刚刚我让你们推荐一下人选,你们就是不推荐。我也很难办啊。”

这番骇人的论调一抛出来,只把两位夏家蛊仙惊得呆住,相互对望。

合着这半天墨迹,推三阻四,反而是我们夏家的问题了?!

另一旁,五转蛊师张开醉,看着方源也是瞪圆了双眼。

这就是蛊仙的……风姿吗?

简直是无赖无耻到了极点啊。

张开醉顿时有了一种叹为观止之感,心中对于蛊仙的美好印象,直接粉碎了一地。

“得了,我算是看出来了。这位明摆着是要拖延时间。”夏琢磨传音道。

“他拖延时间干什么?”夏飞快问。

“很显然,武家完全不占理,处于极其被动的局面。武遗海便如此胡搅蛮缠,拖延时间。等到武家渡过这个难关,抽出手来对付我们,我们就难办了。别的不说,光是等到凰鸟繁衍之后,武镇便能抽出空闲,来这里交涉。他若是来了,局面就又不同了。”夏琢磨分析道。

“你说的是。可我们该怎么办?他武遗海摆明了要拖延时间,但又义正言辞,我们根本挑不出理由来啊。他要是想拖延时间,理由多的是。比如求证得有个时间,求证好了,如何验证身份也有个过程。验证了张开醉的身份后呢?他会不会又提出来,他是否是张三峰的唯一子嗣后代?这些问题扯皮起来,还真的没完没了!”夏飞快意识到了问题的严重性。

“我有一计。”夏琢磨道。

“何计?”夏飞快连忙问他。

夏琢磨便答了一句,夏飞快听得喜笑颜开,当即挺直身躯对方源道:“久闻武遗海仙友实力出众,乃是武家之荣耀。在下不才,愿意代表夏家,想向仙友讨教切磋一番!”

\end{this_body}

