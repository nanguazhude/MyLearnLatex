\newsection{一颗死蛋}    %第九十五节:一颗死蛋

\begin{this_body}

方源操纵“黑城”,一步一步,慢慢步入鹰巢之中。,

九色光辉,交替闪烁,映照在“黑城”的脸上。

视野中,都是天晶,营造出一副堂皇绚丽,奢侈梦幻的景象。

鹰巢中,并不大,仿佛一个小房间。

房间正中央,摆放着一颗大蛋。除此之外,空无一物。

“黑城”一览无余。

他的目光,首先集中在这个大蛋上。

这蛋犹如少年高低,椭圆形状,蛋壳宛若琉璃,反射着天晶散发出的九色光辉,卖相奇佳。

但方源敏锐地察觉到,这颗蛋已经死气沉沉,显然绝了生机。

意识到这点之后,他就不再上心,而是将注意力转移到其他方面。

他逐步逐寸地搜寻,耗费了两三个时辰,除了天晶还是天晶,他竟一无所得。

“怎么会?黑凡乃是宙道大能,他留下的真传,怎么会是区区一座空荡荡的天晶鹰巢?他的那些宙道仙蛊呢?还有他的仙道杀招呢?”

方源沉思,很快,他就将注意力重新集中到那颗死蛋上面。

“恐怕,要寻到黑凡真传,这颗死蛋还是关键!”

方源猜测着。

到了这一步,他从黑城处搜刮出来的记忆,已经不管用了。方源已是超越了历代黑家蛊仙,走到了他们从未达到的地步。

要得到黑凡真传,还得靠方源自己努力。

“黑城”围绕着死蛋转圈,动用蛊虫。进行观察,深入研究。

片刻之后。方源再次确定,这的确是颗死蛋。

货真价实的死蛋。

并不是什么伪装。

如此一来。方源之前暗中猜想的,可能黑凡真传就藏在蛋壳之中的假设,也被推翻了。

“难道说,黑凡真传其实已经被其他人偷偷取走。这里原来是有存放仙蛊的,但被人捷足先登了?或许我并不真是第一个进入这里的。”方源忍不住又想。

这不是没有可能的。

这个世界的水,真的太深了!

方源在和毛六暗中交易之后,就再也不敢小觑任何人。

他虽是穿越者,但这个世界上,穿越者远不止他一人。混得好的。都已经成为九转尊者。

正道、魔道、散修,无数的人杰精英、天才鬼才、英雄豪杰、霸者枭雄,贯穿过去,现在,以及将来,宛若群星闪耀。

而方源充其量,只是其中的一颗星罢了。

方源的这个猜测,不是空想,他也有依据。

根据从黑城中搜刮出来的情报。黑家蛊仙们也意识到,态度蛊是关键。他们一边着手搜寻态度蛊,另一边也走琢磨手段,模拟态度蛊的气息。“哄骗”天晶鹰巢开启。

在黑城的记忆中,黑家蛊仙没有成功。

但并不意味着,真的就没有蛊仙成功过!

说不定。就有蛊仙偷偷地开启了天晶鹰巢,私底下拿去了黑凡真传。旁人都不知道。

甚至,未必是黑家蛊仙。

这个例子。在历史记载中就有很多。

最让人熟知的,就是盗天魔尊。

这家伙有偷天盗地之能,一生当中,偷窃了许多蛊仙真传,让无数势力恨得牙痒痒。

方源又检查几遍,真的没有发现什么。

依他如今的手段还有眼界,没有发现的话,恐怕就真的不存在什么黑凡真传了。

方源心中当然有许多失望。

他还指望着黑凡真传,为他延缓仙窍时间呢。

当然,他现在有楚度帮衬,暂时还用不到。

但更远的未来呢?

人无远虑,必有近忧。

与楚度合作,很危险,是与虎谋皮。天意也会在旁边挑唆。

就算方源现在用不到,有这个做一个保险,也能让自己在未来的日子里进退自如。

“即便没有黑凡真传,这座天晶鹰巢,都是八转仙材,也是具有庞大价值的。”

可以说,刨除方源仙窍中的那些仙蛊,其他的资源累加起来,都没有天晶鹰巢的一半!

当然,这里面没有算荡魂山。荡魂山、落魄谷都在琅琊福地,并且胆识蛊的买卖,方源都未插手,让琅琊派出售的。他只接收每月的收益分润。

“不过,这个死蛋,究竟是什么?依凭我的见识,居然也认不出来。如果黑凡真传还在的话,说不定这个死蛋就是黑凡故意留下来的线索呢。”

尽管希望渺茫,方源还是打算顺着这条线,去查一查。

他首先去了宝黄天,这里也贩卖一些修行的见识和经验。

其中就包括什么荒兽志、南疆堪舆图、上古荒植秘录什么的。

方源神念四处穿行,精挑细选之后,搜刮了一些,买来细看。

上一次,他耗费近万块仙元石,买来了大批丹青香。但本月的收益还未到账,使得他手头有些紧张。

可是看了这些之后,方源虽然涨了一些见识,但仍旧没有查出这颗神秘死蛋的跟脚。

方源没有失望,反而心中微微欢喜起来。

这颗死蛋越是神秘,显然来头越大,越让他感到有希望获得黑凡真传。

“这颗死蛋,很有可能,是一颗太古荒兽的蛋!”方源看了许多修行见闻,也不是一无所获。

如此一来,宝黄天中的那些修行见闻,基本上就不起什么作用了。

因为这些见闻,大多局限在六转、七转的层次上。

太古荒兽相当于八转战力,但八转蛊仙不会将自己的见闻,在宝黄天出卖。他们不至于贪图这等小钱。大多数的蛊仙,也不想去买,因为他们的层次,也遇不到什么太古荒兽。

现如今,大多数的太古荒兽,都存在于太古黑天、白天之中。除此之外,就是五域各处的灾地、禁地。例如,北原的十大灾地。或者说地渊、地沟等等这些地方。

“要不要在宝黄天主动吆喝,开高价询问此蛋来历呢?”方源想了想,旋即否决了这个念头。

太招摇了。

黑家蛊仙可没死绝呢。

智道蛊仙一推算,就能算得许多东西。

这对方源很是不利。

万一吸引来那些八转蛊仙,可就糟糕了。

方源现在还惹不起这些存在。

方源在宝黄天中受挫,也不气馁。因为他清楚,在他身边,就有一个存在,很大可能可以帮助他解惑。

这便是琅琊地灵!

琅琊地灵可是长毛老祖的执念所化。而后者可是八转蛊仙,炼道无上大宗师,名垂青史,声威赫赫。

什么流派的蛊仙眼界最广?

智道、信道还有炼道。

智道蛊仙能推算种种隐秘,信道蛊仙擅长探知情报,而炼道蛊仙要炼制蛊虫,须得广知蛊材。

作为历史上屈指可数的炼道大能,长毛老祖的见识广博,绝对是毋庸置疑的。

方源便主动向琅琊地灵求教。

过程并不顺利,方源受到琅琊地灵的好一阵刁难。

最近,琅琊派上下都对方源很有意见,觉得他只顾自己修行,不为门派奉献。到现在,琅琊派开发太丘的大计,都受到落星犬的阻碍,不得寸进,进展甚微。

“方源,你若是铲除落星犬,派中的此类藏书任你参详。若是你要换,就拿出一百贡献罢。”琅琊地灵脸色阴寒,语气低沉。

方源笑了笑:“我愿换。”

一百贡献,几乎已是他手上的全部。

不过,就算如此,方源也要换。太丘他是不打算去的。

琅琊地灵冷哼一声,正欲发作,方源又道:“一切按照门规行事,太上大长老的确处事公允。琅琊派的兴盛,指日可待。”

琅琊地灵气得胸脯耸动,毛发竖立,他狠狠喘了几口粗气,按捺下来:“听闻你和毛十二交情颇深,我记得这孩子曾四处夸赞过你。他最近挑战落星犬,身受重伤,手中的荒兽也是全数战死,意志消沉。你不去看望他一下吗?”

“谢太上大长老的关心。”方源一脸沉重和悲伤,好像受伤的不是毛十二,而是他自己。

“换了藏书,我便立即去看望毛十二。”方源做出保证。

琅琊地灵无奈,拿这样的方源毫无办法。

方源如愿以偿,的确顺路去往毛十二的云城去。结果并未见到毛十二本人,而是一位凡人蛊师:“我家城主不愿见客,大人请回吧。”

方源吃了个闭门羹,却没有立即走,而是关切问询。

结果,这位凡人蛊师十分冷淡,方源讨了个没趣,最终“无奈”离开。

如果不是地灵通晓福地中的一切,方源连这番表演都不会有。

他心中明白得很,区区一个凡人毛民蛊师,怎敢如此对待他这位蛊仙?很显然,是凡人蛊师背后有毛民蛊仙撑腰。这个人不是别人,正是对方源大有意见的毛十二了。

“我和琅琊派的关系,已经到这种地步了么……这当中,应该有毛六的‘功劳’罢。琅琊派还大有利用的价值,将来还需修缮关系,不过不是当务之急。”

回到自家云城,他查看资料,果然没有让他失望。

“原来这颗蛋,竟然是上极天鹰的蛋!”

方源心头震动。

上极天鹰,可是太古荒兽,八转战力!

震动之后,方源满心欢喜。

他意识到:自己的这番努力没有白费。如果这真是上极天鹰的蛋,那么黑凡真传便没有被取走,还在那里,等待着他!

\end{this_body}

