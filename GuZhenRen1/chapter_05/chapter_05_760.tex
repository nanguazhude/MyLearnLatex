\newsection{杀招偷道}    %第七百六十三节:杀招偷道

\begin{this_body}

%1
方源知道,算不尽的身份毕竟只是伪装。要加入房家,最大的障碍就是房睇长。

%2
因为房睇长是智道大宗师。

%3
有大宗师这样境界的,往往便是八转中的强者,难缠的人物。

%4
但是要让方源动用智道手段,来应付房睇长的推算,未免太着痕迹。毕竟方源的智道境界只是宗师级数,而他在智道方面所擅长的是推算和改良杀招。

%5
借助智慧光晕,单在推算杀招方面,他可谓是世间第一人。

%6
但应对其他蛊仙,克制其他智道蛊仙的推算,就不是方源擅长的了。

%7
一旦方源加入房家,房睇长很有可能产生不好的情绪。

%8
智道蛊仙平常的时候都很冷静,若是心湖起波,激荡出某种情绪,意义就很重大。

%9
所以,方源安排了影无邪,伪装成西漠蛊仙来进攻房家的天露绿洲。

%10
果然,房睇长中计,没有怀疑到方源的身上。

%11
安排影无邪侵犯天露绿洲,还有一层用意。

%12
就是帮助方源更快地“融入”到房家当中。

%13
算不尽和房家合作,参与过豆神宫争夺战,但算不尽刚刚加入房家,西漠正道并不了解,房家也对算不尽的忠诚怀有疑虑。

%14
这个时候,方源就直接安排这场交手,展现出自己的战斗力,并且某种程度上向房家表表忠心。

%15
“只是……房睇长的智道境界远比我高,我和他接触的时间越长,他察觉不妥的几率就越大。”方源暗自警惕。

%16
有一个智道大宗师做对手,就是麻烦!

%17
饶是方源拥有见面曾相识杀招,也有被识破的危险。

%18
除此之外,还有一个风险,就是影无邪。

%19
影无邪伪装的手段,并不奇妙,毕竟魂道不擅长这点。另外传奇太古魂兽——青仇恐怕也潜伏在青鬼沙漠的某处。

%20
这些都是潜在的危机和风险。

%21
然而,比起这些危险,方源更需要海量的魂兽来助益自身的魂道修为。

%22
他宁愿冒这些风险!

%23
“虽然我现在加入了房家,房家也答应开拓青鬼沙漠,每隔一段时间向我提供一批魂兽或者魂核。但他们本身就自顾不暇,又有魂核的库藏。合作前期应该是将库存里的魂核,交出来应付我罢。”

%24
方源早已算出这一点,所以青鬼沙漠的开拓,主力还得靠自己。

%25
方源从未真的想过依靠房家的力量,来帮助他捕猎魂兽。房家这种情况,不会尽心尽力。

%26
事实上,方源也不想让房家过多的进入青鬼沙漠里。

%27
和房家合作,主要是避免房家和自己添乱作对。毕竟青鬼沙漠周围,就只有房家这么一个超级势力。

%28
方源又不可能驻守在青鬼沙漠这里,安排影无邪这类棋子。一旦房家和他作对,对付方源有些难,但对付影无邪却是不困难的。

%29
方源早已思考了很多。

%30
这件事情,他还非得先和房家合作。

%31
房家虽然没有利用魂兽的高效手段,但是每年也会捕猎许多魂兽,采集魂核,一方面增加库存,另一方面贩卖出去。

%32
方源就算顶着算不尽的身份,来独自开拓青鬼沙漠,也会被房家视作眼中钉。

%33
作为一个超级势力,又紧贴着青鬼沙漠,房家早已将青鬼沙漠看做自己盘中的菜。

%34
对于这个潜在的敌人,方源提前铲除它并不可行,那就只有和它合作。把敌人变作朋友,这也是消灭敌人的一种方式。

%35
对于方源而言,杀只是一种手段。

%36
除了这种手段之外,还有其他无数的手段。

%37
只要能达到自己的目的,他当然更愿意采取更恰当,更高效的手段,不会去管这个手段具体是什么。

%38
这点就和幽魂魔尊不同。

%39
幽魂魔尊处理事情、矛盾的手段,基本上就只有一个。

%40
那就是杀!

%41
你不服?杀!

%42
你碍眼?杀!

%43
自己心情不好?杀!

%44
感觉无聊?杀!

%45
方源开始视察天露绿洲。

%46
针灸树之前已经看过,它们就在外围,一目了然。而阵内的湖水却是要入阵方能细察。

%47
天露绿洲中的水资源不仅量大,而且种类也比较多。

%48
常见的有玲珑泉、玉蟾涎、苦香水种种,罕见的有神力水、黑白水等。

%49
蛊材量大,仙材类多。

%50
比方说神力水,它虽然是水,但却蕴含力道道痕,是如今市面上的六转力道仙材。

%51
在凡人的故事里,神力水也有很多的名称,常常有凡人主角喝下了一口神力水,变得力大无穷,打击邪恶蛊师的情节。

%52
又因为神力水乃是仙材,单纯喝下肚腹,没有配套的食道手段,水中的力道道痕能够让蛊仙致死!

%53
所以这些故事中的凡人主角,往往是报仇雪恨之后,自己也死了。

%54
最主要的,还是黑白水。

%55
黑白水乃是七转仙材,位于整个天露绿洲的核心地带。

%56
白天的时候,黑白水呈现白色,夜晚的时候,则呈现黑色。

%57
把黑白水放置一段时间,每天直面苍穹,黑白两天轮转不休,黑白水中就会凝聚出一滴、两滴的天露。

%58
天露透明,小水滴模样,上细下圆,一滴天露约有婴儿的小拇指大小。

%59
天露并不相互融合,就算有很多滴天露,都只是挨着一起而已。

%60
天露绿洲中最有价值的,便是天露了。

%61
它是八转仙材!

%62
方源很快在黑白水域,看到这里积累的天露,足有一百多滴。

%63
影无邪入侵这里,并没有攻打到核心,这大半年来积累的天露,都还在呢。

%64
方源笑了笑,毫不犹豫地出手,将其中的三十滴天露,收入自家囊中。

%65
这是正道的潜规则。

%66
但凡外派镇守资源点的蛊仙,都有暗中扣下一部分资源,辅助自家修行的传统。

%67
当然,明面上没有这个明文规定。

%68
但基本上各大超级势力都是默许的。

%69
在外驻守资源点,通常是一件苦差事,伴有风险。若没有这样的甜头在,谁还会情愿外派呢?

%70
方源有五百年前世的经历,自然明白内中分寸。拿下三十滴天露,便是他的极限的极限。这笔资源价值很高,是天露绿洲两三个月的积累!若再超出这个底线,房家就要翻脸,告他贪污了。

%71
“只可惜如今九天破碎,只剩下黑白两天。”

%72
“若是九天聚在,按照天露绿洲的格局,就不是黑白水,而是八转仙材九色水。九色水直面苍穹,九天轮转不休,再有九天的星辰日日夜夜照耀直射,每隔九年就会从九色水中产出一滴九天星汗水了。”

%73
九天星汗水乃是九转仙材,和万物造化水、涨落潮汐水并成为天地三水,价值极高,在九转仙材中也分外罕见。

%74
不同于天地三火,天地三水中九天星汗水已经绝迹。

%75
因为九天中的七天都不在了,只剩下二天。

%76
视察了种种资源,方源暗叹果然不愧是西漠前十的天露绿洲,他又观察这里的大阵。

%77
他现在被任命镇守这里,自然也能借用大阵。

%78
房家也交代他一些大阵催用之法,方源试着一一催起,洞察大阵之妙。

%79
他有阵道宗师境界,又有智道手段,很快就有所得。

%80
和池家相比,房家在天露绿洲的守护大阵,并不怎么样。所以,房家花费大力,栽种了无数的针灸树来辅助协防。

%81
房家和池家一样,都擅长阵道。

%82
但不同的是,房家擅长的是仙蛊屋,可以将仙蛊屋看做是移动的仙阵。

%83
而池家则擅长布置固定的大阵,这才是阵道的主流。

%84
更关键的是,池家拥有一位阵道大宗师——池曲由!

%85
到了大宗师境界的阵道蛊仙,可以凭借自然道痕来布置仙阵。不仅可以节省仙蛊,而且还能借助地利,一面加强防御,一边辅助经营,助长资源点中的产量。

%86
房家在天露绿洲布置大阵,耗用的仙蛊不少。但大阵主要的效用,还是增加天露产量,并不侧重于防御。

%87
“不过房家的特长也很强大。”

%88
“拥有数座仙蛊屋,不仅是机动极高,而且借助仙蛊屋,房家的六转蛊仙都能和八转敌人作战。”

%89
“如果我是房家首脑,面对此时处境,不妨纠集兵力,先狠扑敌方一路,展现出凶狠来,震慑住其他的西漠正道势力。”

%90
“只是这其中的分寸,要十分注意。把握不好,就是引火烧身。力度稍小,也起不到威慑效果。”

%91
方源也在揣摩房家和西漠正道势力。

%92
这段时间他要潜修,不断消化战果,改良杀招,在南疆还是在西漠并不重要。

%93
时间比较紧。

%94
距离南疆正道谈妥,组成追辑他的队伍,已经近了。

%95
翻天沙海。

%96
巨大的沙浪,一波波连绵起伏。在月光的照耀下,白沙如雪,沙浪往往升腾数丈高度,气象浩然,乃是西漠有名的盛景之一。

%97
房家的蛊仙们秘密潜伏进来,接近沙海中的宝月绿洲。

%98
“按照情报,董家的董陆沉已经过来了。行动吧!”房睇长悄然传音。

%99
“好!”房功猛地暴射而出,气势骇人,直接杀入宝月绿洲中去。

%100
董陆沉飞升上来,一身八转气势同样铺天盖地,他惊怒交加:“房功,我不去找你麻烦,你倒来我这里蹦跶!”

%101
房功咆哮一声:“废话少说,来战。”

%102
两位八转蛊仙交手,掀起惊天的雷暴声响。

%103
董陆沉陷入被动,有点束手束脚,因为他还要分心护住身后的宝月绿洲。

%104
他招架着房功的攻势,满脸阴沉:“你们房家等着!你竟然随意攻略其他正道势力,你破坏了规矩,过不了多久,就有西漠各大家族的联合制裁!”

%105
“是么?你们董家不是在最近扬言,宝月绿洲曾经被我族偷袭过吗?只是你们拿不出证据了,我可是好心好意来帮你们,主动送上证据来的。”一座奇形怪状的残破仙蛊屋升腾上空,里面传来房睇长的声音。

%106
董陆沉更怒:“房睇长,你区区七转也敢……”

%107
话还未说完,房家的仙蛊屋就催出一记仙道杀招。

%108
董陆沉脸上的怒意顿时消散不见,取而代之的是震惊和惶恐。

%109
“这记仙道杀招,难道是?!”

%110
“没错,正是盗天魔尊当年用来开创流派的杀招——偷道!”房睇长回答道。

%111
“不!”董陆沉大吼一声,转头向宝月绿洲望去。

%112
在偷道杀招下,宝月绿洲中的无数天然道痕,显露出形象。它们又长又短,绝大多数都是丝线,相互编织着,组成某种充满奥妙的形态。它有的部分像是网,有的则仿佛是纠缠一团的毛线,有的堆积起来仿佛山丘。

%113
董陆沉尝试阻止,但毫无效果。

%114
他睚眦欲裂,只能眼睁睁地看着这些道痕,统统投入到房家的那座偷道仙蛊屋中去。

%115
这正是偷道杀招的威能!

%116
催动此招,便能将道痕直接偷取过来。

%117
当年,盗天魔尊创出这一招后,这才厚积薄发,引出质变,正式创建出偷道这一流派。

\end{this_body}

