\newsection{异人种族传承}    %第七百五十三节:异人种族传承

\begin{this_body}

池曲由这么一大把年纪,平日主持池家大小事务,操心劳累。为大时代的到来殚精竭虑,还要栽培不成器的继承人。

说实在话,池曲由非常不容易,这些流言还给他上了眼药!

池曲由憋着一股夹杂着悲愤的闷气,咬牙切齿地等待着方源再跳出来。

作为罪魁祸首的方源,这些天却活得很平静、安宁并且充实。

他在经营自家仙窍。

仙窍是一个蛊仙的根基,仙窍经营得不好,再强的蛊仙都是无根浮萍。

琅琊福地已经彻底并入至尊仙窍,本源封印解开后,就被方源迅速融合吞并。

方源让琅琊地灵出头,和所有异人蛊仙的盟约也换了。全新的盟约内容严谨,约束力很强,确定了方源至高无上的权利地位,其余的异人蛊仙都是附庸、下属的身份。

并且规定:所有的异人仙窍,在其牺牲之后,都要并入至尊仙窍。

所以,方源的至尊洞天又增添了许多福地。这些福地都来源于琅琊保卫战中,死在陈衣手中的那些倒霉孩子。

毛民蛊仙们对此惊诧,但琅琊地灵威信甚高,加上毛六的配合,安抚住了他们。

其余的异人蛊仙们当然更不愿意,但方源实力摆在这里,他们无法反抗,只能默默忍受。

当然,方源这种老谋深算的家伙耍弄政治手腕起来,比起武庸也是不遑多让的。

他在盟约中也规定了自己:要在将来全力栽培异人,不带任何偏见。盟约起效之时,他当场就给了这些异人蛊仙很多好处。

并且,他还偷偷地将其中一部分异人蛊仙,带去了石莲岛,让他们也享受一下未来身杀招,开阔开阔眼界。

异人蛊仙们皆是震撼无比,他们这才知道方源背后,还站着史上最神秘的红莲魔尊!

这下,他们一点反叛的念头都没有了!

当然,雪儿、冰卓这些人,方源没有放任他们享受未来身。因为按照墨水效应,方源上一世已经杀死了他们,在光阴长河的下游他们是不存在的,他们是没有未来的。

要欺骗他们也很简单,方源直接对他们讲:石莲岛是有极限的,未来身杀招需要继续积攒、酝酿。

反正盟约中,他们必须对方源真诚,方源却可以保留秘密。

这一点当然很不公平。

但在方源萝卜加大棒的手腕下,这些异人蛊仙只得低头就范。

其实,方源是真心想要发展这些异人。

原因主要有三点。

第一,异人数量多了,寿蛊就产出得多,在这点上方源怎么未雨绸缪,都不嫌早。

第二,方源开始尝试领悟人道。豢养更多的人族、异人,对于他领悟人道帮助很大。

第三,则是和异人的种族传承有关。

当年,人族仙尊、魔尊陆续现世,将十大异人种族打压、屠戮,异人前景黯淡无光,一片绝望。异人意识到不妙,所有人联合起来设下种族传承,将整个种族的最精华的宝藏,埋藏起来,留给后代,也留给现在一抹希望。

按照异人的理解:宿命蛊归属于天道,损有余补不足,“人道当兴”我们认了,但不可能总是“人道当兴”吧!

异人们的理解当然不算错误,但他们没有料到星宿仙尊太过赖皮,直接以身合道。

以至于人族一直当兴,异人们傻眼,一直被打压。

不过,这点对于方源是很有利的。

按照琅琊地灵透露出来的情报:

当异人种族的人数、蛊仙数量达到一定的数目,这些种族传承就会显露而出,直接馈赠给异人,帮助这些异人崛起和壮大。

历史过往,异人就没有崛起壮大的时候,所以这些种族传承应当保留了大部分下来。

根据琅琊地灵的猜测,毛民的种族传承之一,就是不败福地!

不败福地就在毛脚山,然而天庭虽然发现,但是攻略它极其艰难。底蕴深厚如海,人才济济一堂的天庭,都有些拿不败福地没有办法。

最终,他们依靠人道手段,另辟蹊径,截取出不败福地的一些财富。但是没有一位天庭蛊仙,能够真正步入这块封闭的福地。不败福地中究竟是何等景象,无人得知。

上一世,方源即便摧毁了毛脚山,不败福地仍旧完好无损。

如此种种,皆可见不败福地的神奇奥妙,更可推测各大异人的种族传承是多么神秘和超凡。

这也是为什么,方源虽然拥有了定仙游,也改良了相关杀招,但到现在仍旧没有去往毛脚山。

按照上一世的记忆,这个时间段,天庭方面还没有在毛脚山铺设出九九连环不绝阵。

但就算如此,方源利用定仙游杀招突袭,也对不败福地没有什么好办法。

“天庭似乎寻找到了摧毁不败福地的方法,但我暂时并没有。”

“就算有,紫薇仙子岂会不防备这里?突袭那里的风险,实在太大了!”

方源深思熟虑后,还是放弃了突袭毛脚山的想法。

这个想法太冒险,太低估对手的智商,也违背了方源谨慎行事的风格。

吞并了琅琊福地,方源收获极大。

不提其他,他在情报上就受益良多。

异人种族传承就是近在眼前的例子。

琅琊福地历史太悠久了,传承至今已有三十多万年的历史。

方源才成仙多久?就算加上五百年前世,也没有六百年啊。

琅琊地灵知道许多历史秘辛,在被吞并之前,他不会和方源掏心掏肺,但现在则是知无不言,言无不尽。

数天后。

南疆,摄心河滩上空。

翠绿的微光悄然浮现,随后一个个绿色的气泡接连产生。

七转杀招的气息,迅速荡漾,并且向四周扩散开来。

“这股气息?”镇守在摄心河滩的,是池家的一位六转蛊仙,他主持着大阵,很快就察觉到了杀招气息。

但没等到他行动,这些翠绿的气泡就啵啵啵的自行崩散。

每一个气泡崩解,都有一股浓郁的绿光,宛若流水一般,扩散开来。

大量的气泡消散,取而代之的是一大团浓郁的翠绿流光。

流光溢彩,忽然绽射出刺眼的光芒,令人不可逼视。

强光来得快去的也快,当绿光消散无踪,一丝不剩后,一个身影出现在半空中。

他一身青袍,面冠如玉,鼻梁高挺,近乎姣丽。皮肤洁白若雪,眼眸深幽如潭,一头长发如瀑布般垂至腰间,发丝乌黑发亮。

镇守在这里的池家六转蛊仙,见到来客这般模样,顿时嘴巴张大,宛若蛤蟆,心惊胆寒,浑身上下皆冒冷汗:“方、方源?!”

方源自从逃离雷鬼真君的追杀,挂了她三根胸骨在宝黄天中,他的名望就一举压过凤九歌,成为公认的当今七转第一人,有匹敌八转的战斗力!

琅琊保卫战后,天庭败北。紫薇仙子虽然隐瞒了陈衣、雷鬼真君阵亡的消息,但却承担责任,并将天庭入侵琅琊福地失利的结果,公布于众。

凤九歌遭受方源追杀,半数仙蛊被盗的影像,更是被她直接放置在宝黄天中。

于是,世人皆知:方源不仅再度拥有了七转定仙游蛊,而且还吞并了琅琊福地,并可能已掌握了偷盗仙蛊的手段!

击败了凤九歌,方源的声势得到了巩固,乃至更胜一筹,成为蛊仙茶前饭后交流的谈资。

池家六转蛊仙见方源这个大魔头出现,哪里还敢逗留?

“快,快,快,再快一点!”他当即运转大阵,传送离开,逃之夭夭。

方源还在品味刚刚的仙道杀招——翠流珠。

这杀招以七转定仙游为核心,动用了方源手头中几乎全部的宇道仙蛊,不少炼道仙蛊,还有大量的辅助凡蛊,组合而成。

方源吞并了琅琊福地,成就八转,道痕大涨,单纯的七转定仙游仙蛊,已经无法带动他。唯有搭建出杀招来,威能暴涨,才能勉强传送出去。

上一世,他也同样构思出了宇道杀招,以定仙游为核心,带动八转修为的他。

但翠流珠杀招又和上一世的杀招,有着许多的不同。

上一世的定仙游杀招是宇道杀招,但这一世却是宇道和炼道的复合杀招。

原因之一,上一世的定仙游杀招是方源在敲诈勒索了南疆正道后,换来的不少特定的宇道七转仙蛊,方才搭建出来。但这一世,方源根本还未埋伏夏槎呢,更谈不上勒索南疆正道蛊仙了。

方源手中的宇道仙蛊,大多来自于琅琊福地的库存,六转居多,并且不成体系。

方源索性结合炼道仙蛊,将其大幅度改良。

有了炼道的威能,方源的炼道道痕不再是传送的拖累。

这是复合杀招的好处。

但复合杀招推算的难度大得多,方源是因为有准无上的炼道境界,搭配智慧光晕,方能迅速功成。

方源按住脑海中逐渐泛滥的思绪,往下一瞥。

数百里范围,崇山峻岭林立对峙。而在他正下方却是一片平坦之地,百亩大小,无数烟雾笼罩。

方源手指一点,群蛟嘶吼飞出,气势凶猛浩荡。

正是仙道杀招——万蛟!

方源又炼成了万我仙蛊,施展万蛟相当便捷。

万蛟冲入烟雾当中,轻易间就撕碎了底下的守护大阵。

和上一世一样,驻守在大阵中的池家蛊仙早就跑路了。

方源微微一笑,这正在他的意料当中,也是他故意暴露身份的动机之一。

上一世的记忆,让他洞悉了此处池家蛊仙的怯弱,这一世出击只是效仿,效果仍旧良好。

当然,方源也不得不暴露身份。

因为他动用了定仙游,并且接下来他还会屡屡动用相关杀招,相瞒都瞒不住。

\end{this_body}

