\newsection{盗天机缘?}    %第四百七十三节:盗天机缘?

\begin{this_body}

一片茫茫的黑暗,方源的魂魄在黑暗中饱受煎熬。

这片黑暗对他的魂魄,竟然有着非同寻常的消融威能。这是方源之前探索梦境的时候,从未碰见多的情况。

幸而,方源此刻的魂魄底蕴,已经超乎寻常,有着千万人魂级数,完全能经受得住梦境对魂魄的消融。

梦境中的时间,难以计算。

方源只感觉他守候了很久,这才见着一片茫茫然的黑暗中,出现了一个白色的光点。

光点旋即延展成一条白色的光线,随即光线向两边拉扯开来,光芒绽射,一时间充斥方源视野。

少年盗天渐渐苏醒过来。

原来白色的光线,正是灼热的阳光。

少年盗天连忙伸出手掌,遮住自己的眼睛。

“我怎么在这里?”他很快发现自己躺在沙漠上,夜已经过了,现在是白日的晌午时分。

少年盗天的疑惑没有持续下去,因为紧接着,他就从他自己的脑海中获取了一股信息。

这股信息,正是族中的蛊师留下来的。

内容大意是:少年盗天在圣典中违背族规,受到流放的惩罚。不过念在他年幼无知,只要他能够独自在这沙漠中支撑三天三夜,然后独自一人赶回族群栖息之地,就能够重新得到族群的接纳,继续生活下去。

“我擦!”明白之后,少年盗天顿时一阵咒骂,吐沫星子飞溅。

“老子只不过开口抱怨了几句,你们就将我驱逐出来,这就是虐待儿童,草菅人命!”

“你们这群混蛋!老子诅咒你们……”

少年盗天竖起中指,但却发现自己不知道该竖向哪个方向。

他不知东南西北,族群栖息地的方向,都无从而知。

这个发现,顿时让少年盗天泄了气。

他一下子瘫坐在沙硕上:“这可该怎么办才好?我虽然刚刚成为蛊师,但一只蛊虫都没有。身边更没有水袋、食物,一旦到了晚上,没有保暖的衣物或者遮风的帐篷,我会直接被冻死的!”

“不,我一定要存活下来。”

“在这个该死的世界中存活下来,然后找到方法,回家!”

少年盗天狠狠咬牙,双眼中仿佛燃烧着两团野火。

与此同时,方源感觉全身一松,仿佛一层束缚他的无形枷锁,忽然消失了。

他活动手脚,少年盗天也跟着活动手脚。

“哦?能控制了?”

“看来是要让我顶替少年盗天,顺利地在这沙漠中存活下去啊。”

方源立即明悟过来。

在绝境中挣扎求生的经历,方源并不少,但此刻他却也犯难。

原因只有一个,巧妇难为无米之炊啊!

少年盗天身无长物,只有两到三成的青铜真元,一只凡蛊都没有。这要让他在沙漠中求生,比登天还难。

“族群这样流放他,是要让他死啊。不,更准确的说,是不把他这个蛊师放在眼里。”

一般而言,每一个蛊师都值得珍重。但方源仔细一想之前目睹的景象,便发现了缘由。

“盗天童年生活的这个族群,并不强大,占据的资源也不多。所以,就算是有蛊师种子,也要择优培养。”

方源苦笑,发现这个对于他接下来面临的困难,毫无帮助。

四下望了望周围,都是一望无边的沙漠,连一丝绿意都没有。

仙道杀招――解梦。

方源使出手段,但等候良久,都未见任何效用。

这个之前帮助方源很多次,堪称无往不利的底牌,居然在盗天梦境中处处受制。

方源沉重地叹了一口气:“看来只能赌一赌运气了。”

他随意地选择了一个方向,开始跋涉。

梦里的一天半后,方源什么都没有遇到,渴死在了路途中。

这一下,方源的魂魄立即遭受重创。

他被迫从梦境中退出来。

探索梦境失败,魂魄重伤,重归至尊仙体之中。

方源眼前一阵阵发黑,连忙动用胆识蛊,治疗魂魄伤势。

“胆识蛊!”一旁的唐方明瞧了,目光中顿时流露出羡慕的神色。

他眼巴巴地望着,方源用了一只只胆识蛊,很快就将魂魄恢复过来。

胆识蛊对于探索梦境,却是是绝佳的辅助蛊虫。

方源重整旗鼓,再次遁出魂魄,进入盗天梦境。

又一次来到梦中的沙漠。

方源叹了一口气,选择了和之前相反的方向,试着前行。

走了一天,这一路上仍旧是什么都没有,方源又累又渴,隐隐要达到身体的极限。

“若是历史上,盗天碰到和我一样的情况,单靠他自身是绝对生存不下来的,唯有依赖外部助力,也就是机缘。”

“这就意味着,这里有一个方向,可以撞得机缘。只是我运气不佳,并未撞见。”

“这种关卡,似乎也考验探索者的运气。可惜我的运道手段,在梦中毫无用处。除非是结合梦道、运道,研究出梦运之类的杀招或者仙蛊来,才有威能效用。”

方源不断深入地分析着。

第二次探索,他仍旧是以失败告终。

休整之后,他继续第三次。

这一次,他选择了另外一个方向,走了没多久,就遇到了一个坑。

这是流沙的坑,方源陷入当中,难以自拔。

“这就又要死一回?”方源郁闷,只能眼睁睁地看着自己被流沙的坑慢慢吞噬。

但出乎他的意料,当流沙就要没过他的头顶时,忽然流沙猛地喷吐起来,仿佛是一股喷泉,将方源喷到了空中去。

方源落下之后,原本流沙的坑,已经成为一个圆洞,不断地向外喷射凉气。

凉气非常适宜,像是清风送爽。

“有戏!”方源连忙接近圆坑边缘,很快便见这一大股凉气中,飞出一只只像是白银浇筑的飞翅小鱼。

这当然不是小鱼,而是凡蛊凉风。

方源连忙出手,想要收服这些凡蛊。但是凉风蛊速度惊人,完全不理睬方源。

圆洞边缘,凉气喷射,不断排挤方源,将他推开。

方源力气薄弱,肚中又是饥饿难挨,根本无法靠近圆洞边缘,接触到这些凉风蛊。

不过方源没有丝毫气馁,他耐心等待。

不久之后,他果然等到机会,一只凉风蛊在和其他同伴碰撞的过程中,似乎受了伤,坠落到了地上,脱离了凉气,远离了圆坑边缘。

方源连忙一扑,将这凉风蛊拾取在手。

结果一激动,捏死了!

方源无语,只好再次等待。

很快,又等到机会,这一次接连坠落下两只凉风蛊。

但还未等方源前去捕捉,它们又重新飞起,汇入冲霄的凉风之中。

方源叹息一声,忽然灵光一闪:“这个时候,我若再次催动解梦会怎样?”

他再次催动仙道杀招解梦,几乎在下一刻,他就看到了成效,一只凉风蛊坠落了下来。

方源拾取在手,还未炼化成功,又有第二只凉风蛊落下来,并且受伤不轻,扑腾鱼翅,就是飞不起来。

方源施施然又将其捉住,耗费一番功夫,将这两只凉风蛊都炼化掉。

等到凉风消失,这群野生的凉风蛊彻底一飞冲天,离开方源的视野时,方源的空窍中已经掌握了五只凉风蛊。

“我终于有一些手段了。”

虽然并未解除生存的难题,但这无疑是一个巨大的好兆头。

更关键的是,这个流沙圆坑无疑是表明,方源选择的方向乃是正确的。

还有一点让方源士气大振的发现,那就是解梦杀招并非无效,只是要在这片盗天梦境中,用到对的地方,才有成果。

方源顺着这个流沙圆坑,继续前行,不久之前,他遇到了一小片绿洲。

绿洲中有一口井,井边却是有一群野兽守护着,里面的野兽首领身上还寄生着一只凡蛊。

方源来了精神,靠着五只凉风蛊和这群野兽博弈。

最终,他利用地形和诡计,剿灭了这群野兽,还收服了野兽首领身上的那只野生蛊虫。这是一只毒道蛊虫,是蝎子形状。

战局已定,方源连忙来到井边,结果让他相当失望,这是一口枯井。

不过没有关系,这一地的野兽尸体就是宝贵的粮食。

方源痛饮野兽鲜血,又生吃了几块血肉,填饱肚皮。随后,他又采集了野兽的胃袋,将野兽的血液尽量都收集起来,野兽的血肉也撕扯出许多块来,用野兽的肚肠串起来,绑在身上。

这时,梦境中已经夜幕降临,沙漠的空气中温度急剧下降。

方源早有对策,不慌不忙。

他先是将剥下来的兽皮和兽骨,都拿在手上,然后来到枯井口,将兽骨搭建起来,再用兽皮一层层铺上。

这些兽皮内层还附着油腻的脂肪,相当保暖。

大概弄妥兽骨兽皮,方源身子顺着故意留下的空洞,小心翼翼地探入枯井当中。

他用双脚撑着两面的井壁,支撑着自己身躯不下坠,然后又将井口的兽皮拉扯,彻底盖住空洞。

然后,他才一步步落到井底。

到了井底之后,方源原本打算在这里熬一夜,不想却有了意外发现。

原来最底处的井壁处开了一个小洞,里面似乎别有洞天!

看这个小洞边缘,明显是有人为的痕迹。

方源心头一震,连忙驾驭着少年盗天的身体,钻入这个小洞中。

顺着这个小洞,约莫走了五六十步,方源就来到了一个房屋般大小的地下空间中。

这个空间很是简陋,但却有泥捏的锅碗瓢盆,还有一口暗泉,泉口很小,却是积蓄着清澈的地下水。

另外还有一个枯朽的干尸,平躺在地上。

“难道说,这才是盗天魔尊少年时期,遇到的真正机缘?”

方源顿时来了兴趣。(未完待续。)

\end{this_body}

