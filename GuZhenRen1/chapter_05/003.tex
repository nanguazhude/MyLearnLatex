\newsection{方源借蛊}    %第三节:方源借蛊

\begin{this_body}

风在耳畔呼啸,方源跋山涉水。哪怕没有动用蛊虫,也是速度飞快。

过了灾劫肆虐的范围,再不是一马平川,而是山林座座。

南疆多山,这本就是这个地域的一大特征。

“我奔跑了这么久,中途一点都没有休息,居然还是没有疲惫之感!最关键的是,我根本没有动用任何蛊虫。”

方源此时,已经初步感受到这副身躯的特殊之处。

蛊师的身躯,再没有利用黑豕蛊、白豕蛊之类的蛊虫改造之前,和凡人的身躯并无多少差别。

正是因为道痕加身,才使得蛊师的身躯有了种种奥妙,非同寻常之处。

“我的这副身躯,本就是由九转仙蛊至尊仙胎而化。看来我身上的道痕,绝对不少。至少力道方面的道痕,绝对比我原来的积累,还要雄厚得多!”

方源心中暗暗评估。

他自从变成仙僵之后,也感觉不到疲累。但若恢复原先肉身鲜活的状态,像今天这般奔跑,肯定跑不了一半,就要停下歇息的。

方源在厚密的山林间奔驰攀登,险峻的山岩在他手脚并用之下,宛若坦途。

方源还赤着身子,仿佛一条白浪,在阴翳浓郁的山林中穿梭。

浓密的山林,在他的视野中,分辨得清清楚楚。哪怕是奔腾的过程中,也宛若静止一样,被方源轻松避开。

“用地球上的术语,就是动态视觉吧。我现在的动态视觉,简直优秀到了极点!这绝对是非人的程度。我原来的身躯,只有动用蛊虫,才能有这样的程度。”

三下五除二,方源攀上这个山头的巅峰。

视野中阴暗的山林,陡然消失不见,视野陡然开阔,天高山雄!

大风吹拂着他的长发,向身后飘扬。

方源微微喘息了三口气。顿时就平复下来,气息悠缓,好像刚刚的一路狂奔,根本没有发生似的。

体力充沛到难以想象的程度!

山风冷冽。方源还赤着身子,不着片缕,但他却感觉不到任何的寒意。

心跳很快就平复下来,每一次心脏的跳动,都沉缓有力至极。

方源展目远眺。

他的视野极其广阔。

遥看万步之内。一切景物,分毫毕现。

轻轻一扫,方源目光顿住,神色微凝。

他发现了战斗的痕迹。

“看来太白云生、黑楼兰带着我的肉身,在这里遭遇了其他蛊仙,双方交手了。”

方源倏地一下,直接从山巅跳下。

砰的一声,他直接落到三丈之下的一处山岩上。

毫发无损。

甚至连双脚,都未感到酥麻!

倒是脚下的山岩表面,布满了蜘蛛网般的裂痕。

方源眼中精芒一闪即逝。开始继续尝试。

高度的落差,越来越大。从起先的三丈,迅速增长到五丈,八丈,十二丈。

用地球上的距离单位换算,三丈就是九米,五丈就是十五米,已经是五层高楼的程度。十二丈,已经是大厦、电塔。

十二丈还远远不是方源的极限,一路上涨到五十丈。

到了这个程度。方源终于感觉到脚底板酥麻起来。他胆量越放越大,有几次,他故意用身体的其他地方着地。

胸膛正面着地,背后着地等等。往往都把坚硬的山石砸出一个个的坑。

方源从坑中站起来,若无其事。

甚至浑身酥酥麻麻的,感觉还挺好!

深呼吸一口气,方源再次跳下。

这次是五十八丈。

这种高度在方源心中已经无所谓,关键在于,这次方源决定用脑袋着地!

砰。

一声闷响。方源整个脑袋深深地砸进石块之中。

双手用力一撑,他就将脑袋从石坑中拔了出来。

头骨是身体中最硬的地方,但方源还是有点感觉的。

他感觉头皮一阵酥麻,脑袋还有点晕。

但这种眩晕之感,持续的时间只有两三个呼吸,并且程度十分轻微。凉爽的山风一吹,就消散无踪了。

方源摸了摸头,又仔细检查了全身。

毫无伤痕,甚至就连头发都没掉一根!

用一个成语概括,那就是毫无发损。

方源试着拔自己的头发,他发现自己的头发十分坚韧,居然用了很大力气,才拔下一根。

这根头发,比常人头发还要粗大一些,黑的发亮,给人的感觉仿佛是一根极细的钢丝。

再用力拉扯,居然拉不断,十分坚韧。

方源若有所思,将这根头发缠绕在自己的手腕上,绕了几个圈,再结结实实地扎起来。

这头发不能乱丢,万一被有心人得去,可是个上佳的推算证据。

方源一边向那边战斗痕迹赶去,一边继续跳山,试探自己的极限。

高度不断提升,很快从五十多丈,达到八十多丈。

这个程度,方源已经感觉到疼痛,但还未到达难以忍受的程度。

最终,达到一百多丈时,方源终于停止了这番试验。

从这个高度跳下,已经让他感受到明显的疼痛,并且着地的部位,皮肤泛红,出现青紫的淤血。

“一百多丈,还不是这具身躯的极限吗?”方源暗暗咋舌。

这个高度已然达到恐怖的地步。地球上的巴黎铁塔,也不过三百多米,也就是一百多丈高。

“光凭肉身,没有催动任何一个防御蛊,就已经达到这个程度。就算是蛊仙,单凭肉身,也极少达到这种程度吧?”

“看来我身上的道痕中,防御性质的拥有不少呢。”

这个发现,稍稍让方源放心了一些。

毕竟他手中,解谜、换魂、态度三大仙蛊,都不是防御仙蛊。

来到战场痕迹的遗留之地,方源蹲下身子,仔细查看。

“好像是他们。”很快,方源站起身来,眉头皱起。

没有侦查蛊虫。很不方便,方源也无法肯定。

但他知道,自己必须追下去!

“时间不多了。”

来不及查看自身的奥妙,刚刚的试验只是顺手施为。方源继续追踪下去。

很快,他又发现第二处、第三处的痕迹。

方源随着痕迹,也慢慢改变了方向。

蛊仙的交手,大多发生在半空之中。波及到地面上的痕迹,并不多。再加上方源无法飞行。导致他追踪的难度十分的高。

随着时间流逝,方源的一颗心也渐渐沉入谷底。

忽然,一道身影,仿佛流星一边,从天而降。

轰的一声,落在方源面前。

火焰喷吐四溢,将周围数百步的山林,瞬间灼烧得干干净净。

“何人?!”方源双眉一蹙,眼中厉芒绽射,冷喝道。

“在下火崆峒。敢问这位朋友。是从义天山那边过来的吗?”来者浑身燃烧着熊熊烈焰,态度并不友善。

方源脸色迅速阴沉下来,换魂仙蛊的气息浮出,不答反问:“你来的正好。刚刚在空中飞行,可见到什么可疑之人没有?”

来者感到换魂仙蛊的气息,顿时知道方源并不好惹,气势一滞。

拥有仙蛊的六转蛊仙,可不占多数。

火崆峒虽是散修出身,但已投靠了南疆柴家。

柴家可是超级势力,此次义天山大战。柴家蛊仙殒命。柴家太上大长老便派遣客卿蛊仙火崆峒,前来探查究竟。

听到方源反问,火崆峒心道:“可疑之人?大白天的你一个人光着屁股,四处乱跑。你就是最可疑的人!”

他沉默了一下,终究还是开口,答道:“方圆百里毫无人烟,除了阁下之外。”

“可恶!可恨!”方源顿时嘶吼一声,脸色扭曲,一片狰狞。似乎要择人欲噬。

随即,他又捏紧双拳,咬牙切齿地自语道:“看来是跑远了!不过就算你们逃到天涯海角,我也要把你们追上。此次的耻辱,我要你们百倍,千倍地偿还!”

火崆峒见方源如此,心中顿时有了猜测:“看来此人赤身奔走,似乎是糟了他人手段,失去了不少蛊虫,吃了个亏。此人性情狠戾,怒火冲心,我还是小心一点。”

方源还没解释什么,火崆峒心里就为方源解释了。

并且,他更不想招惹方源。要是为他人挡灾,自然颇不划算。

“义天山我没深入,你自己去看吧!”抛下这句话,方源拔腿就走。

对于这个答案,火崆峒自然不满意,刚想要拦下方源,方源就顿住脚步,反倒是回首道:“等等!你是火崆峒?柴家的人?”

“正是。”火崆峒一愣,回答道。

方源脸上挤出一丝难看的笑容:“很好。我和你们的太上九长老柴不拆有旧,他的战场杀招万家薪火,可完善了没有?”

火崆峒连忙再答:“九长老仍在闭关。”

“嘿,他这是被他爷爷关禁闭呢。上次调戏依依仙子……嘿!等我处理了这事,我就去拜访他。”方源说完,竟向火崆峒伸出手来:“你拿点蛊虫借我用用。”

“啊?”火崆峒再楞。

“啊什么啊,你堂堂蛊仙,借我点凡蛊而已,这么小气干什么?”方源不耐烦地道。

火崆峒心想:“九长老闭关完善战场杀招,只是个对外的借口。真实情况,的确是调戏了依依仙子。对方既知此事,应当关系不假。我依附于柴家,是个散修外人。而九长老不仅是柴家中人,而且是太上大长老的孙子。我若遇到他的友人,若是不借蛊虫,势必说不过去啊。”

借吧。

反正是凡蛊,也算不得什么。

火崆峒取出蛊虫来,交到方源手中,又问道:“不知这位兄台,尊姓大名呢?”

“好说。我就是东方雄鸡。”方源摆手道。

火崆峒顿时一瞪眼,心道坏了:“他就是东方雄鸡啊?的确是九长老的损友之一,吝啬之名名传南疆,而且睚眦必报,气量极小。不过他相貌不是这般啊?哦!他无衣遮体,只能变化外形么。罢了,这借出去的蛊虫,就别想要了吧。能打发了这个小人,也是好的。”(\~{}\^{}\~{})

\end{this_body}

