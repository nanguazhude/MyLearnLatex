\newsection{仍是凤九歌}    %第九百五十节:仍是凤九歌

\begin{this_body}

%1
石莲岛上,凤九歌将红莲留下的记忆完整地浏览了一遍。

%2
他重新回到现实当中。

%3
红莲意志宛若真人,栩栩如生:“除了这份记忆之外,我还有两样东西交托给你。一份是八转杀招未来身,另一份则是命运蛊的仙蛊方。”

%4
接着,红莲意志又详细地解释道:“八转层次的未来身杀招,能够让你转瞬之间拥有亚仙尊的实力。毕竟你是命中注定的大梦仙尊的护道人。历代的护道人只要成长起来,几乎都是这样的修为。而未来身就是将你未来的最强姿态,照影到现在,为你所用。”

%5
“至于命运蛊的仙蛊方,虽然只是一件残缺的蛊方,但它本就是依照你未来开创的命运歌所创。它和你的命运歌是一脉相承。”

%6
“在未来上千年后,你开创出了完整的命运歌。但我却不能将它直接传授给你,因为宿命蛊并未彻底毁灭,并不允许如此。”

%7
“所以,我转变另一种方法,将其凝聚成这份九转命运蛊残方,它对你必有巨大帮助,足够让你领悟出最初期的命运歌。”

%8
凤九歌怦然心动。

%9
红莲真意交给他的真传,完全符合他的胃口,他根本拒绝不了。

%10
但他并未丧失冷静,敏锐地察觉到红莲真意的言下之意。

%11
“最初期的命运歌?”凤九歌问道。

%12
红莲真意颔首:“没错。你在未来上千年的时间里,多次改良命运歌。最终使得这记杀招变得极其精妙,最后一次改良,竟让这记杀招本身拥有了自我成长的能力。中招的蛊仙越多,它就能汲取到蛊仙身上的命运威能,从而自我壮大。这是一记非常棒的杀招。”

%13
“不过,需要提醒你一点。只要宿命蛊仍在一日,你的命运歌杀招就要遭受一日的压制和封锁。”

%14
凤九歌摇头:“红莲魔尊,多谢你的真传。我知道你的意思,事实上在此之前,我也被方源多次提点。但是我并不会因为你的这份真传,而去摧毁宿命,去对抗天庭。”

%15
红莲意志轻笑出声:“我并没有要求你什么。我只是将我的经历分享给你,将这份挣脱宿命的关键力量放在你的手中,你如何选择,那是你的事情。一切的决定权都在你的手中,不是吗?”

%16
凤九歌离开石莲岛,心情却很沉重。

%17
宿命对于红莲魔尊已经有所优待,虽然夺去了他的父母、爱人,但却给予了他超绝的才情天赋,众生之上的实力和地位。但是红莲却不愿意接受这些。

%18
人不是木偶,人是有思想的。

%19
“身为尊者的红莲尚且如此,那作为护道人的自己呢?”凤九歌不禁反思自己。

%20
看到凤金煌茁长成长,的确能带给父母欣慰和欢喜。

%21
别说护道人的身份,就算没有这层身份,凤九歌又怎么会不守护自己的女儿呢?

%22
“但是,这护道人的身份,真正是自己喜欢,自己想要的东西吗?”凤九歌在心底问自己。

%23
事实上,在此之前没有人问过自己。

%24
龙公没有问,秦鼎菱没有问,灵缘斋的诸位同派太上长老也没有问。

%25
仿佛这个身份就是荣耀,就是天经地义的事情,凤九歌欢喜地接受还来不及,这样的问题似乎太没有必要了。

%26
但真的是这样吗?

%27
凤九歌又想到了一个更深层次的问题:“人为什么活着?”

%28
这个问题,范围着实太大,太过深奥,答案更因人而异,一千个人会有一千个回答,甚至前一刻的答案,下一刻就会发生变化。

%29
凤九歌也没法回答这样的问题。

%30
他只能退而求其次,真诚地问自己:“我为什么活着?”

%31
凤九歌不禁回想自己的至今为止的生命历程。

%32
他自幼便喜好音乐,毫不犹豫地选择修行音道。

%33
修行之初他就立下宏愿,想要创作九首歌曲,唱尽自己、众生和天地。

%34
后来,他在一处山谷中得了奇缘,意外挖掘到了音道大能空响仙翁遗留下来的传承。

%35
他于是在山谷中结芦隐修,独自一人,却从未感到寂寞。他沉浸在音乐的美好中,不断修行,每一天都过得非常充实。就在这个山谷,他轻易地渡过关键灾劫,不声不响到达六转蛊仙境界。

%36
达到蛊仙的修为,凤九歌也没有丝毫得意和炫耀的心情。

%37
一切自然而然,又平平淡淡。

%38
他继续在山谷中修行,经营仙窍,修为从六转提升到了七转。

%39
有一天,两位蛊仙无意来到山谷,和凤九歌对唱。

%40
当时,正值夜晚,明月高悬,清风徐徐,吹得山谷中一座小湖波光粼粼。

%41
三位蛊仙唱和之间,时间飞速流逝,竟一连唱到了天明。

%42
三仙唱罢,纷纷大笑。却不照面,兴尽而归。

%43
后来二仙遭受污蔑,被正道通缉,凤九歌意外得知这个情况,毫无犹豫,义无反顾地挺身而出。

%44
由此,才有了他一鸣惊人,一力挑战十大古派,天下英杰,无人能制的传奇!

%45
十大古派脸面尽失,便指责凤九歌乃是魔道众人,开始联手抵抗他。

%46
凤九歌昂然不惧,一路转战三千万里,忽然调转兵锋,直捣黄龙,将十大派闹得灰头土脸,一片混乱,无可奈何。

%47
在这个过程中,凤九歌遇到了灵缘斋的白晴仙子,两情相悦,凤九歌转投灵缘斋,成为正道中人。

%48
从此,灵缘斋声威大振,依靠凤九歌势力、地盘不断扩张,将其余九大古派牢牢压在下风。

%49
在此之后,八十八角真阳楼倒塌,凤九歌便率领一批人马,前去调查。

%50
于是就有了后续一连串的事情,他被方源意外搭救,在南疆他偿还救命之恩,替方源抵挡武庸。还了恩情之后,凤九歌又开始追杀方源。

%51
直至现在,方源屡屡放过凤九歌一命,并且还几番提点他,指点他来到石莲岛,得到了红莲真传。

%52
“如果我之前的人生,都是宿命蛊安排的轨迹,那我是心甘情愿的吗?”

%53
凤九歌摇了摇头,他更愿意这样想:“这一切都是我自己的选择!我当时为了二仙挺身而出飞,放弃隐修的安祥生活,是出于心中激愤,要讨回公道。”

%54
“我和白晴仙子结合生女,是我爱她。我教导女儿凤金煌,抚养她长大,也是喜爱。并不是什么宿命的规定。”

%55
“如果宿命让我失去白晴,失去凤金煌,我会学红莲吗?”

%56
“如果现在有一个很好的机会,只需要我稍微动一动手,就能毁掉宿命,我会去这样做吗?”

%57
毁掉宿命,似乎违背了自己的利益。毕竟凤金煌可是宿命蛊规定的大梦仙尊!

%58
“但凤金煌真的喜欢自己成为大梦仙尊吗?”

%59
凤九歌忽然发现,就像所有人没有问过他究竟是否喜欢护道人这个身份,也没有人询问过凤金煌是否喜欢自己成为大梦仙尊。

%60
这当中甚至包括他自己!

%61
似乎这一切都是天经地义的事情,从没有质疑的地方。

%62
凤九歌又问了自己一个更尖锐的问题。

%63
“假使煌儿非常喜欢自己成为大梦仙尊,但宿命蛊就在眼前,能被我轻易摧毁。我会去这样做吗?”

%64
这个问题让凤九歌犹豫,让他彷徨,让他疑惑,让他迷茫。

%65
现实却没有给他空间和时间,让他去静静地思考。

%66
宿命蛊修复大战已经进行得如火如荼,凤九歌不得不立即参战。

%67
毛脚山战场上,他救下凤金煌,唱起命运歌。

%68
八转未来身,让他跻身于天底下最强的蛊仙之一。命运歌的威能更是令帝藏生都受挫。

%69
战场中,凤九歌终于再次和方源遭遇。

%70
方源暗中传音:“凤九歌啊,你终于开创了命运歌。你觉得这杀招怎么样?”

%71
凤九歌沉默。

%72
他不得不承认,这的确是属于他的歌,也是他的性情应该创作出来的歌。这歌里有着他的心声。

%73
是的,他并不想承认宿命的存在,但又感慨宿命的无情和玄奇。他阅览无数人生的轨迹,时而落魄时而风光,他看那些人在各自的人生中奢望、挣扎、欢笑、幸福、痛苦、绝望……

%74
“如果说,命是定数,运是变数。那么我渴望人生、万物都要有变化,这种变化应当来源于各自的选择!”

%75
凤九歌明白了自己心底最真的心意。

%76
“我创作的不是宿命歌,是命运歌!这才是我内心的真正想法啊。”

%77
他恍然大悟,明白了方源和红莲魔尊的图谋。

%78
他们知道自己的性情,所以他们从不劝说。他们只是将命运歌交回到凤九歌的手中,让凤九歌自己来劝说自己。

%79
是的。

%80
这首命运歌凤九歌高声唱着,他为天庭蛊仙而场,为他们增长状态。他为敌人而唱,让强敌们衰落下去。但实际上,他是为自己而唱,他要唱明白自己的心!

%81
于是,他明白了。

%82
他爱着白晴仙子,也喜爱自己的女儿凤金煌。但他凤九歌不是为她们而活着的。

%83
妻子和女儿只是他人生历程中的选择的结果,是他生活的一部分,而不是全部。

%84
“我为什么而活着?”

%85
这个问题让他回到了最初,他想要唱出九首歌,唱尽人生和天地!

%86
现在,他已经开创出了九首歌。但他并没有唱尽人生和天地,他还要接着唱下去。

%87
但接着唱,有着宿命的束缚,就像是一双无形的手卡着他的脖颈,这让他怎么能自由地唱?

%88
见凤九歌没有回应自己,激战中的方源仍旧传音过来:“凤九歌啊,你的歌声就是你的志向,创造出命运歌的人,又岂和天庭一路呢?”

%89
凤九歌潇洒从容地暗中回应:“你倒是知我!不过就算如此,我也不会轻易上了你的贼船。除非你营造出能轻易摧毁宿命蛊的情形来,否则我是不会随意出手助你的。中洲和天庭可是待我不薄。所以,一切还得靠你自己。”

%90
方源呵呵一声:“那你可瞧好了。”

%91
接下来,他便故意被龙公击坠,来到绣楼面前,激活了狂蛮魔尊留下的手段。

%92
出乎凤九歌的意料,方源凭借狂蛮魔尊的力量一路杀破了天,竟直接摧毁了宿命蛊。

%93
而后,龙公的一番话却让凤九歌意识到,这场战斗远没有结束。

%94
当红莲意志再一次出现时,凤九歌也渐渐明白了红莲魔尊的布局。

%95
所以,当方源索要命运蛊方的时候,凤九歌终于出手。

%96
他帮助方源,也是在选择做他自己!

%97
曾经,他相助二仙,挑战中洲十大古派,就曾笑唱:“魔不魔,正不正,天地自有凤九歌。走不走,留不留,死生皆在我心头。”

%98
“现在是走,是留?”

%99
这个问题再困扰不住他。

%100
他凤九歌从魔道来到正道,是时候回去了!

%101
因为,凤九歌仍旧是凤九歌。

\end{this_body}

