\newsection{幽魂真传!}    %第四百一十六节:幽魂真传!

\begin{this_body}

片刻之后,黄史的尸体,就被白凝冰等人带到方源的面前。想看的书几乎都有啊,比一般的小说网站要稳定很多更新还快,全文字的没有广告。]

幽魂意志望着,叹息道:“此人也是惊才艳艳之辈,少时就有天才之名,一路成长,乃是风云人物。昔日名传中洲,乃至五域。今朝,却是陨落在此处了。”

方源等人不语。

幽魂意志操纵这座石莲岛,纵观古今,得知许多古往今来的秘密。这是红莲真传的宙道威能,通过光阴长河,获知情报。

幽魂魔尊当年获取之后,也赞其“不可思议”,对红莲魔尊的宙道手段,自叹不如。

到了八转级数,蛊仙都会触类旁通,九转尊者自然更加如此,手段非常全面。

但是幽魂魔尊本身的宙道手段,或许能和其他八转媲美,但和红莲魔尊比较起来,就是小巫见大巫了。

九转之间,各有专修。术业专攻,便各有所长了。

换做红莲魔尊要和幽魂魔尊比试魂道造诣,自然又会轮到红莲魔尊自叹不如。

汩汩……

光阴的河水,从方源等人脚下涌出来。

石莲岛开始漏水了。

不过只是一两处地方,漏洞也只是茶杯大小。

幽魂意志僵持,叹息一声:“再过片刻,这座石莲岛将彻底陨灭,留给我们的时间不多了。”

红莲魔尊布置下七道真传,皆是由石莲岛承载,位于光阴长河的某个位置。

方源等人进入的这座石莲岛,被幽魂魔尊获得之后,便加以改造,成为影宗最大的情报来源渠道。不仅如此,影宗还多次运用红莲真传的威能,最近几次,便是保护春秋蝉顺利重生,以及帮助影无邪、紫山真君。

早已经底蕴将近干涸,刚刚一战,更是竭尽所能,杀死了宙道大能黄史。

现在,石莲岛彻底耗尽了底蕴,陨灭就在眼前。

“有得必有失。”方源感叹道。

“虽然是失去了这座石莲岛,但是令黄史上人陨落,便是巨大斩获。此人战力卓绝,在光阴长河中如鱼得水。若放过这次绝佳机会,就算再次设伏,恐怕也没有斩杀他的良机了。”

群仙点头,方源的这项决定,都得到他们的认同。

事实上,就连幽魂意志也不例外。[看本书最新章节请到棉花糖小说网www.mianhuatang.cc]

他看向方源,颔首道:“梦境一役,紫在生命的最后时刻所作出的选择,是正确。方源啊,我不得不承认,你已经是我影宗最后的选择。拯救本体的重担就交给你了,当然这也要看你的心情。若是你一心想要除去本体,我们也不会责怪你。”

这话说的,令在场的蛊仙们都不禁动容。

影无邪色变,几次张口,欲言又止。

谁能想到现在的局面?

此时,影宗和方源之间,方源完全占据主动。他的确是影宗最后的希望,甚至假设最终攻上了天庭,能不能救出幽魂本体,皆看方源个人的意愿。

幽魂意志说这话,有点破罐子破摔,无奈丧气的意味。

正是因为他知道情报的最多,导致他彻底认清形势。

方源进入红莲真传的过程中,他没有一丝的阻拦,或者是刁难,斩杀黄史上人等过程中,他都是全力配合方源。

方源一瞥身边群仙,回应幽魂意志道:“你放心,天庭势大,你我是天然的盟友。能帮的我一定会帮。”

幽魂意志点点头:“我信你。”

旋即,他又微笑道:“我们可不是什么盟友,而是一体的。你现在是影宗之主,由紫传位,得到众多分魂的一致认可。即便是将来面对本体,他也定然会承认。因为我们就是他,我们的决定,便是他的决定。”

“好了,现在我就将本体的魂道真传等到一切,都传授给你。”

方源点头,催动智道手段,立即飞出一团我意。

我意宛若巨球,扑向幽魂意志。

后者不闪不避,旋即,两股意志交汇在一起。

意志之间的交流非常的迅速,几个呼吸之后,方源我意回归,被方源置入脑海当中。

一瞬间,海量的信息冲击方源的心头!

方源闭上双眼,有条不紊地全盘接受,他可是智道宗师,不是寻常蛊仙可比,此时表现的非常从容。

在所有的信息中,最具有价值的,无疑是幽魂真传。

这是幽魂魔尊所创,最为完整全面的修行内容。

方源只是稍稍浏览了一些内容,就不禁心神摇曳,为开创这道真传的幽魂魔尊暗暗点赞。

毫无疑问,这是整个天下,乃至人族漫漫历史,不,整个世界古往今来,最具有价值的真传之一!

从古至今,这个世界中只出现了九位尊者。而幽魂魔尊就是其中一位。

方源获得的这份幽魂真传,非常完整,只要他按部就班地修行,就可接近幽魂魔尊曾经的高度。

说起来,关于九大尊者当中,方源牵扯的也不再少数了。

春秋蝉是方源最早入手的,它是红莲真传最关键的钥匙,可是直到现在,方源也只继承了红莲真传的七分之一。并且虽然内容完整,但是红莲真意却已是被幽魂魔尊吸收去了。

巨阳真传有三道,方源从琅琊派中获取了其中“我运真传”中的精髓部分,从八十八角真阳楼、马鸿运处获得的“众生运真传”的一些内容。并没有外界传闻,那么多的收获。

盗天真传,方源只继承了鬼不觉。神不知被赵怜云所得。

“没想到,我最终收获的最完整的尊者真传,反而是死对头的幽魂真传。”

“可惜,我的魂道境界不足,并且修行魂道,还得要海量的时间、精力、资源。”

“这其中,资源一项里对魂魄的需求量非常恐怖。幽魂魔尊当年屠戮天下,就是为了一己修行。我若是重复他的道理,也得是要屠戮苍生了,否则根本难以满足我的魂道修行的资源需求。”

幽魂真传之下,便是红莲真传。

这处石莲岛上的红莲真意,虽然是被幽魂汲取,提升了他的宙道境界,但是真传内容则是流传下来。

其中最主要的,就是关于春秋蝉的仙蛊方。

这可不仅仅是六转仙蛊方,还有七转、八转,乃至九转!

也就是说,方源掌握了六转到九转的,炼制春秋蝉的全部仙蛊方!

这一整套仙蛊方,内容极多,洋洋洒洒,有数十万字。单单六转春秋蝉的仙蛊方,就有十七八道。方源五百年前世获取的那道,只是其中之一罢了。

而七转仙蛊方更多,多达八十六。八转仙蛊方,则有四十三道。

九转仙蛊方,也有三道。

多一道仙蛊方,就是多一个途径,炼制出春秋蝉来。

对于方源而言,无疑帮助很大,今后他要炼制春秋蝉,不仅选择更多,并且更加灵活。

“没想到,红莲魔尊对于春秋蝉,研究得这么深邃。”

“原来,春秋蝉的确是红莲魔尊的本命仙蛊!”

方源当然也会提炼春秋蝉,他深知春秋蝉的厉害,绝不会放弃这个翻盘利器!

除了幽魂真传、红莲真传之外,还有海量的其他传承。

这些传承有些残缺,有些则非常完整。

这可是幽魂魔尊生前收集过来,并且死后影宗花费十万年,陆续搜集的成果。全部都便宜了方源!

紫山真君的遗赠,虽然也丰富,但他到底只是分魂之一,真正掌握的内容,不到这里面的百分之一。

尤其是里面,还有无极魔尊亲自布置的律道真传,乐土仙尊一脉的阵道真传。紫山真君的遗藏中就没有。

这所有的传承内容,价值高得难以估计!

只要留给方源充分的时间和资源,方源完全有信心,可以凭此建造出一个不弱于天庭的组织出来。

方源的收获,大得难以想象。

就连他自己,都为此震撼。成为影宗之主的恐怖收益,到此刻,他才接收到。

幽魂意志再次开口:“方源宗主,你现在算是彻底继承了影宗。可惜真传虽多,但修行起来,却是需要海量的时间、精力和资源。而眼下的局势,却偏偏最缺时间,资源也短缺得很!”

“我本体曾经逆天成功,潜伏天庭,拖延了大时代来临将近五百年!但现在本体失败,一切回归原来的轨迹,大约十年之后,就是大时代彻底开幕,五域合一,必定产生历史上最大规模的混战。”

“而在五域当中,中洲底蕴最是雄厚,拥有蛊师、蛊仙最多。并且,他们还拥有着统一的领导者――天庭!这是其他四域,完全不具备的。不管是东海、西漠,还是南疆、北原,都是超级势力林立,一盘散沙。北原虽有长生天,稍微好一些,但对本域的影响,比起中洲的天庭,要差得多。”

“而天庭最恐怖的一点,就在于他们掌握着九转宿命仙蛊。此蛊威能极其恐怖,星宿仙尊能够在死后,设下谋略,抵御三位魔尊,保持天庭不失,大多依仗着此蛊的威能!”

“三大魔尊都在宿命当中,这是他们出身就已经决定的,改变不了。唯有天外之魔,不受到宿命仙蛊的束缚。方源啊,你现在是完整的天外之魔,是最有希望毁灭宿命的人选。事实上,你就是红莲魔尊最期待的传人!”

“十年,只有十年。方源啊,千万不能让天庭复原了宿命仙蛊!否则,天庭将彻底利于不败之地,即便是四域齐攻,也会屹立不倒。到那时,攻上天庭,拯救本体,是彻底没有希望的。”

“留给你的时间不多了,方源。”(未完待续。)

\end{this_body}

