\newsection{海市赌石坊}    %第六百四十一节:海市赌石坊

\begin{this_body}



%1
八转仙蛊屋功德方尖碑前,一只信道凡蛊被方源随手丢下。

%2
在这只蛊虫中,方源坦言相告,将自己之前在小海岛处采集黑油的经过,以及对功德榜的验证,都记录在了上面。后续若有蛊仙回来,必定会得到这个情报。

%3
这个情报没有必要藏着掖着。

%4
一来,这些人早已经有了猜测,二来,这种猜测也很容易验证。

%5
与其这样,反不如在他们没有验证之前,让方源揭露出来。

%6
“或者……更明白点说,这个秘密就不是秘密。乐土仙尊布置这道真传,恐怕就是有意为之,想要导人向善。”

%7
这是乐土仙尊的行事风格。

%8
至此,方源对乐土真传的真实性,再无一丝怀疑。

%9
蛊师、蛊仙留下自己的传承,是这个世界最大的文化特征之一。

%10
每一个传承,都充斥着个人的印记,保留着个人的风格,或者带着强烈的遗愿。

%11
比如花酒行者的遗藏,他就是为了报复古月山寨,纵然身死,也要复仇。所以他留下的光影,复刻当年古月族长的战败,又挖掘地道,最终诱导方源破坏古月山寨的元泉。

%12
又比如八十八角真阳楼,这是巨阳仙尊遗留下来的运道真传,主要的目的是为了整个北原的黄金血脉,维持他们的统治,是巨阳仙尊为他的后代谋求福利。

%13
还有黑凡真传,黑凡原本打算将自己的衣钵传给孙女,可惜的是命运差错,没有如愿。

%14
还有影宗真传,若是魔尊幽魂没有失败,没有被天庭俘虏,这个真传绝不会有。但事与愿违,他被俘虏,方源走马上任,为了增强他的实力,从绝境中争取一线最微弱的希望,这才有了影宗真传。

%15
每一道传承,都不一样。乐土真传也和其他真传大不相同,充满了一种平和、善意、光明、温暖。

%16
“若是哪一天我失败身死,拼尽全力也无任何希望。那么我也会留下传承罢。”方源思绪蔓延,他虽是一个天外之魔,但在这里生存、挣扎、奋斗了这么久,早已经融入了这个世界。

%17
若是他布置出方源真传,必然是要鼓励后来者继续追逐永生。

%18
时间匆匆流逝,七八十天一晃而过。

%19
在这个期间,方源又做了许多任务,功德榜单上他的名字始终名列前茅。

%20
哗哗哗……

%21
潮起潮落,腥湿的海风吹拂在脸上,这一次方源来到了海市。

%22
他接到的任务,就是在这海市上惩治一个奸商。

%23
海市在东海中十分盛行,按照规模划分,统共有小型、中型、大型、超级海市这四种。若按照时间划分,又有临时海市、固定海市之别。

%24
海市乃是修行物资的集散地、交换地,在这片乐土中,同样有着海市。

%25
这是一个位置固定的海市。虽然不是常年开启,但每年的大半时间都敞开门扉。

%26
海市的最核心,是一处小岛。这座小岛有小半年的时间,会被海水淹没,这也是海市关闭的时间。等到小岛显露出来,海市就处于开启的状态。

%27
在小岛之外,还有大量的蛊屋,多是船只模样,相互之间用甲板、铁链等到链接在一起,形成外围海市。

%28
走在甲板上,方源周围都是人流。

%29
有人族蛊师,也有鲛人。

%30
鲛人还很多,几乎占据了一半。这种情况,在五域中可是非常罕见的。毕竟现在是人族为尊,其他的异人都被排挤打压,几乎没有什么生存的空间,夹着尾巴做人。

%31
“来一来,看一看,这是最上好的水晶珊瑚啊。”

%32
“蛊屋河车还剩下三座,要购的从速!”

%33
“收购大明泥,有多少收多少……”

%34
吆喝声、买卖声、讨价还价的声音,密密匝匝,传入方源的耳中,别有一番喧哗和热闹。

%35
大量的蛊师,都在甲板上布置了或大或小的摊子,很少有凡人操持买卖。

%36
方源越过一座又一座的蛊屋,朝海市最中央的核心小岛走去。

%37
他已经调查出来,这一次任务是要严惩奸商,而这个奸商就在中心小岛上。

%38
当然了,完成任务只是方源的次要目的,他最主要的目的则是探听情报。

%39
这片海市最靠近镇魔悔哭海,又集中了大量的蛊师和势力,方源要探听有关悔蛊的消息,希望很大。

%40
虽然来到这里后,过去了这么久,但是方源却一次都没有进入过镇魔悔哭海。

%41
蛊仙这片乐土中,并无太多的行走自由。

%42
就像第一次任务,方源只能在小海岛周围逡巡,有一个距离的极限。每一次任务的地点,都有这样的束缚。

%43
至于回到功德碑前,则很简单,只需要蛊仙心中连续默念三次“回归”即可。

%44
从这一点,方源充分感受到了乐土仙尊的手段。哪怕方源底蕴深厚,修为也在八转层次,但他一直都无法破解乐土仙尊的布置,尽管他从未停止过研究功德碑以及这片乐土。

%45
除此之外,蛊仙之间的沟通也受到了阻碍。首先他们不能沟通宝黄天,也不能联络外界。其次,相互之间还不能随意及时的交流。逼不得已,众人才想出了在功德碑前丢下信道凡蛊的笨方法。

%46
“应该就是这里了。”片刻后,方源停下脚步。

%47
这是一座赌石坊,门上的牌匾写着金玉屋三个字。

%48
金玉屋坐落在小岛的中心地带,代表着这片海市的最高规格。不是随随便便的蛊师,都能步入海岛上来的,这里面就有严格的把守。

%49
当然,凡人的手段怎么可能奈何得了方源呢。

%50
“看来我不妨也来一场赌石好了。”方源暗中笑了笑。

%51
要严惩奸商,并非是要打杀了他,方源当过一段时间的商人,更亲自开过赌石坊,他深深清楚到底要做什么才能让一个赌石坊的商人心痛。

%52
想到这里,方源踏足进去。

%53
心中的一些记忆,在这一刻忽然升腾起来……

%54
“哈哈哈,谁来跟我对赌?圣女都怕了,你们还敢吗?”一个壮硕的鲛人在嚣张的喊叫着。

%55
在他的周围,站着许多的人族蛊师,还有鲛人。绝大多数的鲛人,对他怒目而视,却只能咬牙切齿。

%56
“鲁达这厮,真是可恶,居然在海市上挑衅圣女大人!”

%57
“他自己没有这个胆量,是他背后的寒潮部族族长在为他撑腰呢。”

%58
“这样下去可不好,太损圣女大人的威望了,怎么办?”

%59
蓝鳞、红鳞两位侍卫忧心忡忡。

%60
“有什么关系?”谢晗沫微微而笑,“就要他叫嚣好了,声望损失再多,也动摇不了我的根本。对方使出这样的法子来挑衅我,正说明他们已经慌了。我们无须下场,只需按部就班,即可获胜。”

%61
经她一提点,蓝鳞、红鳞侍卫顿时有一种恍然大悟之感。

%62
“圣女看得最清楚,说的是啊,鲁达就是跳梁的小丑罢了。”

%63
“只是让圣女受委屈,我心底还是有些憋屈。我们鲛人中就不能站出一些人来,抗衡鲁达么?”

%64
谢晗沫笑道:“鲁达毕竟是四转的强者,在这片海市中有十多年的威望。没有鲛人站出来可以理解,我们要宽容才是。”

%65
话音刚落,一道声音就响彻全场:“我来和你对赌!”

%66
是谁?

%67
敢挑鲁达的虎威?

%68
众人纷纷侧目,只见一位人族蛊师越众而出。

%69
谢晗沫楞了一愣,蓝鳞、红鳞侍卫齐声道:“是我们救下来的那个人族蛊师啊!”

%70
鲛人蛊师鲁达也很意外,他皱起眉头,盯着方源,沉声道:“人族蛊师,这是我们鲛人一族的事情,你有必要来趟浑水吗?”

%71
“实不相瞒,你们的圣女救过我一命,此次我就来还她的恩情。”方源态度坚决,毫无畏惧地和鲁达对视。

%72
“这小子……”

%73
“也不枉费我们救他了。”

%74
蓝鳞、红鳞侍卫纷纷点头。

%75
谢晗沫却叹息一声:“我们出去,他是局外人,不能让他白白牺牲。”

%76
蓝鳞、红鳞大感意外,连忙阻止:“圣女大人,这种情况您刚刚也分析过,可不能随意现身啊。一旦现身,就中了对方的算计。”

%77
场中,鲁达面露狰狞之色,怒极反笑:“好!人族蛊师,你既然自己找死,那就不要怪我了。赌斗的规矩我已经说的很明白,来吧。我让你太挑选,看谁赌出来的蛊虫更佳!”

%78
“你先选好了。”方源目蕴神芒,从容笑道。

%79
第一次交锋十分短暂,以方源的胜利告终。

%80
第二次交锋有些漫长,胜利者仍旧是方源。

%81
“人族蛊师,我小看了你。报上名吧,你有资格让我记住你的名字。”鲁达露出凝重之色,对方源另眼相看。

%82
“我乃古月方源,你要记好了。赌石坊我可是也开过的呢。”方源忽然声调一扬,“最后一场了,你请。”

%83
鲁达在沉默中选中一块石头,然后他忽然出手,将场中其余的石块都统统摧毁。

%84
“你耍赖!”方源变色。

%85
鲁达哈哈大笑:“我耍什么赖?我早已明说过了,对赌的双方不能对彼此动手,但这不包括这些石头啊。我现在有石头,并且可以确信这里藏有蛊虫。而你两手空空,所以不管我的蛊虫是什么,是生是死,你都必败无疑。小子,你区区三转修为,就敢来趟浑水,还是在我经营十多年的海市里。呵呵呵,你自裁吧,省得我动手了。”

%86
方源捏起双拳,正要拼死反击。

%87
“且慢。”声音传来,围观的人群分开,谢晗沫俏脸寒霜,稳步走入场中。

\end{this_body}

