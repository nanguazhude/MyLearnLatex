\newsection{忍耐中的巴十八}    %第七百六十八节:忍耐中的巴十八

\begin{this_body}

“杀!”巴家蛊仙咆哮,他身躯膨胀,宛若巨人,用力一撤,血水飚射,竟将一头上古年兽活生生撕烂。呼!大火卷席,将数十头年兽焚烧成火,柴家蛊仙沐浴火海,宛若神将。

方源记得此人,上一世的印象中,此人的炎道仙窍经营得不错,有好几座雄伟的活火山,可以给一个好评。

铁区中沉默不语,手掌翻飞如蝶,金光道道,犀利至极,将一头头年兽斩成两半,轻而易举,如切豆腐。

方源想起了铁区中手中的刃蛊,这可是光阴飞刃杀招的核心蛊之一,上一世是因为大盗鬼手而缴获。这一点靠的是运气,不太可靠。这一世要多注意一点。

“方源魔头,还要多谢你,送给我们这么多的财富。啊哈哈哈!”商虎杖大笑,木道杀招催发出来,陆续将周围的年兽捆绑束缚,拖进自家仙窍当中。

方源很淡定,商虎杖收多少都没有关系,因为不久后连商虎杖本人都会是方源的了。

嗷吼吼……

接连不断的年兽,闯进大阵当中来。南疆蛊仙杀得都手软,但年兽仍旧源源不断!

“怎么会有这么多?!”

“方源这魔头,是想借此消耗我们!”

“打破这座大阵,才是取胜之道啊。”

南疆蛊仙们纷纷传音,都看出此战关键是什么。但要让他们分心他顾,却是有些勉强。

破阵的最大希望,还在陆畏因、夏槎两人身上。

嗷吼!

这个时候,漩涡忽然膨胀数倍,一个庞大的年兽傲慢踱步,进入战场。

方源讶异了一下:“哦,是召出来一头太古年虎啊,看这气势比我上一世的第一头太古年鸡,要强的多了。嗯……是因为我动用了己运真传的手段,增添了自身运势的缘故么?”

南疆群仙则是脸色大变。

“是太古荒兽了!”

“果然这座大阵能够召唤出太古年兽。”

“这大阵究竟是什么?如此凶恶!快快破了大阵啊!”

陆畏因正要出手对付这头太古年虎,铁区中却主动站出来:“让我们来,二位大人还是全力破解此阵,才是明智!”

“或许让我来出手?”巴十八暗中传音。

“不,此时还不是关键时刻,巴兄且先忍耐。”夏槎传音规劝。

巴十八心想,的确是这样,一头太古年虎而已,本方明面上却是有着两位八转蛊仙呢,便继续伪装下去。

铁区中和数位南疆七转联手,迎上太古年鸡。他们都是七转蛊仙中的佼佼者,联起手来,声势赫赫,但交手后,很快就有不敌溃败迹象。

上一世,他们和方源召唤出来的第一头太古年鸡抗衡,暂时势均力敌。

但这一世,方源召唤出来的太古年虎实力却强了许多。

铁区中脸色铁青,心中震撼:“这就是八转级数的实力么,果然是厉害!我竟妄想挑战这样层次的存在……”

“这里有一处阵眼。”忽然,夏槎眼中精芒一闪即逝,她伸手一指,八转杀招催出,玄白奇光绽射,轰鸣声中,大阵剧烈地颤了三颤,这才平稳下来。

南疆群仙大喜,的确是一处阵眼,被夏槎不负众望地摧毁了。

然后,阵眼被摧毁的地方,忽然形成了第二个漩涡巨门,大量的年兽蜂拥而入。

南疆群仙脸上的喜色顿时垮台。

嗷!

一声龙吼,风云变动,太古年龙一跃而入,悍然参入战团。

众仙脸色再变,这是第二头太古荒兽了!

“该我出手了。”巴十八想要出手。

“不要着急,巴十八仙友啊,还让我来吧。”陆畏因传音规劝,他首次正式出手,单手一抓,一道黄褐大手旋即凝成,直接抓住太古年龙,任凭它如何吼叫挣扎,都脱不得身。

陆畏因不愧是乐土传人,举重若轻,上一世他一击之下,擒拿太古年虎。这一世,同样一击就将一头太古年龙轻松擒拿。他虽然没有攻伐手段,但就这一手段,降龙伏虎不在话下,深不可测,令人敬畏。

南疆群仙士气大振,镇守阵眼的影宗以及异人蛊仙们,脸色皆沉。

八转的神威,的确叫人胆寒。

方源却想到了上一世的最终大战,陆畏因真正的战力十分强大,因为他一人导致方源一方受阻极大。此刻陆畏因表现出来的战力,其实还不及他真正战力的一半,即便如此,也令同行的八转多多侧目了。

嘶嘶嘶……

一头太古年蛇,拖动漫长粗壮的蛇躯,吐着猩红的蛇信,钻出漩涡巨门,进入战场。

第三头太古年兽!

咯咯咯咯哒!

一头太古年鸡,神态傲慢,叉腰耸头,也钻了进来。

第四头太古年兽!

一连两头太古年兽的参战,令南疆众仙脸色一白。

这头太古年蛇很不好对付,体格极巨,气势还要超过之前两头。而第四头的太古年鸡身上,则是有野生仙蛊的气息,这就更加危险了。

两头太古年兽钻出来后,海量的年兽顺着漩涡,仿佛潮水大浪喷涌进来。

两方展开激烈的厮杀,蛊仙运用智慧,能催动杀招,相互配合,反观年兽则鲁莽无智,但却有十分庞大的规模。

方源一方则坐镇各大阵眼,静看两方相争。

“这该死的大阵,必须将它破坏,我们才有胜机!”

“其实只要破坏一定程度,让我们能够将此处情报传递出去,必定有大批正道仙友前来支援。”

“这宙道大阵,真是玄妙……”夏槎紧皱眉头。她自从破解了一处阵眼之后,短时间内居然找不到第二处阵眼。

“大阵浑然一体,毫无破绽可言。我在阵道方面建树太少了。或许我应该全力出手一次,使其大阵承受不住,造出破绽来!”夏槎眼中杀机萌动。

陆畏因却是看出夏槎心中所想,连忙劝道:“夏槎大人,切勿中了魔头奸计。方源狠辣狡诈,天庭都奈何不得。影宗掌握过惊鸿乱斗台,这座仙蛊屋虽然毁了,但残余了不少仙蛊,保留在他们手中。你若要全力出手,恐怕会正中方源下怀,为方源所用。”

夏槎犹豫,这时太古年鸡向她逼来。

“好孽畜。”夏槎怒极反笑,身上大量蛊虫气息澎湃喷涌,顷刻间酿成一记仙道杀招。

仙道杀招——春剪!

一柄剪刀飞出,翠绿作色,体大如象,绕着太古年鸡飞舞。

太古年鸡敌不过,被剪刀重创,很快就落入下风。

夏槎展现出惊人的战力,绝不输给天庭成员。眼看着太古年鸡就要惨死在她的手中,忽然,方源调动大阵。

重伤濒死的太古年鸡骤然消失在原地。

“给我臣服罢。”幕后,方源对这头被削弱到极致的太古年鸡下手,一如前世,顺利地将此兽收入麾下。

重生以来,方源就在筹集太古年兽。

他的计划是筹集到十二种不同的太古年兽,虽然明年太古年鸡在光阴长河中会有很多,但这头太古年鸡却是实力超群,方源不想放弃。

方源借助南疆群仙之力,削弱太古年兽,帮助他强行奴役。

夏槎脸色阴沉,不过旋即一笑:“你总算是露出了马脚,再多给你点太古年兽,又有何妨?”

方源挪移走太古年鸡,令夏槎洞悉了大阵更多的奥妙。

不久之后,夏槎勘测出第二处阵眼,并将其摧毁。

南疆蛊仙欢呼起来,然而下一刻毁去的阵眼,再次转变成了漩涡巨门。

战场上,又多了一道年兽输入的渠道。

“怎么会这样?”南疆群仙傻眼。

“实在不行,就只好我现身,真正动手了!”巴十八传音道。

夏槎和陆畏因都劝他。

“不要着急,都已经到了这一步,不妨再隐忍下去。”

“我们要探查清楚这个大阵,关键时刻你来出手,让方源错算,引发破绽!”

巴十八沉默了片刻,方才不甘地回应:“好吧。”

死战持续着。

南疆群仙已有不少人挂彩。

这些年兽他们都要杀吐了,原本针对方源而精心准备的杀招,还有蛊仙之间的绝妙配合,都用在了这些愚蠢的年兽身上,叫他们无奈至极,又疲累欲死。

然而年兽仍旧绵绵不绝,杀不胜杀。

陆畏因、夏槎两人一直对抗太古年兽,虽然艰难,压力如山,但却是坚持了下来。

借助他俩之手,方源又成功地奴役了数头太古年兽。

但这数量比起上一世,要差一些。

方源奴役的太古年兽,都比上一世要强大一些,选择了当中的精英,宁缺毋滥。

最主要的原因,还是他的魂魄修为远远低于上一世,消耗很大,不足以支撑他奴役更多。

南疆一方也有许多进展。

夏槎又摧毁了两处阵眼,这两处阵眼仍旧转变成了漩涡巨门,冲进来的年兽带给南疆群仙更多的威胁。

方源眼眸中闪烁着冷光:“上一世我这年流伏诛阵,上限只有三个漩涡巨门,但这一世经过我的改良,上限增到了五!”

“火候差不多了。”

“上一世是大战了两天一夜,这一世年兽更多,虽然只过了一天一夜,南疆群仙却是被消耗更多。”

“诸位,配合我,催动出此阵的最强手段!”方源陡然传音。

------------

\end{this_body}

