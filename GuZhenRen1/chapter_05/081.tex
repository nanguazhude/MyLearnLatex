\newsection{思谋}    %第八十一节:思谋

\begin{this_body}

“非也非也。”听到楚度的猜测,智道蛊仙田下心摇头不已,“这印记乃是信道手段,根据我的推算,应当是一份信道真传,并且真传的位置不在北原,而在东海。”

“东海,信道真传?”楚度皱眉,心中的期望顿时下降了一大截。

本来剑道真传,他能兼修的希望就很小。

因为他是力道,纯粹单一,一身力道道痕底蕴丰厚,想要兼修其他流派已经十分困难。

不过剑道攻伐强大,又事关历史上的传奇人物“亚仙尊”薄青,楚度自然而然地就涌出许多好奇。

但现在他听到是信道,这种好奇之情就削减了许多倍数。

楚度这样的人物和眼界,自然清楚信道能带给他修行上的种种优势,但他性子直接,雄心壮志,骨子里更喜欢攻伐手段多些。

他号称霸仙,有俗语道:只有起错的姓名,没有叫错的外号。

楚度既然不能兼修剑道,更不可能来修行信道。

全天下,乃至古往今来,能够全派通修的唯有方源一人!

其他蛊仙,要是兼修第二流派,必定分薄精力和时间。若非特殊情景,又有好的仙蛊,否则绝大多数情况都是弊大于利,绝非明智之举。

见楚度脸色,田下心顿知他的想法,不由开口劝道:“楚兄,切不可小看了这道印记。依我推测。只是后来印刻上飞剑仙蛊的。这种信道手段,简直是为所未闻,让我大开眼界,绝不凡俗!”

楚度神色变得郑重起来。

他被这么一提醒,顿时觉得田下心这话不虚!

要是印刻在普通的山石云水之上。并不出奇。但刻印在蛊虫上面,这却极其少见。

皆因这只飞剑仙蛊,乃是剑道七转!

人是万物之灵,蛊乃天地真精。

蛊仙修行,一次次渡劫,将道痕加身。而仙蛊本身,则是一小块一小块的道痕碎片!

不同流派之间的道痕。会相互内耗、掣肘、排斥。这印记是信道手段。也是一丝信道道痕,居然能够在剑道道痕碎片上面,印刻住,并且长存下来。这就像是一艘孤舟,位于海啸中毫不动摇啊。

如此一看,这种信道手段,简直是惊世骇俗!

“楚兄。我和你打个商量如何?”田下心忽然又道。

楚度淡淡一笑。他虽然不是智道蛊仙,但田下心的想法,他当即也是猜到。[看本书最新章节请到棉花糖小说网www.mianhuatang.cc]

楚度摆手,打断田下心的话,道:“田兄,还请勿怪。这飞剑仙蛊本就不是我的,而是我朋友之物。实不相瞒,我这位朋友失踪良久,我有所担心,所以才请你推算他的下落。这上面的仙缘自然也是他的东西。”

智道蛊仙的推算。并非是凭空而得,需要根本的线索。而这些线索越多,推算演化自然越加容易。

所以往往智道蛊仙,对于那些信道手段,也十分渴求。

田下心本来对大多数的信道手段没有什么兴趣,因为他师承田园老仙,智道传承超凡脱色。连带着他眼界都拔高了许多层次。但眼前的这份信道真传着实太过优异,让他这个甘于平淡的蛊仙,也怦然心动。

“既然如此,是我唐突了。”听得楚度直接拒绝,田下心语气无比失落,但还是将手中的飞剑仙蛊归还给了楚度。

这只七转剑道仙蛊,只是用来推算的关键线索,不是楚度支付给田下心的报酬。

当然,楚度请田下心出手推算,付出的报酬也是很丰厚。

“此次推算,未达到楚兄目的。按照规矩,这次的报酬,我退还一半。”田下心留恋地看着飞剑仙蛊被楚度收入仙窍,叹息一声后,又道。

楚度摇头,哈哈一笑:“不必了,虽然没有达成所愿,但我也并非一无所获。田兄不愧是当今北原的第一智道蛊仙,之前的报酬是物有所值啊。在下告辞了。”

楚度号称霸仙,但并非蛮横到不知为人处事。

田下心乃是当今北原可数的智道大能,以后极可能还得请他出手相助,楚度怎可能轻易恶了这层关系?

“这……”田下心犹豫。

楚度却已走了,他行事雷厉风行,十分干脆利落。

他走的时候,其实心底还很开心。

七转飞剑本身就可牵动蛊仙神经,现在上面又有信道真传线索,楚度推己及人,不管怎么想,都觉得方源不会放弃。如此一来,他就更有了要挟方源的可能!

“能够牵引出狂蛮真意的方法,我一定要得到它!”楚度心中的火焰一直熊熊燃烧。

田下心望着楚度离去的背影,心中无奈感叹。

若换做他人,他说不定就要出手抢夺了。但对方却是霸仙!

明明楚度只是修行力道,田下心修行智道,但田下心终究是无法鼓起勇气,去对付谋算楚度。

力道流派日落西山,智道却一直高贵珍稀。但自古以来,从未以流派论高下,只有最强的蛊仙,没有最强的蛊虫或者流派。

人是万物之灵,蛊乃天地真精。对于人而言,蛊虫亦不过是工具罢了!

时间流逝,北原风起云涌。

围绕着黑家的一场包围网,正在渐渐收拢。

正如方源推测的那样,黑家以一己之力独抗几乎整个北原蛊仙界,哪里有什么胜算?

尽管黑家蛊仙激烈反抗,但仍旧节节溃败,连连失利。争斗的血腥气息,黑家所代表的庞大利益,吸引了越来越多的鬣狗恶鲨。

黑家大本营,铁鹰福地。

此时。黑家蛊仙齐聚一堂,一片愁云惨淡。

黑家四大太上长老,居于最高位。黑城已被方源俘虏,他的好友黑柏身上带着伤。除此之外还有其他六位黑家蛊仙,几乎都是六转修为。

气氛压抑得很。

“家族大难临头。诸位都是我黑家肱骨梁柱,都说说看,如何才能渡过此劫?”太上大长老首先开口,打破场中沉默。

但黑家蛊仙你望我,我看你,一时间都未有人发言。

这已经不是第一次商议。

之前商议时,他们踊跃发言。有人建议。必须强势还击。对胆敢将触手伸到黑家地盘上的宵小之徒,给与严厉的制裁和打击。当然根本的想法,还是求得和平。

任何的和平,都是打出来的。

黑家蛊仙并不愚蠢,自然清楚其中道理。

可惜道理虽然清楚,黑家采用之后,却双拳难敌四手。虽然十分强势还击,却节节败退。弄得现在黑家上下,几乎个个带伤。更有一位六转蛊仙,已经陨落。

太上二长老叹息一声:“若非我四位早年因修行青城纵横,出了差错,相互之间不可分离。否则的话,此刻情景必好上许多。”

黑家占据的资源不在少数,但黑家蛊仙稀少,绝大多数只有六转修为。对于围攻的各大势力而言,黑家是兵力稀少。战力又不强,自然无法镇压局面。

“都怪黑城!此人乃是我黑家的罪魁祸首!!他养的什么女儿,出了这么大的纰漏。本人还弄丢的族中的仙蛊屋黑牢,真是死不足惜!”有黑家蛊仙痛声咒骂。

“是啊,是啊。”

“黑城罪大恶极……”

一时间,黑家蛊仙纷纷开口,群情激奋。

黑城弄丢了仙蛊屋黑牢。的确是一项巨大过失。否则的话,有黑牢在手,配合四大太上长老的青城纵横,局面绝不会如此被动。

太上大长老沉默。

黑牢本是由他所有,借给黑城去用的。黑城始终,黑牢丢失,太上大长老也有用人失察的责任。

平时的时候,太上四大长老都对黑城信任有加,让他代为管理族中事务。

黑家蛊仙们此刻的痛骂和批判,其实也是隐隐对四大太上长老的不满,只是不可直接发泄出来而已。

骂了片刻,太上三长老终于不耐,出声道:“好了,此时此刻,事情已经发生,谩骂又能起得了什么作用?本家存亡的危机,若是因为这些咒骂而减轻几分,那老夫早就骂破了天去。大家都是姓黑,身上流着的都是相同血脉,家族若亡,覆巢之下无完卵的道理,你们都应懂的。”

此言一出,骂声渐消。

太上四长老咳嗽一声:“对当今局面,谁有想法,都可说说。”

众仙沉默。

双方实力差距实在太大了,难有什么有效对策。

“在下有些见解。”这时,一位六转蛊仙忽然开口。

众人视之,乃是黑柏。

此仙主修木道,六转修为,年轻时号称黑家石人,拙于言辞,朴拙内秀。平时和黑城走得最近,双方交情深厚。

“说吧,我们洗耳恭听。”

“倒要看看你黑柏有什么独到看法?”

蛊仙们语含冰锋,他们对黑城十分痛恨,连带着黑柏也不受待见。

黑柏面色不变,似乎一块石头,他低沉开口道:“如今局面,在下也无需多言,诸位都一清二楚。不仅是正道、魔道蛊仙,就连散修的蛊仙,都纷纷现身,想在我黑家身上咬下一块血肉。究根结底,还在于利益二字。”

“墙倒众人推,本家存亡首当其冲,乃是重中之重。只有能存活下来,什么资源、利益、名声,都是外物,皆可舍去。”

“你这是主张投降吗?可惜本家出了黑城、黑楼兰这样的叛徒,王庭福地毁了,你觉得其他黄金家族会轻易绕过我们?”有蛊仙当即冷笑讽刺。

黑柏缓缓摇头:“黄金家族自然不会,不过当今北原,却有一人,可助我黑家脱离此劫。”

“谁?”

黑柏缓缓吐出一个名字。

众仙闻言,皆是惊异。

ps:大家可以猜猜这个人是谁。今晚蛊真人微信公众号上推送一个书友制作的风花雪月劫图,欢迎大家一起鉴赏。(未完待续 \~{}\^{}\~{}。)(未完待续。)

\end{this_body}

