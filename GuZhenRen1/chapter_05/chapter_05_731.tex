\newsection{获取偷生仙蛊}    %第七百三十四节:获取偷生仙蛊

\begin{this_body}

%1
琅琊福地。

%2
一大群的异人蛊仙,守候在传送仙阵旁,翘首以盼。

%3
忽然,仙阵运转起来,发出璀璨耀眼的光辉。

%4
异人蛊仙当中,站在最前面的琅琊黑毛地灵顿时大喜,开口喊道:“回来了,回来了!”

%5
站在他身后的有雪民、石人、墨人蛊仙,此刻的神情都有些复杂。

%6
琅琊福地刚刚遭受天庭的入侵,碍于四族大联盟的盟约,这三大异人族群都有帮衬毛民,守护琅琊福地,抵抗天庭的责任。

%7
琅琊地灵向他们求援的时候,这三家异族都老大的不愿意,一肚子的苦水,对于签订这份盟约万分后悔。

%8
但后悔又有什么用呢?

%9
盟约就在自己身上,违背不得。

%10
就算千方百计地成功违背了盟约,又能如何呢?

%11
如今人族乃是五域的统治者,当之无愧的霸主,异人族群生活空间非常狭小,只有相互联合起来,才能互利共存下去。

%12
这个战略是绝对没有错的。

%13
只是三家异族万万没有想到,琅琊派竟然惹来了天庭!

%14
天庭是何等的存在!

%15
这可是人族的第一势力,屹立数百万年不倒的庞然大物啊。

%16
和天庭相比,四族大联盟太过渺小了,小胳膊小腿完全经不起折腾啊。

%17
所以,当琅琊地灵向这三家求援的时候,这三家都很不愿意,推三阻四。

%18
没成想,就在这个时候,方源在宝黄天挂卖雷鬼真君的三根胸骨,此番战绩仿佛飓风一样,充当着异人蛊仙的身心。

%19
他们看到了一丝对抗天庭的希望。

%20
但更重要的是琅琊福地也有了八转级别的战力!

%21
如果拒绝琅琊地灵的求助,三族除了要面对盟约的严惩之外,还要面对方源这样可敌八转的蛊仙。

%22
心中的天平彻底倾斜,首先是雪民改变立场,表示全力支持琅琊派,随后墨人、石人也改口,愿意真正出力。

%23
“我吸纳方源充当客卿太上长老,实在是太正确了!”琅琊地灵达到了目的,心中大为快慰。

%24
他得到方源的消息之后,知道他要回来,便灵机一动,带领这些异人蛊仙前来迎接方源。

%25
光芒散去,方源的身影出现在众人面前。

%26
异人蛊仙脸色各异,琅琊地灵首先和方源打招呼。

%27
方源扫视一眼,心中暗笑:“一如上一世的景象。说起来,琅琊地灵也有些政治手腕,懂得利用我来打压其他的盟友呢。”

%28
但方源不想寒暄,平白无故地浪费时间。和上一世一样,他只对雪儿点了点头,便起身离开。

%29
这个小动作,却是让雪民一族的异人蛊仙们大为兴奋。

%30
也让石人、墨人两方吃味不已。

%31
因为和雷鬼真君一战,方源在众仙的心中地位大大提升,不过此时方源心如止水。

%32
重生归来,他的战力再次回到七转巅峰一级,和天庭比较起来,根本算不了什么。

%33
事实上,就算是上一世的八转巅峰战力,又能如何呢?

%34
方源和琅琊地灵私下一番交流,琅琊地灵归还了方源道可道仙蛊,并告知方源天庭果然种下了手段,琅琊福地残留了许多不利道痕。

%35
琅琊地灵显得非常担忧。

%36
因为他已经拼尽全力,但都拿这些道痕没有办法。

%37
方源心知肚明,知道这些道痕便是天庭大名鼎鼎的星投杀招的一部分,被凤九歌故意留下。

%38
上一世,天庭发动第二次攻势,这些道痕就化作漩涡通道,直接传送了四位八转蛊仙进来。

%39
此战方源虽然获得了定仙游七转仙蛊,但暴露了阎帝杀招,还有算不尽的身份,更永远失去了琅琊派这个盟友。

%40
最终,方源撤退,天庭也不得不撤退,长生天做了得利的渔翁。

%41
方源故作不知,回应琅琊地灵道:“太上大长老切勿太过焦急,还是先让我推算一番,好好谋划,看看如何应对眼下局面。”

%42
琅琊地灵叹息一声:“也好。”

%43
他有点奇怪,方源居然对长毛真意只字不提,这和方源往昔的性情不符啊。

%44
不过琅琊地灵也不想提醒方源,毕竟对于长毛真意,他虽然答应赔偿给了方源,但心中是非常不舍的。

%45
此时的智慧蛊,还在琅琊福地当中。

%46
方源便来到智慧蛊面前,在这里待了三天三夜,利用宙道分身不断推算。

%47
这一次重生,和之前相比,有些特殊。

%48
重生折返的时间较长,有数年的光阴。

%49
不过这也很合理。

%50
因为当时方源的宙道分身已经有了七转修为,春秋蝉也提升到了七转,并且又用了春秋必成这样的杀招。

%51
方源本体被杀死,宙道分身自爆,春秋蝉动力远胜前例,自然就重生到更早一些的时间了。

%52
至于最初的重生,方源只有六转修为,但回到了五百年前,那是天意作祟,乃是特例!

%53
“其实在红莲真传中,也有春秋准定这样的杀招,可以重生到过去特定的时间点。但是这个杀招不能保证重生的成功,因此我才选择春秋必成杀招,杜绝失败的可能性。”

%54
方源看了一眼春秋蝉。

%55
此时的春秋蝉,已经不是上一世的七转,而是回到了六转。

%56
并且因为刚刚重生,状态也不佳,十分萎靡不振,身躯焦黄如枯木,需要时间休整。

%57
“红莲真传中,有不少杀招可以令春秋蝉加速复原。春秋蝉不愧是红莲魔尊的本命仙蛊,效果真的太强大了。”

%58
方源心中感叹不已。

%59
如今的他,早已经不是凡人蛊师,因此没有被春秋蝉撑破空窍的隐忧了。

%60
照这样发展下去,不用方源加速春秋蝉的复原,等到数年后的中洲炼蛊大会,春秋蝉就能自动复原。如此一来,方源又能再次运用春秋必成杀招了。

%61
“但天庭如此强势,我就算能再次重生,若是找不到战胜天庭的法门,重生多少次也是枉然。”

%62
回想上一世,方源心头像是压上一块巨石。

%63
天庭展现出来的底蕴,绝对是深不可测,震撼世间!

%64
仙墓中的蛊仙究竟有多少,方源到现在也没看到极限。

%65
龙公之强,绝对是亚仙尊战力,历史上恐怕只有薄青才能有实力对抗。方源自己都被龙公所杀,龙公简直强势得一塌糊涂!

%66
当然,尊者始终是无敌的存在,就算是龙公在他们的杀招面前也不够看。

%67
但这样一来,尊者之间隔空交手,各自的谋划,令原本就杂乱的局面变得更加复杂。

%68
尊者的伏笔层出不穷,一次次影响整个大局,方源现在回想起来,都一阵阵头疼。

%69
除了龙公之外,还有紫薇仙子、陈衣、厉煌、清夜、袁琼都等人。

%70
这些蛊仙都是精锐,都容不得一丝轻视。

%71
更可怕的还是他们的精神,为了天庭,他们能不计牺牲,前仆后继。

%72
那种对天庭,对人道的疯狂信仰,让方源都感到一丝丝的忌惮。

%73
“找不到战胜天庭的方法,即便有无数次重生的机会,都是虚的。为了战胜天庭,此生我要竭尽全力,不放过任何一个提升自己的机会!”

%74
“我得好生谋算一番。”

%75
方源沉下心思进行推算。

%76
毫无疑问,重生是有技巧的。

%77
方源虽有前知的优势,但这只是一层记忆。他现在已经和之前不一样,是一个名传天下的七转第一人,一举一动都有广泛的影响。

%78
若是他为了蝇头小利,改变了许多事情,那么影响又会再度扩大,令现实发生巨大改变,令方源的重生优势暴降。

%79
但若他因此束手束脚,害怕改变,那么自我提升不够,到头来仍旧不是天庭的对手。

%80
这其中需要把握好平衡。

%81
还有关键的一点,就是要防备其他蛊仙过早的知道方源又重生了。

%82
这个秘密方源保存得越久,对他就越有利。

%83
方源这一番谋划,耗费了七八个时辰,总算是理清头绪,有了一个庞大而又精致的计划。

%84
方源唤来琅琊地灵:“你手中有多少仙蛊?把名单列出来给我看看,我思前想后,觉得炼道真意对我作用不大。若是有合适的仙蛊,兴许我会放弃炼道真意,选择这些仙蛊。”

%85
“啊!”琅琊地灵惊呼一声,大喜,“这好,我立即将清单列给你。”

%86
对于琅琊地灵而言,炼道真意的象征意义十分重大。就如同宿命蛊对于天庭,除了本身的功效威能之外,更有精神上的价值。

%87
只是之前,琅琊地灵弄丢了方源的荡魂山,碍于盟约,不得不将炼道真意赔偿给方源,其实他是很不情愿的。

%88
方源看了清单,发现偷生仙蛊赫然在内!

%89
他暗中点了点头,感到满意。从这一点就可看出来,琅琊地灵没有多少的隐瞒。

%90
之前方源用幽魂真传,和琅琊地灵交换长毛真传,双方都有保留,压箱底的东西都没有交换。

%91
琅琊派中,长毛炼道真意、长毛炼道大阵、四海皆准、炼水、偷生仙蛊、天婆梭罗、长生天求援之约等等,都是扣在手里,不会轻易拿出来和方源交换的。

%92
上一世,方源从琅琊地灵手中获得了炼道真意,令自家的炼道境界攀升到了准无上大宗师境界,一步登天!

%93
这样的选择并没有错,炼道境界带给方源极大的帮助。

%94
但现在方源重生归来,境界和上一世一致,仍旧是炼道准无上大宗师。同样的一份炼道真意,对于方源而言是无用的。

%95
这是因为方源本体的境界,向来是和宙道分身等同共享的。

%96
不像魔尊幽魂的分身,方源打造宙道分身的最大初衷,就是让宙道分身沐浴智慧光晕,帮助他本体推算。这份工作,没有境界怎么行?

%97
如今虽然只是宙道分身重生,但宙道分身境界一如以往,方源的本体境界自然也跟着飙升了。

%98
所以这一世,方源做了一个不同的选择:“我选择偷生仙蛊,还有这些……嗯,宙道的仙蛊,都给我罢。”

%99
琅琊地灵愣了愣,旋即干脆利落地答应下来:“好!”

\end{this_body}

