\newsection{再添八转战力}    %第四百零九节:再添八转战力

\begin{this_body}

%1
太古年兽钓来阵的防御很强,遭受太古年猴重重一击,仍旧不动如山。

%2
但这更加激发了太古年猴的凶信。

%3
轰轰轰!

%4
它高举双拳,不断打砸下来,拳头宛若流星,一次次暴击钓来仙阵。

%5
仙阵开始摇晃,内部更是剧烈的震动,伴随着太古年猴的每一次重击,都会损失大量的凡蛊。

%6
白兔、黑楼兰、影无邪都在迅速地补充凡蛊。

%7
有些蛊阵,损失一成的蛊虫,就会崩溃。有些更优秀的蛊阵,即便损失一半的蛊虫,仍旧会残存。

%8
而太古年兽钓来阵,只要核心仙蛊不失,并且超过六成尚在,它就能继续坚守在原地。

%9
不过守久必失,方源等却不能任由太古年猴如此恣意攻击下去。

%10
仙道杀招——白相!

%11
下一刻,仙道蛊阵打开一道缝隙,白凝冰电射而出。

%12
咻!

%13
她速度极快,宛若一道雪白的冷箭,飞射向太古年猴。

%14
双方的距离本就很短,白凝冰又如此极速,太古年猴根本来不及反应,咽喉就遭受白凝冰重重的轰击。

%15
太古年猴稍稍后退一小步,感受到了疼痛,顿时大吼一声,手臂回收,两只巨手向白凝冰抓去。

%16
瞬间,白凝冰只感觉眼前一黑,两只巨大的猴爪像是两片庞大的黑云,盖压下来。

%17
白凝冰临危不乱,冷哼一声,旋即飞射而出。

%18
太古年猴的两只手从上往下笼罩而来,而白凝冰则直接向下俯冲。

%19
本来,依照她的速度,完全可以摆脱猴手的追击,但古怪的是,当她刚刚冲到太古年猴的腹部时,她就被追上。

%20
太古年猴巨手狠狠一握,将白凝冰像是捏着一颗玻璃球般,捏在手中。

%21
然后它的猴手死死地握紧,不断碾磨了几下,这才展开来。

%22
白凝冰若是平常状态,势必已经被捏成了肉酱血泥。但她开战之前,就已经催动了白相杀招,整个人化为一具高达一丈五六的白色冰霜巨人。身着冰甲,三头六臂,赤着双脚,脚下踩着两朵淡蓝寒云。

%23
太古年猴展开巨手,白相已经被碾碎成渣滓。

%24
“渣渣!”

%25
太古年猴见铲除了敌人,兴奋狰狞地大笑一声,手掌一抖,将手心中的白相碎屑抛飞出去,然后它再次用拳头砸在仙道蛊阵上面。

%26
白相碎渣在半空中飘飞,很快,相互汇聚,见风而涨,几个呼吸之后,还原成完整的白相,一丈五六的小巨人形态。

%27
白相杀招的玄妙之处,便在于此。

%28
化为白相的白凝冰,已经不再是血肉之躯,哪怕只剩下一个指甲盖大小的残躯,也能转瞬重生,恢复如初。

%29
“不过这一次,我不仅飞行的速度慢了,而且白相的恢复速度也变慢了很多。”白凝冰一边估算着,一边再次飞向太古年猴。

%30
太古年兽身负丰富的宙道道痕,一举一动都有宙道效用,在周围刻印宙道道痕,不可小视。

%31
仙道杀招——彻冰刀!

%32
轰!

%33
寒气四溢的锋锐冰刀打在太古年猴的肩头,却看不破太古年猴的表皮。和太古年猴比较起来,巨大的冰刀,也宛若成了一柄小小的水果刀。

%34
太古年猴再次捕捉白凝冰,白凝冰试图躲闪,却屡次失败。

%35
不过她不管灭亡多少次,只要还有残渣,就能迅速复原。

%36
这也是提前侦查了太古年兽,知道它身上没有寄生野生仙蛊,否则的话,白凝冰不会如此大胆,去亲身试探太古年兽的战力。

%37
太古年猴身上有着大量的野生凡蛊,但这些蛊虫,对于白凝冰而言不算什么。

%38
真正具备威胁的,是太古年猴本身。

%39
十几个回合下来,白凝冰的速度越来越慢,白相的恢复时间也越来越长。

%40
反观太古年猴,却是受伤极其轻微。身上毛皮沾着几块冰霜,抖一抖,就能抖落下来。白凝冰的攻势,也能打破它的血肉。不过太古荒兽的恢复能力实在太强,几个呼吸,这些浅薄的伤口,就会复原。

%41
饶是白凝冰也不由地有些气馁。

%42
她感觉自己就像是一只烦人的苍蝇,围绕着太古年猴。然而太古年猴却是将更多的注意力,放在仙阵上面。

%43
轰轰轰!

%44
很多时候,太古年猴甚至任凭白凝冰进攻,自己尽管打砸仙道蛊阵。

%45
“这头太古年兽实力不俗!”方源观战片刻,得出结论。

%46
就像八转蛊仙之间,也分高低上下一样,太古年兽之间也有实力的差别。眼前这头太古年猴实力不俗,让白凝冰奋起全力,都无法造成巨大伤势。

%47
白凝冰虽然天资卓绝,有十绝体之一的北冥冰魄,更兼白相真传,但底蕴方面,完全不能和动辄千万年寿命的太古年猴两相比较了。

%48
太古年兽钓来阵的损失越来越大,方源等人补充蛊虫的速度,渐渐跟不上消耗。

%49
随着太古年猴被白凝冰撩拨得越加爆炸,打砸频率加剧,就连仙阵表面都浮现出了丝丝裂纹。

%50
作为最强的战力,方源并无立即出战的想法。

%51
让太古年猴攻击太古年兽钓来阵,这也是一项战术,可以削弱它的力气。

%52
太古年兽当中,根据形态不同,年兽各个方面的表现也不同。要论力气,太古年猴还只是所有形态的年兽中,处于中游的一种。相同的底蕴下,太古年虎、太古年牛、太古年龙的力量,要比太古年猴大得多。

%53
但现在即便是太古年猴,仙道蛊阵也有支撑不住的趋势。

%54
“妙音,你去支援白凝冰。”方源下令。

%55
妙音仙子点点头,下一刻,便飞出仙道蛊阵。

%56
已经观战良久的她,深知太古年猴的凶猛,不敢学习白凝冰近战,而是立即远远推开。

%57
仙道杀招——勾月!

%58
她的眼眸中,各自浮现出一弯勾月,盯着太古年猴的眼睛。

%59
盯了半天……

%60
太古年猴居然没有反应,就好像妙音仙子完全没有出招一样!它仍旧在打砸仙道蛊阵。

%61
妙音仙子额头浮现冷汗。

%62
太古年猴常年畅游光阴长河,一双眼睛在河流中饱受压迫,充斥着浑身上下最为密集的宙道道痕。

%63
妙音仙子的目击之术,根本无法见笑。

%64
“好。”妙音仙子被激起好胜之心,催动一记仙道杀招——余音绕梁。

%65
此招并非攻伐杀招,而是辅助他用。

%66
然后,妙音仙子口中低吟,婉转霏糜。

%67
仙道杀招——柔骨音。

%68
这音乐曲调,飞到太古年猴的双臂上,然后在余音绕梁的辅助下,纠缠在太古年猴的双臂上,久久不散。

%69
受到这个声音,太古年猴的臂骨开始出现柔软的征召。

%70
打砸中的太古年猴,微微一顿,望着自己的双臂,面现疑惑之色。

%71
然后……

%72
它又继续轰炸,拳头宛若流星坠地,每一击都让仙阵狠狠地颤抖一下。

%73
妙音仙子眼皮直跳。

%74
她辛辛苦苦连催两记杀招,看来只是让太古年猴有些不得劲的感觉而已。

%75
“八转、七转的差距,实在太过巨大。我等攻击,简直是如同蚍蜉撼树。”

%76
“数千年来,能够以七转抗八转的,偌大天下,也只有方源和凤九歌二人而已啊。”

%77
相似的感叹,萦绕在影宗众仙的心头。

%78
仙道杀招——赤红线。

%79
仙道杀招——愤怒的小鸟。

%80
仙道杀招——火焰凤鸟。

%81
黑楼兰也飞了出来,炎道杀招已经熟稔至极,行云流水般地催发出来,轰炸在太古年猴的身上。声势挺大,但是伤害更低。

%82
方源、影无邪、白兔姑娘则仍旧坚守在仙道蛊阵当中。

%83
“七转杀招,即便精良,对于八转存在而言,难有良效。”这点方源深有体会,之前他和武庸,和凤九歌交手,依靠逆流护身印造成的伤害往往最高。

%84
方源凭借种种手段,可以力战八转蛊仙,但是要斩杀八转,很难很难。姑且不论八转蛊仙逃走的情况,单就方源目前的攻伐杀招,都是以七转仙蛊为核心,典型的攻弱守强。

%85
他不行,影宗其他蛊仙就更是如此。

%86
唯有影无邪的引魂入梦,因为独步天下,对八转蛊仙有着极强的威胁。

%87
但此时方源却不会动用。

%88
一来,和太古年猴交手,可以练兵,让大家熟悉和八转存在交手的感觉。

%89
二来,要在正面击败太古年猴,更有利于待会方源催发百八十奴杀招。收服之后,这头太古年猴也会更加听话一些。

%90
“火候差不多了。”又战片刻,仙道蛊阵濒临崩溃,方源飞了出来。

%91
仙道杀招——万我。

%92
仙道杀招——上古剑蛟变。

%93
仙道杀招——逆流护身印!

%94
几个呼吸之后,方源变成上古剑蛟,银鳞璀璨,夭矫不群。

%95
但是和太古年兽比较起来,他不过只是成年人身旁的一条小蛇。

%96
没有关系。

%97
接下来,方源全力催动一只律道仙蛊——

%98
大!

%99
大、大、大。

%100
上古剑蛟身躯疯狂涨大,很快就变成不输给太古年兽的巨大体型。

%101
太古年猴警惕地看着方源。

%102
方源直接冲上去,和太古年猴厮杀。

%103
方源虽有力道手段,但力气方面远不如太古年猴。律道仙蛊大只是让他体型变巨而已。

%104
方源时常被太古年猴击飞,但太古年猴却是伤势更重更多。

%105
全然是因为它轰炸上古剑蛟,被逆流护身印逆反攻势,全部返回,打在他自己的身上。

%106
这一场大战,持续了数个时辰。

%107
方源主攻,其余人辅助。即便是七转杀招难有良效,但积少成多,也是不容小觑。

%108
太古年猴伤势沉重,气喘吁吁,想要逃跑。

%109
它撕开虚空,要逃进光阴长河中去。但却被方源变化的上古剑蛟,死死缠住全身。

%110
太古年猴受阻,逃窜不出。

%111
最终,方源连续催动百八十奴三次,终究让它低下了头颅,臣服在方源的脚下。

%112
太古年猴,收服!

%113
继失去上极天鹰之后,方源终于再添八转战力!

%114
备注:花好月圆人长久,蛊真人在此恭祝大家,中秋节快乐!!

\end{this_body}

