\newsection{关键灾劫}    %第五百九十九节:关键灾劫

\begin{this_body}



%1
一颗颗的魂核,如同鸡蛋大小,通体深蓝,近乎黑幽,从半空中洒到荡魂山上。

%2
魂核乃是一头魂兽的精髓所在,魂兽死后,尸体无用,会化为烟灰散去,只会留下魂核。

%3
魂核凝聚了大量的魂道道痕,乃是上佳的蛊材。

%4
魂核有大有小,普通魂兽的魂核只有米粒大小,到了六转才如鸡蛋。七转更大一些,八转级数的太古魂兽所拥有的魂核,一个个都大如车马。

%5
房家经过一番商议,最终暗里答应方源来信要求,秘密送来大量的魂核。这些魂核基本上都是六转蛊材,少数是七转魂核,八转魂核没有一只。

%6
方源手中倒是有两颗八转魂核,如今收藏着,并不想用于荡魂山产出胆识蛊。

%7
八转的仙材,一般都是用于炼制八转仙蛊,或者是搭建大阵。只是用来产出胆识蛊,未免太大材小用了些。

%8
荡魂山上光辉摇曳,魂核被震荡成粉末,洒下无数精华,沉入荡魂山体。须臾,一颗颗胆识蛊就凝结而出,长势喜人。

%9
方源目睹这样情景,暗暗点头:“如此一来,我的魂道修行又能走上正轨,大步前行了。”

%10
魂道底蕴对方源而言,十分重要。

%11
奴役太古年兽,需要它,并且底蕴越是深厚,方源能够奴役的太古年兽就越多。

%12
阎帝杀招消耗的就是魂道底蕴,每一次有智道蛊仙推算方源,阎帝杀招发动,为方源遮挡,方源的魂魄底蕴就会消耗一部分。

%13
又比如燃魂爆运,亦是如此。还有引魂入梦杀招,这个杀招方源已经研究透彻,牵引他人魂魄入梦有失败的可能。但方源的魂魄底蕴越高于目标,杀招的成功率就越大。

%14
方源直接以本名去信,抛去算不尽的这层伪装,房家的反应并没有出乎他的意料。

%15
对于一个超级势力,方源这种孤家寡人,是很难侵占它的地盘,顶多是抢夺一些资源。但其他超级势力却不同,他们会长期霸占乃至彻底夺取资源点,派遣力量来驻守。

%16
房家的最大威胁,在于其他的虎视眈眈的正道势力。

%17
“并且西漠这块地方,有些特殊。西漠蛊仙比其他四域,更容易妥协一点。”

%18
这倒不是因为西漠蛊仙的骨头更软,而是地域环境造就。

%19
西漠和其他四域不同,绿洲如星点缀浩瀚沙漠。蛊师之间相互斗争,一旦将绿洲打坏,就再无生存的根基,极容易落个同归于尽的下场。另一方面,绿洲的产物各个不同,蛊师的修行,更需要互通有无。所以蛊师之间常常组建商队,一路上风尘仆仆,突破艰难险阻,不同的蛊师之间,以及不同的势力之间就有许多的合作。

%20
所以,长久以往,养成了西漠蛊师之间善于妥协,主动退让,共同求存的习惯。

%21
任何的蛊仙,都是从蛊师一步步修行上来。这种习惯深入骨髓,已经是西漠蛊仙性情中的一部分了。

%22
方源前世在西漠中流浪生活过一段时间,深谙西漠人的性情。

%23
假设方源俘虏的是西漠蛊仙,正道势力妥协的速度必然比南疆更快一些,并且态度还要更现实、端正。

%24
方源若是俘虏北原蛊仙,那就又不一样。北原蛊仙悍不畏死,甚至会认为被赎回来是一种羞辱。方源要在北原俘虏蛊仙,会很头疼的。

%25
接下来的日子里,方源一边重新展开魂道的修行,一边勒索南疆正道势力,同时他还继续搜魂南疆俘虏。

%26
方源把搜魂的重点,放在夏槎的身上。

%27
虽然搜她的魂魄,过程艰难了一点,但好歹不是没有进展。方源的仙窍时间又快,日积月累之下,成果也颇多。

%28
很快,夏槎的种种仙道杀招,就被方源搜刮出来。

%29
夏槎不愧是夏家的太上大长老,围绕着春夏秋冬四大仙蛊,共开发了两套杀招,每一套都有四招。

%30
第一套杀招主要用来攻伐,分别有春剪、夏扇、秋毫、冬裘四大招数。

%31
其中春剪、夏扇方源已经见识过了,两招威力凶猛十足,尽展八转之威。

%32
随后的秋毫杀招,乃是侦查杀招,极大地弥补了方源侦查方面不足的短板。

%33
而冬裘杀招,则是防御杀招,效果也十分不俗。

%34
第二套杀招则是用来经营仙窍的,分别是春耕、夏耘、秋收、冬藏。

%35
每一招都巧妙玄奇,更厉害的是,这四招联合起来,威能翻倍,用来经营仙窍效果拔群。

%36
除此之外,当然还有其他招数,用来治疗或者腾挪躲闪,也都很精妙,但和这两套杀招并不能比。

%37
其实,构思仙道杀招是相当艰难的事情,十分考较蛊仙的天赋和才情。

%38
中洲蛊仙凶雷恶人花了数年的时间闭关,这才推算出杀招雷神子。就这还是在他参考了血神子仙蛊方的基础上。

%39
正常情况下,一位蛊仙推演构思杀招,没有智道的手段辅助,平均得有五六年的时间。

%40
从万千的蛊虫中挑选出合适的蛊虫,并不容易。蛊虫之间组合起来,种种选择更是浩如烟海。就算构思出来,没有实用价值也是不行的,后遗症太大也不能接受。完成杀招之后,还要在接下来的年岁里,不断地加以改良,这就更难了。

%41
当然,五六年的时间并不是说,蛊仙什么事情都不干,就专门一心一意地推算杀招。

%42
蛊仙若是有参考之物,比如仙蛊方,又比如前人的传承等等,也都能缩短这个时间。

%43
但这个时间,是以年为单位来计算的。

%44
方源每一次似乎推算杀招,改良杀招都轻而易举,主要还是智慧光晕,九转智慧蛊可不是盖的。当然,还有一个重要因素,那就是他雄厚的各种流派的境界底蕴。

%45
从夏槎身上获得的仙蛊,以及仙道杀招,极大地补充了方源宙道方面的实力,可以说是突飞猛进,如火山喷发似的暴涨。

%46
但方源仍旧面临着一个颇为尴尬的问题。

%47
“春、夏两只仙蛊都为八转,非得有八转仙元才可催动。我却只是七转蛊仙,不借助其他手段,连炼化的资格都没有。”

%48
“所以还是要将修为提升到八转!”

%49
“如此一来,不仅是能理由春、夏两只仙蛊,还有之前的慧剑仙蛊也能够用上了。”

%50
方源专门针对晋升八转一事进行了推算。

%51
数天后,他推算完毕。

%52
抛开天灾地劫不算,七转晋升八转,总共要渡过三场浩劫,并且这三场浩劫的威力会越来越强。

%53
方源本身的进展,也已经相当接近最后第三场浩劫了。

%54
浩劫本身的威胁并不大。

%55
哪怕天意将其调整到最大极限,方源掌握着石洞天机杀招,已经牢牢占据主动。

%56
方源担心的不是浩劫,而是人劫。

%57
要渡浩劫,必定要落下仙窍,沟通内外,汲取天地二气。

%58
这个时候,方源的位置是直接暴露在天意之下的,同时防备他人推算的难度也随之暴涨。

%59
天庭绝不会放过这个机会,也一定对方源的修为有所估算。

%60
若是方源在南疆渡劫,南疆的蛊仙知道消息后,就算是俘虏在方源手中,也绝对会找方源麻烦,除方源而后快!

%61
因为除掉方源,永无后患,眼光放远去看,南疆正道收获的利益必然更大。

%62
方源的龙鱼生意,已经被天庭阻截,收入暴降。从这件事上来看,可有证明魔尊幽魂已经顶不住,天庭搜魂进展颇多。随着时间推移,天庭获得的情报越多,方源也就越被动,幽魂真传中的种种手段,实用价值将持续暴降。

%63
“眼下,天庭是我的首要大敌。和天庭争锋,关键的一点在于宿命蛊。”

%64
“天庭若彻底修复宿命蛊,我将再无胜机。若被破坏,我方有继续周旋下去的可能。”

%65
“而破坏宿命蛊的关键,在于红莲真传。”方源早就推算过,有极大的可能在红莲真传中,藏着如何破坏宿命蛊的手段和方法。

%66
“而继承红莲真传,就在于光阴长河一战,我能否突破天庭防线。此事的关键,则在于这第三场浩劫。”

%67
“渡过去,我成就八转,能够利用春、夏、慧剑三大八转仙蛊,战力飙升,光阴长河一战才有希望。”

%68
“渡不过去,自然是身死道消。若是侥幸生还,时机也丢了。一步落后,步步落后,天庭除非出现严重失误,否则不会再给我留下翻盘的机会。”

%69
方源心中如冰雪般冷静,对时局更是洞若观火。

%70
怎样去渡第三场浩劫,成就八转,已经成为重中之重。

%71
接下来的日子里,方源除了自身魂修、搜魂夏槎、经营仙窍之外,重点放在探索梦境,练习全新仙道杀招,推算改良旧有杀招上。

%72
改良旧有杀招是必须的,尤其是魂道杀招,需要大幅度改良。否则的话,对战天庭蛊仙,一被他们针对克制,方源将十分被动。

%73
练习全新杀招,方源主要练习的就是夏槎的手段。他现在只有七转修为,但是秋蛊、冬蛊也只是七转层次而已,可以练习相应的仙道杀招。

%74
练习杀招一般有两大难处。

%75
一个是容易受伤,但方源本身至尊仙体,道痕不斥,又有人如故在手,这一点并不难克服。

%76
第二个就是对仙元的消耗大。仙元消耗得快了,就用仙元石转化呗。方源原本手头拮据,但现在多了许多生意,这方面不差钱!

%77
ps:上一章有一个bug,关于房家太上大长老的姓名,不是房狮而是房功。起点正版原文方面已经修改,十分感谢朋友们的指正!另外,我会在最近,对之前章节中的许多bug以及错别字陆续加以修改。朋友们指出来的错误,我都一一记录在小本子上,这一次统一修改一下。最后,再次对批评指正的朋友们说一声谢谢!

\end{this_body}

