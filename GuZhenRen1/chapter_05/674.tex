\newsection{紫薇借蛊}    %第六百七十六节:紫薇借蛊

\begin{this_body}

重生以来,方源的实力一直就在突飞猛进,速度惊世骇人。

尤其是最近这段时间更是如此,修为晋升八转,一身仙蛊丰富至极,举世罕有。

然而,以他的实力还不足以强夺得了龙宫。

争夺八转仙蛊屋龙宫的,有准仙尊实力的龙公,还有东海的多位八转蛊仙,可以说是东海一域中最为巅峰的蛊仙强者。

方源到底是崛起的时间太短,在一域顶峰的强者面前,在龙公面前,抢夺仙蛊屋,尤其是这座仙蛊屋还有反击之力,方源显然是不够看的。

但方源知道,自己绝不能袖手旁观。

若是令天庭得到龙宫,必然是如虎添翼。尤其是那只神秘的梦道仙蛊,绝不简单!恐怕是天地气运酿成,是天道赋予东海的梦道时代的引子。

这个时候,将长生天的力量引进过来,定会让局面更加混乱。

但只有更加的混乱,才能令方源浑水摸鱼。

方源如今晋升八转,但是面对东海的任一八转蛊仙,都不敢轻视大意。八转蛊仙底蕴雄浑,稍不留意就会中了他们的杀招。

至于龙公,方源更心知肚明,自己战力是远远比不上他的。毕竟方源曾经亲眼目睹过,龙公是如何战胜紫山真君的。

“龙公为了保护凤金煌,并未全力出手过。气道的手段,我从未见他用出来过。饶是如此,他单凭变化道的手段,仍旧是场中最强者,时刻遭受他人的合力夹击。”

“白凝冰他们如今还在红莲岛上修行,即便有着未来身加持,也都是七转巅峰战力。”

“可惜我的仙蛊屋雏形损失了,若是有它在手中,经过这段时间的发展,定然已有八转级别的战力,可以帮助我良多。”

“若是从兽灾洞天中盗取了变通仙蛊等等,可令我施展出万物大同变,也能令我暂时拥有更强大的战力。可惜时机相差许多,天相杀招现在还在探查之中。”

方源心中遗憾。

但现实就是这样,不是什么事情都能水到渠成、恰到好处。

龙宫不见疲态,东海中的八转蛊仙也未见全面,龙公显然在等候天庭的援兵,所以关于仙蛊屋的最终争夺,还得有一段时间。

方源对于长生天的态度急转,心中自然也有顾虑。不过就算他们对自己居心不良,将他们引入到这场争夺大战中来,对方源也是利大于弊。

这既是帮助夺取龙宫,又是打击天庭,更是试探长生天,可谓一举多得。

不过几天后,方源没有等来冰塞川,只等来了他的一个关键情报——天庭在光阴长河中,发现了一座石莲岛,如今已是控制住了!

冰塞川建议方源:“你有春秋蝉在手,本身便是红莲真传的正统继承人。若是天庭夺走了你的机缘,对你是损害,对他们也是巨大的提升。红莲魔尊损坏宿命蛊,很可能他的真传中就有帮助我们彻底破坏宿命的捷径!”

冰塞川的这番话没有问题,他并不知道方源已经继承了一份红莲真传。

方源并没有得到什么捷径,但若是另外的红莲真传中有呢?

这也说不准呐。

当然,这也有可能是冰塞川的假消息,是针对方源的陷阱。

现在摆在方源面前有两个选择,一个是继续重视龙宫,另一个是前往光阴长河,和天庭势力争夺红莲真传。

前者一片混乱,危机重重。后者同样如此,风险很大。

方源敏锐地意识到,这一次的选择相当关键。他不禁陷入沉思:这种情况下,他究竟该如何抉择?

南疆。

道德乐土。

两位八转蛊仙并肩登山,气息俱都收敛起来,好似凡人。

左边这位,一身灰色的麻衣,非常朴素,体格强健如熊。头戴斗笠,又宽又大,遮住脸面,只露出宽厚的下巴。

正是名传南疆蛊仙界,当代乐土传人——陆畏因。

而右边的这位,则是一位女子,肤若白雪,青丝垂至腰间,眼若幽潭,眉宇间笼罩一阵哀愁之意。一身紫色丝绸宫裳,华美神秘。

赫然是天庭领袖,执掌星宿棋盘,恐怕是当今天下第一的智道大能——紫薇仙子!

紫薇仙子居然悄无声息地出现在南疆,并和陆畏因接触,究竟是为了什么?

一路上,陆畏因和紫薇仙子交谈,话虽不多,但气氛和谐。

山路一转,从浓郁的绿荫中现出一座漆红的小亭。陆畏因便指着红色小亭道:“紫薇大人,一路登山至此,不妨先去亭中歇一歇再出发罢。”

“好提议。”紫薇仙子微微一笑,欣然前往。

两人入了红亭,亭中自有石凳,紫薇仙子却不去做,而是站在亭边,眺望广阔山河。

这亭子正建在山壁边缘,因此视野极好。

紫薇仙子此时望去,只见道德乐土中一片青山绿水,山脚山腰都有树屋、村庄,炊烟袅袅,鸡鸣狗叫,一片安静祥和的氛围。

而在空中,悬浮着许多座的小山。这些山都是浮土捏造,能天然悬浮于空中,小山上各有特色,环境各异。有的青藤纠结,如发垂下。有的细雨纷飞,映照七彩红光。也有的瀑布宣泄,一股股水流一直垂落下去,直至汇入地面上的宽阔河水之中。

白鹤翻飞,灵猿啼叫,生机勃勃,各种仙材琳琅满目,荒植荒兽寻常可见。

但更叫紫薇仙子心中在意的,是村庄中菇人和人族和谐共处,孩童相互嬉闹,夫妻爱护关照,上慈下孝的场景。

“果然是一片人间乐土!乐土仙尊就是想要这样的世界,可惜人族始终是正统,异人怎可和人族平起平坐?不过即便如此,我仍旧敬佩乐土仙尊。因为他为了自己的这份梦想,彻底地奉献了自己,从未有一份功利之心。”紫薇仙子缓缓地道。

天庭是以人族为主,凌驾于其他异人种族之上的理念,乐土仙尊却希望人人平等,在他看来,不仅是异人和人族,世间的一切生命都应当是平等的。

正是因为这份理念之差,才使得乐土仙尊最终没有入住天庭,而是在五域当中广建乐土,营造自己梦想中的理想乡。

陆畏因点头:“你我两方虽然理念不同,但对于为祸天下的魔道的态度是一致的。紫薇大人此次前来借蛊,若是其他仙蛊也还罢了,但只借仁蛊,造福世间,我方又怎会阻拦呢?只是外人借蛊,按照仙尊当年留下的规矩,须得徒步登山,登山峰巅,才有借蛊的资格。登山辛苦,修为越高越是如此,还望紫薇大人包容体谅。”

“这山的确不好登,不过这也是乐土仙尊的良苦用心。登上山顶,对我也有不少裨益呢。”紫薇仙子轻声一笑,双眼中波光一闪,“好了,我们继续出发罢。”

光阴长河。

天庭四座仙蛊屋各据一方,围绕着仙阵。

蛊仙之间氛围热烈。

“凤九歌大人,我们已是将这座红莲岛困住,只待进去探索了。”上旬子汇报道。

凤九歌点点头,他原本是想利用光阴长河,来巩固、壮大自己心中关于命运之歌的灵感。没想到进来不久后,就有石莲岛主动显现,被天庭及时围拢困住。

“红莲石岛乃是红莲魔尊亲手布置,还是我亲自前往,见识一下这位惊世魔尊的手段。”凤九歌正要动身,进入仙阵当中。

轰!

一声巨响,他身处的仙蛊屋剧烈颤抖起来。

“有敌人!”

“是谁如此胆大包天,竟然敢动我们天庭?!”

众仙顿时惊怒交加。

来者现身,不是方源还是哪位?

天庭一方愣了愣,旋即就有人叫嚣道:“好个魔头,上一次被你逃走,这一次你居然还敢回来!那就来受死吧。”

方源淡淡微笑,他思考了一番后,还是决定先前来光阴长河,阻止天庭染指红莲真传。

毕竟龙宫那边,场面十分混乱,各方都未尽全力,变数太多。

而红莲石岛这边,若没有他来阻止,肯定会被天庭得手。方源拥有春秋蝉,对于继承红莲石岛大有优势。

“方源,我们又见面了。”凤九歌的声音从仙蛊屋中传来,饱含感慨之意。

方源神色微微一凝,看来这一场战斗,他要和凤九歌再次交手了!

\end{this_body}

