\newsection{攻守}    %第六百一十七节:攻守

\begin{this_body}

“呼……”方源吐出一口浊气,缓缓睁开双眼。

这是一处地下的小洞,洞壁上燃烧着点点幽火,形成天然的阵法。虽然防御不强,但也有着良好的示警作用。

方源在光阴长河中了天庭埋伏,拼尽全力奋斗搏杀,总算是杀出一条血路来。

甩开了清夜等人之后,方源顺着一条光阴支流,来到外界南疆。

南疆乃是影宗大本营,紫山真君曾经在各处布置了这种幽火小洞。方源动用定仙游杀招,随意挑选了一座,潜藏起来,疗伤休整。

他虽然有着逆流护身印,但并非全程遮护,还靠着冬裘杀招护体。不提和厉煌激战,就是逃亡途中,方源也硬生生承受了三座仙蛊屋的狂轰滥炸,受伤不轻。

方源视察一番自己的至尊仙体,缓缓站起身来。

经过这几日在小洞中的休养,他已经痊愈。主要因素还是至尊仙体上,道痕相互之间并不互斥,这让方源疗伤起来效果惊人。

仙窍中的八转白荔仙元,也有所增添,不过仍旧在警戒线下。

方源背负双手,在小洞中踱步。他思考眼前的局势,眉头越皱越深。

这一次,他可算是在光阴长河中栽了一个大跟头。用“险死还生”这个词来形容,毫不为过。

方源心中并没有什么后怕、庆幸的情绪,相反这种情况,也在他的估量之内。

天庭的实力实在是太雄浑了!

方源崛起的时间,和天庭的历史比较起来,连一个零头都算不上。

这种雄厚的底蕴,只要稍稍发力,就能让方源吃不了兜着走。数天前的光阴长河突围战,便是最好的证明。

之前方源应对天庭能有优势,是因为他搞突袭,令天庭猝不及防。又保持隐秘,叫天庭无从下手。方源一直在穷尽心力、争分夺秒,实力上不断高歌猛进,令天庭每每应对都有失策。

但随着天庭将方源的底细调查得越来越清楚,越来越重视,真正到了硬碰硬的时刻,方源的实力就显得单薄了。

天庭之强,不仅在于自身的实力底蕴,还有最关键的一点,那就是统一的意志。

南疆正道的实力,其实也非常雄厚庞大。但各大家族林立,结构松散,面对方源难有手段。

天庭却是一直将方源四处撵着打。

若非方源狡诈多变,老谋深算,换做其他魔道蛊仙,早就被天庭或擒或杀了。

随着时间推移,天庭在光阴长河中驻防,必定会越来越强,方源获取红莲真传的希望也越来越渺茫。

面对这样的庞然大物,方源还能怎么办?

“白荔仙元几乎消耗殆尽。”

“逆流河也只剩下一些,被厉煌消耗很多,很难再继续运用。”

不像荡魂山、落魄谷上有着九转杀招井井有条,可以利用江山如故蛊进行快速复原。逆流河就难办了。

方源并不缺恢复它的手段,但无论哪一种,都是一场艰巨的大工程!

可以预见,在未来的很长时间里,方源难以依赖逆流护身印这个手段。

缺少逆流护身印,但方源修为也晋升八转,又增添了夏槎的手段,前后比较起来,实力还上升了不少。

“只是这一次,我暴露的手段也太多了些。”

太古年兽、春剪、夏扇等杀招的暴露,是在方源的计划当中。但五指拳心剑、光阴飞刃暴露的,就有点让方源感到可惜。

但这没有办法。突围战中,方源不得不用尽手段,藏着掖着恐怕此刻已经身死道消了。

“不过此次却也并非没有收获。”

天庭的底细,也被方源试探了出来。

比如说,天庭有神秘手段,居然能够引发春秋蝉的震动!

这一点超出方源的想象。

不过现在想来,方源非常理解,他也想到那座埋伏他的宙道大阵,恐怕针对的不是他,而是红莲真传。

而春秋蝉就是开启红莲真传的钥匙。

天庭针对红莲真传,怎可能不从春秋蝉着手研究呢?虽然他们没法炼出春秋蝉来,但模拟伪装的程度还是能做到的。

“厉煌、清夜……”又有新面孔的八转蛊仙出现了。

天庭统治中洲,可以从十大古派中,随意吸收八转蛊仙成为全新成员。时不时的还从仙墓深处,苏醒一两位资深强者。

从太古时期就创建,一直屹立至今,谁也说不准天庭方面的底蕴究竟有多深。

正是有着这样的底蕴,方源五百年前世,天庭统御中洲,以一敌四,很长一段时间打得其他四域都抬不起头来,接连灭掉超级家族。

或许也是有这个原因,曾经魔尊幽魂逆天成功,也不敢明目张胆地杀上天庭,而是继续潜伏起来,暗中渗透。

“还有那三座仙蛊屋……不,之前我摧毁的那座,仙蛊并未损毁多少,就算损毁了,天庭也能抢炼。恐怕我将来再次面对的,至少是四座,甚至还可能更多!”

八转宙道蛊仙,就算是天庭,也不容易拿得出来。

但宙道仙蛊屋却是可以拆卸其他仙蛊屋,或者从各个蛊仙手中,从库存当中收集过来,进行组装。

组装起来后,天庭就能用大量的其他流派的蛊仙,坐镇在仙蛊屋内,达到封锁整个光阴长河的目的。

什么围魏救赵、调虎离山的诡计,是不起作用的。

天庭乃是五域第一洞天,毫无后顾之忧。方源攻上天庭,不要说连门户都找不到,就算是进去了,惹得天庭震怒,仙墓中随便惊醒两三位八转大能,方源一条小命可能就要交代在这里面了。

若是方源袭扰十大古派,也动摇不了天庭的统治,更有巨大的风险。毕竟天庭中,可是有着紫薇仙子这位智道大能。

方源目光如冰般冷冽,他清楚得很:要想寻得红莲真传,就得和天庭硬拼!

天庭在光阴长河中防守得越严密,就越证明红莲真传的重要性,方源摧毁宿命蛊的希望就在于此。

“还要快!”

天庭掌握着种种手段,很有可能先方源一步谋取到红莲真传。所以,方源必须在天庭没有得逞之前,突破它的重重防御,搜寻到石莲岛,获取红莲魔尊留下的最关键的东西。

至于是什么东西,方源一点都不清楚,恐怕天庭方面也不太了解。

驻防光阴长河,这是天庭阳谋,因为占据着战略优势,天庭不愁方源不撞过来。

宿命蛊修复在即,方源必须冲破天庭防守,得到红莲真传,才有着破坏宿命的可能。就算是撞得头破血流,就算明知有埋伏,九死一生,方源也不得不去做!

“没有什么巧妙的方法,也没有任何的捷径。只有让我的实力不断飙升,不断暴涨,直至超过天庭驻守力量,冲破一切阻碍,夺取到红莲真传!”

这个目的达成了,方源才有继续陪天庭周旋下去的资格。

------------

\end{this_body}

