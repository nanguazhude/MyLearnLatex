\newsection{插手三域}    %第八百三十六节:插手三域

\begin{this_body}

龙人分身先钻进龙宫,随后方才驾驭这座仙蛊屋,进入了福地。

福地乃是古族大本营,旁人的地盘,古族族长又是八转修为,非同小可,还是谨慎稳妥一些最好。

一进入福地,便是一片苍茫的大地。这片福地之宽阔,甚至要超过一般的洞天。

在福地中,种类繁多的仙材随处可见,很多竟都是当今早已灭绝之物,令方源的分身也开了一些眼界。

“虽然这片福地底蕴并不如琅琊福地,但也差距不大了。”龙人分身吴帅心道。

方源碰到过的福地中,底蕴第一的当属琅琊福地,第二则是王庭福地,第三恐怕就是眼前这座了。

“古族福地,经营了至少百万年!古族历代的族长,至少有七转修为,很多任都有八转。有着这样的强者护卫,福地的灾劫并不是什么威胁。”

“况且……这古族并非人族,一直受到天意的照顾。就算有灾劫,恐怕威能也不会很强。”

吴帅心中暗暗思考。

天道损有余而补不足,讲究平衡。当今天下人族乃是霸主,将其他异族逼压到角落里苟延残喘。

所以,任何一支异人种族都受到天意的照料。

“吴帅前辈,这边请。”古族族长相当客气,当前领路。

龙人分身吴帅紧随其后,飞行了一阵后,两人落到丛林深处。

在这片原始丛林当中,有着一个上万人的巨型部落,部落中的子民都是兽人。

没错,所谓的古族,便是兽族。

根据人族史籍记载,兽族已是灭绝,但事实上天机留一线,仍旧有着一支兽人血脉残存下来,苟且偷生,默默生息。

吴帅受到了整个部落的热烈欢迎。

很快,就有酒宴备好,古族族长盛情邀请吴帅入席。

吴帅便走出龙宫,身后有着两大龙将跟随,充当左右护法侍卫。

古族族长瞳孔微微一缩,旋即哈哈大笑,笑容更加热情了几分。

吴帅只暴露了一半的龙将,但每一个龙将都是八转修为,和古族族长相当。而这片兽人福地中,除了古族族长之外,却是再无其他八转的存在了。

因为梦境探索过的缘故,方源分身对这支兽人族群了解许多。

入席交谈片刻,吴帅更对古族现状有了一个清晰判断。而古族族长也从这番对话中,彻底打消了心中仅剩下的一丝怀疑。

这只兽人种族大多都是鱼头人身,脸颊有腮,双耳乃是鱼翅,背生鱼鳍。除此之外,有的人长着蛙腿,有的则有海鸟翅膀,有的背着龟壳,有的头上是一圈绿色的海藻。

这些兽人本来就是生活在海中,当年兽人一族盛极而衰,被大清洗,陆地上的族群全灭,只剩下海中的这一支侥幸生还下来。

兽人一族盛极而衰,人族称霸的大势之下,这支海底兽人族群就只好默默繁衍生息,不敢有一丝一毫大举反攻的念头。

但对人族的仇恨,他们并没有忘记。几乎历代的古族族长都有这样的心思,企图挑起人族内乱,给自家族群创造崛起的机会。

历史上,真正的吴帅碰到的古族族长,正是基于这样的阵营,方才帮助吴帅升仙,又谋夺宿命蛊,也是一个枭雄!

吴帅为了应付龙人寂灭杀招,便和古族族长商议,定下了一道两族盟约。

之前在福地的门户前,方源分身和这一代的古族族长交流谈及的,正是这个盟约。

这是真正的吴帅,留下的一番布置,准备将来复兴龙人一族。如今方源继承了龙宫,得知这条线,便想顺着走下去。

古族在东海经营了这么多年,古族族长又是八转级数,古族的底蕴又是如此丰厚,方源当然想要从中借力。

纠集更多的力量来对付天庭,对于方源而言,自然是喜闻乐见。

就在方源分身吴帅和古族族长相谈渐欢的同时,西漠的扬子河、张阴已经得手。

轰隆!

一声巨响,扬子河完成了最后一击。

放眼望去,曾经锦绣繁华的一片绿洲,已经彻底化为乌有,沦为废墟。

这是房家的一处大型资源点,如今已经毁了,再无任何修复的可能。

“我们该走了!”

“这一处已经是摧毁的第三处,按照主人的吩咐,做到这一步就可复命去了。”

“还是先检查一番,看看是否露出马脚来。”

两大龙将细心检查,确保无误后,方才离去。

房家震动!

一日里,连续失去了三个大型资源点,让房家也损失不小。

董家也震动!

皆因废墟当中,留下几行大字——董陆沉到此一游。

董陆沉乃是董家的太上大长老,之前房家报复他,出动了偷道仙蛊屋,直接将董家的一个绿洲资源点摧毁了。

董陆沉听到这个消息后,脸都青了,破口大骂:“这群不要脸的家族,枉为正道!居然还敢来栽赃我,并且手段还如此拙劣!今后若被我发现真凶,我绝不饶他(她)!”

董陆沉怀疑不到方源身上去,首先怀疑的对象就是抛除房家、董家之外的那些超级势力。

这是很明显的挑拨离间之计,若是房家中计,开始和董家死磕,首先得到便宜的就是这些西漠超级家族。

董陆沉坐不住,当即给房家太上大长老房功去信。

房家拥有智道蛊仙房睇长,自然不会轻易中计,琢磨了片刻,房睇长便建议房功,回信邀请董陆沉,结合两家之力共查真凶。

这一手端的了得。

董陆沉得到信后,顿感棘手。

他明白房家是想趁机拉拢董家,想要将董家拉到自己的阵营里来。

但董陆沉不傻。

董家和房家接壤,将来房家崛起,首先侵吞的便是董家的地盘和利益。

然而,自己若是不去探查真凶,是否会显得自己做贼心虚呢?

董陆沉左思右想一番,只好捏着鼻子前往案发之地,和房家蛊仙共查真凶。

真正的结果当然不会查到,方源既然派遣了两大龙将来办理此事,自然有了万全手段来误导真相。

于是,房家和董家得到的答案,直指房家附近的另外几家势力。

但房家和董家却是偃旗息鼓,没有去找麻烦,只是暗记在心。

双方都十分克制。

董陆沉是将这几个家族暗记在心,并不想曝光,他还要依靠这些人来帮助他对付房家呢。

而房家为了维稳,也不想现在就公然对抗这些超级势力。毕竟势单力孤,豆神宫距离化为己有,还有一长段的距离。

“太上二长老,如今局面越加危险,单靠你一人来破解豆神宫,是否太过劳累了?那算不尽也是智道蛊仙,或许可以辅助你,帮助你分担一点任务。”房功建议房睇长道。

在他看来,反正算不尽参与过豆神宫的争夺战,进出过豆神宫,如今又是房家的太上客卿家老,是自己人,可以信赖。

房睇长沉吟片刻,始终犹豫不决,最终他叹了一口气:“让我再想想吧。”

南疆。

夏家大本营中,夏槎也在叹气。

她已经被方源释放出来,但肉身、魂魄虽然不缺,仙窍却是被方源取走了,融并在至尊仙窍里。

夏槎再也不是八转蛊仙,而是沦为了凡人。

现在她的身上还挂着夏家太上大长老的名头,但在夏槎感觉里,这个名号却更像是一个笑话。

“我已经是一个废人了!”

“就算曾经是八转,又能如何呢?”

“方源!你果然是好手段,这个滔天之仇,我若能报,必定要把你抽筋扒皮,挫骨扬灰!”

想到这里,夏槎再叹一口气。

她自己也知道,这个报仇的希望是多么的渺茫。

“大喜,大喜啊,夏槎大人!”就在这个时候,夏家太上三长老兴冲冲地跑过来。

“何喜之有?”夏槎皱着眉头,望着他。

夏家太上三长老便递给夏槎一只信道凡蛊,道:“这是那魔头方源刚刚利用宝黄天,送达过来的信。”

夏槎冷笑:“这魔头敲诈阴险,贪婪至极,将我等俘虏不说,还将肉身、魂魄等等拆分下来,一次次敲诈我族,实在是罪该万死!”

“眼下方源手中已经没有多少敲诈的筹码了。这封信我也不必看了,无非是这魔头又想出了什么新法子,来继续勒索我南疆正道。”

夏槎摇头不已。

夏家太上三长老点头:“夏槎大人所料极是,方源企图昭然若揭!他最近一直在大规模地收购运道仙材,似乎在大炼运道仙蛊。但他这一次开出的筹码,实在是……”

“怎么说?他开出了仙蛊的筹码?还想以蛊换蛊?”夏槎扬起眉头,又摆手,“这件事情不需要我同意,直接交给太上二长老处理就是了。只要是有利于我族的仙蛊,不妨换来就是。”

“这一次,方源的筹码只是一只凡蛊。”

“哼,一只凡蛊就想要换仙蛊,换仙材?想得美!”

夏家太上三长老微笑着:“夏槎大人,您还是看看详情再说吧。”

夏槎便接过信道蛊虫,起先浏览,她冷笑连连:“这方源是得了失心疯?要换如此之多的仙材,就像凭一只凡蛊。我倒要看看这是什么蛊!”

“嗯?!”夏槎看到信中关键之处,脸色猛地剧变。

她腾地一下站起身来,以不容置疑的口吻道:“这只凡蛊我要了,必须拿下,不管耗费何等代价!方源想要运道仙材?都给他,都给他!”

\end{this_body}

