\newsection{影无邪换魂}    %第三百七十七节:影无邪换魂

\begin{this_body}

%1
一头黑色的波浪长发,披到肩头。

%2
鼻梁高挺,嘴唇线条宛若刀刻,此刻他双唇轻抿,倒有一股冷酷的风姿。

%3
正是影无邪的魂魄。

%4
此时,他已经脱离了纯梦求真体,将魂魄抽离出来,悬浮在半空当中。

%5
“就是这具肉身?”影无邪皱起眉头,看着眼前的一座冰棺。

%6
这座冰棺,无时不刻不散发着冰冷的寒气,半透明,宛若水晶,里面封印着一个蛊仙。

%7
她玲珑娇小,仿佛幼女,一身雕花的绿衣,双耳吊坠着两颗玉珠子,珠子内部仿佛含水,摇曳间,荡漾出绿水澜光。

%8
此刻双目紧闭,被白凝冰死死封印,就连脑海中的念头都被冰封,生死任凭他人定夺。

%9
这位女仙自然不是别人,正是之前眼力不济,见到方源没有及时逃走,被白凝冰捉住俘虏的西漠女蛊仙翠波仙子。

%10
见影无邪有些迟疑,方源催促道:“现在这种情况,哪里能找寻得到符合你要求的肉身?先将就着用吧。”

%11
影无邪点点头:“唉,只能如此了。快开始吧!”

%12
方源便右手一展,从掌心中飞腾起一只七转仙蛊来。

%13
这只仙蛊形象奇特,像一只鹅蛋,蛋壳半透明,从外面望去,可见里面氤氲的碧绿光气。离得近了,还感到一阵刺骨的寒意,从这只仙蛊中散发出来。

%14
这股寒意和冰棺的冰寒,并不一样。

%15
冰棺的寒气,是侵肉刺骨。而这只仙蛊散发出来的寒意,直接作用在人的魂魄上。

%16
影无邪的魂魄,便因为这只仙蛊的出现,而不自禁地打了一个寒颤。

%17
另一边,白凝冰静静地站着,双眼视线不免被这只七转仙蛊吸引。

%18
“这就是换魂仙蛊吗?”白凝冰心底呢喃一声。

%19
没错,正是仙蛊换魂。

%20
这只仙蛊的来历过往,也颇为曲折。

%21
起先,是被第一代幽魂分魂之一的绿,炼制而出。

%22
然后交给了灵缘斋的仙子墨瑶。

%23
墨瑶为了拯救情郎薄青,运用换魂之术,将自己的魂魄和薄青调换。

%24
薄青残魂被影宗所救,藏在中洲地渊深处,一直试图治愈,可惜伤重难返,最终被天庭俘虏。

%25
而墨瑶的残魂则留在薄青的仙僵肉身当中,连带着换魂仙蛊,以及薄青的一系列剑道蛊虫。

%26
在此之后,方源进入落天河底,运用墨瑶残意,勾引出数只仙蛊。其中就包含换魂仙蛊。

%27
但方源虽然利用智慧光晕,炼化了这些仙蛊,但对换魂仙蛊始终一无所知。

%28
直至星宿天意在梦中指点方源,传授到一招梦中换魂的强大杀招。方源这才知晓,这只仙蛊的作用。

%29
方源凭借梦中换魂,在最关键的时刻,占据了影无邪的原本身躯,然后暗算幽魂本体成功,抢夺走了至尊仙胎蛊。

%30
可以说,义天山大战,影宗大败亏输,方源临终翻盘,换魂仙蛊起到了举足轻重的作用!

%31
现在,这只仙蛊被再度运用,不过这一次,方源却不是为了对付影宗,反而是帮助影宗。甚至他自己,都成为了影宗之主!

%32
命运的玄奇,实在是难以预料的。

%33
在此之前,不管是方源,亦或者影宗上下,又怎可能会料到有这样的情景?

%34
在换魂仙蛊的作用下,影无邪的魂魄成功地和翠波仙子的魂魄,进行了调换。

%35
“你们干什么?你们把我怎么了?”翠波仙子的魂魄脱离了冰棺之后,悠悠醒来,惊恐万分。

%36
但是她缺了肉身,魂魄发出的叫声,听在人耳中,只能是一些无意义的刺耳尖啸。

%37
当然,方源是听清楚了。

%38
他有着魂语蛊虫,这只是凡蛊,数量有很多,是紫山真君的遗产之一。

%39
方源冷笑一声,向翠波仙子的魂魄轻轻地招了一下手。

%40
翠波仙子便感觉到一股难以抵挡的巨力,将她整个魂魄都摄到方源的手中去。

%41
“不!”她发出凄厉的惨叫,声音之刺耳,即便是白凝冰都轻轻地皱了一下眉头。

%42
下一刻,她就被方源用手段制住,塞入到至尊仙窍里。

%43
虽然翠波仙子已经被白凝冰搜过魂,早已经没有了秘密,但是蛊仙魂魄本身,也是能够利用的。

%44
荡魂山就是一个利用的方法。

%45
蛊仙魂魄,尤其是人族蛊仙魂魄,修为越高,魂魄底蕴越厚,落入荡魂山被荡碎后,形成的胆识蛊也就越多。

%46
不过现在,方源又有了其他的手段。

%47
自然是得益于紫山真君的遗产,里面记录的魂道手段很多,其中就有一个,能够将蛊仙魂魄炼制成人形魂兽。

%48
七转修为的翠波仙子,转变成魂兽的话,就是上古魂兽。

%49
方源若是将她炼成魂兽,不用任何的奴道手段,就能让翠波仙子听命于他,并且忠心耿耿。

%50
成为影宗之主后,方源真正的巨大收获,在于手段和杀招。仙材、仙蛊虽然也挺多,但总体价值上,却是远远不如的。

%51
只是这些见闻和手段,需要很长的时间来转化成真正的实力。

%52
而方源现在,面临追杀,偏偏最缺少的就是时间。

%53
翠波仙子的魂魄被镇压,方源又将换魂仙蛊收起来,这个时候,白凝冰已经打开冰棺,放出“翠波仙子”。

%54
更准确地讲,是顶着翠波仙子肉身的影无邪。

%55
“好冷。”影无邪哆哆嗦嗦,浑身还沾着厚重的冰霜。

%56
白凝冰便信手一挥,影无邪身上的冰霜顿时消散全无。

%57
影无邪道一声谢,随后又接住方源抛来的众多蛊虫。

%58
大量的凡蛊,当然还有仙蛊,以及仙元石。

%59
“你还想让我催动引魂入梦?”影无邪看着这些蛊虫,有些诧异。

%60
“你都拿着。引魂入梦的杀招我都已知晓,此招繁复至极,我短时间内无法练熟。如今我们身上还中着天庭的侦查杀招,正急需你的战力!”方源冷静地道。

%61
影无邪重重地点点头,双眼放光,拳头握紧:“宗主说的是,我一定不辜负宗主的信任!”

%62
白凝冰目光中流露出一丝讶异之色。

%63
影无邪虽然重振斗志,但是性情好像有了一些变化。

%64
方源嗯了一声,却是心知肚明。

%65
智道三元,分别是念、意、情。影无邪本身是颓丧的,只是因为斗志昂扬才重整旗鼓,所以性情方面有所变化,也是自然的。

%66
“宗主,咱们接下来干什么?你快说吧!”

%67
“哪怕是攻上天庭,我都会紧随其后!”

%68
“****娘的,这些正道的狗崽子来一个杀一个!!”

%69
影无邪一边说,还一边用力地挥舞手臂,她很激动,以至于颇为巨大的胸口,掀起了一阵波涛乳浪。

%70
方源沉吟道:“天庭的追杀,正是我说忧虑的事情。他们是最熟知我们情报的人,派遣过来的蛊仙,一定是会强到我们无法对抗的程度。”

%71
方源说到这里,眉头不由地轻轻皱起。

%72
他想到了凤九歌。

%73
凤九歌说他自己是因为追取一只和盗天魔尊相关的仙蛊,才意外地撞见了方源等人。

%74
但这种撞见,未免太过巧合了一点吧。

%75
凤九歌就是天庭的后手吗?

%76
似乎是,但又不像。

%77
方源设身处地去想:“若我是天庭一方,对付影宗余孽,一定会竭尽全力,实施致命一击。派遣凤九歌过来,似乎是一场拙劣的安排。”

%78
这点方源有点想不明白。

%79
天庭犯错误的可能,是相当小的。

%80
天庭对付影宗,有两次重大的出击。第一次是义天山大战,第二次则是梦境战役。

%81
这两次都是非常关键,因为天庭的出手,导致影宗大败亏输,魔尊幽魂十万年大计分崩瓦解。

%82
“我一成为天庭之主,天庭方面就出了纰漏吗?不会的,一定是有我还未想明白的事情。”方源在心中呢喃。

%83
与此同时,西漠界壁中闯出一个人来。

%84
他身姿挺拔,剑眉星目,神情温和,风度翩翩,让人心折。

%85
正是凤九歌!

%86
“有着这样一只八转仙蛊,穿越界壁,容易了无数倍。”凤九歌回望一眼,心中不由地浮现出之前的一幕。

%87
两位天庭的八转蛊仙,忽然出现,阻止了凤九歌和武庸的战斗。

%88
天庭归还了南疆正道的蛊虫,武庸身负重伤,只能见好就收,直接撤走。

%89
“凤九歌,你既已立誓,从现在起,你便去追辑方源等影宗余孽吧。”

%90
“这只八转仙蛊暂且借你,今后就算是再碰见八转蛊仙,也不会如此为难。”

%91
“不过,对于方源,你还要注意一些。”天庭蛊仙嘱咐凤九歌道。

%92
凤九歌疑惑,便问:“此言何意?”

%93
天庭蛊仙的目光深幽起来:“方源本身是完整的天外之魔,必须铲除杀死。但是要杀死他,还远远不够。”

%94
“红莲魔尊选中的棋子,就是方源。方源拥有春秋蝉,又称为影宗之主,一定会前往光阴长河,寻找石莲岛,继承红莲真传。”

%95
“红莲魔尊的真传,潜藏在光阴长河的深处,隐秘非凡。”

%96
“追踪方源,不断迫使他去追寻红莲真传,然后将方源和红莲真传一同摧毁!”

%97
凤九歌目光一阵闪烁,至此他才明白天庭的深沉用意。

%98
“方源、红莲真传……”

%99
回到现实,凤九歌口中低语一声,旋即一飞冲天,直向方源等人藏身之处径直赶去。

\end{this_body}

