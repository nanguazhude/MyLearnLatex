\newsection{春晓翠鹂}    %第九十一节:春晓翠鹂

\begin{this_body}

%1
鸟鸣清脆悦耳,鸣叫不休。

%2
至尊仙窍中的小北原,原本洁霜一片的荒芜地面,已经泛出无限绿意。

%3
地灾的力量,可谓浩大。十几个呼吸之后,方源附近十多万里,已经化为绿茵茵的草原。

%4
草木丰茂,繁花雕缀其中,简直是天地换了新颜。

%5
方源站在荡魂山巅,满脸肃穆之色,全神戒备。

%6
这一次,是他的第三次地灾。他把荡魂山再次带在身上。

%7
但眼下情景很诡异。

%8
地灾力量所变化的春晓翠鹂,反而帮助方源建设仙窍。

%9
大半个月前,宝黄天重启,方源没有选择前往黑家冒险,而是打道回府,大力经营至尊仙窍。

%10
他经营许多地方,但小北原始终被方源充当做渡劫之地,一直都没有得到建设。

%11
眼下,这场地灾却替方源代劳,实在出乎方源的意料。

%12
这对方源而言,无疑是一件好事。

%13
“不过天意难测,不会这么好心,必定还有埋伏或者后续,我且暂观其变。”方源站在荡魂山巅,谨慎旁观。

%14
小北原还是很大的。

%15
地灾形成的草原,并未铺满整个小北原,只是覆盖了以荡魂山为中心的一片霜白大地。

%16
放眼望去,绿色越加浓郁。很快,一颗颗树苗就拔升而起,并且以肉眼可见的速度生长起来。

%17
按照这个趋势,这里要形成一片方圆十万里的大森林了。

%18
方源心中又惊又喜。

%19
地灾的力量,多以毁灭的形象,面对世人。此番用来建设仙窍,效果极其惊人。若是方源移种这片森林和草地,必定要花费他大量的精力和时间。

%20
春晓翠鹂的生命极其短暂,一只只灭亡,化为一股碧绿精芒,消散在天空中,但与此同时。全新的春晓翠鹂又凝聚形成,前仆后继。

%21
一时间,漫空都是这些荒鸟飞翔的身影,数量之多。宛若麻雀群,蔚为壮观。

%22
“等等!”方源忽然面色一变。

%23
“去。”他想到什么,口中低喝一声,伸手一指。

%24
顿时,一道剑芒飞出。激射如电。

%25
七转仙蛊飞剑!

%26
春晓翠鹂速度虽快,但也跟不上这道剑芒。不过它转向灵活,并不是直来直去,倒教方源费了一些功夫,这才击毁了一只。

%27
一股真意,随之而来,灌注到方源的心头。

%28
刹那间,方源仿佛变成了一只春晓翠鹂,从出生到飞翔,再到死亡。短短的时间里。历经生死,心境也随之变化。

%29
方源将这股真意完全吸纳,双眼再现清明之色。

%30
他如今是变化道宗师境界,吸纳这些狂蛮真意,再不像之前那般勉强。

%31
若说之前,他仿佛是一个空盆,去接纳一桶桶的水。现在他汲取狂蛮真意,便宛若一口井,来容纳一桶桶的水,自然轻松自如了许多倍!

%32
方源眉头皱起。这一下,算是明白了天意的算计。

%33
方源的地灾,很特别,因为仙灾锻窍和北部冰原的特殊原因。导致他的地灾分为两部分。

%34
一部分是天意控制,另一部分是狂蛮真意影响。

%35
之前的两次地灾,天意控制的部分,都是首先形成,狂蛮真意影响的部分随后出现。但第三次地灾,却是狂蛮真意影响的部分地灾先出现了!

%36
这让方源始料未及。

%37
毕竟这是他第一次遇到这种情况!

%38
方源吃了个小小的闷亏。

%39
狂蛮真意和建设福地相比较起来。当然是前者更重要,后者次要。

%40
因为狂蛮真意可以直接提升变化道境界,而且只有每次渡劫时才有,风险和成本都很高。

%41
至于建设福地,平常什么时候都可以做。

%42
“这天意是知道我谨慎心性,真是好算计,忌惮我变化道境界飞速提升,让我浪费了不少狂蛮真意。”方源哈哈一笑,却也不气馁。

%43
他身形如电,一飞冲天。

%44
飞到空中,他雄躯一震,变化出无数力道虚影。

%45
力道杀招万我!

%46
万我冲杀过去,汇聚成磅礴的人潮,一路上卷席春晓翠鹂。

%47
春晓翠鹂毕竟是荒鸟,在万我人潮中穿梭自如,仍旧啼叫不休。

%48
方源见万我成效不大,连忙又转化另外手段。

%49
仙道杀招剑浪三叠!

%50
哗哗哗!

%51
银白色的剑浪,汹涌澎湃,散发无限锐意,朝前推去。

%52
春晓翠鹂的速度,比剑浪还稍胜一筹。方源发出三叠剑浪,虽有扑灭几只,但成效还是不大。

%53
他眉头微蹙,心中了然:“前两次地灾,让天意已经知晓了我许多底细。不管是万我,还是剑浪三叠的弱点,都被知晓。这次地灾,形成的春晓翠鹂,恰好针对这两个杀招。天意刻意针对,可见用心深远。”

%54
上一次地灾,也有风刃袭杀。

%55
不过方源有荡魂山,当做营地,很大程度上起到了防护作用。

%56
但现在情景,方源虽然也带来了荡魂山,但是春晓翠鹂是狂蛮真意所化,方源偏偏不能坐视不管,消极防守,他必须主动攻打,否则就丢失了最大的收益。

%57
“幸好我拿回了飞剑仙蛊!”

%58
方源催动这只七转仙蛊,剑光纵横,全凭他一念之间。

%59
一只只春晓翠鹂死在方源手上,但更多的春晓翠鹂自我消亡。

%60
不是方源终结的春晓翠鹂,就没有狂蛮真意灌注。天意这次故意设计,方源遭到算计,损失不小。

%61
这不单单只是狂蛮真意的损失,还有青提仙元。

%62
飞剑仙蛊虽厉,但却是七转,每次催动,要耗费大量青提仙元!偏偏每一击只得击杀一头春晓翠鹂。方源宛若大炮打蚊子,很不得力。

%63
“看来下一次渡劫,我恐怕得有新的手段。老旧的招数,越来越不顶用。”方源不由地想到智慧蛊。

%64
若是智慧光晕能用,这个难题就很容易得到解决。凭借他此时的宗师境界,智道手段,推算出仙道杀招来。简直如同喝水般简单。

%65
可惜方源虽有仙僵肉身,但不敢轻易涉险。

%66
地灾持续,一炷香后,仍旧有绵绵不绝的春晓翠鹂产生。

%67
小北原上。何止是绿草茵茵,早已经形成一大片森林,林中都是参天大树,高达数丈,树干如柱。

%68
在此期间。方源又试用了其他手段,譬如毒气喷吐。上一场地灾,这个杀招效果拔群,但此次结果对春晓翠鹂却是完全无效。

%69
嗷嗷嗷!

%70
广袤的森林中,一座座大树仰头发出呼号之声。

%71
随后,它们忽然拔地而起,变化成一个个的巨大树人。

%72
方源立即注意到地面上的变化,不由冷笑:“看来这才是天意控制的部分。呵呵,幸好我谨慎,提前铲除了一半森林。现在看来。可谓防备得当。”

%73
无数的树人,抽出深埋在地下的树根,纷纷向着荡魂山杀来。

%74
它们速度缓慢,但各个体格粗壮。方源心念一转,选择回防。他落到荡魂山巅,看着四面八方的树人群,朝着自己拥来。一片片的绿,厚重浓郁,好像是整个森林朝着自己堆压排挤,让他都有些窒息之感。

%75
方源伸手一指。试探攻击。

%76
飞剑仙蛊冲杀过去,化作一道白虹,刹那间贯穿五六十里路,沿途的树人均被洞穿。

%77
这些树人停滞不动。但很快,伤口复原,又恢复过来,继续朝着荡魂山缓慢爬行过来。

%78
方源又用剑浪三叠。

%79
璀璨夺目的剑浪,带着轰鸣声,吞没了大片树人。一丝残渣碎片都没有。效果上佳。但很快,后面的树人又渐渐补上了这片空白。

%80
方源皱起眉头。

%81
“这些树人不算荒兽,但也相差不多。皮糙肉厚不说,还能汲取生机,迅速复原。”

%82
至于生机何来,自然就是春晓翠鹂了。

%83
自从第一次地灾之后,天意虽然不能控制全部地灾的力量,却也能和狂蛮真意所影响的部分,形成搭配。

%84
上一次是冷月和风刃,这一次是树人和春晓翠鹂,都是搭配密切,效果脱俗!

%85
方源忽然张开大口,吐出一口剧烈的毒气。

%86
毒气在树人当中四散开来,当场就有数十头树人行动缓慢,绿油油的叶子被染成紫色。几个呼吸之后,它们行动越来越慢,最终轰的一声倒在地上,整个树干和树叶都开始腐朽,化为一摊枯枝烂叶。

%87
杀招毒气喷吐的效果也是不错。

%88
可惜这杀招,方源不能持续使用,有很强的后遗症,必须排除自身遗留下来的毒气,否则可能连自己都毒死。

%89
树人开始扑上荡魂山。

%90
很快,这些树人就尝到荡魂山的厉害,魂魄震荡,浑身剧颤,一大片一大片的枝叶掉落下来。

%91
但即便如此,密密麻麻的高大树人们,前仆后继,相互推挤着,向山上挤来。

%92
荡魂山不愧是天地秘境之一,才刚过山脚,就有上百头树人灭亡,不费方源一丝仙元。

%93
方源俯瞰山下情景,眉头越皱越深。

%94
这些树人虽然麻烦,但实质上威胁性很小。

%95
因为它们行动缓慢,方源完全可以打游击战术,一步步将这些树人绞杀掉。

%96
方源觉得不妥:“这次的地灾未免太过轻易。难道说,这次天意觉得无法一次除掉我,所以这一次的重点,就在于消耗我的底蕴?”

%97
ps:今晚微信公众号上,有幸运大转盘的活动,:3准时开始,回馈广大的读者朋友们,就看大家的手气啦!祝大家好运!

%98
今晚两更,满一百月票加更一张,不过要到9点稍后了。大家可以玩玩微信,聊聊天,等下一更。

\end{this_body}

