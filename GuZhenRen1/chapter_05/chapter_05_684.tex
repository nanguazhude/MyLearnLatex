\newsection{暂避风头}    %第六百八十七节:暂避风头

\begin{this_body}

%1
白云在身旁穿梭,风在两耳旁呼啸,方源在高空中疾飞。

%2
一丝血迹还挂在他的嘴角,之前和四位八转激战,他不是没有付出代价。

%3
尤其是最后,当他用计斩杀了周雄信之后,剩下的三位八转便决意撤退,他们相互合作,破坏了方源的大阵。

%4
这座大阵并非是方源布置,而是当初影宗暗中出手,结合了附近天地的自然道痕,从而形成的超级大阵。

%5
这座超级仙阵威力不凡,和当初方源坑害夏槎一批人的那座大阵相差仿佛。

%6
这座仙阵平时隐匿不显,不露丝毫痕迹,影宗布置出来是为了当时机成熟的时候,提供给逆组织的一个基地大本营。

%7
中洲十大古派势力各自雄踞一方,侵占了最多的资源。其余的诸多正道、散仙、魔仙都在十大古派的夹缝间生存。

%8
偏偏门派制度不像家族制,擅长的就是挖掘人才。很多修行的天才,不是才情天赋不够,往往是修行的资粮不足,导致自家不能更进一步。

%9
天长日久,这些人自然和中洲十大古派有了间隙,积累了许多的矛盾。

%10
影宗有感于此,便暗地里勾连对天庭、十大古派不满的蛊仙,将他们联合起来,组成一个结构松散,十分隐蔽的逆组织。

%11
逆组织和僵盟性质差不多,算是影宗的下宗。就像是中洲十大古派属于天庭下宗一样。

%12
逆组织被影宗寄予厚望,但可惜的是,就算它隐藏得再深,也不是紫薇仙子的对手。

%13
如今逆组织早已瓦解,荡然无存,紫薇仙子在这里居功至伟,扫荡得极其彻底。

%14
方源既然现身,四处破坏中洲炼蛊大会的分赛场地,自然早就考虑到了天庭的追兵和围剿。

%15
这处超级大阵就是他的底牌。

%16
当他预感不妙,什么地方都不去,就专门逃到这里。

%17
周雄信利用流言笼杀招,将方源困住时,心底雀跃,却是万万没有想到,他早就落入到方源的埋伏圈里。

%18
超级大阵没有辜负方源的期待,当他悍然催起时,便立即将流言笼和外界的联系阻断,随后将这记仙道战场硬生生摧垮。

%19
再之后,仙阵还阻挡住了三位八转一息时间,为方源争取到了最关键的时刻,使其成功地斩杀了周雄信。

%20
最后,在仙阵的辅助下,方源和剩下三位八转激战。奈何仙阵连续爆发,已经是强弩之末,而且这种利用自然道痕建立的仙阵,有一个天然的缺陷,就是害怕周围环境被影响和毁坏。

%21
三位中洲八转看准这点,连续出手,摧毁了仙阵。

%22
仙阵一毁,方源的下场和周雄信一样,也立即遭受了反噬。

%23
仙道杀招就是这样,往往威能越强,被破解后反噬就越厉害,这是一把双刃剑。

%24
不过好在方源早有心理准备,完全和被打了个猝不及防的周雄信不一样。

%25
三位中洲八转没有乘机得到便宜,更加坚定了撤退之心。

%26
方源没有仙道战场,更没有仙阵阻截他们,只能任由这三人逃走。

%27
他的战斗力是八转巅峰,可以击败这三人,但是要留下他们的性命可就难了点。除非是困住他们,或者是利用罕见的绝世良机,比如斩杀周雄信这次。

%28
方源平时的时候,依靠落魄印、春剪、冬裘、五指拳心剑等等手段,战力稳定在八转高阶。

%29
他辛辛苦苦养了这么多的仙蛊,可不是说笑的。

%30
而当他变化成上古年猴的时候,一身战力就会暴涨到八转巅峰。

%31
造成两者之间区别的,就是他身上的道痕!

%32
红莲真传争夺战上,方源不胜不负。龙宫争夺战上,方源失利。但在这之后,天相杀招不负众望,终于查找到了兽劫洞天中关键仙蛊的收藏之地。

%33
在此之间,方源也通过梦境,提升了变化道境界。

%34
方源先运用见面曾相识,伪装成另外身份,混入洞天中。随后又有偷道手段,成功盗取了一切仙蛊。

%35
那兽劫天灵虽然清醒精明,但哪里是狡诈如狐的方源的对手?

%36
做到这步后,方源再无顾虑,出去后换了一个身份又进来,继承了兽劫真传。

%37
可怜的兽劫天灵被他骗得团团转,直至最终整个兽劫洞天被吞并,也没有发现方源的真正身份。

%38
这座兽劫洞天,正是关键之处。

%39
平均而言,一场地灾能为蛊仙带来二百五十道痕,天劫是七百五十,浩劫乃七千二百五十,万劫更多,多达八万六千七百五十。

%40
当然这都是均数,有时候灾劫凶猛,蛊仙若撑得过去,收获会增添许多。

%41
七转蛊仙每十年一场地灾,八转每十年一场天劫。

%42
七转每五十年一场天劫,八转每五十年却是一场浩劫。

%43
七转每百年有一场浩劫,八转每百年迎来一场万劫!

%44
一场万劫的道痕有八万多,一场浩劫所得差点连前者的零头都不到。八转蛊仙往往能碾压七转,根源就在于双方的道痕差距实在太大。

%45
方源吞并的兽劫洞天,是已经超过了一次浩劫的。虽然洞天原主兽灾仙人撑过第一场浩劫时就挂了,但遗留下来的兽劫洞天而后又撑过了许多场灾劫,距离第二场浩劫也是不远了。

%46
也就是说,兽劫洞天中其他的道痕暂且不论,最主流的变化道痕就有十多万!

%47
方源本身的变化道痕有近十万!其实本来只有五万,但在光阴长河中他杀了不少天庭七转,吞噬大量仙窍福地之后,获得了诸多流派上的道痕增益。

%48
所以,当方源吞并了兽劫洞天之后,变化道痕就有二十多万!

%49
这个数字就恐怖了。

%50
专修变化道的八转蛊仙,要拥有这么多的变化道痕,至少得过两次浩劫。

%51
薄青是两次浩劫的八转蛊仙,龙公同样如此。

%52
而当方源变身成太古年猴的时候,这些变化道痕会变化成宙道道痕。

%53
别忘了,方源吞噬了夏槎的洞天后,宙道道痕就有七万多。光阴长河中,斩杀了三旬子,吞并各自仙窍后,宙道道痕超越了十万。

%54
二十多万加十万余,方源变化为太古年猴时,宙道道痕达到了骇人的三十多万!

%55
这是什么概念?

%56
专修宙道的八转蛊仙,若无特殊奇遇,就得渡过第三场浩劫。

%57
而渡过第三场浩劫,那就是九转尊者了。

%58
当然,九转蛊尊的道痕,绝不只是三十万,这个数字只是尊者最初的起点。

%59
即便如此,方源单靠宙道道痕,已然达到了准仙尊的地步。再搭配上八转宙道仙蛊,诸多优良的八转宙道杀招,使得他一身战力自然而然地暴涨到了八转巅峰的程度。

%60
之前,方源和厉煌对战,利用落魄印,都突破不了厉煌的阳莽背火衣。但是现在,当方源化身成太古年猴时,却很期待春剪杀招和阳莽背火衣的对拼!

%61
“但是对付龙公,还比较困难。”方源目光深沉。

%62
影响战力的因素很多,道痕是一方面,仙蛊是一方面,仙窍根基是一方面,手段又是另一方面。

%63
方源在道痕上不缺。宙道仙蛊有三只,马马虎虎。但是仙道杀招上有些薄弱了,春剪等手段虽然经过屡次改良,变得更加强大,但是不改根本。来来去去就是这么几招。

%64
而龙公呢,方源目睹过他和紫山真君之战,知道他身上不论是变化道痕,还是气道道痕,都至少有三十万的程度。

%65
龙公身上的八转仙蛊,绝对比方源的三只宙道要多。

%66
龙公的仙道杀招手段丰富多彩,层出不穷,尤其是三气归来等等杀招,乃是惊天动地鬼神的绝技,极难招架。

%67
如此种种因素,才造就了龙公准仙尊的战力!

%68
准仙尊……能够摊上一个“尊”字的层次,不是那么轻易达到的。

%69
正因为如此,龙公才能在几乎所有东海八转的面前,强夺走了龙宫仙蛊屋。从这一点,便足见他的强大。

%70
当然,他也因此负伤。

%71
准仙尊,毕竟不是仙尊,多少一字,就是天差地别。

%72
方源绝不缺少自信,他对自己向来有充沛的信心。但这种自信的基础前提,是他自知!

%73
他对自己有多少斤两,知道的非常清楚。

%74
所以,方源忌惮龙公,没办法,战力上就是比不过!

%75
“这千夫所指的人道手段,是越来越强了。即便我对人道没有多少涉猎,也能感觉越发的不妙。”

%76
方源眉头轻皱。

%77
别忘了,他的智道造诣浑厚,就连紫薇仙子都不敢轻视。智道蛊仙心中的情绪,向来不是空穴来风,都预示着什么。

%78
“千夫所指单靠阎帝杀招,难以化解。我在中洲四处现身的次数越多,时间越长,千夫所指杀招似乎就越厉害。”

%79
“可是如今,火还未真正点起来,虽然中洲多处出现意外,都是小打小闹。”

%80
想到这里,方源下定了决心:“养兵千日用兵一时,是时候将他们放出来了!”

%81
之前和周雄信对战,方源还耐心地守候了一番,却未等到龙公。

%82
如今超级仙阵一毁,方源也没了自信来单独对战龙公。

%83
不得不说,龙公从始至终都未出现,却带给方源更大的威胁,逼迫方源隐匿自身,转嫁风险。

%84
“接下来,就让白凝冰这些人四处破坏吧。我还是暂避风头。这场旷世大战,才刚刚开始呢!”方源的眼中一片冷酷。

\end{this_body}

