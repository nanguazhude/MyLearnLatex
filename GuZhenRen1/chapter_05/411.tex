\newsection{野蛊来投}    %第四百一十二节:野蛊来投

\begin{this_body}

%1
传闻中,这世界上有一条长河,名为光阴长河!

%2
它从过去发源,流经现在,向未来奔腾。

%3
世间万物都若河中之鱼,河流湍急,几乎所有的鱼都只能顺势而下。

%4
若无这光阴之河,世界将完全静止,成为画面。有了这河,一切才可变化,世界才能生动,或是衰减或是繁华。

%5
“再次来到光阴长河之中了。”方源感慨无比。

%6
他举目四望,只见河面辽阔至极,虽是称之为光阴长河,却好似海面一样的宽广。

%7
在漆黑的空中,一条宽阔的大河,滚滚流淌,奔腾不息。

%8
河水本身是苍白色的,但是亿兆兆的浪花,不断碰撞时,都会迸溅出缤纷的色彩。

%9
流光溢彩,美轮美奂,映照在方源的脸上。

%10
方源微微失神了一下,旋即清醒过来。

%11
他打开仙窍门户,放出太古年猴。

%12
“渣渣渣。”这头庞然大物再次回归家园,显得非常兴奋。

%13
扑通!

%14
它落入河水当中,顿时掀起漫天的浪花。

%15
方源也变作一头上古年猴,体型比太古年猴要小得多,此时悬浮在空中,他可不是正宗的年兽。虽然此时此刻,变化道道痕都悉数转变成了宙道道痕,让他适应了这里的环境。但是对于光阴河水,能不沾染还是不沾的好。

%16
“出发吧。”方源一声令下,太古年猴便泅水而行。

%17
先前痛揍了它一顿,现在就看到了好处,比上极天鹰要听话得多。

%18
不过,虽然依靠百八十奴杀招取巧,但因为方源本身魂魄底蕴大降,驾驭这头太古年猴还是相当勉强。

%19
“现在只是平常时期,就有着奴役艰难,魂魄沉凝的感觉。若是在激斗中,命令太古年猴,恐怕有心无力。”

%20
方源心中暗自警惕。

%21
这是一个破绽,但没有办法,方源已经做到了他力所能及的极致。

%22
一路默默前行。

%23
哗哗哗……

%24
浪潮声不绝于耳。

%25
方源还是头一次,以肉身的形式,进入这里。

%26
前几次都是借助春秋蝉,将一股意志挪移到过去。因为天意等种种影响,他对光阴长河都是惊鸿一瞥,难见全貌,如今却是看了个够,可谓大开眼界。

%27
光阴长河乃是天地秘境,早已经记载于《人祖传》中,一直广为人知。

%28
但是真正能够以肉身进入其中,非得是宙道积累雄厚的蛊仙强者。方源能够进入,本身能力是不足的,托了变化道的福,还借助紫山真君、黑凡的遗藏、传承。

%29
光阴长河当中,并不死寂。

%30
方源先是见到了一两只野生年蛊,相互追逐飞翔,又时而钻入河水当中去。

%31
然后,他又目睹一群群的宙道野生蛊虫,在漫天飞舞,多的宛若蝗群。

%32
“嗯?”五六只野生年蛊,主动飞到方源的面前,然后静静地栖身到方源的身上,不再动弹。

%33
“原来如此。”方源转念一想,便有了明悟。

%34
他现在变作上古年猴,因为八转态度蛊、见面曾相识等等因素,野生蛊虫自然分辨不了真假,都将他当做野生的上古年猴了。

%35
这几只宙道凡蛊,都被方源身上充裕丰富的宙道道痕吸引,认为是适合生存的地方,所以就主动投靠了方源。

%36
的确是这样。

%37
在大自然中,蛊虫本身是很脆弱的。虽然野生蛊虫,可以汲取天地自然间的元气,迸发威能,但仍旧处境危险。

%38
所以,寄生在强大的生命上,就成了这些野生蛊虫的生存之道。

%39
方源前行了半盏茶功夫,身上已经多了数百只野生凡蛊。

%40
大部分都是日蛊、月蛊,没有一只野生年蛊。

%41
至于方源下方的太古年猴,则收获更多。

%42
这头太古年猴本身的野生凡蛊,都在战斗中,被方源等人打坏了,残余很少。

%43
眼下,却是得到了大量的补充。

%44
太古年猴自然比方源要更加吃香,深受宙道野生凡蛊的青睐,它泅水而行,身边总是萦绕着一群野生宙道蛊虫,像是围着一层薄薄的黑雾。

%45
不过这种情形,再持续了片刻之后,迅速地减缓下来。

%46
就像是野兽占据领地,当方源和太古年猴身上,寄生了大量的野生凡蛊之后,其他的野蛊便可能觉得生存的空间狭小,因此不再前来驻足停息。

%47
方源想了想,没有对自己身上的这些野生凡蛊动手。

%48
依凭他如今的能耐,要炼化了这些野生宙道凡蛊,自然是可以的。

%49
不过,这却有违了方源伪装的初衷。

%50
再者说,这些野生宙道凡蛊,也不过如此。除非是仙蛊,否则对方源而言,绝大多数的凡蛊都没有太多的吸引力。

%51
“吼!”

%52
行进的路途中,一头上古年虎从水中跃出,然后头也不回地直接撤离,跑到了远处去。

%53
这样的情形,方源已经见过多次。

%54
有着太古年猴护航,就是方便,简直就是一份巨型通行证。

%55
事实上,除了年兽之外,还有月兽、日兽。和年兽差不多,这些月兽、日兽分别以月蛊、日蛊为食。

%56
总共一个时辰过去,光阴长河上仍旧是河水滔滔,周围黑幕深深。

%57
景色没有变化,几乎让人有原地驻足,从未前行的错觉。

%58
一只太古年猴大半个身躯,沉入河水当中,泅水前行,宛若河面上漂浮着的一座小山。

%59
而方源则变作上古年猴,悬浮在河面上空,静静漂行。

%60
方源心里有数,因为他有紫山真君的遗藏,心中的感觉越发强烈。只要循着这份感应,一直前行,就能寻找到魔尊幽魂掌控着的那道红莲真传。

%61
而阻碍他的,除了光阴长河的险峻之外,便是天庭的埋伏和追杀了。

%62
轰隆!

%63
忽然间,一声轰鸣,波涛澎湃的光阴长河中,忽然向上射出一道激流。

%64
激流的爆发位置,就在方源的左前方的不远之处。

%65
“渣。”太古年猴轻呼一声,一直以来惬意悠然的脸色上,显露出一丝凝重之情。

%66
“这是突泉。”方源亦是瞳孔微微一缩。

%67
光阴长河,并非安全,有些河段非常危险,有着不一样的变化。

%68
突泉就是其中之一。

%69
行进在这个河段,要时刻提防河水的突然爆发,形成突泉。一旦被突泉集中,海量的宙道道痕刻印下来,非死即伤,甚至更可能削减寿命。

%70
方源谨慎地挪移到太古年猴的头顶上空,他缺乏对抗突泉的手段,只能利用这头太古年兽的皮糙肉厚来硬抗。

%71
这个河段是必须要渡过去的,是前往红莲真传的必经之路。

%72
就在方源提起十二分精神,艰难渡河的同时,位于西漠的光阴支流附近,来了三位蛊仙。

%73
一位七转蛊仙,风姿卓绝,两位八转蛊仙,渊渟岳峙。

%74
正是凤九歌,以及天庭的两位八转蛊仙。

%75
紫薇仙子的计划,就是通过方源,来顺藤摸瓜,带出红莲真传。

%76
如今方源进入光阴长河,要取红莲真传,此时自然要派遣重兵,前来收缴。

%77
“不想这天外之魔,真正厉害,居然和凤贤弟分庭抗礼。”

%78
“有我等二人相助,即便是他才情再高,也无法蹦跶下去了。”

%79
两位八转蛊仙和凤九歌汇合的时间,并没有多久,凤九歌并非光顾脸面之人,直接将方源战力暴涨的情报,详细地提供给了这两位蛊仙。

%80
“我等联手,应当能擒杀了方源。”凤九歌微微点头。

%81
方源的战力,和他相仿。面对一位八转蛊仙,已经勉强,同时面对两位,自然更加艰难。

%82
尤其是现在,凤九歌的鼓点乱心音杀招,得到紫薇仙子的再次改良,已然成为了对付方源的王牌手段。

%83
“嗯?这是梦境!”片刻后,凤九歌等三人却是不得不驻足,无奈地看向由梦境构成的防线。

%84
“看来我们只有等待了。”

%85
“梦境不断自行流转,不久后防线就会出现缝隙漏洞,让我们进入的。”

%86
“除了我等之外,光阴长河当中还有黄史上人。他可是近万年来,最擅长在光阴长河中争斗的人物!方源不会是他的对手。”

%87
三仙商议一番,决定留守,等待战机的出现。

%88
光阴长河当中。

%89
“渣渣渣。”

%90
太古年猴龇牙咧嘴地叫着,身上负伤。

%91
方源扭头,望着背后的河面,心有余悸。

%92
“幸亏有太古年猴为我护航,否则的话,怎么可能如此轻易地闯过这片天险?”

%93
突泉威能浩瀚,至少有七转杀招的厉害,大部分都是八转级数,方源甚至见到了一个媲美九转手段的突泉。

%94
这突泉猛地爆发出来,简直就是一道逆流冲天的巨大瀑布。

%95
好在它爆发的位置,距离方源较远,有惊无险。

%96
“太古年猴伤势不轻,可惜我手段有限,根本无法为它治疗。”方源查探了一下,在心中叹息。

%97
其他流派的治疗手段,在这里饱受压制。宙道的治疗法门,方源只有人如故仙蛊。可惜这只仙蛊,只对人族方有良效。

%98
方源只能让太古年猴自行恢复。

%99
“要不要暂时停留,先休整一下?让太古年猴恢复完全?”

%100
方源不免犹豫。

%101
要去往那道红莲真传,突泉河段只是第一关而已,接下来方源还要面对阴织蛛,一指流鲨群,以及一片较为特殊的刀剑河段。

%102
这三关比突泉河段,都要危险得多。

%103
“突泉河段?”黄史上人望着眼前河面,眉头皱了起来。

%104
“依照紫薇大人的指点,方源等人就在前方了。”旋即,黄史上人眼中爆闪出一阵精芒,然后他坚定地飞入突泉河段中去。

\end{this_body}

