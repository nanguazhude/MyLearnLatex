\newsection{自己吓自己}    %第二十节:自己吓自己

\begin{this_body}

%1
方源的情形,尤其古怪,让戚灾不得不在意。

%2
“他究竟身怀何种道痕,为何使用各种流派的杀招,都能增幅威能,而不见任何损耗?!”

%3
戚灾表面上运筹帷幄,稳坐钓鱼台,实则在暗中苦思冥想。

%4
“据我所知,历史上只有一个人,做到这种程度……那就是狂蛮魔尊!”

%5
人族历史上,有数的魔尊强者狂蛮,他开创力道,是力道始祖,同时也兼修变化道。

%6
根据历史中的明确记载,他能千变万化,同时道痕可以随意转变。

%7
举个例子,当狂蛮魔尊变成电狼,他就能催发雷道杀招。当他又变成泥沼蟹,他又会运用土道杀招,不受其他道痕的抑制和减损。

%8
而寻常的变化道蛊师,通常只能专修一种变化。比如在北原的王庭争霸中,成龙、成虎两位蛊师,就只能变作龙形和虎形。

%9
至于变化道的蛊仙,诸如曾和方源交手的贺狼子,他能变化三种狼的形态,身负不同道痕。但相互切换的时候,就要等候一段时间,趁机运用手段,将原本的道痕清理了,防止不同流派的道痕之间相互干扰。

%10
这个弊端,可谓为难着从古至今,修行变化道的蛊师、蛊仙们。

%11
除了一人。

%12
那就是狂蛮魔尊。

%13
他似乎可以随意变化,不考虑道痕之间的干扰。

%14
可惜,狂魔魔尊陨落之后,他的这个手段并没有随之流传下来。无数先贤,才智卓绝之辈,想要领悟到狂魔魔尊的手段,皆以失败告终,毫无所获。

%15
“难道说,我眼前这个敌人,恰巧继承了狂魔魔尊的这个变化道的关键传承?”戚灾想到这里,心头顿时火热无比。

%16
除了这个解释。他想不到更大的可能。

%17
“当然,蛊师修行,繁杂难测,龙蛇混杂。手段之多宛若夜空繁星,不可用常理估算。”戚灾眯起双眼,仍旧是相当的冷静。

%18
其实,他猜测方源获得狂蛮魔尊的真传,还是低估了方源。

%19
方源之所以能达到如此程度。是因为魔尊幽魂炼出的九转至尊仙胎蛊。

%20
这只仙蛊可不一般!

%21
不仅是魔尊幽魂借以重生的关键宝物,而且还是他寄托希望,今后可以超越其他所有尊者的基石。

%22
当年,幽魂魔尊死后,不甘心现状。

%23
于是,建立影宗,辛辛苦苦筹谋。期间,他不仅自己整理了平生所学,而且还四处搜刮到无数的传承。这些传承中,有蛊师传承。不乏精品,能带给魔尊幽魂许多灵感和全新的思路。更有蛊仙传承,甚至囊括不少其他尊者的传承。

%24
就连最神秘的红莲魔尊,影宗都能搞到一道他留下来的真传。更何况其他尊者呢?

%25
魔尊幽魂驾驭影宗,又开设下宗僵盟,分部五域,历时数万年,潜心积累,不断研究,付出无数心血和牺牲。

%26
最终。才海纳百川,提取精髓,融汇一炉,得到了九转至尊仙胎蛊。

%27
可以说。这只仙蛊,已经不单单只是幽魂魔尊的魂道结晶,更是他兼收并蓄,有容乃大,囊括无数流派精华,许多尊者真传。凝结而成的绝世奇珍。

%28
天意虽然安排方源为棋子,但方源最后关头,却是凭借天外之魔的本能,挣脱枷锁,将至尊仙胎蛊纳为己用。

%29
戚灾心中存疑,虽然方源只是六转气息,但他却丝毫不敢大意。

%30
他在试探方源。

%31
他是七转蛊仙,就算没有七转仙蛊,六转通常也是有的。

%32
之前从烂泥浆中炼出怨气泥球,而后化成泥人倪健,正是仙道手段。

%33
但他没有动用仙蛊。

%34
仙蛊是蛊仙的底牌,动用起来,消耗的可是仙元!

%35
戚灾不过是下等福地,等闲时刻,他不会动用仙蛊。

%36
当然,仙元不是主要原因,主要因素在于一旦动用仙蛊,就会被敌人判知。

%37
要知道蛊师手中的蛊虫,究竟有多少,是什么,都是严格保密的。一旦情报流露出去,就会被针对和克制。

%38
蛊仙也同样如此。

%39
初次对战,不知根知底,蛊仙之间都要有一番试探。

%40
戚灾试探方源的同时,方源也在试探他。

%41
各种凡道杀招,在方源手中层出不穷,再配合他清秀俊逸的形象,很是吸引了戚荷的目光。

%42
和戚灾一样,方源手中虽然也有仙蛊,但不能滥用。

%43
虽然七转剑道仙蛊,威能强大,但方源只是六转青提仙元,这点远不如七转蛊仙戚灾。

%44
若非方源投靠了琅琊派,他没有大量的仙元石及时补充仙元,此战他必输无疑。

%45
双方你来我往,攻防互换,声势煊赫,让戚荷看得目不转睛,心中连连惊叹。

%46
不可避免的,戚灾被方源渐渐压入下风。

%47
凡道杀招,就算再优秀,也没有一只仙蛊。因此受到戚灾身上道痕增幅,也有极限,绝不会超出单纯动用仙蛊的范畴。

%48
漫漫历史长河之中,唯有一例凡道杀招,催发出来,能媲美仙蛊。

%49
那就是傲骨魔君沈桀骜所创的白骨战车。

%50
此人是八转魔道蛊仙,的确有自傲的资本。单纯论这项成就,就连历史上的十大尊者,都比不上他。

%51
仙尊、魔尊虽然曾经无敌天下,但并非代表无所不能。

%52
术业有专攻,当初巨阳仙尊、盗天魔尊,都要寻找长毛老祖来帮助他们炼蛊呢。

%53
反观方源,虽然身上的道痕积累,远不如戚灾,凡道杀招的威能增幅有限,但胜在手段层出不穷,应对起来游刃有余。

%54
所以,渐渐的,方源攻多防少,越加占据主动。

%55
戚灾心中雪亮,也知道战局如何。他眉头微皱,开始反击。

%56
战场杀招气锁一方。

%57
顿时,附近长空被禁锢起来,空气凝住不动,形成一个半透明的临时战场。

%58
方源淡淡一笑。毫不惊慌。

%59
这种战场杀招,只是凡道级别,对于他这种掌握仙蛊的蛊仙而言,威胁性很小。

%60
除非是没有仙蛊的六转蛊仙。才会如临大敌。

%61
而且这种战场杀招,方源也有。

%62
当即,他使出雪松子的招牌,战场杀招雪境。

%63
于是,气锁一方被排挤开来。空中寒气四溢。

%64
戚荷终于忍不住,惊叹出声。

%65
在她的眼前,战场被泾渭分明地划分为两块。一块冰天雪地,一块气息凝固。

%66
戚荷的眼界,还局限在凡人阶段。这种战场杀招,虽然只是凡道级数,但也是每个凡俗势力的底牌,不到决战的关头,一般不会轻易动用,以防次数多了。被其他有心人破解了去。

%67
这一次,她骤然看到两个战场杀招,在她眼前上演,顿时让她感觉此行不虚,有大开眼界的满足感。

%68
见到方源仍旧游刃有余,戚灾终于不耐烦,率先动用仙蛊。

%69
他也是个谨慎的人。

%70
没有直接动用仙道杀招,而是催动仙蛊。

%71
一只六转的气道仙蛊。

%72
方源不知道是什么,戚灾才不会傻到一边出招,一边开口报出仙蛊的名号。

%73
仙蛊祭出。场面顿时发生翻天覆地的剧变。

%74
战场气锁一方,陡然全消,为仙蛊增势。雪境没有抵挡得住,被一股入侵的阴沉气索。直接洞穿。

%75
这股长索一般的阴气,缭绕身姿,仿佛一条碧绿毒蛇,长达数丈,向方源迅速袭来。

%76
方源叹息一声,直接催动七转仙蛊剑遁。

%77
嗖!

%78
瞬间。他好似化为一柄利剑,刺穿空气,直接飞走。

%79
“好快!”戚荷张开嘴巴,十分吃惊。几乎一眨眼间,方源已经飞到她的视野尽头了。

%80
毕竟是七转层数的移动仙蛊,速度自然卓绝惊艳。

%81
戚灾也为之一愣。

%82
他原本估计,方源会是雪道、风道或者暗道,因为他之前使用的凡道杀招,多是这三个流派。

%83
但方源真正运用仙蛊的时候,却是剑道!

%84
并且,还是一只七转剑道仙蛊。

%85
“搞什么鬼?难道他也不是什么变化道蛊仙,而是剑道蛊仙?”戚灾只感觉脑筋要有一种打结的趋势。

%86
试探了这么久,他觉得已经多多少少对方源了解了一些。

%87
但真正动手了,方源居然掏出一只七转剑道仙蛊!

%88
还他喵的飞得这么快!

%89
“若是换做我之前,就只有望而兴叹。可惜,你运气不好,刚好不久前,我设想三年的仙道杀招成功了。”

%90
戚灾冷笑三声,开口长啸。

%91
与此同时,他浑身喷涌出澎湃的气息,气息包裹着他和戚荷,同时渗透进气宗狮的每一寸皮毛骨肉之中。

%92
气宗狮痛得惨嚎一声,双眼赤红,向着方源的背影疾奔而去。

%93
在戚灾的仙道杀招的作用下,气宗狮的速度极其惊人!

%94
半刻钟不到,气宗狮就载着戚家两位蛊仙,追赶上了方源。

%95
方源见此,顿感头疼。

%96
戚灾再次催动仙蛊,发出阴气长索。

%97
很显然,就是刚刚的那只仙蛊。

%98
阴气长索,向方源飞绕过来,迅速接近。

%99
“他没有动用其他仙道杀招,只是单纯运用了一只六转仙蛊,难道是因为他的这个移动杀招,包含蛊虫数量很多,需要牵扯去大量心神吗?”

%100
方源眼中精芒四射。

%101
危机关头,他的心仍旧是冰雪般冷静,不断思索,显示出他底蕴雄厚的战斗造诣。

%102
催动仙蛊的时候,蛊虫气息就会澎湃外溢。而催动仙道杀招时,外溢的气息就会复杂庞大许多。

%103
所以,戚灾很轻易地就分辨出方源所用的仙蛊,高达七转,隶属剑道。

%104
而方源也很容易,就得出此刻戚灾出手,只是单纯用的一只六转仙蛊。

%105
当然,这只是常识,不是绝对的。

%106
也有一些少见的手段,可以遮掩仙蛊气息,达到误导的作用。

%107
飞剑蛊!

%108
方源手指一指,仙蛊射出。

%109
飞剑仙蛊锋锐无当,阴气长索在瞬间被绞碎成渣。

%110
戚灾心头轻颤,再次一惊:“怎么又是一只七转剑道仙蛊?”

%111
“难道他真的是剑道蛊仙?”

%112
“那这如何解释,他刚刚使用凡道杀招时的状况?”

%113
“他的仙蛊,怎么都是七转?”

%114
“难道说,他本身就是七转蛊仙,故意掩盖气息成六转,想扮猪吃老虎?”

%115
戚灾哪里料得到方源的真实情况。

%116
他性情谨慎,思维缜密,越是分析,越是惊疑,自己吓住了自己!

\end{this_body}

