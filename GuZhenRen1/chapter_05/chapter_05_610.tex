\newsection{石岛陷阱}    %第六百一十三节:石岛陷阱

\begin{this_body}

%1
“咳咳咳……”上旬子趴在恒舟的甲板上,不断咳嗽,呛水得很严重,满身水渍。

%2
而在他的旁边,这时下旬子,已然是昏死过去,伤势沉重,状态极为不妙。

%3
八转蛊仙清夜走了过来,面色平静地看向上旬子:“另外两位仙友,已经是被光阴长河卷走,实在搜刮不到。”

%4
上旬子狠狠喘息了几口气,勉强站起身来,看了一眼下旬子,脸色惨白地道:“他二人已经战死,方源魔头实在是厉害!今古亭的残余,搜救得如何?”

%5
四旬子乃是亲兄妹,情义深厚,但此刻上旬子居然更关心今古亭,而对濒死的下旬子不闻不问,更没有悲伤愤怒之情。

%6
清夜微微一笑:“最核心的仙蛊,并未有损,都已经收拢上来了。”

%7
“这就好!”上旬子吐出一口浊气,脸色缓和了许多。

%8
清夜点头:“静室我已经给你准备好了。”

%9
上旬子:“有劳了,我这边下去做法,复活我的弟弟、妹妹。”

%10
原来这四旬子乃是同胞兄妹,血缘最亲,练就了一份绝妙杀招。任凭三人阵亡,只需要当中任何一人活着,就能动用杀招,将其他三人迅速复活。

%11
正因如此,上旬子才没有丝毫的悲伤之情。

%12
片刻之后,四旬子伤势全无,焕然一新地登上甲板,加入了战斗当中。

%13
方源留下的第二批太古年兽,正围绕着恒舟,从四面八方夹攻,拼命纠缠……

%14
河水奔腾,水浪声充斥方源双耳。

%15
此刻的他,已经还原本来面貌,进入了仙蛊屋雏形中休整。

%16
有了这座仙蛊屋雏形,方源就等若有了一份基地,方便得很。

%17
“那恒舟没有追来么……”此刻已经脱离战场有一段时间了,方源心中微微放松下来。

%18
仙蛊屋雏形的速度,肯定是比不上恒舟的,它擅长的是隐匿行迹。

%19
有今古亭在,仙蛊屋雏形会被识破踪迹。但是换做恒舟,恐怕是察觉不到仙蛊屋雏形了。否则的话,这么长时间,又有着速度上的优势,恒舟怎么还不追来?

%20
“暂时是安全了。”影无邪深呼吸一口气,心中还有余悸,“刚刚那时刻真是紧张。”

%21
“没错。若非宗主大人及时将今古亭击溃,我们恐怕就要陷入战斗的泥潭,无法自拔。”妙音仙子微笑道。

%22
白凝冰在一旁沉默不语。

%23
黑楼兰亦是静静地凝望着仙蛊屋外,那浩瀚磅礴的光阴河流。

%24
方源的宙道分身,一直紧闭双目,眉头微皱。他全力感应,但仙窍中的本命蛊春秋蝉并无多少反应。

%25
事实上,按照幽魂魔尊的推测,春秋蝉的转数越高,感应的范围就越广阔,更容易搜寻到红莲真传。

%26
方源的春秋蝉转数只有六转。虽然有着大量的相应仙蛊方,方源也尝试炼制了几次,都以失败告终。

%27
这着实遗憾,但方源已经尽力了。

%28
数月以来,他绝对是拼尽全力,魂道修行、探索梦境、改良杀招、勒索南疆正道、谋划全局、防备推算、吞并各大仙窍等等,耗尽了他的精力和时间。

%29
能够从中挤出一些时间来炼制春秋蝉,可谓是不大不小的奇迹。但真正的奇迹没有发生,方源的尝试皆已失败告终。

%30
“春秋蝉虽是六转,但我却已经从其他方面进行了弥补。”方源眼中闪过自信的光芒。

%31
此次进入光阴长河,乃是他蓄谋已久。方源早就在运道方面催用了燃魂爆运等等手段,达到了能力的极限后方才善罢甘休。

%32
在这茫茫的光阴长河中搜寻一两座红莲石岛,当然要看运气。

%33
嗷吼!

%34
河水忽然湍急起来,兽吼声传入众人耳中。

%35
“前方有两头太古年兽正在争斗!”视察之后,影无邪讶然。

%36
再细细一看,这两头太古年兽已经争斗了好一会儿,均是强弩之末。

%37
方源哈哈一笑,这运道不就来了么?

%38
他飞出仙蛊屋雏形,见面曾相识发送,变作一片河水,混入光阴长河之中。

%39
随后,他打开仙窍门户一丝,催动太古年兽钓来阵!

%40
那两头太古年兽察觉到美食的强烈气息,顿时动作一滞,同时停下,然后又争先恐后地向方源奔来。

%41
它们没有发现任何敌人,但却感应到了一道光阴支流,潜藏在河水当中。

%42
太古年兽想都没想,立即冲进这道光阴支流里去。

%43
这道光阴支流当然属于方源的至尊仙窍。

%44
换做之前,方源肯定是不敢就这样钓来太古年兽,太古年兽放进来,一定会对至尊仙窍造成破坏。

%45
但今时不同往日,方源已经有了年华池,并且还是两座!

%46
年华池沟通光阴支流,等若是巨大水坝。

%47
太古年兽被大阵引诱,顺着光阴支流,一路冲进了“水坝”当中,直接成为瓮中之鳖。

%48
方源畅笑一声,收了变化杀招,飞回到仙蛊屋雏形中去。

%49
那两头太古年猴陷入到年华池中,左右都找寻不到食物,察觉到不妙,想返回光阴长河,但哪里可能?

%50
方源施展奴道手段,很快就将这两头太古年兽收服。

%51
仙蛊屋雏形一路潜行,搜寻红莲石岛。

%52
方源陆续又遭遇了不少年兽,都被他来者不拒,直接诱骗到年华池中,全部收服。

%53
很快,方源手中的太古年兽数量就上涨到六头。

%54
上古年兽、荒级年兽就更多了。

%55
除了年兽之外,光阴长河当中还有其他生命,比如日兽、月兽,又比如阴织蛛,都被方源明智舍弃。

%56
年华池的作用,在这一刻得到了充分的显现!

%57
有了它,方源能够直接诱骗太古年兽进来,而至尊仙窍仍旧非常安全。没有年华池约束的话,太古年兽可以顺着光阴支流,来到至尊仙窍中的任何一个地方。因为光阴支流贯穿整个至尊仙窍。

%58
“等等!”行进的过程当中,方源宙道分身忽然睁开双眼,眼中精芒爆炸。

%59
他好像有了重大发现!

%60
宙道分身旋即又闭上双眼,眉头皱得更深,极力感应。

%61
然后,他伸出手指,指向一个方向:“那边似乎有什么东西,让春秋蝉微微颤动了一下。”

%62
宙道分身的语气也不太确定。

%63
但这是唯一的线索,方源立即操纵仙蛊屋雏形,转折方向,向前方驶去。

%64
随着距离缩短,方源分身脸上喜色逐渐浓郁起来:“感应正逐渐加强,我们的运气真的不错,这么快就有了成果!”

%65
又行进了片刻,方源等人视野的前方,忽然出现了一层薄薄的迷雾。

%66
仙蛊屋雏形速度微微一缓,旋即不再犹豫,直接射进迷雾之中。

%67
迷雾十分不凡,能颠倒方向,令情意迷乱,但方源分身依靠春秋蝉,等若掌握着最关键的钥匙,时刻指明方向。

%68
迷雾重重,河水在迷雾的覆盖下,也变得缓和起来,并无外面那边汹涌澎湃。

%69
隐约中,一座石头岛的模糊轮廓,出现在方源等人的视野当中。

%70
“石莲岛!红莲真传!!”影无邪瞪大双眼,惊呼出声。

%71
白凝冰等人亦不免呼吸微促起来。

%72
当仙蛊屋雏形到达石莲岛的边缘,忽然石岛绽射出白炽之光,狠狠射中仙蛊屋雏形。

%73
“怎么回事?!”妙音仙子尖叫起来。

%74
“不好!这是一处陷阱。”方源眼中厉芒闪烁,但此刻想走,已是晚了。

%75
整个仙蛊屋雏形剧烈的震荡,艰难地抵挡外来的强猛攻势。

%76
“魔头,老夫在此等候你多时了!”一位八转蛊仙,身躯威武雄壮,陡然出现在石莲岛上。

%77
下一刻,他酝酿已久的杀招,正中仙蛊屋雏形。

%78
轰!

%79
仙蛊屋雏形在瞬间崩溃。

\end{this_body}

