\newsection{八耳猴}    %第六百零五节:八耳猴

\begin{this_body}

%1
一进入五界山脉,南疆群仙顿时感到强烈的排斥或者吸摄之力。

%2
他们催动着杀招前行,又支撑着防御手段,自家仙窍无不震动,受到仙道杀招的反噬。

%3
“这魔头倒真是选了一个好地方!”当即就有南疆蛊仙咬牙切齿,恨恨地道。

%4
“就算是我这仙窍全毁了,也要将这魔头杀掉,为天下除害!”更多的蛊仙反而战意暴涨,杀方源之心更加坚定。

%5
方源如此狡诈、凶残,对天庭都毫无敬畏之心,胆大妄为,恣意嚣张,更何况其他四域的正道?

%6
这种人物才七转修为,就埋伏南疆正道,俘虏人质,大肆勒索,若是让他成为八转,那还了得?

%7
方源若是正道,那还好说。

%8
正道嘛,都按照游戏规矩来。但他是魔道,孤家寡人一个,修行起来需要海量的修行资源,就会向正道伸手,去烧杀劫掠。这是无法调和的矛盾!

%9
而方源不仅有着烧杀劫掠的能力,更能防备推算,战力凶猛,这就太吓人了。

%10
必须要及早地铲除掉这个祸害!

%11
“杀!杀进去!!为我族人报仇雪恨呐。”

%12
“谁能斩掉方源项上人头,不管是何人,我商无界必有重赏!”

%13
“我池曲由会亲自出手,为他建造仙阵!”

%14
“我武家开放库藏,可任由斩杀方源之人,挑选三只仙蛊!!”

%15
重赏之下,南疆蛊仙士气狂飙。为防止方源逃窜,他们分出好几路人马,分别从五界山脉各个方位,直闯进去,直捣中央黄龙。

%16
咯咯咯!

%17
一连串的鸡鸣,太古年鸡庞大的身躯忽然出现,横亘在商无界一行人的面前。

%18
方源自然有所布置,太古年兽已经分散各处,守护自己,抵挡强敌。

%19
这是太古年兽,威势凶猛,向南疆蛊仙啄来。

%20
随行的七转蛊仙,俱都心中一凛,正要拼力抵挡,忽然耳畔传来商无界的一声冷哼。

%21
“小小鸡仔,也妄图拦我!”商无界咆哮一声,八转杀招悍然催动。

%22
轰!

%23
一声巨响,整个太古年鸡竟直接被拍到山体中去,竟当场昏死过去。

%24
随行的七转蛊仙见此,顿时都倒吸一口凉气,再次看向商无界的目光已经改变。

%25
“商无界虽是八转蛊仙,但从未以战力闻名天下,没想到他的攻伐杀招居然如此凶悍!”

%26
“一头太古年鸡,到底是太古荒兽,居然连出手的机会都没有,就被商无界给打昏了。”

%27
“凶猛,太凶猛了!”

%28
“商家可是南疆中排列前五的超级势力,但商无界展现出来的实力,也太猛了些啊。”

%29
商无界的表现,让人惊骇,超出想象。太古年鸡再不济,也是八转层次,没想到居然不是他的一合之将。

%30
“哦?是将弱鸡蛊带在身上了么……”方源一边渡劫,一边洞察着战况。

%31
太古年鸡大失水准的表现,让方源很快就回忆出五百年前世的一场著名的交易。巴家七转蛊仙巴鸡,不惜耗费巨资和商家交易,得到八转弱鸡蛊。因为特殊手段,竟能越阶使用这只仙蛊,从而使得自己实力飞速暴涨,成为交易的大赢家,继而在五域乱战中搅动了好大一阵风云。

%32
“现在蛊仙巴鸡还在闭关当中,声名不显,无人问津。弱鸡蛊仍旧在商家手中,这一次被商无界拿出来对付我。”

%33
商无界的战力,在南疆八转当中,并不出众,但有了弱鸡蛊,却是大出风头了一次。

%34
好在那只太古年鸡只是被打败,并未被击杀,还有拯救的希望。

%35
方源立即调遣一头太古年蛇,扑向商无界一行人。

%36
两相交锋,商无界顿时受阻,一马当先的突进戛然而止。

%37
和商无界有着同样遭遇的,还有其他几路,俱都有八转蛊仙存在。显然方源重点防御的,就是这些八转存在。

%38
单对单,这些太古年兽当然不是八转蛊仙的对手,不过也能对抗一时。

%39
而在五界山脉当中,却是不一样的战斗环境。

%40
在这里动用任何的仙道杀招,难度都要上涨一个档次,并且蛊仙自己先要承受反噬。还有无处不在的引力、斥力,一直干扰着蛊仙。使得他们不仅要战斗,更要时刻分出心神,来看这种庞大的引力、斥力。

%41
这种引力、斥力,随着蛊仙修为越高,威能越大。八转蛊仙几乎是举步维艰,这也是为什么通常情况下,八转蛊仙都要通过黑天、白天,不辞辛苦地去绕路。

%42
当然了,五界山脉到底还是仿制五域界壁,并不是真的五域界壁,所以对八转蛊仙的阻挠并没有正版界壁那么强大。饶是如此,也极大地压制了八转蛊仙的发挥。

%43
一时间,就连武庸在内的这些八转蛊仙,全力对抗太古年兽,不能前进分毫。

%44
这些太古年兽,都是来自光阴长河,并非五域生长,所以都不受遏制。

%45
至于仙蛊屋玉清滴风小竹楼,本身就是固化的仙道杀招,在五界山脉中更加受到严重克制,被武庸收入仙窍之中,没有轻易拿出来用。

%46
“魔头算计真正狡诈奸猾,难怪他选择这里渡劫!”感受到战斗的艰难,有蛊仙吐血喊道。

%47
“别管我们,你们突入进去,干扰方源渡劫。万万不能让他渡过此劫!”池曲由大声呐喊。

%48
方源尽管布置了太古年兽,但论个数,可完全没有南疆蛊仙来得多。

%49
太古年兽纠缠住南疆群仙的时候,它同时也被纠缠限制。许多南疆蛊仙绕过太古年兽,继续杀向方源。

%50
“大阵起!”方源临危不乱,一声令下,周围忽然升腾起一座仙道大阵。

%51
一时间,南疆蛊仙五感顿失,一片混乱,眼前迷雾重重,失去了前进的方向。

%52
“这魔头果然还有后手!”

%53
“不用怕,五界山脉中不能轻易动用仙蛊屋,仙阵也同样如此。催动时间一久,就会自行崩散了,到那时,方源还要承受大阵溃败的反噬!”

%54
“不,这正是方源魔头的拖延之计。只要他撑过这段时间,成功渡劫,就算大阵被破,他受了反噬。他也能利用定仙游,及时撤走了。”

%55
南疆蛊仙们闪电般交流,立即明白了方源的拖延战术。

%56
“我们必须尽快地突破此阵!”

%57
“那位同道有这样的手段?”

%58
“不妙,池家蛊仙具被挡住,没有前来。”

%59
“可恨,这魔头是有意如此的。”

%60
“哈哈哈。”忽然间,却有一位蛊仙大笑。

%61
众人尽皆侧目,见是一位头大身小,双眼狡黠之人。当即,有人便问道:“侯家仙友何故发笑?”

%62
侯家蛊仙猴子耍收住笑声:“诸位勿忧,我观此阵并无空间局限威能,只是单纯地混乱我们的感应,让我们错失正确方向。我有一兽,正能派上用场。出来罢!”

%63
话音刚落,就见一道奇光从猴子耍的仙窍中飞射出来,落到他的肩头,化为一个猴子。

%64
这猴子体型小巧,毛发浓密,银光灿烂,有八只耳朵,左边四只,右边四只。

%65
当即便有蛊仙惊呼起来:“莫非这便是太古荒兽八耳猴?!”

%66
猴子耍哈哈大笑:“正是此兽也!”

%67
耳猴乃是奇异珍兽,隶属信道,就算是普通的凡兽也数量稀少,仙级耳猴就更加罕见。

%68
耳猴当生长出六只耳朵时,便是荒兽。七只耳朵,便是上古荒兽。八只耳朵,则是太古荒兽!

%69
一时间,南疆群仙看着猴子耍的目光都起了变化。

%70
猴子耍表面上只是七转修为,战力并不出众,声名不显,没想到竟然雪藏了一头太古荒兽作为帮手!

%71
这隐藏得也太深了,若非是此次围剿方源,恐怕还不能逼他露出这样的底牌。

%72
“猴儿猴儿,敌人位处何方?”猴子耍摸了摸八耳猴的小脑袋瓜儿,轻声下达指令。

%73
八耳猴身为太古荒兽,居然非常温驯,乖乖听猴子耍的话。它立即侧耳倾听了一下,然后吱吱一叫,伸出手来指明方向。

%74
众仙得到它的指点,立即继续突进。

%75
忽然,仙阵骤变,八耳猴立即大叫一声,又侧耳倾听一番,指出另外方向。

%76
方源顿时皱起眉头,如此一来,此阵哪怕是接连变化,也无法挡下这批南疆蛊仙了。

%77
“居然是八耳猴!虽然南疆侯家向来以豢养、奴役猴兽著称,但没想到居然有着这样的稀罕神兽。”君神光见到这一幕,心中杀意顿时弥漫。

%78
此次围攻方源,事关重大,君神光稍稍犹豫了一下,就决定冒险。早在南疆蛊仙突入五界山脉时,他就率先催动杀招,隐形匿迹,悄悄潜入。

%79
而另一位天庭蛊仙卫风,也在迅速赶来,可做黄雀。

%80
“有这种八耳猴,便可以监天视地,将来我中洲攻略南疆,绝大多数的潜行之法统统失效。将其除去,对我天庭大计大为有利!”

%81
君神光想到这里,注视着八耳猴的目光中透露出一丝杀机。

%82
八耳猴顿时有所感应,忽然惊叫一声,恐慌起来。

%83
“这畜牲好生敏锐!”君神光连忙抽回目光,又投向前方,“罢了,先放过它。当务之急,还是以方源为首要目标。”

%84
“莫怕,莫怕!”猴子耍连忙安抚肩头的八耳猴。

%85
正在这时,忽然视野骤然清明,耳畔就有蛊仙大叫:“我们冲出来了!”

%86
呼啦啦。

%87
天地二气还在卷席,至尊仙窍大门洞开,不断鲸吞。

%88
“这就是方源魔头的仙窍?”

%89
“我们接近了!”

%90
强敌在前,南疆群仙反而愣住,面面相觑,不敢向前。

%91
方源赫赫凶威,让这些七转蛊仙都有些惴惴不安。

%92
“哼,南联虽立,但大局岂是一时之间就能改变的?乌合之众,到底还是乌合之众!”君神光冷笑。

%93
他身形隐匿着,继续突飞猛进,竟想要一步窜入方源的仙窍当中去。

%94
“一旦我进入方源的仙窍,他就彻底陷入被动,投鼠忌器。”

%95
“我根本不去和方源对战,只要四处游击破坏,方源就万万承受不了!”

%96
“更妙的是,方源此次浩劫还未成形,我若在旁干扰,他顾此失彼,真有可能被我斩杀。”

%97
“不过更保险的,还是我继续隐藏在他的仙窍中,等到方源被灾劫和南疆群仙力战一番,我再猛地出手。哈哈!真期待他发现自家仙窍里,忽然窜出一个大敌的神情。”

%98
君神光拥有着极其厉害的隐匿手段,这也是他最擅长的领域。

%99
当然,他也深知,此番冒险绝对也有着被方源识破、发现的可能。

%100
但如此绝世良机,就在眼前,他若不去把握,那就是贻误战机!相比较而言,方源识破的可能性,远比无法识破来得更小。

%101
世间哪有什么毫无风险的事情?更何况兵凶战危呢。

%102
君神光贵为八转蛊仙,但从未失去拼命的勇气。只不过什么时候该拼命冒险,什么时候不该,他心中有着一把明确的标尺。

%103
“现在,正是拼命奋战之时了!”君神光眼蕴奇光,狂催仙道杀招,承受着杀招反噬,混入到天地二气当中,就要进入至尊仙窍。

%104
南疆群仙各自催动着侦查杀招,愣是没有发现君神光的丝毫踪迹。

\end{this_body}

