\newsection{琅琊束手,九歌纵横}    %第五百二十九节:琅琊束手,九歌纵横

\begin{this_body}

“来吧,定真枝丫变!”面对大同风,琅琊地灵低吼一声,施展出一记仙道杀招来。

此招一出,整个银色巨人发生翻天覆地的变化。

原本庞巨的人形身躯,陡然间变化成一株参天大树,树干银白,枝干万千,无数叶片宛若白银雕琢,璀璨无比,相互碰撞,发出叮叮脆响。

树枝树叶疯长,迎向大同风。

大同风正向四周肆虐铺散,遭遇到巨树阻截,居然扩张势态骤然减半!

“这?!”

“大同风竟然真的被阻挡住了!”

“难以置信!琅琊福地的底蕴,果然深不可测啊!”

雪民、墨人、石人三族蛊仙,看得此景,惊喜至极。

“嗯?这是……定真树?”一道红白相间的长虹,从大同风心飞射出来,正是凤九歌。

他虽然开创大风歌,能产生大同风。但此风一出,凤九歌就得立即脱离风心,否则等到大同风蔓延,填满风心中空地带,凤九歌也要阵亡。

凤九歌只能发出大同风,而不能操纵它。他也不是没有尝试过,但任何的操纵手段,落到大同风上,都会被风同化,令风壮大。

凤九歌有仙道杀招一曲之士,能够将天地歌、离歌等等,凝练成分身作战。但大风歌却无法形成一曲之士。

凤九歌远远脱离战场,琅琊地灵等人都在集中对付大同风,完全没有精力理睬他。

“真的是定真树啊,想不到,他们居然真的有克制大同风的手段。”凤九歌远远观望,眼中闪烁精芒。

这定真树乃是太古传奇荒植,历史上赫赫有名。它从太古时代,人祖出世前,就生存在天地之间,之后历经远古、上古时代,陨落在中古时代,死于元莲仙尊之手。

此树身上,有诸多野生仙蛊,其中八转仙蛊就有数只。元莲仙尊还只是八转修为时,在东海游历,海啸水患,为祸世间,元莲仙尊单凭手中仙蛊,无法平复这滔天水患,便向东海深处的定真树借取风平、浪静两大八转仙蛊。

但这株太古传奇荒植,自恃身份和实力,不想借给元莲仙尊,提出苛刻条件。元莲大怒,与其大战,杀死定真树,尽取其蛊,平定了水患,拯救了万千凡民,活人无数。

元莲虽然杀死了定真树,但对此树赞不绝口。后来成尊,偶尔感叹:当时自己除去此树,实在有些可惜。此树乃天地独一份,有着克制大同风的玄妙天赋!

方源乃是重生之人,早就知晓凤九歌手中的大风歌。此招惊世绝伦,在五域乱战时期,不知斩杀多少蛊仙。因此,也有无数蛊仙想方设法,要寻求此招破绽。

这世间没有无解的招数,只有无解的人。

大风歌虽强,但招是死的,总会有地方能被针对。

西漠八转蛊仙千变老祖,寻根究底,得到另外一位宙道八转大能相助,从光阴长河之中,见证了定真树的威能。归来之后,推演出了变化道杀招定真变!

动用此招,便能克制大同风。

其时,凤九歌只是七转修为,却要引出八转蛊仙,专门研究对付,可见大风歌的强大之处。

方源记得此事,他未雨绸缪,想到前世凤九歌攻击琅琊福地,为了对付凤九歌手中最强的大风歌,便依靠智慧光晕还有变化道境界,推衍出了定真枝丫变。

此招只是定真变的简略版本,但凤九歌此时掌握的大风歌,也远远不如五域乱战时那般恢弘和完善。

因此,当琅琊地灵催动此招后,使得整个银色巨人变化成定真树的枝丫,果然逐渐地挡下了大同风。

“你们去对付凤九歌,不要让他恣意妄为。待我处理了这片大同风,就去帮助你们,斩杀了他!”琅琊地灵大吼,指挥其他三大异族的蛊仙。

异人蛊仙立即响应,操纵超级仙阵,去为难凤九歌。

琅琊地灵满头冷汗,竭尽心神,维持杀招,对付大同风。

他心中暗暗叫苦:“方源教了我这杀招,可惜我平时嫌这杀招太过繁琐,没有多加练习。没有想到,居然真的会被凤九歌攻打,这么快我就真的要面对大同风!唉,早知如此,我就要更加苦练这记手段了。”

琅琊一方完全不被大同风牵制,苦苦支撑。无数枝丫接触到大同风,被大同风摧毁,但与此同时,又有全新的无数枝丫生长出来。

这一招能抵御大同风,催动的代价也是巨大,非常消耗仙元,还有蛊仙的心神精力。

很快,银色巨树体内的许多毛民蛊仙,就有了萎靡不振,精神极端匮乏的样子。

“坚持,坚持住!”琅琊地灵大声鼓劲。

“大人,为什么我们不能将这块福地舍弃掉呢?”有毛民蛊仙出声建议。

琅琊地灵源自琅琊福地,自然有本事,可以将这块福地割舍掉,如此一来,里面的大同风也就不能威胁到琅琊福地了。

和整个琅琊福地比较起来,舍弃的那一块真的不算什么,九牛一毛都称不上。

“不行!”琅琊地灵断然拒绝,他想到方源早前郑重其事的叮嘱,解释道,“一旦这样做,就会落入敌人算计。割舍福地,就要勾通外界,在那一刻和开启门户无异。如此一来,我们琅琊福地的位置就会被敌人发现了!万万不行!!”

“咦?这琅琊地灵居然甘愿耗费巨大代价,对付大同风,也不愿割舍这一小块福地?”凤九歌目光惊异。

若是琅琊地灵这样做,必然会泄露关键线索,让徘徊在外界附近的凤仙太子立即感知到。如此一来,战局就彻底决定下来。

有着八转蛊仙凤仙太子出手,这些琅琊毛民蛊仙不会有什么阻挡力量。

但琅琊地灵居然咬牙死撑,也不愿这样去做,这就让凤九歌的算计落了空。

咔嚓。

一道憾世惊雷,凭空而生,打向凤九歌头顶。

凤九歌身形如烟,轻轻一晃,消失在原地,成功闪避开来。

他仰望天空,脑海中念头迅速闪过:“天婆梭罗被大同风牵制,剩下的就是这座超级仙阵了。对付它,可比对付天婆梭罗容易得多。”

仙道杀招离歌!

凤九歌催动离歌,果然超级仙阵就开始分离崩溃了。

离歌虽然没有任何的杀伤力量,但是绝妙的威能,能够让任何蛊阵拆分分离。

不过好在,方源建设此阵时,也有充分的思谋,将这仙阵布置成了多重。

凤九歌催动离歌,只是拆分了这座超级仙阵的最外层。三大异族的蛊仙们一直在最中心,分毫无损。

凤九歌只好再催离歌,很快,又拆分了最外的第二层。

但仙阵就露出了第三层。

“哦?我倒要看看这仙阵还有几层!”凤九歌眼中精芒暴涨,连续催动离歌,这一次他直接又拆了五六层,终于停手了。

即便强如凤九歌,心中也几乎要吐血。

“这仙阵究竟有多少层!?”

离歌消耗仙元极其庞大,这几番离歌下来,凤九歌的仙元储备已经不足战前三成。

当然,消耗最大的还是那一记大风歌。

“用大风歌的话,能顶得上离歌无数,但此招风险太大了……”凤九歌微微摇头,打消了再次催动这记杀招的念头。

此招威力虽大,但太过凶险,凤九歌向来都很少使用。

当初,他和凤仙太子切磋,虽然也是催出大风歌,但那时是切磋。凤仙太子之前书信灵缘斋,不满他和凤九歌并列的双凤说法,带给凤九歌非常巨大的政治压力。凤九歌去往北原,便主动和凤仙太子提出一招之约。那一次比试,凤九歌拥有足够的时间,慢慢来,一步步催动蛊虫,外界没有一丝干扰,风险压得很低。

但这个时候,凤九歌是在战斗当中,外界干扰极大,并且催动杀招时,必须分秒必争。刚刚那一次,凤九歌是逼不得已,才会冒险。但现在他面对超级仙阵,所受压力,可小得多了。

凤九歌没有再用大风歌,他目光扫射,银色巨树还在对付大同风,此时的大同风已经缩减了一半还多。

再过一段时间,琅琊一方就能平定大同风,腾出手来对付凤九歌。

“时间不多了。”凤九歌心中一叹。

天婆梭罗也是攻防一体,连大同风都奈何不得,凤九歌现在去夹攻银色巨树,不会有什么良效。

此次进攻琅琊福地,琅琊一方展现出来的实力,大大超出天庭意料,凤九歌心中已有退意。

“不过在撤退之前,我还有事情去做。”

他微微一笑,催动仙道侦查杀招,身形如光,一闪即逝。转瞬间,跨越非凡距离,落到古月方正的面前。

“就是你了。”古月方正还未反应过来,就被凤九歌镇压,收入仙窍。

“还有这个地方!”凤九歌飞射长空,落到云盖大陆某处,见到荡魂山。

“收!”他催动早已提前准备的仙道杀招,竟将这座天地秘境也收入仙窍当中。

“这些云城,虽然都是凡蛊屋,但里面生活了不少毛民精锐,都是琅琊派的蛊仙种子。不错,不错,都收了!”接下来,凤九歌又接连动手,将云盖大陆上的许多云城,都收入囊中。

他做出这些动作,都遭受了超级仙阵的疯狂阻击,可惜此阵已经被凤九歌分离了许多层,威能大减。

至于天婆梭罗,被大同风牢牢牵制,一旦脱离,分秒中大同风就能膨胀数倍,让琅琊地灵之前的苦功尽数化为泡影,因此根本不能赶过来阻止。

“这是什么气息?!”纵横之际,凤九歌查探到了智慧蛊的气息,震惊不已。

他很快就来到智慧蛊面前,凤九歌惊得瞳眸都微缩起来:“非是亲眼所见,我绝不会信。这是九转仙蛊智慧?!”(未完待续。)

\end{this_body}

