\newsection{天庭震撼五域}    %第四百一十九节:天庭震撼五域

\begin{this_body}

中洲,天庭。

汉白玉质地的宫殿巍峨林立,银白色的天空,始终光明堂皇。

在这宫殿群落当中,有一座漆黑的大殿,分外显眼。

殿前的牌匾上,书写着三个大字――镇魂殿!

从宫殿本身外溢而出的气息,彰显着这座宫殿的不凡,毫无疑问,这是一座八转仙蛊屋!

在镇魂殿的中央,矗立着五根巨大的庭柱。

赤红的庭柱之间,用条条紫金锁链串联,结成一片巨网。

而在这巨网中央,束缚着一团残魂。

正是幽魂本体!

紫薇仙子面无表情地站在五根巨柱的边缘:“幽魂魔尊,你生前是九转蛊尊,不甘灭亡,逆天而行。你今日落到如此下场,完全是你亲手酿造的苦果。你知错吗?”

她的声音非常清亮,在偌大的宫殿中回荡。

反观那团残魂,却始终沉默,不做声响。

紫薇仙子又道:“昔日你屠戮天下,祸害苍生,满足一己私欲,铸就你个人的辉煌。如今你成为阶下囚,还不悔过自新,认真忏悔,争取为天下苍生谋其福祉,为自身赎罪吗?”

残魂沉默。

紫薇仙子叹息一声,催动仙蛊屋镇魂殿。

顿时,磅礴的蛊虫气息,澎湃而起,宛若浪潮翻滚。五大巨柱散发出通红的光和热,宛若熔岩铸就。

无数根紫金锁链编织成的巨网,亦流转起绚烂的光彩,引人入胜。

幽魂本体则开始颤抖,并且随着时间的推移,他颤抖的幅度越来越大。

当颤抖的幅度,达到某个程度之后,一股股的信息便被搜刮出来,顺着无数根紫金锁链,传达到镇魂殿中。

于是,巨柱的表面,浮现出一幕幕的影像,宛若人皮画影,演绎着幽魂生前的某些幕场景。

其中有他修行的景象,有他战斗的英姿,有和人对话,林林种种,不一而足。

整个搜魂的过程,持续了一炷香的时间,这才缓缓停息下来。

紫薇仙子却很不满意。搜刮出来的情报,都是流于肤浅,真正宝贵的太少。

下一刻,她目光沉凝,冷笑出声道:“幽魂呐,自古成王败寇,你这又是何必?你已经毫无希望,你还指望有谁能够攻入天庭,救你出来吗?呵呵。你如此负隅顽抗,又能如何?你的结局已定,无法改变了。”

幽魂仍旧不语,就像是一块顽铁,一个石头。

紫薇仙子冷哼一声,转身而走。

轰!

一声巨响,镇魂殿的大门在她身后重重关闭。

望着空荡荡的天庭,悄无声息的重重宫殿,紫薇仙子的眉头微微皱起。

这些天来,她心中的不妙之感,越发浓郁。

而凤九歌那边,始终未等候到方源等人出现。

不仅如此,天庭收集到的西漠情报中,也没有出现过有人强闯光阴支流的消息。

一切都很平静。

但紫薇仙子却从这种平静中,感受到了不妥。

“可惜幽魂落到如此地步,还在负隅顽抗!搜刮的情报,对我推算方源一行人的行踪,根本毫无帮助。”

“到底是曾经的九转尊者,魂道的巅峰传奇。就算是一缕残魂,竟然能抵抗着镇魂殿的搜刮。真是厉害!易位而处,换做是我只剩下一缕残魂,别说是仙蛊屋,就是一只六转仙蛊,也全然没有抵挡之力的。”

“也罢了。就算是幽魂始终不合作,镇魂殿仍旧是能搜魂的,只是进展颇小而已。但假以时日,终究会有一天,将幽魂身上的情报都掏空。”

“至于现在……”

紫薇仙子仰望天空,嘴角微微翘起,露出一抹极美的微笑。

“是时候让我天庭的威名,在全天下传颂了。”

在梦境大战结束的数月之后,渐渐沉寂下去的天庭,猛地爆出一个惊天动地的骇人消息!

曾经的幽魂魔尊,如今的残魂本体,被天庭捉拿,逆天的阴谋被天庭蛊仙破坏。

这个消息一经传出,立即在整个蛊仙界中掀起轩然大波。

无数蛊仙为之震撼。

九转尊者盖压一个时代,是无敌的存在。

没有想到,天庭居然擒拿俘虏了一位九转尊者。而且还是所有尊者当中,杀性最重,屠戮苍生的幽魂魔尊!

尽管这只是残魂,但也足以让人震惊。

当然,质疑消息真假的声音也有,但很快就消失。

天庭方面自然做了充分的准备,列出种种证据,宣布僵盟、影宗以及之前的义天山大战、梦境大战等。

铁证如山,令人信服。

一时间,整个天下都将注意力,集中在了中洲天庭身上。

天庭过往的辉煌,被无数蛊仙翻出。

它是蛊仙界的第一势力,从始至终,都是最强!

它最早是由元始仙尊所创,而后历经数代仙尊,始终屹立不倒。

即便是三大魔尊攻过天庭,但最后都不了了之。

它神秘,它强大,它高高在上,它深不可测!

它就是天庭!!

……

北原,长生天。

南荒仙人躺在病榻之上,苍老的面庞上残留着最后的生气。

“唉,想不到堂堂幽魂魔尊,死后晚节不保,竟是栽在天庭手中。”南荒仙人深深叹息。

病榻之前,药皇轻声安慰道:“南荒大人,您还是闭眼养神,多休息一会儿吧。”

南荒仙人淡笑一声:“我寿元无多,只剩下数天,无妨了。三秋啊,记住我叮嘱过你的事情,我死后,你要接任南荒之位。”

“是,大人。”

南荒仙人脸上涌起一层忧色:“时代更迭,整个五域到了一个史无前例的阶段。我已经感受到时代的汹涌浪潮,那不是寻常的浪,超越以往的尊者时代,是前所未有的海啸。即便是我长生天,若是行将踏错,也会在这个时代的海啸中覆灭。天庭如此强势,必是我方的最大敌人,你切要小心。”

“大人,您的话,晚辈定牢记在心。”

……

南疆,尸皇芋顶天。

这是一座非常特殊的山峰。在古代,一位南疆八转蛊仙转变成仙僵,在这里受到超级势力的联手围攻,最终战死。天长日久,八转仙僵破碎的尸体,在江边磅礴的灵气滋养下,再配合江水冲刷上岸的水草,渐渐生根发芽,最终长成了一座尸山。

尸皇芋顶天高达数百丈,紧靠着赤黄江心漩涡,元气充沛磅礴,资源产出巨大,本身更是战略要冲之地。

此刻,武庸站在尸皇芋顶天的峰巅,眺望着恢弘旋转的赤黄两江的交汇处。

在他身后,屹立着武家蛊仙武镇。

武镇望着武庸的背影,毫不掩饰心中的崇敬之情。

这座尸皇芋顶天本来是武家之物,但在不久前,被姚家夺走。梦境大战之后,武庸携玉清滴风小竹楼,展现出超强战力,震慑南疆正道群雄。并且,在随后的交涉中,成功地从天庭处得回之前南疆正道丢失的仙蛊,武庸将其大多分散还给正道家族。

由此,武庸声威大振,武家虽是损失惨重,但却一扫之前颓势,坐稳南疆第一的宝座。

一手大棒,一手糖果,武庸无往不利,在随后的日子里,成功地收服失地,让诸多正道势力纷纷退却,几乎是兵不血刃。

如今,在武家当中,已有人将武庸与武独秀媲美,武家士气高昂,认定武庸贤明,必定能带领武家延续辉煌!

“武镇啊,这处尸皇芋顶天乃是战略要地。武家上下,唯有你最令我放心,我便将这里交予你了。”武庸头也不回,望着远方漩涡,开口道。

“是,在下一定不负太上大长老所托。”武镇语气隐含颤抖。

“大人……”他顿了顿,继续道,“尸皇芋顶天乃是我武家收复的最后一块失地,接下来,我们是否反攻过去呢?”

武庸皱了皱眉头。

反攻的计划,他一直都在暗中谋算。

但是在接到了天庭擒获幽魂的情报之后,武庸却是决定选择放弃。

“此次武家上下,折损了不少家族蛊仙。失地既然都已经收复,领地广阔,原本人手就十分勉强,如今更是不足。接下来便是休养生息,多多栽培武家后代,成就蛊仙之位。”

“是,太上大长老高见,在下谨记。”武镇恭敬地道。

现在武庸声威极高,他的决断,武家蛊仙都会支持。

“不过,之前冒犯我武家的事情,也不能这么算了。虽然不会扩张领土,但是赔偿资源,必定是要有的。”武庸继续道,声音里藏着一丝战意。

武镇顿时神色振奋起来:“太上大长老英明!”

……

西漠,唐家大本营。

“天庭雄威,叫人心惊啊。”唐家太上二长老忧心忡忡。

唐家太上大长老,淡淡地瞥了他一眼,心知肚明对方为何而来。

之前和方源等人合作,太上二长老就持反对的态度。如今天庭擒获幽魂的消息一公开,唐家蛊仙都饱受巨大压力。

“我们和方源等人合作的事情,会不会因此遭受天庭的追责?”太上二长老唉声叹气。

“追责了如何?不追责又如何?”太上大长老笑着道。

太上二长老皱着眉头,不断踱步:“不追责,那当然皆大欢喜,我们可以和影宗继续合作。但若追责下来,我们唐家到底是西漠蛊仙界的一份超级势力,可以联合其他家族共同对抗天庭。当然,必须严守秘密,不能流出一丝证据。否则,天庭占据大义,逼问我等,我族就无法借到援兵了。”

“还有,实在不行,就只有弃车保帅,舍弃唐方明、唐烂柯二人了。”

太上大长老笑道:“哈哈,你看,你已是知道应对举措,还慌张什么?”

“呃。”

太上大长老收敛笑意,正色道:“我唐家若要崛起,非得如此冒险不可。富贵险中求,盗天魔尊的梦境啊,我们若不把握住,岂不愧对了这份仙缘?至于唐方明、唐烂柯,亦早有觉悟。为家族牺牲,也是你我正道的最好归宿。”

二长老长长叹息一声:“大长老明见!”(未完待续。)

\end{this_body}

