\newsection{家大业大}    %第二百七十一节:家大业大

\begin{this_body}

%1
躲避推算的时候,方源也不闲着,趁机视察自家的仙窍。

%2
五域九天。

%3
格局出众,已经超出常理。

%4
原本五亿亩的广袤空间,就已经大到了惊世骇俗的地步,说出去基本上没有人会相信。

%5
而现在,方源的至尊仙窍,则已经超过了五亿亩的范畴。

%6
这全然是因为,在前一段时间里,他大肆屠戮蛊仙,吞并他人的仙窍福地。

%7
因为至尊仙体的特性,导致方源可以完全吞并他人的仙窍福地,吸收道痕,无一浪费。

%8
这些仙窍福地的增加,使得至尊仙窍中,多了许多片特殊的领地。

%9
这些领地,大多数都有地灵调控管理,省去了方源大量的精力。它们就像是一颗颗的珍珠,洒落在五域九天当中。

%10
“宇道方面的资源,极其庞巨。寻常蛊仙都千方百计想要扩张自家仙窍,我却有这么多的空间,还来不及开发。”

%11
方源对这方面,相当满意,但对于宙道资源,却是另外一番感观了。

%12
至尊仙窍中的时间流速,已经下降了很多倍。

%13
原本是一比六十,宙道资源丰富到没朋友!

%14
但现在不行了,甚至连寻常的标准都达不到。

%15
这也是没有办法的事情。

%16
方源改变了修行方法,为了延缓灾劫来临,做出了如此抉择。

%17
他利用从黑凡真传中得到的宙道手段,延缓仙窍时间,这样宙道资源一少,导致了方源的仙元产量也跟着大大降低,不仅如此,还有其余的资源收益,也下降幅度巨大。

%18
仔细考量。

%19
小中洲中,有血芝林、镜柳林各一片,这两者源自狐仙福地,来源悠久。除此之外,还增添了一处璇光小坑。

%20
这个璇光小坑中,生存着各种光道凡蛊,数量众多。

%21
这是方源参照了北原刘家的一处资源产地璇光坑,在自家仙窍中仿造出来的。

%22
不久前,方源洗劫了璇光坑,将里面的光道凡蛊都一网打尽,洗劫一空。

%23
其中一部分的光道蛊虫,被方源放进了九天极光之中。仍旧剩下大部分,一时间不好脱手。

%24
方源就只好打造出了一个璇光小坑,暂时存储这些海量的光道凡蛊。

%25
当然了,这个璇光小坑和真正的璇光坑不能相比。后者是真正的资源产地,蕴含丰富的光道道痕,任何的虫群都能在里面,被潜移默化,晋升成蛊。但前者只是方源草创,专门存储光道凡蛊用的,一点产量都没有。

%26
小中洲是至尊五域中开发程度最低的地方。

%27
在其之上,是小北原。

%28
小北原中,地貌平坦,大半面积都覆盖了冰雪,形成了雪原。

%29
但冰雪并非唯一的色彩。

%30
在靠近东海的一端,有一片绿水浅滩,一块块的石头,堆砌成特殊的水泊。里面水草丰盛,灵蛇栖息。

%31
这片绿水浅滩,乃是原先的变化道蛊仙韩东的福地。被方源吞并过来,就连福地的地灵粉红灵蛇,都随同而来,帮助方源管理这片地域。

%32
而在靠近西漠的一端,有一片骨葬场。

%33
这里堆积着山一般的骨头,各种类型,有肩胛骨、腿骨、头骨,涵盖飞禽走兽,包含很多仙材,价值巨大。

%34
这片骨葬场,是方源夺自刘家的资源产地骨葬场。

%35
他在击杀了刘家蛊仙刘勇之后,直接将这片骨葬场搬迁到了自己的仙窍里头。

%36
因此,他就有了这一片盛产骨道蛊虫的资源点。

%37
“不过和原来的骨葬场相比,我的这片骨葬场虽然有海量骨头,但骨道道痕终究比不上原版。在原来的骨葬场中,就连周围的土地,都蕴含骨道道痕。刘家完全可以在那片土地中,重建骨葬场。”

%38
方源心知肚明。

%39
当然了,骨葬场这块大蛋糕,方源已经拿走了八成,真正的大头都是这些骨头。

%40
余下两成,就是土地环境,刘家想要重建骨葬场不难,但想要恢复到过去的标准,那非得耗费巨大代价不可。

%41
而在靠近中洲的一端。

%42
是一片稀稀疏疏的草原,各种野草野花,主要的是血镰草、赤斧花。

%43
这些也是来自于狐仙福地。

%44
这片草原面积广大,不是韩东福地所能媲美的。里面放养了一些花粉兔、狐狸群、水狼群、地皮猪群、毒须狼群等,还有一头荒兽鱼翅狼、一头荒兽巨角羊,经常在这片草原上徘徊觅食。

%45
小西漠中,气温最高。

%46
因为大多数的炎道道痕凝聚在这一块。土地多呈现沙化。

%47
这里主要有三个资源点。

%48
第一个是豢养幽火龙蟒的地坑,数量众多,集中在一块。

%49
第二个是一片沙鸥土滩。

%50
这里盛产沙鸥土。

%51
沙鸥土是一种蛊材,更是仙蛊连运的食材。

%52
不过现在方源已无连运仙蛊了,所以沙鸥土已经作为一个贩卖的资源,帮助方源获取仙元石。

%53
沙鸥土滩的中央,自然是那颗天地沙鸥的死蛋。蛋壳上裂痕满布,同时还有数个小洞,不断地从裂缝间,从小洞内往外流出透明的蛋清。利用这颗蛋中的生命精华,就能将普通的沙土转化为沙鸥土。

%54
第三个则是腐黑沼泽。这里是方源渡劫之后,形成的地方,充斥暗道道痕。如今已经被方源改造,形成了一个初步铲除暗道蛊虫的资源点,但是因为才刚刚起步,产量极低。

%55
除了这三片资源点外,就是两片福地。

%56
一片是土道福地,烂泥沼泽的地貌,里面生活着五头荒兽泥怪,各种土道凡蛊,由地灵泥巴小人主持管理。

%57
另外一片是刘勇的奴道福地,里面生活着二十多头的犬类荒兽。这些荒兽,大部分是被方源从宝黄天中购买而来。因为当初刘勇福地的认主条件,就是二十头犬类荒兽。这当中,包含了刘勇本身的六头荒兽骨甲犬。整个福地,由一个外表是黑色卷毛犬的地灵掌管着。

%58
小东海的经营程度,和小西漠相差不多。

%59
武遗海的福地,就落在此处。

%60
这是一片汪洋,被武遗海经营有加,里面有上古荒兽角神龟三头,一大片的上古荒植静音珊瑚群,还有六头荒兽白信蓝羽鸥六头,由海龟地灵执掌。

%61
除此之外,还有原先的龙鱼群,荒兽龙鱼两头,气泡鱼群,一下群的珍稀散文鲤,还存备了一些油水。大量的青玉鲫鱼、血玉鲫鱼以及一头上古荒兽藏娇蚌(在乱流海域所获)。

%62
另外有一小片血湖,在这里掺杂了大量的荒兽血液、蛊仙血液,曾经方源就是将血本仙蛊,投入到这里修养。

%63
如今血本仙蛊状态全复,这片血湖却遗留了下来。

%64
整个小五域中,最富有的当属小南疆。

%65
五光山、继仙山、封天山等十多座山峦。

%66
石钟乳洞窟、长恨蛛群、少量石人,大量的原始丛林中,曲丽木随处可见,茶溪道道清澈见底,气死鸟在林中高歌。

%67
还有一座成龙丘,方源从北原直接用拔山仙蛊,搬到仙窍中的。

%68
而在小九天里。

%69
一小片的极光,存在于小橙天当中。极光中许多流光果载沉载浮,因为璇光坑中缴获了大量的光道蛊虫,导致极光和流光果的数量规模,都在原本的基础上,上涨了许多。

%70
而在小黄天中,是一小段的碎金河。

%71
小青天里,储备着相当多的丹青香,这是喂养换魂仙蛊的食材。还有一个天晶蓄养池,得自黑凡洞天,暂时无用。

%72
小黑天中有走肉树,小白天中,有大量的斑斓霸王花,还有上古天残犬三头,上古鹰犬一头,荒兽鹰犬七头。

%73
小紫天中,上极天鹰翱翔着,目前它还只是荒兽战力。八十多座鹰巢,静静地悬浮在这里。这些都是方源攻打铁鹰福地的战利品。至于原先的那座天晶鹰巢,早已经被上极天鹰吃干抹净,一点不剩。

%74
小蓝天中,则有大量的云土。云土上栽种着大量的箭竹林,还有一片片的星屑草场,落星犬幼体一头。之前的两头荒兽刺脊星龙鱼,则在黑凡洞天的防御战中,被方源甩卖掉了。

%75
还有一片陨石群坑,盛产星火蛊、流星天陨蛊。斩杀耶律群星,收获的群星福地,也被方源安插在这里,形成了一片悬浮的大陆,陆地上全是由星屑铺满,星核地灵掌管。

%76
还有一些福地,零散各处,一些其他的荒兽荒植等等,不做赘述。

%77
而荡魂山、落魄谷,留在琅琊福地当中。

%78
逆流河被方源暂时搁置在小东海里。

%79
蛊虫方面

%80
九转的智慧蛊,仍旧留在琅琊福地里。至尊仙胎蛊已被方源耗用,不记录此内。

%81
八转仙蛊有态度、慧剑、似水流年(封印状态中)。

%82
七转仙蛊有换魂、剑眉、浪剑、飞剑、剑遁、招灾(借给了楚度)、年蛊、以后、龙息、爱意、坚持。

%83
六转仙蛊有解谜、妇人心、血本、暗渡、狗屎运、变形、力气、我力、飞熊之力、拔山、挽澜、江山如故、人如故、星眸、春秋蝉(天意寄存,存放在仙僵肉身之中)、骨刺、道可道。

%84
其余凡蛊,种类就多了,不做过多赘述。

%85
黑凡的暗箭仙蛊,早已自爆,方源并没有缴获。

%86
俘虏有古月方正(留在琅琊福地),黑凡(肉身完好),马鸿运魂魄、刘勇魂魄、耶律群星魂魄。至于石奴之魂,方源并未获得。影宗在魂道方面的造诣,方源只能仰望,虽然他杀了石奴,却无法阻止他的魂魄自行消弭。

%87
至于其他的,诸如东方长凡魂魄、羽民蛊仙郑灵魂魄,东海群仙魂魄,都已经被方源甩卖给了琅琊地灵,换取了大量的门派贡献。如今恐怕都已被荡魂山消化,形成了胆识蛊。

%88
还有一名羽民蛊仙奴隶周中,早已被方源派遣去了西漠,实施另外一个计划去了。

%89
家大业大!

%90
这个词,完全可以评价方源现在的资本,甚至还稍显浅薄。

%91
从总体价值而言,他纵然只是七转修为,但完全能和普通八转蛊仙并驾齐驱了。

%92
单单几个天地秘境,就价值非凡。更何况那些蛊虫,真的是一大堆,数量极其惊人!

%93
想想蛊仙黑凡,八转蛊仙,才有多少只仙蛊?

%94
不过仙蛊太多,喂养是个大难题。

%95
方源至今,仍旧停留在这里,为了喂养仙蛊而做出努力。

%96
其余的六转、七转仙蛊,已经基本解决。难点在于态度蛊、慧剑蛊这两只八转仙蛊。

%97
新得手的坚持蛊,食料是逆流河水,方源完全可以满足。爱意蛊的食物,和爱情蛊一样,都是希望和恐惧,方源也好解决,从宝黄天中收购大量希望蛊、恐惧蛊即可。

%98
对于方源而言,经营仙窍已然不是当前重点。

%99
只要他不断屠戮蛊仙,吞并仙窍,仙窍的底蕴都会蹭蹭暴涨。

%100
他现在面临的问题是仙元不足。

%101
追杀影无邪等人,又经历了逆流河大战,其后又摆脱毛里球的追击,这一路上方源耗费的仙元非常巨量。

%102
导致他的红枣仙元储备,已经下降到了一个非常危险的境地。

%103
“原本有着宙道资源,时间流速很快,通过自身仙窍就能产出不少红枣仙元。”

%104
“但施展过宙道手段后,仙窍本身时间流速相当贴近外界五域,我要补充红枣仙元,只有通过交易买卖了。”

%105
逆流护身印尤其耗费蛊仙的心力和仙元。

%106
幸亏方源成为七转蛊仙,若是单凭之前的青提仙元,恐怕早已干涸。

%107
没有红枣仙元,方源便没有多少底气。

%108
趁着休整期间,他索性将手中的荒兽尸体,都放入宝黄天中贩卖。

%109
这些荒兽尸躯,都是仙材,数量还不少。乃是方源之前在地沟中追杀影无邪等人,铲除的那好几批的荒兽群。

%110
其中还有一群梦魇魔驹,这种梦道仙材若是放入宝黄天中,必定引发巨大的轰动,乃至蛊仙哄抢。

%111
不过方源见这些梦魇魔驹的尸身,都扣留在手中,他还要有更大的用途。

%112
在休整期间,陆续有几笔交易做成。

%113
刨除了给予宝黄天的手续费用之后,方源收获了不少仙元石。

%114
一块仙元石,相当于一颗青提仙元。

%115
方源是七转蛊仙,耗费百颗仙元石,才能凝练出一颗红枣仙元。

%116
就在他不断提升红枣仙元储备的时候,一份消息传达到了他这里。

%117
“是毛六的来信。”

%118
方源感到有些奇怪。

%119
浏览之后,更让他惊异:“影宗居然要与我和谈罢战,甚至展开合作?”

%120
ps:今天着重统计了一下方源的资产,统计得好累!大家看看,有没有什么遗漏的,可以告诉我。最近状态有点不佳,明天双更。

\end{this_body}

