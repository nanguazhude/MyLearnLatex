\newsection{重获肉身和春秋}    %第五十八节:重获肉身和春秋

\begin{this_body}

%1
一汪魂水,幽幽沉沉,表面荡漾着无数涟漪。

%2
影无邪睁大双眼,眼中鬼火燃烧,盯着魂水表面察看。

%3
在他的视野中,有外人难以观察到的景象。

%4
魂水中浮现出毛六的身影,他半跪在地上,满脸懊悔羞愧之色:“大人,我有大罪!在方源渡劫之时,若非我动用”

%5
“我说了,此事不怪你。说起来,也是我动用了底牌,增长运道,让方源占得了便宜。”影无邪的声音,直接传到蛊仙毛六的心底。

%6
这是毛六和影无邪相互沟通的手段,本是魂道力量,却有信道的效果,极其厉害。

%7
双方各自处在中洲、北原,间隔两个界壁,更有一个琅琊福地。不仅可以做到实时通讯,而且还能屏蔽琅琊地灵的感知。

%8
此时的影无邪,已经不再是方源面貌。

%9
他已经换了另外一个力道仙僵的躯壳。

%10
但这个躯壳因为年代久远,原本的仙窍已经散灭,影无邪得到的只是纯粹的肉身。不像之前的方源肉身,有一个空窍,还有一个仙窍。

%11
这样一来,影无邪用蛊就很不方便。不论仙蛊、凡蛊,都只能寄生在他的皮血肉骨骼等等之中,不仅会令气息泄露,让敌人容易侦查,而且很不安全。肉身遭到打击,蛊虫也会倒霉遭殃。

%12
蛊虫本身是很脆弱的。

%13
影无邪这一次和方源交易,可谓大亏血亏,不过此刻的他脸色却很平淡。

%14
他反而激励毛六:“你这一次任务,办得很好。整个交易过程,不停地向方源示弱,很不错。尤其是最后的昏厥若没有你的周旋和努力,我绝不会得到想要的仙材。”

%15
毛六叹息道:“我故意昏厥,试探方源,可惜方源没有向我动手。”

%16
“他既然知道我们的跟脚,自然会十分谨慎。再加上我们已经交易多次。双方各有把柄,他也忌惮我们向琅琊地灵告。”影无邪道。

%17
他此次身陷绝境,左思右想,现除了方源这条路。再无他法。

%18
决定和方源交易之前,他就做好了充分的心理准备。

%19
他交给方源的仙材清单,内容夸张,数目扩大,将真正想要的仙材隐藏在其中。

%20
总之。他的目的达到了。

%21
他从绝境中挣扎,求得了一线生机。

%22
只要渡过这个难关,接下来又是一番海阔天空!

%23
“别的仙蛊倒也罢了,关键是方源不仅重获他的肉身,而且还得到了春秋蝉!他现在的底蕴,未免太大了!”毛六还是很担忧。

%24
影无邪冷哼一声:“方源就算得他的肉身,他敢用吗?春秋蝉也不必顾虑,它本身已经被封印,都快要饿死了。方源就算解决了这个麻烦,春秋蝉本身就有失败的概率。内里还蕴藏天意,很长一段时间他都无法运用。更何况天庭蛊仙的注意力,也会被吸引到那里去。”

%25
毛六神色一震,眉宇间的忧愁少了许多。

%26
他连连点头:“大人说得是。”

%27
影无邪长叹一声:“接下来,你仍旧要好生监视方源。你已经暴露,但方源也不会动你。若无法遏制他,那就算了,也无所谓。他这一次得到那么多的仙蛊,必是极大负担。而且一次次的灾劫,更牵扯了他主要精力。就让他展展。等我汇聚了力量,救得本体,再找他好好算账!”

%28
吃一堑长一智,经历得多了。影无邪正在迅成长。

%29
他懂得了取舍,更学会了忍耐。

%30
他到底是魔尊幽魂的分魂之一,有成为强者的巨大资质。

%31
折磨、挫折都是人生的财富,他正在迅的成熟。

%32
北原,琅琊福地。

%33
距离和影无邪的交易,已经结束了一段时间。

%34
方源望着眼前的仙僵肉身。叹了一口气。

%35
“不敢用啊”方源摩挲着下巴,感到麻烦。

%36
虽然得到手后,他就动用了几乎现阶段所能用的所有手段,对这具仙僵肉身进行了大量、仔细的排查,没有现任何问题。

%37
但是影无邪是什么来历?影宗是什么来历?

%38
幽魂魔尊!

%39
他可是魂道的创始人,有数的蛊尊,曾经统治天地,制霸苍生的人物!

%40
他的分魂之一影无邪,会不会拥有什么魂道手段,将这具肉身布置成了一个险恶的陷阱呢?

%41
大有可能!

%42
方源和影无邪的交易,至始至终都不在信道的约束下。

%43
因为双方都不信任对方。

%44
而且方源的信道手段,肯定不如对方。动用信道手段约束自己,很可能对方却暗中解开,这就对方源更加不利!

%45
所以,完全不能保证这具肉身的安全。

%46
方源要这具肉身,主要目的就是得到智慧蛊的承认,引智慧光晕。所以他要将魂魄投入到这具仙僵躯壳里。

%47
这就很危险了。

%48
涉及到魂魄的事情,都是影宗最为拿手的。

%49
方源虽然没有查出一点问题,或许影无邪没有做手脚,但更有可能的是,这具仙僵肉身上隐藏着巨大的问题,但方源没有能力现。

%50
幽魂魔尊是魂道的祖师爷,方源在他面前玩魂道,简直就是班门弄斧。

%51
“看来这具肉身,还要再检查检查,不能草率。一不小心,就会着了影宗的算计。”方源心中告诫自己。

%52
只有到了十分确信安全的时候,方源才会运用这具肉身。

%53
或者到了迫不得已的情境,方源会选择冒险。

%54
“虽然不能利用智慧蛊,但重夺肉身,还有春秋蝉,已经算是上佳的成果了。”

%55
春秋蝉一直都是方源重生以来的最大底牌。

%56
现在方源要肉身,它作为本命蛊,一直固定在第一空窍中,自然也跟着来。

%57
之前谈判时,方源刚刚提出要索春秋蝉时,就遭到了毛六的坚决否认。

%58
他死不承认影无邪那边拥有春秋蝉。

%59
因为双方没有信道约束,他怎么说,方源也辨别不了真假。随意撒谎毫无恶果。

%60
但方源比他更蛮横,直接道:“不可能没有!如果真没有,那你们就炼。我反正不管其他的,必须还我春秋蝉!”

%61
方源摊开来玩,毫无底线,态度极其坚决。

%62
毛六就算是自残赌咒,声嘶力竭的誓,都没有让方源改变态度。

%63
费了好一番周折,方源才重得肉身和春秋蝉。

%64
不过得到的春秋蝉,也有问题。

%65
它被天庭蛊仙的诡异手段封印起来,已经隔绝了光阴长河,目前正处于相当饥饿的状态。

%66
而且方源还知道,春秋蝉的内部藏有天意,这可是最大的内患!

%67
仙僵肉身、春秋蝉,都有祸患潜伏着,不能利用。

%68
但即便如此,方源也很开心了。

%69
我就算用不了,只要控制在我的手中,那就是一场巨大的胜利!

%70
尤其是春秋蝉可是大杀器!

%71
要了春秋蝉和肉身之后,方源第二次就开口索要定仙游。

%72
结果毛六哪里能拿得出来?

%73
方源就算再强行逼迫也没有用。而且他多少也有点猜到,定仙游恐怕真的毁了。若是有定仙游,对方早就可以借助定仙游跑路了,何必主动找上门来,被方源勒索呢?

%74
当然是,之前方源和毛六已经做了一次交易。

%75
只要做了交易,就算是背叛琅琊派。

%76
毛六手上有了方源的把柄,而地灵单纯固执,若现此点,方源就算再怎么解释,恐怕也是无用。

%77
所以,方源最终放弃了定仙游。

%78
虽然没有得到定仙游,但方源在此次交易中还是收获一大把的仙蛊。

%79
一大把!

%80
其中就有原本属于黑楼兰的力气仙蛊、我力仙蛊、飞熊之力仙蛊。还有太白云生手中的江山如故、人如故。方源自己的拔山仙蛊、挽澜仙蛊、星眸仙蛊、招灾仙蛊。

%81
铁冠鹰力仙蛊、连运仙蛊都被影无邪,先后放入宝黄天贩卖了。这一次宝黄天关闭,这两只仙蛊也都陷入宝黄天中。

%82
而吃力仙蛊、净魂仙蛊、定仙游、星芽、星痕、星光都毁在方源的特意之下,影无邪没有来得及拯救。

%83
当然,以上情况或许不都是真实情况。

%84
方源和毛六交易,是没有信道约束的。双方的底线,都是各自试探出来的。也许影无邪还暗中扣下了一些仙蛊。但方源已经尽力,没有办法让影无邪再吐出来了。

%85
毕竟方源每一次交易,他就算是背叛了琅琊派一次,他的强势就要消散一些。而影宗方面,得到了一部分的仙材,弱势也会减轻一些。

%86
当然,至始至终,都是方源牢牢占据上风。谁叫影无邪境况不妙呢。

%87
但方源也不想影无邪这么快就死了。

%88
他若死了,只要留下一些线索给了天庭蛊仙,那方源就要紧接着遭殃。

%89
有他在前面吸引火力,是很好的。天庭和影宗之间,可以相互削弱。而方源隔岸观火。

%90
虽然方源清楚,自己夺了至尊仙胎蛊,影宗绝对会向自己下杀手。但这一点,也是影宗暗中维护自己的理由。

%91
马鸿运的例子,就在眼前。方源知道,自己就是影宗眼中的马鸿运,影宗定是想要重炼至尊仙胎蛊的。只是现在时局不允许他们这样做罢了。

%92
影无邪既然主动找方源来交易,这就说明他真的已经山穷水尽了。方源必须保留给他一个希望。

%93
若是真逼急了他,让他绝望之下破罐子破摔,方源也得跟着倒霉,甚至玩完。

%94
其实,毛六最开始的那番话,还是有作用的。

\end{this_body}

