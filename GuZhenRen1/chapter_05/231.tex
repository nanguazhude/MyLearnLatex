\newsection{逆命祭炼大阵}    %第二百三十二节:逆命祭炼大阵

\begin{this_body}

“这个大阵,叫做逆命祭炼大阵。乃是由孙名录所立,采取仙蛊十多只,更关键的是,灌输了逆流河的河水,因此威能极其惊人!”

赵怜云沉声解释道。

爱情仙蛊主动催,耗费了赵怜云虚窍内的大量仙元,效用千奇百怪,这一次居然能让赵怜云直接获知整个逆命祭炼大阵的跟脚来历。

“逆流河水?难道就是传说中的那条逆流河不成?”余艺冶子瞪大双眼。

“不错,正是人祖传中的逆流河。”赵怜云点点头。

《人祖传》中记载,荡魂山、落魄谷、逆流河,乃是生死门后的三关。当初人祖为了救活自己的大儿子太日阳莽,去往沉迷死境,结果失败。人祖只好自己先闯三关,重新复活。

结果,人祖先后闯过了荡魂山、落魄谷后,却在逆流河中功亏一篑。

“逆流河也是天地秘境之一,没想到居然在雪胡老祖的手中。”不真子感慨。

沐凌澜喟叹道:“据说当年,是盗天魔尊依靠鬼不觉,进入生死门中,将荡魂山、落魄谷偷了出来,惟独对逆流河无法下手。没想到,逆流河也不知何时被何人取了出来。”

沐凌澜乃是灵蝶谷的太上长老,这个中洲古派情报最为灵通,所以沐凌澜知道天地间的不少秘辛。

不过,中洲群仙很快就接受了这个事实。

既然荡魂山、落魄谷都能被取走,逆流河出世,也不奇怪了。

“现在的关键是,如何破开这个逆命祭炼大阵。”不真子强调道。

但赵怜云却摇头苦笑:“要破开这个大阵,几乎不可能。就算是八转蛊仙,也要陷入于此,不能自拔。组成这个级蛊阵的仙蛊,虽然没有一只八转,但是这个级蛊阵却是完美地运用了逆流河,让它挥无限的作用。”

赵怜云徐徐解释,其余四仙听闻之后,无不瞠目结舌,再也不却打什么破阵的主意。

原来这个级蛊阵,已经将大雪山福地彻底改造,以十五座雪峰为阵眼,形成固若金汤的防守。

蛊仙闯入蛊阵当中后,便会根据修为的不同,被迫传送到不同的雪峰之上。

若是七转蛊仙,会被送到七转蛊仙的雪峰上作战。

若是八转蛊仙,则被送到雪胡老祖的面前。

姬之前说的话,有真有假,并非是完全胡言乱语。

逆命祭炼大阵,的确能够增长阵眼中的己方蛊仙的战斗能力。让六转蛊仙有七转威能,七转蛊仙达不到八转程度,但也得到巨大提升。

而且这座逆命祭炼大阵,还仿造了悔池,有一些宙道上的玄机妙能。

真正歹毒的是,任何在蛊阵中阵亡的蛊仙,都视作将身家性命奉献给了这座大阵,帮助大阵最中心万寿娘子炼化马鸿运。

“难怪我们除掉姬,她的肉身和仙窍,都很快化为飞灰,旋即消散全无。”

“姬虽然逃出了魂魄,但想来此阵中,她恐怕已经魂飞魄散了。”

“没想到孙名录居然能建出这等绝蛊阵出来?此阵之强,十大古派中也难有比肩。”

“孙名录乃是北原公认的第一阵道宗师。传闻中,他性情刚强,极重情义。不和魔道同流合污。当年,雪胡老祖都舍下身段,三请四邀,他都不屈服于这等强势。最终还是看在雪胡老祖为了爱妻万寿娘子,才因此感怀,出手帮助万寿娘子创建了炼道蛊阵。现在看来,能创建出这等歹毒的炼道蛊阵,孙名录恐怕也不是什么好人。”

中洲蛊仙们议论纷纷。

“这么说来,只要我们出了这个雪峰,我们就会被迫分离,孤军奋战喽?”不真子又问。

赵怜云点点头:“应该是这样子的。我们来到这里,是因为爱情仙蛊的力量。”

爱情仙蛊,乃是九转仙蛊,威能凌驾于整座级蛊阵,所以一下子让五位中洲蛊仙,都来到了第九雪峰。

但现在,爱情仙蛊不挥作用,中洲五人一旦踏出这座雪峰的范围,就会被蛊阵带走,各自分开。

不真子紧紧皱起眉头。这是个级坏消息。

他是为了给赵怜云保驾护航,保护住赵怜云,也就是护住了爱情仙蛊。但现在这座级蛊阵会让各自分离,实在难办。

“这样吧,怜云仙子,你钻入我的仙窍。我不信这座级蛊阵还能影响到仙窍不成?”不真子犹豫了一下,下定了决心。

蛊仙的仙窍,乃是自家最为重大的秘密,一般都不会引进外人。

但不真子念及事关重大,决心做出牺牲,放赵怜云进来。

他们两人之间,也只有这样做。

因为赵怜云的并非真正的仙窍,而是虚窍。天庭的虚窍有优点也有弊端。弊端就是,只能装载仙元、蛊虫、意志,不能像正常的仙窍那般可以经营栽培。每隔一段时间,都要重新补充,否则不能长存。优点则是,可以装载任何品级的仙蛊,就像赵怜云现在,能够直接转再看爱情仙蛊,不虞这只九转仙蛊挤爆仙窍。

但结果尝试了一下之后,中洲蛊仙们却现,他们根本连仙窍的门户都打不开。

很显然,孙名录在创建这个级蛊阵的时候,也考虑到了这个问题。这并不是这座蛊阵的漏洞所在。

中洲蛊仙一时间陷入两难之间。

他们现,虽然铲除掉了姬,但己方已经被困在了这里。

不过有一个好消息就是,他们不担心有任何的援军,或者是八转雪胡老祖向他们出手。

因为这座级蛊阵当中,雪峰峰主都镇守一座峰巅,不能远离。入侵大雪山福地的八转蛊仙不能向六转、七转的雪峰峰主们出手,只能和雪胡老祖对战。

这个规矩也限制了雪胡老祖。

雪胡老祖也不能对来犯的六转、七转蛊仙为难。

皆因此项规矩,来源于对逆流河的利用。

人往高处走,水往低处流。

这是世间规律。

八转为高,七转、六转为低,高处的蛊仙只能寻找同等高高强度的蛊仙作战。

整个蛊阵当中,第一雪峰最高,所以闯入蛊阵的生命,都必须一路往高处进,不能走回头路。

这就是逆流河水的威能。

《人祖传》中记载,即便是人祖,都没有成功地逆流而出。

但这个逆流河,终究只是一处天地秘境而已。孙名录是利用了这个天地秘境中的丰富道痕,创建了逆命祭炼大阵。

所以,作为主阵方,雪胡老祖和大雪山的峰主们,也不得不遵守这项规矩。

若不是逆流河,而是换做效用相同的一只或数只仙蛊,那么蛊仙就可以自主操控,做出有利于自家的战斗环境,不像现在这样死板了。

但不管怎么说,即便是有这个死板的缺憾,大雪山福地也绝对是固若金汤,易守难攻。有此蛊阵在手,难怪雪胡老祖信心十足,根本不遮掩任何炼蛊的动静。

遮掩的话,也未必能遮掩得住。更关键的是,他并不害怕蛊仙们攻击过来。反而甚至会有些期待,因为死在蛊阵中的蛊仙,不论敌我,都会化为纯粹的力量,帮助他炼出鸿运齐天仙蛊!

“既然如此,那我们不如就先在这里驻扎,等待后续队伍的接应。”不真子建议道。

这个提议,让其余中洲蛊仙们都点头认同。

虽然他们洞悉了这座蛊阵的跟脚,但却没有什么大用。

赵怜云皱起眉头,但她亦没有反驳。毕竟马鸿运就在第一雪峰,那里就是蛊阵的中央,用来炼蛊的地方。就算赵怜云等人闯过了不少雪峰,最终还是要到第一雪峰上去,面对八转蛊仙雪胡老祖的。

黑天,镇运天宫。

“现了一小拨人。”南荒仙人稍稍吐出一口浊气。

“啊?在哪里?”药皇忙问。

“居然跑到了大雪山福地当中去了。奇怪……”南荒仙人陷入疑惑当中。

他虽然明白,中洲蛊仙们进入了无间道中,但并不清楚,无间道中生了意外。

他原以为能现中洲的大部队,但结果只现了一小撮人。

不过,这里面却有赵怜云这条大鱼。

爱情仙蛊。

南荒仙人苏醒的原因,就在于此。

“大雪山福地。”药皇皱起眉头,“中洲蛊仙们没有全部出现吗?”

南荒仙人摇摇头:“或许这是中洲蛊仙的诱敌之计,不管是八转蛊仙,还是仙蛊屋,都没有出现。”

药皇眉头皱得更深。

就像之前,方源和影无邪的对峙,方源始终不拿出上极天鹰,反而让影无邪忌惮不已。

药皇和南荒仙人面临的情况,也大抵类似。

中洲蛊仙们要全部出现,那他们就肯定动用全力,对付这帮家伙了。

但现在只出现了区区五人,反倒是让南荒仙人和药皇搞不清楚,中洲蛊仙们在搞什么鬼。

会不会有什么阴谋?他们是不是在酝酿什么特别的大计划?

南荒仙人想了想,决定:“就让这波中洲蛊仙和大雪山福地先对峙着吧。重中之重,还是中洲的大部队。这群蛊仙不出现,我们也静观其变。”

“是。”药皇点头,心中却在猜测,“南荒大人对大雪山福地中的情况,都一目了然。这绝非是镇运天宫之能,难道说我方在大雪山福地中也有内应不成?”

ps:数据的事情,有了新的进展,详情大家看我的微信公众号“作者蛊真人”就可以了。今天两更,深夜的时候,修改一下bug,把详细的修订表公布出来给大家看看。

\end{this_body}

