\newsection{方源献计}    %第六十节:方源献计

\begin{this_body}



%1
板栗牦牛被方源一巴掌拍进坑里去,就再也不敢爬出来。

%2
毛十二死命调动手段,都无法再让这头板栗牦牛爬起来再战。

%3
“荒兽不行,蛊仙也不行。”方源在心底摇了摇头,主动撤销仙道杀招。

%4
转眼间,巨大的飞熊化作一团光辉。

%5
从光辉中,方源的身影徐徐降下,直至落到地面。

%6
“这一次,又是我输了。方兄真是厉害!”毛十二努力半天,不见任何成效,只好抱拳,向方源认输。

%7
他和方源的切磋,已经进行了数次。

%8
起先,他输了之后,还不服输,有点倔强。但次数多了,已经麻木了,输是很正常的,赢了反而不正常。

%9
“十二兄,你这仙道杀招还要多加练习才是。而这头板栗牦牛,又弱又怯,还是尽早换个其他荒兽。”方源随意指点两句。

%10
毛十二连连点头,叹息一声:“这不是宝黄天关闭了么!等到它重新开启,我就立即去那里买下一头能战的荒兽。这头牦牛已经被养得野性全无,我早就想扔了。不过,收购荒兽的费用,得由我自己承担。这是建派之后的规矩。唉,我的奴兽仙蛊还不是自己的,还需要向门派支付费用。不聊了,不聊了,我赶回去炼蛊,争取能多卖点仙元石。”

%11
“那就不送了。”方源点头。

%12
毛十二收起板栗牦牛,风风火火的走了。

%13
“好,二十的门派贡献。又到手了。”方源望着毛十二离去的背影,心中荡漾着微微的喜悦。

%14
他这些天来。不仅是和毛十二切磋,还和其余毛民蛊仙。都有类似的交往。

%15
原来,方源和毛六暗中交易,虽然交易结束,但毛六屡次主动前往方源镇守的云城,却是个小小的破绽。

%16
方源虽然加入了琅琊派,成为客卿太上长老,但是却一直和毛民蛊仙之间有着一层隔阂。

%17
毕竟当今人族势大,毛民作为异人的一支,在历史上也惨遭人族的迫害。

%18
方源虽然是自己人。但毛民蛊仙都对他敬而远之。为什么毛六却忽然去主动造访方源呢?这就惹来不少疑虑。

%19
为了弥补这个破绽,毛六在交易之后,就有意无意地对其他毛民蛊仙透露,说他主动找寻方源,是为了提升自己的战力。因为之前的影宗突袭,还有这一次的渡劫,让他感觉自己在战斗能力上的巨大缺憾,他想要弥补,才找到人族蛊仙方源讨教一二。

%20
为了增加可信程度。毛六还主动和其他毛民蛊仙切磋,一一击败他们。

%21
毛六乃是琅琊派内奸,魔尊幽魂分魂之一,本身就有深厚的战斗造诣。稍微发挥一点真实水准,就很容易大败四方。

%22
这件事情,在毛民蛊仙的小圈子里引发了轰动。就连琅琊地灵都有所耳闻,并当众肯定了毛六的作为。还鼓励其他毛民蛊仙。也向方源求教。毕竟毛民蛊仙的战斗力极低,一直以来都是琅琊地灵的心病。

%23
毛六之后。又有其他毛民蛊仙来自己这里讨教切磋,方源思考之后,便知道了毛六的想法,于是热情接待其他毛民蛊仙。

%24
因为这对方源而言,也是有利无害的事情。

%25
首先,他要在未来很长一段时间里依靠琅琊派,让自己的修行更顺利,渡劫更安全,所以和这些毛民蛊仙处好关系,很是必要。

%26
其次,和毛六的出发点一样,方源这边也要配合,弥补这场交易的破绽。

%27
再次,方源需要琅琊派的门派贡献。

%28
今后,他要请琅琊地灵出手相助,借力琅琊派,总不可能次次和琅琊地灵谈判交易。

%29
之前他能和琅琊地灵交易,是因为琅琊地灵看中了他的落魄谷、荡魂山还有智慧蛊。

%30
今后,要让琅琊派为自己出力,主要就得用门派贡献。

%31
这是琅琊派的门规。就连琅琊地灵都主动遵守。

%32
琅琊地灵早先时候发布过门派任务,主要一种就是指点毛民蛊仙,提升他们的战斗能力。

%33
这个任务,就连琅琊地灵本身都不太合适,方源是唯一合适的人选。

%34
这几天来,方源指点了不少毛民蛊仙,反响极其热烈。

%35
哪怕后续的毛民蛊仙们,都受训效果不佳,也不怨方源,而是自己找自己的原因。

%36
一来是毛六竖立了榜样,二来也是方源摸清了这些毛民的底细,动用精妙的交际手腕,与他们热情相处,使得毛民蛊仙们对方源的印象大为改观。

%37
至于方源为什么能摸清这些毛民蛊仙的底细,那是因为他在交易中,从影宗那边得到了不少琅琊福地的情报。

%38
方源算了算:“这样一来,我的门派贡献,就已经有一百四十多了。还真是好赚呢。”

%39
每指点一次,方源就有二十的门派贡献。

%40
之前方源借出荡魂山、落魄谷什么的,也不过只是两三百贡献而已。

%41
“这一百四十的门派贡献,已经可以换取一部分的流光果了。”怀着这个念想,方源往第一云城赶去。

%42
到了第一云城,他见到琅琊地灵。

%43
琅琊地灵上下打量方源,交口称赞道:“不愧是变形仙蛊!你变化成毛民蛊仙,几乎毫无破绽,除非我动用侦查仙蛊。”

%44
方源来时,忽然灵光一闪。他便在半路上,催动变形仙蛊,变化成了一位毛民蛊仙。

%45
目的很明显,就是为了和琅琊地灵套近乎。

%46
琅琊地灵有大毛民主义,第一眼见到这样的方源,果然主动微笑起来,随后对变形仙蛊称赞不已。实则心中对方源看得更加顺眼,心情舒畅。

%47
方源却带着一抹可惜的语气,叹道:“这变化还不算完全,我虽然有变化仙蛊,但却缺乏仙蛊永固。若是能有永固仙蛊,我就能彻底转变成毛民,真正的融入到琅琊派中,成为毛民的一份子!太上大长老大人,你可能不知道,这些天来我和其他毛民蛊仙相处,感触很深。我们人族蛊仙各个狡诈,相互提防。但毛民率真,空前团结,温暖亲切,这才是一个真正的家园。对于这里,我已经开始留恋了。”

%48
琅琊地灵闻言,哈哈大笑,连拍方源的肩膀:“方源啊,你能有这样的觉悟,很好很好!”

%49
若是他知道,方源前几天还和影宗交易,对琅琊派内奸知情不报,绝对会把方源这个阴险小人生吞活剥!

%50
但可惜,琅琊地灵对此一无所知。

%51
“你这些天的表现,我都看在眼里。你指点那些孩子,令他们战力上涨很多,尤其是毛六!不错,不错。几天来我已经不止一次,从其他毛民蛊仙那里,听到关于你的好话了。嗯,你这次来,是有什么事情?”琅琊地灵问道。

%52
“是想用一些门派贡献,换取一些流光果。”方源答道。

%53
“这是小事!只要你贡献足够,换取流光果不会有任何问题。”琅琊地灵大手一挥,十分豪爽。

%54
方源装作欲言又止的样子:“我对琅琊派的发展,还有些微不足道的浅见。不知道当说不当说?”

%55
“说说吧,我听着呢。”琅琊地灵眯着眼睛,笑着道。

%56
方源便道:“这些天来,我和不少毛民蛊仙切磋了,因而略有心得。毛民蛊仙的战力急待提高,但如何提高,却是麻烦。因为几乎所有人,都主修炼道。炼道虽然也有攻伐手段,但到底不如其他流派强势。”

%57
琅琊地灵点头,方源的话说到了他的心坎里去了,不由感慨道:“这正是我所担忧的事啊。”

%58
方源再道:“其实速增战力,有一个好方法,那就是变化道。”

%59
“变化道?”

%60
“不错。”方源缓了缓,继续说道,“纵观所有蛊修流派,要说哪一道能令战力暴涨速成,自然首推血道。但血道危害极大,对他人太有威胁!在修行界中,名声已经彻底坏了,人人喊打。况且血道修为稍微精深一点,就要上天庭的诛魔榜,这就会暴露了我们。所以,血道不可取。”

%61
“你说的不错!”琅琊地灵点头赞同。

%62
其实就他本身而言,也对血道比较忌惮。若是毛民蛊仙修行血道,他还担心会不会对琅琊派内部,造成矛盾和隐患。

%63
方源继续侃侃而谈:“血道之外,最能速成战力的,就是变化道了。变化道门槛很低,仙道杀招还不需要谋划,任何一种变化,就是仙道杀招。正常的仙道杀招,至少演算个好几年吧。”

%64
“而琅琊派中,有大量的仙蛊方,其中就有很多变化道的仙蛊方。仙蛊方多,仙道杀招少。毛民蛊仙主修炼道,兼修变化道,正可以扬此长避其短。”

%65
“毛民蛊仙淳朴,要提拔他们的战力,绝非一日之功。若他们变化成荒兽,皮糙肉厚,就算战斗中失误几次,也不打紧,可以翻身再战。不像其他流派,诸如炎道、水道,人体脆弱不堪,没有护身手段,在战场上十分危险。”

%66
“变化道的仙蛊方,容易推算,仙材容易取,因此仙蛊也较容易炼制。”

%67
“毛民蛊仙们变化成荒兽战斗,也是对毛民身份的一种伪装。虽然我琅琊派底蕴深厚雄浑,但是五域中仍旧是人族势大。能少暴露的话,就应该尽量减少暴露的机会。”

\end{this_body}

