\newsection{被自己给坑了}    %第七百五十六节:被自己给坑了

\begin{this_body}

池曲由全力赶到凤焰山,便见护山大阵已是被方源攻破大半。

方源正在拔取晚霞梧桐,搜刮梧桐林中的火凤凰在内的诸多鸟禽,动作熟练至极!

镇守这里的蛊仙没有逃走,而是借助残存的大阵,正对方源发动猛攻。

方源也应付得游刃有余。

他一边对付残存大阵,一边大肆劫掠。

“住手!”池曲由大喝,又惊又怒。

“太上大家老,您终于来了!”镇守这里的七转蛊仙激动万分,热泪盈眶。

他起先还对护山大阵颇有自信,但交手片刻,就被方源攻破了大阵,吓得心肝儿直颤,胆寒无比,再不敢放肆。池曲由来之前,他都是以守为主。见池曲由来了,他立即发起一些声势浩大的攻势。

池曲由杀奔过来,方源主动收手。

“方源,你做的好事!现在报应来了!”池曲由双眼喷火。

凤焰山这处巨型资源点有些特殊。

它除了养凤凰之外,每年到了特定的时期,还会吸引到许多野生凤凰,前来这里,停留产蛋。一旦凤焰山遭受剧变,野生凤凰极可能就另选他址,会令池家损失很大。

方源却是面带微笑。

和上一世相比,池曲由更加愤怒,因为方源搜刮得力,这一世的池家受损更加严重,让池曲由心头滴血。

“池曲由前辈,我也等你多时,且看我这一招!”方源说着,催动阎罗战场。

池曲由犹豫了一下,没有躲闪,而是任凭阎罗战场将自己包裹,口中嚷道:“我倒要看看你的本事!”

阎罗战场迅速成形,形成一大片的黄褐云团,将方源和池曲由都包裹进去。

随后,阎罗战场迅速收缩,前一刻它仿佛是黄褐色的巨大云团,下一刻就浓缩成一个点,模仿仙窍收缩寄托虚空,消失无踪。

“好一个魂道战场。”池曲由赞叹。

池曲由双手背后,好整以暇,态度发生剧烈改变,虽然陷入阎罗战场却没有着急动手,而是悠闲从容。

他心中暗暗忌惮:没想到方源连战场杀招都有了,之前根本就没有相关的情报。

和上一世的同期相比,方源这一世的阎罗战场没有遭到曝光。

他在上一世,用阎罗战场围困凤九歌,不仅暴露这一手段,而且还被紫薇仙子推算、破解成功。

琅琊福地保卫战后,紫薇仙子更是将阎罗战场的种种玄妙,还有如何破击此招,都无偿的公布到宝黄天中去,导致方源的阎罗战场实战价值大大缩减,风险成倍提高。

而这一世,方源虽然也动用过阎罗战场数次。但分别是蒙屠、睡姑、钻头鳖,这些对象的结果是死的死,降的降。而对付凤九歌的时候,则是身处于超级大阵空间之中,不方便催动阎罗战场。因此这个手段就一直保密,世人不知。

此刻方源用出来,才算是真正意义上的曝光。

方源不可能将池曲由杀了此战后阎罗战场的情报,必将广为流传。

当然,不杀池曲由,不是方源没有这个能力,而是杀了池曲由风险太高,收益太小。这就和池曲由之前不愿意对付方源一样。

两虎相争,必有一伤。

方源站在战场的另一侧,遥遥打量池曲由,面露微笑:“池曲由大人,你如今身陷我的仙道战场,却无一丝进攻的架势,而是想和我闲聊吗?你意欲何为?”

池曲由冷哼一声:“休要装蒜!”

“方源你四处劫掠,无非是想引起我的注意罢了。”

“仔细想想,你既然可以轻易攻破守护大阵,为什么就没有取走池家蛊仙的性命?这对于你而言,并不困难。但你却从未害过一位池家蛊仙的性命。”

“其次,不管是在摄心河滩、云竹山脉、凤焰山,你劫掠虽凶,但每一地都保留下了根基,没有破坏。”

“我和太宇寺出击,也在你的预料当中。见我们出动,你便故意攻击凤焰山和褴褛洞,诱导我方分兵。”

“你无非是想找到一个机会,来和我交流对话,却又没有什么其他的隐秘方法。”

“说吧,你到底有什么事情。”

池曲由侃侃而谈。早在分兵之前,他就猜到了方源的想法,之后的一切怒恨都是他的演技。

池曲由分析的大体上是没有错的。

上一世,掠影地沟仍在,池曲由坐镇了一段时间,直到方源进攻凤焰山,才将他单独吸引过来。

这一世,方源早就将掠影地沟给攻破了,池曲由和太宇寺一直在一起,这就让方源有些难办。

太宇寺这座宇道仙蛊屋,专门克制仙道战场。方源上一世就和它交手,就连大盗鬼手都拿它没有办法。

方源要和池曲由交流,最好就是要避开此屋,创造机会,吸引池曲由出来单独详谈。

这点难不倒方源。方源的智道造诣深厚,脑海中念头一翻,稍稍做了一些改变,便达成了这个目标。

只是有一点,池曲由猜错了。

方源在每一处资源点都留下根基,不是因为他故意保留,而是自身没有收取的手段。

池曲由是会错了情。

当然,方源是绝不会点明的。

“啪啪啪。”方源轻轻地鼓起掌来,“和聪明人讲话,就是轻松惬意。池曲由前辈,且先看看这个如何。”

池曲由眼中精芒闪烁不定,小心翼翼地接过方源抛出的信道凡蛊。

他表面上云淡风轻,实际上心中戒备非常。方源虽然是“七转”修为,但池曲由深知此子面厚心黑,狡诈凶残,心狠手辣,必须得时刻防备他可能的阴谋诡计,一刻都不能放松。

但很快,池曲由的面色就变了。

他眉头皱起,眼中寒芒闪烁,脸上笼罩着一层青色。

方源抛来的信道凡蛊中,记载着池家诸多要地的守护大阵,以及方源的猜测和针对破解的手法。其中就有凤焰山的大阵,方源清楚得一塌糊涂。按照方源对此阵的理解,他完全可以直接攻破大阵,尽取凤焰山的资源,但他却没有这么做。

从这一点上,池曲由看到了方源的“诚意”,证实了自己刚刚的猜测,同时也暗暗心惊于方源的阵道造诣!

而真正让他心寒的,却是信道凡蛊中的后面部分内容。

这部分内容,详细地讲述了池家的种种情报,令池曲由都心中发冷。

这绝不是打探一天两天就能打探出来的,非得是长时间的积累,才能积累出来的丰厚情报。

而在这些情报中,关于池家内患,池曲由后继无人的内容,乃是重点中的重点!

这由不得池曲由不变色,不动容!

因为这份情报,十分深刻,很明显是有南疆的人帮助方源,暗中提供。

究竟是谁?

魔道蛊仙的嫌疑反而很小,他们独来独往,和池家结仇的并不多,也犯不着长期监视池家而不顾自身修行。

“除了那些居心叵测的正道势力,还能有谁?”

池曲由心中惊怒交加,羊家很快就在他心中划过,成为头号嫌疑犯!

池曲由恨啊!

如果有南疆正道的某个超级势力,暗助方源一臂之力,那他的处境就更加被动了。

“很可能我池家没有出内鬼,而是这些狼心狗肺的正道家族,要借方源这把利刃,来对付我池家啊!”

“这些蠢货,这些白痴,这些内鬼,这些南疆正道最大的蛀虫!”

池曲由心中的愤怒,像是火山在喷发,岩浆不断爆涌而出。

他已经强忍住,若非方源就在眼前,说不定他早已气得浑身颤抖了。

方源暗笑。

他察言观色,对于池曲由现在的心情,还有内心的想法,都能洞悉大概。

他给出来的这些情报,绝对是需要大规模的人力物力,长时间搜刮,才能积累出来。

这倒并非是有哪一家的南疆正道大族帮助了他,而是他上一世俘虏了南疆群仙,从他们的魂魄中搜刮出来的。

有了这张好牌,方源当然要利用!

果然不出他的所料,这张牌一用,效果立竿见影。池曲由脸色铁青,之前的稳重和风姿已是荡然无存。

“想必此刻,在池曲由的心中,对所谓那些南疆正道家族的恨意,要远远超过我吧。”

“这是当然的。自古以来,几乎所有的组织成员都最恨二五仔!”

“但实际上,这只是我上一世所得。而我上一世之所以能够俘虏南疆群仙,还多亏了池曲由暗中输送给我大量梦境。所以罪魁祸首还是池曲由自己啊。”

池曲由脸色很难看,他浑身上下都散发出浓浓悲剧的味道。

他是被自己给坑了,但他根本无法察觉。

方源再度开口:“池曲由大人成名已久,经营池家这么多年,想必十分清楚这份情报的含义了。给我这份情报的势力,不仅是小看了我,而且还小看了池大人你呢。”

池曲由听了这话,心中念头忽起:“看来方源也不甘心被人利用。”

不知不觉间,他觉得方源比之前顺眼多了。

方源继续道:“现在你当更相信我的诚意了吧?”

池曲由顿时又想起池家的损失,恨得肝疼,他眯着眼睛看方源,冷笑道:“呵呵,这么看来,我还应该感谢你留手了?”

方源微笑。

\end{this_body}

