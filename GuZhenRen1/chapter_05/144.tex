\newsection{叶凡vs白凝冰}    %第一百四十四节:叶凡vs白凝冰

\begin{this_body}

%1
“呼……”方源看着眼前一片黑色的腐臭坑地,长长地舒了一口气。

%2
“第五次地灾,总算是渡过去了!”

%3
腐蚀暗流让方源疲于应付,不过好在他掌握了黑凡真传,没有一块短板。坚持下来之后,将此劫难渡过。

%4
这里的地貌,彻底改变,需要花大力气治理一番,才能重现之前的状态。

%5
好在方源的至尊仙窍足够广大,这片地方位于小西漠当中,原本也是片荒凉之地,所以方源的损失微乎其微。

%6
甚至,他完全可以保留这片地域,不做任何治理。

%7
福兮祸兮,凡事都要两面看待。

%8
这块地方虽然不能栽种寻常植株,但却可以看做是一种极端的环境,可以让许多特殊的猛兽、植株,在这里生存繁衍。

%9
当然,这里残留的天意,方源是需要用大量的我意凡蛊,来彻底清除的。

%10
不过,这只是小事一桩了。

%11
重点是第五次地灾算是渡过去了。

%12
顺利地返回琅琊福地,方源总结一下,他发现第五次地灾虽然威力强盛,超过之前任何一次,但却也没有超出自己的原先估算。

%13
能够渡过地灾,关键的原因在于方源的实力增长程度,大大超过地灾威力的增长。

%14
“看来地灾并不可怕,目前为止,地灾已经阻挡不了我了。可怕的是人劫!”

%15
方源回想起第四次地灾,人劫的恐怖,他心底还是冒寒气。

%16
那场人劫相当可怕,连太古石龙都登场了。人劫威能大大超过方源的极限,全靠他坑蒙拐骗,临危不乱,才险险渡过。

%17
“这么一想,天意之所以酝酿人劫的原因,岂不就是单靠地灾已经扼杀不了我吗?”

%18
“正因如此,天意才千方百计地布局。利用人劫来毁灭我。”

%19
“我这一次渡劫,就没有遇到什么人劫。那是因为,我是完整的天外之魔,天意无法捉摸出我的计划图谋。再加上我行动迅猛。让天意没有足够的时间来酝酿人劫!”

%20
天意不是假意,影响能力没有假意那么显著。尤其是对修为更高,灵性最强的人族蛊仙而言。

%21
这么一想,方源觉得以后,还是不要固定在某个地方渡劫为好。

%22
总是在北部冰原渡劫。才让天意有机可乘,潜移默化地影响到了许多异人蛊仙,真的差点害死方源。

%23
“只是这一次,我刻意选择地沟,就是想遇到土道灾劫,帮助我的至尊仙窍增添大量的土道道痕。结果却碰到暗道地灾。”

%24
地沟中,土道、暗道两种道痕最为浓郁。

%25
方源这一次成功催动了仙劫锻窍,灾劫的类属就在这两者中选择。

%26
但天意显然不想让方源如愿以偿,没有形成方源最想要的土道地灾,而是暗道。

%27
方源用道可道仙蛊侦查一番。发现自己身上的暗道道痕,数额暴涨,已经十分接近一万了。

%28
“之前暗道道痕有些基础,但十分薄弱。这一次地灾,起码增长了九千道痕!也不算毫无收获。”

%29
方源虽然暗道境界很渣,但是他却掌握着一只暗道仙蛊暗渡!

%30
这只仙蛊对他大有用处,能够遮掩自身,一定程度上屏蔽天意感知。姜钰仙子曾用此,为黑楼兰完美地遮掩过十绝气息。

%31
这一次渡劫,方源只花费了数天而已。

%32
黑凡洞天那边。还在僵持状态。

%33
楚度是七转巅峰战力,黑凡洞天又被他布置严谨,防守滴水不漏。

%34
不过,方源却可以看出。真正占据上风的,还是百足天君,虽然他屡次突进没有成功,看起来有些灰头土脸。

%35
因为百足天君纯粹是进攻,是打是退,什么时候打。什么时候退,完全由他决定。而楚度只能依赖洞天进行防守,不知道什么时候百足天君动手,又用什么方法。这就很被动了。

%36
“楚度守久必失。不过距离他的极限,肯定还有一段时日。”方源暗暗揣摩战况。

%37
他开动脑筋,开始琢磨楚度若是战死,对他自己有利还是有害?

%38
一番思考,方源觉得:楚度还是不死为妙。

%39
原因主要有几点。

%40
第一,楚度身上有着方源的投资。最大的一笔投资,就是七转仙蛊招灾。楚度若死,招灾必亡。

%41
第二,楚度和自己身上有着盟约,见死不救是不行的。

%42
第三,今后灾劫会越来越强,楚度掌握招灾,能够替方源分担很大一部分压力。他的存在,更有利于方源的发展。

%43
第四,若是帮助楚度护卫住黑凡洞天,那么方源就可以移栽天晶的蓄养池,利用黑凡洞天,源源不断地产出天晶,为再次孵化、豢养上极天鹰做准备。

%44
“看来我还是要出手,对付百足天君的。”方源暗下了这个决定。

%45
不过,此时他却不忙出手。

%46
就让楚度再苦捱一段时间,自己刚刚渡劫,需要休整,青提仙元更需储备积累。

%47
这理由正正当当,不算违背盟约,就算楚度得知,也无法诘难方源。

%48
南疆,日冠山。

%49
此山巍峨葱茏,伫立大地之上。每当朝阳时分,便有日晕笼罩山顶,宛若一顶光明高冠,因而得名为日冠山。

%50
此刻却是夜晚。

%51
明月清辉,洒遍山间。

%52
“还有谁来?”叶凡傲立,双臂怀抱在胸口,驻足在一块巨大的山石之上。

%53
在他的对面,一群蛊师畏畏缩缩,犹豫不前。

%54
他们已经被叶凡杀怕了!

%55
“叶凡,你是给家族驱除流放出来的,你也是一个散修。我们都是散修,何必自己人为难自己人?”一位蛊师开口道。

%56
叶凡冷笑:“交手之前,你们依仗人多势众,怎么不说这话?更何况,你们算是散修吗?霸占此山,劫掠四周,身为蛊师,恃强凌弱,恣意欺辱甚至屠戮凡人。你们统统罪大恶极,根本就是魔修,算不上散修!”

%57
“冤枉啊。罪魁祸首都已经伏诛在叶凡的手中,我们只是被逼才这么干的。”

%58
“还有,自从我们俯首于白煞老大,我们已经再不为非作歹了!”

%59
蛊师们嚎叫起来。

%60
叶凡的脸色微微一缓。

%61
据他打探得到的情报,情况的确如此。

%62
自从白凝冰执掌日冠、静流两山,压服群魔,就约束收下,没有做过什么令人发指的恶行。

%63
“哼,若非如此,你们现在都已经丧命,我怎可能留你们到现在?”叶凡冷哼一声,又继续道,“此次我代表商家,就是来打通日冠、静流两山的。我在这里等候,你们速去叫白煞出来见我。否则,到了天明时分,我就将这里屠戮一空!”

%64
叶凡十分机敏,不是单凭勇猛的莽夫。

%65
他知道静流山的防御,远比日冠山要周到严谨数倍,自己孤身一人前来,不可强攻蛮干。

%66
打草惊蛇,把白煞主动吸引出来,方是良策。

%67
不过他此话话音刚落,就听到一个冰冷清澈的声音响起:“不必叫了,因我早已经来了。”

%68
说着,十几道身影,出现在山腰那处。

%69
群魔见之,狂喜大叫:“白煞大人!是白煞大人来救我们了!”

%70
叶凡举目视之,只见来者皆是蛊师,修为方面比日冠山要普遍高出一层。其中不乏魔道、散修中的好手,还有几位,已经成名多年,各具特异的风度。

%71
但真正吸引他眼球的,乃是最中央的那一名女子。

%72
此女一身白衣,银发晶莹,宛若流苏,垂至腰际,深蓝的眼眸清澈如湖,波澜不惊,肌肤若雪,面容冷漠,难掩绝美的容颜。

%73
此刻,她正躺卧在一个竹制的躺椅上,好整以暇,双眼似闭,仿佛假寐。竹椅前后,由四位雪人担负。

%74
叶凡心中一动。

%75
他流浪在外,也算见多识广。

%76
单论外貌,白凝冰的容颜绝对是绝美!能和其相提并论的,唯有商心慈一人。

%77
“白煞大人,小的们拼尽全力防守,等你等的好苦啊,您终于是来了。”这时,一位魔道蛊师跌跌撞撞,爬到白凝冰的面前,口中大叫,满脸谄媚之色。

%78
“你,畏敌避战,该死。”白凝冰微微睁开一丝眼帘,眼眸中蓝光一闪,下一刻那位跪在地上的蛊师,就彻底冻成了冰霜,死得不能再死了。

%79
群魔顿惊,白凝冰身后的几位强者更是身躯发颤,回忆起了当初初见白凝冰的恐怖。

%80
叶凡也是一惊,旋即勃然大怒。第一眼看见白凝冰的好感荡然无存,取而代之的是怒火熊熊。

%81
“果然是魔头!连自己人都杀!”叶凡口中呼喝,跳下巨石,怒视白凝冰。

%82
白凝冰微微一笑,伸出如葱玉指,对着叶凡轻轻一点。

%83
刹那间,叶凡只感觉一股冰冷之气,笼罩自己的左腿。

%84
他定睛一瞧,只见自己的左腿上,竟然在转瞬间结成了一大块的透明冰块。

%85
“这是什么招数?我早就催着防御杀招,但竟然连一丝抵挡之力都没有?!”叶凡十分震惊。

%86
一直以来,他运用这些手段,无往不利。但现在,却在白凝冰的手中,连一个回合都撑不过。

%87
“白煞究竟是什么修为?我的手段,可是蛊仙周青青大人亲自授予的!竟然如此不济!!看她模样,似乎还留有余力。难道我这一次要战死于此?”叶凡心头猛震。

%88
他太低估白凝冰了,没有料到双方的差距如此巨大。不过震惊之后,叶凡很快平定心境,绝境中战意更加旺盛。

\end{this_body}

