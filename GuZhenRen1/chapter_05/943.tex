\newsection{毁了又如何?}    %第九百四十六节:毁了又如何?

\begin{this_body}

宿命蛊被方源摧毁,整个战场顿时恢复了原状。

命败杀招形成的白光,陡然消散一空。

咔嚓嚓。

一连串的脆响,整个监天塔上迅速蔓延遍布无数裂纹。然后下一刻,砰的一声,整个监天塔破碎!

偌大的天庭,陷入一片死寂当中。

战场上没有一丝的风,但众仙的心头却是有飓风呼啸!

监天塔化为无数碎片,或大或小,伴随着蛊仙的尸躯,无助地坠落下去。元莲仙尊埋藏着的画道手段,则化作一道流光,划破天际,远射而去。

方源也不管它,静静地悬停半空,转身看向龙公、凤九歌等人。

他面无表情,没有说一句话,只是缓缓地展开手掌,让众仙以及众目睽睽杀招之下,全五域的人们都看到——宿命蛊被彻底摧毁的碎片!

“宿命蛊……毁了?”

“我们胜利了!?”

三域蛊仙面面相觑,皆看到彼此脸上的惊愕和迷茫之色。

“难以置信,我们真的做到了!”

“这一场战斗,绝对会后人铭记。”

反应过来后,三域蛊仙们狂喜呐喊,欢呼咆哮!

反观天庭一方,剩余的蛊仙们无不脸色灰败,一个个形如木偶。

宿命蛊乃是天庭的旗帜,精神的支柱。

它的作用不仅仅是一只九转蛊虫,从元始仙尊开始,就依靠此蛊来彰显人族的地位,给予当时的人族自由的希望、反抗的勇气。

经过三百多万年的继承和发展,宿命蛊对于天庭而言,意义更加重大。是天庭的底蕴,是象征,更是一面永久飘扬的旗帜。

然而现在,这面旗帜却是被方源硬生生摧垮、撕烂!

天庭诸仙看到这一幕,仿佛心中的支柱被陡然抽走,一时间天都塌了!

在三域蛊仙没有攻击的时候,他们没有外在的刺激,一个个都茫然失措,不知道应该怎么办。

就连凤九歌都沉默,一直响彻战场的命运歌也在此刻停息下来。

远处,凤金煌藏身在一个隐秘的地下仙蛊屋中,已经泣不成声:“宿命蛊……被毁了!为什么会这样?明明大家都这么努力,都不惜牺牲自己!”

秦鼎菱目光呆滞,也在发怔,仿佛一具毫无生命气息的石像。

残破不堪的诛魔榜缓缓升空,蛊屋内的方正仰望着方源,神情复杂至极:“方源,你居然真的做到了。在天庭,在尊者手段的围追堵截下,摧毁了宿命蛊!我与你的差距原来仍旧是这样的巨大。”

这种差距让方正感到绝望,但微妙的是,方正此刻却不感觉到痛苦,反而有一丝轻松。

是该放弃了。

方源这样的人,从一开始就该与众不同。自己想和他媲美,从始至终只是一场奢望,一场痴想吧。

“可恶,可恨……”

“失败了,宿命蛊被毁了。”

天庭蛊仙中许多人泪流满面。他们拼尽了全力,付出了全部的心血,有的从墓中爬起再战,有的坚守到底,但是最终还是失败了。

“真是愧对牺牲的同伴呐!”

“让我无颜以对师门先辈!”

羞愧之色浮现在天庭众仙的脸上,很多人竟都有一种当场自尽的冲动。

“都不要慌!”就在这个时候,龙公忽然开口,声振千里,让所有蛊仙都心头一震。

龙公拔升到最高空,他面色凝重,心知眼下乃是关键时刻。

宿命蛊被毁了,龙公无法懊恼和自责,他是天庭的领袖,他需要去尽力解决这个问题!

让他十分焦虑的是,百战之师的天庭蛊仙尚且如此动摇,那么看到这一幕的全天下的人,又该如何去想呢?

如果说,方源之前三番五次地坑害天庭,是在狠狠地扇天庭脸面的巴掌。那么现在,方源摧毁了宿命蛊,简直是将天庭的根基彻底动摇,将天庭积累了三百多万年的地位和名誉摧毁殆尽!

龙公必须重整军心,振奋士气。

“方源,你真以为你摧毁了宿命蛊吗?”龙公忽然哈哈大笑,一点都看不出他有丝毫的动摇和颓丧。

方源没有答话,只是静静地看着他,微微地将手掌抬了抬。

当即三域蛊仙中就传出一声嘲弄:“龙公,你的眼睛是瞎的吗?没有看到宿命蛊的残渣吗?”

天庭蛊仙中有人双眼闪烁,浮现起希望的花火:“难道方源摧毁的宿命蛊是假的?”

但龙公却摇头,仍旧笑道:“呵呵呵,被摧毁的宿命蛊当然是真的。但是方源啊,就算宿命蛊被摧毁了,难道它就永远毁了,再不能被我们炼制出来吗?”

“嗯?”群仙皆愣了愣。

仙蛊唯一,乃是天地间的一份真理,颠扑不破。我有了一只仙蛊,其他人就不可能有相同的一只。

但是当这只仙蛊摧毁了,全天下人便都有了拥有它的机会。

宿命蛊现在是被方源摧毁了,这是铁的事实。但这并非是真正的了结,因为天庭还可以再炼出来啊!

“难道天庭已有宿命蛊的蛊方?”沈从声阴沉着脸,发问。

“宿命蛊被我天庭执掌了三百多万年,期间历经无数炼道大能,还有三代仙尊。你们猜猜看,我们天庭是否掌握了宿命蛊蛊方呢?”龙公微笑着,全身上下都散发出一股自信和从容。

“该死!”当即就有三域蛊仙气急败坏。

龙公再笑:“然而,就算天庭没有宿命蛊的仙蛊方,又如何呢?诸位,我不妨问问你们,太古时代的宿命蛊,难道是人祖炼制的吗?”

“当然不是。”秦鼎菱出声回答,她明白了龙公的意思,眼中再现生气,“宿命蛊本就是天地而生!”

龙公点头:“不错,仙蛊天然而生,所有才有众多的野生仙蛊。就算我们不炼出来,天地也会把宿命蛊炼出来。新生的宿命蛊可能是六转,也有可能直接是九转。谁又说得准呢?”

这次轮到三域蛊仙们面面相觑了。

被龙公这么一点拨,他们忽然发现,好像摧毁了宿命蛊,也不是什么大事!

宿命蛊毁了也就罢了,今后还可再炼呐。

“就算炼出来,我们也能再毁掉它!”有蛊仙不忿地呼喊,但他自己也没有多少底气。

这一次摧毁宿命蛊,费了多大的劲!这样的成功真的可以再复制出来吗?别的不说,单单只有天外之魔才能摧毁宿命蛊这一点,就不知卡死了多少英才枭雄。没看到三大魔尊都没有成功吗?

“该死,怎么就没有想到这一节呢?”有蛊仙无比懊恼,但仔细想想,却也不能怪自己思虑不周吧。

毕竟攻上天庭,再摧毁宿命蛊,这个目标本身就已经高大如天,需要蛊仙们扬起脖子来奢望。要实现这个目标,就已经让蛊仙们穷竭心力,好比凡人要登天。这完全是不可能完成的目标,即便是拼尽全力,穷尽心志,也难以达成。

在这种情况下,谁又能想到登天之后的事情呢?

事实上,中洲举办炼蛊大会,四域蛊仙齐攻,谁又能真正肯定可以摧毁得了宿命蛊?

没有人!

就算是方源,即便是作局的历代尊者都没有这个自信。

所有人都是在尝试。

尝试成功后,蛊仙们欣喜若狂。但实现了这个目标之后呢?

龙公一番话,说的也是事实。三域蛊仙们陡然发现,似乎他们如此付出,如此拼杀,如此努力,都是一场空?都会轻易地化为泡影?

三域蛊仙们沉默,脸色逐渐变得难看。

武庸敏锐地察觉到人心变易,士气低落,立即冷声反驳:“龙公大人好一番口舌。然而天庭的失败就是失败,你们煞费苦心,做出重重防备,哪怕把宿命蛊藏在天庭中,我们都能毁掉。将来就算让你们重新得了宿命蛊,又能如何?”

“呵呵,事实上,你们还要当心。别赶在天地自然产生了宿命蛊后,你们也炼制不出来。到那时,宿命蛊成了野生仙蛊,那就是人人可得,未必是你天庭之物了。”

三域蛊仙听闻这番话,纷纷眼前发亮,士气回升。

龙公仍旧从容而笑:“武庸,你说的话一点都没有错。退一万步来讲,就算宿命蛊成了野生仙蛊,又被你们当中的人所得,又能如何呢?你们……能用吗?会用吗?”

面对这个问题,就算是武庸也不得不哑然。

龙公傲然一笑:“宿命蛊从来都不能被蛊仙运用!远古时代之前,它掌握在异人的手中,不能被他们所用。三百万来,我天庭历代仙尊也从未有人能直接催使此蛊。无数贤达能才设想出了监天塔,三百万年来也才研究出一种利用宿命蛊的杀招,即是命败!”

“你们懂命败杀招吗?你们知道监天塔如何搭建吗?”

三域蛊仙都陷入沉默。

龙公继续道:“能够使用宿命蛊的,唯有天意。当年星宿仙尊为了人族谋算,不惜牺牲自己,和天意融合一体,干扰和影响天意。从而才使得人族三百万年前崛起,三百万年后人族仍旧是天地霸主,昌盛不断。”

“而我天庭为了支持星宿意志,设立仙墓。天庭成员都会尽量沉眠仙墓,为星宿意志提供源源不断的帮助。因此,我们才能影响宿命蛊,从而影响天下格局!”

“你们谁能做到这一点?”

龙公扫视全场,三域蛊仙默然无音,而天庭的蛊仙们一个个都昂首挺胸,有人傲然接话:“没有人!偌大的五域,只有我天庭这样的势力方能做到这一点!”

“不,你错了,龙公。”劫运坛中传出冰塞川的反驳,“就算我们无法催用宿命蛊,但我们可以用它为一桩炼蛊材料,炼制出命运蛊!宿命蛊是无法被人操纵,但命运蛊却是可以。这正是我家巨阳仙尊大人的谋算。”

龙公望向劫运坛,轻蔑一笑:“就算你炼成了命运蛊,难道宿命蛊就会不存在吗?”

冰塞川无法回应。

宿命蛊若是充作蛊材,即便长生天一方炼出了命运蛊,那么宿命蛊就算是毁掉了。将来仍旧可以出现一只全新的宿命蛊来。

龙公大叹:“不管你们如何做,宿命蛊仍旧会存在。即便现在毁了,将来也会重新出现。只要它一出现,你们今天所有的努力,对于你们而言又有什么意义呢?”

三域蛊仙默然。

“当然是有意义!”龙公自问自答,一脸沉重和痛恨之色,“这当中的意义,就是给异人崛起的机会,给我们人族造成难以想象的糟糕影响。”

“你们用不了宿命蛊,你们何时能炼出命运蛊?你们甚至连命运蛊的半成蛊方都没有!而宿命蛊一旦重新出现,就会遭受天意的掌控。人族将衰落,异族将崛起。”

“而造成这一切的,就是你们!”龙公痛骂,忽然手指着方源,“尤其是你!你真是罪大恶极!你摧毁了仙墓,星宿意志无法再像之前那般干扰天意。如果将来人族灭亡,你就是始作俑者、罪魁祸首!!”

“方源啊方源,你以为你摧毁了宿命蛊,就得到了胜利?”

“呵呵呵,充其量,这只是一场闹剧罢了。”

“我天庭处处为人族着想,维护人族利益,为人族而流血、而牺牲。而你们这些人为了一己私利,破坏了我天庭兢兢业业维持了数百万年的大局。”

“宿命蛊我天庭以后可以重炼,秘密炼制!再不想修复宿命蛊搞得五域皆知,你们能听得到风声吗?”

“你们能及时阻止我们吗?”

“即便你们第二次、第三次阻止住我们。以后呢?谁能保证你们次次都能阻止我们炼蛊?”

“我们只是炼制一只六转宿命蛊,也远比修复九转要容易轻松得多!”

“而我们每一次围绕宿命蛊的争斗,都会人族的惨重损失,极大地消耗人族的底蕴。这就给其他异族崛起的空间和良机啊!”

龙公说到这里,发出一声深深的哀叹。

“这个老人是谁?没想到天庭做出这么多的事情,他们一直在为我们全部人族考虑!”

“为什么要攻打天庭,这些攻打天庭的蛊仙究竟是怎么想的?”

“这些人都是魔头,为了自己的利益,置大局于不顾!都是千杀的混蛋!”

五域的凡人们被引发了共鸣。

大多数的凡人对蛊仙的情况知之甚少。

而那些散落五域的蛊仙,此刻听闻龙公的这一番话,也纷纷陷入沉默之中。

天庭强大,天庭霸道,天庭掌控几乎中洲全部的修行资源,天庭手伸得很长,其他四域的重大事情他们都要积极地插一手。

但是天庭维护人族的功绩,所有人都无法抹灭。

这是事实!

“哎呀呀,这下真有点麻烦了。虽然捏爆了宿命蛊真的很爽,但是爽过之后就有点空虚,就有点麻烦了啊。”血战披风中,狂蛮意志捂住额头嘟嘟囔囔起来。

“呵呵,龙公,说完了?”方源神情淡漠,却是在捏爆了宿命蛊后,首次淡淡开口,打破了全场的沉默。

龙公看向方源,眉头皱起,心中越发感到不妥。

方源太冷静了!按照方源的智谋水准,怎么会看不清楚关键?龙公换位思考,若是他是方源,必定会在刚刚全力出手,不给自己任何发言的机会。

一旦自己从容开口,就会造成眼下的局面。

是的。

龙公的一番话,让天庭蛊仙们再度恢复了斗志,并且打压敌方士气,又拔高天庭声誉名望,绝对是力挽狂澜!

龙公不愧是天庭的第一领袖,他不仅拥有支柱般的恐怖战力,更能在关键时刻,发挥领导才华,鼓舞士气,改易人心。

方源怎么会看不出来?

但他偏偏没有阻止龙公,甚至从他的目光中,龙公竟隐晦地品尝到一丝期待的意味?

他方源究竟在期待什么?

下一刻,方源轻声一笑:“龙公啊,是谁告诉你,我捏碎摧毁了宿命蛊,就是此战的终结?你似乎忘记了一个人呢。”

龙公听了这话,瞳孔猛地一缩,意识到了什么。

他顺着方源眺望的目光望去,于是他看到了一片河。

那是一段光阴长河的虚影!

劫运坛上携带着红莲魔尊的手段——前无古人杀招,从开战之处,这个杀招的力量就在天庭的空中,凝造成了这么一片光阴长河的虚影。

从这段虚影中走出历史上的北原蛊仙强者,给天庭带来了极大的威胁。

但是伴随着战斗进行下去,光阴长河的虚影似乎逐渐无力,越来越少的北原强者被召唤而出。

等到战斗后期,这片光阴长河虚影几乎毫无动静,不见丝毫新的人影。

然而,当方源摧毁了宿命蛊后,这片光阴长河虚影就好像是挣脱了牢笼的猛兽,变得汹涌澎湃起来。

哗哗哗……

河水激荡,掀起惊涛骇浪。

一个身影潇洒从容,从巨浪风波中,徐徐而来。

场中群仙便听他高歌道:

当时年少掷春光,花马踏蹄酒溅香。

爱恨情仇随浪来,夏蝉歌醒夜未央。

光阴长河种红莲,韶光重回泪已干。

今刻沧桑登舞榭,万灵且待命无缰!

他少年模样,目光沧桑,一身朱袍,面冠如玉,眼若明星,肤白似雪。

他从光阴长河的虚影中脱身而出,步入天庭战场。

全场瞩目,鸦雀无声。

环视左右,他的目光停留在龙公的身上。

龙公一脸震动,神色极其复杂,张口欲言,嘴唇翕动,却说不出话来。

少年微笑:“师父,别来无恙?”

------------

\end{this_body}

