\newsection{七幻真传脉轮瓜}    %第三百一十八节:七幻真传脉轮瓜

\begin{this_body}

黑楼兰还未说完,就被紫山真君打断:“没错,这片福地已经被乐土仙尊改造,从此再无灾劫,从菇人福地,变成了菇人乐土。<strong>最新章节全文阅读www.QiuSHU.cc</strong>当然,世间的乐土有很多,但这里却很特别,因为乐土仙尊在未成就尊者之前,便在这里修行学艺。”

“唉!”菇人蛊仙这时长叹一口气,“从我继承了那道真传的时候,我就知道:早晚有一天,我会为此付出最沉重的代价。不过你若想要我出山,助你为恶,那是不可能的,你干脆杀了我好了。”

“我不杀你。”紫山真君流露出一抹狡猾的笑容,“我要让你带着他们两个,偷偷潜入瓜熟地洞,也就是乐土仙尊曾经待过的地方,修行一段时间。这个要求,你不会不答应吧?”

菇人蛊仙脸色越加愁苦,他沉默了一会儿,最后沉重地点头道:“好,我答应你。”

菇人乐土,瓜熟地洞。

这片地洞,正是这位菇人蛊仙把守的。

紫山真君与天意同化,感知到了天意的布局,利用这个菇人蛊仙,他顺利地将白凝冰、黑楼兰二人送进了这个传奇之地。

“这里是我族豢养脉轮瓜的地方。你们在这里偷偷修行,不得损害其中任何一只脉轮瓜。”菇人蛊仙关照道。

脉轮瓜?

白凝冰和黑楼兰对视一眼,均瞧见彼此眼中的惊异之色。

这脉轮瓜乃是七转仙材,蕴藏的是炼道道痕,可以用于任何一种炼蛊的过程中。

所以,这种仙材的用途,非常广泛。

现在,黑白二仙看这个瓜熟地洞的规模,简直是大的恐怖。这里的脉轮瓜还有多少只?

一旁的紫山真君适时地解惑道:“你们不用猜了,这里正是宝黄天中最大的脉轮瓜的出产地,几乎霸占了六成市场。”

“当然,我带你们来这里,脉轮瓜不是目的,更不是重点,而是这里的超级蛊阵。”

“这座超级蛊阵中,以律道仙蛊熟为核心,又有七转仙露蛊为辅助。你们在这里锻炼仙道杀招,效率会大大提高,会让你们更加熟练。”

“并且因为有仙露仙蛊,你们置身在蛊阵中,仙元的恢复速度将变得极高。”

黑楼兰、白凝冰听闻此言,俱都流露出喜悦之色。

这样的修炼宝地,简直是对她们俩个量身订造。

黑楼兰如今主修力道,兼修炎道,力道仙蛊稀少,但是炎道仙蛊却多,更有她的大姨妈留下的炎道真传,无数炎道仙级杀招。

白凝冰则刚刚继承了白相真传,同样不弱,仙蛊、仙道杀招一点不少。<strong>最新章节全文阅读WWW.MianHuatang.cc</strong>

如今制约她们两人作战能力的,正是她们刚刚接触继承,没有时间将优秀的真传转化为实实在在的战斗力。

在界壁中,和方源一战,这个弊端在她们俩人的身上暴露得极其明显。

紫山真君苏醒之后,执掌大权,为了提升己方战力,特意将两人带到这里来,利用这片风水宝地,训练黑白二仙。

瓜熟地洞空间宽敞,菇人蛊仙很轻易地就划分出了一大块地方,供白凝冰、黑楼兰修行。

安置好这两位之后,紫山真君便离开了菇人乐土。

时间一天天过去。

方源还在超级蛊阵中,偷偷的探索梦境。

表面上他以闭关苦修为借口,隐瞒了所有人,没有人知道他竟然是探索梦境,到如今收获已是极其巨大。

类比池伤就可明了。

池伤号称阵痴,一百多年寿命,苦心钻研阵道,才有了阵道宗师境界。这份年龄、才情和成就,已经是蛊仙当中稀有。

而方源只是一两个月的时间,攻破了一个蕴藏阵道真意的梦境,便成为了阵道宗师。

正是因为梦境这样的厉害,大大缩短了积累的时间,才使得五域乱战时期,各种蛊仙强者层出不穷,宛若繁星般交相辉映,璀璨万分。

如今,方源不只是阵道宗师,暗道流派也晋升成为宗师。

拥有这样的境界,方源对于暗渡仙蛊的使用,立即就有了全新的想法。

“我现在完全可以用暗渡仙蛊为核心,形成一个仙道杀招,针对自己,防备其他智道蛊仙的推算。结合自身暗道道痕,这个仙道杀招的威能必定十分惊人。”

“我还可以,以暗渡仙蛊为核心,组成仙道杀招,打出暗箭仙蛊的攻伐效果。”

“还有一个想法,就是将暗渡仙蛊增添到其他的仙道杀招中去,应当可以将这些杀招的气息波动都尽数收敛起来。以后催发,能达到让人猝不及防、出人意料的效果。”

其实,对方源而言,他最需要的还是食道、信道两大流派的梦境。

食道可以让他喂养仙蛊更加便捷,成本更低。节省出来的资源,就能推动他修行或者经营资源。

信道就更加重要。

方源现在身上的信道盟约,有很多。

这让他丧失了许多自由,有时候做事,都束手束脚,颇为不便。

还有一点,如果信道境界提升上来,方源就能识破许多信道陷阱和手段,说不定真的能和影宗合作。

但世间的事情,不是你想要就能得到的。

方源镇守的这片地方,就像是一个窗口。当他探索成功之后,有关左夜灰的梦境消散,填充进来的梦境,都是光怪陆离的荒诞类型。

这种类型的梦境,方源可不想随意尝试。即便是有解梦杀招在手。

方源耐心等待。

梦境是不断流转互换的。

但是接二连三都是荒诞梦境,方源只得将精力转移到其他方面。

仙窍的建设经营是主旨,其次便是利用刚刚晋升的暗道境界,尝试推演出一些实用的仙道杀招,最后则是收集外在的情报。

天庭失败,损失八转蛊仙三位之后,一直没有动静。

五域仍旧处于和平稳定的旧景当中。

不过武家的麻烦,始终没有消退。尽管武庸坐镇中枢,不断调兵遣将,化解延缓各类矛盾,但是其他超级家族不断出手,让武家上下疲于应对。

最近又出了一件麻烦的事情。

起因是一张有关浴火仙蛊的仙蛊方,被一个凡人蛊师意外地发掘出来。

这个消息一经传播,立即牵动了姚家、武家两大超级家族的注意力。

这个仙蛊方,是惊艳仙留下的炎道真传中的内容之一。

在一千多年前,武家和姚家就争夺这份炎道真传,结下了仇怨。最终双方死伤颇重,但却都没有得到这份炎道真传。

炎道真传据说被毁,忽然出现的浴火仙蛊方,却是陡然给世人增添了一份全新的希望。

尤其是武家、姚家,这两大超级势力因为历史原因,必须出手争夺这份仙蛊方。就算夺不走,也不能让对方得到。

为此,武庸不惜派遣出了武艺髯,放弃了对资源点血潮天坑的镇守。

武艺髯七转修为,战力出众,他奉命行事,本着速战速决的想法,很快就查明没有什么惊艳仙的真传,只剩下这份浴火仙蛊的仙蛊方。

如此一来,很多蛊仙都纷纷撤离,不愿为了一个仙蛊方,而得罪两大超级势力。

姚家当然不会放手,派遣了重量级人物姚耕。

武艺髯和姚耕四次切磋,两胜两负,只剩下最后一场最关键的比拼。

在这样关键的时刻,紫山真君秘密来到双方切磋地点附近。

一处无名的小山谷中,他见到一位神秘蛊仙。

神秘蛊仙浑身上下一片七彩霞光,遮掩了相貌:“就是你唤我过来?”

紫山真君点头:“不错。不管你是如何继承了完整的七幻真传,就已经缔结了盟约,按照约定,你需要为我做七件事。”

“好,哪七件?”神秘蛊仙答应得非常干脆。

紫山真君嘴角微翘:“首先第一件,便是杀了那武艺髯。”

神秘蛊仙一愣:“你确定?”

紫山真君嘴角的笑容扩大一分,又继续道:“第二位,是取走姚耕的命。”

神秘蛊仙警惕起来:“你想要做什么?”

“我做什么,你不用管。你只需要先去杀了这两人即可。”紫山真君呵呵一笑。

神秘蛊仙冷哼一声,身形猛地消散在原地,无影无踪。

紫山真君的笑容渐渐收敛起来,他背负双手,目光深幽,望着远方的山峦。

“时不我待,时间仓促所剩无多,必须出手了。”他轻声自语道。

几日后,一个震撼的消息传出。

武艺髯和姚耕双双失踪,切磋结果不明。

武家、姚家大为震动,因为从命牌蛊、魂灯蛊等手段中,得到信息,武艺髯、姚耕已经命丧黄泉!

究竟是谁如此狠辣?

武庸震怒,就在他要派遣蛊仙追查此事的时候,血潮天坑喷发,形成万顷血浪,危害一方。

武家连忙派遣蛊仙,前往镇压,没想到竟从现场的痕迹中,查探出仙道杀招崩坏诀的痕迹。

崩坏诀乃是姚耕自创,独有的仙道杀招。现场的痕迹,更是一目了然。

局势顿时复杂起来。

“姚耕究竟有没有死?”

“这是否是姚家的阴谋?”

“除了姚家之外,还有其他的家族出手吗?”

武庸感到不妙,他从中嗅到了一丝危险的气味!(未完待续。)<!--80txt.com-ouoou-->

------------

\end{this_body}

