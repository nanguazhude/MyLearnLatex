\newsection{房家力保}    %第七百八十六节:房家力保

\begin{this_body}

万家大本营。

“爹——!你死的好惨呐!”哭泣声在议事堂中回荡,震得在场的蛊仙耳膜都嗡嗡作响。

一位少年披着丧服,涕泪交加。

这人便是万追青,有着六转蛊仙修为,是万良翰的亲身儿子。

万追青此刻跪倒在地上,一时嚎哭,谁都劝阻不了。

堂中,万家的蛊仙一个个眉头紧锁,脸色阴沉,保持着沉默。

万家太上二长老性情仁和,不禁叹息道:“事情竟至于此!我族出动三位七转强者,都输了,实在叫人难以置信。”

对于此话,万家群仙亦深有同感。

若非是方源主动暴露,在宝黄天中挂起了战斗的全程影像,这些万家群仙可能还要一番猜疑后,才能接受过来。

事实上,万良翰此次的行动,也是得到了万家绝大多数蛊仙的同意,这才发动的。

三位同转的蛊仙,又有仙道战场,原本以为十拿九稳,没想到竟碰上这么一个硬茬子!

一直沉默的太上大长老终于开口:“算不尽杀了我族蛊仙,我万家绝饶不了他。万追青,你父可给你留下了传承?”

见到太上大长老询问,万追青也不敢放肆,收敛哭声道:“不敢隐瞒太上大长老,家父留下了全部传承的内容,种种杀招、蛊方我都熟知。但是……此中的仙蛊可是一只都没有啊!”

太上大长老面色又沉一分。

这种情况很不常见。算不尽是七转修为,万良翰同样如此,但前者却能瞬杀后者,令后者无法反抗。

一般而言,同转的蛊仙交手,很少有这么干脆的战果。一般而言,生死斗都会陷入僵持战。这个时候,若是一方想走,又没有陷入什么仙道战场,另一方很难留下对手。

仙道战场的价值,就体现在这里,让敌人无法逃脱。

万家三仙拿来对付方源,就是让他逃不了。但没想到,方源根本就不逃,反而连仙道战场都没有用,就直接瞬杀了万良翰,把其他两位万家蛊仙打跑。

这结果对于万家而言,真的有些讽刺。

万家太上大长老感到后悔了。

早知如此,他应当将手中的仙蛊屋借给万良翰等人使用。

但太上大长老也为难。

他只有七转修为,并非所有的正道首脑都有八转修为的。正因如此,才需要家族的仙蛊屋来镇守大本营。

他却不知,自己的这番安排其实是合理的,也是正确的。方源若是看到了这座仙蛊屋,定有手段来搜刮。太上大长老的安排反而减少了万家的损失。

太上大长老心中叹息一声,对万追青道:“万良翰之死,是我等决策之误。万良翰将得家族追赠,他的仙蛊需要何种蛊材,开启家族库藏无偿供应。现在我只但愿这些仙蛊已经被毁。”

“谢太上大长老!谢诸位长老!!”万追青连忙拜服在地上,对太上大长老,以及其他万家蛊仙磕头行礼。

这本来就是他哭诉的主要目的,现在达到了。

他年纪最小,这样磕头也不算错。

太上大长老又继续开口,声音阴森,带着凛冽的寒意:“算不尽是他房家的人,施压房家,让房家交出此人。经过之前一战,我万家已可肯定,算不尽就是裂神老人的传人!这是魔道的贼子!当年裂神老人身上的血债,就要从他的身上找回来。”

不少万家蛊仙当即眼冒精芒。

只有这么办了!

当年的裂神老人,已经成为诸多西漠正道势力的隐痛。把这个罪名安在方源的身上,就能让更多的西漠正道势力有了对付房家的借口。

之前,这只是一个借口。

万家是想捉了算不尽,让房家百口莫辩。现在不成了,但万家太上大长老还是决定坐实它。

缺乏证据?

我就自己制造铁证!

虽然很是无耻,但各大超级势力想要对付房家,担心房家崛起的心理完全可以利用。

只是此计的成功概率,远没有之前计划那般高了。

万家的蛊仙战死了,并且还被广而告之。这口气绝不能忍!

这场戏万家硬着头皮,也要演下去!

太上二长老深叹一声:“情况或许没有那么糟糕。说不定房家此刻已经打算放弃算不尽了。”

“哼,就算放弃了,房家窝藏魔道蛊仙的账,也是绝逃不了的。”

然而第二天,就有噩耗传到万家。

房家宣告自身立场,他们将力保方源,并且证明方源的真正来历,是和郑惊神有关。

万家当即否认,拿出种种“铁证”。

这些铁证,房睇长压根都不屑推算其中的破绽,直接甩出更多的“铁证”!

可笑!

你捏造证据,我就不能捏造吗?

一时间,西漠正道的两大势力,围绕着算不尽究竟是郑惊神传人,还是裂神老人的传人的身份,针锋相对,激烈辩驳。

这一边说我理有据!那一边说我才是真实可靠。

这一边刚刚甩出一个证据,那一边毫不示弱,直接甩出两个来。

作为当事人的方源,身处在政治漩涡之中,却在青鬼沙漠连连攻克太古魂兽,悠闲自得,意气风发。

当然,他也时刻关注着局势的发展。

想想看,其实有点搞笑。

方源自己一点都不着急,根本没有拿出什么证据来,但房家、万家却是又激动又积极,有证据拿出来,没有证据捏造证据也要拿出来!

两大超级势力都争先恐后,为一个魔头作证,并且各自还都非常自信,言之凿凿。

若真让他们知道方源的真正身份,不知道会有什么心理变化?

房家肯定是如坠冰窟,但万家也要担惊受怕!

这可是魔头方源!

连天庭,还有整个南疆正道都奈何不了的人物!

方源这些天不仅压服了好几头太古魂兽,而且也抽空将万良翰的智道传承都消化殆尽。

万良翰被他瞬杀,实力差距太大,根本来不及摧毁仙蛊。

方源便得了他三只仙蛊,两只六转,一只七转。

七转的智道仙蛊,名为智障。智障仙蛊有着蜗牛的外形,黑不溜秋,偶尔闪烁一丝紫色光晕,看起来就是笨笨的样子。

这只仙蛊的威能,方源已经在战斗中领略了。

单纯催发智障仙蛊,能够在蛊仙周围巨大的范围内,瞬间形成障碍。消耗的是仙元,还有大量的意志。

以智障仙蛊为核心的杀招,也有不少。

其中有一个杀招,最为优秀,方源也为之侧目。

这个杀招就是将智障仙蛊造成的障碍,直接降临到敌人的脑海中。受到巨大的阻碍,敌人脑海中的念头碰撞严重受阻,思考起来就相当困难。

往往这招成功之后,会立即造成敌人猛地变蠢,脑子里就像是塞了一个巨石,想什么问题都想不通。

这招很优秀,但首先要感应到敌人,并且刺探到敌人的脑海中。

所以,有另外几个杀招,专门是针对敌人脑海的。

可惜的是,方源有着智道造诣,保护脑海的手段很多。万良翰的这些杀招想要攻克这些难关,希望实在太小。

而且他中了方源的杀招杂念丛生,丧失了先手,明明有杀招却不能用,本事直接废掉了一大半。

回顾之前的战斗,方源心底十分满意。

经过了实战的检验,他推算的两记杀招,效果都是上佳。

一招是杂念丛生,它是智道杀招,但也借助了魂道的奥义。在杀招催发的时候,方源就靠着身边的四头太古魂兽,遮掩了杀招的气息。

若是气息暴露,让敌人察觉就不好了。因为杂念丛生此招,是需要一定时间的积累,威能才会增强到客观的程度。

还有一招智取,乃是方源参照了大盗鬼手杀招,还有偷生真传,推算出来的智道杀招。

大盗鬼手是偷取仙蛊,而智取杀招则是偷取性命,瞬杀强敌。

不过此招只有七转程度,对付八转蛊仙效果很差,对付万良翰这种对手恰到好处。为了伪装成大盗鬼手的模样,方源还在智取杀招中增添了一些变化道的蛊虫,加以辅佐。

方源曾经在豆神宫争夺战中,在房功的眼前施展过大盗鬼手,也在房睇长的配合下,收服过太古魂兽。

所以,方源暴露出来的种种手段,并不会让房家震惊,而是展现出自己的价值。

同时,也巧妙地展现出自己的薄弱一面——方源明明有操纵魂兽的手段,却没有动用。

如此一来,就加深了房家对算不尽这个人还在自己掌控中的错觉。接下来的力保和招揽,也就自然而然的事情了。

这场战斗,单纯来看,只是方源和万家三仙的激战。但实际上,却是方源针对眼前局势,可以激化双方矛盾,是对房家、万家这两个庞然大物加以影响,对整个西漠局势施以操纵。

方源真正的目的,在于战斗的背后。把这个计划进行下去,未来的收益将超乎想象。区区智障仙蛊等等,只是大餐前的开胃小菜而已。

不知不觉间,方源已经不再顾及一城一池的得失,而是带着超然。

身在局中,八方纵横。

心在局上,四野俯望。

“不知不觉间,已经走到这种程度了。”

“古往今来,那些引领时代,操纵天下的枭雄英杰,是否就是这样的呢?”

“呵。”

方源收起泛滥的思绪。

夜晚下的青鬼沙漠,更加深幽黑暗。

头顶阴云密布,一丝星光都未透得下来。

方源坐在一只狰狞恐怖的太古魂兽的头上,身边是密密麻麻的魂兽大军。

大军在方源的约束下,寂寞无声,向着前方无尽的黑暗中开赴而去。

记住手机版网址:m.

\end{this_body}

