\newsection{猴获功成}    %第六十七节:猴获功成

\begin{this_body}

%1
一天后,太丘。

%2
“这里……”方源双手牢牢攀住一根高达六七十丈的巨人草,登高望远,望着远处。

%3
一座大树的残骸,像是搁浅在沙滩上的巨舰。

%4
但只剩下半截,乌黑残破,不成样子,似乎是饱受了雷电的轰击。

%5
方源此时的形象,也已经大变。

%6
他变成一只猿猴。

%7
吞火猴。

%8
上古级荒兽。

%9
体型虽然不大,但绝不好惹。

%10
盘山羊的形态,方源早已经不用了。盘山羊无法深入到太丘的深处,这里已经是上古荒兽经常出没的地带。

%11
吞火猴是其中较为特殊的一个。

%12
它实力强大,但只吞食火焰,因此四处游荡。不和其他荒兽抢食,所以是比较合适方源来伪装的对象。

%13
不变成吞火猴,在这里方源几乎寸步难行。

%14
“千蛇阴嬛树……”方源望着远处,低声呢喃。

%15
这已经是太丘地形图上的第三个目标地点了。

%16
第一个地点,现在被一群荒兽黑血狼占据了。第二个地点,已经彻底荒废,成为两大食肉兽群的夹缝地带。

%17
方源现在的位置,已经是最后一个在地形图上,标注出来的目的地。

%18
这里曾经倒下了一株千蛇阴嬛树。

%19
这种树大如山岳,笼罩四野。太古级数,有七万七千七百七十根树枝,宛若柳条,更似长蛇。枝条末端,长有蛇。

%20
这种树的树根,深深地扎在地底,深达数百丈,近千丈。它以过路的荒兽、上古荒兽为食,捕猎时,上万条枝条飞舞,宛若万蟒穿行,缠绕。绞杀缠死之后。汲取血水生存。

%21
长期以往,树下惨死无数生灵,皮肉腐烂,白骨堆山。怨气冲天。??一?看书1·阴气层叠。

%22
阴阳相吸,每到雷雨天气,千蛇阴嬛树都会引得浩荡天雷劈下。

%23
遇到寻常的雷雨还好,但若运气不好,遭遇到非同寻常的气象形成的天雷。那就糟糕了。

%24
千蛇阴嬛树几乎没有天敌,太古荒兽级,霸占一方,或许也是惹来了天意关注,才降下劫电。

%25
总之三十万年前,太丘的这片地带,燃烧起一片火海。这株千蛇阴嬛树,烧得仿若一座正在喷的火山,照亮了半天夜空,长达数月之久。

%26
“不过这株千蛇阴嬛树。还未死啊!”方源的瞳孔深处,闪过一抹凝重。

%27
长毛老祖生前,留下这个太丘地形图,已经是距今三十多万年的事情了。

%28
这株千蛇阴嬛树,苟延残喘了三十多万年,居然未死,还有生命迹象!

%29
“人虽然是万物之灵,但就生命力、寿命、体魄、魂魄等等,都比其他生命差远了。这株千蛇阴嬛树的生命力,就是绝对的强大。被雷劈。被火烧,居然还健在。”方源心底感叹不已。

%30
如今的千蛇阴嬛树,完全倒在地上。粗壮的树干,也腐朽了大半。只剩下数里长的一截了。

%31
完整状态的千蛇阴嬛树,若竖直伫立,堪比高山,枝条笼罩的攻击范围,涵盖百里方圆。

%32
方源敏锐地观察到,这截千蛇阴嬛树的断木上。还有数十条树枝活动着。它们就像是数十条巨蟒,相互纠缠、盘绕,缓缓蠕动。一旦有猎物进入了它们的狩猎范围,就闪电般出击,将猎物杀死。

%33
虽然千蛇阴嬛树已经凄惨如此,但它到底是太古荒植,猎杀上古荒兽、荒兽群都不在话下。

%34
方源观察了一阵,又有新现:“真是祸兮福所伏,完整的千蛇阴嬛树猎杀大量生命,造成树下尸骨如山,引雷劫劈下。但这株千蛇阴嬛树只剩下这么一小截,可谓百不存一。猎杀的猎物有限,反而没有积累出什么阴气怨气,再没有天雷轰击。”

%35
这就是这株千蛇阴嬛树还残留在世的原因。

%36
方源渐渐皱起眉头。

%37
他冒着巨大风险,此行来到太丘,就是为了寻找到合适的地点,可以布置传送蛊阵。壹??看书看?·1?·

%38
太丘地形图的标注,让方源一共有三个目标地点。

%39
但前两个地点已经消失,第三个却也不太合适。

%40
因为千蛇阴嬛树还活着。

%41
这是太古荒植,战斗力惊世骇俗,可战八转蛊仙。即便是八转中最弱的层次,也不是方源能啃下的。对于偌大的琅琊派而言,也是一个长满尖刺的硬骨头。

%42
而且在这里战斗,搞不好会引恐怖兽潮。

%43
“这么说来,我此行任务是成功了一小半,失败了一大半。虽然将太丘地形图完善了一些部分,排除了三个目标地点。但是却没有找到适合琅琊派,布置传送蛊阵的合适地点。”

%44
“没办法。暗渡仙蛊的力量在不断减弱,我还是趁机出去。过段时间,再探太丘。”

%45
方源暗暗叹息一声。

%46
若是这一次能够成功,那就很好了,毕竟方源近况不错,内忧外患还不是当务之急。

%47
但这次不成功,下次的话,方源就没有太多的时间和精力了。

%48
毕竟他可是很忙的。

%49
经营仙窍就是复杂庞大的计划,更何况还有自身的修行,解决仙僵肉身的问题,调教方源等等事情。

%50
不过,没有办法。人活在这个世间,不如意者十之八九。

%51
方源悄悄退去。

%52
他选择一个最近的方向,进行撤离。

%53
但不妙的是,刚刚走了一段路程,方源就现了许多异样。

%54
先是两只上古荒兽在搏斗,声势惊人。然后三只荒兽群不知道为什么,对峙在一起,狂躁不安,大战一触即。

%55
好巧不巧,这三只荒兽群阻挡了方源的去路。

%56
“这一切都是兽潮形成的征兆啊。”

%57
“原来如此。”

%58
“我身上的暗渡仙蛊的护持,已经削弱到这种程度了么?天意虽然不能具体地甄别出我在哪个位置,但大体的范围已经感知到了。所以正在酝酿兽潮,在太丘中掀起狂澜。想要借助这个机会,让我暴露。”

%59
“嗯……对了,我现在身上还有春秋蝉、大量雪怪,这两者身上都有天意。虽然都被局限在我的仙窍世界中,但和外界的天意本是一体,似乎也能相互呼应啊。”

%60
方源眉头越皱越深。

%61
他还是有些低估了天意的威能。

%62
按照道理来讲。任何的仙窍世界,不管福地还是洞天,都是独立的,和外界五域九天毫不相干。天意也管不到这些小世界里。

%63
但现在方源知道了。若是小世界中也有天意,那么和外界天意之间,就能相互呼应、吸引,就好像是里应外合。

%64
靠着天意之间的里应外合,还有暗渡仙蛊的力量衰落。天意虽然找不到方源的具体位置,但酝酿出一场规模浩大的兽潮,要将方源找出来,然后毫不留情地铲除掉!

%65
“实践出真知!或者说,影宗方面给我的天意情报,还是有所保留。此地不可久留!”方源这样想着,立即行动起来。

%66
他伸展猿臂,在巨人草中跳跃,企图绕开前方对峙的兽群,远离这片天意设下的陷阱。

%67
但天不从人愿。

%68
他还是迟了。

%69
那两头上古荒兽打到对峙的兽群当中。立即引了惊天动地的大混战。

%70
混战的余波,又影响辐射周围,混乱引了更多的混乱。

%71
层层递加之后,形成兽潮,将方源席卷进去。

%72
兽潮恐怖。

%73
不管是荒兽,还是上古荒兽,都陷入到极度的狂乱当中。

%74
失去了平时的理智,一切都听从生存的本能在身体中疯狂的咆哮、呐喊。

%75
原先的猎物,竟然敢于攻击猎杀者。许多平时结伴生存的兽群,则分崩离析。秩序不存。

%76
大量的荒兽惊惶奔腾,渐渐形成了一股强势的冲击。这股冲击势力,又夹裹了其他猛兽。就算猛兽不愿意被夹裹,此时也要身不由己。

%77
片刻后。这股冲击势力越壮大,形成一股洪水般不可阻挡的巨大冲势。

%78
席卷一切!任何敢于拦在这股冲势面前,不管是多少荒兽还是多强的上古荒兽,都要当场饮恨。

%79
方源此时此刻,感觉自己就像是惊涛骇浪中的小舢板。

%80
他身不由己,只能被兽潮夹裹。冲向前方。

%81
他必须继续伪装,一旦暴露,天意就会令周围的兽潮将他瞬间淹没。到那时,就算他有无数仙元,大量仙蛊在身,也要惨死。

%82
他毕竟还只是一位六转蛊仙,渡过一次地灾。

%83
虽然方源拥有变形,拥有态度,更拥有见面曾相识,但一味的伪装,也是不行。

%84
暗渡仙蛊的力量在衰落,到了一定程度,方源就要在天意下暴露了!

%85
不能主动暴露,那是提前找死。也不能一味的坚持隐忍,那是在等死。

%86
方源深陷险境之中,一时间找不到什么方法解决。

%87
“或许冒险方有一线生机。”他的心中迅闪过这个念头。

%88
如果没有办法,他只能这样选择。

%89
寄希望于血漂流和剑遁仙蛊。

%90
不过,这可是在太丘深处啊。

%91
几乎到处都是荒兽,上古荒兽也是层出不穷。天意有太多的目标,可以去选择,去影响,然后从容地拦截方源。

%92
就在这时,兽潮忽然改变了方向,原本是直线前行,此时却转了一个轻微的弧线。

%93
“这竟然是?!”方源远望,一双猴眼瞪圆,惊喜交加。

%94
他远远地看到一座山一般的赤红尸骨。骨头上还燃烧着朵朵蓝焰,温度内敛,一丝火热都没有,但却给方源无比危险的感应。

%95
这是一座太古荒兽的尸骸。

%96
似乎是死去不久,太古气息澎湃洋溢,即便是兽潮,也在下意识避让。

%97
“真是踏破铁鞋无觅处,柳暗花明又一村!”这一刻,方源真想大笑三声。

\end{this_body}

