\newsection{如梦令}    %第八百三十二节:如梦令

\begin{this_body}

%1
“这便是如梦令仙蛊了。”方源视察着龙宫。

%2
如梦令形如蜻蜓,圆脑袋,长身躯,一对粉红宝石般的复眼,有四对透明的翼翅。轻盈的翅膀上细细查看,还会发现有着一层淡淡的七彩斑斓的光晕。

%3
方源心中升腾起一股喜悦之情。

%4
长久以来,他就对梦道的仙蛊抱有期待。他虽有不少的梦道杀招,但事实上梦道的仙蛊一直都是零。

%5
就比方说解梦杀招,仙蛊也只是智道流派的解谜而已,方源采用了大量的梦道凡蛊来取代应该有的梦道仙蛊的作用。这也导致了,每一次动用解梦杀招,都会消耗许许多多的梦道凡蛊。

%6
在得到解梦杀招的前期,方源得每次苦心劳累,自己去一只只炼出梦道凡蛊。

%7
若是有一只梦道的仙蛊来替代,就没有这样麻烦的事情了。

%8
方源一直就没有一只梦道仙蛊,终于到了现在,入手了一只,还是高达八转的如梦令!

%9
“回想此次梦境探索,真是惊险,险些让我迷失自我。”一旁的龙人分身感叹不已,此刻仍旧是心有余悸,“若非我提前察觉到了不妥之处,暗中布置解梦杀招,并且延时发动,否则我恐怕已是叛变了。”

%10
对于解梦杀招,方源一直用于梦境,但这一次却是针对他自己。

%11
不过,龙宫之所以能够奴役八转,打造新主,归根结底也是梦境的效用。所以,方源分身拿解梦用在自己身上,也是针锋相对,对症下药。

%12
杀招是死的,人是活的。

%13
蛊虫妙用,存乎一心。

%14
当然,单凭解梦杀招的层数,还不足够抗衡八转层次的如梦令。这点就要感谢容婆、张阴等东海四位蛊仙了。

%15
方源本体微微点头:“此次的确是我探索梦境以来,最为吃力、艰难的一次。不过之前我观察气运,那丝青紫之气作不得假,预示着你能够成为龙宫之主,又不迷失的可能。”

%16
之前察运,正是因为看到了成功的可能,方源这才令分身入梦,进行考验。

%17
若是没有那丝青紫之气,只有失败的结果,方源早就本体出手了。

%18
此时,方源本体和分身早已经沟通完毕,双方都获悉了各自之前的经历。

%19
方源本体点评道:“龙宫梦境,的确艰深复杂,这一次肉身、魂魄都投入梦境,被勾动情绪也再所难免。还有梦中之梦,更会让人沉迷其中,无法自拔。”

%20
“而在外面,还有天庭为难。稍不留意,就会爆发大战,导致场面失控,一发不可收拾。”

%21
别看东海二仙似乎人畜无害的样子,其实绝非什么好货色。上一世,他们为了争夺龙宫,和龙公作战。龙公得胜了之后,中洲炼蛊大会的时候,他们还伙同其他八转一起入侵藏龙窟,从未放弃过龙宫的争抢。

%22
事实上,龙公之所以选择和谈,一方面是方源扮演的气海老祖实力卓绝,另一方面也是有这方面的考虑。

%23
从整个局面上看,方源这一次行动很是危险,犹如悬崖边上走钢丝,稍不留意局势就崩了。

%24
最终有惊无险,让局面一直在把控当中,也是有赖于方源的运筹帷幄。

%25
首先,他探索梦境,经验丰富,乃是世间第一人。此次虽然是分身出手,但方源也交给他纯梦求真变、解梦的手段。

%26
其次,方源组建了煮运锅,可以察运。方源的确是观察到了青紫之气,知道有成功的可能性,所以才会尝试。

%27
第三,方源在光阴长河也派遣了白凝冰等若,万年斗飞车和天庭的五座宙道仙蛊屋纠缠,牵扯住了天庭的注意力。再依靠之前的谋略,方源成功地哄骗了龙公,巧妙地周旋在天庭和东海本土超级势力之间,两处借力。

%28
最后,退一万步讲,就算失败了,方源也只是损失一个分身而已,本体还是安全的,基本盘仍在。若是龙人分身叛变,方源完全可以依靠本体和分身之间的联系,进行追杀。别的不说,单单运气方面的线索,龙宫就无法解除。

%29
因为龙宫是红莲时代的产物,而运道则是在之后由巨阳仙尊开创出来的。

%30
正是有如此种种缘由,方源这一次面对天庭、东海蛊仙、四大龙将,才能面面出力,点点落子,相互照应,形成进可攻,退可守的优势,始终把持住了大局。

%31
当然,风险也是有的。

%32
若是龙人分身损失,方源现如今的梦道手段几乎全部丧失。

%33
毕竟解梦杀招、纯梦求真变杀招都是在分身的手中。

%34
这点风险还是要冒的。

%35
干大事哪会没有风险?况且这一次是关乎龙宫这样的一座八转仙蛊屋!

%36
至于吴帅的意志,那是没有的。

%37
没有一种意志,能够长存在梦境之中,长达百万年。

%38
龙人分身之所以一度成为吴帅,实际上是如梦令的改造,扭曲了他本来的意志,使得龙人分身一度误以为自己就是吴帅,重生回来。

%39
方源的意志,怎么可能这么容易被扭曲?

%40
就算一时蒙蔽,日后也必有所察觉,然后渐渐醒悟过来。除非是龙人分身整天待在龙宫当中,自我囚禁,时刻受到如梦令的影响。

%41
吴帅当初考量和布置,的确苦心孤诣,这份心思值得赞赏。

%42
方源本体知道分身的经历后,也为之动容,称赞一声:“是个人物!”

%43
可惜的是,吴帅失败了。

%44
这并非他手腕不高明。若不高明,怎么可能对抗得了中洲十大派的刁难,将南华岛经营起来?

%45
他的才情天赋也非常强悍。正因如此,才被其父从小栽培,又被绿蚁居士看中,最终成就八转,虽是得到外力帮助,但他自身的能力是最主要的。

%46
遗憾的是,他要面对的敌人势力太大。且不说天庭,单单一个龙公,就足以解决龙人全族。更何况不只是一个龙公,龙公的背后还有星宿。

%47
天意昭示,龙人当兴,这是天道所趋。然而星宿仙尊着实可怕,以身合道,以自身意志掺和天意,将天道混淆。

%48
正是这一手,导致星宿之后的三大魔尊继往开来,也没有摧毁了天庭,直到最后红莲魔尊,才伤害了宿命蛊。饶是这份小小的成果,也已经超出了世人的想象。毕竟红莲魔尊可不是什么天外之魔!

%49
所以,吴帅的失败一开始就注定了。

%50
连魔尊都失败,更何况他吴帅呢?

%51
吴帅能够牺牲,天庭万万千千的成员同样不计牺牲。龙公寂灭了龙人一族,又何尝不是牺牲?星宿仙尊以身合道,更是牺牲了自己。

%52
从吴帅的人生中,方源本体更能感受到天庭的强大!

%53
这个组织的底蕴,实在是太过深厚了。

%54
仙墓深深,不知沉眠了多少八转……

%55
百万年过去,龙公仍旧健在……

%56
元始仙尊、星宿仙尊、元莲仙尊、龙公、紫薇仙子、陈衣、雷鬼真君这些人物都是举足轻重……

%57
如此种种,早已经形成大势,镇压古往今来一切种种。就算是魔尊,也逆反不得。

%58
光是想一想这点,就令人窒息!难怪宋启元、沈从声之流忌惮无比,眼看着地沟频发,五域合一,日渐焦躁。

%59
方源就是要逆反这样的大势!

%60
“只有摧毁了宿命蛊,方能有永生的可能。”

%61
“我若是失败了,是不是将来,也会有后人知道了我的事迹,也为之动容呢?”

%62
“呵呵,也只是动容而已了。”

%63
简单地视察了一番龙宫,方源便将分身和龙宫都收入仙窍,又唤出雪儿、石狮诚、墨坦桑三人。

%64
方源和他们一齐催动上古战阵四通八达,迅速撤离东海。

%65
自从方源吞并了气相洞天之后,翠流珠杀招就显得不实用了。他只得捡起四通八达杀招代步。

%66
上古战阵的确是有着独到的优势,当然方源用万年斗飞车也可以代步,这座八转仙蛊屋的速度也不差。不过此屋现在仍旧在光阴长河作战。

%67
就算方源手头上有,也不想拿出来赶路代步,毕竟太过显眼了些。

\end{this_body}

