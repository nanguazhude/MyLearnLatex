\newsection{琅琊任务}    %第四百四十三节:琅琊任务

\begin{this_body}

荡魂山上,胆识蛊俯拾皆是。(WWW.mianhuatang.CC 好看的小说

方源一路拾取,不断壮大自家魂魄。

他的魂魄底蕴开始节节暴涨,须臾功夫后,从原先的一百万人魂,直接上涨到了五百万人魂级。

一种鼓胀至极,即将要被撑爆的感觉,充盈方源的心头。

方源心知已然达到极限,便走出荡魂山,来到落魄谷。

一番修行之后,他的魂魄得到锤炼,仿佛是钢铁在炉火中不断地烤打磨砺,变得越加精湛。

这一次下来,他的魂魄底蕴稳定在了一百五十万人魂上。

“到了百万人魂级,每一次修行,就是上涨五十万人魂么?”

“再算上期间需要安魂,还有胆识蛊生产的时间,我至少需要半年,才能达到千万人魂的级别。”

方源估算了一下,微微皱起眉头。

这种度,他还是嫌慢。

十年时间,十年之后,就是天庭将宿命仙蛊修复完成。到那时,天庭将占据绝对的优势,方源和影宗上下几乎毫无希望。

十年中要花半年,才能达到千万人魂,之后还有亿人魂。亿人魂之上,还有更高的层级。

方源并不满意这样的度。

“若是我能够让胆识蛊增产,这方面的度就能迅提升。”

现在制约他魂魄底蕴飙升的,就是胆识蛊的产量。

荡魂山毫无疑问是一个聚宝盆,但它需要大量并且优质的魂魄,才能产生源源不断的胆识蛊。

按照当初的约定,这些胆识蛊有一部分是归于琅琊派,琅琊地灵将其贩卖,赚取利润。另一部分则属于方源。

如今,方源将这一部分的胆识蛊,都拿来用于自家的魂道修行。导致他这方面的经济支柱,顿时缺少了一大块。

不过好在之前年蛊生意,大赚了一笔,方源手中还有不少的本钱。

方源算计了一下,和琅琊地灵交涉,改变双方的胆识蛊的占有率,这并不明智。因为方源还要依靠琅琊派,不仅是躲在琅琊福地中安心展,而且还要依赖琅琊派的炼蛊能力,帮助自己炼蛊。

之前第一批炼蛊计划,那三只仙蛊都是带给方源巨大的帮助。现在是第二批炼蛊计划,主要是大量的凡蛊,白莲巨蚕蛊就是其中之一,重点是两只仙蛊。[看本书最新章节请到其中一只名为天机,它是乐土仙尊所创,被影宗的砚石老人获得,从而研出一记仙道杀招,名为石洞天机。

此招的效果,极其玄妙,竟是可以预料出下一次灾劫的具体内容!

琅琊派若是炼出天机仙蛊,便意味着方源可以催动石洞天机,这对于他渡劫有着极大的帮助。

“不过……布任务,让琅琊派炼制仙蛊,一直都在剧烈地消耗着我的门派贡献。这方面不可不查!”

所以这一次魂道修行结束之后,方源便催动信道凡蛊,查看琅琊派的门派任务。

不得不说,当代的琅琊地灵远比上一任,要更加适合经营琅琊派。

在他一系列的政策之下,整个琅琊派充分地挖掘自身潜力,爆出了勃勃生机。

此时,琅琊派的门派任务很多,其中大部分都是方源布的炼蛊任务,然后之下,占据很大比重的,都是一种类型太丘的探索。

琅琊派自从搭建了传送仙阵之后,就致力于探索太丘,挖掘其中的海量自然资源。

这是琅琊派目前的主要展大略。

但是太丘乃是险地,寻常蛊仙都不想涉足的地方,虽然蕴藏资源丰富,但里面猛兽扎堆,怪木丛生。偏偏琅琊派的蛊仙,虽然炼蛊都有一手,但在战斗方面,却是先天不足。

尽管最近一段时期,都在努力的提升,但是真正的战斗能手,并没有几个。

所以,遭遇到很多强大的猛兽或者植株,这些毛民蛊仙就难受了。探索太丘的进展,一直都不顺利。

看着这些类型的任务,方源脑海中念头纷飞,不断碰撞。

几个呼吸的功夫,他就从中选择出了十几个性价比最高的任务来。

下一刻,他将自己选中的任务,全都统统接下!

“太上大长老。”方源通过信道凡蛊,沟通琅琊地灵。

琅琊地灵第一时间知晓了方源的举动,十分欣慰和开怀:“方源,有你出马,解决这些任务,简直是手到擒来!”

但方源却摇头:“这正是我要和你沟通的事情。这些任务,我打算派遣手下,替我完成。我想,这并不违背琅琊派规吧?”

琅琊地灵微微一愣,他当初布置门规的时候,的确没有想到这一层。

细究起来,这方面却也不算是门规漏洞。因为接下任务的蛊仙,自己寻求帮手,又有什么不可以呢?

琅琊地灵只得对方源道:“那你可得保证,你的这些属下不会做危害琅琊派的事情!”

方源便笑:“这是当然的了。”

和琅琊地灵有了很好的沟通,方源便开始派遣影宗成员。

黑楼兰、影无邪、妙音仙子、白兔姑娘,这些战力若是闲置下来,分外浪费。这一下,都得到了很好的利用。

最后,就连白凝冰都参与进来。

因为站在她的角度,她之所以和方源联盟,最大的原因,就是想恢复男身,因此要炼制出仙级的阴阳转身蛊。

但当她现,方源炼蛊都要依赖琅琊派时,她就将主意打到琅琊派的身上。

遗憾的是,琅琊派上下对白凝冰这个陌生人,并不信任。方源也暗中阻挠两方勾连,导致白凝冰只能依靠方源这条线。

她身为方源的盟友,方源并无命令她的权力,但是方源又岂会浪费她这样的宝贵战力?

在方源的特意交流后,白凝冰相信方源:只要她斩杀越多的猛兽植株,她替方源完成的那份任务说得的门派贡献,方源都会将之积攒起来,用于布炼制仙蛊阴阳转身的门派任务。

数天后。

至尊仙窍,小北原。

大片大片的圆形雪花,在空中不断地飞舞飘扬。

这种特别的雪花,纯白晶莹,有成年人的巴掌大小。

这是六转仙材舞遥雪,乃是天地凝练而出,只要温度保持一定的微寒,它就不会融化。并且还会永久地飘扬着,即便是坠落到地面上,也会旋即飞扬上半空中去。

方源最后查看了一下,没有现什么不妥之处,心道:“这些天来,终于是将雪民一族的聘礼都布置下来。这项工作,可以告一段落了。”

雪民一族既然要想和方源联姻,雪儿自然不会两手空空而来。

她带来了一份彩礼。

其中重点,便是雪莲花精。这是雪民一族特有的仙材,宝黄天中异常稀有,方源又得手了不少,全都被他储藏在小北原中。

其次数量最多的,就是舞遥雪了。它正巧是一只六转智道仙蛊的食材,方源收获了这一批,足够喂养四五次了。

六转仙蛊的喂养难度,自然和八转仙蛊有着天壤之别。

这是一个小小的惊喜,至少方源不用再费周折,去往宝黄天收购舞遥雪了。

除此之前,其他的一些资源,就有一些泛善可陈。被方源卖的卖,留的留,处理得干干净净。

“还是太少。若是有更多一点的资源,我在经营仙窍方面,就会有更快的进展了。”

“雪民一族既然想和我联姻,光这些彩礼可是不够的。”

方源心中冷笑。

他将这些彩礼都一口吞下,但对雪民一族的联姻请求,却置若罔闻。

胆识蛊生产出了一些规模,于是方源再次展开魂道修行。

日复一日,他沉浸在修行当中。他的魂魄底蕴,以肉眼可见的度,迅上涨。除了魂道修行之外,他还利用智慧光晕,开始改良杀招。

他掌握的杀招非常非常的多,涉及世间所有流派。他需要改良其中优秀的杀招,能够用自己手中的仙蛊搭配凡蛊,催用起来。这将极大地增长他的手段和战力!

方源埋头修行,影宗等人致力于太丘探索和战斗,毛六、琅琊地灵等人则为方源炼制仙蛊。每个人都有自己的事情,雪儿渐渐成了最着急的人。

这位雪民一族的大美人,带着雪民一族的重大任务前来,结果方源将她的彩礼都吞了,却连她的面都不见一次,一直都把她晾在这里,不闻不问。

雪儿多次求见方源,但她不过是六转蛊仙,方源已有八转战力,双方层次有着差距,又在琅琊福地中,所以方源都一一推拒。

如此一来,雪儿求见不得,心中郁闷难当,每日脸上的忧愁和愤恼都越来越重。

这一天,雪民一族那边又来信催促,雪儿却有了希望。

原来,雪民一族见雪儿毫无进展,根本得不到方源的接见,便私底下商量了一番。

他们达成共识:方源既然收下了彩礼,就有联姻的意思。但为什么不接见雪儿,恐怕是因为彩礼不够份量,匹配不了他堂堂身份。

于是,雪民一族便又筹集了一批,这一次跟着来信,一块带给了雪儿。

“这算个什么事儿?”

“我雪儿乃是族中天之骄女,美名远播。这一次,却要主动和人族联姻,甚至彩礼还要足够。不足够,连面都见不到一次!”

雪儿带着郁闷,来求见方源。

方源总算是接见了她。

但孰料,方源却对她说了一番出乎她意料之外的话。(未完待续。)8<!--80txt.com-ouoou-->

------------

\end{this_body}

