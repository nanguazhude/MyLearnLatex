\newsection{龙公招揽方源}    %第八百二十二节:龙公招揽方源

\begin{this_body}

%1
龙公中了杀招偷生!

%2
但方源却是面色微沉,因为龙公毫无变化。

%3
之前,方源在琅琊福地伏击雷鬼真君,偷生效果立竿见影。上一世琅琊地灵运用偷生仙蛊,也是重伤紫薇仙子、陈衣、凤九歌,杀死了雷鬼真君。

%4
然而这一次,用来对付龙公,方源寄予厚望的偷生杀招,居然没有效果。这实在出乎他的意料。

%5
不过很快,他就恢复过来,神态自如:“这当然是方源所赠,不过看起来,对你却是无用。这是为何?”

%6
龙公淡淡一笑:“老夫曾催使过杀招龙御上宾,自身的寿元会不断衰减,从而令实力不断上涨。这就像是过度运用了其他增寿之法的蛊仙,再用寿蛊毫无效果,龙御上宾杀招也使得其他种种增寿手段,毫无用处。”

%7
“而偷生杀招,无非也是对代表寿命的相应道痕施加影响。这或许就是它失效的缘由了。”

%8
“原来如此。”方源点头,忽然展颜一笑,“这等小把戏对你龙公无效,但对其他天庭诸仙应该还是效果不错的。我且留着,以待将来。”

%9
龙公面色顿时一沉,他担忧的正是这个。

%10
偷生杀招一出,虽然对他无效,但是对于其他大多数的天庭成员,却是仍旧有效的。

%11
眼下是方源伏击龙公失败,因为沈从声、宋启元意外搅局。龙公心中清楚:这大概就是秦鼎菱为他转运,同时又有方正的引导所致。

%12
然而,龙公想要摧毁偷生杀招和仙蛊,也是万分困难的。

%13
因为他已是和方源交手多次,十分明白:这位气海老祖实力卓绝,绝不能把他看做寻常的八转巅峰的蛊仙。自己要击败他是可能的,但是要在这一次斩杀掉他,却很不容易。

%14
因为来不及铺设仙道战场,拘束不了气海老祖。同时在战场边缘,还有两位东海八转。

%15
所以,龙公中了一记偷生杀招之后,就再也没有动手。

%16
他和颜悦色地对方源道:“气海老祖,你我之间是不打不上相识。你奈何不了我,我也暂时奈何不了你。其实我们之间,何必打生打死呢?”

%17
方源心头顿时一动:“咦,这龙公是想要罢战吗?”

%18
他智道造诣十分深厚,转念一想,就明白了龙公的心理,顿时就有些暗自佩服。

%19
“这龙公被我埋伏,之前交手,也是多数挨打。他如此资辈,又有如此实力,位高权重,却居然毫无傲气,心中始终冷静,关注大局。一旦他明白罢战似乎更有利,就立即停手。这份心性端的恐怖。”

%20
心中这样想着,方源嘴上去道:“哼,这还不是天庭平白无故地来惹我。老夫与世无争,一心潜修,而你天庭却是狼子野心,意欲吞并五域,一统天下。老夫这样的存在自然就是你们最大的阻碍,让你们先是动用阴谋诡计暗算老夫,随后又纠集许多八转,针对老夫进行围攻。”

%21
龙公听了这话,心中郁郁。

%22
这让他感到很冤枉,因为过去、现在天庭都未做过这种事情。他本人还是第一次见到气海老祖呢!

%23
但是要让他反驳,他也无法反驳。

%24
因为很可能,在未来天庭的确会这样做。

%25
“我们且不说这个事情是否是方源捏造的。”龙公索性点头,“就算这件事情是真的,那也是上一世的事情,放在光阴长河当中,也是在此刻的下游,还未发生过。我们既然都已经知晓,那为什么不能改变呢?”

%26
方源嘲讽地笑了声:“哈哈,我没有听错吧?你们天庭不是一直标榜自己乃是正道,维护天意,千方百计地想要修复宿命吗?你居然想要改变未来,不正是要改变宿命的轨迹吗?”

%27
方源故作不知,此举十分狡诈。

%28
龙公微笑:“天庭的确实在维护天道,修复宿命。但这天道是人道当兴的天道,这宿命是维护人族的宿命蛊。气海老祖,你我之间其实并没有什么大的矛盾,完全可以和平共处。我代表天庭,可以和你结盟,互不侵犯!”

%29
龙公郑重其事,诚意满满。

%30
战场边缘,东海两位八转蛊仙却是闻声色变。

%31
沈从声暗叫糟糕,宋启元同样脸色不好。

%32
本来他们见到东海竟有气海老祖这样的人物,可以和龙公单打独斗,而不落下风,都是心中欢喜。

%33
在这五域当中,东海资源最为丰富,但东海蛊仙并不团结,战力水准也参差不齐。

%34
眼下地脉频动,形成一条条全新地沟,五域界壁不断削减,五域合并为一,已经不是什么秘密。

%35
一旦五域真正合一,那么四域都要面临中洲的强烈威胁。

%36
宋启元、沈从声二人都是太上大长老,八转蛊仙,身居高位多年,自然是看到未来的种种隐忧。

%37
在这种情况下,若是气海老祖和天庭不对付,那么对于东海的超级势力是非常有利的。

%38
首先,气海老祖战力超绝,和龙公几乎不相上下,将来天庭入侵东海,必然就有巨大的顾忌。

%39
其次,气海老祖乃是隐修,潜藏了这么多年都不露面,证明他心中毫无野心,又是孤家寡人一个,完全可以和东海本土的超级势力和平共存。

%40
最后,气海老祖明显修行气道,气道早就式微很多年了,东海超级势力哪里会有主修气道的?所以,在这修行资源方面,也不存在矛盾。

%41
“和天庭结盟?”方源故作一愣,也收住手脚,浑身气势按捺不发。

%42
龙公精神一振:“是的,气海仙友,你我两方之间的矛盾,完全可以调解。对于仙友将来可能有的损失,我天庭完全可以补偿。”

%43
龙公到底是做大事,态度很干脆,胸怀广阔,十分大器。

%44
“不可!”沈从声忍不住喊道,直接疾飞过来。

%45
宋启元紧随其后,也道:“天庭家大业大,所谓的补偿,对于天庭而言不过九牛一毛。然则天庭狼子野心,只是要暂时稳住局面,前辈不可不察。”

%46
“是啊,前辈!之前我二人鲁莽,坏了前辈之事,现在都懊悔万分。”沈从声又道。

%47
“再者,所谓的盟约都是放屁,任何的盟约只要给予充分的时间,付出足够的代价,就能够违背。前辈乃是气道,不擅长信道,但天庭却不是这样。天庭绝不可信啊,前辈。”宋启元也附和出声。

%48
他们俩完全以晚辈自居,皆因见到方源实力高超。

%49
方源哪里会答应!他之所以这样做,只不过是故意拖延时间而已。结果东海二仙比他还急躁。

%50
方源暗暗一笑,只是心念一转,就将这两人的心思洞悉彻底。

%51
龙公点头:“的确,到了我们这种层次,盟约束缚不了,不过,气海仙友你不是还有偷生杀招吗?仙友当知此中价值,有了偷生杀招,就等若握住我天庭的把柄。这才是你我双方结盟的基石。我方有此顾忌,怎可能轻易违背盟约呢?”

%52
此言一出,方源不禁深深打量龙公。上一世,方源领略到了龙公的恐怖战力,没想到他也如此精明,口才决不下一般的智道蛊仙。

%53
按照龙公的认知,偷生仙蛊以及杀招都是方源赠给气海老祖,用来对付他的。但现在这些东西,反而被龙公利用,用来劝和气海老祖。

%54
方源沉吟不语。

%55
他当然知道龙公的打算。龙公一定觉得:气海老祖不是天外之魔,只要宿命蛊修复成功,这样的人物根本就没有威胁。甚至还可以招揽,反成为天庭的打手。这就是门派制度的优越,换做家族势力,怎么可能?甚至,一般的超级门派也无法招揽气海老祖。气海老祖这种层次,和龙公相差不多,唯有天庭这样的人道第一势力,才有这样的底气来招揽。

%56
东海二仙着急得不得了。

%57
他们旋即开口,提到宿命,指出:宿命蛊若修复成功,天意时刻影响之下,天庭将立于不败之地。天庭所图甚大,野心勃勃,前辈您前往不能被蒙蔽了。

%58
这两位东海八转也不是好相与的。

%59
方源暗笑,有了这两个蛊仙,倒省了自己好一番口舌,他索性一言不发,直接看向龙公。

%60
龙公冷瞥东海二仙一眼,施加压力。

%61
沈从声、宋启元毫无畏惧,和他对视。

%62
虽然明知道己方二人不会是龙公的对手,但上一世他们敢争夺龙宫,这一世自然也敢针锋相对。

%63
这无关勇气,而是利益。

%64
气绝老祖的位置,牵扯极大,该站出来还是站出来,东海二仙没有缩卵。

%65
沈从声冷笑:“怎么,龙公大人,还不允许我们说了?是我们说到了你的痛脚?”

%66
宋启元接着道:“前辈您看看,这就是天庭的作风,真是霸道。”

%67
龙公嗤笑一声:“若没有我天庭,怎么会有人族的今天。天意就算影响,能影响到哪里去?难道还能教仙友直接送死不成?若是天意有这样的本事,我天庭早就统一五域了,何须还要等到现在?”

%68
“既然话说到这个份上,那我不妨更进一步。气海仙友你真的不如加入天庭!你独修这么多年,能够有如此实力,的确是天资决定。但你更应当清楚散修的苦楚。别的不说,就说渡劫。试问,这世间能帮你的人有多少呢?纵观东海,何人是你对手?你指望这些人来帮你的忙吗?事实上,这两人才刚刚坏了仙友的气道战场呢。”

%69
“仙友若是加入天庭,我就可以帮助仙友渡劫。就算没有我,天庭仙墓中还沉眠着无数的八转蛊仙。我天庭还有智道大能紫薇仙子,掌握星宿棋盘,帮助仙友推算。气道早已式微,天庭中却收藏着元始仙尊的传承。仙友若是将来有杰出贡献,获得元始仙尊也不是不可能啊!”

\end{this_body}

