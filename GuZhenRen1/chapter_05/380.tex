\newsection{年兽军团}    %第三百八十节:年兽军团

\begin{this_body}

沙流漩涡中心的深邃同道,似乎直插向地底。

凤九歌越是深入,越感受到浓郁的宙道气息。

忽然,从深沉的黑暗中,闪烁出一个亮点。

在黑暗的映衬下,这个亮点分外显眼,它闪烁着黄铜一般的光泽。

“是一只野生的凡蛊?”凤九歌飞近,发现这是一只三转的日蛊。

日蛊形如贝壳,有婴孩的手掌大小,圆形的贝壳两侧,还有两个鱼一样的鳍翅。鱼翅不断扇动,带着贝壳般的小巧身躯,在黑暗中游荡。

和年兽相比较起来,日蛊就很常见了,在五域两天的任何地点,都有可能被发现。

让凤九歌在意的是,光阴长河中本就充斥着海量的日蛊、月蛊、年蛊。

这里有野生的日蛊,会不会如凤九歌猜测的那样,藏着一条光阴支流?

光阴支流,是常见却又罕见的修行资源。

说是常见,是因为蛊仙的仙窍中,都有一条光阴支流,使得仙窍世界中,有了时间的流转。

说罕见,则是在五域两天中,自然形成的光阴支流,数量稀少,但规模巨大,远超仙窍中的光阴支流。一旦出现,往往会惹来超级势力间的大力争夺。若是散仙发觉,常常会尽全力隐藏遮掩,自己占据,闷声发大财。

宙道的蛊虫,不仅是宙道流派的蛊师、蛊仙需要,对于炼蛊而言,都有广泛运用。

就比如这只日蛊,三转的日蛊,蕴藏三天的时间。

如果有相应的炼道的手段,耗用这只三转日蛊的话,就能炼蛊时间直接缩减三天。

还有宙道的延寿法门,利用日蛊、月蛊、年蛊等等为人延寿。当然这种延寿之法,留有弊端,在身上印刻宙道道痕。这种道痕多了,将来运用寿蛊都不可行。

但寿蛊难寻,若是手头上没有寿蛊,寿命又到了极限,难道这类法门还能不用吗?

“这里若是有一条光阴支流,必然是影宗发觉,偷偷隐藏下来的。”凤九歌继续深入。

很快,他就看到越来越多的野生宙道蛊虫。

日蛊、月蛊、年蛊等等。

一片苍白的斑斓,出现在凤九歌的前进路线上。

这片光斑,有床般大小,像是一片水光在静静地波动着,幽若无息。

凤九歌谨慎地放慢了脚步。

这是光阴斑斓。

一种气象之变化。

就好像是沙漠中起风,群山中升雾,天空中下雨一样。

光阴斑斓也是一种自然气象。

任何的生命,只要进入光阴斑斓之中,就会受到影响。时间会变快,加速衰老。或者变慢,世界在其眼中,会发展“飞快”。

不管是快慢,都会多多少少削减寿命。

凤九歌都不敢轻易尝试,他放慢脚步,轻轻地绕了过去。

远离这片光阴斑斓之后,凤九歌吐出一口浊气,心中越加肯定,这里恐怕真的有一条光阴支流。

天庭蛊仙关照他的话,已经明确指出,红莲真传就隐藏在石莲岛中,方源要继承的话,就得身入光阴长河。而进入光阴长河,最稳定可靠的方法,就是寻找到一条光阴支流。

这条光阴支流,就相当于入口,必须足够规模,让他能够轻松出入。所以蛊仙仙窍中的光阴支流,都不够格。

只有存在于五域两天中的光阴支流,时间流速和光阴长河本身几乎统一,才能容纳得下蛊仙的进出。

越来越多的光阴斑斓,开始出现在凤九歌的视野当中。

越是深入,黑幽的通道,就越加扩大。

起先只是一两只宙道凡蛊,然后断断续续数量增多,到了最深处,出现了一只只的蛊群,每一支蛊群的数量都上百。

宙道的气息越加浓郁,凤九歌甚至听到了水流的哗哗响声。

终于,在他的视野中,出现了一道修长的光线。

随着凤九歌拉近距离,这道光线也随着在他的视野中扩大,渐渐变成了一条河流模样。

“光阴支流!”凤九歌心头微微一震,这里果然有着一条光阴支流。

“但是方源他们在哪里?”凤九歌刚转过这个念头,忽然从附近的一片光阴斑斓当中,猛地跃出一个庞大的身影,直接向他扑来。

凤九歌连忙躲闪,眼底深处精芒一闪即逝。

一记仙道杀招,在瞬间催动起来。

音道——此时无声胜有声!

凤九歌专修音道,矢志创造出九首歌曲,唱尽天地万物。如今他已经开创了七首歌,离歌正是其中一首。

但是除了这七首歌曲之外,他亦有其他的音道手段。

音道杀招此时无声胜有声,便是其中之一。

凤九歌勤加锤炼,熟得不能再熟,此时一经催动,顿时成功。

他拳掌频出,对准黑影不断遥击。

这亦是杀招手段,拳为鼓拳,掌为钟掌,乃是灵缘斋的一份三音真传中的内容。

三音真传当中,除了鼓拳、钟掌之外,还有一记杀招,名为哨指。凤九歌却未学习,他只是拣了感兴趣的练习,补充自己手段的不足。

那巨大的阴影怪兽,被凤九歌打的嗷嗷直叫。

但诡异的是,它的叫声刚刚喊出口,就消散全无。

凤九歌的鼓拳,每一击都能爆发出咚咚鼓音,钟掌每一击都能打出铛铛钟鸣。但是在此刻,却都悄无声息。

很快,凤九歌就占据了上风,攻势越来越猛,那头偷袭的怪兽被打得蒙圈,越发没有还手之力。

凤九歌也看清楚了,偷袭他的是一头年兽。

猴形年兽。

而且只是普通荒兽,不是上古荒级。若是上古年兽,就不会这么容易,被他压下去了。

凤九歌眯起双眼。

光阴长河中,不仅有野生的宙道蛊虫,还有大量的野兽、植株。年兽在五域外界很罕见,但是在光阴长河中,却很平常。

只是……

这头荒级年兽,是方源等人的安排,还是纯粹的野兽年兽呢?

凤九歌脑海中思考着这个问题。

这个问题的结果,代表着方源是否已经发现了他的踪迹。

“年兽……”方源一边观战,一边将神念沟通宝黄天。

宝黄天中,居然有人公开贩卖大量的年兽。这种买卖,可是相当罕见的。

方源不禁怦然心动。

这些年兽,不是荒兽就是上古荒兽。他如果能够收购过来,并且驾驭住,将是一股巨大的力量!

别忘了他掌握着百八十奴仙道杀招。

到时候,借助这些年兽和百八十奴杀招,方源就能奴役太古年兽!

上极天鹰虽然失去了,但是掌握太古年兽的话,也能为方源这一方增添一个八转战力了。

但是让方源感到麻烦的是,卖方却是一心想做大买卖。

方源当然也想一口吞下这些年兽,然而有意者,并非他一人。

他最大的劣势在于,自己并没有多少的资金,要想买下这么多的年兽,最好将之前狂购下来的荒鹰、上古荒鹰卖掉。

当然,方源也可以卖掉仙蛊。

但不到万不得已,他不想这么做。

之前在梦境大战的时候,方源只有这么做,才能存活下来。现在却并非生死攸关的时刻。

千变老祖正做抉择。

他接收到了很多的购价,但是真正让他心动的,只有三位。

一位直接出卖仙元石,海量的仙元石,难以想象居然有人手中,囤积了如此庞大规模的仙元石。即便是千变老祖都很吃惊。

第二位则是可以付出许多的仙材,种类五花八门,让千变老祖心动的是,这些仙材当中,有很多变化道仙材。

第三位也很有诚意,想用同等的鹰兽来换取年兽。千变老祖目前正修行一种飞鹰变化,所以这个买家也很适合他。

就在千变老祖犹豫不决的时候,这三方卖家中的一位,忽然开出了另外的高价。

“居然愿意付出仙蛊?!”千变老祖又惊又喜,当即不再迟疑,答应了这笔买卖。

所有的年兽,都卖给了一位蛊仙。

交易很快完成,不管是买家的雄厚资本和大胃口,还是千变老祖的决定速度,都让人吃惊。

西漠,沙流通道当中。

吱吱吱!

猴形年兽发出惨烈的尖叫声。

但是叫声,在凤九歌的杀招之下,立即化为虚无。

在这场寂静的杀戮中,猴形年兽终于不敌凤九歌,惨遭屠戮。

凤九歌查看一番后,确认这头猴形年兽乃是野生荒兽,便将其当做战利品,收入自家的仙窍当中去。

整场战斗,只是持续了一小会儿。

凤九歌很显然没有动用真正的本事,消灭一头荒级年兽,显得轻轻松松。

他没有休整,继续朝着光阴支流进发。

距离越来越近,这条光阴支流也在他的眼中,越加清晰。

河水流逝,波光粼粼,在一片黑暗中静静地流淌。

吼吼!

忽然,从光阴支流中,飞出好几头的年兽出来,一齐扑向凤九歌。

凤九歌微微一惊。

“被发现了吗?”刚刚那一瞬间,他感受到了明显的仙道杀招的气息。

功过来的年兽阵容很强大,荒级年兽五头,上古年兽则有三头!

凤九歌冷哼一声,澎湃的战意几乎漫溢而出。

既然被发现了,那他就不用束手束脚地作战了。

“出来吧,方源。”凤九歌轻喝一声。

但下一刻,他的瞳孔微微一缩。

因为还有年兽,正从光阴支流中奔腾而出。一下子,年兽的数量暴涨到了十几头。

并且,更多的年兽还在源源不断地,从光阴支流中杀将出来!

“怎么回事?有一支年兽群,刚好在这条光阴支流附近吗?”(未完待续。)

\end{this_body}

