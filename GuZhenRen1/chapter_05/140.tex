\newsection{试用道可道仙蛊}    %第一百四十节:试用道可道仙蛊

\begin{this_body}

道可道仙蛊!

这只仙蛊头尾圆润,通体漆黑,分有多节,一被方源催动起来,便开始绽射着微微星芒。[看本书最新章节请到

湛蓝的星光,映照在方源的全身。

方源屏气凝神,细心观察,全神戒备,以防任何不测。

蛊师用蛊,并不安全。很多蛊虫,都有后遗症或者弊端。

仙蛊同样如此,甚至因为极端强大,后遗症、弊端也随之更大。

这尚是方源第一次运用道可道仙蛊,虽然事先用过智道手段进行了推算,但多加小心总不为过。

起初,星光只是覆盖方源全身。但几个呼吸之后,星光积累起来,像是一层薄薄的流水,覆盖在方源浑身上下。

星辉凝聚成的流光,越积越厚,积蓄在方源的面庞、四肢、躯干之上。

然后,随着时间推移,流光又渐渐减少,十几个呼吸之后,逐渐减缓至无。

道可道仙蛊周身的星芒,消失无踪,变成漆黑无光的状态。

但是方源的脑海中,却多出一股信息,让他知道了自己身上的道痕多寡。

雪道、冰道的道痕,数量最多,竟有一万六千余!

方源乍然得知,心头也是微微一跳。

一百道痕,就能增幅相应的仙蛊效果,多达一成。一千道痕,便能增幅效果达到原有的一倍。那么一万六千冰雪道痕,自然就是多了十六倍数的增幅效果!

这也就意味着,方源运用某个冰雪流派的仙蛊,效果将十分拔群。假设一位蛊仙,浑身并无道痕,单纯催动一只冰雪仙蛊,那么方源催动的威能,是这位蛊仙整整十七倍!

最关键的一点在于,威能是原先单纯的十七倍,但方源消耗的青提仙元,仍旧是和其他蛊仙一模一样。

不像市井。虽然仙道杀招的威力膨胀了数十倍,乃至上百倍,方源消耗的青提仙元也随之上涨,消耗暴涨到了数十倍、上百倍!

这点就非常可怕了。

蛊仙为什么修为越高。战斗力之间的差距就越大,正是因为如此。

冰雪道痕之下,最多的道痕,竟然是运道道痕,高达一万五千余!

“我的运道道痕。基本上都是吸收的冯军身上的积累。此人的底蕴强大,是亲自闯荡继仙山成功,从传承中直接获得了大量运道道痕加身。”

“哦,对了,有一次地灾,生成巨祸焚木,渡过之后,也为我增加了运道道痕。”

居于运道道痕之下的,是气道道痕和音道道痕。

两者皆有一万三千多。[看本书最新章节请到棉花糖小说网www.mianhuatang.cc]

“气道道痕主要来源于戚灾,他是七转蛊仙。主修气道,一身气道道痕几乎都落到我身上,被我汲取。”

“音道道痕来自汤诵,此人也是七转蛊仙,而且是东海超级势力汤家的太上长老。”

方源思考了一下,发现接下来的道痕,数量上万的都是来自七转蛊仙的尸躯。

其中,律道道痕有一万二的样子,来自黑凡洞天中的律道蛊仙陈尺。

水道道痕在一万一左右,是方源吸收了散仙周礼的水道底蕴。

再下面。就是变化道、力道的道痕,总和加起来,有九千余,接近一万。

其余流派的道痕。杂七杂八。

多的,例如风道、木道,还有食道。

“我渡过的灾劫中,有地灾风花、春晓翠鹂,所以有风道、木道的道痕数额增长。食道居然比这两者还多,看来玄白飞盐劫。增长的就是食道道痕了。”

少一些的,有土道、宙道、信道。

“之前加入异族大联盟,他们用的手段就是土道手段,达到信道效果。所以土道道痕增添了许多。”

“至于宙道,应当是我和楚度定下的百年好合吧,情况和异族联盟类似。”

“而这些信道道痕,应当就是加入琅琊派的盟约所致了。”

最少的,比如虚道,还是一百的数量。这是至尊仙胎自带的道痕基础。

方源根据这些数据,进行思考,顺势洞察到了一些灾劫的本质。

他总结了一下。

最多的是冰雪道痕,下面依次是运道、气道、音道、律道、水道,都有一万以上。

之下是变化道、力道。

再下面则是食道、风道、木道、土道、宙道等等。

“我作战时主要是力道、剑道、宙道手段。见面曾相识、血本仙蛊这些,只能算是补充。”

“四大宗师境界,则分别是智道、力道、血道、变化道。”

方源对照了一下自己的具体情况,不免有点尴尬。

道痕超出一万的流派,和他的主要手段和四大境界,几乎都是岔开的。

按照方源原先的修行计划,重点放在变化道上。

但现在,因为加入了五族大联盟,这个计划执行方面就有了不小的阻碍。无法在北部冰原渡劫,方源就没法抽取狂蛮真意。影响不了地灾,变化道痕的收获就没有了,更别提变化境界的上涨。

方源的思维继续发散。

“以此看来,一般的七转蛊仙,身上的主修道痕底蕴,几乎都在一万以上。”

“但我只是渡劫四场,却有冰雪道痕一万六千多,变化道、力道道痕接近一万。再加上木道、风道、食道道痕数量。渡劫的总体收益,就有三万多。”

很显然,这个数字十分不正常。

蛊仙修为达到七转,平均要渡地灾四五十次,天劫四五次,浩劫在一次到三次之间。

一路下来,增添的主修道痕,才一万多。

方源才仅仅渡过地灾四场,道痕收益就超过了他们的三倍!

灾劫的威力越强,渡劫之后,蛊仙收获的道痕就越多。

普通的蛊仙,一路渡劫,较为顺利。修行到七转后,主修道痕是一万以上。

资质更高一些,拥有上等福地,甚至十绝仙窍。或者说福地中资源极多,福分极强,惹来更强灾劫,渡过之后得到的道痕也会更多。

方源的情况。世间只此一例!

至尊仙窍就算没有资源在里面,本身就是巨大福分,惹来的地灾威力要提升无数档次。

再加上完整的天外之魔,天意千方百计地要打杀方源,导致他每一次地灾都是提升到了天道运转的极限程度。

前四次地灾。不管哪一次,方源渡过去都是十分惊险,仿佛悬崖上走钢丝。不过现在看来,风险虽大,但收获也绝对不小!

“尽管和我现在的路数、手段不相符合,但道痕是实实在在的。照这样下去,一次次地灾、天劫积累过来,我的底蕴将提升到一种恐怖的境地!”

这么一想,方源忽然觉得之前他品尝的艰辛,付出的血汗都完全值得。

就连对待地灾的态度。方源隐隐有点发生了变化。

每一次地灾,都可看做是道痕极大增长的良机!

“再渡几次地灾,我能成长到什么地步?底蕴能暴涨到什么程度?”就连方源自己,都估算不到以后的情景。

至尊仙窍,实在是超过寻常太多太多。

成长的潜力和速度,绝对惊世骇俗!

方源渡过四次地灾的道痕收获,就是普通七转蛊仙一身主修道痕的数倍!

探查清楚自己身上的道痕之后,方源又取出其他仙蛊,利用道可道分别对其侦查。

但可惜的是,虽然有星光流水覆盖在各种仙蛊的表面。但是方源却得不到任何结果。

他心中了然:“人是万物之灵,蛊是天地真精。凡蛊有道痕,仙蛊则是大道碎片。而道可道只能侦查道痕,对于大道碎片是无能为力的么。”

果然。接下来他用道可道仙蛊,针对一些凡蛊进行侦查,皆是无往不利。

这还远远不是结束。

方源接下来,又取出各种仙材,运用道可道仙蛊依次察看。

他发现,虽然仙材身上的道痕。都能被他探知清楚,但消耗的仙元却随着仙材的品级而升高。

“道可道仙蛊是信道仙蛊,侦查纯粹的信道仙材,仙元消耗要少。除此之外,察看任何一种其他仙材,都要考虑到不同道痕相互掣肘的因素。所以,仙元才耗费更多。”

“至于用在我身上,为什么不会有仙元更多损耗。那定是因为我身上的道痕,从不相互内耗罢。”

念及于此,方源的脸上闪过一抹凝重之色。

至尊仙体上的道痕,不相互掣肘内耗,这对于方源而言,既是好事,也是坏事。

对于通常蛊仙而言,异种流派的杀招打中自己,首先会被自己一身道痕抵消许多威力。但方源若是中了,却是实实在在,百分之百,甚至会因为自身道痕中有一部分和杀招流派相同,还会增长威能!

单这一点,方源就和近战基本上无缘了。

他适合的是远战,距离拉远之后,才能有充分的时间进行思考和闪躲。

“道可道仙蛊,侦查仙蛊没有效果,侦查蛊仙、仙材都要考虑异种道痕的抵消。总体而言,算是一个鸡肋吧。”方源一番试用下来,在心中做出了评估。

他暂时能想到的,有三个能利用道可道仙蛊的地方。

一个是辨别仙材,防止别人弄虚作假。

第二个是侦查蛊仙的道痕,推算他的底蕴和实力。不过很不实用,需要时间并不短,而且星光流水的效果未免太过明显,容易被他人针对。

第三个就是炼制仙蛊。

道可道的侦查效果,能够让蛊仙在炼制过程中,十分有效地查看到每一个步骤的道痕种类,以及数额对比。

不过道可道仙蛊,无法查看到道痕的具体位置,更没有直接影响道痕的效果,所以提供的帮助其实很有限。

顶多,让蛊仙从“盲人”变成一个“睁眼瞎子”罢了。

ps:这些数据让我头疼,用计算器算了好多遍,终于有个较为合理的数据了。希望不会再成为bug。因为和之前的设定有冲突,我至少花了大半个小时,在算这玩意。如果还有错误,希望大家多多指正。这书写了370多万字了,有bug在所难免,但我会拼尽全力,尽量减少bug!谢谢大家支持!(未完待续。)<!--80txt.com-ouoou-->

\end{this_body}

