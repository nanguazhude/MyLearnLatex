\newsection{欲取信道传承}    %第一百二十八节:欲取信道传承

\begin{this_body}

%1
大风迎面吹来,夹裹着海水的咸味。

%2
方源端坐在一朵祥云之上,一边赶路,一边分出心神,探入手中的信道凡蛊之中。

%3
这凡蛊,来源于冰卓。

%4
那位手持双枪,擅长近战的雪人蛊仙,在信中告知方源:泪冰对于雪人一族的意义,绝非单纯的蛊材那么简单,而是表达爱意的定情信物。

%5
“哼。”方源冷笑一声,随手将这只信道凡蛊捏碎。

%6
洁白的祥云载着他一路飞驰。风吹得他一头长飘扬,白皙如玉的面庞,英俊至极,近乎姣丽,深邃的黑眸俯瞰云下的深蓝海面,透射出一缕缕的漠然冷芒。

%7
泪冰对于雪人的意义,方源又岂会不知?

%8
别忘了,他可是有五百年重生过来。前世修为不算多高,但见识绝对丰富。

%9
不仅这泪冰,方源对于异人的酒文化,也知之颇深。

%10
曾经在酒宴上,毛六指责方源不知道异人饮酒的风俗。事实上,方源只是装作不知道,借机试探雪人、石人一方的态度而已。

%11
有时候,装装糊涂,更能令自己进退自如。

%12
泪冰一事,也是如此。方源故作不知,其实心中知道得一清二楚。

%13
接受泪冰,就是接受雪儿的情义。但若不接受,势必破坏自己和雪人、石人联合部族的关系。

%14
和雪儿结合?

%15
此事弊大于利!

%16
雪人部族图谋的,是方源的八转战力太古上极天鹰。

%17
且不说方源能从雪人一族中,获得多少利益,单这一点,方源就满足不了对方。

%18
上极天鹰现在已经成了鸟蛋。

%19
方源虽然花销了部分琅琊门派贡献,换取了一些天晶,但储量远远不够。孵化上极天鹰并不适宜。

%20
“和雪儿成婚,就等若将自己捆绑在了雪人一族的战车上。必定还要定下婚约。我现在身上,有和琅琊派的门约,有和楚度的盟约。有四大异族联合的条约,已经极大地限制了我的自由。再加上婚约,岂不是自讨苦吃?”

%21
方源从这些盟约中得利的同时,也背负着相应的义务。

%22
比方说。琅琊福地被强敌攻打,方源必须前去支援,要与琅琊福地共存亡。

%23
琅琊地灵还算守规矩,毕竟是地灵,没有人类的花花肠子。凡事按照琅琊派的门规行事,并没有过多地拘束方源。

%24
雪人一族就不一样了。

%25
他们是异人,灵性仅次于人族。被打压了这么多年,在北部冰原苟延残喘,巨大的生存压力,养成了他们强大的战力,和精明的思维。

%26
方源若要和雪人成婚,雪人一族必定会想方设法,来榨取方源身上的利益!

%27
到那时,方源就身不由己了。

%28
“雪人和石人。在北部冰原地下生存。石人一族拥有太古石龙,八转战力,而雪人一族却是没有的。所以,他们更希望我和雪儿结合,能够间接地掌握八转战力,从而和石人一族分庭抗礼。”

%29
别看雪人、石人蛊仙们,在招待方源期间,表现得十分融洽。

%30
但真的有这么和谐融洽吗?

%31
双方挤在冰原底下,一块土地上生存,空间狭小。平时怎么可能没有摩擦?毕竟是两个种族。

%32
只是碍于大局,心存理智,才相互扶持容忍。

%33
可想而知,拥有太古石龙的石人一族。在平时的矛盾中,是往往得利的一方。雪人吃的瘪不会少,只是都隐忍下来了。

%34
石人一族的地位还是比较高的。

%35
这点从联盟前后的细枝末节中,都可以看出来。

%36
就比如,石宗这位蛊仙,往往都是他第一个代表言。并非是其他的雪人蛊仙。

%37
“我若和雪儿结合,不,哪怕是公开接受雪儿的情义,就代表着亲近雪人一族。不可避免地,就会被牵扯到雪人、石人两族的内部矛盾之中了。”

%38
一旦有了内部矛盾,雪儿便找到方源,请求方源出面调解,方源作为盟友之一,不好拒绝。

%39
偏偏石人一族给与方源的利益更大。

%40
毕竟双方有胆识蛊的贸易往来。

%41
方源怎可能因为一点美色,恶了和石人一族的关系,让自己的收益损失?

%42
这些天来,其实不止冰卓的来信,还有其他墨人、雪人、石人蛊仙的信。

%43
其中就包括雪儿的。

%44
背后的意图,都是想要拉拢方源。

%45
八转战力的诱惑,实在是太惊人了!

%46
方源离开北原,前往东海的原因之一,就是主动避开这些蛊仙。也就意味着,提前避免了将来很多的麻烦。

%47
“对于这些异人,纯粹是利用。可惜现在我和他们的关系,是陷得越来越深。一旦生意外,我也要跟着遭殃。必须尽快脱离,方能进退自如。”方源心中早有觉悟。

%48
并且,琅琊地灵想要招揽霸仙楚度的意图,也让方源感觉到越加的不妙。

%49
一旦楚度加入进来,将极大地威胁到方源在琅琊派中的地位。

%50
琅琊地灵想要招揽楚度的原因,也值得深思。

%51
虽然琅琊地灵交代了这个任务,但方源私下决定,能拖就拖。他绝不会傻乎乎地全力策反楚度。

%52
“东海信道传承”方源在心底呢喃。

%53
这正是他此次前往东海的目的之一。

%54
若是方源能够在这个信道传承中,获得什么手段,让他单方面悄悄地解除身上的盟约,那就最好不过了。

%55
方源的外交,将彻底占据主动!

%56
不过,这支信道传承只是次要目标,方源此行还有另外的主要图谋。

%57
数天之后。

%58
东海,乱流海域。

%59
“饶命,我,我投降了!”刘青玉双膝跪在地上,他伤痕累累,斗志全无。

%60
他是七转散仙,此行更纠结了三位六转同道。

%61
没想到在这乱流海域中,却遭逢埋伏,三位六转蛊仙受他邀请出手,此刻都已阵亡。

%62
他们的敌人,是四位来历神秘的蛊仙。

%63
刘青玉也算是见多识广之辈,但他从未听闻过东海蛊仙界,有这样的四位强者!

%64
这四个蛊仙,都很特别。

%65
修为最高的,有七转气息,却是一位石人蛊仙。

%66
其余三位,都是六转蛊仙。一位极美的英武女仙,一位看起来相当和气的老者,还有一位面容普通的仙僵。

%67
不管当中的哪一位,都是相当厉害的蛊仙强者。但叫刘青玉感到疑惑的是,他们中的脑却不是石人蛊仙,也不是英武女仙或者老者,而是那位貌不惊人的六转仙僵!

%68
“哦?主动投降,呵呵,也算是识时务了。”六转仙僵摩挲着下巴,笑嘻嘻地道。

%69
若是方源在场,必定认出他的身份。

%70
没有错,他就是影无邪!

%71
“哼,毫无骨气之徒,难怪不是我的对手。干脆直接杀了,给石奴增添土道道痕罢。”黑楼兰的身上,缭绕着橘黄的火焰。

%72
石人蛊仙石奴面无表情,看了一旁的影无邪一眼,缓缓地道:“一切听凭大人做主。”

%73
“别,别!”刘青玉连忙大喊,神情惶急,又有委屈,“我虽然是七转蛊仙,但我身上的仙蛊,只有六转而已。反倒是仙子你,却有七转仙蛊,还有那么多的炎道杀招。我敌不过你,也是很正常的啊!”

%74
影无邪哈哈一笑,摆摆手:“现在我影宗正值用人之际,不妨就给你一次活命的机会就是了。乖乖地不要反抗,我来催动仙道杀招,订下盟约!”

%75
“谢大人不杀之恩,谢大人不杀之恩!”刘青玉大喜,感激涕零,一时磕头不止。

%76
影无邪的盟约,自然极为苛刻。但刘青玉是要保存性命,其他的东西已经是次要的了。

%77
“的确是魂道手段”一旁的黑楼兰注视着整个过程,影无邪是用魂道手段,达到信道效果。一如方源的百年好合,乃是其他流派触类旁通的成果。

%78
“从现在起,你就是影宗的一员了。起来吧。”订下盟约,影无邪笑道。

%79
“我愿将仙蛊都献给大人。”刘青玉却并未起身,而是取出自己的两只六转仙蛊,举过头顶。

%80
影无邪哈哈大笑:“这两只仙蛊我还不放在眼里,你都留着,今后好好表现,赐给你七转仙蛊或者真传什么的,也是有可能的。”

%81
“谢大人恩德!”刘青玉一脸感激涕零之状。

%82
“话说来,这里地处偏僻,周围海岛也不少,你们为何偏偏要来这处小岛?”影无邪旋即问道,他怀疑自己这边是不是有什么行迹泄露了。

%83
刘青玉提到这事,苦笑不已:“启禀大人,这一切都是个误会。我不过是想寻找一个歇脚之地罢了。唉,这一切还得从头说起。在这乱流海域,我现了一个信道真传的线索”

%84
原来当初,刘青玉追杀血道魔仙,想要夺得信道传承的关键线索。结果他是杀了此人,却毫无成果,反倒被参与此事的其他两位七转蛊仙怀疑。

%85
他事后反复想,觉得方源嫌疑最大,极可能信道真传的线索被他暗中得了。

%86
刘青玉非常不甘心,便纠结了一批蛊仙,再度秘密潜入乱流海域,企图寻找其他线索,或者等候方源上门。

%87
听完刘青玉的故事,黑楼兰、影无邪等人,都不由地浮现出笑意。

%88
这刘青玉也太过倒霉了。

%89
想寻找一个歇脚的地方,结果好死不死地撞到影宗在这里布置的秘密基地上。

%90
影无邪忽然面色微变:“等等,你刚刚说你那最怀疑的蛊仙,长的是何种模样?”

%91
ps:晚上临时有事,今天只好一更了。欠的一更,会找时间补上。

\end{this_body}

