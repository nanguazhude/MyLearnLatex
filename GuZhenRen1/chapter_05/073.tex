\newsection{天大地大我最大}    %第七十三节:天大地大我最大

\begin{this_body}

方源缓缓飞上天空。

离开之前,他再次,望了一眼身后下方的巍峨云城。

他的脸上,浮现出满满的苦意,深深的长叹一声,才转头飞离。

上一次,他接受琅琊地灵的任务,探索太丘,让琅琊地灵非常高兴,皆大欢喜。但这一次,方源坚持拒绝,没有接受讨伐落星犬的任务,双方不欢而散。

云城中,琅琊地灵坐在主位上,也是咬牙切齿。

他强忍怒气,拿起杯盏,但是火虹茶还未入嘴,就被他啪的一声摔到地砖上去。

“方源你这个家伙!”琅琊地灵捏了捏拳头,磨着牙,心中十分不快。

不过,他也感知到方源临走前的脸上神情,心中的愤怒还是稍微舒缓了一点的。

“方源啊方源,或许你真的很忙,刻苦修行,有自己的安排计划。但此事关乎我琅琊派展大计,你可是我派中的一份子。关键时刻,怎么就不能舍弃自身,维护门派利益呢?说到底,你终究还是外人!不是我毛民蛊仙啊!”

琅琊地灵想到这里,眯起双眼,寒芒烁烁。

站在他这个位置,方源不愿接受这个门派任务,琅琊地灵也不能强硬压迫。

他创建了琅琊派,派中盟约制约所有的成员。但盟约可没有强行规定,可以逼迫派中成员去做他们不愿意做的任务。当然,当琅琊派面临重大危机时,这个规定就没有了。

但现在,琅琊派只是开太丘,和重大危机毫无干系。方源不愿意,琅琊地灵还真的没有什么办法好想。

琅琊地灵坐不住,站起身来,在会客厅中四处踱步。

他背负双手,低头沉思,心中的苦恼和烦躁。让他的眉头越皱越紧:“方源不愿动手,那落星犬这个事情该怎么办呢?”

左思右想,也没有什么好办法。

琅琊地灵本身,是绝对没有方法。离开琅琊福地的。??

还有一个方法,就是上古战阵天婆梭罗。

但琅琊地灵掌握着最关键的组阵仙蛊,在当今炼炉仙蛊屋残缺的情况下,这可是保护琅琊福地的有力手段。

琅琊地灵不想就因为这个局面,就轻易地将天婆梭罗交托出去。万一有个意外怎么办?

琅琊地灵这点认知还是很清醒的。

“唉!难道真要像方源离开前。劝说的那些话?不妨将这头落星犬当做磨刀石,磨砺毛民蛊仙们的战斗造诣吗?”琅琊地灵仰头长叹。

方源在空中疾飞,衣袂翻飞。

呼啸的狂风,扑面而来,风声响彻耳畔。

他的心中一片冰冷。

之前的苦笑和长叹,只是一次表演。

方源深知:在这琅琊福地中,琅琊地灵几乎能达到全知的地步,感应侦查能力极为强大。要不然,他也不会收集到那么多的寿蛊了。

这也是因为,琅琊福地历史太过悠久。底蕴极其丰厚,有难以计数的道痕。这才使得琅琊地灵在这片福地中,能力十分突出。其他地灵和他不能相比。

“我目前的计划,就是依靠琅琊派,帮助我顺利修行。自己是棋手,琅琊派是我利用的棋子。如果我接受那个任务,岂不是自己成了马前卒,为琅琊派出生入死吗?”

“哼!太丘那里得到天意的关注,十分危险。我怀疑天意已经布置了陷阱,只是没有现我。又被太古尸骸克制,所以引而不”

方源谨慎,这种情况下,他是绝不会轻易前往太丘的。

他要修行变化道。要炼制变化仙蛊,斩杀荒兽获取仙材等等。但这些事情,不需要他亲自去做。

琅琊派开太丘,自然会取得这些。方源只要用门派贡献交易,稳居大后方,安心修行即可。

别说危险。可能有天意布置下来的陷阱。就算没有,方源也不打算因此耽误自己的修行!

天大地大我最大。

方源态度坚决。

对于今天的不欢而散,他也早有心理预料。

“我是绝不会为琅琊派奔波效劳,不过,我和琅琊地灵的关系也要处理好。”

“这样的分歧闹得次数越多,我和琅琊地灵的关系也就越恶劣。真要到了极端,琅琊地灵搞不好会无法忍耐,将我逐出门派,甚至斩杀我。”

“况且短时间内,我还要借助琅琊派的力量。至少眼前的第二次地灾,就要借用他的那些水道仙蛊。”

方源冷静思考,分析自己所处的局面。

他曾经和上一任琅琊地灵闹过矛盾,关系降到冰点。这是一个失误,方源警惕自己,不想再犯第二次。

“有时候,情况允许的话,能帮衬出力,还是要的。”

“我对琅琊派只是纯粹的利用,目的是方便自身修行。”

“只有自己强大,修为提高,才是最可靠的。”

天意可以影响方方面面,万事万物。但影响的程度,是有极限的。天意不像假意,可以直接参与脑海中念头碰撞的过程,歪曲生命的思考结果。

这点,天意是比不上假意的。

从影宗处了解到了天意之后,方源知道,天意本质上只是意志的一种。

天意最多是潜移默化。

所以,天意以方源为棋子,对付影宗、魔尊幽魂时,从未抛头露面,而是长期布局,一步步层层递进,团团围拢,最终才图穷匕见。

但义天山大战,最终还是方源靠着半个天外之魔的直觉,脱离了天意的掌控,从而酿成如今的局面。

所以对付天意,最王道的根本方法,就是提高蛊仙的修为,壮大自己的实力。

举个例子,当初戚灾追杀方源,是因为方源只是六转蛊仙,比他弱小。若方源是八转,他绝不会傻到来追杀。

同样的,对于那群云兽,方源若是有实力,自可以轻易剿除。正是因为他实力弱,才被撵得四处乱跑。

“所以,就算地灾天劫,是天意亲自动手铲除我的良机。我也要临危不惧,迎难而上!每当我渡过一次灾劫,就是突破天意的一次次封锁。一次次的壮大,最终能够抗衡天意。等我成就魔尊,天下无敌,看天意能奈我如何?”

方源心中,早有如此明悟。

到自家云城,得知一位毛民使者已经来到。

方源宣见了这位毛民使者。

他是一位五转巅峰的蛊师。

“大人,这是我家城主命令小的,送达此地的东西。”毛民蛊师毕恭毕敬地道。

说话的同时,他双手奉上一块木盒。

方源伸手接过,却不开启木盒,而是上下打量了一下眼前的毛民使者。

就这么一眼,毛民蛊师顿时感到心底凉,好像是赤身裸.体,毛都被扒光了似的,什么秘密都被对方一览无余。

“从今天,你就在我这里住下,我会指点你奴道上的修行。”方源悠悠地道。

“谢上仙指点!”毛民蛊师立即拜倒在地,激动地浑身都在颤抖。

“退下罢。”方源对这位毛民五转巅峰的蛊师可没有什么兴致。

挥退他下去,方源便转身来到静室,继续修行。

他拍开木盒,盒中躺着一只仙蛊,不是其他,正是六转奴兽仙蛊!

原来,方源在几日前,自觉交情到位,时机成熟,便向毛十二商量,想要借他的奴兽仙蛊,愿意用门派贡献支付。

毛十二虽然答应,但却没有接受方源的门派贡献。而是转而提了另外一个要求。

他前不久,在海上大6上接引了一个血脉后裔。毛十二和这位后裔的祖母有过一段情史,加上这位后裔资质不错,毛十二就想将他继续栽培。而这位后裔修行的正是奴道,毛十二就想让方源来亲自指点他。

方源自无不可。

说实在话,这个交易方式对他更有利。

毕竟,他最近门派贡献越用越少,能节省一些是一些。

仙窍中,冰雪的范围已经停止了扩张。

这说明,上一次地灾增添的冰雪道痕,已经充分挥了影响,完成了对仙窍的改造。

方源控制的仙僵,小心翼翼地接近一头荒兽雪怪。

在仙窍中,因为时光流不同,距离上一次围剿,已经过去了很长一段时间。雪怪们又重新松懈下来,所以方源面对的是一头游散在外的雪怪。

驭兽仙蛊!

方源念头凭空,直接往这仙蛊中灌注青提仙元。

荒兽雪怪顿时身躯一震,像是遭受到了无形的一记重击。

但旋即,它仰头咆哮,现了隐藏行迹的力道仙僵,向他奔杀而来。

“失败了么撤!”

方源立即撤退。

按照常理来讲,凭他此时的魂魄底蕴,碾压一头荒兽雪怪是妥妥的。但现在却非如此。

“果然如影宗所言,这些天意充斥的生命,根本不受我的奴役和掌控。留着它们,是最大的隐患!”

方源叹息一声。

虽然从交易中得知了这些情报,但凭他的个性,不自己尝试一下,是绝对不会罢休的。

“若是能够奴役,不管是控制荒兽雪怪,以夷制夷,还是卖到宝黄天去,都是好的。可惜如此一来,就只能强杀了。”

“等等!或许我可以先用我意,冲刷掉天意,再来用奴兽仙蛊尝试奴役?”方源念头又一闪。(未完待续。)

地一下云.来.阁即可获得观看】

\end{this_body}

