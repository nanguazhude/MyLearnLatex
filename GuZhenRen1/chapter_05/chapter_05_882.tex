\newsection{阵平海域}    %第八百八十六节:阵平海域

\begin{this_body}

%1
五团白光,在高空中骤然一闪,将方源、庙明神等五人从功德碑带到了这里。

%2
五人都熟知任务,纷纷朝下鸟瞰,各自催动侦查杀招。

%3
方源率先觅览全局,细细打量。

%4
他虽然在侦查这块,一直都是短板,但只是针对他自己而言。和庙明神等人一比起来,他的短板之处,也不是庙明神这些人能够媲美的。

%5
这是一片广阔的海域。

%6
海面上波涛汹涌,掀起万千重浪,浪花滔滔,无数海兽在波涛中火影获悉。

%7
在这海上,还有五座小岛。

%8
一座玉瑶岛,岛上有无数玉柱,恶风吹鼓,叮当作响。

%9
一座白骨岛,苍白一片,骨兽零星。

%10
一座空隐岛,若隐若现,似乎虚无。

%11
一座剑风岛,岛上有三股龙卷风,终年不息。

%12
一座宝月岛,岛呈月牙形态,小岛上空始终有一片夜幕笼罩,全岛每一寸沙土都闪耀着淡淡的月光。

%13
这五座小岛中有三座,已经开始随波逐流,缓缓漂动。其中剑风岛的流速最快,方向混乱。

%14
剩下两座海岛,玉瑶岛岌岌可危,岛下的支柱已经破损不堪,再过一段时间,就会和那三座小岛一样,开始漂流海面了。

%15
至于空隐岛却是非常稳定,一直牢牢固守在远处。这是因为这座海岛充斥虚道道痕。

%16
造成海岛漂流的罪魁祸首,乃是海底深处蔓延纵横的撞击海流。

%17
这种海流已经庞大到危急海底的地步。镇压海流,将这片海域生态重整,恢复昔日的五岛岛链,正是此次大型任务的内容。

%18
方源开始计算,在心中不断地推演。

%19
种种迹象表明,这片海域原本是以五岛岛链为核心,形成的巧妙生态。撞击海流之前并不存在,而今撞击海流就像是平静的荷塘中忽然闯进一头鲨鱼来,给予海域生态十分剧烈的伤害。

%20
在方源推算的时候,庙明神等人也逐渐将整个海域纳入侦查范围。

%21
他们纷纷皱起眉头,面色凝重。

%22
“这就是大型任务?难度比中型任务要高出数倍啊!”花蝶女仙心中陡沉。

%23
蜂将脸色难看,暗想:“我之前平衡蜂巢岛的生态,拼尽全力,都不完善。这次大型任务,不是平衡生态,而是重整生态。每一座海岛的情况,都比蜂巢岛更加严重。并且五座海岛只是任务的一部分,接下来还有大片的海域,还有海底的撞击海流!”

%24
“盘根错节,要重整这一片的海域生态,需要抽丝剥茧,一条条地进行梳理。没有智道蛊仙,单靠我们恐怕要不断实践,一步步地改进。这当中,错误绝不会少的。”鬼七爷在心中叹息。

%25
庙明神则在计算投入和产出:“我刚刚完成的中型任务,消耗了我大把七转仙材。要重整这里的生态,投入的资源,耗费的仙元还有精力,恐怕是一笔极为庞大的数字!”

%26
庙明神等人越是观察,心中的斗志就滑落得越快。

%27
大型任务的难度,让他们却步。

%28
事实上,他们已经开始后悔了,盲目跟从方源接取大型任务,似乎不是一件明智的事情。

%29
“楚兄,我们合计合计吧。”事已至此,庙明神只得苦笑,找方源商量。

%30
他并不打算就此放弃,接取了任务,不代表他们可以放弃任务。事实上,要放弃任务,还得需要领取另外的名号。

%31
方源眼中精芒一闪即逝,借助宙道分身和自己本体同时推演,他已经得到了一份完美的方案。

%32
他对庙明神摆摆手:“庙兄,我已有成算,照我安排去做,必定能将这个任务完成。”

%33
庙明神愣了愣,虽然心中不信,但他却也不反驳,点头:“还请楚兄吩咐,在下等人一定竭力而为。”

%34
方源哈哈一笑:“庙兄客气了,我们联手合作,共同完成这个任务,对你我都有好处。大型任务完成得好,收益可是中型任务的数倍!”

%35
“就怕完成不好啊。”庙明神心中一叹,却是没有说出口,他已是没有多少信心,眼下的希望也只能寄托在方源身上了。

%36
不过方源的话,倒是听得他很舒服。

%37
方源没有高高在上,而是一副平等合作的态度。这点就不像沈从声。

%38
“请庙兄占据玉瑶岛,防止撞击海流。请鬼七爷驻扎剑风岛,务必使它不要随意流窜。请蜂将、花蝶仙子分别镇守玉瑶岛、白骨岛,防备海流。我先去处理空隐岛。”

%39
方源做了安排,庙明神一伙人本来就毫无头绪,便按照方源吩咐行事。

%40
有了四仙出手,海岛的情况迅速得到缓解,不再随波逐流,漂流速度陡降,被拘束在某个范围内。

%41
看着方源落入空隐岛,四仙心中不免生疑。

%42
“空隐岛乃是虚道小岛,是整片海域生态中最稳定的一角,也是整个任务最轻松的部分。”

%43
“楚瀛为甚要先处理这座小岛呢?应当先把其他小岛都处理好了,再来解决空隐岛。毕竟这座小岛近乎不存在,先处理这座小岛,反而会对接下来的几座小岛形成阻碍。”

%44
“依我之见,还是应当处理撞击海流。这才是导致五岛岛链崩溃的罪魁祸首啊。要重整这里的生态,必定是要从源头开始一步步解决。”

%45
方源落入空隐岛,很快就飞出来,速度之快,让庙明神等人又一番诧异。

%46
方源进入白骨岛。

%47
在花蝶女仙的注视下,方源从仙窍中掏出一份份的仙材,随意抛洒。

%48
“他竟然在布阵!”花蝶女仙第一次看到方源布阵。

%49
方源布阵手段娴熟老练至极,整个过程如行云流水,让花蝶女仙看了,有一种发自内心的赏心悦目的感受。

%50
很快,方源的分阵就布置妥当。

%51
花蝶女仙震惊得张开樱桃小口,合拢不上。

%52
“楚瀛居然能够利用仙材布阵!这岂不是说明,他在阵道方面竟有着宗师级的造诣?”

%53
“原来他不是变化道蛊仙,而是阵道蛊仙。”

%54
“不,不对,当初他击败葛温,救下我等,明明是变化道的手段。”

%55
“难道,他是兼修两道?”

%56
蛊仙兼修两道罕见,但也不是没有。但像方源这般两个流派,居然都如此精通,造诣如此深厚,那就是非常难得的了。

%57
白骨岛的分阵已经布置好,方源便调动这个分阵,和之前空隐道上的主阵联系起来。

%58
在仙阵的作用下,白骨岛咯噔一下,狠狠一抖,旋即就固定在海水上面,再无漂流的任何迹象。

%59
花蝶女仙这才有些明白方源的深意:“原来他是用空隐岛为根基,牵引其他小岛,让它们都固定下来。”

%60
“妙啊,空隐岛作为根基,真的再合适不过了。因为撞击海流也不能拿这座小岛怎样。只是……”

%61
花蝶女仙旋即又想到此法不足之处:“只是其他的海岛,都是依靠空隐岛的牵引。这是阵法之功,而这些海岛仍旧会受到撞击海流的冲撞。岛上的蛊阵不断受损遭创,时间一长,蛊阵被破坏,这些小岛仍旧会随波逐流而去的。”

%62
方源却没有为花蝶女仙解惑的意向,随后,他又分别落在剩下的三座小岛上,一一布阵,将这些小岛暂时固定下来。

%63
庙明神等人看向方源的目光都有些复杂。

%64
方源展露出的阵道造诣,让他们感到震惊。

%65
阵道宗师,可是不多见的!

%66
而之前方源展现出来的变化道造诣,显然也是非同寻常。

%67
如此实力,让方源在众仙心中的神秘程度又浓郁了一大截。

%68
“不过,暂时固定了五座海岛,只是开了一个好头而已。”

%69
“真正的麻烦,在于海岛的撞击海流啊。”

%70
“没有水道的手段,怎么处理这样的海流?”

%71
庙明神等人传信议论。在龙鲸乐土中,他们不能直接传音交流,但是动用信鸽蛊之类的信道凡蛊,还是可以的。

%72
恣意纵横在这片海域中的撞击海流,规模实在庞大,庙明神等人能想到的办法,就只有动用仙道手段,不断地轰炸撞击海流,令它一部分一部分的崩解。

%73
这样的方法十分蠢笨,并且耗费的时间、仙元等等都非常庞大,还有一点很大的弊端,就是崩解的撞击海流会对周围的海域,造成十分严重的破坏。

%74
然而除此之外,庙明神等人也没有什么好办法。

%75
轰!

%76
就在这时,方源悬浮高空,尽起蛊阵。

%77
五座海岛分别绽射出冲天的各色光柱,蛊阵的威能猛地暴涨上去,冲破了原有的极限。

%78
“原来这才是大阵的真正威能!”庙明神等人面色皆变,目光惊骇。

%79
这五座海岛道痕浓郁,也可以看做是一份体积巨大的仙材。方源布置的仙阵,自然也将它们囊括在了可以利用的范围内。

%80
“我要开始接引撞击海流了。诸位还请镇守海岛,期间有什么海兽侵袭,或者巨大的海浪,但凡危及到海岛的,都需要你们出手。”方源大吼一声,声音传遍整个海域。

%81
庙明神等人连忙回应,表示一定拼尽全力。

%82
大阵嗡嗡作响,五根光柱逐渐消散,化为漫天的洁白光雪。

%83
光雪浸入海水当中,将海岛周围的海水染上一片片的白光幻影,美轮美奂。

%84
在这片白光幻影当中,撞击海流宛若被鱼虾勾动的蓝鲸,缓缓调动身躯,然后越游越快,向最前方的玉瑶岛撞来。

%85
玉瑶岛上的庙明神咬住牙关,心里承受着巨大的压力,紧张起来,全神戒备。

%86
若是仙阵防护不足,这样的海流直接撞击过来,能够完全将玉瑶岛撞碎!

%87
轰!

%88
撞击海流接触到大阵,发生惊天的爆响。

%89
一道巨大的海浪,高达十丈,向玉瑶岛扑来。

%90
庙明神立即飞上去,大喝一声,催动杀招死死顶住。

%91
撞击海流被这一撞,缓缓掉转方向,折向玉瑶岛的左方。但它还未远离,就又被第二座白骨岛的仙阵牵引回来。

%92
镇守白骨岛的乃是花蝶女仙,她见到海流朝自己而来,差点忍不住要呻吟一声。

%93
在庙明神一伙人的讨论中,撞击海流是要被排斥、削弱出去的,最好的结果,是将它彻底打散击溃。

%94
但方源此举却是主动勾引撞击海流,来撞击海岛。

%95
这恰恰是庙明神等人极力想要避免的情况。

%96
“楚瀛这家伙未免也太疯狂了!”花蝶女仙很快就来不及思考,撞击海流撞上白骨岛后,整个大阵都响起令人心悸的吱呀之声,然后又掀起巨大的海浪。

%97
这海浪当中还夹裹了不少海兽,花蝶女仙奋起全力,出手抵挡。

%98
就这样,撞击海流在方源的不断牵引下,陆续撞击到五座海岛上。

%99
每撞击一次,海流的威能就减少一层,随后就绕过海岛,被牵引着前往下一处。

%100
撞击海流撞过最后的一座海岛后,它的威能和冲势已经不足之前的一半。

%101
这个时候,方源再次催发大阵,竟让撞击海流绕过海岛,掉了个头,又往倒数第二座海岛撞去。

%102
于是,镇守海岛的蛊仙们又一番心惊肉跳。

%103
当撞击海流撞击了五次之后,重新回到玉瑶岛,海流撞击的威能已经几乎不存,差点就要化为普通的海流。

%104
五座海岛轻轻一震,仙阵彻底稳定下来。撞击海流迅速平复,从头至尾开始均匀流速。

%105
但岛上的四位蛊仙心中却是掀起了惊涛骇浪!

%106
他们万万没有料到,方源居然是这样处理了撞击海流。

%107
撞击海流虽然仍旧存在,却成了环绕海岛的保护层。在五座海岛周围,撞击海流迅速流动,流速远超寻常海水一大截,将五座海岛牢牢护住,同时又反过来帮助仙阵,固定五座海岛的位置。

%108
若是采用庙明神等人的方案,摧毁撞击海流,这片海域定然被摧毁得不成样子。

%109
方源此举却是巧妙地避开了这个弊端,反过来将撞击海流充分利用,原本的灾祸反而成了一种福分,大大提升了这片海域的自然底蕴。

%110
这简直是神来之笔!

%111
庙明神等人无不叹服。

%112
要做到这一点,难度真的太大了。

%113
仙阵的布置且不去说,牵引撞击海流这个事,水道蛊仙都没有这样的本事。而要做到完美的衰减撞击海流,需要严密的计算推导。当撞击海流绕回到玉瑶岛时,它的威能必须被削弱到几乎没有的程度。多一分,少一点,都会极大地影响全局,不会得到如今近乎艺术般完美的平衡。

%114
“大局已定,接下来就会请诸位修补仙阵,同时修补海岛上的生态,撞击海流之外的事情,可以不去管它。”方源笑着道。

%115
他对这份结果也比较满意。

%116
将来若是再有撞击海流出现,这海流若是撞击海岛,就会在仙阵的作用下,被周围的保护海流吞并。

%117
方源会在仙阵中留下意志和大量的仙元,只要这些不缺,五岛海链就会非常的安全。

\end{this_body}

