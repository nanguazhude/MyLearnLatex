\newsection{舆情汹涌}    %第七百九十一节:舆情汹涌

\begin{this_body}

%1
顺着光阴支流,方源等人回到了南疆。

%2
尽管群仙状态极好,斗志昂扬,但战后休整还是必要的。

%3
“此战收益其实还不错。”方源检查了一番打捞上来的尸体。

%4
除了八转蛊仙清夜之外,这些中洲七转的蛊仙,大多来自于中洲十大古派,本身还不够加入天庭的资格,所以仙窍也没有贡献上去。

%5
绝大部分的蛊虫都被摧毁了,但这些仙窍却是实打实的。

%6
还有一两只仙蛊,是两座仙蛊屋被毁后,幸运至极保留下来的。

%7
接下来一段时间,方源准备仔细检查这些仙窍,看看有无问题,没有问题的话,就将他们吞下。

%8
当然,有一部分的蛊仙,方源因为流派境界不足,还不能吞窍。

%9
事实上,南疆俘虏中也有一部分这样的人物。

%10
“战利品只是添头,此战真正的意义在于我终于甩脱了被动的局面,在和天庭的对抗中,第一次把握了主动啊。”

%11
方源此战,时机把握得极好。

%12
若是提前战斗,他没有仙蛊屋万年斗飞车,只有暴露八转修为。

%13
这是不可取的。因为八转修为一旦暴露,天庭方面就会发觉方源没有渡劫,连带着至尊仙窍的秘密也会跟着暴露。

%14
若是延后作战,天庭方面就绝不止这两座仙蛊屋了。上一世方源遭遇的就是四座宙道仙蛊屋。

%15
摧毁了恒舟和今古亭,天庭陷入了极为被动的局面。

%16
天庭唯恐方源获得红莲真传,绝对会在光阴长河中增兵,阻止方源。

%17
然而继黄史上人之后,天庭就缺乏宙道八转,只好用清夜以及宙道仙蛊屋等等来替代。

%18
如此一来,就给了方源再次击破的机会。

%19
就像方源这一次击毁了恒舟、今古亭一样。

%20
然而,天庭方面就算明知如此,也需要尽快侦查,尽力阻止,哪怕付出再大的代价。

%21
“上一世,天庭将光阴长河防守得固若金汤。但这一世……呵呵。”

%22
“光阴长河已经成为天庭必救之地,也是我削弱天庭,让天庭持续放血的上佳战场。”

%23
天庭方面的反应,方源几乎可以确定。

%24
就像天庭掌握了宿命蛊这样的战略优势,逼得四域蛊仙不得不远征,进攻中洲和天庭。

%25
方源靠着红莲真传的幌子,也逼得天庭必须增兵光阴长河。

%26
“只是这样一来,我重生的秘密应当是保不住了。”

%27
上一次琅琊大战,紫薇仙子就已经严重怀疑。这段时间,方源俘虏了南疆追兵,又在光阴长河中大杀特杀。

%28
紫薇仙子若再不能确定方源重生,那她就不是紫薇仙子!

%29
方源绝不会低估紫薇仙子,也特别了解她。

%30
往往了解你的,不是你的朋友,而是你的敌人。

%31
果然,数天后,紫薇仙子在宝黄天中公开宣布方源是重生归来,并且已经掌握了一座八转宙道仙蛊屋。

%32
消息一经传出,迅速四处播散,方源影响力更上一层楼,天下哗然!

%33
“这才多久?方源竟已经成长为魔道巨擘,擎天巨柱般的人物了!”

%34
“之前方源有那逆流河参与的杀招,能逆反八转蛊仙的攻势,利于不败之地。现在又掌握了一座八转层次的仙蛊屋?”

%35
“这也太快了吧?仙蛊屋就像是从地底蹦出来的一样。”

%36
“既然他是重生回来,仙蛊屋就不算突兀。别忘了,他可是吞并了琅琊福地,这里的底蕴可深着呢。”

%37
舆论自然波及到天庭。

%38
“天庭似乎很弱啊,对方源从未有过什么有效的手段。”

%39
“堂堂天庭,只会一败再败,然后在事后公布方源的情报吗?”

%40
“不是天庭弱小,而是拿方源没有办法。”

%41
“一般而言,像魔道蛊仙就算修为高达八转,也不会像方源这般高调。”

%42
“没错。魔道蛊仙向来势单力孤,惹恼了正道,多个势力就会联手,用信道收集情报,用智道推算位置,再联合追辑。就像之前南疆群仙追捕他。”

%43
“但方源呢?他竟然能在智道方面,防备住天庭的推算。又有定仙游四处往来,难以捕杀。他的阵道实力也非常强悍,定能识破埋伏起来的大阵,并且还利用宙道大阵埋伏,把南疆的追辑队伍直接俘虏了!”

%44
“他还吞并了琅琊福地,他的炼道境界至少是大宗师的层次!”

%45
蛊仙们讨论逐渐深入,惊悚地发现方源居然如此多才多艺,兼修多道,才情骇人,惊艳世间。

%46
“我算是看明白了,他就是一头怪物!”许多人倒吸一口冷气。绝大多数的蛊仙,哪怕是八转,都是专修一道,谁能像方源这般兼顾各个方面?

%47
兼顾各个方面,精力、物力、时间就都分散了,蛊仙往往一事无成。

%48
其实,方源之前也想专修一道,但他没有办法!在此之前,他独自作战,始终在外敌的压迫下。不修行智道能行吗?不修行魂道,战力能有提升吗?没有阵道能俘虏南疆群仙吗?

%49
“不能这么说,方源的天赋才情就算再糟糕,只要他有春秋蝉,不断重生,完全可以利用相同的时间,去积累不同的东西啊。”

%50
“是啊,有了春秋蝉,他就算再失败,也有无数次重来的机会!失败的经验还可以积累下来,就算是蠢猪,失败了多次后,也能自然而然地成功了吧?”

%51
“春秋蝉……不愧是红莲魔尊的本命蛊,真的厉害至极了。”

%52
“奇蛊榜上,春秋蝉的名次应当再提高,现在的名次对于这样的奇蛊,还是太低了。”

%53
“因为方源,春秋蝉的名次的确一提再提,但我觉得诸位不要过度高估春秋蝉了。它毕竟只是一只蛊虫,有很多手段能克制他。方源之所以如此厉害,很大程度上是因为他是一位天外之魔!”

%54
“的确。天外之魔的身份和春秋蝉这两者搭配起来,真的太完美了!然而除了天外之魔,其他人运用春秋蝉,自己的人生很难有巨大的改变。”

%55
方源见此,深思熟虑之后,果断将之前光阴长河一战的影像,也散播到宝黄天中去。

%56
更发言道:“天庭,我的反攻才刚刚开始!待我从光阴长河中取得红莲真传,就来彻底捣毁你们的宿命蛊。上一世你们修复宿命蛊成功,那是你们底蕴深厚,出乎我的意料。而四域蛊仙中有识之士太少,有许多八转强者还被你们暗中牵制。还有许多蛊仙,残留着天真,不知道宿命蛊的致命威胁。你们……给我等着!”

%57
这番言论一出,五域蛊仙界都暂为之沉寂。

%58
随后,更恐怖的舆情仿佛是火山喷发,山洪倾泻!

%59
“我的天,方源直接挑衅天庭,这是多少年都没有发生的事情了?”

%60
“何止是多少年?纵观历史,也没有几个!”

%61
“很少有人这么做,更少有人真的能够对天庭产生威胁!”

%62
“方源现在虽然只是七转修为,但已经匹敌八转。他吞并琅琊福地,有大批异人蛊仙效忠,还有八转宙道仙蛊屋!更可怕的是,他似乎还掌握了红莲真传的确切信息,真让他得到魔尊真传……可怕!”

%63
“真的照此发展下去,方源将成为新一代的魔尊!”

%64
“唉,苍天万民何其无辜,又要有一位魔尊出现祸害乾坤吗?”

%65
“不能轻易和方源为敌。好好看看这番影像,天庭的两大仙蛊屋种种杀招,都被方源破解了。明显是方源上一世经历过,这一世找到了破解之法。春秋蝉和天外之魔的配合,未免也太过赖皮了!”

%66
蛊仙界议论纷纷,渐渐的形成了一种恐慌。

%67
一种对方源的恐惧。

%68
以往的情况下,蛊仙对方源这个存在更多的是忌惮、憎恨、嫉妒。

%69
但现在,恐惧的情绪开始逐渐占据上风。

%70
蛊仙们都知道方源不好惹,但方源现在也太不好惹了!连天庭这样的庞然大物,整个五域最强大的正道势力,都被方源三番五次在头顶上拉屎撒尿,威严被狠狠的践踏!到现在,还拿方源没有办法!

%71
南疆,武家。

%72
武庸目光阴沉:“方源……天庭……”

%73
池家。

%74
池曲由脸色铁青,他终于意识到一种可能性:“内鬼一直查不出来,或许……这是上一世暴露出来的秘密。如果上一世我也和方源达成了交易,那不就是说——我就是那个内鬼?”

%75
想到这里,池曲由自己都傻了眼。

%76
西漠,唐家。

%77
太上大长老、二长老密谈。

%78
“方源这个魔头,真的是越来越强了。若是他和我族的关系暴露出来的话,天庭方面必定是要剿除我族了。”

%79
“我等已是和方源合作了,方源如今势大,这种关系不可终止,接下来更要小心谨慎。若是将来他真的成就了魔尊,对于我唐家,也是一个绝好的发展良机啊。”

%80
习家。

%81
闭关的密室中,太上大长老笑了笑:“有意思。天庭方面希望全天下对方源群起而攻之。而方源也宣扬天庭的弱小和威胁,也是希望四域蛊仙对天庭加以围攻。”

%82
“且不管他,我自闭关。待修成之后,倒要看看天下群雄在老夫剑下的姿态。”

\end{this_body}

