\newsection{天鹰逃窜}    %第三百六十七节:天鹰逃窜

\begin{this_body}

%1
“锁风?!”

%2
“我们什么时候中了这个杀招?”

%3
“根本毫无察觉!!”

%4
影宗群仙尽皆惊愕诧异。

%5
这杀招怎么中的,他们从始至终都没有发现。若非是动用上古战阵四通八达效果欠佳,事实摆在了眼前,否则光听武庸的话,恐怕众人都会以为武庸在虚张声势呢!

%6
永远不要低估一位八转蛊仙的手段。

%7
武庸虽然资辈并不高,甚至在南疆蛊仙界中,论资排辈还低于其他八转蛊仙一头。

%8
但是他家学渊源,更是继承了武独秀的衣钵,因此手段丰富至极。

%9
方源眉头大皱!

%10
“这个锁风杀招,是如何运作的?”

%11
“它究竟能持续多久?”

%12
“该如何寻得此招破绽,加以针对?”

%13
他脑海翻腾巨浪,念头此起彼伏,这些问题困扰着他,关键是必须尽快解决,关系身家性命。

%14
玉清滴风小竹楼不断逼近,武庸的笑声传来,他显然不会给方源充足的思考时间。

%15
情况紧迫无比,需要方源立刻做出决断。

%16
他想了一下,立即对其他人下令:“你们都进我仙窍!”

%17
“什么意思?”

%18
“你们实力太低,和八转蛊仙对战,等若直接送死。还不如进入我的仙窍,抓紧时间研究锁风,尽快想出克制的法门!”方源直言,旋即打开仙窍门户。

%19
众仙面面相觑。

%20
不过很快,黑楼兰第一个钻进去。不用送死,自然是好的。黑楼兰刚刚还担心方源会利用她身上的盟约,让她迎上武庸,主动送死,为方源逃生拖延时间去。结果没想到,方源选择这样做。

%21
妙音仙子、黑菟迟疑了一下,也都紧随其后。

%22
她们因为紫山真君的真传,而有所成就,如今成为影宗成员。但是对于方源的认可,主要还是因为方源乃是影宗之主。

%23
白凝冰反而留在了最后。

%24
她当然不是关心方源,而是看向武庸的目光中,带着一丝跃跃欲试的神情。

%25
白凝冰追求精彩的生活,就好像是方源追求永生一样,这种感情炙热如火,甚至可以说是疯狂。

%26
“你不要胡思乱想了,就算你有白相,武庸想要在一瞬间将你绞杀成渣,也是轻而易举的事情。”方源语气急切。

%27
白凝冰冷哼一声,终究还是向前跨步,迈进了至尊仙窍的门户里头。

%28
方源连忙闭合仙窍门户,然后狂催脚下的上极天鹰,命它不断疾飞。

%29
武庸的叹息声从身后清晰地传来:“方源,你也是一个人杰了,居然有手段能够伪装成武遗海,潜伏在我武家多时。你胆大包天,是个干大事的料子。可惜,可惜啊,你若是真正的武遗海该有多好。”

%30
武庸的感叹情真意切,他还是惜才的。

%31
方源在前面逃,武庸在后方追。

%32
太古荒兽上极天鹰爆发出来的速度,在此刻拼命的情况下,还是非常的迅猛。

%33
但是玉清滴风小竹楼的速度,比它还快。

%34
毕竟是八转仙蛊屋!

%35
同时,玉清滴风小竹楼还有一个巨大的优势,那就是仙蛊屋是不会累的。

%36
只要仙元足够,任何仙蛊屋都能永久地保持一个平稳的状态。

%37
但上极天鹰就不同了。

%38
它飞久了,是会疲累的,是会感到乏困的。

%39
不过这一点,方源似乎不用考虑了。

%40
因为按照现在两者的速度来看,远在上极天鹰感到疲惫,速度下降之前,玉清滴风小竹楼就会追上来。

%41
“还有多少人,一并叫出来吧!”方源背对前方,面向武庸,身上各种蛊虫的气息迅速升腾,并且不断萦绕。

%42
“你放心,只有我一人而已。”武庸笑了笑,然后催动仙蛊屋。

%43
玉清滴风小竹楼上,青翠的竹叶,不断飞射出来,化为一道道风箭,射向方源。

%44
方源操纵上极天鹰,忽而拔升,忽而迫降,在空中不断飞绕,娴熟地躲过风箭的攻击。

%45
“不错。你这飞行功底扎实。奴道手段,更是叫人眼前一亮。”武庸不吝夸赞地道。

%46
方源却意识到不妙。

%47
他身上萦绕着一团微风,起先微不可察,如今却是越来越壮大。

%48
不仅是他,就连上极天鹰的双翼,也缠绕着两团青色的风,大大阻挠着它的飞行速度。

%49
“武庸不断出击,迫使我不断躲闪,改变方向。上极天鹰的速度虽然没有丝毫的降低,但是玉清滴风小竹楼却是直线追击。”

%50
“还有这身上的风团,恐怕就是所谓的锁风杀招了。居然可以吸收离散在空气中的凛冽狂风,增长威能!”

%51
武庸的攻击,就是射不中方源,也达到了他的目的。

%52
双方的距离在迅速缩短。

%53
“还没有想到解决之法吗?”方源询问仙窍内的影宗众人。

%54
“难!”

%55
“刚刚有一点头绪。”

%56
“很显然,这是八转杀招,非同小可。”

%57
一堆废话。

%58
方源暗自咬牙,这时他便又听武庸开口道:“你现在一定是在想如何突破我的杀招锁风吧?实话告诉你也无妨,此招一旦布下,便不能移动,范围却是相当广阔,覆盖方圆十多万里。它也有时限,目前还能持续半炷香的功夫。”

%59
武庸竟向坦言。

%60
方源闻声,一颗心直往下沉。

%61
武庸如此作为,显示出了他强大的自信心。他已觉得方源已落入他的手掌,无法再逃出去!

%62
玉清滴风小竹楼一番狂轰滥炸,终于追了上来。

%63
武庸微微一笑,站在竹楼二层的窗口处,伸出食指,对着方源轻轻一指。

%64
仙道杀招——指风龙!

%65
叮咚一声脆响。

%66
一条碧墨小虫,从他指尖,立即飞了出来。

%67
小虫速度极快,飞向方源。

%68
飞行途中,它猛地涨大,身躯急速膨胀,一尺、五尺、一丈、五丈、十五丈。

%69
几个呼吸的时间,它化为一头二十二丈的凶恶风龙,张牙舞爪,狠狠地撞向上极天鹰。

%70
上极天鹰躲闪不及,眼看就要被指风龙击中,关键时刻,方源挺身而出,他竟然主动挡刀,撞向指风龙。

%71
武庸愕然,继而震惊!

%72
因为指风龙撞向方源之后,竟然没有对后者造成任何的损伤,更匪夷所思的是,指风龙掉转了方向,原路返回,竟然直朝着他这个主人反噬过来。

%73
轰!

%74
一声巨响,指风龙撞在玉清滴风小竹楼上。

%75
仙蛊屋剧烈的震荡起来,之前的速度顿时瓦解大半。

%76
而指风龙也彻底崩碎。

%77
方源趁机,飞到上极天鹰的背上,又拉开一大段距离。

%78
武庸见他身上附着了一层仙衣,绶带飘飘,气息缥缈,又强势旺盛,顿时惊讶到难以附加的程度。

%79
“这个杀招,难道是?!”武庸也为之失声。

%80
他虽然没有亲自参加过北原逆流河大战,但是此战的情报,随着柳贯一扬名五域,早已泄露很多。

%81
武庸认出了这个杀招,正是逆流护身印!

%82
这也代表着他,知道了方源就是柳贯一,柳贯一就是方源的秘密!

%83
方源望见武庸神情诧异,心中暗想,看来天庭方面没有将柳贯一的秘密告知武庸。或许天庭也还不清楚呢?

%84
方源可惜了一下,柳贯一的这个身份也暴露了,不能再用,很是影响他在北原方面的人际交往。

%85
不过这完全没有办法!

%86
面对八转大能,方源只能动用逆流护身印,才能站住脚跟。若动用其他手段,会被武庸轻易杀害。

%87
武庸虽然受挫,但他看向方源的目光,很快变得无比的火热起来。

%88
方源不仅是武遗海,有着影宗背景,而且还拥有逆流河,拥有逆流护身印。只要擒获了他,他身上的底蕴和财富,绝对能够让整个武家的实力,都为之暴涨。

%89
巨大的利益,让武庸心动。

%90
更何况,方源还事关武家近年来的最大丑闻。

%91
“很好,不枉费我一番心机,调度全局,营造出你我单独对战的局面。”武庸低啸一声,驾驭仙蛊屋,不断对方源狂轰滥炸,再次追赶上来。

%92
风锁之害,虽然被方源逆反到了武庸身上去。但上极天鹰身上仍旧有风锁,武庸却是被玉清滴风小竹楼承载,根本对危局无效。

%93
这一次,武庸直接操纵仙蛊屋,对着上极天鹰撞去。

%94
方源叹息一声,不闪不避,挡在仙蛊屋的面前。

%95
武庸轻笑一声,直接飞出仙蛊屋,来战方源。

%96
与此同时,仙蛊屋玉清滴风小竹楼则对准上极天鹰。

%97
武庸虽不在玉清滴风小竹楼中,但他在仙蛊屋内预留了意志,还有大批仙元。

%98
上极天鹰和玉清滴风小竹楼纠缠,很快处于下风。

%99
方源的情势更加不济,交手之后,他彻彻底底体会到八转蛊仙的威势。

%100
在武庸的攻势之下,方源只能被动挨打,毫无还手之力。

%101
十几个回合下来,武庸彻底认识到方源的强大防御。逆流护身印叫他也感到极其头疼,任何的攻击打到方源身上,都会被逆反回来,哪怕是拳脚打击,也是一样。

%102
武庸旋即便将目光集中在上极天鹰的身上。

%103
这头太古荒兽,就是此战的突破口。

%104
意识到这一点后,武庸便分出一些心神,将方源死死镇压,然后动用仙道杀招轰击上极天鹰。

%105
上极天鹰哀鸣连连,它速度惊人,但主人就陷落在这里,怎可能容它逃窜?

%106
不过太古荒兽皮糙肉厚,上极天鹰硬挨了武庸数招,仍旧活蹦乱跳。

%107
“不好!”但就在这时,方源忽然面色骤变。

%108
上极天鹰面临强敌,本身的意志逐渐占据上风,想要逃生,就在此刻,这股意志终于达到质变,让方源的杀招百八十奴失败,上极天鹰失去束缚,立即扑扇翅膀,逃离此地。

%109
方源被丢下来,独身一人面对武庸和玉清滴风小竹楼。

%110
“你们还没有研究出来?!”方源对仙窍中的影宗蛊仙急吼。

%111
“这么短的时间,如何能有成果!”白凝冰等人也非常郁闷。

%112
武庸扑来:“在我面前,你就别想放出他们,再用四通八达了。”

%113
果然,接下来的战斗,让方源毫无机会。

%114
又过了数个回合,武庸忽然双臂一展,蓄谋已久的仙道战场杀招发动起来。

%115
方源视野骤变,被困在一片陌生的战场中,再无法逃脱。

%116
“你若束手就擒,献上你所有的修行积累,还有一线生机,方源。”武庸发出通牒。

%117
方源面色如铁。

%118
真的是绝境了!

%119
他现在唯一的希望,就是依靠影无邪等人,寻找机会,施展出四通八达,也应该能穿透这片战场。

%120
不过就在武庸想要再动手的时候,他忽然神色微变,瞧向某个方向:“什么人?出来!”

%121
一个叹息声传来,显露出一位蛊仙,中洲的气息四处洋溢。

%122
方源和武庸看见这人,尽皆惊异。

%123
“你是……凤九歌?!”

\end{this_body}

