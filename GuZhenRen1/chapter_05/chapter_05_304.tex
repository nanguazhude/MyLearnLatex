\newsection{方源的两大利器}    %第三百零四节:方源的两大利器

\begin{this_body}

%1
虽然这图事成和现在方源扮演的男孩,乃是父子关系,但是这父亲完全不靠谱啊。

%2
什么叫做凭感觉?

%3
天赋才情也不是这样体现的。

%4
要知道一旦建造蛊阵不成,轻则蛊虫受损毁灭,重则蛊师本身也会受到反噬。若是凭感觉就能建造出蛊阵的话,现在任何五域中的蛊仙界,阵道蛊仙也就不会这么稀少了。

%5
布置蛊阵,是一项非常严谨周密的事情,就和推算仙蛊方一样。推算、建设蛊阵也是如此,需要一个过程,需要思虑周详,还需要不断试验。

%6
但是梦境就是这样,方源平静思绪,继续深入思考,面对这样的情景,似乎真的要按照图事成飞指点去做,自己该怎么办?

%7
方源想了想,问道:“这四只蛊虫,我必须要全部用到吗?”

%8
“当然不是这样。但布置蛊阵,至少要从这四只蛊虫中挑选出两只。”图事成回答道。

%9
“我根本不清楚这只土道蛊虫,如此一来,不如直接放弃掉。”方源继续思索,这样的话,摆在他面前的就只有三个选项。

%10
“快一点,不要多想,凭感觉!”方源思考的时间稍微长了点,图事成的脸色便微沉下来,催促道,“再给你三个呼吸的时间。”

%11
三个呼吸时间?

%12
而且还只有一次机会!

%13
这不存心刁难人么。

%14
方源的脑海中顿时闪过一个念头:“要用杀招解梦吗?”

%15
他现在手头上的积累,已经今非昔比,上一次他来到超级梦境探索,手头上的梦道凡蛊有限得很,导致解梦杀招的次数也有限,方源得掐算着用,能省则省。

%16
但现在不一样了。

%17
方源底气很足。

%18
因为他曾经捕获了不少的梦魇魔驹,借助这些天然的梦道仙材,再加上他坚持不懈的炼制,他拥有了非常多的梦道凡蛊。

%19
因此,解梦杀招可以使用的次数,是有史以来最多的。

%20
正因如此,方源有着前所未有的雄厚底气。

%21
“还是算了。这才是梦境中的第一层,就算失败,我还可以继续探索。解梦杀招虽然可以催动很多次,但也不是这样浪费的。”

%22
方源想了想,还是放弃了直接催动解梦杀招的念头。

%23
他开始试着催动蛊虫。

%24
先是阵心蛊。

%25
这只黄色瓢虫已经被他炼化,真元消耗下去,它顿时悬浮在方源的面前,一动不动,闪烁着淡黄色的光辉。

%26
“炎道、水道、风道……”

%27
方源沉吟了一下,既然土道蛊虫已经放弃的话,那么就只剩下这三个选择。

%28
他选中了水道的那只一转蛊虫。

%29
至于这样选择的原因很简单,因为炎道、风道的境界,都很普通,但是水道却是宗师!

%30
宗师境界,可不是随便说说的。

%31
水道蛊虫也成功地催动起来,它沐浴在黄色的光辉中,环绕着阵心蛊,不断地飞舞,时上时下。

%32
“炎道、风道……”

%33
方源选择了风道。

%34
风道蛊虫掺和进来,也和水道蛊虫一样,环绕着阵心蛊飞舞。只是风道蛊虫在外圈,水道蛊虫则更靠近阵心蛊多一点。

%35
方源等待了一会儿工夫,这三只蛊虫一只悬停在半空中,两只不断飞舞,各自催动着,还是非常的稳健。

%36
“虽然没有组合成一个真正的蛊阵,但是进展不错。”方源心中有些欣慰。

%37
打个比方的话,铺设蛊阵,就好像是搭积木。

%38
各种蛊虫是各种奇形怪状的积木,蛊师要把这些积木搭成自己心中想要的形状。这些零散的积木,就会形成高塔、楼房等等形态,发挥出远超个体的不一样的效能。

%39
方源虽然没有将这些积木搭建成功,形成一种形态,但是至少这三只蛊虫相互搭配起来,并没有倒塌。

%40
这是一个很可喜的发展。

%41
但是他的父亲图事成,却是眼中闪过一丝阴沉、失望之色。

%42
一直专注于铺设蛊阵的方源,却是没有注意到。

%43
“现在,我就要面临调整。”

%44
“阵心蛊是否一直在停留在中心位置?”

%45
“水道凡蛊和风道凡蛊的相对位置,是否可以调换?水道在内圈,风道在外围,是否可以调节成水道在外围,风道在内圈呢?”

%46
“或者说,水道蛊虫和风道蛊虫,可以相互团结,让风道蛊虫围绕着水道蛊虫转动,而不是围绕阵心蛊?”

%47
方源深入思考下去,顿时有各种各样的方案产生了。

%48
偏偏他对这些方案,能够形成什么样的结果,有什么样的效用,都是一无所知。

%49
他的阵道造诣实在是太弱了。

%50
“凭你的感觉,不要再犹豫。”图事成又催促道,神情有些不耐烦。

%51
方源咬咬牙,其实不需要他催促,方源也知道必须要行动了。

%52
原因无他,他自身的真元已经快要干涸。

%53
没办法,在梦境中,他才刚刚成为蛊师,虽然资质是甲等,但是现在同时催动了三只蛊虫,真元消耗得非常剧烈。

%54
他已经没有时间多想。

%55
“那就先试一试吧。”方源暗自咬牙,将炎道蛊虫也催动起来,加入到整个刚刚建设的体系当中去。

%56
他对刚刚的基础,没有做出什么调整。

%57
因为他一点都不清楚,调整有什么用,也不明白是不是需要调整。

%58
他是一个纯粹的初学者,什么感觉都没有,一切的尝试都是盲目的。

%59
“不过我运气不错,兴许我运气好,这一次能瞎猫碰到死耗子呢?”方源心中还残留着一些希望。

%60
但很快,他的希望就破灭了。

%61
炎道蛊虫添加进去,立即引起了连锁反应。

%62
炎道的力量和风道蛊虫相互接触,立即被增添了威力。旋即又和内圈的水道蛊虫相互碰撞,水火不相容,直接发生了爆炸。

%63
砰的一声轻响,爆炸并不剧烈,但还是把方源的脸面都炸成了一堆乌黑。

%64
整个头发都竖起来,冒着袅袅白烟。

%65
与此同时,阵心蛊直接毁灭,其余的三只蛊虫纷纷受损,受伤程度不一,水道、炎道蛊虫处于濒死的边缘,奄奄一息,风道蛊虫也是受损不轻。

%66
“你失败了!”图事成立即变了脸色,满脸寒霜,“你太令我失望了。”

%67
噗。

%68
方源眼前视野大变,被梦境排斥出来。

%69
第一次探索,宣告失败。

%70
他魂魄归体,身躯微微一晃,原本健康血色的脸面骤然苍白起来。

%71
方源立即检查伤势,发现魂魄受伤的程度,严重得有点超出他的意料。

%72
不过没有关系。

%73
胆识蛊!

%74
方源取出胆识蛊,直接捏碎用了,魂魄上的伤势顿时得到了巨大的缓解。方源又取出一只用在自己身上,顷刻间,他魂魄伤势彻底痊愈。

%75
原本微微头晕目眩的感觉,也一下子消失无踪了。

%76
正是因为有胆识蛊在手,方源才不惧魂魄因为梦境探索失败而造成的伤势。

%77
“目前我应当是天下探索梦境的第一人了。就算是影宗残余,也及不上我吧。”方源暗自估计。

%78
这全然因为他的手中,有探索梦境的两大利器。

%79
第一个就是胆识蛊,可以治疗魂魄伤势。

%80
第二个是解梦杀招,一旦使用出来,效果直达谜底,好用得有些不讲道理。

%81
胆识蛊产自于荡魂山,方源将荡魂山一直放在琅琊福地当中,就连胆识蛊的买卖,也和琅琊派合作。

%82
胆识蛊贸易是垄断贸易,独此一家别无分店,一直以来都是供不应求。

%83
最近这段时间,方源却没有利用胆识蛊去赚取利润,而是将属于自己的胆识蛊那一部分,基本上都保留了下来。

%84
目的只有一个,就是为探索梦境做充足准备。

%85
因此他手中保留了大量的胆识蛊,底蕴雄厚。也正是如此,才让他经济拮据,不得不扩大一些资源的生产规模,让自身经济好转起来。

%86
第一次失败了,没有关系。

%87
有着胆识蛊,方源魂魄立即痊愈,现在就可以进行第二次的探索。

%88
不过他没有着急,而是催动通天蛊,将心神探入其中。

%89
他开始收集情报。

%90
首先就是那只土道凡蛊。

%91
方源并不认识,他需要了解。

%92
一份《土道蛊虫大全》,方源很快从一位土道蛊仙手中买下来。

%93
这是情报交易,宝黄天中买卖的一种。

%94
有的是针对蛊虫、流派的介绍和理解,有的是对当今政局各种实事的贩卖,还有是宙道、智道蛊仙对于过去、现在、将来的推算。

%95
方源当然也可以做,凭借他五百年的眼界,做出来的情报一定很吸引人。

%96
不过这种情报,价格一般都很便宜,买卖并不长久,还会削弱方源的重生优势,所以方源一直都没有看得上。

%97
买下这份某位蛊仙著作的《土道蛊虫大全》,方源并不满足,又购买了一些类似的情报,当然还有一些阵道的凡人传承。

%98
这些花费了他近十块仙元石。

%99
对于方源而言,完全是毛毛雨。

%100
他很快在土道蛊虫的情报中,获悉了那只土道凡蛊的消息。这是一只近古时代早期,出现的蛊虫。很快就绝迹了,因为炼制它的蛊材灭亡了。情报中大略地介绍了这只土道凡蛊的效用,甚至在另外一份比较偏门的情报中,方源还获得了这只凡级土道蛊虫的蛊方。

\end{this_body}

