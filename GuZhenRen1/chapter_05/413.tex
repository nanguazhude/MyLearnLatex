\newsection{怎么可以这样赖皮?!}    %第四百一十四节:怎么可以这样赖皮?!

\begin{this_body}

%1
白凝冰等人正被困刀剑河域当中,暂时无法脱身,黄史上人见了这一良机,怎么可能轻易放过?

%2
他本来一路追杀,就积攒了一肚子的火气,如今见到了猎物,立即飞身扑杀过去。

%3
哗哗哗!

%4
光阴长河中掀起一阵阵的惊涛骇浪。

%5
其中夹杂、迸溅出无数的剑气、刀光,向影宗群仙攒射。

%6
黄史上人也不例外,遭到剑气、刀光的“照顾”,不过依凭他的防御手段,可比下方的影宗群仙,要从容得多。

%7
仙道杀招——日月神梭。

%8
黄史上人居高临下,双掌一推,无数个尖头小梭,宛若暴雨倾盆,向着影宗等人倾斜而去。

%9
“有敌!”白凝冰大声示警。

%10
杀招日月神梭攻势惊人,火龙凶猛,白凝冰等人竭力防御,被一时间打得连头都抬不起来。

%11
吼!

%12
这可惹恼了太古年猴,它猛地咆哮起来,下肢用力,整个身躯就陡然脱离了河水,扑上高空去。

%13
日月神梭打在它的身上,打出一个个的小血洞,不过对于体型庞大的太古年猴而言,只能算是轻伤。

%14
巨大的猴爪,掀起凛冽的狂风,向黄史上人抓去。

%15
感受到这股恐怖的威势,黄史上人眼皮子一跳,脸上显露出一抹惊诧的神色。

%16
“这头太古年猴竟是货真价实,不是那方源所变?!”

%17
“他们居然掌握了太古年猴这样的八转战力?”

%18
黄史上人并非力道,也非变化道蛊仙,不擅长进展。哪怕他是天庭八转,面对太古年猴的抓击,也要退避三舍。

%19
他速度非常地快,一下子就避让到了远处。

%20
在这里,宙道道痕相当浓郁,正适合他作战。

%21
太古年猴一扑未中,重新落到了河水当中,但旋即它又怒吼一声,再次猛跃,跳上空中,杀向黄史上人。

%22
黄史上人眼眸中精芒一闪。

%23
他看到了太古年猴身上的累累伤害,明白此猴已经不是完整状态。想想也是,方源等人能够到达这里,太古年猴必定出力不小,一身战力恐怕已经去了三四成。

%24
“既然如此!”黄史上人的战意越发坚决起来。

%25
他深呼吸一口气,猛地催动另一记仙道杀招。

%26
顿时,他的头顶处出现了一座土黄石块。石块缓缓流转,带动着黄史上人周围的时间,也变得缓慢起来。

%27
随后,从石块上忽然流出无数烟尘。这些烟尘尽数飞下,像是一股流光,附着在太古年猴的身上。

%28
太古年猴受到黄褐烟尘的影响,身上时间的流速下降很快,速度越来越慢,腾空的过程也被极大地延长。

%29
黄史上人见此,吐出一口浊气,面露喜色。

%30
他对自家的这记手段有着自信。随着附着的烟尘越多,太古年兽的时间就过得越慢。

%31
照此下去,很快场面就会被黄史上人彻底控制。

%32
但就在这时!

%33
轰隆。

%34
一道磅礴至极的浪水,冲击向空,竟然直奔黄史上人而去。

%35
光阴长河苍白的水浪中,一股强烈的刀光,刺眼至极,也顺势轰出。

%36
刀光正中黄史上人,黄史上人身躯猛震,立即吐出一小口鲜血!

%37
黄史上人受创,导致整个烟尘都是一抖,烟尘中的太古年猴趁机挣扎。

%38
黄史上人连忙屏住呼吸,加紧催发仙招,顿时又让烟尘更加浓厚,死死罩住太古年猴。

%39
太古年猴身上,白凝冰等人亦都在各展其能,打出冰霜、炙炎、音波,可惜都撼动不了这滚滚烟尘。

%40
在这光阴长河当中,宙道生命得到极大的加成,白凝冰等人本来就修为远差黄史上人,更有着宙道道痕的压制,战力远不够看。

%41
关键时刻,影无邪一脸沉着,就要运用无往不利的杀招引魂入梦。

%42
“嗯?”但这一刻,影无邪吃了一瘪。

%43
影无邪竟然感受不到黄史上人的存在。

%44
“引魂入梦杀招,是要需要感应对方,才能施展。否则没有对象,就施展不了。这黄史上人已经有手段,可以屏蔽我的魂道感知了吗?”

%45
“哦?紫薇大人赐予的防御杀招发动了么……呵呵。”黄史上人心中冷笑。

%46
在影无邪意欲发动引魂入梦的时候,他这边就有所感应。

%47
引魂入梦杀招的确是很强大,但是也有破绽,绝非完美的杀招。

%48
这一次,紫薇仙子既然独自派遣黄史上人,前来追击方源等人,自然要防备引魂入梦这一杀招。毕竟天庭方面,可是在这个杀招上吃了不少苦头。

%49
所谓吃一堑长一智,更何况是天庭的紫薇仙子。

%50
黄史上人有备而来,引魂入梦都无从施展。

%51
“糟糕!”影无邪等人无不变色,重重压力由心而生。

%52
他们难以挽回败局,这个时候只能依靠方源。

%53
但方源又在哪里?

%54
黄史上人思考着这个问题。

%55
他原本以为,这头太古年兽是方源的变化,借此虚张声势,但是接触之后,才知道这是货真价实的太古年猴。

%56
交战以来,方源还未现身。拥有着逆流护身印,如今战力更是和凤九歌类同的方源,无疑是黄史上人心中,最需要戒备的对象。

%57
方源越是不出现,他心中就越是谨慎。

%58
“方源你还不出来?再不出来的话,呵呵……呃?”黄史上人错愕。

%59
就在这一刻,光阴长河中掀起滔天的巨浪,直接向他盖来。

%60
“又来?!”黄史上人不由气闷。

%61
这巨浪不扑击体型巨大的太古年猴,反倒是三番两次,来找他的麻烦。

%62
这运气也太差了!

%63
轰隆!

%64
巨浪打中黄史上人,黄史上人再次吐出一口鲜血,伤上加伤。

%65
他躲闪不及。不仅是因为催动这这个仙道杀招,而且在这个浪涛扑来之时,时间骤缓,就算是他躲闪了水浪,里面迸射而出的剑气、刀光却是迅疾无比,难以闪躲。

%66
黄史上人选择硬抗,争取战机,用来擒拿活捉太古年猴和白凝冰等人。

%67
“现在占据上风的是我,我就不信你方源还能再憋下去!”黄史上人心中发狠。

%68
哗、哗、哗!

%69
三声巨响,又起三道水浪,直冲黄史上人而去。

%70
“什么?还来我这里?!”一瞬间,黄史上人瞪大双眼,骂娘的心都有了。

%71
按照常理而言,体型巨大的太古年猴,会更加可能遭受巨浪拍击。但自从开战之后,就只有黄史上人屡遭浪拍。

%72
“他们的运气也太好了吧?没有一道水浪拍中他们?”

%73
“等一等。”黄史上人忽然心中一凛,“难道是方源的运道手段在作祟?”

%74
他想到一种可能。

%75
“哼!就算如此……区区运道,也难以改变大局。”

%76
“运道在这里,本来就受到压制。方源要操纵这里全部的巨浪,凭借运道根本不可能,唯有宙道的无上手段才行。”

%77
“我在这里还是有着地利的优势!”

%78
黄史上人稳定心神,继续镇压太古年猴和白凝冰等人,逼迫方源现身。

%79
“黄史上人中计了。”

%80
“一如宗主所料……我们继续轰击,不要暴露出破绽。”

%81
白凝冰等人一脸沉重的神色,但实际上却在暗中相互交流。

%82
与此同时,在远处河底的岛上。

%83
方源的面前,影像翻腾,正演绎着黄史上人和太古年猴的战斗景象。

%84
这里正是石莲岛,幽魂魔尊曾经掌握的红莲真传所在。而方源已经提前到达!

%85
方源的身上,紫薇仙子的侦查杀招已经解除。而白凝冰等人,虽然有洁身自好仙道蛊阵,但在时间上却远远来不及。

%86
带着她们,方源的行踪便是暴露的。即便是塞在至尊仙窍当中,也会被推算而出。

%87
就像当初,方源依凭气运交感杀招,能够推算出影无邪等人的位置。即便是她们钻入仙窍躲避,也会被方源算到。

%88
仙窍虽然是小世界,隔绝内外。但若是仙窍中有着明显的线索,就会被推算到。

%89
方源能够做到,紫薇仙子当然也能。天意也可以。

%90
只要仙窍中有着天意残留,那么五域两天的天意,都会察觉到方源的位置。

%91
这点情报,方源早就在很久之前,就亲自探查明白了。

%92
这一点虽是弱点,但在方源看来,也有可以利用的地方。

%93
借助影无邪等人,稍稍麻痹了紫薇仙子,让黄史上人错误估算方源的行程。虽然这点错误微不足道,但是在方源借助红莲真传之后,便将这点小小的优势迅速放大。

%94
没有错。

%95
红莲真传有着可以操纵附近河域的威能。

%96
甚至,这片刀剑河域,便是红莲魔尊刻意谋划出来的。

%97
只是要发动这项威能,会损耗石莲岛的根本。而如今一战之后,这座原本就残破不堪的石莲岛将彻底陨灭。

%98
“幽魂意志,我来助你一臂之力吧。”方源望了身边的幽魂意志一眼,见他形态稀薄,操纵艰难,便伸出手来。

%99
顿时,一大股意志喷涌而出,汇入到幽魂意志的体内。

%100
幽魂意志得此强助,立即低喝一声。

%101
与此同时,战场上猛地掀起一股骇人的巨浪。

%102
“这这这!”黄史上人看得眼珠子都要瞪出来。

%103
这巨浪的规模前所未有,里面蕴藏的刀光、剑气还未喷发,就让黄史上人感受到强烈的生命威胁。

%104
黄史上人咬牙发狠,这样规模的巨浪,虽然会拍中他自己,但无疑也会殃及太古年猴,甚至逼出方源。

%105
“我要和你继续耗下去!”黄史上人正这样想着,然后就看到巨浪分出一道缝隙,直接让过太古年猴,然后全数冲向他。

%106
“什么!?”

%107
黄史上人几乎要大骂出声。

%108
怎么可以这样赖皮?!

\end{this_body}

