\newsection{家族里不是只有规矩}    %第五十八节:家族里不是只有规矩

\begin{this_body}

“族长,您找我有什么事?”学堂家老恭敬地站立在一旁。

“也没什么事,坐罢,我这里有一个故事想说给你听。”古月博眯着双眼,语气悠悠。

“属下洗耳恭听!”学堂家老便选择了下首,最靠近族长的位置坐下。

族长古月博便说了人祖的故事。

话说――

自从人祖道出了正确的名字之后,就降服了规矩二蛊。他的第一个命令,就是叫它们为自己捕捉一只寿蛊。

规矩二蛊一方一圆,合力之下,能捕捉天下万蛊,一只寿蛊自然不在话下。

人祖用了寿蛊,顿时重获青春,回到了二十岁的年龄。

但此时规蛊却对他道:“人啊,你虽然降服了我们,但是你每次命令我们,都要增加一条规矩。

矩蛊也跟着道:“我们为你捕捉了寿蛊,这是第一次命令。我们的新规矩就是,不会再为你捕捉重复的蛊虫。”

也就是说,若是今后人祖再要捕捉寿蛊,规矩二蛊是不会出手的。

人祖点点头,只能答应下来。

他开口下了第二个命令:“那么除了寿蛊之外,就请你们将天下万蛊,都替我捕捉过来吧。”

规矩二蛊得了这个命令,顿时规蛊变成无穷大的一个圆,囊括了宇和宙。矩蛊变成无穷大的一个方,涵盖了大世界。

一方一圆组成一张巨网,将整个天地乾坤都笼罩住。

当它们重新缩小,回到人祖手掌心的时候,天下万蛊除了寿蛊之外,都被它们捕捉到了。

人祖大喜,这样一来,天下所有的蛊虫都归属自己了,从此以后他将是世界之主!

然而当他打开方圆的丝网时,哗啦一声,一股巨大的虫流向外喷涌而出。规矩二蛊辛辛苦苦捕捉来的蛊虫,都争先恐后地飞走了。

当人祖连忙地合上方圆的网时,里面只剩下五只蛊虫了。

“这是为什么?”人祖很惊诧。

规矩二蛊便回答他:“人啊,天底下的蛊虫成千上万,各式各样,你一没有力量,二没有智慧,如何降服得了它们呢?我们只能替你捕捉蛊虫,你要靠自己降服它们,才能让它们为你效力。”

然后它们又道:“这是你第二个命令,我们也要再加第二条规矩――从今往后,我们一次只能为你捕捉一只蛊。”

人祖只好点点头,他小心地掀开丝网,只露出一条缝隙。

剩下的五头蛊虫中,就有力量蛊,还有智慧蛊。人祖看到这里,很是欣喜。

他就对力量蛊说:“力量蛊啊,你当年离开了我,现在有没有后悔?你现在臣服我,我就能还你自由。”

力量蛊却说:“人啊,你错了。我之所以没有飞走,不是没有机会,而是想留下来。你要降服我,是不可能的。我只臣服于力量比我强大的存在,而你不行。不过我们可以再做交易,你把你的青年给我,我就可以暂时听从你的命令。”

人祖听了这话,有些不舍。自己刚刚得到了青春,难道就要失去了吗?

但他十分渴望力量,他知道拥有了力量之后,自己就会变得强大,生活也会变得容易。

再说,拥有了力量,才能降服更多的蛊虫。

于是人祖就再次答应了力量蛊,和它达成了第二次交易。

人祖一下子就到了中年,力量蛊从规矩的网中飞了出来,落在了人祖的肩头。

人祖有了力量,底气顿时足了。

他又对智慧蛊道:“智慧蛊啊,你当年离开了我,现在有没有后悔?你现在臣服我,我就能还你自由。”

智慧蛊却道:“人啊,你错了。我之所以没有飞走,不是没有机会,而是想留下来。你要降服我,是不可能的。我只臣服于比我更智慧的存在,而你不行。不过我们可以再做交易,你把你的中年给我,我就可以暂时听从你的命令。”

人祖听了这话,却不想再交易了。

他比上一次更珍惜生命了,而且他也知道,一旦中年也卖出去,他就只剩下老年。然后过不了多久,力量蛊和智慧蛊又会离他而去,就像上次那样。

人祖不愿意做交易,但又不想放了智慧蛊。

智慧蛊有些急了,只好退让一步,道:“好吧,人,你赢了。我这一次败在了你的手上,只要你告诉我你是用什么方法捉到的我,我就承认失败,不收你任何的东西,从此为你所用。”

人祖听了这话顿时大喜,规矩二蛊都没有来得及阻止,他就脱口而出:“我是用规矩二蛊,捕捉到你的。”

智慧蛊听了哈哈大笑:“我记住了,原来这两只蛊的名字叫做规矩。哈哈,我现在知道了你们的名字,你们再也捉不住我了。”

说完,它就化作一道光,飞了出去,一眨眼就消失不见了。

规矩二蛊都抱怨起来:“人啊,我们老早就告诉过你。我们的名字你最好一个人知晓,不要让其他存在知道。否则我们就要为别的存在所用了。现在好了吧,智慧蛊已经知道了我们的名字,事情麻烦了。”

人祖这才知道自己上了智慧蛊的当,他十分懊悔,他知道他丧失了用规矩捕捉智慧的唯一机会。

说到这里,古月博的故事也暂告一个段落。他看向学堂家老,目光中蕴藏着深意。

学堂家老不禁从座位上站起来,人祖的蛊师他早就听过,但是现在古月博所讲,自然有其深意。

他目光闪了闪,心中早有所领悟,当即向古月博微微弯腰,恭声道:“族长大人,您说这个故事,莫非是打了一个比喻?把方源比作智慧蛊,把家族当做人祖。人祖虽然用规矩将智慧蛊捉住,困在了网中,但是最终还是让智慧蛊逃走了。”

说到这里,学堂家老顿了一顿,思索了一下,又看向古月博:“难道族长大人,是要我放过方源,不再打压他?可是他做的越来越过分了……”

古月博止住学堂家老的话,伸手示意让他坐着说话。

学堂家老重新坐下,就听古月博一声叹息:“你啊,还是这么有悟性,是个聪明人,说话一点就透。可惜心胸格局还是不大,你只看着你那一亩三分地。现在我告诉你,学堂是小的,家族才是大的。”

“我知道你担心什么,害怕其他学员被方源死死压着,最终心气劲儿都被压没了。呵呵呵。”古月博微微摇头,指了指学堂家老,“你多虑了。”

“你以为家族是干什么的?就单靠你这个学堂培养新人吗?不是的,在每一个学员的身后,还有他们的父母,他们的长辈,他们的好友。只要有这些人在支撑着,鼓励着,期望着,我们古月一族的后辈的心中就有底气和劲头。”

“方源的确屡次出我意料,有一丝横空出世的意味。我一直在暗中关注方正,也早就知道他方源抢劫勒索的事情。就让他继续抢好了,用他好好磨一磨方正、漠北还有赤城这些璞玉。这很有好处啊,至少此届学员是我见到的拳脚功夫最扎实的一届了。”

学堂家老却面现忧色:“可是族长大人,挫折太过也不好啊,会把璞玉磨碎的。尤其是方源现在有了酒虫,此虫对一转蛊师有极大的帮助。我担心在这一转期间,在方源的压迫下,其他学员都翻不了身。”

“那就让他们翻不了身!”古月博冷哼一声,流露出一丝上位者的大气和冷酷,“这点挫折算什么?比死亡更可怕吗?有家中长辈的扶持,心气劲还存不住,就说明根本不是璞玉,也就没有培养的价值。家族每年都有新的学员进入学堂,此届不成,还有下一届。至于古月方正……从明晚开始,我会暗中亲自指导他。”

“有族长亲自教导,那是古月方正的幸运啊。”学堂家老适时地拍了一记马屁。

古月博脸色缓和了一丝,看着学堂家老,叮嘱道:“你知道你当了数十年的学堂家老,现在仍旧还是学堂家老,是什么原因吗?气量放大一点,我知道方源触及了你的威严,扫了你的面子,但是你何必跟一个后生晚辈斤斤计较呢?”

“我也知道这方源是有些早智,但终究还是少年性子,年轻冲动。要不然也不会伤了侍卫,当众让你下不了台。他心里是憋着一股气啊,这可以理解。从天才堕落到凡人,敌视家族很正常。”

“他其实还是很幼稚的,你看他试图隐瞒酒虫就知道了。酒虫怎么可能隐瞒得住?他并不成熟,还有天真的想法,不要把他想得那么可怕。我把他比作智慧蛊,是高抬了他。他顶多是有些小聪明,大智慧是没有的。若是他不动声色地隐瞒中阶修为,或者无动于衷地接受了班头职位,那才真的叫心机深沉。”

“族长你的意思是?”学堂家老扬起眉头

“我的意思就是,方源对家族不满,那就让他发泄去吧。蚂蚁向大象吐吐沫,大象会在乎吗?情绪宜疏不宜堵,他发泄完之后,自然会融入家族之中的。我们古月一族,自从一代开创以来,已经近千年。对家族不满的,大有人在,但是最终他们掀翻家族了吗?”

“家族不是只有规矩而已的,还有血脉之情。人祖想要用规矩逮住智慧,结果他不仅失败了,而且规矩都被智慧知晓了。此中的寓意深远啊……规矩是死的,人是活的,情是深的。你若一味地用规矩捉人,反而会增加仇恨,让方源离家族更远。方源只是丙等,侥幸的话,数十年后就是个低层家老罢了。但他毕竟是古月方正的孪生哥哥,你明白吗?”

“懂了!”听到古月博最后一句话,学堂家老顿时恍然大悟。

“嗯……家族如果只是规矩,那就是一摊生硬的死物。但是若增添血脉之情,那就活了。”古月博缓缓点头,“还有一句话,要送给你,希望你能牢记。”

“请族长大人训示。”

古月博目光悠悠,望着窗外的一地月光:“海纳百川,有容乃大……记住它,然后退下罢。”

“是,族长大人,属下告退了。”学堂家老唯唯而退。

\end{this_body}

