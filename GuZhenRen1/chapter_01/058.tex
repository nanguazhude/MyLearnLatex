\newsection{不管三转四转,都是猴子}    %第五十九节:不管三转四转,都是猴子

\begin{this_body}

今夜的月亮,特别的圆。

月光如辉如纱,披洒在青茅山上。

宝气黄铜蟾每一次纵跃,都足足有一百米的距离。因为是跳跃式前进,陡峭狭窄的山路并不能对它形成限制和阻碍。www.13800100.com

贾富一行人,都坐在宝气黄铜蟾的身上,从古月山寨出来,重新往商队赶去。

耳边的风呼呼作响,视野随着宝气黄铜蟾的纵跃而上下颠簸。

月光照耀在众人的脸上,每个人都沉默着,贾富更是面寒如霜。

过了好一会儿,一位心腹之士忍不住这种死寂,对贾富觐言道:“主子,这可如何是好?贾金生死了,主子回去之后,该怎么向老爷交代呢?是不是先找个替罪羊……”

贾富摇摇头,却左顾言他:“你知道人祖的故事吗?”

心腹一愣,未料到贾富会如此回应。一时间竟不知道该怎么回答他好。

贾富却继续道:“人祖有规矩二蛊,就捕捉天下万蛊,得了力量,失了智慧。此时他的网中,还剩下三只蛊。他打开一看,这三只蛊分别是态度蛊、相信蛊还有怀疑蛊。人祖不愿意放它们离去,这三只蛊只好和人祖商量赌一把。只要人祖一打开网口,它们就分成三个方向同时分出去。谁被人祖抓到谁就降服。你说,人祖最后抓到了什么?”

心腹似有所悟,答道:“是态度蛊!”

“知道为什么吗?”贾富又问。

心腹摇头。

贾富嘿然一笑:“因为态度能说明一切啊。不管父亲是‘相信’也好,‘怀疑’也罢,我已经表明了‘态度’。贾金生失踪,我立即在商队展开调查,一有线索,就马不停蹄赶回古月山寨。在山寨中,我冒着被古月一族围攻的危险,当面对质。坐都不坐,为了验证方源的话,连四转的竹君子都用了。”

“回到家族之后,我还会重金聘用捕神,邀请铁血冷调查此事。不管贾金生是死是活,我这个当哥哥的已经做到了该做的事情,态度已经表明了!我刚刚已经想通了,不需要替罪羊,就这样坦诚地回去,因为这事情的确不是我做的!找了替罪羊,说不定就落入了他贾贵的算计里。我能找人顶罪,他自然也能找人翻案。”

心腹暗暗吃惊:“主子,你是真的怀疑贾贵少爷干的?”

“哼,不是他还能有谁做得这么出色?”说到这里,贾富脸色扭曲,双眼中怒火直欲喷射而出,“先前我是顾忌兄弟之情,没有对他这么做。想不到他如此阴险,既然你不仁,就不要怪我贾富不义!”

他此时却不知道,在远处一直有一双眼睛,远远地目送着自己远去。

方源站在山坡上,静静地看着。

今夜月色真是美丽皎洁。

金黄色的圆月高悬在夜空,照耀得群山大地一片空明。

近处,青山郁郁葱葱,百草茂盛。满山的松柏杉树,还有青茅山特有的青矛竹,一片连着一片,一丛接着一丛。大片的黑绿色从山顶一直倾泻下来,流淌到山麓下。

远山,则连绵一片,在月光下模糊成一片沉沉的黑影。

盘曲的山路如羊肠般迤逦而下,时而被树林遮挡住,一直延展出去。

贾富一行人乘坐着宝气黄铜蟾,沿着山路前行。不断地纵跃之后,最终身影被树林遮盖。

虽然山地,不能限制宝气黄铜蟾的速度。但是贾富也不敢胡乱穿越青茅山,若是闯入兽巢当中,凭他四转修为也要狼狈不堪。因此顺着山路前进,最为妥当。

就在不久前,方源就是在这片山坡上撑着雨伞,目送商队离开。如今他又站在这里,看着贾富远去。

“杀死贾金生的麻烦,终于解决了。”他双目幽幽,心湖波澜不惊,一片平静。

自从那晚杀死贾金生之后,他就在思考,如何善后。

他心中十分清楚,自己毫无根基,若是真相揭露出来,古月一族必定会牺牲自己。但是一味的隐瞒,更是纸包不住火。

高明的谎言,都是真真假假,假中有真,真中带假。

必须祸水东引!

这局面就好似一个棋盘,两边对峙。一边是贾富的商队,一边是古月一族的山寨。在这个局里,不管是古月博,还是学堂家老,亦或者贾富都是棋子,哪怕是他方源也是其中的棋子。

想要保住代表自己的这颗棋子,唯有利用两方棋子的对立,在缝隙中寻找到一丝机会。

从几天前,方源就开始布局。

他先借助那两个侍卫,在学堂家老搭起来的戏台上上演了一出好戏。又隐藏了酒虫的存在,就是勾引出族人的好奇心,引起广大反响,吸引高层瞩目。同时,让学堂家老进行暗中调查。

其次打劫同窗,表现出自己的冲动、桀骜和对家族的不满,“示弱”给古月高层看。

再然后算计着日子,就等到了贾富。

当堂对质中,他表演出了自己的幼稚和惊惶,这反而在无形中引导了众人的思维。让众人自己发现了“真相”。

最后他利用古月一族和贾富之间的利益对立,让怀疑并一直调查他的学堂家老,反过来为自己作证。

竹君子是个小小的意外,但终究也是四转,春秋蝉的气息镇压下,竹君子反倒讽刺地成为了方源的最佳证明。

最终,方源不仅完美解释了酒虫的由来,而且将黑锅盖在了无辜的贾贵身上,自己则安然无恙地破局而出。

“学堂家老被留下,看来古月博是想插手学堂事务,取消对我的打压。依他的格局,倒是有这种气量。不过他的真正目的也不在于我,主要的出发点应该是古月方正。我把事情闹大的目的之一,就是引出风波,让高层注意。古月博不出面,也会有古月漠尘、古月赤练为了维护名誉而出手。”

“至于贾富,他现在应该已经认定,贾贵就是凶手了。复仇的火焰已经在他胸中燃烧,呵呵,真是期待啊。经过我这么一插手,这对兄弟的斗争立马就要升级了。不知道那场斗蛊大会是否会提前呢?”

“对了,还有那个神捕铁血冷。铁血冷……哼。”方源在嘴中咀嚼这两个字,半晌后,轻轻一笑,“在正道,此人倒是个人物。可惜他事务缠身,繁忙得很。单因为这件事,要请他来可不容易。贾富要表明态度,一定会请他过来,不过时间就说不准了,至少得排在两三年之后吧。”

两三年后,他就是二转、甚至三转的修为。到那时,人生就是另一番天地了。

夜风袭来,山间清新的空气,透着一股芬芳。

方源呼吸着,越加神清气爽。

极目远眺,视野开阔,漫山如画,月色下一片宁静祥和。

“明月松间照,清泉石上流。”方源轻吟一声,不禁就想起地球上的一则寓言。

猴群逐月,看到井中有月,就想捞月。后面的猴子抓住前边猴子的尾巴,前边的猴子又抓住更前方猴子的尾巴,就这样一个个串起来,最前面的那只猴子终于碰触到井中水面。

一伸手,水晃月散。

这世间的人,亦常常如此。看到月影,就以为是真月。

殊不知,只是井中之月,眼中之月,心中之月罢了。

“此生就愿成真月,出天山,戏云海,照古今,行走在黑暗的诸天之上。”方源双目澄澈,瞳眸中倒映着锦绣青山。

山坡上,身材瘦削的少年默立着。

一轮黄金的月轮,如一张圆盘,高悬在夜空之中。

它万古长存,经行夜空,把少年渺小的影子,淡淡地印在青石上。

(ps:今天是三江榜最后一天,大家有三江票的,不要浪费了,都投给我吧。就目前而言,看来今天是要五更了,我正在努力!)

------------

\end{this_body}

