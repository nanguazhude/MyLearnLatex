\newsection{扫强敌方正展风采}    %第八十三节:扫强敌方正展风采

\begin{this_body}

方正和漠尘双双走上了擂台。

“方正,别以为你有二转修为,我就会输!今天我就要越级挑战。”漠北咬着牙,脸色凝重,在心中默默地为自己打气。面对二转修为的方正,他的确感到了一股压力。www.13800100.com

“来吧。”方正低喝一声,猛地冲出。

漠北心头一跳,这方正不按套路出牌啊。一般都是先对射,再拳脚。他这一次,居然一出手就冲过来,是想比拼拳脚吗?

“他不怕在比斗拳脚的时候,被我的月刃射伤吗?”漠北大为疑惑。

他当然不是为方正担心,而是知道自己如果和方正贴身战斗,在那么近的距离下,方正若是射出月刃,他根本来不及躲避。

漠北连忙后退,企图拉开距离,同时手腕翻转,切出一记月刃。

方正临危不乱,一个扑地打滚,闪过了月刃,继续冲上去。同时他的手中,也浮出一团月华。

漠北看着他掌中凝而不发的月光,心中不由地生出一股紧张情绪,连连后退。

他虽然苦练基础拳脚,也锻炼过月刃,但终究比一直受到族长亲自培养的方正,要稍逊一筹。

漠北适应不了这样的战斗方式,顿时就陷入下风。

“咦?那边有点意思。”擂台上的这场战斗,吸引了不少人。

“居然如此接近,这个方正胆子不小。”药红现在可以分辨出方源和方正。方源表情冷漠,散发着一种成熟阴郁的气质。而方正则面容坚毅,透着阳光。

“应该是族长调教的功劳吧。普通学员对决,双方距离往往是十米。超过这个距离,月刃就会消散。少于这个距离,月刃射过来,学员们又来不及反应。”古月青书目光闪了闪,“方正现在的战斗距离,已经被压缩到了六米。躲闪月刃的工作,也是熟稔无比。看来不仅是族长重视方正,方正也下了苦功,吃了很多苦头。”

“小弟!”漠颜看着台上漠北被方正逼入下风,她一脸的担忧和焦急,真想上去帮忙,揍方正一顿。

而赤山面无表情,只是看着,没有说话,

方正接近漠北,将距离压缩到六米之后,就不再近身,而是使用月光蛊,进行对射。

漠北仓促应付,手忙脚乱,有好几次都差点被月刃扫中,险象环生。

反观方正,他却有底气。

即便是躲闪不及,他还有玉皮蛊这个底牌。只要撑起翠绿玉光,就能防住月刃。

看着漠北被自己如此压着打,方正的脑海里不由地就浮现出往昔。

月夜下,族长对谆谆教导,手把手地教导躲闪动作,毫无保留的传授经验。

“族长大人,我不会让您失望的。”方正双目精芒闪烁,他越战越勇!

“方正你既有天资,又吃苦耐劳,勤学苦练。能有这样的战果,是你一滴滴的汗水积累而成的。这都是你努力的结果。就是这样,方正,就是这股气势,去吧,去展现你的光彩!”棚子下,族长心湖荡起微微涟漪,表面上他坐在那里,静静地观看着,嘴角上露出一丝微笑。

漠北虽然极力挣扎,顽强抵抗,但是一刻钟后,他的身上已经增添了无数的伤口。涌出的血液,染红了他的衣衫。

场下主持局面的蛊师,看到此情景,开口宣布:“此场,古月方正胜。”

“我还没有输!”漠北执拗地大叫着,他浑身浴血,摇摇晃晃。但挣扎了几下后,还是被上台的几位蛊师强制带下去治疗了。

“这种程度的战斗,已经是毕业一年后的水准了。”

“甲等天才,到底是天才。”

“据说被族长大人亲自教导的,能不厉害么?”

看到这样的结果,擂台下的蛊师们发出一阵阵的赞叹声。

古月方正喘着气,走下擂台。三个蛊师也围了上来,为他治疗,同时免费提供元石,供他快速回复真元。

休息了一段时间后,方正恢复到完全状态,再次登上了擂台。

这一场,他要对战古月赤城。

赤城看着方正,干笑了两声道:“很好!方正,你败了漠北那小子,现在我击败你,就一举两得了。”

他似乎信心十足。

方正抿着嘴,一言不发,继续冲了过去。

“龙丸蛐蛐蛊!”赤城意念一动,顿时双腿上浮现出一层橙红色的微光。他轻轻一蹦,瞬间就后撤十米之远。

方正刚刚接近的距离,顿时就被拉大了。

“嘿嘿嘿。”赤城得意地笑起来,“方正,你没有蛊虫增速,单靠两条腿,是追不上我的。这擂台虽然不大,但足够我腾挪的了。你的战术对付漠北有用,对我来讲,就失效了。”

“是这样子的吗?”方正索性停住了脚步,站在原地,目光灼灼地盯着赤城。

他的脸上现出笑容,双眼流露出坚毅之色,大声地道:“你尽管就这样躲闪好了,你每使用一次龙丸蛐蛐蛊,你就要消耗一定的真元。你不过是一转巅峰的青铜真元,我却已经是二转的赤铁真元,比你耐用三倍。你的资质又不如我,最后没有真元,输的一定是你!”

你!”赤城不禁动容,他只看到了自己的优势,却没有看清自己的弱点。现在,他不得不承认方正说的有理,斗志立遭重挫。

“什么,方正晋升二转了?!”周围蛊师一片哗然,修为检测的结果,昨天才刚刚得到,只在小范围内流传,他们中的大部分人还不知道这事。

“甲等天才到底是甲等,了不起。这个方正,搞不好真的是我族的希望之星啊。”

“白家的那个天才白凝冰,真的是太强大了。如果方正能成长起来,兴许就能对抗白凝冰。”

“这个小子有点意思。在学堂就修炼到二转,又有如此扎实的基本功夫,这情况真的比较少见。难怪族长倾注了这么多的心血。”药红口中呢喃。

青书亦是感慨:“族长的培养只是一部分,你不要小看这个小子,自从他从暗杀中惊险生还,他就变得相当刻苦,修行十分努力。他是个好苗子,既有天赋,又十分踏实。若真的能成长起来……唉,我感觉到我的担子越来越重了。”

“呵呵呵。”族长古月博忍不住轻笑出声。

赤城的性格弱点,他曾经向方正剖析过。如今看到方正能实际运用,这让古月博感到分外的欣慰。

“看来方正,应该就是此届第一了。”在族长的身边,古月漠尘开口道。

古月赤练冷哼了一声,紧紧地盯着场中。他当然希望自己的孙子,能够战胜方正。这样一来,就为赤家挣了脸面。

但是事与愿违,擂台上,赤城斗志受挫,和方正正式交手之后,他反而发挥不出原有的水准,失误频出。

最终,赤城被方正抓住一个失误,扫下来了擂台。

“古月方正胜!”蛊师叫道。

古月赤练脸色铁青。

一时间,方正万众瞩目。风头无两。

“方正连续击败漠北、赤城,有勇有谋,看来此次第一非他莫属了。”有人赞叹着。

“有道理,我也很看好他。可惜他是不可能加入我们小组的。”小组挑选学员,学员也在挑选小组。像方正这样的优秀种子,早就被内定了。

“方正是族长一系,漠北、赤城分别是漠脉、赤脉的未来掌权人。看来今后,古月一族将是族长一系力压两大家老了。”有人则看得更加深远。

另外的两个擂台上,战斗仍旧继续着。

方正早早地走下擂台,听着周围人的对自己赞叹话语,他心潮澎湃,明显地感到自己有一种脱胎换骨的感觉。

不同了,和以前不一样了。

他的心中充满了被认同的激动,被赏识的兴奋,被肯定的开心。

一股股冬风吹来,在冷冽的风中,他却感到了春天般的温暖。

“最后一战,古月方正对战古月方源!”片刻之后,一位主持擂台的蛊师喊道。

\end{this_body}

