\newsection{大自然没有无辜}    %第六十八节:大自然没有无辜

\begin{this_body}

两个年轻猎人拿着这些竹纸,双手都在颤抖,双眼都在隐隐放光。

“这可是王猎头一生经验和心血啊。我们猎户之间,虽然交流,但也是交流陷阱摆放的位置。却不会告诉对方兽群的分布情况。这竹纸上的,都是王猎头从祖上就开始积累的情报啊。”

“原来这山谷中有一群野鹿,哈哈,我杀了这群野鹿,至少三个月不愁吃喝!啊,这溪流旁边,有一窝山熊?好险,我上次就在这附近狩猎的。记下来,都要记下来!”

这些珍贵的情报,可以说是猎人吃饭的饭碗!

往往不是一代人,而是祖上父辈不断地用鲜血和生命换来的经验,积累下来的东西。

而王老汉一家,一直以来都是打猎为生。到了王老汉,祖业就达到了巅峰,是公认的猎手第一人。

这样的人手中的情报,自然是最详实的。

两个年轻的猎手足足查看了一刻多钟,翻开了几遍,直到方源催促了一声,他们俩这才依依不舍地交还了竹纸。

在这期间,王老汉一直跪在地上,额头触地,表示着恭敬。而少女则仍旧躺在地上,傻了一般。

“没有问题,大人。”

“这竹纸上面的陷阱位置,都是正确的。”

两人回答道。

“蛊师大人,事关老汉和女儿的性命,老汉绝对不敢欺骗你的啊!”地上跪着的王老汉连忙喊着,不断向方源磕头。

“嗯,不错。”方源抖了抖手中的这叠竹纸,忽然话锋一转,“可是我,不信啊。”

王老汉触电一般,猛地抬头,就看到一片幽蓝的月刃,在他的瞳孔里越扩越大。

哧。

头颅飞起,鲜血飞溅。

“啊!!!”

“大人,这!”

两个年轻猎手猝不及防这样的变故,脸上充满了震骇之色。

“爹――!”少女发出凄厉无比的惨叫,她扑向王老汉的无头尸体。但就在半途中,一片月刃射来,正中她的脸。

扑通。

她一下子就倒在了地上,气息全无。

她那娇美的脸庞,无力地从额头眉心到嘴唇下巴,慢慢地沁出一条细细的红线。

红线越扩越大,猩红的血液缓缓地渗出,顺着她半边鼻子,半边的嘴唇,往下流淌。流淌到黑色的泥土当中,将她的半张脸面都涂染成了鲜红。

而她的另一半脸,则仍旧娇美如初,白里透红,映照在蓝色的天空下,显得晶莹剔透,就像是一个艺术品。

“倒是有点姿色。”方源淡淡地看了地上的少女一眼,满意地点点头。

用一转中阶的真元,催发出的月刃,就能削蛊。如今,用了高阶真元催发月刃,能直接断骨,甚至能斩铁!

“王家妹子!”一位年轻的猎人亲眼目睹着心上人香消玉殒,顿时无力瘫倒在地上。

“蛊师大人,饶我们一命啊!”另一个猎人看到方源转身盯着他们俩,差点吓得魂飞魄散,扑通一下就跪倒在地上。

“都给我起来,进去,搜!”方源冷声道,“我知道但凡猎户家里,都保留着一张兽皮地图。上面就画着地形,还有陷阱位置,以及野兽分布。你们把它给我搜出来,我就放你们一条生路。”

“是是是,我们这就去搜。蛊师大人,请您稍等片刻!”两人手忙脚乱地爬起来,跌跌撞撞地跑进了木屋。

木屋中很快就传来了一阵翻箱倒柜的声响。

但是过了片刻,两个年轻人将木屋里的东西翻遍了,都没有找到兽皮地图。

“大人,您再等等,我们马上就能找到了!”两个年轻人脸上充满了恐惧和慌乱,他们的动作越来越粗暴,桌子碗筷等等都被他们砸碎。

“该死的,到底在哪里?”

“快出来,快出来啊!”

他们口中喃喃,紧张得浑身颤抖,双眼充满了血丝。

“没用的东西。”方源慢慢地走进这间木屋。

“大人!大人!饶了我们吧,呜呜呜……”两个年轻人浑身巨颤,像是触电一般,软倒在地上,对着方源叩拜哭泣。

方源没有搭理跪在地上的这两个人,而是打量这个木屋。

木屋内分了四间卧室,一个大厅,一个厨房。所有的家具摆设,都像垃圾般胡乱地堆积着,如同被洗劫过一般。

方源慢慢地踏着步,他的脚步在木屋的地板上发出咚咚的轻响。

“的确是被翻遍了,但是不应该啊。几乎每个猎人,都拥有一块兽皮地图,从祖辈开始延续下来,一代又一代人不断在地图上刻画修改,记录着兽群的分布,还有陷阱布置,这是猎手的饭碗。怎么可能会没有?”

方源双目幽幽,不断思量:“况且我刚刚试探了那王老汉,故意让这两个年轻猎人进屋去搜纸笔。王老汉立即叫出纸笔的摆放位置,他应该是在担心这两个人搜出兽皮地图。兽皮地图一定就在这个木屋当中!”

方源扫视一圈,忽然灵光一闪,目光定格在大厅里的壁炉上。

这壁炉连着烟囱,是在秋冬季节用来取暖的。壁炉中还有一些烧剩下来的黑色木碳。

方源走到这壁炉旁,慢慢地蹲下,随手拿起一旁靠着壁炉的铁钳,将一一根根烧成黑灰色的木炭翻开。

大多数的木炭还保留着木条的形状,质地很脆,稍微一用力,就断成了两截。

“嗯?”方源翻检了一番后,忽然发现其中有一块木炭,质地很硬,而且很重,不像其他的木炭那么轻。

他用火钳将这块长条木炭取了出来,摔在地上,表面的木炭立即碎成了无数的黑渣。然后露出里面的竹筒。

那两个年轻猎人看到此景,顿时都发出一声低沉的惊呼。

方源拿起这个竹筒,找到一端的盖子,将其拨开。稍微一倾,就倒出了里面一个地图。

这地图比那叠竹纸重多了,是用一张白色的兽皮做的。兽皮地图很大,长度要有一米,宽度有半米,上面用黑色、绿色、红色、黄色、蓝色等等线条,勾勒出一副复杂的地形图。

方源双眼一扫,微微有些惊奇。

这地图范围很广,有些边角都已经离着山寨很远,作为一个凡人,能探测到这些地方,还真的是不容易。

他目光一凝,在这地图上发现了五个山猪群的分布地点。

两个小型野猪群,两个中型野猪群,还有一个大型野猪群。在这大型野猪群的中央,用红色的线画着一个叉。

看到这个叉,方源不由地冷笑一声。在他的竹纸上,就绝对没有这个叉!

那两个年轻猎人也看过竹纸,没有发现问题。这是他们见识有限,只清楚少数地方罢了。这个红色的叉,已经远在地图的边角。从这点上,就可以看出那王老汉的奸猾!

这张兽皮地图就是方源动手杀人的原因。

他要猎杀野猪,就需要一张这样的兽皮地图,但是别人画出来的他怎么能放心?这样亲手得到的地图,才更可靠啊。

方源前世,什么样的话没听过?单凭一个“滚”字,还不至于让他杀了王二。

他在陷阱旁,听到那四人的对话之后,当即就起了杀心。

杀死王二,减少阻力,才能更顺利地得到这张地图。那为什么不杀呢?

方源不为杀而杀,杀只是一种手段。用了这个手段,能更直接地解决问题,那为什么不用呢?

王老汉也肯定是要死的,杀子之仇就算他能放下,方源也放不下。岂不闻“斩草不除根,春风吹又生”的道理?

什么,你说滥杀无辜?

呵呵,不管是何世界,只要活在世上,就是因果纠缠,哪有什么无辜?人杀猪,猪是不是无辜?

大鱼吃小鱼小鱼吃虾米,大自然只有食物链,没有无辜这两个字。

这天底下,谁都可以活着,谁都可以死,谁都不是无辜!

------------

\end{this_body}

