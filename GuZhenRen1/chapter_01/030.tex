\newsection{方源!你又来抢?}    %第三十节:方源!你又来抢?

\begin{this_body}

几乎与此同时,在另一边。

“父亲母亲大人,事情基本就是这样子的。”方正站在笔直,口中恭谨地道。

堂中,方源的舅父古月冻土,以及舅母坐在宽背大椅上,均皱着眉头。

舅母咬牙切齿,一边为方正抱打不平,一边又有些幸灾乐祸:“方源这个小兔崽子,他勒索其他人也就罢了,想不到他连亲弟弟都不放过。竟然如此绝情绝义!不过他这次放下如此大错,估计不久后就要被学堂开除了。”

“好了,你少说几句吧。”舅父叹了一口气,又对方正道,“你失了元石,不过也只是一块,不要紧。下去到账房那里再领一枚,这里没有你的事,你下去好好修行。以你的甲等资质,成为第一个中阶蛊师,极有可能。你不要浪费你的天资,我和母亲都期待着你成为第一呢。”

“是,父亲母亲,孩儿告退了。”方正满怀心事地退了下去。

他暗暗思考:“哥哥今天堵住学堂大门,抢了所有学员。造成了如此恶劣的影响,恐怕真的要被开除了。到那个时候,我该不该为他求情呢?”

他脑海中有两个声音在对峙。

一个声音说道:“不用求情,他连你这个亲弟弟的元石都要抢。就算被开除,也是他咎由自取。天作孽尤可活,自作孽不可活啊!”

另一个声音则道:“他可是你的亲哥哥,长着相似的脸,血浓于水啊。好吧,即便是你不认他,也得求情。你若不求情,外人会怎么看你呢,恐怕会觉得你无情无义吧。”

看着方正离开了厅堂,舅母忍耐不住,高兴地道:“老爷,我们断了方源的生活费。这个小兔崽子忍耐不住,这次犯下了大错了!居然敢堵在学堂大门当众斗殴,还勒索,这是挑衅学堂家老啊。我看他被开除,是八九不离十的事情了。”

舅父却摇头:“你把事情想得太简单了。方源不会被开除的,甚至可能任何惩罚都没有。”

“为什么?”舅母大为不解。

舅父冷笑一声:“斗殴打架是受到鼓励的,只要不出严重后果。这次斗殴,有学员死了吗?没有。”

舅母有些不服“老爷怎么知道就没有?打斗这种事情,总是有意外发生的。”

舅父闭上双眼,倚在靠背上:“你这婆娘,就是天真。你真当学堂家老是摆设么?侍卫什么时候出动的?他们在最后出动,这就说明场面一直在控制之下。若是有人重伤,他们早就冲出来了,不会等到最后的。”

“你不是蛊师,不会明白,学堂里并不禁止学员之间的争斗,甚至保持鼓励态度。打斗越多,对战斗就越有帮助。有的学员,甚至还能打出铁交情。长辈们也不会追究这个事情。这已经是惯例了。若谁要护犊子出头,谁就坏了这规矩。”

舅母听得傻眼,不甘心地道:“那方源抢了这么一大笔元石,什么屁事都没有?就这样放过他了?有了这么一大笔元石,对他的修行帮助太大了。”

舅父睁开双眼,满脸的阴霾:“还能怎么办?难道你让我过去亲手把他的元石抢过来吗?不过此事也不是不可以利用。方源连弟弟方正都抢劫勒索,这就是他的败笔。方正是甲等资质,总有一天会比他强大,我们就利用这件事情,分化挑拨方正。让方正彻底远离方源,为我们所用!”

就这样,过去了三天。

方源抢劫勒索的风波,没有扩散,没有闹大,反而有了渐渐平息的趋势。

没有什么长辈破坏规矩,来亲自找方源的麻烦,学堂家老自然也睁一只眼闭一只眼。

虽然这期间,有过两三个少年,不忿元石被强抢的结果,重新挑战了方源。

但在方源轻而易举地将他们打趴下后,所有人都意识,自己若是不勤学苦练拳脚,是赢不了方源的。

在这些少年中,刮起了一阵苦练拳脚基本功的热潮。

拳脚教头乐坏了,他从未见过有这么一届的学员,对基本拳脚有如此的热情和执着。以前他教导的时候,学员们无不是兴趣缺缺,哈欠连天。如今却是炯炯有神,不断求教。

学堂家老特意来询问他这边的情况。

拳脚教头带着兴奋的语气,禀告道:“学员们表现出了出人意料的热情,转变太大了。只是其中有一个叫做方源的,还是和以前一样的懒散。”

学堂家老笑起来,拍拍他的肩膀道:“你所说的这个学员,就是导致其他人转变的源头啊。”

拳脚教头诧异不解。

当然变化远不止这些。

经此一事,方源毫无疑问地成了全体学员的公敌,被所有人敌视和孤立。

再没有人跟他讲一句话,没有人和他打一声招呼。

少年们无不卯足了劲,私底下勤学苦练基础拳脚。在身后长辈们的鼓励和授意下,他们已经决定,务必要亲手把场子找回来。

平静的表面下,暗流在汹涌。

又四天过去。

学堂家老再次分发元石补贴,方源也到了再次动手的时候。

“方源,你抢一次不够,还想抢我们的元石?!”学员们被方源又堵在门口,惊怒交加。

方源站在大门中央,束手在背后,表情冷酷,语气平淡:“每人一块元石,就可免受皮肉之苦。”

“方源,你欺人太甚。我要向你挑战!”古月漠北怒吼一声,率先战了出来。

“哦?”方源眉角微微扬起。

漠北举起双拳冲了过去,几个回合后,他昏倒在地。

“漠北你太没用了,看我的!”古月赤城大吼一声,冲向方源。

攻防转换了一下,他就步入了漠北的后尘。

方源的战斗经验是他们的千万倍还不止,虽然刚刚修行,但每一份力量均是用得恰到好处。

这群少年才刚刚起步,若一拥而上,还可能带给他点小小的麻烦。但是这样一个个上来挑战,比第一次抢劫还要轻松。

一刻钟之后,他带着一个鼓鼓囊囊的钱袋子,悠然而去。留下一地的少年,有的昏迷一动不动,有的抱着肚子或者捂着裤裆,在呻吟嚎叫。

“兄弟们,快来收拾场子了。”侍卫们呼喊着,纷纷涌了上去。

------------

\end{this_body}

