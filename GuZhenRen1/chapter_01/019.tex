\newsection{六转本命春秋蝉!}    %第十九节:六转本命春秋蝉!

\begin{this_body}

%1
炼化过程中,竟遭蛊虫反噬!

%2
酒虫继承着花酒行者的强大意志,此时入侵空窍,悍然反噬方源。

%3
这股强大的意志力量,从上而下,向空窍底部的青铜元海狠狠地冲击下来。

%4
而青铜元海,亦是波涛翻滚,掀起一阵阵的高潮。在方源的意志下,大股的真元逆天而起,汇聚一起,形成一股冲天狂浪,悍然迎上酒虫的意志。

%5
眼看着双方就要在空窍的中央,凶狠地碰撞到一起。但就在这时,两股力量的中央,空窍中的空白地带,浮现出一抹淡淡的蛊虫虚影。

%6
这是一头蝉。

%7
这蝉体型不大,如果说月光蛊是一枚弯弯如月的蓝色水晶,那么这蝉就好似工匠大师用了棕木和树叶制成的精致工艺品。

%8
它的头部、腹部都是棕黄色的,表面上有树木年轮般的纹理,似乎见证了岁月。

%9
它的背部的双翼很宽大,半透明,就好像是两片树叶交叠着。两片翅膀上都有纹路,且十分相似。这纹路就像是树叶的典型网状叶脉,中央是一根粗茎,从粗茎向两边辐射出网状的叶纹。

%10
春秋蝉!

%11
它被惊动了。

%12
就好像是一头巨兽,平时潜藏在山洞中沉睡。但是忽然被吵醒,并且发现自己的地盘被侵犯。

%13
谁敢来我的地盘撒野!

%14
似乎是尊严被冒犯,春秋蝉愤怒了,它散发出一股气息。

%15
这股气息是微弱的,又是如此的强大。仿佛是滔滔天河,滚滚向前,波涛浩荡,要席卷万里青山,要淹没苍天大漠!

%16
和这股气息相比,酒虫的意志就如同蚂蚁见到了大象!

%17
气息向四周席卷扩张,就好像是掀起的无形绝世大海啸。酒虫入侵进来的意志,碰到这海啸,连抵挡的能力都没有,直接被这股气息吞没个干干净净。

%18
方源胸口一闷。

%19
他极力鼓动逆冲而上的青铜真元,碰到这股气息,仿佛是一股海浪撞上了天山。一下子凝聚在一起的真元就分崩离析,溃散成雨,尽数洒下,落回到真元海面当中。

%20
哗啦啦。

%21
真元海面上浪潮此起彼伏,仿佛是下了一场暴雨,更添动荡。

%22
但几个呼吸后,春秋蝉的气息扩散下来,压在元海上。

%23
轰!

%24
方源仿佛听到一声嗡鸣,在瞬间,波涛翻滚的元海一下子静止下来。

%25
春秋蝉的气息死死地压迫住整片元海,像是一座无形的大山镇压着,海面兴不起一丝波涛,平静得如一面镜子。

%26
又好像是原本一张充满褶皱的纸,被这股气息化成的巨人的无边大手,一下子盖住,抹平。

%27
这简直是无以伦比的力量!

%28
方源感到心头像是压上了一座无形的大山,仿佛是孙猴子被压在五行山下,他一丝真元都调动不起来。

%29
不过,他虽然震惊,却并不害怕,反而心中涌现出一股浓烈的喜悦之情。

%30
“没有想到春秋蝉竟然跟随着我一起重生了!原来它并非是一次性的消耗蛊,而是可以重复利用的。”

%31
春秋蝉品级高达六转,是方源前世的第一只六转蛊虫,也是最后一只。为了炼制它,方源动用了全部的手段和资源,费劲了九牛二虎之力,足足三十年的酝酿,才侥幸成功。

%32
但是成功不久之后,春秋蝉还未捂热,正道群雄都感到了方源的威胁,群起而攻,将方源杀死。

%33
重生之后,方源没有发现春秋蝉,就以为它已经消亡。但事实上,它寄托在方源的身体内,陷入了沉睡。

%34
一下子逆转光阴五百年,它元气大损,实在太虚弱了,虚弱到连方源这个主人都感应不到。

%35
现在春秋蝉虽然出现,但情况仍旧不妙。

%36
重生以来,它一直在沉眠中静养。此时出现,只是觉察到空窍危机,算是被酒虫的意志惊醒的。

%37
它虚弱的很,十分虚弱,极其虚弱。

%38
在方源的记忆中,原先的春秋蝉是生机盎然的,它的身体躯干就好像是名贵地板,散发着温润的油光。它的双翅,是嫩绿嫩绿的,好像是两片刚抽芽的鲜嫩树叶。

%39
但是此刻,,一股浓郁的肃杀死寂的气息从蝉身上散发出来。它的躯干没有一点光华,显得粗糙黯淡,如同枯木一样。它的双翅也不是嫩绿的树叶,而是充满了枯黄之色,像是秋天即将凋零的枯萎叶子。而且翅尖都微微卷起,有残缺,就如同落叶的边角。

%40
对此,方源既心疼又庆幸。

%41
心疼的是,春秋蝉遭此重创,离消亡只有一线之隔,如一脚踏在悬崖边上。

%42
庆幸的是,幸亏春秋蝉虚弱到这种程度,否则自己麻烦就大了!

%43
要知道,蛊师和蛊虫之间,必须相辅相成,最好是同等境界。

%44
一转的蛊师,就用一转的蛊虫,这是最合适的。

%45
若是蛊虫品级低于蛊师,蛊师用起来,就相当于壮汉拎着小木棍,很不给力。

%46
若是蛊虫品级高于蛊师,蛊师用着,就如同孩童扛着沉重的大斧头,心有余而力不足。

%47
春秋蝉是六转蛊虫,而方源只是一转初阶的小小蛊师。打个形象点的比喻,春秋蝉就是一座山,方源是一只松鼠。要让松鼠扛着山去打砸敌人,第一时间松鼠就被山压扁了。

%48
若春秋蝉是巅峰状态,方源小小的一转蛊师的空窍,根本就容不下它,直接就会被它的磅礴气息撑破致死。

%49
幸好它虚弱到了极点,才能被方源此时的空窍所容纳。

%50
“我放弃了月光蛊,千方百计找到了酒虫,就是为了炼成本命蛊。但其实,我早就有了本命蛊,春秋蝉就是我的本命蛊!”感受着自己和春秋蝉之间的那股亲切联系,方源心中感慨万千。

%51
本命蛊,是蛊师第一只炼化的蛊虫,十分重要,很大程度上都影响着蛊师的今后发展。

%52
本命蛊挑选的好,蛊师发展得就越顺利。本命蛊品质差,对于蛊师来讲,就会拖累修行,被同龄人赶超。最关键的,还会影响战斗这样的生死大事。

%53
这点方源心知肚明,所以即便是选择了古月山寨中镇寨的月光蛊,他还不满意。千方百计地寻到了酒虫。

%54
在他记忆中,对一转新嫩蛊师来讲,酒虫已经是上等之选。月光蛊只能算是中上等。

%55
但人生之所以迷人,就是因为人们永远不知道,下一刻等待自己的是什么。

%56
方源在前世,炼化了春秋蝉。重生之后,春秋蝉陷入沉眠,但是这种炼化的关系,仍旧存在着。

%57
并且似乎是经过光阴之河的洗练,方源发现,自己和春秋蝉之间的联系,比前世还要更加玄妙,更加紧密。只是因为春秋蝉太虚弱了,才导致方源没有觉察到它。

%58
所以,从真正意义上讲,春秋蝉是他第一只炼化的蛊虫。

%59
只不过,春秋蝉不是方源今生所炼,而是他前世五百年努力的成果。

%60
春秋蝉就是方源的本命蛊。

%61
一个一转蛊师,拥有六转的本命蛊!

%62
此事说出去,估计都不会有人相信!这已经打破世人认知的极限!

%63
但这事,的的确切发生了。

%64
事实是不容置疑的。

%65
“酒虫是本命蛊的上上之选,但是和春秋蝉一比较,它就是地上的泥渣!我今生的本命蛊竟然就是春秋蝉,哈哈哈……”

\end{this_body}

