\newsection{我可是班头啊!}    %第五十四节:我可是班头啊!

\begin{this_body}



%1
一轮红日,缓缓地向西方的大地群山滑落。

%2
它的光,不在刺眼炫目,而是透着一种柔和明亮。

%3
西边的天空,都被它染成一片通红,晚霞连绵成一片,宛若妃子得到了赏赐,欣喜地簇拥着帝王,要一起晚睡。

%4
青茅山的一切,都笼罩在一片模糊的玫瑰色之中。

%5
一座座的高脚吊楼,也披上了一层金纱。

%6
学堂周围栽种的树林,仿佛抹上了一层淡淡的油。

%7
风徐徐地吹着,学员们怀揣着刚刚领到的元石补贴,走出学室,各个神清气爽。

%8
“真不知道方源是怎么想的,居然放弃了班头的位置!”

%9
“呵呵呵,他脑袋坏掉了,估计整天都在想着杀人,我们不要理会这种疯子。”

%10
“说起来,那天他闯进学堂,我真的被吓一跳。太恐怖了,那天回去之后,我就做了一晚上的噩梦。”

%11
学员们三三两两地,结伴而行。

%12
“班头好。”

%13
“嗯。”

%14
“班头好。”

%15
“嗯嗯。”

%16
古月漠北大摇大摆地走着,所到之处,学员们无不向他鞠躬致礼。

%17
他的脸色有着压抑不住的兴奋和陶醉。

%18
这就是权力的魅力。

%19
哪怕是一点点的区别待遇,都能让人更加肯定自身的价值。

%20
此时落日西下,残阳如血,漠北看着,心中欢喜地想到:“怎么以前就没觉察过,这夕阳红润的真是可爱啊……”

%21
“哼,当了一个班头就抖起来了,有什么了不起的。”古月赤城故意走在后面,就是不想向古月漠北行礼问好。

%22
“真不知道方源究竟在想什么,竟然放着好好的班头不做。不过也幸好如此,否则第三的我,怎么能得到副班头的位置?”古月赤城心中有不解,也有庆幸。

%23
“副班头好。”这时一个普通学员走过他的身边,立即向他鞠躬问好。

%24
“嘿嘿,你也好。”古月赤城顿时点点头,脸上笑开了花。

%25
学员走了过去,他就自然而然地想到:“这副班头的滋味不错,想来班头的滋味就更妙了。如果我不是副班头,而是班头,那该多好!”

%26
刚刚还为此庆幸的赤城,此时已经得陇望蜀,对班头的位置产生了期待。

%27
家族的体制下,一层高过一层的权位,就像是一根比一根大的胡萝卜,在前方深深诱惑住他。

%28
“虽然我只有丙等资质,但是我相信,一切都会变得越来越好的。”古月赤城对未来心怀希望。

%29
然而此刻,同为副班头的古月方正,心情却很糟糕,脸色也十分的难看。

%30
“哥哥,你!”他瞠目结舌地看着学堂的大门口,一个孤独的身影就堵在那里。

%31
“老规矩,每人一块元石。”方源抱臂站着,语气平淡。

%32
方正张了张嘴,几番努力后这才说道:“哥哥,我可是副班头了!”

%33
“的确。”方源面无表情地点点头,淡淡地看了方正一眼,“副班头每次补贴多达五块。那你就交三块出来罢。”

%34
方正瞠目结舌,一时间竟说不出话来。

%35
一群少年簇拥着古月漠北走了过来。

%36
看到方源堵在校门口,古月漠北大怒,手指向方源:“方源!你好大的胆子,居然还敢堵我们?!我现在已经是班头,你这个普通学员见了我,首先要鞠躬问好!”

%37
回答他的是方源的拳头。

%38
古月漠北猝不及防,被一拳打中,不禁倒退几大步,一脸的难以置信:“你打我,你居然敢打我?我可是班头啊!”

%39
再次回答他的,仍旧是方源的拳头。

%40
砰砰砰。

%41
几次攻防转换之后,古月漠北被方源击倒在地,昏迷过去。

%42
周围的少年们,统统目瞪口地看着,一时间都不知道该怎么反应才好。

%43
这一切都和他们想象的不一样!

%44
门口的侍卫看着这一切在他们的眼皮子底下发生,不禁窃窃私语起来。

%45
“方源把新任的班头,都给打倒了,我们怎么办?”

%46
“凉拌!”

%47
“什么意思?”

%48
“就是看着呗,然后招呼其他人,收拾场子。”

%49
“可是……”

%50
“哼哼,方源是什么样的人物,你也想去招惹?想想王大、吴二两个人现在是什么的下场吧!”

%51
提问的侍卫顿时一个激灵,再也不说什么了。

%52
大门口的两个侍卫,都把身躯挺得笔直。任由一场劫案在身边发生,仿佛他们都是聋子和瞎子,什么也听不见,什么都看不到。

%53
方源收拾了古月漠北,又收拾了方正和赤城。

%54
其他的少年们这才意识到,原来这一切都没有改变。方源还是那个方源,抢劫还是会如期而至。

%55
“每人一块元石,副班头三块,班头八块。”方源公布了新的规矩。

%56
少年们叹着气,乖乖地掏出元石。

%57
当他们走出学堂大门,忽然有人一拍脑袋,大叫起来:“我想到了,难怪方源不要班头的职位,他是想继续勒索我们呀!”

%58
“不错。他每次勒索我们都有五十九块元石,现在则增长到六十八块。他要是班头,顶多就只有十块而已。”不少人跟着恍然大悟。

%59
“太阴险了,太狡诈了,太狠毒了!”有人拍着大腿,愤恨不平。

%60
“唉,这样一来,班头、副班头也没有什么了不起的。他们照样被抢,剩下的元石仍旧只有两块,和我们完全一样呢。”

%61
不知谁说的这句话,少年们听了,都不由地沉默了下来。

%62
砰!

%63
学堂家老狠狠地一拍桌子,勃然大怒。

%64
“这个方源太过分了,他想干什么?居然还敢抢,抢了班头八块元石,副班头三块元石。这样一来,班头、副班头和其他普通学员有什么区别?!”学堂家老努力压低声音,他的话语中充满了愤怒的情绪。

%65
方源拒绝班头职位,就是拒绝将自己纳入家族的体制。说的严重点,就是对家族的一种背叛。

%66
这就已经让学堂家老十分生气了。

%67
紧接着,方源又抢劫同窗。他的手伸得越来越长了,已经超出了学堂家老的底线。

%68
被这么一勒索,班头、副班头的权势地位就被彻底地削弱下去。

%69
久而久之,普通学员们也会对这两个职位失去敬畏和兴趣。

%70
方源此举,看似微小,意义却重大。

%71
这已经是在以一己之力,挑战家族的体制!

%72
这是学堂家老绝不愿意看到的情景。他培养的是家族的新希望,而不是家族的背叛者。

%73
然而尽管方源此举挑战了他的底线,但是学堂家老却知道,自己并不能出手处理此事。

%74
若他真的这样做了,第一个绕不过他的就是族长。第二个第三个对他有意见的,就是古月赤练和古月漠尘。

%75
族长寄希望于古月方正,方正毕竟是三年来唯一的甲等天才。族长需要一个顽强自立的天才,不需要一个被照顾关怀的娇嫩花朵。

%76
同时对赤练和漠尘来讲,他们也希望各自的孙子,能够在这种挫折中成长。

%77
要是学堂家老出手,替学员们惩处了方源。这话传出去,就是“漠家、赤家的未来接班人,打不过方源,只好让长辈帮忙。”

%78
多难听啊。

%79
这对漠家、赤家的名誉,必将是一次重大的打击。

%80
学堂家老当然不怕一个小小的方源,但他却担心一旦插手此事,将引来族长、漠脉、赤脉的三重压力,这就几乎是整个古月高层了。他一个小小的家老哪里能承受得起?

%81
“这事情的根源,还是在于方源的那个秘密。他究竟是依靠什么,率性晋升到中阶的呢?”学堂家老按捺住心中的火气,又将目光集中在书桌上的三份调查报告上。

%82
第一份报告上,是方源的详细家庭背景。

%83
方源此次根正苗红,身份没有蹊跷,身世也一清二白。双亲亡故,被舅父舅母收留。但是并不和睦,上了学堂之后方源就一直住在学堂宿舍。

%84
第二份报告上,是方源的生平过往的记录。

%85
他年少便有早智,被族人看好,认为可能是甲等资质。但是开窍大典之后,却测出丙等,令族人大为失望。

%86
第三方报告上,是方源近期的踪迹。

%87
他的起居生活非常之简单,几乎是三点一线。白天他都在学堂上课,晚上都在宿舍睡眠。他修行十分刻苦,每天晚上都要进行蛊师修行,温养空窍。有时候会出去一趟,到山寨中唯一的一家客栈改善伙食,买酒喝。

%88
他对酒情有独钟,喜欢喝青竹酒。他的宿舍床下,就摆放了数十坛的青竹酒。

%89
学堂家老又详细看了这三份报告,心中对于方源的印象又深刻生动了一分。

%90
“双亲早死,又和舅父舅母不和……难怪方源这个小子,对家族没有归属感。他被族人亲手冠上天才之名,又亲手摘取下来,从高空摔到地上……他难怪桀骜不驯,又怪癖冷漠。他生活如此简单,修行刻苦,就是憋着一口气,不服输,想向族人证明他能行!所以我打压他的时候,他才如此激烈的反击吧……”

%91
学堂家老思考到这里,不禁轻轻地叹了一口气。

%92
越了解方源,他就越理解方源。

%93
当然,理解并不代表原谅。方源顶撞他,触犯了他的尊威,又拒绝担任班头,还大肆抢劫同窗,这都是他不能容忍的。

%94
抖了抖手中的这些资料,学堂家老又皱起眉头:“这些资料虽然详细,但是却对方源的晋升秘密毫无涉及。这都几天了,这帮人也太不像话了!”

%95
咚咚咚。

%96
就在这时,敲门声响起。

%97
“进来。”学堂家老道。

%98
门开了。

%99
却是族长古月博的亲卫:“族长有命,请家老大人速去家主阁,有要事商议。”

%100
“哦,是什么要事?”学堂家老从座位上站起身来,他从亲卫的语气和神态中感受到了此事的重大。

%101
“是四转蛊师贾富大人又回来了,他的弟弟贾金生行踪不明!”亲卫答道。

%102
“嘶……”学堂家老顿时倒抽一口冷气。

\end{this_body}

