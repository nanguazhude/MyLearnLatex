\newsection{我的解释你只能接受}    %第五十二节:我的解释你只能接受

\begin{this_body}

学堂中,所有人的目光都集中在方源的身上。

这些目光中有震惊,有惶恐,有嘲讽,有冷酷。

方源对这些目光视若无睹,他看着学堂家老,一手指着地上昏死过去的侍卫,神情严肃。

“禀告家老,这两个侍卫包藏祸心,其心可诛!他们在我冲击中阶最紧要的关头,强行破门而入。众所之知,蛊师修行不能受到干扰。尤其是冲击更高境界,更是如此。稍有不慎,不仅冲击失败,空窍更会受到损伤。不过幸好学生我运气不错,在他们闯入的一刹那,侥幸踏上了中阶。”

“但是!”还没等其他人反应过来,方源又紧接着道,“这两个人还不承认他们刚刚犯下的错误,居然大言不惭地想要对我动手,甚至辱骂我族先祖,谎称此次干扰晚辈修行的举动,是家老大人你的意思。学生不信,激烈反抗。这两人武艺高强,学生浴血奋战,这才将这两人击败。”

“不过看在这两人是学堂侍卫的身份,学生并没有痛下杀手。只是一个被削断了胳膊,一个被斩切了大腿而已。虽然失血有点多,但都还没有死呢。这件事就是这样,还请家老为学生主持公道啊!”说完,方源对着学堂家老一抱拳。

他语气急促,说了一大堆,其他人都没有插嘴的份儿。

说完之后,周围的人这才慢慢地反应过来。

“方源刚刚说什么,我好像没听清楚。”

“他好像说他晋升中阶了!”

“怎么可能,他一个丙等资质的废材,居然第一个晋升中阶。”

“一定是骗人的,他是害怕受到学堂的惩罚,所以撒了谎!”

学员们都大声地议论起来。

相对于方源晋升中阶这个事情,两个侍卫的生死已经不重要了。

他们又不姓古月,谁管这两人死活?

“你说你已经晋升到了一转中阶?”学堂家老声音冰冷,目光中寒芒闪烁,“方源,这话可不能乱说。你要是现在承认错误,我还能念在你初犯,宽大处理你。但是你要再错下去,企图用谎言遮盖掩饰,那么老夫现在就告诉你,谎言是最容易被戳穿的。”

方源没有任何的分辨,只是轻笑一声,对学堂家老道:“请家老检查。”

不用他说,学堂家老已经走了过来。

他把手贴到方源的小腹位置,分出一缕心思投入进去。顿时就看到方源的空窍景象。

空窍里,空无一蛊。

春秋蝉已经隐藏起来,六转蛊虫的层次远远高于三转的学堂家老,有心隐藏,就不会被轻易发现。

至于酒虫,方源则临时放进了宿舍的酒坛当中,并没有随身携带。

学堂家老闭着双目,就见一片青铜元海,波澜不惊。

点滴元水,都是中阶真元才有的苍青色。

再看周围窍壁,白色的窍壁波光泛滥,似乎全部是由水流组成。一股股的水流,都在急速的流动着。

水膜!

“真的晋升了中阶,这怎么可能!?”学堂家老心中叫了一声,眼皮子底下闪过一丝震惊的光。但他极力地掩饰住,面色沉凝如水。

片刻后,他消化了这个事实,缓缓地抽回手,以低沉的声音道:“的确是中阶。”

学员们早就屏息,在静待结果。

学堂家老此言一出,整个学堂像是炸开了锅。

学员们惊疑不定,各个脸上都流露出难以置信的神色。

方源只是丙等,却第一个冲破中阶,这是打破常识的事情!

蛊师修行,冲击境界,首要应该看资质才对。有没有搞错,丙等都能率先晋升?这让那些甲等、乙等资质的人情何以堪!

“这!”古月方正面色陡白,他昨晚还信心十足,但是现在的事实摆在他的眼前,他不堪这样的冲击,一屁股坐了下去。

古月漠北握紧双拳,古月赤城狠狠地咬着牙关。

学堂家老是不可能被蒙蔽的,他方源究竟是怎么做到的?

一时间,所有少年都死死地盯着方源,他们的心中都泛起同一个疑问――凭他方源丙等资质,他晋升的怎么这么快?

学堂家老的心中,同样也充满了疑惑。

在这样巨大的疑惑下,他将先前打压方源的想法抛之脑后,他直接发问:“方源,我希望你能解释一下,你究竟是如何晋升道中阶的。”

方源无声微笑道:“天道酬勤,学生勤学苦练,日积月累,所以水到渠成。”

“骗子!”

“切,要是天道酬勤,我早就第一了!”

“还勤学苦练?我前些时候,还看到他在商铺里优哉游哉地闲逛呢。”

学员们显然不满意这样的答案。

“是吗?”学堂家老不置可否,目光冷冽,逼向方源。

方源面色坦然,毫不畏惧地和家老对视。

他浑身浴血,麻布衣衫凌乱,似乎是经历过了一场激烈的搏斗。

一双眸子深幽如潭,透露出一种平静,一种淡然,甚至似乎还藏着一丝戏谑。

看到这样的双眸,学堂家老的内心,不禁动摇了。

“这个方源不惧、不畏、不恐、不惊,怎么可能会被我当场逼问出来?以他丙等资质,率先晋升到中阶,一定有着秘密。不过这个秘密他既然不想说,我身为学堂家老,也不能强行逼问,看来只能秘密调查了。”

想到这里,学堂家老只好收回目光,冷峻严肃的脸色也缓和下来。

方源却没有罢休:“学生惶恐,家老大人,不知道你怎么处理这两个侍卫。这两人失血过多,再不救治,可就要死了。”

“就你还惶恐?”学堂家老心中不由地一声冷哼。他的眉头深深地皱起来。

事情发展到这一步,他身为学堂的负责人,必须要站出来处理。

“但是该如何处理呢?”学堂家老不由地感到有些棘手。

他不禁沉吟起来。

方源将学堂家老的神情变化,都尽收眼底。他心中暗笑,学堂家老此时一定很为难罢。

这两个侍卫,只是外姓奴仆,命贱如草。若在平常时期,死了也就算了,没有谁在乎。

但是现在这情形不同,这两个人是学堂家老亲自派遣出去的。要真死了,就是丢他家老的脸面!

所以这两个侍卫死不了,学堂家老会实施救治。

真正让学堂家老感到为难的,是对方源的处置。

本来在他的计划中,方源旷课在先,又打杀侍卫在后,可谓违逆师长,狂妄自大。按照族规,就要被关进家族的监牢,在牢房内反省认错。

但是这些过错,再按上了方源晋升中阶的这个大前提后,就陡然变得不一样了。

方源他是因为修行,才不得不旷课,才打杀侍卫。这就是情有可原。

更关键的是,他成功地晋升中阶,成为此届第一人。这就占住了大义。

究竟方源靠着什么,这么快晋升中阶的,这个秘密且放在一边。

成王败寇,世人都注重结果。没有人会指责一个如此优秀的后生晚辈。

学堂家老更不能对他有任何的惩处。

你学堂是干什么的?就是为了培养优秀的蛊师,为家族注入新血。

出现了如此优秀的少年,你学堂家老还要打压,那就是你的失职!

就好像是一个学生考出了好成绩,作为师长就得去表扬鼓励,而非惩罚批评。一个因为学生成绩优异而去批评惩罚的老师,从来都不会被认可的。

也许其他家老,或因为忌惮方源的发展前景,或因为恩怨情仇,对方源进行暗中的压制。但惟独你学堂家老不能!

因为你负责学堂,你就得做到公平公正,至少是表面上的公平公正。

这就是规矩!

“难道就这样放了他?好不容易才抓到他的一个把柄。”学堂家老感到很不甘心。他心知肚明,整个学堂中这些少年,都是局外人。

局外人,只能看热闹,看不出其中的门道,体会不到这暗中的精彩!

事实上,这是一次他学堂家老,和学子方源的一次交手!

他首先抓住规矩不放,要整治方源,打掉他在其他少年心中的强势形象。

然后方源悍然反击!他此举看似莽撞,却一针见血。借着晋升中阶的名义,立即就找回了场子。

至于那两个倒霉的侍卫,不过是两人斗争中遭到无辜牵连,而牺牲的棋子。

“这个方源,心机太深了!若是他真的杀了这两个侍卫,我还能凭此理由来进行反击。他资质虽然不怎么样,但是这手段如此老辣和周到,真难以想象这是一个十五岁的少年做出的事情。偏偏我还反击不得。难怪早些年,族中就盛传他的早智和诗才!”学堂家老忽然意识到,他已经败了。

他的失败,就在于他的身份,他是负责学堂的家老。

这是他的强势之处,又是他的弱势的地方。

最强的一点,就是最弱的一点。

方源早就窥破了这个道理!

学堂家老的心中既无奈又恼火。

他曾要方源解释,方源解释的话,其实漏洞百出,根本经不起推敲。

这些侍卫都是学堂家老选拔的,心性不可能这么鲁莽,更不会脑残到辱骂古月先祖。

方源的话,是故意这么说的,是赤裸裸的诬蔑,是当着知情人的面在栽赃和陷害!

学堂家老很清楚这点,但是他更清楚,自己不能追究这件事情。

这是个陷阱。

一旦追究,真相大白,他该怎么处理这件事情呢?

如果他不惩罚方源,两个侍卫如此蒙冤,他这个学堂家老处事不公,今后如何能服众?

如果他惩罚了方源,他就是打压良才,嫉贤妒能!为了两个外姓的奴仆,就打压族中后辈,这事情传出去将会引发族人的不满。

所以最好的处理方法就是,睁一只眼闭一只眼,把这两个侍卫当做弃子。就认定这两人犯下了大错。同时对方源进行表彰。

这样一来,族人们会满意,侍卫们在被蒙蔽的情况下,也会认为此举公正。

如此处置,无疑就符合学堂家老的最大利益。

理智告诉学堂家老,就应该这么处理。但是感情上,学堂家老却有些跨不过这个坎。

这个方源太可恶了!

学堂家老这次不仅没有压住方源的风头,反而自己成了踏脚石,被方源踩了一脚,当众丢了一次脸!

方源丝毫不敬畏他,竟然敢当面,如此的针对他。这让他堂堂的学堂家老,感到一种耻辱和羞恼。

更关键的是,今后若这两个侍卫觉得冤屈,想要揭露真相,他学堂家老为了维护形象和身份,就得第一个跳出来,把这两个侍卫镇压下去!

可这明明都是方源一手造成的!

这是一种什么感觉呢?

打个形象的比喻,就好像是方源在学堂家老的脸上拉了一泡屎,学堂家老还要表扬方源,同时反过来为他擦屁股。若是有外人想要指出来,说他学堂家老脸上有一泡屎,他就必须第一时间把这外人的嘴给堵住。

这种憋屈和腻味,让学堂家老几乎不能忍受。

心中有一股越来越强烈的冲动,想要直接动手去抽方源几个大巴掌!

但是最终,学堂家老只是将手伸出来,轻轻地拍拍方源的肩膀。

“好小子。”学堂家老的脸色阴沉如水,从牙缝中挤出这句话。

“都是有赖学堂的栽培。”方源的语气淡淡。

学堂家老的眼角顿时一抽搐。

(ps:哎呀呀,三江被赶超的说……求票,急求三江票票。抹泪呀,昨天努力码字一直到深夜的,但是这本书比较注重质量,因此速度仍旧有限。其实这章都有近四千字了。揉脸呀,决定了!若事不济,周日仍旧会加一更的,也就是三更,也算是回报大家的三江票的支持了。)

\end{this_body}

