\newsection{古月青书VS白凝冰(下)}    %第一百四十二节:古月青书VS白凝冰(下)

\begin{this_body}

方正站在远处,看得大气都不敢喘一声。

此战已经到了最关键的时刻,谁胜谁负就看这一次对拼的结果了。

“青书大人,加油啊!”方正紧张得浑身颤抖,他知道自己过去只会添乱,只能大声地为古月青书鼓气。

仿佛是听到了方正的声音,冰刃风暴越来越小,硬生生地被青书压制下来。

“该死,我竟也能碰到真元不足的情况……”白凝冰咬着牙,越转越慢,但他真元的恢复已经跟不上消耗,渐渐无计可施。

北冥冰魄体的真元恢复速度很快,但是三转修为时的回复速度,比木魅直接提用天然元气的效果,还是稍微差了一筹。但如果白凝冰修行到四转,那么回复速度就要远远凌驾于木魅蛊之上了。

只是,生死激斗,从未有如果可言。

不论胜败,结果必须接受。

最终冰刃风暴停顿下来,但古月青书也付出了惨重的代价。

他的双手如马车般大小,此刻左手上仅剩两根手指,右手余下三根,两个掌心都被冰刃削掉了大半。

但当他双手渐渐向中间合拢的时候,新的树干从手掌中迅速地生长出来,并相互交织在一起。

两只手掌形成一个木制的牢笼,将白凝冰困在当中。

“可恶!”白凝冰咬牙,他体内真元已经干涸,只能任由青书施为。

“赢了!!”远远地看到这一幕,方正雀跃大叫。

“我要死了吗……”白凝冰在心中大喊。他眼睁睁地看着两只手掌越来越近。只要当两只手掌彻底合实,他就将被巨力挤压成一块血肉模糊的肉饼。

但是手掌的速度越来越慢,在半途中缓缓地停住。

白凝冰愣了愣,旋即意识到是古月青书身上出现了问题,顿时大喜。

“可恶,就差最后一下……”这一刻,古月青书心中充满了无奈。他的双手已经失去了知觉。完全变成了树木。

同时,他也渐渐地感觉不到自己的内脏和肺腑。木魅蛊的力量侵蚀了他的全身,他即将迈入死亡。

“不。不能就这样结束了!我还可以动用青藤蛊!”古月青书强撑着精神,调动青藤蛊。

一根根粗大的青藤,从牢笼的缝隙中向白凝冰袭杀过去。

白凝冰连连躲闪。但他历经几场战斗,体力消耗实在太大。加上牢笼中空间有限,不好腾挪,终究被一根藤蔓缠住右脚,绊倒在地上。

“结束了。”青书心中欣慰一叹,连忙催动十数根青藤紧跟而上。

生死存亡之际,白凝冰的空窍中,真元终于恢复到可利用的最低程度。

他毫不犹豫地将这些真元消耗一空,灌注到冰刃蛊中,形成一柄新的冰刃。

冰刃锋锐。将缠着右脚的青藤斩断。白凝冰狼狈一滚,堪堪避过射来的十多根青藤。

青藤射在地上,瞬间就扎破厚实的地面。泥土飞溅中,青藤再度袭来。

白凝冰大喘着粗气,用冰刃艰难抵挡。

青藤从四面八方袭来。生死就在间隙之中。白凝冰一旦出现一点失误,就将被青藤擒杀。

但他到底是天才人物,在死亡的刺激之下,他压榨出全部的潜力,抵挡躲闪的动作变得简洁干脆。

虽然时不时地被青藤绊倒,险象环生。但他终究保住了性命。

一根根的青藤被冰刃斩断,青藤的数量也越来越少。

不是古月青书不想催动青藤蛊,生长出更多的青藤。而是周围空气中的天然元气,几乎已经被他吸收殆尽了。

虽然在更外围,空气中的元气正在向此处弥漫,但微薄的元气含量终究满足不了青藤蛊的需求。

还有一个更坏的消息,木魅蛊的力量已经彻底侵蚀了古月青书的身体,并且开始侵蚀他的意识。

古月青书的意识开始出现了恍惚和不时的间断。

死亡的气息已经喷吐在他的脸上。

“要结束了吗?不……”他不甘心,强振精神,为擒杀白凝冰做出最后的努力。

他已经看不到了,木魅蛊的力量早就侵蚀了他的双眼,也听不到了,他的耳朵也形同虚设。

他只剩下一丁点的触觉。

依靠着白凝冰的反击,他判断后者的方位,然后施展进攻。

他的努力得到了回报,白凝冰终于力竭被擒。一根藤蔓绕过他的脖颈,将他整个人提起来,并且渐渐用力收缩。

白凝冰顿时感到呼吸困难,哪怕张着大口也无济于事。和青书同时,他也正在步入死亡。

……

方源喘着粗气,一场激战已经结束。

地上躺着五具死不瞑目的尸体,都是白家的蛊师。

依靠着隐鳞蛊进行偷袭,月芒蛊和双猪巨力的优势,在五百年的战斗经验下,发挥出了惊人的效果。

虽然他对青书说要回归山寨,但这不过是一个托辞罢了。

离开一段距离后,他就用隐鳞蛊隐去身形,绕过山道战场,先后去了蛮石和熊力等人死的地方。

他将蛮石等五人尸体上的蛊虫回收,待到了熊力失去的地方时,已经发现尸体不见了。更谈不上他们的蛊虫。

“看来是熊林收了他们的尸体。可惜了,我本来还希望得到熊力的棕熊本力蛊呢。”方源心中叹息。

留下熊林一条性命,并非他所愿。

但当时,他杀了熊姜,熊林已经警觉无比,要杀他必须得费一番手脚。

那时,白凝冰就在身侧。方源若和熊林内讧,白凝冰就要得意了。

“不过棕熊本力蛊,也未必在熊力的身上。他已经养成了一熊之力,恐怕早就上缴给家族也说不定。”

方源凝目,远眺山道战场。

青书等人和白凝冰的激战,自然动静巨大,瞒不住周围过往的狼群以及蛊师。

方源虽然不<B>①3\&\#56;看\&\#26360;网</B>一方,但木魅蛊的威力他在前世,也是亲眼所见。和白凝冰的一场龙争虎斗,必定是少不了的。

他当然不愿意放弃可能出现的机会,因此选择在周围等待。

不时有蛊师被激战的声音吸引过来,方源就引来狼群,将他们牵制住。

实在来不及牵制的,方源就自己动手。

“山道战场的战斗声已经几乎停止了,看来已经快要分出胜负。”他的右耳参须探出,扎根在旁边的山壁上,使得方源能探听到山道战场中的一些底细。

老实讲,古月青书的表现,算得上超水平发挥。而白凝冰断去右臂后,对他的战力影响比方源想象中还要更大一些。

但旋即,方源面色一变。

他听到有大量的脚步声,正从两个方向上向山道战场赶去。

其中一个方向,来源于古月山寨。另一个方向上的蛊师,则显然来自于白家寨。

每一个方向上的蛊师,都至少有二十多人。根本不是豪电狼群能够阻挡的,方源也不能同时牵引两只狼群去牵制住这群人。

“看来此战的消息,不知道被谁探知到了。这些就是两家特意派遣的援兵,我得赶紧进入山道。”

方源的距离最近,第一个进入山道。

战场中的景象,并没有出乎他的预料。

从牢笼缝隙中,他看到白凝冰被藤蔓吊着,奄奄一息,仍旧有一口气在。

“冰肌的防御力量,支撑着白凝冰的性命。哼,可惜,你遇上了我。”方源心中杀机涌现,脚下连踏,就向白凝冰奔去。

刷刷刷。

忽然一阵松针,如一蓬骤雨,向方源笼罩过来。

竟然是古月青书的松针蛊,在对方源发动攻击。

“怎么回事?”方源连连后退,避过这场松针暴雨。他凝目看向白凝冰所化的巨大树精,心中了然了,“看来古月青书的意识已经模糊到敌友不分的地步。仅靠着心中的执念,企图擒杀白凝冰。任何闯入这片战场的人,都将被他视作了干扰者。”

就在这时,山道那端出现了白家蛊师的身影。

看到这样的惨烈战场,他们的脸上都露出惊骇之情。

“那边的小子,你最好不要妄动!”一位三转蛊师对方源喊道,语气充满了警告威胁的意味。

“这个白凝冰还真是命大啊。”方源看到这一幕,心中不由地冷笑一声,顿时知道自己已经补不上这最后一刀。

首先青书的意识已经模糊,一根青藤根本不足以杀掉拥有冰肌防御的白凝冰。

方源接近白凝冰,必要突破青书的封锁。这样一来,无形中就牵引了他的大部分注意力,说不定反会帮助白凝冰脱困。

就算是强行突破进去,那么白家的这些三转、二转的精英蛊师,绝不是吃干饭的,必会阻止。

况且对付白凝冰,也是极其危险。

白凝冰空窍中真元已经恢复,至少蓝鸟冰棺蛊是可以使用的。

哪怕是方源最后杀了白凝冰,那么这些白家的蛊师精锐,绝不会放过方源。必定要杀之而后快。

方源心中不禁暗暗叹息一声:“距离太长了,和白凝冰的距离至少有二十多步。而我的月芒蛊的攻击范围,只有十步。而且……为了区区一个将死的白凝冰,而搭上自己的性命,严重干扰重生大计,这并不合算。”

想到这里,方源不禁退后几步。

这个显得懦弱的动作,让赶来的白家蛊师都松了一口气。

(ps:月初,求个保底裤的月票……)(未完待续。如果您喜欢这部作品,欢迎您来投推荐票、月票,您的支持,就是我最大的动力。)

------------

\end{this_body}

