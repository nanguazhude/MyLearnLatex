\newsection{既是妥协又是威胁}    %第三十七节:既是妥协又是威胁

\begin{this_body}

%1
而此时,在漠家。

%2
“我叮嘱过你什么?看看你做的好事!”书房中,古月漠尘拍着桌面,在大发雷霆。

%3
漠颜就站在这个老人的对面,低着脑袋,眼眸中流露出既吃惊又愤怒的感情。她也是刚刚知道这个消息,高碗居然被方源杀了!

%4
那个十五岁的少年,居然有这样的手段和心志。高碗可是我堂堂漠家的家奴,方源杀了他,简直是不把漠家看在眼中!

%5
“爷爷,您不用发火。这高碗不过是个家奴,死了也就算了,反正他又不姓古月。倒是那个方源,胆子太大了,打狗也得看主人。他不仅打了我们漠家的狗,还一下子把狗打死了!”漠颜愤愤不平地道。

%6
古月漠尘怒喝起来:“你还好意思说!你是翅膀硬了,不把我说的话放在心上了,是不是?我告诫过你什么,你都忘了一干二净了!”

%7
“孙女不敢。”漠颜悚然一惊,这才知道爷爷是真的发怒,连忙跪了下来。

%8
古月漠尘手指着窗外,训斥道:“哼,那个什么家奴死了也就算了,但你现在还盯着方源不放,真是鼠目寸光,不知轻重!你知道你此举的意义吗?小辈争斗,是他们的事情。我们做长辈的,不要去插手。这是规矩!现在你找方源的麻烦,就是坏了规矩。如今不知道有多少人在外面,冷眼看我们漠家的笑话呢!”

%9
“爷爷请息怒,怒气伤身。是漠颜不好,拖累了漠家。爷爷让漠颜怎么做,漠颜就怎么做!只是孙女实在咽不下这口气,那个方源实在太可恶了,太无耻了。他先是诓骗我,进入学堂。后来又躲在宿舍里,任凭我万般叫骂,都不出来。我一走,他就杀了高碗。实在是阴险无耻至极!”漠颜禀告道。

%10
“哦,是这样?”古月漠尘眉头皱了一下,他还是首次听到这个信息,眼中不禁闪过一抹精光。

%11
他深呼吸一口气,压下心中的怒火,抚须沉吟起来:“这个方源我也听说过一些,早些年做过诗歌,有早智。不想资质却是丙等,难堪大用,因此放弃了对他的招揽。现在看来,倒是有些意思。”

%12
顿了一顿,古月漠尘用手指敲敲桌面:“来人,把那个盒子拿过来。”

%13
门外自有人伺候着。很快就捧进来一个箱子。盒子不大不小,但有些沉,下人用两只手捧着,站在了书桌旁。

%14
“爷爷,这是什么?”漠颜看到这木盒,疑惑地问道。

%15
“你何不打开看看?”古月漠尘眯着双眼,语气有些复杂。

%16
漠颜站起来,掀开木盖,往里面一看。

%17
顿时,她面色骤变,瞳孔猛地缩成针尖大小,忍不住倒退一大步,口中发出一声难以抑制的惊呼。手中的木盖也失手掉在了地上。

%18
没有了木盖,木盒子里装的东西便呈现在众人面前。

%19
竟是一堆血肉!

%20
这些血肉,显然是被人削成一片片,一块块,装在了盒子里。猩红的血液积蓄在里面,有的是惨白的皮肉,有的是长条的肚肠,其中还夹杂着一两块骨头,不是腿骨或者就是肋骨。周边角落的血泊中,还浮着两根手指头,半根脚趾。

%21
呕……

%22
漠颜花容失色,再倒退一大步,肚腹一阵沸腾,差点当场就吐出来。

%23
她虽然是二转蛊师,历练过一番,也杀过人,但还是首次看到如此恶心变态的一幕。

%24
这盒子里的血肉,显然是人的尸体被切碎,然后塞进去的。

%25
一股冲天的血腥气息,顿时弥漫开来,充斥整个书房。

%26
端着盒子的家奴,双手都在抖,脸色一片惨白。虽然先前已经看过这个盒子,也吐过了,但是现在端着它,仍旧感到一阵阵的惊悸和恶心。

%27
书房中三人,唯有家老古月漠尘面色不变,他淡淡地扫了一眼这盒子的血肉,对漠颜缓缓地道:“这个盒子,就是方源今早摆放在我家的后门处。”

%28
“什么,真的是他?!”漠颜大为震惊,脑海中忍不住浮现出方源的形象。

%29
她第一次看到方源,是在客栈。

%30
那时候,方源坐在窗边,静静地吃着饭菜。他面容普通,双目黑沉,身型消瘦,肤色带着一种少年特有的苍白。

%31
明明是一个如此普通安静的少年,竟然做出如此变态疯狂的举动!

%32
惊恐之后就是狂怒,漠颜大叫道:“这个方源太猖狂了,吃了雄心豹子胆了!居然敢如此做,这是对我们漠家的挑衅啊!我这就过去,把他押过来问罪!!”说着就要往外走。

%33
“混账东西,你给我站住!”古月漠尘比她更怒,随手抓住书桌上的一块砚台,就甩手扔了出去。

%34
坚硬沉重的砚台打在漠颜的肩膀上,又砰的一声,掉在地上。

%35
“爷爷!”漠颜捂住肩膀,惊呼一声。

%36
古月漠尘站起来,手指着自家孙女,语气很激动:“看来这些年你是白白历练了,你真是令我失望!对付一个小小的一转初阶的蛊师,你劳师动众不说,还一直被对方牵着鼻子走。现在又被愤怒冲昏了头脑,到现在你还明白方源此举的含义吗?”

%37
“什么含义?”漠颜大惑不解。

%38
古月漠尘哼了一声:“方源若是一心想要挑衅,把此事闹大,何不将这箱子放在人来人往的正门,反而放在了人迹罕至的后门?”

%39
“难道他是想要和解?不对,既然和解,当面赔罪不是更好,为什么要送这个箱子的碎尸。这根本就是挑衅!”漠颜道。

%40
古月漠尘摇摇头,又点点头:“他是想和解,又的确在挑衅。他将木盒放在后门,是想和解。在木盒中装了碎尸,是在挑衅。”

%41
“你看。”老人指着盒子,“这个木盒并不大,装不了一具完整的尸体。所以里面只是一部分碎尸。他是想告诉我们,他不愿意闹大此事,想要息事宁人。但是若我们漠家还要抓住这事不放,他就会将剩下的碎尸抛洒在正门,彻底闹大此事。到那时,就是两败俱伤。全族都会知道,我们漠家先破坏了规矩,我们漠家未来的掌权人,居然孱弱到需要长辈如此的溺爱和维护。”

%42
漠颜听了这番话,一时间都有些目瞪口呆。她从未料到,方源此举竟然有如此深意。

%43
“这手段真是高明啊。”古月漠尘感慨道,“只是一个举动,就刚柔并济,进退有据。这个简简单单的木盒子,既表示了方源的妥协之意,又是他针对我们漠家的威胁。偏偏我们漠家,还真的被他捏住了软肋。漠家的名誉若是因此受损,紧接而来的,就是赤家的发难,族长一脉的打击。”

%44
漠颜不信邪地道:“爷爷,你是否太高看他了?就凭他,他不过才十五岁而已。”

%45
“高看?”漠尘不悦地看了孙女一眼,“看来你这些年顺风顺水惯了,养成了自大的毛病,有些看不清现实。这方源先是临危不乱,诓骗你进入学堂。而后急中生智,在宿舍避祸。接着任你辱骂却不逞强,这是隐忍冷静。你走后他立刻杀了高碗,是坚毅勇敢。现在又送来这箱子,分明是智计谋算。你说我是不是高看他?”

%46
漠颜听得目瞪口呆,她实在没有料到爷爷居然如此欣赏方源,当即不服气地道:“爷爷,他不过只是个丙等罢了。”

%47
古月漠尘抚须长叹:“是啊,他只是个丙等。拥有如此心智,却只是丙等资质,实在是可惜。只要资质再高一层,是个乙等,他必将是我古月一族未来的风云弄潮儿。可惜是丙等啊。”

%48
老人的叹息中,充满了感慨。似在遗憾,又似在庆幸。

%49
漠颜沉默不语,她的脑海中不禁再次浮现出方源的形象。在她的心理作用下,方源那原先文弱的面孔,此时却笼罩了一层诡秘凶险的阴影。

%50
“这件事情,是你一手造成的。你觉得怎么处理?”古月漠尘忽然打破沉默,开始考较漠颜。

%51
漠颜沉思了一会儿,方才带着冷漠的语气道:“高碗一个奴才,死了就算了。方源不过是个丙等,也只是小事。关键是要维护我漠家的名誉。为了平息此事,不妨将高碗全家老小都杀了,向全族表明我们维护规矩的态度。”

%52
“嗯,你能以大局出发,暂时抛开个人感情,维护家族的利益,这点很好。不过这个处理手段还是欠妥。”古月漠尘抚须点评道。

%53
“还请爷爷训下。”漠颜行礼。

%54
古月漠尘沉吟道:“此事由你而起,爷爷就罚你禁闭七天,从此以后不要再找方源的麻烦。高碗以下犯上,一介奴仆冒犯主子,该死,其罪当诛!因为他是漠家的家奴,漠家也有管教不严的责任,就赔偿那方源三十块元石吧。至于高碗的家人,给予他们五十块元石的补贴,再把他们都逐出府去。”

%55
顿了一顿,他又道:“七天之内,你好好在家休息,不要出去了。同时也好好想想,爷爷如此处置的深意。”

%56
“是,爷爷。”

\end{this_body}

