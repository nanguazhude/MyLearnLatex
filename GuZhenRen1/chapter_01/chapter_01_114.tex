\newsection{洞中有猴王}    %第一百一十四节:洞中有猴王

\begin{this_body}

%1
商队离开山寨的三天后。

%2
山体石林,晦暗的红光充斥其中。

%3
一根根的巨大石柱,从洞顶延伸下来,宛若倒长的巨木,组成一片巍然的灰色石林。

%4
方源在石林中且战且退。

%5
吱吱吱……

%6
一群石猴瞪着绿色的圆瞳猴眼,对方源紧追不舍。

%7
“月芒蛊!”方源念头一动,右手朝着猴群的方向,轻轻一劈。

%8
呼。

%9
一片幽蓝色的月刃,足有脸盆大小,骤然形成,穿透空气,直接切入猴群当中。

%10
一只玉眼石猴飞跃在空中,无法借力,被这片月刃击中。

%11
它还没有来得及发出惨叫,下一刻,它的整个身躯,从从头到脚,被月刃劈成了两半。

%12
生命的气息已经离它远去,浓重的死亡笼罩下来。

%13
在转瞬之间,它灵动的双眸,变化成一对玉珠。身躯在落下的过程中,化为石雕。

%14
砰。

%15
一声脆响之后,石雕摔在地上,碎成一块又一块。

%16
而月刃只是光辉黯淡了一些,斩了这只石猴,余势不减,又劈中后面的时候。

%17
嚓嚓嚓……

%18
几声脆响之后,又有五六只的石猴被当场斩杀。

%19
吱吱吱!

%20
同伴的惨死,让猴群更加愤怒,它们厉声嘶叫,声势暴涨一倍,气势汹汹地向方源扑来。

%21
方源临危不乱,心中冰雪一般冷静。且战且退。石猴渐渐接近,他就用月刃反击。

%22
以前的月光蛊,哪怕是动用小光蛊增加威力,一记月刃也只能斩杀一两只石猴。但现在的月芒蛊,只要催发出一记,往往就能收割掉五六只石猴的性命。

%23
不过有利有弊,月芒蛊对于现在的方源来讲。消耗并不低。

%24
每一记月刃,就要消耗掉他一成的赤铁真元。方源的空窍中,元海最多只有四成四。

%25
这也就意味着。他一口气最多只能爆发四记月刃。

%26
“如果合炼了四味酒虫,精炼了真元,我就能连续催发八片月刃。可惜。第二只酒虫虽然到手,酸甜苦辣四种美酒,也收集了三种,就差一种,卡在最后这一步。”方源暗暗叹息。

%27
三记月刃之后,他的空窍当中只剩下一成四分的淡红真元。

%28
保险起见,他不再催发月刃,而是催动了白玉蛊。

%29
石猴群包围了上来,冲在最前面的一只猴子跳到方源的脚边,然后猛地暴起。由下而上,用坚硬的猴头向方源下巴撞去。

%30
方源冷哼一声,正想用拳头击爆这只自不量力的小猴子。

%31
但是心念一动,却住了手,用下巴硬生生地承受住这次攻击。

%32
在碰撞的前一刻。他的下巴处泛起一抹白玉的冷光。

%33
砰的一声闷响。

%34
撞击的力量迸发出来,方源不禁向后一仰头。而那只石猴则倒在地上,抱住脑袋,惨叫着,满地打滚。

%35
若是没有白玉蛊,下巴绝对就要被撞碎。不过现在。方源除了有些微微的眩晕之感外,毫发无伤。

%36
不过,虽然有白玉蛊的防御,但是撞击的力道还是要承受的。

%37
方源连退几步,这才缓过劲来,双眼一片清明。

%38
他刚刚有意地承受了石猴的这记头槌,目的就是要让这个身躯习惯这样的攻击,适应这样的眩晕感觉。

%39
以后若是在生死关头,遭到这样的攻击,他就能更快地回过神,挣扎出一线生机。

%40
方源一向手段狠辣,这份狠意,不仅针对敌人,更针对他自己!

%41
他几乎每隔三天,就来这里,斩杀石猴。

%42
目的不仅仅是为了花酒行者的力量传承,还有一个重要的目的,就是借助石猴群来锤炼自己的战斗能力。

%43
蛊师的身体素质,拳脚功夫,空窍真元,战斗经验,每一只蛊虫,都是影响整体战斗力的因素。

%44
只有将这些因素紧密地统合在一起,才能发挥出最强大的战斗力。

%45
石猴群就像是一个铁锤,而方源就像是刚出炉的铁锭。铁锤的每一次敲打,都能让铁锭更加坚硬、纯粹、凝练。

%46
一刻钟之后,这场战斗结束了。

%47
地面上,是随处可见的碎石块,其中掩藏着颗颗玉珠。

%48
“这次杀了四十一头玉眼石猴。”方源心中有数,每一次都会统计战果,从每一次战果中检讨自己,挖掘不足加以改进。同时也能感受到自己的进步程度。

%49
“在刚刚的战斗中,月芒蛊居功至伟,三记月刃,至少杀了十七八头的石猴,已经几乎占了一半。剩余的石猴,都是我动用拳脚击杀的。”

%50
月光蛊对于石猴的攻击效果,并不是很明显。但晋升为月芒蛊后,它就一跃成为方源目前最犀利的攻击手段。

%51
不仅攻击力强大,更关键的是,效率很高。

%52
方源催发了三记月刃,前后不过短短几个呼吸的功夫。他用拳脚击碎石猴,却花费了十多分钟。

%53
这些石猴真的很滑溜,身手灵活得很。

%54
石猴立在地上的时候,根本就不要想用拳脚击中他们。它们往往一蹬腿,就能轻而易举地窜出去,躲避方源的攻击。

%55
唯一的破绽,就是当它们蹦跳在半空中时,无法借力。方源击杀它们,都是抓住这个破绽。

%56
但这也是仰仗了他丰富的战斗经验。若换做其他的二转蛊师,就算是赤山、漠颜、青书之流,也做不到像方源这样,次次都能抓住破绽。

%57
前世的记忆,让方源能敏锐地捕捉到,战斗中稍纵即逝的战机。他能精细地使用每一分力量,虽然只是二转。却能将自身的战斗力发挥到了极致。

%58
绝不会像方正,明明有玉皮蛊,但是在擂台上,被方源气势震慑,结果应该有的战斗力没有发挥出来。

%59
当然,方源目前的修为还是比较薄弱的。面对猴群,还做不到直接强推。

%60
每一次都是边打边退。

%61
幸好石猴智慧不高。不懂得修改自己的攻击方式。明明看到许多次,同伴在半空中被方源击碎,它们仍旧前仆后继地蹦蹦跳跳地向方源进攻。

%62
同时。它们追杀方源,每次距离过远,对家园的留恋之情。就会取代心中的愤怒。很多石猴都会主动地放弃追杀。

%63
蛊是天地之精,人是万物之灵。

%64
方源正是靠着人的灵慧,抓住石猴习性,采取正确的战术,才能不断地深入石林,如今他已经身处在石林的中央地带。

%65
如此三番五次之后,方源终于将这根石柱中的猴群全部剿灭。

%66
现在,垂在他眼前的,是最后一根石柱。

%67
它是这片石林中,最为粗壮庞大的石柱。堪称石柱之王。就算是五十个人合抱,也抱不过来。

%68
石柱从壁顶延展下来,几乎要触及地面。在静默中,隐隐散发着一股巍然之气势。

%69
方源数了数上面的石洞,略微估算了一下。至少得有五百只石猴。这是他目前为止,碰到的数目最多的一只猴群。

%70
不过玉眼石猴的数量再多,对于方源来讲,也不过只是一点小麻烦罢了,最多是耗费些功夫,多折腾几次罢了。

%71
真正让他目光凝重的。是最上层的一个石洞。

%72
这个石洞的洞口,比周围任何一个石洞都要大,至少要大出两倍不止。

%73
在它下面的石洞,紧密地排布着,有一种众星拱月,百鸟朝凤的格局架势。

%74
“看来这石洞里面住着一只猴王。”方源皱起了眉头。

%75
这才是问题所在。

%76
只要兽群一大,就有兽王产生。野猪群中,有野猪王。石猴群中,当然也会有石猴之王。

%77
兽王的威胁,比普通的野兽要高出许多倍。

%78
原因在于,它们的身上都会有一两只蛊虫寄生着。这些蛊虫和兽王之间,是一种共生合作的关系。兽王一旦遭受攻击,蛊虫也会出力帮助。

%79
“这只玉眼石猴王,应该不是很强,至少比野猪王要弱。否则周围的猴群,早就被它收服了。”方源靠着经验,暗自揣测石猴王的实力。

%80
一般而言,兽群越大,兽王就越强。弱小的兽王,是没有能力统御大量的同类的。

%81
若是按照兽群的规模大小,粗略地划分一下兽王实力的话,那么,从低到高,就可以划分为百兽王,千兽王,万兽王。

%82
病蛇小组围杀的野猪王,是一头千兽王。它统御的野猪,多达上千头。

%83
方源眼前的石猴王,则是百兽王,麾下的石猴有数百只。

%84
掀起狼潮的罪魁祸首雷冠头狼,就是万兽王。每一只雷冠头狼,至少统率一万头的电狼。

%85
这三等兽王之间的实力差距,相差很大。

%86
千兽王往往需要三支小组合作,才能艰难狩猎。病蛇小组之所以对付野猪王,有个大前提,就是野猪王早受了重伤。

%87
万兽王则需要一干家老和族长合力,才能正面对抗。

%88
至于百兽王,一支寻常的五人小组,就能妥妥的收拾掉。

%89
但是方源要对付这只石猴王,当然不可能借助外力,他只能依靠自己一个人的力量。

%90
“二转初阶的真元太不经用。看来是时候,用了那只赤铁舍利蛊了。”方源深深地看了一眼那个石洞,然后退回到第二密室,关上了石门。

%91
若拦在他面前的,是野猪王这种千兽王,那方源想都不想,就会选择避退。

%92
但现在是一只百兽王,若是方源是二转中阶的修为,倒是可以一试。

%93
当然结果怎么样,也说不好。哪怕是晋升了中阶,这事失败仍有七成半的可能,成功机会只有三成不到。

\end{this_body}

