\newsection{都鼓励牺牲}    %第一百四十五节:都鼓励牺牲

\begin{this_body}

作为当事人,方源自然受到了家族的调查。

但这调查并不深入,家族高层被越来越严重的狼潮,吸引了几乎全部的注意力。方源的地听肉耳草、隐鳞蛊都仍旧隐藏着,但就算是暴露出来,方源也可以将其全部推到商队的身上。

反正狼潮之下,交通已经断绝,商队也不会到来。这种情况下,家族得不到商队那边的确认,调查只会拖延下来。

真正能有调查结果的时候,那方源早就修行到三转,走出青茅山了。

当然,若是四味酒虫和春秋蝉,暴露出来,就另当别论了。

不论哪一个暴露出来,都会引起全寨的大震动。

四味酒虫,代表着全新的合炼秘法,意义极其重大。哪怕方源将这推到商队身上,也解释不了。必会受到家族的全面调查。

春秋蝉一旦被发现,什么狼潮都死一边去。这可是六转蛊虫!哪怕是族长古月博见了,也要立即撕破脸皮,什么狗屁的亲情,什么族长之位都可以不要。势必逼迫抢夺之,没有二话。

……

七月,炎热的夏季,空气若流火,弥漫着铁腥的血气。

狼潮越来越严重,战斗也越来越激烈。

很多人都开始意识到,此次的狼潮规模在过往的历史中,也属罕见的庞大。

豪电狼群已经沦为配角,上千头的狂电狼群一支支出没在山寨周围。

人类的生存空间,被压缩到极低的程度。

就连古月一族的大本营也是如此。山脚下的村庄就更别提了。

十室九空,一部分幸运的村民,通过各种关系,生存到山寨当中。但更多的村民,只好离开家园,为了躲避狼群,开始跋山涉水。

他们的目的地是另一处山上的山寨。但沿途的猛兽、野蛊。几乎漫山遍野的电狼,都让他们迁徙的希望压缩成渺小的微粒。

这是一场绝望之旅。

这些凡人已经被古月一族不得不舍弃,他们将在半途中全数死去。不是为猛兽果腹。就是被虫群杀死。

不管是凡人还是蛊师,都在生死间的挣扎。哪怕是平日里位高权重的家老们,也不得不亲自披挂上阵。

先前退隐下来的蛊师们。被重新征召起来。当狼潮结束之后,真正能活下来的绝不超过十分之一。

大自然的残酷,在此刻显露无疑。弱肉强食、优胜劣汰,不是喊两句温情的话,就能避免的。

……

方源盘坐在床榻之上,双目紧闭,心神沉入空窍。

空窍中,四成四的赤铁海波涛生灭,潮起潮落。这些真元,都是红的发黑。是暗红色的二转巅峰之真元。

二转真元,统称为赤铁真元。但在初阶、中阶、高阶、巅峰四个不同的小境界,赤铁真元亦存在着微微的差别。

二转初阶的真元,是浅红色的。二转中阶的真元,是绯红色的。到了高阶。则是深红。巅峰才是暗红。

许多天前,方源就从中阶修为晋升到了高阶。本来的深红真元,经过四味酒虫的提炼,以及元石的补充,全部转换为如今的暗红真元。

此时空窍中,四周窍壁再不是波光流动的水膜。而是白色光斑层层叠叠的厚实石膜。

白玉蛊、隐鳞蛊,都沉于赤铁海中。

四味酒虫在海水中嬉戏玩耍,但当春秋蝉的身影渐渐浮现而出的时候,四味酒虫立即嗖的一声,潜入到深深的海底去了。

这个时候,整个真元海面都会在春秋蝉的气场下,被镇得平滑如镜,一丝波澜也不产生。

春秋蝉的情况,越来越好。

两片翅膀已经完全恢复,仿佛一对新鲜嫩绿的树叶。只是主干身躯还是枯槁如前。

方源每隔一段时间,都会检查一番春秋蝉。现在他明显的感觉到,春秋蝉的恢复速度越来越快了。

以前的春秋蝉,就如同一个奄奄一息的病人,张不开口,只能喝流质食物,维持生命之火。

如今这个病人,却是可以下床,张开大口吃更多的饭菜补充营养。

恢复的速度自然越来越快。

除了春秋蝉之外,方源的空窍中还有新的两成员。

都是得自于白凝冰,一只是水罩蛊,如同水母一般,漂浮在海水当中。另一只则是赤铁舍利蛊。

方源从中阶晋升为高阶,是因为四味酒虫精炼了真元,先前也一直温养空窍,积累到质变的结果。方源还并没有使用这只赤铁舍利蛊呢。

单论修为,方源已经是古月一族二转蛊师中的第三人。熊力、青书已经陨落,就算是在整个青茅山,方源也是前五之内。

至于赤山、漠颜,原本是高阶修为,但在不久前已经双双晋升巅峰。算是给青书牺牲之后,士气低迷的族人们强振了一份精神。

方源在进步,别人当然也在进步。

之前,赤山、漠颜停留在高阶已经有一段时间,一直被巅峰的青书压着一头。

尤其是狼潮带来的死亡刺激之下,更令蛊师们对力量产生严重的渴求,压榨出他们的潜力,导致修为提升。

“但若战斗力,我必是整个青茅山的二转蛊师第一人。现在用了这赤铁舍利蛊后,修为拔升到巅峰,依靠丰富经验,甚至能在一定程度上对战三转蛊师。”方源心中思忖着。

先前的低调和隐忍,此时带给他丰硕的成果。

以丙等资质,能达到这样的进步,说出去族人都要震惊的。就算是甲等资质的方正,此时也不过二转中阶。

当然,就算是晋升巅峰。方源还做不到战胜三转蛊师的程度。

白凝冰当初能越级斩杀三转蛊师,是因为有北冥冰魄体。古月青书也能做到,是因为有木魅蛊这只强大而又特殊的三转蛊虫。

其实,说起来,方源也有一张比他们两人更强大的底牌――春秋蝉。

但这只六转蛊虫相当的特殊,方源逼不得已是不会使用的。春秋蝉还未完全恢复,若强行使用。是否能重生,也将打上一个大大的问号。

赤铁舍利蛊只对二转蛊师有效,留着不用是毫无价值的。

方源正欲用了这枚蛊虫时。门外传来一阵敲门声。

咚咚咚。

“方源大人,是我啊,古月江牙。”敲门声之后。一个人喊道。

方源皱起眉头,这个江牙最近越来越过分了,三番五次地来找自己,意欲收购更多的生机叶。

死亡和伤残越来越多,导致生机叶的价格也越来越高,甚至是有市无货。

“告诉你多少次了,我这里没有多余的生机叶,给我滚。”方源冷哼一声,他岂会为了区区元石收益,而放弃自我修行的时间。

门外。江牙谄笑起来:“方源大人,您消消气。现在的情况您也知道的,我这不是也没办法吗?许多蛊师知道我这里外卖生机叶,都找过来。小的我日子也不好过啊。这样吧,我把收购价再提高一成。方源大人呐,求你行行好,给我十几片救急吧。”

说到最后,他哀求起来,声音带着哭腔。

方源无动于衷:“那是你的事情,与我何干?哼。你现在胆子越来越大了,忘了我们之间的协议,居然私自带了其他人过来。”

“呃……”江牙站在门外,苦笑起来。他看向身边的这位蛊师老者,他也是没有办法,这位蛊师太强势了,硬是要跟着他来到这里。

“方源小友。”这位蛊师老者开口道,“我是古月野,相信你也听说过我。此次想向你求购一些生机叶,希望小友能看在老夫的面子上,抽出一点时间,生产一些。”

“看你的面子……呵,你有什么面子?”方源嗤笑一声。这古月野的确是有些名气,原先已经退隐,因为狼潮的缘故,被家族征兆复出。

他最高峰时,达到三转修为。但是因为受伤的缘故,修为落到二转巅峰。如今年龄老了,修为又下降到二转高阶。

虽然是和方源同样修为,但是战斗力早就不够看了。

古月野脸色铁青,他早就听人说方源此子孤独怪僻,脾气又臭又硬,傲慢无礼,目中无人。他来之前,也有了心理准备,但是没有想到,仍旧是低估了方源。

他想要卖个老脸,但这种无往而不利的法子,在方源这里不好使了。

一时间,他感觉到脸上火辣辣的,心中充满了羞恼之意。自己的脸面,这一次都被丢尽了!

“这个目无尊长的狗东西!”他心中咒骂一声,却没有因此离去。

他需要生机叶!

他是老江湖了,深知生机叶的重要。有时候,一片生机叶能挽救自己的一条命。

人越老,胆越小。

以前,年轻的时候,他会被鼓动,被热血充斥头脑。想要守护族人,想要改变世界,想要成为家族的大英雄!那时他有着觉悟,可以说是视死如归!

但现在他老了,冷静下来,这些年看透了。总算是回味过来。

尤其是几个儿女都牺牲之后,心更凉了。

任何的组织,都需要牺牲。

因为资源都是有限的。虽然每时每刻,都在生产资源,但资源也总是被消耗。这样一加一减,总量仍旧是有限的。

人要生存,需要衣食住行,这就是资源。蛊师要修行,需要蛊虫、元石、食料,这都是资源。

要变强,就需要更多的资源。但你不牺牲,我哪里来更多的资源?

死道友不死贫道!

什么守护、荣耀、亲情、梦想、热血,都是给牺牲一个光明正大的理由。

任何一个组织,没有一个不鼓励自我牺牲的。但是高层从不会明说,他们都会宣扬“守护、荣耀、亲情、梦想、热血、幸福”之类等等,并给与种种福利。

但人都死了,要福利有个屁用,一个死去的“大英雄”能享受到什么?

看那古月青书罢。

他“幸福”地死去了,被埋在土里,名字刻在墓碑上,精神感染着后来的“英雄”们。

(ps:次奥,你们以为我不想更新稳定?难道我不想再多更新一点?人人都有无奈之处的。另外,此书宣扬的东西,比较“偏激”,相较于主流观点,很有区别。大家看看,仁者见仁智者见智罢。这个星期补更新,预计每天三更,但更新时间可能不太稳定,实话实说,我已经尽了最大努力。你们不可以把全职写手的标准,安在我的身上,这不公平!当然,我也知道你们想看到更多的更新,我理解,也十分感谢你们的支持!)(未完待续。如果您喜欢这部作品,欢迎您来投推荐票、月票,您的支持,就是我最大的动力。)

------------

\end{this_body}

