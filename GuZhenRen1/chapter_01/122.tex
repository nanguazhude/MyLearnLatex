\newsection{人生几多风雪}    %第一百二十二节:人生几多风雪

\begin{this_body}

嗤嗤嗤!

接连三片脸盆大小的月刃,在空中划出幽蓝的光线。

嚓嚓嚓。

顷刻之间,就有十六七只玉眼石猴当场死亡。

追击方源的石猴群,顿时就少了一小半。

方源站在原地,并没有后退,而是再次举起右掌,劈空三次。

又是三道月刃,直直地撞入石猴群中,所到之处石猴翻倒一片。

石猴的尸体摔在地上,摔成一块块的碎石。眼球所化的玉珠,也滚跳在赤红色的地上。

方源查看了一下空窍,空窍中还剩下一大半的深红真元。

月芒蛊的一片月刃,需要一成的浅红真元才能催动出来。方源若是二转初阶,只能连续催动四记。到了中阶,上升到八记。到高阶,数量再增加一倍,提升到十六记。

方源虽然没有达到二转高阶,但是此刻有四味酒虫精炼出高阶真元,算得是伪高阶,因此战斗力暴涨。

原先这七八十只的猴群追杀,他需要且战且退,但是如今单靠月刃攻击,就能迅速剿灭绝大部分。只余下十多只,往往又望风而逃。

“不过才两天时间,就打通了三根石柱。这样的速度,比先前要快了数倍!这么算的话,一个半月后,我就能再次打通通往石林中☆央的道路了。”方源暗暗寻思着。

“按照huā酒行者的布置风格,石林中☆央的地洞,就是下一道关卡。在关卡之前,极有可能种下地藏huā蛊。不过到了此步,估计huā酒行者的力量传承也差不多了。毕竟他身受重伤,状态极为不妙,才仓促之间设下的这道传承。最大可能估计,往后也最多只有一两道的关卡了。”

有着前世的丰富经验,方源又想到了当初影壁上,huā酒行者那般浑身浴血,奄奄一息的景象,做出了判断。

huā酒行者布下这道传承,毕竟时间太短,没有办法做得更多。但这只是一个传承的特例。

其实正常的传承,都要经过蛊师经年累月的设计和布置。有的规模宏大,有的每隔十多年才开启一次。有的在传承的过程中,地点并不是只有一处,而是分割成数个地点,分布在五湖四海,相互间距可能相隔天涯海角。

后来者要继承这样的传承,非得经历十数年甚至数十年的时间,进行探索以及历经各种考验。

一些传承,终其蛊师的一生,也未必能探索成功。往往就将这未完成的事业留给了后辈子孙。

“huā酒行者的这道传承,属于微型传承,有个缺点,就是传承的东西比较稀少。但也因此有个优点,那就-< 书 海 阁 >-☆文字首发布置的关卡往往因地制宜,因此相对简单。我从这道传承中,先后得到了白豕蛊、玉皮蛊、酒虫。隐石蛊勉强也能算上。接下来剩下的,恐怕只有一两朵地藏huā了。但愿接下来的这些蛊虫中,有侦察类或者辅助移动的蛊虫!”时间匆匆,秋去冬来。

初冬,第一场雪。

天空阴沉,雪huā飘飘,洒落在青茅山上。

方源独自一人,行进在雪地中,他刚刚从石缝秘洞处出来,现在正往山寨中赶回去。

“算算时间,已经过去了两个多月,但是我打通石林的进展,却一直不佳。”方源的眉眼中藏着一股阴郁。

这并非是他不肯努力,而是狼潮的前奏,已经打响。

冬天,寒风中食物稀少,日益壮大的狼群为了收集更多的食物果腹,开始大范围的猎食。

这导致狼巢周围的野兽群,都被肃清。混乱不堪的小型兽潮频频发生,同时还有残狼潮。

这些残狼,都是被狼巢淘汰出来的。为了生存,它们成群结队,开始在山寨周围显形露迹,频繁活动。

虽然还没有达到冲击山村的猖獗地步,但是凡人猎户已经不敢上山打猎,同时村庄中不时的有村民被狼叼走而丧命。

古月山寨为此,调动大量的蛊师,进行清剿。这样一来,来来往往的人多了,其中又有很多的侦察蛊师,这让方源不得不明智地,大量减少前去石缝秘洞的次数。

这样一来,毫无疑问地,石林推进的速度就暴降了许多。

呼呼的寒风越吹越大,雪势亦是在变大。

吼……

一声声低沉的兽吼,夹杂在风雪中忽然传来。

方源倏地停下脚步,警惕地四处观望。

一支小型的残狼群,大约有二十多头的电狼,很快就出现在他的视野里。

“又来了么……”方源口中喃喃,这已经是他这个月碰到的第八次了。

但是这一次,又有些不同。

“靠着山寨这么近的距离,都有狼群在活动了。看来接下来,家族蛊师出击的次数会越加频繁。石缝秘洞又距离不远,看来接下来的这些天,我是不能再去了。”想到这里,方源心中不禁一沉。

行路难,这世上总会出现一些风雪,阻挡住人们一时前进的脚蛊真人吧☆清逸尔雅步。

狼群很快就包围了方源。

吼吼吼!

它们低沉地咆哮着,纷纷向方源扑杀过来。

“月芒蛊。”方源心头一动,甩手一记月刃呼的飞射☆出去。

幽蓝的月刃劈开风雪,连续斩中残狼,瞬间解决了三头,到了第四头时,残狼忽的就地一滚,狡猾地躲过了这记月刃。

这些残狼,虽然大多数都缺胳膊少腿,瞎眼少尾等等,但是它们有着丰富的战斗经验,极为狡诈。

若是一位普通的二转中阶蛊师,碰到这样的一群残狼,尤其是在被它们包围之后,性命就要受到强烈的威胁。

不过方源却临危不惧。

他丰富的战斗经验,以及四味酒虫精炼出来的高阶的深红真元,都是他的依仗。

杀杀杀!

他在残狼的围攻中,敏捷地腾挪,冷静地闪避,果断地出手。

一头头残狼,死在了他的手上。

很快,残狼群中电狼的数量,就锐减了一半。

嗷呜――!

一只头狼发出凄厉的嗥叫,狼群顿时收住攻势,开始撤退。

这就是残狼的狡诈。

一发现方源是个硬骨头,它们就果断撤退,放弃了围杀狩猎方源的打算。

这些老狼、病狼、伤狼,虽然没有完美的身体状态,但是能生存到今天,都有一套生存的智慧。

方源站在原地,静静地看着残狼群消失在风雪当中。自己的实力,能不暴露,就无须暴露。

确认了狼群已经彻底遁走之后,方源这才蹲下☆身子,匆匆地采集狼尸上的东西。

狼皮、狼牙等等,都是有价值的。

虽然市价比较低,但也耐不住量多。

方源这一两个月里,铲除残狼,靠着战利品也小赚了一笔。

雪地上,狼尸流淌出来的血液,还带着温热。有了残狼,虽然倒在了地上,但还没有死透,狼眼中还残留着一丝的华彩。

“在这自然中,万物都在抗争,都在生存,并不是仅仅只有人而已。这个天地,就是用生和死来演绎精彩的大舞台啊!”方源心中感叹着,毫不留情地给这些奄奄一息的残狼补上最后一刀。

一只残狼的战斗力,就已经高出了两只玉眼石猴。狼群相互配合之下,战斗力又会提升一倍。

“我对付这样的小型残狼群还好,要是对付大型的残狼群,或者今后对付一支小型的健康狼群,恐怕就要有麻烦了。”

方源隐隐感到一股压力。

“接下来,狼潮爆发,整个家族都要动员起来,我自然也不会独善其身。今后我独自一人,要在野外狩猎电狼,必须要有一只侦察类蛊虫,或者是辅助移动的蛊。否则的话,说不定就能陨落在这次狼潮当中。”

越是经验丰富,方源越是能认清自己的不足。

有了四味酒虫之后,他的战斗力已经暴涨许多。月芒蛊、白玉蛊在手,有攻有防,再靠着前世之积累,方源完全不属于青书、赤山、漠颜一流。

可以说,他已经勉强站在了家族中,二转蛊师的一流行列当中。

之所以是勉强,是因为他毕竟不是真正的高阶,同时资质也只是丙等,有限得很。

战力这块已经做到了目前最好,但要在狼潮中生存下来,战斗力只是一方面罢了。

“侦察手段不可或缺,若是我有一只侦察蛊虫,就可以提前感知到狼群的接近,而迅速远离,改变路线。或者利用蛊虫加速奔跑,脱离狼群的围杀,甩掉它们。”方源暗忖。

只有有这两种蛊虫中的一只,他的生存几率就会大增。两只在手,基本上就能做到游刃有余了。

“希望huā酒行者的传承中,有这样的蛊虫。若是没有,也不打紧。按照记忆,每逢狼潮,三大家族会联手布置战功榜,抛售库存的蛊虫,其中就有许多较为珍稀的蛊。到那时,我甚至可以利用战功,兑换白家寨或者熊家寨里的蛊虫。”

方源在心中谋算着,起了身。

这一会儿功夫,他就将战利品都麻利地打理妥当,装在一个袋子里,背在身上。

白雪纷纷扬扬,很快就将狼血冻结,将狼尸覆盖。

“快看,是方源回来了。”

“他背着袋子,又是出去狩猎残狼的么?”

“就是他拯救了我们山寨?”

“嘿,只是机缘巧合罢了。过程我们都清楚,如果我有那么大的力气,也足以做到。这不算什么。”

方源走入山寨,一路上行人们纷纷侧面,有的赞颂,有的好奇,有的嫉妒。

“方源。”赤山忽然出现在一处拐角处,叫喊了一声。

(ps:昨天重新立大纲,huā费了许多时间。过年嘛,应酬也多,你们懂的。已经是很抓紧时间了。有人反映看到了不同的第121章节,标题是“地听肉耳草”的,那是我之前的存稿,现在已经删掉了的。现在立的大纲,更紧凑了,情节也通畅了许多。衷心谢谢大家的支持!接下来,情况会变好的。)(未完待续!\~{}!

\end{this_body}

