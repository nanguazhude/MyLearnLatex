\newsection{命贱如草}    %第六十九节:命贱如草

\begin{this_body}

%1
兽皮地图很厚,不像是纸张折叠起来就能方便携带。方源只能将兽皮卷起来,放回到竹筒当中,再将竹筒两端用麻绳系着,背在背上。

%2
两个猎人紧紧地盯着竹筒,眼中不可避免地流露出贪婪的神色。他们也不是蠢蛋,自然知道其中的价值。

%3
这兽皮地图虽然蛊师们看不上,但是对于凡人,尤其是他们这样的猎户来讲,就是一个至宝了。

%4
王家从祖辈开始,就传承下来的东西。王老汉能成为远近闻名的猎头,这张图的功劳绝对不少。这就是名副其实的传家宝啊。

%5
“我问你们,这王老汉的家中,还有其他人么?”方源目光幽幽闪着,冷声喝问道。

%6
跪在地上的这两个年轻人,听了方源的话,顿时浑身一颤,想起了目前的处境,脸上的贪婪迅速褪去,被畏惧的神情所取代。

%7
“没有了,他们家全死光了,蛊师大人!”

%8
“王猎头原本有个婆娘,但是十多年前,就被闯入村中的野狼给杀了。他婆娘死之前,给王老头生了两男一女。但是大儿子王大,在三年前打猎,死在了山上。王家没人了。”

%9
两个年轻猎手连忙答道。

%10
“是这样……”方源眯起了双眼,他看着跪在地上的这两人,知道他们所言应当不假。生死都捏在自己的手中,骗自己的可能性很小。

%11
不过他仍旧问道:“你们没有骗我?”

%12
“不敢有丝毫欺骗啊,大人!”

%13
“我,我想起来了!王老汉其实还有一个媳妇,就是王大的老婆。但是王大失踪之后,那婆娘也殉情死了,那一年,山寨上面还特意送下来一个贞洁牌坊呢。不过我听说,其实王大的老婆想要改嫁,是被王老汉逼死的。大人您杀了王老汉,是除暴安良,为民造福啊。”

%14
另一个人赶忙附和道:“不错,不错。其实大人,我们也老早看这王老头不顺眼了。哼,有什么了不起的,不就是比我们会打猎么?明明都是凡人,搞的自己好像很特别似的,特意搬出村子,到这里来住。我们作为后辈,有时候想向他请教经验,他直接将我们赶走,还不允许我们再出现在木屋附近!”

%15
方源一边默默地听着,一边点点头,虽说这两人为了保命,开始贬低王老汉,但亦不难察觉这两人口中的怨气。

%16
方源暗自猜测,这两个年轻人请教狩猎经验是假,估计是觊觎王家女儿的姿色。结果被王老汉发现,狠狠地教训了几次。

%17
“除暴安良,为民造福的话不用多说了,我此次杀人,本来就是贪图这张兽皮罢了。嗯,你们二人的表现让我还算满意,你们现在就可以走了。”方源的语气缓和了一丝,同时背在身后的右手上,则亮起了幽幽的月光。

%18
跪在地上的两个年轻人听了方源这话,顿时又惊又喜。

%19
“谢谢大人的不杀之恩!”

%20
“大人,您的宽容和仁厚,我们永记在心!!”

%21
两人涕泪并流,额头碰撞在地上,发出咚咚的轻响。磕了几个头,他们立即转身就走。

%22
方源虽然比他们年龄还要小,但是他们亲眼目睹方源的行事风格之后,无比的心惊胆寒,再也不想面对方源了。

%23
“慢着。”就在这时,忽然传来一个声音

%24
话音刚落,刷的一声,从树梢上跳下一名蛊师。

%25
“你们不能走,把这里发生的事情都说清楚了。”跳下来的这名蛊师穿着一身深蓝劲装,系着赤色腰带,腰带中央镶着铁片,铁片上刻着一个大大的“二”字。

%26
这名二转的蛊师,身形瘦削,双眼细长。手腕上都带着护臂,小腿上有结实的绑脚,整个人透露出一丝精干的意味。

%27
“小民,拜见江鹤大人!”那两个年轻猎手刚刚站起来想走,见到这名蛊师就立即又跪下去,行五体投地的大礼。

%28
这名蛊师他们都认识,是村中的驻扎蛊师。

%29
每年,古月山寨为了加强对周边的控制,防止其他势力的渗透,同时强化边境防御,都会像下属村庄派遣蛊师驻扎在那里。

%30
这名叫做江鹤的蛊师,没有理会跪在地上的两个猎户,而是看向方源,冷声问道:“本人是古月一族驻村蛊师江鹤,你是?”

%31
方源淡淡地笑了一声,将右掌伸出来,掌心中月光蛊正散发着一团盈盈的月辉。

%32
蛊师江鹤看到方源掌中的月光,目光顿时柔和了许多,月光蛊乃古月一族的标志蛊虫,是作伪不得的。

%33
“这事情说来也简单,王老头的二儿子冒犯了我,我一怒之下杀了这家人。他们二人可以为我作证。”方源理直气壮,直接承认,同时指了指地上跪着的两个青年猎户。

%34
方源说的不假,这俩猎人忙不迭地点头,没有一丝的犹豫。

%35
江鹤楞了一下,便哈哈大笑:“杀的好!区区的一介农奴,居然敢冒犯主子,该杀!!”

%36
但是紧接着,他话锋一转,饱含深意地道:“不过学弟啊,你这样做了也叫我有些难办啊。虽说这王老头离群索居,独自一家住在这里,但他们几个毕竟也是我负责的村民。我被族中派遣过来,驻扎在村子里,就是要保护村民,警戒防御。现在你杀了这几人,村中人口就减少,年末考核时,族中对我的评价就会下降啊。”

%37
方源目光一闪,顿时就知道这江鹤是想要借此讹诈自己的钱财。

%38
他笑了笑,直接道:“这有什么难办的。学长你照实回报就是了。就说这一家全是我杀的,和学长你没有任何关系。”

%39
江鹤听了这话,眼皮子抽动了一下,心中暗怒这方源不识抬举,语气变得十分严肃:“那我就秉公办理了。学弟,你要是不怕家族的追究,就报上你的名字吧,我会详细记录下来,如实上书的。”

%40
他这话中威胁的意味很浓,若是别的少年兴许就被这话吓住了。

%41
但方源却从中看到了他的虚弱,当即就道:“学长如实禀告就是,对了,我姓古月,名方源。”

%42
“原来你就是方源!”江鹤脸上明显地诧异了一下,“我最近总是听弟弟说起你。你痛殴全部同窗,当众勒索,每次都能有六十块的元石。我弟弟每次说起来,都在羡慕你这钱赚得真容易。还有你竟然在赌石中,连续开出了癞土蛤蟆和酒虫。这运气实在叫人嫉妒啊。对了,我弟弟就是江牙,你们应该早就见过面了。”

%43
“原来是他。”方源点点头,承认道,“的确,我每次购买月兰花瓣,都是在江牙那里的店铺。”

%44
“哈哈哈,既然如此,那我们就是熟人了。算了,这件事情江鹤担了!”江鹤说到这里,用手拍着胸脯,做出一副义气凛然的样子。

%45
他是借坡下驴,怎么可能真的汇报这事。

%46
汇报了之后,家族对他的评价仍旧会降低。索性不如卖个人情给方源,至于王老汉一家的死,直接上报一个野兽侵袭就可以了。

%47
谁叫这王老头特立独行,偏偏要搬出村子,在这里搭木屋离群索居呢!

%48
江鹤也不怕这事被捅出来,左右不过几个凡人农奴,命贱如草,死了就死了,族中就算发现隐瞒,也根本不会在意这种小事的。

%49
“只是学弟啊,你还没有从学堂走出来,没有学长我的这身蛊师的衣裳。否则那王二看见这身衣裳,就知道你蛊师的身份,又怎么会冒犯你呢?学弟你是个聪明人,你说是不是,呵呵。”江鹤又道。

%50
方源目光闪了一下,江鹤说的含蓄,其实是在告诫自己——不要没事从学堂跑出来晃悠,今后也最好不要在他负责的这块区域,再杀人闹事了。

%51
“那就谢谢学长指教了。”方源抱拳一礼,辞了江鹤,直接便离开了这里。

\end{this_body}

