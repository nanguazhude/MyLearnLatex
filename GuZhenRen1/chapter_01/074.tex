\newsection{魔道的觉悟!}    %第七十五节:魔道的觉悟!

\begin{this_body}

树屋中方源念头翻腾不定。

凭借着前世经验积累出来的智慧,方源察觉到了王大的存在,

那么王大的实力是如何的呢?www.13800100.com

方源没有见过这个王大一面,但是单凭眼前的这些东西,就足以让他分析出很多情报来。

“勇气都是建立在实力的基础上的,他收集了我的资料,仍旧要向我展开报复,如此有把握,看来他的修为必定超过一转!”

“他这三年都辗转在这三处红圈标记的地方,在三大山寨的夹缝中艰难生存,几乎每天都要冒着被发现,被围剿的危险,这说明他没有能力独自一人外出闯荡。而独自闯荡需要的蛊师修为,至少得是三转。”

“这样一来,初步估计,他的修为是二转。”

方源双眼中闪烁着冷冽的光芒:“算算看,他三年前失踪,如今修行到二转,资质应该是丙等至乙等,空窍中的真元总量在四成至七成的范围内。”

“这三年来,他能够在夹缝中生存。同时能在江鹤的眼皮子底下,住进家中,他拥有的蛊虫中一定有潜行匿迹作用的。”

蛊师对战,最重要的就是情报。

方源虽然没有任何侦测类的蛊虫,但是凭借自身的经验和智慧,硬生生地将王大的修为和蛊虫,都大致地分析了出来。

很快,他在心中就勾勒出了一个魔道的二转蛊师,怀着家人被杀的滔天仇恨,潜伏在某处想要猎杀自己的形象。

“我能杀别人,别人自然也能杀我。这也没有什么。”方源不禁轻笑出声。

这个世界上,谁都有权利活着,谁都有机会死去。

杀人者,人恒杀之。

既然自己杀了人,也就要有被杀的准备。

若是就这样被杀,那么死就死吧,没有什么大不了的,也绝没有后悔。这都是自己选择的路。

这点,方源早已有了至深的觉悟。

这就是魔道的觉悟!

“王大想要杀我,那么现在这次年中考核,绝对是一个大好良机。平常的时候,学员们都居住在山寨当中,凭他二转修为,不可能潜入山寨,那是纯粹找死。”

“他也许分析到了,我会出来狩猎的可能。但是青茅山这么大,他一个人一边要隐藏行迹,一边要独自一人搜索我的踪迹,太困难了。现在的这个机会,他最有可能动手。”

“他是二转蛊师,而且是那种三年来都挣扎生存,过着朝不保夕的日子,锻炼出来的魔道蛊师。以我目前的战力,必然不是他的对手。但这并不代表,我没有了一丝生机。”

逃!

方源瞬间决定了这个方向。

为了生存没有什么可耻的,既然不能力拼,那就逃。

临阵突破这种事情,对于蛊师来讲,根本不可能发生。越级挑战,倒是可以,但那也是建立在蛊师手中有特殊蛊虫的基础上的。

方源手中有不少蛊虫,但是春秋蝉濒死沉睡,不能使用。酒虫、白豕蛊、小光蛊、月光蛊,都不是能越级挑战的底牌。

明知道不敌,仍旧死战,那是名为“热血”的愚蠢。就算是胜了,也不过是命运的垂青。

方源一生唯谨慎,哪怕是有底牌,只要是胜算小,他也会选择尽量避免交战。

他喜欢掌控局面,用各种手段尽量将胜率放大到极限。他最喜欢打的,就是必胜的战斗。

只有到了万不得已的时候,他才会冒险激战。

因此他常做的事情,就是欺凌弱小,掠夺资源,不断强大。强大到超越原来敌人的程度,再回来找回场子,也就是继续欺凌弱小。

这没有什么可耻,那些为了证明自己勇敢,而去主动挑战,冒着生命危险和强敌死磕的,才是真正的蠢货。

但偏偏这个价值观一直得到宣扬表彰,这是因为任何的组织,都需要个体的不断牺牲,来维护组织高层的利益。

只要想想就知道,生存才是一切活动的前提。

为了生存下来,实现心中的理想,才是一个人最大的勇敢。

为理想而死,那是蠢货。为理想而苟且偷生的活着,那才是勇士!

地球上,韩信受胯下之辱,曹操被追杀的割须断袍,越王勾践为了活下去,给仇敌尝大粪表忠心……

所以,去他嘛的荣耀、名誉和面子!

不管哪个世界的组织,都会宣扬这种价值观。越是需要牺牲的地方,就越会宣扬这个。比如军队。

“那么该往哪里走,才能将遭遇王大的可能减少到最小?”方源的脑海中浮现出一张地图。

“王大已经知道我有了兽皮地图,现在他应该潜伏在山林当中,按照地图上野猪的分布,在四处寻找我。我不能去这些地方,只有反其道而行,才能搏出一线生机。”想到这里,一个略显疯狂的撤退路线,在方源的脑海中隐约成形。

……

黄昏下的山林,树影叠叠,野草茂盛。

一双猩红的眼睛,隐藏在深深的阴影里。双眼中蕴含的仇恨和愤怒,简直滔滔江水都洗不净,扑不灭。

“方源,终于让我找到你了……”王大咬着牙齿,在肚子里撕吞咀嚼这个名字。

在他的目光注视之下,不远处,一位身材瘦削,脸色苍白的少年在山林中急速地奔行穿梭着。

仇人近在眼前,但是王大却没有动手,而是将目光隐晦地转向其他几个位置。

在这几个位置,都隐藏着一位监考蛊师。

为了防止作弊,第一时间救治伤亡等原因,附近这片区域,隐藏分散着数十位二转蛊师。还有高达三转的几位家老,在远处的山坡上坐镇。

王大小心翼翼地在这片山林中潜行穿梭,已经收集到了不少相关的情报。

“我要杀了方源,就必须先将周围的这三位蛊师都消灭掉。否则一旦现身,就会遭到阻击。或许在第一时间,就可能杀死方源,但是我也会被随后赶来的蛊师围攻杀死。”

“我是二转中阶的修为,体内的真元还有五成。要先发制人,先杀死这三位蛊师,很有难度。必须要在短时间之内,连续出手。否则一旦让他们发觉同伴的死亡,警惕性大增,我的行踪也就暴露了……”

“幽影随行蛊。”王大慢慢地闭起双眼,心中默念一声。

旋即,他的身体很快就化为了一团浓郁的黑暗,在树影中流淌穿行。

一切都悄无声息。

一处茂密的草丛中,古月一族的一位二转蛊师,蹲在里面,无聊地打了一声哈切。

“还真是无聊啊,陪着这群学弟,真像是保姆似的。”这位蛊师小声嘟囔着,丝毫没有察觉一团阴影已经笼罩住他。

一双瘦骨嶙峋的手,从黑暗中慢慢地伸了出来。

这双手苍白无比,骨节宽大,十片指甲锐利地凸出,指甲片都染了一层暗紫色,微微散发出一股腥气。

“这是什么味道?”古月族的蛊师抽动了一下鼻子,下意识地皱起眉头。

他刚要察看,但早已经晚了。

王大如毒蛇扑食的瞬间,闪电般出手!

一手捂住蛊师的鼻口。另一只手直接从背后阴狠一插。暗紫的指甲锋锐如刃,他的五指轻松地没入蛊师的身体内,抵挡他的心脏。

指甲上的剧毒,在一刹那的时间里,就侵蚀了心脏。并通过血液,传遍蛊师的全身。

蛊师浑身一紧,就再也没有了气息。

虽然同为二转蛊师,但是一个偷袭,有心算无心,战斗刚刚开始,就已经结束了。

“总共用了一成真元,还剩下四成。”王大留神了一下空窍,没有停留,再次融化为一团黑暗。

片刻之后,隐蔽在山石后的第二位蛊师,也惨遭王大的毒手。他的瞳孔缩成针尖大小,倒在了地上。

毒素在他的身体内肆虐,他很快就浑身泛紫,从鼻腔中蜿蜒流淌出两道暗紫色的血液。

“还剩三成真元。”王大默念一声,身形又再次化为了一团黑暗。

“什么人!”第三位蛊师隐蔽在一棵大树的树枝上,他在关键时刻察觉到了不对劲,在王大动手之时,猛地转身,两只手牢牢架住王大的两只毒手。

“该死!”王大狞笑一声,十片指甲猛地疯长起来,在瞬间伸长五厘米,插进蛊师的前臂,刺破他的皮肤。

小臂上流淌出的鲜血,迅速地转变成了暗紫色。

“这是爱生离?!”第三位蛊师看到这个情形,顿时惊骇欲绝。巨毒的紫气已经染上他的脸庞。

他自知身上没有抗毒的蛊虫,此次必死无疑,猛地流露出绝然之色,猛地大吼:“那就一起死吧!”

他张开嘴巴,猛地伸出舌头。

舌头上有一个月牙的印记,正是寄居着的月光蛊。

一道月刃闪出,正中王大的左肩,然而从他的后背穿射而出。

血液飚溅。

王大闷哼一声,身体摇晃了几下,而这蛊师却已经双眼翻白,再无一丝气息。

“没错,这就是爱生离……”王大在结实的树枝上,摇晃地站起来,露出一丝惨笑。

爱生离,号称二转蛊虫第一毒!要炼成,须得需要一转生息草、寡妇蛛、红针蝎,还有一颗爱人之心。

为了炼成这只蛊虫,王大亲手杀死了深爱自己的妻子,挖出了她的心!

“一切为了生存,只好选择力量……这就是我魔道的觉悟!”王大双目赤红,紧紧地盯着远处的少年。

“我放弃了爱情,只剩下亲情,你却将它剥夺!方源……”他低沉地嘶吼着,“我要让你无比后悔你所做的事情!”

------------

\end{this_body}

