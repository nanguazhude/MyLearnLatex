\newsection{智窥迷雾现杀机}    %第七十四节:智窥迷雾现杀机

\begin{this_body}

树屋的空间并不大,里面的东西却很多,一眼望去显得杂乱无比。

中央的地板上,铺着一张暗黄色的厚实毛毯。www.13800100.com

屋内靠着墙壁,是一个烧木柴的铁炉子,炉子上还架着一个铜水壶。炉子中有着黑色的灰,旁边则是一小堆未烧的干柴。

别看是夏天,在这山中的夜晚,照样阴冷。铁炉子虽然小,升上火,就能保持整个树屋的温暖。

树屋开着两扇窗户,两三根麻绳绕着一侧窗户的外沿,横穿树屋,然后固定在另一侧窗户的外沿。

麻绳上则担着几件破旧的衣服,衣服上打着补丁,明显是成人的衣服,还带着一些水气,没有彻底晾干。

落日的余晖透过窗户,映照在树屋中。

树屋内光线昏暗,墙角边上的铁斧、猎刀的把柄上,都缠着兽皮。刀刃上还有暗红的血迹。

在另一处的墙上,贴着一张竹纸,一把匕首正插在中央。

竹纸上画着一个少年的脸,赫然就是方源的面目!

这一切都在表明,在最近一段时间,有一个人来到这处隐蔽的小树屋中,隐居生活着。这个人的目的,显而易见,就是竹纸上画着的方源。

纸上插着的一把匕首,将他的险恶意图,表达得淋漓尽致!

这样的场景,由不得当事人方源不动容。

“这个人想要干什么,为什么针对我?不,也许不是我,而是方正。”方源脑海中思绪剧烈翻腾。

方正是甲等资质,古月一族三年中唯一的一位甲等天才,是古月一族的希望。若真能培养出来,就是下一代的家族旗帜。

但是培养总需要一个过程。

在这个过程中,有天灾,更有人祸。

天灾且不去谈它,关键是人祸。众所周知,青茅山不是只有古月山寨一家,而是还有着白家寨、熊家寨。这两股势力绝对不会乐意看到,传统霸主古月一族中顺利培养出一位甲等天才。

因此,派遣刺客暗杀,是常有的事情。

这个世界上,天才少,能顺利成长的更是少之又少。

甲等资质的蛊师,不是没有。三年以前,古月一族就出现过。在更早的年代,也屡次涌现过。

但是青茅山上三大山寨,在这些年加起来,甲等天才只培养出一位,那就是白家寨的白凝冰,如今已经是三转修为。

单单这个现象,就足以说明许多的东西了。

“这个人,是白家寨,还是熊家寨的?现在就要开始抹杀古月方正了么?”方源皱起眉头,盯着木墙上的肖像。

“但为什么,王老汉的兽皮地图上,会有这个树屋的标记?难道王老汉是其他势力私下发展出来的一条暗线?不对,这个人是针对我来的!”

方源眼中猛地绽放出一抹厉芒。

这一刻,他回想到了很多的画面。

第一次,在陷阱旁边,他听到那四位年轻猎手的对话――

一个猎手说:“王二哥,要我说你过年也十九了,这么大也该娶个老婆生娃子了。”

王二却道:“哼,男子汉大丈夫,怎么能贪图这点小小的美色!总有一天,我要走出这个青茅山,外出闯荡,见识天下,才不愧是我这男人身!”

第二次,是自己动手之后,王二的异常镇定。他弯弓搭箭,直指方源。而其他三人则已经在求饶了。

第三次,是方源问话。

“我问你们,这王老汉的家中,还有其他人么?”

有个猎手答道:“王猎头原本有个婆娘,但是十多年前,就被闯入村中的野狼给杀了。他婆娘死之前,给王老头生了两男一女。但是大儿子王大,在三年前打猎,死在了山上。王家没人了。”

“我,我想起来了!王老汉其实还有一个媳妇,就是王大的老婆。但是王大失踪之后,那婆娘也殉情死了,那一年,山寨上面还特意送下来一个贞洁牌坊呢。不过我听说,其实王大的老婆想要改嫁,是被王老汉逼死的。大人您杀了王老汉,是除暴安良,为民造福啊。”

另一个人赶忙附和道:“不错,不错。其实大人,我们也老早看这王老头不顺眼了。哼,有什么了不起的,不就是比我们会打猎么?明明都是凡人,搞的自己好像很特别似的,特意搬出村子,到这里来住。我们作为后辈,有时候想向他请教经验,他直接将我们赶走,还不允许我们再出现在木屋附近!”

王老汉一家,搬出村子,离群索居……

大儿子王二,在三年前打猎,死在了山上……

王大的媳妇,想要改嫁,被王老汉逼死,得了一个贞节牌坊……

王老汉将向其请教狩猎经验的年轻人都赶走……

王老汉一心隐瞒兽皮地图,在交给方源的竹纸上,根本就没有画这三个红圈……

王二年纪轻轻,面对蛊师却很镇定。同时不娶妻,心中有超越凡人层次的壮志……

更关键的一点,兽皮地图上标记的红圈位置,是一个隐秘的栖息地。在这里明显有人类活动的迹象。同时这个人对方源或者方正有强烈的恶意……

这一条条的线索,若分割开来,不容易为人察觉。但是一旦联系在一起,就蹊跷了!

方源越是思索,越是觉得眼前笼罩着的层层迷雾,越来越稀薄。

落日的余晖,透过窗户,映照在他的脸上,浓重得仿佛是一抹血色。

周围似乎陷入了一种死寂,背后是不是有人一直在偷窥着自己?

忽然间,方源双眼骤亮,视线仿佛穿透了时光,他看到了真相!

“王大,并没有死。”

这一刻,他的眼中精芒四射!

“不仅没有死,他还机缘巧合,成了一名魔道蛊师!”

凡人并不是没有修行资质,而是家族体制一直将蛊师的修行方法,严密地掌控在手中。

但世间的事情,从未有绝对。

凡人也能有修成蛊师的例子,比如在野外意外遭遇希望蛊,开了空窍。比如继承了某个力量传承,又比如得到了某个家族中人的私相传授等等。

但这样的蛊师,向来都不被家族所容,最多只能成为家族外围的打手。于是,并不甘心的他们,就成了独来独往的蛊师,修行条件艰辛无比,久而久之,为了争夺资源,就得烧杀抢掠,这就步入了魔道。

“因为某种机缘,可能性最大的是在某个蛊师尸体上,得到了一笔遗产,王大在三年前成了一名蛊师。为了掩人耳目,他放出死亡的假消息,实际上根本没有人见到过他的尸体。王老汉一家知道这件事情之后,就搬出了村子,冒着被野兽侵袭的风险而离群索居,为的就是继续掩盖这个真相。”

“但是其中出现了一些分歧,王大的妻子不同意,也许是想要上报古月山寨。王家就只好将这妻子杀害,又故意放出一些似真似假的小道消息。什么想要改嫁,什么逼死之类的,将真相牢牢地掩盖在这些流言蜚语之下。”

“王大每隔一段时间,会回家居住。王老汉就赶走那些上门求教经验的年轻人。王二和哥哥王大相处之后,对蛊师的惧怕之情渐渐消除,自然而然地生出了自己也要成为蛊师,外出闯荡的雄心壮志!”

“因为担心王大的行迹泄露出去,王二这么大的年纪,都没有冒然娶妻。王大并不能总在家中居住,所以王老汉在兽皮地图上画着三个红圈,这三个地方,应该都是这样的隐秘居所。狡兔三窟,王大就在这些住处轮流居住,生活在三大山寨的势力夹缝之中。”

这三个红圈的意义,终于到此刻明了了!

红色往往代表着醒目,重要,对于王老汉来讲,这是他大儿子的居所。所以方源要王老汉画地图,他没有在竹纸上画这三个红圈,为的是保护大儿子。同样也少画了一些红叉,想要报杀子之仇。

“也许每年的固定时间,王大都要回家居住一段时间。他回来之后,看到家人都死了。秘密打听之后,就查出了我,所以墙上用匕首钉着我的画像,就是为了找我报仇!”

凭借前世的经历,方源几乎可以确信,这就是真相了。

若是其他势力,想要暗杀方正,完全可以更隐秘一些。也不必发展王老汉这样的凡人暗线。毕竟王老汉是住在山脚下,而不是古月山寨子里。

“想不到我为了一张兽皮地图,惹来了一个魔道的杀手。这世间之事,真是奇妙啊。”方源情不自禁地冷笑起来。

他先杀王二,后杀王老汉、王家妹子,是为了更直接的得到兽皮地图。王老汉是附近村庄中第一的猎人,他的兽皮地图当然最有价值。

当时想着:三条凡人的性命罢了,有什么了不起的,顺手杀了就杀了。

这个世界上,谁都有权利活下去,谁都可以死。

现在没想到,居然蹦出来一个魔道蛊师!

对此,方源没有丝毫的后悔,反而感到庆幸。

如果他当时心慈手软,想要获得王家的兽皮地图,王二、王老汉、王家妹子都将是巨大的阻力。他们为了保护王大的秘密,绝对不会掏出真品。

王二的战力,已经能灭杀一般的一转高阶蛊师。王老汉更是老而弥辣,真正厮杀的话,比王二的威胁还要更大。

就算是得了真品,王老汉将详情告知王大,王大会立即得知方源的准确信息。不管他怎么处理,他在暗方源在明,主动权都牢牢地掌握在他的手中。

“幸亏自己一动手就杀人!不管如何,一直将局面掌控在了自己手中。这样一来,哪怕最终得不到真品,自己也没有损失,还可以再抢其他猎户的。而且干脆利落地杀了王老汉和那丫头,让这王大耗费时间搜集自己的情报。不用说了,那活下来的两个年轻猎户,一定已经被他杀人灭口了。”方源心中一片了然。

知情者只有几个,王大不会对江鹤动手,江鹤一死,族内就会派遣人手调查。而那两个猎户在山上失踪的话,就很好操作。江鹤为了族内关于自己的评价,也不会暴露这种事情,反而会替王大遮掩。

\end{this_body}

