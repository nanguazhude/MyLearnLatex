\newsection{有些可爱}    %第四十八节:有些可爱

\begin{this_body}

这场大雨连下了四天,这才停息。

太阳高高地升上天空,掀开这雨帘,仿佛在为夏季揭幕。

夏天的气息,已经若有若无地传来。

天气变得越加晴朗,一扫春天的缠绵悱恻的气息,温度也渐渐升高。

春夜中,活跃的龙丸蛐蛐退场了,它们潜伏到地底深处进行产卵。青茅山上特有的青矛竹,则开始疯长,几乎每天都有明显的增高。

草木树叶开始由翠绿色,往深绿色转变。青山绵延,看起来更加郁郁葱葱。

万里晴空,蓝如水晶。

啪啪啪。

学堂的演武场中,传来一阵拳脚交击的声音。

交手了十几招后,古月漠北被方源一脚踢中腹部,踉跄着倒退五六步,正好退出了地上画着的圆圈。

拳脚教头站在场下,一直在主持着场面。看到此景,立即宣布:“古月漠北被击出场外,古月方源连胜三十三场!”

“哼,这次又输给你了。”古月漠北咬咬牙,双眼紧紧地盯着方源,“不过你不要嚣张。终有一天,我会打败你的。而我已经感觉到,这天已经越来越近了!”

方源面无表情地看了他一眼,随后垂下眼皮:“刚刚那一脚,已经把你踢出了内出血。你还是先把伤治好再说罢。”

“这点小伤算什么?”古月漠北刚反驳了半句,忽然面色一变,喉咙不禁一滚,呕出一口鲜血。

他面色陡白,这还是他首次受到这样的伤!他的双眼中不必避免地流露出惊恐之色。

拳脚教头连忙走了过来,安慰道:“不要紧,这点伤势经过治疗,静静地修养几天就好了。只是这几天要暂停练拳,不能再剧烈活动。”

话音刚落,两位待在场外的治疗蛊师,就赶了过来,小心翼翼地将古月漠北搀扶了出去。

古月漠尘不敢再说话,他深深地看了方源一眼,目光中充满了恼怒、愤恨、遗憾,以及不甘心。

“漠北虽然拳脚功夫很好,但还是打不过方源啊。”

“方源太厉害了,根本就没有人能打得过他!”

“漠北居然被打得吐血,真是可怕呀。我可不想跟这个家伙打。”

“唉,可是教头说了,今天是实战练习,打擂台!每个人都必须上去打一遍的。”

学员们站在场外,有的看着方源流露出惊惧之色,有的唉声叹气,有的脸色苍白,有的提心吊胆。

这其中,还有一些人都受了伤。有的捂住鼻青脸肿的脸面,有些摁住胳膊,嘴里倒抽着冷气。还有些直接躺在地上,揉捏着大腿。

“下一位!”迟迟没有人再上去,教头喊叫一声。

但是没人应答,平日里有勇气挑战方源的,只有古月漠北、古月赤城,还有古月方正三人。但是这三人,都已经被打了下去。

学员们一片寂静,有些人甚至微微地向后退缩。

教头看着学员们胆怯畏缩的表情,眉头紧紧皱起来。

他不由地想起学堂家老关照他的话:“这些天,方源的风头太盛了,必须压一压。其他所有的学员都被他压得抬不起头来,长久下去,他们心中的勇气也会被磨灭。我们学堂培养的是敢于对抗强敌的虎狼,而不是怯战的羔羊。”

“你们都怎么了?方源再强,也只是十五岁,是你们的同龄人!他和你们一样的岁数,和你们吃着一样的饭菜,喝着一样的水。他没有长三头六臂,他又不是怪物!拿起你们的勇气来,让我看看你们心中埋藏着的古月一族的骄傲!”教头咆哮着,极力鼓动场下的学员。

“但是他真的太强了,我们根本就打不过他啊。”

“和他交手的同窗,都好惨的。漠北甚至还被打得吐血了。”

“方源出手越来越重了,教头,我们不敢跟他对打啊。”

学员们小声地,懦弱地反驳道。

教头气得差点跺脚。这帮无知的小儿!

他在场下看得很清楚,方源三十三场车轮战打下来,一点都没有得到休息。虽然方源一直在调整着呼吸,但是体力已经所剩无几了。

方源下手越来越重,这更说明:他再也不像先前那样的得心应手,他对场面已经逐渐失去了掌控。

若再加一把劲,就能逼他露出疲态。再上几个人,就能将他当场击倒!

一旦击倒了方源,他的威慑力就会暴降,学员们的勇气就会被激发出来,打压方源的目的也就达到了。

但是现在,学员们都被方源撑起来的虎皮给唬住了。

有时候,击败自己的,不是强敌,而是自己的心。

教头心中焦急,又继续鼓动。

但他这个人并不善言辞,先前说的也是这番话,引发了少年们的热血,激出了一些挑战者。但是现在这话说得太多遍了,少年们都听得麻木了。

方源环抱双臂,冷眼旁观这一切。他虽然站在场地最中央,但是此时却像是一个毫不相干的外人。

教头鼓动了半天,学员们仍旧面面相觑,没有一个人动弹。

拳脚教头不禁既气愤又无奈。他转身面对方源,不满地喝斥道:“方源,你也有错。你出手越来越狠辣了,同窗之间切磋,应该温和友爱一些,怎么下得了如此重手呢?接下来,你要多注意一些,小心出手。若再把同窗打得吐血,我就判定你输,让你下场!”

“教头,你错了。”

方源冷哼一声,目光毫不示弱,对上拳脚教头,“切磋打斗,自然要全力以赴,否则怎么能起到锻炼的作用呢?难道将来作战的时候,我们也能要求敌人温和友爱么?”

拳脚教头大怒:“哼,你出手歹毒,伤害同窗,还敢强词夺理!”

“教头,你又错了。”

方源冷笑着:“是你安排了这场切磋,又将头名奖励拔高到二十块元石的地步。没有你的鼓动,这些人能受伤吗?”

“混蛋!”拳脚教头并不善言辞,指着方源,咆哮起来,“你还想不想要头名奖励了?你再狡辩的话,即便得了第一,我也会判你输!你如此不团结友爱,又擅自顶撞师长,怎么能有资格得到二十块元石奖励!”

方源哈哈一笑:“不过一场输赢,二十块元石而已,你以为我稀罕?”

说着,转身就走,在众人错愕的目光中,走出了中央的场地。

虽然影壁没有卖出去,但是方源手中却有数百多块元石。况且他此次上场,也不是为了元石而来。

“你!”看到方源真的走下台,教头不禁当场瞠目,流露出吃惊和迷茫的神色。

十五岁的少年,不应该都是血气方刚,好斗争勇的么?

方源有如此战斗才情,性情更应该如此才对啊。怎么就直接退场了呢?

而且,这方源没有资助,元石应该很吃紧才对。怎么这二十块元石,都吸引不了他?

一时间,拳脚教头愣在原地,不知道怎么办才好。

方源不入瓮,直接就下了场。

教头忽然发现:还真拿方源没有办法。以他身份,总不能明目张胆的找茬,强逼方源上场吧?

周围的学员纷纷后退,跟方源保持距离。方源站在场下,身边没有一人。以他为圆心,周围五步之间,呈现出一片圆形的空白地带。

可惜。

若是他们站在方源的身边,就会听到方源尽全力压制的喘息声。

“体力不够用了。”方源在心中叹息。表面上他一副精力充沛的样子,其实在衣服覆盖下的身躯,都在微微颤抖。

他的身体到底只是十五岁,又没有相关蛊虫的辅助。连续对战三十三场,已经接近了极限。

虽然有前世丰富的战斗经验,但是这段时间内,其他少年们的拳脚功夫,都有很大进步。从他们身上,方源已经感到了一股越来越强的压迫力。

这种压力直接体现在方源的出手。他出手越来越重,渐渐收不住手了。以前比斗,学员们太弱,因此他得心应手,少年们顶多也只是跌打损伤。但是现在,他对场面的掌控程度越来越弱,必须下重手,才能稳住场面。

“经验到底不是万能的。任何思想中的力量,都需要物质的基础,才有发挥的介质。”方源眯着眼睛。其实拳脚教头的心思,他早就看破了。

他并不意外,亦早就料到,会有来自学堂方面的打压。

自从他杀了高碗之后,敢向他挑战的人,就越来越少。勒索的时候,更多的人,都迫于他的威压,不敢反抗,乖乖地贡献了元石。

长期以往下去,方源不可战胜的形象,就会确立。给其他少年心中留下阴影,让他们从最初就对自己的拳脚功夫不自信。这是学堂家老绝不想看到的事情。他需要方源来促进学员们进步,但他不想方源彻底磨灭了学员们的战斗激情。

他想看到方源的失败。

方源一旦失败,他在前期树立起来的不可战胜的形象,就崩塌毁灭了。

与此同时,就唤醒了其他学员们的战斗激情。并且挫折之后,更能塑造他们百折不饶的战斗意志。

但是对方源来讲,他需要这样的威慑,才能让勒索元石的行动更方便。

若是他败了,少年们意识到他的虚弱,就会群起而攻。方源虽然手头上元石颇多,但是抢劫勒索,是他元石的主要来源。失去了这个源头,他就会成为无源之水。

所以方源上场,连战三十三场。真正的目的,并不是那二十块元石的奖励,而是维持威慑力。

若一开始就避战,只会让人看出他的虚弱。但若是一直战下去,恐怕就露馅了。

“都愣着干什么,还有谁没上擂台的,继续上。第一名可是二十块元石的奖励,你们不想要了?”拳脚教头楞了半晌,这才大吼出声。

剩下的学员们顿时踊跃起来。

方源已经下了场,他们的心中就像搬开了一块无形的巨石。

“我上!”

“我来!”

两个少年争先恐后,跳入场地,砰砰砰地交上了手。

“唉,早知道如此,我也等等,不用急着上场,也不至于被方源轰下台来。”

“真是可惜,没想到方源下场了。”

“他真敢啊,你们看连教头都拿他没办法。”

……

教头听到身后的窃窃私语,顿时感到自己的威信在不断下降。他心中烦躁得很,真想狠狠地惩治方源一番。但方源根本就没有犯错,主动下场是被允许的行为。

教头郁闷又无奈,最终他看向方源,狠狠地瞪了后者一眼。

方源的嘴角微微翘起一个角度,心想:“如此粗浅手段,这个教头还真有些可爱。”

(ps:三江票每天下午14点,就可以领。本书每天14点第二更,一般大家看完第二更,就可以领票投了。三江票只有一天的有效期,到了明天14点,就得再领了。)

\end{this_body}

