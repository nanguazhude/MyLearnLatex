\newsection{不过是些许风霜罢了}    %第九十节:不过是些许风霜罢了

\begin{this_body}

%1
厅堂中,灯火明亮。

%2
圆形的饭桌上,酒已冷,菜已凉。

%3
红艳艳的烛火跳动着,将舅父舅母的影子照在墙壁上。

%4
两人的影子连成一片,随着烛光,阴沉地晃动着。

%5
在他们的面前,跪着沈嬷嬷。

%6
舅父打破沉寂:“不想这方源,真的是死心踏地地要和我作对啊。唉,原本是想好言相劝,想稳住他,让他先住在家中,再找理由将他逐出家门。没想到那小崽子居然不上当!他是铁了心,居然一口拒绝我的邀请,一点商量的余地都没有!连我这个门槛都不跨进一步!”

%7
舅母咬牙,神色有些惊慌失措:“这个养不熟的小狼崽子,如今已经十六岁了。要是他有要求家产,我们不能不给啊。当年我们得到的家产,笔笔都在内务堂都明确登记着,也不能赖账。这可怎么办?!”

%8
“你先下去罢。”舅父挥退沈嬷嬷后,冷笑起来,“你也不要着急。这一年来,我早就在思索谋划了。首先要分家,就必须得有一转中阶的修为。这点方源早已经达到了,并且已经是巅峰修为,甚至还夺得了年末第一,真是叫人刮目相看。嘿嘿……”

%9
“但是,想要成功分得家产,绝非如此容易!一转中阶的修为,不过只是一个前提条件罢了。要分到家产,方源还要申请,内务堂审批下来,就会发布任务,考察他方源有无资格。这也是家族为了防止胡乱分割了家产,导致内耗过重,家族实力减弱的政策。”

%10
舅母恍然大悟:“这么说来,方源要完成这个任务,才能得到他双亲的遗产。”

%11
“不错。”舅父阴笑道,“但是内务堂的任务,都是针对小组发布的。这个家产任务,也不例外。方源要完成它,就必须依靠小组的力量,单单靠他自己是不行的。家族如此做,也是千方百计地整合小组,促使组员们团结在一起,提高凝聚力。”

%12
舅母哈哈大笑:“老爷,您真是太英明了。让角三将方源收入组内,这样一来,方源要完成家产任务,就得需要他们的力量。但是角三是我们的人,方源一个人根本就完成不了这个任务的。”

%13
舅父眼中闪过一抹得意的光,道:“哼,就算是招不到他进组,我也有其他法子对付他。别说完成任务了,就算是他想申请分家,接到这个家产任务,也未必有可能!”

%14
……

%15
夜幕降临,雪停了下来。

%16
方源走在街道上,沿途的竹楼都覆盖了一层白色霜雪。

%17
脚踩在雪面上,发出轻微的声响。清冷的空气呼吸入体,让方源的脑海清醒无比。

%18
在拒绝了沈嬷嬷后,方源不顾角三等人的挽留,辞别了众人,独自行动。

%19
“原来如此。”他边走边思索着,“舅父舅母是想卡着我,拖延我,让我丧失夺回遗产的机会。”

%20
“过了年后,我就是十六岁,有资格分家。双亲已死,弟弟已经重新认了新父母,只要成功,遗产就都是我的。但是要拿回这遗产,这过程得分两步,每一步都大有关隘。”

%21
“第一步,是在本身没有任务的前提下,到内务堂进行申请。第二步,是完成内务堂发布下来的家产任务,这样才有资格获得家产。”

%22
“角三和舅父舅母都是一丘之貉,且不说第二步,单就第一步他就要卡住我。”

%23
家族规定,蛊师只能一次完成一件任务。这是防止蛊师滥接任务,造成家族内部的恶性竞争。

%24
角三连续接收任务,刚刚完成了采集腐泥冻土的任务后,就立即接了一个捕捉野鹿的新任务。

%25
家族的任务都是针对整个小组发布的,也就是说,按照族中规定,方源必须完成了这个捕捉野鹿的任务之后,才有权申请分家。

%26
“但到了那时,相信角三一定会再接任务吧。他身为组长,在交接任务上,总会比我快一步。我差一步,才能申请到家产任务,但是他就是卡在我的前面。”想到这里,方源眼中清冽的目光一闪。

%27
这些阴谋诡计,真是烦人,像是一个无形的绳索栓住方源前进的脚步。

%28
不过方源并不后悔进入这组。

%29
当时演武场的局面,让他陷入两难之中。角三的邀请,反倒为方源解围。

%30
若不进这组,相信舅父舅母定有其他手段,防不胜防。如今身在组中,反而看清了他们的布局,就可从容反击。

%31
“要解决这个麻烦,也不是没有办法。最简单也是最直接的办法,就是将角三直接铲除,将舅父舅母直接暗杀,家产方面就没人和我争了。不过这个办法,风险太大。他们都是二转蛊师,我一转的修为还是有些低。而且就算是杀了他们,也无法善后。除非有一个绝好的机会,可以趁势而为……但是这种机会,往往可遇而不可求啊。”

%32
方源可以杀死家奴高碗,可以处死王老汉一家。那是因为他们都是凡人,奴仆,性命卑贱如草,杀死他们就相当于杀死一条狗,折断一根草,无所谓。

%33
但是要暗杀蛊师,那麻烦就大了。

%34
蛊师都是姓古月,都是族人。不管死了哪个,都会引来家族刑堂的彻查。方源评估了一下自己的实力,现在杀人风险太大,说不定就被反杀。就算是杀掉了,刑堂的调查将是个更大的麻烦。自己今后行动受到监视不说,甚至还可能被调查出花酒行者的遗藏。

%35
“为了铲除一个小麻烦,而引来一个麻烦百倍的大麻烦,这实非智者所为。嗯?到了。”方源口中轻声自语,停在一栋破旧的竹楼前。

%36
这栋竹楼已经破烂不堪,就像是行将就木的老人,在寒冷的冬风中弯着腰,苟延残喘。

%37
看着这栋竹楼,方源的脸色不由地浮现出一丝追忆之色。

%38
这就是他前世租的房子。

%39
前世他被舅父舅母逐出家门时,手头的元石不足十五块。在睡了几天的大街之后,他找到了这里。

%40
这里的房间太破旧了,因此租金比其他地方,要低得多。而且其他地方都是按月结算,按季度结算,惟独这里是按天数结算。

%41
“我不知道其他的地方,是否有舅父舅母的布置。但前世的记忆告诉我,至少这里没有。”方源扣响了门扉。

%42
半个小时,他定了租房协议,被房东带到二楼的一间房里。

%43
老旧的地板,踩在脚下,发出令人担忧的声音。

%44
房间中设施简单,只有一张床,一张棉被。棉被上打了许多补丁,仍旧还有破洞,露出里面泛黄的棉絮。

%45
床头有一盏油灯,房东点亮之后,就离开了。

%46
方源并没有睡下,而是盘坐在床榻上,开始修行。

%47
空窍中元海潮起潮落,波涛生灭。每一滴元水都是墨绿。

%48
空窍四壁都是坚固的白色晶膜,半透明状。

%49
正是一转巅峰的景象。

%50
忽然间,青铜元海开始掀起巨大的波澜。就好像是一头头野兽,陡然间狂暴了,向着四周的窍壁发起自杀式的冲击。

%51
轰轰轰……

%52
巨浪掀起,狠狠地冲撞在窍壁上,飞溅的浪花碎成点点绿色的晶莹,然后彻底消散。

%53
短短片刻功夫,四成四的真元海面迅速降低,大量的真元剧烈消耗。

%54
厚实坚固的晶膜上,也出现了一道道裂纹。

%55
但仅仅是裂纹,这还远远不够。

%56
方源要突破一转巅峰,晋升到二转,就必须将这晶膜彻底冲垮,破而后立!

%57
墨绿真元不断冲击晶膜,晶膜上的裂纹也越来越多,渐渐地连成了一整片。有的地方,裂纹加深,形成了更加明显的裂痕。

%58
但是随着元海彻底消耗,没有真元继续冲击之后,晶膜上的这些裂痕开始愈合,裂纹开始消失。

%59
方源并不意外,收回心神,睁开双眼。

%60
油灯已经灭了,本来灯油就不多。

%61
房间中一片黑暗,只有窗户那边的缝隙,透进来一些微弱的雪光。

%62
房间里没有火炉,并不温暖。方源盘坐在床上,久未活动,已经感到了身上的寒意在渐渐加重。

%63
他的黑眸与黑暗融为一体。

%64
“其实要突破角三的封锁,还有一个比杀人更简单更安全的方法。那就是晋升二转!一转蛊师没有放弃任务的权利,二转蛊师却每年都有一次。若我晋升二转,直接放弃身上的任务,就可以申请分家。”

%65
“但是突破二转,也不是那么容易的事情。”想到这里,方源幽幽一叹。从床上下来,在狭小的房间内慢慢踱步。

%66
从初阶晋升中阶,从中阶晋升高阶,这是晋升小境界。从一转巅峰晋升二转初阶,这是突破大境界。两者之间,难度不可同日而语。

%67
简单来讲,要冲破晶膜,就得需要爆发力,在短时间内形成足够强的冲击力量,将晶膜冲碎。

%68
但是方源只有丙等资质,元海中的真元只有四成四的量。若是爆发全力,冲击晶膜,不到片刻之后,真元就消耗殆尽了。

%69
就像刚刚那样,真元消耗光之后,再没有余力继续冲击。而晶膜具有自我恢复的能力,过不了多久,就能痊愈。方源之前做的努力,就成了无用功。

%70
“要冲破晶膜,达到二转,不算特殊情况,一般至少需要五成五的墨绿真元。而我资质有限,最多只有四成四。所以人们常说,资质是蛊师修行第二重点!”

%71
想到这里,方源慢慢地停下脚步。

%72
不知不觉间,他已经走到窗边,于是信手推开了窗户。

%73
隔牖风惊竹,开窗雪满山。

%74
月光下,雪如白玉,铺得眼前世界如水晶宫殿,纤尘不染。

%75
雪光映照在方源年轻的脸上,他面色沉静,双眉舒展,一对眸子好似月下幽泉。

%76
寒风扑面而来,少年忽的一笑:“不过是些许风霜罢了。”

%77
(ps:怒了,明明定时了八点就能发布,但就是不起作用啊。昨天也是这样,已经连续三次了。有谁知道的告诉我一声哈!)

\end{this_body}

