\newsection{猴儿酒,酒虫机缘不相让}    %第四十四节:猴儿酒,酒虫机缘不相让

\begin{this_body}

第二天中午,趁着吃午饭的功夫,方源又来到了山寨外的商铺区。

因为白天需要工作的缘故,此时帐篷地摊间的山民,并不太多。

方源按照记忆,走到昨晚贩卖知心草的地方。却见一辆空空的板车,停在原地。拖着板车的,是一只驼鸡。

它骄傲地立足原地,体型大如鸵鸟,外形如鸡,但是背部隆起,形成一个弧度。

一对宽大的翅膀收拢在身侧,七彩的羽毛鲜艳绚烂。

鸡头高高地昂起,大红的鸡冠如玛瑙王冠,在阳光的照耀下,闪着莹润的光。

“看来终究是晚了一步,知心草已经被售卖一空了。可惜,若是买上几斤的知心草,能节省不少的元石。”方源脚步顿了一顿,便离开这里,继续朝里面走。

“来啊,都来品尝一下各个山寨的美酒吧。这里的美酒超过一百种,有上好的灯草酒,有后劲绵绵的九曲酒,清幽淡雅的古龙井,酸甜的花石大曲,入口生津的百泉老窖,酒香浓郁的醉三秋……”在一个蓝色的圆桶状的帐篷前,伙计在卖力地吆喝着。

方源目光一闪,立即起了兴趣。方向一转,就进了这家酒铺。

酒铺里的设施很有特色。

在帐篷最里面,是一个长长的柜台。一位蛊师镇守着,他的背后有数十只栲栳大小的水晶瓢虫,攀附在帐篷的布壁上。

帐篷的地面,并没有铺上地毯,而是直接裸露出山石泥土。在土中生长出一群色彩绚烂的蘑菇。

这些蘑菇五颜六色,圆润润的,有些可爱。有的大如桌面,有的小如矮凳。常常是一个桌面大的蘑菇周围,围绕生长着几只矮凳般的小蘑菇。

“这是天真蘑菇,被蛊师有意催生出来。具有吸收粉尘浊气,净化空气的作用,是草蛊的一种。”方源一眼就认出了这些蘑菇的来历。

他选择一个小蘑菇坐下。蘑菇的表面立即微微凹陷下去,让方源有一种在地球上,坐沙发的错觉。

“这位公子,这是酒单。您看看需要什么酒?”一位伙计走了过来。

方源看了一下酒单,发现这里的酒,比青竹酒都还要贵一些。

“就来一杯猴儿酒。”方源放下酒单道。

“一杯猴儿酒!”伙计转头,高声叫道。

柜台处,那位一转的蛊师听见了,便立即弯腰,从柜台里取出一个竹筒酒杯。

然后他拿着酒杯,转过身,面对身后的帐篷。帐篷的蓝色帆布上,有数十只硕大的水晶瓢虫,头朝下,尾朝上,静静地攀附着。就好像是帐篷布壁上的挂饰。

这些水晶瓢虫也是一种蛊,肚内中空,常常被蛊师用来装载珍贵液水。

它们通体透明,仿佛是由水晶做的。从外面就可以看到瓢虫圆滚滚的肚子里,装着何种酒液。

蛊师很快就从中,寻找到装有猴儿酒的那只水晶瓢虫。

他将竹筒酒杯放在瓢虫的口器之下,然后另一只手轻轻抚摸瓢虫的水晶背甲。

一丝极少量的真元涌入到水晶瓢虫体内,随后,瓢虫就张开口,一股酒水就顺流下来,倒入竹筒酒杯当中。

酒水哗哗响了一阵,倏地停住。

蛊师将装满了猴儿酒的竹筒酒杯放在了柜台上,柜台外的伙计早就等候多时,连忙小心翼翼地捧起来,快走几步,端给了方源。

方源只轻轻地抿了一口,猴儿酒类属果酒,酒香清醇香甜,入口绵柔。

他没有再喝,而是心念一动,唤出了酒虫。

白白胖胖的酒虫,化成一道白光,在空中划出半个弧线,扑通一声,投入到酒杯当中。

酒水四溅,洒在蘑菇桌面上。

酒虫则在酒杯中欢快地打滚,猴儿酒以肉眼可见的速度,迅速下降。几个呼吸的功夫,酒杯中就干涸见底,涓滴不剩了。

“是酒虫!”柜台那边的蛊师惊呼一声,双眼骤亮。他是一转蛊师,资质只有丁等,跟随商队,在这家酒肆中做工。就是想一边游历,一边寻找机缘。

酒虫能精炼真元,提升一个小境界。对一转蛊师来讲,可以说是极其珍贵的蛊虫。这不正是他苦苦寻求的机缘吗?

“这位公子,不知道您的酒虫能不能割爱?”他激动地走过来,一脸诚恳地问道。

方源摇摇头,态度坚决地拒绝了他,随后起身就走。

他来此处的目的,就是主动暴露酒虫,从没想过要贩卖它。

“公子,公子,请留步。我是很有诚意的,也许我们可以坐下来慢慢商谈。”蛊师不舍地跟随方源来到帐篷入口,但是方源没有任何的回应。

最终他只能定定地站在原地,万分遗憾地看着方源的背影转入拐角,消失在视野当中。

……

不知不觉间,太阳渐渐落下,月牙飞旋而上。

夜色中,月光如水洒下,却被无数商铺的彻亮灯火给蒸发出去。

今天晚上的商铺,人群汹涌生意依旧火爆。方源被人群夹裹着前进,各种议论声不可避免地传入他的耳中。

“商铺一般会开设三天三夜。今夜已经是第二个晚上,到了后天早上,商队就会重新启程了。所以想要买点东西,我们就得赶紧了。”

“昨天看上了一只金钟蛊,唉,太贵了。和掌柜的砍了半天价格,都没有便宜多少。今晚再去看看。”

“你们听说了吗,就在昨晚。有一个少年,居然在赌石场开出了一只癞土蛤蟆,足足赚了五百块元石!”

……

方源细心地听着,心中不免有些失望,他没听到任何关于酒虫的消息。

“酒虫只是一转蛊虫,对于一转蛊师意义重大,但是对于二转三转的蛊师,就不能精炼真元了。因此没有人关注,也属正常情况。不过主动暴露酒虫这事,还不能急于一时,太过做作的话,反而会露馅。”方源一边走,一边在心中暗暗思忖。

就在这时,前方人群一阵急速涌动。

紧接着,方源便听见有人高喊:“快来看啊,居然有黑心商铺,向我们族人贩卖假蛊!”

人群中顿时有人义愤填膺。

“嗯?居然有这样的事情发生。”

“快去看看,是哪家商铺敢蒙骗我们的族人!”

方源随着人流,也跟了过去。

只见一群人围在一座大红色的帐篷入口处,里三层,外三层地拥堵着。有的在好奇观望,有的抱臂冷看,但大多数人的脸上都笼罩着一层薄薄的怒气。

帐篷的入口处站着两个人。

一个青年二转蛊师,看其装束,很明显就是古月族人。

另一人还是熟人,正是赌石场的掌柜老板贾金生。

青年蛊师手中捏着一只黑黝黝的蛊虫,高高地举起,对着周围人们高喊:“族人们,就是我面前的这个人,昨天卖给我一只假蛊。哄骗我说是黒豕蛊,卖了我两百五十块元石。没想到买回去一炼化,结果当晚就发现这哪里是什么黑豕蛊,明明是最普通的臭屁肥虫!”

贾金生冷笑连连:“你不要血口喷人。我什么时候说这蛊是黒豕蛊的?你有什么证据?”

青年蛊师见贾金生矢口否认,顿时大怒,一手抓住贾金生的胳膊:“你这奸商,还敢抵赖!居然敢在青茅山上,蒙骗你家古月大爷,是想找死么?!”

“你放手!”贾金生也是怒了,一甩袖子,将青年蛊师的手挥打出去,“你想要闹事,敲诈钱财,也不分清楚对象。我可不怕你!我哥哥就是贾富,四转蛊师,你能奈我何?”

“你!”青年蛊师怒目圆瞪,却不敢再动手了。四转蛊师的名头已经吓住了他。

“啊呸!”贾金生往地上吐了一口唾沫,他昂起头,望着青年蛊师,发出一声不屑的嗤笑,“是你自己想要贪小便宜,你也不动脑筋想想,黒豕蛊能从根本上增加蛊师的力量。是多么珍稀的一转蛊虫,市价比酒虫还要贵,通常都要卖到六百块元石。你以为两百五十块,就能买到黒豕蛊?做梦去吧你!”

“混蛋……”青年蛊师将一口钢牙咬得嘎吱作响,满脸赤红,浑身被气得颤抖,胸中充满了被羞辱的愤怒之火。

人群嗡嗡作响,躁动不安,议论纷纷。却没有一个人敢出头,贾富四转蛊师的名头就像是一座无形的大山,镇住了场面。

“这小子太可恶了,真是个奸商!”

“难怪敢在青茅山这样嚣张啊,原来是贾富的弟弟。”

“听说是同父异母,本身只有一转修为,但是仗着这层关系在商队里作威作福惯了。”

……

“这就究竟发生了什么事情?”就在这时,一个洪亮的声音响起来。

“是贾富来了!”

“头领来处理纠纷了,大家让让。”

议论声顿时一滞,人群连忙分散开来,形成一条狭窄的过道。

一个身材矮壮,盯着肥肥的大肚子的中年蛊师,顺着过道,走了进来。他穿着一声长袖黄袍,正是这支商队的头领贾富。

“贾富大人,有礼了。”青年蛊师纵然有满腔的怒火,也不敢发作,硬生生地忍耐住,主动向贾富抱拳行礼。

贾金生却僵立在原地,他似乎没有料到哥哥的到来,脸色骤然苍白,眼中闪过一丝惊怒。

这个微小的神情变化,让一直在远处暗暗观察的方源看在眼中,顿时若有所思起来。

\end{this_body}

