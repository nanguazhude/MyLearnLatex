\newsection{尽收囊中}    %第十六节:尽收囊中

\begin{this_body}

“你动一下试试?”

“你已经中了我的独门毒蛊,没有我的对应解蛊,七天之后,必定化为脓血而死。”

“和花酒大人相比,晚辈算个屁!晚辈是昏了头,竟然有眼不识泰山,冒犯了花酒大人。花酒大人,请您看在先前我族倾情招待的份上,饶了一命啊!”

墙壁上,画面又开始重复回放第二遍。

方源默然无声,直到画面又开始重复第三遍时,他这才幽幽一叹,道一声:“原来如此。”

这将影像声音留在石壁上的手段,应该是花酒行者布下的留影存声蛊。此蛊能刻印影像,并且投射出来。

留影存声蛊靠吸食光芒和声音为生,这个山壁秘洞不知为何,总散发着红光,同时石缝联通着外界,也不会隔绝外界的声音响动。方源在此处,耳边尽是小瀑布的轰鸣声。

因此留影存声蛊便在这个山壁秘洞中,生存了下来。

能方才方源拨开枯藤,应该是惊动了藏在石壁中的留影存声蛊。

只要脑袋不笨,稍稍推测,就知道这段影像应该是真的。

当年,四代族长暗算花酒行者不成,战败后又偷袭,虽然击退了后者,但是他也因此身亡。这段历史很不光彩,活下来的几位家老,便遮盖篡改了真相。

他们把四代族长和花酒行者的角色,颠倒了一下。

花酒行者成了战败偷袭,被当场击毙的魔头。四代则成了光明磊落的英雄人物。

但这故事,本身有个大漏洞。

那就是花酒行者明明已经被当场击毙,尸骨也应该掌控在古月一族手中,但是为什么还发现了另一具骸骨呢?

前世,那个发现此处的蛊师,想必看到这段影像后,就觉察到了恐惧。

那几位活下来的家老早已经作古,但是为了防止花酒行者去而复返,这个真相应该在家族高层秘密流传着。

那蛊师发现,自己若是独吞了遗藏,就有巨大的风险。日后若被人察觉他和花酒行者有牵扯,家族高层自然要清洗他。

所以,取舍一番之后,他不敢隐瞒这处遗藏,做出了禀告高层的决定。

这样的表现,更证明了他对家族的忠诚。他后来的境遇,也表明了他做了一次明智的选择。

不过他这么做,并不代表方源会这么做。

“好不容易探索得来的遗藏,就应该一个人独吞,凭什么要分给其他人?就算被发现又如何?不冒风险,哪来收益,那蛊师真是胆小。”方源冷酷一笑,不再管石壁上继续重复着的影像,而是转身伸手,用力将枯藤死根彻底拉开。

花酒行者的尸骸,也被殃及,原本完整得很,此时却被成了破碎的数段。

方源毫不在意,将脚边碍事的一根腿骨踢开,重新蹲下,寻找遗藏。

首先,他发现了一袋元石,打开一看,只有十五块。

“穷鬼。”方源吐槽了一句,花酒行者外表光鲜,没想到积蓄这么少。

不过他很快想到了原因,花酒行者激战之后,又中了月影蛊,定然会利用元石疗伤。能剩下十五块,已经算得上不错了。

然后,方源又发现了几只死蛊的残骸。大多是花草之流的蛊,都已经彻底枯萎了。

蛊也是生灵,也是需要食物豢养的,而且大多都很挑剔。草蛊花蛊对食物的要求虽然少,但是这处秘洞却连一丝阳光都没有。

再然后……

再然后,就什么都没有了。

花酒行者先是和同级别的四代族长,进行了一场激战,而后又和近十位家老对战。本身的蛊虫就已经消耗了很多,到了这里后,他想要疗伤,因此催发生长了酒囊花蛊,以及饭袋草蛊。但是最终却被月影蛊拖累死。

经过三百年的光阴,他身上仅剩下的蛊也都死了。

唯一剩下的,就只有石壁中的留影存声蛊,以及酒虫。

这只酒虫,应该是靠着酒囊花蛊,艰难地生存了下来。但是随着酒囊花蛊一根根的枯死,它也丧失了食物来源。

这促使它不得不外出,寻找野生的酒囊花。

然后在这一晚,它被青竹酒浓郁的酒香吸引,来到了方源的面前。

“留影存声蛊只能记录一次,属于消耗型的蛊虫。看来酒虫才是此行最大的收获,难怪那蛊师要禀告家族。看来是因为利益太小,不值得冒这么大的风险。”方源心中升起一股明悟。

记忆中,那蛊师已经是三转,而酒虫不过是一转蛊虫罢了。对于方源来讲,比较珍贵,对于那蛊师而言,却可有可无了。

不过很显然,因为蛊师的通报,家族也给了他不小的奖赏。

“我是不是也应该禀告家族?”方源想了想,就掐断了这个念头。

花酒行者的遗藏似乎只有酒虫和元石,其实不然。

真正有价值的,是这处藏了留影存声蛊的山壁。

或者说是山壁上不断重复播放的影像。

这影像完全可以卖给其他山寨。相信这种打击家族信念的确凿证据,青茅山上其他两家山寨的高层一定十分感兴趣。开出的价格,必然比家族的奖赏更多。

什么?

你说家族的荣誉感和归属感?

还真是抱歉,方源可一点都没有呢。

况且这段影像,又不是什么抄家灭族的强大武力,造成不了多大的实质伤害。

而冷漠的家族也不会重视方源,他需要自己努力,开拓修行资源,在修行前期更需要四方借力。

“指望家族?呵呵。”方源心中冷笑,“怎么能再像前世那般天真呢。”

别指望任何人,这世间的一切都得靠自己。

确认这秘洞已经搜刮干净之后,方源便按照原路返回。

顶着水压,挤出巨石,他又回到山外。回头望望这巨石,方源忽然又想到前世记忆中,是说在地下秘洞中发现的尸骨。但这哪里是地下?分明是山壁内部。

难怪自己废了这么大周折,连续七天都找不到。

看来前世家族发现了这处之后,定然在第一时间摧毁了影壁,又发出真假掺杂的消息,误导族人。

今晚能够发现这里,一部分是运气,一部分是积累,最大的因素恐怕还是青竹酒。

这酒香真是浓郁,可以说是青茅山之最。

也许前世,那蛊师失恋之后,喝的就是这个酒也说不定。

不过这些都已经不重要了,花酒行者的遗藏被方源挖掘殆尽,虽然结果有些差强人意,但是也在情理之中。最重要的是,方源最初的目标(酒虫)已经到手,并且他最需要的东西(元石)也有了。

“接下来,就一门心思窝在客栈炼化蛊虫。只要有了本命蛊,就可以回归学堂,有资格住在学员宿舍,并借助家族资源修行。这客栈住一两次也就罢了,住太久,花费太大。”方源思量着,脚步不停,赶往山寨。

他本来剩余两块元石,又新得十五块,统共十七块。但是对于一名蛊师来讲,这点元石算得了什么?

------------

\end{this_body}

