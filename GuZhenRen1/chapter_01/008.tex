\newsection{物是人非}    %第八节:物是人非

\begin{this_body}

学堂旁设立了一个蛊室。蛊室并不大,只有六十平方米。

蛊师修行,蛊虫就是实力的关键。

课程一结束,兴奋的少年们就向蛊室蜂拥而来。

“排队,一个个的进来。”怒喝声骤起,蛊室门外自然有人把守。

少年们一个个的进去,又出来。

轮到方源走进蛊室。

只见这房间内大有乾坤,四壁都做成隔洞,这些内嵌的方格子一个挨着一个。格子有大有小,大的超不过砂锅,小的小不过拳头。

密密麻麻的格子里,摆放着各式各样的器皿,有的是灰色的石盆,有的是青翠的玉盘,有的是精致的草笼,有的是陶制的暖炉。

这些器皿中就存养着各种各样的蛊虫。

有些蛊虫静默无声,有些蛊虫却吵闹得很,发出吱吱吱、咯咯咯、窸窣窸窣等等各种各样的声音,汇集成一支生命的交响曲。

“蛊虫也分九大层次,照应蛊师九转境界。这些蛊虫都是一转蛊虫。”方源扫视一圈,顿时心知肚明。

一般来讲,一转境界的蛊师,只能用一转层次的蛊虫。若是越级催动高等蛊虫,蛊师往往要付出极其惨重的代价。

而且蛊虫也需要喂养,喂养高等蛊虫所耗费的代价,往往也不是低等蛊师能承受得起的。

对于新人蛊师来讲,不是特殊情况,都会选择一转蛊虫进行首次炼化。

而蛊师炼化的第一只蛊虫,意义重大,称之为本命蛊,性命交修。一旦灭亡,蛊师必定遭受重创。

“唉,原本指望能得到花酒行者的酒虫,炼化它成为我的本命蛊。但是现在,寻找花酒行者尸骨仍旧未见端倪,也不知道什么时候能找到,或者被别人发现。保险起见,还是选一只月光蛊吧。”

方源一边心中暗叹,一边径直走向左手墙侧。

在这面墙洞位置稍稍靠上的其中一层,是一排的白银盘子。每个盘子上都摆放着一只蛊虫。

这蛊虫晶莹剔透,弯弯如月,好像是一块蓝水晶,在白银底盘的映衬下,显现出一股清幽之气。

蛊名月光,是古月一族的镇族蛊虫,绝大多数的族人都选择它成为本命蛊。它并非是天然蛊虫,而是经过古月一族的秘法培育而来,其他地方没有,可以说是古月一族的标志象征。

都是一转的月光蛊,差别极其细微。方源随意挑选了一只,拿在手上。

月光蛊很轻,堪比一张薄纸的重量。只占据掌心一块,如寻常玉坠大小。方源放在手中,能透过它,看到自己被遮掩的掌纹。

最后看了一眼,发现没有什么问题了,方源将其放在口袋中,就出了蛊室。

蛊室外还排着长长的队伍,随后的一个少年见方源走了出来,连忙兴冲冲地跑进蛊室去。

若换做其他人,得了蛊虫,第一反应就是拿回家赶紧炼化。但是方源却没有急着这么做,他的心中还惦记着那只酒虫呢。

酒虫更珍贵,月光蛊虽然是古月山寨的特产,但是却没有酒虫对蛊师的帮助大。

离开了蛊室,方源直接去了酒肆。

“掌柜的,来两坛陈酒。”方源掏掏口袋,将仅剩的元石碎块,放在柜台上。

这些天来,他都会来此买酒,然后再去山寨周边晃荡搜索,意图吸引那只酒虫现身。

掌柜的是个矮矮的中年胖子,满脸油光,经过这些天已经记住了方源。

“客官,你来啦。”打招呼的同时,他伸出又粗又短的胖手,娴熟地将方源的元石碎块抹走。

又放在手中颠了颠,觉得份量不差,掌柜脸上的笑容又亲切几分。

元石是这个世界的货币,用于衡量一切商品的价值。同时它是天地精华的凝练物,自身也能被使用,帮助蛊师更好的修行。

既有货币属性,又有商品属性,极为类似地球上的黄金。地球上有过单一金本位制,这个世界上就是单一元石本位制。

对比一下黄金,因此元石的购买力也相当的惊人。

不过,再多的元石也经不起方源这样连续的消耗。

“每天两坛酒,已经整整七天了。原先积攒的元石,差不多都花完了。”拎着两坛酒出了酒店,方源眉头微微皱着。

一旦成为蛊师,就可以从元石中直接抽取出纯净真元,来补充空窍中的元海。

因此对于蛊师来讲,元石不仅是货币,更是修行的帮手。

有了充足的元石,修行的速度能提升不少,这或多或少能弥补一些资质上的短板。

“明天就没有元石购买酒水了,酒虫却迟迟不现身。难道真的要我把月光蛊炼化成本命蛊?”方源心中有些不甘心。

出了酒肆,方源手中提着两坛酒,一边走路一边思量:“学堂家老说,此次考核第一个炼化本命蛊的人,就有二十块元石的奖励。现在恐怕许多人,都在家卯足劲,炼化蛊虫,争取第一吧。可惜,炼化本命蛊极为考验资质。资质好的人,优势极大。以我丙等资质,有没有其他手段,根本就没有得胜的希望。”

就在这时,身后响起古月方正的声音:“哥哥,你果真又来酒馆买醉!跟我来,舅父舅母要见你。”

方源停下脚步,回身望去。

发现弟弟再没有像从前那般低着头说话。

兄弟俩视线相撞。

一阵风呼啸地吹来,拂起哥哥散乱的黑发,吹起弟弟的衣摆。

短短的一个月,却已物是人非了。

一周前的开窍大典,不管是对于哥哥还是弟弟,都是巨大的改变。

哥哥方源从云端跌下,天才的光环被人无情地剥夺。而弟弟则开始绽放光芒,如一颗新星,冉冉升起。

这种改变对于弟弟古月方正来讲,更是有着一种天翻地覆的意味。

他终于品尝到了哥哥当初的感受,被人寄托着希望,被人用羡慕或者嫉妒的目光看着。

他感觉自己就好像忽然从幽暗的角落里,置身到了充满光的天堂。

每一天醒来,他都感觉自己仿佛在做着一个美梦。天差地别的待遇,让他至今都有些难以置信,同时还有着强烈的不适感。

不适应。

一下子从默默无闻,到被人密切关注,指指点点。

有时候方正走在路上,听到身边路人议论自己、赞叹自己的声音,都会感到脸上发烫,手足无措,眼神躲闪,差点都连路不知道怎么走了!

最初的十几天下来,古月方正莫名其妙的瘦了一圈,不过精气神却越加旺盛。

从他的内心最深处,开始滋生出一种叫做“自信”的东西。

“这就是哥哥以前的感觉啊,真是美妙而又痛苦!”他不可避免地想到自己的哥哥古月方源,面对这样的议论和关注,哥哥他以前是怎么应对的呢?

他下意识地开始模仿方源,装作面无表情,但很快发现自己不是那块料。

有时候在学堂,一声女孩的叫喊,就能让他闹出个大红脸。在路上,大妈大婶的调戏,更让他多次落荒而逃。

他像是一个婴儿学步,跌跌撞撞地适应着新的生活。

在这个过程中,他不可避免地听到有关哥哥的传闻——消沉颓废,变得酗酒,夜不归宿,学堂大睡。

他起先十分震惊,自己的哥哥,那么强大那么天才的存在,竟然变成了这样子?!

但是渐渐的,他开始有点理解了。哥哥也是常人啊,遭遇到这样的挫折和打击,消沉也是必然的。

伴随着这种理解,方正隐隐地感到一阵难以言表的痛快。

这种痛快的情绪,是他极为不想承认的,但的确存在着。

被称赞为天才的哥哥,曾经如阴影般镇压自己的哥哥,如今如此落魄颓丧。这从反面,更见证了自己的成长,不是吗?

自己是优秀的,这才是真相啊!

因此,看到方源拎着酒坛,头发散乱,衣衫不整的模样,古月方正心中狠狠地舒了一口气,呼吸莫名地轻松了许多。

但他嘴上却又说着:“哥哥,你不能再喝酒了,不能再这样下去了。你不知道关心你的人会多么的担心,你要振作起来!”

方源面无表情,没有开口。

兄弟俩四目相对。

弟弟古月方正的眼中闪闪发亮,透出一股锐利之意。而哥哥方源的双眸,却黑的深沉,如幽幽之古潭。

这样的眸子,让方正不由地感到一种莫名其妙的压抑。对视没有多久,他下意识地转移了视线,望向另一侧。

但当他反应过来时,心中瞬间升腾起一股愤怒。

一股对自己的愤怒。

自己这是怎么了?连和哥哥对视的勇气都没有么?

我已经变了,我已经彻底改变了!

这样想着,眼神就锐利起来,重新扫射过去。

但是方源却已不看他,而是一手拎着一坛酒,走过他的身边,平淡的声音传来:“还愣着做什么,走吧。”

方正呼吸一乱,心底积蓄起来的那口气,再没有了发泄的地方,这让他感到一种难以表达的郁闷。

眼看哥哥已经走远,他只好快步跟上去。

只是这一次,他的头不再低着,而是昂起面朝着夕阳。

他的目光则注视着自己的脚,正一步步踩在哥哥方源的影子上。

------------

\end{this_body}

