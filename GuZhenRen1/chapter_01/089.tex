\newsection{一条病蛇盘踞在脚边}    %第八十九节:一条病蛇盘踞在脚边

\begin{this_body}

五人站在街道旁。

古月角三温和地笑着,对方源道:“方源老弟啊,你的表现真是让我们大家刮目相看。看来,我们能邀请你加入,真是做对了。你刚刚从学堂出来,有些东西不太了解。下面我就跟你介绍一下。”

“首先家族任务,每个蛊师至少每个月完成一个。完成的越多,完成的越完美,评价就越高。”

“第二,任务由内务堂或者外事堂发布,蛊师每次只能选择一个,一旦接受就必须完成。”

“第三,在特殊情况下,蛊师可以选择放弃这个任务。但是评价会因此大大降低。同时,一转蛊师没有放弃的权利,只有二转蛊师才有这个资格。但是每年也只能放弃一件任务。”

“第四,家族评价非常重要。关系着你在家族中的前途。评价越高,你的前途就越广大,越光明。”

方源听着,点点头。这些东西他早就滚瓜烂熟了。这个古月角三说得没有错,也没有误导,但是却有很多其他的关键内容,没有一齐说出来。

“好了,腐泥冻土的采集任务,我们刚刚完成了。接下来,我接了一个捕捉野鹿的任务。危险性很低,很适合方源你练手呢。”角三又对方源道。

方源心头冷笑,嘴上则道:“谢谢组长的关照了。”

古月空井立即接道:“方源老弟,你是该好好感谢一下组长。简单的任务,虽然容易完成,但是报酬很低的。组长这么做,都是为了照顾你啊。”

另外的两个女蛊师也跟着附和:“是呀是呀,方源小老弟,口头上的感谢不如实际行动哦。不如就请组长吃个饭吧。”

“方源小弟,你刚走出学堂,社会上的事呀,很复杂。人际关系是相当重要的。你有什么不懂的,尽管在吃饭的时候问我们好了。”

古月角三哈哈一笑,故意摆手道:“你们不要为难方源了。他刚刚走出学堂,手头上元石不多,很拮据呢。”

“怎么可能!别的不说,单单是年末考核的第一,就有一百块元石的奖励吧?”一位女蛊师故意叫了声,然后露出微微嫉妒的神色,“说起来还真是羡慕啊,一百块元石对于我们来讲也算是一笔大款子呢。”

“组长,这你可就看错我们方源老弟了。他怎么可能会这么小气。方源老弟,你说对吧?”古月空井朗声笑着,似乎很开怀。

另一位女蛊师则靠近方源,一副亲近熟稔的样子,道:“方源小弟,好心告诉你哦。我们几个你可以不巴结,但是组长大人,你必须巴结啊。你知道么,我们组员的评价有一部分是组长评写的。他说优秀就是优秀,说差等就是差等。”

“不错,不错。评价对于我们蛊师来讲非常重要,评价上优秀,就会被家族高层看中。将来资历熬上去了,能少奋斗许多年呢!”空井立即附和道。

整个过程,古月角三都笑眯眯地看着。

最后他摆手,脸色慈善,笑容和蔼:“哎,你们可以不要乱讲哦。我的评写都是公正客观的,可不会被你们轻易贿赂的。不过,方源你放心,你是新人,既然加入了我们组,你的评价我一定会酌情关照的。”

若是平常少年,恐怕早就已经被这伙人说得五迷三道,暗暗感动了。欺负的就是新人不懂规矩,天真幼稚。

但是方源的双眸一直清冽平静。

组长参与组员的评价,这的确是事实。这就是家族的体制,将权力下放,赋予组长这个位置的权威性,方便组长管理每个小组。

但是绝不会像这些人说得那么严重。

家族的评价,组长的评论只是很小的一部分,其余的大部分还是家族内务堂的事情。

退一万步讲,就算是组长对组员的评价影响甚大,又能怎样?

方源一点都不把这评价放在心中。

他早就打定主意,尽快提升修为,达到三转,离开这里。他根本就没有想过,在这家族中蹉跎百年,为家族卖命一生。

在这些鼠目寸光的蛊师眼中,评价是重中之重。但是对方源,所谓的评价屁都不是!

所以这些蛊师隐隐胁迫的话语,都方源来讲,根本毫无效力。

“你们说的,我都知道了。我还有点事情,先走了。”方源沉默了一下,说道。

呃?

一时间,这四个蛊师都露出错愕的神色。

“这什么意思?我们刚刚说的话,你都没有听进去吗?”一位女蛊师瞪大了眼睛,在心中咆哮。

“你的理解力真的没有问题吗?”古月空井在这一刻,真的想抓住方源的衣领,当面诘问。

角三的嘴角忍不住抽搐了一下。

方源的无动于衷,让他们的一唱一和,看起来像是一场闹剧。让他们的布置谋划,变成了一个笑话。

看着方源转身就走的背影,他心中的恼火就别提了。

“哎呀,方源,不要急着走嘛,可别听这群家伙胡说八道的。”角三强行咽下这口闷气,堆起笑容,快步走上去,拦住方源,“你这是要去租房子吧,我们陪你一块去。我也算是有点人际关系,租房的情况我也比较熟悉的。”

“的确是要租房,学堂宿舍已经住不了。不知组长有什么见教?”方源微微扬起眉头,面色平淡从容。

“我知道几套房子,比较便宜,而且位置也好。”角三热情地笑起来,当仁不让地在前头领路。

……

“我这房子,一个月十五块元石的租金,绝不二价。”

“呸,八块元石就想租我的房?做梦去吧。”

“必须要交押金,押一付三。别家都是这么干的。”

“我这房子,风水很好的,晚上又僻静。关键是靠着家主阁,你也知道这地段有多好了。价格真的不贵,每个月只要二十五块元石。”

一直到天色灰暗,方源都没有决定下来租哪套屋子。

“方源小弟,你这也太抠了。山寨就这么大,市价都是这个样子的。”

“要我说啊,就租那套临近家主阁的房子,二楼很宽敞,风景不错。方源小弟,你也不要太节俭了,才二十五块元石罢了。你手头上那个第一奖励,就足够你住上整整四个月的了。”

两个女蛊师不怀好意地劝说着。

方源摇头:“我手中的元石并不多,不能这么浪费。”

“那就租地下室吧,那个比较便宜。”古月空井眼中闪过一丝寒芒,建议道。

方源心中冷哼一声,这个空井心思恶毒!

这山间本来湿气就重,如今又是冬天,住在地下室中空气不流通,极易感染风寒。就算是没有染病,住久了,关节也会生毛病。

见方源没有答话,角三开口道:“其实那个大堂的一楼,也不错。和其他人合租的话,租金也会便宜一些的。”

方源摇头:“我想一个住,不喜欢合租。”

古月空井哼了一声,不悦地道:“这又不想,那又不想的。不是我说你呀,方源老弟。你也太挑了,眼界太高,标准也高,不符合实际啊。”

方源听了,心中不由的冷笑:“就算是真有符合标准的,但凡是你们推荐的房子,我都绝不会采纳的。”

他这番虚以委蛇,不过是谨慎起见,想看清可能存在的陷阱罢了。

当即,方源正要告别这些人,沈嬷嬷出现了。

“方源少爷,可找到你了。学堂宿舍不能住了,老爷和太太已经备了酒席,特意嘱托我来请你回家呢。在家里多好,你何必在外面租房子呢。”她劝说道。

终于来了么……

方源心中连连冷笑,这个沈嬷嬷找得可真准啊,一下子就找到了这里,必有人通风报信无疑!

这事果真不出自己所料,幕后的黑手就是舅父舅母!!

而这个角三,就是舅父舅母钳制自己的枷锁。

方源的沉默,看起来就成了犹豫。

“方源啊,要不然你就住在舅父舅母家算了。何必这么费事呢?租屋子可得花费不少的元石呢。”古月角三适时地“关怀”道,他望着方源的侧脸,眼中闪着一道隐晦的阴芒。

\end{this_body}

