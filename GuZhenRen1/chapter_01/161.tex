\newsection{心甘情愿被剥削}    %第一百六十一节:心甘情愿被剥削

\begin{this_body}

方源放下酒杯,坐下。

众人这才敢坐下。这些人,并非是全部同窗。譬如漠北、赤城等有深厚背景的,皆不再此列。

“时间差不多了,我也该走了。这酒宴办的不错。”方源表明了去意。

古月定宗被这一夸,顿时心花怒放,连忙站起来,从怀中掏出一个钱袋子。

袋子里自然装的满满的元石。

他弯腰谄笑道:“今日聆听大人一番教诲,实在叫小人茅塞顿开,大有所获。区区薄礼,聊表下人的感激之情。”

他满嘴胡说八道,从酒宴开始,他一直溜须拍马,哪里讨教了什么修行话题。

但众人却仿佛这事情真的发生是的,大声起哄着,鼓动方源家老收下。

方源也不推却,微微一笑,自然而然地将这钱袋接过手中。

紧接着,就是第二位,第三位一个个上来送礼,皆是元石!

“好说,好说。”方源眯着眼脸上微笑,一个个都收了。

数十袋的元石,方源哪里拿得过来,古月定宗瞧见,立即贴心地唤来几位家奴,替方源在身后捧着。

短短功夫,方源就收了近万块的元石!

最后,方源施施然站起来身,他再次举起酒杯:“相逢即是有缘,这份同窗之情,你我铭记在心,值得饮上一杯。”

“是!”

“方源大人说得好极了。”

“语言精辟,妙到毫巅。正说出了我们心中所想啊!”

……

众人纷纷站起,一个个赞叹着,亦举起酒杯。

他们或是没有背景,或是背景并不深厚,方源晋升家老,都害怕方源的报复,同时也想搭上方源这条线。

方源浅浅笑着。微微抬手,举起酒杯。

此时,天外阴云消散。露出月光如纱,照盖外面的庭院里。清冷的空气中夹杂着血气,真实得残酷。

而这厅堂中。布置贵雅,灯火辉煌,酒色财气充溢,各个脸上浮着虚夸的笑容。似乎是温暖的人间天堂。

“这就是组织体制的魅力了。”方源的眼眸清光闪烁,盯着杯中醇厚的酒液,心思浮泛开来。

以前,他抢掠勒索同窗,不过是区区几块元石,就引来众怒。

现在,他根本提都不提一声。这些人就眼巴巴地,争先恐后地送来元石。一袋子都是上百块!

前后区别,表面上似乎是方源有了家老的身份。

但其实是因为,先前他游离于体制之外,如今他出于组织高层。

在体制之下。成员们都是心甘如怡地被剥削。甚至不需要方源暗示些什么,就会主动有人来贿赂,主动有人脉来投靠依附,有女色投怀送抱。

这世界如此,地球上更是一样。

“这世人几多可笑。被偷被盗白抢,稍稍有丁点损失。就反抗激烈,大呼不平。向上层贿赂,送礼送身体送贞操,却都心甘情愿。还唯恐做不到位!我今夜能收获这么多的元石,无非是借助体制之力罢了。”

方源心中冷笑,不由地想起古月青书、漠颜、赤山等人。

像古月青书这样的才俊,拥有乙等资质,在修行天赋上比方源其实要高多了!

但他们这些人,却修行缓慢,很长一段时间拖在二转境界。

这是他们不努力吗?

呵呵。

冷笑两声。

这就是体制的剥削和压迫。

然而这种剥削和压迫,却往往是隐形的。常人万万难以看穿!

就拿方源眼前来讲,这些人送来的元石,若用于他们自身,对自身的修为绝对有推动力。

所以只要有贿赂,它就是一种剥削!

无数下层争先恐后地对高层贿赂,就是高层的集资,更增加高层的权威。

除了钱财,还有时间上的剥削。

类似古月青书这等精英,自然不用太贿赂他人,但他们的时间却被占用。平日里叫你做这做那,叫你跑腿,叫你奔波,还美名其曰——这是高层的重视和青睐!

如果将这时间用来修行,古月青书早就突破二转巅峰,达到三转。再利用木魅蛊,说不得就能杀了白凝冰!

微妙处就在于,家老们内心并不愿意古月青书这样的后辈,这么快就晋升三转。

这样得力的棋子,真要成了三转,和他们平起平坐,还怎么使唤?

谁愿意自身的权利被瓜分?

所以要有意识地拖着压着,还美名其曰——此子我很看好,但需要打磨,才能成玉啊……

呵呵。

“这就是体制中的真相。若是看不穿这点,任其多么英雄豪杰,多么天赋才情,都不过是被枷锁套着的龙虎,只是奴隶罢了。类似古月青书,古月赤钟等等这些人,哪怕再有才智能干,又如何呢?”

虽是想到这么多,但思绪如电,外界不过一恍惚之间。

“请诸位满饮此杯。”方源将酒杯移到唇边,然后一仰脖子,饮尽。

众人亦忙饮了,不敢剩下半滴。

“告辞。”方源抱拳一礼,迈步就走。身后家奴各捧着元石,亦步亦趋。

众人连忙相送。

“你们喝,不必送我。”方源虽是这样说着,但众人却不敢,纷纷离座,恭维马屁此起彼伏。

方源又说:“我这人喜欢清静。”

众人看其神色,这才作罢,停留在厅堂当中。

看着方源的背影渐行渐远,有人唏嘘,有人沉默,亦有人叹息一声:“方源家老真是奇人,潇洒啊……”

他们都是井底之蛙,雾中看月。只是觉得方源潇洒,还看不透体制这层。

其实只要加入体制,就会被剥削,利益就会被牺牲。

哪怕是族长,也在牺牲,为管理家族奉献大量的时间和精力。

只是底层成员,被剥削的情况更加严重。越到高层。享受的利益就越大。

方源起初之时,抢元石搞对立,特立独行。连亲弟弟都不放过。就是为了避开这层剥削,因此有了充足的时间和精力,冲刺到了三转。成了家老,这结果让无数人惊异连连。

如今他则摇身一变,成了家老,温文尔雅,位高权重,享受家老的福利,叫人艳羡无比。

这游离和加入,一出一进之间,充满了深沉的智慧。

但又有多少人,能看得清呢?

方源少了被剥削。却享用了利益,这放在凡人的眼中,就是潇洒了。

……

“好了,把东西放在桌上就可以走了。”方源关照道。

几位家奴不敢有丝毫意见,沉默地放下后。向方源躬身而退。

这住处已经不再是方源当初租的屋子。

自从方源晋升家老之后,家族就给他拨调了一栋崭新的竹楼。

竹楼中专门有书房,有修行闭关之用的密室。但家奴没有,需要方源自己寻找。

“兜率花,出来。”

方源心中念头一动,白银色的真元灌注。寄居在舌苔上的兜率花的印记,顿时就鲜活起来。

他张口一吐,只见红光一现,兜率花如灯笼般缓缓旋转着,悬浮在半空中,出现在他的面前。

方源催动兜率花,顿时红芒飞涨,将周围映照得一片红霞泛滥。

一块块的元石,被这股红芒覆射照住,受到无形中的牵引,纷纷飞离袋子,投入到兜率花中。

片刻之后,红芒消褪。方源微微张口,这兜率花就重新投入他的嘴中,落在舌苔上,化为一道红色的花灯印记。

“这兜率花是三转蛊,能存元石,也能藏其他东西。属于存储类三转蛊虫中的佼佼者,收藏的元石最多能有三万块。但考虑到也要储藏其他东西,那么最多能存一万五千的元石。”

方源虽然是第一次运用这蛊,但依靠前世丰富的经验,很快就推算出它的极限。

元石是蛊师修行最基础的资源,没有之一。

少了元石,蛊师就严重缺乏了推动力。

而且元石能够快速地恢复真元,对于战斗来讲,帮助也极大。

尤其是孤身在外的蛊师,元石是行走野外最基本的保障。一般而言,元石最少得有一万,才能保证蛊师在一段时间内的基本供给。并且每隔一段时间,都需要及时的补充。

一万五千块元石的储备,对于方源来讲,这数量有些少了。但处于仍旧可以接受的程度。

“先前向赤练借了三千块元石,加上今天这笔收获,短时间内不愁元石了。如今六大方面,攻防有血月天蓬,有雷翼蛊辅助移动,存储有兜率花,地听肉耳草用来侦察。唯独缺少了治疗。”方源暗暗盘算。

之前有一株九叶生机草,但被方源上缴了。因此才有了兜率花。

不过这二转的九叶生机草,就算是留在手中,治疗能力方源也是不太满意的。

“三转的治疗蛊虫中,有几种比较理想。如生生不息蛊,能持续治疗,所耗真元也少。在这点上,最适合我这种资质不好的蛊师使用。还有不死草,能存一线生机,吊住一口气,属于上佳的保命之蛊。最理想的则是自力更生蛊。这蛊虫奇特,依附于蛊师自身的力量。只要蛊师力气越大,它就能促进刺激蛊师的新陈代谢,因此来治愈伤势。”

但这三种蛊虫,方源哪里能寻?

古月一族方面,连地下虫洞都被他探查过了,没有发现。

物质榜上,更不会外流出这等珍稀蛊虫。

真正有些希望的,还是花酒行者的遗藏。

但这种可能性也是微乎其微,方源几乎就不抱希望的。花酒行者的遗藏,他已经感觉到快要结束了。怎么可能恰好最后,就有方源需要的蛊?

如果真是这样,那真是太理想太完美了。

但方源知道,这世事多残酷残缺。真寄予了期望,那才是天真!

“不过即便如此,也该将这遗藏一探究竟了。至少那只锯齿金蜈,应该收服到手里。”方源心中有了思量。(未完待续。如果您喜欢这部作品,欢迎您来投推荐票、月票,您的支持,就是我最大的动力。)

\end{this_body}

