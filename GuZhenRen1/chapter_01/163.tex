\newsection{雷冠头狼}    %第一百六十三节:雷冠头狼

\begin{this_body}

方源看着面前的这株天元宝莲,一直以来,心中积压的疑云顿时消散了大半。推算出了历史的大概。

时光回溯,将近千年之前。

一位五转蛊师强者,孑然一身,来到青茅山上,偶然间发现了地下溶洞中的这道天然元泉。

他大喜,于此安营扎寨,将山脚的凡人村子合并迁徙,形成了古月山寨的雏形。

他大肆娶妻,妻妾上百人,播撒血脉。

他就是古月一族的先祖,古月山寨的创建者。

一代逝世之后,时光流逝,二代、三代,到了四代。

四代族长拥有甲等天资,亦是修行到五转,将家族带上另一个鼎盛时期。

有一天,山寨外来了一位魔道蛊师。

他是光头大汉,穿着一身粉衣,独来独往,最喜欢坏良家女子贞洁。便是当时,魔道中赫赫有名的魔头――花酒行者。

这花酒行者不知得了什么机缘,知晓了能合炼出天元宝莲的秘方。又做了许多准备,就差一口天然元泉,就能炼出这花蛊。

他左挑右选,最终选中了古月山寨中的这口元泉。

起初,他先以交易月兰花做幌子,刻意接近古月高层,终于打探出古月一族的虚实。

然后,在和四代族长的大战中,他以强大战力,几乎得到了完胜。不仅杀了四代,更铲除了大部分的家老,只是身中月影蛊。

月影蛊只是限制真元使用的蛊虫。并不足以致命。但花酒行者顾及着合炼天元宝莲,并不想大肆屠杀,引来不必要的关注,妨碍了合炼。因此选择暗中行事。

最后,他偷偷用千里地狼蛛,挖开地道,秘密潜行到这里。利用手中早已准备好的充足材料。在元泉中成功合炼出天元宝莲。

这天元宝莲来头甚大,其合炼的秘方,是由一位数千年前的正道蛊师――元莲仙尊所创。

天元宝莲本身。只是三转花蛊。但是往后晋升,成为六转的天元宝皇莲之后,位列十大仙蛊排行第六。价值和春秋蝉不相上下!

天元宝莲,号称为移动元泉,能为蛊师产元石。

但合炼它的代价极为高昂。

合炼出天元宝莲,须得利用一道天然元泉。这泉还得元力饱满,不能是那种使用了多年,底蕴不足的元泉。

合炼成功之后,这道元泉就彻底废掉,丧失了生产元石的能力,成为一口最普通不过的泉水。

一道天然元泉的价值,多么巨大。只看它养活了古月一族。无数蛊师将近千年,就可明了。

合炼天元宝莲,就要废掉这么一口元泉。但这才是刚刚开始,往后晋升,到四转。须废七口元泉。到五转,再废九口。到六转,再废十一口!

除此之外,还有其他珍贵蛊虫充当辅料,每一只均价值连城。

“我若有这么一株天元宝莲,带在身上。就等若有一道微型元泉。天元宝莲只是三转,日产元石并不能和正常的元泉相比,但足以支撑我的修行消耗!”

这其中好处多多。

有了天元宝莲,就直接产出元石,收入绝对比九叶生机草要多得多。

有了元石,不仅能推进修为,而且有了充足兑换的钱财。

方源有这宝莲,就能减少元石的携带。兜率花中,完全可以只存储食物等等,给他的后勤减轻极大的负担。

“不过……我听说,合炼这宝莲过程繁杂玄妙,宝莲由无到有,期间似虚似实。常人肉眼根本就观察不到,只有透过水晶,才能瞧得清楚。这蛊娇贵,须得温养在元泉当中,长达九天九夜。直到这宝莲长出九片完整的莲叶,方可采摘,纳入空窍元海。若操之过急,就将功亏一篑。先前苦功,尽皆化为苦水全数流淌。”

方源并不知晓天元宝莲的详细秘方,只是晓得一些秘辛传闻。对他来讲,如何往后合炼,他根本就不清楚。

但即便如此,这三转的天元宝莲,仍旧对他有极大的帮助。

此时他透着水晶墙壁看过去,细细观察,却发现这花骨朵儿,竟然只有八片半的莲叶,有一片残叶,只余一半,连九片的完整状态还差一些。

方源并不奇怪。

时间已经过去数百年了,元泉的底蕴经过消耗,早已经大不如四代时期。

这天元宝莲等若是凝聚了元泉的大多数精华,元泉产出的元石,不断地被消耗,因此底蕴越来越少。所以宝莲渐渐地反过来,却补充元泉的损耗。自身有了损失,这才有了残叶。

“天元宝莲必须是九片莲叶,才可摘取。如今只有八片半,我要采摘,就得往这泉中投入元石!”

元石是元泉的结晶,能恩泽滋润宝莲,使之再生长。

但别看这只是半片残叶,要想生长出来,恐怕得要投入大量的元石!

“如果我所料不差,这水晶墙应该是通堑蛊的作用……”方源试着用手摸了摸,发现这墙似实还虚,仿佛是一片光影。自己手探入墙体里去,竟然毫无阻碍。

但他很快就收了手,不敢真的深入元泉之中。

元泉最忌污染。

他灌注真元,催动兜率花,取出一块元石。

他将元石往水晶墙壁一投,墙壁仿佛不复存在,元石穿透墙壁,很快就没入到泉水当中。撞上天元宝莲的虚影。

天元宝莲蓝白相间的花骨朵儿,顿时一阵水波般晃动。

几乎是一瞬间,这块元石就被天元宝莲消化。

待花影恢复平静后,方源凝神一看,却不见那残叶有任何变化。

他面色沉静,又接连投入数十块元石,但残叶仍旧不见生长。

方源继续投入元石,同时心中默数元石之数,到了五百多块的时候,才见这残叶稍稍生长了一丝。

看到这一幕,方源心中不禁微微一沉。

按照这样的程度推算,他至少得要一次性投进五万余块的元石。

若是分批投入,时间间隔久了,上面家族不断取出元石,天元宝莲又会消耗自己,来补充这口元泉。

“五万多块元石……我手中已有一万余,还差四万的缺口。”

凭借家老的身份,方源要筹措出这四万的元石,倒也并非困难之事。

但真正的问题在于,一旦摘取了这株天元宝莲的话,这元泉就彻底废掉了。到时候,势必将引来整个家族的震怒和疯狂的追查。

方源经验丰富,但手段有限。真要不顾后果的追查,总会被查到蛛丝马迹。事实上,家族高层早已经暗暗怀疑上他,只是碍于狼潮,暂时压下没有发作。

花酒行者的遗藏一旦暴露出去,方源必将是第一个怀疑对象。

就算是方源偷偷潜逃,也会遭到整个家族的倾力追杀。

“天元宝莲我是不会放弃的,就算是没有合炼秘方,但未来的事情谁说得准呢?只是一旦摘取了这蛊,就是捅了马蜂窝,招来杀身之祸。”

方源暗自思量,要收取这天元宝莲,还得等待成熟的良机出现。

“这天元宝莲,应该就是花酒行者的最后一道遗产了。只是这整件事情,还有许多疑点。花酒行者为了合炼天元宝莲,来到此处。但之后又遭到什么变故,让他最终重伤,在临死之前匆匆设下这道传承呢?”

花酒行者设下传承的目的,方源已经知晓了。

就是为了报复古月一族。

天元宝莲若是从元泉中取出来,不管成功与否,这口天然元泉必定是废掉了。

没有了天然元泉,古月一族就没有了驻扎在此地的根基。分崩离析是迟早的事情。

“算了,没有新证据,纠结这处疑点,也不会有所进展。还是先回去山寨罢。”最终,方源摇摇头,顺着原路开始返回。

但当他还未出得石缝秘洞,一道凄厉嘹亮的狼嚎声,就连绵不断地传来。

“这声音!”方源心头一沉,连忙快步走出去。

走到石缝外的河滩上,就有浓郁的血腥气味扑鼻。

虽然距离山寨还有一段距离,但是喊杀声、狼嚎声、爆炸声交织在一起,嘈杂无比。

方源隐去身形,登上一个山坡。

此时凌晨,天空蒙蒙亮。

无数的狼群,如潮水一般,一波波地冲向古月山寨。

方源目光一扫,身躯一震。

在狼群的大后方,他发现了一头体型如小山般巨大的电狼。

万兽王――雷冠头狼!

这狼身躯修长,肢体矫健,浑身长着幽蓝色的鳞片。一撮撮的金黄色狼毛,只分布在狼爪和头尾上。

尤其是头部的狼毛,都炸起来,形成高耸的王冠之状。

它半蹲在地上,沉静得仿佛一尊雕像。四周不断狂吠的狂电狼和豪电狼,反而衬托出它的优雅和高贵。

它只是坐在地上,但已经带给古月一族极大的心理压力。

“终于迎来了万兽王,古月山寨到了生死存亡的关键时刻了!”方源望向山寨,无数的蛊师正在激烈战斗,竭力抵挡疯狂的狼潮冲击。

忽然近十位身影,从寨墙飞跃而出,顶着狼潮,逆流而上,向雷冠头狼杀去。

这些人俱都是家老,其中族长古月博一马当先!

(ps:临时有事,今天就这一更了。)。。)

------------

\end{this_body}

