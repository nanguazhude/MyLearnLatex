\newsection{时间长河唯一条,春秋蝉做舟来渡}    %第一百八十七节:时间长河唯一条,春秋蝉做舟来渡

\begin{this_body}

%1
水幕天华蛊,只有四转,一经使用,就化为一面巨大的圆球水罩。水罩防御卓越,甚至可挡五转蛊的攻伐。但它有个巨大的弊端,那就是无法移动,无法停止。水罩内外隔绝,就算是水幕天华蛊的主人,也不能自由出入。很多势力,都采用水幕天华蛊用来护城。

%2
方源对春秋蝉并不熟悉。

%3
在前世,他刚刚成功合炼了春秋蝉,就被正道围杀,根本没有时间试验把玩。那时候,春秋蝉也不是他的本命蛊,这点特性也没有显露。

%4
到了今生,他修为低微,更不敢乱放春秋蝉出来。

%5
春秋蝉隶属天下奇蛊之一,本来就极其神秘,根本没有使用者的任何心得、经验,流传在世上。

%6
况且,蛊虫一达到六转及以上级数,都是天下唯一,仅有一只。旁人若要合炼出来,只有前一只死了,才能有成功的可能性。否则百分百失败。

%7
这一来二去,便导致方源到如今,才知道这个特性。

%8
“重生非易事,春秋藏灾厄啊。春秋蝉的恢复速度越来越快,就算是甲等修为,拥有海量资源,恐怕修为增长速度也跟不上它的速度。蛊师的空窍,早晚有一天要被春秋蝉撑破!”

%9
方源暗暗咬牙。

%10
重生虽然美好,但拥有春秋蝉,就相当于半个十绝体。天生就被架在绞刑架下,时间一到,就要行刑!

%11
“不能外放春秋蝉,难道只能催动它,再次重生么?”方源深深皱眉。

%12
这似乎是唯一的办法了。再次重生的话,春秋蝉将重新陷入虚弱状态,同时方源也能变相地脱离这里的险境。

%13
但以上情况,看似理想,其实大有问题,风险巨大。

%14
首先,不能确保重生次次成功。

%15
方源重生过一次,也时常回味唯一的宝贵经验。

%16
他结合地球上的观念进行理解——世界是三维立体的空间。时间就是一个轴,贯穿古往今来。没有时间,空间是静止的。一切事物的运动,都需要过程,即意味着消耗时间。

%17
世界唯一。不存在平行空间。动用春秋蝉重生。就是从时间轴的后半段上一点(即未来),跨越到前半段某一点(即过去)。

%18
但方源在“未来”的老迈身躯,在“过去”那个时间是不存在的。

%19
受到天地大道的约束,身躯不能带到过去。只能自爆。自爆产生的能量是动力,而春秋蝉蕴藏着时间法则的碎片,如同孤舟。载着方源的意识,从“将来”重生到“过去”。

%20
意识并非如人的身躯那般,不是纯粹的物质。但严格来讲。这股“未来”的意识,在“过去”那个时间点上,也是不存在的。

%21
但事情巧妙之处,就在于此!

%22
“未来”的意识,会引导蛊师改变自己,进而影响周围,然后影响渐渐扩展到整个世界。这就是蝴蝶效应。

%23
当蝴蝶效应产生,世界就和原来不同,“未来”的意识。也就有了存在的意义,得到大道天地的承认。

%24
有人说过,历史如同一道长河,在上游改变一个事件的结果,下游就会变得面目全非。

%25
这个神奇的蛊之世界。就如同一道长河中的水。绝大多数的人,只能从上游到下游,顺流而行。而方源的意识,却依靠春秋蝉。从下游逆流到上游。

%26
当他在上游做出改变的时候,下游的河水也发生了变化。但河水仍旧是那道河水。蛊师世界仍旧是蛊师世界。只是历史转了一个拐角,变成了另外一种可能的结果。

%27
这样形象比喻,就可以容易理解。

%28
然而,春秋蝉还未恢复完全,只能相当于一艘有漏洞的破船。

%29
方源的修为只有三转初阶,自爆后的动力,和前世六转有难以想象的巨大差距,根本推动不了春秋蝉,逆流时间长河过多的距离。

%30
“我若自爆重生,未必能成功。说不定破船就搁浅在长河当中,让我的意识记忆,都被无情的时间冲刷,化为乌有。若要提高成功几率,那么最好是等到空窍极限。尽量拖延时间,让春秋蝉恢复更好,破船上的漏洞更少。同时自身修为增加,自爆之后,也能提供更多的动力进行逆流。”

%31
想到这里,方源长长地叹了一口气。

%32
春秋蝉这事变化,有些出乎了他的意料。但他向来谨慎,早留了一手准备。

%33
身后血蝠群飞来,方源却将心神沉入空窍。

%34
空窍中充斥黄绿光芒,春秋蝉气势磅礴,白银真元海如镜子般平静,四周窍壁光膜却呈现出不支的危险丝丝裂纹。

%35
其他所有的蛊,都被春秋蝉的气息,镇压到了海底。

%36
方源意念调动之下,一只蛊顶住压力,缓缓飞出海面。

%37
这蛊形如骰子,正正方方,通体灰白,坚硬无比。

%38
正是当初,方源利用强取蛊,强取了白凝冰空窍中的一些蛊虫,将其夺到了手中。

%39
这蛊是消耗蛊,用一次就消失。但作用非凡,一旦用了,就将蛊师空窍中的底蕴和潜力完全榨干净,令蛊师的修为,在瞬间达到同转的巅峰。

%40
“石窍蛊,爆。”

%41
方源心念萌动,石窍蛊顿时炸开,形成一股灰白石粉,如烟似雾,瞬间弥漫在真元之海上。

%42
空窍四壁原本是光膜,但被这灰白石粉一沾染,光芒顿时黯淡下去。石粉附着在光膜的表面,光膜渐渐增厚,从光质变成石质。

%43
几秒之后,方源的空窍四壁皆增厚数倍,化为沉重厚实的石窍。

%44
春秋蝉的黄绿光辉,仍旧变幻不定,但它的气息却是暂时可以承受了。

%45
方源原本只是三转初阶,乃是淡银真元,真元如水,银光淡淡。但这一刻,他修为暴涨,从初阶一举跃升到三转巅峰,拥有雪银真元!

%46
“用了石窍蛊,等若是断了前进之路,空窍中潜力都被榨干。很难再跨入四转了。不过,修为增加,空窍转为石壁,比先前坚实厚重了数倍,倒可以暂时支撑住春秋蝉的压力!白凝冰也是想要利用石窍蛊。来应对北冥冰魄体的大限吧。可惜十绝体比春秋蝉还要麻烦。潜力近乎无穷无尽,就算是一时转化为石窍,但很快窍壁就会复原了。”

%47
这时,刀翅血蝠群已经扑杀过来。

%48
方源冷哼一声。抽出锯齿金蜈,边杀边退。

%49
幸好这山洞空间狭窄,方源仗着天蓬蛊的防御,以及锯齿金蜈宽大的身躯,尽力堵住血蝠群。令它们不能包围自己。

%50
这就令它们的威胁大降。

%51
一时间,山洞中砰砰作响。

%52
锯齿金蜈的拍击,以及刀翅血蝠撞在白芒虚甲上,速度过快扎进山洞墙壁的声音,连成一片。

%53
方源空窍中的真元,急剧消耗。

%54
血蝠群虽然有近百只,但没有蛊师在临场操纵,反而内耗很大,形成不了默契的配合。实际情况是。方源只需要同时对付三十四只。

%55
但这数量,也不是他能抵挡,只能边战边退。

%56
比较尴尬的是,他虽然跃升到三转巅峰,但空窍中还存储着初阶的淡银真元。单单依靠丙等资质的空窍。自我产生的雪银真元速度实在太过于缓慢。现在的情形,也不允许方源取出元石在手上,分心去汲取天然真元。

%57
消耗元石,快速回复真元。这个方法不能应用于实战之中。

%58
在生死较量中分心分神,这是自取其辱。极其愚蠢,等若自己找死。同时,汲取天然真元的效率也大大降低。

%59
这个法子,只能用在平时修行,或者是利用脱离战斗的短暂间隙,来快速回复真元。

%60
万幸的是,方源在不久前,获得了一株草蛊——天元宝莲。

%61
天元宝莲,能产出元石,十分珍稀。但事实上,这个效用只是它本质能力的一种表达。

%62
天元宝莲,号称“移动元泉”,本质上的能力是可以产生天然真元。这些真元,凝聚浓缩在一起之后,就形成元石。

%63
方源拥有一株天元宝莲,沉在空窍真元海内,就如同空窍中,出现了一道极微型的元泉!

%64
天元宝莲的秘方,是元莲仙尊所创。

%65
只有达到九转层次的蛊师,才被天下共尊,若在正道,则称为仙尊。若在魔道,则称为魔尊。

%66
数千年的元莲仙尊,号称古往今来,真元回复第一人。在这个方面,凌驾于其他仙尊、魔尊,究其原因,正是因为元莲之功!

%67
方源拥有的天元宝莲,还只是三转级数,刚刚炼得,最为低等。但已经能为方源,源源不断地提供天然真元。

%68
这天然真元一出现在空窍中,就被方源的空窍自动炼化,成为雪银真元。

%69
方源倘若汲取元石中的真元,还得分出一部分的心神。但这天元宝莲本身就是他的蛊,他催动天元宝莲,就像是舞动手指那般简单轻易!

%70
方源且战且退,有了天元宝莲的帮助,他在真元回复方面,已经能和乙等资质的蛊师媲美。

%71
“杀!”

%72
他忽然爆喝一声,突然改变战斗风格,一头冲入血蝠群中。

%73
锯齿金蜈暴动起来,银边锯齿嚓嚓狂转,砍中一只龟缩在后方的刀翅血蝠。

%74
这血蝠,比寻常同类,要精壮一些。乃是这只血蝠群中唯一的雄蝠,被古月一代炼化。操纵它,就能间接地操控这支血蝠群。

%75
这一击方源暗暗蓄势,用心观察良久,以无心算有心。古月一代也不在现场,立即就收到了奇效。

%76
雄蝠当场被劈杀,绞成一堆血泥。

%77
剩余的雌蝠,顿时轰然崩散!

\end{this_body}

