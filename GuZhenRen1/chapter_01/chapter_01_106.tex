\newsection{半生积累成泡影,唯有本命长久存}    %第一百零六节:半生积累成泡影,唯有本命长久存

\begin{this_body}

%1
一天之后。

%2
床榻上,方源盘坐着。

%3
一团脸庞大的白光,在距离他脸面一臂的半空中,静静地悬浮着。

%4
方源伸手从钱袋中取出一块块的元石,不断地丢进光团里去。

%5
白光慢慢收缩,却越来越耀眼。

%6
当它收缩到拳头大小的时候,白光变得刺眼,令方源只能眯起双眼观察。

%7
“应该是最后一块了。”方源手中捏着元石,心知到了关键时刻。

%8
他将元石投入光团当中。

%9
隐约可见,元石漂浮在光团中,好像是冰雪进入了沸水里,不断地消融。

%10
大量的石粉簌簌地往下落。

%11
元石全部消失,光团猛地一爆!

%12
砰的一声轻响,三只蛊虫分别向三个方向弹射出去,一只掉在床上,其余两只则撞上墙壁,然后掉落都了地板上。

%13
合炼月芒蛊失败了!

%14
方源心中一沉,他赶忙勾勾手指,召回蛊虫。

%15
月光蛊和一只小光蛊都摇摇晃晃地漂浮起来,缓缓地飞到他的手中。但是剩下的那只小光蛊,却没有响应。

%16
它静静地躺在地板上,乳白色的五角星形状的身躯,开始渐渐地消散在空气中。

%17
几个呼吸的功夫,它彻底消散了,再没有任何的印记留下来。

%18
这就是合炼失败的代价——根据合炼秘方的不同,蛊虫会因此受伤,运气差的时候。蛊虫会直接死亡。

%19
哪怕是方源有着丰富经验,同时一心己用,又有正确的秘方,也有失败的几率。

%20
方源也不气馁,这种事情他见多了。自己已经做到了最好,结果失败了,那只能归结于运气。

%21
“不过幸好。不是月光蛊消亡,小光蛊死掉一只,还可以去商铺购买。容易补充。月光蛊消亡的话,那就不能轻易补充了。”他现在颇有钱财,损失掉一只小光蛊。重新购买就是了。

%22
接着,方源查看了一下月光蛊和剩下的那只小光蛊。两只蛊的表面,都有些黯淡无光。这是炼化失败,蛊虫本身受了损伤的表现。

%23
“蛊虫一旦受伤,合炼成功的概率就要下降。还是等到这两只蛊虫恢复了,再继续合炼吧。”方源深知欲速则不达的道理,将这两只蛊收了起来。

%24
他估算了一下时间,至少三天后,才能再次进行合炼。

%25
修行并没有结束。

%26
方源摊开右手。

%27
他的左手白皙,在手掌的部位。有一棵草的绿色印记,仿佛是一个墨绿色的纹身。

%28
方源心念一动,空窍中的真元就调动而出,如同淡红色的气雾,顺着手臂涌入到印记当中。

%29
绿色的印记顿时鲜活起来。从方源的手掌心上冒出草尖,然后是九片碧绿的圆形叶子,最后是翠玉般透明的草茎。至于参须似的草根,则没有暴露出来。

%30
手掌上,原先墨绿色的纹身印记,已经消失。只留下一缕缕的墨绿纹路。代表着草根,和方源的掌纹纵横交错。

%31
正是二转草蛊——九叶生机草。

%32
此刻,方源的手掌就像是一块土地,而一株九叶生机草就生长在上面,仿佛是玉石雕刻的精致工艺品。

%33
方源伸出右手手指,将一片片的草叶采摘下来。

%34
每摘下一片圆形的草叶,方源就感觉自己微微的疼痛一下,类似于一根头发被拔掉的感觉。

%35
九片叶子摘下来后,都被方源随手放在床榻上,而手掌中的九叶生机草则只剩下一根光秃秃的草茎。

%36
方源继续催动真元,淡红色的二转真元,不断从手掌心上升腾起来,宛若烟雾一般,包裹着翠绿草茎。

%37
草茎不断地吸收这些真元,渐渐地一片嫩芽,从草茎的底层处冒出了头。

%38
这片嫩芽是粉绿色的,细腻而又小巧,十分脆弱,一捏就碎。

%39
方源继续催动真元,嫩芽便渐渐生长变大,颜色也变得越来越深。最终它长成了一片深碧色的,完全成熟的叶片。

%40
“消耗了两成真元。”方源检视了一下自己的空窍,得出结论。

%41
他只有四成四的真元海,也就意味着,他只能一口气催生出两片生机草叶。

%42
又催生出一片后,方源拿捏住一块元石,快速地回复空窍中的真元。

%43
当真元海面上涨到四成之后,他又接着催生生机草叶。

%44
如此循环往复,半天的功夫,他重新让九叶生机草上长满了九片草叶。

%45
他不再采摘这上面的草叶,念头一动,九叶生机草重新缩回左手掌心,化为了绿色印记。

%46
他将采摘下来的九片生机叶,用小袋子装好,贴身放着。

%47
一片生机叶,就是一转蛊虫,每一片的市价能卖到五十元石。也就是说,单单这九片叶蛊,就能让方源收获四百五十块元石。

%48
当然,方源催生它们也有成本。但即便抛开这些成本不算,方源也有四百多块的元石收益!

%49
所有的家产当中,真正最有价值的,非这株九叶生机草莫属了。掌握了它,就等于掌握了一条金矿!而且这九叶生机草有个优点,就是容易喂养。只需水和阳光,它就能存活,因此喂养起来,几乎不花成本。

%50
对于方源来讲,其他的家产,都可以舍弃,惟独这株九叶生机草,必须牢牢抓在手中!

%51
当然,这种九叶生机草蛊,并非方源一人有。山寨中,也有数人掌握着这种草蛊。

%52
甚至还有五株九叶生机草,都是家族集体的财产。每天都有专门的后勤蛊师,执行生产任务,轮流催生出大量的生机叶。

%53
对于方源来讲,这是一件好事。

%54
若是只有他一人拥有九叶生机草。那么家族势必要出手,将这株草蛊收购。就像古月青书代表家族,来收购他的酒虫一样。

%55
类似酒虫、黑白豕蛊、九叶生机草这类的珍稀蛊虫,家族高层都希望掌控住,为整个家族服务。

%56
三天之后。

%57
一片光团在方源的注视下,猛然一爆,一只全新的蛊虫悠悠地漂浮在半空中。

%58
它晶莹剔透。弯弯如月,好像是一块蓝水晶。简而言之,就是月光蛊。体型放大一倍之后的形态。

%59
但它并非是月光蛊,而是更高一层,达到二转的月芒蛊。

%60
这一次。方源合炼成功了。

%61
月芒蛊由一只月光蛊,两只小光蛊合炼而成。一只小光蛊能增幅月刃一倍的攻击力,两只小光蛊仍旧是一倍,这种增幅不能叠加。

%62
但是合炼成的二转月芒蛊,攻击力却达到了月光蛊的三倍!

%63
其实,月光蛊的合炼秘方有很多,晋升路线也有很多。

%64
方源走的这条路线,是最大化增幅月刃的攻击力,月刃的攻击范围仍旧是十米远,并没有增加。

%65
有一种路线。是用月光蛊和痕石蛊合炼。合炼成的月痕蛊,攻击力不变,但是攻击范围却能增长一倍,达到二十米远。

%66
还有一种常见的路线,则是用月光蛊和旋风蛊合炼。炼成月旋蛊。使用出来,月刃由蓝变绿,同时攻击方式由直线变化为曲线。古月青书就是走的这个路线。

%67
至于古月方正,他用月光蛊和玉皮蛊合炼,炼成月霓裳。这是比较罕见的路线,最高能晋升到五转。成就宝月光王蛊。

%68
然而有五转的秘方,并不意味着一定就能炼成五转的蛊虫。

%69
很多五转的蛊师,甚至浑身上下都没有一只五转蛊虫。

%70
造成这种尴尬局面的最大原因,不是材料不足,而是成功率。

%71
合炼蛊虫,并非是百分百成功率。合炼越高端的蛊虫,成功率就越低。方源前世合炼春秋蝉,成功率不到百分之一,失败无数次。有时候运气好,蛊虫死的少。有的运气差,蛊虫就全死光了。

%72
要合炼出六转的春秋蝉,需要的都是五转的蛊虫。这些蛊虫一死,等若方源前边辛辛苦苦积累出来的东西,都打了水漂,成了泡影挥发幻灭。

%73
方源一次次失败,就得一次次从头开始,重新合炼,重新收刮蛊虫和特殊材料。最终动静大了,闹得天怒人怨,人心涣散,血海漂尸。

%74
也算他运气好,最终合炼成功,得到了春秋蝉。

%75
不过这只六转蛊虫他一得到手,就被早早窥视觊觎的正道人士围攻,宝贝还没捂热,他就自爆了。

%76
千万年来,就是这个万恶的失败率,让无数高转蛊师功亏一篑,被打回原形。

%77
唯有一个方法,能稍稍地克制这种失败率。

%78
那就是——

%79
本命蛊。

%80
不管合炼的结果,是失败还是成功。蛊师的本命蛊是不会死亡的,最多不过受伤罢了。

%81
为什么会这样?

%82
很多人猜测,可能是因为本命蛊是蛊师的第一只蛊虫,和蛊师性命交修,形成了一种神秘玄妙的生命联系。

%83
只要蛊师还活着,本命蛊在合炼失败后,至多也只是奄奄一息的状态。

%84
当然,和本命蛊合炼的其他蛊虫,仍会有死伤的可能。

%85
但即便这样,蛊师一部分的成果,就能保存着,陆续积累下来了。

%86
本命蛊是蛊师的最大财富和依靠,本命蛊是什么,将很大程度上影响蛊师的发展方向。反过来,蛊师也会积极不断地寻找和发现秘方,来提升本命蛊的层次。

%87
一些二转、三转的低级秘方,对于蛊师来讲,反而缩减了他们的发展前景。

%88
当方源发现,春秋蝉成了他的本命蛊,他为什么如此高兴呢?

%89
原因就在于此了。

%90
春秋蝉是罕见珍稀的蛊虫,能令人重生,能力逆天。不管怎么合炼,它都不会灭亡。若在方源的前世,春秋蝉不是他的本命蛊,若要继续合炼下去,就会有消亡的危机。

%91
春秋蝉高达六转,已经是绝大多数蛊师终生都难以企及的成就。非常多的蛊师连一个六转的秘方也没有,都在苦苦追寻呢!

%92
春秋蝉虽然方源目前发挥不出真正的作用,但却是他最大的宝藏。花酒行者的力量传承,和它一比,简直是地和天之间的差距。

\end{this_body}

