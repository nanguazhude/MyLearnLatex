\newsection{小神捕}    %第一百七十四节:小神捕

\begin{this_body}

%1
铁血冷又道!”现在,你来分析一下方源这个人罢。”

%2
“父亲,你是怀疑方源吗?”铁若男愣了一下,旋即又道“他是很理智很聪明的人,为我们介绍山寨时,说的每一句话都恰到好处,条理分明。嘶……”

%3
铁若男忽然微微抽了一口冷气。

%4
她皱起眉头:“刚刚没有觉得,但是父亲你现在一提醒,我发现了!这个方源太会说话了,他说的每一句话,都是客观事实,抛弃了个人感情。站在旁观者的角度,冷静述说。让人找不到他话中的把柄,也觉察不出话外的线索。他的话,很……

%5
……

%6
很干净。”少女犹豫了一下,最终用了“干净”这个词。

%7
铁血冷点点头,又摇摇头:“他不是抛弃个人感情,任何人都有感情,就算是再冷血的杀手,也有感情。他只是将个人的感情,隐藏得很好,掌控得很好。这个少年身上,有一股魔性。”

%8
“魔性?”

%9
“不错。想想看,他在酒席上直接坦诚,说自己畏惧,不敢上战场。你说,换做一般的人,会这么做吗?”铁血冷问道。

%10
铁若鼻摇摇头:“不会的。盅师都将家族的荣誉,自身的名誉,看得比性命还要重要。但是,也不一定啊。自毁声誉的人,历史上也有很多,不是吗?”

%11
“不错。但那些是什么人呢?”铁血冷目光深邃。

%12
铁若男思索一阵,脸色有些震动:“无”不是人杰!”

%13
“正是如此。历史上但凡自毁声誉者,无非是有两个目的。一个是图谋远大和目标相比,名誉也算不得什么。另一个则是为了自保,自污而避猜忌。”

%14
铁若男双眼骤亮:“父亲你是说?”“你多想了,只是觉得这少年很有意思。可惜他只有丙等资质……”铁血冷却道。

%15
这一夜,月光如水。

%16
方源走在无人的街道上,脚步有些沉重,却又坚定。

%17
刚刚和铁血冷接触了一番,果然是盛名之下无虚士。这铁血冷有洞悉世事的目光智谋和城府皆有深不可测的气象。他纵横南疆数十载,闯下赫赫威名,真的是一方人杰。

%18
要在这样的人面前摆脱嫌疑,千难万难。只要给他们足够的时间,绝对会查明真相!

%19
“现在就要拼时间了。不过,漠脉的招揽,倒的确可以利用。”就在不久前,在漠家庭院,方源狮子大开口,要娶漠颜自是可以但是却要有十万元石,十只珍稀盅虫,每一只至少都是三转。

%20
这个要求惹得古月漠尘大怒。

%21
自己要把宝贝孙女下嫁给你,你居然讨了便宜还卖乖?!

%22
而且大言不惭地索要聘礼,还如此贪婪,真的是忍无可忍!

%23
所以直接就谈崩了将方源赶了出去。

%24
方源掉头就走,没有任何的留念。

%25
他知道了漠尘的想法,就笃定他会妥协。而自己狮子大开口,无非是张口要价,坐地还钱罢了。

%26
“不过此事,利弊参半。虽然有了元石,可以喂给天元宝莲。但是自己原先退出政治漩涡的计划也被打乱了。今夜的酒席上,古月漠尘牺牲自己换取了我的政治前途。接下来,恐怕就会有各方家老的打压。”现在古月山寨的政治格局是:族长健在,双重臣之一的古月赤练身死,但赤脉并非无人,继承人古月赤城还活着,同时还有古月赤钟这样的同出一脉的家老。而漠脉死了继承人,漠尘重伤落入二阶,连家老的身份都保不住。

%27
昔日的两大势力,都落魄了。反倒是药脉,多数作为治疗盅师,

%28
处于后方牺牲较少,实力保存完好有上升之趋势。

%29
药脉原本是在族长派系里,现在却完全有了自立的资格。但不管古月药姬是选择自立,还是继续依附,想要壮大自身,都需要抢夺和吞并。而落魄的赤、漠两脉,是最好的下手对象。

%30
现在不下手,等到两大势力缓过来,就不好说了。

%31
“红尘漩涡不由己,何朝散发弄扁舟?乘风破浪三万里,方是我辈魔道人!”方源仰头望月,苦叹一声。

%32
他想要脱离这场政治漩涡,但是古月漠尘却将他硬生生地拉回去。

%33
各方的压力向他涌来,铁血冷已经开始着手破案,而另一边,白凝冰也获得了新的臂助。

%34
仿佛是在暗礁中行船,危机四伏,如何能闯出一条大道来?

%35
翌日。

%36
“什么?你说杀害贾金生的凶手,已经查到了。甚至已经被击毙?”铁若男诧异至极。

%37
今天早晨,她早早起床,开始正式地了解案情。

%38
但是没有想到,得到的第一个消息,却是凶手已经伏法。

%39
“不错。凶手是一位魔道盅师。曾经甚至要来刺杀我族新星,企图扼杀天才,结果被我族盅师当场击杀。”一位家老提供线索道。

%40
“真的是这样吗?难道他直接承认,自己就是杀死贾金生的凶手?”铁若男深深地皱起眉头。在她身旁,铁血冷带着青铜面具,如雕塑般默立在一边。

%41
“这倒没有。不过,如果不是他,还会有谁呢?”家老耸耸肩道。

%42
铁若男心中沉吟:“这一切都是猜测,没有证据。但不管真相是否如此,这个魔道盅师必须调查清楚。很有可能,这就是一条直达真相的线索!”

%43
想到这里,铁若男猛地抬头:“他葬在哪里?我要开棺验尸!”

%44
破曰简陋的棺材中,躺着一具尸体。

%45
恶臭扑鼻,帮忙开馆的家奴和蛊师,都嫌恶地躲得老远。

%46
铁家父女却仿佛闻不到似的,铁若男双眼更是亮起一层光芒极为感兴趣地俯身探手。

%47
人的尸体上,残留着许多痕迹。很多时候,一两处微小的痕迹,却是直指凶手的铁证!

%48
这具尸骨上,有许多的伤痕,依稀可见此人的相貌,身上也穿着他原先的服饰。

%49
铁若男翻查半天,这才意犹未尽地站起身来。

%50
“有什么收获吗?”铁血冷轻声问道隐隐考较。

%51
“古月一族认为这人极有可能就是杀害贾金生的凶手,因此将这尸体保存得很好。这具尸体大有问题。”铁若男答道。

%52
“他是个中年男性,右臂比左臂粗壮,双手都有一层厚厚的老茧。

%53
老茧的分布情况,却并不一致。他身上伤痕累累,致命伤很多,临死之前有过一场剧烈的激斗。但他身上,暗伤也有很多。尤其是左脚缺了三根脚趾,这是许多年前的伤口了。”

%54
说到这里,铁若男推断道:“他身前很有可能是个猎户证据有很多。左右不对称的体格,还有手上的老茧,都说明他是经常开弓之人。他身上有许多野兽的爪痕和牙印,常和野兽打交道。他身上的服饰穿着,也并非正统盅师。

%55
尤其是他脚上的草鞋更有意思,编织草鞋的草是竹麻草。这种草只伴生于青矛竹周围。而青茅山盛产青矛竹,除此之外,方圆千里,都没有竹麻草。”

%56
“你的意思是?”铁血冷追问一声。

%57
“这人在成为魔道盅师之前,必是一位猎户。从他身上的穿戴,极有可能就是青茅山此处的土著猎户。”铁若男眼中精芒一闪。

%58
“何以见得是这里的原住民呢?如果是草鞋,他很有可能是杀了这里的村民,自己穿上的。“铁血冷故意反驳道。

%59
“不是这样的。在所以的服饰中鞋很特殊。若是抢来的,大多都不合脚。但你看他的这双鞋,不仅异常合脚,而且编织紧凑,甚至是量身定做。他的左脚缺了三根脚趾,左脚的草鞋就相应地短了一截。

%60
他这脚趾伤口,齐根而段锐利非常。我猜测,极有可能是多年前误踩了陷阱所制。”铁若男道。

%61
铁血冷不置可否,没有肯定什么,也没有批评什么。

%62
正如他先前所说,一切都交由铁若男去破案。

%63
铁若男自顾自地继续说道:“有了这层推断我们完全可以去周围的村庄,进行排查。也许能发现更进一步的线索呃!”

%64
正说着话少女忽然神情一僵。

%65
她猛然想到,这青茅山刚刚经历了一场狼潮浩劫,就算是山寨都损失惨重,更何况山脚下的这些村庄?

%66
她想要凭此方法,调查出此人的身份和情报,恐怕希望很渺茫了。

%67
“但就算是希望渺茫,只要有成功的可能,我还是要试一试!”

%68
少女首次独立破案,干劲十足。

%69
然而大半天的时间过去,她铩羽而归。这一次的狼潮,是有史以来最庞大恐怖的一次,一些村庄甚至没有任何幸存者。这给她的调查带来了非常大的困难。

%70
“这条线索算是断了。接下来你要怎么做?”铁血冷适时问道。

%71
少女咬牙,语气透露着倔强和顽强:“不,还不算完。父亲你不是也说过,真正的线索,其实就隐藏更深处,只要继续挖掘,就会出现。”

%72
“这个魔道盅师的死,透着蹊跷。首先,他为什么要袭击方正呢?方正在什么地方惹到了他,令他奋不顾身,在重重强敌之中,还要拼死刺杀?其次,他是本地人的身份。但是为何死亡之后,却无人认出他来?”

%73
她这番话,令铁血冷都不禁侧目。

%74
“孩子,你真的长大了。”神捕感叹一声,语气唏嘘又有欣慰。

\end{this_body}

