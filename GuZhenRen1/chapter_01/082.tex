\newsection{年末考核开始}    %第八十二节:年末考核开始

\begin{this_body}

雪后自是晴空。

一大早,陆续就有蛊师,走入学堂。

“一年一届的年末考核,又开始了。呵呵呵,想想看,十几年前离开学堂的情形,还历历在目呢。”一个中年男蛊师,一头青色长发垂在背后,站在学堂门前,发出一声感慨。

“头儿,快进去吧,你就爱多愁善感。”他身边一个年轻女蛊师,双唇如血,双手插在裤子两边的口袋,嘴里叼着一根草,翻了一下白眼。

“呵呵呵,药红,不要着急。进去早晚都不要紧。反正族长已经关照了,我们今年的新成员早已经定下来了。”青发男蛊师笑道。

“就是那个叫做方正的甲等天才?”叫做古月药红的女蛊师“切”了一声,不满地道,“族长的意思,不就是让我们小组当保姆么!”

“但这个保姆的任务,可不好完成啊。”青发男蛊师叹了一口气,“算了,进去再说吧。”

随着时间的推移,越来越多的蛊师,走进学堂大门,在演武场中纷纷站定。

这些蛊师有男有女,有年轻的面孔,也有中年汉子,甚至还有不少的老者。

蛊师从学堂学满一年之后,就会出来,组成小组,完成家族发布的任务。这些来到演武场的蛊师,都是每个小组的代表,前来观测学员们的表现,看中了就吸纳到自己的小组里面。

对于小组来讲,这就是吸收新血,发展壮大。

对于加入小组的新成员,在老成员的带领和指导下,也能迅速地适应新环境,更有效率地完成家族任务,同时也能减少伤亡。

太阳渐渐升高,学员们陆续进场。

“今天好多人啊。”少年们赞叹着。

“你们快看,那是青书大人呐。我们古月山寨的二转第一人,青书大人的脾气是出了名的温和。”有人指着青发男蛊师惊呼道。

“赤山学长也来了。”

“那边是漠家的漠颜大小姐!”

青书、赤山、漠颜是蛊师中的璀璨明星,为学员们所熟知。

“唉,他们三位学长的小组,我可进不去。我只有丁等资质,本命蛊又只是一只温丝蛛,将来只能做后勤。”一位少年唉声叹气,又问身边的好友,“你呢?”

“哦,我托了关系,已经联系好了。是我舅舅的堂哥的姐姐的干儿子。”

……

学员们在观察这些蛊师的同时,青书、赤山、漠颜等人也在观察这些学员。

“咦?居然有两个古月方正。”药红看到人群中的方源和方正,不禁惊奇地道。

青发男蛊师古月青书无奈地叹了一口气,“昨天给你的资料你没看么?方正有一个孪生哥哥,他们相貌很相似。不过他的哥哥只是丙等资质。”

“竟然是这样。哦,我好像听说过,就是那个早年作诗的方源?那我们要收他进来么?”药红忽然用手掌拍拍额头,想到哪里说到哪里。

青书却微微摇头:“族长特意关照了,不要招纳他。似乎想要观察什么的样子。而且,这两兄弟的关系并不和睦。我们就算有招纳的意向,估计方源也不会加入的。”

药红却不以为意,撇嘴道:“在所有的小组中,我们可是公认的第一,加入我们,就代表光明的前途。只要是学员就会心动。他怎么可能不愿意?”

青书轻笑一声:“那是你不了解他,你还是先看看我给你的资料再说吧。”

这时,族长古月博,当权家老古月赤练、古月漠尘等人,鱼贯而入。他们走到棚子下,依次坐下。

“今年不仅族长亲临,还有赤练、漠尘两位老大人,都来了。”

看到这里,不管是学员,还是这些蛊师都有些激动,往届可没有这等情况出现。。

“这也不奇怪,赤练、漠尘的孙子都在此届。”

“方正是族长的培养的接班人,将来抗争那个白凝冰的希望种子。族长自然也要好好观察了。”

人群中传来嘈杂的议论声。

“弟弟,你可要好好表现啊。”漠颜望着人群中的漠北,心中默念着。她的小组人数最多,规模最为庞大。此时身边围着一圈蛊师,更彰显出了她的气势。

而作为她的死对头,来自赤脉的古月赤山,则一个人站着,他身材高大威猛,足足高出周围一截,像是一座红塔站着,鹤立鸡群。

再暗自打量了一下赤城之后,他就收回了视线。

族长一通简短的讲话之后,年末考核便开始了。

三个擂台上同时进行着战斗。

一时间,呼喝声,呐喊声,月刃飞射时嗤嗤的响声,拳脚的击撞声,台下蛊师们的议论声汇成了一片。

“这届的拳脚水准,都有些高啊。”很快,药红就看出了不同。

“呵呵呵,这一切都亏了方源呢。”古月青书就笑。

“什么意思?”药红大为不解。

青书就解释了一遍,

药红听了不禁啧啧称奇:“方源这个小子,胆子还蛮大的,有些无法无天的意味。呵呵,他连亲弟弟都欺负,有些意思啊。”

她看向人群中的方源和方正,心中不禁琢磨起来――这两人到底谁是哥哥,谁是弟弟。

“下一战,古月金珠对决古月漠北。”一个擂台上,主持的蛊师高喊出声。

古月漠北一跃而上,古月金珠则一脸凝重地,顺着台阶,走了上去。

双方行了一礼,也不多话,直接就交手,顷刻间蓝色的月刃,在空中飞舞。

双方不断对射,同时,不断移动和躲闪。

金珠虽是女孩,但基本功相当的扎实,短时间内竟然和漠北斗得个旗鼓相当。但是随着时间的推移,她体力渐渐不济,落入下风。

最终,她娇汗淋漓,没有力气,只能认输。

反观漠北,仍旧脸不红气不喘。

“增强耐力的蛊虫么,搞不好就是黄骆天牛蛊……”台下,方源观察着,顿时就看出了古月漠北的虚实。

方源手中有六只蛊虫,但这是特例。在同龄人的手上,每个人一般都只有两只蛊虫。

不仅是因为喂养蛊虫,经济压力的关系,还有一点,那就是蛊虫的使用,也需要不断的练习,积累出丰富的经验。

贪多嚼不烂,学员们初步接触蛊虫,才刚刚开始蛊师修行,两只蛊虫就够他们练习掌握的了。

只有方源这样的特例,因为有丰富的前世经验,导致几乎每只蛊虫到了他的手上,都能迅速地掌握,使用起来有板有眼。

考核在继续。

“可恶,跳得像个兔子似的!”另一个擂台上,一名少年恼怒地大叫起来,“古月赤城,你是不是爷们,敢不敢和小爷来肉搏!?”

“切,傻子才和你玩肉搏。”擂台上,古月赤城不屑地笑了声。他仗着赤丸蛐蛐蛊,跳来跳去,身手敏捷得很。

而他的对手空有一枚花豕蛊,虽然能暂时暴涨一猪之力,但是却英雄无用武之地。精明的赤城根本就不和他玩近战。

最终他胸膛中了赤城的一记月刃,失血过多,败下阵来。

场下的治疗蛊师们,立即拥上去,为他治疗伤势。

随着时间的推移,许多学员被无情地淘汰,在此同时,不少的少年则开始展露头角。

赤城、漠北、方源、方正……

到了中午时分。

丁等资质的学员们,都已经被全部刷下。他们资质有限,选择的都是后勤类蛊虫,这些蛊虫要做一些生产、运输等等的事情,还比较合适。用作战斗的话,对蛊师起到的帮助,实在是太低了。

“小学妹,你的本命蛊是一只生息草吧,我们小组正需要一名治疗蛊师。”

“这位学长,我想加入你们的小组,我的本命蛊是月光蛊。”

“对不起,我们小组暂时还不缺进攻蛊师。”

……

各个小组早已经开始收纳新人,而学员们也在挑选着小组。

擂台上的战斗,其实观赏性并不高,几场下来,亮点也不多,乏善可陈得很。因为很多人都选择了月光蛊,一般而言,双方对战,首先都是月刃对射。往往谁的真元耗尽,谁就失败。

若是双方真元都同时不济,那就比拼拳脚功夫。总之最终,总会有一人倒下。

不管是学员、蛊师,还是那些家老,看得都有些兴趣泛泛,有些人都快要打瞌睡了。

到了傍晚时分,留在场中的,只剩下个位数的学员。

“终于要结束了。”有些蛊师打着哈切,勉强提起精神。

就在这时,蛊师大喊道:“下一场,古月方正对战古月漠尘!”

\end{this_body}

