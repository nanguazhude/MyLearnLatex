\newsection{我可是班头啊!}    %第五十四节:我可是班头啊!

\begin{this_body}

www.13800100.com

一切都在方源的意料当中。事后,学堂家老就令治疗蛊师,救助了那两个侍卫。侍卫的命保住了,但是落下了重等伤残,被学堂家老驱逐出了学堂。方源没有受到任何的惩罚,反而被表扬了一通。这样的结果,更令其他少年们感到惊惧。但这事还远远没有结束,随着时间的推移,风波扩大到整个族群。方源以丙等资质率先晋升中阶,成了全族中一个不大不小的奇闻和谈资。茶余饭后,很多人都在讨论这件事情。从最初的惊异之后,人们开始揣测方源晋升之谜。“其实以丙等资质,超越甲等、乙等,率先晋升中阶,也不是什么奇怪的事情。”“不错,这世界上有许多方法都能做到这点。”“就比如说是舍利蛊。此蛊一用,就能升华窍壁,助推修为,升上一阶小境界,是小菜一碟的事情。”……一时间众说纷纭,各种可能的答案喧嚣尘上。至于什么酒虫,异种真元等等,自然也被不少人提了出来。方源若是第一时间坦诚酒虫的存在,还不会有这样的轰动效果。但他这样一遮掩,就在无形当中勾动了人们的好奇心。古月山寨表面风平浪静,其实已经涌起一股暗流无数双眼睛都盯着学堂,等着学堂家老的解释。作为学堂家老,如果连自己教导的学员是如何晋升的都不清楚。那也太失职了。所以学堂家老必须做出解释。时间一天天的过去。第二位突破初阶,成功晋升到中阶的少年出现了。是古月漠北。紧接着,仅仅只是三个小时的差距,古月方正也成功晋升。他到底是被元石缺少给拖累了,当然,也不乏被方源此事打击的因素存在。第三位是古月赤城。纵然他有古月赤练亲自灌注真元,但是此法效率不高,三天才能进行一次,并且凶险非常。不过能以丙等资质,夺得第三,已是成功。到了第五天,学堂家老再次分发元石补贴。“古月方源。”他站在前面,第一个便叫出了方源的名字。方源便站起身,一脸的平淡,走上前去。少年们的视线都跟随他而动,目光中流露着嫉妒、羡慕、怀疑、探究、仇恨种种,不一而足。“今天家老大人不仅是发放补贴,还要安排班头、副班头之位!”“果然,家老大人第一个就叫上了方源。”“他第一个晋升,班头要属于他的了。”“难以想象会是这个结果。之前我还以为是方正呢。”“他率先晋升中阶,实在太蹊跷了,一定有重大秘密,可他就是不告诉我们!”“呵呵,换做我也不会告诉你们的。闷声发大财多好。”在学员们交头接耳的过程中,方源走到了学堂家老的面前。“古月方源,你是此届的第一位一转中阶的蛊师,这是你的奖励。”说着,学堂家老递给他一个蓝白相间的钱袋。方源接过钱袋,当众就打开钱袋,朝里面看。“你放心,里面一共是三十块元石,学堂方面不会少你的。”学堂家老的脸上挤出一丝笑容。说实话,他根本就没有料到,第一个晋升中阶的少年会是方源。方源却没听到学堂家老的话。他很少相信别人,向来最信自己。他细心检查了一下,发现的确是三十块元石,一块不少。这才将这钱袋子揣进怀里。学堂家老看着方源这样的表现,还以为他手头紧,脸上的笑容不由地扩大一分。“也是,他只是丙等资质,冲击中阶,消耗的元石势必更多。他又没有其他人的资助,元石拮据是肯定的。只要他对元石有需求,就不怕他脱离掌控。只要他进入家族的体制,他保守的那个秘密,就算是调查不出来,将来总有一天,他也会说出来的。”对此,学堂家老心中自信十足。事实上,自从那天开始,他就差人着手,秘密调查方源。几乎每天,调查都会有新的进展。但显然那些人此时还没调查出来――方源手头上足有数百块的元石,其实宽裕得很。学堂家老又接着道:“方源你第一个踏入一转中阶,按照学堂的规定,不仅会获得三十块元石之奖励,同时还能在不久后,优先挑选第二头蛊虫。现在,我就任命你为班头!”“终究是任命方源为班头了!”此言一出,学员中顿时就有人叹息一声。“可恶。”古月漠北咬着牙,很不甘心。“哼!”古月赤城抱着双臂,冷眼旁观。受到打击最大的还是方源的弟弟,古月方正。他的脸色有些苍白,目光闪烁不定,心头笼上了一层阴云:“普通学员见了班头、副班头,都要鞠躬问好。依我的成绩,必定能落到一个副班头的位置。但是今后我遇到哥哥,还是要行礼的。”“慢。”但就在方源忽然开口。他对学堂家老微微一笑,慢条斯理地道:“家老大人,学生才疏学浅,难以胜任班头这个职务。这样的职位,就让给有才华的人去担任吧。”“什么?你的意思是――你不想担当班头?当了班头,每次都将享受十块元石的补贴。你真要拒绝,你确定吗?”学堂家老眉头顿时拧成了一个疙瘩。他执教数十年,从未见过有人主动拒绝这个职位!其实他也思考过了,让方源担任班头也有好处。一旦成了班头,就是纳入了家族的体制之内。方源就必须在享受待遇的同时,履行义务。不管班头的具体义务是什么,至少他必须停止勒索抢劫同窗的行为。这事绝不是班头能做的。这就像前番,学堂家老即便被方源削了脸皮,也要因为他的修行成绩,而去褒奖方源。身在体制下,处理事情就不能完全的随心所欲,很多时候甚至会身不由己。当然,学堂家老并非看不惯方源,每次都勒索那么多的元石。他是为全体学员着想。一旦方源成为了班头,就不能再暴力欺压同窗。这就能让其他的少年,在方源的重重压迫之下,缓上一口气。之后再因势利导,就能营造出百家争鸣的景象。只要能培养出方正、漠北、赤城这些希望种子,牺牲一个小小的班头位置算得了什么呢?然而学堂家老的打算虽好,但是事实走向却和他预想中的大相径庭。方源拒绝了!他居然拒绝了!!班头这职位虽小,但是在这群年轻气盛的少年们心中,可是代表着第一的荣耀。不仅是荣耀,成为班头,每次补贴更是多达十块元石!这样的诱惑,还从来没有一个少年抵挡得了的。但是方源竟然拒绝了。方源将学堂家老愣神,便向他反问一句:“难道学堂规定了,必须是第一名才能担任班头吗?班头的职位,难道就不能拒绝吗?”学堂家老面沉如水:“当然没有这样不近人情的规定。”方源哈哈一笑:“谢家老体谅。”说完,拱拱手,就下了去,走回到座位上。亲眼目睹此事发生的学员们,顿时一片哗然。一时间,学堂中沸腾了!“方源居然拒绝了?有没有搞错?!”“他脑子有病吧?”“不知道他哪根筋打错了,嘿嘿,今后有的他后悔的。”……“方源放弃了成为班头,也就是说,我就是班头了?!”惊喜来得太快了,成绩第二位的古月漠北有些反应不过来。古月赤城满脸不可相信的神色,他实在难以理解,居然有人主动放弃了班头职位,真是傻瓜透顶!“哥哥……”古月方正瞪着双眼,失神地望着方源一步步走回座位上。按照他方正的成绩,他必定稳坐副班头之位。但是当方源主动放弃了班头之位时,方正忽然觉得自己的这个副班头职位,是如此的索然无味。学堂家老脸色铁青,他这次是真的铁青。前番方源旷课那次,他脸色也很难看,但这都是他做给别人看的样子。他的心情很糟糕,比上一次还要糟糕。方源放弃了班头位置,就是拒绝加入家族体制。学堂家老执掌学堂数十年,还是头一次看到这样的学生,居然能拒绝这么大的诱惑!原本想请君入瓮,但是方源就是不入,对此学堂家老也无可奈何。补贴发放完毕。最终古月漠北以第二的成绩,史无前例地取得了班头之外。方正和赤城则各为副班头。班头的补贴是十块元石。副班头的补贴则增长到五块。一些家境不怎样的学员,看到这三人的补贴,羡慕得几乎要流下口水来。

------------

一轮红日,缓缓地向西方的大地群山滑落。

它的光,不在刺眼炫目,而是透着一种柔和明亮。

西边的天空,都被它染成一片通红,晚霞连绵成一片,宛若妃子得到了赏赐,欣喜地簇拥着帝王,要一起晚睡。

青茅山的一切,都笼罩在一片模糊的玫瑰色之中。

一座座的高脚吊楼,也披上了一层金纱。

学堂周围栽种的树林,仿佛抹上了一层淡淡的油。

风徐徐地吹着,学员们怀揣着刚刚领到的元石补贴,走出学室,各个神清气爽。

“真不知道方源是怎么想的,居然放弃了班头的位置!”

“呵呵呵,他脑袋坏掉了,估计整天都在想着杀人,我们不要理会这种疯子。”

“说起来,那天他闯进学堂,我真的被吓一跳。太恐怖了,那天回去之后,我就做了一晚上的噩梦。”

学员们三三两两地,结伴而行。

“班头好。”

“嗯。”

“班头好。”

“嗯嗯。”

古月漠北大摇大摆地走着,所到之处,学员们无不向他鞠躬致礼。

他的脸色有着压抑不住的兴奋和陶醉。

这就是权力的魅力。

哪怕是一点点的区别待遇,都能让人更加肯定自身的价值。

此时落日西下,残阳如血,漠北看着,心中欢喜地想到:“怎么以前就没觉察过,这夕阳红润的真是可爱啊……”

“哼,当了一个班头就抖起来了,有什么了不起的。”古月赤城故意走在后面,就是不想向古月漠北行礼问好。

“真不知道方源究竟在想什么,竟然放着好好的班头不做。不过也幸好如此,否则第三的我,怎么能得到副班头的位置?”古月赤城心中有不解,也有庆幸。

“副班头好。”这时一个普通学员走过他的身边,立即向他鞠躬问好。

“嘿嘿,你也好。”古月赤城顿时点点头,脸上笑开了花。

学员走了过去,他就自然而然地想到:“这副班头的滋味不错,想来班头的滋味就更妙了。如果我不是副班头,而是班头,那该多好!”

刚刚还为此庆幸的赤城,此时已经得陇望蜀,对班头的位置产生了期待。

家族的体制下,一层高过一层的权位,就像是一根比一根大的胡萝卜,在前方深深诱惑住他。

“虽然我只有丙等资质,但是我相信,一切都会变得越来越好的。”古月赤城对未来心怀希望。

然而此刻,同为副班头的古月方正,心情却很糟糕,脸色也十分的难看。

“哥哥,你!”他瞠目结舌地看着学堂的大门口,一个孤独的身影就堵在那里。

“老规矩,每人一块元石。”方源抱臂站着,语气平淡。

方正张了张嘴,几番努力后这才说道:“哥哥,我可是副班头了!”

“的确。”方源面无表情地点点头,淡淡地看了方正一眼,“副班头每次补贴多达五块。那你就交三块出来罢。”

方正瞠目结舌,一时间竟说不出话来。

一群少年簇拥着古月漠北走了过来。

看到方源堵在校门口,古月漠北大怒,手指向方源:“方源!你好大的胆子,居然还敢堵我们?!我现在已经是班头,你这个普通学员见了我,首先要鞠躬问好!”

回答他的是方源的拳头。

古月漠北猝不及防,被一拳打中,不禁倒退几大步,一脸的难以置信:“你打我,你居然敢打我?我可是班头啊!”

再次回答他的,仍旧是方源的拳头。

砰砰砰。

几次攻防转换之后,古月漠北被方源击倒在地,昏迷过去。

周围的少年们,统统目瞪口地看着,一时间都不知道该怎么反应才好。

这一切都和他们想象的不一样!

门口的侍卫看着这一切在他们的眼皮子底下发生,不禁窃窃私语起来。

“方源把新任的班头,都给打倒了,我们怎么办?”

“凉拌!”

“什么意思?”

“就是看着呗,然后招呼其他人,收拾场子。”

“可是……”

“哼哼,方源是什么样的人物,你也想去招惹?想想王大、吴二两个人现在是什么的下场吧!”

提问的侍卫顿时一个激灵,再也不说什么了。

大门口的两个侍卫,都把身躯挺得笔直。任由一场劫案在身边发生,仿佛他们都是聋子和瞎子,什么也听不见,什么都看不到。

方源收拾了古月漠北,又收拾了方正和赤城。

其他的少年们这才意识到,原来这一切都没有改变。方源还是那个方源,抢劫还是会如期而至。

“每人一块元石,副班头三块,班头八块。”方源公布了新的规矩。

少年们叹着气,乖乖地掏出元石。

当他们走出学堂大门,忽然有人一拍脑袋,大叫起来:“我想到了,难怪方源不要班头的职位,他是想继续勒索我们呀!”

“不错。他每次勒索我们都有五十九块元石,现在则增长到六十八块。他要是班头,顶多就只有十块而已。”不少人跟着恍然大悟。

“太阴险了,太狡诈了,太狠毒了!”有人拍着大腿,愤恨不平。

“唉,这样一来,班头、副班头也没有什么了不起的。他们照样被抢,剩下的元石仍旧只有两块,和我们完全一样呢。”

不知谁说的这句话,少年们听了,都不由地沉默了下来。

砰!

学堂家老狠狠地一拍桌子,勃然大怒。

“这个方源太过分了,他想干什么?居然还敢抢,抢了班头八块元石,副班头三块元石。这样一来,班头、副班头和其他普通学员有什么区别?!”学堂家老努力压低声音,他的话语中充满了愤怒的情绪。

方源拒绝班头职位,就是拒绝将自己纳入家族的体制。说的严重点,就是对家族的一种背叛。

这就已经让学堂家老十分生气了。

紧接着,方源又抢劫同窗。他的手伸得越来越长了,已经超出了学堂家老的底线。

被这么一勒索,班头、副班头的权势地位就被彻底地削弱下去。

久而久之,普通学员们也会对这两个职位失去敬畏和兴趣。

方源此举,看似微小,意义却重大。

这已经是在以一己之力,挑战家族的体制!

这是学堂家老绝不愿意看到的情景。他培养的是家族的新希望,而不是家族的背叛者。

然而尽管方源此举挑战了他的底线,但是学堂家老却知道,自己并不能出手处理此事。

若他真的这样做了,第一个绕不过他的就是族长。第二个第三个对他有意见的,就是古月赤练和古月漠尘。

族长寄希望于古月方正,方正毕竟是三年来唯一的甲等天才。族长需要一个顽强自立的天才,不需要一个被照顾关怀的娇嫩花朵。

同时对赤练和漠尘来讲,他们也希望各自的孙子,能够在这种挫折中成长。

要是学堂家老出手,替学员们惩处了方源。这话传出去,就是“漠家、赤家的未来接班人,打不过方源,只好让长辈帮忙。”

多难听啊。

这对漠家、赤家的名誉,必将是一次重大的打击。

学堂家老当然不怕一个小小的方源,但他却担心一旦插手此事,将引来族长、漠脉、赤脉的三重压力,这就几乎是整个古月高层了。他一个小小的家老哪里能承受得起?

“这事情的根源,还是在于方源的那个秘密。他究竟是依靠什么,率性晋升到中阶的呢?”学堂家老按捺住心中的火气,又将目光集中在书桌上的三份调查报告上。

第一份报告上,是方源的详细家庭背景。

方源此次根正苗红,身份没有蹊跷,身世也一清二白。双亲亡故,被舅父舅母收留。但是并不和睦,上了学堂之后方源就一直住在学堂宿舍。

第二份报告上,是方源的生平过往的记录。

他年少便有早智,被族人看好,认为可能是甲等资质。但是开窍大典之后,却测出丙等,令族人大为失望。

第三方报告上,是方源近期的踪迹。

他的起居生活非常之简单,几乎是三点一线。白天他都在学堂上课,晚上都在宿舍睡眠。他修行十分刻苦,每天晚上都要进行蛊师修行,温养空窍。有时候会出去一趟,到山寨中唯一的一家客栈改善伙食,买酒喝。

他对酒情有独钟,喜欢喝青竹酒。他的宿舍床下,就摆放了数十坛的青竹酒。

学堂家老又详细看了这三份报告,心中对于方源的印象又深刻生动了一分。

“双亲早死,又和舅父舅母不和……难怪方源这个小子,对家族没有归属感。他被族人亲手冠上天才之名,又亲手摘取下来,从高空摔到地上……他难怪桀骜不驯,又怪癖冷漠。他生活如此简单,修行刻苦,就是憋着一口气,不服输,想向族人证明他能行!所以我打压他的时候,他才如此激烈的反击吧……”

学堂家老思考到这里,不禁轻轻地叹了一口气。

越了解方源,他就越理解方源。

当然,理解并不代表原谅。方源顶撞他,触犯了他的尊威,又拒绝担任班头,还大肆抢劫同窗,这都是他不能容忍的。

抖了抖手中的这些资料,学堂家老又皱起眉头:“这些资料虽然详细,但是却对方源的晋升秘密毫无涉及。这都几天了,这帮人也太不像话了!”

咚咚咚。

就在这时,敲门声响起。

“进来。”学堂家老道。

门开了。

却是族长古月博的亲卫:“族长有命,请家老大人速去家主阁,有要事商议。”

“哦,是什么要事?”学堂家老从座位上站起身来,他从亲卫的语气和神态中感受到了此事的重大。

“是四转蛊师贾富大人又回来了,他的弟弟贾金生行踪不明!”亲卫答道。

“嘶……”学堂家老顿时倒抽一口冷气。

------------

\end{this_body}

