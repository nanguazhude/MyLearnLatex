\newsection{方源就任命你为班头}    %第五十三节:方源就任命你为班头

\begin{this_body}

%1
一切都在方源的意料当中。

%2
事后,学堂家老就令治疗蛊师,救助了那两个侍卫。

%3
侍卫的命保住了,但是落下了重等伤残,被学堂家老驱逐出了学堂。

%4
方源没有受到任何的惩罚,反而被表扬了一通。

%5
这样的结果,更令其他少年们感到惊惧。

%6
但这事还远远没有结束,随着时间的推移,风波扩大到整个族群。

%7
方源以丙等资质率先晋升中阶,成了全族中一个不大不小的奇闻和谈资。

%8
茶余饭后,很多人都在讨论这件事情。从最初的惊异之后,人们开始揣测方源晋升之谜。

%9
“其实以丙等资质,超越甲等、乙等,率先晋升中阶,也不是什么奇怪的事情。”

%10
“不错,这世界上有许多方法都能做到这点。”

%11
“就比如说是舍利蛊。此蛊一用,就能升华窍壁,助推修为,升上一阶小境界,是小菜一碟的事情。”

%12
……

%13
一时间众说纷纭,各种可能的答案喧嚣尘上。

%14
至于什么酒虫,异种真元等等,自然也被不少人提了出来。

%15
方源若是第一时间坦诚酒虫的存在,还不会有这样的轰动效果。

%16
但他这样一遮掩,就在无形当中勾动了人们的好奇心。

%17
古月山寨表面风平浪静,其实已经涌起一股暗流无数双眼睛都盯着学堂,等着学堂家老的解释。

%18
作为学堂家老,如果连自己教导的学员是如何晋升的都不清楚。那也太失职了。

%19
所以学堂家老必须做出解释。

%20
时间一天天的过去。

%21
第二位突破初阶,成功晋升到中阶的少年出现了。

%22
是古月漠北。

%23
紧接着,仅仅只是三个小时的差距,古月方正也成功晋升。

%24
他到底是被元石缺少给拖累了,当然,也不乏被方源此事打击的因素存在。

%25
第三位是古月赤城。纵然他有古月赤练亲自灌注真元,但是此法效率不高,三天才能进行一次,并且凶险非常。

%26
不过能以丙等资质,夺得第三,已是成功。

%27
到了第五天,学堂家老再次分发元石补贴。

%28
“古月方源。”

%29
他站在前面,第一个便叫出了方源的名字。

%30
方源便站起身,一脸的平淡,走上前去。少年们的视线都跟随他而动,目光中流露着嫉妒、羡慕、怀疑、探究、仇恨种种,不一而足。

%31
“今天家老大人不仅是发放补贴,还要安排班头、副班头之位!”

%32
“果然,家老大人第一个就叫上了方源。”

%33
“他第一个晋升,班头要属于他的了。”

%34
“难以想象会是这个结果。之前我还以为是方正呢。”

%35
“他率先晋升中阶,实在太蹊跷了,一定有重大秘密,可他就是不告诉我们!”

%36
“呵呵,换做我也不会告诉你们的。闷声发大财多好。”

%37
在学员们交头接耳的过程中,方源走到了学堂家老的面前。

%38
“古月方源,你是此届的第一位一转中阶的蛊师,这是你的奖励。”

%39
说着,学堂家老递给他一个蓝白相间的钱袋。方源接过钱袋,当众就打开钱袋,朝里面看。

%40
“你放心,里面一共是三十块元石,学堂方面不会少你的。”

%41
学堂家老的脸上挤出一丝笑容。说实话,他根本就没有料到,第一个晋升中阶的少年会是方源。

%42
方源却没听到学堂家老的话。他很少相信别人,向来最信自己。他细心检查了一下,发现的确是三十块元石,一块不少。这才将这钱袋子揣进怀里。

%43
学堂家老看着方源这样的表现,还以为他手头紧,脸上的笑容不由地扩大一分。

%44
“也是,他只是丙等资质,冲击中阶,消耗的元石势必更多。他又没有其他人的资助,元石拮据是肯定的。只要他对元石有需求,就不怕他脱离掌控。只要他进入家族的体制,他保守的那个秘密,就算是调查不出来,将来总有一天,他也会说出来的。”

%45
对此,学堂家老心中自信十足。

%46
事实上,自从那天开始,他就差人着手,秘密调查方源。

%47
几乎每天,调查都会有新的进展。但显然那些人此时还没调查出来——方源手头上足有数百块的元石,其实宽裕得很。

%48
学堂家老又接着道:“方源你第一个踏入一转中阶,按照学堂的规定,不仅会获得三十块元石之奖励,同时还能在不久后,优先挑选第二头蛊虫。现在,我就任命你为班头!”

%49
“终究是任命方源为班头了!”此言一出,学员中顿时就有人叹息一声。

%50
“可恶。”古月漠北咬着牙,很不甘心。

%51
“哼!”古月赤城抱着双臂,冷眼旁观。

%52
受到打击最大的还是方源的弟弟,古月方正。他的脸色有些苍白,目光闪烁不定,心头笼上了一层阴云:“普通学员见了班头、副班头,都要鞠躬问好。依我的成绩,必定能落到一个副班头的位置。但是今后我遇到哥哥,还是要行礼的。”

%53
“慢。”但就在方源忽然开口。

%54
他对学堂家老微微一笑,慢条斯理地道:“家老大人,学生才疏学浅,难以胜任班头这个职务。这样的职位,就让给有才华的人去担任吧。”

%55
“什么?你的意思是——你不想担当班头?当了班头,每次都将享受十块元石的补贴。你真要拒绝,你确定吗?”

%56
学堂家老眉头顿时拧成了一个疙瘩。他执教数十年,从未见过有人主动拒绝这个职位!其实他也思考过了,让方源担任班头也有好处。一旦成了班头,就是纳入了家族的体制之内。方源就必须在享受待遇的同时,履行义务。

%57
不管班头的具体义务是什么,至少他必须停止勒索抢劫同窗的行为。这事绝不是班头能做的。这就像前番,学堂家老即便被方源削了脸皮,也要因为他的修行成绩,而去褒奖方源。身在体制下,处理事情就不能完全的随心所欲,很多时候甚至会身不由己。

%58
当然,学堂家老并非看不惯方源,每次都勒索那么多的元石。他是为全体学员着想。一旦方源成为了班头,就不能再暴力欺压同窗。这就能让其他的少年,在方源的重重压迫之下,缓上一口气。

%59
之后再因势利导,就能营造出百家争鸣的景象。只要能培养出方正、漠北、赤城这些希望种子,牺牲一个小小的班头位置算得了什么呢?

%60
然而学堂家老的打算虽好,但是事实走向却和他预想中的大相径庭。方源拒绝了!他居然拒绝了!!班头这职位虽小,但是在这群年轻气盛的少年们心中,可是代表着第一的荣耀。不仅是荣耀,成为班头,每次补贴更是多达十块元石!

%61
这样的诱惑,还从来没有一个少年抵挡得了的。但是方源竟然拒绝了。

%62
方源将学堂家老愣神,便向他反问一句:“难道学堂规定了,必须是第一名才能担任班头吗?班头的职位,难道就不能拒绝吗?”

%63
学堂家老面沉如水:“当然没有这样不近人情的规定。”

%64
方源哈哈一笑:“谢家老体谅。”

%65
说完,拱拱手,就下了去,走回到座位上。

%66
亲眼目睹此事发生的学员们,顿时一片哗然。一时间,学堂中沸腾了!

%67
“方源居然拒绝了?有没有搞错?!”

%68
“他脑子有病吧?”

%69
“不知道他哪根筋打错了,嘿嘿,今后有的他后悔的。”

%70
……

%71
“方源放弃了成为班头,也就是说,我就是班头了?!”惊喜来得太快了,成绩第二位的古月漠北有些反应不过来。

%72
古月赤城满脸不可相信的神色,他实在难以理解,居然有人主动放弃了班头职位,真是傻瓜透顶!

%73
“哥哥……”古月方正瞪着双眼,失神地望着方源一步步走回座位上。

%74
按照他方正的成绩,他必定稳坐副班头之位。但是当方源主动放弃了班头之位时,方正忽然觉得自己的这个副班头职位,是如此的索然无味。学堂家老脸色铁青,他这次是真的铁青。

%75
前番方源旷课那次,他脸色也很难看,但这都是他做给别人看的样子。他的心情很糟糕,比上一次还要糟糕。

%76
方源放弃了班头位置,就是拒绝加入家族体制。学堂家老执掌学堂数十年,还是头一次看到这样的学生,居然能拒绝这么大的诱惑!原本想请君入瓮,但是方源就是不入,对此学堂家老也无可奈何。

%77
补贴发放完毕。最终古月漠北以第二的成绩,史无前例地取得了班头之外。方正和赤城则各为副班头。班头的补贴是十块元石。副班头的补贴则增长到五块。一些家境不怎样的学员,看到这三人的补贴,羡慕得几乎要流下口水来。

\end{this_body}

