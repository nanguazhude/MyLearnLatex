\newsection{方源就任命你为班头}    %第五十三节:方源就任命你为班头

\begin{this_body}

www.13800100.com

一切都在方源的意料当中。

事后,学堂家老就令治疗蛊师,救助了那两个侍卫。

侍卫的命保住了,但是落下了重等伤残,被学堂家老驱逐出了学堂。

方源没有受到任何的惩罚,反而被表扬了一通。

这样的结果,更令其他少年们感到惊惧。

但这事还远远没有结束,随着时间的推移,风波扩大到整个族群。

方源以丙等资质率先晋升中阶,成了全族中一个不大不小的奇闻和谈资。

茶余饭后,很多人都在讨论这件事情。从最初的惊异之后,人们开始揣测方源晋升之谜。

“其实以丙等资质,超越甲等、乙等,率先晋升中阶,也不是什么奇怪的事情。”

“不错,这世界上有许多方法都能做到这点。”

“就比如说是舍利蛊。此蛊一用,就能升华窍壁,助推修为,升上一阶小境界,是小菜一碟的事情。”

……

一时间众说纷纭,各种可能的答案喧嚣尘上。

至于什么酒虫,异种真元等等,自然也被不少人提了出来。

方源若是第一时间坦诚酒虫的存在,还不会有这样的轰动效果。

但他这样一遮掩,就在无形当中勾动了人们的好奇心。

古月山寨表面风平浪静,其实已经涌起一股暗流无数双眼睛都盯着学堂,等着学堂家老的解释。

作为学堂家老,如果连自己教导的学员是如何晋升的都不清楚。那也太失职了。

所以学堂家老必须做出解释。

时间一天天的过去。

第二位突破初阶,成功晋升到中阶的少年出现了。

是古月漠北。

紧接着,仅仅只是三个小时的差距,古月方正也成功晋升。

他到底是被元石缺少给拖累了,当然,也不乏被方源此事打击的因素存在。

第三位是古月赤城。纵然他有古月赤练亲自灌注真元,但是此法效率不高,三天才能进行一次,并且凶险非常。

不过能以丙等资质,夺得第三,已是成功。

到了第五天,学堂家老再次分发元石补贴。

“古月方源。”

他站在前面,第一个便叫出了方源的名字。

方源便站起身,一脸的平淡,走上前去。少年们的视线都跟随他而动,目光中流露着嫉妒、羡慕、怀疑、探究、仇恨种种,不一而足。

“今天家老大人不仅是发放补贴,还要安排班头、副班头之位!”

“果然,家老大人第一个就叫上了方源。”

“他第一个晋升,班头要属于他的了。”

“难以想象会是这个结果。之前我还以为是方正呢。”

“他率先晋升中阶,实在太蹊跷了,一定有重大秘密,可他就是不告诉我们!”

“呵呵,换做我也不会告诉你们的。闷声发大财多好。”

在学员们交头接耳的过程中,方源走到了学堂家老的面前。

“古月方源,你是此届的第一位一转中阶的蛊师,这是你的奖励。”

说着,学堂家老递给他一个蓝白相间的钱袋。方源接过钱袋,当众就打开钱袋,朝里面看。

“你放心,里面一共是三十块元石,学堂方面不会少你的。”

学堂家老的脸上挤出一丝笑容。说实话,他根本就没有料到,第一个晋升中阶的少年会是方源。

方源却没听到学堂家老的话。他很少相信别人,向来最信自己。他细心检查了一下,发现的确是三十块元石,一块不少。这才将这钱袋子揣进怀里。

学堂家老看着方源这样的表现,还以为他手头紧,脸上的笑容不由地扩大一分。

“也是,他只是丙等资质,冲击中阶,消耗的元石势必更多。他又没有其他人的资助,元石拮据是肯定的。只要他对元石有需求,就不怕他脱离掌控。只要他进入家族的体制,他保守的那个秘密,就算是调查不出来,将来总有一天,他也会说出来的。”

对此,学堂家老心中自信十足。

事实上,自从那天开始,他就差人着手,秘密调查方源。

几乎每天,调查都会有新的进展。但显然那些人此时还没调查出来――方源手头上足有数百块的元石,其实宽裕得很。

学堂家老又接着道:“方源你第一个踏入一转中阶,按照学堂的规定,不仅会获得三十块元石之奖励,同时还能在不久后,优先挑选第二头蛊虫。现在,我就任命你为班头!”

“终究是任命方源为班头了!”此言一出,学员中顿时就有人叹息一声。

“可恶。”古月漠北咬着牙,很不甘心。

“哼!”古月赤城抱着双臂,冷眼旁观。

受到打击最大的还是方源的弟弟,古月方正。他的脸色有些苍白,目光闪烁不定,心头笼上了一层阴云:“普通学员见了班头、副班头,都要鞠躬问好。依我的成绩,必定能落到一个副班头的位置。但是今后我遇到哥哥,还是要行礼的。”

“慢。”但就在方源忽然开口。

他对学堂家老微微一笑,慢条斯理地道:“家老大人,学生才疏学浅,难以胜任班头这个职务。这样的职位,就让给有才华的人去担任吧。”

“什么?你的意思是――你不想担当班头?当了班头,每次都将享受十块元石的补贴。你真要拒绝,你确定吗?”

学堂家老眉头顿时拧成了一个疙瘩。他执教数十年,从未见过有人主动拒绝这个职位!其实他也思考过了,让方源担任班头也有好处。一旦成了班头,就是纳入了家族的体制之内。方源就必须在享受待遇的同时,履行义务。

不管班头的具体义务是什么,至少他必须停止勒索抢劫同窗的行为。这事绝不是班头能做的。这就像前番,学堂家老即便被方源削了脸皮,也要因为他的修行成绩,而去褒奖方源。身在体制下,处理事情就不能完全的随心所欲,很多时候甚至会身不由己。

当然,学堂家老并非看不惯方源,每次都勒索那么多的元石。他是为全体学员着想。一旦方源成为了班头,就不能再暴力欺压同窗。这就能让其他的少年,在方源的重重压迫之下,缓上一口气。

之后再因势利导,就能营造出百家争鸣的景象。只要能培养出方正、漠北、赤城这些希望种子,牺牲一个小小的班头位置算得了什么呢?

然而学堂家老的打算虽好,但是事实走向却和他预想中的大相径庭。方源拒绝了!他居然拒绝了!!班头这职位虽小,但是在这群年轻气盛的少年们心中,可是代表着第一的荣耀。不仅是荣耀,成为班头,每次补贴更是多达十块元石!

这样的诱惑,还从来没有一个少年抵挡得了的。但是方源竟然拒绝了。

方源将学堂家老愣神,便向他反问一句:“难道学堂规定了,必须是第一名才能担任班头吗?班头的职位,难道就不能拒绝吗?”

学堂家老面沉如水:“当然没有这样不近人情的规定。”

方源哈哈一笑:“谢家老体谅。”

说完,拱拱手,就下了去,走回到座位上。

亲眼目睹此事发生的学员们,顿时一片哗然。一时间,学堂中沸腾了!

“方源居然拒绝了?有没有搞错?!”

“他脑子有病吧?”

“不知道他哪根筋打错了,嘿嘿,今后有的他后悔的。”

……

“方源放弃了成为班头,也就是说,我就是班头了?!”惊喜来得太快了,成绩第二位的古月漠北有些反应不过来。

古月赤城满脸不可相信的神色,他实在难以理解,居然有人主动放弃了班头职位,真是傻瓜透顶!

“哥哥……”古月方正瞪着双眼,失神地望着方源一步步走回座位上。

按照他方正的成绩,他必定稳坐副班头之位。但是当方源主动放弃了班头之位时,方正忽然觉得自己的这个副班头职位,是如此的索然无味。学堂家老脸色铁青,他这次是真的铁青。

前番方源旷课那次,他脸色也很难看,但这都是他做给别人看的样子。他的心情很糟糕,比上一次还要糟糕。

方源放弃了班头位置,就是拒绝加入家族体制。学堂家老执掌学堂数十年,还是头一次看到这样的学生,居然能拒绝这么大的诱惑!原本想请君入瓮,但是方源就是不入,对此学堂家老也无可奈何。

补贴发放完毕。最终古月漠北以第二的成绩,史无前例地取得了班头之外。方正和赤城则各为副班头。班头的补贴是十块元石。副班头的补贴则增长到五块。一些家境不怎样的学员,看到这三人的补贴,羡慕得几乎要流下口水来。

\end{this_body}

