\newsection{真是好魄力}    %第一百一十二节:真是好魄力

\begin{this_body}

虽然说是继承了双亲的遗产,但是对于方源来讲,积累的时间还是短了一点。

催生生机叶,他也不是每天都进行,毕竟很损耗时间。往往催生出九片生机叶,大半天的时间就消耗殆尽了。

方源思考了一下,这枚赤铁舍利蛊的出售时间,只有一天。要在这么短的时间内,筹措到这么一大笔元石,唯一的方法就是将自己手中的酒肆或者竹楼抵押出去。

这也没有什么可惜的。

一年之后,就是狼潮。记忆中,在狼群的围攻之下,古月山寨几次都是摇摇欲坠,最凶险的一次,连大门都被破开。族长和一众家老牵制雷冠头狼,古月青书用自己的生命,堵住大门,这才堪堪稳住局面。

狼潮将造成青茅山三大家族的严重减员,虽不说十室九空,但至少也去了五成人口。

到那时,房多人少,还谈什么竹楼出租?酒肆又靠着东大门,谁还敢到前线处去喝酒?就算有人想喝,酒肆也早就被家族征用了,改造成防御塔楼。

现在,家族中的许多人都大大低估了狼潮的严重程度。这个时候,若能抛掉手中的酒肆和竹楼,反而能卖到最好的价格。

“钱财只是身外物,只有自身修行才是根本。不过,卖给家族还稍显便宜了一点。卖给个人,价格上会更多一些。但是谁手头上刚好有这么一大笔钱,能买我的竹楼和酒肆?这样的大买卖。也不是第一次见面就能谈成的,双方总归要考察,要讨价还价,这就耗了时间。而我只有一天的时间啊。等一等,也许有个人可以……”

方源忽然灵光一闪,想到了某个人。

这个人,不是别人。正是他的舅父古月冻土。

舅父舅母精明而又吝啬,这十几年来,经营酒肆、竹楼还有售卖生机叶。手中一定有大量的积蓄。

再者,这份产业,本来就是他们经营的。知根知底,也就省下了考察的时间。

更关键的是,现在他们也迫切地需要一份家产,来经营下去。再多的钱财,没有进项,就是无源之水,看着元石越用越少,谁都会犯愁的。

可以说,舅父舅母是目前最适合的交易对象。

想到这里,方源再不迟疑。出了树屋,就走向古月冻土的住处。

为他开门的是沈翠,他曾经的丫鬟。

“啊,是,是你!”看到方源。她很是吃惊。

很快,她意识到自己的失言,害怕得脸色骤然一白。方源如今是二转蛊师,她却不过是一介凡人,双方差距已经是天差地别。

更关键是,方源可是连漠家的家奴都敢杀了。然后碎尸之后,还送还给漠家的凶人呐。

“奴婢见过方源少爷,欢迎方源少爷回家。”沈翠惊恐得浑身颤颤,双膝一软,跪倒在地上。

“家?”方源跨步迈进庭院,他看着这熟悉的一切,脸上露出一丝嘲讽之色,毫无缅怀和留恋之情。

时隔一年,他再次来到这里。

和印象中相比起来,这里明显冷清了许多。就像方正说的,一些家仆已经被转卖或者辞退了。

方源忽然到来,自然惊动了舅父舅母。

作为管家的沈嬷嬷,第一时间赶了过来,卑躬屈膝地将方源迎进了客厅,并亲手奉茶。

方源坐在椅子上,环顾这个会客的厅堂。

许多的家具都已经不见了,布置上简朴寒酸了许多。

不过这并不意味着舅父舅母手中没有积蓄。

“古月冻土还是精明的,这是他的自保之举啊。他已经退隐,战斗力早已经急剧下滑。最关键的是,他失去了九叶生机草,就意味着失去了维系人际网的底牌,再不能对外施加影响力。”

匹夫无罪怀璧其罪。

方源继承了遗产之后,引来了许多族人的眼红和觊觎。

对于舅父舅母来讲,他们同样面对着这个问题。他们手中的大笔积蓄,既是福又是祸。

财不露白,对他们来讲,乃是正确的生存之道。

这时,一阵蹬蹬蹬的脚步声传来。

脚步声越来越近,随即,舅母就出现在门口。

“方源,你还居然还敢过来!”看到方源,她顿时气不打一处来,尖声骂道,“你个养不熟的小狼崽子,我们夫妇是怎么抚养你长大的。结果你这样对待我们,你还有没有良心,你的良心是不是被狗吃了!”

“你还好意思过来,还好意思坐在这里喝茶?你是专门来看我们落魄的样子是吗,现在你看到了,你满意了吗?!”

她一手指着方源,一手叉腰,泼妇一样喝骂着。

若不是方源穿着一身醒目的二转蛊师的武服,提醒着她,恐怕她早就扑上去,撕扯扭掐方源了。

方源被舅母手指着,遭到喝斥怒骂,面色却不变,仿佛没有听到似的。

一年不见,舅母那一张黄脸,虽然充满了愤怒和狰狞,却难以掩盖她的憔悴。

她身上的衣服已经换成了简约的麻衣,头上的发饰也少了。没有涂脂抹粉,显得尖嘴猴腮。

方源夺回了家产,对她的生活造成了相当大的冲击和影响。

对于她的怒骂,方源根本就没有放在心上,他好整以暇地端起杯盏,喝了一口茶水,语气悠悠地道:“我这一次来,是想出售酒肆和竹楼,不知道舅母和舅父有没有兴趣?”

“呸!你这个白眼狼,会安什么好心,哼,想要出售酒肆和竹楼……”舅母语气忽的一滞,她终于反应过来,脸上露出不可置信的神色。“什么,你要出售酒肆和竹楼?”

方源放下手中的茶杯,背往后靠在椅背上,闭目养神:“还是叫舅父来跟我谈吧。”

舅母咬牙,犹自不信,她双眼喷火似的,狠狠地瞪着方源。咬牙切齿地道:“我知道了,你是故意想戏耍我,才这么说的吧!我一旦答应下来。就会遭受你狠狠的奚落和嘲讽。你真当我是傻子,让你这般耍弄?”

这纯粹就是小人之心了。

方源心中叹了一口气,然后说了一句话。就让舅母改变了态度――

“你若再废话,那我就走了。我相信其他人对这份家产也会很感兴趣,到时候,我卖给了别人,你们可不要后悔。”

舅母顿时愣住:“你真要卖了这些产业?”

“我只等五分钟时间。”方源睁开一丝眼缝说道,旋即又闭上了双眼。

他听到舅母一跺脚,然后是一连串,越来越小的脚步声。

不多时,舅父古月冻土出现在方源的面前。至于舅母却没有同行。

方源看了看他。

舅父已经明显的衰老了许多,原本保养的很好的脸。也消瘦下来,双鬓增添了许多白发。

这些天他愁坏了。

失去了这笔家产,他一下子就失去了经济来源。尤其是没有了九叶生机草,让他失去了对外的影响力。

所谓“隐家老”的名头,已经名不副实。

手中虽然捏着一大笔积蓄。但没了这层影响力,这些元石就显得有些烫手了。

家族的政策,隐隐有鼓励族人相互竞争资源的意思。尤其是斗蛊的政策,显得无情又缺乏人性。但这却能杜绝寄生虫,以及纨绔子弟的出现,使得家族人人都保持着危机意识。让家族的战斗力始终处于一种强盛之势。

在这个世界上,只有强盛的战斗力才能保证生存。飓风、洪涝、猛兽可不会和人讲什么道理。

古月冻土这些年来,生活安稳,个人战斗力早已经下滑很多。早年一些得力的蛊虫,他为了减少喂养的耗费,早已经将它们卖掉了。

要是有人向他下战书,他绝对是输多赢少。

面对舅父,方源直接阐明来意。

“方源,那我就直言不讳了。有些事情我不太明白,你何必要卖了酒肆和竹楼呢?保留着它们,今后的元石就会源源不断的。”舅父也有些不相信,但是语气比舅母委婉多了。

“因为我想要购买一只赤铁舍利蛊。”方源坦诚,这事情也不必隐瞒。

“原来是这样。”舅父目光闪了闪,“那么,九叶生机草你也想出售吗?”

“这是绝不可能的。”方源摇头,一点都没有犹豫,“我只出售酒肆、竹楼还有田地以及那八位家奴。”

九叶生机草才是遗产中最具有价值的东西,方源需要它的治疗作用,同时出售生机叶赚取元石,能支撑他的修行,还有喂养其他蛊虫。

而且,明年的狼潮来袭,生机叶的价格肯定要暴涨。方源有这株九叶生机草在手,二转修行的元石就不愁。

但若是舅父得了九叶生机草,那么他“隐家老”的影响力就要再度恢复。方源也不愿意看到这种事情发生。

见方源态度如此坚决,古月冻土心中很是失望。同时,也相当的无奈。

双方密谈了两个多小时,这才签订了一份严密的转让契约。

古月冻土重新得到了酒肆、竹楼、家奴还有田地,而方源则领着三个家奴,每个家奴都抬着一个装满元石的箱子,向树屋走去。

双方算是各取所需。

舅母听到这消息,赶过来。她看着古月冻土手中的一叠房契、地契,瞪圆了眼睛,露出狂喜之色:“老爷,那小子修行傻了,居然把这生钱的产业都给卖了!真是愚蠢,为了鸡蛋,不要下蛋的母鸡。”

“你不说话能死啊,给我闭嘴。”古月冻土却显得有些烦躁。

“老爷……”舅母嘟囔着,“我这不是高兴嘛。”

“得意不要忘形!有了这酒肆和竹楼,更应该谨慎行事,低调做人。树大招风啊。虽然说方正是我们的义子。但这层关系,不能擅用。毕竟方正还没有成长起来,谁知道未来能发生什么?”古月冻土发出一声深深的叹息。

“知道了,老爷!”舅母一边听着,一边拿过这叠房契观看,笑得嘴都合不拢了。

古月冻土的脸色,却一直阴沉着。

虽说是做成了这笔交易,他就有了进项。花费掉的元石,经营个两三年就能补起来。但是他心中却没有一丝高兴。

他满脑子都是方源的身影。

方源为了一只舍利蛊,毫不犹豫地出售了家产,这就等若他直接放弃了今后安逸而舒适的生活。

舅父古月冻土设身处地一想,自己能做到这事情吗?

不能啊。

哪怕他对方源不待见,有着厌恶和憎恨,但是此刻心中却也不禁感叹一声:“能舍能弃,真是好魄力!”

(ps:捏个……求一张月票,好不?)(未完待续。。)

------------

\end{this_body}

