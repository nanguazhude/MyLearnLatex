\newsection{蛤蟆商队}    %第三十九节:蛤蟆商队

\begin{this_body}

五月,是春和夏的过渡。

花香弥漫,大山青青。阳光开始逐渐地绽放出它热烈的一面。

湛蓝的晴空下,白云如棉絮般漂浮。

青茅山上,青茅竹林仍旧是笔直如枪,遥指苍穹。遍地都是野草在疯长,在草丛中点缀着不知名的野花。微风一吹,野草起伏,浓郁的花粉和青草的气息,就扑面而来。

在山腰处,则是大量的梯田。一层层,一阶阶,嫩绿的麦苗栽种了下去,远远看去就像是一片青翠嫩绿的海。

梯田里不少农人,在忙活着。有的在清理沟渠水道进行引水,有的卷起裤脚,站在田地里栽下秧苗。

这些人自然都是外姓凡人,古月族人是不操持这等贱业的。

叮铃铃……

春风中隐隐传来驼铃的声音。

农人们都直起身子,转头往下山脚。

只见一只商队犹如一只色彩斑斓的长虫,从山道那边,缓缓地探出头来。

“是商队啊!”

“是了,如今已经是五月份,商队也该来了。”

大人们心中了然。顽童们则直接放弃了戏水,和手中的泥巴,蹦蹦跳跳地跑向商队去。

南疆有十万大山,青茅山不过是其中之一。每座山上,栖息着一座座的山寨,人们以血脉亲情维系着寨子。

山与山之间,山林深幽,险石峻壁。环境复杂,栖息着大量的猛兽或者稀奇诡异的蛊虫。

凡人根本难以通过。单独的个人,想要闯过这些艰难阻碍,也至少得有三转蛊师的修为。

因此经济凋敝,贸易困难。最主要的贸易形式,就是商队。

只有组成商队这种庞大的规模,才能成群结队的蛊师,有能力互帮互助,克服途中艰难险阻,从一座山,行到另一座山。

商队的到来,像是一碗沸水,陡然倒进了平静宁和的青茅山。

“往年都是四月,今年到了五月,这商队才来。不过总算是来了。”客栈的掌柜听到这个消息后,着实舒了一口气。客栈的生意在其他月份,都极其清淡。只有指望着商队到来,能带来支撑一年的收益。

同时在他库存里还有一些青竹酒,可以向商队兜售了。

不仅是客栈,酒肆的生意也会跟着红火起来。

商队陆续开进了古月山寨,打头的是一只宝气黄铜蟾。这头蟾高达两米五,浑身橘黄色,蟾背宽厚,上面是疣粒疙瘩,如同古代城门上的那一颗颗硕大的铜铆钉。

宝气黄铜蟾的背上,用根根粗麻绳索固定着一大堆的货物。乍一眼看上去,就像是宝蟾背着一个硕大的包袱。

一个中年人,长着一张圆饼麻子脸,顶着肥滚滚的肚子,盘坐在宝蟾的头上。双眼笑着眯成了一条缝,抱拳向周遭的古月寨民打招呼。

此人姓贾名富,有四转修为,是此次商队的领头人。

宝蟾微微蹦跳着前进,贾富坐在蟾头,四平八稳。蹦的时候,他这高度能和二楼的窗口齐平。就算是落地,也要高过竹楼的第一层。

原本宽敞的街道,此时忽然显得有些狭窄。宝气黄铜蟾像是一只怪兽,闯入了林立的竹楼当中。

宝蟾过后,是一只大肥虫。双眼犹如彩色玻璃窗,色彩鲜艳斑斓。长达十五米,体型类似于蚕。但是表面上覆盖着一层厚厚的黑釉皮甲。皮甲上同样堆着一蓬蓬的货物,用麻绳绕圈系着。货物的间隙处,坐着一个个的蛊师,有年老的,也有年轻人。

还有凡人,均是健壮的武者,在地上跟随着黑皮肥甲虫缓缓向前。

肥甲虫之后,又有彩羽斑斓的驼鸡,毛茸茸的山地大蜘蛛,长着两片羽翼的翼蛇等等。但这些只是少数,大多还是以蛤蟆为主。

这些蛤蟆类似宝气黄铜蟾,只是个头要小一些,有牛马般体型。驮着货物和人,鼓着肚子,一蹦一蹦地走着。

商队蜿蜿蜒蜒地深入山寨。

一路上孩童们瞪大了双眼,好奇地看着,欢叫着,惊叹着。

二楼的窗户一个个接连打开,山民近距离地观察。有的双眼闪着忌惮的光,有的在挥手表示热烈的欢迎。

“贾老弟,今年来的有些迟啊,一路辛苦了。”古月博以族长的身份,亲自欢迎了此次商队的领袖。

贾富是四转蛊师的身份,若是让三转的家老负责接待,无疑是一种怠慢和轻视。

贾富双手抱拳,叹了一口气:“今年走的不大顺,路上碰到了一群幽血蝙蝠,损失了不少好手。又在绝壁山,遇到了山雾,实在是不敢走啊。因此拖延了不少时间。教古月兄久等了。”

言语间,十分客气。

古月山寨需要商队每年都来贸易,而商队也需要和气生财。

“呵呵呵,能来就好。请,族中已经备好就酒菜,让我为老弟接风洗尘。”古月博伸手邀请道。

“族长客气了,太客气了。”贾富作受宠若惊状。

商队是早晨到的青茅山地界,中午驻扎进了古月山寨。到了傍晚时分,山寨周边就形成了一片面积广大的临时商铺。各种红蓝黄绿的高大帐篷搭建着,帐篷之间还见缝插针地塞着无数的小地摊。

夜晚降临了,但这里的却一片灯火通明。

络绎不绝的行人,从寨子里涌进这里。有凡人,也有蛊师。小孩子们雀跃蹦跳着,大人们的脸上也涌现出过节一般的喜悦神色。

方源随着人流,独自一人走进这里。

人群熙熙攘攘,一堆堆地围着地摊,或者在帐篷口不断进出。

周围传来彼起彼伏的叫卖声。

“来一来,看一看了啊。上等的蓝海云茶砖,喝上这一口茶,快活似神仙哎!就算是人不喝,喂养茶蛊,也是物廉价美。一块只需要五元石!”

“蛮力天牛蛊,蛊师催动起来,能暴涨一牛之力。走过路过,不能错过!”

“知心草,上等的知心草,大家伙看看这成色,新鲜得像刚采摘下来的一样。一斤两块元石,多便宜的价格啊……”

方源听到这里,脚步微微一顿,循声看了过去。

只见一只鸵鸡拉着一个两轮板车。板车上堆着一堆粉绿色的草。每根草都长达一米,很细长,平均只有指甲盖的宽度。有些草的尖端还长着红心状的花蕾。

知心草是蛊虫的辅助食料之一,其价值在于它能和一些食物搭配起来,喂养蛊虫。

比方说,方源喂养月光蛊,每顿需要喂食两片花瓣。若是掺和上一根知心草,月光蛊吃上一片月兰花瓣就饱了。

知心草一斤只有两块元石,月兰花瓣每十片就得一块元石。稍微算一下,就知道掺和知心草喂养蛊虫更为划算些。

“半个月前,我杀了高碗,因为在学堂动用了月光蛊,罚款三十块块元石。不过后来漠家赔偿了我三十块,算不上损失。这些天,我抢劫了两次,总计一百一十八块元石。但是最近我不断消耗精炼出的中阶真元,温养空窍四壁,每天都要消耗三块元石。再加上蛊虫的喂养费用,自己的生活费用,陆续不断地购买青竹酒而投入的元石。现在手中还有九十八块。”

自从方源杀了人之后,凶残冷酷的形象深入学员内心,一时间再无人敢向他挑战。导致他抢劫变得大为容易。每次只有极少数的人敢于向他对抗。

方源心中算计了一下,就转移了视线,继续向这临时商铺区的深处走去。

知心草摊上,围着一堆人。都是蛊师或者学员,手中拿着元石正在哄抢购买。

方源不是没有钱购买这知心草,而是没有时间。

“如果记得没有错,那只癞土蛤蟆,就在那个店铺中吧。前世就有蛊师就在第一晚赌到了它,因此大赚了一笔。我可得赶紧,不能因小失大了。”

------------

\end{this_body}

