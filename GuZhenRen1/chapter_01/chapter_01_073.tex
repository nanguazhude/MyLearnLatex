\newsection{神秘的红圈标记}    %第七十三节:神秘的红圈标记

\begin{this_body}

%1
“你说方源此次勒索学员,虽然打败了方正、赤城还有漠北三人,但是却放过了他们,没有收取他们的元石?”听到侍卫的禀告,学堂长老露出了微微诧异的表情。

%2
“不敢欺瞒大人,的确是这样子的。”跪在地上的侍卫立即回答道。

%3
“嗯。”学堂长老不置可否,挥手道,“此事我知道了,你下去罢。”

%4
“属下告退。”

%5
侍卫走过,学堂家老立即陷入了沉思。

%6
原本他关注此事,是担心方正得了三十块元石奖励,却被方源勒索过去。如果是这样,学堂的元石奖励根本就没有任何的意义了,干脆全部颁发给方源好了。

%7
方源如果真这样做,那是学堂方面绝不允许的,学堂家老已经做了严惩方源的准备。

%8
但是他没有想到的是,方源此次不仅没有打这三十块元石的主意,还主动放过了漠北、赤城还有方正三人。

%9
“方源有数百块的元石在身,也许看不上这三十块元石,这可以理解。但是主送放弃勒索漠北、赤城还有方正,这是什么缘由?”

%10
学堂家老思索着,皱起的眉头渐渐舒展。

%11
他有些明白了。

%12
漠北、赤城、方正这三人,可以说代表着家族的三大势力。方源放过这三人,意义不言而喻,是主动向这三大势力示好。这亦可以理解为,方源心理的转变,他想要对家族低头的信号。

%13
“可以理解的。随着不断地修行,方源纵然有酒虫,但也足以让他更加清晰地认识到,丙等资质的不足。先前闹腾了几次,不满不甘的情绪发泄了大半之后,现在的他估计也有许多的气馁和抑郁吧。”

%14
“看来族长的话还是有道理的,方源毕竟只是一个十五岁的少年,怎么能挑战得动家族的体制?他现在已经初步地接受了现世,当他找到属于自己的位置,融入家族就是必然的事情了。”

%15
想到这里,学堂家老舒缓了一口气,心情不由地变得愉悦起来。

%16
三天的时间,一晃而过。

%17
很快,年中考核来到了。

%18
“快快快,野猪被我引来了!”一个少年一边狂奔,一边焦急地大吼。

%19
他的两只小腿上,各缠绕着一圈淡绿色的旋风。就是这两圈旋风,使得他年纪轻轻,就有了超越一般凡人的速度。

%20
不过,他身后冲上来的野猪,速度越来越快,和他的距离也越来越近。

%21
烈日透过树林,照在野猪的身上,将它的两颗獠牙照的雪亮。

%22
“野猪来了,把绳子拉紧了!”埋伏着的四位少年,猛地将隐藏在草丛中的粗麻绳拉起来,瞬间形成一道绊马索。

%23
飞奔的少年,轻轻一个纵跃,跳过这道绊马索,继续朝前狂奔。

%24
但是随后的野猪,却被结结实实地绊了一跤,狠狠地摔倒在地上,划出五六米的距离。

%25
“哎哟!”四位少年也被绳子拖着,跟着野猪一样,摔倒在地上。

%26
“上啊!”在前面飞奔的少年已经回转,大叫着。

%27
倒在地上的几人,手忙脚乱地爬起来,纷纷向野猪围去。

%28
……

%29
咔嚓。

%30
一棵小树苗,在野猪的冲撞下,直接树干断裂,树冠一头栽倒在地。

%31
“好险!”古月赤城擦了擦额头的冷汗,心生余悸,“幸亏刚刚我及时地动用龙丸蛐蛐蛊,向左跳了三米远,否则这棵小树,就是我的下场啊。”

%32
……

%33
嗤嗤嗤!

%34
月刃在空中飞舞,射在野猪的身上,造成一个又一个的细长伤口。

%35
古月漠尘面色激动,双目炯炯,心神完全沉浸在了这场战斗当中。

%36
半个小时之后,野猪失血过多,终于倒下了。

%37
古月漠北喘了粗气,也一屁股坐倒在地上,浑身上下都是泥土或者青色的草屑,同时还有淋漓的大汗。

%38
“和活生生的野猪作战,果然和木人傀儡、草人傀儡的练习战完全不同。我花半个小时,才能杀一头野猪,不晓得其他人的情况是什么?”

%39
……

%40
一个隐蔽的小山岗上,临时搭了一个棚子。棚子顶着烈日的照射,顽强地投下一片阴凉的影子。

%41
在棚子下,摆着几张座椅,学堂家老就端坐在主位上。在他的身边,坐着的都是家老。还有几位蛊师,站在他们的身后。

%42
在棚子的周围的树林中,也隐蔽着一些蛊师。

%43
这时,前方的树林忽然一阵异常的骚动。

%44
嗖。

%45
一位蛊师身形如影,从林中破出,快速奔来,然后跪倒在棚子外。

%46
“情况如何了?”学堂家老问他。

%47
“启禀家老大人,目前为止,学员们还没有任何的伤亡。”蛊师连忙答道。

%48
“不错,不错。”

%49
“年中考核已经过了一个上午,居然还没有人受伤。这情形在往年并不多见。”

%50
“看来,还是学堂家老你教导有方啊。”

%51
其他的几位家老听到这话,都满意地点点头,交口称赞。

%52
学堂家老微微摇了摇头,他知道原因。这都是方源横空登场,勒索同窗,使得这届学员都苦练过基础拳脚,因此才有如今的现象。

%53
他看着跪在地上的这名蛊师,继续问道:“那么到如今为止,哪几位的成绩比较不错的?”

%54
那蛊师立即答道:“禀告大人,目前为止,古月方源、方正、漠北、赤城这四人名列前茅。其中赤城已经杀了三头野猪,方正、漠北斩了五头,方源最多,已经有八头的斩获。”

%55
“哦?想不到是这方源暂时领先!”

%56
“历届以来,甲等乙等学员,居然被一个丙等压着,这样的情况还是极少见的。”

%57
“他不是有酒虫吗,也就是说他有达到高阶的青铜真元,能有这样的成绩,也可以理解。”

%58
“我相信接下来,方正、漠北、赤城这三人定会超过他。虽然他有酒虫,但是真元的恢复速度,还远远不是甲等、乙等学员的对手。”

%59
其他家老们议论起来。

%60
“你下去罢。”学堂家老则对跪在地上的蛊师挥手,“叮嘱其他人,一定要做好防护工作。尤其是方正、赤城、漠北这三人的安全,需要严格的注意!”

%61
“是,大人。”蛊师退了下去。

%62
这种野外战斗,对于大多数学员们来讲,真的是第一次。因此具有风险,家族中自然早有安排。数十位二阶蛊师,此刻都隐藏在山林之中,监控着整个考核的安全进行。同时几位三转修为的家老,也都坐镇在这里,随时准备应对突发情况。

%63
火热的太阳从最高空,缓缓地落下,此时已经接近西北的群山。

%64
晚霞是一片片燃烧着的云朵,这是太阳最后流淌出来的热情。

%65
余晖照在山林当中,又一头野猪倒在了地上。

%66
“第二十三头了。”方源心中算计着,然后蹲下身,动作熟稔地将一对獠牙剥离到手。

%67
他的背后背着一个袋子,袋中已经装了不少的野猪牙。

%68
同时,他还有另一个袋子,装着前些时候自己暗自杀猪得到的獠牙。这些獠牙原本存储在石缝秘洞当中,被方源在前天晚上秘密取出,装在袋子里,埋藏在一处隐蔽的地方。

%69
“我熟知地形和山猪的分布情况,同时有高阶真元催动月光蛊,还搭配了小光蛊。其他人猎猪的效率,一定没有我高。单凭我背后的这个袋子,我就能稳获第一。不知道待会,我再拿出另一个袋子,其他人会是一副怎样的表情呢?呵呵。”

%70
方源抬头望了一眼天色,是该去将另一个袋子取出来了。

%71
想到这里,他的脑海中,顿时浮现出一张地图。

%72
这些天,他已经将兽皮地图都记在了脑子里。方源十分清楚自己现在的位置,只要往左拐,顺着一道山溪行进一刻钟,就能到达埋藏袋子的地点。

%73
但是他刚刚要动身,忽然犹豫了一下。

%74
“依我现在的位置,距离最近的一个红圈标志,只有五六百米远。机会难得,我是不是该去那里看看?”

%75
此念一动,就一发不可收拾。

%76
反正已经赢定了,方源还有一段比较充裕的时间。

%77
“那张兽皮地图有三个红圈的标记,这三个地方对王老汉来讲,非常重要。也是我唯一猜测不到含义的地图标记,不知道是什么意思。看看去!”

%78
方源自然知道,考核当中会有蛊师在暗中监控,但是他正需要如此的证人。

%79
当即,他就装作一副要继续猎杀山猪的样子,向山林深处行去。

%80
半刻钟后,他到达了地图上红圈标注的地点。

%81
一座树屋,隐蔽在一棵巨树的繁盛枝叶中,若不仔细观察,还真察觉不出来。

%82
“这是王老汉狩猎时的临时居住点么?”方源皱起眉头,心中不由地升腾起一股疑虑。

%83
他爬进了树屋。

%84
当他看到树屋中的景象时,他脸色骤变!

\end{this_body}

