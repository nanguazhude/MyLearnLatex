\newsection{寻路}    %第一百六十九节:寻路

\begin{this_body}

%1
白凝冰浑身伤痕累累,气喘吁吁,形态狼狈。

%2
当方源赶来时,两个人都楞了一楞。

%3
命运这玩意,真是奇妙。前一刻,两人还是生死仇敌,要将对方置之死地。但这一刻,他们却需要联手,才有生存逃亡的机会。

%4
和白凝冰联手?

%5
方源目光幽幽一闪,心中思量:“白凝冰虽然癫狂,也意识到自己的命运,但并不代表他不想活下去。”

%6
求生是一个人的原始本能,最基本的需求。

%7
事实上,也正是因为白凝冰一方面有着求生的强烈欲望,另一方面又面临着无法改变的毁灭命运,才会形成如此性情。

%8
在这个世界上,绝没有永恒的敌人。和白凝冰联手,大有基础。但要怎么开口,才能说服他?

%9
“呵呵呵,方源,想不到竟然是你!”白凝冰先开口,大笑起来,语气强硬,“那你就陪我一块死吧。能有你一齐陪葬,我这人生结束的也挺有趣。”

%10
“有趣么?”方源心中有了思量,微微笑着,缓步走向白凝冰。

%11
周围电狼袭来,方源甩手,锯齿金蜈呼啸,将两三只电狼当场拍死,击飞出去。

%12
战到如今,锯齿金蜈的两排锯齿,已经损毁大半,切割搅锯的能力大打折扣。只能用拍击。

%13
“在这群狼环伺之下,我们来一场生死激战,不是更有趣吗?”方源缓缓向白凝冰逼近。嘴角勾勒出一抹冷酷的笑意。

%14
白凝冰眼皮子不禁抖了抖,没有想到方源比他更强势。

%15
不过这却和他的心意。如果方源态度软下来,为了生存,一味地要和他合作逃生,他反而会看不起方源,甚至产生一种羞辱感,会忍不住动手想杀了方源。

%16
这世界上,有些人就是这样。你一味地对他和善。他反而觉得你好欺负,看不起你。对他态度强硬,却能得到尊重。

%17
“你真的想死?那我就成全你!”白凝冰眯起双眼,流露出危险的气息。

%18
方源朗声一笑,脚步放缓,以悠然沧桑的语气道:“人生匆匆百年,如梦幻泡影。人活在这个世界上是为了什么?无非是走上一遭。见证精彩罢了。我虽然不想死,但却不畏惧死亡。我已走在路上。纵死不悔。”

%19
这倒是方源的心底话。

%20
人生自古谁无死?

%21
就算是九转蛊师。就算是人祖,也不过只是长生,不是永生,终究也要面临灭亡。

%22
死就死罢,有什么大不了的?就算是下一刻,方源真的死在这狼潮当中,他亦不会后悔。

%23
皆因他已经为自己的目标奋斗过。努力过,一切都按照自己的意愿活过!

%24
把生死放下。人生才见大宽宏,才有真潇洒。

%25
白凝冰闻言。浑身剧震!

%26
他口口声声不怕死,却不是真洒脱,而是看不透,放不下这生死。

%27
当一个人惧怕的时候,他就成了奴隶。

%28
想他白凝冰,不过是生死之下的一奴隶罢了。

%29
但这亦不怪他,他毕竟还太年轻。许多事情,需要经历很多,才能真正看透看破。

%30
然而,方源的这番话,却着实给一直纠结于此的他开了一扇窗。

%31
“见证精彩……已在路上……纵死不悔?”白凝冰口中喃喃,突然问道,“路!什么是路?”

%32
方源冷笑,继续逼近:“个人有个人的路,我的路不必向你说,你的路我怎么能知道?”

%33
这世间,许多人从生到死都没有路,有些人走在路上,不断摸索,在黑暗中走向心中圣地。

%34
白凝冰的天蓝双眸,猛地爆发出一阵夺目的光泽。

%35
“路……不错,我要寻到我的路!”

%36
这一刻,他心中的激动,旁人万难理解。

%37
就像是一位男子,苦苦追寻一位女郎而不得,忽然有一天发现了正确的方法。又像是一位寻宝者,被挡在最后一道关卡很长时间,忽然有一天他发现了能够破关而入的门径。还仿佛是解一道难题,苦苦思索数年没有进展,忽然发现了能解题的正确手段。

%38
白凝冰没有路,寻不到生活的意义,因此他迷茫。

%39
方源不可能直接解除他的迷茫,但却旁敲侧击,给了他一个希望。给他一个面临死亡的排解劝慰——只要在路上,纵死不悔,死亡也会变得不可怕。

%40
“我感到我就要寻找到我的路了!”白凝冰握紧双拳,神情变得振奋无比。

%41
他看向方源,意味深长地道:“我终于明白了你和我的不同。你已在路上,而我却在徘徊。”

%42
“呵呵呵。”他忽然又笑起来,兴奋得近乎狰狞,“方源,你要打我绝对奉陪到底,但现在不行!我们不妨合作,我有电眼蛊,但视线受阻,只能窥探三十步。逃出这里,我们择日再大战一场,和昔日的仇敌通力合作一场,你不觉得这样更精彩,更有趣吗?”

%43
“哦,我如何信你?”

%44
“我没有让你信我。你可以选择相信,也可以选择不信。你可以把后背交给我,但也可以随时出手,偷袭我一招。呵呵,这完全看你当时的心意变化!”白凝冰笑着耸耸肩,竟生出一股洒脱气。

%45
浓烟滚滚,周围群狼嘶吼。

%46
方源微微垂下眼帘,似在思考白凝冰的建议。

%47
其实说服一个人很困难,但也很简单。关键要准确击中这人的心思。

%48
“也好。”方源伸手抚摸着锯齿金蜈的暗金甲壳,抬起眼,“不过你可要做好被我偷袭的准备!”

%49
“呵呵呵。”白凝冰咧开嘴,笑得很邪。一阵气浪袭来。黑烟重重,断臂处的衣袖,在风中飘荡。

%50
在浓烟中,要判断方向,极为不易。视线越狭小,就越容易迷失方向。

%51
但白凝冰有电眼蛊,侦察范围达五十步,如今被浓烟限制。侦察距离就缩小到三十步。但这也比方源的肉眼,好太多了。

%52
不过白凝冰空有电眼蛊,却在大局上,没有清晰的认知。

%53
他只能看到眼前的景象,有时候冲杀着,反而一头撞入狼群的包围网中。

%54
反观方源,他有地听肉耳草。

%55
浓烟能削弱视野。不过却阻挡不了声音的传播。

%56
周围都是声音,地听肉耳草能侦测达两百步。但方源却只能随波逐流。他视野太狭小。只能看清楚身边的一株树,一块山石,没有参照物对比,无法分辨方向。

%57
合作!

%58
白凝冰的电眼蛊,加上方源的地听肉耳草。

%59
两蛊叠加起来,相互辅助,顿时令场面一缓。

%60
“这边是南方。朝这个方向,正对你们古月山寨。”白凝冰双眼电芒一闪。随即道。

%61
“不行,那里狼群太多。得绕道而行。”方源右耳参须飘飘。

%62
“嘿嘿……那就往东南拐过去,如何?”白凝冰舔舔嘴唇。

%63
方源蹲下身子,参须扎根在泥土中,仔细倾听。

%64
期间,电狼冲来,都被白凝冰打发。

%65
方源听了一会儿,站直了身子:“东南方有个缺口,不过得尽快,它快要合拢了!”

%66
“那就冲吧。”白凝冰说着,却没有急着动身。

%67
他还对方源颇有忌惮,不敢在前面冲杀,把后背暴露在方源的面前。

%68
方源冷笑一声,他同样对白凝冰有所顾忌。

%69
最终,两人间距五步,并肩杀过去。

%70
电狼嘶吼,企图围杀他俩。

%71
但靠着电眼蛊和地听肉耳草的搭配,方源和白凝冰二人避实击虚,不断游走,捕捉到良机,再猛地突围。

%72
情报的优势,在此时展露无疑。

%73
白凝冰或者方源两人,单个作战,无不狼狈困窘。但如今一联手,竟然就掌握了主动,变得游刃有余起来。

%74
冲杀了好一阵子,眼前陡然开阔,明亮的阳光照得两人同时眯起了双眼。

%75
“冲出来了!”白凝冰仰头大笑。

%76
方源回望过去,只见身后一团浓重的黑幕,仿佛是黑漆漆的锅底倒盖住一片广袤的山林。

%77
浓烟中不断地传来剧烈的爆破声,怒吼声。显然两位族长还在和狡电狈交战。

%78
“想不到跟你合作,也蛮愉快的。”白凝冰微微侧身,微笑着。

%79
“我也有同感啊。”方源的嘴角也浮现出微笑。

%80
然后下一刻,两人眼中突绽厉芒。

%81
冰刃蛊!

%82
锯齿金蜈!

%83
修长的冰刃,在空中划出一道寒光。

%84
粗壮的金蜈,横扫拍击,带出一股呼啸之风。

%85
砰。

%86
两者相撞在一起,冰刃在金蜈的背上划出一道伤痕,然后崩解碎裂。

%87
方源和白凝冰各向后跳跃一步,双眼中均流露出浓郁的杀意。

%88
短暂的合作,难改敌对之心。

%89
方源黑发飞舞,白凝冰白衣飘飘,彼此之间充满了太多的相似之处。但正因为如此,两人成了天生的宿敌。

%90
黑眸和蓝瞳对视,在空中几乎要碰撞出火星。

%91
双方的杀意却渐渐收敛。

%92
“哼,将死之人,不用我出手,老天就要收掉他的性命。现在最关键的不是白凝冰,而是天元宝莲!一旦狡电狈袭击古月山寨,恐怕凶多吉少。必须趁此时机,果断出手……”方源心中思量,眼帘低垂。

%93
白凝冰的双眼却越来越亮,他口中喃喃:“路……不悔……是了,纵然是人祖也要死亡。人有生必有死,只要过得精彩,死去又何妨?”

%94
念及于此,他眼中骤然爆发出耀眼的光辉。

%95
“哈哈哈。我也找到了我的路,那就是见证这世间精彩!方源,我们择日再战。到那时希望你的死,能给我的人生带来精彩!”

%96
说完,他连连后跃,拉开距离之后,转身就走。

%97
他虽然狼狈不堪,浑身是伤,脸色黑灰满布,独臂残疾。但他腰杆挺拔如剑,他不再迷茫了。

%98
他已寻得了他的路。

%99
换句话讲,他真正成了他自己!

\end{this_body}

