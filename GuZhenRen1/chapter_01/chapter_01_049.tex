\newsection{不愁方源脱离掌控}    %第四十九节:不愁方源脱离掌控

\begin{this_body}

%1
一对深沉的目光,遥遥地注视着演武场。

%2
学堂家老站在三楼的窗口,将演武场中发生的一切,都看在眼中。

%3
他深深地皱起了眉头。

%4
方源主动下场的一瞬间,他的心中也是不由地诧异了一下。他也没有料到方源会这么干。

%5
“这个小子,有些滑溜啊。本身精通学堂规定,平时又没有犯错。课上虽然常常睡觉,但是一旦被提问,回答总是面面俱到,让人挑不出错误来。想抓住他的把柄,把他的气势压下去,真是有点难办。”

%6
不由地,学堂家老心中就对方源,产生了一股淡淡的厌恶。

%7
作为师长,自然是喜欢听话乖巧,又聪明的学子。讨厌那种不守规矩,调皮捣蛋的坏孩子。

%8
但学堂家老执掌学堂多年,经验丰富至极,看到无数的学员。其中有乖巧得不得了,唯命是从的。也有一天到晚单闯祸,总是违反纪律的。

%9
他早已经做到心如止水,一视同仁。并且同时,他将“为人师表,有教无类”八字,刻在自己书桌的右手边角,作为座右铭。

%10
他从未对一个学生,产生如此的厌恶。

%11
觉察到心中的这股厌恶之情,学堂家老也有一些吃惊。

%12
在往届,哪怕是最调皮的学子,他都能宽怀容忍,怎么面对方源,却失去了这份平衬?

%13
他仔细品味,反思了片刻,终于发现原因。

%14
方源这个小子,骨子里藏着一种傲慢!

%15
似乎是从根本上,他就不把师长放在眼里。刚刚拳脚教头的话,他不但不听,还公然反驳。

%16
其实当众反驳师长的现象,往届也常常发生。但是那些孩子,总是有激动的情绪。不是逆反心理,就是愤怒,桀骜等等。

%17
学堂家老清楚得很,情绪越激动的少年,就意味着他们心中越恐惧。

%18
但是方源没有。

%19
他心中没有一丁点的恐惧,似乎已经看透了学堂的把戏。

%20
他一脸的冷漠,就算是下了场,脸上的表情也没有丝毫的变化。仿佛做了一件微不足道的小事。

%21
是的,他把违抗师长的事情,当做了一件微不足道的小事!

%22
简而言之——

%23
他不怕。

%24
就是这点,让学堂家老都感到了不爽,产生了厌恶之情!

%25
学堂家老可以容忍比方源更刺头,更调皮十倍的少年,那是因为这些学子,都知道怕,都是被激动的情绪支配着。

%26
只要怕,只要冲动,就容易操纵,不会脱离掌控。

%27
但是方源不是。

%28
他冷静,不动声色,不把师长放在眼中。

%29
他不敬畏!

%30
不对家族产生敬畏的人,就算是培养出来,怎么能够为家族所用?

%31
“这样的人,一旦出现,就得镇压,必须镇压!否则,他的存在,就会给其他学员的心中埋下反抗的种子”间一长,感染了其他人,都不敬畏师长。那么学堂,还怎么管理学员呢?”

%32
学堂家老双眼眯起来,在心中下定了决心。

%33
但他很快,脸上又浮现出一股愁容。

%34
怎么镇压方源呢?

%35
方源没有犯错,找不到一丁点的把柄啊。

%36
方源的滑溜,让他有一种无从下手的无奈感觉。他从未遇到这样的一个学生,对学堂的各项规定,都如此熟悉通晓。

%37
作为学堂家老,向来都以公平公正的面目示人,不能像个地痞流氓,故意在方源这个少年的身上找茬。

%38
他本来寄希于拳脚教头,但他现在深深的失望了。

%39
“看起来,要压住方源的风头,只有等到其他学员率先晋升到一转中阶了。”

%40
蛊师要晋升,最大的影响因素就是资质。

%41
学堂家老凭借着丰富的经验,心中早已估算过:真正有希望率先晋升的,只有古月方正、赤城还有漠北这三人。

%42
他们一个是甲等,两个是乙等,身后有长辈的资助,不缺元石。不管哪一个,都有可能成为本届的第一个一转中阶。

%43
“古月方正、赤城还有漠北这三人,才是此届的种子啊。”学堂家老望着演武场,感叹一声。

%44
以他老辣的眼光,已经发现:在演武场上,学员们看似随意地站立着,实际上已经隐隐化为了三个圈子。

%45
一个圈子里,古月赤城正捂着肩膀,周围有一群同龄人簇拥着他。

%46
第二个圈子的中心,是古月方正族长一系的后辈,都隐隐依附着这个甲等天才。

%47
第三个圈子,则是古月漠北。他已经被治好了内伤,此时站立在场中,脸色苍白周边的同窗在向他嘘寒问暖。

%48
“这就是放纵他们争斗的意义啊。”看着这三个圈子,学堂家老满怀欣慰地笑了。

%49
放纵学员争斗,不仅是培养他们的战斗意识,更重要的是提前筛选出领袖人物。

%50
往届的话,必须要到下半年,才能分出小圈子。但是本届,因为方源横空出世,抢劫勒索,导致争斗大大提前。

%51
面对方源,一直敢于正面对抗挑战的,只有方正、漠北以及赤城。

%52
天长日久,潜移默化之后,其他的少年们就自然而然地以这三人为首了。

%53
只要不出意外,这三个圈子,就是未来的家族高层的格局缩影。

%54
“不过这圈子现在还不稳固,圈子之间的人员还在流通。等他们三人率先晋升中阶,安上班头、副班头的职务,区分出来,有了权势,这就初步巩固了圈子了。”学堂家老心道。

%55
当然,也有人不在这三个圈子里。

%56
只有一个,那就是方源。

%57
依附强者是人的本性实际上,虽然方源勒索众学员,站在了所有人的对立面上。但也不是没有少年,主动想要依附他。

%58
然而这些人,都被方源拒绝了。对他来讲,有用的才是棋子,这些同龄人的利用价值太低了。

%59
这正是学堂家老厌恶方源的另一个原因。方源他太孤僻,不融入这个集体。对于这样的人物,家族对他的掌控力,就没有对其他少年那么多。

%60
学堂家老的目光,再一次投向场中的方源。

%61
方源孤独地站在一处,背负双手,眼皮微合,任凭场上少年们为了争夺奖励,打得一片火热,面色上都没有丝毫波动。

%62
他的周围是一片空白,没有少年愿意和他站一起。

%63
很明显,他也不想这些人,和他站在一起。

%64
他一个人站着,孤独包裹着他。

%65
他游离在集体之外。

%66
“不过,也不用太过,方源年纪还小,可以慢慢调教。”学堂家老双目精光一闪,暗忖着。

%67
“接下来就是设立班头、副班头。一年之后,又要分小组,设立组长,副组长。每个学年,还有各种荣耀和奖励,诸如小红花奖,蓝领巾奖,五好学员奖。他要修行,就需要资源,总得会竞争这些职位和奖励。日子久了,相处下来,他总会有亲情、友情、爱情的羁绊。不愁他今后脱离家族的掌控。”

%68
这些年,学堂家老也渐渐地看明白了。

%69
新的家族成员一出生,家族就会给他们洗脑。

%70
先是灌输家族至上的价值观。然后给他们普及道德观,让他们意识到亲情、友情、爱情的美好和重要。

%71
还给他们灌注荣誉观,在成长过程中,又不断地利用物质奖励,来吸引他们。利用家族设立的权位,来筛选和招揽最忠心的族人。

%72
别看班头、副班头这些职位太小,一旦成为了其一,就是纳入了家族的体制。

%73
在这种体制下,耳濡目染。一边是权势的好处、甜头,一边是脱离体制的坏处。一手萝卜,一手大棒,谁能脱离得了?

%74
再桀骜不驯的人,再孤僻的人,也要一步步地同化成家族的一员。没有忠心,也会培养出忠心。没有亲情友情爱情,也会牵扯出来。

%75
这就是体制的威能。

%76
这就是规矩的力量。

%77
这就是家族的生存之道!

\end{this_body}

