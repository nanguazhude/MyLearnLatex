\newsection{我不需要理解}    %第一百二十四节:我不需要理解

\begin{this_body}

“原来如此。”方源摸摸下巴,却又摇摇头道,“我这酒虫是不卖的。既然药姬执意要收购,不妨去找那位买了酒虫的蛊师去。”

听了此话,青书顿时一朵愁云罩脸,他深深叹息道:“唉,也不知道是哪个族人,收购了这只酒虫,一直秘而不宣。我们是调查不出来的,总不能随意探索其他人的空窍吧?这是侵犯**,犯忌讳的事情。为了酒虫这么一件小事,惹了众怒可不好。不过也可以理解这个族人,有了宝物遮掩起来,也是人之常情。”

青书并不知道,真正的收购者此时就坐在他的面前。

但青书从未怀疑过方源,在他的想法中,方源有了酒虫,不可能购买第二只酒虫。买来有什么用呢?

若他知道合炼四味酒虫的秘方,那么他绝对会首先怀疑方源。但就现在而言,四味酒虫的秘方也就方源独一份。

真正知道方源是购买者的人,是贾富。但在贾富看来,方源完全可能是给别人代买的。可能是为了亲人,可能是为了情人,这无可厚非。药姬不也是这种情况,为了孙女而购买酒虫的吗?

“不管怎么说,酒虫的事情,我是不会让步的。”方源态度坚决,没有松口,心中冷笑。

这就是体制了。

一方面体制强大,另一方面体制也是一种束缚。

古月药姬是三转的蛊师,明明比方源强大,但她碍于体制,却不好明抢。碍于体制中的规矩,顾及自己的名声,也不能强购。

一旁。方正忽然开口劝说道:“哥哥。酒虫你也没用了,何必守着它呢。药姬奶奶很慈祥,药乐妹妹我见过好几次面。人很好的,她一定会善待酒虫的。而且酒虫对于她来讲,也会很有帮助。帮助他人是快乐之本。哥哥你拯救了村子。我实在为你感到高兴。这令我也面上生光。但你这次何必守着不放,这样做,未免太小气了些。”

方源脸色顿时一板,冷声道:“我的好弟弟,你这是在教训我吗?酒虫是我的事情,哪怕烂在手心里,也容不得你来指手画脚。”

他倒不至于真的生气,只是态度是心的面具,表明了这个态度。更能让青书明白他拒绝的决心。

“看来方源是铁了心拒绝啊。这次故意带着方正前来,是打错了算盘。这兄弟俩的关系一直不和睦,弄巧成拙了。”古月青书目光一闪。

“方正。你先出去逛逛罢。”方正还想说什么。但被青书伸手阻挡。

方正咬咬牙,终究是听了青书的命令。

“总之这事情我不能理解。哥哥。”他打开门,留下这句话。

“我做事情,还不需要你的理解,方正。”方源面无表情。

方正开门的动作一顿,然后头也不回,快步走出房间,砰的一声关上房门。

这个动作令房间中的气氛更加尴尬。

“若是没有其他的事情,还是请青书兄回吧。”方源直接下了逐客令。

“呵呵呵。”青书干笑了两声,企图缓解气氛,奈何方源脸色冷淡如冰,并没有变化。

不过他秉性温和,也不着恼。

摸摸鼻子,青书尴尬地笑起来:“的确还有一件事情。是关于九叶生机草”

“九叶生机草不卖。”方源翻了一下白眼。

“我知道,我知道。”青书连连点头,“是生机叶的事情,这是我个人小组的私事。希望你能将生产出来的生机叶供应给我们。当然我们会给你一定的补偿的。”

有生意上门,方源当然并不排斥:“那好,每片生机叶六十五块元石。”

青书闻言,顿时咂舌。

按照市价,家族中原本的卖价是五十五块元石。生机叶虽然是一转消耗蛊,用一次就没了的,但是这保命的东西,谁也不嫌多呀。

而且狼潮将近,家族也提价,一片生机叶卖到了六十块。其实不止生机叶,很多东西的物价都在上涨,这是局势动荡,蛊师们也只好捏着鼻子认了。

但青书没有想到,方源的出价竟然比家族的价格还要高。

“嫌我这价格贵?可以不买嘛。但据我所知,狼潮之后,家族管控物资,生机叶供不应求。到时候,这方面的价格还会上涨,你想要还未必有货呢。你说呢?”方源语气笃定,稳坐钓鱼台。

青书一噎,语气有些无奈:“你倒是看得通透。只是你这样提价,未免有些过分。就不怕得罪人吗?你若降下价格,就能借势搭建人脉。但你如此提价,反而会让族人们怨恨你这样子发财。”

方源仰头哈哈一笑:“狼潮将近,我这等小人物,说不定哪天就死了。哪里顾得上这些细枝末节呢?”

“你早就不是小人物了。人际关系更加不是细枝末节。”青书深深地看了方源一眼,然后轻轻地摇摇头,“不过,个人有个人的想法和选择,我也不强求。只是你要多注意一些,药姬大人那边可不会如此善罢甘休……告辞了。”

青书再不提购买生机叶的事情,方源的价格吓到了他。

他是个聪明人,聪明人购买东西,不是靠购物的冲动**,而是理智。聪明人心中都会有个心理的价位,超过这个价位,他们就会冷静地收手。

方源看似只涨了五块元石,但是青书要购买的绝不止一片生机叶。狼潮至少要持续整整一年,消耗的生机叶的数量会很大,这样积累起来,就很多了。

“谢谢你的提醒,慢走不送。”方源看着青书离去,心中一片清明,知道青书必定回来。

皆因他大大低估了此次狼潮的严重程度。

在这样的狼潮当中,死亡会随时降临在每一个人的身上,生机叶是不愁卖的,在方源的记忆中。甚至炒到过一百块的高价!

当然。这价格也是在狼潮最猛烈的时候。现在方源要做的,就是顺应时局,徐徐增价。

随着时间流逝。冬风越加寒冷。

今年的冬天对于青茅山的三家山寨来讲,显得比往年更冷一些。

就拿古月山寨来讲。

越来越多的残狼,出现在山寨周围。

家族中发布大量的任务。几乎清一色的都是剿灭残狼的内容。

到了十二月,残狼群的数量达到最高峰,这令局势直转而下,达到恶劣的程度。甚至出现了,山脚的一座村庄,被一大股的狼群屠尽的情况。

好在驻扎在村子里的十多位蛊师,及时主动的撤退了。这令家族高层暗松了一口气,死一位蛊师,他们都要心疼半天。至于凡人。都是奴仆,死就死吧。

这个世界上,可没有什么人权可讲。

一个蛊师的命。比一百个凡人的命都要高贵。这是所有人共同的观点。

只是伤亡是绝对少不了的,因此山寨中一到晚上。就常有隐隐的哀哭和悲泣。

山寨中弥漫着一股悲伤和压抑。

残狼群还只是前奏,真正的狼潮要更加可怕。

越来越多的人意识到,今年的狼潮可能非同以往。

在这样的压力下,僵持的谈判得到了迅速的进展,三寨联盟促成了。

一月,冬末。

会盟坡,三寨会盟。

小雪如白色绒毛,慢慢地飘零而下。

成百上千的蛊师,汇聚在这里,形成一股巨大规模。

会盟坡本是一个普普通通的山坡。但在历史上,古月一族的二代族长就在此处促成了首次的三寨联盟。从此以后,接下来的所有联盟,都在此地进行。

历经多年的改造,如今的会盟坡已经比原本面貌大了数十倍,简直就是一个巨大的广场。

广场中靠着山壁的那边,竖立着高大的巨石。

巨石台上,有石头雕刻而成的楼台,楼台中石桌石椅俱全,三家的高层此时正坐在里面,密切地商谈。

巨石楼台下,三寨的蛊师泾渭分明地站着,形成三个庞大的团体。

蛊师们都穿着相似的武服,带着腰带,各自的修为一目了然。方源亦在其中。

他暗暗打量观察。

台下的这些蛊师,都是二转蛊师。一转蛊师大多都负责后勤,二转蛊师才是真正的主流部队。至于三转蛊师,皆是家老。像古月山寨这样的中型家族,每一代辛辛苦苦地积累下来,也就那么二十几个家老。

方源在观察,其他的蛊师也在观察。无数的目光相互交杂,有仇恨,有好奇,有警惕。

古月一族的蛊师,普遍身形偏瘦。熊家寨的蛊师,则几乎都很壮实,从外形看上去就知道气力很大。白家寨的人,可能是处于后山瀑布,肌肤都偏白,带着一股阴气。

“那个长头发的就是古月青书,二转巅峰,是古月一族的二转第一人。”有人指指点点。

“漠颜!哼,就是你这贱人,断去我的一只手臂。狼潮之后,我一定要讨回来!!”有人心中冷哼。

“哇,那个骑在巨熊身上的女子,应该就是熊家寨的熊骄嫚了。不要被她可爱的外表迷惑了,她战斗起来十分疯狂。”

“看到那个白白的女胖子了吗?那是白家的白重水,二转高阶的强者,拥有水豚蛊,为人很浪,极好男色。你们新人都要小心,不要被她给推倒了。”

……

三家山寨的恩怨情仇,早已经深厚无比。

会盟坡上议论声越来越大,噪杂一片。很多蛊师前辈都在指指点点,为新人后辈介绍其他两家的强者,提醒他们今后要小心注意。

\end{this_body}

