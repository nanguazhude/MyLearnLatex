\newsection{牺牲的觉悟}    %第一百七十六节:牺牲的觉悟

\begin{this_body}



%1
“方源家老,你残杀甚老汉一家的事情是属实的么?”主位上,古月博沉声问道。

%2
所有的视线都集中在方源的身上,少年家老冷笑一声:“的确是属实的。”

%3
古月方正心情沉痛地闭上双眼。

%4
他虽然斩杀了许多电狼,但从未伤及人命。在他听到方源亲口承认的这一刻,他忽然觉得自己的亲哥哥是如此的陌生和疏远。

%5
这种疏远,有夹杂着狠辣手段的畏惧,还有自己被无辜殃及的愤怒。

%6
“方源,你这样滥杀无辜,就没有什么愧,疚之情吗?或许你有些苦衷,你可以说出来。”铁若男紧皱着眉头道。充满正义感的她,对方源这样的人最为厌恶。

%7
“杀了便杀了。我的故事,并不需要倾述。但是我事先并不知道,失踪的王二会是一位魔道蛊师。也算计不到,方正会遭受牵连。”方源实话实说道。

%8
“哥哥,你就不打算对我说些什么吗?”方正睁开双眼,眼眶通红。

%9
“要我说什么?安慰你的话,还是表达歉意……”哼,弟弟,你还是太嫩了。”方源冷笑。

%10
“可恶啊,哥哥……别以为你晋升了家老就了不起了,告诉你,我已经有了冲击三转的资格……”方正咬牙切齿,双拳都捏出青筋。

%11
“够了!”古月博再看不下去,低喝道,“方正你退下,在此大呼小叫算什么体统?”

%12
他说话另有所指,隐含不满。铁若男没有品味出来,但神捕铁血冷则第一时间反应过来,上前一步,拱手道:“古月族长,以及诸位家老,小女擅闯贵族的议事堂,的确是欠妥。冒犯了诸位,在下致歉!”

%13
家老们连忙站起身来,连说不敢当。

%14
古月博面色亦转缓。

%15
但铁血冷接着道:“然而这位方源小兄弟,因为涉及到贾金生一案,身上还有嫌疑。在下希望他能留在山寨,尽量不出使外寨。

%16
古月博揉了揉太阳穴,叹了一口气:“我们古月一族也十分希望给贾富大人一个交代,既然是神捕所求,那方源家老在没有洗清嫌疑之前,就只能委屈你一点了。希望你能理解。”

%17
古月博看向方源,表情诚恳,目光却深沉。

%18
方源虽然杀了王老汉一家,但他们不过是些凡人,蛊师杀凡人有什么错?尤其这蛊师,还是家族的家老。因此没有任何的惩处。

%19
“是。”方源看了一眼古月博,面无表情地答道。

%20
“可恶,又是这样!”铁若男一拳捣在树干上,震得树叶飞落。

%21
她愤恨不已,咬紧牙关:“明明是杀了人,却视若无睹,没有任何的表示。父亲,难道说凡人就不是人吗?为什么蛊师杀凡人,都觉得天经地义呢?”

%22
铁血冷在一旁沉默,如雕塑一般。

%23
天气有些阴沉,风阵阵吹来,山林树叶沙沙作响。

%24
铁若男忽的垂首,神情颓然:“对不起,父亲。”

%25
她道歉道;“我没有听从您的嘱咐,还是动用了直觉蛊。”

%26
“唉……。”神捕长叹一口气,深沉的目光落在少女身上,“你这孩子,如此嫉恶如仇,正义感十足,就像是年轻时的我。做父亲的为此欣慰,又感到担心啊。”

%27
“担心什么?”

%28
“你的理想,比我当初还要大。年轻时的我,志向是逮捕天下罪犯,将镇魔塔填满。而你呢,却要使人人平等,使蛊师和凡人都一视同仁,使理法遍行天下。这样的志向和理想,太大了,也太沉重了。”铁血冷的话中透着沧桑。

%29
“但是父亲,所谓的法,公道,正义,如果不一视同仁,那正义和法律还有什么意义呢?年轻人没有一些略显狂妄的理想,又怎么算得上是年轻人呢?我相信事在人为。只要我努力,未必不能实现!”铁若男语气激动,双眼中满怀幢憬之光。

%30
铁血冷沉默下来,半晌才道:“总有一天你会明白的,若男。不过也好,年轻人的路,应该自己来走。挫折将引来成熟。父亲老啦,不会再干涉你了。只望你能过自己想要的生活!”

%31
说着,他从怀中掏出一封信,递给了铁若男。

%32
“这是?!”铁若男拆信一看,顿时喜出望外。

%33
这信是贾富的回复函,记录着关于贾金生一案的所有信息,包括昔日方源赌石,拥有酒虫,如何用竹君子蛊审问方源,甚至还有第二次行商,方源出价,他在交易中表现出来的才情,受到贾富招揽的事实。

%34
铁若男将信中关于“方源出价”的内容,反复看了三遍,双目中精光灼灼。

%35
“这个方源不仅手段狠辣,心性残酷,更有智计谋算。直觉告诉我,他有很大的嫌疑。如果他真的是凶手,那他真的有些恐怖。他撒谎却没有令竹君子蛊变色,这是何等手段?”铁若男自言自语。

%36
“你接下来准备怎么做?”铁血冷问道。

%37
“贾金生的案件,时隔很久,处处透着蹊跷和神秘。到目前为止,贾金生的尸体都没有出现,甚至连案发现场都不能推断。这件案子太干净了,让我无处着手。只有一个最大的嫌疑人方源。尽管王二的线索也断了,也没有任何的证据指证他就是凶手。但是在没有其他目标的情况下,我只能凭直觉先调查这个方源了!”铁若男干劲十足地答道。

%38
“你觉得方源身上有疑点?”铁血冷问。

%39
“疑点重重!”铁若男立即道,“这个方源明明只有丙等资质,为什么修行速度比方正还快?虽说有酒虫,还有舍利蛊相助,但这速度也过快了一些。古月一族也许没察觉什么不妥,但旁观者清,这就是一项疑点。”

%40
“除此之外,还有第二个疑点。那就是他的运气。生平第一次赌石,买了六颗紫金石一下子就开出了两只活蛊,一只是癞土蛤蟆,一只是酒虫。这个运气也太好了点吧?”

%41
铁血冷点头:“嗯,接着说下去。”

%42
“这个方源,看似普通但深究起来就会发现他神秘得如山间浓雾。许多地方不经意间显露出来的东西,都要让人琢磨很久。就比方说他解石的手法吧。

%43
居然直接用月光蛊解石,而不损石心分毫。这样精微控制的手法,放在一个还在学堂中的学员真的不可想象”等等!”

%44
铁若男顿住,这一刻她发现了什么,眼中精光骤盛,如鹰阜般锐利!

%45
她将目光集中在信中的一行字上,久久凝视,目光越来越亮。

%46
“我发现破绽了。这个方源大有问题!”半晌后,她猛地抬起头来兴奋地叫道。

%47
杯中茶水的香气,随着热气蒸腾而出,弥散在这处书房之中。

%48
方源好整以暇地端起茶杯,吹了吹浮起的茶叶,喝了一口热茶,然后悠然地吐出一口浊气。

%49
看着面前姿态悠闲的方源,古月漠尘辛苦地忍着感到自己的额头上青筋都在跳。

%50
先前方源狮子大开口,他气得将方源当场赶了出去。

%51
但今天,他又不得不将方源重新邀请过来。

%52
皆因形势比人强。药脉处处紧逼,而他落到二转的事实也迟早要暴露漠脉岌岌可危,急需一位家老来克当上门女婿,稳住人心就是稳住了自身阵脚。

%53
“但方源这小兔崽子也太可恶了,漫天要价,真的以为我漠脉日进斗金吗?”古月漠尘一边在心中狠狠地咒骂着,一边脸上堆起和煦的微笑,以商量的口吻道,“方源家老,你这价格实在太狠了点,大大超出了我漠脉的承受底线。不能再降低一点吗?”

%54
方源看了古月漠尘一眼,面前这老人能屈能伸,倒值得敬佩。

%55
其实他现在的状况也越来越糟糕了。

%56
铁家父女正将他逼入绝境。一旦被发现他就是杀害贾金生的凶手,那么古月一族必定要将他交出去,来平息贾家的愤怒,同时换来贾家商队每年的贸易。

%57
火候也差不多了,方源口气松动下来:“那就再降三成罢。但是有一个条件,必须先预付四万元石,同时还要有一只生铁蛊,一株往生草蛊。这样才能显出你的诚意。”

%58
古月漠尘听了,忍不住又揉了揉眉头,沉声道:“生铁蛊可以先给你,但往生革蛊,我漠脉库存中真的没有。四万元石一时间也拿不出全部,只能分期给你。”

%59
方源知道古月漠尘这老狐狸没有说实话,但他却也明白再咄咄逼人,反而会适得其反。

%60
“也好。等到你先兑现了这些,再谈婚约吧。在此之前,我会在议会上有所表态的。”方源留下这句话后,便离开了此处。

%61
书房中陷入一阵沉默。

%62
半晌后,古月漠尘蓦地开口:“你出来吧。”

%63
暗门被推开,一位少女已经哭红了双眼,带着泪痕走了出来。

%64
“爷爷。”她行礼道,正是古月漠颜。

%65
古月漠尘长叹一口气:“家族的情况你也相当的清楚,漠颜,家族需要你的牺牲,你能理解吗?”

%66
“是。”少女哽咽着,低下了头。

%67
不管哪个世界,都没有白吃的午餐。就连方源担当家老,也为家族贡献了许多力量。只是付出和得到的具体情况,因人而异罢了。

%68
而对于这些含着金汤匙出生的少年,同样也没有无偿的权利,他们享受家族的培养,就要有奉献牺牲的觉悟。

%69
而这点觉悟,古月漠颜早就有了。

%70
哪怕她对方源没有一丁点的感情,甚至厌恶到憎恨。但她知道,为了家族,她必须嫁给他为妻!

\end{this_body}

