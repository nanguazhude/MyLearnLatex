\newsection{鹤灾}    %第一百九十一节:鹤灾

\begin{this_body}

%1
碧蓝的高空中,浮云朵朵。

%2
蛊师老者骑在白鹤之上,凌厉如刀锋的白眉之下,双眸透射出深沉的杀机。

%3
“呵呵呵,这份仇,就从你的子孙后辈身上,开始算吧。”他笑起来,俯视下方的战场,伸出枯瘦如柴的手指,往下轻轻一指。

%4
座下的白鹤,顿时仰起修长优雅的脖颈,发出一声嘹亮悠长的鸣叫。

%5
声音在广阔的天宇中扩散,余音袅袅中,无数的应和声传来。

%6
“这是什么声音?”方源正催动着地听肉耳草,此刻率先听到,顿时心中一惊。

%7
鹤唳声绵延不绝,此起彼伏,气势磅礴恢弘。这不是一两百只鹤群,也不是两三千只飞鹤一起鸣叫,至少有一万头的飞鹤,才能形成这样的效果。

%8
“难道有什么鹤群在迁徙吗?”莫名的,方源感到一阵极度的不妙。

%9
鹤唳声也吸引了场中众位蛊师的目光,纷纷仰头望去。

%10
“天空中那是什么?”

%11
“听声音,应该是有大型的飞禽族群在迁徙。告诫所有蛊师,不要胡乱出手,招惹麻烦!”白家族长正说着话,忽然声音一滞。

%12
他的眼眶慢慢撑大,就看到天空中出现一只、两只、三只……成千上万只的飞鹤,密密麻麻,向这块地面俯冲杀来。

%13
“怎么会这样?”

%14
“赶紧戒备,防御!”

%15
“还是快逃吧,这飞鹤有上万只。必定有万兽王。”

%16
“狼潮刚去,又有鹤灾吗?老天爷。我青茅山真是多灾多难啊……”

%17
蛊师们一片哗然,无不心中震动,斗志动摇。

%18
好不容易将狼潮抵抗过去,现在却出现了鹤灾。青茅山三大家族,俱都伤亡惨重,哪里有力量来对付这样庞大的鹤群?

%19
飞鹤收缩翅膀,如漫天箭雨一般,暴射而下。

%20
爆喝声。惊惶声,惨叫声一齐爆发,各色光芒涌动,月刃、水弹、铁刺等等反射苍穹。

%21
一阵短暂而激烈的抵抗之后,蛊师们死伤大半。

%22
这飞鹤长喙如铁锥,翅膀每一次拍击都有野猪冲撞之力,足爪尖锐能裂石。普通的飞鹤就很难缠。更何况鹤群当中,还有大量的百兽王级的飞鹤,千兽王级的飞鹤也不在少数。

%23
家族抵御狼潮,拥有历史积累下来的丰富经验,更关键是有坚硬的山寨可以依托防守。但在此处,山野空旷。哪里有什么防御建筑?

%24
几乎第一波攻击,蛊师就减员了一半。

%25
飞鹤长喙刺穿心脏,鹤爪抓破头颅,鹤翅一拍,人就吐着血。远远地抛飞出去,浑身骨骼碎裂。

%26
方源也遭到攻击。他双眼皆是白茫茫的一片,只有靠地听肉耳草,来躲避攻击。

%27
“方源,支撑住!”这时,他的身后传来古月博的喊声。

%28
方源感到很纳闷。

%29
这古月博是怎么了,刚刚叫喊自己的时候,语气就不对劲,竟然有一种维护自己的意思。现在又特意赶过来,支援自己。

%30
方源虽是老谋深算,但也做不到料事如神。这般仓促间,哪里能想得到铁若男会将自己认成十绝体呢。

%31
古月博乃四转强者,围攻方源的飞鹤却是普通猛禽,轻易间就被古月博击杀或者驱散。

%32
“方源,是你吗?”古月博来到迷津雾外。

%33
方源脑海中思绪电转:如今处境危险至极,依托在古月博的身边,可极大地增加存活几率。便当即回答道:“是我。”

%34
古月博听出是方源的声音,顿时心中一块巨石落下:“很好!方源,曾经的事情我们就不提了,不管如何,家族都会护你周全的。我们回山寨,我护着你撤退!”

%35
他却不知,山寨对方源来讲,更是龙潭虎穴。

%36
但鹤灾和山寨相比,前者近在咫尺,不逃就死亡。后者还稍远一些,不到火烧眉毛的程度。

%37
方源叹息一声,也不犹豫:“族长请带路,我尽量跟上!”

%38
但就在这时,一只巨鹤从天而降,白眉老蛊师端坐在鹤背上,声音冰寒:“谁都逃不了,都给老夫死在这里罢。”

%39
方源看不见,却听到身边的古月博惊呼一声:“五转蛊师!”

%40
显然,古月族长有侦察手段,能判断陌生人的修为。

%41
方源不由地心中一震:怎么又出现了一位五转强者?小小的青茅山,也不是什么名山大川,钟灵毓秀之地。怎么一个个的五转强者,接连出现呢?

%42
“难道说,和古月一代有关?”方源脑海中闪过一道灵光。

%43
他心中砰然一动!

%44
若是寻常的鹤灾,他已经没有机会。野生的飞禽难以利用,他自身修为虽然高达三转巅峰,但对比五转,却实在有限,难以破局。

%45
但现在却有一位五转蛊师,给他带来无限危机的同时,又带来一丝破局的希望。

%46
如今青茅山这局,三位五转蛊师是关键中的关键,其他人皆是陪衬。

%47
只有五转蛊师,才能对付五转蛊师。

%48
一瞬间,方源心中就下了决议。

%49
是时候了,必须赌上这一把!

%50
“族长大人,一代先祖已经在地底复苏。我们回到山寨就安全了!”方源开口道。

%51
“什么?”耳边顿时传来古月博的惊叫声。

%52
他的震惊反而让方源心中一定。

%53
“这种事情,我怎么可能乱说。只有去了山寨,才能保住一命。”方源接着道。

%54
古月博也是有决断之人,当即拽着方源,向山寨方向飞奔而去。

%55
但飞鹤不断地飞来,阻挡在路上。百兽王级,千兽王级。接连涌现。

%56
古月博浴血奋战,力护方源。渐渐举步维艰,陷入到飞鹤的重重包围当中。方源被古月博护住,倒暂时安全得很。

%57
时间一到,迷津雾自行消散了。

%58
方源扫视战场,只见战场中横尸遍野,极为惨烈。蛊师牺牲巨大,但鹤群也折损了许多,除了人的碎肢断臂之外。就是黑白相间的鹤尸。

%59
“这不是铁喙飞鹤吗?”方源心中诧异。

%60
别人认不出来,皆因这飞鹤并非南疆本土飞禽。但他却知道,这铁喙飞鹤乃是源自中洲。

%61
“嗯?万兽王,五转强者!”旋即,方源看到半空中,巨鹤缓缓浮动双翅,漂浮着。在它的背上。坐着一位白眉白发的冷酷老者。

%62
方源将目光收回,再看身边的古月博。

%63
这位古月族长,已经浑身是伤,满身是血,拼死奋战。很多次明明可以躲闪,但是为了方源的安危。宁愿自己硬抗而受伤。

%64
“族长!如今局面不妙至极,蛊师们各自奋战,被飞鹤切成各个小块,迟早要被吞并。我们只有集合他们的力量,拧成一股。才有望冲出重围,回到山寨!”方源对古月博道。

%65
“你说的有理。”古月博目光重重一扫战场。随后高声呼喊,“诸位,大敌当前,我古月山寨中却有制敌的手段。速速来与我汇合,一起冲杀出去!”

%66
声音在战场中回荡,顿时吸引了无数的目光。

%67
“什么?古月家还有制服五转蛊师的底牌?”

%68
“宁可信其有不可信其无啊!”

%69
“兄弟们,冲出去,和古月族长汇合!!”

%70
蛊师们本来已经绝望,此时却从古月博的话中,看到了一丝希望。

%71
在死亡的威胁下,这些曾经敌对的蛊师们联手起来,很快就汇合到了一处。

%72
“古月家……呵呵。都是师兄你的后人呐。”巨鹤上,白眉老者冷笑,正要指挥鹤群拦截,但转念一想,却止住了这个打算。

%73
“不妨就让这些人逃回去,更方便一网打尽。这些都是他的后人,待会激斗之时,也能让他投鼠忌器一些。但是这三个四转蛊师,却有干扰战局的能力,不能留着,先杀了再说!”

%74
想到这里,白眉老者怪啸一声,屈指一弹,三道白色光圈飞射而出。

%75
“这是什么蛊?”熊家族长首先中招,被这白色光圈一罩,整个人速度暴降,堪比蜗牛爬路。

%76
其余两位族长,亦是如此。

%77
“方源,你快走。古月族人听命,誓死保护方源的安危,只有他知道那个手段!”古月博尝试了数种手段,都解开不了这光圈,只得大叫一声,反身直面白眉老者。

%78
方源回首,深深地看了一眼这位古月当代族长。

%79
“方源家老,我们来护你!”立即就有一大批的古月族人,汇集到方源的身边,将他牢牢护住。

%80
治疗的光波,以及增加速度的旋风都加持在方源的身上。

%81
身后传来轰鸣声,在玄奇而残酷的命运下,原本看彼此都不顺眼的三位族长,此刻却紧密地团结在一起,和神秘老者展开生死大战。

%82
这场战斗的结果,没有悬念。

%83
三位族长接连战死,白眉老者拂拂衣袖,稳坐巨鹤之上。飞鹤大军漫天盖地,缓缓地向古月山寨压去。

%84
古月山寨中一片混乱,传来凄凄的哭声。

%85
大量的竹楼倒塌,废墟间一排排的死尸铺上白布,伤员发出痛苦的呻吟,就地躺着。治疗蛊师忙的满头大汗。

%86
家主阁塌陷了大半,广场上已经积满了一层血水,这样的异象令族人十分恐慌。

%87
铁血冷和古月一代激战,引发了山体动荡,自然就波及到地面正上方的山寨了。

%88
留守在山寨中的古月药姬,没有等来古月博,却等到了这批三族残军。

%89
“这是怎么回事?”她寒声喝问。

%90
方源没有说话,因为身后空中蜂拥而来的飞鹤,已经是最好的解释。

%91
“这?!”

%92
“天呐……”

%93
“难道我古月一族,要在今日陨灭了吗?”

%94
一时间,古月山寨大乱。

%95
“师兄,师弟我千里迢迢,特意赶来看你。你怎么不出来迎接呢?”白眉老者高居鹤背上,语气充满了冰寒的杀机。

%96
他余音未了,山寨广场上,血水陡然喷涌十米高度,朱红的棺材竖直地冒出来。

%97
化身为血鬼尸的古月一代,就站在棺椁当中,血红的双眼死死地盯着白眉老者。

%98
“你竟然也没有死……你是怎么找到这里来的?果然,刚刚那个蛊师,是受你的指引!”古月一代恨声问道。

\end{this_body}

