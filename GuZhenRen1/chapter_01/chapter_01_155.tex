\newsection{借}    %第一百五十五节:借

\begin{this_body}



%1
“恭喜恭喜啊。”

%2
“方源家老实在是年轻有为,实乃后辈之楷模!”

%3
“今后共事,真是期待方源家老的风采啊,哈哈。”

%4
……

%5
一众家老围着方源,说着漂亮话。

%6
学堂家老站在人群的外围,看着方源,眼神十分复杂。

%7
他从未想过有这么一天,方源竟然晋升成了家老。那一届中,他最看好方正,然后是赤城和漠北。

%8
想不到第一个有出息的,居然是方源。

%9
“我这点小小的成绩,和诸位前辈们怎么能比呢。能有今天,还得多亏了家族的栽培。学堂家老,您对我的教导我一直铭记于心呢。”方源脸上浮现着温和的笑,谦虚而又谨慎。

%10
学堂家老没有料到方源这个刺头,居然主动和他找招呼。

%11
他楞了一楞,脸上露出欣慰的神情,道:“看来这些年你成熟了很多,方源家老,好好努力罢。家族需要你这样的新血,我为你感到骄傲!”

%12
方源再次向学堂家老诚挚道谢,同时应对着其他家老。

%13
他有五百年经历,这等逢场作戏简直是手到擒来。

%14
他进退有据,言语温和,态度谦逊,让人听了无不如沐春风。

%15
古月赤练在旁冷眼看着,越看越心寒。这方源处理应对,一言一语都如此老道含蓄,他真的只有十几岁吗?难道他天生就是做政客的料子?

%16
学堂家老则心中惊异。他想到当年学堂的时候。方源是那样的桀骜不驯,甚至连同窗都要剥削。他为此感到头疼不已,没有想到在方源身上竟然有这般翻天覆地的改变。

%17
倒是古月漠尘,对方源如此表现,却并不奇怪。

%18
他早就领略过方源的心机。

%19
此时看着方源温文尔雅,成为人群中的焦点和中心,心中越加感叹古月赤练下了一步好棋。

%20
这场交流只是泛泛飞,时间并不长。但不管家老们是有什么心思。站在什么立场,都不由地对方源刮目相看起来。纷纷在心中感叹,传言果真不可靠!

%21
最终,方源婉言推辞了一些家老的邀请,和古月赤练一起微笑着,离开了家主阁。

%22
“哼,这下你满意了?扳倒了古月药姬。还把我们赤家拖下水!”在书房中,赤练终于不再伪装。脸上的微笑被愤怒所取代。

%23
方源坐在他的对面。悠悠笑道:“说起来,这件事情你还得感谢我啊。古月药姬一倒,是你赤脉占了大便宜。”

%24
古月赤练目光闪了闪:“哼,年轻人,你想的太简单了。赤钟虽然是我赤脉成员,但他的妻子却是药脉中人。族长任命他暂代药堂家老的职位,只是想平衡我赤脉和药脉之间的内斗罢了。你究竟是怎么知道赤城这件事情的?”

%25
说到最后。古月赤练忽然问道。

%26
他的一双眼睛,紧紧地盯住方源。闪着如鹰隼般锐利的光。

%27
方源不以为意地耸耸肩,而是道:“老人家。我身上的元石不够用了,你先给我三千块。”

%28
砰。

%29
古月赤练狠狠地拍了一下桌面,他压低声音,低吼道:“方源!你不要以为知道了那么秘密,你就吃定了我赤脉。我老了,没有多少年岁可活了,大不了舍了这条命不要!哼,我可以接受合作,但绝不受威胁!”

%30
“今天这种事情,我绝不允许有第二次发生!如果你再胡乱树敌,然后随随便便地把我赤脉拖下水,事后你绝对会后悔的!你真以为,那个秘密能毁灭整个赤脉?呵呵,你别天真了。”

%31
方源沉默不语,目光幽幽,任由古月赤练喝斥着。

%32
古月赤练刚刚拍桌子的事后,气势惊人如虎,但是越说气势就越弱,直至最后明显变得气虚,底气不足。

%33
直到他不再开口,方源才悠然笑道:“老人家,火气不要这么大么。我最近的确手头紧,这三千块元石,我也不是白要你的,是向你借的。我可以给你打上借条。”

%34
古月赤练冷哼一声,语气稍缓:“你不会缺少元石的,你刚刚成为家老不久,并不知道家族对家老的优待。只要是家老,每周都有一百块元石补贴。这还是和平时期的数额。如今狼潮时期,你每周将有三百块元石的补贴。”

%35
“不仅如此,你还可以免费取得一只三转蛊虫。同时家族的秘方,从一转到三转,都会向你开放。你可以尽情地选择其中的秘方,来合炼出你的三转蛊虫。还有一些其他的特权,比如说:寻常蛊师,只能娶一位妻子。成为家老之后,可娶一妻两妾。”

%36
“原来是这样。”方源对这些自然心知肚明,但是表面上则摆出一副第一次听说的表情。

%37
“不过,即便如此,我仍旧想要借三千块元石。你也明白,我才刚刚晋升三转。合炼三转蛊虫,自然需要消耗大量的元石。”方源“诚恳”地道。

%38
古月赤练沉吟不语。

%39
他思考着:“依方源家老的身份,倒不至于欠债不还。那他的名誉还要不要了?只是万一,若是他在狼潮中死掉,那么我这三千块元石岂不是打了水漂?等等,他死掉不是更好?赤城资质的事情就能继续隐瞒下去了。但是这个秘密,他到底是怎么知道的?究竟还有什么其他人知道呢?不妨先借给他,降低他的防备之心,然后再试探他。”

%40
念及于此,古月赤练也就不再坚持,当即取出纸笔。

%41
方源写了借条,并且画押手印。

%42
古月赤练叫来管家,嘱咐下去。很快就将几个满满的钱袋,取了过来。

%43
方源将每个钱袋,都掂了掂,并没发现问题。

%44
他的确需要这笔元石。

%45
为了合炼人兽葬生蛊,他几乎消耗了全部积累。这三千块元石算得上一场及时雨。

%46
晋升三转,才是刚刚开始。他需要合炼出三转的蛊虫,才能真正拥有三转蛊师的战斗力和生存能力。

%47
对此,他心中已经有大致的谋划。三千块元石还未必够用。

%48
但并不要紧,赤脉将是他的大钱庄。

%49
这一次借元石,只是刚开始。有第一次就会有第二次,一回生二回熟嘛。

%50
至于还债,呵呵……

%51
得了这笔元石,方源却没有急着走,而是笑道:“我还想再借一样东西。”

%52
“你最好不要得寸进尺。”古月赤练脸色一沉,但终究还是说道,“说吧,是什么?”

%53
“净水蛊。”方源双眼眯起来,坦言道。

%54
商队中曾经售卖过一只净水蛊,如果谁最有可能购买了这蛊,极大可能便是古月赤练。

%55
因为他用自己的真元,替孙子赤城温养空窍,提拔他的修为。令其空窍内存了异种气息,非得用净水蛊才能化去。

%56
“这绝对不行!”古月赤练断然拒绝。

%57
他的确买下了那只净水蛊,但这蛊虫是给他亲孙子古月赤城准备的。要想再买到,非得靠些机缘了。

%58
“话不要说的这么绝嘛。”方源呵呵笑起来,“一只净水蛊和赤脉声誉,孰轻孰重,我相信古月赤练大人,身为赤脉家主,自然能分得清楚。”

%59
古月赤练的脸色完全沉下来,阴寒一片,他狠狠地盯住方源,咬牙切齿地道:“方源,你要搞清楚你在干什么。你在勒索我,勒索敲诈堂堂赤脉的家主!”

%60
“不不不,我这不是勒索,是商讨。我只是借净水蛊一用,今后会还给你一只新的。这个我也可以打欠条。”方源微笑着,语气却很坚定,显示出他志在必得的决心。

%61
“你休想!”赤练的态度也很坚决。

%62
……

%63
半个时辰之后,方源带着三千块元石,以及一只净水蛊离开了赤家。

%64
而古月赤练,则坐在书房当中,看着桌面上方源写的两张轻飘飘的借条,心中的郁愤之情简直如滚滚江河,滔滔不尽。

%65
赤脉的把柄被方源掌握住,这让赤练处于极大的被动当中。方源的胜利,也是情理之中的事情。

%66
三天后。

%67
方源盘坐在床榻上,脸上映照着白光。

%68
一片白色的光团,悬浮在半空当中——合炼蛊虫已经进入尾声。

%69
方源一边用意识维持着光团,一边将一块块元石接连抛入光团当中。

%70
光团骤然消失,一只全新的蛊虫飞到方源的手掌中。

%71
它的外形,就像是一只大瓢虫。半圆形乳白色的甲壳上,点缀着点点黑斑。

%72
它体积比较大,足有一个成年人的拳头大小。

%73
三转天蓬蛊!

%74
“终于合炼成了。”方源满意地点点头,这是他第二次合炼。

%75
天蓬蛊是由二转的白玉蛊,以及一只水行防御蛊虫一起合炼而成。

%76
方源第一次合炼,用的是水罩蛊和白玉蛊。结果合炼失败,水罩蛊因此死亡。

%77
这一次用的水行防御蛊虫,是方源用最后的一点战功换过来的。

%78
然而,这只天蓬蛊并非方源的第一只三转蛊虫。他的第一只三转蛊虫,是直接从家族中取来。

%79
晋升三转成了家老之后,家族就会免费地奖赏一只三转蛊虫。

%80
于是,方源就选取了雷翼蛊。

%81
这蛊还是家老们斩杀了一只狂电狼后,得到的战利品。能凝聚出一对雷光羽翼,令蛊师在短时间内拥有飞翔之能。

%82
有了雷翼蛊辅助移动,方源最后一块的战力短板也就被补齐了。

\end{this_body}

