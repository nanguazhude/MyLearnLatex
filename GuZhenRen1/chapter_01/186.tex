\newsection{春秋不出}    %第一百八十六节:春秋不出

\begin{this_body}

血湖中,两强对立!

“纵然是五转血鬼尸,终究也不过是一头僵尸。古月一代,你已经不是活人,体内充盈死气。空窍也已经死了,虽然能存储真元,但一分少一分,用一成少一成。再不能自我恢复。”巨傀冷声道。

“呵呵呵……虽然真元不能自我回复,但那又怎样?吾照样能通过元石,汲取真元力!汝身上有伤,居然还敢来扰吾……受死吧!”古月一代暴喝。

刹那间,血雾狂涌,血浪滔天,从血湖中飞出一大股虫群。

赫然是五转血滴子!

同时,一支支的刀翅血蝠群从山壁洞顶飞来,从血湖深处钻出,集结起来,纷纷向巨傀杀去。

庞大的蝠群,约有上千头,一下子就汇成了大军。

它们虽然只有三转,属于肉搏型的蛊,但数量一多弥补了质量,就算是五转蛊师铁血冷,也要为此头疼。

但这远远还没有结束,古月一代意识彻底苏醒,在他的召唤下,一支支新的血蝠群,不断出现,朝这里汇集。

他在这里经营了近千年,有不可告人的企图计划。早已经把这里经营成了老巢,占据强大的地利优势。

血蝠群法度森严,排布在空中,不断盘旋,宛若精锐大军,将巨傀包围起来。

血河蟒不再挣扎,反而尽量收缩蟒身。这反倒让铁血冷隐隐忌惮,打起十二分的注意力。

蛊的力量虽然强悍,但智慧低下。有无蛊师主持血蝠群大军,其效果有天地云泥之差的。

“古月一代刚刚出场,就将局面扳回来,有压制铁血冷的趋势。他地利优势极大,以逸待劳,铁血冷又有伤在身,恐怕有些不妙。”

方源已经缩到这处山壁的洞口中,借着阴影,暗暗观察战局。

“但铁血冷行走***多年,身上负伤,仍旧还敢入这魔穴,必有凭借。不管如何,接下来必将有一场剧烈的激战。搞不好这方地界,都要塌陷。我该不该留在这里观战?留下来十分危险,这是地底深处,说不定就被活埋。若留下来,坐山观虎斗,渔翁得利的几率有多大?”

方源脑经急速转动,思索利弊。

他现在修为只是三转初阶,观战风险极高,很可能一场碰撞的余波,就要令其重伤。

但真若是渔翁得利,让方源笑到最后,那利益就大了去了。毕竟是五转蛊师,若有所得,可省去百年苦功积累!

“高风险高利润……”方源长叹一声,打了退堂鼓。

局面早已经不在他掌控之中,留下来的风险太大了。

鸟为食亡,人为财死。这种事情,他前世五百年,见得多了。

他生性谨慎,留得青山在不愁没柴烧,况且他知晓无数秘辛,许多未暴露的传承地点,都有丰厚利润,何必在这里拼命?

正当方源要离去,忽然一支血蝠群从大军中分离出来,向他飞来。

这支血蝠群有近百只,方源连忙转身速退。

“吾族小辈,休要担心。那洞中设下机关陷阱,又有地下猛兽盘踞。这些血蝠群可保你安危。”古月一代的声音远远传来。

方源闻言,逃得更快。

古月一代轻咦一声,未料到方源年纪轻轻,如此滑溜,竟然看出自己的恶意。当下,心念一动,又有一支近百头的血蝠群,向方源追去。

他这一分心,让铁血冷觑得一丝不算破绽的破绽。

天地宏音蛊!

吼――!

巨傀张口,发出震撼九天的吼叫。刹那间,形成磅礴浩瀚的声浪,冲刷四野八极,如雷霆轰鸣。

离得近的刀翅血蝠蛊,在第一时间就被这股雄浑之音震杀,纷纷掉落下去。

较远的血蝠群,也被震得七荤八素,在空中胡乱上下飞窜。

原先还密集的血蝠大军,在顷刻之间,就分崩离析,短时间之内再难有作为。

天地宏音蛊高达五转,又是群攻利器,最擅长对付血蝠群这样的攻势。先前铁血冷用了,威力不显,是他故意克制。如今借助巨傀之身,这才爆发出天地宏音蛊的真正力量。

就算是漫天的血滴子,也被震爆了无数。弥漫的血雾消退,还一片清明。

声浪排击周遭山壁,整个空间,整个山体都在剧烈震荡。

以巨傀为中心,血湖水面都被挤压向下,整个血湖形成碗状。血水漫过洞口,向里面倾泻。

但在此之前,就有声浪波及过去。

方源也因此遭殃,白芒虚甲一阵闪烁,险些崩溃。为了抵抗住这股声浪,空窍中真元骤然降低一成有余。

声波在狭窄的山洞中回响,方源双耳不住地嗡鸣,一个趔趄,差点栽倒在地上。

不过这股声浪,也对他不无帮助。

追赶他的两拨血蝠群,第二波已经湮灭,第一波则被声浪震得七荤八素,一时间在山洞中乱撞乱飞,无法追击方源。

机不可失,方源趁机奔跑,和血蝠群拉开距离。

这波血蝠群距离声源最远,很快就回复过来,扑扇着两对翅膀,重新展开追杀。

血蝠群有近百只,皆是三转,方源实难抗衡,只得埋头逃跑。

先前觉得山洞漆黑,不知通向哪里。但当他眼睛逐渐适应后,这山洞情形也能朦胧看清。

还都是赤土散发的微光之功。

雷翼蛊虽然已经变得不靠谱,但好歹也能增加一点助力,被方源极力催动。

但即便如此,双方速度差距较大,距离很快就被拉近。

“距离差不多了,是该使用春秋蝉了!”方源咬牙,眼看着血蝠群不断逼近,只能出此下策!

刀翅血蝠蛊只是三转,六转蛊的气息能将它们死死的震慑住。但铁血冷和古月一代,就在不远处。

春秋蝉一出,势必将引发一阵异象,动静太大,足以引发他们俩的注意。

但方源也是无奈,眼下情形,只能寄希望于他们双方战斗胶着,无法分心他顾。

吱吱吱……

血蝠群迫近,已距离不到百步远。

方源吐出一口浊气,心中念道一声:“春秋蝉,出来吧!”

一秒,两秒,三秒……

方源发愣,立在原地,春秋蝉稳居空窍中央,岿然不动,散发着绚烂多姿的黄绿之光。

“怎么会这样?!”方源心中大惊。

……

远处的天空蒙蒙发亮。

黎明到来了。

山坡上,古月博以及白家、熊家三位族长,并肩而立。

“虽然真正的三族大比,还在明天。不过预选赛还是少不了的。二位,时间刚好,可以开始了么?”熊家族长微笑着道。

白家族长冷哼一声,没有理他。

“那就开始罢。”古月博心不在焉地回了一句。他将目光转移下去,看向山坡下聚集的蛊师们,仔细搜寻,却仍旧没有发现方源的身影。

这令他心中的担忧更深了。

这些蛊师皆是三十岁之下的年轻面孔,相互站着,形成三个泾渭分明的团体。

放眼望去,三家实力一目了然。

熊家蛊师最多,他们主动撤离,保存了大量的战力。古月家和白家的人数偏少,但白家阵营中有一位白凝冰,只他一人,就将白家的整体实力,拉高到三族之首。

熊家族长大喝道:“此次比试,范围方圆百里,时间持续到晚上,太阳落山便结束。交手不论生死,但也希望你们克制。你们的手中都分发了一块铭牌,收集到三十块铭牌者,方有资格参加接下来的三族大比!下面开始!”

生死激战,只要符合标准,人数不限,方圆百里都是比试场。甚至就连中途参入,都被允许。

这不是一场公平的比试。但三位族长都没有意见或者抱怨。

在这个世界上生存,靠的是拳头,是实力。实力强,那就有资格占据更多的利益。实力弱,那就自认倒霉,低调行事,暗暗积蓄力量,变成强者罢。

……

短短的几个呼吸的时间,方源却感到像两三年那般漫长。

他额头渗出冷汗,春秋蝉竟然催动不出,这是怎么一回事?

春秋蝉是他的本命蛊,关系重大,高达六转,是最终底牌,居然不受自己控制!这个情况太严重了,让方源必须去高度重视。

昏暗的山洞中,他眯起了眼睛。

心中的慌乱只维持了不到半秒,他就强自镇定下来。

脑海中思绪如电光频闪,他心神沉入空窍,春秋蝉并无异状,仍旧在不断地快速复原着。

但任凭方源如何调动,它仍旧只是稳居中央,身躯都不带一丝颤动。

“我明白了!”方源恍然大悟,“这春秋蝉一旦成为本命蛊,居于空窍,就不能再移动分毫了。”

蛊是万物真精,有无穷奥妙,可谓千奇百怪。

蛊虫养、用、炼,三方面博大精深,常常在这些方面有着特殊标准。

在“养”上面,蛊虫都只食用特定食料。“炼”方面,亦有各种要求。

在“用”上,那竹君子非得一生都不撒谎的蛊师,才可催用。而要用正气蛊,非得蛊师心存正义。

又比如留影存声蛊,用一次就死亡,影像附着在石壁上持续一段时间。

春秋蝉一旦被炼化,居于空窍,就不能再移动。这点特性,令方源联想到一只著名的蛊――水幕天华。

\end{this_body}

