\newsection{天地我在独行}    %第一百二十三节:天地我在独行

\begin{this_body}

%1
“有什么事情?”方源看着赤山。

%2
即便是在冬天,他仍旧赤裸着上身,赤红的肌肤散发着暖意,让人仿佛觉得靠着一只火炉。

%3
白雪飘落在他的身上,瞬间就消融掉。

%4
这是因为他体内的空窍中藏着一只双窍火炉蛊。

%5
火炉蛊是二转蛊虫,蕴藏火气,用于进攻。御寒只是一个附作用罢了。

%6
赤山的眼神有些复杂,他盯着方源沉声道:“你知道这些天,熊家寨的熊力,要找你比拼力气的事情吗?”

%7
“知道。”方源点点头。

%8
赤山长叹一声:“熊力要找你比气力,不是单纯的比斗,而是关系到此次三寨联盟的利益分配。狼潮之下,不联盟就是灭亡。但是联盟之后,利益如何分配却是最关键的问题。因此这些天来,三寨都为此事胶着僵持。”

%9
方源看了赤山一眼,顿时明白他主动找上门来的意思。

%10
谈判是最艰辛的事情,为了利益,三家都不会主动退让,寸土必争。谈判桌上早已经硝烟弥漫,战火如荼。

%11
这个世界的价值观,一是力量,二是亲情。

%12
青茅山三大家族,各是各家,摩擦已久,仇怨早深,自然不能动用亲情和稀泥。那么要打破谈判的僵局,就得靠力量。

%13
地球上有军事演习,彰显力量。这个世界异曲同工,也有切磋斗蛊,来展现自己的强势一面,从而争夺更大利益。

%14
熊力要找方源切磋,比斗气力,就是这个缘由。

%15
果然,接下来赤山便道:“我和熊力交过手,他有棕熊本力蛊。已经养出一熊之力。又有熊豪蛊。能再暴涨一熊之力。叠加起来,就是双熊之力。我远不是他的对手,尽管不想承认。但是青茅山第一大力士的名头,他实至名归。”

%16
他顿了一顿,继续说道:“咱们就事论事。你我力气相差并不大,所以你也不会是他的对手。但你却不能输,因为你是推醒了吞江蟾,将青茅山拯救的英雄。你一旦输了,我们古月一族的利益就要受损。所以请你为了家族的利益,舍弃掉个人的名誉,选择避战!”

%17
方源沉默地看着赤山。

%18
赤山低垂下目光:“我知道这件事情对你而言,十分为难。毕竟选择避战,对个人的勇名简直是毁灭性的损害。但是家族的利益为重。若是你输了,恐怕家族要退让得更多。家族培养了我们,我们自然也要为家族贡献。不是吗?家族需要你。你为了家族牺牲个人的名誉,这也是理所应当的事情!但此事因我而起。我以个人的名义,补偿你一点东西,算是我的心意。”

%19
说着,赤山就递给方源一个大型钱袋。

%20
方源接过来掂了一下,不由地哂笑出声:“原来我的名誉,只值两百块元石?”

%21
赤山听出了他话中的嘲讽之意,他眼中厉芒一闪,肃容道:“方源,你不要有愤懑之心!先前是对你好言相劝,事实上我是带着任务而来。要你避战,这是家族高层的秘令。你听也得听,不听也得听。请你好自为之。”

%22
说完,他转身离去,深深的脚印印在雪地之上。

%23
方源看着赤山的背影,眼中露出了然的光。

%24
“家族为了争取最大的利益,恐怕早已经将我驱赶吞江蟾的事情,当做了谈判桌上的一个筹码。毕竟吞江蟾在这里,对于青茅山上下都是一种危害。熊家寨方面,为了打消掉这个筹码,就秘令熊力来挑战我。”

%25
“对于家族来讲,我不过是个棋子罢了。熊力也是棋子,赤山更是棋子。可笑有些人心甘情愿成棋子,还引以为荣,认为理所应当,这是被家族成功洗了脑的。”

%26
“不过我本来就不想和熊力比斗。所谓的名誉,不过是他人的赞赏罢了。这把羁绊人的枷锁,不知拘拿了古往今来多少英雄豪杰。但对我而言,舍掉又有什么可惜?呵呵,倒是要谢谢赤山,凭白送了我两百块元石。”

%27
想到这里,方源暗暗冷笑一声。

%28
自己为什么会招来熊力的挑战,无非是因为拯救山寨名誉罢了。赤山为什么要挑战熊力,无非是为了青茅山第一大力士的名誉罢了。

%29
所谓的名,只不过是一块虚荣的大饼。诱惑了多少人,羁绊了多少人,圈住了多少人。

%30
可叹可叹!

%31
雪仍旧在徐徐地下着。

%32
整个古月山寨,静静地立在雪中。身边的行人,在路上匆匆而行。

%33
“可笑这些人,都被一张虚幻的大饼套牢了身躯!”方源眼帘低垂,睫毛下漆黑幽深的眼眸半遮半掩。

%34
雪地中的光映照在他的脸上,少年的脸显的苍白,透着一股清冷。

%35
忽的呵呵一笑,方源轻吟道:“白雪尽皑皑,天地我独行。独行无牵挂,孤影任去来。”

%36
他迈开脚步,继续行走。

%37
一路上行人匆匆,方源却在独行。

%38
不管是族人、白雪、山寨,都不过是模糊的背景罢了。

%39
片刻之后,他回到自己的租房。

%40
竹楼酒肆等等,都被他转卖掉了。他依旧住在租房当中,虽然环境十分简陋,但方源并不苛求,只要有一个落脚点就行。

%41
盘坐在床上,方源开始修行。

%42
蛊师修行中,突破大境界,需要天资和才情。但是突破小境界,就纯粹是水磨工夫。只要时间足够,窍壁总会被不断地温养,总能不断地提升。

%43
按照约定,到了傍晚时分,江牙来到了方源的住处。

%44
“方源大人,这是这次的元石,请您查收。”他进了房间,恭敬地递过五个钱袋,里面自然装着满满的元石。

%45
元石远超四百块。狼潮渐渐来临,蛊师们对于生机叶的需求,也增大了。这就导致了方源贩卖生机叶的价钱,也越来越多。

%46
方源交给江牙九片生机叶。同时问道:“我先前要你收购的东西。你办妥了吗?”

%47
江牙却露出赫然之色,摇摇头道:“方源大人,时机不巧啊。狼潮将近。族中已经实施了物资的管制。其他的东西还好说,但是那只鱼鳞蛊,价值只比玉皮蛊稍差。在下已经尽了全力。但恐怕一时之间难有成果。”

%48
方源皱了皱眉头。

%49
鱼鳞蛊是用来和隐石蛊一起合炼,炼成隐鳞蛊所用。没有了鱼鳞蛊,他就合成不了隐鳞蛊。

%50
“不过就算是物资管制,也未必弄不来鱼鳞蛊。归根结底,也是这江牙能量太小。看来合炼隐鳞蛊这事,恐怕也得拖延下去了。”方源心中一叹。

%51
不过他也不气馁。

%52
世间之事,不如意者十之八九。

%53
此乃人生之常态,所谓“一帆风顺”不过是句美好的祝福语罢了。

%54
“白家寨盛产鱼鳞蛊,家族中也有鱼鳞蛊。只是数量较少罢了。看来这事情,得等到正式联盟之后了。”

%55
方源并不急躁,他知道一旦三家正式联盟之后。就会设立战功榜。鼓励蛊师们积极猎狼。到那时,用战功就能换取三家的物资。

%56
当然。古月一族的月光蛊,熊家寨的熊力蛊,白家寨的溪流蛊,都是各家标志,并不在这换取的范围内。

%57
但是鱼鳞蛊,必定有的。

%58
危机常常伴有机遇。

%59
对于蛊师们来讲,狼潮是一次严峻的考验,但同样也是一次崛起的良机。

%60
在狼潮的冲击下,无数的成名蛊师死去,无数的蛊师因此成名。家族中旧有的势力也许因此而衰弱,新生势力昂首走上政治舞台。

%61
到了晚上,又有意外的客人到访。

%62
是古月青书和古月方正。

%63
青书开门见山,旧事重提,意欲收购方源的酒虫。

%64
但除此之外,他还想一同收购方源的黒豕蛊,甚至是九叶生机草。

%65
九叶生机草绝对是非卖品,酒虫早已经合炼成了四味酒虫,方源想拿也拿不出来。因此都做了拒绝。

%66
倒是黒豕蛊……

%67
“我已经又增长了一猪之力,黒豕蛊对我的利用价值已经很低了。黒豕蛊的最优晋升结果,是钢鬃蛊,虽然说此蛊攻防一体,但对我来讲,有了白玉蛊在手,用处并不大。不妨换一只鱼鳞蛊。”

%68
方源想到这里,就提了出来。

%69
“鱼鳞蛊?”青书微微皱了皱眉头,旋即又点点头,“我知道了,你的确缺少一件防御性的蛊虫。鱼鳞蛊可以合炼成二转鳞甲蛊,倒是能起到不错的防御效果。”

%70
利用鱼鳞蛊和隐石蛊合炼出隐鳞蛊的秘方,也是前世后两百年后,一场意外而发现的优秀秘方。现在青书不知道也很正常。

%71
方源也不揭破:“黒豕蛊比鱼鳞蛊要贵的多,若是换了,你必须得补偿我其中的差价。”

%72
“这是自然。”青书点头又问,“九叶生机草就算了,但是酒虫你真的不卖吗?它对你来讲,也已经无用了。白白养着,还耗费元石。”

%73
方源摇摇头:“酒虫的事情你就不用再提了,它是非卖品。”

%74
青书摸摸鼻子,不禁苦笑:“方源,这事情比你想象中的要复杂。古月药乐你知道吗?她是药堂家老古月药姬的孙女,今年的学堂新生,有着乙等资质。古月药姬十分疼爱这个孙女,曾经在树屋中竞价过一只酒虫,可惜没有成功。”

%75
“酒虫你也用过,其中的好处你自然体会最深,我也不多说了。只是药姬大人,为了孙女,真的很想求购这只蛊虫。老人的舐犊之情,完全可以理解。所以就找到你这里来了。她是真心想要求购,开的价格很高,甚至允诺,你若在狼潮中受伤,必能得到药堂的悉心照顾。请你好好再考虑一下吧。”

%76
:没有存稿,过年也比较忙,今天是一更,明天是一更,后天可以两更了。这是删除了存稿的后遗症,我也早就料到了。让诸君失望的后果,我也在默默承受。缺的四更会补上的,只是时间上要长一点。来日方长,总归是要补的。唉,这次败了一下人品,却还有这么多童鞋支持我,很感动!)

\end{this_body}

