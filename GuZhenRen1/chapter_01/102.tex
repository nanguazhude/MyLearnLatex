\newsection{冬风吹来春天(保质量第三更)}    %第一百零二节:冬风吹来春天(保质量第三更)

\begin{this_body}

方源到了内务堂,便上缴了黄金蜜酒。

负责接待的中年男盅师很是惊异,执笔问道:“你这是完成了家产任务?”www.13800100.com

“你说呢?”方源反问道。

中年男盅师顿时皱起眉头,这任务就是他特意挑选的,专为了为难方源。想不到方源这么快就完成了!

他看着方源,眼中厉芒闪烁,严肃地道:“少年郎,我问你什么你就要答什么。语焉不详的话,可是要拖累对你的评价的。我来问你,你这任务是你独自一人完成的么?要说实话,我们是会调查的。”“当然是独自完成的。”方源答道。

“好,那可我记下来了。”中年男盅师一边记录着,一边心中冷笑:就凭他一个区区新人,怎么可能独自一人完成这样的任务?如此谎报结果,家族必会找人调查。到时候麻烦就大了。

哪知方源又接着道:“不过这情况有些特殊,那天我不过是为了侦察,没有想到遇到野熊掏蜂窝。我趁机就取了这蜜酒。”

“什么?”中年男盅师笔下一顿,抬头看向方源。

方源耸耸肩,微笑着:“要不然你以为我一个人,能完成这任务?说起来,还是多亏了你,给我挑选了这个任务呢。”

中年男盅师顿时就楞在那里,心中复杂之情难以用言语来表凶好一会儿,他这才干笑两声,继续埋头书写记录。

方源用沉静的目光盯着这人,心知肚明一自己接的这个家产任务如此之难显然是这男盅师的“功劳”。可惜自己已经完成了这任务,就算是家族起疑来查证,方源亦有了相应的布置。这个人哪怕再想要阻止,也阻止不得了。

哪怕此事是他亲手办理的。

这就是入了体制的悲哀,身份成了桎梏行为的枷锁。

“好了,你的任务完成了,已经可以继承家产。只是,你双亲的这些遗产目前都被你的舅父鼻母掌管着。内务堂会替你索回,三日后你再来吧。”片刻之后,中年男盅师道。

方源点点头,这规矩他知道,但他却盯着中年男盅师记录的纸张,道:“按照家族规定,内务堂的任务记录,需要当场给完成者确认。

请把这份记录给我看一看。”中年男盅师脸色微变,没有料到方源如此熟悉这流程。他轻哼一多,将记录递给了方源。

方源接过一看这记录倒无不妥之处。洋洋数百字,最后末尾是评价――良。

他顿时看清了这盅师和古月冻土的亲密程度。

这份评价很是中肯,可见中年盅师并没有因为古月冻土,而放弃职业原则。他之所以帮助古月冻土,应该只是看在交情份上,展开的一场交易。

属于拿人钱财替人消灾的那种。

将手中的这份记录交还给中年盅师后方源便离开了内务堂。

出了门口古月冻土已经不见了踪影。

方源不由地冷笑一声,古月冻土的影响力,还没有大到能影响内务堂运转的程度。就算是族长古月博要做到这点,也得顶住家老团的巨大压力,在政治利益上付出巨大代价。

地球上有句话,叫做“人在江湖,身不由己”。

江湖中有规矩,其实就是一种体制。一入体制不管任何人都是棋子,相互制约,身不由己。

除非个人的力量能达到抗衡整个组织的地步,否则加入了组织,还想要无拘无束的自由,那是痴心妄想!

借助了家族体制的力量,方源夺回家产已经是板上钉钉的事实了。

不要说古月冻土就是族长也不会为了这区区小事,付出集治代价。

会客厅。

“冻土老哥,这件事情我实在是爱莫能助了。”中年男盅师叹着气,站在古月冻土的面前。

古月冻土面沉如水,坐在主位上沉默不语。

“真的就没有办法了吗?”一旁,舅母惊惶地问道语气充满了焦急和不甘。

中年男盅师缓缓地摇了摇头:“此事已经是定局,步入了内务堂的处理流程,除非是当权的两大家老,或者是族长才有这能量阻挡下来。

冻土老哥,这份内务堂的单子上,记录着遗产的详细,还请你全部归还出来,不要让我难做啊。”说着,就递过来一份清单。

清单上密密麻麻地记录着,大到房产,小到桌椅板凳,除此之外还有方源双亲遗留下来的盅虫。

盅师战死后,他们的盅虫若是被回收了,都将作为遗产,留给盅师的继承人。这也是家族的一项政策。

舅母只是瞟了这清单一眼,就失态地尖叫起来:“该死的,你不能这么做!这都是我们的东西,我们的!老爷,你也不说说话,你快想想办法呀。没了这些财产,我们家还剩什么?恐怕连家奴都要辞退大半,供养不起了啊!”啪!

舅父古月冻土猛地站起来,甩手一个巴掌,将舅母打得从座位上跌到地上。

“你嚷嚷个屁!”舅父勃然大怒,气急败坏“没见识的东西,家族的规矩在那摆着呢,你想不还就不还吗?无知,愚蠢!”舅母用手捂住脸颊,一时间被打懵了,瘫倒在地上,呆呆地望着自己的丈夫。

“哼!”古月冻土一把夺过清单,咬着牙,翻看了一遍。

他双目充斥着血丝,气喘如牛,恨声道:“还!这些东西,我都还,必定一样不少!!只是……”

他额头青筋暴跳,脸上肌肉抽动,神情带着一股狰狞:“只是方源啊,我阻止不了内务堂,但是我却能对付你。别以为你拿了这家产就万事大吉了,哼!”三天之后,方源从内务堂走了出来,手中拿着一叠的房契、地契、

卖身契。

“想不到,这份遗产如此丰厚。”他微微有些愣神。

尽管方源已经有了不小的预期,但是这份家产拿到手后,仍旧出乎了他的预料。

十余亩农田,八名家奴三处竹楼,以及一处酒肆!

“难怪舅父如此刁难,想方设法地阻止我。”方源忽然理解了古月冻土的做法。

有了这份丰厚的家产,就算是在这样的世界,也足以做到衣食无忧了。

十余亩农田,八名家奴,这暂且不提。那三处竹楼,就是房产,单单是用作出租,每个月的租金也能支撑方源如今的修行了。

除此之外还有一处酒肆。要知道整个山寨当中,也不过四家酒肆。

这样的家产,换做在地球上,就是拥有几处别墅、一处酒店,蓄养仆从的小豪门。

值得一提的是,这世界生活环境恶劣而且艰难事关性命山寨又是最安全的所在,因此房价比地球上的还要贵。

“据说我这方家一脉,三辈之上,曾经是一位当权家老,留给后人许多家产。我有了这些家产,别说是七只盅虫,就算这数目再暴涨一倍,也能养得起!不过最关键的还不是这些家产,而是这只草盅!”

此时,在方源的怀中,静静地躺着一棵草盅。

它小巧玲珑,翠绿的根须,如人参的参须。有些半透明的根茎,宛如翡翠。九片圆形的叶子碧绿碧绿的,相互掩映,围绕着根茎,组成一个圆盘形状。

此乃九叶生机草,二转盅虫具备治疗作用。

但如果只是普通的治疗作用,那它和生息草也就没有什么太大的区别了。

它真正的价值在于,它的每一片叶子,若撕下来,就是一棵生机叶。

生机叶也是草盅的一种,一转级数,属于消耗型,用一次就消失。

它还有一个缺陷,那就是使用了一片生机叶疗伤之后,一个小时之内,其他的生机叶将没有治疗的效果。

但是,瑕不掩瑜,它易于炼化,瞬间治疗,喂养便宜,是最受二转盅师欢迎的治疗手段。

治疗盅师,每一个小组,都只有一名。但若有时候组员同时受伤,一名治疗盅师怎么照顾得来?若是治疗盅师首先牺牲,或者与治疗盅师失散,又该如何?

所以,盅师们常常自备着一些治疗手段,生机叶就是最好最基本的一种。基本上在外行走的盅师,都会常备一两片。

“我炼化了九叶生机草后,每撕下一片叶子,就是一片生机叶。

动用真元灌入到生机草中,就能让它生长出新的叶子。可以说,这棵九叶生机草,就是一个移动的金矿。是最重要的遗产,没有之一一。

掌握了九叶生机草,就是掌握了一条商脉。

这个世界生存环境十分恶劣,外出执行任务的盅师,哪有不受伤的?对生机叶的需求,一直持续不断。

“难怪古月冻土被称之为“隐家老”这些年来虽然退隐了,但是对外仍旧有影响力。原来根源就出在这棵九叶生机草上。”

盅师们需要生机叶,而古月冻土贩卖生机叶,有着如此的需求关系,这就让舅父的影响力一直保持着。

方源恍然大悟。

“也就是这个世界,家族亲情的价值观被提高到一个全新的高度,制约住了舅父舅母。否则换做地球上,如此重利,我和方正早就被他们暗杀了。不妙啊,看来接下来还有麻烦。舅父舅母绝不会甘心就这样罢手的。”

“不过我现在已经成长起来,有了二转修为。他们要对付我,又要顾及族规,必定要束手束脚。哼,不管什么手段,放马过来好了。

兵来将挡水来土掩,我都一一接着。”

方源走在街道上,双眼中精芒烁烁不定。

一阵冬风吹来,寒意拂面。

距离春季还有些遥远,但是方源却知道,自己已经迎来了人生的春天。(未完待续

------------

\end{this_body}

