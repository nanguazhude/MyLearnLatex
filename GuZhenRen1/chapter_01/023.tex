\newsection{养蛊就像养情妇}    %第二十三节:养蛊就像养情妇

\begin{this_body}

太阳落山了。

晚霞却还在天边燃烧着,放眼望去远处的群山,都蒙上了一层厚厚的灰,并且正在向黑色过渡。

学堂中一天的课程结束了,学员们三三两两地走出学堂。

“今天真是开心,学到了不少东西。尤其是学会了使用月光蛊。”

“月刃飞在空中,真是帅呆了。可惜我资质不够,将来只能做后勤蛊师,不能上战场呢。”少年们兴趣盎然地交谈着。

一些人则呼朋引伴。

“一起去吃饭吧,顺便喝点米酒,怎么样?”

“好啊,真是不错的建议。”

“你们先走,我得去学堂蛊室那边的铺子里买具草人傀儡,好方便回家练习。”

……

方源独自一人来到蛊室。

学堂的蛊室中存放着不少一转蛊虫,种类繁多,方源的月光蛊就是从里面免费选取的。

每隔一段时间,学员们都会有一次免费领取蛊虫的机会。若要额外获取蛊虫,就要付费了。

方源短时间内,没有想要炼化其他蛊虫的心思。他走到蛊室的隔壁,这是一间不大的铺子。

铺子里有七位学员,正在为购买草人傀儡与店主还价。

“是学弟呀。”负责店铺的一转蛊师,二十几岁的样子,看到方源后,一边做着买卖,一边向方源主动打招呼。

方源意外了一下,发现这蛊师就是江牙。那个在客栈中教训猎户的青年蛊师。

“原来是学长。”方源点点头,面无表情。

江牙一边从身后的柜台上取出一具草人傀儡,递给一位购买的学员,一边向方源友善地笑着,问:“学弟也是来购买草人傀儡的么?要给你留一个吗,只要三块元石。这东西卖得很快的,现在只剩下七具了,再迟可就没货了哦。”

他对于凡人他态度傲慢恶劣,但是对于方源他们,则态度亲切得很。

方源摇摇头,心中好笑:这江牙还挺会做生意。草人傀儡是用草人蛊制成的,算上真元的投入,成本价也不过一块半元石。

“学长,你这就不地道了。先来后到,凭什么给他留?”

“不错,我们早就来了。做买卖也要讲规矩啊。”

“三块就三块吧,元石给你,给我傀儡。”

店铺中的少年们听到傀儡只剩下七具,都着急了,也不继续砍价,纷纷掏出元石购买。

很快,七个人心满意足的走了。

“学弟要买具草人傀儡么?”江牙笑着问,“说是刚刚卖完了,其实还有第八具,压箱底的。学弟不买,可就要错失良机喽。”

方源对草人傀儡毫无兴趣,他摇摇头,掏出一块元石,放在柜台上:“我买十片月兰花瓣。”

江牙一愣,深看了方源一眼,摸走元石,抽开柜台的抽屉,取出一个纸包:“十片月兰花瓣,一个不少,你点点。”

方源当面查看了一番,发现无误,这才离开了小铺子。

蛊虫是需要喂养的。

蛊师炼蛊、用蛊,同时也得养蛊。

炼蛊艰难,有着反噬的危险。用蛊不易,需要多多的练习。养蛊的学问,更是博大精深,皆因蛊虫各种各样,它们的食物也千奇百怪。

有的需要吞服泥土,有的需要星光,有的服用眼泪,有的吸食九天云气。

就拿方源现在拥有的三只蛊虫来讲,月光蛊需要吞食月兰花瓣,每天两顿,早晚一顿,每顿两片花瓣。酒虫呢,则需要饮酒。一坛青竹酒,能支撑它四天。而春秋蝉则更奇特,它直接从光阴之河中喝水,维持生机。

光阴之河支撑着这个世界的运转,它并不是远在天边,而是近在咫尺,流淌在每个人的身边。

万物生灵的每一个动作,都需要时间的推动。

时光如流水,匆匆流逝。光阴之河,无形无色,而万物生灵其实都在光阴的河水中生存、生活。

买了这包月兰花瓣,方源又去客栈,购买青竹酒。

酒虫也可以喝一些浊酒、米酒为生。但是一旦是这种次等酒,喝的量就大了,每天都得要数坛。方源算了下,还不如直接买青竹酒,不仅比买次等酒划算一些,而且也不会惹人怀疑。

“公子,您来啦。”客栈的伙计已经认识了方源。

方源直接抛给他三块元石,轻车熟路地道:“给我上坛青竹酒,再给我弄几个好菜。不用找零钱,先放这儿,等月末一块儿结了,多退少补。”

他虽然现在已经不住在客栈,搬到了学堂宿舍。但是每次买酒的话,都会顺便在这里吃饭。

“好咧。公子您这边请坐,酒菜马上就好。”伙计应和了一声,领着方源到了座位。又拿出肩膀上的抹布,殷勤地擦擦桌凳,这才离开。

果真如伙计所讲,酒菜很快就端了上来。

方源一边吃着,一边心里算着账:“一块元石,能购买十片花瓣,月光蛊每天消耗四片。一坛青竹酒价值两块元石,能支撑酒虫四天所需。也就是说,单单喂养这两只蛊,每天就要消耗将近一块元石。”

这看起来少,但其实已经很高昂了。凡人一家三口,一个月的生活费,才是一块元石。

自从炼化蛊虫到现在,已经十六天了。单单养蛊方面,就耗费了方源十四块半的元石。

“我得了花酒遗藏,收走方正的一袋元石,又拿了头名奖励,元石资产曾经一度高达四十四块半。但是炼蛊前期耗费了六块半,喂养蛊虫耗费了十四块半,生活费半块,如今应该剩下二十块。”

方源取出钱袋子,打开一看,袋子里面装着一块块的元石。

这些元石都是灰白色泽,一个个椭圆体,体积都相等,大小如同鸭蛋。

数了一数,果真只剩下二十块了。

也就是说,若无进项,方源所剩下的元石只够他维持大半个月的。他不像其他的同龄人,身边或多或少都有亲朋好友帮衬着,尤其是古月漠北、古月赤城这种学员,元石根本不缺。

而方源只能自己想办法。

“舅父舅母已经断绝了我的生活费,不过每周末,家族学堂都会向学员发放三块元石补贴。看来三天后的月刃考核,我该表现一下,拿下那十块元石奖励。”方源一边嘴里嚼着饭菜,一边思忖着。

他这年龄,正是长身体的时候。不知不觉间,所有的饭菜都落到了肚子里。

拿起没有开封的青竹酒,方源抬脚迈步,走出客栈。

“公子,公子。”身后的店家伙计却追了上来,“告诉公子一个事情,再过一个月不到的时间,就有商队过来山寨。按照惯例,他们都会收购我们店里的青竹酒。公子独爱青竹酒,每周都来买几坛。掌柜的吩咐了,要小的告诉公子这事。我们店里的青竹酒有限,卖了商队,恐怕就所剩无几了。”

“是这样?”方源闻言,轻轻皱起了眉头。

识人辩话,方源有五百年的经验。店家伙计和青年蛊师江牙说的是相似的意思,但是方源自然能分辨出,江牙的虚话,店家伙计的真话。

这事有些麻烦,方源需要喂养酒虫,按照长远来讲,所需的青竹酒的量是很大的。

这客栈若是缺货,恐怕将来只能用大量的次等酒,来喂养酒虫了。

他不可能一天喝上数坛酒,日子久了,就会引人怀疑。

想了想,方源取出十块元石:“那就再买五坛,叫我拿了,跟我一起,放到学堂宿舍去。”

“是,公子。”伙计忙接过元石。

月兰花瓣若无存储手段,只能存放五天,因此方源每次只买一包。不过青竹酒,能存放很长时间,倒没有这方面的问题。

几个伙计跟着方源进入学堂宿舍,将这酒坛摆放到床下,就都告辞了。

看着手中骤然瘪下去的钱袋,方源叹了口气。

炼蛊艰辛,养蛊也不容易啊。

这还是他有着五百年的前世经历,不用练习使用蛊虫,也就意味着真元消耗的减少,这就省下了一大笔开销。

像身边的同龄人,要练习使用月光蛊,就要消耗真元。要提升熟练度,就得多练习几次。真元消耗得太大,恢复又太慢,只能用元石补充。买个草人傀儡,都得三个元石呢。这都是钱呐。

“幸亏我的春秋蝉,是食用光阴,而不是其他食物。要不然我早就破产了,根本支撑不起。”方源忽然感到很庆幸。

越高端的蛊虫,食量越大,或者吃食越珍贵稀少,越是难养。一只二转级数的普通蛊虫,每天的元石消耗就得达到一块到两块之间。

能买到食物,还算好的。有些蛊虫的食物,特别难以找寻,市面上根本就没有此等货物流通。

就像春秋蝉的食物是光阴,这其实更加珍贵。

俗话说,寸金难买寸光阴。

你有再多的钱财,能买到光阴么?

买不到!

理论上讲,蛊师炼化蛊虫的数量,是不限的。只要你能够炼化,十只,一百只,甚至一千只都可以。想要炼化多少只蛊,就可以炼化多少蛊。

但事实上,一位蛊师通常也只有四五只蛊虫。

为什么?

最大的原因,就是养不起啊。

蛊虫品级越高,喂养的代价越昂贵。常常使蛊师捉襟见肘,为此头疼不已。

还有一个原因,是用不起。

催动月光蛊,发出一次月刃攻击,就得消耗一成真元。一个丙等资质的蛊师,发动三四次攻击,空窍中的真元就消耗将尽了。

养再多的蛊虫,用不出来,不都是白养活么?

所以,蛊师修行中流传着一个说法。

养蛊就像养情妇。

养一个情妇,就得买吃的,买穿的,买房子等等。很昂贵,养多了实在耗费巨大,常人都养不起。

就算能养那么多,一个男人的精力总是有限的,又用不起。白白养着过眼瘾么?

蛊师境界提升了,但蛊虫的食物标准也随着提高了。所以,别看蛊师炼蛊没有数量限制,一般蛊师大体上只养四五头同等级的蛊虫。

要是数量再多点,蛊师就要破产喽!

\end{this_body}

