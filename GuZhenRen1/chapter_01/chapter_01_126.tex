\newsection{木魅之殇}    %第一百二十六节:木魅之殇

\begin{this_body}

%1
认输……

%2
方源的话,还回绕在众人的耳中。

%3
一时间,石楼下几乎所有的二转蛊师,都将惊异的目光集中在方源的身上。

%4
方源视若无睹,抱臂立在原地,一脸平静。

%5
“认输?居然直接认输了,我没有听错吧?”

%6
“那个就是方源?连出手的胆量都没有,呵呵。”

%7
“不得不承认熊力的气势,的确强大。但即便是输,总也得走个过场才是。直接认输,不仅是自己懦弱,更丢家族的脸面啊。”

%8
……

%9
渐渐地,窃窃私语声宛若平静的湖面上泛起的一阵阵涟漪。

%10
从起初的惊异之后,众多蛊师的目光,都变成了鄙夷、轻视、幸灾乐祸。

%11
不少古月一族的蛊师,都站立不安起来。来自熊家以及白家蛊师的目光,就好像是一根根无形的针,刺痛了他们心中的自尊心。

%12
方源是古月一族中人,他直接认输,连带着他们都感到颜面无光。

%13
“怎么可以认输呢?方源,你可是古月家的男儿,站出去,和熊力勇敢一战!”

%14
“不过,就算是输了也没有什么大不了的。”

%15
“连出手的勇气都没有,那才是真正的丢脸呢!”

%16
古月族的一些蛊师,都叫起来,撺掇着方源出去应战。

%17
方源面不改色,听着这些话就像是听一群狗在叫。

%18
所谓的名誉、脸面、荣耀,不过是一张画饼。上位者套在组织成员脖子上的枷锁罢了。

%19
这群蛊师,亦不过是一条条脖子上拴着绳索的狗。

%20
熊力定定地看着方源,忽然一笑:“真是令我失望啊。想不到古月家的勇武,就是这样子的吗?”

%21
这话让古月一族的蛊师们的脸色,都难看起来。

%22
熊家寨那边则传来一阵阵哄笑。就连白家寨的蛊师们,也用嘲讽的目光打量着古月一族。

%23
方源周围的人,亦是纷纷移动脚步。尽皆远离他,一副羞与为伍的神色。

%24
很快方源周围五步以内,就都空无一人。

%25
方源独自站在原地。我行我素,格格不入却神色平静。

%26
他人看中勇武的名誉,但方源却嗤之以鼻。这就不由自主地,引发了众人对方源的厌恶。

%27
因为方源对名誉的蔑视,是对大众价值观的否定。否定了这种价值观,就是否定了一直以来遵循着这种价值观生活的大众。

%28
他人自然不愿意否定自身的价值,觉得自己活错了。所以自然而然地,潜意识里就开始抵制方源,排斥方源。

%29
心灵不强大的人,会被这种排斥打败,从而改变自己,使自己大众化。

%30
但方源正需要这种排斥。他身上的秘密越来越多了,他需要形单影只。同时,这些人也没有结交的价值。记忆里,青茅山三大山寨虽然从狼潮中艰难地保存下来,但是两年后那场事故后统统灭亡。从此青茅山化成一片冰寂死地。

%31
方源所需要做的,只是趁着这段时间,尽最大努力成长起来,及时脱离家族,避开杀身之祸。

%32
眼看着如此不利的局面,作为古月山寨中二转第一人的青书。不得不站出来。

%33
“熊力,不如我们来切磋一下吧。”

%34
“嘿嘿,你想怎么比?”熊力虽然笑着,但是表情上却不免流露出凝重之意。

%35
古月青书并不看着熊力,而是微微抬起手掌,暗中催动真元,目光低垂注视着手掌中缓缓生长出的青藤。

%36
他缓缓地道:“就试试你的气力吧。如果你能从我青藤的束缚中挣脱开来,我就算输了。可好?”

%37
“嘿嘿,不错的提议,那就来吧。”熊力咧嘴一笑,眼中露着精芒。

%38
他暗忖:自己本身已经养了一熊之力,又有熊豪蛊爆发出的力量,叠加起来就是双熊之力。青藤虽然坚韧,但是依靠双熊之力绝对能挣脱出来。熊姜已经胜了白病已,若自己再胜青书,绝对大涨脸面!

%39
青书温和一笑,不再说话,只是伸出双掌,平摊开来,两棵青藤从手掌心中越来越长,如两条灵活的绿蟒,攀上熊力的身躯,缠绕在他的身上。

%40
熊力双手垂在腰间,青藤就绕着他的双臂,顷刻间缠了十几道,将他的两只手和腰背都紧紧地捆绑在一起。

%41
其他蛊师都目不转睛地看着这场对决。

%42
“请。”青书对熊力道。

%43
熊力顿时双目圆瞪,猛地攥紧双拳,开始发力。

%44
他的身上一块块的肌肉,突显出来,仿佛是巨石块垒。

%45
一熊之力!

%46
嘣嘣嘣。

%47
一根根的青藤,在他巨力之下,开始一根根的崩断。

%48
“哈哈哈,青书兄,看来这场切磋是我赢了!”熊力一边爆发着力量,一边仍有余力开口说话。

%49
“青书前辈……”一旁的方正看得双手冒汗,一脸紧张。

%50
其他古月族的蛊师,亦是如此。古月青书是他们的第一人,如果切磋再败,那这次古月一族的脸面就丢光了!

%51
“那可未必。”古月青书微微一笑,双眼透出一股自信的亮光。

%52
他话音刚落,手中的青藤就起了变化。原本碧绿的颜色,陡然间变成了墨绿色。同时藤蔓变得更粗壮,周边都开始长出了绿色的宽大藤叶。

%53
熊力面色一变,察觉到来自青藤上的绞力突然加重了十倍不止。

%54
更让他心惊的是,那些被他崩断的青藤,生长起来,又将断裂的地方链接在了一起。

%55
熊豪蛊!

%56
他心知不妙,连忙催动空窍内的熊豪蛊。

%57
顿时,他头发如钢针般直直地竖起。浑身的肌肉都鼓胀开来,整个人都似膨胀高大了许多。

%58
双熊之力!

%59
青藤嘎吱作响,在这样的恐怖力量之下,却仍旧坚韧不断,牢牢禁锢住熊力。

%60
熊力脸色涨得通红,使出了最大的力气,企图挣断青藤。但是最终。他也只是崩断了一根,就再也难以为继了。

%61
“我输了。”他主动撤掉熊豪蛊的加持之力,气喘吁吁地道。

%62
“承让承让。”青书拱拱手。收回了青藤。

%63
“青书大人,您真是好样的。”古月一族的蛊师们一片欢腾。

%64
“真不愧是青书前辈,哥哥和他比较起来。根本就不能比啊。”方正站在青书的身侧,看着后者,双目闪闪,带着一丝崇拜。

%65
熊力用复杂的眼光看着古月青书。自己在进步,收获了棕熊本力蛊,而青书也在进步。虽然不知道他用的什么手段,但这才是自己的对手啊,至于那个方源不过是一介新人,根本不足为虑。

%66
“果然是木魅蛊。”方源遥遥注视着青书,心中暗道。

%67
正面对着古月青书的熊力并没有发现。古月青书的头发上,已经长出了一两片新鲜翠绿的树叶。

%68
这正是使用了木魅蛊的一个迹象。

%69
所谓木魅,就是树精。

%70
使用了木魅蛊,就能化身为树精,进行战斗。

%71
树精是很奇特的生灵。它能直接汲取天地中的元气,化为己用。

%72
而蛊师是不行的,只能运用空窍中的真元。

%73
一旦使用了木魅蛊,化身为木魅树精,蛊师就能直接从空气中汲取天地元气,补充自身真元。等若是汲取元石中的天然真元。

%74
寻常蛊师战斗的时候。不可能分出心神来,一边战斗一边汲取元石中的天然真元。

%75
但是化身为树精,汲取天地元气就是一种本能,根本不需要分心。这就意味着,使用木魅蛊的蛊师,最擅长持久战。真元虽然谈不上取之不尽用之不竭,但因为有着不断地补充,导致战斗时间能暴涨三倍!

%76
同时一旦化为树精,对青藤蛊、松针蛊之类的蛊虫,也有一种增幅的作用。

%77
方源瞬间联想到了很多东西:“木魅蛊的合炼晋升,可以说是天底下最奢侈的晋升路线之一。它是三转蛊,需要和百年寿蛊合炼,晋升为四转百年木魅。再和千年寿蛊合炼,成为五转千年木魅。这个合炼的秘方,众所皆知,流传极广,但绝少有蛊师用上这个秘方。皆因寿蛊更加珍贵,寻到了寿蛊,通常蛊师都是直接用了,增长自身的寿命。”

%78
这个世界上,人类若无灾无病,最多只能活到一百岁。一百年,是人类寿命的极限。

%79
但是若用了寿蛊,就能增长寿命。

%80
百年寿蛊,就是增长百年阳寿。千年寿蛊,就意味着能多活一千年。

%81
寿蛊珍稀无比,为世人所共求。

%82
方源前世活了五百多年,就是先后使用了五只百年寿蛊。共增加了五百年的阳寿。本来连着他本来的百年寿岁,可以活到六百岁。但是在中途,遭到正派围攻,不得不自爆身亡了。

%83
寿蛊代表着“长生”,能让人活得更久。但不意味“不死”。

%84
“木魅蛊虽然强大,但是亦有缺陷。那就是蛊师每一次使用的时间,不能太久。太久了,木魅蛊的力量就会影响蛊师的身躯,使得肌肉木化,最终变成一具树人尸体。前世的古月青书,就是这样死的。”

%85
想到这里,方源的目光闪了闪。

%86
一般强大的蛊虫,弊端也会很大,需要和其他的蛊虫搭配起来使用。否则的话,会对蛊师本身产生不利的影响。

%87
熊姜的游僵蛊,最好和血气蛊搭配使用。否则使用过多,自身的血液越来越少,真的就僵尸化了。

%88
熊力一组回到自家阵营,他们先是战胜了白病已,又被青书所败。但即便如此,仍旧收获了许多掌声和赞叹。

%89
“漠颜,我要向你挑战。”

%90
“来来来,熊骄嫚,我来和你大战一场!”

%91
……

%92
继熊力小组之后,整个场面陡然间热闹起来。许多蛊师纷纷出马,挑战心目中的强敌,彰显自身的勇武。

%93
会盟坡上一阵混乱,切磋斗蛊精彩纷呈。

%94
真正的斗蛊切磋这才刚刚开始。

\end{this_body}

