\newsection{非方源莫属!}    %第一百七十二节:非方源莫属!

\begin{this_body}

%1
酒席上,一时间鸦雀无声。

%2
所有人都愣住,呆呆地看向方源。

%3
他们没有想到方源这么回答,直接承认,还如此坦诚。

%4
“还是太年轻啊,居然这种话都说出了口。”

%5
“哼,这话一说,就意味着整个政治前途的毁灭。”

%6
“今后这方源再也不足为虑了……”

%7
家老们心中思绪翻腾。

%8
狼潮进行到这里,大局已定了。接下来的几个月内,只会有小股狼群出没,并且随着狡电狈的不断召回,以及蛊师们的清剿,到了年末电狼将会基本消失。

%9
但争斗却从不会停歇。

%10
有人的地方,就有利益。有利益的地方,就有争斗。

%11
狼潮下,是人和狼的争斗。如今狼潮中最艰难的时刻已过,家族之间的内斗则露出水面,转为主要矛盾。

%12
狼潮冲击山寨,很多蛊师身死,旧有的势力被打破,这些势力原先掌控的利益,失去原主人,自然需要重新分配和瓜分。

%13
在古月一族的高层,原先众多家老,分食着整个山寨的利益蛋糕。但如今只剩下方源等不到十位家老,蛋糕却仍旧在那里。

%14
要瓜分这块大蛋糕,自然就需要较量。政治上的斗争,虽然没有狼潮这般刀光剑影,血雨腥风,但是阴谋算计,冷风陷阱,亦是深沉艰险。

%15
原本方源作为唯一的新晋家老。势头很足,风头极盛。但如今他坦然承认,自毁政治前程,就等若放弃了和其他家老竞争利益蛋糕的机会。

%16
一下子,方源在众家老心中的威胁程度,就降低了很多倍。

%17
看向方源的众多目光中,蕴含的压力明显在减缓。

%18
这时,族长古月博轻叹一口气:“既然方源你已经承认。那么身为族长,不得不对你此番的临阵脱逃进行处置。按照祖宗传下来的家法,临阵脱逃者,将剥削家老的职位。但最终的结果,将由我和其他几位家老联合商议,酌情处理。但不管结果如何,都希望你能够接受。”

%19
其他人脸色各异。

%20
方源点点头。没有开口,似已认命。

%21
古月药姬自断一臂。保住性命。这样的狠辣和决断。方源亦有。只是他舍弃的更多,将家老的身份都舍去了。

%22
他现在最大的麻烦,是突如其来的铁家父女。若是掺和到政治漩涡当中,受到政治倾轧,势必情形将更加危险。

%23
“舍得,舍得,取舍之间。就是人生。家老这身份,本来就是为了更好的修行。才拿来用的。我为了永生而踏上魔道,连命都可以舍弃。还有什么不可舍的?家老之位,哼……”

%24
方源心中没有一丝懊恼和后悔。

%25
这是最明智的抉择。

%26
而且,虽说会有惩罚,但力度必定轻微。

%27
毕竟现在家老稀缺,他身为三转蛊师,力量必须得到重视。族长一方面要惩罚,另一方面也要借助方源的力量,来稳住山寨大局,因此也要安抚他。

%28
至于其他家老,方源已经退出了这场关键性的政治博弈,已经没有威胁性了。更不会对方源赶尽杀绝,万一逼得方源反击,岂不是自找苦吃?

%29
“虽说有着家族制度,但制度是什么?呵呵,制度都是上位者维护利益的工具。一方面它主宰和分割下层群体的利益,另一方面也是上位者之间协调彼此的游戏规则。”方源心中冷笑,对于这些方面,他洞若观火,看得极为透彻。

%30
“现在最关键的,还是铁家父女。真是该死,居然比我预料中来得更快。狼潮还未退去,他们就到了。哼,不过这样的举止行径,倒是符合铁血冷嫉恶如仇,奋不顾身的性情。”

%31
一想到这里,方源就心生压力。

%32
尽管铁血冷受了伤,但其战力绝非方源可以比敌。瘦死的骆驼比马大,就是这个道理了。

%33
“我要离开山寨,又需要避免铁家父女的追捕,该怎么破局?”

%34
方源苦恼。

%35
这铁家父女,绝非贾富那般容易糊弄。再者,他方源修为低微,纵然有千般妙计,但没有实施的能力,为之奈何?

%36
三转和五转,这实力差距很大。

%37
“诸位我有话说。”就在这时,一直沉默的古月漠尘忽然开口。

%38
他脸色苍白,拖着重伤之体来此,也没有能饮酒,只是喝茶。

%39
但他接下来,却是语出惊人死不休:“有一件事情老夫必须坦诚,方源大人之所以未能及时出现战场,乃是老朽所为。”

%40
“什么?”

%41
一时间,其他家老都微微吃了一惊。

%42
“哦,此话怎讲?”古月博问道。

%43
方源亦投去一道目光,只是隐藏住了惊讶。

%44
这古月漠尘怎么会为自己说话?

%45
要知道自己和漠脉早有旧怨,之前甚至杀了一位漠脉家奴,分尸送礼过去。

%46
古月漠尘接着叹气:“实不相瞒。我的孙女古月漠颜已经深深地爱上了方源家老,在此之前,她亲自老求我,跪在地上哭泣,不愿看到方源去战死沙场。老夫就这么一个孙女,起了私心,就将方源强留在府内,迷昏了他。直到追击雷冠头狼,老夫才放其出去。所以千错万错。都是老夫的错。和方源大人没有关系。”

%47
“什么?”

%48
“哦,真的是这样?”

%49
一众家老都露出怀疑的神情。

%50
古月漠尘的话,也太扯了点,可信度一听就不高。

%51
“年轻人的情情爱爱,实属正常。”古月博点点头,意味深长地看向方源,想从他的脸上看出一些端倪来。

%52
但方源已经垂下眼帘,脸色平静。看不出任何喜怒。

%53
族长有些不大托底了。

%54
其他家老也在交换眼色,一时间没有搞清楚古月漠尘为何这么偏袒方源。

%55
他这样说话,为方源开脱,是在牺牲自己的名誉,毁坏自己的政治前途!

%56
古月漠尘接着道:“方源家老,为了维护老夫的名誉,甘愿自己承担骂名。但老夫之前已经错过一次。怎么能再错一次。真相就是这样,该怎么处罚。请族长大人明示。老夫认罪伏法。若是要剔除家老身份。老夫也是甘愿。”

%57
族长连忙摆手:“漠尘家老劳苦功高,这么做也可理解,人非草木,孰能无情?如今家族百废待兴,正需要您这等肱骨之臣。只是这到底是私情,如何处罚,还需要商量。今日有贵客临门。先不说这些,来。铁兄,在下和全体家老一齐敬您一杯!”

%58
说着。古月博就站起来。

%59
其余家老也跟着站起,举起酒杯。

%60
“诸位客气了。今后还要叨扰诸位,希望诸位能够体谅。”铁血冷不端架子,亦站起来,饮下一杯酒。

%61
……

%62
漠脉大院内,草木芬芳,假山清泉,流水潺潺。

%63
夜空中,明月如盘而高悬。

%64
酒席早已结束。

%65
方源端坐在院中的凉亭内,倾听着耳边的泉水之音,面色平淡地放下手中的茶杯。

%66
在他对面坐着的,正是古月漠尘。酒席散场后,他就邀请方源来此坐谈。

%67
“来,再喝一杯茶,这竹叶青水茶,正好解酒。”古月漠尘微笑着,亲自为方源斟水。

%68
方源神情平淡,看着茶水又添满,只说了一声:“的确是解酒的茶。”

%69
说完,他就把视线移向亭外。

%70
只见明月苍白,洒下一片清辉。月下庭院静谧幽雅,但风中阴影斑驳,隐约可见这赤脉大院的正由盛转衰的落魄气象。

%71
在酒席期间,方源已经窥破了古月漠尘的想法。现在看这景象,更是心中笃定。

%72
漠脉没人了!

%73
古月漠北一死,整个漠脉就失去了继承人。

%74
虽然漠北还有亲姐姐漠颜,但家族体制,祖宗传法,历来重男轻女,家业亦只传男不传女。

%75
就算将来古月漠颜,成了三转,晋升为家老。但她的家业,却不代表漠脉正统,只属于她自己。若将来嫁人,这家业就转为她的夫君名下。

%76
一个家族中的政治势力,若丧失了继承人,那就是没有了前景,不会有人追随的。

%77
漠脉已经陷入了艰难处境,面临着崩溃的危机。

%78
但天机常留一线,漠脉并非绝境,还留有一丝希望。

%79
这丝希望就在古月漠颜的身上。

%80
她虽是女儿身,但却可以嫁人。

%81
这女婿若是入了漠脉的门,在身份上就能说得通,也能继承漠脉!

%82
看方源没有开口的迹象,古月漠尘心中暗骂一声小子狡诈,却不得不首先开口:“不知道方源家老,对我漠脉如何看法?”

%83
他身上有说不得的苦衷。

%84
自从重伤之后,他的修为就已经落到二转境地,再无一丝重回三转的可能。

%85
他现在只是靠着蛊虫,遮掩了真实气息。但纸遮不住火,总有暴露的一天。

%86
唯一的继承人已经陨灭,自己落到二转,也会失去家老身份。古月漠尘现在急需一位撑得住场面的外援,来镇守住场面。

%87
昔日,他位高权重,漠脉占据了庞大的利益蛋糕。如今家族势力重新洗牌,他不求更多的利益,只要将手中的这部分守住,就是最大的胜利。

%88
他左思右想,最理想的人选非方源莫属!

\end{this_body}

