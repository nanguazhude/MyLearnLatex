\newsection{熟能生巧掂元石,酒肆当中恶客来}    %第一百零七节:熟能生巧掂元石,酒肆当中恶客来

\begin{this_body}

寒雪梅中尽,春风柳上归。

不知不觉,冬天已过,春天到来。

冻结的山溪,重新开始潺潺的流动。竹楼屋檐下的冰锥,树挂,都在阳光中滴着晶莹的水。www.13800100.com

早晨,酒肆中有些冷清,并没有多少酒客。

方源坐在里面,靠近窗户的位置上。应他的要求,位置周围用木板屏风竖着,做成了一个隔间。

一阵微风从窗户外吹进来,夹裹着一阵清新而又芳香的泥土气息,让人闻之心旷神怡。

江牙就坐在方源的对面,堆着满脸的笑。

“这是这次的元石,请您查收。”他将装的满满的四个钱袋,放在桌上,推给方源。

钱袋子里,装的自然都是元石。

方源没有一一的打开来看,而是放在手中称量了一下。

他前世做过近一百多年的买卖,多少块元石放在手中掂量掂量,心中就有数了。少一块元石,他都能立即察觉到。

这本事也没有什么。

地球上,有卖油翁,往瓶口处摆个铜钱,他滴油进去,油成一线,透过铜钱中间的小孔,丝毫不溅出一滴来。又有神射手,于百步外射树叶,百发百中。干了多年的老屠夫,用手称称肉,就知道几斤几两,用秤一称量,分毫不差。

怎么练成的这等本事?

无他,熟能生巧罢了!

经验的积累。有时候就能堆砌起一场奇迹。

重生之后,这种源自经验的手感自然也带了过来。方源用手分别称了称,发现没有问题,便从怀中取出一个小布袋,将其递给江牙。

江牙连忙双手接过去,打开袋口,仔细清点。

方源虽然有九叶生机草在手,但他并不直接贩卖。要是古月冻土。巴不得这样做,有利于维系社会关系,增加他的影响力。

但是方源却不愿意这么干。这样做,实在太浪费时间和精力了。所以他将生机叶都卖给江牙,江牙作为商铺的店主,去对外贩卖这些一转治疗草蛊。

江牙是江鹤的弟弟,方源在寻找酒虫的时候。就和他见过面。他的哥哥江鹤,更是方源的半个盟友。因此让他做代理。出面贩卖。是比较安全可靠的。

“一、二、三……九。的确是九片生机叶不错。”江牙数了三遍,这才闭上袋口,将袋子小心地贴身收起来。

然后他举起酒杯,向方源敬道:“方源大人,合作愉快,我敬你一杯!”

他看向方源的目光中,隐藏着浓厚的羡慕。甚至化为了一丝嫉妒。

就在一年之前,也是春天。他第一次见到方源,那时方源还只是学堂的一介学员。连蛊师的武服都没有资格穿着。

但是如今,方源不仅一身武服,同时腰间已经系着赤色腰带,腰带中间镶着方形铁片――已经是二转蛊师了!

而他现在却仍旧只是一转,戴着青色腰带。

这些倒也罢了,更让他江牙眼红的是,方源得了遗产之后,一跃从穷小子成了一名富翁。

他掌握的酒肆、竹楼还有九叶生机草,都是他江牙一辈子打拼,也可能打拼不到的财富!

不过,江牙却不敢将这嫉妒的情绪表现出来。

方源将生机叶售卖给他,而他赚取其中的差价。方源已经成了他的金主,江牙如今可不敢得罪眼前的这个晚辈。

“唉,人比人气死人啊……”江牙举着杯子,脸上堆着笑,心中却是深深地叹息着。

方源也举起杯子,然后一饮而尽。

江牙的神情虽然隐晦,但是他年老成精,怎么可能看不出来?

方源并没有放在心上,江牙若不嫉妒,说明他心怀远大,反倒是能让方源高看一眼。

不过他盯着方源的这份际遇而眼红,单单这份格局就显得小了,不足挂齿。和他喝杯酒,只是他目前有些许的利用价值罢了。

江牙放下酒杯,兴奋地道:“家族中的生机叶,每一片要卖五十五块元石。听了您的吩咐后,我们的草蛊,只卖五十块元石,果然是供不应求!大人,不如你每天都催生一些生机叶,这样一来,我们就能赚得更多了!”

方源听了缓缓摇头,断然拒绝:“不行,催生出九片生机叶,已经是极限了,浪费了我许多修行的时间。”

这就是方源和江牙这等俗人的区别了。

在方源看来,元石不过是修行的资源,是道具。一切都是为修行服务的。而江牙则将元石当成了人生追逐的目标,之所以修行就是为了赚取更多的元石。

不过即便方源每天只催产九片生机叶,每天赚取四百多的元石,经过这些天的积累,他手中的财富也上涨到了一种可观的程度。

见方源拒绝,江牙也不敢强劝,只好可惜地咂咂嘴,殷勤地给方源斟酒,然后再给自己倒上。

“也是。”他感慨道,“大人您坐拥如此财富,何必每天劳心劳累。照我看呀,大人您又何必仍旧住那破旧的租房呢?不如把一座竹楼腾空,自己住了。再娶一个貌美的媳妇,招揽七八个家奴伺候着。这日子过得就美满了,啧啧。”

方源轻笑一声,没有言语。

燕雀安知鸿鹄之志!

他转过头,看向窗外。

一栋栋竹楼,顶出残雪,沐浴在明媚的春光下。远处有一株柳树,舒展着黄绿嫩叶的枝条,在微微的春风中轻柔地拂动。

方源目光有些失神,他想着自己如今的处境。

解决了方正的问题后,家产可以说已经保住了。

白玉蛊、月芒蛊都已经合炼成功。可以说已经攻守兼备。接下来就是合炼酒虫。

但是酒虫这事情比较麻烦,为了合炼出四味酒虫,他必须要有第二只酒虫,以及酸甜苦辣四种美酒。这些东西,大部分他都没有头绪呢。

“酒虫是必须要合炼的,没有酒虫,我的进度就会减慢一倍不止。但是要合炼四味酒虫,至少得等到商队的到来。借助商队的机会。我也可以暴露出白玉蛊。这样一来,我的战斗力就能展现出来,不必要缩手缩脚了。”

方源现在有月芒蛊、白玉蛊在手,再搭配上他五百年的战斗经验,已经凌驾于大部分的二转蛊师。

病蛇角三这等小些名气的组长,若是和他单打独斗,也并非方源的对手。

但是对于赤山、漠颜、青书一流。方源还是弱的。

一来是修为不足,方源只有二转初阶。而他们却都是高阶甚至巅峰。二来是强力的蛊虫不够多。方源用作战斗的蛊虫。只有两只。而赤山、漠颜、青书一流,至少都有三只,同时还有雪藏的底牌。

而同龄人中,方正、漠北、赤城都毫无意外地开始展露头角了。

尤其是方正,拥有了二转蛊虫月霓裳之后,已经有了和方源一战的实力。并且随着时间的流逝,他的修为会越来越高。渐渐地将方源甩远。

除非方源尽快地合炼成四味酒虫,才能在修行的速度上跟得住方正。

至于更高一层。那些三转、四转的蛊师。

对于方源来讲,不提战平。就算是保住性命也是一件难事,更别提什么越级挑战了。

越级挑战是很难做到的,方源一没底牌,二没资质,就算是有丰富的战斗经验,但巧妇难为无米之炊,没有强力的蛊虫,这些经验根本体现不出什么优势来。

“如果四味酒虫能合练出来,修行的速度就比较满意了。但是还需要补充一些蛊虫,防御有白玉蛊,进攻有月芒蛊,治疗有九叶生机草。还缺侦测类以及移动类的蛊虫,这两种蛊虫虽然只是辅助,但是一旦拥有,就补全我的短板,能让我的战斗力至少翻上三倍!”方源思量着。

他不需要在实践中一步步的认知,丰富的人生经验,已经让他知道自己的准确定位。

耳边,江牙的声音传来:“我听说了,最近似乎有人在找大人您的麻烦?专门在您出租的竹楼和酒肆里闹事?”

方源皱了皱眉头,被打断了思绪。

不过,江牙说的倒是没错。

方源也已经调查清楚了,这是舅父古月冻土在后面挑事。

他被古月青书警告过之后,再也不敢利用方正来闹事。但是沉寂了一段时间后,他怀中不甘和愤恨,利用他的关系网,雇了一些蛊师来闹方源的场子。

做生意的,最怕这种麻烦。

所以他最近,都抽空过来看看场子。

“少东家,又有人来闹事了。”就在这时,一个伙计带着焦急害怕的神色,走进隔间。

“哦?”方源眉头一展,想不到今天被他撞到了。

不待方源有所动作,江牙却腾的一下站起来,殷切地道:“大人您且稍作,待我去看看。”

他走出屏风,然后眨眼间就转身回来。

“是古月蛮石!”江牙脸色变白,压低声音道。他的目光中也流露出惊恐担忧之色。

古月蛮石?

方源行事谨慎,深知情报的重要。这段时间,已经将家族中全部的二转蛊师的情报都收集到手,并且大略都记住了。

古月蛮石是个略有声明的蛊师,擅长防御,有一把力气,担当着蛮石小组的组长。论名声地位,比病蛇角三还要稍强一筹。

砰!

从外传来酒坛砸碎在地上的声音。

随后一个粗犷嚣张的声音响起:“呸,你这是什么酒?像马尿一样,也敢卖给大爷我喝?”

“哼!”方源眼中寒芒一闪,站了起来。

\end{this_body}

