\newsection{君子的谎言}    %第五十七节:君子的谎言

\begin{this_body}

%1
贾富很纠结。

%2
他现在心中已经排除了方源的嫌疑,几乎确定贾贵就是幕后黑手了。

%3
“但我就算知道了真相,又能如何呢?”贾富的心头涌起一股悲愤,“我手中没有任何的证据,若是我空口无凭地到父亲面前指证贾贵,恐怕父亲还认为我要陷害他呢!”

%4
贾富很精明,他看向方源,双眼中精芒一阵爆闪。

%5
贾金生是随着他一起走南闯北,贾金生如今失踪,他贾富自然就有照顾不周的责任!既然指证不了贾贵,那么他就必须给父亲有个交代。

%6
而这个交代,就在眼前!

%7
“不错,若把这个方源当做替罪羊,也算能勉强交代过去。只要过了这个坎儿,我就有可能再双倍地讨回来。”贾富心中恶念顿生。

%8
他猛地提高声调,又对方源厉声逼问:“方源,你怎么证明你没有暗害贾金生?”

%9
众家老不由地一愣,这事情明摆着是你贾家内斗,怎么还抓住我族人不放呢?

%10
唯有古月族长面色一沉,目光转厉,紧紧地盯向贾富。

%11
“方源,你有什么证人,证明你不在场,没有时间暗害贾金生?若没有的话,你就是凶手!”贾富一手指向方源,怒目圆瞪,气势逼人。

%12
“他贾富这是要把我族的方源推出去,当做替罪羔羊啊。真是岂有此理!”这一会儿,众家老也都反应过来了,各个脸色都变得不善。

%13
他们长期勾心斗角,稍稍一想,也就领会到了贾富的立场和打算。

%14
“证人?当然有!我早就安排好了。”方源暗自一声冷笑,表面却作出一副百口莫辩,张口欲言却说不出话的逼真神情。

%15
“其他的都不用说,你只要告诉我有没有!”贾富声音再度拔高,逼压方源。

%16
方源一副愤慨不平的模样,最终似无奈地从牙缝中挤出两个字:“没有。”

%17
“哈哈。那你就是——”贾富刚要大声宣判,就在这时。

%18
“慢!”学堂家老一个跨步,挡在了方源的面前,一脸肃容,“他当然有证人,我就是他的证人!”

%19
“你?”贾富惊疑。

%20
“不错,就是老夫。”学堂家老面对四转的贾富,语气有点虚。但是看到族长古月博投来的鼓励目光,他底气顿足,把头一昂,“这些天方源意外地率先晋升中阶,我便差人暗中调查。他的每一天的行踪都记录在案,根本就没有暗害贾金生的时间。”

%21
“对,就是这样……”方源隐藏在学堂家老的背后,谁也没有看到他此时的嘴角,微微勾勒出的一抹笑意。

%22
贾富脸色铁青,他没有料到学堂家老会忽然站出来,担保方源。

%23
更关键的是,古月族长也没有反对。这意义可就重大了,代表着古月一族要保这个方源。

%24
“我懂了!我一心想要让方源当做替罪羊,只是从自身出发,却没有考虑到他们的感受。不错,方源一被顶罪,古月一族就要承担谋害贾家族人的恶名。今后就要面临贾家的报复,名誉还会折损,未来商队也不敢过来交易,那损失就太大了!”

%25
想到这点,贾富懊恼地恨不得拍打自己的脑门。

%26
古月高层,正是如此的想法。

%27
方源只是丙等,若真是他害了贾金生,把他交出去,也不算什么。但是关键是,现在他的嫌疑已经被洗净了,若还把他交出来,古月一族岂不是要蒙受大量的,不必要的损失么?

%28
知道这个矛盾已经不可调和,贾富咬了咬牙,决定坚持到底,他开口道:“既然如此,那不妨让我动用一下足迹蛊。此蛊用了,便可在地面上显示出最近三万步的足迹。”

%29
学堂家老立即不悦地冷哼了一下。

%30
贾富说这话,明显是不相信自己。但他也没有阻拦的道理,于是侧身让过。

%31
“来测吧!”方源望着贾富冷笑,昂首走到他的面前。

%32
他自信十足,因为早就料到了这个因素。因此这些天,都缩在山寨中活动,石缝秘洞根本就没有涉足。

%33
在古月高层的密切注视之下,贾富没有耍弄花样。

%34
足迹蛊形状十分特别,就像是人的一只脚。它的材质就像是一块半透明的冻乳,给人滑嫩的感觉,表面还泛着一层黄绿色的荧光。

%35
体型倒是不大,只有掌心大小。

%36
贾富拿在手中,真元喷涌而出,灌注到足迹蛊之中。

%37
足迹蛊越来越亮,忽然砰的一声轻响,炸成了一大片黄绿荧粉。

%38
荧粉呼的一下,罩住方源,在他的身边旋转一圈,然后就飞出了议事堂的大门。

%39
在荧粉飞过的一路,地上顿时就显现出一连串的脚印。

%40
这些脚印都散发着黄绿色的荧光,大小和方源的两只脚一模一样。正是他刚刚进来议事堂的足迹。

%41
足迹一直从家主阁延伸出去,到达学堂宿舍,然后就在学堂这块绕圈圈。除此之外,就是到达山寨的客栈。

%42
黄绿荧粉越飞越少,最终在第三万个足迹上,彻底消失。

%43
结果出来,众人查看,顿知方源一片清白,毫无疑点。

%44
贾富深深地叹了口气,又从怀中取出一方小巧的玉盒。

%45
他把玉盒打开,玉盒中只存了一个玉片。

%46
玉片呈现半透明的翠绿色,里面则封印了一只蛊虫。

%47
这是一只竹节虫,它身躯纤细而又修长,如碧玉一般的颜色,整个身躯就像是一段圆溜溜的竹管。

%48
竹节虫一般体长都要超过一个巴掌,但是这只却不是,只有指甲盖的长度。在它的表面,还微微散发着白色的光芒。

%49
“青玉为躯,白华覆体,这是竹君子啊!”当即,就有家老认出了这蛊虫,惊叹一声。

%50
就连族长古月博都动容了,不禁劝道:“贾老弟,这竹君子乃是四转蛊虫,炼之不易。何必要浪费在此处呢?”

%51
贾富摇头,看向方源:“这竹君子是我年少时,一次偶然解石而得。石头只解了一半,没有再解下去。众所周知,这蛊虫以真诚为食,天生就能测谎。只有从小到大都没有说过一句谎言的至诚君子才能炼化、喂养这蛊。”

%52
“方源,你只要把石头解开,将虚弱的竹君子收到空窍当中去。我问你什么,你就答什么。然后再把此蛊取出来,让我们大家看看这虫有无变色。只要变色了,就说明你在撒谎!”

%53
“毫无问题。”方源没有一丝犹豫,当即解开玉片,按照贾富所说的做了。

%54
竹君子一出现在空窍之中,立即散发出微微的绿芒,照彻整个真元海。

%55
方源顿时感到,只要他说出一句谎话,这竹君子就能感应到,从而身躯由绿色变成其他颜色。

%56
但是他之所以应承下来,也是有底气的。

%57
“春秋蝉!”他一个念头,沉眠中的春秋蝉顿时苏醒过来,散发出一缕气息。

%58
这气息恢弘无比,立即死死地镇压住竹君子。

%59
竹君子散发出的绿色光芒,顿时咻的一下缩回到体内。整个身躯都蜷缩起来,害怕得瑟瑟发抖。哪里还有余心余力能感应谎言?

%60
贾富开始发问,他的第一句话就直指中心:“方源,你有没有害我的弟弟贾金生?”

%61
“没有!”方源斩钉截铁地答道。

%62
贾富又问:“你知不知其他关于他的消息?”

%63
方源摇头:“不知。”

%64
贾富再问:“你刚刚对我们大家说的话,有无不实之处?”

%65
方源再摇头:“没有。”

%66
“好了,你可以把竹君子取出来了。”三句问完,贾富道。

%67
方源将竹君子取了出来,众人望去,只见竹君子表面仍旧是一片碧绿之色,毫无变化。

%68
一干家老都松了口气。

%69
贾富的面色也缓和下来,他收好竹君子,向古月博一拱手:“这次多有得罪了,古月兄。”

%70
“无妨,能让真相大白,也是我们想要看到的。”古月博摆摆手,随后又叹息一声,“只是可惜了这只竹君子。”

%71
竹君子有测谎之能,品阶高达四转,价值非常之大。但是喂养、炼化都十分不易。它必须是由至诚君子才能炼化。只要其他蛊师,说过一句谎话,炼化竹君子必定失败,竹君子也会当场死亡。

%72
它的食物,就是真诚。至于寄居在至诚君子的空窍当中,食用君子的真诚才能生存。

%73
现在这竹君子被开出来,虚弱至极,但是却没有食物来恢复元气。又经过方源刚刚这一使用,死亡已经是定局了。

%74
贾富却摇摇头,望着手中将死的竹君子,似乎并不可惜。

%75
他沉声道:“此事我已经尽了全力调查,可惜力有未逮。这次回转家族,我会重金聘用捕神铁血冷,一定会将此事调查清楚!告辞了。”

%76
说完对古月博一拱手,转身便走,干净利落,倒也显得有些风采。

%77
望着贾富一干人等离去的背影,古月博长长地舒了口气:“你们也都可以走了。”

%78
他向众家老挥挥手,仿佛想到什么,又道:“学堂家老留下。”

%79
没有少一根汗毛,方源安然无恙地走出了家主阁。

\end{this_body}

