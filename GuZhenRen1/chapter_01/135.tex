\newsection{北冥冰魄,十绝天资}    %第一百三十五节:北冥冰魄,十绝天资

\begin{this_body}

%1
“什么?!”熊林顿时失声。

%2
“强取蛊?青茅山上只有一人有这蛊虫,就是我的表弟熊毡!!”熊姜的脸色也顿时阴沉下来。

%3
熊力捏紧了双眼,眼中凶芒闪烁,别人怕白凝冰,他当然也忌惮。但是忌惮并不代表软弱和惧怕。在他的心中,早有为家族捐躯,死战白凝冰的觉悟!

%4
“白凝冰,强取蛊到底在不在你的手上?”他大步跨出,任由方源错身而过,直面白凝冰。

%5
白凝冰呵呵冷笑:“在又如何?”

%6
熊姜脸色阴沉如水,质问道:“说!我表弟究竟是怎么死的?他的蛊虫,为什么在你的手中?”

%7
熊林亦道:“白凝冰,我敬你是前辈。但是你私自吞没了我们熊家寨的蛊虫,却不上缴。你这是公然违反盟约!”

%8
说罢,扬手一只信号蛊。

%9
砰的一声,彩色的焰火在空中迸发。

%10
烟火的彩光映照在白凝冰的脸上,他哈哈大笑:“什么狗屁的盟约,我从未放在心上。只是看有人用了强取蛊,过程有趣罢了,才拿来收藏的。”

%11
熊姜听了这话,再也按捺不住心中的愤怒。低沉地怒吼一声,奔杀向白凝冰:“白凝冰,你眼睁睁地看着我表弟战死,却不出手。你这个混蛋!”

%12
小组五人,俱是一体。于公于私,熊力小组都要和白凝冰势不两立。

%13
战斗难以避免。场面顿时一片混乱。

%14
一方面,是白凝冰遭受着熊力小组五人合力的进攻。另一方面,他们身处在狼群的包围当中,仍旧要面对电狼的撕咬。

%15
方源脱离战场,站在远处好整以暇地观望着。

%16
狼群攒动,好像是一个大磨盘,六位蛊师忘我地拼杀着,稍有不慎。就是落入狼口的凄惨下场。

%17
熊姜催动了游僵蛊,双瞳一片惨绿。熊力则双目赤红,白凝冰的双瞳如水晶一般湛蓝。这三人之间的交手,成为整个战场上最激烈的地方。

%18
在游僵蛊的力量下,熊姜化身僵尸,对水、冰的防御力暴涨,堪堪能挡住白凝冰的进攻。而熊力则催动熊豪蛊。双熊之力下,砂钵大小的拳头威猛无俦。两三拳下去。哪怕是白凝冰的水球护罩,也要被轰散。

%19
至于其余的三位熊家蛊师,则难以插手这边的战斗,只能竭尽全力地应付着电狼。那只被方源引来的豪电狼,本来能轻松拿下,但是如今却也成了一个巨大的威胁。

%20
“白凝冰,你要为你的所作所为付出代价!”熊姜嘶吼着。冲向白凝冰。

%21
“哼,就凭你?”白凝冰冷笑一声。灵活地往后一跃,拉开距离。同时左手一甩,甩出五根手指大小的冰锥。

%22
冰锥正中熊姜的身躯,但熊姜却没有感受到丝毫的疼痛。当他化身僵尸的时候,哪怕是断手断脚,也不会有任何的痛感。

%23
冰锥的寒气,能冻得正常人行动迟缓,但他却感到一阵冰凉的舒爽。僵尸本就是阴体,对火焰、雷霆、烈日不堪对抗,但是对这等冰寒之气,却有足够的承受能力。

%24
“白凝冰,你这样的处境,还想玩弄我们?使出你的真正实力吧!”熊力怒喝一声。

%25
交战一来,白凝冰一直压制着自己的修为,只呈现二转蛊师的实力,同时使用的蛊虫也大多都是二转。

%26
这让熊力感到自己遭到了蔑视,由此产生的屈辱,更令他怒火熊熊燃烧。

%27
“呵呵呵,你们这些小角色,怎么有资格让我使出全力?”白凝冰冷笑几声,攻势越发凌厉,但仍旧压制着实力,没有动用一只三转蛊虫。

%28
站在远处,抱臂旁观的方源,心中却是一片了然。

%29
“他不是不想用,而是不能用。白凝冰,呵,他可是北冥冰魄体啊……”

%30
按照这个世界上最古老的传说,所有的人类,都是人祖的子孙后代。

%31
但俗话说的好,龙生九子各有不同。绝没有两个相同的人,哪怕是双胞胎,也有着差异。

%32
在这个世界上,人们最关注的差异,就是资质。

%33
有修行资质的人,成为蛊师,就能成为人上人。没有修行资质的,就是凡人,生活在社会中的最底层,被践踏被玩弄。

%34
修行资质分为四等,甲等、乙等、丙等、丁等。这一点已经为世人所知。

%35
但实际上,甲等之上,还有更优秀的资质。

%36
关于这个消息,就是秘闻了。家族从不会大肆宣传,只有社会地位达到一定程度的人,才会得知。

%37
熊力等人当然不会知道,甚至家老、族长都未必知道。但方源前世高达六转,已经脱离凡体,成为蛊仙之流,自然对此心知肚明。

%38
这个甲等之上的资质,有十种,被统称之为十绝体。

%39
“人祖在彻底死亡之前,前前后后一共生有十子。大儿子太日阳莽,二女儿古月阴荒……里面就有一子,名为北冥冰魄。人祖的传说,似真似假,影射了蛊师修行中的许多秘密。人祖十子,分别代表着十种绝顶的资质。”方源回忆着。

%40
“十绝体中的任何一种,都凌驾于甲等资质。最优秀的甲等资质,空窍真元能存储到九成九。但任何一种十绝体,空窍中的真元都是圆满的十成!”

%41
“但是,万物平衡,十成真元的十绝体太过于完美了,天地都不容许他们的存在。人祖的故事中,他的十位儿女都没有长寿的例子。现实中,身怀十绝资质的蛊师,则几乎都是英年早逝,很难成长起来。当然,若真正成长到六转,则必能横扫同阶,甚至在六转时。还能做到越级挑战的奇迹!”

%42
“拥有北冥冰魄体的白凝冰就是如此。十成的真元,让他的空窍不堪重负,随时有崩塌毁灭的危机。为了降低这种危机,白凝冰就必须修行,耗费真元温养空窍窍壁,让其底蕴增厚,能承担住十成的真元。因此他修行精进的速度出类拔萃,十分惊人。”

%43
“然而。修为越高,真元的质量就越高,对空窍的压迫力也随之增大,反而令本身的危机扩大。白凝冰就像是孤舟飘荡在海面上的遇难者,他没有淡水,只能喝海水解口,但是海水是咸的。并且会吸收身体内原本的水分,令他越来越渴。”

%44
“白凝冰越是修行。就离自我的毁灭越来越近。但他又不能不修行。家族的期盼,来自熊家寨、古月寨的暗杀,逼迫他必须越来越强。身为北冥冰魄体的他,想必也明白自己的处境。知道自己时日无多,必死无疑,所以才会养成了这样的性格啊。”

%45
想到这里,方源心中叹息一声。

%46
这无疑是一种讽刺。

%47
过于优秀的资质。没有令蛊师们平步青云,反而是他们灭亡的罪魁祸首。

%48
过犹不及。人都需要喝水吃饭,但是吃喝多了。就会撑死。

%49
从另一种角度理解,不管任何世界,都没有真正的完美。没有真正完美的爱情,没有真正完美的作品。

%50
过于完美的话,就会引来毁灭。

%51
在方源的前世,狼潮过去之后的三年后,白凝冰的修为不可避免地达到了四转境界。终于让他的空窍无法承受真元的重负,轰然自爆。

%52
十绝体不容于天地,逆天之物的自爆,像是一声绝唱,威力不同凡响。直接将三寨中人全部杀死,将整个青茅山都化为绝死冰域。

%53
幸好当时,修为平平的方源,被方正挤兑刁难,只好参加了商队外出,这才侥幸地躲过一劫。

%54
为了拖延死亡的来临,白凝冰主动利用蛊虫,将三转的白银真元稀释成二转的赤铁真元。同时也极少使用三转蛊虫。

%55
皆因催动一次三转蛊虫,将消耗大量的赤铁真元,造成后力不济。反而不如连续使用多只二转蛊虫,在战斗中,对他白凝冰的帮助更大。

%56
这才是白凝冰压制修为的真正原因。

%57
否则明明有实力,却偏偏自缚手脚,把自己陷入险地,那是傻缺脑残才会干的事情。

%58
白凝冰聪颖灵动,又被大力栽培,受到良好的教育,怎么可能做出这样的傻事呢?

%59
只是他如此年轻,这样的性格的确异于常人。不过对于将死之人,他行事之间还有什么好顾忌的呢?

%60
就是这样的毫无顾忌、肆无忌惮,才没有被组织的体制同化,让他养成了一颗魔道之心。

%61
否则,以他的这种生存环境,万众期待、大力栽培、荣耀包拢、强敌环伺、前途光明,早已经被组织同化,养成了领导者般的心性。

%62
白凝冰其实是个很可怜的孩子,方源先前并未想过要对付他。只是他既然要追杀方源,那方源也不介意先利用他一把,再提前将这个祸端除去。

%63
场中的战斗仍旧在继续着。

%64
这么一会儿工夫,局势已经发生了转变。

%65
豪电狼被白凝冰消灭,狼群溃散奔逃。熊力小组中的治疗蛊师也倒在了白凝冰的刀下,而他为此也付出了沉重代价。右臂被熊力狠狠击中一拳,似乎是骨折了,在接下来的激战中一直无力地垂着。

%66
但是这一些,都并不妨碍他现在占据上风。

%67
熊力是二转蛊师中第一流的精英,实力和青书、赤山相当。熊姜是新近崛起的防御能手,熊林是那一届的天才新人,如今也是二转战力。再加上另外一位蛊师,四人合力,却仍旧被白凝冰压制。

%68
要知道,白凝冰先前已经战斗了一场,耗费了不少赤铁真元。同时他出手斩杀了豪电狼,杀了治疗蛊师,右臂丧失行动能力,只能舍弃惯用的右手冰刃,转为左手。这样一来,他用左手催发冰锥的手段,就只能暂时搁置了。因此进攻能力,几乎就降低了一半。

%69
在这样的情况下,他仍旧占据上风,并且优势越来越稳固。

%70
“到底是北冥冰魄体啊,虽然是用蛊虫稀释了真元,也封印了北冥冰魄体的真正优势,但是真元恢复速度,却还保留着。战斗越久,他的优势就越大。”方源看着,暗暗感叹。

%71
“凭借我如今的战力,还不足以击败了他。”对于这点,方源坦然承认。

%72
方源只有丙等资质,四成四的真元。就算是乙等资质的熊力、熊林、熊姜,合力作战之下,也要被白凝冰压制住。可想而知,方源独自一人,面对白凝冰,情况只会比熊力他们更艰难。

%73
“但是,不能败你并不代表,不能杀你啊。”想到这里,方源一声冷笑。

%74
这就是五百年经验积淀出的智慧了。

%75
尽管是有些相似的影子。

%76
但和他这个百年老魔相比,白凝冰不过是个被残酷的命运,提前催熟的小魔头罢了。

\end{this_body}

