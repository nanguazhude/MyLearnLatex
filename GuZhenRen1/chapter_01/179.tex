\newsection{历史由人书写}    %第一百七十九节:历史由人书写

\begin{this_body}

“花酒行者跪地求饶,四代族长心慈仁善,不妨他突然偷袭。四代族长将花酒行者当场击毙,但自身亦受重创,不久逝世!哀哉!魔道中人,果然背信忘义,翻脸无常……”

铁若男读着这行内容,双眼却渐渐地黯淡下来。

“可惜。这花酒行者是被当场击毙,不可能留下传承了。若是他能留下传承,这时间刚好对上呢。”少女心中叹息。

她并不放弃,又将古月史典细心翻看,直到时间耗尽,古月药姬来催促。

铁家父女一走出家主阁,方正就从迎了上去:“有什么发现吗?”

铁血冷沉默,铁若男苦笑着摇摇头。

但方正却道:“我刚刚想起来一件事情,也许对你们会有帮助。曾经哥哥和赤脉走的很近,赤脉的家老古月赤练曾当众维护过他。铁姑娘,你说,这酒虫会不会是赤练私下里,送给哥哥的呢?”

“赤脉的人?”铁若男皱起了眉头。

“没错。在狼潮之前,赤脉、漠脉是我们古月一族,最大的两个势力。”方正解释道。

怎么又牵扯到赤脉了?

铁若男沉吟不语。

她感到头疼。

这是她第一次破案,原本自信满满,但真正自己实施起来,却感到困难重重。

以前看父亲破案,如抽丝剥茧,有条不紊,条理分明,简直是水到渠成。但现在当她自己实践,这才知道破案的艰难。

有时候毫无线索,有时候各种线索胡乱涌现,叫人不知所措。

真相一直笼罩在迷雾当中,她也不知道自己是距离真相更近了,还是更远了。

赤脉的这条线索,究竟有没有用呢?

铁若男不禁感到茫然,她下意识地看向父亲铁血冷。

“父亲一定已经掌握了线索了吧。”少女心中对父亲的佩服之情,越加深厚。

“若换做父亲,这个案子应该早就破了。我距离父亲,还是太遥远了。不过,就算是自己失败了,只要有父亲在身旁,终究会真相大白,将凶手绳之以法的。”铁若男心中既羞愧,又感到一种骄傲。

有铁血冷在,她并不担心罪犯会逍遥法外。

但少女很快就摇摇头,心中有些气恼。

这些气恼,完全是针对她自己。

“若男,你真的是太没有出息了。你不是想要超越父亲的吗?总是想要依赖父亲,这样的心理,绝不会有超越的一天的!”

“若男,你要加油,你可以的!”少女抿起嘴,暗暗为自己打气。

她的斗志又涌上了全身。

她决定将先前的假设推翻,重头再来。

“如果酒虫,不是方源从遗藏传承中所得,而是别人给的。那么赤脉有最大的嫌疑。但是,这又有一点。赤脉为何对一位丙等资质的普通学员,刮目相看,要秘密地给他酒虫呢?”

“方源的身上,有什么特殊的优势,值得赤脉去投资的呢?就因为他是方正的哥哥吗,不对,这些价值太渺小。等等,未必是赤脉主动投资,还有一种可能,那就是赤脉被方源要挟!”

“如果是这点,方源究竟掌握了什么样的把柄,使得赤脉捏着鼻子,乖乖地向一位连蛊师都不是的学员妥协呢?”

铁若男苦思冥想,一个答案渐渐地显现在她的脑海中。

但这个答案太模糊了,仿佛是窗纸后的一抹灯光。铁若男能感应到,知道它就在那里,但却接触不到,看不清楚。

“父亲。”少女忽然抬头,看向铁血冷,“我想借用一下仙人指。”

铁血冷语气低缓:“这仙人指,所给的提示,都是依据栽种者所知道的线索。它只是替你思考,并不一定正确。你确定要用么?”

少女点点头。

仙人指是一颗种子。

铁若男取之埋入脚下的泥土当中,然后手掌心贴住地面,灌注白银真元。

几个呼吸之后,她撤回手掌。很快,就看到泥土松动,一棵幼苗破土而出。

初生的幼苗,带着半透明的绿,十分鲜嫩。它见风生长,越长越大,从嫩嫩的黄绿色,变成苍翠,最后变成碧绿。

它长成的形状,仿佛一株仙人掌。表面带着尖尖的黑刺,皮肉厚实,碧绿盎然。

须臾间,在它的顶部zhongyāng,就冒出一朵花蕾。

白色的花蕾渐渐长大,成为花苞,娇艳欲滴。但仙人掌主体,却萎缩下去,仿佛是缺少水分干瘪。

铁若男伸出芊芊手指,轻轻地摘下这个花苞。仙人掌顿时灰败下去,几秒的功夫,就彻底死亡。

仙人指,高达三转,是消耗型蛊。

铁若男取出这个花苞,小心翼翼地展开。

花苞却非是片片花瓣包裹而成,反而类似一个纸团。

铁若男将花苞完全展开,就成了一张正正方方的花瓣白纸。

白纸上写着两个字——“资质”。

这两个字,对于其他人来讲,平淡无奇,莫名其妙。但是对于铁若男来讲,却是最重要的提示。

仙人指这种草蛊,本来就是运用于此,给苦思冥想而不得的蛊师提供灵感,常常有捅破窗户纸,令蛊师恍然大悟的作用。

“对,就是资质!”铁若叫一声。

她旋即从怀中取出信件,这信件中记载着贾富收集到的全部情报。

在这情报当中,有这么一行内容,写着净水蛊在何时何价被古月赤练收购。

“对,就是这个!我先前也扫到这处内容,却没有想到更深层含义。总觉得有个地方忽略了。仙人指替我思考,给了答案。没错……净水蛊的作用只有一个,那就是消除空窍中异种真元的气息。”

“古月赤练为什么需要他?呵呵,这个情况太常见了,很多山寨都时常发生着。应该是给赤脉的继承人用的罢。赤脉的继承人资质不堪,古月赤练就灌注真元,拔高他的修为。为了消除此举的后遗症,就需要净水蛊。方源不知为何掌握了这个秘密,就来要挟赤脉,因此逼得赤脉妥协,给了他酒虫。”

铁若男口中喃喃不断,说出了她的假设。

“铁姑娘,你真的太厉害了。仅仅凭着这点,就能推断出这么多东西来!但是古月赤城明明是乙等资质啊。”方正道。

“乙等资质,呵呵,难道就不可以作弊么?要证明这些,太简单不过了。只要将赤城的空窍一查,就真相大白了。”铁若男目光灼灼,嘴角微翘。

“不妥。”铁血冷却摇头相阻。

这真相查出来,对赤脉将是严重打击,将引发古月山寨的高层动荡,政治倾轧。

铁血冷道:“我们是来破案的,不是来搞破坏的。我们到底是外人,除非无可奈何,否则不要插手他族政治。”

铁若男点点头:“父亲说的是。不过除了这个法子,还有其他方法可以证明。我记得,但凡重大的祭祀典礼,家族都会用留影存声蛊记录下来,这是传统惯例。相信古月家族也不会例外吧。我要查阅那一届开窍大典的影像!”

……

“铁神捕,你们破案奔波了这么多天,真是辛苦了,请喝茶!至于你你要查看的影像,存在我族禁地地下溶洞的密室里。二位却是不方便进去,但我已经派人去取了。在此等候片刻即可。”古月博坐着,微笑着说。

“叨扰族长了。”铁血冷客套道。

铁若男和古月方正则站在一旁。

“铁神捕,在下有一个不情之请。”古月博忽道。

铁血冷:“哦?族长请讲。”

“这就是我们青茅山的内事了。我族,白家,熊家一直僵持不下,如今狼潮令三家实力变动,熊家恶意躲避,企图利用狼潮来削弱其他两族实力。先前谈判,要求赔偿却无理想结果。于是我们三家就商议,举办一场三族大比武。只有三十岁以下的年轻蛊师参加,以各族的未来力量来决此胜负。”

说到这里,古月博叹息一声:“方源身上的古怪,其实我早有察觉,但因为狼潮没有详细调查。如果他真是凶手,我族绝不袒护。但希望铁神捕能宽延几ri,毕竟三族大比就在后天了。”

方源杀了王老汉一家,那都是凡人,根本不值得追究。他杀就杀了,也没有什么大不了的。

但若方源真杀了贾金生,那情况就完全两样了。

如果古月一族还护着方源,那就要和贾家交恶,再无贾家商队前来贸易,甚至还会引来贾家的报复。

组织会保护内部成员,但若涉及到巨大的利益危害,那么舍弃一两名组织成员,只要是成熟的组织,都会做得轻巧熟稔。

组织的最早形成,是远古时期,人族迫于生存压力,而集结在一起,借助彼此的力量,来采集更多的食物,来分工使得狩猎更成功。

任何一个组织的本质,都是更大程度地获取利益。

如果一两个成员妨碍了利益的获取,那么牺牲掉他们,也是自然而然的事情。

因此,哪怕古月一族家老稀少,但真要舍弃方源时,族长会毫不犹豫。

但后ri,就是三族大比武。比武的结果决定今后几年,青茅山利益分配。方源作为三转蛊师,是古月山寨的一个重要战力。

如同白家族长的做法一样,古月博亦同样想要压榨出方源的最大价值来。

♂♂

\end{this_body}

