\newsection{突破遗藏第六蛊}    %第七十九节:突破遗藏第六蛊

\begin{this_body}

时间如白驹过隙,夏去秋来。

甬道中,方源再一次站在堵路的巨石面前。

因为天气渐冷,他穿起了长袖的朴素衣衫。但他身材已经不再像几个月前,那般的消瘦了。

他的胸膛、双臂、双腿、腹部,都有了明显的肌肉群。

这些肌肉并不向石块那样凸出,而是呈现一种流线型,搭配方源渐渐长高的身躯,以及白皙的肌肤,让人看了有一种青春的,充满活力的感觉。

“从三天前开始,白豕蛊已经再也不能增强我的力量了。也就是说,我已经有了一猪之力,算是达到了花酒行者的目标。今天,就再推推这块圆石!”

方源眼中精芒一闪,右腿往前一踏,左腿在后,就形成一个弓步。

他双手搭在圆石的表面,深吸了一口气,猛地用力。

巨石在他双手的推动下,先是极缓慢地挪动,然后徐徐开动,最终渐渐地向前滚动。

被巨石挡住的路,是一个斜向上的甬道。被花酒行者加工成圆球状的圆石,最适合滚动。恐怕这也是他的用意,就是想要继承者把圆石推动,并且滚上去。

“十步、二十步、三十步……”方源一边一步步向前推动,一边在心中默默计算着,“我上个星期,推了四十五步,就体力不继,只好放弃。这一次,不知道能推出多远?”

四十步、四十五步……

片刻之后,方源突破了原来的记录。但他也累得够呛。

四十六步、四十七步……

方源明显地感觉到,到了这一步,自身的体力,已经所剩无几了。

四十八、四十九步……

他奋起余力,又前行了两步。终于走不动了,他浑身都是大汗。用肩膀和腿抵着圆石后,他狠狠地喘了几口粗气。

“要不要放弃?”方源忍不住这样想。这是个斜向上的甬道,他回去的途中,也得消耗不少体力。毕竟圆石还要滚下来,在这个过程中,他还要抵着。

若直接放手飞奔,圆石会越滚越快,他可不想躲闪不及,被圆石压成一滩肉泥。

但想了想后,方源又觉得有些不甘心,再推几步吧。

第五十步。

他忽然感到,来自巨石的压力骤然一轻。原来巨石已经滚上了一个平面台阶。

方源再推几步,绕过巨石,他发现自己来到了一个密室。

这个密室,和石缝秘洞差不多大小。方源暂且将其命名第二密室。

密室中空无一物,四壁亦都是奇怪的赤红泥土,发着昏暗红光。在密室的另一边,是一扇粗陋的灰色石门。应该是花酒行者仓促之下的作品。

方源休息了一会儿,并没有立即就推开石门,而是又有新发现。

他发现石门前的一块地面,看上去湿漉漉的。

“难道说……”方源脑海中顿时浮现出一个念头。他蹲下身子,伸出双手,拨开松软又潮湿的泥土。

第二朵地藏花!

方源朗声一笑,小心翼翼地拨开花瓣,取出浸泡在黄金花液中的蛊虫。

真元一催,顷刻间就炼化了。

这是一只玉皮蛊,它外形如臭虫,又扁又宽,头很小,淡绿的身体呈现椭圆形状,散发着一层玉色光晕。

“我得了白豕蛊后,还在琢磨,从哪里再弄一只玉皮蛊,这样一来,才能合炼晋升成白玉蛊。想不到花酒行者已经为我准备好了。”方源心思电转,思索着这只玉皮蛊带来的影响。

这已经是方源的第六只蛊虫。

先前他虽然有五只蛊虫,但是没有一只能用于防御。有了玉皮蛊,他总算填补了防御上的漏洞。

有时候,防御往往又意味着进攻。

这不难理解,就拿方源本身来讲,他通过白豕蛊,力量上涨到一猪之力。按理说这样的力量,能让他一拳就捣碎质地较为松软的石块,但是方源从未有这么做过。

因为他知道自己防御力不够,一拳下去,石块会碎,但他的拳头也会血肉模糊。

现在有了玉皮蛊的防御,他就能充分地发挥出力量上的长处。

当然,影响有好的一面,也有糟糕的一面。

玉皮蛊价值高,防御性能在一转蛊虫中属于佼佼者,但它并不容易喂养,它每十天就要吞食二两的玉石。

玉石市价较为贵重,且不去谈,关键是来源问题。

方正也有一只玉皮蛊,但他有族长在背后支撑,供给他玉石。方源要获取玉石,只有购买,但这太容易露出马脚了。

“本来喂养白豕蛊,需要我定时斩杀野猪,这已经比较麻烦了。如今有了这玉皮蛊,难道叫我到处去挖矿么?”方源苦笑一声,一个新的问题摆在了他的面前。

将玉皮蛊收好,暂时温养在体内空窍当中,方源便缓缓地推开石门。

石门沉重,方源若是没有白豕蛊增强力量,必定推之不动。但是现在,在方源双手轻推之下,这扇石门缓慢地打开来。

门外的景象展现在方源的面前,他的视野陡然间一扩!

此处不再是狭小的甬道,或者是密洞,而是一片宽敞的地下石林。

方源稍微目测,仅仅是初步估算,就得出这片地下石林的面积,至少有三十多亩!在地球上,一个标准的足球场,也不过十一亩。

“我现在应该身处在青茅山山内,这处空间,应该是自然形成的。”方源仰望周围石壁。

这处空间的石壁均高十六米以上,顶上自然也是石壁。

从顶壁上,往下垂下一根根巨大的暗红石柱。每一根石柱都散发着微微的红光,周围的石壁也是如此,就如同甬道和秘洞一样,空间中光线虽然暗淡,但足以让方源看清一些东西。

方源一眼望去,一根根的石柱,就像是倒长的大树,剔除了分枝,只剩下孤独的树干。

石柱的表面,并不光滑,充斥着一个又一个的黑色幽深的小洞。无数的石柱上大下小,垂下来,形成一片倒置的山内石林。

大自然就是这样的鬼斧神工。

方源见多识广,也不惊奇,只是双目紧紧地盯着石柱上的这些黑洞,眉头越皱越紧。

他有一些明白花酒行者布置玉皮蛊的意图了。

“如果我所料没错的话……”方源右手一翻,射出一道月刃。

幽蓝的月刃,在空中划出一道笔直的光线,精准无比地射入石柱上的一个黑洞当中。

黑洞中顿时传来尖锐而又恼怒的叫声。

嗖的一声,一只灰色的猴子从洞中飞奔而出,骤然扑向方源。

嗤嗤嗤。

方源连射三道月刃。

猴子在空中无法借力,但是尾巴灵活至极,连连甩动,竟然带动身躯在半空中翻腾。避开了两道月刃后,终于被第三道月刃击中,扑通一声,倒在地上,没了气息。

它死了,竟不流一滴血。

只见灰色的身躯忽然石化,在顷刻之间,就从血肉之躯变成了一个石猴雕像。

雕像的姿势,脸上的神情,一如猴子死之前,摔到在地上的模样,惟妙惟肖。

两个呼吸之后,石猴表面发出咔崩的脆响,裂纹由内而外迅速布满它的全身,然后跨啦一声,从完整的形状,碎裂成一块块的小石头块儿。

“果然是地下兽群玉眼石猴。”方源走近蹲下,拨开碎石堆,从中取出两个碧绿的圆珠。这两颗圆珠,就是玉眼石猴的两颗眼珠子。

这种奇特的生物死后,浑身都会化为灰色碎石,只有一对眼珠子,会形成两颗翠玉圆珠。每颗玉珠都十分沉重,至少重达一两。

这就意味着,只要斩杀玉眼石猴,玉皮蛊的食物问题就能得到解决。

“但是我不仅需要喂养玉皮蛊,还需要继续继承花酒行者的传承啊。花酒行者传承的下一步线索,应该就在这石林的某处吧。”

方源皱紧了眉头,这事情麻烦。

他尝试着前进几步,一对目光密切关注着石柱。

走到第七步的时候,离他最近的一根石柱上,密密麻麻的石洞里头,渐渐亮起了一对对绿莹莹的眼珠子。

顿时,一滴冷汗从方源的额头流下来。

他赶紧缩回脚,小黑洞中无数绿莹莹的眼珠子,又暗淡了下去。

很显然,每一对眼珠子,就代表着一只玉眼石猴。

玉眼石猴灵巧无比,方源斩杀了一头,就先后用了四记月刃。

一根石柱上,至少栖息着上百只的玉眼石猴。而整片石林,不知道有多少只这样的猴子了。

以方源目前的实力,遭到四只猴子的围攻,就要饮恨身死。若是算上玉皮蛊的防御,方源独自一人,最多能和十二头玉眼石猴周旋。

不过幸好这些猴子,平时都在黑洞中沉睡,饿了渴了就啃身边的石头。石柱既是它们的家,又是它们的食物。除非接近它们到十米以内,或者主动招惹,它们是不会出来的。

用地球上的话讲,就是一群宅猴。

\end{this_body}

