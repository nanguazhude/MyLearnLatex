\newsection{血滴子}    %第一百六十五节:血滴子

\begin{this_body}

%1
祠堂中一片死寂,家老和族长都低垂着头,心中像是压着一块巨石。

%2
蛊师死亡之后,遗留下来的蛊虫,仍旧残存着人的意志。不能算是野生蛊虫,更失去了直接运用大自然中元气的能力。

%3
这些蛊虫,可以认作为蛊师的生命,以另外一种方式的延续。

%4
所有人都心中惴惴不安。

%5
一代族长已经仙逝近千年,他留下的手段是否还有效?这是个未知数。

%6
毕竟距离上一次的危难,也已经过去了两百年。

%7
“怎么撤了回去?”山坡上,方源将这一幕看在眼中,心中不禁生疑。

%8
他前世在山寨时,修为太低,根本接触不到家族的秘辛。

%9
但他很快浑身一震,发现一群飞虫,从天而降。

%10
“这竟然是……”方源双眼眯成了一条缝,冷芒乍现。

%11
这蛊虫有上百只,相互飞旋缭绕,形成一涡红云,盘旋而下,落入山寨广场。

%12
嗡嗡嗡……

%13
虫群嘈杂的声音,传入祠堂,一些家老猛地抬起头来,显露出欣喜若狂之色。

%14
“谢先祖庇佑!”族长古月博心中一块石头落下,恭谨一拜,这才起身。

%15
“走,去看看罢。”族长叹了一口气,脸上除了欣喜,还有着沉痛和哀伤,神情复杂。

%16
离开祠堂,站在高楼走廊,众人便看到广场上,仿佛红色旋风刮起来,蛊虫在肆虐。

%17
这些蛊虫。只有手指尖那么大小,形如飞蚊,长相狰狞,浑身赤红。

%18
它们钻入到坐在广场中的蛊师身体内,汲取他们的血液和真元,在短短几个呼吸之间,从一只繁衍成数只。

%19
蛊虫因此越来越多。不断有新生的飞蛊,钻破蛊师的肌肤,飞出来。然后又钻进去。

%20
广场中惨叫声、冷哼声此起彼伏,但没有一位蛊师逃走。

%21
他们在来之前,就已经被家老们告知了此事。为了家族。他们愿意以身育蛊,用自己的牺牲来换取家族的长存。

%22
“这些都是我族的好儿郎!”古月博在楼上看着这一幕,声音极为低沉,扶着窗棂的手也在微微的颤抖。

%23
其余家老神色沉痛,皆沉默不语。

%24
这情形,和家族记载中的一模一样。一代先祖留下的这群蛊虫,必须事先用蛊师的性命去喂养,才得以满足,之后才会帮助家族抵抗外敌。

%25
须臾,这群飞蛊吃饱喝足。重新盘旋而起。它们声势更加浩大,在短短功夫,虫群数量暴涨了数十倍!

%26
留下广场上一地的皑皑白骨,它们形成一股赤红飓风,向山寨外的狼群呼啸而去。

%27
“果然是血滴子……”方源远在山坡。看着这一幕,心中暗道。

%28
这血滴子,乃是五转蛊虫,养用合一,十分奇异。

%29
它专门以蛊师的本命精血为食,饱餐之后。就会裂变分化,从一变二,从二变四……

%30
若是饿了,暂时找不到食物,它们就会相互吞噬,减少族群规模,来维持自身生命活动的消耗。

%31
血滴子强盛时,飞虫铺天盖地,能灭村寨,比许多六转蛊虫还要恐怖。弱小时,一只两只,孤苦伶仃,连一只三转蛊都不如。

%32
尤其是这血滴子再往后晋升,就是赫赫有名的六转魔蛊血神子。在天下十大魔蛊排位中,名列第七。

%33
方源前世创建血翼魔教,曾经首先想炼的并非是春秋蝉,而是血神子。可惜世事多无奈,因为各种原因,只好退而求其次,合炼春秋蝉。

%34
这群血滴子,数量足有数万。如飓风一般席卷战场,所到之处,群狼哀嚎。

%35
它们左右横扫,钻入电狼体内,只是几个呼吸,电狼一身的血液就被血滴子吸摄一空。

%36
但这兽血,并不能让血滴子分化。唯有包含真元气息的蛊师精血,才有这作用。

%37
一只只电狼被抽成干尸,倒在地上,失去了生命气息。

%38
唯有豪电狼、狂电狼喷吐的电流,才能克制这血滴子。

%39
但它们往往只是电倒一小片飞虫后,就被其后密密麻麻的血滴子蜂拥而上,吸成干尸。

%40
嗷呜!

%41
铁网悉数崩断,雷冠头狼重得自由,发出愤怒的咆哮。

%42
血滴子有感,纷纷汇聚起来,凝聚成一朵数亩的大红云,结结实实地将雷冠头狼罩住。

%43
雷冠头狼狼尾四甩,蓝色的电浆迸溅,成百上千的血滴子在瞬间被电成焦炭。

%44
噼里啪啦的声音,像是鞭炮炸响。

%45
一股风吹来,夹裹着血滴子被烤糊的难闻焦味。

%46
血滴子的确是五转蛊虫,但却无人操纵,只是硬打强攻。雷冠头狼的身上,则寄居着好些四转蛊虫,有一些攻击范围广大,正克这血滴子。

%47
飞虫如红云剧烈翻腾着,小山般巨大的雷冠头狼在其中咆哮挣扎,嘶吼腾挪。

%48
雷冠头狼巨大的身躯,此刻成了它最弱的短板,被指尖大小的血滴子针对。

%49
战况惨烈,大片大片的血滴子掉在地上,雷冠头狼鳞甲被冲破,一些血滴子钻入它的体内,大肆吸血。

%50
雷冠头狼无奈之下,只好往自身身上挥洒电浆,将这些血滴子电死。

%51
这样一来,它表面上的肌肉也渐渐被烤熟,散发出一股浓郁的肉香。

%52
狼潮对山寨的冲击,已经停止了。

%53
无数的电狼,在雷冠头狼的呼唤下,向虫群发起冲击。

%54
蛊师们可谓险死还生,无不屏住呼吸,紧张地看着眼前这场壮阔惨烈的厮杀。

%55
虫群仿佛是一股死亡旋风,无数的电狼刚刚冲进去不远,就颓然倒下。

%56
然而电狼前赴后继的牺牲。终究也给虫群带来消耗。

%57
血滴子越来越少,先前浓郁如云,渐渐地成了薄雾,到后来只是成股飞舞,仿佛一团团旋风。

%58
雷冠头狼挣扎而出,嘴里发出呜咽之声,拔腿飞奔。

%59
它全身血液被吸摄了大半。如今四肢酸软,逃跑的速度根本不及原来的十分之一,身上闪烁的电流。也稀疏到了极致。

%60
作为狼群的统领,雷冠头狼这一走,其余的电狼顿时斗志涣散。也夹着尾巴,四散奔逃。

%61
“终于顶住了……”蛊师们看到这一幕,愣了愣,一些人当场就瘫倒在地上,不愿起身。

%62
“我居然还活着!”许多蛊师的神色复杂,有欣喜,也有哀伤。

%63
狼潮中,牺牲了多少族人!

%64
嗡嗡嗡……

%65
稀疏的血滴子虫群,遥遥飞向高空,在空中绕着古月山寨盘旋了一圈后。钻入云层中消失不见。

%66
方源看到这一幕,眼中闪过若有所思的光。

%67
雷冠头狼一败,古月山寨算是安全了。事实上,狼群也死伤得所剩无几,再也不具规模。

%68
这一次的狼潮。可以说是渡过了。

%69
然而狼群记仇,这头万兽王,一日不死,在下一次狼潮中定会卷土重来。经过数年的休养,到那时它会更狡诈,更强大。

%70
现在的它。是最弱的时候,杀了它,收取它身上的众多蛊虫,也可以稍稍弥补一些家族的损失。

%71
“药钟,歌燕,你二位留下来,处理残局。其余家老,都随我动身,追杀雷冠头狼!”古月博匆匆安排之后,就立即率领其余七位还能一战的家老,出了寨子,顺着顺着雷冠头狼逃走的方向追去。

%72
方源目光闪了闪,心中权衡:“家族力量虽然空虚,但场面残而不乱。且这群血滴子亦是来的蹊跷,还是跟上族长一行,看看情况。”

%73
他隐去身形,就走下山坡。

%74
隐鳞蛊和雷翼蛊不能同时使用,雷翼蛊是三转蛊虫,一经催动起来,形成的一双雷翼,凭二转的隐鳞蛊还遮掩不住。

%75
面对三转家老,隐鳞蛊的隐形能力,并不可靠。

%76
方源只能先顺着痕迹,远远跟着,不敢太靠近。

%77
雷冠头狼来袭时,他还在石缝秘洞,一直没有参战。若现在忽然出现,恐怕会引起家老和族长的愤怒和叱问,而且许多东西方源也解释不清楚。

%78
雷冠头狼伤势很重,速度并不快。大约追了半个小时,方源隐隐听到前方的山谷中,传来打斗和怒骂声。

%79
他潜行过去,攀上一处山石,就窥见一群蛊师正在捉对厮杀。

%80
那头雷冠头狼则趴在地上,身上增添了许多新伤口,血流不止,奄奄一息。双眼黯淡,似乎濒临死亡。

%81
“老白毛,你们真有脸,居然敢来捡便宜!”

%82
“呵呵呵,古月博,你这话就不对了。这雷冠头狼明明是我们拦下的,识相点,就赶紧让开!”

%83
古月博和白家族长相互对撼,声势猛烈。

%84
狼巢中有三头雷冠头狼,均是万兽王,各统领着数万狼群,彼此之间并不统属,谁也压不过谁。

%85
狼群行动,向来讲究协同一致。三头雷冠头狼亦衍生出智慧,分别同时对青茅山上残留的三大“兽群”展开狩猎。

%86
在它们眼中,人类也是野兽,更是猎物。

%87
狼潮在它们的理解中,则是一场倾巢而出的大型狩猎。

%88
青茅山中,古月、熊、白三家相持数百年,自然各有底牌。

%89
熊家实力最弱,如今还在艰难地抵御着狼潮。白家近些年崛起,总体实力已经超过古月一族,在斩杀了雷冠头狼,击退狼潮之后,白家族长就带领着一群家老,赶往古月山寨,看看有无便宜可捡。

%90
没有想到,还真捡了个漏子,将雷冠头狼堵在这处山谷。

%91
雷冠头狼的身上,寄生着许多蛊虫,有一些高达四转。雷冠头狼的血、头骨、狼眼、冠毛亦都是珍稀的合炼辅料。古月一族当然不可能拱手相让,气愤之下,于是就展开了激战。

%92
“呵呵呵,方源,你果真来了。我等你多时了!”

%93
山壁上,忽然传来一阵冷笑。

%94
方源抬头望去,只见一个白衣少年从天而降,一把冰刃带出风声,照着他的脸面,冷酷地斩下。(未完待续。如果您喜欢这部作品,欢迎您来投推荐票、月票,您的支持,就是我最大的动力。)

\end{this_body}

