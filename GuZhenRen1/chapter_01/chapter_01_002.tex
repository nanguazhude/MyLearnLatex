\newsection{逆光阴五百年觉悟}    %第二节:逆光阴五百年觉悟

\begin{this_body}

%1
传说中,这个世界存在着一条光阴之河,支撑着这个世界的流转。而利用春秋蝉的力量,就能逆流而上,回到从前。

%2
对这个传闻,世人众说纷纭。很多人并不相信,有些人则将信将疑。

%3
几乎没有人真的确信。

%4
因为每一次使用春秋蝉,都必须付出生命,将整个身躯和所有的修为统统献祭,作为驱动的力量。

%5
这个代价实在太昂贵了,更让人无法接受的是——往往付出了生命,也不知道结果怎样。

%6
就算是有人得到了春秋蝉,也不敢闲来无事胡乱使用。

%7
万一传闻有假,只是个骗局呢?

%8
若不是方源走投无路,也不会这么快就使用它。

%9
不过现在,方源是彻彻底底的相信。

%10
因为铁的事实摆在眼前,不容反驳。他的确是重生了!

%11
“只是可惜了这个好蛊,当初可是费了九牛二虎之力,屠杀了数十万人,弄得天怒人怨,千辛万苦才炼制而得……”方源心中暗叹,虽然是重生了,但是春秋蝉并没有带来。

%12
人是万物之灵长,蛊是天地之精华。

%13
蛊千奇百怪,数不胜数。有的蛊用一次或者两三次,就会彻底消散。而有的蛊,只要不太过度使用,就能重复利用着。

%14
也许春秋蝉就是那种只能使用一次的消耗类蛊虫。

%15
“不过就算是没了,也可以再炼制一只。前世我能炼制,今生难道就不能吗?”可惜之后,方源的心中又不禁涌起一阵壮志豪情。

%16
自己能够重生,春秋蝉的损失完全可以接受。

%17
而且他还身怀珍宝,并非一无所有。

%18
这个珍宝,就是他五百年的记忆和经验。

%19
他的记忆中存在着许许多多的宝藏,如今还没有人开启。存在着一个个的大事件,让他能轻松掌握历史的脉络。存在着无数的人影,有些是前辈隐修,有些是天俊奇才,有些人甚至没有出生呢。还存在着这五百年来,苦修的沉重经历,丰富的战斗经验。

%20
有了这些,无疑就掌握了大局和先机。只要操作的好,纵横人间,重现巨魔枭雄之风采,完全不是问题,甚至能更进一步,冲击更高之境界!

%21
“那么该如何操作呢……”方源十分理智,迅速收拾情怀,面对窗外的夜雨沉思起来。

%22
这么一想,就觉得千头万绪。

%23
思考了片刻,他的眉头越皱越深。

%24
五百年的时间,实在有些漫长。不说那些已经模糊的,想不起来的记忆。就是那些记着的宝藏密地、仙师机缘,虽然很多,但大多不是间隔十万八千里,就是需要在特定的时间才能开启。

%25
“最关键的还是修为啊。自己如今元海未开,还没有踏上蛊师的修行之路,根本就是个凡人!必须得尽快修行,增长修为,赶在历史之前,尽可能的抢占先机,捞够好处。”

%26
而且很多的密藏,修为不够,即便得到了也消化不了。反而是烫手山芋,怀璧之罪。

%27
摆在方源面前的第一个难题,就是修为。

%28
必须要尽快提升修为,若是像上一世慢腾腾的话,黄花菜都凉了。

%29
“要尽快提升修为,就必须借助家族的资源。以我现在的情况,根本就没有能力在危机重重的群山中穿梭,一头普通的山猪都能够要了我的性命。若能达到三转蛊师的修为,就有基本能力自保,在这方世界中跋山涉水了。”

%30
以五百年锻炼出来的魔道巨擘的目光来看,这个青茅山真的是太小了,古月山寨更像是个牢笼。

%31
不过牢笼囚禁自由的同时,坚固的牢房也往往代表着某种安全。

%32
“哼,短时间之内,就姑且在这牢笼里折腾拳脚吧。只要晋升蛊师三转,就离开这穷山僻壤。不过幸好,明天就是开窍大典,此后不久就能正式开启蛊师的修行。”

%33
一想到开窍大典,方源心中那尘封已久的记忆就在心底浮现上来。

%34
“资质么……”望着窗外,他不由冷笑三声。

%35
就在这时,房门被轻轻的推开,走进一位少年。

%36
“哥哥,你怎么站在窗边淋雨?”

%37
这少年体型消瘦,比方源要稍矮一些,面容极似方源。

%38
方源回头看着这个少年,脸上闪过一丝复杂之色。

%39
“是你啊,我的孪生弟弟。”他微微扬起眉头,表情恢复了一贯的冷漠。

%40
方正低下头,看着自己的脚尖,这是他的招牌动作:“看到哥哥的窗户没有关,就想悄悄的进来关了。明天就是开窍大典,哥哥你这么晚还不休息,舅父舅母知道了,恐怕会担心的。”

%41
他对方源的冷漠并不奇怪,皆因从小到大,他的哥哥一向如此。

%42
有时候他会想,也许天才就是这样的非同常人吧。虽然和哥哥有着极为相似的相貌,但是自己却平凡的像个蝼蚁一样。

%43
同时从一个娘胎里生出来,为什么上天就如此不公。赋予了哥哥钻石般的才情,而自己普通的就像个石子。

%44
身边的每个人,提到自己,都会说“这是方源的弟弟。”

%45
舅父舅母也常教育自己,要向你哥哥学习呀。

%46
甚至就连自己有时候照镜子,看着自己的这张脸,都觉得有些厌恶!

%47
这些念头已经有许多年了,日积月累地积压在内心深处。像是一块巨石压着心胸,这些年方正的头垂得越来越低,也越加沉默寡言。

%48
“担心……”想到舅父舅母,方源在心中发出一记无声的嗤笑。

%49
他记得很清楚,自己这个身体的双亲因为一次家族任务,而双双陨落。在三岁的时候,就和弟弟一起成了孤儿。

%50
舅父舅母就依着抚养的名义,堂而皇之地侵占了双亲的遗产,并且苛刻地对待自己和弟弟。

%51
本来作为穿越众,还计划着韬光养晦。但是生活的艰辛,让方源不得不选择展露异于常人的“才华”。

%52
所谓的天才,其实不过是一个成熟灵魂的理智,以及地球上几篇流芳百世的唐诗宋词罢了。

%53
就是这样小试身手,也被惊为天人,受到广泛关注。外在的压力下,也让年幼的方源不得不选择冷漠的表情,来伪装保护自己,减少露馅的可能。

%54
久而久之,冷漠反而成了自己的习惯表情了。

%55
就这样,舅父舅母再也不好苛刻自己和弟弟,随着年龄越大,前途越被看好,待遇也跟着增加。

%56
不过这并非是爱,而是一种投资。

%57
可笑这个弟弟,却没看清这个真相,不仅被舅父舅母蒙蔽,还对自己埋藏着怨恨。别看他现在这样乖巧老实,记忆中被测出甲等资质后,被家族大力培育,隐藏的仇恨嫉妒都释放出来,可没少针对、刁难、打压过自己这个亲哥哥。

%58
而至于自己的资质嘛……

%59
呵呵,最高的只是个丙等罢了。

%60
命运总是爱开玩笑。

%61
一胎双胞,哥哥资质只是丙等,却独享天才之名十几年。弟弟默默无闻,反而有着甲等天资。

%62
开窍的结果,让族人大跌眼镜。也让兄弟俩的处境待遇,彻底颠倒。

%63
弟弟如卧龙升天,哥哥似凤雏落地。

%64
其后,是来自弟弟的多番刁难,舅父舅母的冷眼,族人的轻视。

%65
恨吗?

%66
方源前世恨过,恨自己资质不足,恨家族无情,恨命运不公。

%67
但是现在,他以五百年的人生经历,重新审视这段历程,心中却波澜不惊,没有一点恨意。

%68
有什么好愤恨的呢?

%69
换位思考一下,他也能理解弟弟,舅父舅母,以及五百年后那些围攻他的正派强敌。

%70
弱肉强食,适者生存,本来就是这世间的本质。

%71
况且人各有志,都争那天机一线,彼此间打压杀伐有什么不理解的呢?

%72
五百年的经历,早就让他看透了这一切,心中唯有长生大道。

%73
若是有人阻挡在他的这个追求,不管是谁,无非是你死我活罢了。

%74
心中的野望太大,踏上这条路,就注定举世皆敌,就注定独来独往,就注定杀劫重重。

%75
这就是五百年人生凝练的觉悟。

%76
“复仇不是我打算,邪魔的道路亦从没有妥协二字。”想到这里,方源不禁失笑。回过头对着这个弟弟,淡淡地看了一眼,道:“你退下罢。”

%77
方正不禁心中一悸,感觉哥哥的目光如冰刃般犀利,似乎洞穿到了他内心的最深处。

%78
在这样的目光下,他如赤身裸体在雪地里,没有丝毫的秘密可言。

%79
“那明天见,哥哥。”当下再不敢多话,方正缓缓关上房门,诺诺而退。

%80
(感谢爱gml、报炉吃瓜、明晓言语0525朋友们的打赏,感谢网上飞的豬、梦里惊鸿照影来两位同学的满分评价票。大家一直以来的支持,是我最大的感动和动力。谢谢大家如此相信我,支持我,投推荐票给我。)

%81
(新书期间,稳定更新。目前一天两更,上午八点一更,下午两点一更。)

%82
(还有一个通知,以前老书《御妖至尊》的书友群也会做改变。原《御妖至尊》一群、二群,转为《蛊真人》一群、二群,原《御妖至尊》vip群改变为《蛊真人》三群。在今后大约40天的时间内,我会亲自主持管理工作。所以在节假日期间,请老朋友们冒个泡泡,这不是件难事。在本书上架当天,进行总的剔除工作,踢去一直没有发言的人。同时会设立《蛊真人》vip群,此群会要订阅验证。)

\end{this_body}

