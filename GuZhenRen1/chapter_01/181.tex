\newsection{阴差阳错(大章 )}    %第一百八十一节:阴差阳错(大章 )

\begin{this_body}

“铁姑娘,你这是什么意思?”方正不解。

铁若男伸出食指,指向墙壁:“你们看他,古月方源!不觉得他有些太过于平静了吗?自己被测出丙等资质,亲生弟弟却反而是甲等资质,换做常人都会有心湖的波动。但是他呢?没有丝毫的动容,周围的赞叹声,嘲讽声,都动摇不了他的内心。你们不觉得很奇怪吗?在整个过程中,他都太平静了。”www.13800100.Com 文字首发 /文字首发

铁若男的话,吸引了众人的目光。

这时画面回放,数道视线集中在画壁中,方源的身影上。

方源站在人群当中,孤身一人,茕茕孑立。一片阴影笼罩住他的大半个身躯。

而这番情形,形成鲜明对比的是在对岸。

他的亲生弟弟古月方正,正一步步地走着,浑身都沐浴希望蛊的洁白光辉中。

这一刻,哥哥弟弟,形势立转。前者从高空而坠,陷入人生低谷中,笼罩黑暗。后者则被命运垂青,光辉照耀,要攀上高峰。

“一个十五岁的少年,经历人生如此大变,却从他的身上看不到一丝一毫的失落、迷茫、嫉妒。只有平静。他就站在人群中,这么静静地看着,仿佛局外人,像在看一场戏。”铁若男的声音适时传来。

是的。

方源就这么静静地看着。

阴影笼罩着他大半的身躯,他脸色淡漠,有着少年特有的苍白肤色。

古月博盯着画壁,沉吟不语。

方正却感到一股从心底产生的寒意。

哥哥,你究竟……

“就算他对开窍结果有预料,心中也会有波澜,不会如此平静。我先前动用了仙人指,一直以为上面的“资质”二字,是说古月赤城。现在想来,未必是古月赤城,很有可能是方源!”铁若男道。

“铁姑娘,你这是什么意思?我怎么有点听不懂?”方正更加疑惑。

铁若男竖起一根手指:“只有一种情况,才令方源如此平静。那就是他有不为人知的依仗。只有一种情况,会令他不嫉妒,那就是他有更强的底牌。两个人结伴而行,一人在路上捡到一块元石,另外一人只有捡到十块、百块的元石,才会一点都不嫉妒,心中平静如常。而方源正是这样的情况。”

“你是说,方源作弊,隐藏了他真实的资质?其实他并非是丙等资质?”古月博听出了铁若男话中的味道“但他若是乙等、甲等的资质,为什么不光明正大的展现出来呢?”

“方源,有早智!”铁若男嘴角渐渐翘起,神情自信,焕发出一种迷人光彩“这些天我也拜读了他早年的一些诗词,气魄宏大,胸有锦绣,连父亲也不禁赞叹。这样的天才,自有他心中谋算。”

古月博摇摇头:“单凭这点,不能说明什么。隐瞒资质和公开资质,将是两种截然相反的待遇。就算方源他之前不知道,在学堂一年,他也应该清楚了。”

“他当然清楚,所以更加不敢暴露。”铁若男的话干脆利落,带着一丝斩钉截铁的意味。

“你这是什么意思?”古月博也弄不懂了。

“古月族长。”铁若男转过身,郑重地看向古月博“你可知道,人祖十子?”

古月博先是一愣,但旋即他反应过来,明白了铁若男的真正意思。

他震惊了!

他的瞳孔猛然一扩,嘴巴张大,没有一丝平常的族长风姿。

方正还是第一次看到,古月博出现这样的神态。

人祖十子,大子太日阳莽,二女古月阴荒……十子资质逆天,为天地所忌,没有长寿者。在蛊师界,十子的名字代表着十种最顶端的天资!

那就是――十绝体!

“人祖十子相继灭亡,人祖也要老死。在最后关头,人祖取来十子尸体,又牺牲自己,一起投身衍化蛊的肚子里。衍化蛊撑破肚皮,爆炸开来,无数的生命之光落于大地,就形成了第一批的凡人。这些凡人,没有人祖,以及十子的天资,但是却可以开窍修行。人类因此世代繁衍,形成今天的规模。当然,这都是神话故事了。”

“但按照故事所言,我们每个人体内,都流淌着人祖和十子的血脉!”铁若男继续侃侃而谈“只是有些人的体内,各个血脉都稀疏,又相互牵制,因此不显。而有些人出生之时,其中一道血脉就相当浓郁。或者随着年龄或者修行,慢慢呈现出极端,其中一道血脉压制其他所有。表现出来,就是十绝体!”

“那么这十绝体究竟是什么?”方正问道。

“难道说,这方源竟然是十绝天资?!”古月博震撼。

“极有可能如此!只有十绝天资,才能令方源如此平静,对修行资源要求不高。更只有凌驾于甲等的十绝体,才能让方源对方正没有丝毫的嫉妒,羡慕。也只有十绝体,才令方源不敢公开出来,害怕被提前扼杀,选择作弊隐藏!”铁若男语速极快。

就算是铁血冷,也不会联想到重生。少女更是如此,各种机缘巧合之下,使得她得出一个和真相恰恰相反的推理结果。

这番话后,其他人都呆若木鸡。

铁若男的推论,有凭有据,令人不由自主地去相信。如果不是这个原因,还能有什么解释呢?

“是了,是了!十绝体,十绝体……应该就是古月阴荒体了!”

古月博心潮澎湃,激动得浑身颤抖。

别人兴许不明白,但他掌握着家族秘史,知晓很多的秘辛。

数百年前,一代先祖在这里创下山寨基业,命名为古月山寨。事实上,在此前,一代并不姓古月!

一代为何要取“古月”这个名字呢?这是个迷。

但一代死前,曾长叹:“血脉流传,百年大计,古月阴荒,天下惊惶!”

他留下遗嘱,在遗嘱中,预告着古月山寨在未来,将出现一位蛊师,拥有十绝之一的古月阴荒体!他将令天下惊惶,将古月一族带入辉煌。遗嘱中又告知后人,若是有朝一日,真的出现了此体,就将其带入血湖墓地。

作为族长,古月博自然明白十绝体的弊端。但他深究这份遗嘱,发现一代先祖似乎有克服弊端的手段。而这个手段,他留在了墓地的棺椁中,伴随他长眠。

如果真有古月阴荒体的天才出现,就将他带到棺椁那去……

“想不到,古月方源就是那个预言中的天才――古月阴荒体!”古月博在心中咆哮。

“这不可能。大庭广众之下,哥哥怎么能在众人的眼皮子底下作弊呢?”方正心中极度震动,连连摇头,不能接受这个推论。

铁若男怜悯地看了他一眼:“既然赤城都能作弊,为什么方源不可以。十绝体有无穷奥妙,难以想象。也许方源早已经提前开窍,也许他骗过竹君子,就是因为十绝体的缘故。也许贾金生被方源所杀,就是因为他无意中窥破了方源的这个秘密。”

“铁家丫头,说话要谨慎啊。”古月博脸色很不悦,声音低沉“话不能随便乱说的。贾金生是不是方源杀的,还没有确定。方源是我族一员,就算真的是他杀的,怎么向贾家交代,也是我们的事情。你们只是负责查案罢了。”

古月族长的态度,顿时有了翻天覆地的转变。

先前,他认为方源是丙等资质,没有投资的可能。但现在,方源居然是古月阴荒体?!

涉及到一代先祖的预言和遗嘱,这身价立马就不同了。

必须要护住他,哪怕是得罪了贾家,也在所不惜啊!

“糟糕。这个古月族长翻脸不认人,太不要脸,态度变化这么快!我不应该这么早,就把十绝体的秘密披露出来的。十绝体是凌驾于甲等资质的天资,看来古月族长是想保住方源了。这将给我破案带来极大的困难。父亲我该怎么办?”

铁若男心中一沉,咬牙,下意识地寻找父亲的身影。

“咦,父亲呢?”少女愣住。

铁神捕刚刚还在场,但转眼间,却不见了踪影。

这个发现,让古月博瞬间紧张起来。

铁血冷这个时刻能去哪里?古月族长自然而然,就联想到方源身上。

“方正,我先去寻你哥哥。你去通知其他家老,快速集合来找我!”丢下这句话后,古月博便调动蛊虫,从窗外直接飞射出去。

铁若男冷哼一声,也旋即跑出房门,向方源的住处赶去。

整个大堂,只留下方正一人。

“怎么……怎么会这样?!”

古月方正的状态,却极为不妙。他满脸苍白,身躯摇摇欲坠,仿佛脊椎被人抽空了一般。

他感觉世界一片黑暗。

他感觉天都要塌下来了!

怎么竟然是这样子呢?

“哥哥的资质,居然是十绝体?我的甲等资质,和十绝体根本相比,根本就是土鸡瓦狗啊!”

想着想着,方正就流下泪来。

这一刻,他的心气劲,都被抽空了。

一直以来,他最大的骄傲,就是资质上胜过方源。他的自信,全部建立在这个方面。但是如今,有人告诉他这样一个事实――他的资质其实远不如他的哥哥!

他最骄傲最自信的地方,被方源压过去。

他好不容易竖立起来的自信,在这一刻猛地全数崩溃。

“哥哥!”他仰头,任脸上泪水横流。

他在心中无助地嘶吼:“至始至终,你都是在看戏吗!看我上蹿下跳,看我小丑般的表演?”

他眼角的余光,再次瞄向画壁。

影像变化中,人影斑驳,无数的声音在惊叹对岸那方正的表现。惟独方源站在人群中,一脸平静地看着。

就这么平静地看着……

方正忽然觉得,这影像中包裹着希望之光的自己,是多么的幼稚,多么的刺眼。而在他心中,方源平静的黑眸中,仿佛流露出神一般的目光。

他的身影渐渐拔高,越来越高,形成无可攀越,抵达天际的高山。

那双平静的双眸,就在高山之巅俯视着山脚下,方正卑微的身影。

那双眼睛,仿佛在说――方正,我可笑的弟弟,你不行,你不行,不行……

宛若山谷中的回音,在方正的心中回响。

他感觉压抑。

无以伦比的压抑,压抑到自己呼吸都不通畅。

他曾经以为自己已经完全摆脱了这种感觉,但没有想到,今天这种感觉又袭上心头。

甚至比以前,要强烈一百倍!

“啊啊啊啊!我快要死了,我快要死了!”方正脸色抽搐,额头青筋直冒,他沉溺于自己的幻想当中,用手紧紧地扼住自己的咽喉。

扑通一声,他跪倒在地上。

因为缺氧,他脸色涨红,伸长了舌头,双眼都渐渐凸出眼眶。

“不!我不想死,我真的不想死!!”他的另一只手扣住地砖,用力太猛,将指甲都剥裂开,鲜血流淌,钻心的疼痛更让方正求生欲望大炙!

他的身体没有一点伤势,但是他的心中,却已经遭受了致命的创伤。

他爬不起来了,支撑他的骄傲已经被粉碎。

他觉得自己无脸见人,无法在面对现实,面对哥哥。他已经成了一个巨大的笑话,一个辛辣的讽刺。

他心若死灰,但这灰烬中却还残留着点点未熄的嫣红。

“就算是有十绝资质又怎样?哥哥,你犯了错。你杀人了,你太不应该了!对,对的!杀人是要偿命的,哥哥,贾金生是你杀的吧?你走错路了,你做错事了,就算有再好的资质又怎样呢?!”

蓬……

耳边似乎有一声轻响。

一团殷红的火焰,在方正心中燃烧起来。

死灰复燃!

扼住咽喉的手,渐渐松开。

方正浑身是汗,他跪倒在地上,一只手支撑着,将地砖抓得满是血痕。

一滴滴的汗,顺着他的脸颊,鼻尖,滴在地砖上。

他扭曲的恐怖脸色,渐渐地平复下来。

一阵风,顺着敞开的大门和窗户,吹进这厅堂。

夜间的冷风,刺激得他浑身一颤。

“咯咯咯咯咯咯……”方正几乎趴在地上,笑出声来,声音尖锐而又诡异。

随着风而跳跃的火光,映照在少年端正的脸上,此刻却不是温暖和光明,却像是魔鬼在舞蹈。

\end{this_body}

