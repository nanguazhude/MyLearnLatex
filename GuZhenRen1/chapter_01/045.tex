\newsection{洞心机,已在瓮中不自觉}    %第四十五节:洞心机,已在瓮中不自觉

\begin{this_body}

“你好,这位年轻的蛊师,有什么问题吗?”贾富走到人群中央,和颜悦色地问道。

青年蛊师有些受宠若惊,又行了一礼。他看了一眼周围的族人们,便壮着胆子,将整个事情和盘托出。

“原来是这么一回事!”贾富听了,点点头。他又问一旁的贾金生,“弟弟,是有这么一回事吗?”

贾金生别过头,冷哼一声,不去看他。

贾富沉吟起来。

周围人群静悄悄的,不敢打扰他的思考,亦都在翘首以待他的判定。

这件事情说起来,总体还是贾金生商业诈骗。但是这个青年蛊师也有过失,若不是被贪婪蒙蔽了心智,怎么可能会中招呢?

贾富若是一心要维护自己的弟弟,凭借他四转的修为,就算是古月族长也拿他没有办法。

贾富沉吟半晌,终于开口:“此事我已明了,这件事完全错在我弟,教这位小哥蒙受了损失,买了假货,实在是对不住了!”说着,就向青年蛊师拱手一礼。

“贾富大人!”青年蛊师大感意外,连忙谦让道,“您可是四转蛊师,我不过只是个二转。这可使不得,使不得啊!”

贾富摆手:“呵呵,这事和蛊师修为没有关系,我向来只对事不对人。错就是错了,我代表商队向小哥你道歉。至于赔偿,这样吧,小哥损失了两百五十块元石,我代表贾家双倍赔偿你。”

他言出必行,立即身边就有随从,取出了五个钱袋子,当众交到青年蛊师的手中。

每个钱袋都是饱满鼓囊,各装有一百块元石。

青年蛊师接过钱袋子,顿时激动得什么话都说不出来。

“但是小兄弟,老哥我也有一句话要劝你。”贾富又关照道,“黒豕蛊十分珍稀,能从根本上增加蛊师的力量。这种蛊虽然只有一转,但是市面上难以寻找。只要在市场上一出现,往往第一时间就被人收购。价格都在六百块元石左右。想要靠两百多元石买一只黒豕蛊,并不太现实。”

“晚辈受教了!”青年蛊师心悦诚服地对贾富一躬到底。

人群中传来一阵欢呼。

“贾富大人英明!”

“了不起,不愧是贾富大人!”

“身为四转蛊师,却不恃强凌弱,贾富大人真乃是正道的楷模啊。”

……

“哪里,哪里。”贾富笑眯眯地,向四方抱拳,谦和地道,“我们贾家经商,素来以诚信为本,童叟无欺。各位父老乡亲,我这弟弟只是年少无知,喜欢捉弄别人,其实心底还是善良的。希望大家能多多海涵,多多海涵啊。”

周围欢呼声更加热烈。

“哼!”贾金生脸色铁青,恨恨地一跺脚,直接转身进了帐篷。然后穿过帐篷,从帐篷的后帘走了出去。

方源冷眼旁观,看到这里,心中大定:“看来花酒行者留下的那影壁,可以出手了。”

花酒行者用留影存声蛊,记录下了古月一族四代族长的丑态。

他在死之前,心怀不忿,将这留影存声蛊用了,拍在石壁上,就形成了一块影壁。

影壁上画面不断循环,向世人展现出当时最真实的一幕。

秉着利益最大化的原则,方源早就想贩卖这块影壁了。他确信青茅山上的另两家:白家寨、熊家寨都会对这影壁很感兴趣。

但是要让他亲自贩卖这块影壁,很不妥当。他修为还是太弱了,带着影壁到了其他寨子,很有可能就被人灭口。

就算是交易成功,安然返回。天下没有不透风的墙,一旦消息泄露到古月一族高层,他最轻也要被逐出家族。

按照方源的计划,他现在还需要利用古月一族。

因此最保险的方法,就是卖给商队中的某个商家。这些商家都是外人,不参与青茅山的势力斗争,是最理想不过的选择。

再过一日,这支商队就会启程,离开古月山寨,前往熊家寨,然后便是白家寨。

卖给他们,方源就能将买卖中的风险将减至最小,是最安全稳妥的方法。

……

“再来一杯酒!”

“酒,酒呢?”

“快给我端上来,还怕公子我付不起钱?”

贾金生大力拍着蘑菇桌面,口中咆哮着。

“贾公子,您的酒!”伙计连忙将酒端了上来。

贾金生第一时间抓住竹筒酒杯,仰起脖子,一饮而尽。

“好酒!”他大笑一声,声音苍凉又嘶哑。

砰的一声,他将酒杯顿在桌面上,又大吼起来:“再给爷来一杯,不,有多少来多少!”

酒铺中的伙计自然不敢得罪他,只得照办。

所幸这酒肆中,已经人满为患。不仅是蘑菇桌凳上围坐了一圈。就连周围的过道,都是人挤人。贾金生发着酒疯,大吼大叫,在这人声鼎沸的帐篷里,也并不是很突出。

贾金生一杯接着一杯,想要借酒消愁。他背对着众人,谁也没有发现他喝着喝着,就流下了两行清泪。

谁能知他的苦,谁能知他的愁?

可怜之处必有可恨之人,反之亦然。

每个人都有自己背后的故事。

兄弟几个当中,他年龄最小,长得最英俊,和父亲最像,因此受到父亲的宠爱。但是偏偏上天的捉弄,让他只有区区丁等的天赋。

从小到大,他就活在几位哥哥的压迫挤兑当中。他不甘心,他想要反抗,但是碍于资质,他做不到。

父亲感到大限将至,想要分割家产。让一干兄弟,每两人领一只商队。许诺依照各自的成绩,分割家产。

贾金生想要依靠自己的方式,来获取家产,以及家族的承认。但是没有想到,他再一次成了哥哥贾富的踏脚石。

在贾富出现的那一刻,他就知道自己又中计了。这事情从头到尾就是个阴谋。但是他又能怎样?自从进入这商队,他就一直被贾富牢牢地压在底下。四转和一转的巨大差距,更让他无力斗争。

“贾富!”他从牙缝中挤出这个名字,眼中燃烧着仇恨的火焰,真是不甘心啊!

“想要对付你的哥哥吗?我可以帮你。”一个声音恰在这时,传入他的耳中。

贾金生楞了一下,转头一看,发现不知什么时候起,自己的身边坐着一个人。

他摇晃了几下脑袋,眨眨眼睛,终于看清此人。

不是方源,又是何人?

“是你!”他瞪向方源,有些恼怒,“我记得你!好运的小子,居然从给我的赌场里开出了一只癞土蛤蟆!你是想来戏耍我吗?”

方源看着贾富,目光冷冽如水:“我有一笔大生意。你若想取得更好的成绩,分到更多的家产,不妨听听看。”

贾金生的脸上顿时流露出惊疑之色,他背部一挺,坐直了:“你怎么知道家产的事情?”

这事情秘而不宣,外人不可能得知。偏偏方源却一语道破。

“贾家寨的这些破事,又不是什么机密,怎么瞒不住这世上的有心人?”方源冷笑一声,不禁想起记忆中的前世之事。

贾家家主是个传奇人物,白手起家,以商队发迹,振兴了贾家寨。他垂垂老朽,感到大限将至,就让几个儿女每两人领一只商队,依照他们的成绩,来分割庞大的家产。成绩越好,自然分到的家产就越多。

但他的大儿子贾富,二儿子贾贵,都极为优秀。一直比拼六七年,不分上下,直到贾家家主老死,都没有分出胜负。

贾家家主死后,留下一笔极为庞大的家产。贾富和贾贵为了争夺家产,兄弟阋墙,内斗升级,引进外援。激发了一场大规模的斗蛊大会,最终双双同归于尽。贾家寨强盛一时,很快又败落下去,让世人唏嘘不已。

贾金生眯起双眼,对方源的解释不置可否。他暗暗寻思:去年父亲宣布了分割家产的规矩,现在已经是第二年。世上没有不透风的墙,此事泄露出去,也不奇怪。

他真正担心的是,这是否会是贾富的又一个陷阱呢?但不管如何,先听听看也无妨。

方源却没有立即开口,他环顾周围,这酒铺正是他中午时分进来的那个。酒铺的掌柜经营独到,晚上的生意几乎爆棚,帐篷中声音嘈杂,人满为患。

在这里谈话,反而比僻静的地方更安全,能避免一些蛊虫的窃听。

他向贾金生勾勾手指:“你且附耳过来。”

贾金生不悦地冷哼一声,但是上身还是倾斜过来。

他听了方源的叙述之后,眉头顿时皱紧,看向方源的目光中闪着寒芒:“这生意牵涉到青茅山的三大山寨,我们做商人最忌讳掺和到地方势力的内斗里去。哼,你是贾富派来陷害我的吧?”

方源早料到他有此怀疑,他丝毫不做解释,起身就走:“呵呵,既然如此,那我就找你哥谈好了。”

贾金生眯着双眼,一直盯着方源。直到方源快要走出酒铺时,他终于坐不住了。起身追出,到了帐篷外面赶上方源:“别走啊,我们可以再谈一谈。”

方源把双手放在背后,斜看了他一眼,冷冰冰地道:“我知道你对我有所怀疑,但是现在的你被你哥压得死死的,已经快输的一干二净了。你若相信我还有一丝希望,若不信什么希望都没有了。就看你敢不敢赌这一把。”

贾金生面色一变,纠正道:“贾富不过是年龄大点,我从来未承诺过他是我哥!不过你说的不错,这把我赌了。”

方源肃容道:“两千块元石,不二价。”

贾金生苦笑起来:“这价格太高了,这买卖可是冒风险的。”

“风险越高,收益越高。”方源摇摇头,态度坚决,“你若贩到那两家山寨,保管赚得更多。”

贾金生点点头,脸色浮现出一丝认真之色:“这点我倒是相信。这些年白家寨实力膨胀的很快,又出了一个甲等天才,叫做白凝冰,大有前景。青茅山的势力格局,已经在逐渐的改变了。你们古月山寨的霸主地位已经动摇。我若卖给白家,相信至少能赚一倍!”

听到贾金生对青茅山的局势掌握得如此透彻,方源不禁重新打量了一下他,心道:“这个贾金生,到底是有出生背景,受到家庭熏陶,不是那种一无是处的纨绔。”

贾金生叹了一口气:“不管这次是不是一个坑,我都跳了。我答应你,两千块元石成交!不过,我先要看看货。”

“应该的,跟我来吧。”方源笑了一声,当即转身带路。贾金生已入瓮中,一切都尽在他的掌握当中。

(ps:感谢大家的支持,老早就新书榜前三了。同期成绩貌似比上本《御妖至尊》还要好一些。魔道不孤啊,诸位同道只是稍试牛刀,就能如此厉害,壮栽壮栽!有你们的支持,我可以预见未来的成绩会更好!

因为是新书期的关系,一般都是每天稳定两更。所以请大家不要投催更票了呀,含泪打滚呀。投了我也吃不到呀。等上了架,我会加快更新速度的。这本书至少会每天两更!

之所以敢这样担保,是因为我感觉自己越写越爽了!

大家可以看看本书的序,里面已经说了缘由。我似乎又找到了当初的感动,越写手脚就越舒展,心里就越爽。小说是作者写的,作者爽了,小说能不爽吗?

请诸位相信我,后面将越来越爽,会越来越精彩!

当然,诸君若看得不爽,那就算了。若看得爽,请多多收藏,多多投推荐票。新书需要诸位的鼎力支持!真的太需要了!

最后,感谢最近一段时间,打赏的,收藏的,每天都投推荐票的,投满分评价票的,积极评论的,指正错误的同学们。谢谢你们!)

\end{this_body}

