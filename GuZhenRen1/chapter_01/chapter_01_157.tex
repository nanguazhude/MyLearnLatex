\newsection{血月蛊}    %第一百五十七节:血月蛊

\begin{this_body}

%1
第二天,密室。

%2
彩光在白玉盘上聚焦,形成一篇秘方——

%3
血月蛊。

%4
利用二转月芒蛊和血气蛊相互合炼,形成的三转蛊虫。

%5
一经催发,月刃通体血红,脸盆大小,若形成伤口,有血流不止之效。

%6
“就是它了。”方源目光上下扫视,将这秘方记在心中。又闭上双眼默默回忆,然后又睁开双眼对照之,如此反复三五次,他确信将这秘方统统记在心中,没有半点差错。

%7
和黄金月、霜霖月、幻影月三大经典相比,这血月蛊就显得偏门许多了。

%8
在前三者的合炼秘方中,有洋洋洒洒近十万字的合炼心得和经验。但这血月蛊秘方里,只有数千字不到。

%9
可见在过往的历史中,选择合炼血月蛊的蛊师,有多么的少了。

%10
血月蛊攻击能力差强人意,射程也只是十米,就算是攻击之后,形成血流不止的特效,真实效果也并不佳。

%11
一转到五转的蛊师,因为真元有限,通常战斗并不长久,就能分出胜负。血流不止的效果,在真正的战场中,也只是一点小麻烦。碰到擅长治疗的蛊师,也会有克制的相应手段。

%12
而且血月蛊,还有一个最大的弊端。

%13
在每个月的特定几天内,它就会向外渗出鲜血。鲜血不断地流淌出来,在这期间,它的攻击力暴降到平时的三分之一。

%14
但它有个令方源看中的最大优点。

%15
它好养活。

%16
比黄金月、霜霖月、幻影月容易养活多了。

%17
它需要的食料,不再是月兰花瓣。而是新鲜的血液。

%18
血液的量虽然多,但是种类不限。在西漠不好说,但在南疆这片地方,万里山峦起伏,丛林深处随处可见各种野兽。

%19
斩杀掉它们,就能就地提取到血液。对于血月蛊来讲,它的食物遍布南疆各个地方。随处可见。

%20
“接下来就合炼这只血月蛊罢。”方源心中迅速拿定了主意。

%21
合炼的详细步骤,以及注意事项,都已经被他牢记着。他手中已经有了月芒蛊。但另一种蛊虫血气蛊,却是有些麻烦。

%22
血气蛊珍贵,能补充蛊师的血气。拥有血气蛊的蛊师。常常精力旺盛,就算是受伤,失血过多,也有血气蛊补充之。因此战场生存能力比寻常蛊师,都要高出一筹。

%23
熊姜曾经极度渴望,拥有一只血气蛊。

%24
若是血气蛊和游僵蛊搭配时候,后者的后遗症将锐减至无。使得他化身僵尸的时间大大延长,再无后顾之忧。

%25
他已经是二转蛊师中的佼佼者了,地位较高,但直到他战死。也没有得偿所愿。

%26
密室中,方源又将目光转向中央石桌上,放着的那些照影蛊。

%27
今天时间还绰绰有余,一刻钟的时间,他只用了五分钟。还有十分钟的余地。

%28
黄皮的照影蛊中,记载着的是四转秘方。紫色照影蛊中,则收录了五转秘方。

%29
这些照影蛊的主人,都是历届的秘堂家老,但真正负责喂养的,是家族。

%30
蛊虫是可以借用的。只需要得到蛊虫身上意志的承认。

%31
照影蛊中的意志,和秘堂家老一体。方源晋升家老,其身份已经得到秘堂家老的承认,因此令他能够自由使用其中的一部分照影蛊。

%32
但秘堂家老并不认为方源有权利能阅览四转、五转的秘方。因此,黄紫二色的照影蛊,就算是他催动真元,也得不到回应。

%33
事实上,就算是野生的天然蛊虫,也能得到它们的意志认可。

%34
那些兽王就是如此,因此借用了蛊虫的力量。人类中也有如此情况,就比如那五转吞江蟾传说中的江凡。

%35
方源若是凭借春秋蝉的气息,顷刻间炼化这些蛊虫,得到秘方,自然无不可以。

%36
不过,这样做的后果,显然方源目前还承担不起。而得到的利益,也并不能让他心动。

%37
“其实最珍贵的秘方,不是这些四转、五转的秘方,而是如何逆炼出一转的月光蛊。此蛊源自一代族长,正是在此基础上,经历数百年,才发展出如此庞大的规模和盛况。”方源暗忖着。

%38
合炼是将低阶蛊虫,晋升为高阶。逆炼则是将高阶蛊虫,降为低阶。

%39
一升一降之间,因为过程不同,得到的也许就是全新的蛊虫。

%40
月光蛊并非是天然蛊虫,而是一代族长逆炼而得的。

%41
这个世界中的蛊虫,很多都是蛊师在天然蛊虫的基础上,创造出来的全新物种。因此就算是方源有五百年经验,对于整个蛊虫的世界来说,也认知有限。

%42
而在家族方面,都会有一只或者几只特有蛊虫。这些蛊虫,不是稀有的天然蛊虫,就是逆炼出来的全新物种。

%43
蛊师们在此基础上,发展出一个家族的特有力量。

%44
古月一族的月光蛊,熊家寨的熊力蛊,白家的溪流蛊,皆是如此情形。

%45
若是使用寻常大众的蛊虫,那么就容易被针对。

%46
一个家族的根基,在于元泉,能产出元石。其次便是至少一种的特有蛊虫,这样才能做到自身力量不被完全解读,然后是血脉。血脉亲情是维系家族的重要纽带。

%47
因此,别看月光蛊只是一转蛊虫,但逆炼的秘方价值甚至能超过很多四转、五转的秘方。

%48
逆炼月光蛊的秘方,向来都被家族族长亲自保管。当然除去族长之外,历届都会有一位最忠心的家老,秘密知晓此秘方。同时记载月光蛊秘方的照影蛊,也被妥善隐藏着。

%49
方源显然不可能从这密室中,找到这秘方。

%50
“这秘方价值极高。若是能在临走之前,得到手自然最好不过了。不过也不必强求。”方源对此看得透彻淡然。

%51
对于他来讲,是不准备组建组织和势力的。月光蛊的秘方对他而言,并非是必须之物。

%52
“倒是血月蛊之类的三转蛊虫,才是我最需要的。”

%53
方源现在虽然是三转修为,有了白银真元,但是蛊虫还并非都是三转。发挥不出三转蛊师真正的战斗力。

%54
“我现在拥有雷翼蛊、天蓬蛊,血月蛊若合炼出来,就有三只三转蛊虫。但这还远远不足。”

%55
普通蛊师依附在家族里头。周围有族人配合,又有资源配给,因此只需要三四只蛊虫。就可以了。

%56
但是方源要闯荡南疆,远走他乡,至少需要六只蛊虫,才能勉强应付各个方面。

%57
依他的经验,这六只蛊虫要在攻击、防御、治疗、存储、侦察、移动,这六个方面给他提供帮助。

%58
攻击上,血月蛊勉强够格。防御方面,天蓬蛊可以胜任。移动上,雷翼蛊虽然消耗真元较多,但是能令方源短暂飞行。十分强大。

%59
但是治疗方面,九叶生机草就显得孱弱了。它毕竟是二转蛊虫,再往上晋升的结果,并不令方源满意。

%60
二转的九叶生机草,本身治疗能力也并不出众。它的优势只在于。能够催生出一转生机叶。售卖生机叶,能得到源源不断的元石,等若一棵摇钱树。

%61
但是方源今后,要长久行走于荒无人烟之处,就算是催生出生机叶,也找不到其他蛊师换取元石。

%62
侦察方面。地听肉耳草范围广大,虽然是二转,倒是可以勉强用着。

%63
存储方面,方源根本就是空白。这点却是重中之重,一人独行后勤为先,三军未动粮草先行。可以说,是支撑其他五个方面的基础。

%64
存储喂养蛊虫的食料,人的食物是一个方面,关键是储藏元石。

%65
没有元石,蛊师就失去了修行的动力。

%66
在这点上,方源还未有丝毫之进展。再没有搞到理想中的储藏蛊虫之前,他是不会出走山寨的。

%67
“用于存储的后勤蛊虫,首先存储的范围要广,能存储食料以及元石。然后本身也要容易喂养。最后,最好能有延长保质期的功用。但就算是在三家的物资榜中,也没有合我心意的蛊。看来只能利用赤脉,再榨一榨他们的底蕴。”

%68
一刻钟的时间到了,方源一边走出地下溶洞,一边在心中思量着。

%69
“方源大人,你好。”一位中年男蛊师,就站在出口处,显然在专门等待着方源。

%70
“你是?”

%71
来者微微一笑:“我是古月赤钟,目前暂代药堂家老之职。”

%72
“原来是他。”方源心中恍然,不由地打量这人。

%73
古月赤钟相貌端正,国字脸,浑身上下流露出一股沉稳之气。和方源一样,他也是家老身份,但是修为上已经是三转中阶。

%74
方源把古月药姬气昏过去后,古月赤钟就被族长任命暂时统领药堂。而他的妻子则是药脉的重要成员,此举便是古月博平衡两脉纷争的政治举措。

%75
但不管如何,古月赤钟凭此上位了。

%76
“这里是三百块元石,这一周家老的补贴。我知道你在这里,已经替你随手带来了。希望你不要太介意我的自主主张才好。”古月赤钟说着,递给方源一个钱袋子。

%77
“这个男子……”方源眯了眯眼,接过钱袋。

%78
家老的补贴,非得家老亲自领才行。但古月赤钟却能代领,从某种程度上讲,这是暗示方源他在家族中的人脉和地位。

%79
但这种暗示又恰到好处,掺杂着主动示好的承认,让人觉得并不咄咄逼人。

%80
“实不相瞒,这次主动来找方源大人,实在是有一事相求。”

%81
随后,他开门见山,直接说了来意。

%82
“哦,你要我上缴九叶生机草?”方源露出意味深长的神情。(未完待续。如果您喜欢这部作品,欢迎您来投推荐票、月票,您的支持,就是我最大的动力。)

%83
蛊真人157\_蛊真人全文免费阅读\_第一百五十七节:血月蛊更新完毕!

\end{this_body}

