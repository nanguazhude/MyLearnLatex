\newsection{此一时彼一时}    %第一百五十八节:此一时彼一时

\begin{this_body}

古月药姬曾经看上方源手中的九叶生机草,为此利用职权,设了一道上缴九叶生机草的政策。

但方源晋升三转,成为家老,对此作出有力还击,将古月药姬气得当场昏厥过去,令古月赤钟上台。

古月赤钟新官上任三把火,这一次主动找上方源,就是想要说服方源上缴九叶生机草。

“上缴九叶生机草,也不是不可以。”方源沉吟了一番,道。

有一句俗语说得好:此一时彼一时。

今时不同往日了……

曾经古月药姬要方源上缴九叶生机草,那是上驭下,你缴也得缴,不缴也得缴。

但如今,方源已经是家老,在身份上只弱于族长古月博,和古月药姬等人已经平起平坐。因此古月赤钟劝方源上缴九叶生机草,那就不是上驭下,而是一种平等的交易,利益的交换。

九叶生机草的确非常珍贵,催生生机叶,是供不应求的微型财源。方源如今手头上,养了这么多的蛊,很大一部分的喂养费用,都是靠它支撑下来的。

然而,这个世界上没有什么非卖品,只有利益太小,动不了心罢了。

九叶生机草虽然珍贵,但只要价格合适,为什么就不能卖?

所谓的蛊虫,也不过只是工具。是为了达到心中野望的手段。就算是春秋蝉又如何?只要情况合适。舍弃掉能换取更大利益,为何不这么做呢?

想要得到,就得先学会放弃。

方源自然有着如此觉悟,于是他看向古月赤钟。

古月赤钟了然一笑:“如果阁下愿意上缴九叶生机草,那么我愿意用这个令牌来补偿阁下。”

说着,他取出一面令牌。

这令牌形制简单,三角形状,角尖圆润。正面三个字,堆砌成塔形——“奖赐令”,后面同样如此,也有三个字——“赏有功”。

奖赐令,赏有功。

“阁下是新晋家老,有些事情可能不太清楚。这面奖赐令,是专门颁发给对家族有重大功绩的蛊师。就算是家老。也很少有人拥有。用这面令牌,就可去家族的地下虫洞。选取任意一只蛊虫。虫洞中的蛊虫。都是珍稀之物,物资榜上的前十只蛊虫,有四只就挑选自这个地下虫洞。”古月赤钟解释道。

方源顿时有些意动。

他对此并不意外,古月一族是屹立了数百年不倒的家族,这么长的时间下来,没有一些底蕴和积累,那是不可能的。

事实上。只要是历史悠久点的家族,都会有类似地下虫洞的存蛊密地。

只是要得到这里的蛊虫。必须是有功之臣,还得是有大功劳。同时忠心可嘉。

方源才刚刚晋升家老,离这资格还差十万八千里。

“我要离开家族,得需要存储蛊虫。这令牌倒是一个不错的选择。只是……”方源心中有些担忧。

九叶生机草舍了也就舍了,它的确带给方源相当大的帮助,但是却不适用于将来。

它的治疗效果并不出众,催生出来的生机叶,也不能持续治疗。

方源担忧的是,换了这令牌,去了虫洞,也未必能有理想中的蛊。

古月赤钟一直在察言观色,他误解了方源的犹豫,道:“虫洞中,不仅有三转蛊虫,更有四转蛊。方源家老你不会吃亏的,如果你日后后悔,我们也可再交换回来。只是需要一段时间,同时也得秘密进行。”

方源闻言抬眼,不禁再次看向赤钟。

“这是个人杰。”他心中轻轻喟叹。

新官上任三把火,这赤钟将自己的上位理解得很通透,不仅要向族长,还得向赤脉、药脉妥协。但同时还得表现出自身的能力。

方源上缴九叶生机草,对他来讲有特殊的意义,对此他甘愿付出高昂的代价。

“天地广袤,人杰何其多也!兽潮之下,不断有老人牺牲,有新人上位。从某种程度上讲,这亦是家族的更新换代,如此才更能保持活力,是家族屹立不倒。”方源心道。

太多的老人,会使一个组织腐朽暗沉。因此淘汰机制,是组织维持生命的不二法门。

古月赤钟只是这其中的一个代表。

大自然优胜劣汰,才能生机勃勃。人体新陈代谢,才能维持活力。蛊师手中的蛊虫更不应该一成不变,有变化才有进步。

想到这里,方源主动接过古月赤钟手中的令牌,同时将九叶生机草转交给对方。

方源主动的转让,让赤钟很快就炼化了九叶生机草。

他将这株草蛊放入空窍当中,这才在心中大大地松了一口气。

他心中有野心,暂代药堂家老对他而言,是一个极其重要的良机。若是能把握住,从暂代晋升为正式,也不是不可能。

但是要做到这点,就得需要向三方妥协。

不过政治嘛,都是妥协的艺术。

药姬在位时,拿方源没有办法。他上位之后,却能令方源上缴了九叶生机草。这不仅是再向药脉示好,将药姬的遗政贯彻,更在无形当中显露出自己的能力。

为此,舍弃了那好不容易积累出来的奖赐令,也在所不惜!

“赤钟大人,和你的交谈真是令人愉快。不知道你接下来有没有时间呢?”将令牌揣入怀中,方源道。

古月赤钟的目光不由地闪了闪,有些疑惑。

……

片刻之后。

会客厅中。

“古月冻土携妻拜见两位家老大人!”方源的舅父舅母,垂首弯腰。对方源和赤钟行礼,态度极为恭谨,甚至带着一丝惶恐。

方源晋升为家老的消息传出去后,给很多的年轻蛊师树立了标杆和榜样,也带给舅父舅母一片惊愕和恐慌。

这方源,明明只是个丙等,怎么突然就晋升三转,成了家老了呢?

就算是古月方正。甲等资质,如今也只是二转高阶啊。

这样的惊愕之后,就是惶恐了。

由不得他们不惶恐。

只要一想到,平日过往时候如何对方源刻薄欺压的,他们就有一种心惊胆战之感。

三十年河东三十年河西,莫欺少年穷!

有了家老的身份,就不一样了。

他们害怕方源发迹之后。会找上门报复。

但方源终究还是找上了门。不仅如此,还带着另一位家老。

“来者不善。来者不善啊!”古月冻土心中哀叹。

方源却微笑道:“舅父舅母不必多礼。我虽然晋升了家老,但仍旧还是你们二老的侄子嘛。来,都坐下吧。”

说着,他坐在了主位上。

古月赤钟也随后坐在另一旁。

这两个位置,本来是舅父舅母坐的,是主人家的位置。

但方源他们做了,舅父舅母却丝毫没有异议或者不满。甚至连坐在下首位置。都有些犹豫、畏缩。

这就是家老的权势。

舅父舅母对望一眼,都提心吊胆地坐下。他们正襟危坐。神态拘谨,并且只坐一半的椅面。

这时。家奴上来送茶。

古月赤钟保持沉默,并没有喝。方源则好整以暇品了茶之后,慰问道:“不知道舅父舅母,近来状况如何。”

方源微笑地问着,但放在舅父舅母的眼中,却比雷霆大怒还要可怕一些。

尤其是舅母,一想到曾经辱骂过方源的那些话,她此时害怕得身体都在微颤。

“唉,狼潮来了,生活动荡。尤其是酒铺被迫停业,几座竹楼的租金也在下降。说实话,家里都穷得揭不开锅了。”舅父说到这里,竟是垂下泪来。

他往昔里保养得很好,但如今已经再不是红光满面,双鬓白发,皱纹增多,尤其是响应征召,使得他的困顿窘迫一目了然。

但方源亦知道,他掌管酒肆多年,积蓄还是有的。如今说的可怜兮兮,恐怕是害怕自己报复,想用可怜的样子来博取同情。

“这个舅父,平日里也精明,怎么事情轮到自己身上,就愚了呢。我若是来报复,岂会还带着古月赤钟?”

方源心中冷笑,他对舅父舅母好感欠奉,但这并不代表他们身上就没有了利用价值。

“舅父舅母,你们平日照顾我良多,我能有今天的成就,还多亏了你们二老的栽培。今天我刚刚领到家老的补贴,这里是三百块元石,请你们收下它。”说着,方源就将钱袋递给舅父。

“什么?这……”古月冻土这一刻的表情,实在是太精彩了。

惊愕中含有不安,不安中近乎惶恐,惶恐之下又有些不可思议。

一旁的舅母,也呆住,愣愣地看着方源手中的钱袋。

这是怎么回事?

想象中的报复没有,反而送来了三百块元石?

方源刚刚说的一番话,他们俩怎么听,怎么刺耳。“照顾”、“栽培”,这些词,透着一股浓郁的讽刺意味,让他们更觉得蹊跷古怪。

“这个方源是有什么打算?”

“他究竟想干什么?究竟想怎么样来折磨我们!”

舅父舅母对视一眼,均踌躇犹豫,一时间不敢上前接过钱袋。

“既然是方源家老的孝心,你们就收下罢。”一旁,古月赤钟面无表情地开口道。

“是,是,是。”舅父连连点头,连忙接过。他虽然外号隐家老,但是面对如今的药堂家老,却还算不上什么。

他用双手捧着这三百块元石,以他重财抠门的心性,此时却觉得这钱袋子真是烫手无比,真想把它立即抛掉!

“既然舅父已经收下,那我就告辞了。”方源说走就走。

舅父舅母连忙要相送,却被方源止住。

“别再看了。”角落里,沈嬷嬷叹了一口气,对身旁的女儿沈翠道。

沈翠面色黯然,直到方源和赤钟的背影彻底消失,这才收回目光。

“妈,我是不是选择错了?”她道。

家老能娶一妻两妾,如果她从一而终,说不定此时就成了方源的小妾了。

“真是没有想到,这方源区区丙等资质,竟然达到了这步。”沈嬷嬷无奈地摇头,“丫头,看开点。方正可是甲等资质,绝不会比他差的!”

“嗯。”沈翠答应着,黯淡的双眸又绽放出微微的光彩来。

在这个世界上,凡人是多么的卑微!

就她身上,能让蛊师大人动心的东西,又有多少?

既然把身子都已经给了方正,那就是下了人生的赌注,再也没办法回头了。。。)

\end{this_body}

