\newsection{暗事好做,明事难成}    %第六十四节:暗事好做,明事难成

\begin{this_body}

%1
这几天,气温越来越高了。

%2
正午,烈日高照,恣意地挥洒下热情。

%3
商队走了,客栈中的生意又冷清下来。

%4
方源步入饭堂,立即引起了伙计们的注意。

%5
一个熟脸立即屁颠屁颠地跑过来,脸上堆着谄媚的笑意:“哎,公子,您来啦!快请里面坐。”

%6
“给我一坛酒,再切一斤牛肉,再来几个小菜。”方源迈进了饭堂,走向窗口的老位置。

%7
伙计的脸色露出难色:“公子,不好意思,上次商队过来,掌柜的都把青竹酒给抛售了。如今咱这里已经没这酒了。”

%8
方源点点头,倒也不意外:“那就给我来一坛米酒吧,顺便告诉掌柜的,今年多酿些青竹酒,我要预订上百坛。至于要多少定金,你们先算算,再报给我。”

%9
如今酒虫已经暴露,也没有惹人怀疑,根本就不用顾忌买酒的事情了。

%10
“好咧,这话小的一定带到!公子您就一百个放宽心吧。”伙计拍拍胸脯,语气凿凿。

%11
很快,酒菜都上来。

%12
方源一边吃喝着,一边望着窗外。

%13
大热的天气,加上又是吃饭的时候,街道上行人稀少。

%14
太阳光照在地上,以及绿油油的吊脚竹楼上。

%15
一些赤着脚,沾着泥水的凡人农夫,或是扛着铁锹,或是提着扁担,走在前面。显然刚刚干完农活,现在往家赶。

%16
两个小孩子,前边高举着竹架的小风车,小腿儿一阵急跑。后边的追着,哭喊着。似乎是前面的那个顽童抢走了风车。

%17
这时两位青年蛊师,缠着青色腰带,疾步走着,行事匆匆的样子。

%18
“滚开,别挡道!”一个蛊师猛地推开前面的农夫。

%19
农夫们仓惶避让。

%20
“哼。”两位青年蛊师一脸冷傲地走了过去。

%21
方源看着,眼中略有失神,一部分的心神已经分到空窍里面。

%22
空窍中,水膜无声地流动着,青铜真元海面波涛生灭。

%23
酒虫在元海中载沉载浮,时而舒服地打着滚儿,时而完全成一个圆团子。

%24
春秋蝉陷入沉眠,隐藏了身形。

%25
体型圆润的白豕蛊则振翅飞在上空,盘旋着。

%26
白豕蛊和黒豕蛊齐名,都是一转中的珍稀蛊虫。它们在市场上的售价,比酒虫还要高。

%27
不过它们虽然作用相同,外形相似,但是它们往后的晋升路线却不一样。

%28
黒豕蛊和青丝蛊合炼成二转的黑鬃蛊,再晋升就是三转的钢鬃蛊。

%29
而白豕蛊,最佳的晋升路线是和玉皮蛊合练,炼成二转的白玉蛊,再升为三转的天蓬蛊。

%30
钢鬃蛊能让蛊师的毛发硬如钢针,攻防一体。天蓬蛊令蛊师全身皮肤硬如白玉,同时削减类似月刃这样的攻击效果。

%31
方源心中充满了淡淡的欣喜。

%32
得了白豕蛊只是一个方面,真正的令他高兴的,是花酒行者的力量传承。

%33
“白豕蛊能增强气力,花酒行者又设下巨石堵路,看来是让我炼化这白豕蛊,将来有了力气,推开巨石,继续前行。这应该就是第一个考验了。”

%34
“顺着花酒行者布置这关卡的用心,就能推测出,在接下来的传承中,必定有第二道,第三道关卡。最关键的是可以肯定,他设置的这个力量传承,不是巨坑陷阱,而是很有诚意的。”

%35
“借助这个传承,我就能更快地达到三转,离开这青茅山,外出闯荡,占据先机了!”

%36
蛊师修行,最需要的是什么?

%37
答案只有两个字——资源。

%38
方源要修行,就需要资源。但是家族的资源是有限的,想要资源,就得争夺。

%39
不仅需要争夺,还必须在争夺中得胜。

%40
对他来讲,竞争越多,胜利越多,他的底牌就暴露越多,就越让人忌惮。

%41
忌惮累积到一定的程度,就会形成打压,阻挡他前进的脚步。

%42
方源杀了家奴,漠脉为什么不追究?抢劫了所有同龄人,那些长辈为什么不问罪?方源反抗家族,不入体制,为什么族长选择宽容?

%43
都是因为他弱小,他是丙等资质。

%44
他们自恃强大,不屑于打杀弱小。在家族的体制下,和弱小的方源斤斤计较,不仅丢人丢脸,而且会让人觉得冷酷无情,破坏了自己的交际网络。

%45
弱小,就是方源目前的一层保护伞。

%46
但是随着不断地争抢到资源,方源会表现得越来越强大。这就会让众人瞩目,忌惮,从而拉拢。不管方源选择哪一方阵营,他都会遭到另外阵营的打压和牵制。

%47
而牵制和打压,将拖慢他成长的速度。

%48
方源很清楚自己此时的处境。他现在的处境很玄妙,他看似所有人都得罪了,其实他什么人都没有得罪。

%49
但是随着时间,修为的提升,这个矛盾就会产生,就会激化。

%50
方源知道,这个矛盾迟早会激化,但是激化越晚,对他的成长就越有好处。

%51
因此,这个花酒行者的力量传承,出现的真是太妙了!

%52
有了这个力量传承,他就有隐形的资源。借助这个资源,他就可以游离在体制之外,走独自路线,从容修行,暗暗积蓄实力。

%53
一入体制,就要站队,哪怕自己再无争,也绝对会被政治以及党争牵连。

%54
入了体制的人,就是棋子。你首先得成为一个合格的棋子,别人才放心用你。放心用你,你才能有机会往上爬,爬的时候还得小心,别被当成了弃子。

%55
方源对这个过程,知道得太清楚了。纵然有千般智慧,也奈何不了这种格局。这就是规矩!

%56
最关键是他只有丙等资质,对家族来讲,根本就没有投资他的欲望。常常就有可能被当做了弃子。

%57
所以最佳的发展路线,就是独自发展,这样一来,大部分的竞争也就可以避免。在山寨高层心中,先前精心营造出来的形象也可以得到维持。

%58
“这个世界上的事情,都是暗事好做,明事难成。我借助花酒行者的力量传承,就能暗自修行,积蓄力量,不惹人注意,不招来打压。不过抢劫勒索还是要继续的,忽然中断,会惹来怀疑,同时我也需要元石。”方源思考着今后的打算。

%59
他的确需要元石。

%60
别的同龄人,是开始炼化、喂养第二只蛊虫。他则是炼化了小光蛊,又得了白豕蛊。足足有四只蛊虫。

%61
以前他喂养月光蛊和酒虫,每天将近一块元石。现在他喂养蛊虫,算下来,每天元石的消耗,要比两块还要多一点!

%62
在算上他修行所需,还有生活费用,一天下来,元石至少得要五块!

%63
五块元石,足够凡人三口之家,五个月的生活费。

%64
他手头上,虽有数百块元石,但也禁不住这样的长期花销。

%65
更关键是,越到后期,蛊师的花销就越大。尤其是升上二转之后,蛊虫每次合练,都是一笔不菲的费用。

%66
想到这里,方源就忧心忡忡。

%67
元石是个问题,单靠抢劫勒索同窗还有手头上的积蓄,只能延缓这个问题的爆发。

%68
除此之外,他还有一个麻烦,那就是白豕蛊的喂养。

%69
白豕蛊的食物,是猪肉。

%70
豕这个字,就是猪的意思。豕蛊系列的食物,都是猪肉。

%71
黑白豕蛊的食量都很大,每五天一顿,每顿都得要吃上一整头成年猪的肉。

%72
这个世界猪肉价格是不便宜的,凡人只有过年的时候,才会宰杀一头猪,沾点荤腥。没有地球上大规模的养殖技术,猪肉牛肉等等的价格,都是凡人吃不起的。同时,青茅山地形陡峭,居住面积狭小,能有多少饲养家畜的空间?

%73
靠山吃山靠水吃水,山民们平常偶尔间能吃到的猪肉,大多都是猎人逮杀的野生山猪。

%74
“看来,今后我得亲自捕猎,杀猪取肉了。”方源眼中闪过一抹光。

%75
单靠在山寨中收购猪肉,第一耗费元石,第二会引来怀疑的目光,你一个人就算是爱吃猪肉,也不会吃这么频繁,吃这么多吧?

%76
若是自己亲自捕猎,不仅这个麻烦迎刃而解,同时还能更大程度上缓解经济上的压力。

%77
“伙计,结账!”想到这里,方源再不迟疑,结了账之后,就走出了客栈。

%78
这几日学堂已经放假,目的是让学员们好好炼化第二只蛊虫。方源决定,自己正好利用这时间,外出山寨,摸清情况,然后杀猪取肉。

\end{this_body}

