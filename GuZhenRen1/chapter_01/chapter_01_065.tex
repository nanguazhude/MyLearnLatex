\newsection{还不快滚}    %第六十五节:还不快滚

\begin{this_body}

%1
漫漫山林,方源谨慎地行进着。

%2
阳光透过高大的树木,照进来,一片斑驳零碎的树影。

%3
脚边青草茵茵,野花烂漫。

%4
耳边是鸟鸣声,或是潺潺的流水声。

%5
越远离山寨,向外围行进,就越是危险。所以方源更是小心翼翼。

%6
野外是十分危险的,蛊师至少得有三转修为,才能独自一人,在野外探险。但这并不意味着安全,很多三转蛊师都死在野外,甚至还有四转。猛兽、毒虫、人祸,还有时不时的极恶天气,都会导致生命的消逝。

%7
不过,方源要斩杀的山猪,在山寨的附近就有出没。否则山民猎户,也不会偶尔捉捕到野猪了。

%8
“山寨周围的环境,每隔一段时间,就有家族蛊师出动,清理一遍。这样的环境,对于我一转中阶的蛊师来讲,还算是安全。不过仍旧要注意,野兽、蛊虫都是有流动性的。”

%9
方源凭借着前世的经验,谨慎搜索。

%10
时间渐渐地过去,方源却一无所获。

%11
“可恶,青矛山对现在的我来讲,范围还是太大了。我没有侦测蛊虫,又不熟悉这里的环境,再加上山寨附近都要被定期的清理,想要找到山猪,还是太困难了些。去山脚下!”

%12
方源搜索无获,便立即改变了主意,向山脚行去。

%13
青茅山有三大山寨,分别是古月山寨,熊家寨,白家寨。其中熊家寨在前山,古月山寨在山腰,白家寨在后山瀑布。

%14
除了这三大寨子之外,山脚下还分布有数十个村庄,生活的都是凡人。

%15
三大山寨瓜分了这些村庄,成为他们幕后的掌控者。一旦缺少家奴,都会在这些村庄中选择。

%16
但是不会选择蛊师。

%17
培养蛊师,都只会培养族人。哪怕他们也知道,这些凡人村民当中,也有修行资质的人物。虽然比例很少,但是仍旧不取。

%18
这个世界,极为注重血脉亲情,将力量掌握在亲族手中,才是家族稳固的政策根本。

%19
很多山寨,为了扩大规模,胡乱将外人纳入族群,最终导致力量外流,引发内乱,因此破败或者衰落下去。

%20
不论是哪个世界,政权的根基都是军队,这是真理。

%21
而军队就是暴力机关,就是力量。掌握了力量,才有地位和权利。

%22
当然,家族体制也不是一成不变,也会纳入新血。每年都会有外族人嫁入家族,脱离奴籍,她们生下的孩子,就姓古月,也就是新一代的族人。

%23
这就像是一口深潭,引入了一道溪流进去。别小看这道溪流,没有它,深潭就是死水,早晚要腐臭。有了它,就是活水,深潭也会慢慢壮大。

%24
方源曾经的贴身丫鬟沈翠,就是打的这个主意,要攀龙附凤,脱离奴籍。

%25
方源下了山,走了半个小时,就隐约看到了从山下升腾起来的袅袅炊烟。

%26
又走了片刻,在一处视野开阔的山坡上,他就遥遥看到了一个村庄,依着一道小河,盘踞着。

%27
这里附近的村庄,都受古月山寨的统治。虽然村庄附近,没有山寨周围那么安全,但是也定期清理。凡人在这样的环境下都能够生存,对方源来讲,自然也可以接受。

%28
“嗯?”在接近村庄的山道旁,方源敏锐地发现了地上的踪迹。过往的经验告诉他,这是一头山猪的痕迹。

%29
“追!”方源精神一振,跟着这踪迹,渐渐深入到山林当中。

%30
苍苔碧藓铺阴石,古桧高槐结大林。

%31
夏日的山林,在烈日的照耀下,反而更显得深幽。

%32
茂盛的草丛忽然动了一动。

%33
几只正在吃着丰盛野草的山鹿,立即抬起了头,目光谨慎地盯着草丛,耳朵一动一动。

%34
半人高的草丛忽然分开,从草丛中钻出一个少年。

%35
这少年肌肤苍白,黑色短发,穿着朴素的麻布衣衫,正是方源。

%36
唰唰唰。

%37
几只野鹿受到惊吓,四条纤细而又矫健的腿陡然间迈动起来,纵跃配合小跑,一下子就窜出去,消失在方源的视野里。

%38
“这些野鹿,都是雌鹿,毛皮割了,能制成保暖的皮袄,鹿肉也是一些蛊虫的食物。要是雄鹿的话,鹿角鹿茸便更加珍贵,有些金毫鹿茸,还是蛊虫晋升的必须之物。”

%39
野外是危机四伏的,同样也蕴藏着丰富的宝藏。

%40
方源看了一眼这些野鹿逃走的方向,就收回了视线。他此行的目标是野生山猪,而不是这些野鹿。

%41
他继续前行。

%42
嗡嗡嗡。

%43
前方传来这样的声音,令方源停下脚步。

%44
“蜂窝。”他在远处看了看,只见在一棵树上,挂着一个蜂窝。

%45
蜂窝硕大,有栲栳大小,呈现一种灰暗的黄色。蜂窝外不断缭绕飞行着数十只兵蜂,还时不时地有工蜂进出。

%46
“蜂窝中藏有蜂蜜,蜂蜜是熊力蛊的食物,因此熊家寨对蜂蜜需要量很大。这个蜂窝不过是个小型蜂窝,产生蛊的概率不大。我若有铜皮蛊之类的蛊虫,倒是可以冒着被蜂蜇的危险,去采集蜂蜜,可惜。”方源暗道。

%47
在这世界,不是所有昆虫都是蛊。

%48
蛊是天地之精,法则载体,哪有那么的廉价?一群昆虫中,往往只有虫王,才是蛊虫。

%49
但这也要看虫群的规模,规模太小,也是没有的。

%50
就拿这眼前的蜂群,因为族群太小,有蛊的可能性并不大。

%51
方源远远地绕过这个蜂窝,继续前行。

%52
地上山猪的踪迹越来越明显,方源知道自己离这头山猪越来越近了。他也越来越小心,山猪也是很危险的。

%53
山猪不是家猪,一头成年山猪和一头老虎相斗,未必老虎能赢。

%54
在这个奇妙的世界里,野兽也大不简单。

%55
“嗯?这个情况!”当方源终于发现了山猪时,山猪正倒在一个大坑当中,坑底竖着一根根削尖了的青矛竹,山猪被青矛竹插着,血液咕咕地往外流淌着。

%56
看这坑中积着的血液,这头山猪已经落入陷阱至少一刻钟了。

%57
不过这山猪仍旧在奋力挣扎,嘴里吭哧吭哧地嚎叫着,仍旧还有相当的活力。

%58
“这个陷坑,明显是猎人设下来的。想不到给我捡了一个便宜。”方源嘴上笑了笑,神情却有些凝重。

%59
这些陷阱,对他来讲也是一个威胁。

%60
若是自己中了这陷阱,估计半刻钟后,就要死亡。

%61
村庄周围虽然定期被清理,但是猎户也常来这里狩猎,布下了不少的陷阱。有些陷阱可以辨认出来,但有些陷阱设计得隐蔽,方源若是不察,很有可能就中招了。

%62
“这漫山遍地,任何角落都有可能被猎户们埋设陷阱。不过猎户之间,每设下一个陷阱,往往都会相互告知地点。看来我得找到一个老猎户,让他说清楚这里的陷阱布置。还有附近野兽的活动范围,有了这些情报,我就不至于像现在这样大海捞针了。”方源在心中暗忖道。

%63
这都是他身上没有侦测类蛊虫的弊端。

%64
不过要找到一只优良的侦测蛊虫,也相当不容易。

%65
短时间之内,若有猎户的情报,也勉强能应付了。

%66
这般想着,方源手腕一转,一道月刃就飞了出去,轻而易举地射中山猪。

%67
刷。

%68
只听一声轻响,整片月刃没入山猪的颈部,瞬间造成一道纤细的长线伤口。然后扑哧一声,一道血泉呈片状喷射而出。

%69
喷射的血液,造成伤口越来越大,血泉越涌越多。

%70
山猪嚎叫着,回光返照似的折腾着,最终动静越来越小……

%71
方源沉默地看着,面色一片沉静。

%72
山猪的生命流逝殆尽,它双眼瞪得溜圆,身上的肌肉还在抽搐,温热的血液填满了陷坑,浓郁的血腥气息扑面而来。

%73
“生存或者死亡,这就是自然的主旋律啊。”他在心中一叹。

%74
就在这时,隐约有人声传来。

%75
“王二哥,论我们村,打猎的本事谁都不如您啊。尤其是猎杀山猪,你们家的王老爷子可是方圆百里的猎王,大名鼎鼎,谁人不晓?”

%76
“是啊是啊,王二哥你继承了老爷子的真传,要猎一头山猪,自然是手到擒来!”

%77
“王二哥,今天王小妹怎么没有跟着你来呀?”

%78
奉承的声音后,一个刚硬冷峻的声音响起:“哼!猎一头山猪,怎么能显出我的本事?今天我要猎杀三头山猪,让你们好好看看!还有,二狗蛋,你别惦记我妹妹,小心我揍你!!”

%79
二狗蛋却道:“男大当婚女大当嫁,我喜欢王家妹子犯法么?再说了,又不是只有我一个人喜欢,我们村那个男的不喜欢?王二哥,要我说你过年也十九了,这么大也该娶个老婆生娃子了。”

%80
那个刚硬冷峻的声音又响起:“哼,男子汉大丈夫,怎么能贪图这点小小的美色!总有一天,我要走出这个青茅山,外出闯荡,见识天下,才不愧是我这男人身!”

%81
说话间,四个年轻的猎手从树林那边走了出来。

%82
当头一位猎人,身高体长,背着弓箭,肌肉贲发,双目有神,透着一股精悍气势。

%83
当他看到方源之后,他立即眉头一拧,喝道:“嗯?你是哪个村哪户人家的小孩,居然也想捡我王二的便宜。还不快滚!”

\end{this_body}

