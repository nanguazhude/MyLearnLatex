\newsection{白玉蛊}    %第一百节:白玉蛊

\begin{this_body}

转眼,就又过去了十多天。

山体内部,地下石林。

吱吱吱!

数十只玉眼石猴,在半空中闪出模糊的身影。

它们不断地纵跃跳动,向方源发动着猛烈的攻击。

若换做以前,方源必然撤退无疑。但是如今,他站在原地,神情冷漠,岿然不动。

石猴撞击、抓挠、撕咬在方源的身上,发出叮叮的脆响,仿佛它们围攻的不是一个人,而是一个坚硬的玉石巨柱。

一片微亮的白玉之光,紧紧地贴着方源的全身。这层光,虽然比玉皮蛊的翠绿玉光,要薄很多。但是防御力,却高出了两倍有余。

玉皮蛊的防御,最多只能支撑十六只石猴的同时进攻,但是现在方源却独自面对着三十多只石猴。

“我在擂台上,能赤手空拳打破了方正的玉皮蛊的防御,但若是方正使用的是这白玉蛊,恐怕就算是我打折了手骨,也破不了这防御。”

方源心中思索着,同时也分出一部分心神,关注着空窍元海。

赤铁真元海中,白玉蛊沉在了海底,时刻吸取着真元,它的表面也在散发着微亮的白色玉光,就好像是个灯泡。

每当石猴攻击到方源一次,这颗像是椭圆形鹅卵石的白玉蛊表面,也会微微的一闪。

同时,方源感觉到,真元的消耗也加剧了一丝。

“白玉蛊的防御,和玉皮蛊一样,都需要不断地向它灌入真元。同时,承受的攻击越多,真元的消耗就越剧烈。”方源心中总结着。

同时,他蓦地展开了反攻。

拳打脚踢,掀起一片风声。他的招数简单而狠辣,动势凌厉。

白豕蛊虽然消失了,但是带给方源的力量却还保留在方源的体内。

不断有石猴,被方源击中。有的被踢飞,撞在石柱上,有的在半空中就直接被打杀,变作石头后,落在地上,摔得四分五裂。

同时,方源甩手之间,月刃也一片片地飞出来,宛若死神之镰刀,收割着石猴的生命。

在赤铁真元的催动下,一转的月光蛊每一击都给石猴,造成了最大的攻击效果。

吱吱……

石猴惊恐地大叫着,不断后退。

方源反击的片刻功夫,大量的石猴灭亡,只剩下了五六只。

方源又杀一只,剩下的石猴终于崩溃了,惊惶地乱窜,逃入了石林深处。

方源也不追赶这些残兵败将,而是继续前进,向石林中央深入过去。

这些天,他都在努力地探寻传承的下一步线索。随着不断地探索,石林周边都已经踏足过,并没有任何的发现。

方源就有了明悟,隐隐猜到了花酒行者的想法。他觉得,下一步的线索很可能是被花酒行者,布置在了这片地下石林的最中央。

越往石林深处走,石柱就越是巨大,居住在里面的石猴也越多。

方源一边走,一边遥望――在石林中央,有一根最巨型的石柱,就算是十几个成年人手拉手围抱,也抱不住。

这根石柱就是他接下来的目的地。

但是,越往里面深入,石猴群的规模就越是庞大,难度自然就越高。

方源踏出关键一步,踏破了一支猴群的警戒线。

吱吱吱!

石柱上的一个个黑洞中,顿时钻出一只只愤怒的玉眼石猴,足有上百只,向方源扑来。

方源转身就跑。

同时面对这么多石猴的进攻,他纵然有白玉蛊,也招架不住。

石猴群追击了方源一阵子,陆续就有石猴放弃了追击,掉头回转。渐渐地,只剩下三十几只石猴,吊在方源的身后。

方源见时机成熟,转身便战。

一阵打杀之后,只余下几只石猴慌忙逃窜,连原先的石洞都不敢回去了。

如此轮番几次,方源斩杀了一百多只的石猴,一路上到处可见猴子死后碎裂成渣的石块。

“真元不足了。”方源检查了一下元海,叹了一口气,不得已停下了前进的脚步。

换做以往,他早就用元石进行快速的恢复。但是现在,自从合炼了白玉蛊之后,他的元石严重不足,可以说经济已经彻底崩溃。

方源将地上散落着的玉石眼球,都一一拾取,装入袋子中。

“应该就在中央那根石柱底下,但是我要到达那里,就必须打通出一条路来。”这种感觉越来越强烈,方源最后回望了一眼,打开石门,回到了第二密室。

密室的墙角,堆放着一些杂物。

一个小麻袋中,装了数百颗的眼球玉珠。方源打开它,将今天斩获的玉珠都倒了进去。

玉珠相互碰撞,发出一连串的脆响。

麻袋还有一个,里面装的是野猪牙。不过现在,方源已经不需要再斩杀野猪了。

他用白豕蛊和玉皮蛊成功地合炼出了白玉蛊,白豕蛊的消失,使得他对于猪肉的需求降至为零。

新生的白玉蛊,更偏向于玉皮蛊,食物同样是玉石。

玉皮蛊每十天要吞食二两的玉石,而白玉蛊则是每隔二十天,喂养玉石八两。

通常而言,越高级的蛊虫,摄取食物的间隔就越长。二转的蛊虫,大多数都超过半个月。三转蛊虫会超过一个月。

当然了,蛊虫越高级,食量也增多了。平均算下来,白玉蛊的喂养费用,比玉皮蛊和白豕蛊加起来,还要更多。

但是方源来讲,他有这片地下石林,根本不缺玉石。同时不需要猪肉,他也不必斩杀野猪,这就省下了他许多的麻烦,节省了许多时间。

将小麻袋的口扎起来,方源捡起旁边的一个牛皮水囊。

水囊鼓鼓的,里面装满了黄金蜜酒。早在几天前,方源就靠着玉皮蛊,顶着黄金蜂的进攻,从蜂巢中采集了足够多的蜜酒。

“我现在身上的元石,只剩下两块半。是时候去内务堂,交还家产任务了。”

方源将牛皮水囊收入怀中,退回地下甬道,挤出石缝,回到外界。

此时,才是黄昏。

冬日里的黄昏,天气晴好,并不觉寒冷,斜阳呈现出暖融融的橙红色,细腻的日光透过了松树的树冠,照在山地之上。

一路独行,向山寨走去。

不过方源并未有直接前行,而是故意绕了几个弯路,以期减少石缝秘洞被发现的几率。

冬风吹打在他的脸上,这都是自由的味道。

以前在学堂时,只能晚上偷偷摸摸的过来。现在方源晋升二转了,在白天自由出入并不会引起怀疑。

当然,更关键的还是病蛇四人的死亡,让方源独处,减少了一层束缚。

唯一的关隘就是,方源如今只是一人,身边没有组员的支持,接下来要完成家族每月至少一次的任务,势必更加艰难。

如今,经过小兽潮后,各个小组的重整已经完成,方源已经错过了这个时机。

而且因为名声的关系,方源也被其他蛊师隐隐地排挤在外。想要加入其他小组,可不容易。

“排挤就排挤罢,我越游离在外,关注我的目光就越少,对我越有利。至于家族每月的任务,都是强制的,我当然必须得接。不过……”

想到这里,方源双眸透出一抹寒光,他对此早有打算。

家族方面,强制要求蛊师每月至少要接一个任务,但是却没有强制要求,这个任务一定要完成。

完成不了任务,评价就会降低。这是几乎所有的蛊师都不愿意看到的,所以他们无不竭尽全力地去完成。

但是对于方源来讲,所谓的评价算个屁啊!

走进山寨,青石板铺就的街道上,行人络绎不绝。

一般到这个时候,都是人流的高峰期。

许多的蛊师,完成了任务,带着伤势和狼狈,回转山寨。劳作了一天的农人,赤着泥泞的脚,拖着疲惫的身躯,默默前行。

在这个世界上,人的生活并不容易,都有辛酸和痛楚。

落日渐渐垂到了山头,散播着最后一点温和的光。然后这些光,又被参差交错的枯树梢头减碎,成了一摊时光的碎片,撒在碧青的竹楼墙壁上。

“啊呀,我的玩具。”一个女童叫着,在人群中追逐着陀螺。

陀螺恰好滚动到方源的脚下,女童也撞在方源的腿上,跌倒在地。

“对不起,对不起!冲撞了蛊师大人,请您恕罪!”女童的父亲连忙赶到,看到方源的服饰,脸色惨白如纸。带着女童,当场就跪下,对方源磕头不止。

女童吓得哭了,白嫩粉红的脸蛋上眼珠扑簌地落下。

路旁的凡人看到这一幕,无不绕路而行,避退左右。

一些蛊师投来冷漠的视线,旋即又转移开去。

“不要哭了,闯祸精!”父亲又惊又怒又怕,甩手就是一个巴掌,却被方源伸手抓住,不能动弹分毫。

“只是一件小事,无须挂怀。”方源淡淡一笑,伸手摸了摸女童的小脑袋,轻声安慰道“不用害怕,没有事的。”

女童停住了哭泣,用水汪汪的眼睛仰望着方源,觉得眼前这个大哥哥真是温柔。

“谢大人宽恕,谢大人宽恕!”女童的父亲喜极而泣,向方源磕头不止。

方源继续前行。

他租的房子,就在不远处。

而在那栋竹楼下,舅父古月冻土站着,遥遥地望着方源,显然在等待着他。

\end{this_body}

