\newsection{所谓的道德和感情}    %第一百二十七节:所谓的道德和感情

\begin{this_body}



%1
一直到傍晚时分,斗蛊大会这才将将结束。

%2
会盟坡上,蛊师们的三大家族阵营,已经再不像之前那般泾渭分明。蛊师们布满了整个山坡,只是隐隐残留着三个较为密集的人群。

%3
斗蛊切磋,是一种发泄,亦是一种妥协。

%4
在这个以力量为第一价值观的世界里,有了力量就能得到尊重,才有合作的基础。

%5
自从熊力挑战过方源之后,就再也没有蛊师来找他。

%6
方源只是个新人,还没有和其他两家的蛊师产生多少交接,更没有什么恩怨情仇。至于本寨子里的蛊师,在这样的场合下,也从不会挑战同族之人。

%7
尤其是,方源主动认输,更让其他的蛊师失去了对他的兴趣。这样的一个“惫懒”、“懦弱”的对手,就算是战胜了又有什么值得吹嘘的呢?

%8
不想被人踩,有两个途径。一个是变成强者,强大到让任何人都不敢踩。第二个是变成狗屎,让其他人都不屑踩。

%9
方源并不在乎什么狗屎、强者的名誉,他向来行事不择手段,并且只在乎结果。没有人来约斗他,这让他甚是轻松。

%10
周围人鄙视、轻蔑的目光,他更是视若无睹。若是连这些目光都承受不了,他还谈什么魔道追求?

%11
会盟成功地结束了,古月一族、白家、熊家暂时达成了盟约。

%12
第一,颁布法令。三族暂时抛去过往恩怨,共同对抗狼潮。危急时必须相互援助。同时严禁内部残杀,设立调查督战组,一经发现,杀人者首先被驱逐家族,同时三家联手制裁,杀人者必须偿命。若杀人者逃亡在外,将以其亲属抵命。

%13
就算是蛊师死亡。从尸体上回收的蛊虫也必须上缴,私自吞没使用,一律以杀人罪处理。蛊虫上缴之后。可兑换战功。

%14
第二,设立战功榜。以小组为单位,时刻显示三大家族各个小组的战绩排名。一颗电狼眼珠兑换十点战功。战功可兑换蛊虫、元石等等物资。

%15
战功榜的出现,自然激发了蛊师们无以伦比的热情。

%16
斗蛊切磋的结果,并不能说明双方真正战斗起来的结果。就拿熊姜和白病已来说,除了水钻蛊之外,白病已自然还有其他的进攻手段。而熊姜的影殇蛊的有效范围,却只有十米。

%17
并且任何的一场战斗,对于结果的影响因素都会很多。

%18
斗蛊切磋缺乏一定的说服力,但是战功榜却提供了一个更为公平的竞争舞台。

%19
三大家族的蛊师们,无不奋力狩猎电狼,争取战功榜上更好的名次。

%20
这不仅关乎自身的荣耀。更关系着家族的脸面。

%21
尤其是战功榜的前三名,呈现出空前激烈的竞争景象。几乎每隔一天,前三位的名次都会有所变化。

%22
很快,一个多月的时间就过去了。

%23
冬去春来,万物萌发。

%24
一场战斗之后。

%25
方源踩着地上的残雪。鼻息微喘,环顾着身边的战场。

%26
战场上,倒着十多头电狼,都已经死了。狼血和碎尸随处可见,空气中亦是弥漫着浓浓的血腥气味。

%27
嗷呜……

%28
就在这时,不远处忽然又传来狼嚎之声。

%29
方源面色微微一变。凭借着经验,他知道有一支狼群正在迅速接近这里。

%30
若换做其他蛊师,历经一场激战,空窍中真元不足,恐怕已经打起了退堂鼓。但方源却不管这些,仍旧蹲下身子,采集狼尸中的眼珠子。

%31
他动作熟练,极有效率,但即使如此,当他采集完毕之后,已经显露了狼群的半包围当中。

%32
这是一支中型狼群,足有上百只的残狼,绿莹莹的狼眼流露出凶残的气息,紧紧地盯住方源毫不放松。

%33
方源将狼眼收好,微笑着站起来,然后身影如水波荡漾了一下,缓缓地消失在了原地。

%34
奔袭而来的狼群,顿时骚动了一阵。不少残狼,脚步都顿住,都露出疑惑犹豫之态。

%35
毕竟都是野兽,对于这样的神奇一幕,难以理解。

%36
“不过,这也是因为电狼主要利用一双狼眼观察,而并非是利用鼻子。电狼是这个世界的奇妙生物,它们的视力有如苍鹰般锐利,但是嗅觉却不比人类好到哪里去。我的隐鳞蛊恰好克制电狼这种野兽,但是却瞒不过一只狗的鼻子。”方源心中有数的很。

%37
隐鳞蛊是他在斗蛊会之前,就已经合炼成功。它就像是一块鲤鱼化石,灰扑扑的,躺在方源的真元海中,任由真元之水在它表面栩栩如生的鱼鳞上流过。

%38
而方源先前缺少的鱼鳞蛊,自然是用黒豕蛊和青书兑换而来的。

%39
有着这只隐鳞蛊,方源在狼群的眼皮子底下安然远去。

%40
这种情形,在这些天里,已经发生了许多次了。

%41
一般的狼或者犬,嗅觉都会很灵敏。但是电狼不同,它的速度很快,没有锐利的视线,是会撞到树木和山石上的。

%42
不过大自然是公平的,在赋予了它们锐利的狼瞳的同时,剥夺了它们的灵敏的嗅觉。

%43
然而,雷冠头狼又不同。

%44
方源有了隐鳞蛊,完全能在普通的狼潮中游刃有余。但是在雷冠头狼这样的万兽王的面前,却是无所遁迹。

%45
皆因雷冠头狼的双眼中,往往寄居着电眼蛊,此蛊能窥破隐行之身。

%46
除了电眼蛊之外,其实还有其他的许多蛊虫,能察觉到隐去身形的方源。

%47
就比如说蛇信蛊,可以感知热量。兽语蛊,能让蛊师从和野兽的对话中得到讯息。顺风耳蛊,能使蛊师的听觉能力变得极其灵敏。

%48
所以。有了隐鳞蛊,并不代表方源就能安全无恙,顶多是有了一个小小的保命手段。

%49
回到山寨时,还是晌午。

%50
春日明媚,山寨门前人来人往。和往年不同,路上的行人多数都是蛊师,只有极少数的凡人。

%51
在狼潮的影响下。野外并不安全。猎户们完全不敢上山狩猎,农田也都荒废了。

%52
街道上,蛊师们却是士气高涨。一个个血迹斑斑地从外归来,或者精神焕发地出发。

%53
他们或是谈论着战功榜上的名次,或是探讨着猎杀电狼的经验心得。或是议论着其他寨子的优秀蛊师。

%54
方源夹杂在人群中走入山寨,来到家主阁前的广场。

%55
广场上已经搭上了台子,兑换战功就在此处。

%56
大量的蛊师汇集在这里,一转蛊师们记录着。一部分二转蛊师则相互拥挤着,用沾着血迹的狼眼换取战功。

%57
还有一部分,则在消去战功,用来兑换元石或者蛊虫,食料等等。

%58
广场中央,一根巨幡竖立起来,幡面上游动着这个世界的文字。时时刻刻地变化着。

%59
这是战功榜,上面寄托着十多只游字蛊。

%60
一转的游字蛊,被一转的后勤蛊师们操纵着,能在幡面上随心所欲地转化成文字内容。

%61
“哼,第一名怎么还是白家的白病已小组?”有人望着战功榜。皱起眉头。

%62
“我看看,白病已是第一,我族的青书小组是第二,熊力小组是第三……我的小组,是第一百三十七位。”有的蛊师则默数着。

%63
这时忽然有人叫道:“变了,变了!青书小组上了第一位次。将白家的病秧子挤下去了!”

%64
战功榜上,标志着第二位青书小组的文字,忽然由静化动,向上攀登。将第一名的白病已小组一“手”扯了下来,然后自己登上第一位,同时还用“脚”踩了踩。

%65
文字上这种人性化的动作,当然是操纵游字蛊的蛊师所为。

%66
看到这番有趣的变化,广场中的蛊师都爆发出一阵哈哈大笑,对在场的青书小组竖起了大拇指,交口称赞不绝。那位操纵游字蛊的一转蛊师更是满面红光,带着兴奋的神色。

%67
“青书大人,您真不愧是我们二转蛊师的第一人!”

%68
“青书大人,您真是好样的。”

%69
古月青书走在人群中,微笑以待。身后的古月方正则握紧双拳,亦步亦趋,神色中流露出激动的情怀。

%70
周围的赞誉扑面而来,充斥了少年渴望荣耀的火热心田。这让方正觉得自己正走在光明的坦途上,无形中他对家族的认同更加深了一步。

%71
“哥哥……”他意外地瞥见人群中默默站立,抱臂旁观的方源。

%72
“哥哥,你还是形单影只,独自一人啊。难怪你的小组从一开始,就在战功榜上垫底呢。只有自己主动敞开心扉,融入集体,才能感觉到家族的温暖和快乐啊。”方正心中叹息一声,忽然觉得方源有些可怜可悲。

%73
像方源这样,独自一人打拼,身边没有同伴,根本就感觉不到那种家族中的亲情,战场中的战友情。

%74
一个人面对所有的一切,不仅危险,更是无趣。

%75
一个人活着,若没有友情、爱情、亲情,他活着还有什么意思?

%76
方源站在人群中,默默地看着高高竖立着的战功榜。毫无疑问,在战功榜最后一位,写着“方源小组”四个大字。

%77
别的蛊师,看到这里,都以为耻辱。但方源目光平淡得很,不以为意。

%78
他被古月博提拔为组长之后,就从未打算过招揽其他组员。他是光杆组长,青茅山三大家族中,最特殊的小组。

%79
只有他一人,当然每天狩猎的成果就没有其他小组的多。当然,依方源目前的个人实力,真正爆发出来,排位绝非是最后一位。

%80
但这对方源来讲毫无意义。

%81
他不是太需要战功的,生活物资、蛊虫的食料,他都有所囤积。先前打算用战功兑换一只鱼鳞蛊,但现在他已经不需要了。

%82
他现在每天狩猎电狼,一个是应景儿,做做样子,第二个就是积累一些战功,用来换取月兰花瓣。

%83
毕竟,他的月芒蛊还需要这种食物。

%84
大半个月的时间一晃即逝,春意渐浓。

%85
树木抽出新绿,路边野花盛开。

%86
温暖的春风醺醺欲醉,蛊师们在战斗中凯歌高唱,战局形势越来越好。

%87
方源走在街道上,看到的都是众人洋溢的笑脸,听到的都是相互夸赞,标榜勇武的话。

%88
当然,也有一些蛊师忧心忡忡,愁眉不展。这样的蛊师,大多都是老者,人生经验丰富,知道真正的狼潮还在夏秋两季。

%89
方源心中一片了然。

%90
“造成这样的情形,一个是因为三寨联盟,降低了相互间的防备,因此有更多的蛊师投入到了抵抗狼潮的事情中。第二个原因,是剿灭的电狼,只是残狼。第三是春天到了,狼潮中电狼群忙着交配繁衍。等到了夏季,真正的精壮狼群将如潮水般从四处涌来。对三家山寨都将造成巨大的伤亡。”

%91
想到这里,方源眼中寒芒一闪。

%92
今年的狼潮厉害程度,将超越历史,达到空前的地步。就算是家族高层,也大大低估了狼潮的严重程度。

%93
记忆中,大部分的蛊师都将死去。三大家族积累出的底蕴,都几乎将消耗一空。

%94
方源从未想过去提醒家族高层。一来就算是提醒了,家族也未必重视他的这个建议。重视了更麻烦,他解释不清楚这个消息的来源。二来提醒了也没用,这是实力差距。三者,也是最关键的原因,提醒家族并不符合他的最大利益!

%95
宁我负天下人,不教天下人负我!!

%96
什么亲情、友情、爱情,不过是人生中的一些点缀,能和男人的大业相比么?

%97
地球上,项羽要烹了刘邦的父亲,汉太祖刘邦笑着说:烹吧,不要忘了到时候分我一块肉。

%98
唐太宗李世民玄武门杀兄弟,曹操的军队没有军粮了,杀人煮成人肉干,当做口粮。刘备借荆州有借无还。

%99
高层宣扬鼓吹的价值观、道德观,都不过是维护统治的工具罢了。

%100
真要被这些束缚住,还能成什么大事?

%101
踏上真正高位的人,哪一个双手不沾着血,踏着尸骨?资本家的资金原始积累,都是血腥的。任何的政治家都是肮脏的。所谓的慈善家,都是拿金钱换取社会名望的。

%102
只是成功者,大多矫饰遮掩了过往的不堪。被成功者的自传骗住的人才是蠢材。

%103
“这些蠢材比比皆是,被感情和道德框住,活该被统治被愚弄被摆布。更可悲的是,见到其他人没有被框住,往往就会跳出来,不断指责和批评,想要将这样的价值观灌输给自由者,见不得其他人比他们自由。并且在这个过程中,还享受着一种莫名其妙的道德优越感、幸福感。”

%104
方源想到这里,望着走在身边的蛊师们。

%105
这些人,空有一身本领,有的人修为比他都要高。但又有什么用?

%106
都不过是些棋子,被框住的奴犬罢了。

%107
其实真正禁锢一个人成就的,往往不是天资,而是思想。

%108
任何一个组织,人一生下来,就被不断地灌输价值观,不断地洗脑。要想有超凡脱俗的成就,就得打破思想上的项圈。可惜大多数人,都被项圈牢牢套着,驱策着奋力前进,并且还把这项圈当做一个荣耀的标志。

%109
方源暗想着,不由地冷笑几声。

%110
出了山寨大门后,他将散漫的思绪收回来。

%111
今天有一件重要的事情,他准备再入石缝秘洞!(未完待续。如果您喜欢这部作品,欢迎您来投推荐票、月票,您的支持,就是我最大的动力。)

\end{this_body}

