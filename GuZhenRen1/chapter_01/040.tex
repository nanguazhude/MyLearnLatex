\newsection{紫金石中蟾蛊眠}    %第四十节:紫金石中蟾蛊眠

\begin{this_body}

越往深处走,越是锦绣繁华。

小地摊越来越少,大帐篷越来越多。

只见红蓝黄绿各种大帐篷,有的搭成方形,有的是圆桶状。有的在入口卷帘处,竖着两根门柱子,有的则挂着大红的灯笼。有的里面灯火辉煌,有的里面却暗淡无光。

方源边走边看,最终在一座灰色的帐篷处停住脚步。

“是这里了。”他抬眼打量,只见这帐篷门口设着两个立柱,立柱上用阴刻手法,刻着一副对联。

左边:小施勇气,得春夏秋冬禄。

右边:大展身手,获东南西北财。

中间还有一个横批:时来运转。

没错,这是一家赌场。

这赌场占地一亩,已经属于大型的帐篷。

方源走了进去。在帐篷里面依着边,摆放着三排柜台。柜台上放着一块块的琥珀或者化石。小的只有巴掌大小,大的有脸盆之大。也有更大的,足有一人高。柜台上自然摆不下,就直接立在地上。

和其他帐篷商铺的热闹不同,这里面却是悄然寂静。

有三三两两的蛊师,站在柜台前,有的细心端详着柜台上摆放的石头,有的拿起化石在手掌中小心地搓动,感受手感,有的和同伴小声讨论,有的在和店家伙计商量价格。

但是不管他们说什么话,都是轻声细语,尽量不打扰其他人。

这是一家赌石场。

在这个蛊的世界里,有各种各样,千奇百怪,数不胜数的蛊。蛊虫有各自特定的食物,没有食物,蛊虫只能坚持一段时间,就会死亡。

但是大自然,对于生命,是既冷酷又仁慈的。

若是缺乏食物,蛊虫也有一线之生机。那就是陷入沉睡,自我封印。

例如月光蛊没有了月兰花瓣,可能就会封印自己。它会将力量尽量的收缩,类似于冬眠一般,陷入最深沉的睡眠当中。这个时候,它的身体不仅会蓝芒消散,而且会从透明的水晶状褪变成一块灰石,笼罩上一层石壳。天长日久,石壳越来越厚重,就会形成一块顽石。

再例如酒虫,若是它自我封印,就会结成一块白色的蚕茧。它会蜷缩身躯,在蚕茧中沉睡。

当然这种封印沉睡的情况,并不是在所有蛊虫的身上都会发生。它发生的概率很小,正常情况下,蛊虫都不会沉睡,而是被饿死。只有少数个别的蛊虫,才会在特定的情况下,自我封印。

一些蛊师意外地得到这些封印了蛊虫的顽石或者虫茧,唤醒其中沉睡的蛊虫。有的因此小发了一笔横财。有的飞黄腾达,生命轨迹迎来了转折。这些情况,在蛊师世界屡屡发生,常常有或真或假的流言风语,引人遐想。

这家赌石场中的顽石来源,就在于此。当然这些石头,都是外形疑似。要开出来,才能确定里面是否真的藏有蛊虫。

“像这种小型赌石场,十块石头有八九块都是实心的,里面没有蛊虫。就算是有蛊虫的石块,也未必都是活虫,十有八九都是死蛊。不过一旦赌到了活蛊,大部分情况下,都能大赚一笔。如是蛊虫极为珍稀,那么不是从此飞黄腾达,就是被杀人越货。”

方源心中透亮,他对这里面的门道都很清楚。

前世的时候,他也参加过商队,在赌石场当过伙计。后来一段时间,他甚至经营过一家赌石场,比这场子还大,是中型的赌石场。坑过不少赌徒,也走过眼,几次被其他人赌出价值珍贵的蛊虫。

方源在门口站了一会,目光扫视了一圈,这才慢慢走到左边的柜台。

柜台后,间隔几米就站着一位店家的伙计,有男有女。他们腰间都系着青色腰带,不是凡人,都是一转蛊师。大多数是初阶,有个别的是中阶。

见到方源来到柜台前面,一个离着他最近的女蛊师便走过来,脸上浮现出笑容,轻声地道:“这位公子您需要什么蛊虫?这边的柜台每块石头都均售十块元石。您要是首次尝试,小赌怡情的话,不妨去右边的柜台,那里的石头只卖五块元石。若是您想要刺激,不妨去正前方的高等柜台,那里的顽石一块售价二十元石。”

这是个有经验的女蛊师,在赌石场做工已经不少时间了。

她看到方源进来后,从他的外貌,年龄,身高等等方面,就推测出他是一名学员。

来赌石的都是蛊师,不会有凡人。而学员只能算是预备蛊师,刚刚踏上修行,通常因为喂养蛊虫而经济拮据,哪里会有什么闲钱来赌石呢?

像这样的学员,往往只是进来看一看,开开眼界,图个新奇。绝大多数都是只看不买,有个别少数家境不错的,也许会买个尝试一下。不过通常也只买最便宜的化石。

因此,对于方源能购买多少石头,女蛊师并不期待。

“我先看看。”方源面无表情地向她点点头,然后埋头观看。

记忆中,应该就是这家赌石场的这边柜台。

但五百年,真的太久了。很多东西都模糊无比,尤其是五百年的记忆实在是个庞大的储量。老实讲方源记得不是很清晰。

只是隐约记得,就是在这一年,商队到来的第一晚,有一个幸运儿花了十块元石买了一块石皮上泛着紫金色泽的化石。

他当场解石之后,得到了一只癞土蛤蟆。而后这只蟾蛊被人收购,他因此赚了不少的元石。

方源看了一会儿,眉头微微皱起。

这柜台上,外表泛着紫金光彩的化石,有二十多枚。究竟在哪一块里面,藏有癞土蛤蟆呢?

这边的每块石头,均售价十块元石。方源如今身上有九十八块元石,按这样估算最多能购买九块。

但是实际上并不能这样算。

任何的冒险和赌博,都得考虑后果。

方源早就不是那种愣头青,亦不是自命不凡的赌徒。以为自己被命运垂青的人,通常都折在命运的捉弄之下。

“我独自一人,没有亲朋好友的资助。必须留下一些元石支撑生活,以及购买食料喂养蛊虫。”他稍稍算了一下,在基本保障之后,他最多能购买七块化石。

“这块石头,紫金如星点缀,但是扁平如饼,里面是不会藏有癞土蛤蟆的。”

“这块紫金耀眼,但是只有拳头大小。若真藏有癞土蛤蟆,应该至少比它还要大上三成。”

“这块紫金化石,大是大了,但是表面光滑至极,而癞土蛤蟆的石皮都是坑坑洼洼。显然不是……”

方源不断打量观察,排除筛选。

蛊虫封印沉眠之后,形成的化石浑然一体,本身杜绝世间大多数的探测手段。剩下的一些探测方法几乎都强硬蛮横,一旦用了,就会直接杀死里面奄奄一息的蛊虫。

因此蛊师来选石,只能靠猜测,靠经验,靠运气,靠偶尔间的灵光一闪。

若非如此,这里也不会叫做赌场了。

当然这大千世界,无奇不有,也不排除有极个别的温和的探测手段,能让蛊师事先知晓石头中是否藏有蛊虫。

方源前世就听说过一些风声,但是后来经过验证,发现都是谣传。

方源私下设想:若真有这种手段存在,也必定是秘密传承,只会掌握在极少数的神秘人手中。对整个赌业大局是没有影响的。

在青茅山这一带还好些,越往东,赌场就越盛行。到了白头山区域,每家山寨都设有赌场。有些大型山寨中,还有大型赌场林立。在以赌石远见闻名的三大山寨:磐石寨、古墓寨、苍鲸寨中,更有超大型的赌石场。

这三家超大型的赌石场,每一家都有上千年的历史。如今仍旧生意红火,赌徒络绎不绝。从未发生过什么被人扫场的事情。

如今方源所在的这间帐篷,充其量只能勉强算是小型赌场。

但若是其他十五岁的少年来这里,必定会被这些化石缭乱了双眼,就算是挑选化石也是纯粹瞎蒙乱指。

但是方源不同。

首先,他事先就已经知道一部分的答案,因此选择范围就骤然缩小到三十块以下。

当然从这二十多块化石中,挑选出唯一的那块,也极不容易。

但是方源却有五百年的经验打底,靠着这样雄厚的底蕴,他观察了片刻,筛选出最符合标准的六块紫金化石。

他有八成的把握可以肯定――癞土蛤蟆就在其中的一块紫金化石中沉眠着!

\end{this_body}

