\newsection{二转初阶!}    %第八十一节:二转初阶!

\begin{this_body}



%1
脚步声越来越近,很快陡坡边缘的深绿树丛被强硬地分开,一个高大的大汉踏上陡坡,现身在方源的视野中。

%2
他一头黑色短发,每一根头发都刚劲笔直。赤裸着上身,虎背熊腰,浑身肌肤都是赤红色的。

%3
他身高近两米,深秋寒意重,但是他给人的感觉,就像是一个移动的火炉,似乎他的每一次鼻息,都能将周围的温度提升。

%4
他的腰间挂着一群的野兽尸体。有狐狸,有野兔,有山鸡,还有那只刚刚离去的老狼。

%5
看到方源,他微微愣了一下,随即迈开大步,和方源擦肩而过。

%6
“古月赤山。”望着巨汉离去的背影,方源的心底浮现出一个名字。

%7
此人是赤之一脉的代表人物,二转高阶修为。说起来,他的经历和方源还有一些相似呢。

%8
这人也是天赋异禀,小时候时力量就大,十岁那年失手将一个成年家奴打死,十二岁时能拿起沉重的石磨,当成飞盘玩耍。

%9
在当时,被族人普遍看好,以为是甲等资质。但是开窍大典之后,测出他资质只是乙等。

%10
他原本性情狂妄,目中无人,经历此变故之后,心性转变,变得沉稳有加。虽是乙等,但在他们那届,是名副其实的第一学员。

%11
一年学满之后,从学堂出来,不断打拼,很快就崭露头角。数年之后的如今,他已成为家族二转蛊师中的精英。

%12
幸福不能教给人生活的真理,往往只有痛苦才能。

%13
“在家族,少年在十五岁参加开窍大典,进入学堂。十六岁从学堂毕业,组成五人小组,完成家族任务。同时也有了继承家产,分家独立的资格。从十六岁开始打拼,修为不断进步,任务的危险度不断升级,地位不断提高。有的人死了,有的人活着。有的人伤残,修为下降,苟延残喘。有的人经过打熬磨砺,成为三转蛊师,晋升家老,成为高层。”

%14
方源目光闪烁着,联想到了很多。

%15
蛊师越往后修行,越是困难,晋升的难度呈几何倍数暴涨。加上艰难困苦的生存环境,能够晋升三转的,真是少之又少。

%16
“说起来,已经快要入冬。我在学堂中,已经快满一年了。每届学员都有两项重大考核,第一项是年中考核,内容每届不同。第二项是年末考核,内容不变,都是擂台战。擂台战之后,我就不能再居住在学堂宿舍,必须搬出来。”

%17
搬出来,住在哪里呢?

%18
方源不可能和舅父舅母住在一起,他们巴不得这样呢。

%19
在这个世界,十六岁就是成年人,开始独立的年龄。再加上方源自身秘密太多,也必须独自生活。

%20
“前世从学堂出来,我刚刚一转中阶。今生的话,情况要好很多,到那时定然已经是一转巅峰。不过能以丙等资质,取得如此成就,也不是没有代价,这过程中消耗了大量的元石。”

%21
方源眉头悄然皱起,他手头中的元石,已经不多了。

%22
资质所限,他修行所耗费的元石,要远比方正、赤城还有漠北要多得多。

%23
他一个人就养了六只蛊!

%24
除此之外,酒虫精炼,温养空窍,催动白豕蛊增添力量等等,都需要消耗真元。真元消耗之后,单凭丙等资质的恢复速度,必然满足不了他的需求,只好动用元石,汲取其中的天然真元进行补充。

%25
幸好他有春秋蝉,又两次从地藏花中取蛊,因此炼化蛊虫不需要耗费大量元石。这才让他稍微好过一些。

%26
但是接下来,他从学堂出来,至少得租房子,添置一些家用。

%27
巅峰之后,就是冲击二转。这个过程,会耗费大量元石。

%28
二转之后,他还要合炼蛊虫,每一次合炼都是一笔不菲的开销。

%29
单单以上这些,他越加拮据的财政状况,就已经负担不了。更何况,就算是捱过这些,他还要继续喂养蛊虫,继续修行。

%30
之前在年中考核,如果不是他用了野猪牙换了不少的元石,缓解了很大一部分压力。否则他现在就已经支撑不住了。

%31
“元石、元石……花酒行者的传承中,并没有元石供给,这真是个遗憾。从同窗勒索过来的元石,是最主要的支持力。但是学员从学堂出来后,学堂补贴就会停止了,我也不能继续再勒索下去。不过年末考核的第一,似乎可以获得一百五十块的元石奖励。”方源在心中琢磨着。

%32
如果能得了第一,有了这一百五十块元石,他那巨大的经济压力就能稍微缓解一下了。

%33
……

%34
时间匆匆,秋去冬来。

%35
学堂的演武场上,三个擂台都已经搭建好。

%36
在擂台旁边,靠着演武场的竹墙边上,还搭建了棚子,摆放了长桌和靠椅。

%37
学堂家老、族长,以及其他几位家老,此时都坐在棚子下。

%38
天空下着小雪。

%39
五十七位学员,笔直地站在场中。一个个的鼻子都冻得通红,随着呼吸,吐出一口口的热气。

%40
学堂家老大声道:“转眼间一年结束了,这一年里你们在学堂中受到训练,已经初步具备了蛊师的素质。明天,就是年末擂台考,将检验你们的学习成果!届时,不仅族长和几位家老大人都将亲临现场,同时还有你们的前辈,你们的学长旁观考察,挑选出表现优异的人,吸纳进不同的小组。”

%41
“明天你们的表现,将很大程度上影响着你们的前途。考核的第一名,不仅有一百五十块元石之奖励,而且还有挑选蛊虫的优先选择权!现在,开始你们在学堂中的,最后一次修为检测!”

%42
说完这话,学堂家老就向一旁站着的蛊师点点头。

%43
那中年女蛊师接到示意,便照着名单,念出第一个名字:“古月金珠!”

%44
一名少女带着一脸的紧张越众而出,走到蛊师面前。

%45
蛊师伸出手,抚摸在少女的腹部,闭眼感受了一番后,睁开双眼宣布道:“古月金珠,一转中阶。下一个,古月鹏。”

%46
一个又一个的少年,上去检测,又走下来,归入队伍。

%47
他们表情不一,或是失落,或是欣喜。

%48
最差的成绩,自然是一转初阶。无一例外都是丁等资质的少年。

%49
大多数人的修为则是一转中阶,这些人中少数是丁等资质,大部分资质都是丙等。

%50
“古月赤城。”中年女蛊师叫出名字。

%51
在所有同龄人中,个子最矮的古月赤城挺着胸膛,走了出来。

%52
检测了一番后,中年女蛊师睁开双眼:“古月赤城,一转巅峰!”

%53
检测至今,这尚是第一个一转巅峰的蛊师。

%54
在座的家老们都不由地轻轻颔首。

%55
有家老认出来,轻声道:“这个是古月赤练的孙子,有乙等资质,也难怪了。”

%56
棚子外,少年们也在交头接耳。

%57
“赤城是一转巅峰了,不晓得漠北是不是?他们俩可是死对头啊。”

%58
“能升到巅峰,都是乙等、甲等的人。唉,我们这些丁等、丙等的苦哈哈,是羡慕不来的。”

%59
“哼!”古月漠北冷哼一声,看着赤城得意洋洋的样子,他就感到不爽。

%60
古月方正则暗捏双拳,双唇抿着,似乎憋着一股气。

%61
“古月漠北。”很快,检测蛊师就叫道。

%62
长着一副马脸的漠北,快步走出去。

%63
“古月漠北,一转巅峰。”在这样的声音下,他又走了回来,在途中不甘示弱地瞟了古月赤城一眼。

%64
检测继续着,空中的雪越下越小,最终停止。

%65
空气冰冷而又清新。

%66
“古月方源。”中年女蛊师这时叫道。

%67
方源面无表情,走了上去。

%68
片刻后,女蛊师睁开双眼,有些意外地打量了方源一眼,大声地道:“古月方源,一转巅峰!”

%69
“一转巅峰,有没有听错啊?方源居然修到了这样的程度?”少年们一阵惊奇。

%70
“唉,他走了狗屎运,有酒虫帮助温养空窍。虽然只是丙等,但和乙等、甲等比起来,仍旧不落下风。”一些少年嫉妒地道。

%71
尤其是一些同样是丙等资质的少年,语气酸溜溜的自我安慰着:“这也没有什么。酒虫不能精炼二转的真元,方源以后就没有这个优势了。”

%72
“虽然是巅峰,但终究只是丙等资质,不足为虑。”漠北和赤城望了方源一眼,又很快转移了视线,看向仍旧站在队伍中的方正。

%73
在他们的心中,拥有甲等资质的方正,才是竞争对手。

%74
“哥哥,你真是让我有些意外。不过接下来,你可看好了……”方正看着方源一步步走下来,双眼炯炯发亮,流露出十分期待的神情。

%75
“古月方正。”女蛊师的声音,姗姗来迟。

%76
“是那个甲等天才?”一时间,家老们都将目光集中在方正的身上。

%77
方正越众而出,他能感受到这些目光传达过来的压力,这让他不禁微微地紧张起来。

%78
不过,当他看到族长古月博在对他微笑的时候,他心中的紧张顿时冰消瓦解。

%79
他走到中年女蛊师的面前。

%80
女蛊师闭上双眼,然后猛地睁开双眼,用惊讶的口气叫道:“古月方正,修为——二转初阶!”

%81
轰。

%82
少年们顿时爆发出一股嘈杂的音浪。

%83
“什么,竟然达到了二转?!”

%84
“不愧是甲等资质的天才啊。”

%85
“了不起,这样一来漠北、赤城、方源都被他甩下去了。”

%86
“这个方正!”一时间,漠北和赤城都震惊地望着方正。

%87
“呵呵呵,居然比前世还要高了一筹……”方源垂下眼帘,笑了笑。他并没有太意外,刚刚他留意了一下方正的神色,就隐隐猜测出这个检测结果了。

%88
“到底是甲等资质。”

%89
“我族的希望啊。”

%90
“还是族长大人培养之功……”

%91
家老们交口称赞。

%92
一时间,方正成为了众人瞩目的焦点。

%93
半年前,古月博交给他一只玉皮蛊,让他成为晋升二转的第一人。他做到了!

%94
“族长大人,我没有辜负您的期望,我做到了!从今以后,我还会做到更多,获得诸位家老大人们更多的认同,获得身边人更多的认可。哥哥,你已经被我甩在了身后,你再也不是我心中的阴影了。我古月方正,已经不再是从前的那个古月方正了!”

%95
方正心中振奋地叫着,他的双眸中闪灼着一种绚烂的光彩。

%96
这种光彩,就叫做自信!

\end{this_body}

