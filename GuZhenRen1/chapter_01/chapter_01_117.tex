\newsection{苦贝酒和吞江蟾}    %第一百一十七节:苦贝酒和吞江蟾

\begin{this_body}



%1
大厅内气氛凝重至极。

%2
一众家老默默地坐着,脸色或是冷漠,或是阴沉,或是沉重。

%3
族长古月博坐在主位上,亦是难掩眼中的忧愁:“三日前,山脚的村庄附近,出现了一头吞江蟾。此蟾似是从黄龙江逆流而上,无意间流落此处。它现在堵住一处河道,睡在里面。若是放任它,山寨就要时刻处在危机当中。在座的诸位家老,有什么良策,能驱赶了此蟾?”

%4
家老你望我,我看你,一时间无人说话。

%5
吞江蟾乃五转蛊虫,威力宏大,张口一吐,就是大江横流。若此事处理不好,惹怒了它,恐怕大半个青茅山都要被水淹没,整个山寨都要被冲垮。

%6
沉默良久,古月赤练开口道:“事情很严重,必须要尽快解决。一旦消息被走漏出去,说不定会有居心叵测的歹人,偷偷前来,故意招惹这吞江蟾,陷害我古月一族。”

%7
“赤练家老说得很对。”古月漠尘点点头,他虽然是古月赤练的政敌,但是值此村子生死存亡的关键时刻,他彻底放下了往日的成见。

%8
顿了一顿,他继续又道:“还有一个更严重的情况。一旦吞江蟾水淹青茅山,狼巢就要淹没。为了逃生,狼群自然要往山上迁移。到那时,狼潮就要提前爆发。我们就得和无数的野兽争夺山顶的生存空间。”

%9
众家老听了这话。俱都脸色一白。

%10
古月博以沉重的语气补充道:“大家不要忘了,我们寨子的根基。当初一代先祖之所以在这里立下山寨,就是因为我们脚下的这道灵泉。一旦水淹青茅山,这道灵泉恐怕也要毁了。”

%11
“这可该如何是好啊?”

%12
“唉……即便抵挡住兽潮,在山顶生存下来。洪水退去之后,灵泉消失,大量兽群的灭亡,周围一片荒芜。修行的资源将严重不足啊。”

%13
“要死一块死,不如向熊家寨、白家寨求援?大家都是拴在一根线上的三只蚂蚱,我不信他们不出力!”

%14
家老们交头接耳,隐现慌乱。有的人,已经开始想着要求援兵。

%15
“现在求援,还为时过早了。”古月博摇摇头,第一时间否定了这个念头。“现在还不是最艰难的时刻。当年一代先祖,刚刚立下山寨时。有一只五转的血河蟒袭击山寨。被一代族长斩杀。相比较血河蟒,吞江蟾要可爱许多了。”

%16
“它脾气温和,对凡人都秋毫无犯。只有感受到其他蛊虫的气息,才会警惕。受到重创之后,才会发怒发狂,喷吐水流。我在偶然间,曾经听上代族长说过。有关于吞江蟾的传闻……”

%17
大厅中,古月博侃侃而谈。声音徐徐。

%18
众家老专注地听着,脸上紧张慌乱的神情。不由地舒缓下来。

%19
“真不愧是族长啊。一席话,就稳定了军心。”古月药姬察觉到氛围的变化,深深地看了眼古月博,心中赞叹一声。

%20
“若按照族长大人刚刚所讲,那么驱赶这吞江蟾也不是很难的一件事情。”一位家老开口道。

%21
“也不能这么说。”古月博摇摇头,“这些都只是传闻,没有亲眼见过,更没有实践过。事关重大,马虎不得。我想,还是暂且派遣一组蛊师,先去试一试吧。”

%22
众家老无不点头。

%23
古月赤练道:“要做这事,非我赤脉的一人不可。他若是不行,恐怕我族中就没人可行了。”

%24
其他人都知道赤练说的是谁,纷纷赞同。

%25
族长古月博微微笑道:“既然如此,那就命赤山小组走一遭吧。”

%26
……

%27
时值初秋,天气渐渐地凉爽起来。

%28
酒肆中靠着窗户的位置上,方源独自一人坐着,静静地品着酒。

%29
酒肆的掌柜,则站在他的身边,卑躬屈膝。

%30
“掌柜的,前几天我让打听有关苦贝酒的事情,你有了眉目没有?”方源问道。

%31
方源合炼酒虫,就差一份苦酒。

%32
然而苦酒难寻,之前又因为赤铁舍利蛊的关系,导致他受人瞩目,一走到哪里,就被人指指点点。因此也不好打听苦酒的事情。

%33
也就是这些天,风波才渐渐平息。也许是否极泰来,方源在无意中打听到苦酒的一丝线索。

%34
掌柜老者连忙答道:“禀告公子,您要我打听的苦贝酒,有人在白家寨喝到过。这种酒的原料,是深潭中的一种贝壳。这种贝壳,浑身黝黑,壳上有一圈圈的白色纹路,仿佛树木年轮。我们叫它苦贝。寻常的贝壳,能酝酿出珍珠。它吞吃水中的沙石,却只能将沙石溶解,化为苦水。有人撬开它的贝壳,得到这种苦水,用来酿酒。酿造出的苦贝酒,口感又苦又香,十分独特。”

%35
方源闻言,微微扬起眉头:“这么说,白家寨里就有这种苦贝酒了?”

%36
掌柜的连忙弯下腰:“小人也不敢担保,只是偶尔间听人谈到过。不过真要说起来,白家寨真正有名的,还是白粮液。这酒和我族的青竹酒,熊家寨的熊胆酒,并称为青矛三酒。苦贝酒……下人觉得,恐怕就算是白家寨即便有,也没有多少罢。”

%37
“没有多少,也得寻找。”方源心道。

%38
可是这事情麻烦,白家寨这些年来有渐渐崛起之迹象,开始渐渐动摇古月山寨传统霸主的位置。

%39
方源要擅自进入白家寨,恐怕还没有见到山寨的大门,就被警戒巡逻的白家蛊师打杀了。

%40
但即便如此,方源也想要尝试一番。毕竟这苦贝酒,比十万里之遥的绿艾酒,要靠谱多了。

%41
从沉思中回过神来。方源却发现掌柜老者仍旧站在自己的身边。他便挥挥手道:“好了,你下去吧,这里没有你什么事情了。”

%42
老者却没有走,脸上流露出犹豫的神情,欲言又止。

%43
最终他鼓起勇气道:“公子,您能不能把这酒肆再盘回来呀。小的和全部伙计,都想在您底下干活呢。您不知道,老东家一回来。就克扣了我们一大半的酬劳,仅靠每月那么点的元石,小的们都很难养家糊口啊。”

%44
方源摇摇头,面无表情:“这家酒肆我已经卖给了他,按照约定,是盘不回来的。再者,我也不想经营这酒肆的生意。你下去吧。”

%45
“可是。少东家……”老者犹自驻足。

%46
方源不悦地皱起眉头:“记住,我已经不是你们的少东家了!”

%47
他先前为这些人涨了薪酬。不过是想调动出他们工作的热情。为自己所用罢了。这些人却以为他好说话,就得寸进尺。

%48
自己现在靠着贩卖生机叶,只能做到自给自足。又一直为苦酒烦心不已,凭什么要为这些人盘回酒肆?

%49
“可是少东家,我们真的是活不下去了呀!您大慈大悲,可怜可怜我们吧。”掌柜的扑通一声,跪倒在地上。苦苦哀求。

%50
这番响动,顿时引来了周围酒客的关注。

%51
方源哈的冷笑一声。随手拿起桌上的酒坛,砸在掌柜的头上。

%52
夸嚓一声。

%53
顿时。坛身破碎,酒水四溅,老汉头破血流。

%54
“真以为我不敢杀你?没眼色的东西,滚。”方源眼中冷芒四射。

%55
掌柜老者被这杀气一激,霎时浑身一颤,猛地惊醒过来,慌忙退下。

%56
不论哪个世界上,总有一群弱者,乞讨强者的施舍,死皮赖脸又不知分寸。好像帮助他们才是强者的风范,不帮助他们就是不对的事情。

%57
弱小者就该有弱小者的样子,要么认命,卑贱如奴,要么就奋发,低调地努力。

%58
强者对弱者的帮助,只是心情好时的施舍罢了。

%59
弱者自己不努力,死皮赖脸地向强者乞讨,还一定要有个结果,纠缠不清。那么受到拒绝也是活该。

%60
甘于弱小,而不自发努力,只想向强者乞讨的人,根本就不值得同情。

%61
“掌柜的……”

%62
“快给掌柜的包扎伤口。”

%63
伙计们围着满脸污血的老者,一阵忙乱。

%64
掌柜老汉不过是一个凡人,就算是当场杀了,也不要紧。

%65
此事如此收场,周围的酒客们顿感无趣,纷纷收回视线,继续他们的谈话。

%66
“你知道吗?最近出了一件大事!”

%67
“你是指那只吞江蟾吗,这事情现在谁不知道?”

%68
“这可是五转蛊虫,要是处理不当,恐怕就要有灭寨的危机了!”

%69
“据说这吞江蟾,以水为食。饿了的时候,就张开大嘴,直接吞吸一条江河!”

%70
“它要是发怒,能水漫青茅山,威能恐怖至极,我们恐怕都得死!”

%71
“那可怎么办啊?”

%72
“唉,这事情就看家族高层怎么处理吧。反正我们也逃不了,能逃哪里去?”

%73
……

%74
酒肆中,弥漫着一种慌张迷茫的气氛。

%75
“吞江蟾么……”方源听着,心中存着一股笑意。

%76
恐惧是会传染的,并且越传越恐惧。

%77
其实吞江蟾性情十分温和,并不可怕。它生性嗜睡,常常有人看到江河中随波逐流的吞江蟾。它们往往将白色的肚皮朝上,仰躺在水面上呼呼大睡。

%78
有时候睡醒了,就吞吸江水果腹。饱了之后,就继续睡觉。

%79
它们对战斗和杀戮不敢兴趣,要是碰到敌人,第一反应就是逃避。除非是身陷绝境,实在躲避不了,它们这才会悍然反击。

%80
它们战力雄浑,嘴巴一张就能喷涌出一条滔滔长河,河水席卷大地,顷刻之间,就能营造出一片泽国。

%81
“这只吞江蟾,应该是睡着了。随着黄龙江水,无意中进入了支流,然后被水浪带到了青茅山脚下。”方源猜中了事实。(未完待续。如果您喜欢这部作品,欢迎您来投推荐票、月票,您的支持,就是我最大的动力。)

\end{this_body}

