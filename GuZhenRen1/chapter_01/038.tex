\newsection{魔头在光明中行走}    %第三十八节:魔头在光明中行走

\begin{this_body}

%1
天空中,阴云层层,春雨缠绵地下着。

%2
细如牛毛的细雨,洒落下来,将青茅山都笼罩上一层薄烟。

%3
客栈中的第一层饭堂,仍旧是冷冷清清,只有四桌客人。

%4
方源坐的位置正靠着窗户边,一阵风吹来,夹杂着诗意和花香。

%5
“天街小雨润如酥,草色遥看近却无。”方源遥望窗口,轻轻低吟了一句,这才收回视线。

%6
他面前的桌上摆着好酒好菜,色香味俱全。尤其是青竹酒,酒香香醇又透着一股清新。碧绿色的酒液静静地盛在竹杯中,从这个角度看去,散着琥珀般的温润光泽。

%7
最靠近他的一桌上,是祖孙二人。穿着朴素的衣服,都是凡人。

%8
老爷子一边嘬着米酒,一边羡慕看着方源这边。显然是被青竹酒香勾动,却没有钱财来买。

%9
孙子则吃着卤汁香豆,咬在嘴里咔蹦卡蹦的作响。同时又纠缠着他的爷爷,摇晃着老人的膀子:“爷爷,爷爷,给我讲讲人祖的故事吧。你要是不讲,我就回去告诉奶奶,说你出来偷酒喝!”

%10
“唉,吃个酒都不安稳。”老爷子叹了一口气,脸上却浮现出慈爱之色。他用枯瘦如柴的手摸摸孙子的脑袋,“那我就讲讲吧。话说那人祖把心交给了希望蛊,脱离了困境的围捕……”

%11
人祖的故事,是这个世界流传最广,也是最古老的传说。

%12
老人说的故事,大概是这样子的。

%13
话说人祖因为希望,而摆脱了困境。但是他终究老态龙钟了,没有力量和智慧,不能再继续狩猎,甚至牙齿都掉落光了,很多野果野菜都咀嚼不动了。

%14
人祖已经感觉到死亡的渐渐逼近。

%15
这个时候,希望蛊告诉他:“人啊,你不能死啊。你死了,心就没了,我希望就将失去了栖息之所。”

%16
人祖很无奈:“谁想死呢,但是天地要我死,我不得不死啊。”

%17
希望蛊就说:“凡事都有希望。只要你抓住寿蛊,你就能增添新的寿命。”

%18
人祖早就听说过寿蛊的存在,但是他很无奈地摊手:“寿蛊静止时,谁都察觉不了它。当它飞行时,比光还快。我怎么可能捕捉到它呢。这太难了!”

%19
希望蛊便告诉人祖一个秘密:“人啊,凡事都不要放弃希望。我告诉你,就在大地的西北角上,有一座大山。山顶有一个洞,洞中生活着一圆一方两只蛊虫。你只要能降服这两只蛊虫,天底下没有什么蛊虫不能捕捉的,哪怕寿蛊也不例外!”

%20
人祖已经走投无路了,这是他仅剩下的唯一的希望。

%21
他耗尽千辛万苦,找到了这座大山。又冒着万般惊险,攀上了山顶。在山顶的洞口处,他仅剩下最后一点点的力气,步履蹒跚地挪了进去。

%22
山洞中一片黑暗,伸手不见五指,人祖在黑暗中走啊走。时而磕磕碰碰,不知道撞到了什么东西,磕得头破血流。时而又觉得这黑暗广大无边,仿佛是另一个世界,除了他自己,周围空无一物。

%23
他耗费了许多时间,都走不出这茫茫的黑暗。更谈不上降服那一圆一方两只蛊虫了。

%24
就在他陷入迷茫的时候,有两个声音从黑暗中传来。

%25
一个说:“人啊,你也想来捕捉我们吗?你回去吧,就算是你有力量蛊在身,都没有可能呢。”

%26
另一个声音则说:“人啊,你退去吧,我们不取你的性命。就算你有智慧蛊帮助,也未必能找得到我们。”

%27
人祖无力地倒在地上,气喘吁吁:“力量蛊和智慧蛊早就都离我远去了,我已经寿元无多,走投无路了。不过只要我心中还有一丝希望,就不会放弃!”

%28
听到人祖的话,那两个声音沉默了下来。

%29
过了好一会儿,才说道:“我明白了,人,你已经将心交给了希望蛊。你是说什么都不会放弃的。”

%30
另一个声音接道:“那既然如此,我们就给你一个机会。只要你说出我们俩的名字,我们就为你所用。”

%31
人祖愣了愣,要从词海中准确地叫出两个准确的名字来,这根本就是大海捞针。

%32
而且,他连这两只蛊的名字中,究竟有多少字,都不知道。

%33
人祖连忙问希望蛊,但是希望蛊也不知道。

%34
人祖已经没有其他的出路,只好硬着头皮说名字。他说了许许多多的名字,耗费了很长很长的时间,但是黑暗中没有丝毫的回应,显然都叫错了。

%35
渐渐地,人祖的气息越来越弱,在这个山洞中他从老年,迈向了暮年。就仿佛是傍晚的落日,渐渐降下,已经有一半落下了天边,成了夕阳。

%36
他随身带来的食物越来越少,脑筋转得越来越慢,说话的力气也不多了。

%37
黑暗中的声音劝道:“人啊,你快死了,我们放你走吧。趁着你最后的时间,可以爬到山洞外,看看这个世界最后一眼。但是你冒犯我们,作为惩罚,就把希望蛊留在这里给我们做伴吧。”

%38
人祖紧捂住心口,断然回绝:“我就算是死,也不会放弃希望!”

%39
希望蛊很感动,奋尽全力回应人祖,发出了洁白的光辉。人祖的心口处,就冒出了点点白光。

%40
但这白光太弱了,根本不能照亮黑暗,甚至连人祖的全身都照顾不到,只能照亮胸膛这点地方。

%41
人祖却感到一股无形的崭新的力气,从希望蛊中涌入自己的身体当中。

%42
他继续开口,说出名字。他已经老糊涂了,有很多名字先前都说过,但他记不清楚,又说一遍,白白浪费了很多功夫。

%43
生命随着时光在流逝,人祖的寿命所剩无多了。

%44
终于在他只剩下一天的寿命时,他说出了一个“矩”字。

%45
黑暗中传来一声叹息,一个声音道:“人啊,我矩佩服你的坚持。你说出了我的名字,从今天开始,我就听你的命令。但是我只有和我的兄弟在一起,才能为你捕捉全天下的蛊虫。否则单靠我一个蛊的能力,是不行的。所以你放弃吧,你已经濒死了,还不如利用这个世界,最后看一眼这个世界。”

%46
人祖却坚定地摇摇头,他抓紧一切时间,继续说话,猜另一只蛊虫的名字。

%47
时间一分一秒的流逝,他只剩下最后一个小时的时间。

%48
但就在这时,他无意中说出了“规”这个字。

%49
霎时间,黑暗消失了。

%50
两只蛊出现在他的面前。正如希望蛊所讲,一只蛊是方的,叫做“矩”。一只蛊是圆的,叫做“规”。合起来,就是“规矩”。

%51
两只蛊一齐开口:“不管是谁,只要是知道了我们的名字,我们就听命于他。人啊,你既然已经知道了我们的名字,我们就为你所用了。但是你要记住,最好不要让其他生灵,知道我们的名字。知道我们名字的越多,我们就得同时降服他们。现在,你是第一个降服我们的人,说出你的要求吧。”

%52
人祖大喜:“那我就命令你们,先给我捕捉一只寿蛊吧。”

%53
规矩二蛊合力,捕捉来了一只八十年的寿蛊。

%54
人祖已经是一百岁了,吃了这只寿蛊后,顿时脸上的皱纹全部消除,枯瘦的四肢又填充上了健美的肌肉,青春的气息又重新散发出来。

%55
他一个鲤鱼打挺,就跳了起来。

%56
他欣喜地看着自己的身体,知道自己重新成了二十岁的青年!

%57
……

%58
“今天就讲到这里,回家喽,乖孙子。”老人说完这个故事,酒也喝完了。

%59
“爷爷,再继续讲嘛,后来人祖怎么样了?”孙子却不依,摇晃着老人的胳膊。

%60
“走吧,有时间再讲。”老人戴上斗笠和蓑衣,又给孙子披上一件小号的。

%61
祖孙俩出了客栈,步入了雨帘,渐行渐远。

%62
“规矩……”方源目光幽幽,转着酒杯,凝望着杯中的青竹酒液,心中颇有感触。

%63
人祖的传说,在这个世界上广为流传,几乎没有不知道的人。方源自然也早就听说过。

%64
只是不管是传说,还是故事,都是仁者见仁智者见智。刚刚那祖孙二人仅仅只是把它当做故事,但是方源却能品味出其中蕴含的深意。

%65
就像那人祖。

%66
当他不知道规矩的时候,就在黑暗中摸索。有时候跌跌撞撞,磕磕碰碰,弄得头破血流,狼狈不堪。有的时候,又一片广袤,陷入迷茫,毫无方向和目标。

%67
这种黑暗不是单纯的黑或者暗,力量、智慧和希望,都克制不了它。

%68
只有当人祖知道了“规矩”,道破了规矩的名字,黑暗这才豁然消失,人祖迎来光明。

%69
黑暗是规矩的黑暗,而这光明也是规矩的光明。

%70
方源将视线从酒杯挪移出来,看向窗外。

%71
只见窗外远处,天空阴沉昏暗,青山苍翠连绵,细雨纷飞如雾。在近处,高脚竹楼一个接着一个,延展出去。路上两三个行人,脚上沾着雨水泥泞,不是穿着灰绿蓑衣,就是打着黄油布伞。

%72
方源心中总结着:“这天地就像个大棋盘,万物生灵都是棋子,按照各自的规矩在运行。四季时节有规矩,春夏秋冬依序循环。水流有规矩,往低处流。热气有规矩,往高处升。人自然也有规矩。”

%73
“每个人都有自己的立场,欲望和原则。就拿古月山寨来讲,奴仆命贱,主子命贵,这就是规矩。因此沈翠想攀附权贵,费尽心机也要脱离奴籍。高碗千方百计地要讨主子的欢心,狗仗人势。”

%74
“对于舅父舅母,就是秉性贪婪,想吞并双亲遗产。学堂家老则主要是培养蛊师,维护学堂中自身之地位。”

%75
“每个人都有每个人的规矩,每一行业都有行业的规矩,每个社会群体都有各自的规矩。只要洞彻这其中的规矩,就能旁观者清。去黑暗而得光明,在此中周旋,而游刃有余。”

%76
方源又联想自身境况,心中早已运筹帷幄:“对漠家掌权人古月漠尘来讲,就是要维护本脉的繁荣和利益。漠颜找我麻烦,已经是坏了规矩,为了维护家族名誉,他必定不会对我动手。甚至还可能主动补偿我。”

%77
“其实漠家势力雄厚,凭着名誉受损,一心想要惩罚我,我也无力抵抗。但是古月漠尘心中却是怕的。他不是怕自己坏了规矩,而是担心其他人跟着他也开始坏规矩。小辈争斗,若是长辈插手,事态就会扩大,若是波及高层,对整个山寨就是巨大威胁。古月漠尘担忧害怕就在于,若是今后争斗,有其他人对自己的孙子古月漠北下手。他这一脉,就这一个男丁,若是死了那该怎么办?这种害怕,也许连他自己都不知道吧。他只是下意识地要维护规矩。”

%78
方源双目一片清澈,对此事从始到终都洞若观火。

%79
高碗不姓古月,就是外人,就是奴仆。

%80
主子处死一个奴仆,有什么要紧的。在这个世界,正常的很。

%81
方源杀死高碗。高碗的死不是关键,关键在于他的主子,他背后的漠家。

%82
“不过,我此番送了一盒子的碎尸,估计古月漠尘会解读成妥协和威胁之意罢。这也正是我想要他这么理解的。不出意外,漠家不会追究高碗的死了。当然,若是我资质再高一层,是个乙等,漠家就会感觉到威胁,凭着名誉受损,也要打压我这个未来的威胁。”方源心中冷笑。

%83
强大可以凭借,弱小也可以利用。

%84
方源此时虽然身在局中,也是个棋子。但是心中精通规矩,已经有了棋手的心。

%85
寻常人物,大多也只是类似古月漠尘或者学堂家老,也只是知道自己的规矩,隔行如隔山。像方源这样,洞察大局,精通规矩何其难也!

%86
要知道规矩,非得如人祖一样,在黑暗中摸爬滚打,迷茫徘徊一番。

%87
这个时候,力量、智慧和希望,都不太管用。非得消耗时间,切身体悟,才能取得经验。

%88
人祖能道破规矩二蛊的名字,是耗费了时间,在死亡的气息笼罩下,尝试了无数次。

%89
方源能精通规矩,也是他前世五百年累积的体验。

%90
重生以来,他自信能创出给辉煌的未来。不是因为春秋蝉的重现,不是因为他知晓了度许多秘境宝藏,也不是掌握了未来的大致走向。

%91
而是前世五百年来,为人处世的人生经验啊。

%92
如同人祖掌握了规矩二蛊,就能轻易捕捉天下万虫!

%93
而方源精通规和矩,就能居高临下,目光洞穿繁芜世事。或抽丝剥茧,或一针见血。我自居高笑傲,冷看世间之人如颗颗棋子,遵循着各自的规矩,按部就班地行进。

%94
黑暗是规矩的黑暗,光明是规矩的光明。

%95
而重生的魔头已在光明中行走。

%96
(ps:四千字大章哦,求推荐票猛砸!)

\end{this_body}

