\newsection{下马威、刁难和打压}    %第八十八节:下马威、刁难和打压

\begin{this_body}

%1
雪地中,五人小组在奔跑。

%2
古月角三看了一下天空,便道:“天色不早了,此行任务要采集腐泥冻土,虽然简单,但是耗时长久。我们必须加快速度,大家都跟紧我,尽量不要掉队。方源,你若是坚持不住,就和我们说一声,没有关系的。你是新人,我们会照顾你的。”

%3
古月角三的笑容十分和善。

%4
方源点点头,没有说话。

%5
其他三个组员却是相互交换了一下眼神,其实天色尚早,角三此举毫无必要。实际上,就是想给方源一个下马威。

%6
他们三人心知肚明,但都没有揭破这事情。

%7
事实上,这种下马威普遍存在。新人加入小组,老组员们都会多多少少地施展一些下马威,打掉新人的气焰,方便今后的命令和管理。

%8
“走。”角三轻喝一声,脚步急踏,率先冲出。

%9
方源目光一闪,和其他三人同样加快速度,紧随其后。

%10
竹芒鞋踩踏在积雪上,印下一连串深深的脚印。

%11
山地多坎坷,本来就不好走。尤其是覆盖了一层积雪之后,更容易滑倒。同时雪下难测,谁知道这积雪下面,是锐利的石子,还是凹坑?

%12
若是踩中了猎户设下的陷阱,那就更加倒霉了。

%13
在这世界中生活艰辛,赶路看似轻巧,实际上在这方面大有学问。许多新人就因此栽了不少跟头。

%14
唯有经过长期锻炼,吃过不少苦头之后,积累出经验的蛊师,才能有效地避开这些障碍。

%15
冰冷的冬风扑面而来,方源在雪地中疾驰着。

%16
时而小跃,时而长跑,时而攀援,时而挪步,紧紧地跟在角三的身后。

%17
整个青茅山都裹了一层雪衣,许多树木的枝干都是光秃秃的,没有一片叶子。

%18
不时的有松鼠,或者野鹿,被这群人惊扰,急速地窜到远处。

%19
半个小时之后,角三忽然停下了脚步,已经到达了目的地。

%20
他回过头,看向方源,嘴角带笑,交口称赞道:“不错!方源你不愧是本届状元,居然一直跟在我的身后没有掉队。”

%21
方源笑了笑,没有说话。这种下马威,他早已熟知。事实上,雪地奔行已经演变成了一个“传统节目”。很多小组都是利用这个,打压住新人的气性。

%22
二人站在原地等了一会儿,其余的那三位组员才堪堪赶到。

%23
呼,呼,呼……

%24
他们大喘着粗气,额头满是汗渍,都是脸皮涨红,两位弯腰支手,剩下一位直接瘫坐到了雪地上。

%25
角三狠狠地瞪了这三人一眼,低喝道:“都给我站直了!你们丢脸不丢脸?看看人家方源,再看看你们自己。哼,此次任务完成之后,都回去好好反省反省。”

%26
三人连忙挺身直立,头却低着,被角三训得不敢抬头,也不敢反驳什么。

%27
只是他们看向方源的目光,都发生了一些变化。

%28
“真是怪了,这个方源怎么如此老道?至始至终都没看到他摔过一跤!”

%29
“唉,我们的力气都只是寻常,怎么能和这个怪胎相比?”

%30
“哼,好戏没看成,反而害得我们成了牺牲品。这个家伙……”

%31
“好了,都打起精神来。”角三手指向前方,“这个小型山谷就是我们的目的地。里面有大量的腐泥冻土可以采集。我们就此分开,采集冻土,一个小时之后在这里集合。空井,现在分发工具。”

%32
角三话音刚落,叫做古月空井的男组员便站了出来。

%33
他手掌平伸开来,一道黄色的光便从他腹部空窍射出,倏地悬停在他的手掌中央。

%34
黄光消散,露出真容,是一只金背蛙。

%35
这只金蛙肥嘟嘟的,雪白的肚皮极其的大,鼓起来,使得它的整个身躯看上去仿佛是个圆球。蛤蟆的嘴巴,眼睛,都被这肚皮顶到了头顶,压缩到了一块去。

%36
方源目光一闪,瞬间认出这只蛊虫。

%37
这是二转蛊虫——大肚蛙。

%38
随后,空井的手上涌出丝丝的赤铁真元,被大肚蛙吸收。

%39
呱。

%40
大肚蛙响亮地叫了一声,张口一吐,吐出一个微型小铁锹。

%41
铁锹飞在空中,很快变大,眨眼功夫,落在地面上,成了一个半人高的大铁锹。

%42
呱呱呱……

%43
它连续叫了几声,每次都吐出一件工具。

%44
最终,众人面前的雪地上,堆了五只铁锹,五个木盒。木盒上都穿着两根麻绳背带。

%45
蛊师喂养蛊虫,负担很大,因此蛊虫数量有限。蛊师在前期,很难独自面对复杂的环境,以及层出不穷的各种麻烦,因此常常以小组形式行动。

%46
在一个小组中,有人侦察,有人进攻,有人防守,有人治疗,还有一人就是后勤。

%47
这个站出来的男蛊师空井,显然就是后勤蛊师。他掌握的这只大肚蛙,就是典型的后勤蛊虫,肚子里另有宽敞空间,具有存储的功用。

%48
当然,每只蛊虫都有长处和短处。

%49
大肚蛙的缺陷,除了存储空间有限之外,每次吐出事物,都要叫一声,这点实在讨厌。尤其是当蛊师在战场上潜伏时,处理不好,就会因此暴露自己的位置。

%50
还有就是,大肚蛙不能存储蛊虫,同时不免疫毒素,有毒的物品不能存储。

%51
分发了工具之后,小组五人每人手中都有了一把铁锹,一个木盒。

%52
“出发。”角三挥了挥手,率先步入山谷。

%53
方源提着铁锹,背着木盒,选择了另一个方向。

%54
“到底是新人,兴冲冲地去了呢。呵呵。”

%55
“腐泥冻土岂是这么好采集的?不仔细分辨,很有可能就采集到普通的冻土,凭白做了无用功。”

%56
“事实上,分辨也很困难。腐泥冻土的颜色,和寻常冻土的差距不大。尤其是在积雪的覆盖下,新人往往只能靠运气挖掘出来。”

%57
三位组员看着方源离去的背影,都在暗暗冷笑。

%58
然而一个小时之后,当他们看到方源满载而归,盛满了整整一盒子的腐泥冻土时,都傻眼了。

%59
包括角三在内,他们的木盒当中,最多也不过半盒的腐泥冻土。

%60
看到方源的木盒,他们甚至都羞于展示自己的成果。

%61
“全是腐泥冻土!”一位组员仔细查看了一下,更加吃惊。

%62
“方源,你怎么会采集到这么多的冻土?”一位女组员耐不住心中的疑惑和好奇。

%63
方源微微挑起眉头,雪光映照着他的眸子,显得清冽而又透彻。

%64
他淡淡地笑着:“学堂家老讲过的。腐泥冻土,是冰雪冻结了泥沼之后,才会产生的一种常见资源。它在黑中透出一丝紫,气味其实很臭,只是被冰雪冻住,暂时闻不到。它是臭屁肥虫的食物,同时它十分肥沃,常常混杂在泥土中,用来栽培麦苗、蔬果。家族方面发布这个任务,应该是用在地下溶洞,给月兰花施肥吧。”

%65
一席话将四人说得楞在原地。

%66
“这些理论,学堂上自然会传授。但是理论和实际之间要联系起来,也有相当大的难度。这个方源难道以前亲手采集过腐泥冻土?”三位组员不禁面面相觑。

%67
古月角三的目光闪了闪,笑道:“方源,做的不错。”

%68
他赞美着,只是脸上一直以来的温和笑容,此时也有些僵硬了。

%69
角三转过头,继续对众人道:“这样一来,我们的任务应该是完成了。大家将铁锹和木盒都交给空井,我们回去吧。”

%70
回到山寨,已经是下午。

%71
五人从内务堂出来,角三将任务所得的六块元石都分发出去。他自己得两块,其余人各得一块。

%72
元石得的如此容易,一些组员的嘴角都带着笑意。

%73
方源默默地收起元石,不动声色。

%74
只在心中思索:“新人加入小组,家族都会相应地加大任务报酬,算是扶持新人的补贴。采集腐泥冻土的任务报酬,最多只有两块元石。因为我的缘故,多了两倍。按照道理,应该多多照顾我才是。如果雪地奔行算是下马威,故意单独采集冻土是刁难,那么这样子分发元石,就是打压了。”

%75
一两块元石罢了,方源还并不放在心上。他只是奇怪,自己和古月角三素未平生,他为什么打压自己?

%76
“难道说……”一道灵光在方源的心头闪过。

%77
(ps:靠……谁能告诉我为什么,明明已经定时发布的章节。上一次也是这样,定时发布没有成功。)

\end{this_body}

