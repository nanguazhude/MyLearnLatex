\newsection{四味酒虫}    %第一百二十一节:四味酒虫

\begin{this_body}

“古月一族……”熊力站在山坡上,遥望着远处的古月山寨,脸上浮现出一丝复杂之色。

秋风飒爽,徐徐吹来。

此时望去,秋意染遍山峦。

树叶红黄交杂,野果累累。唯有青矛竹,碧绿如玉,依旧挺立。

“曾几何时,古月一族就像是这山上的青矛竹,四季常青,第一霸主。现在嘛,竟有一种落魄。”熊力的嘴角扯动出一丝冷讽的弧度。

但是很快,他又想到自家的山寨,嘴角的弧线抹平,他的心情沉重起来。

白家寨的崛起,已经打破了青茅山旧的平衡。传统霸主古月一族的暗弱,熊家寨的经营不佳,都让青茅山的格局,趋于一种动荡当中。

熊力知道,这个问题之所以没有彻底爆发出来,都是因为狼潮在上面压着。三家山寨必须通力合作,才能渡过这次狼潮,所以都有默契地选择合作,暂时抛去过往恩怨。

“狼潮一过,青茅山的陈旧格局,想必就要被打破了吧。白凝冰,这才几年,已经是三转修为,真是恐怖啊……”熊力的心中浮现出一个白衣少年的身影,心中更是像压了巨石一般,沉重压抑。

他熊力是熊家寨二转蛊师第一人,生平大小战数十次,胜多败少,立下赫赫之威。身怀熊豪蛊,爆发出来,有一熊之力,号称青茅山第一大力士。

他早已出道,算是亲眼目睹着白凝冰火箭般的崛起,更明白此人的恐怖。

“组长,那个就是古月山寨啊!还那么远呢,为什么我们要停在这里呢?”一旁,熊林双手搭在一起。抱着后脑勺。好奇地道。

在这一行五人的小组当中,熊林最为年轻,是刚刚出道的新人。和方源同岁,是熊家寨此届的第一天才。

他身材矮小,剃了一个光头。在阳光下显得白亮亮的。

熊力目光扫了一下这个家族中的后起之秀,沉重的心情稍微舒缓了一丝。他沉声答道:“我们此行是在执行出使任务,不是侦察任务。这里已经是古月一族的精戒线,我们如果冒然进去,恐怕会当做敌人来处理。”

“哦,原来是这样。”熊林恍然。

“我们这次来,有两个目的。一个是将族长大人的亲笔信,交给古月族长。另一个,是调查吞江蟾事件。古月山寨。毕竟不是我们的地方,待会到了那里,都把你们的臭脾气收敛起来。但是也绝不能堕了我们熊家寨的威风。听明白了吗?”熊力目光扫视身边四人。轻轻一喝。

其余蛊师无不神情一凛,默默点头。

“组长。有人来了。”组中的侦察蛊师,忽然开口。

“我们暴露行迹这么长时间,也该来了。就是不知道是谁……嗯?原来是赤山。”不久,熊力也发现了赤山小组,不由地目光一闪。

“哇!那个人好高大,他就是赤山吗?比熊力组长还要高啊,这身肌肉,一块块的……组长,他就是那个天生巨力,一直想要抢夺青茅山第一大力士称号的那个人吗?”熊林顿时看直了眼。

“哼,就凭他……”yin测测的熊姜不屑地撇撇嘴。

“熊力!”

“赤山。”

两支小组距离缩短到五十步,两位组长照面,眼神锐利似在半空中喷撞出了火花。

“看来这一次,你是熊家寨的特使。”赤山冷哼一声,他没少和熊力交手。

“正是如此。白家寨的特使来了么?”熊力面色如铁。

“问那么多干什么,随我来吧。”赤山带着戒备,微微侧身,邀请道。

……

与此同时。

第二密室中,四个酒坛摆在方源的面前。

酸甜苦辣四味美酒,甜的是黄金蜜酒,辣的白粮液,酸的是杨梅酒,苦的是苦贝酒。

方源盘坐在地上,心念一动,空窍中的两只酒虫便飞了出来。

合炼四味酒虫的过程,和普通的合炼稍稍有所区别。

两只酒虫在方源的意志下,一齐钻入到杨梅酒坛当中。

在杨梅酒液中,它们开始尝试着融合。白色的光团在酒坛中产生,豪光从坛口冲出来,映照在壁顶上。

方源往酒坛中投入元石,一块,十块,五十块……

一直到一百块的时候,光团凝缩成拳头大小,悬浮在酒坛当中。

此时,杨梅酒已经消耗一空,方源便拿起第二坛酒,将浓稠如油的黄金蜜酒倾倒其中。

在蜜酒的浸泡下,白色的光团忽然涨大成原状。

方源头上浸出汗渍,他要一直维持着两只酒虫的意识融合,因此一心多用,极耗心力。

他继续往酒坛中投去元石。

每投一块元石,白色光团就会缩小凝聚一分,直到再次凝聚成拳头大小,达到极限。

方源如法炮制,再依次倒进苦贝酒、白粮液。

当四坛美酒通通消耗之后,酒坛中白光骤然一盛,旋即消散至无。

“成了。”方源不用往酒坛中看,就知道已经成功了。

他心念一动,酒坛中就晃晃悠悠地飞出一只蛊虫。

正是四味酒虫。

和酒虫相比较起来,它的外型并没有太多的变化,只是比酒虫稍大一些。

同样酷似蚕宝宝,一对黑亮的小眼睛。

只是酒虫通体ru白无暇,而这只四位酒虫的身上却是四种颜色不断地渐变,代表辣的红色,代表苦的蓝色,代表酸的绿色,代表甜的黄色,让方源不禁联想到地球上的霓虹灯。

“呼……”方源长出一口气,这次运气不错,没有失败,第一次就成功了。

怕就怕失败之后,酒虫受损严重,死掉一只。或者苦贝酒消耗光。那就麻烦了。

庆幸的是没有这种情况发生。

蛊师用蛊、养蛊、炼蛊。那一方面都不容易。合炼蛊虫方面,很多蛊师都要千辛万苦地寻找秘方,筹集材料。

秘方各有不同。未必就有适合的。有的蛊师为了筹集材料,甚至会耗费十几年的功夫。就算是找到了秘方,收集全了材料。但是合炼失败了,材料损失消耗,那么之前的努力和准备往往就打了水漂。

“蛊师修行艰难啊……”方源在心中深深地一叹。

合炼蛊虫,在修行前期,还算是容易的。到了四转、五转,往往十次未必能成功一次。

六转的成功率更是降到百分之一。合炼高级蛊虫,每一次失败,都意味着损失大笔的资源。

不过,一旦成功。收益将极为可观。

就拿方源新炼成的这只四味酒虫来说,它能精炼二转真元,提升一个小境界。

方源使用了一只赤铁舍利蛊。晋升到中阶。如今用了四味酒虫。就是高阶真元。

这就意味着,他的战斗力猛地暴涨两倍。同时温养空窍。修行起来事半功倍。

不过凡事,皆有利有弊。

方源用四味酒虫精炼真元,必会导致元石消耗的增加。单靠贩卖生机叶的收入,已经不足以持平他的修行消耗了。

“接下来,还要将隐石蛊,合炼晋升为隐鳞蛊。这又是一笔开支。”

每次合练蛊师,不管成功还是失败,都会消耗一笔元石。合练四味酒虫,方源就先消耗了四百多块元石。

他这次驱赶吞江蟾之后,族里奖了他五百块元石。五百块元石足够其他蛊师花销很长一阵子,但是方源几乎就全用在了这里。

幸好他之前转卖了家产,用了大半收购了赤铁舍利蛊后,手头上还余着一笔钱财。短时间之内,倒不用过于担心。

只是,这隐鳞蛊必要合练。

方源斩杀了石猴王后,得到了这只隐石蛊。但是这只蛊缺乏实用xing。

它仅仅只能隐去本。也就是说方源用了之后,他的躯干、脑袋、头发等都会隐去形迹,让人瞧之不见。

但是他身上穿着的衣服,带着的护腕绑脚,踩着的竹芒鞋仍旧在那里,能被肉眼观察到。

石猴王当然没有这个顾虑,它是野兽,不需要服饰。

但是方源就尴尬了。要想发挥出隐石蛊的最佳效果,让人看不见他,他就必须脱去身上的所有衣衫。若是不脱,即便隐身了,别人也会发现一套在“行走”的二转蛊师的武服。

隐石蛊只是一转蛊虫,当它晋升到二转的隐鳞蛊时,这个问题就能解决了。

隐鳞蛊能将蛊师的衣服也同时隐形,如果石猴王催动的是隐鳞蛊,那么方源的上衣即便盖在它的身上,也会消失不见。

倘若石猴王身上是一只隐鳞蛊,那么方源能否战胜石猴王,还是一件颇有悬念的事情。

合炼隐鳞蛊的,除了隐石蛊之外,当然还有其他材料。不过这些材料都比较普通,方源已经拜托江牙帮忙收集了。

“若是合练成了隐鳞蛊,不仅出入石缝秘洞方便了许多,而且在狼潮中更能游刃有余,有此保命手段,进可攻退可守。”方源寻思着。

时候也不早了,他收了四味酒虫,归入空窍,就出了洞口,往山寨走去。

他成功驱赶了吞江蟾,一时间成了众人关注的焦点,这些天行动并不方便,因此有所顾忌。在这秘洞中待久了,会引起他人的怀疑。

山寨的门口,一场有关气力的较量已经结束。

熊力一组人面色傲然地站着,赤山一组以及门口守卫蛊师,都是一脸的凝重。

熊力身材没有赤山高大,此时目光中却带着居高临下的意味。他缓缓地道:“赤山,你的确天生巨力,有着天赋。但是我有棕熊本力蛊,已经养了一熊之力。刚刚比斗的结果你也看到了,你还不是我的对手。”

“哼,想要青茅山第一力士的名头,做梦去吧。”一旁,熊姜冷哼着。

赤山脸色铁青,他知道这是对方故意约战,行为举动充满了政治意图。此刻,他输了,已经不是个人的私事,而是堕了古月一族的名头。

“战胜我有什么可得意的。你们不知道,我早已经不是家族中最有力气之人。有本事的话,胜了方源再说。”赤山逼不得已,不得不动抬出了方源。

“哦,方源?我听说古月一族出了一位甲等天才,名叫方正。方源又是何人?”熊力疑惑地问道。

赤山冷哼一声便答:“方源便是方正的哥哥,亦是天赋异禀,有着天生的巨力,同时还与蛊虫增添本身力气。先前五转吞江蟾,就是被他一人之力,推出百步之远。最终赶跑了吞江蟾,你若不信,可去寨子里随便打听。”

熊力小组脸色不由一变。

五转吞江蟾!

方源!

这个名字一下子就刻在他们的心中。

(ps:这是修整大纲后的第一更,呼……不会断更的,这点大家放心。缺的更新,以后会爆发的。错过双倍月票真是可惜,但是为了长远打算,这也算不得什么了。呵,酸甜苦辣,这不过是人生的一场小雨。)

\end{this_body}

