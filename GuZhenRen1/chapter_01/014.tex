\newsection{山缝之中藏玄机}    %第十四节:山缝之中藏玄机

\begin{this_body}

酒虫体型如蚕宝宝,通体散发着珍珠一样的白光,有点胖胖的,外形很可爱。

它以酒水为食物,能够凌空飞行。飞行时,会把身躯团成一团,速度还很快。

它虽然是一转蛊虫,但是其价值,却比一些二转蛊虫还要珍贵。

若用它来作为本命蛊,可比月光蛊要好得多。

此时,这样的一只酒虫就贴在一根距离方源,仅仅只有五六十步远的青矛竹上。

方源屏住呼吸,没有冒然接近,而是慢慢地后退。

他知道这个距离虽然很近,但是真正要直接捕捉酒虫,对于自己一个刚刚开窍的蛊师来讲,是千难万难,或者说,根本就没有成功的希望。

以方源此时的目力虽还不能看清酒虫的样子,但是他却在冥冥之中,感受到酒虫对自己的警惕之意。

方源退得很慢,很轻柔,尽量不惊动酒虫。

他知道,酒虫若要飞走,以自己的速度根本就赶不上,只有等到它喝酒喝得醉醺醺,飞行速度慢下来,才有机会捕捉。

见方源退得越来越远,趴在竹竿上的酒虫,身子忍不住骚动起来。

前面强烈的酒香,诱惑着它,吸引着它,让它为之魂牵梦绕。若是它有口水,只怕此时早就滴下口水一大滩了。

但酒虫的警惕性依旧很高,方源一直退了两百步,它这才弹缩身躯,一跃到了空中。

它凌空飘行的时候,把身躯团成一个团子,就好像是白乎乎的小汤圆。

汤圆从空中划过一道圆润的弧线,落到方源滴洒青竹酒的草丛上。

美食近在眼前,酒虫戒备放下大半,它猴急地爬到蕴含酒液的花骨朵前,探进去脑袋,只留下胖乎乎的尾巴在外面。

它饿极了,青竹酒又是如此的美味,它大口吸允着,很快就沉浸在食物的美味当中,把方源遗忘到脑后。

方源这个时候,才开始小心翼翼地悄悄接近。

他看到花骨朵外酒虫的尾巴。这尾巴就像是蚕宝宝的尾巴,胖乎乎的,又圆润,散发的光晕让人联想到珍珠。

起先它的尾巴,垂在外面,一动不动。

然后过了不久,这尾巴开始一翘一翘的,显然酒虫喝得很高兴。

到最后,方源已经接近到十步距离的时候,它的尾巴左右摇晃起来,一荡一荡的,带着欢快的节奏。

完全喝嗨了!

看到此景,方源险些笑出声。

他没有继续前进,而是耐心地等待着。若是现在一扑而去,必有相当大的把握捕捉到酒虫,但是方源还想让这酒虫领路,带他去花酒行者的尸骸之处呢。

不一会儿,酒虫从花骨朵中退了出来。它身躯胖了一圈,脑袋摇摇晃晃,像是醉汉一般,竟然对方源的存在毫无察觉。

它又爬到另一棵嫩黄的野花上,栖息在花蕊中,饱餐酒露。

这次喝完之后,它终于感到饱了。身躯在花瓣上慢慢地缩成一团,然后缓缓飞起,一直上升到距离地面一米五左右的高度,这才悠悠地向竹林深处飞去。

方源连忙拔腿跟上。

酒虫已经醉醺醺的,飞行速度降低到了往常的一半。但就算这样,方源也要全力跑动,才能跟得上它的身影。

夜色如洗,少年在竹林中快速穿梭,追逐着前方不远处的一点珠雪。

月色温柔,清风徐徐。竹林里如积水空明,一棵棵翠绿的青矛竹,在方源的眼前快速闪现,又落到少年的身后去。

地上,是绿油油的草丛,点缀着野花朵朵。

还有长着青苔的小石块,未长成的嫩黄竹笋。

方源淡淡的虚影也在地面上快速穿行,越过青矛竹投在地上的一道道笔直黑线。

他紧紧地盯住那点雪影,在酒的暗香中,大口呼吸着山林间的清新空气,迈动双腿紧跟在后。

因为急速穿行,眼底下月光如水,光影频动,竟似在长满水草的水中奔驰。

酒虫飞出竹林,方源也冲出竹林。一大片白灿灿,花心微黄的花朵,在他的腿边借着劲风,飞散出纷纷花瓣。

一群龙丸蛐蛐好似一曲流动的诗篇,恰巧流淌到了前方,方源直冲而过,顿时哗的一声,眼前红霞绽放,冲散出一片赤星流萤。

一条静静流淌的山溪,铺着鹅卵石,潺潺水面,映照着夜空的春月。啪啪几声,被方源涉水踩碎成万千银色涟漪。

可惜一溪风月,踏碎了琼瑶。

方源紧追不舍,牢牢地跟在酒虫身后。

顺着山溪向上,他已经隐约听到瀑布的声响,又转过一片稀疏的树林,便看到酒虫飞入一块巨石狭缝当中。

方源眼前顿时骤亮,这才停下脚步。

“原来是在此处。”他大口喘着粗气,心脏砰砰直跳,这一停,顿时感到满身是汗,一股热气随着血液的加速流淌,激荡在全身。

环顾四周,发现这是片浅浅的河滩。

大大小小的鹅卵石满布地面,河水只高出鹅卵石一指左右。也有一块块的灰白巨石,随意地分散在这里。

青茅山的后山,是一道大瀑布。

瀑布的水流,随着气象变化而变化,它一落千丈,冲击出一块深潭。深潭旁,就是白家山寨,势力雄浑,和古月山寨只强不弱。

瀑布也有分支,显然方源面对的就是一条分支中的分支。

这处河滩平常时候,是干的。但是最近下了一场大雨,连续三天三夜,便导致这里积蓄了一股浅浅的流水。

流水的源头,就是那块酒虫钻进去的巨石。

巨石倚着垂直的山壁。一道细长的微型瀑布,从大瀑布分支过来,像是一条银色巨蟒贴着山壁垂落而下,打在巨石上,天长日久就将这巨石中央,冲刷出了一条缝隙。

此时瀑布冲刷下来,水流微微轰鸣着。像是一道洁白卷帘,把这巨石缝隙完全挡住。

借着观察打量的功夫,方源气息已经不再那么急喘了。他眼中闪过一抹坚定的光,走到巨石边,深吸一口气,埋头冲了进去。

巨石缝隙颇大,两个成年人并排走着,都没问题。更何况方源的身体,只是一个十五岁的少年。

一冲进去,急速的水流就将方源的身躯往下一压,同时冰冷的水一下子就将方源浑身上下都淋个湿透。

方源扛着水压,疾步前行,走了几十步,水压渐渐小了下去。

但是缝隙间距也随着缩小,方源只好侧着身躯走。

耳边是轰鸣的水声,头顶上是白亮一片,巨石更深处则是一团黑暗。

黑暗中隐藏着什么?

也许是一条阴腻的毒蛇,也许是剧毒的壁虎,也许是魔头花酒行者的机关陷阱,也许空无一物。

方源就这样侧身,慢慢地挤进黑暗。

头上的水流,已经没有了。石壁上长满了青苔,擦着方源的身躯皮肤,极为滑腻。

方源完全被黑暗吞没,石缝也越来越窄,渐渐地,让他的头颅都不能自由转动。

方源咬咬牙,继续前行。

走了二十多步,他发现黑暗中似乎有一团红色的光影。

起初他还以为是幻觉,但是眨眨眼,再定睛一看,他这才确认,这的确是光亮!

这个发现让他精神一振。

继续走了五六十步,红光越来越亮,在方源的视网膜中渐渐扩张成了一条长长的竖直细缝。

他伸直的左臂,突然感到前方石壁一空,弯曲了下来。

方源顿时大喜,知道这巨石果然内有空间。他疾走几步,终于挤进这条光缝。

眼前豁然开朗,一个大约有八十平方米的空间,展现在方源面前。

“我走了这么长的距离,早就过了巨石,现在应该是山壁当中了。”他一边活动手脚,舒展四肢,一边打量这处隐秘空间。

整个空间充斥着昏暗的红光,也不知道光源是从哪里来的。

四周的石壁很潮湿,长满了青苔,但是这里的空气却很干燥。

在这石壁上,还依附着一条条已经枯死的藤蔓。藤蔓相互纠结交错,将四面大半的石壁都编织起来。上面还长着一些凋零枯萎的花朵根茎。

方源看着这些残花败叶,觉得有些眼熟。

“是酒囊花蛊,和饭袋草蛊。”忽然,他灵光一现,认出了这枯死的花茎和藤蔓。

蛊的形态,有很多种。有像矿石的,比如蓝水晶模样的月光蛊。有类似虫子的,比如蚕宝宝般的酒虫。还有花草形态,就是方源眼前的酒囊花蛊,和饭袋草蛊。

这两种蛊,都是一转的天然蛊。只有灌注真元,就能生长。长成之后,花心中会分泌出花蜜酒,草袋中会生长出香喷喷的米饭。

方源顺着藤蔓的根系,移动视线,果然在墙角发现枯死的根包成一团,鼓成一块。

酒虫就栖息在这团死根之上,呼呼睡去,已经唾手可得了。

方源走过去,先把酒虫收入怀中,又蹲下去,拨开死根枯藤,便发现一具白骨骷髅被包裹在里面。

“终于找到你了,花酒行者。”看到此处,他的嘴角泛起一丝微笑。

正要伸手将枯藤全部剥去,就在这时!

“你动一下试试?”一个充满杀机的声音,陡然在方源的背后响起。

------------

\end{this_body}

