\newsection{请一边玩蛋去}    %第三节:请一边玩蛋去

\begin{this_body}

邦、邦邦,邦、邦邦。

巡游的更夫,敲着有节奏的梆子。

声音传入高脚吊楼,方源睁开干涩的眼皮,心中暗道:“是五更天了。”

昨夜躺在床上思索了很久,计划安排了一大堆,算起来只睡了一个时辰多丁点。

这个身体还没有开始修行,精力并不旺盛,因此一阵阵的疲累困乏之意,仍旧笼罩着身心。

不过五百多年的经历,早就打造了方源钢铁般深沉的毅志。这点嗜睡之意,根本就算不了什么。

当即便推开身上的薄丝被褥,干净利落地起了身。

推开窗户,春雨已经停了。

混合着泥土、树木和野花的香味的清新湿气,顿时扑面而来。方源顿感头脑一清,昏沉的睡意被驱除了干净。

此时太阳还未升起,天空蓝的深沉,似暗似亮。

放眼望去,用绿竹和树木搭建的高脚吊楼,和群山相衬着,一片幽静苍绿之色。

高脚吊楼至少有两层,是山民居住屋的特有结构。因为山上崎岖不平,因此一楼是巨大的木桩,二楼才是人的居所。

方源和弟弟方正是住在二楼。

“方源少爷,您醒了。奴家这就上楼来,伺候您洗漱。”就在此刻,楼下传来一个少女的声音。

方源低头一看,是自己的贴身丫鬟沈翠。

她姿容只能算上中等,但打扮得好,穿着一身绿衫,长袖长裤,脚下是绣花鞋,黑发上还有一个珍珠簪子,全身上下都散发出青春的活力。

她欢喜地望了一眼方源,端着一盆水,蹬蹬蹬的就上了楼。

水是调好的温水,用来洗脸。漱口则用柳条沾着雪盐,能净齿白牙。

沈翠温柔的伺候着,脸上带着笑颜,眉目含春。而后又为方源穿衣结扣,在这过程中时不时地用丰满的胸脯蹭方源的胳膊,或者后背。

方源面无表情,心如止水。

这个丫鬟不仅是舅父舅母的眼线,而且爱慕虚荣,性情薄凉。上一世曾被其蒙蔽,到了开窍大典之后,自己地位一落千丈,她顿时就翻了脸,没少给过自己白眼。

方正来的时候,正看到沈翠为方源抚平胸口衣衫上的褶皱,眼中不由地闪过一丝羡慕嫉妒的光。

这些年跟着哥哥一起生活,受方源的照顾,他也有个奴仆伺候着。不过却不是沈翠这样的年轻丫鬟,而是个体型肥肿的老妈子。

“若是哪天,沈翠能伺候我这样,该是什么滋味?”方正心中有些想,又有些不敢想。

舅母舅父偏爱方源,这是府上众所周知的事情。

本来他都没有奴仆伺候,还是方源主动为方正要求来的。

虽说有着主仆的身份区别,但是平日里方正也不敢小瞧这个沈翠。皆因沈翠的母亲,就是舅母身边的沈嬷嬷,也是整个府里的管家,深受舅母之信任,有着不小的权柄。

“好了,不用收拾了。”方源不耐地拂开沈翠的柔软小手,衣衫早就平整,沈翠更多的是在引诱。

对她来讲,自己前途光明,甲等资质的可能性极大,若是能成为方源的侧室,就能从奴转为主,可谓一步登天。

上一世方源被蒙蔽过,甚至喜欢上这个婢女。重生之后却是洞若观火,心冷似霜。

“你退下罢。”方源看也不看沈翠,整理着自己的袖口。

沈翠微微撅嘴,为方源今日的不解风情感到有些奇怪和委屈。想要说什么撒娇的话,但是被方源若有若无的莫名气质震慑着,张口几次,最终说了声“是”,乖乖地退下。

“你准备好了?”方源看向方正。

弟弟呆呆地站在门口,低下头看着自己的脚尖,轻轻地嗯了一声。

他其实四更时就醒了,紧张的睡不着,偷偷起床早早就准备好了,两个眼圈都是黑的。

方源点点头,弟弟心中的想法,在前世他并不清楚,不过今生他又怎么不明白?

但此时点破毫无意义,淡淡地吩咐着:“那就走吧。”

于是兄弟俩就走出了居所,一路上,碰到不少的同龄人,三三两两的,显然有着相同的目的地。

“你们看,那是方家两兄弟。”小耳边传来小心翼翼的议论声。

“前面走着的就是那个方源,就是那个作诗的方源。”有人强调着。

“原来是他呀,面无表情、旁若无人的样子,果真和传闻中一样拽。”有人语气酸酸,带着嫉妒和羡慕。

“哼,你要是能像他一样,你也可以这样拽!”有人冷哼着这样回答,隐藏着一种不满。

方正面无表情地听着,这样的议论声他早已习惯了。

他低着头,跟在哥哥的身后,默默走着。

此时天边已经亮起晨光,方源的影子就投在他的脸上。

朝阳在渐渐升起,但是方正却忽然觉得,自己正走向黑暗。

这个黑暗来源于他的哥哥,也许这一辈子,自己都不能挣脱哥哥笼罩自己的巨大阴影。

他感到胸口传来一阵阵的压抑,甚至是呼吸不畅,这该死的感觉让他甚至联想到“窒息”这个词!

“哼,这样的议论,果真是木秀于林风必催之。”听着耳边的议论声,方源心中冷笑着。

难怪在测出自己的丙等资质后,会四面环敌,很长一段时间都受着苛刻、白眼、冷遇。

身后弟弟方正越来越沉闷的喘息声,他也尽收耳底。

前世没有察觉到的,今生则是明察秋毫。

这都是五百年人生经历带来的敏锐洞察力。

他忽然想到舅父舅母,真是有些手段。给自己配了沈翠来贴身监控,给弟弟配的老嬷嬷。其实还有其他生活细节上的差别待遇。

这都是有意为之,就是要挑起弟弟心中的不平之气,挑拨和自己的兄弟情谊。

世人皆不患寡,而患不均。

前世自己经历太少,弟弟又太傻太天真,被舅父舅母挑拨成功。

重生以来,眼看着就要开窍大典,局面看似积重难返,但是以方源魔道巨擘的手段和智慧,也不是不可以改变。

这弟弟完全可以镇压收服,沈翠一个小小的丫头片子,更能提早收入后.宫。还有舅父舅母、族长家老,敲打他们至少有数百种方案。

“但是,我却不想这么做呀……”方源在心中悠然一叹。

就算是亲弟弟又如何,没有亲情可言,只是个外人罢了,舍了也就舍了。

就算是沈翠长得再漂亮又如何,没有爱和忠心,不过是一具肉体。收入后.宫?她还不配。

就算是舅父舅母,族长家老又如何,都是生命中的过客,何必费尽心机,耗散精力,来敲打这些路人?

呵呵。

只要不阻碍我赶路,那就一边玩自己的蛋去,踩都不屑踩。

------------

\end{this_body}

