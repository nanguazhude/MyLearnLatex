\newsection{得来全不费工夫}    %第一百二十节:得来全不费工夫

\begin{this_body}

%1
秋高气爽,风和日丽。

%2
瓦蓝的天空,如水晶般澄澈干净。

%3
风徐徐地吹着,拂动方源的发梢。

%4
远远望去,附近的村庄已经人烟稀少。近处,小山一般的吞江蟾沉睡着,卡在河床中。方源站在河岸处,和这只五转的巨型蛊虫对比起来,就仿佛是大象身边的小猴子。

%5
方源心境平和,没有一丝的紧张,他暗暗思忖:“我先后用了白豕蛊、黒豕蛊,淬炼身躯,提升力量。有了两猪之力,再加上我本身的力量,足以超过五六个成年人。但是要推醒这只吞江蟾,恐怕得有两牛之力。单凭我自身之力,当然不行。不过,只要借助浪涛的力量……来了。”

%6
哗哗哗……

%7
方源侧身望去,一阵浪涛起伏而来,平静的河水迎来了动荡。

%8
浪潮越来越大,不断地拍击着吞江蟾,激起冲天的浪花。

%9
方源只是站着一会儿,身上就被打湿了。

%10
他也不在意,开始奋起全力推动吞江蟾。

%11
吞江蟾仍旧在死睡,它本来就是生活在江河湖海中的蛊虫,浪潮的打击对它来讲,太平常不过,根本就不能唤醒它。

%12
接着一股股的水浪的冲势,方源努力了半晌,终于将吞江蟾缓缓推离原来的位置。

%13
这条河越到下游,河床就越是宽敞。再加上两侧的水流,使得方源越推越轻松。

%14
大约推了三百多米远,吞江蟾睁开了它朦胧的睡眼。

%15
一对深绿色的瞳眸。从迷茫散光的状态,渐渐地收束起来,然后盯住了身边的方源。

%16
方源毫不畏惧,与它对视。

%17
从它深幽的瞳孔中,方源能清晰地看到自己的倒影。

%18
“江昂!”吞江蟾扬起脑袋,忽的张开大嘴,发出一声古怪的蛙鸣。

%19
蛙鸣声传播开去。在青茅山中回响。

%20
一时间,方源感到双耳嗡嗡作响。

%21
吞江蟾将蛙头垂下,大嘴对准冲刷而来的河水猛吸。

%22
哗哗哗!

%23
河水的流速顿时加快了十倍不止。纷纷涌入到吞江蟾的肚皮里去。河面以肉眼可见的速度,迅速下降。

%24
方源站在吞江蟾的身边,清晰地看到无数的鱼虾鳝鳖。也随着河流被吞江蟾吸入肚子中。

%25
察觉到河水的异变,赤山小组赶了过来,一个个看到正在进食的吞江蟾,皆是动容。

%26
“真是壮观呐!”赤城望着,难掩震惊的神色。

%27
“你成功了?”赤山则看向方源。

%28
“应该是吧。”方源神情淡漠地点点头。

%29
河水越降越低,直至断流,吞江蟾又再次高高地昂起头颅,肚皮一涨一缩,吐出无数的鱼虾龟鳖。

%30
啪啪啪。

%31
一时间,大量的河鲜落在地上。发出噼里啪啦的脆响。

%32
一条鱼儿在地上蹦跳,龟鳖摔得七荤八素,螃蟹在横走,然后又被落下的河鲜砸中身躯。

%33
方源起先没有太在意,只是随意的看着。忽然闻到了一丝酒香。

%34
“奇怪,怎么会有酒的香味?”赤城嗅了嗅鼻子,一脸惊奇。

%35
“应该是百年苦贝。”组中的女蛊师手指着一个磨盘大小,破损的黑色贝壳。

%36
这贝壳浑身漆黑如墨,壳上一圈圈的白色纹路,好像是树木的年轮。

%37
“不错。苦贝能将沙石化为苦水汁液。百年苦贝中的苦水,经过时间的积累达到质变,便变成了酒。白家寨的当代族长,就很喜欢喝这种苦贝酒。”赤山在一旁补充道。

%38
无须他细说,方源早就拿起这只贝壳,开始采集苦贝酒。

%39
不久前,他还在困恼如何获得这种苦贝酒,想不到却以这种形式出现在他的面前。

%40
真是踏破铁鞋无觅处,得来全不费工夫!

%41
吞江蟾几乎吞吸光了这条河的河水。河水深处,埋藏在泥沙中生活的百年苦贝,也因此暴露出来。

%42
方源很快就收集了六只百年苦贝。其中两只贝壳已经破损,其余四只却是完整无缺。

%43
“终于收集到苦酒,如此一来,就可以开始合练四味酒虫!”这一刻,方源心中的欢喜,不足以外人道也。

%44
“江昂!”

%45
吞江蟾吐完这些河鲜,又叫了一声,然后它深深地看了方源一眼,缓缓地转过巨型身躯,沿着河道,向下游而去。

%46
“真的成功了!”赤山口中喃喃,心石落地。他一直注视着吞江蟾离去,直至它的的背影,消失在视野中。

%47
“什么嘛,居然这么简单就赶跑了它。早知道这样,我们自己就可以完成了。现在却让方源这般容易,就成了英雄!”赤城撇撇嘴,语气充满了嫉妒,很不甘心。

%48
“方源,不管怎么说,你这次立下大功了。你是我们古月一族的英雄!”赤山复杂地看着方源说道。

%49
“哦。”方源心不在焉地应和一声,充满了敷衍的味道。同时,他双目炯炯,在满地的河鲜当中,继续翻找着百年苦贝。

%50
什么英雄,不过是一个赞誉罢了。

%51
而赞誉和诋毁,都不过是外人对自己的看法和观念。

%52
外人对自己的看法,方源根本就不在乎。

%53
你认为你的,我自活我的。

%54
英雄?狗熊?呵呵,还不如一只苦贝来的实在。

%55
赶走吞江蟾的消息,第一时间传入山寨。

%56
古月博连道三声好,厅堂中沉闷气氛一扫而空。

%57
唯有内务堂家老脸上神情复杂,他对方源并不看好,更在不久之前,对方源大肆批判。如今古月山寨危机,却是方源站出来解难。一前一后,这不是打他的脸么?

%58
“古月方源驱赶吞江蟾有功。破格提升为一组之长,奖五百元石。”古月博沉吟了一番,下了这道饱含深意的命令。

%59
酒肆中。

%60
“什么,方源竟然成功了?!”

%61
“奇怪,他不过是区区新人,如何能驱赶了一只五转蛊虫?”

%62
“就连赤山都要铩羽而归,他却做到了……”

%63
消息传来。众人惊异万分。

%64
“方源成了拯救我族的英雄?这……”和方源有仇怨的男蛊师听到这个消息后,不知所措。

%65
他的组长却忽然大喝一声,手指着酒肆掌柜还有一众伙计:“你们这些区区凡人。诋毁我族英雄,该杀!”

%66
话还未说完,他就是一道月刃。

%67
掌柜老者哪里料得到杀身之祸来得这么突然。被这记月刃射中脖颈,顿时身首异处。

%68
“大人饶命啊!”伙计们看到这一幕,先是楞了楞,然后猛地跪倒在地上,大声哭喊求饶。

%69
“组长,你这是干什么?”男蛊师站起身来。

%70
“干什么?”他的组长抖了抖眉头,语气沉重地叹息道,“今时不同往日了,阿海。方源一下子成了英雄,必定被高层看中。你说。如果我们在此诋毁他的事情,被有心人宣传了出去,会怎么样?在场的侦察蛊师大有人在,若是有看我们不顺眼的家伙,对家老们说上几句坏话。我们的前途就毁了!”

%71
男蛊师听得浑身冷汗。

%72
的确是这样,家族亲情至高无上。方源在外面对五转蛊虫,冒着生命的危险,保卫家族。而在此同时,他们却当众诋毁他,咒骂他。侮辱他。这是什么心态?这是狼心狗肺,不识好歹的无情冷漠!

%73
就好像是地球上一段历史,岳飞在外干仗,保家卫国,秦桧在朝廷中当内奸陷害。

%74
这些蛊师虽然还达不到陷害的地步,但这事情要真被人宣传出去,家族高层能放心他们这样的人吗?

%75
若要是漠颜、赤城这样有背景跟脚的,也就罢了。偏偏他们几个,都是上头没人的货色。

%76
在体制内往上爬,无非是人挤人,人踩人。这事情若被其他人利用,对他们的前途而言,将会造成极坏的影响!

%77
“现在挽救还来得及,只要表明态度,外人也不会多说什么。这些凡人贱命一条,死不足惜。不,他们能为我们牺牲,这是他们的荣幸。你们立刻就动手,一人杀一个,杀完之后,夸赞方源,表明态度!”族长低声喝道。

%78
“该死的!”男蛊师狠狠地咒骂了一声,在仇恨和前途中,他毫不犹豫地选择了后者。

%79
一记月刃砍下去,顿时一位伙计就惨死当中。

%80
“大人,求求你们放过我吧。”一时间,其他的伙计都瘫倒在地上,吓得屎尿横流。

%81
男蛊师却不管他们,众目睽睽之下,他义正言辞地手指着这些可怜的伙计,喝斥道:“你们这群人真是该死。古月方源是何等英雄,单凭一己之力,保全家族,你们吃了什么雄心豹子胆?竟然敢诋毁他!”

%82
男蛊师说着,紧紧地皱起眉头。

%83
这神情倒不作伪,方源是他深恨之人,但他却得当众夸赞方源,他为自己的话感到一阵的腻味和恶心!

%84
“大人,这都不是你叫我们……呃!”一位伙计感到冤屈的不得了,高声喊着。

%85
但他刚喊了一半,声音就戛然而止。

%86
一记月刃飞来,将他劈死。

%87
“一群贱民,自己诋毁也就罢了,还想倒打一耙,牵连我们!”出手的是一位女蛊师,此时她面罩寒霜,冷喝出声。

%88
其他蛊师看着这边,像是看一场闹剧。

%89
有的冷笑,有的淡漠,有的继续交谈,但没有人来上来劝阻。

%90
死些凡人算得了什么?

%91
大不了赔偿一些家奴罢了。

%92
大家都是一族中人,都是亲人,不会为了这些外人,而去干扰或者追究,凭白无故地生了什么间隙的。(未完待续。如果您喜欢这部作品,欢迎您来投推荐票、月票,您的支持,就是我最大的动力。)

%93
------------

%94
这不是求月票!

%95
现在是凌晨三点半。

%96
从下午开始写,一直写到现在。

%97
晚没有看一眼,听着窗外的鞭炮声,写着小说,就这样挺进了蛇年。

%98
看到这里,兴许你们以为我要求月票。

%99
不是的。

%100
我把这些天码出来的五万多字的存稿全删了。

%101
就在刚刚……

%102
呼!

%103
吐出一口浊气。

%104
老实讲,这些天我写的很烦躁,越写越烦躁。

%105
来自更新的压力,来自成绩的压力。

%106
说实在话,这本书从写开始,成绩好的有点出乎我意料。在这里,十分感谢海星编辑的支持!

%107
所以写着写着,心境就变化了。

%108
变得急躁,变得不安。

%109
变得想让成绩更好!想让支持的朋友们更开心!

%110
人总是这样,走在一条路上,走着走着,就忘记了出发时的初衷。

%111
人心总是如此善变,这就是我的痛楚和烦躁的来由。

%112
我写着写着,就忘记了当初为什么写,抱着什么样的理念在写。

%113
刚刚我重新看了一下《序言》,心情真的很复杂!

%114
这篇《序言》是写给大家看的,也是提醒我自己而写的。这是我当初为什么写这本书的缘由!

%115
再审视了一下这些天码出来的存稿,我越看越不满意。

%116
如果真的发出来,跟其他的网文有什么区别?今后最后悔的人,必定是我。因为是我背叛了自己,抛弃了当初的理想。这本书也就毁了!

%117
打开窗户,乡下老家的冷风,将头脑发热的我吹醒了。

%118
新的一年,就该有新的气象。如果重走以前的老路,有什么意思?

%119
舍得,舍得,没有舍,哪里来的得?

%120
我要写一部小说,完成一个六年前的梦!这是最大的前提。

%121
所以……我把这些存稿给删了,并且决定重新修整大纲。

%122
很对不住各位,原本准备新年爆更新的,但是恐怕是爆不了了。

%123
这都是我的错。

%124
但我不后悔,我知道如果我爆了这次更新,我会更后悔。

%125
大家热忱的支持,让我汗颜和惭愧。

%126
新年伊始,就让诸君失望了……

%127
但我确信,更新出来,诸君反而会更失望。因为那不是我想要写的“魔”。

%128
和大家做一个约定,这一次的先记下来,将来一定还!

%129
呼……

%130
写到这里,已经是三点五十三分。

%131
心神激荡,没有睡意。

%132
现在,我要沉下心来,洗去心中的浮躁,来重新写一个心中之魔。

\end{this_body}

