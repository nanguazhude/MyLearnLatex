\newsection{不择手段}    %第二十九节:不择手段

\begin{this_body}

“把酒都放到床下。”方源指挥着客栈的四个伙计。

他们每人手中都提着数坛青竹酒。这是方源勒索了其他学员之后,就来到客栈,一口气又购买了二十坛。

每坛两块元石,方源为了酒虫,一下子砸出四十块元石。

刚刚鼓起来的钱袋,瞬间就瘪了一半,只剩下三十九块元石。不过也算是物有所值,这些酒能够支撑酒虫一段较长的时间了。

“好咧。”伙计们连忙应道,对蛊师他们不敢有丝毫的不敬。

再加上方源购买了这么多的酒,可以说是客栈的大顾客了。在掌柜面前轻轻一句话,就能让这些伙计轻而易举地丢掉工作。

伙计们走后,方源关上宿舍房门,盘坐在床榻上。

已经是夜里了。

天空中月明星稀,夜风中流淌着暗香。

房间里没有点灯,方源抚平心绪,心神投入元海。

元海波涛生灭,海水散发着青铜色的光泽。每一股海水,都是一转蛊师特有的青铜真元。

元海只达到整个空窍的四成四,这是方源丙等资质的局限。

空窍四壁,是一层薄薄的白色光膜,支撑又包裹着空窍。

在元海上空,空无一物。春秋蝉在方源的调动下,已经再次隐藏,在沉睡中修养去了。

倒是真元海面上,漂浮着一只白胖可爱的酒虫。

它在真元海水中肆意撒欢,有时候潜游入海,有时候摆头甩尾,溅起点点水滴。

方源念头一动,酒虫顿时响应,停止了嬉戏,团成一个汤圆形状,悠悠地漂浮起来,升到空窍中央地带,脱离了青铜海面。

“去。”方源调动一成真元,形成一股细流,逆冲而上,悉数灌注到酒虫体内。

酒虫早已经被他炼化,因此来者不拒,把这股真元全数吸收进身体里面。

顿时,四成四的海面,就下降了一小截。

将真元化为动力,团成一团的酒虫开始绽放出白色的毫光。毫光中氤氲的酒气渐渐生出,最终汇成一团淡白色的酒雾。

酒雾奇妙,也不飘散,而是笼罩在酒虫的身边。

“起。”方源念头一动,再调动出一成真元。

青铜海水扑上酒雾,酒雾融入海水当中,越来越少,最终一丝都不剩。而那一成青铜真元,也凭空消失了一般体积,只剩下了半成。

但这半成真元,却比先前的更加凝练。原本的真元,是翠绿色,散发着铜的光泽。现在的这股真元,虽然同样有一种铜的色泽,但是绿色却更深一层,是苍绿色。

苍绿色真元,是一转中阶蛊师才具备的真元。酒虫的作用,就是凝练真元,将蛊师的真元提升一个小境界!

蛊师有九大境界,从低到高分外一转、二转,直至九转。每一大境界中又细分四个小境界,分别是初阶、中阶、高阶和巅峰。

方源此时只是一转初阶蛊师,但是在酒虫的帮助下,却有了半成的一转中阶的蛊师真元!

“我要凝练出半成中阶真元,就得消耗两成初阶真元。我要将空窍中四成四的元海,都转换成中阶真元,就得耗费近十八成的初阶真元。要想尽快地达成这个目标,就得借助元石了。”

想到这里,方源睁开双眼,从袋子里掏出一颗鸭蛋大小的完整元石。

元石是一种椭圆体半透明的石块,灰白色,随着石内天然真元不断损耗,它的体型也会不断缩小。

他右手慢慢合拢,将元石紧握在手心中,汲取着里面的天然真元,不断地补充到自己的空窍之中。

空窍中原本下降的海面,随之缓慢地上升起来。

元石就是拿来用的,方源一点也不吝啬,更不会节省。

“我没有靠山,没有亲朋好友的资助,所以只能靠抢劫勒索。今天只是第一次,以后每隔七天学堂发放元石补贴的时候,我就继续堵住学堂大门口。”

抢劫勒索一次,怎么能满足方源的胃口。蛊师修行,元石最缺少不得了。

至于此次抢劫的影响,方源一点都不担心。

这个世界,不同于地球。

在地球上,学校往往都禁止斗殴,以稳定和谐为主。但是这个世界,战斗是主题。

不管是蛊师还是凡人,都要为了生存而战斗。有时候是和恐怖的野兽搏斗,有时候是拼战恶劣狂暴的天气,有时候为了争夺资源,和其他蛊师交战。

因此有节制的打架斗殴,反而被人们所鼓励和提倡。

从小到大,从打架斗殴到生死激战,这是大多数人类的生活写照。

这个世界面积广大无边,单单方源生活的南疆,就比七八个地球的总面积还要广阔。这里的生活环境十分恶劣,人类通常以家族的形式,建造山寨,龟缩一隅。

每隔一段时间,都有兽潮,或者是极恶天气,来冲击山寨。

蛊师成为守护山寨的中坚力量,每年减员的状况都比较严重。

生活需要有强大战斗意志的人。家族需要战斗蛊师,从不嫌多。

况且,方源出手是有分寸的。

他从不攻击下巴,因为这会容易造成颅底骨折,容易出人命。也不攻击别人的后脑勺,打斗的时候,他没有用拳,也没有用肘,或手指戳,而是用的手掌。踢脚的次数也屈指可数。

倒下去的学员,没有重伤,顶多是轻伤。

方源并不嗜杀,他只是把杀当做一种手段。

每次动手,他都有明确的目标。什么样的手段,是能让他达到目标的捷径,他就用哪一个。

换句话说,他行事不择手段。

……

阴云漂浮过来,遮盖住月光。

一层阴影笼罩住古月山寨。

更夫梆、梆、梆地敲着梆子,提示人们已经是深夜,小心火烛,小心防范野兽袭击,以及可能潜入进来的外寨蛊师。

山寨中还有不少灯火未熄。

在赤之分家,古月赤练的书房中,灯火明亮。

这个位高权重的老人,正以一种温和的语气,慰问自己的孙子古月赤城:“听说你今天被那方源打了?”

古月赤城右眼黑了一圈,他气愤地道:“是的,爷爷。方源那个家伙,只是区区丙等,竟然如此嚣张。他把我们都堵在门口,不顾同窗的情谊,抢劫我们的元石。更可气的是,学堂方面居然睁一只眼闭一只眼。直到方源扬长而去,侍卫们才赶过来。爷爷,这次您可得帮孙儿出这口恶气啊!”

古月赤练却摇摇头:“这是你们小辈之间的事情。你也只不过被勒索了区区一块元石,更没有受到重伤,爷爷我师出无名啊。况且就算你被打成重伤,爷爷也不会为你出头的,你明白为什么吗?”

古月赤城愣住了,他苦苦思索,半晌口中迟疑地道:“爷爷,我有些懂你的意思。您是希望我靠自己的力量,找回场子是吗?”

“你只是理解了一个方面。”古月赤练点点头,“你要记住,你不仅仅只代表个人,还代表我们赤之分脉的形象。我们赤家和漠家对峙多年,你的一举一动就代表着赤家未来的希望。爷爷可以暗地里帮助你,但是你必须竖立起自强自立的形象,否则支持我们的家老们看不到未来的希望,都会舍我们赤家而去。”

说到这里,古月赤练叹息一声:“这也是为什么,爷爷要帮助你作弊,让你冒充乙等资质的原因。我们赤家需要一个强有力的继承人,才能让那些支持我们的人坚持下去。”

古月赤城这才恍然:“爷爷,孙儿懂了。”

古月赤练摇头:“光懂还没有用,要去努力奋发。方源这件事是个小麻烦,你接下来须勤学苦练基本拳脚,把场子找回来。同时,也不要忘了,努力修行,早日晋升中阶。最好是夺得班头的位置,这是莫大的荣耀,对我们赤家也是一种帮助。”

“是的,爷爷。”古月赤城大声应和。

“呵呵呵,这股精神气才像我们赤脉的继承人模样。孙儿你好好努力,爷爷为尽全力帮助你的。”

\end{this_body}

