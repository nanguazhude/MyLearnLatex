\newsection{想买酒虫?}    %第一百零四节:想买酒虫?

\begin{this_body}

%1
最终,方源没有走下楼,弟弟也没有主动登上去。

%2
两方的坚持,上下楼的距离,似乎也预示着这兄弟俩之间的隔阂越来越大。

%3
交谈并不愉快。

%4
“哥哥,你太过分了!想不到你竟然是这样的人!”楼下,方正站着,眉头紧皱,大声呵斥。

%5
方源却不动怒,反而轻笑:“哦,我是哪样的人?”

%6
“哥哥!”方正长叹一口气,“我们爹娘死后,是舅父舅母收养的我们。他们对我们俩有养育之恩呐。想不到你却如此不念旧情,恩将仇报,哥哥,你的心是铁石做的吗?”

%7
到这里,方正的语调都在微微地颤抖。

%8
“真是奇怪,这些家产,本来就是我的,有什么恩将仇报之说呢。”方源淡淡地反驳一句。

%9
方正咬牙,点头承认:“是!我知道,这些家产是双亲留下来的。但是你也不能全夺了去,总得留一些给舅父舅母养老吧!你这样做,真是让人寒心,让我看不起你!”

%10
顿了一顿,他继续道:“你有没有回家看看,就看一眼他们两位老人现在生活的样子。家里现在连家奴都退了一大半,养不起了。哥哥,你怎么能如此绝情!”

%11
方正双目通红,捏紧了双拳,对方源大声质问。

%12
方源不由地冷笑一声。他知道舅父舅母这些年来,掌管这些家产,必定有大量的钱财累积。就算是没有。单凭酒肆月末的进账,也定然能养得起那些家奴。他们之所以如此哭穷,无非是撺掇方正来这里闹而已。

%13
方源用目光打量着方正,索性直接道:“我可爱的弟弟,如果我执意不还,那你又能怎样呢?你虽然也是十六岁,但别忘了,你已经认了他们俩为父母。你已经丧失了继承家产的资格。”

%14
“我知道!”方正双目绽放出一抹神光,“所以我找你来斗蛊,我要对你下战书。在擂台上一决胜负,如果我赢了,请你把一部分的家产还给他们二老。”

%15
这个世界上的斗蛊,就如同地球武林中的比武、搭手。

%16
族人之间,如果有矛盾不可调和。就用这个方法解决问题。斗蛊也分许多种,有单斗双斗。有文斗武斗。还有生死斗。

%17
当然,方源和方正之间,如果要斗蛊,不会严重到生死斗的程度。

%18
看着楼下神情坚决的弟弟,方源忽然笑了起来:“看来此行之前,舅父舅母都特意嘱咐你了吧。不过,作为我的手下败将。你这么有信心能赢得了我?”

%19
方正眯了眯眼,不由地想到不久前。擂台上那屈辱的一幕。

%20
从那之后,他每次回想起来。心中都会涌起一股愤怒。这股愤怒,既是针对方源,又是针对自己。

%21
他恨自己不争气,临战慌乱,事实上,那场比斗他发挥失常了。气势被方源所夺,玉皮蛊到了最后关头,才想到而使用出来。最后,他败得很突然,也很憋屈。

%22
方正这种对自己的恨怒,更衍生出了强烈的不甘心。

%23
于是不可避免的,就有了这样的想法——“如果事情能够重来,我一定能表现得更好,击败哥哥的!”

%24
所以,当舅父舅母向他哭诉时,方正除了想替二老讨回部分家产之外,还有想重新和方源当众比试一场,重新证明自己的意思。

%25
“今时不同往日了,哥哥。”方正看着方源,双眼中斗志如火,熊熊燃烧,“上一次我发挥失常,让你得胜了。这一次,我已经成功地合炼出二转蛊虫月霓裳,你再也不能突破我的防御!”

%26
话音刚落,他的身体周围,便浮现出一片朦胧的淡蓝雾气。

%27
这些雾气包裹着他,在雾气中,渐渐地凝练成一条修长飘荡的绶带。

%28
飘带绕过腰背一圈,缠在他的双臂上。绶带中段,在他的脑后高高飘荡,使方正不由地散发着一种飘逸神秘的仙气。

%29
“果真是月霓裳,真是愚蠢啊,居然直接把底牌暴露出来。”方源站在楼上,看到如此情形,目光闪烁了一下。

%30
月霓裳是二阶蛊虫,防御类型。虽然在防御上,月霓裳比白玉蛊要差上一筹,但是它却有能帮助其他人防御的能力,小组作战时,对整个团队的贡献很大。

%31
方正有了这只蛊,方源的确不能再用双拳打破他的防御。拳头打过去,就像是打中厚棉絮一样,完全丧失了爆发力。

%32
就算是用月光蛊也不行,除非是月芒蛊。所以方正真要下了战术,约方源斗蛊,按照族规方源必须得接受。不能暴露白玉蛊的情况下,方源说不得还真的要输。

%33
甲等资质毕竟是甲等,再加上族长的悉心栽培,方正成长的很快。如果说,在学堂,方源压着方正,但是如今不得不承认,方正已经逐步绽放出了天才的光辉,渐渐对方源产生了威胁。

%34
“但是,你以为我没有料到么?”方源俯视着楼下的弟弟,嘴角微微翘起。

%35
他对方正道:“我执着的弟弟,你要约斗我当然可以。但是你征求了你的组员同意了么?如果在约斗期间,你们小组刚巧要外出执行任务,你该如何抉择呢?”

%36
方正顿时一愣,他的确没有想过这个方面。

%37
他不得不承认,哥哥方源说的很有道理。小组自然要一齐行动,组员若要单独行动,事先得要汇报。

%38
“所以,你不妨回去,找到你那组长古月青书,说一说这个情况。我在东门的酒肆等你们。”方源道。

%39
方正稍稍犹豫了一下,最终咬咬牙:“哥哥,我这就去!不过我也得告诉你。拖延计策是没有用的。”

%40
他来到古月青书的住处,自有家奴领他进门。

%41
古月青书正在练习用蛊。

%42
他的身影,在自家庭院的演武场上跳转腾挪,矫健无比。

%43
“青藤蛊。”他轻喝一声,右手掌心中呼的一下,冒出一条碧绿的藤条。藤条长达十五米,被青书顺势抓在手中,当做鞭子甩、劈、撩、扫。

%44
啪啪啪!

%45
鞭影纵横。扫在地上,把一片片的青石方砖都打得开裂。

%46
“松针蛊。”他忽然收了藤鞭,顺势一甩青色的长发。

%47
顿时从散漫开来的长发中,咻咻咻地射出一阵密集如雨般的松针。

%48
一蓬松针打在不远处的一具木人傀儡上,顿时将其全身都洞穿,造成密密麻麻的针眼。

%49
“月旋蛊。”紧接着,他左手平伸。掌心处有一片绿色月牙印记,忽然散发出盈盈的绿芒。

%50
青书紧接着一甩手。一片绿色的月刃就被甩飞了出去。

%51
不同于寻常月刃直线的攻击。这片绿色月刃,弧度更弯,飞在空中,划出一道弯弯的曲线,这无疑更令敌人难以防御。

%52
“青书前辈不愧是我们寨子里,二转蛊师中的第一人!这样的攻击下,我连十个呼吸都撑不住吧。真的太强了。”方正看得目瞪口呆。一时间都忘了自己来到此处的目的。

%53
“哦?方正,你怎么来了。上一个任务刚刚结束。要多注意休息,劳逸结合哟。”古月青书发现了方正。便收了架势,温和地笑着。

%54
“青书前辈。”方正恭敬地向他行了一礼。

%55
这份恭敬发自方正的内心。因为从入组以来,方正就受到青书的细心关照,在方正眼中,青书就是他的哥哥。

%56
“方正啊,看来你是有什么事情来找我?”青书一边拿着布巾擦着额头的汗渍,一边笑眯眯地走过来。

%57
“是这样子的……”方正说明了来意,以及整个事情的来龙去脉。

%58
青书听了,双眉地微微扬起。事实上,他对方源了解很多,比起方正,他对方源更感兴趣。

%59
“不妨会一会他。”

%60
念及于此,古月青书点点头:“正好我也有事情要找你哥哥谈呢,既然如此,那我们就一块去吧。”

%61
方正大喜:“谢谢前辈!”

%62
“呵呵呵,谢什么,我们可是一组的。”青书拍拍方正的肩膀。

%63
方正顿时感到一股暖流涌动在心田,眼眶都不禁泛红了。

%64
两人来到酒肆门口,便有伙计早早地盼着,领着二人来到酒肆里面。

%65
一张靠着窗户的方桌上,摆着几碟小菜,两个酒杯,一坛酒。

%66
方源坐在一侧,看到古月青书,微微一笑,伸手示意:“请坐。”

%67
古月青书对方源点点头,坐了下来,又对方正道:“方正,随便逛逛去吧。我和你哥谈一谈。”

%68
他是个聪明人,看到只有两个酒杯,就知道方源是想和自己单独交谈。

%69
事实上,他也正有此意。

%70
方正哦了一声,只好心不甘情不愿地出了去。

%71
“我知道你,方源。”青书微笑着,自来熟地拍开酒坛,给方源倒了一杯,又自己倒了一杯。

%72
“你很有意思,是个聪明人。”接着,他把酒杯举起来,遥遥对着方源。

%73
方源哈哈一笑,同样举杯相敬。

%74
两人同时一饮而尽。

%75
青书又为方源斟酒,同时自己也斟了一杯。

%76
他一边倒酒,一边说道:“和聪明人讲话,饶弯子没有意思。我就直接说了,我想买你的酒虫。不知道你要卖多少?”

%77
他没有问方源,你卖不卖酒虫。而是直接问——不知道你要卖多少。显示出了强烈的信心。

%78
他是二转蛊师第一人,年纪轻轻,修为以及达到二转巅峰。将二转高阶的赤山以及漠颜,都压在身下。

%79
他一出场,就反客为主,为方源斟酒,敬酒,占据主动。

%80
他的自信搭配上他温和的笑容,形成一种独特的气质。不会咄咄逼人,让人觉得反感,但又让人觉出他的坚持。

%81
他一头青色的长发,冬日的阳光透过窗户照在他白皙的,线条柔和的脸庞。让方源不由地联想到明媚的春光。

%82
“是个人杰,不过可惜了。”方源心道。

%83
他对青书的反客为主毫不在意,带着些微的欣赏,轻轻一叹。

\end{this_body}

