\newsection{元石不过身外物}    %第一百一十一节:元石不过身外物

\begin{this_body}

%1
“奶奶,这只是什么蛊?”少女指着三层的中央柜台,好奇地问道。

%2
这树屋中,分三层。其中第一层,专买一转级别的蛊虫。第二层,贩卖二转蛊虫。第三层,则卖着三转蛊虫。

%3
越往上层,蛊虫越少,价格也越高。

%4
当然,能摆放在树屋中出售的蛊虫,都是比较珍稀的。

%5
古月药姬顺着孙女的眼光看去,只见一个圆桶状的高瘦树墩,树墩上生长着五条枝叶,好像是人的五根手指,又在中间交错。

%6
一只圆球状的蛊虫,只有拇指大小。被细小的树枝缠绕着,在翠绿的树叶掩映下,散发着银白色的光。

%7
“这是白银舍利蛊,只能使用一次。能让三转蛊师的修为,瞬间提升一个小境界。”古月药姬缓缓地解释道。

%8
舍利蛊,也是一个系列的蛊虫。

%9
一转的是青铜舍利蛊,专门针对一转蛊师。二转的是赤铁舍利蛊,只对二转蛊师有效。三转的就是这白银舍利蛊了。

%10
到了四转,还有黄金舍利蛊。

%11
“标价就要三万块元石,好贵!”少女吓得吐了吐舌头。

%12
古月药姬点点头:“这只蛊最后至少能卖到五万块元石。好了,这里我们已经逛得差不多了,去一层大门处的总台,酒虫应该有结果了。”

%13
树屋中的一转蛊虫,只要是被投了价格的,每一只在柜台上摆放的时间,都只有半天。无人问津的蛊虫。则仍旧摆放着,直到有人投价。

%14
二转蛊虫,会摆放一天。三转蛊虫则是两天。

%15
这规矩乍看之下,比较古怪。但其实却是实践积累出来,最适合买卖的方式。

%16
总台。

%17
“什么,酒虫已经被其他人买走了?”古月药姬得到结果后,顿时蹙起眉头。她觉得自己开价已经很高了。对于得到酒虫她至少有八九成的把握。但是没有想到,居然失手了。

%18
“哼!是谁这么坏,抢了我的酒虫宝宝?”少女气鼓鼓地问道。

%19
“药乐。”古月药姬提醒了少女一句。

%20
少女撅起樱桃小嘴。乖乖地不再说话。

%21
柜台后的店员,是一名二转的女蛊师,她微微鞠躬。回答少女的话,道:“很抱歉,顾客的信息是一概不公布的。行有行规,还请见谅。”

%22
正是因为不公布,才能打消了许多顾客的顾虑,使得他们能够放开手脚来竞价。

%23
有些时候,很多东西明明想要,但是碍于情面,只能礼让。毕竟家族中人嘛,抬头不见低头见的。

%24
但是用了这样的售卖法子。暗地里买卖,就能绕开这个情面问题。

%25
凭什么这么好的东西,必须要让给你,就因为你是我的长辈亲戚朋友?

%26
永远不要小看每个人心中的阴暗面。

%27
而私密的交易,更能让这种阴暗面展现出来。

%28
古月药姬沉吟了一下:“老身知道行规。你放心,小姑娘,老身也不问购买酒虫的那人姓名,只想打听一下最后成交的价格。”

%29
女蛊师又鞠了一躬:“实在对不起,价格也是要保密的。但成交的价格一定是所有价格中最高的,请您老放心。贾家行商。向来以诚意为本。”

%30
“哼,小姑娘你知道我是谁吗?”古月药姬脸沉下来,不悦地冷哼一声。

%31
“发生了什么事情?”这时,一位三转的中年男蛊师赶了过来。

%32
这间树屋一直在蛊师的监控之下,发生了什么事情,自然一清二楚。

%33
“管事大人。”女蛊师立即向这位中年男子问好。

%34
男子对她挥挥手:“你先下去罢,这里由我来处理。”

%35
然后转身正对古月药姬,笑道:“原来是药姬大人呐,这位一定是您的孙女吧,真是灵慧可人。”

%36
见这位男子也是三转蛊师,古月药姬的脸色柔和了下来,但是却仍旧想知道成交价。

%37
男管事感到有些棘手。

%38
他是商队的老人了,也是贾富的心腹,行商多年,对古月山寨的情况了解很深。知道眼前这位老人家的来头。

%39
对于他们来讲,哪怕得罪了古月赤练或者古月漠尘,也不愿得罪古月药姬。后者的影响力,仅次于族长古月博。

%40
男管事想了想,道:“这样吧,药姬大人既然这么想购买酒虫,那我就做主,私下里秘密地调一只过来。实不相瞒,库存中有三只酒虫,每一只售卖的地点,贾富大人都是亲自敲定的。大人您应该也知道,酒虫的珍贵。至于多少元石,就按照您老的出价算账。”

%41
古月药姬却微微摇头,将手中的拐杖顿在地上,发出砰的一声轻响。

%42
她说道:“老身可不想贪图这个便宜。价格……就按照刚才那只酒虫的成交价算吧。”

%43
“这……”管事的犹豫起来,他当然看得出古月药姬的目的。

%44
古月药姬故作不悦,继续向男管事施压:“怎么,难道这价格很高,怕老身付不起吗?”

%45
“当然不是这个意思了。唉,那好吧,就按照您说的办。”管事的叹了口气,说出了一个价格。

%46
少女听了,先是轻舒一口气,旋即又有些不忿:“什么呀,只比我们多二十块元石罢了。”

%47
古月药姬则眯了眯双眼,却是没有说话。

%48
于此同时,出了树屋的方源,已经来到了酒铺。

%49
第二只酒虫已经到手,现在剩下的就是酸甜苦辣四种美酒。

%50
“甜酒我已经有了,当初完成家产任务时,还多出不少的黄金蜜酒。辣酒、酸酒应该不成问题,现在担心的就是苦酒。”方源心中想着,隐隐有些担忧。

%51
如果有苦酒。他今晚就能开始合炼四味酒虫。若是没有,不知道要等到猴年马月了。

%52
人生很多事情,都是怕什么,来什么。

%53
方源的担忧成了真。他花费了数个小时在无数的帐篷中奔波,寻到了辣酒和酸酒,但却没有找得到苦酒。

%54
“世间之事不如意者,十之八九啊。”方源无奈得很。这样一来,他只好搁置了酒虫的合炼计划了。

%55
没有了四味酒虫,他的修为晋升速度。就很平常了。

%56
当天下午,他又来到树屋。

%57
一层的许多柜台上,都换上了新的蛊虫。

%58
原本摆放着酒虫的中央柜台上。已经被一只净水蛊占据了。

%59
净水蛊就像是地球上的水蛭,俗称就是蚂蟥。只是它比蚂蟥可爱多了,浑身都是淡蓝色的,泛着水光。

%60
“净水蛊能消除空窍中的异种气息,对于赤城来讲,是必得的蛊虫。”看到这只净水蛊,方源就想到了赤城。

%61
他知道赤城只是丙等资质,靠着爷爷古月赤练的真元强行提拔修为,因此空窍中掺杂了赤练的气息。如果不洗练掉,对于此次赤城的前景来讲。危害颇大。

%62
“赤练一定会为赤城购买这只蛊虫的。让我算算看,嗯……他的报价应该在六百三十到六百四十之间。”

%63
这价格,已经比酒虫的市价还要贵了。主要还是因为赤城特别需要这只蛊虫。

%64
“我如果能出价六百五十,应该就能拿下这只净水蛊了。若再加十块,这只净水蛊必定落入我手!至于今天上午买的酒虫。我的出价应该比古月药姬多出大约二十块元石。”方源心中冷笑。

%65
他有这种自信。

%66
这种自信,是他五百年的经验,再加上地球上发达数倍的商业理论,凝结起来的。已经凌驾于凡俗。

%67
按照方源在前世的实践经验,往往当他多出十块元石,就有八成把握能拿下物品。买酒虫的时候。之所以又多出十块,完全是方源谨慎的行事作风。

%68
方源最终没有出价,净水蛊并不是他所需要的。同时一旦得到它,也会引来赤练的调查。当然最主要的原因,还在于方源需要留着钱财,看看接下来的几天里,有什么好的蛊虫。

%69
“我现在缺少两只蛊虫,一只用于侦察,一只辅助移动。来年青茅山上就会爆发狼潮,商队不会再来了。虽说有着花酒行者的遗藏,但那毕竟是花酒行者受伤之后,仓促留下的东西。谁知道完整不完整,接下来会有什么蛊虫?”

%70
记忆中,来年的狼潮十分凶险。方源可不想由于缺少蛊虫的原因,导致自己应对能力不足,在狼潮中伤残甚至殒命。

%71
现在的他,一旦陷入狼群的包围当中,基本上就是凶多吉少了。

%72
所以,在此之前,他需要做到最充足的准备,修为和蛊虫一个都能少。

%73
此后连续三天,他多次来到树屋。

%74
第三天时,在树屋的第一层,他发现了一个惊喜——一只黒豕蛊!

%75
黑白豕蛊,都是能从根本上增加蛊师力量的蛊虫。方源已经利用白豕蛊,为自己增添了一猪之力。如果他继续使用第二只白豕蛊,那么不会有任何力量增加的效果。但是黒豕蛊却不同,黑白豕蛊之间的力量是可以相互叠加的。

%76
于是当中午来临时,他的手中再次多了一只蛊虫。

%77
接下来就没有什么了。

%78
柜台上出现了一些用于侦察和辅助移动的蛊虫,但都不是方源满意的。

%79
这些蛊虫都在普通柜台上展出,行情并不好,没有多少人购买。方源又打听到这次商队将要停留长达八天的时间,所以他一直耐心地等待着,并不着急。

%80
一直到了第七天。

%81
在树屋的第二层,方源发现了一枚赤铁舍利蛊。

%82
二转蛊师用之,能立即提升一个小境界!

%83
标价三千块元石,引得无数二转蛊师争相出价,纸片投入柜台当中,场面十分火爆。

%84
“我若得了这只赤铁舍利蛊,就能将自身修为立即推到中阶地步。有了中阶的绯红真元,不管是月芒蛊还是白玉蛊,我都能催动更多次。”

%85
修为是蛊师的根本,修为一旦提升上去,战斗力就会随之上涨。在效果方面,远比方源补上侦察和移动蛊虫的,要好上许多。

%86
况且树屋中这两类的蛊虫,对方源来讲,都是普通货色,还没有令他看中的。

%87
“但是我之前收购了酒虫和黒豕蛊,还买了一些酒,这赤铁舍利蛊最后的成交价,一定超过五千块元石,甚至能达到八千左右。毕竟大家都知道狼潮即将到来,这个时候提升一个小境界,对自己的帮助太大了。要将这舍利蛊得到手,我的元石恐怕不够用!”

%88
方源瞬间意识到,一个难题已经摆在了他的面前。

\end{this_body}

