\newsection{影灭壁破现甬道,魔道传承岂易}    %第六十节:影灭壁破现甬道,魔道传承岂易

\begin{this_body}

%1
初夏的夜空是美丽的。

%2
天上没有一朵浮云,晶莹的星星闪烁着动容的光芒。

%3
檀香蟋蟀粉墨登场,取代了龙丸蛐蛐的位置,它们在草丛中、山溪边、树枝上唱着一曲曲抒情的歌。

%4
古月山寨中亮着点点的灯光,似乎在和星空遥相辉映。

%5
一座座碧绿的竹楼在夜风中挺立,星空之下别有一番祥和静谧的悠然氛围。

%6
方源此时却不在山寨中,而是潜进了石缝秘洞。

%7
他半跪在地面上,右手都抚上影壁,掌心中散发着一团蓝色的月光。

%8
影壁上,原先的画面早已经消失,和周围的石壁别无二致。若不是方源清楚地记得位置,谁会知道这影壁下藏有花酒行者的力量传承呢?

%9
一个多月前的夜晚,影壁发生了异变,突然出现了花酒行者的密藏。影壁上先是现出了花酒行者浑身浴血,扬言要留下传承的影像。然后是一行血字,提示后来人打破影壁,就会出现洞口。再之后血字消失不见,影壁上留影存声蛊的力量完全耗散,影壁也就还原成了普通的山壁。

%10
虽然知道了花酒行者的传承,但是方源一直没有时间,对此进行探索。

%11
因为这个意外,他当场杀了贾金生,那天晚上就忙着处理现场。而后为了应对即将到来的审讯,精心设局,活动范围一直局限在山寨之中。

%12
直到贾富离开,学堂家老撤销暗中调查,如此过了十多天,风声渐渐消停,方源这才秘密地潜回这里。

%13
空窍中的青铜海在缓缓地下降,方源调动着一股股的真元,不断地涌入到右手掌心中的月光蛊中。

%14
月光蛊散发着一团柔和的月光,月光微微地闪烁不停,在它的作用下,石壁不断被削减,大量的石粉洒落而下。

%15
这是方源对月光蛊的精微操控,当初他解石时就是用的这一手。

%16
相对于解石,这种手法还是显得相当粗犷的。但是对于破开这厚厚的石壁,这手段又显得太温柔了些。

%17
这已经是方源连续第六天,动用月光蛊在碾磨这石壁了。

%18
在地上,已经堆了一堆厚厚的暗红色石粉。

%19
按道理来讲,青茅山的土都是青色的泥。但是这里的山土却古怪地显现出赤红,还散发着暗光。

%20
不过也幸亏如此,有了这光源,方源才不用准备火把。

%21
为了尽最大可能不引起怀疑,方源没有动用任何的工具。铁锤和铁镐无疑更方便他破开石壁,但是这样一来,破壁的声音就会在夏夜中回荡。

%22
不管从外面听起来,这声音微弱或是响亮,方源都要杜绝这些疑点的出现。

%23
细节往往决定成败。

%24
对于方源来讲,谨慎与其说是优点,倒不如说是习惯。

%25
前世他大大咧咧过,但很快就受到惨重的教训。常常都说,人越老胆越小。其实不是胆子小,而是谨慎耐心。五百年的时光,更是将谨慎二字沉淀到他的骨子里了。

%26
“呼……”喘了一口粗气,方源慢慢停止了真元的灌输。

%27
他一屁股坐到了地上,实在是有些累了。

%28
整片半人高的影壁,已经被他整整磨去了三寸的厚度。

%29
他擦了擦满头的汗,一边活动肩膀,一边舒展双腿。因为长时间蹲着,此时一阵阵酥麻的感觉从他的腿上不断传来。

%30
咚咚咚。

%31
方源扣指,又敲了敲石壁。

%32
听着这个声音,他心中微微泛喜,知道这石壁越来越薄了。

%33
再闭目安神,感受了一下空窍中的情景。

%34
青铜元海还剩下两成不到。

%35
“继续努力!”方源咬咬牙,再度伸出右手,抚上石壁。

%36
水蓝色的月光持续闪亮了一刻钟,方源忽然动作一顿。他拿开右手,欣喜地发现石壁已经破开了一个小洞。

%37
他当即站起身来,用脚一踹。

%38
哗啦一声,洞口顿时扩大,形成竹篮一般大小。

%39
方源谨慎地后退几步,感觉到一股沉闷的郁气,渐渐从洞口中散发出来,然后弥漫到这秘洞当中。

%40
秘洞的通风效果并不好,方源想了想,便选择钻出石缝,回到了外界。

%41
过了好一阵子,他这才回转。

%42
秘洞中气闷的感觉,比先前舒缓了许多。方源继续扩宽洞口,时而用月光蛊,时而用手扒拉,时而用脚踹。片刻后,他终于将洞口扩宽到自己能从容进入的尺寸。

%43
从洞口望去,里面是个斜向下的笔直甬道。

%44
甬道初始时不宽,但是越往下越是宽敞。人刚刚钻入甬道,必须弯腰低头,但是走到后半段就能直起身子,大跨步地走了。

%45
这洞口四壁也是古怪的赤土,散着一层略显昏暗的红光。这使得甬道中的景象清晰可见。

%46
但甬道很长,延展到视野之外,因为角度的原因,方源也看不到甬道的尽头是什么。

%47
他立足在洞口,没有立即就跨进去,而是双目眯着,站在原地。

%48
力量传承,不同于遗藏。

%49
遗藏是指一名蛊师死了,遗留下来的东西。发现遗藏者,往往一下子就得到了尸体上所有的东西。

%50
力量传承,则是蛊师将死之前,不想自己的流派灭绝,或者福泽后人,或者想在世界留个最后的印记等等原因,主动设下关卡,考验后来之人。

%51
拿花酒行者来讲,他之所以设立这个传承,目的很明确,就是为了培养一名复仇者,替他想古月一族复仇!

%52
后来之人若能通过这些考验,就能获得种种好处。通过了最终考验的人,就意味着他(她)获得了完整的力量传承。

%53
按照力量传承者的阵营划分,力量传承自然就分为正道传承和魔道传承。

%54
正道传承通常设计精巧,考验后来人的心性品德。中途失败的人,也不会有性命之忧。

%55
魔道传承就复杂了。

%56
魔道中人,往往疯癫执着,或者冷酷无情,或者杀人如麻,不能以常理判断。

%57
有的魔道传承,设计得极考验心智。谜题重重,很多人一生都陷在其中,苦苦思索不得结果。

%58
有的魔道传承,简单至极,就是一间密室,里面直接摆放着蛊虫和元石。

%59
有的魔道传承,关卡残酷至极,中途失败往往就意味着死亡。

%60
更有甚者,一些魔君魔头的传承根本是个谎言,本身就是个巨大的陷阱。他们遵循着损人不利己的行事原则,临死之前都要奋力设计,坑别人一把。

%61
“花酒行者的这道传承,会是哪一种呢?”方源思索着。

%62
他有前世的记忆,未来很多著名的魔道传承,他都一清二楚。但是偏偏这道传承,前世并没有人发掘,他对此毫无所知。

%63
“按照道理来说,陷阱的可能性并不大。否则花酒行者也不用设计这道影壁了。不过是否会有机关呢?”

%64
方源拾起一块扒拉出来的石头,朝着洞中扔去。

%65
石头在甬道中磕磕碰碰,很快就滚出了方源的视野,只听到一阵一连串的声音,在通道里回响。

%66
方源若有所思。投石问路的结果,看来是安全的。

%67
但是他仍旧没有进入甬道,而是取出石粉,均匀地洒在刚刚开出的甬道洞口附近。同时在秘洞的石缝入口处,也洒了薄薄的一层。

%68
然后他钻出入口处的狭窄石缝,离开了这里。

%69
在距离河滩数百米外的一处极隐蔽的草丛中,他找到了此次随时携带来的青竹酒。他拍开酒坛,狠狠地灌了一口,又故意洒出些酒液,沾湿衣裳,弄得一身酒气。

%70
拎着酒坛,他回到山寨,到了宿舍,刚刚好是午夜时分。

%71
常常一夜不归的话,是会惹人怀疑的。这样就很好,上半夜外出,下半夜归来。

%72
在寻找酒虫那会儿,方源就已经这么做了,有了前科,如今这种行为倒也不算突兀。

%73
星消日出,一夜无话。

\end{this_body}

