\newsection{蛊师有九转,花酒留遗藏}    %第七节:蛊师有九转,花酒留遗藏

\begin{this_body}

很快,一个星期就过去了。

“人是万物之灵,蛊是天地之精。在这个世界上存在着成千上万种,数不胜数的蛊。它们就生活在我们的周围,在矿土里,在草丛里,甚至在野兽的体内。”

“在人类繁衍生息的过程中,先贤们逐步发现了蛊虫的奥妙。已经开辟空窍,运用本身真元来喂养、炼化、操控这些蛊,达到各种目的的人,我们统称为蛊师。”

“而你们在七天前的开窍大典中,都已经成功开辟了空窍,凝聚了真元海,如今已经都是一转蛊师了。”

山寨中的学堂中,学堂家老正侃侃而谈。

在他的对面,端坐着五十七位少年,一个个都聚精会神的听着。

蛊师的神奇和强大,早就深入少年们的内心。因此家老讲述的一切,都深深地吸引着他们。

这时,一位少年举手,得到家老允许后,便站起来发问:“家老大人,我很小时就知道,蛊师有一转,二转等等之分,您能为我们详细讲述一下吗?”

古月师点点头,摆手示意少年坐下:“蛊师一共有九大境界,从下到上,分别是一转、二转、三转直至九转。每一转大境界中又分初阶、中阶、高阶、巅峰四个小境界。你们刚刚成为蛊师,都是一转初阶。”

“若是今后你们努力修行,修为自然就会提高,晋升二转、三转也有可能。当然了,资质越高,晋升的可能性就越大。”

“丁等资质,元海占据空窍两三成,往往最高能修行到一转二转。丙等资质,元海是空窍的四五成,通常会停在二转境界,只有很少一部分能侥幸突破到三转初阶。乙等资质,元海占据整个空窍的六七成,能修到三转,甚至四转。甲等资质,元海充足,是空窍的八九成,这样的人自然天赋最高,最适合蛊师修行,能修行到五转。”

“至于六转向上的蛊师,每一个都是传奇,具体的我也不太清楚。我们古月一族,也没有出现过六转蛊师,但是五转、四转蛊师都有过。”

少年们的耳朵都竖起来,双眼炯炯发亮地听着。

许多人不由自主地看向第一排正襟危坐的古月方正,这可是甲等资质啊,目光中无不充满了羡慕嫉妒的情感。

同时也有一部分目光飘向学堂最后一排的那个角落。

那靠着窗户的角落里,古月方源正趴在桌子上呼呼大睡。

“看,还在睡呢。”有人轻轻地道。

“已经连续睡了一个星期了吧,还没缓过来?”有人撇嘴。

“何止呢,听说他晚上都夜不归宿,在村外周边游荡。”

“有人还不止一次看到,在晚上他抱着个酒坛,烂醉在外面呢。幸好这些年,村子周围已经被肃清,比较安全。”同窗们交头接耳,各种小道消息迅速流传着。

“唉,打击的确太大了。自己顶着天才的名称那么多年,想不到到头来只是个丙等,呵呵。”

“要是这样也就罢了。偏偏自己的那个亲弟弟,被测出了甲等,如今万众瞩目,享受最好的待遇。弟弟在天,哥哥在地呀,啧啧……”

听着耳边越来越大的议论声,学堂家老的眉头已经凝成了一个疙瘩。

整个教室内,少年们无不正襟危坐,焕发着生机活力,因此更显得摊睡在桌上的方源越加醒目刺眼。

“已经一周过去了,还这么颓废。哼,当初也是看走了眼,这样的人怎么可能是个天才!”家老在心中不悦地冷哼。对于这个情况,他已经说过方源很多次了。但是毫无效果,方源仍旧我行我素。每节课都是睡过去的,让负责教学的家老十分头疼和恼火。

“算了,不过是个丙等。连这点打击都承受不住,就这样的心性培养出来,也难堪大用,反而是浪费家族资源。”家老心中对方源十分失望。

方源不过是丙等资质,相比较而言,他弟弟方正拥有甲等资质,这才是值得让家族花大力气培养的对象。

学堂家老一边想着,一边口中又继续刚刚的话题:“在我族的历史上,出现过许多的强者。其中五转强者,就有两位。一位是一代族长,是我们的老祖宗,就是他创立了古月山寨。还有一位,是四代族长。天资卓越,一直修行到了五转蛊师的境界。要不是那个卑鄙无耻的魔头花酒行者偷袭的话,兴许能晋升成六转蛊师也说不定。唉……”

说到这里,古月师深深一叹。

讲台下,少年们都义愤填膺地叫嚷起来。

“都是那花酒行者,太阴险狡诈了!”

“可惜我们四代族长宅心仁厚,英年早逝。”

“只恨我没有早生几百年,否则见到那个魔头,定要拼死揭破他的丑恶嘴脸。”

四代族长和花酒行者的典故,古月族人没有一个不知道的。

花酒行者同样是五转蛊师,是为恶多年的采花大盗,在当时的魔道中赫赫有名。数百年前,他流窜到青茅山,企图在古月山寨中作案,结果被四代族长识破。一场惊天动地的大激战之后,花酒行者被打的跪地求饶,四代族长心慈仁厚,打算饶他一命。结果花酒行者突然发难偷袭,重伤四代族长。

族长大怒,当场击毙了花酒行者,但是随后也重伤不治,撒手人寰。

因此,在所有古月族人的心中,四代族长是为了山寨而牺牲的英雄人物。

“花酒行者么……”被学堂中的声讨吵醒,角落里方源睁开了朦胧的睡眼。

他结结实实地伸了个大懒腰,心中也不无怨念:“这个花酒行者,到底死在哪里?为什么我将山寨周围都转了个遍,还未找到他的遗产?”

记忆中,大约是两个月后,一位因为失恋而醉酒的族内蛊师,烂醉如泥地躺在山寨外,结果四溢的酒香气息,意外地引来了一头酒虫。

蛊师大喜,想要捕捉。酒虫慌忙逃窜,蛊师紧追不舍,顺着酒虫的踪迹,发现了一处隐秘的洞口,通往地下。

酒虫是很珍贵的一种蛊,这蛊师带着酒意,就冒险进入洞口,来到地下秘洞。然后就发现了花酒行者的尸骸,还有留下来的遗产。

蛊师回到山寨后,汇报了所有的发现,立即引起了整个家族的大轰动。

而那蛊师也因此得益,修为越加突出,反而吸引了那个曾经抛弃他的情侣回转了心意。成为一时的风云人物。

“可惜这个消息,我也只听说了大概,并不知道确切的位置。当时也没想到会有重生的这一天啊。花酒行者,你到底死在哪里?”

他这些天来,买了许多酒,一到夜晚就在山寨周围闲逛。想借着散发的酒气,来吸引到酒虫露面。可惜,就是不见那酒虫,结果令人十分失望。

“若是找到那酒虫,炼化为本命蛊,比家族中的月光蛊要好多了。眨眼间,已经到了四月,时不我待呀。”方源叹了一口气,视线转向窗外。

只见蓝天白云下,群山葱茏延绵开去。近处则是一片竹林。

这是青茅山特有的矛竹,各个笔直得像一条直线,同时尖端锋锐异常,如同枪尖。

不远处的树林,已经泛起了新绿。抽出的嫩芽,黄绿一片。不时有漂亮的彩雀儿,落到枝干上。

春风袭来,将青山绿水的清新气息包裹着,吹洒人间。

不知不觉间,这堂课接近了尾声。学堂家老最后通知道:“这一周来,我教会你们如何冥想,察看自身的空窍元海。如何打坐,调动体内的真元。现在是时候炼化你们的本命蛊了。这节课结束后,你们就去学堂里的蛊室,挑选蛊虫。选了蛊虫之后,就回家潜修。直到炼化了蛊虫,再来学堂继续上课。同时,这也是你们的第一场考核。谁能拔得头筹,就会有二十块元石的丰厚奖励。”

喔!

下一刻,整个学堂都欢呼起来。

“终于要炼化蛊虫了,那我该挑选什么蛊虫好呢?”方源双眼中精芒一闪而逝。

------------

\end{this_body}

