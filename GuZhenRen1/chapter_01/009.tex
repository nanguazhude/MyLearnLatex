\newsection{渐行渐远}    %第九节:渐行渐远

\begin{this_body}

红日西沉,但还未完全落下。

天空还有着光,只是所有的事物都像是被蒙上了一层灰色。透过窗户远眺,远处的山,正渐渐地向沉重的黑色靠拢。

客厅内光线暗淡,舅父舅母高座在主位上,面目表情都笼罩着一层阴影,看不大分明。

看到方源随身带来的那两坛酒,舅父古月冻土的眉头拧成了疙瘩,他开口道:“时间一晃,你们已经十五岁了。竟然都有蛊师资质,尤其是方正,舅父舅母都替你们感到骄傲。我给你们每人六块元石,你们兄弟俩拿去。炼化蛊虫,极耗真元,这些元石你们需要。”

说着,就有奴仆过来,交给方源方正两兄弟每人一个小袋子。

方源收起袋子,沉默不语。

方正则立即展开袋口一看,只见里面装着六块椭圆的灰白色元石。顿时脸色涌现出感激之色,从座位上站起来,对舅父舅母行礼道:“谢谢舅父舅母,侄儿正需要元石来补充真元呢。你们把侄儿养这么大,养育之恩侄儿铭记在心,永生不忘!”

舅父笑着点点头。

舅母则连忙摆手,对方正温言道:“快坐下,快坐下。你们兄弟俩虽然不是我们亲生的,但我们一直都把你们当做亲生儿子抚养。你们能有出息,我们也感到骄傲。唉,我们膝下无子,有时候在想你们真能成为我们的孩子,就好了。”

这话说的大有深意,方正没听出来,方源却微微皱起眉头。

果然舅父接着就道:“我和你们舅母商量过,想把你们过继到我们家来,成为真正的一家人。方正,不知道你愿不愿意?”

方正楞了一下,但他很快脸上就涌现出欣喜之色,一口应承下来:“说实在话,自从双亲死后,侄儿就很渴望一家团圆的日子。能和舅父舅母成为一家人,这太好不过了!”

舅母神情一松,笑起来:“那你就是我们的乖儿子了,还叫舅父舅母么?”

“父亲,母亲。”方正恍然,连忙改口。

舅父舅母都哈哈一笑。

“好儿子,不枉费我们夫妻从五岁就抚养你,可养了你整整十年啊。”舅母抹着泪。

舅父则看向沉默不语的方源,温和地说着:“方源,我的意向呢?”

方源摇头不语。

“哥哥。”古月方正想劝,却被舅父阻止。

舅父语气不变,又道:“既然如此,方源侄儿,我们也不会勉强你。只是你已经十五岁了,也该独立门户,这样一来也方便继承你方家支脉。舅父这里为你准备了两百块元石,算是给你的资助。”

“两百块元石!”方正顿时瞪圆了眼睛,他从未见过这么多的元石,不禁流露出羡慕的神情。

哪知方源却仍旧摇头。

方正大惑不解,舅父的面色却微微一变,舅母的脸也阴沉下来。

“舅父舅母,若没其他事情,侄儿就先告辞了。”方源没有给他们再说话的机会,丢下这句话,拎起酒坛,直接就出了厅堂。

方正起身:“父亲,母亲,哥哥是一时想不通,不如让我来劝劝他?”

舅父摆手,故意长叹:“唉,这事不能强求,你有这个心,为父已经很欣慰了。来人,把方正少爷待下去,好生安住着。”

“那儿子告退了。”方正退下,客厅便陷入了沉寂。

太阳彻底落下山去,客厅中越加昏暗。

半晌,昏暗中传来舅父冷冷的声音:“看来方源这个小兔崽子,已经看破了我们的谋算。”

古月一族的族规中,有明文规定:十六岁的长子,有继承家产的资格。

方源的双亲,已经亡故,留下一笔不菲的遗产,都被舅父舅母“保管”着。

这笔遗产的价值,可不是区区两百块元石可比的。

若是方源也像方正一样过继给舅父舅母,那就没有资格继承这笔遗产。若是方源今年十五岁就独立门户,也不符合族中继承家产的规定。

“幸亏啊,我们笼络住了方正,而方源只有丙等资质。”舅父又叹一声,感到庆幸无比。

“那老爷,方源摆明了是要在十六岁独立出去,我们该怎么办呢?”舅母一想到那笔遗产,语气就急了。

“哼,他既然心怀不轨,也就怪不得我们了。只要我们在他独立出去之前,抓住他的大错,将他逐出家门,也就剥夺了他继承遗产的资格。”舅父冷哼道。

“可是方源这小兔崽子,聪明得很,怎么会犯错呢?”舅母不解。

舅父顿时翻了个白眼,低声呵斥:“你真是蠢笨!他不会犯错,难道我们就不能陷害么?就让沈翠那个丫头先去勾引方源,然后再大叫非礼,我们当场人赃俱获,再栽赃他个酒后乱性,丧心病狂的罪名,还怕逐不出方源?”

“老爷还是你有办法,妙计啊!”舅母顿时大喜过望。

浓郁的夜色铺盖下来,漫天的繁星被飘来的阴云遮挡住大半。山寨中各家各户渐渐亮起了灯火。

古月方正被领进一间房内。

“方正少爷,这可是老爷亲自叮嘱老奴,特意为您整理,专门腾出来的房间。”沈嬷嬷殷勤地介绍着,她弓着腰,脸上堆满了谄媚的笑容。

方正环视一周,眼睛发亮。这房间比他原先住的还要大上两倍,中央是宽大的床铺,窗台一侧是檀木书桌,摆着精致的笔墨纸砚,四周墙壁是精美的挂饰。甚至脚下也不是普通的地板,而是覆盖了一层柔软的手工地毯。

从小到大,方正还从未住过这样的房间。当即连连点头:“这很好,真是不错。谢谢沈嬷嬷了。”

这沈嬷嬷是舅母最器重的人,管理着家里上下的奴仆,是名副其实的管家。

方源的贴身丫鬟沈翠,就是她的女儿。

沈嬷嬷呵呵地笑起来:“奴婢哪里敢当得起少爷您的谢,应该的,应该的!少爷您尽管吃好睡好,想要什么就摇摇床边的铃铛,立即就会有下人上来听候吩咐。老爷吩咐了,这些日子少爷您就一门心思的修行,其他的琐事都交给我们下人们办理。”

方正心中再度涌出一股感激之情,他没有再说什么,只在心中默默下定决心:这一次一定要夺得第一,不让舅父舅母失望!

……

天空中的阴云越来越重,夜色也因此越发深沉。夜空中的星辰几乎都被云翳遮蔽,只余下几颗闪着微弱的光芒,在天空中挣扎着。

“舅父舅母应该在合计着,怎么将我逐出家门吧。前世是暗中唆使下人挑衅我,然后栽赃我,最后把我逐出家门,不知道这一世会有什么变化。”方源走在街道上,心中冷笑不止。

对于舅父舅母的真面目,他早就看清了。

不过也能理解。

人为财死鸟为食亡,不管是地球还是这个世界上,总有那么多的人为了利益而践踏亲情、友情、爱情。

事实上,亲情根本就没有。当初舅父舅母收养方源方正,根本目的就是贪图遗产。只是方源方正两兄弟频频让他们意外。

“万事开头难,对我而言,更是如此。我一没有过人资质,二没有师长关照,等于是白手起家。双亲的遗产,可以说是我的一个大跳板。前世遗产被舅父舅母夺去,害得自己整整耗费了两年,才修行到一转巅峰。这一世,这个错误不能再犯了。”

方源就这样一边走,一边思考着。

他没有在居所待着,而是提着两坛酒,方向直指寨外。

夜空越来越阴沉,乌云遮蔽了星光,山风呼呼的吹着,有渐渐增强的趋势。

山雨欲来啊。

不过还是要探索。双亲遗产要夺回来,那也得等到他明年十六岁。而花酒行者遗藏,才是近期就可能得手的东西。

街道上,行人很少。路边房屋中透出昏暗的光,一些琐碎的生活垃圾,以及树叶尘土,被风卷吹,随意飘零。

方源单薄的衣服,有些挡不住这山风,不由地感到一阵冷意。

他索性将拎着的酒坛打开,小小的喝了一口。虽是浊酒,但是咽下去后,就有一股暖意升腾上来。

这还是他这些天,第一次真的饮酒。

越要出山寨,路边的房屋就越稀疏,灯火就越昏暗。

前方,更是黑暗重重。风吹压着山林,夜色中树枝摇曳,呼呼作响,像是群兽在咆哮。

方源的步伐没有半点迟疑,出了山寨大门,在黑暗的路中渐行渐远。

而在他的背后,是明媚辉煌的万家灯火。

在这灯火中,有个温暖的角落。

弟弟古月方正坐在书桌前,温习着课上记下的笔记。房屋中灯火明亮,坚实的墙壁阻挡了冷风,在他的手边摆着一杯温热的参茶,热气袅袅地升腾着。

“方正少爷,洗澡的热水已经为您准备好了。”门外,沈翠的声音轻轻传来。

方正心中一动:“那就拿进来吧。”

沈翠带着一脸的媚意,扭着腰走进了房间。

“奴婢见过方正少爷。”她满眼秋波地向方正望过去。方源只是个丙等,方正可是甲等资质。能攀上他,才是真正的大富贵!

\end{this_body}

