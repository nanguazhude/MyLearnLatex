\newsection{草绳上的人生}    %第六十一节:草绳上的人生

\begin{this_body}



%1
耀眼的朝阳,照亮了青茅山。

%2
学堂中,家老在详细讲述着要点:“明天,我们就要选择第二只蛊虫进行炼化。大家都有炼化蛊虫的成功经验,这一次可以进行巩固。对于第二头蛊虫的选择,大家要仔细思考。结合这些天的修行心得,以及对自身的了解,进行综合考虑。一般而言,最好是能和本命蛊搭配使用。”

%3
蛊师的第一只蛊虫,称之为本命蛊,一经选择,无疑就确立了发展的基石。随后的第二头蛊虫,第三头蛊虫等等,都是在这基石上确定蛊师修行的具体方向。

%4
听了学堂家老的话,少年们都不由地沉思起来。唯有方源一人趴在桌上呼呼大睡。

%5
他昨天辛苦了半夜,回到宿舍后,仍旧进行蛊师修行,温养空窍。天刚亮,这才开始入眠。

%6
学堂家老扫了方源一眼,微微皱起眉头,却没有多说什么。

%7
自从族长古月博对他讲了那番话后,他就对方源采取了听之任之,放任不管的态度。

%8
“我该选择什么蛊虫呢?”许多学员思考的时候,都不由自主地看向方源。

%9
“说起来,方源早已经有了第二只蛊虫了。”

%10
“是啊,那可是酒虫啊,居然解石能解出酒虫,这运气真是太好了!”

%11
“要是我有酒虫,也能率先晋升中阶吧?”

%12
学员们心中的想法此起彼伏,羡慕者有之,嫉妒者更有之。

%13
自从那天,通过审讯之后,方源的酒虫也顺利曝光了。酒虫的来历,没有引起怀疑。族人们释然的同时,也对方源的运气很是感慨。

%14
“我怎么就没有这么好的运气呢,唉!”实质同为丙等资质的古月赤城,心中叹息着。

%15
很早之前,他的爷爷就四处托人,为他采购酒虫。没有想到,他身为一个分脉的继承人都没有得到,古月方源却优先得到了。

%16
相比较赤城的哀叹嫉妒,同为副班头的方正却精神奕奕。

%17
“哥哥,我一定会超越你的。”他看了一眼方源,在心中说了一句,就收回了视线。

%18
这些天他的眼睛都闪着光,对生活充满了一种澎湃的激情。他的脸颊红润,额头泛光,甚至走路的步伐都是轻快的。

%19
学堂家老把这些看在眼里,立即明白这是古月族长已经开始暗中教导方正。

%20
这种开小灶的事情,当然不能明说。

%21
学堂家老对此睁一只眼闭一只眼。

%22
又到了晚上。

%23
方源再次挤进了石缝秘洞。

%24
叮铃铃……

%25
在他的手中,一只野兔剧烈地挣扎着,在野兔的脖子上系着一只铃铛。

%26
这是方源在山上捕捉的野兔,铃铛自然也是他系上去的。

%27
经过一天,秘洞中的闷气已经完全散去,空气很清新。

%28
甬道的洞口敞开着,里面静寂无声。方源半蹲在地上,先是仔细地查看了一下地面。他昨晚在两处地面都洒了石粉,这层薄薄的石粉,并不惹人瞩目。

%29
“甬道入口处的石粉,保持原状,看来我离开的这段时间,甬道中没有爬出什么不干净的东西。秘洞的石缝入口处,倒是有一个脚印,但这是我刚刚踩上去的。可见并没有外人来过这里。”方源观察了一下,就放下心来。

%30
他站起身,伸手用力,将墙壁上的枯藤死蔓扯下一把来,

%31
然后他坐在地上,用腿弯将野兔压制住,空出双掌搓动这些藤蔓。

%32
这活计一般的蛊师都不会,但方源有着太多丰富的人生经验,前世有好几次穷困潦倒的时候,连蛊虫都喂养不起,纷纷饿死。

%33
有一段时间,他空有真元,毫无蛊虫,跟凡人一般,连生活都困难。万般无奈之下,就学会了搓草绳编织草鞋、草帽等等贩卖,换取一些元石碎屑勉强糊口。

%34
此时搓起草绳,方源心中的记忆又浮现出来。

%35
那时的苦涩和煎熬,化为了此刻他嘴角无声的笑。腿弯下野兔不时地挣扎,铃铛叮啷作响。

%36
一双两好缠绵久,万转千回缱绻多。

%37
细细的,慢慢地,经年累月,把岁月汇聚在一起,有曲折,有翻搓,有纠缠。

%38
搓草绳,不就是经历人生吗?

%39
秘洞中,赤光晦暗,年轻和沧桑交错在方源的脸上。

%40
时间也仿佛在此驻足,静静地欣赏着少年搓草绳。

%41
叮铃铃……

%42
半个时辰之后,野兔快速地窜进了甬道之中,脖子上的铃铛叮当作响,几个呼吸就出了方源的视野。

%43
方源手中临时编制的草绳,一端系在野兔的后腿上,此刻被野兔拖拽着,急速地向外游走。

%44
过了一会儿,草绳停止了游动。

%45
但这并不意味着野兔到达了甬道的尽头,有可能被陷阱所杀,有可能只是中途驻足。

%46
方源开始往回收绳,绳子渐渐收紧,他用力一拽。

%47
草绳那端,立即传来一股力量。接着草绳又接着向外游走。

%48
显然是那边的野兔,忽然受到拖拽的力量,在惊惶之下,又开始向里面急窜。

%49
如此三番五次,野兔似乎终于走到了甬道的尽头,不管方源再拽草绳,草绳也只是松了又紧,紧了又松。

%50
也许是野兔窜到了甬道的尽头,也有可能是野兔落到了什么陷阱机关当中,被困住了。

%51
要验证这当中的答案,也十分简单。

%52
方源开始收绳,他的力量野兔子哪里抵得过,最终他用草绳把野兔硬生生地拽出来。

%53
野兔在草绳那端不断地奋力挣扎,但是草绳取材酒囊花蛊和饭袋草蛊,虽然枯死多年,但仍旧坚韧,不是寻常的稻草可比。

%54
野兔再次活蹦乱跳地落到方源的手中,方源细细地检查了一番野兔,见它的身上并没有伤口,这才吐出一口浊气。

%55
“目前看来,这段甬道应该是安全的。”

%56
得到这个结果,野兔也就失去了作用,方源一把捏死,随手抛尸在地上。

%57
不能放生这野兔,动物也是有记忆的,万一它再回来这里,像是酒虫一样,引来外人,那就糟糕了。

%58
他深吸一口气,经过几番试探,他这才小心翼翼地踏入甬道。

%59
尽管有野兔探路,但是很多陷阱机关,专门针对人类。野兔这种小巧的动物,反而触发不了。因此方源不得不防。

%60
甬道呈直线,斜向着地底延伸出去。并且越往下,甬道就越宽敞。

%61
方源刚刚进入甬道,得弯腰低头,走出五十多步之后,就能昂首挺胸。再走到一百步左右,能甩开膀子左右挥舞。

%62
甬道其实并不长,只有三百米左右,但是方源却足足花费了一个时辰,才探到了甬道的尽头。

%63
这一路上,他都是小心谨慎,一步一探。走到尽头时,他已经累得浑身大汗。

%64
“没有用来侦测的蛊虫,就是麻烦。”方源擦了擦额头的冷汗,确认安全之后,这才定下心来,仔细打量甬道尽头。

%65
这一观察,他就楞住了。

%66
甬道尽头处,堵着一块巨大的石头。石头表面光滑,向方源的方向凸出来,如同贾富那圆滚滚的肚皮。

%67
就是这块巨石,挡住了方源继续前进的脚步。

%68
除了这块巨石之外,方源身边空无一物。

%69
“难道因为意外,甬道中段塌方,导致了堵塞?”方源目光一凝,这是很有可能的事情。

%70
花酒行者在临死之前,急切间创立这个力量传承。他利用千里地狼蛛,仓促地做出了一条山体甬道。甬道通往山体深处,引导着继承者前行。

%71
然而数百年过去了,这条甬道经不起时间的侵蚀,在某一刻,其中的一段甬道年久失修,塌方了。

%72
生命中总会有各种意外发生。

%73
“若真是这样,我岂不是要止步于此了?”他走上前摸了摸石头,这阻挡住他前进的巨石,单单露出的面积,就和门一样大。可以想象它整个形体的厚度。

%74
方源可以用月光蛊磨掉石壁,但是要磨掉这样的巨石,非得有一两年的苦功不可。

%75
“看来必须动用工具,利用铁镐和铁锹,将巨石破开。只是这样一来,难免就暴露了许多痕迹。一些敲打的声音,也会传出去。”想到这里,方源深深地皱起眉头。他在考虑其中的风险和收益。

%76
若是风险太大,他宁愿放弃这个力量传承。

%77
毕竟若是被其他人发现这里的秘密,他先前做出一番布局和苦功,都要破费不说,自己都可能有生命的危险!

\end{this_body}

