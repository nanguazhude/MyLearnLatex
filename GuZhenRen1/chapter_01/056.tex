\newsection{洗尽嫌疑}    %第五十六节:洗尽嫌疑

\begin{this_body}

“开出来的?”古月博深深地皱起眉头。

“不错,事到如今,我也不隐瞒了!”方源像是豁出去了,语气急促,“我在赌石场买了六块紫金石,因为真元不够,当场只解开了五块。还剩下一块,带到宿舍,解开之后,发现是只酒虫。我大喜过望,因为之前查过资料,知道酒虫是能弥补我资质短板的珍稀蛊虫。于是就立即炼化了它。”

“等等,你说你赌石,买了六块。一块赌出了癞土蛤蟆,另一块还开出了酒虫?”家老中一人听到这里,忍耐不住,带着难以置信的语气,反问道。

“这又怎么了?”方源理所当然地反问了一句,然后一指女蛊师,大声道,“她可以给我作证!”

场中众人都是一愣,齐齐看向女蛊师。

女蛊师感受到目光中传来的压迫力,她不敢撒谎,便实话实说:“的确是这样,方源买了六块,然后第五块开出了癞土蛤蟆。第六块也的确带走了,不过开出什么我就不知道了。”

“就买了六块紫金石,连续开出两只蛊虫,这运气也太好了吧?”家老中不少人嘀咕起来。

“这有什么不可理解的?运气嘛,谁也说不好。嘿,想当年我赌石那会……”

“等一等,蛊虫难炼,怎么听方源这话,一下子就把蛊虫炼化了?”有家老疑惑地道。

“你是老糊涂了吧。不知道蛊虫解开来,都是极其虚弱,濒临死亡的么?就算是越级炼化也有可能。他炼化了酒虫有什么稀奇的?”身旁立即就有人答道。

方源又继续说道:“我炼了酒虫,第二天就又到商铺去。的确是在中午到了一家酒铺,买了一杯猴儿酒。晚上时分又去,正巧看到一场欺诈纠纷,贾金生将臭屁肥虫冒充成黒豕蛊,卖给了我族的一名蛊师。后来贾富大人出现,解决了这个纠纷。”

“我再到酒铺去,没有想到正好碰上贾金生在那里喝闷酒。我刚刚得了酒虫,心中欢喜的不得了,就想问问看这酒虫能卖多少元石。哪知贾金生得知我有酒虫,就想强买。我当然不愿意了,我根本就不想卖酒虫的,只是想明确酒虫的价值而已。要卖也至少得到我二转之后,所以当场我就走了。”

方源这一席话,将贾富和贾金生的矛盾公布了出来,这让家老们看向贾富的目光,都变得有些意味深长。

在这些目光的压力下,贾富咳嗽一声,双眼精芒一阵闪烁,问向方源:“那我弟弟贾金生,之后有没有再去追你?”

方源点点头,半真半假地道:“他不仅追了过来,还加了五十块元石。但我根本就不想卖,他很愤怒,扬言说古月一族算什么,叫我今后小心一些。说完狠话,他就走了。我就再也没有见过他了。”

贾富暗暗点头,以他对贾金生的性格了解,肯定会追上去。放狠话也是贾金生的一贯作风。

若是方源说贾金生没有追出来,他就料到这必是假话。

但方源既然如此说了,这就让贾富有些为难。他调查的结果,就止步在此。贾金生是不是真的就没有再找方源?也许贾金生后来又找到了方源,双方谈不拢,结果被方源所杀——这完全是有可能的。

“说,贾金生是不是你杀的!”想到这里,贾富厉色逼问,企图以气势压迫方源。

方源则矢口否认,一口咬定从此就再也没有见过贾金生。

贾富再没有其他的证据,一方逼问,一方否定。事情到了这里,便陷入了僵局。

古月博听着听着,脸色有些不快了,这个贾金生居然敢在青茅山,如此威胁古月一族的人。这明显是不把古月一族放在眼里!现在贾富又当着古月高层的面,如此逼问古月族人。要是有确凿证据也就罢了,现在明显是没有关键证据,这事要传出去,自己的脸面往哪里搁?

“贾老弟,不是老哥多嘴啊。”族长打断了贾富的逼问,道,“贾金生失踪这么多天,恐怕已经凶多吉少了。凶手造成这场血案,那必定就会有蛛丝马迹。不知道老弟你还查到什么没有?”

贾富狠狠地瞪了方源一眼,仰头长叹一声:“老哥的话,我又岂会不知!若是有蛛丝马迹,贾某人就不会到老哥你这儿对质来了。那凶手显然是个惯犯,手段毒辣又周密。不瞒老哥,所有线索都断了,我们离去那天,又下着大雨,就算是有血腥气也被洗刷了。”

古月博淡淡一笑:“贾老弟,我听说你们贾家有一种追踪蛊虫冥路蝶,能散发魂香,种在蛊虫上。此香无色无味,历久弥新。你们贾家族人的蛊虫上,都沾染了一丝魂香。只要利用冥路蝶,循着这缕魂香,就能找到蛊虫,从而找到族人。”

贾富脸色阴沉:“冥路蝶我早就用过了,根本没有效果。想必古月老哥也听说过,只要蛊虫一死,魂香就散了。显然那凶手已经把我弟弟身上的蛊虫,都一一灭杀了!”

古月博话锋顿时一转:“这就奇了。那凶手害了你的弟弟,一不要他的蛊虫,二没有递来绑票,勒索元石。那凶手杀他一个小小的一转蛊师,是图什么呢?”

是啊,图什么呢?

不管贾金生有没有死,害他的凶手总得有动机吧。

一不为蛊虫,二不为元石,难道是情杀?

但若是情杀,总得有个时间的积累过程,他贾富就不应该找上门来。商队中人和贾金生朝夕相处,才更有嫌疑啊。

一时间,议事堂中陷入了沉默。

方源不着痕迹地扫视众人一眼,忽然对贾富道:“也许贾金生就是你干掉的呢。我早就听说,你们贾家要分家产,死了一个兄弟,你分到的家产不就多了吗?”

“住嘴!”

“空口无凭,不得随意指责贾富大人。”

立即就有家老低喝出声。

方源立即住口不说,他目光隐晦地闪了闪,其实他已经达到目的了。

他刚刚的一句话,就像是一个小石子,投在家老们思维的湖泊当中,荡漾起一圈圈的涟漪。

家老们顺着这层涟漪,思维不由地发散开来:“贾富是不可能杀死贾金生的,这对他来讲,损失要大于收益。等等,他不做,未必其他人不会做……”

“贾家内斗!”不知是哪位家老灵光一闪,轻声地道。

他声音不大,但是在寂静的大堂中却很是清晰。

一时间,众多家老的目光都骤亮起来。

“终于想到这方面了。”方源撇撇嘴,眼帘垂下,掩盖住眼里的一抹冷光。

贾家族长要分家产,传族长之位,因此几个儿女都展开了激烈的竞争。尤其是贾富和贾贵两位,皆是四转蛊师修为,身边都拥有一批拥护者。

这些年,贾家这些情报许多山寨都多少知道一些。

贾金生遇害,这事情太蹊跷了。目前没有任何直接的证据,证明方源就是杀人凶手。单单为了酒虫杀人,动机明显不足。同时凶手手段也不会这么机密严谨。

但如果是贾贵暗中出手,那就可以解释了。

在场的几乎都是高层人物,但凡身居高位者,必有过人之处。至少对于政治阴谋,有这敏锐的嗅觉和洞察力。

贾家内斗这四个字,无疑给联想丰富的众人插上了一对想象的翅膀。

贾家族长安排贾金生加入商队,其中一个用意就是为了考察贾富的性情,是否仁爱厚道,只打压而不欺压兄弟。

贾金生出事了,贾富也会受到牵连,真正受益者是谁?

明显是和贾富的最大竞争对手——贾贵!

在加上凶手行事如此老辣,所有的线索都几乎被掐断了,可见凶手经验之丰富。怎么可能是方源这个十五岁的少年做的呢?

所以一切的答案,就呼之欲出了!

议事堂仍旧被沉默笼罩着,但是在场的家老都相互交换着饱含深意的眼神。

“让人从内心深处相信某个可能,不是靠说服,而是引导啊。”方源敏锐地察觉到这些眼神,心中冷笑一声,脸上则仍旧呈现出一副不甘忍受冤屈的倔强神色。

贾富的面色阴沉得能滴下水来。

“贾家内斗”四字一出,他瞬间就想到了贾贵。

在那个刹那,他的整个灵魂都开始颤抖!

还有谁,比贾贵更有动手的可能?

没有了!

“我看明白了。一切都明白了。”学堂家老站在方源的身后,看着方源,眼中精芒一闪,“方源既幸运又倒霉,碰巧在最后的时间遇到了贾金生。就凭他还刚刚上学堂的年纪,怎么可能把线索都掐灭掉?若是他有这份深沉周密的心机,怎么可能表现的如此桀骜倔强呢。他刚刚矢口否认,无非是想掩藏酒虫的存在罢了。”

一时间,所有人对方源的怀疑都已尽去!

“洗净嫌疑只是第二步,下面才是关键的时刻了。”方源运筹帷幄,事态发展皆在胸中,不出所料。他在心中一叹,看向贾富。

贾富也望着他,眼中的不善已经越来越明显。

------------

\end{this_body}

