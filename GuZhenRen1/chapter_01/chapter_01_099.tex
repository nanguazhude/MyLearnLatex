\newsection{是人不是神}    %第九十九节:是人不是神

\begin{this_body}

%1
哗哗哗……

%2
真元海中波涛起伏,潮起潮落。

%3
放眼纵览,一片淡红之色,已经不是原先一转的青铜海,而是赤铁海。

%4
空窍四壁是一片光明的薄膜,正是二转初阶的气象。

%5
整个赤铁真元海,占据空窍的四成四。海上半空处,春秋蝉显现出身形来。

%6
通过近一年的休养沉眠,它已经有了起色。

%7
原先它的躯干没有一点的光泽,粗糙暗哑,如同枯朽的木柴。如今却是沾了一点油光似的。

%8
原本它的双翅,宛若在秋风中凋零漂落的枯叶,边角亦多有残缺。如今却是微微泛出点绿色的新意,翅膀边角也像是描了一道黑线,有了个完整的弧线,不再有缺口。

%9
“春秋蝉,春秋……原来如此,它要恢复,就得历经春天和秋天。重生以来,过去一年,也就是一轮春秋,所以有了恢复。”

%10
方源盯着这只春秋蝉,心机不禁萌动,对春秋蝉的理解又加深了一层。

%11
蛊师炼蛊、养蛊、用蛊,其中的“用”,分门别类,包含万千。方源和春秋蝉朝夕相处,对春秋蝉的理解在慢慢地积累中,不断地加深。

%12
“不过春秋蝉仍旧还很虚弱,只是从濒死的悬崖上往前爬了几步远。能利用的就是它的气息,威压蛊虫,增强单炼之效果。至于合炼蛊虫,它是帮不上忙的。”

%13
能提高合炼成功率的,另有其他神奇蛊虫。术业有专攻,春秋蝉只是具有重生之能。

%14
除了春秋蝉之外,在海面上,肥嘟嘟的酒虫缩成一个圆团,在海水中飘动,玩得不亦乐乎。

%15
如同一只瓢虫般的白豕蛊,和翠玉样儿的玉皮蛊相互环绕着飞行。

%16
方源缓缓地睁开双眼,慢慢摊开自己的右手手掌。手掌中央,是一个月牙和两颗五角晨星的印记。

%17
正是两只小光蛊和月光蛊,栖息寄居着。

%18
方源盘坐在床榻上,视线挪向床榻上。

%19
床榻上放着三只钱袋子,两只鼓鼓的,另一只瘪了大半。除此之外,还有一只野猪王的雪白獠牙,好像是一颗象牙,静静地挨着方源的腿,躺在床单上。

%20
病蛇小组费尽全力,杀了野猪王之后,却遇到了电狼群的袭击,野猪王的大部分皮肉都被电狼啃噬掉了,只余下两颗雪白獠牙是最有价值的战利品。

%21
按照族中的规矩,方源作为杀死野猪王的成员之一,免费获得了一颗野猪王的雪獠牙。

%22
方源望着面前的这些东西,神情带着微微的凝重:“我的元石已经不多了,只够一次合炼的消耗。这次合炼之后,不管成功与否,我的经济都将崩溃。但是如果我不及时合炼的话,过了不了十几天,我的元石不断消耗,连合炼的机会都将丧失。”

%23
方源养了七只蛊虫,经济负担有些大。又因为丙等资质,为了追求快速修行,而屡次动用酒虫。因此元石的消耗比一般人都要高。

%24
最近,他也不再用元石来快速补充空窍的真元了。他现在体内的赤铁海,都是他自我恢复的成果。

%25
方源早就开始掐着元石,计算着生活,元石不能随意滥用。

%26
现在的情况,就相当于他已经快要跌入悬崖,就靠手中抓着的一丛野草,使得自己的身躯暂时保持在悬崖边上。

%27
但是随着时间的推移,他手中的这丛野草也不断地崩断。如果不做任何冒险的努力,他用不了多久,就要跌入悬崖。

%28
他现在要做的,就是趁着手中还有这野草,借助这野草,奋力攀上悬崖。

%29
如果他成功,那他就能趁势获得家产,登上另一个台阶,一切都有新气象。

%30
如果他失败,那他就跌落下去,想要再攀爬到这个地步,就得重新耗费大量的时间和精力。

%31
“不管怎么样,开始吧。”方源深吸一口气,眼神一定。

%32
白豕蛊、玉皮蛊!

%33
两只蛊虫随着他的心念,从空窍当中钻了出来,悬停在方源的面前。

%34
“合!”方源心中暗喝一声,白豕蛊和玉皮蛊便骤然爆发出一阵刺眼的白光,相互直直地撞在了一起。

%35
这一撞,无声无息,却撞出了一片光团。

%36
白色的光,比先前更耀眼。

%37
这预示着方源的两股意识,正在彼此融合。

%38
方源一边用自身的意识维持着白色光团,另一边则从袋子中取出元石,一块块地扔入光团。

%39
光团吞没了元石,只剩下一蓬蓬的石粉,洒落在床榻上。每吞没一块,光团的边缘就扩张一点距离。

%40
光团吸收了天然真元,越来越大。

%41
渐渐地,它从原先的脸盆大小,变为了石磨大小。

%42
“差不多了。”方源眯着眼睛,将那颗野猪王的雪獠牙抓在手中,然后果断地抛入光团当中。

%43
若是世人目睹这一幕,恐怕都得惊叫起来。因为白豕蛊搭配玉皮蛊,合炼成白玉蛊的秘方,已经众所周知,流传了上千年,但从未有人听过,还要添加野猪王的雪獠牙。

%44
然而,过去没有,并不代表将来没有。

%45
在此后一百五十年后,有一位蛊师改善了这个秘方,他发现只要添加一颗野猪牙,就能大大地增添合炼的成功率。

%46
方源重生五百年前,自然知晓了这个窍门。

%47
雪獠牙一投入光团当中,顿时发生了奇妙的变化。

%48
原本刺眼的白光,在瞬间变得柔和起来。原先只是一味的向四周威射,但是现在却有一种波光流转,明暗转换的自然妙韵。

%49
在方源的注视下,光团缓缓地收缩,最终彻底消散在空气中。

%50
玉皮蛊和白豕蛊已经都不见了,一只全新的蛊虫,静静地悬浮在方源的面前。

%51
它就像是一颗椭圆的鹅卵石,通体俱白,但这种白不是宣纸的苍白,也不是牛奶的乳白,而是一种润白,透着玉的光泽。

%52
这就是——二转的白玉蛊!

%53
至此,方源才轻吐出一口浊气,心石落地。

%54
别看这过程,似乎很简单。其实不然。

%55
第一,意识融合,就得做到一心多用。

%56
寻常人一手画圈,一手画方,这就是一心两用。很多人都难以做到,更何况更高难度的一心多用?

%57
非得是无数的积累,刻苦的修行,承受无数次的失败和挫折,还得有一定的天份,才能做到一心多用。

%58
方源能做到驾轻就熟,都是五百年的浓重积淀,一丝一毫都造不得假。

%59
第二,就是对蛊虫的理解和认知。

%60
蛊师对蛊虫的理解越深刻,合炼的成功性就越大。

%61
这点,也是在将来,大约三百年后被广泛认同的真知灼见。

%62
所以,往往使用越久的蛊虫,合炼起来的成功率就越大。

%63
第三,是正确而且独到的秘方。

%64
就例如这次,雪獠牙的添加,如同画龙点睛,能将成功率骤然提升两成。效果着实非凡。

%65
一些秘方在这个世界上广泛流传,但也有许多秘方,被人珍藏着,并不流传。

%66
就比如古月山寨的月光蛊,炼制它的秘方,就只掌握在少数的几位家老,以及历代的族长手中。

%67
尤其是那些能五转,和五转以上的秘方,都是如此。被蛊师们秘藏着,珍惜若命。很多蛊师不到将死之时,是绝不会传授出去的。

%68
但就算是有了这三点,也无法做到绝对的成功。哪怕是方源这样的人物,有着前世五百年记忆,经验丰富至极,对蛊虫理解深刻,能一心多用,知晓许多的秘方。但他合炼蛊虫,仍旧有失败的可能。

%69
只能说,他的失败率比较低罢了。

%70
蛊虫合炼,可以说是生命的一种升华,一种创造。它把时间浓缩到极限,让漫长的进化过程,在瞬间开花结果。

%71
在地球上,能做到这点的,只有神明。

%72
毫无疑问,这是一场生命的奇迹。蛊师以凡人展示神迹,怎么可能次次成功?

%73
若是每次都成功,那就不是人,而是神了。

\end{this_body}
\newsectionindepend{写在上架前——不会让大家失望!}
\begin{this_body}
%74
写这本书之前,就知道会有争论。写了之后,果然争论很多。

%75
其实魔道文,比正派文难写多了。正派主角的行径、风格,大家都有一致的认知,因为这些道德标准,早已经被社会宣扬了无数年。但是魔的行为、风格,每个人都有每个人的观点和想法,没有形成过统一。

%76
书评区中,大多数的评论,都在争论着这个问题。

%77
可以说,写魔道文,是一件吃力不讨好的事情。

%78
但是我仍旧要写,原因在本书的《序》中已经详细说明了。

%79
不像其他的许多网文,在纯粹地讲故事。这本书寄托了我很多的东西。打个比方,那些网文如同一张白纸,可以承载读者的很多东西。这本书呢,已经在白纸上画了些浓墨重彩的画,带着强烈的个人风格。

%80
大家能看到这里,又保持着喜欢,那真的算是同道中人了。

%81
很感谢大家的认可和支持,下面本书将要上架,也希望大家伙仍旧支持我。

%82
接下来,有几个问题说明一下,方便大家今后的阅读。

%83
1求月初的月票。

%84
呵呵,写书的目的之一就是想赚些生活费,毕竟我也要生活。

%85
恳请大家支持我!你们的月票是我的动力之源,热情之源,也能让我看清楚自己的定位,和这本书的市场价值,拜托诸位了。

%86
上架当日,自然会爆更新,回报大家的支持!

%87
2求首订、全订。

%88
首订是很重要的数据,关系着接下来本书的发展。

%89
所以请大家,能多多订阅。尤其是首章的订阅,谢谢。

%90
3更新。

%91
目前更新情况,我还是很满意的。从发书开始,都是稳定更新,一天至少两章,这点大家有目共睹。期间虽然出了些意外情况,但那都是起点后台的定时更新的功能出错。

%92
接下来vip章节的更新,一天至少两章更新,然后爆更新。推荐好的话,会一天三更。求月票的话,也会适时爆更新。

%93
当然命运无常,若是遇到突发情况,我会及时请假。

%94
但更新情况,一定会比上一本要好很多很多的。诸位放心,努力做到更新不断。

%95
4完本。

%96
这本书绝对会完本。

%97
我写的书,都是完本,从来没有太监的情况。

%98
退一万步讲,就算是只有一个人订阅,全部骂声一片,我还是会照样写的。

%99
至少写出一个结局出来。

%100
他们骂就骂呗,我写我的。有时候,坚持就是一种成功。

%101
这就是我的风格,对得起自己,对得起支持我的人。

%102
5成绩。

%103
本书成绩好不好,一靠我写,二靠大家的鼎力支持。

%104
成绩能争就争,就比方三江的第一,诸位牛刀小试,就争来了,轻轻松松的。

%105
成绩越好,作为写手而言,自然热情更高,状态更好,更新更多。

%106
上月票榜前十,会爆更新的。

%107
6qq书友群。

%108
目前三个老群,都在清人。

%109
有些人一进群,从未说过话,那实在有些过分了。qq群是为了方便交流用的,鼓励大家讨论书。不要乱发广告、诅咒图等等,会被踢的。

%110
还有一个vip群,群号是250906315。是需要全部订阅的哦,亲们。进群之后,请大家报一下起点id,并截图验证。

%111
三个老群都是普通群,清人之后,会通知大家群号。

%112
7致谢。

%113
新书能走到现在,说实话,成绩比我想象中的好。这都是诸位读者朋友们的鼎力支持!

%114
感谢新老朋友们对蛊真人的厚爱。

%115
本书会越写越顺,越写越爽,越写越好。

%116
因为我本人的手已经开始热了,已经渐渐地脱离以往的藩篱和框架,而且我越写越爽。甚至可以说,曾经的感动,我已经找到了一部分。

%117
本书也会越来越精彩。

%118
春秋蝉的真正作用,一旦发挥出来,相信诸位一定会看得很爽的。嘿嘿……

%119
随着方源的实力成长,他将逐步展现出魔头的真正风采。现在的他,不过是在青茅山这个牢笼里,将魔瞳慢慢地睁开了一丝,露出了些许寒芒罢了。

%120
相信不会让大家失望!

\end{this_body}

