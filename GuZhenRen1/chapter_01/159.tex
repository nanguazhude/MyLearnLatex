\newsection{天元宝莲}    %第一百六十二节:天元宝莲

\begin{this_body}

虽是打算探查花酒遗藏,但方源一直脱不得身,他成为家老风头正劲,狼潮又频繁活动,许多事情委派过来,让他没有机会。

真正再到石缝秘洞中来时,已经是十几天之后。

夏末夜晚。

小雨刚停,带来秋意。

天空中,一轮黄金之月,圆满如盘,高高悬挂着。

耳边隐约的狼嗥声,和残留的蝉鸣相互交融。方源隐着身站在山坡上,回头一望。

古月山寨中亮着无数灯火,残破的寨墙补了又补,早已经失去了往昔的和平和安宁的气息,仿佛是一头历经无数大战的巨兽,趴在地上喘息着。

“重生之后,就连狼潮的进程都改变了许多。记忆当中,雷冠头狼早在三天前就到了。如今却是不见影踪。”

方源看了一眼,就转过视线。今夜是他好不容易争取出来的时间,得好好利用。

片刻之后,他再入石缝秘洞。

洞口处故意撒下的灰尘一片,并未有脚印出现,可见这处还未被发现。

这种检测的小手段,似乎上不了台面,但在方源的体会里,却出奇的好用。

当然,他不仅仅只设下这一个手段,几个检查下来,他确认这秘洞暂时仍旧是安全的。

他轻轻地松了一口气,自己重生以来,改变的东西越来越多。尤其是狼潮之下,蛊师出动频繁。说不定就会被人发现这里。

他走入甬道,进入第二密室,推开石门,来到山体石林。

石林中,曾经打通的路线上,又迁徙了不少的玉眼石猴群。

不过如今方源,已经是三转蛊师。血月蛊虽然在三转中,攻击力并不优秀。但绝对远远比二转的月芒蛊好多了。

方源花了三个时辰,覆灭近十支猴群,重新打开通路。

他来到最中央,踏着粗糙的石阶,深入到第三密室。石门挡在他的面前,石门上刻着――“金蜈洞中杀身祸,可用地听避凶灾。”上一次他就止步于此。

但这次。他毅然推开石门,迈入进入其中。

他手持火把。照亮周围十步之远。

这金蜈洞宽敞。主道高有至少三米,宽两米。还有许多略显狭窄的支道岔开,四通八达。

方源所到之处,火光照亮,黑暗消退。起初洞中只回荡着他的脚步声,但不久后悉悉索索的声音,就从四面八方涌来。

声音交汇一体。连绵不绝。火光的边缘,很快就涌现出无数只蜈蚣。

它们凶性十足。只是一时碍于明亮的火光,没有向方源发动进攻。但是方源知道。随着时间的推移,蜈蚣越来越多,后面挤前面,这种僵持的局面很快就会被打破。

但他并不在意。

若是二转时期,只有白玉蛊防御,他绝不会故意造成这样的动静,引来蜈蚣群的躁动。但如今他已经晋升三转,天蓬蛊的防御力,已经足够他支撑群虫噬咬,唯一要顾及的只有这里的虫王――锯齿金蜈。

它已经出现了!

方源故意从空窍中调动出一丝白银真元,流出体外,泄露出他三转蛊师的气息。

这样的气息,让锯齿金蜈感到了强烈的威胁。对于它来讲,方源这个踏入它领地的强大“野兽”,必须要它来第一时间进行绞杀。

方源和它对峙着。

这锯齿金蜈,长达一米多,身躯有双拳宽。起先落在火光照明出的范围边缘,盘曲着身躯,仿佛是一头潜伏在阴影中伺机而动的蟒。

但旋即,它缓缓动了,无数的节足支撑着身躯,向方源渐渐逼压而来。

方源的三转气息,令其警觉,却不会忌惮。若是四转,恐怕就不会主动逼压了。若是五转,只要气息稍稍一流露,它必然转身就逃。

方源高举着火把,火把上火焰燃烧着,照着周围的洞窟光影波动。

在火光中,锯齿金蜈暗金的甲壳,泛着幽光。它的身躯两旁,长满了银色的锯齿。随着它慢慢压来,锯齿也在缓缓转动,仿佛是放缓了的电锯一样,发出嗡嗡的声音。

其他的小蜈蚣,则从地上,墙壁上,向着方源围拢过来。

一些蜈蚣攀上洞顶,然后掉下来,落在方源的肩头、背上。

方源浑不在意,他撑起天蓬蛊,浑身上下浮现出一层厚实的白晶之光,隐约可见铠甲的雏形,牢牢地包裹着他。

蜈蚣的毒肢,丝毫奈何不得这层白晶护甲。

扭扭曲曲的蜈蚣,有的爬在脸颊上,耳背上,或许有些恶心,但方源心理承受能力早已经视若无睹了。前世落魄在野外,他什么都吃,无毒的蜈蚣他甚至生吃过。其实味道还挺不错的,当初吃时味道有些怪,但吃着吃着就习惯了。

他只把注意力集中在锯齿金蜈的身上。

锯齿金蜈缓缓推进,和方源的距离越来越短。

离着还有三四步远的时候,方源停住白银真元的外泄,这就导致他的气息顿时一弱。

锯齿金蜈敏锐地感觉到,顿时速度暴涨,宛若一条金线。

呼!

一眨眼的功夫,它就跨越了距离,从方源的腿肚子上攀绕上去。

这速度真是快,不动则已,一动就是金光一闪。

待方源反应过来时,这锯齿金蜈已经如蛇一般,绕过他的腰,张开口器,向方源的脸部袭来。

方源连忙伸出双手,抓住这金蜈的头部。

锯齿金蜈剧烈挣扎,方源有双猪大力,却竟然感到渐渐力不从心。

尤其是锯齿金蜈的两侧各一排的锯齿。此刻开始急速转动。

咔咔咔!

强大的撕裂力量,磨着天蓬蛊的白晶之光。

一时间,方源空窍内白银真元急速下降,同时白光如火花点点,被锯齿绞磨得飞溅出来。

方源这真元,还只是初阶的淡银真元,又只有四成二的储备,自然耐不住这般的消耗。

但方源临危不乱。即便摆脱不了锯齿金蜈的纠缠,但他还有杀手锏!

春秋蝉!

他心念一动,空窍中顿时浮现出春秋蝉的身影。

春秋蝉恢复得更好了,两片羽翅不仅新嫩如翠叶,同时身躯也泛出一抹高贵名木的温润油光。只是总体上,仍旧还是给人枯燥干死之感。

它已经恢复了两成多一些,气息自然更加强大。

这气息一流露。强劲挣扎的锯齿金蜈,顿时就萎了!

它只是三转的野生蛊虫。面对六转春秋蝉的气息。根本就不敢动弹。

方源感觉最明显,前一刻他还牢牢抓着锯齿金蜈,就仿佛抓着一只毒蟒,千方百计地阻止它的噬咬。下一刻,它就变成了一根软趴趴的草绳。

方源微微一笑,白银真元催动过去,锯齿金蜈根本已经缴械投降。方源的意志立即摧枯拉朽。将它身体中的野生意志绞灭个干干净净。

几个呼吸的功夫,锯齿金蜈已经被方源化为己用。

方源松开双手。锯齿金蜈无数的节足,有韵律地动起来。攀着白晶护甲,绕过方源腰部一圈,然后缠绕在他的手臂上。

周围的蜈蚣群,如潮水般散去。

野生的锯齿金蜈,因为有天生的意志,因此能统御虫群。但如今方源的意志已经取代了它,因此锯齿金蜈也丧失了和虫群交流并控制的能力。

方源也不剿除这蜈蚣群,任其离去。过个十几年,这虫群中或许又会有一只新的锯齿金蜈产生。不过这已经和方源的关系不大了。

他任由锯齿金蜈攀附在自己的肩膀上,又向洞窟深处探去。

这蜈蚣洞四通八达,走了一会儿,先前的主道就分化成了三条分支岔道。

方源先是动用地听肉耳草,倾听了片刻,便排除了中间那一支道。选了右边一道后,走了半个小时,发现到了死路。只好回撤,换到左边这道。

他收了锯齿金蜈,有着金蜈的气息,最克制这些蜈蚣,因此所到之处,群虫辟易。

这大大方便了他的探索。

进入左道不久后,他就从虫群移动开后,裸露出来的洞壁,发现了某些痕迹端倪。

“这是人工开采的迹象!”方源心头一振。

很显然,这条通道就应该是当初,花酒行者利用千里地狼蛛,开辟出来的。

方源顺着这条道,慢慢踱步下去,耐心探索。

这道中也生存着大量的蜈蚣,这对方源来讲,是一个好消息。

因为有着虫群生活在这里,他就可以排除许多的陷阱布置的可能性。

这通道比他料想得要长得多,方源探索足足花了六个多时辰,走了近三里的距离!

甬道坡度越来越向下,方源渐渐深入地底深处。

每隔一段距离,他就停下来,动用地听肉耳草倾听,来尽量地排除掉可能的危险。

哗哗哗。

“这是什么声音?”方源渐渐地听到一种奇怪的声音。

旋即,他就意识到这声音是什么。

“这是水声……难道说?”他心头一动,已经有了模糊的猜想。

走到道路的尽头,他看到了一面水晶之墙。

水晶之墙的后面,是水。

水中,有一股股灰白色的水流,呈现螺旋状相互自转着。仿佛是一道道微型龙卷风,此起彼灭,生生不息。

“果然和我的猜想一样,这是天然元泉!”看到此处,方源不禁目光一凝。

旋即,他又看到这水晶墙壁之后的元泉中,还有一物。

一朵蓝白相间的花骨朵儿,在泉水中悠然飘荡。

“这……竟然是天元宝莲!”方源心头一震!(未完待续。如果您喜欢这部作品,欢迎您来投推荐票、月票,您的支持,就是我最大的动力。)

------------

\end{this_body}

