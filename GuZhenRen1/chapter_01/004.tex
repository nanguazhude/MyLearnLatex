\newsection{古月方源!}    %第四节:古月方源!

\begin{this_body}

%1
朝阳升起来,霞光烂漫。

%2
山雾不是很浓,被利剑般的阳光轻易洞穿。

%3
一百多位十五岁的少年,此刻汇集在家主阁前。

%4
家主阁就处在山寨的正中央,高达五层,飞檐翘角,重兵把守。阁前就是广场,阁内供奉着古月先人的牌位。每代族长也都生活起居在这里面,每逢重大典礼,或者有突发大事,也会在这里召集家老们商讨议论。这是整个山寨的权利中枢。

%5
“很好,都准时来了。今天是开窍大典,是你们人生的重大转折点。闲话不多说了,随我来吧。”负责此行的,是学堂家老。他须发皆白,精神矍铄地领着少年们进入家主阁。

%6
不过却没有上楼,而是通过一层大堂的入口,往下走。

%7
顺着打造好的石梯,就进入地下溶洞。

%8
少年们纷纷发出惊叹之声。地下溶洞美轮美奂,钟乳石散发着赤橙黄绿青蓝紫七色光华,这光彩映照在少年们的脸上,霓虹般绚烂。

%9
方源混杂在人群中央,静静地审视这一切,心中暗暗思量:“数百年前,古月一族从中土迁徙到南疆,在这青茅山驻扎下来。就是看中了这里地下溶洞的一口灵泉。这灵泉产出大量元石,可以说是古月山寨的根基。”

%10
行了数百步,却是越来越暗,并且依稀听到了水声。

%11
转过转角,一条宽三丈有余的地下河,就展现在众人眼前。

%12
此地钟乳石的彩光,已经彻底消失了。

%13
但是黑暗中,河水却散发着淡淡的幽蓝之光,好像是夜空中的星河。

%14
河水从溶洞的黑暗深处流淌过来,清澈无比,甚至可以看到里面的游鱼,水草,以及河底的沙石。

%15
在河的对岸,是一片花海。

%16
这是古月一族有意栽培的月兰花,花瓣如月牙,呈现出清丽淡雅的蓝粉色。花茎如玉,花心闪耀着,好像是珍珠在光下的折射出来的温润光华。

%17
乍一眼看上去,在黑暗的背景中,河畔花海就好像是一大片的蓝绿地毯,点缀着数不清的珍珠。

%18
“月兰花,是很多蛊虫的食材。这片花海,可以说是家族最大的培养基地了。”方源对此心知肚明。

%19
“好美。”

%20
“真是漂亮呀。”

%21
少年们算是开了眼界,一个个双眼放光,既兴奋又紧张。

%22
“好了,下面听我报名,叫到的人穿过这河,到对岸去。能走多远,就走多远,当然越远越好。都听清楚了吗?”家老此刻说着。

%23
“清楚了。”少年们纷纷应是。其实来之前,都听家人或者前辈们讲过,知道走的越远,代表资质越好,日后的成就也就越大。

%24
“古月陈博。”家老拿着名单点出第一人。

%25
河水虽宽却并不深,只及少年膝盖。陈博一脸的严肃,踏上河畔花海。

%26
顿时他就感觉到一股隐形的压力,好像面前有一面看不见的墙,在阻挡他前进。

%27
正举步维艰之时,脚畔的花海中忽然浮起一蓬光点,光点很稀薄,呈现素白之色。

%28
光辉向陈博汇集过去,并投入到他的体内。

%29
陈博瞬间感觉到压力剧减。那堵无形的墙壁,忽然变得柔软起来。

%30
他咬牙用力向前走,硬生生的挤进去。走了三步之后,前方的压力又大增,一如之前如墙壁一样,不能再进分毫。

%31
见到此景,家老一叹,当场一边记录,一边道:“古月陈博,三步,没有蛊师资质。下一个,古月藻榭。”

%32
陈博脸色顿时苍白,咬着牙,穿过河水,回到原处。没有资质,今后就只能作为一个凡人活着,在家族中也只能是最底层的地位。

%33
他身躯摇摇欲坠,打击太大了,等于是扼杀了一生的希望。

%34
很多人都向他投来怜悯的目光,更多的人则关注着第二位登上彼岸的少年。

%35
可惜这个少年,也只能前进四步,同样没有资质。

%36
并非所有人都有修行的资质,一般而言,十个人中有五人能修行,就已经不错了。在古月家族里,这个比例还要高一些,达到六人的程度。

%37
这是因为古月先祖,也就是一代族长,是一位大名鼎鼎的传奇强者,因为修行的缘故导致他的血脉中隐藏着承载力量的基因。古月族人因为有着他的血脉,因此资质普遍较高。

%38
连续两个没有资质的情况,让暗中关注的其他家老们都脸色难看起来,就是老成持重的古月族长,也微微蹙眉。

%39
就在这时,学堂家老喊出第三个名字:“古月漠北。”

%40
“在!”一个身穿麻布衣衫的马脸少年,轻喝一声,越众而出。

%41
他身材高大,比同龄人要粗壮得多,透着一丝彪悍气息。

%42
三两步过了河,踏上对岸。

%43
十步,二十步,三十步,陆续有微光投入到他的体内。

%44
一直走到三十六步,终于走不动了。

%45
少年们隔岸看得目瞪口呆,学堂家老欢喜得大叫起来:“好,古月漠北,乙等资质,来这里,让我看看你的元海。”

%46
古月漠北便又回到学堂家老的身边,后者伸出手,搭在少年的肩膀,闭目凝神探查了一番,便收回手,点点头,在纸上记录起来:“古月漠北,元海六成六,可大力栽培。”

%47
这资质从上到下,分甲乙丙丁四等。

%48
一位丁等资质的少年,培养个三年,就能晋升成一转的资深蛊师,成为家族的基石。

%49
一位丙等资质的少年,培养两年,大多都能成为二转的蛊师,成为家族的中坚存在。

%50
一位乙等资质,就要呵护了。往往要作为未来的家老培养,六七年的功夫,能成为三转蛊师。

%51
至于甲等资质,哪怕出现一位,都是整个家族的幸运。要细心关照,倾斜资源,十年左右就能成为四转蛊师,到那时便能竞争族长之位!

%52
也就是说,这古月漠北只要成长起来,就是今后古月一族的家老。难怪学堂长老喜得哈哈一笑,而暗中关注的众家老们都统统舒了一口气,而后又纷纷向其中一位家老投去羡慕的目光。

%53
这家老也是一副马脸,正是古月漠北的爷爷古月漠尘。他脸上早已经荡漾起笑意,又挑衅地看了一眼自己的老对头:“怎么样,我的孙儿不差吧,古月赤练。”

%54
家老古月赤练一头红发,此时冷哼一声,并未答话,脸色阴沉得很是难看。

%55
半个时辰之后,已有一半少年踏足过花海,涌现了不少丙等、丁等的资质,当然毫无资质的占了几乎一半。

%56
“唉,血脉越来越稀薄,加上这些年来,家族也没有出现几位四转强者,来增强血脉。四代族长是唯一的五转强者,结果却和花酒行者同归于尽,没有留下血脉后裔。古月一族后辈的资质是越来越弱了。”族长深深的叹息着。

%57
就在这时,学堂家老喊道:“古月赤城。”

%58
听到这个名字,家老们纷纷看向古月赤练,这是古月赤练的孙子。

%59
古月赤城身材矮小,满脸麻子,捏着拳头,满脸出汗,显得特别的紧张。

%60
他踏上对岸,光点纷纷投入他的体内,一连走到三十六步,这才停步。

%61
“又一个乙等!”学堂家老叫喊着。

%62
少年们骚动起来,纷纷向古月赤城投来羡慕的目光。

%63
“哈哈哈,三十六步,三十六步!”古月赤练大叫着,示威地瞪着古月漠尘。

%64
这次轮到古月漠尘脸色铁青了。

%65
“古月赤城么……”人群中,方源若有所思的摩挲着下巴。

%66
记忆中,他因为在开窍大典中作弊,而受到了族中的严厉惩罚。

%67
事实上,他的资质只有丙等,但是他的爷爷古月赤练为他作假,因此有了乙等资质的假象。

%68
其实要作弊,方源心中有数十种方案,有些方案比古月赤城的更加完美。若是表现出乙等,或者甲等的资质,必然受到家族的大力栽培。

%69
但是一来,方源重生的时间太短,以他的境况难以准备作弊手段。

%70
二来,就算是作弊成功,日后修行时的速度是掩盖不住的,照样会露相。

%71
而这个古月赤城却不一样,他的爷爷就是古月赤练,是家族中权势最重的两个家老之一,能够为他遮掩。

%72
“古月赤练一直和古月漠尘敌对,这两个家老是家族中最大的两派势力。为了打压对手,他需要自己的孙子资质出众。也正是因为他在背后掩护,古月赤城才能够隐瞒一时。记忆中要不是那场意外,也不会暴露。”

%73
方源眼中闪烁着精芒,思量着该怎么利用这点来谋夺最大的利益呢?

%74
当场揭露,虽然会得到家族的一点奖赏,但是却会得罪位高权重的古月赤练,绝不可取。

%75
短时间之内,也不能敲诈勒索。因为实力太低,反而会自取其祸。

%76
正思量着,忽然听到学堂家老叫出自己的名字:“古月方源!”

\end{this_body}

