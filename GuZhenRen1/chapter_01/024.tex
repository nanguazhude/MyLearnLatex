\newsection{近战蛊师}    %第二十四节:近战蛊师

\begin{this_body}

三天之后。

“矮身闪躲,是克制摆拳的通常技法。当敌人进攻过来的时候,迅速下蹲,并且趁势做出反击,攻击他的裆部和腹部。不要害怕摆拳,常常一上来就使摆拳的,都是些没脑子,又冲动的人。”

演武场上,学堂的拳脚教头一边说着,一边做出动作进行示范。

一个木人傀儡右拳横扫过来,拳脚教习猛地下蹲,躲过攻击,然后出拳击打傀儡腹部,砰砰砰几下,就将木人傀儡击倒。

学员们围成一圈看着,但是大多提不起精神,显得兴趣缺缺。

学堂中传授多种课程,这节就是教授的基础拳脚。动用拳脚卖力气,可比又帅又酷的月刃攻击逊色多了,少年们几乎都有些心不在焉。

“下节课就是月光蛊的使用考核了。你最近练得怎么样?”

“还好吧,三记月刃都能发出来,但很少能全中。一般能有两记打在草人傀儡上。”

“嗯,跟我差不多。这些天为了练习这个,特意买了一具草人傀儡呢。”

……

少年们窃窃私语着,心思早已经不在这里,都牵挂着下节课的考核。

为了这次考核,他们可是在课下辛苦练习了很久,现在都卯足了劲,对考核十分期待。

学员们议论的声音传入教头耳中,拳脚教头猛地回头,大喝:“课堂上禁止讲话,都给我闭嘴,好好看着!”

他是二转蛊师,浑身肌肉发达,赤裸着雄健的上身,古铜色的皮肤上布满了疤痕。一声大喝,威风凛凛,顿时镇住场中所有少年。

演武场中一片寂静。

“基础拳脚,是重中之重。尤其是在蛊师修行的前期,比其他的任何东西都要重要。都给我集中注意力!”

训斥了一番,拳脚教头招来另一只木人傀儡。

这只淡黄色的木人傀儡高达两米,木头大脚踩在青石地砖上,发出脆亮的声响。它张开双臂,笨拙地向教头冲来。

教头闪过它的攻击,然后猛地抱住他的腰,用力前推,将高大的木人傀儡推到在地。紧接着,他顺势骑在木人傀儡的腰上,抡起拳头快速击打木人脑袋。

木人傀儡抵挡了片刻,就被教头如雨般的拳头打碎了脑袋,瘫痪在地上,一动不动。

拳脚教头站起身来,气息仍旧平和悠长,对学员们讲解道:“近身战中面对高大的敌人,不要害怕。破坏对方的重心是一种制敌的明智战术,就像我刚刚做的,抱住敌人的腰,控制对方的髋骨,然后用力前推。之后顺势骑在对方身上,施以猛烈拳击,无防守意念的人都会瞬间崩溃。”

学员们连连点头,只是目光中大多流露出不以为然的神色。

教头将一切都看在眼里,心中苦笑。

往届都是这样,年轻人的心性,自然容易被绚烂华丽的东西吸引过去。没有切身的体会和经历,少年们是难以理解基本拳脚的重要性的。

事实上,尤其对前期蛊师来讲,基本拳脚虽然看起来不起眼,却比月刃攻击更加重要。

“……记住,在近身战中,目光不要一味地注视敌人的眼睛,而是要留意对方的肩膀。敌人不论出拳,还是出脚,肩膀都会先动……”

“……近身战中速度很重要,这里的速度不是出手的速度,而是脚下移动的速度。……”

“……距离是最好的防守……”

“……腿要保持弹性,这样才能更容易爆发出力量……”

“……移动出拳的时候,要保持三角支撑。否则下盘失控,敌人没有被击倒,你反而倒下了……”

拳脚教头一边演示,一边耐心地解说着。这都是他长期作战中,用血泪积累出来的宝贵经验。

可惜场外的学员们,却没有意识到这一点。

他们又渐渐地交头接耳起来,谈论的焦点仍旧是下节课的月刃考核。

“这个拳脚教习很务实啊,但是教法却有问题。”方源在人群中静静地看着,点点头又摇摇头。

这教头教的毫无章法,纯粹是兴趣所致,想到哪里教到哪里。因此传授出来的东西,杂乱无章,又多又繁,很多学员起初听得很认真,但是渐渐地失去了兴趣,将注意力转移到其他方面去了。

只有方源一直在一丝不苟地听着,别人是学习,他是复习。他的战斗经验比拳脚教头丰富得多,但是听别人讲述,也是一种修行上的验证。

蛊师的战斗方式,通常分为近战和远程。

月刃攻击,就是远程类型,但严格上讲,只能算是中远程。因为它的攻击有效距离,只有十米。

近战蛊师的话,这位拳脚教头就是最好的例子。他们通常会选择增幅自身的蛊虫,进行修行。这些蛊虫带给他们超越常人的力量,敏捷,反应力,耐力等等。

就像这位拳脚教头,他浑身的皮肤都是古铜色。这当然不是他的肌肤本色,而是一种铜皮蛊的效果。

铜皮蛊会让蛊师的皮肤防御力大增,能让蛊师承受更多的伤害。

“一记月刃,就要耗费一成真元。一个蛊师在战斗中能发几次?次数少之又少,尤其是新手,根本就能形成有效打击。只能作为一种杀手锏,威慑性要大于杀伤力。对于一转蛊师来讲,真正有用的还是拳脚功夫。因为拳脚攻击更持久,更可靠。可惜这些道理,他们现在没有切身体会,是不知道的。”

方源用目光淡淡地扫视身边的同龄人,嘴角不由地泛起一丝冷笑。

基础拳脚课程终于结束了,中途休息了片刻后,在少年们期盼的目光中,学堂家老姗姗来迟。

他大手一摆,指着竹墙下的一排草人傀儡,直入主题:“好了,今天是检查成果的日子。你们五人一组,依次上来,使用月刃进行三次攻击。”

刷刷刷。

第一组学员走了上去,月刃在空中飞舞。

三轮之后,只有九记月刃,打中了草人傀儡。

学堂家老微微摇头,有些不满意。这命中率太低了,关键是五人中有两人只成功发出了两记月刃。

“你们接下来要好好练习,尤其是你,还有你。”家老稍稍训斥了一句,大手一挥,“下一组。”

被训斥的两位垂头丧气地走下场。其中一位少女,眼眶都泛红了,心中有些委屈。

自己只有丙等资质,又舍不得用元石快速恢复真元,因此这三天的练习量很少,导致自己催发月刃并不熟练。

蛊师炼蛊用钱,养蛊用钱,练习使用蛊虫也得耗钱。但她哪里来的那么多钱?

虽然有双亲在背后支持着,但是家家都有一笔难念的经。手头拮据是蛊师常常面临的困境。

“反正自己争夺第一也没有希望,干脆放弃,省下元石,不是更好么。”想到这里,少女心中又坦然了。

和少女一样想法的人,并不在少数。

因为练习量的缺乏,许多学员表现得差强人意。

学堂家老的眉头皱得越来越深。

方源看在眼里,心中摇头:“这些人真是可惜又可悲,为了些许元石,自己放弃了进取的机会。元石是拿来用的,像个守财奴积蓄元石,干嘛成为蛊师呢?”

换句话讲,往往鼠目寸光的人,才会锱铢必较,常常舍本逐末。而那些志存高远之人,通常都能表现出一副宽容大方、能舍能弃的气量。

“终于轮到我了。”这时,古月漠北的马脸上满带自信微笑,走上了场。

他身材粗壮,透露出一股彪悍气息。站定之后,扬手三记月刃,三记全中。其中两记打在傀儡的胸膛上,一记印在傀儡的左臂,削飞了几根青色的草屑。

这样的成绩,自然引起了少年们的一阵赞叹。

“做的不错。”家老的眉头也为此微微舒展。

又一组上来,其中就有古月赤城。

古月赤城身材矮小,满脸麻子,表情带着微微紧张。

他接连催发出三记月刃,都打在傀儡的胸口处,削出三记交错的伤痕。伤痕由深变浅,几个呼吸之后,因为草人傀儡的自愈,而恢复原貌。

不过这样的表现,已经和古月漠北持平,同样得到了家老的表扬。

赤城昂着头走下场,途中挑衅地看了漠北一眼。

“哼!”场下,古月漠北一声冷哼,却没回瞪赤城,而是看向仍旧还未上场的古月方正。

他心里清楚,真正有威胁的,只有古月赤城和古月方正。

前者和他一样,乙等资质,又有元石的不断供应。后者是甲等资质,虽然元石可能没有他们丰富,但是靠着本身资质带来的真元恢复速度,也能在短时间内进行大量练习。

现在古月赤城的成绩已经出来了,和他漠北相平,只剩下古月方正了。

到了最后几组,古月方正终于上场。

\end{this_body}

