\newsection{答案}    %第一百四十三节:答案

\begin{this_body}

%1
随后,方正也赶了过来。

%2
“青书大人!”他带着欣喜的神情,无知无畏地奔向古月青书。但紧接着就被一阵松针打了回来。

%3
“组长,是我方正啊!!”他惊疑地叫喊起来,第一次意识到不妥。

%4
但古月青书哪里会答应他。

%5
“哥哥,青书大人到底怎么了?”方正迷茫和惊惶之下,只好求助方源。

%6
但方源根本不理他,而是半蹲下来,伸出右手虚握成爪,对着白凝冰的方向。

%7
强取蛊!

%8
他暗暗催动这只蛊虫,很快就感觉自己抓住了一件东西似的。

%9
他连忙往回拽,从白凝冰的身躯中就飞出了一只蛊虫。

%10
竟然是一只赤铁舍利蛊!

%11
这是白家族长曾经交给白凝冰的,但是白凝冰并未使用它,就迅速地达到了三转修为。

%12
如今却是便宜了方源。

%13
看到赤铁舍利蛊从白凝冰的身上,遥遥飞出,然后落到方源的手中。一群白家的蛊师顿时都急红了眼,纷纷叫嚷起来。

%14
“混蛋,不想死的,就快住手!”

%15
“居然敢当面抢我们白家的蛊虫!!”

%16
“还是一只赤铁舍利蛊啊……”

%17
方源冷笑不语,赤铁舍利蛊一到手,他就利用春秋蝉顷刻炼化。但并不收入空窍,而是揣入怀中,给人一种并没有当场炼化的错觉。

%18
他紧接着又抓,这一次飞出了一只灰色的甲虫。

%19
“石窍蛊……”方源认出了这只蛊虫的来历。目光一闪,再次将其炼化,收入怀中。

%20
“可恶,又是一只蛊虫!”

%21
“拦住他,阻止他,他如此肆无忌惮,根本就是没有把我们白家放在眼里。”

%22
“救下白凝冰。杀了这些家伙!”

%23
白家的蛊师们嘶吼着,从山道那端狂奔过来。

%24
白凝冰受着家族的大力栽培,身上的蛊虫无一不是精品。此时被方源当众夺走,让众人无不心中滴血。

%25
这比杀了他们还要难受。

%26
看着这群人气势汹汹地杀过来,方正惊得后退一步。方源却无动于衷。

%27
此刻情形,他和方正站在山道的东面,白家蛊师们则在西端,两方之间堵着古月青书和白凝冰。

%28
嗖嗖嗖!

%29
松针如雨,延绵而下。

%30
“可恶……”白家蛊师们怒骂连连,却一时被古月青书阻挡住。

%31
“古月青书离死不远,剩下的时间只够我再催一次强取蛊的。这一次,又会是什么?”方源沉下心来,再催强取蛊。

%32
强取蛊每次催动,都要消耗不少的真元。针对的蛊虫目标越强大。过程就越艰难,真元的消耗就越多。如果强取失败,蛊师就会遭到力量反噬。

%33
因此强取蛊比较鸡肋,用途并不广泛。

%34
但是此刻,白凝冰奄奄一息。意识也是迷迷糊糊,接近油尽灯枯的状态。要强取他的蛊虫,却并不困难。

%35
白凝冰身上的蛊虫当中,最有价值的无疑是霜妖蛊。此蛊可以媲美木魅蛊,能令人化身霜妖。但是使用时间过长,也会令蛊师生命之火熄灭。化为一座冰人雕塑。

%36
白凝冰也知道它的弊端,从未像古月青书一样,彻底使用过霜妖蛊。

%37
除了霜妖蛊之外,价值第二高的是蓝鸟冰棺蛊,三转蛊虫。现今,寄居在白凝冰的咽喉处。

%38
若是能抓来蓝鸟冰棺蛊,就是最佳的情形。但是强取蛊毕竟只是二转,想要随蛊师的心愿,那它还有力未逮。

%39
最终,方源抓来的是白凝冰的水罩蛊。

%40
这也不错了,水罩蛊配合白玉蛊,将给方源提供更好的防御力量。

%41
古月青书化身的树精,最终被白家蛊师们推倒。

%42
他们劈开木制的牢笼,将失去了右臂,已经昏死过去的白凝冰救下。

%43
正要向方源、方正杀来的时候,古月一族的援军也赶了过来。

%44
双方对峙了一阵子后,皆默契地罢手。

%45
青书死亡,白凝冰重伤,狼潮之下,这样的牺牲已经足够触目惊心。如果发生大规模的火并,对于家族而言,那生存的压力就太大了。

%46
不论在哪个世界上,人们之所以争斗,大多都是为了利益。

%47
然而世间上最大的利益,无疑是“生存”。

%48
最终,古月青书的尸首和蛊虫,由古月一族的蛊师们带回。

%49
双方相互戒备着,离开这处战场。

%50
……

%51
天空下着雨,阴沉一片。

%52
一群人站在山寨后的一处山坡上,这里便是墓地。

%53
几乎每隔一段时间,这里总会增添几座新坟。

%54
在这个世界,人类生存艰难,不管是外力还是内因,总会有几多的牺牲。

%55
家老的声音,嘶哑低沉,回响在众人的耳中,更增添人们心中的压抑。

%56
“……我们有着同一个姓名,我们来自同一个家族,我们的身上流着相同的血脉。”

%57
“我们距离近在咫尺,却已经生死永隔。”

%58
“此刻悲痛充斥我心。”

%59
“等着我。”

%60
“将来的某一天,我也将躺在你的身边。”

%61
“让我们化为尘埃和泥土,托起血脉后代……”

%62
一片新坟前,这群人低垂着头颅,不少在低声地抽泣,一些则满怀哀伤地看着墓碑上的名字。

%63
生死的残酷,像一只白骨之手,在所有人的心中撕开一条血淋淋的伤口。

%64
只是有些人早已被伤害得麻木,有些人的心却还稚嫩。

%65
古月方正身处在人群当中,低垂的眼神愣愣地盯着墓碑上“古月青书”这四个字。

%66
死了?

%67
他的眼中,有着无尽的迷茫。

%68
昨天的战斗。整个过程和情形他还历历在目,深印在心。

%69
他的经验有限,没有看懂古月青书的悲壮和牺牲。

%70
当如今现实摆在他的面前时,他一时间竟然接受不过来。

%71
“死了?那个温柔地笑着,总是提点我,照顾我,宽怀我的青书大人……竟然真的死了吗?”

%72
“为什么会这样?”

%73
“为什么这个世界上。总是好人容易死去,而坏人长存?”

%74
“这难道是一个梦,我现在正在做梦吗?”

%75
方正下意识地捏了捏双拳。真实的触觉,更让他心中悲痛无比。

%76
耳边传来周围蛊师的议论声。

%77
“唉,想不到这一次。居然连青书大人都牺牲了。”

%78
“人总是会死的。只是太可惜了,听说白凝冰那家伙还剩下一口气,终究被救活了。”

%79
“愿他在地下安眠,保佑我们能渡过这场狼潮罢……”

%80
人群渐渐地散去,最终只剩下方正。

%81
少年孤零零的身影,独自面对着满山的墓碑和坟墓。

%82
“青书大人!”他忽然跪倒在地上,眼泪扑簌地掉落下来。

%83
他迷茫,他懊悔,他无奈,他悲痛!

%84
啪。啪啪,啪啪啪。

%85
豆大的雨滴从阴云中垂落,击打在地面上,将绿草和树枝压垂。

%86
泥土的气息逆冲到方正的鼻腔,他痛声哀哭。哭声和雨声混杂在一起。十指嵌入泥泞当中,想抓回青书的生命。但他最终只抓住了两团泥土。

%87
雨下起来了,白凝冰躺在柔软床榻上,无神地看着这场雨。

%88
他的右臂断处已经做了处理,绑着白色绷带。他的双眼也恢复了黝黑之色,修为却是三转。没有再度压制到二转。

%89
当他从昏睡中苏醒过来,他忽然觉得意兴阑珊,了无生趣。

%90
他静静地躺着,睁着双眼,已经有十多个小时。任由三转的白银真元,在温养着空窍。他都懒得去管。

%91
这场雨沟动了他内心最深处的回忆。

%92
就是在这般的夏雨当中,他正式被白家族长收养。族长慈爱又寄托希望的目光,落在他的身上,周围家老们恭贺的声音,如潮水般涌来。

%93
小小的他,赤着脚站在冰冷的地板上,盯着窗外的雨,感到的只有迷茫还有孤单。

%94
“人活着,究竟是为了什么呢?”这个困扰他二十多年的问题,并且极有可能将继续困扰他,直到他自爆而亡的问题,又再度浮现出来。

%95
“是为了亲情、家族么……”白凝冰不可避免地想到了古月青书。

%96
从小到大,这样的牺牲,他见过很多次很多次。有些是白家的族人,有些是熊家的、古月家的。

%97
他难以理解这等狂热,他似乎天生就是一个冷漠无情的人。

%98
古月青书并不能带给他答案,于是白凝冰的心头就浮现出方源的影子。

%99
他第一次见到方源时,方源正靠着背后的树,吃着随手摘来的野果,无动于衷地看着山脚下的战斗。

%100
他激动得全身颤栗,兴奋的发抖。皆因他从方源的那双同样幽深的黑眸中,看到了他自己。

%101
但现在回想起来,方源的眼眸中似乎比他多了一些东西。

%102
那个东西,就是能回答这个问题的答案。

%103
雨越下越大,雷霆轰鸣,狂电闪烁。

%104
“人活着,究竟是为了什么呢?”昏暗的书房中,古月方正同样在问这个问题。

%105
族长古月博叹了一口气,他哀怜地看着眼前失魂落魄的少年一眼。然后便将目光投向窗外的雨。

%106
方正的迷茫可以理解。不可避免的死亡,往往迫使人去思考生命存在的价值。

%107
“你知道吗?十多年前,也有一个少年和你这般情形,问了我相同的问题。”良久,古月博缓缓开口道。

%108
“那个人,就是你的组长,我的义子——古月青书。”

%109
方正微微一愣,抬起头来。一双通红浮肿的双眼,透出一股对答案的渴求。(未完待续。如果您喜欢这部作品,欢迎您来投推荐票、月票,您的支持,就是我最大的动力。)

\end{this_body}

