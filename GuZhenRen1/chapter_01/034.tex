\newsection{压着你打!}    %第三十四节:压着你打!

\begin{this_body}

方源充耳不闻,一心两用。

他一边汲取元石中的天然真元,一边留心观察空窍。

空窍中原本低落下去的真元海面,随着天然真元的不断注入,已经开始缓慢地回升。

这种回升速度有些缓慢,但是方源也不着急。

修行就是累积,是急不得的。

真正着急的反而是屋外的中年男性个家奴。

大约过了半个小时,方源的青铜真元海又达到了四成四的极限。

不过这还不算完。

此时的真元海,呈现出一种翠绿色,这只是一转初阶的青铜真元。

方源先前温养窍壁的真元,并不是初阶真元。而是经过酒虫精炼之后的中阶真元。

“酒虫。”方源念头一动,元海中的酒虫顿时就飞腾而起,悬停在半空中,缩成一个白色的汤圆团子。

刷。

一成的初阶真元调动而出,投入酒虫体内,很快被它吸收的涓滴不剩。

然后,一股醇香的雾状酒气,就从酒虫体表散发而出,凝成一团。

方源再度调动出一成真元,投入这酒雾当中。

酒雾完全被消耗之后,原先的一成初阶真元,体积骤然缩小了一半,同时颜色也从翠绿色转变成了苍绿色。

这是中阶真元了。

“寻常学员,要晋升修为,都是用一转初阶的真元。而我去是用中阶真元,效率比他们要高出至少两倍。同时用中阶真元催动月光蛊,发出的月刃,也比初阶真元催动出来的更强大。”

直到元海中都被提纯精炼成一转中阶真元,方源这才睁开双眼。

修行无岁月,此刻已经是凌晨时分。

天空已经不是纯粹的黑,而是一种深沉的黑蓝之色。

月亮已经隐去,只剩下几颗残星。

大门几乎敞开了一夜,木制门扉下面的边角被浸湿了,显露出一种水渍带来的黑色。

学堂宿舍就是这点不好,它不像普通的吊脚竹楼那样舒适,直接坐落在地上,因此湿意较重。

回过神来,方源也感到了一股凉意包裹着自己的身躯。盘坐久了,两条腿似乎都有些麻木了。

他松开合拢的右手,顿时洒下一片灰白石粉。

这是元石中的天然真元被完全吸收之后,残留下来的渣粉。

“一晚上的修行,我先后消耗了三块元石。”方源心中暗暗计算着。

他本身只有丙等资质,但为了追求修行速度,一直在利用元石补充真元。而后更关键的是,他运用酒虫,来精炼出中阶真元。

这就大大加剧了元石的消耗。

“虽然昨天又抢了一笔,但是一晚上就消耗元石高达三块。这样一来,看起来元石很多,但是也经受不住长期的修行损耗……不过追求修行的速度和效率,自然是要付出代价的。”

方源再看门外。

只见那叫高碗的健壮家奴,此时正蹲在墙角,身子缩成一团,似乎是睡去了。

“看来那个二转女蛊师,早已经离开了。留下这个高碗,在这里看守我。呵呵。”方源的嘴角浮现出一丝冷笑,他走下床,开始悄悄地活动拳脚。

身体渐热,他就踏出宿舍。

“小子,你终于舍得出来了。怎么样,乖乖地束手就擒,跟我走,向我家大小姐磕头认错吧。”高碗双耳一动,捕捉到了方源的脚步声,倏地一下站了起来。

他身躯粗壮,个头几乎是方源的两倍。

身上肌肉贲发,吊梢眉毛下,一双细细的眼缝闪着残忍恶毒的光,仿佛是一条饿极了的鬣狗。

方源面无表情,继续走向他。

“小子,你早点出来也就算了。现在才出来,知道大爷为了看守你,吃了多少苦头么?”他一边嘿嘿地笑着,一边迈开步伐迎向方源,神情很不怀好意。

就在这时,方源忽然轻喝一声,猛地一跃,举起双拳冲向高碗。

“臭小子,不知死活!!”高碗脸色扭曲,他现在一肚子的火气,抡起砂钵大小的拳头,向方源砸去。

这拳头有力至极,突破空气,竟带出呼呼的风声。

方源目光清冽如水,眼看着拳头打来,脚下一错,就晃到了高碗的身侧。

伸出手指,向高碗的腰际一戳。

高碗回臂格挡,方源没有戳中,打在了高碗的左前臂上。

方源顿感手指好像是戳中了一块铁板,又痛又麻。

“这个高碗,已经处在凡人武艺的巅峰。我现在只有月光蛊能够战斗,没有其他蛊虫辅助,单论拳脚不是他的对手!”方源目光一闪,立即明智地放弃攻击,连退几步,拉开了距离。

古月山寨中,只有古月族人才有修行蛊师的资格。

其余外姓人,不管是有无修行资质,都没有参加开窍大典的资格。

但是这些凡人,却可以修行拳脚功夫。

就像眼前的高碗,他虽然不是蛊师,但是必定苦练过拳脚,基本功相当扎实,又是中年汉子,凡人一生中最年富力强的时候。

方源除开月光蛊能够用作战斗,身躯不过是个十五岁的少年。不论是力量,敏捷还是承受力,都不是高碗的对手。

像高碗这样的武人,足以能杀死一转初阶的蛊师。就算是中阶蛊师,也能造成一定的威胁。

“这个小子,太阴狠了!”看到方源和自己拉开距离,高碗心中还残留着余悸。

腰是一个人的要害部位,被人用力一戳,伤害非浅。若是用力过大,甚至能造出人命。

高碗在院子外待了大半个晚上,身子都被寒潮湿气包裹,反应有些缓慢。刚刚一击,就差点被方源得手了。幸亏他平时虽然溜须拍马,但是也苦练不辍,关键时候身体的本能凌驾于头脑的反应,惊险万分地防住了方源的进攻。

“不能再大意了。这小子是个狼崽子,出手又狠又阴,一不小心就要着了道。难怪小少爷两次被他击昏。”高碗抹了一把头上的冷汗,一扫先前的轻蔑,对方源彻底重视起来。

“抓住这个小子,我就立下大功了。到时候大小姐必定会有赏赐!一转初阶的月刃,顶多只能算把刀子,只要不打在要害部位,不过是破皮流血的浅伤。”

想到这里,高碗心头炙热起来。伸出铁钳一般的双手,就向方源抓来。

砰砰砰。

方源丝毫不惧,迎上高碗,与其近身搏斗。一时间拳脚相击,攻防转换,传出连续不断的闷响。

打劫学员们时,他只用手掌,以制服控场为主。现在和高碗相斗,他再不留手。

时而用手指抠眼睛,戳咽喉,时而用手掌根部击打下巴,用边沿砍他后脑,时而用膝盖顶撞他的胯裆,时而用手肘戳他的腰侧。

高碗打得冷汗直冒。

方源招招不离高碗的要害,手段又狠又辣,简直是要置高碗于死地。

高碗只是凡人,不像蛊师,虽然苦练拳脚,但是要害仍旧是要害。凡人不可能把眼皮子连成钢铁,这就是凡人练武的局限。

反而高碗却不敢对方源下杀手。

方源是古月族人,要杀了方源,高碗就犯下众怒,铁定被五马分尸,到时候漠家会第一个清理门户。

因此他的想法是生擒方源,最好能在生擒的过程中,给他吃些苦头。

一方心有顾忌,一方心存杀机,场面上竟然是方源压着高碗在打!

(ps:哎呀呀,悲剧了,昨晚码字太晚了,跳了一个章节,幸好有诸位朋友们的提醒。立马改正,立马改正!2013年了,祝大家新年快乐哦!)

\end{this_body}

