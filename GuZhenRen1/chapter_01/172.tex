\newsection{非方源莫属!}    %第一百七十二节:非方源莫属!

\begin{this_body}

酒席上,一时间鸦雀无声。

所有人都愣住,呆呆地看向方源。

他们没有想到方源这么回答,直接承认,还如此坦诚。

“还是太年轻啊,居然这种话都说出了口。”

“哼,这话一说,就意味着整个政治前途的毁灭。”

“今后这方源再也不足为虑了……”

家老们心中思绪翻腾。

狼潮进行到这里,大局已定了。接下来的几个月内,只会有小股狼群出没,并且随着狡电狈的不断召回,以及蛊师们的清剿,到了年末电狼将会基本消失。

但争斗却从不会停歇。

有人的地方,就有利益。有利益的地方,就有争斗。

狼潮下,是人和狼的争斗。如今狼潮中最艰难的时刻已过,家族之间的内斗则露出水面,转为主要矛盾。

狼潮冲击山寨,很多蛊师身死,旧有的势力被打破,这些势力原先掌控的利益,失去原主人,自然需要重新分配和瓜分。

在古月一族的高层,原先众多家老,分食着整个山寨的利益蛋糕。但如今只剩下方源等不到十位家老,蛋糕却仍旧在那里。

要瓜分这块大蛋糕,自然就需要较量。政治上的斗争,虽然没有狼潮这般刀光剑影,血雨腥风,但是阴谋算计,冷风陷阱,亦是深沉艰险。

原本方源作为唯一的新晋家老。势头很足,风头极盛。但如今他坦然承认,自毁政治前程,就等若放弃了和其他家老竞争利益蛋糕的机会。

一下子,方源在众家老心中的威胁程度,就降低了很多倍。

看向方源的众多目光中,蕴含的压力明显在减缓。

这时,族长古月博轻叹一口气:“既然方源你已经承认。那么身为族长,不得不对你此番的临阵脱逃进行处置。按照祖宗传下来的家法,临阵脱逃者,将剥削家老的职位。但最终的结果,将由我和其他几位家老联合商议,酌情处理。但不管结果如何,都希望你能够接受。”

其他人脸色各异。

方源点点头。没有开口,似已认命。

古月药姬自断一臂。保住性命。这样的狠辣和决断。方源亦有。只是他舍弃的更多,将家老的身份都舍去了。

他现在最大的麻烦,是突如其来的铁家父女。若是掺和到政治漩涡当中,受到政治倾轧,势必情形将更加危险。

“舍得,舍得,取舍之间。就是人生。家老这身份,本来就是为了更好的修行。才拿来用的。我为了永生而踏上魔道,连命都可以舍弃。还有什么不可舍的?家老之位,哼……”

方源心中没有一丝懊恼和后悔。

这是最明智的抉择。

而且,虽说会有惩罚,但力度必定轻微。

毕竟现在家老稀缺,他身为三转蛊师,力量必须得到重视。族长一方面要惩罚,另一方面也要借助方源的力量,来稳住山寨大局,因此也要安抚他。

至于其他家老,方源已经退出了这场关键性的政治博弈,已经没有威胁性了。更不会对方源赶尽杀绝,万一逼得方源反击,岂不是自找苦吃?

“虽说有着家族制度,但制度是什么?呵呵,制度都是上位者维护利益的工具。一方面它主宰和分割下层群体的利益,另一方面也是上位者之间协调彼此的游戏规则。”方源心中冷笑,对于这些方面,他洞若观火,看得极为透彻。

“现在最关键的,还是铁家父女。真是该死,居然比我预料中来得更快。狼潮还未退去,他们就到了。哼,不过这样的举止行径,倒是符合铁血冷嫉恶如仇,奋不顾身的性情。”

一想到这里,方源就心生压力。

尽管铁血冷受了伤,但其战力绝非方源可以比敌。瘦死的骆驼比马大,就是这个道理了。

“我要离开山寨,又需要避免铁家父女的追捕,该怎么破局?”

方源苦恼。

这铁家父女,绝非贾富那般容易糊弄。再者,他方源修为低微,纵然有千般妙计,但没有实施的能力,为之奈何?

三转和五转,这实力差距很大。

“诸位我有话说。”就在这时,一直沉默的古月赤练忽然开口。

他脸色苍白,拖着重伤之体来此,也没有能饮酒,只是喝茶。

但他接下来,却是语出惊人死不休:“有一件事情老夫必须坦诚,方源大人之所以未能及时出现战场,乃是老朽所为。”

“什么?”

一时间,其他家老都微微吃了一惊。

“哦,此话怎讲?”古月博问道。

方源亦投去一道目光,只是隐藏住了惊讶。

这古月赤练怎么会为自己说话?

虽然方源先前掌握了他的把柄,但是此次狼潮,赤脉的继承人古月赤城,却已经死在战场中,不幸丧生了。

尽管赤脉已经尽了最大可能,对赤城进行了保护。但是战场上,总是意外最多的地方。并且狼潮之下,人人自危,很多时候连自己都照顾不了,更何谈照顾他人?

人死灯灭,方源先前掌握的把柄,也就失去了作用。但为何古月赤练,反而主动站出来为自己遮谎?

古月赤练接着叹气:“实不相瞒。我的孙女古月漠颜已经深深地爱上了方源家老,在此之前,她亲自老求我,跪在地上哭泣,不愿看到方源去战死沙场。老夫就这么一个孙女,起了私心,就将方源强留在府内,迷昏了他。直到追击雷冠头狼,老夫才放其出去。所以千错万错。都是老夫的错。和方源大人没有关系。”

“什么?”

“哦,真的是这样?”

一众家老都露出怀疑的神情。

古月赤练的话,也太扯了点,可信度一听就不高。

“年轻人的情情爱爱,实属正常。”古月博点点头,意味深长地看向方源,想从他的脸上看出一些端倪来。

但方源已经垂下眼帘,脸色平静。看不出任何喜怒。

族长有些不大托底了。

其他家老也在交换眼色,一时间没有搞清楚古月赤练为何这么偏袒方源。

他这样说话,为方源开脱,是在牺牲自己的名誉,毁坏自己的政治前途!

古月赤练接着道:“方源家老,为了维护老夫的名誉,甘愿自己承担骂名。但老夫之前已经错过一次。怎么能再错一次。真相就是这样,该怎么处罚。请族长大人明示。老夫认罪伏法。若是要剔除家老身份。老夫也是甘愿。”

族长连忙摆手:“赤练家老劳苦功高,这么做也可理解,人非草木,孰能无情?如今家族百废待兴,正需要您这等肱骨之臣。只是这到底是私情,如何处罚,还需要商量。今日有贵客临门。先不说这些,来。铁兄,在下和全体家老一齐敬您一杯!”

说着。古月博就站起来。

其余家老也跟着站起,举起酒杯。

“诸位客气了。今后还要叨扰诸位,希望诸位能够体谅。”铁血冷不端架子,亦站起来,饮下一杯酒。

……

赤脉大院内,草木芬芳,假山清泉,流水潺潺。

夜空中,明月如盘而高悬。

酒席早已结束。

方源端坐在院中的凉亭内,倾听着耳边的泉水之音,面色平淡地放下手中的茶杯。

在他对面坐着的,正是古月赤练。酒席散场后,他就邀请方源来此坐谈。

“来,再喝一杯茶,这竹叶青水茶,正好解酒。”古月赤练微笑着,亲自为方源斟水。

方源神情平淡,看着茶水又添满,只说了一声:“的确是解酒的茶。”

说完,他就把视线移向亭外。

只见明月苍白,洒下一片清辉。月下庭院静谧幽雅,但风中阴影斑驳,隐约可见这赤脉大院的正由盛转衰的落魄气象。

在酒席期间,方源已经窥破了古月赤练的想法。现在看这景象,更是心中笃定。

赤脉没人了!

古月赤城一死,整个赤脉就失去了继承人。

虽然赤城还有亲姐姐漠颜,但家族体制,祖宗传法,历来重男轻女,家业亦只传男不传女。

就算将来古月漠颜,成了三转,晋升为家老。但她的家业,却不代表赤脉正统,只属于她自己。若将来嫁人,这家业就转为她的夫君名下。

一个家族中的政治势力,若丧失了继承人,那就是没有了前景,不会有人追随的。

赤脉已经陷入了艰难处境,面临着崩溃的危机。

但天机常留一线,赤脉并非绝境,还留有一丝希望。

这丝希望就在古月漠颜的身上。

她虽是女儿身,但却可以嫁人。

这女婿若是入了赤脉的门,在身份上就能说得通,也能继承赤脉!

看方源没有开口的迹象,古月赤练心中暗骂一声小子狡诈,却不得不首先开口:“不知道方源家老,对我赤脉如何看法?”

他身上有说不得的苦衷。

自从重伤之后,他的修为就已经落到二转境地,再无一丝重回三转的可能。

他现在只是靠着蛊虫,遮掩了真实气息。但纸遮不住火,总有暴露的一天。

唯一的继承人已经陨灭,自己落到二转,也会失去家老身份。古月赤练现在急需一位撑得住场面的外援,来镇守住场面。

昔日,他位高权重,赤脉占据了庞大的利益蛋糕。如今家族势力重新洗牌,他不求更多的利益,只要将手中的这部分守住,就是最大的胜利。

他左思右想,最理想的人选非方源莫属!。。)

\end{this_body}

