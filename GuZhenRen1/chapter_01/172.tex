\newsection{压力渐生}    %第一百七十五节:压力渐生

\begin{this_body}

你问我为什么那惠师来杀我”我怎么知道!”面对铁若男的提问,方正眨眨眼,很是无辜。

“若是你做了一些事情,真的希望你不要隐瞒。因为很可能你无心的一句话,却对破案产生巨大帮助。”少女诚恳地道。

方正摇摇头:“我也纳闷呢,那段时间我都在闷头修炼,怎么就招来刺杀。不过事后,身边的人替我分析,这魔道蛊师很有可能受到其他两家族雇佣,特意来扼杀我这样的新星。你也知道,白家和熊家,向来都对我古月家很仇视的。尤其是熊家最有嫌疑,他们曾经就招揽吸收过魔道蛊师。”

“熊家么……”铁若男听了不禁有些丧气。熊家已经被狼潮第一百七十五节:压力渐生吞灭了,看来这条线索又断了。

然而就在这时,门外传来一阵喧哗声。

“快看,那不是熊家寨的人吗?”

“熊家寨不是被灭了,怎么还有使者前来!?”

熊家寨使者的出现,引发了一场波及整个山寨的波澜,人们议论纷纷。

很快就有消息,从家主阁传出。

“熊家寨还有大量的幸存者。”

“他们是主动撤离的,利用了先祖留下的一只蛊,同时隐去了许多人的身形,瞒天过海了!”

“这群混账东西,消极避战,让狼潮蔓延到我们这边来”

“哼,熊家这些人五大三粗,其实心中阴险得很。想要借助狼潮,来削弱我们。太卑鄙了!”

古月族人皆义愤填膺。

熊家使者的出现,令青茅山的整个局面都发生了翻天覆地的变化。

原本以为今后将是白家和古月家的双雄之争,没有想到,仍旧是三家争霸。

不过,想想也释然。熊家寨是屹立数百年而不倒的家族,同样怀有底蕴。哪一家没有老祖宗,没有压箱底的手段?

熊家使者走后,古月博立即召开家老议会。第一百七十五节:压力渐生

“熊家寨的这些狗崽子,真不是个东西。居然直接撤退了!”

“果然不能小看任何人啊。熊家寨一直排在我族和白家之后,在青茅山属于垫底势力。但竟然有这样的图谋,今后更要当心。”

“他们想要借刀杀人,借助狼潮来铲除我们。还真被他们得逞了,要不是那只狡电狈,我们未必会牺牲那么多的家老。这些人真是该死!”

“要不是铁神捕出现,恐怕连两家族长都要殒命。这些人不能白白地放过他们。”

“要求赔偿是必须的。是我们和白家一齐出力,将狡电狈的麻烦处理掉。但是如何索要赔偿,还需要细细推敲。”

家老们你一言,我一语,商议出最后结果。

古月一族将派遣使者,出使熊家寨。务必打探出熊家的虚实。

若熊家强大,就得和白家联合。若熊家过于弱小,说不定就直接派人剿除,夺取他们的元泉。

“那么,谁出使熊家寨比较合适呢?”古月博扫视周围,问道“哪位家老可担当此重任!”

堂中顿时沉默下来。

家老们你望我,我望你,谁都不愿意去。

现在正是内斗紧张,分派利益蛋糕的关键时刻。若出使熊家,使得自己这一脉群龙无首,被他人所趁,回来后大局已定,找谁哭去?

“老身觉得,要出使熊家,必须得有一个老成持重,经验丰富,可独当一面之人。在座的诸位当中,唯有漠尘家可担当此重任!”古月药姬忽然道。

古月漠尘冷哼一声,立即反驳:“要说资格,药姬大人比老夫更加深厚。尤其在人望方面,老夫望尘莫及,甘拜下风。出使熊家寨,看来还得劳烦药姬大人出力了。”

“漠尘家老所言极是,我亦推荐药姬大人。”一位家老站出来。

“我反而觉得漠尘大人,更为适合。”另一位家老则立即出言反驳。

场面一阵混乱。

古月博高座主位,冷眼看着,没有做声。

药脉已经有脱离自己的意向,不在受他掌控了。所以他两不想帮,静看场中局势。

这是以药脉和漠脉的首次较量。

双方皆有政治盟友,可见在场下,两方首脑都做了许多的妥协,和利益交换。但总体而言,药脉更为强势一些。

古月药姬的人望,以及赤脉的倾向,是造成这个局面的主要原因。

古月博冷眼旁观,将个人阵营都暗暗记在心头。

作为族长,他当然不希望看到大权旁落,这些家老都是他的对手。但他决定先静观其变,隐而不发。

“漠脉掌握的资源和权柄太多了,又丧失了继承人,因此药脉才急着跳出来,想啃上一口。所以这场争斗的关键,就在一个人的身上。”

古月博暗暗思量,将目光移到方源的身上。

方源一直端坐在位置上,沉默不言。

“看来这个方源和漠脉走的还不是很近,还没有达成利益一致的协约。否则早就出声相助了。这是否是我的机会呢?”古月博不禁寻思。

但就在这时,方源忽然从座位上直接站起来。

他的这个动作,顿时吸引了众人的目光。

然后他一语惊人:“出使山寨,事关重要,关系到我族兴衰存亡。我愿请命,担当出使之责,为家族探尽熊家虚实!”

“什么?”

“方源竟然主动要求去?”

“他这是什么意思?他是真傻还是假傻?不怕回来时,利益已经被瓜分光了吗!”

众家老纷纷流露出惊疑之色。

方源自有其打算。若是出使熊家寨,或许能寻找到机会,将三家冲突挑起。即便不能,也有离开的机会。

“等一等!谁都能担当使者,惟独方源不能!”大门忽然打开,铁若男一马当先,径直闯入。

方源侧身回头,瞳孔微微一缩。就看见铁家父女迈步而来,同时身后还跟着两人,一位和方源相貌酷似,正是他的弟弟古月方正。

另一位则是古月江鹤。

“不知铁神捕此来,有何见教?”古月博站起身来相迎,他语气有些不悦。这是古月家族内议,你们怎么能直接闯进来。

“古月族长,以及诸位家老,小女已经调查出来,那个曾经袭杀古月方正的魔道蛊师的真正身份。”铁血冷开口道。

“哦?是这样……。”

“那个魔道蛊师,不就是熊家寨指使的么?”

“难道这其中有什么内幕不成?”

“不错。这名魔道蛊师的真正身份,其实是山脚村子中的一名猎人。只是机源巧合,成了一名魔道蛊师。他名叫王二,之所以袭杀古月方正,却是因为方正的亲身哥哥方源!”说着,铁若男紧紧地逼视方源。

“哥哥,想不到你竟然是这种人!”一旁的方正,攥紧拳头,眼中流露出一股愤怒。

“小姑娘,你这话是什么意思?”古月漠尘声音低沉下来。

“难道说,是方源雇佣了这魔道蛊师,来刺杀他的亲身弟弟方正不成?”古月药姬稳含兴奋之色。

就连古月博也面色动容,在座位上微微调整了姿势。

“你们想岔了。”铁若男却摇头“真实的情况,是方正滥杀无辜,杀害了王老汉一家,引起王二的报复。但王二并不知道方源有一个胞弟,他将方正误认为是方源,因此出手袭杀报仇。”

“小姑娘,凡事要讲究证据的。”有家老开口道。

“我当然有证据。古月江鹤,请你把你所知道的,都说出来吧。”铁若男有备而来,并不慌乱。

古月江鹤叹了一口气,他畏惧地看了铁家父女一眼,哆哆嗦嗦地走上前来,然后扑通一下跪在地上,哭号起来:“这是属下失职,请族长责罚!”

古月博脸色阴沉如水:“有什么事情,你先说出来。不得有任何一丝隐瞒!”

当初,方源杀死王老汉一家,正是古月江鹤的管辖范围,被赶到现场的他发现。因为考核评价的缘故,他将这件事情压了下来,隐瞒不报。没有想到,今日东窗事发,被铁若男捅破。

“事情是这样子的……”江鹤结结巴巴地说下去,内容倒没有添油加醋,十分朴实,符合实情。

此时此刻,他不敢撒谎。方源也是家老,他亦不敢夸大其词。

“没有想到,真相竟然是这样子的!”

“方源杀了王二的父亲,王二来报复,结果找到了方正。原来如此……。”

“方正是受了无妄之灾啊,替方源挡了一劫呢。”

众家老窃窃私语。

方正将拳头攥得更紧,心中有怒火在不断地升腾,他终于忍不住,对方源低吼起来:“哥哥,你怎么能这样子草营人命呢。那老人和女孩,都是无辜的凡人,你怎么能下得了手?!”

面对弟弟的职责,方源无动于衷,充耳未闻一般。

古月方正并不是重点。

方源看向铁家父女。能够在这么短时间内,就挖掘出这层真相,真不愧神捕之名。

不管他们用了什么蛊虫,何等手段,能令江鹤屈服,主动揭露出自己的秘密,的确是有本事。

这样的本事,让方源更加笃定,他杀死贾金生的事情,必然会被铁家父女揭破。这只是时间问题罢了。毕竟这是蛊的世界,可以用蛊来作案,亦可以用蛊来破案。

压力渐生啊,(未完待续。!!!

------------

\end{this_body}

