\newsection{再看开窍大典}    %第一百八十节:再看开窍大典

\begin{this_body}

“时候差不多了,不能再等了。”

夜幕早已降临,黑暗的房间中,方源睁开双眼。

他已经褪下蛊师服,穿了一身黑袍,遮住腿脚。再配合他养起来的黑色长发,走在黑夜里,宛若幽灵。

他在前世,就已经习惯长发。有些蛊虫,需要长发,威力才能体现。譬如黑鬓蛊,钢鬃蛊等等。

长发很方便,有时候需要改头换面,他就会将长发剪短。但短发只能靠蛊虫的力量,才能在短时间内生长。

不久前,他和古月漠尘协约。到如今,生铁蛊,以及那四万元石都分批次交到了方源的手中。唯有那株用来治疗的草蛊,还未到手。

“没有治疗蛊,也只能算了。怎能事事如意呢,现实充斥着无奈啊……”

方源叹息,站起身来,轻轻地推开房门,动用隐鳞蛊,消失在夜色之中。

他这也是情势所逼,不得不有所行动了。

铁家父女越逼越紧,他出使熊家寨的打算,也被阻止下来。

熊家寨的力量都大部分保存着,使得白家、古月家不敢去过分逼迫,因此索赔要求无疾而终,三方约定进行三族大比武。

虽然漠脉有意招揽方源,但此举却令方源深陷政治斗争当中,引起其余各个家老的敌视。

再加上越接近死亡,就越强大的白凝冰,整个局势对方源越来越不利,简直是已经把他逼上了绝境。

方源纵是老谋深算,但计谋也要靠自身实力来支撑。面对这样的局面,尽管他已经尽了最大心力,比前世同期有了长足进步,但是三转初阶的修为仍旧不够分量,难以破局。

“局势倾颓,唯有兵行险招,下一剂猛药!”方源左思右想,将心思放在了天元宝莲身上。

只要他摘下天元宝莲,地下溶洞的元泉就会废掉。家族必定会疯狂调查,但是除了调查之外呢?

元泉已经被废掉,就算是将天元宝莲取回来,重新放入元泉当中,甚至毁掉,也无法复原元泉。

家族要生存下去,还能有什么办法呢?

唯一的办法就只有一个,那就是夺取新的元泉!

但青茅山上的元泉,只有三道,各被三家占据。其中一道如果被方源毁了,那么摆在古月一族面前的只有一个选择,两个选项。

这个选择,就是战争。

两个选项,第一是白家,第二是熊家。

只有夺取了其中一家的元泉,古月一族才有存在的物质根基。

没有元泉的支撑,根本谈不上蛊师修行。

然而此举,危险重重。方源也是没有办法,chun秋蝉恢复得越来越快,空窍已经不堪重负。他没有多少时间,只能绝地反击,从死中求活,挣出一线生机。

……

厅堂中,点亮了灯火。

蛊师已经将留影存声蛊取来,但此蛊却被古月博捏在手中。

“铁神捕,对于我刚刚的请求,你意下如何?”古月博笑着道。

铁若男轻哼一声。

铁血冷沉吟片刻,点头道:“也罢,若查出方源真是那凶手,我也会留出充足的时间,让他有参加三族大比的时间。”

“父亲……”铁若男眼中闪现异色,这不是铁血冷的风格。

“呵呵呵。铁神捕一言九鼎,从不食言。在下完全信得过你,也谢谢你的体谅。”古月博笑容越发和煦,但心中却在冷哼。

古月药姬私自带领铁家父女,潜入地下溶洞,观看家族正史典籍,这事情他身为族长岂会不知?

只是大比在即,同时家族中政治斗争复杂,他按捺不发罢了。

铁血冷虽然是五转强者,但强大的力量,并不能扼杀古月博心中的不满。

“幸好最真实的内容,都记载在家族秘史当中,从来都只有历代族长掌握。那本正史,不过是给外人看的罢了。”古月博心中暗暗得意。

古月一族的史籍,分有正史和秘史。

正史收藏在地下溶洞的密室中,内容被后人粉饰遮掩,真真假假,掩人耳目。

而秘史,则记载着最真实的内容,没有一丝作假,甚至还有许多不可告人的秘闻,都记录在案。

就比如那血滴子如何召唤,正史上是绝对没有的,只有秘史上详细记载着。

“古月族长,我父已经答应了你的请求,现在该给我们看看影像了吧。”铁若男语气不佳。

“就算是铁神捕不答应,我也会竭力配合你们查案的。”古月博笑着澄清一句,轻轻一捏,就将留影存声蛊捏碎。

这蛊碎了,却化为一团七彩烟气,夹杂着各种噪音。

古月博张口轻轻一吹,这烟气便飘到一面墙壁上,然后没入进去。

仿佛是墨滴落入水面,洁白的墙壁上很快就出现了一片彩色印记。

印记越扩越大,形成一片影像,正是昔ri开窍大典的情形。

在这影像中,方正很快就找到了自己,还有许多熟悉的面孔。

这些面孔,都在用兴奋的目光四处观望周围溶洞,脸上都充斥着少年的稚嫩。宛若雏鸟刚刚振翅,飞出了巢穴。

“那就是方源……”同样的,铁若男也很快发现了方源。

方源在队伍中行走,虽然四处观看,但目光平淡而冷漠。站在同龄人中,仿佛鹤立鸡群。不过若不特意关注他,常人也不会发现他身上的这点异样。

但现在堂中众人,都将目光集中在他的身上,使得这点异样无所遁形。

“咦?这个方源,真的有古怪。”此刻,就连古月博,也不禁流露出微微讶异的神情。

墙壁上画面变化,众少年来到花海。

要开窍了,一个个少年跨越地下河流,走到对岸的花海中去。

希望蛊的光辉,此起彼伏。

少年们还有家老的声音,也清晰地传出来。

当时情景重现,使人身临其境一般。

先是古月漠北测出乙等资质,引起轰动。然后紧接着是古月赤城,也被测出是乙等。

“果然有古怪,这个古月赤城神色太过紧张,身躯僵硬,肤色并不寻常,只是地下溶洞光线不亮,但若仔细观察,就可发现他身上是涂了某种东西,来吸引希望蛊。呵呵,应该是作弊无疑了。”铁若男看到此处,心中已笃定。

但她紧接着眉头皱起来。

方源出场了,他淌过河水,登上对岸。

希望蛊的光辉,并不是很旺盛。希望越大,失望越大,少年和家老的失望叹息声,也随之传入众人耳帘。

一切似乎都很正常,但铁若男的眉头却皱得更紧。

方源的神色一直很平静。

在当时,他背对着家老和少年们,因此旁人看不清他的神色。但如今,铁若男作为旁观者,她发现方源的神色一直漠然,没有变化。

就像是,像是……他仿佛早就知道了这个结果一样!

“这不可能!设身处地想想,如果是我的话,十五岁身受众望,却被检查出只有丙等资质。怎么可能没有一丝的失落、失望、气馁呢?怎么会这样,怎么会这样?”铁若男的眉头几乎都要拧成了疙瘩。

巨大的疑云,笼罩在她的心头,让她的呼吸都压抑几分。

她的心脏,在此刻砰砰直跳。一个个的念头,在她脑海中如电光闪现。

究竟为什么?

怎么会这样?

“等一等,资质……难道说?!”铁若男猛地抬头,直觉得自己心中一炸,一个极为大胆而疯狂的猜测,浮现在她的脑中。

……

灿烂的光影,此刻映照在方正的脸上。

开窍大典对他来讲,是他人生最重要的转折点。

开窍大典之前,他有一种人生,卑微渺小,默默无闻。开窍大典之后,他有另一种人生,光芒照耀,自信勃发。

在他的印象中,开窍大典是模糊的,稀里糊涂就过来了。

此刻,方正以旁观者的角度,再看这场盛典,他心中的复杂情怀,难以用言语表述。

方正看到自己登场,那时候的自己是多么的自卑和软弱呀。

紧接着,他又看到自己跌倒在河水中,挣扎扑腾,被哥哥方源拎起来,满身是水狼狈不堪的样子。

他的嘴角浮现起笑容,这就是曾经的自己啊,受到多少人的嘲弄!

然后他看到自己踏入彼岸,闷头走着,身上笼罩的希望之光,让无数人震惊赞叹。

这是荣耀一刻,这是奇迹一刻!

甲等资质,从此天地都不同!

“方正,我有一个问题,想要问你。”铁若男忽然开口,打乱方正心中情怀。

“什么事情?请问吧,我一定知无不言言无不尽。”方正回首,面带微笑。

“是你哥哥的事情。在你哥哥搀扶你的时候,我看到他嘴唇翕动,但周围的声音太嘈杂了。他对你说了什么,能如实地告诉我吗?”铁若男目光灼灼地追问道。

“当时的话……”方正陷入回忆,“好像是说……路?”

“对了。就是路。”方正眼中一阵清明,“我记起来了。他对我说:未来的路会很精彩。嗯?好奇怪啊,当时我没有觉得,现在回忆起来,感觉哥哥的这句话含义很深,仿佛,仿佛他事先知道我有甲等资质似的!”

“不,他不是知道甲等资质,而是另有所指。”铁若男娇躯一震,神情复杂,吐出一口浊气。未完待续。

♂♂

------------

\end{this_body}

