\newsection{你放心,我会饶了你们的}    %第六十七节:你放心,我会饶了你们的

\begin{this_body}

%1
树,根扎山土,将翠绿的手伸向天空。

%2
一棵棵粗壮的树干,遥相对望。繁茂的枝叶则在半空中交错。

%3
在这些树木的环绕之下,有一间木屋。

%4
木屋全有粗壮的树干打造而成,透露出一股敦厚坚实的味道。木屋并不是新近打造的,而已经有些年岁,因此木屋的表面爬着青苔,甚至有些树干上还发出了细嫩的枝丫。

%5
木屋一周,是用青矛竹围成的高大的竹栅栏。前后边都是菜地,菜地中央是打造出的一口井。

%6
此时,一位年轻貌美的姑娘,在井边打着水。

%7
她身上的服装虽然极为朴素,但亦难以遮掩她的容颜。她年方二八,一双乌溜溜的大眼睛,黑白分明,纯净如水晶。

%8
阳光透过重叠繁茂的树叶,照在她的脸上,将她的肌肤映照得宛若白雪,又显现出一种般透明的红晕和温润。

%9
她黑色的发鬓,调皮地垂下,将可爱的耳垂半遮半掩。

%10
粉红的双唇此时抿着,咬着一口贝齿,脸上都是努力的神情。

%11
她吃力地将装得满满的水桶,从井中深处提了上来。又提起一口气,将木制的水桶挪到了井边的灰砖地上。

%12
“呼!”少女鼓起粉嫩的腮帮,吐出一口浊气,又伸出白皙的手当做扇子,对着自己的脸颊扇着风。

%13
听到木桶顿在地上的声音,木屋的门吱呀一声打开了,从里面走出一个老人。

%14
老人头发黑白掺杂,脸上皱纹叠叠,一对老眼虽然沧桑,但是偶然间会闪烁出一抹厉芒。就像是一头年迈的老虎,虽然老了,但是虎威犹在。

%15
“丫头,这水桶太重了,都说了让爹来提。你怎么又背着我偷偷浇菜?”老人看着井旁的少女,脸上流露出慈爱的神色。

%16
“爹!”少女甜甜地叫了一声,“你昨天打猎那么晚才回来,今天早上你就多睡一会儿嘛。不过是个水桶而已,你看,我这不就提上来了吗?”

%17
“你啊,就爱逞强!”老人语气透着无奈,目光中又带着宠溺。

%18
他迈开大步,走到井边,伸出一只手,轻而易举地就抓起了水桶:“来,丫头,爹和你一起浇菜。”

%19
空气中充满了野草野花的芬芳,夏日的风,热烈地吹过来,拂过树梢,就转为了一股清新和深幽。

%20
山中木屋前的菜地上,女儿用瓢舀水,弯着腰,细心地浇着菜。父亲则负责打水,两个水桶轮流交换。一股家庭的温馨氛围,弥漫在这个小小的空间。

%21
“唉,终究是老了,提了几下,就提不动了。”过了片刻,老人站在井边,擦了擦头上的汗渍,深深地叹了一口气。

%22
少女回过头来,笑魇如花,嗔道:“爹,你终于知道了呀。已经上了岁数,一天到晚就爱逞强,告诉你多少回,打猎就让二哥去嘛,你这把年纪就该躺在家里,享享清福了。”

%23
“呵呵呵。”老人笑起来,点点头,“依你二哥的本事,闯荡这片山林,的确已经足够了。尤其是他那一手箭术,比我年轻时还好。不过有一点我还不放心他,他心太野了,自恃武力,一心想飞。唉,年轻人爱幻想,多少都有这个毛病。”

%24
“爹……”少女拖长了音调。

%25
老人笑得更欢畅了,打趣道:“对,还有你。你年纪不小了,也该找个婆家了。爹为你好好物色物色,咱闺女长得是这片儿的独一份,不愁找不到好人家!”

%26
少女脸上陡然升起了两朵红云,顿时羞得说不出话来。

%27
老人遥望天空,像是望见了美好的未来,他悠然而叹:“等你二哥吃点亏,收敛了性子,我就收手了,再也不上山了。再给你找个好婆家,看着你嫁人生子,最好生个大胖小子,嘿嘿,你爹我带带孙子,就满足啦。人这一辈子啊,真的不容易。做猎户的,能有几个善始善终的?唉,年轻时候的伙伴,到现在已经都没了,就剩下你爹我了。”

%28
“爹。你这话说错了。”少女笑着安慰,“什么只剩下你了,你不是还有我们嘛。”

%29
“呵呵……嗯?”老人笑着,刚要说话,忽然听到了动静,猛地回首。

%30
竹制的栅栏小门,被人从外面猛地踹开。

%31
“你就是王老汉?”方源一脸冷酷,双目幽幽,右手上托这一团月光,当先而来。

%32
老人大吃一惊,看到方源手中的月光,连忙跪倒在地:“老汉拜见蛊师大人!”

%33
“王老汉,你儿子居然敢冒犯我,已经被我杀了。把他的尸体带上来!”方源居高临下,盯着地上跪着的老人,直接开门见山。

%34
他话音刚落,从栅栏外就走进来两个年轻猎手,他们一人在前,一人在后抬着王二的尸体。

%35
看到这个尸体,王老汉身躯猛地一颤!

%36
“二哥——!”少女则凄凉地大叫一声,立即冲了过去,扑上王二的尸体,瞬间泪流满面。

%37
“王家妹子……”两个年轻的猎手,看到心仪的女子在自己面前如此痛哭,心中都很不忍,想要劝说什么,却说不出口。

%38
“王老汉,我听说你是猎头,附近几个村子中最出名的猎人。年纪这么大,还能上山打猎,每次都是收获丰富。这很好。”

%39
方源说到这里顿了顿,又面无表情地继续道:“你现在就给我画一张图,标明这附近山上所有的陷阱位置,还有你这些年来打猎过程中,观察总结的野兽分布情况。你画出来,我就赦免你儿子冒犯我的大罪。画不出来——哼。”

%40
这些村子,都受着古月山寨的控制。村子里的村民,都是古月一族的农奴。

%41
现在王二以下犯上,冒犯主子,按照族规,全家都得受到牵连!

%42
王老汉身躯再次一抖,差点瘫倒在地上。这样的打击,对于他来讲,实在是太大太突然了。

%43
“凶手,你是杀害我二哥的凶手!你草菅人命,还来兴师问罪?我要为我二哥报仇!!”少女大叫着,声音中充满了对方源的憎恨和愤怒,她猛地冲向方源。

%44
但是她还在途中,就被一个身影猛地扑出,将其阻挡下来。

%45
阻挡她的不是那两个年轻猎人,而是她的老父亲。

%46
“混账东西!”王老汉发出一声低沉的怒吼,甩手一个巴掌,啪的一声,就把少女打倒在地上。

%47
“你二哥已经死了,难道你也想死吗?难道你想让我这个当爹的,孤苦终老吗?!”老人说着这话,老泪纵横,浑身都在剧烈的颤抖。

%48
“爹!”少女被打清醒了,双眼泪如泉涌,声音中充满了痛楚、不甘、委屈、可怜、仇恨、无奈种种复杂的情绪。

%49
老人转过身,面对方源,双膝一软,又跪倒在地上,并且额头触地,深深地跪拜在方源的脚下。他的声音哽咽着,颤抖着:“蛊师大人在上,我那儿子冒犯了您,的确死有余辜!老朽这就为您画出地图,您宽宏大量,请您饶了我们罢。”

%50
方源脸色温和了一丝,他居高临下地看着老人道:“你放心,只要你如实画出来,我一定会饶了你们。不过你最好不要骗我,若是让我发现一丝的疑点,你们的性命就难保了!”

%51
“老汉明白,老汉明白。”王猎头磕头不止,“请蛊师大人,容小的回去拿纸和笔。”

%52
“不用。”方源摆手,望着此行威逼过来的两个年轻猎手,命令道,“你们去屋里,给我搜出纸笔,带出来。”

%53
“是,蛊师大人。”两个年青人两股颤颤,在方源的威慑下,不敢有丝毫的违抗。

%54
“大人,纸笔就在厨房的方桌上。”王老汉在旁道。

%55
方源目光幽幽一闪,没有说话。

%56
两个猎手闯进木屋,很快就拿着纸笔出来。

%57
南疆这里的纸,都是特制的竹纸,纸质很硬,纸色带着淡淡的绿。这样的纸,才适合南疆气候潮湿的环境。

%58
若是宣纸,只怕七八天之后,就被潮气打湿了。

%59
老人提着笔,跪在地上,画出一道道的黑色线条。或弯曲,或笔直。

%60
他足足画了十多张竹纸,片刻之后,双手将这些竹纸奉给方源。

%61
方源匆匆一览,就将这些竹分开两半,递给那两个年轻的猎手:“你们看看,有什么不妥的地方。每查出一个错误来,我就奖赏你们一块元石!”

%62
(ps:就像序言所讲,本书会很邪恶,大家把它当做一个纯粹的故事看就行,没有必要代入到现实当中。还有一个问题要说明一下,人祖的故事是一条暗线,写出来不是为了教育大家,没这心思也没这功夫更没这资本,只是作为一个世界观的展现,对世界底蕴的一种侧面描写,同时通过这个神话,也会抛出本书的部分设定。人祖的故事,会贯穿本书始终。大家看看就行,不喜欢看就跳过吧。还是这句话,看书没必要代入太深,看得爽就行了。更不要代入错误,代入错误是自己找罪受啊。当然,不排除有受虐癖好的大大们。哈哈……)

\end{this_body}

