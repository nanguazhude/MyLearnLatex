\newsection{牺牲常在,而信念不死}    %第一百四十四节:牺牲常在,而信念不死

\begin{this_body}

“我对他说,一个人活着,会有千千万万的理由。你为什么活着,我回答不了,只有你自己才能够回答。自己去寻找罢。”古月博答道。

“那么族长您自己的答案是什么呢?”方正疑惑地眨眨眼,问道。

古月博呵呵一笑,在他眼中,方正和青书的形象似乎重合到了一起。曾经,古月青书也问了相似的问题。

族长沉吟了一番,回忆了一下,然后重复了当年的答案:“一个组织总会有牺牲,一个人从出生的那一刻起,就意味着死亡。在生死之间,人脆弱不堪,但有一样东西却能温暖人心,照耀我们的心灵。那便是爱——这就是我的答案。”

牺牲常有。

古月青书是他古月博的义子,抚养了许多年,此时牺牲,作为义父的古月博,自然心存悲痛。

但是作为族长,他看过太多的牺牲。

当有了生死的觉悟,悲伤和痛苦就变得能够接受了。

方正重新低下头来,沉默不语,似乎正在思考。

族长笑了笑,从书桌的抽屉中取出一封信,递给方正。

“这是古月青书的信,记录了他思考多年的答案,现在我将它交给你,你可以看看。这就是他的答案。”

毫无疑问,这封信在方正的心中,有着无以伦比的吸引力。

他当即拆开来,看到第一行时,眼泪就止不住地往下流。

正是古月青书熟悉的笔迹。字里行间中透着他独有的温柔气息。

信中的开头,记录着他的迷茫和痛楚。

然后是这些年来,他不断思索的过程,给他触动的事件。

方正看着这封信,就仿佛历经古月青书的一生。他跟随着古月青书的人生步伐,一直读到信的末尾。

在末尾记录着这么一段话。

“家族就仿佛是一片森林,我们每一个成员就如同森林中的一棵树木。老树延伸枝干。为新生的树苗遮风挡雨。当新生的树木长成高大的参天大树时,老树将倒下化为泥土中的养分,滋养土地。孕育新的树木。人总是会死的,天和地不会记住我们。但是新生的树木,将是老树存在过的见证。就在这样不断地见证中。家族的这片森林将越加广袤,走向昌盛和繁荣。”

“人总是会死的。作为蛊师,亦难掩死亡的结局。即便是七转、八转,甚至九转的蛊师,也只是活得更长一些罢了。面对死亡,我感到恐惧。但我深深的明白,终于有一天,我古月青书也会死亡。或许生老病死,或许战死沙场。但愿到那时,我能平静而又无憾地走向死亡。”

在信的最后。

“义父大人。我曾经问你的那个问题,我想我已经找到了答案。”

看完这封信,方正泣不成声。

他的脑海中,充满了对青书的回忆。犯了错,他不责怪反而宽慰的话。受了挫。他鼓励的目光。失意之下,他用温暖的手摸摸自己的脑袋。

古月博收起这封信:“将来,当你思考出答案后,也可以写一封信,告诉我。去吧,回去好好休息一下。狼潮的危机还没有结束,需要你贡献一份力量。”

“不。”方正缓缓抬起头来,捏紧双拳。

“怎么?”古月博问道。

“我已经找到了我的答案。”方正的语气中透着难以言喻的坚毅,“我想要力量!用来守护身边的亲人,保护他们再不受伤害。我想守护家族,壮大家族!我想要看到狼潮再不能折磨我们,想要看到同伴们欣慰快乐和幸福!这样的悲痛,我不想再上演再继续。我要用我的这双手,我的身躯,我的灵魂,来守护住身边的人!”

古月博露出意外的神情。在这一刻,他仿佛看到了古月青书。

“青书,你并没有白死……”看着方正熠熠闪光的双眼,族长在心中长叹一声。

一株老树倒下,就在它渐渐腐烂的泥土中,一株新苗已经开始飞速地成长。

……

人祖难以忍受孤独之心,因此抠下双眼,化为一儿一女。这才稍稍排解了孤苦寂寞。

但是好景不长,这对子女渐渐贪恋起世间的景色,将父亲人祖望之脑后,常常嬉戏玩耍忘了时间,更忘了照顾人祖。

人祖看不见任何的东西,眼中一片黑暗。

但有时候,他竟能看到一点点的光亮。

对此,他疑惑万分,向态度蛊请教。

态度蛊便告诉他说:“哦,这是信念蛊发出来的不朽光芒。”

“信念?”白凝冰看到这里,哂笑一声,将手中这本记载着古代传说的书,随手抛飞出去。

房门恰巧打开,来人险些被这本书打中脸面。

“凝冰,你这是做什么?”进来的正是白家的族长。

他皱着眉头,劝慰道:“我知道你心情不好,但失了右臂,也算不了什么。这个世界上,有许多的蛊虫,都能治好你这样的伤势。”

“以前我让家老跟在你身边,保护你,你总是拒绝,甚至对家老大打出手。这次吃亏了吧?”

“不过这也是好事。从小到大,你走的太顺了,只要不是死亡牺牲,这点亏也没有什么大不了的。你的伤势已经无大碍了,但狼潮越来越凶猛,家族现在需要你的力量!”

“一群小狼崽子,算得了什么?”白凝冰闭上双眼,躺在床上,无所谓地道。

族长的脸上却显现出凝重的神色:“情况并不妙,甚至说是不容乐观。据侦察,已经有不下三只的狂电狼群,出没在山寨周围了。你的失败。给族人带来了极大的影响。我希望你今天傍晚,就要公开露面。只要你不倒下,就能大大振奋我族的士气。你听明白了吗?”

“明白了,明白了,这点小事而已。”白凝冰嘟囔着,一脸的不耐烦。

若换做其他人这般态度应对族长,恐怕早就遭到严惩了。但白凝冰却又是另一回事。

白家族长无奈地叹了一口气。关上房门,退了出去。

房间中又只剩白凝冰一人,他缓缓地睁开双眼。眼中流露出孤独迷茫之色。

他并没有告诉其他人,自己身体中的麻烦,以及对死亡的预感。

从家族的典籍中。他倒是查出北冥冰魄体的名称。在那些有限的资料中,十绝体也称之为必死天资,当窍壁到达极限后,自爆的威力将极为强大。

别看白家族长抚养了白凝冰长大,对他如此宽纵。若是将北冥冰魄体的事情说出来,白凝冰毫不怀疑,第一个想要杀他的就是这位白家的族长。

“一个人活着,究竟是为了什么?”

以前思考这个问题的时候,白凝冰会感到十分的迷茫。进而产生无聊、焦躁、愤懑等等负面情绪。

但是如今,他的心中却有了一股平静。

人总是会成长的。更何况他这样的天才。

以前,他心知自己必死无疑,在绝望中留恋生命,在内心深处对死亡存在一种惧怕。

但是如今,当他真的差点死了。他反而看开了。

任由三转的白银真元时刻温养着空窍,他也再不心焦。

皆因他不再惧怕死亡。

尽管他仍旧迷茫于生存的意义,但他却知道这个问题的答案在哪里。

这个答案,早就在方源的心中。

这种感觉,说不清道不明,玄而又玄。但他就是清楚。

况且,如今石窍蛊也落在了方源的手中。

“方源……我们必会再相见。”他轻声喃喃,眼中绽放出一抹闪亮的光辉,如同钻石般璀璨。

……

“石窍蛊……”在租房中,方源看着手中的蛊虫,陷入沉思。

石窍蛊形如骰子,正正方方,通体灰白,坚硬无比。

这蛊是消耗蛊,用一次就消失,作用是将蛊师的窍壁转变成坚硬的石壁。

此举是将空窍的底蕴和潜力完全榨干,令蛊师能在瞬间达到巅峰修为。

举个例子,方源如今是二转中阶的修为。用了此蛊之后,就能在瞬间达到二转巅峰的境界。

但因此付出的代价是,方源终生将没有晋升三转的可能性,同时丧失自我回复真元的能力。今后,只能利用元石进行补充。

石窍蛊是给一些走投无路的蛊师们使用。一些蛊师的空窍受到了难以恢复的损伤,有了裂痕,却无法医治,只好使用这蛊。

或者在特殊的情况下,蛊师晋升无望,必须在短时间之内提升修为,才能活命。因此就使用石窍蛊了。

“合炼石窍蛊的成本相当高。白凝冰合炼这只蛊虫,恐怕是想将自身的空窍转变成石窍,来阻止死亡的降临。可惜这个方法,只能延缓死亡的到来,根本阻挡不了自我的毁灭。北冥冰魄体如果能这么简单就破解的话,那还称为什么十绝体呢?”

这石窍蛊对方源并无用处,倒是从白凝冰身上得来的赤铁舍利蛊,以及水罩蛊都有大用。

至于方源从古月蛮石等人身上,收刮出来的蛊虫,都并不出色,回到山寨后就上缴给了家族高层,换取了大量战功。

因为狼潮的关系,青书和白凝冰激战的事情被三家山寨都默契地压制下来,隐忍不发。三方都需要彼此的力量,来渡过难关。

熊林的汇报,让方源拥有白玉蛊的事实暴露出来。但他以在商队购买的理由,暂时搪塞了过去。

(ps:唉,想当年俺也是日更几万的强者,只要给我充足的时间!但是时间啊,总是不够用。俺也不是真正全职的写手,生活总是充满了各种各样的无奈、压迫和心酸。只是男人嘛,许多事情当然得自己咬牙挺着,何必向外人多说。俺向来不喜欢用此来博取同情,不喜欢这种弱小的感觉。)

\end{this_body}

