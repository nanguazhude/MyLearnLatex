\newsection{树屋藏酒虫}    %第一百零九节:树屋藏酒虫

\begin{this_body}

古月蛮石憾败于新人方源!

这个消息,很快就宣传开来,在二转蛊师当中掀起一阵小小的风浪。

事件的两个主人翁,大家都比较熟悉。

蛮石是小有名气的二转蛊师,两年前,曾经在白凝冰的手下逃得性命,不容小觑。

而方源则是本届的第一,年末考核之时,许多人目睹了他击败方正的情景。而后又因为他继承了遗产,一夜暴富,更让众人眼红嫉妒。

双方的差距应该很明显才对,但是偏偏强者蛮石败给了弱小的方源。这样的反差,实在有些叫人大跌眼镜。

许多人都纷纷打听事情的经过,方源因此声名鹊起。

二转蛊师们都开始正视这个年轻的后辈新人。

“居然一言不发,就动手了。小年轻,容易冲动。”

“手中有财富,合成了月芒蛊,也算是小有能力了。”

“这是个疯子,行事狠辣。据说古月蛮石败走后,躺在床上足足修养了三天!”

人们评论着方源。

虽然他和蛮石交战的过程中,忽然出手,占据先机,首先就将蛮石重创,确立了极大优势,似乎有些胜之不武的味道。

但是胜利就是胜利,失败就是失败。

结果说明一切。

或许在地球上,会有许多人注重过程,不注重结果。但在这个世界,生活困苦艰辛,周围充满了致命的危险。胜了往往就表示活着。败了就是死去,丧失一切。

胜者为王,败者为寇的理念,得到了几乎所有人的强烈认同。

方源胜利了,不管过程如何,事实就是这样。

一个新人崛起了,踩在蛮石的肩膀上。正式踏入众人的视野当中。

而蛮石则成了垫脚石,名声毁于一旦,回去之后。就辞去了组长的职务。

这就是失败者的下场。

亲人们会同情这样的失败者,但是他们更崇拜和认可胜利者。胜利者代表着强大,而强大则意味着人们的生命更加安全。

此件事情之后。古月冻土也明智地停止了小动作。

古月蛮石的下场,终于让精明的舅父认识到现实。方源的成长,让他无奈、愤恨、不甘。

他知道自己再无可能夺回这笔遗产了。再坚持下去,毫无意义。

自己动用关系网,雇佣其他人找方源的麻烦,这是在消耗元石。而方源则财源滚滚。

一旦僵持下去,哪怕他有大量的元石积蓄,最终失败的人也一定是他。

因为他失去了竹楼、酒肆和九叶生机草,已经成了无源之水,元石消耗出去就很难补充回来了。反观方源。虽然元石缺少,但却日渐增多。

更关键的是,古月冻土沮丧地发现,这样的僵持毫无利益可言。

所以当他听到蛮石败逃的消息后,他立即停止了这种无谓的举动。

事实上。当方正闹事被方源化解之后,就已经意味着他古月冻土的失败。

这样一来,方源的酒肆又能照常运转了,算得上一件高兴事。

还有一件喜事,那就是商队的提前到来。

三月。

春光明媚,春天的轻歌踏着欢快的节拍而来。

春暖花开。草长莺飞。

青茅山上,放眼望去,都是新生的碧绿。有些向阳的山坡上,盛开的一朵朵野花儿,形成色彩斑斓而又绚丽的花海。汩汩流淌着的花流,似烈火般熊熊绽放,与阳光相互交织。

新生的龙丸蛐蛐从一颗颗的细卵慢慢地长成,组成一支支新的虫群,开始在夜间活跃。

而在白天,彩雀鹦鹉大群地出没,盘旋在半空中,喳喳地鸣叫。

阳春布德泽,万物生光辉。

在这样的风景下,一支商队缓缓地开进青茅山。

黑皮肥甲虫,缓慢地蠕动着,上面坐满了人和货物。

骄傲的驼鸡,七彩羽毛绚烂光鲜,拖着一辆辆的板车。山地大蜘蛛无视地形,翼蛇扭曲着身躯,蜿蜒前行,时而睁开双翼,飞翔一段距离。

一头宝气黄铜蟾,高达两米五,浑身橘黄色,打头而来,上面坐着的正是四转蛊师强者贾富。

知道商队进驻山寨后,方源在心中轻轻一叹:“又改变了。关于前世的记忆里,这支商队应该是在夏季才到。按照往年的惯例,也是夏季的时候商队到达这里。但是如今,却是提前了两三个月份,在春季就到了。并且规模更大。”

方源的重生,改变了自己的现状,同样也影响了周围,导致了未来发生了改变。

其实究其根本,还是因为他杀了贾金生。

在蒙骗了众人之后,贾富便误认为,贾金生的死是竞争对手贾贵的一场阴谋。

贾富回到家族之后,便采取了一系列的激进措施,这使得他们兄弟间的竞争变得更加激烈。

为了争取更优异的商队成绩,今年雪还没有完全解化时,贾家的几位兄弟,就争先恐后地出发,领着各支商队开始四面行商了。

族长古月博接见贾富。

两位四转的蛊师,各是两方的首脑。

“古月老哥,别来无恙乎?”贾富脸上堆着笑,热情洋溢,只是他的脸上增添了一道长长的疤痕。

“哈哈哈,贾富老弟,今年来的挺早。”古月博看着贾富脸上的伤疤,心头一动,却没有发问。

“早起的鸟儿有虫吃嘛。这次我带来了许多珍贵的货物,相信高贵的古月一族会有大量的需要。”贾富为了争取成绩,这次是下了血本。

“哦,这可是个好消息。”古月博目光一闪。又接着道,“正巧后天就是我族的开窍大典,邀请贾老弟观礼。”

“哈哈,能见证古月一族的繁荣昌盛,是在下的无比荣幸。”贾富立即抱拳,语气诚挚地道。

能邀请旁人观看本族的开窍大典,这是真的把这外人当做了贵宾对待。贾富从这个邀请上。感受到了古月一族的诚意。

“事实上,还有一件事情。”贾富欲言又止。

“贵客远来,有什么要求尽管说就是。我族一定竭力而为。”古月博道。

贾富叹息:“唉,还是贾金生的事情。这次特地从族中带了几个侦察能手,希望在调查中。能打开方便之门。”

古月博顿时露出了然的神色。

看来贾金生的死,让贾富在家产的竞争中陷入了尴尬的被动境地。听说,回去家族之后,贾富就和贾贵当众发生了口角,爆发了一场激战。他脸上的伤疤,很有可能就是那场激战留下的印记。

也难怪他开春时就跑了过来,可见他贾富双肩扛着的压力颇大啊。

方源在各个帐篷商铺间游走闲逛。

今年的商队规模,比往年任何一次都更加庞大。不仅是帐篷增多了,而且还出现了蛊屋。

蛊屋是大型商队才有的事物,往往一个大型商队。有两三座蛊屋。贾富的商队规模顶多只能算是中等,但是却有了一座蛊屋。

这座蛊屋,是一棵大树。

它高达十八米,名副其实的参天巨木。树根粗壮,根根虬枝如龙蛇纠缠。一小部分裸露在地表,其余则深深地根植于地表之下。

底部树干,直径有十米。往上递减,但是减少的幅度并不明显。褐色的树干,并不密实一体,而是在树干里开有三层空间。

树干表面。也开了窗户。阳光和清新的空气,透过窗户,进入树干的三层空间里去。

大树枝干稀疏,枝叶也显得稀少,只有树冠如盖,碧绿一片,郁郁葱葱。春风吹来,树叶摇晃,有轻微的沙沙之音。

这是最常见的一种屋蛊。

三转草木蛊,名为三星洞。

被后勤蛊师种下去,灌溉真元而顷刻长成。树干中的三层空间,就是三个上下排列的房间。防御力绝非帐篷之流可比。

绵延一片的帐篷里,一颗高大的巨树耸立其中,如同一座塔楼,颇有鹤立鸡群的意味。

巨树底部,特意开了两个宽敞的门户,供人进入。

方源顺着人流,走进巨树。

树屋里的三层空间,都被改造成商铺的格局。一座座柜台里,摆放着各式各样的蛊虫。

这些柜台,都是木头,是整个巨树的一部分。上面长了绿色的枝叶。三星洞树蛊可以按照蛊师的意愿,进行特定造型的生长。

除了这些柜台之外,还有圆凳,以及长椅,供顾客休息之用。

一名三转的后勤蛊师,不知在这巨树的什么地方,时刻操纵着,监控着。

一旦有什么人抢走柜台上的蛊虫时,他就操纵这巨树生长,立即就能闭合树干底部的大门,在瞬间打造成密室。无数的枝干会疯狂生长,形成密集的攻击。同时树屋中驻守的蛊师,也会参加战斗。

树屋比帐篷要安全得多,因此里面贩卖的商品,就更加贵重。

方源刚刚到达一层,首先看到中央的一个单独的柜台上,静静地摆放着一只酒虫。

已经有不少的蛊师围着这只酒虫,在评头论足,或者发出啧啧的赞叹之声。

方源扫视一圈,其他的柜台上亦摆放了许多较为珍贵的蛊虫。

有玉皮蛊、旋风蛊、痕石蛊等等。

这些蛊虫,都是能和月光蛊搭配起来,合炼成更高级数的蛊虫。

贾富虽然并不完全清楚这些合炼的秘方,但是这么多年行商,经验已经积累下来,知道古月一族对哪些蛊虫比较有需求。

“贾富行商自然不是针对古月山寨一家,由此可见,这一次他真的是动了全力,看来回去之后是被刺激到了。”方源看到这里,心头一动。(未完待续。如果您喜欢这部作品,欢迎您来投推荐票、月票,您的支持,就是我最大的动力。)

------------

\end{this_body}

