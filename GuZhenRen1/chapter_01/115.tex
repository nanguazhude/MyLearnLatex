\newsection{晋升中阶}    %第一百一十五节:晋升中阶

\begin{this_body}

光膜透亮,淡红色的真元海面,波涛生灭,潮起又落。

海面上,两只白胖胖的酒虫在吸水。海面上空,黑色瓢虫一般的黒豕蛊,在绕着悬停着的赤铁舍利蛊周围,不断地振翅飞旋。

白玉蛊如鹅卵石一般,沉在海底深处,一动不动。

春秋蝉则隐了身形,仍旧在沉眠休养。

“是时候了。”方源心念一动,海浪顿起,一股真元逆冲而上,直接灌入到赤铁舍利蛊当中。

赤铁舍利蛊顿时摇摇飞升,散发出一股赤红色的光芒。

很快,舍利蛊就仿佛是一颗冉冉升起的太阳,光芒映照在整个空窍的窍壁上。

光芒如火一般炽热,如刀剑般刺眼逼人。

黒豕蛊很快就受不了,扑通一声,钻入真元海中去了。

两只酒虫也没入了元海深处。

白玉蛊则在海底深处一闪一闪。

若按照正常手段,方源要进军二转中阶,只有用那水磨的功夫,不断催动淡红真元冲刷周围的光膜窍壁。

但是如今,赤铁舍利蛊爆发出气势磅礴的逼人红光,取代了淡红真元,直接灌注到周围的窍壁当中,效果惊人。

方源心神注视之下,就看到整个光膜,都以肉眼可见的速度在不断地增厚。

光膜中的光,凝结成一股股的光流,最后光膜渐变成水膜。白色的波光在上面流转不定。时而明亮时而晦暗。

这一刻,方源晋升中阶!

但是舍利蛊却仍旧在绽放着赤色华光。

光芒充斥着整个空窍,取代真元,不断地将精华和底蕴注入到方源的空窍当中去。

水膜全数接受过来,上面的波光盈盈如水,流动得越来越畅快。

这个过程,又持续了一刻钟左右。

赤铁舍利蛊彻底消耗了所有的底蕴,它的身躯变得透明。然后彻底消失在红光当中。

它一消失,逼人的红色光芒也陡然消散。

空窍又恢复了往日的平静。

只是水膜变得更加厚实,赤铁舍利蛊的这番作为,省去了方源大量的时间和苦功。

一丝绯红色的真元,出现在了元海之中。

这是二转中阶的真元,它比淡红色的初阶真元要更加凝练,沉在海底深处。萦绕在白玉蛊的周围。

赤铁舍利蛊,能直接增强空窍的底蕴。表现在蛊师身上。就是提升一个小境界的效果。

这种蛊虫自然是越早用越好。

蛊师修为越高,战斗力越高,生存几率就越大,同时完成的任务越多,赚的元石也就越多。对各个方面,都有有益的影响。

到达中阶之后,方源又取出几块元石。快速补充真元。直到将空窍中的真元海,全部积蓄成四成四的中阶绯红真元。他这才罢休。

半个小时之后,他再次踏入石林。深入中央。

一踏入猴群的警戒线,顿时石柱中冒出一只只愤怒的玉眼石猴。

它们吱吱大叫着,向方源扑去。

方源面不改色,大部分的注意力都盯在最高层的那个石洞上。

普通的玉眼石猴,只要不陷入它们的围攻当中,并不要紧。问题的关键,在于这只石猴王。

究竟有什么蛊虫寄居在它的身上呢?

这点方源也不好揣度。

方源一边徐徐后退,一边谨慎观察,但是这只石猴王一直没有露面。

方源心中暗暗觉得奇怪:“难道这只猴群没有猴王?如果存在猴王,家园被侵犯,它势必第一个出来。等一等,也许它已经出来了!”

他刚想到这里,空窍中原本一直沉睡着的春秋蝉,猛地浮现出来,身躯不停地颤抖着,发出一种微弱的,只回响在方源内心当中的惊鸣。

本命蛊示警!

这是当本命蛊感到蛊师的生命,受到强烈的威胁的时候,才会发生的现象。

霎时间,方源汗毛炸立。他想也不想,直接下意识地全力催动起白玉蛊。

他浑身都笼罩住一层白玉的光晕。

就在下一刻,一只比寻常石猴要大出三倍的石猴王,忽然在方源的左侧出现,尖锐的猴爪猛地抓在方源的左肩上。

砰的一声,石猴王的攻击被白玉蛊的防御抵挡住,无功而返。

受到攻击的这一刹那,方源空窍内的白玉蛊骤然一亮,蓦地吸收了多达半成的绯红真元。

这要换成方源二转初阶时,一成的淡红真元就消耗殆尽了。

由此可见,石猴王暴然偷袭的一击,是多么的阴损狠辣!

饶是方源,心性沉稳,此刻也不由地出了一身冷汗。若不是这些天他努力磨砺自己,将自己打磨到一种巅峰的战斗状态,还真要着了石猴王的道了。

刚刚如果有一点点的反应不及,那方源的左肩势必就不保,整个左臂就不能再用了。方源的下场,就会和不久前的古月蛮石一样凄惨了。

“这个石猴王的身上,竟然寄居着一只可以令其隐身的野生蛊虫!”方源急速爆退,他没有能侦测隐形的蛊虫,一下子就落入了下风。

那只石猴王似乎也比普通的石猴要更加狡诈,一击不中之后,就重新隐去了身形,不知道藏到哪里去了。

这无疑带给方源一股庞大的心理压力。

他催动白玉蛊,形成全身防御,每时每刻都在消耗真元。他不可能一直维持着它。

就算是曾经他和石猴群战斗,也是在关键时刻,才启动白玉蛊进行防御。

如果一直维持着这样的状态,那么过不了多久,他的真元就要被消耗殆尽了。

五百多只的石猴,气势汹汹,向方源围剿过来。

方源尽最大速度,往后退去,拉开距离。

一些石猴的气势越来越弱,一些石猴顿足原处,开始回首望着家园。

“吱!”就在这时,玉眼石猴王又现出了身形,大声号令。

“吱吱!!”石猴群立即响应,迷茫和犹豫顿消,重新对方源展开追杀。

看着五百多只玉眼石猴,锲而不舍地向自己追杀过来,方源并不慌乱,反而嘴角流露出一丝冷笑。

这个变故,他早已经在预料当中。

他向石林中央深入,只是选择了一个最容易的路线,打通了一个通道罢了。在通道的周围,还生存着大量的石猴群。

这个通道,对于方源来讲,十分熟悉。

但是对于这些智力不高的石猴,它们怎么可能知道?在石猴王的督促下,石猴群在石林中横冲直闯,自然入侵了其他猴群的警戒线,很快就惹来了其他猴群的反击。

石林陷入大混乱!

无数的玉眼石猴站在自己的立场上,保家卫国,开始自相残杀。

若是再等个十几年,石猴王说不定就能成长为千兽王,完全统一这片石林。但是现在,它只是百兽王,还没有能力驾驭住这么多的石猴。

不同的石猴群,相互之间,陷入了大乱斗当中。

一时间,方源耳中全是石猴吱吱喳喳的乱叫。

追杀他的五百多只石猴,很快就被其他石猴群绊住。但是那头石猴王却是对方源紧追不舍。

方源且战且退,这期间,石猴王多次偷袭他,每一次都造成他真元的大量损耗。幸亏他在这之前,晋升到了中阶。否则初阶的那点真元,怎么能支撑这样的消耗。

方源陷入绝对的下风,他捕捉不到石猴王的破绽。

唯一反击的机会,在于石猴王攻击自己的一瞬间。但是方源即便反应过来,也来不及做出反击的动作。

石猴王有着隐身蛊虫,牢牢地掌握了主动权。想什么时候袭击方源,就什么时候袭击。就算是方源斩伤了它,它也能利用隐身蛊虫安然逃遁,可以说已经立于不败之地。

“我没有侦破隐形的蛊虫,这场战斗胜算极小!若是有个大范围的攻击手段,兴许可以一试。但是月芒蛊……除非走了运气,正好击中这只石猴王,但这可能性实在太小了。”

方源洞悉了战局,立即就要撤退。

但是石猴王却铁了心似的,要击杀他。

方源退到距离石门一百米远的距离,忽然停住脚步。

“我的真元只剩下一成多一点。一百米的距离,根本支撑不到。就算是进入了第二密室,关上石门,这只石猴王也有可能破门而入!”

方源原本以为,这只石猴王追杀自己这么长一段时间,应该要放弃了。但没有想到它竟然还是这般执着。

此时,他已经退出了石林,周围是一片空地。

无数的石猴,在石林中乱战,群情沸腾,它们发出嘈杂无比的声音,嗡嗡地回荡在这片山体空间当中。

方源不再动弹,一股战斗直觉告诉他,石猴王就隐身在某处地方,等待着他的破绽,然后实施致命一击。

方源知道自己已经陷入绝境。

若是普通的二转蛊师,此刻恐怕已经崩溃了,受不了这种无形的压力。

但是方源依旧冷静。

这个情形,亦在他的料想当中,只是可能性很小罢了。按照道理来讲,石猴王也具备着石猴的习性,留恋家园。但这只石猴王不知道为什么,非要追着方源杀不可。

“既然选择冒险,就要有付出生命的觉悟!”方源眼中寒芒一闪,开始脱下上衣。(未完待续。。)

\end{this_body}

