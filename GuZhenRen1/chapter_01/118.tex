\newsection{吞江蟾的传说}    %第一百一十八节:吞江蟾的传说

\begin{this_body}

据说,每一只吞江蟾的肚子里都存着一条江河。

方源前世并没有用过吞江蟾,但是这只蛊他印象很深刻。这都源于一人。

一个普通人,一个家奴。

前世的两百多年后,出现了一名极特殊的蛊师——江凡。

他的存在,让蛊师们大跌眼镜,让凡人们争相传诵。

他一出现,就成了传奇。

造就他的,就是一只吞江蟾。

江凡本是一介家奴,独自一人替主人掌管了一片渔场。有一天,一只吞江蟾搁浅到河滩上,肚皮朝上,仰头躺着,一直在沉睡。

江凡起初又惊又怕,但是慢慢地,他觉得这只是不是蟾死了,怎么一动都不动?

“蟾尸”堵着上流的河水,给掌管渔场的江凡造成了相当大的困扰。

江凡千方百计,想把这“蟾尸”弄走。他不过一介凡人,哪里能弄得走这般沉重的吞江蟾呢。

主人家苛刻残暴,完不成每月规定的份额,是要掉脑袋的。江凡不敢禀告上去,不久前就有一人,没有完成份额,禀告了一个正当的原因,结果却被主家当场杀死。

眼看着日期将近,“蟾尸”却一直堵着,极大地影响着他的收益。不由地,江凡越来越恐慌,脾气也越来越暴躁。

他虽然知道自己搬不动这“蟾尸”,但是每天都会过去,对“蟾尸”拳打脚踢,又哭又闹。宣泄着死亡来临前的惊恐和愤怒。

然而有一天,吞江蟾忽然醒来,睁开朦胧的睡眼,盯着江凡看。

江凡当场吓得腿都软了。

吞江蟾半睡半醒,仍旧躺在那里“挺尸”。江凡过了良久,才重新镇定。

他不害怕了,一个将死之人。还有什么好怕的?

他直接爬到吞江蟾的肚皮上,仰头躺下,望着星空:“蛤蟆呀蛤蟆。你也和我一样,只留着一口气,快要死了吗?”

他哪里知道吞江蟾的习性。看着吞江蟾半死不活的样子,只以为它也是奄奄一息。江凡说着说着,就泪流满面。

吞江蟾半眯着双眼,听着江凡的话,也看着星空。

此后几天,他每天都躺倒在吞江蟾雪白柔软的肚皮上,一边哭泣,一边说话,倾吐着一个凡人的痛楚和压抑。

终于,到达了期限。管事从山寨处下来,来到他的渔场,想要收鱼。

江凡哪有鱼来上缴?万般无奈之下,只得推脱需要时间收拾一下,然后又跑到吞江蟾那边告别。

他拍着吞江蟾的肚皮。道:“老蟾啊,想不到我先要死了。能够和你相识,也算是一场缘分。希望你最后的日子里,也能好过一些吧。”

就在这时,吞江蟾开始动了。

江凡吓了一大跳,吞江蟾动作越来越大。他连忙跳了下来。

扑通!

吞江蟾翻过身来,肚子朝下,背部朝上,它终于完全醒了。

江凡浑身都湿透了,看到这一幕气得直跺脚:“老蟾啊老蟾,原来你能动啊。啊呀呀,你可害死我了,你早动几天,我就不用死了!”

吞江蟾并没理会他的话,它醒来了,就感到肚子饿了。

它半个身躯沉入水中,然后张开大口,开始吞吸河水进食。

这一刻的情景,看得江凡目瞪口呆。他震惊地看着河水的水位,以肉眼可见的速度,迅速下降。

大量的河水被吞江蟾吸进了肚皮里去,但它的肚皮半分不见涨大,好像里面是一个无底洞。

半晌之后,吞江蟾才悠然地停止了进食。河水已经暴降到很低的程度,充满淤泥的河床绝大部分都裸露出来。人站在河底,河水仅仅只达到人的膝盖处。

江凡站在河岸出,呆若木鸡。

吞江蟾看了他一眼,忽的打了一个饱嗝。然后肚皮一鼓一缩,嘴巴张得老大,猛地向外喷吐出大量的河鲜。

什么鱼虾龟鳖,田螺黄鳝大螃蟹应有尽有!

吞江蟾只是以水为食,并不吃这些河鲜水产,它将这些东西统统都吐了出来。

这一刻,天空就好像是下了一场河鲜豪雨。

转眼之间,这些河鲜就堆成了一座小山。江凡看到这里,狂喜得一蹦三尺高。他大叫道:“我有救了,我有救了!这些鱼鳖,足够我三个月的份额。老蟾啊老蟾,多亏了你呀!”

他将这河鲜都收拾起来,交到了管事的手中。

管事顿时又惊又疑,怎么会有这么多的份量?他赶忙禀告上去,山寨中的蛊师也察觉到了河水的突变。

调查之后,他们很快就发觉了吞江蟾的存在。

这可是五转的蛊啊!

山寨中一片恐慌,组成大部队,要驱赶吞江蟾。

江凡不愿吞江蟾遭到伤害,这些天他已经把吞江蟾当做了唯一的朋友看待。

他跪在蛊师们的面前,痛苦哀求。蛊师们哪里会把这个凡人看在眼里呢?一脚把他踹开,正要痛下杀手。

就在这时,吞江蟾赶了过来。

也不知道,它是把江凡也当做了朋友,还是觉得留着江凡在身边,比较有趣,能唠嗑解闷。

总之它出手了。

它背起江凡,吐出的江河席卷了整个山寨,淹没了大半个山峰。

这一战,震动南疆!

从此以后,江凡的名字传遍了十万大山。吞江蟾留在了他的身边,他拥有了一只五转的蛊虫!

要知道,就算是五转的蛊师,也未必有一只五转蛊虫。

五转蛊师稀少罕见,就算是古月一族整个历史上,也只出现过两人。一位是一代族长,第二位是四代族长。

但是他江凡,根本就没有开启空窍,只是一个普通人,却豢养了一头吞江蟾。

他的存在,震动了蛊师界。

后来,江凡在原来的山寨地址上。建立了一个村子。他宽容待人,对凡人抱有同情,立志建立一个人人平等。没有压迫的山寨。

他成了一面旗帜,周围的山寨中的普通人,都向他这边涌来。都想要依附于他。

但他最终,还是被人刺杀。

空有一只五转的吞江蟾,并不能让他真的成为强者。他终究不是蛊师,他死后,吞江蟾也走了。

蛊师们夷平了他的山寨,将那些胆大包天的凡人都屠杀个干净。

江凡以凡人之躯,挑战整个社会的体制,自然引起了蛊师们的愤怒。

“不知道这一世,因为我的影响,江凡还会不会出现。”回忆结束。方源笑了一笑。

赤山却笑不出来。

他沉着脸,铩羽而归。

山脚下的村民们,一直期待着蛊师大人能解决掉这个麻烦。

但是堂堂的古月赤山亲自出马,亦解决不了这个问题。这让村民的恐慌之情,迅速蔓延开来。达到了高峰。

他们拖家带口,带着大包小包,纷纷涌上了山寨。他们自然不敢擅闯山寨,所以越来越多的村民,跪在了山寨大门处,祈求蛊师大人们开恩。放他们进来。

大厅中。

“什么?这样的一群贱民,居然敢包围了大门。真是岂有此理,胆子越来越大了,杀了,都杀了!”刑堂家老咆哮着。

药堂家老古月药姬脸色也阴沉着:“这群贱民虽然死不足惜,但杀一就能儆百。杀了几个不顺眼的,就能驱散了这人群了。可是却让其他山寨看了一个笑话。”

古月赤练道:“现在的关键,还不是这个。若是连赤山都推不醒这吞江蟾,我族还有什么人能行?看来真的要请援兵了。熊家寨以力量见长,唉,为了山寨的安危,请他们出手,即便花费一点代价,也是值得的。”

这番话引起了其他家老的赞同,族长古月博也是意动。

“族长以及诸位家老大人,晚辈有事禀告。”古月赤山立在堂中,听着家老们的话,忽然行了一礼,开口道。

古月博点点头,他对古月赤山亦是抱有欣赏的态度:“赤山,你有什么建议,不妨事,都说来。”

赤山不答反问:“诸位大人,要推醒这头吞江蟾,是不是非得靠一个人的力量?”

古月博:“按照上代族长偶然间提到过,吞江蟾脾气温顺,且嗜睡。就算被推动身躯而醒来,也不会发怒。所以,才命山寨中力气最大的你,去推醒它。结果却是失败了。”

赤山便道:“那就请族长大人,赐我一只蛮力天牛蛊。有了这一牛之力,再加上晚辈的这天生的气力,定然能推动这只吞江蟾。”

“绝不能动用蛊虫的力量。”赤山话音刚落,就有一位家老断然否定了他的请求,“蛊虫的气息会引来吞江蟾的警惕,若是让这蟾感觉到威胁,而暴动起来,那结果谁负责?”

“不错。”古月博点点头道,“动用了蛊虫,就算是推醒了吞江蟾,也得不到它的认可。必须是单独一人,靠着自身的力量,推醒它,才能让它认可。”

蛊是天地真精,但是习性近乎于野兽。野兽有各自的领土,流浪的猛兽遭遇到这片领土的兽王,往往会展开一战。胜者夺得领地,败者则去流浪。

兽潮的形成,也是基于野兽的这个习性。强大的兽群,侵吞周围的领地。弱小的兽群被躯干出来,就形成了前期的兽潮。

要驱赶吞江蟾,就是在这习性上着手。

吞江蟾性情温顺,不爱争斗,只要让它认可了这片领土的“兽王”的能力,它就会退去。

因此动用蛊虫不行,蛊虫气息的会让吞江蟾觉察到,后果难以预测。调集众人合力,也不行。人一多,即便推动了吞江蟾,它也不会走。

因为这是众人的力量,胜之不武,它不会认可。

所以,族长才让赤山前去走上一遭。皆因他的本身力气是古月山寨中最大的。

“原来如此,我了解了。”赤山终于弄懂原委,他抱拳道,“既是这样,那晚辈向诸位家老推荐一人,此人力气比我还大。”

“哦,是谁?”

“竟然还有此人,为何我们不知?”

“赤山,不要卖关子,快快说来!”

“此人就是古月方源。”赤山说出了一个名字。(未完待续。。)

\end{this_body}

