\newsection{青竹酒香,蛊师逞威}    %第十二节:青竹酒香,蛊师逞威

\begin{this_body}

“现在所有的问题,都归结在花酒行者的遗藏上。我若能找到它,一切困难就迎刃而解。若是发现不了,这些难题将极大地拖慢我修行的速度。会让我在修行之初,就被同龄人甩得远远的。费解啊,我用了一周多的时间,来吸引酒虫出现,为什么就一直不见成效?”

方源皱着眉头,苦思冥想着。吃到嘴里的饭菜,也不知道什么味道。

就在这时,一阵喧闹声传来,打断了他的思绪。

方源循声望去,发现是中央的那桌,六个猎户围着桌子,已经喝得酒气熏天,气氛热烈如火,各个脸红脖子粗。

“张老弟,来,再喝一杯!”

“峰大哥,兄弟们佩服你的本事,一个人就搞定一头黑皮野猪,真是好汉!这杯酒你必须得喝,不喝就是不给兄弟们面子。”

“谢兄弟们抬爱,但我实在喝不下了。”

“峰大哥喝不下,是嫌弃这酒不好吗?小二,你过来,给爷几个上好酒!”

声音越来越大,很显然几个猎户都喝多了。

跑堂的小二连忙走过去:“几位大哥,好酒是有,但是可有些贵呢。”

“怎么,怕爷几个不给钱怎么的?!”猎户们听了这话,好几个都站起来,瞪向店小二。他们不是五大三粗,身材魁梧,就是黑瘦粗壮,精悍逼人,带着山民特有的彪悍之气。

小二连忙打招呼,叫屈道:“小的哪敢瞧不起诸位英雄好汉,只是这酒真的有些贵,一坛可值两块元石呢!”

猎户们都一愣。

两块元石,那可不便宜,是寻常人家两个月的生活费。猎户虽然打猎,赚得比寻常凡人较多,有时候一头黑皮野猪,就值半块元石了。但这狩猎也是有风险的,有时候搞不好自己就成了猎物。

为喝一坛酒,就耗费两块元石,对猎户来讲,太不值得了。

“真有这么贵的酒?”

“小子,你可不是骗我们的吧?”

猎户们大呼小叫,但是声音都有点虚,有些下不了台的尴尬。

小儿连叫不敢。

那猎户中叫峰哥的,看场面不对,连忙打圆场:“诸位兄弟,不要再破费了。今天已经喝不下了,这酒改日再喝吧。”

“哥哥说哪里的话!”

“这哪成……”

其余猎户们叫着,但是声音已经渐渐弱小下去,一个个也都坐回座位上。

小二也是个精明人物,看这架势,也知道买卖做不成了。

不过这情形,他也已经见怪不怪。正要退走,冷不防那角落里的一桌,传来年轻人的声音:“呵呵,真是好笑,一个个瞎咋呼什么,买不起酒,就乖乖地闭嘴,缩一边去!”

那猎户们听了这话,其中一个顿时被刺激得大叫起来:“谁说我们买不起,小儿,就上那坛酒来,老子给你元石,不就两块嘛!”

“哎,客官稍等,这就来!”小二没料到峰回路转,立马接口,转身就下去抱了一坛酒上来。

这酒坛只有寻常酒坛的一半大小,但是拍开封泥,顿时就有一股清醇的酒香飘散出来,弥漫整个饭堂。

那坐在窗户边独酌的老人,也因为这酒香,不由地转过头来,将目光投放在这坛酒上。

的确是好酒。

“几位客官,不是小的吹牛啊。这可是上好的青竹酒,整个山寨就我们客栈独一家。你们闻闻这酒香!”店小二一边说着,一边深深地吸了一口气,满脸的享受和满足。

方源心中一动,这店小二说的也不算吹牛。

古月山寨中共有三家酒肆,卖的都是寻常的米酒、浊酒,种类大同小异。方源为了吸引酒虫现身,连续买了七天的酒水,自然清楚其中行情。

几个猎户望着面前的酒坛,都被勾起了酒瘾,一个个抽动起鼻子,滚动喉结。

而那个一时口快,买了酒的猎户,脸上神情更加精彩,多了一抹懊恼之色。

就这坛酒,可值两块元石呐!

“自己一时冲动,就买了此酒。这店小二也太不地道,立即就上了酒,现在封泥都开了,就算是想退货都不行了。”

猎户越想越心疼,想要退,却着实抹不开这面子。

最终只好拍了一下桌,强笑道:“妈的,这酒好!哥哥们,敞开了喝,今天这酒,兄弟我请了!”

恰在这时,那角落一桌的年轻人又发出一声嗤笑:“就这一小坛酒,哪够六个人喝的?有种的再买几坛啊。”

猎户被这话挤兑得青筋暴跳,腾地一下站起身来,勃然大怒,双目圆瞪向发话的年轻人:“小兔崽子,话挺多呀。来,站出来,来跟哥哥练两手!”

“哦?那我可站出来了。”青年听了这话,还真起了身,阴笑着走出角落阴影。

他身材高瘦,面皮苍白,穿得一身深蓝武服,显得干净利落。

他头上戴着宝蓝色的头带,上身穿着短衣,露出瘦弱的肩膀。下身穿着长裤,脚上是竹芒鞋,小腿处还有绑脚。

最关键的是,他腰间系着青布腰带。腰带中段镶嵌着一块闪亮的铜片,铜片上刻着黑色的“一”字。

“一转蛊师?!”叫嚣的猎户显然明白这身服饰所代表的意思,他倒抽一口冷气,脸上的怒色消退了,变成了惊惧。

他没有想到,自己居然招惹到了一名蛊师!

“你不是想找我练练手吗?来啊,动手啊。”青年蛊师踱步走来,带着一脸戏谑的笑。

但是刚刚挑衅的猎户,却像是个雕塑一样,站在原地一动不动。

“或者你们一起上,也行啊。”青年蛊师慢慢地走到猎户一桌,很随意地说着。

猎户们脸色都变了,一些喝红了脸的,霎时间脸色就白了。一个个额头都淌下了冷汗,坐立不安,大气都不敢喘一声。

青年蛊师伸出一只手,提起青竹酒的酒坛,放到鼻翼下闻了闻,笑起来:“还真是香啊……”

“蛊师大人若喜欢,拿过去喝好了。就当是小的冒犯大人,向大人赔罪。”挑衅叫嚣的那个猎户,连忙拱手行了个礼,脸上堆起笑容说道。

不料青年蛊师猛地变色,啪的一声,把酒坛摔到地上。

蛊师脸色如冰,目光如剑,低声地怒吼起来:“就凭你也有资格向我赔罪?你们这些猎户,真是有钱啊,比我还有钱啊,居然花了两块元石买酒喝?!你知不知道,我正为元石发愁呢!居然敢在这个时候,在我面前炫富!你们这些凡人也配?!”

“不敢,不敢!”

“冲撞了大人,我们罪该万死!”

“小的们都是无意冒犯啊,这是小的们身上的元石,请蛊师大人笑纳。”

猎户们都触电一般站起来,从怀中掏出元石。但是这些凡人,哪有什么钱财,掏出的都是零零碎碎的元石,最大的也超不过四分之一。

青年蛊师却没接过这些元石,只是不停地冷笑,用鹰隼般的目光,扫视整个饭堂。

被他扫视到的猎户,都一个个低下了头。窗前那桌看热闹的老人,也赶忙转头,避开蛊师的目光。

只有方源静静地看着,毫无顾忌。

这个青年蛊师一身的服饰,只有正式蛊师才能穿戴,就算是方源也还没资格。只有方源从学堂毕业之后,才能从家族中领取。

青年蛊师腰带铜片上的“一”字,表明了他一转蛊师的身份。

但他已经有二十好几岁的模样,从他身上散发出来的真元气息,应该是一转高阶。

十五岁开始修行,到了二十多岁,还只是一转高阶,这说明青年的资质只有丁等,比方源还要差一筹。很有可能,只是一位后勤蛊师,连战斗蛊师都算不上。

但就算如此,对付六个猎户壮汉,仍旧绰绰有余。

这就是蛊师和凡人之间的力量差距。

“有了力量,就能高高在上,这就是这个世界的本质。不,任何世界都是这样,大鱼吃小鱼小鱼吃虾米。只是这个世界表现得更赤裸裸一些罢了。”方源心中暗暗感慨。

“好了江牙,教训一下就得了,不要为难这些凡人,传出去你不嫌丢人,我还嫌丢人呢。”角落里坐着的另一个年轻人,这时开口道。

众人听她说话,这才知道这个年轻人是个女子。

名叫江牙的青年蛊师被女同伴说的没趣,停止了冷笑,看也不看猎户们供奉出来的零碎元石。这些元石加起来,还不够两块,他当然没有兴趣。

他一拂袖,走向原来座位,一边迈动脚步,一边放下狠话:“你们有种的继续喝,就喝青竹酒。我倒要看看谁还敢喝这酒?”

猎户们都低垂着头,被训斥得像六个乖孙子。

浓郁的酒香弥漫在整个饭堂,那买了酒的猎户闻着酒香,心疼得脸上肌肉抽动。

这酒他可是花了两块元石,却没喝上一口啊!

方源停下筷子,他已经吃饱了。闻着这股酒香,他目光闪烁了几下,忽然掏出两块元石,放在桌上,淡然道:“小二,给我上坛青竹酒。”

全场一愣。

那青年蛊师江牙顿时停下步子,嘴角一抽,丝了一口气。他刚刚放下狠话,方源就要了这坛酒,这不是专门拆他的台,打他的脸么?

他转过身,眯起双眼,用阴冷的目光射向方源。

方源坦然地和他对视,一脸淡然,毫无所惧。

江牙目光一闪,阴冷之气渐渐消退,他感受到了方源身上真元的气息。

他知道了方源的身份,顿时笑了起来,春风般和煦:“原来是位学弟。”

其他人恍然,顿时看向方源的目光已经发生了变化。

难怪这少年一点都不怕蛊师,原来他也是蛊师。虽然还在上着学堂,但是本质上已经不同了。

“蛊师大人,您的酒!”小二屁颠屁颠地跑过来,一脸谄笑着。

方源向青年蛊师江牙点点头,拎着这坛酒,走出了客栈。

------------

\end{this_body}

