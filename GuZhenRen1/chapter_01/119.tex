\newsection{此子顽劣,需要打磨}    %第一百一十九节:此子顽劣,需要打磨

\begin{this_body}

%1
“古月方源?”一听到这名字,众家老不禁面面相觑。

%2
他们却很清楚这人,事实上,自从一开始,古月方源的名字,就时不时地回响在他们的耳中。

%3
尤其是开窍大典之后,展开了蛊师修行,此子越加能折腾了,时常都会闹出一些事情来,惹人瞩目。

%4
“哦,我有印象。前段时间,就是此子卖了双亲的遗产,买了一只赤铁舍利蛊吧?”一位家老恍然道。

%5
古月赤练、古月漠尘听着这话,两人的脸色都有些难看。

%6
赤铁舍利蛊若是被赤山,或者漠颜用了,就能助其中一人登上二转巅峰,自此和古月青书分庭抗礼。

%7
这反映到高层来,不管对于漠脉,还是赤脉,都是一种政治上的胜利。没有想到,结果却被这败家小子坏了事!

%8
“不过话说回来,这个小子的确有股气力。当初在擂台上,两拳打破玉皮蛊的防御,将方正揍趴下,夺得此届的状元。”一个家老回忆道。

%9
这次轮到族长古月博脸色有一丝的些微难堪了。

%10
古月方正是他特意培养出来的,方正的失败,从某种意义上来讲,就是他族长一系的失败。

%11
只要入了体制,任何人都会被打上阵营的标签。政治中有中间派系,但绝没有无派系的人。

%12
“但真要论力量,恐怕他的气力,还比不上赤山你的吧?”赤练疑惑地问道。

%13
赤山恭声答道:“诸位家老也许有所不知,方源不仅买了赤铁舍利蛊。而且还收购了一只黒豕蛊。这几个月,他一直在购买野猪肉,喂养黒豕蛊,增强自身力量。有一次,我看到他在山坡上搬弄巨石,丈量自身的气力。我不知道,他最大的力气有多大。但就我所见。已知道他的气力绝不会输给我。”

%14
“原来是这样。方源这孩子,想不到已经成长到这一步了。”古月博点点头,道。“那就命令方源那组,再去试一试罢。”

%15
听到这里,内务堂家老脸色尴尬地从座位上起身。站了起来:“族长大人容禀,这方源至今孤单一人,还没有加入小组。”

%16
“这是什么意思?”古月博微微皱起眉头。

%17
“是这样的。自从第一次兽潮之后,他所在的小组几乎全军覆没,只剩下他独自一人存活。”内务堂家老回答道。

%18
“即便如此,那重组时,怎么没有算上他?”有家老好奇地问道。

%19
“唉!”内务堂家老深深地叹了口气,“这事我也叮嘱过他,但是他却没有加入小组的意向。老实讲,我很看不惯这小子。他最擅偷奸耍滑,恐怕是因为继承了那笔遗产,就丧失了奋斗之志。”

%20
“偷奸耍滑?这不可能吧,那他没有小组,如何完成族中每月规定的任务?”一位家老怀疑道。

%21
内务堂家老的脸色完全沉下来:“他每月都会接强制任务。但每一次都是失败。他的履历是我见过最差的,几乎都是任务失败的记录。我曾经专门找他谈过几次,但他仍旧我行我素,毫不悔改。但他并没有违反族中的规矩,致使我亦无妨惩处他这等顽劣刁钻的小子!”

%22
众家老听得面面相觑,他们也从来没有见过如此不求上进的后生晚辈。

%23
任务失败越多。就代表着在族中前途越小。

%24
“这小子糊涂啊……”

%25
“哼,简直是顽劣不堪!”

%26
“他这是在自毁前程!”

%27
“我若生出这么个惫懒晚辈,直接一巴掌把他拍死!”

%28
“好了。”古月博抬起手,制止了众家老的窃窃私语,脸上看不出喜怒。

%29
古月博目光环视一圈,最后停顿在内务堂家老的身上:“强制命令,古月方源前去吞江蟾处,让他出出力。此子顽劣且桀骜不驯,又自由散漫惯了,需要打磨。若是失败,也可借此稍作惩戒。”

%30
“遵命,族长大人。”内务堂家老连忙应道。

%31
……

%32
酒肆中人声鼎沸。

%33
“你们知道吗,就在刚刚,家族派遣赤山小组前去山脚,结果失败归来了。”

%34
“山脚下的村民都堵到了寨子门口,现在跪了一地呢。”

%35
“哼,这群贱民,一点见识都没有。吞江蟾是什么,那是五转蛊虫。真以为躲到寨子里就安全了?”

%36
虽然是这么说着,但是恐慌的氛围已经越来越浓郁。这些蛊师们都在强自镇定。

%37
方源又听了片刻,已经毫无新鲜的消息。正要起身离开,就在这时,酒肆中走进一人。

%38
他身躯高大,虎背熊腰,赤裸上身,肌肤赤红,肌肉贲发。

%39
正是古月赤山。

%40
酒肆中的议论声,顿时停息下来,无数双目光注视到赤山的身上。

%41
赤山不管这些视线,只是扫视一圈,看到了方源。

%42
“你原来在这里。”众目睽睽之下,他走到方源的面前,“走吧,家族已经下了强制命令,详情路上再说,先和我到山脚下走一遭吧。”

%43
方源目光闪了闪,这等强制命令他不好推却。再者就算是面对吞江蟾,危险性也不大,便点头答应下来。

%44
直到方源、赤山二人出了酒肆,酒肆中这才再次喧闹起来。

%45
“掌柜的,老天有眼啊。你看看,这现世报来的这么快!那可是五转的蛊虫啊,连其他蛊师大人都束手无策,他又这么年轻,过去了不就是白白送命么!”

%46
“原本想着,方源公子会和其他蛊师大人不一样,能体恤我们这些下人的辛酸和痛苦。哼,想不到也是一路货色。死吧死吧,死了也不可惜。”

%47
“掌柜的。你这伤也不算白挨的,赔上一条蛊师大人的命,绝对是赚了的。”

%48
掌柜老者的头部已经缠上了一圈圈的白色绷带,此时他口中哼哼不断,无力地靠在墙角处。

%49
几位伙计围着他,说着开解的话。

%50
老汉的眼中闪过一丝怨毒之色,听着这些话。心情稍微好了一些。

%51
但他听了一阵后,假意低声喝斥道:“都给我闭嘴,这话是我们能说的吗?不怕被别的蛊师听到掉脑袋吗!”

%52
伙计们都嬉笑着:“掌柜的你太小心了。酒肆里这么吵闹,谁会听到我们这么低的声音呢?”

%53
这话音刚落,坐在最近位置上的一个蛊师便插进话来。道:“我听到了。”

%54
掌柜的,和几个伙计顿时面色大变,惶恐到无以复加的地步。

%55
“大人……”掌柜老者也不顾头部的昏沉眩晕,赶紧走到这蛊师的身旁求饶。

%56
这位男蛊师却抬手,制止他的话。

%57
“你们说的很好,我喜欢听。方源这小畜生,死不足惜!再说说这样的话,说的好,我大大有赏!”男蛊师取出一块元石,啪的一声。拍在桌子上。

%58
如果方源在这里,就能认出这人。就是当初小兽潮的那名治疗蛊师,方源将他爱慕的女子当做了盾牌,挡在了身前。他因此深恨方源,一直不得排解。

%59
几位伙计面面相觑。一个胆子大的,看着桌子上的这块元石,眼睛都直了。

%60
男蛊师的三位同伴,却都皱起眉头,但是又不好明劝。只能听着几位伙计,争先恐后地痛骂方源的话。

%61
初秋。一片好风景。

%62
山林中,树叶一丛深,一丛浅,绿叶点点映着黄叶,黄叶淡淡衬着红叶。

%63
稻田里,一片片黄橙橙的稻谷,随着秋风泛起金涛。

%64
一些绿油油的菜地中,肥嫩的菜叶新鲜可人。

%65
方源从山腰一路疾驰,随着赤山小组来到山脚下,见到了这只五转的吞江蟾。

%66
它体型巨大,简直像是一座小山。仰躺在河床中,直接堵住了河道。上游积着河水,已经快要溢出河岸。而下游却几乎断流,只有浅浅的一小股滋润着河床。

%67
吞江蟾肚皮朝上,雪白细腻,笼罩着一层光泽。它的背部,则是晴空万里时蓝天的颜色。亦是光滑,没有寻常蛤蟆背上的疣粒。

%68
它此时躺着,呼呼大睡着。但是却没有打呼噜,睡得很安静很温和。

%69
感受到它的气息,方源空窍中的两只酒虫,都缩成了一团。黒豕蛊一扫欢快的气象,飞得很低。寄居在右手掌心的月芒蛊,则收敛了光芒。

%70
唯有春秋蝉,仍旧安然沉眠着。

%71
方源将月芒蛊,也收入空窍当中。只要他不主动催动这些蛊虫,就不会泄露出它们的气息出来。放在空窍当中,十分保险安全。

%72
“方源,接下来就看你的了。”赤山在一旁道。

%73
在来路上,他将大部分的情况都做了说明。

%74
方源也认可这个法子。当然最简单的方法,就是动用春秋蝉。只要六转蛊虫的气息一泄露,这只吞江蟾必定仓惶逃窜。

%75
当然,这也是因为吞江蟾不喜欢战斗的缘故。若是换做血河蟒这样残暴的蛊,春秋蝉的气息反而让它陷入狂暴,展开疯狂的攻击。

%76
方源站在河岸上,先试着推了推。吞江蟾皮肤滑腻,有一种使不上劲头的感觉。

%77
加之它体重死沉,根本没有移动一丝。

%78
“你行不行啊?”一旁,赤城说着风凉话。

%79
方源并不理会他,而是对赤山道:“我虽然有黒豕蛊增长气力,但是真要论力量大小,恐怕也只比你多一丝。不过要推动这吞江蟾,也并非毫无希望,还需要你们的帮助。”

%80
“怎么帮?”赤山立即问道。

%81
方源慢慢道来,赤山露出一丝疑虑:“这样一来,岂不是我们在合作?吞江蟾即便醒来,是否还会认可你,然后心甘情愿地认输远离呢?”

%82
方源便笑道:“这个你大可放心。只要你们距离远一些,让它感应不到,自然就可以了。它毕竟也是蛊虫,不要把它想得有多么聪明。”

\end{this_body}

