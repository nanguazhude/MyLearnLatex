\newsection{最后的第六块紫金石}    %第四十三节:最后的第六块紫金石

\begin{this_body}

%1
青铜元海中波涛生灭,潮起潮落。

%2
海面上空,酒虫团成一团,散发出来的酒气如飘渺之白雾。

%3
一股真元哗的一声,逆冲而上,直入酒雾当中。待落下时,已经少了一半,颜色更加深邃。从初阶的翠绿,转变成了中阶的苍绿之色。

%4
中阶的真元落入元海,却并不和初阶真元相互混杂。仿佛更重一些,沉淀到下面去了。

%5
元海便形成了上半层是初阶真元水,下半层是中阶真元的格局。

%6
随着时间的推移,空窍中酒雾缭绕不断。在酒虫的精炼之下,渐渐的,初阶真元不断减少,中阶真元逐渐增多。

%7
可以明显地看到,下半层的中阶真元水位越来越高,上半层的初阶水位,则不断下降,又不断上升。

%8
方源一边精炼真元,一边又在外汲取元石中的天然真元,快速地补充到体内的空窍里。

%9
最终,空窍中四成半的元海,再次全部精炼成中阶真元。

%10
“多亏了中阶真元,否则在赌石场中,我还无法连续解石五次。”趺坐在宿舍的床榻上,方源缓缓地睁开双眼。

%11
此时已经深夜。

%12
他从赌石场出来后,就没有再逛其他的铺子,而是直接回到了学堂。

%13
虽然是在古月山寨周边,但是作为一转初阶的蛊师,身怀五百三十八块元石,还是太多了。

%14
这不仅是因为这些元石有些重,携带不方便。还有引人觊觎,会危机生命的另一层意思。

%15
若是有一转高阶,乃至二转蛊师起了歹心,以方源目前之能,还不能招架。

%16
“钱财都是身外物,人因财而死,是可悲的。可笑这许多世人都看不破这点。利益之船装了多少人,又沉了多少人。”方源嘴角勾勒出一丝冷笑,看了看手中握着的灰白元石。

%17
完整的元石,颗颗都有鸭蛋大小。但是他手中这颗,因为被汲取了一半真元,已经整整小了一圈。

%18
方源一点都不心疼。

%19
凡事有得必有失。他只是丙等资质,又要用酒虫精炼真元,元石的消耗是同龄人的数倍。但也因此,他能克服资质上的不足,若算真正的修行进度,他能名列前三。

%20
方源将元石重新放入钱袋里,又取出那最后一块紫金化石。

%21
他一共在赌场购买了六块,当场解开五块,还有一块随身带回了这里。

%22
他双眼精芒一闪,再次催动月光蛊,五指磨搓,进行解石。

%23
紫金化石在蓝色的波光中,迅速削减,最终消磨成空,留下的只是床边地上的一小摊的石粉。

%24
方源并不意外,赌石这事,十赌九输。

%25
就算是他有五百年的经历,顶多也只能做到十赌八输。剩下的两成赢面中,还分死蛊和活蛊。

%26
死蛊基本上没有多少价值。活蛊的话,也未必是那种珍稀的蛊虫。若真开出了价值巨大的活蛊,反而会惹来杀身大祸。

%27
方源现在这种修为,是很低微的,只是蛊师中的最底层。刚刚开出的癞土蛤蟆,若非是在古月山寨周边,他说不定就被那贾金生抢了。

%28
赌博,从来都不是发家致富的途径,反而为此倾家荡产的居多。这并不是方源的发展路线。

%29
虽然最后一块紫金化石,没有开出蛊虫。但是方源并不失望,反而看着地上的这摊石粉,渐渐露出了微笑。

%30
没有错,他进入赌场的最终目的,就在这摊石粉上。

%31
至于那只癞土蛤蟆不过是顺手而为的事情。

%32
他私下解石,除了他之外,没有人知道这个真实的结果。

%33
从今以后,他大可以托辞,就说酒虫就是从这紫金化石中唤醒收服的。

%34
这主意很妙。

%35
首先,谁也无法确定化石中封印着什么样的蛊虫。谁敢说酒虫不能沉眠在紫金化石当中呢?这完全有可能!

%36
其次,他有一些目击证人,他开出癞土蛤蟆,势必已经给赌场的蛊师们留下了深刻鲜明的印象。

%37
第三,就算是有人穷追不放,他也能将这一切都归结在运气上。运气这种东西,是最难理解的。就算是有人怀疑这就是花酒行者的酒虫,但是面对“运气”这个借口也无从下手。

%38
黑暗的房间中,方源目光幽幽。

%39
一味的隐瞒,就像是用纸包住火,终有一天是会露馅的。

%40
要处理掉酒虫这个隐患,就得主动出击。这才是方源的风格。

%41
况且,他细心思量过,在接下来的修行过程中,他也需要暴露酒虫。

%42
“酒虫这种一转蛊虫,对于一转蛊师来讲,十分珍贵。但到了二转,就不合用了。因此暴露出去,顶多会引发一些人的重视,无伤大局,并不要紧。它不像春秋蝉,若是春秋蝉暴露了,我说不定下一刻,就死无葬身之地了。”

%43
五百年的处世经验,早就让方源对人的心理洞如观火,掌握得了如指掌。

%44
“花酒行者遗藏,还有癞土蛤蟆,记忆里也就这两样便宜,如今都被我捡了。接下来就只能靠我自身按部就班的修行。”

%45
方源叹了一口气,舒展身子,就感到一种疲累困乏。

%46
蛊师的元海修行,并不能取代睡眠。

%47
方源抽起被褥,躺在了床上,仍旧半睁着双眼。

%48
虽然床头就藏着那五百多块的元石,床下又储藏了许多坛的青竹酒,但是他却感到一股隐隐的危机感。

%49
这五百多块元石,已经是一种极限。盛极而衰,方源清楚今后元石的消耗会越来越大。

%50
而他的收入,绝大部分都来源于勒索同窗。

%51
他已经越来越感觉到,周围同窗各自明显的进步。尤其是在最近的几次勒索中,古月漠北、赤城,还有弟弟古月方正这三人,拳脚功夫进步的程度很大。以前只要一两招就能收拾,现在却需要五六招。

%52
“再抢劫个三四次,他们的拳脚功夫就会被磨练出来。一个个向我挑战,以我现在的体力,还不能承受那样的车轮战。五百多块元石看起来多,但以我每天有四块元石的剧烈消耗,其实也不算什么了。”

%53
“青茅山这边已经没有什么宝藏,倒是附近的白骨山,有一位正道的四转蛊师秘密立下的力量传承。但是要得到这个传承,也很麻烦。其中有重要关卡,需要两人同心协力才能通过。”

%54
“唉,主要还是花酒行者的遗藏太少了,只得了一个酒虫。嗯……还有那个影壁呢。也许我可以卖给商队中的某个商家……”

%55
方源思考着,眼皮子越来越沉重,直至沉睡过去。

\end{this_body}

