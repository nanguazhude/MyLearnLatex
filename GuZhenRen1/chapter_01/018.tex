\newsection{就让往事如烟飘散}    %第十八节:就让往事如烟飘散

\begin{this_body}

面对弟弟的质询,方源没有说话,仍旧吃着早餐。他清楚弟弟的性格,方正是沉不住气的。

果然,方正见哥哥看也不看自己一眼,似乎把自己当做了空气。下一刻,他就带着不满的语气,叫起来:“哥哥,你对沈翠做了什么?自从她昨天从你房间里出来后,就大哭了一场,我安慰她,她哭的越凶。”

方源抬眼看了看弟弟,面无表情。

方正则皱着眉头,紧紧地盯着哥哥,等待着他的回答。

气氛越来越紧张。

但是方源只是看了他一眼,就低下头,继续吃饭。

弟弟方正顿时气急,方源这种态度,简直就是对他赤裸裸的藐视。他羞恼之下,一拍桌子,大声吼道:“古月方源,你怎么能这个样子!人家一个小丫头,服侍了你这么多年,对你的温柔体贴我都一一看在眼里。是,我知道你很失落,也能理解你的颓废。你只有丙等资质嘛,但是你也不能因为自己的遭遇,而去迁怒他人啊。这对她是不公平的!”

他话音还未落,方源腾的一下就站起身来,扬手如电。

啪!

一声脆响,给了方正一个结结实实的巴掌。

方正捂住右脸,蹬蹬倒退两步,带着一脸的惊愕。

“混账东西,你这是在用什么口气,对自己哥哥这么说话?!那沈翠不过是个小丫鬟,你为了这么一个小女子,就忘记我是你哥了么?”方源低声训斥着。

方正反应过来,脸上的疼痛一波波地传达到他的神经中枢。他瞪圆了眼睛,喘着粗气,难以置信地道:“哥,你打我?从小到大你都没打过我!是,我是被测出甲等资质,你只是丙等。但是你也不能怪我呀,这都是上天的安排……”

啪!

方正话还未说完,方源反手又是一个巴掌打过去。

方正双手捂住两边面颊,他懵了。

“天真的蠢货,你还记得什么!从小到大,我是怎么照顾你的?双亲刚死的时候,我们生活困苦,过新年舅父舅母只给我们俩一件新衣,我自己穿了吗?我给谁穿了?你小时候爱吃蜜饯粥,我每天都吩咐厨房给你多做一碗。你被旁人欺负了,是谁带着你找回场子?还有其他种种,我都不屑说了。嗯,现在你为了一个婢女,就这么跟我讲话,来质问我?”

方正涨红了脸,他的嘴唇哆嗦着,既羞恼又惊怒,却说不出一句反驳的话来。

因为方源说的都是事实!

“也罢了。”方源冷笑连连,“你既然连亲生父母都弃了,重新认了别人,我这个当哥哥的又算得了什么呢?”

“哥,你怎么能这么说呢。你也知道我从小就很渴望家庭的温暖,我……”方正连忙分辨。

方源摆手,阻止他讲下去:“从今天起,你不是我弟弟,我也不再是你哥。”

“哥!”方正大惊,张口欲言。

这时,方源又开口:“你不是喜欢沈翠吗?你放心,我没有对她做什么,她还是处子,黄花闺女。你给我六块元石,我把她转给你了,从今以后她就是你的贴身女仆了。”

“哥,你怎么……”陡然间被说破心思,方正一阵慌乱,有些猝不及防。

但同时,他心中也一定,他最担心的事情并没有成真。

在不久前的那一晚,沈翠亲自伺候他洗澡。

虽然没有发生什么实质性的东西,但是方正永远也忘不了那一晚的温柔。每次想到沈翠,想起她灵巧的双手,柔润的红唇,他的心中都涌起一阵悸动。

青春的情愫,早已经在少年的胸中积蓄,并且发芽。

所以当他昨天傍晚得知沈翠的异状时,他的心中顿时涌现出一股气。他当即放弃炼化月光蛊,转而满山寨寻找方源,要来讨个说法。

见方正不答应,方源皱起眉头:“男欢女爱,正常的很,你坦诚一点,躲躲闪闪的算什么。当然,你若不要换,那就算了。”

方正顿时着急了:“换,怎么不换。但我这元石,已经不够六块了。”

说着,他掏出钱袋子,面皮涨得通红。

方源接过袋子,发现里面有六块,但是其中一块比完整的元石要小上一半。他顿时知道,这是方正汲取了元石中的真元,好尽快地炼化月光蛊。

随着天然真元被抽取得越多,元石的体积就变得越小,重量也就越轻。

虽然只有五块半,但是方源也知道:这是方正手上所有的元石了。他本身就没有积蓄,这六块元石还是不久前舅父舅母给他的。

“元石我收下了,你可以走了。”方源表情冷漠,将袋子揣进怀里。

“哥哥……”方正还要说话。

方源微微扬起眉头,慢条斯理地道:“趁我还未改变主意,你最好从我眼前消失。”

方正心中一紧,咬咬牙,终于扭头走了。

跨出客栈门口,他下意识地捂住胸口,觉得心中一阵阵发虚。冥冥中有种感觉告诉他,他好像在此刻失去了一件很重要的东西。

但他很快心头一热,他想到了沈翠,以及那个魂牵梦绕的晚上。

“我终于可以名正言顺地得到你了,翠翠。”他头也不回,走出了方源的视野。

方源面无表情,站着好一会儿,这才慢慢坐下。

明媚的阳光,透过窗户,照在他冷漠的脸上,让人看了有种冰凉的感觉。

饭堂的生意有些冷清,街道上行人越来越多,喧闹声传来,更凸显出此处的寂默。

早餐已经凉了,伙计殷勤地走过来,问是否需要重新加热一下。

方源充耳未闻,他的眼神如烟云变幻不定,似乎在回忆着什么。

伙计候了一会儿,却见方源在发怔,始终没有回应,他只好摸摸鼻子,悻悻地走了。

半晌之后,方源眼神一定。

心中的回忆如烟,已经渐渐消褪。

他又回到了现实世界,阳光洒进来,照亮了一半多的桌面。饭菜上飘散的热气已经散去,街道上行人喧闹的声音也传入耳畔。

隔着衣服,伸手摸摸那揣在怀中的五颗半元石,嘴角露出一丝苦涩的,嘲讽的笑意。

但笑意一放即收。

“伙计,把这饭菜拿下去热一下。”方源看了一眼饭菜,淡淡地开口,喊了声。

这一刻,他的双眸清冽无比。

……

“什么!你哥是这么说的?”厅堂中,舅父皱着眉头,声音透着冷意。

舅母则坐在一旁,看着方正脸蛋上一边一个的鲜红掌印,很无语。

“是的,我见到哥哥时,他就在客栈吃早点,整个事情就是这样的。”方正恭谨地答道。

舅父的眉头皱得更深了,都凝成了一个川字。

几个呼吸功夫之后,他深深地叹息一声,用严肃的口气道:“方正,我的孩子,你要记住,婢女沈翠不是他方源的私有财物,而是我们调配给他的,怎么能够买卖呢?况且你想要的话,你就应该早早地对我们明说,我们把她调给你就是了。”

“啊?”方正听得目瞪口呆。

舅父挥手:“你先下去罢,你元石都给了方源,我就再拨给你六块。记住,这次要好好用在炼蛊上面,夺得此次第一,我们会为你感到骄傲的。”

“父亲大人,孩儿惭愧……”方正顿时感动得流下眼泪。

舅父叹气:“下去罢,下去罢,赶紧回房间去炼蛊,你已经浪费不少时间了。”

方正退了下去,舅父这才显露出愤怒狰狞的脸色。

砰!

他狠狠地一巴掌,拍在桌子上,低声地吼道:“哼,这个小兔崽子,居然拿着我们的人来做买卖,真是狡诈奸猾!”

舅母在一旁忙劝道:“老爷,消消气。不过是六块元石罢了。”

“你个妇道人家,懂个什么!他方源只是丙等资质,要炼化月光蛊,就得用元石。以他第一次炼蛊的稚嫩手法,六块元石肯定不够。但是现在他有了十二块,已经绰绰有余了。”舅父恨得咬牙切齿。

他继续又道:“蛊师的修行,只要资源充足,不被瓶颈卡住,就会很迅速。两三年的功夫,家族就能培养出一个个的二转蛊师。他方源修为越低,一年后夺回家产的希望就越小。现在他年纪轻轻,刚刚修行,我们卡住他,让他在起步阶段就落后同龄人。学堂的资源都是奖励给优秀的学员,凭他的资质,一落后就得不到资源,没有资源辅助修行越加落后。如此恶性循环下去,一年之后看他还能有实力继承家产不?”

舅母不解:“就算我们不遏制他,他一年之后最多不过一转中阶。老爷你可是二转修为,还怕他?”

舅父气得跺脚:“妇道人家,果真头发长见识短!以我堂堂长辈的身份,难道要和一个晚辈对殴不成?他要索回遗产,合情合理,根本就不能直接阻拦,只有从族规上着手。族规中明文规定:十六岁要成家立业者,就至少得有一转中阶的修为。否则,就说明他方源没有资格来浪费家族资源。我这么说,你懂吗?”

舅母恍然大悟。

舅父眯起双眼,里面阴芒闪烁。他微微摇头,感慨道:“这个方源实在太精明了,太狡诈了。色诱都被他看破,这是什么心智?小小年纪就老谋深算,恐怖啊!本来还想设计谋算他,他就直接搬出去了。还想靠着沈翠,来监视他干扰他,结果被他打发走了,还赚了六块元石。”

“唉,要是他能和方正一样傻就好了。对了,今后你要对方正好一点,他可是甲等资质。而且我看得出来,他对方源很不服气,很不甘心。这情绪很好,要好好引导。我有一种预感,他将来会是对付方源的最好利器!”

……

转眼,又是两天过去。

客栈房间中,并没有点灯。月光洒进来,照出一地霜华。

在床榻上,方源闭目盘坐着,调动青铜真元,聚精会神地炼化酒虫。

酒虫的一小截身躯,被染成了青铜的绿色,但是意志仍旧顽强不息,在烟雾状的真元包裹中,不断地挣扎。

方源的炼化进行得很不顺利,可以说是举步维艰。

“我耗费了足足两天两夜,每天只休息两个小时,耗费十二块元石,也不过只炼化到十五分之一的进度。按照时间来算,也就是最近这几天,就会有人炼蛊成功了吧。”

方源对局势洞若观火。

不过本来他资质就差一筹,现在炼化的酒虫求生意志又极度顽强,比寻常的月光蛊要高出太多。造成现在这个落后局面,也十分正常。

“一时的落后不算什么,只要有了酒虫……”方源心湖如镜,没有一丝焦躁和气馁。

但就在这时,酒虫忽然蜷缩起来,团成了一团。

“不好,蛊虫反噬!”方源陡然睁开双眼,眼中闪过一丝惊诧。眼前,酒虫团成一个圆溜溜的汤圆,猛地绽放出刺眼的白色毫光。

它孤注一掷了!

顿时,方源感到一股强大的意志,从酒虫的身上传来,直接越过真元,降临到他的空窍元海。

蛊虫反噬的情况,十分罕见。只有那种意志极为刚强的蛊虫,才会奋尽全力,不成功则成仁。

若换做其他少年,面对这种情况,必定会惊慌失措。

但是方源虽惊不慌,反而有些微喜:“孤注一掷也好,只要我接下这次反噬,就能让酒虫的意志大大削弱了。不过接下来我就要凝聚全部精神,全力对抗这股意志冲击,不能有丝毫的外界干扰。否则就糟糕了,唉……但愿这期间没有人来打扰我。”

他思量完毕,正要催动空窍真元,迎上那股意志。好好与它纠缠,大战三百个回合。

但就在这时,异变再生!

一只蛊虫,在他的空窍中央,元海上空显现出来。

轰隆!

这只蛊虫暴发出强大至极的气息。

这股气息就好像是天河倾泻,山洪暴发,又像是威严被触犯的恐怖巨兽,睁开猩红的双眼,查看是谁胆敢冒犯它的地盘!

“这是春秋蝉?!”看到这只蛊虫,方源彻底的震惊了!!

(ps:感谢粉丝789、失落の光、ㄨ太歲乂、魏陆辉、水淡云清的打赏,感谢老朋友冥枫魂同学的一直支持。新书发布刚刚好一周了,好多老朋友都来了,有的甚至是好几年前的。看见你们熟悉的id,我心中感到很温馨。可以说,没有你们的支持,我走不到今天。

我还会一直走下去,三年,六年,九年……这期间,有些朋友会暂时离开,有些朋友会一直在。在茫茫人生的旅途中,我们彼此印证彼此的存在,我们相互证明我们活过。

我幻想这样的一个情形:当我们老了,朋友们看到“蛊真人”这个id,会心一笑,自言自语:“哦,是他啊,我年轻的时候看过他的小说。我还给他投过推荐票呢。”或者我再翻开以前的版面,看到一个个熟悉的id,曾经打赏的,投月票的,评论的。我会回想,在我独自码字的时候,这些个名字就是**跋涉,给我温暖的点点灯火。

本书到这里,是一个小小的转折。方源将开始真正展露出自己的独特风采。能看到这里的朋友们,都是有缘人。本人在这里担保,本书会越来越精彩,更新也会很稳定,比上一本会好很多。请大家多投点推荐票吧,目前正需要呢。)

------------

\end{this_body}

