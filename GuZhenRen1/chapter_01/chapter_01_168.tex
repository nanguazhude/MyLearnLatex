\newsection{狡电狈}    %第一百六十八节:狡电狈

\begin{this_body}

%1
群狼涌动,呼啸而来。

%2
情势急转直下,让两位族长以及家老们都脸色骤变。

%3
他们两家虽然成功抵挡住狼潮,但也是岌岌可危,极为勉强的。如今再出现一波如此强势的狼潮,根本就无余力抵抗了。

%4
一时间,家老和两位族长都下意识地停止了战斗。

%5
“情报上说,不是只有三头雷冠头狼的吗?”有家老失声大叫。

%6
“不对,这头雷冠头狼身上负着伤,狼群的规模也不是很大。”古月博强制镇定地道。

%7
“难道说,这是冲击熊家寨的那头雷冠头狼?”有家老一拍脑门,喊道。

%8
这可能性相当的大,八九不离十。

%9
家老们心情仍旧沉重无比,有人涩声道:“这狼群已经出现在这里,那岂不是说熊家寨已经……”

%10
“该死的熊家寨,怎么如此不济?居然连一波狼潮都抵御不住!”有人大声咒骂。

%11
但也有人心存乐观:“你们看,这头雷冠头狼两前肢,都受了重伤,肌肉骨骼都萎缩了!”

%12
众人循声看去,失落的士气微微一振。

%13
的确如此。

%14
这雷冠头狼后肢发达,两前肢却短小,肌肉萎缩。这使得它奔跑时,只能利用两条后肢,像是袋鼠一样蹦跳着前进。

%15
“等等,这好像不是雷冠头狼……”白家族长忽然想到了什么,浑身一震。

%16
“是狡电狈!”方源心中已给出答案。

%17
五虎一彪,三犬一獒。十狼一狈。

%18
狈也是狼属,但比狼要聪明数倍,往往担任狼群的军师。眼前的这头狡电狈,外形上极为类似雷冠头狼,被侦察蛊师误解也很正常。但它是货真价实的万兽王,有着不亚于人的智慧!

%19
这狡电狈虽然在肉搏方面,稍稍逊色于寻常的雷冠头狼。但是有类人的智慧。让它的危险程度远远超过雷冠头狼。再加上无数的电狼大军可供驱策,难怪熊家寨被灭。

%20
“走!”方源一振雷翼,立即飞上半空。

%21
眼前的这些家老。早已经激战良久,战力损耗极大,根本就难以对狼群构成威胁。更关键的是。他们之间彼此猜疑,难以合作在一起。

%22
方源掉头就走,万兽王不是闹着玩的。尽管场中还留着两只野生蛊在乱飞,但他也不顾上了。

%23
必须撤离,晚一点恐怕都走不掉!

%24
自知之明,能舍能弃,才是行走世界的第一要义。

%25
“撤吧,狼潮势大,我们打不动的。”

%26
“回去山寨,赶紧布防!”

%27
尽管家老们还没有认出狡电狈的身份。但他们也都萌生了退意。

%28
但就在这时,狡电狈忽然嗥叫一声,张开巨大的狼口。

%29
狼牙参差如刃,在狼牙之间,一团黑气从无到有。眨眼睛就凝成一团黑球。

%30
嗖!

%31
黑球暴射而出,在空中划出一道略带弯曲的黑色弧线,然后砸落在地面上。

%32
“射空了?”

%33
“这雷冠头狼不行了,准头太烂了!”

%34
家老们大叫着,方源则飞得更疾。

%35
轰——!

%36
黑色烟球猛地爆炸开来,剧烈的轰鸣声中。黑烟向四周辐射扩散。

%37
势头迅猛无比,几乎眨眼间,就覆盖了方圆百里。

%38
“这是四转狼烟蛊!”方源心中一沉,他已经第一时间,做了最准确最明智的反应。但是这黑色狼烟扩散得太快,将他直接笼罩住。

%39
一时间,他仿佛置身黑夜当中,伸手不见五指。到处是呛人的浓烟,呼吸不畅,难受至极。

%40
但好在他有雷翼蛊,只要顺着上方直飞,总会能脱离这狼烟的范围。

%41
咔嚓。

%42
就在下一刻,一道霹雳闪电,劈开层层黑烟,如雷蛇似狂龙,跨越百里距离,轰向方源。

%43
这是狡电狈出手。

%44
闪电是多么的快,几乎让人难以反应。

%45
但千钧一发之际,方源的战斗意识超越了思维速度,优先做出了应对。

%46
雷盾蛊!

%47
天蓬蛊!

%48
一道圆形电光护盾,乍然出现在方源的身侧。同时他的身上,亮起一层白光虚甲。

%49
狂暴的闪电,亮得刺眼至极,仿佛咆哮的天龙,首先轰击在电光护盾上。

%50
护盾只坚持了一秒不到,就被摧枯拉朽的闪电霹雳,撕裂撞破。

%51
闪电轰击在方源的身上,一刹那间,方源尽管已经紧闭了双眼,但仍旧觉得刺眼至极。

%52
一股巨大的力量涌来,将他击落。

%53
他几乎都要昏厥过去,电流缠绕,麻痹着他的浑身肌肉,几乎让他忘记了呼吸!

%54
扑通一声,他掉落在地上。

%55
剧痛袭来,这才让他痛醒过来。

%56
他忍住浑身剧痛和酥麻,连忙爬起来。

%57
雷盾蛊已经死亡了,雷翼蛊亦是被殃及,奄奄一息,难堪再用。天蓬蛊也受伤不轻,有些萎靡,毕竟承受了那么强大的电流冲击。

%58
狡电狈刚刚那一击,必定是四转蛊虫发威。

%59
蛊虫越往后晋升,差一阶威能就是天差地别。

%60
四转的攻击蛊虫,至少得用两只三转才能防住。当然雷盾蛊之所以死去,也是因为之前鏖战很久,积累了许多伤势。

%61
“想不到,这狡电狈如此看得起我……”方源苦笑一声,观察四周。

%62
四周黑漆漆一片,全是浓烟,根本分不清方位。

%63
“小心,那雷冠头狼变小了,和普通电狼一样大,就藏在狼群里。”这时,从黑烟深处,传来一位家老的叫喊声。

%64
方源听了,瞳孔一缩。

%65
这狡电狈太狡诈了,真的很阴毒。它似乎想把这些蛊师都一网打尽。刚刚出手针对方源,是一个都不想放过。

%66
周围很快亮起无数双狼瞳。

%67
低吼声,狼群奔袭带出的风声,都向方源传来。

%68
在这样漆黑的环境中,蛊师们的视野受到了极大的阻碍。但是狼群却毫无关系,因为这些电狼舍弃了嗅觉,视力极好。

%69
“必须尽快地逃出去。谁知道会不会碰到那头狡电狈?就算暂时碰不上,被狼群包围住,我只剩下四成的真元。根本不能支撑消耗战!”

%70
方源心思电转,催动地听肉耳草。

%71
十多根参须,从他右耳廓生长出来。往外蔓延。

%72
无数的声音传来,有狼嚎声,有战斗声,有家老惊惶的低吼,有电狼临死的惨叫。

%73
太乱了!

%74
地听肉耳草侦察氛围很广,但并不能区分细节。

%75
方源皱了一下眉头,只能顺着声音少的一侧转移。

%76
但很快,他就遭遇到了一只上百头的电狼群。

%77
电狼从黑烟中奔袭而来,一只只凶悍无比。

%78
方源唤出锯齿金蜈,同时撑起天蓬蛊。

%79
锯齿疯狂转动。金蜈如大剑,绞动黑烟,砍在狼躯上,无不掀起一阵血雨白骨。

%80
方源如溺水行舟,披荆斩棘。

%81
一只只电狼。惨死在锯齿金蜈之下。但旋即又有第二只,第三只……接连不断地向它扑杀而来。

%82
“这狡电狈太阴险毒辣,居然让狼群绕后包围。”方源顺着一个方向,冲了一阵子,却总是遭遇电狼,心中顿时恍然。

%83
他且战且退。不一会儿,就浑身浴血。

%84
压力太大,四周黑幕重重,伸手不见五指,狼群从四面八方冲杀过来,叫他一个人难以应对周全。

%85
“古月博,你怎么说?”这时,浓烟深处忽然传出白家族长的声音。

%86
“也罢,先联合一起,突围出去再说!”古月博的大喝声,也紧接着传来。

%87
这也是形势所逼,只有联手才能有生存的机会。

%88
否则单打独斗,很快就被电狼消耗光真元,被狼群分食。下场势必将凄惨无比。

%89
吼!

%90
“该死!”

%91
下一刻,一声狼吼,激烈的爆炸声传来。两位族长同时爆喝,再无法组织家老。

%92
很显然,是狡电狈出手偷袭了。

%93
它有不属于人的智慧,一出手就破坏了两位族长的意图,打乱了家老们反抗的步骤。

%94
没有两位族长压着场面,家老们之间能否精诚合作呢?

%95
这是个巨大的疑问。

%96
“不行了。真元消耗得太多,我必须借助他人之力!”又冲杀片刻之后,方源感到累了。

%97
他有双猪之力,但仍旧感到浑身肌肉酸痛。

%98
他的真元已经不足,浑身都出现伤口,天蓬蛊只能时停时用。

%99
锯齿金蜈浑身黯淡,两旁锯齿已经破损不堪。短短功夫,它砍了不下千头电狼,其中还有几头豪电狼。

%100
其中几头,因为有防御蛊虫,比铁石还硬。

%101
锯齿金蜈也并非无坚不摧,没有锯齿,它的劈砍能力顿时暴降,变得惨不忍睹。

%102
方源没有恋战,且战且走。

%103
他靠着地听肉耳草,极力分辨。一旦有狂电狼的沉重脚步声,就立即转变方向,努力避开。

%104
狂电狼这等千兽王,他一人还独战不了。一旦被牵制,落入重重包围当中,那绝对是十死无生。

%105
“我不甘啊!”浓烟中忽然传来家老的惨叫,声音旋即戛然而止。

%106
黑烟中,狼潮汹涌,许多家老都惨死狼口,发出不甘无奈的怒吼。

%107
“我也要坚持不住了!”方源感受到死亡的气息,但他面色冷酷,越是危机,心中越是沉静如冰雪。

%108
他的心灵没有一丝一毫的动摇,前世比这危险的情境多得去了。

%109
目前情况还不算太糟糕,两位族长还和狡电狈战斗着,方源还有希望。

%110
“嗯?前方有打斗声!”方源听到声音,立即转了方向。他快支撑不住了,不管是古月一族,还是白家蛊师,对他都有利。

%111
“杀!杀!杀!”远远就听见这蛊师大声地咆哮着,正在酣战。

%112
方源疾奔过去,冲势忽然一滞。

%113
白凝冰!

%114
(ps:哎哟喂,又涌现一个秀道德优越感的喷子,被恶心了一下。卫道士们,不满意请绕道。本文简介中写了明明白白的四个字“三观不正”,本文序言也不是白写的,写的清清楚楚。这本书我写到现在满意的很,也会继续写下去。抱歉让你们失望了,我是不会太监的。呵呵。)(未完待续。如果您喜欢这部作品,欢迎您来投推荐票、月票,您的支持,就是我最大的动力。)

\end{this_body}

