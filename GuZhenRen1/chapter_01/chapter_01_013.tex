\newsection{月下竹林,一点珠雪}    %第十三节:月下竹林,一点珠雪

\begin{this_body}

%1
大约是在三百多年前,古月一族中出现了一个天才人物。

%2
他才华横溢,还是青年,就已经修炼到了五转蛊师的地步,还有更进一步的可能。名扬青茅山,风光无限,被族人们寄托希望和重任。

%3
他就是古月族史上,最为人传诵的四代族长。

%4
可惜的是,四代族长为保护族人,对战同样是五转蛊师的魔头花酒行者。虽然一场大激战后,他击败了花酒行者,并且让这个魔头跪地求饶。

%5
但是最终,四代族长不小心,被花酒行者阴险偷袭。

%6
四代怒毙花酒行者,但是自己也伤重难返,英年早逝。

%7
这个充满悲剧的故事,一直流传至今,为古月族人广为传颂。

%8
但是方源却明白,这故事不可尽信。因为它本身就有一个巨大的漏洞。

%9
在前世的一个月后,一位因失恋而醉酒的蛊师,烂醉如泥地躺在山寨外,结果四溢的酒香气息,意外地引来了一头酒虫。

%10
蛊师追着这酒虫,在一处地下密洞里发现了花酒行者的遗骸,同时还有花酒行者的随身遗产。

%11
蛊师赶忙回到家族,禀告了此事,引起巨大轰动。

%12
风波渐渐平息之后,他也因此受益,得到了酒虫,修为提高,抛弃他的女友反而倒追他,此事使他成为了寨中一时的风云人物。

%13
故事一代传一代,走了样也属于正常。但是方源记忆中,那个蛊师发现遗藏的说法,虽然极为真实可信,但他也怀疑其中可能隐藏着其他真相。

%14
“原先还没有察觉,但是这些天我一边探索,一边分析,发现了这事件别有蹊跷。”夜色渐浓,方源在山寨周围的竹林中走着,心中则在回顾整合所有的已知线索。

%15
“设身处地去想,假设我就是那个蛊师,发现了花酒行者的遗藏,为什么不一个人独吞,而是禀告了家族呢?别提什么家族的荣誉感,人都是有贪欲的。是什么促使那个蛊师,违背心中的欲望,而心甘情愿地舍弃全部的利益,把这发现禀告家族高层呢?”

%16
真相总隐藏在历史的迷雾当中,方源苦思冥想,也没有结果。

%17
毕竟线索太少了。仅有的两条线索,本身也是真真假假,不可全信。

%18
方源又不禁想到自身:“不管如何,买了这坛青竹酒后,我现在身上只剩下两块元石。若是再找不到遗藏,可就麻烦了。今日此举,算得上孤注一掷!”

%19
不过,他本身炼化蛊虫,元石就不够。反而不如投资在这酒上,增加此行的成功率。

%20
若换做别人,大多数都会按部就班地攒元石吧。

%21
但此举对方源来讲,效率太低了。他宁愿去冒着风险,来搏一搏。

%22
魔道中人嘛,总是喜欢冒险。

%23
此刻。

%24
夜色渐浓,春月如弓。

%25
浮云遮蔽月光,仿佛给月牙罩上一层淡淡的薄纱。

%26
因为刚刚下了场连续三天三夜的大雨,山间的浊气都被洗刷个干净,只留下最纯粹的清新。

%27
这样的清新空气,纯的如同一张白纸,更利于酒香的传播。这是方源今夜信心十足的原因之一。

%28
“就剩下这片领域没有探索了。”走到一处竹林,他停下脚步。

%29
前七天的探索,不是没有收获的。至少证明了,花酒行者并没有死在那些地方。

%30
这是方源信心的原因之二。

%31
竹林中,芳草萋萋,白花漫漫,青矛竹挺得笔直,如根根玉竿。

%32
方源拍开小酒坛的泥封,顿时一阵浓郁的酒香扑鼻而来。

%33
青竹酒,可以说是古月山寨中的第一好酒。这是方源今夜信心十足的第三原因。

%34
“如此三大原因,积累起来,若要成功,必是今夜!”方源一边在心中暗暗打气,一边将酒坛慢慢倾斜,倒下一小股青竹酒,滴在石头上。

%35
若是那几个猎户看到此景,必定要心疼坏了。这酒可足足价值两块元石啊……

%36
但是方源却是无动于衷。

%37
香醇的酒气很快就扩散开来,夜色里,清风徐徐,暗香浮动,浸染竹林一片。

%38
方源站在原地,闻着酒香,等了片刻,却不见任何动静。

%39
只听见一头夜莺般的鸟儿,在不远处啼叫,声音像是一串银铃。

%40
他目光沉静,也不意外,挪开脚步,转身又去了几百米外的下一处。

%41
在这处,他如法炮制,又倒出一股酒来,站在原地等待片刻。

%42
如此三番五次,转了几处方位,倒了几次酒,酒坛中青竹酒已经所剩无几。

%43
“最后一次了。”方源心中叹了一口气,将手中酒坛彻底倒置,坛底朝天,将仅剩下的一股酒全数倾倒下来。

%44
酒水洒在草丛中,青草一阵摇曳,野花沾湿了酒露,微微垂下了头。

%45
方源站在原地,怀着最后的希望,凝视周围。

%46
此时夜已深沉。

%47
一片浓郁的云翳,遮蔽了这里的月光。

%48
阴影黑暗像是一层幕布,盖住这片竹林。

%49
四周寂静无声,一根根青矛竹孤零零地立着,在方源的眼眸中留下一道道直上直下的剪影。

%50
他静静地站在原地,听着自己极为清晰的呼吸,然后感受着自己胸中那点残留的希望,不断地流逝,流空。

%51
“还是失败了么。”他心中喃喃,“今夜三大优势积累在一起,都没有成功,寻不到酒虫的影子。这就意味着今后的成功率会更低。我现在身上仅剩下两块元石,还要炼化月光蛊。不能再冒险了。”

%52
冒险的结果,常常都不理想。但一旦是理想的结果,那么收益必定可观。

%53
方源喜欢冒险,但他不嗜赌,他不是那种输红了眼,一心想要翻本的赌徒。

%54
他有自己的底线,他清楚自己的本钱。

%55
现在,五百年的生活经验,告诉他,是时候收手了。

%56
有的时候,人生就是这样,常常有那么一个目标,它是那么美好,充满了诱惑力。看似近在咫尺,但是费尽周折,却总达不到。让人辗转发侧,寤寐思服。

%57
“这是生活的无奈,也是生活的魅力所在啊。”方源苦笑地摇摇头,转身就走。

%58
就在此时。

%59
一阵风吹来,像是一只温柔的臂膀,轻轻地拂开了夜空中的浮云。

%60
浮云悠悠而过,露出遮蔽的月牙。

%61
月牙弯弯,悬挂于天空,如一白玉盏,将澄澈如水般的月光倾倒下来。

%62
月光倾洒在青茅竹林中,倾洒在山石上,倾洒在溪流山涧中,倾洒在方源的身上。

%63
方源穿着一身朴素的衣衫,年轻的脸庞在月光的轻柔触摸下,显得更加白皙。

%64
黑暗仿佛在刹那间消退了,取而代之的是一地的似雪霜华。

%65
似乎是受到月光的感染,息声的夜莺开始重新鸣叫,这一次不止一只,而是有数只,分散在这竹林中,此起彼伏的响应着。

%66
同时,一种生活在大山,在月色下活动的龙丸蛐蛐,也唱响了窸窸窣窣的生命之歌。

%67
它们是夜间才出来活动,身上都散发红光的昆虫,此刻成群结队地跳跃而出,一个个身上闪现出红玛瑙般的光辉。

%68
方源乍一眼看去,这群龙丸蛐蛐就像是一股会跳跃的赤水,踏着青草野花,在月下的竹林中跃进。

%69
竹林中如积水空明,碧玉的青矛竹在月色下,闪现着莹润如玉的光华。

%70
柳暗花明,大自然在这一刻,向方源展现出她的美好皎丽。

%71
方源不自觉地停下脚步,感觉自己如置身仙境。

%72
他本已要转身离去,但此刻下意识地又转身一望。

%73
那倾倒了青竹酒水的野花草丛,在风中微微颤动着,仍旧空无一物。

%74
方源自嘲一笑,收回视线。

%75
然而。

%76
不想就在这抬眼的过程中,他看到了一点白色的雪影。

%77
这雪影贴在不远处的一根青矛竹竿上,在月光下,如同悬挂着的一颗浑圆的珍珠。

%78
方源双眼瞳孔猛地一扩,身躯微微一颤,心中砰然一动,旋即越跳越快。

%79
是酒虫!

\end{this_body}

