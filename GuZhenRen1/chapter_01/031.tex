\newsection{方源,你大祸临头了!}    %第三十一节:方源,你大祸临头了!

\begin{this_body}

%1
“我苦练基本拳脚连续七天,想不到只在方源的手中支持了两招,就被他再次击昏。耻辱,耻辱啊!”古月漠北心中充满了羞愤恼怒。

%2
在家里的院子中,他对着木人傀儡,不断拳打脚踢,打得砰砰作响。

%3
这时一声轻笑传来:“小弟,跟木人傀儡有深仇大恨吗,这么恨它?”

%4
听到这个熟悉的声音,古月漠北脸色一松,停止了攻击,转过头去:“姐姐,你回来了!”

%5
“嗯,家族委派了侦察任务,出去了十几天……”古月漠颜笑着答道,她是漠北的亲姐姐,目前已经是二转中阶的蛊师。

%6
但她很快脸色一变,眼光中透出一股凌厉:“小弟,你脸上的伤痕是怎么回事?谁欺负的你?”

%7
“啊,没什么事。这是我不小心走路摔跤,磕到的。”漠北脸上闪过一丝慌乱之色,胡乱搪塞了一个理由,他可不希望姐姐知道自己这么糗的事情。堂堂漠之分家未来的掌权人,古月漠尘的亲孙子,居然连续两次被人打昏。

%8
不过万幸的是,倒霉的也不是他一个,其他人都有份。

%9
“哦,是这样,那你可得小心点。对了,你要练拳脚,这样可不行。你没有增加防御的蛊,就要缠上厚厚的布条。这样才能保护你的手脚不受到伤害。”古月漠颜叮嘱了几句,就离开了。

%10
“大小姐好。”

%11
“大小姐安好。”

%12
“是大小姐回来了呀,奴才拜见大小姐。”

%13
古月漠颜面色清冷,疾步走着。一路上碰到的家奴,无不躬身行礼。

%14
她走到书房前,也不禀告,直接推门而入。

%15
书房中,古月漠尘背对着她,正站在书桌前练字。

%16
“回来了?”古月漠尘并没有回头,而是直接问道,“在外面侦察了半个多月,那狼巢的情况怎么样?”

%17
“爷爷,你怎么知道是我?”漠颜愣了楞。

%18
“哼,全家上下,除了你能仗着我的宠爱不守规矩,连门都不敲一下。还能有谁?”古月漠尘带着责备的语气,不过转过身来,脸色却很温和,望着漠颜眼中带着笑意。

%19
漠颜撇撇嘴:“要说宠爱,其实爷爷更宠弟弟多些。只是弟弟将来要掌权,爷爷对他要求严格,旁人体会不到这份宠爱罢了。”

%20
顿了顿,又问道:“爷爷,小弟怎么被人打了!我刚问他,他撒谎。我也不好强求,只好来问爷爷了。”

%21
古月漠尘脸色一肃:“你还没有回答我的话呢。”

%22
他放下笔,抽过椅子,慢慢坐下。

%23
漠颜只好禀告:“狼巢差不多要满了,按照这种繁衍速度,虽然今年不会有狼潮,但是来年必定就有狼潮冲击我们山寨。”

%24
古月漠尘又问:“每三年基本上都会有一次狼潮,这并不稀奇。只是那狼群中,有多少只雷冠头狼呢?”

%25
“大约有三只。”漠颜答道。

%26
古月漠尘点点头,放下心来。雷冠头狼是狼群中的首领,狼潮冲击山寨时,最棘手的存在。

%27
三只雷冠头狼并不太多,因为青茅山上有三家山寨。每个山寨分担一条,狼潮冲击的压力就会大减。

%28
“爷爷,您还没告诉我小弟的事情呢。”漠颜不甘心地问道。

%29
“告诉你也无妨,漠北是被人打了,七天前一次,今天第二次。是在学堂大门口被揍趴下的,两次都是当场昏迷。”古月漠尘笑着道。

%30
“什么人竟然胆大包天,打昏小弟?”漠颜瞪眼。

%31
“是他的一个同窗,叫做方源。他打得好啊……”古月漠尘呵呵笑出了声。

%32
漠颜眼睛瞪得更大了,不解地问道:“爷爷,您这是什么话?漠北可是你亲孙子!”

%33
古月漠尘深深地看了孙女一眼,语重心长地道:“漠颜啊,你是女孩子,可能不懂。失败和耻辱会使人进步。没有失败,是淬炼不出一个成熟的男人的。”

%34
“漠北被打了,是他自己的失败。他一清醒,就开始向护院们讨教拳脚功夫,这是一种进步。而这种进步,就是方源带来的,打醒他的。你作为他的亲姐姐,若要真正意义上爱护他,不要去干涉他的成长。方源只是个丙等的穷小子,漠北是乙等资质,身后又有我们,迟早有一天会把方源踩在脚下。”

%35
“把这个对手留给漠北吧。女人的一生中,需要亲友爱人。而男人的一生中,亲友爱人可有可无,但惟独不能缺少对手。你不要去找方源的麻烦,知道吗?这是小一辈的事情。你参合进去,就是以大欺小。打破其中的规矩惯例,别人也会看不起我们漠家的。”

%36
漠颜几次张口欲言,但在古月漠尘的目光注视下,她最终还是低下了头,回答:“是,爷爷。孙女明白了。”

%37
她垂首退出书房,就算是古月漠北也没有发现——她的眼中一直闪着别样的光。

%38
“爷爷,这是你宠爱孙子的方式。而我漠颜也有爱护我小弟的方式!”漠颜心中早已经另有主意。

%39
……

%40
客栈的饭厅中,有几桌都有人坐着,吃着饭菜,因此显得有些热闹。

%41
一两个伙计端着盘子上菜,在桌子间自如地穿梭游走。

%42
方源就坐在靠近窗户的桌子旁,点了几个菜,一边吃着一边透过窗户远眺。

%43
天边,晚霞如火,静静地燃烧着。

%44
太阳已经落下大半,它留恋地最后凝望大地,余晖就是它不舍的回眸。

%45
远处的群山,已经被暮色掩盖。近处的街道上,是归家的人群。有些是赤着脚,沾满泥土的农人,有些是背着药篓的采药者,有些猎户扛着山鸡、野猪等等猎物,还有一些则是蛊师。他们往往穿着一身蓝色武服,干练精悍。头上绑着头带,腰间系着宽边腰带。

%46
这腰带很有讲究,一转蛊师就是青色腰带,正面是镶着铜片,铜片上有一个“一”字。

%47
二转蛊师的腰带,是赤色的,正中央镶着铁片,上面刻着“二”字。

%48
方源坐在窗边,观察了一会,看到了六七位一转蛊师,大多都是年轻人。还有一位二转蛊师,是个中年汉子。

%49
至于三转蛊师,通常已经是家老。四转就是族长,一寨之主。

%50
五转的蛊师很少见了,古月一族的历史上,也只有一代族长和四代族长。

%51
“其实观测一个家族的实力,是很简单的事情。只要在山寨中找个位置坐下,精心观察行人个把小时,看其中有多少一转蛊师、二转蛊师,就能看出这个家族的底蕴和实力。”方源结合五百年的阅历,在心中总结出了一个经验。

%52
就古月山寨来讲,街道上走过二十个行人,其中就有六位蛊师。这六位蛊师当中,有一半的几率,会有一名二转蛊师。

%53
古月山寨就是凭借这样的实力和底蕴,霸占了青茅山最好的资源点之一。堪称青茅山的霸主。

%54
但是青茅山,不过是广袤的南疆中的一个小角落。古月山寨在整个南疆来讲,只能算是中下等的族群。

%55
“我现在不过才刚刚修行,一转初阶的修为,连独自一人闯荡南疆的资格都没有。至少得有三转实力,才能远游啊。”方源吃了口菜,在心中幽幽一叹。

%56
青茅山太小了,装载不了他的野心,他已经注定离去。

%57
“哈哈,古月方源,终于找到了你了。”就在这时,一位中年男子狞笑着走了过来。

%58
“嗯?”方源微微转头看去,只见对方脸色蜡黄,吊梢眉,但是身材高大,身上肌肉发达。此时大步走过来,双手抱臂在胸,居高临下地盯着正坐着吃饭的少年方源,细长的双眼中闪过一抹寒光。

%59
“方源,你大祸临头了知不知道?嘿嘿嘿,你居然敢打我们漠家的小少爷,现在大小姐漠颜到处在找你算账呢。”中年男子冷笑不断。他用目光上下打量着方源,隐隐散发出一股逼人的气势。

\end{this_body}

