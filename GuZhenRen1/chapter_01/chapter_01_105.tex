\newsection{我会输得很惨}    %第一百零五节:我会输得很惨

\begin{this_body}

%1
和方源、方正一样,古月青书也是孤儿。

%2
他的双亲,在童年之时,就都牺牲在浪潮之中。

%3
他被族长古月博亲自抚养成人,测出乙等资质之后,又受到古月博的亲自教导。他的资质优异,在乙等中算得上出类拔萃,几乎可称之为“伪甲等”,古月博一直将其作为族长候选人栽培。

%4
古月青书脾气温和,富有亲和力,很受周围族人的好评和欢迎。他对家族十分忠诚,方正的出现,断绝了他继承族长之位的希望,但是他反而高兴,一心一意照看方正。

%5
这样的人物,放在地球上,就是岳飞、魏征、包拯之流。

%6
可惜的是,在一年后的狼潮中,北大门失陷。他为了堵住这个缺口,保护族人不受伤害,挺身而出。最终,他以二转修为强行催动三转蛊虫,一夫当关万夫莫开,成功守住了山寨。

%7
但是他本人,也因此空窍损灭,最后化为一株树人而亡。

%8
也因此,古月青书给方源留下了一个较为深刻的印象。

%9
古月青书见方源叹气,当然想不到方源正在回顾自己的死亡,还以为方源也在为酒虫而苦恼。

%10
他微笑道:“想来方源你也早就清楚酒虫的局限了。没错,酒虫只是一转蛊虫,只能精炼一转的青铜真元。而你已经是二转蛊师,对于赤铁真元酒虫是没有办法精炼的。虽然你现在有了酒肆。可以更方便地喂养酒虫,但是对自己没有用处的东西,白白养着干什么呢?”

%11
接着他话锋一转:“但是,酒虫对你无用,对其他的一转蛊师却有效果。尤其是明年春天开窍大典后,家族会涌现出的新一批的学员,酒虫对他们的帮助会很大。所以不如将酒虫卖给家族,给家族贡献一份力量。”

%12
方源沉默不语。

%13
青书沉吟了一下。猜中了方源的心意:“我明白了,你是不舍得酒虫,还想合炼它。如果我料得不错,你应该是想走二转白虫茧,再走三转蒙汗蝶的合炼路线吧。”

%14
“这个秘方,是流传最广,也是最实用的秘方。蒙汗蝶也是比较好的蛊虫。但是白虫茧却毫无能力。对你而言,这个合炼的路线价值并不大。你是丙等资质。如今是二转修为。而白虫茧没有任何能力,只是耗费食物喂养,对你来讲没有帮助。”

%15
“你晋升三转的可能有多大呢?就算你成功晋升了三转,恐怕已经到了中年了。你要一直喂养无用的白虫茧数十年吗?用喂养白虫茧的这笔费用,还不如喂养其他蛊虫,更加实用,对你更有帮助不是吗?”

%16
“酒虫的真正价值。在于精炼真元,提高一个小境界。你若是这样合炼。只是将酒虫当做一种材料,你不觉得可惜吗?”

%17
任何一个蛊虫。都只有一个能力。

%18
譬如春秋蝉,虽然高达六转,但只有重生之能。

%19
合炼后的新蛊虫,常常只能留取一只蛊虫的能力,并加以强化。例如白玉蛊,就是把玉皮蛊的防御能力保留了下来,并强化,同时丧失了白豕蛊增强力量的效果。

%20
也就是说,如果有人得到白玉蛊,白玉蛊只会带来防御上的帮助,不会提升蛊师的力量。

%21
青书说的不错,酒虫最有价值的地方,在于精炼真元,提升一个小境界。

%22
这对于蛊师来讲,不啻于在另一种形式,增大了真元的储备。同时对温养空窍,推动修为有巨大帮助。

%23
如果按照“白虫茧、蒙汗蝶”的这个路线,得到的蛊虫,都没有精炼真元的能力。的确是可惜。

%24
事实上,花酒行者就是采用的这个路线。他将酒虫晋升为蒙汗蝶之后,随身携带。多次将女性迷倒,实施恶行。只是后来他死了,蒙汗蝶没有充足的食物来源,不断退化,最终还原成酒虫。

%25
见方源没有说话,青书双目一闪,又接着道:“其实,我族中还收录了一个秘方。按照此等秘方,可将酒虫晋升为二转的邀月蛊,三转的七香酒虫。七香酒虫,亦有精炼真元的能力。”

%26
“方源啊,如果你不想贩卖酒虫,那么我们可以换另一种交易方法。你将酒虫卖给家族,家族若成功地合炼出了七香酒虫,那么你就有五年的使用权。如果失败,家族还会另给你一笔补偿。你看如何?”

%27
按这法子,无疑就将合炼的风险完全转嫁给了家族。这样优越的条件,若是其他人,恐怕都要怦然心动,答应下来。

%28
但方源却是心中冷笑。

%29
他有自知之明。

%30
以他四成四的丙等资质,晋升三转,几乎不可能。前世方源卡在这儿上百年,最终机缘巧合,得到了一只提升资质的蛊虫,这才成了三转蛊师。

%31
五年的使用权看似美好,其实对于方源来讲,飘渺得如雾中花水中月。

%32
古月青书之所以这么说,也是看出来方源有冲击三转的野心,故意抛出这个甜头,想让方源上钩。

%33
但是他从一开始,就失算了啊!

%34
合炼酒虫的秘方,方源的记忆中就存有一个最佳的。

%35
先是晋升二转四味酒虫,再晋升三转七香酒虫。不管是四味酒虫,还是七香酒虫,都具有精炼真元的能力。

%36
只是要合炼成四味酒虫,可不容易。

%37
首先,它的合炼需要两只酒虫,而方源现在手中只有一只。其次,合炼当中必须有四种美酒,这四种酒,分别要有酸甜苦辣四种味道。

%38
先不说酒虫有价无市,难以购买。

%39
就说这四种酒吧。

%40
辣酒最常见,一般的白酒都是辣酒。酸酒可取杨梅酒。葡萄酒,那都是酸的。甜酒,可取糯米酒。但是苦酒,就要费思量了。

%41
就方源所知,只有一种绿色的苦酒,用艾草酿造,在艾家寨。可惜这艾家寨,远在十万八千里。怎么得?

%42
酒虫一直被方源滞留在手中,并不是他待价而沽,想卖个高价。而是方源从一开始,就想走这条合炼晋升的路线,若换做其他路线,酒虫也被糟蹋了。

%43
古月青书怎么可能料到方源的心中所想。

%44
他见方源始终没有点头,终于抛出杀手锏:“方源啊。如果你出售酒虫,那么你和方正的关系我也可以稍作调解。至少不会让他用家产的名义。来和你斗蛊。你也知道。按照家族之规定,一旦下达战书,必须接受。哪怕是斗蛊的请求,得不到高层的批准,也得先接受下来。哪怕明知必败无疑,不上擂台,就直接认输。还得先接受挑战。”

%45
这个世界尚武,家族中不需要懦夫。一旦有战书,蛊师就必须接受。接受了就证明你不是懦夫。哪怕在在大庭广众之下承认失败,那也是勇气的行径。

%46
这在残酷的自然环境的压迫下,自然而然形成的价值观。

%47
而家族高层会根据斗蛊的结果,做个仲裁,解决问题和纠纷。

%48
当然,斗蛊的前提是必须事出有因,挑战方理直气壮、合情合理,或者双方达成了一致的类似赌约的协议,斗蛊才会得到批准。

%49
“方正的斗蛊请求,合情合理,应该会得到批准。这样一来,不管结果如何,是输是赢,都是家老评判。你觉得你和方正之间,家老会偏袒哪一个呢?”

%50
青书笑意越加浓厚,他目光灼灼地盯着方源,继续向其施加压力,“方源啊,你若赢了,也只是交还的家产少一点罢了。但是如果你将酒虫卖给家族,就是对家族的贡献。家族会记得你的。我在此保证,方正今后不会以家产的理由,来向你挑战。”

%51
言外之意,就是方正仍旧会找方源斗蛊,只是会用不同的理由。

%52
这也是古月青书,以及古月博乐意看到的事情。他们当然希望方正能击败方源,从而驱散心中阴影,树立自信。

%53
方源忽的一笑,从一开始就听着古月青书喋喋不休的劝说,他这时才开口说话。

%54
“如果交战,你觉得我会输吗?”方源问青书。

%55
青书也笑着,答道:“战斗嘛,自然充满了变数,谁也料不准。不过有必要提醒你一下,方正已经合炼成了二转蛊虫月霓裳,恐怕你的优势不大。”

%56
“呵呵呵。”方源摇摇头,脸上的笑容在扩大,“我会输的,一定会输的。”

%57
青书一愣。

%58
方源盯着他的双眼,又继续补充道:“我不仅输,还会输的很惨。双亲的遗产我将全部交出来,从此露宿街头,在山寨中流浪乞讨。”

%59
“你……”古月青书是个聪明人,他听出了方源真正想要表达的意思,顿时面色一变,再无见面以来就一直保持的自信风采,神情变得很凝重。

%60
方源的话,是赤裸裸的威胁。

%61
方正是作为下一代族长培养的,如果被爆出他重新认他人为父母,仗着修为和资质,欺压亲哥哥,夺走遗产的消息,将对方正的名誉将产生毁灭性的打击。

%62
就算是在地球上,做出这种行为的人也会被世人不耻、鄙视。更何况这个世界里,亲情的价值观被提高到一种全新的高度。

%63
若是方正是做魔头,那还罢了。要他做家族族长,正道领袖,那得维护道德,爱惜羽翼。

%64
一时间,古月青书怔怔地看着方源,他发现尽管自己已经很了解方源,但还是低估了他。

%65
从见面一来,他一句句话累积出来的优势,在这一刻轰然崩塌。

%66
方源一针见血之语,已经直指古月青书的要害之处。

%67
若是换做另一个人,方源自然会换另一种说法。但是他古月青书,对家族无比的忠诚,前世宁愿牺牲自己,也要保存家族。因此方源的威胁,令他不得不顾忌。

%68
但他很快就镇定下来,双目紧紧地盯着方源,咬牙道:“但你不会这么做的。因为遗产一直就是你的目的,你放弃了这笔遗产,如何修行?”

%69
方源毫不畏惧,迎上青书的目光,嘴角上翘,笑着:“所以,我相信你也会放弃收购酒虫的想法,同时劝说方正不要约斗我,不是吗?”

%70
换做其他人,还真说服不了方正。但是他古月青书却有这个能耐。

%71
这点方源毫不怀疑。

%72
场面一时间僵持住。

%73
片刻后,古月青书主动转过视线,垂下眼帘。

%74
他盯着手中的酒杯好一会儿,忽然笑起来。

%75
“有意思,那就这么办罢。”言语中带着一股郁气。

\end{this_body}

