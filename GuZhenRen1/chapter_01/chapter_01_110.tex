\newsection{老太婆,你太嫩了!}    %第一百一十节:老太婆,你太嫩了!

\begin{this_body}

%1
“酒虫……”方源口中轻声喃喃,走到中央的柜台旁。

%2
他只要有了这酒虫,再补足酸甜苦辣四种美酒,就能合炼成四味酒虫。当然这种合炼,自然存在失败的可能性。

%3
但是如果方源手中没有第二只酒虫,他连尝试一下的机会都没有。

%4
人生就是这样,努力不一定有好的结果,不一定成功。但是不努力,一定是失败。

%5
魔道亦是如此,魔道中人大多都擅长披荆斩棘,勇猛精进。这在那些大多数世人眼中,就显得偏激和冒险了。

%6
“我先前还在担忧,如何能找到第二只酒虫。想不到,命运就将这只酒虫送到了我的面前。机会就在眼前,我怎么能放弃?得到这只酒虫!”方源眼中闪过一抹坚定的光芒。

%7
“如果我修为达到五转四转,战力雄浑,那自然二话不说,抢了就走。神挡杀神佛挡杀佛!如果是三转四转,手段丰富,就可以偷盗窃取,神不知鬼不觉。可惜我现在只有二转修为,还只是初阶……”

%8
方源在心中暗暗叹气:“如此只能老老实实地购买了。”

%9
他看了一眼柜台上的标价。

%10
“酒虫——五百块元石。”

%11
酒虫的正常市价,是五百八十块元石。这里的标价比市价竟然还要便宜八十块元石。

%12
但方源若真觉得,自己花费五百块元石就能买到这只酒虫,那他前世五百年都白活了。

%13
标价之所以这样低。只是贾富想吸引人气,勾动人心理的购买欲望。

%14
这座树屋,很明显就是他贾富的产业。

%15
“奶奶,这只酒虫竟然只买五百块元石呀!”一个少女走到这里,小小的惊呼一声。

%16
少女的双眼泛着亮彩,摇晃着身旁母亲的手臂:“奶奶,明天就是开窍大典了。你不是答应过我,要送我一个礼物的吗?不如你就买了这只酒虫送给我吧。”

%17
少女的奶奶赫然系着白色腰带,腰带正面镶着方形银片。上面刻着大大的“三”字。

%18
蛊师一旦修行到三转,将自动晋升到家老之位。

%19
只是家老之间,也有区别。有的是当权家老。位高权重。有的则不是,管理着油水稀少的部门。

%20
但这个三转的老人家,显然不是那种蹩脚的家老。

%21
“古月药姬……”方源认出了她。此人是药堂家老,药堂是整个家族后勤的中心,可以说是最有油水的部门。古月药姬资格很老,就算是面对族长,也可以不行礼,坐着答话。她是族中治疗蛊师第一,曾经救过许多家老的性命,在族中人脉极强。

%22
“好好好。乖孙女想要,奶奶就给你买。”老人家满脸皱纹,勾勒着背,一手拄着拐杖,无奈地叹了口气。慈祥地笑着。

%23
“奶奶最好了,就知道奶奶最疼我了。”少女高兴地当场搂住古月药姬,开心地撅起小嘴,在奶奶的脸颊上亲了一口。

%24
“那奶奶,我们把这的店员叫出来,赶紧把这酒虫买下吧!”

%25
古月药姬摇摇头:“乖孙女啊。这里的蛊虫可不是这么买的。奶奶来教你,你看到柜台上的那叠纸和笔了吗?”

%26
少女懵懂地点点头:“看到了。”

%27
古月药姬:“你去取一张纸,用笔写上自己购买这酒虫的价钱。然后塞进柜台侧面的洞口里去。如果在所有想要购买的人中,你的价格是最高的,那么这只酒虫就归你所有了。”

%28
“原来这样啊,蛮有趣的嘛。”少女拿起一张竹纸,执着笔,写的时候却犹豫了。

%29
她皱起可爱的眉头,苦思冥想了好一会儿,终于还是微微撅嘴,问道,“奶奶,我究竟该写多高的价格才合适呢?我怕价格低了,酒虫被人买走了。价格高出别人太多,自己就吃亏啦。”

%30
古月药姬哈哈一笑,故意地逗趣道:“想要买酒虫哪有那么容易?就看乖孙女你的啦……”

%31
“奶奶!”少女撒娇,抱着古月药姬的胳膊一阵摇晃。

%32
“好了,好了,不要晃了。奶奶的身子骨都要被你晃散架啦。”老人叹着气,“奶奶帮你填就是了。”

%33
少女顿时小小的一蹦,娇声叫道:“我就知道奶奶对我最好啦!”

%34
古月药姬执笔写了一个价格,另外附属了自己的姓名。少女在一旁直勾勾地看着。

%35
老人写完之后,将纸折叠起来,向少女挤挤眼:“去吧,把这纸投进去。”

%36
少女乖巧地接过来,寻到柜台侧面的一个方口,将纸片塞了进去。

%37
她回到古月药姬的身边,有些不放心地问道:“奶奶,这样就可以了吗?”

%38
老人家点点头:“应该差不多了。但是世事难料,也许有人出价比奶奶的还要高呢。不过那个价格,就高太多了。真要出的比奶奶的还要高,那购买这酒虫的人真的是冤大头了。放心吧,这件事情已经十拿九稳了。”

%39
“哦。”少女点点头,神情可爱。

%40
“走吧。陪奶奶到上层去看看。”

%41
“好的,奶奶。”

%42
……

%43
望着祖孙俩离去的背影,方源的眼中闪过一丝凝重。

%44
这个古月药姬,对他来讲,的确是一个强大的竞争对手,不容忽视。

%45
不过这个情况,方源心中早已经有所预料和准备。

%46
酒虫珍贵,虽然只能对一转蛊师起到作用,但是酒虫能精炼真元,这点实在是优秀。真元精炼,提升一个小境界,这就意味着真元储备的增加,同时对于蛊师的修行进展有很强力的推动作用。

%47
唯一的缺陷,就是酒虫的发展前景不大。

%48
按照流传最广的晋升秘方。酒虫只是作为一种合炼的材料,合炼出来的新蛊虫,不再具有精炼真元的能力。

%49
这点实在是可惜,甚至有些得不偿失。

%50
所以很多家族有了酒虫,并不会将其合炼晋升,而是用于学堂,专门给新学员轮流使用。

%51
如果方源将他的晋升秘方。暴露出来,那么酒虫的市价必定将要暴涨许多。

%52
“想要得到这只酒虫,不容易啊。这对祖孙只是其中的一个对手罢了。不知道还有多少个其他的竞争对手,已经在这柜台中投了纸片呢?”

%53
酒虫是个好东西,好东西自然人人都想要。

%54
只是这些竞争者当中。有些人是诚心想要购买,有些人只是想撞撞运气。有些人财力丰厚,如古月药姬。有些人财力薄弱,如方源。

%55
“幸好我夺回了家产,这些天通过贩卖一转生机叶,还有酒肆、竹楼出租,积累了一些元石。否则我连竞争的资格都没有。”

%56
然而他终究积累的时间太短了,另一方面也有不少的蛊虫需要喂养,论财富哪里会是古月药姬这等家老的对手呢。

%57
“唉,走吧。药姬大人刚刚当众投了纸片呢。”

%58
“我也看到了,看来这酒虫注定和我无缘了。”

%59
……

%60
围在这柜台周围的蛊师,都一个个垂头丧气地离开。

%61
只有方源还站着。

%62
他的双眼如幽泉,闪着冷光。

%63
离开的蛊师,都被古月药姬的气势所吓倒。主动退出。但是方源怎么可能被吓住?

%64
“有时候机会就摆在眼前,只是人们主动地放弃了。我还有机会!”方源脑海中思绪翻腾,陷入沉思。

%65
要比拼财富,方源绝对不是古月药姬的对手。

%66
然而……

%67
这并不意味着,古月药姬的出价,就一定比方源要高!

%68
酒虫虽然珍贵。但它终究只是一转蛊虫。任何的一件商品价格都会浮动,但是绝不会无限制地上涨或者下跌,所有的价格浮动,都处在一个范围内。

%69
所以现在的关键是,古月药姬出了什么价格!

%70
她并不缺钱,为了疼爱的孙女,她能出多少?

%71
只要方源的出价,哪怕高出药姬出价一丁点,都是方源胜了。

%72
这是一场别开生面的战斗!强大者不一定是胜利者,弱小者未必会失败。猜测和赌博,让战斗显出别样的精彩。

%73
“如果是别人,兴许会猜不透。古月药姬,你刚刚故意说出了一些话,就是想要吓退一些竞争者吗?但是在我面前,你还太嫩了!”方源的嘴角微微翘起一个弧度,露出一抹自信的微笑。

%74
这个世界的贸易,其实很有意思。

%75
若放在地球上,卖家要贩卖酒虫这类珍稀的蛊虫,一般都会采取拍卖的形式。

%76
但是这个世界,拍卖并不盛行。

%77
一个重要的原因,就是亲情至高的价值观,即家族的凝聚力。

%78
如果举办一场拍卖会,家族的成员,面对贾富这种外人,都会在潜意识内升起一种同仇敌忾的情绪。

%79
拍卖的商品,一旦价格高了一些,许多竞争者就会自动退出。甚至会当场温言协商,做一些利益的妥协、交换和补偿。

%80
这个世界里的人,几乎都有一个观点——输给家里人不算什么,但是要让外人赚了钱去,那是对整个家族的羞辱。

%81
除非是,几个家族同时参加一个拍卖会。这样一来,就有竞争,充满了火药味。

%82
但这种拍卖会,很难举办。

%83
因为交通不便利。

%84
交通是商业的基础,交通不发达,商业就凋敝。因为商业从根本上说,就是货物的流通。

%85
各个山寨占据着一座座山头,相互距离遥远。中间的路途难以行走,有兽群出没,有悬崖峭壁,有恶劣气候,有危险的野生蛊虫,充满了艰难险阻。

%86
如此不便利的交通,很难将各方人士组织起来,参加一场拍卖会。

%87
就算是在青茅山,有三家山寨相邻。贾富也不敢组织拍卖会。

%88
首先拍卖会在哪里举行呢?在野外举行不安全,在古月山寨举办,其他两方不放心。

%89
他只是四转修为,三家山寨的族长也都是四转,他也难以镇住场面。

%90
比起地球来讲,这个世界的商业并不发达,有着独特的规则。

%91
方源前世靠着地球上的熏陶出来的商业理念,赚过钱财,也亏过本。历经了不少血泪教训,从实践得出真知。

%92
结合地球上健全丰富的商业理论,再加上亲身实践的经验,毫无夸大地讲,方源对于这个世界商业的认知,绝对占据世界一流水准。

%93
就凭区区一个生活在山寨中,从未出过青茅山的老太婆,也想阻止我得到酒虫?

%94
老太婆,你太嫩了!

\end{this_body}

