\newsection{春光正明媚}    %第二十五节:春光正明媚

\begin{this_body}

%1
“他是方源,还是方正?”有些学员在嘀咕,仍旧有人分辨不出方源和方正这两个孪生兄弟。

%2
“是方正。方源总是一脸冷漠,脸上绝不会出现紧张的神情。”有人解答道。

%3
“哦,那就有看头了。方正可是我们山寨三年唯一的甲等天才呢。”众人纷纷投去目光。

%4
方正感受到了这些目光中蕴含的压力,这让他更加紧张。

%5
站在场上,他手指都在微微颤抖。

%6
他打出第一记月刃,原本是瞄准的草人胸膛,但是却因为紧张的缘故,而打偏了。最终月刃印在草人傀儡的脖颈部位。

%7
少年们立即传出一阵轻微的惊讶声。

%8
他们以为这是方正有意为之,不打最容易命中的傀儡胸膛,而是那脖颈,这是对自己攻击手法自信的表现。

%9
不由地,更加期待方正接下来的表现。

%10
古月漠北和古月赤城二人,亦是面色微沉

%11
能看出方正失误的,场中只有学堂家老和方源二人。

%12
“好险!”看到这记月刃,方正心中惊呼一声,暗暗觉得侥幸。

%13
他几口深呼吸,强制镇定下来,再发两道月刃。这次他没有再失误,两记月刃都打在草人傀儡的胸膛,准确命中。

%14
这个结果,让学堂家老点点头,漠北和赤城则镇定下来。方正的这个成绩,和他们不相伯仲,就看学堂家老怎么评分了。

%15
其他的学员们,则发出声声叹息。方正之后的表现,并不出彩,让他们有些微微的失望。

%16
接下来的几组,就没有什么看点了。再没有人能表现得比漠北、赤城、方正他们三人更好。

%17
少年们开始窃窃私语起来。

%18
“这样看来,今天考核的头名应该在他们三人当中产生了。”

%19
“他们三人都击中了草人傀儡,不知道家老大人会更看好谁。”

%20
“等等,到最后一组了,方源上场了。”

%21
“哦,就是那个丙等的‘冷酷天才’?呵呵。”

%22
直到最后一组,方源才施施然上场。

%23
“是那个方源……”古月漠北抬头看了一眼方源,又垂下眼帘,不是很在意。

%24
“上次让你走了狗屎运,意外选了一个意志薄弱的月光蛊,才让你夺得第一。这次看你怎么表现!”古月赤城环抱双臂,等着看方源的笑话。

%25
“哥哥……这次可不比上次了,我努力练习了那么久,一定能超越你。”人群中,古月方正抿着嘴唇,双拳下意识地握紧。

%26
上次炼化本命蛊的考核,他以甲等资质却屈居第二,自然并不服气。

%27
尤其是当他了解到方源能够获胜的原因,竟然是因为运气好,才夺得第一的。这让他更加不甘心。

%28
对于古月方正来讲,战胜自己的哥哥方源,对他有着特殊的重大意义。

%29
不少视线集中在方源的身上,学堂家老的目光,也凝视向他。

%30
方源毫无动容,表情冷漠。

%31
他站定之后,真元涌入掌心中的月光蛊,手掌一切,发出第一记月刃。

%32
月刃飞得很高,不仅越过了草人傀儡的头顶,还高出了竹墙。飞了将近十五米的距离,这才光芒黯淡下去,消失在空气中。

%33
“噗嗤……”有人忍不住笑出声来。

%34
“这也偏得太离谱了吧。”有人嘿然冷笑。

%35
“的确是天才呀,难怪得了炼蛊第一呢。”有人说着嘲讽的话。

%36
早些年,方源创作诗词,展现出早智的时候,就引起了这些人的不满。后来又靠着“运气”得了炼蛊第一,更让他们不满的情绪中,又增添了一份嫉妒。

%37
很多人都等着看好戏,等着方源这个“天才”出丑,而方源这记月刃也没有让他们失望。

%38
人群中嗤笑声连成一片。

%39
学堂家老微微摇头,心中也笑自己,凭白无故关注方源做什么?他不过是个丙等,只是因为一时运气夺了炼蛊头名罢了。

%40
他在心中已经打定主意,虽然漠北、赤城、方正的成绩都是一样,但他会选方正为第一。

%41
古月漠北和古月赤城的争斗,就是家族中两大当权家老的政治斗争的缩影。学堂家老一直是中间派,不想参合到政治漩涡中去。

%42
学堂家老更倾向于族长古月博,而方正正是族长一系的人。加之他是甲等资质,选他为第一,对他有些偏颇关怀,也能让家族高层接受。

%43
一阵温暖的春风吹来,花香阵阵,飘入演武场。

%44
阳光照在方源的身上,在地面上照射出一个孤零零的黑影。

%45
他表情仍旧冷漠,静静地望着十米开外的草人傀儡,手中心的月刃印记正幽幽地散发着水蓝光辉。

%46
第一记月刃,当然是他有意打偏。现在他只剩下两次出手机会,再考虑到学堂家老的立场,他要夺得第一,就必须在仅有的两次攻击中,制造出远超众人的攻击效果。

%47
“仅剩下两次出手机会,不可能了。哥哥,我终于赢你了。”古月方正双眼一闪不闪,盯着方源。

%48
从小到大,哥哥带给他的人生阴影,终于在此刻渐渐消褪。

%49
方正感受到胜利已经近在咫尺,他双拳下意识地捏紧,全身都激动得微微颤抖起来。

%50
“哥哥,这一次赢你,只是一个开始。接下来,我会一次次的赢你,直到将我心中的阴影全部驱除。我要像族人们证明,甲等天才的优秀!”方正在心中对自己说道。

%51
然而就在这时,方源出手了。

%52
右掌如刀,虚空一劈。

%53
哧的一声轻响,笼罩在手掌上的水蓝光辉,便脱离而出,飞到空中,化为一弯蓝光月刃,射向草人傀儡。

%54
仅仅是在下一秒中,方源的右掌上又再次亮起一层蓝芒。

%55
他手掌一翻,便斜劈出去第三道月刃。

%56
他这两次攻击衔接得行云流水,恰到好处。

%57
两道月刃接连飞出,在空中相距仅仅不到半米之远。

%58
在众人惊愕的目光中,两道月刃都准确级命中草人傀儡的脖颈。

%59
“这……”方正瞳孔猛缩,他的心中陡然涌现出一股不妙的感觉。

%60
在下一刻,学员们带着惊讶的神色,缓缓地张大了嘴巴。

%61
他们看到草人的头颅先是慢慢倾斜到一边,然后从脖子上落下去,最后掉在地上,弹了一下,滚出两三米的距离。

%62
方源斩落了头颅!

%63
这样的结果,出乎在场所有人的意料。

%64
“这是运气还是实力?”学堂家老皱起眉头。

%65
这样的疑惑,同样盘旋在其余学员们的心中。

%66
一时间,演武场上陷入了沉默。

%67
“怎么会这样……”方正失声喃喃,他呆呆地看着方源,心中的澎湃一下子落空,陷入了深深的低谷。

%68
方源眯了眯眼睛,似乎根本就没有察觉到众人落在他身上的目光。

%69
咕咕咕……

%70
蓝天白云之下,一群彩雀鹦鹉忽然扑腾着翅膀,飞上半空中。它们拖着华美修长的雀尾,咕咕地叫着,在空中飞旋嘻戏。

%71
方源站在演武场的中央,仰头望去。在灿烂的阳光下,鸟儿的七彩羽毛,显然更加耀眼绚烂。

%72
他表情一片淡然,仿佛刚刚斩断草人头颅的,根本不是他一样。

%73
“春天的阳光,还真是明媚啊……”他在心中叹息一声。

\end{this_body}

