\newsection{中阶!}    %第五十节:中阶!

\begin{this_body}

夜幕降临下来。

月亮如一块银盘,出现在云间。稀疏的星辰,点缀在周围。

古月漠北站在院子里,仰首看着月亮,双目炯炯发亮。

“小弟,听说你今天受伤了。”身后忽然传来姐姐古月漠颜的声音。

“姐姐,你是担心我今天被打得吐血,会因此有了阴影吗?”漠北转身,嘴角一翘。

漠颜看着弟弟笑着,心中顿时舒了一口气。她虽然真的有这份担忧,但是嘴上却道:“哪里有,姐姐我最了解你的。好弟弟,你有百折不挠的斗志,是我们漠家未来的掌权人,怎么可能会被这样的小伤势吓住?”

“呵呵呵,就知道姐姐最疼爱我了。”漠北挠挠头皮,嘻嘻笑着。

“你知道吗,姐姐。”明亮的月光下,这个十五岁的少年双目绽放着明亮的光泽,“虽然我这次又失败了,但是我却在交战中,听到了方源那家伙的急促喘息。以前我都是三两下就被他打趴下,而他总是气定神闲的。他的喘息声,已经透露出他的虚弱。他绝没有大多数人想象的那般强大。总有一天,我会堂堂正正地击败他!”

“好,不愧是我漠之一脉的好男儿!”漠颜朗笑一声,摸摸弟弟的脑袋,脸上露出关切的神情,“不过你受了内伤,最近几天就不要练拳脚了。”

“姐姐,你别这么摸我头,我已经长大了。”漠北摇晃了一下脑袋,用略带不满的语气道,“你说的我都知道。我已经有了计划,正好趁着这些天,来温养空窍四壁,完成从初阶到中阶的晋升。先取得班头的位置,把方源那家伙的风头压下去。让他知道,蛊师修行真正注重的,还是资质!”

“你这么想,姐姐我就放心了。姐姐以前也只是副班头。弟弟你若当了班头,也算是为姐姐弥补了遗憾。”

“姐姐,你就放心吧。班头的位置,我必取之!”

几乎同一时间,在赤家。

密室里,只有一把火炬,插着青石墙壁的凹槽上。

火熊熊地燃烧着,照亮这间不大的密室。

两大当权家老之一的古月赤练,正面对着他的亲孙子古月赤城。两人盘坐在蒲团上,影子投射在地上,随着火焰的跳动而不断摇曳。

古月赤练伸出大手,用掌心紧贴赤城的小腹部位。一缕缕的白银真元,顺着老人的心念,倒灌进古月赤城的空窍当中。

古月赤城满脸的紧张,心神全部投入到自己的空窍当中,极力压制着自己的元海波动。

这个世界上,没有两片相同的树叶。对于不同的蛊师来讲,亦绝没有完全相同的真元。

异种真元一旦进入空窍,就会引发空窍中原有真元的自然反抗。

古月赤城若不压制,任由真元反抗的话,必定会引发真元对冲。对冲引发的剧烈波动,会对空窍造成损伤。

空窍元海是蛊师修行的一切根基之地,重中之重。

空窍一旦受伤,轻则修为下降,重则资质降低。一旦空窍完全破碎,蛊师就是直接死亡。

过了好一会儿,古月赤练这才缓缓停下真元的灌注,慢慢地收回手。

古月赤城顿时长长地舒了一口气,紧张的身躯也放松下来:“谢谢爷爷,每隔三天都要给孙儿灌注真元,温养空窍。让您劳累了!”

古月赤练满头的大汗,他叹了一口气,道:“这也是没有办法的事情。你的资质只有丙等,若单靠你的力量,想要晋升中阶,需要花费很长的时间。这个时间,常常是乙等的两倍,甲等的四倍。这样一来,你就会露馅。所以这个方法虽然凶险,也不得不用了。”

“孙儿明白爷爷的良苦用心。”

“你明白就好。”老人又叹一口气,“此法还有一个很大的后遗症。就是窍壁是被我的白银真元温养的。白银真元虽然效果卓越,但是对你来讲,毕竟是异种真元。今后,就算是窍壁从光膜转变成水膜,但也掺杂了我的气息。异种气息越来越多,空窍就不存,是对你的拔苗助长,会缩小你将来的发展空间。”

古月赤城紧紧地抿着嘴:“爷爷,为了我赤家的将来,孙儿愿意牺牲个人的前景!”

古月赤练顿时老怀大畅,点头抚须道:“你能有这样的心思,很好,很好。不过天机常留一线,你也不是毫无希望。若是将来能寻到净水蛊,就能洗练你的窍壁,将空窍元海中的异种气息都化去,将此等后遗症消除。”

“还有。爷爷已经托关系,让人为你四处寻购酒虫。此虫能在一转境界的修行中,帮助蛊师精炼真元,提升一层小境界。如此精炼出来的真元,是你的本身真元,而不是异种真元。用来温养空窍的话,没有丝毫的后遗症,亦没有风险,温养的效果更佳!”

古月赤练大喜:“谢谢爷爷!”

“不过,酒虫难寻啊。一转蛊虫当中,酒虫、豕蛊、书虫等等这些,都是最珍稀的蛊。一旦在市面上出现,几乎第一时间就被人购走了。当然,这个世界上还有一些蛊虫,传说能改变蛊师的资质。但是爷爷这大半辈子,也只是隐约听见过风声,从未有亲眼见过。”老人语气悠悠。

夜风徐徐,从窗户中吹进了房间。

古月方正趺坐在床榻之上,双目紧闭,两手各握紧一块元石。

青铜元海无风生潮,一波波冲刷着白色的窍壁。

他有甲等资质,真元海占据整个空窍八成以上。本身真元的恢复速度,整整是方源的两倍!

这种得天独厚的优势,使得他已经接近一转中阶。

呼。

良久之后,古月方正吐出一口浊气,缓缓地睁开双眼。

窗外月明星稀,碧青的竹楼排开而去。

一片安静和祥和。

“修行的时候,时间总是过的非常快。转眼间,就到了深夜。”方正口中翕动,轻声喃喃。他慢慢地松开双手,顿时就有两捧灰白色的石粉,从手中洒落到床前的地板上。

元石中的天然真元一旦被完全汲取,就会化为一堆石粉。

看着石粉洒落,方正微微皱起了眉头。

他从怀中掏出钱袋,钱袋早已经干瘪。

打开袋口,看到袋子里只剩下三块元石。

方正每七天都会从学堂领取三块元石的补贴,但是要被方源抢去一块,因此落到他手中的,只有两块。

舅父舅母会给他生活补贴,同样是七天三块元石。

就这些元石,哪里能够用?

方正一心想要超越哥哥方源,因此数次主动找上舅父舅母,求了一些元石。

次数多了,舅母就找他谈心,告诉他现在家境困窘,支出困难,没有闲钱。

方正也就不好意思求了。

“父亲母亲为了我的修行,已经尽了全力来资质我。我不能再为难他们,向他们索要元石了。只剩下三块元石,那我就更得节省着用。每天使用一块。这样就能使用三天。”

“我有一种感觉,三四天之后,我必定能晋升中阶!就是不知道哥哥那里,是什么进度?”想到这里,方正就下意识地望向窗外,看着学堂宿舍的方向。

“我有甲等资质,哥哥只是丙等,他的进展一定没有我快。哥哥此次绝不会是我的对手!哥哥,我要让你知道,甲等资质的厉害!”

想到这里,方正不禁握紧了双拳。

学堂宿舍。

方源的房门,关的紧紧的。

黑暗中,他并没有睡下,而是盘坐在床上。

蛊师的修行,并不能取代睡眠。以往这个时候,方源已经躺下入眠了。

但是今天在刚刚的修行中,他感觉到只差一丝,就能踏入中阶。

“索性今晚就不睡了,直接冲刺中阶!”他的眼中闪过一抹坚定的光。

随后,他闭上双眼,心神投入到空窍当中。

四成四的青铜元海,都在刚刚,被酒虫精炼成苍绿色的中阶真元。

“起。”方源念头一动,平静的青铜元海顿时荡漾起一波波的涟漪。

涟漪越来越多,越来越大,形成一波波的浪潮。

哗哗哗……

浪潮前仆后继,向四周的窍壁冲撞过去。

就像是撞在了礁石上,大部分的真元碎成了翠绿的浪花,高高飞溅,然后又融入海中。

少部分的真元则消耗掉,化为一丝丝奇妙的无形力量,渗透到白色的光膜窍壁当中。

“再起。”方源心神鼓动,苍绿的波涛规模越来越大。先前的浪潮,顶多是狗兔奔逐,现在却像是一支支的马群,朝着空窍光膜撞去。

马如龙,浪飞天!

真元发生剧烈的消耗,水面不断地下降。

哗哗哗!

浪潮波涛连绵不绝,终于量变引发了质变。

那白色的光膜忽然一阵颤动,原本柔和的白光,骤然间绽放出咄咄逼人的光亮。

见此情景,方源心中大喜,知道是关键时刻,连忙鼓动真元,继续冲刷不停。

白光越来越亮,光线发生扭曲,纠缠在一起,给人越来越浓稠的感觉。

十几个呼吸的功夫,在光膜之上出现了一条条的光带。光带好似一股股水流不断流转,不断相撞。

在撞击的过程中,它们不断地合并壮大,形成白色的光流。

终于,光流连成一片,完全覆盖了光膜。

白光黯淡下去,原先空窍周壁的白色光膜已经消失了,取而代之的是一层球形的白色水膜。

光膜表面一片光滑,毫无杂质。而水膜明显比光膜更厚,上面波光流转,明灭不定。

真元海面恢复了平静,空窍中还剩下两成海水。

“晋升中阶了!”方源朗声一笑,睁开双眼。

明亮的阳光透过窗户的缝隙,钻了进来。

不知不觉,一夜过去,已经是上午了。

\end{this_body}

