\newsection{魔性(第一更)}    %第一百五十一节:魔性(第一更)

\begin{this_body}

“什么?”古月药乐闻言,不禁一愣。

方源已闪电般出手,手掌往少女细嫩的脖颈上一切,便将其击昏。

少女软倒下来,方源手臂快速一捞,将其拦腰夹住。隐鳞蛊催动起来,他俩的身影消失在原处。

当古月药乐模模糊糊地睁开双眼时,她发现自己身处在一个阴暗的山洞里。

她甩了甩昏昏沉沉的脑袋,下意识想站起来。

但她很快发现,自己被五花大绑着,绑在一块石头上。

她身上的所有蛊虫,都被方源利用强取蛊取走,炼化成了他的东西。

她一个十五岁的少女,身娇体弱,怎么能挣脱一指粗细,还绕了几匝,打上了死结的麻绳?

被捆在这陌生的环境当中,少女心中自然涌起一阵惊恐。

她回想起自己被击昏的那一幕,就算是再天真的人,也知道方源要对她不利了。

“但是方源会怎么对付我呢?他究竟要对我干什么?难道是因为艘向奶奶打了他的小报告,要来报复我?”少女四肢动弹不得,但思绪却在急速地翻腾不休。

她越想越恐惧,不禁流下泪来。

“奶奶,你在哪里呀?快来救救乐乐呀……”她哭泣着,感受到前所未有的孤单和害怕。

方源不知去了哪里,山洞中回荡着她的哭泣声。

“难道方源是想把我囚禁在这里?饿了七八天,让我尝到苦头。以后再不敢说他的坏话?”古月药乐哭了一会儿,忽然想到一种可能。

太坏了!

方源,我一定不会放过你的!!

她恶狠狠地咬牙,本来心中对方源的印象就极为不好。如今已经差到谷底去了。

古月药乐从小到大,还没这么恨过一个人。

就在这时,一阵脚步声传来。

不一会儿,方源的身影就从阴暗中显露而出。

“方源,你想要干什么,你快放了我!不然,我奶奶会收拾你的。”看到方源,古月药乐极力挣扎。一双嫩腿蹬着地面,就像是一只掉入陷坑的小鹿。

“倒是很有活力嘛。”方源冷哼一声。

古月药乐刚要开口,继续怒骂,忽然看到方源身后。缓缓走来一只大熊。

“熊,熊……”她瞪大双眼,惊慌的说不出话来。

方源呵呵冷笑,伸出手来,抚摸着熊的黑色毛皮。充满寒气的声音在洞中如阴风穿梭:“狼潮之下,要找到这样一头野熊,可不太容易,费了我不少时间呢。”

古月药乐旋即反应过来。她心思灵动,很快就想到方源曾经从熊骄嫚手中。索要了一只驭熊蛊。

“原来是这样……”她冷笑一声,张开欲言。冷不防方源忽然走过来,蹲到自己的面前。

“你想要干什么?!”少女尽量的往后缩,但方源轻而易举地就用右手,牢牢捏住了她的脸颊。

“长相可爱水灵,的确讨人喜欢。”方源淡淡地评价了一句。

哧!

他右手顺势而下,拽住药乐的衣领,用力一扯。

衣衫顿时破裂,露出里面的粉色肚兜。

“啊——!”少女愣了一下后,猛地发出剧烈的尖叫,疯狂挣扎。就算是细嫩的肌肤被麻绳勒出了道道血痕,她也不管不顾。

方源冷笑一声,继续撕扯。

嗤嗤嗤。

很快,少女衣衫褴褛,只剩下一些可怜的残破布片,露出大片大片牛奶般白嫩的少女肌肤。

“不要,不要!”她感到害怕极了,大声哭号。想到方源要教训自己的某种可能,她整个身躯都在颤抖着。

方源却没有再如她料想般,继续动作下去,而是站起身来,慢慢后退。

少女的哭号转为哽咽。

但就在这时,黑熊迈开熊掌,缓缓走了过来。

少女惊得瞳孔缩成针尖,在这一刻,她感到了强烈的死亡气息。

呼!

熊掌拍出,发出破空的风声。

咔嚓一声脆响,少女的头颅遭到熊掌一拍,巨大的力量让她娇嫩的脖颈瞬间折断。

她的头以一种诡异的扭曲角度,折到了一边。

前一刻,还活色生香的小美人,这一刻就已经香消玉殒。温热的尸体被绑在巨石上,仿佛是一件被玩残的娃娃玩具。

此时不消方源指挥驭熊蛊,处于觅食的本能,黑熊就已经低下头,开始享用这份丰盛的美食。

它先从少女的咽喉啃噬,鲜血顿时喷涌出来,飞溅在它黑色的毛皮上。

然后是少女白皙细嫩的胸脯,如一对未盛开的花苞。

黑熊一口咬下少女的右胸,撕开皮肉,露出惨白的肋骨骨架。

这时,黑熊动用熊掌一拍,就拍断这些肋骨,少女的内脏受到挤压,顿时一阵鲜血狂涌。

没了骨骼的阻碍,黑熊将嘴深进少女的体内,咬在仍旧怦怦跳的心脏上,然后一口吞下。

心脏通过咽喉,滚入肚子里,这只因为狼潮不得不躲藏着,好久没有进食的黑熊,顿时满足地嚎叫起来。

叫了一声之后,它再次低下脑袋,大肆食用少女的五脏六腑。

扑哧扑哧。

黑熊张嘴咬合,大量的血液就从嘴中泄出来,发出水声。

好一会儿,黑熊这才钻出头来。

少女的胸腔已经一空,巨大的伤口,拉扯到腹部。但对于白花花的肠子,黑熊似乎没有多少兴趣。

它开始将注意力,转移到少女如藕般白嫩纤细的腿脚上。

少女的手,根根都是芊芊玉指。被黑熊一口要下,几下咀嚼。吞入腹中,发出嘎嘣嘎嘣的脆响。

少女的大腿,亦是美味。

皮肉细嫩,带着一股处子芳香。黑熊意犹未尽地吃完之后。只剩下白色的腿骨。

摇晃当中,少女的头颅终于掉在了地上。

坦白来讲,她长得的确可爱。有着一双黑亮明丽的大眼睛,鼻子圆润,微微翘起,肌肤粉红若花,樱桃小嘴中两排洁白的贝齿。

但现在,她脸上血色褪尽。肌肤变得惨白。一头青丝披散开来,遮住大半个脸面,一对眼睛瞪着,充满了惊恐和怒怨。

死不瞑目!

方源抱臂旁观。欣赏着古月药乐的表情,不由地想到地球上一句佛语,曰:无我相,无人相,无众生相。无寿者相,红粉骷髅,白骨皮肉!

我即是自我,没有自我。打破自我主义,认知自己的普通平凡。“无我相”就是“人人平等。没有区别”。

人是人类,不再把人当做高贵。把其他生物贬斥为低贱。“无人相”便是“众生平等,没有区别”。

众生是一切的生命,不再把生命当做高贵,认为其他无生命的山石水流都具有灵性。就是“无众生相”,即“世间一切都平等,没有区别”。

任何事物都有各自的寿命,“无寿者相”即是“不管存在的,还是不存在的,都是平等,皆无区别”。

再漂亮的男女,最终都要变成骷髅。白骨和皮肉皆是一体,但世人多恋皮肉,而恐白骨,这就是着相,没有认识到此中的平等。

这佛语就是叫人破一切相,见真相。

美色是相,人我众生寿者亦是相。见诸相非相,即见如来。

看透看破,一视同仁,众生平等。

所以佛祖舍身喂虎,割肉喂鹰。这是他心存大慈悲,视万物为己出,普爱一切,大爱任何。

不管是我,他人,还是动物植物,甚至无生命的山石水土,以及不存在的东西,都去爱它们。

凡人站这里,看着熊吃人。一些热血男儿,丁当跳出来,大吼一声:“畜生,休得吃人!”亦或者“美少女,不要怕,叔叔来救你!”等等。

这都是凡人的爱恨,爱少女恨巨熊。还没有看透,着相执迷,看不穿红粉骷髅。

佛祖若站在这里,看着熊吃人。会叹息一声,唱一声佛诺:“我不入地狱,谁入地狱?”把少女救下来,把自己喂给黑熊。

这就是佛的爱恨,爱少女亦爱巨熊,一视同仁。

但此刻是方源站在这里。

看着少女的惨死,他的心中涟漪不生。

这倒并非是见惯了生死的麻木,而是破相而出,没有了执迷。无我相,无人相,无众生相,无寿者相……

看万物都是一视同仁,众生皆是平等。

所以少女的死,和一只蚂蚱,一头狐狸,一棵树的死,没有区别。

但在凡夫俗子的眼中,少女的死会引起他们的愤怒、仇恨和惋惜。若换做少女吃熊,他们却不会觉得有多过分。换做一个老太婆被吃,他们心中的惋惜就会大大降低。换做一个恶贯满盈的杀人凶手被吃,他们会拍手称快,反而大声叫好。

其实,万物平等,天地不仁以万物为刍狗。

大自然是公平的,不讲爱恨,是无情的,从不会区别对待万物。

弱肉强食,优胜劣汰!

一个生命体的消失,对于整个广袤的自然界,对于深邃浩瀚的星空,对于滚滚而流的历史长河,又算得什么呢?

死了就死了,谁能不死呢?什么少女、巨熊、蚂蚱、狐狸、树木、老太婆、杀人犯,都是卑微!都是低贱!都是刍狗!

只要认识到这一点,破一切相,见真相,就有了神性。

这神性往光明处踏出微微一步,就是佛。往黑暗迈出半步,就是魔。

魔性!

(ps:本来想写个名副其实的人兽葬生蛊。但考虑到河蟹,就只写“葬生”了。前面那部分,魔道中人请自行脑补之。另外:130985234,这是官方1群,普通群,不需要全订阅,也不需要验证。今晚俺开始在线了,有兴趣的同学可以加一加,大家聊聊剧情蛮好的。)

\end{this_body}

