\newsection{力可扛猪思低调}    %第七十一节:力可扛猪思低调

\begin{this_body}

%1
时间过的飞,似乎是弹指一挥间,就到了六月中旬。

%2
夏夜,金色的月轮高高地悬挂在空中,普照青山大地。

%3
风一阵阵地吹着,树叶和月光一起浮动。蝉鸣蛙鼓齐齐上阵,偶尔,也会有一声狼嚎,从很远的地方传来,在青山上回响。

%4
一处河滩,溪水冲刷出一片白色的圆滑岩石。在河滩旁,一场战斗在进行着。

%5
一只山猪浑身都是细长深邃的伤口,它四脚踢踏,再度冲向方源,伤口处血液飚射,印下一路的鲜红。

%6
方源与其周旋,毫不慌乱。

%7
这只山猪已经重伤濒死,但恰恰是这种状态的野兽为危险。它们将迸发出生命的后余光,更加疯狂和难缠,一不小心就会遭个跟头,被獠牙捣烂肚肠。

%8
方源面色平静,幽幽的黑眸在月光的照耀下,闪着冷冽的光。

%9
他有五百年的经验沉淀,一方面沉浸在战斗中,另一方面一部分心神又脱离了这场战斗,时刻警惕着周围的动静。

%10
有好几次猎杀山猪的时候,都有其他的生物过来搅场。有一次是一只山猪,还有几次是野狼,甚至有过一次,是流浪过来的一只老虎。

%11
随着时间的流逝,山猪的攻势终于缓慢了下来。

%12
方源双眼精光大放,猛地几个跨步,忽然接近山猪,然后矮身提肩,将山猪呼的一下就扛了起来。

%13
喝!

%14
方源低喝一声,涨红了脸面,双手一撑,就将山猪高举起来。

%15
山猪微弱地挣扎着。

%16
方源身躯摇晃不定,使出了全身力气,猛地掷出山猪。

%17
砰的一声闷响,山猪被砸在河滩的一块巨大岩石上,它凄惨地嚎叫一声,其中伴随着肋骨折断的脆响。

%18
它从巨石上摔落下来,从耳鼻口中,向外喷涌出一股股滚烫的鲜血。

%19
它又挣扎了几下,终于没有了气息。

%20
周围恢复了平静。

%21
河水潺潺地流动着,将猩红的猪血顺带着,流淌到远方。

%22
“我现在的力量,已经能抗起一头猪了!今晚就试试那个甬道的巨石。”方源站在原地,喘着粗气,眼中藏着一股兴奋。

%23
这些天他不断利用白豕蛊,映照自己的身躯,增长力量。他可以明显地感受到,自己的气力越来越大。

%24
以前和野猪对战,只能利用月刃游斗,现在他甚至能扛起了野猪。力量的成长有了长足的进展。

%25
当然,白豕蛊不会无限制地给方源增长力量,极限的程度是一猪之力。到达这个极限,就不能在利用白豕蛊,增长气力了。

%26
“我现在能扛起一整只野猪,但并不意味着能直接和猪角力。就像是一个壮汉可以抱起另一个壮汉,但未必两个壮汉之间的力量存在差距。我的力量,还能再成长。”

%27
将猪肉全部喂给了白豕蛊,方源又用猎刀取下猪牙,后将原本就破损不堪的野猪皮割碎,这去了石缝秘洞。

%28
至于这猪尸,并不需要他处理。夏夜里,野兽出没频繁,估计过不了一会儿,就有野兽闻到这血腥味,赶过来为方源清场。

%29
退一步讲,就算是有人发现也不打紧。野猪身上的伤口,方源都用刀“加工”了一下,看不出月刃攻击的痕迹了。

%30
回到赤光笼罩的秘洞中,方源随手将两个野猪牙抛在了角落里。

%31
野猪牙相互碰撞,发出咚的一声脆响。

%32
在这个角落里,已经有了一小堆的野猪牙。都是近方源狩猎的成果。

%33
方源径直钻进了甬道,再次来到甬道底部。

%34
走在甬道中,脚步声在甬道里面嗡嗡地回响。视野中一片昏暗的红光。

%35
一切都没有变化,巨石仍旧在那里,静静地横亘在前方。至于挖出地藏花的那个坑,已经被方源重填上了。

%36
喝。

%37
方源来到巨石跟前,举起双手,奋力猛推。

%38
然而他脸庞发涨,使出了吃奶的力气,巨石却纹丝不动。

%39
“我如今奋起全力,能够勉勉强强扛起一头野猪。但是这巨石,至少是五六头野猪合起来的重量。我推不动,也不奇怪。花酒行者的力量传承,没有那般容易!”方源目光闪了一下,心中估算着。

%40
他也不气馁,钻出甬道,回到上面的密室。

%41
将靠在墙角的竹筒取来,方源直接盘坐在地上,拔开竹筒的盖,将里面的兽皮地图和一张张竹纸都倒了出来。

%42
他将兽皮地图展开,又开始了默记,时不时地用手指在地面上比划出浅浅的痕迹,帮助自己记忆。

%43
从得到这张图的那天起,他就开始这样做了。

%44
方源没有储物功用的蛊,不能将着兽皮地图随身携带。身上背着一个竹筒进行战斗的话,会很不方便。所以,方源就下了苦功,准备将着兽皮地图装在脑里。

%45
有些事情虽然麻烦,但是好都做了。在人的一生当中,往往害怕麻烦而不积极解决的人,往往后面临着相当麻烦的处境。这个道理,方源前世就懂了。

%46
“年轻的时候,记忆力就是好啊。我现在已经将大半个地图,都默默地暗记在脑中了。若是在老年时候,背这玩意,估计一边背着,一边忘着。呵呵呵……当然如果我有书虫,就将这兽皮地图喂给它吃,从此以后只要书虫不失,我就能永远清晰地记得这张地图。”

%47
方源有了酒虫、白豕蛊,又开始得陇望蜀,希望得到书虫。

%48
书虫的价值,和酒虫,白豕蛊相当,都是一转蛊虫中珍稀品种,市价昂贵,并且一直是脱销状态。

%49
前世的时候,方源没有得到过酒虫、白豕蛊,反而意外地得到过一只书虫。这只书虫后来不断晋升,一直陪伴着他整整六十年。

%50
“算了,书虫向来数量稀少,短时间之内空想也得不到。其实说起来,重生之后我如今的处境,比前世同期不知道要好上多少倍了。前世这个时候,我还在一转初阶,而其他人诸如方正、赤城、漠北,都已经到了高阶,远远地把我甩在身后了。”方源不是那种钻牛角尖的人,想想也就释然了。

%51
他对如今的进度,还是比较满意的。

%52
自己是中阶,其他人也是中阶。凭借丙等资质,能吊住这些甲等、乙等资质的同龄人,非常不容易。这其中,酒虫的作用居功至伟,除此之外,就是方源丰富老道的修行经验的辅助。

%53
还有一方面,也是他的原因。

%54
他抢劫勒索所有同窗,这些人无不在长辈的指导下,奋发努力地锻炼拳脚功夫。因此在无形中,就牵扯了他们的精力,缩减了他们温养窍壁的时间。这就导致今生,他们的修为比方源前一世还有薄弱一些。

%55
不过就算如此,他们现在距离高阶,也不远了。

%56
蛊师的前期修行,算是比较容易,短时间之内就能看到效果的。尤其是古月方正、古月漠北还有古月赤城这三人,其修为已经有隐隐反超方源的架势。

%57
随着不断的修行,这三人或者资质,或者背后强大的支持力,已经开始显露出了明显的优势。方源利用酒虫,艰难营造出的领先局面,已经几乎不存在了。

%58
当然,这也和他近狩猎,利用白豕蛊增强力量,分散了相当大的精力和时间有关。

%59
“按照这个时间估计,过不了多久,就会有人率先晋升到高阶吧。高阶第一人,会得到三十块元石的奖励。不过这个奖励,我不打算去争了。”方源早已经做了决定。

%60
他如果现在暂时抛开白豕蛊,全力冲击高阶,兴许还有获胜的希望。但是这并不是方源想要的,三十块元石虽然好,但是他目前还不缺这点元石。

%61
主要的原因还在于,他现在需要低调,隐藏自己,只有减少了别人对自己的关注度,他能安全地继承花酒行者的力量传承。

%62
这个是他主要的目的。

%63
“学堂中的各项奖励,不过是为了激发学员发奋修行的甜头,亦是家族体制中的一部分。纠缠于这些蝇头小利,实非智者所为。”

%64
方源收回心神,又凝目看向兽皮地图。

%65
兽皮地图分有两面,正面记录着白天,反面记录着夜晚的情况。上面各种颜色的线条纵横交错。

%66
这些线条,或是直线,或是曲线,都有着特殊的代表含义。只有王老汉清楚,可惜他已经死了。不过就算是活着,逼着他讲解,他也未必能说真话。

%67
这些天,方源依靠着自己的见识和经验,又照着竹纸对比,几乎成功地解读了全部。

%68
“红色的叉,应该是代表着危险、禁地。这块画着红叉的地方,隐隐被野猪群包裹着,应该是一头野猪王。以我目前的实力,碰到野猪王就是一个死。哼!”

%69
方源想到了王老汉,不由地冷哼一声。

%70
这块地方的红叉,竹纸上并没有。若方源真信了这竹纸,恐怕某一天就死在野猪王的獠牙之下了。这个王老汉,真是老辣精明,想报杀之仇,却自己不动手,想借助野猪王的力。这样一来,若是杀了方源,他自己还能置身事外。

%71
“不过这三个地方,画着红色圆圈,又代表着什么含义呢?”方源疑惑不解。

%72
这是兽皮地图上后的一个疑点。

%73
三个红色的圆圈,标记在三个很偏僻的地方,同时相距很很远,周围兽群不多,算是在野外比较安全的地方。

%74
“红叉代表禁地,红圈代表什么呢?”方源再次陷入了沉思,“一般而言,红色都是醒目的。王老头在这三个地方标了红圈,就代表这三个地方对他很重要。可惜这三个地方太远了点,否则我亲自探探,就知道了。”

\end{this_body}

