\newsection{月刃飞舞}    %第二十二节:月刃飞舞

\begin{this_body}

%1
天空蓝得空明,纯净得仿佛洗过一般。

%2
太阳金灿灿的。

%3
朵朵白云飘浮着,一群彩雀鹦鹉喳喳地叫着,在蓝天白云下组成箭矢阵型,振翅翱翔。

%4
这种彩雀鹦鹉,只有在春天才会大群出没。它们浑身羽毛七彩斑斓,体型如鹰般大小,头颅鸟喙都是鹦鹉,尾部则像是孔雀的尾巴,拖得长长的。

%5
距离方源在炼化本命蛊的考核中夺得第一,已经有十来天过去了。

%6
春风吹绿了满山的青草,野花争相绽放,蜂群和蝶群交相飞舞。一派生机澎湃的春情妙景。

%7
春天的气息是如此的浓郁,以至于演武场四周的高大竹墙,都遮不住。

%8
这个演武场占地三亩,地面平整,用一层又厚又宽的青灰石铺着。四周则移栽上青矛竹,这些碧绿的竹竿一个紧挨着一个,又直又挺,组成一圈绿色高墙。

%9
墙角下虽然也铺着石砖,但是许多地方已经冒出了一丛丛青草。竹与竹之间,还有野蔷薇,从外面钻进来,甚至攀上墙头。

%10
五十七名十五岁的少年,此时就站在演武场中,围成半圈,注视着中央的学堂家老。

%11
这是一堂课,教导少年们如何使用月光蛊。

%12
“月光蛊是我们古月一族的标志蛊虫,就像熊家的熊力蛊,白家的溪流蛊。在场的大多数人,都是选择月光蛊作为本命蛊,你们好好看着,接下来就老夫就亲自演示,如何催动月光蛊进行攻击。本命蛊不是月光蛊的学员,你们也要集中注意力,这种经典的远程攻击方法可以运用到其他蛊虫身上,运用范围非常广泛。”

%13
说着,学堂家老就伸出右手,五指张开,将手掌放低,使得少年们都能看到他的掌心。

%14
“首先,用意念调动月光蛊,转移到自己的手心里。”随着他的话音,代表月光蛊的月牙印记,就顺着家老的手臂,移动到他的掌心中。

%15
“然后,调动空窍中的真元,灌入到月光蛊中。”一丝白银色泽的真元,从家老的体内涌出,几乎细不可察,投入到掌心处的月光蛊中。

%16
学堂家老是三转境界,只有三转蛊师,凝练出来的真元才是白银色泽。

%17
一转蛊师的真元俗称青铜真元,二转蛊师则号称赤铁真元。到了三转,才是白银真元。

%18
吸收了这丝白银真元,家老手中中心的月牙印记,顿时越来越亮,即便是在白天,仍旧发出了一团明亮的淡蓝色光辉。

%19
“太厉害了!”

%20
“好美啊。”少年们目睹此景,不由地发出一阵惊奇赞叹。

%21
淡蓝色的光芒,清澈如水,在家老的手中幽幽闪烁,乍一眼看过去,仿佛学堂家老手中掬着一捧月光。

%22
学堂家老微微带笑:“都看仔细了,最后一步,就是像我这样,将它发射出去。”

%23
说着,他张开的五指缓缓并拢,然后抬起手臂,慢慢向前伸直,最后挥掌轻轻一切。

%24
整个动作沉稳有力。

%25
刷。

%26
少年们耳边都听到了一声轻响。

%27
随着学堂家老的动作,手掌上凝聚的淡蓝如水的光芒,就这样被甩了出去。

%28
光芒在空中形成了一道小型月刃,幽蓝的月刃只有巴掌大小,就如同夜空中的月牙。它在空中划出一道直线,然后打在十米之外的一具草人傀儡上。

%29
只听哧的一声,草人傀儡厚达三十厘米的颈部就被月刃切断,傀儡的身躯晃了晃,硕大的头部则一下子掉在地上。

%30
切断了这草人,月刃顿时就显得黯淡下去。不过,它仍旧在空中又飞行了大约六米的距离,月牙这才渐渐隐去,最终消散在空气中。

%31
再看草人傀儡上的脖子,只见切面极其平整,就好像是用最锋利的镰刀割断的。

%32
少年们几乎个个都瞪大了双眼,惊讶地看着这一幕。其中有几个,还不由自主地摸了摸自己的脖子,为这月刃的攻击力感到吃惊。

%33
短暂的寂静之后,惊叹声迭传。少年们有的双目放光,盯着草人傀儡,有的盯着家老的手掌。也有的交头接耳,兴奋地谈论。

%34
唯有方源一人隐没在人群中,面色冷漠,不动声色。

%35
他前世修行达到六转,在中洲创建血翼魔教,教众数万人,被誉为魔道巨头,声名赫赫。

%36
学堂家老不过是三转蛊师,这点手段根本就是小儿科,不会引发方源心中的一丝波澜。

%37
“下面,只要是炼成了月光蛊的,都站出来。每人一个草人傀儡,按照我刚刚的方法,催发月刃,练习攻击。”

%38
学堂家老话音刚落,就有三十多位少年站了出来。

%39
这一届,全族有一百多位少年,参加开窍大典。有修行资质的,有五十七学员。这其中选择月光蛊的就占据了三十五人。经过这些天的努力,他们都炼化了月光蛊。

%40
其余的少年,都是丁等资质,不是不想炼化月光蛊,而是资质不行,只能知难而退。

%41
对于古月一族的少年们来讲,月光蛊已经不是单纯的蛊虫,而是象征着家族的荣耀。

%42
很快,三十五位少年站成了一排。每个人十米开外的正对面,都竖立着一具草人傀儡。

%43
方源站在队伍中段,并不惹人注目。

%44
练习开始。

%45
少年们纷纷伸出右手,将月光蛊转移到手掌心中。蓝色的月牙印记,随着青铜色泽的真元注入,一个接一个地散发出水蓝的光辉。

%46
但是当他们竖掌空切的时候,却只有七八片月牙飞了出去。这些月牙有的只现出一瞬,就很快消散。有的刚飞出两三米,忽然砰的一声,崩溃成蓝色流光。有的飞的倒远,只是方向偏得厉害,直接飞上了天空。

%47
少年们纷纷皱起眉头,刚刚看家老演示,似乎挺简单。但是到他们亲手实践,这才发现其中的门道。要发出一记月刃,又要打在草人上,还真不容易。

%48
家老看着,微微带笑,这个情景他每年都会看到,并不意外。

%49
剩下的二十二位少年,则只能站在场外,羡慕地看着。

%50
练习了五分钟,少年们渐渐地都能发出月刃了。一时间,演武场空中,到处都是淡蓝色的月刃在飞射。

%51
一些月刃在半途就消散,一些不幸地撞在一起,一些飞出场外,歪七斜八的。真正能够打在草人傀儡上的,只有偶尔的几片月刃。当然这都是属于瞎猫碰到死耗子的情况。

%52
学堂家老开始进行单独辅导。

%53
他重点照顾方正、漠北、赤城这些资质优异的学员,耐心地纠正他们的姿势错误,教授自己的心得。而对方源这种丙等的学员,只是稍微提点两句。

%54
方源一直凝聚着手中的蓝光,几次挥掌空切,都是凝而不发,做做样子。此时见场面混乱,没人注意到自己,他意念便一动,松开对月光蛊的控制,手掌微微一斜,做了个掌刀劈空的动作。

%55
为了不引人瞩目,他没有瞄准自己对面的草人傀儡,而是斜着瞄准了左前方的一只。

%56
刷的一声,一片月刃快速飞出,穿过中央地带的混乱区域,在空中划出一条直线,准确地命中一具草人傀儡的颈部。

%57
草人颤抖了一下,颈部被这记月刃切割大半,但很快被切断的绿色草茎又开始重新生长,纠缠在一起,伤口愈合了。

%58
这草人傀儡当然不是普通的草人,而是一转的草人蛊,本身拥有草木系的轻微自愈能力。

%59
除非一下子将草人砍断成两截,否则一会儿伤口就会复合如初。

%60
“哇,你看这记月牙!”

%61
“好厉害,是谁发的?”

%62
能击中草人傀儡的月刃,本来就稀少。方源只是随手一击,却造成了至今为止最显著的攻击效果,顿时就引起场外学员们的惊讶呼声。

%63
就连学堂家老的注意力,也被吸引过来,问道:“刚刚这记月牙发挥的不错,是你发的吗?”

%64
他用询问的眼光,看向一位丙等学员。因为那具傀儡,就是和他对应的。

%65
这个男学员双眼连连眨动,面对众人突然而来的注视目光,显得有些手足无措。说实在话,刚刚场面太混乱了,都是月刃在飞,他也不知道这是不是他发的。

%66
“不过,看上去应该好像可能是的吧?”少年想着,就下意识地点点头。

%67
周围少年顿时对他有些刮目相看。

%68
“他是谁呀,叫什么名字?”一些女学员纷纷向周围人打听。

%69
“连他都能发出月刃,我更不能输!”古月漠北双眼闪过一丝坚定。

%70
“原来不是哥哥发的。”古月方正莫名其妙地松了一口气。经过舅父舅母的安慰,他已经从前次的打击中恢复过来了。

%71
“哥哥,你上次得了第一,是因为运气好,选了一个意志薄弱的月光蛊。蛊师修行不可能总依赖运道,我会赢你的。”方正在心中暗暗为自己打气。

%72
“做的不错,要继续努力,抓住刚刚的那个感觉。”学堂家老拍拍这个学员的肩膀,微笑鼓励道。

%73
少年顿时激动地连连点头,眼中涌现出不一样的光彩。

%74
家老借此机会,当众宣布:“都听好了,这就是你们的作业,课下好好练习,三天后检查成果。谁的成绩最好,谁就能获得十块元石奖励。都听到了吗?”

%75
“哦!”少年们大声应喝着,听到有元石奖励,不禁更加兴奋。

%76
然而仅仅三分钟之后,空中飞舞的月刃却渐渐稀疏起来。

%77
“可恶,每一记月刃都要消耗掉一成的真元。”

%78
“月刃的消耗太大了,我只是丙等修为,空窍中能存储三成八的青铜真元。只能连续催发出三片月刃。”

%79
停下手来的少年们,都唉声叹气。

%80
学堂家老目睹这一切,不动声色,却在心中感慨:“这就是修行资质高的好处了。要使用月刃,无非是四个字——熟能生巧。资质高的人,空窍中存储的真元就多,真元的恢复速度也快,就有更多的练习机会。资质若是差点,也可以用元石弥补,加强练习量。资质不高,又没有元石的,纵然有练习的想法,也是心有余而力不足。唉,蛊师修行就是这么残酷,我还是多多关照那些资质高的学生罢。”

\end{this_body}

