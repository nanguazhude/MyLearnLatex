\newsection{炼蛊的艰难}    %第九十八节:炼蛊的艰难

\begin{this_body}



%1
“什么?那个小子,现在居然是二转蛊师了,还向内务堂申请了分家任务!”舅父古月冻土惊怒交加的声音,在会客堂中回荡。

%2
“的确是这样的。我虽然收了你的元石,但是也不好阻拦他,只能告诉他三天后来取任务。帮你拖延这么多天,已经是我所能做到的极限了。”坐在一旁的中年男蛊师说道。

%3
古月冻土心中冷哼一声,他听出了这位内务堂蛊师的意思。

%4
什么“拖延三天,已经是极限”,说这样的话,无非是想在来趁机讹诈他一些钱财罢了。

%5
“不过,当务之急还是竭尽全力,保住家产。这个小崽子实在太不让人省心了!”古月冻土额头滴下冷汗,方源的成长速度让他有一种心惊胆跳之感。

%6
“我是特地过来告诉你一声的,告辞了。”中年男蛊师起身欲走。

%7
“别急着走啊,老弟。有些事情,还要请你行个方便呢。”古月冻土连忙也站起来,取出一个钱袋,塞到男蛊师的手中。

%8
男蛊师将满满一袋的元石,塞到自己的怀中,话锋顿时一转,大笑道:“冻土老哥,你太客气啦。我们哥俩什么关系啊,早在十多年前就认识了。你放心,我会准备一个最难的任务,交给方源。但是方源也许会聘用其他的蛊师,这点你可要注意呀。”

%9
“呵呵呵,这点你放心。老哥我虽然退了,但是关系网还是有的,早就请人盯着他了。那小子若是聘请外人,就是违反族规,正愁着没有他的把柄呢。嘿嘿……”

%10
“那这样我就放心了,告辞。”

%11
“我送送你。”

%12
“不用了,请留步。”

%13
古月冻土望着蛊师离去的背影,脸上堆起的笑容渐渐地垮了下来。

%14
“这个方源,明明是丙等,怎么就这么快晋升二转了?!真是该死啊,角三他们干什么吃的!连个大活人都看不住。”

%15
“唉,现在角三四个,居然都死在兽潮中了,真是不中用啊。方源又晋升二转,这样一来,他必定能接到家产之任务。先前限制他的手段,不能再用了。不过,他现在孤家寡人一个,要独自一人完成家产任务,着实困难。”

%16
“不妥!这小子运气有点邪门,刚请角三那几人压住他,就来了小兽潮。万一他又靠着运气,完成了任务怎么办?我得做最坏的准备!”

%17
人老成精,古月冻土能善始善终,活到现在,本身就代表着成功。

%18
和初出茅庐的方源相比,他的人际关系要强得太多了。

%19
……

%20
“采集蜜酒?”方源接到这个家产任务,眼中寒芒一闪。

%21
这任务很麻烦,是要采集五两的黄金蜂的蜜酒。黄金蜂一个个都有拳头大小,身躯黑金斑斓,蜂刺锐利,富有很强的攻击性。

%22
这还不算,一般的小型蜂群,只有蜂蜜。唯有大中型的蜂巢中,蜂蜜积累多了,才会酝酿成更珍贵的蜜酒。

%23
“这个任务,就算是对五人小组,也是难度很高的。因为人多也没用,采集蜜酒的蛊师,必须得有防御蛊虫,来顶住蜂刺攻击。看来是舅父在背后出手,就是欺负我没有防御性的蛊虫。可惜……他失算了。”方源心中冷笑着。

%24
现在,就体现出了花酒行者的遗藏,又暗藏不说的好处了。

%25
暗事易行,明事难成。

%26
行事越光明正大,就越让人看透,阻止起来就越容易。反观暗中行事,隐藏底牌,就让人摸不清情况,不能对症下药了。

%27
“不过,我要采集蜜酒,单靠玉皮蛊还不够。玉皮蛊只是一转蛊虫,若是能晋升成二转的白玉蛊,相信就能手到擒来了。”

%28
根本不用冒险去尝试,丰富的人生经验就让方源少走了一条弯路。

%29
说起来,方源如今已经是二转初阶的蛊师,但是身上的蛊虫,除去春秋蝉之外,都是一转级数的蛊虫。

%30
这种情况,就像是能拿大关刀砍人的大汉,却捏着一把小巧的匕首。匕首并不能充分发挥出大汉最大的战斗力,只有打造成大关刀,才是最趁手。

%31
方源的手中,有七只蛊虫。

%32
本命蛊春秋蝉、月光蛊、酒虫、白豕蛊、玉皮蛊以及两只小光蛊。

%33
在这其中,月光蛊和两只小光蛊可以合炼成二转的月芒蛊,白豕蛊和玉皮蛊合练成白玉蛊。

%34
月芒蛊代表着进攻能力的上涨,而白玉蛊则是增长防御性能。

%35
若是方源元石充足,自然两者都会炼制。但是不久前,他为了将修为推到二转,消耗了大部分的元石。如今手头中的元石,只够一次所用。

%36
“毫无疑问,选择合练出白玉蛊更为明智。有了白玉蛊,我就能采集蜜酒。而且拥有白玉蛊,对于花酒行者的传承,也有帮助作用。但若是我这次合炼失败,结果就严重了。我的经济已经濒临崩溃,没有白玉蛊,就无法继承遗产。卡在这一步的话,将严重拖延我的成长速度。”

%37
方源隐隐感到一股压力。

%38
他知道自己是到了一个关键时刻,若是合炼成功,自然前景一片光明。若是失败,就好似跌入深渊,想要再爬到现在这样的地步,需要更加努力的经营,以及更长的时间。

%39
……

%40
“蛊师以自身元海为基石,以蛊虫为手段。蛊是蛊师的必备之物,没有蛊,就谈不上蛊师。蛊师除了自身的修行之外,还得炼蛊、养蛊、用蛊。”

%41
房间中,族长古月博细心地为方正讲解着。

%42
“炼、养、用,这三个方面,任何一面都博大精深,究其人的一生,都探索不尽。其中炼蛊方面,你已经知道了如何炼化一只蛊虫,将其收为己用。然而这个,只是单炼。单炼之外,还有更重要的合炼。”

%43
“通过合炼,就能将不同的多种蛊虫,合并成一只更高级数的蛊虫。这是生命的进化和升华。方正啊,你早已经是二转修为,手中的蛊虫却都是一转。该合炼一只二转蛊虫了。”

%44
方正便问:“族长大人,那我该如何合炼呢?”

%45
古月博道:“要合炼就得知道秘方。有些蛊虫之间,是不能合炼的。经过无数岁月的发展和沉淀,先祖们通过不断的实践,数不清的失败后,才总结出来许多的秘方。我们古月一族,对月光蛊的研究最为深入,如今已经掌握了两个五转秘方。”

%46
“什么是五转秘方?”

%47
“按照这个秘方,不断合练,最终就能合练出五转的蛊虫来。方源啊,你手中有一只玉皮蛊,还有一只月光蛊,正符合其中的一个五转秘方。若是你遵循这个秘方,最终将合炼出五转蛊虫——宝月光王蛊!”

%48
“宝月光王蛊?”方正的脸上顿时浮现出向往之情。

%49
“呵呵呵,现在给你说五转蛊虫,还太早了。来,你取出玉皮蛊和月光蛊,我来指导你如何将这两只蛊合炼成二转蛊虫月霓裳!”

%50
说到这里,古月博的脸色转为肃穆:“合炼蛊虫最重要的,就是一心多用,意识融合。现在你手中的月光蛊、玉皮蛊,原先的野生意识都没有了,都被你的意识所取代。你现在所要做的,就是将这两股意识融合在一起。”

%51
“融合在一起?”方正眨眨眼,表情很是困惑。

%52
古月博笑了笑:“不要紧,多练习,你就能把握住这种感觉了。下面开始吧。”

%53
“嗯。”方正点点头,在古月博的指点下,淡红色的赤铁真元如烟雾升起,将玉皮蛊和月光蛊托在半空当中。

%54
方正闭上双眼,开始感受并调动这两只蛊虫的意识。

%55
古月博在外面时刻注意着,就看到玉皮蛊和月光蛊如两颗星辰,相互环绕着转动。

%56
随着意识越来越融合,两只蛊虫的距离也越来越近。

%57
三个小时的不断尝试之后,它们体内的意识终于被方正调动起来,完全融合在了一起。

%58
顿时,玉皮蛊和月光蛊都爆发出刺眼的白色光芒。

%59
两片白光链接起来,很快就形成脸盆大小的一团。

%60
“保持住现在的状态,然后往光团中扔元石。”古月博及时指点着。

%61
方正便取出一块元石,投入到光团当中。

%62
说来也奇怪,元石投了进去之后,就化为了一股纯净的天然真元,彻底融入了光团当中。只余下一些白色石粉,纷纷扬扬地落到床单之上。

%63
“继续扔,直到合炼成二转蛊虫。”古月博道。

%64
方正便又扔出第二块,但就在这时,白光骤然消失,两只蛊虫像是相互被对方狠狠地推了一把,分别向两个相反的方向飞落出去。

%65
合炼失败了。

%66
“糟了,我刚刚忘了维持住意识的融合。”方正旋即就认识到了自己犯下的错误。

%67
“不要紧,前几次失败十分正常。”古月博安慰了方正一句,然后又提醒道,“但是你要注意,失败的次数不能太多。否则月光蛊和玉皮蛊都会消亡的。”

%68
方正召回玉皮蛊和月光蛊,果然就发现,月光蛊的表面出现了微微的裂痕,而玉皮蛊则显得有些萎靡不振。

%69
他顿时心中一沉,真正体会到了炼蛊的艰难。

\end{this_body}

