\newsection{五转激战}    %第一百八十四节:五转激战

\begin{this_body}

%1
“铁血冷,他怎么会突然出现在这里?”方源听出这个声音,不由地心中一惊。

%2
这股巨大的声音,在血湖上空嗡嗡回响,引发汹涌的血浪波涛。

%3
“这是……天地宏音蛊么?”方源咬牙,巨大的声音灌入他的双耳,让他觉得整个脑袋都似乎在发嗡。

%4
天地宏音蛊,高达五转,乃是音类蛊。一旦催发,能形成音浪攻势,范围极为广阔,属于极强的群杀类的蛊。

%5
现在方源听到的声音,不过是铁血冷牛刀小试,轻轻一喝,并未真正催动。

%6
这种程度,就如同方源动用月光蛊,碾磨紫金石头一样。只是一种对蛊虫的细微操控。

%7
当然,这天地宏音蛊也有弊端,催发久了,对蛊师的声带,咽喉都会有过重的负担。这些负担超越极限之后,往往就会令蛊师彻底哑掉,丧失说话的能力。

%8
砰!

%9
一股巨浪陡然爆发,血水四溅飞射。

%10
巨蟒的上半身,出现在方源的视野当中。

%11
血河蟒!

%12
方源瞳孔一缩,只见这蟒浑身长着光滑的血红鳞甲,蛇头巨大如象,眉眼出生长着一片赤金色的骨头刀刺,令其狰狞狂暴之气尽显无漏。

%13
“血河蟒……我记得古月一族的历史中,一代族长立下山寨不久,就有一只血河蟒出现,对古月山寨造成巨大威胁。传说中,一代将其斩杀。难道说……”

%14
方源心中一动,联想到了一些东西。

%15
血河蟒庞大无比,蛇头高耸,紫色的蛇瞳中两道残暴的目光射向方源。

%16
它生性凶残,桀骜不驯,以血为食,哪怕是用春秋蝉气息震慑,也只会让它更狂暴。在五转蛊虫中,是最难被蛊师炼化的蛊之一。

%17
但血海蟒瞟了一眼山壁上的方源之后,就将头高高昂起,转向洞顶。

%18
洞顶处,一个同样渺小的人影,从其中某处洞口,缓缓飘飞而下。

%19
他面带青铜面具,背负双手,目光冷酷。虽然和血河蟒的对比,他渺小若蚁,但一股磅礴的气势从他身上散发出来,巨大的无形压力,充斥四面八方,宛若天神下凡。

%20
正气蛊!

%21
属于心类蛊,只有人秉持正义,才能催动使用。

%22
宵小之徒,心志不坚的人,常常会在正气蛊下,胆战心惊,做贼心虚,心中战意急剧下滑,生出无法和铁血冷对战的感觉,从而不战自溃。

%23
吱吱吱……

%24
在正气的压迫下,刀翅血蝠蛊惊惶失措,发出尖锐的叫声。一只只宛若身上被压上了重担,两对翅膀连连拍动,在半空中挣扎飞腾,忽上忽下。

%25
它们自顾不暇,却没有再接近方源。

%26
甚至就连凶猛的血河蟒,都微微垂首,感到铁血冷的无上威势。

%27
“好个正气蛊!”方源攀在山壁上,嘴角冷笑。

%28
这正气的压迫,对他毫无效果。只有意志越不坚的人,才会受到更严重的削弱。他是魔道巨擘,心意如钢似铁,怎么可能会被这虚无缥缈的气势所吓倒呢?

%29
“嗯?”方源的若无其事,令铁血冷忍不住轻咦一声,感到惊讶。

%30
正气蛊已经伴随他有许多年头,他深深的明白此蛊的效用。

%31
正气蛊效果可大可小,攻势直指人心,有时候能收到奇效,有时候却无功而返。

%32
但无功而返的现象,却少之又少。哪怕是正道人物,也多感到心头压抑。毕竟阵营说明不了善恶,人无完人。

%33
但铁血冷深知方源的罪恶,早在他第一次见面时,他就在方源身上中了蛊,用来追踪。从那之后,铁血冷就能若隐若现地感应到,方源周边的环境变动。

%34
犯下罪孽的人,在正气蛊下,绝大多数都会心志动摇。不过,铁血冷也碰到过不少的魔道蛊师,在正气之下昂首酣战,不受影响。

%35
“只有一种人即便犯罪,也不受正气蛊的影响。这些都是具有真正的魔性的人,他们从内心最深处,就偏执疯狂,根本不把自己犯下的罪孽当成罪孽,而认为是理所应当的事情。想不到这个方源,魔性是如此深重。哼,先把这边的大魔处理掉,若他还能侥幸活着,不妨再来收拾!”

%36
铁血冷嫉恶如仇,冷哼一声,眼眸转向血河蟒。

%37
他一出现,就被血河蟒锁定,只要分心击杀方源,势必就会出现破绽。

%38
铁血冷身负恐怖伤势,而强敌隐于幕后,说不定暗暗偷窥他,此时自然不敢分散精力。

%39
他凝视这血河蟒片刻,目光便越过血河蟒,投向其身后的血湖。

%40
在不久之前,他收到一份神秘来信,信中证据凿凿,指明古月山寨当中,藏有血祸。

%41
这信是被一只白鹤嘴里衔着,从天而降得来,来源也很可疑。

%42
但铁血冷宁可信其有不可信其无,皆因血祸非同小可,绝不能麻痹轻忽。稍不留神,就会骤然扩大,波及四方,为祸众生。

%43
最佳的应对手段,就是趁着血祸没有成势之时,将这个动乱的苗头掐灭。

%44
正巧的是,他手头上还有一个委托,正是贾富以重金,要查出贾金生的死因。

%45
铁血冷便带着女儿,提前赶到了青茅山。

%46
他令女儿来调查贾金生的案件,一方面是为了培养和锻炼铁若男,另一方面也是掩人耳目,拖延时间,自己居于幕后,调查详细情况。

%47
他和方源第一次见面,就给方源中下感动身受蛊,虽是信手而为,但也源自长期以来,积累出来的经验和感觉。

%48
感同身受蛊,无形无色,仿佛印记,他布置了数十只。真正起作用的,却还是他信手布置的第一只。

%49
“古月一代,我知道你没有死。你潜藏了近千年,布下这个局,可惜到如今,要功亏一篑了。”铁血冷一开口,整个空间都嗡嗡作响。

%50
但血湖没有任何异变,反倒是血河蟒张开血盆大口,抬头怒吼。

%51
它生性残暴,受不了任何的压迫。正气蛊更激发了它的凶性!

%52
砰。

%53
它猛地探身,巨大的蛇躯激起冲天的血浪,带洞穿天际的杀势,向半空中的铁血冷扑去。

%54
铁血冷早防备着血河蟒,身躯猛动,闪过血河蟒的吞咬。

%55
血河蟒身躯太长,冲势太猛,一头撞到洞顶。

%56
轰隆的巨响声中,洞顶崩塌一大块,大量的碎石砸落下来,顿时激起血湖的滔天大浪。

%57
“哼,古月一代,你以为留着一只血河蟒把守,就万事无忧了吗?你还是出现罢。”铁血冷哂笑一声,他在空中腾挪,血河蟒的狂暴攻势,在他面前仿佛是清风细雨。

%58
血河蟒疯狂嘶吼,越加暴躁。

%59
它有无与伦比的力量,恣意施展,引发巨大的破坏。

%60
整个空间中,越加动荡,简直是地动山摇,血湖亦掀起惊涛巨浪。

%61
“可恶!”方源被殃及池鱼,赤土松软,原先的立足点早就被毁。

%62
他只好利用锯齿金蜈,以及雷翼蛊在山壁上游走。碎石一块块的砸下来,如雨般密集。

%63
果然如他先前所料,雷翼蛊状态很不好。方源背后的雷翅,十分虚弱颓靡,无法提供强大的上升力量。

%64
但古怪的是,雷翼蛊向方源汲取的真元却变得稀少。在方源腾挪期间,它竟然时不时地开始吞吸周围空气中的元气。

%65
起初这种现象,发生的时间很短,次数也很少。

%66
方源努力闪避,并未注意到。但渐渐的,次数越来越多,每一次持续的时间也越来越长。

%67
同时,雷翅越来越虚弱。

%68
“明白了,是血狂蛊!”方源脑海中闪过一道灵光。

%69
只有被血狂蛊污染,才会有这样的现象发生。

%70
扑通!

%71
方源尽了全力,但终究还是掉落到血湖当中。

%72
血河蟒在四处乱撞,用粗大的蟒尾横扫狂抽。碎石如雨,绵绵不绝地砸落下来。

%73
方源催动天蓬蛊,全身罩住一层白光虚甲。他水性很不错,前世曾在东海生活很长一段时间,得到过充分的锤炼。

%74
大如房屋的巨石,他都极力躲闪。时而游弋,时而沉入血湖里,利用血水来缓冲撞击力量。

%75
但那些琐碎的石块,小的有拳头大,大的如石磨,他就无暇躲闪了。

%76
白光虚甲坚如磐石,但却无法减缓冲击力量,砸在他身上,一阵阵生疼。同时,他空窍中真元也在因为每一次的承受,而缓慢的减少。

%77
万幸的是,正气蛊的力量笼罩全场,导致先前的那群刀翅血蝠蛊,仍旧在四处乱飞,没有来找方源的麻烦。

%78
但方源面色却相当凝重。

%79
雷翼蛊已经半废了。被血狂蛊污染,没有特殊蛊虫清洗,过不了多久,就会化为一滩血水,成为新的污染源。

%80
它究竟是如何被血狂蛊沾染的?

%81
方源思前想后,只有一个解释。那就是这血水中,依附着血狂蛊。

%82
雷翼蛊越来越虚弱,且开始出现不受操纵的苗头。

%83
幸好他及时撑起了白光虚甲,否则血狂蛊沾上他的身体,气息渗透到空窍中。他大部分的蛊虫都要被污染了。

%84
但若是他真元耗尽,或者这层光甲被打碎,那后果就会变得相当的严重!

%85
“必须尽快地离开这里!”方源咬紧牙关,在血浪中沉浮,四处张望寻找出路。

%86
山壁出大量的赤土,被震落下来。

%87
洞顶也面目全非,铁血冷悬浮在半空中,在和血河蟒纠缠。

%88
五转级数的战斗,根本就不是方源能够插手的。

%89
蛊师越到后阶,不同修为间,战力差距将越来越明显。

%90
也许血河蟒的一次甩尾,就能令光甲破碎,天蓬蛊被反噬重伤,方源全身骨折。也许铁血冷的随手一击,就能让方源陷入绝境。

%91
现在的局面,就好像是两头巨象争斗,方源是附近的一只花猫。虽也有小小利爪,但难登大堂。

%92
“等等,一个洞口?”

\end{this_body}

