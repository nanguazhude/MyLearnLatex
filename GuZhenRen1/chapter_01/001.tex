\newsection{纵身亡魔心仍不悔}    %第一节:纵身亡魔心仍不悔

\begin{this_body}

%1
“方源,乖乖地交出春秋蝉,我给你个痛快!”

%2
“方老魔,你不要妄图反抗了,今日我们正道各大派联合起来,就是要踏破你的魔窟。这里早已经布下天罗地网,这次你必定身首异处!”

%3
“方源你个该死的魔头,你为了练成春秋蝉,杀了千万人的性命。你已经犯下了滔天的罪孽,罪无可恕,罄竹难书!”

%4
“魔头,三百年前你侮辱了我,夺走了我的清白之身,杀光我全家,诛了我的九族。从那刻起,我恨不得吃你肉,喝你的血!今天,我要让你生不如死!!”

%5
……

%6
方源一身残破的碧绿大袍,披头散发,浑身浴血,环顾四周。

%7
山风吹得血袍飘荡,如战旗般嚯嚯作响。

%8
鲜红的血液,从身上数百道伤口向外涌着。只是站着一会儿,方源脚下已经积了一大滩的血水。

%9
群敌环伺,早已经没有了生路。

%10
大局已定,今日必死无疑。

%11
方源对局势洞若观火,不过即便死亡将临,他仍旧是面不改色,神情平淡。

%12
他目光幽幽,如古井深潭一般,一如既往的深不见底。

%13
围攻他的正道群雄,不是堂堂一派之长者尊贵,就是名动四方之少年英豪。此时牢牢包围着方源,有的在咆哮,有的在冷笑,有的双眼眯起闪着警惕的光,有的捂着伤口恐惧地望着。

%14
他们没有动手,都忌惮着方源的临死反扑。

%15
就这样紧张地对峙了三个时辰,夕阳西下,落日的余晖将山边的晚霞点燃,一时间绚烂如火。

%16
一直静如雕塑的方源,慢慢转身。

%17
群雄顿时一阵骚动,齐齐后退一大步。

%18
此刻,方源脚下的灰白山石,早已经被鲜血染成了暗红。因为失血过多而显得苍白的脸,被晚霞映照着,忽然增添了一份嫣然之光彩。

%19
看着这青山落日,方源轻声一笑:“青山落日,秋月春风。当真是朝如青丝暮成雪,是非成败转头空。”

%20
说这话的时候,眼前忽的就浮现出前世地球上的种种。

%21
他本是地球上的华夏学子,机缘巧合穿越到这方世界。辗转颠簸三百年,纵横世间两百余年,五百多年光阴悠悠,却是晃眼即逝。

%22
深埋在心底的许多记忆,在此刻鲜活起来,栩栩如生地在眼前回现着。

%23
“终究是失败了呀。”方源心中叹着,有些感慨,却并不后悔。

%24
这种结果,他也早有预见。当初选择时,就有了心理准备。

%25
所谓魔道,就是不修善果,杀人放火。天地不容,举世皆敌,还要纵情纵横。

%26
“若是刚炼成的春秋蝉有效,来生还是要做邪魔!”这般想着,方源情不自禁放声大笑。

%27
“老魔,你笑什么?”

%28
“大家小心,魔头死到临头要反扑了!”

%29
“快快交出春秋蝉!!”

%30
群雄逼迫而来,恰在这时,轰的一声,方源悍然自爆。

%31
……

%32
春雨绵绵,悄无声息地滋润着青茅山。

%33
夜已经深了,丝丝凉风吹拂着细雨。

%34
青茅山却不黑暗,从山腰至山脚,闪着许多莹莹的微光,好像是披着一条灿烂的光带。

%35
这些光来源于一座座高脚吊楼,虽称不上万家灯火,却也有数千的规模。

%36
正是坐落在青茅山的古月山寨,给广袤幽静的山峦增添了一份浓郁的人烟气息。

%37
古月山寨的最中央,是一座大气辉煌的楼阁。此时正举办着祭祀大典,因此更是灯火通明,光辉绚烂。

%38
“列祖列宗保佑,希望此次开窍大典中能多多涌现出资质优秀的少年,为家族增添新血和希望!”古月族长中年模样,两鬓微霜,一身素白庄重的祭祀服装,跪在棕黄色的地板上,直着上身,双手合十,紧闭双目诚心祈祷。

%39
他面对着高高的黑漆台案,在台案有三层,供奉着先祖的牌位。牌位两侧摆着赤铜香炉,香烟袅袅。

%40
在他的身后,也同样跪着十余人。他们穿着宽大的白色祭服,都是家族中的家老、话事人,执掌着各方面的权柄。

%41
祈祷了一番后,古月族长率先弯腰,双手平摊,掌心紧紧贴着地板,磕头。额头碰在棕色的地板上,发出轻轻的砰砰声。

%42
身后的家老们各个表情肃穆,也随着默默效仿。

%43
一时间,宗族祠堂中尽是额头碰撞地板的轻响。

%44
大典完毕,众人慢慢地从地板上站起身来,静静地走出庄严的祠堂。

%45
在走廊中,众家老默默地舒了一口气,气氛为之一松。

%46
议论声渐渐地起来。

%47
“时间过的真是太快了,一眨眼,一年就都过去了。”

%48
“上一届的开窍大典就像是昨天发生的一样,依旧历历在目呢。”

%49
“明日就是一年一度的开窍大典了,不知道今年会涌现出什么样的家族新血呢?”

%50
“唉,希望有甲等资质的少年出现。我们古月一族已经有三年没有这样的天才出现了。”

%51
“不错,白家寨、熊家寨这些年都有天才涌现。尤其是白家的白凝冰,天资真是恐怖。”

%52
不知是谁,提及到白凝冰这个名字,众家老的脸上不由地浮现出一层忧色。

%53
此子的资质极端出色,短短两年功夫,就已经修行到三转蛊师。在年轻一辈中,可谓独领风骚。甚至就连老一辈们都感觉到了这位后起之秀的压力。

%54
假以时日,他必然是白家寨的顶梁柱。至少也是独当一面的强者。没有人怀疑过这一点。

%55
“不过今年参加开窍大典的少年里,也不是没有希望。”

%56
“不错,方之一脉出现了一个天才少年。三月能言,四月能走。五岁时就能作诗诵词,聪慧异常,才华横溢。可惜就是父母死的早,现在被其舅父舅母抚养着。”

%57
“嗯,这是有早智的,而且有大志向。近些年他创作的《将敬酒》、《咏梅》,还有《江城子》我也听说过,真是天才!”

%58
古月族长最后一个走出宗祖祠堂,慢慢地关上门,便听到走廊中家老们的议论声音。

%59
顿时就知道,家老们此时议论的是一位叫做古月方源的少年。

%60
作为一族之长,对于那些优秀而突出的子弟自然会关注。而古月方源就是小辈当中,最为出色耀眼的一位。

%61
而经验表明,往往从小就过目不忘,或者力气大如成人等等天赋异禀的人,都有优秀的修行资质。

%62
“若是此子测出甲等资质,好生培养,也未必不能和白凝冰抗衡。就算是乙等资质,日后定也能独当一面,成为古月一族的一面旗帜。不过他这样的早慧,乙等资质的可能性不大,极有可能就是甲等。”这一念生出,古月族长的嘴角不由地微微翘起,浮现出一抹微笑。

%63
旋即,咳嗽一声,对诸位家老们道:“诸位,时候不早了,为了明日的开窍大殿,今晚请务必好好休息,保养精神。”

%64
家老们听了这话,都微微一怔。看向彼此的目光中都隐藏了一丝警惕之色。

%65
族长这话说的含蓄,但大家都深晓其意。

%66
每年为了争夺这些天才后辈,家老们彼此之间都是争的面红耳赤,头破血流。

%67
是该好好的养精蓄锐,待到明天,争上那一番。

%68
尤其是那个古月方源,甲等资质的可能性非常的大。而且他双亲已经亡故,是方之一脉仅剩下的双孤之一。若是能收入自己这一脉中,好好培养,可保自己这一支百年的昌盛不衰!

%69
“不过,丑话先说在前头。争要堂堂正正的争,不可以动用阴谋手段,损害家族的团结。诸位家老们请牢记在心!”族长语气严肃地关照道。

%70
“不敢。不敢。”

%71
“一定牢记在心。”

%72
“这就告辞了,族长大人请留步。”

%73
家老们满怀心思,一一散去。

%74
不久,长长的走廊上就冷清下来。春雨斜风透过窗户吹拂过来,族长轻轻举步,走到窗前。

%75
顿时,满口都是清新湿润的山间空气,沁人心脾。

%76
这是阁楼第三层,族长放眼望去,大半个古月山寨都一览无余。

%77
此刻深夜,寨中大多数人家却还有着灯火,和平时大不相同。

%78
明天就是开窍大典,关乎着每个人的切身利益。一种兴奋、紧张的氛围,笼罩着族人的心,自然有很多人睡不安稳。

%79
“这就是家族未来的希望啊。”眼中倒映着点点灯光,族长长叹一声。

%80
而此时,同样有一对清亮的眸子,静静地看着这些深夜中闪闪的灯光,满怀复杂情怀。

%81
“古月山寨,这是五百年前?!春秋蝉果真起作用了……”方源眼神幽幽,站在窗户旁,任凭风雨打在身上。

%82
春秋蝉的作用,就是逆转时光。在十大奇蛊排名中,能名列第七,自然非同小可。

%83
简而言之,就是重生。

%84
“利用春秋蝉重生了,回到了五百年前!”方源伸出手,目光定定地看着自己年轻稚嫩的有些苍白的手掌,然后慢慢握紧,用力感受着这份真实。

%85
耳畔是细雨打在窗扉上发出的微微声响,他缓缓地闭上眼,半晌后才睁开,喟然一叹:“五百年的经历,真像是个梦啊。”

%86
但他却清楚的知道,这绝不是梦。

%87
(ps:好了,新书终于和大家见面,就像在《御妖至尊》结尾和大家打的招呼,主角很邪恶,本书会饱受争议。本书的序,希望读者朋友们能够看看,知道这是一本什么样的书。该说的话,我都在序里面说了。如果朋友们喜欢,那就请我们一起携手共进这场奇妙的冒险旅程。新书伊始,朋友们的收该藏、推荐票以及评论,是最好的鼓励。)

\end{this_body}

