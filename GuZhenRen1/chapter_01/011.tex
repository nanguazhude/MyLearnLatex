\newsection{不过是色诱罢了}    %第十一节:不过是色诱罢了

\begin{this_body}

方源眉头微微地皱起来,凭借直觉和五百年的人生体验,他闻到了一种阴谋的味道。

他眼中冷芒一闪即逝,眉头舒展开来:“我正有些饿,你来的好,给我端进来吧。”

门外沈翠提着食盒,听到这话,嘴角泄露出一丝不屑的冷笑。

但当她推开房门时,她的脸上就只剩下柔顺之色。

“方源少爷,这酒菜可香了,奴婢隔着食盒都闻到了呢。”她的声音甜腻腻的,透着一股春情媚意。

将食盒放在小桌上,沈翠又一一取出餐盘,摆放好了。

餐盘中的确是色香俱全的美味。

她接着又取出两个酒杯,斟上酒。

“来,少爷,坐嘛。奴婢今天斗胆,想陪少爷您喝一杯呢。”她笑魇如花,走到方源的身边,大胆地拉住方源的手,将他拖到桌边的凳上坐好。

然后她直接坐到方源的大腿上,娇柔的身躯都倚靠在方源的胸膛上,小鸟依人地在方源耳边道:“方源少爷,奴婢一直都喜欢着你。不管是什么资质,奴婢都想陪着你,依赖你,慰藉你。今晚,奴婢就想把身子给了你。”

她今天可谓盛装打扮。

抹了胭脂,唇如樱粉,因为是贴着耳根说话,一股娇柔青春的气息,就撩拨着方源的耳垂上。

因为她坐在怀里,方源可以明显地感觉到沈翠的丰满的身躯。

她那充满弹性的大腿,她柔细的小蛮腰,她胸前的柔软。

“少爷,让奴婢来喂您酒吧。”沈翠端起酒杯,却一仰头,将酒抿入口中。然后双眼似含着水般,定定地看着方源,樱桃小嘴虚张着,向方源的嘴唇慢慢地靠了过来。

方源面色冷漠,好像怀中坐着的不是一个少女,而是一块雕塑。

沈翠看着方源这个表情,初始时心中还有点惴惴不安,但当她的嘴唇就只差一指头的距离,就要贴上方源的嘴唇时,她笃定了,心中不屑地一笑:“还装。”

恰在此时,方源冷笑一声,语气中带着不屑:“原来不过是色诱罢了。”

沈翠脸上神情一僵,咽下口中酒水,假意嗔道:“方源少爷,您说什么呢。”

方源双目幽幽散发着冷光,盯着沈翠的眼睛,同时右手搭在她雪白的脖颈上,缓缓用力。

沈翠瞳孔猛缩,声音带着惊惶:“少爷,您弄疼我了。”

方源不答话,只是手上的力量越来越大。

“方源少爷,奴婢有些害怕!”沈翠已经有些喘不过气来,神色慌乱,一双娇嫩的手下意识地搭在方源的手上,想要将他的手掰开。但是方源的手如铁钳一般,哪里掰得开。

“看来舅父舅母是让你色诱陷害我?这么说来,楼下应该也安排好了人马了吧。”方源轻蔑地冷笑一声,“不过你算什么东西,也来色诱我?就凭你胸前的这两堆垃圾般的烂肉。”

说着,左手就攀上沈翠的胸口,恶狠狠地捏住胸前的柔软,一下子就让它发生了剧烈形变。

强烈的剧痛从胸口传来,沈翠双眼圆瞪,疼得满眼含泪,她想要叫喊,但是喉咙被方源掐住,最后只能呜咽几声,她开始强烈的反抗,再不反抗她就真的要窒息了!

但就在这时,方源却缓缓放松了手劲。

沈翠立即张开大口,贪婪地呼吸空气,她呼吸得太急切了,以至于引发了一阵剧烈的干咳。

方源轻轻地笑起来,伸出手掌温柔地抚摸沈翠的脸颊,悠然地道:“沈翠,你觉得我能不能杀你?”

若是方源恶声恶气地大吼咆哮,沈翠说不定还会剧烈反抗。

但是当方源如此低笑浅语,柔声地问她能不能杀她的时候,沈翠感到一种由衷的恐惧。

她害怕了!

她惊恐地看着方源,看着这个少年笑眯眯地望着她。

在这一刻,沈翠发誓自己永远不会忘了方源的双眼。这双眼睛,不夹杂半点情绪,漆黑深邃,像是隐藏着恐怖巨兽的古潭。

在这双眼睛的注视下,沈翠觉得自己如同赤身裸体,置身在冰天雪地当中!

眼前这个人,绝对敢杀自己,能杀自己……

天呐!我为什么要来招惹这样的一个魔鬼?!

沈翠的心中充满了懊悔,在这一刻她恨不得转身就逃。

但她此刻投身在方源的怀中,却不敢逃,甚至不敢做出任何的一个动作。

她浑身的肌肉都在紧张,娇躯颤抖着,脸色苍白如纸,说不出一句话来。

“念在你作为贴身丫鬟,服侍我这么多年的份上,我这次就不杀你了。你不是想脱离奴籍么,去找我弟弟吧,他可是又傻又天真。”方源收起笑容,拍拍沈翠的脸颊,语气平淡如水。

叹了一口气,他最后说道――

“你走吧。”

沈翠便呆若木鸡乖乖地走了出去。她失魂落魄,也不知道是怎么逃离了方源这个魔鬼身边的。

那隐藏在暗处的人马,见到沈翠这般出来,都疑惑得面面相觑。

“居然安排了个美色陷阱,倒是比前世有新意。呵呵,舅父舅母,你们这番恩情我深深地记下了!”

沈翠走后不久,方源就站起来,出了门。

不管如何,这住处是不能呆了。

君子不立危墙之下,更何况魔头?力量不足,只有傻子才置身险境。

“掌柜的,有房间么?”来到山寨中唯一的一间客栈,方源问了价格。

“有的,有的,有上房,在二层、三层,不仅便宜,而且都收拾得很干净。一层是饭堂,客官可以在这里用饭,也可以叫店家伙计专门送到房间里去。”掌柜的殷勤地招待方源。

这客栈是山寨中唯一的一家,生意并不好,显得有些冷清。只有每年商队来到青茅山贸易的时候,客栈才会充满人气。

方源真的有些饿了,便抛给掌柜的两块完整的元石:“给我一间上房我先住下,再给我准备两坛酒,三四样小菜,多退少补。”

“好咧。”掌柜的接过两块元石,又问,“客官是要在房间里吃,还是在大厅图个热闹?”

方源看了一眼天色,雨已经停了,而且接近傍晚。干脆在大厅吃完,然后直接出寨,探索花酒行者遗藏。便对掌柜的道:“就在大厅吃吧。”

这客栈一层饭堂,摆着十几张方桌,桌子一圈四张长长的板凳。桌子之间,还有粗大的柱子,支撑着客栈。地面上铺着一块块的大理石,但湿漉漉的,难掩山间的湿气。

饭堂里有三桌人。

靠着窗户的一桌,只有一个老汉喝着小酒,看着窗外山间晚霞,在慢慢独酌。

正中央的一桌,是五六个猎户,围成一圈坐着,大声交谈着打猎的经历,脚边的地上还摆放着一堆山鸡野兔什么的猎物。

还有角落里的一桌,是两个年轻人,似乎在密谈什么。他们的身形隐没在阴影中,看不分明,分不清男女。

方源就选了个靠近门口的位置坐着,不一会儿就有酒菜端上了桌。

“以我丙等的资质,要炼化月光蛊,必须要借助元石。若是运气好,这月光蛊意志不顽强,只需要五块。若是顽固不化,就麻烦了,至少需要八块。”

蛊虫也是生灵,自然有求生的意志。

有的意志强大,会一直抵抗蛊师的炼化。有的意志弱小,炼到最后它就绝望投降了,一旦没有了抵抗的意志,那炼化就极其轻松了。

“我现在身上只有六块元石,两块已经给了店家,还只剩下四块。有些不够啊。”

在这个世界上,元石是硬通货,购买力很强。

一个凡人三口之家,一个月的生活费,最多也就一块元石。

但是对于蛊师来讲,元石的消耗就大了。就像方源,单单炼蛊,就需要平均七块元石的样子。

这还只是月光蛊,若是真的找到了酒虫,要炼化它,以方源的资质,至少得再添上十几块元石的支出!

“也就是说,现在的情况是――我即便找到酒虫,也未必有元石来帮助我炼化它。不过还是要继续探索,因为花酒行者的遗藏中很有可能,拥有大量的元石。”

这点不难推断。

花酒行者是五转蛊师,著名的魔道强者,怎么可能身上没有元石这种蛊师的必备修行用品呢?

\end{this_body}

