\newsection{前人开道后人跑路}    %第一百八十八节:前人开道后人跑路

\begin{this_body}

%1
血蝠群中,只有一只雄蝠,其余皆是雌蝠,受到雄蝠的指挥命令。

%2
雄蝠和雌蝠之间,外形相似,差别并不大。但方源前世对这种刀翅血蝠蛊,熟悉到了骨子里,只要给他足够的时间,他就能分辨出来。

%3
没有了雄蝠,这群血蝠陷入混乱当中。

%4
方源趁机,挥动锯齿金蜈大杀特杀。

%5
片刻功夫,二十多只刀翅血蝠皆殒灭,可谓战果累累。

%6
但剧烈的激战,也让方源空窍真元消耗巨大。

%7
“该撤了!”方源果断撤退,拔腿就走。

%8
当身后的血蝠群反应过来时,方源已经逃出百步之远。约有二十多只刀翅血蝠,仍旧向方源追杀而来,剩余的则各自飞散。

%9
“呼呼呼……”在狭窄的山洞中,方源喘着粗气,一边狂奔,一边催动天元宝莲。

%10
经过刚刚一战的消耗,空窍中只剩下雪银真元,同时低落的真元海面,正缓慢而决绝地回升。

%11
时间拖延得越久,方源的战力就恢复得越快。

%12
突然,方源背后雷翅猛地一扇,冲力带动方源的身躯,差点让他一头撞到洞壁上。

%13
雷翼蛊被血狂蛊污染,已经达到某种的极限,开始出现不受操纵的迹象。

%14
“过不了多久,雷翼蛊就会化为一滩血水,成为新的污染源头了。”想到这里,方源再不迟疑,连催三次,这才让不听话的雷翼蛊,脱离他的后背。

%15
“去吧。”方源毅然舍掉雷翼蛊,抛向身后。

%16
身后的刀翅血蝠群,旋即将雷翼蛊团团包围,然后一拥而上,将雷翼蛊斩碎。

%17
这么一阻,也算为方源争取了些微喘息的时间。

%18
待到方源被刀翅血蝠追上,他空窍中的真元,已经回复到将近一半。

%19
雪银真元比淡银真元。要耐用许多倍。方源此时的战力,已经超出初阶良多。

%20
他朗笑一声,挥动锯齿金蜈再酣战。

%21
杀了八九只刀翅血蝠之后,剩下的皆轰然飞逃。

%22
“可惜我没有铁手擒拿蛊这类的蛊虫,刀翅血蝠蛊飞得太快了。飞翅如刀刃锋锐。实难徒手捕捉。若是能捕捉两三只,也是好的。”

%23
方源收起锯齿金蜈,转身就向山洞深处走去。

%24
他夺了天元宝莲,却被血水卷入下来。这肯定是古月一代的手段。

%25
古月一代利用血鬼尸蛊,变成了飞僵。空窍已死,再不能自我回复真元。除非动用元石补充,否则空窍中的真元用多少,就多少。

%26
但他若是有一株天元宝莲的话。很大程度上,将弥补他的这个弱点。

%27
因此,在不久之前,他特意调动两群刀翅血蝠,飞向方源,企图将他捉拿。好在方源见机不妙,及时转移,同时铁血冷也间接地帮了他一个忙。

%28
“不晓得铁血冷和古月一代的战斗,进行得如何了。”方源目光沉凝。

%29
不管是哪方胜利了。都要找他麻烦。

%30
铁血冷要抓他归案,而古月一代则觊觎他摘取的天元宝莲。

%31
虽然刀翅血蝠群溃散了,但方源的危机并没有解除。

%32
“必须尽快逃离青茅山,越快越好!”方源咬牙,他不可能再回去。只有顺着这个山洞向前走,看看是否真有出路。

%33
这山洞明显是有人为迹象,但过去很久了,有些地段都发生了小塌方。

%34
方源埋头前进。遇到这些塌方,只能动用锯齿金蜈。

%35
锯齿金蜈本来就是地底生物。善于钻洞,此时发挥极大用途。

%36
方源本身也有两猪之力,挖开泥土,继续前进。

%37
这大大拖慢了他的速度,三四个小时之后,他走到了山洞尽头。

%38
一块刚硬的石壁,完全挡住了他的路。

%39
就算是用锯齿金蜈,也钻不破这山壁。

%40
“难道这山洞早已经被古月一代封死?”方源心中一沉。

%41
……

%42
砰!

%43
双方对掌,熊家蛊师如破麻袋一般,高高抛飞,然后重重地落在地上。

%44
噗。

%45
熊家蛊师吐出一口鲜血后,当即陷入昏迷。

%46
天空中,太阳当空照耀,洒下热烈的光辉。阳光透过树野,斑驳的光点映照在白重水的肥脸上。

%47
这位白家的年轻精锐,得意地一笑:“哼,熊家蛊师又怎样?敢和老娘我拼力气!”

%48
她迈动肥胖的“娇躯”,正要上前补上一刀,顺便收刮铭牌,忽然一道月刃袭来。

%49
砰!

%50
白重水身躯一转,张口吐出一记水弹。

%51
水弹和月刃在空中相撞,双双爆炸。

%52
以方正为首的三位蛊师,从树梢上跳跃下来。

%53
“古月家的,这人可是老娘我的战利品。”白重水眯起了双眼,眼中闪烁着危险的光。

%54
方正充斥血丝的双眼,紧紧地盯住白重水:“这样又如何?现在你是我们的猎物!”

%55
这时树林另一侧,又有脚步声响起。

%56
几个人影走出来,却是熊家的天才少年熊林,他的光头反射着阳光,看着耀眼炫目。

%57
“呵呵呵,这局面就有趣多了!”白重水哈哈大笑,但下一刻,她的笑声戛然而止。

%58
两边人马一起展开攻势,将白重水打得措手不及,团团包围。

%59
白重水吐出一口鲜血,脸色难看至极:“怎么?古月家的,你们居然和熊家这样卑鄙的家伙们联手?”

%60
方正面无表情,杀意腾腾,一言不发,向白重水逼近。

%61
熊林则嘿嘿一笑:“此战关系到今后百年的三族大局,实话告诉你,古月家已经和我们熊家建立了联盟。白重水,今日你的死期到了。”

%62
“呸!”白重水吐出一口血水,她昂首不屑地扫视周围,“老娘死了又何妨?嘿,联手……是害怕我族的白凝冰罢。没有用的,这才是初战,接下来还有个人演武。你们两家有谁是白凝冰的对手?我们白家注定是第一!”

%63
“呵呵呵。所以我们决定,就在这场初战时,就一起联手,将白凝冰干掉啊!”熊林哈哈大笑。

%64
……

%65
嗡嗡嗡……

%66
锯齿金蜈的银边锯齿。在石壁上剧烈磨绞,山洞中回荡着阵阵噪音。

%67
这石壁坚硬厚重,出现得十分古怪。方源满头大汗,轮番动用锯齿金蜈以及血月蛊轰斩,如此半天功夫。也只碾磨了半米之距。

%68
“难道真的是绝路不成……咦?”方源心中一动。忽然感觉到石壁中存有一股生机。

%69
他连忙动用地听肉耳草,贴着石壁倾听,石壁另一侧真的藏有神秘生物,但是气息很虚弱。

%70
半个小时之后。方源挖开一个洞口,神秘生物显露原形。

%71
它全身都是漆黑,散发着金属光泽,如钢似铁。不管是胸板还是背甲,都是粗硬的线条。彰显霸意。它有三对触脚,每一对触脚尖端,都是螺旋形态。让方源不由地联想到,地球上的电转。

%72
“原来是千里地狼蛛!”方源恍然大悟,立即联想到花酒行者。

%73
方源心中的迷雾顿时消散了大半。

%74
当初花酒行者,为了栽培天元宝莲,来到古月山寨。但在最后关头,被古月一代阻止,施展手段。将其拽入到血河墓地。

%75
花酒行者同样是五转强者,必定和古月一代展开了一场激战。

%76
古月一代经营了数百年,占据地利,花酒行者不敌,只能逃遁。

%77
他利用千里地狼蛛开辟出这条山洞。出了战场之后,却已经重伤难治。在生命的最后关头,他匆忙留下传承,进行最后的报复。企图让后人摘取天元宝莲。破坏掉古月一族的根基。

%78
这就解释了花酒行者,之所以浑身浴血。满身都是伤痕的原因。

%79
但千里地狼蛛,为何在这里沉眠自我封印,却仍旧是个迷。还有古月一代,到底有什么图谋,也是个疑问。

%80
“这些疑问,现在都只是细枝末节,还是先离开这里再说!”方源伸出手来,真元一吐,就将千里地狼蛛炼化。

%81
这坚硬的石壁,其实是千里地狼蛛陷入休眠沉睡,自我包裹的石茧。它虚弱至极,如同方源从地藏花中取出酒虫。

%82
因此,虽然是五转蛊,但却被方源轻易炼化入手。

%83
方源丢了雷翼蛊,正犯愁自身移动方面,又有短板。此刻,却得到了千里地狼蛛,堪称柳暗花明。

%84
这千里地狼蛛,属于坐骑之类的巨型蛊。本身以土为食,容易养活。

%85
方源将雪银真元灌注过去,千里地狼蛛便渐渐复苏,气息开始强大起来。

%86
它开始进食,就地吞噬大量的泥土。

%87
当状态恢复到一定程度,担心夜长梦多的方源便跳上它的背,驱动它立即前行。

%88
千里地狼蛛虽然只恢复了一部分,但终究是五转蛊,三对触角轮番交替,速度很快。

%89
再加上石茧碎后,之后山洞又显现出来。

%90
这条路就是当初,花酒行者逃跑开辟的。前人栽树后人乘凉,前人开道后人跑路,极大地方便了方源。

%91
唯一令方源焦虑的是,坐骑类的蛊虫任何行动,都会消耗蛊师的真元。

%92
千里地狼蛛是五转蛊,三转真元根本支撑不起。每隔一段时间,方源都得停下来,坐在千里地狼蛛的背上,一边催动天元宝莲,一边汲取元石,提取天然真元。

%93
这双管齐下,再加上丙等资质本身的回复能力,方源空窍的真元回复的速度,变得相当迅猛。和之前不可同日而语。

%94
驱动千里地狼蛛前进,停下来回复真元,遇到小塌方,就直接钻过去。。如此过程循环往复,方源渐渐地远离血湖墓地,向着地面靠拢。

\end{this_body}

