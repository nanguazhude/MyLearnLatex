\newsection{照影秘方,花酒迷离}    %第一百五十六节:照影秘方,花酒迷离

\begin{this_body}

%1
“方源大人,这边就是密室了,您请进。”幽暗的地下甬道中,一位二转蛊师老者,在前面走着,为方源领路。

%2
收藏着家族中蛊虫合炼秘方的密室,就在前方不远处。

%3
方源晋升家老,身份地位就有了翻天覆地的改变。家族中收藏的合炼秘方,只要不是四转、五转,他都有权利可以尽情查看。

%4
若是普通的一转、二转蛊师,要阅览秘方,就必须利用元石以及战功换取。

%5
这密室位处山寨地下,十分隐秘。即便是山寨都被抹平了,这处密室还会存在着。

%6
当初,古月山寨的创始者,一代族长,就是在此地下溶洞中发现了元气灵泉,才下定决心,在此立寨。

%7
经过历代族长的经营和发展,古月山寨的地下溶洞已经成为古月一族的秘密基地。

%8
平日里,寻常蛊师都没有资格出入这里,只有家老和族长,以及负责把守这里的暗堂蛊师才能有这样的权利。

%9
很多蛊师终其一生,也只来过一次。

%10
那就是资质大典,少年们都会来到元泉花海处,通过希望蛊开启空窍。

%11
除此之外,若非特殊原因,地下溶洞是绝对禁止闲杂人等进入。就连把守这里的蛊师,都是经过严格甄选。

%12
毕竟地下溶洞中的元泉,是整个古月山寨的根基。从它泉眼中凝结而出的大量元石,支撑着古月一族所有蛊师的修行。

%13
走在甬道中。两人的脚步声不断地在回响。

%14
依稀传来河水流淌的潺潺之音——那是溶洞中的地下河。

%15
片刻后,蛊师领着方源来到一处石门。

%16
他一拍肚皮,从空窍中飞出一只蛊,电光火石之间,就撞上石门。

%17
石门表面波荡了一下,如水面投入一颗石子。紧接着,石门表面光影变幻了一番后。渐渐消失,露出里面的密室。

%18
密室面积并不小,有一亩大小。每隔几步。就有一座半人高的石桌。数十张石桌上摆着白玉盘子,盘子中往往静静地栖息着一只蛊虫。

%19
这些蛊虫,各种颜色都有。约有一个拳头大小,外形也很相似,仿佛是蚕。头部上长有两只蜜蜂蜻蜓似的复眼。复眼色彩斑斓,如珐琅琉璃。只是表面并不柔软,披着一层幽亮的甲壳,甲壳上还泛着金属光泽。

%20
“方源大人,您是第一次来这密室。我就给您介绍一下。这些二转的照影蛊中,都记载着秘方。青绿颜色的照影蛊中,是家族中几乎所有的一转秘方。红黑颜色的照影蛊中,则是二转秘方。白色的蛊里是三转秘方。橘黄色的照影蛊中。则是四转秘方。至于五转秘方,当然就在最中央石台中的那只紫色照影蛊中了。”

%21
一旁,蛊师老者适时地介绍道。

%22
照影蛊是一种能记录影像的蛊虫,再进一步,就是三转的留影存声蛊了。

%23
合炼成留影存声蛊之后。不仅能回溯影像,更能听得响动。在石缝秘洞中,花酒行者就用一只留影存声蛊,给方源留下了关于传承的遗言。

%24
但记录秘方,并不需要声音,因此照影蛊也就足够了。

%25
合炼秘方这种东西。十分珍贵。用寻常的竹纸记载的话,被人偷盗去,不仅能合炼出古月一族的特有蛊虫,而且对家族的情报是一种严重的泄露。

%26
知己知彼百战不殆,经验丰富的蛊师从一个秘方中,就能看出合炼出来的蛊虫的优劣长短。

%27
秘方若是被泄露,对于一个家族来讲,损失是重大的。将来蛊师作战,本族的蛊师很有可能就被敌方针对克制。

%28
所以秘方向来严格控制,分级控制,并且特殊保存。

%29
照影蛊比较容易炼制,成本不高,是最常见的一种保存秘方的方式。

%30
“方源大人,这些照影蛊都是秘堂家老所有,但并不妨碍你催动真元,进行阅览。只是以您如今的身份,却还不能阅览四转、五转的秘方。还有两点:密室是家族重地,您在这里的一举一动,都受到暗中的监视。您每天每次在这密室,只能待上一刻钟。时间耗尽之后,您就得出来了。”蛊师老者这时又道。

%31
“嗯,我明白了。”方源点头。

%32
“方源大人,属下没有权利进去,就站在门口。时间一到,就会唤您出来的。”老者躬身道。

%33
方源举步迈进密室,刚走进去,背后的石门就由虚转实,将他一个人关在了密室当中。

%34
走在密室中,一片沉静。脚步声在耳边回响,四周墙壁上,镶嵌着几只水光蛊。

%35
从它们身上散发的光,波荡如水,光影变幻间,层层澜澜,明晦不定。

%36
方源信手取出一只红黑色的蛊虫,白银真元灌注进去,顿时从照影蛊的两只复眼中照射出两柱彩光。

%37
他稍稍高举照影蛊,将复眼对准石台上的白玉盘。

%38
两道彩光在白玉盘上交汇,一阵变幻之后,显现出文字来。

%39
这是一篇如何合炼出月痕蛊的秘方。

%40
首先是名列出月痕蛊的优劣,月痕蛊的优点在于攻击距离,是月光蛊的一倍。缺点在于攻击力并不出众。

%41
然后是合炼秘方的内容——月痕蛊是月光蛊和痕石蛊合炼而出。

%42
其次是合炼的注意事项,合炼时的窍门——若在合炼中途,增添一些玉石。或者在月光充足的夜晚,露天合炼,将有更高的成功率。

%43
最后是历代合炼者的心得体会。这处内容最多,洋洋洒洒,上万余字。

%44
方源看了,将一些东西暗记在心。

%45
到底是经过无数人的经营和实践,得出来的经验结晶。一些内容,就算是方源也不知道。

%46
毕竟,月光蛊是古月一族的特有蛊虫。在他前世,也从未得到过家老之位,来到这里阅览。

%47
时间有限,方源匆匆一瞥,就把手中的照影蛊放回原处,然后专挑白色的照影蛊观看。

%48
白色的照影蛊中,记载的都是三转蛊虫的合炼秘方。

%49
大部分并不适合方源,所需的蛊虫都是以月旋蛊,月痕蛊等等为底子。

%50
方源没有细看,为了节省时间,匆匆一览。

%51
他先前月光蛊配合小光蛊,合炼成月芒蛊。在月芒蛊的基础上,真正让方源看得入眼的合炼秘方,只有三个。

%52
第一个是三转蛊虫黄金月,射程仍旧是十步不变,但是攻击威力再度增强。一经射出,金黄色的月牙,足有半人高,透着一丝威武霸气。

%53
第二个是霜霖月蛊,月刃幽白冰冷,透着一股寒霜之气。中者将冰寒入体,行动缓慢。

%54
第三个是幻影月蛊。这蛊比较特殊,不是用来进攻,一经催动,就能令蛊师化出一道幻影,起到分散火力,迷惑敌人的作用。

%55
尤其是这幻影月蛊,若是合炼出来,以此为晋升基础,再往上就是四转的月影蛊。

%56
“月影蛊,能种入蛊师空窍,压制住蛊师真元的使用。”在关于幻影蛊的记载中,也稍稍透露了一些月影蛊的信息,提供一种参考。

%57
“月影蛊……不就是四转族长暗算了花酒行者,使用的蛊虫么?”方源看到这里,心中暗动,眉头则微皱起来。

%58
这世界蛊虫繁多芜杂,就算是他有五百年经历,真正所知的蛊虫也只是万千之一。

%59
他原先对月影蛊一知半解,如今知道这蛊的作用,心中疑惑顿生。

%60
月影蛊作用特殊,能压制四转蛊师三成真元,五转蛊师一成半,三转蛊师六成真元。也就是说,类似方源的丙等资质,只有四五成真元的蛊师,若是中了这月影蛊,一点真元都动用不了,战斗力暴降,等若直接被废。

%61
当然,若是方源本人,月影蛊对付他,就是肉包子打狗有去无回。

%62
皆因月影蛊一旦进入他的空窍当中,春秋蝉就会发威,气势压迫之下,月影蛊反而被方源瞬间炼化,成为他手中之物。

%63
让方源疑惑的是,不是月影蛊的作用,而是花酒行者。

%64
“我记得那影壁上,花酒行者的形象是浑身浴血,伤口满布。先前他和四代以及家老激战,都是些轻伤。那些重伤,原本以为是这月影蛊造成的,没有想到月影蛊的作用只是压制真元,令蛊师不能动用。那么花酒行者那严重伤势究竟是怎么来的?”

%65
方源心思谨慎,越想越疑惑。花酒行者当年战败家老,暂时撤离后,到底发生了什么事情?

%66
月影蛊不是致他死亡的主要原因,那么究竟是什么原因呢?

%67
一下子,原本清晰的花酒行者遗藏,又在方源眼中,显得扑朔迷离起来。

%68
“方源大人,已经一刻钟了,相信您定有一些收获。不管如何,请明天再来罢。”就在这时,石门又幻化为虚无,蛊师老者站在门口恭谨地说道。

%69
“好。”方源目光闪了闪,放下了手中的照影蛊,就出了密室。

%70
不管是黄金月,还是霜霖月、幻影月,这些合炼配方都不是他想要的。皆因这些蛊虫,食用的都是大量的月兰花瓣。

%71
月兰花瓣极难存储,只能保质数天。方源的计划中,他要离开古月山寨,闯荡世间的。要是合炼了这些蛊,没有食物喂养它们,大半年之后就死了。还不如不炼制呢。

%72
不过,还有些三转秘方,他没有观看。

%73
“明天再来罢。”这样的念头在他心中闪过。

\end{this_body}

