\newsection{野心一大,世界就小}    %第一百零三节:野心一大,世界就小

\begin{this_body}

酒肆并不大,但位置很好,位于山寨东侧,靠着东大门。

东大门和北大门,是人流量最大的两个寨门。因此每当午后,或者傍晚,酒肆的生意都会不错。

“少东家,您请坐。”一位老汉对方源点头哈腰。

几个伙计将板凳和桌子都用力擦了擦,对这方源谄笑,一脸讨好。

方源摇摇头,并没有入座,而是在这家酒肆中,随意漫步,四处打量。心中暗道:“这就是我的酒肆了。”

这酒肆只有一层,但有地下酒窖。

地面上铺着大而方的青砖,摆着八张方桌,有两张方桌靠着墙壁,其余六张桌子,都围着四条长凳子。

入了酒肆的门,就能看到一个暗棕色的长条柜台。柜台上摆着笔墨纸砚,还有算盘。柜台后就是酒柜,上面摆着大大小小的酒坛。有的是黑陶的大酒坛。有的是亮瓷的小酒瓶。

方源随意地走着,老汉和几个伙计当然不敢坐下,都亦步亦趋地跟着。

他们都忐忑不安,酒肆易主的消息来得很突然。上一个东家古月冻土,精明似鬼,严苛刻薄,他们被压得喘不过气来。眼前这少年,居然能把酒肆从古月冻土的手中。硬生生地抢过来,这个手段可不得了。所以这些人望向方源的目光中,就带着不安和畏惧。

方源忽然停下脚步:“不错,但就是这铺子小了点。”

老汉连忙上前,躬身答道:“是这样的,少东家。每年到了夏天,我们都会在外门搭上棚子。再摆些桌凳。但现在这会冬风凛冽,搭了棚子也没人坐。所以就撤了。”

方源微微侧身。扫了这老汉一眼:“你就是掌柜?”

老汉腰弯得更低了,越加恭谨地道:“不敢当,不敢当。少东家这家酒肆是您的,您属意谁当掌柜,谁就是掌柜。”

方源点点头,又扫了其他伙计一眼,看起来都是些精明能干的伙计。

若是在地球上。他要担心这掌柜和伙计联合起来,欺瞒自己这个主人。但是在这个世界。蛊师高高在上,打杀凡人只是一念之间。就算是舅父舅母撺掇指使。这些凡人也绝不敢针对方源。

“好了,去把账簿拿来,再给我倒上一壶茶。”方源坐了下来。

“是的,少东家。”掌柜和伙计都是一阵忙乱。

账簿足有十六册,每一册都用的竹纸,透着淡淡的绿意。拿在手中,比宣纸要脆硬许多,适合南疆这样潮湿的气候。

方源随手取出几册,稍微浏览了一些,问了一些问题。

掌柜的连忙对答,不多时,就满头的大汗。

方源前世创建血翼魔教,教众数十万,自然经验丰富,目光老辣,区区账簿别人看了或许一头雾水,眼花缭乱。但在他眼中,什么疑点都是明察秋毫,洞若观火。

这酒肆是排在九叶生机草后,第二大进项,方源自然要牢牢抓着。

这账簿上的问题并不大,属于一些错漏。这些凡人还不敢放肆。

只是方源翻了最后一页,发现这个月的进账都被古月冻土提走了。

“少东家,这是上一位东家亲自来提的,小的们也不敢违抗啊。”掌柜一边回答,一边擦着汗,他年迈的身躯,早已经在微微的颤抖,脸色苍白得很。

方源沉默不语,轻轻地把账簿放在桌上,扫了掌柜的一眼。

掌柜的顿时感到一股庞大的压力,简直是山一样的压来。他心惊胆战,扑通一声跪倒在地上。

见掌柜的都跪了,其他几位伙计也是精明伶俐的,旋即一个个都跪了下来。

方源安然地坐着,转过视线,扫视他们。

伙计们顿感自己如置身冰天雪地当中,难以抵挡方源的气场,俱都噤若寒蝉。

酒肆的工作,对于他们这些凡人来讲,不仅稳定而且安全,最为理想不过。所以他们都不想丢掉这份工作。

方源见立威的效果已到,过犹不及,便缓缓开口:“过去的事情,就算了,我不计较。我看了账簿,你们的工钱有些低了,今后伙计的工钱提高两成,掌柜的提高四成。好好的干活,会有你们的好处的。”

完,方源起身,向门口走去。

众人跪在地上,楞了半晌,这才明白过来,各个泪水喷涌,在脸上横流。

“谢谢少东家的大恩大德啊!”

“少东家仁慈宽厚,小的们一定全力干活!”

“少东家您就是我们的再生父母,请您慢走。”

身后传来众人感激涕零的声音,还有不断磕头,额头碰到青砖的响声。

恩威并施,不论哪个世界,都是上位者驭下的不二法门。这其中,威才是基础,在威凌之下,任何一点点小小的恩惠,都会被放大千万倍。

没有威的施恩,只会得到一个烂好人的名号。甚至久而久之,如此施恩,不仅不会引来感激,反而会引起觊觎和灾祸。

“不过这些御人的手段。都是些小道。放在地球上会被世人推崇,但在这里,唯有提高自身的力量,才是大道。不,就算是地球上,也是力量第一。”

方源不禁想到红朝赤祖。

当年赤祖亦是历经一番磨难,得出一个结论:枪杆子里出政权!

这就是赤裸裸的真理了――任何政权的基础,都是力量。所谓的权利。不过是力量的附属品罢了。

事实上,不仅是权利,财富和美色,亦都是力量的衍生物。

离开这间酒肆,方源又去了那三处竹楼。

这三栋竹楼都被舅父舅母租了出去,几乎住满了人。

这个世界,讲究多生多育。对于日益庞大的人口。山寨的空间就显得狭小了。

家族中,都是大子继承制。其他的子女就得自力更生。很多人就算是仗着家族政策。分了些微薄的家产,在外辛苦打拼,但终其一生,其积蓄也不够买一栋竹楼。

一来,养蛊消耗甚大。二来,山寨中空间有限,房价金贵。

当然了山寨外。自然可以随意搭建房屋,但是那很不安全。总会有野兽毒蛇出没。闯入民宅。尤其是每次兽潮一过,不管寨外什么房屋。都会被摧毁冲垮。

山寨扩建是唯一的解决之道。

但是一旦扩建,外围防御的地方就多了,兽潮袭来,以山寨的力量,根本就防御不住。再者,地方大了,若是有其他山寨的蛊师潜入,也难以警戒搜查。

历史上,古月山寨几次扩建,都被兽潮摧垮。汲取了经验教训之后,现在的山寨已经是最大的规模。

方源浏览了一遍,了解了一些情况后,心里就有数了。

这三栋竹楼都被舅父舅母经营得很好,租金也定制得敲到好处,索性就照此发展。他算了一下,这三栋竹楼给他带来的收益,虽然没有酒肆的多,但也相差不了多少。

总体情况,比方源原先料想的,要乐观得多。

就在前天,他还只是一个两手空空,身上元石不足五块的落魄穷小子。如今却一下子跻身到族中小富豪的行列。

那些租房的女房客,都是辗转颠沛的二转女蛊师,得知了方源的身份后,看向他的目光都带着媚色。

只要能傍上方源,嫁给方源,对于这些女蛊师来讲,今后也就不用在外奔波,冒着死亡的危险,至此过上安稳的生活。

这种生活,也是她们奋力拼搏,一直想追求的东西。

也就是说,现在只要方源愿意,他甚至可以马上退隐,过上舅父曾经享受的富足生活。

他勾一勾手指头,就会有许多的女蛊师蜂拥而至。

“但这一切都不是我想要的呀。”方源站在竹楼二层,任由女蛊师们火辣辣的视线落在自己的身上,他眉头紧锁,手扶栏杆远望。

远处,青山连绵一片,宛若横卧的巨人,把灰蓝色的苍穹当做被褥盖在身上而酣睡。

万里江河,苍莽大地,何时才能任我纵横?

风云变幻,龙蛇起陆,何时才能睥睨众生?

“重生以来,如无根浮萍,随波漂流。如今竭心尽力,有了这家产,可称基业,算得上成功自立,站住了脚跟。接下来,就是挖掘出花酒传承,奋发图强,修到三转,闯荡天下!”方源深黑的双眸中,燃烧着火焰。

青茅山,不过是南疆十万大山的一座。而南疆,亦不过只是天下的一隅。

太小了,太小了!这种地方怎么能承载住他的勃勃野心?

和他的野心相比起来,这些许的家产,旁人争得头破血流,朝思暮想的东西,渺小得如同一粒尘埃。

“哥哥,你下来,我有话跟你说。”不知何时,古月方正来到了竹楼下,仰着头对方源喊道。

“嗯?”方源被打断了思绪,面目冷漠,看向楼下的方正。

兄弟俩四目相对,一阵无言……

弟弟方正站在楼下,被另一栋楼的阴影罩住,他仰起的脸庞上,眉头紧皱,一双眼睛闪烁着光。

哥哥方源站在楼上,阳光沐浴着他,他微垂的眼帘下,瞳眸漆黑如夜。

相似的面容,彼此倒映在相似的眼眸中。

对于弟弟的出现,方源丝毫不觉得意外。方正就是舅父舅母争夺家产的一件利器。

然而,又能如何呢?

方源居高临下,俯视着方正,心中则发出一声轻轻的叹息:“甲等资质又如何,左右不过是颗棋子……真是渺小啊。”

\end{this_body}

