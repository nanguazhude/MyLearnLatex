\newsection{不出算计收获丰}    %第七十八节:不出算计收获丰

\begin{this_body}

厅堂中,族长古月博面无表情,双眼深邃,高居在主位。

其余的十多位家老,则各自端坐着,眼观鼻鼻观心,只用眼角的余光隐晦地扫视左右,企图从周围人的表情上分辨出些线索。www.13800100.com

气氛一下子微妙起来。

“方源能夺得第一,其中蕴藏的深意绝不简单。他居然说途中捡到一个藏着野猪牙的袋子,这话也太假了。”

“这袋子只可能是有人暗中准备好的,单靠方源一个人是不可能完成的事情。也就是说,有人暗中帮助方源。”

“此届的年中考核,不同以往。因为要调动数十位二转蛊师,监视考察,所以这个考核内容,不仅学堂家老知道,家族中的许多其他家老都知道。”

“要提前安排好这个袋子,也只有在场的诸位家老,甚至可能直接就是族长!”

家老们各个年老成精,在政坛上摸爬滚打多年,立即就想到了很多东西。

方正是甲等资质,若真培养到四转,那是什么?

那就是下一代的族长!

方源是方正的哥哥,虽然只有丙等资质,但单靠这点血脉亲缘,就有投资的价值!

对于族长古月博来讲,若是暗助方源,将其纳入自己体系,将来对方正来讲,也是一个很好的维系纽带。

对于家老们来讲,方正是这些年唯一的甲等天才,已经归入族长一脉。若真培养起来,族长一脉必然越加强势。若是家老们将方源收为己用,利用这层血缘关系,将来应对方正也是个极为好用的棋子。

所以,在这厅堂中坐着的十多人,都有帮助方源的动机。

但是会是哪一个人呢?

古月赤练沉思着:“我没有招揽方源,会是谁暗中帮助他?是漠尘那老家伙么,有些可能。虽然方源杀了他家的家奴,但家奴算什么,就算全死光了也不心疼。族长更有可能,他收容了古月方正,再将古月方源拉过去,这样对方正的掌控力就大了!只是……过往惯例,都是年末拉拢收纳,如今年中就动手,这是坏了规矩啊。”

“也不算坏了规矩,但此举的确是打了擦边球。是谁这么看好方源,比我还看好他?”古月漠尘也在思索。

事实上,方源杀死高碗,碎尸送礼之后,他就对方源刮目相看,也有些招揽的心思。

但是这种招揽的行径,一般都发生在年末,学员们从学堂毕业之时。

方源被人提前招揽,这让古月漠尘也有一丝的猝不及防。

古月博的目光,则主要集中在古月漠尘和古月赤练这两个当权家老的脸上。

这位族长想得更多,更深一些。

方源明目张胆地撒了一个谎言,从而赢得了第一。这行为有一种肆无忌惮的意味,更是一个对外发出的讯号,就是要让你们看出来――方源我罩了!被我提前收容了,你们最好不要动他。

那这个人会是谁呢?

古月家族中,政治格局是三足鼎立。除了族长之外,就是古月赤练代表的赤脉,古月漠尘代表的漠脉。

古月博心知肚明,自己没有招揽方源的举止。那么最大的嫌疑人,就是古月赤练和古月漠尘。

“这两个老家伙,演技越来越高了。单看神情,真的看不出来。难道不是他们,而是其他的小势力?”

古月博不着痕迹地观察着,揣测着。他并不知道,其他家老和他一样,也在如此观测着、怀疑着、揣摩着。

学堂家老也在揣测,但是他一直保持中立,游离于政治斗争之外,因此他的思想却简单多了:“原来方源被某一个家老招揽,难怪他在勒索时,主动放过了方正、漠北和赤城。估计招揽他的不是族长,就是赤练或者漠尘家老。这是好事啊!这说明他已经认清了现实,初步融入了家族。不管怎么样,他都是家族的一份子。将来完全融入进去,会甘愿为家族的奉献一生的!”

沉默了片刻,古月博见也看不出什么端倪,只好开口道:“来而不往非礼也,对方居然现在就开始针对方正,我们古月一族也不是好欺负的,必须还以颜色!暗堂家老,这件事情你先计划一下,然后汇报我。”

“遵命,族长大人。”暗堂家老立即点头应承下来。

“至于古月方正,我担心他遭逢这样的变故,心灵受到打击。他是甲等资质,对我族来讲,意义不需再说。从今以后,我将单独辅导他。”古月博又道。

没有家老提出什么反对意见。

其实很多人都知道,族长早就暗中给古月方正开小灶了。现在把这事提出来,虽然有违公平的原则,但是理由充足,家老们也不好阻止。

“至于古月方源……”古月博故意拖长了音调。

霎时间,全体家老都竖起了耳朵,难道就是族长暗中出手帮助方源的么?

古月博扫视一圈,他也在观察周围家老们的神色。但他注定失望了。

于是,他只好继续道:“他能以丙等资质,夺得第一,很不容易。我就以个人名义,奖励他三十块元石。学堂家老,你带一句话给他,让他好好努力。”

“是,族长大人。”学堂家老躬身领命。

“三十块元石,这个不痛不痒的奖励,算是什么意思?”家老们都皱起了眉头。

“不管是谁暗中吸纳了方源,这三十块元石,就是我善意的信号。毕竟古月之外,还有白家寨、熊家寨啊。”古月博心中叹着。

方正被刺,是外敌。方源作弊,是内斗。

对付外敌自然要用强硬的手段反击,对于内斗,族长古月博则以怀柔为主,为的就是避免内斗太多,引发家族整体实力的下降。

“好了,这事情就先这样处理吧。大家都退下,好好主持手中的工作,家族的兴盛和各位的表现是分不开的。”古月博挥了挥手。

“是,族长大人。那我等告退了。”

家老们陆续走出,几个呼吸之后,厅堂中只剩下古月博一人。

他深深地叹了一口气,用手指揉捏着自己的两个太阳穴。

身为族长,算是古月一族权位最高的第一人了,但是他并不轻松,需要协调多方利益。更不能随心所欲,族中的势力盘根错节,一代传承一代,都有悠久历史,掣肘颇多。

对外,他要面对行事蛮横的熊家寨,渐渐崛起的白家寨。

对内,他要处理族内复杂的政治斗争。他虽是中年,但已经双鬓花白。

“担当族长的这些年,虽然资源充足,但是修为却进展缓慢,心神已经被繁芜的家族事务拖累了。有时候,真想当一名独行侠啊,自由自在,一身轻松。没有了负担,自然步履轻快,也许我还能在修行的道路上走得更远。”

古月博心中叹息一声。

只要入了家族体制,身上就有责任,有了责任就不轻松,再也不能一心一意进行修行。

但反过来讲,不入家族体制,那家族的资源就不会供应。没有资源供应,独行的蛊师修行也会举步维艰。

这就形成了矛盾,一个怪圈。

就是这个怪圈,不知道禁锢了多少人的前途,埋没了多少人的才华和天赋!

王大死了。

三天之后,方源得到了这个消息。

于此同时,他还从江鹤那里打听到,那两个年轻猎人果然也在上山狩猎的时候,失踪了。至于那个被方源斩断右手的猎手,也因为抑郁,在家中“自杀”了。

江鹤说这话的时候,看着方源的目光十分意味深长。他后来见过了王大的尸体,自然就认出了他的身份。

但是他不敢说出王大的真实身份。

他是驻村的蛊师,担负着责任。只要是入了体制,不管位置高低,都有责任。

王大成为魔道蛊师三年多,他身为驻村蛊师却一直不察,真要追究下来,就是他履历的一大污点,在家族中几乎前途就毁了。

那三个年轻猎手蹊跷的死,他也尽全力压下来。

“方源,咱们都是熟人了,今后到我表弟江牙店铺里买东西,一律五折!”某一次,江鹤对方源如此道。

其他人都死了,知道这件事情真相的只有他和方源两人。但这事如果揭露出来,对方源来讲,没有太大影响。

因为他只是杀了三个农奴罢了,就算杀三十个,族中也不会追究,顶多是做一些“十几块元石”之类的象征性惩罚。

江鹤主动的贿赂,方源自然来者不拒。对他方源来讲,这事情从头到尾都充满了意外和惊险,但好在结果不错。

此事之后,在家族中毫无根基的方源,得到了半个盟友,还有一个并不存在的靠山。

就是这个神秘的靠山,让方源的身上披上了第二层保护伞。当他将来渐渐地不再显得弱小的时候,这个保护伞将起到关键性的作用,至少能支撑着他修行到二转,而不引来太大的打压。

现如今,他可以明显地感觉到,学堂家老对于自己态度的缓和。

十几天一晃而过。

在古月方正、古月漠北之后,方源几乎和赤城同时,顺利地晋升到一转高阶。

虽然抢劫勒索仍然在继续,但他从此不再抢方正、漠北和赤城三人的元石,他越来越低调,自身的实力也在以超越前世数倍的速度,快速增长着。

------------

\end{this_body}

