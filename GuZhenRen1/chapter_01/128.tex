\newsection{锯齿金蜈}    %第一百二十九节:锯齿金蜈

\begin{this_body}

方源一觉醒来,已经是第二天的中午。

头已经不疼了,剧痛已经消失。

他下意识地摸了摸自己的耳朵,触感和先前别无二致。仿佛昨夜的割耳的事情根本就没有发生的一样。

他从床上爬起身来,首先是寻了镜子,照了一下。

镜子中,显露出一个少年的面目,并不英俊,但是一双漆黑深幽的眸子,却让他脱于凡俗,泄露出一丝不同寻常的冷酷魅力。

少年的双耳一般大小,别无二致。

昨晚刚刚种下地听肉耳草时,方源的右耳又肥又大,耳垂几乎已经抵到了下巴。但是如今却变得普普通通,从外形上看去,根本就发现不了异常。

这是因为身体和地听肉耳草已经相互适应了。

方源心中一动,从空窍中调出一股些微的赤铁真元。真元顺着他的身体,一路往上,最终流入到右耳当中。

顿时,他的听力暴涨数倍,无数人的脚步声都传入他的耳中。

他虽在二楼,但是却有一种脚踩大地的错觉。

方源用心倾听,真元继续催动,听力也随之增强。在镜子中,他的右耳也慢慢地生长出了根茎。

这些根茎,像是千年老参的参须,从耳廓处向外延展,稀稀疏疏,不断地变长,垂向地面。

同时,右耳也有一种要变大变肥的迹象。

方源试着停止催动地听肉耳草,一秒之后。耳廓向外生长的参须全部缩了回去,右耳也变成了普通形状。

当然,听力也回落到原先的状态。

“这样一来,我就有了侦察之蛊。”方源穿上衣衫,从床底下取出水盆。

昨晚殷红的血水,因为投入进去的煤石,已经化成了一片黑沉的污水。浸泡在其中的血巾也是如此。原先的血红还有白色的底色,都被黑灰色遮盖。让人很容易地就联想到,厨房间里沾染油污的抹布。

这盆水端出去。就算是当着其他人的面倾倒出去,也不会露出马脚。

夏天就要到了,许多族人都要擦拭清洗用了一个冬季的火炉子。经常会清洗出这样的一盆污水。

趁热打铁,方源再入石缝秘洞。

这次他没有空手而来,而是在野外捉到一头幼鹿,用麻绳捆住四蹄,用铁罩套着嘴,再用隐鳞蛊隐没身形,带到了石门前。

他没有忙着推门,而是首先催动了地听肉耳草。

参须从耳廓周边延展而出,他的听力顿时得到了数倍的增幅。

咚咚咚……

首先,他听到一阵阵轻微的。缓慢的心跳声。

随着参须越来越长,心跳声也越来越响,同时声源数量也在增加。

方源几乎不用思考就知道,这些心跳声都是来自后方石林的玉眼石猴。

他闭上双眼,脑海中就能想象出。这些奇特的声音蜗居在自己的石洞中,盘缩着身躯,沉眠的景象。

但这些并不是他想要感知的。

他继续倾听,右耳已经变得有些肥大了,耳廓上的参须足有半米的长度,根须有灵性的。都蔓延到石门上,扎进去一个很浅的深度。

在这一瞬间,方源感到自己的听力得到了一种巨大的提升。

以他为圆心,他感知到方圆三百步的范围内的无数声音!

这才是地听肉耳草的真正用法,先前在山寨中只不过是浅尝辄止。

地听肉耳草的参须若不接触大地,侦测范围在二转蛊虫中,只能说是普通水准。但是一旦参须扎根在土地里,它的侦察范围就能得到质的提升。

这点很好理解。

用地球上的理论,声音的传播速度和介质有关。声音在大地,在水中的传播速度,要远远大于空气中的传播速度。

在中国古代,一些打仗的士兵,在睡觉的时候,都会枕着木制的箭囊。一旦有骑兵奔袭,士兵们透过大地传来的声音,就能及时地被惊醒。若是单靠空气传播,根本就反应不及。

参须扎根石门,方源顿时感到石门后的动静。

那是一阵阵稀稀疏疏的声音,很是细微繁芜密集。和这种声音相比,石猴的心跳声就宛若敲击大鼓的鼓声。

换做新人第一次用这地听肉耳草,恐怕听到这里,就要费尽思量的揣摩猜测。但这声音亦在方源的意料当中,只是他听了一阵后,眉头渐渐皱起。

他索性推开石门。

石门沉重,但是他如今有二猪之力,却是毫不费劲。

石门洞开,展现在他眼前的是一个幽深黑暗的甬道,一直水平地延伸出去,通向未知和神秘。

方源将捉来的幼鹿解开绳套,抛向前方的甬道。

幼鹿灵性十足,在冥冥中感应到前方黑暗中的危险,不敢前进。它大大的双眼看向方源,似乎流露出恐惧和哀求之色。

方源冷哼一声,甩手一记月刃。

这月刃被方源控制的威力不大,斜斜一切,在幼鹿的身躯上,划出一道浅浅的血口。

血液涌现出来,疼痛下,幼鹿对于方源的恐惧顿时占了上风,它慌乱地就向黑暗的前方奔去。

黑暗很快就将它吞没。

方源再催地听肉耳草,这一次参须都扎根在身侧的墙壁上。

听力得到暴涨,首先方源听到的是幼鹿急窜的脚步声,它的心跳,然后是骤然而起的稀疏声潮。

金蜈洞中杀身祸,可用地听避凶灾……

方源心知肚明,这些稀疏的声潮便是一只只蜈蚣在爬行的声音。

耳中,忽然传来幼鹿的惊嘶。

显然它闯入了虫穴,遭遇了蜈蚣。

方源的脑海中顿时想象出一个画面来:幼鹿惊慌失措,蜈蚣群如潮水般涌来,包围了它。幼鹿在原地惊惶地打转,踏着幼蹄,感受到死亡的气息而发出哀鸣。

密密麻麻的蜈蚣攀上了它的身躯。幼鹿倒在地上,满地打滚,剧烈挣扎。

只是过了一小会儿,它的心脏就停止了跳动。

蜈蚣群覆盖在它的身躯上,开始啃噬幼鹿的血肉。

方源目光突地一凝,他忽然听到了一个特别的响声。

这声音嗡嗡嗡的,就像是电锯开动起来一样,充斥着张狂、蛮横、粗野的气息。

若换做新手听来,必定要疑惑万分。但是方源凭借着丰富的经验,却在第一时间猜中了这个声源的身份。

一只三转的野生蛊虫――锯齿金蜈!

这是蜈蚣群的虫王,金蜈洞中的真正杀机。

方源完全可以想象出来:有着一米多长,双拳之宽的金色大蜈蚣,忽然由静转动,蜿蜒爬行起来。

蜈蚣的身躯两侧,是一排锐利的银边锯齿。随着它的爬行,这些锯齿也在急速地转动,仿佛是电锯一般。

锯齿金蜈的到来,顿时让蜈蚣群一静。

它汹汹而来,所到之处,蜈蚣群无比辟易四散,露出还剩下大半的幼鹿身躯。

它爬到幼鹿的身躯上,张开狰狞的口器,汲取鹿血,吞食鹿肉。遇到鹿的骨架,它就卷起身躯,利用银边锯齿轻轻一绞,就将骨头轻而易举地搅成骨粉。

“就算是白玉蛊的防御,也吃不住这锯齿金蜈的一阵锉锯。看来花酒行者的意思,就是让我用这地听肉耳草,避过这只锯齿金蜈。不过我有春秋蝉,要收服这只锯齿金蜈,也不是不可能!”方源脑海中,一个念头萌生出来。

他是见猎心喜了。

锯齿金蜈也是较为优秀的蛊虫,若是纳为己用,必定是一个强力手段。

只是要在地底,收服这样的一只锯齿金蜈,恐怕就算是四转蛊师也得大费周章。

捕捉和围杀完全是两码事,前者的难度要远远高于后者。

野生蛊虫大多很是狡猾,锯齿金蜈见机不妙,就会地遁而走。蛊师若没有地遁的手段,怎么追击?

不过方源有着春秋蝉,只要抓住这只锯齿金蜈,泄露一丝春秋蝉的气息,必定能使得这虫畏缩万分,不敢有丝毫的动弹。

春秋蝉高达六转,它的气息对于一转到四转的蛊虫,都有强烈的震慑效果。但是对于五转蛊虫,效果就不大了。对于同级别的其他六转,就更无一丝威慑效果。

这个现象很有意思,其实想想,人类社会也是如此。

面对异常优秀的伟人,人们常常会敬佩、羡慕和崇拜。但是对于只优秀自己一点的人物,就敢于冒犯了,因此通常都是竞争和嫉妒。

“只是我现在才是二转中阶,现在要收服三转的锯齿金蜈,虽然也不是不可以,但这时机却有些过早了。”方源沉吟起来。

二转蛊师,当然用二转的蛊虫最合适,也最趁手。当然,二转蛊师们也照样能拥有三转、甚至四转的蛊虫。

只是这个现象并不常见。

一来蛊虫越是高级,喂养的费用就大大增加。二来越是高阶,蛊虫用起来就越不方便,需要付出一定的代价。就好像是婴儿挥舞大铁锤,强行舞动的话,很有可能被铁锤的重量拉伤筋肉,砸中自己的脚。

“幼鹿已经惊动了整个虫群,恐怕得有十天半个月,才能让这虫群真正的平静下来。现在一番试探,已经验证了我心中的猜想,但却并非接着探索的良机。不如先放一放,缓一缓,这事情不能太着急。”

方源想到这里,便重新关上了石门,利用隐鳞蛊退了出去。(未完待续。如果您喜欢这部作品,欢迎您来投推荐票、月票,您的支持,就是我最大的动力。)

------------

\end{this_body}

