\newsection{严冬不肃杀,何以见阳春!}    %第一百二十八节:严冬不肃杀,何以见阳春!

\begin{this_body}

山体石林仍旧充斥着一片晦暗的赤光。根根石柱从洞顶垂下,如倒长的参天巨木,蔚为壮观。

“距离上一次进入这里,已经是一个多月了吧。“”方源心中想着,有些无奈。

自三寨联盟之后,不断扫荡周边的狼群,以至于石缝秘洞附近,总是不断地有蛊师出没。在这种情况下,即便方源有隐鳞蛊隐身,也有暴露的危险。

方源做事谨慎,五百年的生涯也孕养了足够的耐心,这段时间就一直克制着欲望,没有过来这边。

直到冬去春来,残狼被扫荡干净,蛊师进出的少了,方源这才在外门绕了一个大圈,然后动用了隐鳞蛊,一路隐身来到此处。

一个多月过去,曾经打通的通路上,又有一些石猴群迁徙过来。

不过庆幸的是,猴群的数量较为稀少。

方源花费了一些功夫,斩杀猴群一路挺进,终于再次抵挡石林的最中央。

巨大的石柱阴影下,一个明显是人工开辟的洞口,出现在他的眼前。

洞口中,粗糙的石阶一直通往黑暗的地下。

方源催动白玉蛊,浑身上下罩住一层微亮的玉光,小心翼翼地拾阶而下。

他的左手高举着一只火把,右手掌心则亮着盈盈的月光――月芒蛊已经蓄势待发。

周围一片黑暗,就算是有火把,也只能照亮周围五步远的距离。

这种情况下,若是有一只照明的蛊虫就好多了。可惜方源身家还没有如此丰厚。

一步一探。走了半晌,他这才走到石阶的尽头。

一扇做工粗糙的石门出现在方源的面前。

“金蜈洞中杀身祸,可用地听避凶灾。”方源举起火把,发现石门上刻着一些字。

地听……

金蜈……

方源眯起双眼,精光闪烁一阵,隐有所悟。

“如果我所料不差……”他蹲下身子,用手抚摸地面。很快就在石门前摸到一处湿润的土地。

“有了。”他心中微微一喜,用手挖开这片泥土,果真发现了一朵地藏花。

小心翼翼地拨开花瓣。他从花心中取出一只蛊虫。

这只蛊相当特别。

它就像是一只人的耳朵,只是更小一些,通体都是土黄色泽。干瘪且暗哑

捏在手中,有些干瘪,就像是腌制的萝卜干,带着一点微微的温热。

耳朵周边上,向外延伸出根茎。这些细长的根须,足有数十根,像是人参的参须。

这是一棵二转的草蛊,人们称之为――地听肉耳草。

看着手中的地听肉耳草,方源目光闪了几闪。

这地听肉耳草来的及时,可用于侦察。正适合他用。

地听肉耳草有一大优点,就是侦察距离远达三百步,这个距离在二转侦察蛊虫中,相当的出类拔萃。

而且地听肉耳草又容易喂养。

它的食料,便是人参的参须。

这世界的南疆多是深山老林。人参比地球上要多得多。猎户上山狩猎,时不时地就会采到一些。

尤其是人参很容易贮藏,只要确实是干透了的,密封之后常温之下,就能存储很久。

月兰花瓣几天内就会枯萎,但参须却能长久储藏。

“蛊虫养、用、炼。这三个方面博大精深,奥妙非凡。地听肉耳草虽然容易养,但是用起来却比较麻烦。”方源暗忖。他手捏着这棵地听肉耳草,真元暗暗一催,顷刻间就炼化了它。

地听肉耳草十分优秀,在侦察的距离上简直可以媲美许多三转蛊虫。但上天是公平的,要使用地听肉耳草,并非炼化了就行,还需要付出某些代价。

就像是僵尸蛊、木魅蛊,必须搭配一些其他的蛊虫一齐使用。否则单个用久了的话,蛊师的身躯就会受到侵蚀,变成真正的僵尸或者树人。

“很多蛊虫,单单炼化成功,却也不能使用,还需要一些特殊条件。这棵地听肉耳草就是如此。我若是有了这蛊,就能探听虚实,真正在狼潮中做到游刃有余,甚至能利用这场狼潮,达到一些目的……”

方源思索了一下,就决定采用这棵地听肉耳草。

收益已经大于损失,因此哪怕是要付出那个代价,为了未来,也算不了什么。

“不论哪个世界,没有付出,哪来收获?”方源心中冷笑一声,将地听肉耳草收入空窍之内。

他深深地望了石门一眼,转身离开。

如果所料不差,石门的另一侧将有一番凶险。需要用地听肉耳草,才可避过。

方源退出秘洞,却不忙回转山寨,而是在外转了几圈,猎杀了几头孤狼后,得了几对狼眼珠子,这才回去。

战功榜上他仍旧是最后一名,走在街道上,一些认识他的蛊师,都对他投来隐隐嬉笑、轻视的目光。

方源也不在意,仍旧我行我素。

其后几天,他用微薄的战功兑了参须,好生喂养了那棵地听肉耳草,让它恢复到生机饱满的状态。

家族的大部分注意力都被狼潮所牵扯,无人关注方源。若是以前,兴许舅父舅母那边还会有所纠缠,但自从方源将家产转卖过去后,那边就再无动静了。

明事难成,暗事好作。

很快,方源低调且顺利地完成了一些准备工作。

这一夜,月明星稀。

月亮如玉盘高高悬挂,温柔夜幕如纱,罩住青茅山。

时不时地,就有一阵狼嚎声从远方隐约传来。

方源将门窗关紧,赤身裸体地站在租房的地板上。在他面前的小桌上。摆着一张盛满温水的面盆,面盆上旁是一张白色的巾布,白棉布上摆着一把锋锐的匕首。

就连他站着的这块地板上,都铺了一层厚布。

一缕缕的月光,透过窗户的缝隙,映照在桌子上。

方源面色淡漠,伸手抓住匕首。雪亮的刀刃上。透着寒芒,能当做镜子照。

在微光下,少年的眉目映在匕首上。透着一种冰冷。

就在此刻,方源不由想到地球上的一部武学秘籍《葵花宝典》。

《葵花宝典》第一页第一句话,即是“欲练此功。挥刀自宫”。

要获得速成的力量,就必有舍弃和付出!

自宫又如何?

没有这种近乎残酷的决断和割舍,如何能成就野望,实现霸业?

想要不付出,就得到,那都是欺骗小孩子的童话。

换到方源如今的状况,为了能用这地听肉耳草,付出点代价算得了什么!?

想到这里,方源忽的一声冷笑。

他用手指指腹,轻轻地抚摸着冰冷的刀刃。轻吟出声――

月如霜满夜,刀光尤冷寒。

严冬不肃杀,何以见阳春!

话音刚落,他闪电般出手。

手起刀落,血光迸现。

一块肉就掉在了桌上――

方源整个的右耳都被切掉。血液喷涌而出。

霎时间,他先是感到耳朵一凉,然后是一股强烈的剧痛猛地袭来。

他咬着牙关,微微倒抽一口冷气,强忍住痛楚,从空窍中召出地听肉耳草。

这棵地听肉耳草被他养得生机勃勃。已经和刚取出来的并不相同。

刚从地藏花中取出时,它色泽暗哑且干瘪,如今却是温润肉呼,又肥又大,涨到成年人的巴掌大小。

摸在手中,富有弹性,像是地球上佛像的垂下的耳朵。

方源将地听肉耳草按到头侧的伤口上,一股赤铁真元紧随着灌注而入。

地听肉耳草的根须顿时得到了一股动力,以肉眼可见的速度开始生长,扎根到了方源的伤口处。

这又是一股疼痛!

方源仿佛感到,有数十根蚯蚓从伤口处向自己的脑袋中钻进来。

这种感觉不仅痛楚,更有一种恶心。

一般而言,进行这个过程,蛊师都会利用一些其他的蛊虫麻痹自己的神经。但是方源却无这种条件,只能靠着自身钢铁般的意志力,硬生生地扛起。

到底是少年的身躯,方源承受着这样的剧痛,身体也不由地开始晃动。

越来越多的根须,延伸进伤口当中。慢慢的,地听肉耳草和血淋淋的伤口黏合起来,成了方源新的右耳。

到最后,伤口处不再流血,甚至就连疤痕都没有。

但是方源脸色惨白,剧痛只是稍减,仍旧折磨着他。

他头冒青筋,心脏咚咚的急速跳动。

到此处,已经成功大半,但方源的身躯仍旧需要时间来适应这地听肉耳草。

他先取出镜子,借着微微的月光,照了照。

只见镜子中,自己面色苍白,眉头微微皱着。左边的耳朵小,右边的耳朵却是肥大了两倍不止,有些畸形。

方源也不意外,反而照了片刻后,没有发现问题,感到一阵满意。

他放下镜子,又取布巾,沾上盆中的温水开始擦拭身上的血迹。

他没有穿衣服,血迹清理得极为方便。一些血迹流到脚边,也都被地板上事先铺着的棉布吸收住。

方源将血迹清除干净,最后拿起桌上自己原先的右耳。

他冷哼一声,手掌心中月芒蛊一催。就将自己的右耳,绞成了碎肉末儿,来个毁尸灭迹。

剩下的满盆的血水,则被方源端到床底下,又投了一块煤石进去。

做完这一切,方源这才躺倒床上。

疼痛已经削弱大半,但是仍旧折磨着他。

方源感到自己的头一阵阵的痛,随着心跳,突突突。

也不知过了多久,他终于沉沉地睡去。

(ps:这些天重新看了一下前文,可能是专注于抒发了心中的郁愤,导致了类似说教的感觉,还有节奏缓慢。进行了一些思考,今后将做些调整,重视情节推进,,更纯粹地来讲故事,让行文更流畅一些。)(未完待续。如果您喜欢这部作品,欢迎您来投推荐票、月票,您的支持,就是我最大的动力。)

\end{this_body}

