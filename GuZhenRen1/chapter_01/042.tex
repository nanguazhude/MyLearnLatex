\newsection{竟然真的开出了蛊?!}    %第四十二节:竟然真的开出了蛊?!

\begin{this_body}

%1
“咦?”

%2
“不会又是石中石吧。”

%3
“看样子应该是的。不过有些奇怪,这黄土被坚硬的紫金石皮包裹在里面,应该被挤压的圆滑,怎么表面凹凸不平呢?”围观的蛊师们疑惑不解。

%4
看到手中的黄泥土球,方源表情不变,心中却微微一动。

%5
他继续磨搓,蓝色光辉如水,泥土成粉落下。粉状的泥土中,还夹杂着了不少的土疙瘩,接连掉落在他脚边的石粉堆上。

%6
“不会真有料吧?!”一些蛊师看到这里,都惊奇地瞪大了双眼。

%7
“难说得很。”一些人语气不太确定了。

%8
“我感觉是有,好像真的有。”有人小声地叫着。

%9
黄球泥土渐渐被磨小,快要接近巴掌大小的时候。一个人忽然闯进了帐篷:“小子,悠着点。这土球我贾金生买了!”

%10
方源手中动作顿止,一时间,帐篷中的蛊师都将目光集中到此人身上。

%11
此人外貌年轻,大约有二十岁至二十五岁之间。身穿一件金色长袍,腰间系着丝绸腰带,腰带中间镶着方形玉片。玉片中有着一条横状的玉纹,形成罕见的“一”字。

%12
很显然这是位一转蛊师。

%13
二十多岁了,还是一转蛊师,看来资质并不怎么样。

%14
但是此人地位却有些特殊。见到此人,帐篷中的蛊师都躬身行礼,齐声道:“属下见过二公子。”

%15
“二公子?”

%16
“他刚刚又自称贾金生,莫非就是商队领袖贾富的那个同父异母的弟弟……”

%17
“这么说,这家赌石场应该就是他开的了。不过他现在冒然出来干涉,似乎是坏了赌场的规矩啊。”

%18
蛊师小声地议论起来。

%19
“不错,我就是这家商铺的掌柜。小弟弟,这么小年纪就出来赌石啊。不怕你家里人追究责骂吗?我现在出四十块元石,买你手中的土球。你看怎么样?四十块元石已经不少了,里面未必会有蛊虫,只是本公子今天心情好,念你第一次赌石,不想你血本无归,算是给你回点本钱。”贾金生快步到方源的面前道。

%20
“四十块元石?”方源微微扬起眉头,斜看了贾金生一眼,冷笑道,“看来你想要强买我手中的泥球化石了?强买是要坏赌场规矩的。而且还是在青茅山上,你是想当众欺负我这个姓古月的?”

%21
“嗯?”听到方源最后这话,在场的其他蛊师站不住了,不禁都升起同仇敌忾的心意,纷纷向方源的地方拥来。看向贾金生的目光也变得不善。

%22
贾金生原以为方源这样的十五岁少年,比较容易对付,三言两语就能撬动他的心。没有料到方源手段这般了得,一句话举轻若重,就将他陷入到不利的局面下!

%23
看到蛊师们纷纷拥上来的架势,贾金生顿时脸色一变,改了口风,急忙摆手道:“小兄弟,你误会了!我是赌石场的掌柜,怎么能自己拆自己的台子,坏自己的规矩呢?那我以后还做不做生意了?呵呵呵。只是看你的土球有趣,想买下来罢了。你要是不卖,那就算了。不过待会要是没有料,可不要怪我事先没有提醒你。”

%24
方源不再理他,转过头,又专心地摩挲手中的泥球。

%25
他的动作很缓慢,很细致。常常片刻之后,才有一丝丝一缕缕的干燥泥粉洒落下来。

%26
随着他的动作,一只沉眠的蛊虫逐步展现在众人的眼前。

%27
“我的老天,真有蛊虫啊!”

%28
“真开了一只蛊!!”

%29
“有没有搞错,这样赌石都能行?”

%30
“这少年运气要爆了,居然真被他硬生生用运气撞出一只蛊来。”

%31
一时间,蛊师中惊叹声迭传。

%32
女蛊师下意识地捂着嘴,难以置信地看着这一幕。

%33
她成为店员,一路辗转了许多山寨,遇到过形形色色的人,许许多多的顾客,但从来没有见过这么戏剧性的一幕。

%34
“果然真有蛊虫!”贾金生双眼闪过一抹寒光,心中暗恨不已。他心胸狭窄,善嫉好妒。最喜欢干的事,是讨别人便宜。最痛恨讨厌的事,是被人占便宜。

%35
他开了这家赌石场,在里面布置了严密的眼线。一旦有客人似乎要开出蛊虫,他得到消息就会出现,一般都会强买下来。

%36
现在方源就在他的赌场中,在他的眼皮子底下,开出一头蛊虫。贾金生感到自己的心在滴血。

%37
开出来的,是一只蟾蛊。

%38
它浑身黄不拉几,肚皮淡黄,背部褐黄,疙疙瘩瘩,长满了蟾蜍特有的疣粒。乍一眼看上去,有些渗人。

%39
它并不大,只有巴掌大小。托在掌心中,如同托着两三颗鸡蛋。

%40
方源在各种惊叹、羡慕、嫉妒的目光中,面色平静,小心调动真元,注入到蛤蟆的体内。

%41
顿时,这只蛤蟆就被他炼化。

%42
解开化石得到的蛊虫,都是极其衰弱的。不仅全身的力量所剩无几,而且意志也混混沌沌,没有反抗能力,能被蛊师轻松的炼化。

%43
蟾蛊被方源从沉睡中唤醒,它慢腾腾地睁开双眼,肚皮微微一鼓,轻轻地叫了一声。

%44
呱。

%45
这声音虽然轻微虚弱,但是却让在场的其他人的脸色神情,陡然间变得十分精彩。

%46
一只活蛊和一只死蛊之间的价值差距,是相当巨大的。

%47
“是活蛊,开出了活蛊了!!”有人揉擦着眼睛,不敢相信。

%48
“这是癞土蛤蟆啊,该死的,真是癞土蛤蟆啊。”有人认出了蟾蛊的身份,激动地吼叫起来。

%49
“这少年撞大运了。我怎么就没这运气!”有人叹气,情绪复杂,包含着羡慕嫉妒恨。

%50
“公子,真是太恭喜你了。这,这,这是我至今为止,看到的最珍贵的蛊虫了!”女蛊师激动得有些语文伦次,双眼熠熠生辉。

%51
“竟然是癞土蛤蟆!这可是稀有的二转蛊虫,足足价值五百块元石啊。该死的,该死的。竟然在我的店铺中,被人开出了这样的蛊虫。我亏大了,亏大了!”贾金生面色苍白,死死地瞪着蛤蟆,心中涌起一股强烈的冲动,想要把它抢过来。

%52
但是他知道不能,若真这样做了,那就是自找死路。

%53
这可不是本家的寨子,而是外地,古月一族的地盘。

%54
“也许我应该多出几十块元石,兴许就能让他转让给我。不错,他不过只是个学员。我出到一百块元石,不怕他不心动。我怎么就没有这么做呢?”贾金生胸中充满了懊恼。

%55
“不,也许这小子并不识货。虽然开出了这只癞土蛤蟆,但我应该能压住价格,收购下来!”贾金生心中浮现出一丝新的希望。

%56
但是下一刻,这丝希望就被方源的一席话,给无情的击碎了。

%57
方源淡然凝望着手中托着的癞土蛤蟆,不管旁人如何惊呼艳羡。

%58
他以一种平静地语气,对贾金生道:“癞土蛤蟆,二转蛊虫,每顿食用一斤黄泥,黄泥越肥沃越佳。它数量稀少,是炼成宝气黄铜蟾的必须主蛊。市价五百块元石。贾金生,你要收购么?”

%59
“你,竟然知道的这么清楚……”贾金生嘴皮子哆嗦着,被这样一打击,一时间差点说不出话来。

%60
方源轻笑一声,继续道:“你若不愿意,那就算了。我卖给其他人好了,相信会有人买的。

%61
“慢着,慢着。我收购的,收购的。只是这价钱能不能便宜点?”贾金生的脸上浮现出苦涩的笑容。

%62
方源转身就走。

%63
贾金生连忙追上去:“别!别走啊。我买,我买了!”

%64
方源并没有计划培养这只癞土蛤蟆。

%65
它是二转蛊虫,目前方源才只是一转初阶。虽是吃的黄泥,但是青茅山上到处都是青土,弄到食物很是麻烦。

%66
再者,若是不卖这只蟾蛊,方源就得以一人之力同时喂养三只蛊虫。元石消耗增大不说,他目前手中的元石,也根本就不够喂养。

%67
所以方源的目标,一直就是出售癞土蛤蟆,得到五百块元石,赚上一笔。

%68
五百块元石,对于方源这样的一转初阶的蛊师来讲,已经算得上一大笔款子了。

%69
交易很快就完成,方源当众将癞土蛤蟆转给贾金生,同时收下五个沉甸甸的钱袋。每个钱袋里面都装有整整一百块的元石。

%70
方源原先财产是九十八块,赌石耗费了六十块,剩下三十八块。这样一来,财产瞬间翻了几番,拥有五百三十八块元石!

%71
许多蛊师亲眼看着这一幕的发生,双眼都红了。

%72
方源将五个钱袋都揣入怀中,拿起最后一块紫金化石,悠然走出了帐篷。

%73
“公子,您这块元石不解了吗?”女蛊师连连眨眼,望着方源的背影,大声提醒道。

%74
方源充耳不闻,头也不回地离开了赌石场。

%75
留下一众错愕的蛊师,相视默然。

\end{this_body}

