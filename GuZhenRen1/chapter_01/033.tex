\newsection{你骂吧}    %第三十三节:你骂吧

\begin{this_body}

“嗯?”漠颜眉头一皱,马脸上怒气顿现,她终于明白自己被方源耍了一通。

“呵,真是狗胆包天,连你家的姑奶奶都敢骗!”说着,她伸出右手,就要进来抓拿方源。

方源一步不退,昂首哈哈一笑:“漠颜你可要想清楚了!”

漠颜动作顿止,她还驻足在门外,伸出的右手悬停在半空中,脸上却闪过一丝犹豫和恼怒。

家族中有相关的明文规定。学员在学堂中是受到保护的,任何人不得擅自闯入宿舍,擒拿学员。漠颜今天只想稍微教训一下方源,让他知道尝尝苦头。绝不想担当违背族规的风险。

“单单我自己违背了族规,也还罢了。就怕连累到家里,牵连爷爷。”想到这里,漠颜恨恨地收回了手。

她看向屋内的方源,双眼瞪圆。目光中的愤怒,若是化成火焰,能将方源顷刻之间烧成灰烬。

“我可从未骗你。我说带你找到方源,现在你已经找到了。看来,你有话对我说了。”方源背负双手,微微带笑,丝毫不惧二转蛊师的威势,凛然不惧地迎上漠颜愤怒的目光。

他和漠颜的距离,只有一步之远。一个人在屋内,一个人在屋外。

但就是这样的距离,却成了天堑般遥远。

“呵呵呵,方源啊方源,你倒是对族规研究得很透彻嘛。”漠颜压住怒火,满脸都是森冷的笑,“可惜你靠着族规,也只能暂且拖延罢了。你不可能不出宿舍,姑奶奶倒要看看你究竟能躲到什么时候!”

方源朗笑一声,不屑地看向漠颜:“那我更要看看,你能堵我到什么时候。啊,已经晚上了,你有床可以睡,你呢?若是第二天的课堂上没有我的身影,学堂家老追查下来,你觉得我会怎么说?”

“你!”漠颜勃然大怒,手指着方源,蠢蠢欲动,“你真以为我不敢进来拿你?”

吱呀。

方源将宿舍房门完全敞开,他的嘴角勾勒出一抹笑意,双眼幽幽如潭,语气充满了一种局势一手掌握的笃定,对漠颜似挑衅似坦言地道:“那你动手罢。”

“呵呵呵……”漠颜反而冷静下来,她眯着眼睛,看向方源,“你以为我会中你的激将法?”

方源耸耸肩,他早已经看破了漠颜的心性。

他若是将房门关上,或者半遮半掩,漠颜至少有一半的可能会强行闯进来。但当他故意将房门彻底打开时,反而会让漠颜冷静下来,更加顾忌,强闯的可能性已经微乎其微了。

五百年的经历,早已经让他洞悉人性之弱点。

他堂而皇之地转过身,将后背完全暴露在漠颜的面前。此时若漠颜来抓他,势必极可能一击得手。但是漠颜却站在门外没有动弹,好像前方有一座无形的大山阻碍着她。

直到方源盘坐上床榻,漠颜都一直愤愤地看着,咬牙切齿,始终没有动手。

“这就是人的可悲之处啊。”方源趺坐着,看着门外傻傻站着的漠颜,心中感慨,“有的时候,阻止人行动的,往往不是物质上的难题,而是心灵的枷锁。”

论修为,方源此时绝不是漠颜的对手。但是她空有二转修为,却只能眼睁睁地看着方源,而不敢动手。

她距离方源只有几步之远,房门洞开,毫无阻碍。真正牵绊她手脚的,是她自己。

“人千方百计的学习,来认知世界,知晓规则,就是要利用规则。若是被规则牵绊,反而因为自身所学而束手束脚,这才是真正的悲剧。”方源最后看了漠颜一眼,然后缓缓合上眼皮,将心神沉入到元海中去了。

“这个方源居然当着我的面,进行修炼!简直是丝毫不把我放在眼里!!”看到这一幕,漠颜气得胸口一阵烦躁,差点要吐血。

她恨不得直冲上去,立刻就给方源几拳!

但是她知道自己不能。

漠颜忽然感到有一丝后悔,她站在门外,有一种骑虎难下的尴尬。

她并不甘心就此罢手,但是面子上又实在说不过去。自己劳师动众,要来教训方源,结果自己却闹得个灰头土脸的。

尤其是还有个家奴看着自己。

“真是该死!方源这个臭小子实在是太不配合,太奸诈了!”漠颜愤愤地想了想,开始在门外喝斥痛骂,企图将方源激出来。

“方源,你这个小兔崽子,有种的就出来!”

“方源,你一个男子汉,就要敢作敢当。现在你居然当了缩头乌龟,真是令人不齿!”

“你别给姑奶奶我装模作样,识相点快给我滚出来。”

“你个孬种,窝囊废物!!”

方源充耳不闻,根本就没有一丝回应。

漠颜骂了一会儿,却毫无往常骂人的舒爽快感,反而越来越憋闷。

她越来越感觉自己就像是个小丑,或者泼妇,堵在门口的行径,实在有些丢份儿。

“啊啊啊,真是气死我了!”漠颜抓狂了,终于放弃了激将法。

“方源,你躲得了初一,躲不过十五。”她恨恨地一跺脚,不甘心地走了。临走前又抛下一个命令,“高碗,你给我站这里,死死的盯着!姑奶奶我就不信他不出来。”

“小的遵命!”壮汉家奴高碗连忙答应着,目光恭送漠颜离开。心中却暗暗叫苦,这山间的夜晚潮湿寒冷,自己却要通宵苦守,一不小心就要染上风寒。这可是个苦差事。

哗哗哗……

元海中潮起潮落,波涛生灭。

青铜色的真元,汇聚如水,形成一波波的浪潮,在方源的意念调动下,连绵不绝地向四周窍壁冲刷过去。

一转初阶蛊师的空窍周壁,都是白色光膜。此时在青铜真元的冲击之下,显出斑斓光影,生出难以言述的一股妙韵。

时间在一点一滴的流逝,青铜元海的水面也在缓缓地下降。

从原先的四成四,降至一成二。

“蛊师要提升境界,就得不断地消耗真元,来温养空窍。初阶蛊师的窍壁是光膜,中阶蛊师的窍壁是水膜,高阶是石膜。我若要从初阶晋升到中阶,就得将光膜窍壁冲刷温养成水膜窍壁。”

前世五百年的记忆,让方源对此时期的修行掌握得了如指掌,洞若观火。

他缓缓睁开双眼。

只见此时,已经是深夜。

月牙高高地悬挂在夜空,洒下月辉洁白如水。

门口敞开着,月光照进来,让方源联想到地球上的一句著名的诗――床前明月光,疑是地上霜。

夜风徐动,吹来一股寒凉之意。

方源可没有保温功能的蛊虫,单凭十五岁的少年身子骨,也忍不住抖索了一下。

山间的夜还是很凉的。

“臭小子,你终于睁开眼了。你要修炼到什么时候?!赶紧出来吧,伸头一刀缩头也是一刀。你打了我们家的漠北少爷,被我家大小姐教训是早晚的事情。”看到方源终于有了动静,一直站在门口处的豪奴高碗顿时来了精神。

方源眯了眯眼睛,看这样子,那个二转的女蛊师已经走了?

“臭小子,你听到没有?快给老子滚出来,你有房间住,有床睡,老子却得站在这院子里。你再不出来,信不信老子就闯进去?!”见方源没有反应,高碗恐吓道。

方源仍旧无动于衷。

“混蛋东西,快滚出来束手就擒。你得罪了我们漠之一脉,今后不会有好日子过了。赶紧向大小姐磕头认错,兴许还有原谅你的可能。”高碗继续斥骂道。

方源充耳未闻,他从钱袋子里施施然取出一枚元石,握在手心。又缓缓闭上了双眼。

高碗见方源还要继续修行,顿时急了,破口大骂:“你个区区丙等的资质,充其量不过是个二转蛊师,有什么好修炼的。单凭你一个人,怎么能和我们整个的漠之分家抗衡?小子,你耳朵聋了?听到大爷我说话没有?!”

------------

\end{this_body}

