\newsection{战力大涨}    %第八十节:战力大涨

\begin{this_body}

%1
秋风萧瑟,红叶飘零。

%2
野草枯黄,树枝上挂着通红或者橙黄的颗颗野果。

%3
“哼吼!”黑色的野猪,身上的鬃毛直竖起来,四蹄在地上狂奔。

%4
山上的地面,积了一层厚厚的落叶。

%5
野猪冲撞而来,夹裹着一股风,落叶就在它身后飞舞。

%6
方源静静地站立着,看着野猪越来越近,脸上一片冷酷。

%7
杀!

%8
他忽然向前一大步跨出,然后双腿轮番迈步,竟然不闪不躲,悍然冲向野猪。

%9
野猪的两颗雪白獠牙,刺破空气,杀机腾腾。

%10
方源侧身让过獠牙,沉肩猛撞野猪的脑袋。

%11
在即将相撞的时候,方源肩膀处闪过一道淡绿色的玉光。

%12
玉皮蛊!

%13
砰。

%14
一声闷响,两者狠狠地撞在一起。

%15
方源连续后退三步,野猪则后退一步。

%16
真算起来,双方的力量,还是方源大一些。但是方源两条腿奔驰,不及野猪四蹄的支撑力道,同时野猪的重心比方源要低沉稳重得多。

%17
不过野猪被方源狠狠地撞上头部,虽然仍旧站着,但是肥厚的身躯已经在摇晃。

%18
方源喝了一声,再次冲了上去,左手摁住野猪的獠牙,右拳高高举起,淡绿色的玉光形成一个薄薄的光膜,笼罩在拳头表面。

%19
砰。

%20
拳头狠狠砸下,野猪立即发生一声痛叫,开始极力挣扎。

%21
方源左臂肌肉鼓动,青筋暴起,如一根根蚯蚓盘绕,死死地制住野猪。

%22
同时,他的右拳不断地高高举起,然后重重落下。

%23
砰砰砰。

%24
拳头每击打在野猪的头部,拳头上翠绿的玉光就爆闪一次。

%25
野猪被拳头砸得七荤八素,挣扎的力道越来越小。

%26
“最后一击!”方源双眼中电芒一闪,他舒展开上身,右臂伸直,举到最高处,然后屈起右肘,轰然砸下。

%27
绿色的玉光紧紧地贴在方源的右肘上,随着方源的动作,在半空中划出一片绿影。

%28
砰!

%29
方源单膝跪地,手肘凶狠地砸碎了野猪的头骨。野猪连最后一声惨叫都没有来得及发出,骤然静止下来。

%30
一颗猪头完全变形,断裂的白色头骨刺破黑皮,裸露出来。鲜血和脑浆缓缓流淌而出,在凋零的层层落叶上,渲染出一片绚烂的猩红之色。

%31
一阵萧瑟的秋风吹来。

%32
卷起几片落叶,吹散猪血的热气。

%33
“生,当如夏花之绚烂。死,当如秋叶之精美。”方源口中喃喃,欣赏着这幅画面。

%34
生者浓烈灿烂,死者凄冷静寂。

%35
一生一死之间,充满了多么强烈的对比,彰显了自然的残酷,以及生命的精彩。

%36
“不管哪个世界,胜利者灿烂绚丽,而失败者落灰败凄冷。胜败之别,对我而言就是生死之别。因为行走魔道,一旦失败,往往就是死亡。”

%37
方源直接靠着猪尸,盘坐在地。他放出白豕蛊,让它吞食猪肉,又分出心神沉入到体内空窍。

%38
空窍中墨绿色的青铜元海,潮生潮落,波涛生灭。

%39
元海圆满时,占据四成四的体积。刚刚一次激战,方源多次利用玉皮蛊,增加自己防御,因此真元损耗,如今真元剩着三成六分。

%40
算一算的话,只消耗了八分真元,连一成都不到。但因为这是一转巅峰的墨绿真元,已经消耗很大了。

%41
一转初阶是翠绿真元。

%42
一转中阶是苍绿真元。

%43
一转高阶是深绿真元。

%44
一转巅峰是墨绿真元。

%45
浓缩就是精华。

%46
月光蛊催发一次,需要动用一成的翠绿真元,换算成苍绿真元,只需要半成。深绿真元再缩减一倍,墨绿真元亦是如此。

%47
也就是说,一成的墨绿真元,相当于两成深绿,四成苍绿,八成翠绿。

%48
使用玉皮蛊消耗了八分墨绿真元,换算成一转初阶的翠绿真元,就高达六成四分!

%49
若是方源还在初阶,整个空窍只有四成四的真元,使用到一半,空窍中的元海就彻底消耗光了。

%50
“蛊师修为越高,战斗力越强,就体现在真元上面。越高阶,真元的颜色越深,越是耐用。不过我这的墨绿真元,是在高阶真元的基础上,被酒虫精炼成的。不像方正,如今已经晋升到一转巅峰了。”想到这里,方源目光闪了闪。

%51
时间匆匆,如今已是深秋。

%52
距离王大刺杀,已经过去了两个多月了。

%53
方正中了毒,昏睡了七天七夜,醒来之后,就像是变了一个人。变得十分努力,修行起来极其刻苦。

%54
有人说,生活每一次磨难,都是一笔金子般的财富。

%55
不管这话是否正确,方正的确从这次磨难中,汲取到了很多的东西。他就像是一个璞玉,经过打磨,已经露出里面的华美玉质。

%56
他是晋升高阶的第一位学员,不久前,又第一个晋升到一转巅峰,再一次将同龄人甩在身后。甲等资质的光彩,在他身上开始绽放。

%57
“我距离巅峰也不远了,最迟十天半个月吧。说起来我每天也从未停止过温养空窍,只是丙等资质实在没有办法和甲等、乙等相提并论。同时,还有另一个原因……”方源想到这里,咧开了嘴,发出无声的苦笑。

%58
他每隔一段时间,都要猎杀玉眼石猴,给玉皮蛊喂食。同时,他还要在石林中不断探寻,以求得花酒行者下一步力量传承的线索。

%59
石林地形相当复杂,一根根巨石从顶壁垂直下来。方源稍不留意,就会距离某根石柱过近,从而引发玉眼石猴群的攻击。

%60
有好几次,方源都被数十只的玉眼石猴追得逃窜。最危险的一次,是在撤退的过程中,又踏入另一根石柱的警戒线内,引发了两拨石猴,足有上百只的追杀。

%61
幸亏这些玉眼石猴,有一种宅的本性,每次追杀出来,都不会追得太远。追了一段距离之后,往往就会回归到家中,继续宅起来。

%62
饶是如此,也有数次让方源处在生死关头。关键时候,玉皮蛊起了很好的防御作用。

%63
如此探索,不得不让方源投入大量的时间和精力,这就是他修为进展较为缓慢的最大原因了。

%64
“不过即便如此,也比我前世要好太多了。还有石林探索,也不是没有成果。至少已经知道,四周一圈的石壁,都是毫无问题的。初步估计,力量传承的下一步线索,应该在石林当中的某个地方。”

%65
方源正沉思着,这时一个黑影踏过枯枝,接近过来。

%66
这是一头流浪的老狼。

%67
它有一身棕黄色的毛皮,跛着脚,脸上一只眼睛也残了,仅剩下的左眼散发着残忍和机警的绿芒。

%68
它紧紧地盯着方源,鼻子抽动着。狼和狗一样,向来嗅觉敏锐,它应该是被猪血腥味吸引过来的。

%69
狼一般都是群体性的动物,但是也有这样流浪的残狼。狼群内部也有竞争,为了保持整个狼群的活力,都会剔除一些战斗力低下的残疾老狼。

%70
方源迅速地站了起来,静静地看着这头老狼。

%71
以前他杀死一头野猪,体内的真元也就所剩无几,战斗力大幅度下降,遇到搅局的野兽,都会选择避让。

%72
但是这些个月,他的战斗力飞速提升,如今又有玉皮蛊在手上,对付一头残狼,绰绰有余得很。

%73
漫山树叶红遍。

%74
夕阳晚照。

%75
一人一狼相距五十步,静静地对视着。

%76
狼眼中,绿芒闪烁,流露出一股残忍而又狡诈的意味。而方源的双眼,则一片深沉,漆黑的眸子,透着冷意。

%77
白豕蛊钻了出来,它吃饱了,心满意足地飞回了方源的空窍当中。

%78
老狼看了一眼地上的野猪,只剩下骨头和猪皮,肉几乎已经被白豕蛊吃光了。

%79
狼眼中绿色的光芒缩了缩,它先是后退几步,然后掉头钻入了树丛当中。

%80
这头老狼能存活至今,自然也有一定的智慧,它敏锐地察觉出方源的可怕,谨慎地选择了退走。

%81
它来的突然,去也匆匆。

%82
没有野猪奔驰冲撞的闹腾,没有老虎咧嘴低吼的嘈杂。

%83
和方源的对决,在无声中开始,又在沉默中结束。

%84
“生死存亡的主题下,大自然不知孕育了多少的精彩。”方源站在原地,没有去追杀。这头老狼的身上,没有他值得出手的地方。

%85
嗷呜!

%86
不过就在下一刻,忽然传来老狼的惨嚎声。

%87
狼嚎声忽然爆发,又瞬间戛然而止,如此之间,流露出一股浓郁的死亡气息。

%88
咔嚓咔嚓。

%89
树丛那边,很快传来一阵枯树枝被踩断的声音。

%90
声音肆无忌惮,越来越近,方源瞳孔缩了缩。

%91
“能顷刻之间,就解决了那头狡猾的老狼……”他的目光越发清冷了。

\end{this_body}

