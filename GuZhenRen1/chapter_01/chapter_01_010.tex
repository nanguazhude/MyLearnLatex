\newsection{天有不测风云,炼蛊别具艰辛}    %第十节:天有不测风云,炼蛊别具艰辛

\begin{this_body}

%1
啪啪啪……

%2
豆大的雨点,密集地下着。打在苍翠的竹楼上,发出脆响。

%3
楼前池塘中,水面被雨点激打着,鱼儿在水里快活地穿梭着,水草在池底摇曳。

%4
阴云密布,整个视野都被一层浓密厚重的雨帘遮挡起来。

%5
在有些昏暗的房间里,窗户开着,方源静静地看着这场大雨,心中幽幽一叹:“已经三天三夜了。”

%6
三天前的那晚,他拎着两坛酒,出了山寨,在周边继续探索。

%7
但到了深夜就下起了瓢泼大雨。

%8
被淋成落汤鸡且不去谈他,关键是如此情况下,就再也不能四处探索了。

%9
雨水能将酒气迅速地冲刷掉,同时若顶着雨强行探索,只怕会引起怀疑。

%10
先前是伪装成失意酗酒的样子,来遮掩真正的动机。但千万不要低估旁人的智慧,往往只有蠢材才会认为别人愚蠢。

%11
所以无奈之下,方源就只能终止了探索。

%12
而这场雨下起来,就一直持续着,期间或大或小,或稀或密,却没有停止过。

%13
“这样一来,酒虫短时间之内就寻不到了。保险起见,只能先着手炼化月光蛊。在这炼化的过程中,能寻到酒虫最好,得不到也只好这样将就了。不过此事也很平常,天有不测之风云。这世间哪有人做事能一直顺风顺水,尽善尽美的?”方源心境很平稳,五百年的经历,早就洗去了他性情中本来就少的浮躁。

%14
他关上窗户和门,趺坐到床榻之上。缓缓闭上双眼,呼吸调匀后,便将心神一沉。

%15
下一刻,脑海中就展现出自身空窍的景象。

%16
空窍虽然寄托在体内,但是玄妙异常,无限大,又无限小。

%17
空窍外侧,是一层光膜。白色的光膜给人很纤薄的感觉,但是却实实在在地支撑着空窍。

%18
空窍中,是一片真元海洋。

%19
海水呈青铜色,海面平静如镜,水位差不多是空窍的一半高。整个元海体积,占据空窍的四成四。

%20
这就是一转蛊师的青铜元海,每一滴海水,都是真元。是方源的生命元力,是方源的精气神的凝结。

%21
每一滴真元都是宝贵的,因为它是一个蛊师的根本,是力量之源。就是靠着真元,蛊师才能炼化和催动蛊虫。

%22
心神从元海中退出来,方源睁开双眼,从怀中取出那只月光蛊。

%23
月光蛊静静地躺在方源的掌心中,就好像一个弯弯的蓝色月亮,小巧玲珑,晶莹剔透。

%24
方源心念一动,顿时空窍中的元海翻滚起来,一股真元水流冲破海面,调动到体外,尽数涌入到月光蛊之中。

%25
月光蛊猛地绽放出幽蓝的光辉,在方源的手掌中微微地颤动,抗拒着方源真元的涌入。

%26
蛊是天地之精华,大道之密码,法则之载体。它也是生灵,天生自由自在,存在着本身的意志。现在方源要炼化它,就是要抹去它的意志,感应到这股危机,月光蛊自然要反抗。

%27
炼化的过程十分艰难。

%28
月光蛊就像是一片弯弯的月牙,青铜色的真元注入到月牙当中,首先就将月牙两个尖端染绿。

%29
随后这股青铜绿意,开始向月牙中段蔓延。

%30
不到三分钟,方源的脸上就呈现出一抹苍白。大量的真元不断地涌入到月光蛊中,导致一股股抽经伐髓的虚弱感,连绵不绝地向他心头袭来。

%31
一分、两分、三分……八方、九分、一成。

%32
十分钟后,方源元海就消耗掉了整整一成真元。但是蓝水晶一样的月光蛊表面,处在月牙尖端的两点青铜绿意,却只向中段扩张了一丁点的面积。

%33
月光蛊的反抗力,十分顽强。

%34
好在方源对此早有预料,也不意外,坚持向月光蛊中灌输真元。

%35
一成,两成,三成。

%36
又过了二十分钟,方源体内的元海只剩下一成四分,月光蛊上青铜绿意扩张了一丝,两片绿意若叠加起来算,大约有整个月光蛊表面的十二分之一。

%37
至于其余部分,还是月光蛊本身的淡蓝之色。

%38
“炼蛊艰难呐。”看到此景,方源暗叹一声,断了真元供给,不再继续炼化这月光蛊。

%39
到此刻,炼蛊整整半小时,空窍元海已经消耗了一大半,只剩下一成四的真元残留着。而月光蛊才刚刚炼化了十二分之一。

%40
而且,更叫人难以接受的是,月光蛊仍旧散发着幽蓝的光晕。方源虽然停止了炼化,但是它却没有停止反抗,仍旧在驱逐方源的青铜真元。

%41
方源可以清晰地感到,汇入到月光蛊中的真元,正在一点点被月光蛊驱除出去,逸散到体外。月光蛊表面,两个月牙尖端的青铜绿意也在慢慢地缩小。

%42
照这减少的速度估算,大约六个小时之后,月光蛊就能将方源的真元全部驱除。到那时再来炼化这蛊虫,和重头祭炼没有分别。

%43
“每一次炼蛊,就像是两军交战,打阵地战,消耗战。蛊虫只祭炼了十二分之一,而我的真元却消耗了整整三成。蛊师炼蛊就必须一边补充元海真元,一边持续不断地祭炼,巩固胜利成果。炼蛊考较的不仅是调动真元的技巧,还有持久战的耐心。”

%44
方源一边想着,一边从钱袋中取出一块元石。

%45
蛊师要补充消耗掉的真元,通常有两种方法。

%46
一种是自然恢复。每过一段时间,元海就会自动补充真元。像方源这种丙等资质,大约一个小时补充四分真元。六个小时,就能恢复二成四分的真元总量。

%47
第二种方法,就是汲取元石中的自然元力。

%48
元石是大自然的瑰宝,凝聚着天然真元,可以被蛊师吸收。

%49
方源手握元石,从里面源源不断地汲取出天然真元,汇入到自己的空窍元海之中。

%50
元石表面的细腻光华,慢慢地暗淡下去,方源的元海水位却以肉眼可见的速度,不断攀升。

%51
大约半个小时之后,元海重新恢复到四成四的体积。到了这程度,海面增高的趋势,就戛然而止。虽然空窍中还有空间,但是方源却再也存储不了更多真元。这就是他丙等资质的局限了。

%52
由此,就可体现出修行资质的重要作用了。

%53
资质越高,空窍中存储的真元就越多,并且真元自然恢复的速度也越快。

%54
对于方源来讲,要炼化蛊虫,巩固成果,必须得吸收元石,因为他真元自然恢复的速度,比不上月光蛊驱逐真元的速度。

%55
但是对于甲等资质的方正来讲,他每小时能补充八分真元。六个小时,就能恢复四成八分的真元总量,而月光蛊在同样的六个小时内,只能驱逐三成真元。

%56
他甚至不需要元石这样的外力帮助,就这样一直炼化,期间休息几次,过个几天,就能将月光蛊成功炼化。

%57
所以,方源从一开始就知道,在这场炼化月光蛊的考核中,自己根本就没可能夺得第一。这无关实力,而是资质才是此中的第一因素。

%58
第二因素则是元石。

%59
若是元石充沛,不惜损耗,乙等资质的人,也可能超越甲等,夺得第一。

%60
“我手中只有六块元石,比不上古月漠北,古月赤城这种背后有长辈支持的人。我资质只有丙等,也比不上甲等资质的古月方正。这场考核本来就没有一丁点胜算,还不如分散精力,去寻找酒虫。若是能将酒虫炼成本命蛊,可比月光蛊要好多了。嗯?窗外的雨声小了,似乎有停息的迹象。这场大雨已经连下了三天三夜,也该停了。”

%61
方源收起月光蛊,下了床榻,正要打开窗,此时却有敲门声响起。

%62
门外传来贴身丫鬟沈翠的声音:“方源少爷,是奴婢。这三天雨连下着,奴婢这就给您带来了一些酒菜,少爷吃喝一些,也能解些乏闷。”

%63
(ps:感谢粉丝789、蓝浪魔、颓废wudi、甜甜圈现象、q62552217几位同学的打赏。感谢粉丝789、蓝浪魔等同学们的满分评价票。感谢一直投着推荐票的朋友们。人气在回归,老朋友们也越来越多,我们的力量在慢慢回复。新书就像婴儿、幼草,需要点击、收藏和推荐票的灌溉,这样才能茁壮成长。请诸君努力灌溉和支持啊!)

\end{this_body}

