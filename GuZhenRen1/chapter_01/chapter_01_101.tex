\newsection{心老人便老}    %第一百零一节:心老人便老

\begin{this_body}

%1
依古月冻土的能量,自然查得到方源的住处。

%2
他今天罕见地穿起了蛊师的武服,腿脚上有绑腿,腰间系着赤色的腰带。一身整洁干净,透着淡淡的威仪。

%3
他看着方源慢慢走来,视线往少年的腰带上瞥了一眼,心中自然生出一阵感慨。

%4
“只是丙等资质,就已经在十六岁突破到二转,真是出人意料。真不知道他是如何冲击成功的。不过……这样快的速度,大部分也归功于酒虫吧。可惜酒虫到了二转,就不能使用了。”

%5
紧接着,又看到方源嘴角挂着的淡淡微笑。

%6
就是这笑意,却让古月冻土心中一寒。

%7
然后,一股薄怒涌上心头:“这个小子,如此从容,是以为吃定我了吗?!”

%8
方源缓缓走来,停在古月冻土的面前,他知道对方必是来找自己的。

%9
果然,舅父古月冻土开口道:“方源,我想我们可以谈一谈。”

%10
“谈什么?”方源微微一挑眉头。

%11
古月冻土笑了笑,却是换了一个话题:“你知道么,我和你不同,我十五岁就出道了。”

%12
“那时候,正值狼潮,蛊师大量死伤,不得已的情况下,我们这些学员必须要顶上去。我是乙等资质,在十六岁到达二转初阶,十七岁高阶,十八岁巅峰。十九岁时,我已经开始冲击三转。那个时候,我认为自己能在二十岁,晋升到三转蛊师。”

%13
“呵呵呵,我太轻狂了,目中无人,总以为自己能做到一切。甚至无所不能,根本不知道天高地厚。就在二十岁那年。我外出执行任务,被一位熊家寨的蛊师击败。当时重伤濒死,亏得药堂家老亲自出手,才捡回一条命。但自那之后,我的资质废了,落到了丙等,我一蹶不振了整整八年。”

%14
“在我二十九的生日那天,我开始重新审视自己以及这个世界。我发现,一个人的力量终究是有限的。纵然我晋升三转又如何呢?人生活在这个社会上,最重要的不是个人实力。而是与其他人的交际关系。”

%15
“我从三十岁开始重新起步。四十五时退居二线。期间数十次,被百位蛊师联名上书,要推举我为家老。我虽然只有二转巅峰的修为,终身也不能跨出那一步,但是也没有必要了。我已经取得了成功。族人称呼我为‘隐家老’。绝大多数的同龄人都已经死去,而我却活得很安稳。对于那些在外打拼的蛊师,我的影响力仍旧存在。”

%16
长篇大论到了这里,古月冻土这才转回正题,他看着方源,嘴角弯成一个弧度:“方源啊,你还是太年轻了,初出茅庐,就像是当年的我一样。以为什么都可以自己做主,什么事情自己都可以完成。呵呵。”

%17
古月冻土摇摇头,继续道:“但是当你的人生阅历再丰富一些,你就会明白,人终究是社会的一员,不是独行的野兽。有时候人生是需要低头和让步的。偏激和极端,桀骜和张扬,只会引来孤立和毁灭。我相信你现在已经感觉到了吧,没有一个小组肯接纳你,你周围的人都排斥你。你就算是接到了分家任务,又能怎样?你已经是孤家寡人一个,绝对没有机会完成任务的。你放弃吧。”

%18
方源淡淡地看着面前的这个中年男人,脸色一片平静。

%19
“如果他知道自己有着五百年的人生经历,不知道又会是什么表情呢?”

%20
想到这里,方源的眼中就不禁有些流露出一丝笑意。

%21
事实上,舅父的想法,曾经也是他前世很长一段时间的生活理念。

%22
因此,他创建了血翼魔教,依靠制度和人情,殚精竭虑地打造出一个庞然势力,一呼万应,霸占资源,对抗强敌。

%23
然而当他突破到六转之后,他看到了截然不同的风景。

%24
蛊师中五转为凡,六转成仙。当他站在这个高度,再看人生时,他恍然而悟——庞大的势力的确是他的臂助,但同时也是一个巨大的累赘和拖累。

%25
不管是哪个世界,真正最靠得住的人只有一个,那就是自己。

%26
只是这世人,常常软弱。总耐不住寂寞,总受不了孤独,喜欢追寻亲情、友情、爱情,来填充自己的心灵。迷恋集体,害怕独处。

%27
一旦受到挫折,就躲到集体当中去,向亲朋诉苦,向好友倾诉,不敢孤独地面对恐惧和失败。有了痛苦就忙于分摊,有了快乐就急于炫耀。

%28
古月冻土成功吗,毫无疑问,他是成功的。

%29
他在原先的路上走不下去了,换了一条路,走出了一片天。

%30
但他同样也是一个失败者。

%31
他因为一个挫折而低头,他不过是个懦夫,却还在为他的逃避而沾沾自得。

%32
古月冻土当然不知道,方源已经把他认作了一个懦夫。他见方源一直没有说话,还以为方源已经被他的话所摄。

%33
他继续道:“方源,你不是方正,我就打开天窗说亮话。如果你能放弃你继承家产的打算,你就能获得我的友谊,我的人脉关系你都可以利用。同时,我还会补偿你一千块的元石。我知道的,你最近手头拮据,连房租都拖欠两天了吧。”

%34
方源淡然一笑,开口道:“舅父大人,这身衣服不常穿吧?”

%35
古月冻土一愣,没有料到方源会突然谈这个话题。

%36
不过方源的确说的没错,他已经退隐很长时间了,这身服装压着箱底。今日来见方源,特意穿了这身,就是为给自己增添说服力和威慑力。

%37
方源叹了一口气,打量着古月冻土的衣衫,缓缓地道:“蛊师的衣服,没有这么干净整洁的。它要沾着汗水、泥浆和鲜血,它要破烂不堪,那才有蛊师的味道。”

%38
“你已经老了。舅父大人。你的雄心壮志,早在年轻时就已经消失了。这些年安逸的生活。已经腐蚀了你的心。你争夺家产,不是为了修行,而是为了维持富足的生活。单凭这种心态,你怎么可能阻挡我?”

%39
古月冻土脸色顿时变得铁青,心中泛起一股怒气。

%40
这个世界上,总会有一群“老”人。他们四处兜售着社会的经验,把他人的理想当做幻想,把他人的热情当做轻狂,把他人的坚持当做桀骜。他们常在教训后辈中,寻找自己的存在感和优越感。

%41
毫无疑问。古月冻土就是这样的人。

%42
他想要教育方源。但是没有想到,方源不仅没有听从他,屈服他,反倒过来把他教育了一通!

%43
“方源!”他低喝道:“我身为你的长辈,好心好意地开导你。劝说你,你却这样不识好歹。哼,既然你一心想跟我作对,那就来吧。不怕告诉你,你那分家任务的内容,我早就知道了。年轻人,不知道天多高地多厚。呵呵,我倒要看看你怎么完成这个任务!”

%44
方源戏谑一笑,这时已没有隐藏之必要。反正矛盾绝不可调和,不妨欣赏一下古月冻土接下来的精彩表情。

%45
于是,他取出牛皮水囊,拔开盖子,飘出一丝蜜酒特有的香甜。

%46
“你觉得这里面装的什么?”他道。

%47
舅父大惊失色,一颗心顿时沉入最谷底。

%48
“怎么可能。你这是哪里弄来的蜜酒?!”他吼起来,脸上浮现出难以置信的惊容。

%49
方源不再理他,盖上盖子,将水囊重新揣入怀中,迈步向内务堂走去。

%50
舅父满头的冷汗,脑海中思绪剧烈翻腾。

%51
“他是哪里弄来的蜜酒?我已经关照过,只要他找上其他小组,我就会第一时间得到消息。难道他是独自完成的?不,这怎么可能,他又没有防御性的蛊虫。一定是有人帮助他。不对!现在的关键,不是找什么原因。方源这个小子已经有蜜酒了,他要交接那个任务去!”

%52
想到这里,古月冻土心中已经一片慌乱,再也没有刚刚的从容。

%53
他快步地追上方源:“方源,你等等,凡事都好商量的。”

%54
方源不语,继续走着,古月冻土也只得紧跟在他的身边。

%55
“一千块元石不行,那就两千块,不两千五百块。”古月冻土不断加价。

%56
方源充耳不闻,心中倒是对这家产生出了些期待。古月冻土如此急迫,不断加价,看来真正的家产一定很客观。

%57
古月冻土急得满头大汗,他见方源毫不动容,脸上显现狰狞,低沉地威胁道:“方源,你可要想清楚了!你得罪我会有什么下场,哼哼,将来若是缺胳膊少腿,别怪我这当舅父的无情。”

%58
方源哈哈一笑。

%59
这古月冻土真是可悲的人。被规矩束缚着,明明水囊近在咫尺,却不敢强抢。如此没有胆量,又想争夺利益,怎么可能成功?

%60
富贵险中求,不管哪个世界,想要得到什么,必定要付出代价的。

%61
“方源,你别以为拿了家产,就能万事大吉!我告诉你,你还太年轻,不懂得什么是社会,什么是险恶!”古月冻土在方源耳边低沉地怒吼着。

%62
方源摇摇头并不理他,在他的怒目瞪视中,迈入了内务堂。

%63
对于这个舅父,其实他还谈不上憎恨,甚至也没有厌恶。

%64
这样的人,他见多了,也可以理解这种人。

%65
如果方源的元石够用,能支撑修行,他甚至不会争夺这个家产。不过区区小利,给了舅父又能如何?

%66
他重生为的什么?

%67
不是为了争这一时长短,而是想迈步最强之巅峰。这种路人,只要不碍着此事,任由其在一旁咆哮,他方源踩都不屑踩。

%68
但是偏偏这古月冻土,阻碍了方源前进的脚步。

%69
既然如此,那就踩着你前进好了。

%70
“方源!方源……”舅父古月冻土眼睁睁地看着方源迈入内务堂,他浑身都颤抖着,额头上青筋暴跳。

%71
夕阳的光,照在他的双鬓,都已经泛白。

%72
他的确是老了。

%73
二十岁受伤那年,他就老了。

\end{this_body}

