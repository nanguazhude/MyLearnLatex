\newsection{血海老祖}    %第一百八十三节:血海老祖

\begin{this_body}

%1
等一等,赤土?”

%2
方源看到这里,心中猛然一动。

%3
他伸出手来,抓住身边石壁,用力一扣,就扣出了一块赤土。

%4
这赤土质地松软,散发着微微红光。方源又轻轻用手一捏,毫不费力地就将其捏碎。

%5
“原来如此。”他恍然大悟。

%6
记得第一次,他进入石缝秘洞,就发现秘洞中全是这样的赤土,散发着微光,根本不用其他照明。

%7
他当时就觉得古怪,因为青芽山的土,都是青黑色的土。他原以为这是花酒行者的布置。但现在看来,源头应该是这片诡异的血湖。

%8
方源感到越加不妥,前世五百年积累的人生经验,已经沉淀得近乎成直觉。

%9
“这地方不仅古第一百八十三节:血海老祖怪,而且危机四伏。我现在时间紧迫,该如何离开这里?”方源抬头看向洞壁,壁顶的洞口足有数百,联通着元泉的洞,究竟是哪个?

%10
方源一阵迟疑。

%11
水能流淌进来,并不代表地下河道就宽敞得能过人。

%12
“而且……”方源面色凝重,试着震动背后双翅。

%13
这雷电之翼,却不像先前那般如臂使指。原本幽蓝的电流雷光,此时掺杂着丝丝诡异的猩红之色。透露出一种虚弱和强大并存的矛盾感觉。

%14
雷翼蛊这状态很不可靠,极可能在飞行中掉链子,令方源从空中坠落。

%15
哗……,血湖中,一股暗流莫名涌动起来。

%16
来自五转蛊的庞大气息,从血湖里渗透而出。

%17
“那是…。”方源瞳孔猛缩,只见一条长长的黑影,在血水中渐渐显现。

%18
它长达四十多米,直径有六米多的粗细。

%19
这是一条巨大的蟒,栖息在血湖深处,如今闻到方源身上的血肉气味,似乎要出来狩猎!

%20
“该死的……”方源心中一阵紧迫。

%21
此刻,他长发黑袍,靠着锯齿金蜈戳穿山壁,勉强吊在松软的赤土第一百八十三节:血海老祖上。和偌大的血海相比,仿佛是一只黑色的蚂蚁。

%22
数百的黑点,也从湖水底部出现,上升,宛若鱼群游戈。

%23
嗖嗖…

%24
它们速度比巨蟒更快,须臾间就飞出湖面,显露出身形来。

%25
这却不是鱼,而是蝙蝠。

%26
这些暗红色的蝙蝠,双耳尖长,长有两对翅膀。一对主翅较大,一对副翅稍小,在主翅的下方。

%27
它们没有脚爪,但两对翅膀边缘,都是尖锐锋利,堪比钢刀。

%28
“三转刀翅血蝠蛊?”方源心中顿时升腾出一个答案。

%29
看着这些刀翅血蝠群,杀气腾腾地向自己扑来,他脑海中首先浮现的却是那段影像。

%30
在留影存声蛊的影像中,花酒行者浑身浴血,重伤濒死。

%31
月影蛊是绝不可能造成那样的伤势,但这群刀翅血蝠,却非常吻合。

%32
“难道说,花酒行者曾经来过这里?事实上,是被这里的刀翅血蝠所伤?”一时间,方源思绪电转。

%33
花酒行者的死因,一直是个谜团。但现在看来,极可能就在于此了。

%34
“刀翅血蝠……”方源心中喃喃,对于这种蛊,他其实一点都不陌生。

%35
这种蛊,虽然只是三转,但极易喂养,只需要血液即可。

%36
在他前世,他建立血翼魔教,就是以刀翅血蝠为标志。以魔教资源,足足供养了有近万头刀翅血蝠蛊,凶威赫赫,播撒恐怖更准确的说,他就是以刀翅血蝠立业的。

%37
在前世的四百多年后,他误打误撞,得到了血海老祖的一处传承。依靠刀翅血蝠群,以及自身的五转修为,成为一方霸主。

%38
这血海老祖,乃是七转魔道蛊师,恶名昭昭,杀人如麻,载于史册,遗臭万年。

%39
他原先是凡人,机缘巧合下成了魔道蛊师。一路从最低层攀登,用了八百多年,成为魔道巨擘。

%40
他的资质并不高,空窍内真元有限。对合炼蛊虫,一直都有狂热的研究兴趣。

%41
野生蛊,有天然意志,自身能汲取空气中的天然真元。但当野生蛊被蛊师炼化,身躯被人的意志主宰之后,它们就失去了汲取周围元气的能力,只能吞吸蛊师空窍内的真元。

%42
血海老祖一直想要研炼出,能够被蛊师炼化,却又能有吸纳自然元气的蛊。

%43
正道蛊师们,对此十分恐惧,极其担心血海老祖研炼成功。先后组织了数次正道包围网,来围杀他。

%44
血海老祖最终没有成功,但他也没有完全失败。

%45
他探索出了刀翅血蝠、血滴子、血狂蛊的合炼秘方。

%46
三转的刀翅血蝠蛊,是他最初的成果。极易豢养,但仍旧需要蛊师来提供真元。不过这刀翅血蝠群,乃是一个十分特殊的集群。蛊师只需要操纵其中唯一的雄蝠,就能间接地号令其他所有的雌蝠。

%47
血狂蛊是他的第二成果。此蛊无形无体,乃是一团血气,依附在其他物体上才能存活。此蛊高达四转,效用十分奇特。但凡蛊虫沾染了它的气息,便能时不时的吸收自然元气。但它却有极大弊端,受到血狂蛊影响的蛊虫,都会渐渐的不受蛊师操控。一段时间之后,就会化为一滩血水。

%48
血滴子则是血海老祖的最后成果。这蛊高达五转,比前两者无疑更加成熟。它养用合一,以战养战,吞噬蛊师鲜血来繁衍分化。已经完全不需要主人来提供真元。

%49
可惜的是,血海老祖创造出血滴子之后,终究在辗转乱战中,被正道围攻,力竭战败。

%50
他受到无法治愈的致命伤,在重重包围中血遁逃离。

%51
正道人士担心他临时反扑,危及自己,没有追击的心思,看着他逃之天天。从此之后,这些正道人士每次回想,都后悔不迭当时轻易纵敌的举动。

%52
血海老祖自知必死无疑,开始广布传承。利用临死前有限的时间,他以七转蛊师之能,竟然布置了数十万个传承密地,地点遍及中洲、南疆等地。

%53
他在死前曾怪笑:“血道不孤,遗毒万世!”

%54
他此言一点都不假,此后无数蛊师因此受益,魔道大昌。

%55
不管是刀翅血蝠蛊,还是血狂蛊,血滴子,都极容易豢养和繁衍。也许在某个不起眼的山谷,在踅脚落魄的村庄,在无人的沙漠,在山道旁,就留着血海老祖信手布置的两三只蛊。

%56
这些蛊,容易豢养,对真元需求不高,极容易被资质普通的蛊师所用。

%57
在这样的世界中生存,朝不保夕,有哪个蛊师不渴望更强大的力量呢?血海老祖留下的蛊虫,就代表着一种全新的力量。

%58
这种力量快捷方便,比其他蛊虫,无疑更受欢迎。

%59
力量本身是没有罪过的,用之善则为善,用之恶则为恶。但世间又有多少人,能有坚定的心性,能把持得住凭空暴涨的力量?

%60
就像男人有钱,常会变得花心。暴涨的力量,必滋养出先前未曾有过的野望!

%61
因此,许多蛊师得到血海老祖的传承之后,成为了大杀四方的魔头。甚至还有许多原先的正道人士,也因此改换了阵营。

%62
血蛊的传承,带给全天下极大的动荡和危害。

%63
血海老祖布置下的一个传承中,往往只有两三只蛊。但这些传承,就宛若星星之火,稍不留神,就会燎原!

%64
几乎每隔一段时间,都会有掌握血蛊的魔道蛊师,出来作乱。这些人有的失败了,只在山寨中就被扑杀。有的暂时成功了,形成大势,毒害一方。

%65
不管暂时成功的,还是失败的,亦都会在某个困顿之时,不甘地留下新的血之传承。

%66
因此,血祸绵绵不绝,不仅没有因为血海老祖的死亡,以及正道的全力剿杀而衰落,反而更加昌盛,大有底蕴深藏,无法狠除,永远不断的趋势。

%67
以至于正道人士都常痛声大骂:“这些该死的血蛊师!我明明记得,已经杀了一波又一波。

%68
但稍微不留神,再抬头一看,不知从什么地方,又会冒出新的一茬!”

%69
到如今,血海传承已经被公认为,是全天下最普及,数量最多的传承。没有之一!

%70
严格意义上讲,方源前世也是受了血海老祖的遗泽。

%71
“前世,我在四百年之后,才找到一处血海传承,开始称霸。今生我若收服这些血蝠,等若提前了四百年的功夫啊。”

%72
血蝠冲来,方源临危不惧。

%73
他有春秋蝉,六转气息足以让这些蝙蝠不战自溃。唯一需要忌惮的,就是那只隐藏在血湖中的五转蛇蟒。

%74
“但这情形,又有些不对劲。当初血海老祖布平传承,往往只是两三只蛊。怎么这里,却有上百只的刀翅血蝠?难道……那个传闻是真的?”

%75
传闻说,血海老祖故布疑阵,虽然设了数十万道传承,但真正要传下的只有几道。

%76
这几道传承中,藏有血海老祖最得意的几只蛊虫,或者研究的心得,或者合炼的秘方。

%77
“难道这里,便是血海老祖真正传承之地?”方源自然而然地想到这里,心头怦怦直跳。

%78
他思绪电转,虽然想了一大堆,但外界时间却无多少流逝。

%79
刀翅血蝠群纷纷向他杀来,方源面色平淡,他长发黑袍,攀在山壁上,正要唤出春秋蝉。

%80
但就在这时,异变突生!

%81
“果然啊,这里藏着血祸!”一个低沉坚定,如铁石般的洪亮声音,从洞顶传来,然后在血海上嗡嗡回荡。

%82
神捕铁血冷!!!!

\end{this_body}

