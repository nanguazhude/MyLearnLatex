\newsection{未来的路,会很精彩}    %第六节:未来的路,会很精彩

\begin{this_body}

%1
空窍玄妙异常,虽然寄托在方源体内,但是却不和五脏六腑同处一个空间。你可以说它无限大,又可以说它无限小。

%2
有人称之为紫府,有人称之为华池。不过更多人称之为元海空窍。

%3
整个空窍呈球形,空窍表面流动着白光,是一层光膜。就是先前的希望蛊炸裂开来,凝聚的一层光。

%4
正是因为有了这层光膜,才支撑着空窍,不会塌陷。

%5
空窍之中,自然就是元海。

%6
海水平滑如镜,呈现一片碧青之色,却又浓稠无比,带着铜的光泽。

%7
这是一转蛊师才有的青铜真元凝结,俗称青铜海。

%8
海面不到空窍的一半高度,只有四成四。这也是丙等资质的局限。

%9
每一滴的海水,都是真元,代表着方源精气神的凝结,象征着他十五年来积蓄出来的生命潜能。

%10
蛊师就是用这真元来催动蛊虫,也就是说,从此刻起,方源就正式迈入一转蛊师的行列。

%11
空窍已开,再没有希望蛊汇入方源体内。

%12
方源收起心神,只觉得前方的压力坚厚如壁,也不能再进一步。

%13
“一如前世啊。”对这个结果他淡然一笑。

%14
“不能再走了吗?”学堂家老抱着万分之一的希望,隔岸喊道。

%15
方源直接转身往回走,以实际行动回答了他。

%16
这下子,就连那些少年们也反应过来了。

%17
顿时,就听嗡的一声,人群炸开了锅。

%18
“什么?方源走了二十七步?”

%19
“原来他只有丙等资质?!”

%20
“难以置信,他这么天才,只是个丙等?”

%21
人群中掀起轩然大波。

%22
“哥哥……”在人群中,古月方正抬起头来,震惊地看着趟河回来的方源。他不敢相信眼前的一幕,自己的哥哥居然只是一个丙等?

%23
他一直认为,哥哥会是个甲等资质。

%24
不,不仅是他,舅父舅母还有族中的许多人都这么认为着。

%25
但是现在,结果居然是这样!

%26
“可恶,居然只是个丙等!”古月族长暗捏双拳,深深地叹了一口气,失望之情溢于言表。

%27
暗中关注的家老们,有的皱起眉头,有的低头谈论,有人仰天长叹。

%28
“会不会测试有误?”

%29
“怎么可能?这方法准确无比,况且有我们一直盯着,作弊都难。”

%30
“可是,那他先前的表现和才情,又怎么解释呢?”

%31
“元海资质越高的少年,的确会表现出超越常人的特性。比如聪颖、悟性、记忆力、力量、敏捷等等。但是反过来,这些特性并不代表资质一定就高,一切还得以测试结果为准。”

%32
“唉,希望越大失望越大。古月一族是一代不如一代了。”

%33
……

%34
冰凉的河水,浸湿了足袜,冷得有些刺骨。

%35
方源依旧面无表情地走着,距离越来越近,他可以清晰地看到学堂家老沉重的表情,可以敏锐地觉察到上百位少年投来的目光。

%36
这目光中有惊诧,有震动,有嘲讽,有幸灾乐祸,有恍然,有冷漠。

%37
一模一样的情境,让方源不由地想到了前世。

%38
那时自己感觉天都塌下来了,过河时失足摔了一跤,浑身湿透,失魂落魄。却没有一个人搀扶自己。

%39
这些失望、冷漠的表情和眼神,像是尖刀凌迟着自己的心。思绪纷乱无比,胸口更是隐隐作痛。

%40
就好像是从云端摔到地上,站得越高,摔得越狠。

%41
不过,今生。

%42
重临这样的场景,方源的心却平静无比。

%43
他想到了那个传说,当困境来临时,要把心交给希望。

%44
如今这个希望,就在自己的体内。虽然这希望不大,但已经好过那些毫无修行资质的人。

%45
别人因此而失望,那就失望吧,又能如何呢?

%46
别人的失望与我何干?重要的是自己心存希望!

%47
五百年的生涯,让他明白了一个道理:人的一生之精彩,在于自己追逐梦想的过程。不必苛求旁人的不失望或者喜欢。

%48
走自己的路,让旁人失望和不喜欢去吧!

%49
“唉……”学堂家老深深地叹了一口气,又喊道,“下一位,古月方正。”

%50
却无人应答。

%51
“古月方正!”家老大喝,声音在溶洞中嗡嗡回响。

%52
“啊?我在,我在!”方正从震惊的情绪中挣脱,连忙跑出,不幸脚下一个踉跄,顿时摔了个跟头,噗通一声,恰巧滚落到河里。

%53
顿时,哄堂大笑。

%54
“方家兄弟,不过如此。”古月族长冷哼一声,对古月方正也产生了一种厌烦。

%55
“这下丢人丢大了!”方正在河水中奋力扑腾了几下,奈何这河底甚滑,根本站不起来。努力的结果,反而显得他更加笨手笨脚。耳中听到一阵阵的哄笑声,这让他心中越加慌乱。

%56
但就在这时,他忽然感觉一股大力把他拽住,脑袋终于脱离了水面,身体也重新掌握了平衡。

%57
狼狈地抹了一把脸,再定睛一看,却是哥哥方源提着自己的衣领,将自己提了起来。

%58
“哥……”他张口欲言,换来的却是水呛到的结果,继而又引发了一阵剧烈的咳嗽。

%59
“哈哈,方家的难兄难弟!”岸上有人嘲笑出声。

%60
哄笑声愈加响亮,学堂家老也没有站出来阻止,他深皱着眉头,心中充满了失望。

%61
方正手足无措,就听到耳边传来哥哥的声音:“去吧,未来的路,会很精彩呢。”

%62
方正不由地惊讶地张开嘴,此时方源背对着众人,因此岸上的人都看不清楚。但是方正却清楚地感觉到方源的平静,说话的时候嘴角甚至都微微翘起,流露出一抹玩味深沉的笑意。

%63
“明明只是个丙等资质,为什么哥哥还这么平静?”不由地,方正的心中充满了疑惑。

%64
方源再没有多说,拍拍方正的背,转身就走。

%65
方正表情愣愣的,走向花海。

%66
“想不到哥哥竟然这么平静,要是我的话,恐怕……”他低着头,无意识地向前走着。却并不知道自己正上演着一场奇迹式的表演。

%67
等到他惊醒的时候,他已经站在花海的深处,一个先前并未有人达到的距离。

%68
四十三步!

%69
“天呐,甲等资质!!!”学堂家老失态地大叫着。

%70
“甲等,竟是甲等?!”

%71
“三年了,古月一族终于又出现了甲等资质的天才!”

%72
暗中关注的家老们,也同时叫喊着,大失风范。

%73
“嗯,方之一脉本来就是从我们赤脉分出去的,这个古月方正就由我赤之一脉收养了。”古月赤练当即宣布道。

%74
“怎么可能?你赤练老鬼何德何能,误人子弟的本事倒是不错。这个孩子还不如交给我古月漠尘抚养。”古月漠尘像触电一般,立即咆哮起来。

%75
“谁都别争了。这个孩子谁栽培,都没有本族长亲自栽培的好。谁有异议,就是反对我古月博!”古月族长双眼赤红,一扫先前的失望颓废,整个人都癫狂了。

%76
(ps:新书真心需要大家鼎力支持啊,多点击一下,都是好的。虽然说隔了两个月才开新书,人气有点断,随着时间流逝,人气就会回升。但是……目前在冲新书榜的说,急需大家支持呀。混贴吧的同学都来起点支持一下,公众章节是不收费的。)

\end{this_body}

