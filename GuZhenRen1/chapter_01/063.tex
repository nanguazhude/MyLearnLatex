\newsection{月下赠玉皮,地花藏白豕}    %第六十三节:月下赠玉皮,地花藏白豕

\begin{this_body}

日落月升,夜幕降临。

这是古月山寨中一处隐蔽的院落里。

一株高大的槐树,枝繁叶茂,雄厚宽大的树冠,宛若碧玉华盖,将整个院子都笼罩住。

月光温柔如水,透过槐树的枝叶,洒在庭院中。

一阵风袭来,树叶沙沙作响,树影微微晃动。

就在这树影下,站着两个人。

族长古月博用柔和的目光注视着方正:“方正,今天你选择了第二只蛊,不知是什么?”

“禀告族长,今天我选了一只铜皮蛊。”古月方正站得笔直,带着一脸的崇敬之色答道。

古月博点点头,沉吟一声,道:“不错,选的很好。”

古月方正站在古月博的面前,身躯都紧张地绷着。听了古月博夸奖的话,一时间不知道给怎么回应好,只好赫然一笑:“族长大人,我也是随便选的。”

“你以为我这是胡乱夸赞你吗?不是的。”古月博盯着方正,嘴角上挂着淡淡的笑意,“你知道么,从选择一只蛊虫,往往就能看出一个人的秉性。”

“你选择了铜皮蛊,是用于防守。搭配月光蛊,就是一攻一守。这说明你秉性纯厚。世间之事,一攻一守,就是一正一奇,一阴一阳,一柔一刚,这就王道。”

“而古月漠北选的是黄骆天牛蛊,此蛊增加耐力,能持久战。这就透露出漠北坚韧顽强的个性。”

“至于古月赤城选择龙丸蛐蛐蛊,使得他闪避能力增加。这说明他不喜欢强攻,为人精明,善于钻营,但同时也表露出的他性格软弱的一面。”

古月方正听得目瞪口呆,他从未想过,可以从这么一件简简单单的事情上,看出这么多通透的道理来。

不由地,他看向古月博的目光变得更加崇敬起来。

“族长大人,那我哥哥选的是什么蛊?”方正忽然想到了方源,立即问道。

古月博笑了笑:“你哥哥选择的是小光蛊,用此蛊辅助月光蛊,能令月刃攻击更强。这就说明他的性情激进,饱含侵略性,容易走极端。”

“的确,哥哥好像就是这样子的。”方正轻声喃喃。

古月博将方正的神情看在眼中,记在心头。

但凡身居高位者,必有过人之处。虽然古月博和方正之间,面对面相处的日子并不久。但是古月博老辣犀利的目光,却已经看透了方正。

他告诉方正漠北、赤城两人的选择,自有其深意。

就是帮助方正分析此二人,期待着他将这二人击败,以甲等资质奠定领导地位。

但是古月博不会明说,不会直接指使教唆方正去干什么。

身为族长,一言一行,都有着政治属性。若是直接指使方正,对付赤城和漠北,这话传出去,说不定就被人误解成族长的政治意图。这样的话,影响可就大了,搞不好家族内斗,还会祸及整个山寨。

还有一点,就是古月博也期待方正能独自领悟他的意图。他花费时间和精力,亲自培养方正,不是培养一个四转、五转的打手。没有政治智慧的打手,就是一把双刃剑。他要培养的是未来家族的领袖!

“我替方正分析漠北和赤城的性情,他根本就没有意识到里面蕴藏的深意。反而问我古月方源的情况。看来方源留给他不少的阴影,不过他正是少年叛逆的时期,一心想压过方源,也可以理解。唉,若是方正有方源的智慧就好了,这些年我见过不少的少年,要论政治智慧,方源绝对是第一。可惜,他只有丙等资质。”

古月博心中叹了一口气,脸上的笑容却越发的温和。

他从怀中掏出一只蛊虫来。

“这是――玉皮蛊?”方正看到这只蛊虫,顿时眼前一亮,轻轻地叫出声。

古月博道:“和铜皮蛊相比起来,这只玉皮蛊更加优秀,不仅消耗的真元少,而且防御力也比铜皮蛊更强一些。方正,你想要么?”

“族长!”方正吃了一惊,他看向古月博,结结巴巴地道,“我,我当然想要了。”

“想要我可以给你。”古月博笑得越加柔和,“但是我身为族长,向来公平公正,不能凭空无故地赠送给你。所以,我有一个条件。”

方正连连点头,瞪大双眼:“什么条件?”

古月博脸上笑意消失,露出肃容:“我要你率先突破一转,达到二转,成为修为第一人!而这只玉皮蛊就是对你的提前奖励。”

“啊,晋升二转?”古月方正不禁露出迟疑之色。他只是刚刚晋升一转中阶,中阶之后是高阶,高阶之后还有巅峰。

结果古月博现在却要求他,成为此届的二转第一人。

“怎么,你害怕了?那这只蛊就只能给其他人了。”古月博作势收回玉皮蛊。

古月方正被这话一激,顿时头脑发热,喊道:“不,我答应你!我会击败所有人,成为二转第一人!”

“这才对嘛。”古月博再次露出温和的笑容,将玉皮蛊放在方正的手中。

心中则道:“方正啊方正,我知道你有些自卑,这些自卑对你的成长太有害了。而要打消这些自卑的最好方法,就是成功。你是甲等资质,修为晋升二转,就属你最有优势,也是你最能轻易达到的一个成功。你一定要好好努力,若是连这个都失败的话,未免就让我太失望了。”

而同时,方源再次进入石缝秘洞,深入到甬道尽头。

这一次,他并没有带什么铁锹、铲子、铁锤等等工具,而是细心地观察周围。

前一晚他在这里受挫,在回去山寨的路上,就感觉有些不对。

到了今天在宿舍,利用春秋蝉,炼化了小光蛊之后,他忽然灵光一现,悟到了其中的蹊跷之处。

“这挡路的巨石,未免太圆润,太光滑了,这是人力加工过的。也就是说,是花酒行者故意设置的拦路巨石。他为什么要在这里,设下如此关卡?”方源露出思索之色。

他再次打量周围。

甬道中的地面平整,顶部是圆顶,两边的墙壁都是浑然一体的赤红泥土,散发着昏暗的光。

“咦?”当他目光再次扫向地面时,他发现了一点可疑的地方。

靠着拦路巨石的一块地面,颜色有些深重。这个色差并不明显,在如此昏暗的光线下,若不仔细观察的话,绝不容易发现。

方源蹲下身子,出手摸了摸这片地面,顿时就有一种湿漉漉的感觉。

难怪颜色有些深重,原来是沾着水。

但这甬道干燥,哪里来的水?

方源又用手指头,捻了捻这片潮湿的泥土。他发现这块地面的泥土确实有问题,十分松散柔软,不像干燥的红泥那样粘聚。

方源目光闪了一闪,经验和直觉都告诉他,这处地方很有可能就埋藏着花酒行者的一把“钥匙”。

而这把“钥匙”,就是关键,能让方源继续前进下去。

方源开始挖泥,泥土很松散,倒是没有费多大力气。

挖到地下一寸的时候,就有一股特殊的幽香,似有似无地传入方源的鼻腔之内。

“这股幽香浓郁奢华,却又不庸俗,显得高致雅贵,难道是……”方源心中一动,想到了某种可能,手上的动作顿时又加快了几分。

挖着挖着,泥土下忽然现出一丝暗金的光。

“果然是它!”方源双眼骤亮,手上动作变得细腻,小心翼翼地将周围的泥土挖开,将这坑口扩大。

片刻之后,一朵埋藏在地中的,暗金色的花苞呈现在他的面前。

它深入地面两寸,体积有寻常石磨大小,花苞表面细腻如绸,暗金作色,显得幽静神秘而又高贵典雅。

“果然是地藏花蛊!”方源见此,长长地吐出一口浊气。

他并没有急着打开花瓣,而是坐在地上休息了片刻,将双手的泥土都擦拭干净,这才慢慢伸手,将暗金色的巨大花瓣轻轻地揭开。

地藏花蛊,就像是荷花和卷心菜的结合体。它的花瓣一片又一片,紧紧地贴在一起,厚厚的,手感滑润。方源揭开一片片的花瓣,就仿佛揭开一卷卷的丝绸。

而这暗金色的巨大花瓣,一旦脱离了本体,就迅速消散。好像是一片片的雪花,融化在空气当中。

方源揭开外围五六十片的花瓣后,花苞的体积削减了一大半,露出里面的花心。

花心处的花瓣,形体较小,厚度也纤细下来,质地越加柔软细腻。不像是丝绸,更像是薄薄的一张纸。

方源动作越加缓和,往往几个呼吸之后,才成功地揭开一片花瓣。

花瓣越来越透明,片刻之后,方源将一张似宣纸般轻薄的花瓣掀开之后,他停下了动作。

此时地藏花蛊,只剩下了最中心的薄薄一层花瓣。

这些花瓣相互叠加,包裹成一个拳头大小的圆球形状。

花瓣半透明,轻薄如宣纸。花瓣的里面充斥着一股黄金色的液体。在这黄金的花液中央,一只蛊虫在里面沉睡着。

方源凝神细看,却只能看到这蛊虫的模糊影像,不能辨认究竟是何种蛊虫。

离得近了,他的气息就喷到花心之上。花心圆球顿时颤颤巍巍,黄金的液体在花瓣的包裹下,也轻轻地晃动起来。

蛊虫没有食物就会饿死,只有极少数的蛊虫身上,能出现自我封印的情形。为了保养蛊虫,蛊师们想出了许多的方法。

地藏花蛊就是其中一种。

地藏花蛊,是一次性的消耗蛊,一旦种植在地上,就不能再移动了。

它的食物来源十分简单,就是地气。只要种植在地面下,有着充足的地气,就能存活。

它的作用也只有一个,那就是把其他蛊虫包养在花心当中,浸泡在黄金色的花液里。

这种黄金花液能在一定程度上,模拟封印状态,让浸泡在其中的蛊虫陷入沉眠。

“花酒行者在这里种下了地藏花蛊,这花心里面的蛊虫,应该就是留给继承者的。”方源伸出手指,轻轻地捏着剩下的花瓣,小心地撕开一道口子。

黄金液便顺着他的手指流淌出来,如同豆油的感觉。

随着黄金液的流逝,花心慢慢地瘪软下去。方源手指搓动,捻开柔嫩至极的花瓣,从中取出那只沉睡的蛊虫。

这是一只很可爱的瓢虫。

只有大拇指的指甲盖那么大。

它浑身都是乳白色的,从背面俯瞰的话,就像是一个圆。

它的头只占据圆的很小部分,其余的都是它肥肥的肚子和亮亮的甲壳。

它的六只细小的肢脚,也是乳白色的,藏在肚子底下。

“白豕蛊!”方源眼中喜色一闪即逝。

(ps:大家的支持如此给力,我也不能玩虚的。三江第一,五更完毕,兑现诺言,让魔道崛起!)

c!\~{}!

\end{this_body}

