\newsection{贾金生,其实我不想杀你的}    %第四十七节:贾金生,其实我不想杀你的

\begin{this_body}

雨哗哗地下着。

天空中乌云盖顶,远方的群山绵延,融成一团墨色。

雨帘将天地交织在一起。

咔嚓!

天空骤然一亮,一道闪电如银蛇划破天空,又骤然消失不见。

要到夏天了,这番春末的大雨也似乎带上了一丝夏天的热烈。

青茅山上,大片大片的碧色矛竹高昂挺立,对抗着风雨,竹身仍旧笔直如枪,竹尖直指苍穹。

古月山寨中,无数的高脚吊楼鳞次栉比,埋着头忍受着大雨的洗刷。山寨外,商队已经重新启程。

“雨大了,注意路面。”

“别掉队,蛊师牵引好蛊虫,尤其是肥甲虫,别卡在山道上了!”

“你们这些凡人武者,都被把招子放亮点,照看好货物。丢了一件,就找你们算账!”

商队中不断有此起彼伏的吆喝声。

在古月山寨中停留了三天,这只商队就离开此地,顺着青茅山的山道,前往下一个目的地。

大雨冲刷着天地,山寨周遭的道路上都铺着鹅卵石,这还好些。但是出了五百米之外,就是一片泥泞狭窄的山道。

骄傲的驼鸡,此时把头垂下,鲜艳的彩羽被雨水打湿,粘连成一片片,成了名符其实的落汤鸡。

肥甲虫扭动着肥大的身躯,前进的十分缓慢。雨水打在它的黑色皮甲上,形成股股水流,滑落到两边地上。

毛茸茸的山地大蜘蛛,也被淋湿了,青黑色的毛都黏在了一起。

倒是那些蟾蛊欢快地大叫着,驮着货物和蛊师,在山上蹦跳着前进。

还有翼蛇,已经收起了双翼。粗大的蛇身快活地在泥水中穿梭行走。

为了保护货物,避免被雨水淋湿,蛊师们此时亦各显神通。

在几头身躯庞大臃肿的肥甲虫的身上,都有蛊师站立在中端。他们双手高举,距离手掌一寸的高度,各悬空漂浮着一只一气金光虫。

青铜真元如水汽升腾,灌注到一气金光虫的体内。蛊虫全身都闪着如金豆般的光,以此圆心,撑起一个庞大的淡金色气罩。

半球形的气罩笼罩范围颇大,将一头肥甲虫完全遮盖外,还有绰绰有余的空间。

而雨滴砸在气罩上,就都被弹开来,好像是打在了雨伞上。

不过这种一气金光虫,持续消耗真元,时间一长,一转的蛊师就受不了了。

果然不一会儿,就有一位蛊师喊道:“不行了,我真元快耗尽了。谁来接替我?”

“我来!”几乎是第一时间,就有一名蛊师赶了过来,接替了他的位置。

一些拉着板车,或驾驭山地大蜘蛛的蛊师,则催动了体内的青丝蛊。

在青丝蛊的力量下,蛊师的头发疯长起来。

一个正常人的头发,至少有十万根。十万根的发丝,根根都伸长成五六米。它们相互纠缠交织在一起,将蛊师的身躯,还有屁股下的坐骑蛊都包裹住,形成一件临时的密不透水的黑发蓑衣。

青丝蛊,是一转蛊虫,常被蛊师用来防御打击。它使用起来,一次性消耗三成的青铜真元,不像一气金光虫那样需要持续的真元输出。

这青丝蛊若是和一转的黒豕蛊合并精炼,就会晋升成二转的黑鬃蛊。

黑鬃蛊催动起来,就不仅仅只是头发,全身的汗毛都能变得又黑又粗,在几个呼吸之内,在蛊师的身上生长成一片黑鬃护甲。

黑鬃蛊若继续晋升,就能成为三转蛊虫中大名鼎鼎的钢鬃蛊。

除了一气金光虫、青丝蛊之外,商队中的许多蛊师选择了水蛛蛊。可以看到,这些蛊师的身上,都覆盖着一层薄薄的淡蓝色水衣。

水衣表面,水流不断地汩汩流转。雨滴打在水衣上,立即就和水衣融为一体。

蛊师不断淋雨,身上的水衣越变越厚。每隔一段时间,蛊师就要催动水蛛蛊,将水衣中多余的水分排出去。这个时候,厚厚的水衣,就会削减成原来的薄薄一层。

至于那些凡人武者,都在泥泞的道路上来回奔波着,照看着货物。他们大多数都穿着蓑衣,但忙乱中蓑衣避雨的效果很是有限,他们都被雨水淋透了。

“这鬼天气!”武者们都在心中狠狠地咒骂着。

雨天,山路更加难走。

在这样的天气中,武者哪怕再强健,终究还是凡人之躯。浑身都被淋湿,又过度劳累的话,极容易就感染风寒,大病一场都是轻的,说不定就染上了后遗症,甚至病重的,会直接被蛊师抛弃在路途中。

若是在山道中遇到坍塌滑坡,或者是遭遇野兽、蛊虫的侵袭,更有可能直接丢掉性命。

商队虽然规模庞大,有许多的蛊师。但是每次行商,都会有大量的减员。凡人武者死的最多,蛊师也会有伤亡。

若是商队不幸遇到一股大型兽群迁徙,全军覆没也有可能。

事实上,除了这些天灾,还有人祸。

沿途的山寨,未必都欢迎商队的到来。有一些山寨,就喜欢打劫外来人。

“走了,来年再见!”一些蛊师坐在蛊虫身上,侧过身子,向古月山寨挥手告别。

山寨外的大门口,不少人聚集在一起,目送着商队离开。

“来年一定要再来哦。”小孩子们依依不舍地大喊着。

大人们的目光则复杂了许多。

“前途未卜,世道艰难。明年能来山寨的,不知还能剩下多少的熟面孔?”

“不管是行商,还是生活在寨子里,讨个生活都是不易啊。”

商队越行越远,众人也渐渐散去。

欢快热闹的集市氛围,也随之消失无踪。原先搭建着帐篷,摆着地摊的地方,留下了一大片的狼藉。

草皮被络绎不绝的人群,踩踏出草根和泥土。雨水打在上面,立刻形成泥泞,还有无数坑坑洼洼的浑浊积水。

此外,还有不少的生活垃圾残留了下来。

方源独自一人,站在僻静的山坡上,遥遥远望着商队。

商队就像一只肥胖的彩色花蟒,在灰色的大雨下,沿着狭窄的山道,缓缓钻入茂密的山林。

“真是天公作美啊……”方源轻叹一声。

他撑着一把黄油纸伞,在雨中静静地立着。

他身上穿着最普通的麻布衣衫,身型瘦削,皮肤带着十五岁少年的那种苍白,一头干净利落的黑色短发。末端的发梢随着风在伞下微微颤动。

别人诅咒这种鬼天气,他却感叹这雨下的及时。

他在昨晚杀了贾金生,处理了现场,但是事发突然,总有些不妥当的地方。尤其是血腥气味,因为秘洞不通风,并不容易驱除。

这大雨一下,冲刷天地,洗净空气,大大减弱了依靠气味的侦测手段。石缝那边必定又有小瀑布垂落下来,新鲜的水汽稀释之后,短时间内几乎就不可能暴露。

当然,时间一长,暴露的可能就越大。

这个世界上,存在着各种各样的奇妙蛊虫,侦察手段丰富多彩,就算是方源,也只是知道其中的一部分罢了。

雨水打在黄油伞面之上,发出滴滴答答的声音。然后顺着伞骨,一股股水流垂落而下,又打在方源脚下的青石上,啪啪的,溅起一朵朵的水花。

看着商队转过拐角,完全钻入茂密的山林当中,方源却没有一丝喜色,反而眼光有些凝重。

“贾金生修为虽然薄弱,资质低下,但是地位特殊。商队中每个人均忙得焦头烂额,因此短时间之内没有察觉到他的失踪。但是过不了多久,必定就有人发觉。到那时,贾富必回来调查,真正的挑战才会到来。”

“贾家家主刻意安排贾金生和贾富共领商队,此举大有深意。论修为,贾金生对贾富有天壤之别。论心智,前者更是被甩出了几条大街。贾家家主如此安排,是就是要让贾金生遭受打击,让他认清楚现实,今后安安分分过日子。同时也在考验贾富的心性,若是打击太过,连自己兄弟都不容,怎么可能把贾家族长之位给他呢?”

“贾金生从未真正了解过其父的用心良苦。他虽然有些才智,可惜只练了一张外皮,真是可惜啊。可惜了这么好的一个棋子。”

方源在心中暗暗感叹。凭借五百年的经历,他早已看破了表象,觉察出这事深层的本质。

当他在昨晚的那场纠纷中,看出了贾金生和贾富的复杂关系时,当场他的心中就产生了一个模糊的计划。

在他这个计划中,贾金生是个很合适的棋子。他修为薄弱,地位很高,虽然有些小聪明,但是阅历太浅了,完全能掌握在手中。

这个棋子,一旦掌控住,将很有用处。

一来,在他身上就能建立起一个稳定的销赃渠道,为将来杀人夺宝做准备。

二来,方源隐居幕后,利用他和影壁,挑拨青茅山上三大山寨的矛盾,引发内战,自己渔翁得利。

三来,依靠他,来打入贾家内部。未来贾家家产纠纷,引发的斗蛊大会,是一个重大事件。好处多多,方源完全可以在其中,谋求到最大利益。

“我现在的修为太低下了,做起事情来束手束脚。若是有个棋子为我所用,就可以干一些我不能出面的事情。不仅方便,而且风险大大的降低。若是将来暴露了,直接抛弃这棋子,自身仍旧逍遥。”

“周围的人都是知根知底,忠于家族,不好操纵。唯有像贾金生这样的外人,才能更方便破局。可惜了,没有料到花酒行者居然留下了力量传承。”

花酒行者是五转蛊师,他的遗产自然比贾金生这个棋子更加珍贵。

当然,若是两者兼得,自然是最好。但是面对这样的重宝,贾金生已经不受控制了,所以只能舍弃掉。

“世间不如意之事,十之八九啊。”方源摇头叹息。

花酒行者的传承出现,打破了方源的原计划。而影壁异变之后,先前的画面统统消失,只出现了一行血字,提示方源打破影壁,会出现一个洞口。沿着洞口走下去,就能得到力量的传承。

血字只出现了几个呼吸的时间,就消失不见。影壁也还原成了最普通的山壁。

方源昨天一夜都忙着处理杀人现场,根本就没有时间打破影壁。

“仓促之间杀了贾金生,这件事后遗症颇多,只是暂时没有显露出来罢了。我虽然成功毁尸灭迹,但必定还有一番大麻烦。这样一来,暴露酒虫的方式,就要修改了。还有石缝秘洞暂时也不能去。近期都要缩在山寨之中,防备不久后的调查。”

方源转过身,撑着伞,在雨中向山寨走去。

“不过这样也好。最近这段时间,我大量消耗元石,精炼出中阶真元。利用中阶真元,温养空窍,近期就能突破到中阶。到了中阶,我的实力就能增长一倍。到时候继承花酒行者的传承,也会更有把握。”

魔道中人的传承,可不像正道人士那么温和,通常都会有惊险的考验。若是考验不通过,往往付出的就是生命的代价。

“世事难料,不过也正是如此,才显得有趣啊。”方源嘴角勾勒出一丝冷笑。

大雨下的青山,延绵不绝,泛着灰色的绿,显得压抑而又沉重。

一阵风吹来,雨点倾斜,打在方源的肩头,一阵寒意袭来。

他又想到了贾金生。

心中一叹:“贾金生,其实我……不想杀你的。”

可惜了一个好棋子。

(ps:立马就三江第一了,我魔道不衰!这章将近四千字,奉献给大家。不过接下来还有七天,时间说长不长,说短不短,魔道志士们,咱们不能懈怠,被赶超就太丢脸鸟……)

------------

\end{this_body}

