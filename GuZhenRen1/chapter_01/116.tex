\newsection{斩杀猴王得新蛊}    %第一百一十六节:斩杀猴王得新蛊

\begin{this_body}

“一成一分的真元,也就是两记月刃,或者承受石猴王两次的偷袭。单靠月芒蛊或者白玉蛊,是不行的。唯一的机会,就是在石猴王攻击我的瞬间,抓住战机,催出月刃,斩杀掉它!”方源脑海中电光火石般,闪出此刻最佳的战术。

石猴的防御能力并不出众,石猴王既然选择这种偷袭的攻击方式,也从侧面暴露出它防御低下的弱点。www.13800100.com

一记月刃,能一下子斩杀五六只玉眼石猴。即便不能能斩杀掉石猴王,亦能重创它。

但别以为很容易。做到这点相当的难,就算是一组的蛊师过来,没有侦破隐形的蛊虫,照样要饮恨当场。

“这猴头狡诈,一直不攻击,是想等着我真元耗尽吗?也罢,就相信春秋蝉一次,赌了这一把!”方源瞬间便有了决断,双眼中闪过一抹冷酷的光。

他站在原地,双手垂下,提着上衣的领口。同时他缓缓地合上眼帘,只留出一条眼缝。更惊人的是,他撤掉了白玉蛊的防御。

空窍中真元的消耗,顿时停止下来。但与此同时,他浑身上下再无白玉之光的保护。

石林中不断传来石猴的怒叫和惨嚎,但是方源却感觉这些声音,离自己越来越遥远。

一种静寂笼罩住他的心神。

他在静心地等待着石猴王的攻击。

当它攻击的时候,就是这场战斗决出胜负的一刻!

等待……

等待……

陡然间,空窍中春秋蝉再出震动。

吱!

下一刻。方源耳边一声炸响,石猴王陡然出现在他的左侧!!

“白玉蛊!”方源双眼一抹精芒暴射,白玉之光笼罩住他的全身。

砰。

石猴王打在方源的身上,力道凶猛,几乎把方源打了个趔趄,空窍中的真元骤然缩减半成,只剩下另一半!

狡诈的石猴王一击不中。立即遁走!

方源根本来不及反击,但这时间足够他将手中的上衣一扬。

旋即,他就感到上衣兜住一个东西。一股力道拖拽着上衣向外跑。

上衣并不是铁丝网,为了防止上衣的破裂,方源及时地松开双手。就看到上衣裹着一个东西。以惊人的速度在四处乱窜。

“就是此刻!”方源眼中寒芒一闪,此战成败,就看他手中这记月刃,他心中冰雪般冷静。

石猴王到底是野兽,被上衣遮住脸面,顿时陷入慌乱当中。

它发出吱吱的尖叫,呼唤麾下石猴相助,同时顶着上衣不断地变向,突兀转折,四处乱窜。

一道幽蓝的月刃斜斜飞来。正中石猴王。

石猴王发出一声凄厉的惨叫,显出身形。

它外形和普通的玉眼石猴,没有什么两样。但是体型比它们大了三倍,同时双眼绽放着血红的光芒。

一道细长又深重的伤口从它的胸膛处,一直延伸到左大腿。鲜血不断地向身外涌出来。

虽然没有死亡。但是它已经身受重伤,死亡的气息已经笼罩住它的全身。它惊恐地捂住伤口,重新隐去身形。

方源的上衣被月刃斩出一道长长的缺口,落在地上。但是血迹却仍旧暴露了石猴的动向——它慌张地后退,再没有追杀方源的欲望。如此重伤,再不处理。恐怕性命不保。

趁着这个功夫,方源也退向石门。催出月刃之后,他空窍中的真元只残留一丝,战斗力急剧下滑。

此战看似平手,其实是方源胜了。

石猴王的伤势,一时半刻必定恢复不了,血流的越多,它就越虚弱。

反观方源,依靠元石就能快速地补充真元,将战斗力恢复过来。

即便没有侦破隐身的蛊虫,亦没有大范围的攻击手段,但是凭借着丰富的战斗经验,以及临危不乱的钢铁意志,方源做到了以弱胜强。

“猴、狐、狈……此类野兽,有着超越寻常兽类的智慧,因此狡诈。但正因为如此,它们缺乏一种蛮勇,受了重伤,就会远遁。若是野牛、野猪这种生物,越是受伤越是狂暴。这只猴王身上,看来只有一只蛊虫。这只蛊虫虽然能隐身,但是却连血迹都遮掩不住,若我所料不差,应该是一转的隐石蛊。”

方源心中思量着,依靠脑海中的记忆,石猴王对他来讲,已经再无秘密可言。

“战局已定了。”方源退回到石室,关上石门,利用元石补充真元。

片刻之后,他真元重新补充到巅峰状态,推开石门,他再次来到石林中。

石林中仍旧是一片混乱,但是比之前的程度要好许多。

“这场混乱之后,恐怕整个石林的猴群势力都要重新洗牌。石猴的迁徙和重整,流浪的孤单石猴将组成新的猴群。我辛辛苦苦打通的通道,恐怕也要因此消失了。”

方源心中一沉,他必须趁着这个通道没有彻底消失之前,斩杀了石猴王。

否则重新打通这通道,将耗费他大量的时间。当他再次到达石林中心的时候,恐怕要面对一只痊愈之后的石猴王。

宜将剩勇追穷寇,不可沽名学霸王。

方源沿着开辟出来的路线,闯入石林。沿途中不时的有石猴蹦跶出来,都被他一一绞杀。

一刻钟后,他再次来到最中央的巨大石柱跟前。

石猴王倒在地上,化为了石雕,已经死了。

一只玉眼石猴,一脚踏在它的尸体上,吱吱的乱叫着。

王位更替,旧王已死,新王上位。不管是兽群内部,还是人类社会,都有冷酷的淘汰机制。

“倒是省去了我一些功夫。”方源慢慢走近。

就在这时,一只蛊虫悠悠地从石猴王的尸体上悬浮而起,向着新王飞去。

月芒蛊!

方源及时地发出一道月刃,赶跑了石猴新王,然后走上前去,一把抓住这只蛊虫。

此蛊外形极为平凡普通,就是一个灰色的石块儿。表面凹凸不平,不是正立方,也不是圆珠状。估计把这蛊随处丢在路边,单看外表,将无人注意。

但实际上,它却是石中之精,大自然孕育而生的天然蛊虫。

它看起来是个石头死物,不过事实上,却是货真价实的生灵,有着自己的灵智意识。

一如方源所料,正是隐石蛊。

它被方源抓住,不断挣扎着,还想脱离方源的魔掌。

春秋蝉。

方源心念一动,春秋蝉在空窍中浮现出来,气息泄露了一丝出去。

隐石蛊顿时死了一般,再也不敢挣扎,就像是老鼠见到了猫。

方源绯红的真元一催,瞬间将它炼化。

又得一蛊!

隐石蛊被方源收入空窍,沉入真元海底,和白玉蛊靠在一起。

石猴新王在一旁眼睁睁地看着,方源将隐石蛊收入体内,急得在原地乱蹦,吱吱喳喳地尖叫着。

它才刚刚上位不久,没有多少的石猴响应它。

方源又一记月刃扫过去,立即就收了四五天猴头的性命。那些聚集在它身边的猴群顿时轰然崩溃,四散开来。

新的石猴王冲着方源龇牙咧嘴。

“滚。”方源盯着它,说了一个字,眼神寒冷如冰。

石猴王浑身一颤,真正感受到方源散发出来的恐怖杀机。它呆呆地看着方源一眼,旋即呜咽一声,转身而逃。显示出它超越其他野兽的灵智。

方源驱散了这群石猴,也不理会它们。而是抓紧时间走到石柱底下。

离得近了,他发现了石柱下的一个洞口。

洞口不大,一排石阶从洞口延伸往下,一直没入黑暗当中。

方源没有侦察蛊虫,自然不知道地洞下面有什么东西。

情况不明,方源没有进入地洞,走下石阶。他方才直闯进来,自身状态并不是很好。更关键的是,石林中的混乱正在消失,已经趋于稳定。

他花费了大量时间和精力,才打通的路线,已经有许多的石猴在路线上的石柱中定居。

“欲速则不达,找到了接下来的传承线索,就已经达到了目的。是时候回去了。”方源忍住一探究竟的欲望,按照原路返回。

一路上,前进的压力明显增大。但最终,方源顶住压力,被数百只石猴撵着跑,狼狈不堪地冲出石林。

时间匆匆,春夏交替。

不知不觉间,又到了炎炎夏日。

方源勤练不辍,抓紧每时每刻刻苦修行。赤铁舍利蛊的使用,使得他一下子就追上了方正的修为进度。

他没有特殊蛊虫,中阶的气息是隐藏不住的。在斩杀了石猴王,获得隐石蛊的一天之后,他的修为就被人发现。

族人这才知道,原来得到了赤铁舍利蛊的人,竟然是方源!

同时,方源也故意暴露出黒豕蛊。

方源为了购买黒豕蛊和赤铁舍利蛊,将那么一大笔遗产都给卖掉了。很多人都不能理解他的想法,一时间“大傻瓜”、“蠢蛋”、“疯子”、“目光短浅”成了方源的代名词。

关注度的上升,让方源不得不减少对花酒行者传承的探索次数。

他一面继续温养空窍,向着二转高阶迈出稳健的步伐,另一面收集酒虫以及隐石蛊晋升合练的材料,同时催生生机叶,赚取元石,维持修行。

七月,初秋。

在山脚下的村庄附近,一只野生的五转蛊虫忽然出现,引发了整个古月山寨的强烈震动!

\end{this_body}

