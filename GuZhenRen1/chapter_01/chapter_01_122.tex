\newsection{人生几多风雪}    %第一百二十二节:人生几多风雪

\begin{this_body}



%1
嗤嗤嗤!

%2
接连三片脸盆大小的月刃,在空中划出幽蓝的光线。

%3
嚓嚓嚓。

%4
顷刻之间,就有十六七只玉眼石猴当场死亡。

%5
追击方源的石猴群,顿时就少了一小半。

%6
方源站在原地,并没有后退,而是再次举起右掌,劈空三次。

%7
又是三道月刃,直直地撞入石猴群中,所到之处石猴翻倒一片。

%8
石猴的尸体摔在地上,摔成一块块的碎石。眼球所化的玉珠,也滚跳在赤红色的地上。

%9
方源查看了一下空窍,空窍中还剩下一大半的深红真元。

%10
月芒蛊的一片月刃,需要一成的浅红真元才能催动出来。方源若是二转初阶,只能连续催动四记。到了中阶,上升到八记。到高阶,数量再增加一倍,提升到十六记。

%11
方源虽然没有达到二转高阶,但是此刻有四味酒虫精炼出高阶真元,算得是伪高阶,因此战斗力暴涨。

%12
原先这七八十只的猴群追杀,他需要且战且退,但是如今单靠月刃攻击,就能迅速剿灭绝大部分。只余下十多只,往往又望风而逃。

%13
“不过才两天时间,就打通了三根石柱。这样的速度,比先前要快了数倍!这么算的话,一个半月后,我就能再次打通通往石林中央的道路了。”方源暗暗寻思着。

%14
“按照花酒行者的布置风格,石林中央的地洞,就是下一道关卡。在关卡之前,极有可能种下地藏花蛊。不过到了此步,估计花酒行者的力量传承也差不多了。毕竟他身受重伤,状态极为不妙,才仓促之间设下的这道传承。最大可能估计,往后也最多只有一两道的关卡了。”

%15
有着前世的丰富经验,方源又想到了当初影壁上,花酒行者那般浑身浴血,奄奄一息的景象,做出了判断。

%16
花酒行者布下这道传承,毕竟时间太短,没有办法做得更多。但这只是一个传承的特例。

%17
其实正常的传承,都要经过蛊师经年累月的设计和布置。有的规模宏大,有的每隔十多年才开启一次。有的在传承的过程中,地点并不是只有一处,而是分割成数个地点,分布在五湖四海,相互间距可能相隔天涯海角。

%18
后来者要继承这样的传承,非得经历十数年甚至数十年的时间,进行探索以及历经各种考验。

%19
一些传承,终其蛊师的一生,也未必能探索成功。往往就将这未完成的事业留给了后辈子孙。

%20
“花酒行者的这道传承,属于微型传承,有个缺点,就是传承的东西比较稀少。但也因此有个优点,那就布置的关卡往往因地制宜,因此相对简单。我从这道传承中,先后得到了白豕蛊、玉皮蛊、酒虫。隐石蛊勉强也能算上。接下来剩下的,恐怕只有一两朵地藏花了。但愿接下来的这些蛊虫中,有侦察类或者辅助移动的蛊虫!”时间匆匆,秋去冬来。

%21
初冬,第一场雪。

%22
天空阴沉,雪花飘飘,洒落在青茅山上。

%23
方源独自一人,行进在雪地中,他刚刚从石缝秘洞处出来,现在正往山寨中赶回去。

%24
“算算时间,已经过去了两个多月,但是我打通石林的进展,却一直不佳。”方源的眉眼中藏着一股阴郁。

%25
这并非是他不肯努力,而是狼潮的前奏,已经打响。

%26
冬天,寒风中食物稀少,日益壮大的狼群为了收集更多的食物果腹,开始大范围的猎食。

%27
这导致狼巢周围的野兽群,都被肃清。混乱不堪的小型兽潮频频发生,同时还有残狼潮。

%28
这些残狼,都是被狼巢淘汰出来的。为了生存,它们成群结队,开始在山寨周围显形露迹,频繁活动。

%29
虽然还没有达到冲击山村的猖獗地步,但是凡人猎户已经不敢上山打猎,同时村庄中不时的有村民被狼叼走而丧命。

%30
古月山寨为此,调动大量的蛊师,进行清剿。这样一来,来来往往的人多了,其中又有很多的侦察蛊师,这让方源不得不明智地,大量减少前去石缝秘洞的次数。

%31
这样一来,毫无疑问地,石林推进的速度就暴降了许多。

%32
呼呼的寒风越吹越大,雪势亦是在变大。

%33
吼……

%34
一声声低沉的兽吼,夹杂在风雪中忽然传来。

%35
方源倏地停下脚步,警惕地四处观望。

%36
一支小型的残狼群,大约有二十多头的电狼,很快就出现在他的视野里。

%37
“又来了么……”方源口中喃喃,这已经是他这个月碰到的第八次了。

%38
但是这一次,又有些不同。

%39
“靠着山寨这么近的距离,都有狼群在活动了。看来接下来,家族蛊师出击的次数会越加频繁。石缝秘洞又距离不远,看来接下来的这些天,我是不能再去了。”想到这里,方源心中不禁一沉。

%40
行路难,这世上总会出现一些风雪,阻挡住人们一时前进的脚步。

%41
狼群很快就包围了方源。

%42
吼吼吼!

%43
它们低沉地咆哮着,纷纷向方源扑杀过来。

%44
“月芒蛊。”方源心头一动,甩手一记月刃呼的飞射出去。

%45
幽蓝的月刃劈开风雪,连续斩中残狼,瞬间解决了三头,到了第四头时,残狼忽的就地一滚,狡猾地躲过了这记月刃。

%46
这些残狼,虽然大多数都缺胳膊少腿,瞎眼少尾等等,但是它们有着丰富的战斗经验,极为狡诈。

%47
若是一位普通的二转中阶蛊师,碰到这样的一群残狼,尤其是在被它们包围之后,性命就要受到强烈的威胁。

%48
不过方源却临危不惧。

%49
他丰富的战斗经验,以及四味酒虫精炼出来的高阶的深红真元,都是他的依仗。

%50
杀杀杀!

%51
他在残狼的围攻中,敏捷地腾挪,冷静地闪避,果断地出手。

%52
一头头残狼,死在了他的手上。

%53
很快,残狼群中电狼的数量,就锐减了一半。

%54
嗷呜——!

%55
一只头狼发出凄厉的嗥叫,狼群顿时收住攻势,开始撤退。

%56
这就是残狼的狡诈。

%57
一发现方源是个硬骨头,它们就果断撤退,放弃了围杀狩猎方源的打算。

%58
这些老狼、病狼、伤狼,虽然没有完美的身体状态,但是能生存到今天,都有一套生存的智慧。

%59
方源站在原地,静静地看着残狼群消失在风雪当中。自己的实力,能不暴露,就无须暴露。

%60
确认了狼群已经彻底遁走之后,方源这才蹲下身子,匆匆地采集狼尸上的东西。

%61
狼皮、狼牙等等,都是有价值的。

%62
虽然市价比较低,但也耐不住量多。

%63
方源这一两个月里,铲除残狼,靠着战利品也小赚了一笔。

%64
雪地上,狼尸流淌出来的血液,还带着温热。有了残狼,虽然倒在了地上,但还没有死透,狼眼中还残留着一丝的华彩。

%65
“在这自然中,万物都在抗争,都在生存,并不是仅仅只有人而已。这个天地,就是用生和死来演绎精彩的大舞台啊!”方源心中感叹着,毫不留情地给这些奄奄一息的残狼补上最后一刀。

%66
一只残狼的战斗力,就已经高出了两只玉眼石猴。狼群相互配合之下,战斗力又会提升一倍。

%67
“我对付这样的小型残狼群还好,要是对付大型的残狼群,或者今后对付一支小型的健康狼群,恐怕就要有麻烦了。”

%68
方源隐隐感到一股压力。

%69
“接下来,狼潮爆发,整个家族都要动员起来,我自然也不会独善其身。今后我独自一人,要在野外狩猎电狼,必须要有一只侦察类蛊虫,或者是辅助移动的蛊。否则的话,说不定就能陨落在这次狼潮当中。”

%70
越是经验丰富,方源越是能认清自己的不足。

%71
有了四味酒虫之后,他的战斗力已经暴涨许多。月芒蛊、白玉蛊在手,有攻有防,再靠着前世之积累,方源完全不属于青书、赤山、漠颜一流。

%72
可以说,他已经勉强站在了家族中,二转蛊师的一流行列当中。

%73
之所以是勉强,是因为他毕竟不是真正的高阶,同时资质也只是丙等,有限得很。

%74
战力这块已经做到了目前最好,但要在狼潮中生存下来,战斗力只是一方面罢了。

%75
“侦察手段不可或缺,若是我有一只侦察蛊虫,就可以提前感知到狼群的接近,而迅速远离,改变路线。或者利用蛊虫加速奔跑,脱离狼群的围杀,甩掉它们。”方源暗忖。

%76
只有有这两种蛊虫中的一只,他的生存几率就会大增。两只在手,基本上就能做到游刃有余了。

%77
“希望花酒行者的传承中,有这样的蛊虫。若是没有,也不打紧。按照记忆,每逢狼潮,三大家族会联手布置战功榜,抛售库存的蛊虫,其中就有许多较为珍稀的蛊。到那时,我甚至可以利用战功,兑换白家寨或者熊家寨里的蛊虫。”

%78
方源在心中谋算着,起了身。

%79
这一会儿功夫,他就将战利品都麻利地打理妥当,装在一个袋子里,背在身上。

%80
白雪纷纷扬扬,很快就将狼血冻结,将狼尸覆盖。

%81
“快看,是方源回来了。”

%82
“他背着袋子,又是出去狩猎残狼的么?”

%83
“就是他拯救了我们山寨?”

%84
“嘿,只是机缘巧合罢了。过程我们都清楚,如果我有那么大的力气,也足以做到。这不算什么。”

%85
方源走入山寨,一路上行人们纷纷侧面,有的赞颂,有的好奇,有的嫉妒。

%86
“方源。”赤山忽然出现在一处拐角处,叫喊了一声。

%87
(ps:昨天重新立大纲,花费了许多时间。过年嘛,应酬也多,你们懂的。已经是很抓紧时间了。有人反映看到了不同的第121章节,标题是“地听肉耳草”的,那是我之前的存稿,现在已经删掉了的。现在立的大纲,更紧凑了,情节也通畅了许多。衷心谢谢大家的支持!接下来,情况会变好的。)(未完待续!\~{}!

\end{this_body}

