\newsection{再次重生}    %第一百九十六节:再次重生

\begin{this_body}

方源立即催动锯齿金蜈,以及血月蛊,想要破冰而出。44rc。m8 9 阅 138看书网

奈何冰层深厚至极,寒气弥漫浓郁,削去一寸冰霜,就凝出两寸。方源被困绝境,无法脱身。

恰又在此时,冰川之下忽的冒出血光。

血光起初只是一抹,旋即扩散,越来越盛,形成血色霞光,泛滥一片。

“哈哈哈。”长笑声中,血光冲天,古月一代破冰而出。他虽然狼狈,却更显猖狂,“这个北冥冰魄体,修为若达到四转,说不定就能将我镇杀了。可惜啊,他只有三转巅峰……”

说完,他就将目光集中在冰川中的方源身上。

“北冥冰魄体杀不了你,那师弟我就代劳吧。”远处天空,天鹤上人悠悠飞来。他座下的是那只铁喙飞鹤王。之前他飞天远去,相助铁喙飞鹤王斩杀了血河蟒,此时挟持胜势而来。

此刻,整个青矛山,就只剩下这三人。

白凝冰已经化为这片冰山,意识也消散殆尽。方源如琥珀中的昆虫,被困在冰川当中。

反观两位五转蛊师,古月一代提升了资质,空窍中存储的真元更多。天鹤上人也休养过,战力恢复大半。

两个人的目光,都集中在方源的身上。

古月一代要杀了方源,尽取其血。天鹤上人要阻止古月一代,自然不会保护方源而导致自己束手束脚,因此他只有先下手为强,将方源先杀掉。

以他俩的心性,更不会容忍一个旁观者。万一两败俱伤,被第三人捡了便宜去呢?

方源长叹一声,看到这两人的目光,就知道自己已经必死无疑了。

他此刻失了雷翼蛊,千里地狼蛛,就算是有,也未必能逃得过两位五转强者的追杀。

他只有三转巅峰修为,和五转强者根本不能相比。此般情景,他如鱼肉,人为刀俎。还是两把锋利无双的刀刃!

战又战不过,逃又逃不走。但方源还有一个法门!

那就是——春秋蝉!

方源将心神投入空窍,空窍中光膜不再,只有一片粗糙的石窍。

雪银色的真元海,倒还残留有大半。石窍已经不具有恢复真元之能,但方源有天元宝莲,因此才有真元海的这般景象。

这些都不是关键,方源将注意力都集中到石窍最中央的那只蛊虫身上。

那是他的本命蛊,高达六转的春秋蝉!

唯有靠此蛊,逆流光阴之河而上,才可再创奇迹!!以不可能的手段,来篡改命运结果!

然而——

行此手段,凶险异常。

首先,光阴之河乃是大道禁区,凡人不可涉及。一旦侵入其中,等若触犯天地法则,必遭受天谴地灾。

其次,春秋蝉还未恢复完全,如一艘破船漏船,强渡光阴之河,说不得半道就要倾覆,沉没。

最后,方源不过区区三转巅峰,驾驭六转春秋蝉,简直是婴儿耍大刀,耍不好就要被刀锋所伤。

“一旦我用了这春秋蝉,就要自爆。以全部修为,一身皮肉精血,所有其他蛊虫,都要毁灭,化为一股动力,推动春秋蝉前进。和前世相比,我就算自爆,这股力量也太小了。唉,很大可能就是直接自杀。但此刻此时,我已走投无路,非得用此蛊不可了!”

方源也是万般无奈。

先前,他宁愿动用石窍蛊,也不想动用春秋蝉,就是因为太冒险了。

十成中,成功几率未必有一成。

很多时候,蛊师催动高转蛊虫,都要遭受反噬的代价。好比古月青书。现在方源也只好寄希望于“春秋蝉是本命蛊”这点上。

“小鬼头,把命拿来,给你的老祖宗贡献一份血力!”

“小子,你命不好,怪只能怪你摊上了这个卑鄙的祖宗。我来给你解脱了罢!!”

古月一代、天鹤上人同时扑杀过来。

方源被逼入悬崖边缘,他只有纵身一跳。

“春秋蝉,来吧!”他眼冒奇光,呐喊一声,身上猛地爆发出青黄二色。

“这样的气息?!”

“怎么可能?竟然是六转蛊虫!”

这一刻,两位五转蛊师惊诧万分。但旋即,贪婪的神色涌现在他们的脸上。

“杀了他,取得六转蛊!”

“这是天降机缘,好小子,乖乖贡献出来,绕你一命!”

他们速度更快三分。

但就在此刻,轰的一声。

方源自爆!

“什么?!”在临死之前,他仿佛听到两位五转蛊师的惊呼声。

传闻中,这世界有一条长河,名为光阴!人若河中之鱼,河流湍急,几乎所有的鱼都只能顺势而下。偶尔,有一两只鱼偶尔跃出河面,看到下游情景,就是预知未来。

若无这光阴之河,世界将完全静止,成为画面。有了这河,一切才可变化,世界才能生动,或是衰减或是繁华。

光阴长河,江水滔滔。每一滴浪花,都是一股故事,一个曾经发生的画面。

在湍急的河水当中,一只小小的蛊虫,正在逆流而上。

它振奋双翅,举步维艰。澎湃汹涌的浪潮,每一次拍击过来,都让它险险倾覆。

它载着方源达到意识记忆,身上绽放着一圈黄绿相见的微光,光芒在潮水中摇摇晃晃,如风中残烛。

终于,它只逆流了微小的一段,黄绿微光几乎消散不见。春秋蝉达到了极限,嗖的一下,化作一道光辉,钻入到一朵浪花当中。

方源浑身一抖,双眼瞳孔深处闪现出一抹黄绿之芒。

这光芒一闪即逝,方源如打了一个寒颤。

意识和记忆冲击他的脑海,并在瞬间交融在一起。

成功了!

他心中一阵狂喜,自己又重生了!

意识到这一点后,他立即目光扫视四周,观察自身处境。

他发现自己真元不断消耗,手腕、身躯都被白眉缠住。

再一看,哦!

原来是这个时候。

铁血冷的后手布置,已经发动。古月一代全身都被铁索缠绕,动弹不得。额头也贴着一张黄符,真是镇魔铁索蛊,以及符底抽薪蛊。

而那天鹤上人也落在地上,身上罩着一个白色光圈,正全力催动扬眉吐气蛊,企图耗尽古月一代空窍中的全部真元!

方源双眼眯起来,此时他被白眉缠住,动弹不得,只能对耗真元,等待良机。

在符底抽薪蛊的作用下,一团黄光从古月一代的体内飞出来。

这黄光比拳头更大,比脸盆要小。悠悠地飘落到地面上,显露起光团中的血颅蛊。

“血颅蛊啊!时隔数百年,我终于又见到你了!”天鹤上人远远看到,喜极而泣,神色激动至极。

古月一代急得把满口獠牙咬得咔嚓作响,却万般无奈,被五花大绑,动弹不得。

又有一团黄光从黄色符面拘拿出来,落在地上,化为一枚黑白相间的太极光球。

这光球中,两只奇特的蛊虫相互盘旋,你追我逐,乃是阴阳转身蛊。

按照方源的记忆,天鹤上人在那大叫:“阴阳转身蛊!!我的好师兄,真难为你寻得如此好蛊。哈哈哈,你居然还想转身成人,可惜了,真是遗憾啊,被我破坏了!”

方源再看古月一代。

果然见他坐在地上,急得乱蹬腿,连连嘶吼,披毛散发,完全失态了。

“再等等,时机不远了。”方源眼中精芒闪烁,按兵不动。

第三道黄芒拘出来,落在地上,是一只猩红色的蛊,半透明如水球。

方源心头一震:“血幕天华蛊!”

正是此蛊,改变了局势,让古月一代翻身。

血幕天华蛊乃古月一代开发,方源先前也不认得,如今却对它的特性清楚无比。

果然,紧接着,古月一代就大叫道:“快快来人,斩掉这蛊虫!”

几位蛊师闻言,立即上前,把这蛊击毁。

嗡!

一声轻吟声响,血幕天华“再”现天地。血色球罩隔绝内外。一部分人被隔在罩子外,另外一部分则身处罩内。

血罩隔断了白眉,方源等人不再和天鹤上人对耗真元,一下子脱困,重获自由身。

天鹤上人被这护罩阻隔,踉跄地站起来,冷笑不断。

一番对话后,他攻打血罩未果,只得停下手来,问道:“这是何蛊?”

古月一代得意洋洋“好教你知,这是我独家合炼而出的血幕天华蛊。水幕天华只有四转,能挡五转蛊攻伐。我这蛊高达五转,比之防御更坚。水幕天华蛊连其主人都不能进出,我这血幕天华蛊,我却能出不进。师弟,你慢慢打,用力打。待我恢复好了,再出来收割你的狗头,啊哈哈!”

天鹤上人大怒,再次攻打,又不成。冷静下来,他选择就地补充真元,等待血罩时效耗尽。

众蛊师见血罩稳如泰山,均大喜过望,觉得自己保住了性命,纷纷对古月一代大拍马屁。唯有白凝冰冷哼。

古月一代顺势要求大量元石,众人纷纷慷慨解囊,贡献出来。

古月一代虽被铁索绑着,行动不便,但血盆大嘴咬碎元石,真元补充得极为快速。

天鹤上人看到此处,大叫:“你们这群蠢货!他一恢复行动,就要杀掉你们,以你们的血来洗练提升他的资质。你们这是在自寻死路!”

众人自然不信。

“哼哼,这等低级的离间计还使用出来,不怕人笑掉大牙嘛!”

“快快快,这老贼叫我们不要给,我们更要给元石了。”

“一切都依赖古月一代大人了!”。

\end{this_body}

