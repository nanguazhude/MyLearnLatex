\newsection{哥哥,你怎么能这样对我?!}    %第一百七十七节:哥哥,你怎么能这样对我?!

\begin{this_body}

%1
生铁蛊是二转蛊,形如煤球,拳头大小,漆黑一片,表面还有无数细小的孔洞。

%2
方源将白银真元灌注进去,这生铁蛊就悬浮起来,悠悠自转,才孔洞中喷涌出黑雾似的铁气。

%3
锯齿金蜈盘曲在方源的脚边,它暗金色的背甲伤痕满布,两排银边锯齿也破损不堪。

%4
但铁气涌动过来,覆盖在这些伤痕上,竟渐渐地将伤痕消抹。

%5
黑雾铁气不断消耗,锯齿金蜈的两排锯齿,也在以肉眼可见的速度生长出来。

%6
治疗蛊分门别类,有些治疗蛊师,有些专门治疗各种伤势,有些则针对蛊虫。

%7
对于锯齿金蜈来讲,生铁蛊就是它的治疗蛊。

%8
半个时辰之后,生铁蛊越来越小,从拳头大小的煤球状,削减成弹珠大小,最终彻底消散。

%9
它是消耗类的蛊。

%10
但它的牺牲,换来了锯齿金蜈的恢复。

%11
此时的锯齿金蜈,仿佛焕然一新。两排的银边锯齿已经崭亮如新,闪耀着凌厉的寒芒。暗金色的背甲处,伤痕也平复了绝大多数。只有五六道浅浅的伤痕,仍旧残留着。

%12
但这已经无伤大雅,过不了几个星期,这些伤痕也会因为锯齿金蜈本身的康复能力,而消失不见。

%13
不过,如果没有生铁蛊,但靠锯齿金蜈本身的恢复力,恐怕至少得有大半年,才能令锯齿全部长全。

%14
锯齿金蜈刚强而少柔,虽然有消耗真元少,攻击力强等优点,但过刚亦折,在恢复方面有着缺陷。

%15
万物平衡,这个世界上没有全方外强盛的蛊,有优点必有缺点。哪怕是六转、七转等等以上的蛊,亦都遵循着大自然的这道法规。

%16
“这样一来锯齿金蜈的战力算是完全回复了……”方源伸手轻轻抚摸着锯齿金蜈冰冷的甲壳,脸色有些苍白。

%17
他苍白的脸上,隐现冷汗。

%18
“可恶,居然这个时候……”方源咬牙,左手下意识地捂住自己的腹部。

%19
心神投入空窍只见白银真元海死寂一片,整个空窍中充满了压迫力量。

%20
其他所有的蛊,都被狠狠地镇压在一旁。唯有海面上空,空窍中央的春秋蝉,绽放着时黄时绿的绚丽光辉。

%21
此刻的春秋蝉不仅双翅都已经恢复,而且就连主躯干也增添了许多光泽。

%22
就像是高空坠物,越到下方,坠物的速度就越快。春秋蝉的恢复速度,也同样如此。渡过了前期艰难缓慢的时刻,时间越往后推够,它恢复的速度就越快。

%23
因此,麻烦就来了。

%24
春秋蝉高达六转,而方源不过是三转蛊师,他的空窍渐渐难以装载春秋蝉。

%25
以前春秋蝉虚弱无比,空窍的负担并不重。但如今,春秋蝉渐渐恢复,展露出六转蛊的强势,使得方源的空窍如小庙难容大神!

%26
“这样下去,说不定在铁家父女查出真相之前,我就要被春秋蝉撑破空窍死亡了!真是屋漏偏逢连夜雨……”

%27
真正解决之道,就是尽快地提升自己的修为。当他成为六转蛊师时,空窍就有能力来装载春秋蝉了。

%28
但这个办法,太过于漫长。方源前世五百年,总共用了四百多年,才修到六转。

%29
他现在是丙等资质,三转修为要修行到六转,严重缺乏时间。

%30
除此之外,还有一个暂时的解决之道。

%31
就是将春秋蝉调出空窍,放养在体外。

%32
但这举措,也有极大弊端。

%33
首先春秋蝉并非战斗蛊无自保之力,不如藏在空窍中安全。其次六转蛊虫一旦出现,都会干扰法则,形成一定区域内的天地异象。只要在某个地区逗留时间一长,就会有蜂拥而至的蛊师强者被吸引过来。

%34
方源现在身处在山寨当中,人多耳杂,又被铁家父女盯住,这春秋蝉一放出体外,几乎立即就会被外人察觉。

%35
这样一来,他只能苦挨。

%36
“春秋蝉恢复速度越来越快,照此下去,恐怕我没有多少时间了。等到古月漠尘那边的四万元石到手,就摘下天元宝莲,离开这里。铁家父女那边,也只能走一步算一步了。”

%37
方源心中叹息。

%38
铁家父女的事情,他只能拖延。但现在,春秋蝉却不给他拖延的时间。

%39
他已经被逼入绝境,时间之紧迫,浪费一分一秒,都是在减少他的生机。

%40
蛊师被自身蛊虫害死的情况,绝不是少数。很多蛊师强行运用,受到蛊虫力量反噬而丧命的例子,比比皆是。远的不说,古月青书的例子,就近在咫尺。

%41
“六块紫金石,各个都有拳头大小。以方源当时的修为,居然当场连续解开了五块。他怎么会有这么多的真元呢?”铁若男的目米牢牢地锁定在信笺中的相关一行上,得意地笑起来。

%42
铁血冷点点头:“你终于发现了这里的疑点,不错,细心留意才能发现常人看不见的东西。但是你又能从这个疑点中,推断出什么东西来呢?”

%43
铁若男闭上双眼,暗中催动直觉蛊。

%44
在黑暗中,她就觉得脑海中灵光一现。她猛地睁开双眼:“直觉告诉我,方源也许早就有酒虫了!”

%45
“但直觉有时候也是错误的,它并不能代表证据。”铁血冷提醒道。

%46
“要证据还不容易?呵呵,只要他有酒虫,就得去喂养。只要喂养,总会留下蛛丝马迹。铁若男的嘴角渐渐弯成一个弧度“走!我们再去找古月方正,作为弟弟,他应该最熟悉方源不过了。”

%47
“你问哥哥以前的事情?”方正脸色流露出复杂之色。

%48
他叹息一声,回忆道:“哥哥以前,就是很优秀的人。他从小就展现出才情,作了很多诗,令整个山寨都侧目关注。那时候,我敬畏他,敬佩他。在我心中,他就像是一座我无法攀越的高山口可能是站得高,摔得疼吧。在后来的资质大典中,他被测出只有丙等资质,因此颓废了很长一段时间,上课时都在睡觉,晚上都夜不归宿,整天买醉。从那刻起,我才明白,原来哥哥也是人……”

%49
“等等,你说买醉?”铁若男敏锐地察觉到这个关键词,她的双眼眯起来。

%50
“是啊,有一段时间,他酗酒烂醉。唉,可能是现实太残酷了。自己是丙等,而亲弟弟却是甲等,接受不过来吧。其实设身处地地想一想,我也能理解他的心情和感受。”方正道。

%51
“那我问你,从那时候起,方源都隔一段时间,都会买酒喝吗?”铁若男又问。

%52
“是啊。从那刻起,哥哥他就爱上了杯中物,为此耗费了不少钱财。有段时间,他迷上了青竹酒。那可是我们山寨特产,很贵的一种酒。他咨意抢掠同窗的元石补贴,就是为了买酒喝。这种行为真的是很过分,以至于没有一个学员喜欢他的。

%53
怎么,这里面有什么问题吗?”在最后,方正表示了疑惑。

%54
“有大大的问题。我现在怀疑,你哥哥的酒虫并非是赌石赌出来的,而是原来早就拥有。你哥哥烂醉颓唐,只是他的一场表演。真实的目的,在于掩盖自己得到酒虫,喂养它的事实。”铁若男沉声答道。

%55
“什么?!”方正闻言,吃惊地从座位上直起身。

%56
这消息太让人意外了!

%57
“你刚刚的一番话,更令我怀疑加重了许多。你哥哥平时在哪里买酒?我需要再去调查一下。”铁若男也站起来,她争分夺秒,干劲十足,行事雷厉风行。

%58
“青竹酒整个山寨,只有一家售卖,就是唯一的那家客栈。”

%59
“那先告辞了。”铁若男转身就走。

%60
“等一等,我……我和你们一起去!”方正犹豫了一下,追了上来。

%61
半个时辰之后。

%62
铁若男走在青石街道上,总结道:“刚刚已经问明白客栈掌柜,情况已经很显然了。方源买这么多的酒,是有他深层次的目的。那就是喂养酒虫。在之后,他故意去赌石,就是将酒虫合理地暴露在众人眼前。这一切,都是他计划安排好的。”

%63
一旁,古月方正有些失神落魄地走着,脸色显得有些呆滞。

%64
他没有想到,真相竟然是这样的!

%65
曾经一度,他看不起方源,认为他颓废,自甘堕落。从那刻起,他觉得昔日的高峰,不再高不可攀。

%66
但真实的情况是,这一切都是方源他的伪装,他的表演,他的布局!

%67
周围的人,都被他蒙在鼓里,骗得团团转。

%68
而他古月方正也不例外!

%69
昔日对哥哥的看不起,轻视,现在看来,就好像是个充满嘲讽意味的笑话。

%70
“哥哥……在你的心中,我算什么呢?你假装烂醉的眼眸中,我就是一个笑话吗?哥哥!你是如此的心机深沉,在你心中,我是不是幼稚的让你冷笑呢!”方正在心中咆哮。

%71
他羞恼,他愤怒。

%72
他感觉自己就像是被方源玩弄戏耍着,一直以来,表演着可笑的幼稚的话剧。

%73
他感到方源对他的不屑。

%74
“哥哥,你怎么能这样对我?!”

%75
“如果不是铁姑娘,我还被你蒙在鼓里。你到底要欺骗我,欺骗族人到什么时候?你滥杀无辜,草管人命。欺骗和谎言,冷漠和残酷,这才是真实的你吗?”!!!

\end{this_body}

