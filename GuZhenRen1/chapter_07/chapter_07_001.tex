\newsectionindepend{狂蛮传1}    %第一节 狂蛮传1

\begin{this_body}

%1
一,人族奴隶

%2
秋日正要落下,天边残霞如血。

%3
一支长长的队伍,走在莺飞草长的草原上。

%4
啪、啪、啪!

%5
带着倒刺的长鞭,在空中挥舞,击打空气,每一击都发出尖锐的声音。

%6
握住长鞭的,是一位猪头人身的兽人,两颗尖锐的牙齿向外突出,黑色的猪毛覆盖满脸,目光凶神恶煞。

%7
他张开口吼叫着:“快走,你们这些低贱的人族奴隶!”

%8
被他所驱赶的人族,有高矮胖瘦,亦有男女老少,不一而足。

%9
“走快一点,再走快一点,给老子动起来!嗯?!”猪头兽人忽然双目圆瞪,杀气四溢,看向一位跌倒在路边上的老人。

%10
“臭老头,居然敢停下来,不把我的话听在心里,找死!”猪头兽人大怒,皮鞭狠狠抽下。

%11
啊!

%12
老人发出一声凄厉的惨嚎,血光四溅,背上大半的皮肉都被皮鞭搅碎,四处飞扬。

%13
伤口深得能见到惨白的脊椎骨头。

%14
“哼!”猪头兽人冷哼一声,手臂一挥,皮鞭又袭来。

%15
像是灵动的毒蛇,皮鞭落到老人的身上,直接将他卷起来。

%16
随后,猪头兽人轻轻一抽手臂,将老人整个都抽回到战车上,落到他自己的脚边。

%17
然后,他抬起粗硬的猪蹄,狠狠地踩在老人的头颅上。

%18
啪。

%19
一声轻响,就好像是西瓜碎裂一样,老人的头颅被踩碎,脑浆、鲜血还有半颗破碎的眼珠,混杂在一起。

%20
几点鲜血飞溅到猪头兽人的嘴边,他微微张开嘴,伸出粗苯的舌头,舔了舔血渍,随后双眼微微眯起,露出享受的神情。

%21
在他的战车周围,许多凶恶勇猛的兽人,闻着血腥气味,都开始骚动起来。

%22
“嘿嘿,人族就是皮细肉嫩。这老货虽然年龄大了点,但也有一些嚼劲。朱爷我赏赐给你们了!”

%23
说着,猪头兽人一脚把老人的无头尸体,踢到战车外。

%24
啊吼!

%25
下一刻,十多位强壮的兽人,哄抢在一起,一阵抢夺之后,老人被当场分尸,其状惨不忍睹。

%26
一阵狼吞虎咽,老人就这样被充当食物,填入了兽人的肚子里去。

%27
兽人们兴奋地嚎叫起来,一些狼头兽人更是长鸣不止,另外一些没有来得及抢到嘴的兽人们,则不甘心,眼巴巴地望着他们的头领朱爷。

%28
猪头兽人朱爷不满地冷哼一声,眼中闪烁着狡诈的光。

%29
他心想:“这群没脑子的东西,真当我朱爷慷慨无私吗?这老货已经病入膏肓,活不长了,所以才废物利用。杀了他,是为了震慑其他人族奴隶。若非如此,一个人族,卖到主城中,至少得有百块元石啊。”

%30
想到这里,朱爷开口大吼:“快走,谁走的慢,拖了后腿,朱爷我就把他(她)当场吃掉!”

%31
此言一出,几乎所有的人族俘虏,都是心头一颤。

%32
整个队伍的行进速度,顿时上涨了许多。

%33
朱爷满意地暗中点头:“按照这种速度,天黑之前就能到达主城了。不过,还是要小心人族蛊师的突击队。最近几年,人族越来越嚣张了!捕奴贩卖的生意,也越来越难做。”

%34
队伍向主城接近,行进了半个时辰之后,道途上出现了两位羽民。

%35
朱爷从这两位蛊师身上,感受到了五转的气息,他顿时面露恭谨之色,主动招呼道:“两位大人胃口好。”

%36
祝福朋友和贵客胃口好,这是兽人打招呼,表达善意的方式。

%37
两位羽民其中一位点点头,背负双手,态度倨傲,貌似是主,另外一位,态度则平和许多,问道:“兽人的朋友,你是要去灰石主城吗?”

%38
“是的。”朱爷连忙点头,整个队伍都因此暂停下来。

%39
搭话的那位羽民笑了:“正好我们也要去灰石主城,一起走吧。”

%40
朱爷大喜:“有两位加入,是俺求之不得的好事哩,快请上来坐。”

%41
说着,朱爷主动挪动他那肥大粗壮的身躯,好不容易在战车上让出一些空间来。

%42
两位羽民蛊师却嫌弃战车上环境的肮脏,他们展开翅膀,飞在半空中,一路跟随。

%43
朱爷自然知晓羽民的秉性之一,便是好洁。

%44
他尴尬地笑了笑,仍旧十分热情地来套近乎。

%45
他自己只是三转修为,若是能搭上四转、五转蛊师的人脉,必将带给他飞黄腾达的可能。

%46
这可是难得一遇的机缘啊!

%47
五转气息的羽民,仍旧是一副高傲的样子,并不开口。

%48
和朱爷交流的,是那位看似仆从的四转羽民蛊师。

%49
“老朱啊,你这趟货挺多啊,赚头应该不少吧。”四转羽民蛊师随口道。

%50
朱爷笑了笑,得意地道:“不是小的吹啊,论及灰石主城中的奴隶商,我朱爷算第二,没人有资格称第一。”

%51
不过很快,他笑声就收敛起来:“可惜,这些年奴隶生意越来越不好做了。人族那边强者越来越多,不仅防范有加,甚至还会反攻我兽人一族的驻地。我这批货,是好不容易得到了线报,穿越了上千里地,才偷袭了一个人族的村庄。”

%52
“当然,这一次回到主城,卖出去,一定能引发轰动的。因为主城缺少人族奴隶,已经很长时间了。”

%53
“老朱本领不弱。”四转羽民蛊师随意夸赞了一声。

%54
猪头兽人谄媚得笑出声,神态能让人作呕:“哪里,哪里,和二位大人比较起来,小的只是赚一点辛苦钱而已,过得都是把脑袋提在裤腰带上的苦日子。”

%55
羽民蛊师点点头:“这年头,战火纷飞,谁不是过着朝不保夕的日子呢?”

%56
老朱深深地叹了一口气,极为共鸣地道:“大人说得是啊,说到老朱我心坎子里去了。”

%57
“老朱,你对人族怎么看?最近几年,有不少流言,说人族将会是天下之主,把其他所有的异人种族,都将镇压下去。”羽民蛊师问道。

%58
“屁!”猪头兽人立即冷哼一声,神情非常不屑,“就凭这些脆弱懦弱的东西?在过去,他们连给我们***的资格都没有!”

%59
羽民蛊师眯起双眼,平静地道:“可是如今,在中洲,人族已经成就了霸权,中洲的各大异族,都被人族一力打压。”

%60
“哼,那是中洲,不是北原。在北原,还是我们兽人说了算。这仗打了多少年了,人族还不是这个样儿?中洲那是例外,你看西漠、东海、南疆,不都是咱们异人说了算吗?”猪头兽人反驳道。

%61
羽民蛊师点点头:“你说的有点道理。那么老朱,在你心中,人族强大吗?你觉得该用什么法子对付他们效果最好?”

%62
“当然是拳头!”老朱狠狠地挥了一下拳,不假思索,“这些东西,就该狠狠地教训,让他们知道,什么是实力和底蕴,异人就是高人一等。人族标榜自己是万物之灵,哼,这算个屁。我们兽人的勇武,羽民天生就能飞翔,鲛人能自由入海,他们人族身板太弱,算个屁。”

%63
“可是人族一旦修行有成,蛊师强者并不容易对付。”羽民蛊师又道。

%64
“是这样的,不过人族还有一个巨大的弱点,嘿嘿,那就是人族怕死啊。我们兽人勇武,悍不畏死,但人族却懦弱似鼠,这个种族怎么可能成为天地霸主?大人不信,我随便挑一个人出来。”

%65
老朱说到这里,鞭子一卷,随意挑选出了队伍中的一个人族少年,拽到了车上。

%66
“说,你想死,还想活?”朱爷瞪着双眼,对这少年吼道。

%67
少年满脸惨白之色,差点被吓傻,反应过来后,连连点头:“想活,想活啊。”

%68
“想活就给我磕头,叫我三声爷爷。”

%69
“爷爷,亲爷爷,好爷爷!”少年磕头不止。

%70
“哈哈哈!”朱爷大笑,对那两位羽民蛊师道,“大人你看看。这样的人族,怎可能战胜得了我大兽人部族呢?!”

%71
四转羽民蛊师一阵无语,只得点头。

%72
倒是那位五转蛊师,看了一眼人族少年,忽然打破沉默,开口问道:“人族的少年,你叫什么?”

%73
“我、我、我叫乖。我是最乖的,最听话的,求几位大人不要吃我,不要吃我啊……”说着,竟然是涕泪交流,懦弱至极。

%74
五转蛊师收回目光,继续保持沉默。

%75
朱爷一脚将少年踢到地上去,得意地命令道:“给我好好跟着,你这么乖,待会到了主城,朱爷我一定把你卖出个好价钱,哈哈哈!”

%76
“是,是,是,谢朱爷,谢朱爷!”人族少年乖诚惶诚恐地应声。

%77
这一下,那位四转羽民蛊师彻底对人族少年失去了兴趣。

%78
而另一位五转羽民,却仍旧将目光投注在少年身上,他问道:“我听说,最近你们人族出现了一个流言。流言中说,人族中将会出现一个王,带领你们人族,成就天下霸业,将各大异族都踩在脚下。这个王的名字,就叫做——狂。有没有这个流言?”

%79
不待少年回答,朱爷的鞭子一把抽过去,擦着少年的脸颊,险之又险地抽到地面上。

%80
朱爷大吼:“大人问你话,你要实话实说,敢有半句虚言,直接把你吃了!”

%81
那人族少年吓得身躯直颤,面上毫无血色,结结巴巴地道:“有,是有这个流言。但,但,但我从来是不信的……”

%82
“啊?还真有这个流言啊。这些人族本事没有几个,倒真是能够奢望。这就是懦弱者的表现啊。啊哈哈哈。”朱爷大笑起来,对这个流言十分不屑。

%83
队伍持续前行,一路上,猪头兽人极力巴结两位羽民蛊师,对羽民蛊师的提问,可谓知无不言言无不尽。

%84
快到灰石主城的时候,那位五转羽民蛊师终于说话:“我们先走一步。”

%85
老朱吃了一惊,忙道:“两位大人这就走吗?何不随小的一同入城,小的也算是熟悉灰石主城中的情况,愿为二位向导。”

%86
那位五转羽民蛊师却不搭理他,径直飞走。

%87
四转羽民蛊仙,倒是临走之前,对老朱笑了笑。

%88
老朱站在车上,一直目送两位羽民蛊师飞走,这才换了一副颇为复杂的表情,小小的切了一声,嘟囔道:“有什么了不起的,不就是修为高了点吗?拽得二五八万,我兽人一族的蛊师强者,远比你们羽民要多得多!”

%89
继而声调一扬,手中皮鞭高高抽起来:“还不快走,你们这些低贱肮脏的人族奴隶!”

%90
老朱却不知道,那两位羽民蛊师离开了车队之后,竟从身上逸散出蛊仙的气息来。

%91
那原本的五转蛊师,此刻流露出七转气息,那位四转羽民,从气息上判断则是六转修为。

%92
七转羽民蛊仙叹道:“见微而知著,云理理,你也看到了兽人现在的情况。他们仍旧自大、粗鲁,人族日益强大,他们却对如今种族面临的危情,视而不见。”

%93
云理理点头,面色凝重:“郸大人,在下认为:兽人的狂傲,是由来已久的历史,毕竟他们一直都是天地间的霸主。不过那头老猪,只是兽人底层的见识,我们这一次会见灰石城主,他是蛊仙,眼界应当更高的。”

%94
“希望如此罢。如今中洲已经失陷,天庭建立,横霸中洲。人族的大势,已经渐起。我们各大异族必须联手起来,趁着人族中没有涌现出九转的蛊尊,将人族打压下去。否则的话,一旦再出现一位九转蛊尊,那结果不堪设想!”七转羽民蛊仙郸沉郁忧心忡忡。

%95
一提到九转蛊尊,云理理也不由地心头一沉。

%96
他握紧双拳:“怎么唯有人族,能涌现九转的存在?我们异人历史上,却始终没有一个出现!”

%97
“唉……蛊是天地真精,人为万物之灵,这句话不假。这种灵性,兽人有时候会误以为懦弱,就像刚刚的那个普通的人族少年。他委屈求生,知晓进退,这不正是灵性的表现吗?正因如此,他换来了宝贵的生存的机会。而只要生存下去,就会有希望,不是吗?”郸沉郁道。

%98
云理理点头,他更关心的是另外一个话题:“关于人族狂王的预言,如今已经泛滥到了人族底层中去了。若真的出现一位人族的狂王,他是否就是郸大人你推算出的,人族将要诞生的第四位尊者呢?”

%99
郸沉郁想了想,沉吟道:“有这样的可能。这个预言的源头,乃是人族八转智道大能弃思仙。此人实力深不可测,曾经一度被人族蛊仙公认为智道第一,他在临死 之前,甚至继承了星宿仙尊的真传。不过幸好,我们异族蛊仙早就得到情报,联手出击,不计牺牲,终是将此人成功围杀。否则的话,留着此人在人族内部,不知会 搅起多少的风云,给我等异族带来多少的损失。”

%100
云理理闻言,面色凝重:“那我们接下来,要密切关注狂王的出现了。”

%101
“嗯,蛊尊没有成长起来,就有着提前扼杀的可能。我们还可从另外一个角度出发,提前发觉人族蛊尊。”

%102
“每一代蛊尊的成长历程中,都有一个共同点,那就是拥有一个护道之人。护道人的修为往往高超,为年轻的蛊尊遮风挡雨。若狂王出现,并且他的身边拥有护道人似的角色,那极可能就是未来的蛊尊了。”

%103
郸沉郁说道这里,便住了口。

%104
因为兽人蛊仙灰石城主,已经提前出来,迎接他们。

%105
三位蛊仙见面的时候,在灰石城外距离不到十里的地方,一场屠戮已经结束了。

%106
之前凶残的兽人蛊师们,已经死伤一片,只有老朱等极个别的蛊师强者,正在拼死防守。

%107
他们的敌人只有一个。

%108
这是一个年轻的人族少年,浓眉大眼,虎头虎脑。

%109
老朱等人却害怕到了极致,结结巴巴地吼道:“你,你是人族的战士蛮!你,你居然敢深入我兽人部族的腹地,在灰石城外截杀车队?!”

%110
“这有什么?”蛮面无表情,迈出脚步,就要逼近。

%111
在他身后,被他解救的人族奴隶们,无不为他呐喊欢呼。

%112
但就在这时,一支骑着狼的兽人蛊师陡然出现,并且快速袭来。

%113
蛮皱起眉头。

%114
带这些人族走,完全没有问题,但是老朱的身后,还有一个人族少年没有被他拯救。

%115
他身后的那些人族慌张了。

%116
“蛮,我们快走吧。”

%117
“不要救他了,他一条命,我们这里上百条命啊!”

%118
“那个少年乖,是个小偷,不是我们本村的人,偷我们的东西。”

%119
“没错,还摇尾乞怜,之前跪在兽人蛊师的脚下,像是一条狗!”

%120
人族俘虏们大叫、咒骂。

%121
蛮冷哼一声,瞟了老朱身边的少年一眼,带着厌恶:“懦夫!”

%122
然后,他转身便走。

%123
人族俘虏们紧随其后。

%124
老朱等残兵败将,完全不敢追击。

%125
只能看着蛮带着他的财富,迅速地撤离,又看着灰石城的兽人精锐蛊师们,骑乘着恶狼,追杀过去。

%126
“可恶,这次损失大了!”

%127
“赔本,朱爷我要亏死了!!”

%128
“怎么只剩下你这个玩意儿?你只是一个人,我能卖多少钱?”

%129
回过神来,老朱望着身边的人族少年乖,杀气四溢,气急败坏。

\end{this_body}

