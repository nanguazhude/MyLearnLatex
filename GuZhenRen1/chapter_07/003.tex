\newsection{狂蛮传3:赚了}    %第三节 狂蛮传3:赚了

\begin{this_body}

%1
灰石城坐落在北原东南部,是方圆百万里内最大的城池。

%2
灰石城中以兽人为主,蛊师无数,城主更是七转蛊仙层次的强者。

%3
灰石城主府。

%4
先前乔装打扮成蛊师,询问过朱爷的两位羽民蛊仙,分别是七转蛊仙郸沉郁、六转蛊仙云理理,正受到灰石城主的招待。

%5
灰石城主乃是一位兽人,熊头人身,皮发灰白,双眼皆瞎,只能借助蛊虫来视物。

%6
“眼下人族势大,必须加以遏制。灰石城主,你是一方镇守,当明白天下大势。若不提前扼杀,人族必将崛起!今后的北原就是眼下的中洲啊,要想独善其身,真的是一场奢望。”云理理用沉重的语调劝说道。

%7
灰石城主摸了摸头顶上的灰毛,瓮声瓮气地道:“两位羽民蛊仙,我只是个瞎子,看不出什么大势,你们高抬我了。”

%8
“自从我们见面交谈,这熊瞎子就一味推托,可恶!”云理理心中暗骂,自己白费那么多口水,结果这灰石城主简直是油盐不进。

%9
这是,郸沉郁开口道:“灰石仙友,此次我等前来邀约,并非是我等个人意图,也不是单单我羽族的意向,而是包括羽族、雪民、石人、鲛人、毛民、墨人、蛋 人、小人、菇人的共识!自从联络以来,各族的八转蛊仙强者都纷纷响应,兽族中的勤蜂、太极、巨象三位大人都亲口答应参与此事。”

%10
灰石城主顿时心头一震,暗想:“这次各族串联,似乎和往昔不同,竟然声势如此浩大,并且规格如此之高。就连我族的天榜强者都几乎全都参与进来了。”

%11
此时,异族蛊仙之间盛行天地人三榜。人榜是那些蛊仙种子,蛊师级的罕见强者。人榜中的前几位,甚至能对抗六转底层蛊仙。

%12
地榜乃是六转、七转中的强者,不是战力卓绝,就是有一技之长。

%13
天榜共有十八位名额,乃是五域两天中最强的八转蛊仙。其中兽族强者就有四位,足见兽族的强势。

%14
勤蜂、太极、巨象便是其中三人,还有一位玄龟,行踪神秘,很难得见。

%15
相比他们,灰石城主虽然是七转中的强者,坐拥一方,但也无法相提并论,只能算是小角色了。

%16
郸沉郁又接着道:“此次各族联合,共讨人族,有功之人皆得厚赏。各族都必须开放族库,只要功勋足够,兑换仙蛊也是寻常。”

%17
灰石城主怦然心动,不禁问道:“难道我有熊氏也开放族库吗?”

%18
“当然。”郸沉郁道,“若不是先得到有熊氏的允许,我们俩人会专程到你这里来邀请吗?我们也知道你和你的部族之间,存在一些矛盾的。”

%19
灰石城主点点头,叹息一声:“唉,让两位使者见笑了。我很想一口答应二位,但是我女儿生了怪病,想必二位也清楚不过吧?”

%20
云理理点头:“正因为如此,我认为城主你更应当参加此次联军。只要城主你立下足够的功勋,兑换到治疗仙蛊,亦或者请求某位蛊仙出手,治好你女儿的怪病,绝对大有希望。”

%21
灰石城主哀叹一声:“可是我十分担心,若是我离开此地参战,我家女儿却要留在这里,我十分放心不下啊。”

%22
两位羽民蛊仙对视一眼,心中均道:这灰石城主果然和传闻一样,极其溺爱他那唯一的女儿啊。传闻中,他这女儿怪病难治,即便是蛊仙出手,也救治不好。此女 常年卧病在床,昏迷不醒。不能随意移动身躯,一有外力碰触,便浑身发颤,体内空窍濒临自爆,只有依靠蛊仙仙元灌输镇压,方能缓解病症,保住性命。

%23
灰石城主因此几乎寸步不离灰石城,就是想时时刻刻照顾爱女,尽全力保住女儿性命。

%24
“这可怎么相劝呢?”两位羽民蛊仙犯难了。灰石城主实力卓绝,七转巅峰强者,若他不参加,此次两位羽民蛊仙的功勋就要大打折扣了。

%25
张红狐站在城主府的门口。

%26
他是兽人五转蛊师,狐头人身,年岁已经很高,鼻子下端拖着两根花白的胡须,一直下垂到胸前。

%27
年岁不饶人,但张红狐仍旧努力站得笔直。因为他身为灰石城主府的大总管,以此为荣,兢兢业业地要求自己,绝不可因自己的形象而堕了城主府的威风。

%28
城主府的门旁墙壁上,张贴着一幅巨大的文榜。文榜已经张贴在这里好多年了,就是灰石城主高价悬赏,寻求良医为爱女治病的文贴。

%29
从张贴文榜到现在,城主府门口因此拥挤不堪,许多慕名而来的揭榜人络绎不绝。

%30
“可惜这些人,都是蛊师而已,来到这里揭榜不过是贪图老爷的好处。真正有本事治好大小姐的,恐怕只有那些蛊仙大医师了。”

%31
“唉,大小姐命苦啊,让人心痛!若真能治好大小姐,我老张宁愿舍了这身性命。”张红狐心道,他对城主一家忠心耿耿。

%32
“怎么会有这么多人?”拥挤的人群外围,走来一座马车。马车中的少年乖,透过窗帘看到这一幕,十分惊疑。

%33
和之前惨兮兮的状态不同,此刻的乖不仅浑身无伤,而且一身锦绣衣裳,披着一层貂皮披风,满脸红光,嘴里的牙齿也长齐了。双眼冒光,精力旺盛至极。

%34
他一手端着酒杯,另一手拿着一个烧烤得外香里脆的烤肉,小桌上还有一盆盆的各式水果。

%35
他旁边坐着的则是朱爷。

%36
朱爷的心情很不美丽,冷哼一声:“你见识少,别大惊小怪!文榜张贴之后天天如此。”

%37
乖龇起了牙,一边用小拇指剔牙上的肉丝,一边道:“朱爷,您看,我们若是排队,不知何时才能轮到我们。我有一个好想法。”

%38
朱爷嘬牙花子,瞪眼看向乖:“你这龟孙,又想了什么鬼花样?”

%39
朱爷不禁想起来的这一路上,少年乖是如何一步步讲条件的。

%40
起先,乖只是要求坐上马车。但随后不久,他恬着脸对朱爷道:“朱爷,小的有个好想法。我想喝点水。”

%41
不待朱爷说话,乖又开口道:“您瞧我的嘴唇,都干裂出口子了。我真的太渴了,喝了水能让我状态更好,到了城主府给城主女儿治病,更有把握啊。”

%42
朱爷冷哼一声,没有多想,就取出水来扔给乖。

%43
片刻后,乖道:“朱爷,我有个好想法。您瞧,我不喝水也就罢了,喝水开了胃,饿的肚子都叫了,不如给我点吃的吧。”

%44
朱爷便给了他吃的。

%45
于是,这就一发不可收拾起来。

%46
“朱爷,您老不妨给我件好看的衣裳罢。您想啊,我这个模样去给城主爱女治病,估计连门都不给进啊。”

%47
“朱爷,我这痛啊,身上伤势大大小小,真的不轻呢。这要是被查出来,城主府那边能信我吗?我连自己都治疗不好,更何况给城主的女儿治病?”

%48
于是,这一路上,乖不仅是吃饱喝足,而且上等皮衣加身,全身也都被治疗得当,整个人焕然一新。

%49
朱爷将回本的希望寄托在了乖的身上,愿意一试,但也恶心的不行。现在乖又用那种讨厌至极的语气,向朱爷开口。

%50
朱爷心中顿时感到了一点不妙。

%51
果然,乖又继续开口道:“朱爷,我就算吃的再好,穿的再好,出去了,也只是您的俘虏。城主府那边对我能有多少信心呢?城主府肯定是要拿最有希望的蛊师,进入城主府内尝试了。”

%52
朱爷冷哼一声:“别说城主府了,就是老子我对你也是信心不足啊。不过你放心,就算你治不好城主爱女,城主也不会拿我怎么样。城主宽宏大量,就算是最名不副实的蛊师也只是被赶出去而已。若真有一些手段,就算医治无效,城主府也会在临走前送上一份酬劳。”

%53
“嘿嘿。”乖干笑两声,“朱爷您的想法,小的懂!正因为如此啊,咱们不妨玩大一点。虽然这对朱爷您有点冒犯,但我们能立即进入城主府啊。就算小的失败了,也更能和朱爷您脱清干系呢。”

%54
朱爷眯起双眼,龇牙咧嘴:“小子,有什么话直接讲。再敢废话,我就把你的骨头一寸寸的全部捏碎!”

%55
于是,片刻之后。

%56
一辆马车径直闯到城主府前。

%57
被冲挤到两旁的蛊师们无不喝骂。

%58
“是谁啊,这么没规矩!”

%59
“哟,我还以为是谁。原来是你啊,朱头三!”

%60
“老朱,你这是不给我们面子。你究竟想干嘛。”

%61
朱爷冷哼一声,跳下车,打开车门,用恭敬的语气道:“大人,地方到了,您这边请。”

%62
少年乖施施然走了出去。

%63
全场一静,随后哗然。

%64
“什么鬼!一个人类的小崽子?”

%65
“老朱,这就是你的打算?你疯了吧你!”

%66
“不对劲!老朱我熟悉,怎么可能随意被人骗。看来这人族小子恐怕是有几把刷子。”

%67
“哼,我看悬!老朱这次出去,商队被劫了。只怕他现在已经疯了。”

%68
被人说疯了的朱爷,却是一点都不急躁,笑眯眯地来到城主府大总管的面前,拱手弯腰:“狐爷,您老胃口好啊。”

%69
张红狐深深地看了一眼朱爷,又看向少年乖。

%70
他渐渐皱起眉头。

%71
这个少年乖身上的蛊师气息若隐若现,似有似无,张红狐一时间也琢磨不透。

%72
犹豫了一下,张红狐还是点头道:“你们先进来试一试吧。”

%73
少年乖面带微笑,在众多蛊师惊诧、凶狠、猜疑的目光中,从容淡定地走进了城主府。

%74
小小少年的心中这样想着:“他娘的,这一路上我是吃饱喝足了,现在还过了一把瘾!爽!就算是死,也是赚了。”

\end{this_body}

